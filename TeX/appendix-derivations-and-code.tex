\chapter{Addenda, Derivations, and Extended Code}


\section{Constrained Rosenbrock Problem}
\label{sect:rosen-derivation}
Here, we derive a closed-form solution to the constrained Rosenbrock problem, restated from Eq. \ref{eq:constrained-rosenbrock} as:

\begin{mini}
    |l|
        {x, y}{(1-x)^2 + 100 (y - x^2)^2}
        {}{}
    \addConstraint{x^2 + y^2 \leq 1}
\end{mini}

The objective function, which we denote $f(x, y)$ is shown in Figure \ref{fig:unconstrained-rosen}; the constraint limits the feasible region to the unit ball.

\begin{figure}[H]
    \centering
    \ifdraft{}{\input{figures/rosen.pgf}}
    \caption{Rosenbrock Function}
    \label{fig:unconstrained-rosen}
\end{figure}


We find the gradient of $f$ to be:

\begin{equation}
    \nabla f =
    \begin{bmatrix}
        \diff{f}{x} \\
        \diff{f}{y}
    \end{bmatrix} =
    \begin{bmatrix}
        400x(x^2 - y) + 2x - 2 \\
        -200x^2 + 200y
    \end{bmatrix}
\end{equation}

Ignoring the constraint to begin, we find the unconstrained critical points by setting $\nabla f = \vec{0}$. Thus, the $\diff{f}{y}=0$ condition yields $x^2 = y$, which is substituted into the remaining equation to yield:

\begin{equation}
    400x(y-y)+2x-2 = 0 \implies 2x-2 = 0 \implies x=1
\end{equation}

Given $x=1$, the $\diff{f}{y}=0$ conditions implies $y=1$. Thus, $(x, y) = (1, 1)$ is the only critical point of the unconstrained problem.\footnote{Further analysis or graphical inspection reveals this to be the global minimum of the unconstrained problem.}

Thus, the only critical point for the unconstrained problem lies outside the feasible region. The complementary slackness optimality condition implies that criticality ($\nabla f = 0$ locally) is a requirement for optimality in the absence of an active constraint. Thus, we infer that the constrained optimum must lie on the constraint boundary and that the constraint have a nonzero associated dual variable.

We now prepare to formulate and directly solve optimality conditions including the constraint using the KKT conditions. We first rewrite the constraint in the form $g(x) \leq 0$, where here, $g(x) = x^2+y^2-1$. The constraint gradient is then:

\begin{equation}
    \nabla g =
    \begin{bmatrix}
        2x \\
        2y
    \end{bmatrix}
\end{equation}

Now, the optimality conditions can be formed. For the unknowns $x, y, \lambda$, we obtain a set of two equations from the stationarity requirement of the Lagrangian:

\begin{equation}
    \nabla f + \lambda \nabla g(x) = \vec{0}
\end{equation}

A third equation comes from the constraint itself, which we now know to be tight:

\begin{equation}
    g(x) = 0
\end{equation}

This system of three equations does not readily admit an analytic solution, but a numerical solution can be easily obtained:

\begin{equation}
    x = 0.7864, y = 0.6177, \lambda = 0.1215
    \label{eq:constrained-rosen-solution}
\end{equation}

Finally, we can illustrate the poor scaling in this problem by evaluating the condition number of the Hessian at the optimum. The Hessian matrix of the objective function is found to be:

\begin{equation}
    \mat{H} =
    \begin{bmatrix}
        \ddiff{f}{x} & \ddiffm{f}{x}{y} \\
        \ddiffm{f}{x}{y} & \ddiff{f}{y}
    \end{bmatrix} =
    \begin{bmatrix}
        1200x^2 - 400y + 2 & -400x \\
        -400x & 200
    \end{bmatrix}
\end{equation}

Which, evaluated at the optimum of $(x, y) = (0.7864, 0.6177)$ yields:

\begin{equation}
    \mat{H} \approx
    \begin{bmatrix}
        497 & -315 \\
        -315 & 200
    \end{bmatrix}
\end{equation}

After eigenvalue factorization, we evaluate the condition number to find:

\begin{equation}
    \mathrm{cond}(\mat{H}) \approx 1054
\end{equation}


\section{Simple Wing}
\label{sect:simple-wing-code}

The Simple Wing problem described in Section \ref{sect:simple-wing} can be solved using the following AeroSandbox code:

\begin{minted}{python}

import aerosandbox as asb
import aerosandbox.numpy as np

### Constants
form_factor = 1.2  # form factor [-]
oswalds_efficiency = 0.95  # Oswald efficiency factor [-]
viscosity = 1.78e-5  # viscosity of air [kg/m/s]
density = 1.23  # density of air [kg/m^3]
airfoil_thickness_fraction = 0.12  # airfoil thickness to chord ratio [-]
ultimate_load_factor = 3.8  # ultimate load factor [-]
airspeed_takeoff = 22  # takeoff speed [m/s]
CL_max = 1.5  # max CL with flaps down [-]
wetted_area_ratio = 2.05  # wetted area ratio [-]
W_W_coeff1 = 8.71e-5  # Wing Weight Coefficient 1 [1/m]
W_W_coeff2 = 45.24  # Wing Weight Coefficient 2 [Pa]
drag_area_fuselage = 0.031  # fuselage drag area [m^2]
weight_fuselage = 4940.0  # aircraft weight excluding wing [N]

opti = asb.Opti()  # initialize an optimization environment

### Variables
aspect_ratio = opti.variable(init_guess=10)  # aspect ratio
wing_area = opti.variable(init_guess=10)  # total wing area [m^2]
airspeed = opti.variable(init_guess=100)  # cruising speed [m/s]
weight = opti.variable(init_guess=10000)  # total aircraft weight [N]
CL = opti.variable(init_guess=1)  # Lift coefficient of wing [-]

### Models
# Aerodynamics model
CD_fuselage = drag_area_fuselage / wing_area
Re = (density / viscosity) * airspeed * (wing_area / aspect_ratio) ** 0.5
Cf = 0.074 / Re ** 0.2
CD_profile = form_factor * Cf * wetted_area_ratio
CD_induced = CL ** 2 / (np.pi * aspect_ratio * oswalds_efficiency)
CD = CD_fuselage + CD_profile + CD_induced
dynamic_pressure = 0.5 * density * airspeed ** 2
drag = dynamic_pressure * wing_area * CD
lift_cruise = dynamic_pressure * wing_area * CL
lift_takeoff = 0.5 * density * wing_area * CL_max * airspeed_takeoff ** 2

# Wing weight model
weight_wing_structural = W_W_coeff1 * (
        ultimate_load_factor * aspect_ratio ** 1.5 *
        (weight_fuselage * weight * wing_area) ** 0.5
) / airfoil_thickness_fraction
weight_wing_surface = W_W_coeff2 * wing_area
weight_wing = weight_wing_surface + weight_wing_structural

### Constraints
opti.subject_to([
    weight <= lift_cruise,
    weight <= lift_takeoff,
    weight == weight_fuselage + weight_wing
])

### Objective
opti.minimize(drag)

sol = opti.solve()

\end{minted}


\section{Simple Aircraft (SimpleAC)}
\label{sect:simpleac-code}

The Simple Aircraft (SimpleAC) problem described in Section \ref{sect:simpleac} can be solved using the following AeroSandbox code:

\begin{minted}{python}

import aerosandbox as asb
import aerosandbox.numpy as np

opti = asb.Opti()

### Env. constants
g = 9.81  # gravitational acceleration, m/s^2
mu = 1.775e-5  # viscosity of air, kg/m/s
rho = 1.23  # density of air, kg/m^3
rho_f = 817  # density of fuel, kg/m^3

### Non-dimensional constants
C_Lmax = 1.6  # stall CL
e = 0.92  # Oswald's efficiency factor
k = 1.17  # form factor
N_ult = 3.3  # ultimate load factor
S_wetratio = 2.075  # wetted area ratio
tau = 0.12  # airfoil thickness to chord ratio
W_W_coeff1 = 2e-5  # wing weight coefficient 1
W_W_coeff2 = 60  # wing weight coefficient 2

### Dimensional constants
Range = 1000e3  # aircraft range, m
TSFC = 0.6 / 3600  # thrust specific fuel consumption, 1/sec
V_min = 25  # takeoff speed, m/s
W_0 = 6250  # aircraft weight excluding wing, N

### Free variables (same as SimPleAC, with extraneous variables removed)
AR = opti.variable(init_guess=10, log_transform=True)  # aspect ratio
S = opti.variable(init_guess=10, log_transform=True)  # total wing area, m^2
V = opti.variable(init_guess=100, log_transform=True)  # cruise speed, m/s
W = opti.variable(init_guess=10000, log_transform=True)  # total aircraft weight, N
C_L = opti.variable(init_guess=1, log_transform=True)  # lift coefficient
W_f = opti.variable(init_guess=3000, log_transform=True)  # fuel weight, N
V_f_fuse = opti.variable(init_guess=1, log_transform=True)  # fuel volume in the fuselage, m^3

### Wing weight
W_w_surf = W_W_coeff2 * S
W_w_strc = W_W_coeff1 / tau * N_ult * AR ** 1.5 * np.sqrt(
    (W_0 + V_f_fuse * g * rho_f) * W * S
)
W_w = W_w_surf + W_w_strc

### Entire weight
opti.subject_to(W >= W_0 + W_w + W_f)

### Lift equals weight constraint
opti.subject_to([
    W_0 + W_w + 0.5 * W_f <= 0.5 * rho * S * C_L * V ** 2,
    W <= 0.5 * rho * S * C_Lmax * V_min ** 2,
])

### Flight duration
T_flight = Range / V

### Drag
Re = (rho / mu) * V * (S / AR) ** 0.5
C_f = 0.074 / Re ** 0.2

CDA0 = V_f_fuse / 10

C_D_fuse = CDA0 / S
C_D_wpar = k * C_f * S_wetratio
C_D_ind = C_L ** 2 / (np.pi * AR * e)
C_D = C_D_fuse + C_D_wpar + C_D_ind
D = 0.5 * rho * S * C_D * V ** 2

opti.subject_to(W_f >= TSFC * T_flight * D)

V_f = W_f / g / rho_f
V_f_wing = 0.03 * S ** 1.5 / AR ** 0.5 * tau

V_f_avail = V_f_wing + V_f_fuse

opti.subject_to(V_f_avail >= V_f)

opti.minimize(W_f)

sol = opti.solve()

\end{minted}

\section{Discontinuities at Non-Optimal Points}
\label{sect:discontinuities-at-non-optimal}

Continuing from Section \ref{sect:differentiability}, we note that discontinuity and non-differentiability is still admissible if it does not occur at the optimum. For example, we can pose the following problem, which has countably infinite non-continuous, non-differentiable points:

\begin{mini}
    |l|
        {x}{x \cdot \left\lfloor x + \frac{1}{2} \right\rfloor + \frac{1}{10} x^2}
        {}{}
    \label{eq:nondiff}
\end{mini}

\noindent
where the objective function and its derivative are visualized in Figure \ref{fig:nondiff}. This problem is easily solved as-written in AeroSandbox, because the region surrounding the optimum ($x=0$, $f(x)=0$) is locally $C_1$-continuous. Convergence is easily achieved even if the initial guess is one of these discontinuous points: with an initial guess of $x_0=10.5$, solution is achieved in just 6 iterations.

\begin{figure}[H]
    \centering
    %% Creator: Matplotlib, PGF backend
%%
%% To include the figure in your LaTeX document, write
%%   \input{<filename>.pgf}
%%
%% Make sure the required packages are loaded in your preamble
%%   \usepackage{pgf}
%%
%% Figures using additional raster images can only be included by \input if
%% they are in the same directory as the main LaTeX file. For loading figures
%% from other directories you can use the `import` package
%%   \usepackage{import}
%%
%% and then include the figures with
%%   \import{<path to file>}{<filename>.pgf}
%%
%% Matplotlib used the following preamble
%%   \usepackage{fontspec}
%%
\begingroup%
\makeatletter%
\begin{pgfpicture}%
\pgfpathrectangle{\pgfpointorigin}{\pgfqpoint{6.000000in}{3.000000in}}%
\pgfusepath{use as bounding box, clip}%
\begin{pgfscope}%
\pgfsetbuttcap%
\pgfsetmiterjoin%
\definecolor{currentfill}{rgb}{1.000000,1.000000,1.000000}%
\pgfsetfillcolor{currentfill}%
\pgfsetlinewidth{0.000000pt}%
\definecolor{currentstroke}{rgb}{1.000000,1.000000,1.000000}%
\pgfsetstrokecolor{currentstroke}%
\pgfsetstrokeopacity{0.000000}%
\pgfsetdash{}{0pt}%
\pgfpathmoveto{\pgfqpoint{0.000000in}{0.000000in}}%
\pgfpathlineto{\pgfqpoint{6.000000in}{0.000000in}}%
\pgfpathlineto{\pgfqpoint{6.000000in}{3.000000in}}%
\pgfpathlineto{\pgfqpoint{0.000000in}{3.000000in}}%
\pgfpathclose%
\pgfusepath{fill}%
\end{pgfscope}%
\begin{pgfscope}%
\pgfsetbuttcap%
\pgfsetmiterjoin%
\definecolor{currentfill}{rgb}{0.917647,0.917647,0.949020}%
\pgfsetfillcolor{currentfill}%
\pgfsetlinewidth{0.000000pt}%
\definecolor{currentstroke}{rgb}{0.000000,0.000000,0.000000}%
\pgfsetstrokecolor{currentstroke}%
\pgfsetstrokeopacity{0.000000}%
\pgfsetdash{}{0pt}%
\pgfpathmoveto{\pgfqpoint{0.464152in}{0.650389in}}%
\pgfpathlineto{\pgfqpoint{2.888860in}{0.650389in}}%
\pgfpathlineto{\pgfqpoint{2.888860in}{2.450897in}}%
\pgfpathlineto{\pgfqpoint{0.464152in}{2.450897in}}%
\pgfpathclose%
\pgfusepath{fill}%
\end{pgfscope}%
\begin{pgfscope}%
\pgfpathrectangle{\pgfqpoint{0.464152in}{0.650389in}}{\pgfqpoint{2.424708in}{1.800508in}}%
\pgfusepath{clip}%
\pgfsetroundcap%
\pgfsetroundjoin%
\pgfsetlinewidth{1.606000pt}%
\definecolor{currentstroke}{rgb}{1.000000,1.000000,1.000000}%
\pgfsetstrokecolor{currentstroke}%
\pgfsetdash{}{0pt}%
\pgfpathmoveto{\pgfqpoint{0.941746in}{0.650389in}}%
\pgfpathlineto{\pgfqpoint{0.941746in}{2.450897in}}%
\pgfusepath{stroke}%
\end{pgfscope}%
\begin{pgfscope}%
\definecolor{textcolor}{rgb}{0.150000,0.150000,0.150000}%
\pgfsetstrokecolor{textcolor}%
\pgfsetfillcolor{textcolor}%
\pgftext[x=0.941746in,y=0.518445in,,top]{\color{textcolor}\sffamily\fontsize{11.000000}{13.200000}\selectfont \(\displaystyle {\ensuremath{-}2}\)}%
\end{pgfscope}%
\begin{pgfscope}%
\pgfpathrectangle{\pgfqpoint{0.464152in}{0.650389in}}{\pgfqpoint{2.424708in}{1.800508in}}%
\pgfusepath{clip}%
\pgfsetroundcap%
\pgfsetroundjoin%
\pgfsetlinewidth{1.606000pt}%
\definecolor{currentstroke}{rgb}{1.000000,1.000000,1.000000}%
\pgfsetstrokecolor{currentstroke}%
\pgfsetdash{}{0pt}%
\pgfpathmoveto{\pgfqpoint{1.676506in}{0.650389in}}%
\pgfpathlineto{\pgfqpoint{1.676506in}{2.450897in}}%
\pgfusepath{stroke}%
\end{pgfscope}%
\begin{pgfscope}%
\definecolor{textcolor}{rgb}{0.150000,0.150000,0.150000}%
\pgfsetstrokecolor{textcolor}%
\pgfsetfillcolor{textcolor}%
\pgftext[x=1.676506in,y=0.518445in,,top]{\color{textcolor}\sffamily\fontsize{11.000000}{13.200000}\selectfont \(\displaystyle {0}\)}%
\end{pgfscope}%
\begin{pgfscope}%
\pgfpathrectangle{\pgfqpoint{0.464152in}{0.650389in}}{\pgfqpoint{2.424708in}{1.800508in}}%
\pgfusepath{clip}%
\pgfsetroundcap%
\pgfsetroundjoin%
\pgfsetlinewidth{1.606000pt}%
\definecolor{currentstroke}{rgb}{1.000000,1.000000,1.000000}%
\pgfsetstrokecolor{currentstroke}%
\pgfsetdash{}{0pt}%
\pgfpathmoveto{\pgfqpoint{2.411266in}{0.650389in}}%
\pgfpathlineto{\pgfqpoint{2.411266in}{2.450897in}}%
\pgfusepath{stroke}%
\end{pgfscope}%
\begin{pgfscope}%
\definecolor{textcolor}{rgb}{0.150000,0.150000,0.150000}%
\pgfsetstrokecolor{textcolor}%
\pgfsetfillcolor{textcolor}%
\pgftext[x=2.411266in,y=0.518445in,,top]{\color{textcolor}\sffamily\fontsize{11.000000}{13.200000}\selectfont \(\displaystyle {2}\)}%
\end{pgfscope}%
\begin{pgfscope}%
\pgfpathrectangle{\pgfqpoint{0.464152in}{0.650389in}}{\pgfqpoint{2.424708in}{1.800508in}}%
\pgfusepath{clip}%
\pgfsetroundcap%
\pgfsetroundjoin%
\pgfsetlinewidth{0.702625pt}%
\definecolor{currentstroke}{rgb}{1.000000,1.000000,1.000000}%
\pgfsetstrokecolor{currentstroke}%
\pgfsetdash{}{0pt}%
\pgfpathmoveto{\pgfqpoint{0.574366in}{0.650389in}}%
\pgfpathlineto{\pgfqpoint{0.574366in}{2.450897in}}%
\pgfusepath{stroke}%
\end{pgfscope}%
\begin{pgfscope}%
\pgfpathrectangle{\pgfqpoint{0.464152in}{0.650389in}}{\pgfqpoint{2.424708in}{1.800508in}}%
\pgfusepath{clip}%
\pgfsetroundcap%
\pgfsetroundjoin%
\pgfsetlinewidth{0.702625pt}%
\definecolor{currentstroke}{rgb}{1.000000,1.000000,1.000000}%
\pgfsetstrokecolor{currentstroke}%
\pgfsetdash{}{0pt}%
\pgfpathmoveto{\pgfqpoint{1.309126in}{0.650389in}}%
\pgfpathlineto{\pgfqpoint{1.309126in}{2.450897in}}%
\pgfusepath{stroke}%
\end{pgfscope}%
\begin{pgfscope}%
\pgfpathrectangle{\pgfqpoint{0.464152in}{0.650389in}}{\pgfqpoint{2.424708in}{1.800508in}}%
\pgfusepath{clip}%
\pgfsetroundcap%
\pgfsetroundjoin%
\pgfsetlinewidth{0.702625pt}%
\definecolor{currentstroke}{rgb}{1.000000,1.000000,1.000000}%
\pgfsetstrokecolor{currentstroke}%
\pgfsetdash{}{0pt}%
\pgfpathmoveto{\pgfqpoint{2.043886in}{0.650389in}}%
\pgfpathlineto{\pgfqpoint{2.043886in}{2.450897in}}%
\pgfusepath{stroke}%
\end{pgfscope}%
\begin{pgfscope}%
\pgfpathrectangle{\pgfqpoint{0.464152in}{0.650389in}}{\pgfqpoint{2.424708in}{1.800508in}}%
\pgfusepath{clip}%
\pgfsetroundcap%
\pgfsetroundjoin%
\pgfsetlinewidth{0.702625pt}%
\definecolor{currentstroke}{rgb}{1.000000,1.000000,1.000000}%
\pgfsetstrokecolor{currentstroke}%
\pgfsetdash{}{0pt}%
\pgfpathmoveto{\pgfqpoint{2.778646in}{0.650389in}}%
\pgfpathlineto{\pgfqpoint{2.778646in}{2.450897in}}%
\pgfusepath{stroke}%
\end{pgfscope}%
\begin{pgfscope}%
\definecolor{textcolor}{rgb}{0.150000,0.150000,0.150000}%
\pgfsetstrokecolor{textcolor}%
\pgfsetfillcolor{textcolor}%
\pgftext[x=1.676506in,y=0.327223in,,top]{\color{textcolor}\sffamily\fontsize{12.000000}{14.400000}\selectfont \(\displaystyle x\)}%
\end{pgfscope}%
\begin{pgfscope}%
\pgfpathrectangle{\pgfqpoint{0.464152in}{0.650389in}}{\pgfqpoint{2.424708in}{1.800508in}}%
\pgfusepath{clip}%
\pgfsetroundcap%
\pgfsetroundjoin%
\pgfsetlinewidth{1.606000pt}%
\definecolor{currentstroke}{rgb}{1.000000,1.000000,1.000000}%
\pgfsetstrokecolor{currentstroke}%
\pgfsetdash{}{0pt}%
\pgfpathmoveto{\pgfqpoint{0.464152in}{0.732219in}}%
\pgfpathlineto{\pgfqpoint{2.888860in}{0.732219in}}%
\pgfusepath{stroke}%
\end{pgfscope}%
\begin{pgfscope}%
\definecolor{textcolor}{rgb}{0.150000,0.150000,0.150000}%
\pgfsetstrokecolor{textcolor}%
\pgfsetfillcolor{textcolor}%
\pgftext[x=0.256166in, y=0.679205in, left, base]{\color{textcolor}\sffamily\fontsize{11.000000}{13.200000}\selectfont \(\displaystyle {0}\)}%
\end{pgfscope}%
\begin{pgfscope}%
\pgfpathrectangle{\pgfqpoint{0.464152in}{0.650389in}}{\pgfqpoint{2.424708in}{1.800508in}}%
\pgfusepath{clip}%
\pgfsetroundcap%
\pgfsetroundjoin%
\pgfsetlinewidth{1.606000pt}%
\definecolor{currentstroke}{rgb}{1.000000,1.000000,1.000000}%
\pgfsetstrokecolor{currentstroke}%
\pgfsetdash{}{0pt}%
\pgfpathmoveto{\pgfqpoint{0.464152in}{1.062893in}}%
\pgfpathlineto{\pgfqpoint{2.888860in}{1.062893in}}%
\pgfusepath{stroke}%
\end{pgfscope}%
\begin{pgfscope}%
\definecolor{textcolor}{rgb}{0.150000,0.150000,0.150000}%
\pgfsetstrokecolor{textcolor}%
\pgfsetfillcolor{textcolor}%
\pgftext[x=0.256166in, y=1.009879in, left, base]{\color{textcolor}\sffamily\fontsize{11.000000}{13.200000}\selectfont \(\displaystyle {2}\)}%
\end{pgfscope}%
\begin{pgfscope}%
\pgfpathrectangle{\pgfqpoint{0.464152in}{0.650389in}}{\pgfqpoint{2.424708in}{1.800508in}}%
\pgfusepath{clip}%
\pgfsetroundcap%
\pgfsetroundjoin%
\pgfsetlinewidth{1.606000pt}%
\definecolor{currentstroke}{rgb}{1.000000,1.000000,1.000000}%
\pgfsetstrokecolor{currentstroke}%
\pgfsetdash{}{0pt}%
\pgfpathmoveto{\pgfqpoint{0.464152in}{1.393567in}}%
\pgfpathlineto{\pgfqpoint{2.888860in}{1.393567in}}%
\pgfusepath{stroke}%
\end{pgfscope}%
\begin{pgfscope}%
\definecolor{textcolor}{rgb}{0.150000,0.150000,0.150000}%
\pgfsetstrokecolor{textcolor}%
\pgfsetfillcolor{textcolor}%
\pgftext[x=0.256166in, y=1.340553in, left, base]{\color{textcolor}\sffamily\fontsize{11.000000}{13.200000}\selectfont \(\displaystyle {4}\)}%
\end{pgfscope}%
\begin{pgfscope}%
\pgfpathrectangle{\pgfqpoint{0.464152in}{0.650389in}}{\pgfqpoint{2.424708in}{1.800508in}}%
\pgfusepath{clip}%
\pgfsetroundcap%
\pgfsetroundjoin%
\pgfsetlinewidth{1.606000pt}%
\definecolor{currentstroke}{rgb}{1.000000,1.000000,1.000000}%
\pgfsetstrokecolor{currentstroke}%
\pgfsetdash{}{0pt}%
\pgfpathmoveto{\pgfqpoint{0.464152in}{1.724242in}}%
\pgfpathlineto{\pgfqpoint{2.888860in}{1.724242in}}%
\pgfusepath{stroke}%
\end{pgfscope}%
\begin{pgfscope}%
\definecolor{textcolor}{rgb}{0.150000,0.150000,0.150000}%
\pgfsetstrokecolor{textcolor}%
\pgfsetfillcolor{textcolor}%
\pgftext[x=0.256166in, y=1.671228in, left, base]{\color{textcolor}\sffamily\fontsize{11.000000}{13.200000}\selectfont \(\displaystyle {6}\)}%
\end{pgfscope}%
\begin{pgfscope}%
\pgfpathrectangle{\pgfqpoint{0.464152in}{0.650389in}}{\pgfqpoint{2.424708in}{1.800508in}}%
\pgfusepath{clip}%
\pgfsetroundcap%
\pgfsetroundjoin%
\pgfsetlinewidth{1.606000pt}%
\definecolor{currentstroke}{rgb}{1.000000,1.000000,1.000000}%
\pgfsetstrokecolor{currentstroke}%
\pgfsetdash{}{0pt}%
\pgfpathmoveto{\pgfqpoint{0.464152in}{2.054916in}}%
\pgfpathlineto{\pgfqpoint{2.888860in}{2.054916in}}%
\pgfusepath{stroke}%
\end{pgfscope}%
\begin{pgfscope}%
\definecolor{textcolor}{rgb}{0.150000,0.150000,0.150000}%
\pgfsetstrokecolor{textcolor}%
\pgfsetfillcolor{textcolor}%
\pgftext[x=0.256166in, y=2.001902in, left, base]{\color{textcolor}\sffamily\fontsize{11.000000}{13.200000}\selectfont \(\displaystyle {8}\)}%
\end{pgfscope}%
\begin{pgfscope}%
\pgfpathrectangle{\pgfqpoint{0.464152in}{0.650389in}}{\pgfqpoint{2.424708in}{1.800508in}}%
\pgfusepath{clip}%
\pgfsetroundcap%
\pgfsetroundjoin%
\pgfsetlinewidth{1.606000pt}%
\definecolor{currentstroke}{rgb}{1.000000,1.000000,1.000000}%
\pgfsetstrokecolor{currentstroke}%
\pgfsetdash{}{0pt}%
\pgfpathmoveto{\pgfqpoint{0.464152in}{2.385590in}}%
\pgfpathlineto{\pgfqpoint{2.888860in}{2.385590in}}%
\pgfusepath{stroke}%
\end{pgfscope}%
\begin{pgfscope}%
\definecolor{textcolor}{rgb}{0.150000,0.150000,0.150000}%
\pgfsetstrokecolor{textcolor}%
\pgfsetfillcolor{textcolor}%
\pgftext[x=0.180124in, y=2.332576in, left, base]{\color{textcolor}\sffamily\fontsize{11.000000}{13.200000}\selectfont \(\displaystyle {10}\)}%
\end{pgfscope}%
\begin{pgfscope}%
\pgfpathrectangle{\pgfqpoint{0.464152in}{0.650389in}}{\pgfqpoint{2.424708in}{1.800508in}}%
\pgfusepath{clip}%
\pgfsetroundcap%
\pgfsetroundjoin%
\pgfsetlinewidth{0.702625pt}%
\definecolor{currentstroke}{rgb}{1.000000,1.000000,1.000000}%
\pgfsetstrokecolor{currentstroke}%
\pgfsetdash{}{0pt}%
\pgfpathmoveto{\pgfqpoint{0.464152in}{0.897556in}}%
\pgfpathlineto{\pgfqpoint{2.888860in}{0.897556in}}%
\pgfusepath{stroke}%
\end{pgfscope}%
\begin{pgfscope}%
\pgfpathrectangle{\pgfqpoint{0.464152in}{0.650389in}}{\pgfqpoint{2.424708in}{1.800508in}}%
\pgfusepath{clip}%
\pgfsetroundcap%
\pgfsetroundjoin%
\pgfsetlinewidth{0.702625pt}%
\definecolor{currentstroke}{rgb}{1.000000,1.000000,1.000000}%
\pgfsetstrokecolor{currentstroke}%
\pgfsetdash{}{0pt}%
\pgfpathmoveto{\pgfqpoint{0.464152in}{1.228230in}}%
\pgfpathlineto{\pgfqpoint{2.888860in}{1.228230in}}%
\pgfusepath{stroke}%
\end{pgfscope}%
\begin{pgfscope}%
\pgfpathrectangle{\pgfqpoint{0.464152in}{0.650389in}}{\pgfqpoint{2.424708in}{1.800508in}}%
\pgfusepath{clip}%
\pgfsetroundcap%
\pgfsetroundjoin%
\pgfsetlinewidth{0.702625pt}%
\definecolor{currentstroke}{rgb}{1.000000,1.000000,1.000000}%
\pgfsetstrokecolor{currentstroke}%
\pgfsetdash{}{0pt}%
\pgfpathmoveto{\pgfqpoint{0.464152in}{1.558904in}}%
\pgfpathlineto{\pgfqpoint{2.888860in}{1.558904in}}%
\pgfusepath{stroke}%
\end{pgfscope}%
\begin{pgfscope}%
\pgfpathrectangle{\pgfqpoint{0.464152in}{0.650389in}}{\pgfqpoint{2.424708in}{1.800508in}}%
\pgfusepath{clip}%
\pgfsetroundcap%
\pgfsetroundjoin%
\pgfsetlinewidth{0.702625pt}%
\definecolor{currentstroke}{rgb}{1.000000,1.000000,1.000000}%
\pgfsetstrokecolor{currentstroke}%
\pgfsetdash{}{0pt}%
\pgfpathmoveto{\pgfqpoint{0.464152in}{1.889579in}}%
\pgfpathlineto{\pgfqpoint{2.888860in}{1.889579in}}%
\pgfusepath{stroke}%
\end{pgfscope}%
\begin{pgfscope}%
\pgfpathrectangle{\pgfqpoint{0.464152in}{0.650389in}}{\pgfqpoint{2.424708in}{1.800508in}}%
\pgfusepath{clip}%
\pgfsetroundcap%
\pgfsetroundjoin%
\pgfsetlinewidth{0.702625pt}%
\definecolor{currentstroke}{rgb}{1.000000,1.000000,1.000000}%
\pgfsetstrokecolor{currentstroke}%
\pgfsetdash{}{0pt}%
\pgfpathmoveto{\pgfqpoint{0.464152in}{2.220253in}}%
\pgfpathlineto{\pgfqpoint{2.888860in}{2.220253in}}%
\pgfusepath{stroke}%
\end{pgfscope}%
\begin{pgfscope}%
\pgfpathrectangle{\pgfqpoint{0.464152in}{0.650389in}}{\pgfqpoint{2.424708in}{1.800508in}}%
\pgfusepath{clip}%
\pgfsetroundcap%
\pgfsetroundjoin%
\pgfsetlinewidth{1.505625pt}%
\definecolor{currentstroke}{rgb}{0.258824,0.521569,0.956863}%
\pgfsetstrokecolor{currentstroke}%
\pgfsetdash{}{0pt}%
\pgfpathmoveto{\pgfqpoint{0.574366in}{2.369056in}}%
\pgfpathlineto{\pgfqpoint{0.671044in}{2.213566in}}%
\pgfpathlineto{\pgfqpoint{0.758055in}{2.075583in}}%
\pgfpathmoveto{\pgfqpoint{0.758056in}{1.662240in}}%
\pgfpathlineto{\pgfqpoint{0.835399in}{1.575953in}}%
\pgfpathlineto{\pgfqpoint{0.912742in}{1.491132in}}%
\pgfpathlineto{\pgfqpoint{0.990085in}{1.407777in}}%
\pgfpathlineto{\pgfqpoint{1.067428in}{1.325887in}}%
\pgfpathlineto{\pgfqpoint{1.125435in}{1.265431in}}%
\pgfpathmoveto{\pgfqpoint{1.125436in}{1.017425in}}%
\pgfpathlineto{\pgfqpoint{1.202779in}{0.972908in}}%
\pgfpathlineto{\pgfqpoint{1.280122in}{0.929856in}}%
\pgfpathlineto{\pgfqpoint{1.357465in}{0.888270in}}%
\pgfpathlineto{\pgfqpoint{1.434808in}{0.848150in}}%
\pgfpathlineto{\pgfqpoint{1.492816in}{0.819021in}}%
\pgfpathmoveto{\pgfqpoint{1.492816in}{0.736352in}}%
\pgfpathlineto{\pgfqpoint{1.570159in}{0.733604in}}%
\pgfpathlineto{\pgfqpoint{1.647502in}{0.732322in}}%
\pgfpathlineto{\pgfqpoint{1.724845in}{0.732505in}}%
\pgfpathlineto{\pgfqpoint{1.802188in}{0.734154in}}%
\pgfpathlineto{\pgfqpoint{1.860196in}{0.736352in}}%
\pgfpathmoveto{\pgfqpoint{1.860196in}{0.819021in}}%
\pgfpathlineto{\pgfqpoint{1.937539in}{0.858042in}}%
\pgfpathlineto{\pgfqpoint{2.014883in}{0.898529in}}%
\pgfpathlineto{\pgfqpoint{2.092226in}{0.940482in}}%
\pgfpathlineto{\pgfqpoint{2.169569in}{0.983900in}}%
\pgfpathlineto{\pgfqpoint{2.227576in}{1.017425in}}%
\pgfpathmoveto{\pgfqpoint{2.227577in}{1.265431in}}%
\pgfpathlineto{\pgfqpoint{2.304920in}{1.346222in}}%
\pgfpathlineto{\pgfqpoint{2.382263in}{1.428478in}}%
\pgfpathlineto{\pgfqpoint{2.459606in}{1.512200in}}%
\pgfpathlineto{\pgfqpoint{2.536949in}{1.597387in}}%
\pgfpathlineto{\pgfqpoint{2.594956in}{1.662240in}}%
\pgfpathmoveto{\pgfqpoint{2.594957in}{2.075583in}}%
\pgfpathlineto{\pgfqpoint{2.691635in}{2.229012in}}%
\pgfpathlineto{\pgfqpoint{2.778646in}{2.369056in}}%
\pgfpathlineto{\pgfqpoint{2.778646in}{2.369056in}}%
\pgfusepath{stroke}%
\end{pgfscope}%
\begin{pgfscope}%
\pgfsetrectcap%
\pgfsetmiterjoin%
\pgfsetlinewidth{1.254687pt}%
\definecolor{currentstroke}{rgb}{1.000000,1.000000,1.000000}%
\pgfsetstrokecolor{currentstroke}%
\pgfsetdash{}{0pt}%
\pgfpathmoveto{\pgfqpoint{0.464152in}{0.650389in}}%
\pgfpathlineto{\pgfqpoint{0.464152in}{2.450897in}}%
\pgfusepath{stroke}%
\end{pgfscope}%
\begin{pgfscope}%
\pgfsetrectcap%
\pgfsetmiterjoin%
\pgfsetlinewidth{1.254687pt}%
\definecolor{currentstroke}{rgb}{1.000000,1.000000,1.000000}%
\pgfsetstrokecolor{currentstroke}%
\pgfsetdash{}{0pt}%
\pgfpathmoveto{\pgfqpoint{2.888860in}{0.650389in}}%
\pgfpathlineto{\pgfqpoint{2.888860in}{2.450897in}}%
\pgfusepath{stroke}%
\end{pgfscope}%
\begin{pgfscope}%
\pgfsetrectcap%
\pgfsetmiterjoin%
\pgfsetlinewidth{1.254687pt}%
\definecolor{currentstroke}{rgb}{1.000000,1.000000,1.000000}%
\pgfsetstrokecolor{currentstroke}%
\pgfsetdash{}{0pt}%
\pgfpathmoveto{\pgfqpoint{0.464152in}{0.650389in}}%
\pgfpathlineto{\pgfqpoint{2.888860in}{0.650389in}}%
\pgfusepath{stroke}%
\end{pgfscope}%
\begin{pgfscope}%
\pgfsetrectcap%
\pgfsetmiterjoin%
\pgfsetlinewidth{1.254687pt}%
\definecolor{currentstroke}{rgb}{1.000000,1.000000,1.000000}%
\pgfsetstrokecolor{currentstroke}%
\pgfsetdash{}{0pt}%
\pgfpathmoveto{\pgfqpoint{0.464152in}{2.450897in}}%
\pgfpathlineto{\pgfqpoint{2.888860in}{2.450897in}}%
\pgfusepath{stroke}%
\end{pgfscope}%
\begin{pgfscope}%
\definecolor{textcolor}{rgb}{0.150000,0.150000,0.150000}%
\pgfsetstrokecolor{textcolor}%
\pgfsetfillcolor{textcolor}%
\pgftext[x=1.676506in,y=2.642261in,,base]{\color{textcolor}\sffamily\fontsize{12.000000}{14.400000}\selectfont Value: \(\displaystyle f(x) = x \cdot \big\lfloor x + \frac{1}{2} \big\rfloor + \frac{1}{10} x^2\)}%
\end{pgfscope}%
\begin{pgfscope}%
\pgfsetbuttcap%
\pgfsetmiterjoin%
\definecolor{currentfill}{rgb}{0.917647,0.917647,0.949020}%
\pgfsetfillcolor{currentfill}%
\pgfsetlinewidth{0.000000pt}%
\definecolor{currentstroke}{rgb}{0.000000,0.000000,0.000000}%
\pgfsetstrokecolor{currentstroke}%
\pgfsetstrokeopacity{0.000000}%
\pgfsetdash{}{0pt}%
\pgfpathmoveto{\pgfqpoint{3.395292in}{0.650389in}}%
\pgfpathlineto{\pgfqpoint{5.820000in}{0.650389in}}%
\pgfpathlineto{\pgfqpoint{5.820000in}{2.450897in}}%
\pgfpathlineto{\pgfqpoint{3.395292in}{2.450897in}}%
\pgfpathclose%
\pgfusepath{fill}%
\end{pgfscope}%
\begin{pgfscope}%
\pgfpathrectangle{\pgfqpoint{3.395292in}{0.650389in}}{\pgfqpoint{2.424708in}{1.800508in}}%
\pgfusepath{clip}%
\pgfsetroundcap%
\pgfsetroundjoin%
\pgfsetlinewidth{1.606000pt}%
\definecolor{currentstroke}{rgb}{1.000000,1.000000,1.000000}%
\pgfsetstrokecolor{currentstroke}%
\pgfsetdash{}{0pt}%
\pgfpathmoveto{\pgfqpoint{3.869649in}{0.650389in}}%
\pgfpathlineto{\pgfqpoint{3.869649in}{2.450897in}}%
\pgfusepath{stroke}%
\end{pgfscope}%
\begin{pgfscope}%
\definecolor{textcolor}{rgb}{0.150000,0.150000,0.150000}%
\pgfsetstrokecolor{textcolor}%
\pgfsetfillcolor{textcolor}%
\pgftext[x=3.869649in,y=0.518445in,,top]{\color{textcolor}\sffamily\fontsize{11.000000}{13.200000}\selectfont \(\displaystyle {\ensuremath{-}2}\)}%
\end{pgfscope}%
\begin{pgfscope}%
\pgfpathrectangle{\pgfqpoint{3.395292in}{0.650389in}}{\pgfqpoint{2.424708in}{1.800508in}}%
\pgfusepath{clip}%
\pgfsetroundcap%
\pgfsetroundjoin%
\pgfsetlinewidth{1.606000pt}%
\definecolor{currentstroke}{rgb}{1.000000,1.000000,1.000000}%
\pgfsetstrokecolor{currentstroke}%
\pgfsetdash{}{0pt}%
\pgfpathmoveto{\pgfqpoint{4.607646in}{0.650389in}}%
\pgfpathlineto{\pgfqpoint{4.607646in}{2.450897in}}%
\pgfusepath{stroke}%
\end{pgfscope}%
\begin{pgfscope}%
\definecolor{textcolor}{rgb}{0.150000,0.150000,0.150000}%
\pgfsetstrokecolor{textcolor}%
\pgfsetfillcolor{textcolor}%
\pgftext[x=4.607646in,y=0.518445in,,top]{\color{textcolor}\sffamily\fontsize{11.000000}{13.200000}\selectfont \(\displaystyle {0}\)}%
\end{pgfscope}%
\begin{pgfscope}%
\pgfpathrectangle{\pgfqpoint{3.395292in}{0.650389in}}{\pgfqpoint{2.424708in}{1.800508in}}%
\pgfusepath{clip}%
\pgfsetroundcap%
\pgfsetroundjoin%
\pgfsetlinewidth{1.606000pt}%
\definecolor{currentstroke}{rgb}{1.000000,1.000000,1.000000}%
\pgfsetstrokecolor{currentstroke}%
\pgfsetdash{}{0pt}%
\pgfpathmoveto{\pgfqpoint{5.345643in}{0.650389in}}%
\pgfpathlineto{\pgfqpoint{5.345643in}{2.450897in}}%
\pgfusepath{stroke}%
\end{pgfscope}%
\begin{pgfscope}%
\definecolor{textcolor}{rgb}{0.150000,0.150000,0.150000}%
\pgfsetstrokecolor{textcolor}%
\pgfsetfillcolor{textcolor}%
\pgftext[x=5.345643in,y=0.518445in,,top]{\color{textcolor}\sffamily\fontsize{11.000000}{13.200000}\selectfont \(\displaystyle {2}\)}%
\end{pgfscope}%
\begin{pgfscope}%
\pgfpathrectangle{\pgfqpoint{3.395292in}{0.650389in}}{\pgfqpoint{2.424708in}{1.800508in}}%
\pgfusepath{clip}%
\pgfsetroundcap%
\pgfsetroundjoin%
\pgfsetlinewidth{0.702625pt}%
\definecolor{currentstroke}{rgb}{1.000000,1.000000,1.000000}%
\pgfsetstrokecolor{currentstroke}%
\pgfsetdash{}{0pt}%
\pgfpathmoveto{\pgfqpoint{3.500650in}{0.650389in}}%
\pgfpathlineto{\pgfqpoint{3.500650in}{2.450897in}}%
\pgfusepath{stroke}%
\end{pgfscope}%
\begin{pgfscope}%
\pgfpathrectangle{\pgfqpoint{3.395292in}{0.650389in}}{\pgfqpoint{2.424708in}{1.800508in}}%
\pgfusepath{clip}%
\pgfsetroundcap%
\pgfsetroundjoin%
\pgfsetlinewidth{0.702625pt}%
\definecolor{currentstroke}{rgb}{1.000000,1.000000,1.000000}%
\pgfsetstrokecolor{currentstroke}%
\pgfsetdash{}{0pt}%
\pgfpathmoveto{\pgfqpoint{4.238647in}{0.650389in}}%
\pgfpathlineto{\pgfqpoint{4.238647in}{2.450897in}}%
\pgfusepath{stroke}%
\end{pgfscope}%
\begin{pgfscope}%
\pgfpathrectangle{\pgfqpoint{3.395292in}{0.650389in}}{\pgfqpoint{2.424708in}{1.800508in}}%
\pgfusepath{clip}%
\pgfsetroundcap%
\pgfsetroundjoin%
\pgfsetlinewidth{0.702625pt}%
\definecolor{currentstroke}{rgb}{1.000000,1.000000,1.000000}%
\pgfsetstrokecolor{currentstroke}%
\pgfsetdash{}{0pt}%
\pgfpathmoveto{\pgfqpoint{4.976644in}{0.650389in}}%
\pgfpathlineto{\pgfqpoint{4.976644in}{2.450897in}}%
\pgfusepath{stroke}%
\end{pgfscope}%
\begin{pgfscope}%
\pgfpathrectangle{\pgfqpoint{3.395292in}{0.650389in}}{\pgfqpoint{2.424708in}{1.800508in}}%
\pgfusepath{clip}%
\pgfsetroundcap%
\pgfsetroundjoin%
\pgfsetlinewidth{0.702625pt}%
\definecolor{currentstroke}{rgb}{1.000000,1.000000,1.000000}%
\pgfsetstrokecolor{currentstroke}%
\pgfsetdash{}{0pt}%
\pgfpathmoveto{\pgfqpoint{5.714642in}{0.650389in}}%
\pgfpathlineto{\pgfqpoint{5.714642in}{2.450897in}}%
\pgfusepath{stroke}%
\end{pgfscope}%
\begin{pgfscope}%
\definecolor{textcolor}{rgb}{0.150000,0.150000,0.150000}%
\pgfsetstrokecolor{textcolor}%
\pgfsetfillcolor{textcolor}%
\pgftext[x=4.607646in,y=0.327223in,,top]{\color{textcolor}\sffamily\fontsize{12.000000}{14.400000}\selectfont \(\displaystyle x\)}%
\end{pgfscope}%
\begin{pgfscope}%
\pgfpathrectangle{\pgfqpoint{3.395292in}{0.650389in}}{\pgfqpoint{2.424708in}{1.800508in}}%
\pgfusepath{clip}%
\pgfsetroundcap%
\pgfsetroundjoin%
\pgfsetlinewidth{1.606000pt}%
\definecolor{currentstroke}{rgb}{1.000000,1.000000,1.000000}%
\pgfsetstrokecolor{currentstroke}%
\pgfsetdash{}{0pt}%
\pgfpathmoveto{\pgfqpoint{3.395292in}{1.095637in}}%
\pgfpathlineto{\pgfqpoint{5.820000in}{1.095637in}}%
\pgfusepath{stroke}%
\end{pgfscope}%
\begin{pgfscope}%
\definecolor{textcolor}{rgb}{0.150000,0.150000,0.150000}%
\pgfsetstrokecolor{textcolor}%
\pgfsetfillcolor{textcolor}%
\pgftext[x=3.069018in, y=1.042623in, left, base]{\color{textcolor}\sffamily\fontsize{11.000000}{13.200000}\selectfont \(\displaystyle {\ensuremath{-}2}\)}%
\end{pgfscope}%
\begin{pgfscope}%
\pgfpathrectangle{\pgfqpoint{3.395292in}{0.650389in}}{\pgfqpoint{2.424708in}{1.800508in}}%
\pgfusepath{clip}%
\pgfsetroundcap%
\pgfsetroundjoin%
\pgfsetlinewidth{1.606000pt}%
\definecolor{currentstroke}{rgb}{1.000000,1.000000,1.000000}%
\pgfsetstrokecolor{currentstroke}%
\pgfsetdash{}{0pt}%
\pgfpathmoveto{\pgfqpoint{3.395292in}{1.550643in}}%
\pgfpathlineto{\pgfqpoint{5.820000in}{1.550643in}}%
\pgfusepath{stroke}%
\end{pgfscope}%
\begin{pgfscope}%
\definecolor{textcolor}{rgb}{0.150000,0.150000,0.150000}%
\pgfsetstrokecolor{textcolor}%
\pgfsetfillcolor{textcolor}%
\pgftext[x=3.187306in, y=1.497629in, left, base]{\color{textcolor}\sffamily\fontsize{11.000000}{13.200000}\selectfont \(\displaystyle {0}\)}%
\end{pgfscope}%
\begin{pgfscope}%
\pgfpathrectangle{\pgfqpoint{3.395292in}{0.650389in}}{\pgfqpoint{2.424708in}{1.800508in}}%
\pgfusepath{clip}%
\pgfsetroundcap%
\pgfsetroundjoin%
\pgfsetlinewidth{1.606000pt}%
\definecolor{currentstroke}{rgb}{1.000000,1.000000,1.000000}%
\pgfsetstrokecolor{currentstroke}%
\pgfsetdash{}{0pt}%
\pgfpathmoveto{\pgfqpoint{3.395292in}{2.005649in}}%
\pgfpathlineto{\pgfqpoint{5.820000in}{2.005649in}}%
\pgfusepath{stroke}%
\end{pgfscope}%
\begin{pgfscope}%
\definecolor{textcolor}{rgb}{0.150000,0.150000,0.150000}%
\pgfsetstrokecolor{textcolor}%
\pgfsetfillcolor{textcolor}%
\pgftext[x=3.187306in, y=1.952635in, left, base]{\color{textcolor}\sffamily\fontsize{11.000000}{13.200000}\selectfont \(\displaystyle {2}\)}%
\end{pgfscope}%
\begin{pgfscope}%
\pgfpathrectangle{\pgfqpoint{3.395292in}{0.650389in}}{\pgfqpoint{2.424708in}{1.800508in}}%
\pgfusepath{clip}%
\pgfsetroundcap%
\pgfsetroundjoin%
\pgfsetlinewidth{0.702625pt}%
\definecolor{currentstroke}{rgb}{1.000000,1.000000,1.000000}%
\pgfsetstrokecolor{currentstroke}%
\pgfsetdash{}{0pt}%
\pgfpathmoveto{\pgfqpoint{3.395292in}{0.868134in}}%
\pgfpathlineto{\pgfqpoint{5.820000in}{0.868134in}}%
\pgfusepath{stroke}%
\end{pgfscope}%
\begin{pgfscope}%
\pgfpathrectangle{\pgfqpoint{3.395292in}{0.650389in}}{\pgfqpoint{2.424708in}{1.800508in}}%
\pgfusepath{clip}%
\pgfsetroundcap%
\pgfsetroundjoin%
\pgfsetlinewidth{0.702625pt}%
\definecolor{currentstroke}{rgb}{1.000000,1.000000,1.000000}%
\pgfsetstrokecolor{currentstroke}%
\pgfsetdash{}{0pt}%
\pgfpathmoveto{\pgfqpoint{3.395292in}{1.323140in}}%
\pgfpathlineto{\pgfqpoint{5.820000in}{1.323140in}}%
\pgfusepath{stroke}%
\end{pgfscope}%
\begin{pgfscope}%
\pgfpathrectangle{\pgfqpoint{3.395292in}{0.650389in}}{\pgfqpoint{2.424708in}{1.800508in}}%
\pgfusepath{clip}%
\pgfsetroundcap%
\pgfsetroundjoin%
\pgfsetlinewidth{0.702625pt}%
\definecolor{currentstroke}{rgb}{1.000000,1.000000,1.000000}%
\pgfsetstrokecolor{currentstroke}%
\pgfsetdash{}{0pt}%
\pgfpathmoveto{\pgfqpoint{3.395292in}{1.778146in}}%
\pgfpathlineto{\pgfqpoint{5.820000in}{1.778146in}}%
\pgfusepath{stroke}%
\end{pgfscope}%
\begin{pgfscope}%
\pgfpathrectangle{\pgfqpoint{3.395292in}{0.650389in}}{\pgfqpoint{2.424708in}{1.800508in}}%
\pgfusepath{clip}%
\pgfsetroundcap%
\pgfsetroundjoin%
\pgfsetlinewidth{0.702625pt}%
\definecolor{currentstroke}{rgb}{1.000000,1.000000,1.000000}%
\pgfsetstrokecolor{currentstroke}%
\pgfsetdash{}{0pt}%
\pgfpathmoveto{\pgfqpoint{3.395292in}{2.233153in}}%
\pgfpathlineto{\pgfqpoint{5.820000in}{2.233153in}}%
\pgfusepath{stroke}%
\end{pgfscope}%
\begin{pgfscope}%
\pgfpathrectangle{\pgfqpoint{3.395292in}{0.650389in}}{\pgfqpoint{2.424708in}{1.800508in}}%
\pgfusepath{clip}%
\pgfsetroundcap%
\pgfsetroundjoin%
\pgfsetlinewidth{1.505625pt}%
\definecolor{currentstroke}{rgb}{0.917647,0.262745,0.207843}%
\pgfsetstrokecolor{currentstroke}%
\pgfsetdash{}{0pt}%
\pgfpathmoveto{\pgfqpoint{3.505506in}{0.732230in}}%
\pgfpathlineto{\pgfqpoint{3.680294in}{0.753783in}}%
\pgfpathmoveto{\pgfqpoint{3.694860in}{0.983083in}}%
\pgfpathlineto{\pgfqpoint{4.044437in}{1.026188in}}%
\pgfpathmoveto{\pgfqpoint{4.063859in}{1.256086in}}%
\pgfpathlineto{\pgfqpoint{4.413436in}{1.299192in}}%
\pgfpathmoveto{\pgfqpoint{4.432857in}{1.529090in}}%
\pgfpathlineto{\pgfqpoint{4.782434in}{1.572196in}}%
\pgfpathmoveto{\pgfqpoint{4.801856in}{1.802094in}}%
\pgfpathlineto{\pgfqpoint{5.151433in}{1.845200in}}%
\pgfpathmoveto{\pgfqpoint{5.170855in}{2.075098in}}%
\pgfpathlineto{\pgfqpoint{5.520431in}{2.118204in}}%
\pgfpathmoveto{\pgfqpoint{5.534998in}{2.347503in}}%
\pgfpathlineto{\pgfqpoint{5.709786in}{2.369056in}}%
\pgfpathlineto{\pgfqpoint{5.709786in}{2.369056in}}%
\pgfusepath{stroke}%
\end{pgfscope}%
\begin{pgfscope}%
\pgfsetrectcap%
\pgfsetmiterjoin%
\pgfsetlinewidth{1.254687pt}%
\definecolor{currentstroke}{rgb}{1.000000,1.000000,1.000000}%
\pgfsetstrokecolor{currentstroke}%
\pgfsetdash{}{0pt}%
\pgfpathmoveto{\pgfqpoint{3.395292in}{0.650389in}}%
\pgfpathlineto{\pgfqpoint{3.395292in}{2.450897in}}%
\pgfusepath{stroke}%
\end{pgfscope}%
\begin{pgfscope}%
\pgfsetrectcap%
\pgfsetmiterjoin%
\pgfsetlinewidth{1.254687pt}%
\definecolor{currentstroke}{rgb}{1.000000,1.000000,1.000000}%
\pgfsetstrokecolor{currentstroke}%
\pgfsetdash{}{0pt}%
\pgfpathmoveto{\pgfqpoint{5.820000in}{0.650389in}}%
\pgfpathlineto{\pgfqpoint{5.820000in}{2.450897in}}%
\pgfusepath{stroke}%
\end{pgfscope}%
\begin{pgfscope}%
\pgfsetrectcap%
\pgfsetmiterjoin%
\pgfsetlinewidth{1.254687pt}%
\definecolor{currentstroke}{rgb}{1.000000,1.000000,1.000000}%
\pgfsetstrokecolor{currentstroke}%
\pgfsetdash{}{0pt}%
\pgfpathmoveto{\pgfqpoint{3.395292in}{0.650389in}}%
\pgfpathlineto{\pgfqpoint{5.820000in}{0.650389in}}%
\pgfusepath{stroke}%
\end{pgfscope}%
\begin{pgfscope}%
\pgfsetrectcap%
\pgfsetmiterjoin%
\pgfsetlinewidth{1.254687pt}%
\definecolor{currentstroke}{rgb}{1.000000,1.000000,1.000000}%
\pgfsetstrokecolor{currentstroke}%
\pgfsetdash{}{0pt}%
\pgfpathmoveto{\pgfqpoint{3.395292in}{2.450897in}}%
\pgfpathlineto{\pgfqpoint{5.820000in}{2.450897in}}%
\pgfusepath{stroke}%
\end{pgfscope}%
\begin{pgfscope}%
\definecolor{textcolor}{rgb}{0.150000,0.150000,0.150000}%
\pgfsetstrokecolor{textcolor}%
\pgfsetfillcolor{textcolor}%
\pgftext[x=4.607646in,y=2.642261in,,base]{\color{textcolor}\sffamily\fontsize{12.000000}{14.400000}\selectfont First Derivative: \(\displaystyle \frac{\partial f}{\partial x} \big|_x\)}%
\end{pgfscope}%
\end{pgfpicture}%
\makeatother%
\endgroup%

    \caption{Illustration of the function in Eq. \ref{eq:nondiff}, which exhibits an infinite number of discontinuities in value and derivative.}
    \label{fig:nondiff}
\end{figure}
