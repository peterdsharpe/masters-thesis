\chapter{Modeling Tools}
\label{chapter:modeling}


\section{Geometry Stack}

To be written\dots % TODO


\section{Surrogate Modeling Tools}
\label{sect:surrogate}

It is critical that a general-purpose engineering design optimization framework such as AeroSandbox allows the use of user-defined physics models. These models can generally be classified into one of three categories:

\begin{enumerate}[noitemsep]
    \item \textbf{Analytical models}, which are often derived theoretically and can be specified concisely in closed-form. For example, an analytical relation for aircraft drag, as given by Drela \cite{drela-performance-lab} is:
    \begin{equation}
        C_D(\AR, S, \dots) = \frac{\text{CDA}_0}{S} + c_d(C_L, \text{Re}, \tau) + \frac{C_L^2}{\pi \AR}
        \label{eq:analytical-drag}
    \end{equation}
    \begin{multicols}{2}
        \begin{eqexpl}
            \item{$C_D$} Drag coefficient
            \item{$\AR$} Aspect ratio
            \item{$S$} Wing area
            \item{$\text{CDA}_0$} Fuselage drag area
            \item{$c_d$} Profile drag coefficient
            \item{$C_L$} Lift coefficient
            \item{Re} Reynolds number
            \item{$\tau$} Taper ratio
        \end{eqexpl}
    \end{multicols}

    \item \textbf{Expensive models} that are the result of laborious, black-box external computation. Examples of this might include:

    \begin{itemize}[noitemsep]

        \item A RANS CFD code, or other analysis that requires the solution of 3D nonlinear PDEs
        \item Aerostructural analysis with explicit time-domain dynamics
        \item An external code that computes engine performance with non-equilibrium gas dynamics

    \end{itemize}

    \item \textbf{Data-driven models}, where the underlying input to the model is not a set of equations but rather a dataset. Common sources of this data include wind tunnel runs, meteorological data, or experimental testing on prototype components.

\end{enumerate}

Analytical models are trivially implemented into a differentiable optimization framework such as AeroSandbox using the techniques described in Chapter \ref{chapter:core}. However, the remaining two categories cannot generally be used in a straightforward way.

In the case of expensive models, there are two key problems that make direct implementation unattractive. First, black-box functions break the differentiability trace, so they must be re-coded from scratch in a differentiable numerics framework. Although this is usually possible, it can be tedious. Secondly, expensive models in a SAND framework may yield an optimization problem that takes many minutes or hours to solve, precluding the practice of \textit{interactive design}.

Data-driven models also come with their own set of challenges. First, one must obtain data of sufficient quantity (it spans the input space of the model) and quality (the data has minimal noise). Secondly, one must form some kind of strategy to evaluate the model between known data points (interpolation) and beyond known data (extrapolation).

Both expensive models and data-driven models can be implemented into a differentiable optimization framework using \textit{surrogate modeling}. Surrogate modeling is a collection of techniques that aims to replace an expensive, data-driven, or otherwise unusable model with a differentiable, cheaply-evaluated model that approximates the original one.

There are two general approaches to surrogate modeling: \textit{fitting} and \textit{interpolation}. Fitting aims to replace an expensive model or dataset with an analytical expression. Interpolation forgoes the need for an explicit analytical expression, instead interpolating from known data points using a piecewise spline.

The present work provides computational tools for surrogate modeling using both of these approaches. In the following section, we detail both of these. We restrict our focus here to models of the form $f\ :\ \R^n \to \R^1$, because these are by far the most common types of models, and because vector-valued functions can be constructed by combining several scalar-valued functions.

\subsection{Fitted Models}
\label{sect:fitting}

One approach to creating a surrogate model is curve fitting, or more formally, \textit{regression}. Fitting is the process of deriving an analytical model that approximates a function from which samples have been drawn. An infinite number of fitted models could be regressed from a given dataset, and choosing which of these models is best is an optimization problem. Specifically, we can write the fitting problem as follows, following notation from \cite{koch2019}:

\begin{example}

    \noindent
    \textbf{The Canonical Fitting Problem} (Least-squares Regression)

    \noindent
    We are given a dataset that consists of $m$ entries. Each entry maps from $\vec{x} \in \R^n $ to $ y \in \R^1 $. We collectively refer to the inputs and outputs of the dataset as $\mat{X}$ and $\vec{y}$, respectively.

    A model format is also provided, which includes some unknown parameters $\vec{\theta}$. We then denote the model outputs as:

    \begin{equation}
        \vec{y}_\text{model} = \text{model}\big(\mat{X}, \vec{\theta}\big)
        \label{eq:fitting-model}
    \end{equation}

    \noindent
    The error of this model at each of the $m$ data points is then:

    \[ \vec{e} = \vec{y}_\text{model} - \vec{y} \]

    \noindent
    We then seek to find the optimal value of the model parameters $\vec{\theta}$ by solving the following optimization problem:

    \begin{argmini}
    {\vec{\theta}}{ \big\Vert \vec{e} \big\Vert _2 }
    {}{\vec{\theta}^* = }
        \label{eq:fitting}
    \end{argmini}

    The vector norm in the objective Eq. \ref{eq:fitting} can be rewritten as $ \big\Vert \vec{e} \big\Vert _2 = \sqrt{\sum_{i=1}^m e_i^2 } $. Because the square root function is monotonic in the positive domain, it can be removed without changing the value of $\vec{\theta}^*$. This is convenient, as the objective function now tends to be more closely approximated by a quadratic\footnote{In the case of \textit{linear} least-squares regression, the objective is now exactly quadratic and admits closed-form solution.}, dramatically improving numerical performance. Therefore, the fitting problem can be expressed as:

    \begin{argmini}
    {\vec{\theta}}{ \sum_{i=1}^m e_i^2 }
    {}{\vec{\theta}^* = }
        \label{eq:fitting-l2}
    \end{argmini}

\end{example}

This forms the canonical fitting problem, also known as least-squares regression. Because Equation \ref{eq:fitting} is an optimization problem with entirely glass-box functions, it is efficiently differentiated and solved by AeroSandbox. This functionality is provided by the \mintinline{python}{asb.FittedModel} class, which acts both as the fitting solver (performed upon instantiation) and the callable model itself.

The generality of the model format in Eq. \ref{eq:fitting-model} is quite powerful, as the fitting routine presented here can use any composition of elementary operators as its model format. In addition, this optimization approach can fit piecewise functions and expressions that can only be tractably expressed in code (e.g. model formats with loops, complicated conditionals). The AeroSandbox \mintinline{python}{FittedModel} implementation can also be used to fit datasets with general multidimensional inputs. Furthermore, because the fitting optimization problem can utilize fast gradients via automatic differentiation, fitting performance scales efficiently with parameter dimensionality\footnote{analogous to the scaling seen in Figure \ref{fig:nd-rosen}}.

\subsubsection{Example: Wind Analysis}

The power of this generality is demonstrated in Figure \ref{fig:fitting-wind}, where the AeroSandbox \mintinline{python}{FittedModel} routine is used to regress a model for peak\footnote{Quantified as the 99th-percentile of wind speed over time} wind speeds at various points in the atmosphere above the continental United States in August\footnote{Averaged over years 1979-2020}. This model holds great importance for high-altitude long-endurance (HALE) aircraft design: the primary failure mode of HALE aircraft in the past two decades has been in-flight aerostructural failure exacerbated by wind gusts. The underlying 2D dataset in this example was obtained via statistical analysis of the ERA5 Global Reanalysis meteorological dataset \cite{era5}.

\begin{figure}[H]
    \centering
%    \input{figures/fitting-wind.pgf}
    \ifdraft{}{\input{figures/fitting-wind.pgf}}
    \caption{An demonstration of \mintinline{python}{asb.FittedModel}, where an 18-parameter analytical model is fitted to a multidimensional, aerospace-relevant example dataset.}
    \label{fig:fitting-wind}
\end{figure}

There is a clear and strong nonlinearity present in this dataset, evidenced by the sharp rise in peak wind speeds near (15 km altitude, 50 deg. N latitude). This nonlinearity, which depicts the Arctic polar vortex of the jet stream, makes this a challenging dataset to fit. Here, an 18-parameter model consisting of a combination of polynomials and Gaussian-like terms was used to fit the data.

The resulting model from this fitting process is not only cheaply evaluated, but it is also end-to-end automatic differentiable. This means that it can be used as desired in the optimization framework described in Chapter \ref{chapter:core}. Fitting also tends to remove noise from the dataset, which makes optimization much more well behaved. This noise removal occurs because the fit is essentially a projection of the dataset noise (generally relatively high-spatial-frequency\footnote{In the common case of uncorrelated random error, the noise follows a \textit{white noise} spectrum}) onto the mode shapes associated with model linearization with respect to $\vec{\theta}$ (generally relatively low-spatial-frequency). Because of these desirable properties, curve fitting is a good tool for creating surrogate models from experimental or synthetic datasets.

The classical ordinary least-squares fitting problem described in Equation \ref{eq:fitting} can be extended in several interesting ways in order to obtain fits with more desirable characteristics. Several of these generalizations have been implemented into the AeroSandbox surrogate modeling toolkit via the \mintinline{python}{asb.FittedModel} class. These features are described in the following sections.

\subsubsection{Generalization to Various $L_p$ Norms}

The fitting problem specified in Eq. \ref{eq:fitting} minimizes the $L_2$ norm of the error vector $\vec{e}$, a process known as least-squares fitting. This fitting problem can be generalized by instead minimizing various $L_p$ norms of the error vector. In general, the $L_p$ norm of the error vector can be expressed as:

\begin{equation}
    \big\Vert \vec{e} \big\Vert _p = \bigg( \sum_{i=1}^m e_i^p \bigg)^{1/p}, \qquad p \in [1, \infty)
    \label{eq:norms}
\end{equation}

\noindent
By analogy to Eq. \ref{eq:fitting-l2}, the fit optimization problem associated with this equation is often solved more easily after elimination of the (monotonic) root function.

Of particular interest are the $L_1$ and $L_\infty$ norms, which can be expressed in special form derived from limit analysis of Eq. \ref{eq:norms}:

\begin{equation*}
    \big\Vert \vec{e} \big\Vert _1      = \sum_{i=1}^m |e_i|
    \qquad\qquad
    \big\Vert \vec{e} \big\Vert _\infty = \max\big(|e_1|, |e_2|, \dots, |e_m|\big)
\end{equation*}

\noindent
These norms are of special interest for two reasons. First, they represent the extremes of the $L_p$ norm family. Secondly, the fitting problem associated with them can be more efficiently expressed with a reasonable\footnote{more precisely, a finite number of constraints that is of $\order(m)$} number of constraints\footnote{and in particular, \textit{linear} constraints in the case of linear regression}, as shown by analogy to Figure \ref{fig:opti-norms}.

The primary practical distinction between fits using these various norms is their response to outliers. This is demonstrated in Figure \ref{fig:fitting-norm}, where fits are made to a synthetic example dataset that contains an outlier. The $L_1$ fit largely eschews the influence of the outlier, essentially discarding the outlier as a \textit{systematic} error rather than a \textit{random} one. The $L_\infty$ fit is the opposite, as it seeks to minimize the maximum deviation; essentially this treats the outlier as a random error that still conveys useful information about the underlying model.

\begin{figure}[H]
    \centering
%    %% Creator: Matplotlib, PGF backend
%%
%% To include the figure in your LaTeX document, write
%%   \input{<filename>.pgf}
%%
%% Make sure the required packages are loaded in your preamble
%%   \usepackage{pgf}
%%
%% Figures using additional raster images can only be included by \input if
%% they are in the same directory as the main LaTeX file. For loading figures
%% from other directories you can use the `import` package
%%   \usepackage{import}
%%
%% and then include the figures with
%%   \import{<path to file>}{<filename>.pgf}
%%
%% Matplotlib used the following preamble
%%   \usepackage{fontspec}
%%
\begingroup%
\makeatletter%
\begin{pgfpicture}%
\pgfpathrectangle{\pgfpointorigin}{\pgfqpoint{5.000000in}{4.000000in}}%
\pgfusepath{use as bounding box, clip}%
\begin{pgfscope}%
\pgfsetbuttcap%
\pgfsetmiterjoin%
\definecolor{currentfill}{rgb}{1.000000,1.000000,1.000000}%
\pgfsetfillcolor{currentfill}%
\pgfsetlinewidth{0.000000pt}%
\definecolor{currentstroke}{rgb}{1.000000,1.000000,1.000000}%
\pgfsetstrokecolor{currentstroke}%
\pgfsetdash{}{0pt}%
\pgfpathmoveto{\pgfqpoint{0.000000in}{0.000000in}}%
\pgfpathlineto{\pgfqpoint{5.000000in}{0.000000in}}%
\pgfpathlineto{\pgfqpoint{5.000000in}{4.000000in}}%
\pgfpathlineto{\pgfqpoint{0.000000in}{4.000000in}}%
\pgfpathclose%
\pgfusepath{fill}%
\end{pgfscope}%
\begin{pgfscope}%
\pgfsetbuttcap%
\pgfsetmiterjoin%
\definecolor{currentfill}{rgb}{0.917647,0.917647,0.949020}%
\pgfsetfillcolor{currentfill}%
\pgfsetlinewidth{0.000000pt}%
\definecolor{currentstroke}{rgb}{0.000000,0.000000,0.000000}%
\pgfsetstrokecolor{currentstroke}%
\pgfsetstrokeopacity{0.000000}%
\pgfsetdash{}{0pt}%
\pgfpathmoveto{\pgfqpoint{0.750000in}{0.520000in}}%
\pgfpathlineto{\pgfqpoint{4.900000in}{0.520000in}}%
\pgfpathlineto{\pgfqpoint{4.900000in}{3.720000in}}%
\pgfpathlineto{\pgfqpoint{0.750000in}{3.720000in}}%
\pgfpathclose%
\pgfusepath{fill}%
\end{pgfscope}%
\begin{pgfscope}%
\pgfpathrectangle{\pgfqpoint{0.750000in}{0.520000in}}{\pgfqpoint{4.150000in}{3.200000in}}%
\pgfusepath{clip}%
\pgfsetroundcap%
\pgfsetroundjoin%
\pgfsetlinewidth{1.606000pt}%
\definecolor{currentstroke}{rgb}{1.000000,1.000000,1.000000}%
\pgfsetstrokecolor{currentstroke}%
\pgfsetdash{}{0pt}%
\pgfpathmoveto{\pgfqpoint{0.750000in}{0.520000in}}%
\pgfpathlineto{\pgfqpoint{0.750000in}{3.720000in}}%
\pgfusepath{stroke}%
\end{pgfscope}%
\begin{pgfscope}%
\definecolor{textcolor}{rgb}{0.150000,0.150000,0.150000}%
\pgfsetstrokecolor{textcolor}%
\pgfsetfillcolor{textcolor}%
\pgftext[x=0.750000in,y=0.388056in,,top]{\color{textcolor}\sffamily\fontsize{11.000000}{13.200000}\selectfont 0}%
\end{pgfscope}%
\begin{pgfscope}%
\pgfpathrectangle{\pgfqpoint{0.750000in}{0.520000in}}{\pgfqpoint{4.150000in}{3.200000in}}%
\pgfusepath{clip}%
\pgfsetroundcap%
\pgfsetroundjoin%
\pgfsetlinewidth{1.606000pt}%
\definecolor{currentstroke}{rgb}{1.000000,1.000000,1.000000}%
\pgfsetstrokecolor{currentstroke}%
\pgfsetdash{}{0pt}%
\pgfpathmoveto{\pgfqpoint{1.165000in}{0.520000in}}%
\pgfpathlineto{\pgfqpoint{1.165000in}{3.720000in}}%
\pgfusepath{stroke}%
\end{pgfscope}%
\begin{pgfscope}%
\definecolor{textcolor}{rgb}{0.150000,0.150000,0.150000}%
\pgfsetstrokecolor{textcolor}%
\pgfsetfillcolor{textcolor}%
\pgftext[x=1.165000in,y=0.388056in,,top]{\color{textcolor}\sffamily\fontsize{11.000000}{13.200000}\selectfont 10}%
\end{pgfscope}%
\begin{pgfscope}%
\pgfpathrectangle{\pgfqpoint{0.750000in}{0.520000in}}{\pgfqpoint{4.150000in}{3.200000in}}%
\pgfusepath{clip}%
\pgfsetroundcap%
\pgfsetroundjoin%
\pgfsetlinewidth{1.606000pt}%
\definecolor{currentstroke}{rgb}{1.000000,1.000000,1.000000}%
\pgfsetstrokecolor{currentstroke}%
\pgfsetdash{}{0pt}%
\pgfpathmoveto{\pgfqpoint{1.580000in}{0.520000in}}%
\pgfpathlineto{\pgfqpoint{1.580000in}{3.720000in}}%
\pgfusepath{stroke}%
\end{pgfscope}%
\begin{pgfscope}%
\definecolor{textcolor}{rgb}{0.150000,0.150000,0.150000}%
\pgfsetstrokecolor{textcolor}%
\pgfsetfillcolor{textcolor}%
\pgftext[x=1.580000in,y=0.388056in,,top]{\color{textcolor}\sffamily\fontsize{11.000000}{13.200000}\selectfont 20}%
\end{pgfscope}%
\begin{pgfscope}%
\pgfpathrectangle{\pgfqpoint{0.750000in}{0.520000in}}{\pgfqpoint{4.150000in}{3.200000in}}%
\pgfusepath{clip}%
\pgfsetroundcap%
\pgfsetroundjoin%
\pgfsetlinewidth{1.606000pt}%
\definecolor{currentstroke}{rgb}{1.000000,1.000000,1.000000}%
\pgfsetstrokecolor{currentstroke}%
\pgfsetdash{}{0pt}%
\pgfpathmoveto{\pgfqpoint{1.995000in}{0.520000in}}%
\pgfpathlineto{\pgfqpoint{1.995000in}{3.720000in}}%
\pgfusepath{stroke}%
\end{pgfscope}%
\begin{pgfscope}%
\definecolor{textcolor}{rgb}{0.150000,0.150000,0.150000}%
\pgfsetstrokecolor{textcolor}%
\pgfsetfillcolor{textcolor}%
\pgftext[x=1.995000in,y=0.388056in,,top]{\color{textcolor}\sffamily\fontsize{11.000000}{13.200000}\selectfont 30}%
\end{pgfscope}%
\begin{pgfscope}%
\pgfpathrectangle{\pgfqpoint{0.750000in}{0.520000in}}{\pgfqpoint{4.150000in}{3.200000in}}%
\pgfusepath{clip}%
\pgfsetroundcap%
\pgfsetroundjoin%
\pgfsetlinewidth{1.606000pt}%
\definecolor{currentstroke}{rgb}{1.000000,1.000000,1.000000}%
\pgfsetstrokecolor{currentstroke}%
\pgfsetdash{}{0pt}%
\pgfpathmoveto{\pgfqpoint{2.410000in}{0.520000in}}%
\pgfpathlineto{\pgfqpoint{2.410000in}{3.720000in}}%
\pgfusepath{stroke}%
\end{pgfscope}%
\begin{pgfscope}%
\definecolor{textcolor}{rgb}{0.150000,0.150000,0.150000}%
\pgfsetstrokecolor{textcolor}%
\pgfsetfillcolor{textcolor}%
\pgftext[x=2.410000in,y=0.388056in,,top]{\color{textcolor}\sffamily\fontsize{11.000000}{13.200000}\selectfont 40}%
\end{pgfscope}%
\begin{pgfscope}%
\pgfpathrectangle{\pgfqpoint{0.750000in}{0.520000in}}{\pgfqpoint{4.150000in}{3.200000in}}%
\pgfusepath{clip}%
\pgfsetroundcap%
\pgfsetroundjoin%
\pgfsetlinewidth{1.606000pt}%
\definecolor{currentstroke}{rgb}{1.000000,1.000000,1.000000}%
\pgfsetstrokecolor{currentstroke}%
\pgfsetdash{}{0pt}%
\pgfpathmoveto{\pgfqpoint{2.825000in}{0.520000in}}%
\pgfpathlineto{\pgfqpoint{2.825000in}{3.720000in}}%
\pgfusepath{stroke}%
\end{pgfscope}%
\begin{pgfscope}%
\definecolor{textcolor}{rgb}{0.150000,0.150000,0.150000}%
\pgfsetstrokecolor{textcolor}%
\pgfsetfillcolor{textcolor}%
\pgftext[x=2.825000in,y=0.388056in,,top]{\color{textcolor}\sffamily\fontsize{11.000000}{13.200000}\selectfont 50}%
\end{pgfscope}%
\begin{pgfscope}%
\pgfpathrectangle{\pgfqpoint{0.750000in}{0.520000in}}{\pgfqpoint{4.150000in}{3.200000in}}%
\pgfusepath{clip}%
\pgfsetroundcap%
\pgfsetroundjoin%
\pgfsetlinewidth{1.606000pt}%
\definecolor{currentstroke}{rgb}{1.000000,1.000000,1.000000}%
\pgfsetstrokecolor{currentstroke}%
\pgfsetdash{}{0pt}%
\pgfpathmoveto{\pgfqpoint{3.240000in}{0.520000in}}%
\pgfpathlineto{\pgfqpoint{3.240000in}{3.720000in}}%
\pgfusepath{stroke}%
\end{pgfscope}%
\begin{pgfscope}%
\definecolor{textcolor}{rgb}{0.150000,0.150000,0.150000}%
\pgfsetstrokecolor{textcolor}%
\pgfsetfillcolor{textcolor}%
\pgftext[x=3.240000in,y=0.388056in,,top]{\color{textcolor}\sffamily\fontsize{11.000000}{13.200000}\selectfont 60}%
\end{pgfscope}%
\begin{pgfscope}%
\pgfpathrectangle{\pgfqpoint{0.750000in}{0.520000in}}{\pgfqpoint{4.150000in}{3.200000in}}%
\pgfusepath{clip}%
\pgfsetroundcap%
\pgfsetroundjoin%
\pgfsetlinewidth{1.606000pt}%
\definecolor{currentstroke}{rgb}{1.000000,1.000000,1.000000}%
\pgfsetstrokecolor{currentstroke}%
\pgfsetdash{}{0pt}%
\pgfpathmoveto{\pgfqpoint{3.655000in}{0.520000in}}%
\pgfpathlineto{\pgfqpoint{3.655000in}{3.720000in}}%
\pgfusepath{stroke}%
\end{pgfscope}%
\begin{pgfscope}%
\definecolor{textcolor}{rgb}{0.150000,0.150000,0.150000}%
\pgfsetstrokecolor{textcolor}%
\pgfsetfillcolor{textcolor}%
\pgftext[x=3.655000in,y=0.388056in,,top]{\color{textcolor}\sffamily\fontsize{11.000000}{13.200000}\selectfont 70}%
\end{pgfscope}%
\begin{pgfscope}%
\pgfpathrectangle{\pgfqpoint{0.750000in}{0.520000in}}{\pgfqpoint{4.150000in}{3.200000in}}%
\pgfusepath{clip}%
\pgfsetroundcap%
\pgfsetroundjoin%
\pgfsetlinewidth{1.606000pt}%
\definecolor{currentstroke}{rgb}{1.000000,1.000000,1.000000}%
\pgfsetstrokecolor{currentstroke}%
\pgfsetdash{}{0pt}%
\pgfpathmoveto{\pgfqpoint{4.070000in}{0.520000in}}%
\pgfpathlineto{\pgfqpoint{4.070000in}{3.720000in}}%
\pgfusepath{stroke}%
\end{pgfscope}%
\begin{pgfscope}%
\definecolor{textcolor}{rgb}{0.150000,0.150000,0.150000}%
\pgfsetstrokecolor{textcolor}%
\pgfsetfillcolor{textcolor}%
\pgftext[x=4.070000in,y=0.388056in,,top]{\color{textcolor}\sffamily\fontsize{11.000000}{13.200000}\selectfont 80}%
\end{pgfscope}%
\begin{pgfscope}%
\pgfpathrectangle{\pgfqpoint{0.750000in}{0.520000in}}{\pgfqpoint{4.150000in}{3.200000in}}%
\pgfusepath{clip}%
\pgfsetroundcap%
\pgfsetroundjoin%
\pgfsetlinewidth{1.606000pt}%
\definecolor{currentstroke}{rgb}{1.000000,1.000000,1.000000}%
\pgfsetstrokecolor{currentstroke}%
\pgfsetdash{}{0pt}%
\pgfpathmoveto{\pgfqpoint{4.485000in}{0.520000in}}%
\pgfpathlineto{\pgfqpoint{4.485000in}{3.720000in}}%
\pgfusepath{stroke}%
\end{pgfscope}%
\begin{pgfscope}%
\definecolor{textcolor}{rgb}{0.150000,0.150000,0.150000}%
\pgfsetstrokecolor{textcolor}%
\pgfsetfillcolor{textcolor}%
\pgftext[x=4.485000in,y=0.388056in,,top]{\color{textcolor}\sffamily\fontsize{11.000000}{13.200000}\selectfont 90}%
\end{pgfscope}%
\begin{pgfscope}%
\pgfpathrectangle{\pgfqpoint{0.750000in}{0.520000in}}{\pgfqpoint{4.150000in}{3.200000in}}%
\pgfusepath{clip}%
\pgfsetroundcap%
\pgfsetroundjoin%
\pgfsetlinewidth{1.606000pt}%
\definecolor{currentstroke}{rgb}{1.000000,1.000000,1.000000}%
\pgfsetstrokecolor{currentstroke}%
\pgfsetdash{}{0pt}%
\pgfpathmoveto{\pgfqpoint{4.900000in}{0.520000in}}%
\pgfpathlineto{\pgfqpoint{4.900000in}{3.720000in}}%
\pgfusepath{stroke}%
\end{pgfscope}%
\begin{pgfscope}%
\definecolor{textcolor}{rgb}{0.150000,0.150000,0.150000}%
\pgfsetstrokecolor{textcolor}%
\pgfsetfillcolor{textcolor}%
\pgftext[x=4.900000in,y=0.388056in,,top]{\color{textcolor}\sffamily\fontsize{11.000000}{13.200000}\selectfont 100}%
\end{pgfscope}%
\begin{pgfscope}%
\pgfpathrectangle{\pgfqpoint{0.750000in}{0.520000in}}{\pgfqpoint{4.150000in}{3.200000in}}%
\pgfusepath{clip}%
\pgfsetroundcap%
\pgfsetroundjoin%
\pgfsetlinewidth{0.702625pt}%
\definecolor{currentstroke}{rgb}{1.000000,1.000000,1.000000}%
\pgfsetstrokecolor{currentstroke}%
\pgfsetdash{}{0pt}%
\pgfpathmoveto{\pgfqpoint{0.957500in}{0.520000in}}%
\pgfpathlineto{\pgfqpoint{0.957500in}{3.720000in}}%
\pgfusepath{stroke}%
\end{pgfscope}%
\begin{pgfscope}%
\pgfpathrectangle{\pgfqpoint{0.750000in}{0.520000in}}{\pgfqpoint{4.150000in}{3.200000in}}%
\pgfusepath{clip}%
\pgfsetroundcap%
\pgfsetroundjoin%
\pgfsetlinewidth{0.702625pt}%
\definecolor{currentstroke}{rgb}{1.000000,1.000000,1.000000}%
\pgfsetstrokecolor{currentstroke}%
\pgfsetdash{}{0pt}%
\pgfpathmoveto{\pgfqpoint{1.372500in}{0.520000in}}%
\pgfpathlineto{\pgfqpoint{1.372500in}{3.720000in}}%
\pgfusepath{stroke}%
\end{pgfscope}%
\begin{pgfscope}%
\pgfpathrectangle{\pgfqpoint{0.750000in}{0.520000in}}{\pgfqpoint{4.150000in}{3.200000in}}%
\pgfusepath{clip}%
\pgfsetroundcap%
\pgfsetroundjoin%
\pgfsetlinewidth{0.702625pt}%
\definecolor{currentstroke}{rgb}{1.000000,1.000000,1.000000}%
\pgfsetstrokecolor{currentstroke}%
\pgfsetdash{}{0pt}%
\pgfpathmoveto{\pgfqpoint{1.787500in}{0.520000in}}%
\pgfpathlineto{\pgfqpoint{1.787500in}{3.720000in}}%
\pgfusepath{stroke}%
\end{pgfscope}%
\begin{pgfscope}%
\pgfpathrectangle{\pgfqpoint{0.750000in}{0.520000in}}{\pgfqpoint{4.150000in}{3.200000in}}%
\pgfusepath{clip}%
\pgfsetroundcap%
\pgfsetroundjoin%
\pgfsetlinewidth{0.702625pt}%
\definecolor{currentstroke}{rgb}{1.000000,1.000000,1.000000}%
\pgfsetstrokecolor{currentstroke}%
\pgfsetdash{}{0pt}%
\pgfpathmoveto{\pgfqpoint{2.202500in}{0.520000in}}%
\pgfpathlineto{\pgfqpoint{2.202500in}{3.720000in}}%
\pgfusepath{stroke}%
\end{pgfscope}%
\begin{pgfscope}%
\pgfpathrectangle{\pgfqpoint{0.750000in}{0.520000in}}{\pgfqpoint{4.150000in}{3.200000in}}%
\pgfusepath{clip}%
\pgfsetroundcap%
\pgfsetroundjoin%
\pgfsetlinewidth{0.702625pt}%
\definecolor{currentstroke}{rgb}{1.000000,1.000000,1.000000}%
\pgfsetstrokecolor{currentstroke}%
\pgfsetdash{}{0pt}%
\pgfpathmoveto{\pgfqpoint{2.617500in}{0.520000in}}%
\pgfpathlineto{\pgfqpoint{2.617500in}{3.720000in}}%
\pgfusepath{stroke}%
\end{pgfscope}%
\begin{pgfscope}%
\pgfpathrectangle{\pgfqpoint{0.750000in}{0.520000in}}{\pgfqpoint{4.150000in}{3.200000in}}%
\pgfusepath{clip}%
\pgfsetroundcap%
\pgfsetroundjoin%
\pgfsetlinewidth{0.702625pt}%
\definecolor{currentstroke}{rgb}{1.000000,1.000000,1.000000}%
\pgfsetstrokecolor{currentstroke}%
\pgfsetdash{}{0pt}%
\pgfpathmoveto{\pgfqpoint{3.032500in}{0.520000in}}%
\pgfpathlineto{\pgfqpoint{3.032500in}{3.720000in}}%
\pgfusepath{stroke}%
\end{pgfscope}%
\begin{pgfscope}%
\pgfpathrectangle{\pgfqpoint{0.750000in}{0.520000in}}{\pgfqpoint{4.150000in}{3.200000in}}%
\pgfusepath{clip}%
\pgfsetroundcap%
\pgfsetroundjoin%
\pgfsetlinewidth{0.702625pt}%
\definecolor{currentstroke}{rgb}{1.000000,1.000000,1.000000}%
\pgfsetstrokecolor{currentstroke}%
\pgfsetdash{}{0pt}%
\pgfpathmoveto{\pgfqpoint{3.447500in}{0.520000in}}%
\pgfpathlineto{\pgfqpoint{3.447500in}{3.720000in}}%
\pgfusepath{stroke}%
\end{pgfscope}%
\begin{pgfscope}%
\pgfpathrectangle{\pgfqpoint{0.750000in}{0.520000in}}{\pgfqpoint{4.150000in}{3.200000in}}%
\pgfusepath{clip}%
\pgfsetroundcap%
\pgfsetroundjoin%
\pgfsetlinewidth{0.702625pt}%
\definecolor{currentstroke}{rgb}{1.000000,1.000000,1.000000}%
\pgfsetstrokecolor{currentstroke}%
\pgfsetdash{}{0pt}%
\pgfpathmoveto{\pgfqpoint{3.862500in}{0.520000in}}%
\pgfpathlineto{\pgfqpoint{3.862500in}{3.720000in}}%
\pgfusepath{stroke}%
\end{pgfscope}%
\begin{pgfscope}%
\pgfpathrectangle{\pgfqpoint{0.750000in}{0.520000in}}{\pgfqpoint{4.150000in}{3.200000in}}%
\pgfusepath{clip}%
\pgfsetroundcap%
\pgfsetroundjoin%
\pgfsetlinewidth{0.702625pt}%
\definecolor{currentstroke}{rgb}{1.000000,1.000000,1.000000}%
\pgfsetstrokecolor{currentstroke}%
\pgfsetdash{}{0pt}%
\pgfpathmoveto{\pgfqpoint{4.277500in}{0.520000in}}%
\pgfpathlineto{\pgfqpoint{4.277500in}{3.720000in}}%
\pgfusepath{stroke}%
\end{pgfscope}%
\begin{pgfscope}%
\pgfpathrectangle{\pgfqpoint{0.750000in}{0.520000in}}{\pgfqpoint{4.150000in}{3.200000in}}%
\pgfusepath{clip}%
\pgfsetroundcap%
\pgfsetroundjoin%
\pgfsetlinewidth{0.702625pt}%
\definecolor{currentstroke}{rgb}{1.000000,1.000000,1.000000}%
\pgfsetstrokecolor{currentstroke}%
\pgfsetdash{}{0pt}%
\pgfpathmoveto{\pgfqpoint{4.692500in}{0.520000in}}%
\pgfpathlineto{\pgfqpoint{4.692500in}{3.720000in}}%
\pgfusepath{stroke}%
\end{pgfscope}%
\begin{pgfscope}%
\definecolor{textcolor}{rgb}{0.150000,0.150000,0.150000}%
\pgfsetstrokecolor{textcolor}%
\pgfsetfillcolor{textcolor}%
\pgftext[x=2.825000in,y=0.196833in,,top]{\color{textcolor}\sffamily\fontsize{12.000000}{14.400000}\selectfont \(\displaystyle x\)}%
\end{pgfscope}%
\begin{pgfscope}%
\pgfpathrectangle{\pgfqpoint{0.750000in}{0.520000in}}{\pgfqpoint{4.150000in}{3.200000in}}%
\pgfusepath{clip}%
\pgfsetroundcap%
\pgfsetroundjoin%
\pgfsetlinewidth{1.606000pt}%
\definecolor{currentstroke}{rgb}{1.000000,1.000000,1.000000}%
\pgfsetstrokecolor{currentstroke}%
\pgfsetdash{}{0pt}%
\pgfpathmoveto{\pgfqpoint{0.750000in}{0.520000in}}%
\pgfpathlineto{\pgfqpoint{4.900000in}{0.520000in}}%
\pgfusepath{stroke}%
\end{pgfscope}%
\begin{pgfscope}%
\definecolor{textcolor}{rgb}{0.150000,0.150000,0.150000}%
\pgfsetstrokecolor{textcolor}%
\pgfsetfillcolor{textcolor}%
\pgftext[x=0.275185in, y=0.466986in, left, base]{\color{textcolor}\sffamily\fontsize{11.000000}{13.200000}\selectfont \ensuremath{-}100}%
\end{pgfscope}%
\begin{pgfscope}%
\pgfpathrectangle{\pgfqpoint{0.750000in}{0.520000in}}{\pgfqpoint{4.150000in}{3.200000in}}%
\pgfusepath{clip}%
\pgfsetroundcap%
\pgfsetroundjoin%
\pgfsetlinewidth{1.606000pt}%
\definecolor{currentstroke}{rgb}{1.000000,1.000000,1.000000}%
\pgfsetstrokecolor{currentstroke}%
\pgfsetdash{}{0pt}%
\pgfpathmoveto{\pgfqpoint{0.750000in}{0.977143in}}%
\pgfpathlineto{\pgfqpoint{4.900000in}{0.977143in}}%
\pgfusepath{stroke}%
\end{pgfscope}%
\begin{pgfscope}%
\definecolor{textcolor}{rgb}{0.150000,0.150000,0.150000}%
\pgfsetstrokecolor{textcolor}%
\pgfsetfillcolor{textcolor}%
\pgftext[x=0.350046in, y=0.924129in, left, base]{\color{textcolor}\sffamily\fontsize{11.000000}{13.200000}\selectfont \ensuremath{-}50}%
\end{pgfscope}%
\begin{pgfscope}%
\pgfpathrectangle{\pgfqpoint{0.750000in}{0.520000in}}{\pgfqpoint{4.150000in}{3.200000in}}%
\pgfusepath{clip}%
\pgfsetroundcap%
\pgfsetroundjoin%
\pgfsetlinewidth{1.606000pt}%
\definecolor{currentstroke}{rgb}{1.000000,1.000000,1.000000}%
\pgfsetstrokecolor{currentstroke}%
\pgfsetdash{}{0pt}%
\pgfpathmoveto{\pgfqpoint{0.750000in}{1.434286in}}%
\pgfpathlineto{\pgfqpoint{4.900000in}{1.434286in}}%
\pgfusepath{stroke}%
\end{pgfscope}%
\begin{pgfscope}%
\definecolor{textcolor}{rgb}{0.150000,0.150000,0.150000}%
\pgfsetstrokecolor{textcolor}%
\pgfsetfillcolor{textcolor}%
\pgftext[x=0.543194in, y=1.381272in, left, base]{\color{textcolor}\sffamily\fontsize{11.000000}{13.200000}\selectfont 0}%
\end{pgfscope}%
\begin{pgfscope}%
\pgfpathrectangle{\pgfqpoint{0.750000in}{0.520000in}}{\pgfqpoint{4.150000in}{3.200000in}}%
\pgfusepath{clip}%
\pgfsetroundcap%
\pgfsetroundjoin%
\pgfsetlinewidth{1.606000pt}%
\definecolor{currentstroke}{rgb}{1.000000,1.000000,1.000000}%
\pgfsetstrokecolor{currentstroke}%
\pgfsetdash{}{0pt}%
\pgfpathmoveto{\pgfqpoint{0.750000in}{1.891429in}}%
\pgfpathlineto{\pgfqpoint{4.900000in}{1.891429in}}%
\pgfusepath{stroke}%
\end{pgfscope}%
\begin{pgfscope}%
\definecolor{textcolor}{rgb}{0.150000,0.150000,0.150000}%
\pgfsetstrokecolor{textcolor}%
\pgfsetfillcolor{textcolor}%
\pgftext[x=0.468333in, y=1.838415in, left, base]{\color{textcolor}\sffamily\fontsize{11.000000}{13.200000}\selectfont 50}%
\end{pgfscope}%
\begin{pgfscope}%
\pgfpathrectangle{\pgfqpoint{0.750000in}{0.520000in}}{\pgfqpoint{4.150000in}{3.200000in}}%
\pgfusepath{clip}%
\pgfsetroundcap%
\pgfsetroundjoin%
\pgfsetlinewidth{1.606000pt}%
\definecolor{currentstroke}{rgb}{1.000000,1.000000,1.000000}%
\pgfsetstrokecolor{currentstroke}%
\pgfsetdash{}{0pt}%
\pgfpathmoveto{\pgfqpoint{0.750000in}{2.348571in}}%
\pgfpathlineto{\pgfqpoint{4.900000in}{2.348571in}}%
\pgfusepath{stroke}%
\end{pgfscope}%
\begin{pgfscope}%
\definecolor{textcolor}{rgb}{0.150000,0.150000,0.150000}%
\pgfsetstrokecolor{textcolor}%
\pgfsetfillcolor{textcolor}%
\pgftext[x=0.393472in, y=2.295558in, left, base]{\color{textcolor}\sffamily\fontsize{11.000000}{13.200000}\selectfont 100}%
\end{pgfscope}%
\begin{pgfscope}%
\pgfpathrectangle{\pgfqpoint{0.750000in}{0.520000in}}{\pgfqpoint{4.150000in}{3.200000in}}%
\pgfusepath{clip}%
\pgfsetroundcap%
\pgfsetroundjoin%
\pgfsetlinewidth{1.606000pt}%
\definecolor{currentstroke}{rgb}{1.000000,1.000000,1.000000}%
\pgfsetstrokecolor{currentstroke}%
\pgfsetdash{}{0pt}%
\pgfpathmoveto{\pgfqpoint{0.750000in}{2.805714in}}%
\pgfpathlineto{\pgfqpoint{4.900000in}{2.805714in}}%
\pgfusepath{stroke}%
\end{pgfscope}%
\begin{pgfscope}%
\definecolor{textcolor}{rgb}{0.150000,0.150000,0.150000}%
\pgfsetstrokecolor{textcolor}%
\pgfsetfillcolor{textcolor}%
\pgftext[x=0.393472in, y=2.752700in, left, base]{\color{textcolor}\sffamily\fontsize{11.000000}{13.200000}\selectfont 150}%
\end{pgfscope}%
\begin{pgfscope}%
\pgfpathrectangle{\pgfqpoint{0.750000in}{0.520000in}}{\pgfqpoint{4.150000in}{3.200000in}}%
\pgfusepath{clip}%
\pgfsetroundcap%
\pgfsetroundjoin%
\pgfsetlinewidth{1.606000pt}%
\definecolor{currentstroke}{rgb}{1.000000,1.000000,1.000000}%
\pgfsetstrokecolor{currentstroke}%
\pgfsetdash{}{0pt}%
\pgfpathmoveto{\pgfqpoint{0.750000in}{3.262857in}}%
\pgfpathlineto{\pgfqpoint{4.900000in}{3.262857in}}%
\pgfusepath{stroke}%
\end{pgfscope}%
\begin{pgfscope}%
\definecolor{textcolor}{rgb}{0.150000,0.150000,0.150000}%
\pgfsetstrokecolor{textcolor}%
\pgfsetfillcolor{textcolor}%
\pgftext[x=0.393472in, y=3.209843in, left, base]{\color{textcolor}\sffamily\fontsize{11.000000}{13.200000}\selectfont 200}%
\end{pgfscope}%
\begin{pgfscope}%
\pgfpathrectangle{\pgfqpoint{0.750000in}{0.520000in}}{\pgfqpoint{4.150000in}{3.200000in}}%
\pgfusepath{clip}%
\pgfsetroundcap%
\pgfsetroundjoin%
\pgfsetlinewidth{1.606000pt}%
\definecolor{currentstroke}{rgb}{1.000000,1.000000,1.000000}%
\pgfsetstrokecolor{currentstroke}%
\pgfsetdash{}{0pt}%
\pgfpathmoveto{\pgfqpoint{0.750000in}{3.720000in}}%
\pgfpathlineto{\pgfqpoint{4.900000in}{3.720000in}}%
\pgfusepath{stroke}%
\end{pgfscope}%
\begin{pgfscope}%
\definecolor{textcolor}{rgb}{0.150000,0.150000,0.150000}%
\pgfsetstrokecolor{textcolor}%
\pgfsetfillcolor{textcolor}%
\pgftext[x=0.393472in, y=3.666986in, left, base]{\color{textcolor}\sffamily\fontsize{11.000000}{13.200000}\selectfont 250}%
\end{pgfscope}%
\begin{pgfscope}%
\pgfpathrectangle{\pgfqpoint{0.750000in}{0.520000in}}{\pgfqpoint{4.150000in}{3.200000in}}%
\pgfusepath{clip}%
\pgfsetroundcap%
\pgfsetroundjoin%
\pgfsetlinewidth{0.702625pt}%
\definecolor{currentstroke}{rgb}{1.000000,1.000000,1.000000}%
\pgfsetstrokecolor{currentstroke}%
\pgfsetdash{}{0pt}%
\pgfpathmoveto{\pgfqpoint{0.750000in}{0.748571in}}%
\pgfpathlineto{\pgfqpoint{4.900000in}{0.748571in}}%
\pgfusepath{stroke}%
\end{pgfscope}%
\begin{pgfscope}%
\pgfpathrectangle{\pgfqpoint{0.750000in}{0.520000in}}{\pgfqpoint{4.150000in}{3.200000in}}%
\pgfusepath{clip}%
\pgfsetroundcap%
\pgfsetroundjoin%
\pgfsetlinewidth{0.702625pt}%
\definecolor{currentstroke}{rgb}{1.000000,1.000000,1.000000}%
\pgfsetstrokecolor{currentstroke}%
\pgfsetdash{}{0pt}%
\pgfpathmoveto{\pgfqpoint{0.750000in}{1.205714in}}%
\pgfpathlineto{\pgfqpoint{4.900000in}{1.205714in}}%
\pgfusepath{stroke}%
\end{pgfscope}%
\begin{pgfscope}%
\pgfpathrectangle{\pgfqpoint{0.750000in}{0.520000in}}{\pgfqpoint{4.150000in}{3.200000in}}%
\pgfusepath{clip}%
\pgfsetroundcap%
\pgfsetroundjoin%
\pgfsetlinewidth{0.702625pt}%
\definecolor{currentstroke}{rgb}{1.000000,1.000000,1.000000}%
\pgfsetstrokecolor{currentstroke}%
\pgfsetdash{}{0pt}%
\pgfpathmoveto{\pgfqpoint{0.750000in}{1.662857in}}%
\pgfpathlineto{\pgfqpoint{4.900000in}{1.662857in}}%
\pgfusepath{stroke}%
\end{pgfscope}%
\begin{pgfscope}%
\pgfpathrectangle{\pgfqpoint{0.750000in}{0.520000in}}{\pgfqpoint{4.150000in}{3.200000in}}%
\pgfusepath{clip}%
\pgfsetroundcap%
\pgfsetroundjoin%
\pgfsetlinewidth{0.702625pt}%
\definecolor{currentstroke}{rgb}{1.000000,1.000000,1.000000}%
\pgfsetstrokecolor{currentstroke}%
\pgfsetdash{}{0pt}%
\pgfpathmoveto{\pgfqpoint{0.750000in}{2.120000in}}%
\pgfpathlineto{\pgfqpoint{4.900000in}{2.120000in}}%
\pgfusepath{stroke}%
\end{pgfscope}%
\begin{pgfscope}%
\pgfpathrectangle{\pgfqpoint{0.750000in}{0.520000in}}{\pgfqpoint{4.150000in}{3.200000in}}%
\pgfusepath{clip}%
\pgfsetroundcap%
\pgfsetroundjoin%
\pgfsetlinewidth{0.702625pt}%
\definecolor{currentstroke}{rgb}{1.000000,1.000000,1.000000}%
\pgfsetstrokecolor{currentstroke}%
\pgfsetdash{}{0pt}%
\pgfpathmoveto{\pgfqpoint{0.750000in}{2.577143in}}%
\pgfpathlineto{\pgfqpoint{4.900000in}{2.577143in}}%
\pgfusepath{stroke}%
\end{pgfscope}%
\begin{pgfscope}%
\pgfpathrectangle{\pgfqpoint{0.750000in}{0.520000in}}{\pgfqpoint{4.150000in}{3.200000in}}%
\pgfusepath{clip}%
\pgfsetroundcap%
\pgfsetroundjoin%
\pgfsetlinewidth{0.702625pt}%
\definecolor{currentstroke}{rgb}{1.000000,1.000000,1.000000}%
\pgfsetstrokecolor{currentstroke}%
\pgfsetdash{}{0pt}%
\pgfpathmoveto{\pgfqpoint{0.750000in}{3.034286in}}%
\pgfpathlineto{\pgfqpoint{4.900000in}{3.034286in}}%
\pgfusepath{stroke}%
\end{pgfscope}%
\begin{pgfscope}%
\pgfpathrectangle{\pgfqpoint{0.750000in}{0.520000in}}{\pgfqpoint{4.150000in}{3.200000in}}%
\pgfusepath{clip}%
\pgfsetroundcap%
\pgfsetroundjoin%
\pgfsetlinewidth{0.702625pt}%
\definecolor{currentstroke}{rgb}{1.000000,1.000000,1.000000}%
\pgfsetstrokecolor{currentstroke}%
\pgfsetdash{}{0pt}%
\pgfpathmoveto{\pgfqpoint{0.750000in}{3.491429in}}%
\pgfpathlineto{\pgfqpoint{4.900000in}{3.491429in}}%
\pgfusepath{stroke}%
\end{pgfscope}%
\begin{pgfscope}%
\definecolor{textcolor}{rgb}{0.150000,0.150000,0.150000}%
\pgfsetstrokecolor{textcolor}%
\pgfsetfillcolor{textcolor}%
\pgftext[x=0.219629in,y=2.120000in,,bottom,rotate=90.000000]{\color{textcolor}\sffamily\fontsize{12.000000}{14.400000}\selectfont \(\displaystyle f(x)\)}%
\end{pgfscope}%
\begin{pgfscope}%
\pgfpathrectangle{\pgfqpoint{0.750000in}{0.520000in}}{\pgfqpoint{4.150000in}{3.200000in}}%
\pgfusepath{clip}%
\pgfsetbuttcap%
\pgfsetroundjoin%
\definecolor{currentfill}{rgb}{0.000000,0.000000,0.000000}%
\pgfsetfillcolor{currentfill}%
\pgfsetlinewidth{1.003750pt}%
\definecolor{currentstroke}{rgb}{0.000000,0.000000,0.000000}%
\pgfsetstrokecolor{currentstroke}%
\pgfsetdash{}{0pt}%
\pgfsys@defobject{currentmarker}{\pgfqpoint{-0.020833in}{-0.020833in}}{\pgfqpoint{0.020833in}{0.020833in}}{%
\pgfpathmoveto{\pgfqpoint{0.000000in}{-0.020833in}}%
\pgfpathcurveto{\pgfqpoint{0.005525in}{-0.020833in}}{\pgfqpoint{0.010825in}{-0.018638in}}{\pgfqpoint{0.014731in}{-0.014731in}}%
\pgfpathcurveto{\pgfqpoint{0.018638in}{-0.010825in}}{\pgfqpoint{0.020833in}{-0.005525in}}{\pgfqpoint{0.020833in}{0.000000in}}%
\pgfpathcurveto{\pgfqpoint{0.020833in}{0.005525in}}{\pgfqpoint{0.018638in}{0.010825in}}{\pgfqpoint{0.014731in}{0.014731in}}%
\pgfpathcurveto{\pgfqpoint{0.010825in}{0.018638in}}{\pgfqpoint{0.005525in}{0.020833in}}{\pgfqpoint{0.000000in}{0.020833in}}%
\pgfpathcurveto{\pgfqpoint{-0.005525in}{0.020833in}}{\pgfqpoint{-0.010825in}{0.018638in}}{\pgfqpoint{-0.014731in}{0.014731in}}%
\pgfpathcurveto{\pgfqpoint{-0.018638in}{0.010825in}}{\pgfqpoint{-0.020833in}{0.005525in}}{\pgfqpoint{-0.020833in}{0.000000in}}%
\pgfpathcurveto{\pgfqpoint{-0.020833in}{-0.005525in}}{\pgfqpoint{-0.018638in}{-0.010825in}}{\pgfqpoint{-0.014731in}{-0.014731in}}%
\pgfpathcurveto{\pgfqpoint{-0.010825in}{-0.018638in}}{\pgfqpoint{-0.005525in}{-0.020833in}}{\pgfqpoint{0.000000in}{-0.020833in}}%
\pgfpathclose%
\pgfusepath{stroke,fill}%
}%
\begin{pgfscope}%
\pgfsys@transformshift{3.027576in}{2.757289in}%
\pgfsys@useobject{currentmarker}{}%
\end{pgfscope}%
\begin{pgfscope}%
\pgfsys@transformshift{3.718036in}{2.906160in}%
\pgfsys@useobject{currentmarker}{}%
\end{pgfscope}%
\begin{pgfscope}%
\pgfsys@transformshift{3.251468in}{2.747962in}%
\pgfsys@useobject{currentmarker}{}%
\end{pgfscope}%
\begin{pgfscope}%
\pgfsys@transformshift{3.011265in}{2.535412in}%
\pgfsys@useobject{currentmarker}{}%
\end{pgfscope}%
\begin{pgfscope}%
\pgfsys@transformshift{2.508167in}{2.158410in}%
\pgfsys@useobject{currentmarker}{}%
\end{pgfscope}%
\begin{pgfscope}%
\pgfsys@transformshift{3.430461in}{2.857966in}%
\pgfsys@useobject{currentmarker}{}%
\end{pgfscope}%
\begin{pgfscope}%
\pgfsys@transformshift{2.565987in}{2.496336in}%
\pgfsys@useobject{currentmarker}{}%
\end{pgfscope}%
\begin{pgfscope}%
\pgfsys@transformshift{4.450858in}{3.179958in}%
\pgfsys@useobject{currentmarker}{}%
\end{pgfscope}%
\begin{pgfscope}%
\pgfsys@transformshift{4.749200in}{3.586789in}%
\pgfsys@useobject{currentmarker}{}%
\end{pgfscope}%
\begin{pgfscope}%
\pgfsys@transformshift{2.341282in}{2.185322in}%
\pgfsys@useobject{currentmarker}{}%
\end{pgfscope}%
\begin{pgfscope}%
\pgfsys@transformshift{4.035659in}{3.069052in}%
\pgfsys@useobject{currentmarker}{}%
\end{pgfscope}%
\begin{pgfscope}%
\pgfsys@transformshift{2.944914in}{2.567151in}%
\pgfsys@useobject{currentmarker}{}%
\end{pgfscope}%
\begin{pgfscope}%
\pgfsys@transformshift{3.107385in}{2.795993in}%
\pgfsys@useobject{currentmarker}{}%
\end{pgfscope}%
\begin{pgfscope}%
\pgfsys@transformshift{4.591226in}{3.444004in}%
\pgfsys@useobject{currentmarker}{}%
\end{pgfscope}%
\begin{pgfscope}%
\pgfsys@transformshift{1.044800in}{1.761204in}%
\pgfsys@useobject{currentmarker}{}%
\end{pgfscope}%
\begin{pgfscope}%
\pgfsys@transformshift{1.111587in}{1.811040in}%
\pgfsys@useobject{currentmarker}{}%
\end{pgfscope}%
\begin{pgfscope}%
\pgfsys@transformshift{0.833906in}{1.572945in}%
\pgfsys@useobject{currentmarker}{}%
\end{pgfscope}%
\begin{pgfscope}%
\pgfsys@transformshift{4.205372in}{2.958546in}%
\pgfsys@useobject{currentmarker}{}%
\end{pgfscope}%
\begin{pgfscope}%
\pgfsys@transformshift{3.979351in}{3.008249in}%
\pgfsys@useobject{currentmarker}{}%
\end{pgfscope}%
\begin{pgfscope}%
\pgfsys@transformshift{4.360550in}{3.222317in}%
\pgfsys@useobject{currentmarker}{}%
\end{pgfscope}%
\begin{pgfscope}%
\pgfsys@transformshift{4.485000in}{1.434286in}%
\pgfsys@useobject{currentmarker}{}%
\end{pgfscope}%
\end{pgfscope}%
\begin{pgfscope}%
\pgfpathrectangle{\pgfqpoint{0.750000in}{0.520000in}}{\pgfqpoint{4.150000in}{3.200000in}}%
\pgfusepath{clip}%
\pgfsetroundcap%
\pgfsetroundjoin%
\pgfsetlinewidth{1.505625pt}%
\definecolor{currentstroke}{rgb}{0.258824,0.521569,0.956863}%
\pgfsetstrokecolor{currentstroke}%
\pgfsetstrokeopacity{0.800000}%
\pgfsetdash{}{0pt}%
\pgfpathmoveto{\pgfqpoint{0.750000in}{1.632293in}}%
\pgfpathlineto{\pgfqpoint{0.834694in}{1.669329in}}%
\pgfpathlineto{\pgfqpoint{0.919388in}{1.706364in}}%
\pgfpathlineto{\pgfqpoint{1.004082in}{1.743399in}}%
\pgfpathlineto{\pgfqpoint{1.088776in}{1.780434in}}%
\pgfpathlineto{\pgfqpoint{1.173469in}{1.817469in}}%
\pgfpathlineto{\pgfqpoint{1.258163in}{1.854504in}}%
\pgfpathlineto{\pgfqpoint{1.342857in}{1.891539in}}%
\pgfpathlineto{\pgfqpoint{1.427551in}{1.928574in}}%
\pgfpathlineto{\pgfqpoint{1.512245in}{1.965609in}}%
\pgfpathlineto{\pgfqpoint{1.596939in}{2.002644in}}%
\pgfpathlineto{\pgfqpoint{1.681633in}{2.039679in}}%
\pgfpathlineto{\pgfqpoint{1.766327in}{2.076715in}}%
\pgfpathlineto{\pgfqpoint{1.851020in}{2.113750in}}%
\pgfpathlineto{\pgfqpoint{1.935714in}{2.150785in}}%
\pgfpathlineto{\pgfqpoint{2.020408in}{2.187820in}}%
\pgfpathlineto{\pgfqpoint{2.105102in}{2.224855in}}%
\pgfpathlineto{\pgfqpoint{2.189796in}{2.261890in}}%
\pgfpathlineto{\pgfqpoint{2.274490in}{2.298925in}}%
\pgfpathlineto{\pgfqpoint{2.359184in}{2.335960in}}%
\pgfpathlineto{\pgfqpoint{2.443878in}{2.372995in}}%
\pgfpathlineto{\pgfqpoint{2.528571in}{2.410030in}}%
\pgfpathlineto{\pgfqpoint{2.613265in}{2.447065in}}%
\pgfpathlineto{\pgfqpoint{2.697959in}{2.484101in}}%
\pgfpathlineto{\pgfqpoint{2.782653in}{2.521136in}}%
\pgfpathlineto{\pgfqpoint{2.867347in}{2.558171in}}%
\pgfpathlineto{\pgfqpoint{2.952041in}{2.595206in}}%
\pgfpathlineto{\pgfqpoint{3.036735in}{2.632241in}}%
\pgfpathlineto{\pgfqpoint{3.121429in}{2.669276in}}%
\pgfpathlineto{\pgfqpoint{3.206122in}{2.706311in}}%
\pgfpathlineto{\pgfqpoint{3.290816in}{2.743346in}}%
\pgfpathlineto{\pgfqpoint{3.375510in}{2.780381in}}%
\pgfpathlineto{\pgfqpoint{3.460204in}{2.817416in}}%
\pgfpathlineto{\pgfqpoint{3.544898in}{2.854451in}}%
\pgfpathlineto{\pgfqpoint{3.629592in}{2.891487in}}%
\pgfpathlineto{\pgfqpoint{3.714286in}{2.928522in}}%
\pgfpathlineto{\pgfqpoint{3.798980in}{2.965557in}}%
\pgfpathlineto{\pgfqpoint{3.883673in}{3.002592in}}%
\pgfpathlineto{\pgfqpoint{3.968367in}{3.039627in}}%
\pgfpathlineto{\pgfqpoint{4.053061in}{3.076662in}}%
\pgfpathlineto{\pgfqpoint{4.137755in}{3.113697in}}%
\pgfpathlineto{\pgfqpoint{4.222449in}{3.150732in}}%
\pgfpathlineto{\pgfqpoint{4.307143in}{3.187767in}}%
\pgfpathlineto{\pgfqpoint{4.391837in}{3.224802in}}%
\pgfpathlineto{\pgfqpoint{4.476531in}{3.261837in}}%
\pgfpathlineto{\pgfqpoint{4.561224in}{3.298873in}}%
\pgfpathlineto{\pgfqpoint{4.645918in}{3.335908in}}%
\pgfpathlineto{\pgfqpoint{4.730612in}{3.372943in}}%
\pgfpathlineto{\pgfqpoint{4.815306in}{3.409978in}}%
\pgfpathlineto{\pgfqpoint{4.900000in}{3.447013in}}%
\pgfusepath{stroke}%
\end{pgfscope}%
\begin{pgfscope}%
\pgfpathrectangle{\pgfqpoint{0.750000in}{0.520000in}}{\pgfqpoint{4.150000in}{3.200000in}}%
\pgfusepath{clip}%
\pgfsetroundcap%
\pgfsetroundjoin%
\pgfsetlinewidth{1.505625pt}%
\definecolor{currentstroke}{rgb}{0.917647,0.262745,0.207843}%
\pgfsetstrokecolor{currentstroke}%
\pgfsetstrokeopacity{0.800000}%
\pgfsetdash{}{0pt}%
\pgfpathmoveto{\pgfqpoint{0.750000in}{1.687067in}}%
\pgfpathlineto{\pgfqpoint{0.834694in}{1.719033in}}%
\pgfpathlineto{\pgfqpoint{0.919388in}{1.750999in}}%
\pgfpathlineto{\pgfqpoint{1.004082in}{1.782965in}}%
\pgfpathlineto{\pgfqpoint{1.088776in}{1.814931in}}%
\pgfpathlineto{\pgfqpoint{1.173469in}{1.846897in}}%
\pgfpathlineto{\pgfqpoint{1.258163in}{1.878864in}}%
\pgfpathlineto{\pgfqpoint{1.342857in}{1.910830in}}%
\pgfpathlineto{\pgfqpoint{1.427551in}{1.942796in}}%
\pgfpathlineto{\pgfqpoint{1.512245in}{1.974762in}}%
\pgfpathlineto{\pgfqpoint{1.596939in}{2.006728in}}%
\pgfpathlineto{\pgfqpoint{1.681633in}{2.038695in}}%
\pgfpathlineto{\pgfqpoint{1.766327in}{2.070661in}}%
\pgfpathlineto{\pgfqpoint{1.851020in}{2.102627in}}%
\pgfpathlineto{\pgfqpoint{1.935714in}{2.134593in}}%
\pgfpathlineto{\pgfqpoint{2.020408in}{2.166559in}}%
\pgfpathlineto{\pgfqpoint{2.105102in}{2.198525in}}%
\pgfpathlineto{\pgfqpoint{2.189796in}{2.230492in}}%
\pgfpathlineto{\pgfqpoint{2.274490in}{2.262458in}}%
\pgfpathlineto{\pgfqpoint{2.359184in}{2.294424in}}%
\pgfpathlineto{\pgfqpoint{2.443878in}{2.326390in}}%
\pgfpathlineto{\pgfqpoint{2.528571in}{2.358356in}}%
\pgfpathlineto{\pgfqpoint{2.613265in}{2.390323in}}%
\pgfpathlineto{\pgfqpoint{2.697959in}{2.422289in}}%
\pgfpathlineto{\pgfqpoint{2.782653in}{2.454255in}}%
\pgfpathlineto{\pgfqpoint{2.867347in}{2.486221in}}%
\pgfpathlineto{\pgfqpoint{2.952041in}{2.518187in}}%
\pgfpathlineto{\pgfqpoint{3.036735in}{2.550153in}}%
\pgfpathlineto{\pgfqpoint{3.121429in}{2.582120in}}%
\pgfpathlineto{\pgfqpoint{3.206122in}{2.614086in}}%
\pgfpathlineto{\pgfqpoint{3.290816in}{2.646052in}}%
\pgfpathlineto{\pgfqpoint{3.375510in}{2.678018in}}%
\pgfpathlineto{\pgfqpoint{3.460204in}{2.709984in}}%
\pgfpathlineto{\pgfqpoint{3.544898in}{2.741951in}}%
\pgfpathlineto{\pgfqpoint{3.629592in}{2.773917in}}%
\pgfpathlineto{\pgfqpoint{3.714286in}{2.805883in}}%
\pgfpathlineto{\pgfqpoint{3.798980in}{2.837849in}}%
\pgfpathlineto{\pgfqpoint{3.883673in}{2.869815in}}%
\pgfpathlineto{\pgfqpoint{3.968367in}{2.901781in}}%
\pgfpathlineto{\pgfqpoint{4.053061in}{2.933748in}}%
\pgfpathlineto{\pgfqpoint{4.137755in}{2.965714in}}%
\pgfpathlineto{\pgfqpoint{4.222449in}{2.997680in}}%
\pgfpathlineto{\pgfqpoint{4.307143in}{3.029646in}}%
\pgfpathlineto{\pgfqpoint{4.391837in}{3.061612in}}%
\pgfpathlineto{\pgfqpoint{4.476531in}{3.093579in}}%
\pgfpathlineto{\pgfqpoint{4.561224in}{3.125545in}}%
\pgfpathlineto{\pgfqpoint{4.645918in}{3.157511in}}%
\pgfpathlineto{\pgfqpoint{4.730612in}{3.189477in}}%
\pgfpathlineto{\pgfqpoint{4.815306in}{3.221443in}}%
\pgfpathlineto{\pgfqpoint{4.900000in}{3.253410in}}%
\pgfusepath{stroke}%
\end{pgfscope}%
\begin{pgfscope}%
\pgfpathrectangle{\pgfqpoint{0.750000in}{0.520000in}}{\pgfqpoint{4.150000in}{3.200000in}}%
\pgfusepath{clip}%
\pgfsetroundcap%
\pgfsetroundjoin%
\pgfsetlinewidth{1.505625pt}%
\definecolor{currentstroke}{rgb}{0.203922,0.658824,0.325490}%
\pgfsetstrokecolor{currentstroke}%
\pgfsetstrokeopacity{0.800000}%
\pgfsetdash{}{0pt}%
\pgfpathmoveto{\pgfqpoint{0.750000in}{0.647910in}}%
\pgfpathlineto{\pgfqpoint{0.834694in}{0.688704in}}%
\pgfpathlineto{\pgfqpoint{0.919388in}{0.729498in}}%
\pgfpathlineto{\pgfqpoint{1.004082in}{0.770291in}}%
\pgfpathlineto{\pgfqpoint{1.088776in}{0.811085in}}%
\pgfpathlineto{\pgfqpoint{1.173469in}{0.851879in}}%
\pgfpathlineto{\pgfqpoint{1.258163in}{0.892672in}}%
\pgfpathlineto{\pgfqpoint{1.342857in}{0.933466in}}%
\pgfpathlineto{\pgfqpoint{1.427551in}{0.974260in}}%
\pgfpathlineto{\pgfqpoint{1.512245in}{1.015053in}}%
\pgfpathlineto{\pgfqpoint{1.596939in}{1.055847in}}%
\pgfpathlineto{\pgfqpoint{1.681633in}{1.096640in}}%
\pgfpathlineto{\pgfqpoint{1.766327in}{1.137434in}}%
\pgfpathlineto{\pgfqpoint{1.851020in}{1.178228in}}%
\pgfpathlineto{\pgfqpoint{1.935714in}{1.219021in}}%
\pgfpathlineto{\pgfqpoint{2.020408in}{1.259815in}}%
\pgfpathlineto{\pgfqpoint{2.105102in}{1.300609in}}%
\pgfpathlineto{\pgfqpoint{2.189796in}{1.341402in}}%
\pgfpathlineto{\pgfqpoint{2.274490in}{1.382196in}}%
\pgfpathlineto{\pgfqpoint{2.359184in}{1.422990in}}%
\pgfpathlineto{\pgfqpoint{2.443878in}{1.463783in}}%
\pgfpathlineto{\pgfqpoint{2.528571in}{1.504577in}}%
\pgfpathlineto{\pgfqpoint{2.613265in}{1.545371in}}%
\pgfpathlineto{\pgfqpoint{2.697959in}{1.586164in}}%
\pgfpathlineto{\pgfqpoint{2.782653in}{1.626958in}}%
\pgfpathlineto{\pgfqpoint{2.867347in}{1.667752in}}%
\pgfpathlineto{\pgfqpoint{2.952041in}{1.708545in}}%
\pgfpathlineto{\pgfqpoint{3.036735in}{1.749339in}}%
\pgfpathlineto{\pgfqpoint{3.121429in}{1.790132in}}%
\pgfpathlineto{\pgfqpoint{3.206122in}{1.830926in}}%
\pgfpathlineto{\pgfqpoint{3.290816in}{1.871720in}}%
\pgfpathlineto{\pgfqpoint{3.375510in}{1.912513in}}%
\pgfpathlineto{\pgfqpoint{3.460204in}{1.953307in}}%
\pgfpathlineto{\pgfqpoint{3.544898in}{1.994101in}}%
\pgfpathlineto{\pgfqpoint{3.629592in}{2.034894in}}%
\pgfpathlineto{\pgfqpoint{3.714286in}{2.075688in}}%
\pgfpathlineto{\pgfqpoint{3.798980in}{2.116482in}}%
\pgfpathlineto{\pgfqpoint{3.883673in}{2.157275in}}%
\pgfpathlineto{\pgfqpoint{3.968367in}{2.198069in}}%
\pgfpathlineto{\pgfqpoint{4.053061in}{2.238863in}}%
\pgfpathlineto{\pgfqpoint{4.137755in}{2.279656in}}%
\pgfpathlineto{\pgfqpoint{4.222449in}{2.320450in}}%
\pgfpathlineto{\pgfqpoint{4.307143in}{2.361244in}}%
\pgfpathlineto{\pgfqpoint{4.391837in}{2.402037in}}%
\pgfpathlineto{\pgfqpoint{4.476531in}{2.442831in}}%
\pgfpathlineto{\pgfqpoint{4.561224in}{2.483624in}}%
\pgfpathlineto{\pgfqpoint{4.645918in}{2.524418in}}%
\pgfpathlineto{\pgfqpoint{4.730612in}{2.565212in}}%
\pgfpathlineto{\pgfqpoint{4.815306in}{2.606005in}}%
\pgfpathlineto{\pgfqpoint{4.900000in}{2.646799in}}%
\pgfusepath{stroke}%
\end{pgfscope}%
\begin{pgfscope}%
\pgfsetrectcap%
\pgfsetmiterjoin%
\pgfsetlinewidth{1.254687pt}%
\definecolor{currentstroke}{rgb}{1.000000,1.000000,1.000000}%
\pgfsetstrokecolor{currentstroke}%
\pgfsetdash{}{0pt}%
\pgfpathmoveto{\pgfqpoint{0.750000in}{0.520000in}}%
\pgfpathlineto{\pgfqpoint{0.750000in}{3.720000in}}%
\pgfusepath{stroke}%
\end{pgfscope}%
\begin{pgfscope}%
\pgfsetrectcap%
\pgfsetmiterjoin%
\pgfsetlinewidth{1.254687pt}%
\definecolor{currentstroke}{rgb}{1.000000,1.000000,1.000000}%
\pgfsetstrokecolor{currentstroke}%
\pgfsetdash{}{0pt}%
\pgfpathmoveto{\pgfqpoint{4.900000in}{0.520000in}}%
\pgfpathlineto{\pgfqpoint{4.900000in}{3.720000in}}%
\pgfusepath{stroke}%
\end{pgfscope}%
\begin{pgfscope}%
\pgfsetrectcap%
\pgfsetmiterjoin%
\pgfsetlinewidth{1.254687pt}%
\definecolor{currentstroke}{rgb}{1.000000,1.000000,1.000000}%
\pgfsetstrokecolor{currentstroke}%
\pgfsetdash{}{0pt}%
\pgfpathmoveto{\pgfqpoint{0.750000in}{0.520000in}}%
\pgfpathlineto{\pgfqpoint{4.900000in}{0.520000in}}%
\pgfusepath{stroke}%
\end{pgfscope}%
\begin{pgfscope}%
\pgfsetrectcap%
\pgfsetmiterjoin%
\pgfsetlinewidth{1.254687pt}%
\definecolor{currentstroke}{rgb}{1.000000,1.000000,1.000000}%
\pgfsetstrokecolor{currentstroke}%
\pgfsetdash{}{0pt}%
\pgfpathmoveto{\pgfqpoint{0.750000in}{3.720000in}}%
\pgfpathlineto{\pgfqpoint{4.900000in}{3.720000in}}%
\pgfusepath{stroke}%
\end{pgfscope}%
\begin{pgfscope}%
\pgfsetfillopacity{0.900000}%
\pgfsetstrokeopacity{0.900000}%
\definecolor{textcolor}{rgb}{0.232941,0.469412,0.861176}%
\pgfsetstrokecolor{textcolor}%
\pgfsetfillcolor{textcolor}%
\pgftext[x=0.874650in, y=1.775406in, left, base,rotate=23.568304]{\color{textcolor}\sffamily\fontsize{10.000000}{12.000000}\selectfont \(\displaystyle L_1\) fit (best outlier rejection)}%
\end{pgfscope}%
\begin{pgfscope}%
\pgfsetfillopacity{0.900000}%
\pgfsetstrokeopacity{0.900000}%
\definecolor{textcolor}{rgb}{0.825882,0.236471,0.187059}%
\pgfsetstrokecolor{textcolor}%
\pgfsetfillcolor{textcolor}%
\pgftext[x=0.952706in, y=1.592065in, left, base,rotate=20.632598]{\color{textcolor}\sffamily\fontsize{10.000000}{12.000000}\selectfont \(\displaystyle L_2\) fit (least-squares)}%
\end{pgfscope}%
\begin{pgfscope}%
\pgfsetfillopacity{0.900000}%
\pgfsetstrokeopacity{0.900000}%
\definecolor{textcolor}{rgb}{0.183529,0.592941,0.292941}%
\pgfsetstrokecolor{textcolor}%
\pgfsetfillcolor{textcolor}%
\pgftext[x=0.878114in, y=0.794177in, left, base,rotate=25.664382]{\color{textcolor}\sffamily\fontsize{10.000000}{12.000000}\selectfont \(\displaystyle L_\infty\) fit (minimizes peak error)}%
\end{pgfscope}%
\begin{pgfscope}%
\definecolor{textcolor}{rgb}{0.150000,0.150000,0.150000}%
\pgfsetstrokecolor{textcolor}%
\pgfsetfillcolor{textcolor}%
\pgftext[x=2.825000in,y=3.803333in,,base]{\color{textcolor}\sffamily\fontsize{12.000000}{14.400000}\selectfont Impact of Norm Choice for Outlier Treatment}%
\end{pgfscope}%
\end{pgfpicture}%
\makeatother%
\endgroup%

    \ifdraft{}{%% Creator: Matplotlib, PGF backend
%%
%% To include the figure in your LaTeX document, write
%%   \input{<filename>.pgf}
%%
%% Make sure the required packages are loaded in your preamble
%%   \usepackage{pgf}
%%
%% Figures using additional raster images can only be included by \input if
%% they are in the same directory as the main LaTeX file. For loading figures
%% from other directories you can use the `import` package
%%   \usepackage{import}
%%
%% and then include the figures with
%%   \import{<path to file>}{<filename>.pgf}
%%
%% Matplotlib used the following preamble
%%   \usepackage{fontspec}
%%
\begingroup%
\makeatletter%
\begin{pgfpicture}%
\pgfpathrectangle{\pgfpointorigin}{\pgfqpoint{5.000000in}{4.000000in}}%
\pgfusepath{use as bounding box, clip}%
\begin{pgfscope}%
\pgfsetbuttcap%
\pgfsetmiterjoin%
\definecolor{currentfill}{rgb}{1.000000,1.000000,1.000000}%
\pgfsetfillcolor{currentfill}%
\pgfsetlinewidth{0.000000pt}%
\definecolor{currentstroke}{rgb}{1.000000,1.000000,1.000000}%
\pgfsetstrokecolor{currentstroke}%
\pgfsetdash{}{0pt}%
\pgfpathmoveto{\pgfqpoint{0.000000in}{0.000000in}}%
\pgfpathlineto{\pgfqpoint{5.000000in}{0.000000in}}%
\pgfpathlineto{\pgfqpoint{5.000000in}{4.000000in}}%
\pgfpathlineto{\pgfqpoint{0.000000in}{4.000000in}}%
\pgfpathclose%
\pgfusepath{fill}%
\end{pgfscope}%
\begin{pgfscope}%
\pgfsetbuttcap%
\pgfsetmiterjoin%
\definecolor{currentfill}{rgb}{0.917647,0.917647,0.949020}%
\pgfsetfillcolor{currentfill}%
\pgfsetlinewidth{0.000000pt}%
\definecolor{currentstroke}{rgb}{0.000000,0.000000,0.000000}%
\pgfsetstrokecolor{currentstroke}%
\pgfsetstrokeopacity{0.000000}%
\pgfsetdash{}{0pt}%
\pgfpathmoveto{\pgfqpoint{0.750000in}{0.520000in}}%
\pgfpathlineto{\pgfqpoint{4.900000in}{0.520000in}}%
\pgfpathlineto{\pgfqpoint{4.900000in}{3.720000in}}%
\pgfpathlineto{\pgfqpoint{0.750000in}{3.720000in}}%
\pgfpathclose%
\pgfusepath{fill}%
\end{pgfscope}%
\begin{pgfscope}%
\pgfpathrectangle{\pgfqpoint{0.750000in}{0.520000in}}{\pgfqpoint{4.150000in}{3.200000in}}%
\pgfusepath{clip}%
\pgfsetroundcap%
\pgfsetroundjoin%
\pgfsetlinewidth{1.606000pt}%
\definecolor{currentstroke}{rgb}{1.000000,1.000000,1.000000}%
\pgfsetstrokecolor{currentstroke}%
\pgfsetdash{}{0pt}%
\pgfpathmoveto{\pgfqpoint{0.750000in}{0.520000in}}%
\pgfpathlineto{\pgfqpoint{0.750000in}{3.720000in}}%
\pgfusepath{stroke}%
\end{pgfscope}%
\begin{pgfscope}%
\definecolor{textcolor}{rgb}{0.150000,0.150000,0.150000}%
\pgfsetstrokecolor{textcolor}%
\pgfsetfillcolor{textcolor}%
\pgftext[x=0.750000in,y=0.388056in,,top]{\color{textcolor}\sffamily\fontsize{11.000000}{13.200000}\selectfont 0}%
\end{pgfscope}%
\begin{pgfscope}%
\pgfpathrectangle{\pgfqpoint{0.750000in}{0.520000in}}{\pgfqpoint{4.150000in}{3.200000in}}%
\pgfusepath{clip}%
\pgfsetroundcap%
\pgfsetroundjoin%
\pgfsetlinewidth{1.606000pt}%
\definecolor{currentstroke}{rgb}{1.000000,1.000000,1.000000}%
\pgfsetstrokecolor{currentstroke}%
\pgfsetdash{}{0pt}%
\pgfpathmoveto{\pgfqpoint{1.165000in}{0.520000in}}%
\pgfpathlineto{\pgfqpoint{1.165000in}{3.720000in}}%
\pgfusepath{stroke}%
\end{pgfscope}%
\begin{pgfscope}%
\definecolor{textcolor}{rgb}{0.150000,0.150000,0.150000}%
\pgfsetstrokecolor{textcolor}%
\pgfsetfillcolor{textcolor}%
\pgftext[x=1.165000in,y=0.388056in,,top]{\color{textcolor}\sffamily\fontsize{11.000000}{13.200000}\selectfont 10}%
\end{pgfscope}%
\begin{pgfscope}%
\pgfpathrectangle{\pgfqpoint{0.750000in}{0.520000in}}{\pgfqpoint{4.150000in}{3.200000in}}%
\pgfusepath{clip}%
\pgfsetroundcap%
\pgfsetroundjoin%
\pgfsetlinewidth{1.606000pt}%
\definecolor{currentstroke}{rgb}{1.000000,1.000000,1.000000}%
\pgfsetstrokecolor{currentstroke}%
\pgfsetdash{}{0pt}%
\pgfpathmoveto{\pgfqpoint{1.580000in}{0.520000in}}%
\pgfpathlineto{\pgfqpoint{1.580000in}{3.720000in}}%
\pgfusepath{stroke}%
\end{pgfscope}%
\begin{pgfscope}%
\definecolor{textcolor}{rgb}{0.150000,0.150000,0.150000}%
\pgfsetstrokecolor{textcolor}%
\pgfsetfillcolor{textcolor}%
\pgftext[x=1.580000in,y=0.388056in,,top]{\color{textcolor}\sffamily\fontsize{11.000000}{13.200000}\selectfont 20}%
\end{pgfscope}%
\begin{pgfscope}%
\pgfpathrectangle{\pgfqpoint{0.750000in}{0.520000in}}{\pgfqpoint{4.150000in}{3.200000in}}%
\pgfusepath{clip}%
\pgfsetroundcap%
\pgfsetroundjoin%
\pgfsetlinewidth{1.606000pt}%
\definecolor{currentstroke}{rgb}{1.000000,1.000000,1.000000}%
\pgfsetstrokecolor{currentstroke}%
\pgfsetdash{}{0pt}%
\pgfpathmoveto{\pgfqpoint{1.995000in}{0.520000in}}%
\pgfpathlineto{\pgfqpoint{1.995000in}{3.720000in}}%
\pgfusepath{stroke}%
\end{pgfscope}%
\begin{pgfscope}%
\definecolor{textcolor}{rgb}{0.150000,0.150000,0.150000}%
\pgfsetstrokecolor{textcolor}%
\pgfsetfillcolor{textcolor}%
\pgftext[x=1.995000in,y=0.388056in,,top]{\color{textcolor}\sffamily\fontsize{11.000000}{13.200000}\selectfont 30}%
\end{pgfscope}%
\begin{pgfscope}%
\pgfpathrectangle{\pgfqpoint{0.750000in}{0.520000in}}{\pgfqpoint{4.150000in}{3.200000in}}%
\pgfusepath{clip}%
\pgfsetroundcap%
\pgfsetroundjoin%
\pgfsetlinewidth{1.606000pt}%
\definecolor{currentstroke}{rgb}{1.000000,1.000000,1.000000}%
\pgfsetstrokecolor{currentstroke}%
\pgfsetdash{}{0pt}%
\pgfpathmoveto{\pgfqpoint{2.410000in}{0.520000in}}%
\pgfpathlineto{\pgfqpoint{2.410000in}{3.720000in}}%
\pgfusepath{stroke}%
\end{pgfscope}%
\begin{pgfscope}%
\definecolor{textcolor}{rgb}{0.150000,0.150000,0.150000}%
\pgfsetstrokecolor{textcolor}%
\pgfsetfillcolor{textcolor}%
\pgftext[x=2.410000in,y=0.388056in,,top]{\color{textcolor}\sffamily\fontsize{11.000000}{13.200000}\selectfont 40}%
\end{pgfscope}%
\begin{pgfscope}%
\pgfpathrectangle{\pgfqpoint{0.750000in}{0.520000in}}{\pgfqpoint{4.150000in}{3.200000in}}%
\pgfusepath{clip}%
\pgfsetroundcap%
\pgfsetroundjoin%
\pgfsetlinewidth{1.606000pt}%
\definecolor{currentstroke}{rgb}{1.000000,1.000000,1.000000}%
\pgfsetstrokecolor{currentstroke}%
\pgfsetdash{}{0pt}%
\pgfpathmoveto{\pgfqpoint{2.825000in}{0.520000in}}%
\pgfpathlineto{\pgfqpoint{2.825000in}{3.720000in}}%
\pgfusepath{stroke}%
\end{pgfscope}%
\begin{pgfscope}%
\definecolor{textcolor}{rgb}{0.150000,0.150000,0.150000}%
\pgfsetstrokecolor{textcolor}%
\pgfsetfillcolor{textcolor}%
\pgftext[x=2.825000in,y=0.388056in,,top]{\color{textcolor}\sffamily\fontsize{11.000000}{13.200000}\selectfont 50}%
\end{pgfscope}%
\begin{pgfscope}%
\pgfpathrectangle{\pgfqpoint{0.750000in}{0.520000in}}{\pgfqpoint{4.150000in}{3.200000in}}%
\pgfusepath{clip}%
\pgfsetroundcap%
\pgfsetroundjoin%
\pgfsetlinewidth{1.606000pt}%
\definecolor{currentstroke}{rgb}{1.000000,1.000000,1.000000}%
\pgfsetstrokecolor{currentstroke}%
\pgfsetdash{}{0pt}%
\pgfpathmoveto{\pgfqpoint{3.240000in}{0.520000in}}%
\pgfpathlineto{\pgfqpoint{3.240000in}{3.720000in}}%
\pgfusepath{stroke}%
\end{pgfscope}%
\begin{pgfscope}%
\definecolor{textcolor}{rgb}{0.150000,0.150000,0.150000}%
\pgfsetstrokecolor{textcolor}%
\pgfsetfillcolor{textcolor}%
\pgftext[x=3.240000in,y=0.388056in,,top]{\color{textcolor}\sffamily\fontsize{11.000000}{13.200000}\selectfont 60}%
\end{pgfscope}%
\begin{pgfscope}%
\pgfpathrectangle{\pgfqpoint{0.750000in}{0.520000in}}{\pgfqpoint{4.150000in}{3.200000in}}%
\pgfusepath{clip}%
\pgfsetroundcap%
\pgfsetroundjoin%
\pgfsetlinewidth{1.606000pt}%
\definecolor{currentstroke}{rgb}{1.000000,1.000000,1.000000}%
\pgfsetstrokecolor{currentstroke}%
\pgfsetdash{}{0pt}%
\pgfpathmoveto{\pgfqpoint{3.655000in}{0.520000in}}%
\pgfpathlineto{\pgfqpoint{3.655000in}{3.720000in}}%
\pgfusepath{stroke}%
\end{pgfscope}%
\begin{pgfscope}%
\definecolor{textcolor}{rgb}{0.150000,0.150000,0.150000}%
\pgfsetstrokecolor{textcolor}%
\pgfsetfillcolor{textcolor}%
\pgftext[x=3.655000in,y=0.388056in,,top]{\color{textcolor}\sffamily\fontsize{11.000000}{13.200000}\selectfont 70}%
\end{pgfscope}%
\begin{pgfscope}%
\pgfpathrectangle{\pgfqpoint{0.750000in}{0.520000in}}{\pgfqpoint{4.150000in}{3.200000in}}%
\pgfusepath{clip}%
\pgfsetroundcap%
\pgfsetroundjoin%
\pgfsetlinewidth{1.606000pt}%
\definecolor{currentstroke}{rgb}{1.000000,1.000000,1.000000}%
\pgfsetstrokecolor{currentstroke}%
\pgfsetdash{}{0pt}%
\pgfpathmoveto{\pgfqpoint{4.070000in}{0.520000in}}%
\pgfpathlineto{\pgfqpoint{4.070000in}{3.720000in}}%
\pgfusepath{stroke}%
\end{pgfscope}%
\begin{pgfscope}%
\definecolor{textcolor}{rgb}{0.150000,0.150000,0.150000}%
\pgfsetstrokecolor{textcolor}%
\pgfsetfillcolor{textcolor}%
\pgftext[x=4.070000in,y=0.388056in,,top]{\color{textcolor}\sffamily\fontsize{11.000000}{13.200000}\selectfont 80}%
\end{pgfscope}%
\begin{pgfscope}%
\pgfpathrectangle{\pgfqpoint{0.750000in}{0.520000in}}{\pgfqpoint{4.150000in}{3.200000in}}%
\pgfusepath{clip}%
\pgfsetroundcap%
\pgfsetroundjoin%
\pgfsetlinewidth{1.606000pt}%
\definecolor{currentstroke}{rgb}{1.000000,1.000000,1.000000}%
\pgfsetstrokecolor{currentstroke}%
\pgfsetdash{}{0pt}%
\pgfpathmoveto{\pgfqpoint{4.485000in}{0.520000in}}%
\pgfpathlineto{\pgfqpoint{4.485000in}{3.720000in}}%
\pgfusepath{stroke}%
\end{pgfscope}%
\begin{pgfscope}%
\definecolor{textcolor}{rgb}{0.150000,0.150000,0.150000}%
\pgfsetstrokecolor{textcolor}%
\pgfsetfillcolor{textcolor}%
\pgftext[x=4.485000in,y=0.388056in,,top]{\color{textcolor}\sffamily\fontsize{11.000000}{13.200000}\selectfont 90}%
\end{pgfscope}%
\begin{pgfscope}%
\pgfpathrectangle{\pgfqpoint{0.750000in}{0.520000in}}{\pgfqpoint{4.150000in}{3.200000in}}%
\pgfusepath{clip}%
\pgfsetroundcap%
\pgfsetroundjoin%
\pgfsetlinewidth{1.606000pt}%
\definecolor{currentstroke}{rgb}{1.000000,1.000000,1.000000}%
\pgfsetstrokecolor{currentstroke}%
\pgfsetdash{}{0pt}%
\pgfpathmoveto{\pgfqpoint{4.900000in}{0.520000in}}%
\pgfpathlineto{\pgfqpoint{4.900000in}{3.720000in}}%
\pgfusepath{stroke}%
\end{pgfscope}%
\begin{pgfscope}%
\definecolor{textcolor}{rgb}{0.150000,0.150000,0.150000}%
\pgfsetstrokecolor{textcolor}%
\pgfsetfillcolor{textcolor}%
\pgftext[x=4.900000in,y=0.388056in,,top]{\color{textcolor}\sffamily\fontsize{11.000000}{13.200000}\selectfont 100}%
\end{pgfscope}%
\begin{pgfscope}%
\pgfpathrectangle{\pgfqpoint{0.750000in}{0.520000in}}{\pgfqpoint{4.150000in}{3.200000in}}%
\pgfusepath{clip}%
\pgfsetroundcap%
\pgfsetroundjoin%
\pgfsetlinewidth{0.702625pt}%
\definecolor{currentstroke}{rgb}{1.000000,1.000000,1.000000}%
\pgfsetstrokecolor{currentstroke}%
\pgfsetdash{}{0pt}%
\pgfpathmoveto{\pgfqpoint{0.957500in}{0.520000in}}%
\pgfpathlineto{\pgfqpoint{0.957500in}{3.720000in}}%
\pgfusepath{stroke}%
\end{pgfscope}%
\begin{pgfscope}%
\pgfpathrectangle{\pgfqpoint{0.750000in}{0.520000in}}{\pgfqpoint{4.150000in}{3.200000in}}%
\pgfusepath{clip}%
\pgfsetroundcap%
\pgfsetroundjoin%
\pgfsetlinewidth{0.702625pt}%
\definecolor{currentstroke}{rgb}{1.000000,1.000000,1.000000}%
\pgfsetstrokecolor{currentstroke}%
\pgfsetdash{}{0pt}%
\pgfpathmoveto{\pgfqpoint{1.372500in}{0.520000in}}%
\pgfpathlineto{\pgfqpoint{1.372500in}{3.720000in}}%
\pgfusepath{stroke}%
\end{pgfscope}%
\begin{pgfscope}%
\pgfpathrectangle{\pgfqpoint{0.750000in}{0.520000in}}{\pgfqpoint{4.150000in}{3.200000in}}%
\pgfusepath{clip}%
\pgfsetroundcap%
\pgfsetroundjoin%
\pgfsetlinewidth{0.702625pt}%
\definecolor{currentstroke}{rgb}{1.000000,1.000000,1.000000}%
\pgfsetstrokecolor{currentstroke}%
\pgfsetdash{}{0pt}%
\pgfpathmoveto{\pgfqpoint{1.787500in}{0.520000in}}%
\pgfpathlineto{\pgfqpoint{1.787500in}{3.720000in}}%
\pgfusepath{stroke}%
\end{pgfscope}%
\begin{pgfscope}%
\pgfpathrectangle{\pgfqpoint{0.750000in}{0.520000in}}{\pgfqpoint{4.150000in}{3.200000in}}%
\pgfusepath{clip}%
\pgfsetroundcap%
\pgfsetroundjoin%
\pgfsetlinewidth{0.702625pt}%
\definecolor{currentstroke}{rgb}{1.000000,1.000000,1.000000}%
\pgfsetstrokecolor{currentstroke}%
\pgfsetdash{}{0pt}%
\pgfpathmoveto{\pgfqpoint{2.202500in}{0.520000in}}%
\pgfpathlineto{\pgfqpoint{2.202500in}{3.720000in}}%
\pgfusepath{stroke}%
\end{pgfscope}%
\begin{pgfscope}%
\pgfpathrectangle{\pgfqpoint{0.750000in}{0.520000in}}{\pgfqpoint{4.150000in}{3.200000in}}%
\pgfusepath{clip}%
\pgfsetroundcap%
\pgfsetroundjoin%
\pgfsetlinewidth{0.702625pt}%
\definecolor{currentstroke}{rgb}{1.000000,1.000000,1.000000}%
\pgfsetstrokecolor{currentstroke}%
\pgfsetdash{}{0pt}%
\pgfpathmoveto{\pgfqpoint{2.617500in}{0.520000in}}%
\pgfpathlineto{\pgfqpoint{2.617500in}{3.720000in}}%
\pgfusepath{stroke}%
\end{pgfscope}%
\begin{pgfscope}%
\pgfpathrectangle{\pgfqpoint{0.750000in}{0.520000in}}{\pgfqpoint{4.150000in}{3.200000in}}%
\pgfusepath{clip}%
\pgfsetroundcap%
\pgfsetroundjoin%
\pgfsetlinewidth{0.702625pt}%
\definecolor{currentstroke}{rgb}{1.000000,1.000000,1.000000}%
\pgfsetstrokecolor{currentstroke}%
\pgfsetdash{}{0pt}%
\pgfpathmoveto{\pgfqpoint{3.032500in}{0.520000in}}%
\pgfpathlineto{\pgfqpoint{3.032500in}{3.720000in}}%
\pgfusepath{stroke}%
\end{pgfscope}%
\begin{pgfscope}%
\pgfpathrectangle{\pgfqpoint{0.750000in}{0.520000in}}{\pgfqpoint{4.150000in}{3.200000in}}%
\pgfusepath{clip}%
\pgfsetroundcap%
\pgfsetroundjoin%
\pgfsetlinewidth{0.702625pt}%
\definecolor{currentstroke}{rgb}{1.000000,1.000000,1.000000}%
\pgfsetstrokecolor{currentstroke}%
\pgfsetdash{}{0pt}%
\pgfpathmoveto{\pgfqpoint{3.447500in}{0.520000in}}%
\pgfpathlineto{\pgfqpoint{3.447500in}{3.720000in}}%
\pgfusepath{stroke}%
\end{pgfscope}%
\begin{pgfscope}%
\pgfpathrectangle{\pgfqpoint{0.750000in}{0.520000in}}{\pgfqpoint{4.150000in}{3.200000in}}%
\pgfusepath{clip}%
\pgfsetroundcap%
\pgfsetroundjoin%
\pgfsetlinewidth{0.702625pt}%
\definecolor{currentstroke}{rgb}{1.000000,1.000000,1.000000}%
\pgfsetstrokecolor{currentstroke}%
\pgfsetdash{}{0pt}%
\pgfpathmoveto{\pgfqpoint{3.862500in}{0.520000in}}%
\pgfpathlineto{\pgfqpoint{3.862500in}{3.720000in}}%
\pgfusepath{stroke}%
\end{pgfscope}%
\begin{pgfscope}%
\pgfpathrectangle{\pgfqpoint{0.750000in}{0.520000in}}{\pgfqpoint{4.150000in}{3.200000in}}%
\pgfusepath{clip}%
\pgfsetroundcap%
\pgfsetroundjoin%
\pgfsetlinewidth{0.702625pt}%
\definecolor{currentstroke}{rgb}{1.000000,1.000000,1.000000}%
\pgfsetstrokecolor{currentstroke}%
\pgfsetdash{}{0pt}%
\pgfpathmoveto{\pgfqpoint{4.277500in}{0.520000in}}%
\pgfpathlineto{\pgfqpoint{4.277500in}{3.720000in}}%
\pgfusepath{stroke}%
\end{pgfscope}%
\begin{pgfscope}%
\pgfpathrectangle{\pgfqpoint{0.750000in}{0.520000in}}{\pgfqpoint{4.150000in}{3.200000in}}%
\pgfusepath{clip}%
\pgfsetroundcap%
\pgfsetroundjoin%
\pgfsetlinewidth{0.702625pt}%
\definecolor{currentstroke}{rgb}{1.000000,1.000000,1.000000}%
\pgfsetstrokecolor{currentstroke}%
\pgfsetdash{}{0pt}%
\pgfpathmoveto{\pgfqpoint{4.692500in}{0.520000in}}%
\pgfpathlineto{\pgfqpoint{4.692500in}{3.720000in}}%
\pgfusepath{stroke}%
\end{pgfscope}%
\begin{pgfscope}%
\definecolor{textcolor}{rgb}{0.150000,0.150000,0.150000}%
\pgfsetstrokecolor{textcolor}%
\pgfsetfillcolor{textcolor}%
\pgftext[x=2.825000in,y=0.196833in,,top]{\color{textcolor}\sffamily\fontsize{12.000000}{14.400000}\selectfont \(\displaystyle x\)}%
\end{pgfscope}%
\begin{pgfscope}%
\pgfpathrectangle{\pgfqpoint{0.750000in}{0.520000in}}{\pgfqpoint{4.150000in}{3.200000in}}%
\pgfusepath{clip}%
\pgfsetroundcap%
\pgfsetroundjoin%
\pgfsetlinewidth{1.606000pt}%
\definecolor{currentstroke}{rgb}{1.000000,1.000000,1.000000}%
\pgfsetstrokecolor{currentstroke}%
\pgfsetdash{}{0pt}%
\pgfpathmoveto{\pgfqpoint{0.750000in}{0.520000in}}%
\pgfpathlineto{\pgfqpoint{4.900000in}{0.520000in}}%
\pgfusepath{stroke}%
\end{pgfscope}%
\begin{pgfscope}%
\definecolor{textcolor}{rgb}{0.150000,0.150000,0.150000}%
\pgfsetstrokecolor{textcolor}%
\pgfsetfillcolor{textcolor}%
\pgftext[x=0.275185in, y=0.466986in, left, base]{\color{textcolor}\sffamily\fontsize{11.000000}{13.200000}\selectfont \ensuremath{-}100}%
\end{pgfscope}%
\begin{pgfscope}%
\pgfpathrectangle{\pgfqpoint{0.750000in}{0.520000in}}{\pgfqpoint{4.150000in}{3.200000in}}%
\pgfusepath{clip}%
\pgfsetroundcap%
\pgfsetroundjoin%
\pgfsetlinewidth{1.606000pt}%
\definecolor{currentstroke}{rgb}{1.000000,1.000000,1.000000}%
\pgfsetstrokecolor{currentstroke}%
\pgfsetdash{}{0pt}%
\pgfpathmoveto{\pgfqpoint{0.750000in}{0.977143in}}%
\pgfpathlineto{\pgfqpoint{4.900000in}{0.977143in}}%
\pgfusepath{stroke}%
\end{pgfscope}%
\begin{pgfscope}%
\definecolor{textcolor}{rgb}{0.150000,0.150000,0.150000}%
\pgfsetstrokecolor{textcolor}%
\pgfsetfillcolor{textcolor}%
\pgftext[x=0.350046in, y=0.924129in, left, base]{\color{textcolor}\sffamily\fontsize{11.000000}{13.200000}\selectfont \ensuremath{-}50}%
\end{pgfscope}%
\begin{pgfscope}%
\pgfpathrectangle{\pgfqpoint{0.750000in}{0.520000in}}{\pgfqpoint{4.150000in}{3.200000in}}%
\pgfusepath{clip}%
\pgfsetroundcap%
\pgfsetroundjoin%
\pgfsetlinewidth{1.606000pt}%
\definecolor{currentstroke}{rgb}{1.000000,1.000000,1.000000}%
\pgfsetstrokecolor{currentstroke}%
\pgfsetdash{}{0pt}%
\pgfpathmoveto{\pgfqpoint{0.750000in}{1.434286in}}%
\pgfpathlineto{\pgfqpoint{4.900000in}{1.434286in}}%
\pgfusepath{stroke}%
\end{pgfscope}%
\begin{pgfscope}%
\definecolor{textcolor}{rgb}{0.150000,0.150000,0.150000}%
\pgfsetstrokecolor{textcolor}%
\pgfsetfillcolor{textcolor}%
\pgftext[x=0.543194in, y=1.381272in, left, base]{\color{textcolor}\sffamily\fontsize{11.000000}{13.200000}\selectfont 0}%
\end{pgfscope}%
\begin{pgfscope}%
\pgfpathrectangle{\pgfqpoint{0.750000in}{0.520000in}}{\pgfqpoint{4.150000in}{3.200000in}}%
\pgfusepath{clip}%
\pgfsetroundcap%
\pgfsetroundjoin%
\pgfsetlinewidth{1.606000pt}%
\definecolor{currentstroke}{rgb}{1.000000,1.000000,1.000000}%
\pgfsetstrokecolor{currentstroke}%
\pgfsetdash{}{0pt}%
\pgfpathmoveto{\pgfqpoint{0.750000in}{1.891429in}}%
\pgfpathlineto{\pgfqpoint{4.900000in}{1.891429in}}%
\pgfusepath{stroke}%
\end{pgfscope}%
\begin{pgfscope}%
\definecolor{textcolor}{rgb}{0.150000,0.150000,0.150000}%
\pgfsetstrokecolor{textcolor}%
\pgfsetfillcolor{textcolor}%
\pgftext[x=0.468333in, y=1.838415in, left, base]{\color{textcolor}\sffamily\fontsize{11.000000}{13.200000}\selectfont 50}%
\end{pgfscope}%
\begin{pgfscope}%
\pgfpathrectangle{\pgfqpoint{0.750000in}{0.520000in}}{\pgfqpoint{4.150000in}{3.200000in}}%
\pgfusepath{clip}%
\pgfsetroundcap%
\pgfsetroundjoin%
\pgfsetlinewidth{1.606000pt}%
\definecolor{currentstroke}{rgb}{1.000000,1.000000,1.000000}%
\pgfsetstrokecolor{currentstroke}%
\pgfsetdash{}{0pt}%
\pgfpathmoveto{\pgfqpoint{0.750000in}{2.348571in}}%
\pgfpathlineto{\pgfqpoint{4.900000in}{2.348571in}}%
\pgfusepath{stroke}%
\end{pgfscope}%
\begin{pgfscope}%
\definecolor{textcolor}{rgb}{0.150000,0.150000,0.150000}%
\pgfsetstrokecolor{textcolor}%
\pgfsetfillcolor{textcolor}%
\pgftext[x=0.393472in, y=2.295558in, left, base]{\color{textcolor}\sffamily\fontsize{11.000000}{13.200000}\selectfont 100}%
\end{pgfscope}%
\begin{pgfscope}%
\pgfpathrectangle{\pgfqpoint{0.750000in}{0.520000in}}{\pgfqpoint{4.150000in}{3.200000in}}%
\pgfusepath{clip}%
\pgfsetroundcap%
\pgfsetroundjoin%
\pgfsetlinewidth{1.606000pt}%
\definecolor{currentstroke}{rgb}{1.000000,1.000000,1.000000}%
\pgfsetstrokecolor{currentstroke}%
\pgfsetdash{}{0pt}%
\pgfpathmoveto{\pgfqpoint{0.750000in}{2.805714in}}%
\pgfpathlineto{\pgfqpoint{4.900000in}{2.805714in}}%
\pgfusepath{stroke}%
\end{pgfscope}%
\begin{pgfscope}%
\definecolor{textcolor}{rgb}{0.150000,0.150000,0.150000}%
\pgfsetstrokecolor{textcolor}%
\pgfsetfillcolor{textcolor}%
\pgftext[x=0.393472in, y=2.752700in, left, base]{\color{textcolor}\sffamily\fontsize{11.000000}{13.200000}\selectfont 150}%
\end{pgfscope}%
\begin{pgfscope}%
\pgfpathrectangle{\pgfqpoint{0.750000in}{0.520000in}}{\pgfqpoint{4.150000in}{3.200000in}}%
\pgfusepath{clip}%
\pgfsetroundcap%
\pgfsetroundjoin%
\pgfsetlinewidth{1.606000pt}%
\definecolor{currentstroke}{rgb}{1.000000,1.000000,1.000000}%
\pgfsetstrokecolor{currentstroke}%
\pgfsetdash{}{0pt}%
\pgfpathmoveto{\pgfqpoint{0.750000in}{3.262857in}}%
\pgfpathlineto{\pgfqpoint{4.900000in}{3.262857in}}%
\pgfusepath{stroke}%
\end{pgfscope}%
\begin{pgfscope}%
\definecolor{textcolor}{rgb}{0.150000,0.150000,0.150000}%
\pgfsetstrokecolor{textcolor}%
\pgfsetfillcolor{textcolor}%
\pgftext[x=0.393472in, y=3.209843in, left, base]{\color{textcolor}\sffamily\fontsize{11.000000}{13.200000}\selectfont 200}%
\end{pgfscope}%
\begin{pgfscope}%
\pgfpathrectangle{\pgfqpoint{0.750000in}{0.520000in}}{\pgfqpoint{4.150000in}{3.200000in}}%
\pgfusepath{clip}%
\pgfsetroundcap%
\pgfsetroundjoin%
\pgfsetlinewidth{1.606000pt}%
\definecolor{currentstroke}{rgb}{1.000000,1.000000,1.000000}%
\pgfsetstrokecolor{currentstroke}%
\pgfsetdash{}{0pt}%
\pgfpathmoveto{\pgfqpoint{0.750000in}{3.720000in}}%
\pgfpathlineto{\pgfqpoint{4.900000in}{3.720000in}}%
\pgfusepath{stroke}%
\end{pgfscope}%
\begin{pgfscope}%
\definecolor{textcolor}{rgb}{0.150000,0.150000,0.150000}%
\pgfsetstrokecolor{textcolor}%
\pgfsetfillcolor{textcolor}%
\pgftext[x=0.393472in, y=3.666986in, left, base]{\color{textcolor}\sffamily\fontsize{11.000000}{13.200000}\selectfont 250}%
\end{pgfscope}%
\begin{pgfscope}%
\pgfpathrectangle{\pgfqpoint{0.750000in}{0.520000in}}{\pgfqpoint{4.150000in}{3.200000in}}%
\pgfusepath{clip}%
\pgfsetroundcap%
\pgfsetroundjoin%
\pgfsetlinewidth{0.702625pt}%
\definecolor{currentstroke}{rgb}{1.000000,1.000000,1.000000}%
\pgfsetstrokecolor{currentstroke}%
\pgfsetdash{}{0pt}%
\pgfpathmoveto{\pgfqpoint{0.750000in}{0.748571in}}%
\pgfpathlineto{\pgfqpoint{4.900000in}{0.748571in}}%
\pgfusepath{stroke}%
\end{pgfscope}%
\begin{pgfscope}%
\pgfpathrectangle{\pgfqpoint{0.750000in}{0.520000in}}{\pgfqpoint{4.150000in}{3.200000in}}%
\pgfusepath{clip}%
\pgfsetroundcap%
\pgfsetroundjoin%
\pgfsetlinewidth{0.702625pt}%
\definecolor{currentstroke}{rgb}{1.000000,1.000000,1.000000}%
\pgfsetstrokecolor{currentstroke}%
\pgfsetdash{}{0pt}%
\pgfpathmoveto{\pgfqpoint{0.750000in}{1.205714in}}%
\pgfpathlineto{\pgfqpoint{4.900000in}{1.205714in}}%
\pgfusepath{stroke}%
\end{pgfscope}%
\begin{pgfscope}%
\pgfpathrectangle{\pgfqpoint{0.750000in}{0.520000in}}{\pgfqpoint{4.150000in}{3.200000in}}%
\pgfusepath{clip}%
\pgfsetroundcap%
\pgfsetroundjoin%
\pgfsetlinewidth{0.702625pt}%
\definecolor{currentstroke}{rgb}{1.000000,1.000000,1.000000}%
\pgfsetstrokecolor{currentstroke}%
\pgfsetdash{}{0pt}%
\pgfpathmoveto{\pgfqpoint{0.750000in}{1.662857in}}%
\pgfpathlineto{\pgfqpoint{4.900000in}{1.662857in}}%
\pgfusepath{stroke}%
\end{pgfscope}%
\begin{pgfscope}%
\pgfpathrectangle{\pgfqpoint{0.750000in}{0.520000in}}{\pgfqpoint{4.150000in}{3.200000in}}%
\pgfusepath{clip}%
\pgfsetroundcap%
\pgfsetroundjoin%
\pgfsetlinewidth{0.702625pt}%
\definecolor{currentstroke}{rgb}{1.000000,1.000000,1.000000}%
\pgfsetstrokecolor{currentstroke}%
\pgfsetdash{}{0pt}%
\pgfpathmoveto{\pgfqpoint{0.750000in}{2.120000in}}%
\pgfpathlineto{\pgfqpoint{4.900000in}{2.120000in}}%
\pgfusepath{stroke}%
\end{pgfscope}%
\begin{pgfscope}%
\pgfpathrectangle{\pgfqpoint{0.750000in}{0.520000in}}{\pgfqpoint{4.150000in}{3.200000in}}%
\pgfusepath{clip}%
\pgfsetroundcap%
\pgfsetroundjoin%
\pgfsetlinewidth{0.702625pt}%
\definecolor{currentstroke}{rgb}{1.000000,1.000000,1.000000}%
\pgfsetstrokecolor{currentstroke}%
\pgfsetdash{}{0pt}%
\pgfpathmoveto{\pgfqpoint{0.750000in}{2.577143in}}%
\pgfpathlineto{\pgfqpoint{4.900000in}{2.577143in}}%
\pgfusepath{stroke}%
\end{pgfscope}%
\begin{pgfscope}%
\pgfpathrectangle{\pgfqpoint{0.750000in}{0.520000in}}{\pgfqpoint{4.150000in}{3.200000in}}%
\pgfusepath{clip}%
\pgfsetroundcap%
\pgfsetroundjoin%
\pgfsetlinewidth{0.702625pt}%
\definecolor{currentstroke}{rgb}{1.000000,1.000000,1.000000}%
\pgfsetstrokecolor{currentstroke}%
\pgfsetdash{}{0pt}%
\pgfpathmoveto{\pgfqpoint{0.750000in}{3.034286in}}%
\pgfpathlineto{\pgfqpoint{4.900000in}{3.034286in}}%
\pgfusepath{stroke}%
\end{pgfscope}%
\begin{pgfscope}%
\pgfpathrectangle{\pgfqpoint{0.750000in}{0.520000in}}{\pgfqpoint{4.150000in}{3.200000in}}%
\pgfusepath{clip}%
\pgfsetroundcap%
\pgfsetroundjoin%
\pgfsetlinewidth{0.702625pt}%
\definecolor{currentstroke}{rgb}{1.000000,1.000000,1.000000}%
\pgfsetstrokecolor{currentstroke}%
\pgfsetdash{}{0pt}%
\pgfpathmoveto{\pgfqpoint{0.750000in}{3.491429in}}%
\pgfpathlineto{\pgfqpoint{4.900000in}{3.491429in}}%
\pgfusepath{stroke}%
\end{pgfscope}%
\begin{pgfscope}%
\definecolor{textcolor}{rgb}{0.150000,0.150000,0.150000}%
\pgfsetstrokecolor{textcolor}%
\pgfsetfillcolor{textcolor}%
\pgftext[x=0.219629in,y=2.120000in,,bottom,rotate=90.000000]{\color{textcolor}\sffamily\fontsize{12.000000}{14.400000}\selectfont \(\displaystyle f(x)\)}%
\end{pgfscope}%
\begin{pgfscope}%
\pgfpathrectangle{\pgfqpoint{0.750000in}{0.520000in}}{\pgfqpoint{4.150000in}{3.200000in}}%
\pgfusepath{clip}%
\pgfsetbuttcap%
\pgfsetroundjoin%
\definecolor{currentfill}{rgb}{0.000000,0.000000,0.000000}%
\pgfsetfillcolor{currentfill}%
\pgfsetlinewidth{1.003750pt}%
\definecolor{currentstroke}{rgb}{0.000000,0.000000,0.000000}%
\pgfsetstrokecolor{currentstroke}%
\pgfsetdash{}{0pt}%
\pgfsys@defobject{currentmarker}{\pgfqpoint{-0.020833in}{-0.020833in}}{\pgfqpoint{0.020833in}{0.020833in}}{%
\pgfpathmoveto{\pgfqpoint{0.000000in}{-0.020833in}}%
\pgfpathcurveto{\pgfqpoint{0.005525in}{-0.020833in}}{\pgfqpoint{0.010825in}{-0.018638in}}{\pgfqpoint{0.014731in}{-0.014731in}}%
\pgfpathcurveto{\pgfqpoint{0.018638in}{-0.010825in}}{\pgfqpoint{0.020833in}{-0.005525in}}{\pgfqpoint{0.020833in}{0.000000in}}%
\pgfpathcurveto{\pgfqpoint{0.020833in}{0.005525in}}{\pgfqpoint{0.018638in}{0.010825in}}{\pgfqpoint{0.014731in}{0.014731in}}%
\pgfpathcurveto{\pgfqpoint{0.010825in}{0.018638in}}{\pgfqpoint{0.005525in}{0.020833in}}{\pgfqpoint{0.000000in}{0.020833in}}%
\pgfpathcurveto{\pgfqpoint{-0.005525in}{0.020833in}}{\pgfqpoint{-0.010825in}{0.018638in}}{\pgfqpoint{-0.014731in}{0.014731in}}%
\pgfpathcurveto{\pgfqpoint{-0.018638in}{0.010825in}}{\pgfqpoint{-0.020833in}{0.005525in}}{\pgfqpoint{-0.020833in}{0.000000in}}%
\pgfpathcurveto{\pgfqpoint{-0.020833in}{-0.005525in}}{\pgfqpoint{-0.018638in}{-0.010825in}}{\pgfqpoint{-0.014731in}{-0.014731in}}%
\pgfpathcurveto{\pgfqpoint{-0.010825in}{-0.018638in}}{\pgfqpoint{-0.005525in}{-0.020833in}}{\pgfqpoint{0.000000in}{-0.020833in}}%
\pgfpathclose%
\pgfusepath{stroke,fill}%
}%
\begin{pgfscope}%
\pgfsys@transformshift{3.027576in}{2.757289in}%
\pgfsys@useobject{currentmarker}{}%
\end{pgfscope}%
\begin{pgfscope}%
\pgfsys@transformshift{3.718036in}{2.906160in}%
\pgfsys@useobject{currentmarker}{}%
\end{pgfscope}%
\begin{pgfscope}%
\pgfsys@transformshift{3.251468in}{2.747962in}%
\pgfsys@useobject{currentmarker}{}%
\end{pgfscope}%
\begin{pgfscope}%
\pgfsys@transformshift{3.011265in}{2.535412in}%
\pgfsys@useobject{currentmarker}{}%
\end{pgfscope}%
\begin{pgfscope}%
\pgfsys@transformshift{2.508167in}{2.158410in}%
\pgfsys@useobject{currentmarker}{}%
\end{pgfscope}%
\begin{pgfscope}%
\pgfsys@transformshift{3.430461in}{2.857966in}%
\pgfsys@useobject{currentmarker}{}%
\end{pgfscope}%
\begin{pgfscope}%
\pgfsys@transformshift{2.565987in}{2.496336in}%
\pgfsys@useobject{currentmarker}{}%
\end{pgfscope}%
\begin{pgfscope}%
\pgfsys@transformshift{4.450858in}{3.179958in}%
\pgfsys@useobject{currentmarker}{}%
\end{pgfscope}%
\begin{pgfscope}%
\pgfsys@transformshift{4.749200in}{3.586789in}%
\pgfsys@useobject{currentmarker}{}%
\end{pgfscope}%
\begin{pgfscope}%
\pgfsys@transformshift{2.341282in}{2.185322in}%
\pgfsys@useobject{currentmarker}{}%
\end{pgfscope}%
\begin{pgfscope}%
\pgfsys@transformshift{4.035659in}{3.069052in}%
\pgfsys@useobject{currentmarker}{}%
\end{pgfscope}%
\begin{pgfscope}%
\pgfsys@transformshift{2.944914in}{2.567151in}%
\pgfsys@useobject{currentmarker}{}%
\end{pgfscope}%
\begin{pgfscope}%
\pgfsys@transformshift{3.107385in}{2.795993in}%
\pgfsys@useobject{currentmarker}{}%
\end{pgfscope}%
\begin{pgfscope}%
\pgfsys@transformshift{4.591226in}{3.444004in}%
\pgfsys@useobject{currentmarker}{}%
\end{pgfscope}%
\begin{pgfscope}%
\pgfsys@transformshift{1.044800in}{1.761204in}%
\pgfsys@useobject{currentmarker}{}%
\end{pgfscope}%
\begin{pgfscope}%
\pgfsys@transformshift{1.111587in}{1.811040in}%
\pgfsys@useobject{currentmarker}{}%
\end{pgfscope}%
\begin{pgfscope}%
\pgfsys@transformshift{0.833906in}{1.572945in}%
\pgfsys@useobject{currentmarker}{}%
\end{pgfscope}%
\begin{pgfscope}%
\pgfsys@transformshift{4.205372in}{2.958546in}%
\pgfsys@useobject{currentmarker}{}%
\end{pgfscope}%
\begin{pgfscope}%
\pgfsys@transformshift{3.979351in}{3.008249in}%
\pgfsys@useobject{currentmarker}{}%
\end{pgfscope}%
\begin{pgfscope}%
\pgfsys@transformshift{4.360550in}{3.222317in}%
\pgfsys@useobject{currentmarker}{}%
\end{pgfscope}%
\begin{pgfscope}%
\pgfsys@transformshift{4.485000in}{1.434286in}%
\pgfsys@useobject{currentmarker}{}%
\end{pgfscope}%
\end{pgfscope}%
\begin{pgfscope}%
\pgfpathrectangle{\pgfqpoint{0.750000in}{0.520000in}}{\pgfqpoint{4.150000in}{3.200000in}}%
\pgfusepath{clip}%
\pgfsetroundcap%
\pgfsetroundjoin%
\pgfsetlinewidth{1.505625pt}%
\definecolor{currentstroke}{rgb}{0.258824,0.521569,0.956863}%
\pgfsetstrokecolor{currentstroke}%
\pgfsetstrokeopacity{0.800000}%
\pgfsetdash{}{0pt}%
\pgfpathmoveto{\pgfqpoint{0.750000in}{1.632293in}}%
\pgfpathlineto{\pgfqpoint{0.834694in}{1.669329in}}%
\pgfpathlineto{\pgfqpoint{0.919388in}{1.706364in}}%
\pgfpathlineto{\pgfqpoint{1.004082in}{1.743399in}}%
\pgfpathlineto{\pgfqpoint{1.088776in}{1.780434in}}%
\pgfpathlineto{\pgfqpoint{1.173469in}{1.817469in}}%
\pgfpathlineto{\pgfqpoint{1.258163in}{1.854504in}}%
\pgfpathlineto{\pgfqpoint{1.342857in}{1.891539in}}%
\pgfpathlineto{\pgfqpoint{1.427551in}{1.928574in}}%
\pgfpathlineto{\pgfqpoint{1.512245in}{1.965609in}}%
\pgfpathlineto{\pgfqpoint{1.596939in}{2.002644in}}%
\pgfpathlineto{\pgfqpoint{1.681633in}{2.039679in}}%
\pgfpathlineto{\pgfqpoint{1.766327in}{2.076715in}}%
\pgfpathlineto{\pgfqpoint{1.851020in}{2.113750in}}%
\pgfpathlineto{\pgfqpoint{1.935714in}{2.150785in}}%
\pgfpathlineto{\pgfqpoint{2.020408in}{2.187820in}}%
\pgfpathlineto{\pgfqpoint{2.105102in}{2.224855in}}%
\pgfpathlineto{\pgfqpoint{2.189796in}{2.261890in}}%
\pgfpathlineto{\pgfqpoint{2.274490in}{2.298925in}}%
\pgfpathlineto{\pgfqpoint{2.359184in}{2.335960in}}%
\pgfpathlineto{\pgfqpoint{2.443878in}{2.372995in}}%
\pgfpathlineto{\pgfqpoint{2.528571in}{2.410030in}}%
\pgfpathlineto{\pgfqpoint{2.613265in}{2.447065in}}%
\pgfpathlineto{\pgfqpoint{2.697959in}{2.484101in}}%
\pgfpathlineto{\pgfqpoint{2.782653in}{2.521136in}}%
\pgfpathlineto{\pgfqpoint{2.867347in}{2.558171in}}%
\pgfpathlineto{\pgfqpoint{2.952041in}{2.595206in}}%
\pgfpathlineto{\pgfqpoint{3.036735in}{2.632241in}}%
\pgfpathlineto{\pgfqpoint{3.121429in}{2.669276in}}%
\pgfpathlineto{\pgfqpoint{3.206122in}{2.706311in}}%
\pgfpathlineto{\pgfqpoint{3.290816in}{2.743346in}}%
\pgfpathlineto{\pgfqpoint{3.375510in}{2.780381in}}%
\pgfpathlineto{\pgfqpoint{3.460204in}{2.817416in}}%
\pgfpathlineto{\pgfqpoint{3.544898in}{2.854451in}}%
\pgfpathlineto{\pgfqpoint{3.629592in}{2.891487in}}%
\pgfpathlineto{\pgfqpoint{3.714286in}{2.928522in}}%
\pgfpathlineto{\pgfqpoint{3.798980in}{2.965557in}}%
\pgfpathlineto{\pgfqpoint{3.883673in}{3.002592in}}%
\pgfpathlineto{\pgfqpoint{3.968367in}{3.039627in}}%
\pgfpathlineto{\pgfqpoint{4.053061in}{3.076662in}}%
\pgfpathlineto{\pgfqpoint{4.137755in}{3.113697in}}%
\pgfpathlineto{\pgfqpoint{4.222449in}{3.150732in}}%
\pgfpathlineto{\pgfqpoint{4.307143in}{3.187767in}}%
\pgfpathlineto{\pgfqpoint{4.391837in}{3.224802in}}%
\pgfpathlineto{\pgfqpoint{4.476531in}{3.261837in}}%
\pgfpathlineto{\pgfqpoint{4.561224in}{3.298873in}}%
\pgfpathlineto{\pgfqpoint{4.645918in}{3.335908in}}%
\pgfpathlineto{\pgfqpoint{4.730612in}{3.372943in}}%
\pgfpathlineto{\pgfqpoint{4.815306in}{3.409978in}}%
\pgfpathlineto{\pgfqpoint{4.900000in}{3.447013in}}%
\pgfusepath{stroke}%
\end{pgfscope}%
\begin{pgfscope}%
\pgfpathrectangle{\pgfqpoint{0.750000in}{0.520000in}}{\pgfqpoint{4.150000in}{3.200000in}}%
\pgfusepath{clip}%
\pgfsetroundcap%
\pgfsetroundjoin%
\pgfsetlinewidth{1.505625pt}%
\definecolor{currentstroke}{rgb}{0.917647,0.262745,0.207843}%
\pgfsetstrokecolor{currentstroke}%
\pgfsetstrokeopacity{0.800000}%
\pgfsetdash{}{0pt}%
\pgfpathmoveto{\pgfqpoint{0.750000in}{1.687067in}}%
\pgfpathlineto{\pgfqpoint{0.834694in}{1.719033in}}%
\pgfpathlineto{\pgfqpoint{0.919388in}{1.750999in}}%
\pgfpathlineto{\pgfqpoint{1.004082in}{1.782965in}}%
\pgfpathlineto{\pgfqpoint{1.088776in}{1.814931in}}%
\pgfpathlineto{\pgfqpoint{1.173469in}{1.846897in}}%
\pgfpathlineto{\pgfqpoint{1.258163in}{1.878864in}}%
\pgfpathlineto{\pgfqpoint{1.342857in}{1.910830in}}%
\pgfpathlineto{\pgfqpoint{1.427551in}{1.942796in}}%
\pgfpathlineto{\pgfqpoint{1.512245in}{1.974762in}}%
\pgfpathlineto{\pgfqpoint{1.596939in}{2.006728in}}%
\pgfpathlineto{\pgfqpoint{1.681633in}{2.038695in}}%
\pgfpathlineto{\pgfqpoint{1.766327in}{2.070661in}}%
\pgfpathlineto{\pgfqpoint{1.851020in}{2.102627in}}%
\pgfpathlineto{\pgfqpoint{1.935714in}{2.134593in}}%
\pgfpathlineto{\pgfqpoint{2.020408in}{2.166559in}}%
\pgfpathlineto{\pgfqpoint{2.105102in}{2.198525in}}%
\pgfpathlineto{\pgfqpoint{2.189796in}{2.230492in}}%
\pgfpathlineto{\pgfqpoint{2.274490in}{2.262458in}}%
\pgfpathlineto{\pgfqpoint{2.359184in}{2.294424in}}%
\pgfpathlineto{\pgfqpoint{2.443878in}{2.326390in}}%
\pgfpathlineto{\pgfqpoint{2.528571in}{2.358356in}}%
\pgfpathlineto{\pgfqpoint{2.613265in}{2.390323in}}%
\pgfpathlineto{\pgfqpoint{2.697959in}{2.422289in}}%
\pgfpathlineto{\pgfqpoint{2.782653in}{2.454255in}}%
\pgfpathlineto{\pgfqpoint{2.867347in}{2.486221in}}%
\pgfpathlineto{\pgfqpoint{2.952041in}{2.518187in}}%
\pgfpathlineto{\pgfqpoint{3.036735in}{2.550153in}}%
\pgfpathlineto{\pgfqpoint{3.121429in}{2.582120in}}%
\pgfpathlineto{\pgfqpoint{3.206122in}{2.614086in}}%
\pgfpathlineto{\pgfqpoint{3.290816in}{2.646052in}}%
\pgfpathlineto{\pgfqpoint{3.375510in}{2.678018in}}%
\pgfpathlineto{\pgfqpoint{3.460204in}{2.709984in}}%
\pgfpathlineto{\pgfqpoint{3.544898in}{2.741951in}}%
\pgfpathlineto{\pgfqpoint{3.629592in}{2.773917in}}%
\pgfpathlineto{\pgfqpoint{3.714286in}{2.805883in}}%
\pgfpathlineto{\pgfqpoint{3.798980in}{2.837849in}}%
\pgfpathlineto{\pgfqpoint{3.883673in}{2.869815in}}%
\pgfpathlineto{\pgfqpoint{3.968367in}{2.901781in}}%
\pgfpathlineto{\pgfqpoint{4.053061in}{2.933748in}}%
\pgfpathlineto{\pgfqpoint{4.137755in}{2.965714in}}%
\pgfpathlineto{\pgfqpoint{4.222449in}{2.997680in}}%
\pgfpathlineto{\pgfqpoint{4.307143in}{3.029646in}}%
\pgfpathlineto{\pgfqpoint{4.391837in}{3.061612in}}%
\pgfpathlineto{\pgfqpoint{4.476531in}{3.093579in}}%
\pgfpathlineto{\pgfqpoint{4.561224in}{3.125545in}}%
\pgfpathlineto{\pgfqpoint{4.645918in}{3.157511in}}%
\pgfpathlineto{\pgfqpoint{4.730612in}{3.189477in}}%
\pgfpathlineto{\pgfqpoint{4.815306in}{3.221443in}}%
\pgfpathlineto{\pgfqpoint{4.900000in}{3.253410in}}%
\pgfusepath{stroke}%
\end{pgfscope}%
\begin{pgfscope}%
\pgfpathrectangle{\pgfqpoint{0.750000in}{0.520000in}}{\pgfqpoint{4.150000in}{3.200000in}}%
\pgfusepath{clip}%
\pgfsetroundcap%
\pgfsetroundjoin%
\pgfsetlinewidth{1.505625pt}%
\definecolor{currentstroke}{rgb}{0.203922,0.658824,0.325490}%
\pgfsetstrokecolor{currentstroke}%
\pgfsetstrokeopacity{0.800000}%
\pgfsetdash{}{0pt}%
\pgfpathmoveto{\pgfqpoint{0.750000in}{0.647910in}}%
\pgfpathlineto{\pgfqpoint{0.834694in}{0.688704in}}%
\pgfpathlineto{\pgfqpoint{0.919388in}{0.729498in}}%
\pgfpathlineto{\pgfqpoint{1.004082in}{0.770291in}}%
\pgfpathlineto{\pgfqpoint{1.088776in}{0.811085in}}%
\pgfpathlineto{\pgfqpoint{1.173469in}{0.851879in}}%
\pgfpathlineto{\pgfqpoint{1.258163in}{0.892672in}}%
\pgfpathlineto{\pgfqpoint{1.342857in}{0.933466in}}%
\pgfpathlineto{\pgfqpoint{1.427551in}{0.974260in}}%
\pgfpathlineto{\pgfqpoint{1.512245in}{1.015053in}}%
\pgfpathlineto{\pgfqpoint{1.596939in}{1.055847in}}%
\pgfpathlineto{\pgfqpoint{1.681633in}{1.096640in}}%
\pgfpathlineto{\pgfqpoint{1.766327in}{1.137434in}}%
\pgfpathlineto{\pgfqpoint{1.851020in}{1.178228in}}%
\pgfpathlineto{\pgfqpoint{1.935714in}{1.219021in}}%
\pgfpathlineto{\pgfqpoint{2.020408in}{1.259815in}}%
\pgfpathlineto{\pgfqpoint{2.105102in}{1.300609in}}%
\pgfpathlineto{\pgfqpoint{2.189796in}{1.341402in}}%
\pgfpathlineto{\pgfqpoint{2.274490in}{1.382196in}}%
\pgfpathlineto{\pgfqpoint{2.359184in}{1.422990in}}%
\pgfpathlineto{\pgfqpoint{2.443878in}{1.463783in}}%
\pgfpathlineto{\pgfqpoint{2.528571in}{1.504577in}}%
\pgfpathlineto{\pgfqpoint{2.613265in}{1.545371in}}%
\pgfpathlineto{\pgfqpoint{2.697959in}{1.586164in}}%
\pgfpathlineto{\pgfqpoint{2.782653in}{1.626958in}}%
\pgfpathlineto{\pgfqpoint{2.867347in}{1.667752in}}%
\pgfpathlineto{\pgfqpoint{2.952041in}{1.708545in}}%
\pgfpathlineto{\pgfqpoint{3.036735in}{1.749339in}}%
\pgfpathlineto{\pgfqpoint{3.121429in}{1.790132in}}%
\pgfpathlineto{\pgfqpoint{3.206122in}{1.830926in}}%
\pgfpathlineto{\pgfqpoint{3.290816in}{1.871720in}}%
\pgfpathlineto{\pgfqpoint{3.375510in}{1.912513in}}%
\pgfpathlineto{\pgfqpoint{3.460204in}{1.953307in}}%
\pgfpathlineto{\pgfqpoint{3.544898in}{1.994101in}}%
\pgfpathlineto{\pgfqpoint{3.629592in}{2.034894in}}%
\pgfpathlineto{\pgfqpoint{3.714286in}{2.075688in}}%
\pgfpathlineto{\pgfqpoint{3.798980in}{2.116482in}}%
\pgfpathlineto{\pgfqpoint{3.883673in}{2.157275in}}%
\pgfpathlineto{\pgfqpoint{3.968367in}{2.198069in}}%
\pgfpathlineto{\pgfqpoint{4.053061in}{2.238863in}}%
\pgfpathlineto{\pgfqpoint{4.137755in}{2.279656in}}%
\pgfpathlineto{\pgfqpoint{4.222449in}{2.320450in}}%
\pgfpathlineto{\pgfqpoint{4.307143in}{2.361244in}}%
\pgfpathlineto{\pgfqpoint{4.391837in}{2.402037in}}%
\pgfpathlineto{\pgfqpoint{4.476531in}{2.442831in}}%
\pgfpathlineto{\pgfqpoint{4.561224in}{2.483624in}}%
\pgfpathlineto{\pgfqpoint{4.645918in}{2.524418in}}%
\pgfpathlineto{\pgfqpoint{4.730612in}{2.565212in}}%
\pgfpathlineto{\pgfqpoint{4.815306in}{2.606005in}}%
\pgfpathlineto{\pgfqpoint{4.900000in}{2.646799in}}%
\pgfusepath{stroke}%
\end{pgfscope}%
\begin{pgfscope}%
\pgfsetrectcap%
\pgfsetmiterjoin%
\pgfsetlinewidth{1.254687pt}%
\definecolor{currentstroke}{rgb}{1.000000,1.000000,1.000000}%
\pgfsetstrokecolor{currentstroke}%
\pgfsetdash{}{0pt}%
\pgfpathmoveto{\pgfqpoint{0.750000in}{0.520000in}}%
\pgfpathlineto{\pgfqpoint{0.750000in}{3.720000in}}%
\pgfusepath{stroke}%
\end{pgfscope}%
\begin{pgfscope}%
\pgfsetrectcap%
\pgfsetmiterjoin%
\pgfsetlinewidth{1.254687pt}%
\definecolor{currentstroke}{rgb}{1.000000,1.000000,1.000000}%
\pgfsetstrokecolor{currentstroke}%
\pgfsetdash{}{0pt}%
\pgfpathmoveto{\pgfqpoint{4.900000in}{0.520000in}}%
\pgfpathlineto{\pgfqpoint{4.900000in}{3.720000in}}%
\pgfusepath{stroke}%
\end{pgfscope}%
\begin{pgfscope}%
\pgfsetrectcap%
\pgfsetmiterjoin%
\pgfsetlinewidth{1.254687pt}%
\definecolor{currentstroke}{rgb}{1.000000,1.000000,1.000000}%
\pgfsetstrokecolor{currentstroke}%
\pgfsetdash{}{0pt}%
\pgfpathmoveto{\pgfqpoint{0.750000in}{0.520000in}}%
\pgfpathlineto{\pgfqpoint{4.900000in}{0.520000in}}%
\pgfusepath{stroke}%
\end{pgfscope}%
\begin{pgfscope}%
\pgfsetrectcap%
\pgfsetmiterjoin%
\pgfsetlinewidth{1.254687pt}%
\definecolor{currentstroke}{rgb}{1.000000,1.000000,1.000000}%
\pgfsetstrokecolor{currentstroke}%
\pgfsetdash{}{0pt}%
\pgfpathmoveto{\pgfqpoint{0.750000in}{3.720000in}}%
\pgfpathlineto{\pgfqpoint{4.900000in}{3.720000in}}%
\pgfusepath{stroke}%
\end{pgfscope}%
\begin{pgfscope}%
\pgfsetfillopacity{0.900000}%
\pgfsetstrokeopacity{0.900000}%
\definecolor{textcolor}{rgb}{0.232941,0.469412,0.861176}%
\pgfsetstrokecolor{textcolor}%
\pgfsetfillcolor{textcolor}%
\pgftext[x=0.874650in, y=1.775406in, left, base,rotate=23.568304]{\color{textcolor}\sffamily\fontsize{10.000000}{12.000000}\selectfont \(\displaystyle L_1\) fit (best outlier rejection)}%
\end{pgfscope}%
\begin{pgfscope}%
\pgfsetfillopacity{0.900000}%
\pgfsetstrokeopacity{0.900000}%
\definecolor{textcolor}{rgb}{0.825882,0.236471,0.187059}%
\pgfsetstrokecolor{textcolor}%
\pgfsetfillcolor{textcolor}%
\pgftext[x=0.952706in, y=1.592065in, left, base,rotate=20.632598]{\color{textcolor}\sffamily\fontsize{10.000000}{12.000000}\selectfont \(\displaystyle L_2\) fit (least-squares)}%
\end{pgfscope}%
\begin{pgfscope}%
\pgfsetfillopacity{0.900000}%
\pgfsetstrokeopacity{0.900000}%
\definecolor{textcolor}{rgb}{0.183529,0.592941,0.292941}%
\pgfsetstrokecolor{textcolor}%
\pgfsetfillcolor{textcolor}%
\pgftext[x=0.878114in, y=0.794177in, left, base,rotate=25.664382]{\color{textcolor}\sffamily\fontsize{10.000000}{12.000000}\selectfont \(\displaystyle L_\infty\) fit (minimizes peak error)}%
\end{pgfscope}%
\begin{pgfscope}%
\definecolor{textcolor}{rgb}{0.150000,0.150000,0.150000}%
\pgfsetstrokecolor{textcolor}%
\pgfsetfillcolor{textcolor}%
\pgftext[x=2.825000in,y=3.803333in,,base]{\color{textcolor}\sffamily\fontsize{12.000000}{14.400000}\selectfont Impact of Norm Choice for Outlier Treatment}%
\end{pgfscope}%
\end{pgfpicture}%
\makeatother%
\endgroup%
}
    \caption{Fitting with various norms on an synthetic dataset with an outlier.}
    \label{fig:fitting-norm}
\end{figure}

Neither approach is universally superior; they simply represent different prior beliefs about the likely source of error as a function of model deviation. Surrogate modeling from a synthetic dataset derived from high-fidelity computational simulation is likely best served by $L_\infty$ fitting, as the noise in the dataset is generally assumed to be zero. The $L_\infty$ fit will enforce the tightest possible bound on the deviation from the high-fidelity dataset.

On the other hand, an $L_1$ fit might be expected to produce superior results for an experimental dataset. This is because the systematic errors sometimes found in experiment can yield outliers that convey no useful information about the underlying physics. The example dataset in Figure \ref{fig:fitting-norm}, which simulates an experimental dataset with measurement dropout, is clearly better served by the robust $L_1$ fit.

%\subsubsection{Weights}

% TODO

\subsubsection{Parameter and Model Bounds}

Another useful feature of the AeroSandbox fitting submodule is the ability to easily perform constrained fitting. This can take two forms:

\begin{enumerate}
    \item \textbf{Parameter bounds}: the vector of fit parameters $\vec{\theta}$ can be directly given bounds constraints, which can be used to stabilize the fit process on nonconvex problems.
    \item \textbf{Model bounds}: the error vector $\vec{e}$ can be constrained such that the fitted model represents either an upper or lower bound on the dataset. This is quite useful in engineering practice, as it allows the creation of surrogate models that can be guaranteed to be conservative with respect to the original dataset. This reduces the likelihood that an optimization problem that includes a fitted model will result in an optimum that is infeasible according to the true underlying physics.
\end{enumerate}

This second process of adding model bounds is demonstrated in the fits in Figure \ref{fig:constrained-fitting}. Here, a drag polar for a SD7032 airfoil at $\text{Re}=10^6$ is fit using \mintinline{python}{asb.FittedModel} and a quadratic model. This quadratic model is a common approximation for an airfoil's profile drag polar; for example, this approximation is seen in the QProp propeller design code by Drela \cite{qprop}. Here, using an upper-bound fit to model profile drag means that downstream optimization using this model will be more robust to surrogate modeling error.

\begin{figure}[H]
    \centering
%    %% Creator: Matplotlib, PGF backend
%%
%% To include the figure in your LaTeX document, write
%%   \input{<filename>.pgf}
%%
%% Make sure the required packages are loaded in your preamble
%%   \usepackage{pgf}
%%
%% Figures using additional raster images can only be included by \input if
%% they are in the same directory as the main LaTeX file. For loading figures
%% from other directories you can use the `import` package
%%   \usepackage{import}
%%
%% and then include the figures with
%%   \import{<path to file>}{<filename>.pgf}
%%
%% Matplotlib used the following preamble
%%   \usepackage{fontspec}
%%
\begingroup%
\makeatletter%
\begin{pgfpicture}%
\pgfpathrectangle{\pgfpointorigin}{\pgfqpoint{6.000000in}{4.000000in}}%
\pgfusepath{use as bounding box, clip}%
\begin{pgfscope}%
\pgfsetbuttcap%
\pgfsetmiterjoin%
\definecolor{currentfill}{rgb}{1.000000,1.000000,1.000000}%
\pgfsetfillcolor{currentfill}%
\pgfsetlinewidth{0.000000pt}%
\definecolor{currentstroke}{rgb}{1.000000,1.000000,1.000000}%
\pgfsetstrokecolor{currentstroke}%
\pgfsetstrokeopacity{0.000000}%
\pgfsetdash{}{0pt}%
\pgfpathmoveto{\pgfqpoint{0.000000in}{0.000000in}}%
\pgfpathlineto{\pgfqpoint{6.000000in}{0.000000in}}%
\pgfpathlineto{\pgfqpoint{6.000000in}{4.000000in}}%
\pgfpathlineto{\pgfqpoint{0.000000in}{4.000000in}}%
\pgfpathclose%
\pgfusepath{fill}%
\end{pgfscope}%
\begin{pgfscope}%
\pgfsetbuttcap%
\pgfsetmiterjoin%
\definecolor{currentfill}{rgb}{0.917647,0.917647,0.949020}%
\pgfsetfillcolor{currentfill}%
\pgfsetlinewidth{0.000000pt}%
\definecolor{currentstroke}{rgb}{0.000000,0.000000,0.000000}%
\pgfsetstrokecolor{currentstroke}%
\pgfsetstrokeopacity{0.000000}%
\pgfsetdash{}{0pt}%
\pgfpathmoveto{\pgfqpoint{0.845000in}{0.692500in}}%
\pgfpathlineto{\pgfqpoint{5.734687in}{0.692500in}}%
\pgfpathlineto{\pgfqpoint{5.734687in}{3.572667in}}%
\pgfpathlineto{\pgfqpoint{0.845000in}{3.572667in}}%
\pgfpathclose%
\pgfusepath{fill}%
\end{pgfscope}%
\begin{pgfscope}%
\pgfpathrectangle{\pgfqpoint{0.845000in}{0.692500in}}{\pgfqpoint{4.889687in}{2.880167in}}%
\pgfusepath{clip}%
\pgfsetroundcap%
\pgfsetroundjoin%
\pgfsetlinewidth{1.606000pt}%
\definecolor{currentstroke}{rgb}{1.000000,1.000000,1.000000}%
\pgfsetstrokecolor{currentstroke}%
\pgfsetdash{}{0pt}%
\pgfpathmoveto{\pgfqpoint{1.132629in}{0.692500in}}%
\pgfpathlineto{\pgfqpoint{1.132629in}{3.572667in}}%
\pgfusepath{stroke}%
\end{pgfscope}%
\begin{pgfscope}%
\definecolor{textcolor}{rgb}{0.150000,0.150000,0.150000}%
\pgfsetstrokecolor{textcolor}%
\pgfsetfillcolor{textcolor}%
\pgftext[x=1.132629in,y=0.560556in,,top]{\color{textcolor}\sffamily\fontsize{11.000000}{13.200000}\selectfont \ensuremath{-}2}%
\end{pgfscope}%
\begin{pgfscope}%
\pgfpathrectangle{\pgfqpoint{0.845000in}{0.692500in}}{\pgfqpoint{4.889687in}{2.880167in}}%
\pgfusepath{clip}%
\pgfsetroundcap%
\pgfsetroundjoin%
\pgfsetlinewidth{1.606000pt}%
\definecolor{currentstroke}{rgb}{1.000000,1.000000,1.000000}%
\pgfsetstrokecolor{currentstroke}%
\pgfsetdash{}{0pt}%
\pgfpathmoveto{\pgfqpoint{1.707886in}{0.692500in}}%
\pgfpathlineto{\pgfqpoint{1.707886in}{3.572667in}}%
\pgfusepath{stroke}%
\end{pgfscope}%
\begin{pgfscope}%
\definecolor{textcolor}{rgb}{0.150000,0.150000,0.150000}%
\pgfsetstrokecolor{textcolor}%
\pgfsetfillcolor{textcolor}%
\pgftext[x=1.707886in,y=0.560556in,,top]{\color{textcolor}\sffamily\fontsize{11.000000}{13.200000}\selectfont 0}%
\end{pgfscope}%
\begin{pgfscope}%
\pgfpathrectangle{\pgfqpoint{0.845000in}{0.692500in}}{\pgfqpoint{4.889687in}{2.880167in}}%
\pgfusepath{clip}%
\pgfsetroundcap%
\pgfsetroundjoin%
\pgfsetlinewidth{1.606000pt}%
\definecolor{currentstroke}{rgb}{1.000000,1.000000,1.000000}%
\pgfsetstrokecolor{currentstroke}%
\pgfsetdash{}{0pt}%
\pgfpathmoveto{\pgfqpoint{2.283143in}{0.692500in}}%
\pgfpathlineto{\pgfqpoint{2.283143in}{3.572667in}}%
\pgfusepath{stroke}%
\end{pgfscope}%
\begin{pgfscope}%
\definecolor{textcolor}{rgb}{0.150000,0.150000,0.150000}%
\pgfsetstrokecolor{textcolor}%
\pgfsetfillcolor{textcolor}%
\pgftext[x=2.283143in,y=0.560556in,,top]{\color{textcolor}\sffamily\fontsize{11.000000}{13.200000}\selectfont 2}%
\end{pgfscope}%
\begin{pgfscope}%
\pgfpathrectangle{\pgfqpoint{0.845000in}{0.692500in}}{\pgfqpoint{4.889687in}{2.880167in}}%
\pgfusepath{clip}%
\pgfsetroundcap%
\pgfsetroundjoin%
\pgfsetlinewidth{1.606000pt}%
\definecolor{currentstroke}{rgb}{1.000000,1.000000,1.000000}%
\pgfsetstrokecolor{currentstroke}%
\pgfsetdash{}{0pt}%
\pgfpathmoveto{\pgfqpoint{2.858401in}{0.692500in}}%
\pgfpathlineto{\pgfqpoint{2.858401in}{3.572667in}}%
\pgfusepath{stroke}%
\end{pgfscope}%
\begin{pgfscope}%
\definecolor{textcolor}{rgb}{0.150000,0.150000,0.150000}%
\pgfsetstrokecolor{textcolor}%
\pgfsetfillcolor{textcolor}%
\pgftext[x=2.858401in,y=0.560556in,,top]{\color{textcolor}\sffamily\fontsize{11.000000}{13.200000}\selectfont 4}%
\end{pgfscope}%
\begin{pgfscope}%
\pgfpathrectangle{\pgfqpoint{0.845000in}{0.692500in}}{\pgfqpoint{4.889687in}{2.880167in}}%
\pgfusepath{clip}%
\pgfsetroundcap%
\pgfsetroundjoin%
\pgfsetlinewidth{1.606000pt}%
\definecolor{currentstroke}{rgb}{1.000000,1.000000,1.000000}%
\pgfsetstrokecolor{currentstroke}%
\pgfsetdash{}{0pt}%
\pgfpathmoveto{\pgfqpoint{3.433658in}{0.692500in}}%
\pgfpathlineto{\pgfqpoint{3.433658in}{3.572667in}}%
\pgfusepath{stroke}%
\end{pgfscope}%
\begin{pgfscope}%
\definecolor{textcolor}{rgb}{0.150000,0.150000,0.150000}%
\pgfsetstrokecolor{textcolor}%
\pgfsetfillcolor{textcolor}%
\pgftext[x=3.433658in,y=0.560556in,,top]{\color{textcolor}\sffamily\fontsize{11.000000}{13.200000}\selectfont 6}%
\end{pgfscope}%
\begin{pgfscope}%
\pgfpathrectangle{\pgfqpoint{0.845000in}{0.692500in}}{\pgfqpoint{4.889687in}{2.880167in}}%
\pgfusepath{clip}%
\pgfsetroundcap%
\pgfsetroundjoin%
\pgfsetlinewidth{1.606000pt}%
\definecolor{currentstroke}{rgb}{1.000000,1.000000,1.000000}%
\pgfsetstrokecolor{currentstroke}%
\pgfsetdash{}{0pt}%
\pgfpathmoveto{\pgfqpoint{4.008915in}{0.692500in}}%
\pgfpathlineto{\pgfqpoint{4.008915in}{3.572667in}}%
\pgfusepath{stroke}%
\end{pgfscope}%
\begin{pgfscope}%
\definecolor{textcolor}{rgb}{0.150000,0.150000,0.150000}%
\pgfsetstrokecolor{textcolor}%
\pgfsetfillcolor{textcolor}%
\pgftext[x=4.008915in,y=0.560556in,,top]{\color{textcolor}\sffamily\fontsize{11.000000}{13.200000}\selectfont 8}%
\end{pgfscope}%
\begin{pgfscope}%
\pgfpathrectangle{\pgfqpoint{0.845000in}{0.692500in}}{\pgfqpoint{4.889687in}{2.880167in}}%
\pgfusepath{clip}%
\pgfsetroundcap%
\pgfsetroundjoin%
\pgfsetlinewidth{1.606000pt}%
\definecolor{currentstroke}{rgb}{1.000000,1.000000,1.000000}%
\pgfsetstrokecolor{currentstroke}%
\pgfsetdash{}{0pt}%
\pgfpathmoveto{\pgfqpoint{4.584173in}{0.692500in}}%
\pgfpathlineto{\pgfqpoint{4.584173in}{3.572667in}}%
\pgfusepath{stroke}%
\end{pgfscope}%
\begin{pgfscope}%
\definecolor{textcolor}{rgb}{0.150000,0.150000,0.150000}%
\pgfsetstrokecolor{textcolor}%
\pgfsetfillcolor{textcolor}%
\pgftext[x=4.584173in,y=0.560556in,,top]{\color{textcolor}\sffamily\fontsize{11.000000}{13.200000}\selectfont 10}%
\end{pgfscope}%
\begin{pgfscope}%
\pgfpathrectangle{\pgfqpoint{0.845000in}{0.692500in}}{\pgfqpoint{4.889687in}{2.880167in}}%
\pgfusepath{clip}%
\pgfsetroundcap%
\pgfsetroundjoin%
\pgfsetlinewidth{1.606000pt}%
\definecolor{currentstroke}{rgb}{1.000000,1.000000,1.000000}%
\pgfsetstrokecolor{currentstroke}%
\pgfsetdash{}{0pt}%
\pgfpathmoveto{\pgfqpoint{5.159430in}{0.692500in}}%
\pgfpathlineto{\pgfqpoint{5.159430in}{3.572667in}}%
\pgfusepath{stroke}%
\end{pgfscope}%
\begin{pgfscope}%
\definecolor{textcolor}{rgb}{0.150000,0.150000,0.150000}%
\pgfsetstrokecolor{textcolor}%
\pgfsetfillcolor{textcolor}%
\pgftext[x=5.159430in,y=0.560556in,,top]{\color{textcolor}\sffamily\fontsize{11.000000}{13.200000}\selectfont 12}%
\end{pgfscope}%
\begin{pgfscope}%
\pgfpathrectangle{\pgfqpoint{0.845000in}{0.692500in}}{\pgfqpoint{4.889687in}{2.880167in}}%
\pgfusepath{clip}%
\pgfsetroundcap%
\pgfsetroundjoin%
\pgfsetlinewidth{1.606000pt}%
\definecolor{currentstroke}{rgb}{1.000000,1.000000,1.000000}%
\pgfsetstrokecolor{currentstroke}%
\pgfsetdash{}{0pt}%
\pgfpathmoveto{\pgfqpoint{5.734687in}{0.692500in}}%
\pgfpathlineto{\pgfqpoint{5.734687in}{3.572667in}}%
\pgfusepath{stroke}%
\end{pgfscope}%
\begin{pgfscope}%
\definecolor{textcolor}{rgb}{0.150000,0.150000,0.150000}%
\pgfsetstrokecolor{textcolor}%
\pgfsetfillcolor{textcolor}%
\pgftext[x=5.734687in,y=0.560556in,,top]{\color{textcolor}\sffamily\fontsize{11.000000}{13.200000}\selectfont 14}%
\end{pgfscope}%
\begin{pgfscope}%
\pgfpathrectangle{\pgfqpoint{0.845000in}{0.692500in}}{\pgfqpoint{4.889687in}{2.880167in}}%
\pgfusepath{clip}%
\pgfsetroundcap%
\pgfsetroundjoin%
\pgfsetlinewidth{0.702625pt}%
\definecolor{currentstroke}{rgb}{1.000000,1.000000,1.000000}%
\pgfsetstrokecolor{currentstroke}%
\pgfsetdash{}{0pt}%
\pgfpathmoveto{\pgfqpoint{0.845000in}{0.692500in}}%
\pgfpathlineto{\pgfqpoint{0.845000in}{3.572667in}}%
\pgfusepath{stroke}%
\end{pgfscope}%
\begin{pgfscope}%
\pgfpathrectangle{\pgfqpoint{0.845000in}{0.692500in}}{\pgfqpoint{4.889687in}{2.880167in}}%
\pgfusepath{clip}%
\pgfsetroundcap%
\pgfsetroundjoin%
\pgfsetlinewidth{0.702625pt}%
\definecolor{currentstroke}{rgb}{1.000000,1.000000,1.000000}%
\pgfsetstrokecolor{currentstroke}%
\pgfsetdash{}{0pt}%
\pgfpathmoveto{\pgfqpoint{1.420257in}{0.692500in}}%
\pgfpathlineto{\pgfqpoint{1.420257in}{3.572667in}}%
\pgfusepath{stroke}%
\end{pgfscope}%
\begin{pgfscope}%
\pgfpathrectangle{\pgfqpoint{0.845000in}{0.692500in}}{\pgfqpoint{4.889687in}{2.880167in}}%
\pgfusepath{clip}%
\pgfsetroundcap%
\pgfsetroundjoin%
\pgfsetlinewidth{0.702625pt}%
\definecolor{currentstroke}{rgb}{1.000000,1.000000,1.000000}%
\pgfsetstrokecolor{currentstroke}%
\pgfsetdash{}{0pt}%
\pgfpathmoveto{\pgfqpoint{1.995515in}{0.692500in}}%
\pgfpathlineto{\pgfqpoint{1.995515in}{3.572667in}}%
\pgfusepath{stroke}%
\end{pgfscope}%
\begin{pgfscope}%
\pgfpathrectangle{\pgfqpoint{0.845000in}{0.692500in}}{\pgfqpoint{4.889687in}{2.880167in}}%
\pgfusepath{clip}%
\pgfsetroundcap%
\pgfsetroundjoin%
\pgfsetlinewidth{0.702625pt}%
\definecolor{currentstroke}{rgb}{1.000000,1.000000,1.000000}%
\pgfsetstrokecolor{currentstroke}%
\pgfsetdash{}{0pt}%
\pgfpathmoveto{\pgfqpoint{2.570772in}{0.692500in}}%
\pgfpathlineto{\pgfqpoint{2.570772in}{3.572667in}}%
\pgfusepath{stroke}%
\end{pgfscope}%
\begin{pgfscope}%
\pgfpathrectangle{\pgfqpoint{0.845000in}{0.692500in}}{\pgfqpoint{4.889687in}{2.880167in}}%
\pgfusepath{clip}%
\pgfsetroundcap%
\pgfsetroundjoin%
\pgfsetlinewidth{0.702625pt}%
\definecolor{currentstroke}{rgb}{1.000000,1.000000,1.000000}%
\pgfsetstrokecolor{currentstroke}%
\pgfsetdash{}{0pt}%
\pgfpathmoveto{\pgfqpoint{3.146029in}{0.692500in}}%
\pgfpathlineto{\pgfqpoint{3.146029in}{3.572667in}}%
\pgfusepath{stroke}%
\end{pgfscope}%
\begin{pgfscope}%
\pgfpathrectangle{\pgfqpoint{0.845000in}{0.692500in}}{\pgfqpoint{4.889687in}{2.880167in}}%
\pgfusepath{clip}%
\pgfsetroundcap%
\pgfsetroundjoin%
\pgfsetlinewidth{0.702625pt}%
\definecolor{currentstroke}{rgb}{1.000000,1.000000,1.000000}%
\pgfsetstrokecolor{currentstroke}%
\pgfsetdash{}{0pt}%
\pgfpathmoveto{\pgfqpoint{3.721287in}{0.692500in}}%
\pgfpathlineto{\pgfqpoint{3.721287in}{3.572667in}}%
\pgfusepath{stroke}%
\end{pgfscope}%
\begin{pgfscope}%
\pgfpathrectangle{\pgfqpoint{0.845000in}{0.692500in}}{\pgfqpoint{4.889687in}{2.880167in}}%
\pgfusepath{clip}%
\pgfsetroundcap%
\pgfsetroundjoin%
\pgfsetlinewidth{0.702625pt}%
\definecolor{currentstroke}{rgb}{1.000000,1.000000,1.000000}%
\pgfsetstrokecolor{currentstroke}%
\pgfsetdash{}{0pt}%
\pgfpathmoveto{\pgfqpoint{4.296544in}{0.692500in}}%
\pgfpathlineto{\pgfqpoint{4.296544in}{3.572667in}}%
\pgfusepath{stroke}%
\end{pgfscope}%
\begin{pgfscope}%
\pgfpathrectangle{\pgfqpoint{0.845000in}{0.692500in}}{\pgfqpoint{4.889687in}{2.880167in}}%
\pgfusepath{clip}%
\pgfsetroundcap%
\pgfsetroundjoin%
\pgfsetlinewidth{0.702625pt}%
\definecolor{currentstroke}{rgb}{1.000000,1.000000,1.000000}%
\pgfsetstrokecolor{currentstroke}%
\pgfsetdash{}{0pt}%
\pgfpathmoveto{\pgfqpoint{4.871801in}{0.692500in}}%
\pgfpathlineto{\pgfqpoint{4.871801in}{3.572667in}}%
\pgfusepath{stroke}%
\end{pgfscope}%
\begin{pgfscope}%
\pgfpathrectangle{\pgfqpoint{0.845000in}{0.692500in}}{\pgfqpoint{4.889687in}{2.880167in}}%
\pgfusepath{clip}%
\pgfsetroundcap%
\pgfsetroundjoin%
\pgfsetlinewidth{0.702625pt}%
\definecolor{currentstroke}{rgb}{1.000000,1.000000,1.000000}%
\pgfsetstrokecolor{currentstroke}%
\pgfsetdash{}{0pt}%
\pgfpathmoveto{\pgfqpoint{5.447059in}{0.692500in}}%
\pgfpathlineto{\pgfqpoint{5.447059in}{3.572667in}}%
\pgfusepath{stroke}%
\end{pgfscope}%
\begin{pgfscope}%
\definecolor{textcolor}{rgb}{0.150000,0.150000,0.150000}%
\pgfsetstrokecolor{textcolor}%
\pgfsetfillcolor{textcolor}%
\pgftext[x=3.289844in,y=0.369333in,,top]{\color{textcolor}\sffamily\fontsize{12.000000}{14.400000}\selectfont Angle of Attack \(\displaystyle \alpha\)}%
\end{pgfscope}%
\begin{pgfscope}%
\pgfpathrectangle{\pgfqpoint{0.845000in}{0.692500in}}{\pgfqpoint{4.889687in}{2.880167in}}%
\pgfusepath{clip}%
\pgfsetroundcap%
\pgfsetroundjoin%
\pgfsetlinewidth{1.606000pt}%
\definecolor{currentstroke}{rgb}{1.000000,1.000000,1.000000}%
\pgfsetstrokecolor{currentstroke}%
\pgfsetdash{}{0pt}%
\pgfpathmoveto{\pgfqpoint{0.845000in}{0.692500in}}%
\pgfpathlineto{\pgfqpoint{5.734687in}{0.692500in}}%
\pgfusepath{stroke}%
\end{pgfscope}%
\begin{pgfscope}%
\definecolor{textcolor}{rgb}{0.150000,0.150000,0.150000}%
\pgfsetstrokecolor{textcolor}%
\pgfsetfillcolor{textcolor}%
\pgftext[x=0.446917in, y=0.639486in, left, base]{\color{textcolor}\sffamily\fontsize{11.000000}{13.200000}\selectfont 0.00}%
\end{pgfscope}%
\begin{pgfscope}%
\pgfpathrectangle{\pgfqpoint{0.845000in}{0.692500in}}{\pgfqpoint{4.889687in}{2.880167in}}%
\pgfusepath{clip}%
\pgfsetroundcap%
\pgfsetroundjoin%
\pgfsetlinewidth{1.606000pt}%
\definecolor{currentstroke}{rgb}{1.000000,1.000000,1.000000}%
\pgfsetstrokecolor{currentstroke}%
\pgfsetdash{}{0pt}%
\pgfpathmoveto{\pgfqpoint{0.845000in}{1.219423in}}%
\pgfpathlineto{\pgfqpoint{5.734687in}{1.219423in}}%
\pgfusepath{stroke}%
\end{pgfscope}%
\begin{pgfscope}%
\definecolor{textcolor}{rgb}{0.150000,0.150000,0.150000}%
\pgfsetstrokecolor{textcolor}%
\pgfsetfillcolor{textcolor}%
\pgftext[x=0.446917in, y=1.166409in, left, base]{\color{textcolor}\sffamily\fontsize{11.000000}{13.200000}\selectfont 0.01}%
\end{pgfscope}%
\begin{pgfscope}%
\pgfpathrectangle{\pgfqpoint{0.845000in}{0.692500in}}{\pgfqpoint{4.889687in}{2.880167in}}%
\pgfusepath{clip}%
\pgfsetroundcap%
\pgfsetroundjoin%
\pgfsetlinewidth{1.606000pt}%
\definecolor{currentstroke}{rgb}{1.000000,1.000000,1.000000}%
\pgfsetstrokecolor{currentstroke}%
\pgfsetdash{}{0pt}%
\pgfpathmoveto{\pgfqpoint{0.845000in}{1.746346in}}%
\pgfpathlineto{\pgfqpoint{5.734687in}{1.746346in}}%
\pgfusepath{stroke}%
\end{pgfscope}%
\begin{pgfscope}%
\definecolor{textcolor}{rgb}{0.150000,0.150000,0.150000}%
\pgfsetstrokecolor{textcolor}%
\pgfsetfillcolor{textcolor}%
\pgftext[x=0.446917in, y=1.693332in, left, base]{\color{textcolor}\sffamily\fontsize{11.000000}{13.200000}\selectfont 0.02}%
\end{pgfscope}%
\begin{pgfscope}%
\pgfpathrectangle{\pgfqpoint{0.845000in}{0.692500in}}{\pgfqpoint{4.889687in}{2.880167in}}%
\pgfusepath{clip}%
\pgfsetroundcap%
\pgfsetroundjoin%
\pgfsetlinewidth{1.606000pt}%
\definecolor{currentstroke}{rgb}{1.000000,1.000000,1.000000}%
\pgfsetstrokecolor{currentstroke}%
\pgfsetdash{}{0pt}%
\pgfpathmoveto{\pgfqpoint{0.845000in}{2.273268in}}%
\pgfpathlineto{\pgfqpoint{5.734687in}{2.273268in}}%
\pgfusepath{stroke}%
\end{pgfscope}%
\begin{pgfscope}%
\definecolor{textcolor}{rgb}{0.150000,0.150000,0.150000}%
\pgfsetstrokecolor{textcolor}%
\pgfsetfillcolor{textcolor}%
\pgftext[x=0.446917in, y=2.220255in, left, base]{\color{textcolor}\sffamily\fontsize{11.000000}{13.200000}\selectfont 0.03}%
\end{pgfscope}%
\begin{pgfscope}%
\pgfpathrectangle{\pgfqpoint{0.845000in}{0.692500in}}{\pgfqpoint{4.889687in}{2.880167in}}%
\pgfusepath{clip}%
\pgfsetroundcap%
\pgfsetroundjoin%
\pgfsetlinewidth{1.606000pt}%
\definecolor{currentstroke}{rgb}{1.000000,1.000000,1.000000}%
\pgfsetstrokecolor{currentstroke}%
\pgfsetdash{}{0pt}%
\pgfpathmoveto{\pgfqpoint{0.845000in}{2.800191in}}%
\pgfpathlineto{\pgfqpoint{5.734687in}{2.800191in}}%
\pgfusepath{stroke}%
\end{pgfscope}%
\begin{pgfscope}%
\definecolor{textcolor}{rgb}{0.150000,0.150000,0.150000}%
\pgfsetstrokecolor{textcolor}%
\pgfsetfillcolor{textcolor}%
\pgftext[x=0.446917in, y=2.747177in, left, base]{\color{textcolor}\sffamily\fontsize{11.000000}{13.200000}\selectfont 0.04}%
\end{pgfscope}%
\begin{pgfscope}%
\pgfpathrectangle{\pgfqpoint{0.845000in}{0.692500in}}{\pgfqpoint{4.889687in}{2.880167in}}%
\pgfusepath{clip}%
\pgfsetroundcap%
\pgfsetroundjoin%
\pgfsetlinewidth{1.606000pt}%
\definecolor{currentstroke}{rgb}{1.000000,1.000000,1.000000}%
\pgfsetstrokecolor{currentstroke}%
\pgfsetdash{}{0pt}%
\pgfpathmoveto{\pgfqpoint{0.845000in}{3.327114in}}%
\pgfpathlineto{\pgfqpoint{5.734687in}{3.327114in}}%
\pgfusepath{stroke}%
\end{pgfscope}%
\begin{pgfscope}%
\definecolor{textcolor}{rgb}{0.150000,0.150000,0.150000}%
\pgfsetstrokecolor{textcolor}%
\pgfsetfillcolor{textcolor}%
\pgftext[x=0.446917in, y=3.274100in, left, base]{\color{textcolor}\sffamily\fontsize{11.000000}{13.200000}\selectfont 0.05}%
\end{pgfscope}%
\begin{pgfscope}%
\pgfpathrectangle{\pgfqpoint{0.845000in}{0.692500in}}{\pgfqpoint{4.889687in}{2.880167in}}%
\pgfusepath{clip}%
\pgfsetroundcap%
\pgfsetroundjoin%
\pgfsetlinewidth{0.702625pt}%
\definecolor{currentstroke}{rgb}{1.000000,1.000000,1.000000}%
\pgfsetstrokecolor{currentstroke}%
\pgfsetdash{}{0pt}%
\pgfpathmoveto{\pgfqpoint{0.845000in}{0.955961in}}%
\pgfpathlineto{\pgfqpoint{5.734687in}{0.955961in}}%
\pgfusepath{stroke}%
\end{pgfscope}%
\begin{pgfscope}%
\pgfpathrectangle{\pgfqpoint{0.845000in}{0.692500in}}{\pgfqpoint{4.889687in}{2.880167in}}%
\pgfusepath{clip}%
\pgfsetroundcap%
\pgfsetroundjoin%
\pgfsetlinewidth{0.702625pt}%
\definecolor{currentstroke}{rgb}{1.000000,1.000000,1.000000}%
\pgfsetstrokecolor{currentstroke}%
\pgfsetdash{}{0pt}%
\pgfpathmoveto{\pgfqpoint{0.845000in}{1.482884in}}%
\pgfpathlineto{\pgfqpoint{5.734687in}{1.482884in}}%
\pgfusepath{stroke}%
\end{pgfscope}%
\begin{pgfscope}%
\pgfpathrectangle{\pgfqpoint{0.845000in}{0.692500in}}{\pgfqpoint{4.889687in}{2.880167in}}%
\pgfusepath{clip}%
\pgfsetroundcap%
\pgfsetroundjoin%
\pgfsetlinewidth{0.702625pt}%
\definecolor{currentstroke}{rgb}{1.000000,1.000000,1.000000}%
\pgfsetstrokecolor{currentstroke}%
\pgfsetdash{}{0pt}%
\pgfpathmoveto{\pgfqpoint{0.845000in}{2.009807in}}%
\pgfpathlineto{\pgfqpoint{5.734687in}{2.009807in}}%
\pgfusepath{stroke}%
\end{pgfscope}%
\begin{pgfscope}%
\pgfpathrectangle{\pgfqpoint{0.845000in}{0.692500in}}{\pgfqpoint{4.889687in}{2.880167in}}%
\pgfusepath{clip}%
\pgfsetroundcap%
\pgfsetroundjoin%
\pgfsetlinewidth{0.702625pt}%
\definecolor{currentstroke}{rgb}{1.000000,1.000000,1.000000}%
\pgfsetstrokecolor{currentstroke}%
\pgfsetdash{}{0pt}%
\pgfpathmoveto{\pgfqpoint{0.845000in}{2.536730in}}%
\pgfpathlineto{\pgfqpoint{5.734687in}{2.536730in}}%
\pgfusepath{stroke}%
\end{pgfscope}%
\begin{pgfscope}%
\pgfpathrectangle{\pgfqpoint{0.845000in}{0.692500in}}{\pgfqpoint{4.889687in}{2.880167in}}%
\pgfusepath{clip}%
\pgfsetroundcap%
\pgfsetroundjoin%
\pgfsetlinewidth{0.702625pt}%
\definecolor{currentstroke}{rgb}{1.000000,1.000000,1.000000}%
\pgfsetstrokecolor{currentstroke}%
\pgfsetdash{}{0pt}%
\pgfpathmoveto{\pgfqpoint{0.845000in}{3.063653in}}%
\pgfpathlineto{\pgfqpoint{5.734687in}{3.063653in}}%
\pgfusepath{stroke}%
\end{pgfscope}%
\begin{pgfscope}%
\definecolor{textcolor}{rgb}{0.150000,0.150000,0.150000}%
\pgfsetstrokecolor{textcolor}%
\pgfsetfillcolor{textcolor}%
\pgftext[x=0.391361in,y=2.132583in,,bottom,rotate=90.000000]{\color{textcolor}\sffamily\fontsize{12.000000}{14.400000}\selectfont Drag Coefficient \(\displaystyle C_D\)}%
\end{pgfscope}%
\begin{pgfscope}%
\pgfpathrectangle{\pgfqpoint{0.845000in}{0.692500in}}{\pgfqpoint{4.889687in}{2.880167in}}%
\pgfusepath{clip}%
\pgfsetbuttcap%
\pgfsetroundjoin%
\definecolor{currentfill}{rgb}{0.000000,0.000000,0.000000}%
\pgfsetfillcolor{currentfill}%
\pgfsetfillopacity{0.600000}%
\pgfsetlinewidth{1.003750pt}%
\definecolor{currentstroke}{rgb}{0.000000,0.000000,0.000000}%
\pgfsetstrokecolor{currentstroke}%
\pgfsetstrokeopacity{0.600000}%
\pgfsetdash{}{0pt}%
\pgfsys@defobject{currentmarker}{\pgfqpoint{-0.020833in}{-0.020833in}}{\pgfqpoint{0.020833in}{0.020833in}}{%
\pgfpathmoveto{\pgfqpoint{0.000000in}{-0.020833in}}%
\pgfpathcurveto{\pgfqpoint{0.005525in}{-0.020833in}}{\pgfqpoint{0.010825in}{-0.018638in}}{\pgfqpoint{0.014731in}{-0.014731in}}%
\pgfpathcurveto{\pgfqpoint{0.018638in}{-0.010825in}}{\pgfqpoint{0.020833in}{-0.005525in}}{\pgfqpoint{0.020833in}{0.000000in}}%
\pgfpathcurveto{\pgfqpoint{0.020833in}{0.005525in}}{\pgfqpoint{0.018638in}{0.010825in}}{\pgfqpoint{0.014731in}{0.014731in}}%
\pgfpathcurveto{\pgfqpoint{0.010825in}{0.018638in}}{\pgfqpoint{0.005525in}{0.020833in}}{\pgfqpoint{0.000000in}{0.020833in}}%
\pgfpathcurveto{\pgfqpoint{-0.005525in}{0.020833in}}{\pgfqpoint{-0.010825in}{0.018638in}}{\pgfqpoint{-0.014731in}{0.014731in}}%
\pgfpathcurveto{\pgfqpoint{-0.018638in}{0.010825in}}{\pgfqpoint{-0.020833in}{0.005525in}}{\pgfqpoint{-0.020833in}{0.000000in}}%
\pgfpathcurveto{\pgfqpoint{-0.020833in}{-0.005525in}}{\pgfqpoint{-0.018638in}{-0.010825in}}{\pgfqpoint{-0.014731in}{-0.014731in}}%
\pgfpathcurveto{\pgfqpoint{-0.010825in}{-0.018638in}}{\pgfqpoint{-0.005525in}{-0.020833in}}{\pgfqpoint{0.000000in}{-0.020833in}}%
\pgfpathclose%
\pgfusepath{stroke,fill}%
}%
\begin{pgfscope}%
\pgfsys@transformshift{1.742977in}{0.987050in}%
\pgfsys@useobject{currentmarker}{}%
\end{pgfscope}%
\begin{pgfscope}%
\pgfsys@transformshift{1.842784in}{0.984415in}%
\pgfsys@useobject{currentmarker}{}%
\end{pgfscope}%
\begin{pgfscope}%
\pgfsys@transformshift{1.942591in}{0.978092in}%
\pgfsys@useobject{currentmarker}{}%
\end{pgfscope}%
\begin{pgfscope}%
\pgfsys@transformshift{2.042398in}{0.978092in}%
\pgfsys@useobject{currentmarker}{}%
\end{pgfscope}%
\begin{pgfscope}%
\pgfsys@transformshift{2.142205in}{0.973877in}%
\pgfsys@useobject{currentmarker}{}%
\end{pgfscope}%
\begin{pgfscope}%
\pgfsys@transformshift{2.242012in}{0.984942in}%
\pgfsys@useobject{currentmarker}{}%
\end{pgfscope}%
\begin{pgfscope}%
\pgfsys@transformshift{2.341820in}{0.994954in}%
\pgfsys@useobject{currentmarker}{}%
\end{pgfscope}%
\begin{pgfscope}%
\pgfsys@transformshift{2.441627in}{1.007600in}%
\pgfsys@useobject{currentmarker}{}%
\end{pgfscope}%
\begin{pgfscope}%
\pgfsys@transformshift{2.541434in}{1.019192in}%
\pgfsys@useobject{currentmarker}{}%
\end{pgfscope}%
\begin{pgfscope}%
\pgfsys@transformshift{2.641241in}{1.030258in}%
\pgfsys@useobject{currentmarker}{}%
\end{pgfscope}%
\begin{pgfscope}%
\pgfsys@transformshift{2.741048in}{1.045538in}%
\pgfsys@useobject{currentmarker}{}%
\end{pgfscope}%
\begin{pgfscope}%
\pgfsys@transformshift{2.840855in}{1.055550in}%
\pgfsys@useobject{currentmarker}{}%
\end{pgfscope}%
\begin{pgfscope}%
\pgfsys@transformshift{2.940663in}{1.075046in}%
\pgfsys@useobject{currentmarker}{}%
\end{pgfscope}%
\begin{pgfscope}%
\pgfsys@transformshift{3.040470in}{1.085584in}%
\pgfsys@useobject{currentmarker}{}%
\end{pgfscope}%
\begin{pgfscope}%
\pgfsys@transformshift{3.140277in}{1.106661in}%
\pgfsys@useobject{currentmarker}{}%
\end{pgfscope}%
\begin{pgfscope}%
\pgfsys@transformshift{3.240084in}{1.118781in}%
\pgfsys@useobject{currentmarker}{}%
\end{pgfscope}%
\begin{pgfscope}%
\pgfsys@transformshift{3.339604in}{1.141438in}%
\pgfsys@useobject{currentmarker}{}%
\end{pgfscope}%
\begin{pgfscope}%
\pgfsys@transformshift{3.439411in}{1.233123in}%
\pgfsys@useobject{currentmarker}{}%
\end{pgfscope}%
\begin{pgfscope}%
\pgfsys@transformshift{3.539218in}{1.293192in}%
\pgfsys@useobject{currentmarker}{}%
\end{pgfscope}%
\begin{pgfscope}%
\pgfsys@transformshift{3.639025in}{1.349046in}%
\pgfsys@useobject{currentmarker}{}%
\end{pgfscope}%
\begin{pgfscope}%
\pgfsys@transformshift{3.738832in}{1.401211in}%
\pgfsys@useobject{currentmarker}{}%
\end{pgfscope}%
\begin{pgfscope}%
\pgfsys@transformshift{3.838639in}{1.428611in}%
\pgfsys@useobject{currentmarker}{}%
\end{pgfscope}%
\begin{pgfscope}%
\pgfsys@transformshift{3.938446in}{1.450742in}%
\pgfsys@useobject{currentmarker}{}%
\end{pgfscope}%
\begin{pgfscope}%
\pgfsys@transformshift{4.038254in}{1.475507in}%
\pgfsys@useobject{currentmarker}{}%
\end{pgfscope}%
\begin{pgfscope}%
\pgfsys@transformshift{4.138061in}{1.501853in}%
\pgfsys@useobject{currentmarker}{}%
\end{pgfscope}%
\begin{pgfscope}%
\pgfsys@transformshift{4.237868in}{1.530307in}%
\pgfsys@useobject{currentmarker}{}%
\end{pgfscope}%
\begin{pgfscope}%
\pgfsys@transformshift{4.337675in}{1.558234in}%
\pgfsys@useobject{currentmarker}{}%
\end{pgfscope}%
\begin{pgfscope}%
\pgfsys@transformshift{4.437482in}{1.590376in}%
\pgfsys@useobject{currentmarker}{}%
\end{pgfscope}%
\begin{pgfscope}%
\pgfsys@transformshift{4.537289in}{1.627261in}%
\pgfsys@useobject{currentmarker}{}%
\end{pgfscope}%
\begin{pgfscope}%
\pgfsys@transformshift{4.637096in}{1.670996in}%
\pgfsys@useobject{currentmarker}{}%
\end{pgfscope}%
\begin{pgfscope}%
\pgfsys@transformshift{4.736904in}{1.721580in}%
\pgfsys@useobject{currentmarker}{}%
\end{pgfscope}%
\begin{pgfscope}%
\pgfsys@transformshift{4.836711in}{1.780069in}%
\pgfsys@useobject{currentmarker}{}%
\end{pgfscope}%
\begin{pgfscope}%
\pgfsys@transformshift{4.936230in}{1.847515in}%
\pgfsys@useobject{currentmarker}{}%
\end{pgfscope}%
\begin{pgfscope}%
\pgfsys@transformshift{5.036037in}{1.926026in}%
\pgfsys@useobject{currentmarker}{}%
\end{pgfscope}%
\begin{pgfscope}%
\pgfsys@transformshift{5.135845in}{2.018238in}%
\pgfsys@useobject{currentmarker}{}%
\end{pgfscope}%
\begin{pgfscope}%
\pgfsys@transformshift{5.235652in}{2.128365in}%
\pgfsys@useobject{currentmarker}{}%
\end{pgfscope}%
\begin{pgfscope}%
\pgfsys@transformshift{5.335459in}{2.260095in}%
\pgfsys@useobject{currentmarker}{}%
\end{pgfscope}%
\begin{pgfscope}%
\pgfsys@transformshift{5.435266in}{2.436088in}%
\pgfsys@useobject{currentmarker}{}%
\end{pgfscope}%
\begin{pgfscope}%
\pgfsys@transformshift{5.535073in}{2.676891in}%
\pgfsys@useobject{currentmarker}{}%
\end{pgfscope}%
\begin{pgfscope}%
\pgfsys@transformshift{5.634880in}{3.011487in}%
\pgfsys@useobject{currentmarker}{}%
\end{pgfscope}%
\begin{pgfscope}%
\pgfsys@transformshift{5.734687in}{3.445672in}%
\pgfsys@useobject{currentmarker}{}%
\end{pgfscope}%
\begin{pgfscope}%
\pgfsys@transformshift{1.643457in}{0.996008in}%
\pgfsys@useobject{currentmarker}{}%
\end{pgfscope}%
\begin{pgfscope}%
\pgfsys@transformshift{1.543650in}{1.000750in}%
\pgfsys@useobject{currentmarker}{}%
\end{pgfscope}%
\begin{pgfscope}%
\pgfsys@transformshift{1.443843in}{1.013923in}%
\pgfsys@useobject{currentmarker}{}%
\end{pgfscope}%
\begin{pgfscope}%
\pgfsys@transformshift{1.344036in}{1.022881in}%
\pgfsys@useobject{currentmarker}{}%
\end{pgfscope}%
\begin{pgfscope}%
\pgfsys@transformshift{1.244229in}{1.051861in}%
\pgfsys@useobject{currentmarker}{}%
\end{pgfscope}%
\begin{pgfscope}%
\pgfsys@transformshift{1.144421in}{1.066088in}%
\pgfsys@useobject{currentmarker}{}%
\end{pgfscope}%
\begin{pgfscope}%
\pgfsys@transformshift{1.044614in}{1.073992in}%
\pgfsys@useobject{currentmarker}{}%
\end{pgfscope}%
\begin{pgfscope}%
\pgfsys@transformshift{0.944807in}{1.080842in}%
\pgfsys@useobject{currentmarker}{}%
\end{pgfscope}%
\begin{pgfscope}%
\pgfsys@transformshift{0.845000in}{1.088219in}%
\pgfsys@useobject{currentmarker}{}%
\end{pgfscope}%
\end{pgfscope}%
\begin{pgfscope}%
\pgfpathrectangle{\pgfqpoint{0.845000in}{0.692500in}}{\pgfqpoint{4.889687in}{2.880167in}}%
\pgfusepath{clip}%
\pgfsetbuttcap%
\pgfsetroundjoin%
\definecolor{currentfill}{rgb}{0.258824,0.521569,0.956863}%
\pgfsetfillcolor{currentfill}%
\pgfsetfillopacity{0.200000}%
\pgfsetlinewidth{1.003750pt}%
\definecolor{currentstroke}{rgb}{0.258824,0.521569,0.956863}%
\pgfsetstrokecolor{currentstroke}%
\pgfsetstrokeopacity{0.200000}%
\pgfsetdash{}{0pt}%
\pgfsys@defobject{currentmarker}{\pgfqpoint{0.845000in}{0.905793in}}{\pgfqpoint{5.734687in}{3.445673in}}{%
\pgfpathmoveto{\pgfqpoint{0.845000in}{1.467906in}}%
\pgfpathlineto{\pgfqpoint{0.845000in}{1.088097in}}%
\pgfpathlineto{\pgfqpoint{0.854799in}{1.085253in}}%
\pgfpathlineto{\pgfqpoint{0.864598in}{1.082431in}}%
\pgfpathlineto{\pgfqpoint{0.874397in}{1.079632in}}%
\pgfpathlineto{\pgfqpoint{0.884196in}{1.076855in}}%
\pgfpathlineto{\pgfqpoint{0.893995in}{1.074101in}}%
\pgfpathlineto{\pgfqpoint{0.903794in}{1.071368in}}%
\pgfpathlineto{\pgfqpoint{0.913593in}{1.068659in}}%
\pgfpathlineto{\pgfqpoint{0.923392in}{1.065971in}}%
\pgfpathlineto{\pgfqpoint{0.933191in}{1.063306in}}%
\pgfpathlineto{\pgfqpoint{0.942990in}{1.060663in}}%
\pgfpathlineto{\pgfqpoint{0.952789in}{1.058043in}}%
\pgfpathlineto{\pgfqpoint{0.962588in}{1.055445in}}%
\pgfpathlineto{\pgfqpoint{0.972387in}{1.052869in}}%
\pgfpathlineto{\pgfqpoint{0.982186in}{1.050316in}}%
\pgfpathlineto{\pgfqpoint{0.991985in}{1.047785in}}%
\pgfpathlineto{\pgfqpoint{1.001784in}{1.045277in}}%
\pgfpathlineto{\pgfqpoint{1.011583in}{1.042790in}}%
\pgfpathlineto{\pgfqpoint{1.021382in}{1.040327in}}%
\pgfpathlineto{\pgfqpoint{1.031180in}{1.037885in}}%
\pgfpathlineto{\pgfqpoint{1.040979in}{1.035466in}}%
\pgfpathlineto{\pgfqpoint{1.050778in}{1.033069in}}%
\pgfpathlineto{\pgfqpoint{1.060577in}{1.030695in}}%
\pgfpathlineto{\pgfqpoint{1.070376in}{1.028343in}}%
\pgfpathlineto{\pgfqpoint{1.080175in}{1.026013in}}%
\pgfpathlineto{\pgfqpoint{1.089974in}{1.023706in}}%
\pgfpathlineto{\pgfqpoint{1.099773in}{1.021421in}}%
\pgfpathlineto{\pgfqpoint{1.109572in}{1.019158in}}%
\pgfpathlineto{\pgfqpoint{1.119371in}{1.016918in}}%
\pgfpathlineto{\pgfqpoint{1.129170in}{1.014700in}}%
\pgfpathlineto{\pgfqpoint{1.138969in}{1.012504in}}%
\pgfpathlineto{\pgfqpoint{1.148768in}{1.010331in}}%
\pgfpathlineto{\pgfqpoint{1.158567in}{1.008180in}}%
\pgfpathlineto{\pgfqpoint{1.168366in}{1.006052in}}%
\pgfpathlineto{\pgfqpoint{1.178165in}{1.003945in}}%
\pgfpathlineto{\pgfqpoint{1.187964in}{1.001862in}}%
\pgfpathlineto{\pgfqpoint{1.197763in}{0.999800in}}%
\pgfpathlineto{\pgfqpoint{1.207562in}{0.997761in}}%
\pgfpathlineto{\pgfqpoint{1.217361in}{0.995744in}}%
\pgfpathlineto{\pgfqpoint{1.227160in}{0.993750in}}%
\pgfpathlineto{\pgfqpoint{1.236959in}{0.991778in}}%
\pgfpathlineto{\pgfqpoint{1.246758in}{0.989828in}}%
\pgfpathlineto{\pgfqpoint{1.256557in}{0.987901in}}%
\pgfpathlineto{\pgfqpoint{1.266356in}{0.985996in}}%
\pgfpathlineto{\pgfqpoint{1.276155in}{0.984114in}}%
\pgfpathlineto{\pgfqpoint{1.285954in}{0.982253in}}%
\pgfpathlineto{\pgfqpoint{1.295753in}{0.980416in}}%
\pgfpathlineto{\pgfqpoint{1.305552in}{0.978600in}}%
\pgfpathlineto{\pgfqpoint{1.315351in}{0.976807in}}%
\pgfpathlineto{\pgfqpoint{1.325150in}{0.975036in}}%
\pgfpathlineto{\pgfqpoint{1.334949in}{0.973288in}}%
\pgfpathlineto{\pgfqpoint{1.344748in}{0.971562in}}%
\pgfpathlineto{\pgfqpoint{1.354547in}{0.969858in}}%
\pgfpathlineto{\pgfqpoint{1.364346in}{0.968177in}}%
\pgfpathlineto{\pgfqpoint{1.374145in}{0.966518in}}%
\pgfpathlineto{\pgfqpoint{1.383944in}{0.964881in}}%
\pgfpathlineto{\pgfqpoint{1.393742in}{0.963267in}}%
\pgfpathlineto{\pgfqpoint{1.403541in}{0.961675in}}%
\pgfpathlineto{\pgfqpoint{1.413340in}{0.960105in}}%
\pgfpathlineto{\pgfqpoint{1.423139in}{0.958558in}}%
\pgfpathlineto{\pgfqpoint{1.432938in}{0.957033in}}%
\pgfpathlineto{\pgfqpoint{1.442737in}{0.955531in}}%
\pgfpathlineto{\pgfqpoint{1.452536in}{0.954050in}}%
\pgfpathlineto{\pgfqpoint{1.462335in}{0.952593in}}%
\pgfpathlineto{\pgfqpoint{1.472134in}{0.951157in}}%
\pgfpathlineto{\pgfqpoint{1.481933in}{0.949744in}}%
\pgfpathlineto{\pgfqpoint{1.491732in}{0.948353in}}%
\pgfpathlineto{\pgfqpoint{1.501531in}{0.946985in}}%
\pgfpathlineto{\pgfqpoint{1.511330in}{0.945639in}}%
\pgfpathlineto{\pgfqpoint{1.521129in}{0.944315in}}%
\pgfpathlineto{\pgfqpoint{1.530928in}{0.943014in}}%
\pgfpathlineto{\pgfqpoint{1.540727in}{0.941735in}}%
\pgfpathlineto{\pgfqpoint{1.550526in}{0.940479in}}%
\pgfpathlineto{\pgfqpoint{1.560325in}{0.939244in}}%
\pgfpathlineto{\pgfqpoint{1.570124in}{0.938033in}}%
\pgfpathlineto{\pgfqpoint{1.579923in}{0.936843in}}%
\pgfpathlineto{\pgfqpoint{1.589722in}{0.935676in}}%
\pgfpathlineto{\pgfqpoint{1.599521in}{0.934531in}}%
\pgfpathlineto{\pgfqpoint{1.609320in}{0.933409in}}%
\pgfpathlineto{\pgfqpoint{1.619119in}{0.932309in}}%
\pgfpathlineto{\pgfqpoint{1.628918in}{0.931231in}}%
\pgfpathlineto{\pgfqpoint{1.638717in}{0.930176in}}%
\pgfpathlineto{\pgfqpoint{1.648516in}{0.929143in}}%
\pgfpathlineto{\pgfqpoint{1.658315in}{0.928132in}}%
\pgfpathlineto{\pgfqpoint{1.668114in}{0.927144in}}%
\pgfpathlineto{\pgfqpoint{1.677913in}{0.926178in}}%
\pgfpathlineto{\pgfqpoint{1.687712in}{0.925234in}}%
\pgfpathlineto{\pgfqpoint{1.697511in}{0.924313in}}%
\pgfpathlineto{\pgfqpoint{1.707310in}{0.923414in}}%
\pgfpathlineto{\pgfqpoint{1.717109in}{0.922538in}}%
\pgfpathlineto{\pgfqpoint{1.726908in}{0.921683in}}%
\pgfpathlineto{\pgfqpoint{1.736707in}{0.920852in}}%
\pgfpathlineto{\pgfqpoint{1.746506in}{0.920042in}}%
\pgfpathlineto{\pgfqpoint{1.756304in}{0.919255in}}%
\pgfpathlineto{\pgfqpoint{1.766103in}{0.918490in}}%
\pgfpathlineto{\pgfqpoint{1.775902in}{0.917748in}}%
\pgfpathlineto{\pgfqpoint{1.785701in}{0.917028in}}%
\pgfpathlineto{\pgfqpoint{1.795500in}{0.916330in}}%
\pgfpathlineto{\pgfqpoint{1.805299in}{0.915655in}}%
\pgfpathlineto{\pgfqpoint{1.815098in}{0.915002in}}%
\pgfpathlineto{\pgfqpoint{1.824897in}{0.914372in}}%
\pgfpathlineto{\pgfqpoint{1.834696in}{0.913764in}}%
\pgfpathlineto{\pgfqpoint{1.844495in}{0.913178in}}%
\pgfpathlineto{\pgfqpoint{1.854294in}{0.912614in}}%
\pgfpathlineto{\pgfqpoint{1.864093in}{0.912073in}}%
\pgfpathlineto{\pgfqpoint{1.873892in}{0.911554in}}%
\pgfpathlineto{\pgfqpoint{1.883691in}{0.911058in}}%
\pgfpathlineto{\pgfqpoint{1.893490in}{0.910584in}}%
\pgfpathlineto{\pgfqpoint{1.903289in}{0.910132in}}%
\pgfpathlineto{\pgfqpoint{1.913088in}{0.909703in}}%
\pgfpathlineto{\pgfqpoint{1.922887in}{0.909296in}}%
\pgfpathlineto{\pgfqpoint{1.932686in}{0.908911in}}%
\pgfpathlineto{\pgfqpoint{1.942485in}{0.908549in}}%
\pgfpathlineto{\pgfqpoint{1.952284in}{0.908209in}}%
\pgfpathlineto{\pgfqpoint{1.962083in}{0.907891in}}%
\pgfpathlineto{\pgfqpoint{1.971882in}{0.907596in}}%
\pgfpathlineto{\pgfqpoint{1.981681in}{0.907323in}}%
\pgfpathlineto{\pgfqpoint{1.991480in}{0.907073in}}%
\pgfpathlineto{\pgfqpoint{2.001279in}{0.906845in}}%
\pgfpathlineto{\pgfqpoint{2.011078in}{0.906639in}}%
\pgfpathlineto{\pgfqpoint{2.020877in}{0.906455in}}%
\pgfpathlineto{\pgfqpoint{2.030676in}{0.906294in}}%
\pgfpathlineto{\pgfqpoint{2.040475in}{0.906156in}}%
\pgfpathlineto{\pgfqpoint{2.050274in}{0.906039in}}%
\pgfpathlineto{\pgfqpoint{2.060073in}{0.905945in}}%
\pgfpathlineto{\pgfqpoint{2.069872in}{0.905874in}}%
\pgfpathlineto{\pgfqpoint{2.079671in}{0.905824in}}%
\pgfpathlineto{\pgfqpoint{2.089470in}{0.905797in}}%
\pgfpathlineto{\pgfqpoint{2.099269in}{0.905793in}}%
\pgfpathlineto{\pgfqpoint{2.109068in}{0.905811in}}%
\pgfpathlineto{\pgfqpoint{2.118866in}{0.905851in}}%
\pgfpathlineto{\pgfqpoint{2.128665in}{0.905913in}}%
\pgfpathlineto{\pgfqpoint{2.138464in}{0.905998in}}%
\pgfpathlineto{\pgfqpoint{2.148263in}{0.906105in}}%
\pgfpathlineto{\pgfqpoint{2.158062in}{0.906235in}}%
\pgfpathlineto{\pgfqpoint{2.167861in}{0.906387in}}%
\pgfpathlineto{\pgfqpoint{2.177660in}{0.906561in}}%
\pgfpathlineto{\pgfqpoint{2.187459in}{0.906758in}}%
\pgfpathlineto{\pgfqpoint{2.197258in}{0.906977in}}%
\pgfpathlineto{\pgfqpoint{2.207057in}{0.907218in}}%
\pgfpathlineto{\pgfqpoint{2.216856in}{0.907482in}}%
\pgfpathlineto{\pgfqpoint{2.226655in}{0.907768in}}%
\pgfpathlineto{\pgfqpoint{2.236454in}{0.908076in}}%
\pgfpathlineto{\pgfqpoint{2.246253in}{0.908407in}}%
\pgfpathlineto{\pgfqpoint{2.256052in}{0.908760in}}%
\pgfpathlineto{\pgfqpoint{2.265851in}{0.909136in}}%
\pgfpathlineto{\pgfqpoint{2.275650in}{0.909534in}}%
\pgfpathlineto{\pgfqpoint{2.285449in}{0.909954in}}%
\pgfpathlineto{\pgfqpoint{2.295248in}{0.910396in}}%
\pgfpathlineto{\pgfqpoint{2.305047in}{0.910861in}}%
\pgfpathlineto{\pgfqpoint{2.314846in}{0.911349in}}%
\pgfpathlineto{\pgfqpoint{2.324645in}{0.911858in}}%
\pgfpathlineto{\pgfqpoint{2.334444in}{0.912390in}}%
\pgfpathlineto{\pgfqpoint{2.344243in}{0.912945in}}%
\pgfpathlineto{\pgfqpoint{2.354042in}{0.913521in}}%
\pgfpathlineto{\pgfqpoint{2.363841in}{0.914120in}}%
\pgfpathlineto{\pgfqpoint{2.373640in}{0.914742in}}%
\pgfpathlineto{\pgfqpoint{2.383439in}{0.915386in}}%
\pgfpathlineto{\pgfqpoint{2.393238in}{0.916052in}}%
\pgfpathlineto{\pgfqpoint{2.403037in}{0.916740in}}%
\pgfpathlineto{\pgfqpoint{2.412836in}{0.917451in}}%
\pgfpathlineto{\pgfqpoint{2.422635in}{0.918184in}}%
\pgfpathlineto{\pgfqpoint{2.432434in}{0.918940in}}%
\pgfpathlineto{\pgfqpoint{2.442233in}{0.919718in}}%
\pgfpathlineto{\pgfqpoint{2.452032in}{0.920518in}}%
\pgfpathlineto{\pgfqpoint{2.461831in}{0.921341in}}%
\pgfpathlineto{\pgfqpoint{2.471630in}{0.922186in}}%
\pgfpathlineto{\pgfqpoint{2.481428in}{0.923053in}}%
\pgfpathlineto{\pgfqpoint{2.491227in}{0.923943in}}%
\pgfpathlineto{\pgfqpoint{2.501026in}{0.924855in}}%
\pgfpathlineto{\pgfqpoint{2.510825in}{0.925789in}}%
\pgfpathlineto{\pgfqpoint{2.520624in}{0.926746in}}%
\pgfpathlineto{\pgfqpoint{2.530423in}{0.927725in}}%
\pgfpathlineto{\pgfqpoint{2.540222in}{0.928727in}}%
\pgfpathlineto{\pgfqpoint{2.550021in}{0.929751in}}%
\pgfpathlineto{\pgfqpoint{2.559820in}{0.930797in}}%
\pgfpathlineto{\pgfqpoint{2.569619in}{0.931865in}}%
\pgfpathlineto{\pgfqpoint{2.579418in}{0.932956in}}%
\pgfpathlineto{\pgfqpoint{2.589217in}{0.934070in}}%
\pgfpathlineto{\pgfqpoint{2.599016in}{0.935205in}}%
\pgfpathlineto{\pgfqpoint{2.608815in}{0.936363in}}%
\pgfpathlineto{\pgfqpoint{2.618614in}{0.937544in}}%
\pgfpathlineto{\pgfqpoint{2.628413in}{0.938746in}}%
\pgfpathlineto{\pgfqpoint{2.638212in}{0.939972in}}%
\pgfpathlineto{\pgfqpoint{2.648011in}{0.941219in}}%
\pgfpathlineto{\pgfqpoint{2.657810in}{0.942489in}}%
\pgfpathlineto{\pgfqpoint{2.667609in}{0.943781in}}%
\pgfpathlineto{\pgfqpoint{2.677408in}{0.945095in}}%
\pgfpathlineto{\pgfqpoint{2.687207in}{0.946432in}}%
\pgfpathlineto{\pgfqpoint{2.697006in}{0.947791in}}%
\pgfpathlineto{\pgfqpoint{2.706805in}{0.949173in}}%
\pgfpathlineto{\pgfqpoint{2.716604in}{0.950577in}}%
\pgfpathlineto{\pgfqpoint{2.726403in}{0.952003in}}%
\pgfpathlineto{\pgfqpoint{2.736202in}{0.953452in}}%
\pgfpathlineto{\pgfqpoint{2.746001in}{0.954923in}}%
\pgfpathlineto{\pgfqpoint{2.755800in}{0.956416in}}%
\pgfpathlineto{\pgfqpoint{2.765599in}{0.957932in}}%
\pgfpathlineto{\pgfqpoint{2.775398in}{0.959470in}}%
\pgfpathlineto{\pgfqpoint{2.785197in}{0.961030in}}%
\pgfpathlineto{\pgfqpoint{2.794996in}{0.962613in}}%
\pgfpathlineto{\pgfqpoint{2.804795in}{0.964218in}}%
\pgfpathlineto{\pgfqpoint{2.814594in}{0.965846in}}%
\pgfpathlineto{\pgfqpoint{2.824393in}{0.967496in}}%
\pgfpathlineto{\pgfqpoint{2.834192in}{0.969168in}}%
\pgfpathlineto{\pgfqpoint{2.843990in}{0.970863in}}%
\pgfpathlineto{\pgfqpoint{2.853789in}{0.972580in}}%
\pgfpathlineto{\pgfqpoint{2.863588in}{0.974319in}}%
\pgfpathlineto{\pgfqpoint{2.873387in}{0.976080in}}%
\pgfpathlineto{\pgfqpoint{2.883186in}{0.977864in}}%
\pgfpathlineto{\pgfqpoint{2.892985in}{0.979671in}}%
\pgfpathlineto{\pgfqpoint{2.902784in}{0.981500in}}%
\pgfpathlineto{\pgfqpoint{2.912583in}{0.983351in}}%
\pgfpathlineto{\pgfqpoint{2.922382in}{0.985224in}}%
\pgfpathlineto{\pgfqpoint{2.932181in}{0.987120in}}%
\pgfpathlineto{\pgfqpoint{2.941980in}{0.989038in}}%
\pgfpathlineto{\pgfqpoint{2.951779in}{0.990978in}}%
\pgfpathlineto{\pgfqpoint{2.961578in}{0.992941in}}%
\pgfpathlineto{\pgfqpoint{2.971377in}{0.994927in}}%
\pgfpathlineto{\pgfqpoint{2.981176in}{0.996934in}}%
\pgfpathlineto{\pgfqpoint{2.990975in}{0.998964in}}%
\pgfpathlineto{\pgfqpoint{3.000774in}{1.001016in}}%
\pgfpathlineto{\pgfqpoint{3.010573in}{1.003091in}}%
\pgfpathlineto{\pgfqpoint{3.020372in}{1.005188in}}%
\pgfpathlineto{\pgfqpoint{3.030171in}{1.007307in}}%
\pgfpathlineto{\pgfqpoint{3.039970in}{1.009449in}}%
\pgfpathlineto{\pgfqpoint{3.049769in}{1.011613in}}%
\pgfpathlineto{\pgfqpoint{3.059568in}{1.013800in}}%
\pgfpathlineto{\pgfqpoint{3.069367in}{1.016008in}}%
\pgfpathlineto{\pgfqpoint{3.079166in}{1.018240in}}%
\pgfpathlineto{\pgfqpoint{3.088965in}{1.020493in}}%
\pgfpathlineto{\pgfqpoint{3.098764in}{1.022769in}}%
\pgfpathlineto{\pgfqpoint{3.108563in}{1.025067in}}%
\pgfpathlineto{\pgfqpoint{3.118362in}{1.027388in}}%
\pgfpathlineto{\pgfqpoint{3.128161in}{1.029731in}}%
\pgfpathlineto{\pgfqpoint{3.137960in}{1.032096in}}%
\pgfpathlineto{\pgfqpoint{3.147759in}{1.034484in}}%
\pgfpathlineto{\pgfqpoint{3.157558in}{1.036894in}}%
\pgfpathlineto{\pgfqpoint{3.167357in}{1.039326in}}%
\pgfpathlineto{\pgfqpoint{3.177156in}{1.041781in}}%
\pgfpathlineto{\pgfqpoint{3.186955in}{1.044258in}}%
\pgfpathlineto{\pgfqpoint{3.196754in}{1.046757in}}%
\pgfpathlineto{\pgfqpoint{3.206552in}{1.049279in}}%
\pgfpathlineto{\pgfqpoint{3.216351in}{1.051823in}}%
\pgfpathlineto{\pgfqpoint{3.226150in}{1.054390in}}%
\pgfpathlineto{\pgfqpoint{3.235949in}{1.056978in}}%
\pgfpathlineto{\pgfqpoint{3.245748in}{1.059590in}}%
\pgfpathlineto{\pgfqpoint{3.255547in}{1.062223in}}%
\pgfpathlineto{\pgfqpoint{3.265346in}{1.064879in}}%
\pgfpathlineto{\pgfqpoint{3.275145in}{1.067557in}}%
\pgfpathlineto{\pgfqpoint{3.284944in}{1.070258in}}%
\pgfpathlineto{\pgfqpoint{3.294743in}{1.072981in}}%
\pgfpathlineto{\pgfqpoint{3.304542in}{1.075726in}}%
\pgfpathlineto{\pgfqpoint{3.314341in}{1.078494in}}%
\pgfpathlineto{\pgfqpoint{3.324140in}{1.081284in}}%
\pgfpathlineto{\pgfqpoint{3.333939in}{1.084097in}}%
\pgfpathlineto{\pgfqpoint{3.343738in}{1.086932in}}%
\pgfpathlineto{\pgfqpoint{3.353537in}{1.089789in}}%
\pgfpathlineto{\pgfqpoint{3.363336in}{1.092668in}}%
\pgfpathlineto{\pgfqpoint{3.373135in}{1.095570in}}%
\pgfpathlineto{\pgfqpoint{3.382934in}{1.098494in}}%
\pgfpathlineto{\pgfqpoint{3.392733in}{1.101441in}}%
\pgfpathlineto{\pgfqpoint{3.402532in}{1.104410in}}%
\pgfpathlineto{\pgfqpoint{3.412331in}{1.107401in}}%
\pgfpathlineto{\pgfqpoint{3.422130in}{1.110415in}}%
\pgfpathlineto{\pgfqpoint{3.431929in}{1.113451in}}%
\pgfpathlineto{\pgfqpoint{3.441728in}{1.116509in}}%
\pgfpathlineto{\pgfqpoint{3.451527in}{1.119590in}}%
\pgfpathlineto{\pgfqpoint{3.461326in}{1.122693in}}%
\pgfpathlineto{\pgfqpoint{3.471125in}{1.125818in}}%
\pgfpathlineto{\pgfqpoint{3.480924in}{1.128966in}}%
\pgfpathlineto{\pgfqpoint{3.490723in}{1.132136in}}%
\pgfpathlineto{\pgfqpoint{3.500522in}{1.135329in}}%
\pgfpathlineto{\pgfqpoint{3.510321in}{1.138544in}}%
\pgfpathlineto{\pgfqpoint{3.520120in}{1.141781in}}%
\pgfpathlineto{\pgfqpoint{3.529919in}{1.145040in}}%
\pgfpathlineto{\pgfqpoint{3.539718in}{1.148322in}}%
\pgfpathlineto{\pgfqpoint{3.549517in}{1.151627in}}%
\pgfpathlineto{\pgfqpoint{3.559316in}{1.154953in}}%
\pgfpathlineto{\pgfqpoint{3.569114in}{1.158302in}}%
\pgfpathlineto{\pgfqpoint{3.578913in}{1.161674in}}%
\pgfpathlineto{\pgfqpoint{3.588712in}{1.165067in}}%
\pgfpathlineto{\pgfqpoint{3.598511in}{1.168483in}}%
\pgfpathlineto{\pgfqpoint{3.608310in}{1.171922in}}%
\pgfpathlineto{\pgfqpoint{3.618109in}{1.175383in}}%
\pgfpathlineto{\pgfqpoint{3.627908in}{1.178866in}}%
\pgfpathlineto{\pgfqpoint{3.637707in}{1.182371in}}%
\pgfpathlineto{\pgfqpoint{3.647506in}{1.185899in}}%
\pgfpathlineto{\pgfqpoint{3.657305in}{1.189449in}}%
\pgfpathlineto{\pgfqpoint{3.667104in}{1.193022in}}%
\pgfpathlineto{\pgfqpoint{3.676903in}{1.196617in}}%
\pgfpathlineto{\pgfqpoint{3.686702in}{1.200234in}}%
\pgfpathlineto{\pgfqpoint{3.696501in}{1.203874in}}%
\pgfpathlineto{\pgfqpoint{3.706300in}{1.207536in}}%
\pgfpathlineto{\pgfqpoint{3.716099in}{1.211220in}}%
\pgfpathlineto{\pgfqpoint{3.725898in}{1.214927in}}%
\pgfpathlineto{\pgfqpoint{3.735697in}{1.218656in}}%
\pgfpathlineto{\pgfqpoint{3.745496in}{1.222407in}}%
\pgfpathlineto{\pgfqpoint{3.755295in}{1.226181in}}%
\pgfpathlineto{\pgfqpoint{3.765094in}{1.229977in}}%
\pgfpathlineto{\pgfqpoint{3.774893in}{1.233796in}}%
\pgfpathlineto{\pgfqpoint{3.784692in}{1.237637in}}%
\pgfpathlineto{\pgfqpoint{3.794491in}{1.241500in}}%
\pgfpathlineto{\pgfqpoint{3.804290in}{1.245386in}}%
\pgfpathlineto{\pgfqpoint{3.814089in}{1.249293in}}%
\pgfpathlineto{\pgfqpoint{3.823888in}{1.253224in}}%
\pgfpathlineto{\pgfqpoint{3.833687in}{1.257176in}}%
\pgfpathlineto{\pgfqpoint{3.843486in}{1.261151in}}%
\pgfpathlineto{\pgfqpoint{3.853285in}{1.265149in}}%
\pgfpathlineto{\pgfqpoint{3.863084in}{1.269168in}}%
\pgfpathlineto{\pgfqpoint{3.872883in}{1.273211in}}%
\pgfpathlineto{\pgfqpoint{3.882682in}{1.277275in}}%
\pgfpathlineto{\pgfqpoint{3.892481in}{1.281362in}}%
\pgfpathlineto{\pgfqpoint{3.902280in}{1.285471in}}%
\pgfpathlineto{\pgfqpoint{3.912079in}{1.289602in}}%
\pgfpathlineto{\pgfqpoint{3.921878in}{1.293756in}}%
\pgfpathlineto{\pgfqpoint{3.931676in}{1.297933in}}%
\pgfpathlineto{\pgfqpoint{3.941475in}{1.302131in}}%
\pgfpathlineto{\pgfqpoint{3.951274in}{1.306352in}}%
\pgfpathlineto{\pgfqpoint{3.961073in}{1.310595in}}%
\pgfpathlineto{\pgfqpoint{3.970872in}{1.314861in}}%
\pgfpathlineto{\pgfqpoint{3.980671in}{1.319149in}}%
\pgfpathlineto{\pgfqpoint{3.990470in}{1.323459in}}%
\pgfpathlineto{\pgfqpoint{4.000269in}{1.327792in}}%
\pgfpathlineto{\pgfqpoint{4.010068in}{1.332147in}}%
\pgfpathlineto{\pgfqpoint{4.019867in}{1.336525in}}%
\pgfpathlineto{\pgfqpoint{4.029666in}{1.340924in}}%
\pgfpathlineto{\pgfqpoint{4.039465in}{1.345347in}}%
\pgfpathlineto{\pgfqpoint{4.049264in}{1.349791in}}%
\pgfpathlineto{\pgfqpoint{4.059063in}{1.354258in}}%
\pgfpathlineto{\pgfqpoint{4.068862in}{1.358747in}}%
\pgfpathlineto{\pgfqpoint{4.078661in}{1.363259in}}%
\pgfpathlineto{\pgfqpoint{4.088460in}{1.367793in}}%
\pgfpathlineto{\pgfqpoint{4.098259in}{1.372349in}}%
\pgfpathlineto{\pgfqpoint{4.108058in}{1.376928in}}%
\pgfpathlineto{\pgfqpoint{4.117857in}{1.381529in}}%
\pgfpathlineto{\pgfqpoint{4.127656in}{1.386152in}}%
\pgfpathlineto{\pgfqpoint{4.137455in}{1.390798in}}%
\pgfpathlineto{\pgfqpoint{4.147254in}{1.395466in}}%
\pgfpathlineto{\pgfqpoint{4.157053in}{1.400156in}}%
\pgfpathlineto{\pgfqpoint{4.166852in}{1.404869in}}%
\pgfpathlineto{\pgfqpoint{4.176651in}{1.409604in}}%
\pgfpathlineto{\pgfqpoint{4.186450in}{1.414362in}}%
\pgfpathlineto{\pgfqpoint{4.196249in}{1.419141in}}%
\pgfpathlineto{\pgfqpoint{4.206048in}{1.423944in}}%
\pgfpathlineto{\pgfqpoint{4.215847in}{1.428768in}}%
\pgfpathlineto{\pgfqpoint{4.225646in}{1.433615in}}%
\pgfpathlineto{\pgfqpoint{4.235445in}{1.438484in}}%
\pgfpathlineto{\pgfqpoint{4.245244in}{1.443376in}}%
\pgfpathlineto{\pgfqpoint{4.255043in}{1.448290in}}%
\pgfpathlineto{\pgfqpoint{4.264842in}{1.453227in}}%
\pgfpathlineto{\pgfqpoint{4.274641in}{1.458185in}}%
\pgfpathlineto{\pgfqpoint{4.284440in}{1.463166in}}%
\pgfpathlineto{\pgfqpoint{4.294238in}{1.468170in}}%
\pgfpathlineto{\pgfqpoint{4.304037in}{1.473196in}}%
\pgfpathlineto{\pgfqpoint{4.313836in}{1.478244in}}%
\pgfpathlineto{\pgfqpoint{4.323635in}{1.483314in}}%
\pgfpathlineto{\pgfqpoint{4.333434in}{1.488407in}}%
\pgfpathlineto{\pgfqpoint{4.343233in}{1.493522in}}%
\pgfpathlineto{\pgfqpoint{4.353032in}{1.498660in}}%
\pgfpathlineto{\pgfqpoint{4.362831in}{1.503820in}}%
\pgfpathlineto{\pgfqpoint{4.372630in}{1.509002in}}%
\pgfpathlineto{\pgfqpoint{4.382429in}{1.514207in}}%
\pgfpathlineto{\pgfqpoint{4.392228in}{1.519434in}}%
\pgfpathlineto{\pgfqpoint{4.402027in}{1.524683in}}%
\pgfpathlineto{\pgfqpoint{4.411826in}{1.529955in}}%
\pgfpathlineto{\pgfqpoint{4.421625in}{1.535249in}}%
\pgfpathlineto{\pgfqpoint{4.431424in}{1.540565in}}%
\pgfpathlineto{\pgfqpoint{4.441223in}{1.545904in}}%
\pgfpathlineto{\pgfqpoint{4.451022in}{1.551265in}}%
\pgfpathlineto{\pgfqpoint{4.460821in}{1.556649in}}%
\pgfpathlineto{\pgfqpoint{4.470620in}{1.562055in}}%
\pgfpathlineto{\pgfqpoint{4.480419in}{1.567483in}}%
\pgfpathlineto{\pgfqpoint{4.490218in}{1.572933in}}%
\pgfpathlineto{\pgfqpoint{4.500017in}{1.578406in}}%
\pgfpathlineto{\pgfqpoint{4.509816in}{1.583902in}}%
\pgfpathlineto{\pgfqpoint{4.519615in}{1.589419in}}%
\pgfpathlineto{\pgfqpoint{4.529414in}{1.594959in}}%
\pgfpathlineto{\pgfqpoint{4.539213in}{1.600522in}}%
\pgfpathlineto{\pgfqpoint{4.549012in}{1.606106in}}%
\pgfpathlineto{\pgfqpoint{4.558811in}{1.611713in}}%
\pgfpathlineto{\pgfqpoint{4.568610in}{1.617343in}}%
\pgfpathlineto{\pgfqpoint{4.578409in}{1.622995in}}%
\pgfpathlineto{\pgfqpoint{4.588208in}{1.628669in}}%
\pgfpathlineto{\pgfqpoint{4.598007in}{1.634365in}}%
\pgfpathlineto{\pgfqpoint{4.607806in}{1.640084in}}%
\pgfpathlineto{\pgfqpoint{4.617605in}{1.645825in}}%
\pgfpathlineto{\pgfqpoint{4.627404in}{1.651589in}}%
\pgfpathlineto{\pgfqpoint{4.637203in}{1.657375in}}%
\pgfpathlineto{\pgfqpoint{4.647002in}{1.663183in}}%
\pgfpathlineto{\pgfqpoint{4.656800in}{1.669014in}}%
\pgfpathlineto{\pgfqpoint{4.666599in}{1.674867in}}%
\pgfpathlineto{\pgfqpoint{4.676398in}{1.680742in}}%
\pgfpathlineto{\pgfqpoint{4.686197in}{1.686640in}}%
\pgfpathlineto{\pgfqpoint{4.695996in}{1.692560in}}%
\pgfpathlineto{\pgfqpoint{4.705795in}{1.698503in}}%
\pgfpathlineto{\pgfqpoint{4.715594in}{1.704467in}}%
\pgfpathlineto{\pgfqpoint{4.725393in}{1.710454in}}%
\pgfpathlineto{\pgfqpoint{4.735192in}{1.716464in}}%
\pgfpathlineto{\pgfqpoint{4.744991in}{1.722496in}}%
\pgfpathlineto{\pgfqpoint{4.754790in}{1.728550in}}%
\pgfpathlineto{\pgfqpoint{4.764589in}{1.734627in}}%
\pgfpathlineto{\pgfqpoint{4.774388in}{1.740726in}}%
\pgfpathlineto{\pgfqpoint{4.784187in}{1.746847in}}%
\pgfpathlineto{\pgfqpoint{4.793986in}{1.752991in}}%
\pgfpathlineto{\pgfqpoint{4.803785in}{1.759157in}}%
\pgfpathlineto{\pgfqpoint{4.813584in}{1.765345in}}%
\pgfpathlineto{\pgfqpoint{4.823383in}{1.771556in}}%
\pgfpathlineto{\pgfqpoint{4.833182in}{1.777789in}}%
\pgfpathlineto{\pgfqpoint{4.842981in}{1.784044in}}%
\pgfpathlineto{\pgfqpoint{4.852780in}{1.790322in}}%
\pgfpathlineto{\pgfqpoint{4.862579in}{1.796622in}}%
\pgfpathlineto{\pgfqpoint{4.872378in}{1.802945in}}%
\pgfpathlineto{\pgfqpoint{4.882177in}{1.809289in}}%
\pgfpathlineto{\pgfqpoint{4.891976in}{1.815657in}}%
\pgfpathlineto{\pgfqpoint{4.901775in}{1.822046in}}%
\pgfpathlineto{\pgfqpoint{4.911574in}{1.828458in}}%
\pgfpathlineto{\pgfqpoint{4.921373in}{1.834892in}}%
\pgfpathlineto{\pgfqpoint{4.931172in}{1.841349in}}%
\pgfpathlineto{\pgfqpoint{4.940971in}{1.847828in}}%
\pgfpathlineto{\pgfqpoint{4.950770in}{1.854329in}}%
\pgfpathlineto{\pgfqpoint{4.960569in}{1.860853in}}%
\pgfpathlineto{\pgfqpoint{4.970368in}{1.867399in}}%
\pgfpathlineto{\pgfqpoint{4.980167in}{1.873968in}}%
\pgfpathlineto{\pgfqpoint{4.989966in}{1.880559in}}%
\pgfpathlineto{\pgfqpoint{4.999765in}{1.887172in}}%
\pgfpathlineto{\pgfqpoint{5.009564in}{1.893807in}}%
\pgfpathlineto{\pgfqpoint{5.019362in}{1.900465in}}%
\pgfpathlineto{\pgfqpoint{5.029161in}{1.907145in}}%
\pgfpathlineto{\pgfqpoint{5.038960in}{1.913848in}}%
\pgfpathlineto{\pgfqpoint{5.048759in}{1.920573in}}%
\pgfpathlineto{\pgfqpoint{5.058558in}{1.927320in}}%
\pgfpathlineto{\pgfqpoint{5.068357in}{1.934090in}}%
\pgfpathlineto{\pgfqpoint{5.078156in}{1.940882in}}%
\pgfpathlineto{\pgfqpoint{5.087955in}{1.947696in}}%
\pgfpathlineto{\pgfqpoint{5.097754in}{1.954533in}}%
\pgfpathlineto{\pgfqpoint{5.107553in}{1.961392in}}%
\pgfpathlineto{\pgfqpoint{5.117352in}{1.968273in}}%
\pgfpathlineto{\pgfqpoint{5.127151in}{1.975177in}}%
\pgfpathlineto{\pgfqpoint{5.136950in}{1.982103in}}%
\pgfpathlineto{\pgfqpoint{5.146749in}{1.989052in}}%
\pgfpathlineto{\pgfqpoint{5.156548in}{1.996023in}}%
\pgfpathlineto{\pgfqpoint{5.166347in}{2.003016in}}%
\pgfpathlineto{\pgfqpoint{5.176146in}{2.010031in}}%
\pgfpathlineto{\pgfqpoint{5.185945in}{2.017069in}}%
\pgfpathlineto{\pgfqpoint{5.195744in}{2.024130in}}%
\pgfpathlineto{\pgfqpoint{5.205543in}{2.031212in}}%
\pgfpathlineto{\pgfqpoint{5.215342in}{2.038317in}}%
\pgfpathlineto{\pgfqpoint{5.225141in}{2.045445in}}%
\pgfpathlineto{\pgfqpoint{5.234940in}{2.052594in}}%
\pgfpathlineto{\pgfqpoint{5.244739in}{2.059767in}}%
\pgfpathlineto{\pgfqpoint{5.254538in}{2.066961in}}%
\pgfpathlineto{\pgfqpoint{5.264337in}{2.074178in}}%
\pgfpathlineto{\pgfqpoint{5.274136in}{2.081417in}}%
\pgfpathlineto{\pgfqpoint{5.283935in}{2.088678in}}%
\pgfpathlineto{\pgfqpoint{5.293734in}{2.095962in}}%
\pgfpathlineto{\pgfqpoint{5.303533in}{2.103268in}}%
\pgfpathlineto{\pgfqpoint{5.313332in}{2.110597in}}%
\pgfpathlineto{\pgfqpoint{5.323131in}{2.117948in}}%
\pgfpathlineto{\pgfqpoint{5.332930in}{2.125321in}}%
\pgfpathlineto{\pgfqpoint{5.342729in}{2.132717in}}%
\pgfpathlineto{\pgfqpoint{5.352528in}{2.140135in}}%
\pgfpathlineto{\pgfqpoint{5.362327in}{2.147575in}}%
\pgfpathlineto{\pgfqpoint{5.372126in}{2.155038in}}%
\pgfpathlineto{\pgfqpoint{5.381924in}{2.162523in}}%
\pgfpathlineto{\pgfqpoint{5.391723in}{2.170031in}}%
\pgfpathlineto{\pgfqpoint{5.401522in}{2.177560in}}%
\pgfpathlineto{\pgfqpoint{5.411321in}{2.185112in}}%
\pgfpathlineto{\pgfqpoint{5.421120in}{2.192687in}}%
\pgfpathlineto{\pgfqpoint{5.430919in}{2.200284in}}%
\pgfpathlineto{\pgfqpoint{5.440718in}{2.207903in}}%
\pgfpathlineto{\pgfqpoint{5.450517in}{2.215545in}}%
\pgfpathlineto{\pgfqpoint{5.460316in}{2.223209in}}%
\pgfpathlineto{\pgfqpoint{5.470115in}{2.230895in}}%
\pgfpathlineto{\pgfqpoint{5.479914in}{2.238604in}}%
\pgfpathlineto{\pgfqpoint{5.489713in}{2.246335in}}%
\pgfpathlineto{\pgfqpoint{5.499512in}{2.254088in}}%
\pgfpathlineto{\pgfqpoint{5.509311in}{2.261864in}}%
\pgfpathlineto{\pgfqpoint{5.519110in}{2.269662in}}%
\pgfpathlineto{\pgfqpoint{5.528909in}{2.277482in}}%
\pgfpathlineto{\pgfqpoint{5.538708in}{2.285325in}}%
\pgfpathlineto{\pgfqpoint{5.548507in}{2.293190in}}%
\pgfpathlineto{\pgfqpoint{5.558306in}{2.301078in}}%
\pgfpathlineto{\pgfqpoint{5.568105in}{2.308987in}}%
\pgfpathlineto{\pgfqpoint{5.577904in}{2.316920in}}%
\pgfpathlineto{\pgfqpoint{5.587703in}{2.324874in}}%
\pgfpathlineto{\pgfqpoint{5.597502in}{2.332851in}}%
\pgfpathlineto{\pgfqpoint{5.607301in}{2.340851in}}%
\pgfpathlineto{\pgfqpoint{5.617100in}{2.348872in}}%
\pgfpathlineto{\pgfqpoint{5.626899in}{2.356916in}}%
\pgfpathlineto{\pgfqpoint{5.636698in}{2.364983in}}%
\pgfpathlineto{\pgfqpoint{5.646497in}{2.373071in}}%
\pgfpathlineto{\pgfqpoint{5.656296in}{2.381182in}}%
\pgfpathlineto{\pgfqpoint{5.666095in}{2.389316in}}%
\pgfpathlineto{\pgfqpoint{5.675894in}{2.397472in}}%
\pgfpathlineto{\pgfqpoint{5.685693in}{2.405650in}}%
\pgfpathlineto{\pgfqpoint{5.695492in}{2.413850in}}%
\pgfpathlineto{\pgfqpoint{5.705291in}{2.422073in}}%
\pgfpathlineto{\pgfqpoint{5.715090in}{2.430318in}}%
\pgfpathlineto{\pgfqpoint{5.724889in}{2.438586in}}%
\pgfpathlineto{\pgfqpoint{5.734687in}{2.446876in}}%
\pgfpathlineto{\pgfqpoint{5.734687in}{3.445673in}}%
\pgfpathlineto{\pgfqpoint{5.734687in}{3.445673in}}%
\pgfpathlineto{\pgfqpoint{5.724889in}{3.431691in}}%
\pgfpathlineto{\pgfqpoint{5.715090in}{3.417750in}}%
\pgfpathlineto{\pgfqpoint{5.705291in}{3.403850in}}%
\pgfpathlineto{\pgfqpoint{5.695492in}{3.389989in}}%
\pgfpathlineto{\pgfqpoint{5.685693in}{3.376169in}}%
\pgfpathlineto{\pgfqpoint{5.675894in}{3.362389in}}%
\pgfpathlineto{\pgfqpoint{5.666095in}{3.348649in}}%
\pgfpathlineto{\pgfqpoint{5.656296in}{3.334949in}}%
\pgfpathlineto{\pgfqpoint{5.646497in}{3.321290in}}%
\pgfpathlineto{\pgfqpoint{5.636698in}{3.307671in}}%
\pgfpathlineto{\pgfqpoint{5.626899in}{3.294092in}}%
\pgfpathlineto{\pgfqpoint{5.617100in}{3.280553in}}%
\pgfpathlineto{\pgfqpoint{5.607301in}{3.267055in}}%
\pgfpathlineto{\pgfqpoint{5.597502in}{3.253597in}}%
\pgfpathlineto{\pgfqpoint{5.587703in}{3.240179in}}%
\pgfpathlineto{\pgfqpoint{5.577904in}{3.226801in}}%
\pgfpathlineto{\pgfqpoint{5.568105in}{3.213463in}}%
\pgfpathlineto{\pgfqpoint{5.558306in}{3.200166in}}%
\pgfpathlineto{\pgfqpoint{5.548507in}{3.186909in}}%
\pgfpathlineto{\pgfqpoint{5.538708in}{3.173692in}}%
\pgfpathlineto{\pgfqpoint{5.528909in}{3.160516in}}%
\pgfpathlineto{\pgfqpoint{5.519110in}{3.147379in}}%
\pgfpathlineto{\pgfqpoint{5.509311in}{3.134283in}}%
\pgfpathlineto{\pgfqpoint{5.499512in}{3.121227in}}%
\pgfpathlineto{\pgfqpoint{5.489713in}{3.108212in}}%
\pgfpathlineto{\pgfqpoint{5.479914in}{3.095236in}}%
\pgfpathlineto{\pgfqpoint{5.470115in}{3.082301in}}%
\pgfpathlineto{\pgfqpoint{5.460316in}{3.069406in}}%
\pgfpathlineto{\pgfqpoint{5.450517in}{3.056552in}}%
\pgfpathlineto{\pgfqpoint{5.440718in}{3.043737in}}%
\pgfpathlineto{\pgfqpoint{5.430919in}{3.030963in}}%
\pgfpathlineto{\pgfqpoint{5.421120in}{3.018229in}}%
\pgfpathlineto{\pgfqpoint{5.411321in}{3.005535in}}%
\pgfpathlineto{\pgfqpoint{5.401522in}{2.992881in}}%
\pgfpathlineto{\pgfqpoint{5.391723in}{2.980268in}}%
\pgfpathlineto{\pgfqpoint{5.381924in}{2.967695in}}%
\pgfpathlineto{\pgfqpoint{5.372126in}{2.955162in}}%
\pgfpathlineto{\pgfqpoint{5.362327in}{2.942669in}}%
\pgfpathlineto{\pgfqpoint{5.352528in}{2.930217in}}%
\pgfpathlineto{\pgfqpoint{5.342729in}{2.917805in}}%
\pgfpathlineto{\pgfqpoint{5.332930in}{2.905433in}}%
\pgfpathlineto{\pgfqpoint{5.323131in}{2.893101in}}%
\pgfpathlineto{\pgfqpoint{5.313332in}{2.880810in}}%
\pgfpathlineto{\pgfqpoint{5.303533in}{2.868558in}}%
\pgfpathlineto{\pgfqpoint{5.293734in}{2.856347in}}%
\pgfpathlineto{\pgfqpoint{5.283935in}{2.844177in}}%
\pgfpathlineto{\pgfqpoint{5.274136in}{2.832046in}}%
\pgfpathlineto{\pgfqpoint{5.264337in}{2.819956in}}%
\pgfpathlineto{\pgfqpoint{5.254538in}{2.807906in}}%
\pgfpathlineto{\pgfqpoint{5.244739in}{2.795896in}}%
\pgfpathlineto{\pgfqpoint{5.234940in}{2.783926in}}%
\pgfpathlineto{\pgfqpoint{5.225141in}{2.771997in}}%
\pgfpathlineto{\pgfqpoint{5.215342in}{2.760108in}}%
\pgfpathlineto{\pgfqpoint{5.205543in}{2.748259in}}%
\pgfpathlineto{\pgfqpoint{5.195744in}{2.736450in}}%
\pgfpathlineto{\pgfqpoint{5.185945in}{2.724682in}}%
\pgfpathlineto{\pgfqpoint{5.176146in}{2.712953in}}%
\pgfpathlineto{\pgfqpoint{5.166347in}{2.701265in}}%
\pgfpathlineto{\pgfqpoint{5.156548in}{2.689618in}}%
\pgfpathlineto{\pgfqpoint{5.146749in}{2.678010in}}%
\pgfpathlineto{\pgfqpoint{5.136950in}{2.666443in}}%
\pgfpathlineto{\pgfqpoint{5.127151in}{2.654916in}}%
\pgfpathlineto{\pgfqpoint{5.117352in}{2.643429in}}%
\pgfpathlineto{\pgfqpoint{5.107553in}{2.631982in}}%
\pgfpathlineto{\pgfqpoint{5.097754in}{2.620576in}}%
\pgfpathlineto{\pgfqpoint{5.087955in}{2.609210in}}%
\pgfpathlineto{\pgfqpoint{5.078156in}{2.597884in}}%
\pgfpathlineto{\pgfqpoint{5.068357in}{2.586598in}}%
\pgfpathlineto{\pgfqpoint{5.058558in}{2.575353in}}%
\pgfpathlineto{\pgfqpoint{5.048759in}{2.564148in}}%
\pgfpathlineto{\pgfqpoint{5.038960in}{2.552983in}}%
\pgfpathlineto{\pgfqpoint{5.029161in}{2.541858in}}%
\pgfpathlineto{\pgfqpoint{5.019362in}{2.530773in}}%
\pgfpathlineto{\pgfqpoint{5.009564in}{2.519729in}}%
\pgfpathlineto{\pgfqpoint{4.999765in}{2.508725in}}%
\pgfpathlineto{\pgfqpoint{4.989966in}{2.497761in}}%
\pgfpathlineto{\pgfqpoint{4.980167in}{2.486838in}}%
\pgfpathlineto{\pgfqpoint{4.970368in}{2.475954in}}%
\pgfpathlineto{\pgfqpoint{4.960569in}{2.465111in}}%
\pgfpathlineto{\pgfqpoint{4.950770in}{2.454308in}}%
\pgfpathlineto{\pgfqpoint{4.940971in}{2.443546in}}%
\pgfpathlineto{\pgfqpoint{4.931172in}{2.432823in}}%
\pgfpathlineto{\pgfqpoint{4.921373in}{2.422141in}}%
\pgfpathlineto{\pgfqpoint{4.911574in}{2.411499in}}%
\pgfpathlineto{\pgfqpoint{4.901775in}{2.400897in}}%
\pgfpathlineto{\pgfqpoint{4.891976in}{2.390336in}}%
\pgfpathlineto{\pgfqpoint{4.882177in}{2.379814in}}%
\pgfpathlineto{\pgfqpoint{4.872378in}{2.369333in}}%
\pgfpathlineto{\pgfqpoint{4.862579in}{2.358893in}}%
\pgfpathlineto{\pgfqpoint{4.852780in}{2.348492in}}%
\pgfpathlineto{\pgfqpoint{4.842981in}{2.338132in}}%
\pgfpathlineto{\pgfqpoint{4.833182in}{2.327812in}}%
\pgfpathlineto{\pgfqpoint{4.823383in}{2.317532in}}%
\pgfpathlineto{\pgfqpoint{4.813584in}{2.307292in}}%
\pgfpathlineto{\pgfqpoint{4.803785in}{2.297093in}}%
\pgfpathlineto{\pgfqpoint{4.793986in}{2.286933in}}%
\pgfpathlineto{\pgfqpoint{4.784187in}{2.276814in}}%
\pgfpathlineto{\pgfqpoint{4.774388in}{2.266736in}}%
\pgfpathlineto{\pgfqpoint{4.764589in}{2.256697in}}%
\pgfpathlineto{\pgfqpoint{4.754790in}{2.246699in}}%
\pgfpathlineto{\pgfqpoint{4.744991in}{2.236741in}}%
\pgfpathlineto{\pgfqpoint{4.735192in}{2.226823in}}%
\pgfpathlineto{\pgfqpoint{4.725393in}{2.216946in}}%
\pgfpathlineto{\pgfqpoint{4.715594in}{2.207108in}}%
\pgfpathlineto{\pgfqpoint{4.705795in}{2.197311in}}%
\pgfpathlineto{\pgfqpoint{4.695996in}{2.187554in}}%
\pgfpathlineto{\pgfqpoint{4.686197in}{2.177838in}}%
\pgfpathlineto{\pgfqpoint{4.676398in}{2.168161in}}%
\pgfpathlineto{\pgfqpoint{4.666599in}{2.158525in}}%
\pgfpathlineto{\pgfqpoint{4.656800in}{2.148929in}}%
\pgfpathlineto{\pgfqpoint{4.647002in}{2.139374in}}%
\pgfpathlineto{\pgfqpoint{4.637203in}{2.129858in}}%
\pgfpathlineto{\pgfqpoint{4.627404in}{2.120383in}}%
\pgfpathlineto{\pgfqpoint{4.617605in}{2.110948in}}%
\pgfpathlineto{\pgfqpoint{4.607806in}{2.101553in}}%
\pgfpathlineto{\pgfqpoint{4.598007in}{2.092198in}}%
\pgfpathlineto{\pgfqpoint{4.588208in}{2.082884in}}%
\pgfpathlineto{\pgfqpoint{4.578409in}{2.073610in}}%
\pgfpathlineto{\pgfqpoint{4.568610in}{2.064376in}}%
\pgfpathlineto{\pgfqpoint{4.558811in}{2.055183in}}%
\pgfpathlineto{\pgfqpoint{4.549012in}{2.046029in}}%
\pgfpathlineto{\pgfqpoint{4.539213in}{2.036916in}}%
\pgfpathlineto{\pgfqpoint{4.529414in}{2.027843in}}%
\pgfpathlineto{\pgfqpoint{4.519615in}{2.018810in}}%
\pgfpathlineto{\pgfqpoint{4.509816in}{2.009818in}}%
\pgfpathlineto{\pgfqpoint{4.500017in}{2.000866in}}%
\pgfpathlineto{\pgfqpoint{4.490218in}{1.991954in}}%
\pgfpathlineto{\pgfqpoint{4.480419in}{1.983082in}}%
\pgfpathlineto{\pgfqpoint{4.470620in}{1.974251in}}%
\pgfpathlineto{\pgfqpoint{4.460821in}{1.965459in}}%
\pgfpathlineto{\pgfqpoint{4.451022in}{1.956708in}}%
\pgfpathlineto{\pgfqpoint{4.441223in}{1.947997in}}%
\pgfpathlineto{\pgfqpoint{4.431424in}{1.939327in}}%
\pgfpathlineto{\pgfqpoint{4.421625in}{1.930696in}}%
\pgfpathlineto{\pgfqpoint{4.411826in}{1.922106in}}%
\pgfpathlineto{\pgfqpoint{4.402027in}{1.913556in}}%
\pgfpathlineto{\pgfqpoint{4.392228in}{1.905047in}}%
\pgfpathlineto{\pgfqpoint{4.382429in}{1.896577in}}%
\pgfpathlineto{\pgfqpoint{4.372630in}{1.888148in}}%
\pgfpathlineto{\pgfqpoint{4.362831in}{1.879759in}}%
\pgfpathlineto{\pgfqpoint{4.353032in}{1.871410in}}%
\pgfpathlineto{\pgfqpoint{4.343233in}{1.863102in}}%
\pgfpathlineto{\pgfqpoint{4.333434in}{1.854833in}}%
\pgfpathlineto{\pgfqpoint{4.323635in}{1.846605in}}%
\pgfpathlineto{\pgfqpoint{4.313836in}{1.838418in}}%
\pgfpathlineto{\pgfqpoint{4.304037in}{1.830270in}}%
\pgfpathlineto{\pgfqpoint{4.294238in}{1.822163in}}%
\pgfpathlineto{\pgfqpoint{4.284440in}{1.814096in}}%
\pgfpathlineto{\pgfqpoint{4.274641in}{1.806069in}}%
\pgfpathlineto{\pgfqpoint{4.264842in}{1.798082in}}%
\pgfpathlineto{\pgfqpoint{4.255043in}{1.790136in}}%
\pgfpathlineto{\pgfqpoint{4.245244in}{1.782229in}}%
\pgfpathlineto{\pgfqpoint{4.235445in}{1.774363in}}%
\pgfpathlineto{\pgfqpoint{4.225646in}{1.766538in}}%
\pgfpathlineto{\pgfqpoint{4.215847in}{1.758752in}}%
\pgfpathlineto{\pgfqpoint{4.206048in}{1.751007in}}%
\pgfpathlineto{\pgfqpoint{4.196249in}{1.743302in}}%
\pgfpathlineto{\pgfqpoint{4.186450in}{1.735637in}}%
\pgfpathlineto{\pgfqpoint{4.176651in}{1.728013in}}%
\pgfpathlineto{\pgfqpoint{4.166852in}{1.720428in}}%
\pgfpathlineto{\pgfqpoint{4.157053in}{1.712884in}}%
\pgfpathlineto{\pgfqpoint{4.147254in}{1.705380in}}%
\pgfpathlineto{\pgfqpoint{4.137455in}{1.697917in}}%
\pgfpathlineto{\pgfqpoint{4.127656in}{1.690493in}}%
\pgfpathlineto{\pgfqpoint{4.117857in}{1.683110in}}%
\pgfpathlineto{\pgfqpoint{4.108058in}{1.675767in}}%
\pgfpathlineto{\pgfqpoint{4.098259in}{1.668464in}}%
\pgfpathlineto{\pgfqpoint{4.088460in}{1.661202in}}%
\pgfpathlineto{\pgfqpoint{4.078661in}{1.653980in}}%
\pgfpathlineto{\pgfqpoint{4.068862in}{1.646798in}}%
\pgfpathlineto{\pgfqpoint{4.059063in}{1.639656in}}%
\pgfpathlineto{\pgfqpoint{4.049264in}{1.632554in}}%
\pgfpathlineto{\pgfqpoint{4.039465in}{1.625493in}}%
\pgfpathlineto{\pgfqpoint{4.029666in}{1.618472in}}%
\pgfpathlineto{\pgfqpoint{4.019867in}{1.611491in}}%
\pgfpathlineto{\pgfqpoint{4.010068in}{1.604550in}}%
\pgfpathlineto{\pgfqpoint{4.000269in}{1.597650in}}%
\pgfpathlineto{\pgfqpoint{3.990470in}{1.590790in}}%
\pgfpathlineto{\pgfqpoint{3.980671in}{1.583970in}}%
\pgfpathlineto{\pgfqpoint{3.970872in}{1.577190in}}%
\pgfpathlineto{\pgfqpoint{3.961073in}{1.570451in}}%
\pgfpathlineto{\pgfqpoint{3.951274in}{1.563751in}}%
\pgfpathlineto{\pgfqpoint{3.941475in}{1.557092in}}%
\pgfpathlineto{\pgfqpoint{3.931676in}{1.550474in}}%
\pgfpathlineto{\pgfqpoint{3.921878in}{1.543895in}}%
\pgfpathlineto{\pgfqpoint{3.912079in}{1.537357in}}%
\pgfpathlineto{\pgfqpoint{3.902280in}{1.530859in}}%
\pgfpathlineto{\pgfqpoint{3.892481in}{1.524401in}}%
\pgfpathlineto{\pgfqpoint{3.882682in}{1.517983in}}%
\pgfpathlineto{\pgfqpoint{3.872883in}{1.511606in}}%
\pgfpathlineto{\pgfqpoint{3.863084in}{1.505269in}}%
\pgfpathlineto{\pgfqpoint{3.853285in}{1.498972in}}%
\pgfpathlineto{\pgfqpoint{3.843486in}{1.492715in}}%
\pgfpathlineto{\pgfqpoint{3.833687in}{1.486499in}}%
\pgfpathlineto{\pgfqpoint{3.823888in}{1.480323in}}%
\pgfpathlineto{\pgfqpoint{3.814089in}{1.474187in}}%
\pgfpathlineto{\pgfqpoint{3.804290in}{1.468091in}}%
\pgfpathlineto{\pgfqpoint{3.794491in}{1.462035in}}%
\pgfpathlineto{\pgfqpoint{3.784692in}{1.456020in}}%
\pgfpathlineto{\pgfqpoint{3.774893in}{1.450045in}}%
\pgfpathlineto{\pgfqpoint{3.765094in}{1.444110in}}%
\pgfpathlineto{\pgfqpoint{3.755295in}{1.438216in}}%
\pgfpathlineto{\pgfqpoint{3.745496in}{1.432361in}}%
\pgfpathlineto{\pgfqpoint{3.735697in}{1.426547in}}%
\pgfpathlineto{\pgfqpoint{3.725898in}{1.420773in}}%
\pgfpathlineto{\pgfqpoint{3.716099in}{1.415039in}}%
\pgfpathlineto{\pgfqpoint{3.706300in}{1.409346in}}%
\pgfpathlineto{\pgfqpoint{3.696501in}{1.403693in}}%
\pgfpathlineto{\pgfqpoint{3.686702in}{1.398080in}}%
\pgfpathlineto{\pgfqpoint{3.676903in}{1.392507in}}%
\pgfpathlineto{\pgfqpoint{3.667104in}{1.386975in}}%
\pgfpathlineto{\pgfqpoint{3.657305in}{1.381482in}}%
\pgfpathlineto{\pgfqpoint{3.647506in}{1.376030in}}%
\pgfpathlineto{\pgfqpoint{3.637707in}{1.370618in}}%
\pgfpathlineto{\pgfqpoint{3.627908in}{1.365247in}}%
\pgfpathlineto{\pgfqpoint{3.618109in}{1.359916in}}%
\pgfpathlineto{\pgfqpoint{3.608310in}{1.354624in}}%
\pgfpathlineto{\pgfqpoint{3.598511in}{1.349374in}}%
\pgfpathlineto{\pgfqpoint{3.588712in}{1.344163in}}%
\pgfpathlineto{\pgfqpoint{3.578913in}{1.338993in}}%
\pgfpathlineto{\pgfqpoint{3.569114in}{1.333862in}}%
\pgfpathlineto{\pgfqpoint{3.559316in}{1.328772in}}%
\pgfpathlineto{\pgfqpoint{3.549517in}{1.323723in}}%
\pgfpathlineto{\pgfqpoint{3.539718in}{1.318713in}}%
\pgfpathlineto{\pgfqpoint{3.529919in}{1.313744in}}%
\pgfpathlineto{\pgfqpoint{3.520120in}{1.308815in}}%
\pgfpathlineto{\pgfqpoint{3.510321in}{1.303926in}}%
\pgfpathlineto{\pgfqpoint{3.500522in}{1.299077in}}%
\pgfpathlineto{\pgfqpoint{3.490723in}{1.294269in}}%
\pgfpathlineto{\pgfqpoint{3.480924in}{1.289501in}}%
\pgfpathlineto{\pgfqpoint{3.471125in}{1.284773in}}%
\pgfpathlineto{\pgfqpoint{3.461326in}{1.280086in}}%
\pgfpathlineto{\pgfqpoint{3.451527in}{1.275438in}}%
\pgfpathlineto{\pgfqpoint{3.441728in}{1.270831in}}%
\pgfpathlineto{\pgfqpoint{3.431929in}{1.266264in}}%
\pgfpathlineto{\pgfqpoint{3.422130in}{1.261737in}}%
\pgfpathlineto{\pgfqpoint{3.412331in}{1.257251in}}%
\pgfpathlineto{\pgfqpoint{3.402532in}{1.252805in}}%
\pgfpathlineto{\pgfqpoint{3.392733in}{1.248399in}}%
\pgfpathlineto{\pgfqpoint{3.382934in}{1.244033in}}%
\pgfpathlineto{\pgfqpoint{3.373135in}{1.239707in}}%
\pgfpathlineto{\pgfqpoint{3.363336in}{1.235422in}}%
\pgfpathlineto{\pgfqpoint{3.353537in}{1.231177in}}%
\pgfpathlineto{\pgfqpoint{3.343738in}{1.226972in}}%
\pgfpathlineto{\pgfqpoint{3.333939in}{1.222807in}}%
\pgfpathlineto{\pgfqpoint{3.324140in}{1.218683in}}%
\pgfpathlineto{\pgfqpoint{3.314341in}{1.214599in}}%
\pgfpathlineto{\pgfqpoint{3.304542in}{1.210555in}}%
\pgfpathlineto{\pgfqpoint{3.294743in}{1.206551in}}%
\pgfpathlineto{\pgfqpoint{3.284944in}{1.202588in}}%
\pgfpathlineto{\pgfqpoint{3.275145in}{1.198665in}}%
\pgfpathlineto{\pgfqpoint{3.265346in}{1.194782in}}%
\pgfpathlineto{\pgfqpoint{3.255547in}{1.190939in}}%
\pgfpathlineto{\pgfqpoint{3.245748in}{1.187136in}}%
\pgfpathlineto{\pgfqpoint{3.235949in}{1.183374in}}%
\pgfpathlineto{\pgfqpoint{3.226150in}{1.179652in}}%
\pgfpathlineto{\pgfqpoint{3.216351in}{1.175970in}}%
\pgfpathlineto{\pgfqpoint{3.206552in}{1.172328in}}%
\pgfpathlineto{\pgfqpoint{3.196754in}{1.168727in}}%
\pgfpathlineto{\pgfqpoint{3.186955in}{1.165166in}}%
\pgfpathlineto{\pgfqpoint{3.177156in}{1.161645in}}%
\pgfpathlineto{\pgfqpoint{3.167357in}{1.158164in}}%
\pgfpathlineto{\pgfqpoint{3.157558in}{1.154724in}}%
\pgfpathlineto{\pgfqpoint{3.147759in}{1.151324in}}%
\pgfpathlineto{\pgfqpoint{3.137960in}{1.147964in}}%
\pgfpathlineto{\pgfqpoint{3.128161in}{1.144644in}}%
\pgfpathlineto{\pgfqpoint{3.118362in}{1.141364in}}%
\pgfpathlineto{\pgfqpoint{3.108563in}{1.138125in}}%
\pgfpathlineto{\pgfqpoint{3.098764in}{1.134926in}}%
\pgfpathlineto{\pgfqpoint{3.088965in}{1.131767in}}%
\pgfpathlineto{\pgfqpoint{3.079166in}{1.128649in}}%
\pgfpathlineto{\pgfqpoint{3.069367in}{1.125570in}}%
\pgfpathlineto{\pgfqpoint{3.059568in}{1.122532in}}%
\pgfpathlineto{\pgfqpoint{3.049769in}{1.119534in}}%
\pgfpathlineto{\pgfqpoint{3.039970in}{1.116577in}}%
\pgfpathlineto{\pgfqpoint{3.030171in}{1.113659in}}%
\pgfpathlineto{\pgfqpoint{3.020372in}{1.110782in}}%
\pgfpathlineto{\pgfqpoint{3.010573in}{1.107945in}}%
\pgfpathlineto{\pgfqpoint{3.000774in}{1.105148in}}%
\pgfpathlineto{\pgfqpoint{2.990975in}{1.102392in}}%
\pgfpathlineto{\pgfqpoint{2.981176in}{1.099676in}}%
\pgfpathlineto{\pgfqpoint{2.971377in}{1.097000in}}%
\pgfpathlineto{\pgfqpoint{2.961578in}{1.094364in}}%
\pgfpathlineto{\pgfqpoint{2.951779in}{1.091768in}}%
\pgfpathlineto{\pgfqpoint{2.941980in}{1.089213in}}%
\pgfpathlineto{\pgfqpoint{2.932181in}{1.086698in}}%
\pgfpathlineto{\pgfqpoint{2.922382in}{1.084223in}}%
\pgfpathlineto{\pgfqpoint{2.912583in}{1.081788in}}%
\pgfpathlineto{\pgfqpoint{2.902784in}{1.079394in}}%
\pgfpathlineto{\pgfqpoint{2.892985in}{1.077040in}}%
\pgfpathlineto{\pgfqpoint{2.883186in}{1.074726in}}%
\pgfpathlineto{\pgfqpoint{2.873387in}{1.072452in}}%
\pgfpathlineto{\pgfqpoint{2.863588in}{1.070218in}}%
\pgfpathlineto{\pgfqpoint{2.853789in}{1.068025in}}%
\pgfpathlineto{\pgfqpoint{2.843990in}{1.065872in}}%
\pgfpathlineto{\pgfqpoint{2.834192in}{1.063759in}}%
\pgfpathlineto{\pgfqpoint{2.824393in}{1.061687in}}%
\pgfpathlineto{\pgfqpoint{2.814594in}{1.059654in}}%
\pgfpathlineto{\pgfqpoint{2.804795in}{1.057662in}}%
\pgfpathlineto{\pgfqpoint{2.794996in}{1.055711in}}%
\pgfpathlineto{\pgfqpoint{2.785197in}{1.053799in}}%
\pgfpathlineto{\pgfqpoint{2.775398in}{1.051928in}}%
\pgfpathlineto{\pgfqpoint{2.765599in}{1.050096in}}%
\pgfpathlineto{\pgfqpoint{2.755800in}{1.048305in}}%
\pgfpathlineto{\pgfqpoint{2.746001in}{1.046555in}}%
\pgfpathlineto{\pgfqpoint{2.736202in}{1.044844in}}%
\pgfpathlineto{\pgfqpoint{2.726403in}{1.043174in}}%
\pgfpathlineto{\pgfqpoint{2.716604in}{1.041544in}}%
\pgfpathlineto{\pgfqpoint{2.706805in}{1.039954in}}%
\pgfpathlineto{\pgfqpoint{2.697006in}{1.038405in}}%
\pgfpathlineto{\pgfqpoint{2.687207in}{1.036895in}}%
\pgfpathlineto{\pgfqpoint{2.677408in}{1.035426in}}%
\pgfpathlineto{\pgfqpoint{2.667609in}{1.033997in}}%
\pgfpathlineto{\pgfqpoint{2.657810in}{1.032609in}}%
\pgfpathlineto{\pgfqpoint{2.648011in}{1.031260in}}%
\pgfpathlineto{\pgfqpoint{2.638212in}{1.029952in}}%
\pgfpathlineto{\pgfqpoint{2.628413in}{1.028684in}}%
\pgfpathlineto{\pgfqpoint{2.618614in}{1.027457in}}%
\pgfpathlineto{\pgfqpoint{2.608815in}{1.026269in}}%
\pgfpathlineto{\pgfqpoint{2.599016in}{1.025122in}}%
\pgfpathlineto{\pgfqpoint{2.589217in}{1.024015in}}%
\pgfpathlineto{\pgfqpoint{2.579418in}{1.022948in}}%
\pgfpathlineto{\pgfqpoint{2.569619in}{1.021922in}}%
\pgfpathlineto{\pgfqpoint{2.559820in}{1.020935in}}%
\pgfpathlineto{\pgfqpoint{2.550021in}{1.019989in}}%
\pgfpathlineto{\pgfqpoint{2.540222in}{1.019084in}}%
\pgfpathlineto{\pgfqpoint{2.530423in}{1.018218in}}%
\pgfpathlineto{\pgfqpoint{2.520624in}{1.017393in}}%
\pgfpathlineto{\pgfqpoint{2.510825in}{1.016607in}}%
\pgfpathlineto{\pgfqpoint{2.501026in}{1.015863in}}%
\pgfpathlineto{\pgfqpoint{2.491227in}{1.015158in}}%
\pgfpathlineto{\pgfqpoint{2.481428in}{1.014493in}}%
\pgfpathlineto{\pgfqpoint{2.471630in}{1.013869in}}%
\pgfpathlineto{\pgfqpoint{2.461831in}{1.013285in}}%
\pgfpathlineto{\pgfqpoint{2.452032in}{1.012741in}}%
\pgfpathlineto{\pgfqpoint{2.442233in}{1.012238in}}%
\pgfpathlineto{\pgfqpoint{2.432434in}{1.011775in}}%
\pgfpathlineto{\pgfqpoint{2.422635in}{1.011352in}}%
\pgfpathlineto{\pgfqpoint{2.412836in}{1.010969in}}%
\pgfpathlineto{\pgfqpoint{2.403037in}{1.010626in}}%
\pgfpathlineto{\pgfqpoint{2.393238in}{1.010324in}}%
\pgfpathlineto{\pgfqpoint{2.383439in}{1.010062in}}%
\pgfpathlineto{\pgfqpoint{2.373640in}{1.009840in}}%
\pgfpathlineto{\pgfqpoint{2.363841in}{1.009658in}}%
\pgfpathlineto{\pgfqpoint{2.354042in}{1.009517in}}%
\pgfpathlineto{\pgfqpoint{2.344243in}{1.009416in}}%
\pgfpathlineto{\pgfqpoint{2.334444in}{1.009355in}}%
\pgfpathlineto{\pgfqpoint{2.324645in}{1.009334in}}%
\pgfpathlineto{\pgfqpoint{2.314846in}{1.009353in}}%
\pgfpathlineto{\pgfqpoint{2.305047in}{1.009413in}}%
\pgfpathlineto{\pgfqpoint{2.295248in}{1.009513in}}%
\pgfpathlineto{\pgfqpoint{2.285449in}{1.009653in}}%
\pgfpathlineto{\pgfqpoint{2.275650in}{1.009834in}}%
\pgfpathlineto{\pgfqpoint{2.265851in}{1.010054in}}%
\pgfpathlineto{\pgfqpoint{2.256052in}{1.010315in}}%
\pgfpathlineto{\pgfqpoint{2.246253in}{1.010616in}}%
\pgfpathlineto{\pgfqpoint{2.236454in}{1.010958in}}%
\pgfpathlineto{\pgfqpoint{2.226655in}{1.011339in}}%
\pgfpathlineto{\pgfqpoint{2.216856in}{1.011761in}}%
\pgfpathlineto{\pgfqpoint{2.207057in}{1.012223in}}%
\pgfpathlineto{\pgfqpoint{2.197258in}{1.012726in}}%
\pgfpathlineto{\pgfqpoint{2.187459in}{1.013268in}}%
\pgfpathlineto{\pgfqpoint{2.177660in}{1.013851in}}%
\pgfpathlineto{\pgfqpoint{2.167861in}{1.014474in}}%
\pgfpathlineto{\pgfqpoint{2.158062in}{1.015137in}}%
\pgfpathlineto{\pgfqpoint{2.148263in}{1.015841in}}%
\pgfpathlineto{\pgfqpoint{2.138464in}{1.016584in}}%
\pgfpathlineto{\pgfqpoint{2.128665in}{1.017368in}}%
\pgfpathlineto{\pgfqpoint{2.118866in}{1.018192in}}%
\pgfpathlineto{\pgfqpoint{2.109068in}{1.019057in}}%
\pgfpathlineto{\pgfqpoint{2.099269in}{1.019961in}}%
\pgfpathlineto{\pgfqpoint{2.089470in}{1.020906in}}%
\pgfpathlineto{\pgfqpoint{2.079671in}{1.021891in}}%
\pgfpathlineto{\pgfqpoint{2.069872in}{1.022916in}}%
\pgfpathlineto{\pgfqpoint{2.060073in}{1.023982in}}%
\pgfpathlineto{\pgfqpoint{2.050274in}{1.025088in}}%
\pgfpathlineto{\pgfqpoint{2.040475in}{1.026234in}}%
\pgfpathlineto{\pgfqpoint{2.030676in}{1.027420in}}%
\pgfpathlineto{\pgfqpoint{2.020877in}{1.028646in}}%
\pgfpathlineto{\pgfqpoint{2.011078in}{1.029913in}}%
\pgfpathlineto{\pgfqpoint{2.001279in}{1.031220in}}%
\pgfpathlineto{\pgfqpoint{1.991480in}{1.032567in}}%
\pgfpathlineto{\pgfqpoint{1.981681in}{1.033955in}}%
\pgfpathlineto{\pgfqpoint{1.971882in}{1.035382in}}%
\pgfpathlineto{\pgfqpoint{1.962083in}{1.036850in}}%
\pgfpathlineto{\pgfqpoint{1.952284in}{1.038358in}}%
\pgfpathlineto{\pgfqpoint{1.942485in}{1.039906in}}%
\pgfpathlineto{\pgfqpoint{1.932686in}{1.041495in}}%
\pgfpathlineto{\pgfqpoint{1.922887in}{1.043124in}}%
\pgfpathlineto{\pgfqpoint{1.913088in}{1.044793in}}%
\pgfpathlineto{\pgfqpoint{1.903289in}{1.046502in}}%
\pgfpathlineto{\pgfqpoint{1.893490in}{1.048252in}}%
\pgfpathlineto{\pgfqpoint{1.883691in}{1.050041in}}%
\pgfpathlineto{\pgfqpoint{1.873892in}{1.051871in}}%
\pgfpathlineto{\pgfqpoint{1.864093in}{1.053741in}}%
\pgfpathlineto{\pgfqpoint{1.854294in}{1.055652in}}%
\pgfpathlineto{\pgfqpoint{1.844495in}{1.057602in}}%
\pgfpathlineto{\pgfqpoint{1.834696in}{1.059593in}}%
\pgfpathlineto{\pgfqpoint{1.824897in}{1.061624in}}%
\pgfpathlineto{\pgfqpoint{1.815098in}{1.063696in}}%
\pgfpathlineto{\pgfqpoint{1.805299in}{1.065807in}}%
\pgfpathlineto{\pgfqpoint{1.795500in}{1.067959in}}%
\pgfpathlineto{\pgfqpoint{1.785701in}{1.070151in}}%
\pgfpathlineto{\pgfqpoint{1.775902in}{1.072383in}}%
\pgfpathlineto{\pgfqpoint{1.766103in}{1.074656in}}%
\pgfpathlineto{\pgfqpoint{1.756304in}{1.076969in}}%
\pgfpathlineto{\pgfqpoint{1.746506in}{1.079322in}}%
\pgfpathlineto{\pgfqpoint{1.736707in}{1.081715in}}%
\pgfpathlineto{\pgfqpoint{1.726908in}{1.084148in}}%
\pgfpathlineto{\pgfqpoint{1.717109in}{1.086622in}}%
\pgfpathlineto{\pgfqpoint{1.707310in}{1.089136in}}%
\pgfpathlineto{\pgfqpoint{1.697511in}{1.091690in}}%
\pgfpathlineto{\pgfqpoint{1.687712in}{1.094284in}}%
\pgfpathlineto{\pgfqpoint{1.677913in}{1.096919in}}%
\pgfpathlineto{\pgfqpoint{1.668114in}{1.099594in}}%
\pgfpathlineto{\pgfqpoint{1.658315in}{1.102309in}}%
\pgfpathlineto{\pgfqpoint{1.648516in}{1.105064in}}%
\pgfpathlineto{\pgfqpoint{1.638717in}{1.107859in}}%
\pgfpathlineto{\pgfqpoint{1.628918in}{1.110695in}}%
\pgfpathlineto{\pgfqpoint{1.619119in}{1.113571in}}%
\pgfpathlineto{\pgfqpoint{1.609320in}{1.116487in}}%
\pgfpathlineto{\pgfqpoint{1.599521in}{1.119444in}}%
\pgfpathlineto{\pgfqpoint{1.589722in}{1.122440in}}%
\pgfpathlineto{\pgfqpoint{1.579923in}{1.125477in}}%
\pgfpathlineto{\pgfqpoint{1.570124in}{1.128554in}}%
\pgfpathlineto{\pgfqpoint{1.560325in}{1.131672in}}%
\pgfpathlineto{\pgfqpoint{1.550526in}{1.134829in}}%
\pgfpathlineto{\pgfqpoint{1.540727in}{1.138027in}}%
\pgfpathlineto{\pgfqpoint{1.530928in}{1.141265in}}%
\pgfpathlineto{\pgfqpoint{1.521129in}{1.144544in}}%
\pgfpathlineto{\pgfqpoint{1.511330in}{1.147862in}}%
\pgfpathlineto{\pgfqpoint{1.501531in}{1.151221in}}%
\pgfpathlineto{\pgfqpoint{1.491732in}{1.154620in}}%
\pgfpathlineto{\pgfqpoint{1.481933in}{1.158059in}}%
\pgfpathlineto{\pgfqpoint{1.472134in}{1.161539in}}%
\pgfpathlineto{\pgfqpoint{1.462335in}{1.165058in}}%
\pgfpathlineto{\pgfqpoint{1.452536in}{1.168618in}}%
\pgfpathlineto{\pgfqpoint{1.442737in}{1.172218in}}%
\pgfpathlineto{\pgfqpoint{1.432938in}{1.175859in}}%
\pgfpathlineto{\pgfqpoint{1.423139in}{1.179539in}}%
\pgfpathlineto{\pgfqpoint{1.413340in}{1.183260in}}%
\pgfpathlineto{\pgfqpoint{1.403541in}{1.187021in}}%
\pgfpathlineto{\pgfqpoint{1.393742in}{1.190823in}}%
\pgfpathlineto{\pgfqpoint{1.383944in}{1.194664in}}%
\pgfpathlineto{\pgfqpoint{1.374145in}{1.198546in}}%
\pgfpathlineto{\pgfqpoint{1.364346in}{1.202468in}}%
\pgfpathlineto{\pgfqpoint{1.354547in}{1.206430in}}%
\pgfpathlineto{\pgfqpoint{1.344748in}{1.210433in}}%
\pgfpathlineto{\pgfqpoint{1.334949in}{1.214475in}}%
\pgfpathlineto{\pgfqpoint{1.325150in}{1.218558in}}%
\pgfpathlineto{\pgfqpoint{1.315351in}{1.222681in}}%
\pgfpathlineto{\pgfqpoint{1.305552in}{1.226845in}}%
\pgfpathlineto{\pgfqpoint{1.295753in}{1.231048in}}%
\pgfpathlineto{\pgfqpoint{1.285954in}{1.235292in}}%
\pgfpathlineto{\pgfqpoint{1.276155in}{1.239576in}}%
\pgfpathlineto{\pgfqpoint{1.266356in}{1.243901in}}%
\pgfpathlineto{\pgfqpoint{1.256557in}{1.248265in}}%
\pgfpathlineto{\pgfqpoint{1.246758in}{1.252670in}}%
\pgfpathlineto{\pgfqpoint{1.236959in}{1.257115in}}%
\pgfpathlineto{\pgfqpoint{1.227160in}{1.261600in}}%
\pgfpathlineto{\pgfqpoint{1.217361in}{1.266126in}}%
\pgfpathlineto{\pgfqpoint{1.207562in}{1.270691in}}%
\pgfpathlineto{\pgfqpoint{1.197763in}{1.275297in}}%
\pgfpathlineto{\pgfqpoint{1.187964in}{1.279944in}}%
\pgfpathlineto{\pgfqpoint{1.178165in}{1.284630in}}%
\pgfpathlineto{\pgfqpoint{1.168366in}{1.289357in}}%
\pgfpathlineto{\pgfqpoint{1.158567in}{1.294124in}}%
\pgfpathlineto{\pgfqpoint{1.148768in}{1.298931in}}%
\pgfpathlineto{\pgfqpoint{1.138969in}{1.303778in}}%
\pgfpathlineto{\pgfqpoint{1.129170in}{1.308666in}}%
\pgfpathlineto{\pgfqpoint{1.119371in}{1.313593in}}%
\pgfpathlineto{\pgfqpoint{1.109572in}{1.318561in}}%
\pgfpathlineto{\pgfqpoint{1.099773in}{1.323570in}}%
\pgfpathlineto{\pgfqpoint{1.089974in}{1.328618in}}%
\pgfpathlineto{\pgfqpoint{1.080175in}{1.333707in}}%
\pgfpathlineto{\pgfqpoint{1.070376in}{1.338836in}}%
\pgfpathlineto{\pgfqpoint{1.060577in}{1.344005in}}%
\pgfpathlineto{\pgfqpoint{1.050778in}{1.349214in}}%
\pgfpathlineto{\pgfqpoint{1.040979in}{1.354464in}}%
\pgfpathlineto{\pgfqpoint{1.031180in}{1.359754in}}%
\pgfpathlineto{\pgfqpoint{1.021382in}{1.365084in}}%
\pgfpathlineto{\pgfqpoint{1.011583in}{1.370455in}}%
\pgfpathlineto{\pgfqpoint{1.001784in}{1.375865in}}%
\pgfpathlineto{\pgfqpoint{0.991985in}{1.381316in}}%
\pgfpathlineto{\pgfqpoint{0.982186in}{1.386807in}}%
\pgfpathlineto{\pgfqpoint{0.972387in}{1.392338in}}%
\pgfpathlineto{\pgfqpoint{0.962588in}{1.397910in}}%
\pgfpathlineto{\pgfqpoint{0.952789in}{1.403522in}}%
\pgfpathlineto{\pgfqpoint{0.942990in}{1.409173in}}%
\pgfpathlineto{\pgfqpoint{0.933191in}{1.414866in}}%
\pgfpathlineto{\pgfqpoint{0.923392in}{1.420598in}}%
\pgfpathlineto{\pgfqpoint{0.913593in}{1.426371in}}%
\pgfpathlineto{\pgfqpoint{0.903794in}{1.432184in}}%
\pgfpathlineto{\pgfqpoint{0.893995in}{1.438037in}}%
\pgfpathlineto{\pgfqpoint{0.884196in}{1.443930in}}%
\pgfpathlineto{\pgfqpoint{0.874397in}{1.449864in}}%
\pgfpathlineto{\pgfqpoint{0.864598in}{1.455838in}}%
\pgfpathlineto{\pgfqpoint{0.854799in}{1.461852in}}%
\pgfpathlineto{\pgfqpoint{0.845000in}{1.467906in}}%
\pgfpathclose%
\pgfusepath{stroke,fill}%
}%
\begin{pgfscope}%
\pgfsys@transformshift{0.000000in}{0.000000in}%
\pgfsys@useobject{currentmarker}{}%
\end{pgfscope}%
\end{pgfscope}%
\begin{pgfscope}%
\pgfsetrectcap%
\pgfsetmiterjoin%
\pgfsetlinewidth{1.254687pt}%
\definecolor{currentstroke}{rgb}{1.000000,1.000000,1.000000}%
\pgfsetstrokecolor{currentstroke}%
\pgfsetdash{}{0pt}%
\pgfpathmoveto{\pgfqpoint{0.845000in}{0.692500in}}%
\pgfpathlineto{\pgfqpoint{0.845000in}{3.572667in}}%
\pgfusepath{stroke}%
\end{pgfscope}%
\begin{pgfscope}%
\pgfsetrectcap%
\pgfsetmiterjoin%
\pgfsetlinewidth{1.254687pt}%
\definecolor{currentstroke}{rgb}{1.000000,1.000000,1.000000}%
\pgfsetstrokecolor{currentstroke}%
\pgfsetdash{}{0pt}%
\pgfpathmoveto{\pgfqpoint{5.734687in}{0.692500in}}%
\pgfpathlineto{\pgfqpoint{5.734687in}{3.572667in}}%
\pgfusepath{stroke}%
\end{pgfscope}%
\begin{pgfscope}%
\pgfsetrectcap%
\pgfsetmiterjoin%
\pgfsetlinewidth{1.254687pt}%
\definecolor{currentstroke}{rgb}{1.000000,1.000000,1.000000}%
\pgfsetstrokecolor{currentstroke}%
\pgfsetdash{}{0pt}%
\pgfpathmoveto{\pgfqpoint{0.845000in}{0.692500in}}%
\pgfpathlineto{\pgfqpoint{5.734687in}{0.692500in}}%
\pgfusepath{stroke}%
\end{pgfscope}%
\begin{pgfscope}%
\pgfsetrectcap%
\pgfsetmiterjoin%
\pgfsetlinewidth{1.254687pt}%
\definecolor{currentstroke}{rgb}{1.000000,1.000000,1.000000}%
\pgfsetstrokecolor{currentstroke}%
\pgfsetdash{}{0pt}%
\pgfpathmoveto{\pgfqpoint{0.845000in}{3.572667in}}%
\pgfpathlineto{\pgfqpoint{5.734687in}{3.572667in}}%
\pgfusepath{stroke}%
\end{pgfscope}%
\begin{pgfscope}%
\pgfpathrectangle{\pgfqpoint{0.845000in}{0.692500in}}{\pgfqpoint{4.889687in}{2.880167in}}%
\pgfusepath{clip}%
\pgfsetroundcap%
\pgfsetroundjoin%
\pgfsetlinewidth{1.505625pt}%
\definecolor{currentstroke}{rgb}{0.718353,0.818196,0.983608}%
\pgfsetstrokecolor{currentstroke}%
\pgfsetstrokeopacity{0.700000}%
\pgfsetdash{}{0pt}%
\pgfpathmoveto{\pgfqpoint{0.845000in}{1.467906in}}%
\pgfpathlineto{\pgfqpoint{0.903794in}{1.432184in}}%
\pgfpathlineto{\pgfqpoint{0.962588in}{1.397910in}}%
\pgfpathlineto{\pgfqpoint{1.021382in}{1.365084in}}%
\pgfpathlineto{\pgfqpoint{1.080175in}{1.333707in}}%
\pgfpathlineto{\pgfqpoint{1.138969in}{1.303778in}}%
\pgfpathlineto{\pgfqpoint{1.197763in}{1.275297in}}%
\pgfpathlineto{\pgfqpoint{1.256557in}{1.248265in}}%
\pgfpathlineto{\pgfqpoint{1.315351in}{1.222681in}}%
\pgfpathlineto{\pgfqpoint{1.364346in}{1.202468in}}%
\pgfpathlineto{\pgfqpoint{1.413340in}{1.183260in}}%
\pgfpathlineto{\pgfqpoint{1.462335in}{1.165058in}}%
\pgfpathlineto{\pgfqpoint{1.511330in}{1.147862in}}%
\pgfpathlineto{\pgfqpoint{1.560325in}{1.131672in}}%
\pgfpathlineto{\pgfqpoint{1.609320in}{1.116487in}}%
\pgfpathlineto{\pgfqpoint{1.658315in}{1.102309in}}%
\pgfpathlineto{\pgfqpoint{1.707310in}{1.089136in}}%
\pgfpathlineto{\pgfqpoint{1.756304in}{1.076969in}}%
\pgfpathlineto{\pgfqpoint{1.805299in}{1.065807in}}%
\pgfpathlineto{\pgfqpoint{1.854294in}{1.055652in}}%
\pgfpathlineto{\pgfqpoint{1.903289in}{1.046502in}}%
\pgfpathlineto{\pgfqpoint{1.952284in}{1.038358in}}%
\pgfpathlineto{\pgfqpoint{2.001279in}{1.031220in}}%
\pgfpathlineto{\pgfqpoint{2.050274in}{1.025088in}}%
\pgfpathlineto{\pgfqpoint{2.099269in}{1.019961in}}%
\pgfpathlineto{\pgfqpoint{2.148263in}{1.015841in}}%
\pgfpathlineto{\pgfqpoint{2.197258in}{1.012726in}}%
\pgfpathlineto{\pgfqpoint{2.246253in}{1.010616in}}%
\pgfpathlineto{\pgfqpoint{2.295248in}{1.009513in}}%
\pgfpathlineto{\pgfqpoint{2.344243in}{1.009416in}}%
\pgfpathlineto{\pgfqpoint{2.393238in}{1.010324in}}%
\pgfpathlineto{\pgfqpoint{2.442233in}{1.012238in}}%
\pgfpathlineto{\pgfqpoint{2.491227in}{1.015158in}}%
\pgfpathlineto{\pgfqpoint{2.540222in}{1.019084in}}%
\pgfpathlineto{\pgfqpoint{2.589217in}{1.024015in}}%
\pgfpathlineto{\pgfqpoint{2.638212in}{1.029952in}}%
\pgfpathlineto{\pgfqpoint{2.687207in}{1.036895in}}%
\pgfpathlineto{\pgfqpoint{2.736202in}{1.044844in}}%
\pgfpathlineto{\pgfqpoint{2.785197in}{1.053799in}}%
\pgfpathlineto{\pgfqpoint{2.834192in}{1.063759in}}%
\pgfpathlineto{\pgfqpoint{2.883186in}{1.074726in}}%
\pgfpathlineto{\pgfqpoint{2.932181in}{1.086698in}}%
\pgfpathlineto{\pgfqpoint{2.981176in}{1.099676in}}%
\pgfpathlineto{\pgfqpoint{3.030171in}{1.113659in}}%
\pgfpathlineto{\pgfqpoint{3.079166in}{1.128649in}}%
\pgfpathlineto{\pgfqpoint{3.128161in}{1.144644in}}%
\pgfpathlineto{\pgfqpoint{3.177156in}{1.161645in}}%
\pgfpathlineto{\pgfqpoint{3.226150in}{1.179652in}}%
\pgfpathlineto{\pgfqpoint{3.275145in}{1.198665in}}%
\pgfpathlineto{\pgfqpoint{3.324140in}{1.218683in}}%
\pgfpathlineto{\pgfqpoint{3.373135in}{1.239707in}}%
\pgfpathlineto{\pgfqpoint{3.431929in}{1.266264in}}%
\pgfpathlineto{\pgfqpoint{3.490723in}{1.294269in}}%
\pgfpathlineto{\pgfqpoint{3.549517in}{1.323723in}}%
\pgfpathlineto{\pgfqpoint{3.608310in}{1.354624in}}%
\pgfpathlineto{\pgfqpoint{3.667104in}{1.386975in}}%
\pgfpathlineto{\pgfqpoint{3.725898in}{1.420773in}}%
\pgfpathlineto{\pgfqpoint{3.784692in}{1.456020in}}%
\pgfpathlineto{\pgfqpoint{3.843486in}{1.492715in}}%
\pgfpathlineto{\pgfqpoint{3.902280in}{1.530859in}}%
\pgfpathlineto{\pgfqpoint{3.961073in}{1.570451in}}%
\pgfpathlineto{\pgfqpoint{4.019867in}{1.611491in}}%
\pgfpathlineto{\pgfqpoint{4.078661in}{1.653980in}}%
\pgfpathlineto{\pgfqpoint{4.137455in}{1.697917in}}%
\pgfpathlineto{\pgfqpoint{4.196249in}{1.743302in}}%
\pgfpathlineto{\pgfqpoint{4.255043in}{1.790136in}}%
\pgfpathlineto{\pgfqpoint{4.313836in}{1.838418in}}%
\pgfpathlineto{\pgfqpoint{4.372630in}{1.888148in}}%
\pgfpathlineto{\pgfqpoint{4.431424in}{1.939327in}}%
\pgfpathlineto{\pgfqpoint{4.490218in}{1.991954in}}%
\pgfpathlineto{\pgfqpoint{4.549012in}{2.046029in}}%
\pgfpathlineto{\pgfqpoint{4.607806in}{2.101553in}}%
\pgfpathlineto{\pgfqpoint{4.666599in}{2.158525in}}%
\pgfpathlineto{\pgfqpoint{4.725393in}{2.216946in}}%
\pgfpathlineto{\pgfqpoint{4.784187in}{2.276814in}}%
\pgfpathlineto{\pgfqpoint{4.842981in}{2.338132in}}%
\pgfpathlineto{\pgfqpoint{4.901775in}{2.400897in}}%
\pgfpathlineto{\pgfqpoint{4.960569in}{2.465111in}}%
\pgfpathlineto{\pgfqpoint{5.019362in}{2.530773in}}%
\pgfpathlineto{\pgfqpoint{5.078156in}{2.597884in}}%
\pgfpathlineto{\pgfqpoint{5.146749in}{2.678010in}}%
\pgfpathlineto{\pgfqpoint{5.215342in}{2.760108in}}%
\pgfpathlineto{\pgfqpoint{5.283935in}{2.844177in}}%
\pgfpathlineto{\pgfqpoint{5.352528in}{2.930217in}}%
\pgfpathlineto{\pgfqpoint{5.421120in}{3.018229in}}%
\pgfpathlineto{\pgfqpoint{5.489713in}{3.108212in}}%
\pgfpathlineto{\pgfqpoint{5.558306in}{3.200166in}}%
\pgfpathlineto{\pgfqpoint{5.626899in}{3.294092in}}%
\pgfpathlineto{\pgfqpoint{5.695492in}{3.389989in}}%
\pgfpathlineto{\pgfqpoint{5.734687in}{3.445673in}}%
\pgfpathlineto{\pgfqpoint{5.734687in}{3.445673in}}%
\pgfusepath{stroke}%
\end{pgfscope}%
\begin{pgfscope}%
\pgfpathrectangle{\pgfqpoint{0.845000in}{0.692500in}}{\pgfqpoint{4.889687in}{2.880167in}}%
\pgfusepath{clip}%
\pgfsetroundcap%
\pgfsetroundjoin%
\pgfsetlinewidth{1.505625pt}%
\definecolor{currentstroke}{rgb}{0.258824,0.521569,0.956863}%
\pgfsetstrokecolor{currentstroke}%
\pgfsetstrokeopacity{0.700000}%
\pgfsetdash{}{0pt}%
\pgfpathmoveto{\pgfqpoint{0.845000in}{1.200296in}}%
\pgfpathlineto{\pgfqpoint{0.913593in}{1.174378in}}%
\pgfpathlineto{\pgfqpoint{0.982186in}{1.149832in}}%
\pgfpathlineto{\pgfqpoint{1.040979in}{1.129886in}}%
\pgfpathlineto{\pgfqpoint{1.099773in}{1.110947in}}%
\pgfpathlineto{\pgfqpoint{1.158567in}{1.093018in}}%
\pgfpathlineto{\pgfqpoint{1.217361in}{1.076097in}}%
\pgfpathlineto{\pgfqpoint{1.276155in}{1.060184in}}%
\pgfpathlineto{\pgfqpoint{1.334949in}{1.045280in}}%
\pgfpathlineto{\pgfqpoint{1.393742in}{1.031385in}}%
\pgfpathlineto{\pgfqpoint{1.452536in}{1.018498in}}%
\pgfpathlineto{\pgfqpoint{1.511330in}{1.006620in}}%
\pgfpathlineto{\pgfqpoint{1.570124in}{0.995750in}}%
\pgfpathlineto{\pgfqpoint{1.628918in}{0.985888in}}%
\pgfpathlineto{\pgfqpoint{1.687712in}{0.977035in}}%
\pgfpathlineto{\pgfqpoint{1.746506in}{0.969191in}}%
\pgfpathlineto{\pgfqpoint{1.805299in}{0.962355in}}%
\pgfpathlineto{\pgfqpoint{1.864093in}{0.956528in}}%
\pgfpathlineto{\pgfqpoint{1.922887in}{0.951709in}}%
\pgfpathlineto{\pgfqpoint{1.981681in}{0.947899in}}%
\pgfpathlineto{\pgfqpoint{2.040475in}{0.945097in}}%
\pgfpathlineto{\pgfqpoint{2.099269in}{0.943303in}}%
\pgfpathlineto{\pgfqpoint{2.158062in}{0.942519in}}%
\pgfpathlineto{\pgfqpoint{2.216856in}{0.942742in}}%
\pgfpathlineto{\pgfqpoint{2.275650in}{0.943975in}}%
\pgfpathlineto{\pgfqpoint{2.334444in}{0.946215in}}%
\pgfpathlineto{\pgfqpoint{2.393238in}{0.949465in}}%
\pgfpathlineto{\pgfqpoint{2.452032in}{0.953722in}}%
\pgfpathlineto{\pgfqpoint{2.510825in}{0.958989in}}%
\pgfpathlineto{\pgfqpoint{2.569619in}{0.965263in}}%
\pgfpathlineto{\pgfqpoint{2.628413in}{0.972547in}}%
\pgfpathlineto{\pgfqpoint{2.687207in}{0.980839in}}%
\pgfpathlineto{\pgfqpoint{2.746001in}{0.990139in}}%
\pgfpathlineto{\pgfqpoint{2.804795in}{1.000448in}}%
\pgfpathlineto{\pgfqpoint{2.863588in}{1.011765in}}%
\pgfpathlineto{\pgfqpoint{2.922382in}{1.024091in}}%
\pgfpathlineto{\pgfqpoint{2.981176in}{1.037426in}}%
\pgfpathlineto{\pgfqpoint{3.039970in}{1.051769in}}%
\pgfpathlineto{\pgfqpoint{3.098764in}{1.067120in}}%
\pgfpathlineto{\pgfqpoint{3.157558in}{1.083480in}}%
\pgfpathlineto{\pgfqpoint{3.216351in}{1.100848in}}%
\pgfpathlineto{\pgfqpoint{3.275145in}{1.119225in}}%
\pgfpathlineto{\pgfqpoint{3.333939in}{1.138611in}}%
\pgfpathlineto{\pgfqpoint{3.392733in}{1.159005in}}%
\pgfpathlineto{\pgfqpoint{3.461326in}{1.184073in}}%
\pgfpathlineto{\pgfqpoint{3.529919in}{1.210513in}}%
\pgfpathlineto{\pgfqpoint{3.598511in}{1.238326in}}%
\pgfpathlineto{\pgfqpoint{3.667104in}{1.267512in}}%
\pgfpathlineto{\pgfqpoint{3.735697in}{1.298070in}}%
\pgfpathlineto{\pgfqpoint{3.804290in}{1.330001in}}%
\pgfpathlineto{\pgfqpoint{3.872883in}{1.363305in}}%
\pgfpathlineto{\pgfqpoint{3.941475in}{1.397982in}}%
\pgfpathlineto{\pgfqpoint{4.010068in}{1.434031in}}%
\pgfpathlineto{\pgfqpoint{4.078661in}{1.471453in}}%
\pgfpathlineto{\pgfqpoint{4.147254in}{1.510248in}}%
\pgfpathlineto{\pgfqpoint{4.215847in}{1.550415in}}%
\pgfpathlineto{\pgfqpoint{4.284440in}{1.591955in}}%
\pgfpathlineto{\pgfqpoint{4.353032in}{1.634868in}}%
\pgfpathlineto{\pgfqpoint{4.421625in}{1.679153in}}%
\pgfpathlineto{\pgfqpoint{4.490218in}{1.724811in}}%
\pgfpathlineto{\pgfqpoint{4.558811in}{1.771842in}}%
\pgfpathlineto{\pgfqpoint{4.627404in}{1.820246in}}%
\pgfpathlineto{\pgfqpoint{4.695996in}{1.870022in}}%
\pgfpathlineto{\pgfqpoint{4.764589in}{1.921171in}}%
\pgfpathlineto{\pgfqpoint{4.833182in}{1.973693in}}%
\pgfpathlineto{\pgfqpoint{4.901775in}{2.027587in}}%
\pgfpathlineto{\pgfqpoint{4.970368in}{2.082854in}}%
\pgfpathlineto{\pgfqpoint{5.038960in}{2.139494in}}%
\pgfpathlineto{\pgfqpoint{5.107553in}{2.197506in}}%
\pgfpathlineto{\pgfqpoint{5.176146in}{2.256892in}}%
\pgfpathlineto{\pgfqpoint{5.244739in}{2.317649in}}%
\pgfpathlineto{\pgfqpoint{5.313332in}{2.379780in}}%
\pgfpathlineto{\pgfqpoint{5.381924in}{2.443283in}}%
\pgfpathlineto{\pgfqpoint{5.450517in}{2.508159in}}%
\pgfpathlineto{\pgfqpoint{5.519110in}{2.574408in}}%
\pgfpathlineto{\pgfqpoint{5.587703in}{2.642029in}}%
\pgfpathlineto{\pgfqpoint{5.666095in}{2.720992in}}%
\pgfpathlineto{\pgfqpoint{5.734687in}{2.791555in}}%
\pgfpathlineto{\pgfqpoint{5.734687in}{2.791555in}}%
\pgfusepath{stroke}%
\end{pgfscope}%
\begin{pgfscope}%
\pgfpathrectangle{\pgfqpoint{0.845000in}{0.692500in}}{\pgfqpoint{4.889687in}{2.880167in}}%
\pgfusepath{clip}%
\pgfsetroundcap%
\pgfsetroundjoin%
\pgfsetlinewidth{1.505625pt}%
\definecolor{currentstroke}{rgb}{0.040118,0.284471,0.689294}%
\pgfsetstrokecolor{currentstroke}%
\pgfsetstrokeopacity{0.700000}%
\pgfsetdash{}{0pt}%
\pgfpathmoveto{\pgfqpoint{0.845000in}{1.088097in}}%
\pgfpathlineto{\pgfqpoint{0.913593in}{1.068659in}}%
\pgfpathlineto{\pgfqpoint{0.982186in}{1.050316in}}%
\pgfpathlineto{\pgfqpoint{1.050778in}{1.033069in}}%
\pgfpathlineto{\pgfqpoint{1.119371in}{1.016918in}}%
\pgfpathlineto{\pgfqpoint{1.187964in}{1.001862in}}%
\pgfpathlineto{\pgfqpoint{1.256557in}{0.987901in}}%
\pgfpathlineto{\pgfqpoint{1.325150in}{0.975036in}}%
\pgfpathlineto{\pgfqpoint{1.393742in}{0.963267in}}%
\pgfpathlineto{\pgfqpoint{1.462335in}{0.952593in}}%
\pgfpathlineto{\pgfqpoint{1.530928in}{0.943014in}}%
\pgfpathlineto{\pgfqpoint{1.599521in}{0.934531in}}%
\pgfpathlineto{\pgfqpoint{1.668114in}{0.927144in}}%
\pgfpathlineto{\pgfqpoint{1.736707in}{0.920852in}}%
\pgfpathlineto{\pgfqpoint{1.805299in}{0.915655in}}%
\pgfpathlineto{\pgfqpoint{1.873892in}{0.911554in}}%
\pgfpathlineto{\pgfqpoint{1.942485in}{0.908549in}}%
\pgfpathlineto{\pgfqpoint{2.011078in}{0.906639in}}%
\pgfpathlineto{\pgfqpoint{2.079671in}{0.905824in}}%
\pgfpathlineto{\pgfqpoint{2.148263in}{0.906105in}}%
\pgfpathlineto{\pgfqpoint{2.216856in}{0.907482in}}%
\pgfpathlineto{\pgfqpoint{2.285449in}{0.909954in}}%
\pgfpathlineto{\pgfqpoint{2.354042in}{0.913521in}}%
\pgfpathlineto{\pgfqpoint{2.422635in}{0.918184in}}%
\pgfpathlineto{\pgfqpoint{2.491227in}{0.923943in}}%
\pgfpathlineto{\pgfqpoint{2.559820in}{0.930797in}}%
\pgfpathlineto{\pgfqpoint{2.628413in}{0.938746in}}%
\pgfpathlineto{\pgfqpoint{2.697006in}{0.947791in}}%
\pgfpathlineto{\pgfqpoint{2.765599in}{0.957932in}}%
\pgfpathlineto{\pgfqpoint{2.834192in}{0.969168in}}%
\pgfpathlineto{\pgfqpoint{2.902784in}{0.981500in}}%
\pgfpathlineto{\pgfqpoint{2.971377in}{0.994927in}}%
\pgfpathlineto{\pgfqpoint{3.039970in}{1.009449in}}%
\pgfpathlineto{\pgfqpoint{3.108563in}{1.025067in}}%
\pgfpathlineto{\pgfqpoint{3.177156in}{1.041781in}}%
\pgfpathlineto{\pgfqpoint{3.245748in}{1.059590in}}%
\pgfpathlineto{\pgfqpoint{3.314341in}{1.078494in}}%
\pgfpathlineto{\pgfqpoint{3.382934in}{1.098494in}}%
\pgfpathlineto{\pgfqpoint{3.451527in}{1.119590in}}%
\pgfpathlineto{\pgfqpoint{3.520120in}{1.141781in}}%
\pgfpathlineto{\pgfqpoint{3.588712in}{1.165067in}}%
\pgfpathlineto{\pgfqpoint{3.657305in}{1.189449in}}%
\pgfpathlineto{\pgfqpoint{3.725898in}{1.214927in}}%
\pgfpathlineto{\pgfqpoint{3.794491in}{1.241500in}}%
\pgfpathlineto{\pgfqpoint{3.863084in}{1.269168in}}%
\pgfpathlineto{\pgfqpoint{3.931676in}{1.297933in}}%
\pgfpathlineto{\pgfqpoint{4.000269in}{1.327792in}}%
\pgfpathlineto{\pgfqpoint{4.068862in}{1.358747in}}%
\pgfpathlineto{\pgfqpoint{4.137455in}{1.390798in}}%
\pgfpathlineto{\pgfqpoint{4.206048in}{1.423944in}}%
\pgfpathlineto{\pgfqpoint{4.274641in}{1.458185in}}%
\pgfpathlineto{\pgfqpoint{4.343233in}{1.493522in}}%
\pgfpathlineto{\pgfqpoint{4.421625in}{1.535249in}}%
\pgfpathlineto{\pgfqpoint{4.500017in}{1.578406in}}%
\pgfpathlineto{\pgfqpoint{4.578409in}{1.622995in}}%
\pgfpathlineto{\pgfqpoint{4.656800in}{1.669014in}}%
\pgfpathlineto{\pgfqpoint{4.735192in}{1.716464in}}%
\pgfpathlineto{\pgfqpoint{4.813584in}{1.765345in}}%
\pgfpathlineto{\pgfqpoint{4.891976in}{1.815657in}}%
\pgfpathlineto{\pgfqpoint{4.970368in}{1.867399in}}%
\pgfpathlineto{\pgfqpoint{5.048759in}{1.920573in}}%
\pgfpathlineto{\pgfqpoint{5.127151in}{1.975177in}}%
\pgfpathlineto{\pgfqpoint{5.205543in}{2.031212in}}%
\pgfpathlineto{\pgfqpoint{5.283935in}{2.088678in}}%
\pgfpathlineto{\pgfqpoint{5.362327in}{2.147575in}}%
\pgfpathlineto{\pgfqpoint{5.440718in}{2.207903in}}%
\pgfpathlineto{\pgfqpoint{5.519110in}{2.269662in}}%
\pgfpathlineto{\pgfqpoint{5.597502in}{2.332851in}}%
\pgfpathlineto{\pgfqpoint{5.675894in}{2.397472in}}%
\pgfpathlineto{\pgfqpoint{5.734687in}{2.446876in}}%
\pgfpathlineto{\pgfqpoint{5.734687in}{2.446876in}}%
\pgfusepath{stroke}%
\end{pgfscope}%
\begin{pgfscope}%
\definecolor{textcolor}{rgb}{0.150000,0.150000,0.150000}%
\pgfsetstrokecolor{textcolor}%
\pgfsetfillcolor{textcolor}%
\pgftext[x=3.289844in,y=3.656000in,,base]{\color{textcolor}\sffamily\fontsize{12.000000}{14.400000}\selectfont Constrained Fitting of Drag on SD7032 Airfoil (\(\displaystyle \mathrm{Re}=10^6\))}%
\end{pgfscope}%
\begin{pgfscope}%
\pgfsetbuttcap%
\pgfsetmiterjoin%
\definecolor{currentfill}{rgb}{0.917647,0.917647,0.949020}%
\pgfsetfillcolor{currentfill}%
\pgfsetfillopacity{0.800000}%
\pgfsetlinewidth{1.003750pt}%
\definecolor{currentstroke}{rgb}{0.800000,0.800000,0.800000}%
\pgfsetstrokecolor{currentstroke}%
\pgfsetstrokeopacity{0.800000}%
\pgfsetdash{}{0pt}%
\pgfpathmoveto{\pgfqpoint{0.951944in}{2.598556in}}%
\pgfpathlineto{\pgfqpoint{2.495611in}{2.598556in}}%
\pgfpathquadraticcurveto{\pgfqpoint{2.526167in}{2.598556in}}{\pgfqpoint{2.526167in}{2.629112in}}%
\pgfpathlineto{\pgfqpoint{2.526167in}{3.465722in}}%
\pgfpathquadraticcurveto{\pgfqpoint{2.526167in}{3.496278in}}{\pgfqpoint{2.495611in}{3.496278in}}%
\pgfpathlineto{\pgfqpoint{0.951944in}{3.496278in}}%
\pgfpathquadraticcurveto{\pgfqpoint{0.921389in}{3.496278in}}{\pgfqpoint{0.921389in}{3.465722in}}%
\pgfpathlineto{\pgfqpoint{0.921389in}{2.629112in}}%
\pgfpathquadraticcurveto{\pgfqpoint{0.921389in}{2.598556in}}{\pgfqpoint{0.951944in}{2.598556in}}%
\pgfpathclose%
\pgfusepath{stroke,fill}%
\end{pgfscope}%
\begin{pgfscope}%
\pgfsetbuttcap%
\pgfsetroundjoin%
\definecolor{currentfill}{rgb}{0.000000,0.000000,0.000000}%
\pgfsetfillcolor{currentfill}%
\pgfsetfillopacity{0.600000}%
\pgfsetlinewidth{1.003750pt}%
\definecolor{currentstroke}{rgb}{0.000000,0.000000,0.000000}%
\pgfsetstrokecolor{currentstroke}%
\pgfsetstrokeopacity{0.600000}%
\pgfsetdash{}{0pt}%
\pgfsys@defobject{currentmarker}{\pgfqpoint{-0.020833in}{-0.020833in}}{\pgfqpoint{0.020833in}{0.020833in}}{%
\pgfpathmoveto{\pgfqpoint{0.000000in}{-0.020833in}}%
\pgfpathcurveto{\pgfqpoint{0.005525in}{-0.020833in}}{\pgfqpoint{0.010825in}{-0.018638in}}{\pgfqpoint{0.014731in}{-0.014731in}}%
\pgfpathcurveto{\pgfqpoint{0.018638in}{-0.010825in}}{\pgfqpoint{0.020833in}{-0.005525in}}{\pgfqpoint{0.020833in}{0.000000in}}%
\pgfpathcurveto{\pgfqpoint{0.020833in}{0.005525in}}{\pgfqpoint{0.018638in}{0.010825in}}{\pgfqpoint{0.014731in}{0.014731in}}%
\pgfpathcurveto{\pgfqpoint{0.010825in}{0.018638in}}{\pgfqpoint{0.005525in}{0.020833in}}{\pgfqpoint{0.000000in}{0.020833in}}%
\pgfpathcurveto{\pgfqpoint{-0.005525in}{0.020833in}}{\pgfqpoint{-0.010825in}{0.018638in}}{\pgfqpoint{-0.014731in}{0.014731in}}%
\pgfpathcurveto{\pgfqpoint{-0.018638in}{0.010825in}}{\pgfqpoint{-0.020833in}{0.005525in}}{\pgfqpoint{-0.020833in}{0.000000in}}%
\pgfpathcurveto{\pgfqpoint{-0.020833in}{-0.005525in}}{\pgfqpoint{-0.018638in}{-0.010825in}}{\pgfqpoint{-0.014731in}{-0.014731in}}%
\pgfpathcurveto{\pgfqpoint{-0.010825in}{-0.018638in}}{\pgfqpoint{-0.005525in}{-0.020833in}}{\pgfqpoint{0.000000in}{-0.020833in}}%
\pgfpathclose%
\pgfusepath{stroke,fill}%
}%
\begin{pgfscope}%
\pgfsys@transformshift{1.135278in}{3.381694in}%
\pgfsys@useobject{currentmarker}{}%
\end{pgfscope}%
\end{pgfscope}%
\begin{pgfscope}%
\definecolor{textcolor}{rgb}{0.150000,0.150000,0.150000}%
\pgfsetstrokecolor{textcolor}%
\pgfsetfillcolor{textcolor}%
\pgftext[x=1.410278in,y=3.328222in,left,base]{\color{textcolor}\sffamily\fontsize{11.000000}{13.200000}\selectfont XFoil Data}%
\end{pgfscope}%
\begin{pgfscope}%
\pgfsetroundcap%
\pgfsetroundjoin%
\pgfsetlinewidth{1.505625pt}%
\definecolor{currentstroke}{rgb}{0.718353,0.818196,0.983608}%
\pgfsetstrokecolor{currentstroke}%
\pgfsetstrokeopacity{0.700000}%
\pgfsetdash{}{0pt}%
\pgfpathmoveto{\pgfqpoint{0.982500in}{3.168722in}}%
\pgfpathlineto{\pgfqpoint{1.288056in}{3.168722in}}%
\pgfusepath{stroke}%
\end{pgfscope}%
\begin{pgfscope}%
\definecolor{textcolor}{rgb}{0.150000,0.150000,0.150000}%
\pgfsetstrokecolor{textcolor}%
\pgfsetfillcolor{textcolor}%
\pgftext[x=1.410278in,y=3.115250in,left,base]{\color{textcolor}\sffamily\fontsize{11.000000}{13.200000}\selectfont Upper Bound Fit}%
\end{pgfscope}%
\begin{pgfscope}%
\pgfsetroundcap%
\pgfsetroundjoin%
\pgfsetlinewidth{1.505625pt}%
\definecolor{currentstroke}{rgb}{0.258824,0.521569,0.956863}%
\pgfsetstrokecolor{currentstroke}%
\pgfsetstrokeopacity{0.700000}%
\pgfsetdash{}{0pt}%
\pgfpathmoveto{\pgfqpoint{0.982500in}{2.955750in}}%
\pgfpathlineto{\pgfqpoint{1.288056in}{2.955750in}}%
\pgfusepath{stroke}%
\end{pgfscope}%
\begin{pgfscope}%
\definecolor{textcolor}{rgb}{0.150000,0.150000,0.150000}%
\pgfsetstrokecolor{textcolor}%
\pgfsetfillcolor{textcolor}%
\pgftext[x=1.410278in,y=2.902278in,left,base]{\color{textcolor}\sffamily\fontsize{11.000000}{13.200000}\selectfont Best Fit}%
\end{pgfscope}%
\begin{pgfscope}%
\pgfsetroundcap%
\pgfsetroundjoin%
\pgfsetlinewidth{1.505625pt}%
\definecolor{currentstroke}{rgb}{0.040118,0.284471,0.689294}%
\pgfsetstrokecolor{currentstroke}%
\pgfsetstrokeopacity{0.700000}%
\pgfsetdash{}{0pt}%
\pgfpathmoveto{\pgfqpoint{0.982500in}{2.742778in}}%
\pgfpathlineto{\pgfqpoint{1.288056in}{2.742778in}}%
\pgfusepath{stroke}%
\end{pgfscope}%
\begin{pgfscope}%
\definecolor{textcolor}{rgb}{0.150000,0.150000,0.150000}%
\pgfsetstrokecolor{textcolor}%
\pgfsetfillcolor{textcolor}%
\pgftext[x=1.410278in,y=2.689306in,left,base]{\color{textcolor}\sffamily\fontsize{11.000000}{13.200000}\selectfont Lower Bound Fit}%
\end{pgfscope}%
\end{pgfpicture}%
\makeatother%
\endgroup%

    \ifdraft{}{%% Creator: Matplotlib, PGF backend
%%
%% To include the figure in your LaTeX document, write
%%   \input{<filename>.pgf}
%%
%% Make sure the required packages are loaded in your preamble
%%   \usepackage{pgf}
%%
%% Figures using additional raster images can only be included by \input if
%% they are in the same directory as the main LaTeX file. For loading figures
%% from other directories you can use the `import` package
%%   \usepackage{import}
%%
%% and then include the figures with
%%   \import{<path to file>}{<filename>.pgf}
%%
%% Matplotlib used the following preamble
%%   \usepackage{fontspec}
%%
\begingroup%
\makeatletter%
\begin{pgfpicture}%
\pgfpathrectangle{\pgfpointorigin}{\pgfqpoint{6.000000in}{4.000000in}}%
\pgfusepath{use as bounding box, clip}%
\begin{pgfscope}%
\pgfsetbuttcap%
\pgfsetmiterjoin%
\definecolor{currentfill}{rgb}{1.000000,1.000000,1.000000}%
\pgfsetfillcolor{currentfill}%
\pgfsetlinewidth{0.000000pt}%
\definecolor{currentstroke}{rgb}{1.000000,1.000000,1.000000}%
\pgfsetstrokecolor{currentstroke}%
\pgfsetstrokeopacity{0.000000}%
\pgfsetdash{}{0pt}%
\pgfpathmoveto{\pgfqpoint{0.000000in}{0.000000in}}%
\pgfpathlineto{\pgfqpoint{6.000000in}{0.000000in}}%
\pgfpathlineto{\pgfqpoint{6.000000in}{4.000000in}}%
\pgfpathlineto{\pgfqpoint{0.000000in}{4.000000in}}%
\pgfpathclose%
\pgfusepath{fill}%
\end{pgfscope}%
\begin{pgfscope}%
\pgfsetbuttcap%
\pgfsetmiterjoin%
\definecolor{currentfill}{rgb}{0.917647,0.917647,0.949020}%
\pgfsetfillcolor{currentfill}%
\pgfsetlinewidth{0.000000pt}%
\definecolor{currentstroke}{rgb}{0.000000,0.000000,0.000000}%
\pgfsetstrokecolor{currentstroke}%
\pgfsetstrokeopacity{0.000000}%
\pgfsetdash{}{0pt}%
\pgfpathmoveto{\pgfqpoint{0.845000in}{0.692500in}}%
\pgfpathlineto{\pgfqpoint{5.734687in}{0.692500in}}%
\pgfpathlineto{\pgfqpoint{5.734687in}{3.572667in}}%
\pgfpathlineto{\pgfqpoint{0.845000in}{3.572667in}}%
\pgfpathclose%
\pgfusepath{fill}%
\end{pgfscope}%
\begin{pgfscope}%
\pgfpathrectangle{\pgfqpoint{0.845000in}{0.692500in}}{\pgfqpoint{4.889687in}{2.880167in}}%
\pgfusepath{clip}%
\pgfsetroundcap%
\pgfsetroundjoin%
\pgfsetlinewidth{1.606000pt}%
\definecolor{currentstroke}{rgb}{1.000000,1.000000,1.000000}%
\pgfsetstrokecolor{currentstroke}%
\pgfsetdash{}{0pt}%
\pgfpathmoveto{\pgfqpoint{1.132629in}{0.692500in}}%
\pgfpathlineto{\pgfqpoint{1.132629in}{3.572667in}}%
\pgfusepath{stroke}%
\end{pgfscope}%
\begin{pgfscope}%
\definecolor{textcolor}{rgb}{0.150000,0.150000,0.150000}%
\pgfsetstrokecolor{textcolor}%
\pgfsetfillcolor{textcolor}%
\pgftext[x=1.132629in,y=0.560556in,,top]{\color{textcolor}\sffamily\fontsize{11.000000}{13.200000}\selectfont \ensuremath{-}2}%
\end{pgfscope}%
\begin{pgfscope}%
\pgfpathrectangle{\pgfqpoint{0.845000in}{0.692500in}}{\pgfqpoint{4.889687in}{2.880167in}}%
\pgfusepath{clip}%
\pgfsetroundcap%
\pgfsetroundjoin%
\pgfsetlinewidth{1.606000pt}%
\definecolor{currentstroke}{rgb}{1.000000,1.000000,1.000000}%
\pgfsetstrokecolor{currentstroke}%
\pgfsetdash{}{0pt}%
\pgfpathmoveto{\pgfqpoint{1.707886in}{0.692500in}}%
\pgfpathlineto{\pgfqpoint{1.707886in}{3.572667in}}%
\pgfusepath{stroke}%
\end{pgfscope}%
\begin{pgfscope}%
\definecolor{textcolor}{rgb}{0.150000,0.150000,0.150000}%
\pgfsetstrokecolor{textcolor}%
\pgfsetfillcolor{textcolor}%
\pgftext[x=1.707886in,y=0.560556in,,top]{\color{textcolor}\sffamily\fontsize{11.000000}{13.200000}\selectfont 0}%
\end{pgfscope}%
\begin{pgfscope}%
\pgfpathrectangle{\pgfqpoint{0.845000in}{0.692500in}}{\pgfqpoint{4.889687in}{2.880167in}}%
\pgfusepath{clip}%
\pgfsetroundcap%
\pgfsetroundjoin%
\pgfsetlinewidth{1.606000pt}%
\definecolor{currentstroke}{rgb}{1.000000,1.000000,1.000000}%
\pgfsetstrokecolor{currentstroke}%
\pgfsetdash{}{0pt}%
\pgfpathmoveto{\pgfqpoint{2.283143in}{0.692500in}}%
\pgfpathlineto{\pgfqpoint{2.283143in}{3.572667in}}%
\pgfusepath{stroke}%
\end{pgfscope}%
\begin{pgfscope}%
\definecolor{textcolor}{rgb}{0.150000,0.150000,0.150000}%
\pgfsetstrokecolor{textcolor}%
\pgfsetfillcolor{textcolor}%
\pgftext[x=2.283143in,y=0.560556in,,top]{\color{textcolor}\sffamily\fontsize{11.000000}{13.200000}\selectfont 2}%
\end{pgfscope}%
\begin{pgfscope}%
\pgfpathrectangle{\pgfqpoint{0.845000in}{0.692500in}}{\pgfqpoint{4.889687in}{2.880167in}}%
\pgfusepath{clip}%
\pgfsetroundcap%
\pgfsetroundjoin%
\pgfsetlinewidth{1.606000pt}%
\definecolor{currentstroke}{rgb}{1.000000,1.000000,1.000000}%
\pgfsetstrokecolor{currentstroke}%
\pgfsetdash{}{0pt}%
\pgfpathmoveto{\pgfqpoint{2.858401in}{0.692500in}}%
\pgfpathlineto{\pgfqpoint{2.858401in}{3.572667in}}%
\pgfusepath{stroke}%
\end{pgfscope}%
\begin{pgfscope}%
\definecolor{textcolor}{rgb}{0.150000,0.150000,0.150000}%
\pgfsetstrokecolor{textcolor}%
\pgfsetfillcolor{textcolor}%
\pgftext[x=2.858401in,y=0.560556in,,top]{\color{textcolor}\sffamily\fontsize{11.000000}{13.200000}\selectfont 4}%
\end{pgfscope}%
\begin{pgfscope}%
\pgfpathrectangle{\pgfqpoint{0.845000in}{0.692500in}}{\pgfqpoint{4.889687in}{2.880167in}}%
\pgfusepath{clip}%
\pgfsetroundcap%
\pgfsetroundjoin%
\pgfsetlinewidth{1.606000pt}%
\definecolor{currentstroke}{rgb}{1.000000,1.000000,1.000000}%
\pgfsetstrokecolor{currentstroke}%
\pgfsetdash{}{0pt}%
\pgfpathmoveto{\pgfqpoint{3.433658in}{0.692500in}}%
\pgfpathlineto{\pgfqpoint{3.433658in}{3.572667in}}%
\pgfusepath{stroke}%
\end{pgfscope}%
\begin{pgfscope}%
\definecolor{textcolor}{rgb}{0.150000,0.150000,0.150000}%
\pgfsetstrokecolor{textcolor}%
\pgfsetfillcolor{textcolor}%
\pgftext[x=3.433658in,y=0.560556in,,top]{\color{textcolor}\sffamily\fontsize{11.000000}{13.200000}\selectfont 6}%
\end{pgfscope}%
\begin{pgfscope}%
\pgfpathrectangle{\pgfqpoint{0.845000in}{0.692500in}}{\pgfqpoint{4.889687in}{2.880167in}}%
\pgfusepath{clip}%
\pgfsetroundcap%
\pgfsetroundjoin%
\pgfsetlinewidth{1.606000pt}%
\definecolor{currentstroke}{rgb}{1.000000,1.000000,1.000000}%
\pgfsetstrokecolor{currentstroke}%
\pgfsetdash{}{0pt}%
\pgfpathmoveto{\pgfqpoint{4.008915in}{0.692500in}}%
\pgfpathlineto{\pgfqpoint{4.008915in}{3.572667in}}%
\pgfusepath{stroke}%
\end{pgfscope}%
\begin{pgfscope}%
\definecolor{textcolor}{rgb}{0.150000,0.150000,0.150000}%
\pgfsetstrokecolor{textcolor}%
\pgfsetfillcolor{textcolor}%
\pgftext[x=4.008915in,y=0.560556in,,top]{\color{textcolor}\sffamily\fontsize{11.000000}{13.200000}\selectfont 8}%
\end{pgfscope}%
\begin{pgfscope}%
\pgfpathrectangle{\pgfqpoint{0.845000in}{0.692500in}}{\pgfqpoint{4.889687in}{2.880167in}}%
\pgfusepath{clip}%
\pgfsetroundcap%
\pgfsetroundjoin%
\pgfsetlinewidth{1.606000pt}%
\definecolor{currentstroke}{rgb}{1.000000,1.000000,1.000000}%
\pgfsetstrokecolor{currentstroke}%
\pgfsetdash{}{0pt}%
\pgfpathmoveto{\pgfqpoint{4.584173in}{0.692500in}}%
\pgfpathlineto{\pgfqpoint{4.584173in}{3.572667in}}%
\pgfusepath{stroke}%
\end{pgfscope}%
\begin{pgfscope}%
\definecolor{textcolor}{rgb}{0.150000,0.150000,0.150000}%
\pgfsetstrokecolor{textcolor}%
\pgfsetfillcolor{textcolor}%
\pgftext[x=4.584173in,y=0.560556in,,top]{\color{textcolor}\sffamily\fontsize{11.000000}{13.200000}\selectfont 10}%
\end{pgfscope}%
\begin{pgfscope}%
\pgfpathrectangle{\pgfqpoint{0.845000in}{0.692500in}}{\pgfqpoint{4.889687in}{2.880167in}}%
\pgfusepath{clip}%
\pgfsetroundcap%
\pgfsetroundjoin%
\pgfsetlinewidth{1.606000pt}%
\definecolor{currentstroke}{rgb}{1.000000,1.000000,1.000000}%
\pgfsetstrokecolor{currentstroke}%
\pgfsetdash{}{0pt}%
\pgfpathmoveto{\pgfqpoint{5.159430in}{0.692500in}}%
\pgfpathlineto{\pgfqpoint{5.159430in}{3.572667in}}%
\pgfusepath{stroke}%
\end{pgfscope}%
\begin{pgfscope}%
\definecolor{textcolor}{rgb}{0.150000,0.150000,0.150000}%
\pgfsetstrokecolor{textcolor}%
\pgfsetfillcolor{textcolor}%
\pgftext[x=5.159430in,y=0.560556in,,top]{\color{textcolor}\sffamily\fontsize{11.000000}{13.200000}\selectfont 12}%
\end{pgfscope}%
\begin{pgfscope}%
\pgfpathrectangle{\pgfqpoint{0.845000in}{0.692500in}}{\pgfqpoint{4.889687in}{2.880167in}}%
\pgfusepath{clip}%
\pgfsetroundcap%
\pgfsetroundjoin%
\pgfsetlinewidth{1.606000pt}%
\definecolor{currentstroke}{rgb}{1.000000,1.000000,1.000000}%
\pgfsetstrokecolor{currentstroke}%
\pgfsetdash{}{0pt}%
\pgfpathmoveto{\pgfqpoint{5.734687in}{0.692500in}}%
\pgfpathlineto{\pgfqpoint{5.734687in}{3.572667in}}%
\pgfusepath{stroke}%
\end{pgfscope}%
\begin{pgfscope}%
\definecolor{textcolor}{rgb}{0.150000,0.150000,0.150000}%
\pgfsetstrokecolor{textcolor}%
\pgfsetfillcolor{textcolor}%
\pgftext[x=5.734687in,y=0.560556in,,top]{\color{textcolor}\sffamily\fontsize{11.000000}{13.200000}\selectfont 14}%
\end{pgfscope}%
\begin{pgfscope}%
\pgfpathrectangle{\pgfqpoint{0.845000in}{0.692500in}}{\pgfqpoint{4.889687in}{2.880167in}}%
\pgfusepath{clip}%
\pgfsetroundcap%
\pgfsetroundjoin%
\pgfsetlinewidth{0.702625pt}%
\definecolor{currentstroke}{rgb}{1.000000,1.000000,1.000000}%
\pgfsetstrokecolor{currentstroke}%
\pgfsetdash{}{0pt}%
\pgfpathmoveto{\pgfqpoint{0.845000in}{0.692500in}}%
\pgfpathlineto{\pgfqpoint{0.845000in}{3.572667in}}%
\pgfusepath{stroke}%
\end{pgfscope}%
\begin{pgfscope}%
\pgfpathrectangle{\pgfqpoint{0.845000in}{0.692500in}}{\pgfqpoint{4.889687in}{2.880167in}}%
\pgfusepath{clip}%
\pgfsetroundcap%
\pgfsetroundjoin%
\pgfsetlinewidth{0.702625pt}%
\definecolor{currentstroke}{rgb}{1.000000,1.000000,1.000000}%
\pgfsetstrokecolor{currentstroke}%
\pgfsetdash{}{0pt}%
\pgfpathmoveto{\pgfqpoint{1.420257in}{0.692500in}}%
\pgfpathlineto{\pgfqpoint{1.420257in}{3.572667in}}%
\pgfusepath{stroke}%
\end{pgfscope}%
\begin{pgfscope}%
\pgfpathrectangle{\pgfqpoint{0.845000in}{0.692500in}}{\pgfqpoint{4.889687in}{2.880167in}}%
\pgfusepath{clip}%
\pgfsetroundcap%
\pgfsetroundjoin%
\pgfsetlinewidth{0.702625pt}%
\definecolor{currentstroke}{rgb}{1.000000,1.000000,1.000000}%
\pgfsetstrokecolor{currentstroke}%
\pgfsetdash{}{0pt}%
\pgfpathmoveto{\pgfqpoint{1.995515in}{0.692500in}}%
\pgfpathlineto{\pgfqpoint{1.995515in}{3.572667in}}%
\pgfusepath{stroke}%
\end{pgfscope}%
\begin{pgfscope}%
\pgfpathrectangle{\pgfqpoint{0.845000in}{0.692500in}}{\pgfqpoint{4.889687in}{2.880167in}}%
\pgfusepath{clip}%
\pgfsetroundcap%
\pgfsetroundjoin%
\pgfsetlinewidth{0.702625pt}%
\definecolor{currentstroke}{rgb}{1.000000,1.000000,1.000000}%
\pgfsetstrokecolor{currentstroke}%
\pgfsetdash{}{0pt}%
\pgfpathmoveto{\pgfqpoint{2.570772in}{0.692500in}}%
\pgfpathlineto{\pgfqpoint{2.570772in}{3.572667in}}%
\pgfusepath{stroke}%
\end{pgfscope}%
\begin{pgfscope}%
\pgfpathrectangle{\pgfqpoint{0.845000in}{0.692500in}}{\pgfqpoint{4.889687in}{2.880167in}}%
\pgfusepath{clip}%
\pgfsetroundcap%
\pgfsetroundjoin%
\pgfsetlinewidth{0.702625pt}%
\definecolor{currentstroke}{rgb}{1.000000,1.000000,1.000000}%
\pgfsetstrokecolor{currentstroke}%
\pgfsetdash{}{0pt}%
\pgfpathmoveto{\pgfqpoint{3.146029in}{0.692500in}}%
\pgfpathlineto{\pgfqpoint{3.146029in}{3.572667in}}%
\pgfusepath{stroke}%
\end{pgfscope}%
\begin{pgfscope}%
\pgfpathrectangle{\pgfqpoint{0.845000in}{0.692500in}}{\pgfqpoint{4.889687in}{2.880167in}}%
\pgfusepath{clip}%
\pgfsetroundcap%
\pgfsetroundjoin%
\pgfsetlinewidth{0.702625pt}%
\definecolor{currentstroke}{rgb}{1.000000,1.000000,1.000000}%
\pgfsetstrokecolor{currentstroke}%
\pgfsetdash{}{0pt}%
\pgfpathmoveto{\pgfqpoint{3.721287in}{0.692500in}}%
\pgfpathlineto{\pgfqpoint{3.721287in}{3.572667in}}%
\pgfusepath{stroke}%
\end{pgfscope}%
\begin{pgfscope}%
\pgfpathrectangle{\pgfqpoint{0.845000in}{0.692500in}}{\pgfqpoint{4.889687in}{2.880167in}}%
\pgfusepath{clip}%
\pgfsetroundcap%
\pgfsetroundjoin%
\pgfsetlinewidth{0.702625pt}%
\definecolor{currentstroke}{rgb}{1.000000,1.000000,1.000000}%
\pgfsetstrokecolor{currentstroke}%
\pgfsetdash{}{0pt}%
\pgfpathmoveto{\pgfqpoint{4.296544in}{0.692500in}}%
\pgfpathlineto{\pgfqpoint{4.296544in}{3.572667in}}%
\pgfusepath{stroke}%
\end{pgfscope}%
\begin{pgfscope}%
\pgfpathrectangle{\pgfqpoint{0.845000in}{0.692500in}}{\pgfqpoint{4.889687in}{2.880167in}}%
\pgfusepath{clip}%
\pgfsetroundcap%
\pgfsetroundjoin%
\pgfsetlinewidth{0.702625pt}%
\definecolor{currentstroke}{rgb}{1.000000,1.000000,1.000000}%
\pgfsetstrokecolor{currentstroke}%
\pgfsetdash{}{0pt}%
\pgfpathmoveto{\pgfqpoint{4.871801in}{0.692500in}}%
\pgfpathlineto{\pgfqpoint{4.871801in}{3.572667in}}%
\pgfusepath{stroke}%
\end{pgfscope}%
\begin{pgfscope}%
\pgfpathrectangle{\pgfqpoint{0.845000in}{0.692500in}}{\pgfqpoint{4.889687in}{2.880167in}}%
\pgfusepath{clip}%
\pgfsetroundcap%
\pgfsetroundjoin%
\pgfsetlinewidth{0.702625pt}%
\definecolor{currentstroke}{rgb}{1.000000,1.000000,1.000000}%
\pgfsetstrokecolor{currentstroke}%
\pgfsetdash{}{0pt}%
\pgfpathmoveto{\pgfqpoint{5.447059in}{0.692500in}}%
\pgfpathlineto{\pgfqpoint{5.447059in}{3.572667in}}%
\pgfusepath{stroke}%
\end{pgfscope}%
\begin{pgfscope}%
\definecolor{textcolor}{rgb}{0.150000,0.150000,0.150000}%
\pgfsetstrokecolor{textcolor}%
\pgfsetfillcolor{textcolor}%
\pgftext[x=3.289844in,y=0.369333in,,top]{\color{textcolor}\sffamily\fontsize{12.000000}{14.400000}\selectfont Angle of Attack \(\displaystyle \alpha\)}%
\end{pgfscope}%
\begin{pgfscope}%
\pgfpathrectangle{\pgfqpoint{0.845000in}{0.692500in}}{\pgfqpoint{4.889687in}{2.880167in}}%
\pgfusepath{clip}%
\pgfsetroundcap%
\pgfsetroundjoin%
\pgfsetlinewidth{1.606000pt}%
\definecolor{currentstroke}{rgb}{1.000000,1.000000,1.000000}%
\pgfsetstrokecolor{currentstroke}%
\pgfsetdash{}{0pt}%
\pgfpathmoveto{\pgfqpoint{0.845000in}{0.692500in}}%
\pgfpathlineto{\pgfqpoint{5.734687in}{0.692500in}}%
\pgfusepath{stroke}%
\end{pgfscope}%
\begin{pgfscope}%
\definecolor{textcolor}{rgb}{0.150000,0.150000,0.150000}%
\pgfsetstrokecolor{textcolor}%
\pgfsetfillcolor{textcolor}%
\pgftext[x=0.446917in, y=0.639486in, left, base]{\color{textcolor}\sffamily\fontsize{11.000000}{13.200000}\selectfont 0.00}%
\end{pgfscope}%
\begin{pgfscope}%
\pgfpathrectangle{\pgfqpoint{0.845000in}{0.692500in}}{\pgfqpoint{4.889687in}{2.880167in}}%
\pgfusepath{clip}%
\pgfsetroundcap%
\pgfsetroundjoin%
\pgfsetlinewidth{1.606000pt}%
\definecolor{currentstroke}{rgb}{1.000000,1.000000,1.000000}%
\pgfsetstrokecolor{currentstroke}%
\pgfsetdash{}{0pt}%
\pgfpathmoveto{\pgfqpoint{0.845000in}{1.219423in}}%
\pgfpathlineto{\pgfqpoint{5.734687in}{1.219423in}}%
\pgfusepath{stroke}%
\end{pgfscope}%
\begin{pgfscope}%
\definecolor{textcolor}{rgb}{0.150000,0.150000,0.150000}%
\pgfsetstrokecolor{textcolor}%
\pgfsetfillcolor{textcolor}%
\pgftext[x=0.446917in, y=1.166409in, left, base]{\color{textcolor}\sffamily\fontsize{11.000000}{13.200000}\selectfont 0.01}%
\end{pgfscope}%
\begin{pgfscope}%
\pgfpathrectangle{\pgfqpoint{0.845000in}{0.692500in}}{\pgfqpoint{4.889687in}{2.880167in}}%
\pgfusepath{clip}%
\pgfsetroundcap%
\pgfsetroundjoin%
\pgfsetlinewidth{1.606000pt}%
\definecolor{currentstroke}{rgb}{1.000000,1.000000,1.000000}%
\pgfsetstrokecolor{currentstroke}%
\pgfsetdash{}{0pt}%
\pgfpathmoveto{\pgfqpoint{0.845000in}{1.746346in}}%
\pgfpathlineto{\pgfqpoint{5.734687in}{1.746346in}}%
\pgfusepath{stroke}%
\end{pgfscope}%
\begin{pgfscope}%
\definecolor{textcolor}{rgb}{0.150000,0.150000,0.150000}%
\pgfsetstrokecolor{textcolor}%
\pgfsetfillcolor{textcolor}%
\pgftext[x=0.446917in, y=1.693332in, left, base]{\color{textcolor}\sffamily\fontsize{11.000000}{13.200000}\selectfont 0.02}%
\end{pgfscope}%
\begin{pgfscope}%
\pgfpathrectangle{\pgfqpoint{0.845000in}{0.692500in}}{\pgfqpoint{4.889687in}{2.880167in}}%
\pgfusepath{clip}%
\pgfsetroundcap%
\pgfsetroundjoin%
\pgfsetlinewidth{1.606000pt}%
\definecolor{currentstroke}{rgb}{1.000000,1.000000,1.000000}%
\pgfsetstrokecolor{currentstroke}%
\pgfsetdash{}{0pt}%
\pgfpathmoveto{\pgfqpoint{0.845000in}{2.273268in}}%
\pgfpathlineto{\pgfqpoint{5.734687in}{2.273268in}}%
\pgfusepath{stroke}%
\end{pgfscope}%
\begin{pgfscope}%
\definecolor{textcolor}{rgb}{0.150000,0.150000,0.150000}%
\pgfsetstrokecolor{textcolor}%
\pgfsetfillcolor{textcolor}%
\pgftext[x=0.446917in, y=2.220255in, left, base]{\color{textcolor}\sffamily\fontsize{11.000000}{13.200000}\selectfont 0.03}%
\end{pgfscope}%
\begin{pgfscope}%
\pgfpathrectangle{\pgfqpoint{0.845000in}{0.692500in}}{\pgfqpoint{4.889687in}{2.880167in}}%
\pgfusepath{clip}%
\pgfsetroundcap%
\pgfsetroundjoin%
\pgfsetlinewidth{1.606000pt}%
\definecolor{currentstroke}{rgb}{1.000000,1.000000,1.000000}%
\pgfsetstrokecolor{currentstroke}%
\pgfsetdash{}{0pt}%
\pgfpathmoveto{\pgfqpoint{0.845000in}{2.800191in}}%
\pgfpathlineto{\pgfqpoint{5.734687in}{2.800191in}}%
\pgfusepath{stroke}%
\end{pgfscope}%
\begin{pgfscope}%
\definecolor{textcolor}{rgb}{0.150000,0.150000,0.150000}%
\pgfsetstrokecolor{textcolor}%
\pgfsetfillcolor{textcolor}%
\pgftext[x=0.446917in, y=2.747177in, left, base]{\color{textcolor}\sffamily\fontsize{11.000000}{13.200000}\selectfont 0.04}%
\end{pgfscope}%
\begin{pgfscope}%
\pgfpathrectangle{\pgfqpoint{0.845000in}{0.692500in}}{\pgfqpoint{4.889687in}{2.880167in}}%
\pgfusepath{clip}%
\pgfsetroundcap%
\pgfsetroundjoin%
\pgfsetlinewidth{1.606000pt}%
\definecolor{currentstroke}{rgb}{1.000000,1.000000,1.000000}%
\pgfsetstrokecolor{currentstroke}%
\pgfsetdash{}{0pt}%
\pgfpathmoveto{\pgfqpoint{0.845000in}{3.327114in}}%
\pgfpathlineto{\pgfqpoint{5.734687in}{3.327114in}}%
\pgfusepath{stroke}%
\end{pgfscope}%
\begin{pgfscope}%
\definecolor{textcolor}{rgb}{0.150000,0.150000,0.150000}%
\pgfsetstrokecolor{textcolor}%
\pgfsetfillcolor{textcolor}%
\pgftext[x=0.446917in, y=3.274100in, left, base]{\color{textcolor}\sffamily\fontsize{11.000000}{13.200000}\selectfont 0.05}%
\end{pgfscope}%
\begin{pgfscope}%
\pgfpathrectangle{\pgfqpoint{0.845000in}{0.692500in}}{\pgfqpoint{4.889687in}{2.880167in}}%
\pgfusepath{clip}%
\pgfsetroundcap%
\pgfsetroundjoin%
\pgfsetlinewidth{0.702625pt}%
\definecolor{currentstroke}{rgb}{1.000000,1.000000,1.000000}%
\pgfsetstrokecolor{currentstroke}%
\pgfsetdash{}{0pt}%
\pgfpathmoveto{\pgfqpoint{0.845000in}{0.955961in}}%
\pgfpathlineto{\pgfqpoint{5.734687in}{0.955961in}}%
\pgfusepath{stroke}%
\end{pgfscope}%
\begin{pgfscope}%
\pgfpathrectangle{\pgfqpoint{0.845000in}{0.692500in}}{\pgfqpoint{4.889687in}{2.880167in}}%
\pgfusepath{clip}%
\pgfsetroundcap%
\pgfsetroundjoin%
\pgfsetlinewidth{0.702625pt}%
\definecolor{currentstroke}{rgb}{1.000000,1.000000,1.000000}%
\pgfsetstrokecolor{currentstroke}%
\pgfsetdash{}{0pt}%
\pgfpathmoveto{\pgfqpoint{0.845000in}{1.482884in}}%
\pgfpathlineto{\pgfqpoint{5.734687in}{1.482884in}}%
\pgfusepath{stroke}%
\end{pgfscope}%
\begin{pgfscope}%
\pgfpathrectangle{\pgfqpoint{0.845000in}{0.692500in}}{\pgfqpoint{4.889687in}{2.880167in}}%
\pgfusepath{clip}%
\pgfsetroundcap%
\pgfsetroundjoin%
\pgfsetlinewidth{0.702625pt}%
\definecolor{currentstroke}{rgb}{1.000000,1.000000,1.000000}%
\pgfsetstrokecolor{currentstroke}%
\pgfsetdash{}{0pt}%
\pgfpathmoveto{\pgfqpoint{0.845000in}{2.009807in}}%
\pgfpathlineto{\pgfqpoint{5.734687in}{2.009807in}}%
\pgfusepath{stroke}%
\end{pgfscope}%
\begin{pgfscope}%
\pgfpathrectangle{\pgfqpoint{0.845000in}{0.692500in}}{\pgfqpoint{4.889687in}{2.880167in}}%
\pgfusepath{clip}%
\pgfsetroundcap%
\pgfsetroundjoin%
\pgfsetlinewidth{0.702625pt}%
\definecolor{currentstroke}{rgb}{1.000000,1.000000,1.000000}%
\pgfsetstrokecolor{currentstroke}%
\pgfsetdash{}{0pt}%
\pgfpathmoveto{\pgfqpoint{0.845000in}{2.536730in}}%
\pgfpathlineto{\pgfqpoint{5.734687in}{2.536730in}}%
\pgfusepath{stroke}%
\end{pgfscope}%
\begin{pgfscope}%
\pgfpathrectangle{\pgfqpoint{0.845000in}{0.692500in}}{\pgfqpoint{4.889687in}{2.880167in}}%
\pgfusepath{clip}%
\pgfsetroundcap%
\pgfsetroundjoin%
\pgfsetlinewidth{0.702625pt}%
\definecolor{currentstroke}{rgb}{1.000000,1.000000,1.000000}%
\pgfsetstrokecolor{currentstroke}%
\pgfsetdash{}{0pt}%
\pgfpathmoveto{\pgfqpoint{0.845000in}{3.063653in}}%
\pgfpathlineto{\pgfqpoint{5.734687in}{3.063653in}}%
\pgfusepath{stroke}%
\end{pgfscope}%
\begin{pgfscope}%
\definecolor{textcolor}{rgb}{0.150000,0.150000,0.150000}%
\pgfsetstrokecolor{textcolor}%
\pgfsetfillcolor{textcolor}%
\pgftext[x=0.391361in,y=2.132583in,,bottom,rotate=90.000000]{\color{textcolor}\sffamily\fontsize{12.000000}{14.400000}\selectfont Drag Coefficient \(\displaystyle C_D\)}%
\end{pgfscope}%
\begin{pgfscope}%
\pgfpathrectangle{\pgfqpoint{0.845000in}{0.692500in}}{\pgfqpoint{4.889687in}{2.880167in}}%
\pgfusepath{clip}%
\pgfsetbuttcap%
\pgfsetroundjoin%
\definecolor{currentfill}{rgb}{0.000000,0.000000,0.000000}%
\pgfsetfillcolor{currentfill}%
\pgfsetfillopacity{0.600000}%
\pgfsetlinewidth{1.003750pt}%
\definecolor{currentstroke}{rgb}{0.000000,0.000000,0.000000}%
\pgfsetstrokecolor{currentstroke}%
\pgfsetstrokeopacity{0.600000}%
\pgfsetdash{}{0pt}%
\pgfsys@defobject{currentmarker}{\pgfqpoint{-0.020833in}{-0.020833in}}{\pgfqpoint{0.020833in}{0.020833in}}{%
\pgfpathmoveto{\pgfqpoint{0.000000in}{-0.020833in}}%
\pgfpathcurveto{\pgfqpoint{0.005525in}{-0.020833in}}{\pgfqpoint{0.010825in}{-0.018638in}}{\pgfqpoint{0.014731in}{-0.014731in}}%
\pgfpathcurveto{\pgfqpoint{0.018638in}{-0.010825in}}{\pgfqpoint{0.020833in}{-0.005525in}}{\pgfqpoint{0.020833in}{0.000000in}}%
\pgfpathcurveto{\pgfqpoint{0.020833in}{0.005525in}}{\pgfqpoint{0.018638in}{0.010825in}}{\pgfqpoint{0.014731in}{0.014731in}}%
\pgfpathcurveto{\pgfqpoint{0.010825in}{0.018638in}}{\pgfqpoint{0.005525in}{0.020833in}}{\pgfqpoint{0.000000in}{0.020833in}}%
\pgfpathcurveto{\pgfqpoint{-0.005525in}{0.020833in}}{\pgfqpoint{-0.010825in}{0.018638in}}{\pgfqpoint{-0.014731in}{0.014731in}}%
\pgfpathcurveto{\pgfqpoint{-0.018638in}{0.010825in}}{\pgfqpoint{-0.020833in}{0.005525in}}{\pgfqpoint{-0.020833in}{0.000000in}}%
\pgfpathcurveto{\pgfqpoint{-0.020833in}{-0.005525in}}{\pgfqpoint{-0.018638in}{-0.010825in}}{\pgfqpoint{-0.014731in}{-0.014731in}}%
\pgfpathcurveto{\pgfqpoint{-0.010825in}{-0.018638in}}{\pgfqpoint{-0.005525in}{-0.020833in}}{\pgfqpoint{0.000000in}{-0.020833in}}%
\pgfpathclose%
\pgfusepath{stroke,fill}%
}%
\begin{pgfscope}%
\pgfsys@transformshift{1.742977in}{0.987050in}%
\pgfsys@useobject{currentmarker}{}%
\end{pgfscope}%
\begin{pgfscope}%
\pgfsys@transformshift{1.842784in}{0.984415in}%
\pgfsys@useobject{currentmarker}{}%
\end{pgfscope}%
\begin{pgfscope}%
\pgfsys@transformshift{1.942591in}{0.978092in}%
\pgfsys@useobject{currentmarker}{}%
\end{pgfscope}%
\begin{pgfscope}%
\pgfsys@transformshift{2.042398in}{0.978092in}%
\pgfsys@useobject{currentmarker}{}%
\end{pgfscope}%
\begin{pgfscope}%
\pgfsys@transformshift{2.142205in}{0.973877in}%
\pgfsys@useobject{currentmarker}{}%
\end{pgfscope}%
\begin{pgfscope}%
\pgfsys@transformshift{2.242012in}{0.984942in}%
\pgfsys@useobject{currentmarker}{}%
\end{pgfscope}%
\begin{pgfscope}%
\pgfsys@transformshift{2.341820in}{0.994954in}%
\pgfsys@useobject{currentmarker}{}%
\end{pgfscope}%
\begin{pgfscope}%
\pgfsys@transformshift{2.441627in}{1.007600in}%
\pgfsys@useobject{currentmarker}{}%
\end{pgfscope}%
\begin{pgfscope}%
\pgfsys@transformshift{2.541434in}{1.019192in}%
\pgfsys@useobject{currentmarker}{}%
\end{pgfscope}%
\begin{pgfscope}%
\pgfsys@transformshift{2.641241in}{1.030258in}%
\pgfsys@useobject{currentmarker}{}%
\end{pgfscope}%
\begin{pgfscope}%
\pgfsys@transformshift{2.741048in}{1.045538in}%
\pgfsys@useobject{currentmarker}{}%
\end{pgfscope}%
\begin{pgfscope}%
\pgfsys@transformshift{2.840855in}{1.055550in}%
\pgfsys@useobject{currentmarker}{}%
\end{pgfscope}%
\begin{pgfscope}%
\pgfsys@transformshift{2.940663in}{1.075046in}%
\pgfsys@useobject{currentmarker}{}%
\end{pgfscope}%
\begin{pgfscope}%
\pgfsys@transformshift{3.040470in}{1.085584in}%
\pgfsys@useobject{currentmarker}{}%
\end{pgfscope}%
\begin{pgfscope}%
\pgfsys@transformshift{3.140277in}{1.106661in}%
\pgfsys@useobject{currentmarker}{}%
\end{pgfscope}%
\begin{pgfscope}%
\pgfsys@transformshift{3.240084in}{1.118781in}%
\pgfsys@useobject{currentmarker}{}%
\end{pgfscope}%
\begin{pgfscope}%
\pgfsys@transformshift{3.339604in}{1.141438in}%
\pgfsys@useobject{currentmarker}{}%
\end{pgfscope}%
\begin{pgfscope}%
\pgfsys@transformshift{3.439411in}{1.233123in}%
\pgfsys@useobject{currentmarker}{}%
\end{pgfscope}%
\begin{pgfscope}%
\pgfsys@transformshift{3.539218in}{1.293192in}%
\pgfsys@useobject{currentmarker}{}%
\end{pgfscope}%
\begin{pgfscope}%
\pgfsys@transformshift{3.639025in}{1.349046in}%
\pgfsys@useobject{currentmarker}{}%
\end{pgfscope}%
\begin{pgfscope}%
\pgfsys@transformshift{3.738832in}{1.401211in}%
\pgfsys@useobject{currentmarker}{}%
\end{pgfscope}%
\begin{pgfscope}%
\pgfsys@transformshift{3.838639in}{1.428611in}%
\pgfsys@useobject{currentmarker}{}%
\end{pgfscope}%
\begin{pgfscope}%
\pgfsys@transformshift{3.938446in}{1.450742in}%
\pgfsys@useobject{currentmarker}{}%
\end{pgfscope}%
\begin{pgfscope}%
\pgfsys@transformshift{4.038254in}{1.475507in}%
\pgfsys@useobject{currentmarker}{}%
\end{pgfscope}%
\begin{pgfscope}%
\pgfsys@transformshift{4.138061in}{1.501853in}%
\pgfsys@useobject{currentmarker}{}%
\end{pgfscope}%
\begin{pgfscope}%
\pgfsys@transformshift{4.237868in}{1.530307in}%
\pgfsys@useobject{currentmarker}{}%
\end{pgfscope}%
\begin{pgfscope}%
\pgfsys@transformshift{4.337675in}{1.558234in}%
\pgfsys@useobject{currentmarker}{}%
\end{pgfscope}%
\begin{pgfscope}%
\pgfsys@transformshift{4.437482in}{1.590376in}%
\pgfsys@useobject{currentmarker}{}%
\end{pgfscope}%
\begin{pgfscope}%
\pgfsys@transformshift{4.537289in}{1.627261in}%
\pgfsys@useobject{currentmarker}{}%
\end{pgfscope}%
\begin{pgfscope}%
\pgfsys@transformshift{4.637096in}{1.670996in}%
\pgfsys@useobject{currentmarker}{}%
\end{pgfscope}%
\begin{pgfscope}%
\pgfsys@transformshift{4.736904in}{1.721580in}%
\pgfsys@useobject{currentmarker}{}%
\end{pgfscope}%
\begin{pgfscope}%
\pgfsys@transformshift{4.836711in}{1.780069in}%
\pgfsys@useobject{currentmarker}{}%
\end{pgfscope}%
\begin{pgfscope}%
\pgfsys@transformshift{4.936230in}{1.847515in}%
\pgfsys@useobject{currentmarker}{}%
\end{pgfscope}%
\begin{pgfscope}%
\pgfsys@transformshift{5.036037in}{1.926026in}%
\pgfsys@useobject{currentmarker}{}%
\end{pgfscope}%
\begin{pgfscope}%
\pgfsys@transformshift{5.135845in}{2.018238in}%
\pgfsys@useobject{currentmarker}{}%
\end{pgfscope}%
\begin{pgfscope}%
\pgfsys@transformshift{5.235652in}{2.128365in}%
\pgfsys@useobject{currentmarker}{}%
\end{pgfscope}%
\begin{pgfscope}%
\pgfsys@transformshift{5.335459in}{2.260095in}%
\pgfsys@useobject{currentmarker}{}%
\end{pgfscope}%
\begin{pgfscope}%
\pgfsys@transformshift{5.435266in}{2.436088in}%
\pgfsys@useobject{currentmarker}{}%
\end{pgfscope}%
\begin{pgfscope}%
\pgfsys@transformshift{5.535073in}{2.676891in}%
\pgfsys@useobject{currentmarker}{}%
\end{pgfscope}%
\begin{pgfscope}%
\pgfsys@transformshift{5.634880in}{3.011487in}%
\pgfsys@useobject{currentmarker}{}%
\end{pgfscope}%
\begin{pgfscope}%
\pgfsys@transformshift{5.734687in}{3.445672in}%
\pgfsys@useobject{currentmarker}{}%
\end{pgfscope}%
\begin{pgfscope}%
\pgfsys@transformshift{1.643457in}{0.996008in}%
\pgfsys@useobject{currentmarker}{}%
\end{pgfscope}%
\begin{pgfscope}%
\pgfsys@transformshift{1.543650in}{1.000750in}%
\pgfsys@useobject{currentmarker}{}%
\end{pgfscope}%
\begin{pgfscope}%
\pgfsys@transformshift{1.443843in}{1.013923in}%
\pgfsys@useobject{currentmarker}{}%
\end{pgfscope}%
\begin{pgfscope}%
\pgfsys@transformshift{1.344036in}{1.022881in}%
\pgfsys@useobject{currentmarker}{}%
\end{pgfscope}%
\begin{pgfscope}%
\pgfsys@transformshift{1.244229in}{1.051861in}%
\pgfsys@useobject{currentmarker}{}%
\end{pgfscope}%
\begin{pgfscope}%
\pgfsys@transformshift{1.144421in}{1.066088in}%
\pgfsys@useobject{currentmarker}{}%
\end{pgfscope}%
\begin{pgfscope}%
\pgfsys@transformshift{1.044614in}{1.073992in}%
\pgfsys@useobject{currentmarker}{}%
\end{pgfscope}%
\begin{pgfscope}%
\pgfsys@transformshift{0.944807in}{1.080842in}%
\pgfsys@useobject{currentmarker}{}%
\end{pgfscope}%
\begin{pgfscope}%
\pgfsys@transformshift{0.845000in}{1.088219in}%
\pgfsys@useobject{currentmarker}{}%
\end{pgfscope}%
\end{pgfscope}%
\begin{pgfscope}%
\pgfpathrectangle{\pgfqpoint{0.845000in}{0.692500in}}{\pgfqpoint{4.889687in}{2.880167in}}%
\pgfusepath{clip}%
\pgfsetbuttcap%
\pgfsetroundjoin%
\definecolor{currentfill}{rgb}{0.258824,0.521569,0.956863}%
\pgfsetfillcolor{currentfill}%
\pgfsetfillopacity{0.200000}%
\pgfsetlinewidth{1.003750pt}%
\definecolor{currentstroke}{rgb}{0.258824,0.521569,0.956863}%
\pgfsetstrokecolor{currentstroke}%
\pgfsetstrokeopacity{0.200000}%
\pgfsetdash{}{0pt}%
\pgfsys@defobject{currentmarker}{\pgfqpoint{0.845000in}{0.905793in}}{\pgfqpoint{5.734687in}{3.445673in}}{%
\pgfpathmoveto{\pgfqpoint{0.845000in}{1.467906in}}%
\pgfpathlineto{\pgfqpoint{0.845000in}{1.088097in}}%
\pgfpathlineto{\pgfqpoint{0.854799in}{1.085253in}}%
\pgfpathlineto{\pgfqpoint{0.864598in}{1.082431in}}%
\pgfpathlineto{\pgfqpoint{0.874397in}{1.079632in}}%
\pgfpathlineto{\pgfqpoint{0.884196in}{1.076855in}}%
\pgfpathlineto{\pgfqpoint{0.893995in}{1.074101in}}%
\pgfpathlineto{\pgfqpoint{0.903794in}{1.071368in}}%
\pgfpathlineto{\pgfqpoint{0.913593in}{1.068659in}}%
\pgfpathlineto{\pgfqpoint{0.923392in}{1.065971in}}%
\pgfpathlineto{\pgfqpoint{0.933191in}{1.063306in}}%
\pgfpathlineto{\pgfqpoint{0.942990in}{1.060663in}}%
\pgfpathlineto{\pgfqpoint{0.952789in}{1.058043in}}%
\pgfpathlineto{\pgfqpoint{0.962588in}{1.055445in}}%
\pgfpathlineto{\pgfqpoint{0.972387in}{1.052869in}}%
\pgfpathlineto{\pgfqpoint{0.982186in}{1.050316in}}%
\pgfpathlineto{\pgfqpoint{0.991985in}{1.047785in}}%
\pgfpathlineto{\pgfqpoint{1.001784in}{1.045277in}}%
\pgfpathlineto{\pgfqpoint{1.011583in}{1.042790in}}%
\pgfpathlineto{\pgfqpoint{1.021382in}{1.040327in}}%
\pgfpathlineto{\pgfqpoint{1.031180in}{1.037885in}}%
\pgfpathlineto{\pgfqpoint{1.040979in}{1.035466in}}%
\pgfpathlineto{\pgfqpoint{1.050778in}{1.033069in}}%
\pgfpathlineto{\pgfqpoint{1.060577in}{1.030695in}}%
\pgfpathlineto{\pgfqpoint{1.070376in}{1.028343in}}%
\pgfpathlineto{\pgfqpoint{1.080175in}{1.026013in}}%
\pgfpathlineto{\pgfqpoint{1.089974in}{1.023706in}}%
\pgfpathlineto{\pgfqpoint{1.099773in}{1.021421in}}%
\pgfpathlineto{\pgfqpoint{1.109572in}{1.019158in}}%
\pgfpathlineto{\pgfqpoint{1.119371in}{1.016918in}}%
\pgfpathlineto{\pgfqpoint{1.129170in}{1.014700in}}%
\pgfpathlineto{\pgfqpoint{1.138969in}{1.012504in}}%
\pgfpathlineto{\pgfqpoint{1.148768in}{1.010331in}}%
\pgfpathlineto{\pgfqpoint{1.158567in}{1.008180in}}%
\pgfpathlineto{\pgfqpoint{1.168366in}{1.006052in}}%
\pgfpathlineto{\pgfqpoint{1.178165in}{1.003945in}}%
\pgfpathlineto{\pgfqpoint{1.187964in}{1.001862in}}%
\pgfpathlineto{\pgfqpoint{1.197763in}{0.999800in}}%
\pgfpathlineto{\pgfqpoint{1.207562in}{0.997761in}}%
\pgfpathlineto{\pgfqpoint{1.217361in}{0.995744in}}%
\pgfpathlineto{\pgfqpoint{1.227160in}{0.993750in}}%
\pgfpathlineto{\pgfqpoint{1.236959in}{0.991778in}}%
\pgfpathlineto{\pgfqpoint{1.246758in}{0.989828in}}%
\pgfpathlineto{\pgfqpoint{1.256557in}{0.987901in}}%
\pgfpathlineto{\pgfqpoint{1.266356in}{0.985996in}}%
\pgfpathlineto{\pgfqpoint{1.276155in}{0.984114in}}%
\pgfpathlineto{\pgfqpoint{1.285954in}{0.982253in}}%
\pgfpathlineto{\pgfqpoint{1.295753in}{0.980416in}}%
\pgfpathlineto{\pgfqpoint{1.305552in}{0.978600in}}%
\pgfpathlineto{\pgfqpoint{1.315351in}{0.976807in}}%
\pgfpathlineto{\pgfqpoint{1.325150in}{0.975036in}}%
\pgfpathlineto{\pgfqpoint{1.334949in}{0.973288in}}%
\pgfpathlineto{\pgfqpoint{1.344748in}{0.971562in}}%
\pgfpathlineto{\pgfqpoint{1.354547in}{0.969858in}}%
\pgfpathlineto{\pgfqpoint{1.364346in}{0.968177in}}%
\pgfpathlineto{\pgfqpoint{1.374145in}{0.966518in}}%
\pgfpathlineto{\pgfqpoint{1.383944in}{0.964881in}}%
\pgfpathlineto{\pgfqpoint{1.393742in}{0.963267in}}%
\pgfpathlineto{\pgfqpoint{1.403541in}{0.961675in}}%
\pgfpathlineto{\pgfqpoint{1.413340in}{0.960105in}}%
\pgfpathlineto{\pgfqpoint{1.423139in}{0.958558in}}%
\pgfpathlineto{\pgfqpoint{1.432938in}{0.957033in}}%
\pgfpathlineto{\pgfqpoint{1.442737in}{0.955531in}}%
\pgfpathlineto{\pgfqpoint{1.452536in}{0.954050in}}%
\pgfpathlineto{\pgfqpoint{1.462335in}{0.952593in}}%
\pgfpathlineto{\pgfqpoint{1.472134in}{0.951157in}}%
\pgfpathlineto{\pgfqpoint{1.481933in}{0.949744in}}%
\pgfpathlineto{\pgfqpoint{1.491732in}{0.948353in}}%
\pgfpathlineto{\pgfqpoint{1.501531in}{0.946985in}}%
\pgfpathlineto{\pgfqpoint{1.511330in}{0.945639in}}%
\pgfpathlineto{\pgfqpoint{1.521129in}{0.944315in}}%
\pgfpathlineto{\pgfqpoint{1.530928in}{0.943014in}}%
\pgfpathlineto{\pgfqpoint{1.540727in}{0.941735in}}%
\pgfpathlineto{\pgfqpoint{1.550526in}{0.940479in}}%
\pgfpathlineto{\pgfqpoint{1.560325in}{0.939244in}}%
\pgfpathlineto{\pgfqpoint{1.570124in}{0.938033in}}%
\pgfpathlineto{\pgfqpoint{1.579923in}{0.936843in}}%
\pgfpathlineto{\pgfqpoint{1.589722in}{0.935676in}}%
\pgfpathlineto{\pgfqpoint{1.599521in}{0.934531in}}%
\pgfpathlineto{\pgfqpoint{1.609320in}{0.933409in}}%
\pgfpathlineto{\pgfqpoint{1.619119in}{0.932309in}}%
\pgfpathlineto{\pgfqpoint{1.628918in}{0.931231in}}%
\pgfpathlineto{\pgfqpoint{1.638717in}{0.930176in}}%
\pgfpathlineto{\pgfqpoint{1.648516in}{0.929143in}}%
\pgfpathlineto{\pgfqpoint{1.658315in}{0.928132in}}%
\pgfpathlineto{\pgfqpoint{1.668114in}{0.927144in}}%
\pgfpathlineto{\pgfqpoint{1.677913in}{0.926178in}}%
\pgfpathlineto{\pgfqpoint{1.687712in}{0.925234in}}%
\pgfpathlineto{\pgfqpoint{1.697511in}{0.924313in}}%
\pgfpathlineto{\pgfqpoint{1.707310in}{0.923414in}}%
\pgfpathlineto{\pgfqpoint{1.717109in}{0.922538in}}%
\pgfpathlineto{\pgfqpoint{1.726908in}{0.921683in}}%
\pgfpathlineto{\pgfqpoint{1.736707in}{0.920852in}}%
\pgfpathlineto{\pgfqpoint{1.746506in}{0.920042in}}%
\pgfpathlineto{\pgfqpoint{1.756304in}{0.919255in}}%
\pgfpathlineto{\pgfqpoint{1.766103in}{0.918490in}}%
\pgfpathlineto{\pgfqpoint{1.775902in}{0.917748in}}%
\pgfpathlineto{\pgfqpoint{1.785701in}{0.917028in}}%
\pgfpathlineto{\pgfqpoint{1.795500in}{0.916330in}}%
\pgfpathlineto{\pgfqpoint{1.805299in}{0.915655in}}%
\pgfpathlineto{\pgfqpoint{1.815098in}{0.915002in}}%
\pgfpathlineto{\pgfqpoint{1.824897in}{0.914372in}}%
\pgfpathlineto{\pgfqpoint{1.834696in}{0.913764in}}%
\pgfpathlineto{\pgfqpoint{1.844495in}{0.913178in}}%
\pgfpathlineto{\pgfqpoint{1.854294in}{0.912614in}}%
\pgfpathlineto{\pgfqpoint{1.864093in}{0.912073in}}%
\pgfpathlineto{\pgfqpoint{1.873892in}{0.911554in}}%
\pgfpathlineto{\pgfqpoint{1.883691in}{0.911058in}}%
\pgfpathlineto{\pgfqpoint{1.893490in}{0.910584in}}%
\pgfpathlineto{\pgfqpoint{1.903289in}{0.910132in}}%
\pgfpathlineto{\pgfqpoint{1.913088in}{0.909703in}}%
\pgfpathlineto{\pgfqpoint{1.922887in}{0.909296in}}%
\pgfpathlineto{\pgfqpoint{1.932686in}{0.908911in}}%
\pgfpathlineto{\pgfqpoint{1.942485in}{0.908549in}}%
\pgfpathlineto{\pgfqpoint{1.952284in}{0.908209in}}%
\pgfpathlineto{\pgfqpoint{1.962083in}{0.907891in}}%
\pgfpathlineto{\pgfqpoint{1.971882in}{0.907596in}}%
\pgfpathlineto{\pgfqpoint{1.981681in}{0.907323in}}%
\pgfpathlineto{\pgfqpoint{1.991480in}{0.907073in}}%
\pgfpathlineto{\pgfqpoint{2.001279in}{0.906845in}}%
\pgfpathlineto{\pgfqpoint{2.011078in}{0.906639in}}%
\pgfpathlineto{\pgfqpoint{2.020877in}{0.906455in}}%
\pgfpathlineto{\pgfqpoint{2.030676in}{0.906294in}}%
\pgfpathlineto{\pgfqpoint{2.040475in}{0.906156in}}%
\pgfpathlineto{\pgfqpoint{2.050274in}{0.906039in}}%
\pgfpathlineto{\pgfqpoint{2.060073in}{0.905945in}}%
\pgfpathlineto{\pgfqpoint{2.069872in}{0.905874in}}%
\pgfpathlineto{\pgfqpoint{2.079671in}{0.905824in}}%
\pgfpathlineto{\pgfqpoint{2.089470in}{0.905797in}}%
\pgfpathlineto{\pgfqpoint{2.099269in}{0.905793in}}%
\pgfpathlineto{\pgfqpoint{2.109068in}{0.905811in}}%
\pgfpathlineto{\pgfqpoint{2.118866in}{0.905851in}}%
\pgfpathlineto{\pgfqpoint{2.128665in}{0.905913in}}%
\pgfpathlineto{\pgfqpoint{2.138464in}{0.905998in}}%
\pgfpathlineto{\pgfqpoint{2.148263in}{0.906105in}}%
\pgfpathlineto{\pgfqpoint{2.158062in}{0.906235in}}%
\pgfpathlineto{\pgfqpoint{2.167861in}{0.906387in}}%
\pgfpathlineto{\pgfqpoint{2.177660in}{0.906561in}}%
\pgfpathlineto{\pgfqpoint{2.187459in}{0.906758in}}%
\pgfpathlineto{\pgfqpoint{2.197258in}{0.906977in}}%
\pgfpathlineto{\pgfqpoint{2.207057in}{0.907218in}}%
\pgfpathlineto{\pgfqpoint{2.216856in}{0.907482in}}%
\pgfpathlineto{\pgfqpoint{2.226655in}{0.907768in}}%
\pgfpathlineto{\pgfqpoint{2.236454in}{0.908076in}}%
\pgfpathlineto{\pgfqpoint{2.246253in}{0.908407in}}%
\pgfpathlineto{\pgfqpoint{2.256052in}{0.908760in}}%
\pgfpathlineto{\pgfqpoint{2.265851in}{0.909136in}}%
\pgfpathlineto{\pgfqpoint{2.275650in}{0.909534in}}%
\pgfpathlineto{\pgfqpoint{2.285449in}{0.909954in}}%
\pgfpathlineto{\pgfqpoint{2.295248in}{0.910396in}}%
\pgfpathlineto{\pgfqpoint{2.305047in}{0.910861in}}%
\pgfpathlineto{\pgfqpoint{2.314846in}{0.911349in}}%
\pgfpathlineto{\pgfqpoint{2.324645in}{0.911858in}}%
\pgfpathlineto{\pgfqpoint{2.334444in}{0.912390in}}%
\pgfpathlineto{\pgfqpoint{2.344243in}{0.912945in}}%
\pgfpathlineto{\pgfqpoint{2.354042in}{0.913521in}}%
\pgfpathlineto{\pgfqpoint{2.363841in}{0.914120in}}%
\pgfpathlineto{\pgfqpoint{2.373640in}{0.914742in}}%
\pgfpathlineto{\pgfqpoint{2.383439in}{0.915386in}}%
\pgfpathlineto{\pgfqpoint{2.393238in}{0.916052in}}%
\pgfpathlineto{\pgfqpoint{2.403037in}{0.916740in}}%
\pgfpathlineto{\pgfqpoint{2.412836in}{0.917451in}}%
\pgfpathlineto{\pgfqpoint{2.422635in}{0.918184in}}%
\pgfpathlineto{\pgfqpoint{2.432434in}{0.918940in}}%
\pgfpathlineto{\pgfqpoint{2.442233in}{0.919718in}}%
\pgfpathlineto{\pgfqpoint{2.452032in}{0.920518in}}%
\pgfpathlineto{\pgfqpoint{2.461831in}{0.921341in}}%
\pgfpathlineto{\pgfqpoint{2.471630in}{0.922186in}}%
\pgfpathlineto{\pgfqpoint{2.481428in}{0.923053in}}%
\pgfpathlineto{\pgfqpoint{2.491227in}{0.923943in}}%
\pgfpathlineto{\pgfqpoint{2.501026in}{0.924855in}}%
\pgfpathlineto{\pgfqpoint{2.510825in}{0.925789in}}%
\pgfpathlineto{\pgfqpoint{2.520624in}{0.926746in}}%
\pgfpathlineto{\pgfqpoint{2.530423in}{0.927725in}}%
\pgfpathlineto{\pgfqpoint{2.540222in}{0.928727in}}%
\pgfpathlineto{\pgfqpoint{2.550021in}{0.929751in}}%
\pgfpathlineto{\pgfqpoint{2.559820in}{0.930797in}}%
\pgfpathlineto{\pgfqpoint{2.569619in}{0.931865in}}%
\pgfpathlineto{\pgfqpoint{2.579418in}{0.932956in}}%
\pgfpathlineto{\pgfqpoint{2.589217in}{0.934070in}}%
\pgfpathlineto{\pgfqpoint{2.599016in}{0.935205in}}%
\pgfpathlineto{\pgfqpoint{2.608815in}{0.936363in}}%
\pgfpathlineto{\pgfqpoint{2.618614in}{0.937544in}}%
\pgfpathlineto{\pgfqpoint{2.628413in}{0.938746in}}%
\pgfpathlineto{\pgfqpoint{2.638212in}{0.939972in}}%
\pgfpathlineto{\pgfqpoint{2.648011in}{0.941219in}}%
\pgfpathlineto{\pgfqpoint{2.657810in}{0.942489in}}%
\pgfpathlineto{\pgfqpoint{2.667609in}{0.943781in}}%
\pgfpathlineto{\pgfqpoint{2.677408in}{0.945095in}}%
\pgfpathlineto{\pgfqpoint{2.687207in}{0.946432in}}%
\pgfpathlineto{\pgfqpoint{2.697006in}{0.947791in}}%
\pgfpathlineto{\pgfqpoint{2.706805in}{0.949173in}}%
\pgfpathlineto{\pgfqpoint{2.716604in}{0.950577in}}%
\pgfpathlineto{\pgfqpoint{2.726403in}{0.952003in}}%
\pgfpathlineto{\pgfqpoint{2.736202in}{0.953452in}}%
\pgfpathlineto{\pgfqpoint{2.746001in}{0.954923in}}%
\pgfpathlineto{\pgfqpoint{2.755800in}{0.956416in}}%
\pgfpathlineto{\pgfqpoint{2.765599in}{0.957932in}}%
\pgfpathlineto{\pgfqpoint{2.775398in}{0.959470in}}%
\pgfpathlineto{\pgfqpoint{2.785197in}{0.961030in}}%
\pgfpathlineto{\pgfqpoint{2.794996in}{0.962613in}}%
\pgfpathlineto{\pgfqpoint{2.804795in}{0.964218in}}%
\pgfpathlineto{\pgfqpoint{2.814594in}{0.965846in}}%
\pgfpathlineto{\pgfqpoint{2.824393in}{0.967496in}}%
\pgfpathlineto{\pgfqpoint{2.834192in}{0.969168in}}%
\pgfpathlineto{\pgfqpoint{2.843990in}{0.970863in}}%
\pgfpathlineto{\pgfqpoint{2.853789in}{0.972580in}}%
\pgfpathlineto{\pgfqpoint{2.863588in}{0.974319in}}%
\pgfpathlineto{\pgfqpoint{2.873387in}{0.976080in}}%
\pgfpathlineto{\pgfqpoint{2.883186in}{0.977864in}}%
\pgfpathlineto{\pgfqpoint{2.892985in}{0.979671in}}%
\pgfpathlineto{\pgfqpoint{2.902784in}{0.981500in}}%
\pgfpathlineto{\pgfqpoint{2.912583in}{0.983351in}}%
\pgfpathlineto{\pgfqpoint{2.922382in}{0.985224in}}%
\pgfpathlineto{\pgfqpoint{2.932181in}{0.987120in}}%
\pgfpathlineto{\pgfqpoint{2.941980in}{0.989038in}}%
\pgfpathlineto{\pgfqpoint{2.951779in}{0.990978in}}%
\pgfpathlineto{\pgfqpoint{2.961578in}{0.992941in}}%
\pgfpathlineto{\pgfqpoint{2.971377in}{0.994927in}}%
\pgfpathlineto{\pgfqpoint{2.981176in}{0.996934in}}%
\pgfpathlineto{\pgfqpoint{2.990975in}{0.998964in}}%
\pgfpathlineto{\pgfqpoint{3.000774in}{1.001016in}}%
\pgfpathlineto{\pgfqpoint{3.010573in}{1.003091in}}%
\pgfpathlineto{\pgfqpoint{3.020372in}{1.005188in}}%
\pgfpathlineto{\pgfqpoint{3.030171in}{1.007307in}}%
\pgfpathlineto{\pgfqpoint{3.039970in}{1.009449in}}%
\pgfpathlineto{\pgfqpoint{3.049769in}{1.011613in}}%
\pgfpathlineto{\pgfqpoint{3.059568in}{1.013800in}}%
\pgfpathlineto{\pgfqpoint{3.069367in}{1.016008in}}%
\pgfpathlineto{\pgfqpoint{3.079166in}{1.018240in}}%
\pgfpathlineto{\pgfqpoint{3.088965in}{1.020493in}}%
\pgfpathlineto{\pgfqpoint{3.098764in}{1.022769in}}%
\pgfpathlineto{\pgfqpoint{3.108563in}{1.025067in}}%
\pgfpathlineto{\pgfqpoint{3.118362in}{1.027388in}}%
\pgfpathlineto{\pgfqpoint{3.128161in}{1.029731in}}%
\pgfpathlineto{\pgfqpoint{3.137960in}{1.032096in}}%
\pgfpathlineto{\pgfqpoint{3.147759in}{1.034484in}}%
\pgfpathlineto{\pgfqpoint{3.157558in}{1.036894in}}%
\pgfpathlineto{\pgfqpoint{3.167357in}{1.039326in}}%
\pgfpathlineto{\pgfqpoint{3.177156in}{1.041781in}}%
\pgfpathlineto{\pgfqpoint{3.186955in}{1.044258in}}%
\pgfpathlineto{\pgfqpoint{3.196754in}{1.046757in}}%
\pgfpathlineto{\pgfqpoint{3.206552in}{1.049279in}}%
\pgfpathlineto{\pgfqpoint{3.216351in}{1.051823in}}%
\pgfpathlineto{\pgfqpoint{3.226150in}{1.054390in}}%
\pgfpathlineto{\pgfqpoint{3.235949in}{1.056978in}}%
\pgfpathlineto{\pgfqpoint{3.245748in}{1.059590in}}%
\pgfpathlineto{\pgfqpoint{3.255547in}{1.062223in}}%
\pgfpathlineto{\pgfqpoint{3.265346in}{1.064879in}}%
\pgfpathlineto{\pgfqpoint{3.275145in}{1.067557in}}%
\pgfpathlineto{\pgfqpoint{3.284944in}{1.070258in}}%
\pgfpathlineto{\pgfqpoint{3.294743in}{1.072981in}}%
\pgfpathlineto{\pgfqpoint{3.304542in}{1.075726in}}%
\pgfpathlineto{\pgfqpoint{3.314341in}{1.078494in}}%
\pgfpathlineto{\pgfqpoint{3.324140in}{1.081284in}}%
\pgfpathlineto{\pgfqpoint{3.333939in}{1.084097in}}%
\pgfpathlineto{\pgfqpoint{3.343738in}{1.086932in}}%
\pgfpathlineto{\pgfqpoint{3.353537in}{1.089789in}}%
\pgfpathlineto{\pgfqpoint{3.363336in}{1.092668in}}%
\pgfpathlineto{\pgfqpoint{3.373135in}{1.095570in}}%
\pgfpathlineto{\pgfqpoint{3.382934in}{1.098494in}}%
\pgfpathlineto{\pgfqpoint{3.392733in}{1.101441in}}%
\pgfpathlineto{\pgfqpoint{3.402532in}{1.104410in}}%
\pgfpathlineto{\pgfqpoint{3.412331in}{1.107401in}}%
\pgfpathlineto{\pgfqpoint{3.422130in}{1.110415in}}%
\pgfpathlineto{\pgfqpoint{3.431929in}{1.113451in}}%
\pgfpathlineto{\pgfqpoint{3.441728in}{1.116509in}}%
\pgfpathlineto{\pgfqpoint{3.451527in}{1.119590in}}%
\pgfpathlineto{\pgfqpoint{3.461326in}{1.122693in}}%
\pgfpathlineto{\pgfqpoint{3.471125in}{1.125818in}}%
\pgfpathlineto{\pgfqpoint{3.480924in}{1.128966in}}%
\pgfpathlineto{\pgfqpoint{3.490723in}{1.132136in}}%
\pgfpathlineto{\pgfqpoint{3.500522in}{1.135329in}}%
\pgfpathlineto{\pgfqpoint{3.510321in}{1.138544in}}%
\pgfpathlineto{\pgfqpoint{3.520120in}{1.141781in}}%
\pgfpathlineto{\pgfqpoint{3.529919in}{1.145040in}}%
\pgfpathlineto{\pgfqpoint{3.539718in}{1.148322in}}%
\pgfpathlineto{\pgfqpoint{3.549517in}{1.151627in}}%
\pgfpathlineto{\pgfqpoint{3.559316in}{1.154953in}}%
\pgfpathlineto{\pgfqpoint{3.569114in}{1.158302in}}%
\pgfpathlineto{\pgfqpoint{3.578913in}{1.161674in}}%
\pgfpathlineto{\pgfqpoint{3.588712in}{1.165067in}}%
\pgfpathlineto{\pgfqpoint{3.598511in}{1.168483in}}%
\pgfpathlineto{\pgfqpoint{3.608310in}{1.171922in}}%
\pgfpathlineto{\pgfqpoint{3.618109in}{1.175383in}}%
\pgfpathlineto{\pgfqpoint{3.627908in}{1.178866in}}%
\pgfpathlineto{\pgfqpoint{3.637707in}{1.182371in}}%
\pgfpathlineto{\pgfqpoint{3.647506in}{1.185899in}}%
\pgfpathlineto{\pgfqpoint{3.657305in}{1.189449in}}%
\pgfpathlineto{\pgfqpoint{3.667104in}{1.193022in}}%
\pgfpathlineto{\pgfqpoint{3.676903in}{1.196617in}}%
\pgfpathlineto{\pgfqpoint{3.686702in}{1.200234in}}%
\pgfpathlineto{\pgfqpoint{3.696501in}{1.203874in}}%
\pgfpathlineto{\pgfqpoint{3.706300in}{1.207536in}}%
\pgfpathlineto{\pgfqpoint{3.716099in}{1.211220in}}%
\pgfpathlineto{\pgfqpoint{3.725898in}{1.214927in}}%
\pgfpathlineto{\pgfqpoint{3.735697in}{1.218656in}}%
\pgfpathlineto{\pgfqpoint{3.745496in}{1.222407in}}%
\pgfpathlineto{\pgfqpoint{3.755295in}{1.226181in}}%
\pgfpathlineto{\pgfqpoint{3.765094in}{1.229977in}}%
\pgfpathlineto{\pgfqpoint{3.774893in}{1.233796in}}%
\pgfpathlineto{\pgfqpoint{3.784692in}{1.237637in}}%
\pgfpathlineto{\pgfqpoint{3.794491in}{1.241500in}}%
\pgfpathlineto{\pgfqpoint{3.804290in}{1.245386in}}%
\pgfpathlineto{\pgfqpoint{3.814089in}{1.249293in}}%
\pgfpathlineto{\pgfqpoint{3.823888in}{1.253224in}}%
\pgfpathlineto{\pgfqpoint{3.833687in}{1.257176in}}%
\pgfpathlineto{\pgfqpoint{3.843486in}{1.261151in}}%
\pgfpathlineto{\pgfqpoint{3.853285in}{1.265149in}}%
\pgfpathlineto{\pgfqpoint{3.863084in}{1.269168in}}%
\pgfpathlineto{\pgfqpoint{3.872883in}{1.273211in}}%
\pgfpathlineto{\pgfqpoint{3.882682in}{1.277275in}}%
\pgfpathlineto{\pgfqpoint{3.892481in}{1.281362in}}%
\pgfpathlineto{\pgfqpoint{3.902280in}{1.285471in}}%
\pgfpathlineto{\pgfqpoint{3.912079in}{1.289602in}}%
\pgfpathlineto{\pgfqpoint{3.921878in}{1.293756in}}%
\pgfpathlineto{\pgfqpoint{3.931676in}{1.297933in}}%
\pgfpathlineto{\pgfqpoint{3.941475in}{1.302131in}}%
\pgfpathlineto{\pgfqpoint{3.951274in}{1.306352in}}%
\pgfpathlineto{\pgfqpoint{3.961073in}{1.310595in}}%
\pgfpathlineto{\pgfqpoint{3.970872in}{1.314861in}}%
\pgfpathlineto{\pgfqpoint{3.980671in}{1.319149in}}%
\pgfpathlineto{\pgfqpoint{3.990470in}{1.323459in}}%
\pgfpathlineto{\pgfqpoint{4.000269in}{1.327792in}}%
\pgfpathlineto{\pgfqpoint{4.010068in}{1.332147in}}%
\pgfpathlineto{\pgfqpoint{4.019867in}{1.336525in}}%
\pgfpathlineto{\pgfqpoint{4.029666in}{1.340924in}}%
\pgfpathlineto{\pgfqpoint{4.039465in}{1.345347in}}%
\pgfpathlineto{\pgfqpoint{4.049264in}{1.349791in}}%
\pgfpathlineto{\pgfqpoint{4.059063in}{1.354258in}}%
\pgfpathlineto{\pgfqpoint{4.068862in}{1.358747in}}%
\pgfpathlineto{\pgfqpoint{4.078661in}{1.363259in}}%
\pgfpathlineto{\pgfqpoint{4.088460in}{1.367793in}}%
\pgfpathlineto{\pgfqpoint{4.098259in}{1.372349in}}%
\pgfpathlineto{\pgfqpoint{4.108058in}{1.376928in}}%
\pgfpathlineto{\pgfqpoint{4.117857in}{1.381529in}}%
\pgfpathlineto{\pgfqpoint{4.127656in}{1.386152in}}%
\pgfpathlineto{\pgfqpoint{4.137455in}{1.390798in}}%
\pgfpathlineto{\pgfqpoint{4.147254in}{1.395466in}}%
\pgfpathlineto{\pgfqpoint{4.157053in}{1.400156in}}%
\pgfpathlineto{\pgfqpoint{4.166852in}{1.404869in}}%
\pgfpathlineto{\pgfqpoint{4.176651in}{1.409604in}}%
\pgfpathlineto{\pgfqpoint{4.186450in}{1.414362in}}%
\pgfpathlineto{\pgfqpoint{4.196249in}{1.419141in}}%
\pgfpathlineto{\pgfqpoint{4.206048in}{1.423944in}}%
\pgfpathlineto{\pgfqpoint{4.215847in}{1.428768in}}%
\pgfpathlineto{\pgfqpoint{4.225646in}{1.433615in}}%
\pgfpathlineto{\pgfqpoint{4.235445in}{1.438484in}}%
\pgfpathlineto{\pgfqpoint{4.245244in}{1.443376in}}%
\pgfpathlineto{\pgfqpoint{4.255043in}{1.448290in}}%
\pgfpathlineto{\pgfqpoint{4.264842in}{1.453227in}}%
\pgfpathlineto{\pgfqpoint{4.274641in}{1.458185in}}%
\pgfpathlineto{\pgfqpoint{4.284440in}{1.463166in}}%
\pgfpathlineto{\pgfqpoint{4.294238in}{1.468170in}}%
\pgfpathlineto{\pgfqpoint{4.304037in}{1.473196in}}%
\pgfpathlineto{\pgfqpoint{4.313836in}{1.478244in}}%
\pgfpathlineto{\pgfqpoint{4.323635in}{1.483314in}}%
\pgfpathlineto{\pgfqpoint{4.333434in}{1.488407in}}%
\pgfpathlineto{\pgfqpoint{4.343233in}{1.493522in}}%
\pgfpathlineto{\pgfqpoint{4.353032in}{1.498660in}}%
\pgfpathlineto{\pgfqpoint{4.362831in}{1.503820in}}%
\pgfpathlineto{\pgfqpoint{4.372630in}{1.509002in}}%
\pgfpathlineto{\pgfqpoint{4.382429in}{1.514207in}}%
\pgfpathlineto{\pgfqpoint{4.392228in}{1.519434in}}%
\pgfpathlineto{\pgfqpoint{4.402027in}{1.524683in}}%
\pgfpathlineto{\pgfqpoint{4.411826in}{1.529955in}}%
\pgfpathlineto{\pgfqpoint{4.421625in}{1.535249in}}%
\pgfpathlineto{\pgfqpoint{4.431424in}{1.540565in}}%
\pgfpathlineto{\pgfqpoint{4.441223in}{1.545904in}}%
\pgfpathlineto{\pgfqpoint{4.451022in}{1.551265in}}%
\pgfpathlineto{\pgfqpoint{4.460821in}{1.556649in}}%
\pgfpathlineto{\pgfqpoint{4.470620in}{1.562055in}}%
\pgfpathlineto{\pgfqpoint{4.480419in}{1.567483in}}%
\pgfpathlineto{\pgfqpoint{4.490218in}{1.572933in}}%
\pgfpathlineto{\pgfqpoint{4.500017in}{1.578406in}}%
\pgfpathlineto{\pgfqpoint{4.509816in}{1.583902in}}%
\pgfpathlineto{\pgfqpoint{4.519615in}{1.589419in}}%
\pgfpathlineto{\pgfqpoint{4.529414in}{1.594959in}}%
\pgfpathlineto{\pgfqpoint{4.539213in}{1.600522in}}%
\pgfpathlineto{\pgfqpoint{4.549012in}{1.606106in}}%
\pgfpathlineto{\pgfqpoint{4.558811in}{1.611713in}}%
\pgfpathlineto{\pgfqpoint{4.568610in}{1.617343in}}%
\pgfpathlineto{\pgfqpoint{4.578409in}{1.622995in}}%
\pgfpathlineto{\pgfqpoint{4.588208in}{1.628669in}}%
\pgfpathlineto{\pgfqpoint{4.598007in}{1.634365in}}%
\pgfpathlineto{\pgfqpoint{4.607806in}{1.640084in}}%
\pgfpathlineto{\pgfqpoint{4.617605in}{1.645825in}}%
\pgfpathlineto{\pgfqpoint{4.627404in}{1.651589in}}%
\pgfpathlineto{\pgfqpoint{4.637203in}{1.657375in}}%
\pgfpathlineto{\pgfqpoint{4.647002in}{1.663183in}}%
\pgfpathlineto{\pgfqpoint{4.656800in}{1.669014in}}%
\pgfpathlineto{\pgfqpoint{4.666599in}{1.674867in}}%
\pgfpathlineto{\pgfqpoint{4.676398in}{1.680742in}}%
\pgfpathlineto{\pgfqpoint{4.686197in}{1.686640in}}%
\pgfpathlineto{\pgfqpoint{4.695996in}{1.692560in}}%
\pgfpathlineto{\pgfqpoint{4.705795in}{1.698503in}}%
\pgfpathlineto{\pgfqpoint{4.715594in}{1.704467in}}%
\pgfpathlineto{\pgfqpoint{4.725393in}{1.710454in}}%
\pgfpathlineto{\pgfqpoint{4.735192in}{1.716464in}}%
\pgfpathlineto{\pgfqpoint{4.744991in}{1.722496in}}%
\pgfpathlineto{\pgfqpoint{4.754790in}{1.728550in}}%
\pgfpathlineto{\pgfqpoint{4.764589in}{1.734627in}}%
\pgfpathlineto{\pgfqpoint{4.774388in}{1.740726in}}%
\pgfpathlineto{\pgfqpoint{4.784187in}{1.746847in}}%
\pgfpathlineto{\pgfqpoint{4.793986in}{1.752991in}}%
\pgfpathlineto{\pgfqpoint{4.803785in}{1.759157in}}%
\pgfpathlineto{\pgfqpoint{4.813584in}{1.765345in}}%
\pgfpathlineto{\pgfqpoint{4.823383in}{1.771556in}}%
\pgfpathlineto{\pgfqpoint{4.833182in}{1.777789in}}%
\pgfpathlineto{\pgfqpoint{4.842981in}{1.784044in}}%
\pgfpathlineto{\pgfqpoint{4.852780in}{1.790322in}}%
\pgfpathlineto{\pgfqpoint{4.862579in}{1.796622in}}%
\pgfpathlineto{\pgfqpoint{4.872378in}{1.802945in}}%
\pgfpathlineto{\pgfqpoint{4.882177in}{1.809289in}}%
\pgfpathlineto{\pgfqpoint{4.891976in}{1.815657in}}%
\pgfpathlineto{\pgfqpoint{4.901775in}{1.822046in}}%
\pgfpathlineto{\pgfqpoint{4.911574in}{1.828458in}}%
\pgfpathlineto{\pgfqpoint{4.921373in}{1.834892in}}%
\pgfpathlineto{\pgfqpoint{4.931172in}{1.841349in}}%
\pgfpathlineto{\pgfqpoint{4.940971in}{1.847828in}}%
\pgfpathlineto{\pgfqpoint{4.950770in}{1.854329in}}%
\pgfpathlineto{\pgfqpoint{4.960569in}{1.860853in}}%
\pgfpathlineto{\pgfqpoint{4.970368in}{1.867399in}}%
\pgfpathlineto{\pgfqpoint{4.980167in}{1.873968in}}%
\pgfpathlineto{\pgfqpoint{4.989966in}{1.880559in}}%
\pgfpathlineto{\pgfqpoint{4.999765in}{1.887172in}}%
\pgfpathlineto{\pgfqpoint{5.009564in}{1.893807in}}%
\pgfpathlineto{\pgfqpoint{5.019362in}{1.900465in}}%
\pgfpathlineto{\pgfqpoint{5.029161in}{1.907145in}}%
\pgfpathlineto{\pgfqpoint{5.038960in}{1.913848in}}%
\pgfpathlineto{\pgfqpoint{5.048759in}{1.920573in}}%
\pgfpathlineto{\pgfqpoint{5.058558in}{1.927320in}}%
\pgfpathlineto{\pgfqpoint{5.068357in}{1.934090in}}%
\pgfpathlineto{\pgfqpoint{5.078156in}{1.940882in}}%
\pgfpathlineto{\pgfqpoint{5.087955in}{1.947696in}}%
\pgfpathlineto{\pgfqpoint{5.097754in}{1.954533in}}%
\pgfpathlineto{\pgfqpoint{5.107553in}{1.961392in}}%
\pgfpathlineto{\pgfqpoint{5.117352in}{1.968273in}}%
\pgfpathlineto{\pgfqpoint{5.127151in}{1.975177in}}%
\pgfpathlineto{\pgfqpoint{5.136950in}{1.982103in}}%
\pgfpathlineto{\pgfqpoint{5.146749in}{1.989052in}}%
\pgfpathlineto{\pgfqpoint{5.156548in}{1.996023in}}%
\pgfpathlineto{\pgfqpoint{5.166347in}{2.003016in}}%
\pgfpathlineto{\pgfqpoint{5.176146in}{2.010031in}}%
\pgfpathlineto{\pgfqpoint{5.185945in}{2.017069in}}%
\pgfpathlineto{\pgfqpoint{5.195744in}{2.024130in}}%
\pgfpathlineto{\pgfqpoint{5.205543in}{2.031212in}}%
\pgfpathlineto{\pgfqpoint{5.215342in}{2.038317in}}%
\pgfpathlineto{\pgfqpoint{5.225141in}{2.045445in}}%
\pgfpathlineto{\pgfqpoint{5.234940in}{2.052594in}}%
\pgfpathlineto{\pgfqpoint{5.244739in}{2.059767in}}%
\pgfpathlineto{\pgfqpoint{5.254538in}{2.066961in}}%
\pgfpathlineto{\pgfqpoint{5.264337in}{2.074178in}}%
\pgfpathlineto{\pgfqpoint{5.274136in}{2.081417in}}%
\pgfpathlineto{\pgfqpoint{5.283935in}{2.088678in}}%
\pgfpathlineto{\pgfqpoint{5.293734in}{2.095962in}}%
\pgfpathlineto{\pgfqpoint{5.303533in}{2.103268in}}%
\pgfpathlineto{\pgfqpoint{5.313332in}{2.110597in}}%
\pgfpathlineto{\pgfqpoint{5.323131in}{2.117948in}}%
\pgfpathlineto{\pgfqpoint{5.332930in}{2.125321in}}%
\pgfpathlineto{\pgfqpoint{5.342729in}{2.132717in}}%
\pgfpathlineto{\pgfqpoint{5.352528in}{2.140135in}}%
\pgfpathlineto{\pgfqpoint{5.362327in}{2.147575in}}%
\pgfpathlineto{\pgfqpoint{5.372126in}{2.155038in}}%
\pgfpathlineto{\pgfqpoint{5.381924in}{2.162523in}}%
\pgfpathlineto{\pgfqpoint{5.391723in}{2.170031in}}%
\pgfpathlineto{\pgfqpoint{5.401522in}{2.177560in}}%
\pgfpathlineto{\pgfqpoint{5.411321in}{2.185112in}}%
\pgfpathlineto{\pgfqpoint{5.421120in}{2.192687in}}%
\pgfpathlineto{\pgfqpoint{5.430919in}{2.200284in}}%
\pgfpathlineto{\pgfqpoint{5.440718in}{2.207903in}}%
\pgfpathlineto{\pgfqpoint{5.450517in}{2.215545in}}%
\pgfpathlineto{\pgfqpoint{5.460316in}{2.223209in}}%
\pgfpathlineto{\pgfqpoint{5.470115in}{2.230895in}}%
\pgfpathlineto{\pgfqpoint{5.479914in}{2.238604in}}%
\pgfpathlineto{\pgfqpoint{5.489713in}{2.246335in}}%
\pgfpathlineto{\pgfqpoint{5.499512in}{2.254088in}}%
\pgfpathlineto{\pgfqpoint{5.509311in}{2.261864in}}%
\pgfpathlineto{\pgfqpoint{5.519110in}{2.269662in}}%
\pgfpathlineto{\pgfqpoint{5.528909in}{2.277482in}}%
\pgfpathlineto{\pgfqpoint{5.538708in}{2.285325in}}%
\pgfpathlineto{\pgfqpoint{5.548507in}{2.293190in}}%
\pgfpathlineto{\pgfqpoint{5.558306in}{2.301078in}}%
\pgfpathlineto{\pgfqpoint{5.568105in}{2.308987in}}%
\pgfpathlineto{\pgfqpoint{5.577904in}{2.316920in}}%
\pgfpathlineto{\pgfqpoint{5.587703in}{2.324874in}}%
\pgfpathlineto{\pgfqpoint{5.597502in}{2.332851in}}%
\pgfpathlineto{\pgfqpoint{5.607301in}{2.340851in}}%
\pgfpathlineto{\pgfqpoint{5.617100in}{2.348872in}}%
\pgfpathlineto{\pgfqpoint{5.626899in}{2.356916in}}%
\pgfpathlineto{\pgfqpoint{5.636698in}{2.364983in}}%
\pgfpathlineto{\pgfqpoint{5.646497in}{2.373071in}}%
\pgfpathlineto{\pgfqpoint{5.656296in}{2.381182in}}%
\pgfpathlineto{\pgfqpoint{5.666095in}{2.389316in}}%
\pgfpathlineto{\pgfqpoint{5.675894in}{2.397472in}}%
\pgfpathlineto{\pgfqpoint{5.685693in}{2.405650in}}%
\pgfpathlineto{\pgfqpoint{5.695492in}{2.413850in}}%
\pgfpathlineto{\pgfqpoint{5.705291in}{2.422073in}}%
\pgfpathlineto{\pgfqpoint{5.715090in}{2.430318in}}%
\pgfpathlineto{\pgfqpoint{5.724889in}{2.438586in}}%
\pgfpathlineto{\pgfqpoint{5.734687in}{2.446876in}}%
\pgfpathlineto{\pgfqpoint{5.734687in}{3.445673in}}%
\pgfpathlineto{\pgfqpoint{5.734687in}{3.445673in}}%
\pgfpathlineto{\pgfqpoint{5.724889in}{3.431691in}}%
\pgfpathlineto{\pgfqpoint{5.715090in}{3.417750in}}%
\pgfpathlineto{\pgfqpoint{5.705291in}{3.403850in}}%
\pgfpathlineto{\pgfqpoint{5.695492in}{3.389989in}}%
\pgfpathlineto{\pgfqpoint{5.685693in}{3.376169in}}%
\pgfpathlineto{\pgfqpoint{5.675894in}{3.362389in}}%
\pgfpathlineto{\pgfqpoint{5.666095in}{3.348649in}}%
\pgfpathlineto{\pgfqpoint{5.656296in}{3.334949in}}%
\pgfpathlineto{\pgfqpoint{5.646497in}{3.321290in}}%
\pgfpathlineto{\pgfqpoint{5.636698in}{3.307671in}}%
\pgfpathlineto{\pgfqpoint{5.626899in}{3.294092in}}%
\pgfpathlineto{\pgfqpoint{5.617100in}{3.280553in}}%
\pgfpathlineto{\pgfqpoint{5.607301in}{3.267055in}}%
\pgfpathlineto{\pgfqpoint{5.597502in}{3.253597in}}%
\pgfpathlineto{\pgfqpoint{5.587703in}{3.240179in}}%
\pgfpathlineto{\pgfqpoint{5.577904in}{3.226801in}}%
\pgfpathlineto{\pgfqpoint{5.568105in}{3.213463in}}%
\pgfpathlineto{\pgfqpoint{5.558306in}{3.200166in}}%
\pgfpathlineto{\pgfqpoint{5.548507in}{3.186909in}}%
\pgfpathlineto{\pgfqpoint{5.538708in}{3.173692in}}%
\pgfpathlineto{\pgfqpoint{5.528909in}{3.160516in}}%
\pgfpathlineto{\pgfqpoint{5.519110in}{3.147379in}}%
\pgfpathlineto{\pgfqpoint{5.509311in}{3.134283in}}%
\pgfpathlineto{\pgfqpoint{5.499512in}{3.121227in}}%
\pgfpathlineto{\pgfqpoint{5.489713in}{3.108212in}}%
\pgfpathlineto{\pgfqpoint{5.479914in}{3.095236in}}%
\pgfpathlineto{\pgfqpoint{5.470115in}{3.082301in}}%
\pgfpathlineto{\pgfqpoint{5.460316in}{3.069406in}}%
\pgfpathlineto{\pgfqpoint{5.450517in}{3.056552in}}%
\pgfpathlineto{\pgfqpoint{5.440718in}{3.043737in}}%
\pgfpathlineto{\pgfqpoint{5.430919in}{3.030963in}}%
\pgfpathlineto{\pgfqpoint{5.421120in}{3.018229in}}%
\pgfpathlineto{\pgfqpoint{5.411321in}{3.005535in}}%
\pgfpathlineto{\pgfqpoint{5.401522in}{2.992881in}}%
\pgfpathlineto{\pgfqpoint{5.391723in}{2.980268in}}%
\pgfpathlineto{\pgfqpoint{5.381924in}{2.967695in}}%
\pgfpathlineto{\pgfqpoint{5.372126in}{2.955162in}}%
\pgfpathlineto{\pgfqpoint{5.362327in}{2.942669in}}%
\pgfpathlineto{\pgfqpoint{5.352528in}{2.930217in}}%
\pgfpathlineto{\pgfqpoint{5.342729in}{2.917805in}}%
\pgfpathlineto{\pgfqpoint{5.332930in}{2.905433in}}%
\pgfpathlineto{\pgfqpoint{5.323131in}{2.893101in}}%
\pgfpathlineto{\pgfqpoint{5.313332in}{2.880810in}}%
\pgfpathlineto{\pgfqpoint{5.303533in}{2.868558in}}%
\pgfpathlineto{\pgfqpoint{5.293734in}{2.856347in}}%
\pgfpathlineto{\pgfqpoint{5.283935in}{2.844177in}}%
\pgfpathlineto{\pgfqpoint{5.274136in}{2.832046in}}%
\pgfpathlineto{\pgfqpoint{5.264337in}{2.819956in}}%
\pgfpathlineto{\pgfqpoint{5.254538in}{2.807906in}}%
\pgfpathlineto{\pgfqpoint{5.244739in}{2.795896in}}%
\pgfpathlineto{\pgfqpoint{5.234940in}{2.783926in}}%
\pgfpathlineto{\pgfqpoint{5.225141in}{2.771997in}}%
\pgfpathlineto{\pgfqpoint{5.215342in}{2.760108in}}%
\pgfpathlineto{\pgfqpoint{5.205543in}{2.748259in}}%
\pgfpathlineto{\pgfqpoint{5.195744in}{2.736450in}}%
\pgfpathlineto{\pgfqpoint{5.185945in}{2.724682in}}%
\pgfpathlineto{\pgfqpoint{5.176146in}{2.712953in}}%
\pgfpathlineto{\pgfqpoint{5.166347in}{2.701265in}}%
\pgfpathlineto{\pgfqpoint{5.156548in}{2.689618in}}%
\pgfpathlineto{\pgfqpoint{5.146749in}{2.678010in}}%
\pgfpathlineto{\pgfqpoint{5.136950in}{2.666443in}}%
\pgfpathlineto{\pgfqpoint{5.127151in}{2.654916in}}%
\pgfpathlineto{\pgfqpoint{5.117352in}{2.643429in}}%
\pgfpathlineto{\pgfqpoint{5.107553in}{2.631982in}}%
\pgfpathlineto{\pgfqpoint{5.097754in}{2.620576in}}%
\pgfpathlineto{\pgfqpoint{5.087955in}{2.609210in}}%
\pgfpathlineto{\pgfqpoint{5.078156in}{2.597884in}}%
\pgfpathlineto{\pgfqpoint{5.068357in}{2.586598in}}%
\pgfpathlineto{\pgfqpoint{5.058558in}{2.575353in}}%
\pgfpathlineto{\pgfqpoint{5.048759in}{2.564148in}}%
\pgfpathlineto{\pgfqpoint{5.038960in}{2.552983in}}%
\pgfpathlineto{\pgfqpoint{5.029161in}{2.541858in}}%
\pgfpathlineto{\pgfqpoint{5.019362in}{2.530773in}}%
\pgfpathlineto{\pgfqpoint{5.009564in}{2.519729in}}%
\pgfpathlineto{\pgfqpoint{4.999765in}{2.508725in}}%
\pgfpathlineto{\pgfqpoint{4.989966in}{2.497761in}}%
\pgfpathlineto{\pgfqpoint{4.980167in}{2.486838in}}%
\pgfpathlineto{\pgfqpoint{4.970368in}{2.475954in}}%
\pgfpathlineto{\pgfqpoint{4.960569in}{2.465111in}}%
\pgfpathlineto{\pgfqpoint{4.950770in}{2.454308in}}%
\pgfpathlineto{\pgfqpoint{4.940971in}{2.443546in}}%
\pgfpathlineto{\pgfqpoint{4.931172in}{2.432823in}}%
\pgfpathlineto{\pgfqpoint{4.921373in}{2.422141in}}%
\pgfpathlineto{\pgfqpoint{4.911574in}{2.411499in}}%
\pgfpathlineto{\pgfqpoint{4.901775in}{2.400897in}}%
\pgfpathlineto{\pgfqpoint{4.891976in}{2.390336in}}%
\pgfpathlineto{\pgfqpoint{4.882177in}{2.379814in}}%
\pgfpathlineto{\pgfqpoint{4.872378in}{2.369333in}}%
\pgfpathlineto{\pgfqpoint{4.862579in}{2.358893in}}%
\pgfpathlineto{\pgfqpoint{4.852780in}{2.348492in}}%
\pgfpathlineto{\pgfqpoint{4.842981in}{2.338132in}}%
\pgfpathlineto{\pgfqpoint{4.833182in}{2.327812in}}%
\pgfpathlineto{\pgfqpoint{4.823383in}{2.317532in}}%
\pgfpathlineto{\pgfqpoint{4.813584in}{2.307292in}}%
\pgfpathlineto{\pgfqpoint{4.803785in}{2.297093in}}%
\pgfpathlineto{\pgfqpoint{4.793986in}{2.286933in}}%
\pgfpathlineto{\pgfqpoint{4.784187in}{2.276814in}}%
\pgfpathlineto{\pgfqpoint{4.774388in}{2.266736in}}%
\pgfpathlineto{\pgfqpoint{4.764589in}{2.256697in}}%
\pgfpathlineto{\pgfqpoint{4.754790in}{2.246699in}}%
\pgfpathlineto{\pgfqpoint{4.744991in}{2.236741in}}%
\pgfpathlineto{\pgfqpoint{4.735192in}{2.226823in}}%
\pgfpathlineto{\pgfqpoint{4.725393in}{2.216946in}}%
\pgfpathlineto{\pgfqpoint{4.715594in}{2.207108in}}%
\pgfpathlineto{\pgfqpoint{4.705795in}{2.197311in}}%
\pgfpathlineto{\pgfqpoint{4.695996in}{2.187554in}}%
\pgfpathlineto{\pgfqpoint{4.686197in}{2.177838in}}%
\pgfpathlineto{\pgfqpoint{4.676398in}{2.168161in}}%
\pgfpathlineto{\pgfqpoint{4.666599in}{2.158525in}}%
\pgfpathlineto{\pgfqpoint{4.656800in}{2.148929in}}%
\pgfpathlineto{\pgfqpoint{4.647002in}{2.139374in}}%
\pgfpathlineto{\pgfqpoint{4.637203in}{2.129858in}}%
\pgfpathlineto{\pgfqpoint{4.627404in}{2.120383in}}%
\pgfpathlineto{\pgfqpoint{4.617605in}{2.110948in}}%
\pgfpathlineto{\pgfqpoint{4.607806in}{2.101553in}}%
\pgfpathlineto{\pgfqpoint{4.598007in}{2.092198in}}%
\pgfpathlineto{\pgfqpoint{4.588208in}{2.082884in}}%
\pgfpathlineto{\pgfqpoint{4.578409in}{2.073610in}}%
\pgfpathlineto{\pgfqpoint{4.568610in}{2.064376in}}%
\pgfpathlineto{\pgfqpoint{4.558811in}{2.055183in}}%
\pgfpathlineto{\pgfqpoint{4.549012in}{2.046029in}}%
\pgfpathlineto{\pgfqpoint{4.539213in}{2.036916in}}%
\pgfpathlineto{\pgfqpoint{4.529414in}{2.027843in}}%
\pgfpathlineto{\pgfqpoint{4.519615in}{2.018810in}}%
\pgfpathlineto{\pgfqpoint{4.509816in}{2.009818in}}%
\pgfpathlineto{\pgfqpoint{4.500017in}{2.000866in}}%
\pgfpathlineto{\pgfqpoint{4.490218in}{1.991954in}}%
\pgfpathlineto{\pgfqpoint{4.480419in}{1.983082in}}%
\pgfpathlineto{\pgfqpoint{4.470620in}{1.974251in}}%
\pgfpathlineto{\pgfqpoint{4.460821in}{1.965459in}}%
\pgfpathlineto{\pgfqpoint{4.451022in}{1.956708in}}%
\pgfpathlineto{\pgfqpoint{4.441223in}{1.947997in}}%
\pgfpathlineto{\pgfqpoint{4.431424in}{1.939327in}}%
\pgfpathlineto{\pgfqpoint{4.421625in}{1.930696in}}%
\pgfpathlineto{\pgfqpoint{4.411826in}{1.922106in}}%
\pgfpathlineto{\pgfqpoint{4.402027in}{1.913556in}}%
\pgfpathlineto{\pgfqpoint{4.392228in}{1.905047in}}%
\pgfpathlineto{\pgfqpoint{4.382429in}{1.896577in}}%
\pgfpathlineto{\pgfqpoint{4.372630in}{1.888148in}}%
\pgfpathlineto{\pgfqpoint{4.362831in}{1.879759in}}%
\pgfpathlineto{\pgfqpoint{4.353032in}{1.871410in}}%
\pgfpathlineto{\pgfqpoint{4.343233in}{1.863102in}}%
\pgfpathlineto{\pgfqpoint{4.333434in}{1.854833in}}%
\pgfpathlineto{\pgfqpoint{4.323635in}{1.846605in}}%
\pgfpathlineto{\pgfqpoint{4.313836in}{1.838418in}}%
\pgfpathlineto{\pgfqpoint{4.304037in}{1.830270in}}%
\pgfpathlineto{\pgfqpoint{4.294238in}{1.822163in}}%
\pgfpathlineto{\pgfqpoint{4.284440in}{1.814096in}}%
\pgfpathlineto{\pgfqpoint{4.274641in}{1.806069in}}%
\pgfpathlineto{\pgfqpoint{4.264842in}{1.798082in}}%
\pgfpathlineto{\pgfqpoint{4.255043in}{1.790136in}}%
\pgfpathlineto{\pgfqpoint{4.245244in}{1.782229in}}%
\pgfpathlineto{\pgfqpoint{4.235445in}{1.774363in}}%
\pgfpathlineto{\pgfqpoint{4.225646in}{1.766538in}}%
\pgfpathlineto{\pgfqpoint{4.215847in}{1.758752in}}%
\pgfpathlineto{\pgfqpoint{4.206048in}{1.751007in}}%
\pgfpathlineto{\pgfqpoint{4.196249in}{1.743302in}}%
\pgfpathlineto{\pgfqpoint{4.186450in}{1.735637in}}%
\pgfpathlineto{\pgfqpoint{4.176651in}{1.728013in}}%
\pgfpathlineto{\pgfqpoint{4.166852in}{1.720428in}}%
\pgfpathlineto{\pgfqpoint{4.157053in}{1.712884in}}%
\pgfpathlineto{\pgfqpoint{4.147254in}{1.705380in}}%
\pgfpathlineto{\pgfqpoint{4.137455in}{1.697917in}}%
\pgfpathlineto{\pgfqpoint{4.127656in}{1.690493in}}%
\pgfpathlineto{\pgfqpoint{4.117857in}{1.683110in}}%
\pgfpathlineto{\pgfqpoint{4.108058in}{1.675767in}}%
\pgfpathlineto{\pgfqpoint{4.098259in}{1.668464in}}%
\pgfpathlineto{\pgfqpoint{4.088460in}{1.661202in}}%
\pgfpathlineto{\pgfqpoint{4.078661in}{1.653980in}}%
\pgfpathlineto{\pgfqpoint{4.068862in}{1.646798in}}%
\pgfpathlineto{\pgfqpoint{4.059063in}{1.639656in}}%
\pgfpathlineto{\pgfqpoint{4.049264in}{1.632554in}}%
\pgfpathlineto{\pgfqpoint{4.039465in}{1.625493in}}%
\pgfpathlineto{\pgfqpoint{4.029666in}{1.618472in}}%
\pgfpathlineto{\pgfqpoint{4.019867in}{1.611491in}}%
\pgfpathlineto{\pgfqpoint{4.010068in}{1.604550in}}%
\pgfpathlineto{\pgfqpoint{4.000269in}{1.597650in}}%
\pgfpathlineto{\pgfqpoint{3.990470in}{1.590790in}}%
\pgfpathlineto{\pgfqpoint{3.980671in}{1.583970in}}%
\pgfpathlineto{\pgfqpoint{3.970872in}{1.577190in}}%
\pgfpathlineto{\pgfqpoint{3.961073in}{1.570451in}}%
\pgfpathlineto{\pgfqpoint{3.951274in}{1.563751in}}%
\pgfpathlineto{\pgfqpoint{3.941475in}{1.557092in}}%
\pgfpathlineto{\pgfqpoint{3.931676in}{1.550474in}}%
\pgfpathlineto{\pgfqpoint{3.921878in}{1.543895in}}%
\pgfpathlineto{\pgfqpoint{3.912079in}{1.537357in}}%
\pgfpathlineto{\pgfqpoint{3.902280in}{1.530859in}}%
\pgfpathlineto{\pgfqpoint{3.892481in}{1.524401in}}%
\pgfpathlineto{\pgfqpoint{3.882682in}{1.517983in}}%
\pgfpathlineto{\pgfqpoint{3.872883in}{1.511606in}}%
\pgfpathlineto{\pgfqpoint{3.863084in}{1.505269in}}%
\pgfpathlineto{\pgfqpoint{3.853285in}{1.498972in}}%
\pgfpathlineto{\pgfqpoint{3.843486in}{1.492715in}}%
\pgfpathlineto{\pgfqpoint{3.833687in}{1.486499in}}%
\pgfpathlineto{\pgfqpoint{3.823888in}{1.480323in}}%
\pgfpathlineto{\pgfqpoint{3.814089in}{1.474187in}}%
\pgfpathlineto{\pgfqpoint{3.804290in}{1.468091in}}%
\pgfpathlineto{\pgfqpoint{3.794491in}{1.462035in}}%
\pgfpathlineto{\pgfqpoint{3.784692in}{1.456020in}}%
\pgfpathlineto{\pgfqpoint{3.774893in}{1.450045in}}%
\pgfpathlineto{\pgfqpoint{3.765094in}{1.444110in}}%
\pgfpathlineto{\pgfqpoint{3.755295in}{1.438216in}}%
\pgfpathlineto{\pgfqpoint{3.745496in}{1.432361in}}%
\pgfpathlineto{\pgfqpoint{3.735697in}{1.426547in}}%
\pgfpathlineto{\pgfqpoint{3.725898in}{1.420773in}}%
\pgfpathlineto{\pgfqpoint{3.716099in}{1.415039in}}%
\pgfpathlineto{\pgfqpoint{3.706300in}{1.409346in}}%
\pgfpathlineto{\pgfqpoint{3.696501in}{1.403693in}}%
\pgfpathlineto{\pgfqpoint{3.686702in}{1.398080in}}%
\pgfpathlineto{\pgfqpoint{3.676903in}{1.392507in}}%
\pgfpathlineto{\pgfqpoint{3.667104in}{1.386975in}}%
\pgfpathlineto{\pgfqpoint{3.657305in}{1.381482in}}%
\pgfpathlineto{\pgfqpoint{3.647506in}{1.376030in}}%
\pgfpathlineto{\pgfqpoint{3.637707in}{1.370618in}}%
\pgfpathlineto{\pgfqpoint{3.627908in}{1.365247in}}%
\pgfpathlineto{\pgfqpoint{3.618109in}{1.359916in}}%
\pgfpathlineto{\pgfqpoint{3.608310in}{1.354624in}}%
\pgfpathlineto{\pgfqpoint{3.598511in}{1.349374in}}%
\pgfpathlineto{\pgfqpoint{3.588712in}{1.344163in}}%
\pgfpathlineto{\pgfqpoint{3.578913in}{1.338993in}}%
\pgfpathlineto{\pgfqpoint{3.569114in}{1.333862in}}%
\pgfpathlineto{\pgfqpoint{3.559316in}{1.328772in}}%
\pgfpathlineto{\pgfqpoint{3.549517in}{1.323723in}}%
\pgfpathlineto{\pgfqpoint{3.539718in}{1.318713in}}%
\pgfpathlineto{\pgfqpoint{3.529919in}{1.313744in}}%
\pgfpathlineto{\pgfqpoint{3.520120in}{1.308815in}}%
\pgfpathlineto{\pgfqpoint{3.510321in}{1.303926in}}%
\pgfpathlineto{\pgfqpoint{3.500522in}{1.299077in}}%
\pgfpathlineto{\pgfqpoint{3.490723in}{1.294269in}}%
\pgfpathlineto{\pgfqpoint{3.480924in}{1.289501in}}%
\pgfpathlineto{\pgfqpoint{3.471125in}{1.284773in}}%
\pgfpathlineto{\pgfqpoint{3.461326in}{1.280086in}}%
\pgfpathlineto{\pgfqpoint{3.451527in}{1.275438in}}%
\pgfpathlineto{\pgfqpoint{3.441728in}{1.270831in}}%
\pgfpathlineto{\pgfqpoint{3.431929in}{1.266264in}}%
\pgfpathlineto{\pgfqpoint{3.422130in}{1.261737in}}%
\pgfpathlineto{\pgfqpoint{3.412331in}{1.257251in}}%
\pgfpathlineto{\pgfqpoint{3.402532in}{1.252805in}}%
\pgfpathlineto{\pgfqpoint{3.392733in}{1.248399in}}%
\pgfpathlineto{\pgfqpoint{3.382934in}{1.244033in}}%
\pgfpathlineto{\pgfqpoint{3.373135in}{1.239707in}}%
\pgfpathlineto{\pgfqpoint{3.363336in}{1.235422in}}%
\pgfpathlineto{\pgfqpoint{3.353537in}{1.231177in}}%
\pgfpathlineto{\pgfqpoint{3.343738in}{1.226972in}}%
\pgfpathlineto{\pgfqpoint{3.333939in}{1.222807in}}%
\pgfpathlineto{\pgfqpoint{3.324140in}{1.218683in}}%
\pgfpathlineto{\pgfqpoint{3.314341in}{1.214599in}}%
\pgfpathlineto{\pgfqpoint{3.304542in}{1.210555in}}%
\pgfpathlineto{\pgfqpoint{3.294743in}{1.206551in}}%
\pgfpathlineto{\pgfqpoint{3.284944in}{1.202588in}}%
\pgfpathlineto{\pgfqpoint{3.275145in}{1.198665in}}%
\pgfpathlineto{\pgfqpoint{3.265346in}{1.194782in}}%
\pgfpathlineto{\pgfqpoint{3.255547in}{1.190939in}}%
\pgfpathlineto{\pgfqpoint{3.245748in}{1.187136in}}%
\pgfpathlineto{\pgfqpoint{3.235949in}{1.183374in}}%
\pgfpathlineto{\pgfqpoint{3.226150in}{1.179652in}}%
\pgfpathlineto{\pgfqpoint{3.216351in}{1.175970in}}%
\pgfpathlineto{\pgfqpoint{3.206552in}{1.172328in}}%
\pgfpathlineto{\pgfqpoint{3.196754in}{1.168727in}}%
\pgfpathlineto{\pgfqpoint{3.186955in}{1.165166in}}%
\pgfpathlineto{\pgfqpoint{3.177156in}{1.161645in}}%
\pgfpathlineto{\pgfqpoint{3.167357in}{1.158164in}}%
\pgfpathlineto{\pgfqpoint{3.157558in}{1.154724in}}%
\pgfpathlineto{\pgfqpoint{3.147759in}{1.151324in}}%
\pgfpathlineto{\pgfqpoint{3.137960in}{1.147964in}}%
\pgfpathlineto{\pgfqpoint{3.128161in}{1.144644in}}%
\pgfpathlineto{\pgfqpoint{3.118362in}{1.141364in}}%
\pgfpathlineto{\pgfqpoint{3.108563in}{1.138125in}}%
\pgfpathlineto{\pgfqpoint{3.098764in}{1.134926in}}%
\pgfpathlineto{\pgfqpoint{3.088965in}{1.131767in}}%
\pgfpathlineto{\pgfqpoint{3.079166in}{1.128649in}}%
\pgfpathlineto{\pgfqpoint{3.069367in}{1.125570in}}%
\pgfpathlineto{\pgfqpoint{3.059568in}{1.122532in}}%
\pgfpathlineto{\pgfqpoint{3.049769in}{1.119534in}}%
\pgfpathlineto{\pgfqpoint{3.039970in}{1.116577in}}%
\pgfpathlineto{\pgfqpoint{3.030171in}{1.113659in}}%
\pgfpathlineto{\pgfqpoint{3.020372in}{1.110782in}}%
\pgfpathlineto{\pgfqpoint{3.010573in}{1.107945in}}%
\pgfpathlineto{\pgfqpoint{3.000774in}{1.105148in}}%
\pgfpathlineto{\pgfqpoint{2.990975in}{1.102392in}}%
\pgfpathlineto{\pgfqpoint{2.981176in}{1.099676in}}%
\pgfpathlineto{\pgfqpoint{2.971377in}{1.097000in}}%
\pgfpathlineto{\pgfqpoint{2.961578in}{1.094364in}}%
\pgfpathlineto{\pgfqpoint{2.951779in}{1.091768in}}%
\pgfpathlineto{\pgfqpoint{2.941980in}{1.089213in}}%
\pgfpathlineto{\pgfqpoint{2.932181in}{1.086698in}}%
\pgfpathlineto{\pgfqpoint{2.922382in}{1.084223in}}%
\pgfpathlineto{\pgfqpoint{2.912583in}{1.081788in}}%
\pgfpathlineto{\pgfqpoint{2.902784in}{1.079394in}}%
\pgfpathlineto{\pgfqpoint{2.892985in}{1.077040in}}%
\pgfpathlineto{\pgfqpoint{2.883186in}{1.074726in}}%
\pgfpathlineto{\pgfqpoint{2.873387in}{1.072452in}}%
\pgfpathlineto{\pgfqpoint{2.863588in}{1.070218in}}%
\pgfpathlineto{\pgfqpoint{2.853789in}{1.068025in}}%
\pgfpathlineto{\pgfqpoint{2.843990in}{1.065872in}}%
\pgfpathlineto{\pgfqpoint{2.834192in}{1.063759in}}%
\pgfpathlineto{\pgfqpoint{2.824393in}{1.061687in}}%
\pgfpathlineto{\pgfqpoint{2.814594in}{1.059654in}}%
\pgfpathlineto{\pgfqpoint{2.804795in}{1.057662in}}%
\pgfpathlineto{\pgfqpoint{2.794996in}{1.055711in}}%
\pgfpathlineto{\pgfqpoint{2.785197in}{1.053799in}}%
\pgfpathlineto{\pgfqpoint{2.775398in}{1.051928in}}%
\pgfpathlineto{\pgfqpoint{2.765599in}{1.050096in}}%
\pgfpathlineto{\pgfqpoint{2.755800in}{1.048305in}}%
\pgfpathlineto{\pgfqpoint{2.746001in}{1.046555in}}%
\pgfpathlineto{\pgfqpoint{2.736202in}{1.044844in}}%
\pgfpathlineto{\pgfqpoint{2.726403in}{1.043174in}}%
\pgfpathlineto{\pgfqpoint{2.716604in}{1.041544in}}%
\pgfpathlineto{\pgfqpoint{2.706805in}{1.039954in}}%
\pgfpathlineto{\pgfqpoint{2.697006in}{1.038405in}}%
\pgfpathlineto{\pgfqpoint{2.687207in}{1.036895in}}%
\pgfpathlineto{\pgfqpoint{2.677408in}{1.035426in}}%
\pgfpathlineto{\pgfqpoint{2.667609in}{1.033997in}}%
\pgfpathlineto{\pgfqpoint{2.657810in}{1.032609in}}%
\pgfpathlineto{\pgfqpoint{2.648011in}{1.031260in}}%
\pgfpathlineto{\pgfqpoint{2.638212in}{1.029952in}}%
\pgfpathlineto{\pgfqpoint{2.628413in}{1.028684in}}%
\pgfpathlineto{\pgfqpoint{2.618614in}{1.027457in}}%
\pgfpathlineto{\pgfqpoint{2.608815in}{1.026269in}}%
\pgfpathlineto{\pgfqpoint{2.599016in}{1.025122in}}%
\pgfpathlineto{\pgfqpoint{2.589217in}{1.024015in}}%
\pgfpathlineto{\pgfqpoint{2.579418in}{1.022948in}}%
\pgfpathlineto{\pgfqpoint{2.569619in}{1.021922in}}%
\pgfpathlineto{\pgfqpoint{2.559820in}{1.020935in}}%
\pgfpathlineto{\pgfqpoint{2.550021in}{1.019989in}}%
\pgfpathlineto{\pgfqpoint{2.540222in}{1.019084in}}%
\pgfpathlineto{\pgfqpoint{2.530423in}{1.018218in}}%
\pgfpathlineto{\pgfqpoint{2.520624in}{1.017393in}}%
\pgfpathlineto{\pgfqpoint{2.510825in}{1.016607in}}%
\pgfpathlineto{\pgfqpoint{2.501026in}{1.015863in}}%
\pgfpathlineto{\pgfqpoint{2.491227in}{1.015158in}}%
\pgfpathlineto{\pgfqpoint{2.481428in}{1.014493in}}%
\pgfpathlineto{\pgfqpoint{2.471630in}{1.013869in}}%
\pgfpathlineto{\pgfqpoint{2.461831in}{1.013285in}}%
\pgfpathlineto{\pgfqpoint{2.452032in}{1.012741in}}%
\pgfpathlineto{\pgfqpoint{2.442233in}{1.012238in}}%
\pgfpathlineto{\pgfqpoint{2.432434in}{1.011775in}}%
\pgfpathlineto{\pgfqpoint{2.422635in}{1.011352in}}%
\pgfpathlineto{\pgfqpoint{2.412836in}{1.010969in}}%
\pgfpathlineto{\pgfqpoint{2.403037in}{1.010626in}}%
\pgfpathlineto{\pgfqpoint{2.393238in}{1.010324in}}%
\pgfpathlineto{\pgfqpoint{2.383439in}{1.010062in}}%
\pgfpathlineto{\pgfqpoint{2.373640in}{1.009840in}}%
\pgfpathlineto{\pgfqpoint{2.363841in}{1.009658in}}%
\pgfpathlineto{\pgfqpoint{2.354042in}{1.009517in}}%
\pgfpathlineto{\pgfqpoint{2.344243in}{1.009416in}}%
\pgfpathlineto{\pgfqpoint{2.334444in}{1.009355in}}%
\pgfpathlineto{\pgfqpoint{2.324645in}{1.009334in}}%
\pgfpathlineto{\pgfqpoint{2.314846in}{1.009353in}}%
\pgfpathlineto{\pgfqpoint{2.305047in}{1.009413in}}%
\pgfpathlineto{\pgfqpoint{2.295248in}{1.009513in}}%
\pgfpathlineto{\pgfqpoint{2.285449in}{1.009653in}}%
\pgfpathlineto{\pgfqpoint{2.275650in}{1.009834in}}%
\pgfpathlineto{\pgfqpoint{2.265851in}{1.010054in}}%
\pgfpathlineto{\pgfqpoint{2.256052in}{1.010315in}}%
\pgfpathlineto{\pgfqpoint{2.246253in}{1.010616in}}%
\pgfpathlineto{\pgfqpoint{2.236454in}{1.010958in}}%
\pgfpathlineto{\pgfqpoint{2.226655in}{1.011339in}}%
\pgfpathlineto{\pgfqpoint{2.216856in}{1.011761in}}%
\pgfpathlineto{\pgfqpoint{2.207057in}{1.012223in}}%
\pgfpathlineto{\pgfqpoint{2.197258in}{1.012726in}}%
\pgfpathlineto{\pgfqpoint{2.187459in}{1.013268in}}%
\pgfpathlineto{\pgfqpoint{2.177660in}{1.013851in}}%
\pgfpathlineto{\pgfqpoint{2.167861in}{1.014474in}}%
\pgfpathlineto{\pgfqpoint{2.158062in}{1.015137in}}%
\pgfpathlineto{\pgfqpoint{2.148263in}{1.015841in}}%
\pgfpathlineto{\pgfqpoint{2.138464in}{1.016584in}}%
\pgfpathlineto{\pgfqpoint{2.128665in}{1.017368in}}%
\pgfpathlineto{\pgfqpoint{2.118866in}{1.018192in}}%
\pgfpathlineto{\pgfqpoint{2.109068in}{1.019057in}}%
\pgfpathlineto{\pgfqpoint{2.099269in}{1.019961in}}%
\pgfpathlineto{\pgfqpoint{2.089470in}{1.020906in}}%
\pgfpathlineto{\pgfqpoint{2.079671in}{1.021891in}}%
\pgfpathlineto{\pgfqpoint{2.069872in}{1.022916in}}%
\pgfpathlineto{\pgfqpoint{2.060073in}{1.023982in}}%
\pgfpathlineto{\pgfqpoint{2.050274in}{1.025088in}}%
\pgfpathlineto{\pgfqpoint{2.040475in}{1.026234in}}%
\pgfpathlineto{\pgfqpoint{2.030676in}{1.027420in}}%
\pgfpathlineto{\pgfqpoint{2.020877in}{1.028646in}}%
\pgfpathlineto{\pgfqpoint{2.011078in}{1.029913in}}%
\pgfpathlineto{\pgfqpoint{2.001279in}{1.031220in}}%
\pgfpathlineto{\pgfqpoint{1.991480in}{1.032567in}}%
\pgfpathlineto{\pgfqpoint{1.981681in}{1.033955in}}%
\pgfpathlineto{\pgfqpoint{1.971882in}{1.035382in}}%
\pgfpathlineto{\pgfqpoint{1.962083in}{1.036850in}}%
\pgfpathlineto{\pgfqpoint{1.952284in}{1.038358in}}%
\pgfpathlineto{\pgfqpoint{1.942485in}{1.039906in}}%
\pgfpathlineto{\pgfqpoint{1.932686in}{1.041495in}}%
\pgfpathlineto{\pgfqpoint{1.922887in}{1.043124in}}%
\pgfpathlineto{\pgfqpoint{1.913088in}{1.044793in}}%
\pgfpathlineto{\pgfqpoint{1.903289in}{1.046502in}}%
\pgfpathlineto{\pgfqpoint{1.893490in}{1.048252in}}%
\pgfpathlineto{\pgfqpoint{1.883691in}{1.050041in}}%
\pgfpathlineto{\pgfqpoint{1.873892in}{1.051871in}}%
\pgfpathlineto{\pgfqpoint{1.864093in}{1.053741in}}%
\pgfpathlineto{\pgfqpoint{1.854294in}{1.055652in}}%
\pgfpathlineto{\pgfqpoint{1.844495in}{1.057602in}}%
\pgfpathlineto{\pgfqpoint{1.834696in}{1.059593in}}%
\pgfpathlineto{\pgfqpoint{1.824897in}{1.061624in}}%
\pgfpathlineto{\pgfqpoint{1.815098in}{1.063696in}}%
\pgfpathlineto{\pgfqpoint{1.805299in}{1.065807in}}%
\pgfpathlineto{\pgfqpoint{1.795500in}{1.067959in}}%
\pgfpathlineto{\pgfqpoint{1.785701in}{1.070151in}}%
\pgfpathlineto{\pgfqpoint{1.775902in}{1.072383in}}%
\pgfpathlineto{\pgfqpoint{1.766103in}{1.074656in}}%
\pgfpathlineto{\pgfqpoint{1.756304in}{1.076969in}}%
\pgfpathlineto{\pgfqpoint{1.746506in}{1.079322in}}%
\pgfpathlineto{\pgfqpoint{1.736707in}{1.081715in}}%
\pgfpathlineto{\pgfqpoint{1.726908in}{1.084148in}}%
\pgfpathlineto{\pgfqpoint{1.717109in}{1.086622in}}%
\pgfpathlineto{\pgfqpoint{1.707310in}{1.089136in}}%
\pgfpathlineto{\pgfqpoint{1.697511in}{1.091690in}}%
\pgfpathlineto{\pgfqpoint{1.687712in}{1.094284in}}%
\pgfpathlineto{\pgfqpoint{1.677913in}{1.096919in}}%
\pgfpathlineto{\pgfqpoint{1.668114in}{1.099594in}}%
\pgfpathlineto{\pgfqpoint{1.658315in}{1.102309in}}%
\pgfpathlineto{\pgfqpoint{1.648516in}{1.105064in}}%
\pgfpathlineto{\pgfqpoint{1.638717in}{1.107859in}}%
\pgfpathlineto{\pgfqpoint{1.628918in}{1.110695in}}%
\pgfpathlineto{\pgfqpoint{1.619119in}{1.113571in}}%
\pgfpathlineto{\pgfqpoint{1.609320in}{1.116487in}}%
\pgfpathlineto{\pgfqpoint{1.599521in}{1.119444in}}%
\pgfpathlineto{\pgfqpoint{1.589722in}{1.122440in}}%
\pgfpathlineto{\pgfqpoint{1.579923in}{1.125477in}}%
\pgfpathlineto{\pgfqpoint{1.570124in}{1.128554in}}%
\pgfpathlineto{\pgfqpoint{1.560325in}{1.131672in}}%
\pgfpathlineto{\pgfqpoint{1.550526in}{1.134829in}}%
\pgfpathlineto{\pgfqpoint{1.540727in}{1.138027in}}%
\pgfpathlineto{\pgfqpoint{1.530928in}{1.141265in}}%
\pgfpathlineto{\pgfqpoint{1.521129in}{1.144544in}}%
\pgfpathlineto{\pgfqpoint{1.511330in}{1.147862in}}%
\pgfpathlineto{\pgfqpoint{1.501531in}{1.151221in}}%
\pgfpathlineto{\pgfqpoint{1.491732in}{1.154620in}}%
\pgfpathlineto{\pgfqpoint{1.481933in}{1.158059in}}%
\pgfpathlineto{\pgfqpoint{1.472134in}{1.161539in}}%
\pgfpathlineto{\pgfqpoint{1.462335in}{1.165058in}}%
\pgfpathlineto{\pgfqpoint{1.452536in}{1.168618in}}%
\pgfpathlineto{\pgfqpoint{1.442737in}{1.172218in}}%
\pgfpathlineto{\pgfqpoint{1.432938in}{1.175859in}}%
\pgfpathlineto{\pgfqpoint{1.423139in}{1.179539in}}%
\pgfpathlineto{\pgfqpoint{1.413340in}{1.183260in}}%
\pgfpathlineto{\pgfqpoint{1.403541in}{1.187021in}}%
\pgfpathlineto{\pgfqpoint{1.393742in}{1.190823in}}%
\pgfpathlineto{\pgfqpoint{1.383944in}{1.194664in}}%
\pgfpathlineto{\pgfqpoint{1.374145in}{1.198546in}}%
\pgfpathlineto{\pgfqpoint{1.364346in}{1.202468in}}%
\pgfpathlineto{\pgfqpoint{1.354547in}{1.206430in}}%
\pgfpathlineto{\pgfqpoint{1.344748in}{1.210433in}}%
\pgfpathlineto{\pgfqpoint{1.334949in}{1.214475in}}%
\pgfpathlineto{\pgfqpoint{1.325150in}{1.218558in}}%
\pgfpathlineto{\pgfqpoint{1.315351in}{1.222681in}}%
\pgfpathlineto{\pgfqpoint{1.305552in}{1.226845in}}%
\pgfpathlineto{\pgfqpoint{1.295753in}{1.231048in}}%
\pgfpathlineto{\pgfqpoint{1.285954in}{1.235292in}}%
\pgfpathlineto{\pgfqpoint{1.276155in}{1.239576in}}%
\pgfpathlineto{\pgfqpoint{1.266356in}{1.243901in}}%
\pgfpathlineto{\pgfqpoint{1.256557in}{1.248265in}}%
\pgfpathlineto{\pgfqpoint{1.246758in}{1.252670in}}%
\pgfpathlineto{\pgfqpoint{1.236959in}{1.257115in}}%
\pgfpathlineto{\pgfqpoint{1.227160in}{1.261600in}}%
\pgfpathlineto{\pgfqpoint{1.217361in}{1.266126in}}%
\pgfpathlineto{\pgfqpoint{1.207562in}{1.270691in}}%
\pgfpathlineto{\pgfqpoint{1.197763in}{1.275297in}}%
\pgfpathlineto{\pgfqpoint{1.187964in}{1.279944in}}%
\pgfpathlineto{\pgfqpoint{1.178165in}{1.284630in}}%
\pgfpathlineto{\pgfqpoint{1.168366in}{1.289357in}}%
\pgfpathlineto{\pgfqpoint{1.158567in}{1.294124in}}%
\pgfpathlineto{\pgfqpoint{1.148768in}{1.298931in}}%
\pgfpathlineto{\pgfqpoint{1.138969in}{1.303778in}}%
\pgfpathlineto{\pgfqpoint{1.129170in}{1.308666in}}%
\pgfpathlineto{\pgfqpoint{1.119371in}{1.313593in}}%
\pgfpathlineto{\pgfqpoint{1.109572in}{1.318561in}}%
\pgfpathlineto{\pgfqpoint{1.099773in}{1.323570in}}%
\pgfpathlineto{\pgfqpoint{1.089974in}{1.328618in}}%
\pgfpathlineto{\pgfqpoint{1.080175in}{1.333707in}}%
\pgfpathlineto{\pgfqpoint{1.070376in}{1.338836in}}%
\pgfpathlineto{\pgfqpoint{1.060577in}{1.344005in}}%
\pgfpathlineto{\pgfqpoint{1.050778in}{1.349214in}}%
\pgfpathlineto{\pgfqpoint{1.040979in}{1.354464in}}%
\pgfpathlineto{\pgfqpoint{1.031180in}{1.359754in}}%
\pgfpathlineto{\pgfqpoint{1.021382in}{1.365084in}}%
\pgfpathlineto{\pgfqpoint{1.011583in}{1.370455in}}%
\pgfpathlineto{\pgfqpoint{1.001784in}{1.375865in}}%
\pgfpathlineto{\pgfqpoint{0.991985in}{1.381316in}}%
\pgfpathlineto{\pgfqpoint{0.982186in}{1.386807in}}%
\pgfpathlineto{\pgfqpoint{0.972387in}{1.392338in}}%
\pgfpathlineto{\pgfqpoint{0.962588in}{1.397910in}}%
\pgfpathlineto{\pgfqpoint{0.952789in}{1.403522in}}%
\pgfpathlineto{\pgfqpoint{0.942990in}{1.409173in}}%
\pgfpathlineto{\pgfqpoint{0.933191in}{1.414866in}}%
\pgfpathlineto{\pgfqpoint{0.923392in}{1.420598in}}%
\pgfpathlineto{\pgfqpoint{0.913593in}{1.426371in}}%
\pgfpathlineto{\pgfqpoint{0.903794in}{1.432184in}}%
\pgfpathlineto{\pgfqpoint{0.893995in}{1.438037in}}%
\pgfpathlineto{\pgfqpoint{0.884196in}{1.443930in}}%
\pgfpathlineto{\pgfqpoint{0.874397in}{1.449864in}}%
\pgfpathlineto{\pgfqpoint{0.864598in}{1.455838in}}%
\pgfpathlineto{\pgfqpoint{0.854799in}{1.461852in}}%
\pgfpathlineto{\pgfqpoint{0.845000in}{1.467906in}}%
\pgfpathclose%
\pgfusepath{stroke,fill}%
}%
\begin{pgfscope}%
\pgfsys@transformshift{0.000000in}{0.000000in}%
\pgfsys@useobject{currentmarker}{}%
\end{pgfscope}%
\end{pgfscope}%
\begin{pgfscope}%
\pgfsetrectcap%
\pgfsetmiterjoin%
\pgfsetlinewidth{1.254687pt}%
\definecolor{currentstroke}{rgb}{1.000000,1.000000,1.000000}%
\pgfsetstrokecolor{currentstroke}%
\pgfsetdash{}{0pt}%
\pgfpathmoveto{\pgfqpoint{0.845000in}{0.692500in}}%
\pgfpathlineto{\pgfqpoint{0.845000in}{3.572667in}}%
\pgfusepath{stroke}%
\end{pgfscope}%
\begin{pgfscope}%
\pgfsetrectcap%
\pgfsetmiterjoin%
\pgfsetlinewidth{1.254687pt}%
\definecolor{currentstroke}{rgb}{1.000000,1.000000,1.000000}%
\pgfsetstrokecolor{currentstroke}%
\pgfsetdash{}{0pt}%
\pgfpathmoveto{\pgfqpoint{5.734687in}{0.692500in}}%
\pgfpathlineto{\pgfqpoint{5.734687in}{3.572667in}}%
\pgfusepath{stroke}%
\end{pgfscope}%
\begin{pgfscope}%
\pgfsetrectcap%
\pgfsetmiterjoin%
\pgfsetlinewidth{1.254687pt}%
\definecolor{currentstroke}{rgb}{1.000000,1.000000,1.000000}%
\pgfsetstrokecolor{currentstroke}%
\pgfsetdash{}{0pt}%
\pgfpathmoveto{\pgfqpoint{0.845000in}{0.692500in}}%
\pgfpathlineto{\pgfqpoint{5.734687in}{0.692500in}}%
\pgfusepath{stroke}%
\end{pgfscope}%
\begin{pgfscope}%
\pgfsetrectcap%
\pgfsetmiterjoin%
\pgfsetlinewidth{1.254687pt}%
\definecolor{currentstroke}{rgb}{1.000000,1.000000,1.000000}%
\pgfsetstrokecolor{currentstroke}%
\pgfsetdash{}{0pt}%
\pgfpathmoveto{\pgfqpoint{0.845000in}{3.572667in}}%
\pgfpathlineto{\pgfqpoint{5.734687in}{3.572667in}}%
\pgfusepath{stroke}%
\end{pgfscope}%
\begin{pgfscope}%
\pgfpathrectangle{\pgfqpoint{0.845000in}{0.692500in}}{\pgfqpoint{4.889687in}{2.880167in}}%
\pgfusepath{clip}%
\pgfsetroundcap%
\pgfsetroundjoin%
\pgfsetlinewidth{1.505625pt}%
\definecolor{currentstroke}{rgb}{0.718353,0.818196,0.983608}%
\pgfsetstrokecolor{currentstroke}%
\pgfsetstrokeopacity{0.700000}%
\pgfsetdash{}{0pt}%
\pgfpathmoveto{\pgfqpoint{0.845000in}{1.467906in}}%
\pgfpathlineto{\pgfqpoint{0.903794in}{1.432184in}}%
\pgfpathlineto{\pgfqpoint{0.962588in}{1.397910in}}%
\pgfpathlineto{\pgfqpoint{1.021382in}{1.365084in}}%
\pgfpathlineto{\pgfqpoint{1.080175in}{1.333707in}}%
\pgfpathlineto{\pgfqpoint{1.138969in}{1.303778in}}%
\pgfpathlineto{\pgfqpoint{1.197763in}{1.275297in}}%
\pgfpathlineto{\pgfqpoint{1.256557in}{1.248265in}}%
\pgfpathlineto{\pgfqpoint{1.315351in}{1.222681in}}%
\pgfpathlineto{\pgfqpoint{1.364346in}{1.202468in}}%
\pgfpathlineto{\pgfqpoint{1.413340in}{1.183260in}}%
\pgfpathlineto{\pgfqpoint{1.462335in}{1.165058in}}%
\pgfpathlineto{\pgfqpoint{1.511330in}{1.147862in}}%
\pgfpathlineto{\pgfqpoint{1.560325in}{1.131672in}}%
\pgfpathlineto{\pgfqpoint{1.609320in}{1.116487in}}%
\pgfpathlineto{\pgfqpoint{1.658315in}{1.102309in}}%
\pgfpathlineto{\pgfqpoint{1.707310in}{1.089136in}}%
\pgfpathlineto{\pgfqpoint{1.756304in}{1.076969in}}%
\pgfpathlineto{\pgfqpoint{1.805299in}{1.065807in}}%
\pgfpathlineto{\pgfqpoint{1.854294in}{1.055652in}}%
\pgfpathlineto{\pgfqpoint{1.903289in}{1.046502in}}%
\pgfpathlineto{\pgfqpoint{1.952284in}{1.038358in}}%
\pgfpathlineto{\pgfqpoint{2.001279in}{1.031220in}}%
\pgfpathlineto{\pgfqpoint{2.050274in}{1.025088in}}%
\pgfpathlineto{\pgfqpoint{2.099269in}{1.019961in}}%
\pgfpathlineto{\pgfqpoint{2.148263in}{1.015841in}}%
\pgfpathlineto{\pgfqpoint{2.197258in}{1.012726in}}%
\pgfpathlineto{\pgfqpoint{2.246253in}{1.010616in}}%
\pgfpathlineto{\pgfqpoint{2.295248in}{1.009513in}}%
\pgfpathlineto{\pgfqpoint{2.344243in}{1.009416in}}%
\pgfpathlineto{\pgfqpoint{2.393238in}{1.010324in}}%
\pgfpathlineto{\pgfqpoint{2.442233in}{1.012238in}}%
\pgfpathlineto{\pgfqpoint{2.491227in}{1.015158in}}%
\pgfpathlineto{\pgfqpoint{2.540222in}{1.019084in}}%
\pgfpathlineto{\pgfqpoint{2.589217in}{1.024015in}}%
\pgfpathlineto{\pgfqpoint{2.638212in}{1.029952in}}%
\pgfpathlineto{\pgfqpoint{2.687207in}{1.036895in}}%
\pgfpathlineto{\pgfqpoint{2.736202in}{1.044844in}}%
\pgfpathlineto{\pgfqpoint{2.785197in}{1.053799in}}%
\pgfpathlineto{\pgfqpoint{2.834192in}{1.063759in}}%
\pgfpathlineto{\pgfqpoint{2.883186in}{1.074726in}}%
\pgfpathlineto{\pgfqpoint{2.932181in}{1.086698in}}%
\pgfpathlineto{\pgfqpoint{2.981176in}{1.099676in}}%
\pgfpathlineto{\pgfqpoint{3.030171in}{1.113659in}}%
\pgfpathlineto{\pgfqpoint{3.079166in}{1.128649in}}%
\pgfpathlineto{\pgfqpoint{3.128161in}{1.144644in}}%
\pgfpathlineto{\pgfqpoint{3.177156in}{1.161645in}}%
\pgfpathlineto{\pgfqpoint{3.226150in}{1.179652in}}%
\pgfpathlineto{\pgfqpoint{3.275145in}{1.198665in}}%
\pgfpathlineto{\pgfqpoint{3.324140in}{1.218683in}}%
\pgfpathlineto{\pgfqpoint{3.373135in}{1.239707in}}%
\pgfpathlineto{\pgfqpoint{3.431929in}{1.266264in}}%
\pgfpathlineto{\pgfqpoint{3.490723in}{1.294269in}}%
\pgfpathlineto{\pgfqpoint{3.549517in}{1.323723in}}%
\pgfpathlineto{\pgfqpoint{3.608310in}{1.354624in}}%
\pgfpathlineto{\pgfqpoint{3.667104in}{1.386975in}}%
\pgfpathlineto{\pgfqpoint{3.725898in}{1.420773in}}%
\pgfpathlineto{\pgfqpoint{3.784692in}{1.456020in}}%
\pgfpathlineto{\pgfqpoint{3.843486in}{1.492715in}}%
\pgfpathlineto{\pgfqpoint{3.902280in}{1.530859in}}%
\pgfpathlineto{\pgfqpoint{3.961073in}{1.570451in}}%
\pgfpathlineto{\pgfqpoint{4.019867in}{1.611491in}}%
\pgfpathlineto{\pgfqpoint{4.078661in}{1.653980in}}%
\pgfpathlineto{\pgfqpoint{4.137455in}{1.697917in}}%
\pgfpathlineto{\pgfqpoint{4.196249in}{1.743302in}}%
\pgfpathlineto{\pgfqpoint{4.255043in}{1.790136in}}%
\pgfpathlineto{\pgfqpoint{4.313836in}{1.838418in}}%
\pgfpathlineto{\pgfqpoint{4.372630in}{1.888148in}}%
\pgfpathlineto{\pgfqpoint{4.431424in}{1.939327in}}%
\pgfpathlineto{\pgfqpoint{4.490218in}{1.991954in}}%
\pgfpathlineto{\pgfqpoint{4.549012in}{2.046029in}}%
\pgfpathlineto{\pgfqpoint{4.607806in}{2.101553in}}%
\pgfpathlineto{\pgfqpoint{4.666599in}{2.158525in}}%
\pgfpathlineto{\pgfqpoint{4.725393in}{2.216946in}}%
\pgfpathlineto{\pgfqpoint{4.784187in}{2.276814in}}%
\pgfpathlineto{\pgfqpoint{4.842981in}{2.338132in}}%
\pgfpathlineto{\pgfqpoint{4.901775in}{2.400897in}}%
\pgfpathlineto{\pgfqpoint{4.960569in}{2.465111in}}%
\pgfpathlineto{\pgfqpoint{5.019362in}{2.530773in}}%
\pgfpathlineto{\pgfqpoint{5.078156in}{2.597884in}}%
\pgfpathlineto{\pgfqpoint{5.146749in}{2.678010in}}%
\pgfpathlineto{\pgfqpoint{5.215342in}{2.760108in}}%
\pgfpathlineto{\pgfqpoint{5.283935in}{2.844177in}}%
\pgfpathlineto{\pgfqpoint{5.352528in}{2.930217in}}%
\pgfpathlineto{\pgfqpoint{5.421120in}{3.018229in}}%
\pgfpathlineto{\pgfqpoint{5.489713in}{3.108212in}}%
\pgfpathlineto{\pgfqpoint{5.558306in}{3.200166in}}%
\pgfpathlineto{\pgfqpoint{5.626899in}{3.294092in}}%
\pgfpathlineto{\pgfqpoint{5.695492in}{3.389989in}}%
\pgfpathlineto{\pgfqpoint{5.734687in}{3.445673in}}%
\pgfpathlineto{\pgfqpoint{5.734687in}{3.445673in}}%
\pgfusepath{stroke}%
\end{pgfscope}%
\begin{pgfscope}%
\pgfpathrectangle{\pgfqpoint{0.845000in}{0.692500in}}{\pgfqpoint{4.889687in}{2.880167in}}%
\pgfusepath{clip}%
\pgfsetroundcap%
\pgfsetroundjoin%
\pgfsetlinewidth{1.505625pt}%
\definecolor{currentstroke}{rgb}{0.258824,0.521569,0.956863}%
\pgfsetstrokecolor{currentstroke}%
\pgfsetstrokeopacity{0.700000}%
\pgfsetdash{}{0pt}%
\pgfpathmoveto{\pgfqpoint{0.845000in}{1.200296in}}%
\pgfpathlineto{\pgfqpoint{0.913593in}{1.174378in}}%
\pgfpathlineto{\pgfqpoint{0.982186in}{1.149832in}}%
\pgfpathlineto{\pgfqpoint{1.040979in}{1.129886in}}%
\pgfpathlineto{\pgfqpoint{1.099773in}{1.110947in}}%
\pgfpathlineto{\pgfqpoint{1.158567in}{1.093018in}}%
\pgfpathlineto{\pgfqpoint{1.217361in}{1.076097in}}%
\pgfpathlineto{\pgfqpoint{1.276155in}{1.060184in}}%
\pgfpathlineto{\pgfqpoint{1.334949in}{1.045280in}}%
\pgfpathlineto{\pgfqpoint{1.393742in}{1.031385in}}%
\pgfpathlineto{\pgfqpoint{1.452536in}{1.018498in}}%
\pgfpathlineto{\pgfqpoint{1.511330in}{1.006620in}}%
\pgfpathlineto{\pgfqpoint{1.570124in}{0.995750in}}%
\pgfpathlineto{\pgfqpoint{1.628918in}{0.985888in}}%
\pgfpathlineto{\pgfqpoint{1.687712in}{0.977035in}}%
\pgfpathlineto{\pgfqpoint{1.746506in}{0.969191in}}%
\pgfpathlineto{\pgfqpoint{1.805299in}{0.962355in}}%
\pgfpathlineto{\pgfqpoint{1.864093in}{0.956528in}}%
\pgfpathlineto{\pgfqpoint{1.922887in}{0.951709in}}%
\pgfpathlineto{\pgfqpoint{1.981681in}{0.947899in}}%
\pgfpathlineto{\pgfqpoint{2.040475in}{0.945097in}}%
\pgfpathlineto{\pgfqpoint{2.099269in}{0.943303in}}%
\pgfpathlineto{\pgfqpoint{2.158062in}{0.942519in}}%
\pgfpathlineto{\pgfqpoint{2.216856in}{0.942742in}}%
\pgfpathlineto{\pgfqpoint{2.275650in}{0.943975in}}%
\pgfpathlineto{\pgfqpoint{2.334444in}{0.946215in}}%
\pgfpathlineto{\pgfqpoint{2.393238in}{0.949465in}}%
\pgfpathlineto{\pgfqpoint{2.452032in}{0.953722in}}%
\pgfpathlineto{\pgfqpoint{2.510825in}{0.958989in}}%
\pgfpathlineto{\pgfqpoint{2.569619in}{0.965263in}}%
\pgfpathlineto{\pgfqpoint{2.628413in}{0.972547in}}%
\pgfpathlineto{\pgfqpoint{2.687207in}{0.980839in}}%
\pgfpathlineto{\pgfqpoint{2.746001in}{0.990139in}}%
\pgfpathlineto{\pgfqpoint{2.804795in}{1.000448in}}%
\pgfpathlineto{\pgfqpoint{2.863588in}{1.011765in}}%
\pgfpathlineto{\pgfqpoint{2.922382in}{1.024091in}}%
\pgfpathlineto{\pgfqpoint{2.981176in}{1.037426in}}%
\pgfpathlineto{\pgfqpoint{3.039970in}{1.051769in}}%
\pgfpathlineto{\pgfqpoint{3.098764in}{1.067120in}}%
\pgfpathlineto{\pgfqpoint{3.157558in}{1.083480in}}%
\pgfpathlineto{\pgfqpoint{3.216351in}{1.100848in}}%
\pgfpathlineto{\pgfqpoint{3.275145in}{1.119225in}}%
\pgfpathlineto{\pgfqpoint{3.333939in}{1.138611in}}%
\pgfpathlineto{\pgfqpoint{3.392733in}{1.159005in}}%
\pgfpathlineto{\pgfqpoint{3.461326in}{1.184073in}}%
\pgfpathlineto{\pgfqpoint{3.529919in}{1.210513in}}%
\pgfpathlineto{\pgfqpoint{3.598511in}{1.238326in}}%
\pgfpathlineto{\pgfqpoint{3.667104in}{1.267512in}}%
\pgfpathlineto{\pgfqpoint{3.735697in}{1.298070in}}%
\pgfpathlineto{\pgfqpoint{3.804290in}{1.330001in}}%
\pgfpathlineto{\pgfqpoint{3.872883in}{1.363305in}}%
\pgfpathlineto{\pgfqpoint{3.941475in}{1.397982in}}%
\pgfpathlineto{\pgfqpoint{4.010068in}{1.434031in}}%
\pgfpathlineto{\pgfqpoint{4.078661in}{1.471453in}}%
\pgfpathlineto{\pgfqpoint{4.147254in}{1.510248in}}%
\pgfpathlineto{\pgfqpoint{4.215847in}{1.550415in}}%
\pgfpathlineto{\pgfqpoint{4.284440in}{1.591955in}}%
\pgfpathlineto{\pgfqpoint{4.353032in}{1.634868in}}%
\pgfpathlineto{\pgfqpoint{4.421625in}{1.679153in}}%
\pgfpathlineto{\pgfqpoint{4.490218in}{1.724811in}}%
\pgfpathlineto{\pgfqpoint{4.558811in}{1.771842in}}%
\pgfpathlineto{\pgfqpoint{4.627404in}{1.820246in}}%
\pgfpathlineto{\pgfqpoint{4.695996in}{1.870022in}}%
\pgfpathlineto{\pgfqpoint{4.764589in}{1.921171in}}%
\pgfpathlineto{\pgfqpoint{4.833182in}{1.973693in}}%
\pgfpathlineto{\pgfqpoint{4.901775in}{2.027587in}}%
\pgfpathlineto{\pgfqpoint{4.970368in}{2.082854in}}%
\pgfpathlineto{\pgfqpoint{5.038960in}{2.139494in}}%
\pgfpathlineto{\pgfqpoint{5.107553in}{2.197506in}}%
\pgfpathlineto{\pgfqpoint{5.176146in}{2.256892in}}%
\pgfpathlineto{\pgfqpoint{5.244739in}{2.317649in}}%
\pgfpathlineto{\pgfqpoint{5.313332in}{2.379780in}}%
\pgfpathlineto{\pgfqpoint{5.381924in}{2.443283in}}%
\pgfpathlineto{\pgfqpoint{5.450517in}{2.508159in}}%
\pgfpathlineto{\pgfqpoint{5.519110in}{2.574408in}}%
\pgfpathlineto{\pgfqpoint{5.587703in}{2.642029in}}%
\pgfpathlineto{\pgfqpoint{5.666095in}{2.720992in}}%
\pgfpathlineto{\pgfqpoint{5.734687in}{2.791555in}}%
\pgfpathlineto{\pgfqpoint{5.734687in}{2.791555in}}%
\pgfusepath{stroke}%
\end{pgfscope}%
\begin{pgfscope}%
\pgfpathrectangle{\pgfqpoint{0.845000in}{0.692500in}}{\pgfqpoint{4.889687in}{2.880167in}}%
\pgfusepath{clip}%
\pgfsetroundcap%
\pgfsetroundjoin%
\pgfsetlinewidth{1.505625pt}%
\definecolor{currentstroke}{rgb}{0.040118,0.284471,0.689294}%
\pgfsetstrokecolor{currentstroke}%
\pgfsetstrokeopacity{0.700000}%
\pgfsetdash{}{0pt}%
\pgfpathmoveto{\pgfqpoint{0.845000in}{1.088097in}}%
\pgfpathlineto{\pgfqpoint{0.913593in}{1.068659in}}%
\pgfpathlineto{\pgfqpoint{0.982186in}{1.050316in}}%
\pgfpathlineto{\pgfqpoint{1.050778in}{1.033069in}}%
\pgfpathlineto{\pgfqpoint{1.119371in}{1.016918in}}%
\pgfpathlineto{\pgfqpoint{1.187964in}{1.001862in}}%
\pgfpathlineto{\pgfqpoint{1.256557in}{0.987901in}}%
\pgfpathlineto{\pgfqpoint{1.325150in}{0.975036in}}%
\pgfpathlineto{\pgfqpoint{1.393742in}{0.963267in}}%
\pgfpathlineto{\pgfqpoint{1.462335in}{0.952593in}}%
\pgfpathlineto{\pgfqpoint{1.530928in}{0.943014in}}%
\pgfpathlineto{\pgfqpoint{1.599521in}{0.934531in}}%
\pgfpathlineto{\pgfqpoint{1.668114in}{0.927144in}}%
\pgfpathlineto{\pgfqpoint{1.736707in}{0.920852in}}%
\pgfpathlineto{\pgfqpoint{1.805299in}{0.915655in}}%
\pgfpathlineto{\pgfqpoint{1.873892in}{0.911554in}}%
\pgfpathlineto{\pgfqpoint{1.942485in}{0.908549in}}%
\pgfpathlineto{\pgfqpoint{2.011078in}{0.906639in}}%
\pgfpathlineto{\pgfqpoint{2.079671in}{0.905824in}}%
\pgfpathlineto{\pgfqpoint{2.148263in}{0.906105in}}%
\pgfpathlineto{\pgfqpoint{2.216856in}{0.907482in}}%
\pgfpathlineto{\pgfqpoint{2.285449in}{0.909954in}}%
\pgfpathlineto{\pgfqpoint{2.354042in}{0.913521in}}%
\pgfpathlineto{\pgfqpoint{2.422635in}{0.918184in}}%
\pgfpathlineto{\pgfqpoint{2.491227in}{0.923943in}}%
\pgfpathlineto{\pgfqpoint{2.559820in}{0.930797in}}%
\pgfpathlineto{\pgfqpoint{2.628413in}{0.938746in}}%
\pgfpathlineto{\pgfqpoint{2.697006in}{0.947791in}}%
\pgfpathlineto{\pgfqpoint{2.765599in}{0.957932in}}%
\pgfpathlineto{\pgfqpoint{2.834192in}{0.969168in}}%
\pgfpathlineto{\pgfqpoint{2.902784in}{0.981500in}}%
\pgfpathlineto{\pgfqpoint{2.971377in}{0.994927in}}%
\pgfpathlineto{\pgfqpoint{3.039970in}{1.009449in}}%
\pgfpathlineto{\pgfqpoint{3.108563in}{1.025067in}}%
\pgfpathlineto{\pgfqpoint{3.177156in}{1.041781in}}%
\pgfpathlineto{\pgfqpoint{3.245748in}{1.059590in}}%
\pgfpathlineto{\pgfqpoint{3.314341in}{1.078494in}}%
\pgfpathlineto{\pgfqpoint{3.382934in}{1.098494in}}%
\pgfpathlineto{\pgfqpoint{3.451527in}{1.119590in}}%
\pgfpathlineto{\pgfqpoint{3.520120in}{1.141781in}}%
\pgfpathlineto{\pgfqpoint{3.588712in}{1.165067in}}%
\pgfpathlineto{\pgfqpoint{3.657305in}{1.189449in}}%
\pgfpathlineto{\pgfqpoint{3.725898in}{1.214927in}}%
\pgfpathlineto{\pgfqpoint{3.794491in}{1.241500in}}%
\pgfpathlineto{\pgfqpoint{3.863084in}{1.269168in}}%
\pgfpathlineto{\pgfqpoint{3.931676in}{1.297933in}}%
\pgfpathlineto{\pgfqpoint{4.000269in}{1.327792in}}%
\pgfpathlineto{\pgfqpoint{4.068862in}{1.358747in}}%
\pgfpathlineto{\pgfqpoint{4.137455in}{1.390798in}}%
\pgfpathlineto{\pgfqpoint{4.206048in}{1.423944in}}%
\pgfpathlineto{\pgfqpoint{4.274641in}{1.458185in}}%
\pgfpathlineto{\pgfqpoint{4.343233in}{1.493522in}}%
\pgfpathlineto{\pgfqpoint{4.421625in}{1.535249in}}%
\pgfpathlineto{\pgfqpoint{4.500017in}{1.578406in}}%
\pgfpathlineto{\pgfqpoint{4.578409in}{1.622995in}}%
\pgfpathlineto{\pgfqpoint{4.656800in}{1.669014in}}%
\pgfpathlineto{\pgfqpoint{4.735192in}{1.716464in}}%
\pgfpathlineto{\pgfqpoint{4.813584in}{1.765345in}}%
\pgfpathlineto{\pgfqpoint{4.891976in}{1.815657in}}%
\pgfpathlineto{\pgfqpoint{4.970368in}{1.867399in}}%
\pgfpathlineto{\pgfqpoint{5.048759in}{1.920573in}}%
\pgfpathlineto{\pgfqpoint{5.127151in}{1.975177in}}%
\pgfpathlineto{\pgfqpoint{5.205543in}{2.031212in}}%
\pgfpathlineto{\pgfqpoint{5.283935in}{2.088678in}}%
\pgfpathlineto{\pgfqpoint{5.362327in}{2.147575in}}%
\pgfpathlineto{\pgfqpoint{5.440718in}{2.207903in}}%
\pgfpathlineto{\pgfqpoint{5.519110in}{2.269662in}}%
\pgfpathlineto{\pgfqpoint{5.597502in}{2.332851in}}%
\pgfpathlineto{\pgfqpoint{5.675894in}{2.397472in}}%
\pgfpathlineto{\pgfqpoint{5.734687in}{2.446876in}}%
\pgfpathlineto{\pgfqpoint{5.734687in}{2.446876in}}%
\pgfusepath{stroke}%
\end{pgfscope}%
\begin{pgfscope}%
\definecolor{textcolor}{rgb}{0.150000,0.150000,0.150000}%
\pgfsetstrokecolor{textcolor}%
\pgfsetfillcolor{textcolor}%
\pgftext[x=3.289844in,y=3.656000in,,base]{\color{textcolor}\sffamily\fontsize{12.000000}{14.400000}\selectfont Constrained Fitting of Drag on SD7032 Airfoil (\(\displaystyle \mathrm{Re}=10^6\))}%
\end{pgfscope}%
\begin{pgfscope}%
\pgfsetbuttcap%
\pgfsetmiterjoin%
\definecolor{currentfill}{rgb}{0.917647,0.917647,0.949020}%
\pgfsetfillcolor{currentfill}%
\pgfsetfillopacity{0.800000}%
\pgfsetlinewidth{1.003750pt}%
\definecolor{currentstroke}{rgb}{0.800000,0.800000,0.800000}%
\pgfsetstrokecolor{currentstroke}%
\pgfsetstrokeopacity{0.800000}%
\pgfsetdash{}{0pt}%
\pgfpathmoveto{\pgfqpoint{0.951944in}{2.598556in}}%
\pgfpathlineto{\pgfqpoint{2.495611in}{2.598556in}}%
\pgfpathquadraticcurveto{\pgfqpoint{2.526167in}{2.598556in}}{\pgfqpoint{2.526167in}{2.629112in}}%
\pgfpathlineto{\pgfqpoint{2.526167in}{3.465722in}}%
\pgfpathquadraticcurveto{\pgfqpoint{2.526167in}{3.496278in}}{\pgfqpoint{2.495611in}{3.496278in}}%
\pgfpathlineto{\pgfqpoint{0.951944in}{3.496278in}}%
\pgfpathquadraticcurveto{\pgfqpoint{0.921389in}{3.496278in}}{\pgfqpoint{0.921389in}{3.465722in}}%
\pgfpathlineto{\pgfqpoint{0.921389in}{2.629112in}}%
\pgfpathquadraticcurveto{\pgfqpoint{0.921389in}{2.598556in}}{\pgfqpoint{0.951944in}{2.598556in}}%
\pgfpathclose%
\pgfusepath{stroke,fill}%
\end{pgfscope}%
\begin{pgfscope}%
\pgfsetbuttcap%
\pgfsetroundjoin%
\definecolor{currentfill}{rgb}{0.000000,0.000000,0.000000}%
\pgfsetfillcolor{currentfill}%
\pgfsetfillopacity{0.600000}%
\pgfsetlinewidth{1.003750pt}%
\definecolor{currentstroke}{rgb}{0.000000,0.000000,0.000000}%
\pgfsetstrokecolor{currentstroke}%
\pgfsetstrokeopacity{0.600000}%
\pgfsetdash{}{0pt}%
\pgfsys@defobject{currentmarker}{\pgfqpoint{-0.020833in}{-0.020833in}}{\pgfqpoint{0.020833in}{0.020833in}}{%
\pgfpathmoveto{\pgfqpoint{0.000000in}{-0.020833in}}%
\pgfpathcurveto{\pgfqpoint{0.005525in}{-0.020833in}}{\pgfqpoint{0.010825in}{-0.018638in}}{\pgfqpoint{0.014731in}{-0.014731in}}%
\pgfpathcurveto{\pgfqpoint{0.018638in}{-0.010825in}}{\pgfqpoint{0.020833in}{-0.005525in}}{\pgfqpoint{0.020833in}{0.000000in}}%
\pgfpathcurveto{\pgfqpoint{0.020833in}{0.005525in}}{\pgfqpoint{0.018638in}{0.010825in}}{\pgfqpoint{0.014731in}{0.014731in}}%
\pgfpathcurveto{\pgfqpoint{0.010825in}{0.018638in}}{\pgfqpoint{0.005525in}{0.020833in}}{\pgfqpoint{0.000000in}{0.020833in}}%
\pgfpathcurveto{\pgfqpoint{-0.005525in}{0.020833in}}{\pgfqpoint{-0.010825in}{0.018638in}}{\pgfqpoint{-0.014731in}{0.014731in}}%
\pgfpathcurveto{\pgfqpoint{-0.018638in}{0.010825in}}{\pgfqpoint{-0.020833in}{0.005525in}}{\pgfqpoint{-0.020833in}{0.000000in}}%
\pgfpathcurveto{\pgfqpoint{-0.020833in}{-0.005525in}}{\pgfqpoint{-0.018638in}{-0.010825in}}{\pgfqpoint{-0.014731in}{-0.014731in}}%
\pgfpathcurveto{\pgfqpoint{-0.010825in}{-0.018638in}}{\pgfqpoint{-0.005525in}{-0.020833in}}{\pgfqpoint{0.000000in}{-0.020833in}}%
\pgfpathclose%
\pgfusepath{stroke,fill}%
}%
\begin{pgfscope}%
\pgfsys@transformshift{1.135278in}{3.381694in}%
\pgfsys@useobject{currentmarker}{}%
\end{pgfscope}%
\end{pgfscope}%
\begin{pgfscope}%
\definecolor{textcolor}{rgb}{0.150000,0.150000,0.150000}%
\pgfsetstrokecolor{textcolor}%
\pgfsetfillcolor{textcolor}%
\pgftext[x=1.410278in,y=3.328222in,left,base]{\color{textcolor}\sffamily\fontsize{11.000000}{13.200000}\selectfont XFoil Data}%
\end{pgfscope}%
\begin{pgfscope}%
\pgfsetroundcap%
\pgfsetroundjoin%
\pgfsetlinewidth{1.505625pt}%
\definecolor{currentstroke}{rgb}{0.718353,0.818196,0.983608}%
\pgfsetstrokecolor{currentstroke}%
\pgfsetstrokeopacity{0.700000}%
\pgfsetdash{}{0pt}%
\pgfpathmoveto{\pgfqpoint{0.982500in}{3.168722in}}%
\pgfpathlineto{\pgfqpoint{1.288056in}{3.168722in}}%
\pgfusepath{stroke}%
\end{pgfscope}%
\begin{pgfscope}%
\definecolor{textcolor}{rgb}{0.150000,0.150000,0.150000}%
\pgfsetstrokecolor{textcolor}%
\pgfsetfillcolor{textcolor}%
\pgftext[x=1.410278in,y=3.115250in,left,base]{\color{textcolor}\sffamily\fontsize{11.000000}{13.200000}\selectfont Upper Bound Fit}%
\end{pgfscope}%
\begin{pgfscope}%
\pgfsetroundcap%
\pgfsetroundjoin%
\pgfsetlinewidth{1.505625pt}%
\definecolor{currentstroke}{rgb}{0.258824,0.521569,0.956863}%
\pgfsetstrokecolor{currentstroke}%
\pgfsetstrokeopacity{0.700000}%
\pgfsetdash{}{0pt}%
\pgfpathmoveto{\pgfqpoint{0.982500in}{2.955750in}}%
\pgfpathlineto{\pgfqpoint{1.288056in}{2.955750in}}%
\pgfusepath{stroke}%
\end{pgfscope}%
\begin{pgfscope}%
\definecolor{textcolor}{rgb}{0.150000,0.150000,0.150000}%
\pgfsetstrokecolor{textcolor}%
\pgfsetfillcolor{textcolor}%
\pgftext[x=1.410278in,y=2.902278in,left,base]{\color{textcolor}\sffamily\fontsize{11.000000}{13.200000}\selectfont Best Fit}%
\end{pgfscope}%
\begin{pgfscope}%
\pgfsetroundcap%
\pgfsetroundjoin%
\pgfsetlinewidth{1.505625pt}%
\definecolor{currentstroke}{rgb}{0.040118,0.284471,0.689294}%
\pgfsetstrokecolor{currentstroke}%
\pgfsetstrokeopacity{0.700000}%
\pgfsetdash{}{0pt}%
\pgfpathmoveto{\pgfqpoint{0.982500in}{2.742778in}}%
\pgfpathlineto{\pgfqpoint{1.288056in}{2.742778in}}%
\pgfusepath{stroke}%
\end{pgfscope}%
\begin{pgfscope}%
\definecolor{textcolor}{rgb}{0.150000,0.150000,0.150000}%
\pgfsetstrokecolor{textcolor}%
\pgfsetfillcolor{textcolor}%
\pgftext[x=1.410278in,y=2.689306in,left,base]{\color{textcolor}\sffamily\fontsize{11.000000}{13.200000}\selectfont Lower Bound Fit}%
\end{pgfscope}%
\end{pgfpicture}%
\makeatother%
\endgroup%
}
    \caption{Robust fitting of an example drag polar with model bounds.}
    \label{fig:constrained-fitting}
\end{figure}

\subsubsection{Log-transformed Errors}

A final note here is that many outputs of engineering models are more physically relevant when considered in a log-transformed sense. In other words, the goodness of fit is a function of the relative (multiplicative) error rather than the absolute (additive) error.

A common example quantity that demonstrates this phenomenon is aerodynamic drag, which tends to approximately follow a power law with respect to Reynolds number. For illustration, we consider the case of the drag coefficient on a cylinder in crossflow. Experimental data plotted in Panton \cite{Panton} (reproduced here in Fig. \ref{fig:cylinder-drag}) found drag coefficients ranging from approximately 0.25 to 500, depending on Reynolds number. Indeed, in the $\text{Re} \to 0$ (i.e. Stokes flow) limit, this drag coefficient exactly follows a power law and is unbounded\footnote{This is because the drag force $D$ becomes linearly proportional to the freestream velocity $U_\infty$ in the Stokes limit, rather than the usual $D \propto U_\infty^2$.}.

Using the AeroSandbox fitting module, a relatively parsimonious analytical model can be fit that accurately predicts (and extrapolates) cylinder drag for any Reynolds number. Surprisingly, the author has not found any other such universal model for cylinder drag in the literature.

Because of the logarithmic importance of drag coefficient, the fitting module \mintinline{python}{asb.FittedModel} is supplied with \mintinline{python}{put_residuals_in_logspace=True} at initialization. The resulting model, which minimizes log-transformed error with respect to the experimental dataset, is as follows:

\newcommand{\logt}{\log_{10}}
\newcommand{\logtr}{\logt(\text{Re})}

\begin{example}
    \noindent
    \textbf{Cylinder Drag Fitted Model}

    \noindent
    First, models for subcritical (i.e. below drag crisis) and supercritical drag are computed:
    \begin{equation*}
        \begin{aligned}
            r &= \logtr \\
            C_{D\text{, subcrit}} &= 10^ {-0.6739 r + 1.0355} + 0.6325 + 0.1006 r \\
            \logt(C_{D\text{, supercrit}}) &= -0.1200 - 0.04615 \ln\bigg[\exp\big(10 \cdot (6.7016 - r)\big) + 1 \bigg] \\
        \end{aligned}
    \end{equation*}

    \noindent
    Then, these equations are blended together using a sigmoid:
    \begin{equation}
        \begin{aligned}
            \sigma(r) &= \frac{\tanh \big[12.597 \cdot (r - 5.547)\big] + 1}{2} \\
            C_D &= \sigma(r) \cdot C_{D\text{, supercrit}} + \big[1 - \sigma(r)\big] \cdot C_{D\text{, subcrit}} \\
        \end{aligned}
    \end{equation}
\end{example}

\noindent
The fit of this model to the experimental data from \cite{Panton} is shown in Figure \ref{fig:cylinder-drag}.

\begin{figure}[H]
    \centering
%    \centerline{%% Creator: Matplotlib, PGF backend
%%
%% To include the figure in your LaTeX document, write
%%   \input{<filename>.pgf}
%%
%% Make sure the required packages are loaded in your preamble
%%   \usepackage{pgf}
%%
%% Figures using additional raster images can only be included by \input if
%% they are in the same directory as the main LaTeX file. For loading figures
%% from other directories you can use the `import` package
%%   \usepackage{import}
%%
%% and then include the figures with
%%   \import{<path to file>}{<filename>.pgf}
%%
%% Matplotlib used the following preamble
%%   \usepackage{fontspec}
%%
\begingroup%
\makeatletter%
\begin{pgfpicture}%
\pgfpathrectangle{\pgfpointorigin}{\pgfqpoint{6.000000in}{3.500000in}}%
\pgfusepath{use as bounding box, clip}%
\begin{pgfscope}%
\pgfsetbuttcap%
\pgfsetmiterjoin%
\definecolor{currentfill}{rgb}{1.000000,1.000000,1.000000}%
\pgfsetfillcolor{currentfill}%
\pgfsetlinewidth{0.000000pt}%
\definecolor{currentstroke}{rgb}{1.000000,1.000000,1.000000}%
\pgfsetstrokecolor{currentstroke}%
\pgfsetstrokeopacity{0.000000}%
\pgfsetdash{}{0pt}%
\pgfpathmoveto{\pgfqpoint{0.000000in}{0.000000in}}%
\pgfpathlineto{\pgfqpoint{6.000000in}{0.000000in}}%
\pgfpathlineto{\pgfqpoint{6.000000in}{3.500000in}}%
\pgfpathlineto{\pgfqpoint{0.000000in}{3.500000in}}%
\pgfpathclose%
\pgfusepath{fill}%
\end{pgfscope}%
\begin{pgfscope}%
\pgfsetbuttcap%
\pgfsetmiterjoin%
\definecolor{currentfill}{rgb}{0.917647,0.917647,0.949020}%
\pgfsetfillcolor{currentfill}%
\pgfsetlinewidth{0.000000pt}%
\definecolor{currentstroke}{rgb}{0.000000,0.000000,0.000000}%
\pgfsetstrokecolor{currentstroke}%
\pgfsetstrokeopacity{0.000000}%
\pgfsetdash{}{0pt}%
\pgfpathmoveto{\pgfqpoint{0.832441in}{0.754477in}}%
\pgfpathlineto{\pgfqpoint{5.820000in}{0.754477in}}%
\pgfpathlineto{\pgfqpoint{5.820000in}{3.115885in}}%
\pgfpathlineto{\pgfqpoint{0.832441in}{3.115885in}}%
\pgfpathclose%
\pgfusepath{fill}%
\end{pgfscope}%
\begin{pgfscope}%
\pgfpathrectangle{\pgfqpoint{0.832441in}{0.754477in}}{\pgfqpoint{4.987559in}{2.361409in}}%
\pgfusepath{clip}%
\pgfsetroundcap%
\pgfsetroundjoin%
\pgfsetlinewidth{1.606000pt}%
\definecolor{currentstroke}{rgb}{1.000000,1.000000,1.000000}%
\pgfsetstrokecolor{currentstroke}%
\pgfsetdash{}{0pt}%
\pgfpathmoveto{\pgfqpoint{1.109528in}{0.754477in}}%
\pgfpathlineto{\pgfqpoint{1.109528in}{3.115885in}}%
\pgfusepath{stroke}%
\end{pgfscope}%
\begin{pgfscope}%
\definecolor{textcolor}{rgb}{0.150000,0.150000,0.150000}%
\pgfsetstrokecolor{textcolor}%
\pgfsetfillcolor{textcolor}%
\pgftext[x=1.109528in,y=0.622532in,,top]{\color{textcolor}\sffamily\fontsize{11.000000}{13.200000}\selectfont \(\displaystyle {10^{-1}}\)}%
\end{pgfscope}%
\begin{pgfscope}%
\pgfpathrectangle{\pgfqpoint{0.832441in}{0.754477in}}{\pgfqpoint{4.987559in}{2.361409in}}%
\pgfusepath{clip}%
\pgfsetroundcap%
\pgfsetroundjoin%
\pgfsetlinewidth{1.606000pt}%
\definecolor{currentstroke}{rgb}{1.000000,1.000000,1.000000}%
\pgfsetstrokecolor{currentstroke}%
\pgfsetdash{}{0pt}%
\pgfpathmoveto{\pgfqpoint{1.663701in}{0.754477in}}%
\pgfpathlineto{\pgfqpoint{1.663701in}{3.115885in}}%
\pgfusepath{stroke}%
\end{pgfscope}%
\begin{pgfscope}%
\definecolor{textcolor}{rgb}{0.150000,0.150000,0.150000}%
\pgfsetstrokecolor{textcolor}%
\pgfsetfillcolor{textcolor}%
\pgftext[x=1.663701in,y=0.622532in,,top]{\color{textcolor}\sffamily\fontsize{11.000000}{13.200000}\selectfont \(\displaystyle {10^{0}}\)}%
\end{pgfscope}%
\begin{pgfscope}%
\pgfpathrectangle{\pgfqpoint{0.832441in}{0.754477in}}{\pgfqpoint{4.987559in}{2.361409in}}%
\pgfusepath{clip}%
\pgfsetroundcap%
\pgfsetroundjoin%
\pgfsetlinewidth{1.606000pt}%
\definecolor{currentstroke}{rgb}{1.000000,1.000000,1.000000}%
\pgfsetstrokecolor{currentstroke}%
\pgfsetdash{}{0pt}%
\pgfpathmoveto{\pgfqpoint{2.217874in}{0.754477in}}%
\pgfpathlineto{\pgfqpoint{2.217874in}{3.115885in}}%
\pgfusepath{stroke}%
\end{pgfscope}%
\begin{pgfscope}%
\definecolor{textcolor}{rgb}{0.150000,0.150000,0.150000}%
\pgfsetstrokecolor{textcolor}%
\pgfsetfillcolor{textcolor}%
\pgftext[x=2.217874in,y=0.622532in,,top]{\color{textcolor}\sffamily\fontsize{11.000000}{13.200000}\selectfont \(\displaystyle {10^{1}}\)}%
\end{pgfscope}%
\begin{pgfscope}%
\pgfpathrectangle{\pgfqpoint{0.832441in}{0.754477in}}{\pgfqpoint{4.987559in}{2.361409in}}%
\pgfusepath{clip}%
\pgfsetroundcap%
\pgfsetroundjoin%
\pgfsetlinewidth{1.606000pt}%
\definecolor{currentstroke}{rgb}{1.000000,1.000000,1.000000}%
\pgfsetstrokecolor{currentstroke}%
\pgfsetdash{}{0pt}%
\pgfpathmoveto{\pgfqpoint{2.772047in}{0.754477in}}%
\pgfpathlineto{\pgfqpoint{2.772047in}{3.115885in}}%
\pgfusepath{stroke}%
\end{pgfscope}%
\begin{pgfscope}%
\definecolor{textcolor}{rgb}{0.150000,0.150000,0.150000}%
\pgfsetstrokecolor{textcolor}%
\pgfsetfillcolor{textcolor}%
\pgftext[x=2.772047in,y=0.622532in,,top]{\color{textcolor}\sffamily\fontsize{11.000000}{13.200000}\selectfont \(\displaystyle {10^{2}}\)}%
\end{pgfscope}%
\begin{pgfscope}%
\pgfpathrectangle{\pgfqpoint{0.832441in}{0.754477in}}{\pgfqpoint{4.987559in}{2.361409in}}%
\pgfusepath{clip}%
\pgfsetroundcap%
\pgfsetroundjoin%
\pgfsetlinewidth{1.606000pt}%
\definecolor{currentstroke}{rgb}{1.000000,1.000000,1.000000}%
\pgfsetstrokecolor{currentstroke}%
\pgfsetdash{}{0pt}%
\pgfpathmoveto{\pgfqpoint{3.326221in}{0.754477in}}%
\pgfpathlineto{\pgfqpoint{3.326221in}{3.115885in}}%
\pgfusepath{stroke}%
\end{pgfscope}%
\begin{pgfscope}%
\definecolor{textcolor}{rgb}{0.150000,0.150000,0.150000}%
\pgfsetstrokecolor{textcolor}%
\pgfsetfillcolor{textcolor}%
\pgftext[x=3.326221in,y=0.622532in,,top]{\color{textcolor}\sffamily\fontsize{11.000000}{13.200000}\selectfont \(\displaystyle {10^{3}}\)}%
\end{pgfscope}%
\begin{pgfscope}%
\pgfpathrectangle{\pgfqpoint{0.832441in}{0.754477in}}{\pgfqpoint{4.987559in}{2.361409in}}%
\pgfusepath{clip}%
\pgfsetroundcap%
\pgfsetroundjoin%
\pgfsetlinewidth{1.606000pt}%
\definecolor{currentstroke}{rgb}{1.000000,1.000000,1.000000}%
\pgfsetstrokecolor{currentstroke}%
\pgfsetdash{}{0pt}%
\pgfpathmoveto{\pgfqpoint{3.880394in}{0.754477in}}%
\pgfpathlineto{\pgfqpoint{3.880394in}{3.115885in}}%
\pgfusepath{stroke}%
\end{pgfscope}%
\begin{pgfscope}%
\definecolor{textcolor}{rgb}{0.150000,0.150000,0.150000}%
\pgfsetstrokecolor{textcolor}%
\pgfsetfillcolor{textcolor}%
\pgftext[x=3.880394in,y=0.622532in,,top]{\color{textcolor}\sffamily\fontsize{11.000000}{13.200000}\selectfont \(\displaystyle {10^{4}}\)}%
\end{pgfscope}%
\begin{pgfscope}%
\pgfpathrectangle{\pgfqpoint{0.832441in}{0.754477in}}{\pgfqpoint{4.987559in}{2.361409in}}%
\pgfusepath{clip}%
\pgfsetroundcap%
\pgfsetroundjoin%
\pgfsetlinewidth{1.606000pt}%
\definecolor{currentstroke}{rgb}{1.000000,1.000000,1.000000}%
\pgfsetstrokecolor{currentstroke}%
\pgfsetdash{}{0pt}%
\pgfpathmoveto{\pgfqpoint{4.434567in}{0.754477in}}%
\pgfpathlineto{\pgfqpoint{4.434567in}{3.115885in}}%
\pgfusepath{stroke}%
\end{pgfscope}%
\begin{pgfscope}%
\definecolor{textcolor}{rgb}{0.150000,0.150000,0.150000}%
\pgfsetstrokecolor{textcolor}%
\pgfsetfillcolor{textcolor}%
\pgftext[x=4.434567in,y=0.622532in,,top]{\color{textcolor}\sffamily\fontsize{11.000000}{13.200000}\selectfont \(\displaystyle {10^{5}}\)}%
\end{pgfscope}%
\begin{pgfscope}%
\pgfpathrectangle{\pgfqpoint{0.832441in}{0.754477in}}{\pgfqpoint{4.987559in}{2.361409in}}%
\pgfusepath{clip}%
\pgfsetroundcap%
\pgfsetroundjoin%
\pgfsetlinewidth{1.606000pt}%
\definecolor{currentstroke}{rgb}{1.000000,1.000000,1.000000}%
\pgfsetstrokecolor{currentstroke}%
\pgfsetdash{}{0pt}%
\pgfpathmoveto{\pgfqpoint{4.988740in}{0.754477in}}%
\pgfpathlineto{\pgfqpoint{4.988740in}{3.115885in}}%
\pgfusepath{stroke}%
\end{pgfscope}%
\begin{pgfscope}%
\definecolor{textcolor}{rgb}{0.150000,0.150000,0.150000}%
\pgfsetstrokecolor{textcolor}%
\pgfsetfillcolor{textcolor}%
\pgftext[x=4.988740in,y=0.622532in,,top]{\color{textcolor}\sffamily\fontsize{11.000000}{13.200000}\selectfont \(\displaystyle {10^{6}}\)}%
\end{pgfscope}%
\begin{pgfscope}%
\pgfpathrectangle{\pgfqpoint{0.832441in}{0.754477in}}{\pgfqpoint{4.987559in}{2.361409in}}%
\pgfusepath{clip}%
\pgfsetroundcap%
\pgfsetroundjoin%
\pgfsetlinewidth{1.606000pt}%
\definecolor{currentstroke}{rgb}{1.000000,1.000000,1.000000}%
\pgfsetstrokecolor{currentstroke}%
\pgfsetdash{}{0pt}%
\pgfpathmoveto{\pgfqpoint{5.542913in}{0.754477in}}%
\pgfpathlineto{\pgfqpoint{5.542913in}{3.115885in}}%
\pgfusepath{stroke}%
\end{pgfscope}%
\begin{pgfscope}%
\definecolor{textcolor}{rgb}{0.150000,0.150000,0.150000}%
\pgfsetstrokecolor{textcolor}%
\pgfsetfillcolor{textcolor}%
\pgftext[x=5.542913in,y=0.622532in,,top]{\color{textcolor}\sffamily\fontsize{11.000000}{13.200000}\selectfont \(\displaystyle {10^{7}}\)}%
\end{pgfscope}%
\begin{pgfscope}%
\pgfpathrectangle{\pgfqpoint{0.832441in}{0.754477in}}{\pgfqpoint{4.987559in}{2.361409in}}%
\pgfusepath{clip}%
\pgfsetroundcap%
\pgfsetroundjoin%
\pgfsetlinewidth{0.702625pt}%
\definecolor{currentstroke}{rgb}{1.000000,1.000000,1.000000}%
\pgfsetstrokecolor{currentstroke}%
\pgfsetdash{}{0pt}%
\pgfpathmoveto{\pgfqpoint{0.889000in}{0.754477in}}%
\pgfpathlineto{\pgfqpoint{0.889000in}{3.115885in}}%
\pgfusepath{stroke}%
\end{pgfscope}%
\begin{pgfscope}%
\pgfpathrectangle{\pgfqpoint{0.832441in}{0.754477in}}{\pgfqpoint{4.987559in}{2.361409in}}%
\pgfusepath{clip}%
\pgfsetroundcap%
\pgfsetroundjoin%
\pgfsetlinewidth{0.702625pt}%
\definecolor{currentstroke}{rgb}{1.000000,1.000000,1.000000}%
\pgfsetstrokecolor{currentstroke}%
\pgfsetdash{}{0pt}%
\pgfpathmoveto{\pgfqpoint{0.942705in}{0.754477in}}%
\pgfpathlineto{\pgfqpoint{0.942705in}{3.115885in}}%
\pgfusepath{stroke}%
\end{pgfscope}%
\begin{pgfscope}%
\pgfpathrectangle{\pgfqpoint{0.832441in}{0.754477in}}{\pgfqpoint{4.987559in}{2.361409in}}%
\pgfusepath{clip}%
\pgfsetroundcap%
\pgfsetroundjoin%
\pgfsetlinewidth{0.702625pt}%
\definecolor{currentstroke}{rgb}{1.000000,1.000000,1.000000}%
\pgfsetstrokecolor{currentstroke}%
\pgfsetdash{}{0pt}%
\pgfpathmoveto{\pgfqpoint{0.986585in}{0.754477in}}%
\pgfpathlineto{\pgfqpoint{0.986585in}{3.115885in}}%
\pgfusepath{stroke}%
\end{pgfscope}%
\begin{pgfscope}%
\pgfpathrectangle{\pgfqpoint{0.832441in}{0.754477in}}{\pgfqpoint{4.987559in}{2.361409in}}%
\pgfusepath{clip}%
\pgfsetroundcap%
\pgfsetroundjoin%
\pgfsetlinewidth{0.702625pt}%
\definecolor{currentstroke}{rgb}{1.000000,1.000000,1.000000}%
\pgfsetstrokecolor{currentstroke}%
\pgfsetdash{}{0pt}%
\pgfpathmoveto{\pgfqpoint{1.023685in}{0.754477in}}%
\pgfpathlineto{\pgfqpoint{1.023685in}{3.115885in}}%
\pgfusepath{stroke}%
\end{pgfscope}%
\begin{pgfscope}%
\pgfpathrectangle{\pgfqpoint{0.832441in}{0.754477in}}{\pgfqpoint{4.987559in}{2.361409in}}%
\pgfusepath{clip}%
\pgfsetroundcap%
\pgfsetroundjoin%
\pgfsetlinewidth{0.702625pt}%
\definecolor{currentstroke}{rgb}{1.000000,1.000000,1.000000}%
\pgfsetstrokecolor{currentstroke}%
\pgfsetdash{}{0pt}%
\pgfpathmoveto{\pgfqpoint{1.055823in}{0.754477in}}%
\pgfpathlineto{\pgfqpoint{1.055823in}{3.115885in}}%
\pgfusepath{stroke}%
\end{pgfscope}%
\begin{pgfscope}%
\pgfpathrectangle{\pgfqpoint{0.832441in}{0.754477in}}{\pgfqpoint{4.987559in}{2.361409in}}%
\pgfusepath{clip}%
\pgfsetroundcap%
\pgfsetroundjoin%
\pgfsetlinewidth{0.702625pt}%
\definecolor{currentstroke}{rgb}{1.000000,1.000000,1.000000}%
\pgfsetstrokecolor{currentstroke}%
\pgfsetdash{}{0pt}%
\pgfpathmoveto{\pgfqpoint{1.084170in}{0.754477in}}%
\pgfpathlineto{\pgfqpoint{1.084170in}{3.115885in}}%
\pgfusepath{stroke}%
\end{pgfscope}%
\begin{pgfscope}%
\pgfpathrectangle{\pgfqpoint{0.832441in}{0.754477in}}{\pgfqpoint{4.987559in}{2.361409in}}%
\pgfusepath{clip}%
\pgfsetroundcap%
\pgfsetroundjoin%
\pgfsetlinewidth{0.702625pt}%
\definecolor{currentstroke}{rgb}{1.000000,1.000000,1.000000}%
\pgfsetstrokecolor{currentstroke}%
\pgfsetdash{}{0pt}%
\pgfpathmoveto{\pgfqpoint{1.276351in}{0.754477in}}%
\pgfpathlineto{\pgfqpoint{1.276351in}{3.115885in}}%
\pgfusepath{stroke}%
\end{pgfscope}%
\begin{pgfscope}%
\pgfpathrectangle{\pgfqpoint{0.832441in}{0.754477in}}{\pgfqpoint{4.987559in}{2.361409in}}%
\pgfusepath{clip}%
\pgfsetroundcap%
\pgfsetroundjoin%
\pgfsetlinewidth{0.702625pt}%
\definecolor{currentstroke}{rgb}{1.000000,1.000000,1.000000}%
\pgfsetstrokecolor{currentstroke}%
\pgfsetdash{}{0pt}%
\pgfpathmoveto{\pgfqpoint{1.373936in}{0.754477in}}%
\pgfpathlineto{\pgfqpoint{1.373936in}{3.115885in}}%
\pgfusepath{stroke}%
\end{pgfscope}%
\begin{pgfscope}%
\pgfpathrectangle{\pgfqpoint{0.832441in}{0.754477in}}{\pgfqpoint{4.987559in}{2.361409in}}%
\pgfusepath{clip}%
\pgfsetroundcap%
\pgfsetroundjoin%
\pgfsetlinewidth{0.702625pt}%
\definecolor{currentstroke}{rgb}{1.000000,1.000000,1.000000}%
\pgfsetstrokecolor{currentstroke}%
\pgfsetdash{}{0pt}%
\pgfpathmoveto{\pgfqpoint{1.443173in}{0.754477in}}%
\pgfpathlineto{\pgfqpoint{1.443173in}{3.115885in}}%
\pgfusepath{stroke}%
\end{pgfscope}%
\begin{pgfscope}%
\pgfpathrectangle{\pgfqpoint{0.832441in}{0.754477in}}{\pgfqpoint{4.987559in}{2.361409in}}%
\pgfusepath{clip}%
\pgfsetroundcap%
\pgfsetroundjoin%
\pgfsetlinewidth{0.702625pt}%
\definecolor{currentstroke}{rgb}{1.000000,1.000000,1.000000}%
\pgfsetstrokecolor{currentstroke}%
\pgfsetdash{}{0pt}%
\pgfpathmoveto{\pgfqpoint{1.496878in}{0.754477in}}%
\pgfpathlineto{\pgfqpoint{1.496878in}{3.115885in}}%
\pgfusepath{stroke}%
\end{pgfscope}%
\begin{pgfscope}%
\pgfpathrectangle{\pgfqpoint{0.832441in}{0.754477in}}{\pgfqpoint{4.987559in}{2.361409in}}%
\pgfusepath{clip}%
\pgfsetroundcap%
\pgfsetroundjoin%
\pgfsetlinewidth{0.702625pt}%
\definecolor{currentstroke}{rgb}{1.000000,1.000000,1.000000}%
\pgfsetstrokecolor{currentstroke}%
\pgfsetdash{}{0pt}%
\pgfpathmoveto{\pgfqpoint{1.540758in}{0.754477in}}%
\pgfpathlineto{\pgfqpoint{1.540758in}{3.115885in}}%
\pgfusepath{stroke}%
\end{pgfscope}%
\begin{pgfscope}%
\pgfpathrectangle{\pgfqpoint{0.832441in}{0.754477in}}{\pgfqpoint{4.987559in}{2.361409in}}%
\pgfusepath{clip}%
\pgfsetroundcap%
\pgfsetroundjoin%
\pgfsetlinewidth{0.702625pt}%
\definecolor{currentstroke}{rgb}{1.000000,1.000000,1.000000}%
\pgfsetstrokecolor{currentstroke}%
\pgfsetdash{}{0pt}%
\pgfpathmoveto{\pgfqpoint{1.577858in}{0.754477in}}%
\pgfpathlineto{\pgfqpoint{1.577858in}{3.115885in}}%
\pgfusepath{stroke}%
\end{pgfscope}%
\begin{pgfscope}%
\pgfpathrectangle{\pgfqpoint{0.832441in}{0.754477in}}{\pgfqpoint{4.987559in}{2.361409in}}%
\pgfusepath{clip}%
\pgfsetroundcap%
\pgfsetroundjoin%
\pgfsetlinewidth{0.702625pt}%
\definecolor{currentstroke}{rgb}{1.000000,1.000000,1.000000}%
\pgfsetstrokecolor{currentstroke}%
\pgfsetdash{}{0pt}%
\pgfpathmoveto{\pgfqpoint{1.609996in}{0.754477in}}%
\pgfpathlineto{\pgfqpoint{1.609996in}{3.115885in}}%
\pgfusepath{stroke}%
\end{pgfscope}%
\begin{pgfscope}%
\pgfpathrectangle{\pgfqpoint{0.832441in}{0.754477in}}{\pgfqpoint{4.987559in}{2.361409in}}%
\pgfusepath{clip}%
\pgfsetroundcap%
\pgfsetroundjoin%
\pgfsetlinewidth{0.702625pt}%
\definecolor{currentstroke}{rgb}{1.000000,1.000000,1.000000}%
\pgfsetstrokecolor{currentstroke}%
\pgfsetdash{}{0pt}%
\pgfpathmoveto{\pgfqpoint{1.638343in}{0.754477in}}%
\pgfpathlineto{\pgfqpoint{1.638343in}{3.115885in}}%
\pgfusepath{stroke}%
\end{pgfscope}%
\begin{pgfscope}%
\pgfpathrectangle{\pgfqpoint{0.832441in}{0.754477in}}{\pgfqpoint{4.987559in}{2.361409in}}%
\pgfusepath{clip}%
\pgfsetroundcap%
\pgfsetroundjoin%
\pgfsetlinewidth{0.702625pt}%
\definecolor{currentstroke}{rgb}{1.000000,1.000000,1.000000}%
\pgfsetstrokecolor{currentstroke}%
\pgfsetdash{}{0pt}%
\pgfpathmoveto{\pgfqpoint{1.830524in}{0.754477in}}%
\pgfpathlineto{\pgfqpoint{1.830524in}{3.115885in}}%
\pgfusepath{stroke}%
\end{pgfscope}%
\begin{pgfscope}%
\pgfpathrectangle{\pgfqpoint{0.832441in}{0.754477in}}{\pgfqpoint{4.987559in}{2.361409in}}%
\pgfusepath{clip}%
\pgfsetroundcap%
\pgfsetroundjoin%
\pgfsetlinewidth{0.702625pt}%
\definecolor{currentstroke}{rgb}{1.000000,1.000000,1.000000}%
\pgfsetstrokecolor{currentstroke}%
\pgfsetdash{}{0pt}%
\pgfpathmoveto{\pgfqpoint{1.928109in}{0.754477in}}%
\pgfpathlineto{\pgfqpoint{1.928109in}{3.115885in}}%
\pgfusepath{stroke}%
\end{pgfscope}%
\begin{pgfscope}%
\pgfpathrectangle{\pgfqpoint{0.832441in}{0.754477in}}{\pgfqpoint{4.987559in}{2.361409in}}%
\pgfusepath{clip}%
\pgfsetroundcap%
\pgfsetroundjoin%
\pgfsetlinewidth{0.702625pt}%
\definecolor{currentstroke}{rgb}{1.000000,1.000000,1.000000}%
\pgfsetstrokecolor{currentstroke}%
\pgfsetdash{}{0pt}%
\pgfpathmoveto{\pgfqpoint{1.997347in}{0.754477in}}%
\pgfpathlineto{\pgfqpoint{1.997347in}{3.115885in}}%
\pgfusepath{stroke}%
\end{pgfscope}%
\begin{pgfscope}%
\pgfpathrectangle{\pgfqpoint{0.832441in}{0.754477in}}{\pgfqpoint{4.987559in}{2.361409in}}%
\pgfusepath{clip}%
\pgfsetroundcap%
\pgfsetroundjoin%
\pgfsetlinewidth{0.702625pt}%
\definecolor{currentstroke}{rgb}{1.000000,1.000000,1.000000}%
\pgfsetstrokecolor{currentstroke}%
\pgfsetdash{}{0pt}%
\pgfpathmoveto{\pgfqpoint{2.051051in}{0.754477in}}%
\pgfpathlineto{\pgfqpoint{2.051051in}{3.115885in}}%
\pgfusepath{stroke}%
\end{pgfscope}%
\begin{pgfscope}%
\pgfpathrectangle{\pgfqpoint{0.832441in}{0.754477in}}{\pgfqpoint{4.987559in}{2.361409in}}%
\pgfusepath{clip}%
\pgfsetroundcap%
\pgfsetroundjoin%
\pgfsetlinewidth{0.702625pt}%
\definecolor{currentstroke}{rgb}{1.000000,1.000000,1.000000}%
\pgfsetstrokecolor{currentstroke}%
\pgfsetdash{}{0pt}%
\pgfpathmoveto{\pgfqpoint{2.094932in}{0.754477in}}%
\pgfpathlineto{\pgfqpoint{2.094932in}{3.115885in}}%
\pgfusepath{stroke}%
\end{pgfscope}%
\begin{pgfscope}%
\pgfpathrectangle{\pgfqpoint{0.832441in}{0.754477in}}{\pgfqpoint{4.987559in}{2.361409in}}%
\pgfusepath{clip}%
\pgfsetroundcap%
\pgfsetroundjoin%
\pgfsetlinewidth{0.702625pt}%
\definecolor{currentstroke}{rgb}{1.000000,1.000000,1.000000}%
\pgfsetstrokecolor{currentstroke}%
\pgfsetdash{}{0pt}%
\pgfpathmoveto{\pgfqpoint{2.132032in}{0.754477in}}%
\pgfpathlineto{\pgfqpoint{2.132032in}{3.115885in}}%
\pgfusepath{stroke}%
\end{pgfscope}%
\begin{pgfscope}%
\pgfpathrectangle{\pgfqpoint{0.832441in}{0.754477in}}{\pgfqpoint{4.987559in}{2.361409in}}%
\pgfusepath{clip}%
\pgfsetroundcap%
\pgfsetroundjoin%
\pgfsetlinewidth{0.702625pt}%
\definecolor{currentstroke}{rgb}{1.000000,1.000000,1.000000}%
\pgfsetstrokecolor{currentstroke}%
\pgfsetdash{}{0pt}%
\pgfpathmoveto{\pgfqpoint{2.164169in}{0.754477in}}%
\pgfpathlineto{\pgfqpoint{2.164169in}{3.115885in}}%
\pgfusepath{stroke}%
\end{pgfscope}%
\begin{pgfscope}%
\pgfpathrectangle{\pgfqpoint{0.832441in}{0.754477in}}{\pgfqpoint{4.987559in}{2.361409in}}%
\pgfusepath{clip}%
\pgfsetroundcap%
\pgfsetroundjoin%
\pgfsetlinewidth{0.702625pt}%
\definecolor{currentstroke}{rgb}{1.000000,1.000000,1.000000}%
\pgfsetstrokecolor{currentstroke}%
\pgfsetdash{}{0pt}%
\pgfpathmoveto{\pgfqpoint{2.192517in}{0.754477in}}%
\pgfpathlineto{\pgfqpoint{2.192517in}{3.115885in}}%
\pgfusepath{stroke}%
\end{pgfscope}%
\begin{pgfscope}%
\pgfpathrectangle{\pgfqpoint{0.832441in}{0.754477in}}{\pgfqpoint{4.987559in}{2.361409in}}%
\pgfusepath{clip}%
\pgfsetroundcap%
\pgfsetroundjoin%
\pgfsetlinewidth{0.702625pt}%
\definecolor{currentstroke}{rgb}{1.000000,1.000000,1.000000}%
\pgfsetstrokecolor{currentstroke}%
\pgfsetdash{}{0pt}%
\pgfpathmoveto{\pgfqpoint{2.384697in}{0.754477in}}%
\pgfpathlineto{\pgfqpoint{2.384697in}{3.115885in}}%
\pgfusepath{stroke}%
\end{pgfscope}%
\begin{pgfscope}%
\pgfpathrectangle{\pgfqpoint{0.832441in}{0.754477in}}{\pgfqpoint{4.987559in}{2.361409in}}%
\pgfusepath{clip}%
\pgfsetroundcap%
\pgfsetroundjoin%
\pgfsetlinewidth{0.702625pt}%
\definecolor{currentstroke}{rgb}{1.000000,1.000000,1.000000}%
\pgfsetstrokecolor{currentstroke}%
\pgfsetdash{}{0pt}%
\pgfpathmoveto{\pgfqpoint{2.482282in}{0.754477in}}%
\pgfpathlineto{\pgfqpoint{2.482282in}{3.115885in}}%
\pgfusepath{stroke}%
\end{pgfscope}%
\begin{pgfscope}%
\pgfpathrectangle{\pgfqpoint{0.832441in}{0.754477in}}{\pgfqpoint{4.987559in}{2.361409in}}%
\pgfusepath{clip}%
\pgfsetroundcap%
\pgfsetroundjoin%
\pgfsetlinewidth{0.702625pt}%
\definecolor{currentstroke}{rgb}{1.000000,1.000000,1.000000}%
\pgfsetstrokecolor{currentstroke}%
\pgfsetdash{}{0pt}%
\pgfpathmoveto{\pgfqpoint{2.551520in}{0.754477in}}%
\pgfpathlineto{\pgfqpoint{2.551520in}{3.115885in}}%
\pgfusepath{stroke}%
\end{pgfscope}%
\begin{pgfscope}%
\pgfpathrectangle{\pgfqpoint{0.832441in}{0.754477in}}{\pgfqpoint{4.987559in}{2.361409in}}%
\pgfusepath{clip}%
\pgfsetroundcap%
\pgfsetroundjoin%
\pgfsetlinewidth{0.702625pt}%
\definecolor{currentstroke}{rgb}{1.000000,1.000000,1.000000}%
\pgfsetstrokecolor{currentstroke}%
\pgfsetdash{}{0pt}%
\pgfpathmoveto{\pgfqpoint{2.605225in}{0.754477in}}%
\pgfpathlineto{\pgfqpoint{2.605225in}{3.115885in}}%
\pgfusepath{stroke}%
\end{pgfscope}%
\begin{pgfscope}%
\pgfpathrectangle{\pgfqpoint{0.832441in}{0.754477in}}{\pgfqpoint{4.987559in}{2.361409in}}%
\pgfusepath{clip}%
\pgfsetroundcap%
\pgfsetroundjoin%
\pgfsetlinewidth{0.702625pt}%
\definecolor{currentstroke}{rgb}{1.000000,1.000000,1.000000}%
\pgfsetstrokecolor{currentstroke}%
\pgfsetdash{}{0pt}%
\pgfpathmoveto{\pgfqpoint{2.649105in}{0.754477in}}%
\pgfpathlineto{\pgfqpoint{2.649105in}{3.115885in}}%
\pgfusepath{stroke}%
\end{pgfscope}%
\begin{pgfscope}%
\pgfpathrectangle{\pgfqpoint{0.832441in}{0.754477in}}{\pgfqpoint{4.987559in}{2.361409in}}%
\pgfusepath{clip}%
\pgfsetroundcap%
\pgfsetroundjoin%
\pgfsetlinewidth{0.702625pt}%
\definecolor{currentstroke}{rgb}{1.000000,1.000000,1.000000}%
\pgfsetstrokecolor{currentstroke}%
\pgfsetdash{}{0pt}%
\pgfpathmoveto{\pgfqpoint{2.686205in}{0.754477in}}%
\pgfpathlineto{\pgfqpoint{2.686205in}{3.115885in}}%
\pgfusepath{stroke}%
\end{pgfscope}%
\begin{pgfscope}%
\pgfpathrectangle{\pgfqpoint{0.832441in}{0.754477in}}{\pgfqpoint{4.987559in}{2.361409in}}%
\pgfusepath{clip}%
\pgfsetroundcap%
\pgfsetroundjoin%
\pgfsetlinewidth{0.702625pt}%
\definecolor{currentstroke}{rgb}{1.000000,1.000000,1.000000}%
\pgfsetstrokecolor{currentstroke}%
\pgfsetdash{}{0pt}%
\pgfpathmoveto{\pgfqpoint{2.718342in}{0.754477in}}%
\pgfpathlineto{\pgfqpoint{2.718342in}{3.115885in}}%
\pgfusepath{stroke}%
\end{pgfscope}%
\begin{pgfscope}%
\pgfpathrectangle{\pgfqpoint{0.832441in}{0.754477in}}{\pgfqpoint{4.987559in}{2.361409in}}%
\pgfusepath{clip}%
\pgfsetroundcap%
\pgfsetroundjoin%
\pgfsetlinewidth{0.702625pt}%
\definecolor{currentstroke}{rgb}{1.000000,1.000000,1.000000}%
\pgfsetstrokecolor{currentstroke}%
\pgfsetdash{}{0pt}%
\pgfpathmoveto{\pgfqpoint{2.746690in}{0.754477in}}%
\pgfpathlineto{\pgfqpoint{2.746690in}{3.115885in}}%
\pgfusepath{stroke}%
\end{pgfscope}%
\begin{pgfscope}%
\pgfpathrectangle{\pgfqpoint{0.832441in}{0.754477in}}{\pgfqpoint{4.987559in}{2.361409in}}%
\pgfusepath{clip}%
\pgfsetroundcap%
\pgfsetroundjoin%
\pgfsetlinewidth{0.702625pt}%
\definecolor{currentstroke}{rgb}{1.000000,1.000000,1.000000}%
\pgfsetstrokecolor{currentstroke}%
\pgfsetdash{}{0pt}%
\pgfpathmoveto{\pgfqpoint{2.938870in}{0.754477in}}%
\pgfpathlineto{\pgfqpoint{2.938870in}{3.115885in}}%
\pgfusepath{stroke}%
\end{pgfscope}%
\begin{pgfscope}%
\pgfpathrectangle{\pgfqpoint{0.832441in}{0.754477in}}{\pgfqpoint{4.987559in}{2.361409in}}%
\pgfusepath{clip}%
\pgfsetroundcap%
\pgfsetroundjoin%
\pgfsetlinewidth{0.702625pt}%
\definecolor{currentstroke}{rgb}{1.000000,1.000000,1.000000}%
\pgfsetstrokecolor{currentstroke}%
\pgfsetdash{}{0pt}%
\pgfpathmoveto{\pgfqpoint{3.036455in}{0.754477in}}%
\pgfpathlineto{\pgfqpoint{3.036455in}{3.115885in}}%
\pgfusepath{stroke}%
\end{pgfscope}%
\begin{pgfscope}%
\pgfpathrectangle{\pgfqpoint{0.832441in}{0.754477in}}{\pgfqpoint{4.987559in}{2.361409in}}%
\pgfusepath{clip}%
\pgfsetroundcap%
\pgfsetroundjoin%
\pgfsetlinewidth{0.702625pt}%
\definecolor{currentstroke}{rgb}{1.000000,1.000000,1.000000}%
\pgfsetstrokecolor{currentstroke}%
\pgfsetdash{}{0pt}%
\pgfpathmoveto{\pgfqpoint{3.105693in}{0.754477in}}%
\pgfpathlineto{\pgfqpoint{3.105693in}{3.115885in}}%
\pgfusepath{stroke}%
\end{pgfscope}%
\begin{pgfscope}%
\pgfpathrectangle{\pgfqpoint{0.832441in}{0.754477in}}{\pgfqpoint{4.987559in}{2.361409in}}%
\pgfusepath{clip}%
\pgfsetroundcap%
\pgfsetroundjoin%
\pgfsetlinewidth{0.702625pt}%
\definecolor{currentstroke}{rgb}{1.000000,1.000000,1.000000}%
\pgfsetstrokecolor{currentstroke}%
\pgfsetdash{}{0pt}%
\pgfpathmoveto{\pgfqpoint{3.159398in}{0.754477in}}%
\pgfpathlineto{\pgfqpoint{3.159398in}{3.115885in}}%
\pgfusepath{stroke}%
\end{pgfscope}%
\begin{pgfscope}%
\pgfpathrectangle{\pgfqpoint{0.832441in}{0.754477in}}{\pgfqpoint{4.987559in}{2.361409in}}%
\pgfusepath{clip}%
\pgfsetroundcap%
\pgfsetroundjoin%
\pgfsetlinewidth{0.702625pt}%
\definecolor{currentstroke}{rgb}{1.000000,1.000000,1.000000}%
\pgfsetstrokecolor{currentstroke}%
\pgfsetdash{}{0pt}%
\pgfpathmoveto{\pgfqpoint{3.203278in}{0.754477in}}%
\pgfpathlineto{\pgfqpoint{3.203278in}{3.115885in}}%
\pgfusepath{stroke}%
\end{pgfscope}%
\begin{pgfscope}%
\pgfpathrectangle{\pgfqpoint{0.832441in}{0.754477in}}{\pgfqpoint{4.987559in}{2.361409in}}%
\pgfusepath{clip}%
\pgfsetroundcap%
\pgfsetroundjoin%
\pgfsetlinewidth{0.702625pt}%
\definecolor{currentstroke}{rgb}{1.000000,1.000000,1.000000}%
\pgfsetstrokecolor{currentstroke}%
\pgfsetdash{}{0pt}%
\pgfpathmoveto{\pgfqpoint{3.240378in}{0.754477in}}%
\pgfpathlineto{\pgfqpoint{3.240378in}{3.115885in}}%
\pgfusepath{stroke}%
\end{pgfscope}%
\begin{pgfscope}%
\pgfpathrectangle{\pgfqpoint{0.832441in}{0.754477in}}{\pgfqpoint{4.987559in}{2.361409in}}%
\pgfusepath{clip}%
\pgfsetroundcap%
\pgfsetroundjoin%
\pgfsetlinewidth{0.702625pt}%
\definecolor{currentstroke}{rgb}{1.000000,1.000000,1.000000}%
\pgfsetstrokecolor{currentstroke}%
\pgfsetdash{}{0pt}%
\pgfpathmoveto{\pgfqpoint{3.272516in}{0.754477in}}%
\pgfpathlineto{\pgfqpoint{3.272516in}{3.115885in}}%
\pgfusepath{stroke}%
\end{pgfscope}%
\begin{pgfscope}%
\pgfpathrectangle{\pgfqpoint{0.832441in}{0.754477in}}{\pgfqpoint{4.987559in}{2.361409in}}%
\pgfusepath{clip}%
\pgfsetroundcap%
\pgfsetroundjoin%
\pgfsetlinewidth{0.702625pt}%
\definecolor{currentstroke}{rgb}{1.000000,1.000000,1.000000}%
\pgfsetstrokecolor{currentstroke}%
\pgfsetdash{}{0pt}%
\pgfpathmoveto{\pgfqpoint{3.300863in}{0.754477in}}%
\pgfpathlineto{\pgfqpoint{3.300863in}{3.115885in}}%
\pgfusepath{stroke}%
\end{pgfscope}%
\begin{pgfscope}%
\pgfpathrectangle{\pgfqpoint{0.832441in}{0.754477in}}{\pgfqpoint{4.987559in}{2.361409in}}%
\pgfusepath{clip}%
\pgfsetroundcap%
\pgfsetroundjoin%
\pgfsetlinewidth{0.702625pt}%
\definecolor{currentstroke}{rgb}{1.000000,1.000000,1.000000}%
\pgfsetstrokecolor{currentstroke}%
\pgfsetdash{}{0pt}%
\pgfpathmoveto{\pgfqpoint{3.493043in}{0.754477in}}%
\pgfpathlineto{\pgfqpoint{3.493043in}{3.115885in}}%
\pgfusepath{stroke}%
\end{pgfscope}%
\begin{pgfscope}%
\pgfpathrectangle{\pgfqpoint{0.832441in}{0.754477in}}{\pgfqpoint{4.987559in}{2.361409in}}%
\pgfusepath{clip}%
\pgfsetroundcap%
\pgfsetroundjoin%
\pgfsetlinewidth{0.702625pt}%
\definecolor{currentstroke}{rgb}{1.000000,1.000000,1.000000}%
\pgfsetstrokecolor{currentstroke}%
\pgfsetdash{}{0pt}%
\pgfpathmoveto{\pgfqpoint{3.590628in}{0.754477in}}%
\pgfpathlineto{\pgfqpoint{3.590628in}{3.115885in}}%
\pgfusepath{stroke}%
\end{pgfscope}%
\begin{pgfscope}%
\pgfpathrectangle{\pgfqpoint{0.832441in}{0.754477in}}{\pgfqpoint{4.987559in}{2.361409in}}%
\pgfusepath{clip}%
\pgfsetroundcap%
\pgfsetroundjoin%
\pgfsetlinewidth{0.702625pt}%
\definecolor{currentstroke}{rgb}{1.000000,1.000000,1.000000}%
\pgfsetstrokecolor{currentstroke}%
\pgfsetdash{}{0pt}%
\pgfpathmoveto{\pgfqpoint{3.659866in}{0.754477in}}%
\pgfpathlineto{\pgfqpoint{3.659866in}{3.115885in}}%
\pgfusepath{stroke}%
\end{pgfscope}%
\begin{pgfscope}%
\pgfpathrectangle{\pgfqpoint{0.832441in}{0.754477in}}{\pgfqpoint{4.987559in}{2.361409in}}%
\pgfusepath{clip}%
\pgfsetroundcap%
\pgfsetroundjoin%
\pgfsetlinewidth{0.702625pt}%
\definecolor{currentstroke}{rgb}{1.000000,1.000000,1.000000}%
\pgfsetstrokecolor{currentstroke}%
\pgfsetdash{}{0pt}%
\pgfpathmoveto{\pgfqpoint{3.713571in}{0.754477in}}%
\pgfpathlineto{\pgfqpoint{3.713571in}{3.115885in}}%
\pgfusepath{stroke}%
\end{pgfscope}%
\begin{pgfscope}%
\pgfpathrectangle{\pgfqpoint{0.832441in}{0.754477in}}{\pgfqpoint{4.987559in}{2.361409in}}%
\pgfusepath{clip}%
\pgfsetroundcap%
\pgfsetroundjoin%
\pgfsetlinewidth{0.702625pt}%
\definecolor{currentstroke}{rgb}{1.000000,1.000000,1.000000}%
\pgfsetstrokecolor{currentstroke}%
\pgfsetdash{}{0pt}%
\pgfpathmoveto{\pgfqpoint{3.757451in}{0.754477in}}%
\pgfpathlineto{\pgfqpoint{3.757451in}{3.115885in}}%
\pgfusepath{stroke}%
\end{pgfscope}%
\begin{pgfscope}%
\pgfpathrectangle{\pgfqpoint{0.832441in}{0.754477in}}{\pgfqpoint{4.987559in}{2.361409in}}%
\pgfusepath{clip}%
\pgfsetroundcap%
\pgfsetroundjoin%
\pgfsetlinewidth{0.702625pt}%
\definecolor{currentstroke}{rgb}{1.000000,1.000000,1.000000}%
\pgfsetstrokecolor{currentstroke}%
\pgfsetdash{}{0pt}%
\pgfpathmoveto{\pgfqpoint{3.794551in}{0.754477in}}%
\pgfpathlineto{\pgfqpoint{3.794551in}{3.115885in}}%
\pgfusepath{stroke}%
\end{pgfscope}%
\begin{pgfscope}%
\pgfpathrectangle{\pgfqpoint{0.832441in}{0.754477in}}{\pgfqpoint{4.987559in}{2.361409in}}%
\pgfusepath{clip}%
\pgfsetroundcap%
\pgfsetroundjoin%
\pgfsetlinewidth{0.702625pt}%
\definecolor{currentstroke}{rgb}{1.000000,1.000000,1.000000}%
\pgfsetstrokecolor{currentstroke}%
\pgfsetdash{}{0pt}%
\pgfpathmoveto{\pgfqpoint{3.826689in}{0.754477in}}%
\pgfpathlineto{\pgfqpoint{3.826689in}{3.115885in}}%
\pgfusepath{stroke}%
\end{pgfscope}%
\begin{pgfscope}%
\pgfpathrectangle{\pgfqpoint{0.832441in}{0.754477in}}{\pgfqpoint{4.987559in}{2.361409in}}%
\pgfusepath{clip}%
\pgfsetroundcap%
\pgfsetroundjoin%
\pgfsetlinewidth{0.702625pt}%
\definecolor{currentstroke}{rgb}{1.000000,1.000000,1.000000}%
\pgfsetstrokecolor{currentstroke}%
\pgfsetdash{}{0pt}%
\pgfpathmoveto{\pgfqpoint{3.855036in}{0.754477in}}%
\pgfpathlineto{\pgfqpoint{3.855036in}{3.115885in}}%
\pgfusepath{stroke}%
\end{pgfscope}%
\begin{pgfscope}%
\pgfpathrectangle{\pgfqpoint{0.832441in}{0.754477in}}{\pgfqpoint{4.987559in}{2.361409in}}%
\pgfusepath{clip}%
\pgfsetroundcap%
\pgfsetroundjoin%
\pgfsetlinewidth{0.702625pt}%
\definecolor{currentstroke}{rgb}{1.000000,1.000000,1.000000}%
\pgfsetstrokecolor{currentstroke}%
\pgfsetdash{}{0pt}%
\pgfpathmoveto{\pgfqpoint{4.047217in}{0.754477in}}%
\pgfpathlineto{\pgfqpoint{4.047217in}{3.115885in}}%
\pgfusepath{stroke}%
\end{pgfscope}%
\begin{pgfscope}%
\pgfpathrectangle{\pgfqpoint{0.832441in}{0.754477in}}{\pgfqpoint{4.987559in}{2.361409in}}%
\pgfusepath{clip}%
\pgfsetroundcap%
\pgfsetroundjoin%
\pgfsetlinewidth{0.702625pt}%
\definecolor{currentstroke}{rgb}{1.000000,1.000000,1.000000}%
\pgfsetstrokecolor{currentstroke}%
\pgfsetdash{}{0pt}%
\pgfpathmoveto{\pgfqpoint{4.144802in}{0.754477in}}%
\pgfpathlineto{\pgfqpoint{4.144802in}{3.115885in}}%
\pgfusepath{stroke}%
\end{pgfscope}%
\begin{pgfscope}%
\pgfpathrectangle{\pgfqpoint{0.832441in}{0.754477in}}{\pgfqpoint{4.987559in}{2.361409in}}%
\pgfusepath{clip}%
\pgfsetroundcap%
\pgfsetroundjoin%
\pgfsetlinewidth{0.702625pt}%
\definecolor{currentstroke}{rgb}{1.000000,1.000000,1.000000}%
\pgfsetstrokecolor{currentstroke}%
\pgfsetdash{}{0pt}%
\pgfpathmoveto{\pgfqpoint{4.214039in}{0.754477in}}%
\pgfpathlineto{\pgfqpoint{4.214039in}{3.115885in}}%
\pgfusepath{stroke}%
\end{pgfscope}%
\begin{pgfscope}%
\pgfpathrectangle{\pgfqpoint{0.832441in}{0.754477in}}{\pgfqpoint{4.987559in}{2.361409in}}%
\pgfusepath{clip}%
\pgfsetroundcap%
\pgfsetroundjoin%
\pgfsetlinewidth{0.702625pt}%
\definecolor{currentstroke}{rgb}{1.000000,1.000000,1.000000}%
\pgfsetstrokecolor{currentstroke}%
\pgfsetdash{}{0pt}%
\pgfpathmoveto{\pgfqpoint{4.267744in}{0.754477in}}%
\pgfpathlineto{\pgfqpoint{4.267744in}{3.115885in}}%
\pgfusepath{stroke}%
\end{pgfscope}%
\begin{pgfscope}%
\pgfpathrectangle{\pgfqpoint{0.832441in}{0.754477in}}{\pgfqpoint{4.987559in}{2.361409in}}%
\pgfusepath{clip}%
\pgfsetroundcap%
\pgfsetroundjoin%
\pgfsetlinewidth{0.702625pt}%
\definecolor{currentstroke}{rgb}{1.000000,1.000000,1.000000}%
\pgfsetstrokecolor{currentstroke}%
\pgfsetdash{}{0pt}%
\pgfpathmoveto{\pgfqpoint{4.311624in}{0.754477in}}%
\pgfpathlineto{\pgfqpoint{4.311624in}{3.115885in}}%
\pgfusepath{stroke}%
\end{pgfscope}%
\begin{pgfscope}%
\pgfpathrectangle{\pgfqpoint{0.832441in}{0.754477in}}{\pgfqpoint{4.987559in}{2.361409in}}%
\pgfusepath{clip}%
\pgfsetroundcap%
\pgfsetroundjoin%
\pgfsetlinewidth{0.702625pt}%
\definecolor{currentstroke}{rgb}{1.000000,1.000000,1.000000}%
\pgfsetstrokecolor{currentstroke}%
\pgfsetdash{}{0pt}%
\pgfpathmoveto{\pgfqpoint{4.348724in}{0.754477in}}%
\pgfpathlineto{\pgfqpoint{4.348724in}{3.115885in}}%
\pgfusepath{stroke}%
\end{pgfscope}%
\begin{pgfscope}%
\pgfpathrectangle{\pgfqpoint{0.832441in}{0.754477in}}{\pgfqpoint{4.987559in}{2.361409in}}%
\pgfusepath{clip}%
\pgfsetroundcap%
\pgfsetroundjoin%
\pgfsetlinewidth{0.702625pt}%
\definecolor{currentstroke}{rgb}{1.000000,1.000000,1.000000}%
\pgfsetstrokecolor{currentstroke}%
\pgfsetdash{}{0pt}%
\pgfpathmoveto{\pgfqpoint{4.380862in}{0.754477in}}%
\pgfpathlineto{\pgfqpoint{4.380862in}{3.115885in}}%
\pgfusepath{stroke}%
\end{pgfscope}%
\begin{pgfscope}%
\pgfpathrectangle{\pgfqpoint{0.832441in}{0.754477in}}{\pgfqpoint{4.987559in}{2.361409in}}%
\pgfusepath{clip}%
\pgfsetroundcap%
\pgfsetroundjoin%
\pgfsetlinewidth{0.702625pt}%
\definecolor{currentstroke}{rgb}{1.000000,1.000000,1.000000}%
\pgfsetstrokecolor{currentstroke}%
\pgfsetdash{}{0pt}%
\pgfpathmoveto{\pgfqpoint{4.409209in}{0.754477in}}%
\pgfpathlineto{\pgfqpoint{4.409209in}{3.115885in}}%
\pgfusepath{stroke}%
\end{pgfscope}%
\begin{pgfscope}%
\pgfpathrectangle{\pgfqpoint{0.832441in}{0.754477in}}{\pgfqpoint{4.987559in}{2.361409in}}%
\pgfusepath{clip}%
\pgfsetroundcap%
\pgfsetroundjoin%
\pgfsetlinewidth{0.702625pt}%
\definecolor{currentstroke}{rgb}{1.000000,1.000000,1.000000}%
\pgfsetstrokecolor{currentstroke}%
\pgfsetdash{}{0pt}%
\pgfpathmoveto{\pgfqpoint{4.601390in}{0.754477in}}%
\pgfpathlineto{\pgfqpoint{4.601390in}{3.115885in}}%
\pgfusepath{stroke}%
\end{pgfscope}%
\begin{pgfscope}%
\pgfpathrectangle{\pgfqpoint{0.832441in}{0.754477in}}{\pgfqpoint{4.987559in}{2.361409in}}%
\pgfusepath{clip}%
\pgfsetroundcap%
\pgfsetroundjoin%
\pgfsetlinewidth{0.702625pt}%
\definecolor{currentstroke}{rgb}{1.000000,1.000000,1.000000}%
\pgfsetstrokecolor{currentstroke}%
\pgfsetdash{}{0pt}%
\pgfpathmoveto{\pgfqpoint{4.698975in}{0.754477in}}%
\pgfpathlineto{\pgfqpoint{4.698975in}{3.115885in}}%
\pgfusepath{stroke}%
\end{pgfscope}%
\begin{pgfscope}%
\pgfpathrectangle{\pgfqpoint{0.832441in}{0.754477in}}{\pgfqpoint{4.987559in}{2.361409in}}%
\pgfusepath{clip}%
\pgfsetroundcap%
\pgfsetroundjoin%
\pgfsetlinewidth{0.702625pt}%
\definecolor{currentstroke}{rgb}{1.000000,1.000000,1.000000}%
\pgfsetstrokecolor{currentstroke}%
\pgfsetdash{}{0pt}%
\pgfpathmoveto{\pgfqpoint{4.768213in}{0.754477in}}%
\pgfpathlineto{\pgfqpoint{4.768213in}{3.115885in}}%
\pgfusepath{stroke}%
\end{pgfscope}%
\begin{pgfscope}%
\pgfpathrectangle{\pgfqpoint{0.832441in}{0.754477in}}{\pgfqpoint{4.987559in}{2.361409in}}%
\pgfusepath{clip}%
\pgfsetroundcap%
\pgfsetroundjoin%
\pgfsetlinewidth{0.702625pt}%
\definecolor{currentstroke}{rgb}{1.000000,1.000000,1.000000}%
\pgfsetstrokecolor{currentstroke}%
\pgfsetdash{}{0pt}%
\pgfpathmoveto{\pgfqpoint{4.821917in}{0.754477in}}%
\pgfpathlineto{\pgfqpoint{4.821917in}{3.115885in}}%
\pgfusepath{stroke}%
\end{pgfscope}%
\begin{pgfscope}%
\pgfpathrectangle{\pgfqpoint{0.832441in}{0.754477in}}{\pgfqpoint{4.987559in}{2.361409in}}%
\pgfusepath{clip}%
\pgfsetroundcap%
\pgfsetroundjoin%
\pgfsetlinewidth{0.702625pt}%
\definecolor{currentstroke}{rgb}{1.000000,1.000000,1.000000}%
\pgfsetstrokecolor{currentstroke}%
\pgfsetdash{}{0pt}%
\pgfpathmoveto{\pgfqpoint{4.865798in}{0.754477in}}%
\pgfpathlineto{\pgfqpoint{4.865798in}{3.115885in}}%
\pgfusepath{stroke}%
\end{pgfscope}%
\begin{pgfscope}%
\pgfpathrectangle{\pgfqpoint{0.832441in}{0.754477in}}{\pgfqpoint{4.987559in}{2.361409in}}%
\pgfusepath{clip}%
\pgfsetroundcap%
\pgfsetroundjoin%
\pgfsetlinewidth{0.702625pt}%
\definecolor{currentstroke}{rgb}{1.000000,1.000000,1.000000}%
\pgfsetstrokecolor{currentstroke}%
\pgfsetdash{}{0pt}%
\pgfpathmoveto{\pgfqpoint{4.902898in}{0.754477in}}%
\pgfpathlineto{\pgfqpoint{4.902898in}{3.115885in}}%
\pgfusepath{stroke}%
\end{pgfscope}%
\begin{pgfscope}%
\pgfpathrectangle{\pgfqpoint{0.832441in}{0.754477in}}{\pgfqpoint{4.987559in}{2.361409in}}%
\pgfusepath{clip}%
\pgfsetroundcap%
\pgfsetroundjoin%
\pgfsetlinewidth{0.702625pt}%
\definecolor{currentstroke}{rgb}{1.000000,1.000000,1.000000}%
\pgfsetstrokecolor{currentstroke}%
\pgfsetdash{}{0pt}%
\pgfpathmoveto{\pgfqpoint{4.935035in}{0.754477in}}%
\pgfpathlineto{\pgfqpoint{4.935035in}{3.115885in}}%
\pgfusepath{stroke}%
\end{pgfscope}%
\begin{pgfscope}%
\pgfpathrectangle{\pgfqpoint{0.832441in}{0.754477in}}{\pgfqpoint{4.987559in}{2.361409in}}%
\pgfusepath{clip}%
\pgfsetroundcap%
\pgfsetroundjoin%
\pgfsetlinewidth{0.702625pt}%
\definecolor{currentstroke}{rgb}{1.000000,1.000000,1.000000}%
\pgfsetstrokecolor{currentstroke}%
\pgfsetdash{}{0pt}%
\pgfpathmoveto{\pgfqpoint{4.963383in}{0.754477in}}%
\pgfpathlineto{\pgfqpoint{4.963383in}{3.115885in}}%
\pgfusepath{stroke}%
\end{pgfscope}%
\begin{pgfscope}%
\pgfpathrectangle{\pgfqpoint{0.832441in}{0.754477in}}{\pgfqpoint{4.987559in}{2.361409in}}%
\pgfusepath{clip}%
\pgfsetroundcap%
\pgfsetroundjoin%
\pgfsetlinewidth{0.702625pt}%
\definecolor{currentstroke}{rgb}{1.000000,1.000000,1.000000}%
\pgfsetstrokecolor{currentstroke}%
\pgfsetdash{}{0pt}%
\pgfpathmoveto{\pgfqpoint{5.155563in}{0.754477in}}%
\pgfpathlineto{\pgfqpoint{5.155563in}{3.115885in}}%
\pgfusepath{stroke}%
\end{pgfscope}%
\begin{pgfscope}%
\pgfpathrectangle{\pgfqpoint{0.832441in}{0.754477in}}{\pgfqpoint{4.987559in}{2.361409in}}%
\pgfusepath{clip}%
\pgfsetroundcap%
\pgfsetroundjoin%
\pgfsetlinewidth{0.702625pt}%
\definecolor{currentstroke}{rgb}{1.000000,1.000000,1.000000}%
\pgfsetstrokecolor{currentstroke}%
\pgfsetdash{}{0pt}%
\pgfpathmoveto{\pgfqpoint{5.253148in}{0.754477in}}%
\pgfpathlineto{\pgfqpoint{5.253148in}{3.115885in}}%
\pgfusepath{stroke}%
\end{pgfscope}%
\begin{pgfscope}%
\pgfpathrectangle{\pgfqpoint{0.832441in}{0.754477in}}{\pgfqpoint{4.987559in}{2.361409in}}%
\pgfusepath{clip}%
\pgfsetroundcap%
\pgfsetroundjoin%
\pgfsetlinewidth{0.702625pt}%
\definecolor{currentstroke}{rgb}{1.000000,1.000000,1.000000}%
\pgfsetstrokecolor{currentstroke}%
\pgfsetdash{}{0pt}%
\pgfpathmoveto{\pgfqpoint{5.322386in}{0.754477in}}%
\pgfpathlineto{\pgfqpoint{5.322386in}{3.115885in}}%
\pgfusepath{stroke}%
\end{pgfscope}%
\begin{pgfscope}%
\pgfpathrectangle{\pgfqpoint{0.832441in}{0.754477in}}{\pgfqpoint{4.987559in}{2.361409in}}%
\pgfusepath{clip}%
\pgfsetroundcap%
\pgfsetroundjoin%
\pgfsetlinewidth{0.702625pt}%
\definecolor{currentstroke}{rgb}{1.000000,1.000000,1.000000}%
\pgfsetstrokecolor{currentstroke}%
\pgfsetdash{}{0pt}%
\pgfpathmoveto{\pgfqpoint{5.376091in}{0.754477in}}%
\pgfpathlineto{\pgfqpoint{5.376091in}{3.115885in}}%
\pgfusepath{stroke}%
\end{pgfscope}%
\begin{pgfscope}%
\pgfpathrectangle{\pgfqpoint{0.832441in}{0.754477in}}{\pgfqpoint{4.987559in}{2.361409in}}%
\pgfusepath{clip}%
\pgfsetroundcap%
\pgfsetroundjoin%
\pgfsetlinewidth{0.702625pt}%
\definecolor{currentstroke}{rgb}{1.000000,1.000000,1.000000}%
\pgfsetstrokecolor{currentstroke}%
\pgfsetdash{}{0pt}%
\pgfpathmoveto{\pgfqpoint{5.419971in}{0.754477in}}%
\pgfpathlineto{\pgfqpoint{5.419971in}{3.115885in}}%
\pgfusepath{stroke}%
\end{pgfscope}%
\begin{pgfscope}%
\pgfpathrectangle{\pgfqpoint{0.832441in}{0.754477in}}{\pgfqpoint{4.987559in}{2.361409in}}%
\pgfusepath{clip}%
\pgfsetroundcap%
\pgfsetroundjoin%
\pgfsetlinewidth{0.702625pt}%
\definecolor{currentstroke}{rgb}{1.000000,1.000000,1.000000}%
\pgfsetstrokecolor{currentstroke}%
\pgfsetdash{}{0pt}%
\pgfpathmoveto{\pgfqpoint{5.457071in}{0.754477in}}%
\pgfpathlineto{\pgfqpoint{5.457071in}{3.115885in}}%
\pgfusepath{stroke}%
\end{pgfscope}%
\begin{pgfscope}%
\pgfpathrectangle{\pgfqpoint{0.832441in}{0.754477in}}{\pgfqpoint{4.987559in}{2.361409in}}%
\pgfusepath{clip}%
\pgfsetroundcap%
\pgfsetroundjoin%
\pgfsetlinewidth{0.702625pt}%
\definecolor{currentstroke}{rgb}{1.000000,1.000000,1.000000}%
\pgfsetstrokecolor{currentstroke}%
\pgfsetdash{}{0pt}%
\pgfpathmoveto{\pgfqpoint{5.489208in}{0.754477in}}%
\pgfpathlineto{\pgfqpoint{5.489208in}{3.115885in}}%
\pgfusepath{stroke}%
\end{pgfscope}%
\begin{pgfscope}%
\pgfpathrectangle{\pgfqpoint{0.832441in}{0.754477in}}{\pgfqpoint{4.987559in}{2.361409in}}%
\pgfusepath{clip}%
\pgfsetroundcap%
\pgfsetroundjoin%
\pgfsetlinewidth{0.702625pt}%
\definecolor{currentstroke}{rgb}{1.000000,1.000000,1.000000}%
\pgfsetstrokecolor{currentstroke}%
\pgfsetdash{}{0pt}%
\pgfpathmoveto{\pgfqpoint{5.517556in}{0.754477in}}%
\pgfpathlineto{\pgfqpoint{5.517556in}{3.115885in}}%
\pgfusepath{stroke}%
\end{pgfscope}%
\begin{pgfscope}%
\pgfpathrectangle{\pgfqpoint{0.832441in}{0.754477in}}{\pgfqpoint{4.987559in}{2.361409in}}%
\pgfusepath{clip}%
\pgfsetroundcap%
\pgfsetroundjoin%
\pgfsetlinewidth{0.702625pt}%
\definecolor{currentstroke}{rgb}{1.000000,1.000000,1.000000}%
\pgfsetstrokecolor{currentstroke}%
\pgfsetdash{}{0pt}%
\pgfpathmoveto{\pgfqpoint{5.709736in}{0.754477in}}%
\pgfpathlineto{\pgfqpoint{5.709736in}{3.115885in}}%
\pgfusepath{stroke}%
\end{pgfscope}%
\begin{pgfscope}%
\pgfpathrectangle{\pgfqpoint{0.832441in}{0.754477in}}{\pgfqpoint{4.987559in}{2.361409in}}%
\pgfusepath{clip}%
\pgfsetroundcap%
\pgfsetroundjoin%
\pgfsetlinewidth{0.702625pt}%
\definecolor{currentstroke}{rgb}{1.000000,1.000000,1.000000}%
\pgfsetstrokecolor{currentstroke}%
\pgfsetdash{}{0pt}%
\pgfpathmoveto{\pgfqpoint{5.807321in}{0.754477in}}%
\pgfpathlineto{\pgfqpoint{5.807321in}{3.115885in}}%
\pgfusepath{stroke}%
\end{pgfscope}%
\begin{pgfscope}%
\definecolor{textcolor}{rgb}{0.150000,0.150000,0.150000}%
\pgfsetstrokecolor{textcolor}%
\pgfsetfillcolor{textcolor}%
\pgftext[x=3.326221in,y=0.431310in,,top]{\color{textcolor}\sffamily\fontsize{12.000000}{14.400000}\selectfont Reynolds Number \(\displaystyle \mathrm{Re}_D\)}%
\end{pgfscope}%
\begin{pgfscope}%
\pgfpathrectangle{\pgfqpoint{0.832441in}{0.754477in}}{\pgfqpoint{4.987559in}{2.361409in}}%
\pgfusepath{clip}%
\pgfsetroundcap%
\pgfsetroundjoin%
\pgfsetlinewidth{1.606000pt}%
\definecolor{currentstroke}{rgb}{1.000000,1.000000,1.000000}%
\pgfsetstrokecolor{currentstroke}%
\pgfsetdash{}{0pt}%
\pgfpathmoveto{\pgfqpoint{0.832441in}{0.754477in}}%
\pgfpathlineto{\pgfqpoint{5.820000in}{0.754477in}}%
\pgfusepath{stroke}%
\end{pgfscope}%
\begin{pgfscope}%
\definecolor{textcolor}{rgb}{0.150000,0.150000,0.150000}%
\pgfsetstrokecolor{textcolor}%
\pgfsetfillcolor{textcolor}%
\pgftext[x=0.390618in, y=0.701463in, left, base]{\color{textcolor}\sffamily\fontsize{11.000000}{13.200000}\selectfont \(\displaystyle {10^{-1}}\)}%
\end{pgfscope}%
\begin{pgfscope}%
\pgfpathrectangle{\pgfqpoint{0.832441in}{0.754477in}}{\pgfqpoint{4.987559in}{2.361409in}}%
\pgfusepath{clip}%
\pgfsetroundcap%
\pgfsetroundjoin%
\pgfsetlinewidth{1.606000pt}%
\definecolor{currentstroke}{rgb}{1.000000,1.000000,1.000000}%
\pgfsetstrokecolor{currentstroke}%
\pgfsetdash{}{0pt}%
\pgfpathmoveto{\pgfqpoint{0.832441in}{1.541613in}}%
\pgfpathlineto{\pgfqpoint{5.820000in}{1.541613in}}%
\pgfusepath{stroke}%
\end{pgfscope}%
\begin{pgfscope}%
\definecolor{textcolor}{rgb}{0.150000,0.150000,0.150000}%
\pgfsetstrokecolor{textcolor}%
\pgfsetfillcolor{textcolor}%
\pgftext[x=0.482440in, y=1.488599in, left, base]{\color{textcolor}\sffamily\fontsize{11.000000}{13.200000}\selectfont \(\displaystyle {10^{0}}\)}%
\end{pgfscope}%
\begin{pgfscope}%
\pgfpathrectangle{\pgfqpoint{0.832441in}{0.754477in}}{\pgfqpoint{4.987559in}{2.361409in}}%
\pgfusepath{clip}%
\pgfsetroundcap%
\pgfsetroundjoin%
\pgfsetlinewidth{1.606000pt}%
\definecolor{currentstroke}{rgb}{1.000000,1.000000,1.000000}%
\pgfsetstrokecolor{currentstroke}%
\pgfsetdash{}{0pt}%
\pgfpathmoveto{\pgfqpoint{0.832441in}{2.328749in}}%
\pgfpathlineto{\pgfqpoint{5.820000in}{2.328749in}}%
\pgfusepath{stroke}%
\end{pgfscope}%
\begin{pgfscope}%
\definecolor{textcolor}{rgb}{0.150000,0.150000,0.150000}%
\pgfsetstrokecolor{textcolor}%
\pgfsetfillcolor{textcolor}%
\pgftext[x=0.482440in, y=2.275735in, left, base]{\color{textcolor}\sffamily\fontsize{11.000000}{13.200000}\selectfont \(\displaystyle {10^{1}}\)}%
\end{pgfscope}%
\begin{pgfscope}%
\pgfpathrectangle{\pgfqpoint{0.832441in}{0.754477in}}{\pgfqpoint{4.987559in}{2.361409in}}%
\pgfusepath{clip}%
\pgfsetroundcap%
\pgfsetroundjoin%
\pgfsetlinewidth{1.606000pt}%
\definecolor{currentstroke}{rgb}{1.000000,1.000000,1.000000}%
\pgfsetstrokecolor{currentstroke}%
\pgfsetdash{}{0pt}%
\pgfpathmoveto{\pgfqpoint{0.832441in}{3.115885in}}%
\pgfpathlineto{\pgfqpoint{5.820000in}{3.115885in}}%
\pgfusepath{stroke}%
\end{pgfscope}%
\begin{pgfscope}%
\definecolor{textcolor}{rgb}{0.150000,0.150000,0.150000}%
\pgfsetstrokecolor{textcolor}%
\pgfsetfillcolor{textcolor}%
\pgftext[x=0.482440in, y=3.062872in, left, base]{\color{textcolor}\sffamily\fontsize{11.000000}{13.200000}\selectfont \(\displaystyle {10^{2}}\)}%
\end{pgfscope}%
\begin{pgfscope}%
\pgfpathrectangle{\pgfqpoint{0.832441in}{0.754477in}}{\pgfqpoint{4.987559in}{2.361409in}}%
\pgfusepath{clip}%
\pgfsetroundcap%
\pgfsetroundjoin%
\pgfsetlinewidth{0.702625pt}%
\definecolor{currentstroke}{rgb}{1.000000,1.000000,1.000000}%
\pgfsetstrokecolor{currentstroke}%
\pgfsetdash{}{0pt}%
\pgfpathmoveto{\pgfqpoint{0.832441in}{0.991428in}}%
\pgfpathlineto{\pgfqpoint{5.820000in}{0.991428in}}%
\pgfusepath{stroke}%
\end{pgfscope}%
\begin{pgfscope}%
\pgfpathrectangle{\pgfqpoint{0.832441in}{0.754477in}}{\pgfqpoint{4.987559in}{2.361409in}}%
\pgfusepath{clip}%
\pgfsetroundcap%
\pgfsetroundjoin%
\pgfsetlinewidth{0.702625pt}%
\definecolor{currentstroke}{rgb}{1.000000,1.000000,1.000000}%
\pgfsetstrokecolor{currentstroke}%
\pgfsetdash{}{0pt}%
\pgfpathmoveto{\pgfqpoint{0.832441in}{1.130036in}}%
\pgfpathlineto{\pgfqpoint{5.820000in}{1.130036in}}%
\pgfusepath{stroke}%
\end{pgfscope}%
\begin{pgfscope}%
\pgfpathrectangle{\pgfqpoint{0.832441in}{0.754477in}}{\pgfqpoint{4.987559in}{2.361409in}}%
\pgfusepath{clip}%
\pgfsetroundcap%
\pgfsetroundjoin%
\pgfsetlinewidth{0.702625pt}%
\definecolor{currentstroke}{rgb}{1.000000,1.000000,1.000000}%
\pgfsetstrokecolor{currentstroke}%
\pgfsetdash{}{0pt}%
\pgfpathmoveto{\pgfqpoint{0.832441in}{1.228380in}}%
\pgfpathlineto{\pgfqpoint{5.820000in}{1.228380in}}%
\pgfusepath{stroke}%
\end{pgfscope}%
\begin{pgfscope}%
\pgfpathrectangle{\pgfqpoint{0.832441in}{0.754477in}}{\pgfqpoint{4.987559in}{2.361409in}}%
\pgfusepath{clip}%
\pgfsetroundcap%
\pgfsetroundjoin%
\pgfsetlinewidth{0.702625pt}%
\definecolor{currentstroke}{rgb}{1.000000,1.000000,1.000000}%
\pgfsetstrokecolor{currentstroke}%
\pgfsetdash{}{0pt}%
\pgfpathmoveto{\pgfqpoint{0.832441in}{1.304661in}}%
\pgfpathlineto{\pgfqpoint{5.820000in}{1.304661in}}%
\pgfusepath{stroke}%
\end{pgfscope}%
\begin{pgfscope}%
\pgfpathrectangle{\pgfqpoint{0.832441in}{0.754477in}}{\pgfqpoint{4.987559in}{2.361409in}}%
\pgfusepath{clip}%
\pgfsetroundcap%
\pgfsetroundjoin%
\pgfsetlinewidth{0.702625pt}%
\definecolor{currentstroke}{rgb}{1.000000,1.000000,1.000000}%
\pgfsetstrokecolor{currentstroke}%
\pgfsetdash{}{0pt}%
\pgfpathmoveto{\pgfqpoint{0.832441in}{1.366988in}}%
\pgfpathlineto{\pgfqpoint{5.820000in}{1.366988in}}%
\pgfusepath{stroke}%
\end{pgfscope}%
\begin{pgfscope}%
\pgfpathrectangle{\pgfqpoint{0.832441in}{0.754477in}}{\pgfqpoint{4.987559in}{2.361409in}}%
\pgfusepath{clip}%
\pgfsetroundcap%
\pgfsetroundjoin%
\pgfsetlinewidth{0.702625pt}%
\definecolor{currentstroke}{rgb}{1.000000,1.000000,1.000000}%
\pgfsetstrokecolor{currentstroke}%
\pgfsetdash{}{0pt}%
\pgfpathmoveto{\pgfqpoint{0.832441in}{1.419684in}}%
\pgfpathlineto{\pgfqpoint{5.820000in}{1.419684in}}%
\pgfusepath{stroke}%
\end{pgfscope}%
\begin{pgfscope}%
\pgfpathrectangle{\pgfqpoint{0.832441in}{0.754477in}}{\pgfqpoint{4.987559in}{2.361409in}}%
\pgfusepath{clip}%
\pgfsetroundcap%
\pgfsetroundjoin%
\pgfsetlinewidth{0.702625pt}%
\definecolor{currentstroke}{rgb}{1.000000,1.000000,1.000000}%
\pgfsetstrokecolor{currentstroke}%
\pgfsetdash{}{0pt}%
\pgfpathmoveto{\pgfqpoint{0.832441in}{1.465331in}}%
\pgfpathlineto{\pgfqpoint{5.820000in}{1.465331in}}%
\pgfusepath{stroke}%
\end{pgfscope}%
\begin{pgfscope}%
\pgfpathrectangle{\pgfqpoint{0.832441in}{0.754477in}}{\pgfqpoint{4.987559in}{2.361409in}}%
\pgfusepath{clip}%
\pgfsetroundcap%
\pgfsetroundjoin%
\pgfsetlinewidth{0.702625pt}%
\definecolor{currentstroke}{rgb}{1.000000,1.000000,1.000000}%
\pgfsetstrokecolor{currentstroke}%
\pgfsetdash{}{0pt}%
\pgfpathmoveto{\pgfqpoint{0.832441in}{1.505595in}}%
\pgfpathlineto{\pgfqpoint{5.820000in}{1.505595in}}%
\pgfusepath{stroke}%
\end{pgfscope}%
\begin{pgfscope}%
\pgfpathrectangle{\pgfqpoint{0.832441in}{0.754477in}}{\pgfqpoint{4.987559in}{2.361409in}}%
\pgfusepath{clip}%
\pgfsetroundcap%
\pgfsetroundjoin%
\pgfsetlinewidth{0.702625pt}%
\definecolor{currentstroke}{rgb}{1.000000,1.000000,1.000000}%
\pgfsetstrokecolor{currentstroke}%
\pgfsetdash{}{0pt}%
\pgfpathmoveto{\pgfqpoint{0.832441in}{1.778564in}}%
\pgfpathlineto{\pgfqpoint{5.820000in}{1.778564in}}%
\pgfusepath{stroke}%
\end{pgfscope}%
\begin{pgfscope}%
\pgfpathrectangle{\pgfqpoint{0.832441in}{0.754477in}}{\pgfqpoint{4.987559in}{2.361409in}}%
\pgfusepath{clip}%
\pgfsetroundcap%
\pgfsetroundjoin%
\pgfsetlinewidth{0.702625pt}%
\definecolor{currentstroke}{rgb}{1.000000,1.000000,1.000000}%
\pgfsetstrokecolor{currentstroke}%
\pgfsetdash{}{0pt}%
\pgfpathmoveto{\pgfqpoint{0.832441in}{1.917172in}}%
\pgfpathlineto{\pgfqpoint{5.820000in}{1.917172in}}%
\pgfusepath{stroke}%
\end{pgfscope}%
\begin{pgfscope}%
\pgfpathrectangle{\pgfqpoint{0.832441in}{0.754477in}}{\pgfqpoint{4.987559in}{2.361409in}}%
\pgfusepath{clip}%
\pgfsetroundcap%
\pgfsetroundjoin%
\pgfsetlinewidth{0.702625pt}%
\definecolor{currentstroke}{rgb}{1.000000,1.000000,1.000000}%
\pgfsetstrokecolor{currentstroke}%
\pgfsetdash{}{0pt}%
\pgfpathmoveto{\pgfqpoint{0.832441in}{2.015516in}}%
\pgfpathlineto{\pgfqpoint{5.820000in}{2.015516in}}%
\pgfusepath{stroke}%
\end{pgfscope}%
\begin{pgfscope}%
\pgfpathrectangle{\pgfqpoint{0.832441in}{0.754477in}}{\pgfqpoint{4.987559in}{2.361409in}}%
\pgfusepath{clip}%
\pgfsetroundcap%
\pgfsetroundjoin%
\pgfsetlinewidth{0.702625pt}%
\definecolor{currentstroke}{rgb}{1.000000,1.000000,1.000000}%
\pgfsetstrokecolor{currentstroke}%
\pgfsetdash{}{0pt}%
\pgfpathmoveto{\pgfqpoint{0.832441in}{2.091797in}}%
\pgfpathlineto{\pgfqpoint{5.820000in}{2.091797in}}%
\pgfusepath{stroke}%
\end{pgfscope}%
\begin{pgfscope}%
\pgfpathrectangle{\pgfqpoint{0.832441in}{0.754477in}}{\pgfqpoint{4.987559in}{2.361409in}}%
\pgfusepath{clip}%
\pgfsetroundcap%
\pgfsetroundjoin%
\pgfsetlinewidth{0.702625pt}%
\definecolor{currentstroke}{rgb}{1.000000,1.000000,1.000000}%
\pgfsetstrokecolor{currentstroke}%
\pgfsetdash{}{0pt}%
\pgfpathmoveto{\pgfqpoint{0.832441in}{2.154124in}}%
\pgfpathlineto{\pgfqpoint{5.820000in}{2.154124in}}%
\pgfusepath{stroke}%
\end{pgfscope}%
\begin{pgfscope}%
\pgfpathrectangle{\pgfqpoint{0.832441in}{0.754477in}}{\pgfqpoint{4.987559in}{2.361409in}}%
\pgfusepath{clip}%
\pgfsetroundcap%
\pgfsetroundjoin%
\pgfsetlinewidth{0.702625pt}%
\definecolor{currentstroke}{rgb}{1.000000,1.000000,1.000000}%
\pgfsetstrokecolor{currentstroke}%
\pgfsetdash{}{0pt}%
\pgfpathmoveto{\pgfqpoint{0.832441in}{2.206820in}}%
\pgfpathlineto{\pgfqpoint{5.820000in}{2.206820in}}%
\pgfusepath{stroke}%
\end{pgfscope}%
\begin{pgfscope}%
\pgfpathrectangle{\pgfqpoint{0.832441in}{0.754477in}}{\pgfqpoint{4.987559in}{2.361409in}}%
\pgfusepath{clip}%
\pgfsetroundcap%
\pgfsetroundjoin%
\pgfsetlinewidth{0.702625pt}%
\definecolor{currentstroke}{rgb}{1.000000,1.000000,1.000000}%
\pgfsetstrokecolor{currentstroke}%
\pgfsetdash{}{0pt}%
\pgfpathmoveto{\pgfqpoint{0.832441in}{2.252468in}}%
\pgfpathlineto{\pgfqpoint{5.820000in}{2.252468in}}%
\pgfusepath{stroke}%
\end{pgfscope}%
\begin{pgfscope}%
\pgfpathrectangle{\pgfqpoint{0.832441in}{0.754477in}}{\pgfqpoint{4.987559in}{2.361409in}}%
\pgfusepath{clip}%
\pgfsetroundcap%
\pgfsetroundjoin%
\pgfsetlinewidth{0.702625pt}%
\definecolor{currentstroke}{rgb}{1.000000,1.000000,1.000000}%
\pgfsetstrokecolor{currentstroke}%
\pgfsetdash{}{0pt}%
\pgfpathmoveto{\pgfqpoint{0.832441in}{2.292732in}}%
\pgfpathlineto{\pgfqpoint{5.820000in}{2.292732in}}%
\pgfusepath{stroke}%
\end{pgfscope}%
\begin{pgfscope}%
\pgfpathrectangle{\pgfqpoint{0.832441in}{0.754477in}}{\pgfqpoint{4.987559in}{2.361409in}}%
\pgfusepath{clip}%
\pgfsetroundcap%
\pgfsetroundjoin%
\pgfsetlinewidth{0.702625pt}%
\definecolor{currentstroke}{rgb}{1.000000,1.000000,1.000000}%
\pgfsetstrokecolor{currentstroke}%
\pgfsetdash{}{0pt}%
\pgfpathmoveto{\pgfqpoint{0.832441in}{2.565701in}}%
\pgfpathlineto{\pgfqpoint{5.820000in}{2.565701in}}%
\pgfusepath{stroke}%
\end{pgfscope}%
\begin{pgfscope}%
\pgfpathrectangle{\pgfqpoint{0.832441in}{0.754477in}}{\pgfqpoint{4.987559in}{2.361409in}}%
\pgfusepath{clip}%
\pgfsetroundcap%
\pgfsetroundjoin%
\pgfsetlinewidth{0.702625pt}%
\definecolor{currentstroke}{rgb}{1.000000,1.000000,1.000000}%
\pgfsetstrokecolor{currentstroke}%
\pgfsetdash{}{0pt}%
\pgfpathmoveto{\pgfqpoint{0.832441in}{2.704309in}}%
\pgfpathlineto{\pgfqpoint{5.820000in}{2.704309in}}%
\pgfusepath{stroke}%
\end{pgfscope}%
\begin{pgfscope}%
\pgfpathrectangle{\pgfqpoint{0.832441in}{0.754477in}}{\pgfqpoint{4.987559in}{2.361409in}}%
\pgfusepath{clip}%
\pgfsetroundcap%
\pgfsetroundjoin%
\pgfsetlinewidth{0.702625pt}%
\definecolor{currentstroke}{rgb}{1.000000,1.000000,1.000000}%
\pgfsetstrokecolor{currentstroke}%
\pgfsetdash{}{0pt}%
\pgfpathmoveto{\pgfqpoint{0.832441in}{2.802652in}}%
\pgfpathlineto{\pgfqpoint{5.820000in}{2.802652in}}%
\pgfusepath{stroke}%
\end{pgfscope}%
\begin{pgfscope}%
\pgfpathrectangle{\pgfqpoint{0.832441in}{0.754477in}}{\pgfqpoint{4.987559in}{2.361409in}}%
\pgfusepath{clip}%
\pgfsetroundcap%
\pgfsetroundjoin%
\pgfsetlinewidth{0.702625pt}%
\definecolor{currentstroke}{rgb}{1.000000,1.000000,1.000000}%
\pgfsetstrokecolor{currentstroke}%
\pgfsetdash{}{0pt}%
\pgfpathmoveto{\pgfqpoint{0.832441in}{2.878934in}}%
\pgfpathlineto{\pgfqpoint{5.820000in}{2.878934in}}%
\pgfusepath{stroke}%
\end{pgfscope}%
\begin{pgfscope}%
\pgfpathrectangle{\pgfqpoint{0.832441in}{0.754477in}}{\pgfqpoint{4.987559in}{2.361409in}}%
\pgfusepath{clip}%
\pgfsetroundcap%
\pgfsetroundjoin%
\pgfsetlinewidth{0.702625pt}%
\definecolor{currentstroke}{rgb}{1.000000,1.000000,1.000000}%
\pgfsetstrokecolor{currentstroke}%
\pgfsetdash{}{0pt}%
\pgfpathmoveto{\pgfqpoint{0.832441in}{2.941260in}}%
\pgfpathlineto{\pgfqpoint{5.820000in}{2.941260in}}%
\pgfusepath{stroke}%
\end{pgfscope}%
\begin{pgfscope}%
\pgfpathrectangle{\pgfqpoint{0.832441in}{0.754477in}}{\pgfqpoint{4.987559in}{2.361409in}}%
\pgfusepath{clip}%
\pgfsetroundcap%
\pgfsetroundjoin%
\pgfsetlinewidth{0.702625pt}%
\definecolor{currentstroke}{rgb}{1.000000,1.000000,1.000000}%
\pgfsetstrokecolor{currentstroke}%
\pgfsetdash{}{0pt}%
\pgfpathmoveto{\pgfqpoint{0.832441in}{2.993956in}}%
\pgfpathlineto{\pgfqpoint{5.820000in}{2.993956in}}%
\pgfusepath{stroke}%
\end{pgfscope}%
\begin{pgfscope}%
\pgfpathrectangle{\pgfqpoint{0.832441in}{0.754477in}}{\pgfqpoint{4.987559in}{2.361409in}}%
\pgfusepath{clip}%
\pgfsetroundcap%
\pgfsetroundjoin%
\pgfsetlinewidth{0.702625pt}%
\definecolor{currentstroke}{rgb}{1.000000,1.000000,1.000000}%
\pgfsetstrokecolor{currentstroke}%
\pgfsetdash{}{0pt}%
\pgfpathmoveto{\pgfqpoint{0.832441in}{3.039604in}}%
\pgfpathlineto{\pgfqpoint{5.820000in}{3.039604in}}%
\pgfusepath{stroke}%
\end{pgfscope}%
\begin{pgfscope}%
\pgfpathrectangle{\pgfqpoint{0.832441in}{0.754477in}}{\pgfqpoint{4.987559in}{2.361409in}}%
\pgfusepath{clip}%
\pgfsetroundcap%
\pgfsetroundjoin%
\pgfsetlinewidth{0.702625pt}%
\definecolor{currentstroke}{rgb}{1.000000,1.000000,1.000000}%
\pgfsetstrokecolor{currentstroke}%
\pgfsetdash{}{0pt}%
\pgfpathmoveto{\pgfqpoint{0.832441in}{3.079868in}}%
\pgfpathlineto{\pgfqpoint{5.820000in}{3.079868in}}%
\pgfusepath{stroke}%
\end{pgfscope}%
\begin{pgfscope}%
\definecolor{textcolor}{rgb}{0.150000,0.150000,0.150000}%
\pgfsetstrokecolor{textcolor}%
\pgfsetfillcolor{textcolor}%
\pgftext[x=0.335062in,y=1.935181in,,bottom,rotate=90.000000]{\color{textcolor}\sffamily\fontsize{12.000000}{14.400000}\selectfont Drag Coefficient \(\displaystyle C_D\)}%
\end{pgfscope}%
\begin{pgfscope}%
\pgfpathrectangle{\pgfqpoint{0.832441in}{0.754477in}}{\pgfqpoint{4.987559in}{2.361409in}}%
\pgfusepath{clip}%
\pgfsetbuttcap%
\pgfsetroundjoin%
\definecolor{currentfill}{rgb}{0.000000,0.000000,0.000000}%
\pgfsetfillcolor{currentfill}%
\pgfsetfillopacity{0.800000}%
\pgfsetlinewidth{1.003750pt}%
\definecolor{currentstroke}{rgb}{0.000000,0.000000,0.000000}%
\pgfsetstrokecolor{currentstroke}%
\pgfsetstrokeopacity{0.800000}%
\pgfsetdash{}{0pt}%
\pgfsys@defobject{currentmarker}{\pgfqpoint{-0.010417in}{-0.010417in}}{\pgfqpoint{0.010417in}{0.010417in}}{%
\pgfpathmoveto{\pgfqpoint{0.000000in}{-0.010417in}}%
\pgfpathcurveto{\pgfqpoint{0.002763in}{-0.010417in}}{\pgfqpoint{0.005412in}{-0.009319in}}{\pgfqpoint{0.007366in}{-0.007366in}}%
\pgfpathcurveto{\pgfqpoint{0.009319in}{-0.005412in}}{\pgfqpoint{0.010417in}{-0.002763in}}{\pgfqpoint{0.010417in}{0.000000in}}%
\pgfpathcurveto{\pgfqpoint{0.010417in}{0.002763in}}{\pgfqpoint{0.009319in}{0.005412in}}{\pgfqpoint{0.007366in}{0.007366in}}%
\pgfpathcurveto{\pgfqpoint{0.005412in}{0.009319in}}{\pgfqpoint{0.002763in}{0.010417in}}{\pgfqpoint{0.000000in}{0.010417in}}%
\pgfpathcurveto{\pgfqpoint{-0.002763in}{0.010417in}}{\pgfqpoint{-0.005412in}{0.009319in}}{\pgfqpoint{-0.007366in}{0.007366in}}%
\pgfpathcurveto{\pgfqpoint{-0.009319in}{0.005412in}}{\pgfqpoint{-0.010417in}{0.002763in}}{\pgfqpoint{-0.010417in}{0.000000in}}%
\pgfpathcurveto{\pgfqpoint{-0.010417in}{-0.002763in}}{\pgfqpoint{-0.009319in}{-0.005412in}}{\pgfqpoint{-0.007366in}{-0.007366in}}%
\pgfpathcurveto{\pgfqpoint{-0.005412in}{-0.009319in}}{\pgfqpoint{-0.002763in}{-0.010417in}}{\pgfqpoint{0.000000in}{-0.010417in}}%
\pgfpathclose%
\pgfusepath{stroke,fill}%
}%
\begin{pgfscope}%
\pgfsys@transformshift{1.146092in}{2.879068in}%
\pgfsys@useobject{currentmarker}{}%
\end{pgfscope}%
\begin{pgfscope}%
\pgfsys@transformshift{1.150337in}{2.905586in}%
\pgfsys@useobject{currentmarker}{}%
\end{pgfscope}%
\begin{pgfscope}%
\pgfsys@transformshift{1.199913in}{2.824314in}%
\pgfsys@useobject{currentmarker}{}%
\end{pgfscope}%
\begin{pgfscope}%
\pgfsys@transformshift{1.245400in}{2.762493in}%
\pgfsys@useobject{currentmarker}{}%
\end{pgfscope}%
\begin{pgfscope}%
\pgfsys@transformshift{1.255985in}{2.819056in}%
\pgfsys@useobject{currentmarker}{}%
\end{pgfscope}%
\begin{pgfscope}%
\pgfsys@transformshift{1.291025in}{2.739548in}%
\pgfsys@useobject{currentmarker}{}%
\end{pgfscope}%
\begin{pgfscope}%
\pgfsys@transformshift{1.311635in}{2.693610in}%
\pgfsys@useobject{currentmarker}{}%
\end{pgfscope}%
\begin{pgfscope}%
\pgfsys@transformshift{1.353043in}{2.652986in}%
\pgfsys@useobject{currentmarker}{}%
\end{pgfscope}%
\begin{pgfscope}%
\pgfsys@transformshift{1.353320in}{2.730763in}%
\pgfsys@useobject{currentmarker}{}%
\end{pgfscope}%
\begin{pgfscope}%
\pgfsys@transformshift{1.444071in}{2.543472in}%
\pgfsys@useobject{currentmarker}{}%
\end{pgfscope}%
\begin{pgfscope}%
\pgfsys@transformshift{1.452616in}{2.610646in}%
\pgfsys@useobject{currentmarker}{}%
\end{pgfscope}%
\begin{pgfscope}%
\pgfsys@transformshift{1.465229in}{2.653081in}%
\pgfsys@useobject{currentmarker}{}%
\end{pgfscope}%
\begin{pgfscope}%
\pgfsys@transformshift{1.497920in}{2.497566in}%
\pgfsys@useobject{currentmarker}{}%
\end{pgfscope}%
\begin{pgfscope}%
\pgfsys@transformshift{1.527219in}{2.557680in}%
\pgfsys@useobject{currentmarker}{}%
\end{pgfscope}%
\begin{pgfscope}%
\pgfsys@transformshift{1.548145in}{2.600122in}%
\pgfsys@useobject{currentmarker}{}%
\end{pgfscope}%
\begin{pgfscope}%
\pgfsys@transformshift{1.551764in}{2.448118in}%
\pgfsys@useobject{currentmarker}{}%
\end{pgfscope}%
\begin{pgfscope}%
\pgfsys@transformshift{1.564238in}{2.451660in}%
\pgfsys@useobject{currentmarker}{}%
\end{pgfscope}%
\begin{pgfscope}%
\pgfsys@transformshift{1.584914in}{2.423394in}%
\pgfsys@useobject{currentmarker}{}%
\end{pgfscope}%
\begin{pgfscope}%
\pgfsys@transformshift{1.589109in}{2.434005in}%
\pgfsys@useobject{currentmarker}{}%
\end{pgfscope}%
\begin{pgfscope}%
\pgfsys@transformshift{1.601562in}{2.430479in}%
\pgfsys@useobject{currentmarker}{}%
\end{pgfscope}%
\begin{pgfscope}%
\pgfsys@transformshift{1.605640in}{2.409273in}%
\pgfsys@useobject{currentmarker}{}%
\end{pgfscope}%
\begin{pgfscope}%
\pgfsys@transformshift{1.618154in}{2.423426in}%
\pgfsys@useobject{currentmarker}{}%
\end{pgfscope}%
\begin{pgfscope}%
\pgfsys@transformshift{1.630534in}{2.398686in}%
\pgfsys@useobject{currentmarker}{}%
\end{pgfscope}%
\begin{pgfscope}%
\pgfsys@transformshift{1.647342in}{2.451723in}%
\pgfsys@useobject{currentmarker}{}%
\end{pgfscope}%
\begin{pgfscope}%
\pgfsys@transformshift{1.671947in}{2.359833in}%
\pgfsys@useobject{currentmarker}{}%
\end{pgfscope}%
\begin{pgfscope}%
\pgfsys@transformshift{1.692673in}{2.345712in}%
\pgfsys@useobject{currentmarker}{}%
\end{pgfscope}%
\begin{pgfscope}%
\pgfsys@transformshift{1.696962in}{2.384596in}%
\pgfsys@useobject{currentmarker}{}%
\end{pgfscope}%
\begin{pgfscope}%
\pgfsys@transformshift{1.713360in}{2.320980in}%
\pgfsys@useobject{currentmarker}{}%
\end{pgfscope}%
\begin{pgfscope}%
\pgfsys@transformshift{1.721595in}{2.299775in}%
\pgfsys@useobject{currentmarker}{}%
\end{pgfscope}%
\begin{pgfscope}%
\pgfsys@transformshift{1.729941in}{2.310393in}%
\pgfsys@useobject{currentmarker}{}%
\end{pgfscope}%
\begin{pgfscope}%
\pgfsys@transformshift{1.746500in}{2.292730in}%
\pgfsys@useobject{currentmarker}{}%
\end{pgfscope}%
\begin{pgfscope}%
\pgfsys@transformshift{1.767237in}{2.282143in}%
\pgfsys@useobject{currentmarker}{}%
\end{pgfscope}%
\begin{pgfscope}%
\pgfsys@transformshift{1.783824in}{2.271548in}%
\pgfsys@useobject{currentmarker}{}%
\end{pgfscope}%
\begin{pgfscope}%
\pgfsys@transformshift{1.796248in}{2.260953in}%
\pgfsys@useobject{currentmarker}{}%
\end{pgfscope}%
\begin{pgfscope}%
\pgfsys@transformshift{1.817163in}{2.299853in}%
\pgfsys@useobject{currentmarker}{}%
\end{pgfscope}%
\begin{pgfscope}%
\pgfsys@transformshift{1.829388in}{2.232695in}%
\pgfsys@useobject{currentmarker}{}%
\end{pgfscope}%
\begin{pgfscope}%
\pgfsys@transformshift{1.841807in}{2.218566in}%
\pgfsys@useobject{currentmarker}{}%
\end{pgfscope}%
\begin{pgfscope}%
\pgfsys@transformshift{1.854198in}{2.197368in}%
\pgfsys@useobject{currentmarker}{}%
\end{pgfscope}%
\begin{pgfscope}%
\pgfsys@transformshift{1.862483in}{2.190308in}%
\pgfsys@useobject{currentmarker}{}%
\end{pgfscope}%
\begin{pgfscope}%
\pgfsys@transformshift{1.883242in}{2.186789in}%
\pgfsys@useobject{currentmarker}{}%
\end{pgfscope}%
\begin{pgfscope}%
\pgfsys@transformshift{1.895634in}{2.165584in}%
\pgfsys@useobject{currentmarker}{}%
\end{pgfscope}%
\begin{pgfscope}%
\pgfsys@transformshift{1.908064in}{2.154989in}%
\pgfsys@useobject{currentmarker}{}%
\end{pgfscope}%
\begin{pgfscope}%
\pgfsys@transformshift{1.912292in}{2.176202in}%
\pgfsys@useobject{currentmarker}{}%
\end{pgfscope}%
\begin{pgfscope}%
\pgfsys@transformshift{1.920499in}{2.147928in}%
\pgfsys@useobject{currentmarker}{}%
\end{pgfscope}%
\begin{pgfscope}%
\pgfsys@transformshift{1.937036in}{2.123196in}%
\pgfsys@useobject{currentmarker}{}%
\end{pgfscope}%
\begin{pgfscope}%
\pgfsys@transformshift{1.970164in}{2.091412in}%
\pgfsys@useobject{currentmarker}{}%
\end{pgfscope}%
\begin{pgfscope}%
\pgfsys@transformshift{1.991073in}{2.130312in}%
\pgfsys@useobject{currentmarker}{}%
\end{pgfscope}%
\begin{pgfscope}%
\pgfsys@transformshift{2.003066in}{1.995987in}%
\pgfsys@useobject{currentmarker}{}%
\end{pgfscope}%
\begin{pgfscope}%
\pgfsys@transformshift{2.036682in}{2.102070in}%
\pgfsys@useobject{currentmarker}{}%
\end{pgfscope}%
\begin{pgfscope}%
\pgfsys@transformshift{2.065443in}{2.010179in}%
\pgfsys@useobject{currentmarker}{}%
\end{pgfscope}%
\begin{pgfscope}%
\pgfsys@transformshift{2.082312in}{2.080896in}%
\pgfsys@useobject{currentmarker}{}%
\end{pgfscope}%
\begin{pgfscope}%
\pgfsys@transformshift{2.107040in}{2.024356in}%
\pgfsys@useobject{currentmarker}{}%
\end{pgfscope}%
\begin{pgfscope}%
\pgfsys@transformshift{2.115341in}{2.020821in}%
\pgfsys@useobject{currentmarker}{}%
\end{pgfscope}%
\begin{pgfscope}%
\pgfsys@transformshift{2.140196in}{1.999632in}%
\pgfsys@useobject{currentmarker}{}%
\end{pgfscope}%
\begin{pgfscope}%
\pgfsys@transformshift{2.144263in}{1.974892in}%
\pgfsys@useobject{currentmarker}{}%
\end{pgfscope}%
\begin{pgfscope}%
\pgfsys@transformshift{2.160872in}{1.971374in}%
\pgfsys@useobject{currentmarker}{}%
\end{pgfscope}%
\begin{pgfscope}%
\pgfsys@transformshift{2.177492in}{1.971381in}%
\pgfsys@useobject{currentmarker}{}%
\end{pgfscope}%
\begin{pgfscope}%
\pgfsys@transformshift{2.189894in}{1.953718in}%
\pgfsys@useobject{currentmarker}{}%
\end{pgfscope}%
\begin{pgfscope}%
\pgfsys@transformshift{2.194100in}{1.967863in}%
\pgfsys@useobject{currentmarker}{}%
\end{pgfscope}%
\begin{pgfscope}%
\pgfsys@transformshift{2.210509in}{1.907781in}%
\pgfsys@useobject{currentmarker}{}%
\end{pgfscope}%
\begin{pgfscope}%
\pgfsys@transformshift{2.210642in}{1.946665in}%
\pgfsys@useobject{currentmarker}{}%
\end{pgfscope}%
\begin{pgfscope}%
\pgfsys@transformshift{2.231368in}{1.932544in}%
\pgfsys@useobject{currentmarker}{}%
\end{pgfscope}%
\begin{pgfscope}%
\pgfsys@transformshift{2.243782in}{1.918407in}%
\pgfsys@useobject{currentmarker}{}%
\end{pgfscope}%
\begin{pgfscope}%
\pgfsys@transformshift{2.247877in}{1.900736in}%
\pgfsys@useobject{currentmarker}{}%
\end{pgfscope}%
\begin{pgfscope}%
\pgfsys@transformshift{2.260280in}{1.883073in}%
\pgfsys@useobject{currentmarker}{}%
\end{pgfscope}%
\begin{pgfscope}%
\pgfsys@transformshift{2.268642in}{1.897217in}%
\pgfsys@useobject{currentmarker}{}%
\end{pgfscope}%
\begin{pgfscope}%
\pgfsys@transformshift{2.276860in}{1.872478in}%
\pgfsys@useobject{currentmarker}{}%
\end{pgfscope}%
\begin{pgfscope}%
\pgfsys@transformshift{2.293480in}{1.872493in}%
\pgfsys@useobject{currentmarker}{}%
\end{pgfscope}%
\begin{pgfscope}%
\pgfsys@transformshift{2.297586in}{1.858356in}%
\pgfsys@useobject{currentmarker}{}%
\end{pgfscope}%
\begin{pgfscope}%
\pgfsys@transformshift{2.310055in}{1.858364in}%
\pgfsys@useobject{currentmarker}{}%
\end{pgfscope}%
\begin{pgfscope}%
\pgfsys@transformshift{2.314234in}{1.865441in}%
\pgfsys@useobject{currentmarker}{}%
\end{pgfscope}%
\begin{pgfscope}%
\pgfsys@transformshift{2.322508in}{1.854846in}%
\pgfsys@useobject{currentmarker}{}%
\end{pgfscope}%
\begin{pgfscope}%
\pgfsys@transformshift{2.363871in}{1.801848in}%
\pgfsys@useobject{currentmarker}{}%
\end{pgfscope}%
\begin{pgfscope}%
\pgfsys@transformshift{2.405429in}{1.805422in}%
\pgfsys@useobject{currentmarker}{}%
\end{pgfscope}%
\begin{pgfscope}%
\pgfsys@transformshift{2.413569in}{1.755934in}%
\pgfsys@useobject{currentmarker}{}%
\end{pgfscope}%
\begin{pgfscope}%
\pgfsys@transformshift{2.463428in}{1.755974in}%
\pgfsys@useobject{currentmarker}{}%
\end{pgfscope}%
\begin{pgfscope}%
\pgfsys@transformshift{2.492301in}{1.695899in}%
\pgfsys@useobject{currentmarker}{}%
\end{pgfscope}%
\begin{pgfscope}%
\pgfsys@transformshift{2.513199in}{1.731265in}%
\pgfsys@useobject{currentmarker}{}%
\end{pgfscope}%
\begin{pgfscope}%
\pgfsys@transformshift{2.537909in}{1.667657in}%
\pgfsys@useobject{currentmarker}{}%
\end{pgfscope}%
\begin{pgfscope}%
\pgfsys@transformshift{2.546377in}{1.713618in}%
\pgfsys@useobject{currentmarker}{}%
\end{pgfscope}%
\begin{pgfscope}%
\pgfsys@transformshift{2.641767in}{1.664201in}%
\pgfsys@useobject{currentmarker}{}%
\end{pgfscope}%
\begin{pgfscope}%
\pgfsys@transformshift{2.683247in}{1.643027in}%
\pgfsys@useobject{currentmarker}{}%
\end{pgfscope}%
\begin{pgfscope}%
\pgfsys@transformshift{2.762117in}{1.621877in}%
\pgfsys@useobject{currentmarker}{}%
\end{pgfscope}%
\begin{pgfscope}%
\pgfsys@transformshift{2.783053in}{1.667854in}%
\pgfsys@useobject{currentmarker}{}%
\end{pgfscope}%
\begin{pgfscope}%
\pgfsys@transformshift{2.807808in}{1.618382in}%
\pgfsys@useobject{currentmarker}{}%
\end{pgfscope}%
\begin{pgfscope}%
\pgfsys@transformshift{2.837030in}{1.657290in}%
\pgfsys@useobject{currentmarker}{}%
\end{pgfscope}%
\begin{pgfscope}%
\pgfsys@transformshift{2.882649in}{1.632582in}%
\pgfsys@useobject{currentmarker}{}%
\end{pgfscope}%
\begin{pgfscope}%
\pgfsys@transformshift{2.936687in}{1.639698in}%
\pgfsys@useobject{currentmarker}{}%
\end{pgfscope}%
\begin{pgfscope}%
\pgfsys@transformshift{2.961558in}{1.622042in}%
\pgfsys@useobject{currentmarker}{}%
\end{pgfscope}%
\begin{pgfscope}%
\pgfsys@transformshift{3.081991in}{1.604458in}%
\pgfsys@useobject{currentmarker}{}%
\end{pgfscope}%
\begin{pgfscope}%
\pgfsys@transformshift{3.115130in}{1.576207in}%
\pgfsys@useobject{currentmarker}{}%
\end{pgfscope}%
\begin{pgfscope}%
\pgfsys@transformshift{3.185716in}{1.562126in}%
\pgfsys@useobject{currentmarker}{}%
\end{pgfscope}%
\begin{pgfscope}%
\pgfsys@transformshift{3.239681in}{1.548028in}%
\pgfsys@useobject{currentmarker}{}%
\end{pgfscope}%
\begin{pgfscope}%
\pgfsys@transformshift{3.339338in}{1.530428in}%
\pgfsys@useobject{currentmarker}{}%
\end{pgfscope}%
\begin{pgfscope}%
\pgfsys@transformshift{3.401649in}{1.526941in}%
\pgfsys@useobject{currentmarker}{}%
\end{pgfscope}%
\begin{pgfscope}%
\pgfsys@transformshift{3.439045in}{1.526972in}%
\pgfsys@useobject{currentmarker}{}%
\end{pgfscope}%
\begin{pgfscope}%
\pgfsys@transformshift{3.497166in}{1.512882in}%
\pgfsys@useobject{currentmarker}{}%
\end{pgfscope}%
\begin{pgfscope}%
\pgfsys@transformshift{3.588577in}{1.512953in}%
\pgfsys@useobject{currentmarker}{}%
\end{pgfscope}%
\begin{pgfscope}%
\pgfsys@transformshift{3.650927in}{1.520069in}%
\pgfsys@useobject{currentmarker}{}%
\end{pgfscope}%
\begin{pgfscope}%
\pgfsys@transformshift{3.700836in}{1.534253in}%
\pgfsys@useobject{currentmarker}{}%
\end{pgfscope}%
\begin{pgfscope}%
\pgfsys@transformshift{3.725807in}{1.544879in}%
\pgfsys@useobject{currentmarker}{}%
\end{pgfscope}%
\begin{pgfscope}%
\pgfsys@transformshift{3.759030in}{1.541369in}%
\pgfsys@useobject{currentmarker}{}%
\end{pgfscope}%
\begin{pgfscope}%
\pgfsys@transformshift{3.846312in}{1.548508in}%
\pgfsys@useobject{currentmarker}{}%
\end{pgfscope}%
\begin{pgfscope}%
\pgfsys@transformshift{3.850568in}{1.576790in}%
\pgfsys@useobject{currentmarker}{}%
\end{pgfscope}%
\begin{pgfscope}%
\pgfsys@transformshift{3.871355in}{1.580340in}%
\pgfsys@useobject{currentmarker}{}%
\end{pgfscope}%
\begin{pgfscope}%
\pgfsys@transformshift{3.942062in}{1.601608in}%
\pgfsys@useobject{currentmarker}{}%
\end{pgfscope}%
\begin{pgfscope}%
\pgfsys@transformshift{3.950225in}{1.559197in}%
\pgfsys@useobject{currentmarker}{}%
\end{pgfscope}%
\begin{pgfscope}%
\pgfsys@transformshift{3.983464in}{1.559221in}%
\pgfsys@useobject{currentmarker}{}%
\end{pgfscope}%
\begin{pgfscope}%
\pgfsys@transformshift{4.045853in}{1.576947in}%
\pgfsys@useobject{currentmarker}{}%
\end{pgfscope}%
\begin{pgfscope}%
\pgfsys@transformshift{4.046025in}{1.626443in}%
\pgfsys@useobject{currentmarker}{}%
\end{pgfscope}%
\begin{pgfscope}%
\pgfsys@transformshift{4.066613in}{1.573429in}%
\pgfsys@useobject{currentmarker}{}%
\end{pgfscope}%
\begin{pgfscope}%
\pgfsys@transformshift{4.112332in}{1.577002in}%
\pgfsys@useobject{currentmarker}{}%
\end{pgfscope}%
\begin{pgfscope}%
\pgfsys@transformshift{4.145510in}{1.559355in}%
\pgfsys@useobject{currentmarker}{}%
\end{pgfscope}%
\begin{pgfscope}%
\pgfsys@transformshift{4.158240in}{1.633598in}%
\pgfsys@useobject{currentmarker}{}%
\end{pgfscope}%
\begin{pgfscope}%
\pgfsys@transformshift{4.182906in}{1.559379in}%
\pgfsys@useobject{currentmarker}{}%
\end{pgfscope}%
\begin{pgfscope}%
\pgfsys@transformshift{4.183105in}{1.615942in}%
\pgfsys@useobject{currentmarker}{}%
\end{pgfscope}%
\begin{pgfscope}%
\pgfsys@transformshift{4.199675in}{1.601813in}%
\pgfsys@useobject{currentmarker}{}%
\end{pgfscope}%
\begin{pgfscope}%
\pgfsys@transformshift{4.212066in}{1.580615in}%
\pgfsys@useobject{currentmarker}{}%
\end{pgfscope}%
\begin{pgfscope}%
\pgfsys@transformshift{4.212089in}{1.587684in}%
\pgfsys@useobject{currentmarker}{}%
\end{pgfscope}%
\begin{pgfscope}%
\pgfsys@transformshift{4.220412in}{1.591226in}%
\pgfsys@useobject{currentmarker}{}%
\end{pgfscope}%
\begin{pgfscope}%
\pgfsys@transformshift{4.220462in}{1.605371in}%
\pgfsys@useobject{currentmarker}{}%
\end{pgfscope}%
\begin{pgfscope}%
\pgfsys@transformshift{4.237082in}{1.605379in}%
\pgfsys@useobject{currentmarker}{}%
\end{pgfscope}%
\begin{pgfscope}%
\pgfsys@transformshift{4.249373in}{1.555899in}%
\pgfsys@useobject{currentmarker}{}%
\end{pgfscope}%
\begin{pgfscope}%
\pgfsys@transformshift{4.249473in}{1.584181in}%
\pgfsys@useobject{currentmarker}{}%
\end{pgfscope}%
\begin{pgfscope}%
\pgfsys@transformshift{4.249545in}{1.605395in}%
\pgfsys@useobject{currentmarker}{}%
\end{pgfscope}%
\begin{pgfscope}%
\pgfsys@transformshift{4.270149in}{1.555915in}%
\pgfsys@useobject{currentmarker}{}%
\end{pgfscope}%
\begin{pgfscope}%
\pgfsys@transformshift{4.270271in}{1.591265in}%
\pgfsys@useobject{currentmarker}{}%
\end{pgfscope}%
\begin{pgfscope}%
\pgfsys@transformshift{4.274466in}{1.601876in}%
\pgfsys@useobject{currentmarker}{}%
\end{pgfscope}%
\begin{pgfscope}%
\pgfsys@transformshift{4.282591in}{1.548855in}%
\pgfsys@useobject{currentmarker}{}%
\end{pgfscope}%
\begin{pgfscope}%
\pgfsys@transformshift{4.290936in}{1.559465in}%
\pgfsys@useobject{currentmarker}{}%
\end{pgfscope}%
\begin{pgfscope}%
\pgfsys@transformshift{4.291025in}{1.584213in}%
\pgfsys@useobject{currentmarker}{}%
\end{pgfscope}%
\begin{pgfscope}%
\pgfsys@transformshift{4.291058in}{1.594815in}%
\pgfsys@useobject{currentmarker}{}%
\end{pgfscope}%
\begin{pgfscope}%
\pgfsys@transformshift{4.307678in}{1.594831in}%
\pgfsys@useobject{currentmarker}{}%
\end{pgfscope}%
\begin{pgfscope}%
\pgfsys@transformshift{4.311795in}{1.584228in}%
\pgfsys@useobject{currentmarker}{}%
\end{pgfscope}%
\begin{pgfscope}%
\pgfsys@transformshift{4.328432in}{1.587778in}%
\pgfsys@useobject{currentmarker}{}%
\end{pgfscope}%
\begin{pgfscope}%
\pgfsys@transformshift{4.336617in}{1.552436in}%
\pgfsys@useobject{currentmarker}{}%
\end{pgfscope}%
\begin{pgfscope}%
\pgfsys@transformshift{4.336667in}{1.566573in}%
\pgfsys@useobject{currentmarker}{}%
\end{pgfscope}%
\begin{pgfscope}%
\pgfsys@transformshift{4.349169in}{1.577191in}%
\pgfsys@useobject{currentmarker}{}%
\end{pgfscope}%
\begin{pgfscope}%
\pgfsys@transformshift{4.353358in}{1.587802in}%
\pgfsys@useobject{currentmarker}{}%
\end{pgfscope}%
\begin{pgfscope}%
\pgfsys@transformshift{4.357404in}{1.555986in}%
\pgfsys@useobject{currentmarker}{}%
\end{pgfscope}%
\begin{pgfscope}%
\pgfsys@transformshift{4.361483in}{1.534781in}%
\pgfsys@useobject{currentmarker}{}%
\end{pgfscope}%
\begin{pgfscope}%
\pgfsys@transformshift{4.369945in}{1.577207in}%
\pgfsys@useobject{currentmarker}{}%
\end{pgfscope}%
\begin{pgfscope}%
\pgfsys@transformshift{4.378169in}{1.552467in}%
\pgfsys@useobject{currentmarker}{}%
\end{pgfscope}%
\begin{pgfscope}%
\pgfsys@transformshift{4.390654in}{1.559544in}%
\pgfsys@useobject{currentmarker}{}%
\end{pgfscope}%
\begin{pgfscope}%
\pgfsys@transformshift{4.390743in}{1.584291in}%
\pgfsys@useobject{currentmarker}{}%
\end{pgfscope}%
\begin{pgfscope}%
\pgfsys@transformshift{4.390854in}{1.616107in}%
\pgfsys@useobject{currentmarker}{}%
\end{pgfscope}%
\begin{pgfscope}%
\pgfsys@transformshift{4.411519in}{1.584307in}%
\pgfsys@useobject{currentmarker}{}%
\end{pgfscope}%
\begin{pgfscope}%
\pgfsys@transformshift{4.411630in}{1.616123in}%
\pgfsys@useobject{currentmarker}{}%
\end{pgfscope}%
\begin{pgfscope}%
\pgfsys@transformshift{4.415515in}{1.538354in}%
\pgfsys@useobject{currentmarker}{}%
\end{pgfscope}%
\begin{pgfscope}%
\pgfsys@transformshift{4.415886in}{1.644405in}%
\pgfsys@useobject{currentmarker}{}%
\end{pgfscope}%
\begin{pgfscope}%
\pgfsys@transformshift{4.424093in}{1.616139in}%
\pgfsys@useobject{currentmarker}{}%
\end{pgfscope}%
\begin{pgfscope}%
\pgfsys@transformshift{4.428211in}{1.605536in}%
\pgfsys@useobject{currentmarker}{}%
\end{pgfscope}%
\begin{pgfscope}%
\pgfsys@transformshift{4.432306in}{1.587865in}%
\pgfsys@useobject{currentmarker}{}%
\end{pgfscope}%
\begin{pgfscope}%
\pgfsys@transformshift{4.432345in}{1.598468in}%
\pgfsys@useobject{currentmarker}{}%
\end{pgfscope}%
\begin{pgfscope}%
\pgfsys@transformshift{4.453093in}{1.591415in}%
\pgfsys@useobject{currentmarker}{}%
\end{pgfscope}%
\begin{pgfscope}%
\pgfsys@transformshift{4.453182in}{1.616155in}%
\pgfsys@useobject{currentmarker}{}%
\end{pgfscope}%
\begin{pgfscope}%
\pgfsys@transformshift{4.469624in}{1.566683in}%
\pgfsys@useobject{currentmarker}{}%
\end{pgfscope}%
\begin{pgfscope}%
\pgfsys@transformshift{4.469713in}{1.591423in}%
\pgfsys@useobject{currentmarker}{}%
\end{pgfscope}%
\begin{pgfscope}%
\pgfsys@transformshift{4.482065in}{1.559623in}%
\pgfsys@useobject{currentmarker}{}%
\end{pgfscope}%
\begin{pgfscope}%
\pgfsys@transformshift{4.490350in}{1.552554in}%
\pgfsys@useobject{currentmarker}{}%
\end{pgfscope}%
\begin{pgfscope}%
\pgfsys@transformshift{4.490478in}{1.587904in}%
\pgfsys@useobject{currentmarker}{}%
\end{pgfscope}%
\begin{pgfscope}%
\pgfsys@transformshift{4.494506in}{1.552562in}%
\pgfsys@useobject{currentmarker}{}%
\end{pgfscope}%
\begin{pgfscope}%
\pgfsys@transformshift{4.523789in}{1.609141in}%
\pgfsys@useobject{currentmarker}{}%
\end{pgfscope}%
\begin{pgfscope}%
\pgfsys@transformshift{4.540165in}{1.538456in}%
\pgfsys@useobject{currentmarker}{}%
\end{pgfscope}%
\begin{pgfscope}%
\pgfsys@transformshift{4.573587in}{1.591509in}%
\pgfsys@useobject{currentmarker}{}%
\end{pgfscope}%
\begin{pgfscope}%
\pgfsys@transformshift{4.594324in}{1.580922in}%
\pgfsys@useobject{currentmarker}{}%
\end{pgfscope}%
\begin{pgfscope}%
\pgfsys@transformshift{4.594391in}{1.598594in}%
\pgfsys@useobject{currentmarker}{}%
\end{pgfscope}%
\begin{pgfscope}%
\pgfsys@transformshift{4.610811in}{1.542046in}%
\pgfsys@useobject{currentmarker}{}%
\end{pgfscope}%
\begin{pgfscope}%
\pgfsys@transformshift{4.635876in}{1.580954in}%
\pgfsys@useobject{currentmarker}{}%
\end{pgfscope}%
\begin{pgfscope}%
\pgfsys@transformshift{4.652446in}{1.566825in}%
\pgfsys@useobject{currentmarker}{}%
\end{pgfscope}%
\begin{pgfscope}%
\pgfsys@transformshift{4.673111in}{1.535025in}%
\pgfsys@useobject{currentmarker}{}%
\end{pgfscope}%
\begin{pgfscope}%
\pgfsys@transformshift{4.677339in}{1.556238in}%
\pgfsys@useobject{currentmarker}{}%
\end{pgfscope}%
\begin{pgfscope}%
\pgfsys@transformshift{4.693887in}{1.535040in}%
\pgfsys@useobject{currentmarker}{}%
\end{pgfscope}%
\begin{pgfscope}%
\pgfsys@transformshift{4.693987in}{1.563322in}%
\pgfsys@useobject{currentmarker}{}%
\end{pgfscope}%
\begin{pgfscope}%
\pgfsys@transformshift{4.714663in}{1.535064in}%
\pgfsys@useobject{currentmarker}{}%
\end{pgfscope}%
\begin{pgfscope}%
\pgfsys@transformshift{4.739190in}{1.421960in}%
\pgfsys@useobject{currentmarker}{}%
\end{pgfscope}%
\begin{pgfscope}%
\pgfsys@transformshift{4.755378in}{1.298246in}%
\pgfsys@useobject{currentmarker}{}%
\end{pgfscope}%
\begin{pgfscope}%
\pgfsys@transformshift{4.775866in}{1.216958in}%
\pgfsys@useobject{currentmarker}{}%
\end{pgfscope}%
\begin{pgfscope}%
\pgfsys@transformshift{4.787980in}{1.117984in}%
\pgfsys@useobject{currentmarker}{}%
\end{pgfscope}%
\begin{pgfscope}%
\pgfsys@transformshift{4.804555in}{1.103855in}%
\pgfsys@useobject{currentmarker}{}%
\end{pgfscope}%
\begin{pgfscope}%
\pgfsys@transformshift{4.812901in}{1.114465in}%
\pgfsys@useobject{currentmarker}{}%
\end{pgfscope}%
\begin{pgfscope}%
\pgfsys@transformshift{4.816991in}{1.096794in}%
\pgfsys@useobject{currentmarker}{}%
\end{pgfscope}%
\begin{pgfscope}%
\pgfsys@transformshift{4.821175in}{1.103871in}%
\pgfsys@useobject{currentmarker}{}%
\end{pgfscope}%
\begin{pgfscope}%
\pgfsys@transformshift{4.825276in}{1.089734in}%
\pgfsys@useobject{currentmarker}{}%
\end{pgfscope}%
\begin{pgfscope}%
\pgfsys@transformshift{4.829532in}{1.118015in}%
\pgfsys@useobject{currentmarker}{}%
\end{pgfscope}%
\begin{pgfscope}%
\pgfsys@transformshift{4.829582in}{1.132160in}%
\pgfsys@useobject{currentmarker}{}%
\end{pgfscope}%
\begin{pgfscope}%
\pgfsys@transformshift{4.846151in}{1.118031in}%
\pgfsys@useobject{currentmarker}{}%
\end{pgfscope}%
\begin{pgfscope}%
\pgfsys@transformshift{4.846201in}{1.132168in}%
\pgfsys@useobject{currentmarker}{}%
\end{pgfscope}%
\begin{pgfscope}%
\pgfsys@transformshift{4.850158in}{1.075612in}%
\pgfsys@useobject{currentmarker}{}%
\end{pgfscope}%
\begin{pgfscope}%
\pgfsys@transformshift{4.854464in}{1.118039in}%
\pgfsys@useobject{currentmarker}{}%
\end{pgfscope}%
\begin{pgfscope}%
\pgfsys@transformshift{4.858670in}{1.132184in}%
\pgfsys@useobject{currentmarker}{}%
\end{pgfscope}%
\begin{pgfscope}%
\pgfsys@transformshift{4.863081in}{1.206419in}%
\pgfsys@useobject{currentmarker}{}%
\end{pgfscope}%
\begin{pgfscope}%
\pgfsys@transformshift{4.867005in}{1.139260in}%
\pgfsys@useobject{currentmarker}{}%
\end{pgfscope}%
\begin{pgfscope}%
\pgfsys@transformshift{4.871416in}{1.213495in}%
\pgfsys@useobject{currentmarker}{}%
\end{pgfscope}%
\begin{pgfscope}%
\pgfsys@transformshift{4.883586in}{1.128665in}%
\pgfsys@useobject{currentmarker}{}%
\end{pgfscope}%
\begin{pgfscope}%
\pgfsys@transformshift{4.883636in}{1.142802in}%
\pgfsys@useobject{currentmarker}{}%
\end{pgfscope}%
\begin{pgfscope}%
\pgfsys@transformshift{4.883697in}{1.160481in}%
\pgfsys@useobject{currentmarker}{}%
\end{pgfscope}%
\begin{pgfscope}%
\pgfsys@transformshift{4.900078in}{1.093331in}%
\pgfsys@useobject{currentmarker}{}%
\end{pgfscope}%
\begin{pgfscope}%
\pgfsys@transformshift{4.904395in}{1.139284in}%
\pgfsys@useobject{currentmarker}{}%
\end{pgfscope}%
\begin{pgfscope}%
\pgfsys@transformshift{4.912957in}{1.209992in}%
\pgfsys@useobject{currentmarker}{}%
\end{pgfscope}%
\begin{pgfscope}%
\pgfsys@transformshift{4.921065in}{1.153444in}%
\pgfsys@useobject{currentmarker}{}%
\end{pgfscope}%
\begin{pgfscope}%
\pgfsys@transformshift{4.941879in}{1.164063in}%
\pgfsys@useobject{currentmarker}{}%
\end{pgfscope}%
\begin{pgfscope}%
\pgfsys@transformshift{4.941940in}{1.181734in}%
\pgfsys@useobject{currentmarker}{}%
\end{pgfscope}%
\begin{pgfscope}%
\pgfsys@transformshift{4.949976in}{1.103973in}%
\pgfsys@useobject{currentmarker}{}%
\end{pgfscope}%
\begin{pgfscope}%
\pgfsys@transformshift{4.950314in}{1.199421in}%
\pgfsys@useobject{currentmarker}{}%
\end{pgfscope}%
\begin{pgfscope}%
\pgfsys@transformshift{4.954343in}{1.164071in}%
\pgfsys@useobject{currentmarker}{}%
\end{pgfscope}%
\begin{pgfscope}%
\pgfsys@transformshift{4.966812in}{1.164079in}%
\pgfsys@useobject{currentmarker}{}%
\end{pgfscope}%
\begin{pgfscope}%
\pgfsys@transformshift{4.971029in}{1.181758in}%
\pgfsys@useobject{currentmarker}{}%
\end{pgfscope}%
\begin{pgfscope}%
\pgfsys@transformshift{4.983442in}{1.167629in}%
\pgfsys@useobject{currentmarker}{}%
\end{pgfscope}%
\begin{pgfscope}%
\pgfsys@transformshift{4.983464in}{1.174697in}%
\pgfsys@useobject{currentmarker}{}%
\end{pgfscope}%
\begin{pgfscope}%
\pgfsys@transformshift{4.983514in}{1.188842in}%
\pgfsys@useobject{currentmarker}{}%
\end{pgfscope}%
\begin{pgfscope}%
\pgfsys@transformshift{5.008397in}{1.174721in}%
\pgfsys@useobject{currentmarker}{}%
\end{pgfscope}%
\begin{pgfscope}%
\pgfsys@transformshift{5.012403in}{1.132302in}%
\pgfsys@useobject{currentmarker}{}%
\end{pgfscope}%
\begin{pgfscope}%
\pgfsys@transformshift{5.012653in}{1.203003in}%
\pgfsys@useobject{currentmarker}{}%
\end{pgfscope}%
\begin{pgfscope}%
\pgfsys@transformshift{5.025016in}{1.174736in}%
\pgfsys@useobject{currentmarker}{}%
\end{pgfscope}%
\begin{pgfscope}%
\pgfsys@transformshift{5.025055in}{1.185339in}%
\pgfsys@useobject{currentmarker}{}%
\end{pgfscope}%
\begin{pgfscope}%
\pgfsys@transformshift{5.025077in}{1.192408in}%
\pgfsys@useobject{currentmarker}{}%
\end{pgfscope}%
\begin{pgfscope}%
\pgfsys@transformshift{5.025116in}{1.203010in}%
\pgfsys@useobject{currentmarker}{}%
\end{pgfscope}%
\begin{pgfscope}%
\pgfsys@transformshift{5.037480in}{1.174744in}%
\pgfsys@useobject{currentmarker}{}%
\end{pgfscope}%
\begin{pgfscope}%
\pgfsys@transformshift{5.045643in}{1.132333in}%
\pgfsys@useobject{currentmarker}{}%
\end{pgfscope}%
\begin{pgfscope}%
\pgfsys@transformshift{5.045831in}{1.185355in}%
\pgfsys@useobject{currentmarker}{}%
\end{pgfscope}%
\begin{pgfscope}%
\pgfsys@transformshift{5.049949in}{1.174752in}%
\pgfsys@useobject{currentmarker}{}%
\end{pgfscope}%
\begin{pgfscope}%
\pgfsys@transformshift{5.054277in}{1.224247in}%
\pgfsys@useobject{currentmarker}{}%
\end{pgfscope}%
\begin{pgfscope}%
\pgfsys@transformshift{5.054299in}{1.231316in}%
\pgfsys@useobject{currentmarker}{}%
\end{pgfscope}%
\begin{pgfscope}%
\pgfsys@transformshift{5.054349in}{1.245461in}%
\pgfsys@useobject{currentmarker}{}%
\end{pgfscope}%
\begin{pgfscope}%
\pgfsys@transformshift{5.066679in}{1.206584in}%
\pgfsys@useobject{currentmarker}{}%
\end{pgfscope}%
\begin{pgfscope}%
\pgfsys@transformshift{5.066890in}{1.266682in}%
\pgfsys@useobject{currentmarker}{}%
\end{pgfscope}%
\begin{pgfscope}%
\pgfsys@transformshift{5.075091in}{1.234866in}%
\pgfsys@useobject{currentmarker}{}%
\end{pgfscope}%
\begin{pgfscope}%
\pgfsys@transformshift{5.079120in}{1.199523in}%
\pgfsys@useobject{currentmarker}{}%
\end{pgfscope}%
\begin{pgfscope}%
\pgfsys@transformshift{5.091412in}{1.150044in}%
\pgfsys@useobject{currentmarker}{}%
\end{pgfscope}%
\begin{pgfscope}%
\pgfsys@transformshift{5.091744in}{1.245484in}%
\pgfsys@useobject{currentmarker}{}%
\end{pgfscope}%
\begin{pgfscope}%
\pgfsys@transformshift{5.091794in}{1.259629in}%
\pgfsys@useobject{currentmarker}{}%
\end{pgfscope}%
\begin{pgfscope}%
\pgfsys@transformshift{5.104385in}{1.294987in}%
\pgfsys@useobject{currentmarker}{}%
\end{pgfscope}%
\begin{pgfscope}%
\pgfsys@transformshift{5.108414in}{1.259645in}%
\pgfsys@useobject{currentmarker}{}%
\end{pgfscope}%
\begin{pgfscope}%
\pgfsys@transformshift{5.166885in}{1.344530in}%
\pgfsys@useobject{currentmarker}{}%
\end{pgfscope}%
\begin{pgfscope}%
\pgfsys@transformshift{5.220989in}{1.369317in}%
\pgfsys@useobject{currentmarker}{}%
\end{pgfscope}%
\begin{pgfscope}%
\pgfsys@transformshift{5.283472in}{1.415325in}%
\pgfsys@useobject{currentmarker}{}%
\end{pgfscope}%
\begin{pgfscope}%
\pgfsys@transformshift{5.316766in}{1.429493in}%
\pgfsys@useobject{currentmarker}{}%
\end{pgfscope}%
\begin{pgfscope}%
\pgfsys@transformshift{5.345788in}{1.411838in}%
\pgfsys@useobject{currentmarker}{}%
\end{pgfscope}%
\begin{pgfscope}%
\pgfsys@transformshift{5.354107in}{1.415380in}%
\pgfsys@useobject{currentmarker}{}%
\end{pgfscope}%
\begin{pgfscope}%
\pgfsys@transformshift{5.358501in}{1.482546in}%
\pgfsys@useobject{currentmarker}{}%
\end{pgfscope}%
\begin{pgfscope}%
\pgfsys@transformshift{5.366797in}{1.479020in}%
\pgfsys@useobject{currentmarker}{}%
\end{pgfscope}%
\begin{pgfscope}%
\pgfsys@transformshift{5.383384in}{1.468425in}%
\pgfsys@useobject{currentmarker}{}%
\end{pgfscope}%
\begin{pgfscope}%
\pgfsys@transformshift{5.399987in}{1.464906in}%
\pgfsys@useobject{currentmarker}{}%
\end{pgfscope}%
\begin{pgfscope}%
\pgfsys@transformshift{5.424819in}{1.436648in}%
\pgfsys@useobject{currentmarker}{}%
\end{pgfscope}%
\begin{pgfscope}%
\pgfsys@transformshift{5.462165in}{1.422535in}%
\pgfsys@useobject{currentmarker}{}%
\end{pgfscope}%
\begin{pgfscope}%
\pgfsys@transformshift{5.503778in}{1.440245in}%
\pgfsys@useobject{currentmarker}{}%
\end{pgfscope}%
\end{pgfscope}%
\begin{pgfscope}%
\pgfpathrectangle{\pgfqpoint{0.832441in}{0.754477in}}{\pgfqpoint{4.987559in}{2.361409in}}%
\pgfusepath{clip}%
\pgfsetroundcap%
\pgfsetroundjoin%
\pgfsetlinewidth{1.505625pt}%
\definecolor{currentstroke}{rgb}{0.258824,0.521569,0.956863}%
\pgfsetstrokecolor{currentstroke}%
\pgfsetstrokeopacity{0.800000}%
\pgfsetdash{}{0pt}%
\pgfpathmoveto{\pgfqpoint{0.867092in}{3.120885in}}%
\pgfpathlineto{\pgfqpoint{1.132850in}{2.868639in}}%
\pgfpathlineto{\pgfqpoint{1.321646in}{2.690941in}}%
\pgfpathlineto{\pgfqpoint{1.466020in}{2.556550in}}%
\pgfpathlineto{\pgfqpoint{1.588182in}{2.444396in}}%
\pgfpathlineto{\pgfqpoint{1.688133in}{2.354131in}}%
\pgfpathlineto{\pgfqpoint{1.776979in}{2.275392in}}%
\pgfpathlineto{\pgfqpoint{1.854719in}{2.207951in}}%
\pgfpathlineto{\pgfqpoint{1.921353in}{2.151459in}}%
\pgfpathlineto{\pgfqpoint{1.987987in}{2.096399in}}%
\pgfpathlineto{\pgfqpoint{2.043515in}{2.051778in}}%
\pgfpathlineto{\pgfqpoint{2.099043in}{2.008456in}}%
\pgfpathlineto{\pgfqpoint{2.154572in}{1.966585in}}%
\pgfpathlineto{\pgfqpoint{2.210100in}{1.926318in}}%
\pgfpathlineto{\pgfqpoint{2.254523in}{1.895363in}}%
\pgfpathlineto{\pgfqpoint{2.298946in}{1.865609in}}%
\pgfpathlineto{\pgfqpoint{2.343368in}{1.837129in}}%
\pgfpathlineto{\pgfqpoint{2.387791in}{1.809990in}}%
\pgfpathlineto{\pgfqpoint{2.432214in}{1.784250in}}%
\pgfpathlineto{\pgfqpoint{2.476636in}{1.759959in}}%
\pgfpathlineto{\pgfqpoint{2.521059in}{1.737155in}}%
\pgfpathlineto{\pgfqpoint{2.565482in}{1.715864in}}%
\pgfpathlineto{\pgfqpoint{2.609905in}{1.696099in}}%
\pgfpathlineto{\pgfqpoint{2.654327in}{1.677858in}}%
\pgfpathlineto{\pgfqpoint{2.698750in}{1.661127in}}%
\pgfpathlineto{\pgfqpoint{2.743173in}{1.645879in}}%
\pgfpathlineto{\pgfqpoint{2.787595in}{1.632071in}}%
\pgfpathlineto{\pgfqpoint{2.832018in}{1.619654in}}%
\pgfpathlineto{\pgfqpoint{2.876441in}{1.608566in}}%
\pgfpathlineto{\pgfqpoint{2.920863in}{1.598740in}}%
\pgfpathlineto{\pgfqpoint{2.965286in}{1.590103in}}%
\pgfpathlineto{\pgfqpoint{3.009709in}{1.582576in}}%
\pgfpathlineto{\pgfqpoint{3.054132in}{1.576080in}}%
\pgfpathlineto{\pgfqpoint{3.098554in}{1.570537in}}%
\pgfpathlineto{\pgfqpoint{3.154083in}{1.564826in}}%
\pgfpathlineto{\pgfqpoint{3.209611in}{1.560331in}}%
\pgfpathlineto{\pgfqpoint{3.265139in}{1.556911in}}%
\pgfpathlineto{\pgfqpoint{3.331773in}{1.554042in}}%
\pgfpathlineto{\pgfqpoint{3.398407in}{1.552325in}}%
\pgfpathlineto{\pgfqpoint{3.476147in}{1.551538in}}%
\pgfpathlineto{\pgfqpoint{3.553887in}{1.551820in}}%
\pgfpathlineto{\pgfqpoint{3.642732in}{1.553185in}}%
\pgfpathlineto{\pgfqpoint{3.753789in}{1.556080in}}%
\pgfpathlineto{\pgfqpoint{3.887057in}{1.560798in}}%
\pgfpathlineto{\pgfqpoint{4.053642in}{1.567915in}}%
\pgfpathlineto{\pgfqpoint{4.286862in}{1.579093in}}%
\pgfpathlineto{\pgfqpoint{4.597820in}{1.594269in}}%
\pgfpathlineto{\pgfqpoint{4.620032in}{1.594516in}}%
\pgfpathlineto{\pgfqpoint{4.631137in}{1.594177in}}%
\pgfpathlineto{\pgfqpoint{4.642243in}{1.593260in}}%
\pgfpathlineto{\pgfqpoint{4.653349in}{1.591398in}}%
\pgfpathlineto{\pgfqpoint{4.664454in}{1.588012in}}%
\pgfpathlineto{\pgfqpoint{4.675560in}{1.582199in}}%
\pgfpathlineto{\pgfqpoint{4.686666in}{1.572608in}}%
\pgfpathlineto{\pgfqpoint{4.697772in}{1.557341in}}%
\pgfpathlineto{\pgfqpoint{4.708877in}{1.534024in}}%
\pgfpathlineto{\pgfqpoint{4.719983in}{1.500240in}}%
\pgfpathlineto{\pgfqpoint{4.731089in}{1.454538in}}%
\pgfpathlineto{\pgfqpoint{4.742194in}{1.397817in}}%
\pgfpathlineto{\pgfqpoint{4.764406in}{1.270595in}}%
\pgfpathlineto{\pgfqpoint{4.775511in}{1.214037in}}%
\pgfpathlineto{\pgfqpoint{4.786617in}{1.169461in}}%
\pgfpathlineto{\pgfqpoint{4.797723in}{1.138278in}}%
\pgfpathlineto{\pgfqpoint{4.808828in}{1.119052in}}%
\pgfpathlineto{\pgfqpoint{4.819934in}{1.109038in}}%
\pgfpathlineto{\pgfqpoint{4.831040in}{1.105453in}}%
\pgfpathlineto{\pgfqpoint{4.842145in}{1.106076in}}%
\pgfpathlineto{\pgfqpoint{4.853251in}{1.109340in}}%
\pgfpathlineto{\pgfqpoint{4.864357in}{1.114222in}}%
\pgfpathlineto{\pgfqpoint{4.886568in}{1.126514in}}%
\pgfpathlineto{\pgfqpoint{4.919885in}{1.147387in}}%
\pgfpathlineto{\pgfqpoint{5.019836in}{1.212614in}}%
\pgfpathlineto{\pgfqpoint{5.164210in}{1.306541in}}%
\pgfpathlineto{\pgfqpoint{5.219738in}{1.341658in}}%
\pgfpathlineto{\pgfqpoint{5.264161in}{1.368410in}}%
\pgfpathlineto{\pgfqpoint{5.297478in}{1.386975in}}%
\pgfpathlineto{\pgfqpoint{5.319689in}{1.398269in}}%
\pgfpathlineto{\pgfqpoint{5.341901in}{1.408442in}}%
\pgfpathlineto{\pgfqpoint{5.364112in}{1.417304in}}%
\pgfpathlineto{\pgfqpoint{5.386323in}{1.424737in}}%
\pgfpathlineto{\pgfqpoint{5.408535in}{1.430729in}}%
\pgfpathlineto{\pgfqpoint{5.430746in}{1.435378in}}%
\pgfpathlineto{\pgfqpoint{5.452957in}{1.438862in}}%
\pgfpathlineto{\pgfqpoint{5.475169in}{1.441397in}}%
\pgfpathlineto{\pgfqpoint{5.508486in}{1.443887in}}%
\pgfpathlineto{\pgfqpoint{5.552909in}{1.445653in}}%
\pgfpathlineto{\pgfqpoint{5.608437in}{1.446597in}}%
\pgfpathlineto{\pgfqpoint{5.719494in}{1.447081in}}%
\pgfpathlineto{\pgfqpoint{5.825000in}{1.447145in}}%
\pgfpathlineto{\pgfqpoint{5.825000in}{1.447145in}}%
\pgfusepath{stroke}%
\end{pgfscope}%
\begin{pgfscope}%
\pgfsetrectcap%
\pgfsetmiterjoin%
\pgfsetlinewidth{1.254687pt}%
\definecolor{currentstroke}{rgb}{1.000000,1.000000,1.000000}%
\pgfsetstrokecolor{currentstroke}%
\pgfsetdash{}{0pt}%
\pgfpathmoveto{\pgfqpoint{0.832441in}{0.754477in}}%
\pgfpathlineto{\pgfqpoint{0.832441in}{3.115885in}}%
\pgfusepath{stroke}%
\end{pgfscope}%
\begin{pgfscope}%
\pgfsetrectcap%
\pgfsetmiterjoin%
\pgfsetlinewidth{1.254687pt}%
\definecolor{currentstroke}{rgb}{1.000000,1.000000,1.000000}%
\pgfsetstrokecolor{currentstroke}%
\pgfsetdash{}{0pt}%
\pgfpathmoveto{\pgfqpoint{5.820000in}{0.754477in}}%
\pgfpathlineto{\pgfqpoint{5.820000in}{3.115885in}}%
\pgfusepath{stroke}%
\end{pgfscope}%
\begin{pgfscope}%
\pgfsetrectcap%
\pgfsetmiterjoin%
\pgfsetlinewidth{1.254687pt}%
\definecolor{currentstroke}{rgb}{1.000000,1.000000,1.000000}%
\pgfsetstrokecolor{currentstroke}%
\pgfsetdash{}{0pt}%
\pgfpathmoveto{\pgfqpoint{0.832441in}{0.754477in}}%
\pgfpathlineto{\pgfqpoint{5.820000in}{0.754477in}}%
\pgfusepath{stroke}%
\end{pgfscope}%
\begin{pgfscope}%
\pgfsetrectcap%
\pgfsetmiterjoin%
\pgfsetlinewidth{1.254687pt}%
\definecolor{currentstroke}{rgb}{1.000000,1.000000,1.000000}%
\pgfsetstrokecolor{currentstroke}%
\pgfsetdash{}{0pt}%
\pgfpathmoveto{\pgfqpoint{0.832441in}{3.115885in}}%
\pgfpathlineto{\pgfqpoint{5.820000in}{3.115885in}}%
\pgfusepath{stroke}%
\end{pgfscope}%
\begin{pgfscope}%
\pgfsetfillopacity{0.500000}%
\pgfsetstrokeopacity{0.500000}%
\definecolor{textcolor}{rgb}{0.150000,0.150000,0.150000}%
\pgfsetstrokecolor{textcolor}%
\pgfsetfillcolor{textcolor}%
\pgftext[x=0.932192in,y=0.801705in,left,base]{\color{textcolor}\sffamily\fontsize{9.000000}{10.800000}\selectfont Data from Panton: Incompressible Flow (4th Ed.)}%
\end{pgfscope}%
\begin{pgfscope}%
\definecolor{textcolor}{rgb}{0.150000,0.150000,0.150000}%
\pgfsetstrokecolor{textcolor}%
\pgfsetfillcolor{textcolor}%
\pgftext[x=3.326221in,y=3.199219in,,base]{\color{textcolor}\sffamily\fontsize{12.000000}{14.400000}\selectfont Cylinder Drag Coefficient}%
\end{pgfscope}%
\begin{pgfscope}%
\pgfsetbuttcap%
\pgfsetmiterjoin%
\definecolor{currentfill}{rgb}{0.917647,0.917647,0.949020}%
\pgfsetfillcolor{currentfill}%
\pgfsetfillopacity{0.800000}%
\pgfsetlinewidth{1.003750pt}%
\definecolor{currentstroke}{rgb}{0.800000,0.800000,0.800000}%
\pgfsetstrokecolor{currentstroke}%
\pgfsetstrokeopacity{0.800000}%
\pgfsetdash{}{0pt}%
\pgfpathmoveto{\pgfqpoint{4.415819in}{2.567719in}}%
\pgfpathlineto{\pgfqpoint{5.713056in}{2.567719in}}%
\pgfpathquadraticcurveto{\pgfqpoint{5.743611in}{2.567719in}}{\pgfqpoint{5.743611in}{2.598275in}}%
\pgfpathlineto{\pgfqpoint{5.743611in}{3.008941in}}%
\pgfpathquadraticcurveto{\pgfqpoint{5.743611in}{3.039497in}}{\pgfqpoint{5.713056in}{3.039497in}}%
\pgfpathlineto{\pgfqpoint{4.415819in}{3.039497in}}%
\pgfpathquadraticcurveto{\pgfqpoint{4.385264in}{3.039497in}}{\pgfqpoint{4.385264in}{3.008941in}}%
\pgfpathlineto{\pgfqpoint{4.385264in}{2.598275in}}%
\pgfpathquadraticcurveto{\pgfqpoint{4.385264in}{2.567719in}}{\pgfqpoint{4.415819in}{2.567719in}}%
\pgfpathclose%
\pgfusepath{stroke,fill}%
\end{pgfscope}%
\begin{pgfscope}%
\pgfsetbuttcap%
\pgfsetroundjoin%
\definecolor{currentfill}{rgb}{0.000000,0.000000,0.000000}%
\pgfsetfillcolor{currentfill}%
\pgfsetfillopacity{0.800000}%
\pgfsetlinewidth{1.003750pt}%
\definecolor{currentstroke}{rgb}{0.000000,0.000000,0.000000}%
\pgfsetstrokecolor{currentstroke}%
\pgfsetstrokeopacity{0.800000}%
\pgfsetdash{}{0pt}%
\pgfsys@defobject{currentmarker}{\pgfqpoint{-0.010417in}{-0.010417in}}{\pgfqpoint{0.010417in}{0.010417in}}{%
\pgfpathmoveto{\pgfqpoint{0.000000in}{-0.010417in}}%
\pgfpathcurveto{\pgfqpoint{0.002763in}{-0.010417in}}{\pgfqpoint{0.005412in}{-0.009319in}}{\pgfqpoint{0.007366in}{-0.007366in}}%
\pgfpathcurveto{\pgfqpoint{0.009319in}{-0.005412in}}{\pgfqpoint{0.010417in}{-0.002763in}}{\pgfqpoint{0.010417in}{0.000000in}}%
\pgfpathcurveto{\pgfqpoint{0.010417in}{0.002763in}}{\pgfqpoint{0.009319in}{0.005412in}}{\pgfqpoint{0.007366in}{0.007366in}}%
\pgfpathcurveto{\pgfqpoint{0.005412in}{0.009319in}}{\pgfqpoint{0.002763in}{0.010417in}}{\pgfqpoint{0.000000in}{0.010417in}}%
\pgfpathcurveto{\pgfqpoint{-0.002763in}{0.010417in}}{\pgfqpoint{-0.005412in}{0.009319in}}{\pgfqpoint{-0.007366in}{0.007366in}}%
\pgfpathcurveto{\pgfqpoint{-0.009319in}{0.005412in}}{\pgfqpoint{-0.010417in}{0.002763in}}{\pgfqpoint{-0.010417in}{0.000000in}}%
\pgfpathcurveto{\pgfqpoint{-0.010417in}{-0.002763in}}{\pgfqpoint{-0.009319in}{-0.005412in}}{\pgfqpoint{-0.007366in}{-0.007366in}}%
\pgfpathcurveto{\pgfqpoint{-0.005412in}{-0.009319in}}{\pgfqpoint{-0.002763in}{-0.010417in}}{\pgfqpoint{0.000000in}{-0.010417in}}%
\pgfpathclose%
\pgfusepath{stroke,fill}%
}%
\begin{pgfscope}%
\pgfsys@transformshift{4.599153in}{2.924913in}%
\pgfsys@useobject{currentmarker}{}%
\end{pgfscope}%
\end{pgfscope}%
\begin{pgfscope}%
\definecolor{textcolor}{rgb}{0.150000,0.150000,0.150000}%
\pgfsetstrokecolor{textcolor}%
\pgfsetfillcolor{textcolor}%
\pgftext[x=4.874153in,y=2.871441in,left,base]{\color{textcolor}\sffamily\fontsize{11.000000}{13.200000}\selectfont Data}%
\end{pgfscope}%
\begin{pgfscope}%
\pgfsetroundcap%
\pgfsetroundjoin%
\pgfsetlinewidth{1.505625pt}%
\definecolor{currentstroke}{rgb}{0.258824,0.521569,0.956863}%
\pgfsetstrokecolor{currentstroke}%
\pgfsetstrokeopacity{0.800000}%
\pgfsetdash{}{0pt}%
\pgfpathmoveto{\pgfqpoint{4.446375in}{2.711941in}}%
\pgfpathlineto{\pgfqpoint{4.751931in}{2.711941in}}%
\pgfusepath{stroke}%
\end{pgfscope}%
\begin{pgfscope}%
\definecolor{textcolor}{rgb}{0.150000,0.150000,0.150000}%
\pgfsetstrokecolor{textcolor}%
\pgfsetfillcolor{textcolor}%
\pgftext[x=4.874153in,y=2.658469in,left,base]{\color{textcolor}\sffamily\fontsize{11.000000}{13.200000}\selectfont Fitted Model}%
\end{pgfscope}%
\end{pgfpicture}%
\makeatother%
\endgroup%
}
    \ifdraft{}{\centerline{%% Creator: Matplotlib, PGF backend
%%
%% To include the figure in your LaTeX document, write
%%   \input{<filename>.pgf}
%%
%% Make sure the required packages are loaded in your preamble
%%   \usepackage{pgf}
%%
%% Figures using additional raster images can only be included by \input if
%% they are in the same directory as the main LaTeX file. For loading figures
%% from other directories you can use the `import` package
%%   \usepackage{import}
%%
%% and then include the figures with
%%   \import{<path to file>}{<filename>.pgf}
%%
%% Matplotlib used the following preamble
%%   \usepackage{fontspec}
%%
\begingroup%
\makeatletter%
\begin{pgfpicture}%
\pgfpathrectangle{\pgfpointorigin}{\pgfqpoint{6.000000in}{3.500000in}}%
\pgfusepath{use as bounding box, clip}%
\begin{pgfscope}%
\pgfsetbuttcap%
\pgfsetmiterjoin%
\definecolor{currentfill}{rgb}{1.000000,1.000000,1.000000}%
\pgfsetfillcolor{currentfill}%
\pgfsetlinewidth{0.000000pt}%
\definecolor{currentstroke}{rgb}{1.000000,1.000000,1.000000}%
\pgfsetstrokecolor{currentstroke}%
\pgfsetstrokeopacity{0.000000}%
\pgfsetdash{}{0pt}%
\pgfpathmoveto{\pgfqpoint{0.000000in}{0.000000in}}%
\pgfpathlineto{\pgfqpoint{6.000000in}{0.000000in}}%
\pgfpathlineto{\pgfqpoint{6.000000in}{3.500000in}}%
\pgfpathlineto{\pgfqpoint{0.000000in}{3.500000in}}%
\pgfpathclose%
\pgfusepath{fill}%
\end{pgfscope}%
\begin{pgfscope}%
\pgfsetbuttcap%
\pgfsetmiterjoin%
\definecolor{currentfill}{rgb}{0.917647,0.917647,0.949020}%
\pgfsetfillcolor{currentfill}%
\pgfsetlinewidth{0.000000pt}%
\definecolor{currentstroke}{rgb}{0.000000,0.000000,0.000000}%
\pgfsetstrokecolor{currentstroke}%
\pgfsetstrokeopacity{0.000000}%
\pgfsetdash{}{0pt}%
\pgfpathmoveto{\pgfqpoint{0.832441in}{0.754477in}}%
\pgfpathlineto{\pgfqpoint{5.820000in}{0.754477in}}%
\pgfpathlineto{\pgfqpoint{5.820000in}{3.115885in}}%
\pgfpathlineto{\pgfqpoint{0.832441in}{3.115885in}}%
\pgfpathclose%
\pgfusepath{fill}%
\end{pgfscope}%
\begin{pgfscope}%
\pgfpathrectangle{\pgfqpoint{0.832441in}{0.754477in}}{\pgfqpoint{4.987559in}{2.361409in}}%
\pgfusepath{clip}%
\pgfsetroundcap%
\pgfsetroundjoin%
\pgfsetlinewidth{1.606000pt}%
\definecolor{currentstroke}{rgb}{1.000000,1.000000,1.000000}%
\pgfsetstrokecolor{currentstroke}%
\pgfsetdash{}{0pt}%
\pgfpathmoveto{\pgfqpoint{1.109528in}{0.754477in}}%
\pgfpathlineto{\pgfqpoint{1.109528in}{3.115885in}}%
\pgfusepath{stroke}%
\end{pgfscope}%
\begin{pgfscope}%
\definecolor{textcolor}{rgb}{0.150000,0.150000,0.150000}%
\pgfsetstrokecolor{textcolor}%
\pgfsetfillcolor{textcolor}%
\pgftext[x=1.109528in,y=0.622532in,,top]{\color{textcolor}\sffamily\fontsize{11.000000}{13.200000}\selectfont \(\displaystyle {10^{-1}}\)}%
\end{pgfscope}%
\begin{pgfscope}%
\pgfpathrectangle{\pgfqpoint{0.832441in}{0.754477in}}{\pgfqpoint{4.987559in}{2.361409in}}%
\pgfusepath{clip}%
\pgfsetroundcap%
\pgfsetroundjoin%
\pgfsetlinewidth{1.606000pt}%
\definecolor{currentstroke}{rgb}{1.000000,1.000000,1.000000}%
\pgfsetstrokecolor{currentstroke}%
\pgfsetdash{}{0pt}%
\pgfpathmoveto{\pgfqpoint{1.663701in}{0.754477in}}%
\pgfpathlineto{\pgfqpoint{1.663701in}{3.115885in}}%
\pgfusepath{stroke}%
\end{pgfscope}%
\begin{pgfscope}%
\definecolor{textcolor}{rgb}{0.150000,0.150000,0.150000}%
\pgfsetstrokecolor{textcolor}%
\pgfsetfillcolor{textcolor}%
\pgftext[x=1.663701in,y=0.622532in,,top]{\color{textcolor}\sffamily\fontsize{11.000000}{13.200000}\selectfont \(\displaystyle {10^{0}}\)}%
\end{pgfscope}%
\begin{pgfscope}%
\pgfpathrectangle{\pgfqpoint{0.832441in}{0.754477in}}{\pgfqpoint{4.987559in}{2.361409in}}%
\pgfusepath{clip}%
\pgfsetroundcap%
\pgfsetroundjoin%
\pgfsetlinewidth{1.606000pt}%
\definecolor{currentstroke}{rgb}{1.000000,1.000000,1.000000}%
\pgfsetstrokecolor{currentstroke}%
\pgfsetdash{}{0pt}%
\pgfpathmoveto{\pgfqpoint{2.217874in}{0.754477in}}%
\pgfpathlineto{\pgfqpoint{2.217874in}{3.115885in}}%
\pgfusepath{stroke}%
\end{pgfscope}%
\begin{pgfscope}%
\definecolor{textcolor}{rgb}{0.150000,0.150000,0.150000}%
\pgfsetstrokecolor{textcolor}%
\pgfsetfillcolor{textcolor}%
\pgftext[x=2.217874in,y=0.622532in,,top]{\color{textcolor}\sffamily\fontsize{11.000000}{13.200000}\selectfont \(\displaystyle {10^{1}}\)}%
\end{pgfscope}%
\begin{pgfscope}%
\pgfpathrectangle{\pgfqpoint{0.832441in}{0.754477in}}{\pgfqpoint{4.987559in}{2.361409in}}%
\pgfusepath{clip}%
\pgfsetroundcap%
\pgfsetroundjoin%
\pgfsetlinewidth{1.606000pt}%
\definecolor{currentstroke}{rgb}{1.000000,1.000000,1.000000}%
\pgfsetstrokecolor{currentstroke}%
\pgfsetdash{}{0pt}%
\pgfpathmoveto{\pgfqpoint{2.772047in}{0.754477in}}%
\pgfpathlineto{\pgfqpoint{2.772047in}{3.115885in}}%
\pgfusepath{stroke}%
\end{pgfscope}%
\begin{pgfscope}%
\definecolor{textcolor}{rgb}{0.150000,0.150000,0.150000}%
\pgfsetstrokecolor{textcolor}%
\pgfsetfillcolor{textcolor}%
\pgftext[x=2.772047in,y=0.622532in,,top]{\color{textcolor}\sffamily\fontsize{11.000000}{13.200000}\selectfont \(\displaystyle {10^{2}}\)}%
\end{pgfscope}%
\begin{pgfscope}%
\pgfpathrectangle{\pgfqpoint{0.832441in}{0.754477in}}{\pgfqpoint{4.987559in}{2.361409in}}%
\pgfusepath{clip}%
\pgfsetroundcap%
\pgfsetroundjoin%
\pgfsetlinewidth{1.606000pt}%
\definecolor{currentstroke}{rgb}{1.000000,1.000000,1.000000}%
\pgfsetstrokecolor{currentstroke}%
\pgfsetdash{}{0pt}%
\pgfpathmoveto{\pgfqpoint{3.326221in}{0.754477in}}%
\pgfpathlineto{\pgfqpoint{3.326221in}{3.115885in}}%
\pgfusepath{stroke}%
\end{pgfscope}%
\begin{pgfscope}%
\definecolor{textcolor}{rgb}{0.150000,0.150000,0.150000}%
\pgfsetstrokecolor{textcolor}%
\pgfsetfillcolor{textcolor}%
\pgftext[x=3.326221in,y=0.622532in,,top]{\color{textcolor}\sffamily\fontsize{11.000000}{13.200000}\selectfont \(\displaystyle {10^{3}}\)}%
\end{pgfscope}%
\begin{pgfscope}%
\pgfpathrectangle{\pgfqpoint{0.832441in}{0.754477in}}{\pgfqpoint{4.987559in}{2.361409in}}%
\pgfusepath{clip}%
\pgfsetroundcap%
\pgfsetroundjoin%
\pgfsetlinewidth{1.606000pt}%
\definecolor{currentstroke}{rgb}{1.000000,1.000000,1.000000}%
\pgfsetstrokecolor{currentstroke}%
\pgfsetdash{}{0pt}%
\pgfpathmoveto{\pgfqpoint{3.880394in}{0.754477in}}%
\pgfpathlineto{\pgfqpoint{3.880394in}{3.115885in}}%
\pgfusepath{stroke}%
\end{pgfscope}%
\begin{pgfscope}%
\definecolor{textcolor}{rgb}{0.150000,0.150000,0.150000}%
\pgfsetstrokecolor{textcolor}%
\pgfsetfillcolor{textcolor}%
\pgftext[x=3.880394in,y=0.622532in,,top]{\color{textcolor}\sffamily\fontsize{11.000000}{13.200000}\selectfont \(\displaystyle {10^{4}}\)}%
\end{pgfscope}%
\begin{pgfscope}%
\pgfpathrectangle{\pgfqpoint{0.832441in}{0.754477in}}{\pgfqpoint{4.987559in}{2.361409in}}%
\pgfusepath{clip}%
\pgfsetroundcap%
\pgfsetroundjoin%
\pgfsetlinewidth{1.606000pt}%
\definecolor{currentstroke}{rgb}{1.000000,1.000000,1.000000}%
\pgfsetstrokecolor{currentstroke}%
\pgfsetdash{}{0pt}%
\pgfpathmoveto{\pgfqpoint{4.434567in}{0.754477in}}%
\pgfpathlineto{\pgfqpoint{4.434567in}{3.115885in}}%
\pgfusepath{stroke}%
\end{pgfscope}%
\begin{pgfscope}%
\definecolor{textcolor}{rgb}{0.150000,0.150000,0.150000}%
\pgfsetstrokecolor{textcolor}%
\pgfsetfillcolor{textcolor}%
\pgftext[x=4.434567in,y=0.622532in,,top]{\color{textcolor}\sffamily\fontsize{11.000000}{13.200000}\selectfont \(\displaystyle {10^{5}}\)}%
\end{pgfscope}%
\begin{pgfscope}%
\pgfpathrectangle{\pgfqpoint{0.832441in}{0.754477in}}{\pgfqpoint{4.987559in}{2.361409in}}%
\pgfusepath{clip}%
\pgfsetroundcap%
\pgfsetroundjoin%
\pgfsetlinewidth{1.606000pt}%
\definecolor{currentstroke}{rgb}{1.000000,1.000000,1.000000}%
\pgfsetstrokecolor{currentstroke}%
\pgfsetdash{}{0pt}%
\pgfpathmoveto{\pgfqpoint{4.988740in}{0.754477in}}%
\pgfpathlineto{\pgfqpoint{4.988740in}{3.115885in}}%
\pgfusepath{stroke}%
\end{pgfscope}%
\begin{pgfscope}%
\definecolor{textcolor}{rgb}{0.150000,0.150000,0.150000}%
\pgfsetstrokecolor{textcolor}%
\pgfsetfillcolor{textcolor}%
\pgftext[x=4.988740in,y=0.622532in,,top]{\color{textcolor}\sffamily\fontsize{11.000000}{13.200000}\selectfont \(\displaystyle {10^{6}}\)}%
\end{pgfscope}%
\begin{pgfscope}%
\pgfpathrectangle{\pgfqpoint{0.832441in}{0.754477in}}{\pgfqpoint{4.987559in}{2.361409in}}%
\pgfusepath{clip}%
\pgfsetroundcap%
\pgfsetroundjoin%
\pgfsetlinewidth{1.606000pt}%
\definecolor{currentstroke}{rgb}{1.000000,1.000000,1.000000}%
\pgfsetstrokecolor{currentstroke}%
\pgfsetdash{}{0pt}%
\pgfpathmoveto{\pgfqpoint{5.542913in}{0.754477in}}%
\pgfpathlineto{\pgfqpoint{5.542913in}{3.115885in}}%
\pgfusepath{stroke}%
\end{pgfscope}%
\begin{pgfscope}%
\definecolor{textcolor}{rgb}{0.150000,0.150000,0.150000}%
\pgfsetstrokecolor{textcolor}%
\pgfsetfillcolor{textcolor}%
\pgftext[x=5.542913in,y=0.622532in,,top]{\color{textcolor}\sffamily\fontsize{11.000000}{13.200000}\selectfont \(\displaystyle {10^{7}}\)}%
\end{pgfscope}%
\begin{pgfscope}%
\pgfpathrectangle{\pgfqpoint{0.832441in}{0.754477in}}{\pgfqpoint{4.987559in}{2.361409in}}%
\pgfusepath{clip}%
\pgfsetroundcap%
\pgfsetroundjoin%
\pgfsetlinewidth{0.702625pt}%
\definecolor{currentstroke}{rgb}{1.000000,1.000000,1.000000}%
\pgfsetstrokecolor{currentstroke}%
\pgfsetdash{}{0pt}%
\pgfpathmoveto{\pgfqpoint{0.889000in}{0.754477in}}%
\pgfpathlineto{\pgfqpoint{0.889000in}{3.115885in}}%
\pgfusepath{stroke}%
\end{pgfscope}%
\begin{pgfscope}%
\pgfpathrectangle{\pgfqpoint{0.832441in}{0.754477in}}{\pgfqpoint{4.987559in}{2.361409in}}%
\pgfusepath{clip}%
\pgfsetroundcap%
\pgfsetroundjoin%
\pgfsetlinewidth{0.702625pt}%
\definecolor{currentstroke}{rgb}{1.000000,1.000000,1.000000}%
\pgfsetstrokecolor{currentstroke}%
\pgfsetdash{}{0pt}%
\pgfpathmoveto{\pgfqpoint{0.942705in}{0.754477in}}%
\pgfpathlineto{\pgfqpoint{0.942705in}{3.115885in}}%
\pgfusepath{stroke}%
\end{pgfscope}%
\begin{pgfscope}%
\pgfpathrectangle{\pgfqpoint{0.832441in}{0.754477in}}{\pgfqpoint{4.987559in}{2.361409in}}%
\pgfusepath{clip}%
\pgfsetroundcap%
\pgfsetroundjoin%
\pgfsetlinewidth{0.702625pt}%
\definecolor{currentstroke}{rgb}{1.000000,1.000000,1.000000}%
\pgfsetstrokecolor{currentstroke}%
\pgfsetdash{}{0pt}%
\pgfpathmoveto{\pgfqpoint{0.986585in}{0.754477in}}%
\pgfpathlineto{\pgfqpoint{0.986585in}{3.115885in}}%
\pgfusepath{stroke}%
\end{pgfscope}%
\begin{pgfscope}%
\pgfpathrectangle{\pgfqpoint{0.832441in}{0.754477in}}{\pgfqpoint{4.987559in}{2.361409in}}%
\pgfusepath{clip}%
\pgfsetroundcap%
\pgfsetroundjoin%
\pgfsetlinewidth{0.702625pt}%
\definecolor{currentstroke}{rgb}{1.000000,1.000000,1.000000}%
\pgfsetstrokecolor{currentstroke}%
\pgfsetdash{}{0pt}%
\pgfpathmoveto{\pgfqpoint{1.023685in}{0.754477in}}%
\pgfpathlineto{\pgfqpoint{1.023685in}{3.115885in}}%
\pgfusepath{stroke}%
\end{pgfscope}%
\begin{pgfscope}%
\pgfpathrectangle{\pgfqpoint{0.832441in}{0.754477in}}{\pgfqpoint{4.987559in}{2.361409in}}%
\pgfusepath{clip}%
\pgfsetroundcap%
\pgfsetroundjoin%
\pgfsetlinewidth{0.702625pt}%
\definecolor{currentstroke}{rgb}{1.000000,1.000000,1.000000}%
\pgfsetstrokecolor{currentstroke}%
\pgfsetdash{}{0pt}%
\pgfpathmoveto{\pgfqpoint{1.055823in}{0.754477in}}%
\pgfpathlineto{\pgfqpoint{1.055823in}{3.115885in}}%
\pgfusepath{stroke}%
\end{pgfscope}%
\begin{pgfscope}%
\pgfpathrectangle{\pgfqpoint{0.832441in}{0.754477in}}{\pgfqpoint{4.987559in}{2.361409in}}%
\pgfusepath{clip}%
\pgfsetroundcap%
\pgfsetroundjoin%
\pgfsetlinewidth{0.702625pt}%
\definecolor{currentstroke}{rgb}{1.000000,1.000000,1.000000}%
\pgfsetstrokecolor{currentstroke}%
\pgfsetdash{}{0pt}%
\pgfpathmoveto{\pgfqpoint{1.084170in}{0.754477in}}%
\pgfpathlineto{\pgfqpoint{1.084170in}{3.115885in}}%
\pgfusepath{stroke}%
\end{pgfscope}%
\begin{pgfscope}%
\pgfpathrectangle{\pgfqpoint{0.832441in}{0.754477in}}{\pgfqpoint{4.987559in}{2.361409in}}%
\pgfusepath{clip}%
\pgfsetroundcap%
\pgfsetroundjoin%
\pgfsetlinewidth{0.702625pt}%
\definecolor{currentstroke}{rgb}{1.000000,1.000000,1.000000}%
\pgfsetstrokecolor{currentstroke}%
\pgfsetdash{}{0pt}%
\pgfpathmoveto{\pgfqpoint{1.276351in}{0.754477in}}%
\pgfpathlineto{\pgfqpoint{1.276351in}{3.115885in}}%
\pgfusepath{stroke}%
\end{pgfscope}%
\begin{pgfscope}%
\pgfpathrectangle{\pgfqpoint{0.832441in}{0.754477in}}{\pgfqpoint{4.987559in}{2.361409in}}%
\pgfusepath{clip}%
\pgfsetroundcap%
\pgfsetroundjoin%
\pgfsetlinewidth{0.702625pt}%
\definecolor{currentstroke}{rgb}{1.000000,1.000000,1.000000}%
\pgfsetstrokecolor{currentstroke}%
\pgfsetdash{}{0pt}%
\pgfpathmoveto{\pgfqpoint{1.373936in}{0.754477in}}%
\pgfpathlineto{\pgfqpoint{1.373936in}{3.115885in}}%
\pgfusepath{stroke}%
\end{pgfscope}%
\begin{pgfscope}%
\pgfpathrectangle{\pgfqpoint{0.832441in}{0.754477in}}{\pgfqpoint{4.987559in}{2.361409in}}%
\pgfusepath{clip}%
\pgfsetroundcap%
\pgfsetroundjoin%
\pgfsetlinewidth{0.702625pt}%
\definecolor{currentstroke}{rgb}{1.000000,1.000000,1.000000}%
\pgfsetstrokecolor{currentstroke}%
\pgfsetdash{}{0pt}%
\pgfpathmoveto{\pgfqpoint{1.443173in}{0.754477in}}%
\pgfpathlineto{\pgfqpoint{1.443173in}{3.115885in}}%
\pgfusepath{stroke}%
\end{pgfscope}%
\begin{pgfscope}%
\pgfpathrectangle{\pgfqpoint{0.832441in}{0.754477in}}{\pgfqpoint{4.987559in}{2.361409in}}%
\pgfusepath{clip}%
\pgfsetroundcap%
\pgfsetroundjoin%
\pgfsetlinewidth{0.702625pt}%
\definecolor{currentstroke}{rgb}{1.000000,1.000000,1.000000}%
\pgfsetstrokecolor{currentstroke}%
\pgfsetdash{}{0pt}%
\pgfpathmoveto{\pgfqpoint{1.496878in}{0.754477in}}%
\pgfpathlineto{\pgfqpoint{1.496878in}{3.115885in}}%
\pgfusepath{stroke}%
\end{pgfscope}%
\begin{pgfscope}%
\pgfpathrectangle{\pgfqpoint{0.832441in}{0.754477in}}{\pgfqpoint{4.987559in}{2.361409in}}%
\pgfusepath{clip}%
\pgfsetroundcap%
\pgfsetroundjoin%
\pgfsetlinewidth{0.702625pt}%
\definecolor{currentstroke}{rgb}{1.000000,1.000000,1.000000}%
\pgfsetstrokecolor{currentstroke}%
\pgfsetdash{}{0pt}%
\pgfpathmoveto{\pgfqpoint{1.540758in}{0.754477in}}%
\pgfpathlineto{\pgfqpoint{1.540758in}{3.115885in}}%
\pgfusepath{stroke}%
\end{pgfscope}%
\begin{pgfscope}%
\pgfpathrectangle{\pgfqpoint{0.832441in}{0.754477in}}{\pgfqpoint{4.987559in}{2.361409in}}%
\pgfusepath{clip}%
\pgfsetroundcap%
\pgfsetroundjoin%
\pgfsetlinewidth{0.702625pt}%
\definecolor{currentstroke}{rgb}{1.000000,1.000000,1.000000}%
\pgfsetstrokecolor{currentstroke}%
\pgfsetdash{}{0pt}%
\pgfpathmoveto{\pgfqpoint{1.577858in}{0.754477in}}%
\pgfpathlineto{\pgfqpoint{1.577858in}{3.115885in}}%
\pgfusepath{stroke}%
\end{pgfscope}%
\begin{pgfscope}%
\pgfpathrectangle{\pgfqpoint{0.832441in}{0.754477in}}{\pgfqpoint{4.987559in}{2.361409in}}%
\pgfusepath{clip}%
\pgfsetroundcap%
\pgfsetroundjoin%
\pgfsetlinewidth{0.702625pt}%
\definecolor{currentstroke}{rgb}{1.000000,1.000000,1.000000}%
\pgfsetstrokecolor{currentstroke}%
\pgfsetdash{}{0pt}%
\pgfpathmoveto{\pgfqpoint{1.609996in}{0.754477in}}%
\pgfpathlineto{\pgfqpoint{1.609996in}{3.115885in}}%
\pgfusepath{stroke}%
\end{pgfscope}%
\begin{pgfscope}%
\pgfpathrectangle{\pgfqpoint{0.832441in}{0.754477in}}{\pgfqpoint{4.987559in}{2.361409in}}%
\pgfusepath{clip}%
\pgfsetroundcap%
\pgfsetroundjoin%
\pgfsetlinewidth{0.702625pt}%
\definecolor{currentstroke}{rgb}{1.000000,1.000000,1.000000}%
\pgfsetstrokecolor{currentstroke}%
\pgfsetdash{}{0pt}%
\pgfpathmoveto{\pgfqpoint{1.638343in}{0.754477in}}%
\pgfpathlineto{\pgfqpoint{1.638343in}{3.115885in}}%
\pgfusepath{stroke}%
\end{pgfscope}%
\begin{pgfscope}%
\pgfpathrectangle{\pgfqpoint{0.832441in}{0.754477in}}{\pgfqpoint{4.987559in}{2.361409in}}%
\pgfusepath{clip}%
\pgfsetroundcap%
\pgfsetroundjoin%
\pgfsetlinewidth{0.702625pt}%
\definecolor{currentstroke}{rgb}{1.000000,1.000000,1.000000}%
\pgfsetstrokecolor{currentstroke}%
\pgfsetdash{}{0pt}%
\pgfpathmoveto{\pgfqpoint{1.830524in}{0.754477in}}%
\pgfpathlineto{\pgfqpoint{1.830524in}{3.115885in}}%
\pgfusepath{stroke}%
\end{pgfscope}%
\begin{pgfscope}%
\pgfpathrectangle{\pgfqpoint{0.832441in}{0.754477in}}{\pgfqpoint{4.987559in}{2.361409in}}%
\pgfusepath{clip}%
\pgfsetroundcap%
\pgfsetroundjoin%
\pgfsetlinewidth{0.702625pt}%
\definecolor{currentstroke}{rgb}{1.000000,1.000000,1.000000}%
\pgfsetstrokecolor{currentstroke}%
\pgfsetdash{}{0pt}%
\pgfpathmoveto{\pgfqpoint{1.928109in}{0.754477in}}%
\pgfpathlineto{\pgfqpoint{1.928109in}{3.115885in}}%
\pgfusepath{stroke}%
\end{pgfscope}%
\begin{pgfscope}%
\pgfpathrectangle{\pgfqpoint{0.832441in}{0.754477in}}{\pgfqpoint{4.987559in}{2.361409in}}%
\pgfusepath{clip}%
\pgfsetroundcap%
\pgfsetroundjoin%
\pgfsetlinewidth{0.702625pt}%
\definecolor{currentstroke}{rgb}{1.000000,1.000000,1.000000}%
\pgfsetstrokecolor{currentstroke}%
\pgfsetdash{}{0pt}%
\pgfpathmoveto{\pgfqpoint{1.997347in}{0.754477in}}%
\pgfpathlineto{\pgfqpoint{1.997347in}{3.115885in}}%
\pgfusepath{stroke}%
\end{pgfscope}%
\begin{pgfscope}%
\pgfpathrectangle{\pgfqpoint{0.832441in}{0.754477in}}{\pgfqpoint{4.987559in}{2.361409in}}%
\pgfusepath{clip}%
\pgfsetroundcap%
\pgfsetroundjoin%
\pgfsetlinewidth{0.702625pt}%
\definecolor{currentstroke}{rgb}{1.000000,1.000000,1.000000}%
\pgfsetstrokecolor{currentstroke}%
\pgfsetdash{}{0pt}%
\pgfpathmoveto{\pgfqpoint{2.051051in}{0.754477in}}%
\pgfpathlineto{\pgfqpoint{2.051051in}{3.115885in}}%
\pgfusepath{stroke}%
\end{pgfscope}%
\begin{pgfscope}%
\pgfpathrectangle{\pgfqpoint{0.832441in}{0.754477in}}{\pgfqpoint{4.987559in}{2.361409in}}%
\pgfusepath{clip}%
\pgfsetroundcap%
\pgfsetroundjoin%
\pgfsetlinewidth{0.702625pt}%
\definecolor{currentstroke}{rgb}{1.000000,1.000000,1.000000}%
\pgfsetstrokecolor{currentstroke}%
\pgfsetdash{}{0pt}%
\pgfpathmoveto{\pgfqpoint{2.094932in}{0.754477in}}%
\pgfpathlineto{\pgfqpoint{2.094932in}{3.115885in}}%
\pgfusepath{stroke}%
\end{pgfscope}%
\begin{pgfscope}%
\pgfpathrectangle{\pgfqpoint{0.832441in}{0.754477in}}{\pgfqpoint{4.987559in}{2.361409in}}%
\pgfusepath{clip}%
\pgfsetroundcap%
\pgfsetroundjoin%
\pgfsetlinewidth{0.702625pt}%
\definecolor{currentstroke}{rgb}{1.000000,1.000000,1.000000}%
\pgfsetstrokecolor{currentstroke}%
\pgfsetdash{}{0pt}%
\pgfpathmoveto{\pgfqpoint{2.132032in}{0.754477in}}%
\pgfpathlineto{\pgfqpoint{2.132032in}{3.115885in}}%
\pgfusepath{stroke}%
\end{pgfscope}%
\begin{pgfscope}%
\pgfpathrectangle{\pgfqpoint{0.832441in}{0.754477in}}{\pgfqpoint{4.987559in}{2.361409in}}%
\pgfusepath{clip}%
\pgfsetroundcap%
\pgfsetroundjoin%
\pgfsetlinewidth{0.702625pt}%
\definecolor{currentstroke}{rgb}{1.000000,1.000000,1.000000}%
\pgfsetstrokecolor{currentstroke}%
\pgfsetdash{}{0pt}%
\pgfpathmoveto{\pgfqpoint{2.164169in}{0.754477in}}%
\pgfpathlineto{\pgfqpoint{2.164169in}{3.115885in}}%
\pgfusepath{stroke}%
\end{pgfscope}%
\begin{pgfscope}%
\pgfpathrectangle{\pgfqpoint{0.832441in}{0.754477in}}{\pgfqpoint{4.987559in}{2.361409in}}%
\pgfusepath{clip}%
\pgfsetroundcap%
\pgfsetroundjoin%
\pgfsetlinewidth{0.702625pt}%
\definecolor{currentstroke}{rgb}{1.000000,1.000000,1.000000}%
\pgfsetstrokecolor{currentstroke}%
\pgfsetdash{}{0pt}%
\pgfpathmoveto{\pgfqpoint{2.192517in}{0.754477in}}%
\pgfpathlineto{\pgfqpoint{2.192517in}{3.115885in}}%
\pgfusepath{stroke}%
\end{pgfscope}%
\begin{pgfscope}%
\pgfpathrectangle{\pgfqpoint{0.832441in}{0.754477in}}{\pgfqpoint{4.987559in}{2.361409in}}%
\pgfusepath{clip}%
\pgfsetroundcap%
\pgfsetroundjoin%
\pgfsetlinewidth{0.702625pt}%
\definecolor{currentstroke}{rgb}{1.000000,1.000000,1.000000}%
\pgfsetstrokecolor{currentstroke}%
\pgfsetdash{}{0pt}%
\pgfpathmoveto{\pgfqpoint{2.384697in}{0.754477in}}%
\pgfpathlineto{\pgfqpoint{2.384697in}{3.115885in}}%
\pgfusepath{stroke}%
\end{pgfscope}%
\begin{pgfscope}%
\pgfpathrectangle{\pgfqpoint{0.832441in}{0.754477in}}{\pgfqpoint{4.987559in}{2.361409in}}%
\pgfusepath{clip}%
\pgfsetroundcap%
\pgfsetroundjoin%
\pgfsetlinewidth{0.702625pt}%
\definecolor{currentstroke}{rgb}{1.000000,1.000000,1.000000}%
\pgfsetstrokecolor{currentstroke}%
\pgfsetdash{}{0pt}%
\pgfpathmoveto{\pgfqpoint{2.482282in}{0.754477in}}%
\pgfpathlineto{\pgfqpoint{2.482282in}{3.115885in}}%
\pgfusepath{stroke}%
\end{pgfscope}%
\begin{pgfscope}%
\pgfpathrectangle{\pgfqpoint{0.832441in}{0.754477in}}{\pgfqpoint{4.987559in}{2.361409in}}%
\pgfusepath{clip}%
\pgfsetroundcap%
\pgfsetroundjoin%
\pgfsetlinewidth{0.702625pt}%
\definecolor{currentstroke}{rgb}{1.000000,1.000000,1.000000}%
\pgfsetstrokecolor{currentstroke}%
\pgfsetdash{}{0pt}%
\pgfpathmoveto{\pgfqpoint{2.551520in}{0.754477in}}%
\pgfpathlineto{\pgfqpoint{2.551520in}{3.115885in}}%
\pgfusepath{stroke}%
\end{pgfscope}%
\begin{pgfscope}%
\pgfpathrectangle{\pgfqpoint{0.832441in}{0.754477in}}{\pgfqpoint{4.987559in}{2.361409in}}%
\pgfusepath{clip}%
\pgfsetroundcap%
\pgfsetroundjoin%
\pgfsetlinewidth{0.702625pt}%
\definecolor{currentstroke}{rgb}{1.000000,1.000000,1.000000}%
\pgfsetstrokecolor{currentstroke}%
\pgfsetdash{}{0pt}%
\pgfpathmoveto{\pgfqpoint{2.605225in}{0.754477in}}%
\pgfpathlineto{\pgfqpoint{2.605225in}{3.115885in}}%
\pgfusepath{stroke}%
\end{pgfscope}%
\begin{pgfscope}%
\pgfpathrectangle{\pgfqpoint{0.832441in}{0.754477in}}{\pgfqpoint{4.987559in}{2.361409in}}%
\pgfusepath{clip}%
\pgfsetroundcap%
\pgfsetroundjoin%
\pgfsetlinewidth{0.702625pt}%
\definecolor{currentstroke}{rgb}{1.000000,1.000000,1.000000}%
\pgfsetstrokecolor{currentstroke}%
\pgfsetdash{}{0pt}%
\pgfpathmoveto{\pgfqpoint{2.649105in}{0.754477in}}%
\pgfpathlineto{\pgfqpoint{2.649105in}{3.115885in}}%
\pgfusepath{stroke}%
\end{pgfscope}%
\begin{pgfscope}%
\pgfpathrectangle{\pgfqpoint{0.832441in}{0.754477in}}{\pgfqpoint{4.987559in}{2.361409in}}%
\pgfusepath{clip}%
\pgfsetroundcap%
\pgfsetroundjoin%
\pgfsetlinewidth{0.702625pt}%
\definecolor{currentstroke}{rgb}{1.000000,1.000000,1.000000}%
\pgfsetstrokecolor{currentstroke}%
\pgfsetdash{}{0pt}%
\pgfpathmoveto{\pgfqpoint{2.686205in}{0.754477in}}%
\pgfpathlineto{\pgfqpoint{2.686205in}{3.115885in}}%
\pgfusepath{stroke}%
\end{pgfscope}%
\begin{pgfscope}%
\pgfpathrectangle{\pgfqpoint{0.832441in}{0.754477in}}{\pgfqpoint{4.987559in}{2.361409in}}%
\pgfusepath{clip}%
\pgfsetroundcap%
\pgfsetroundjoin%
\pgfsetlinewidth{0.702625pt}%
\definecolor{currentstroke}{rgb}{1.000000,1.000000,1.000000}%
\pgfsetstrokecolor{currentstroke}%
\pgfsetdash{}{0pt}%
\pgfpathmoveto{\pgfqpoint{2.718342in}{0.754477in}}%
\pgfpathlineto{\pgfqpoint{2.718342in}{3.115885in}}%
\pgfusepath{stroke}%
\end{pgfscope}%
\begin{pgfscope}%
\pgfpathrectangle{\pgfqpoint{0.832441in}{0.754477in}}{\pgfqpoint{4.987559in}{2.361409in}}%
\pgfusepath{clip}%
\pgfsetroundcap%
\pgfsetroundjoin%
\pgfsetlinewidth{0.702625pt}%
\definecolor{currentstroke}{rgb}{1.000000,1.000000,1.000000}%
\pgfsetstrokecolor{currentstroke}%
\pgfsetdash{}{0pt}%
\pgfpathmoveto{\pgfqpoint{2.746690in}{0.754477in}}%
\pgfpathlineto{\pgfqpoint{2.746690in}{3.115885in}}%
\pgfusepath{stroke}%
\end{pgfscope}%
\begin{pgfscope}%
\pgfpathrectangle{\pgfqpoint{0.832441in}{0.754477in}}{\pgfqpoint{4.987559in}{2.361409in}}%
\pgfusepath{clip}%
\pgfsetroundcap%
\pgfsetroundjoin%
\pgfsetlinewidth{0.702625pt}%
\definecolor{currentstroke}{rgb}{1.000000,1.000000,1.000000}%
\pgfsetstrokecolor{currentstroke}%
\pgfsetdash{}{0pt}%
\pgfpathmoveto{\pgfqpoint{2.938870in}{0.754477in}}%
\pgfpathlineto{\pgfqpoint{2.938870in}{3.115885in}}%
\pgfusepath{stroke}%
\end{pgfscope}%
\begin{pgfscope}%
\pgfpathrectangle{\pgfqpoint{0.832441in}{0.754477in}}{\pgfqpoint{4.987559in}{2.361409in}}%
\pgfusepath{clip}%
\pgfsetroundcap%
\pgfsetroundjoin%
\pgfsetlinewidth{0.702625pt}%
\definecolor{currentstroke}{rgb}{1.000000,1.000000,1.000000}%
\pgfsetstrokecolor{currentstroke}%
\pgfsetdash{}{0pt}%
\pgfpathmoveto{\pgfqpoint{3.036455in}{0.754477in}}%
\pgfpathlineto{\pgfqpoint{3.036455in}{3.115885in}}%
\pgfusepath{stroke}%
\end{pgfscope}%
\begin{pgfscope}%
\pgfpathrectangle{\pgfqpoint{0.832441in}{0.754477in}}{\pgfqpoint{4.987559in}{2.361409in}}%
\pgfusepath{clip}%
\pgfsetroundcap%
\pgfsetroundjoin%
\pgfsetlinewidth{0.702625pt}%
\definecolor{currentstroke}{rgb}{1.000000,1.000000,1.000000}%
\pgfsetstrokecolor{currentstroke}%
\pgfsetdash{}{0pt}%
\pgfpathmoveto{\pgfqpoint{3.105693in}{0.754477in}}%
\pgfpathlineto{\pgfqpoint{3.105693in}{3.115885in}}%
\pgfusepath{stroke}%
\end{pgfscope}%
\begin{pgfscope}%
\pgfpathrectangle{\pgfqpoint{0.832441in}{0.754477in}}{\pgfqpoint{4.987559in}{2.361409in}}%
\pgfusepath{clip}%
\pgfsetroundcap%
\pgfsetroundjoin%
\pgfsetlinewidth{0.702625pt}%
\definecolor{currentstroke}{rgb}{1.000000,1.000000,1.000000}%
\pgfsetstrokecolor{currentstroke}%
\pgfsetdash{}{0pt}%
\pgfpathmoveto{\pgfqpoint{3.159398in}{0.754477in}}%
\pgfpathlineto{\pgfqpoint{3.159398in}{3.115885in}}%
\pgfusepath{stroke}%
\end{pgfscope}%
\begin{pgfscope}%
\pgfpathrectangle{\pgfqpoint{0.832441in}{0.754477in}}{\pgfqpoint{4.987559in}{2.361409in}}%
\pgfusepath{clip}%
\pgfsetroundcap%
\pgfsetroundjoin%
\pgfsetlinewidth{0.702625pt}%
\definecolor{currentstroke}{rgb}{1.000000,1.000000,1.000000}%
\pgfsetstrokecolor{currentstroke}%
\pgfsetdash{}{0pt}%
\pgfpathmoveto{\pgfqpoint{3.203278in}{0.754477in}}%
\pgfpathlineto{\pgfqpoint{3.203278in}{3.115885in}}%
\pgfusepath{stroke}%
\end{pgfscope}%
\begin{pgfscope}%
\pgfpathrectangle{\pgfqpoint{0.832441in}{0.754477in}}{\pgfqpoint{4.987559in}{2.361409in}}%
\pgfusepath{clip}%
\pgfsetroundcap%
\pgfsetroundjoin%
\pgfsetlinewidth{0.702625pt}%
\definecolor{currentstroke}{rgb}{1.000000,1.000000,1.000000}%
\pgfsetstrokecolor{currentstroke}%
\pgfsetdash{}{0pt}%
\pgfpathmoveto{\pgfqpoint{3.240378in}{0.754477in}}%
\pgfpathlineto{\pgfqpoint{3.240378in}{3.115885in}}%
\pgfusepath{stroke}%
\end{pgfscope}%
\begin{pgfscope}%
\pgfpathrectangle{\pgfqpoint{0.832441in}{0.754477in}}{\pgfqpoint{4.987559in}{2.361409in}}%
\pgfusepath{clip}%
\pgfsetroundcap%
\pgfsetroundjoin%
\pgfsetlinewidth{0.702625pt}%
\definecolor{currentstroke}{rgb}{1.000000,1.000000,1.000000}%
\pgfsetstrokecolor{currentstroke}%
\pgfsetdash{}{0pt}%
\pgfpathmoveto{\pgfqpoint{3.272516in}{0.754477in}}%
\pgfpathlineto{\pgfqpoint{3.272516in}{3.115885in}}%
\pgfusepath{stroke}%
\end{pgfscope}%
\begin{pgfscope}%
\pgfpathrectangle{\pgfqpoint{0.832441in}{0.754477in}}{\pgfqpoint{4.987559in}{2.361409in}}%
\pgfusepath{clip}%
\pgfsetroundcap%
\pgfsetroundjoin%
\pgfsetlinewidth{0.702625pt}%
\definecolor{currentstroke}{rgb}{1.000000,1.000000,1.000000}%
\pgfsetstrokecolor{currentstroke}%
\pgfsetdash{}{0pt}%
\pgfpathmoveto{\pgfqpoint{3.300863in}{0.754477in}}%
\pgfpathlineto{\pgfqpoint{3.300863in}{3.115885in}}%
\pgfusepath{stroke}%
\end{pgfscope}%
\begin{pgfscope}%
\pgfpathrectangle{\pgfqpoint{0.832441in}{0.754477in}}{\pgfqpoint{4.987559in}{2.361409in}}%
\pgfusepath{clip}%
\pgfsetroundcap%
\pgfsetroundjoin%
\pgfsetlinewidth{0.702625pt}%
\definecolor{currentstroke}{rgb}{1.000000,1.000000,1.000000}%
\pgfsetstrokecolor{currentstroke}%
\pgfsetdash{}{0pt}%
\pgfpathmoveto{\pgfqpoint{3.493043in}{0.754477in}}%
\pgfpathlineto{\pgfqpoint{3.493043in}{3.115885in}}%
\pgfusepath{stroke}%
\end{pgfscope}%
\begin{pgfscope}%
\pgfpathrectangle{\pgfqpoint{0.832441in}{0.754477in}}{\pgfqpoint{4.987559in}{2.361409in}}%
\pgfusepath{clip}%
\pgfsetroundcap%
\pgfsetroundjoin%
\pgfsetlinewidth{0.702625pt}%
\definecolor{currentstroke}{rgb}{1.000000,1.000000,1.000000}%
\pgfsetstrokecolor{currentstroke}%
\pgfsetdash{}{0pt}%
\pgfpathmoveto{\pgfqpoint{3.590628in}{0.754477in}}%
\pgfpathlineto{\pgfqpoint{3.590628in}{3.115885in}}%
\pgfusepath{stroke}%
\end{pgfscope}%
\begin{pgfscope}%
\pgfpathrectangle{\pgfqpoint{0.832441in}{0.754477in}}{\pgfqpoint{4.987559in}{2.361409in}}%
\pgfusepath{clip}%
\pgfsetroundcap%
\pgfsetroundjoin%
\pgfsetlinewidth{0.702625pt}%
\definecolor{currentstroke}{rgb}{1.000000,1.000000,1.000000}%
\pgfsetstrokecolor{currentstroke}%
\pgfsetdash{}{0pt}%
\pgfpathmoveto{\pgfqpoint{3.659866in}{0.754477in}}%
\pgfpathlineto{\pgfqpoint{3.659866in}{3.115885in}}%
\pgfusepath{stroke}%
\end{pgfscope}%
\begin{pgfscope}%
\pgfpathrectangle{\pgfqpoint{0.832441in}{0.754477in}}{\pgfqpoint{4.987559in}{2.361409in}}%
\pgfusepath{clip}%
\pgfsetroundcap%
\pgfsetroundjoin%
\pgfsetlinewidth{0.702625pt}%
\definecolor{currentstroke}{rgb}{1.000000,1.000000,1.000000}%
\pgfsetstrokecolor{currentstroke}%
\pgfsetdash{}{0pt}%
\pgfpathmoveto{\pgfqpoint{3.713571in}{0.754477in}}%
\pgfpathlineto{\pgfqpoint{3.713571in}{3.115885in}}%
\pgfusepath{stroke}%
\end{pgfscope}%
\begin{pgfscope}%
\pgfpathrectangle{\pgfqpoint{0.832441in}{0.754477in}}{\pgfqpoint{4.987559in}{2.361409in}}%
\pgfusepath{clip}%
\pgfsetroundcap%
\pgfsetroundjoin%
\pgfsetlinewidth{0.702625pt}%
\definecolor{currentstroke}{rgb}{1.000000,1.000000,1.000000}%
\pgfsetstrokecolor{currentstroke}%
\pgfsetdash{}{0pt}%
\pgfpathmoveto{\pgfqpoint{3.757451in}{0.754477in}}%
\pgfpathlineto{\pgfqpoint{3.757451in}{3.115885in}}%
\pgfusepath{stroke}%
\end{pgfscope}%
\begin{pgfscope}%
\pgfpathrectangle{\pgfqpoint{0.832441in}{0.754477in}}{\pgfqpoint{4.987559in}{2.361409in}}%
\pgfusepath{clip}%
\pgfsetroundcap%
\pgfsetroundjoin%
\pgfsetlinewidth{0.702625pt}%
\definecolor{currentstroke}{rgb}{1.000000,1.000000,1.000000}%
\pgfsetstrokecolor{currentstroke}%
\pgfsetdash{}{0pt}%
\pgfpathmoveto{\pgfqpoint{3.794551in}{0.754477in}}%
\pgfpathlineto{\pgfqpoint{3.794551in}{3.115885in}}%
\pgfusepath{stroke}%
\end{pgfscope}%
\begin{pgfscope}%
\pgfpathrectangle{\pgfqpoint{0.832441in}{0.754477in}}{\pgfqpoint{4.987559in}{2.361409in}}%
\pgfusepath{clip}%
\pgfsetroundcap%
\pgfsetroundjoin%
\pgfsetlinewidth{0.702625pt}%
\definecolor{currentstroke}{rgb}{1.000000,1.000000,1.000000}%
\pgfsetstrokecolor{currentstroke}%
\pgfsetdash{}{0pt}%
\pgfpathmoveto{\pgfqpoint{3.826689in}{0.754477in}}%
\pgfpathlineto{\pgfqpoint{3.826689in}{3.115885in}}%
\pgfusepath{stroke}%
\end{pgfscope}%
\begin{pgfscope}%
\pgfpathrectangle{\pgfqpoint{0.832441in}{0.754477in}}{\pgfqpoint{4.987559in}{2.361409in}}%
\pgfusepath{clip}%
\pgfsetroundcap%
\pgfsetroundjoin%
\pgfsetlinewidth{0.702625pt}%
\definecolor{currentstroke}{rgb}{1.000000,1.000000,1.000000}%
\pgfsetstrokecolor{currentstroke}%
\pgfsetdash{}{0pt}%
\pgfpathmoveto{\pgfqpoint{3.855036in}{0.754477in}}%
\pgfpathlineto{\pgfqpoint{3.855036in}{3.115885in}}%
\pgfusepath{stroke}%
\end{pgfscope}%
\begin{pgfscope}%
\pgfpathrectangle{\pgfqpoint{0.832441in}{0.754477in}}{\pgfqpoint{4.987559in}{2.361409in}}%
\pgfusepath{clip}%
\pgfsetroundcap%
\pgfsetroundjoin%
\pgfsetlinewidth{0.702625pt}%
\definecolor{currentstroke}{rgb}{1.000000,1.000000,1.000000}%
\pgfsetstrokecolor{currentstroke}%
\pgfsetdash{}{0pt}%
\pgfpathmoveto{\pgfqpoint{4.047217in}{0.754477in}}%
\pgfpathlineto{\pgfqpoint{4.047217in}{3.115885in}}%
\pgfusepath{stroke}%
\end{pgfscope}%
\begin{pgfscope}%
\pgfpathrectangle{\pgfqpoint{0.832441in}{0.754477in}}{\pgfqpoint{4.987559in}{2.361409in}}%
\pgfusepath{clip}%
\pgfsetroundcap%
\pgfsetroundjoin%
\pgfsetlinewidth{0.702625pt}%
\definecolor{currentstroke}{rgb}{1.000000,1.000000,1.000000}%
\pgfsetstrokecolor{currentstroke}%
\pgfsetdash{}{0pt}%
\pgfpathmoveto{\pgfqpoint{4.144802in}{0.754477in}}%
\pgfpathlineto{\pgfqpoint{4.144802in}{3.115885in}}%
\pgfusepath{stroke}%
\end{pgfscope}%
\begin{pgfscope}%
\pgfpathrectangle{\pgfqpoint{0.832441in}{0.754477in}}{\pgfqpoint{4.987559in}{2.361409in}}%
\pgfusepath{clip}%
\pgfsetroundcap%
\pgfsetroundjoin%
\pgfsetlinewidth{0.702625pt}%
\definecolor{currentstroke}{rgb}{1.000000,1.000000,1.000000}%
\pgfsetstrokecolor{currentstroke}%
\pgfsetdash{}{0pt}%
\pgfpathmoveto{\pgfqpoint{4.214039in}{0.754477in}}%
\pgfpathlineto{\pgfqpoint{4.214039in}{3.115885in}}%
\pgfusepath{stroke}%
\end{pgfscope}%
\begin{pgfscope}%
\pgfpathrectangle{\pgfqpoint{0.832441in}{0.754477in}}{\pgfqpoint{4.987559in}{2.361409in}}%
\pgfusepath{clip}%
\pgfsetroundcap%
\pgfsetroundjoin%
\pgfsetlinewidth{0.702625pt}%
\definecolor{currentstroke}{rgb}{1.000000,1.000000,1.000000}%
\pgfsetstrokecolor{currentstroke}%
\pgfsetdash{}{0pt}%
\pgfpathmoveto{\pgfqpoint{4.267744in}{0.754477in}}%
\pgfpathlineto{\pgfqpoint{4.267744in}{3.115885in}}%
\pgfusepath{stroke}%
\end{pgfscope}%
\begin{pgfscope}%
\pgfpathrectangle{\pgfqpoint{0.832441in}{0.754477in}}{\pgfqpoint{4.987559in}{2.361409in}}%
\pgfusepath{clip}%
\pgfsetroundcap%
\pgfsetroundjoin%
\pgfsetlinewidth{0.702625pt}%
\definecolor{currentstroke}{rgb}{1.000000,1.000000,1.000000}%
\pgfsetstrokecolor{currentstroke}%
\pgfsetdash{}{0pt}%
\pgfpathmoveto{\pgfqpoint{4.311624in}{0.754477in}}%
\pgfpathlineto{\pgfqpoint{4.311624in}{3.115885in}}%
\pgfusepath{stroke}%
\end{pgfscope}%
\begin{pgfscope}%
\pgfpathrectangle{\pgfqpoint{0.832441in}{0.754477in}}{\pgfqpoint{4.987559in}{2.361409in}}%
\pgfusepath{clip}%
\pgfsetroundcap%
\pgfsetroundjoin%
\pgfsetlinewidth{0.702625pt}%
\definecolor{currentstroke}{rgb}{1.000000,1.000000,1.000000}%
\pgfsetstrokecolor{currentstroke}%
\pgfsetdash{}{0pt}%
\pgfpathmoveto{\pgfqpoint{4.348724in}{0.754477in}}%
\pgfpathlineto{\pgfqpoint{4.348724in}{3.115885in}}%
\pgfusepath{stroke}%
\end{pgfscope}%
\begin{pgfscope}%
\pgfpathrectangle{\pgfqpoint{0.832441in}{0.754477in}}{\pgfqpoint{4.987559in}{2.361409in}}%
\pgfusepath{clip}%
\pgfsetroundcap%
\pgfsetroundjoin%
\pgfsetlinewidth{0.702625pt}%
\definecolor{currentstroke}{rgb}{1.000000,1.000000,1.000000}%
\pgfsetstrokecolor{currentstroke}%
\pgfsetdash{}{0pt}%
\pgfpathmoveto{\pgfqpoint{4.380862in}{0.754477in}}%
\pgfpathlineto{\pgfqpoint{4.380862in}{3.115885in}}%
\pgfusepath{stroke}%
\end{pgfscope}%
\begin{pgfscope}%
\pgfpathrectangle{\pgfqpoint{0.832441in}{0.754477in}}{\pgfqpoint{4.987559in}{2.361409in}}%
\pgfusepath{clip}%
\pgfsetroundcap%
\pgfsetroundjoin%
\pgfsetlinewidth{0.702625pt}%
\definecolor{currentstroke}{rgb}{1.000000,1.000000,1.000000}%
\pgfsetstrokecolor{currentstroke}%
\pgfsetdash{}{0pt}%
\pgfpathmoveto{\pgfqpoint{4.409209in}{0.754477in}}%
\pgfpathlineto{\pgfqpoint{4.409209in}{3.115885in}}%
\pgfusepath{stroke}%
\end{pgfscope}%
\begin{pgfscope}%
\pgfpathrectangle{\pgfqpoint{0.832441in}{0.754477in}}{\pgfqpoint{4.987559in}{2.361409in}}%
\pgfusepath{clip}%
\pgfsetroundcap%
\pgfsetroundjoin%
\pgfsetlinewidth{0.702625pt}%
\definecolor{currentstroke}{rgb}{1.000000,1.000000,1.000000}%
\pgfsetstrokecolor{currentstroke}%
\pgfsetdash{}{0pt}%
\pgfpathmoveto{\pgfqpoint{4.601390in}{0.754477in}}%
\pgfpathlineto{\pgfqpoint{4.601390in}{3.115885in}}%
\pgfusepath{stroke}%
\end{pgfscope}%
\begin{pgfscope}%
\pgfpathrectangle{\pgfqpoint{0.832441in}{0.754477in}}{\pgfqpoint{4.987559in}{2.361409in}}%
\pgfusepath{clip}%
\pgfsetroundcap%
\pgfsetroundjoin%
\pgfsetlinewidth{0.702625pt}%
\definecolor{currentstroke}{rgb}{1.000000,1.000000,1.000000}%
\pgfsetstrokecolor{currentstroke}%
\pgfsetdash{}{0pt}%
\pgfpathmoveto{\pgfqpoint{4.698975in}{0.754477in}}%
\pgfpathlineto{\pgfqpoint{4.698975in}{3.115885in}}%
\pgfusepath{stroke}%
\end{pgfscope}%
\begin{pgfscope}%
\pgfpathrectangle{\pgfqpoint{0.832441in}{0.754477in}}{\pgfqpoint{4.987559in}{2.361409in}}%
\pgfusepath{clip}%
\pgfsetroundcap%
\pgfsetroundjoin%
\pgfsetlinewidth{0.702625pt}%
\definecolor{currentstroke}{rgb}{1.000000,1.000000,1.000000}%
\pgfsetstrokecolor{currentstroke}%
\pgfsetdash{}{0pt}%
\pgfpathmoveto{\pgfqpoint{4.768213in}{0.754477in}}%
\pgfpathlineto{\pgfqpoint{4.768213in}{3.115885in}}%
\pgfusepath{stroke}%
\end{pgfscope}%
\begin{pgfscope}%
\pgfpathrectangle{\pgfqpoint{0.832441in}{0.754477in}}{\pgfqpoint{4.987559in}{2.361409in}}%
\pgfusepath{clip}%
\pgfsetroundcap%
\pgfsetroundjoin%
\pgfsetlinewidth{0.702625pt}%
\definecolor{currentstroke}{rgb}{1.000000,1.000000,1.000000}%
\pgfsetstrokecolor{currentstroke}%
\pgfsetdash{}{0pt}%
\pgfpathmoveto{\pgfqpoint{4.821917in}{0.754477in}}%
\pgfpathlineto{\pgfqpoint{4.821917in}{3.115885in}}%
\pgfusepath{stroke}%
\end{pgfscope}%
\begin{pgfscope}%
\pgfpathrectangle{\pgfqpoint{0.832441in}{0.754477in}}{\pgfqpoint{4.987559in}{2.361409in}}%
\pgfusepath{clip}%
\pgfsetroundcap%
\pgfsetroundjoin%
\pgfsetlinewidth{0.702625pt}%
\definecolor{currentstroke}{rgb}{1.000000,1.000000,1.000000}%
\pgfsetstrokecolor{currentstroke}%
\pgfsetdash{}{0pt}%
\pgfpathmoveto{\pgfqpoint{4.865798in}{0.754477in}}%
\pgfpathlineto{\pgfqpoint{4.865798in}{3.115885in}}%
\pgfusepath{stroke}%
\end{pgfscope}%
\begin{pgfscope}%
\pgfpathrectangle{\pgfqpoint{0.832441in}{0.754477in}}{\pgfqpoint{4.987559in}{2.361409in}}%
\pgfusepath{clip}%
\pgfsetroundcap%
\pgfsetroundjoin%
\pgfsetlinewidth{0.702625pt}%
\definecolor{currentstroke}{rgb}{1.000000,1.000000,1.000000}%
\pgfsetstrokecolor{currentstroke}%
\pgfsetdash{}{0pt}%
\pgfpathmoveto{\pgfqpoint{4.902898in}{0.754477in}}%
\pgfpathlineto{\pgfqpoint{4.902898in}{3.115885in}}%
\pgfusepath{stroke}%
\end{pgfscope}%
\begin{pgfscope}%
\pgfpathrectangle{\pgfqpoint{0.832441in}{0.754477in}}{\pgfqpoint{4.987559in}{2.361409in}}%
\pgfusepath{clip}%
\pgfsetroundcap%
\pgfsetroundjoin%
\pgfsetlinewidth{0.702625pt}%
\definecolor{currentstroke}{rgb}{1.000000,1.000000,1.000000}%
\pgfsetstrokecolor{currentstroke}%
\pgfsetdash{}{0pt}%
\pgfpathmoveto{\pgfqpoint{4.935035in}{0.754477in}}%
\pgfpathlineto{\pgfqpoint{4.935035in}{3.115885in}}%
\pgfusepath{stroke}%
\end{pgfscope}%
\begin{pgfscope}%
\pgfpathrectangle{\pgfqpoint{0.832441in}{0.754477in}}{\pgfqpoint{4.987559in}{2.361409in}}%
\pgfusepath{clip}%
\pgfsetroundcap%
\pgfsetroundjoin%
\pgfsetlinewidth{0.702625pt}%
\definecolor{currentstroke}{rgb}{1.000000,1.000000,1.000000}%
\pgfsetstrokecolor{currentstroke}%
\pgfsetdash{}{0pt}%
\pgfpathmoveto{\pgfqpoint{4.963383in}{0.754477in}}%
\pgfpathlineto{\pgfqpoint{4.963383in}{3.115885in}}%
\pgfusepath{stroke}%
\end{pgfscope}%
\begin{pgfscope}%
\pgfpathrectangle{\pgfqpoint{0.832441in}{0.754477in}}{\pgfqpoint{4.987559in}{2.361409in}}%
\pgfusepath{clip}%
\pgfsetroundcap%
\pgfsetroundjoin%
\pgfsetlinewidth{0.702625pt}%
\definecolor{currentstroke}{rgb}{1.000000,1.000000,1.000000}%
\pgfsetstrokecolor{currentstroke}%
\pgfsetdash{}{0pt}%
\pgfpathmoveto{\pgfqpoint{5.155563in}{0.754477in}}%
\pgfpathlineto{\pgfqpoint{5.155563in}{3.115885in}}%
\pgfusepath{stroke}%
\end{pgfscope}%
\begin{pgfscope}%
\pgfpathrectangle{\pgfqpoint{0.832441in}{0.754477in}}{\pgfqpoint{4.987559in}{2.361409in}}%
\pgfusepath{clip}%
\pgfsetroundcap%
\pgfsetroundjoin%
\pgfsetlinewidth{0.702625pt}%
\definecolor{currentstroke}{rgb}{1.000000,1.000000,1.000000}%
\pgfsetstrokecolor{currentstroke}%
\pgfsetdash{}{0pt}%
\pgfpathmoveto{\pgfqpoint{5.253148in}{0.754477in}}%
\pgfpathlineto{\pgfqpoint{5.253148in}{3.115885in}}%
\pgfusepath{stroke}%
\end{pgfscope}%
\begin{pgfscope}%
\pgfpathrectangle{\pgfqpoint{0.832441in}{0.754477in}}{\pgfqpoint{4.987559in}{2.361409in}}%
\pgfusepath{clip}%
\pgfsetroundcap%
\pgfsetroundjoin%
\pgfsetlinewidth{0.702625pt}%
\definecolor{currentstroke}{rgb}{1.000000,1.000000,1.000000}%
\pgfsetstrokecolor{currentstroke}%
\pgfsetdash{}{0pt}%
\pgfpathmoveto{\pgfqpoint{5.322386in}{0.754477in}}%
\pgfpathlineto{\pgfqpoint{5.322386in}{3.115885in}}%
\pgfusepath{stroke}%
\end{pgfscope}%
\begin{pgfscope}%
\pgfpathrectangle{\pgfqpoint{0.832441in}{0.754477in}}{\pgfqpoint{4.987559in}{2.361409in}}%
\pgfusepath{clip}%
\pgfsetroundcap%
\pgfsetroundjoin%
\pgfsetlinewidth{0.702625pt}%
\definecolor{currentstroke}{rgb}{1.000000,1.000000,1.000000}%
\pgfsetstrokecolor{currentstroke}%
\pgfsetdash{}{0pt}%
\pgfpathmoveto{\pgfqpoint{5.376091in}{0.754477in}}%
\pgfpathlineto{\pgfqpoint{5.376091in}{3.115885in}}%
\pgfusepath{stroke}%
\end{pgfscope}%
\begin{pgfscope}%
\pgfpathrectangle{\pgfqpoint{0.832441in}{0.754477in}}{\pgfqpoint{4.987559in}{2.361409in}}%
\pgfusepath{clip}%
\pgfsetroundcap%
\pgfsetroundjoin%
\pgfsetlinewidth{0.702625pt}%
\definecolor{currentstroke}{rgb}{1.000000,1.000000,1.000000}%
\pgfsetstrokecolor{currentstroke}%
\pgfsetdash{}{0pt}%
\pgfpathmoveto{\pgfqpoint{5.419971in}{0.754477in}}%
\pgfpathlineto{\pgfqpoint{5.419971in}{3.115885in}}%
\pgfusepath{stroke}%
\end{pgfscope}%
\begin{pgfscope}%
\pgfpathrectangle{\pgfqpoint{0.832441in}{0.754477in}}{\pgfqpoint{4.987559in}{2.361409in}}%
\pgfusepath{clip}%
\pgfsetroundcap%
\pgfsetroundjoin%
\pgfsetlinewidth{0.702625pt}%
\definecolor{currentstroke}{rgb}{1.000000,1.000000,1.000000}%
\pgfsetstrokecolor{currentstroke}%
\pgfsetdash{}{0pt}%
\pgfpathmoveto{\pgfqpoint{5.457071in}{0.754477in}}%
\pgfpathlineto{\pgfqpoint{5.457071in}{3.115885in}}%
\pgfusepath{stroke}%
\end{pgfscope}%
\begin{pgfscope}%
\pgfpathrectangle{\pgfqpoint{0.832441in}{0.754477in}}{\pgfqpoint{4.987559in}{2.361409in}}%
\pgfusepath{clip}%
\pgfsetroundcap%
\pgfsetroundjoin%
\pgfsetlinewidth{0.702625pt}%
\definecolor{currentstroke}{rgb}{1.000000,1.000000,1.000000}%
\pgfsetstrokecolor{currentstroke}%
\pgfsetdash{}{0pt}%
\pgfpathmoveto{\pgfqpoint{5.489208in}{0.754477in}}%
\pgfpathlineto{\pgfqpoint{5.489208in}{3.115885in}}%
\pgfusepath{stroke}%
\end{pgfscope}%
\begin{pgfscope}%
\pgfpathrectangle{\pgfqpoint{0.832441in}{0.754477in}}{\pgfqpoint{4.987559in}{2.361409in}}%
\pgfusepath{clip}%
\pgfsetroundcap%
\pgfsetroundjoin%
\pgfsetlinewidth{0.702625pt}%
\definecolor{currentstroke}{rgb}{1.000000,1.000000,1.000000}%
\pgfsetstrokecolor{currentstroke}%
\pgfsetdash{}{0pt}%
\pgfpathmoveto{\pgfqpoint{5.517556in}{0.754477in}}%
\pgfpathlineto{\pgfqpoint{5.517556in}{3.115885in}}%
\pgfusepath{stroke}%
\end{pgfscope}%
\begin{pgfscope}%
\pgfpathrectangle{\pgfqpoint{0.832441in}{0.754477in}}{\pgfqpoint{4.987559in}{2.361409in}}%
\pgfusepath{clip}%
\pgfsetroundcap%
\pgfsetroundjoin%
\pgfsetlinewidth{0.702625pt}%
\definecolor{currentstroke}{rgb}{1.000000,1.000000,1.000000}%
\pgfsetstrokecolor{currentstroke}%
\pgfsetdash{}{0pt}%
\pgfpathmoveto{\pgfqpoint{5.709736in}{0.754477in}}%
\pgfpathlineto{\pgfqpoint{5.709736in}{3.115885in}}%
\pgfusepath{stroke}%
\end{pgfscope}%
\begin{pgfscope}%
\pgfpathrectangle{\pgfqpoint{0.832441in}{0.754477in}}{\pgfqpoint{4.987559in}{2.361409in}}%
\pgfusepath{clip}%
\pgfsetroundcap%
\pgfsetroundjoin%
\pgfsetlinewidth{0.702625pt}%
\definecolor{currentstroke}{rgb}{1.000000,1.000000,1.000000}%
\pgfsetstrokecolor{currentstroke}%
\pgfsetdash{}{0pt}%
\pgfpathmoveto{\pgfqpoint{5.807321in}{0.754477in}}%
\pgfpathlineto{\pgfqpoint{5.807321in}{3.115885in}}%
\pgfusepath{stroke}%
\end{pgfscope}%
\begin{pgfscope}%
\definecolor{textcolor}{rgb}{0.150000,0.150000,0.150000}%
\pgfsetstrokecolor{textcolor}%
\pgfsetfillcolor{textcolor}%
\pgftext[x=3.326221in,y=0.431310in,,top]{\color{textcolor}\sffamily\fontsize{12.000000}{14.400000}\selectfont Reynolds Number \(\displaystyle \mathrm{Re}_D\)}%
\end{pgfscope}%
\begin{pgfscope}%
\pgfpathrectangle{\pgfqpoint{0.832441in}{0.754477in}}{\pgfqpoint{4.987559in}{2.361409in}}%
\pgfusepath{clip}%
\pgfsetroundcap%
\pgfsetroundjoin%
\pgfsetlinewidth{1.606000pt}%
\definecolor{currentstroke}{rgb}{1.000000,1.000000,1.000000}%
\pgfsetstrokecolor{currentstroke}%
\pgfsetdash{}{0pt}%
\pgfpathmoveto{\pgfqpoint{0.832441in}{0.754477in}}%
\pgfpathlineto{\pgfqpoint{5.820000in}{0.754477in}}%
\pgfusepath{stroke}%
\end{pgfscope}%
\begin{pgfscope}%
\definecolor{textcolor}{rgb}{0.150000,0.150000,0.150000}%
\pgfsetstrokecolor{textcolor}%
\pgfsetfillcolor{textcolor}%
\pgftext[x=0.390618in, y=0.701463in, left, base]{\color{textcolor}\sffamily\fontsize{11.000000}{13.200000}\selectfont \(\displaystyle {10^{-1}}\)}%
\end{pgfscope}%
\begin{pgfscope}%
\pgfpathrectangle{\pgfqpoint{0.832441in}{0.754477in}}{\pgfqpoint{4.987559in}{2.361409in}}%
\pgfusepath{clip}%
\pgfsetroundcap%
\pgfsetroundjoin%
\pgfsetlinewidth{1.606000pt}%
\definecolor{currentstroke}{rgb}{1.000000,1.000000,1.000000}%
\pgfsetstrokecolor{currentstroke}%
\pgfsetdash{}{0pt}%
\pgfpathmoveto{\pgfqpoint{0.832441in}{1.541613in}}%
\pgfpathlineto{\pgfqpoint{5.820000in}{1.541613in}}%
\pgfusepath{stroke}%
\end{pgfscope}%
\begin{pgfscope}%
\definecolor{textcolor}{rgb}{0.150000,0.150000,0.150000}%
\pgfsetstrokecolor{textcolor}%
\pgfsetfillcolor{textcolor}%
\pgftext[x=0.482440in, y=1.488599in, left, base]{\color{textcolor}\sffamily\fontsize{11.000000}{13.200000}\selectfont \(\displaystyle {10^{0}}\)}%
\end{pgfscope}%
\begin{pgfscope}%
\pgfpathrectangle{\pgfqpoint{0.832441in}{0.754477in}}{\pgfqpoint{4.987559in}{2.361409in}}%
\pgfusepath{clip}%
\pgfsetroundcap%
\pgfsetroundjoin%
\pgfsetlinewidth{1.606000pt}%
\definecolor{currentstroke}{rgb}{1.000000,1.000000,1.000000}%
\pgfsetstrokecolor{currentstroke}%
\pgfsetdash{}{0pt}%
\pgfpathmoveto{\pgfqpoint{0.832441in}{2.328749in}}%
\pgfpathlineto{\pgfqpoint{5.820000in}{2.328749in}}%
\pgfusepath{stroke}%
\end{pgfscope}%
\begin{pgfscope}%
\definecolor{textcolor}{rgb}{0.150000,0.150000,0.150000}%
\pgfsetstrokecolor{textcolor}%
\pgfsetfillcolor{textcolor}%
\pgftext[x=0.482440in, y=2.275735in, left, base]{\color{textcolor}\sffamily\fontsize{11.000000}{13.200000}\selectfont \(\displaystyle {10^{1}}\)}%
\end{pgfscope}%
\begin{pgfscope}%
\pgfpathrectangle{\pgfqpoint{0.832441in}{0.754477in}}{\pgfqpoint{4.987559in}{2.361409in}}%
\pgfusepath{clip}%
\pgfsetroundcap%
\pgfsetroundjoin%
\pgfsetlinewidth{1.606000pt}%
\definecolor{currentstroke}{rgb}{1.000000,1.000000,1.000000}%
\pgfsetstrokecolor{currentstroke}%
\pgfsetdash{}{0pt}%
\pgfpathmoveto{\pgfqpoint{0.832441in}{3.115885in}}%
\pgfpathlineto{\pgfqpoint{5.820000in}{3.115885in}}%
\pgfusepath{stroke}%
\end{pgfscope}%
\begin{pgfscope}%
\definecolor{textcolor}{rgb}{0.150000,0.150000,0.150000}%
\pgfsetstrokecolor{textcolor}%
\pgfsetfillcolor{textcolor}%
\pgftext[x=0.482440in, y=3.062872in, left, base]{\color{textcolor}\sffamily\fontsize{11.000000}{13.200000}\selectfont \(\displaystyle {10^{2}}\)}%
\end{pgfscope}%
\begin{pgfscope}%
\pgfpathrectangle{\pgfqpoint{0.832441in}{0.754477in}}{\pgfqpoint{4.987559in}{2.361409in}}%
\pgfusepath{clip}%
\pgfsetroundcap%
\pgfsetroundjoin%
\pgfsetlinewidth{0.702625pt}%
\definecolor{currentstroke}{rgb}{1.000000,1.000000,1.000000}%
\pgfsetstrokecolor{currentstroke}%
\pgfsetdash{}{0pt}%
\pgfpathmoveto{\pgfqpoint{0.832441in}{0.991428in}}%
\pgfpathlineto{\pgfqpoint{5.820000in}{0.991428in}}%
\pgfusepath{stroke}%
\end{pgfscope}%
\begin{pgfscope}%
\pgfpathrectangle{\pgfqpoint{0.832441in}{0.754477in}}{\pgfqpoint{4.987559in}{2.361409in}}%
\pgfusepath{clip}%
\pgfsetroundcap%
\pgfsetroundjoin%
\pgfsetlinewidth{0.702625pt}%
\definecolor{currentstroke}{rgb}{1.000000,1.000000,1.000000}%
\pgfsetstrokecolor{currentstroke}%
\pgfsetdash{}{0pt}%
\pgfpathmoveto{\pgfqpoint{0.832441in}{1.130036in}}%
\pgfpathlineto{\pgfqpoint{5.820000in}{1.130036in}}%
\pgfusepath{stroke}%
\end{pgfscope}%
\begin{pgfscope}%
\pgfpathrectangle{\pgfqpoint{0.832441in}{0.754477in}}{\pgfqpoint{4.987559in}{2.361409in}}%
\pgfusepath{clip}%
\pgfsetroundcap%
\pgfsetroundjoin%
\pgfsetlinewidth{0.702625pt}%
\definecolor{currentstroke}{rgb}{1.000000,1.000000,1.000000}%
\pgfsetstrokecolor{currentstroke}%
\pgfsetdash{}{0pt}%
\pgfpathmoveto{\pgfqpoint{0.832441in}{1.228380in}}%
\pgfpathlineto{\pgfqpoint{5.820000in}{1.228380in}}%
\pgfusepath{stroke}%
\end{pgfscope}%
\begin{pgfscope}%
\pgfpathrectangle{\pgfqpoint{0.832441in}{0.754477in}}{\pgfqpoint{4.987559in}{2.361409in}}%
\pgfusepath{clip}%
\pgfsetroundcap%
\pgfsetroundjoin%
\pgfsetlinewidth{0.702625pt}%
\definecolor{currentstroke}{rgb}{1.000000,1.000000,1.000000}%
\pgfsetstrokecolor{currentstroke}%
\pgfsetdash{}{0pt}%
\pgfpathmoveto{\pgfqpoint{0.832441in}{1.304661in}}%
\pgfpathlineto{\pgfqpoint{5.820000in}{1.304661in}}%
\pgfusepath{stroke}%
\end{pgfscope}%
\begin{pgfscope}%
\pgfpathrectangle{\pgfqpoint{0.832441in}{0.754477in}}{\pgfqpoint{4.987559in}{2.361409in}}%
\pgfusepath{clip}%
\pgfsetroundcap%
\pgfsetroundjoin%
\pgfsetlinewidth{0.702625pt}%
\definecolor{currentstroke}{rgb}{1.000000,1.000000,1.000000}%
\pgfsetstrokecolor{currentstroke}%
\pgfsetdash{}{0pt}%
\pgfpathmoveto{\pgfqpoint{0.832441in}{1.366988in}}%
\pgfpathlineto{\pgfqpoint{5.820000in}{1.366988in}}%
\pgfusepath{stroke}%
\end{pgfscope}%
\begin{pgfscope}%
\pgfpathrectangle{\pgfqpoint{0.832441in}{0.754477in}}{\pgfqpoint{4.987559in}{2.361409in}}%
\pgfusepath{clip}%
\pgfsetroundcap%
\pgfsetroundjoin%
\pgfsetlinewidth{0.702625pt}%
\definecolor{currentstroke}{rgb}{1.000000,1.000000,1.000000}%
\pgfsetstrokecolor{currentstroke}%
\pgfsetdash{}{0pt}%
\pgfpathmoveto{\pgfqpoint{0.832441in}{1.419684in}}%
\pgfpathlineto{\pgfqpoint{5.820000in}{1.419684in}}%
\pgfusepath{stroke}%
\end{pgfscope}%
\begin{pgfscope}%
\pgfpathrectangle{\pgfqpoint{0.832441in}{0.754477in}}{\pgfqpoint{4.987559in}{2.361409in}}%
\pgfusepath{clip}%
\pgfsetroundcap%
\pgfsetroundjoin%
\pgfsetlinewidth{0.702625pt}%
\definecolor{currentstroke}{rgb}{1.000000,1.000000,1.000000}%
\pgfsetstrokecolor{currentstroke}%
\pgfsetdash{}{0pt}%
\pgfpathmoveto{\pgfqpoint{0.832441in}{1.465331in}}%
\pgfpathlineto{\pgfqpoint{5.820000in}{1.465331in}}%
\pgfusepath{stroke}%
\end{pgfscope}%
\begin{pgfscope}%
\pgfpathrectangle{\pgfqpoint{0.832441in}{0.754477in}}{\pgfqpoint{4.987559in}{2.361409in}}%
\pgfusepath{clip}%
\pgfsetroundcap%
\pgfsetroundjoin%
\pgfsetlinewidth{0.702625pt}%
\definecolor{currentstroke}{rgb}{1.000000,1.000000,1.000000}%
\pgfsetstrokecolor{currentstroke}%
\pgfsetdash{}{0pt}%
\pgfpathmoveto{\pgfqpoint{0.832441in}{1.505595in}}%
\pgfpathlineto{\pgfqpoint{5.820000in}{1.505595in}}%
\pgfusepath{stroke}%
\end{pgfscope}%
\begin{pgfscope}%
\pgfpathrectangle{\pgfqpoint{0.832441in}{0.754477in}}{\pgfqpoint{4.987559in}{2.361409in}}%
\pgfusepath{clip}%
\pgfsetroundcap%
\pgfsetroundjoin%
\pgfsetlinewidth{0.702625pt}%
\definecolor{currentstroke}{rgb}{1.000000,1.000000,1.000000}%
\pgfsetstrokecolor{currentstroke}%
\pgfsetdash{}{0pt}%
\pgfpathmoveto{\pgfqpoint{0.832441in}{1.778564in}}%
\pgfpathlineto{\pgfqpoint{5.820000in}{1.778564in}}%
\pgfusepath{stroke}%
\end{pgfscope}%
\begin{pgfscope}%
\pgfpathrectangle{\pgfqpoint{0.832441in}{0.754477in}}{\pgfqpoint{4.987559in}{2.361409in}}%
\pgfusepath{clip}%
\pgfsetroundcap%
\pgfsetroundjoin%
\pgfsetlinewidth{0.702625pt}%
\definecolor{currentstroke}{rgb}{1.000000,1.000000,1.000000}%
\pgfsetstrokecolor{currentstroke}%
\pgfsetdash{}{0pt}%
\pgfpathmoveto{\pgfqpoint{0.832441in}{1.917172in}}%
\pgfpathlineto{\pgfqpoint{5.820000in}{1.917172in}}%
\pgfusepath{stroke}%
\end{pgfscope}%
\begin{pgfscope}%
\pgfpathrectangle{\pgfqpoint{0.832441in}{0.754477in}}{\pgfqpoint{4.987559in}{2.361409in}}%
\pgfusepath{clip}%
\pgfsetroundcap%
\pgfsetroundjoin%
\pgfsetlinewidth{0.702625pt}%
\definecolor{currentstroke}{rgb}{1.000000,1.000000,1.000000}%
\pgfsetstrokecolor{currentstroke}%
\pgfsetdash{}{0pt}%
\pgfpathmoveto{\pgfqpoint{0.832441in}{2.015516in}}%
\pgfpathlineto{\pgfqpoint{5.820000in}{2.015516in}}%
\pgfusepath{stroke}%
\end{pgfscope}%
\begin{pgfscope}%
\pgfpathrectangle{\pgfqpoint{0.832441in}{0.754477in}}{\pgfqpoint{4.987559in}{2.361409in}}%
\pgfusepath{clip}%
\pgfsetroundcap%
\pgfsetroundjoin%
\pgfsetlinewidth{0.702625pt}%
\definecolor{currentstroke}{rgb}{1.000000,1.000000,1.000000}%
\pgfsetstrokecolor{currentstroke}%
\pgfsetdash{}{0pt}%
\pgfpathmoveto{\pgfqpoint{0.832441in}{2.091797in}}%
\pgfpathlineto{\pgfqpoint{5.820000in}{2.091797in}}%
\pgfusepath{stroke}%
\end{pgfscope}%
\begin{pgfscope}%
\pgfpathrectangle{\pgfqpoint{0.832441in}{0.754477in}}{\pgfqpoint{4.987559in}{2.361409in}}%
\pgfusepath{clip}%
\pgfsetroundcap%
\pgfsetroundjoin%
\pgfsetlinewidth{0.702625pt}%
\definecolor{currentstroke}{rgb}{1.000000,1.000000,1.000000}%
\pgfsetstrokecolor{currentstroke}%
\pgfsetdash{}{0pt}%
\pgfpathmoveto{\pgfqpoint{0.832441in}{2.154124in}}%
\pgfpathlineto{\pgfqpoint{5.820000in}{2.154124in}}%
\pgfusepath{stroke}%
\end{pgfscope}%
\begin{pgfscope}%
\pgfpathrectangle{\pgfqpoint{0.832441in}{0.754477in}}{\pgfqpoint{4.987559in}{2.361409in}}%
\pgfusepath{clip}%
\pgfsetroundcap%
\pgfsetroundjoin%
\pgfsetlinewidth{0.702625pt}%
\definecolor{currentstroke}{rgb}{1.000000,1.000000,1.000000}%
\pgfsetstrokecolor{currentstroke}%
\pgfsetdash{}{0pt}%
\pgfpathmoveto{\pgfqpoint{0.832441in}{2.206820in}}%
\pgfpathlineto{\pgfqpoint{5.820000in}{2.206820in}}%
\pgfusepath{stroke}%
\end{pgfscope}%
\begin{pgfscope}%
\pgfpathrectangle{\pgfqpoint{0.832441in}{0.754477in}}{\pgfqpoint{4.987559in}{2.361409in}}%
\pgfusepath{clip}%
\pgfsetroundcap%
\pgfsetroundjoin%
\pgfsetlinewidth{0.702625pt}%
\definecolor{currentstroke}{rgb}{1.000000,1.000000,1.000000}%
\pgfsetstrokecolor{currentstroke}%
\pgfsetdash{}{0pt}%
\pgfpathmoveto{\pgfqpoint{0.832441in}{2.252468in}}%
\pgfpathlineto{\pgfqpoint{5.820000in}{2.252468in}}%
\pgfusepath{stroke}%
\end{pgfscope}%
\begin{pgfscope}%
\pgfpathrectangle{\pgfqpoint{0.832441in}{0.754477in}}{\pgfqpoint{4.987559in}{2.361409in}}%
\pgfusepath{clip}%
\pgfsetroundcap%
\pgfsetroundjoin%
\pgfsetlinewidth{0.702625pt}%
\definecolor{currentstroke}{rgb}{1.000000,1.000000,1.000000}%
\pgfsetstrokecolor{currentstroke}%
\pgfsetdash{}{0pt}%
\pgfpathmoveto{\pgfqpoint{0.832441in}{2.292732in}}%
\pgfpathlineto{\pgfqpoint{5.820000in}{2.292732in}}%
\pgfusepath{stroke}%
\end{pgfscope}%
\begin{pgfscope}%
\pgfpathrectangle{\pgfqpoint{0.832441in}{0.754477in}}{\pgfqpoint{4.987559in}{2.361409in}}%
\pgfusepath{clip}%
\pgfsetroundcap%
\pgfsetroundjoin%
\pgfsetlinewidth{0.702625pt}%
\definecolor{currentstroke}{rgb}{1.000000,1.000000,1.000000}%
\pgfsetstrokecolor{currentstroke}%
\pgfsetdash{}{0pt}%
\pgfpathmoveto{\pgfqpoint{0.832441in}{2.565701in}}%
\pgfpathlineto{\pgfqpoint{5.820000in}{2.565701in}}%
\pgfusepath{stroke}%
\end{pgfscope}%
\begin{pgfscope}%
\pgfpathrectangle{\pgfqpoint{0.832441in}{0.754477in}}{\pgfqpoint{4.987559in}{2.361409in}}%
\pgfusepath{clip}%
\pgfsetroundcap%
\pgfsetroundjoin%
\pgfsetlinewidth{0.702625pt}%
\definecolor{currentstroke}{rgb}{1.000000,1.000000,1.000000}%
\pgfsetstrokecolor{currentstroke}%
\pgfsetdash{}{0pt}%
\pgfpathmoveto{\pgfqpoint{0.832441in}{2.704309in}}%
\pgfpathlineto{\pgfqpoint{5.820000in}{2.704309in}}%
\pgfusepath{stroke}%
\end{pgfscope}%
\begin{pgfscope}%
\pgfpathrectangle{\pgfqpoint{0.832441in}{0.754477in}}{\pgfqpoint{4.987559in}{2.361409in}}%
\pgfusepath{clip}%
\pgfsetroundcap%
\pgfsetroundjoin%
\pgfsetlinewidth{0.702625pt}%
\definecolor{currentstroke}{rgb}{1.000000,1.000000,1.000000}%
\pgfsetstrokecolor{currentstroke}%
\pgfsetdash{}{0pt}%
\pgfpathmoveto{\pgfqpoint{0.832441in}{2.802652in}}%
\pgfpathlineto{\pgfqpoint{5.820000in}{2.802652in}}%
\pgfusepath{stroke}%
\end{pgfscope}%
\begin{pgfscope}%
\pgfpathrectangle{\pgfqpoint{0.832441in}{0.754477in}}{\pgfqpoint{4.987559in}{2.361409in}}%
\pgfusepath{clip}%
\pgfsetroundcap%
\pgfsetroundjoin%
\pgfsetlinewidth{0.702625pt}%
\definecolor{currentstroke}{rgb}{1.000000,1.000000,1.000000}%
\pgfsetstrokecolor{currentstroke}%
\pgfsetdash{}{0pt}%
\pgfpathmoveto{\pgfqpoint{0.832441in}{2.878934in}}%
\pgfpathlineto{\pgfqpoint{5.820000in}{2.878934in}}%
\pgfusepath{stroke}%
\end{pgfscope}%
\begin{pgfscope}%
\pgfpathrectangle{\pgfqpoint{0.832441in}{0.754477in}}{\pgfqpoint{4.987559in}{2.361409in}}%
\pgfusepath{clip}%
\pgfsetroundcap%
\pgfsetroundjoin%
\pgfsetlinewidth{0.702625pt}%
\definecolor{currentstroke}{rgb}{1.000000,1.000000,1.000000}%
\pgfsetstrokecolor{currentstroke}%
\pgfsetdash{}{0pt}%
\pgfpathmoveto{\pgfqpoint{0.832441in}{2.941260in}}%
\pgfpathlineto{\pgfqpoint{5.820000in}{2.941260in}}%
\pgfusepath{stroke}%
\end{pgfscope}%
\begin{pgfscope}%
\pgfpathrectangle{\pgfqpoint{0.832441in}{0.754477in}}{\pgfqpoint{4.987559in}{2.361409in}}%
\pgfusepath{clip}%
\pgfsetroundcap%
\pgfsetroundjoin%
\pgfsetlinewidth{0.702625pt}%
\definecolor{currentstroke}{rgb}{1.000000,1.000000,1.000000}%
\pgfsetstrokecolor{currentstroke}%
\pgfsetdash{}{0pt}%
\pgfpathmoveto{\pgfqpoint{0.832441in}{2.993956in}}%
\pgfpathlineto{\pgfqpoint{5.820000in}{2.993956in}}%
\pgfusepath{stroke}%
\end{pgfscope}%
\begin{pgfscope}%
\pgfpathrectangle{\pgfqpoint{0.832441in}{0.754477in}}{\pgfqpoint{4.987559in}{2.361409in}}%
\pgfusepath{clip}%
\pgfsetroundcap%
\pgfsetroundjoin%
\pgfsetlinewidth{0.702625pt}%
\definecolor{currentstroke}{rgb}{1.000000,1.000000,1.000000}%
\pgfsetstrokecolor{currentstroke}%
\pgfsetdash{}{0pt}%
\pgfpathmoveto{\pgfqpoint{0.832441in}{3.039604in}}%
\pgfpathlineto{\pgfqpoint{5.820000in}{3.039604in}}%
\pgfusepath{stroke}%
\end{pgfscope}%
\begin{pgfscope}%
\pgfpathrectangle{\pgfqpoint{0.832441in}{0.754477in}}{\pgfqpoint{4.987559in}{2.361409in}}%
\pgfusepath{clip}%
\pgfsetroundcap%
\pgfsetroundjoin%
\pgfsetlinewidth{0.702625pt}%
\definecolor{currentstroke}{rgb}{1.000000,1.000000,1.000000}%
\pgfsetstrokecolor{currentstroke}%
\pgfsetdash{}{0pt}%
\pgfpathmoveto{\pgfqpoint{0.832441in}{3.079868in}}%
\pgfpathlineto{\pgfqpoint{5.820000in}{3.079868in}}%
\pgfusepath{stroke}%
\end{pgfscope}%
\begin{pgfscope}%
\definecolor{textcolor}{rgb}{0.150000,0.150000,0.150000}%
\pgfsetstrokecolor{textcolor}%
\pgfsetfillcolor{textcolor}%
\pgftext[x=0.335062in,y=1.935181in,,bottom,rotate=90.000000]{\color{textcolor}\sffamily\fontsize{12.000000}{14.400000}\selectfont Drag Coefficient \(\displaystyle C_D\)}%
\end{pgfscope}%
\begin{pgfscope}%
\pgfpathrectangle{\pgfqpoint{0.832441in}{0.754477in}}{\pgfqpoint{4.987559in}{2.361409in}}%
\pgfusepath{clip}%
\pgfsetbuttcap%
\pgfsetroundjoin%
\definecolor{currentfill}{rgb}{0.000000,0.000000,0.000000}%
\pgfsetfillcolor{currentfill}%
\pgfsetfillopacity{0.800000}%
\pgfsetlinewidth{1.003750pt}%
\definecolor{currentstroke}{rgb}{0.000000,0.000000,0.000000}%
\pgfsetstrokecolor{currentstroke}%
\pgfsetstrokeopacity{0.800000}%
\pgfsetdash{}{0pt}%
\pgfsys@defobject{currentmarker}{\pgfqpoint{-0.010417in}{-0.010417in}}{\pgfqpoint{0.010417in}{0.010417in}}{%
\pgfpathmoveto{\pgfqpoint{0.000000in}{-0.010417in}}%
\pgfpathcurveto{\pgfqpoint{0.002763in}{-0.010417in}}{\pgfqpoint{0.005412in}{-0.009319in}}{\pgfqpoint{0.007366in}{-0.007366in}}%
\pgfpathcurveto{\pgfqpoint{0.009319in}{-0.005412in}}{\pgfqpoint{0.010417in}{-0.002763in}}{\pgfqpoint{0.010417in}{0.000000in}}%
\pgfpathcurveto{\pgfqpoint{0.010417in}{0.002763in}}{\pgfqpoint{0.009319in}{0.005412in}}{\pgfqpoint{0.007366in}{0.007366in}}%
\pgfpathcurveto{\pgfqpoint{0.005412in}{0.009319in}}{\pgfqpoint{0.002763in}{0.010417in}}{\pgfqpoint{0.000000in}{0.010417in}}%
\pgfpathcurveto{\pgfqpoint{-0.002763in}{0.010417in}}{\pgfqpoint{-0.005412in}{0.009319in}}{\pgfqpoint{-0.007366in}{0.007366in}}%
\pgfpathcurveto{\pgfqpoint{-0.009319in}{0.005412in}}{\pgfqpoint{-0.010417in}{0.002763in}}{\pgfqpoint{-0.010417in}{0.000000in}}%
\pgfpathcurveto{\pgfqpoint{-0.010417in}{-0.002763in}}{\pgfqpoint{-0.009319in}{-0.005412in}}{\pgfqpoint{-0.007366in}{-0.007366in}}%
\pgfpathcurveto{\pgfqpoint{-0.005412in}{-0.009319in}}{\pgfqpoint{-0.002763in}{-0.010417in}}{\pgfqpoint{0.000000in}{-0.010417in}}%
\pgfpathclose%
\pgfusepath{stroke,fill}%
}%
\begin{pgfscope}%
\pgfsys@transformshift{1.146092in}{2.879068in}%
\pgfsys@useobject{currentmarker}{}%
\end{pgfscope}%
\begin{pgfscope}%
\pgfsys@transformshift{1.150337in}{2.905586in}%
\pgfsys@useobject{currentmarker}{}%
\end{pgfscope}%
\begin{pgfscope}%
\pgfsys@transformshift{1.199913in}{2.824314in}%
\pgfsys@useobject{currentmarker}{}%
\end{pgfscope}%
\begin{pgfscope}%
\pgfsys@transformshift{1.245400in}{2.762493in}%
\pgfsys@useobject{currentmarker}{}%
\end{pgfscope}%
\begin{pgfscope}%
\pgfsys@transformshift{1.255985in}{2.819056in}%
\pgfsys@useobject{currentmarker}{}%
\end{pgfscope}%
\begin{pgfscope}%
\pgfsys@transformshift{1.291025in}{2.739548in}%
\pgfsys@useobject{currentmarker}{}%
\end{pgfscope}%
\begin{pgfscope}%
\pgfsys@transformshift{1.311635in}{2.693610in}%
\pgfsys@useobject{currentmarker}{}%
\end{pgfscope}%
\begin{pgfscope}%
\pgfsys@transformshift{1.353043in}{2.652986in}%
\pgfsys@useobject{currentmarker}{}%
\end{pgfscope}%
\begin{pgfscope}%
\pgfsys@transformshift{1.353320in}{2.730763in}%
\pgfsys@useobject{currentmarker}{}%
\end{pgfscope}%
\begin{pgfscope}%
\pgfsys@transformshift{1.444071in}{2.543472in}%
\pgfsys@useobject{currentmarker}{}%
\end{pgfscope}%
\begin{pgfscope}%
\pgfsys@transformshift{1.452616in}{2.610646in}%
\pgfsys@useobject{currentmarker}{}%
\end{pgfscope}%
\begin{pgfscope}%
\pgfsys@transformshift{1.465229in}{2.653081in}%
\pgfsys@useobject{currentmarker}{}%
\end{pgfscope}%
\begin{pgfscope}%
\pgfsys@transformshift{1.497920in}{2.497566in}%
\pgfsys@useobject{currentmarker}{}%
\end{pgfscope}%
\begin{pgfscope}%
\pgfsys@transformshift{1.527219in}{2.557680in}%
\pgfsys@useobject{currentmarker}{}%
\end{pgfscope}%
\begin{pgfscope}%
\pgfsys@transformshift{1.548145in}{2.600122in}%
\pgfsys@useobject{currentmarker}{}%
\end{pgfscope}%
\begin{pgfscope}%
\pgfsys@transformshift{1.551764in}{2.448118in}%
\pgfsys@useobject{currentmarker}{}%
\end{pgfscope}%
\begin{pgfscope}%
\pgfsys@transformshift{1.564238in}{2.451660in}%
\pgfsys@useobject{currentmarker}{}%
\end{pgfscope}%
\begin{pgfscope}%
\pgfsys@transformshift{1.584914in}{2.423394in}%
\pgfsys@useobject{currentmarker}{}%
\end{pgfscope}%
\begin{pgfscope}%
\pgfsys@transformshift{1.589109in}{2.434005in}%
\pgfsys@useobject{currentmarker}{}%
\end{pgfscope}%
\begin{pgfscope}%
\pgfsys@transformshift{1.601562in}{2.430479in}%
\pgfsys@useobject{currentmarker}{}%
\end{pgfscope}%
\begin{pgfscope}%
\pgfsys@transformshift{1.605640in}{2.409273in}%
\pgfsys@useobject{currentmarker}{}%
\end{pgfscope}%
\begin{pgfscope}%
\pgfsys@transformshift{1.618154in}{2.423426in}%
\pgfsys@useobject{currentmarker}{}%
\end{pgfscope}%
\begin{pgfscope}%
\pgfsys@transformshift{1.630534in}{2.398686in}%
\pgfsys@useobject{currentmarker}{}%
\end{pgfscope}%
\begin{pgfscope}%
\pgfsys@transformshift{1.647342in}{2.451723in}%
\pgfsys@useobject{currentmarker}{}%
\end{pgfscope}%
\begin{pgfscope}%
\pgfsys@transformshift{1.671947in}{2.359833in}%
\pgfsys@useobject{currentmarker}{}%
\end{pgfscope}%
\begin{pgfscope}%
\pgfsys@transformshift{1.692673in}{2.345712in}%
\pgfsys@useobject{currentmarker}{}%
\end{pgfscope}%
\begin{pgfscope}%
\pgfsys@transformshift{1.696962in}{2.384596in}%
\pgfsys@useobject{currentmarker}{}%
\end{pgfscope}%
\begin{pgfscope}%
\pgfsys@transformshift{1.713360in}{2.320980in}%
\pgfsys@useobject{currentmarker}{}%
\end{pgfscope}%
\begin{pgfscope}%
\pgfsys@transformshift{1.721595in}{2.299775in}%
\pgfsys@useobject{currentmarker}{}%
\end{pgfscope}%
\begin{pgfscope}%
\pgfsys@transformshift{1.729941in}{2.310393in}%
\pgfsys@useobject{currentmarker}{}%
\end{pgfscope}%
\begin{pgfscope}%
\pgfsys@transformshift{1.746500in}{2.292730in}%
\pgfsys@useobject{currentmarker}{}%
\end{pgfscope}%
\begin{pgfscope}%
\pgfsys@transformshift{1.767237in}{2.282143in}%
\pgfsys@useobject{currentmarker}{}%
\end{pgfscope}%
\begin{pgfscope}%
\pgfsys@transformshift{1.783824in}{2.271548in}%
\pgfsys@useobject{currentmarker}{}%
\end{pgfscope}%
\begin{pgfscope}%
\pgfsys@transformshift{1.796248in}{2.260953in}%
\pgfsys@useobject{currentmarker}{}%
\end{pgfscope}%
\begin{pgfscope}%
\pgfsys@transformshift{1.817163in}{2.299853in}%
\pgfsys@useobject{currentmarker}{}%
\end{pgfscope}%
\begin{pgfscope}%
\pgfsys@transformshift{1.829388in}{2.232695in}%
\pgfsys@useobject{currentmarker}{}%
\end{pgfscope}%
\begin{pgfscope}%
\pgfsys@transformshift{1.841807in}{2.218566in}%
\pgfsys@useobject{currentmarker}{}%
\end{pgfscope}%
\begin{pgfscope}%
\pgfsys@transformshift{1.854198in}{2.197368in}%
\pgfsys@useobject{currentmarker}{}%
\end{pgfscope}%
\begin{pgfscope}%
\pgfsys@transformshift{1.862483in}{2.190308in}%
\pgfsys@useobject{currentmarker}{}%
\end{pgfscope}%
\begin{pgfscope}%
\pgfsys@transformshift{1.883242in}{2.186789in}%
\pgfsys@useobject{currentmarker}{}%
\end{pgfscope}%
\begin{pgfscope}%
\pgfsys@transformshift{1.895634in}{2.165584in}%
\pgfsys@useobject{currentmarker}{}%
\end{pgfscope}%
\begin{pgfscope}%
\pgfsys@transformshift{1.908064in}{2.154989in}%
\pgfsys@useobject{currentmarker}{}%
\end{pgfscope}%
\begin{pgfscope}%
\pgfsys@transformshift{1.912292in}{2.176202in}%
\pgfsys@useobject{currentmarker}{}%
\end{pgfscope}%
\begin{pgfscope}%
\pgfsys@transformshift{1.920499in}{2.147928in}%
\pgfsys@useobject{currentmarker}{}%
\end{pgfscope}%
\begin{pgfscope}%
\pgfsys@transformshift{1.937036in}{2.123196in}%
\pgfsys@useobject{currentmarker}{}%
\end{pgfscope}%
\begin{pgfscope}%
\pgfsys@transformshift{1.970164in}{2.091412in}%
\pgfsys@useobject{currentmarker}{}%
\end{pgfscope}%
\begin{pgfscope}%
\pgfsys@transformshift{1.991073in}{2.130312in}%
\pgfsys@useobject{currentmarker}{}%
\end{pgfscope}%
\begin{pgfscope}%
\pgfsys@transformshift{2.003066in}{1.995987in}%
\pgfsys@useobject{currentmarker}{}%
\end{pgfscope}%
\begin{pgfscope}%
\pgfsys@transformshift{2.036682in}{2.102070in}%
\pgfsys@useobject{currentmarker}{}%
\end{pgfscope}%
\begin{pgfscope}%
\pgfsys@transformshift{2.065443in}{2.010179in}%
\pgfsys@useobject{currentmarker}{}%
\end{pgfscope}%
\begin{pgfscope}%
\pgfsys@transformshift{2.082312in}{2.080896in}%
\pgfsys@useobject{currentmarker}{}%
\end{pgfscope}%
\begin{pgfscope}%
\pgfsys@transformshift{2.107040in}{2.024356in}%
\pgfsys@useobject{currentmarker}{}%
\end{pgfscope}%
\begin{pgfscope}%
\pgfsys@transformshift{2.115341in}{2.020821in}%
\pgfsys@useobject{currentmarker}{}%
\end{pgfscope}%
\begin{pgfscope}%
\pgfsys@transformshift{2.140196in}{1.999632in}%
\pgfsys@useobject{currentmarker}{}%
\end{pgfscope}%
\begin{pgfscope}%
\pgfsys@transformshift{2.144263in}{1.974892in}%
\pgfsys@useobject{currentmarker}{}%
\end{pgfscope}%
\begin{pgfscope}%
\pgfsys@transformshift{2.160872in}{1.971374in}%
\pgfsys@useobject{currentmarker}{}%
\end{pgfscope}%
\begin{pgfscope}%
\pgfsys@transformshift{2.177492in}{1.971381in}%
\pgfsys@useobject{currentmarker}{}%
\end{pgfscope}%
\begin{pgfscope}%
\pgfsys@transformshift{2.189894in}{1.953718in}%
\pgfsys@useobject{currentmarker}{}%
\end{pgfscope}%
\begin{pgfscope}%
\pgfsys@transformshift{2.194100in}{1.967863in}%
\pgfsys@useobject{currentmarker}{}%
\end{pgfscope}%
\begin{pgfscope}%
\pgfsys@transformshift{2.210509in}{1.907781in}%
\pgfsys@useobject{currentmarker}{}%
\end{pgfscope}%
\begin{pgfscope}%
\pgfsys@transformshift{2.210642in}{1.946665in}%
\pgfsys@useobject{currentmarker}{}%
\end{pgfscope}%
\begin{pgfscope}%
\pgfsys@transformshift{2.231368in}{1.932544in}%
\pgfsys@useobject{currentmarker}{}%
\end{pgfscope}%
\begin{pgfscope}%
\pgfsys@transformshift{2.243782in}{1.918407in}%
\pgfsys@useobject{currentmarker}{}%
\end{pgfscope}%
\begin{pgfscope}%
\pgfsys@transformshift{2.247877in}{1.900736in}%
\pgfsys@useobject{currentmarker}{}%
\end{pgfscope}%
\begin{pgfscope}%
\pgfsys@transformshift{2.260280in}{1.883073in}%
\pgfsys@useobject{currentmarker}{}%
\end{pgfscope}%
\begin{pgfscope}%
\pgfsys@transformshift{2.268642in}{1.897217in}%
\pgfsys@useobject{currentmarker}{}%
\end{pgfscope}%
\begin{pgfscope}%
\pgfsys@transformshift{2.276860in}{1.872478in}%
\pgfsys@useobject{currentmarker}{}%
\end{pgfscope}%
\begin{pgfscope}%
\pgfsys@transformshift{2.293480in}{1.872493in}%
\pgfsys@useobject{currentmarker}{}%
\end{pgfscope}%
\begin{pgfscope}%
\pgfsys@transformshift{2.297586in}{1.858356in}%
\pgfsys@useobject{currentmarker}{}%
\end{pgfscope}%
\begin{pgfscope}%
\pgfsys@transformshift{2.310055in}{1.858364in}%
\pgfsys@useobject{currentmarker}{}%
\end{pgfscope}%
\begin{pgfscope}%
\pgfsys@transformshift{2.314234in}{1.865441in}%
\pgfsys@useobject{currentmarker}{}%
\end{pgfscope}%
\begin{pgfscope}%
\pgfsys@transformshift{2.322508in}{1.854846in}%
\pgfsys@useobject{currentmarker}{}%
\end{pgfscope}%
\begin{pgfscope}%
\pgfsys@transformshift{2.363871in}{1.801848in}%
\pgfsys@useobject{currentmarker}{}%
\end{pgfscope}%
\begin{pgfscope}%
\pgfsys@transformshift{2.405429in}{1.805422in}%
\pgfsys@useobject{currentmarker}{}%
\end{pgfscope}%
\begin{pgfscope}%
\pgfsys@transformshift{2.413569in}{1.755934in}%
\pgfsys@useobject{currentmarker}{}%
\end{pgfscope}%
\begin{pgfscope}%
\pgfsys@transformshift{2.463428in}{1.755974in}%
\pgfsys@useobject{currentmarker}{}%
\end{pgfscope}%
\begin{pgfscope}%
\pgfsys@transformshift{2.492301in}{1.695899in}%
\pgfsys@useobject{currentmarker}{}%
\end{pgfscope}%
\begin{pgfscope}%
\pgfsys@transformshift{2.513199in}{1.731265in}%
\pgfsys@useobject{currentmarker}{}%
\end{pgfscope}%
\begin{pgfscope}%
\pgfsys@transformshift{2.537909in}{1.667657in}%
\pgfsys@useobject{currentmarker}{}%
\end{pgfscope}%
\begin{pgfscope}%
\pgfsys@transformshift{2.546377in}{1.713618in}%
\pgfsys@useobject{currentmarker}{}%
\end{pgfscope}%
\begin{pgfscope}%
\pgfsys@transformshift{2.641767in}{1.664201in}%
\pgfsys@useobject{currentmarker}{}%
\end{pgfscope}%
\begin{pgfscope}%
\pgfsys@transformshift{2.683247in}{1.643027in}%
\pgfsys@useobject{currentmarker}{}%
\end{pgfscope}%
\begin{pgfscope}%
\pgfsys@transformshift{2.762117in}{1.621877in}%
\pgfsys@useobject{currentmarker}{}%
\end{pgfscope}%
\begin{pgfscope}%
\pgfsys@transformshift{2.783053in}{1.667854in}%
\pgfsys@useobject{currentmarker}{}%
\end{pgfscope}%
\begin{pgfscope}%
\pgfsys@transformshift{2.807808in}{1.618382in}%
\pgfsys@useobject{currentmarker}{}%
\end{pgfscope}%
\begin{pgfscope}%
\pgfsys@transformshift{2.837030in}{1.657290in}%
\pgfsys@useobject{currentmarker}{}%
\end{pgfscope}%
\begin{pgfscope}%
\pgfsys@transformshift{2.882649in}{1.632582in}%
\pgfsys@useobject{currentmarker}{}%
\end{pgfscope}%
\begin{pgfscope}%
\pgfsys@transformshift{2.936687in}{1.639698in}%
\pgfsys@useobject{currentmarker}{}%
\end{pgfscope}%
\begin{pgfscope}%
\pgfsys@transformshift{2.961558in}{1.622042in}%
\pgfsys@useobject{currentmarker}{}%
\end{pgfscope}%
\begin{pgfscope}%
\pgfsys@transformshift{3.081991in}{1.604458in}%
\pgfsys@useobject{currentmarker}{}%
\end{pgfscope}%
\begin{pgfscope}%
\pgfsys@transformshift{3.115130in}{1.576207in}%
\pgfsys@useobject{currentmarker}{}%
\end{pgfscope}%
\begin{pgfscope}%
\pgfsys@transformshift{3.185716in}{1.562126in}%
\pgfsys@useobject{currentmarker}{}%
\end{pgfscope}%
\begin{pgfscope}%
\pgfsys@transformshift{3.239681in}{1.548028in}%
\pgfsys@useobject{currentmarker}{}%
\end{pgfscope}%
\begin{pgfscope}%
\pgfsys@transformshift{3.339338in}{1.530428in}%
\pgfsys@useobject{currentmarker}{}%
\end{pgfscope}%
\begin{pgfscope}%
\pgfsys@transformshift{3.401649in}{1.526941in}%
\pgfsys@useobject{currentmarker}{}%
\end{pgfscope}%
\begin{pgfscope}%
\pgfsys@transformshift{3.439045in}{1.526972in}%
\pgfsys@useobject{currentmarker}{}%
\end{pgfscope}%
\begin{pgfscope}%
\pgfsys@transformshift{3.497166in}{1.512882in}%
\pgfsys@useobject{currentmarker}{}%
\end{pgfscope}%
\begin{pgfscope}%
\pgfsys@transformshift{3.588577in}{1.512953in}%
\pgfsys@useobject{currentmarker}{}%
\end{pgfscope}%
\begin{pgfscope}%
\pgfsys@transformshift{3.650927in}{1.520069in}%
\pgfsys@useobject{currentmarker}{}%
\end{pgfscope}%
\begin{pgfscope}%
\pgfsys@transformshift{3.700836in}{1.534253in}%
\pgfsys@useobject{currentmarker}{}%
\end{pgfscope}%
\begin{pgfscope}%
\pgfsys@transformshift{3.725807in}{1.544879in}%
\pgfsys@useobject{currentmarker}{}%
\end{pgfscope}%
\begin{pgfscope}%
\pgfsys@transformshift{3.759030in}{1.541369in}%
\pgfsys@useobject{currentmarker}{}%
\end{pgfscope}%
\begin{pgfscope}%
\pgfsys@transformshift{3.846312in}{1.548508in}%
\pgfsys@useobject{currentmarker}{}%
\end{pgfscope}%
\begin{pgfscope}%
\pgfsys@transformshift{3.850568in}{1.576790in}%
\pgfsys@useobject{currentmarker}{}%
\end{pgfscope}%
\begin{pgfscope}%
\pgfsys@transformshift{3.871355in}{1.580340in}%
\pgfsys@useobject{currentmarker}{}%
\end{pgfscope}%
\begin{pgfscope}%
\pgfsys@transformshift{3.942062in}{1.601608in}%
\pgfsys@useobject{currentmarker}{}%
\end{pgfscope}%
\begin{pgfscope}%
\pgfsys@transformshift{3.950225in}{1.559197in}%
\pgfsys@useobject{currentmarker}{}%
\end{pgfscope}%
\begin{pgfscope}%
\pgfsys@transformshift{3.983464in}{1.559221in}%
\pgfsys@useobject{currentmarker}{}%
\end{pgfscope}%
\begin{pgfscope}%
\pgfsys@transformshift{4.045853in}{1.576947in}%
\pgfsys@useobject{currentmarker}{}%
\end{pgfscope}%
\begin{pgfscope}%
\pgfsys@transformshift{4.046025in}{1.626443in}%
\pgfsys@useobject{currentmarker}{}%
\end{pgfscope}%
\begin{pgfscope}%
\pgfsys@transformshift{4.066613in}{1.573429in}%
\pgfsys@useobject{currentmarker}{}%
\end{pgfscope}%
\begin{pgfscope}%
\pgfsys@transformshift{4.112332in}{1.577002in}%
\pgfsys@useobject{currentmarker}{}%
\end{pgfscope}%
\begin{pgfscope}%
\pgfsys@transformshift{4.145510in}{1.559355in}%
\pgfsys@useobject{currentmarker}{}%
\end{pgfscope}%
\begin{pgfscope}%
\pgfsys@transformshift{4.158240in}{1.633598in}%
\pgfsys@useobject{currentmarker}{}%
\end{pgfscope}%
\begin{pgfscope}%
\pgfsys@transformshift{4.182906in}{1.559379in}%
\pgfsys@useobject{currentmarker}{}%
\end{pgfscope}%
\begin{pgfscope}%
\pgfsys@transformshift{4.183105in}{1.615942in}%
\pgfsys@useobject{currentmarker}{}%
\end{pgfscope}%
\begin{pgfscope}%
\pgfsys@transformshift{4.199675in}{1.601813in}%
\pgfsys@useobject{currentmarker}{}%
\end{pgfscope}%
\begin{pgfscope}%
\pgfsys@transformshift{4.212066in}{1.580615in}%
\pgfsys@useobject{currentmarker}{}%
\end{pgfscope}%
\begin{pgfscope}%
\pgfsys@transformshift{4.212089in}{1.587684in}%
\pgfsys@useobject{currentmarker}{}%
\end{pgfscope}%
\begin{pgfscope}%
\pgfsys@transformshift{4.220412in}{1.591226in}%
\pgfsys@useobject{currentmarker}{}%
\end{pgfscope}%
\begin{pgfscope}%
\pgfsys@transformshift{4.220462in}{1.605371in}%
\pgfsys@useobject{currentmarker}{}%
\end{pgfscope}%
\begin{pgfscope}%
\pgfsys@transformshift{4.237082in}{1.605379in}%
\pgfsys@useobject{currentmarker}{}%
\end{pgfscope}%
\begin{pgfscope}%
\pgfsys@transformshift{4.249373in}{1.555899in}%
\pgfsys@useobject{currentmarker}{}%
\end{pgfscope}%
\begin{pgfscope}%
\pgfsys@transformshift{4.249473in}{1.584181in}%
\pgfsys@useobject{currentmarker}{}%
\end{pgfscope}%
\begin{pgfscope}%
\pgfsys@transformshift{4.249545in}{1.605395in}%
\pgfsys@useobject{currentmarker}{}%
\end{pgfscope}%
\begin{pgfscope}%
\pgfsys@transformshift{4.270149in}{1.555915in}%
\pgfsys@useobject{currentmarker}{}%
\end{pgfscope}%
\begin{pgfscope}%
\pgfsys@transformshift{4.270271in}{1.591265in}%
\pgfsys@useobject{currentmarker}{}%
\end{pgfscope}%
\begin{pgfscope}%
\pgfsys@transformshift{4.274466in}{1.601876in}%
\pgfsys@useobject{currentmarker}{}%
\end{pgfscope}%
\begin{pgfscope}%
\pgfsys@transformshift{4.282591in}{1.548855in}%
\pgfsys@useobject{currentmarker}{}%
\end{pgfscope}%
\begin{pgfscope}%
\pgfsys@transformshift{4.290936in}{1.559465in}%
\pgfsys@useobject{currentmarker}{}%
\end{pgfscope}%
\begin{pgfscope}%
\pgfsys@transformshift{4.291025in}{1.584213in}%
\pgfsys@useobject{currentmarker}{}%
\end{pgfscope}%
\begin{pgfscope}%
\pgfsys@transformshift{4.291058in}{1.594815in}%
\pgfsys@useobject{currentmarker}{}%
\end{pgfscope}%
\begin{pgfscope}%
\pgfsys@transformshift{4.307678in}{1.594831in}%
\pgfsys@useobject{currentmarker}{}%
\end{pgfscope}%
\begin{pgfscope}%
\pgfsys@transformshift{4.311795in}{1.584228in}%
\pgfsys@useobject{currentmarker}{}%
\end{pgfscope}%
\begin{pgfscope}%
\pgfsys@transformshift{4.328432in}{1.587778in}%
\pgfsys@useobject{currentmarker}{}%
\end{pgfscope}%
\begin{pgfscope}%
\pgfsys@transformshift{4.336617in}{1.552436in}%
\pgfsys@useobject{currentmarker}{}%
\end{pgfscope}%
\begin{pgfscope}%
\pgfsys@transformshift{4.336667in}{1.566573in}%
\pgfsys@useobject{currentmarker}{}%
\end{pgfscope}%
\begin{pgfscope}%
\pgfsys@transformshift{4.349169in}{1.577191in}%
\pgfsys@useobject{currentmarker}{}%
\end{pgfscope}%
\begin{pgfscope}%
\pgfsys@transformshift{4.353358in}{1.587802in}%
\pgfsys@useobject{currentmarker}{}%
\end{pgfscope}%
\begin{pgfscope}%
\pgfsys@transformshift{4.357404in}{1.555986in}%
\pgfsys@useobject{currentmarker}{}%
\end{pgfscope}%
\begin{pgfscope}%
\pgfsys@transformshift{4.361483in}{1.534781in}%
\pgfsys@useobject{currentmarker}{}%
\end{pgfscope}%
\begin{pgfscope}%
\pgfsys@transformshift{4.369945in}{1.577207in}%
\pgfsys@useobject{currentmarker}{}%
\end{pgfscope}%
\begin{pgfscope}%
\pgfsys@transformshift{4.378169in}{1.552467in}%
\pgfsys@useobject{currentmarker}{}%
\end{pgfscope}%
\begin{pgfscope}%
\pgfsys@transformshift{4.390654in}{1.559544in}%
\pgfsys@useobject{currentmarker}{}%
\end{pgfscope}%
\begin{pgfscope}%
\pgfsys@transformshift{4.390743in}{1.584291in}%
\pgfsys@useobject{currentmarker}{}%
\end{pgfscope}%
\begin{pgfscope}%
\pgfsys@transformshift{4.390854in}{1.616107in}%
\pgfsys@useobject{currentmarker}{}%
\end{pgfscope}%
\begin{pgfscope}%
\pgfsys@transformshift{4.411519in}{1.584307in}%
\pgfsys@useobject{currentmarker}{}%
\end{pgfscope}%
\begin{pgfscope}%
\pgfsys@transformshift{4.411630in}{1.616123in}%
\pgfsys@useobject{currentmarker}{}%
\end{pgfscope}%
\begin{pgfscope}%
\pgfsys@transformshift{4.415515in}{1.538354in}%
\pgfsys@useobject{currentmarker}{}%
\end{pgfscope}%
\begin{pgfscope}%
\pgfsys@transformshift{4.415886in}{1.644405in}%
\pgfsys@useobject{currentmarker}{}%
\end{pgfscope}%
\begin{pgfscope}%
\pgfsys@transformshift{4.424093in}{1.616139in}%
\pgfsys@useobject{currentmarker}{}%
\end{pgfscope}%
\begin{pgfscope}%
\pgfsys@transformshift{4.428211in}{1.605536in}%
\pgfsys@useobject{currentmarker}{}%
\end{pgfscope}%
\begin{pgfscope}%
\pgfsys@transformshift{4.432306in}{1.587865in}%
\pgfsys@useobject{currentmarker}{}%
\end{pgfscope}%
\begin{pgfscope}%
\pgfsys@transformshift{4.432345in}{1.598468in}%
\pgfsys@useobject{currentmarker}{}%
\end{pgfscope}%
\begin{pgfscope}%
\pgfsys@transformshift{4.453093in}{1.591415in}%
\pgfsys@useobject{currentmarker}{}%
\end{pgfscope}%
\begin{pgfscope}%
\pgfsys@transformshift{4.453182in}{1.616155in}%
\pgfsys@useobject{currentmarker}{}%
\end{pgfscope}%
\begin{pgfscope}%
\pgfsys@transformshift{4.469624in}{1.566683in}%
\pgfsys@useobject{currentmarker}{}%
\end{pgfscope}%
\begin{pgfscope}%
\pgfsys@transformshift{4.469713in}{1.591423in}%
\pgfsys@useobject{currentmarker}{}%
\end{pgfscope}%
\begin{pgfscope}%
\pgfsys@transformshift{4.482065in}{1.559623in}%
\pgfsys@useobject{currentmarker}{}%
\end{pgfscope}%
\begin{pgfscope}%
\pgfsys@transformshift{4.490350in}{1.552554in}%
\pgfsys@useobject{currentmarker}{}%
\end{pgfscope}%
\begin{pgfscope}%
\pgfsys@transformshift{4.490478in}{1.587904in}%
\pgfsys@useobject{currentmarker}{}%
\end{pgfscope}%
\begin{pgfscope}%
\pgfsys@transformshift{4.494506in}{1.552562in}%
\pgfsys@useobject{currentmarker}{}%
\end{pgfscope}%
\begin{pgfscope}%
\pgfsys@transformshift{4.523789in}{1.609141in}%
\pgfsys@useobject{currentmarker}{}%
\end{pgfscope}%
\begin{pgfscope}%
\pgfsys@transformshift{4.540165in}{1.538456in}%
\pgfsys@useobject{currentmarker}{}%
\end{pgfscope}%
\begin{pgfscope}%
\pgfsys@transformshift{4.573587in}{1.591509in}%
\pgfsys@useobject{currentmarker}{}%
\end{pgfscope}%
\begin{pgfscope}%
\pgfsys@transformshift{4.594324in}{1.580922in}%
\pgfsys@useobject{currentmarker}{}%
\end{pgfscope}%
\begin{pgfscope}%
\pgfsys@transformshift{4.594391in}{1.598594in}%
\pgfsys@useobject{currentmarker}{}%
\end{pgfscope}%
\begin{pgfscope}%
\pgfsys@transformshift{4.610811in}{1.542046in}%
\pgfsys@useobject{currentmarker}{}%
\end{pgfscope}%
\begin{pgfscope}%
\pgfsys@transformshift{4.635876in}{1.580954in}%
\pgfsys@useobject{currentmarker}{}%
\end{pgfscope}%
\begin{pgfscope}%
\pgfsys@transformshift{4.652446in}{1.566825in}%
\pgfsys@useobject{currentmarker}{}%
\end{pgfscope}%
\begin{pgfscope}%
\pgfsys@transformshift{4.673111in}{1.535025in}%
\pgfsys@useobject{currentmarker}{}%
\end{pgfscope}%
\begin{pgfscope}%
\pgfsys@transformshift{4.677339in}{1.556238in}%
\pgfsys@useobject{currentmarker}{}%
\end{pgfscope}%
\begin{pgfscope}%
\pgfsys@transformshift{4.693887in}{1.535040in}%
\pgfsys@useobject{currentmarker}{}%
\end{pgfscope}%
\begin{pgfscope}%
\pgfsys@transformshift{4.693987in}{1.563322in}%
\pgfsys@useobject{currentmarker}{}%
\end{pgfscope}%
\begin{pgfscope}%
\pgfsys@transformshift{4.714663in}{1.535064in}%
\pgfsys@useobject{currentmarker}{}%
\end{pgfscope}%
\begin{pgfscope}%
\pgfsys@transformshift{4.739190in}{1.421960in}%
\pgfsys@useobject{currentmarker}{}%
\end{pgfscope}%
\begin{pgfscope}%
\pgfsys@transformshift{4.755378in}{1.298246in}%
\pgfsys@useobject{currentmarker}{}%
\end{pgfscope}%
\begin{pgfscope}%
\pgfsys@transformshift{4.775866in}{1.216958in}%
\pgfsys@useobject{currentmarker}{}%
\end{pgfscope}%
\begin{pgfscope}%
\pgfsys@transformshift{4.787980in}{1.117984in}%
\pgfsys@useobject{currentmarker}{}%
\end{pgfscope}%
\begin{pgfscope}%
\pgfsys@transformshift{4.804555in}{1.103855in}%
\pgfsys@useobject{currentmarker}{}%
\end{pgfscope}%
\begin{pgfscope}%
\pgfsys@transformshift{4.812901in}{1.114465in}%
\pgfsys@useobject{currentmarker}{}%
\end{pgfscope}%
\begin{pgfscope}%
\pgfsys@transformshift{4.816991in}{1.096794in}%
\pgfsys@useobject{currentmarker}{}%
\end{pgfscope}%
\begin{pgfscope}%
\pgfsys@transformshift{4.821175in}{1.103871in}%
\pgfsys@useobject{currentmarker}{}%
\end{pgfscope}%
\begin{pgfscope}%
\pgfsys@transformshift{4.825276in}{1.089734in}%
\pgfsys@useobject{currentmarker}{}%
\end{pgfscope}%
\begin{pgfscope}%
\pgfsys@transformshift{4.829532in}{1.118015in}%
\pgfsys@useobject{currentmarker}{}%
\end{pgfscope}%
\begin{pgfscope}%
\pgfsys@transformshift{4.829582in}{1.132160in}%
\pgfsys@useobject{currentmarker}{}%
\end{pgfscope}%
\begin{pgfscope}%
\pgfsys@transformshift{4.846151in}{1.118031in}%
\pgfsys@useobject{currentmarker}{}%
\end{pgfscope}%
\begin{pgfscope}%
\pgfsys@transformshift{4.846201in}{1.132168in}%
\pgfsys@useobject{currentmarker}{}%
\end{pgfscope}%
\begin{pgfscope}%
\pgfsys@transformshift{4.850158in}{1.075612in}%
\pgfsys@useobject{currentmarker}{}%
\end{pgfscope}%
\begin{pgfscope}%
\pgfsys@transformshift{4.854464in}{1.118039in}%
\pgfsys@useobject{currentmarker}{}%
\end{pgfscope}%
\begin{pgfscope}%
\pgfsys@transformshift{4.858670in}{1.132184in}%
\pgfsys@useobject{currentmarker}{}%
\end{pgfscope}%
\begin{pgfscope}%
\pgfsys@transformshift{4.863081in}{1.206419in}%
\pgfsys@useobject{currentmarker}{}%
\end{pgfscope}%
\begin{pgfscope}%
\pgfsys@transformshift{4.867005in}{1.139260in}%
\pgfsys@useobject{currentmarker}{}%
\end{pgfscope}%
\begin{pgfscope}%
\pgfsys@transformshift{4.871416in}{1.213495in}%
\pgfsys@useobject{currentmarker}{}%
\end{pgfscope}%
\begin{pgfscope}%
\pgfsys@transformshift{4.883586in}{1.128665in}%
\pgfsys@useobject{currentmarker}{}%
\end{pgfscope}%
\begin{pgfscope}%
\pgfsys@transformshift{4.883636in}{1.142802in}%
\pgfsys@useobject{currentmarker}{}%
\end{pgfscope}%
\begin{pgfscope}%
\pgfsys@transformshift{4.883697in}{1.160481in}%
\pgfsys@useobject{currentmarker}{}%
\end{pgfscope}%
\begin{pgfscope}%
\pgfsys@transformshift{4.900078in}{1.093331in}%
\pgfsys@useobject{currentmarker}{}%
\end{pgfscope}%
\begin{pgfscope}%
\pgfsys@transformshift{4.904395in}{1.139284in}%
\pgfsys@useobject{currentmarker}{}%
\end{pgfscope}%
\begin{pgfscope}%
\pgfsys@transformshift{4.912957in}{1.209992in}%
\pgfsys@useobject{currentmarker}{}%
\end{pgfscope}%
\begin{pgfscope}%
\pgfsys@transformshift{4.921065in}{1.153444in}%
\pgfsys@useobject{currentmarker}{}%
\end{pgfscope}%
\begin{pgfscope}%
\pgfsys@transformshift{4.941879in}{1.164063in}%
\pgfsys@useobject{currentmarker}{}%
\end{pgfscope}%
\begin{pgfscope}%
\pgfsys@transformshift{4.941940in}{1.181734in}%
\pgfsys@useobject{currentmarker}{}%
\end{pgfscope}%
\begin{pgfscope}%
\pgfsys@transformshift{4.949976in}{1.103973in}%
\pgfsys@useobject{currentmarker}{}%
\end{pgfscope}%
\begin{pgfscope}%
\pgfsys@transformshift{4.950314in}{1.199421in}%
\pgfsys@useobject{currentmarker}{}%
\end{pgfscope}%
\begin{pgfscope}%
\pgfsys@transformshift{4.954343in}{1.164071in}%
\pgfsys@useobject{currentmarker}{}%
\end{pgfscope}%
\begin{pgfscope}%
\pgfsys@transformshift{4.966812in}{1.164079in}%
\pgfsys@useobject{currentmarker}{}%
\end{pgfscope}%
\begin{pgfscope}%
\pgfsys@transformshift{4.971029in}{1.181758in}%
\pgfsys@useobject{currentmarker}{}%
\end{pgfscope}%
\begin{pgfscope}%
\pgfsys@transformshift{4.983442in}{1.167629in}%
\pgfsys@useobject{currentmarker}{}%
\end{pgfscope}%
\begin{pgfscope}%
\pgfsys@transformshift{4.983464in}{1.174697in}%
\pgfsys@useobject{currentmarker}{}%
\end{pgfscope}%
\begin{pgfscope}%
\pgfsys@transformshift{4.983514in}{1.188842in}%
\pgfsys@useobject{currentmarker}{}%
\end{pgfscope}%
\begin{pgfscope}%
\pgfsys@transformshift{5.008397in}{1.174721in}%
\pgfsys@useobject{currentmarker}{}%
\end{pgfscope}%
\begin{pgfscope}%
\pgfsys@transformshift{5.012403in}{1.132302in}%
\pgfsys@useobject{currentmarker}{}%
\end{pgfscope}%
\begin{pgfscope}%
\pgfsys@transformshift{5.012653in}{1.203003in}%
\pgfsys@useobject{currentmarker}{}%
\end{pgfscope}%
\begin{pgfscope}%
\pgfsys@transformshift{5.025016in}{1.174736in}%
\pgfsys@useobject{currentmarker}{}%
\end{pgfscope}%
\begin{pgfscope}%
\pgfsys@transformshift{5.025055in}{1.185339in}%
\pgfsys@useobject{currentmarker}{}%
\end{pgfscope}%
\begin{pgfscope}%
\pgfsys@transformshift{5.025077in}{1.192408in}%
\pgfsys@useobject{currentmarker}{}%
\end{pgfscope}%
\begin{pgfscope}%
\pgfsys@transformshift{5.025116in}{1.203010in}%
\pgfsys@useobject{currentmarker}{}%
\end{pgfscope}%
\begin{pgfscope}%
\pgfsys@transformshift{5.037480in}{1.174744in}%
\pgfsys@useobject{currentmarker}{}%
\end{pgfscope}%
\begin{pgfscope}%
\pgfsys@transformshift{5.045643in}{1.132333in}%
\pgfsys@useobject{currentmarker}{}%
\end{pgfscope}%
\begin{pgfscope}%
\pgfsys@transformshift{5.045831in}{1.185355in}%
\pgfsys@useobject{currentmarker}{}%
\end{pgfscope}%
\begin{pgfscope}%
\pgfsys@transformshift{5.049949in}{1.174752in}%
\pgfsys@useobject{currentmarker}{}%
\end{pgfscope}%
\begin{pgfscope}%
\pgfsys@transformshift{5.054277in}{1.224247in}%
\pgfsys@useobject{currentmarker}{}%
\end{pgfscope}%
\begin{pgfscope}%
\pgfsys@transformshift{5.054299in}{1.231316in}%
\pgfsys@useobject{currentmarker}{}%
\end{pgfscope}%
\begin{pgfscope}%
\pgfsys@transformshift{5.054349in}{1.245461in}%
\pgfsys@useobject{currentmarker}{}%
\end{pgfscope}%
\begin{pgfscope}%
\pgfsys@transformshift{5.066679in}{1.206584in}%
\pgfsys@useobject{currentmarker}{}%
\end{pgfscope}%
\begin{pgfscope}%
\pgfsys@transformshift{5.066890in}{1.266682in}%
\pgfsys@useobject{currentmarker}{}%
\end{pgfscope}%
\begin{pgfscope}%
\pgfsys@transformshift{5.075091in}{1.234866in}%
\pgfsys@useobject{currentmarker}{}%
\end{pgfscope}%
\begin{pgfscope}%
\pgfsys@transformshift{5.079120in}{1.199523in}%
\pgfsys@useobject{currentmarker}{}%
\end{pgfscope}%
\begin{pgfscope}%
\pgfsys@transformshift{5.091412in}{1.150044in}%
\pgfsys@useobject{currentmarker}{}%
\end{pgfscope}%
\begin{pgfscope}%
\pgfsys@transformshift{5.091744in}{1.245484in}%
\pgfsys@useobject{currentmarker}{}%
\end{pgfscope}%
\begin{pgfscope}%
\pgfsys@transformshift{5.091794in}{1.259629in}%
\pgfsys@useobject{currentmarker}{}%
\end{pgfscope}%
\begin{pgfscope}%
\pgfsys@transformshift{5.104385in}{1.294987in}%
\pgfsys@useobject{currentmarker}{}%
\end{pgfscope}%
\begin{pgfscope}%
\pgfsys@transformshift{5.108414in}{1.259645in}%
\pgfsys@useobject{currentmarker}{}%
\end{pgfscope}%
\begin{pgfscope}%
\pgfsys@transformshift{5.166885in}{1.344530in}%
\pgfsys@useobject{currentmarker}{}%
\end{pgfscope}%
\begin{pgfscope}%
\pgfsys@transformshift{5.220989in}{1.369317in}%
\pgfsys@useobject{currentmarker}{}%
\end{pgfscope}%
\begin{pgfscope}%
\pgfsys@transformshift{5.283472in}{1.415325in}%
\pgfsys@useobject{currentmarker}{}%
\end{pgfscope}%
\begin{pgfscope}%
\pgfsys@transformshift{5.316766in}{1.429493in}%
\pgfsys@useobject{currentmarker}{}%
\end{pgfscope}%
\begin{pgfscope}%
\pgfsys@transformshift{5.345788in}{1.411838in}%
\pgfsys@useobject{currentmarker}{}%
\end{pgfscope}%
\begin{pgfscope}%
\pgfsys@transformshift{5.354107in}{1.415380in}%
\pgfsys@useobject{currentmarker}{}%
\end{pgfscope}%
\begin{pgfscope}%
\pgfsys@transformshift{5.358501in}{1.482546in}%
\pgfsys@useobject{currentmarker}{}%
\end{pgfscope}%
\begin{pgfscope}%
\pgfsys@transformshift{5.366797in}{1.479020in}%
\pgfsys@useobject{currentmarker}{}%
\end{pgfscope}%
\begin{pgfscope}%
\pgfsys@transformshift{5.383384in}{1.468425in}%
\pgfsys@useobject{currentmarker}{}%
\end{pgfscope}%
\begin{pgfscope}%
\pgfsys@transformshift{5.399987in}{1.464906in}%
\pgfsys@useobject{currentmarker}{}%
\end{pgfscope}%
\begin{pgfscope}%
\pgfsys@transformshift{5.424819in}{1.436648in}%
\pgfsys@useobject{currentmarker}{}%
\end{pgfscope}%
\begin{pgfscope}%
\pgfsys@transformshift{5.462165in}{1.422535in}%
\pgfsys@useobject{currentmarker}{}%
\end{pgfscope}%
\begin{pgfscope}%
\pgfsys@transformshift{5.503778in}{1.440245in}%
\pgfsys@useobject{currentmarker}{}%
\end{pgfscope}%
\end{pgfscope}%
\begin{pgfscope}%
\pgfpathrectangle{\pgfqpoint{0.832441in}{0.754477in}}{\pgfqpoint{4.987559in}{2.361409in}}%
\pgfusepath{clip}%
\pgfsetroundcap%
\pgfsetroundjoin%
\pgfsetlinewidth{1.505625pt}%
\definecolor{currentstroke}{rgb}{0.258824,0.521569,0.956863}%
\pgfsetstrokecolor{currentstroke}%
\pgfsetstrokeopacity{0.800000}%
\pgfsetdash{}{0pt}%
\pgfpathmoveto{\pgfqpoint{0.867092in}{3.120885in}}%
\pgfpathlineto{\pgfqpoint{1.132850in}{2.868639in}}%
\pgfpathlineto{\pgfqpoint{1.321646in}{2.690941in}}%
\pgfpathlineto{\pgfqpoint{1.466020in}{2.556550in}}%
\pgfpathlineto{\pgfqpoint{1.588182in}{2.444396in}}%
\pgfpathlineto{\pgfqpoint{1.688133in}{2.354131in}}%
\pgfpathlineto{\pgfqpoint{1.776979in}{2.275392in}}%
\pgfpathlineto{\pgfqpoint{1.854719in}{2.207951in}}%
\pgfpathlineto{\pgfqpoint{1.921353in}{2.151459in}}%
\pgfpathlineto{\pgfqpoint{1.987987in}{2.096399in}}%
\pgfpathlineto{\pgfqpoint{2.043515in}{2.051778in}}%
\pgfpathlineto{\pgfqpoint{2.099043in}{2.008456in}}%
\pgfpathlineto{\pgfqpoint{2.154572in}{1.966585in}}%
\pgfpathlineto{\pgfqpoint{2.210100in}{1.926318in}}%
\pgfpathlineto{\pgfqpoint{2.254523in}{1.895363in}}%
\pgfpathlineto{\pgfqpoint{2.298946in}{1.865609in}}%
\pgfpathlineto{\pgfqpoint{2.343368in}{1.837129in}}%
\pgfpathlineto{\pgfqpoint{2.387791in}{1.809990in}}%
\pgfpathlineto{\pgfqpoint{2.432214in}{1.784250in}}%
\pgfpathlineto{\pgfqpoint{2.476636in}{1.759959in}}%
\pgfpathlineto{\pgfqpoint{2.521059in}{1.737155in}}%
\pgfpathlineto{\pgfqpoint{2.565482in}{1.715864in}}%
\pgfpathlineto{\pgfqpoint{2.609905in}{1.696099in}}%
\pgfpathlineto{\pgfqpoint{2.654327in}{1.677858in}}%
\pgfpathlineto{\pgfqpoint{2.698750in}{1.661127in}}%
\pgfpathlineto{\pgfqpoint{2.743173in}{1.645879in}}%
\pgfpathlineto{\pgfqpoint{2.787595in}{1.632071in}}%
\pgfpathlineto{\pgfqpoint{2.832018in}{1.619654in}}%
\pgfpathlineto{\pgfqpoint{2.876441in}{1.608566in}}%
\pgfpathlineto{\pgfqpoint{2.920863in}{1.598740in}}%
\pgfpathlineto{\pgfqpoint{2.965286in}{1.590103in}}%
\pgfpathlineto{\pgfqpoint{3.009709in}{1.582576in}}%
\pgfpathlineto{\pgfqpoint{3.054132in}{1.576080in}}%
\pgfpathlineto{\pgfqpoint{3.098554in}{1.570537in}}%
\pgfpathlineto{\pgfqpoint{3.154083in}{1.564826in}}%
\pgfpathlineto{\pgfqpoint{3.209611in}{1.560331in}}%
\pgfpathlineto{\pgfqpoint{3.265139in}{1.556911in}}%
\pgfpathlineto{\pgfqpoint{3.331773in}{1.554042in}}%
\pgfpathlineto{\pgfqpoint{3.398407in}{1.552325in}}%
\pgfpathlineto{\pgfqpoint{3.476147in}{1.551538in}}%
\pgfpathlineto{\pgfqpoint{3.553887in}{1.551820in}}%
\pgfpathlineto{\pgfqpoint{3.642732in}{1.553185in}}%
\pgfpathlineto{\pgfqpoint{3.753789in}{1.556080in}}%
\pgfpathlineto{\pgfqpoint{3.887057in}{1.560798in}}%
\pgfpathlineto{\pgfqpoint{4.053642in}{1.567915in}}%
\pgfpathlineto{\pgfqpoint{4.286862in}{1.579093in}}%
\pgfpathlineto{\pgfqpoint{4.597820in}{1.594269in}}%
\pgfpathlineto{\pgfqpoint{4.620032in}{1.594516in}}%
\pgfpathlineto{\pgfqpoint{4.631137in}{1.594177in}}%
\pgfpathlineto{\pgfqpoint{4.642243in}{1.593260in}}%
\pgfpathlineto{\pgfqpoint{4.653349in}{1.591398in}}%
\pgfpathlineto{\pgfqpoint{4.664454in}{1.588012in}}%
\pgfpathlineto{\pgfqpoint{4.675560in}{1.582199in}}%
\pgfpathlineto{\pgfqpoint{4.686666in}{1.572608in}}%
\pgfpathlineto{\pgfqpoint{4.697772in}{1.557341in}}%
\pgfpathlineto{\pgfqpoint{4.708877in}{1.534024in}}%
\pgfpathlineto{\pgfqpoint{4.719983in}{1.500240in}}%
\pgfpathlineto{\pgfqpoint{4.731089in}{1.454538in}}%
\pgfpathlineto{\pgfqpoint{4.742194in}{1.397817in}}%
\pgfpathlineto{\pgfqpoint{4.764406in}{1.270595in}}%
\pgfpathlineto{\pgfqpoint{4.775511in}{1.214037in}}%
\pgfpathlineto{\pgfqpoint{4.786617in}{1.169461in}}%
\pgfpathlineto{\pgfqpoint{4.797723in}{1.138278in}}%
\pgfpathlineto{\pgfqpoint{4.808828in}{1.119052in}}%
\pgfpathlineto{\pgfqpoint{4.819934in}{1.109038in}}%
\pgfpathlineto{\pgfqpoint{4.831040in}{1.105453in}}%
\pgfpathlineto{\pgfqpoint{4.842145in}{1.106076in}}%
\pgfpathlineto{\pgfqpoint{4.853251in}{1.109340in}}%
\pgfpathlineto{\pgfqpoint{4.864357in}{1.114222in}}%
\pgfpathlineto{\pgfqpoint{4.886568in}{1.126514in}}%
\pgfpathlineto{\pgfqpoint{4.919885in}{1.147387in}}%
\pgfpathlineto{\pgfqpoint{5.019836in}{1.212614in}}%
\pgfpathlineto{\pgfqpoint{5.164210in}{1.306541in}}%
\pgfpathlineto{\pgfqpoint{5.219738in}{1.341658in}}%
\pgfpathlineto{\pgfqpoint{5.264161in}{1.368410in}}%
\pgfpathlineto{\pgfqpoint{5.297478in}{1.386975in}}%
\pgfpathlineto{\pgfqpoint{5.319689in}{1.398269in}}%
\pgfpathlineto{\pgfqpoint{5.341901in}{1.408442in}}%
\pgfpathlineto{\pgfqpoint{5.364112in}{1.417304in}}%
\pgfpathlineto{\pgfqpoint{5.386323in}{1.424737in}}%
\pgfpathlineto{\pgfqpoint{5.408535in}{1.430729in}}%
\pgfpathlineto{\pgfqpoint{5.430746in}{1.435378in}}%
\pgfpathlineto{\pgfqpoint{5.452957in}{1.438862in}}%
\pgfpathlineto{\pgfqpoint{5.475169in}{1.441397in}}%
\pgfpathlineto{\pgfqpoint{5.508486in}{1.443887in}}%
\pgfpathlineto{\pgfqpoint{5.552909in}{1.445653in}}%
\pgfpathlineto{\pgfqpoint{5.608437in}{1.446597in}}%
\pgfpathlineto{\pgfqpoint{5.719494in}{1.447081in}}%
\pgfpathlineto{\pgfqpoint{5.825000in}{1.447145in}}%
\pgfpathlineto{\pgfqpoint{5.825000in}{1.447145in}}%
\pgfusepath{stroke}%
\end{pgfscope}%
\begin{pgfscope}%
\pgfsetrectcap%
\pgfsetmiterjoin%
\pgfsetlinewidth{1.254687pt}%
\definecolor{currentstroke}{rgb}{1.000000,1.000000,1.000000}%
\pgfsetstrokecolor{currentstroke}%
\pgfsetdash{}{0pt}%
\pgfpathmoveto{\pgfqpoint{0.832441in}{0.754477in}}%
\pgfpathlineto{\pgfqpoint{0.832441in}{3.115885in}}%
\pgfusepath{stroke}%
\end{pgfscope}%
\begin{pgfscope}%
\pgfsetrectcap%
\pgfsetmiterjoin%
\pgfsetlinewidth{1.254687pt}%
\definecolor{currentstroke}{rgb}{1.000000,1.000000,1.000000}%
\pgfsetstrokecolor{currentstroke}%
\pgfsetdash{}{0pt}%
\pgfpathmoveto{\pgfqpoint{5.820000in}{0.754477in}}%
\pgfpathlineto{\pgfqpoint{5.820000in}{3.115885in}}%
\pgfusepath{stroke}%
\end{pgfscope}%
\begin{pgfscope}%
\pgfsetrectcap%
\pgfsetmiterjoin%
\pgfsetlinewidth{1.254687pt}%
\definecolor{currentstroke}{rgb}{1.000000,1.000000,1.000000}%
\pgfsetstrokecolor{currentstroke}%
\pgfsetdash{}{0pt}%
\pgfpathmoveto{\pgfqpoint{0.832441in}{0.754477in}}%
\pgfpathlineto{\pgfqpoint{5.820000in}{0.754477in}}%
\pgfusepath{stroke}%
\end{pgfscope}%
\begin{pgfscope}%
\pgfsetrectcap%
\pgfsetmiterjoin%
\pgfsetlinewidth{1.254687pt}%
\definecolor{currentstroke}{rgb}{1.000000,1.000000,1.000000}%
\pgfsetstrokecolor{currentstroke}%
\pgfsetdash{}{0pt}%
\pgfpathmoveto{\pgfqpoint{0.832441in}{3.115885in}}%
\pgfpathlineto{\pgfqpoint{5.820000in}{3.115885in}}%
\pgfusepath{stroke}%
\end{pgfscope}%
\begin{pgfscope}%
\pgfsetfillopacity{0.500000}%
\pgfsetstrokeopacity{0.500000}%
\definecolor{textcolor}{rgb}{0.150000,0.150000,0.150000}%
\pgfsetstrokecolor{textcolor}%
\pgfsetfillcolor{textcolor}%
\pgftext[x=0.932192in,y=0.801705in,left,base]{\color{textcolor}\sffamily\fontsize{9.000000}{10.800000}\selectfont Data from Panton: Incompressible Flow (4th Ed.)}%
\end{pgfscope}%
\begin{pgfscope}%
\definecolor{textcolor}{rgb}{0.150000,0.150000,0.150000}%
\pgfsetstrokecolor{textcolor}%
\pgfsetfillcolor{textcolor}%
\pgftext[x=3.326221in,y=3.199219in,,base]{\color{textcolor}\sffamily\fontsize{12.000000}{14.400000}\selectfont Cylinder Drag Coefficient}%
\end{pgfscope}%
\begin{pgfscope}%
\pgfsetbuttcap%
\pgfsetmiterjoin%
\definecolor{currentfill}{rgb}{0.917647,0.917647,0.949020}%
\pgfsetfillcolor{currentfill}%
\pgfsetfillopacity{0.800000}%
\pgfsetlinewidth{1.003750pt}%
\definecolor{currentstroke}{rgb}{0.800000,0.800000,0.800000}%
\pgfsetstrokecolor{currentstroke}%
\pgfsetstrokeopacity{0.800000}%
\pgfsetdash{}{0pt}%
\pgfpathmoveto{\pgfqpoint{4.415819in}{2.567719in}}%
\pgfpathlineto{\pgfqpoint{5.713056in}{2.567719in}}%
\pgfpathquadraticcurveto{\pgfqpoint{5.743611in}{2.567719in}}{\pgfqpoint{5.743611in}{2.598275in}}%
\pgfpathlineto{\pgfqpoint{5.743611in}{3.008941in}}%
\pgfpathquadraticcurveto{\pgfqpoint{5.743611in}{3.039497in}}{\pgfqpoint{5.713056in}{3.039497in}}%
\pgfpathlineto{\pgfqpoint{4.415819in}{3.039497in}}%
\pgfpathquadraticcurveto{\pgfqpoint{4.385264in}{3.039497in}}{\pgfqpoint{4.385264in}{3.008941in}}%
\pgfpathlineto{\pgfqpoint{4.385264in}{2.598275in}}%
\pgfpathquadraticcurveto{\pgfqpoint{4.385264in}{2.567719in}}{\pgfqpoint{4.415819in}{2.567719in}}%
\pgfpathclose%
\pgfusepath{stroke,fill}%
\end{pgfscope}%
\begin{pgfscope}%
\pgfsetbuttcap%
\pgfsetroundjoin%
\definecolor{currentfill}{rgb}{0.000000,0.000000,0.000000}%
\pgfsetfillcolor{currentfill}%
\pgfsetfillopacity{0.800000}%
\pgfsetlinewidth{1.003750pt}%
\definecolor{currentstroke}{rgb}{0.000000,0.000000,0.000000}%
\pgfsetstrokecolor{currentstroke}%
\pgfsetstrokeopacity{0.800000}%
\pgfsetdash{}{0pt}%
\pgfsys@defobject{currentmarker}{\pgfqpoint{-0.010417in}{-0.010417in}}{\pgfqpoint{0.010417in}{0.010417in}}{%
\pgfpathmoveto{\pgfqpoint{0.000000in}{-0.010417in}}%
\pgfpathcurveto{\pgfqpoint{0.002763in}{-0.010417in}}{\pgfqpoint{0.005412in}{-0.009319in}}{\pgfqpoint{0.007366in}{-0.007366in}}%
\pgfpathcurveto{\pgfqpoint{0.009319in}{-0.005412in}}{\pgfqpoint{0.010417in}{-0.002763in}}{\pgfqpoint{0.010417in}{0.000000in}}%
\pgfpathcurveto{\pgfqpoint{0.010417in}{0.002763in}}{\pgfqpoint{0.009319in}{0.005412in}}{\pgfqpoint{0.007366in}{0.007366in}}%
\pgfpathcurveto{\pgfqpoint{0.005412in}{0.009319in}}{\pgfqpoint{0.002763in}{0.010417in}}{\pgfqpoint{0.000000in}{0.010417in}}%
\pgfpathcurveto{\pgfqpoint{-0.002763in}{0.010417in}}{\pgfqpoint{-0.005412in}{0.009319in}}{\pgfqpoint{-0.007366in}{0.007366in}}%
\pgfpathcurveto{\pgfqpoint{-0.009319in}{0.005412in}}{\pgfqpoint{-0.010417in}{0.002763in}}{\pgfqpoint{-0.010417in}{0.000000in}}%
\pgfpathcurveto{\pgfqpoint{-0.010417in}{-0.002763in}}{\pgfqpoint{-0.009319in}{-0.005412in}}{\pgfqpoint{-0.007366in}{-0.007366in}}%
\pgfpathcurveto{\pgfqpoint{-0.005412in}{-0.009319in}}{\pgfqpoint{-0.002763in}{-0.010417in}}{\pgfqpoint{0.000000in}{-0.010417in}}%
\pgfpathclose%
\pgfusepath{stroke,fill}%
}%
\begin{pgfscope}%
\pgfsys@transformshift{4.599153in}{2.924913in}%
\pgfsys@useobject{currentmarker}{}%
\end{pgfscope}%
\end{pgfscope}%
\begin{pgfscope}%
\definecolor{textcolor}{rgb}{0.150000,0.150000,0.150000}%
\pgfsetstrokecolor{textcolor}%
\pgfsetfillcolor{textcolor}%
\pgftext[x=4.874153in,y=2.871441in,left,base]{\color{textcolor}\sffamily\fontsize{11.000000}{13.200000}\selectfont Data}%
\end{pgfscope}%
\begin{pgfscope}%
\pgfsetroundcap%
\pgfsetroundjoin%
\pgfsetlinewidth{1.505625pt}%
\definecolor{currentstroke}{rgb}{0.258824,0.521569,0.956863}%
\pgfsetstrokecolor{currentstroke}%
\pgfsetstrokeopacity{0.800000}%
\pgfsetdash{}{0pt}%
\pgfpathmoveto{\pgfqpoint{4.446375in}{2.711941in}}%
\pgfpathlineto{\pgfqpoint{4.751931in}{2.711941in}}%
\pgfusepath{stroke}%
\end{pgfscope}%
\begin{pgfscope}%
\definecolor{textcolor}{rgb}{0.150000,0.150000,0.150000}%
\pgfsetstrokecolor{textcolor}%
\pgfsetfillcolor{textcolor}%
\pgftext[x=4.874153in,y=2.658469in,left,base]{\color{textcolor}\sffamily\fontsize{11.000000}{13.200000}\selectfont Fitted Model}%
\end{pgfscope}%
\end{pgfpicture}%
\makeatother%
\endgroup%
}}
    \caption{Analytical fitting of cylinder drag data from Panton \cite{Panton}. Fit minimizes the $L_1$-norm of log-transformed error.}
    \label{fig:cylinder-drag}
\end{figure}

\subsection{Interpolated Models}
\label{sect:interpolation}

Another approach to creating surrogate models is interpolation. Interpolation forgoes the need for an analytic expression, instead representing the surrogate model in the form of a lookup table with rules for computing intermediate values. Interpolation is implemented in AeroSandbox via the syntax \mintinline{python}{asb.InterpolatedModel}, with inputs that are analogous to those for fitted models.

There are several challenges with interpolation that must be addressed in order to use an interpolated model in a differentiable optimization framework:

\begin{enumerate}
    \item Interpolated models must be at least $C_1$-continuous, following the logic in Section \ref{sect:differentiability}. This means that many common interpolation techniques, such as linear interpolation, are not permissible. In AeroSandbox, this is solved by defaulting to a b-spline interpolation that consists of piecewise cubic polynomial patches; this is therefore a $C_2$-continuous representation and can be adequately treated with modern gradient-based optimization algorithms.
    \item Interpolated models must extend to multidimensional datasets with an arbitrary number of inputs.
\end{enumerate}

This combination of requirements is quite tricky to satisfy, and hence few underlying packages support this. For example, the SciPy library that forms the standard toolbox for scientific computing in Python only supports spline interpolation on 1D and 2D datasets \cite{scipy}. Thankfully, the CasADi automatic differentiation library \cite{casadi} includes routines that enable $n$-dimensional spline interpolation, so these have been wrapped for use in AeroSandbox.

These $n$-dimensional spline interpolators are demonstrated in Figure \ref{fig:interpolated-xfoil-3d}, where a synthetic dataset depicting lift of a SD7032 airfoil is turned into a surrogate model using the AeroSandbox interpolator library. Here, the $C_2$ continuity of the piecewise-cubic interpolation is clearly visible.

\begin{figure}[H]
    \centering
%    %% Creator: Matplotlib, PGF backend
%%
%% To include the figure in your LaTeX document, write
%%   \input{<filename>.pgf}
%%
%% Make sure the required packages are loaded in your preamble
%%   \usepackage{pgf}
%%
%% Figures using additional raster images can only be included by \input if
%% they are in the same directory as the main LaTeX file. For loading figures
%% from other directories you can use the `import` package
%%   \usepackage{import}
%%
%% and then include the figures with
%%   \import{<path to file>}{<filename>.pgf}
%%
%% Matplotlib used the following preamble
%%   \usepackage{fontspec}
%%
\begingroup%
\makeatletter%
\begin{pgfpicture}%
\pgfpathrectangle{\pgfpointorigin}{\pgfqpoint{6.000000in}{5.000000in}}%
\pgfusepath{use as bounding box, clip}%
\begin{pgfscope}%
\pgfsetbuttcap%
\pgfsetmiterjoin%
\definecolor{currentfill}{rgb}{1.000000,1.000000,1.000000}%
\pgfsetfillcolor{currentfill}%
\pgfsetlinewidth{0.000000pt}%
\definecolor{currentstroke}{rgb}{1.000000,1.000000,1.000000}%
\pgfsetstrokecolor{currentstroke}%
\pgfsetstrokeopacity{0.000000}%
\pgfsetdash{}{0pt}%
\pgfpathmoveto{\pgfqpoint{0.000000in}{0.000000in}}%
\pgfpathlineto{\pgfqpoint{6.000000in}{0.000000in}}%
\pgfpathlineto{\pgfqpoint{6.000000in}{5.000000in}}%
\pgfpathlineto{\pgfqpoint{0.000000in}{5.000000in}}%
\pgfpathclose%
\pgfusepath{fill}%
\end{pgfscope}%
\begin{pgfscope}%
\pgfsetbuttcap%
\pgfsetmiterjoin%
\definecolor{currentfill}{rgb}{0.917647,0.917647,0.949020}%
\pgfsetfillcolor{currentfill}%
\pgfsetlinewidth{0.000000pt}%
\definecolor{currentstroke}{rgb}{0.000000,0.000000,0.000000}%
\pgfsetstrokecolor{currentstroke}%
\pgfsetstrokeopacity{0.000000}%
\pgfsetdash{}{0pt}%
\pgfpathmoveto{\pgfqpoint{0.887500in}{0.275000in}}%
\pgfpathlineto{\pgfqpoint{5.112500in}{0.275000in}}%
\pgfpathlineto{\pgfqpoint{5.112500in}{4.500000in}}%
\pgfpathlineto{\pgfqpoint{0.887500in}{4.500000in}}%
\pgfpathclose%
\pgfusepath{fill}%
\end{pgfscope}%
\begin{pgfscope}%
\pgfsetbuttcap%
\pgfsetmiterjoin%
\definecolor{currentfill}{rgb}{0.950000,0.950000,0.950000}%
\pgfsetfillcolor{currentfill}%
\pgfsetfillopacity{0.500000}%
\pgfsetlinewidth{1.003750pt}%
\definecolor{currentstroke}{rgb}{0.950000,0.950000,0.950000}%
\pgfsetstrokecolor{currentstroke}%
\pgfsetstrokeopacity{0.500000}%
\pgfsetdash{}{0pt}%
\pgfpathmoveto{\pgfqpoint{4.907673in}{1.316750in}}%
\pgfpathlineto{\pgfqpoint{3.512440in}{2.486262in}}%
\pgfpathlineto{\pgfqpoint{3.531835in}{4.172907in}}%
\pgfpathlineto{\pgfqpoint{4.993837in}{3.106004in}}%
\pgfusepath{stroke,fill}%
\end{pgfscope}%
\begin{pgfscope}%
\pgfsetbuttcap%
\pgfsetmiterjoin%
\definecolor{currentfill}{rgb}{0.900000,0.900000,0.900000}%
\pgfsetfillcolor{currentfill}%
\pgfsetfillopacity{0.500000}%
\pgfsetlinewidth{1.003750pt}%
\definecolor{currentstroke}{rgb}{0.900000,0.900000,0.900000}%
\pgfsetstrokecolor{currentstroke}%
\pgfsetstrokeopacity{0.500000}%
\pgfsetdash{}{0pt}%
\pgfpathmoveto{\pgfqpoint{1.273588in}{1.835516in}}%
\pgfpathlineto{\pgfqpoint{3.512440in}{2.486262in}}%
\pgfpathlineto{\pgfqpoint{3.531835in}{4.172907in}}%
\pgfpathlineto{\pgfqpoint{1.193692in}{3.580256in}}%
\pgfusepath{stroke,fill}%
\end{pgfscope}%
\begin{pgfscope}%
\pgfsetbuttcap%
\pgfsetmiterjoin%
\definecolor{currentfill}{rgb}{0.925000,0.925000,0.925000}%
\pgfsetfillcolor{currentfill}%
\pgfsetfillopacity{0.500000}%
\pgfsetlinewidth{1.003750pt}%
\definecolor{currentstroke}{rgb}{0.925000,0.925000,0.925000}%
\pgfsetstrokecolor{currentstroke}%
\pgfsetstrokeopacity{0.500000}%
\pgfsetdash{}{0pt}%
\pgfpathmoveto{\pgfqpoint{2.534381in}{0.541633in}}%
\pgfpathlineto{\pgfqpoint{4.907673in}{1.316750in}}%
\pgfpathlineto{\pgfqpoint{3.512440in}{2.486262in}}%
\pgfpathlineto{\pgfqpoint{1.273588in}{1.835516in}}%
\pgfusepath{stroke,fill}%
\end{pgfscope}%
\begin{pgfscope}%
\pgfsetroundcap%
\pgfsetroundjoin%
\pgfsetlinewidth{1.254687pt}%
\definecolor{currentstroke}{rgb}{1.000000,1.000000,1.000000}%
\pgfsetstrokecolor{currentstroke}%
\pgfsetdash{}{0pt}%
\pgfpathmoveto{\pgfqpoint{4.907673in}{1.316750in}}%
\pgfpathlineto{\pgfqpoint{2.534381in}{0.541633in}}%
\pgfusepath{stroke}%
\end{pgfscope}%
\begin{pgfscope}%
\definecolor{textcolor}{rgb}{0.150000,0.150000,0.150000}%
\pgfsetstrokecolor{textcolor}%
\pgfsetfillcolor{textcolor}%
\pgftext[x=3.420909in, y=0.225013in, left, base,rotate=18.087038]{\color{textcolor}\sffamily\fontsize{12.000000}{14.400000}\selectfont Reynolds Number}%
\end{pgfscope}%
\begin{pgfscope}%
\pgfsetbuttcap%
\pgfsetroundjoin%
\pgfsetlinewidth{1.003750pt}%
\definecolor{currentstroke}{rgb}{1.000000,1.000000,1.000000}%
\pgfsetstrokecolor{currentstroke}%
\pgfsetdash{}{0pt}%
\pgfpathmoveto{\pgfqpoint{2.584363in}{0.557957in}}%
\pgfpathlineto{\pgfqpoint{1.320536in}{1.849162in}}%
\pgfpathlineto{\pgfqpoint{1.242823in}{3.592709in}}%
\pgfusepath{stroke}%
\end{pgfscope}%
\begin{pgfscope}%
\pgfsetbuttcap%
\pgfsetroundjoin%
\pgfsetlinewidth{1.003750pt}%
\definecolor{currentstroke}{rgb}{1.000000,1.000000,1.000000}%
\pgfsetstrokecolor{currentstroke}%
\pgfsetdash{}{0pt}%
\pgfpathmoveto{\pgfqpoint{3.369619in}{0.814421in}}%
\pgfpathlineto{\pgfqpoint{2.059260in}{2.063880in}}%
\pgfpathlineto{\pgfqpoint{2.015332in}{3.788518in}}%
\pgfusepath{stroke}%
\end{pgfscope}%
\begin{pgfscope}%
\pgfsetbuttcap%
\pgfsetroundjoin%
\pgfsetlinewidth{1.003750pt}%
\definecolor{currentstroke}{rgb}{1.000000,1.000000,1.000000}%
\pgfsetstrokecolor{currentstroke}%
\pgfsetdash{}{0pt}%
\pgfpathmoveto{\pgfqpoint{4.128599in}{1.062304in}}%
\pgfpathlineto{\pgfqpoint{2.775317in}{2.272009in}}%
\pgfpathlineto{\pgfqpoint{2.763111in}{3.978058in}}%
\pgfusepath{stroke}%
\end{pgfscope}%
\begin{pgfscope}%
\pgfsetbuttcap%
\pgfsetroundjoin%
\pgfsetlinewidth{1.003750pt}%
\definecolor{currentstroke}{rgb}{1.000000,1.000000,1.000000}%
\pgfsetstrokecolor{currentstroke}%
\pgfsetdash{}{0pt}%
\pgfpathmoveto{\pgfqpoint{4.862599in}{1.302029in}}%
\pgfpathlineto{\pgfqpoint{3.469735in}{2.473849in}}%
\pgfpathlineto{\pgfqpoint{3.487328in}{4.161626in}}%
\pgfusepath{stroke}%
\end{pgfscope}%
\begin{pgfscope}%
\pgfsetbuttcap%
\pgfsetroundjoin%
\pgfsetlinewidth{1.003750pt}%
\definecolor{currentstroke}{rgb}{1.000000,1.000000,1.000000}%
\pgfsetstrokecolor{currentstroke}%
\pgfsetdash{}{0pt}%
\pgfpathmoveto{\pgfqpoint{2.823594in}{0.636090in}}%
\pgfpathlineto{\pgfqpoint{1.545362in}{1.914510in}}%
\pgfpathlineto{\pgfqpoint{1.478046in}{3.652331in}}%
\pgfusepath{stroke}%
\end{pgfscope}%
\begin{pgfscope}%
\pgfsetbuttcap%
\pgfsetroundjoin%
\pgfsetlinewidth{1.003750pt}%
\definecolor{currentstroke}{rgb}{1.000000,1.000000,1.000000}%
\pgfsetstrokecolor{currentstroke}%
\pgfsetdash{}{0pt}%
\pgfpathmoveto{\pgfqpoint{3.600798in}{0.889924in}}%
\pgfpathlineto{\pgfqpoint{2.277151in}{2.127212in}}%
\pgfpathlineto{\pgfqpoint{2.242983in}{3.846220in}}%
\pgfusepath{stroke}%
\end{pgfscope}%
\begin{pgfscope}%
\pgfsetbuttcap%
\pgfsetroundjoin%
\pgfsetlinewidth{1.003750pt}%
\definecolor{currentstroke}{rgb}{1.000000,1.000000,1.000000}%
\pgfsetstrokecolor{currentstroke}%
\pgfsetdash{}{0pt}%
\pgfpathmoveto{\pgfqpoint{4.352126in}{1.135308in}}%
\pgfpathlineto{\pgfqpoint{2.986589in}{2.333417in}}%
\pgfpathlineto{\pgfqpoint{2.983549in}{4.033932in}}%
\pgfusepath{stroke}%
\end{pgfscope}%
\begin{pgfscope}%
\pgfsetbuttcap%
\pgfsetroundjoin%
\pgfsetlinewidth{1.003750pt}%
\definecolor{currentstroke}{rgb}{1.000000,1.000000,1.000000}%
\pgfsetstrokecolor{currentstroke}%
\pgfsetdash{}{0pt}%
\pgfpathmoveto{\pgfqpoint{2.962389in}{0.681420in}}%
\pgfpathlineto{\pgfqpoint{1.675892in}{1.952450in}}%
\pgfpathlineto{\pgfqpoint{1.614565in}{3.686935in}}%
\pgfusepath{stroke}%
\end{pgfscope}%
\begin{pgfscope}%
\pgfsetbuttcap%
\pgfsetroundjoin%
\pgfsetlinewidth{1.003750pt}%
\definecolor{currentstroke}{rgb}{1.000000,1.000000,1.000000}%
\pgfsetstrokecolor{currentstroke}%
\pgfsetdash{}{0pt}%
\pgfpathmoveto{\pgfqpoint{3.734940in}{0.933735in}}%
\pgfpathlineto{\pgfqpoint{2.403669in}{2.163986in}}%
\pgfpathlineto{\pgfqpoint{2.375124in}{3.879714in}}%
\pgfusepath{stroke}%
\end{pgfscope}%
\begin{pgfscope}%
\pgfsetbuttcap%
\pgfsetroundjoin%
\pgfsetlinewidth{1.003750pt}%
\definecolor{currentstroke}{rgb}{1.000000,1.000000,1.000000}%
\pgfsetstrokecolor{currentstroke}%
\pgfsetdash{}{0pt}%
\pgfpathmoveto{\pgfqpoint{4.481845in}{1.177674in}}%
\pgfpathlineto{\pgfqpoint{3.109277in}{2.369078in}}%
\pgfpathlineto{\pgfqpoint{3.111520in}{4.066369in}}%
\pgfusepath{stroke}%
\end{pgfscope}%
\begin{pgfscope}%
\pgfsetbuttcap%
\pgfsetroundjoin%
\pgfsetlinewidth{1.003750pt}%
\definecolor{currentstroke}{rgb}{1.000000,1.000000,1.000000}%
\pgfsetstrokecolor{currentstroke}%
\pgfsetdash{}{0pt}%
\pgfpathmoveto{\pgfqpoint{3.060357in}{0.713417in}}%
\pgfpathlineto{\pgfqpoint{1.768067in}{1.979241in}}%
\pgfpathlineto{\pgfqpoint{1.710950in}{3.711366in}}%
\pgfusepath{stroke}%
\end{pgfscope}%
\begin{pgfscope}%
\pgfsetbuttcap%
\pgfsetroundjoin%
\pgfsetlinewidth{1.003750pt}%
\definecolor{currentstroke}{rgb}{1.000000,1.000000,1.000000}%
\pgfsetstrokecolor{currentstroke}%
\pgfsetdash{}{0pt}%
\pgfpathmoveto{\pgfqpoint{3.829633in}{0.964662in}}%
\pgfpathlineto{\pgfqpoint{2.493017in}{2.189956in}}%
\pgfpathlineto{\pgfqpoint{2.468425in}{3.903364in}}%
\pgfusepath{stroke}%
\end{pgfscope}%
\begin{pgfscope}%
\pgfsetbuttcap%
\pgfsetroundjoin%
\pgfsetlinewidth{1.003750pt}%
\definecolor{currentstroke}{rgb}{1.000000,1.000000,1.000000}%
\pgfsetstrokecolor{currentstroke}%
\pgfsetdash{}{0pt}%
\pgfpathmoveto{\pgfqpoint{4.573424in}{1.207584in}}%
\pgfpathlineto{\pgfqpoint{3.195927in}{2.394264in}}%
\pgfpathlineto{\pgfqpoint{3.201884in}{4.089274in}}%
\pgfusepath{stroke}%
\end{pgfscope}%
\begin{pgfscope}%
\pgfsetbuttcap%
\pgfsetroundjoin%
\pgfsetlinewidth{1.003750pt}%
\definecolor{currentstroke}{rgb}{1.000000,1.000000,1.000000}%
\pgfsetstrokecolor{currentstroke}%
\pgfsetdash{}{0pt}%
\pgfpathmoveto{\pgfqpoint{3.136060in}{0.738141in}}%
\pgfpathlineto{\pgfqpoint{1.839315in}{1.999950in}}%
\pgfpathlineto{\pgfqpoint{1.785440in}{3.730247in}}%
\pgfusepath{stroke}%
\end{pgfscope}%
\begin{pgfscope}%
\pgfsetbuttcap%
\pgfsetroundjoin%
\pgfsetlinewidth{1.003750pt}%
\definecolor{currentstroke}{rgb}{1.000000,1.000000,1.000000}%
\pgfsetstrokecolor{currentstroke}%
\pgfsetdash{}{0pt}%
\pgfpathmoveto{\pgfqpoint{3.902809in}{0.988561in}}%
\pgfpathlineto{\pgfqpoint{2.562084in}{2.210031in}}%
\pgfpathlineto{\pgfqpoint{2.540537in}{3.921642in}}%
\pgfusepath{stroke}%
\end{pgfscope}%
\begin{pgfscope}%
\pgfsetbuttcap%
\pgfsetroundjoin%
\pgfsetlinewidth{1.003750pt}%
\definecolor{currentstroke}{rgb}{1.000000,1.000000,1.000000}%
\pgfsetstrokecolor{currentstroke}%
\pgfsetdash{}{0pt}%
\pgfpathmoveto{\pgfqpoint{4.644197in}{1.230698in}}%
\pgfpathlineto{\pgfqpoint{3.262912in}{2.413734in}}%
\pgfpathlineto{\pgfqpoint{3.271729in}{4.106978in}}%
\pgfusepath{stroke}%
\end{pgfscope}%
\begin{pgfscope}%
\pgfsetbuttcap%
\pgfsetroundjoin%
\pgfsetlinewidth{1.003750pt}%
\definecolor{currentstroke}{rgb}{1.000000,1.000000,1.000000}%
\pgfsetstrokecolor{currentstroke}%
\pgfsetdash{}{0pt}%
\pgfpathmoveto{\pgfqpoint{3.197727in}{0.758281in}}%
\pgfpathlineto{\pgfqpoint{1.897369in}{2.016824in}}%
\pgfpathlineto{\pgfqpoint{1.846129in}{3.745630in}}%
\pgfusepath{stroke}%
\end{pgfscope}%
\begin{pgfscope}%
\pgfsetbuttcap%
\pgfsetroundjoin%
\pgfsetlinewidth{1.003750pt}%
\definecolor{currentstroke}{rgb}{1.000000,1.000000,1.000000}%
\pgfsetstrokecolor{currentstroke}%
\pgfsetdash{}{0pt}%
\pgfpathmoveto{\pgfqpoint{3.962420in}{1.008030in}}%
\pgfpathlineto{\pgfqpoint{2.618363in}{2.226389in}}%
\pgfpathlineto{\pgfqpoint{2.599290in}{3.936534in}}%
\pgfusepath{stroke}%
\end{pgfscope}%
\begin{pgfscope}%
\pgfsetbuttcap%
\pgfsetroundjoin%
\pgfsetlinewidth{1.003750pt}%
\definecolor{currentstroke}{rgb}{1.000000,1.000000,1.000000}%
\pgfsetstrokecolor{currentstroke}%
\pgfsetdash{}{0pt}%
\pgfpathmoveto{\pgfqpoint{4.701855in}{1.249529in}}%
\pgfpathlineto{\pgfqpoint{3.317496in}{2.429599in}}%
\pgfpathlineto{\pgfqpoint{3.328638in}{4.121402in}}%
\pgfusepath{stroke}%
\end{pgfscope}%
\begin{pgfscope}%
\pgfsetbuttcap%
\pgfsetroundjoin%
\pgfsetlinewidth{1.003750pt}%
\definecolor{currentstroke}{rgb}{1.000000,1.000000,1.000000}%
\pgfsetstrokecolor{currentstroke}%
\pgfsetdash{}{0pt}%
\pgfpathmoveto{\pgfqpoint{3.249735in}{0.775267in}}%
\pgfpathlineto{\pgfqpoint{1.946340in}{2.031058in}}%
\pgfpathlineto{\pgfqpoint{1.897318in}{3.758604in}}%
\pgfusepath{stroke}%
\end{pgfscope}%
\begin{pgfscope}%
\pgfsetbuttcap%
\pgfsetroundjoin%
\pgfsetlinewidth{1.003750pt}%
\definecolor{currentstroke}{rgb}{1.000000,1.000000,1.000000}%
\pgfsetstrokecolor{currentstroke}%
\pgfsetdash{}{0pt}%
\pgfpathmoveto{\pgfqpoint{4.012698in}{1.024451in}}%
\pgfpathlineto{\pgfqpoint{2.665840in}{2.240188in}}%
\pgfpathlineto{\pgfqpoint{2.648849in}{3.949096in}}%
\pgfusepath{stroke}%
\end{pgfscope}%
\begin{pgfscope}%
\pgfsetbuttcap%
\pgfsetroundjoin%
\pgfsetlinewidth{1.003750pt}%
\definecolor{currentstroke}{rgb}{1.000000,1.000000,1.000000}%
\pgfsetstrokecolor{currentstroke}%
\pgfsetdash{}{0pt}%
\pgfpathmoveto{\pgfqpoint{4.750486in}{1.265412in}}%
\pgfpathlineto{\pgfqpoint{3.363545in}{2.442984in}}%
\pgfpathlineto{\pgfqpoint{3.376643in}{4.133570in}}%
\pgfusepath{stroke}%
\end{pgfscope}%
\begin{pgfscope}%
\pgfsetbuttcap%
\pgfsetroundjoin%
\pgfsetlinewidth{1.003750pt}%
\definecolor{currentstroke}{rgb}{1.000000,1.000000,1.000000}%
\pgfsetstrokecolor{currentstroke}%
\pgfsetdash{}{0pt}%
\pgfpathmoveto{\pgfqpoint{3.294691in}{0.789950in}}%
\pgfpathlineto{\pgfqpoint{1.988679in}{2.043364in}}%
\pgfpathlineto{\pgfqpoint{1.941569in}{3.769821in}}%
\pgfusepath{stroke}%
\end{pgfscope}%
\begin{pgfscope}%
\pgfsetbuttcap%
\pgfsetroundjoin%
\pgfsetlinewidth{1.003750pt}%
\definecolor{currentstroke}{rgb}{1.000000,1.000000,1.000000}%
\pgfsetstrokecolor{currentstroke}%
\pgfsetdash{}{0pt}%
\pgfpathmoveto{\pgfqpoint{4.056159in}{1.038645in}}%
\pgfpathlineto{\pgfqpoint{2.706887in}{2.252119in}}%
\pgfpathlineto{\pgfqpoint{2.691692in}{3.959955in}}%
\pgfusepath{stroke}%
\end{pgfscope}%
\begin{pgfscope}%
\pgfsetbuttcap%
\pgfsetroundjoin%
\pgfsetlinewidth{1.003750pt}%
\definecolor{currentstroke}{rgb}{1.000000,1.000000,1.000000}%
\pgfsetstrokecolor{currentstroke}%
\pgfsetdash{}{0pt}%
\pgfpathmoveto{\pgfqpoint{4.792526in}{1.279143in}}%
\pgfpathlineto{\pgfqpoint{3.403358in}{2.454556in}}%
\pgfpathlineto{\pgfqpoint{3.418144in}{4.144090in}}%
\pgfusepath{stroke}%
\end{pgfscope}%
\begin{pgfscope}%
\pgfsetbuttcap%
\pgfsetroundjoin%
\pgfsetlinewidth{1.003750pt}%
\definecolor{currentstroke}{rgb}{1.000000,1.000000,1.000000}%
\pgfsetstrokecolor{currentstroke}%
\pgfsetdash{}{0pt}%
\pgfpathmoveto{\pgfqpoint{3.334272in}{0.802877in}}%
\pgfpathlineto{\pgfqpoint{2.025960in}{2.054201in}}%
\pgfpathlineto{\pgfqpoint{1.980533in}{3.779697in}}%
\pgfusepath{stroke}%
\end{pgfscope}%
\begin{pgfscope}%
\pgfsetbuttcap%
\pgfsetroundjoin%
\pgfsetlinewidth{1.003750pt}%
\definecolor{currentstroke}{rgb}{1.000000,1.000000,1.000000}%
\pgfsetstrokecolor{currentstroke}%
\pgfsetdash{}{0pt}%
\pgfpathmoveto{\pgfqpoint{4.094425in}{1.051143in}}%
\pgfpathlineto{\pgfqpoint{2.743032in}{2.262625in}}%
\pgfpathlineto{\pgfqpoint{2.729417in}{3.969517in}}%
\pgfusepath{stroke}%
\end{pgfscope}%
\begin{pgfscope}%
\pgfsetbuttcap%
\pgfsetroundjoin%
\pgfsetlinewidth{1.003750pt}%
\definecolor{currentstroke}{rgb}{1.000000,1.000000,1.000000}%
\pgfsetstrokecolor{currentstroke}%
\pgfsetdash{}{0pt}%
\pgfpathmoveto{\pgfqpoint{4.829541in}{1.291232in}}%
\pgfpathlineto{\pgfqpoint{3.438418in}{2.464746in}}%
\pgfpathlineto{\pgfqpoint{3.454688in}{4.153352in}}%
\pgfusepath{stroke}%
\end{pgfscope}%
\begin{pgfscope}%
\pgfpathrectangle{\pgfqpoint{0.887500in}{0.275000in}}{\pgfqpoint{4.225000in}{4.225000in}}%
\pgfusepath{clip}%
\pgfsetroundcap%
\pgfsetroundjoin%
\pgfsetlinewidth{1.003750pt}%
\definecolor{currentstroke}{rgb}{1.000000,1.000000,1.000000}%
\pgfsetstrokecolor{currentstroke}%
\pgfsetdash{}{0pt}%
\pgfpathmoveto{\pgfqpoint{3.057095in}{2.444595in}}%
\pgfusepath{stroke}%
\end{pgfscope}%
\begin{pgfscope}%
\pgfsetroundcap%
\pgfsetroundjoin%
\pgfsetlinewidth{1.254687pt}%
\definecolor{currentstroke}{rgb}{0.150000,0.150000,0.150000}%
\pgfsetstrokecolor{currentstroke}%
\pgfsetdash{}{0pt}%
\pgfpathmoveto{\pgfqpoint{2.573314in}{0.569246in}}%
\pgfpathlineto{\pgfqpoint{2.606512in}{0.535328in}}%
\pgfusepath{stroke}%
\end{pgfscope}%
\begin{pgfscope}%
\definecolor{textcolor}{rgb}{0.150000,0.150000,0.150000}%
\pgfsetstrokecolor{textcolor}%
\pgfsetfillcolor{textcolor}%
\pgftext[x=2.686647in,y=0.326577in,,top]{\color{textcolor}\sffamily\fontsize{11.000000}{13.200000}\selectfont \(\displaystyle 10^3\)}%
\end{pgfscope}%
\begin{pgfscope}%
\pgfpathrectangle{\pgfqpoint{0.887500in}{0.275000in}}{\pgfqpoint{4.225000in}{4.225000in}}%
\pgfusepath{clip}%
\pgfsetroundcap%
\pgfsetroundjoin%
\pgfsetlinewidth{1.003750pt}%
\definecolor{currentstroke}{rgb}{1.000000,1.000000,1.000000}%
\pgfsetstrokecolor{currentstroke}%
\pgfsetdash{}{0pt}%
\pgfpathmoveto{\pgfqpoint{3.057095in}{2.444595in}}%
\pgfusepath{stroke}%
\end{pgfscope}%
\begin{pgfscope}%
\pgfsetroundcap%
\pgfsetroundjoin%
\pgfsetlinewidth{1.254687pt}%
\definecolor{currentstroke}{rgb}{0.150000,0.150000,0.150000}%
\pgfsetstrokecolor{currentstroke}%
\pgfsetdash{}{0pt}%
\pgfpathmoveto{\pgfqpoint{3.358179in}{0.825330in}}%
\pgfpathlineto{\pgfqpoint{3.392550in}{0.792557in}}%
\pgfusepath{stroke}%
\end{pgfscope}%
\begin{pgfscope}%
\definecolor{textcolor}{rgb}{0.150000,0.150000,0.150000}%
\pgfsetstrokecolor{textcolor}%
\pgfsetfillcolor{textcolor}%
\pgftext[x=3.472790in,y=0.587852in,,top]{\color{textcolor}\sffamily\fontsize{11.000000}{13.200000}\selectfont \(\displaystyle 10^4\)}%
\end{pgfscope}%
\begin{pgfscope}%
\pgfpathrectangle{\pgfqpoint{0.887500in}{0.275000in}}{\pgfqpoint{4.225000in}{4.225000in}}%
\pgfusepath{clip}%
\pgfsetroundcap%
\pgfsetroundjoin%
\pgfsetlinewidth{1.003750pt}%
\definecolor{currentstroke}{rgb}{1.000000,1.000000,1.000000}%
\pgfsetstrokecolor{currentstroke}%
\pgfsetdash{}{0pt}%
\pgfpathmoveto{\pgfqpoint{3.057095in}{2.444595in}}%
\pgfusepath{stroke}%
\end{pgfscope}%
\begin{pgfscope}%
\pgfsetroundcap%
\pgfsetroundjoin%
\pgfsetlinewidth{1.254687pt}%
\definecolor{currentstroke}{rgb}{0.150000,0.150000,0.150000}%
\pgfsetstrokecolor{currentstroke}%
\pgfsetdash{}{0pt}%
\pgfpathmoveto{\pgfqpoint{4.116801in}{1.072850in}}%
\pgfpathlineto{\pgfqpoint{4.152246in}{1.041165in}}%
\pgfusepath{stroke}%
\end{pgfscope}%
\begin{pgfscope}%
\definecolor{textcolor}{rgb}{0.150000,0.150000,0.150000}%
\pgfsetstrokecolor{textcolor}%
\pgfsetfillcolor{textcolor}%
\pgftext[x=4.232540in,y=0.840355in,,top]{\color{textcolor}\sffamily\fontsize{11.000000}{13.200000}\selectfont \(\displaystyle 10^5\)}%
\end{pgfscope}%
\begin{pgfscope}%
\pgfpathrectangle{\pgfqpoint{0.887500in}{0.275000in}}{\pgfqpoint{4.225000in}{4.225000in}}%
\pgfusepath{clip}%
\pgfsetroundcap%
\pgfsetroundjoin%
\pgfsetlinewidth{1.003750pt}%
\definecolor{currentstroke}{rgb}{1.000000,1.000000,1.000000}%
\pgfsetstrokecolor{currentstroke}%
\pgfsetdash{}{0pt}%
\pgfpathmoveto{\pgfqpoint{3.057095in}{2.444595in}}%
\pgfusepath{stroke}%
\end{pgfscope}%
\begin{pgfscope}%
\pgfsetroundcap%
\pgfsetroundjoin%
\pgfsetlinewidth{1.254687pt}%
\definecolor{currentstroke}{rgb}{0.150000,0.150000,0.150000}%
\pgfsetstrokecolor{currentstroke}%
\pgfsetdash{}{0pt}%
\pgfpathmoveto{\pgfqpoint{4.850472in}{1.312231in}}%
\pgfpathlineto{\pgfqpoint{4.886904in}{1.281580in}}%
\pgfusepath{stroke}%
\end{pgfscope}%
\begin{pgfscope}%
\definecolor{textcolor}{rgb}{0.150000,0.150000,0.150000}%
\pgfsetstrokecolor{textcolor}%
\pgfsetfillcolor{textcolor}%
\pgftext[x=4.967205in,y=1.084521in,,top]{\color{textcolor}\sffamily\fontsize{11.000000}{13.200000}\selectfont \(\displaystyle 10^6\)}%
\end{pgfscope}%
\begin{pgfscope}%
\pgfsetroundcap%
\pgfsetroundjoin%
\pgfsetlinewidth{1.003750pt}%
\definecolor{currentstroke}{rgb}{0.000000,0.000000,0.000000}%
\pgfsetstrokecolor{currentstroke}%
\pgfsetstrokeopacity{0.300000}%
\pgfsetdash{}{0pt}%
\pgfpathmoveto{\pgfqpoint{2.812423in}{0.647262in}}%
\pgfpathlineto{\pgfqpoint{2.845985in}{0.613695in}}%
\pgfusepath{stroke}%
\end{pgfscope}%
\begin{pgfscope}%
\pgfsetroundcap%
\pgfsetroundjoin%
\pgfsetlinewidth{1.003750pt}%
\definecolor{currentstroke}{rgb}{0.000000,0.000000,0.000000}%
\pgfsetstrokecolor{currentstroke}%
\pgfsetstrokeopacity{0.300000}%
\pgfsetdash{}{0pt}%
\pgfpathmoveto{\pgfqpoint{3.589247in}{0.900722in}}%
\pgfpathlineto{\pgfqpoint{3.623951in}{0.868282in}}%
\pgfusepath{stroke}%
\end{pgfscope}%
\begin{pgfscope}%
\pgfsetroundcap%
\pgfsetroundjoin%
\pgfsetlinewidth{1.003750pt}%
\definecolor{currentstroke}{rgb}{0.000000,0.000000,0.000000}%
\pgfsetstrokecolor{currentstroke}%
\pgfsetstrokeopacity{0.300000}%
\pgfsetdash{}{0pt}%
\pgfpathmoveto{\pgfqpoint{4.340226in}{1.145749in}}%
\pgfpathlineto{\pgfqpoint{4.375977in}{1.114381in}}%
\pgfusepath{stroke}%
\end{pgfscope}%
\begin{pgfscope}%
\pgfsetroundcap%
\pgfsetroundjoin%
\pgfsetlinewidth{1.003750pt}%
\definecolor{currentstroke}{rgb}{0.000000,0.000000,0.000000}%
\pgfsetstrokecolor{currentstroke}%
\pgfsetstrokeopacity{0.300000}%
\pgfsetdash{}{0pt}%
\pgfpathmoveto{\pgfqpoint{2.951149in}{0.692525in}}%
\pgfpathlineto{\pgfqpoint{2.984919in}{0.659161in}}%
\pgfusepath{stroke}%
\end{pgfscope}%
\begin{pgfscope}%
\pgfsetroundcap%
\pgfsetroundjoin%
\pgfsetlinewidth{1.003750pt}%
\definecolor{currentstroke}{rgb}{0.000000,0.000000,0.000000}%
\pgfsetstrokecolor{currentstroke}%
\pgfsetstrokeopacity{0.300000}%
\pgfsetdash{}{0pt}%
\pgfpathmoveto{\pgfqpoint{3.723326in}{0.944469in}}%
\pgfpathlineto{\pgfqpoint{3.758221in}{0.912222in}}%
\pgfusepath{stroke}%
\end{pgfscope}%
\begin{pgfscope}%
\pgfsetroundcap%
\pgfsetroundjoin%
\pgfsetlinewidth{1.003750pt}%
\definecolor{currentstroke}{rgb}{0.000000,0.000000,0.000000}%
\pgfsetstrokecolor{currentstroke}%
\pgfsetstrokeopacity{0.300000}%
\pgfsetdash{}{0pt}%
\pgfpathmoveto{\pgfqpoint{4.469887in}{1.188054in}}%
\pgfpathlineto{\pgfqpoint{4.505814in}{1.156869in}}%
\pgfusepath{stroke}%
\end{pgfscope}%
\begin{pgfscope}%
\pgfsetroundcap%
\pgfsetroundjoin%
\pgfsetlinewidth{1.003750pt}%
\definecolor{currentstroke}{rgb}{0.000000,0.000000,0.000000}%
\pgfsetstrokecolor{currentstroke}%
\pgfsetstrokeopacity{0.300000}%
\pgfsetdash{}{0pt}%
\pgfpathmoveto{\pgfqpoint{3.049069in}{0.724474in}}%
\pgfpathlineto{\pgfqpoint{3.082985in}{0.691252in}}%
\pgfusepath{stroke}%
\end{pgfscope}%
\begin{pgfscope}%
\pgfsetroundcap%
\pgfsetroundjoin%
\pgfsetlinewidth{1.003750pt}%
\definecolor{currentstroke}{rgb}{0.000000,0.000000,0.000000}%
\pgfsetstrokecolor{currentstroke}%
\pgfsetstrokeopacity{0.300000}%
\pgfsetdash{}{0pt}%
\pgfpathmoveto{\pgfqpoint{3.817974in}{0.975350in}}%
\pgfpathlineto{\pgfqpoint{3.853003in}{0.943239in}}%
\pgfusepath{stroke}%
\end{pgfscope}%
\begin{pgfscope}%
\pgfsetroundcap%
\pgfsetroundjoin%
\pgfsetlinewidth{1.003750pt}%
\definecolor{currentstroke}{rgb}{0.000000,0.000000,0.000000}%
\pgfsetstrokecolor{currentstroke}%
\pgfsetstrokeopacity{0.300000}%
\pgfsetdash{}{0pt}%
\pgfpathmoveto{\pgfqpoint{4.561424in}{1.217921in}}%
\pgfpathlineto{\pgfqpoint{4.597474in}{1.186865in}}%
\pgfusepath{stroke}%
\end{pgfscope}%
\begin{pgfscope}%
\pgfsetroundcap%
\pgfsetroundjoin%
\pgfsetlinewidth{1.003750pt}%
\definecolor{currentstroke}{rgb}{0.000000,0.000000,0.000000}%
\pgfsetstrokecolor{currentstroke}%
\pgfsetstrokeopacity{0.300000}%
\pgfsetdash{}{0pt}%
\pgfpathmoveto{\pgfqpoint{3.124734in}{0.749162in}}%
\pgfpathlineto{\pgfqpoint{3.158762in}{0.716050in}}%
\pgfusepath{stroke}%
\end{pgfscope}%
\begin{pgfscope}%
\pgfsetroundcap%
\pgfsetroundjoin%
\pgfsetlinewidth{1.003750pt}%
\definecolor{currentstroke}{rgb}{0.000000,0.000000,0.000000}%
\pgfsetstrokecolor{currentstroke}%
\pgfsetstrokeopacity{0.300000}%
\pgfsetdash{}{0pt}%
\pgfpathmoveto{\pgfqpoint{3.891115in}{0.999214in}}%
\pgfpathlineto{\pgfqpoint{3.926247in}{0.967208in}}%
\pgfusepath{stroke}%
\end{pgfscope}%
\begin{pgfscope}%
\pgfsetroundcap%
\pgfsetroundjoin%
\pgfsetlinewidth{1.003750pt}%
\definecolor{currentstroke}{rgb}{0.000000,0.000000,0.000000}%
\pgfsetstrokecolor{currentstroke}%
\pgfsetstrokeopacity{0.300000}%
\pgfsetdash{}{0pt}%
\pgfpathmoveto{\pgfqpoint{4.632166in}{1.241003in}}%
\pgfpathlineto{\pgfqpoint{4.668310in}{1.210046in}}%
\pgfusepath{stroke}%
\end{pgfscope}%
\begin{pgfscope}%
\pgfsetroundcap%
\pgfsetroundjoin%
\pgfsetlinewidth{1.003750pt}%
\definecolor{currentstroke}{rgb}{0.000000,0.000000,0.000000}%
\pgfsetstrokecolor{currentstroke}%
\pgfsetstrokeopacity{0.300000}%
\pgfsetdash{}{0pt}%
\pgfpathmoveto{\pgfqpoint{3.186370in}{0.769272in}}%
\pgfpathlineto{\pgfqpoint{3.220489in}{0.736250in}}%
\pgfusepath{stroke}%
\end{pgfscope}%
\begin{pgfscope}%
\pgfsetroundcap%
\pgfsetroundjoin%
\pgfsetlinewidth{1.003750pt}%
\definecolor{currentstroke}{rgb}{0.000000,0.000000,0.000000}%
\pgfsetstrokecolor{currentstroke}%
\pgfsetstrokeopacity{0.300000}%
\pgfsetdash{}{0pt}%
\pgfpathmoveto{\pgfqpoint{3.950699in}{1.018655in}}%
\pgfpathlineto{\pgfqpoint{3.985914in}{0.986734in}}%
\pgfusepath{stroke}%
\end{pgfscope}%
\begin{pgfscope}%
\pgfsetroundcap%
\pgfsetroundjoin%
\pgfsetlinewidth{1.003750pt}%
\definecolor{currentstroke}{rgb}{0.000000,0.000000,0.000000}%
\pgfsetstrokecolor{currentstroke}%
\pgfsetstrokeopacity{0.300000}%
\pgfsetdash{}{0pt}%
\pgfpathmoveto{\pgfqpoint{4.689798in}{1.259807in}}%
\pgfpathlineto{\pgfqpoint{4.726019in}{1.228931in}}%
\pgfusepath{stroke}%
\end{pgfscope}%
\begin{pgfscope}%
\pgfsetroundcap%
\pgfsetroundjoin%
\pgfsetlinewidth{1.003750pt}%
\definecolor{currentstroke}{rgb}{0.000000,0.000000,0.000000}%
\pgfsetstrokecolor{currentstroke}%
\pgfsetstrokeopacity{0.300000}%
\pgfsetdash{}{0pt}%
\pgfpathmoveto{\pgfqpoint{3.238354in}{0.786233in}}%
\pgfpathlineto{\pgfqpoint{3.272549in}{0.753287in}}%
\pgfusepath{stroke}%
\end{pgfscope}%
\begin{pgfscope}%
\pgfsetroundcap%
\pgfsetroundjoin%
\pgfsetlinewidth{1.003750pt}%
\definecolor{currentstroke}{rgb}{0.000000,0.000000,0.000000}%
\pgfsetstrokecolor{currentstroke}%
\pgfsetstrokeopacity{0.300000}%
\pgfsetdash{}{0pt}%
\pgfpathmoveto{\pgfqpoint{4.000953in}{1.035052in}}%
\pgfpathlineto{\pgfqpoint{4.036238in}{1.003202in}}%
\pgfusepath{stroke}%
\end{pgfscope}%
\begin{pgfscope}%
\pgfsetroundcap%
\pgfsetroundjoin%
\pgfsetlinewidth{1.003750pt}%
\definecolor{currentstroke}{rgb}{0.000000,0.000000,0.000000}%
\pgfsetstrokecolor{currentstroke}%
\pgfsetstrokeopacity{0.300000}%
\pgfsetdash{}{0pt}%
\pgfpathmoveto{\pgfqpoint{4.738408in}{1.275667in}}%
\pgfpathlineto{\pgfqpoint{4.774693in}{1.244859in}}%
\pgfusepath{stroke}%
\end{pgfscope}%
\begin{pgfscope}%
\pgfsetroundcap%
\pgfsetroundjoin%
\pgfsetlinewidth{1.003750pt}%
\definecolor{currentstroke}{rgb}{0.000000,0.000000,0.000000}%
\pgfsetstrokecolor{currentstroke}%
\pgfsetstrokeopacity{0.300000}%
\pgfsetdash{}{0pt}%
\pgfpathmoveto{\pgfqpoint{3.283288in}{0.800894in}}%
\pgfpathlineto{\pgfqpoint{3.317549in}{0.768013in}}%
\pgfusepath{stroke}%
\end{pgfscope}%
\begin{pgfscope}%
\pgfsetroundcap%
\pgfsetroundjoin%
\pgfsetlinewidth{1.003750pt}%
\definecolor{currentstroke}{rgb}{0.000000,0.000000,0.000000}%
\pgfsetstrokecolor{currentstroke}%
\pgfsetstrokeopacity{0.300000}%
\pgfsetdash{}{0pt}%
\pgfpathmoveto{\pgfqpoint{4.044394in}{1.049226in}}%
\pgfpathlineto{\pgfqpoint{4.079740in}{1.017438in}}%
\pgfusepath{stroke}%
\end{pgfscope}%
\begin{pgfscope}%
\pgfsetroundcap%
\pgfsetroundjoin%
\pgfsetlinewidth{1.003750pt}%
\definecolor{currentstroke}{rgb}{0.000000,0.000000,0.000000}%
\pgfsetstrokecolor{currentstroke}%
\pgfsetstrokeopacity{0.300000}%
\pgfsetdash{}{0pt}%
\pgfpathmoveto{\pgfqpoint{4.780429in}{1.289378in}}%
\pgfpathlineto{\pgfqpoint{4.816770in}{1.258629in}}%
\pgfusepath{stroke}%
\end{pgfscope}%
\begin{pgfscope}%
\pgfsetroundcap%
\pgfsetroundjoin%
\pgfsetlinewidth{1.003750pt}%
\definecolor{currentstroke}{rgb}{0.000000,0.000000,0.000000}%
\pgfsetstrokecolor{currentstroke}%
\pgfsetstrokeopacity{0.300000}%
\pgfsetdash{}{0pt}%
\pgfpathmoveto{\pgfqpoint{3.322849in}{0.813802in}}%
\pgfpathlineto{\pgfqpoint{3.357168in}{0.780978in}}%
\pgfusepath{stroke}%
\end{pgfscope}%
\begin{pgfscope}%
\pgfsetroundcap%
\pgfsetroundjoin%
\pgfsetlinewidth{1.003750pt}%
\definecolor{currentstroke}{rgb}{0.000000,0.000000,0.000000}%
\pgfsetstrokecolor{currentstroke}%
\pgfsetstrokeopacity{0.300000}%
\pgfsetdash{}{0pt}%
\pgfpathmoveto{\pgfqpoint{4.082642in}{1.061705in}}%
\pgfpathlineto{\pgfqpoint{4.118041in}{1.029972in}}%
\pgfusepath{stroke}%
\end{pgfscope}%
\begin{pgfscope}%
\pgfsetroundcap%
\pgfsetroundjoin%
\pgfsetlinewidth{1.003750pt}%
\definecolor{currentstroke}{rgb}{0.000000,0.000000,0.000000}%
\pgfsetstrokecolor{currentstroke}%
\pgfsetstrokeopacity{0.300000}%
\pgfsetdash{}{0pt}%
\pgfpathmoveto{\pgfqpoint{4.817428in}{1.301449in}}%
\pgfpathlineto{\pgfqpoint{4.853817in}{1.270753in}}%
\pgfusepath{stroke}%
\end{pgfscope}%
\begin{pgfscope}%
\pgfsetroundcap%
\pgfsetroundjoin%
\pgfsetlinewidth{1.254687pt}%
\definecolor{currentstroke}{rgb}{1.000000,1.000000,1.000000}%
\pgfsetstrokecolor{currentstroke}%
\pgfsetdash{}{0pt}%
\pgfpathmoveto{\pgfqpoint{1.273588in}{1.835516in}}%
\pgfpathlineto{\pgfqpoint{2.534381in}{0.541633in}}%
\pgfusepath{stroke}%
\end{pgfscope}%
\begin{pgfscope}%
\definecolor{textcolor}{rgb}{0.150000,0.150000,0.150000}%
\pgfsetstrokecolor{textcolor}%
\pgfsetfillcolor{textcolor}%
\pgftext[x=0.945471in, y=1.339199in, left, base,rotate=314.257888]{\color{textcolor}\sffamily\fontsize{12.000000}{14.400000}\selectfont Angle of Attack [deg.]}%
\end{pgfscope}%
\begin{pgfscope}%
\pgfsetbuttcap%
\pgfsetroundjoin%
\pgfsetlinewidth{1.003750pt}%
\definecolor{currentstroke}{rgb}{1.000000,1.000000,1.000000}%
\pgfsetstrokecolor{currentstroke}%
\pgfsetdash{}{0pt}%
\pgfpathmoveto{\pgfqpoint{4.961933in}{3.129286in}}%
\pgfpathlineto{\pgfqpoint{4.877339in}{1.342176in}}%
\pgfpathlineto{\pgfqpoint{2.506852in}{0.569885in}}%
\pgfusepath{stroke}%
\end{pgfscope}%
\begin{pgfscope}%
\pgfsetbuttcap%
\pgfsetroundjoin%
\pgfsetlinewidth{1.003750pt}%
\definecolor{currentstroke}{rgb}{1.000000,1.000000,1.000000}%
\pgfsetstrokecolor{currentstroke}%
\pgfsetdash{}{0pt}%
\pgfpathmoveto{\pgfqpoint{4.661589in}{3.348464in}}%
\pgfpathlineto{\pgfqpoint{4.591528in}{1.581748in}}%
\pgfpathlineto{\pgfqpoint{2.247722in}{0.835815in}}%
\pgfusepath{stroke}%
\end{pgfscope}%
\begin{pgfscope}%
\pgfsetbuttcap%
\pgfsetroundjoin%
\pgfsetlinewidth{1.003750pt}%
\definecolor{currentstroke}{rgb}{1.000000,1.000000,1.000000}%
\pgfsetstrokecolor{currentstroke}%
\pgfsetdash{}{0pt}%
\pgfpathmoveto{\pgfqpoint{4.371595in}{3.560088in}}%
\pgfpathlineto{\pgfqpoint{4.315143in}{1.813419in}}%
\pgfpathlineto{\pgfqpoint{1.997586in}{1.092517in}}%
\pgfusepath{stroke}%
\end{pgfscope}%
\begin{pgfscope}%
\pgfsetbuttcap%
\pgfsetroundjoin%
\pgfsetlinewidth{1.003750pt}%
\definecolor{currentstroke}{rgb}{1.000000,1.000000,1.000000}%
\pgfsetstrokecolor{currentstroke}%
\pgfsetdash{}{0pt}%
\pgfpathmoveto{\pgfqpoint{4.091426in}{3.764543in}}%
\pgfpathlineto{\pgfqpoint{4.047727in}{2.037573in}}%
\pgfpathlineto{\pgfqpoint{1.755981in}{1.340462in}}%
\pgfusepath{stroke}%
\end{pgfscope}%
\begin{pgfscope}%
\pgfsetbuttcap%
\pgfsetroundjoin%
\pgfsetlinewidth{1.003750pt}%
\definecolor{currentstroke}{rgb}{1.000000,1.000000,1.000000}%
\pgfsetstrokecolor{currentstroke}%
\pgfsetdash{}{0pt}%
\pgfpathmoveto{\pgfqpoint{3.820590in}{3.962186in}}%
\pgfpathlineto{\pgfqpoint{3.788849in}{2.254570in}}%
\pgfpathlineto{\pgfqpoint{1.522480in}{1.580092in}}%
\pgfusepath{stroke}%
\end{pgfscope}%
\begin{pgfscope}%
\pgfsetbuttcap%
\pgfsetroundjoin%
\pgfsetlinewidth{1.003750pt}%
\definecolor{currentstroke}{rgb}{1.000000,1.000000,1.000000}%
\pgfsetstrokecolor{currentstroke}%
\pgfsetdash{}{0pt}%
\pgfpathmoveto{\pgfqpoint{3.558629in}{4.153353in}}%
\pgfpathlineto{\pgfqpoint{3.538106in}{2.464747in}}%
\pgfpathlineto{\pgfqpoint{1.296682in}{1.811817in}}%
\pgfusepath{stroke}%
\end{pgfscope}%
\begin{pgfscope}%
\pgfsetbuttcap%
\pgfsetroundjoin%
\pgfsetlinewidth{1.003750pt}%
\definecolor{currentstroke}{rgb}{1.000000,1.000000,1.000000}%
\pgfsetstrokecolor{currentstroke}%
\pgfsetdash{}{0pt}%
\pgfpathmoveto{\pgfqpoint{4.810433in}{3.239844in}}%
\pgfpathlineto{\pgfqpoint{4.733225in}{1.462975in}}%
\pgfpathlineto{\pgfqpoint{2.376133in}{0.704034in}}%
\pgfusepath{stroke}%
\end{pgfscope}%
\begin{pgfscope}%
\pgfsetbuttcap%
\pgfsetroundjoin%
\pgfsetlinewidth{1.003750pt}%
\definecolor{currentstroke}{rgb}{1.000000,1.000000,1.000000}%
\pgfsetstrokecolor{currentstroke}%
\pgfsetdash{}{0pt}%
\pgfpathmoveto{\pgfqpoint{4.515332in}{3.455195in}}%
\pgfpathlineto{\pgfqpoint{4.452187in}{1.698547in}}%
\pgfpathlineto{\pgfqpoint{2.121559in}{0.965289in}}%
\pgfusepath{stroke}%
\end{pgfscope}%
\begin{pgfscope}%
\pgfsetbuttcap%
\pgfsetroundjoin%
\pgfsetlinewidth{1.003750pt}%
\definecolor{currentstroke}{rgb}{1.000000,1.000000,1.000000}%
\pgfsetstrokecolor{currentstroke}%
\pgfsetdash{}{0pt}%
\pgfpathmoveto{\pgfqpoint{4.230314in}{3.663188in}}%
\pgfpathlineto{\pgfqpoint{4.180342in}{1.926413in}}%
\pgfpathlineto{\pgfqpoint{1.875745in}{1.217556in}}%
\pgfusepath{stroke}%
\end{pgfscope}%
\begin{pgfscope}%
\pgfsetbuttcap%
\pgfsetroundjoin%
\pgfsetlinewidth{1.003750pt}%
\definecolor{currentstroke}{rgb}{1.000000,1.000000,1.000000}%
\pgfsetstrokecolor{currentstroke}%
\pgfsetdash{}{0pt}%
\pgfpathmoveto{\pgfqpoint{3.954871in}{3.864194in}}%
\pgfpathlineto{\pgfqpoint{3.917246in}{2.146945in}}%
\pgfpathlineto{\pgfqpoint{1.638244in}{1.461290in}}%
\pgfusepath{stroke}%
\end{pgfscope}%
\begin{pgfscope}%
\pgfsetbuttcap%
\pgfsetroundjoin%
\pgfsetlinewidth{1.003750pt}%
\definecolor{currentstroke}{rgb}{1.000000,1.000000,1.000000}%
\pgfsetstrokecolor{currentstroke}%
\pgfsetdash{}{0pt}%
\pgfpathmoveto{\pgfqpoint{3.688528in}{4.058559in}}%
\pgfpathlineto{\pgfqpoint{3.662485in}{2.360491in}}%
\pgfpathlineto{\pgfqpoint{1.408642in}{1.696918in}}%
\pgfusepath{stroke}%
\end{pgfscope}%
\begin{pgfscope}%
\pgfpathrectangle{\pgfqpoint{0.887500in}{0.275000in}}{\pgfqpoint{4.225000in}{4.225000in}}%
\pgfusepath{clip}%
\pgfsetroundcap%
\pgfsetroundjoin%
\pgfsetlinewidth{1.003750pt}%
\definecolor{currentstroke}{rgb}{1.000000,1.000000,1.000000}%
\pgfsetstrokecolor{currentstroke}%
\pgfsetdash{}{0pt}%
\pgfpathmoveto{\pgfqpoint{3.057095in}{2.444595in}}%
\pgfusepath{stroke}%
\end{pgfscope}%
\begin{pgfscope}%
\pgfsetroundcap%
\pgfsetroundjoin%
\pgfsetlinewidth{1.254687pt}%
\definecolor{currentstroke}{rgb}{0.150000,0.150000,0.150000}%
\pgfsetstrokecolor{currentstroke}%
\pgfsetdash{}{0pt}%
\pgfpathmoveto{\pgfqpoint{2.526832in}{0.576394in}}%
\pgfpathlineto{\pgfqpoint{2.466839in}{0.556849in}}%
\pgfusepath{stroke}%
\end{pgfscope}%
\begin{pgfscope}%
\definecolor{textcolor}{rgb}{0.150000,0.150000,0.150000}%
\pgfsetstrokecolor{textcolor}%
\pgfsetfillcolor{textcolor}%
\pgftext[x=2.327268in,y=0.382670in,right,bottom]{\color{textcolor}\sffamily\fontsize{11.000000}{13.200000}\selectfont \ensuremath{-}4}%
\end{pgfscope}%
\begin{pgfscope}%
\pgfpathrectangle{\pgfqpoint{0.887500in}{0.275000in}}{\pgfqpoint{4.225000in}{4.225000in}}%
\pgfusepath{clip}%
\pgfsetroundcap%
\pgfsetroundjoin%
\pgfsetlinewidth{1.003750pt}%
\definecolor{currentstroke}{rgb}{1.000000,1.000000,1.000000}%
\pgfsetstrokecolor{currentstroke}%
\pgfsetdash{}{0pt}%
\pgfpathmoveto{\pgfqpoint{3.057095in}{2.444595in}}%
\pgfusepath{stroke}%
\end{pgfscope}%
\begin{pgfscope}%
\pgfsetroundcap%
\pgfsetroundjoin%
\pgfsetlinewidth{1.254687pt}%
\definecolor{currentstroke}{rgb}{0.150000,0.150000,0.150000}%
\pgfsetstrokecolor{currentstroke}%
\pgfsetdash{}{0pt}%
\pgfpathmoveto{\pgfqpoint{2.267460in}{0.842097in}}%
\pgfpathlineto{\pgfqpoint{2.208197in}{0.823236in}}%
\pgfusepath{stroke}%
\end{pgfscope}%
\begin{pgfscope}%
\definecolor{textcolor}{rgb}{0.150000,0.150000,0.150000}%
\pgfsetstrokecolor{textcolor}%
\pgfsetfillcolor{textcolor}%
\pgftext[x=2.071177in,y=0.652045in,right,bottom]{\color{textcolor}\sffamily\fontsize{11.000000}{13.200000}\selectfont 0}%
\end{pgfscope}%
\begin{pgfscope}%
\pgfpathrectangle{\pgfqpoint{0.887500in}{0.275000in}}{\pgfqpoint{4.225000in}{4.225000in}}%
\pgfusepath{clip}%
\pgfsetroundcap%
\pgfsetroundjoin%
\pgfsetlinewidth{1.003750pt}%
\definecolor{currentstroke}{rgb}{1.000000,1.000000,1.000000}%
\pgfsetstrokecolor{currentstroke}%
\pgfsetdash{}{0pt}%
\pgfpathmoveto{\pgfqpoint{3.057095in}{2.444595in}}%
\pgfusepath{stroke}%
\end{pgfscope}%
\begin{pgfscope}%
\pgfsetroundcap%
\pgfsetroundjoin%
\pgfsetlinewidth{1.254687pt}%
\definecolor{currentstroke}{rgb}{0.150000,0.150000,0.150000}%
\pgfsetstrokecolor{currentstroke}%
\pgfsetdash{}{0pt}%
\pgfpathmoveto{\pgfqpoint{2.017085in}{1.098582in}}%
\pgfpathlineto{\pgfqpoint{1.958537in}{1.080370in}}%
\pgfusepath{stroke}%
\end{pgfscope}%
\begin{pgfscope}%
\definecolor{textcolor}{rgb}{0.150000,0.150000,0.150000}%
\pgfsetstrokecolor{textcolor}%
\pgfsetfillcolor{textcolor}%
\pgftext[x=1.823978in,y=0.912067in,right,bottom]{\color{textcolor}\sffamily\fontsize{11.000000}{13.200000}\selectfont 4}%
\end{pgfscope}%
\begin{pgfscope}%
\pgfpathrectangle{\pgfqpoint{0.887500in}{0.275000in}}{\pgfqpoint{4.225000in}{4.225000in}}%
\pgfusepath{clip}%
\pgfsetroundcap%
\pgfsetroundjoin%
\pgfsetlinewidth{1.003750pt}%
\definecolor{currentstroke}{rgb}{1.000000,1.000000,1.000000}%
\pgfsetstrokecolor{currentstroke}%
\pgfsetdash{}{0pt}%
\pgfpathmoveto{\pgfqpoint{3.057095in}{2.444595in}}%
\pgfusepath{stroke}%
\end{pgfscope}%
\begin{pgfscope}%
\pgfsetroundcap%
\pgfsetroundjoin%
\pgfsetlinewidth{1.254687pt}%
\definecolor{currentstroke}{rgb}{0.150000,0.150000,0.150000}%
\pgfsetstrokecolor{currentstroke}%
\pgfsetdash{}{0pt}%
\pgfpathmoveto{\pgfqpoint{1.775248in}{1.346323in}}%
\pgfpathlineto{\pgfqpoint{1.717401in}{1.328727in}}%
\pgfusepath{stroke}%
\end{pgfscope}%
\begin{pgfscope}%
\definecolor{textcolor}{rgb}{0.150000,0.150000,0.150000}%
\pgfsetstrokecolor{textcolor}%
\pgfsetfillcolor{textcolor}%
\pgftext[x=1.585216in,y=1.163216in,right,bottom]{\color{textcolor}\sffamily\fontsize{11.000000}{13.200000}\selectfont 8}%
\end{pgfscope}%
\begin{pgfscope}%
\pgfpathrectangle{\pgfqpoint{0.887500in}{0.275000in}}{\pgfqpoint{4.225000in}{4.225000in}}%
\pgfusepath{clip}%
\pgfsetroundcap%
\pgfsetroundjoin%
\pgfsetlinewidth{1.003750pt}%
\definecolor{currentstroke}{rgb}{1.000000,1.000000,1.000000}%
\pgfsetstrokecolor{currentstroke}%
\pgfsetdash{}{0pt}%
\pgfpathmoveto{\pgfqpoint{3.057095in}{2.444595in}}%
\pgfusepath{stroke}%
\end{pgfscope}%
\begin{pgfscope}%
\pgfsetroundcap%
\pgfsetroundjoin%
\pgfsetlinewidth{1.254687pt}%
\definecolor{currentstroke}{rgb}{0.150000,0.150000,0.150000}%
\pgfsetstrokecolor{currentstroke}%
\pgfsetdash{}{0pt}%
\pgfpathmoveto{\pgfqpoint{1.541518in}{1.585757in}}%
\pgfpathlineto{\pgfqpoint{1.484359in}{1.568747in}}%
\pgfusepath{stroke}%
\end{pgfscope}%
\begin{pgfscope}%
\definecolor{textcolor}{rgb}{0.150000,0.150000,0.150000}%
\pgfsetstrokecolor{textcolor}%
\pgfsetfillcolor{textcolor}%
\pgftext[x=1.354465in,y=1.405937in,right,bottom]{\color{textcolor}\sffamily\fontsize{11.000000}{13.200000}\selectfont 12}%
\end{pgfscope}%
\begin{pgfscope}%
\pgfpathrectangle{\pgfqpoint{0.887500in}{0.275000in}}{\pgfqpoint{4.225000in}{4.225000in}}%
\pgfusepath{clip}%
\pgfsetroundcap%
\pgfsetroundjoin%
\pgfsetlinewidth{1.003750pt}%
\definecolor{currentstroke}{rgb}{1.000000,1.000000,1.000000}%
\pgfsetstrokecolor{currentstroke}%
\pgfsetdash{}{0pt}%
\pgfpathmoveto{\pgfqpoint{3.057095in}{2.444595in}}%
\pgfusepath{stroke}%
\end{pgfscope}%
\begin{pgfscope}%
\pgfsetroundcap%
\pgfsetroundjoin%
\pgfsetlinewidth{1.254687pt}%
\definecolor{currentstroke}{rgb}{0.150000,0.150000,0.150000}%
\pgfsetstrokecolor{currentstroke}%
\pgfsetdash{}{0pt}%
\pgfpathmoveto{\pgfqpoint{1.315495in}{1.817297in}}%
\pgfpathlineto{\pgfqpoint{1.259010in}{1.800843in}}%
\pgfusepath{stroke}%
\end{pgfscope}%
\begin{pgfscope}%
\definecolor{textcolor}{rgb}{0.150000,0.150000,0.150000}%
\pgfsetstrokecolor{textcolor}%
\pgfsetfillcolor{textcolor}%
\pgftext[x=1.131330in,y=1.640647in,right,bottom]{\color{textcolor}\sffamily\fontsize{11.000000}{13.200000}\selectfont 16}%
\end{pgfscope}%
\begin{pgfscope}%
\pgfsetroundcap%
\pgfsetroundjoin%
\pgfsetlinewidth{1.003750pt}%
\definecolor{currentstroke}{rgb}{0.000000,0.000000,0.000000}%
\pgfsetstrokecolor{currentstroke}%
\pgfsetstrokeopacity{0.300000}%
\pgfsetdash{}{0pt}%
\pgfpathmoveto{\pgfqpoint{2.395991in}{0.710428in}}%
\pgfpathlineto{\pgfqpoint{2.336365in}{0.691230in}}%
\pgfusepath{stroke}%
\end{pgfscope}%
\begin{pgfscope}%
\pgfsetroundcap%
\pgfsetroundjoin%
\pgfsetlinewidth{1.003750pt}%
\definecolor{currentstroke}{rgb}{0.000000,0.000000,0.000000}%
\pgfsetstrokecolor{currentstroke}%
\pgfsetstrokeopacity{0.300000}%
\pgfsetdash{}{0pt}%
\pgfpathmoveto{\pgfqpoint{2.141178in}{0.971461in}}%
\pgfpathlineto{\pgfqpoint{2.082274in}{0.952929in}}%
\pgfusepath{stroke}%
\end{pgfscope}%
\begin{pgfscope}%
\pgfsetroundcap%
\pgfsetroundjoin%
\pgfsetlinewidth{1.003750pt}%
\definecolor{currentstroke}{rgb}{0.000000,0.000000,0.000000}%
\pgfsetstrokecolor{currentstroke}%
\pgfsetstrokeopacity{0.300000}%
\pgfsetdash{}{0pt}%
\pgfpathmoveto{\pgfqpoint{1.895127in}{1.223517in}}%
\pgfpathlineto{\pgfqpoint{1.836931in}{1.205617in}}%
\pgfusepath{stroke}%
\end{pgfscope}%
\begin{pgfscope}%
\pgfsetroundcap%
\pgfsetroundjoin%
\pgfsetlinewidth{1.003750pt}%
\definecolor{currentstroke}{rgb}{0.000000,0.000000,0.000000}%
\pgfsetstrokecolor{currentstroke}%
\pgfsetstrokeopacity{0.300000}%
\pgfsetdash{}{0pt}%
\pgfpathmoveto{\pgfqpoint{1.657395in}{1.467052in}}%
\pgfpathlineto{\pgfqpoint{1.599894in}{1.449752in}}%
\pgfusepath{stroke}%
\end{pgfscope}%
\begin{pgfscope}%
\pgfsetroundcap%
\pgfsetroundjoin%
\pgfsetlinewidth{1.003750pt}%
\definecolor{currentstroke}{rgb}{0.000000,0.000000,0.000000}%
\pgfsetstrokecolor{currentstroke}%
\pgfsetstrokeopacity{0.300000}%
\pgfsetdash{}{0pt}%
\pgfpathmoveto{\pgfqpoint{1.427567in}{1.702489in}}%
\pgfpathlineto{\pgfqpoint{1.370747in}{1.685761in}}%
\pgfusepath{stroke}%
\end{pgfscope}%
\begin{pgfscope}%
\pgfsetroundcap%
\pgfsetroundjoin%
\pgfsetlinewidth{1.254687pt}%
\definecolor{currentstroke}{rgb}{1.000000,1.000000,1.000000}%
\pgfsetstrokecolor{currentstroke}%
\pgfsetdash{}{0pt}%
\pgfpathmoveto{\pgfqpoint{1.273588in}{1.835516in}}%
\pgfpathlineto{\pgfqpoint{1.193692in}{3.580256in}}%
\pgfusepath{stroke}%
\end{pgfscope}%
\begin{pgfscope}%
\definecolor{textcolor}{rgb}{0.150000,0.150000,0.150000}%
\pgfsetstrokecolor{textcolor}%
\pgfsetfillcolor{textcolor}%
\pgftext[x=0.609474in, y=3.246352in, left, base,rotate=272.621908]{\color{textcolor}\sffamily\fontsize{12.000000}{14.400000}\selectfont Lift Coeff. \(\displaystyle C_L\)}%
\end{pgfscope}%
\begin{pgfscope}%
\pgfsetbuttcap%
\pgfsetroundjoin%
\pgfsetlinewidth{1.003750pt}%
\definecolor{currentstroke}{rgb}{1.000000,1.000000,1.000000}%
\pgfsetstrokecolor{currentstroke}%
\pgfsetdash{}{0pt}%
\pgfpathmoveto{\pgfqpoint{1.266388in}{1.992764in}}%
\pgfpathlineto{\pgfqpoint{3.514191in}{2.638566in}}%
\pgfpathlineto{\pgfqpoint{4.915426in}{1.477763in}}%
\pgfusepath{stroke}%
\end{pgfscope}%
\begin{pgfscope}%
\pgfsetbuttcap%
\pgfsetroundjoin%
\pgfsetlinewidth{1.003750pt}%
\definecolor{currentstroke}{rgb}{1.000000,1.000000,1.000000}%
\pgfsetstrokecolor{currentstroke}%
\pgfsetdash{}{0pt}%
\pgfpathmoveto{\pgfqpoint{1.258846in}{2.157445in}}%
\pgfpathlineto{\pgfqpoint{3.516024in}{2.798008in}}%
\pgfpathlineto{\pgfqpoint{4.923549in}{1.646439in}}%
\pgfusepath{stroke}%
\end{pgfscope}%
\begin{pgfscope}%
\pgfsetbuttcap%
\pgfsetroundjoin%
\pgfsetlinewidth{1.003750pt}%
\definecolor{currentstroke}{rgb}{1.000000,1.000000,1.000000}%
\pgfsetstrokecolor{currentstroke}%
\pgfsetdash{}{0pt}%
\pgfpathmoveto{\pgfqpoint{1.251241in}{2.323519in}}%
\pgfpathlineto{\pgfqpoint{3.517873in}{2.958734in}}%
\pgfpathlineto{\pgfqpoint{4.931743in}{1.816596in}}%
\pgfusepath{stroke}%
\end{pgfscope}%
\begin{pgfscope}%
\pgfsetbuttcap%
\pgfsetroundjoin%
\pgfsetlinewidth{1.003750pt}%
\definecolor{currentstroke}{rgb}{1.000000,1.000000,1.000000}%
\pgfsetstrokecolor{currentstroke}%
\pgfsetdash{}{0pt}%
\pgfpathmoveto{\pgfqpoint{1.243572in}{2.491004in}}%
\pgfpathlineto{\pgfqpoint{3.519736in}{3.120759in}}%
\pgfpathlineto{\pgfqpoint{4.940010in}{1.988254in}}%
\pgfusepath{stroke}%
\end{pgfscope}%
\begin{pgfscope}%
\pgfsetbuttcap%
\pgfsetroundjoin%
\pgfsetlinewidth{1.003750pt}%
\definecolor{currentstroke}{rgb}{1.000000,1.000000,1.000000}%
\pgfsetstrokecolor{currentstroke}%
\pgfsetdash{}{0pt}%
\pgfpathmoveto{\pgfqpoint{1.235837in}{2.659917in}}%
\pgfpathlineto{\pgfqpoint{3.521614in}{3.284100in}}%
\pgfpathlineto{\pgfqpoint{4.948349in}{2.161433in}}%
\pgfusepath{stroke}%
\end{pgfscope}%
\begin{pgfscope}%
\pgfsetbuttcap%
\pgfsetroundjoin%
\pgfsetlinewidth{1.003750pt}%
\definecolor{currentstroke}{rgb}{1.000000,1.000000,1.000000}%
\pgfsetstrokecolor{currentstroke}%
\pgfsetdash{}{0pt}%
\pgfpathmoveto{\pgfqpoint{1.228035in}{2.830278in}}%
\pgfpathlineto{\pgfqpoint{3.523508in}{3.448773in}}%
\pgfpathlineto{\pgfqpoint{4.956763in}{2.336152in}}%
\pgfusepath{stroke}%
\end{pgfscope}%
\begin{pgfscope}%
\pgfsetbuttcap%
\pgfsetroundjoin%
\pgfsetlinewidth{1.003750pt}%
\definecolor{currentstroke}{rgb}{1.000000,1.000000,1.000000}%
\pgfsetstrokecolor{currentstroke}%
\pgfsetdash{}{0pt}%
\pgfpathmoveto{\pgfqpoint{1.220167in}{3.002104in}}%
\pgfpathlineto{\pgfqpoint{3.525417in}{3.614794in}}%
\pgfpathlineto{\pgfqpoint{4.965252in}{2.512433in}}%
\pgfusepath{stroke}%
\end{pgfscope}%
\begin{pgfscope}%
\pgfsetbuttcap%
\pgfsetroundjoin%
\pgfsetlinewidth{1.003750pt}%
\definecolor{currentstroke}{rgb}{1.000000,1.000000,1.000000}%
\pgfsetstrokecolor{currentstroke}%
\pgfsetdash{}{0pt}%
\pgfpathmoveto{\pgfqpoint{1.212231in}{3.175415in}}%
\pgfpathlineto{\pgfqpoint{3.527342in}{3.782180in}}%
\pgfpathlineto{\pgfqpoint{4.973818in}{2.690297in}}%
\pgfusepath{stroke}%
\end{pgfscope}%
\begin{pgfscope}%
\pgfsetbuttcap%
\pgfsetroundjoin%
\pgfsetlinewidth{1.003750pt}%
\definecolor{currentstroke}{rgb}{1.000000,1.000000,1.000000}%
\pgfsetstrokecolor{currentstroke}%
\pgfsetdash{}{0pt}%
\pgfpathmoveto{\pgfqpoint{1.204225in}{3.350230in}}%
\pgfpathlineto{\pgfqpoint{3.529282in}{3.950947in}}%
\pgfpathlineto{\pgfqpoint{4.982460in}{2.869764in}}%
\pgfusepath{stroke}%
\end{pgfscope}%
\begin{pgfscope}%
\pgfsetbuttcap%
\pgfsetroundjoin%
\pgfsetlinewidth{1.003750pt}%
\definecolor{currentstroke}{rgb}{1.000000,1.000000,1.000000}%
\pgfsetstrokecolor{currentstroke}%
\pgfsetdash{}{0pt}%
\pgfpathmoveto{\pgfqpoint{1.196150in}{3.526569in}}%
\pgfpathlineto{\pgfqpoint{3.531239in}{4.121113in}}%
\pgfpathlineto{\pgfqpoint{4.991181in}{3.050857in}}%
\pgfusepath{stroke}%
\end{pgfscope}%
\begin{pgfscope}%
\pgfpathrectangle{\pgfqpoint{0.887500in}{0.275000in}}{\pgfqpoint{4.225000in}{4.225000in}}%
\pgfusepath{clip}%
\pgfsetroundcap%
\pgfsetroundjoin%
\pgfsetlinewidth{1.003750pt}%
\definecolor{currentstroke}{rgb}{1.000000,1.000000,1.000000}%
\pgfsetstrokecolor{currentstroke}%
\pgfsetdash{}{0pt}%
\pgfpathmoveto{\pgfqpoint{3.057095in}{2.444595in}}%
\pgfusepath{stroke}%
\end{pgfscope}%
\begin{pgfscope}%
\pgfsetroundcap%
\pgfsetroundjoin%
\pgfsetlinewidth{1.254687pt}%
\definecolor{currentstroke}{rgb}{0.150000,0.150000,0.150000}%
\pgfsetstrokecolor{currentstroke}%
\pgfsetdash{}{0pt}%
\pgfpathmoveto{\pgfqpoint{1.285256in}{1.998185in}}%
\pgfpathlineto{\pgfqpoint{1.228605in}{1.981909in}}%
\pgfusepath{stroke}%
\end{pgfscope}%
\begin{pgfscope}%
\definecolor{textcolor}{rgb}{0.150000,0.150000,0.150000}%
\pgfsetstrokecolor{textcolor}%
\pgfsetfillcolor{textcolor}%
\pgftext[x=1.011837in,y=2.028690in,,top]{\color{textcolor}\sffamily\fontsize{11.000000}{13.200000}\selectfont \ensuremath{-}0.2}%
\end{pgfscope}%
\begin{pgfscope}%
\pgfpathrectangle{\pgfqpoint{0.887500in}{0.275000in}}{\pgfqpoint{4.225000in}{4.225000in}}%
\pgfusepath{clip}%
\pgfsetroundcap%
\pgfsetroundjoin%
\pgfsetlinewidth{1.003750pt}%
\definecolor{currentstroke}{rgb}{1.000000,1.000000,1.000000}%
\pgfsetstrokecolor{currentstroke}%
\pgfsetdash{}{0pt}%
\pgfpathmoveto{\pgfqpoint{3.057095in}{2.444595in}}%
\pgfusepath{stroke}%
\end{pgfscope}%
\begin{pgfscope}%
\pgfsetroundcap%
\pgfsetroundjoin%
\pgfsetlinewidth{1.254687pt}%
\definecolor{currentstroke}{rgb}{0.150000,0.150000,0.150000}%
\pgfsetstrokecolor{currentstroke}%
\pgfsetdash{}{0pt}%
\pgfpathmoveto{\pgfqpoint{1.277798in}{2.162824in}}%
\pgfpathlineto{\pgfqpoint{1.220898in}{2.146676in}}%
\pgfusepath{stroke}%
\end{pgfscope}%
\begin{pgfscope}%
\definecolor{textcolor}{rgb}{0.150000,0.150000,0.150000}%
\pgfsetstrokecolor{textcolor}%
\pgfsetfillcolor{textcolor}%
\pgftext[x=1.003247in,y=2.193086in,,top]{\color{textcolor}\sffamily\fontsize{11.000000}{13.200000}\selectfont 0.0}%
\end{pgfscope}%
\begin{pgfscope}%
\pgfpathrectangle{\pgfqpoint{0.887500in}{0.275000in}}{\pgfqpoint{4.225000in}{4.225000in}}%
\pgfusepath{clip}%
\pgfsetroundcap%
\pgfsetroundjoin%
\pgfsetlinewidth{1.003750pt}%
\definecolor{currentstroke}{rgb}{1.000000,1.000000,1.000000}%
\pgfsetstrokecolor{currentstroke}%
\pgfsetdash{}{0pt}%
\pgfpathmoveto{\pgfqpoint{3.057095in}{2.444595in}}%
\pgfusepath{stroke}%
\end{pgfscope}%
\begin{pgfscope}%
\pgfsetroundcap%
\pgfsetroundjoin%
\pgfsetlinewidth{1.254687pt}%
\definecolor{currentstroke}{rgb}{0.150000,0.150000,0.150000}%
\pgfsetstrokecolor{currentstroke}%
\pgfsetdash{}{0pt}%
\pgfpathmoveto{\pgfqpoint{1.270276in}{2.328854in}}%
\pgfpathlineto{\pgfqpoint{1.213127in}{2.312838in}}%
\pgfusepath{stroke}%
\end{pgfscope}%
\begin{pgfscope}%
\definecolor{textcolor}{rgb}{0.150000,0.150000,0.150000}%
\pgfsetstrokecolor{textcolor}%
\pgfsetfillcolor{textcolor}%
\pgftext[x=0.994584in,y=2.358869in,,top]{\color{textcolor}\sffamily\fontsize{11.000000}{13.200000}\selectfont 0.2}%
\end{pgfscope}%
\begin{pgfscope}%
\pgfpathrectangle{\pgfqpoint{0.887500in}{0.275000in}}{\pgfqpoint{4.225000in}{4.225000in}}%
\pgfusepath{clip}%
\pgfsetroundcap%
\pgfsetroundjoin%
\pgfsetlinewidth{1.003750pt}%
\definecolor{currentstroke}{rgb}{1.000000,1.000000,1.000000}%
\pgfsetstrokecolor{currentstroke}%
\pgfsetdash{}{0pt}%
\pgfpathmoveto{\pgfqpoint{3.057095in}{2.444595in}}%
\pgfusepath{stroke}%
\end{pgfscope}%
\begin{pgfscope}%
\pgfsetroundcap%
\pgfsetroundjoin%
\pgfsetlinewidth{1.254687pt}%
\definecolor{currentstroke}{rgb}{0.150000,0.150000,0.150000}%
\pgfsetstrokecolor{currentstroke}%
\pgfsetdash{}{0pt}%
\pgfpathmoveto{\pgfqpoint{1.262690in}{2.496294in}}%
\pgfpathlineto{\pgfqpoint{1.205289in}{2.480412in}}%
\pgfusepath{stroke}%
\end{pgfscope}%
\begin{pgfscope}%
\definecolor{textcolor}{rgb}{0.150000,0.150000,0.150000}%
\pgfsetstrokecolor{textcolor}%
\pgfsetfillcolor{textcolor}%
\pgftext[x=0.985847in,y=2.526057in,,top]{\color{textcolor}\sffamily\fontsize{11.000000}{13.200000}\selectfont 0.4}%
\end{pgfscope}%
\begin{pgfscope}%
\pgfpathrectangle{\pgfqpoint{0.887500in}{0.275000in}}{\pgfqpoint{4.225000in}{4.225000in}}%
\pgfusepath{clip}%
\pgfsetroundcap%
\pgfsetroundjoin%
\pgfsetlinewidth{1.003750pt}%
\definecolor{currentstroke}{rgb}{1.000000,1.000000,1.000000}%
\pgfsetstrokecolor{currentstroke}%
\pgfsetdash{}{0pt}%
\pgfpathmoveto{\pgfqpoint{3.057095in}{2.444595in}}%
\pgfusepath{stroke}%
\end{pgfscope}%
\begin{pgfscope}%
\pgfsetroundcap%
\pgfsetroundjoin%
\pgfsetlinewidth{1.254687pt}%
\definecolor{currentstroke}{rgb}{0.150000,0.150000,0.150000}%
\pgfsetstrokecolor{currentstroke}%
\pgfsetdash{}{0pt}%
\pgfpathmoveto{\pgfqpoint{1.255040in}{2.665161in}}%
\pgfpathlineto{\pgfqpoint{1.197384in}{2.649417in}}%
\pgfusepath{stroke}%
\end{pgfscope}%
\begin{pgfscope}%
\definecolor{textcolor}{rgb}{0.150000,0.150000,0.150000}%
\pgfsetstrokecolor{textcolor}%
\pgfsetfillcolor{textcolor}%
\pgftext[x=0.977037in,y=2.694667in,,top]{\color{textcolor}\sffamily\fontsize{11.000000}{13.200000}\selectfont 0.6}%
\end{pgfscope}%
\begin{pgfscope}%
\pgfpathrectangle{\pgfqpoint{0.887500in}{0.275000in}}{\pgfqpoint{4.225000in}{4.225000in}}%
\pgfusepath{clip}%
\pgfsetroundcap%
\pgfsetroundjoin%
\pgfsetlinewidth{1.003750pt}%
\definecolor{currentstroke}{rgb}{1.000000,1.000000,1.000000}%
\pgfsetstrokecolor{currentstroke}%
\pgfsetdash{}{0pt}%
\pgfpathmoveto{\pgfqpoint{3.057095in}{2.444595in}}%
\pgfusepath{stroke}%
\end{pgfscope}%
\begin{pgfscope}%
\pgfsetroundcap%
\pgfsetroundjoin%
\pgfsetlinewidth{1.254687pt}%
\definecolor{currentstroke}{rgb}{0.150000,0.150000,0.150000}%
\pgfsetstrokecolor{currentstroke}%
\pgfsetdash{}{0pt}%
\pgfpathmoveto{\pgfqpoint{1.247324in}{2.835475in}}%
\pgfpathlineto{\pgfqpoint{1.189412in}{2.819871in}}%
\pgfusepath{stroke}%
\end{pgfscope}%
\begin{pgfscope}%
\definecolor{textcolor}{rgb}{0.150000,0.150000,0.150000}%
\pgfsetstrokecolor{textcolor}%
\pgfsetfillcolor{textcolor}%
\pgftext[x=0.968151in,y=2.864717in,,top]{\color{textcolor}\sffamily\fontsize{11.000000}{13.200000}\selectfont 0.8}%
\end{pgfscope}%
\begin{pgfscope}%
\pgfpathrectangle{\pgfqpoint{0.887500in}{0.275000in}}{\pgfqpoint{4.225000in}{4.225000in}}%
\pgfusepath{clip}%
\pgfsetroundcap%
\pgfsetroundjoin%
\pgfsetlinewidth{1.003750pt}%
\definecolor{currentstroke}{rgb}{1.000000,1.000000,1.000000}%
\pgfsetstrokecolor{currentstroke}%
\pgfsetdash{}{0pt}%
\pgfpathmoveto{\pgfqpoint{3.057095in}{2.444595in}}%
\pgfusepath{stroke}%
\end{pgfscope}%
\begin{pgfscope}%
\pgfsetroundcap%
\pgfsetroundjoin%
\pgfsetlinewidth{1.254687pt}%
\definecolor{currentstroke}{rgb}{0.150000,0.150000,0.150000}%
\pgfsetstrokecolor{currentstroke}%
\pgfsetdash{}{0pt}%
\pgfpathmoveto{\pgfqpoint{1.239542in}{3.007253in}}%
\pgfpathlineto{\pgfqpoint{1.181371in}{2.991793in}}%
\pgfusepath{stroke}%
\end{pgfscope}%
\begin{pgfscope}%
\definecolor{textcolor}{rgb}{0.150000,0.150000,0.150000}%
\pgfsetstrokecolor{textcolor}%
\pgfsetfillcolor{textcolor}%
\pgftext[x=0.959189in,y=3.036227in,,top]{\color{textcolor}\sffamily\fontsize{11.000000}{13.200000}\selectfont 1.0}%
\end{pgfscope}%
\begin{pgfscope}%
\pgfpathrectangle{\pgfqpoint{0.887500in}{0.275000in}}{\pgfqpoint{4.225000in}{4.225000in}}%
\pgfusepath{clip}%
\pgfsetroundcap%
\pgfsetroundjoin%
\pgfsetlinewidth{1.003750pt}%
\definecolor{currentstroke}{rgb}{1.000000,1.000000,1.000000}%
\pgfsetstrokecolor{currentstroke}%
\pgfsetdash{}{0pt}%
\pgfpathmoveto{\pgfqpoint{3.057095in}{2.444595in}}%
\pgfusepath{stroke}%
\end{pgfscope}%
\begin{pgfscope}%
\pgfsetroundcap%
\pgfsetroundjoin%
\pgfsetlinewidth{1.254687pt}%
\definecolor{currentstroke}{rgb}{0.150000,0.150000,0.150000}%
\pgfsetstrokecolor{currentstroke}%
\pgfsetdash{}{0pt}%
\pgfpathmoveto{\pgfqpoint{1.231692in}{3.180516in}}%
\pgfpathlineto{\pgfqpoint{1.173260in}{3.165201in}}%
\pgfusepath{stroke}%
\end{pgfscope}%
\begin{pgfscope}%
\definecolor{textcolor}{rgb}{0.150000,0.150000,0.150000}%
\pgfsetstrokecolor{textcolor}%
\pgfsetfillcolor{textcolor}%
\pgftext[x=0.950149in,y=3.209215in,,top]{\color{textcolor}\sffamily\fontsize{11.000000}{13.200000}\selectfont 1.2}%
\end{pgfscope}%
\begin{pgfscope}%
\pgfpathrectangle{\pgfqpoint{0.887500in}{0.275000in}}{\pgfqpoint{4.225000in}{4.225000in}}%
\pgfusepath{clip}%
\pgfsetroundcap%
\pgfsetroundjoin%
\pgfsetlinewidth{1.003750pt}%
\definecolor{currentstroke}{rgb}{1.000000,1.000000,1.000000}%
\pgfsetstrokecolor{currentstroke}%
\pgfsetdash{}{0pt}%
\pgfpathmoveto{\pgfqpoint{3.057095in}{2.444595in}}%
\pgfusepath{stroke}%
\end{pgfscope}%
\begin{pgfscope}%
\pgfsetroundcap%
\pgfsetroundjoin%
\pgfsetlinewidth{1.254687pt}%
\definecolor{currentstroke}{rgb}{0.150000,0.150000,0.150000}%
\pgfsetstrokecolor{currentstroke}%
\pgfsetdash{}{0pt}%
\pgfpathmoveto{\pgfqpoint{1.223774in}{3.355281in}}%
\pgfpathlineto{\pgfqpoint{1.165079in}{3.340116in}}%
\pgfusepath{stroke}%
\end{pgfscope}%
\begin{pgfscope}%
\definecolor{textcolor}{rgb}{0.150000,0.150000,0.150000}%
\pgfsetstrokecolor{textcolor}%
\pgfsetfillcolor{textcolor}%
\pgftext[x=0.941032in,y=3.383699in,,top]{\color{textcolor}\sffamily\fontsize{11.000000}{13.200000}\selectfont 1.4}%
\end{pgfscope}%
\begin{pgfscope}%
\pgfpathrectangle{\pgfqpoint{0.887500in}{0.275000in}}{\pgfqpoint{4.225000in}{4.225000in}}%
\pgfusepath{clip}%
\pgfsetroundcap%
\pgfsetroundjoin%
\pgfsetlinewidth{1.003750pt}%
\definecolor{currentstroke}{rgb}{1.000000,1.000000,1.000000}%
\pgfsetstrokecolor{currentstroke}%
\pgfsetdash{}{0pt}%
\pgfpathmoveto{\pgfqpoint{3.057095in}{2.444595in}}%
\pgfusepath{stroke}%
\end{pgfscope}%
\begin{pgfscope}%
\pgfsetroundcap%
\pgfsetroundjoin%
\pgfsetlinewidth{1.254687pt}%
\definecolor{currentstroke}{rgb}{0.150000,0.150000,0.150000}%
\pgfsetstrokecolor{currentstroke}%
\pgfsetdash{}{0pt}%
\pgfpathmoveto{\pgfqpoint{1.215788in}{3.531569in}}%
\pgfpathlineto{\pgfqpoint{1.156826in}{3.516556in}}%
\pgfusepath{stroke}%
\end{pgfscope}%
\begin{pgfscope}%
\definecolor{textcolor}{rgb}{0.150000,0.150000,0.150000}%
\pgfsetstrokecolor{textcolor}%
\pgfsetfillcolor{textcolor}%
\pgftext[x=0.931835in,y=3.559701in,,top]{\color{textcolor}\sffamily\fontsize{11.000000}{13.200000}\selectfont 1.6}%
\end{pgfscope}%
\begin{pgfscope}%
\pgfpathrectangle{\pgfqpoint{0.887500in}{0.275000in}}{\pgfqpoint{4.225000in}{4.225000in}}%
\pgfusepath{clip}%
\pgfsetbuttcap%
\pgfsetroundjoin%
\definecolor{currentfill}{rgb}{0.730889,0.871916,0.156029}%
\pgfsetfillcolor{currentfill}%
\pgfsetfillopacity{0.700000}%
\pgfsetlinewidth{0.501875pt}%
\definecolor{currentstroke}{rgb}{1.000000,1.000000,1.000000}%
\pgfsetstrokecolor{currentstroke}%
\pgfsetstrokeopacity{0.500000}%
\pgfsetdash{}{0pt}%
\pgfpathmoveto{\pgfqpoint{3.492699in}{3.844824in}}%
\pgfpathlineto{\pgfqpoint{3.503390in}{3.845218in}}%
\pgfpathlineto{\pgfqpoint{3.514072in}{3.845374in}}%
\pgfpathlineto{\pgfqpoint{3.524745in}{3.845288in}}%
\pgfpathlineto{\pgfqpoint{3.535409in}{3.844954in}}%
\pgfpathlineto{\pgfqpoint{3.529317in}{3.851359in}}%
\pgfpathlineto{\pgfqpoint{3.523245in}{3.859164in}}%
\pgfpathlineto{\pgfqpoint{3.517194in}{3.868459in}}%
\pgfpathlineto{\pgfqpoint{3.511163in}{3.879334in}}%
\pgfpathlineto{\pgfqpoint{3.500452in}{3.874791in}}%
\pgfpathlineto{\pgfqpoint{3.489737in}{3.870327in}}%
\pgfpathlineto{\pgfqpoint{3.479018in}{3.865950in}}%
\pgfpathlineto{\pgfqpoint{3.468297in}{3.861669in}}%
\pgfpathlineto{\pgfqpoint{3.474377in}{3.856178in}}%
\pgfpathlineto{\pgfqpoint{3.480471in}{3.851577in}}%
\pgfpathlineto{\pgfqpoint{3.486578in}{3.847810in}}%
\pgfpathclose%
\pgfusepath{stroke,fill}%
\end{pgfscope}%
\begin{pgfscope}%
\pgfpathrectangle{\pgfqpoint{0.887500in}{0.275000in}}{\pgfqpoint{4.225000in}{4.225000in}}%
\pgfusepath{clip}%
\pgfsetbuttcap%
\pgfsetroundjoin%
\definecolor{currentfill}{rgb}{0.730889,0.871916,0.156029}%
\pgfsetfillcolor{currentfill}%
\pgfsetfillopacity{0.700000}%
\pgfsetlinewidth{0.501875pt}%
\definecolor{currentstroke}{rgb}{1.000000,1.000000,1.000000}%
\pgfsetstrokecolor{currentstroke}%
\pgfsetstrokeopacity{0.500000}%
\pgfsetdash{}{0pt}%
\pgfpathmoveto{\pgfqpoint{3.439116in}{3.839446in}}%
\pgfpathlineto{\pgfqpoint{3.449848in}{3.840960in}}%
\pgfpathlineto{\pgfqpoint{3.460573in}{3.842260in}}%
\pgfpathlineto{\pgfqpoint{3.471290in}{3.843340in}}%
\pgfpathlineto{\pgfqpoint{3.481999in}{3.844196in}}%
\pgfpathlineto{\pgfqpoint{3.492699in}{3.844824in}}%
\pgfpathlineto{\pgfqpoint{3.486578in}{3.847810in}}%
\pgfpathlineto{\pgfqpoint{3.480471in}{3.851577in}}%
\pgfpathlineto{\pgfqpoint{3.474377in}{3.856178in}}%
\pgfpathlineto{\pgfqpoint{3.468297in}{3.861669in}}%
\pgfpathlineto{\pgfqpoint{3.457572in}{3.857492in}}%
\pgfpathlineto{\pgfqpoint{3.446844in}{3.853429in}}%
\pgfpathlineto{\pgfqpoint{3.436113in}{3.849488in}}%
\pgfpathlineto{\pgfqpoint{3.425378in}{3.845678in}}%
\pgfpathlineto{\pgfqpoint{3.414641in}{3.842008in}}%
\pgfpathlineto{\pgfqpoint{3.420747in}{3.840948in}}%
\pgfpathlineto{\pgfqpoint{3.426861in}{3.840184in}}%
\pgfpathlineto{\pgfqpoint{3.432984in}{3.839691in}}%
\pgfpathclose%
\pgfusepath{stroke,fill}%
\end{pgfscope}%
\begin{pgfscope}%
\pgfpathrectangle{\pgfqpoint{0.887500in}{0.275000in}}{\pgfqpoint{4.225000in}{4.225000in}}%
\pgfusepath{clip}%
\pgfsetbuttcap%
\pgfsetroundjoin%
\definecolor{currentfill}{rgb}{0.741388,0.873449,0.149561}%
\pgfsetfillcolor{currentfill}%
\pgfsetfillopacity{0.700000}%
\pgfsetlinewidth{0.501875pt}%
\definecolor{currentstroke}{rgb}{1.000000,1.000000,1.000000}%
\pgfsetstrokecolor{currentstroke}%
\pgfsetstrokeopacity{0.500000}%
\pgfsetdash{}{0pt}%
\pgfpathmoveto{\pgfqpoint{3.523493in}{3.839711in}}%
\pgfpathlineto{\pgfqpoint{3.534177in}{3.838141in}}%
\pgfpathlineto{\pgfqpoint{3.544850in}{3.836138in}}%
\pgfpathlineto{\pgfqpoint{3.555509in}{3.833699in}}%
\pgfpathlineto{\pgfqpoint{3.566155in}{3.830824in}}%
\pgfpathlineto{\pgfqpoint{3.559969in}{3.831561in}}%
\pgfpathlineto{\pgfqpoint{3.553801in}{3.833253in}}%
\pgfpathlineto{\pgfqpoint{3.547652in}{3.835990in}}%
\pgfpathlineto{\pgfqpoint{3.541520in}{3.839861in}}%
\pgfpathlineto{\pgfqpoint{3.535409in}{3.844954in}}%
\pgfpathlineto{\pgfqpoint{3.524745in}{3.845288in}}%
\pgfpathlineto{\pgfqpoint{3.514072in}{3.845374in}}%
\pgfpathlineto{\pgfqpoint{3.503390in}{3.845218in}}%
\pgfpathlineto{\pgfqpoint{3.492699in}{3.844824in}}%
\pgfpathlineto{\pgfqpoint{3.498832in}{3.842565in}}%
\pgfpathlineto{\pgfqpoint{3.504979in}{3.840977in}}%
\pgfpathlineto{\pgfqpoint{3.511138in}{3.840009in}}%
\pgfpathlineto{\pgfqpoint{3.517309in}{3.839605in}}%
\pgfpathclose%
\pgfusepath{stroke,fill}%
\end{pgfscope}%
\begin{pgfscope}%
\pgfpathrectangle{\pgfqpoint{0.887500in}{0.275000in}}{\pgfqpoint{4.225000in}{4.225000in}}%
\pgfusepath{clip}%
\pgfsetbuttcap%
\pgfsetroundjoin%
\definecolor{currentfill}{rgb}{0.741388,0.873449,0.149561}%
\pgfsetfillcolor{currentfill}%
\pgfsetfillopacity{0.700000}%
\pgfsetlinewidth{0.501875pt}%
\definecolor{currentstroke}{rgb}{1.000000,1.000000,1.000000}%
\pgfsetstrokecolor{currentstroke}%
\pgfsetstrokeopacity{0.500000}%
\pgfsetdash{}{0pt}%
\pgfpathmoveto{\pgfqpoint{3.385341in}{3.828820in}}%
\pgfpathlineto{\pgfqpoint{3.396110in}{3.831336in}}%
\pgfpathlineto{\pgfqpoint{3.406872in}{3.833662in}}%
\pgfpathlineto{\pgfqpoint{3.417627in}{3.835792in}}%
\pgfpathlineto{\pgfqpoint{3.428375in}{3.837721in}}%
\pgfpathlineto{\pgfqpoint{3.439116in}{3.839446in}}%
\pgfpathlineto{\pgfqpoint{3.432984in}{3.839691in}}%
\pgfpathlineto{\pgfqpoint{3.426861in}{3.840184in}}%
\pgfpathlineto{\pgfqpoint{3.420747in}{3.840948in}}%
\pgfpathlineto{\pgfqpoint{3.414641in}{3.842008in}}%
\pgfpathlineto{\pgfqpoint{3.403899in}{3.838485in}}%
\pgfpathlineto{\pgfqpoint{3.393155in}{3.835119in}}%
\pgfpathlineto{\pgfqpoint{3.382407in}{3.831919in}}%
\pgfpathlineto{\pgfqpoint{3.371655in}{3.828893in}}%
\pgfpathlineto{\pgfqpoint{3.360900in}{3.826050in}}%
\pgfpathlineto{\pgfqpoint{3.367001in}{3.826733in}}%
\pgfpathlineto{\pgfqpoint{3.373108in}{3.827427in}}%
\pgfpathlineto{\pgfqpoint{3.379221in}{3.828126in}}%
\pgfpathclose%
\pgfusepath{stroke,fill}%
\end{pgfscope}%
\begin{pgfscope}%
\pgfpathrectangle{\pgfqpoint{0.887500in}{0.275000in}}{\pgfqpoint{4.225000in}{4.225000in}}%
\pgfusepath{clip}%
\pgfsetbuttcap%
\pgfsetroundjoin%
\definecolor{currentfill}{rgb}{0.762373,0.876424,0.137064}%
\pgfsetfillcolor{currentfill}%
\pgfsetfillopacity{0.700000}%
\pgfsetlinewidth{0.501875pt}%
\definecolor{currentstroke}{rgb}{1.000000,1.000000,1.000000}%
\pgfsetstrokecolor{currentstroke}%
\pgfsetstrokeopacity{0.500000}%
\pgfsetdash{}{0pt}%
\pgfpathmoveto{\pgfqpoint{3.469893in}{3.841118in}}%
\pgfpathlineto{\pgfqpoint{3.480635in}{3.841690in}}%
\pgfpathlineto{\pgfqpoint{3.491367in}{3.841836in}}%
\pgfpathlineto{\pgfqpoint{3.502087in}{3.841557in}}%
\pgfpathlineto{\pgfqpoint{3.512796in}{3.840849in}}%
\pgfpathlineto{\pgfqpoint{3.523493in}{3.839711in}}%
\pgfpathlineto{\pgfqpoint{3.517309in}{3.839605in}}%
\pgfpathlineto{\pgfqpoint{3.511138in}{3.840009in}}%
\pgfpathlineto{\pgfqpoint{3.504979in}{3.840977in}}%
\pgfpathlineto{\pgfqpoint{3.498832in}{3.842565in}}%
\pgfpathlineto{\pgfqpoint{3.492699in}{3.844824in}}%
\pgfpathlineto{\pgfqpoint{3.481999in}{3.844196in}}%
\pgfpathlineto{\pgfqpoint{3.471290in}{3.843340in}}%
\pgfpathlineto{\pgfqpoint{3.460573in}{3.842260in}}%
\pgfpathlineto{\pgfqpoint{3.449848in}{3.840960in}}%
\pgfpathlineto{\pgfqpoint{3.439116in}{3.839446in}}%
\pgfpathlineto{\pgfqpoint{3.445255in}{3.839425in}}%
\pgfpathlineto{\pgfqpoint{3.451403in}{3.839606in}}%
\pgfpathlineto{\pgfqpoint{3.457559in}{3.839964in}}%
\pgfpathlineto{\pgfqpoint{3.463722in}{3.840476in}}%
\pgfpathclose%
\pgfusepath{stroke,fill}%
\end{pgfscope}%
\begin{pgfscope}%
\pgfpathrectangle{\pgfqpoint{0.887500in}{0.275000in}}{\pgfqpoint{4.225000in}{4.225000in}}%
\pgfusepath{clip}%
\pgfsetbuttcap%
\pgfsetroundjoin%
\definecolor{currentfill}{rgb}{0.741388,0.873449,0.149561}%
\pgfsetfillcolor{currentfill}%
\pgfsetfillopacity{0.700000}%
\pgfsetlinewidth{0.501875pt}%
\definecolor{currentstroke}{rgb}{1.000000,1.000000,1.000000}%
\pgfsetstrokecolor{currentstroke}%
\pgfsetstrokeopacity{0.500000}%
\pgfsetdash{}{0pt}%
\pgfpathmoveto{\pgfqpoint{3.331395in}{3.813049in}}%
\pgfpathlineto{\pgfqpoint{3.342198in}{3.816678in}}%
\pgfpathlineto{\pgfqpoint{3.352994in}{3.820055in}}%
\pgfpathlineto{\pgfqpoint{3.363783in}{3.823193in}}%
\pgfpathlineto{\pgfqpoint{3.374565in}{3.826110in}}%
\pgfpathlineto{\pgfqpoint{3.385341in}{3.828820in}}%
\pgfpathlineto{\pgfqpoint{3.379221in}{3.828126in}}%
\pgfpathlineto{\pgfqpoint{3.373108in}{3.827427in}}%
\pgfpathlineto{\pgfqpoint{3.367001in}{3.826733in}}%
\pgfpathlineto{\pgfqpoint{3.360900in}{3.826050in}}%
\pgfpathlineto{\pgfqpoint{3.350141in}{3.823363in}}%
\pgfpathlineto{\pgfqpoint{3.339377in}{3.820748in}}%
\pgfpathlineto{\pgfqpoint{3.328608in}{3.818110in}}%
\pgfpathlineto{\pgfqpoint{3.317833in}{3.815357in}}%
\pgfpathlineto{\pgfqpoint{3.307051in}{3.812396in}}%
\pgfpathlineto{\pgfqpoint{3.313128in}{3.812583in}}%
\pgfpathlineto{\pgfqpoint{3.319211in}{3.812757in}}%
\pgfpathlineto{\pgfqpoint{3.325300in}{3.812914in}}%
\pgfpathclose%
\pgfusepath{stroke,fill}%
\end{pgfscope}%
\begin{pgfscope}%
\pgfpathrectangle{\pgfqpoint{0.887500in}{0.275000in}}{\pgfqpoint{4.225000in}{4.225000in}}%
\pgfusepath{clip}%
\pgfsetbuttcap%
\pgfsetroundjoin%
\definecolor{currentfill}{rgb}{0.772852,0.877868,0.131109}%
\pgfsetfillcolor{currentfill}%
\pgfsetfillopacity{0.700000}%
\pgfsetlinewidth{0.501875pt}%
\definecolor{currentstroke}{rgb}{1.000000,1.000000,1.000000}%
\pgfsetstrokecolor{currentstroke}%
\pgfsetstrokeopacity{0.500000}%
\pgfsetdash{}{0pt}%
\pgfpathmoveto{\pgfqpoint{3.554573in}{3.845992in}}%
\pgfpathlineto{\pgfqpoint{3.565280in}{3.844758in}}%
\pgfpathlineto{\pgfqpoint{3.575974in}{3.843076in}}%
\pgfpathlineto{\pgfqpoint{3.586655in}{3.840945in}}%
\pgfpathlineto{\pgfqpoint{3.597322in}{3.838363in}}%
\pgfpathlineto{\pgfqpoint{3.591060in}{3.835656in}}%
\pgfpathlineto{\pgfqpoint{3.584811in}{3.833459in}}%
\pgfpathlineto{\pgfqpoint{3.578577in}{3.831862in}}%
\pgfpathlineto{\pgfqpoint{3.572358in}{3.830954in}}%
\pgfpathlineto{\pgfqpoint{3.566155in}{3.830824in}}%
\pgfpathlineto{\pgfqpoint{3.555509in}{3.833699in}}%
\pgfpathlineto{\pgfqpoint{3.544850in}{3.836138in}}%
\pgfpathlineto{\pgfqpoint{3.534177in}{3.838141in}}%
\pgfpathlineto{\pgfqpoint{3.523493in}{3.839711in}}%
\pgfpathlineto{\pgfqpoint{3.529688in}{3.840273in}}%
\pgfpathlineto{\pgfqpoint{3.535894in}{3.841236in}}%
\pgfpathlineto{\pgfqpoint{3.542110in}{3.842547in}}%
\pgfpathlineto{\pgfqpoint{3.548337in}{3.844150in}}%
\pgfpathclose%
\pgfusepath{stroke,fill}%
\end{pgfscope}%
\begin{pgfscope}%
\pgfpathrectangle{\pgfqpoint{0.887500in}{0.275000in}}{\pgfqpoint{4.225000in}{4.225000in}}%
\pgfusepath{clip}%
\pgfsetbuttcap%
\pgfsetroundjoin%
\definecolor{currentfill}{rgb}{0.730889,0.871916,0.156029}%
\pgfsetfillcolor{currentfill}%
\pgfsetfillopacity{0.700000}%
\pgfsetlinewidth{0.501875pt}%
\definecolor{currentstroke}{rgb}{1.000000,1.000000,1.000000}%
\pgfsetstrokecolor{currentstroke}%
\pgfsetstrokeopacity{0.500000}%
\pgfsetdash{}{0pt}%
\pgfpathmoveto{\pgfqpoint{3.277292in}{3.790926in}}%
\pgfpathlineto{\pgfqpoint{3.288124in}{3.795822in}}%
\pgfpathlineto{\pgfqpoint{3.298950in}{3.800517in}}%
\pgfpathlineto{\pgfqpoint{3.309772in}{3.804972in}}%
\pgfpathlineto{\pgfqpoint{3.320587in}{3.809152in}}%
\pgfpathlineto{\pgfqpoint{3.331395in}{3.813049in}}%
\pgfpathlineto{\pgfqpoint{3.325300in}{3.812914in}}%
\pgfpathlineto{\pgfqpoint{3.319211in}{3.812757in}}%
\pgfpathlineto{\pgfqpoint{3.313128in}{3.812583in}}%
\pgfpathlineto{\pgfqpoint{3.307051in}{3.812396in}}%
\pgfpathlineto{\pgfqpoint{3.296263in}{3.809132in}}%
\pgfpathlineto{\pgfqpoint{3.285468in}{3.805475in}}%
\pgfpathlineto{\pgfqpoint{3.274665in}{3.801401in}}%
\pgfpathlineto{\pgfqpoint{3.263857in}{3.796971in}}%
\pgfpathlineto{\pgfqpoint{3.253043in}{3.792250in}}%
\pgfpathlineto{\pgfqpoint{3.259096in}{3.791960in}}%
\pgfpathlineto{\pgfqpoint{3.265155in}{3.791643in}}%
\pgfpathlineto{\pgfqpoint{3.271221in}{3.791299in}}%
\pgfpathclose%
\pgfusepath{stroke,fill}%
\end{pgfscope}%
\begin{pgfscope}%
\pgfpathrectangle{\pgfqpoint{0.887500in}{0.275000in}}{\pgfqpoint{4.225000in}{4.225000in}}%
\pgfusepath{clip}%
\pgfsetbuttcap%
\pgfsetroundjoin%
\definecolor{currentfill}{rgb}{0.772852,0.877868,0.131109}%
\pgfsetfillcolor{currentfill}%
\pgfsetfillopacity{0.700000}%
\pgfsetlinewidth{0.501875pt}%
\definecolor{currentstroke}{rgb}{1.000000,1.000000,1.000000}%
\pgfsetstrokecolor{currentstroke}%
\pgfsetstrokeopacity{0.500000}%
\pgfsetdash{}{0pt}%
\pgfpathmoveto{\pgfqpoint{3.416033in}{3.831960in}}%
\pgfpathlineto{\pgfqpoint{3.426824in}{3.834626in}}%
\pgfpathlineto{\pgfqpoint{3.437606in}{3.836876in}}%
\pgfpathlineto{\pgfqpoint{3.448378in}{3.838710in}}%
\pgfpathlineto{\pgfqpoint{3.459141in}{3.840124in}}%
\pgfpathlineto{\pgfqpoint{3.469893in}{3.841118in}}%
\pgfpathlineto{\pgfqpoint{3.463722in}{3.840476in}}%
\pgfpathlineto{\pgfqpoint{3.457559in}{3.839964in}}%
\pgfpathlineto{\pgfqpoint{3.451403in}{3.839606in}}%
\pgfpathlineto{\pgfqpoint{3.445255in}{3.839425in}}%
\pgfpathlineto{\pgfqpoint{3.439116in}{3.839446in}}%
\pgfpathlineto{\pgfqpoint{3.428375in}{3.837721in}}%
\pgfpathlineto{\pgfqpoint{3.417627in}{3.835792in}}%
\pgfpathlineto{\pgfqpoint{3.406872in}{3.833662in}}%
\pgfpathlineto{\pgfqpoint{3.396110in}{3.831336in}}%
\pgfpathlineto{\pgfqpoint{3.385341in}{3.828820in}}%
\pgfpathlineto{\pgfqpoint{3.391467in}{3.829502in}}%
\pgfpathlineto{\pgfqpoint{3.397600in}{3.830165in}}%
\pgfpathlineto{\pgfqpoint{3.403738in}{3.830801in}}%
\pgfpathlineto{\pgfqpoint{3.409883in}{3.831402in}}%
\pgfpathclose%
\pgfusepath{stroke,fill}%
\end{pgfscope}%
\begin{pgfscope}%
\pgfpathrectangle{\pgfqpoint{0.887500in}{0.275000in}}{\pgfqpoint{4.225000in}{4.225000in}}%
\pgfusepath{clip}%
\pgfsetbuttcap%
\pgfsetroundjoin%
\definecolor{currentfill}{rgb}{0.709898,0.868751,0.169257}%
\pgfsetfillcolor{currentfill}%
\pgfsetfillopacity{0.700000}%
\pgfsetlinewidth{0.501875pt}%
\definecolor{currentstroke}{rgb}{1.000000,1.000000,1.000000}%
\pgfsetstrokecolor{currentstroke}%
\pgfsetstrokeopacity{0.500000}%
\pgfsetdash{}{0pt}%
\pgfpathmoveto{\pgfqpoint{3.223064in}{3.764696in}}%
\pgfpathlineto{\pgfqpoint{3.233918in}{3.770067in}}%
\pgfpathlineto{\pgfqpoint{3.244768in}{3.775398in}}%
\pgfpathlineto{\pgfqpoint{3.255613in}{3.780676in}}%
\pgfpathlineto{\pgfqpoint{3.266455in}{3.785864in}}%
\pgfpathlineto{\pgfqpoint{3.277292in}{3.790926in}}%
\pgfpathlineto{\pgfqpoint{3.271221in}{3.791299in}}%
\pgfpathlineto{\pgfqpoint{3.265155in}{3.791643in}}%
\pgfpathlineto{\pgfqpoint{3.259096in}{3.791960in}}%
\pgfpathlineto{\pgfqpoint{3.253043in}{3.792250in}}%
\pgfpathlineto{\pgfqpoint{3.242223in}{3.787306in}}%
\pgfpathlineto{\pgfqpoint{3.231399in}{3.782204in}}%
\pgfpathlineto{\pgfqpoint{3.220571in}{3.777011in}}%
\pgfpathlineto{\pgfqpoint{3.209739in}{3.771791in}}%
\pgfpathlineto{\pgfqpoint{3.198903in}{3.766567in}}%
\pgfpathlineto{\pgfqpoint{3.204934in}{3.766195in}}%
\pgfpathlineto{\pgfqpoint{3.210972in}{3.765758in}}%
\pgfpathlineto{\pgfqpoint{3.217015in}{3.765257in}}%
\pgfpathclose%
\pgfusepath{stroke,fill}%
\end{pgfscope}%
\begin{pgfscope}%
\pgfpathrectangle{\pgfqpoint{0.887500in}{0.275000in}}{\pgfqpoint{4.225000in}{4.225000in}}%
\pgfusepath{clip}%
\pgfsetbuttcap%
\pgfsetroundjoin%
\definecolor{currentfill}{rgb}{0.804182,0.882046,0.114965}%
\pgfsetfillcolor{currentfill}%
\pgfsetfillopacity{0.700000}%
\pgfsetlinewidth{0.501875pt}%
\definecolor{currentstroke}{rgb}{1.000000,1.000000,1.000000}%
\pgfsetstrokecolor{currentstroke}%
\pgfsetstrokeopacity{0.500000}%
\pgfsetdash{}{0pt}%
\pgfpathmoveto{\pgfqpoint{3.500858in}{3.845459in}}%
\pgfpathlineto{\pgfqpoint{3.511624in}{3.846456in}}%
\pgfpathlineto{\pgfqpoint{3.522379in}{3.847009in}}%
\pgfpathlineto{\pgfqpoint{3.533122in}{3.847116in}}%
\pgfpathlineto{\pgfqpoint{3.543854in}{3.846777in}}%
\pgfpathlineto{\pgfqpoint{3.554573in}{3.845992in}}%
\pgfpathlineto{\pgfqpoint{3.548337in}{3.844150in}}%
\pgfpathlineto{\pgfqpoint{3.542110in}{3.842547in}}%
\pgfpathlineto{\pgfqpoint{3.535894in}{3.841236in}}%
\pgfpathlineto{\pgfqpoint{3.529688in}{3.840273in}}%
\pgfpathlineto{\pgfqpoint{3.523493in}{3.839711in}}%
\pgfpathlineto{\pgfqpoint{3.512796in}{3.840849in}}%
\pgfpathlineto{\pgfqpoint{3.502087in}{3.841557in}}%
\pgfpathlineto{\pgfqpoint{3.491367in}{3.841836in}}%
\pgfpathlineto{\pgfqpoint{3.480635in}{3.841690in}}%
\pgfpathlineto{\pgfqpoint{3.469893in}{3.841118in}}%
\pgfpathlineto{\pgfqpoint{3.476072in}{3.841868in}}%
\pgfpathlineto{\pgfqpoint{3.482258in}{3.842700in}}%
\pgfpathlineto{\pgfqpoint{3.488451in}{3.843592in}}%
\pgfpathlineto{\pgfqpoint{3.494651in}{3.844519in}}%
\pgfpathclose%
\pgfusepath{stroke,fill}%
\end{pgfscope}%
\begin{pgfscope}%
\pgfpathrectangle{\pgfqpoint{0.887500in}{0.275000in}}{\pgfqpoint{4.225000in}{4.225000in}}%
\pgfusepath{clip}%
\pgfsetbuttcap%
\pgfsetroundjoin%
\definecolor{currentfill}{rgb}{0.783315,0.879285,0.125405}%
\pgfsetfillcolor{currentfill}%
\pgfsetfillopacity{0.700000}%
\pgfsetlinewidth{0.501875pt}%
\definecolor{currentstroke}{rgb}{1.000000,1.000000,1.000000}%
\pgfsetstrokecolor{currentstroke}%
\pgfsetstrokeopacity{0.500000}%
\pgfsetdash{}{0pt}%
\pgfpathmoveto{\pgfqpoint{3.361961in}{3.813258in}}%
\pgfpathlineto{\pgfqpoint{3.372789in}{3.817604in}}%
\pgfpathlineto{\pgfqpoint{3.383611in}{3.821681in}}%
\pgfpathlineto{\pgfqpoint{3.394426in}{3.825455in}}%
\pgfpathlineto{\pgfqpoint{3.405234in}{3.828893in}}%
\pgfpathlineto{\pgfqpoint{3.416033in}{3.831960in}}%
\pgfpathlineto{\pgfqpoint{3.409883in}{3.831402in}}%
\pgfpathlineto{\pgfqpoint{3.403738in}{3.830801in}}%
\pgfpathlineto{\pgfqpoint{3.397600in}{3.830165in}}%
\pgfpathlineto{\pgfqpoint{3.391467in}{3.829502in}}%
\pgfpathlineto{\pgfqpoint{3.385341in}{3.828820in}}%
\pgfpathlineto{\pgfqpoint{3.374565in}{3.826110in}}%
\pgfpathlineto{\pgfqpoint{3.363783in}{3.823193in}}%
\pgfpathlineto{\pgfqpoint{3.352994in}{3.820055in}}%
\pgfpathlineto{\pgfqpoint{3.342198in}{3.816678in}}%
\pgfpathlineto{\pgfqpoint{3.331395in}{3.813049in}}%
\pgfpathlineto{\pgfqpoint{3.337497in}{3.813158in}}%
\pgfpathlineto{\pgfqpoint{3.343604in}{3.813238in}}%
\pgfpathlineto{\pgfqpoint{3.349717in}{3.813284in}}%
\pgfpathlineto{\pgfqpoint{3.355836in}{3.813292in}}%
\pgfpathclose%
\pgfusepath{stroke,fill}%
\end{pgfscope}%
\begin{pgfscope}%
\pgfpathrectangle{\pgfqpoint{0.887500in}{0.275000in}}{\pgfqpoint{4.225000in}{4.225000in}}%
\pgfusepath{clip}%
\pgfsetbuttcap%
\pgfsetroundjoin%
\definecolor{currentfill}{rgb}{0.688944,0.865448,0.182725}%
\pgfsetfillcolor{currentfill}%
\pgfsetfillopacity{0.700000}%
\pgfsetlinewidth{0.501875pt}%
\definecolor{currentstroke}{rgb}{1.000000,1.000000,1.000000}%
\pgfsetstrokecolor{currentstroke}%
\pgfsetstrokeopacity{0.500000}%
\pgfsetdash{}{0pt}%
\pgfpathmoveto{\pgfqpoint{3.168736in}{3.737108in}}%
\pgfpathlineto{\pgfqpoint{3.179609in}{3.742740in}}%
\pgfpathlineto{\pgfqpoint{3.190479in}{3.748310in}}%
\pgfpathlineto{\pgfqpoint{3.201344in}{3.753823in}}%
\pgfpathlineto{\pgfqpoint{3.212206in}{3.759283in}}%
\pgfpathlineto{\pgfqpoint{3.223064in}{3.764696in}}%
\pgfpathlineto{\pgfqpoint{3.217015in}{3.765257in}}%
\pgfpathlineto{\pgfqpoint{3.210972in}{3.765758in}}%
\pgfpathlineto{\pgfqpoint{3.204934in}{3.766195in}}%
\pgfpathlineto{\pgfqpoint{3.198903in}{3.766567in}}%
\pgfpathlineto{\pgfqpoint{3.188063in}{3.761329in}}%
\pgfpathlineto{\pgfqpoint{3.177219in}{3.756065in}}%
\pgfpathlineto{\pgfqpoint{3.166371in}{3.750762in}}%
\pgfpathlineto{\pgfqpoint{3.155519in}{3.745409in}}%
\pgfpathlineto{\pgfqpoint{3.144663in}{3.739992in}}%
\pgfpathlineto{\pgfqpoint{3.150672in}{3.739367in}}%
\pgfpathlineto{\pgfqpoint{3.156688in}{3.738676in}}%
\pgfpathlineto{\pgfqpoint{3.162709in}{3.737921in}}%
\pgfpathclose%
\pgfusepath{stroke,fill}%
\end{pgfscope}%
\begin{pgfscope}%
\pgfpathrectangle{\pgfqpoint{0.887500in}{0.275000in}}{\pgfqpoint{4.225000in}{4.225000in}}%
\pgfusepath{clip}%
\pgfsetbuttcap%
\pgfsetroundjoin%
\definecolor{currentfill}{rgb}{0.772852,0.877868,0.131109}%
\pgfsetfillcolor{currentfill}%
\pgfsetfillopacity{0.700000}%
\pgfsetlinewidth{0.501875pt}%
\definecolor{currentstroke}{rgb}{1.000000,1.000000,1.000000}%
\pgfsetstrokecolor{currentstroke}%
\pgfsetstrokeopacity{0.500000}%
\pgfsetdash{}{0pt}%
\pgfpathmoveto{\pgfqpoint{3.307735in}{3.788592in}}%
\pgfpathlineto{\pgfqpoint{3.318590in}{3.793833in}}%
\pgfpathlineto{\pgfqpoint{3.329441in}{3.798941in}}%
\pgfpathlineto{\pgfqpoint{3.340286in}{3.803896in}}%
\pgfpathlineto{\pgfqpoint{3.351126in}{3.808677in}}%
\pgfpathlineto{\pgfqpoint{3.361961in}{3.813258in}}%
\pgfpathlineto{\pgfqpoint{3.355836in}{3.813292in}}%
\pgfpathlineto{\pgfqpoint{3.349717in}{3.813284in}}%
\pgfpathlineto{\pgfqpoint{3.343604in}{3.813238in}}%
\pgfpathlineto{\pgfqpoint{3.337497in}{3.813158in}}%
\pgfpathlineto{\pgfqpoint{3.331395in}{3.813049in}}%
\pgfpathlineto{\pgfqpoint{3.320587in}{3.809152in}}%
\pgfpathlineto{\pgfqpoint{3.309772in}{3.804972in}}%
\pgfpathlineto{\pgfqpoint{3.298950in}{3.800517in}}%
\pgfpathlineto{\pgfqpoint{3.288124in}{3.795822in}}%
\pgfpathlineto{\pgfqpoint{3.277292in}{3.790926in}}%
\pgfpathlineto{\pgfqpoint{3.283369in}{3.790523in}}%
\pgfpathlineto{\pgfqpoint{3.289452in}{3.790089in}}%
\pgfpathlineto{\pgfqpoint{3.295540in}{3.789623in}}%
\pgfpathlineto{\pgfqpoint{3.301635in}{3.789125in}}%
\pgfpathclose%
\pgfusepath{stroke,fill}%
\end{pgfscope}%
\begin{pgfscope}%
\pgfpathrectangle{\pgfqpoint{0.887500in}{0.275000in}}{\pgfqpoint{4.225000in}{4.225000in}}%
\pgfusepath{clip}%
\pgfsetbuttcap%
\pgfsetroundjoin%
\definecolor{currentfill}{rgb}{0.835270,0.886029,0.102646}%
\pgfsetfillcolor{currentfill}%
\pgfsetfillopacity{0.700000}%
\pgfsetlinewidth{0.501875pt}%
\definecolor{currentstroke}{rgb}{1.000000,1.000000,1.000000}%
\pgfsetstrokecolor{currentstroke}%
\pgfsetstrokeopacity{0.500000}%
\pgfsetdash{}{0pt}%
\pgfpathmoveto{\pgfqpoint{3.585879in}{3.856850in}}%
\pgfpathlineto{\pgfqpoint{3.596627in}{3.857272in}}%
\pgfpathlineto{\pgfqpoint{3.607364in}{3.857341in}}%
\pgfpathlineto{\pgfqpoint{3.618090in}{3.857056in}}%
\pgfpathlineto{\pgfqpoint{3.628804in}{3.856416in}}%
\pgfpathlineto{\pgfqpoint{3.622489in}{3.852504in}}%
\pgfpathlineto{\pgfqpoint{3.616182in}{3.848652in}}%
\pgfpathlineto{\pgfqpoint{3.609884in}{3.844951in}}%
\pgfpathlineto{\pgfqpoint{3.603597in}{3.841492in}}%
\pgfpathlineto{\pgfqpoint{3.597322in}{3.838363in}}%
\pgfpathlineto{\pgfqpoint{3.586655in}{3.840945in}}%
\pgfpathlineto{\pgfqpoint{3.575974in}{3.843076in}}%
\pgfpathlineto{\pgfqpoint{3.565280in}{3.844758in}}%
\pgfpathlineto{\pgfqpoint{3.554573in}{3.845992in}}%
\pgfpathlineto{\pgfqpoint{3.560819in}{3.848017in}}%
\pgfpathlineto{\pgfqpoint{3.567073in}{3.850170in}}%
\pgfpathlineto{\pgfqpoint{3.573335in}{3.852397in}}%
\pgfpathlineto{\pgfqpoint{3.579604in}{3.854642in}}%
\pgfpathclose%
\pgfusepath{stroke,fill}%
\end{pgfscope}%
\begin{pgfscope}%
\pgfpathrectangle{\pgfqpoint{0.887500in}{0.275000in}}{\pgfqpoint{4.225000in}{4.225000in}}%
\pgfusepath{clip}%
\pgfsetbuttcap%
\pgfsetroundjoin%
\definecolor{currentfill}{rgb}{0.824940,0.884720,0.106217}%
\pgfsetfillcolor{currentfill}%
\pgfsetfillopacity{0.700000}%
\pgfsetlinewidth{0.501875pt}%
\definecolor{currentstroke}{rgb}{1.000000,1.000000,1.000000}%
\pgfsetstrokecolor{currentstroke}%
\pgfsetstrokeopacity{0.500000}%
\pgfsetdash{}{0pt}%
\pgfpathmoveto{\pgfqpoint{3.446873in}{3.833842in}}%
\pgfpathlineto{\pgfqpoint{3.457689in}{3.837046in}}%
\pgfpathlineto{\pgfqpoint{3.468496in}{3.839811in}}%
\pgfpathlineto{\pgfqpoint{3.479294in}{3.842135in}}%
\pgfpathlineto{\pgfqpoint{3.490081in}{3.844018in}}%
\pgfpathlineto{\pgfqpoint{3.500858in}{3.845459in}}%
\pgfpathlineto{\pgfqpoint{3.494651in}{3.844519in}}%
\pgfpathlineto{\pgfqpoint{3.488451in}{3.843592in}}%
\pgfpathlineto{\pgfqpoint{3.482258in}{3.842700in}}%
\pgfpathlineto{\pgfqpoint{3.476072in}{3.841868in}}%
\pgfpathlineto{\pgfqpoint{3.469893in}{3.841118in}}%
\pgfpathlineto{\pgfqpoint{3.459141in}{3.840124in}}%
\pgfpathlineto{\pgfqpoint{3.448378in}{3.838710in}}%
\pgfpathlineto{\pgfqpoint{3.437606in}{3.836876in}}%
\pgfpathlineto{\pgfqpoint{3.426824in}{3.834626in}}%
\pgfpathlineto{\pgfqpoint{3.416033in}{3.831960in}}%
\pgfpathlineto{\pgfqpoint{3.422190in}{3.832468in}}%
\pgfpathlineto{\pgfqpoint{3.428352in}{3.832919in}}%
\pgfpathlineto{\pgfqpoint{3.434520in}{3.833303in}}%
\pgfpathlineto{\pgfqpoint{3.440694in}{3.833613in}}%
\pgfpathclose%
\pgfusepath{stroke,fill}%
\end{pgfscope}%
\begin{pgfscope}%
\pgfpathrectangle{\pgfqpoint{0.887500in}{0.275000in}}{\pgfqpoint{4.225000in}{4.225000in}}%
\pgfusepath{clip}%
\pgfsetbuttcap%
\pgfsetroundjoin%
\definecolor{currentfill}{rgb}{0.668054,0.861999,0.196293}%
\pgfsetfillcolor{currentfill}%
\pgfsetfillopacity{0.700000}%
\pgfsetlinewidth{0.501875pt}%
\definecolor{currentstroke}{rgb}{1.000000,1.000000,1.000000}%
\pgfsetstrokecolor{currentstroke}%
\pgfsetstrokeopacity{0.500000}%
\pgfsetdash{}{0pt}%
\pgfpathmoveto{\pgfqpoint{3.114311in}{3.708151in}}%
\pgfpathlineto{\pgfqpoint{3.125203in}{3.714011in}}%
\pgfpathlineto{\pgfqpoint{3.136092in}{3.719850in}}%
\pgfpathlineto{\pgfqpoint{3.146977in}{3.725654in}}%
\pgfpathlineto{\pgfqpoint{3.157858in}{3.731410in}}%
\pgfpathlineto{\pgfqpoint{3.168736in}{3.737108in}}%
\pgfpathlineto{\pgfqpoint{3.162709in}{3.737921in}}%
\pgfpathlineto{\pgfqpoint{3.156688in}{3.738676in}}%
\pgfpathlineto{\pgfqpoint{3.150672in}{3.739367in}}%
\pgfpathlineto{\pgfqpoint{3.144663in}{3.739992in}}%
\pgfpathlineto{\pgfqpoint{3.133803in}{3.734502in}}%
\pgfpathlineto{\pgfqpoint{3.122939in}{3.728945in}}%
\pgfpathlineto{\pgfqpoint{3.112072in}{3.723334in}}%
\pgfpathlineto{\pgfqpoint{3.101200in}{3.717688in}}%
\pgfpathlineto{\pgfqpoint{3.090326in}{3.712019in}}%
\pgfpathlineto{\pgfqpoint{3.096313in}{3.711130in}}%
\pgfpathlineto{\pgfqpoint{3.102307in}{3.710187in}}%
\pgfpathlineto{\pgfqpoint{3.108306in}{3.709193in}}%
\pgfpathclose%
\pgfusepath{stroke,fill}%
\end{pgfscope}%
\begin{pgfscope}%
\pgfpathrectangle{\pgfqpoint{0.887500in}{0.275000in}}{\pgfqpoint{4.225000in}{4.225000in}}%
\pgfusepath{clip}%
\pgfsetbuttcap%
\pgfsetroundjoin%
\definecolor{currentfill}{rgb}{0.751884,0.874951,0.143228}%
\pgfsetfillcolor{currentfill}%
\pgfsetfillopacity{0.700000}%
\pgfsetlinewidth{0.501875pt}%
\definecolor{currentstroke}{rgb}{1.000000,1.000000,1.000000}%
\pgfsetstrokecolor{currentstroke}%
\pgfsetstrokeopacity{0.500000}%
\pgfsetdash{}{0pt}%
\pgfpathmoveto{\pgfqpoint{3.253394in}{3.761084in}}%
\pgfpathlineto{\pgfqpoint{3.264271in}{3.766701in}}%
\pgfpathlineto{\pgfqpoint{3.275143in}{3.772274in}}%
\pgfpathlineto{\pgfqpoint{3.286011in}{3.777793in}}%
\pgfpathlineto{\pgfqpoint{3.296875in}{3.783239in}}%
\pgfpathlineto{\pgfqpoint{3.307735in}{3.788592in}}%
\pgfpathlineto{\pgfqpoint{3.301635in}{3.789125in}}%
\pgfpathlineto{\pgfqpoint{3.295540in}{3.789623in}}%
\pgfpathlineto{\pgfqpoint{3.289452in}{3.790089in}}%
\pgfpathlineto{\pgfqpoint{3.283369in}{3.790523in}}%
\pgfpathlineto{\pgfqpoint{3.277292in}{3.790926in}}%
\pgfpathlineto{\pgfqpoint{3.266455in}{3.785864in}}%
\pgfpathlineto{\pgfqpoint{3.255613in}{3.780676in}}%
\pgfpathlineto{\pgfqpoint{3.244768in}{3.775398in}}%
\pgfpathlineto{\pgfqpoint{3.233918in}{3.770067in}}%
\pgfpathlineto{\pgfqpoint{3.223064in}{3.764696in}}%
\pgfpathlineto{\pgfqpoint{3.229119in}{3.764078in}}%
\pgfpathlineto{\pgfqpoint{3.235179in}{3.763405in}}%
\pgfpathlineto{\pgfqpoint{3.241245in}{3.762680in}}%
\pgfpathlineto{\pgfqpoint{3.247317in}{3.761905in}}%
\pgfpathclose%
\pgfusepath{stroke,fill}%
\end{pgfscope}%
\begin{pgfscope}%
\pgfpathrectangle{\pgfqpoint{0.887500in}{0.275000in}}{\pgfqpoint{4.225000in}{4.225000in}}%
\pgfusepath{clip}%
\pgfsetbuttcap%
\pgfsetroundjoin%
\definecolor{currentfill}{rgb}{0.636902,0.856542,0.216620}%
\pgfsetfillcolor{currentfill}%
\pgfsetfillopacity{0.700000}%
\pgfsetlinewidth{0.501875pt}%
\definecolor{currentstroke}{rgb}{1.000000,1.000000,1.000000}%
\pgfsetstrokecolor{currentstroke}%
\pgfsetstrokeopacity{0.500000}%
\pgfsetdash{}{0pt}%
\pgfpathmoveto{\pgfqpoint{3.059796in}{3.678957in}}%
\pgfpathlineto{\pgfqpoint{3.070706in}{3.684742in}}%
\pgfpathlineto{\pgfqpoint{3.081613in}{3.690565in}}%
\pgfpathlineto{\pgfqpoint{3.092516in}{3.696416in}}%
\pgfpathlineto{\pgfqpoint{3.103415in}{3.702282in}}%
\pgfpathlineto{\pgfqpoint{3.114311in}{3.708151in}}%
\pgfpathlineto{\pgfqpoint{3.108306in}{3.709193in}}%
\pgfpathlineto{\pgfqpoint{3.102307in}{3.710187in}}%
\pgfpathlineto{\pgfqpoint{3.096313in}{3.711130in}}%
\pgfpathlineto{\pgfqpoint{3.090326in}{3.712019in}}%
\pgfpathlineto{\pgfqpoint{3.079447in}{3.706346in}}%
\pgfpathlineto{\pgfqpoint{3.068565in}{3.700683in}}%
\pgfpathlineto{\pgfqpoint{3.057679in}{3.695044in}}%
\pgfpathlineto{\pgfqpoint{3.046789in}{3.689440in}}%
\pgfpathlineto{\pgfqpoint{3.035896in}{3.683877in}}%
\pgfpathlineto{\pgfqpoint{3.041862in}{3.682711in}}%
\pgfpathlineto{\pgfqpoint{3.047835in}{3.681499in}}%
\pgfpathlineto{\pgfqpoint{3.053813in}{3.680245in}}%
\pgfpathclose%
\pgfusepath{stroke,fill}%
\end{pgfscope}%
\begin{pgfscope}%
\pgfpathrectangle{\pgfqpoint{0.887500in}{0.275000in}}{\pgfqpoint{4.225000in}{4.225000in}}%
\pgfusepath{clip}%
\pgfsetbuttcap%
\pgfsetroundjoin%
\definecolor{currentfill}{rgb}{0.824940,0.884720,0.106217}%
\pgfsetfillcolor{currentfill}%
\pgfsetfillopacity{0.700000}%
\pgfsetlinewidth{0.501875pt}%
\definecolor{currentstroke}{rgb}{1.000000,1.000000,1.000000}%
\pgfsetstrokecolor{currentstroke}%
\pgfsetstrokeopacity{0.500000}%
\pgfsetdash{}{0pt}%
\pgfpathmoveto{\pgfqpoint{3.392672in}{3.812310in}}%
\pgfpathlineto{\pgfqpoint{3.403526in}{3.817208in}}%
\pgfpathlineto{\pgfqpoint{3.414373in}{3.821854in}}%
\pgfpathlineto{\pgfqpoint{3.425214in}{3.826204in}}%
\pgfpathlineto{\pgfqpoint{3.436048in}{3.830215in}}%
\pgfpathlineto{\pgfqpoint{3.446873in}{3.833842in}}%
\pgfpathlineto{\pgfqpoint{3.440694in}{3.833613in}}%
\pgfpathlineto{\pgfqpoint{3.434520in}{3.833303in}}%
\pgfpathlineto{\pgfqpoint{3.428352in}{3.832919in}}%
\pgfpathlineto{\pgfqpoint{3.422190in}{3.832468in}}%
\pgfpathlineto{\pgfqpoint{3.416033in}{3.831960in}}%
\pgfpathlineto{\pgfqpoint{3.405234in}{3.828893in}}%
\pgfpathlineto{\pgfqpoint{3.394426in}{3.825455in}}%
\pgfpathlineto{\pgfqpoint{3.383611in}{3.821681in}}%
\pgfpathlineto{\pgfqpoint{3.372789in}{3.817604in}}%
\pgfpathlineto{\pgfqpoint{3.361961in}{3.813258in}}%
\pgfpathlineto{\pgfqpoint{3.368091in}{3.813178in}}%
\pgfpathlineto{\pgfqpoint{3.374228in}{3.813047in}}%
\pgfpathlineto{\pgfqpoint{3.380370in}{3.812861in}}%
\pgfpathlineto{\pgfqpoint{3.386518in}{3.812617in}}%
\pgfpathclose%
\pgfusepath{stroke,fill}%
\end{pgfscope}%
\begin{pgfscope}%
\pgfpathrectangle{\pgfqpoint{0.887500in}{0.275000in}}{\pgfqpoint{4.225000in}{4.225000in}}%
\pgfusepath{clip}%
\pgfsetbuttcap%
\pgfsetroundjoin%
\definecolor{currentfill}{rgb}{0.855810,0.888601,0.097452}%
\pgfsetfillcolor{currentfill}%
\pgfsetfillopacity{0.700000}%
\pgfsetlinewidth{0.501875pt}%
\definecolor{currentstroke}{rgb}{1.000000,1.000000,1.000000}%
\pgfsetstrokecolor{currentstroke}%
\pgfsetstrokeopacity{0.500000}%
\pgfsetdash{}{0pt}%
\pgfpathmoveto{\pgfqpoint{3.531982in}{3.849493in}}%
\pgfpathlineto{\pgfqpoint{3.542781in}{3.851661in}}%
\pgfpathlineto{\pgfqpoint{3.553571in}{3.853481in}}%
\pgfpathlineto{\pgfqpoint{3.564351in}{3.854954in}}%
\pgfpathlineto{\pgfqpoint{3.575120in}{3.856077in}}%
\pgfpathlineto{\pgfqpoint{3.585879in}{3.856850in}}%
\pgfpathlineto{\pgfqpoint{3.579604in}{3.854642in}}%
\pgfpathlineto{\pgfqpoint{3.573335in}{3.852397in}}%
\pgfpathlineto{\pgfqpoint{3.567073in}{3.850170in}}%
\pgfpathlineto{\pgfqpoint{3.560819in}{3.848017in}}%
\pgfpathlineto{\pgfqpoint{3.554573in}{3.845992in}}%
\pgfpathlineto{\pgfqpoint{3.543854in}{3.846777in}}%
\pgfpathlineto{\pgfqpoint{3.533122in}{3.847116in}}%
\pgfpathlineto{\pgfqpoint{3.522379in}{3.847009in}}%
\pgfpathlineto{\pgfqpoint{3.511624in}{3.846456in}}%
\pgfpathlineto{\pgfqpoint{3.500858in}{3.845459in}}%
\pgfpathlineto{\pgfqpoint{3.507071in}{3.846386in}}%
\pgfpathlineto{\pgfqpoint{3.513290in}{3.847277in}}%
\pgfpathlineto{\pgfqpoint{3.519515in}{3.848108in}}%
\pgfpathlineto{\pgfqpoint{3.525746in}{3.848854in}}%
\pgfpathclose%
\pgfusepath{stroke,fill}%
\end{pgfscope}%
\begin{pgfscope}%
\pgfpathrectangle{\pgfqpoint{0.887500in}{0.275000in}}{\pgfqpoint{4.225000in}{4.225000in}}%
\pgfusepath{clip}%
\pgfsetbuttcap%
\pgfsetroundjoin%
\definecolor{currentfill}{rgb}{0.720391,0.870350,0.162603}%
\pgfsetfillcolor{currentfill}%
\pgfsetfillopacity{0.700000}%
\pgfsetlinewidth{0.501875pt}%
\definecolor{currentstroke}{rgb}{1.000000,1.000000,1.000000}%
\pgfsetstrokecolor{currentstroke}%
\pgfsetstrokeopacity{0.500000}%
\pgfsetdash{}{0pt}%
\pgfpathmoveto{\pgfqpoint{3.198955in}{3.732289in}}%
\pgfpathlineto{\pgfqpoint{3.209851in}{3.738146in}}%
\pgfpathlineto{\pgfqpoint{3.220742in}{3.743954in}}%
\pgfpathlineto{\pgfqpoint{3.231630in}{3.749711in}}%
\pgfpathlineto{\pgfqpoint{3.242514in}{3.755421in}}%
\pgfpathlineto{\pgfqpoint{3.253394in}{3.761084in}}%
\pgfpathlineto{\pgfqpoint{3.247317in}{3.761905in}}%
\pgfpathlineto{\pgfqpoint{3.241245in}{3.762680in}}%
\pgfpathlineto{\pgfqpoint{3.235179in}{3.763405in}}%
\pgfpathlineto{\pgfqpoint{3.229119in}{3.764078in}}%
\pgfpathlineto{\pgfqpoint{3.223064in}{3.764696in}}%
\pgfpathlineto{\pgfqpoint{3.212206in}{3.759283in}}%
\pgfpathlineto{\pgfqpoint{3.201344in}{3.753823in}}%
\pgfpathlineto{\pgfqpoint{3.190479in}{3.748310in}}%
\pgfpathlineto{\pgfqpoint{3.179609in}{3.742740in}}%
\pgfpathlineto{\pgfqpoint{3.168736in}{3.737108in}}%
\pgfpathlineto{\pgfqpoint{3.174768in}{3.736240in}}%
\pgfpathlineto{\pgfqpoint{3.180806in}{3.735320in}}%
\pgfpathlineto{\pgfqpoint{3.186850in}{3.734352in}}%
\pgfpathlineto{\pgfqpoint{3.192900in}{3.733340in}}%
\pgfpathclose%
\pgfusepath{stroke,fill}%
\end{pgfscope}%
\begin{pgfscope}%
\pgfpathrectangle{\pgfqpoint{0.887500in}{0.275000in}}{\pgfqpoint{4.225000in}{4.225000in}}%
\pgfusepath{clip}%
\pgfsetbuttcap%
\pgfsetroundjoin%
\definecolor{currentfill}{rgb}{0.616293,0.852709,0.230052}%
\pgfsetfillcolor{currentfill}%
\pgfsetfillopacity{0.700000}%
\pgfsetlinewidth{0.501875pt}%
\definecolor{currentstroke}{rgb}{1.000000,1.000000,1.000000}%
\pgfsetstrokecolor{currentstroke}%
\pgfsetstrokeopacity{0.500000}%
\pgfsetdash{}{0pt}%
\pgfpathmoveto{\pgfqpoint{3.005189in}{3.650995in}}%
\pgfpathlineto{\pgfqpoint{3.016118in}{3.656426in}}%
\pgfpathlineto{\pgfqpoint{3.027044in}{3.661945in}}%
\pgfpathlineto{\pgfqpoint{3.037965in}{3.667547in}}%
\pgfpathlineto{\pgfqpoint{3.048882in}{3.673222in}}%
\pgfpathlineto{\pgfqpoint{3.059796in}{3.678957in}}%
\pgfpathlineto{\pgfqpoint{3.053813in}{3.680245in}}%
\pgfpathlineto{\pgfqpoint{3.047835in}{3.681499in}}%
\pgfpathlineto{\pgfqpoint{3.041862in}{3.682711in}}%
\pgfpathlineto{\pgfqpoint{3.035896in}{3.683877in}}%
\pgfpathlineto{\pgfqpoint{3.024998in}{3.678362in}}%
\pgfpathlineto{\pgfqpoint{3.014097in}{3.672902in}}%
\pgfpathlineto{\pgfqpoint{3.003192in}{3.667504in}}%
\pgfpathlineto{\pgfqpoint{2.992284in}{3.662175in}}%
\pgfpathlineto{\pgfqpoint{2.981371in}{3.656917in}}%
\pgfpathlineto{\pgfqpoint{2.987317in}{3.655467in}}%
\pgfpathlineto{\pgfqpoint{2.993269in}{3.653993in}}%
\pgfpathlineto{\pgfqpoint{2.999226in}{3.652500in}}%
\pgfpathclose%
\pgfusepath{stroke,fill}%
\end{pgfscope}%
\begin{pgfscope}%
\pgfpathrectangle{\pgfqpoint{0.887500in}{0.275000in}}{\pgfqpoint{4.225000in}{4.225000in}}%
\pgfusepath{clip}%
\pgfsetbuttcap%
\pgfsetroundjoin%
\definecolor{currentfill}{rgb}{0.804182,0.882046,0.114965}%
\pgfsetfillcolor{currentfill}%
\pgfsetfillopacity{0.700000}%
\pgfsetlinewidth{0.501875pt}%
\definecolor{currentstroke}{rgb}{1.000000,1.000000,1.000000}%
\pgfsetstrokecolor{currentstroke}%
\pgfsetstrokeopacity{0.500000}%
\pgfsetdash{}{0pt}%
\pgfpathmoveto{\pgfqpoint{3.338323in}{3.785387in}}%
\pgfpathlineto{\pgfqpoint{3.349202in}{3.791006in}}%
\pgfpathlineto{\pgfqpoint{3.360077in}{3.796527in}}%
\pgfpathlineto{\pgfqpoint{3.370947in}{3.801932in}}%
\pgfpathlineto{\pgfqpoint{3.381812in}{3.807203in}}%
\pgfpathlineto{\pgfqpoint{3.392672in}{3.812310in}}%
\pgfpathlineto{\pgfqpoint{3.386518in}{3.812617in}}%
\pgfpathlineto{\pgfqpoint{3.380370in}{3.812861in}}%
\pgfpathlineto{\pgfqpoint{3.374228in}{3.813047in}}%
\pgfpathlineto{\pgfqpoint{3.368091in}{3.813178in}}%
\pgfpathlineto{\pgfqpoint{3.361961in}{3.813258in}}%
\pgfpathlineto{\pgfqpoint{3.351126in}{3.808677in}}%
\pgfpathlineto{\pgfqpoint{3.340286in}{3.803896in}}%
\pgfpathlineto{\pgfqpoint{3.329441in}{3.798941in}}%
\pgfpathlineto{\pgfqpoint{3.318590in}{3.793833in}}%
\pgfpathlineto{\pgfqpoint{3.307735in}{3.788592in}}%
\pgfpathlineto{\pgfqpoint{3.313841in}{3.788025in}}%
\pgfpathlineto{\pgfqpoint{3.319953in}{3.787422in}}%
\pgfpathlineto{\pgfqpoint{3.326071in}{3.786782in}}%
\pgfpathlineto{\pgfqpoint{3.332194in}{3.786104in}}%
\pgfpathclose%
\pgfusepath{stroke,fill}%
\end{pgfscope}%
\begin{pgfscope}%
\pgfpathrectangle{\pgfqpoint{0.887500in}{0.275000in}}{\pgfqpoint{4.225000in}{4.225000in}}%
\pgfusepath{clip}%
\pgfsetbuttcap%
\pgfsetroundjoin%
\definecolor{currentfill}{rgb}{0.699415,0.867117,0.175971}%
\pgfsetfillcolor{currentfill}%
\pgfsetfillopacity{0.700000}%
\pgfsetlinewidth{0.501875pt}%
\definecolor{currentstroke}{rgb}{1.000000,1.000000,1.000000}%
\pgfsetstrokecolor{currentstroke}%
\pgfsetstrokeopacity{0.500000}%
\pgfsetdash{}{0pt}%
\pgfpathmoveto{\pgfqpoint{3.144422in}{3.702367in}}%
\pgfpathlineto{\pgfqpoint{3.155336in}{3.708410in}}%
\pgfpathlineto{\pgfqpoint{3.166246in}{3.714432in}}%
\pgfpathlineto{\pgfqpoint{3.177153in}{3.720425in}}%
\pgfpathlineto{\pgfqpoint{3.188056in}{3.726380in}}%
\pgfpathlineto{\pgfqpoint{3.198955in}{3.732289in}}%
\pgfpathlineto{\pgfqpoint{3.192900in}{3.733340in}}%
\pgfpathlineto{\pgfqpoint{3.186850in}{3.734352in}}%
\pgfpathlineto{\pgfqpoint{3.180806in}{3.735320in}}%
\pgfpathlineto{\pgfqpoint{3.174768in}{3.736240in}}%
\pgfpathlineto{\pgfqpoint{3.168736in}{3.737108in}}%
\pgfpathlineto{\pgfqpoint{3.157858in}{3.731410in}}%
\pgfpathlineto{\pgfqpoint{3.146977in}{3.725654in}}%
\pgfpathlineto{\pgfqpoint{3.136092in}{3.719850in}}%
\pgfpathlineto{\pgfqpoint{3.125203in}{3.714011in}}%
\pgfpathlineto{\pgfqpoint{3.114311in}{3.708151in}}%
\pgfpathlineto{\pgfqpoint{3.120322in}{3.707067in}}%
\pgfpathlineto{\pgfqpoint{3.126338in}{3.705942in}}%
\pgfpathlineto{\pgfqpoint{3.132361in}{3.704782in}}%
\pgfpathlineto{\pgfqpoint{3.138389in}{3.703589in}}%
\pgfpathclose%
\pgfusepath{stroke,fill}%
\end{pgfscope}%
\begin{pgfscope}%
\pgfpathrectangle{\pgfqpoint{0.887500in}{0.275000in}}{\pgfqpoint{4.225000in}{4.225000in}}%
\pgfusepath{clip}%
\pgfsetbuttcap%
\pgfsetroundjoin%
\definecolor{currentfill}{rgb}{0.866013,0.889868,0.095953}%
\pgfsetfillcolor{currentfill}%
\pgfsetfillopacity{0.700000}%
\pgfsetlinewidth{0.501875pt}%
\definecolor{currentstroke}{rgb}{1.000000,1.000000,1.000000}%
\pgfsetstrokecolor{currentstroke}%
\pgfsetstrokeopacity{0.500000}%
\pgfsetdash{}{0pt}%
\pgfpathmoveto{\pgfqpoint{3.477849in}{3.833491in}}%
\pgfpathlineto{\pgfqpoint{3.488693in}{3.837376in}}%
\pgfpathlineto{\pgfqpoint{3.499528in}{3.840919in}}%
\pgfpathlineto{\pgfqpoint{3.510355in}{3.844121in}}%
\pgfpathlineto{\pgfqpoint{3.521173in}{3.846979in}}%
\pgfpathlineto{\pgfqpoint{3.531982in}{3.849493in}}%
\pgfpathlineto{\pgfqpoint{3.525746in}{3.848854in}}%
\pgfpathlineto{\pgfqpoint{3.519515in}{3.848108in}}%
\pgfpathlineto{\pgfqpoint{3.513290in}{3.847277in}}%
\pgfpathlineto{\pgfqpoint{3.507071in}{3.846386in}}%
\pgfpathlineto{\pgfqpoint{3.500858in}{3.845459in}}%
\pgfpathlineto{\pgfqpoint{3.490081in}{3.844018in}}%
\pgfpathlineto{\pgfqpoint{3.479294in}{3.842135in}}%
\pgfpathlineto{\pgfqpoint{3.468496in}{3.839811in}}%
\pgfpathlineto{\pgfqpoint{3.457689in}{3.837046in}}%
\pgfpathlineto{\pgfqpoint{3.446873in}{3.833842in}}%
\pgfpathlineto{\pgfqpoint{3.453058in}{3.833982in}}%
\pgfpathlineto{\pgfqpoint{3.459248in}{3.834025in}}%
\pgfpathlineto{\pgfqpoint{3.465443in}{3.833962in}}%
\pgfpathlineto{\pgfqpoint{3.471644in}{3.833787in}}%
\pgfpathclose%
\pgfusepath{stroke,fill}%
\end{pgfscope}%
\begin{pgfscope}%
\pgfpathrectangle{\pgfqpoint{0.887500in}{0.275000in}}{\pgfqpoint{4.225000in}{4.225000in}}%
\pgfusepath{clip}%
\pgfsetbuttcap%
\pgfsetroundjoin%
\definecolor{currentfill}{rgb}{0.595839,0.848717,0.243329}%
\pgfsetfillcolor{currentfill}%
\pgfsetfillopacity{0.700000}%
\pgfsetlinewidth{0.501875pt}%
\definecolor{currentstroke}{rgb}{1.000000,1.000000,1.000000}%
\pgfsetstrokecolor{currentstroke}%
\pgfsetstrokeopacity{0.500000}%
\pgfsetdash{}{0pt}%
\pgfpathmoveto{\pgfqpoint{2.950485in}{3.624146in}}%
\pgfpathlineto{\pgfqpoint{2.961434in}{3.629583in}}%
\pgfpathlineto{\pgfqpoint{2.972378in}{3.634954in}}%
\pgfpathlineto{\pgfqpoint{2.983319in}{3.640291in}}%
\pgfpathlineto{\pgfqpoint{2.994256in}{3.645627in}}%
\pgfpathlineto{\pgfqpoint{3.005189in}{3.650995in}}%
\pgfpathlineto{\pgfqpoint{2.999226in}{3.652500in}}%
\pgfpathlineto{\pgfqpoint{2.993269in}{3.653993in}}%
\pgfpathlineto{\pgfqpoint{2.987317in}{3.655467in}}%
\pgfpathlineto{\pgfqpoint{2.981371in}{3.656917in}}%
\pgfpathlineto{\pgfqpoint{2.970454in}{3.651708in}}%
\pgfpathlineto{\pgfqpoint{2.959533in}{3.646521in}}%
\pgfpathlineto{\pgfqpoint{2.948608in}{3.641329in}}%
\pgfpathlineto{\pgfqpoint{2.937679in}{3.636105in}}%
\pgfpathlineto{\pgfqpoint{2.926747in}{3.630821in}}%
\pgfpathlineto{\pgfqpoint{2.932673in}{3.629113in}}%
\pgfpathlineto{\pgfqpoint{2.938605in}{3.627430in}}%
\pgfpathlineto{\pgfqpoint{2.944542in}{3.625774in}}%
\pgfpathclose%
\pgfusepath{stroke,fill}%
\end{pgfscope}%
\begin{pgfscope}%
\pgfpathrectangle{\pgfqpoint{0.887500in}{0.275000in}}{\pgfqpoint{4.225000in}{4.225000in}}%
\pgfusepath{clip}%
\pgfsetbuttcap%
\pgfsetroundjoin%
\definecolor{currentfill}{rgb}{0.906311,0.894855,0.098125}%
\pgfsetfillcolor{currentfill}%
\pgfsetfillopacity{0.700000}%
\pgfsetlinewidth{0.501875pt}%
\definecolor{currentstroke}{rgb}{1.000000,1.000000,1.000000}%
\pgfsetstrokecolor{currentstroke}%
\pgfsetstrokeopacity{0.500000}%
\pgfsetdash{}{0pt}%
\pgfpathmoveto{\pgfqpoint{3.617330in}{3.865476in}}%
\pgfpathlineto{\pgfqpoint{3.628126in}{3.867897in}}%
\pgfpathlineto{\pgfqpoint{3.638913in}{3.870103in}}%
\pgfpathlineto{\pgfqpoint{3.649693in}{3.872094in}}%
\pgfpathlineto{\pgfqpoint{3.660463in}{3.873867in}}%
\pgfpathlineto{\pgfqpoint{3.654124in}{3.870905in}}%
\pgfpathlineto{\pgfqpoint{3.647788in}{3.867616in}}%
\pgfpathlineto{\pgfqpoint{3.641455in}{3.864060in}}%
\pgfpathlineto{\pgfqpoint{3.635126in}{3.860299in}}%
\pgfpathlineto{\pgfqpoint{3.628804in}{3.856416in}}%
\pgfpathlineto{\pgfqpoint{3.618090in}{3.857056in}}%
\pgfpathlineto{\pgfqpoint{3.607364in}{3.857341in}}%
\pgfpathlineto{\pgfqpoint{3.596627in}{3.857272in}}%
\pgfpathlineto{\pgfqpoint{3.585879in}{3.856850in}}%
\pgfpathlineto{\pgfqpoint{3.592161in}{3.858966in}}%
\pgfpathlineto{\pgfqpoint{3.598447in}{3.860934in}}%
\pgfpathlineto{\pgfqpoint{3.604738in}{3.862702in}}%
\pgfpathlineto{\pgfqpoint{3.611032in}{3.864229in}}%
\pgfpathclose%
\pgfusepath{stroke,fill}%
\end{pgfscope}%
\begin{pgfscope}%
\pgfpathrectangle{\pgfqpoint{0.887500in}{0.275000in}}{\pgfqpoint{4.225000in}{4.225000in}}%
\pgfusepath{clip}%
\pgfsetbuttcap%
\pgfsetroundjoin%
\definecolor{currentfill}{rgb}{0.783315,0.879285,0.125405}%
\pgfsetfillcolor{currentfill}%
\pgfsetfillopacity{0.700000}%
\pgfsetlinewidth{0.501875pt}%
\definecolor{currentstroke}{rgb}{1.000000,1.000000,1.000000}%
\pgfsetstrokecolor{currentstroke}%
\pgfsetstrokeopacity{0.500000}%
\pgfsetdash{}{0pt}%
\pgfpathmoveto{\pgfqpoint{3.283868in}{3.756383in}}%
\pgfpathlineto{\pgfqpoint{3.294766in}{3.762256in}}%
\pgfpathlineto{\pgfqpoint{3.305662in}{3.768104in}}%
\pgfpathlineto{\pgfqpoint{3.316553in}{3.773919in}}%
\pgfpathlineto{\pgfqpoint{3.327440in}{3.779686in}}%
\pgfpathlineto{\pgfqpoint{3.338323in}{3.785387in}}%
\pgfpathlineto{\pgfqpoint{3.332194in}{3.786104in}}%
\pgfpathlineto{\pgfqpoint{3.326071in}{3.786782in}}%
\pgfpathlineto{\pgfqpoint{3.319953in}{3.787422in}}%
\pgfpathlineto{\pgfqpoint{3.313841in}{3.788025in}}%
\pgfpathlineto{\pgfqpoint{3.307735in}{3.788592in}}%
\pgfpathlineto{\pgfqpoint{3.296875in}{3.783239in}}%
\pgfpathlineto{\pgfqpoint{3.286011in}{3.777793in}}%
\pgfpathlineto{\pgfqpoint{3.275143in}{3.772274in}}%
\pgfpathlineto{\pgfqpoint{3.264271in}{3.766701in}}%
\pgfpathlineto{\pgfqpoint{3.253394in}{3.761084in}}%
\pgfpathlineto{\pgfqpoint{3.259478in}{3.760220in}}%
\pgfpathlineto{\pgfqpoint{3.265567in}{3.759315in}}%
\pgfpathlineto{\pgfqpoint{3.271661in}{3.758372in}}%
\pgfpathlineto{\pgfqpoint{3.277762in}{3.757393in}}%
\pgfpathclose%
\pgfusepath{stroke,fill}%
\end{pgfscope}%
\begin{pgfscope}%
\pgfpathrectangle{\pgfqpoint{0.887500in}{0.275000in}}{\pgfqpoint{4.225000in}{4.225000in}}%
\pgfusepath{clip}%
\pgfsetbuttcap%
\pgfsetroundjoin%
\definecolor{currentfill}{rgb}{0.668054,0.861999,0.196293}%
\pgfsetfillcolor{currentfill}%
\pgfsetfillopacity{0.700000}%
\pgfsetlinewidth{0.501875pt}%
\definecolor{currentstroke}{rgb}{1.000000,1.000000,1.000000}%
\pgfsetstrokecolor{currentstroke}%
\pgfsetstrokeopacity{0.500000}%
\pgfsetdash{}{0pt}%
\pgfpathmoveto{\pgfqpoint{3.089800in}{3.672175in}}%
\pgfpathlineto{\pgfqpoint{3.100732in}{3.678167in}}%
\pgfpathlineto{\pgfqpoint{3.111660in}{3.684199in}}%
\pgfpathlineto{\pgfqpoint{3.122584in}{3.690253in}}%
\pgfpathlineto{\pgfqpoint{3.133505in}{3.696312in}}%
\pgfpathlineto{\pgfqpoint{3.144422in}{3.702367in}}%
\pgfpathlineto{\pgfqpoint{3.138389in}{3.703589in}}%
\pgfpathlineto{\pgfqpoint{3.132361in}{3.704782in}}%
\pgfpathlineto{\pgfqpoint{3.126338in}{3.705942in}}%
\pgfpathlineto{\pgfqpoint{3.120322in}{3.707067in}}%
\pgfpathlineto{\pgfqpoint{3.114311in}{3.708151in}}%
\pgfpathlineto{\pgfqpoint{3.103415in}{3.702282in}}%
\pgfpathlineto{\pgfqpoint{3.092516in}{3.696416in}}%
\pgfpathlineto{\pgfqpoint{3.081613in}{3.690565in}}%
\pgfpathlineto{\pgfqpoint{3.070706in}{3.684742in}}%
\pgfpathlineto{\pgfqpoint{3.059796in}{3.678957in}}%
\pgfpathlineto{\pgfqpoint{3.065786in}{3.677638in}}%
\pgfpathlineto{\pgfqpoint{3.071781in}{3.676296in}}%
\pgfpathlineto{\pgfqpoint{3.077782in}{3.674934in}}%
\pgfpathlineto{\pgfqpoint{3.083788in}{3.673558in}}%
\pgfpathclose%
\pgfusepath{stroke,fill}%
\end{pgfscope}%
\begin{pgfscope}%
\pgfpathrectangle{\pgfqpoint{0.887500in}{0.275000in}}{\pgfqpoint{4.225000in}{4.225000in}}%
\pgfusepath{clip}%
\pgfsetbuttcap%
\pgfsetroundjoin%
\definecolor{currentfill}{rgb}{0.575563,0.844566,0.256415}%
\pgfsetfillcolor{currentfill}%
\pgfsetfillopacity{0.700000}%
\pgfsetlinewidth{0.501875pt}%
\definecolor{currentstroke}{rgb}{1.000000,1.000000,1.000000}%
\pgfsetstrokecolor{currentstroke}%
\pgfsetstrokeopacity{0.500000}%
\pgfsetdash{}{0pt}%
\pgfpathmoveto{\pgfqpoint{2.895689in}{3.595933in}}%
\pgfpathlineto{\pgfqpoint{2.906656in}{3.601564in}}%
\pgfpathlineto{\pgfqpoint{2.917619in}{3.607262in}}%
\pgfpathlineto{\pgfqpoint{2.928578in}{3.612965in}}%
\pgfpathlineto{\pgfqpoint{2.939533in}{3.618611in}}%
\pgfpathlineto{\pgfqpoint{2.950485in}{3.624146in}}%
\pgfpathlineto{\pgfqpoint{2.944542in}{3.625774in}}%
\pgfpathlineto{\pgfqpoint{2.938605in}{3.627430in}}%
\pgfpathlineto{\pgfqpoint{2.932673in}{3.629113in}}%
\pgfpathlineto{\pgfqpoint{2.926747in}{3.630821in}}%
\pgfpathlineto{\pgfqpoint{2.915810in}{3.625450in}}%
\pgfpathlineto{\pgfqpoint{2.904871in}{3.619978in}}%
\pgfpathlineto{\pgfqpoint{2.893927in}{3.614431in}}%
\pgfpathlineto{\pgfqpoint{2.882981in}{3.608843in}}%
\pgfpathlineto{\pgfqpoint{2.872030in}{3.603247in}}%
\pgfpathlineto{\pgfqpoint{2.877937in}{3.601342in}}%
\pgfpathlineto{\pgfqpoint{2.883849in}{3.599488in}}%
\pgfpathlineto{\pgfqpoint{2.889766in}{3.597686in}}%
\pgfpathclose%
\pgfusepath{stroke,fill}%
\end{pgfscope}%
\begin{pgfscope}%
\pgfpathrectangle{\pgfqpoint{0.887500in}{0.275000in}}{\pgfqpoint{4.225000in}{4.225000in}}%
\pgfusepath{clip}%
\pgfsetbuttcap%
\pgfsetroundjoin%
\definecolor{currentfill}{rgb}{0.866013,0.889868,0.095953}%
\pgfsetfillcolor{currentfill}%
\pgfsetfillopacity{0.700000}%
\pgfsetlinewidth{0.501875pt}%
\definecolor{currentstroke}{rgb}{1.000000,1.000000,1.000000}%
\pgfsetstrokecolor{currentstroke}%
\pgfsetstrokeopacity{0.500000}%
\pgfsetdash{}{0pt}%
\pgfpathmoveto{\pgfqpoint{3.423522in}{3.809683in}}%
\pgfpathlineto{\pgfqpoint{3.434400in}{3.814933in}}%
\pgfpathlineto{\pgfqpoint{3.445273in}{3.819968in}}%
\pgfpathlineto{\pgfqpoint{3.456139in}{3.824759in}}%
\pgfpathlineto{\pgfqpoint{3.466998in}{3.829276in}}%
\pgfpathlineto{\pgfqpoint{3.477849in}{3.833491in}}%
\pgfpathlineto{\pgfqpoint{3.471644in}{3.833787in}}%
\pgfpathlineto{\pgfqpoint{3.465443in}{3.833962in}}%
\pgfpathlineto{\pgfqpoint{3.459248in}{3.834025in}}%
\pgfpathlineto{\pgfqpoint{3.453058in}{3.833982in}}%
\pgfpathlineto{\pgfqpoint{3.446873in}{3.833842in}}%
\pgfpathlineto{\pgfqpoint{3.436048in}{3.830215in}}%
\pgfpathlineto{\pgfqpoint{3.425214in}{3.826204in}}%
\pgfpathlineto{\pgfqpoint{3.414373in}{3.821854in}}%
\pgfpathlineto{\pgfqpoint{3.403526in}{3.817208in}}%
\pgfpathlineto{\pgfqpoint{3.392672in}{3.812310in}}%
\pgfpathlineto{\pgfqpoint{3.398831in}{3.811935in}}%
\pgfpathlineto{\pgfqpoint{3.404995in}{3.811490in}}%
\pgfpathlineto{\pgfqpoint{3.411166in}{3.810969in}}%
\pgfpathlineto{\pgfqpoint{3.417341in}{3.810368in}}%
\pgfpathclose%
\pgfusepath{stroke,fill}%
\end{pgfscope}%
\begin{pgfscope}%
\pgfpathrectangle{\pgfqpoint{0.887500in}{0.275000in}}{\pgfqpoint{4.225000in}{4.225000in}}%
\pgfusepath{clip}%
\pgfsetbuttcap%
\pgfsetroundjoin%
\definecolor{currentfill}{rgb}{0.751884,0.874951,0.143228}%
\pgfsetfillcolor{currentfill}%
\pgfsetfillopacity{0.700000}%
\pgfsetlinewidth{0.501875pt}%
\definecolor{currentstroke}{rgb}{1.000000,1.000000,1.000000}%
\pgfsetstrokecolor{currentstroke}%
\pgfsetstrokeopacity{0.500000}%
\pgfsetdash{}{0pt}%
\pgfpathmoveto{\pgfqpoint{3.229318in}{3.726560in}}%
\pgfpathlineto{\pgfqpoint{3.240235in}{3.732592in}}%
\pgfpathlineto{\pgfqpoint{3.251149in}{3.738588in}}%
\pgfpathlineto{\pgfqpoint{3.262059in}{3.744551in}}%
\pgfpathlineto{\pgfqpoint{3.272965in}{3.750482in}}%
\pgfpathlineto{\pgfqpoint{3.283868in}{3.756383in}}%
\pgfpathlineto{\pgfqpoint{3.277762in}{3.757393in}}%
\pgfpathlineto{\pgfqpoint{3.271661in}{3.758372in}}%
\pgfpathlineto{\pgfqpoint{3.265567in}{3.759315in}}%
\pgfpathlineto{\pgfqpoint{3.259478in}{3.760220in}}%
\pgfpathlineto{\pgfqpoint{3.253394in}{3.761084in}}%
\pgfpathlineto{\pgfqpoint{3.242514in}{3.755421in}}%
\pgfpathlineto{\pgfqpoint{3.231630in}{3.749711in}}%
\pgfpathlineto{\pgfqpoint{3.220742in}{3.743954in}}%
\pgfpathlineto{\pgfqpoint{3.209851in}{3.738146in}}%
\pgfpathlineto{\pgfqpoint{3.198955in}{3.732289in}}%
\pgfpathlineto{\pgfqpoint{3.205016in}{3.731201in}}%
\pgfpathlineto{\pgfqpoint{3.211083in}{3.730080in}}%
\pgfpathlineto{\pgfqpoint{3.217155in}{3.728930in}}%
\pgfpathlineto{\pgfqpoint{3.223234in}{3.727756in}}%
\pgfpathclose%
\pgfusepath{stroke,fill}%
\end{pgfscope}%
\begin{pgfscope}%
\pgfpathrectangle{\pgfqpoint{0.887500in}{0.275000in}}{\pgfqpoint{4.225000in}{4.225000in}}%
\pgfusepath{clip}%
\pgfsetbuttcap%
\pgfsetroundjoin%
\definecolor{currentfill}{rgb}{0.647257,0.858400,0.209861}%
\pgfsetfillcolor{currentfill}%
\pgfsetfillopacity{0.700000}%
\pgfsetlinewidth{0.501875pt}%
\definecolor{currentstroke}{rgb}{1.000000,1.000000,1.000000}%
\pgfsetstrokecolor{currentstroke}%
\pgfsetstrokeopacity{0.500000}%
\pgfsetdash{}{0pt}%
\pgfpathmoveto{\pgfqpoint{3.035090in}{3.643480in}}%
\pgfpathlineto{\pgfqpoint{3.046039in}{3.648994in}}%
\pgfpathlineto{\pgfqpoint{3.056985in}{3.654633in}}%
\pgfpathlineto{\pgfqpoint{3.067927in}{3.660389in}}%
\pgfpathlineto{\pgfqpoint{3.078866in}{3.666243in}}%
\pgfpathlineto{\pgfqpoint{3.089800in}{3.672175in}}%
\pgfpathlineto{\pgfqpoint{3.083788in}{3.673558in}}%
\pgfpathlineto{\pgfqpoint{3.077782in}{3.674934in}}%
\pgfpathlineto{\pgfqpoint{3.071781in}{3.676296in}}%
\pgfpathlineto{\pgfqpoint{3.065786in}{3.677638in}}%
\pgfpathlineto{\pgfqpoint{3.059796in}{3.678957in}}%
\pgfpathlineto{\pgfqpoint{3.048882in}{3.673222in}}%
\pgfpathlineto{\pgfqpoint{3.037965in}{3.667547in}}%
\pgfpathlineto{\pgfqpoint{3.027044in}{3.661945in}}%
\pgfpathlineto{\pgfqpoint{3.016118in}{3.656426in}}%
\pgfpathlineto{\pgfqpoint{3.005189in}{3.650995in}}%
\pgfpathlineto{\pgfqpoint{3.011158in}{3.649482in}}%
\pgfpathlineto{\pgfqpoint{3.017133in}{3.647969in}}%
\pgfpathlineto{\pgfqpoint{3.023113in}{3.646460in}}%
\pgfpathlineto{\pgfqpoint{3.029098in}{3.644962in}}%
\pgfpathclose%
\pgfusepath{stroke,fill}%
\end{pgfscope}%
\begin{pgfscope}%
\pgfpathrectangle{\pgfqpoint{0.887500in}{0.275000in}}{\pgfqpoint{4.225000in}{4.225000in}}%
\pgfusepath{clip}%
\pgfsetbuttcap%
\pgfsetroundjoin%
\definecolor{currentfill}{rgb}{0.906311,0.894855,0.098125}%
\pgfsetfillcolor{currentfill}%
\pgfsetfillopacity{0.700000}%
\pgfsetlinewidth{0.501875pt}%
\definecolor{currentstroke}{rgb}{1.000000,1.000000,1.000000}%
\pgfsetstrokecolor{currentstroke}%
\pgfsetstrokeopacity{0.500000}%
\pgfsetdash{}{0pt}%
\pgfpathmoveto{\pgfqpoint{3.563229in}{3.850222in}}%
\pgfpathlineto{\pgfqpoint{3.574065in}{3.853687in}}%
\pgfpathlineto{\pgfqpoint{3.584893in}{3.856947in}}%
\pgfpathlineto{\pgfqpoint{3.595713in}{3.859999in}}%
\pgfpathlineto{\pgfqpoint{3.606525in}{3.862843in}}%
\pgfpathlineto{\pgfqpoint{3.617330in}{3.865476in}}%
\pgfpathlineto{\pgfqpoint{3.611032in}{3.864229in}}%
\pgfpathlineto{\pgfqpoint{3.604738in}{3.862702in}}%
\pgfpathlineto{\pgfqpoint{3.598447in}{3.860934in}}%
\pgfpathlineto{\pgfqpoint{3.592161in}{3.858966in}}%
\pgfpathlineto{\pgfqpoint{3.585879in}{3.856850in}}%
\pgfpathlineto{\pgfqpoint{3.575120in}{3.856077in}}%
\pgfpathlineto{\pgfqpoint{3.564351in}{3.854954in}}%
\pgfpathlineto{\pgfqpoint{3.553571in}{3.853481in}}%
\pgfpathlineto{\pgfqpoint{3.542781in}{3.851661in}}%
\pgfpathlineto{\pgfqpoint{3.531982in}{3.849493in}}%
\pgfpathlineto{\pgfqpoint{3.538223in}{3.849999in}}%
\pgfpathlineto{\pgfqpoint{3.544469in}{3.850348in}}%
\pgfpathlineto{\pgfqpoint{3.550718in}{3.850517in}}%
\pgfpathlineto{\pgfqpoint{3.556972in}{3.850483in}}%
\pgfpathclose%
\pgfusepath{stroke,fill}%
\end{pgfscope}%
\begin{pgfscope}%
\pgfpathrectangle{\pgfqpoint{0.887500in}{0.275000in}}{\pgfqpoint{4.225000in}{4.225000in}}%
\pgfusepath{clip}%
\pgfsetbuttcap%
\pgfsetroundjoin%
\definecolor{currentfill}{rgb}{0.555484,0.840254,0.269281}%
\pgfsetfillcolor{currentfill}%
\pgfsetfillopacity{0.700000}%
\pgfsetlinewidth{0.501875pt}%
\definecolor{currentstroke}{rgb}{1.000000,1.000000,1.000000}%
\pgfsetstrokecolor{currentstroke}%
\pgfsetstrokeopacity{0.500000}%
\pgfsetdash{}{0pt}%
\pgfpathmoveto{\pgfqpoint{2.840784in}{3.570406in}}%
\pgfpathlineto{\pgfqpoint{2.851775in}{3.575249in}}%
\pgfpathlineto{\pgfqpoint{2.862761in}{3.580079in}}%
\pgfpathlineto{\pgfqpoint{2.873742in}{3.585128in}}%
\pgfpathlineto{\pgfqpoint{2.884718in}{3.590434in}}%
\pgfpathlineto{\pgfqpoint{2.895689in}{3.595933in}}%
\pgfpathlineto{\pgfqpoint{2.889766in}{3.597686in}}%
\pgfpathlineto{\pgfqpoint{2.883849in}{3.599488in}}%
\pgfpathlineto{\pgfqpoint{2.877937in}{3.601342in}}%
\pgfpathlineto{\pgfqpoint{2.872030in}{3.603247in}}%
\pgfpathlineto{\pgfqpoint{2.861076in}{3.597676in}}%
\pgfpathlineto{\pgfqpoint{2.850117in}{3.592163in}}%
\pgfpathlineto{\pgfqpoint{2.839154in}{3.586742in}}%
\pgfpathlineto{\pgfqpoint{2.828187in}{3.581383in}}%
\pgfpathlineto{\pgfqpoint{2.817216in}{3.575933in}}%
\pgfpathlineto{\pgfqpoint{2.823100in}{3.574585in}}%
\pgfpathlineto{\pgfqpoint{2.828989in}{3.573211in}}%
\pgfpathlineto{\pgfqpoint{2.834884in}{3.571816in}}%
\pgfpathclose%
\pgfusepath{stroke,fill}%
\end{pgfscope}%
\begin{pgfscope}%
\pgfpathrectangle{\pgfqpoint{0.887500in}{0.275000in}}{\pgfqpoint{4.225000in}{4.225000in}}%
\pgfusepath{clip}%
\pgfsetbuttcap%
\pgfsetroundjoin%
\definecolor{currentfill}{rgb}{0.845561,0.887322,0.099702}%
\pgfsetfillcolor{currentfill}%
\pgfsetfillopacity{0.700000}%
\pgfsetlinewidth{0.501875pt}%
\definecolor{currentstroke}{rgb}{1.000000,1.000000,1.000000}%
\pgfsetstrokecolor{currentstroke}%
\pgfsetstrokeopacity{0.500000}%
\pgfsetdash{}{0pt}%
\pgfpathmoveto{\pgfqpoint{3.369055in}{3.781180in}}%
\pgfpathlineto{\pgfqpoint{3.379958in}{3.787102in}}%
\pgfpathlineto{\pgfqpoint{3.390856in}{3.792934in}}%
\pgfpathlineto{\pgfqpoint{3.401750in}{3.798656in}}%
\pgfpathlineto{\pgfqpoint{3.412638in}{3.804248in}}%
\pgfpathlineto{\pgfqpoint{3.423522in}{3.809683in}}%
\pgfpathlineto{\pgfqpoint{3.417341in}{3.810368in}}%
\pgfpathlineto{\pgfqpoint{3.411166in}{3.810969in}}%
\pgfpathlineto{\pgfqpoint{3.404995in}{3.811490in}}%
\pgfpathlineto{\pgfqpoint{3.398831in}{3.811935in}}%
\pgfpathlineto{\pgfqpoint{3.392672in}{3.812310in}}%
\pgfpathlineto{\pgfqpoint{3.381812in}{3.807203in}}%
\pgfpathlineto{\pgfqpoint{3.370947in}{3.801932in}}%
\pgfpathlineto{\pgfqpoint{3.360077in}{3.796527in}}%
\pgfpathlineto{\pgfqpoint{3.349202in}{3.791006in}}%
\pgfpathlineto{\pgfqpoint{3.338323in}{3.785387in}}%
\pgfpathlineto{\pgfqpoint{3.344458in}{3.784630in}}%
\pgfpathlineto{\pgfqpoint{3.350599in}{3.783832in}}%
\pgfpathlineto{\pgfqpoint{3.356745in}{3.782992in}}%
\pgfpathlineto{\pgfqpoint{3.362897in}{3.782108in}}%
\pgfpathclose%
\pgfusepath{stroke,fill}%
\end{pgfscope}%
\begin{pgfscope}%
\pgfpathrectangle{\pgfqpoint{0.887500in}{0.275000in}}{\pgfqpoint{4.225000in}{4.225000in}}%
\pgfusepath{clip}%
\pgfsetbuttcap%
\pgfsetroundjoin%
\definecolor{currentfill}{rgb}{0.730889,0.871916,0.156029}%
\pgfsetfillcolor{currentfill}%
\pgfsetfillopacity{0.700000}%
\pgfsetlinewidth{0.501875pt}%
\definecolor{currentstroke}{rgb}{1.000000,1.000000,1.000000}%
\pgfsetstrokecolor{currentstroke}%
\pgfsetstrokeopacity{0.500000}%
\pgfsetdash{}{0pt}%
\pgfpathmoveto{\pgfqpoint{3.174677in}{3.695957in}}%
\pgfpathlineto{\pgfqpoint{3.185612in}{3.702114in}}%
\pgfpathlineto{\pgfqpoint{3.196544in}{3.708260in}}%
\pgfpathlineto{\pgfqpoint{3.207472in}{3.714388in}}%
\pgfpathlineto{\pgfqpoint{3.218397in}{3.720491in}}%
\pgfpathlineto{\pgfqpoint{3.229318in}{3.726560in}}%
\pgfpathlineto{\pgfqpoint{3.223234in}{3.727756in}}%
\pgfpathlineto{\pgfqpoint{3.217155in}{3.728930in}}%
\pgfpathlineto{\pgfqpoint{3.211083in}{3.730080in}}%
\pgfpathlineto{\pgfqpoint{3.205016in}{3.731201in}}%
\pgfpathlineto{\pgfqpoint{3.198955in}{3.732289in}}%
\pgfpathlineto{\pgfqpoint{3.188056in}{3.726380in}}%
\pgfpathlineto{\pgfqpoint{3.177153in}{3.720425in}}%
\pgfpathlineto{\pgfqpoint{3.166246in}{3.714432in}}%
\pgfpathlineto{\pgfqpoint{3.155336in}{3.708410in}}%
\pgfpathlineto{\pgfqpoint{3.144422in}{3.702367in}}%
\pgfpathlineto{\pgfqpoint{3.150462in}{3.701121in}}%
\pgfpathlineto{\pgfqpoint{3.156507in}{3.699853in}}%
\pgfpathlineto{\pgfqpoint{3.162558in}{3.698567in}}%
\pgfpathlineto{\pgfqpoint{3.168614in}{3.697267in}}%
\pgfpathclose%
\pgfusepath{stroke,fill}%
\end{pgfscope}%
\begin{pgfscope}%
\pgfpathrectangle{\pgfqpoint{0.887500in}{0.275000in}}{\pgfqpoint{4.225000in}{4.225000in}}%
\pgfusepath{clip}%
\pgfsetbuttcap%
\pgfsetroundjoin%
\definecolor{currentfill}{rgb}{0.616293,0.852709,0.230052}%
\pgfsetfillcolor{currentfill}%
\pgfsetfillopacity{0.700000}%
\pgfsetlinewidth{0.501875pt}%
\definecolor{currentstroke}{rgb}{1.000000,1.000000,1.000000}%
\pgfsetstrokecolor{currentstroke}%
\pgfsetstrokeopacity{0.500000}%
\pgfsetdash{}{0pt}%
\pgfpathmoveto{\pgfqpoint{2.980283in}{3.616468in}}%
\pgfpathlineto{\pgfqpoint{2.991252in}{3.621941in}}%
\pgfpathlineto{\pgfqpoint{3.002217in}{3.627335in}}%
\pgfpathlineto{\pgfqpoint{3.013179in}{3.632694in}}%
\pgfpathlineto{\pgfqpoint{3.024136in}{3.638062in}}%
\pgfpathlineto{\pgfqpoint{3.035090in}{3.643480in}}%
\pgfpathlineto{\pgfqpoint{3.029098in}{3.644962in}}%
\pgfpathlineto{\pgfqpoint{3.023113in}{3.646460in}}%
\pgfpathlineto{\pgfqpoint{3.017133in}{3.647969in}}%
\pgfpathlineto{\pgfqpoint{3.011158in}{3.649482in}}%
\pgfpathlineto{\pgfqpoint{3.005189in}{3.650995in}}%
\pgfpathlineto{\pgfqpoint{2.994256in}{3.645627in}}%
\pgfpathlineto{\pgfqpoint{2.983319in}{3.640291in}}%
\pgfpathlineto{\pgfqpoint{2.972378in}{3.634954in}}%
\pgfpathlineto{\pgfqpoint{2.961434in}{3.629583in}}%
\pgfpathlineto{\pgfqpoint{2.950485in}{3.624146in}}%
\pgfpathlineto{\pgfqpoint{2.956434in}{3.622548in}}%
\pgfpathlineto{\pgfqpoint{2.962388in}{3.620979in}}%
\pgfpathlineto{\pgfqpoint{2.968347in}{3.619443in}}%
\pgfpathlineto{\pgfqpoint{2.974312in}{3.617939in}}%
\pgfpathclose%
\pgfusepath{stroke,fill}%
\end{pgfscope}%
\begin{pgfscope}%
\pgfpathrectangle{\pgfqpoint{0.887500in}{0.275000in}}{\pgfqpoint{4.225000in}{4.225000in}}%
\pgfusepath{clip}%
\pgfsetbuttcap%
\pgfsetroundjoin%
\definecolor{currentfill}{rgb}{0.906311,0.894855,0.098125}%
\pgfsetfillcolor{currentfill}%
\pgfsetfillopacity{0.700000}%
\pgfsetlinewidth{0.501875pt}%
\definecolor{currentstroke}{rgb}{1.000000,1.000000,1.000000}%
\pgfsetstrokecolor{currentstroke}%
\pgfsetstrokeopacity{0.500000}%
\pgfsetdash{}{0pt}%
\pgfpathmoveto{\pgfqpoint{3.508947in}{3.829877in}}%
\pgfpathlineto{\pgfqpoint{3.519817in}{3.834342in}}%
\pgfpathlineto{\pgfqpoint{3.530680in}{3.838611in}}%
\pgfpathlineto{\pgfqpoint{3.541537in}{3.842682in}}%
\pgfpathlineto{\pgfqpoint{3.552387in}{3.846553in}}%
\pgfpathlineto{\pgfqpoint{3.563229in}{3.850222in}}%
\pgfpathlineto{\pgfqpoint{3.556972in}{3.850483in}}%
\pgfpathlineto{\pgfqpoint{3.550718in}{3.850517in}}%
\pgfpathlineto{\pgfqpoint{3.544469in}{3.850348in}}%
\pgfpathlineto{\pgfqpoint{3.538223in}{3.849999in}}%
\pgfpathlineto{\pgfqpoint{3.531982in}{3.849493in}}%
\pgfpathlineto{\pgfqpoint{3.521173in}{3.846979in}}%
\pgfpathlineto{\pgfqpoint{3.510355in}{3.844121in}}%
\pgfpathlineto{\pgfqpoint{3.499528in}{3.840919in}}%
\pgfpathlineto{\pgfqpoint{3.488693in}{3.837376in}}%
\pgfpathlineto{\pgfqpoint{3.477849in}{3.833491in}}%
\pgfpathlineto{\pgfqpoint{3.484060in}{3.833066in}}%
\pgfpathlineto{\pgfqpoint{3.490275in}{3.832505in}}%
\pgfpathlineto{\pgfqpoint{3.496494in}{3.831797in}}%
\pgfpathlineto{\pgfqpoint{3.502718in}{3.830927in}}%
\pgfpathclose%
\pgfusepath{stroke,fill}%
\end{pgfscope}%
\begin{pgfscope}%
\pgfpathrectangle{\pgfqpoint{0.887500in}{0.275000in}}{\pgfqpoint{4.225000in}{4.225000in}}%
\pgfusepath{clip}%
\pgfsetbuttcap%
\pgfsetroundjoin%
\definecolor{currentfill}{rgb}{0.525776,0.833491,0.288127}%
\pgfsetfillcolor{currentfill}%
\pgfsetfillopacity{0.700000}%
\pgfsetlinewidth{0.501875pt}%
\definecolor{currentstroke}{rgb}{1.000000,1.000000,1.000000}%
\pgfsetstrokecolor{currentstroke}%
\pgfsetstrokeopacity{0.500000}%
\pgfsetdash{}{0pt}%
\pgfpathmoveto{\pgfqpoint{2.785829in}{3.537028in}}%
\pgfpathlineto{\pgfqpoint{2.796814in}{3.545777in}}%
\pgfpathlineto{\pgfqpoint{2.807804in}{3.553236in}}%
\pgfpathlineto{\pgfqpoint{2.818797in}{3.559656in}}%
\pgfpathlineto{\pgfqpoint{2.829791in}{3.565294in}}%
\pgfpathlineto{\pgfqpoint{2.840784in}{3.570406in}}%
\pgfpathlineto{\pgfqpoint{2.834884in}{3.571816in}}%
\pgfpathlineto{\pgfqpoint{2.828989in}{3.573211in}}%
\pgfpathlineto{\pgfqpoint{2.823100in}{3.574585in}}%
\pgfpathlineto{\pgfqpoint{2.817216in}{3.575933in}}%
\pgfpathlineto{\pgfqpoint{2.806243in}{3.570224in}}%
\pgfpathlineto{\pgfqpoint{2.795270in}{3.564086in}}%
\pgfpathlineto{\pgfqpoint{2.784296in}{3.557353in}}%
\pgfpathlineto{\pgfqpoint{2.773325in}{3.549855in}}%
\pgfpathlineto{\pgfqpoint{2.762358in}{3.541429in}}%
\pgfpathlineto{\pgfqpoint{2.768215in}{3.540610in}}%
\pgfpathlineto{\pgfqpoint{2.774080in}{3.539585in}}%
\pgfpathlineto{\pgfqpoint{2.779951in}{3.538382in}}%
\pgfpathclose%
\pgfusepath{stroke,fill}%
\end{pgfscope}%
\begin{pgfscope}%
\pgfpathrectangle{\pgfqpoint{0.887500in}{0.275000in}}{\pgfqpoint{4.225000in}{4.225000in}}%
\pgfusepath{clip}%
\pgfsetbuttcap%
\pgfsetroundjoin%
\definecolor{currentfill}{rgb}{0.814576,0.883393,0.110347}%
\pgfsetfillcolor{currentfill}%
\pgfsetfillopacity{0.700000}%
\pgfsetlinewidth{0.501875pt}%
\definecolor{currentstroke}{rgb}{1.000000,1.000000,1.000000}%
\pgfsetstrokecolor{currentstroke}%
\pgfsetstrokeopacity{0.500000}%
\pgfsetdash{}{0pt}%
\pgfpathmoveto{\pgfqpoint{3.314485in}{3.750945in}}%
\pgfpathlineto{\pgfqpoint{3.325406in}{3.757017in}}%
\pgfpathlineto{\pgfqpoint{3.336324in}{3.763089in}}%
\pgfpathlineto{\pgfqpoint{3.347238in}{3.769153in}}%
\pgfpathlineto{\pgfqpoint{3.358148in}{3.775191in}}%
\pgfpathlineto{\pgfqpoint{3.369055in}{3.781180in}}%
\pgfpathlineto{\pgfqpoint{3.362897in}{3.782108in}}%
\pgfpathlineto{\pgfqpoint{3.356745in}{3.782992in}}%
\pgfpathlineto{\pgfqpoint{3.350599in}{3.783832in}}%
\pgfpathlineto{\pgfqpoint{3.344458in}{3.784630in}}%
\pgfpathlineto{\pgfqpoint{3.338323in}{3.785387in}}%
\pgfpathlineto{\pgfqpoint{3.327440in}{3.779686in}}%
\pgfpathlineto{\pgfqpoint{3.316553in}{3.773919in}}%
\pgfpathlineto{\pgfqpoint{3.305662in}{3.768104in}}%
\pgfpathlineto{\pgfqpoint{3.294766in}{3.762256in}}%
\pgfpathlineto{\pgfqpoint{3.283868in}{3.756383in}}%
\pgfpathlineto{\pgfqpoint{3.289980in}{3.755343in}}%
\pgfpathlineto{\pgfqpoint{3.296097in}{3.754276in}}%
\pgfpathlineto{\pgfqpoint{3.302221in}{3.753186in}}%
\pgfpathlineto{\pgfqpoint{3.308350in}{3.752074in}}%
\pgfpathclose%
\pgfusepath{stroke,fill}%
\end{pgfscope}%
\begin{pgfscope}%
\pgfpathrectangle{\pgfqpoint{0.887500in}{0.275000in}}{\pgfqpoint{4.225000in}{4.225000in}}%
\pgfusepath{clip}%
\pgfsetbuttcap%
\pgfsetroundjoin%
\definecolor{currentfill}{rgb}{0.699415,0.867117,0.175971}%
\pgfsetfillcolor{currentfill}%
\pgfsetfillopacity{0.700000}%
\pgfsetlinewidth{0.501875pt}%
\definecolor{currentstroke}{rgb}{1.000000,1.000000,1.000000}%
\pgfsetstrokecolor{currentstroke}%
\pgfsetstrokeopacity{0.500000}%
\pgfsetdash{}{0pt}%
\pgfpathmoveto{\pgfqpoint{3.119948in}{3.665333in}}%
\pgfpathlineto{\pgfqpoint{3.130901in}{3.671392in}}%
\pgfpathlineto{\pgfqpoint{3.141850in}{3.677502in}}%
\pgfpathlineto{\pgfqpoint{3.152795in}{3.683643in}}%
\pgfpathlineto{\pgfqpoint{3.163738in}{3.689797in}}%
\pgfpathlineto{\pgfqpoint{3.174677in}{3.695957in}}%
\pgfpathlineto{\pgfqpoint{3.168614in}{3.697267in}}%
\pgfpathlineto{\pgfqpoint{3.162558in}{3.698567in}}%
\pgfpathlineto{\pgfqpoint{3.156507in}{3.699853in}}%
\pgfpathlineto{\pgfqpoint{3.150462in}{3.701121in}}%
\pgfpathlineto{\pgfqpoint{3.144422in}{3.702367in}}%
\pgfpathlineto{\pgfqpoint{3.133505in}{3.696312in}}%
\pgfpathlineto{\pgfqpoint{3.122584in}{3.690253in}}%
\pgfpathlineto{\pgfqpoint{3.111660in}{3.684199in}}%
\pgfpathlineto{\pgfqpoint{3.100732in}{3.678167in}}%
\pgfpathlineto{\pgfqpoint{3.089800in}{3.672175in}}%
\pgfpathlineto{\pgfqpoint{3.095818in}{3.670789in}}%
\pgfpathlineto{\pgfqpoint{3.101842in}{3.669407in}}%
\pgfpathlineto{\pgfqpoint{3.107872in}{3.668033in}}%
\pgfpathlineto{\pgfqpoint{3.113907in}{3.666673in}}%
\pgfpathclose%
\pgfusepath{stroke,fill}%
\end{pgfscope}%
\begin{pgfscope}%
\pgfpathrectangle{\pgfqpoint{0.887500in}{0.275000in}}{\pgfqpoint{4.225000in}{4.225000in}}%
\pgfusepath{clip}%
\pgfsetbuttcap%
\pgfsetroundjoin%
\definecolor{currentfill}{rgb}{0.955300,0.901065,0.118128}%
\pgfsetfillcolor{currentfill}%
\pgfsetfillopacity{0.700000}%
\pgfsetlinewidth{0.501875pt}%
\definecolor{currentstroke}{rgb}{1.000000,1.000000,1.000000}%
\pgfsetstrokecolor{currentstroke}%
\pgfsetstrokeopacity{0.500000}%
\pgfsetdash{}{0pt}%
\pgfpathmoveto{\pgfqpoint{3.648838in}{3.866170in}}%
\pgfpathlineto{\pgfqpoint{3.659677in}{3.870204in}}%
\pgfpathlineto{\pgfqpoint{3.670511in}{3.874148in}}%
\pgfpathlineto{\pgfqpoint{3.681339in}{3.877999in}}%
\pgfpathlineto{\pgfqpoint{3.692160in}{3.881757in}}%
\pgfpathlineto{\pgfqpoint{3.685824in}{3.881296in}}%
\pgfpathlineto{\pgfqpoint{3.679485in}{3.880218in}}%
\pgfpathlineto{\pgfqpoint{3.673145in}{3.878582in}}%
\pgfpathlineto{\pgfqpoint{3.666804in}{3.876445in}}%
\pgfpathlineto{\pgfqpoint{3.660463in}{3.873867in}}%
\pgfpathlineto{\pgfqpoint{3.649693in}{3.872094in}}%
\pgfpathlineto{\pgfqpoint{3.638913in}{3.870103in}}%
\pgfpathlineto{\pgfqpoint{3.628126in}{3.867897in}}%
\pgfpathlineto{\pgfqpoint{3.617330in}{3.865476in}}%
\pgfpathlineto{\pgfqpoint{3.623629in}{3.866405in}}%
\pgfpathlineto{\pgfqpoint{3.629931in}{3.866978in}}%
\pgfpathlineto{\pgfqpoint{3.636233in}{3.867156in}}%
\pgfpathlineto{\pgfqpoint{3.642536in}{3.866899in}}%
\pgfpathclose%
\pgfusepath{stroke,fill}%
\end{pgfscope}%
\begin{pgfscope}%
\pgfpathrectangle{\pgfqpoint{0.887500in}{0.275000in}}{\pgfqpoint{4.225000in}{4.225000in}}%
\pgfusepath{clip}%
\pgfsetbuttcap%
\pgfsetroundjoin%
\definecolor{currentfill}{rgb}{0.259857,0.745492,0.444467}%
\pgfsetfillcolor{currentfill}%
\pgfsetfillopacity{0.700000}%
\pgfsetlinewidth{0.501875pt}%
\definecolor{currentstroke}{rgb}{1.000000,1.000000,1.000000}%
\pgfsetstrokecolor{currentstroke}%
\pgfsetstrokeopacity{0.500000}%
\pgfsetdash{}{0pt}%
\pgfpathmoveto{\pgfqpoint{2.566888in}{3.264073in}}%
\pgfpathlineto{\pgfqpoint{2.577855in}{3.275621in}}%
\pgfpathlineto{\pgfqpoint{2.588798in}{3.289177in}}%
\pgfpathlineto{\pgfqpoint{2.599724in}{3.304223in}}%
\pgfpathlineto{\pgfqpoint{2.610641in}{3.320240in}}%
\pgfpathlineto{\pgfqpoint{2.621556in}{3.336709in}}%
\pgfpathlineto{\pgfqpoint{2.615684in}{3.344903in}}%
\pgfpathlineto{\pgfqpoint{2.609811in}{3.353519in}}%
\pgfpathlineto{\pgfqpoint{2.603938in}{3.362505in}}%
\pgfpathlineto{\pgfqpoint{2.598064in}{3.371808in}}%
\pgfpathlineto{\pgfqpoint{2.587119in}{3.359346in}}%
\pgfpathlineto{\pgfqpoint{2.576170in}{3.347197in}}%
\pgfpathlineto{\pgfqpoint{2.565213in}{3.335632in}}%
\pgfpathlineto{\pgfqpoint{2.554246in}{3.324922in}}%
\pgfpathlineto{\pgfqpoint{2.543263in}{3.315338in}}%
\pgfpathlineto{\pgfqpoint{2.549189in}{3.300644in}}%
\pgfpathlineto{\pgfqpoint{2.555102in}{3.287219in}}%
\pgfpathlineto{\pgfqpoint{2.561001in}{3.275036in}}%
\pgfpathclose%
\pgfusepath{stroke,fill}%
\end{pgfscope}%
\begin{pgfscope}%
\pgfpathrectangle{\pgfqpoint{0.887500in}{0.275000in}}{\pgfqpoint{4.225000in}{4.225000in}}%
\pgfusepath{clip}%
\pgfsetbuttcap%
\pgfsetroundjoin%
\definecolor{currentfill}{rgb}{0.595839,0.848717,0.243329}%
\pgfsetfillcolor{currentfill}%
\pgfsetfillopacity{0.700000}%
\pgfsetlinewidth{0.501875pt}%
\definecolor{currentstroke}{rgb}{1.000000,1.000000,1.000000}%
\pgfsetstrokecolor{currentstroke}%
\pgfsetstrokeopacity{0.500000}%
\pgfsetdash{}{0pt}%
\pgfpathmoveto{\pgfqpoint{2.925383in}{3.587907in}}%
\pgfpathlineto{\pgfqpoint{2.936371in}{3.593576in}}%
\pgfpathlineto{\pgfqpoint{2.947354in}{3.599351in}}%
\pgfpathlineto{\pgfqpoint{2.958334in}{3.605146in}}%
\pgfpathlineto{\pgfqpoint{2.969310in}{3.610875in}}%
\pgfpathlineto{\pgfqpoint{2.980283in}{3.616468in}}%
\pgfpathlineto{\pgfqpoint{2.974312in}{3.617939in}}%
\pgfpathlineto{\pgfqpoint{2.968347in}{3.619443in}}%
\pgfpathlineto{\pgfqpoint{2.962388in}{3.620979in}}%
\pgfpathlineto{\pgfqpoint{2.956434in}{3.622548in}}%
\pgfpathlineto{\pgfqpoint{2.950485in}{3.624146in}}%
\pgfpathlineto{\pgfqpoint{2.939533in}{3.618611in}}%
\pgfpathlineto{\pgfqpoint{2.928578in}{3.612965in}}%
\pgfpathlineto{\pgfqpoint{2.917619in}{3.607262in}}%
\pgfpathlineto{\pgfqpoint{2.906656in}{3.601564in}}%
\pgfpathlineto{\pgfqpoint{2.895689in}{3.595933in}}%
\pgfpathlineto{\pgfqpoint{2.901617in}{3.594231in}}%
\pgfpathlineto{\pgfqpoint{2.907550in}{3.592578in}}%
\pgfpathlineto{\pgfqpoint{2.913489in}{3.590973in}}%
\pgfpathlineto{\pgfqpoint{2.919434in}{3.589416in}}%
\pgfpathclose%
\pgfusepath{stroke,fill}%
\end{pgfscope}%
\begin{pgfscope}%
\pgfpathrectangle{\pgfqpoint{0.887500in}{0.275000in}}{\pgfqpoint{4.225000in}{4.225000in}}%
\pgfusepath{clip}%
\pgfsetbuttcap%
\pgfsetroundjoin%
\definecolor{currentfill}{rgb}{0.477504,0.821444,0.318195}%
\pgfsetfillcolor{currentfill}%
\pgfsetfillopacity{0.700000}%
\pgfsetlinewidth{0.501875pt}%
\definecolor{currentstroke}{rgb}{1.000000,1.000000,1.000000}%
\pgfsetstrokecolor{currentstroke}%
\pgfsetstrokeopacity{0.500000}%
\pgfsetdash{}{0pt}%
\pgfpathmoveto{\pgfqpoint{2.731001in}{3.477810in}}%
\pgfpathlineto{\pgfqpoint{2.741958in}{3.490923in}}%
\pgfpathlineto{\pgfqpoint{2.752918in}{3.503644in}}%
\pgfpathlineto{\pgfqpoint{2.763882in}{3.515729in}}%
\pgfpathlineto{\pgfqpoint{2.774853in}{3.526938in}}%
\pgfpathlineto{\pgfqpoint{2.785829in}{3.537028in}}%
\pgfpathlineto{\pgfqpoint{2.779951in}{3.538382in}}%
\pgfpathlineto{\pgfqpoint{2.774080in}{3.539585in}}%
\pgfpathlineto{\pgfqpoint{2.768215in}{3.540610in}}%
\pgfpathlineto{\pgfqpoint{2.762358in}{3.541429in}}%
\pgfpathlineto{\pgfqpoint{2.751395in}{3.532048in}}%
\pgfpathlineto{\pgfqpoint{2.740438in}{3.521876in}}%
\pgfpathlineto{\pgfqpoint{2.729483in}{3.511093in}}%
\pgfpathlineto{\pgfqpoint{2.718531in}{3.499877in}}%
\pgfpathlineto{\pgfqpoint{2.707581in}{3.488405in}}%
\pgfpathlineto{\pgfqpoint{2.713424in}{3.486255in}}%
\pgfpathlineto{\pgfqpoint{2.719276in}{3.483726in}}%
\pgfpathlineto{\pgfqpoint{2.725135in}{3.480889in}}%
\pgfpathclose%
\pgfusepath{stroke,fill}%
\end{pgfscope}%
\begin{pgfscope}%
\pgfpathrectangle{\pgfqpoint{0.887500in}{0.275000in}}{\pgfqpoint{4.225000in}{4.225000in}}%
\pgfusepath{clip}%
\pgfsetbuttcap%
\pgfsetroundjoin%
\definecolor{currentfill}{rgb}{0.335885,0.777018,0.402049}%
\pgfsetfillcolor{currentfill}%
\pgfsetfillopacity{0.700000}%
\pgfsetlinewidth{0.501875pt}%
\definecolor{currentstroke}{rgb}{1.000000,1.000000,1.000000}%
\pgfsetstrokecolor{currentstroke}%
\pgfsetstrokeopacity{0.500000}%
\pgfsetdash{}{0pt}%
\pgfpathmoveto{\pgfqpoint{2.621556in}{3.336709in}}%
\pgfpathlineto{\pgfqpoint{2.632472in}{3.353107in}}%
\pgfpathlineto{\pgfqpoint{2.643398in}{3.368918in}}%
\pgfpathlineto{\pgfqpoint{2.654333in}{3.383885in}}%
\pgfpathlineto{\pgfqpoint{2.665276in}{3.398140in}}%
\pgfpathlineto{\pgfqpoint{2.676225in}{3.411849in}}%
\pgfpathlineto{\pgfqpoint{2.670362in}{3.417132in}}%
\pgfpathlineto{\pgfqpoint{2.664506in}{3.422235in}}%
\pgfpathlineto{\pgfqpoint{2.658656in}{3.427065in}}%
\pgfpathlineto{\pgfqpoint{2.652815in}{3.431530in}}%
\pgfpathlineto{\pgfqpoint{2.641860in}{3.420116in}}%
\pgfpathlineto{\pgfqpoint{2.630907in}{3.408497in}}%
\pgfpathlineto{\pgfqpoint{2.619956in}{3.396585in}}%
\pgfpathlineto{\pgfqpoint{2.609009in}{3.384311in}}%
\pgfpathlineto{\pgfqpoint{2.598064in}{3.371808in}}%
\pgfpathlineto{\pgfqpoint{2.603938in}{3.362505in}}%
\pgfpathlineto{\pgfqpoint{2.609811in}{3.353519in}}%
\pgfpathlineto{\pgfqpoint{2.615684in}{3.344903in}}%
\pgfpathclose%
\pgfusepath{stroke,fill}%
\end{pgfscope}%
\begin{pgfscope}%
\pgfpathrectangle{\pgfqpoint{0.887500in}{0.275000in}}{\pgfqpoint{4.225000in}{4.225000in}}%
\pgfusepath{clip}%
\pgfsetbuttcap%
\pgfsetroundjoin%
\definecolor{currentfill}{rgb}{0.404001,0.800275,0.362552}%
\pgfsetfillcolor{currentfill}%
\pgfsetfillopacity{0.700000}%
\pgfsetlinewidth{0.501875pt}%
\definecolor{currentstroke}{rgb}{1.000000,1.000000,1.000000}%
\pgfsetstrokecolor{currentstroke}%
\pgfsetstrokeopacity{0.500000}%
\pgfsetdash{}{0pt}%
\pgfpathmoveto{\pgfqpoint{2.676225in}{3.411849in}}%
\pgfpathlineto{\pgfqpoint{2.687179in}{3.425178in}}%
\pgfpathlineto{\pgfqpoint{2.698134in}{3.438294in}}%
\pgfpathlineto{\pgfqpoint{2.709090in}{3.451364in}}%
\pgfpathlineto{\pgfqpoint{2.720046in}{3.464547in}}%
\pgfpathlineto{\pgfqpoint{2.731001in}{3.477810in}}%
\pgfpathlineto{\pgfqpoint{2.725135in}{3.480889in}}%
\pgfpathlineto{\pgfqpoint{2.719276in}{3.483726in}}%
\pgfpathlineto{\pgfqpoint{2.713424in}{3.486255in}}%
\pgfpathlineto{\pgfqpoint{2.707581in}{3.488405in}}%
\pgfpathlineto{\pgfqpoint{2.696631in}{3.476857in}}%
\pgfpathlineto{\pgfqpoint{2.685679in}{3.465403in}}%
\pgfpathlineto{\pgfqpoint{2.674725in}{3.454087in}}%
\pgfpathlineto{\pgfqpoint{2.663771in}{3.442824in}}%
\pgfpathlineto{\pgfqpoint{2.652815in}{3.431530in}}%
\pgfpathlineto{\pgfqpoint{2.658656in}{3.427065in}}%
\pgfpathlineto{\pgfqpoint{2.664506in}{3.422235in}}%
\pgfpathlineto{\pgfqpoint{2.670362in}{3.417132in}}%
\pgfpathclose%
\pgfusepath{stroke,fill}%
\end{pgfscope}%
\begin{pgfscope}%
\pgfpathrectangle{\pgfqpoint{0.887500in}{0.275000in}}{\pgfqpoint{4.225000in}{4.225000in}}%
\pgfusepath{clip}%
\pgfsetbuttcap%
\pgfsetroundjoin%
\definecolor{currentfill}{rgb}{0.226397,0.728888,0.462789}%
\pgfsetfillcolor{currentfill}%
\pgfsetfillopacity{0.700000}%
\pgfsetlinewidth{0.501875pt}%
\definecolor{currentstroke}{rgb}{1.000000,1.000000,1.000000}%
\pgfsetstrokecolor{currentstroke}%
\pgfsetstrokeopacity{0.500000}%
\pgfsetdash{}{0pt}%
\pgfpathmoveto{\pgfqpoint{2.511638in}{3.236443in}}%
\pgfpathlineto{\pgfqpoint{2.522731in}{3.239334in}}%
\pgfpathlineto{\pgfqpoint{2.533807in}{3.243071in}}%
\pgfpathlineto{\pgfqpoint{2.544862in}{3.248127in}}%
\pgfpathlineto{\pgfqpoint{2.555890in}{3.254971in}}%
\pgfpathlineto{\pgfqpoint{2.566888in}{3.264073in}}%
\pgfpathlineto{\pgfqpoint{2.561001in}{3.275036in}}%
\pgfpathlineto{\pgfqpoint{2.555102in}{3.287219in}}%
\pgfpathlineto{\pgfqpoint{2.549189in}{3.300644in}}%
\pgfpathlineto{\pgfqpoint{2.543263in}{3.315338in}}%
\pgfpathlineto{\pgfqpoint{2.532260in}{3.307107in}}%
\pgfpathlineto{\pgfqpoint{2.521239in}{3.300139in}}%
\pgfpathlineto{\pgfqpoint{2.510200in}{3.294194in}}%
\pgfpathlineto{\pgfqpoint{2.499147in}{3.289033in}}%
\pgfpathlineto{\pgfqpoint{2.488083in}{3.284418in}}%
\pgfpathlineto{\pgfqpoint{2.493989in}{3.270925in}}%
\pgfpathlineto{\pgfqpoint{2.499883in}{3.258456in}}%
\pgfpathlineto{\pgfqpoint{2.505765in}{3.246974in}}%
\pgfpathclose%
\pgfusepath{stroke,fill}%
\end{pgfscope}%
\begin{pgfscope}%
\pgfpathrectangle{\pgfqpoint{0.887500in}{0.275000in}}{\pgfqpoint{4.225000in}{4.225000in}}%
\pgfusepath{clip}%
\pgfsetbuttcap%
\pgfsetroundjoin%
\definecolor{currentfill}{rgb}{0.896320,0.893616,0.096335}%
\pgfsetfillcolor{currentfill}%
\pgfsetfillopacity{0.700000}%
\pgfsetlinewidth{0.501875pt}%
\definecolor{currentstroke}{rgb}{1.000000,1.000000,1.000000}%
\pgfsetstrokecolor{currentstroke}%
\pgfsetstrokeopacity{0.500000}%
\pgfsetdash{}{0pt}%
\pgfpathmoveto{\pgfqpoint{3.454504in}{3.804773in}}%
\pgfpathlineto{\pgfqpoint{3.465404in}{3.810143in}}%
\pgfpathlineto{\pgfqpoint{3.476299in}{3.815345in}}%
\pgfpathlineto{\pgfqpoint{3.487187in}{3.820372in}}%
\pgfpathlineto{\pgfqpoint{3.498070in}{3.825218in}}%
\pgfpathlineto{\pgfqpoint{3.508947in}{3.829877in}}%
\pgfpathlineto{\pgfqpoint{3.502718in}{3.830927in}}%
\pgfpathlineto{\pgfqpoint{3.496494in}{3.831797in}}%
\pgfpathlineto{\pgfqpoint{3.490275in}{3.832505in}}%
\pgfpathlineto{\pgfqpoint{3.484060in}{3.833066in}}%
\pgfpathlineto{\pgfqpoint{3.477849in}{3.833491in}}%
\pgfpathlineto{\pgfqpoint{3.466998in}{3.829276in}}%
\pgfpathlineto{\pgfqpoint{3.456139in}{3.824759in}}%
\pgfpathlineto{\pgfqpoint{3.445273in}{3.819968in}}%
\pgfpathlineto{\pgfqpoint{3.434400in}{3.814933in}}%
\pgfpathlineto{\pgfqpoint{3.423522in}{3.809683in}}%
\pgfpathlineto{\pgfqpoint{3.429708in}{3.808910in}}%
\pgfpathlineto{\pgfqpoint{3.435900in}{3.808045in}}%
\pgfpathlineto{\pgfqpoint{3.442096in}{3.807079in}}%
\pgfpathlineto{\pgfqpoint{3.448298in}{3.805996in}}%
\pgfpathclose%
\pgfusepath{stroke,fill}%
\end{pgfscope}%
\begin{pgfscope}%
\pgfpathrectangle{\pgfqpoint{0.887500in}{0.275000in}}{\pgfqpoint{4.225000in}{4.225000in}}%
\pgfusepath{clip}%
\pgfsetbuttcap%
\pgfsetroundjoin%
\definecolor{currentfill}{rgb}{0.793760,0.880678,0.120005}%
\pgfsetfillcolor{currentfill}%
\pgfsetfillopacity{0.700000}%
\pgfsetlinewidth{0.501875pt}%
\definecolor{currentstroke}{rgb}{1.000000,1.000000,1.000000}%
\pgfsetstrokecolor{currentstroke}%
\pgfsetstrokeopacity{0.500000}%
\pgfsetdash{}{0pt}%
\pgfpathmoveto{\pgfqpoint{3.259825in}{3.720397in}}%
\pgfpathlineto{\pgfqpoint{3.270764in}{3.726549in}}%
\pgfpathlineto{\pgfqpoint{3.281700in}{3.732675in}}%
\pgfpathlineto{\pgfqpoint{3.292632in}{3.738779in}}%
\pgfpathlineto{\pgfqpoint{3.303560in}{3.744867in}}%
\pgfpathlineto{\pgfqpoint{3.314485in}{3.750945in}}%
\pgfpathlineto{\pgfqpoint{3.308350in}{3.752074in}}%
\pgfpathlineto{\pgfqpoint{3.302221in}{3.753186in}}%
\pgfpathlineto{\pgfqpoint{3.296097in}{3.754276in}}%
\pgfpathlineto{\pgfqpoint{3.289980in}{3.755343in}}%
\pgfpathlineto{\pgfqpoint{3.283868in}{3.756383in}}%
\pgfpathlineto{\pgfqpoint{3.272965in}{3.750482in}}%
\pgfpathlineto{\pgfqpoint{3.262059in}{3.744551in}}%
\pgfpathlineto{\pgfqpoint{3.251149in}{3.738588in}}%
\pgfpathlineto{\pgfqpoint{3.240235in}{3.732592in}}%
\pgfpathlineto{\pgfqpoint{3.229318in}{3.726560in}}%
\pgfpathlineto{\pgfqpoint{3.235408in}{3.725347in}}%
\pgfpathlineto{\pgfqpoint{3.241503in}{3.724120in}}%
\pgfpathlineto{\pgfqpoint{3.247604in}{3.722884in}}%
\pgfpathlineto{\pgfqpoint{3.253712in}{3.721642in}}%
\pgfpathclose%
\pgfusepath{stroke,fill}%
\end{pgfscope}%
\begin{pgfscope}%
\pgfpathrectangle{\pgfqpoint{0.887500in}{0.275000in}}{\pgfqpoint{4.225000in}{4.225000in}}%
\pgfusepath{clip}%
\pgfsetbuttcap%
\pgfsetroundjoin%
\definecolor{currentfill}{rgb}{0.668054,0.861999,0.196293}%
\pgfsetfillcolor{currentfill}%
\pgfsetfillopacity{0.700000}%
\pgfsetlinewidth{0.501875pt}%
\definecolor{currentstroke}{rgb}{1.000000,1.000000,1.000000}%
\pgfsetstrokecolor{currentstroke}%
\pgfsetstrokeopacity{0.500000}%
\pgfsetdash{}{0pt}%
\pgfpathmoveto{\pgfqpoint{3.065132in}{3.636525in}}%
\pgfpathlineto{\pgfqpoint{3.076102in}{3.642027in}}%
\pgfpathlineto{\pgfqpoint{3.087069in}{3.647673in}}%
\pgfpathlineto{\pgfqpoint{3.098032in}{3.653453in}}%
\pgfpathlineto{\pgfqpoint{3.108992in}{3.659347in}}%
\pgfpathlineto{\pgfqpoint{3.119948in}{3.665333in}}%
\pgfpathlineto{\pgfqpoint{3.113907in}{3.666673in}}%
\pgfpathlineto{\pgfqpoint{3.107872in}{3.668033in}}%
\pgfpathlineto{\pgfqpoint{3.101842in}{3.669407in}}%
\pgfpathlineto{\pgfqpoint{3.095818in}{3.670789in}}%
\pgfpathlineto{\pgfqpoint{3.089800in}{3.672175in}}%
\pgfpathlineto{\pgfqpoint{3.078866in}{3.666243in}}%
\pgfpathlineto{\pgfqpoint{3.067927in}{3.660389in}}%
\pgfpathlineto{\pgfqpoint{3.056985in}{3.654633in}}%
\pgfpathlineto{\pgfqpoint{3.046039in}{3.648994in}}%
\pgfpathlineto{\pgfqpoint{3.035090in}{3.643480in}}%
\pgfpathlineto{\pgfqpoint{3.041087in}{3.642021in}}%
\pgfpathlineto{\pgfqpoint{3.047089in}{3.640590in}}%
\pgfpathlineto{\pgfqpoint{3.053098in}{3.639193in}}%
\pgfpathlineto{\pgfqpoint{3.059112in}{3.637836in}}%
\pgfpathclose%
\pgfusepath{stroke,fill}%
\end{pgfscope}%
\begin{pgfscope}%
\pgfpathrectangle{\pgfqpoint{0.887500in}{0.275000in}}{\pgfqpoint{4.225000in}{4.225000in}}%
\pgfusepath{clip}%
\pgfsetbuttcap%
\pgfsetroundjoin%
\definecolor{currentfill}{rgb}{0.955300,0.901065,0.118128}%
\pgfsetfillcolor{currentfill}%
\pgfsetfillopacity{0.700000}%
\pgfsetlinewidth{0.501875pt}%
\definecolor{currentstroke}{rgb}{1.000000,1.000000,1.000000}%
\pgfsetstrokecolor{currentstroke}%
\pgfsetstrokeopacity{0.500000}%
\pgfsetdash{}{0pt}%
\pgfpathmoveto{\pgfqpoint{3.594555in}{3.844716in}}%
\pgfpathlineto{\pgfqpoint{3.605423in}{3.849171in}}%
\pgfpathlineto{\pgfqpoint{3.616285in}{3.853546in}}%
\pgfpathlineto{\pgfqpoint{3.627141in}{3.857839in}}%
\pgfpathlineto{\pgfqpoint{3.637993in}{3.862048in}}%
\pgfpathlineto{\pgfqpoint{3.648838in}{3.866170in}}%
\pgfpathlineto{\pgfqpoint{3.642536in}{3.866899in}}%
\pgfpathlineto{\pgfqpoint{3.636233in}{3.867156in}}%
\pgfpathlineto{\pgfqpoint{3.629931in}{3.866978in}}%
\pgfpathlineto{\pgfqpoint{3.623629in}{3.866405in}}%
\pgfpathlineto{\pgfqpoint{3.617330in}{3.865476in}}%
\pgfpathlineto{\pgfqpoint{3.606525in}{3.862843in}}%
\pgfpathlineto{\pgfqpoint{3.595713in}{3.859999in}}%
\pgfpathlineto{\pgfqpoint{3.584893in}{3.856947in}}%
\pgfpathlineto{\pgfqpoint{3.574065in}{3.853687in}}%
\pgfpathlineto{\pgfqpoint{3.563229in}{3.850222in}}%
\pgfpathlineto{\pgfqpoint{3.569490in}{3.849711in}}%
\pgfpathlineto{\pgfqpoint{3.575753in}{3.848928in}}%
\pgfpathlineto{\pgfqpoint{3.582019in}{3.847850in}}%
\pgfpathlineto{\pgfqpoint{3.588286in}{3.846453in}}%
\pgfpathclose%
\pgfusepath{stroke,fill}%
\end{pgfscope}%
\begin{pgfscope}%
\pgfpathrectangle{\pgfqpoint{0.887500in}{0.275000in}}{\pgfqpoint{4.225000in}{4.225000in}}%
\pgfusepath{clip}%
\pgfsetbuttcap%
\pgfsetroundjoin%
\definecolor{currentfill}{rgb}{0.575563,0.844566,0.256415}%
\pgfsetfillcolor{currentfill}%
\pgfsetfillopacity{0.700000}%
\pgfsetlinewidth{0.501875pt}%
\definecolor{currentstroke}{rgb}{1.000000,1.000000,1.000000}%
\pgfsetstrokecolor{currentstroke}%
\pgfsetstrokeopacity{0.500000}%
\pgfsetdash{}{0pt}%
\pgfpathmoveto{\pgfqpoint{2.870371in}{3.563330in}}%
\pgfpathlineto{\pgfqpoint{2.881384in}{3.567852in}}%
\pgfpathlineto{\pgfqpoint{2.892392in}{3.572381in}}%
\pgfpathlineto{\pgfqpoint{2.903394in}{3.577224in}}%
\pgfpathlineto{\pgfqpoint{2.914391in}{3.582427in}}%
\pgfpathlineto{\pgfqpoint{2.925383in}{3.587907in}}%
\pgfpathlineto{\pgfqpoint{2.919434in}{3.589416in}}%
\pgfpathlineto{\pgfqpoint{2.913489in}{3.590973in}}%
\pgfpathlineto{\pgfqpoint{2.907550in}{3.592578in}}%
\pgfpathlineto{\pgfqpoint{2.901617in}{3.594231in}}%
\pgfpathlineto{\pgfqpoint{2.895689in}{3.595933in}}%
\pgfpathlineto{\pgfqpoint{2.884718in}{3.590434in}}%
\pgfpathlineto{\pgfqpoint{2.873742in}{3.585128in}}%
\pgfpathlineto{\pgfqpoint{2.862761in}{3.580079in}}%
\pgfpathlineto{\pgfqpoint{2.851775in}{3.575249in}}%
\pgfpathlineto{\pgfqpoint{2.840784in}{3.570406in}}%
\pgfpathlineto{\pgfqpoint{2.846690in}{3.568987in}}%
\pgfpathlineto{\pgfqpoint{2.852602in}{3.567565in}}%
\pgfpathlineto{\pgfqpoint{2.858520in}{3.566144in}}%
\pgfpathlineto{\pgfqpoint{2.864443in}{3.564730in}}%
\pgfpathclose%
\pgfusepath{stroke,fill}%
\end{pgfscope}%
\begin{pgfscope}%
\pgfpathrectangle{\pgfqpoint{0.887500in}{0.275000in}}{\pgfqpoint{4.225000in}{4.225000in}}%
\pgfusepath{clip}%
\pgfsetbuttcap%
\pgfsetroundjoin%
\definecolor{currentfill}{rgb}{0.239374,0.735588,0.455688}%
\pgfsetfillcolor{currentfill}%
\pgfsetfillopacity{0.700000}%
\pgfsetlinewidth{0.501875pt}%
\definecolor{currentstroke}{rgb}{1.000000,1.000000,1.000000}%
\pgfsetstrokecolor{currentstroke}%
\pgfsetstrokeopacity{0.500000}%
\pgfsetdash{}{0pt}%
\pgfpathmoveto{\pgfqpoint{2.596166in}{3.226712in}}%
\pgfpathlineto{\pgfqpoint{2.607142in}{3.238982in}}%
\pgfpathlineto{\pgfqpoint{2.618095in}{3.253358in}}%
\pgfpathlineto{\pgfqpoint{2.629032in}{3.269320in}}%
\pgfpathlineto{\pgfqpoint{2.639961in}{3.286348in}}%
\pgfpathlineto{\pgfqpoint{2.650886in}{3.303919in}}%
\pgfpathlineto{\pgfqpoint{2.645024in}{3.309210in}}%
\pgfpathlineto{\pgfqpoint{2.639160in}{3.315187in}}%
\pgfpathlineto{\pgfqpoint{2.633294in}{3.321798in}}%
\pgfpathlineto{\pgfqpoint{2.627426in}{3.328990in}}%
\pgfpathlineto{\pgfqpoint{2.621556in}{3.336709in}}%
\pgfpathlineto{\pgfqpoint{2.610641in}{3.320240in}}%
\pgfpathlineto{\pgfqpoint{2.599724in}{3.304223in}}%
\pgfpathlineto{\pgfqpoint{2.588798in}{3.289177in}}%
\pgfpathlineto{\pgfqpoint{2.577855in}{3.275621in}}%
\pgfpathlineto{\pgfqpoint{2.566888in}{3.264073in}}%
\pgfpathlineto{\pgfqpoint{2.572763in}{3.254305in}}%
\pgfpathlineto{\pgfqpoint{2.578628in}{3.245708in}}%
\pgfpathlineto{\pgfqpoint{2.584483in}{3.238259in}}%
\pgfpathlineto{\pgfqpoint{2.590328in}{3.231935in}}%
\pgfpathclose%
\pgfusepath{stroke,fill}%
\end{pgfscope}%
\begin{pgfscope}%
\pgfpathrectangle{\pgfqpoint{0.887500in}{0.275000in}}{\pgfqpoint{4.225000in}{4.225000in}}%
\pgfusepath{clip}%
\pgfsetbuttcap%
\pgfsetroundjoin%
\definecolor{currentfill}{rgb}{0.876168,0.891125,0.095250}%
\pgfsetfillcolor{currentfill}%
\pgfsetfillopacity{0.700000}%
\pgfsetlinewidth{0.501875pt}%
\definecolor{currentstroke}{rgb}{1.000000,1.000000,1.000000}%
\pgfsetstrokecolor{currentstroke}%
\pgfsetstrokeopacity{0.500000}%
\pgfsetdash{}{0pt}%
\pgfpathmoveto{\pgfqpoint{3.399927in}{3.775735in}}%
\pgfpathlineto{\pgfqpoint{3.410852in}{3.781775in}}%
\pgfpathlineto{\pgfqpoint{3.421772in}{3.787722in}}%
\pgfpathlineto{\pgfqpoint{3.432688in}{3.793552in}}%
\pgfpathlineto{\pgfqpoint{3.443599in}{3.799240in}}%
\pgfpathlineto{\pgfqpoint{3.454504in}{3.804773in}}%
\pgfpathlineto{\pgfqpoint{3.448298in}{3.805996in}}%
\pgfpathlineto{\pgfqpoint{3.442096in}{3.807079in}}%
\pgfpathlineto{\pgfqpoint{3.435900in}{3.808045in}}%
\pgfpathlineto{\pgfqpoint{3.429708in}{3.808910in}}%
\pgfpathlineto{\pgfqpoint{3.423522in}{3.809683in}}%
\pgfpathlineto{\pgfqpoint{3.412638in}{3.804248in}}%
\pgfpathlineto{\pgfqpoint{3.401750in}{3.798656in}}%
\pgfpathlineto{\pgfqpoint{3.390856in}{3.792934in}}%
\pgfpathlineto{\pgfqpoint{3.379958in}{3.787102in}}%
\pgfpathlineto{\pgfqpoint{3.369055in}{3.781180in}}%
\pgfpathlineto{\pgfqpoint{3.375218in}{3.780207in}}%
\pgfpathlineto{\pgfqpoint{3.381387in}{3.779188in}}%
\pgfpathlineto{\pgfqpoint{3.387562in}{3.778117in}}%
\pgfpathlineto{\pgfqpoint{3.393742in}{3.776974in}}%
\pgfpathclose%
\pgfusepath{stroke,fill}%
\end{pgfscope}%
\begin{pgfscope}%
\pgfpathrectangle{\pgfqpoint{0.887500in}{0.275000in}}{\pgfqpoint{4.225000in}{4.225000in}}%
\pgfusepath{clip}%
\pgfsetbuttcap%
\pgfsetroundjoin%
\definecolor{currentfill}{rgb}{0.762373,0.876424,0.137064}%
\pgfsetfillcolor{currentfill}%
\pgfsetfillopacity{0.700000}%
\pgfsetlinewidth{0.501875pt}%
\definecolor{currentstroke}{rgb}{1.000000,1.000000,1.000000}%
\pgfsetstrokecolor{currentstroke}%
\pgfsetstrokeopacity{0.500000}%
\pgfsetdash{}{0pt}%
\pgfpathmoveto{\pgfqpoint{3.205075in}{3.689383in}}%
\pgfpathlineto{\pgfqpoint{3.216032in}{3.695578in}}%
\pgfpathlineto{\pgfqpoint{3.226986in}{3.701791in}}%
\pgfpathlineto{\pgfqpoint{3.237936in}{3.708008in}}%
\pgfpathlineto{\pgfqpoint{3.248882in}{3.714215in}}%
\pgfpathlineto{\pgfqpoint{3.259825in}{3.720397in}}%
\pgfpathlineto{\pgfqpoint{3.253712in}{3.721642in}}%
\pgfpathlineto{\pgfqpoint{3.247604in}{3.722884in}}%
\pgfpathlineto{\pgfqpoint{3.241503in}{3.724120in}}%
\pgfpathlineto{\pgfqpoint{3.235408in}{3.725347in}}%
\pgfpathlineto{\pgfqpoint{3.229318in}{3.726560in}}%
\pgfpathlineto{\pgfqpoint{3.218397in}{3.720491in}}%
\pgfpathlineto{\pgfqpoint{3.207472in}{3.714388in}}%
\pgfpathlineto{\pgfqpoint{3.196544in}{3.708260in}}%
\pgfpathlineto{\pgfqpoint{3.185612in}{3.702114in}}%
\pgfpathlineto{\pgfqpoint{3.174677in}{3.695957in}}%
\pgfpathlineto{\pgfqpoint{3.180745in}{3.694640in}}%
\pgfpathlineto{\pgfqpoint{3.186819in}{3.693321in}}%
\pgfpathlineto{\pgfqpoint{3.192898in}{3.692002in}}%
\pgfpathlineto{\pgfqpoint{3.198984in}{3.690688in}}%
\pgfpathclose%
\pgfusepath{stroke,fill}%
\end{pgfscope}%
\begin{pgfscope}%
\pgfpathrectangle{\pgfqpoint{0.887500in}{0.275000in}}{\pgfqpoint{4.225000in}{4.225000in}}%
\pgfusepath{clip}%
\pgfsetbuttcap%
\pgfsetroundjoin%
\definecolor{currentfill}{rgb}{0.214000,0.722114,0.469588}%
\pgfsetfillcolor{currentfill}%
\pgfsetfillopacity{0.700000}%
\pgfsetlinewidth{0.501875pt}%
\definecolor{currentstroke}{rgb}{1.000000,1.000000,1.000000}%
\pgfsetstrokecolor{currentstroke}%
\pgfsetstrokeopacity{0.500000}%
\pgfsetdash{}{0pt}%
\pgfpathmoveto{\pgfqpoint{2.456111in}{3.220779in}}%
\pgfpathlineto{\pgfqpoint{2.467216in}{3.224662in}}%
\pgfpathlineto{\pgfqpoint{2.478321in}{3.228223in}}%
\pgfpathlineto{\pgfqpoint{2.489427in}{3.231318in}}%
\pgfpathlineto{\pgfqpoint{2.500534in}{3.233928in}}%
\pgfpathlineto{\pgfqpoint{2.511638in}{3.236443in}}%
\pgfpathlineto{\pgfqpoint{2.505765in}{3.246974in}}%
\pgfpathlineto{\pgfqpoint{2.499883in}{3.258456in}}%
\pgfpathlineto{\pgfqpoint{2.493989in}{3.270925in}}%
\pgfpathlineto{\pgfqpoint{2.488083in}{3.284418in}}%
\pgfpathlineto{\pgfqpoint{2.477009in}{3.280110in}}%
\pgfpathlineto{\pgfqpoint{2.465930in}{3.275869in}}%
\pgfpathlineto{\pgfqpoint{2.454848in}{3.271505in}}%
\pgfpathlineto{\pgfqpoint{2.443762in}{3.267066in}}%
\pgfpathlineto{\pgfqpoint{2.432672in}{3.262674in}}%
\pgfpathlineto{\pgfqpoint{2.438541in}{3.251266in}}%
\pgfpathlineto{\pgfqpoint{2.444404in}{3.240506in}}%
\pgfpathlineto{\pgfqpoint{2.450260in}{3.230356in}}%
\pgfpathclose%
\pgfusepath{stroke,fill}%
\end{pgfscope}%
\begin{pgfscope}%
\pgfpathrectangle{\pgfqpoint{0.887500in}{0.275000in}}{\pgfqpoint{4.225000in}{4.225000in}}%
\pgfusepath{clip}%
\pgfsetbuttcap%
\pgfsetroundjoin%
\definecolor{currentfill}{rgb}{0.647257,0.858400,0.209861}%
\pgfsetfillcolor{currentfill}%
\pgfsetfillopacity{0.700000}%
\pgfsetlinewidth{0.501875pt}%
\definecolor{currentstroke}{rgb}{1.000000,1.000000,1.000000}%
\pgfsetstrokecolor{currentstroke}%
\pgfsetstrokeopacity{0.500000}%
\pgfsetdash{}{0pt}%
\pgfpathmoveto{\pgfqpoint{3.010221in}{3.609664in}}%
\pgfpathlineto{\pgfqpoint{3.021211in}{3.615113in}}%
\pgfpathlineto{\pgfqpoint{3.032197in}{3.620474in}}%
\pgfpathlineto{\pgfqpoint{3.043179in}{3.625798in}}%
\pgfpathlineto{\pgfqpoint{3.054158in}{3.631132in}}%
\pgfpathlineto{\pgfqpoint{3.065132in}{3.636525in}}%
\pgfpathlineto{\pgfqpoint{3.059112in}{3.637836in}}%
\pgfpathlineto{\pgfqpoint{3.053098in}{3.639193in}}%
\pgfpathlineto{\pgfqpoint{3.047089in}{3.640590in}}%
\pgfpathlineto{\pgfqpoint{3.041087in}{3.642021in}}%
\pgfpathlineto{\pgfqpoint{3.035090in}{3.643480in}}%
\pgfpathlineto{\pgfqpoint{3.024136in}{3.638062in}}%
\pgfpathlineto{\pgfqpoint{3.013179in}{3.632694in}}%
\pgfpathlineto{\pgfqpoint{3.002217in}{3.627335in}}%
\pgfpathlineto{\pgfqpoint{2.991252in}{3.621941in}}%
\pgfpathlineto{\pgfqpoint{2.980283in}{3.616468in}}%
\pgfpathlineto{\pgfqpoint{2.986259in}{3.615033in}}%
\pgfpathlineto{\pgfqpoint{2.992241in}{3.613634in}}%
\pgfpathlineto{\pgfqpoint{2.998228in}{3.612272in}}%
\pgfpathlineto{\pgfqpoint{3.004222in}{3.610948in}}%
\pgfpathclose%
\pgfusepath{stroke,fill}%
\end{pgfscope}%
\begin{pgfscope}%
\pgfpathrectangle{\pgfqpoint{0.887500in}{0.275000in}}{\pgfqpoint{4.225000in}{4.225000in}}%
\pgfusepath{clip}%
\pgfsetbuttcap%
\pgfsetroundjoin%
\definecolor{currentfill}{rgb}{0.202219,0.715272,0.476084}%
\pgfsetfillcolor{currentfill}%
\pgfsetfillopacity{0.700000}%
\pgfsetlinewidth{0.501875pt}%
\definecolor{currentstroke}{rgb}{1.000000,1.000000,1.000000}%
\pgfsetstrokecolor{currentstroke}%
\pgfsetstrokeopacity{0.500000}%
\pgfsetdash{}{0pt}%
\pgfpathmoveto{\pgfqpoint{2.540878in}{3.196772in}}%
\pgfpathlineto{\pgfqpoint{2.551978in}{3.199982in}}%
\pgfpathlineto{\pgfqpoint{2.563061in}{3.204100in}}%
\pgfpathlineto{\pgfqpoint{2.574123in}{3.209608in}}%
\pgfpathlineto{\pgfqpoint{2.585159in}{3.216985in}}%
\pgfpathlineto{\pgfqpoint{2.596166in}{3.226712in}}%
\pgfpathlineto{\pgfqpoint{2.590328in}{3.231935in}}%
\pgfpathlineto{\pgfqpoint{2.584483in}{3.238259in}}%
\pgfpathlineto{\pgfqpoint{2.578628in}{3.245708in}}%
\pgfpathlineto{\pgfqpoint{2.572763in}{3.254305in}}%
\pgfpathlineto{\pgfqpoint{2.566888in}{3.264073in}}%
\pgfpathlineto{\pgfqpoint{2.555890in}{3.254971in}}%
\pgfpathlineto{\pgfqpoint{2.544862in}{3.248127in}}%
\pgfpathlineto{\pgfqpoint{2.533807in}{3.243071in}}%
\pgfpathlineto{\pgfqpoint{2.522731in}{3.239334in}}%
\pgfpathlineto{\pgfqpoint{2.511638in}{3.236443in}}%
\pgfpathlineto{\pgfqpoint{2.517501in}{3.226827in}}%
\pgfpathlineto{\pgfqpoint{2.523356in}{3.218088in}}%
\pgfpathlineto{\pgfqpoint{2.529203in}{3.210190in}}%
\pgfpathlineto{\pgfqpoint{2.535044in}{3.203096in}}%
\pgfpathclose%
\pgfusepath{stroke,fill}%
\end{pgfscope}%
\begin{pgfscope}%
\pgfpathrectangle{\pgfqpoint{0.887500in}{0.275000in}}{\pgfqpoint{4.225000in}{4.225000in}}%
\pgfusepath{clip}%
\pgfsetbuttcap%
\pgfsetroundjoin%
\definecolor{currentfill}{rgb}{0.319809,0.770914,0.411152}%
\pgfsetfillcolor{currentfill}%
\pgfsetfillopacity{0.700000}%
\pgfsetlinewidth{0.501875pt}%
\definecolor{currentstroke}{rgb}{1.000000,1.000000,1.000000}%
\pgfsetstrokecolor{currentstroke}%
\pgfsetstrokeopacity{0.500000}%
\pgfsetdash{}{0pt}%
\pgfpathmoveto{\pgfqpoint{2.650886in}{3.303919in}}%
\pgfpathlineto{\pgfqpoint{2.661813in}{3.321511in}}%
\pgfpathlineto{\pgfqpoint{2.672748in}{3.338605in}}%
\pgfpathlineto{\pgfqpoint{2.683693in}{3.354944in}}%
\pgfpathlineto{\pgfqpoint{2.694645in}{3.370663in}}%
\pgfpathlineto{\pgfqpoint{2.705602in}{3.385933in}}%
\pgfpathlineto{\pgfqpoint{2.699721in}{3.390742in}}%
\pgfpathlineto{\pgfqpoint{2.693842in}{3.395830in}}%
\pgfpathlineto{\pgfqpoint{2.687966in}{3.401105in}}%
\pgfpathlineto{\pgfqpoint{2.682093in}{3.406475in}}%
\pgfpathlineto{\pgfqpoint{2.676225in}{3.411849in}}%
\pgfpathlineto{\pgfqpoint{2.665276in}{3.398140in}}%
\pgfpathlineto{\pgfqpoint{2.654333in}{3.383885in}}%
\pgfpathlineto{\pgfqpoint{2.643398in}{3.368918in}}%
\pgfpathlineto{\pgfqpoint{2.632472in}{3.353107in}}%
\pgfpathlineto{\pgfqpoint{2.621556in}{3.336709in}}%
\pgfpathlineto{\pgfqpoint{2.627426in}{3.328990in}}%
\pgfpathlineto{\pgfqpoint{2.633294in}{3.321798in}}%
\pgfpathlineto{\pgfqpoint{2.639160in}{3.315187in}}%
\pgfpathlineto{\pgfqpoint{2.645024in}{3.309210in}}%
\pgfpathclose%
\pgfusepath{stroke,fill}%
\end{pgfscope}%
\begin{pgfscope}%
\pgfpathrectangle{\pgfqpoint{0.887500in}{0.275000in}}{\pgfqpoint{4.225000in}{4.225000in}}%
\pgfusepath{clip}%
\pgfsetbuttcap%
\pgfsetroundjoin%
\definecolor{currentfill}{rgb}{0.555484,0.840254,0.269281}%
\pgfsetfillcolor{currentfill}%
\pgfsetfillopacity{0.700000}%
\pgfsetlinewidth{0.501875pt}%
\definecolor{currentstroke}{rgb}{1.000000,1.000000,1.000000}%
\pgfsetstrokecolor{currentstroke}%
\pgfsetstrokeopacity{0.500000}%
\pgfsetdash{}{0pt}%
\pgfpathmoveto{\pgfqpoint{2.815312in}{3.528982in}}%
\pgfpathlineto{\pgfqpoint{2.826315in}{3.538540in}}%
\pgfpathlineto{\pgfqpoint{2.837326in}{3.546419in}}%
\pgfpathlineto{\pgfqpoint{2.848340in}{3.552952in}}%
\pgfpathlineto{\pgfqpoint{2.859356in}{3.558476in}}%
\pgfpathlineto{\pgfqpoint{2.870371in}{3.563330in}}%
\pgfpathlineto{\pgfqpoint{2.864443in}{3.564730in}}%
\pgfpathlineto{\pgfqpoint{2.858520in}{3.566144in}}%
\pgfpathlineto{\pgfqpoint{2.852602in}{3.567565in}}%
\pgfpathlineto{\pgfqpoint{2.846690in}{3.568987in}}%
\pgfpathlineto{\pgfqpoint{2.840784in}{3.570406in}}%
\pgfpathlineto{\pgfqpoint{2.829791in}{3.565294in}}%
\pgfpathlineto{\pgfqpoint{2.818797in}{3.559656in}}%
\pgfpathlineto{\pgfqpoint{2.807804in}{3.553236in}}%
\pgfpathlineto{\pgfqpoint{2.796814in}{3.545777in}}%
\pgfpathlineto{\pgfqpoint{2.785829in}{3.537028in}}%
\pgfpathlineto{\pgfqpoint{2.791714in}{3.535552in}}%
\pgfpathlineto{\pgfqpoint{2.797605in}{3.533981in}}%
\pgfpathlineto{\pgfqpoint{2.803502in}{3.532344in}}%
\pgfpathlineto{\pgfqpoint{2.809404in}{3.530668in}}%
\pgfpathclose%
\pgfusepath{stroke,fill}%
\end{pgfscope}%
\begin{pgfscope}%
\pgfpathrectangle{\pgfqpoint{0.887500in}{0.275000in}}{\pgfqpoint{4.225000in}{4.225000in}}%
\pgfusepath{clip}%
\pgfsetbuttcap%
\pgfsetroundjoin%
\definecolor{currentfill}{rgb}{0.945636,0.899815,0.112838}%
\pgfsetfillcolor{currentfill}%
\pgfsetfillopacity{0.700000}%
\pgfsetlinewidth{0.501875pt}%
\definecolor{currentstroke}{rgb}{1.000000,1.000000,1.000000}%
\pgfsetstrokecolor{currentstroke}%
\pgfsetstrokeopacity{0.500000}%
\pgfsetdash{}{0pt}%
\pgfpathmoveto{\pgfqpoint{3.540139in}{3.821301in}}%
\pgfpathlineto{\pgfqpoint{3.551032in}{3.826129in}}%
\pgfpathlineto{\pgfqpoint{3.561921in}{3.830886in}}%
\pgfpathlineto{\pgfqpoint{3.572804in}{3.835571in}}%
\pgfpathlineto{\pgfqpoint{3.583682in}{3.840182in}}%
\pgfpathlineto{\pgfqpoint{3.594555in}{3.844716in}}%
\pgfpathlineto{\pgfqpoint{3.588286in}{3.846453in}}%
\pgfpathlineto{\pgfqpoint{3.582019in}{3.847850in}}%
\pgfpathlineto{\pgfqpoint{3.575753in}{3.848928in}}%
\pgfpathlineto{\pgfqpoint{3.569490in}{3.849711in}}%
\pgfpathlineto{\pgfqpoint{3.563229in}{3.850222in}}%
\pgfpathlineto{\pgfqpoint{3.552387in}{3.846553in}}%
\pgfpathlineto{\pgfqpoint{3.541537in}{3.842682in}}%
\pgfpathlineto{\pgfqpoint{3.530680in}{3.838611in}}%
\pgfpathlineto{\pgfqpoint{3.519817in}{3.834342in}}%
\pgfpathlineto{\pgfqpoint{3.508947in}{3.829877in}}%
\pgfpathlineto{\pgfqpoint{3.515179in}{3.828628in}}%
\pgfpathlineto{\pgfqpoint{3.521414in}{3.827164in}}%
\pgfpathlineto{\pgfqpoint{3.527653in}{3.825467in}}%
\pgfpathlineto{\pgfqpoint{3.533894in}{3.823518in}}%
\pgfpathclose%
\pgfusepath{stroke,fill}%
\end{pgfscope}%
\begin{pgfscope}%
\pgfpathrectangle{\pgfqpoint{0.887500in}{0.275000in}}{\pgfqpoint{4.225000in}{4.225000in}}%
\pgfusepath{clip}%
\pgfsetbuttcap%
\pgfsetroundjoin%
\definecolor{currentfill}{rgb}{0.412913,0.803041,0.357269}%
\pgfsetfillcolor{currentfill}%
\pgfsetfillopacity{0.700000}%
\pgfsetlinewidth{0.501875pt}%
\definecolor{currentstroke}{rgb}{1.000000,1.000000,1.000000}%
\pgfsetstrokecolor{currentstroke}%
\pgfsetstrokeopacity{0.500000}%
\pgfsetdash{}{0pt}%
\pgfpathmoveto{\pgfqpoint{2.705602in}{3.385933in}}%
\pgfpathlineto{\pgfqpoint{2.716563in}{3.400923in}}%
\pgfpathlineto{\pgfqpoint{2.727527in}{3.415802in}}%
\pgfpathlineto{\pgfqpoint{2.738491in}{3.430742in}}%
\pgfpathlineto{\pgfqpoint{2.749455in}{3.445903in}}%
\pgfpathlineto{\pgfqpoint{2.760419in}{3.461218in}}%
\pgfpathlineto{\pgfqpoint{2.754526in}{3.464467in}}%
\pgfpathlineto{\pgfqpoint{2.748637in}{3.467820in}}%
\pgfpathlineto{\pgfqpoint{2.742753in}{3.471207in}}%
\pgfpathlineto{\pgfqpoint{2.736874in}{3.474560in}}%
\pgfpathlineto{\pgfqpoint{2.731001in}{3.477810in}}%
\pgfpathlineto{\pgfqpoint{2.720046in}{3.464547in}}%
\pgfpathlineto{\pgfqpoint{2.709090in}{3.451364in}}%
\pgfpathlineto{\pgfqpoint{2.698134in}{3.438294in}}%
\pgfpathlineto{\pgfqpoint{2.687179in}{3.425178in}}%
\pgfpathlineto{\pgfqpoint{2.676225in}{3.411849in}}%
\pgfpathlineto{\pgfqpoint{2.682093in}{3.406475in}}%
\pgfpathlineto{\pgfqpoint{2.687966in}{3.401105in}}%
\pgfpathlineto{\pgfqpoint{2.693842in}{3.395830in}}%
\pgfpathlineto{\pgfqpoint{2.699721in}{3.390742in}}%
\pgfpathclose%
\pgfusepath{stroke,fill}%
\end{pgfscope}%
\begin{pgfscope}%
\pgfpathrectangle{\pgfqpoint{0.887500in}{0.275000in}}{\pgfqpoint{4.225000in}{4.225000in}}%
\pgfusepath{clip}%
\pgfsetbuttcap%
\pgfsetroundjoin%
\definecolor{currentfill}{rgb}{0.496615,0.826376,0.306377}%
\pgfsetfillcolor{currentfill}%
\pgfsetfillopacity{0.700000}%
\pgfsetlinewidth{0.501875pt}%
\definecolor{currentstroke}{rgb}{1.000000,1.000000,1.000000}%
\pgfsetstrokecolor{currentstroke}%
\pgfsetstrokeopacity{0.500000}%
\pgfsetdash{}{0pt}%
\pgfpathmoveto{\pgfqpoint{2.760419in}{3.461218in}}%
\pgfpathlineto{\pgfqpoint{2.771385in}{3.476375in}}%
\pgfpathlineto{\pgfqpoint{2.782356in}{3.491053in}}%
\pgfpathlineto{\pgfqpoint{2.793334in}{3.504929in}}%
\pgfpathlineto{\pgfqpoint{2.804319in}{3.517679in}}%
\pgfpathlineto{\pgfqpoint{2.815312in}{3.528982in}}%
\pgfpathlineto{\pgfqpoint{2.809404in}{3.530668in}}%
\pgfpathlineto{\pgfqpoint{2.803502in}{3.532344in}}%
\pgfpathlineto{\pgfqpoint{2.797605in}{3.533981in}}%
\pgfpathlineto{\pgfqpoint{2.791714in}{3.535552in}}%
\pgfpathlineto{\pgfqpoint{2.785829in}{3.537028in}}%
\pgfpathlineto{\pgfqpoint{2.774853in}{3.526938in}}%
\pgfpathlineto{\pgfqpoint{2.763882in}{3.515729in}}%
\pgfpathlineto{\pgfqpoint{2.752918in}{3.503644in}}%
\pgfpathlineto{\pgfqpoint{2.741958in}{3.490923in}}%
\pgfpathlineto{\pgfqpoint{2.731001in}{3.477810in}}%
\pgfpathlineto{\pgfqpoint{2.736874in}{3.474560in}}%
\pgfpathlineto{\pgfqpoint{2.742753in}{3.471207in}}%
\pgfpathlineto{\pgfqpoint{2.748637in}{3.467820in}}%
\pgfpathlineto{\pgfqpoint{2.754526in}{3.464467in}}%
\pgfpathclose%
\pgfusepath{stroke,fill}%
\end{pgfscope}%
\begin{pgfscope}%
\pgfpathrectangle{\pgfqpoint{0.887500in}{0.275000in}}{\pgfqpoint{4.225000in}{4.225000in}}%
\pgfusepath{clip}%
\pgfsetbuttcap%
\pgfsetroundjoin%
\definecolor{currentfill}{rgb}{0.730889,0.871916,0.156029}%
\pgfsetfillcolor{currentfill}%
\pgfsetfillopacity{0.700000}%
\pgfsetlinewidth{0.501875pt}%
\definecolor{currentstroke}{rgb}{1.000000,1.000000,1.000000}%
\pgfsetstrokecolor{currentstroke}%
\pgfsetstrokeopacity{0.500000}%
\pgfsetdash{}{0pt}%
\pgfpathmoveto{\pgfqpoint{3.150241in}{3.659125in}}%
\pgfpathlineto{\pgfqpoint{3.161215in}{3.665045in}}%
\pgfpathlineto{\pgfqpoint{3.172185in}{3.671039in}}%
\pgfpathlineto{\pgfqpoint{3.183152in}{3.677099in}}%
\pgfpathlineto{\pgfqpoint{3.194115in}{3.683219in}}%
\pgfpathlineto{\pgfqpoint{3.205075in}{3.689383in}}%
\pgfpathlineto{\pgfqpoint{3.198984in}{3.690688in}}%
\pgfpathlineto{\pgfqpoint{3.192898in}{3.692002in}}%
\pgfpathlineto{\pgfqpoint{3.186819in}{3.693321in}}%
\pgfpathlineto{\pgfqpoint{3.180745in}{3.694640in}}%
\pgfpathlineto{\pgfqpoint{3.174677in}{3.695957in}}%
\pgfpathlineto{\pgfqpoint{3.163738in}{3.689797in}}%
\pgfpathlineto{\pgfqpoint{3.152795in}{3.683643in}}%
\pgfpathlineto{\pgfqpoint{3.141850in}{3.677502in}}%
\pgfpathlineto{\pgfqpoint{3.130901in}{3.671392in}}%
\pgfpathlineto{\pgfqpoint{3.119948in}{3.665333in}}%
\pgfpathlineto{\pgfqpoint{3.125995in}{3.664018in}}%
\pgfpathlineto{\pgfqpoint{3.132047in}{3.662735in}}%
\pgfpathlineto{\pgfqpoint{3.138106in}{3.661487in}}%
\pgfpathlineto{\pgfqpoint{3.144171in}{3.660282in}}%
\pgfpathclose%
\pgfusepath{stroke,fill}%
\end{pgfscope}%
\begin{pgfscope}%
\pgfpathrectangle{\pgfqpoint{0.887500in}{0.275000in}}{\pgfqpoint{4.225000in}{4.225000in}}%
\pgfusepath{clip}%
\pgfsetbuttcap%
\pgfsetroundjoin%
\definecolor{currentfill}{rgb}{0.845561,0.887322,0.099702}%
\pgfsetfillcolor{currentfill}%
\pgfsetfillopacity{0.700000}%
\pgfsetlinewidth{0.501875pt}%
\definecolor{currentstroke}{rgb}{1.000000,1.000000,1.000000}%
\pgfsetstrokecolor{currentstroke}%
\pgfsetstrokeopacity{0.500000}%
\pgfsetdash{}{0pt}%
\pgfpathmoveto{\pgfqpoint{3.345247in}{3.745001in}}%
\pgfpathlineto{\pgfqpoint{3.356190in}{3.751150in}}%
\pgfpathlineto{\pgfqpoint{3.367129in}{3.757311in}}%
\pgfpathlineto{\pgfqpoint{3.378066in}{3.763479in}}%
\pgfpathlineto{\pgfqpoint{3.388998in}{3.769628in}}%
\pgfpathlineto{\pgfqpoint{3.399927in}{3.775735in}}%
\pgfpathlineto{\pgfqpoint{3.393742in}{3.776974in}}%
\pgfpathlineto{\pgfqpoint{3.387562in}{3.778117in}}%
\pgfpathlineto{\pgfqpoint{3.381387in}{3.779188in}}%
\pgfpathlineto{\pgfqpoint{3.375218in}{3.780207in}}%
\pgfpathlineto{\pgfqpoint{3.369055in}{3.781180in}}%
\pgfpathlineto{\pgfqpoint{3.358148in}{3.775191in}}%
\pgfpathlineto{\pgfqpoint{3.347238in}{3.769153in}}%
\pgfpathlineto{\pgfqpoint{3.336324in}{3.763089in}}%
\pgfpathlineto{\pgfqpoint{3.325406in}{3.757017in}}%
\pgfpathlineto{\pgfqpoint{3.314485in}{3.750945in}}%
\pgfpathlineto{\pgfqpoint{3.320626in}{3.749800in}}%
\pgfpathlineto{\pgfqpoint{3.326772in}{3.748643in}}%
\pgfpathlineto{\pgfqpoint{3.332925in}{3.747472in}}%
\pgfpathlineto{\pgfqpoint{3.339083in}{3.746265in}}%
\pgfpathclose%
\pgfusepath{stroke,fill}%
\end{pgfscope}%
\begin{pgfscope}%
\pgfpathrectangle{\pgfqpoint{0.887500in}{0.275000in}}{\pgfqpoint{4.225000in}{4.225000in}}%
\pgfusepath{clip}%
\pgfsetbuttcap%
\pgfsetroundjoin%
\definecolor{currentfill}{rgb}{0.993248,0.906157,0.143936}%
\pgfsetfillcolor{currentfill}%
\pgfsetfillopacity{0.700000}%
\pgfsetlinewidth{0.501875pt}%
\definecolor{currentstroke}{rgb}{1.000000,1.000000,1.000000}%
\pgfsetstrokecolor{currentstroke}%
\pgfsetstrokeopacity{0.500000}%
\pgfsetdash{}{0pt}%
\pgfpathmoveto{\pgfqpoint{3.680317in}{3.854314in}}%
\pgfpathlineto{\pgfqpoint{3.691185in}{3.859059in}}%
\pgfpathlineto{\pgfqpoint{3.702049in}{3.863778in}}%
\pgfpathlineto{\pgfqpoint{3.712908in}{3.868468in}}%
\pgfpathlineto{\pgfqpoint{3.723762in}{3.873129in}}%
\pgfpathlineto{\pgfqpoint{3.717454in}{3.876346in}}%
\pgfpathlineto{\pgfqpoint{3.711140in}{3.878857in}}%
\pgfpathlineto{\pgfqpoint{3.704819in}{3.880594in}}%
\pgfpathlineto{\pgfqpoint{3.698492in}{3.881543in}}%
\pgfpathlineto{\pgfqpoint{3.692160in}{3.881757in}}%
\pgfpathlineto{\pgfqpoint{3.681339in}{3.877999in}}%
\pgfpathlineto{\pgfqpoint{3.670511in}{3.874148in}}%
\pgfpathlineto{\pgfqpoint{3.659677in}{3.870204in}}%
\pgfpathlineto{\pgfqpoint{3.648838in}{3.866170in}}%
\pgfpathlineto{\pgfqpoint{3.655139in}{3.864929in}}%
\pgfpathlineto{\pgfqpoint{3.661438in}{3.863138in}}%
\pgfpathlineto{\pgfqpoint{3.667733in}{3.860761in}}%
\pgfpathlineto{\pgfqpoint{3.674026in}{3.857805in}}%
\pgfpathclose%
\pgfusepath{stroke,fill}%
\end{pgfscope}%
\begin{pgfscope}%
\pgfpathrectangle{\pgfqpoint{0.887500in}{0.275000in}}{\pgfqpoint{4.225000in}{4.225000in}}%
\pgfusepath{clip}%
\pgfsetbuttcap%
\pgfsetroundjoin%
\definecolor{currentfill}{rgb}{0.626579,0.854645,0.223353}%
\pgfsetfillcolor{currentfill}%
\pgfsetfillopacity{0.700000}%
\pgfsetlinewidth{0.501875pt}%
\definecolor{currentstroke}{rgb}{1.000000,1.000000,1.000000}%
\pgfsetstrokecolor{currentstroke}%
\pgfsetstrokeopacity{0.500000}%
\pgfsetdash{}{0pt}%
\pgfpathmoveto{\pgfqpoint{2.955215in}{3.581047in}}%
\pgfpathlineto{\pgfqpoint{2.966224in}{3.586731in}}%
\pgfpathlineto{\pgfqpoint{2.977229in}{3.592527in}}%
\pgfpathlineto{\pgfqpoint{2.988230in}{3.598341in}}%
\pgfpathlineto{\pgfqpoint{2.999227in}{3.604079in}}%
\pgfpathlineto{\pgfqpoint{3.010221in}{3.609664in}}%
\pgfpathlineto{\pgfqpoint{3.004222in}{3.610948in}}%
\pgfpathlineto{\pgfqpoint{2.998228in}{3.612272in}}%
\pgfpathlineto{\pgfqpoint{2.992241in}{3.613634in}}%
\pgfpathlineto{\pgfqpoint{2.986259in}{3.615033in}}%
\pgfpathlineto{\pgfqpoint{2.980283in}{3.616468in}}%
\pgfpathlineto{\pgfqpoint{2.969310in}{3.610875in}}%
\pgfpathlineto{\pgfqpoint{2.958334in}{3.605146in}}%
\pgfpathlineto{\pgfqpoint{2.947354in}{3.599351in}}%
\pgfpathlineto{\pgfqpoint{2.936371in}{3.593576in}}%
\pgfpathlineto{\pgfqpoint{2.925383in}{3.587907in}}%
\pgfpathlineto{\pgfqpoint{2.931338in}{3.586444in}}%
\pgfpathlineto{\pgfqpoint{2.937299in}{3.585027in}}%
\pgfpathlineto{\pgfqpoint{2.943266in}{3.583656in}}%
\pgfpathlineto{\pgfqpoint{2.949238in}{3.582329in}}%
\pgfpathclose%
\pgfusepath{stroke,fill}%
\end{pgfscope}%
\begin{pgfscope}%
\pgfpathrectangle{\pgfqpoint{0.887500in}{0.275000in}}{\pgfqpoint{4.225000in}{4.225000in}}%
\pgfusepath{clip}%
\pgfsetbuttcap%
\pgfsetroundjoin%
\definecolor{currentfill}{rgb}{0.214000,0.722114,0.469588}%
\pgfsetfillcolor{currentfill}%
\pgfsetfillopacity{0.700000}%
\pgfsetlinewidth{0.501875pt}%
\definecolor{currentstroke}{rgb}{1.000000,1.000000,1.000000}%
\pgfsetstrokecolor{currentstroke}%
\pgfsetstrokeopacity{0.500000}%
\pgfsetdash{}{0pt}%
\pgfpathmoveto{\pgfqpoint{2.400511in}{3.201515in}}%
\pgfpathlineto{\pgfqpoint{2.411648in}{3.204913in}}%
\pgfpathlineto{\pgfqpoint{2.422775in}{3.208641in}}%
\pgfpathlineto{\pgfqpoint{2.433892in}{3.212624in}}%
\pgfpathlineto{\pgfqpoint{2.445004in}{3.216718in}}%
\pgfpathlineto{\pgfqpoint{2.456111in}{3.220779in}}%
\pgfpathlineto{\pgfqpoint{2.450260in}{3.230356in}}%
\pgfpathlineto{\pgfqpoint{2.444404in}{3.240506in}}%
\pgfpathlineto{\pgfqpoint{2.438541in}{3.251266in}}%
\pgfpathlineto{\pgfqpoint{2.432672in}{3.262674in}}%
\pgfpathlineto{\pgfqpoint{2.421574in}{3.258450in}}%
\pgfpathlineto{\pgfqpoint{2.410467in}{3.254516in}}%
\pgfpathlineto{\pgfqpoint{2.399348in}{3.250993in}}%
\pgfpathlineto{\pgfqpoint{2.388215in}{3.248004in}}%
\pgfpathlineto{\pgfqpoint{2.377067in}{3.245580in}}%
\pgfpathlineto{\pgfqpoint{2.382945in}{3.233297in}}%
\pgfpathlineto{\pgfqpoint{2.388811in}{3.221907in}}%
\pgfpathlineto{\pgfqpoint{2.394665in}{3.211337in}}%
\pgfpathclose%
\pgfusepath{stroke,fill}%
\end{pgfscope}%
\begin{pgfscope}%
\pgfpathrectangle{\pgfqpoint{0.887500in}{0.275000in}}{\pgfqpoint{4.225000in}{4.225000in}}%
\pgfusepath{clip}%
\pgfsetbuttcap%
\pgfsetroundjoin%
\definecolor{currentfill}{rgb}{0.196571,0.711827,0.479221}%
\pgfsetfillcolor{currentfill}%
\pgfsetfillopacity{0.700000}%
\pgfsetlinewidth{0.501875pt}%
\definecolor{currentstroke}{rgb}{1.000000,1.000000,1.000000}%
\pgfsetstrokecolor{currentstroke}%
\pgfsetstrokeopacity{0.500000}%
\pgfsetdash{}{0pt}%
\pgfpathmoveto{\pgfqpoint{2.485310in}{3.180141in}}%
\pgfpathlineto{\pgfqpoint{2.496424in}{3.184142in}}%
\pgfpathlineto{\pgfqpoint{2.507538in}{3.187861in}}%
\pgfpathlineto{\pgfqpoint{2.518653in}{3.191150in}}%
\pgfpathlineto{\pgfqpoint{2.529768in}{3.193988in}}%
\pgfpathlineto{\pgfqpoint{2.540878in}{3.196772in}}%
\pgfpathlineto{\pgfqpoint{2.535044in}{3.203096in}}%
\pgfpathlineto{\pgfqpoint{2.529203in}{3.210190in}}%
\pgfpathlineto{\pgfqpoint{2.523356in}{3.218088in}}%
\pgfpathlineto{\pgfqpoint{2.517501in}{3.226827in}}%
\pgfpathlineto{\pgfqpoint{2.511638in}{3.236443in}}%
\pgfpathlineto{\pgfqpoint{2.500534in}{3.233928in}}%
\pgfpathlineto{\pgfqpoint{2.489427in}{3.231318in}}%
\pgfpathlineto{\pgfqpoint{2.478321in}{3.228223in}}%
\pgfpathlineto{\pgfqpoint{2.467216in}{3.224662in}}%
\pgfpathlineto{\pgfqpoint{2.456111in}{3.220779in}}%
\pgfpathlineto{\pgfqpoint{2.461957in}{3.211735in}}%
\pgfpathlineto{\pgfqpoint{2.467799in}{3.203188in}}%
\pgfpathlineto{\pgfqpoint{2.473638in}{3.195098in}}%
\pgfpathlineto{\pgfqpoint{2.479475in}{3.187429in}}%
\pgfpathclose%
\pgfusepath{stroke,fill}%
\end{pgfscope}%
\begin{pgfscope}%
\pgfpathrectangle{\pgfqpoint{0.887500in}{0.275000in}}{\pgfqpoint{4.225000in}{4.225000in}}%
\pgfusepath{clip}%
\pgfsetbuttcap%
\pgfsetroundjoin%
\definecolor{currentfill}{rgb}{0.926106,0.897330,0.104071}%
\pgfsetfillcolor{currentfill}%
\pgfsetfillopacity{0.700000}%
\pgfsetlinewidth{0.501875pt}%
\definecolor{currentstroke}{rgb}{1.000000,1.000000,1.000000}%
\pgfsetstrokecolor{currentstroke}%
\pgfsetstrokeopacity{0.500000}%
\pgfsetdash{}{0pt}%
\pgfpathmoveto{\pgfqpoint{3.485595in}{3.795830in}}%
\pgfpathlineto{\pgfqpoint{3.496514in}{3.801141in}}%
\pgfpathlineto{\pgfqpoint{3.507428in}{3.806332in}}%
\pgfpathlineto{\pgfqpoint{3.518337in}{3.811414in}}%
\pgfpathlineto{\pgfqpoint{3.529240in}{3.816400in}}%
\pgfpathlineto{\pgfqpoint{3.540139in}{3.821301in}}%
\pgfpathlineto{\pgfqpoint{3.533894in}{3.823518in}}%
\pgfpathlineto{\pgfqpoint{3.527653in}{3.825467in}}%
\pgfpathlineto{\pgfqpoint{3.521414in}{3.827164in}}%
\pgfpathlineto{\pgfqpoint{3.515179in}{3.828628in}}%
\pgfpathlineto{\pgfqpoint{3.508947in}{3.829877in}}%
\pgfpathlineto{\pgfqpoint{3.498070in}{3.825218in}}%
\pgfpathlineto{\pgfqpoint{3.487187in}{3.820372in}}%
\pgfpathlineto{\pgfqpoint{3.476299in}{3.815345in}}%
\pgfpathlineto{\pgfqpoint{3.465404in}{3.810143in}}%
\pgfpathlineto{\pgfqpoint{3.454504in}{3.804773in}}%
\pgfpathlineto{\pgfqpoint{3.460715in}{3.803389in}}%
\pgfpathlineto{\pgfqpoint{3.466929in}{3.801824in}}%
\pgfpathlineto{\pgfqpoint{3.473148in}{3.800057in}}%
\pgfpathlineto{\pgfqpoint{3.479369in}{3.798066in}}%
\pgfpathclose%
\pgfusepath{stroke,fill}%
\end{pgfscope}%
\begin{pgfscope}%
\pgfpathrectangle{\pgfqpoint{0.887500in}{0.275000in}}{\pgfqpoint{4.225000in}{4.225000in}}%
\pgfusepath{clip}%
\pgfsetbuttcap%
\pgfsetroundjoin%
\definecolor{currentfill}{rgb}{0.824940,0.884720,0.106217}%
\pgfsetfillcolor{currentfill}%
\pgfsetfillopacity{0.700000}%
\pgfsetlinewidth{0.501875pt}%
\definecolor{currentstroke}{rgb}{1.000000,1.000000,1.000000}%
\pgfsetstrokecolor{currentstroke}%
\pgfsetstrokeopacity{0.500000}%
\pgfsetdash{}{0pt}%
\pgfpathmoveto{\pgfqpoint{3.290478in}{3.714151in}}%
\pgfpathlineto{\pgfqpoint{3.301439in}{3.720365in}}%
\pgfpathlineto{\pgfqpoint{3.312397in}{3.726547in}}%
\pgfpathlineto{\pgfqpoint{3.323350in}{3.732708in}}%
\pgfpathlineto{\pgfqpoint{3.334300in}{3.738857in}}%
\pgfpathlineto{\pgfqpoint{3.345247in}{3.745001in}}%
\pgfpathlineto{\pgfqpoint{3.339083in}{3.746265in}}%
\pgfpathlineto{\pgfqpoint{3.332925in}{3.747472in}}%
\pgfpathlineto{\pgfqpoint{3.326772in}{3.748643in}}%
\pgfpathlineto{\pgfqpoint{3.320626in}{3.749800in}}%
\pgfpathlineto{\pgfqpoint{3.314485in}{3.750945in}}%
\pgfpathlineto{\pgfqpoint{3.303560in}{3.744867in}}%
\pgfpathlineto{\pgfqpoint{3.292632in}{3.738779in}}%
\pgfpathlineto{\pgfqpoint{3.281700in}{3.732675in}}%
\pgfpathlineto{\pgfqpoint{3.270764in}{3.726549in}}%
\pgfpathlineto{\pgfqpoint{3.259825in}{3.720397in}}%
\pgfpathlineto{\pgfqpoint{3.265944in}{3.719155in}}%
\pgfpathlineto{\pgfqpoint{3.272069in}{3.717918in}}%
\pgfpathlineto{\pgfqpoint{3.278199in}{3.716685in}}%
\pgfpathlineto{\pgfqpoint{3.284336in}{3.715437in}}%
\pgfpathclose%
\pgfusepath{stroke,fill}%
\end{pgfscope}%
\begin{pgfscope}%
\pgfpathrectangle{\pgfqpoint{0.887500in}{0.275000in}}{\pgfqpoint{4.225000in}{4.225000in}}%
\pgfusepath{clip}%
\pgfsetbuttcap%
\pgfsetroundjoin%
\definecolor{currentfill}{rgb}{0.699415,0.867117,0.175971}%
\pgfsetfillcolor{currentfill}%
\pgfsetfillopacity{0.700000}%
\pgfsetlinewidth{0.501875pt}%
\definecolor{currentstroke}{rgb}{1.000000,1.000000,1.000000}%
\pgfsetstrokecolor{currentstroke}%
\pgfsetstrokeopacity{0.500000}%
\pgfsetdash{}{0pt}%
\pgfpathmoveto{\pgfqpoint{3.095319in}{3.630867in}}%
\pgfpathlineto{\pgfqpoint{3.106311in}{3.636321in}}%
\pgfpathlineto{\pgfqpoint{3.117299in}{3.641876in}}%
\pgfpathlineto{\pgfqpoint{3.128283in}{3.647534in}}%
\pgfpathlineto{\pgfqpoint{3.139264in}{3.653285in}}%
\pgfpathlineto{\pgfqpoint{3.150241in}{3.659125in}}%
\pgfpathlineto{\pgfqpoint{3.144171in}{3.660282in}}%
\pgfpathlineto{\pgfqpoint{3.138106in}{3.661487in}}%
\pgfpathlineto{\pgfqpoint{3.132047in}{3.662735in}}%
\pgfpathlineto{\pgfqpoint{3.125995in}{3.664018in}}%
\pgfpathlineto{\pgfqpoint{3.119948in}{3.665333in}}%
\pgfpathlineto{\pgfqpoint{3.108992in}{3.659347in}}%
\pgfpathlineto{\pgfqpoint{3.098032in}{3.653453in}}%
\pgfpathlineto{\pgfqpoint{3.087069in}{3.647673in}}%
\pgfpathlineto{\pgfqpoint{3.076102in}{3.642027in}}%
\pgfpathlineto{\pgfqpoint{3.065132in}{3.636525in}}%
\pgfpathlineto{\pgfqpoint{3.071158in}{3.635266in}}%
\pgfpathlineto{\pgfqpoint{3.077189in}{3.634064in}}%
\pgfpathlineto{\pgfqpoint{3.083227in}{3.632927in}}%
\pgfpathlineto{\pgfqpoint{3.089270in}{3.631859in}}%
\pgfpathclose%
\pgfusepath{stroke,fill}%
\end{pgfscope}%
\begin{pgfscope}%
\pgfpathrectangle{\pgfqpoint{0.887500in}{0.275000in}}{\pgfqpoint{4.225000in}{4.225000in}}%
\pgfusepath{clip}%
\pgfsetbuttcap%
\pgfsetroundjoin%
\definecolor{currentfill}{rgb}{0.974417,0.903590,0.130215}%
\pgfsetfillcolor{currentfill}%
\pgfsetfillopacity{0.700000}%
\pgfsetlinewidth{0.501875pt}%
\definecolor{currentstroke}{rgb}{1.000000,1.000000,1.000000}%
\pgfsetstrokecolor{currentstroke}%
\pgfsetstrokeopacity{0.500000}%
\pgfsetdash{}{0pt}%
\pgfpathmoveto{\pgfqpoint{3.625908in}{3.830225in}}%
\pgfpathlineto{\pgfqpoint{3.636799in}{3.835087in}}%
\pgfpathlineto{\pgfqpoint{3.647685in}{3.839929in}}%
\pgfpathlineto{\pgfqpoint{3.658567in}{3.844748in}}%
\pgfpathlineto{\pgfqpoint{3.669444in}{3.849543in}}%
\pgfpathlineto{\pgfqpoint{3.680317in}{3.854314in}}%
\pgfpathlineto{\pgfqpoint{3.674026in}{3.857805in}}%
\pgfpathlineto{\pgfqpoint{3.667733in}{3.860761in}}%
\pgfpathlineto{\pgfqpoint{3.661438in}{3.863138in}}%
\pgfpathlineto{\pgfqpoint{3.655139in}{3.864929in}}%
\pgfpathlineto{\pgfqpoint{3.648838in}{3.866170in}}%
\pgfpathlineto{\pgfqpoint{3.637993in}{3.862048in}}%
\pgfpathlineto{\pgfqpoint{3.627141in}{3.857839in}}%
\pgfpathlineto{\pgfqpoint{3.616285in}{3.853546in}}%
\pgfpathlineto{\pgfqpoint{3.605423in}{3.849171in}}%
\pgfpathlineto{\pgfqpoint{3.594555in}{3.844716in}}%
\pgfpathlineto{\pgfqpoint{3.600825in}{3.842613in}}%
\pgfpathlineto{\pgfqpoint{3.607096in}{3.840123in}}%
\pgfpathlineto{\pgfqpoint{3.613366in}{3.837224in}}%
\pgfpathlineto{\pgfqpoint{3.619637in}{3.833918in}}%
\pgfpathclose%
\pgfusepath{stroke,fill}%
\end{pgfscope}%
\begin{pgfscope}%
\pgfpathrectangle{\pgfqpoint{0.887500in}{0.275000in}}{\pgfqpoint{4.225000in}{4.225000in}}%
\pgfusepath{clip}%
\pgfsetbuttcap%
\pgfsetroundjoin%
\definecolor{currentfill}{rgb}{0.606045,0.850733,0.236712}%
\pgfsetfillcolor{currentfill}%
\pgfsetfillopacity{0.700000}%
\pgfsetlinewidth{0.501875pt}%
\definecolor{currentstroke}{rgb}{1.000000,1.000000,1.000000}%
\pgfsetstrokecolor{currentstroke}%
\pgfsetstrokeopacity{0.500000}%
\pgfsetdash{}{0pt}%
\pgfpathmoveto{\pgfqpoint{2.900099in}{3.556724in}}%
\pgfpathlineto{\pgfqpoint{2.911132in}{3.561166in}}%
\pgfpathlineto{\pgfqpoint{2.922161in}{3.565608in}}%
\pgfpathlineto{\pgfqpoint{2.933184in}{3.570392in}}%
\pgfpathlineto{\pgfqpoint{2.944202in}{3.575569in}}%
\pgfpathlineto{\pgfqpoint{2.955215in}{3.581047in}}%
\pgfpathlineto{\pgfqpoint{2.949238in}{3.582329in}}%
\pgfpathlineto{\pgfqpoint{2.943266in}{3.583656in}}%
\pgfpathlineto{\pgfqpoint{2.937299in}{3.585027in}}%
\pgfpathlineto{\pgfqpoint{2.931338in}{3.586444in}}%
\pgfpathlineto{\pgfqpoint{2.925383in}{3.587907in}}%
\pgfpathlineto{\pgfqpoint{2.914391in}{3.582427in}}%
\pgfpathlineto{\pgfqpoint{2.903394in}{3.577224in}}%
\pgfpathlineto{\pgfqpoint{2.892392in}{3.572381in}}%
\pgfpathlineto{\pgfqpoint{2.881384in}{3.567852in}}%
\pgfpathlineto{\pgfqpoint{2.870371in}{3.563330in}}%
\pgfpathlineto{\pgfqpoint{2.876306in}{3.561948in}}%
\pgfpathlineto{\pgfqpoint{2.882246in}{3.560591in}}%
\pgfpathlineto{\pgfqpoint{2.888191in}{3.559265in}}%
\pgfpathlineto{\pgfqpoint{2.894142in}{3.557973in}}%
\pgfpathclose%
\pgfusepath{stroke,fill}%
\end{pgfscope}%
\begin{pgfscope}%
\pgfpathrectangle{\pgfqpoint{0.887500in}{0.275000in}}{\pgfqpoint{4.225000in}{4.225000in}}%
\pgfusepath{clip}%
\pgfsetbuttcap%
\pgfsetroundjoin%
\definecolor{currentfill}{rgb}{0.239374,0.735588,0.455688}%
\pgfsetfillcolor{currentfill}%
\pgfsetfillopacity{0.700000}%
\pgfsetlinewidth{0.501875pt}%
\definecolor{currentstroke}{rgb}{1.000000,1.000000,1.000000}%
\pgfsetstrokecolor{currentstroke}%
\pgfsetstrokeopacity{0.500000}%
\pgfsetdash{}{0pt}%
\pgfpathmoveto{\pgfqpoint{2.625250in}{3.216314in}}%
\pgfpathlineto{\pgfqpoint{2.636250in}{3.228779in}}%
\pgfpathlineto{\pgfqpoint{2.647235in}{3.242658in}}%
\pgfpathlineto{\pgfqpoint{2.658211in}{3.257637in}}%
\pgfpathlineto{\pgfqpoint{2.669180in}{3.273398in}}%
\pgfpathlineto{\pgfqpoint{2.680146in}{3.289626in}}%
\pgfpathlineto{\pgfqpoint{2.674301in}{3.290684in}}%
\pgfpathlineto{\pgfqpoint{2.668452in}{3.292697in}}%
\pgfpathlineto{\pgfqpoint{2.662599in}{3.295609in}}%
\pgfpathlineto{\pgfqpoint{2.656744in}{3.299367in}}%
\pgfpathlineto{\pgfqpoint{2.650886in}{3.303919in}}%
\pgfpathlineto{\pgfqpoint{2.639961in}{3.286348in}}%
\pgfpathlineto{\pgfqpoint{2.629032in}{3.269320in}}%
\pgfpathlineto{\pgfqpoint{2.618095in}{3.253358in}}%
\pgfpathlineto{\pgfqpoint{2.607142in}{3.238982in}}%
\pgfpathlineto{\pgfqpoint{2.596166in}{3.226712in}}%
\pgfpathlineto{\pgfqpoint{2.601995in}{3.222567in}}%
\pgfpathlineto{\pgfqpoint{2.607817in}{3.219477in}}%
\pgfpathlineto{\pgfqpoint{2.613634in}{3.217420in}}%
\pgfpathlineto{\pgfqpoint{2.619444in}{3.216374in}}%
\pgfpathclose%
\pgfusepath{stroke,fill}%
\end{pgfscope}%
\begin{pgfscope}%
\pgfpathrectangle{\pgfqpoint{0.887500in}{0.275000in}}{\pgfqpoint{4.225000in}{4.225000in}}%
\pgfusepath{clip}%
\pgfsetbuttcap%
\pgfsetroundjoin%
\definecolor{currentfill}{rgb}{0.196571,0.711827,0.479221}%
\pgfsetfillcolor{currentfill}%
\pgfsetfillopacity{0.700000}%
\pgfsetlinewidth{0.501875pt}%
\definecolor{currentstroke}{rgb}{1.000000,1.000000,1.000000}%
\pgfsetstrokecolor{currentstroke}%
\pgfsetstrokeopacity{0.500000}%
\pgfsetdash{}{0pt}%
\pgfpathmoveto{\pgfqpoint{2.569990in}{3.175410in}}%
\pgfpathlineto{\pgfqpoint{2.581071in}{3.181538in}}%
\pgfpathlineto{\pgfqpoint{2.592140in}{3.188418in}}%
\pgfpathlineto{\pgfqpoint{2.603194in}{3.196324in}}%
\pgfpathlineto{\pgfqpoint{2.614232in}{3.205531in}}%
\pgfpathlineto{\pgfqpoint{2.625250in}{3.216314in}}%
\pgfpathlineto{\pgfqpoint{2.619444in}{3.216374in}}%
\pgfpathlineto{\pgfqpoint{2.613634in}{3.217420in}}%
\pgfpathlineto{\pgfqpoint{2.607817in}{3.219477in}}%
\pgfpathlineto{\pgfqpoint{2.601995in}{3.222567in}}%
\pgfpathlineto{\pgfqpoint{2.596166in}{3.226712in}}%
\pgfpathlineto{\pgfqpoint{2.585159in}{3.216985in}}%
\pgfpathlineto{\pgfqpoint{2.574123in}{3.209608in}}%
\pgfpathlineto{\pgfqpoint{2.563061in}{3.204100in}}%
\pgfpathlineto{\pgfqpoint{2.551978in}{3.199982in}}%
\pgfpathlineto{\pgfqpoint{2.540878in}{3.196772in}}%
\pgfpathlineto{\pgfqpoint{2.546708in}{3.191179in}}%
\pgfpathlineto{\pgfqpoint{2.552533in}{3.186283in}}%
\pgfpathlineto{\pgfqpoint{2.558354in}{3.182047in}}%
\pgfpathlineto{\pgfqpoint{2.564173in}{3.178434in}}%
\pgfpathclose%
\pgfusepath{stroke,fill}%
\end{pgfscope}%
\begin{pgfscope}%
\pgfpathrectangle{\pgfqpoint{0.887500in}{0.275000in}}{\pgfqpoint{4.225000in}{4.225000in}}%
\pgfusepath{clip}%
\pgfsetbuttcap%
\pgfsetroundjoin%
\definecolor{currentfill}{rgb}{0.208030,0.718701,0.472873}%
\pgfsetfillcolor{currentfill}%
\pgfsetfillopacity{0.700000}%
\pgfsetlinewidth{0.501875pt}%
\definecolor{currentstroke}{rgb}{1.000000,1.000000,1.000000}%
\pgfsetstrokecolor{currentstroke}%
\pgfsetstrokeopacity{0.500000}%
\pgfsetdash{}{0pt}%
\pgfpathmoveto{\pgfqpoint{2.344768in}{3.183387in}}%
\pgfpathlineto{\pgfqpoint{2.355912in}{3.187812in}}%
\pgfpathlineto{\pgfqpoint{2.367063in}{3.191643in}}%
\pgfpathlineto{\pgfqpoint{2.378215in}{3.195074in}}%
\pgfpathlineto{\pgfqpoint{2.389365in}{3.198301in}}%
\pgfpathlineto{\pgfqpoint{2.400511in}{3.201515in}}%
\pgfpathlineto{\pgfqpoint{2.394665in}{3.211337in}}%
\pgfpathlineto{\pgfqpoint{2.388811in}{3.221907in}}%
\pgfpathlineto{\pgfqpoint{2.382945in}{3.233297in}}%
\pgfpathlineto{\pgfqpoint{2.377067in}{3.245580in}}%
\pgfpathlineto{\pgfqpoint{2.365909in}{3.243458in}}%
\pgfpathlineto{\pgfqpoint{2.354745in}{3.241317in}}%
\pgfpathlineto{\pgfqpoint{2.343581in}{3.238832in}}%
\pgfpathlineto{\pgfqpoint{2.332424in}{3.235682in}}%
\pgfpathlineto{\pgfqpoint{2.321279in}{3.231542in}}%
\pgfpathlineto{\pgfqpoint{2.327181in}{3.217656in}}%
\pgfpathlineto{\pgfqpoint{2.333061in}{3.205076in}}%
\pgfpathlineto{\pgfqpoint{2.338923in}{3.193691in}}%
\pgfpathclose%
\pgfusepath{stroke,fill}%
\end{pgfscope}%
\begin{pgfscope}%
\pgfpathrectangle{\pgfqpoint{0.887500in}{0.275000in}}{\pgfqpoint{4.225000in}{4.225000in}}%
\pgfusepath{clip}%
\pgfsetbuttcap%
\pgfsetroundjoin%
\definecolor{currentfill}{rgb}{0.327796,0.773980,0.406640}%
\pgfsetfillcolor{currentfill}%
\pgfsetfillopacity{0.700000}%
\pgfsetlinewidth{0.501875pt}%
\definecolor{currentstroke}{rgb}{1.000000,1.000000,1.000000}%
\pgfsetstrokecolor{currentstroke}%
\pgfsetstrokeopacity{0.500000}%
\pgfsetdash{}{0pt}%
\pgfpathmoveto{\pgfqpoint{2.680146in}{3.289626in}}%
\pgfpathlineto{\pgfqpoint{2.691113in}{3.306003in}}%
\pgfpathlineto{\pgfqpoint{2.702083in}{3.322214in}}%
\pgfpathlineto{\pgfqpoint{2.713058in}{3.338107in}}%
\pgfpathlineto{\pgfqpoint{2.724037in}{3.353772in}}%
\pgfpathlineto{\pgfqpoint{2.735018in}{3.369320in}}%
\pgfpathlineto{\pgfqpoint{2.729135in}{3.371343in}}%
\pgfpathlineto{\pgfqpoint{2.723252in}{3.374108in}}%
\pgfpathlineto{\pgfqpoint{2.717368in}{3.377524in}}%
\pgfpathlineto{\pgfqpoint{2.711485in}{3.381496in}}%
\pgfpathlineto{\pgfqpoint{2.705602in}{3.385933in}}%
\pgfpathlineto{\pgfqpoint{2.694645in}{3.370663in}}%
\pgfpathlineto{\pgfqpoint{2.683693in}{3.354944in}}%
\pgfpathlineto{\pgfqpoint{2.672748in}{3.338605in}}%
\pgfpathlineto{\pgfqpoint{2.661813in}{3.321511in}}%
\pgfpathlineto{\pgfqpoint{2.650886in}{3.303919in}}%
\pgfpathlineto{\pgfqpoint{2.656744in}{3.299367in}}%
\pgfpathlineto{\pgfqpoint{2.662599in}{3.295609in}}%
\pgfpathlineto{\pgfqpoint{2.668452in}{3.292697in}}%
\pgfpathlineto{\pgfqpoint{2.674301in}{3.290684in}}%
\pgfpathclose%
\pgfusepath{stroke,fill}%
\end{pgfscope}%
\begin{pgfscope}%
\pgfpathrectangle{\pgfqpoint{0.887500in}{0.275000in}}{\pgfqpoint{4.225000in}{4.225000in}}%
\pgfusepath{clip}%
\pgfsetbuttcap%
\pgfsetroundjoin%
\definecolor{currentfill}{rgb}{0.906311,0.894855,0.098125}%
\pgfsetfillcolor{currentfill}%
\pgfsetfillopacity{0.700000}%
\pgfsetlinewidth{0.501875pt}%
\definecolor{currentstroke}{rgb}{1.000000,1.000000,1.000000}%
\pgfsetstrokecolor{currentstroke}%
\pgfsetstrokeopacity{0.500000}%
\pgfsetdash{}{0pt}%
\pgfpathmoveto{\pgfqpoint{3.430921in}{3.767283in}}%
\pgfpathlineto{\pgfqpoint{3.441866in}{3.773224in}}%
\pgfpathlineto{\pgfqpoint{3.452805in}{3.779070in}}%
\pgfpathlineto{\pgfqpoint{3.463740in}{3.784799in}}%
\pgfpathlineto{\pgfqpoint{3.474670in}{3.790387in}}%
\pgfpathlineto{\pgfqpoint{3.485595in}{3.795830in}}%
\pgfpathlineto{\pgfqpoint{3.479369in}{3.798066in}}%
\pgfpathlineto{\pgfqpoint{3.473148in}{3.800057in}}%
\pgfpathlineto{\pgfqpoint{3.466929in}{3.801824in}}%
\pgfpathlineto{\pgfqpoint{3.460715in}{3.803389in}}%
\pgfpathlineto{\pgfqpoint{3.454504in}{3.804773in}}%
\pgfpathlineto{\pgfqpoint{3.443599in}{3.799240in}}%
\pgfpathlineto{\pgfqpoint{3.432688in}{3.793552in}}%
\pgfpathlineto{\pgfqpoint{3.421772in}{3.787722in}}%
\pgfpathlineto{\pgfqpoint{3.410852in}{3.781775in}}%
\pgfpathlineto{\pgfqpoint{3.399927in}{3.775735in}}%
\pgfpathlineto{\pgfqpoint{3.406117in}{3.774377in}}%
\pgfpathlineto{\pgfqpoint{3.412312in}{3.772877in}}%
\pgfpathlineto{\pgfqpoint{3.418511in}{3.771210in}}%
\pgfpathlineto{\pgfqpoint{3.424714in}{3.769353in}}%
\pgfpathclose%
\pgfusepath{stroke,fill}%
\end{pgfscope}%
\begin{pgfscope}%
\pgfpathrectangle{\pgfqpoint{0.887500in}{0.275000in}}{\pgfqpoint{4.225000in}{4.225000in}}%
\pgfusepath{clip}%
\pgfsetbuttcap%
\pgfsetroundjoin%
\definecolor{currentfill}{rgb}{0.793760,0.880678,0.120005}%
\pgfsetfillcolor{currentfill}%
\pgfsetfillopacity{0.700000}%
\pgfsetlinewidth{0.501875pt}%
\definecolor{currentstroke}{rgb}{1.000000,1.000000,1.000000}%
\pgfsetstrokecolor{currentstroke}%
\pgfsetstrokeopacity{0.500000}%
\pgfsetdash{}{0pt}%
\pgfpathmoveto{\pgfqpoint{3.235621in}{3.682990in}}%
\pgfpathlineto{\pgfqpoint{3.246599in}{3.689145in}}%
\pgfpathlineto{\pgfqpoint{3.257574in}{3.695368in}}%
\pgfpathlineto{\pgfqpoint{3.268546in}{3.701630in}}%
\pgfpathlineto{\pgfqpoint{3.279514in}{3.707901in}}%
\pgfpathlineto{\pgfqpoint{3.290478in}{3.714151in}}%
\pgfpathlineto{\pgfqpoint{3.284336in}{3.715437in}}%
\pgfpathlineto{\pgfqpoint{3.278199in}{3.716685in}}%
\pgfpathlineto{\pgfqpoint{3.272069in}{3.717918in}}%
\pgfpathlineto{\pgfqpoint{3.265944in}{3.719155in}}%
\pgfpathlineto{\pgfqpoint{3.259825in}{3.720397in}}%
\pgfpathlineto{\pgfqpoint{3.248882in}{3.714215in}}%
\pgfpathlineto{\pgfqpoint{3.237936in}{3.708008in}}%
\pgfpathlineto{\pgfqpoint{3.226986in}{3.701791in}}%
\pgfpathlineto{\pgfqpoint{3.216032in}{3.695578in}}%
\pgfpathlineto{\pgfqpoint{3.205075in}{3.689383in}}%
\pgfpathlineto{\pgfqpoint{3.211173in}{3.688089in}}%
\pgfpathlineto{\pgfqpoint{3.217276in}{3.686812in}}%
\pgfpathlineto{\pgfqpoint{3.223385in}{3.685549in}}%
\pgfpathlineto{\pgfqpoint{3.229500in}{3.684282in}}%
\pgfpathclose%
\pgfusepath{stroke,fill}%
\end{pgfscope}%
\begin{pgfscope}%
\pgfpathrectangle{\pgfqpoint{0.887500in}{0.275000in}}{\pgfqpoint{4.225000in}{4.225000in}}%
\pgfusepath{clip}%
\pgfsetbuttcap%
\pgfsetroundjoin%
\definecolor{currentfill}{rgb}{0.191090,0.708366,0.482284}%
\pgfsetfillcolor{currentfill}%
\pgfsetfillopacity{0.700000}%
\pgfsetlinewidth{0.501875pt}%
\definecolor{currentstroke}{rgb}{1.000000,1.000000,1.000000}%
\pgfsetstrokecolor{currentstroke}%
\pgfsetstrokeopacity{0.500000}%
\pgfsetdash{}{0pt}%
\pgfpathmoveto{\pgfqpoint{2.429649in}{3.161065in}}%
\pgfpathlineto{\pgfqpoint{2.440800in}{3.164307in}}%
\pgfpathlineto{\pgfqpoint{2.451939in}{3.167942in}}%
\pgfpathlineto{\pgfqpoint{2.463069in}{3.171892in}}%
\pgfpathlineto{\pgfqpoint{2.474192in}{3.176008in}}%
\pgfpathlineto{\pgfqpoint{2.485310in}{3.180141in}}%
\pgfpathlineto{\pgfqpoint{2.479475in}{3.187429in}}%
\pgfpathlineto{\pgfqpoint{2.473638in}{3.195098in}}%
\pgfpathlineto{\pgfqpoint{2.467799in}{3.203188in}}%
\pgfpathlineto{\pgfqpoint{2.461957in}{3.211735in}}%
\pgfpathlineto{\pgfqpoint{2.456111in}{3.220779in}}%
\pgfpathlineto{\pgfqpoint{2.445004in}{3.216718in}}%
\pgfpathlineto{\pgfqpoint{2.433892in}{3.212624in}}%
\pgfpathlineto{\pgfqpoint{2.422775in}{3.208641in}}%
\pgfpathlineto{\pgfqpoint{2.411648in}{3.204913in}}%
\pgfpathlineto{\pgfqpoint{2.400511in}{3.201515in}}%
\pgfpathlineto{\pgfqpoint{2.406348in}{3.192368in}}%
\pgfpathlineto{\pgfqpoint{2.412179in}{3.183822in}}%
\pgfpathlineto{\pgfqpoint{2.418006in}{3.175804in}}%
\pgfpathlineto{\pgfqpoint{2.423829in}{3.168243in}}%
\pgfpathclose%
\pgfusepath{stroke,fill}%
\end{pgfscope}%
\begin{pgfscope}%
\pgfpathrectangle{\pgfqpoint{0.887500in}{0.275000in}}{\pgfqpoint{4.225000in}{4.225000in}}%
\pgfusepath{clip}%
\pgfsetbuttcap%
\pgfsetroundjoin%
\definecolor{currentfill}{rgb}{0.678489,0.863742,0.189503}%
\pgfsetfillcolor{currentfill}%
\pgfsetfillopacity{0.700000}%
\pgfsetlinewidth{0.501875pt}%
\definecolor{currentstroke}{rgb}{1.000000,1.000000,1.000000}%
\pgfsetstrokecolor{currentstroke}%
\pgfsetstrokeopacity{0.500000}%
\pgfsetdash{}{0pt}%
\pgfpathmoveto{\pgfqpoint{3.040302in}{3.603877in}}%
\pgfpathlineto{\pgfqpoint{3.051313in}{3.609367in}}%
\pgfpathlineto{\pgfqpoint{3.062321in}{3.614773in}}%
\pgfpathlineto{\pgfqpoint{3.073324in}{3.620132in}}%
\pgfpathlineto{\pgfqpoint{3.084324in}{3.625483in}}%
\pgfpathlineto{\pgfqpoint{3.095319in}{3.630867in}}%
\pgfpathlineto{\pgfqpoint{3.089270in}{3.631859in}}%
\pgfpathlineto{\pgfqpoint{3.083227in}{3.632927in}}%
\pgfpathlineto{\pgfqpoint{3.077189in}{3.634064in}}%
\pgfpathlineto{\pgfqpoint{3.071158in}{3.635266in}}%
\pgfpathlineto{\pgfqpoint{3.065132in}{3.636525in}}%
\pgfpathlineto{\pgfqpoint{3.054158in}{3.631132in}}%
\pgfpathlineto{\pgfqpoint{3.043179in}{3.625798in}}%
\pgfpathlineto{\pgfqpoint{3.032197in}{3.620474in}}%
\pgfpathlineto{\pgfqpoint{3.021211in}{3.615113in}}%
\pgfpathlineto{\pgfqpoint{3.010221in}{3.609664in}}%
\pgfpathlineto{\pgfqpoint{3.016225in}{3.608420in}}%
\pgfpathlineto{\pgfqpoint{3.022236in}{3.607219in}}%
\pgfpathlineto{\pgfqpoint{3.028252in}{3.606060in}}%
\pgfpathlineto{\pgfqpoint{3.034274in}{3.604946in}}%
\pgfpathclose%
\pgfusepath{stroke,fill}%
\end{pgfscope}%
\begin{pgfscope}%
\pgfpathrectangle{\pgfqpoint{0.887500in}{0.275000in}}{\pgfqpoint{4.225000in}{4.225000in}}%
\pgfusepath{clip}%
\pgfsetbuttcap%
\pgfsetroundjoin%
\definecolor{currentfill}{rgb}{0.421908,0.805774,0.351910}%
\pgfsetfillcolor{currentfill}%
\pgfsetfillopacity{0.700000}%
\pgfsetlinewidth{0.501875pt}%
\definecolor{currentstroke}{rgb}{1.000000,1.000000,1.000000}%
\pgfsetstrokecolor{currentstroke}%
\pgfsetstrokeopacity{0.500000}%
\pgfsetdash{}{0pt}%
\pgfpathmoveto{\pgfqpoint{2.735018in}{3.369320in}}%
\pgfpathlineto{\pgfqpoint{2.746001in}{3.384863in}}%
\pgfpathlineto{\pgfqpoint{2.756984in}{3.400512in}}%
\pgfpathlineto{\pgfqpoint{2.767968in}{3.416378in}}%
\pgfpathlineto{\pgfqpoint{2.778951in}{3.432566in}}%
\pgfpathlineto{\pgfqpoint{2.789935in}{3.448955in}}%
\pgfpathlineto{\pgfqpoint{2.784026in}{3.450645in}}%
\pgfpathlineto{\pgfqpoint{2.778120in}{3.452786in}}%
\pgfpathlineto{\pgfqpoint{2.772216in}{3.455308in}}%
\pgfpathlineto{\pgfqpoint{2.766316in}{3.458142in}}%
\pgfpathlineto{\pgfqpoint{2.760419in}{3.461218in}}%
\pgfpathlineto{\pgfqpoint{2.749455in}{3.445903in}}%
\pgfpathlineto{\pgfqpoint{2.738491in}{3.430742in}}%
\pgfpathlineto{\pgfqpoint{2.727527in}{3.415802in}}%
\pgfpathlineto{\pgfqpoint{2.716563in}{3.400923in}}%
\pgfpathlineto{\pgfqpoint{2.705602in}{3.385933in}}%
\pgfpathlineto{\pgfqpoint{2.711485in}{3.381496in}}%
\pgfpathlineto{\pgfqpoint{2.717368in}{3.377524in}}%
\pgfpathlineto{\pgfqpoint{2.723252in}{3.374108in}}%
\pgfpathlineto{\pgfqpoint{2.729135in}{3.371343in}}%
\pgfpathclose%
\pgfusepath{stroke,fill}%
\end{pgfscope}%
\begin{pgfscope}%
\pgfpathrectangle{\pgfqpoint{0.887500in}{0.275000in}}{\pgfqpoint{4.225000in}{4.225000in}}%
\pgfusepath{clip}%
\pgfsetbuttcap%
\pgfsetroundjoin%
\definecolor{currentfill}{rgb}{0.575563,0.844566,0.256415}%
\pgfsetfillcolor{currentfill}%
\pgfsetfillopacity{0.700000}%
\pgfsetlinewidth{0.501875pt}%
\definecolor{currentstroke}{rgb}{1.000000,1.000000,1.000000}%
\pgfsetstrokecolor{currentstroke}%
\pgfsetstrokeopacity{0.500000}%
\pgfsetdash{}{0pt}%
\pgfpathmoveto{\pgfqpoint{2.844933in}{3.521387in}}%
\pgfpathlineto{\pgfqpoint{2.855957in}{3.531451in}}%
\pgfpathlineto{\pgfqpoint{2.866989in}{3.539646in}}%
\pgfpathlineto{\pgfqpoint{2.878025in}{3.546339in}}%
\pgfpathlineto{\pgfqpoint{2.889062in}{3.551907in}}%
\pgfpathlineto{\pgfqpoint{2.900099in}{3.556724in}}%
\pgfpathlineto{\pgfqpoint{2.894142in}{3.557973in}}%
\pgfpathlineto{\pgfqpoint{2.888191in}{3.559265in}}%
\pgfpathlineto{\pgfqpoint{2.882246in}{3.560591in}}%
\pgfpathlineto{\pgfqpoint{2.876306in}{3.561948in}}%
\pgfpathlineto{\pgfqpoint{2.870371in}{3.563330in}}%
\pgfpathlineto{\pgfqpoint{2.859356in}{3.558476in}}%
\pgfpathlineto{\pgfqpoint{2.848340in}{3.552952in}}%
\pgfpathlineto{\pgfqpoint{2.837326in}{3.546419in}}%
\pgfpathlineto{\pgfqpoint{2.826315in}{3.538540in}}%
\pgfpathlineto{\pgfqpoint{2.815312in}{3.528982in}}%
\pgfpathlineto{\pgfqpoint{2.821226in}{3.527314in}}%
\pgfpathlineto{\pgfqpoint{2.827145in}{3.525692in}}%
\pgfpathlineto{\pgfqpoint{2.833069in}{3.524145in}}%
\pgfpathlineto{\pgfqpoint{2.838999in}{3.522700in}}%
\pgfpathclose%
\pgfusepath{stroke,fill}%
\end{pgfscope}%
\begin{pgfscope}%
\pgfpathrectangle{\pgfqpoint{0.887500in}{0.275000in}}{\pgfqpoint{4.225000in}{4.225000in}}%
\pgfusepath{clip}%
\pgfsetbuttcap%
\pgfsetroundjoin%
\definecolor{currentfill}{rgb}{0.964894,0.902323,0.123941}%
\pgfsetfillcolor{currentfill}%
\pgfsetfillopacity{0.700000}%
\pgfsetlinewidth{0.501875pt}%
\definecolor{currentstroke}{rgb}{1.000000,1.000000,1.000000}%
\pgfsetstrokecolor{currentstroke}%
\pgfsetstrokeopacity{0.500000}%
\pgfsetdash{}{0pt}%
\pgfpathmoveto{\pgfqpoint{3.571388in}{3.805637in}}%
\pgfpathlineto{\pgfqpoint{3.582301in}{3.810587in}}%
\pgfpathlineto{\pgfqpoint{3.593209in}{3.815523in}}%
\pgfpathlineto{\pgfqpoint{3.604113in}{3.820442in}}%
\pgfpathlineto{\pgfqpoint{3.615013in}{3.825343in}}%
\pgfpathlineto{\pgfqpoint{3.625908in}{3.830225in}}%
\pgfpathlineto{\pgfqpoint{3.619637in}{3.833918in}}%
\pgfpathlineto{\pgfqpoint{3.613366in}{3.837224in}}%
\pgfpathlineto{\pgfqpoint{3.607096in}{3.840123in}}%
\pgfpathlineto{\pgfqpoint{3.600825in}{3.842613in}}%
\pgfpathlineto{\pgfqpoint{3.594555in}{3.844716in}}%
\pgfpathlineto{\pgfqpoint{3.583682in}{3.840182in}}%
\pgfpathlineto{\pgfqpoint{3.572804in}{3.835571in}}%
\pgfpathlineto{\pgfqpoint{3.561921in}{3.830886in}}%
\pgfpathlineto{\pgfqpoint{3.551032in}{3.826129in}}%
\pgfpathlineto{\pgfqpoint{3.540139in}{3.821301in}}%
\pgfpathlineto{\pgfqpoint{3.546385in}{3.818797in}}%
\pgfpathlineto{\pgfqpoint{3.552634in}{3.815989in}}%
\pgfpathlineto{\pgfqpoint{3.558884in}{3.812859in}}%
\pgfpathlineto{\pgfqpoint{3.565135in}{3.809405in}}%
\pgfpathclose%
\pgfusepath{stroke,fill}%
\end{pgfscope}%
\begin{pgfscope}%
\pgfpathrectangle{\pgfqpoint{0.887500in}{0.275000in}}{\pgfqpoint{4.225000in}{4.225000in}}%
\pgfusepath{clip}%
\pgfsetbuttcap%
\pgfsetroundjoin%
\definecolor{currentfill}{rgb}{0.515992,0.831158,0.294279}%
\pgfsetfillcolor{currentfill}%
\pgfsetfillopacity{0.700000}%
\pgfsetlinewidth{0.501875pt}%
\definecolor{currentstroke}{rgb}{1.000000,1.000000,1.000000}%
\pgfsetstrokecolor{currentstroke}%
\pgfsetstrokeopacity{0.500000}%
\pgfsetdash{}{0pt}%
\pgfpathmoveto{\pgfqpoint{2.789935in}{3.448955in}}%
\pgfpathlineto{\pgfqpoint{2.800921in}{3.465195in}}%
\pgfpathlineto{\pgfqpoint{2.811913in}{3.480922in}}%
\pgfpathlineto{\pgfqpoint{2.822911in}{3.495773in}}%
\pgfpathlineto{\pgfqpoint{2.833918in}{3.509383in}}%
\pgfpathlineto{\pgfqpoint{2.844933in}{3.521387in}}%
\pgfpathlineto{\pgfqpoint{2.838999in}{3.522700in}}%
\pgfpathlineto{\pgfqpoint{2.833069in}{3.524145in}}%
\pgfpathlineto{\pgfqpoint{2.827145in}{3.525692in}}%
\pgfpathlineto{\pgfqpoint{2.821226in}{3.527314in}}%
\pgfpathlineto{\pgfqpoint{2.815312in}{3.528982in}}%
\pgfpathlineto{\pgfqpoint{2.804319in}{3.517679in}}%
\pgfpathlineto{\pgfqpoint{2.793334in}{3.504929in}}%
\pgfpathlineto{\pgfqpoint{2.782356in}{3.491053in}}%
\pgfpathlineto{\pgfqpoint{2.771385in}{3.476375in}}%
\pgfpathlineto{\pgfqpoint{2.760419in}{3.461218in}}%
\pgfpathlineto{\pgfqpoint{2.766316in}{3.458142in}}%
\pgfpathlineto{\pgfqpoint{2.772216in}{3.455308in}}%
\pgfpathlineto{\pgfqpoint{2.778120in}{3.452786in}}%
\pgfpathlineto{\pgfqpoint{2.784026in}{3.450645in}}%
\pgfpathclose%
\pgfusepath{stroke,fill}%
\end{pgfscope}%
\begin{pgfscope}%
\pgfpathrectangle{\pgfqpoint{0.887500in}{0.275000in}}{\pgfqpoint{4.225000in}{4.225000in}}%
\pgfusepath{clip}%
\pgfsetbuttcap%
\pgfsetroundjoin%
\definecolor{currentfill}{rgb}{0.993248,0.906157,0.143936}%
\pgfsetfillcolor{currentfill}%
\pgfsetfillopacity{0.700000}%
\pgfsetlinewidth{0.501875pt}%
\definecolor{currentstroke}{rgb}{1.000000,1.000000,1.000000}%
\pgfsetstrokecolor{currentstroke}%
\pgfsetstrokeopacity{0.500000}%
\pgfsetdash{}{0pt}%
\pgfpathmoveto{\pgfqpoint{3.711747in}{3.830425in}}%
\pgfpathlineto{\pgfqpoint{3.722629in}{3.835092in}}%
\pgfpathlineto{\pgfqpoint{3.733506in}{3.839735in}}%
\pgfpathlineto{\pgfqpoint{3.744379in}{3.844355in}}%
\pgfpathlineto{\pgfqpoint{3.755247in}{3.848952in}}%
\pgfpathlineto{\pgfqpoint{3.748954in}{3.854629in}}%
\pgfpathlineto{\pgfqpoint{3.742660in}{3.859956in}}%
\pgfpathlineto{\pgfqpoint{3.736364in}{3.864863in}}%
\pgfpathlineto{\pgfqpoint{3.730065in}{3.869277in}}%
\pgfpathlineto{\pgfqpoint{3.723762in}{3.873129in}}%
\pgfpathlineto{\pgfqpoint{3.712908in}{3.868468in}}%
\pgfpathlineto{\pgfqpoint{3.702049in}{3.863778in}}%
\pgfpathlineto{\pgfqpoint{3.691185in}{3.859059in}}%
\pgfpathlineto{\pgfqpoint{3.680317in}{3.854314in}}%
\pgfpathlineto{\pgfqpoint{3.686605in}{3.850334in}}%
\pgfpathlineto{\pgfqpoint{3.692892in}{3.845910in}}%
\pgfpathlineto{\pgfqpoint{3.699177in}{3.841087in}}%
\pgfpathlineto{\pgfqpoint{3.705462in}{3.835910in}}%
\pgfpathclose%
\pgfusepath{stroke,fill}%
\end{pgfscope}%
\begin{pgfscope}%
\pgfpathrectangle{\pgfqpoint{0.887500in}{0.275000in}}{\pgfqpoint{4.225000in}{4.225000in}}%
\pgfusepath{clip}%
\pgfsetbuttcap%
\pgfsetroundjoin%
\definecolor{currentfill}{rgb}{0.185783,0.704891,0.485273}%
\pgfsetfillcolor{currentfill}%
\pgfsetfillopacity{0.700000}%
\pgfsetlinewidth{0.501875pt}%
\definecolor{currentstroke}{rgb}{1.000000,1.000000,1.000000}%
\pgfsetstrokecolor{currentstroke}%
\pgfsetstrokeopacity{0.500000}%
\pgfsetdash{}{0pt}%
\pgfpathmoveto{\pgfqpoint{2.514483in}{3.148104in}}%
\pgfpathlineto{\pgfqpoint{2.525595in}{3.153393in}}%
\pgfpathlineto{\pgfqpoint{2.536700in}{3.158837in}}%
\pgfpathlineto{\pgfqpoint{2.547801in}{3.164307in}}%
\pgfpathlineto{\pgfqpoint{2.558899in}{3.169758in}}%
\pgfpathlineto{\pgfqpoint{2.569990in}{3.175410in}}%
\pgfpathlineto{\pgfqpoint{2.564173in}{3.178434in}}%
\pgfpathlineto{\pgfqpoint{2.558354in}{3.182047in}}%
\pgfpathlineto{\pgfqpoint{2.552533in}{3.186283in}}%
\pgfpathlineto{\pgfqpoint{2.546708in}{3.191179in}}%
\pgfpathlineto{\pgfqpoint{2.540878in}{3.196772in}}%
\pgfpathlineto{\pgfqpoint{2.529768in}{3.193988in}}%
\pgfpathlineto{\pgfqpoint{2.518653in}{3.191150in}}%
\pgfpathlineto{\pgfqpoint{2.507538in}{3.187861in}}%
\pgfpathlineto{\pgfqpoint{2.496424in}{3.184142in}}%
\pgfpathlineto{\pgfqpoint{2.485310in}{3.180141in}}%
\pgfpathlineto{\pgfqpoint{2.491144in}{3.173198in}}%
\pgfpathlineto{\pgfqpoint{2.496977in}{3.166560in}}%
\pgfpathlineto{\pgfqpoint{2.502811in}{3.160191in}}%
\pgfpathlineto{\pgfqpoint{2.508646in}{3.154052in}}%
\pgfpathclose%
\pgfusepath{stroke,fill}%
\end{pgfscope}%
\begin{pgfscope}%
\pgfpathrectangle{\pgfqpoint{0.887500in}{0.275000in}}{\pgfqpoint{4.225000in}{4.225000in}}%
\pgfusepath{clip}%
\pgfsetbuttcap%
\pgfsetroundjoin%
\definecolor{currentfill}{rgb}{0.876168,0.891125,0.095250}%
\pgfsetfillcolor{currentfill}%
\pgfsetfillopacity{0.700000}%
\pgfsetlinewidth{0.501875pt}%
\definecolor{currentstroke}{rgb}{1.000000,1.000000,1.000000}%
\pgfsetstrokecolor{currentstroke}%
\pgfsetstrokeopacity{0.500000}%
\pgfsetdash{}{0pt}%
\pgfpathmoveto{\pgfqpoint{3.376141in}{3.736970in}}%
\pgfpathlineto{\pgfqpoint{3.387104in}{3.743048in}}%
\pgfpathlineto{\pgfqpoint{3.398064in}{3.749130in}}%
\pgfpathlineto{\pgfqpoint{3.409020in}{3.755212in}}%
\pgfpathlineto{\pgfqpoint{3.419973in}{3.761271in}}%
\pgfpathlineto{\pgfqpoint{3.430921in}{3.767283in}}%
\pgfpathlineto{\pgfqpoint{3.424714in}{3.769353in}}%
\pgfpathlineto{\pgfqpoint{3.418511in}{3.771210in}}%
\pgfpathlineto{\pgfqpoint{3.412312in}{3.772877in}}%
\pgfpathlineto{\pgfqpoint{3.406117in}{3.774377in}}%
\pgfpathlineto{\pgfqpoint{3.399927in}{3.775735in}}%
\pgfpathlineto{\pgfqpoint{3.388998in}{3.769628in}}%
\pgfpathlineto{\pgfqpoint{3.378066in}{3.763479in}}%
\pgfpathlineto{\pgfqpoint{3.367129in}{3.757311in}}%
\pgfpathlineto{\pgfqpoint{3.356190in}{3.751150in}}%
\pgfpathlineto{\pgfqpoint{3.345247in}{3.745001in}}%
\pgfpathlineto{\pgfqpoint{3.351416in}{3.743654in}}%
\pgfpathlineto{\pgfqpoint{3.357590in}{3.742202in}}%
\pgfpathlineto{\pgfqpoint{3.363769in}{3.740620in}}%
\pgfpathlineto{\pgfqpoint{3.369953in}{3.738884in}}%
\pgfpathclose%
\pgfusepath{stroke,fill}%
\end{pgfscope}%
\begin{pgfscope}%
\pgfpathrectangle{\pgfqpoint{0.887500in}{0.275000in}}{\pgfqpoint{4.225000in}{4.225000in}}%
\pgfusepath{clip}%
\pgfsetbuttcap%
\pgfsetroundjoin%
\definecolor{currentfill}{rgb}{0.762373,0.876424,0.137064}%
\pgfsetfillcolor{currentfill}%
\pgfsetfillopacity{0.700000}%
\pgfsetlinewidth{0.501875pt}%
\definecolor{currentstroke}{rgb}{1.000000,1.000000,1.000000}%
\pgfsetstrokecolor{currentstroke}%
\pgfsetstrokeopacity{0.500000}%
\pgfsetdash{}{0pt}%
\pgfpathmoveto{\pgfqpoint{3.180684in}{3.654071in}}%
\pgfpathlineto{\pgfqpoint{3.191678in}{3.659597in}}%
\pgfpathlineto{\pgfqpoint{3.202669in}{3.665232in}}%
\pgfpathlineto{\pgfqpoint{3.213656in}{3.671006in}}%
\pgfpathlineto{\pgfqpoint{3.224640in}{3.676934in}}%
\pgfpathlineto{\pgfqpoint{3.235621in}{3.682990in}}%
\pgfpathlineto{\pgfqpoint{3.229500in}{3.684282in}}%
\pgfpathlineto{\pgfqpoint{3.223385in}{3.685549in}}%
\pgfpathlineto{\pgfqpoint{3.217276in}{3.686812in}}%
\pgfpathlineto{\pgfqpoint{3.211173in}{3.688089in}}%
\pgfpathlineto{\pgfqpoint{3.205075in}{3.689383in}}%
\pgfpathlineto{\pgfqpoint{3.194115in}{3.683219in}}%
\pgfpathlineto{\pgfqpoint{3.183152in}{3.677099in}}%
\pgfpathlineto{\pgfqpoint{3.172185in}{3.671039in}}%
\pgfpathlineto{\pgfqpoint{3.161215in}{3.665045in}}%
\pgfpathlineto{\pgfqpoint{3.150241in}{3.659125in}}%
\pgfpathlineto{\pgfqpoint{3.156318in}{3.658020in}}%
\pgfpathlineto{\pgfqpoint{3.162400in}{3.656975in}}%
\pgfpathlineto{\pgfqpoint{3.168489in}{3.655987in}}%
\pgfpathlineto{\pgfqpoint{3.174583in}{3.655030in}}%
\pgfpathclose%
\pgfusepath{stroke,fill}%
\end{pgfscope}%
\begin{pgfscope}%
\pgfpathrectangle{\pgfqpoint{0.887500in}{0.275000in}}{\pgfqpoint{4.225000in}{4.225000in}}%
\pgfusepath{clip}%
\pgfsetbuttcap%
\pgfsetroundjoin%
\definecolor{currentfill}{rgb}{0.202219,0.715272,0.476084}%
\pgfsetfillcolor{currentfill}%
\pgfsetfillopacity{0.700000}%
\pgfsetlinewidth{0.501875pt}%
\definecolor{currentstroke}{rgb}{1.000000,1.000000,1.000000}%
\pgfsetstrokecolor{currentstroke}%
\pgfsetstrokeopacity{0.500000}%
\pgfsetdash{}{0pt}%
\pgfpathmoveto{\pgfqpoint{2.289183in}{3.150339in}}%
\pgfpathlineto{\pgfqpoint{2.300287in}{3.157984in}}%
\pgfpathlineto{\pgfqpoint{2.311395in}{3.165299in}}%
\pgfpathlineto{\pgfqpoint{2.322509in}{3.172093in}}%
\pgfpathlineto{\pgfqpoint{2.333632in}{3.178176in}}%
\pgfpathlineto{\pgfqpoint{2.344768in}{3.183387in}}%
\pgfpathlineto{\pgfqpoint{2.338923in}{3.193691in}}%
\pgfpathlineto{\pgfqpoint{2.333061in}{3.205076in}}%
\pgfpathlineto{\pgfqpoint{2.327181in}{3.217656in}}%
\pgfpathlineto{\pgfqpoint{2.321279in}{3.231542in}}%
\pgfpathlineto{\pgfqpoint{2.310154in}{3.226092in}}%
\pgfpathlineto{\pgfqpoint{2.299052in}{3.219195in}}%
\pgfpathlineto{\pgfqpoint{2.287969in}{3.211131in}}%
\pgfpathlineto{\pgfqpoint{2.276899in}{3.202231in}}%
\pgfpathlineto{\pgfqpoint{2.265838in}{3.192827in}}%
\pgfpathlineto{\pgfqpoint{2.271695in}{3.180909in}}%
\pgfpathlineto{\pgfqpoint{2.277537in}{3.169908in}}%
\pgfpathlineto{\pgfqpoint{2.283365in}{3.159743in}}%
\pgfpathclose%
\pgfusepath{stroke,fill}%
\end{pgfscope}%
\begin{pgfscope}%
\pgfpathrectangle{\pgfqpoint{0.887500in}{0.275000in}}{\pgfqpoint{4.225000in}{4.225000in}}%
\pgfusepath{clip}%
\pgfsetbuttcap%
\pgfsetroundjoin%
\definecolor{currentfill}{rgb}{0.657642,0.860219,0.203082}%
\pgfsetfillcolor{currentfill}%
\pgfsetfillopacity{0.700000}%
\pgfsetlinewidth{0.501875pt}%
\definecolor{currentstroke}{rgb}{1.000000,1.000000,1.000000}%
\pgfsetstrokecolor{currentstroke}%
\pgfsetstrokeopacity{0.500000}%
\pgfsetdash{}{0pt}%
\pgfpathmoveto{\pgfqpoint{2.985188in}{3.575278in}}%
\pgfpathlineto{\pgfqpoint{2.996219in}{3.580927in}}%
\pgfpathlineto{\pgfqpoint{3.007245in}{3.586702in}}%
\pgfpathlineto{\pgfqpoint{3.018268in}{3.592511in}}%
\pgfpathlineto{\pgfqpoint{3.029287in}{3.598261in}}%
\pgfpathlineto{\pgfqpoint{3.040302in}{3.603877in}}%
\pgfpathlineto{\pgfqpoint{3.034274in}{3.604946in}}%
\pgfpathlineto{\pgfqpoint{3.028252in}{3.606060in}}%
\pgfpathlineto{\pgfqpoint{3.022236in}{3.607219in}}%
\pgfpathlineto{\pgfqpoint{3.016225in}{3.608420in}}%
\pgfpathlineto{\pgfqpoint{3.010221in}{3.609664in}}%
\pgfpathlineto{\pgfqpoint{2.999227in}{3.604079in}}%
\pgfpathlineto{\pgfqpoint{2.988230in}{3.598341in}}%
\pgfpathlineto{\pgfqpoint{2.977229in}{3.592527in}}%
\pgfpathlineto{\pgfqpoint{2.966224in}{3.586731in}}%
\pgfpathlineto{\pgfqpoint{2.955215in}{3.581047in}}%
\pgfpathlineto{\pgfqpoint{2.961198in}{3.579808in}}%
\pgfpathlineto{\pgfqpoint{2.967187in}{3.578613in}}%
\pgfpathlineto{\pgfqpoint{2.973182in}{3.577459in}}%
\pgfpathlineto{\pgfqpoint{2.979182in}{3.576348in}}%
\pgfpathclose%
\pgfusepath{stroke,fill}%
\end{pgfscope}%
\begin{pgfscope}%
\pgfpathrectangle{\pgfqpoint{0.887500in}{0.275000in}}{\pgfqpoint{4.225000in}{4.225000in}}%
\pgfusepath{clip}%
\pgfsetbuttcap%
\pgfsetroundjoin%
\definecolor{currentfill}{rgb}{0.191090,0.708366,0.482284}%
\pgfsetfillcolor{currentfill}%
\pgfsetfillopacity{0.700000}%
\pgfsetlinewidth{0.501875pt}%
\definecolor{currentstroke}{rgb}{1.000000,1.000000,1.000000}%
\pgfsetstrokecolor{currentstroke}%
\pgfsetstrokeopacity{0.500000}%
\pgfsetdash{}{0pt}%
\pgfpathmoveto{\pgfqpoint{2.373827in}{3.144208in}}%
\pgfpathlineto{\pgfqpoint{2.384989in}{3.148356in}}%
\pgfpathlineto{\pgfqpoint{2.396156in}{3.151909in}}%
\pgfpathlineto{\pgfqpoint{2.407324in}{3.155073in}}%
\pgfpathlineto{\pgfqpoint{2.418489in}{3.158055in}}%
\pgfpathlineto{\pgfqpoint{2.429649in}{3.161065in}}%
\pgfpathlineto{\pgfqpoint{2.423829in}{3.168243in}}%
\pgfpathlineto{\pgfqpoint{2.418006in}{3.175804in}}%
\pgfpathlineto{\pgfqpoint{2.412179in}{3.183822in}}%
\pgfpathlineto{\pgfqpoint{2.406348in}{3.192368in}}%
\pgfpathlineto{\pgfqpoint{2.400511in}{3.201515in}}%
\pgfpathlineto{\pgfqpoint{2.389365in}{3.198301in}}%
\pgfpathlineto{\pgfqpoint{2.378215in}{3.195074in}}%
\pgfpathlineto{\pgfqpoint{2.367063in}{3.191643in}}%
\pgfpathlineto{\pgfqpoint{2.355912in}{3.187812in}}%
\pgfpathlineto{\pgfqpoint{2.344768in}{3.183387in}}%
\pgfpathlineto{\pgfqpoint{2.350598in}{3.174055in}}%
\pgfpathlineto{\pgfqpoint{2.356418in}{3.165583in}}%
\pgfpathlineto{\pgfqpoint{2.362227in}{3.157858in}}%
\pgfpathlineto{\pgfqpoint{2.368030in}{3.150770in}}%
\pgfpathclose%
\pgfusepath{stroke,fill}%
\end{pgfscope}%
\begin{pgfscope}%
\pgfpathrectangle{\pgfqpoint{0.887500in}{0.275000in}}{\pgfqpoint{4.225000in}{4.225000in}}%
\pgfusepath{clip}%
\pgfsetbuttcap%
\pgfsetroundjoin%
\definecolor{currentfill}{rgb}{0.945636,0.899815,0.112838}%
\pgfsetfillcolor{currentfill}%
\pgfsetfillopacity{0.700000}%
\pgfsetlinewidth{0.501875pt}%
\definecolor{currentstroke}{rgb}{1.000000,1.000000,1.000000}%
\pgfsetstrokecolor{currentstroke}%
\pgfsetstrokeopacity{0.500000}%
\pgfsetdash{}{0pt}%
\pgfpathmoveto{\pgfqpoint{3.516757in}{3.780320in}}%
\pgfpathlineto{\pgfqpoint{3.527693in}{3.785505in}}%
\pgfpathlineto{\pgfqpoint{3.538624in}{3.790615in}}%
\pgfpathlineto{\pgfqpoint{3.549550in}{3.795664in}}%
\pgfpathlineto{\pgfqpoint{3.560472in}{3.800667in}}%
\pgfpathlineto{\pgfqpoint{3.571388in}{3.805637in}}%
\pgfpathlineto{\pgfqpoint{3.565135in}{3.809405in}}%
\pgfpathlineto{\pgfqpoint{3.558884in}{3.812859in}}%
\pgfpathlineto{\pgfqpoint{3.552634in}{3.815989in}}%
\pgfpathlineto{\pgfqpoint{3.546385in}{3.818797in}}%
\pgfpathlineto{\pgfqpoint{3.540139in}{3.821301in}}%
\pgfpathlineto{\pgfqpoint{3.529240in}{3.816400in}}%
\pgfpathlineto{\pgfqpoint{3.518337in}{3.811414in}}%
\pgfpathlineto{\pgfqpoint{3.507428in}{3.806332in}}%
\pgfpathlineto{\pgfqpoint{3.496514in}{3.801141in}}%
\pgfpathlineto{\pgfqpoint{3.485595in}{3.795830in}}%
\pgfpathlineto{\pgfqpoint{3.491823in}{3.793329in}}%
\pgfpathlineto{\pgfqpoint{3.498053in}{3.790541in}}%
\pgfpathlineto{\pgfqpoint{3.504286in}{3.787445in}}%
\pgfpathlineto{\pgfqpoint{3.510521in}{3.784036in}}%
\pgfpathclose%
\pgfusepath{stroke,fill}%
\end{pgfscope}%
\begin{pgfscope}%
\pgfpathrectangle{\pgfqpoint{0.887500in}{0.275000in}}{\pgfqpoint{4.225000in}{4.225000in}}%
\pgfusepath{clip}%
\pgfsetbuttcap%
\pgfsetroundjoin%
\definecolor{currentfill}{rgb}{0.175707,0.697900,0.491033}%
\pgfsetfillcolor{currentfill}%
\pgfsetfillopacity{0.700000}%
\pgfsetlinewidth{0.501875pt}%
\definecolor{currentstroke}{rgb}{1.000000,1.000000,1.000000}%
\pgfsetstrokecolor{currentstroke}%
\pgfsetstrokeopacity{0.500000}%
\pgfsetdash{}{0pt}%
\pgfpathmoveto{\pgfqpoint{2.233593in}{3.113343in}}%
\pgfpathlineto{\pgfqpoint{2.244728in}{3.120174in}}%
\pgfpathlineto{\pgfqpoint{2.255854in}{3.127325in}}%
\pgfpathlineto{\pgfqpoint{2.266970in}{3.134819in}}%
\pgfpathlineto{\pgfqpoint{2.278078in}{3.142553in}}%
\pgfpathlineto{\pgfqpoint{2.289183in}{3.150339in}}%
\pgfpathlineto{\pgfqpoint{2.283365in}{3.159743in}}%
\pgfpathlineto{\pgfqpoint{2.277537in}{3.169908in}}%
\pgfpathlineto{\pgfqpoint{2.271695in}{3.180909in}}%
\pgfpathlineto{\pgfqpoint{2.265838in}{3.192827in}}%
\pgfpathlineto{\pgfqpoint{2.254778in}{3.183250in}}%
\pgfpathlineto{\pgfqpoint{2.243713in}{3.173833in}}%
\pgfpathlineto{\pgfqpoint{2.232636in}{3.164904in}}%
\pgfpathlineto{\pgfqpoint{2.221543in}{3.156637in}}%
\pgfpathlineto{\pgfqpoint{2.210435in}{3.148976in}}%
\pgfpathlineto{\pgfqpoint{2.216228in}{3.139665in}}%
\pgfpathlineto{\pgfqpoint{2.222018in}{3.130637in}}%
\pgfpathlineto{\pgfqpoint{2.227807in}{3.121871in}}%
\pgfpathclose%
\pgfusepath{stroke,fill}%
\end{pgfscope}%
\begin{pgfscope}%
\pgfpathrectangle{\pgfqpoint{0.887500in}{0.275000in}}{\pgfqpoint{4.225000in}{4.225000in}}%
\pgfusepath{clip}%
\pgfsetbuttcap%
\pgfsetroundjoin%
\definecolor{currentfill}{rgb}{0.983868,0.904867,0.136897}%
\pgfsetfillcolor{currentfill}%
\pgfsetfillopacity{0.700000}%
\pgfsetlinewidth{0.501875pt}%
\definecolor{currentstroke}{rgb}{1.000000,1.000000,1.000000}%
\pgfsetstrokecolor{currentstroke}%
\pgfsetstrokeopacity{0.500000}%
\pgfsetdash{}{0pt}%
\pgfpathmoveto{\pgfqpoint{3.657268in}{3.806727in}}%
\pgfpathlineto{\pgfqpoint{3.668173in}{3.811516in}}%
\pgfpathlineto{\pgfqpoint{3.679073in}{3.816279in}}%
\pgfpathlineto{\pgfqpoint{3.689969in}{3.821019in}}%
\pgfpathlineto{\pgfqpoint{3.700860in}{3.825734in}}%
\pgfpathlineto{\pgfqpoint{3.711747in}{3.830425in}}%
\pgfpathlineto{\pgfqpoint{3.705462in}{3.835910in}}%
\pgfpathlineto{\pgfqpoint{3.699177in}{3.841087in}}%
\pgfpathlineto{\pgfqpoint{3.692892in}{3.845910in}}%
\pgfpathlineto{\pgfqpoint{3.686605in}{3.850334in}}%
\pgfpathlineto{\pgfqpoint{3.680317in}{3.854314in}}%
\pgfpathlineto{\pgfqpoint{3.669444in}{3.849543in}}%
\pgfpathlineto{\pgfqpoint{3.658567in}{3.844748in}}%
\pgfpathlineto{\pgfqpoint{3.647685in}{3.839929in}}%
\pgfpathlineto{\pgfqpoint{3.636799in}{3.835087in}}%
\pgfpathlineto{\pgfqpoint{3.625908in}{3.830225in}}%
\pgfpathlineto{\pgfqpoint{3.632179in}{3.826168in}}%
\pgfpathlineto{\pgfqpoint{3.638450in}{3.821767in}}%
\pgfpathlineto{\pgfqpoint{3.644722in}{3.817046in}}%
\pgfpathlineto{\pgfqpoint{3.650994in}{3.812025in}}%
\pgfpathclose%
\pgfusepath{stroke,fill}%
\end{pgfscope}%
\begin{pgfscope}%
\pgfpathrectangle{\pgfqpoint{0.887500in}{0.275000in}}{\pgfqpoint{4.225000in}{4.225000in}}%
\pgfusepath{clip}%
\pgfsetbuttcap%
\pgfsetroundjoin%
\definecolor{currentfill}{rgb}{0.845561,0.887322,0.099702}%
\pgfsetfillcolor{currentfill}%
\pgfsetfillopacity{0.700000}%
\pgfsetlinewidth{0.501875pt}%
\definecolor{currentstroke}{rgb}{1.000000,1.000000,1.000000}%
\pgfsetstrokecolor{currentstroke}%
\pgfsetstrokeopacity{0.500000}%
\pgfsetdash{}{0pt}%
\pgfpathmoveto{\pgfqpoint{3.321270in}{3.706332in}}%
\pgfpathlineto{\pgfqpoint{3.332252in}{3.712525in}}%
\pgfpathlineto{\pgfqpoint{3.343230in}{3.718676in}}%
\pgfpathlineto{\pgfqpoint{3.354204in}{3.724794in}}%
\pgfpathlineto{\pgfqpoint{3.365174in}{3.730889in}}%
\pgfpathlineto{\pgfqpoint{3.376141in}{3.736970in}}%
\pgfpathlineto{\pgfqpoint{3.369953in}{3.738884in}}%
\pgfpathlineto{\pgfqpoint{3.363769in}{3.740620in}}%
\pgfpathlineto{\pgfqpoint{3.357590in}{3.742202in}}%
\pgfpathlineto{\pgfqpoint{3.351416in}{3.743654in}}%
\pgfpathlineto{\pgfqpoint{3.345247in}{3.745001in}}%
\pgfpathlineto{\pgfqpoint{3.334300in}{3.738857in}}%
\pgfpathlineto{\pgfqpoint{3.323350in}{3.732708in}}%
\pgfpathlineto{\pgfqpoint{3.312397in}{3.726547in}}%
\pgfpathlineto{\pgfqpoint{3.301439in}{3.720365in}}%
\pgfpathlineto{\pgfqpoint{3.290478in}{3.714151in}}%
\pgfpathlineto{\pgfqpoint{3.296626in}{3.712804in}}%
\pgfpathlineto{\pgfqpoint{3.302780in}{3.711371in}}%
\pgfpathlineto{\pgfqpoint{3.308938in}{3.709831in}}%
\pgfpathlineto{\pgfqpoint{3.315102in}{3.708159in}}%
\pgfpathclose%
\pgfusepath{stroke,fill}%
\end{pgfscope}%
\begin{pgfscope}%
\pgfpathrectangle{\pgfqpoint{0.887500in}{0.275000in}}{\pgfqpoint{4.225000in}{4.225000in}}%
\pgfusepath{clip}%
\pgfsetbuttcap%
\pgfsetroundjoin%
\definecolor{currentfill}{rgb}{0.741388,0.873449,0.149561}%
\pgfsetfillcolor{currentfill}%
\pgfsetfillopacity{0.700000}%
\pgfsetlinewidth{0.501875pt}%
\definecolor{currentstroke}{rgb}{1.000000,1.000000,1.000000}%
\pgfsetstrokecolor{currentstroke}%
\pgfsetstrokeopacity{0.500000}%
\pgfsetdash{}{0pt}%
\pgfpathmoveto{\pgfqpoint{3.125657in}{3.627053in}}%
\pgfpathlineto{\pgfqpoint{3.136670in}{3.632472in}}%
\pgfpathlineto{\pgfqpoint{3.147679in}{3.637855in}}%
\pgfpathlineto{\pgfqpoint{3.158685in}{3.643230in}}%
\pgfpathlineto{\pgfqpoint{3.169686in}{3.648625in}}%
\pgfpathlineto{\pgfqpoint{3.180684in}{3.654071in}}%
\pgfpathlineto{\pgfqpoint{3.174583in}{3.655030in}}%
\pgfpathlineto{\pgfqpoint{3.168489in}{3.655987in}}%
\pgfpathlineto{\pgfqpoint{3.162400in}{3.656975in}}%
\pgfpathlineto{\pgfqpoint{3.156318in}{3.658020in}}%
\pgfpathlineto{\pgfqpoint{3.150241in}{3.659125in}}%
\pgfpathlineto{\pgfqpoint{3.139264in}{3.653285in}}%
\pgfpathlineto{\pgfqpoint{3.128283in}{3.647534in}}%
\pgfpathlineto{\pgfqpoint{3.117299in}{3.641876in}}%
\pgfpathlineto{\pgfqpoint{3.106311in}{3.636321in}}%
\pgfpathlineto{\pgfqpoint{3.095319in}{3.630867in}}%
\pgfpathlineto{\pgfqpoint{3.101375in}{3.629956in}}%
\pgfpathlineto{\pgfqpoint{3.107436in}{3.629133in}}%
\pgfpathlineto{\pgfqpoint{3.113503in}{3.628396in}}%
\pgfpathlineto{\pgfqpoint{3.119577in}{3.627714in}}%
\pgfpathclose%
\pgfusepath{stroke,fill}%
\end{pgfscope}%
\begin{pgfscope}%
\pgfpathrectangle{\pgfqpoint{0.887500in}{0.275000in}}{\pgfqpoint{4.225000in}{4.225000in}}%
\pgfusepath{clip}%
\pgfsetbuttcap%
\pgfsetroundjoin%
\definecolor{currentfill}{rgb}{0.180653,0.701402,0.488189}%
\pgfsetfillcolor{currentfill}%
\pgfsetfillopacity{0.700000}%
\pgfsetlinewidth{0.501875pt}%
\definecolor{currentstroke}{rgb}{1.000000,1.000000,1.000000}%
\pgfsetstrokecolor{currentstroke}%
\pgfsetstrokeopacity{0.500000}%
\pgfsetdash{}{0pt}%
\pgfpathmoveto{\pgfqpoint{2.458763in}{3.128367in}}%
\pgfpathlineto{\pgfqpoint{2.469936in}{3.131053in}}%
\pgfpathlineto{\pgfqpoint{2.481092in}{3.134444in}}%
\pgfpathlineto{\pgfqpoint{2.492234in}{3.138502in}}%
\pgfpathlineto{\pgfqpoint{2.503364in}{3.143098in}}%
\pgfpathlineto{\pgfqpoint{2.514483in}{3.148104in}}%
\pgfpathlineto{\pgfqpoint{2.508646in}{3.154052in}}%
\pgfpathlineto{\pgfqpoint{2.502811in}{3.160191in}}%
\pgfpathlineto{\pgfqpoint{2.496977in}{3.166560in}}%
\pgfpathlineto{\pgfqpoint{2.491144in}{3.173198in}}%
\pgfpathlineto{\pgfqpoint{2.485310in}{3.180141in}}%
\pgfpathlineto{\pgfqpoint{2.474192in}{3.176008in}}%
\pgfpathlineto{\pgfqpoint{2.463069in}{3.171892in}}%
\pgfpathlineto{\pgfqpoint{2.451939in}{3.167942in}}%
\pgfpathlineto{\pgfqpoint{2.440800in}{3.164307in}}%
\pgfpathlineto{\pgfqpoint{2.429649in}{3.161065in}}%
\pgfpathlineto{\pgfqpoint{2.435469in}{3.154196in}}%
\pgfpathlineto{\pgfqpoint{2.441289in}{3.147565in}}%
\pgfpathlineto{\pgfqpoint{2.447111in}{3.141099in}}%
\pgfpathlineto{\pgfqpoint{2.452935in}{3.134724in}}%
\pgfpathclose%
\pgfusepath{stroke,fill}%
\end{pgfscope}%
\begin{pgfscope}%
\pgfpathrectangle{\pgfqpoint{0.887500in}{0.275000in}}{\pgfqpoint{4.225000in}{4.225000in}}%
\pgfusepath{clip}%
\pgfsetbuttcap%
\pgfsetroundjoin%
\definecolor{currentfill}{rgb}{0.636902,0.856542,0.216620}%
\pgfsetfillcolor{currentfill}%
\pgfsetfillopacity{0.700000}%
\pgfsetlinewidth{0.501875pt}%
\definecolor{currentstroke}{rgb}{1.000000,1.000000,1.000000}%
\pgfsetstrokecolor{currentstroke}%
\pgfsetstrokeopacity{0.500000}%
\pgfsetdash{}{0pt}%
\pgfpathmoveto{\pgfqpoint{2.929965in}{3.551297in}}%
\pgfpathlineto{\pgfqpoint{2.941020in}{3.555632in}}%
\pgfpathlineto{\pgfqpoint{2.952070in}{3.560001in}}%
\pgfpathlineto{\pgfqpoint{2.963114in}{3.564723in}}%
\pgfpathlineto{\pgfqpoint{2.974154in}{3.569846in}}%
\pgfpathlineto{\pgfqpoint{2.985188in}{3.575278in}}%
\pgfpathlineto{\pgfqpoint{2.979182in}{3.576348in}}%
\pgfpathlineto{\pgfqpoint{2.973182in}{3.577459in}}%
\pgfpathlineto{\pgfqpoint{2.967187in}{3.578613in}}%
\pgfpathlineto{\pgfqpoint{2.961198in}{3.579808in}}%
\pgfpathlineto{\pgfqpoint{2.955215in}{3.581047in}}%
\pgfpathlineto{\pgfqpoint{2.944202in}{3.575569in}}%
\pgfpathlineto{\pgfqpoint{2.933184in}{3.570392in}}%
\pgfpathlineto{\pgfqpoint{2.922161in}{3.565608in}}%
\pgfpathlineto{\pgfqpoint{2.911132in}{3.561166in}}%
\pgfpathlineto{\pgfqpoint{2.900099in}{3.556724in}}%
\pgfpathlineto{\pgfqpoint{2.906061in}{3.555521in}}%
\pgfpathlineto{\pgfqpoint{2.912028in}{3.554372in}}%
\pgfpathlineto{\pgfqpoint{2.918002in}{3.553281in}}%
\pgfpathlineto{\pgfqpoint{2.923981in}{3.552254in}}%
\pgfpathclose%
\pgfusepath{stroke,fill}%
\end{pgfscope}%
\begin{pgfscope}%
\pgfpathrectangle{\pgfqpoint{0.887500in}{0.275000in}}{\pgfqpoint{4.225000in}{4.225000in}}%
\pgfusepath{clip}%
\pgfsetbuttcap%
\pgfsetroundjoin%
\definecolor{currentfill}{rgb}{0.180653,0.701402,0.488189}%
\pgfsetfillcolor{currentfill}%
\pgfsetfillopacity{0.700000}%
\pgfsetlinewidth{0.501875pt}%
\definecolor{currentstroke}{rgb}{1.000000,1.000000,1.000000}%
\pgfsetstrokecolor{currentstroke}%
\pgfsetstrokeopacity{0.500000}%
\pgfsetdash{}{0pt}%
\pgfpathmoveto{\pgfqpoint{2.318157in}{3.111999in}}%
\pgfpathlineto{\pgfqpoint{2.329277in}{3.119602in}}%
\pgfpathlineto{\pgfqpoint{2.340402in}{3.126795in}}%
\pgfpathlineto{\pgfqpoint{2.351533in}{3.133405in}}%
\pgfpathlineto{\pgfqpoint{2.362675in}{3.139257in}}%
\pgfpathlineto{\pgfqpoint{2.373827in}{3.144208in}}%
\pgfpathlineto{\pgfqpoint{2.368030in}{3.150770in}}%
\pgfpathlineto{\pgfqpoint{2.362227in}{3.157858in}}%
\pgfpathlineto{\pgfqpoint{2.356418in}{3.165583in}}%
\pgfpathlineto{\pgfqpoint{2.350598in}{3.174055in}}%
\pgfpathlineto{\pgfqpoint{2.344768in}{3.183387in}}%
\pgfpathlineto{\pgfqpoint{2.333632in}{3.178176in}}%
\pgfpathlineto{\pgfqpoint{2.322509in}{3.172093in}}%
\pgfpathlineto{\pgfqpoint{2.311395in}{3.165299in}}%
\pgfpathlineto{\pgfqpoint{2.300287in}{3.157984in}}%
\pgfpathlineto{\pgfqpoint{2.289183in}{3.150339in}}%
\pgfpathlineto{\pgfqpoint{2.294990in}{3.141616in}}%
\pgfpathlineto{\pgfqpoint{2.300789in}{3.133499in}}%
\pgfpathlineto{\pgfqpoint{2.306583in}{3.125908in}}%
\pgfpathlineto{\pgfqpoint{2.312371in}{3.118768in}}%
\pgfpathclose%
\pgfusepath{stroke,fill}%
\end{pgfscope}%
\begin{pgfscope}%
\pgfpathrectangle{\pgfqpoint{0.887500in}{0.275000in}}{\pgfqpoint{4.225000in}{4.225000in}}%
\pgfusepath{clip}%
\pgfsetbuttcap%
\pgfsetroundjoin%
\definecolor{currentfill}{rgb}{0.162016,0.687316,0.499129}%
\pgfsetfillcolor{currentfill}%
\pgfsetfillopacity{0.700000}%
\pgfsetlinewidth{0.501875pt}%
\definecolor{currentstroke}{rgb}{1.000000,1.000000,1.000000}%
\pgfsetstrokecolor{currentstroke}%
\pgfsetstrokeopacity{0.500000}%
\pgfsetdash{}{0pt}%
\pgfpathmoveto{\pgfqpoint{2.177789in}{3.082751in}}%
\pgfpathlineto{\pgfqpoint{2.188965in}{3.088511in}}%
\pgfpathlineto{\pgfqpoint{2.200134in}{3.094415in}}%
\pgfpathlineto{\pgfqpoint{2.211295in}{3.100498in}}%
\pgfpathlineto{\pgfqpoint{2.222449in}{3.106796in}}%
\pgfpathlineto{\pgfqpoint{2.233593in}{3.113343in}}%
\pgfpathlineto{\pgfqpoint{2.227807in}{3.121871in}}%
\pgfpathlineto{\pgfqpoint{2.222018in}{3.130637in}}%
\pgfpathlineto{\pgfqpoint{2.216228in}{3.139665in}}%
\pgfpathlineto{\pgfqpoint{2.210435in}{3.148976in}}%
\pgfpathlineto{\pgfqpoint{2.199313in}{3.141845in}}%
\pgfpathlineto{\pgfqpoint{2.188178in}{3.135169in}}%
\pgfpathlineto{\pgfqpoint{2.177032in}{3.128870in}}%
\pgfpathlineto{\pgfqpoint{2.165875in}{3.122873in}}%
\pgfpathlineto{\pgfqpoint{2.154709in}{3.117102in}}%
\pgfpathlineto{\pgfqpoint{2.160470in}{3.108705in}}%
\pgfpathlineto{\pgfqpoint{2.166237in}{3.100171in}}%
\pgfpathlineto{\pgfqpoint{2.172010in}{3.091515in}}%
\pgfpathclose%
\pgfusepath{stroke,fill}%
\end{pgfscope}%
\begin{pgfscope}%
\pgfpathrectangle{\pgfqpoint{0.887500in}{0.275000in}}{\pgfqpoint{4.225000in}{4.225000in}}%
\pgfusepath{clip}%
\pgfsetbuttcap%
\pgfsetroundjoin%
\definecolor{currentfill}{rgb}{0.274149,0.751988,0.436601}%
\pgfsetfillcolor{currentfill}%
\pgfsetfillopacity{0.700000}%
\pgfsetlinewidth{0.501875pt}%
\definecolor{currentstroke}{rgb}{1.000000,1.000000,1.000000}%
\pgfsetstrokecolor{currentstroke}%
\pgfsetstrokeopacity{0.500000}%
\pgfsetdash{}{0pt}%
\pgfpathmoveto{\pgfqpoint{2.654229in}{3.230049in}}%
\pgfpathlineto{\pgfqpoint{2.665250in}{3.243680in}}%
\pgfpathlineto{\pgfqpoint{2.676269in}{3.257595in}}%
\pgfpathlineto{\pgfqpoint{2.687287in}{3.271744in}}%
\pgfpathlineto{\pgfqpoint{2.698303in}{3.286076in}}%
\pgfpathlineto{\pgfqpoint{2.709320in}{3.300541in}}%
\pgfpathlineto{\pgfqpoint{2.703492in}{3.296014in}}%
\pgfpathlineto{\pgfqpoint{2.697661in}{3.292714in}}%
\pgfpathlineto{\pgfqpoint{2.691826in}{3.290586in}}%
\pgfpathlineto{\pgfqpoint{2.685988in}{3.289575in}}%
\pgfpathlineto{\pgfqpoint{2.680146in}{3.289626in}}%
\pgfpathlineto{\pgfqpoint{2.669180in}{3.273398in}}%
\pgfpathlineto{\pgfqpoint{2.658211in}{3.257637in}}%
\pgfpathlineto{\pgfqpoint{2.647235in}{3.242658in}}%
\pgfpathlineto{\pgfqpoint{2.636250in}{3.228779in}}%
\pgfpathlineto{\pgfqpoint{2.625250in}{3.216314in}}%
\pgfpathlineto{\pgfqpoint{2.631052in}{3.217220in}}%
\pgfpathlineto{\pgfqpoint{2.636850in}{3.219069in}}%
\pgfpathlineto{\pgfqpoint{2.642645in}{3.221838in}}%
\pgfpathlineto{\pgfqpoint{2.648438in}{3.225505in}}%
\pgfpathclose%
\pgfusepath{stroke,fill}%
\end{pgfscope}%
\begin{pgfscope}%
\pgfpathrectangle{\pgfqpoint{0.887500in}{0.275000in}}{\pgfqpoint{4.225000in}{4.225000in}}%
\pgfusepath{clip}%
\pgfsetbuttcap%
\pgfsetroundjoin%
\definecolor{currentfill}{rgb}{0.220124,0.725509,0.466226}%
\pgfsetfillcolor{currentfill}%
\pgfsetfillopacity{0.700000}%
\pgfsetlinewidth{0.501875pt}%
\definecolor{currentstroke}{rgb}{1.000000,1.000000,1.000000}%
\pgfsetstrokecolor{currentstroke}%
\pgfsetstrokeopacity{0.500000}%
\pgfsetdash{}{0pt}%
\pgfpathmoveto{\pgfqpoint{2.599064in}{3.167841in}}%
\pgfpathlineto{\pgfqpoint{2.610108in}{3.179336in}}%
\pgfpathlineto{\pgfqpoint{2.621145in}{3.191350in}}%
\pgfpathlineto{\pgfqpoint{2.632177in}{3.203837in}}%
\pgfpathlineto{\pgfqpoint{2.643205in}{3.216752in}}%
\pgfpathlineto{\pgfqpoint{2.654229in}{3.230049in}}%
\pgfpathlineto{\pgfqpoint{2.648438in}{3.225505in}}%
\pgfpathlineto{\pgfqpoint{2.642645in}{3.221838in}}%
\pgfpathlineto{\pgfqpoint{2.636850in}{3.219069in}}%
\pgfpathlineto{\pgfqpoint{2.631052in}{3.217220in}}%
\pgfpathlineto{\pgfqpoint{2.625250in}{3.216314in}}%
\pgfpathlineto{\pgfqpoint{2.614232in}{3.205531in}}%
\pgfpathlineto{\pgfqpoint{2.603194in}{3.196324in}}%
\pgfpathlineto{\pgfqpoint{2.592140in}{3.188418in}}%
\pgfpathlineto{\pgfqpoint{2.581071in}{3.181538in}}%
\pgfpathlineto{\pgfqpoint{2.569990in}{3.175410in}}%
\pgfpathlineto{\pgfqpoint{2.575805in}{3.172937in}}%
\pgfpathlineto{\pgfqpoint{2.581619in}{3.170980in}}%
\pgfpathlineto{\pgfqpoint{2.587433in}{3.169502in}}%
\pgfpathlineto{\pgfqpoint{2.593248in}{3.168468in}}%
\pgfpathclose%
\pgfusepath{stroke,fill}%
\end{pgfscope}%
\begin{pgfscope}%
\pgfpathrectangle{\pgfqpoint{0.887500in}{0.275000in}}{\pgfqpoint{4.225000in}{4.225000in}}%
\pgfusepath{clip}%
\pgfsetbuttcap%
\pgfsetroundjoin%
\definecolor{currentfill}{rgb}{0.926106,0.897330,0.104071}%
\pgfsetfillcolor{currentfill}%
\pgfsetfillopacity{0.700000}%
\pgfsetlinewidth{0.501875pt}%
\definecolor{currentstroke}{rgb}{1.000000,1.000000,1.000000}%
\pgfsetstrokecolor{currentstroke}%
\pgfsetstrokeopacity{0.500000}%
\pgfsetdash{}{0pt}%
\pgfpathmoveto{\pgfqpoint{3.462004in}{3.752956in}}%
\pgfpathlineto{\pgfqpoint{3.472964in}{3.758611in}}%
\pgfpathlineto{\pgfqpoint{3.483920in}{3.764186in}}%
\pgfpathlineto{\pgfqpoint{3.494871in}{3.769669in}}%
\pgfpathlineto{\pgfqpoint{3.505817in}{3.775046in}}%
\pgfpathlineto{\pgfqpoint{3.516757in}{3.780320in}}%
\pgfpathlineto{\pgfqpoint{3.510521in}{3.784036in}}%
\pgfpathlineto{\pgfqpoint{3.504286in}{3.787445in}}%
\pgfpathlineto{\pgfqpoint{3.498053in}{3.790541in}}%
\pgfpathlineto{\pgfqpoint{3.491823in}{3.793329in}}%
\pgfpathlineto{\pgfqpoint{3.485595in}{3.795830in}}%
\pgfpathlineto{\pgfqpoint{3.474670in}{3.790387in}}%
\pgfpathlineto{\pgfqpoint{3.463740in}{3.784799in}}%
\pgfpathlineto{\pgfqpoint{3.452805in}{3.779070in}}%
\pgfpathlineto{\pgfqpoint{3.441866in}{3.773224in}}%
\pgfpathlineto{\pgfqpoint{3.430921in}{3.767283in}}%
\pgfpathlineto{\pgfqpoint{3.437132in}{3.764975in}}%
\pgfpathlineto{\pgfqpoint{3.443346in}{3.762406in}}%
\pgfpathlineto{\pgfqpoint{3.449563in}{3.759552in}}%
\pgfpathlineto{\pgfqpoint{3.455782in}{3.756403in}}%
\pgfpathclose%
\pgfusepath{stroke,fill}%
\end{pgfscope}%
\begin{pgfscope}%
\pgfpathrectangle{\pgfqpoint{0.887500in}{0.275000in}}{\pgfqpoint{4.225000in}{4.225000in}}%
\pgfusepath{clip}%
\pgfsetbuttcap%
\pgfsetroundjoin%
\definecolor{currentfill}{rgb}{0.824940,0.884720,0.106217}%
\pgfsetfillcolor{currentfill}%
\pgfsetfillopacity{0.700000}%
\pgfsetlinewidth{0.501875pt}%
\definecolor{currentstroke}{rgb}{1.000000,1.000000,1.000000}%
\pgfsetstrokecolor{currentstroke}%
\pgfsetstrokeopacity{0.500000}%
\pgfsetdash{}{0pt}%
\pgfpathmoveto{\pgfqpoint{3.266308in}{3.675368in}}%
\pgfpathlineto{\pgfqpoint{3.277307in}{3.681417in}}%
\pgfpathlineto{\pgfqpoint{3.288303in}{3.687585in}}%
\pgfpathlineto{\pgfqpoint{3.299295in}{3.693825in}}%
\pgfpathlineto{\pgfqpoint{3.310284in}{3.700089in}}%
\pgfpathlineto{\pgfqpoint{3.321270in}{3.706332in}}%
\pgfpathlineto{\pgfqpoint{3.315102in}{3.708159in}}%
\pgfpathlineto{\pgfqpoint{3.308938in}{3.709831in}}%
\pgfpathlineto{\pgfqpoint{3.302780in}{3.711371in}}%
\pgfpathlineto{\pgfqpoint{3.296626in}{3.712804in}}%
\pgfpathlineto{\pgfqpoint{3.290478in}{3.714151in}}%
\pgfpathlineto{\pgfqpoint{3.279514in}{3.707901in}}%
\pgfpathlineto{\pgfqpoint{3.268546in}{3.701630in}}%
\pgfpathlineto{\pgfqpoint{3.257574in}{3.695368in}}%
\pgfpathlineto{\pgfqpoint{3.246599in}{3.689145in}}%
\pgfpathlineto{\pgfqpoint{3.235621in}{3.682990in}}%
\pgfpathlineto{\pgfqpoint{3.241748in}{3.681650in}}%
\pgfpathlineto{\pgfqpoint{3.247880in}{3.680239in}}%
\pgfpathlineto{\pgfqpoint{3.254018in}{3.678737in}}%
\pgfpathlineto{\pgfqpoint{3.260160in}{3.677121in}}%
\pgfpathclose%
\pgfusepath{stroke,fill}%
\end{pgfscope}%
\begin{pgfscope}%
\pgfpathrectangle{\pgfqpoint{0.887500in}{0.275000in}}{\pgfqpoint{4.225000in}{4.225000in}}%
\pgfusepath{clip}%
\pgfsetbuttcap%
\pgfsetroundjoin%
\definecolor{currentfill}{rgb}{0.352360,0.783011,0.392636}%
\pgfsetfillcolor{currentfill}%
\pgfsetfillopacity{0.700000}%
\pgfsetlinewidth{0.501875pt}%
\definecolor{currentstroke}{rgb}{1.000000,1.000000,1.000000}%
\pgfsetstrokecolor{currentstroke}%
\pgfsetstrokeopacity{0.500000}%
\pgfsetdash{}{0pt}%
\pgfpathmoveto{\pgfqpoint{2.709320in}{3.300541in}}%
\pgfpathlineto{\pgfqpoint{2.720336in}{3.315087in}}%
\pgfpathlineto{\pgfqpoint{2.731354in}{3.329665in}}%
\pgfpathlineto{\pgfqpoint{2.742372in}{3.344258in}}%
\pgfpathlineto{\pgfqpoint{2.753391in}{3.358900in}}%
\pgfpathlineto{\pgfqpoint{2.764410in}{3.373628in}}%
\pgfpathlineto{\pgfqpoint{2.758535in}{3.370530in}}%
\pgfpathlineto{\pgfqpoint{2.752659in}{3.368645in}}%
\pgfpathlineto{\pgfqpoint{2.746780in}{3.367878in}}%
\pgfpathlineto{\pgfqpoint{2.740900in}{3.368134in}}%
\pgfpathlineto{\pgfqpoint{2.735018in}{3.369320in}}%
\pgfpathlineto{\pgfqpoint{2.724037in}{3.353772in}}%
\pgfpathlineto{\pgfqpoint{2.713058in}{3.338107in}}%
\pgfpathlineto{\pgfqpoint{2.702083in}{3.322214in}}%
\pgfpathlineto{\pgfqpoint{2.691113in}{3.306003in}}%
\pgfpathlineto{\pgfqpoint{2.680146in}{3.289626in}}%
\pgfpathlineto{\pgfqpoint{2.685988in}{3.289575in}}%
\pgfpathlineto{\pgfqpoint{2.691826in}{3.290586in}}%
\pgfpathlineto{\pgfqpoint{2.697661in}{3.292714in}}%
\pgfpathlineto{\pgfqpoint{2.703492in}{3.296014in}}%
\pgfpathclose%
\pgfusepath{stroke,fill}%
\end{pgfscope}%
\begin{pgfscope}%
\pgfpathrectangle{\pgfqpoint{0.887500in}{0.275000in}}{\pgfqpoint{4.225000in}{4.225000in}}%
\pgfusepath{clip}%
\pgfsetbuttcap%
\pgfsetroundjoin%
\definecolor{currentfill}{rgb}{0.964894,0.902323,0.123941}%
\pgfsetfillcolor{currentfill}%
\pgfsetfillopacity{0.700000}%
\pgfsetlinewidth{0.501875pt}%
\definecolor{currentstroke}{rgb}{1.000000,1.000000,1.000000}%
\pgfsetstrokecolor{currentstroke}%
\pgfsetstrokeopacity{0.500000}%
\pgfsetdash{}{0pt}%
\pgfpathmoveto{\pgfqpoint{3.602674in}{3.782402in}}%
\pgfpathlineto{\pgfqpoint{3.613602in}{3.787319in}}%
\pgfpathlineto{\pgfqpoint{3.624525in}{3.792209in}}%
\pgfpathlineto{\pgfqpoint{3.635444in}{3.797074in}}%
\pgfpathlineto{\pgfqpoint{3.646358in}{3.801913in}}%
\pgfpathlineto{\pgfqpoint{3.657268in}{3.806727in}}%
\pgfpathlineto{\pgfqpoint{3.650994in}{3.812025in}}%
\pgfpathlineto{\pgfqpoint{3.644722in}{3.817046in}}%
\pgfpathlineto{\pgfqpoint{3.638450in}{3.821767in}}%
\pgfpathlineto{\pgfqpoint{3.632179in}{3.826168in}}%
\pgfpathlineto{\pgfqpoint{3.625908in}{3.830225in}}%
\pgfpathlineto{\pgfqpoint{3.615013in}{3.825343in}}%
\pgfpathlineto{\pgfqpoint{3.604113in}{3.820442in}}%
\pgfpathlineto{\pgfqpoint{3.593209in}{3.815523in}}%
\pgfpathlineto{\pgfqpoint{3.582301in}{3.810587in}}%
\pgfpathlineto{\pgfqpoint{3.571388in}{3.805637in}}%
\pgfpathlineto{\pgfqpoint{3.577643in}{3.801563in}}%
\pgfpathlineto{\pgfqpoint{3.583899in}{3.797193in}}%
\pgfpathlineto{\pgfqpoint{3.590156in}{3.792536in}}%
\pgfpathlineto{\pgfqpoint{3.596414in}{3.787603in}}%
\pgfpathclose%
\pgfusepath{stroke,fill}%
\end{pgfscope}%
\begin{pgfscope}%
\pgfpathrectangle{\pgfqpoint{0.887500in}{0.275000in}}{\pgfqpoint{4.225000in}{4.225000in}}%
\pgfusepath{clip}%
\pgfsetbuttcap%
\pgfsetroundjoin%
\definecolor{currentfill}{rgb}{0.993248,0.906157,0.143936}%
\pgfsetfillcolor{currentfill}%
\pgfsetfillopacity{0.700000}%
\pgfsetlinewidth{0.501875pt}%
\definecolor{currentstroke}{rgb}{1.000000,1.000000,1.000000}%
\pgfsetstrokecolor{currentstroke}%
\pgfsetstrokeopacity{0.500000}%
\pgfsetdash{}{0pt}%
\pgfpathmoveto{\pgfqpoint{3.743191in}{3.799888in}}%
\pgfpathlineto{\pgfqpoint{3.754083in}{3.804394in}}%
\pgfpathlineto{\pgfqpoint{3.764971in}{3.808864in}}%
\pgfpathlineto{\pgfqpoint{3.775854in}{3.813301in}}%
\pgfpathlineto{\pgfqpoint{3.786731in}{3.817703in}}%
\pgfpathlineto{\pgfqpoint{3.780429in}{3.824167in}}%
\pgfpathlineto{\pgfqpoint{3.774130in}{3.830544in}}%
\pgfpathlineto{\pgfqpoint{3.767834in}{3.836838in}}%
\pgfpathlineto{\pgfqpoint{3.761540in}{3.842998in}}%
\pgfpathlineto{\pgfqpoint{3.755247in}{3.848952in}}%
\pgfpathlineto{\pgfqpoint{3.744379in}{3.844355in}}%
\pgfpathlineto{\pgfqpoint{3.733506in}{3.839735in}}%
\pgfpathlineto{\pgfqpoint{3.722629in}{3.835092in}}%
\pgfpathlineto{\pgfqpoint{3.711747in}{3.830425in}}%
\pgfpathlineto{\pgfqpoint{3.718032in}{3.824677in}}%
\pgfpathlineto{\pgfqpoint{3.724319in}{3.818712in}}%
\pgfpathlineto{\pgfqpoint{3.730607in}{3.812576in}}%
\pgfpathlineto{\pgfqpoint{3.736898in}{3.806302in}}%
\pgfpathclose%
\pgfusepath{stroke,fill}%
\end{pgfscope}%
\begin{pgfscope}%
\pgfpathrectangle{\pgfqpoint{0.887500in}{0.275000in}}{\pgfqpoint{4.225000in}{4.225000in}}%
\pgfusepath{clip}%
\pgfsetbuttcap%
\pgfsetroundjoin%
\definecolor{currentfill}{rgb}{0.709898,0.868751,0.169257}%
\pgfsetfillcolor{currentfill}%
\pgfsetfillopacity{0.700000}%
\pgfsetlinewidth{0.501875pt}%
\definecolor{currentstroke}{rgb}{1.000000,1.000000,1.000000}%
\pgfsetstrokecolor{currentstroke}%
\pgfsetstrokeopacity{0.500000}%
\pgfsetdash{}{0pt}%
\pgfpathmoveto{\pgfqpoint{3.070530in}{3.599185in}}%
\pgfpathlineto{\pgfqpoint{3.081563in}{3.604879in}}%
\pgfpathlineto{\pgfqpoint{3.092592in}{3.610508in}}%
\pgfpathlineto{\pgfqpoint{3.103618in}{3.616077in}}%
\pgfpathlineto{\pgfqpoint{3.114639in}{3.621590in}}%
\pgfpathlineto{\pgfqpoint{3.125657in}{3.627053in}}%
\pgfpathlineto{\pgfqpoint{3.119577in}{3.627714in}}%
\pgfpathlineto{\pgfqpoint{3.113503in}{3.628396in}}%
\pgfpathlineto{\pgfqpoint{3.107436in}{3.629133in}}%
\pgfpathlineto{\pgfqpoint{3.101375in}{3.629956in}}%
\pgfpathlineto{\pgfqpoint{3.095319in}{3.630867in}}%
\pgfpathlineto{\pgfqpoint{3.084324in}{3.625483in}}%
\pgfpathlineto{\pgfqpoint{3.073324in}{3.620132in}}%
\pgfpathlineto{\pgfqpoint{3.062321in}{3.614773in}}%
\pgfpathlineto{\pgfqpoint{3.051313in}{3.609367in}}%
\pgfpathlineto{\pgfqpoint{3.040302in}{3.603877in}}%
\pgfpathlineto{\pgfqpoint{3.046336in}{3.602854in}}%
\pgfpathlineto{\pgfqpoint{3.052375in}{3.601879in}}%
\pgfpathlineto{\pgfqpoint{3.058421in}{3.600950in}}%
\pgfpathlineto{\pgfqpoint{3.064472in}{3.600056in}}%
\pgfpathclose%
\pgfusepath{stroke,fill}%
\end{pgfscope}%
\begin{pgfscope}%
\pgfpathrectangle{\pgfqpoint{0.887500in}{0.275000in}}{\pgfqpoint{4.225000in}{4.225000in}}%
\pgfusepath{clip}%
\pgfsetbuttcap%
\pgfsetroundjoin%
\definecolor{currentfill}{rgb}{0.191090,0.708366,0.482284}%
\pgfsetfillcolor{currentfill}%
\pgfsetfillopacity{0.700000}%
\pgfsetlinewidth{0.501875pt}%
\definecolor{currentstroke}{rgb}{1.000000,1.000000,1.000000}%
\pgfsetstrokecolor{currentstroke}%
\pgfsetstrokeopacity{0.500000}%
\pgfsetdash{}{0pt}%
\pgfpathmoveto{\pgfqpoint{2.543712in}{3.119914in}}%
\pgfpathlineto{\pgfqpoint{2.554804in}{3.128020in}}%
\pgfpathlineto{\pgfqpoint{2.565884in}{3.136940in}}%
\pgfpathlineto{\pgfqpoint{2.576953in}{3.146595in}}%
\pgfpathlineto{\pgfqpoint{2.588012in}{3.156913in}}%
\pgfpathlineto{\pgfqpoint{2.599064in}{3.167841in}}%
\pgfpathlineto{\pgfqpoint{2.593248in}{3.168468in}}%
\pgfpathlineto{\pgfqpoint{2.587433in}{3.169502in}}%
\pgfpathlineto{\pgfqpoint{2.581619in}{3.170980in}}%
\pgfpathlineto{\pgfqpoint{2.575805in}{3.172937in}}%
\pgfpathlineto{\pgfqpoint{2.569990in}{3.175410in}}%
\pgfpathlineto{\pgfqpoint{2.558899in}{3.169758in}}%
\pgfpathlineto{\pgfqpoint{2.547801in}{3.164307in}}%
\pgfpathlineto{\pgfqpoint{2.536700in}{3.158837in}}%
\pgfpathlineto{\pgfqpoint{2.525595in}{3.153393in}}%
\pgfpathlineto{\pgfqpoint{2.514483in}{3.148104in}}%
\pgfpathlineto{\pgfqpoint{2.520322in}{3.142311in}}%
\pgfpathlineto{\pgfqpoint{2.526164in}{3.136633in}}%
\pgfpathlineto{\pgfqpoint{2.532009in}{3.131033in}}%
\pgfpathlineto{\pgfqpoint{2.537858in}{3.125473in}}%
\pgfpathclose%
\pgfusepath{stroke,fill}%
\end{pgfscope}%
\begin{pgfscope}%
\pgfpathrectangle{\pgfqpoint{0.887500in}{0.275000in}}{\pgfqpoint{4.225000in}{4.225000in}}%
\pgfusepath{clip}%
\pgfsetbuttcap%
\pgfsetroundjoin%
\definecolor{currentfill}{rgb}{0.449368,0.813768,0.335384}%
\pgfsetfillcolor{currentfill}%
\pgfsetfillopacity{0.700000}%
\pgfsetlinewidth{0.501875pt}%
\definecolor{currentstroke}{rgb}{1.000000,1.000000,1.000000}%
\pgfsetstrokecolor{currentstroke}%
\pgfsetstrokeopacity{0.500000}%
\pgfsetdash{}{0pt}%
\pgfpathmoveto{\pgfqpoint{2.764410in}{3.373628in}}%
\pgfpathlineto{\pgfqpoint{2.775430in}{3.388482in}}%
\pgfpathlineto{\pgfqpoint{2.786449in}{3.403499in}}%
\pgfpathlineto{\pgfqpoint{2.797469in}{3.418717in}}%
\pgfpathlineto{\pgfqpoint{2.808488in}{3.434168in}}%
\pgfpathlineto{\pgfqpoint{2.819509in}{3.449716in}}%
\pgfpathlineto{\pgfqpoint{2.813590in}{3.448102in}}%
\pgfpathlineto{\pgfqpoint{2.807674in}{3.447289in}}%
\pgfpathlineto{\pgfqpoint{2.801759in}{3.447207in}}%
\pgfpathlineto{\pgfqpoint{2.795846in}{3.447786in}}%
\pgfpathlineto{\pgfqpoint{2.789935in}{3.448955in}}%
\pgfpathlineto{\pgfqpoint{2.778951in}{3.432566in}}%
\pgfpathlineto{\pgfqpoint{2.767968in}{3.416378in}}%
\pgfpathlineto{\pgfqpoint{2.756984in}{3.400512in}}%
\pgfpathlineto{\pgfqpoint{2.746001in}{3.384863in}}%
\pgfpathlineto{\pgfqpoint{2.735018in}{3.369320in}}%
\pgfpathlineto{\pgfqpoint{2.740900in}{3.368134in}}%
\pgfpathlineto{\pgfqpoint{2.746780in}{3.367878in}}%
\pgfpathlineto{\pgfqpoint{2.752659in}{3.368645in}}%
\pgfpathlineto{\pgfqpoint{2.758535in}{3.370530in}}%
\pgfpathclose%
\pgfusepath{stroke,fill}%
\end{pgfscope}%
\begin{pgfscope}%
\pgfpathrectangle{\pgfqpoint{0.887500in}{0.275000in}}{\pgfqpoint{4.225000in}{4.225000in}}%
\pgfusepath{clip}%
\pgfsetbuttcap%
\pgfsetroundjoin%
\definecolor{currentfill}{rgb}{0.616293,0.852709,0.230052}%
\pgfsetfillcolor{currentfill}%
\pgfsetfillopacity{0.700000}%
\pgfsetlinewidth{0.501875pt}%
\definecolor{currentstroke}{rgb}{1.000000,1.000000,1.000000}%
\pgfsetstrokecolor{currentstroke}%
\pgfsetstrokeopacity{0.500000}%
\pgfsetdash{}{0pt}%
\pgfpathmoveto{\pgfqpoint{2.874677in}{3.517786in}}%
\pgfpathlineto{\pgfqpoint{2.885730in}{3.527238in}}%
\pgfpathlineto{\pgfqpoint{2.896787in}{3.534967in}}%
\pgfpathlineto{\pgfqpoint{2.907847in}{3.541317in}}%
\pgfpathlineto{\pgfqpoint{2.918907in}{3.546643in}}%
\pgfpathlineto{\pgfqpoint{2.929965in}{3.551297in}}%
\pgfpathlineto{\pgfqpoint{2.923981in}{3.552254in}}%
\pgfpathlineto{\pgfqpoint{2.918002in}{3.553281in}}%
\pgfpathlineto{\pgfqpoint{2.912028in}{3.554372in}}%
\pgfpathlineto{\pgfqpoint{2.906061in}{3.555521in}}%
\pgfpathlineto{\pgfqpoint{2.900099in}{3.556724in}}%
\pgfpathlineto{\pgfqpoint{2.889062in}{3.551907in}}%
\pgfpathlineto{\pgfqpoint{2.878025in}{3.546339in}}%
\pgfpathlineto{\pgfqpoint{2.866989in}{3.539646in}}%
\pgfpathlineto{\pgfqpoint{2.855957in}{3.531451in}}%
\pgfpathlineto{\pgfqpoint{2.844933in}{3.521387in}}%
\pgfpathlineto{\pgfqpoint{2.850872in}{3.520233in}}%
\pgfpathlineto{\pgfqpoint{2.856816in}{3.519268in}}%
\pgfpathlineto{\pgfqpoint{2.862765in}{3.518519in}}%
\pgfpathlineto{\pgfqpoint{2.868719in}{3.518016in}}%
\pgfpathclose%
\pgfusepath{stroke,fill}%
\end{pgfscope}%
\begin{pgfscope}%
\pgfpathrectangle{\pgfqpoint{0.887500in}{0.275000in}}{\pgfqpoint{4.225000in}{4.225000in}}%
\pgfusepath{clip}%
\pgfsetbuttcap%
\pgfsetroundjoin%
\definecolor{currentfill}{rgb}{0.166383,0.690856,0.496502}%
\pgfsetfillcolor{currentfill}%
\pgfsetfillopacity{0.700000}%
\pgfsetlinewidth{0.501875pt}%
\definecolor{currentstroke}{rgb}{1.000000,1.000000,1.000000}%
\pgfsetstrokecolor{currentstroke}%
\pgfsetstrokeopacity{0.500000}%
\pgfsetdash{}{0pt}%
\pgfpathmoveto{\pgfqpoint{2.262523in}{3.073481in}}%
\pgfpathlineto{\pgfqpoint{2.273662in}{3.080889in}}%
\pgfpathlineto{\pgfqpoint{2.284793in}{3.088473in}}%
\pgfpathlineto{\pgfqpoint{2.295918in}{3.096261in}}%
\pgfpathlineto{\pgfqpoint{2.307039in}{3.104161in}}%
\pgfpathlineto{\pgfqpoint{2.318157in}{3.111999in}}%
\pgfpathlineto{\pgfqpoint{2.312371in}{3.118768in}}%
\pgfpathlineto{\pgfqpoint{2.306583in}{3.125908in}}%
\pgfpathlineto{\pgfqpoint{2.300789in}{3.133499in}}%
\pgfpathlineto{\pgfqpoint{2.294990in}{3.141616in}}%
\pgfpathlineto{\pgfqpoint{2.289183in}{3.150339in}}%
\pgfpathlineto{\pgfqpoint{2.278078in}{3.142553in}}%
\pgfpathlineto{\pgfqpoint{2.266970in}{3.134819in}}%
\pgfpathlineto{\pgfqpoint{2.255854in}{3.127325in}}%
\pgfpathlineto{\pgfqpoint{2.244728in}{3.120174in}}%
\pgfpathlineto{\pgfqpoint{2.233593in}{3.113343in}}%
\pgfpathlineto{\pgfqpoint{2.239379in}{3.105030in}}%
\pgfpathlineto{\pgfqpoint{2.245164in}{3.096910in}}%
\pgfpathlineto{\pgfqpoint{2.250950in}{3.088961in}}%
\pgfpathlineto{\pgfqpoint{2.256736in}{3.081159in}}%
\pgfpathclose%
\pgfusepath{stroke,fill}%
\end{pgfscope}%
\begin{pgfscope}%
\pgfpathrectangle{\pgfqpoint{0.887500in}{0.275000in}}{\pgfqpoint{4.225000in}{4.225000in}}%
\pgfusepath{clip}%
\pgfsetbuttcap%
\pgfsetroundjoin%
\definecolor{currentfill}{rgb}{0.180653,0.701402,0.488189}%
\pgfsetfillcolor{currentfill}%
\pgfsetfillopacity{0.700000}%
\pgfsetlinewidth{0.501875pt}%
\definecolor{currentstroke}{rgb}{1.000000,1.000000,1.000000}%
\pgfsetstrokecolor{currentstroke}%
\pgfsetstrokeopacity{0.500000}%
\pgfsetdash{}{0pt}%
\pgfpathmoveto{\pgfqpoint{2.402811in}{3.115378in}}%
\pgfpathlineto{\pgfqpoint{2.413997in}{3.118968in}}%
\pgfpathlineto{\pgfqpoint{2.425191in}{3.121746in}}%
\pgfpathlineto{\pgfqpoint{2.436386in}{3.124028in}}%
\pgfpathlineto{\pgfqpoint{2.447579in}{3.126130in}}%
\pgfpathlineto{\pgfqpoint{2.458763in}{3.128367in}}%
\pgfpathlineto{\pgfqpoint{2.452935in}{3.134724in}}%
\pgfpathlineto{\pgfqpoint{2.447111in}{3.141099in}}%
\pgfpathlineto{\pgfqpoint{2.441289in}{3.147565in}}%
\pgfpathlineto{\pgfqpoint{2.435469in}{3.154196in}}%
\pgfpathlineto{\pgfqpoint{2.429649in}{3.161065in}}%
\pgfpathlineto{\pgfqpoint{2.418489in}{3.158055in}}%
\pgfpathlineto{\pgfqpoint{2.407324in}{3.155073in}}%
\pgfpathlineto{\pgfqpoint{2.396156in}{3.151909in}}%
\pgfpathlineto{\pgfqpoint{2.384989in}{3.148356in}}%
\pgfpathlineto{\pgfqpoint{2.373827in}{3.144208in}}%
\pgfpathlineto{\pgfqpoint{2.379622in}{3.138060in}}%
\pgfpathlineto{\pgfqpoint{2.385416in}{3.132214in}}%
\pgfpathlineto{\pgfqpoint{2.391211in}{3.126560in}}%
\pgfpathlineto{\pgfqpoint{2.397009in}{3.120985in}}%
\pgfpathclose%
\pgfusepath{stroke,fill}%
\end{pgfscope}%
\begin{pgfscope}%
\pgfpathrectangle{\pgfqpoint{0.887500in}{0.275000in}}{\pgfqpoint{4.225000in}{4.225000in}}%
\pgfusepath{clip}%
\pgfsetbuttcap%
\pgfsetroundjoin%
\definecolor{currentfill}{rgb}{0.150148,0.676631,0.506589}%
\pgfsetfillcolor{currentfill}%
\pgfsetfillopacity{0.700000}%
\pgfsetlinewidth{0.501875pt}%
\definecolor{currentstroke}{rgb}{1.000000,1.000000,1.000000}%
\pgfsetstrokecolor{currentstroke}%
\pgfsetstrokeopacity{0.500000}%
\pgfsetdash{}{0pt}%
\pgfpathmoveto{\pgfqpoint{2.121887in}{3.052355in}}%
\pgfpathlineto{\pgfqpoint{2.133061in}{3.059071in}}%
\pgfpathlineto{\pgfqpoint{2.144241in}{3.065341in}}%
\pgfpathlineto{\pgfqpoint{2.155424in}{3.071292in}}%
\pgfpathlineto{\pgfqpoint{2.166608in}{3.077053in}}%
\pgfpathlineto{\pgfqpoint{2.177789in}{3.082751in}}%
\pgfpathlineto{\pgfqpoint{2.172010in}{3.091515in}}%
\pgfpathlineto{\pgfqpoint{2.166237in}{3.100171in}}%
\pgfpathlineto{\pgfqpoint{2.160470in}{3.108705in}}%
\pgfpathlineto{\pgfqpoint{2.154709in}{3.117102in}}%
\pgfpathlineto{\pgfqpoint{2.143537in}{3.111442in}}%
\pgfpathlineto{\pgfqpoint{2.132362in}{3.105738in}}%
\pgfpathlineto{\pgfqpoint{2.121187in}{3.099835in}}%
\pgfpathlineto{\pgfqpoint{2.110017in}{3.093578in}}%
\pgfpathlineto{\pgfqpoint{2.098856in}{3.086810in}}%
\pgfpathlineto{\pgfqpoint{2.104593in}{3.078884in}}%
\pgfpathlineto{\pgfqpoint{2.110344in}{3.070470in}}%
\pgfpathlineto{\pgfqpoint{2.116109in}{3.061612in}}%
\pgfpathclose%
\pgfusepath{stroke,fill}%
\end{pgfscope}%
\begin{pgfscope}%
\pgfpathrectangle{\pgfqpoint{0.887500in}{0.275000in}}{\pgfqpoint{4.225000in}{4.225000in}}%
\pgfusepath{clip}%
\pgfsetbuttcap%
\pgfsetroundjoin%
\definecolor{currentfill}{rgb}{0.545524,0.838039,0.275626}%
\pgfsetfillcolor{currentfill}%
\pgfsetfillopacity{0.700000}%
\pgfsetlinewidth{0.501875pt}%
\definecolor{currentstroke}{rgb}{1.000000,1.000000,1.000000}%
\pgfsetstrokecolor{currentstroke}%
\pgfsetstrokeopacity{0.500000}%
\pgfsetdash{}{0pt}%
\pgfpathmoveto{\pgfqpoint{2.819509in}{3.449716in}}%
\pgfpathlineto{\pgfqpoint{2.830532in}{3.465049in}}%
\pgfpathlineto{\pgfqpoint{2.841559in}{3.479844in}}%
\pgfpathlineto{\pgfqpoint{2.852592in}{3.493780in}}%
\pgfpathlineto{\pgfqpoint{2.863631in}{3.506535in}}%
\pgfpathlineto{\pgfqpoint{2.874677in}{3.517786in}}%
\pgfpathlineto{\pgfqpoint{2.868719in}{3.518016in}}%
\pgfpathlineto{\pgfqpoint{2.862765in}{3.518519in}}%
\pgfpathlineto{\pgfqpoint{2.856816in}{3.519268in}}%
\pgfpathlineto{\pgfqpoint{2.850872in}{3.520233in}}%
\pgfpathlineto{\pgfqpoint{2.844933in}{3.521387in}}%
\pgfpathlineto{\pgfqpoint{2.833918in}{3.509383in}}%
\pgfpathlineto{\pgfqpoint{2.822911in}{3.495773in}}%
\pgfpathlineto{\pgfqpoint{2.811913in}{3.480922in}}%
\pgfpathlineto{\pgfqpoint{2.800921in}{3.465195in}}%
\pgfpathlineto{\pgfqpoint{2.789935in}{3.448955in}}%
\pgfpathlineto{\pgfqpoint{2.795846in}{3.447786in}}%
\pgfpathlineto{\pgfqpoint{2.801759in}{3.447207in}}%
\pgfpathlineto{\pgfqpoint{2.807674in}{3.447289in}}%
\pgfpathlineto{\pgfqpoint{2.813590in}{3.448102in}}%
\pgfpathclose%
\pgfusepath{stroke,fill}%
\end{pgfscope}%
\begin{pgfscope}%
\pgfpathrectangle{\pgfqpoint{0.887500in}{0.275000in}}{\pgfqpoint{4.225000in}{4.225000in}}%
\pgfusepath{clip}%
\pgfsetbuttcap%
\pgfsetroundjoin%
\definecolor{currentfill}{rgb}{0.896320,0.893616,0.096335}%
\pgfsetfillcolor{currentfill}%
\pgfsetfillopacity{0.700000}%
\pgfsetlinewidth{0.501875pt}%
\definecolor{currentstroke}{rgb}{1.000000,1.000000,1.000000}%
\pgfsetstrokecolor{currentstroke}%
\pgfsetstrokeopacity{0.500000}%
\pgfsetdash{}{0pt}%
\pgfpathmoveto{\pgfqpoint{3.407140in}{3.723939in}}%
\pgfpathlineto{\pgfqpoint{3.418121in}{3.729805in}}%
\pgfpathlineto{\pgfqpoint{3.429098in}{3.735646in}}%
\pgfpathlineto{\pgfqpoint{3.440071in}{3.741461in}}%
\pgfpathlineto{\pgfqpoint{3.451039in}{3.747235in}}%
\pgfpathlineto{\pgfqpoint{3.462004in}{3.752956in}}%
\pgfpathlineto{\pgfqpoint{3.455782in}{3.756403in}}%
\pgfpathlineto{\pgfqpoint{3.449563in}{3.759552in}}%
\pgfpathlineto{\pgfqpoint{3.443346in}{3.762406in}}%
\pgfpathlineto{\pgfqpoint{3.437132in}{3.764975in}}%
\pgfpathlineto{\pgfqpoint{3.430921in}{3.767283in}}%
\pgfpathlineto{\pgfqpoint{3.419973in}{3.761271in}}%
\pgfpathlineto{\pgfqpoint{3.409020in}{3.755212in}}%
\pgfpathlineto{\pgfqpoint{3.398064in}{3.749130in}}%
\pgfpathlineto{\pgfqpoint{3.387104in}{3.743048in}}%
\pgfpathlineto{\pgfqpoint{3.376141in}{3.736970in}}%
\pgfpathlineto{\pgfqpoint{3.382334in}{3.734855in}}%
\pgfpathlineto{\pgfqpoint{3.388530in}{3.732515in}}%
\pgfpathlineto{\pgfqpoint{3.394730in}{3.729925in}}%
\pgfpathlineto{\pgfqpoint{3.400933in}{3.727070in}}%
\pgfpathclose%
\pgfusepath{stroke,fill}%
\end{pgfscope}%
\begin{pgfscope}%
\pgfpathrectangle{\pgfqpoint{0.887500in}{0.275000in}}{\pgfqpoint{4.225000in}{4.225000in}}%
\pgfusepath{clip}%
\pgfsetbuttcap%
\pgfsetroundjoin%
\definecolor{currentfill}{rgb}{0.793760,0.880678,0.120005}%
\pgfsetfillcolor{currentfill}%
\pgfsetfillopacity{0.700000}%
\pgfsetlinewidth{0.501875pt}%
\definecolor{currentstroke}{rgb}{1.000000,1.000000,1.000000}%
\pgfsetstrokecolor{currentstroke}%
\pgfsetstrokeopacity{0.500000}%
\pgfsetdash{}{0pt}%
\pgfpathmoveto{\pgfqpoint{3.211273in}{3.648141in}}%
\pgfpathlineto{\pgfqpoint{3.222286in}{3.653195in}}%
\pgfpathlineto{\pgfqpoint{3.233296in}{3.658399in}}%
\pgfpathlineto{\pgfqpoint{3.244303in}{3.663817in}}%
\pgfpathlineto{\pgfqpoint{3.255307in}{3.669486in}}%
\pgfpathlineto{\pgfqpoint{3.266308in}{3.675368in}}%
\pgfpathlineto{\pgfqpoint{3.260160in}{3.677121in}}%
\pgfpathlineto{\pgfqpoint{3.254018in}{3.678737in}}%
\pgfpathlineto{\pgfqpoint{3.247880in}{3.680239in}}%
\pgfpathlineto{\pgfqpoint{3.241748in}{3.681650in}}%
\pgfpathlineto{\pgfqpoint{3.235621in}{3.682990in}}%
\pgfpathlineto{\pgfqpoint{3.224640in}{3.676934in}}%
\pgfpathlineto{\pgfqpoint{3.213656in}{3.671006in}}%
\pgfpathlineto{\pgfqpoint{3.202669in}{3.665232in}}%
\pgfpathlineto{\pgfqpoint{3.191678in}{3.659597in}}%
\pgfpathlineto{\pgfqpoint{3.180684in}{3.654071in}}%
\pgfpathlineto{\pgfqpoint{3.186791in}{3.653079in}}%
\pgfpathlineto{\pgfqpoint{3.192903in}{3.652023in}}%
\pgfpathlineto{\pgfqpoint{3.199021in}{3.650869in}}%
\pgfpathlineto{\pgfqpoint{3.205144in}{3.649585in}}%
\pgfpathclose%
\pgfusepath{stroke,fill}%
\end{pgfscope}%
\begin{pgfscope}%
\pgfpathrectangle{\pgfqpoint{0.887500in}{0.275000in}}{\pgfqpoint{4.225000in}{4.225000in}}%
\pgfusepath{clip}%
\pgfsetbuttcap%
\pgfsetroundjoin%
\definecolor{currentfill}{rgb}{0.974417,0.903590,0.130215}%
\pgfsetfillcolor{currentfill}%
\pgfsetfillopacity{0.700000}%
\pgfsetlinewidth{0.501875pt}%
\definecolor{currentstroke}{rgb}{1.000000,1.000000,1.000000}%
\pgfsetstrokecolor{currentstroke}%
\pgfsetstrokeopacity{0.500000}%
\pgfsetdash{}{0pt}%
\pgfpathmoveto{\pgfqpoint{3.688654in}{3.776803in}}%
\pgfpathlineto{\pgfqpoint{3.699571in}{3.781497in}}%
\pgfpathlineto{\pgfqpoint{3.710484in}{3.786151in}}%
\pgfpathlineto{\pgfqpoint{3.721391in}{3.790768in}}%
\pgfpathlineto{\pgfqpoint{3.732293in}{3.795346in}}%
\pgfpathlineto{\pgfqpoint{3.743191in}{3.799888in}}%
\pgfpathlineto{\pgfqpoint{3.736898in}{3.806302in}}%
\pgfpathlineto{\pgfqpoint{3.730607in}{3.812576in}}%
\pgfpathlineto{\pgfqpoint{3.724319in}{3.818712in}}%
\pgfpathlineto{\pgfqpoint{3.718032in}{3.824677in}}%
\pgfpathlineto{\pgfqpoint{3.711747in}{3.830425in}}%
\pgfpathlineto{\pgfqpoint{3.700860in}{3.825734in}}%
\pgfpathlineto{\pgfqpoint{3.689969in}{3.821019in}}%
\pgfpathlineto{\pgfqpoint{3.679073in}{3.816279in}}%
\pgfpathlineto{\pgfqpoint{3.668173in}{3.811516in}}%
\pgfpathlineto{\pgfqpoint{3.657268in}{3.806727in}}%
\pgfpathlineto{\pgfqpoint{3.663542in}{3.801173in}}%
\pgfpathlineto{\pgfqpoint{3.669818in}{3.795385in}}%
\pgfpathlineto{\pgfqpoint{3.676095in}{3.789385in}}%
\pgfpathlineto{\pgfqpoint{3.682374in}{3.783190in}}%
\pgfpathclose%
\pgfusepath{stroke,fill}%
\end{pgfscope}%
\begin{pgfscope}%
\pgfpathrectangle{\pgfqpoint{0.887500in}{0.275000in}}{\pgfqpoint{4.225000in}{4.225000in}}%
\pgfusepath{clip}%
\pgfsetbuttcap%
\pgfsetroundjoin%
\definecolor{currentfill}{rgb}{0.955300,0.901065,0.118128}%
\pgfsetfillcolor{currentfill}%
\pgfsetfillopacity{0.700000}%
\pgfsetlinewidth{0.501875pt}%
\definecolor{currentstroke}{rgb}{1.000000,1.000000,1.000000}%
\pgfsetstrokecolor{currentstroke}%
\pgfsetstrokeopacity{0.500000}%
\pgfsetdash{}{0pt}%
\pgfpathmoveto{\pgfqpoint{3.547968in}{3.757345in}}%
\pgfpathlineto{\pgfqpoint{3.558918in}{3.762431in}}%
\pgfpathlineto{\pgfqpoint{3.569864in}{3.767476in}}%
\pgfpathlineto{\pgfqpoint{3.580805in}{3.772484in}}%
\pgfpathlineto{\pgfqpoint{3.591742in}{3.777458in}}%
\pgfpathlineto{\pgfqpoint{3.602674in}{3.782402in}}%
\pgfpathlineto{\pgfqpoint{3.596414in}{3.787603in}}%
\pgfpathlineto{\pgfqpoint{3.590156in}{3.792536in}}%
\pgfpathlineto{\pgfqpoint{3.583899in}{3.797193in}}%
\pgfpathlineto{\pgfqpoint{3.577643in}{3.801563in}}%
\pgfpathlineto{\pgfqpoint{3.571388in}{3.805637in}}%
\pgfpathlineto{\pgfqpoint{3.560472in}{3.800667in}}%
\pgfpathlineto{\pgfqpoint{3.549550in}{3.795664in}}%
\pgfpathlineto{\pgfqpoint{3.538624in}{3.790615in}}%
\pgfpathlineto{\pgfqpoint{3.527693in}{3.785505in}}%
\pgfpathlineto{\pgfqpoint{3.516757in}{3.780320in}}%
\pgfpathlineto{\pgfqpoint{3.522996in}{3.776303in}}%
\pgfpathlineto{\pgfqpoint{3.529236in}{3.771990in}}%
\pgfpathlineto{\pgfqpoint{3.535478in}{3.767389in}}%
\pgfpathlineto{\pgfqpoint{3.541722in}{3.762505in}}%
\pgfpathclose%
\pgfusepath{stroke,fill}%
\end{pgfscope}%
\begin{pgfscope}%
\pgfpathrectangle{\pgfqpoint{0.887500in}{0.275000in}}{\pgfqpoint{4.225000in}{4.225000in}}%
\pgfusepath{clip}%
\pgfsetbuttcap%
\pgfsetroundjoin%
\definecolor{currentfill}{rgb}{0.688944,0.865448,0.182725}%
\pgfsetfillcolor{currentfill}%
\pgfsetfillopacity{0.700000}%
\pgfsetlinewidth{0.501875pt}%
\definecolor{currentstroke}{rgb}{1.000000,1.000000,1.000000}%
\pgfsetstrokecolor{currentstroke}%
\pgfsetstrokeopacity{0.500000}%
\pgfsetdash{}{0pt}%
\pgfpathmoveto{\pgfqpoint{3.015307in}{3.570535in}}%
\pgfpathlineto{\pgfqpoint{3.026359in}{3.576074in}}%
\pgfpathlineto{\pgfqpoint{3.037407in}{3.581788in}}%
\pgfpathlineto{\pgfqpoint{3.048452in}{3.587597in}}%
\pgfpathlineto{\pgfqpoint{3.059493in}{3.593419in}}%
\pgfpathlineto{\pgfqpoint{3.070530in}{3.599185in}}%
\pgfpathlineto{\pgfqpoint{3.064472in}{3.600056in}}%
\pgfpathlineto{\pgfqpoint{3.058421in}{3.600950in}}%
\pgfpathlineto{\pgfqpoint{3.052375in}{3.601879in}}%
\pgfpathlineto{\pgfqpoint{3.046336in}{3.602854in}}%
\pgfpathlineto{\pgfqpoint{3.040302in}{3.603877in}}%
\pgfpathlineto{\pgfqpoint{3.029287in}{3.598261in}}%
\pgfpathlineto{\pgfqpoint{3.018268in}{3.592511in}}%
\pgfpathlineto{\pgfqpoint{3.007245in}{3.586702in}}%
\pgfpathlineto{\pgfqpoint{2.996219in}{3.580927in}}%
\pgfpathlineto{\pgfqpoint{2.985188in}{3.575278in}}%
\pgfpathlineto{\pgfqpoint{2.991200in}{3.574248in}}%
\pgfpathlineto{\pgfqpoint{2.997218in}{3.573258in}}%
\pgfpathlineto{\pgfqpoint{3.003242in}{3.572308in}}%
\pgfpathlineto{\pgfqpoint{3.009271in}{3.571400in}}%
\pgfpathclose%
\pgfusepath{stroke,fill}%
\end{pgfscope}%
\begin{pgfscope}%
\pgfpathrectangle{\pgfqpoint{0.887500in}{0.275000in}}{\pgfqpoint{4.225000in}{4.225000in}}%
\pgfusepath{clip}%
\pgfsetbuttcap%
\pgfsetroundjoin%
\definecolor{currentfill}{rgb}{0.150148,0.676631,0.506589}%
\pgfsetfillcolor{currentfill}%
\pgfsetfillopacity{0.700000}%
\pgfsetlinewidth{0.501875pt}%
\definecolor{currentstroke}{rgb}{1.000000,1.000000,1.000000}%
\pgfsetstrokecolor{currentstroke}%
\pgfsetstrokeopacity{0.500000}%
\pgfsetdash{}{0pt}%
\pgfpathmoveto{\pgfqpoint{2.206759in}{3.037782in}}%
\pgfpathlineto{\pgfqpoint{2.217920in}{3.044868in}}%
\pgfpathlineto{\pgfqpoint{2.229077in}{3.051944in}}%
\pgfpathlineto{\pgfqpoint{2.240230in}{3.059046in}}%
\pgfpathlineto{\pgfqpoint{2.251380in}{3.066213in}}%
\pgfpathlineto{\pgfqpoint{2.262523in}{3.073481in}}%
\pgfpathlineto{\pgfqpoint{2.256736in}{3.081159in}}%
\pgfpathlineto{\pgfqpoint{2.250950in}{3.088961in}}%
\pgfpathlineto{\pgfqpoint{2.245164in}{3.096910in}}%
\pgfpathlineto{\pgfqpoint{2.239379in}{3.105030in}}%
\pgfpathlineto{\pgfqpoint{2.233593in}{3.113343in}}%
\pgfpathlineto{\pgfqpoint{2.222449in}{3.106796in}}%
\pgfpathlineto{\pgfqpoint{2.211295in}{3.100498in}}%
\pgfpathlineto{\pgfqpoint{2.200134in}{3.094415in}}%
\pgfpathlineto{\pgfqpoint{2.188965in}{3.088511in}}%
\pgfpathlineto{\pgfqpoint{2.177789in}{3.082751in}}%
\pgfpathlineto{\pgfqpoint{2.183574in}{3.073891in}}%
\pgfpathlineto{\pgfqpoint{2.189363in}{3.064951in}}%
\pgfpathlineto{\pgfqpoint{2.195158in}{3.055944in}}%
\pgfpathlineto{\pgfqpoint{2.200956in}{3.046883in}}%
\pgfpathclose%
\pgfusepath{stroke,fill}%
\end{pgfscope}%
\begin{pgfscope}%
\pgfpathrectangle{\pgfqpoint{0.887500in}{0.275000in}}{\pgfqpoint{4.225000in}{4.225000in}}%
\pgfusepath{clip}%
\pgfsetbuttcap%
\pgfsetroundjoin%
\definecolor{currentfill}{rgb}{0.175707,0.697900,0.491033}%
\pgfsetfillcolor{currentfill}%
\pgfsetfillopacity{0.700000}%
\pgfsetlinewidth{0.501875pt}%
\definecolor{currentstroke}{rgb}{1.000000,1.000000,1.000000}%
\pgfsetstrokecolor{currentstroke}%
\pgfsetstrokeopacity{0.500000}%
\pgfsetdash{}{0pt}%
\pgfpathmoveto{\pgfqpoint{2.488000in}{3.094297in}}%
\pgfpathlineto{\pgfqpoint{2.499181in}{3.097232in}}%
\pgfpathlineto{\pgfqpoint{2.510342in}{3.101281in}}%
\pgfpathlineto{\pgfqpoint{2.521482in}{3.106464in}}%
\pgfpathlineto{\pgfqpoint{2.532605in}{3.112702in}}%
\pgfpathlineto{\pgfqpoint{2.543712in}{3.119914in}}%
\pgfpathlineto{\pgfqpoint{2.537858in}{3.125473in}}%
\pgfpathlineto{\pgfqpoint{2.532009in}{3.131033in}}%
\pgfpathlineto{\pgfqpoint{2.526164in}{3.136633in}}%
\pgfpathlineto{\pgfqpoint{2.520322in}{3.142311in}}%
\pgfpathlineto{\pgfqpoint{2.514483in}{3.148104in}}%
\pgfpathlineto{\pgfqpoint{2.503364in}{3.143098in}}%
\pgfpathlineto{\pgfqpoint{2.492234in}{3.138502in}}%
\pgfpathlineto{\pgfqpoint{2.481092in}{3.134444in}}%
\pgfpathlineto{\pgfqpoint{2.469936in}{3.131053in}}%
\pgfpathlineto{\pgfqpoint{2.458763in}{3.128367in}}%
\pgfpathlineto{\pgfqpoint{2.464597in}{3.121955in}}%
\pgfpathlineto{\pgfqpoint{2.470436in}{3.115416in}}%
\pgfpathlineto{\pgfqpoint{2.476282in}{3.108675in}}%
\pgfpathlineto{\pgfqpoint{2.482137in}{3.101660in}}%
\pgfpathclose%
\pgfusepath{stroke,fill}%
\end{pgfscope}%
\begin{pgfscope}%
\pgfpathrectangle{\pgfqpoint{0.887500in}{0.275000in}}{\pgfqpoint{4.225000in}{4.225000in}}%
\pgfusepath{clip}%
\pgfsetbuttcap%
\pgfsetroundjoin%
\definecolor{currentfill}{rgb}{0.140210,0.665859,0.513427}%
\pgfsetfillcolor{currentfill}%
\pgfsetfillopacity{0.700000}%
\pgfsetlinewidth{0.501875pt}%
\definecolor{currentstroke}{rgb}{1.000000,1.000000,1.000000}%
\pgfsetstrokecolor{currentstroke}%
\pgfsetstrokeopacity{0.500000}%
\pgfsetdash{}{0pt}%
\pgfpathmoveto{\pgfqpoint{2.066174in}{3.010553in}}%
\pgfpathlineto{\pgfqpoint{2.077305in}{3.019590in}}%
\pgfpathlineto{\pgfqpoint{2.088437in}{3.028516in}}%
\pgfpathlineto{\pgfqpoint{2.099575in}{3.037088in}}%
\pgfpathlineto{\pgfqpoint{2.110724in}{3.045065in}}%
\pgfpathlineto{\pgfqpoint{2.121887in}{3.052355in}}%
\pgfpathlineto{\pgfqpoint{2.116109in}{3.061612in}}%
\pgfpathlineto{\pgfqpoint{2.110344in}{3.070470in}}%
\pgfpathlineto{\pgfqpoint{2.104593in}{3.078884in}}%
\pgfpathlineto{\pgfqpoint{2.098856in}{3.086810in}}%
\pgfpathlineto{\pgfqpoint{2.087707in}{3.079376in}}%
\pgfpathlineto{\pgfqpoint{2.076575in}{3.071139in}}%
\pgfpathlineto{\pgfqpoint{2.065458in}{3.062197in}}%
\pgfpathlineto{\pgfqpoint{2.054350in}{3.052824in}}%
\pgfpathlineto{\pgfqpoint{2.043243in}{3.043298in}}%
\pgfpathlineto{\pgfqpoint{2.048958in}{3.035602in}}%
\pgfpathlineto{\pgfqpoint{2.054685in}{3.027559in}}%
\pgfpathlineto{\pgfqpoint{2.060424in}{3.019199in}}%
\pgfpathclose%
\pgfusepath{stroke,fill}%
\end{pgfscope}%
\begin{pgfscope}%
\pgfpathrectangle{\pgfqpoint{0.887500in}{0.275000in}}{\pgfqpoint{4.225000in}{4.225000in}}%
\pgfusepath{clip}%
\pgfsetbuttcap%
\pgfsetroundjoin%
\definecolor{currentfill}{rgb}{0.175707,0.697900,0.491033}%
\pgfsetfillcolor{currentfill}%
\pgfsetfillopacity{0.700000}%
\pgfsetlinewidth{0.501875pt}%
\definecolor{currentstroke}{rgb}{1.000000,1.000000,1.000000}%
\pgfsetstrokecolor{currentstroke}%
\pgfsetstrokeopacity{0.500000}%
\pgfsetdash{}{0pt}%
\pgfpathmoveto{\pgfqpoint{2.347095in}{3.081018in}}%
\pgfpathlineto{\pgfqpoint{2.358215in}{3.089606in}}%
\pgfpathlineto{\pgfqpoint{2.369344in}{3.097565in}}%
\pgfpathlineto{\pgfqpoint{2.380484in}{3.104661in}}%
\pgfpathlineto{\pgfqpoint{2.391639in}{3.110661in}}%
\pgfpathlineto{\pgfqpoint{2.402811in}{3.115378in}}%
\pgfpathlineto{\pgfqpoint{2.397009in}{3.120985in}}%
\pgfpathlineto{\pgfqpoint{2.391211in}{3.126560in}}%
\pgfpathlineto{\pgfqpoint{2.385416in}{3.132214in}}%
\pgfpathlineto{\pgfqpoint{2.379622in}{3.138060in}}%
\pgfpathlineto{\pgfqpoint{2.373827in}{3.144208in}}%
\pgfpathlineto{\pgfqpoint{2.362675in}{3.139257in}}%
\pgfpathlineto{\pgfqpoint{2.351533in}{3.133405in}}%
\pgfpathlineto{\pgfqpoint{2.340402in}{3.126795in}}%
\pgfpathlineto{\pgfqpoint{2.329277in}{3.119602in}}%
\pgfpathlineto{\pgfqpoint{2.318157in}{3.111999in}}%
\pgfpathlineto{\pgfqpoint{2.323941in}{3.105525in}}%
\pgfpathlineto{\pgfqpoint{2.329726in}{3.099267in}}%
\pgfpathlineto{\pgfqpoint{2.335512in}{3.093149in}}%
\pgfpathlineto{\pgfqpoint{2.341301in}{3.087092in}}%
\pgfpathclose%
\pgfusepath{stroke,fill}%
\end{pgfscope}%
\begin{pgfscope}%
\pgfpathrectangle{\pgfqpoint{0.887500in}{0.275000in}}{\pgfqpoint{4.225000in}{4.225000in}}%
\pgfusepath{clip}%
\pgfsetbuttcap%
\pgfsetroundjoin%
\definecolor{currentfill}{rgb}{0.876168,0.891125,0.095250}%
\pgfsetfillcolor{currentfill}%
\pgfsetfillopacity{0.700000}%
\pgfsetlinewidth{0.501875pt}%
\definecolor{currentstroke}{rgb}{1.000000,1.000000,1.000000}%
\pgfsetstrokecolor{currentstroke}%
\pgfsetstrokeopacity{0.500000}%
\pgfsetdash{}{0pt}%
\pgfpathmoveto{\pgfqpoint{3.352176in}{3.694082in}}%
\pgfpathlineto{\pgfqpoint{3.363177in}{3.700141in}}%
\pgfpathlineto{\pgfqpoint{3.374173in}{3.706151in}}%
\pgfpathlineto{\pgfqpoint{3.385166in}{3.712118in}}%
\pgfpathlineto{\pgfqpoint{3.396155in}{3.718045in}}%
\pgfpathlineto{\pgfqpoint{3.407140in}{3.723939in}}%
\pgfpathlineto{\pgfqpoint{3.400933in}{3.727070in}}%
\pgfpathlineto{\pgfqpoint{3.394730in}{3.729925in}}%
\pgfpathlineto{\pgfqpoint{3.388530in}{3.732515in}}%
\pgfpathlineto{\pgfqpoint{3.382334in}{3.734855in}}%
\pgfpathlineto{\pgfqpoint{3.376141in}{3.736970in}}%
\pgfpathlineto{\pgfqpoint{3.365174in}{3.730889in}}%
\pgfpathlineto{\pgfqpoint{3.354204in}{3.724794in}}%
\pgfpathlineto{\pgfqpoint{3.343230in}{3.718676in}}%
\pgfpathlineto{\pgfqpoint{3.332252in}{3.712525in}}%
\pgfpathlineto{\pgfqpoint{3.321270in}{3.706332in}}%
\pgfpathlineto{\pgfqpoint{3.327443in}{3.704327in}}%
\pgfpathlineto{\pgfqpoint{3.333620in}{3.702119in}}%
\pgfpathlineto{\pgfqpoint{3.339802in}{3.699687in}}%
\pgfpathlineto{\pgfqpoint{3.345987in}{3.697012in}}%
\pgfpathclose%
\pgfusepath{stroke,fill}%
\end{pgfscope}%
\begin{pgfscope}%
\pgfpathrectangle{\pgfqpoint{0.887500in}{0.275000in}}{\pgfqpoint{4.225000in}{4.225000in}}%
\pgfusepath{clip}%
\pgfsetbuttcap%
\pgfsetroundjoin%
\definecolor{currentfill}{rgb}{0.772852,0.877868,0.131109}%
\pgfsetfillcolor{currentfill}%
\pgfsetfillopacity{0.700000}%
\pgfsetlinewidth{0.501875pt}%
\definecolor{currentstroke}{rgb}{1.000000,1.000000,1.000000}%
\pgfsetstrokecolor{currentstroke}%
\pgfsetstrokeopacity{0.500000}%
\pgfsetdash{}{0pt}%
\pgfpathmoveto{\pgfqpoint{3.156144in}{3.622803in}}%
\pgfpathlineto{\pgfqpoint{3.167178in}{3.628089in}}%
\pgfpathlineto{\pgfqpoint{3.178209in}{3.633208in}}%
\pgfpathlineto{\pgfqpoint{3.189234in}{3.638213in}}%
\pgfpathlineto{\pgfqpoint{3.200256in}{3.643168in}}%
\pgfpathlineto{\pgfqpoint{3.211273in}{3.648141in}}%
\pgfpathlineto{\pgfqpoint{3.205144in}{3.649585in}}%
\pgfpathlineto{\pgfqpoint{3.199021in}{3.650869in}}%
\pgfpathlineto{\pgfqpoint{3.192903in}{3.652023in}}%
\pgfpathlineto{\pgfqpoint{3.186791in}{3.653079in}}%
\pgfpathlineto{\pgfqpoint{3.180684in}{3.654071in}}%
\pgfpathlineto{\pgfqpoint{3.169686in}{3.648625in}}%
\pgfpathlineto{\pgfqpoint{3.158685in}{3.643230in}}%
\pgfpathlineto{\pgfqpoint{3.147679in}{3.637855in}}%
\pgfpathlineto{\pgfqpoint{3.136670in}{3.632472in}}%
\pgfpathlineto{\pgfqpoint{3.125657in}{3.627053in}}%
\pgfpathlineto{\pgfqpoint{3.131742in}{3.626377in}}%
\pgfpathlineto{\pgfqpoint{3.137834in}{3.625650in}}%
\pgfpathlineto{\pgfqpoint{3.143931in}{3.624835in}}%
\pgfpathlineto{\pgfqpoint{3.150035in}{3.623898in}}%
\pgfpathclose%
\pgfusepath{stroke,fill}%
\end{pgfscope}%
\begin{pgfscope}%
\pgfpathrectangle{\pgfqpoint{0.887500in}{0.275000in}}{\pgfqpoint{4.225000in}{4.225000in}}%
\pgfusepath{clip}%
\pgfsetbuttcap%
\pgfsetroundjoin%
\definecolor{currentfill}{rgb}{0.128087,0.647749,0.523491}%
\pgfsetfillcolor{currentfill}%
\pgfsetfillopacity{0.700000}%
\pgfsetlinewidth{0.501875pt}%
\definecolor{currentstroke}{rgb}{1.000000,1.000000,1.000000}%
\pgfsetstrokecolor{currentstroke}%
\pgfsetstrokeopacity{0.500000}%
\pgfsetdash{}{0pt}%
\pgfpathmoveto{\pgfqpoint{2.010315in}{2.971915in}}%
\pgfpathlineto{\pgfqpoint{2.021531in}{2.978150in}}%
\pgfpathlineto{\pgfqpoint{2.032722in}{2.985217in}}%
\pgfpathlineto{\pgfqpoint{2.043889in}{2.993123in}}%
\pgfpathlineto{\pgfqpoint{2.055037in}{3.001650in}}%
\pgfpathlineto{\pgfqpoint{2.066174in}{3.010553in}}%
\pgfpathlineto{\pgfqpoint{2.060424in}{3.019199in}}%
\pgfpathlineto{\pgfqpoint{2.054685in}{3.027559in}}%
\pgfpathlineto{\pgfqpoint{2.048958in}{3.035602in}}%
\pgfpathlineto{\pgfqpoint{2.043243in}{3.043298in}}%
\pgfpathlineto{\pgfqpoint{2.032131in}{3.033896in}}%
\pgfpathlineto{\pgfqpoint{2.021008in}{3.024894in}}%
\pgfpathlineto{\pgfqpoint{2.009864in}{3.016569in}}%
\pgfpathlineto{\pgfqpoint{1.998694in}{3.009169in}}%
\pgfpathlineto{\pgfqpoint{1.987496in}{3.002688in}}%
\pgfpathlineto{\pgfqpoint{1.993193in}{2.995073in}}%
\pgfpathlineto{\pgfqpoint{1.998895in}{2.987404in}}%
\pgfpathlineto{\pgfqpoint{2.004603in}{2.979683in}}%
\pgfpathclose%
\pgfusepath{stroke,fill}%
\end{pgfscope}%
\begin{pgfscope}%
\pgfpathrectangle{\pgfqpoint{0.887500in}{0.275000in}}{\pgfqpoint{4.225000in}{4.225000in}}%
\pgfusepath{clip}%
\pgfsetbuttcap%
\pgfsetroundjoin%
\definecolor{currentfill}{rgb}{0.974417,0.903590,0.130215}%
\pgfsetfillcolor{currentfill}%
\pgfsetfillopacity{0.700000}%
\pgfsetlinewidth{0.501875pt}%
\definecolor{currentstroke}{rgb}{1.000000,1.000000,1.000000}%
\pgfsetstrokecolor{currentstroke}%
\pgfsetstrokeopacity{0.500000}%
\pgfsetdash{}{0pt}%
\pgfpathmoveto{\pgfqpoint{3.774682in}{3.765418in}}%
\pgfpathlineto{\pgfqpoint{3.785588in}{3.769990in}}%
\pgfpathlineto{\pgfqpoint{3.796489in}{3.774529in}}%
\pgfpathlineto{\pgfqpoint{3.807385in}{3.779033in}}%
\pgfpathlineto{\pgfqpoint{3.818276in}{3.783505in}}%
\pgfpathlineto{\pgfqpoint{3.811964in}{3.790649in}}%
\pgfpathlineto{\pgfqpoint{3.805653in}{3.797624in}}%
\pgfpathlineto{\pgfqpoint{3.799344in}{3.804448in}}%
\pgfpathlineto{\pgfqpoint{3.793036in}{3.811136in}}%
\pgfpathlineto{\pgfqpoint{3.786731in}{3.817703in}}%
\pgfpathlineto{\pgfqpoint{3.775854in}{3.813301in}}%
\pgfpathlineto{\pgfqpoint{3.764971in}{3.808864in}}%
\pgfpathlineto{\pgfqpoint{3.754083in}{3.804394in}}%
\pgfpathlineto{\pgfqpoint{3.743191in}{3.799888in}}%
\pgfpathlineto{\pgfqpoint{3.749486in}{3.793325in}}%
\pgfpathlineto{\pgfqpoint{3.755783in}{3.786605in}}%
\pgfpathlineto{\pgfqpoint{3.762081in}{3.779720in}}%
\pgfpathlineto{\pgfqpoint{3.768381in}{3.772660in}}%
\pgfpathclose%
\pgfusepath{stroke,fill}%
\end{pgfscope}%
\begin{pgfscope}%
\pgfpathrectangle{\pgfqpoint{0.887500in}{0.275000in}}{\pgfqpoint{4.225000in}{4.225000in}}%
\pgfusepath{clip}%
\pgfsetbuttcap%
\pgfsetroundjoin%
\definecolor{currentfill}{rgb}{0.140210,0.665859,0.513427}%
\pgfsetfillcolor{currentfill}%
\pgfsetfillopacity{0.700000}%
\pgfsetlinewidth{0.501875pt}%
\definecolor{currentstroke}{rgb}{1.000000,1.000000,1.000000}%
\pgfsetstrokecolor{currentstroke}%
\pgfsetstrokeopacity{0.500000}%
\pgfsetdash{}{0pt}%
\pgfpathmoveto{\pgfqpoint{2.150927in}{3.001621in}}%
\pgfpathlineto{\pgfqpoint{2.162097in}{3.008986in}}%
\pgfpathlineto{\pgfqpoint{2.173265in}{3.016274in}}%
\pgfpathlineto{\pgfqpoint{2.184432in}{3.023496in}}%
\pgfpathlineto{\pgfqpoint{2.195597in}{3.030662in}}%
\pgfpathlineto{\pgfqpoint{2.206759in}{3.037782in}}%
\pgfpathlineto{\pgfqpoint{2.200956in}{3.046883in}}%
\pgfpathlineto{\pgfqpoint{2.195158in}{3.055944in}}%
\pgfpathlineto{\pgfqpoint{2.189363in}{3.064951in}}%
\pgfpathlineto{\pgfqpoint{2.183574in}{3.073891in}}%
\pgfpathlineto{\pgfqpoint{2.177789in}{3.082751in}}%
\pgfpathlineto{\pgfqpoint{2.166608in}{3.077053in}}%
\pgfpathlineto{\pgfqpoint{2.155424in}{3.071292in}}%
\pgfpathlineto{\pgfqpoint{2.144241in}{3.065341in}}%
\pgfpathlineto{\pgfqpoint{2.133061in}{3.059071in}}%
\pgfpathlineto{\pgfqpoint{2.121887in}{3.052355in}}%
\pgfpathlineto{\pgfqpoint{2.127677in}{3.042742in}}%
\pgfpathlineto{\pgfqpoint{2.133477in}{3.032817in}}%
\pgfpathlineto{\pgfqpoint{2.139286in}{3.022626in}}%
\pgfpathlineto{\pgfqpoint{2.145103in}{3.012213in}}%
\pgfpathclose%
\pgfusepath{stroke,fill}%
\end{pgfscope}%
\begin{pgfscope}%
\pgfpathrectangle{\pgfqpoint{0.887500in}{0.275000in}}{\pgfqpoint{4.225000in}{4.225000in}}%
\pgfusepath{clip}%
\pgfsetbuttcap%
\pgfsetroundjoin%
\definecolor{currentfill}{rgb}{0.935904,0.898570,0.108131}%
\pgfsetfillcolor{currentfill}%
\pgfsetfillopacity{0.700000}%
\pgfsetlinewidth{0.501875pt}%
\definecolor{currentstroke}{rgb}{1.000000,1.000000,1.000000}%
\pgfsetstrokecolor{currentstroke}%
\pgfsetstrokeopacity{0.500000}%
\pgfsetdash{}{0pt}%
\pgfpathmoveto{\pgfqpoint{3.493146in}{3.731210in}}%
\pgfpathlineto{\pgfqpoint{3.504120in}{3.736537in}}%
\pgfpathlineto{\pgfqpoint{3.515088in}{3.741813in}}%
\pgfpathlineto{\pgfqpoint{3.526053in}{3.747039in}}%
\pgfpathlineto{\pgfqpoint{3.537012in}{3.752216in}}%
\pgfpathlineto{\pgfqpoint{3.547968in}{3.757345in}}%
\pgfpathlineto{\pgfqpoint{3.541722in}{3.762505in}}%
\pgfpathlineto{\pgfqpoint{3.535478in}{3.767389in}}%
\pgfpathlineto{\pgfqpoint{3.529236in}{3.771990in}}%
\pgfpathlineto{\pgfqpoint{3.522996in}{3.776303in}}%
\pgfpathlineto{\pgfqpoint{3.516757in}{3.780320in}}%
\pgfpathlineto{\pgfqpoint{3.505817in}{3.775046in}}%
\pgfpathlineto{\pgfqpoint{3.494871in}{3.769669in}}%
\pgfpathlineto{\pgfqpoint{3.483920in}{3.764186in}}%
\pgfpathlineto{\pgfqpoint{3.472964in}{3.758611in}}%
\pgfpathlineto{\pgfqpoint{3.462004in}{3.752956in}}%
\pgfpathlineto{\pgfqpoint{3.468228in}{3.749211in}}%
\pgfpathlineto{\pgfqpoint{3.474455in}{3.745165in}}%
\pgfpathlineto{\pgfqpoint{3.480683in}{3.740817in}}%
\pgfpathlineto{\pgfqpoint{3.486914in}{3.736166in}}%
\pgfpathclose%
\pgfusepath{stroke,fill}%
\end{pgfscope}%
\begin{pgfscope}%
\pgfpathrectangle{\pgfqpoint{0.887500in}{0.275000in}}{\pgfqpoint{4.225000in}{4.225000in}}%
\pgfusepath{clip}%
\pgfsetbuttcap%
\pgfsetroundjoin%
\definecolor{currentfill}{rgb}{0.668054,0.861999,0.196293}%
\pgfsetfillcolor{currentfill}%
\pgfsetfillopacity{0.700000}%
\pgfsetlinewidth{0.501875pt}%
\definecolor{currentstroke}{rgb}{1.000000,1.000000,1.000000}%
\pgfsetstrokecolor{currentstroke}%
\pgfsetstrokeopacity{0.500000}%
\pgfsetdash{}{0pt}%
\pgfpathmoveto{\pgfqpoint{2.959974in}{3.547667in}}%
\pgfpathlineto{\pgfqpoint{2.971051in}{3.551642in}}%
\pgfpathlineto{\pgfqpoint{2.982122in}{3.555778in}}%
\pgfpathlineto{\pgfqpoint{2.993189in}{3.560307in}}%
\pgfpathlineto{\pgfqpoint{3.004250in}{3.565253in}}%
\pgfpathlineto{\pgfqpoint{3.015307in}{3.570535in}}%
\pgfpathlineto{\pgfqpoint{3.009271in}{3.571400in}}%
\pgfpathlineto{\pgfqpoint{3.003242in}{3.572308in}}%
\pgfpathlineto{\pgfqpoint{2.997218in}{3.573258in}}%
\pgfpathlineto{\pgfqpoint{2.991200in}{3.574248in}}%
\pgfpathlineto{\pgfqpoint{2.985188in}{3.575278in}}%
\pgfpathlineto{\pgfqpoint{2.974154in}{3.569846in}}%
\pgfpathlineto{\pgfqpoint{2.963114in}{3.564723in}}%
\pgfpathlineto{\pgfqpoint{2.952070in}{3.560001in}}%
\pgfpathlineto{\pgfqpoint{2.941020in}{3.555632in}}%
\pgfpathlineto{\pgfqpoint{2.929965in}{3.551297in}}%
\pgfpathlineto{\pgfqpoint{2.935956in}{3.550417in}}%
\pgfpathlineto{\pgfqpoint{2.941952in}{3.549617in}}%
\pgfpathlineto{\pgfqpoint{2.947953in}{3.548901in}}%
\pgfpathlineto{\pgfqpoint{2.953961in}{3.548256in}}%
\pgfpathclose%
\pgfusepath{stroke,fill}%
\end{pgfscope}%
\begin{pgfscope}%
\pgfpathrectangle{\pgfqpoint{0.887500in}{0.275000in}}{\pgfqpoint{4.225000in}{4.225000in}}%
\pgfusepath{clip}%
\pgfsetbuttcap%
\pgfsetroundjoin%
\definecolor{currentfill}{rgb}{0.964894,0.902323,0.123941}%
\pgfsetfillcolor{currentfill}%
\pgfsetfillopacity{0.700000}%
\pgfsetlinewidth{0.501875pt}%
\definecolor{currentstroke}{rgb}{1.000000,1.000000,1.000000}%
\pgfsetstrokecolor{currentstroke}%
\pgfsetstrokeopacity{0.500000}%
\pgfsetdash{}{0pt}%
\pgfpathmoveto{\pgfqpoint{3.633995in}{3.752716in}}%
\pgfpathlineto{\pgfqpoint{3.644937in}{3.757619in}}%
\pgfpathlineto{\pgfqpoint{3.655873in}{3.762478in}}%
\pgfpathlineto{\pgfqpoint{3.666805in}{3.767294in}}%
\pgfpathlineto{\pgfqpoint{3.677732in}{3.772069in}}%
\pgfpathlineto{\pgfqpoint{3.688654in}{3.776803in}}%
\pgfpathlineto{\pgfqpoint{3.682374in}{3.783190in}}%
\pgfpathlineto{\pgfqpoint{3.676095in}{3.789385in}}%
\pgfpathlineto{\pgfqpoint{3.669818in}{3.795385in}}%
\pgfpathlineto{\pgfqpoint{3.663542in}{3.801173in}}%
\pgfpathlineto{\pgfqpoint{3.657268in}{3.806727in}}%
\pgfpathlineto{\pgfqpoint{3.646358in}{3.801913in}}%
\pgfpathlineto{\pgfqpoint{3.635444in}{3.797074in}}%
\pgfpathlineto{\pgfqpoint{3.624525in}{3.792209in}}%
\pgfpathlineto{\pgfqpoint{3.613602in}{3.787319in}}%
\pgfpathlineto{\pgfqpoint{3.602674in}{3.782402in}}%
\pgfpathlineto{\pgfqpoint{3.608936in}{3.776943in}}%
\pgfpathlineto{\pgfqpoint{3.615198in}{3.771236in}}%
\pgfpathlineto{\pgfqpoint{3.621463in}{3.765291in}}%
\pgfpathlineto{\pgfqpoint{3.627728in}{3.759115in}}%
\pgfpathclose%
\pgfusepath{stroke,fill}%
\end{pgfscope}%
\begin{pgfscope}%
\pgfpathrectangle{\pgfqpoint{0.887500in}{0.275000in}}{\pgfqpoint{4.225000in}{4.225000in}}%
\pgfusepath{clip}%
\pgfsetbuttcap%
\pgfsetroundjoin%
\definecolor{currentfill}{rgb}{0.157851,0.683765,0.501686}%
\pgfsetfillcolor{currentfill}%
\pgfsetfillopacity{0.700000}%
\pgfsetlinewidth{0.501875pt}%
\definecolor{currentstroke}{rgb}{1.000000,1.000000,1.000000}%
\pgfsetstrokecolor{currentstroke}%
\pgfsetstrokeopacity{0.500000}%
\pgfsetdash{}{0pt}%
\pgfpathmoveto{\pgfqpoint{2.291496in}{3.036170in}}%
\pgfpathlineto{\pgfqpoint{2.302624in}{3.044923in}}%
\pgfpathlineto{\pgfqpoint{2.313748in}{3.053812in}}%
\pgfpathlineto{\pgfqpoint{2.324865in}{3.062889in}}%
\pgfpathlineto{\pgfqpoint{2.335980in}{3.072035in}}%
\pgfpathlineto{\pgfqpoint{2.347095in}{3.081018in}}%
\pgfpathlineto{\pgfqpoint{2.341301in}{3.087092in}}%
\pgfpathlineto{\pgfqpoint{2.335512in}{3.093149in}}%
\pgfpathlineto{\pgfqpoint{2.329726in}{3.099267in}}%
\pgfpathlineto{\pgfqpoint{2.323941in}{3.105525in}}%
\pgfpathlineto{\pgfqpoint{2.318157in}{3.111999in}}%
\pgfpathlineto{\pgfqpoint{2.307039in}{3.104161in}}%
\pgfpathlineto{\pgfqpoint{2.295918in}{3.096261in}}%
\pgfpathlineto{\pgfqpoint{2.284793in}{3.088473in}}%
\pgfpathlineto{\pgfqpoint{2.273662in}{3.080889in}}%
\pgfpathlineto{\pgfqpoint{2.262523in}{3.073481in}}%
\pgfpathlineto{\pgfqpoint{2.268313in}{3.065906in}}%
\pgfpathlineto{\pgfqpoint{2.274104in}{3.058410in}}%
\pgfpathlineto{\pgfqpoint{2.279898in}{3.050971in}}%
\pgfpathlineto{\pgfqpoint{2.285695in}{3.043565in}}%
\pgfpathclose%
\pgfusepath{stroke,fill}%
\end{pgfscope}%
\begin{pgfscope}%
\pgfpathrectangle{\pgfqpoint{0.887500in}{0.275000in}}{\pgfqpoint{4.225000in}{4.225000in}}%
\pgfusepath{clip}%
\pgfsetbuttcap%
\pgfsetroundjoin%
\definecolor{currentfill}{rgb}{0.175707,0.697900,0.491033}%
\pgfsetfillcolor{currentfill}%
\pgfsetfillopacity{0.700000}%
\pgfsetlinewidth{0.501875pt}%
\definecolor{currentstroke}{rgb}{1.000000,1.000000,1.000000}%
\pgfsetstrokecolor{currentstroke}%
\pgfsetstrokeopacity{0.500000}%
\pgfsetdash{}{0pt}%
\pgfpathmoveto{\pgfqpoint{2.431960in}{3.082942in}}%
\pgfpathlineto{\pgfqpoint{2.443165in}{3.086143in}}%
\pgfpathlineto{\pgfqpoint{2.454379in}{3.088467in}}%
\pgfpathlineto{\pgfqpoint{2.465593in}{3.090326in}}%
\pgfpathlineto{\pgfqpoint{2.476802in}{3.092132in}}%
\pgfpathlineto{\pgfqpoint{2.488000in}{3.094297in}}%
\pgfpathlineto{\pgfqpoint{2.482137in}{3.101660in}}%
\pgfpathlineto{\pgfqpoint{2.476282in}{3.108675in}}%
\pgfpathlineto{\pgfqpoint{2.470436in}{3.115416in}}%
\pgfpathlineto{\pgfqpoint{2.464597in}{3.121955in}}%
\pgfpathlineto{\pgfqpoint{2.458763in}{3.128367in}}%
\pgfpathlineto{\pgfqpoint{2.447579in}{3.126130in}}%
\pgfpathlineto{\pgfqpoint{2.436386in}{3.124028in}}%
\pgfpathlineto{\pgfqpoint{2.425191in}{3.121746in}}%
\pgfpathlineto{\pgfqpoint{2.413997in}{3.118968in}}%
\pgfpathlineto{\pgfqpoint{2.402811in}{3.115378in}}%
\pgfpathlineto{\pgfqpoint{2.408621in}{3.109627in}}%
\pgfpathlineto{\pgfqpoint{2.414439in}{3.103619in}}%
\pgfpathlineto{\pgfqpoint{2.420267in}{3.097244in}}%
\pgfpathlineto{\pgfqpoint{2.426107in}{3.090389in}}%
\pgfpathclose%
\pgfusepath{stroke,fill}%
\end{pgfscope}%
\begin{pgfscope}%
\pgfpathrectangle{\pgfqpoint{0.887500in}{0.275000in}}{\pgfqpoint{4.225000in}{4.225000in}}%
\pgfusepath{clip}%
\pgfsetbuttcap%
\pgfsetroundjoin%
\definecolor{currentfill}{rgb}{0.845561,0.887322,0.099702}%
\pgfsetfillcolor{currentfill}%
\pgfsetfillopacity{0.700000}%
\pgfsetlinewidth{0.501875pt}%
\definecolor{currentstroke}{rgb}{1.000000,1.000000,1.000000}%
\pgfsetstrokecolor{currentstroke}%
\pgfsetstrokeopacity{0.500000}%
\pgfsetdash{}{0pt}%
\pgfpathmoveto{\pgfqpoint{3.297119in}{3.663804in}}%
\pgfpathlineto{\pgfqpoint{3.308137in}{3.669707in}}%
\pgfpathlineto{\pgfqpoint{3.319152in}{3.675736in}}%
\pgfpathlineto{\pgfqpoint{3.330163in}{3.681840in}}%
\pgfpathlineto{\pgfqpoint{3.341172in}{3.687972in}}%
\pgfpathlineto{\pgfqpoint{3.352176in}{3.694082in}}%
\pgfpathlineto{\pgfqpoint{3.345987in}{3.697012in}}%
\pgfpathlineto{\pgfqpoint{3.339802in}{3.699687in}}%
\pgfpathlineto{\pgfqpoint{3.333620in}{3.702119in}}%
\pgfpathlineto{\pgfqpoint{3.327443in}{3.704327in}}%
\pgfpathlineto{\pgfqpoint{3.321270in}{3.706332in}}%
\pgfpathlineto{\pgfqpoint{3.310284in}{3.700089in}}%
\pgfpathlineto{\pgfqpoint{3.299295in}{3.693825in}}%
\pgfpathlineto{\pgfqpoint{3.288303in}{3.687585in}}%
\pgfpathlineto{\pgfqpoint{3.277307in}{3.681417in}}%
\pgfpathlineto{\pgfqpoint{3.266308in}{3.675368in}}%
\pgfpathlineto{\pgfqpoint{3.272461in}{3.673457in}}%
\pgfpathlineto{\pgfqpoint{3.278619in}{3.671364in}}%
\pgfpathlineto{\pgfqpoint{3.284781in}{3.669069in}}%
\pgfpathlineto{\pgfqpoint{3.290948in}{3.666553in}}%
\pgfpathclose%
\pgfusepath{stroke,fill}%
\end{pgfscope}%
\begin{pgfscope}%
\pgfpathrectangle{\pgfqpoint{0.887500in}{0.275000in}}{\pgfqpoint{4.225000in}{4.225000in}}%
\pgfusepath{clip}%
\pgfsetbuttcap%
\pgfsetroundjoin%
\definecolor{currentfill}{rgb}{0.196571,0.711827,0.479221}%
\pgfsetfillcolor{currentfill}%
\pgfsetfillopacity{0.700000}%
\pgfsetlinewidth{0.501875pt}%
\definecolor{currentstroke}{rgb}{1.000000,1.000000,1.000000}%
\pgfsetstrokecolor{currentstroke}%
\pgfsetstrokeopacity{0.500000}%
\pgfsetdash{}{0pt}%
\pgfpathmoveto{\pgfqpoint{2.573059in}{3.090995in}}%
\pgfpathlineto{\pgfqpoint{2.584116in}{3.103619in}}%
\pgfpathlineto{\pgfqpoint{2.595156in}{3.117750in}}%
\pgfpathlineto{\pgfqpoint{2.606179in}{3.133380in}}%
\pgfpathlineto{\pgfqpoint{2.617189in}{3.150430in}}%
\pgfpathlineto{\pgfqpoint{2.628189in}{3.168580in}}%
\pgfpathlineto{\pgfqpoint{2.622352in}{3.168387in}}%
\pgfpathlineto{\pgfqpoint{2.616524in}{3.168002in}}%
\pgfpathlineto{\pgfqpoint{2.610701in}{3.167664in}}%
\pgfpathlineto{\pgfqpoint{2.604881in}{3.167585in}}%
\pgfpathlineto{\pgfqpoint{2.599064in}{3.167841in}}%
\pgfpathlineto{\pgfqpoint{2.588012in}{3.156913in}}%
\pgfpathlineto{\pgfqpoint{2.576953in}{3.146595in}}%
\pgfpathlineto{\pgfqpoint{2.565884in}{3.136940in}}%
\pgfpathlineto{\pgfqpoint{2.554804in}{3.128020in}}%
\pgfpathlineto{\pgfqpoint{2.543712in}{3.119914in}}%
\pgfpathlineto{\pgfqpoint{2.549570in}{3.114318in}}%
\pgfpathlineto{\pgfqpoint{2.555434in}{3.108646in}}%
\pgfpathlineto{\pgfqpoint{2.561303in}{3.102869in}}%
\pgfpathlineto{\pgfqpoint{2.567178in}{3.096985in}}%
\pgfpathclose%
\pgfusepath{stroke,fill}%
\end{pgfscope}%
\begin{pgfscope}%
\pgfpathrectangle{\pgfqpoint{0.887500in}{0.275000in}}{\pgfqpoint{4.225000in}{4.225000in}}%
\pgfusepath{clip}%
\pgfsetbuttcap%
\pgfsetroundjoin%
\definecolor{currentfill}{rgb}{0.132268,0.655014,0.519661}%
\pgfsetfillcolor{currentfill}%
\pgfsetfillopacity{0.700000}%
\pgfsetlinewidth{0.501875pt}%
\definecolor{currentstroke}{rgb}{1.000000,1.000000,1.000000}%
\pgfsetstrokecolor{currentstroke}%
\pgfsetstrokeopacity{0.500000}%
\pgfsetdash{}{0pt}%
\pgfpathmoveto{\pgfqpoint{2.095049in}{2.964109in}}%
\pgfpathlineto{\pgfqpoint{2.106233in}{2.971499in}}%
\pgfpathlineto{\pgfqpoint{2.117411in}{2.979035in}}%
\pgfpathlineto{\pgfqpoint{2.128584in}{2.986623in}}%
\pgfpathlineto{\pgfqpoint{2.139756in}{2.994168in}}%
\pgfpathlineto{\pgfqpoint{2.150927in}{3.001621in}}%
\pgfpathlineto{\pgfqpoint{2.145103in}{3.012213in}}%
\pgfpathlineto{\pgfqpoint{2.139286in}{3.022626in}}%
\pgfpathlineto{\pgfqpoint{2.133477in}{3.032817in}}%
\pgfpathlineto{\pgfqpoint{2.127677in}{3.042742in}}%
\pgfpathlineto{\pgfqpoint{2.121887in}{3.052355in}}%
\pgfpathlineto{\pgfqpoint{2.110724in}{3.045065in}}%
\pgfpathlineto{\pgfqpoint{2.099575in}{3.037088in}}%
\pgfpathlineto{\pgfqpoint{2.088437in}{3.028516in}}%
\pgfpathlineto{\pgfqpoint{2.077305in}{3.019590in}}%
\pgfpathlineto{\pgfqpoint{2.066174in}{3.010553in}}%
\pgfpathlineto{\pgfqpoint{2.071933in}{3.001652in}}%
\pgfpathlineto{\pgfqpoint{2.077700in}{2.992526in}}%
\pgfpathlineto{\pgfqpoint{2.083476in}{2.983206in}}%
\pgfpathlineto{\pgfqpoint{2.089259in}{2.973724in}}%
\pgfpathclose%
\pgfusepath{stroke,fill}%
\end{pgfscope}%
\begin{pgfscope}%
\pgfpathrectangle{\pgfqpoint{0.887500in}{0.275000in}}{\pgfqpoint{4.225000in}{4.225000in}}%
\pgfusepath{clip}%
\pgfsetbuttcap%
\pgfsetroundjoin%
\definecolor{currentfill}{rgb}{0.123444,0.636809,0.528763}%
\pgfsetfillcolor{currentfill}%
\pgfsetfillopacity{0.700000}%
\pgfsetlinewidth{0.501875pt}%
\definecolor{currentstroke}{rgb}{1.000000,1.000000,1.000000}%
\pgfsetstrokecolor{currentstroke}%
\pgfsetstrokeopacity{0.500000}%
\pgfsetdash{}{0pt}%
\pgfpathmoveto{\pgfqpoint{1.953948in}{2.949139in}}%
\pgfpathlineto{\pgfqpoint{1.965251in}{2.952966in}}%
\pgfpathlineto{\pgfqpoint{1.976541in}{2.957039in}}%
\pgfpathlineto{\pgfqpoint{1.987818in}{2.961477in}}%
\pgfpathlineto{\pgfqpoint{1.999076in}{2.966397in}}%
\pgfpathlineto{\pgfqpoint{2.010315in}{2.971915in}}%
\pgfpathlineto{\pgfqpoint{2.004603in}{2.979683in}}%
\pgfpathlineto{\pgfqpoint{1.998895in}{2.987404in}}%
\pgfpathlineto{\pgfqpoint{1.993193in}{2.995073in}}%
\pgfpathlineto{\pgfqpoint{1.987496in}{3.002688in}}%
\pgfpathlineto{\pgfqpoint{1.976273in}{2.996996in}}%
\pgfpathlineto{\pgfqpoint{1.965028in}{2.991961in}}%
\pgfpathlineto{\pgfqpoint{1.953765in}{2.987451in}}%
\pgfpathlineto{\pgfqpoint{1.942485in}{2.983334in}}%
\pgfpathlineto{\pgfqpoint{1.931194in}{2.979478in}}%
\pgfpathlineto{\pgfqpoint{1.936877in}{2.971909in}}%
\pgfpathlineto{\pgfqpoint{1.942563in}{2.964329in}}%
\pgfpathlineto{\pgfqpoint{1.948254in}{2.956739in}}%
\pgfpathclose%
\pgfusepath{stroke,fill}%
\end{pgfscope}%
\begin{pgfscope}%
\pgfpathrectangle{\pgfqpoint{0.887500in}{0.275000in}}{\pgfqpoint{4.225000in}{4.225000in}}%
\pgfusepath{clip}%
\pgfsetbuttcap%
\pgfsetroundjoin%
\definecolor{currentfill}{rgb}{0.751884,0.874951,0.143228}%
\pgfsetfillcolor{currentfill}%
\pgfsetfillopacity{0.700000}%
\pgfsetlinewidth{0.501875pt}%
\definecolor{currentstroke}{rgb}{1.000000,1.000000,1.000000}%
\pgfsetstrokecolor{currentstroke}%
\pgfsetstrokeopacity{0.500000}%
\pgfsetdash{}{0pt}%
\pgfpathmoveto{\pgfqpoint{3.100907in}{3.594708in}}%
\pgfpathlineto{\pgfqpoint{3.111962in}{3.600444in}}%
\pgfpathlineto{\pgfqpoint{3.123014in}{3.606153in}}%
\pgfpathlineto{\pgfqpoint{3.134061in}{3.611803in}}%
\pgfpathlineto{\pgfqpoint{3.145104in}{3.617364in}}%
\pgfpathlineto{\pgfqpoint{3.156144in}{3.622803in}}%
\pgfpathlineto{\pgfqpoint{3.150035in}{3.623898in}}%
\pgfpathlineto{\pgfqpoint{3.143931in}{3.624835in}}%
\pgfpathlineto{\pgfqpoint{3.137834in}{3.625650in}}%
\pgfpathlineto{\pgfqpoint{3.131742in}{3.626377in}}%
\pgfpathlineto{\pgfqpoint{3.125657in}{3.627053in}}%
\pgfpathlineto{\pgfqpoint{3.114639in}{3.621590in}}%
\pgfpathlineto{\pgfqpoint{3.103618in}{3.616077in}}%
\pgfpathlineto{\pgfqpoint{3.092592in}{3.610508in}}%
\pgfpathlineto{\pgfqpoint{3.081563in}{3.604879in}}%
\pgfpathlineto{\pgfqpoint{3.070530in}{3.599185in}}%
\pgfpathlineto{\pgfqpoint{3.076593in}{3.598323in}}%
\pgfpathlineto{\pgfqpoint{3.082663in}{3.597457in}}%
\pgfpathlineto{\pgfqpoint{3.088738in}{3.596574in}}%
\pgfpathlineto{\pgfqpoint{3.094820in}{3.595662in}}%
\pgfpathclose%
\pgfusepath{stroke,fill}%
\end{pgfscope}%
\begin{pgfscope}%
\pgfpathrectangle{\pgfqpoint{0.887500in}{0.275000in}}{\pgfqpoint{4.225000in}{4.225000in}}%
\pgfusepath{clip}%
\pgfsetbuttcap%
\pgfsetroundjoin%
\definecolor{currentfill}{rgb}{0.259857,0.745492,0.444467}%
\pgfsetfillcolor{currentfill}%
\pgfsetfillopacity{0.700000}%
\pgfsetlinewidth{0.501875pt}%
\definecolor{currentstroke}{rgb}{1.000000,1.000000,1.000000}%
\pgfsetstrokecolor{currentstroke}%
\pgfsetstrokeopacity{0.500000}%
\pgfsetdash{}{0pt}%
\pgfpathmoveto{\pgfqpoint{2.628189in}{3.168580in}}%
\pgfpathlineto{\pgfqpoint{2.639185in}{3.187465in}}%
\pgfpathlineto{\pgfqpoint{2.650180in}{3.206718in}}%
\pgfpathlineto{\pgfqpoint{2.661180in}{3.225970in}}%
\pgfpathlineto{\pgfqpoint{2.672186in}{3.244853in}}%
\pgfpathlineto{\pgfqpoint{2.683202in}{3.262998in}}%
\pgfpathlineto{\pgfqpoint{2.677396in}{3.255895in}}%
\pgfpathlineto{\pgfqpoint{2.671600in}{3.248630in}}%
\pgfpathlineto{\pgfqpoint{2.665809in}{3.241677in}}%
\pgfpathlineto{\pgfqpoint{2.660020in}{3.235447in}}%
\pgfpathlineto{\pgfqpoint{2.654229in}{3.230049in}}%
\pgfpathlineto{\pgfqpoint{2.643205in}{3.216752in}}%
\pgfpathlineto{\pgfqpoint{2.632177in}{3.203837in}}%
\pgfpathlineto{\pgfqpoint{2.621145in}{3.191350in}}%
\pgfpathlineto{\pgfqpoint{2.610108in}{3.179336in}}%
\pgfpathlineto{\pgfqpoint{2.599064in}{3.167841in}}%
\pgfpathlineto{\pgfqpoint{2.604881in}{3.167585in}}%
\pgfpathlineto{\pgfqpoint{2.610701in}{3.167664in}}%
\pgfpathlineto{\pgfqpoint{2.616524in}{3.168002in}}%
\pgfpathlineto{\pgfqpoint{2.622352in}{3.168387in}}%
\pgfpathclose%
\pgfusepath{stroke,fill}%
\end{pgfscope}%
\begin{pgfscope}%
\pgfpathrectangle{\pgfqpoint{0.887500in}{0.275000in}}{\pgfqpoint{4.225000in}{4.225000in}}%
\pgfusepath{clip}%
\pgfsetbuttcap%
\pgfsetroundjoin%
\definecolor{currentfill}{rgb}{0.140210,0.665859,0.513427}%
\pgfsetfillcolor{currentfill}%
\pgfsetfillopacity{0.700000}%
\pgfsetlinewidth{0.501875pt}%
\definecolor{currentstroke}{rgb}{1.000000,1.000000,1.000000}%
\pgfsetstrokecolor{currentstroke}%
\pgfsetstrokeopacity{0.500000}%
\pgfsetdash{}{0pt}%
\pgfpathmoveto{\pgfqpoint{2.235827in}{2.992168in}}%
\pgfpathlineto{\pgfqpoint{2.246960in}{3.001215in}}%
\pgfpathlineto{\pgfqpoint{2.258095in}{3.010075in}}%
\pgfpathlineto{\pgfqpoint{2.269230in}{3.018811in}}%
\pgfpathlineto{\pgfqpoint{2.280365in}{3.027488in}}%
\pgfpathlineto{\pgfqpoint{2.291496in}{3.036170in}}%
\pgfpathlineto{\pgfqpoint{2.285695in}{3.043565in}}%
\pgfpathlineto{\pgfqpoint{2.279898in}{3.050971in}}%
\pgfpathlineto{\pgfqpoint{2.274104in}{3.058410in}}%
\pgfpathlineto{\pgfqpoint{2.268313in}{3.065906in}}%
\pgfpathlineto{\pgfqpoint{2.262523in}{3.073481in}}%
\pgfpathlineto{\pgfqpoint{2.251380in}{3.066213in}}%
\pgfpathlineto{\pgfqpoint{2.240230in}{3.059046in}}%
\pgfpathlineto{\pgfqpoint{2.229077in}{3.051944in}}%
\pgfpathlineto{\pgfqpoint{2.217920in}{3.044868in}}%
\pgfpathlineto{\pgfqpoint{2.206759in}{3.037782in}}%
\pgfpathlineto{\pgfqpoint{2.212566in}{3.028656in}}%
\pgfpathlineto{\pgfqpoint{2.218377in}{3.019518in}}%
\pgfpathlineto{\pgfqpoint{2.224191in}{3.010381in}}%
\pgfpathlineto{\pgfqpoint{2.230007in}{3.001260in}}%
\pgfpathclose%
\pgfusepath{stroke,fill}%
\end{pgfscope}%
\begin{pgfscope}%
\pgfpathrectangle{\pgfqpoint{0.887500in}{0.275000in}}{\pgfqpoint{4.225000in}{4.225000in}}%
\pgfusepath{clip}%
\pgfsetbuttcap%
\pgfsetroundjoin%
\definecolor{currentfill}{rgb}{0.964894,0.902323,0.123941}%
\pgfsetfillcolor{currentfill}%
\pgfsetfillopacity{0.700000}%
\pgfsetlinewidth{0.501875pt}%
\definecolor{currentstroke}{rgb}{1.000000,1.000000,1.000000}%
\pgfsetstrokecolor{currentstroke}%
\pgfsetstrokeopacity{0.500000}%
\pgfsetdash{}{0pt}%
\pgfpathmoveto{\pgfqpoint{3.720079in}{3.742007in}}%
\pgfpathlineto{\pgfqpoint{3.731009in}{3.746766in}}%
\pgfpathlineto{\pgfqpoint{3.741935in}{3.751485in}}%
\pgfpathlineto{\pgfqpoint{3.752856in}{3.756166in}}%
\pgfpathlineto{\pgfqpoint{3.763771in}{3.760810in}}%
\pgfpathlineto{\pgfqpoint{3.774682in}{3.765418in}}%
\pgfpathlineto{\pgfqpoint{3.768381in}{3.772660in}}%
\pgfpathlineto{\pgfqpoint{3.762081in}{3.779720in}}%
\pgfpathlineto{\pgfqpoint{3.755783in}{3.786605in}}%
\pgfpathlineto{\pgfqpoint{3.749486in}{3.793325in}}%
\pgfpathlineto{\pgfqpoint{3.743191in}{3.799888in}}%
\pgfpathlineto{\pgfqpoint{3.732293in}{3.795346in}}%
\pgfpathlineto{\pgfqpoint{3.721391in}{3.790768in}}%
\pgfpathlineto{\pgfqpoint{3.710484in}{3.786151in}}%
\pgfpathlineto{\pgfqpoint{3.699571in}{3.781497in}}%
\pgfpathlineto{\pgfqpoint{3.688654in}{3.776803in}}%
\pgfpathlineto{\pgfqpoint{3.694936in}{3.770225in}}%
\pgfpathlineto{\pgfqpoint{3.701220in}{3.763456in}}%
\pgfpathlineto{\pgfqpoint{3.707505in}{3.756496in}}%
\pgfpathlineto{\pgfqpoint{3.713791in}{3.749346in}}%
\pgfpathclose%
\pgfusepath{stroke,fill}%
\end{pgfscope}%
\begin{pgfscope}%
\pgfpathrectangle{\pgfqpoint{0.887500in}{0.275000in}}{\pgfqpoint{4.225000in}{4.225000in}}%
\pgfusepath{clip}%
\pgfsetbuttcap%
\pgfsetroundjoin%
\definecolor{currentfill}{rgb}{0.906311,0.894855,0.098125}%
\pgfsetfillcolor{currentfill}%
\pgfsetfillopacity{0.700000}%
\pgfsetlinewidth{0.501875pt}%
\definecolor{currentstroke}{rgb}{1.000000,1.000000,1.000000}%
\pgfsetstrokecolor{currentstroke}%
\pgfsetstrokeopacity{0.500000}%
\pgfsetdash{}{0pt}%
\pgfpathmoveto{\pgfqpoint{3.438212in}{3.703804in}}%
\pgfpathlineto{\pgfqpoint{3.449208in}{3.709389in}}%
\pgfpathlineto{\pgfqpoint{3.460199in}{3.714923in}}%
\pgfpathlineto{\pgfqpoint{3.471186in}{3.720404in}}%
\pgfpathlineto{\pgfqpoint{3.482168in}{3.725833in}}%
\pgfpathlineto{\pgfqpoint{3.493146in}{3.731210in}}%
\pgfpathlineto{\pgfqpoint{3.486914in}{3.736166in}}%
\pgfpathlineto{\pgfqpoint{3.480683in}{3.740817in}}%
\pgfpathlineto{\pgfqpoint{3.474455in}{3.745165in}}%
\pgfpathlineto{\pgfqpoint{3.468228in}{3.749211in}}%
\pgfpathlineto{\pgfqpoint{3.462004in}{3.752956in}}%
\pgfpathlineto{\pgfqpoint{3.451039in}{3.747235in}}%
\pgfpathlineto{\pgfqpoint{3.440071in}{3.741461in}}%
\pgfpathlineto{\pgfqpoint{3.429098in}{3.735646in}}%
\pgfpathlineto{\pgfqpoint{3.418121in}{3.729805in}}%
\pgfpathlineto{\pgfqpoint{3.407140in}{3.723939in}}%
\pgfpathlineto{\pgfqpoint{3.413349in}{3.720523in}}%
\pgfpathlineto{\pgfqpoint{3.419561in}{3.716812in}}%
\pgfpathlineto{\pgfqpoint{3.425776in}{3.712795in}}%
\pgfpathlineto{\pgfqpoint{3.431993in}{3.708462in}}%
\pgfpathclose%
\pgfusepath{stroke,fill}%
\end{pgfscope}%
\begin{pgfscope}%
\pgfpathrectangle{\pgfqpoint{0.887500in}{0.275000in}}{\pgfqpoint{4.225000in}{4.225000in}}%
\pgfusepath{clip}%
\pgfsetbuttcap%
\pgfsetroundjoin%
\definecolor{currentfill}{rgb}{0.170948,0.694384,0.493803}%
\pgfsetfillcolor{currentfill}%
\pgfsetfillopacity{0.700000}%
\pgfsetlinewidth{0.501875pt}%
\definecolor{currentstroke}{rgb}{1.000000,1.000000,1.000000}%
\pgfsetstrokecolor{currentstroke}%
\pgfsetstrokeopacity{0.500000}%
\pgfsetdash{}{0pt}%
\pgfpathmoveto{\pgfqpoint{2.517466in}{3.050702in}}%
\pgfpathlineto{\pgfqpoint{2.528629in}{3.055742in}}%
\pgfpathlineto{\pgfqpoint{2.539770in}{3.062251in}}%
\pgfpathlineto{\pgfqpoint{2.550887in}{3.070302in}}%
\pgfpathlineto{\pgfqpoint{2.561983in}{3.079886in}}%
\pgfpathlineto{\pgfqpoint{2.573059in}{3.090995in}}%
\pgfpathlineto{\pgfqpoint{2.567178in}{3.096985in}}%
\pgfpathlineto{\pgfqpoint{2.561303in}{3.102869in}}%
\pgfpathlineto{\pgfqpoint{2.555434in}{3.108646in}}%
\pgfpathlineto{\pgfqpoint{2.549570in}{3.114318in}}%
\pgfpathlineto{\pgfqpoint{2.543712in}{3.119914in}}%
\pgfpathlineto{\pgfqpoint{2.532605in}{3.112702in}}%
\pgfpathlineto{\pgfqpoint{2.521482in}{3.106464in}}%
\pgfpathlineto{\pgfqpoint{2.510342in}{3.101281in}}%
\pgfpathlineto{\pgfqpoint{2.499181in}{3.097232in}}%
\pgfpathlineto{\pgfqpoint{2.488000in}{3.094297in}}%
\pgfpathlineto{\pgfqpoint{2.493873in}{3.086513in}}%
\pgfpathlineto{\pgfqpoint{2.499756in}{3.078235in}}%
\pgfpathlineto{\pgfqpoint{2.505651in}{3.069430in}}%
\pgfpathlineto{\pgfqpoint{2.511555in}{3.060206in}}%
\pgfpathclose%
\pgfusepath{stroke,fill}%
\end{pgfscope}%
\begin{pgfscope}%
\pgfpathrectangle{\pgfqpoint{0.887500in}{0.275000in}}{\pgfqpoint{4.225000in}{4.225000in}}%
\pgfusepath{clip}%
\pgfsetbuttcap%
\pgfsetroundjoin%
\definecolor{currentfill}{rgb}{0.647257,0.858400,0.209861}%
\pgfsetfillcolor{currentfill}%
\pgfsetfillopacity{0.700000}%
\pgfsetlinewidth{0.501875pt}%
\definecolor{currentstroke}{rgb}{1.000000,1.000000,1.000000}%
\pgfsetstrokecolor{currentstroke}%
\pgfsetstrokeopacity{0.500000}%
\pgfsetdash{}{0pt}%
\pgfpathmoveto{\pgfqpoint{2.904543in}{3.521209in}}%
\pgfpathlineto{\pgfqpoint{2.915632in}{3.528233in}}%
\pgfpathlineto{\pgfqpoint{2.926720in}{3.534135in}}%
\pgfpathlineto{\pgfqpoint{2.937808in}{3.539171in}}%
\pgfpathlineto{\pgfqpoint{2.948893in}{3.543596in}}%
\pgfpathlineto{\pgfqpoint{2.959974in}{3.547667in}}%
\pgfpathlineto{\pgfqpoint{2.953961in}{3.548256in}}%
\pgfpathlineto{\pgfqpoint{2.947953in}{3.548901in}}%
\pgfpathlineto{\pgfqpoint{2.941952in}{3.549617in}}%
\pgfpathlineto{\pgfqpoint{2.935956in}{3.550417in}}%
\pgfpathlineto{\pgfqpoint{2.929965in}{3.551297in}}%
\pgfpathlineto{\pgfqpoint{2.918907in}{3.546643in}}%
\pgfpathlineto{\pgfqpoint{2.907847in}{3.541317in}}%
\pgfpathlineto{\pgfqpoint{2.896787in}{3.534967in}}%
\pgfpathlineto{\pgfqpoint{2.885730in}{3.527238in}}%
\pgfpathlineto{\pgfqpoint{2.874677in}{3.517786in}}%
\pgfpathlineto{\pgfqpoint{2.880640in}{3.517860in}}%
\pgfpathlineto{\pgfqpoint{2.886608in}{3.518265in}}%
\pgfpathlineto{\pgfqpoint{2.892581in}{3.519007in}}%
\pgfpathlineto{\pgfqpoint{2.898559in}{3.520018in}}%
\pgfpathclose%
\pgfusepath{stroke,fill}%
\end{pgfscope}%
\begin{pgfscope}%
\pgfpathrectangle{\pgfqpoint{0.887500in}{0.275000in}}{\pgfqpoint{4.225000in}{4.225000in}}%
\pgfusepath{clip}%
\pgfsetbuttcap%
\pgfsetroundjoin%
\definecolor{currentfill}{rgb}{0.945636,0.899815,0.112838}%
\pgfsetfillcolor{currentfill}%
\pgfsetfillopacity{0.700000}%
\pgfsetlinewidth{0.501875pt}%
\definecolor{currentstroke}{rgb}{1.000000,1.000000,1.000000}%
\pgfsetstrokecolor{currentstroke}%
\pgfsetstrokeopacity{0.500000}%
\pgfsetdash{}{0pt}%
\pgfpathmoveto{\pgfqpoint{3.579218in}{3.727629in}}%
\pgfpathlineto{\pgfqpoint{3.590183in}{3.732712in}}%
\pgfpathlineto{\pgfqpoint{3.601143in}{3.737765in}}%
\pgfpathlineto{\pgfqpoint{3.612098in}{3.742786in}}%
\pgfpathlineto{\pgfqpoint{3.623049in}{3.747771in}}%
\pgfpathlineto{\pgfqpoint{3.633995in}{3.752716in}}%
\pgfpathlineto{\pgfqpoint{3.627728in}{3.759115in}}%
\pgfpathlineto{\pgfqpoint{3.621463in}{3.765291in}}%
\pgfpathlineto{\pgfqpoint{3.615198in}{3.771236in}}%
\pgfpathlineto{\pgfqpoint{3.608936in}{3.776943in}}%
\pgfpathlineto{\pgfqpoint{3.602674in}{3.782402in}}%
\pgfpathlineto{\pgfqpoint{3.591742in}{3.777458in}}%
\pgfpathlineto{\pgfqpoint{3.580805in}{3.772484in}}%
\pgfpathlineto{\pgfqpoint{3.569864in}{3.767476in}}%
\pgfpathlineto{\pgfqpoint{3.558918in}{3.762431in}}%
\pgfpathlineto{\pgfqpoint{3.547968in}{3.757345in}}%
\pgfpathlineto{\pgfqpoint{3.554215in}{3.751915in}}%
\pgfpathlineto{\pgfqpoint{3.560463in}{3.746222in}}%
\pgfpathlineto{\pgfqpoint{3.566713in}{3.740271in}}%
\pgfpathlineto{\pgfqpoint{3.572965in}{3.734069in}}%
\pgfpathclose%
\pgfusepath{stroke,fill}%
\end{pgfscope}%
\begin{pgfscope}%
\pgfpathrectangle{\pgfqpoint{0.887500in}{0.275000in}}{\pgfqpoint{4.225000in}{4.225000in}}%
\pgfusepath{clip}%
\pgfsetbuttcap%
\pgfsetroundjoin%
\definecolor{currentfill}{rgb}{0.335885,0.777018,0.402049}%
\pgfsetfillcolor{currentfill}%
\pgfsetfillopacity{0.700000}%
\pgfsetlinewidth{0.501875pt}%
\definecolor{currentstroke}{rgb}{1.000000,1.000000,1.000000}%
\pgfsetstrokecolor{currentstroke}%
\pgfsetstrokeopacity{0.500000}%
\pgfsetdash{}{0pt}%
\pgfpathmoveto{\pgfqpoint{2.683202in}{3.262998in}}%
\pgfpathlineto{\pgfqpoint{2.694230in}{3.280175in}}%
\pgfpathlineto{\pgfqpoint{2.705268in}{3.296466in}}%
\pgfpathlineto{\pgfqpoint{2.716315in}{3.311990in}}%
\pgfpathlineto{\pgfqpoint{2.727368in}{3.326870in}}%
\pgfpathlineto{\pgfqpoint{2.738427in}{3.341228in}}%
\pgfpathlineto{\pgfqpoint{2.732603in}{3.331333in}}%
\pgfpathlineto{\pgfqpoint{2.726783in}{3.321946in}}%
\pgfpathlineto{\pgfqpoint{2.720965in}{3.313498in}}%
\pgfpathlineto{\pgfqpoint{2.715144in}{3.306350in}}%
\pgfpathlineto{\pgfqpoint{2.709320in}{3.300541in}}%
\pgfpathlineto{\pgfqpoint{2.698303in}{3.286076in}}%
\pgfpathlineto{\pgfqpoint{2.687287in}{3.271744in}}%
\pgfpathlineto{\pgfqpoint{2.676269in}{3.257595in}}%
\pgfpathlineto{\pgfqpoint{2.665250in}{3.243680in}}%
\pgfpathlineto{\pgfqpoint{2.654229in}{3.230049in}}%
\pgfpathlineto{\pgfqpoint{2.660020in}{3.235447in}}%
\pgfpathlineto{\pgfqpoint{2.665809in}{3.241677in}}%
\pgfpathlineto{\pgfqpoint{2.671600in}{3.248630in}}%
\pgfpathlineto{\pgfqpoint{2.677396in}{3.255895in}}%
\pgfpathclose%
\pgfusepath{stroke,fill}%
\end{pgfscope}%
\begin{pgfscope}%
\pgfpathrectangle{\pgfqpoint{0.887500in}{0.275000in}}{\pgfqpoint{4.225000in}{4.225000in}}%
\pgfusepath{clip}%
\pgfsetbuttcap%
\pgfsetroundjoin%
\definecolor{currentfill}{rgb}{0.421908,0.805774,0.351910}%
\pgfsetfillcolor{currentfill}%
\pgfsetfillopacity{0.700000}%
\pgfsetlinewidth{0.501875pt}%
\definecolor{currentstroke}{rgb}{1.000000,1.000000,1.000000}%
\pgfsetstrokecolor{currentstroke}%
\pgfsetstrokeopacity{0.500000}%
\pgfsetdash{}{0pt}%
\pgfpathmoveto{\pgfqpoint{2.738427in}{3.341228in}}%
\pgfpathlineto{\pgfqpoint{2.749490in}{3.355185in}}%
\pgfpathlineto{\pgfqpoint{2.760555in}{3.368862in}}%
\pgfpathlineto{\pgfqpoint{2.771622in}{3.382336in}}%
\pgfpathlineto{\pgfqpoint{2.782691in}{3.395611in}}%
\pgfpathlineto{\pgfqpoint{2.793761in}{3.408686in}}%
\pgfpathlineto{\pgfqpoint{2.787889in}{3.399458in}}%
\pgfpathlineto{\pgfqpoint{2.782021in}{3.391071in}}%
\pgfpathlineto{\pgfqpoint{2.776152in}{3.383843in}}%
\pgfpathlineto{\pgfqpoint{2.770282in}{3.378034in}}%
\pgfpathlineto{\pgfqpoint{2.764410in}{3.373628in}}%
\pgfpathlineto{\pgfqpoint{2.753391in}{3.358900in}}%
\pgfpathlineto{\pgfqpoint{2.742372in}{3.344258in}}%
\pgfpathlineto{\pgfqpoint{2.731354in}{3.329665in}}%
\pgfpathlineto{\pgfqpoint{2.720336in}{3.315087in}}%
\pgfpathlineto{\pgfqpoint{2.709320in}{3.300541in}}%
\pgfpathlineto{\pgfqpoint{2.715144in}{3.306350in}}%
\pgfpathlineto{\pgfqpoint{2.720965in}{3.313498in}}%
\pgfpathlineto{\pgfqpoint{2.726783in}{3.321946in}}%
\pgfpathlineto{\pgfqpoint{2.732603in}{3.331333in}}%
\pgfpathclose%
\pgfusepath{stroke,fill}%
\end{pgfscope}%
\begin{pgfscope}%
\pgfpathrectangle{\pgfqpoint{0.887500in}{0.275000in}}{\pgfqpoint{4.225000in}{4.225000in}}%
\pgfusepath{clip}%
\pgfsetbuttcap%
\pgfsetroundjoin%
\definecolor{currentfill}{rgb}{0.814576,0.883393,0.110347}%
\pgfsetfillcolor{currentfill}%
\pgfsetfillopacity{0.700000}%
\pgfsetlinewidth{0.501875pt}%
\definecolor{currentstroke}{rgb}{1.000000,1.000000,1.000000}%
\pgfsetstrokecolor{currentstroke}%
\pgfsetstrokeopacity{0.500000}%
\pgfsetdash{}{0pt}%
\pgfpathmoveto{\pgfqpoint{3.241989in}{3.637470in}}%
\pgfpathlineto{\pgfqpoint{3.253021in}{3.642310in}}%
\pgfpathlineto{\pgfqpoint{3.264050in}{3.647323in}}%
\pgfpathlineto{\pgfqpoint{3.275076in}{3.652568in}}%
\pgfpathlineto{\pgfqpoint{3.286099in}{3.658074in}}%
\pgfpathlineto{\pgfqpoint{3.297119in}{3.663804in}}%
\pgfpathlineto{\pgfqpoint{3.290948in}{3.666553in}}%
\pgfpathlineto{\pgfqpoint{3.284781in}{3.669069in}}%
\pgfpathlineto{\pgfqpoint{3.278619in}{3.671364in}}%
\pgfpathlineto{\pgfqpoint{3.272461in}{3.673457in}}%
\pgfpathlineto{\pgfqpoint{3.266308in}{3.675368in}}%
\pgfpathlineto{\pgfqpoint{3.255307in}{3.669486in}}%
\pgfpathlineto{\pgfqpoint{3.244303in}{3.663817in}}%
\pgfpathlineto{\pgfqpoint{3.233296in}{3.658399in}}%
\pgfpathlineto{\pgfqpoint{3.222286in}{3.653195in}}%
\pgfpathlineto{\pgfqpoint{3.211273in}{3.648141in}}%
\pgfpathlineto{\pgfqpoint{3.217407in}{3.646502in}}%
\pgfpathlineto{\pgfqpoint{3.223546in}{3.644637in}}%
\pgfpathlineto{\pgfqpoint{3.229689in}{3.642515in}}%
\pgfpathlineto{\pgfqpoint{3.235837in}{3.640124in}}%
\pgfpathclose%
\pgfusepath{stroke,fill}%
\end{pgfscope}%
\begin{pgfscope}%
\pgfpathrectangle{\pgfqpoint{0.887500in}{0.275000in}}{\pgfqpoint{4.225000in}{4.225000in}}%
\pgfusepath{clip}%
\pgfsetbuttcap%
\pgfsetroundjoin%
\definecolor{currentfill}{rgb}{0.124780,0.640461,0.527068}%
\pgfsetfillcolor{currentfill}%
\pgfsetfillopacity{0.700000}%
\pgfsetlinewidth{0.501875pt}%
\definecolor{currentstroke}{rgb}{1.000000,1.000000,1.000000}%
\pgfsetstrokecolor{currentstroke}%
\pgfsetstrokeopacity{0.500000}%
\pgfsetdash{}{0pt}%
\pgfpathmoveto{\pgfqpoint{2.038945in}{2.932509in}}%
\pgfpathlineto{\pgfqpoint{2.050197in}{2.937884in}}%
\pgfpathlineto{\pgfqpoint{2.061432in}{2.943758in}}%
\pgfpathlineto{\pgfqpoint{2.072651in}{2.950144in}}%
\pgfpathlineto{\pgfqpoint{2.083855in}{2.956959in}}%
\pgfpathlineto{\pgfqpoint{2.095049in}{2.964109in}}%
\pgfpathlineto{\pgfqpoint{2.089259in}{2.973724in}}%
\pgfpathlineto{\pgfqpoint{2.083476in}{2.983206in}}%
\pgfpathlineto{\pgfqpoint{2.077700in}{2.992526in}}%
\pgfpathlineto{\pgfqpoint{2.071933in}{3.001652in}}%
\pgfpathlineto{\pgfqpoint{2.066174in}{3.010553in}}%
\pgfpathlineto{\pgfqpoint{2.055037in}{3.001650in}}%
\pgfpathlineto{\pgfqpoint{2.043889in}{2.993123in}}%
\pgfpathlineto{\pgfqpoint{2.032722in}{2.985217in}}%
\pgfpathlineto{\pgfqpoint{2.021531in}{2.978150in}}%
\pgfpathlineto{\pgfqpoint{2.010315in}{2.971915in}}%
\pgfpathlineto{\pgfqpoint{2.016032in}{2.964104in}}%
\pgfpathlineto{\pgfqpoint{2.021754in}{2.956254in}}%
\pgfpathlineto{\pgfqpoint{2.027480in}{2.948369in}}%
\pgfpathlineto{\pgfqpoint{2.033210in}{2.940452in}}%
\pgfpathclose%
\pgfusepath{stroke,fill}%
\end{pgfscope}%
\begin{pgfscope}%
\pgfpathrectangle{\pgfqpoint{0.887500in}{0.275000in}}{\pgfqpoint{4.225000in}{4.225000in}}%
\pgfusepath{clip}%
\pgfsetbuttcap%
\pgfsetroundjoin%
\definecolor{currentfill}{rgb}{0.130067,0.651384,0.521608}%
\pgfsetfillcolor{currentfill}%
\pgfsetfillopacity{0.700000}%
\pgfsetlinewidth{0.501875pt}%
\definecolor{currentstroke}{rgb}{1.000000,1.000000,1.000000}%
\pgfsetstrokecolor{currentstroke}%
\pgfsetstrokeopacity{0.500000}%
\pgfsetdash{}{0pt}%
\pgfpathmoveto{\pgfqpoint{2.180115in}{2.947539in}}%
\pgfpathlineto{\pgfqpoint{2.191274in}{2.955908in}}%
\pgfpathlineto{\pgfqpoint{2.202421in}{2.964699in}}%
\pgfpathlineto{\pgfqpoint{2.213560in}{2.973769in}}%
\pgfpathlineto{\pgfqpoint{2.224694in}{2.982973in}}%
\pgfpathlineto{\pgfqpoint{2.235827in}{2.992168in}}%
\pgfpathlineto{\pgfqpoint{2.230007in}{3.001260in}}%
\pgfpathlineto{\pgfqpoint{2.224191in}{3.010381in}}%
\pgfpathlineto{\pgfqpoint{2.218377in}{3.019518in}}%
\pgfpathlineto{\pgfqpoint{2.212566in}{3.028656in}}%
\pgfpathlineto{\pgfqpoint{2.206759in}{3.037782in}}%
\pgfpathlineto{\pgfqpoint{2.195597in}{3.030662in}}%
\pgfpathlineto{\pgfqpoint{2.184432in}{3.023496in}}%
\pgfpathlineto{\pgfqpoint{2.173265in}{3.016274in}}%
\pgfpathlineto{\pgfqpoint{2.162097in}{3.008986in}}%
\pgfpathlineto{\pgfqpoint{2.150927in}{3.001621in}}%
\pgfpathlineto{\pgfqpoint{2.156757in}{2.990895in}}%
\pgfpathlineto{\pgfqpoint{2.162592in}{2.980079in}}%
\pgfpathlineto{\pgfqpoint{2.168430in}{2.969219in}}%
\pgfpathlineto{\pgfqpoint{2.174272in}{2.958357in}}%
\pgfpathclose%
\pgfusepath{stroke,fill}%
\end{pgfscope}%
\begin{pgfscope}%
\pgfpathrectangle{\pgfqpoint{0.887500in}{0.275000in}}{\pgfqpoint{4.225000in}{4.225000in}}%
\pgfusepath{clip}%
\pgfsetbuttcap%
\pgfsetroundjoin%
\definecolor{currentfill}{rgb}{0.506271,0.828786,0.300362}%
\pgfsetfillcolor{currentfill}%
\pgfsetfillopacity{0.700000}%
\pgfsetlinewidth{0.501875pt}%
\definecolor{currentstroke}{rgb}{1.000000,1.000000,1.000000}%
\pgfsetstrokecolor{currentstroke}%
\pgfsetstrokeopacity{0.500000}%
\pgfsetdash{}{0pt}%
\pgfpathmoveto{\pgfqpoint{2.793761in}{3.408686in}}%
\pgfpathlineto{\pgfqpoint{2.804833in}{3.421561in}}%
\pgfpathlineto{\pgfqpoint{2.815906in}{3.434236in}}%
\pgfpathlineto{\pgfqpoint{2.826981in}{3.446709in}}%
\pgfpathlineto{\pgfqpoint{2.838056in}{3.458977in}}%
\pgfpathlineto{\pgfqpoint{2.849132in}{3.470964in}}%
\pgfpathlineto{\pgfqpoint{2.843200in}{3.465206in}}%
\pgfpathlineto{\pgfqpoint{2.837274in}{3.460025in}}%
\pgfpathlineto{\pgfqpoint{2.831350in}{3.455634in}}%
\pgfpathlineto{\pgfqpoint{2.825429in}{3.452203in}}%
\pgfpathlineto{\pgfqpoint{2.819509in}{3.449716in}}%
\pgfpathlineto{\pgfqpoint{2.808488in}{3.434168in}}%
\pgfpathlineto{\pgfqpoint{2.797469in}{3.418717in}}%
\pgfpathlineto{\pgfqpoint{2.786449in}{3.403499in}}%
\pgfpathlineto{\pgfqpoint{2.775430in}{3.388482in}}%
\pgfpathlineto{\pgfqpoint{2.764410in}{3.373628in}}%
\pgfpathlineto{\pgfqpoint{2.770282in}{3.378034in}}%
\pgfpathlineto{\pgfqpoint{2.776152in}{3.383843in}}%
\pgfpathlineto{\pgfqpoint{2.782021in}{3.391071in}}%
\pgfpathlineto{\pgfqpoint{2.787889in}{3.399458in}}%
\pgfpathclose%
\pgfusepath{stroke,fill}%
\end{pgfscope}%
\begin{pgfscope}%
\pgfpathrectangle{\pgfqpoint{0.887500in}{0.275000in}}{\pgfqpoint{4.225000in}{4.225000in}}%
\pgfusepath{clip}%
\pgfsetbuttcap%
\pgfsetroundjoin%
\definecolor{currentfill}{rgb}{0.170948,0.694384,0.493803}%
\pgfsetfillcolor{currentfill}%
\pgfsetfillopacity{0.700000}%
\pgfsetlinewidth{0.501875pt}%
\definecolor{currentstroke}{rgb}{1.000000,1.000000,1.000000}%
\pgfsetstrokecolor{currentstroke}%
\pgfsetstrokeopacity{0.500000}%
\pgfsetdash{}{0pt}%
\pgfpathmoveto{\pgfqpoint{2.376178in}{3.047676in}}%
\pgfpathlineto{\pgfqpoint{2.387306in}{3.056808in}}%
\pgfpathlineto{\pgfqpoint{2.398445in}{3.065154in}}%
\pgfpathlineto{\pgfqpoint{2.409598in}{3.072455in}}%
\pgfpathlineto{\pgfqpoint{2.420769in}{3.078452in}}%
\pgfpathlineto{\pgfqpoint{2.431960in}{3.082942in}}%
\pgfpathlineto{\pgfqpoint{2.426107in}{3.090389in}}%
\pgfpathlineto{\pgfqpoint{2.420267in}{3.097244in}}%
\pgfpathlineto{\pgfqpoint{2.414439in}{3.103619in}}%
\pgfpathlineto{\pgfqpoint{2.408621in}{3.109627in}}%
\pgfpathlineto{\pgfqpoint{2.402811in}{3.115378in}}%
\pgfpathlineto{\pgfqpoint{2.391639in}{3.110661in}}%
\pgfpathlineto{\pgfqpoint{2.380484in}{3.104661in}}%
\pgfpathlineto{\pgfqpoint{2.369344in}{3.097565in}}%
\pgfpathlineto{\pgfqpoint{2.358215in}{3.089606in}}%
\pgfpathlineto{\pgfqpoint{2.347095in}{3.081018in}}%
\pgfpathlineto{\pgfqpoint{2.352895in}{3.074850in}}%
\pgfpathlineto{\pgfqpoint{2.358702in}{3.068510in}}%
\pgfpathlineto{\pgfqpoint{2.364517in}{3.061920in}}%
\pgfpathlineto{\pgfqpoint{2.370342in}{3.055001in}}%
\pgfpathclose%
\pgfusepath{stroke,fill}%
\end{pgfscope}%
\begin{pgfscope}%
\pgfpathrectangle{\pgfqpoint{0.887500in}{0.275000in}}{\pgfqpoint{4.225000in}{4.225000in}}%
\pgfusepath{clip}%
\pgfsetbuttcap%
\pgfsetroundjoin%
\definecolor{currentfill}{rgb}{0.595839,0.848717,0.243329}%
\pgfsetfillcolor{currentfill}%
\pgfsetfillopacity{0.700000}%
\pgfsetlinewidth{0.501875pt}%
\definecolor{currentstroke}{rgb}{1.000000,1.000000,1.000000}%
\pgfsetstrokecolor{currentstroke}%
\pgfsetstrokeopacity{0.500000}%
\pgfsetdash{}{0pt}%
\pgfpathmoveto{\pgfqpoint{2.849132in}{3.470964in}}%
\pgfpathlineto{\pgfqpoint{2.860210in}{3.482515in}}%
\pgfpathlineto{\pgfqpoint{2.871290in}{3.493473in}}%
\pgfpathlineto{\pgfqpoint{2.882372in}{3.503680in}}%
\pgfpathlineto{\pgfqpoint{2.893457in}{3.512978in}}%
\pgfpathlineto{\pgfqpoint{2.904543in}{3.521209in}}%
\pgfpathlineto{\pgfqpoint{2.898559in}{3.520018in}}%
\pgfpathlineto{\pgfqpoint{2.892581in}{3.519007in}}%
\pgfpathlineto{\pgfqpoint{2.886608in}{3.518265in}}%
\pgfpathlineto{\pgfqpoint{2.880640in}{3.517860in}}%
\pgfpathlineto{\pgfqpoint{2.874677in}{3.517786in}}%
\pgfpathlineto{\pgfqpoint{2.863631in}{3.506535in}}%
\pgfpathlineto{\pgfqpoint{2.852592in}{3.493780in}}%
\pgfpathlineto{\pgfqpoint{2.841559in}{3.479844in}}%
\pgfpathlineto{\pgfqpoint{2.830532in}{3.465049in}}%
\pgfpathlineto{\pgfqpoint{2.819509in}{3.449716in}}%
\pgfpathlineto{\pgfqpoint{2.825429in}{3.452203in}}%
\pgfpathlineto{\pgfqpoint{2.831350in}{3.455634in}}%
\pgfpathlineto{\pgfqpoint{2.837274in}{3.460025in}}%
\pgfpathlineto{\pgfqpoint{2.843200in}{3.465206in}}%
\pgfpathclose%
\pgfusepath{stroke,fill}%
\end{pgfscope}%
\begin{pgfscope}%
\pgfpathrectangle{\pgfqpoint{0.887500in}{0.275000in}}{\pgfqpoint{4.225000in}{4.225000in}}%
\pgfusepath{clip}%
\pgfsetbuttcap%
\pgfsetroundjoin%
\definecolor{currentfill}{rgb}{0.720391,0.870350,0.162603}%
\pgfsetfillcolor{currentfill}%
\pgfsetfillopacity{0.700000}%
\pgfsetlinewidth{0.501875pt}%
\definecolor{currentstroke}{rgb}{1.000000,1.000000,1.000000}%
\pgfsetstrokecolor{currentstroke}%
\pgfsetstrokeopacity{0.500000}%
\pgfsetdash{}{0pt}%
\pgfpathmoveto{\pgfqpoint{3.045573in}{3.566955in}}%
\pgfpathlineto{\pgfqpoint{3.056648in}{3.572223in}}%
\pgfpathlineto{\pgfqpoint{3.067718in}{3.577684in}}%
\pgfpathlineto{\pgfqpoint{3.078785in}{3.583285in}}%
\pgfpathlineto{\pgfqpoint{3.089848in}{3.588976in}}%
\pgfpathlineto{\pgfqpoint{3.100907in}{3.594708in}}%
\pgfpathlineto{\pgfqpoint{3.094820in}{3.595662in}}%
\pgfpathlineto{\pgfqpoint{3.088738in}{3.596574in}}%
\pgfpathlineto{\pgfqpoint{3.082663in}{3.597457in}}%
\pgfpathlineto{\pgfqpoint{3.076593in}{3.598323in}}%
\pgfpathlineto{\pgfqpoint{3.070530in}{3.599185in}}%
\pgfpathlineto{\pgfqpoint{3.059493in}{3.593419in}}%
\pgfpathlineto{\pgfqpoint{3.048452in}{3.587597in}}%
\pgfpathlineto{\pgfqpoint{3.037407in}{3.581788in}}%
\pgfpathlineto{\pgfqpoint{3.026359in}{3.576074in}}%
\pgfpathlineto{\pgfqpoint{3.015307in}{3.570535in}}%
\pgfpathlineto{\pgfqpoint{3.021348in}{3.569717in}}%
\pgfpathlineto{\pgfqpoint{3.027395in}{3.568948in}}%
\pgfpathlineto{\pgfqpoint{3.033448in}{3.568230in}}%
\pgfpathlineto{\pgfqpoint{3.039508in}{3.567564in}}%
\pgfpathclose%
\pgfusepath{stroke,fill}%
\end{pgfscope}%
\begin{pgfscope}%
\pgfpathrectangle{\pgfqpoint{0.887500in}{0.275000in}}{\pgfqpoint{4.225000in}{4.225000in}}%
\pgfusepath{clip}%
\pgfsetbuttcap%
\pgfsetroundjoin%
\definecolor{currentfill}{rgb}{0.955300,0.901065,0.118128}%
\pgfsetfillcolor{currentfill}%
\pgfsetfillopacity{0.700000}%
\pgfsetlinewidth{0.501875pt}%
\definecolor{currentstroke}{rgb}{1.000000,1.000000,1.000000}%
\pgfsetstrokecolor{currentstroke}%
\pgfsetstrokeopacity{0.500000}%
\pgfsetdash{}{0pt}%
\pgfpathmoveto{\pgfqpoint{3.806197in}{3.726174in}}%
\pgfpathlineto{\pgfqpoint{3.817116in}{3.730834in}}%
\pgfpathlineto{\pgfqpoint{3.828029in}{3.735474in}}%
\pgfpathlineto{\pgfqpoint{3.838939in}{3.740099in}}%
\pgfpathlineto{\pgfqpoint{3.849843in}{3.744710in}}%
\pgfpathlineto{\pgfqpoint{3.843530in}{3.752928in}}%
\pgfpathlineto{\pgfqpoint{3.837217in}{3.760904in}}%
\pgfpathlineto{\pgfqpoint{3.830903in}{3.768649in}}%
\pgfpathlineto{\pgfqpoint{3.824589in}{3.776177in}}%
\pgfpathlineto{\pgfqpoint{3.818276in}{3.783505in}}%
\pgfpathlineto{\pgfqpoint{3.807385in}{3.779033in}}%
\pgfpathlineto{\pgfqpoint{3.796489in}{3.774529in}}%
\pgfpathlineto{\pgfqpoint{3.785588in}{3.769990in}}%
\pgfpathlineto{\pgfqpoint{3.774682in}{3.765418in}}%
\pgfpathlineto{\pgfqpoint{3.780984in}{3.757985in}}%
\pgfpathlineto{\pgfqpoint{3.787287in}{3.750352in}}%
\pgfpathlineto{\pgfqpoint{3.793590in}{3.742511in}}%
\pgfpathlineto{\pgfqpoint{3.799894in}{3.734454in}}%
\pgfpathclose%
\pgfusepath{stroke,fill}%
\end{pgfscope}%
\begin{pgfscope}%
\pgfpathrectangle{\pgfqpoint{0.887500in}{0.275000in}}{\pgfqpoint{4.225000in}{4.225000in}}%
\pgfusepath{clip}%
\pgfsetbuttcap%
\pgfsetroundjoin%
\definecolor{currentfill}{rgb}{0.123444,0.636809,0.528763}%
\pgfsetfillcolor{currentfill}%
\pgfsetfillopacity{0.700000}%
\pgfsetlinewidth{0.501875pt}%
\definecolor{currentstroke}{rgb}{1.000000,1.000000,1.000000}%
\pgfsetstrokecolor{currentstroke}%
\pgfsetstrokeopacity{0.500000}%
\pgfsetdash{}{0pt}%
\pgfpathmoveto{\pgfqpoint{1.897316in}{2.931476in}}%
\pgfpathlineto{\pgfqpoint{1.908656in}{2.934897in}}%
\pgfpathlineto{\pgfqpoint{1.919990in}{2.938353in}}%
\pgfpathlineto{\pgfqpoint{1.931317in}{2.941864in}}%
\pgfpathlineto{\pgfqpoint{1.942637in}{2.945452in}}%
\pgfpathlineto{\pgfqpoint{1.953948in}{2.949139in}}%
\pgfpathlineto{\pgfqpoint{1.948254in}{2.956739in}}%
\pgfpathlineto{\pgfqpoint{1.942563in}{2.964329in}}%
\pgfpathlineto{\pgfqpoint{1.936877in}{2.971909in}}%
\pgfpathlineto{\pgfqpoint{1.931194in}{2.979478in}}%
\pgfpathlineto{\pgfqpoint{1.919894in}{2.975764in}}%
\pgfpathlineto{\pgfqpoint{1.908585in}{2.972147in}}%
\pgfpathlineto{\pgfqpoint{1.897269in}{2.968605in}}%
\pgfpathlineto{\pgfqpoint{1.885947in}{2.965117in}}%
\pgfpathlineto{\pgfqpoint{1.874618in}{2.961661in}}%
\pgfpathlineto{\pgfqpoint{1.880287in}{2.954134in}}%
\pgfpathlineto{\pgfqpoint{1.885959in}{2.946594in}}%
\pgfpathlineto{\pgfqpoint{1.891636in}{2.939041in}}%
\pgfpathclose%
\pgfusepath{stroke,fill}%
\end{pgfscope}%
\begin{pgfscope}%
\pgfpathrectangle{\pgfqpoint{0.887500in}{0.275000in}}{\pgfqpoint{4.225000in}{4.225000in}}%
\pgfusepath{clip}%
\pgfsetbuttcap%
\pgfsetroundjoin%
\definecolor{currentfill}{rgb}{0.886271,0.892374,0.095374}%
\pgfsetfillcolor{currentfill}%
\pgfsetfillopacity{0.700000}%
\pgfsetlinewidth{0.501875pt}%
\definecolor{currentstroke}{rgb}{1.000000,1.000000,1.000000}%
\pgfsetstrokecolor{currentstroke}%
\pgfsetstrokeopacity{0.500000}%
\pgfsetdash{}{0pt}%
\pgfpathmoveto{\pgfqpoint{3.383170in}{3.675149in}}%
\pgfpathlineto{\pgfqpoint{3.394187in}{3.680972in}}%
\pgfpathlineto{\pgfqpoint{3.405200in}{3.686751in}}%
\pgfpathlineto{\pgfqpoint{3.416208in}{3.692484in}}%
\pgfpathlineto{\pgfqpoint{3.427212in}{3.698168in}}%
\pgfpathlineto{\pgfqpoint{3.438212in}{3.703804in}}%
\pgfpathlineto{\pgfqpoint{3.431993in}{3.708462in}}%
\pgfpathlineto{\pgfqpoint{3.425776in}{3.712795in}}%
\pgfpathlineto{\pgfqpoint{3.419561in}{3.716812in}}%
\pgfpathlineto{\pgfqpoint{3.413349in}{3.720523in}}%
\pgfpathlineto{\pgfqpoint{3.407140in}{3.723939in}}%
\pgfpathlineto{\pgfqpoint{3.396155in}{3.718045in}}%
\pgfpathlineto{\pgfqpoint{3.385166in}{3.712118in}}%
\pgfpathlineto{\pgfqpoint{3.374173in}{3.706151in}}%
\pgfpathlineto{\pgfqpoint{3.363177in}{3.700141in}}%
\pgfpathlineto{\pgfqpoint{3.352176in}{3.694082in}}%
\pgfpathlineto{\pgfqpoint{3.358369in}{3.690883in}}%
\pgfpathlineto{\pgfqpoint{3.364565in}{3.687404in}}%
\pgfpathlineto{\pgfqpoint{3.370764in}{3.683630in}}%
\pgfpathlineto{\pgfqpoint{3.376966in}{3.679549in}}%
\pgfpathclose%
\pgfusepath{stroke,fill}%
\end{pgfscope}%
\begin{pgfscope}%
\pgfpathrectangle{\pgfqpoint{0.887500in}{0.275000in}}{\pgfqpoint{4.225000in}{4.225000in}}%
\pgfusepath{clip}%
\pgfsetbuttcap%
\pgfsetroundjoin%
\definecolor{currentfill}{rgb}{0.945636,0.899815,0.112838}%
\pgfsetfillcolor{currentfill}%
\pgfsetfillopacity{0.700000}%
\pgfsetlinewidth{0.501875pt}%
\definecolor{currentstroke}{rgb}{1.000000,1.000000,1.000000}%
\pgfsetstrokecolor{currentstroke}%
\pgfsetstrokeopacity{0.500000}%
\pgfsetdash{}{0pt}%
\pgfpathmoveto{\pgfqpoint{3.665353in}{3.717597in}}%
\pgfpathlineto{\pgfqpoint{3.676308in}{3.722564in}}%
\pgfpathlineto{\pgfqpoint{3.687258in}{3.727488in}}%
\pgfpathlineto{\pgfqpoint{3.698203in}{3.732370in}}%
\pgfpathlineto{\pgfqpoint{3.709143in}{3.737209in}}%
\pgfpathlineto{\pgfqpoint{3.720079in}{3.742007in}}%
\pgfpathlineto{\pgfqpoint{3.713791in}{3.749346in}}%
\pgfpathlineto{\pgfqpoint{3.707505in}{3.756496in}}%
\pgfpathlineto{\pgfqpoint{3.701220in}{3.763456in}}%
\pgfpathlineto{\pgfqpoint{3.694936in}{3.770225in}}%
\pgfpathlineto{\pgfqpoint{3.688654in}{3.776803in}}%
\pgfpathlineto{\pgfqpoint{3.677732in}{3.772069in}}%
\pgfpathlineto{\pgfqpoint{3.666805in}{3.767294in}}%
\pgfpathlineto{\pgfqpoint{3.655873in}{3.762478in}}%
\pgfpathlineto{\pgfqpoint{3.644937in}{3.757619in}}%
\pgfpathlineto{\pgfqpoint{3.633995in}{3.752716in}}%
\pgfpathlineto{\pgfqpoint{3.640264in}{3.746100in}}%
\pgfpathlineto{\pgfqpoint{3.646534in}{3.739274in}}%
\pgfpathlineto{\pgfqpoint{3.652805in}{3.732244in}}%
\pgfpathlineto{\pgfqpoint{3.659078in}{3.725016in}}%
\pgfpathclose%
\pgfusepath{stroke,fill}%
\end{pgfscope}%
\begin{pgfscope}%
\pgfpathrectangle{\pgfqpoint{0.887500in}{0.275000in}}{\pgfqpoint{4.225000in}{4.225000in}}%
\pgfusepath{clip}%
\pgfsetbuttcap%
\pgfsetroundjoin%
\definecolor{currentfill}{rgb}{0.150148,0.676631,0.506589}%
\pgfsetfillcolor{currentfill}%
\pgfsetfillopacity{0.700000}%
\pgfsetlinewidth{0.501875pt}%
\definecolor{currentstroke}{rgb}{1.000000,1.000000,1.000000}%
\pgfsetstrokecolor{currentstroke}%
\pgfsetstrokeopacity{0.500000}%
\pgfsetdash{}{0pt}%
\pgfpathmoveto{\pgfqpoint{2.320570in}{2.998560in}}%
\pgfpathlineto{\pgfqpoint{2.331696in}{3.008330in}}%
\pgfpathlineto{\pgfqpoint{2.342819in}{3.018152in}}%
\pgfpathlineto{\pgfqpoint{2.353939in}{3.028090in}}%
\pgfpathlineto{\pgfqpoint{2.365057in}{3.038017in}}%
\pgfpathlineto{\pgfqpoint{2.376178in}{3.047676in}}%
\pgfpathlineto{\pgfqpoint{2.370342in}{3.055001in}}%
\pgfpathlineto{\pgfqpoint{2.364517in}{3.061920in}}%
\pgfpathlineto{\pgfqpoint{2.358702in}{3.068510in}}%
\pgfpathlineto{\pgfqpoint{2.352895in}{3.074850in}}%
\pgfpathlineto{\pgfqpoint{2.347095in}{3.081018in}}%
\pgfpathlineto{\pgfqpoint{2.335980in}{3.072035in}}%
\pgfpathlineto{\pgfqpoint{2.324865in}{3.062889in}}%
\pgfpathlineto{\pgfqpoint{2.313748in}{3.053812in}}%
\pgfpathlineto{\pgfqpoint{2.302624in}{3.044923in}}%
\pgfpathlineto{\pgfqpoint{2.291496in}{3.036170in}}%
\pgfpathlineto{\pgfqpoint{2.297301in}{3.028763in}}%
\pgfpathlineto{\pgfqpoint{2.303111in}{3.021321in}}%
\pgfpathlineto{\pgfqpoint{2.308925in}{3.013822in}}%
\pgfpathlineto{\pgfqpoint{2.314745in}{3.006243in}}%
\pgfpathclose%
\pgfusepath{stroke,fill}%
\end{pgfscope}%
\begin{pgfscope}%
\pgfpathrectangle{\pgfqpoint{0.887500in}{0.275000in}}{\pgfqpoint{4.225000in}{4.225000in}}%
\pgfusepath{clip}%
\pgfsetbuttcap%
\pgfsetroundjoin%
\definecolor{currentfill}{rgb}{0.926106,0.897330,0.104071}%
\pgfsetfillcolor{currentfill}%
\pgfsetfillopacity{0.700000}%
\pgfsetlinewidth{0.501875pt}%
\definecolor{currentstroke}{rgb}{1.000000,1.000000,1.000000}%
\pgfsetstrokecolor{currentstroke}%
\pgfsetstrokeopacity{0.500000}%
\pgfsetdash{}{0pt}%
\pgfpathmoveto{\pgfqpoint{3.524329in}{3.701857in}}%
\pgfpathlineto{\pgfqpoint{3.535316in}{3.707059in}}%
\pgfpathlineto{\pgfqpoint{3.546298in}{3.712236in}}%
\pgfpathlineto{\pgfqpoint{3.557276in}{3.717389in}}%
\pgfpathlineto{\pgfqpoint{3.568249in}{3.722520in}}%
\pgfpathlineto{\pgfqpoint{3.579218in}{3.727629in}}%
\pgfpathlineto{\pgfqpoint{3.572965in}{3.734069in}}%
\pgfpathlineto{\pgfqpoint{3.566713in}{3.740271in}}%
\pgfpathlineto{\pgfqpoint{3.560463in}{3.746222in}}%
\pgfpathlineto{\pgfqpoint{3.554215in}{3.751915in}}%
\pgfpathlineto{\pgfqpoint{3.547968in}{3.757345in}}%
\pgfpathlineto{\pgfqpoint{3.537012in}{3.752216in}}%
\pgfpathlineto{\pgfqpoint{3.526053in}{3.747039in}}%
\pgfpathlineto{\pgfqpoint{3.515088in}{3.741813in}}%
\pgfpathlineto{\pgfqpoint{3.504120in}{3.736537in}}%
\pgfpathlineto{\pgfqpoint{3.493146in}{3.731210in}}%
\pgfpathlineto{\pgfqpoint{3.499380in}{3.725948in}}%
\pgfpathlineto{\pgfqpoint{3.505615in}{3.720379in}}%
\pgfpathlineto{\pgfqpoint{3.511852in}{3.714500in}}%
\pgfpathlineto{\pgfqpoint{3.518090in}{3.708318in}}%
\pgfpathclose%
\pgfusepath{stroke,fill}%
\end{pgfscope}%
\begin{pgfscope}%
\pgfpathrectangle{\pgfqpoint{0.887500in}{0.275000in}}{\pgfqpoint{4.225000in}{4.225000in}}%
\pgfusepath{clip}%
\pgfsetbuttcap%
\pgfsetroundjoin%
\definecolor{currentfill}{rgb}{0.123444,0.636809,0.528763}%
\pgfsetfillcolor{currentfill}%
\pgfsetfillopacity{0.700000}%
\pgfsetlinewidth{0.501875pt}%
\definecolor{currentstroke}{rgb}{1.000000,1.000000,1.000000}%
\pgfsetstrokecolor{currentstroke}%
\pgfsetstrokeopacity{0.500000}%
\pgfsetdash{}{0pt}%
\pgfpathmoveto{\pgfqpoint{2.124064in}{2.915125in}}%
\pgfpathlineto{\pgfqpoint{2.135308in}{2.920432in}}%
\pgfpathlineto{\pgfqpoint{2.146537in}{2.926235in}}%
\pgfpathlineto{\pgfqpoint{2.157749in}{2.932637in}}%
\pgfpathlineto{\pgfqpoint{2.168941in}{2.939738in}}%
\pgfpathlineto{\pgfqpoint{2.180115in}{2.947539in}}%
\pgfpathlineto{\pgfqpoint{2.174272in}{2.958357in}}%
\pgfpathlineto{\pgfqpoint{2.168430in}{2.969219in}}%
\pgfpathlineto{\pgfqpoint{2.162592in}{2.980079in}}%
\pgfpathlineto{\pgfqpoint{2.156757in}{2.990895in}}%
\pgfpathlineto{\pgfqpoint{2.150927in}{3.001621in}}%
\pgfpathlineto{\pgfqpoint{2.139756in}{2.994168in}}%
\pgfpathlineto{\pgfqpoint{2.128584in}{2.986623in}}%
\pgfpathlineto{\pgfqpoint{2.117411in}{2.979035in}}%
\pgfpathlineto{\pgfqpoint{2.106233in}{2.971499in}}%
\pgfpathlineto{\pgfqpoint{2.095049in}{2.964109in}}%
\pgfpathlineto{\pgfqpoint{2.100843in}{2.954392in}}%
\pgfpathlineto{\pgfqpoint{2.106643in}{2.944604in}}%
\pgfpathlineto{\pgfqpoint{2.112447in}{2.934777in}}%
\pgfpathlineto{\pgfqpoint{2.118254in}{2.924940in}}%
\pgfpathclose%
\pgfusepath{stroke,fill}%
\end{pgfscope}%
\begin{pgfscope}%
\pgfpathrectangle{\pgfqpoint{0.887500in}{0.275000in}}{\pgfqpoint{4.225000in}{4.225000in}}%
\pgfusepath{clip}%
\pgfsetbuttcap%
\pgfsetroundjoin%
\definecolor{currentfill}{rgb}{0.162016,0.687316,0.499129}%
\pgfsetfillcolor{currentfill}%
\pgfsetfillopacity{0.700000}%
\pgfsetlinewidth{0.501875pt}%
\definecolor{currentstroke}{rgb}{1.000000,1.000000,1.000000}%
\pgfsetstrokecolor{currentstroke}%
\pgfsetstrokeopacity{0.500000}%
\pgfsetdash{}{0pt}%
\pgfpathmoveto{\pgfqpoint{2.461451in}{3.034392in}}%
\pgfpathlineto{\pgfqpoint{2.472662in}{3.037800in}}%
\pgfpathlineto{\pgfqpoint{2.483875in}{3.040735in}}%
\pgfpathlineto{\pgfqpoint{2.495084in}{3.043598in}}%
\pgfpathlineto{\pgfqpoint{2.506282in}{3.046787in}}%
\pgfpathlineto{\pgfqpoint{2.517466in}{3.050702in}}%
\pgfpathlineto{\pgfqpoint{2.511555in}{3.060206in}}%
\pgfpathlineto{\pgfqpoint{2.505651in}{3.069430in}}%
\pgfpathlineto{\pgfqpoint{2.499756in}{3.078235in}}%
\pgfpathlineto{\pgfqpoint{2.493873in}{3.086513in}}%
\pgfpathlineto{\pgfqpoint{2.488000in}{3.094297in}}%
\pgfpathlineto{\pgfqpoint{2.476802in}{3.092132in}}%
\pgfpathlineto{\pgfqpoint{2.465593in}{3.090326in}}%
\pgfpathlineto{\pgfqpoint{2.454379in}{3.088467in}}%
\pgfpathlineto{\pgfqpoint{2.443165in}{3.086143in}}%
\pgfpathlineto{\pgfqpoint{2.431960in}{3.082942in}}%
\pgfpathlineto{\pgfqpoint{2.437827in}{3.074790in}}%
\pgfpathlineto{\pgfqpoint{2.443711in}{3.065821in}}%
\pgfpathlineto{\pgfqpoint{2.449612in}{3.055985in}}%
\pgfpathlineto{\pgfqpoint{2.455526in}{3.045441in}}%
\pgfpathclose%
\pgfusepath{stroke,fill}%
\end{pgfscope}%
\begin{pgfscope}%
\pgfpathrectangle{\pgfqpoint{0.887500in}{0.275000in}}{\pgfqpoint{4.225000in}{4.225000in}}%
\pgfusepath{clip}%
\pgfsetbuttcap%
\pgfsetroundjoin%
\definecolor{currentfill}{rgb}{0.121380,0.629492,0.531973}%
\pgfsetfillcolor{currentfill}%
\pgfsetfillopacity{0.700000}%
\pgfsetlinewidth{0.501875pt}%
\definecolor{currentstroke}{rgb}{1.000000,1.000000,1.000000}%
\pgfsetstrokecolor{currentstroke}%
\pgfsetstrokeopacity{0.500000}%
\pgfsetdash{}{0pt}%
\pgfpathmoveto{\pgfqpoint{1.982480in}{2.910974in}}%
\pgfpathlineto{\pgfqpoint{1.993796in}{2.914775in}}%
\pgfpathlineto{\pgfqpoint{2.005102in}{2.918767in}}%
\pgfpathlineto{\pgfqpoint{2.016397in}{2.923012in}}%
\pgfpathlineto{\pgfqpoint{2.027678in}{2.927573in}}%
\pgfpathlineto{\pgfqpoint{2.038945in}{2.932509in}}%
\pgfpathlineto{\pgfqpoint{2.033210in}{2.940452in}}%
\pgfpathlineto{\pgfqpoint{2.027480in}{2.948369in}}%
\pgfpathlineto{\pgfqpoint{2.021754in}{2.956254in}}%
\pgfpathlineto{\pgfqpoint{2.016032in}{2.964104in}}%
\pgfpathlineto{\pgfqpoint{2.010315in}{2.971915in}}%
\pgfpathlineto{\pgfqpoint{1.999076in}{2.966397in}}%
\pgfpathlineto{\pgfqpoint{1.987818in}{2.961477in}}%
\pgfpathlineto{\pgfqpoint{1.976541in}{2.957039in}}%
\pgfpathlineto{\pgfqpoint{1.965251in}{2.952966in}}%
\pgfpathlineto{\pgfqpoint{1.953948in}{2.949139in}}%
\pgfpathlineto{\pgfqpoint{1.959646in}{2.941527in}}%
\pgfpathlineto{\pgfqpoint{1.965349in}{2.933906in}}%
\pgfpathlineto{\pgfqpoint{1.971055in}{2.926273in}}%
\pgfpathlineto{\pgfqpoint{1.976766in}{2.918629in}}%
\pgfpathclose%
\pgfusepath{stroke,fill}%
\end{pgfscope}%
\begin{pgfscope}%
\pgfpathrectangle{\pgfqpoint{0.887500in}{0.275000in}}{\pgfqpoint{4.225000in}{4.225000in}}%
\pgfusepath{clip}%
\pgfsetbuttcap%
\pgfsetroundjoin%
\definecolor{currentfill}{rgb}{0.804182,0.882046,0.114965}%
\pgfsetfillcolor{currentfill}%
\pgfsetfillopacity{0.700000}%
\pgfsetlinewidth{0.501875pt}%
\definecolor{currentstroke}{rgb}{1.000000,1.000000,1.000000}%
\pgfsetstrokecolor{currentstroke}%
\pgfsetstrokeopacity{0.500000}%
\pgfsetdash{}{0pt}%
\pgfpathmoveto{\pgfqpoint{3.186767in}{3.613828in}}%
\pgfpathlineto{\pgfqpoint{3.197820in}{3.618669in}}%
\pgfpathlineto{\pgfqpoint{3.208869in}{3.623402in}}%
\pgfpathlineto{\pgfqpoint{3.219914in}{3.628075in}}%
\pgfpathlineto{\pgfqpoint{3.230954in}{3.632744in}}%
\pgfpathlineto{\pgfqpoint{3.241989in}{3.637470in}}%
\pgfpathlineto{\pgfqpoint{3.235837in}{3.640124in}}%
\pgfpathlineto{\pgfqpoint{3.229689in}{3.642515in}}%
\pgfpathlineto{\pgfqpoint{3.223546in}{3.644637in}}%
\pgfpathlineto{\pgfqpoint{3.217407in}{3.646502in}}%
\pgfpathlineto{\pgfqpoint{3.211273in}{3.648141in}}%
\pgfpathlineto{\pgfqpoint{3.200256in}{3.643168in}}%
\pgfpathlineto{\pgfqpoint{3.189234in}{3.638213in}}%
\pgfpathlineto{\pgfqpoint{3.178209in}{3.633208in}}%
\pgfpathlineto{\pgfqpoint{3.167178in}{3.628089in}}%
\pgfpathlineto{\pgfqpoint{3.156144in}{3.622803in}}%
\pgfpathlineto{\pgfqpoint{3.162258in}{3.621512in}}%
\pgfpathlineto{\pgfqpoint{3.168378in}{3.619992in}}%
\pgfpathlineto{\pgfqpoint{3.174502in}{3.618205in}}%
\pgfpathlineto{\pgfqpoint{3.180632in}{3.616146in}}%
\pgfpathclose%
\pgfusepath{stroke,fill}%
\end{pgfscope}%
\begin{pgfscope}%
\pgfpathrectangle{\pgfqpoint{0.887500in}{0.275000in}}{\pgfqpoint{4.225000in}{4.225000in}}%
\pgfusepath{clip}%
\pgfsetbuttcap%
\pgfsetroundjoin%
\definecolor{currentfill}{rgb}{0.134692,0.658636,0.517649}%
\pgfsetfillcolor{currentfill}%
\pgfsetfillopacity{0.700000}%
\pgfsetlinewidth{0.501875pt}%
\definecolor{currentstroke}{rgb}{1.000000,1.000000,1.000000}%
\pgfsetstrokecolor{currentstroke}%
\pgfsetstrokeopacity{0.500000}%
\pgfsetdash{}{0pt}%
\pgfpathmoveto{\pgfqpoint{2.264956in}{2.947635in}}%
\pgfpathlineto{\pgfqpoint{2.276071in}{2.958363in}}%
\pgfpathlineto{\pgfqpoint{2.287192in}{2.968742in}}%
\pgfpathlineto{\pgfqpoint{2.298317in}{2.978847in}}%
\pgfpathlineto{\pgfqpoint{2.309443in}{2.988760in}}%
\pgfpathlineto{\pgfqpoint{2.320570in}{2.998560in}}%
\pgfpathlineto{\pgfqpoint{2.314745in}{3.006243in}}%
\pgfpathlineto{\pgfqpoint{2.308925in}{3.013822in}}%
\pgfpathlineto{\pgfqpoint{2.303111in}{3.021321in}}%
\pgfpathlineto{\pgfqpoint{2.297301in}{3.028763in}}%
\pgfpathlineto{\pgfqpoint{2.291496in}{3.036170in}}%
\pgfpathlineto{\pgfqpoint{2.280365in}{3.027488in}}%
\pgfpathlineto{\pgfqpoint{2.269230in}{3.018811in}}%
\pgfpathlineto{\pgfqpoint{2.258095in}{3.010075in}}%
\pgfpathlineto{\pgfqpoint{2.246960in}{3.001215in}}%
\pgfpathlineto{\pgfqpoint{2.235827in}{2.992168in}}%
\pgfpathlineto{\pgfqpoint{2.241649in}{2.983120in}}%
\pgfpathlineto{\pgfqpoint{2.247473in}{2.974129in}}%
\pgfpathlineto{\pgfqpoint{2.253299in}{2.965208in}}%
\pgfpathlineto{\pgfqpoint{2.259127in}{2.956373in}}%
\pgfpathclose%
\pgfusepath{stroke,fill}%
\end{pgfscope}%
\begin{pgfscope}%
\pgfpathrectangle{\pgfqpoint{0.887500in}{0.275000in}}{\pgfqpoint{4.225000in}{4.225000in}}%
\pgfusepath{clip}%
\pgfsetbuttcap%
\pgfsetroundjoin%
\definecolor{currentfill}{rgb}{0.945636,0.899815,0.112838}%
\pgfsetfillcolor{currentfill}%
\pgfsetfillopacity{0.700000}%
\pgfsetlinewidth{0.501875pt}%
\definecolor{currentstroke}{rgb}{1.000000,1.000000,1.000000}%
\pgfsetstrokecolor{currentstroke}%
\pgfsetstrokeopacity{0.500000}%
\pgfsetdash{}{0pt}%
\pgfpathmoveto{\pgfqpoint{3.751533in}{3.702478in}}%
\pgfpathlineto{\pgfqpoint{3.762475in}{3.707281in}}%
\pgfpathlineto{\pgfqpoint{3.773413in}{3.712048in}}%
\pgfpathlineto{\pgfqpoint{3.784346in}{3.716784in}}%
\pgfpathlineto{\pgfqpoint{3.795274in}{3.721492in}}%
\pgfpathlineto{\pgfqpoint{3.806197in}{3.726174in}}%
\pgfpathlineto{\pgfqpoint{3.799894in}{3.734454in}}%
\pgfpathlineto{\pgfqpoint{3.793590in}{3.742511in}}%
\pgfpathlineto{\pgfqpoint{3.787287in}{3.750352in}}%
\pgfpathlineto{\pgfqpoint{3.780984in}{3.757985in}}%
\pgfpathlineto{\pgfqpoint{3.774682in}{3.765418in}}%
\pgfpathlineto{\pgfqpoint{3.763771in}{3.760810in}}%
\pgfpathlineto{\pgfqpoint{3.752856in}{3.756166in}}%
\pgfpathlineto{\pgfqpoint{3.741935in}{3.751485in}}%
\pgfpathlineto{\pgfqpoint{3.731009in}{3.746766in}}%
\pgfpathlineto{\pgfqpoint{3.720079in}{3.742007in}}%
\pgfpathlineto{\pgfqpoint{3.726367in}{3.734479in}}%
\pgfpathlineto{\pgfqpoint{3.732657in}{3.726762in}}%
\pgfpathlineto{\pgfqpoint{3.738948in}{3.718857in}}%
\pgfpathlineto{\pgfqpoint{3.745240in}{3.710763in}}%
\pgfpathclose%
\pgfusepath{stroke,fill}%
\end{pgfscope}%
\begin{pgfscope}%
\pgfpathrectangle{\pgfqpoint{0.887500in}{0.275000in}}{\pgfqpoint{4.225000in}{4.225000in}}%
\pgfusepath{clip}%
\pgfsetbuttcap%
\pgfsetroundjoin%
\definecolor{currentfill}{rgb}{0.709898,0.868751,0.169257}%
\pgfsetfillcolor{currentfill}%
\pgfsetfillopacity{0.700000}%
\pgfsetlinewidth{0.501875pt}%
\definecolor{currentstroke}{rgb}{1.000000,1.000000,1.000000}%
\pgfsetstrokecolor{currentstroke}%
\pgfsetstrokeopacity{0.500000}%
\pgfsetdash{}{0pt}%
\pgfpathmoveto{\pgfqpoint{2.990131in}{3.545015in}}%
\pgfpathlineto{\pgfqpoint{3.001230in}{3.548782in}}%
\pgfpathlineto{\pgfqpoint{3.012323in}{3.552815in}}%
\pgfpathlineto{\pgfqpoint{3.023411in}{3.557200in}}%
\pgfpathlineto{\pgfqpoint{3.034494in}{3.561930in}}%
\pgfpathlineto{\pgfqpoint{3.045573in}{3.566955in}}%
\pgfpathlineto{\pgfqpoint{3.039508in}{3.567564in}}%
\pgfpathlineto{\pgfqpoint{3.033448in}{3.568230in}}%
\pgfpathlineto{\pgfqpoint{3.027395in}{3.568948in}}%
\pgfpathlineto{\pgfqpoint{3.021348in}{3.569717in}}%
\pgfpathlineto{\pgfqpoint{3.015307in}{3.570535in}}%
\pgfpathlineto{\pgfqpoint{3.004250in}{3.565253in}}%
\pgfpathlineto{\pgfqpoint{2.993189in}{3.560307in}}%
\pgfpathlineto{\pgfqpoint{2.982122in}{3.555778in}}%
\pgfpathlineto{\pgfqpoint{2.971051in}{3.551642in}}%
\pgfpathlineto{\pgfqpoint{2.959974in}{3.547667in}}%
\pgfpathlineto{\pgfqpoint{2.965993in}{3.547118in}}%
\pgfpathlineto{\pgfqpoint{2.972019in}{3.546594in}}%
\pgfpathlineto{\pgfqpoint{2.978050in}{3.546079in}}%
\pgfpathlineto{\pgfqpoint{2.984088in}{3.545558in}}%
\pgfpathclose%
\pgfusepath{stroke,fill}%
\end{pgfscope}%
\begin{pgfscope}%
\pgfpathrectangle{\pgfqpoint{0.887500in}{0.275000in}}{\pgfqpoint{4.225000in}{4.225000in}}%
\pgfusepath{clip}%
\pgfsetbuttcap%
\pgfsetroundjoin%
\definecolor{currentfill}{rgb}{0.157851,0.683765,0.501686}%
\pgfsetfillcolor{currentfill}%
\pgfsetfillopacity{0.700000}%
\pgfsetlinewidth{0.501875pt}%
\definecolor{currentstroke}{rgb}{1.000000,1.000000,1.000000}%
\pgfsetstrokecolor{currentstroke}%
\pgfsetstrokeopacity{0.500000}%
\pgfsetdash{}{0pt}%
\pgfpathmoveto{\pgfqpoint{2.547062in}{3.003820in}}%
\pgfpathlineto{\pgfqpoint{2.558202in}{3.011623in}}%
\pgfpathlineto{\pgfqpoint{2.569319in}{3.021052in}}%
\pgfpathlineto{\pgfqpoint{2.580415in}{3.032181in}}%
\pgfpathlineto{\pgfqpoint{2.591489in}{3.045022in}}%
\pgfpathlineto{\pgfqpoint{2.602544in}{3.059589in}}%
\pgfpathlineto{\pgfqpoint{2.596636in}{3.066054in}}%
\pgfpathlineto{\pgfqpoint{2.590734in}{3.072430in}}%
\pgfpathlineto{\pgfqpoint{2.584837in}{3.078714in}}%
\pgfpathlineto{\pgfqpoint{2.578945in}{3.084904in}}%
\pgfpathlineto{\pgfqpoint{2.573059in}{3.090995in}}%
\pgfpathlineto{\pgfqpoint{2.561983in}{3.079886in}}%
\pgfpathlineto{\pgfqpoint{2.550887in}{3.070302in}}%
\pgfpathlineto{\pgfqpoint{2.539770in}{3.062251in}}%
\pgfpathlineto{\pgfqpoint{2.528629in}{3.055742in}}%
\pgfpathlineto{\pgfqpoint{2.517466in}{3.050702in}}%
\pgfpathlineto{\pgfqpoint{2.523382in}{3.041057in}}%
\pgfpathlineto{\pgfqpoint{2.529301in}{3.031407in}}%
\pgfpathlineto{\pgfqpoint{2.535222in}{3.021892in}}%
\pgfpathlineto{\pgfqpoint{2.541143in}{3.012650in}}%
\pgfpathclose%
\pgfusepath{stroke,fill}%
\end{pgfscope}%
\begin{pgfscope}%
\pgfpathrectangle{\pgfqpoint{0.887500in}{0.275000in}}{\pgfqpoint{4.225000in}{4.225000in}}%
\pgfusepath{clip}%
\pgfsetbuttcap%
\pgfsetroundjoin%
\definecolor{currentfill}{rgb}{0.855810,0.888601,0.097452}%
\pgfsetfillcolor{currentfill}%
\pgfsetfillopacity{0.700000}%
\pgfsetlinewidth{0.501875pt}%
\definecolor{currentstroke}{rgb}{1.000000,1.000000,1.000000}%
\pgfsetstrokecolor{currentstroke}%
\pgfsetstrokeopacity{0.500000}%
\pgfsetdash{}{0pt}%
\pgfpathmoveto{\pgfqpoint{3.328031in}{3.646079in}}%
\pgfpathlineto{\pgfqpoint{3.339065in}{3.651762in}}%
\pgfpathlineto{\pgfqpoint{3.350097in}{3.657551in}}%
\pgfpathlineto{\pgfqpoint{3.361125in}{3.663405in}}%
\pgfpathlineto{\pgfqpoint{3.372149in}{3.669284in}}%
\pgfpathlineto{\pgfqpoint{3.383170in}{3.675149in}}%
\pgfpathlineto{\pgfqpoint{3.376966in}{3.679549in}}%
\pgfpathlineto{\pgfqpoint{3.370764in}{3.683630in}}%
\pgfpathlineto{\pgfqpoint{3.364565in}{3.687404in}}%
\pgfpathlineto{\pgfqpoint{3.358369in}{3.690883in}}%
\pgfpathlineto{\pgfqpoint{3.352176in}{3.694082in}}%
\pgfpathlineto{\pgfqpoint{3.341172in}{3.687972in}}%
\pgfpathlineto{\pgfqpoint{3.330163in}{3.681840in}}%
\pgfpathlineto{\pgfqpoint{3.319152in}{3.675736in}}%
\pgfpathlineto{\pgfqpoint{3.308137in}{3.669707in}}%
\pgfpathlineto{\pgfqpoint{3.297119in}{3.663804in}}%
\pgfpathlineto{\pgfqpoint{3.303294in}{3.660807in}}%
\pgfpathlineto{\pgfqpoint{3.309473in}{3.657549in}}%
\pgfpathlineto{\pgfqpoint{3.315655in}{3.654018in}}%
\pgfpathlineto{\pgfqpoint{3.321842in}{3.650199in}}%
\pgfpathclose%
\pgfusepath{stroke,fill}%
\end{pgfscope}%
\begin{pgfscope}%
\pgfpathrectangle{\pgfqpoint{0.887500in}{0.275000in}}{\pgfqpoint{4.225000in}{4.225000in}}%
\pgfusepath{clip}%
\pgfsetbuttcap%
\pgfsetroundjoin%
\definecolor{currentfill}{rgb}{0.202219,0.715272,0.476084}%
\pgfsetfillcolor{currentfill}%
\pgfsetfillopacity{0.700000}%
\pgfsetlinewidth{0.501875pt}%
\definecolor{currentstroke}{rgb}{1.000000,1.000000,1.000000}%
\pgfsetstrokecolor{currentstroke}%
\pgfsetstrokeopacity{0.500000}%
\pgfsetdash{}{0pt}%
\pgfpathmoveto{\pgfqpoint{2.602544in}{3.059589in}}%
\pgfpathlineto{\pgfqpoint{2.613581in}{3.075895in}}%
\pgfpathlineto{\pgfqpoint{2.624601in}{3.093955in}}%
\pgfpathlineto{\pgfqpoint{2.635605in}{3.113781in}}%
\pgfpathlineto{\pgfqpoint{2.646596in}{3.135308in}}%
\pgfpathlineto{\pgfqpoint{2.657579in}{3.158198in}}%
\pgfpathlineto{\pgfqpoint{2.651667in}{3.162594in}}%
\pgfpathlineto{\pgfqpoint{2.645773in}{3.165587in}}%
\pgfpathlineto{\pgfqpoint{2.639898in}{3.167421in}}%
\pgfpathlineto{\pgfqpoint{2.634037in}{3.168338in}}%
\pgfpathlineto{\pgfqpoint{2.628189in}{3.168580in}}%
\pgfpathlineto{\pgfqpoint{2.617189in}{3.150430in}}%
\pgfpathlineto{\pgfqpoint{2.606179in}{3.133380in}}%
\pgfpathlineto{\pgfqpoint{2.595156in}{3.117750in}}%
\pgfpathlineto{\pgfqpoint{2.584116in}{3.103619in}}%
\pgfpathlineto{\pgfqpoint{2.573059in}{3.090995in}}%
\pgfpathlineto{\pgfqpoint{2.578945in}{3.084904in}}%
\pgfpathlineto{\pgfqpoint{2.584837in}{3.078714in}}%
\pgfpathlineto{\pgfqpoint{2.590734in}{3.072430in}}%
\pgfpathlineto{\pgfqpoint{2.596636in}{3.066054in}}%
\pgfpathclose%
\pgfusepath{stroke,fill}%
\end{pgfscope}%
\begin{pgfscope}%
\pgfpathrectangle{\pgfqpoint{0.887500in}{0.275000in}}{\pgfqpoint{4.225000in}{4.225000in}}%
\pgfusepath{clip}%
\pgfsetbuttcap%
\pgfsetroundjoin%
\definecolor{currentfill}{rgb}{0.123444,0.636809,0.528763}%
\pgfsetfillcolor{currentfill}%
\pgfsetfillopacity{0.700000}%
\pgfsetlinewidth{0.501875pt}%
\definecolor{currentstroke}{rgb}{1.000000,1.000000,1.000000}%
\pgfsetstrokecolor{currentstroke}%
\pgfsetstrokeopacity{0.500000}%
\pgfsetdash{}{0pt}%
\pgfpathmoveto{\pgfqpoint{2.209328in}{2.895652in}}%
\pgfpathlineto{\pgfqpoint{2.220480in}{2.904908in}}%
\pgfpathlineto{\pgfqpoint{2.231613in}{2.915012in}}%
\pgfpathlineto{\pgfqpoint{2.242733in}{2.925686in}}%
\pgfpathlineto{\pgfqpoint{2.253845in}{2.936653in}}%
\pgfpathlineto{\pgfqpoint{2.264956in}{2.947635in}}%
\pgfpathlineto{\pgfqpoint{2.259127in}{2.956373in}}%
\pgfpathlineto{\pgfqpoint{2.253299in}{2.965208in}}%
\pgfpathlineto{\pgfqpoint{2.247473in}{2.974129in}}%
\pgfpathlineto{\pgfqpoint{2.241649in}{2.983120in}}%
\pgfpathlineto{\pgfqpoint{2.235827in}{2.992168in}}%
\pgfpathlineto{\pgfqpoint{2.224694in}{2.982973in}}%
\pgfpathlineto{\pgfqpoint{2.213560in}{2.973769in}}%
\pgfpathlineto{\pgfqpoint{2.202421in}{2.964699in}}%
\pgfpathlineto{\pgfqpoint{2.191274in}{2.955908in}}%
\pgfpathlineto{\pgfqpoint{2.180115in}{2.947539in}}%
\pgfpathlineto{\pgfqpoint{2.185959in}{2.936809in}}%
\pgfpathlineto{\pgfqpoint{2.191804in}{2.926211in}}%
\pgfpathlineto{\pgfqpoint{2.197647in}{2.915789in}}%
\pgfpathlineto{\pgfqpoint{2.203489in}{2.905588in}}%
\pgfpathclose%
\pgfusepath{stroke,fill}%
\end{pgfscope}%
\begin{pgfscope}%
\pgfpathrectangle{\pgfqpoint{0.887500in}{0.275000in}}{\pgfqpoint{4.225000in}{4.225000in}}%
\pgfusepath{clip}%
\pgfsetbuttcap%
\pgfsetroundjoin%
\definecolor{currentfill}{rgb}{0.935904,0.898570,0.108131}%
\pgfsetfillcolor{currentfill}%
\pgfsetfillopacity{0.700000}%
\pgfsetlinewidth{0.501875pt}%
\definecolor{currentstroke}{rgb}{1.000000,1.000000,1.000000}%
\pgfsetstrokecolor{currentstroke}%
\pgfsetstrokeopacity{0.500000}%
\pgfsetdash{}{0pt}%
\pgfpathmoveto{\pgfqpoint{3.610508in}{3.692218in}}%
\pgfpathlineto{\pgfqpoint{3.621486in}{3.697349in}}%
\pgfpathlineto{\pgfqpoint{3.632460in}{3.702458in}}%
\pgfpathlineto{\pgfqpoint{3.643429in}{3.707539in}}%
\pgfpathlineto{\pgfqpoint{3.654393in}{3.712587in}}%
\pgfpathlineto{\pgfqpoint{3.665353in}{3.717597in}}%
\pgfpathlineto{\pgfqpoint{3.659078in}{3.725016in}}%
\pgfpathlineto{\pgfqpoint{3.652805in}{3.732244in}}%
\pgfpathlineto{\pgfqpoint{3.646534in}{3.739274in}}%
\pgfpathlineto{\pgfqpoint{3.640264in}{3.746100in}}%
\pgfpathlineto{\pgfqpoint{3.633995in}{3.752716in}}%
\pgfpathlineto{\pgfqpoint{3.623049in}{3.747771in}}%
\pgfpathlineto{\pgfqpoint{3.612098in}{3.742786in}}%
\pgfpathlineto{\pgfqpoint{3.601143in}{3.737765in}}%
\pgfpathlineto{\pgfqpoint{3.590183in}{3.732712in}}%
\pgfpathlineto{\pgfqpoint{3.579218in}{3.727629in}}%
\pgfpathlineto{\pgfqpoint{3.585473in}{3.720959in}}%
\pgfpathlineto{\pgfqpoint{3.591729in}{3.714073in}}%
\pgfpathlineto{\pgfqpoint{3.597987in}{3.706980in}}%
\pgfpathlineto{\pgfqpoint{3.604247in}{3.699691in}}%
\pgfpathclose%
\pgfusepath{stroke,fill}%
\end{pgfscope}%
\begin{pgfscope}%
\pgfpathrectangle{\pgfqpoint{0.887500in}{0.275000in}}{\pgfqpoint{4.225000in}{4.225000in}}%
\pgfusepath{clip}%
\pgfsetbuttcap%
\pgfsetroundjoin%
\definecolor{currentfill}{rgb}{0.120638,0.625828,0.533488}%
\pgfsetfillcolor{currentfill}%
\pgfsetfillopacity{0.700000}%
\pgfsetlinewidth{0.501875pt}%
\definecolor{currentstroke}{rgb}{1.000000,1.000000,1.000000}%
\pgfsetstrokecolor{currentstroke}%
\pgfsetstrokeopacity{0.500000}%
\pgfsetdash{}{0pt}%
\pgfpathmoveto{\pgfqpoint{2.067681in}{2.892528in}}%
\pgfpathlineto{\pgfqpoint{2.078973in}{2.896804in}}%
\pgfpathlineto{\pgfqpoint{2.090258in}{2.901149in}}%
\pgfpathlineto{\pgfqpoint{2.101537in}{2.905586in}}%
\pgfpathlineto{\pgfqpoint{2.112806in}{2.910211in}}%
\pgfpathlineto{\pgfqpoint{2.124064in}{2.915125in}}%
\pgfpathlineto{\pgfqpoint{2.118254in}{2.924940in}}%
\pgfpathlineto{\pgfqpoint{2.112447in}{2.934777in}}%
\pgfpathlineto{\pgfqpoint{2.106643in}{2.944604in}}%
\pgfpathlineto{\pgfqpoint{2.100843in}{2.954392in}}%
\pgfpathlineto{\pgfqpoint{2.095049in}{2.964109in}}%
\pgfpathlineto{\pgfqpoint{2.083855in}{2.956959in}}%
\pgfpathlineto{\pgfqpoint{2.072651in}{2.950144in}}%
\pgfpathlineto{\pgfqpoint{2.061432in}{2.943758in}}%
\pgfpathlineto{\pgfqpoint{2.050197in}{2.937884in}}%
\pgfpathlineto{\pgfqpoint{2.038945in}{2.932509in}}%
\pgfpathlineto{\pgfqpoint{2.044684in}{2.924543in}}%
\pgfpathlineto{\pgfqpoint{2.050428in}{2.916557in}}%
\pgfpathlineto{\pgfqpoint{2.056175in}{2.908557in}}%
\pgfpathlineto{\pgfqpoint{2.061926in}{2.900546in}}%
\pgfpathclose%
\pgfusepath{stroke,fill}%
\end{pgfscope}%
\begin{pgfscope}%
\pgfpathrectangle{\pgfqpoint{0.887500in}{0.275000in}}{\pgfqpoint{4.225000in}{4.225000in}}%
\pgfusepath{clip}%
\pgfsetbuttcap%
\pgfsetroundjoin%
\definecolor{currentfill}{rgb}{0.906311,0.894855,0.098125}%
\pgfsetfillcolor{currentfill}%
\pgfsetfillopacity{0.700000}%
\pgfsetlinewidth{0.501875pt}%
\definecolor{currentstroke}{rgb}{1.000000,1.000000,1.000000}%
\pgfsetstrokecolor{currentstroke}%
\pgfsetstrokeopacity{0.500000}%
\pgfsetdash{}{0pt}%
\pgfpathmoveto{\pgfqpoint{3.469331in}{3.675372in}}%
\pgfpathlineto{\pgfqpoint{3.480339in}{3.680739in}}%
\pgfpathlineto{\pgfqpoint{3.491344in}{3.686069in}}%
\pgfpathlineto{\pgfqpoint{3.502343in}{3.691364in}}%
\pgfpathlineto{\pgfqpoint{3.513338in}{3.696626in}}%
\pgfpathlineto{\pgfqpoint{3.524329in}{3.701857in}}%
\pgfpathlineto{\pgfqpoint{3.518090in}{3.708318in}}%
\pgfpathlineto{\pgfqpoint{3.511852in}{3.714500in}}%
\pgfpathlineto{\pgfqpoint{3.505615in}{3.720379in}}%
\pgfpathlineto{\pgfqpoint{3.499380in}{3.725948in}}%
\pgfpathlineto{\pgfqpoint{3.493146in}{3.731210in}}%
\pgfpathlineto{\pgfqpoint{3.482168in}{3.725833in}}%
\pgfpathlineto{\pgfqpoint{3.471186in}{3.720404in}}%
\pgfpathlineto{\pgfqpoint{3.460199in}{3.714923in}}%
\pgfpathlineto{\pgfqpoint{3.449208in}{3.709389in}}%
\pgfpathlineto{\pgfqpoint{3.438212in}{3.703804in}}%
\pgfpathlineto{\pgfqpoint{3.444433in}{3.698811in}}%
\pgfpathlineto{\pgfqpoint{3.450656in}{3.693472in}}%
\pgfpathlineto{\pgfqpoint{3.456879in}{3.687777in}}%
\pgfpathlineto{\pgfqpoint{3.463105in}{3.681731in}}%
\pgfpathclose%
\pgfusepath{stroke,fill}%
\end{pgfscope}%
\begin{pgfscope}%
\pgfpathrectangle{\pgfqpoint{0.887500in}{0.275000in}}{\pgfqpoint{4.225000in}{4.225000in}}%
\pgfusepath{clip}%
\pgfsetbuttcap%
\pgfsetroundjoin%
\definecolor{currentfill}{rgb}{0.122312,0.633153,0.530398}%
\pgfsetfillcolor{currentfill}%
\pgfsetfillopacity{0.700000}%
\pgfsetlinewidth{0.501875pt}%
\definecolor{currentstroke}{rgb}{1.000000,1.000000,1.000000}%
\pgfsetstrokecolor{currentstroke}%
\pgfsetstrokeopacity{0.500000}%
\pgfsetdash{}{0pt}%
\pgfpathmoveto{\pgfqpoint{1.840538in}{2.914279in}}%
\pgfpathlineto{\pgfqpoint{1.851903in}{2.917754in}}%
\pgfpathlineto{\pgfqpoint{1.863264in}{2.921214in}}%
\pgfpathlineto{\pgfqpoint{1.874619in}{2.924653in}}%
\pgfpathlineto{\pgfqpoint{1.885970in}{2.928068in}}%
\pgfpathlineto{\pgfqpoint{1.897316in}{2.931476in}}%
\pgfpathlineto{\pgfqpoint{1.891636in}{2.939041in}}%
\pgfpathlineto{\pgfqpoint{1.885959in}{2.946594in}}%
\pgfpathlineto{\pgfqpoint{1.880287in}{2.954134in}}%
\pgfpathlineto{\pgfqpoint{1.874618in}{2.961661in}}%
\pgfpathlineto{\pgfqpoint{1.863284in}{2.958216in}}%
\pgfpathlineto{\pgfqpoint{1.851944in}{2.954760in}}%
\pgfpathlineto{\pgfqpoint{1.840600in}{2.951278in}}%
\pgfpathlineto{\pgfqpoint{1.829251in}{2.947774in}}%
\pgfpathlineto{\pgfqpoint{1.817898in}{2.944255in}}%
\pgfpathlineto{\pgfqpoint{1.823552in}{2.936785in}}%
\pgfpathlineto{\pgfqpoint{1.829210in}{2.929298in}}%
\pgfpathlineto{\pgfqpoint{1.834872in}{2.921796in}}%
\pgfpathclose%
\pgfusepath{stroke,fill}%
\end{pgfscope}%
\begin{pgfscope}%
\pgfpathrectangle{\pgfqpoint{0.887500in}{0.275000in}}{\pgfqpoint{4.225000in}{4.225000in}}%
\pgfusepath{clip}%
\pgfsetbuttcap%
\pgfsetroundjoin%
\definecolor{currentfill}{rgb}{0.157851,0.683765,0.501686}%
\pgfsetfillcolor{currentfill}%
\pgfsetfillopacity{0.700000}%
\pgfsetlinewidth{0.501875pt}%
\definecolor{currentstroke}{rgb}{1.000000,1.000000,1.000000}%
\pgfsetstrokecolor{currentstroke}%
\pgfsetstrokeopacity{0.500000}%
\pgfsetdash{}{0pt}%
\pgfpathmoveto{\pgfqpoint{2.405545in}{3.003140in}}%
\pgfpathlineto{\pgfqpoint{2.416706in}{3.011081in}}%
\pgfpathlineto{\pgfqpoint{2.427876in}{3.018326in}}%
\pgfpathlineto{\pgfqpoint{2.439056in}{3.024721in}}%
\pgfpathlineto{\pgfqpoint{2.450247in}{3.030112in}}%
\pgfpathlineto{\pgfqpoint{2.461451in}{3.034392in}}%
\pgfpathlineto{\pgfqpoint{2.455526in}{3.045441in}}%
\pgfpathlineto{\pgfqpoint{2.449612in}{3.055985in}}%
\pgfpathlineto{\pgfqpoint{2.443711in}{3.065821in}}%
\pgfpathlineto{\pgfqpoint{2.437827in}{3.074790in}}%
\pgfpathlineto{\pgfqpoint{2.431960in}{3.082942in}}%
\pgfpathlineto{\pgfqpoint{2.420769in}{3.078452in}}%
\pgfpathlineto{\pgfqpoint{2.409598in}{3.072455in}}%
\pgfpathlineto{\pgfqpoint{2.398445in}{3.065154in}}%
\pgfpathlineto{\pgfqpoint{2.387306in}{3.056808in}}%
\pgfpathlineto{\pgfqpoint{2.376178in}{3.047676in}}%
\pgfpathlineto{\pgfqpoint{2.382026in}{3.039866in}}%
\pgfpathlineto{\pgfqpoint{2.387887in}{3.031493in}}%
\pgfpathlineto{\pgfqpoint{2.393762in}{3.022517in}}%
\pgfpathlineto{\pgfqpoint{2.399648in}{3.013027in}}%
\pgfpathclose%
\pgfusepath{stroke,fill}%
\end{pgfscope}%
\begin{pgfscope}%
\pgfpathrectangle{\pgfqpoint{0.887500in}{0.275000in}}{\pgfqpoint{4.225000in}{4.225000in}}%
\pgfusepath{clip}%
\pgfsetbuttcap%
\pgfsetroundjoin%
\definecolor{currentfill}{rgb}{0.783315,0.879285,0.125405}%
\pgfsetfillcolor{currentfill}%
\pgfsetfillopacity{0.700000}%
\pgfsetlinewidth{0.501875pt}%
\definecolor{currentstroke}{rgb}{1.000000,1.000000,1.000000}%
\pgfsetstrokecolor{currentstroke}%
\pgfsetstrokeopacity{0.500000}%
\pgfsetdash{}{0pt}%
\pgfpathmoveto{\pgfqpoint{3.131432in}{3.588810in}}%
\pgfpathlineto{\pgfqpoint{3.142508in}{3.593824in}}%
\pgfpathlineto{\pgfqpoint{3.153579in}{3.598862in}}%
\pgfpathlineto{\pgfqpoint{3.164646in}{3.603895in}}%
\pgfpathlineto{\pgfqpoint{3.175708in}{3.608893in}}%
\pgfpathlineto{\pgfqpoint{3.186767in}{3.613828in}}%
\pgfpathlineto{\pgfqpoint{3.180632in}{3.616146in}}%
\pgfpathlineto{\pgfqpoint{3.174502in}{3.618205in}}%
\pgfpathlineto{\pgfqpoint{3.168378in}{3.619992in}}%
\pgfpathlineto{\pgfqpoint{3.162258in}{3.621512in}}%
\pgfpathlineto{\pgfqpoint{3.156144in}{3.622803in}}%
\pgfpathlineto{\pgfqpoint{3.145104in}{3.617364in}}%
\pgfpathlineto{\pgfqpoint{3.134061in}{3.611803in}}%
\pgfpathlineto{\pgfqpoint{3.123014in}{3.606153in}}%
\pgfpathlineto{\pgfqpoint{3.111962in}{3.600444in}}%
\pgfpathlineto{\pgfqpoint{3.100907in}{3.594708in}}%
\pgfpathlineto{\pgfqpoint{3.107000in}{3.593697in}}%
\pgfpathlineto{\pgfqpoint{3.113100in}{3.592618in}}%
\pgfpathlineto{\pgfqpoint{3.119205in}{3.591456in}}%
\pgfpathlineto{\pgfqpoint{3.125316in}{3.590194in}}%
\pgfpathclose%
\pgfusepath{stroke,fill}%
\end{pgfscope}%
\begin{pgfscope}%
\pgfpathrectangle{\pgfqpoint{0.887500in}{0.275000in}}{\pgfqpoint{4.225000in}{4.225000in}}%
\pgfusepath{clip}%
\pgfsetbuttcap%
\pgfsetroundjoin%
\definecolor{currentfill}{rgb}{0.926106,0.897330,0.104071}%
\pgfsetfillcolor{currentfill}%
\pgfsetfillopacity{0.700000}%
\pgfsetlinewidth{0.501875pt}%
\definecolor{currentstroke}{rgb}{1.000000,1.000000,1.000000}%
\pgfsetstrokecolor{currentstroke}%
\pgfsetstrokeopacity{0.500000}%
\pgfsetdash{}{0pt}%
\pgfpathmoveto{\pgfqpoint{3.837708in}{3.681295in}}%
\pgfpathlineto{\pgfqpoint{3.848636in}{3.685922in}}%
\pgfpathlineto{\pgfqpoint{3.859559in}{3.690552in}}%
\pgfpathlineto{\pgfqpoint{3.870478in}{3.695191in}}%
\pgfpathlineto{\pgfqpoint{3.881393in}{3.699849in}}%
\pgfpathlineto{\pgfqpoint{3.875086in}{3.709338in}}%
\pgfpathlineto{\pgfqpoint{3.868777in}{3.718565in}}%
\pgfpathlineto{\pgfqpoint{3.862467in}{3.727532in}}%
\pgfpathlineto{\pgfqpoint{3.856155in}{3.736246in}}%
\pgfpathlineto{\pgfqpoint{3.849843in}{3.744710in}}%
\pgfpathlineto{\pgfqpoint{3.838939in}{3.740099in}}%
\pgfpathlineto{\pgfqpoint{3.828029in}{3.735474in}}%
\pgfpathlineto{\pgfqpoint{3.817116in}{3.730834in}}%
\pgfpathlineto{\pgfqpoint{3.806197in}{3.726174in}}%
\pgfpathlineto{\pgfqpoint{3.812500in}{3.717668in}}%
\pgfpathlineto{\pgfqpoint{3.818803in}{3.708931in}}%
\pgfpathlineto{\pgfqpoint{3.825106in}{3.699959in}}%
\pgfpathlineto{\pgfqpoint{3.831407in}{3.690749in}}%
\pgfpathclose%
\pgfusepath{stroke,fill}%
\end{pgfscope}%
\begin{pgfscope}%
\pgfpathrectangle{\pgfqpoint{0.887500in}{0.275000in}}{\pgfqpoint{4.225000in}{4.225000in}}%
\pgfusepath{clip}%
\pgfsetbuttcap%
\pgfsetroundjoin%
\definecolor{currentfill}{rgb}{0.121380,0.629492,0.531973}%
\pgfsetfillcolor{currentfill}%
\pgfsetfillopacity{0.700000}%
\pgfsetlinewidth{0.501875pt}%
\definecolor{currentstroke}{rgb}{1.000000,1.000000,1.000000}%
\pgfsetstrokecolor{currentstroke}%
\pgfsetstrokeopacity{0.500000}%
\pgfsetdash{}{0pt}%
\pgfpathmoveto{\pgfqpoint{1.925780in}{2.893462in}}%
\pgfpathlineto{\pgfqpoint{1.937134in}{2.896845in}}%
\pgfpathlineto{\pgfqpoint{1.948481in}{2.900265in}}%
\pgfpathlineto{\pgfqpoint{1.959822in}{2.903746in}}%
\pgfpathlineto{\pgfqpoint{1.971155in}{2.907308in}}%
\pgfpathlineto{\pgfqpoint{1.982480in}{2.910974in}}%
\pgfpathlineto{\pgfqpoint{1.976766in}{2.918629in}}%
\pgfpathlineto{\pgfqpoint{1.971055in}{2.926273in}}%
\pgfpathlineto{\pgfqpoint{1.965349in}{2.933906in}}%
\pgfpathlineto{\pgfqpoint{1.959646in}{2.941527in}}%
\pgfpathlineto{\pgfqpoint{1.953948in}{2.949139in}}%
\pgfpathlineto{\pgfqpoint{1.942637in}{2.945452in}}%
\pgfpathlineto{\pgfqpoint{1.931317in}{2.941864in}}%
\pgfpathlineto{\pgfqpoint{1.919990in}{2.938353in}}%
\pgfpathlineto{\pgfqpoint{1.908656in}{2.934897in}}%
\pgfpathlineto{\pgfqpoint{1.897316in}{2.931476in}}%
\pgfpathlineto{\pgfqpoint{1.903001in}{2.923898in}}%
\pgfpathlineto{\pgfqpoint{1.908689in}{2.916308in}}%
\pgfpathlineto{\pgfqpoint{1.914382in}{2.908705in}}%
\pgfpathlineto{\pgfqpoint{1.920079in}{2.901089in}}%
\pgfpathclose%
\pgfusepath{stroke,fill}%
\end{pgfscope}%
\begin{pgfscope}%
\pgfpathrectangle{\pgfqpoint{0.887500in}{0.275000in}}{\pgfqpoint{4.225000in}{4.225000in}}%
\pgfusepath{clip}%
\pgfsetbuttcap%
\pgfsetroundjoin%
\definecolor{currentfill}{rgb}{0.120081,0.622161,0.534946}%
\pgfsetfillcolor{currentfill}%
\pgfsetfillopacity{0.700000}%
\pgfsetlinewidth{0.501875pt}%
\definecolor{currentstroke}{rgb}{1.000000,1.000000,1.000000}%
\pgfsetstrokecolor{currentstroke}%
\pgfsetstrokeopacity{0.500000}%
\pgfsetdash{}{0pt}%
\pgfpathmoveto{\pgfqpoint{2.153128in}{2.867450in}}%
\pgfpathlineto{\pgfqpoint{2.164423in}{2.871002in}}%
\pgfpathlineto{\pgfqpoint{2.175695in}{2.875351in}}%
\pgfpathlineto{\pgfqpoint{2.186939in}{2.880768in}}%
\pgfpathlineto{\pgfqpoint{2.198150in}{2.887520in}}%
\pgfpathlineto{\pgfqpoint{2.209328in}{2.895652in}}%
\pgfpathlineto{\pgfqpoint{2.203489in}{2.905588in}}%
\pgfpathlineto{\pgfqpoint{2.197647in}{2.915789in}}%
\pgfpathlineto{\pgfqpoint{2.191804in}{2.926211in}}%
\pgfpathlineto{\pgfqpoint{2.185959in}{2.936809in}}%
\pgfpathlineto{\pgfqpoint{2.180115in}{2.947539in}}%
\pgfpathlineto{\pgfqpoint{2.168941in}{2.939738in}}%
\pgfpathlineto{\pgfqpoint{2.157749in}{2.932637in}}%
\pgfpathlineto{\pgfqpoint{2.146537in}{2.926235in}}%
\pgfpathlineto{\pgfqpoint{2.135308in}{2.920432in}}%
\pgfpathlineto{\pgfqpoint{2.124064in}{2.915125in}}%
\pgfpathlineto{\pgfqpoint{2.129876in}{2.905362in}}%
\pgfpathlineto{\pgfqpoint{2.135689in}{2.895682in}}%
\pgfpathlineto{\pgfqpoint{2.141502in}{2.886117in}}%
\pgfpathlineto{\pgfqpoint{2.147316in}{2.876695in}}%
\pgfpathclose%
\pgfusepath{stroke,fill}%
\end{pgfscope}%
\begin{pgfscope}%
\pgfpathrectangle{\pgfqpoint{0.887500in}{0.275000in}}{\pgfqpoint{4.225000in}{4.225000in}}%
\pgfusepath{clip}%
\pgfsetbuttcap%
\pgfsetroundjoin%
\definecolor{currentfill}{rgb}{0.146616,0.673050,0.508936}%
\pgfsetfillcolor{currentfill}%
\pgfsetfillopacity{0.700000}%
\pgfsetlinewidth{0.501875pt}%
\definecolor{currentstroke}{rgb}{1.000000,1.000000,1.000000}%
\pgfsetstrokecolor{currentstroke}%
\pgfsetstrokeopacity{0.500000}%
\pgfsetdash{}{0pt}%
\pgfpathmoveto{\pgfqpoint{2.491123in}{2.978718in}}%
\pgfpathlineto{\pgfqpoint{2.502331in}{2.982938in}}%
\pgfpathlineto{\pgfqpoint{2.513532in}{2.987242in}}%
\pgfpathlineto{\pgfqpoint{2.524724in}{2.991946in}}%
\pgfpathlineto{\pgfqpoint{2.535901in}{2.997366in}}%
\pgfpathlineto{\pgfqpoint{2.547062in}{3.003820in}}%
\pgfpathlineto{\pgfqpoint{2.541143in}{3.012650in}}%
\pgfpathlineto{\pgfqpoint{2.535222in}{3.021892in}}%
\pgfpathlineto{\pgfqpoint{2.529301in}{3.031407in}}%
\pgfpathlineto{\pgfqpoint{2.523382in}{3.041057in}}%
\pgfpathlineto{\pgfqpoint{2.517466in}{3.050702in}}%
\pgfpathlineto{\pgfqpoint{2.506282in}{3.046787in}}%
\pgfpathlineto{\pgfqpoint{2.495084in}{3.043598in}}%
\pgfpathlineto{\pgfqpoint{2.483875in}{3.040735in}}%
\pgfpathlineto{\pgfqpoint{2.472662in}{3.037800in}}%
\pgfpathlineto{\pgfqpoint{2.461451in}{3.034392in}}%
\pgfpathlineto{\pgfqpoint{2.467383in}{3.023043in}}%
\pgfpathlineto{\pgfqpoint{2.473319in}{3.011597in}}%
\pgfpathlineto{\pgfqpoint{2.479256in}{3.000258in}}%
\pgfpathlineto{\pgfqpoint{2.485192in}{2.989230in}}%
\pgfpathclose%
\pgfusepath{stroke,fill}%
\end{pgfscope}%
\begin{pgfscope}%
\pgfpathrectangle{\pgfqpoint{0.887500in}{0.275000in}}{\pgfqpoint{4.225000in}{4.225000in}}%
\pgfusepath{clip}%
\pgfsetbuttcap%
\pgfsetroundjoin%
\definecolor{currentfill}{rgb}{0.835270,0.886029,0.102646}%
\pgfsetfillcolor{currentfill}%
\pgfsetfillopacity{0.700000}%
\pgfsetlinewidth{0.501875pt}%
\definecolor{currentstroke}{rgb}{1.000000,1.000000,1.000000}%
\pgfsetstrokecolor{currentstroke}%
\pgfsetstrokeopacity{0.500000}%
\pgfsetdash{}{0pt}%
\pgfpathmoveto{\pgfqpoint{3.272815in}{3.620426in}}%
\pgfpathlineto{\pgfqpoint{3.283864in}{3.625149in}}%
\pgfpathlineto{\pgfqpoint{3.294910in}{3.630064in}}%
\pgfpathlineto{\pgfqpoint{3.305953in}{3.635193in}}%
\pgfpathlineto{\pgfqpoint{3.316993in}{3.640543in}}%
\pgfpathlineto{\pgfqpoint{3.328031in}{3.646079in}}%
\pgfpathlineto{\pgfqpoint{3.321842in}{3.650199in}}%
\pgfpathlineto{\pgfqpoint{3.315655in}{3.654018in}}%
\pgfpathlineto{\pgfqpoint{3.309473in}{3.657549in}}%
\pgfpathlineto{\pgfqpoint{3.303294in}{3.660807in}}%
\pgfpathlineto{\pgfqpoint{3.297119in}{3.663804in}}%
\pgfpathlineto{\pgfqpoint{3.286099in}{3.658074in}}%
\pgfpathlineto{\pgfqpoint{3.275076in}{3.652568in}}%
\pgfpathlineto{\pgfqpoint{3.264050in}{3.647323in}}%
\pgfpathlineto{\pgfqpoint{3.253021in}{3.642310in}}%
\pgfpathlineto{\pgfqpoint{3.241989in}{3.637470in}}%
\pgfpathlineto{\pgfqpoint{3.248146in}{3.634557in}}%
\pgfpathlineto{\pgfqpoint{3.254307in}{3.631390in}}%
\pgfpathlineto{\pgfqpoint{3.260472in}{3.627976in}}%
\pgfpathlineto{\pgfqpoint{3.266642in}{3.624320in}}%
\pgfpathclose%
\pgfusepath{stroke,fill}%
\end{pgfscope}%
\begin{pgfscope}%
\pgfpathrectangle{\pgfqpoint{0.887500in}{0.275000in}}{\pgfqpoint{4.225000in}{4.225000in}}%
\pgfusepath{clip}%
\pgfsetbuttcap%
\pgfsetroundjoin%
\definecolor{currentfill}{rgb}{0.926106,0.897330,0.104071}%
\pgfsetfillcolor{currentfill}%
\pgfsetfillopacity{0.700000}%
\pgfsetlinewidth{0.501875pt}%
\definecolor{currentstroke}{rgb}{1.000000,1.000000,1.000000}%
\pgfsetstrokecolor{currentstroke}%
\pgfsetstrokeopacity{0.500000}%
\pgfsetdash{}{0pt}%
\pgfpathmoveto{\pgfqpoint{3.696747in}{3.677843in}}%
\pgfpathlineto{\pgfqpoint{3.707714in}{3.682863in}}%
\pgfpathlineto{\pgfqpoint{3.718676in}{3.687834in}}%
\pgfpathlineto{\pgfqpoint{3.729633in}{3.692758in}}%
\pgfpathlineto{\pgfqpoint{3.740586in}{3.697639in}}%
\pgfpathlineto{\pgfqpoint{3.751533in}{3.702478in}}%
\pgfpathlineto{\pgfqpoint{3.745240in}{3.710763in}}%
\pgfpathlineto{\pgfqpoint{3.738948in}{3.718857in}}%
\pgfpathlineto{\pgfqpoint{3.732657in}{3.726762in}}%
\pgfpathlineto{\pgfqpoint{3.726367in}{3.734479in}}%
\pgfpathlineto{\pgfqpoint{3.720079in}{3.742007in}}%
\pgfpathlineto{\pgfqpoint{3.709143in}{3.737209in}}%
\pgfpathlineto{\pgfqpoint{3.698203in}{3.732370in}}%
\pgfpathlineto{\pgfqpoint{3.687258in}{3.727488in}}%
\pgfpathlineto{\pgfqpoint{3.676308in}{3.722564in}}%
\pgfpathlineto{\pgfqpoint{3.665353in}{3.717597in}}%
\pgfpathlineto{\pgfqpoint{3.671629in}{3.709993in}}%
\pgfpathlineto{\pgfqpoint{3.677906in}{3.702210in}}%
\pgfpathlineto{\pgfqpoint{3.684185in}{3.694255in}}%
\pgfpathlineto{\pgfqpoint{3.690465in}{3.686133in}}%
\pgfpathclose%
\pgfusepath{stroke,fill}%
\end{pgfscope}%
\begin{pgfscope}%
\pgfpathrectangle{\pgfqpoint{0.887500in}{0.275000in}}{\pgfqpoint{4.225000in}{4.225000in}}%
\pgfusepath{clip}%
\pgfsetbuttcap%
\pgfsetroundjoin%
\definecolor{currentfill}{rgb}{0.688944,0.865448,0.182725}%
\pgfsetfillcolor{currentfill}%
\pgfsetfillopacity{0.700000}%
\pgfsetlinewidth{0.501875pt}%
\definecolor{currentstroke}{rgb}{1.000000,1.000000,1.000000}%
\pgfsetstrokecolor{currentstroke}%
\pgfsetstrokeopacity{0.500000}%
\pgfsetdash{}{0pt}%
\pgfpathmoveto{\pgfqpoint{2.934563in}{3.526813in}}%
\pgfpathlineto{\pgfqpoint{2.945686in}{3.530689in}}%
\pgfpathlineto{\pgfqpoint{2.956805in}{3.534352in}}%
\pgfpathlineto{\pgfqpoint{2.967919in}{3.537896in}}%
\pgfpathlineto{\pgfqpoint{2.979028in}{3.541419in}}%
\pgfpathlineto{\pgfqpoint{2.990131in}{3.545015in}}%
\pgfpathlineto{\pgfqpoint{2.984088in}{3.545558in}}%
\pgfpathlineto{\pgfqpoint{2.978050in}{3.546079in}}%
\pgfpathlineto{\pgfqpoint{2.972019in}{3.546594in}}%
\pgfpathlineto{\pgfqpoint{2.965993in}{3.547118in}}%
\pgfpathlineto{\pgfqpoint{2.959974in}{3.547667in}}%
\pgfpathlineto{\pgfqpoint{2.948893in}{3.543596in}}%
\pgfpathlineto{\pgfqpoint{2.937808in}{3.539171in}}%
\pgfpathlineto{\pgfqpoint{2.926720in}{3.534135in}}%
\pgfpathlineto{\pgfqpoint{2.915632in}{3.528233in}}%
\pgfpathlineto{\pgfqpoint{2.904543in}{3.521209in}}%
\pgfpathlineto{\pgfqpoint{2.910534in}{3.522493in}}%
\pgfpathlineto{\pgfqpoint{2.916530in}{3.523784in}}%
\pgfpathlineto{\pgfqpoint{2.922534in}{3.524992in}}%
\pgfpathlineto{\pgfqpoint{2.928545in}{3.526031in}}%
\pgfpathclose%
\pgfusepath{stroke,fill}%
\end{pgfscope}%
\begin{pgfscope}%
\pgfpathrectangle{\pgfqpoint{0.887500in}{0.275000in}}{\pgfqpoint{4.225000in}{4.225000in}}%
\pgfusepath{clip}%
\pgfsetbuttcap%
\pgfsetroundjoin%
\definecolor{currentfill}{rgb}{0.143303,0.669459,0.511215}%
\pgfsetfillcolor{currentfill}%
\pgfsetfillopacity{0.700000}%
\pgfsetlinewidth{0.501875pt}%
\definecolor{currentstroke}{rgb}{1.000000,1.000000,1.000000}%
\pgfsetstrokecolor{currentstroke}%
\pgfsetstrokeopacity{0.500000}%
\pgfsetdash{}{0pt}%
\pgfpathmoveto{\pgfqpoint{2.349793in}{2.957948in}}%
\pgfpathlineto{\pgfqpoint{2.360940in}{2.967417in}}%
\pgfpathlineto{\pgfqpoint{2.372089in}{2.976682in}}%
\pgfpathlineto{\pgfqpoint{2.383238in}{2.985787in}}%
\pgfpathlineto{\pgfqpoint{2.394389in}{2.994657in}}%
\pgfpathlineto{\pgfqpoint{2.405545in}{3.003140in}}%
\pgfpathlineto{\pgfqpoint{2.399648in}{3.013027in}}%
\pgfpathlineto{\pgfqpoint{2.393762in}{3.022517in}}%
\pgfpathlineto{\pgfqpoint{2.387887in}{3.031493in}}%
\pgfpathlineto{\pgfqpoint{2.382026in}{3.039866in}}%
\pgfpathlineto{\pgfqpoint{2.376178in}{3.047676in}}%
\pgfpathlineto{\pgfqpoint{2.365057in}{3.038017in}}%
\pgfpathlineto{\pgfqpoint{2.353939in}{3.028090in}}%
\pgfpathlineto{\pgfqpoint{2.342819in}{3.018152in}}%
\pgfpathlineto{\pgfqpoint{2.331696in}{3.008330in}}%
\pgfpathlineto{\pgfqpoint{2.320570in}{2.998560in}}%
\pgfpathlineto{\pgfqpoint{2.326402in}{2.990751in}}%
\pgfpathlineto{\pgfqpoint{2.332239in}{2.982793in}}%
\pgfpathlineto{\pgfqpoint{2.338084in}{2.974670in}}%
\pgfpathlineto{\pgfqpoint{2.343935in}{2.966385in}}%
\pgfpathclose%
\pgfusepath{stroke,fill}%
\end{pgfscope}%
\begin{pgfscope}%
\pgfpathrectangle{\pgfqpoint{0.887500in}{0.275000in}}{\pgfqpoint{4.225000in}{4.225000in}}%
\pgfusepath{clip}%
\pgfsetbuttcap%
\pgfsetroundjoin%
\definecolor{currentfill}{rgb}{0.304148,0.764704,0.419943}%
\pgfsetfillcolor{currentfill}%
\pgfsetfillopacity{0.700000}%
\pgfsetlinewidth{0.501875pt}%
\definecolor{currentstroke}{rgb}{1.000000,1.000000,1.000000}%
\pgfsetstrokecolor{currentstroke}%
\pgfsetstrokeopacity{0.500000}%
\pgfsetdash{}{0pt}%
\pgfpathmoveto{\pgfqpoint{2.657579in}{3.158198in}}%
\pgfpathlineto{\pgfqpoint{2.668557in}{3.182059in}}%
\pgfpathlineto{\pgfqpoint{2.679537in}{3.206497in}}%
\pgfpathlineto{\pgfqpoint{2.690521in}{3.231118in}}%
\pgfpathlineto{\pgfqpoint{2.701513in}{3.255525in}}%
\pgfpathlineto{\pgfqpoint{2.712518in}{3.279322in}}%
\pgfpathlineto{\pgfqpoint{2.706604in}{3.280217in}}%
\pgfpathlineto{\pgfqpoint{2.700719in}{3.278550in}}%
\pgfpathlineto{\pgfqpoint{2.694860in}{3.274805in}}%
\pgfpathlineto{\pgfqpoint{2.689022in}{3.269460in}}%
\pgfpathlineto{\pgfqpoint{2.683202in}{3.262998in}}%
\pgfpathlineto{\pgfqpoint{2.672186in}{3.244853in}}%
\pgfpathlineto{\pgfqpoint{2.661180in}{3.225970in}}%
\pgfpathlineto{\pgfqpoint{2.650180in}{3.206718in}}%
\pgfpathlineto{\pgfqpoint{2.639185in}{3.187465in}}%
\pgfpathlineto{\pgfqpoint{2.628189in}{3.168580in}}%
\pgfpathlineto{\pgfqpoint{2.634037in}{3.168338in}}%
\pgfpathlineto{\pgfqpoint{2.639898in}{3.167421in}}%
\pgfpathlineto{\pgfqpoint{2.645773in}{3.165587in}}%
\pgfpathlineto{\pgfqpoint{2.651667in}{3.162594in}}%
\pgfpathclose%
\pgfusepath{stroke,fill}%
\end{pgfscope}%
\begin{pgfscope}%
\pgfpathrectangle{\pgfqpoint{0.887500in}{0.275000in}}{\pgfqpoint{4.225000in}{4.225000in}}%
\pgfusepath{clip}%
\pgfsetbuttcap%
\pgfsetroundjoin%
\definecolor{currentfill}{rgb}{0.916242,0.896091,0.100717}%
\pgfsetfillcolor{currentfill}%
\pgfsetfillopacity{0.700000}%
\pgfsetlinewidth{0.501875pt}%
\definecolor{currentstroke}{rgb}{1.000000,1.000000,1.000000}%
\pgfsetstrokecolor{currentstroke}%
\pgfsetstrokeopacity{0.500000}%
\pgfsetdash{}{0pt}%
\pgfpathmoveto{\pgfqpoint{3.555555in}{3.666393in}}%
\pgfpathlineto{\pgfqpoint{3.566554in}{3.671568in}}%
\pgfpathlineto{\pgfqpoint{3.577549in}{3.676741in}}%
\pgfpathlineto{\pgfqpoint{3.588540in}{3.681910in}}%
\pgfpathlineto{\pgfqpoint{3.599526in}{3.687070in}}%
\pgfpathlineto{\pgfqpoint{3.610508in}{3.692218in}}%
\pgfpathlineto{\pgfqpoint{3.604247in}{3.699691in}}%
\pgfpathlineto{\pgfqpoint{3.597987in}{3.706980in}}%
\pgfpathlineto{\pgfqpoint{3.591729in}{3.714073in}}%
\pgfpathlineto{\pgfqpoint{3.585473in}{3.720959in}}%
\pgfpathlineto{\pgfqpoint{3.579218in}{3.727629in}}%
\pgfpathlineto{\pgfqpoint{3.568249in}{3.722520in}}%
\pgfpathlineto{\pgfqpoint{3.557276in}{3.717389in}}%
\pgfpathlineto{\pgfqpoint{3.546298in}{3.712236in}}%
\pgfpathlineto{\pgfqpoint{3.535316in}{3.707059in}}%
\pgfpathlineto{\pgfqpoint{3.524329in}{3.701857in}}%
\pgfpathlineto{\pgfqpoint{3.530570in}{3.695147in}}%
\pgfpathlineto{\pgfqpoint{3.536813in}{3.688217in}}%
\pgfpathlineto{\pgfqpoint{3.543058in}{3.681095in}}%
\pgfpathlineto{\pgfqpoint{3.549305in}{3.673811in}}%
\pgfpathclose%
\pgfusepath{stroke,fill}%
\end{pgfscope}%
\begin{pgfscope}%
\pgfpathrectangle{\pgfqpoint{0.887500in}{0.275000in}}{\pgfqpoint{4.225000in}{4.225000in}}%
\pgfusepath{clip}%
\pgfsetbuttcap%
\pgfsetroundjoin%
\definecolor{currentfill}{rgb}{0.886271,0.892374,0.095374}%
\pgfsetfillcolor{currentfill}%
\pgfsetfillopacity{0.700000}%
\pgfsetlinewidth{0.501875pt}%
\definecolor{currentstroke}{rgb}{1.000000,1.000000,1.000000}%
\pgfsetstrokecolor{currentstroke}%
\pgfsetstrokeopacity{0.500000}%
\pgfsetdash{}{0pt}%
\pgfpathmoveto{\pgfqpoint{3.414224in}{3.648008in}}%
\pgfpathlineto{\pgfqpoint{3.425254in}{3.653550in}}%
\pgfpathlineto{\pgfqpoint{3.436280in}{3.659059in}}%
\pgfpathlineto{\pgfqpoint{3.447301in}{3.664532in}}%
\pgfpathlineto{\pgfqpoint{3.458318in}{3.669970in}}%
\pgfpathlineto{\pgfqpoint{3.469331in}{3.675372in}}%
\pgfpathlineto{\pgfqpoint{3.463105in}{3.681731in}}%
\pgfpathlineto{\pgfqpoint{3.456879in}{3.687777in}}%
\pgfpathlineto{\pgfqpoint{3.450656in}{3.693472in}}%
\pgfpathlineto{\pgfqpoint{3.444433in}{3.698811in}}%
\pgfpathlineto{\pgfqpoint{3.438212in}{3.703804in}}%
\pgfpathlineto{\pgfqpoint{3.427212in}{3.698168in}}%
\pgfpathlineto{\pgfqpoint{3.416208in}{3.692484in}}%
\pgfpathlineto{\pgfqpoint{3.405200in}{3.686751in}}%
\pgfpathlineto{\pgfqpoint{3.394187in}{3.680972in}}%
\pgfpathlineto{\pgfqpoint{3.383170in}{3.675149in}}%
\pgfpathlineto{\pgfqpoint{3.389377in}{3.670416in}}%
\pgfpathlineto{\pgfqpoint{3.395586in}{3.665338in}}%
\pgfpathlineto{\pgfqpoint{3.401797in}{3.659902in}}%
\pgfpathlineto{\pgfqpoint{3.408010in}{3.654109in}}%
\pgfpathclose%
\pgfusepath{stroke,fill}%
\end{pgfscope}%
\begin{pgfscope}%
\pgfpathrectangle{\pgfqpoint{0.887500in}{0.275000in}}{\pgfqpoint{4.225000in}{4.225000in}}%
\pgfusepath{clip}%
\pgfsetbuttcap%
\pgfsetroundjoin%
\definecolor{currentfill}{rgb}{0.120081,0.622161,0.534946}%
\pgfsetfillcolor{currentfill}%
\pgfsetfillopacity{0.700000}%
\pgfsetlinewidth{0.501875pt}%
\definecolor{currentstroke}{rgb}{1.000000,1.000000,1.000000}%
\pgfsetstrokecolor{currentstroke}%
\pgfsetstrokeopacity{0.500000}%
\pgfsetdash{}{0pt}%
\pgfpathmoveto{\pgfqpoint{2.011111in}{2.872527in}}%
\pgfpathlineto{\pgfqpoint{2.022441in}{2.876307in}}%
\pgfpathlineto{\pgfqpoint{2.033762in}{2.880208in}}%
\pgfpathlineto{\pgfqpoint{2.045076in}{2.884219in}}%
\pgfpathlineto{\pgfqpoint{2.056382in}{2.888329in}}%
\pgfpathlineto{\pgfqpoint{2.067681in}{2.892528in}}%
\pgfpathlineto{\pgfqpoint{2.061926in}{2.900546in}}%
\pgfpathlineto{\pgfqpoint{2.056175in}{2.908557in}}%
\pgfpathlineto{\pgfqpoint{2.050428in}{2.916557in}}%
\pgfpathlineto{\pgfqpoint{2.044684in}{2.924543in}}%
\pgfpathlineto{\pgfqpoint{2.038945in}{2.932509in}}%
\pgfpathlineto{\pgfqpoint{2.027678in}{2.927573in}}%
\pgfpathlineto{\pgfqpoint{2.016397in}{2.923012in}}%
\pgfpathlineto{\pgfqpoint{2.005102in}{2.918767in}}%
\pgfpathlineto{\pgfqpoint{1.993796in}{2.914775in}}%
\pgfpathlineto{\pgfqpoint{1.982480in}{2.910974in}}%
\pgfpathlineto{\pgfqpoint{1.988198in}{2.903308in}}%
\pgfpathlineto{\pgfqpoint{1.993920in}{2.895630in}}%
\pgfpathlineto{\pgfqpoint{1.999647in}{2.887941in}}%
\pgfpathlineto{\pgfqpoint{2.005377in}{2.880240in}}%
\pgfpathclose%
\pgfusepath{stroke,fill}%
\end{pgfscope}%
\begin{pgfscope}%
\pgfpathrectangle{\pgfqpoint{0.887500in}{0.275000in}}{\pgfqpoint{4.225000in}{4.225000in}}%
\pgfusepath{clip}%
\pgfsetbuttcap%
\pgfsetroundjoin%
\definecolor{currentfill}{rgb}{0.128087,0.647749,0.523491}%
\pgfsetfillcolor{currentfill}%
\pgfsetfillopacity{0.700000}%
\pgfsetlinewidth{0.501875pt}%
\definecolor{currentstroke}{rgb}{1.000000,1.000000,1.000000}%
\pgfsetstrokecolor{currentstroke}%
\pgfsetstrokeopacity{0.500000}%
\pgfsetdash{}{0pt}%
\pgfpathmoveto{\pgfqpoint{2.294122in}{2.905692in}}%
\pgfpathlineto{\pgfqpoint{2.305243in}{2.916970in}}%
\pgfpathlineto{\pgfqpoint{2.316372in}{2.927786in}}%
\pgfpathlineto{\pgfqpoint{2.327508in}{2.938186in}}%
\pgfpathlineto{\pgfqpoint{2.338648in}{2.948222in}}%
\pgfpathlineto{\pgfqpoint{2.349793in}{2.957948in}}%
\pgfpathlineto{\pgfqpoint{2.343935in}{2.966385in}}%
\pgfpathlineto{\pgfqpoint{2.338084in}{2.974670in}}%
\pgfpathlineto{\pgfqpoint{2.332239in}{2.982793in}}%
\pgfpathlineto{\pgfqpoint{2.326402in}{2.990751in}}%
\pgfpathlineto{\pgfqpoint{2.320570in}{2.998560in}}%
\pgfpathlineto{\pgfqpoint{2.309443in}{2.988760in}}%
\pgfpathlineto{\pgfqpoint{2.298317in}{2.978847in}}%
\pgfpathlineto{\pgfqpoint{2.287192in}{2.968742in}}%
\pgfpathlineto{\pgfqpoint{2.276071in}{2.958363in}}%
\pgfpathlineto{\pgfqpoint{2.264956in}{2.947635in}}%
\pgfpathlineto{\pgfqpoint{2.270787in}{2.939011in}}%
\pgfpathlineto{\pgfqpoint{2.276619in}{2.930512in}}%
\pgfpathlineto{\pgfqpoint{2.282451in}{2.922145in}}%
\pgfpathlineto{\pgfqpoint{2.288286in}{2.913882in}}%
\pgfpathclose%
\pgfusepath{stroke,fill}%
\end{pgfscope}%
\begin{pgfscope}%
\pgfpathrectangle{\pgfqpoint{0.887500in}{0.275000in}}{\pgfqpoint{4.225000in}{4.225000in}}%
\pgfusepath{clip}%
\pgfsetbuttcap%
\pgfsetroundjoin%
\definecolor{currentfill}{rgb}{0.762373,0.876424,0.137064}%
\pgfsetfillcolor{currentfill}%
\pgfsetfillopacity{0.700000}%
\pgfsetlinewidth{0.501875pt}%
\definecolor{currentstroke}{rgb}{1.000000,1.000000,1.000000}%
\pgfsetstrokecolor{currentstroke}%
\pgfsetstrokeopacity{0.500000}%
\pgfsetdash{}{0pt}%
\pgfpathmoveto{\pgfqpoint{3.075992in}{3.564660in}}%
\pgfpathlineto{\pgfqpoint{3.087089in}{3.569383in}}%
\pgfpathlineto{\pgfqpoint{3.098181in}{3.574148in}}%
\pgfpathlineto{\pgfqpoint{3.109269in}{3.578966in}}%
\pgfpathlineto{\pgfqpoint{3.120353in}{3.583851in}}%
\pgfpathlineto{\pgfqpoint{3.131432in}{3.588810in}}%
\pgfpathlineto{\pgfqpoint{3.125316in}{3.590194in}}%
\pgfpathlineto{\pgfqpoint{3.119205in}{3.591456in}}%
\pgfpathlineto{\pgfqpoint{3.113100in}{3.592618in}}%
\pgfpathlineto{\pgfqpoint{3.107000in}{3.593697in}}%
\pgfpathlineto{\pgfqpoint{3.100907in}{3.594708in}}%
\pgfpathlineto{\pgfqpoint{3.089848in}{3.588976in}}%
\pgfpathlineto{\pgfqpoint{3.078785in}{3.583285in}}%
\pgfpathlineto{\pgfqpoint{3.067718in}{3.577684in}}%
\pgfpathlineto{\pgfqpoint{3.056648in}{3.572223in}}%
\pgfpathlineto{\pgfqpoint{3.045573in}{3.566955in}}%
\pgfpathlineto{\pgfqpoint{3.051645in}{3.566403in}}%
\pgfpathlineto{\pgfqpoint{3.057722in}{3.565912in}}%
\pgfpathlineto{\pgfqpoint{3.063806in}{3.565481in}}%
\pgfpathlineto{\pgfqpoint{3.069896in}{3.565082in}}%
\pgfpathclose%
\pgfusepath{stroke,fill}%
\end{pgfscope}%
\begin{pgfscope}%
\pgfpathrectangle{\pgfqpoint{0.887500in}{0.275000in}}{\pgfqpoint{4.225000in}{4.225000in}}%
\pgfusepath{clip}%
\pgfsetbuttcap%
\pgfsetroundjoin%
\definecolor{currentfill}{rgb}{0.916242,0.896091,0.100717}%
\pgfsetfillcolor{currentfill}%
\pgfsetfillopacity{0.700000}%
\pgfsetlinewidth{0.501875pt}%
\definecolor{currentstroke}{rgb}{1.000000,1.000000,1.000000}%
\pgfsetstrokecolor{currentstroke}%
\pgfsetstrokeopacity{0.500000}%
\pgfsetdash{}{0pt}%
\pgfpathmoveto{\pgfqpoint{3.783002in}{3.657945in}}%
\pgfpathlineto{\pgfqpoint{3.793953in}{3.662668in}}%
\pgfpathlineto{\pgfqpoint{3.804899in}{3.667358in}}%
\pgfpathlineto{\pgfqpoint{3.815840in}{3.672021in}}%
\pgfpathlineto{\pgfqpoint{3.826776in}{3.676664in}}%
\pgfpathlineto{\pgfqpoint{3.837708in}{3.681295in}}%
\pgfpathlineto{\pgfqpoint{3.831407in}{3.690749in}}%
\pgfpathlineto{\pgfqpoint{3.825106in}{3.699959in}}%
\pgfpathlineto{\pgfqpoint{3.818803in}{3.708931in}}%
\pgfpathlineto{\pgfqpoint{3.812500in}{3.717668in}}%
\pgfpathlineto{\pgfqpoint{3.806197in}{3.726174in}}%
\pgfpathlineto{\pgfqpoint{3.795274in}{3.721492in}}%
\pgfpathlineto{\pgfqpoint{3.784346in}{3.716784in}}%
\pgfpathlineto{\pgfqpoint{3.773413in}{3.712048in}}%
\pgfpathlineto{\pgfqpoint{3.762475in}{3.707281in}}%
\pgfpathlineto{\pgfqpoint{3.751533in}{3.702478in}}%
\pgfpathlineto{\pgfqpoint{3.757826in}{3.693995in}}%
\pgfpathlineto{\pgfqpoint{3.764120in}{3.685307in}}%
\pgfpathlineto{\pgfqpoint{3.770414in}{3.676407in}}%
\pgfpathlineto{\pgfqpoint{3.776708in}{3.667288in}}%
\pgfpathclose%
\pgfusepath{stroke,fill}%
\end{pgfscope}%
\begin{pgfscope}%
\pgfpathrectangle{\pgfqpoint{0.887500in}{0.275000in}}{\pgfqpoint{4.225000in}{4.225000in}}%
\pgfusepath{clip}%
\pgfsetbuttcap%
\pgfsetroundjoin%
\definecolor{currentfill}{rgb}{0.122312,0.633153,0.530398}%
\pgfsetfillcolor{currentfill}%
\pgfsetfillopacity{0.700000}%
\pgfsetlinewidth{0.501875pt}%
\definecolor{currentstroke}{rgb}{1.000000,1.000000,1.000000}%
\pgfsetstrokecolor{currentstroke}%
\pgfsetstrokeopacity{0.500000}%
\pgfsetdash{}{0pt}%
\pgfpathmoveto{\pgfqpoint{1.783630in}{2.896927in}}%
\pgfpathlineto{\pgfqpoint{1.795023in}{2.900373in}}%
\pgfpathlineto{\pgfqpoint{1.806410in}{2.903837in}}%
\pgfpathlineto{\pgfqpoint{1.817792in}{2.907314in}}%
\pgfpathlineto{\pgfqpoint{1.829168in}{2.910797in}}%
\pgfpathlineto{\pgfqpoint{1.840538in}{2.914279in}}%
\pgfpathlineto{\pgfqpoint{1.834872in}{2.921796in}}%
\pgfpathlineto{\pgfqpoint{1.829210in}{2.929298in}}%
\pgfpathlineto{\pgfqpoint{1.823552in}{2.936785in}}%
\pgfpathlineto{\pgfqpoint{1.817898in}{2.944255in}}%
\pgfpathlineto{\pgfqpoint{1.806539in}{2.940732in}}%
\pgfpathlineto{\pgfqpoint{1.795175in}{2.937212in}}%
\pgfpathlineto{\pgfqpoint{1.783806in}{2.933703in}}%
\pgfpathlineto{\pgfqpoint{1.772430in}{2.930215in}}%
\pgfpathlineto{\pgfqpoint{1.761048in}{2.926753in}}%
\pgfpathlineto{\pgfqpoint{1.766687in}{2.919322in}}%
\pgfpathlineto{\pgfqpoint{1.772331in}{2.911874in}}%
\pgfpathlineto{\pgfqpoint{1.777979in}{2.904409in}}%
\pgfpathclose%
\pgfusepath{stroke,fill}%
\end{pgfscope}%
\begin{pgfscope}%
\pgfpathrectangle{\pgfqpoint{0.887500in}{0.275000in}}{\pgfqpoint{4.225000in}{4.225000in}}%
\pgfusepath{clip}%
\pgfsetbuttcap%
\pgfsetroundjoin%
\definecolor{currentfill}{rgb}{0.120081,0.622161,0.534946}%
\pgfsetfillcolor{currentfill}%
\pgfsetfillopacity{0.700000}%
\pgfsetlinewidth{0.501875pt}%
\definecolor{currentstroke}{rgb}{1.000000,1.000000,1.000000}%
\pgfsetstrokecolor{currentstroke}%
\pgfsetstrokeopacity{0.500000}%
\pgfsetdash{}{0pt}%
\pgfpathmoveto{\pgfqpoint{2.238467in}{2.851045in}}%
\pgfpathlineto{\pgfqpoint{2.249627in}{2.860629in}}%
\pgfpathlineto{\pgfqpoint{2.260766in}{2.871222in}}%
\pgfpathlineto{\pgfqpoint{2.271890in}{2.882486in}}%
\pgfpathlineto{\pgfqpoint{2.283006in}{2.894088in}}%
\pgfpathlineto{\pgfqpoint{2.294122in}{2.905692in}}%
\pgfpathlineto{\pgfqpoint{2.288286in}{2.913882in}}%
\pgfpathlineto{\pgfqpoint{2.282451in}{2.922145in}}%
\pgfpathlineto{\pgfqpoint{2.276619in}{2.930512in}}%
\pgfpathlineto{\pgfqpoint{2.270787in}{2.939011in}}%
\pgfpathlineto{\pgfqpoint{2.264956in}{2.947635in}}%
\pgfpathlineto{\pgfqpoint{2.253845in}{2.936653in}}%
\pgfpathlineto{\pgfqpoint{2.242733in}{2.925686in}}%
\pgfpathlineto{\pgfqpoint{2.231613in}{2.915012in}}%
\pgfpathlineto{\pgfqpoint{2.220480in}{2.904908in}}%
\pgfpathlineto{\pgfqpoint{2.209328in}{2.895652in}}%
\pgfpathlineto{\pgfqpoint{2.215164in}{2.886025in}}%
\pgfpathlineto{\pgfqpoint{2.220996in}{2.876752in}}%
\pgfpathlineto{\pgfqpoint{2.226823in}{2.867859in}}%
\pgfpathlineto{\pgfqpoint{2.232646in}{2.859307in}}%
\pgfpathclose%
\pgfusepath{stroke,fill}%
\end{pgfscope}%
\begin{pgfscope}%
\pgfpathrectangle{\pgfqpoint{0.887500in}{0.275000in}}{\pgfqpoint{4.225000in}{4.225000in}}%
\pgfusepath{clip}%
\pgfsetbuttcap%
\pgfsetroundjoin%
\definecolor{currentfill}{rgb}{0.150148,0.676631,0.506589}%
\pgfsetfillcolor{currentfill}%
\pgfsetfillopacity{0.700000}%
\pgfsetlinewidth{0.501875pt}%
\definecolor{currentstroke}{rgb}{1.000000,1.000000,1.000000}%
\pgfsetstrokecolor{currentstroke}%
\pgfsetstrokeopacity{0.500000}%
\pgfsetdash{}{0pt}%
\pgfpathmoveto{\pgfqpoint{2.576588in}{2.969849in}}%
\pgfpathlineto{\pgfqpoint{2.587736in}{2.978337in}}%
\pgfpathlineto{\pgfqpoint{2.598865in}{2.988285in}}%
\pgfpathlineto{\pgfqpoint{2.609977in}{2.999649in}}%
\pgfpathlineto{\pgfqpoint{2.621074in}{3.012300in}}%
\pgfpathlineto{\pgfqpoint{2.632159in}{3.026108in}}%
\pgfpathlineto{\pgfqpoint{2.626226in}{3.032925in}}%
\pgfpathlineto{\pgfqpoint{2.620298in}{3.039699in}}%
\pgfpathlineto{\pgfqpoint{2.614375in}{3.046408in}}%
\pgfpathlineto{\pgfqpoint{2.608457in}{3.053040in}}%
\pgfpathlineto{\pgfqpoint{2.602544in}{3.059589in}}%
\pgfpathlineto{\pgfqpoint{2.591489in}{3.045022in}}%
\pgfpathlineto{\pgfqpoint{2.580415in}{3.032181in}}%
\pgfpathlineto{\pgfqpoint{2.569319in}{3.021052in}}%
\pgfpathlineto{\pgfqpoint{2.558202in}{3.011623in}}%
\pgfpathlineto{\pgfqpoint{2.547062in}{3.003820in}}%
\pgfpathlineto{\pgfqpoint{2.552977in}{2.995539in}}%
\pgfpathlineto{\pgfqpoint{2.558888in}{2.987947in}}%
\pgfpathlineto{\pgfqpoint{2.564793in}{2.981173in}}%
\pgfpathlineto{\pgfqpoint{2.570692in}{2.975191in}}%
\pgfpathclose%
\pgfusepath{stroke,fill}%
\end{pgfscope}%
\begin{pgfscope}%
\pgfpathrectangle{\pgfqpoint{0.887500in}{0.275000in}}{\pgfqpoint{4.225000in}{4.225000in}}%
\pgfusepath{clip}%
\pgfsetbuttcap%
\pgfsetroundjoin%
\definecolor{currentfill}{rgb}{0.119483,0.614817,0.537692}%
\pgfsetfillcolor{currentfill}%
\pgfsetfillopacity{0.700000}%
\pgfsetlinewidth{0.501875pt}%
\definecolor{currentstroke}{rgb}{1.000000,1.000000,1.000000}%
\pgfsetstrokecolor{currentstroke}%
\pgfsetstrokeopacity{0.500000}%
\pgfsetdash{}{0pt}%
\pgfpathmoveto{\pgfqpoint{2.096509in}{2.852473in}}%
\pgfpathlineto{\pgfqpoint{2.107835in}{2.855819in}}%
\pgfpathlineto{\pgfqpoint{2.119164in}{2.858868in}}%
\pgfpathlineto{\pgfqpoint{2.130493in}{2.861654in}}%
\pgfpathlineto{\pgfqpoint{2.141816in}{2.864424in}}%
\pgfpathlineto{\pgfqpoint{2.153128in}{2.867450in}}%
\pgfpathlineto{\pgfqpoint{2.147316in}{2.876695in}}%
\pgfpathlineto{\pgfqpoint{2.141502in}{2.886117in}}%
\pgfpathlineto{\pgfqpoint{2.135689in}{2.895682in}}%
\pgfpathlineto{\pgfqpoint{2.129876in}{2.905362in}}%
\pgfpathlineto{\pgfqpoint{2.124064in}{2.915125in}}%
\pgfpathlineto{\pgfqpoint{2.112806in}{2.910211in}}%
\pgfpathlineto{\pgfqpoint{2.101537in}{2.905586in}}%
\pgfpathlineto{\pgfqpoint{2.090258in}{2.901149in}}%
\pgfpathlineto{\pgfqpoint{2.078973in}{2.896804in}}%
\pgfpathlineto{\pgfqpoint{2.067681in}{2.892528in}}%
\pgfpathlineto{\pgfqpoint{2.073439in}{2.884506in}}%
\pgfpathlineto{\pgfqpoint{2.079201in}{2.876486in}}%
\pgfpathlineto{\pgfqpoint{2.084967in}{2.868472in}}%
\pgfpathlineto{\pgfqpoint{2.090736in}{2.860466in}}%
\pgfpathclose%
\pgfusepath{stroke,fill}%
\end{pgfscope}%
\begin{pgfscope}%
\pgfpathrectangle{\pgfqpoint{0.887500in}{0.275000in}}{\pgfqpoint{4.225000in}{4.225000in}}%
\pgfusepath{clip}%
\pgfsetbuttcap%
\pgfsetroundjoin%
\definecolor{currentfill}{rgb}{0.143303,0.669459,0.511215}%
\pgfsetfillcolor{currentfill}%
\pgfsetfillopacity{0.700000}%
\pgfsetlinewidth{0.501875pt}%
\definecolor{currentstroke}{rgb}{1.000000,1.000000,1.000000}%
\pgfsetstrokecolor{currentstroke}%
\pgfsetstrokeopacity{0.500000}%
\pgfsetdash{}{0pt}%
\pgfpathmoveto{\pgfqpoint{2.435104in}{2.951826in}}%
\pgfpathlineto{\pgfqpoint{2.446302in}{2.958127in}}%
\pgfpathlineto{\pgfqpoint{2.457503in}{2.963973in}}%
\pgfpathlineto{\pgfqpoint{2.468708in}{2.969355in}}%
\pgfpathlineto{\pgfqpoint{2.479915in}{2.974264in}}%
\pgfpathlineto{\pgfqpoint{2.491123in}{2.978718in}}%
\pgfpathlineto{\pgfqpoint{2.485192in}{2.989230in}}%
\pgfpathlineto{\pgfqpoint{2.479256in}{3.000258in}}%
\pgfpathlineto{\pgfqpoint{2.473319in}{3.011597in}}%
\pgfpathlineto{\pgfqpoint{2.467383in}{3.023043in}}%
\pgfpathlineto{\pgfqpoint{2.461451in}{3.034392in}}%
\pgfpathlineto{\pgfqpoint{2.450247in}{3.030112in}}%
\pgfpathlineto{\pgfqpoint{2.439056in}{3.024721in}}%
\pgfpathlineto{\pgfqpoint{2.427876in}{3.018326in}}%
\pgfpathlineto{\pgfqpoint{2.416706in}{3.011081in}}%
\pgfpathlineto{\pgfqpoint{2.405545in}{3.003140in}}%
\pgfpathlineto{\pgfqpoint{2.411449in}{2.992971in}}%
\pgfpathlineto{\pgfqpoint{2.417359in}{2.982639in}}%
\pgfpathlineto{\pgfqpoint{2.423272in}{2.972260in}}%
\pgfpathlineto{\pgfqpoint{2.429188in}{2.961950in}}%
\pgfpathclose%
\pgfusepath{stroke,fill}%
\end{pgfscope}%
\begin{pgfscope}%
\pgfpathrectangle{\pgfqpoint{0.887500in}{0.275000in}}{\pgfqpoint{4.225000in}{4.225000in}}%
\pgfusepath{clip}%
\pgfsetbuttcap%
\pgfsetroundjoin%
\definecolor{currentfill}{rgb}{0.668054,0.861999,0.196293}%
\pgfsetfillcolor{currentfill}%
\pgfsetfillopacity{0.700000}%
\pgfsetlinewidth{0.501875pt}%
\definecolor{currentstroke}{rgb}{1.000000,1.000000,1.000000}%
\pgfsetstrokecolor{currentstroke}%
\pgfsetstrokeopacity{0.500000}%
\pgfsetdash{}{0pt}%
\pgfpathmoveto{\pgfqpoint{2.878903in}{3.500978in}}%
\pgfpathlineto{\pgfqpoint{2.890038in}{3.507310in}}%
\pgfpathlineto{\pgfqpoint{2.901173in}{3.512968in}}%
\pgfpathlineto{\pgfqpoint{2.912306in}{3.518044in}}%
\pgfpathlineto{\pgfqpoint{2.923436in}{3.522628in}}%
\pgfpathlineto{\pgfqpoint{2.934563in}{3.526813in}}%
\pgfpathlineto{\pgfqpoint{2.928545in}{3.526031in}}%
\pgfpathlineto{\pgfqpoint{2.922534in}{3.524992in}}%
\pgfpathlineto{\pgfqpoint{2.916530in}{3.523784in}}%
\pgfpathlineto{\pgfqpoint{2.910534in}{3.522493in}}%
\pgfpathlineto{\pgfqpoint{2.904543in}{3.521209in}}%
\pgfpathlineto{\pgfqpoint{2.893457in}{3.512978in}}%
\pgfpathlineto{\pgfqpoint{2.882372in}{3.503680in}}%
\pgfpathlineto{\pgfqpoint{2.871290in}{3.493473in}}%
\pgfpathlineto{\pgfqpoint{2.860210in}{3.482515in}}%
\pgfpathlineto{\pgfqpoint{2.849132in}{3.470964in}}%
\pgfpathlineto{\pgfqpoint{2.855070in}{3.477088in}}%
\pgfpathlineto{\pgfqpoint{2.861014in}{3.483365in}}%
\pgfpathlineto{\pgfqpoint{2.866968in}{3.489582in}}%
\pgfpathlineto{\pgfqpoint{2.872930in}{3.495525in}}%
\pgfpathclose%
\pgfusepath{stroke,fill}%
\end{pgfscope}%
\begin{pgfscope}%
\pgfpathrectangle{\pgfqpoint{0.887500in}{0.275000in}}{\pgfqpoint{4.225000in}{4.225000in}}%
\pgfusepath{clip}%
\pgfsetbuttcap%
\pgfsetroundjoin%
\definecolor{currentfill}{rgb}{0.120638,0.625828,0.533488}%
\pgfsetfillcolor{currentfill}%
\pgfsetfillopacity{0.700000}%
\pgfsetlinewidth{0.501875pt}%
\definecolor{currentstroke}{rgb}{1.000000,1.000000,1.000000}%
\pgfsetstrokecolor{currentstroke}%
\pgfsetstrokeopacity{0.500000}%
\pgfsetdash{}{0pt}%
\pgfpathmoveto{\pgfqpoint{1.868931in}{2.876470in}}%
\pgfpathlineto{\pgfqpoint{1.880310in}{2.879905in}}%
\pgfpathlineto{\pgfqpoint{1.891685in}{2.883323in}}%
\pgfpathlineto{\pgfqpoint{1.903055in}{2.886720in}}%
\pgfpathlineto{\pgfqpoint{1.914420in}{2.890094in}}%
\pgfpathlineto{\pgfqpoint{1.925780in}{2.893462in}}%
\pgfpathlineto{\pgfqpoint{1.920079in}{2.901089in}}%
\pgfpathlineto{\pgfqpoint{1.914382in}{2.908705in}}%
\pgfpathlineto{\pgfqpoint{1.908689in}{2.916308in}}%
\pgfpathlineto{\pgfqpoint{1.903001in}{2.923898in}}%
\pgfpathlineto{\pgfqpoint{1.897316in}{2.931476in}}%
\pgfpathlineto{\pgfqpoint{1.885970in}{2.928068in}}%
\pgfpathlineto{\pgfqpoint{1.874619in}{2.924653in}}%
\pgfpathlineto{\pgfqpoint{1.863264in}{2.921214in}}%
\pgfpathlineto{\pgfqpoint{1.851903in}{2.917754in}}%
\pgfpathlineto{\pgfqpoint{1.840538in}{2.914279in}}%
\pgfpathlineto{\pgfqpoint{1.846208in}{2.906747in}}%
\pgfpathlineto{\pgfqpoint{1.851883in}{2.899199in}}%
\pgfpathlineto{\pgfqpoint{1.857561in}{2.891637in}}%
\pgfpathlineto{\pgfqpoint{1.863244in}{2.884061in}}%
\pgfpathclose%
\pgfusepath{stroke,fill}%
\end{pgfscope}%
\begin{pgfscope}%
\pgfpathrectangle{\pgfqpoint{0.887500in}{0.275000in}}{\pgfqpoint{4.225000in}{4.225000in}}%
\pgfusepath{clip}%
\pgfsetbuttcap%
\pgfsetroundjoin%
\definecolor{currentfill}{rgb}{0.906311,0.894855,0.098125}%
\pgfsetfillcolor{currentfill}%
\pgfsetfillopacity{0.700000}%
\pgfsetlinewidth{0.501875pt}%
\definecolor{currentstroke}{rgb}{1.000000,1.000000,1.000000}%
\pgfsetstrokecolor{currentstroke}%
\pgfsetstrokeopacity{0.500000}%
\pgfsetdash{}{0pt}%
\pgfpathmoveto{\pgfqpoint{3.641845in}{3.652464in}}%
\pgfpathlineto{\pgfqpoint{3.652833in}{3.657509in}}%
\pgfpathlineto{\pgfqpoint{3.663818in}{3.662590in}}%
\pgfpathlineto{\pgfqpoint{3.674799in}{3.667686in}}%
\pgfpathlineto{\pgfqpoint{3.685775in}{3.672777in}}%
\pgfpathlineto{\pgfqpoint{3.696747in}{3.677843in}}%
\pgfpathlineto{\pgfqpoint{3.690465in}{3.686133in}}%
\pgfpathlineto{\pgfqpoint{3.684185in}{3.694255in}}%
\pgfpathlineto{\pgfqpoint{3.677906in}{3.702210in}}%
\pgfpathlineto{\pgfqpoint{3.671629in}{3.709993in}}%
\pgfpathlineto{\pgfqpoint{3.665353in}{3.717597in}}%
\pgfpathlineto{\pgfqpoint{3.654393in}{3.712587in}}%
\pgfpathlineto{\pgfqpoint{3.643429in}{3.707539in}}%
\pgfpathlineto{\pgfqpoint{3.632460in}{3.702458in}}%
\pgfpathlineto{\pgfqpoint{3.621486in}{3.697349in}}%
\pgfpathlineto{\pgfqpoint{3.610508in}{3.692218in}}%
\pgfpathlineto{\pgfqpoint{3.616772in}{3.684573in}}%
\pgfpathlineto{\pgfqpoint{3.623037in}{3.676765in}}%
\pgfpathlineto{\pgfqpoint{3.629304in}{3.668806in}}%
\pgfpathlineto{\pgfqpoint{3.635574in}{3.660707in}}%
\pgfpathclose%
\pgfusepath{stroke,fill}%
\end{pgfscope}%
\begin{pgfscope}%
\pgfpathrectangle{\pgfqpoint{0.887500in}{0.275000in}}{\pgfqpoint{4.225000in}{4.225000in}}%
\pgfusepath{clip}%
\pgfsetbuttcap%
\pgfsetroundjoin%
\definecolor{currentfill}{rgb}{0.824940,0.884720,0.106217}%
\pgfsetfillcolor{currentfill}%
\pgfsetfillopacity{0.700000}%
\pgfsetlinewidth{0.501875pt}%
\definecolor{currentstroke}{rgb}{1.000000,1.000000,1.000000}%
\pgfsetstrokecolor{currentstroke}%
\pgfsetstrokeopacity{0.500000}%
\pgfsetdash{}{0pt}%
\pgfpathmoveto{\pgfqpoint{3.217512in}{3.598856in}}%
\pgfpathlineto{\pgfqpoint{3.228581in}{3.602973in}}%
\pgfpathlineto{\pgfqpoint{3.239645in}{3.607167in}}%
\pgfpathlineto{\pgfqpoint{3.250706in}{3.611458in}}%
\pgfpathlineto{\pgfqpoint{3.261762in}{3.615870in}}%
\pgfpathlineto{\pgfqpoint{3.272815in}{3.620426in}}%
\pgfpathlineto{\pgfqpoint{3.266642in}{3.624320in}}%
\pgfpathlineto{\pgfqpoint{3.260472in}{3.627976in}}%
\pgfpathlineto{\pgfqpoint{3.254307in}{3.631390in}}%
\pgfpathlineto{\pgfqpoint{3.248146in}{3.634557in}}%
\pgfpathlineto{\pgfqpoint{3.241989in}{3.637470in}}%
\pgfpathlineto{\pgfqpoint{3.230954in}{3.632744in}}%
\pgfpathlineto{\pgfqpoint{3.219914in}{3.628075in}}%
\pgfpathlineto{\pgfqpoint{3.208869in}{3.623402in}}%
\pgfpathlineto{\pgfqpoint{3.197820in}{3.618669in}}%
\pgfpathlineto{\pgfqpoint{3.186767in}{3.613828in}}%
\pgfpathlineto{\pgfqpoint{3.192906in}{3.611265in}}%
\pgfpathlineto{\pgfqpoint{3.199050in}{3.608471in}}%
\pgfpathlineto{\pgfqpoint{3.205199in}{3.605462in}}%
\pgfpathlineto{\pgfqpoint{3.211353in}{3.602252in}}%
\pgfpathclose%
\pgfusepath{stroke,fill}%
\end{pgfscope}%
\begin{pgfscope}%
\pgfpathrectangle{\pgfqpoint{0.887500in}{0.275000in}}{\pgfqpoint{4.225000in}{4.225000in}}%
\pgfusepath{clip}%
\pgfsetbuttcap%
\pgfsetroundjoin%
\definecolor{currentfill}{rgb}{0.430983,0.808473,0.346476}%
\pgfsetfillcolor{currentfill}%
\pgfsetfillopacity{0.700000}%
\pgfsetlinewidth{0.501875pt}%
\definecolor{currentstroke}{rgb}{1.000000,1.000000,1.000000}%
\pgfsetstrokecolor{currentstroke}%
\pgfsetstrokeopacity{0.500000}%
\pgfsetdash{}{0pt}%
\pgfpathmoveto{\pgfqpoint{2.712518in}{3.279322in}}%
\pgfpathlineto{\pgfqpoint{2.723536in}{3.302206in}}%
\pgfpathlineto{\pgfqpoint{2.734567in}{3.324086in}}%
\pgfpathlineto{\pgfqpoint{2.745612in}{3.344899in}}%
\pgfpathlineto{\pgfqpoint{2.756669in}{3.364582in}}%
\pgfpathlineto{\pgfqpoint{2.767739in}{3.383070in}}%
\pgfpathlineto{\pgfqpoint{2.761841in}{3.377178in}}%
\pgfpathlineto{\pgfqpoint{2.755964in}{3.369609in}}%
\pgfpathlineto{\pgfqpoint{2.750104in}{3.360802in}}%
\pgfpathlineto{\pgfqpoint{2.744260in}{3.351196in}}%
\pgfpathlineto{\pgfqpoint{2.738427in}{3.341228in}}%
\pgfpathlineto{\pgfqpoint{2.727368in}{3.326870in}}%
\pgfpathlineto{\pgfqpoint{2.716315in}{3.311990in}}%
\pgfpathlineto{\pgfqpoint{2.705268in}{3.296466in}}%
\pgfpathlineto{\pgfqpoint{2.694230in}{3.280175in}}%
\pgfpathlineto{\pgfqpoint{2.683202in}{3.262998in}}%
\pgfpathlineto{\pgfqpoint{2.689022in}{3.269460in}}%
\pgfpathlineto{\pgfqpoint{2.694860in}{3.274805in}}%
\pgfpathlineto{\pgfqpoint{2.700719in}{3.278550in}}%
\pgfpathlineto{\pgfqpoint{2.706604in}{3.280217in}}%
\pgfpathclose%
\pgfusepath{stroke,fill}%
\end{pgfscope}%
\begin{pgfscope}%
\pgfpathrectangle{\pgfqpoint{0.887500in}{0.275000in}}{\pgfqpoint{4.225000in}{4.225000in}}%
\pgfusepath{clip}%
\pgfsetbuttcap%
\pgfsetroundjoin%
\definecolor{currentfill}{rgb}{0.886271,0.892374,0.095374}%
\pgfsetfillcolor{currentfill}%
\pgfsetfillopacity{0.700000}%
\pgfsetlinewidth{0.501875pt}%
\definecolor{currentstroke}{rgb}{1.000000,1.000000,1.000000}%
\pgfsetstrokecolor{currentstroke}%
\pgfsetstrokeopacity{0.500000}%
\pgfsetdash{}{0pt}%
\pgfpathmoveto{\pgfqpoint{3.869192in}{3.630246in}}%
\pgfpathlineto{\pgfqpoint{3.880125in}{3.634734in}}%
\pgfpathlineto{\pgfqpoint{3.891053in}{3.639235in}}%
\pgfpathlineto{\pgfqpoint{3.901978in}{3.643760in}}%
\pgfpathlineto{\pgfqpoint{3.912899in}{3.648318in}}%
\pgfpathlineto{\pgfqpoint{3.906602in}{3.659178in}}%
\pgfpathlineto{\pgfqpoint{3.900303in}{3.669761in}}%
\pgfpathlineto{\pgfqpoint{3.894002in}{3.680065in}}%
\pgfpathlineto{\pgfqpoint{3.887698in}{3.690093in}}%
\pgfpathlineto{\pgfqpoint{3.881393in}{3.699849in}}%
\pgfpathlineto{\pgfqpoint{3.870478in}{3.695191in}}%
\pgfpathlineto{\pgfqpoint{3.859559in}{3.690552in}}%
\pgfpathlineto{\pgfqpoint{3.848636in}{3.685922in}}%
\pgfpathlineto{\pgfqpoint{3.837708in}{3.681295in}}%
\pgfpathlineto{\pgfqpoint{3.844008in}{3.671595in}}%
\pgfpathlineto{\pgfqpoint{3.850306in}{3.661643in}}%
\pgfpathlineto{\pgfqpoint{3.856603in}{3.651436in}}%
\pgfpathlineto{\pgfqpoint{3.862899in}{3.640970in}}%
\pgfpathclose%
\pgfusepath{stroke,fill}%
\end{pgfscope}%
\begin{pgfscope}%
\pgfpathrectangle{\pgfqpoint{0.887500in}{0.275000in}}{\pgfqpoint{4.225000in}{4.225000in}}%
\pgfusepath{clip}%
\pgfsetbuttcap%
\pgfsetroundjoin%
\definecolor{currentfill}{rgb}{0.202219,0.715272,0.476084}%
\pgfsetfillcolor{currentfill}%
\pgfsetfillopacity{0.700000}%
\pgfsetlinewidth{0.501875pt}%
\definecolor{currentstroke}{rgb}{1.000000,1.000000,1.000000}%
\pgfsetstrokecolor{currentstroke}%
\pgfsetstrokeopacity{0.500000}%
\pgfsetdash{}{0pt}%
\pgfpathmoveto{\pgfqpoint{2.632159in}{3.026108in}}%
\pgfpathlineto{\pgfqpoint{2.643234in}{3.040945in}}%
\pgfpathlineto{\pgfqpoint{2.654300in}{3.056681in}}%
\pgfpathlineto{\pgfqpoint{2.665361in}{3.073186in}}%
\pgfpathlineto{\pgfqpoint{2.676417in}{3.090413in}}%
\pgfpathlineto{\pgfqpoint{2.687467in}{3.108579in}}%
\pgfpathlineto{\pgfqpoint{2.681449in}{3.122179in}}%
\pgfpathlineto{\pgfqpoint{2.675447in}{3.134197in}}%
\pgfpathlineto{\pgfqpoint{2.669468in}{3.144231in}}%
\pgfpathlineto{\pgfqpoint{2.663512in}{3.152158in}}%
\pgfpathlineto{\pgfqpoint{2.657579in}{3.158198in}}%
\pgfpathlineto{\pgfqpoint{2.646596in}{3.135308in}}%
\pgfpathlineto{\pgfqpoint{2.635605in}{3.113781in}}%
\pgfpathlineto{\pgfqpoint{2.624601in}{3.093955in}}%
\pgfpathlineto{\pgfqpoint{2.613581in}{3.075895in}}%
\pgfpathlineto{\pgfqpoint{2.602544in}{3.059589in}}%
\pgfpathlineto{\pgfqpoint{2.608457in}{3.053040in}}%
\pgfpathlineto{\pgfqpoint{2.614375in}{3.046408in}}%
\pgfpathlineto{\pgfqpoint{2.620298in}{3.039699in}}%
\pgfpathlineto{\pgfqpoint{2.626226in}{3.032925in}}%
\pgfpathclose%
\pgfusepath{stroke,fill}%
\end{pgfscope}%
\begin{pgfscope}%
\pgfpathrectangle{\pgfqpoint{0.887500in}{0.275000in}}{\pgfqpoint{4.225000in}{4.225000in}}%
\pgfusepath{clip}%
\pgfsetbuttcap%
\pgfsetroundjoin%
\definecolor{currentfill}{rgb}{0.866013,0.889868,0.095953}%
\pgfsetfillcolor{currentfill}%
\pgfsetfillopacity{0.700000}%
\pgfsetlinewidth{0.501875pt}%
\definecolor{currentstroke}{rgb}{1.000000,1.000000,1.000000}%
\pgfsetstrokecolor{currentstroke}%
\pgfsetstrokeopacity{0.500000}%
\pgfsetdash{}{0pt}%
\pgfpathmoveto{\pgfqpoint{3.359019in}{3.620602in}}%
\pgfpathlineto{\pgfqpoint{3.370066in}{3.625895in}}%
\pgfpathlineto{\pgfqpoint{3.381111in}{3.631330in}}%
\pgfpathlineto{\pgfqpoint{3.392152in}{3.636859in}}%
\pgfpathlineto{\pgfqpoint{3.403190in}{3.642434in}}%
\pgfpathlineto{\pgfqpoint{3.414224in}{3.648008in}}%
\pgfpathlineto{\pgfqpoint{3.408010in}{3.654109in}}%
\pgfpathlineto{\pgfqpoint{3.401797in}{3.659902in}}%
\pgfpathlineto{\pgfqpoint{3.395586in}{3.665338in}}%
\pgfpathlineto{\pgfqpoint{3.389377in}{3.670416in}}%
\pgfpathlineto{\pgfqpoint{3.383170in}{3.675149in}}%
\pgfpathlineto{\pgfqpoint{3.372149in}{3.669284in}}%
\pgfpathlineto{\pgfqpoint{3.361125in}{3.663405in}}%
\pgfpathlineto{\pgfqpoint{3.350097in}{3.657551in}}%
\pgfpathlineto{\pgfqpoint{3.339065in}{3.651762in}}%
\pgfpathlineto{\pgfqpoint{3.328031in}{3.646079in}}%
\pgfpathlineto{\pgfqpoint{3.334223in}{3.641645in}}%
\pgfpathlineto{\pgfqpoint{3.340418in}{3.636883in}}%
\pgfpathlineto{\pgfqpoint{3.346616in}{3.631779in}}%
\pgfpathlineto{\pgfqpoint{3.352816in}{3.626337in}}%
\pgfpathclose%
\pgfusepath{stroke,fill}%
\end{pgfscope}%
\begin{pgfscope}%
\pgfpathrectangle{\pgfqpoint{0.887500in}{0.275000in}}{\pgfqpoint{4.225000in}{4.225000in}}%
\pgfusepath{clip}%
\pgfsetbuttcap%
\pgfsetroundjoin%
\definecolor{currentfill}{rgb}{0.896320,0.893616,0.096335}%
\pgfsetfillcolor{currentfill}%
\pgfsetfillopacity{0.700000}%
\pgfsetlinewidth{0.501875pt}%
\definecolor{currentstroke}{rgb}{1.000000,1.000000,1.000000}%
\pgfsetstrokecolor{currentstroke}%
\pgfsetstrokeopacity{0.500000}%
\pgfsetdash{}{0pt}%
\pgfpathmoveto{\pgfqpoint{3.500500in}{3.640597in}}%
\pgfpathlineto{\pgfqpoint{3.511519in}{3.645739in}}%
\pgfpathlineto{\pgfqpoint{3.522534in}{3.650890in}}%
\pgfpathlineto{\pgfqpoint{3.533545in}{3.656051in}}%
\pgfpathlineto{\pgfqpoint{3.544552in}{3.661220in}}%
\pgfpathlineto{\pgfqpoint{3.555555in}{3.666393in}}%
\pgfpathlineto{\pgfqpoint{3.549305in}{3.673811in}}%
\pgfpathlineto{\pgfqpoint{3.543058in}{3.681095in}}%
\pgfpathlineto{\pgfqpoint{3.536813in}{3.688217in}}%
\pgfpathlineto{\pgfqpoint{3.530570in}{3.695147in}}%
\pgfpathlineto{\pgfqpoint{3.524329in}{3.701857in}}%
\pgfpathlineto{\pgfqpoint{3.513338in}{3.696626in}}%
\pgfpathlineto{\pgfqpoint{3.502343in}{3.691364in}}%
\pgfpathlineto{\pgfqpoint{3.491344in}{3.686069in}}%
\pgfpathlineto{\pgfqpoint{3.480339in}{3.680739in}}%
\pgfpathlineto{\pgfqpoint{3.469331in}{3.675372in}}%
\pgfpathlineto{\pgfqpoint{3.475560in}{3.668751in}}%
\pgfpathlineto{\pgfqpoint{3.481790in}{3.661915in}}%
\pgfpathlineto{\pgfqpoint{3.488023in}{3.654912in}}%
\pgfpathlineto{\pgfqpoint{3.494260in}{3.647790in}}%
\pgfpathclose%
\pgfusepath{stroke,fill}%
\end{pgfscope}%
\begin{pgfscope}%
\pgfpathrectangle{\pgfqpoint{0.887500in}{0.275000in}}{\pgfqpoint{4.225000in}{4.225000in}}%
\pgfusepath{clip}%
\pgfsetbuttcap%
\pgfsetroundjoin%
\definecolor{currentfill}{rgb}{0.119512,0.607464,0.540218}%
\pgfsetfillcolor{currentfill}%
\pgfsetfillopacity{0.700000}%
\pgfsetlinewidth{0.501875pt}%
\definecolor{currentstroke}{rgb}{1.000000,1.000000,1.000000}%
\pgfsetstrokecolor{currentstroke}%
\pgfsetstrokeopacity{0.500000}%
\pgfsetdash{}{0pt}%
\pgfpathmoveto{\pgfqpoint{2.182167in}{2.824637in}}%
\pgfpathlineto{\pgfqpoint{2.193489in}{2.827464in}}%
\pgfpathlineto{\pgfqpoint{2.204785in}{2.831210in}}%
\pgfpathlineto{\pgfqpoint{2.216050in}{2.836212in}}%
\pgfpathlineto{\pgfqpoint{2.227277in}{2.842803in}}%
\pgfpathlineto{\pgfqpoint{2.238467in}{2.851045in}}%
\pgfpathlineto{\pgfqpoint{2.232646in}{2.859307in}}%
\pgfpathlineto{\pgfqpoint{2.226823in}{2.867859in}}%
\pgfpathlineto{\pgfqpoint{2.220996in}{2.876752in}}%
\pgfpathlineto{\pgfqpoint{2.215164in}{2.886025in}}%
\pgfpathlineto{\pgfqpoint{2.209328in}{2.895652in}}%
\pgfpathlineto{\pgfqpoint{2.198150in}{2.887520in}}%
\pgfpathlineto{\pgfqpoint{2.186939in}{2.880768in}}%
\pgfpathlineto{\pgfqpoint{2.175695in}{2.875351in}}%
\pgfpathlineto{\pgfqpoint{2.164423in}{2.871002in}}%
\pgfpathlineto{\pgfqpoint{2.153128in}{2.867450in}}%
\pgfpathlineto{\pgfqpoint{2.158940in}{2.858411in}}%
\pgfpathlineto{\pgfqpoint{2.164749in}{2.849609in}}%
\pgfpathlineto{\pgfqpoint{2.170556in}{2.841063in}}%
\pgfpathlineto{\pgfqpoint{2.176362in}{2.832749in}}%
\pgfpathclose%
\pgfusepath{stroke,fill}%
\end{pgfscope}%
\begin{pgfscope}%
\pgfpathrectangle{\pgfqpoint{0.887500in}{0.275000in}}{\pgfqpoint{4.225000in}{4.225000in}}%
\pgfusepath{clip}%
\pgfsetbuttcap%
\pgfsetroundjoin%
\definecolor{currentfill}{rgb}{0.616293,0.852709,0.230052}%
\pgfsetfillcolor{currentfill}%
\pgfsetfillopacity{0.700000}%
\pgfsetlinewidth{0.501875pt}%
\definecolor{currentstroke}{rgb}{1.000000,1.000000,1.000000}%
\pgfsetstrokecolor{currentstroke}%
\pgfsetstrokeopacity{0.500000}%
\pgfsetdash{}{0pt}%
\pgfpathmoveto{\pgfqpoint{2.823246in}{3.456192in}}%
\pgfpathlineto{\pgfqpoint{2.834371in}{3.467153in}}%
\pgfpathlineto{\pgfqpoint{2.845500in}{3.477043in}}%
\pgfpathlineto{\pgfqpoint{2.856632in}{3.485929in}}%
\pgfpathlineto{\pgfqpoint{2.867767in}{3.493881in}}%
\pgfpathlineto{\pgfqpoint{2.878903in}{3.500978in}}%
\pgfpathlineto{\pgfqpoint{2.872930in}{3.495525in}}%
\pgfpathlineto{\pgfqpoint{2.866968in}{3.489582in}}%
\pgfpathlineto{\pgfqpoint{2.861014in}{3.483365in}}%
\pgfpathlineto{\pgfqpoint{2.855070in}{3.477088in}}%
\pgfpathlineto{\pgfqpoint{2.849132in}{3.470964in}}%
\pgfpathlineto{\pgfqpoint{2.838056in}{3.458977in}}%
\pgfpathlineto{\pgfqpoint{2.826981in}{3.446709in}}%
\pgfpathlineto{\pgfqpoint{2.815906in}{3.434236in}}%
\pgfpathlineto{\pgfqpoint{2.804833in}{3.421561in}}%
\pgfpathlineto{\pgfqpoint{2.793761in}{3.408686in}}%
\pgfpathlineto{\pgfqpoint{2.799639in}{3.418435in}}%
\pgfpathlineto{\pgfqpoint{2.805524in}{3.428383in}}%
\pgfpathlineto{\pgfqpoint{2.811419in}{3.438208in}}%
\pgfpathlineto{\pgfqpoint{2.817326in}{3.447586in}}%
\pgfpathclose%
\pgfusepath{stroke,fill}%
\end{pgfscope}%
\begin{pgfscope}%
\pgfpathrectangle{\pgfqpoint{0.887500in}{0.275000in}}{\pgfqpoint{4.225000in}{4.225000in}}%
\pgfusepath{clip}%
\pgfsetbuttcap%
\pgfsetroundjoin%
\definecolor{currentfill}{rgb}{0.137339,0.662252,0.515571}%
\pgfsetfillcolor{currentfill}%
\pgfsetfillopacity{0.700000}%
\pgfsetlinewidth{0.501875pt}%
\definecolor{currentstroke}{rgb}{1.000000,1.000000,1.000000}%
\pgfsetstrokecolor{currentstroke}%
\pgfsetstrokeopacity{0.500000}%
\pgfsetdash{}{0pt}%
\pgfpathmoveto{\pgfqpoint{2.520644in}{2.939774in}}%
\pgfpathlineto{\pgfqpoint{2.531849in}{2.945129in}}%
\pgfpathlineto{\pgfqpoint{2.543049in}{2.950517in}}%
\pgfpathlineto{\pgfqpoint{2.554242in}{2.956236in}}%
\pgfpathlineto{\pgfqpoint{2.565422in}{2.962581in}}%
\pgfpathlineto{\pgfqpoint{2.576588in}{2.969849in}}%
\pgfpathlineto{\pgfqpoint{2.570692in}{2.975191in}}%
\pgfpathlineto{\pgfqpoint{2.564793in}{2.981173in}}%
\pgfpathlineto{\pgfqpoint{2.558888in}{2.987947in}}%
\pgfpathlineto{\pgfqpoint{2.552977in}{2.995539in}}%
\pgfpathlineto{\pgfqpoint{2.547062in}{3.003820in}}%
\pgfpathlineto{\pgfqpoint{2.535901in}{2.997366in}}%
\pgfpathlineto{\pgfqpoint{2.524724in}{2.991946in}}%
\pgfpathlineto{\pgfqpoint{2.513532in}{2.987242in}}%
\pgfpathlineto{\pgfqpoint{2.502331in}{2.982938in}}%
\pgfpathlineto{\pgfqpoint{2.491123in}{2.978718in}}%
\pgfpathlineto{\pgfqpoint{2.497048in}{2.968924in}}%
\pgfpathlineto{\pgfqpoint{2.502963in}{2.960052in}}%
\pgfpathlineto{\pgfqpoint{2.508866in}{2.952293in}}%
\pgfpathlineto{\pgfqpoint{2.514759in}{2.945609in}}%
\pgfpathclose%
\pgfusepath{stroke,fill}%
\end{pgfscope}%
\begin{pgfscope}%
\pgfpathrectangle{\pgfqpoint{0.887500in}{0.275000in}}{\pgfqpoint{4.225000in}{4.225000in}}%
\pgfusepath{clip}%
\pgfsetbuttcap%
\pgfsetroundjoin%
\definecolor{currentfill}{rgb}{0.535621,0.835785,0.281908}%
\pgfsetfillcolor{currentfill}%
\pgfsetfillopacity{0.700000}%
\pgfsetlinewidth{0.501875pt}%
\definecolor{currentstroke}{rgb}{1.000000,1.000000,1.000000}%
\pgfsetstrokecolor{currentstroke}%
\pgfsetstrokeopacity{0.500000}%
\pgfsetdash{}{0pt}%
\pgfpathmoveto{\pgfqpoint{2.767739in}{3.383070in}}%
\pgfpathlineto{\pgfqpoint{2.778821in}{3.400300in}}%
\pgfpathlineto{\pgfqpoint{2.789913in}{3.416211in}}%
\pgfpathlineto{\pgfqpoint{2.801016in}{3.430788in}}%
\pgfpathlineto{\pgfqpoint{2.812128in}{3.444093in}}%
\pgfpathlineto{\pgfqpoint{2.823246in}{3.456192in}}%
\pgfpathlineto{\pgfqpoint{2.817326in}{3.447586in}}%
\pgfpathlineto{\pgfqpoint{2.811419in}{3.438208in}}%
\pgfpathlineto{\pgfqpoint{2.805524in}{3.428383in}}%
\pgfpathlineto{\pgfqpoint{2.799639in}{3.418435in}}%
\pgfpathlineto{\pgfqpoint{2.793761in}{3.408686in}}%
\pgfpathlineto{\pgfqpoint{2.782691in}{3.395611in}}%
\pgfpathlineto{\pgfqpoint{2.771622in}{3.382336in}}%
\pgfpathlineto{\pgfqpoint{2.760555in}{3.368862in}}%
\pgfpathlineto{\pgfqpoint{2.749490in}{3.355185in}}%
\pgfpathlineto{\pgfqpoint{2.738427in}{3.341228in}}%
\pgfpathlineto{\pgfqpoint{2.744260in}{3.351196in}}%
\pgfpathlineto{\pgfqpoint{2.750104in}{3.360802in}}%
\pgfpathlineto{\pgfqpoint{2.755964in}{3.369609in}}%
\pgfpathlineto{\pgfqpoint{2.761841in}{3.377178in}}%
\pgfpathclose%
\pgfusepath{stroke,fill}%
\end{pgfscope}%
\begin{pgfscope}%
\pgfpathrectangle{\pgfqpoint{0.887500in}{0.275000in}}{\pgfqpoint{4.225000in}{4.225000in}}%
\pgfusepath{clip}%
\pgfsetbuttcap%
\pgfsetroundjoin%
\definecolor{currentfill}{rgb}{0.119699,0.618490,0.536347}%
\pgfsetfillcolor{currentfill}%
\pgfsetfillopacity{0.700000}%
\pgfsetlinewidth{0.501875pt}%
\definecolor{currentstroke}{rgb}{1.000000,1.000000,1.000000}%
\pgfsetstrokecolor{currentstroke}%
\pgfsetstrokeopacity{0.500000}%
\pgfsetdash{}{0pt}%
\pgfpathmoveto{\pgfqpoint{1.954344in}{2.855137in}}%
\pgfpathlineto{\pgfqpoint{1.965712in}{2.858487in}}%
\pgfpathlineto{\pgfqpoint{1.977073in}{2.861877in}}%
\pgfpathlineto{\pgfqpoint{1.988427in}{2.865332in}}%
\pgfpathlineto{\pgfqpoint{1.999773in}{2.868874in}}%
\pgfpathlineto{\pgfqpoint{2.011111in}{2.872527in}}%
\pgfpathlineto{\pgfqpoint{2.005377in}{2.880240in}}%
\pgfpathlineto{\pgfqpoint{1.999647in}{2.887941in}}%
\pgfpathlineto{\pgfqpoint{1.993920in}{2.895630in}}%
\pgfpathlineto{\pgfqpoint{1.988198in}{2.903308in}}%
\pgfpathlineto{\pgfqpoint{1.982480in}{2.910974in}}%
\pgfpathlineto{\pgfqpoint{1.971155in}{2.907308in}}%
\pgfpathlineto{\pgfqpoint{1.959822in}{2.903746in}}%
\pgfpathlineto{\pgfqpoint{1.948481in}{2.900265in}}%
\pgfpathlineto{\pgfqpoint{1.937134in}{2.896845in}}%
\pgfpathlineto{\pgfqpoint{1.925780in}{2.893462in}}%
\pgfpathlineto{\pgfqpoint{1.931485in}{2.885821in}}%
\pgfpathlineto{\pgfqpoint{1.937193in}{2.878169in}}%
\pgfpathlineto{\pgfqpoint{1.942906in}{2.870504in}}%
\pgfpathlineto{\pgfqpoint{1.948623in}{2.862827in}}%
\pgfpathclose%
\pgfusepath{stroke,fill}%
\end{pgfscope}%
\begin{pgfscope}%
\pgfpathrectangle{\pgfqpoint{0.887500in}{0.275000in}}{\pgfqpoint{4.225000in}{4.225000in}}%
\pgfusepath{clip}%
\pgfsetbuttcap%
\pgfsetroundjoin%
\definecolor{currentfill}{rgb}{0.741388,0.873449,0.149561}%
\pgfsetfillcolor{currentfill}%
\pgfsetfillopacity{0.700000}%
\pgfsetlinewidth{0.501875pt}%
\definecolor{currentstroke}{rgb}{1.000000,1.000000,1.000000}%
\pgfsetstrokecolor{currentstroke}%
\pgfsetstrokeopacity{0.500000}%
\pgfsetdash{}{0pt}%
\pgfpathmoveto{\pgfqpoint{3.020442in}{3.541429in}}%
\pgfpathlineto{\pgfqpoint{3.031561in}{3.545959in}}%
\pgfpathlineto{\pgfqpoint{3.042675in}{3.550609in}}%
\pgfpathlineto{\pgfqpoint{3.053785in}{3.555286in}}%
\pgfpathlineto{\pgfqpoint{3.064891in}{3.559965in}}%
\pgfpathlineto{\pgfqpoint{3.075992in}{3.564660in}}%
\pgfpathlineto{\pgfqpoint{3.069896in}{3.565082in}}%
\pgfpathlineto{\pgfqpoint{3.063806in}{3.565481in}}%
\pgfpathlineto{\pgfqpoint{3.057722in}{3.565912in}}%
\pgfpathlineto{\pgfqpoint{3.051645in}{3.566403in}}%
\pgfpathlineto{\pgfqpoint{3.045573in}{3.566955in}}%
\pgfpathlineto{\pgfqpoint{3.034494in}{3.561930in}}%
\pgfpathlineto{\pgfqpoint{3.023411in}{3.557200in}}%
\pgfpathlineto{\pgfqpoint{3.012323in}{3.552815in}}%
\pgfpathlineto{\pgfqpoint{3.001230in}{3.548782in}}%
\pgfpathlineto{\pgfqpoint{2.990131in}{3.545015in}}%
\pgfpathlineto{\pgfqpoint{2.996181in}{3.544436in}}%
\pgfpathlineto{\pgfqpoint{3.002237in}{3.543804in}}%
\pgfpathlineto{\pgfqpoint{3.008299in}{3.543103in}}%
\pgfpathlineto{\pgfqpoint{3.014367in}{3.542318in}}%
\pgfpathclose%
\pgfusepath{stroke,fill}%
\end{pgfscope}%
\begin{pgfscope}%
\pgfpathrectangle{\pgfqpoint{0.887500in}{0.275000in}}{\pgfqpoint{4.225000in}{4.225000in}}%
\pgfusepath{clip}%
\pgfsetbuttcap%
\pgfsetroundjoin%
\definecolor{currentfill}{rgb}{0.896320,0.893616,0.096335}%
\pgfsetfillcolor{currentfill}%
\pgfsetfillopacity{0.700000}%
\pgfsetlinewidth{0.501875pt}%
\definecolor{currentstroke}{rgb}{1.000000,1.000000,1.000000}%
\pgfsetstrokecolor{currentstroke}%
\pgfsetstrokeopacity{0.500000}%
\pgfsetdash{}{0pt}%
\pgfpathmoveto{\pgfqpoint{3.728170in}{3.633551in}}%
\pgfpathlineto{\pgfqpoint{3.739147in}{3.638557in}}%
\pgfpathlineto{\pgfqpoint{3.750118in}{3.643493in}}%
\pgfpathlineto{\pgfqpoint{3.761085in}{3.648364in}}%
\pgfpathlineto{\pgfqpoint{3.772046in}{3.653179in}}%
\pgfpathlineto{\pgfqpoint{3.783002in}{3.657945in}}%
\pgfpathlineto{\pgfqpoint{3.776708in}{3.667288in}}%
\pgfpathlineto{\pgfqpoint{3.770414in}{3.676407in}}%
\pgfpathlineto{\pgfqpoint{3.764120in}{3.685307in}}%
\pgfpathlineto{\pgfqpoint{3.757826in}{3.693995in}}%
\pgfpathlineto{\pgfqpoint{3.751533in}{3.702478in}}%
\pgfpathlineto{\pgfqpoint{3.740586in}{3.697639in}}%
\pgfpathlineto{\pgfqpoint{3.729633in}{3.692758in}}%
\pgfpathlineto{\pgfqpoint{3.718676in}{3.687834in}}%
\pgfpathlineto{\pgfqpoint{3.707714in}{3.682863in}}%
\pgfpathlineto{\pgfqpoint{3.696747in}{3.677843in}}%
\pgfpathlineto{\pgfqpoint{3.703030in}{3.669375in}}%
\pgfpathlineto{\pgfqpoint{3.709314in}{3.660720in}}%
\pgfpathlineto{\pgfqpoint{3.715599in}{3.651871in}}%
\pgfpathlineto{\pgfqpoint{3.721884in}{3.642817in}}%
\pgfpathclose%
\pgfusepath{stroke,fill}%
\end{pgfscope}%
\begin{pgfscope}%
\pgfpathrectangle{\pgfqpoint{0.887500in}{0.275000in}}{\pgfqpoint{4.225000in}{4.225000in}}%
\pgfusepath{clip}%
\pgfsetbuttcap%
\pgfsetroundjoin%
\definecolor{currentfill}{rgb}{0.132268,0.655014,0.519661}%
\pgfsetfillcolor{currentfill}%
\pgfsetfillopacity{0.700000}%
\pgfsetlinewidth{0.501875pt}%
\definecolor{currentstroke}{rgb}{1.000000,1.000000,1.000000}%
\pgfsetstrokecolor{currentstroke}%
\pgfsetstrokeopacity{0.500000}%
\pgfsetdash{}{0pt}%
\pgfpathmoveto{\pgfqpoint{2.379167in}{2.913796in}}%
\pgfpathlineto{\pgfqpoint{2.390346in}{2.922260in}}%
\pgfpathlineto{\pgfqpoint{2.401530in}{2.930291in}}%
\pgfpathlineto{\pgfqpoint{2.412717in}{2.937899in}}%
\pgfpathlineto{\pgfqpoint{2.423909in}{2.945080in}}%
\pgfpathlineto{\pgfqpoint{2.435104in}{2.951826in}}%
\pgfpathlineto{\pgfqpoint{2.429188in}{2.961950in}}%
\pgfpathlineto{\pgfqpoint{2.423272in}{2.972260in}}%
\pgfpathlineto{\pgfqpoint{2.417359in}{2.982639in}}%
\pgfpathlineto{\pgfqpoint{2.411449in}{2.992971in}}%
\pgfpathlineto{\pgfqpoint{2.405545in}{3.003140in}}%
\pgfpathlineto{\pgfqpoint{2.394389in}{2.994657in}}%
\pgfpathlineto{\pgfqpoint{2.383238in}{2.985787in}}%
\pgfpathlineto{\pgfqpoint{2.372089in}{2.976682in}}%
\pgfpathlineto{\pgfqpoint{2.360940in}{2.967417in}}%
\pgfpathlineto{\pgfqpoint{2.349793in}{2.957948in}}%
\pgfpathlineto{\pgfqpoint{2.355656in}{2.949368in}}%
\pgfpathlineto{\pgfqpoint{2.361526in}{2.940654in}}%
\pgfpathlineto{\pgfqpoint{2.367401in}{2.931814in}}%
\pgfpathlineto{\pgfqpoint{2.373281in}{2.922859in}}%
\pgfpathclose%
\pgfusepath{stroke,fill}%
\end{pgfscope}%
\begin{pgfscope}%
\pgfpathrectangle{\pgfqpoint{0.887500in}{0.275000in}}{\pgfqpoint{4.225000in}{4.225000in}}%
\pgfusepath{clip}%
\pgfsetbuttcap%
\pgfsetroundjoin%
\definecolor{currentfill}{rgb}{0.122312,0.633153,0.530398}%
\pgfsetfillcolor{currentfill}%
\pgfsetfillopacity{0.700000}%
\pgfsetlinewidth{0.501875pt}%
\definecolor{currentstroke}{rgb}{1.000000,1.000000,1.000000}%
\pgfsetstrokecolor{currentstroke}%
\pgfsetstrokeopacity{0.500000}%
\pgfsetdash{}{0pt}%
\pgfpathmoveto{\pgfqpoint{1.726577in}{2.879919in}}%
\pgfpathlineto{\pgfqpoint{1.738000in}{2.883296in}}%
\pgfpathlineto{\pgfqpoint{1.749416in}{2.886683in}}%
\pgfpathlineto{\pgfqpoint{1.760827in}{2.890083in}}%
\pgfpathlineto{\pgfqpoint{1.772232in}{2.893497in}}%
\pgfpathlineto{\pgfqpoint{1.783630in}{2.896927in}}%
\pgfpathlineto{\pgfqpoint{1.777979in}{2.904409in}}%
\pgfpathlineto{\pgfqpoint{1.772331in}{2.911874in}}%
\pgfpathlineto{\pgfqpoint{1.766687in}{2.919322in}}%
\pgfpathlineto{\pgfqpoint{1.761048in}{2.926753in}}%
\pgfpathlineto{\pgfqpoint{1.749660in}{2.923315in}}%
\pgfpathlineto{\pgfqpoint{1.738266in}{2.919896in}}%
\pgfpathlineto{\pgfqpoint{1.726866in}{2.916493in}}%
\pgfpathlineto{\pgfqpoint{1.715460in}{2.913104in}}%
\pgfpathlineto{\pgfqpoint{1.704049in}{2.909723in}}%
\pgfpathlineto{\pgfqpoint{1.709674in}{2.902293in}}%
\pgfpathlineto{\pgfqpoint{1.715305in}{2.894849in}}%
\pgfpathlineto{\pgfqpoint{1.720939in}{2.887391in}}%
\pgfpathclose%
\pgfusepath{stroke,fill}%
\end{pgfscope}%
\begin{pgfscope}%
\pgfpathrectangle{\pgfqpoint{0.887500in}{0.275000in}}{\pgfqpoint{4.225000in}{4.225000in}}%
\pgfusepath{clip}%
\pgfsetbuttcap%
\pgfsetroundjoin%
\definecolor{currentfill}{rgb}{0.886271,0.892374,0.095374}%
\pgfsetfillcolor{currentfill}%
\pgfsetfillopacity{0.700000}%
\pgfsetlinewidth{0.501875pt}%
\definecolor{currentstroke}{rgb}{1.000000,1.000000,1.000000}%
\pgfsetstrokecolor{currentstroke}%
\pgfsetstrokeopacity{0.500000}%
\pgfsetdash{}{0pt}%
\pgfpathmoveto{\pgfqpoint{3.586855in}{3.628276in}}%
\pgfpathlineto{\pgfqpoint{3.597860in}{3.632977in}}%
\pgfpathlineto{\pgfqpoint{3.608861in}{3.637735in}}%
\pgfpathlineto{\pgfqpoint{3.619859in}{3.642564in}}%
\pgfpathlineto{\pgfqpoint{3.630854in}{3.647475in}}%
\pgfpathlineto{\pgfqpoint{3.641845in}{3.652464in}}%
\pgfpathlineto{\pgfqpoint{3.635574in}{3.660707in}}%
\pgfpathlineto{\pgfqpoint{3.629304in}{3.668806in}}%
\pgfpathlineto{\pgfqpoint{3.623037in}{3.676765in}}%
\pgfpathlineto{\pgfqpoint{3.616772in}{3.684573in}}%
\pgfpathlineto{\pgfqpoint{3.610508in}{3.692218in}}%
\pgfpathlineto{\pgfqpoint{3.599526in}{3.687070in}}%
\pgfpathlineto{\pgfqpoint{3.588540in}{3.681910in}}%
\pgfpathlineto{\pgfqpoint{3.577549in}{3.676741in}}%
\pgfpathlineto{\pgfqpoint{3.566554in}{3.671568in}}%
\pgfpathlineto{\pgfqpoint{3.555555in}{3.666393in}}%
\pgfpathlineto{\pgfqpoint{3.561808in}{3.658871in}}%
\pgfpathlineto{\pgfqpoint{3.568064in}{3.651275in}}%
\pgfpathlineto{\pgfqpoint{3.574324in}{3.643632in}}%
\pgfpathlineto{\pgfqpoint{3.580588in}{3.635969in}}%
\pgfpathclose%
\pgfusepath{stroke,fill}%
\end{pgfscope}%
\begin{pgfscope}%
\pgfpathrectangle{\pgfqpoint{0.887500in}{0.275000in}}{\pgfqpoint{4.225000in}{4.225000in}}%
\pgfusepath{clip}%
\pgfsetbuttcap%
\pgfsetroundjoin%
\definecolor{currentfill}{rgb}{0.119483,0.614817,0.537692}%
\pgfsetfillcolor{currentfill}%
\pgfsetfillopacity{0.700000}%
\pgfsetlinewidth{0.501875pt}%
\definecolor{currentstroke}{rgb}{1.000000,1.000000,1.000000}%
\pgfsetstrokecolor{currentstroke}%
\pgfsetstrokeopacity{0.500000}%
\pgfsetdash{}{0pt}%
\pgfpathmoveto{\pgfqpoint{2.039843in}{2.833780in}}%
\pgfpathlineto{\pgfqpoint{2.051186in}{2.837541in}}%
\pgfpathlineto{\pgfqpoint{2.062522in}{2.841363in}}%
\pgfpathlineto{\pgfqpoint{2.073854in}{2.845173in}}%
\pgfpathlineto{\pgfqpoint{2.085182in}{2.848901in}}%
\pgfpathlineto{\pgfqpoint{2.096509in}{2.852473in}}%
\pgfpathlineto{\pgfqpoint{2.090736in}{2.860466in}}%
\pgfpathlineto{\pgfqpoint{2.084967in}{2.868472in}}%
\pgfpathlineto{\pgfqpoint{2.079201in}{2.876486in}}%
\pgfpathlineto{\pgfqpoint{2.073439in}{2.884506in}}%
\pgfpathlineto{\pgfqpoint{2.067681in}{2.892528in}}%
\pgfpathlineto{\pgfqpoint{2.056382in}{2.888329in}}%
\pgfpathlineto{\pgfqpoint{2.045076in}{2.884219in}}%
\pgfpathlineto{\pgfqpoint{2.033762in}{2.880208in}}%
\pgfpathlineto{\pgfqpoint{2.022441in}{2.876307in}}%
\pgfpathlineto{\pgfqpoint{2.011111in}{2.872527in}}%
\pgfpathlineto{\pgfqpoint{2.016850in}{2.864802in}}%
\pgfpathlineto{\pgfqpoint{2.022592in}{2.857065in}}%
\pgfpathlineto{\pgfqpoint{2.028338in}{2.849316in}}%
\pgfpathlineto{\pgfqpoint{2.034089in}{2.841554in}}%
\pgfpathclose%
\pgfusepath{stroke,fill}%
\end{pgfscope}%
\begin{pgfscope}%
\pgfpathrectangle{\pgfqpoint{0.887500in}{0.275000in}}{\pgfqpoint{4.225000in}{4.225000in}}%
\pgfusepath{clip}%
\pgfsetbuttcap%
\pgfsetroundjoin%
\definecolor{currentfill}{rgb}{0.876168,0.891125,0.095250}%
\pgfsetfillcolor{currentfill}%
\pgfsetfillopacity{0.700000}%
\pgfsetlinewidth{0.501875pt}%
\definecolor{currentstroke}{rgb}{1.000000,1.000000,1.000000}%
\pgfsetstrokecolor{currentstroke}%
\pgfsetstrokeopacity{0.500000}%
\pgfsetdash{}{0pt}%
\pgfpathmoveto{\pgfqpoint{3.814460in}{3.607626in}}%
\pgfpathlineto{\pgfqpoint{3.825416in}{3.612209in}}%
\pgfpathlineto{\pgfqpoint{3.836367in}{3.616752in}}%
\pgfpathlineto{\pgfqpoint{3.847314in}{3.621266in}}%
\pgfpathlineto{\pgfqpoint{3.858255in}{3.625760in}}%
\pgfpathlineto{\pgfqpoint{3.869192in}{3.630246in}}%
\pgfpathlineto{\pgfqpoint{3.862899in}{3.640970in}}%
\pgfpathlineto{\pgfqpoint{3.856603in}{3.651436in}}%
\pgfpathlineto{\pgfqpoint{3.850306in}{3.661643in}}%
\pgfpathlineto{\pgfqpoint{3.844008in}{3.671595in}}%
\pgfpathlineto{\pgfqpoint{3.837708in}{3.681295in}}%
\pgfpathlineto{\pgfqpoint{3.826776in}{3.676664in}}%
\pgfpathlineto{\pgfqpoint{3.815840in}{3.672021in}}%
\pgfpathlineto{\pgfqpoint{3.804899in}{3.667358in}}%
\pgfpathlineto{\pgfqpoint{3.793953in}{3.662668in}}%
\pgfpathlineto{\pgfqpoint{3.783002in}{3.657945in}}%
\pgfpathlineto{\pgfqpoint{3.789295in}{3.648369in}}%
\pgfpathlineto{\pgfqpoint{3.795588in}{3.638556in}}%
\pgfpathlineto{\pgfqpoint{3.801880in}{3.628498in}}%
\pgfpathlineto{\pgfqpoint{3.808170in}{3.618188in}}%
\pgfpathclose%
\pgfusepath{stroke,fill}%
\end{pgfscope}%
\begin{pgfscope}%
\pgfpathrectangle{\pgfqpoint{0.887500in}{0.275000in}}{\pgfqpoint{4.225000in}{4.225000in}}%
\pgfusepath{clip}%
\pgfsetbuttcap%
\pgfsetroundjoin%
\definecolor{currentfill}{rgb}{0.804182,0.882046,0.114965}%
\pgfsetfillcolor{currentfill}%
\pgfsetfillopacity{0.700000}%
\pgfsetlinewidth{0.501875pt}%
\definecolor{currentstroke}{rgb}{1.000000,1.000000,1.000000}%
\pgfsetstrokecolor{currentstroke}%
\pgfsetstrokeopacity{0.500000}%
\pgfsetdash{}{0pt}%
\pgfpathmoveto{\pgfqpoint{3.162098in}{3.579296in}}%
\pgfpathlineto{\pgfqpoint{3.173190in}{3.583078in}}%
\pgfpathlineto{\pgfqpoint{3.184277in}{3.586923in}}%
\pgfpathlineto{\pgfqpoint{3.195360in}{3.590833in}}%
\pgfpathlineto{\pgfqpoint{3.206438in}{3.594810in}}%
\pgfpathlineto{\pgfqpoint{3.217512in}{3.598856in}}%
\pgfpathlineto{\pgfqpoint{3.211353in}{3.602252in}}%
\pgfpathlineto{\pgfqpoint{3.205199in}{3.605462in}}%
\pgfpathlineto{\pgfqpoint{3.199050in}{3.608471in}}%
\pgfpathlineto{\pgfqpoint{3.192906in}{3.611265in}}%
\pgfpathlineto{\pgfqpoint{3.186767in}{3.613828in}}%
\pgfpathlineto{\pgfqpoint{3.175708in}{3.608893in}}%
\pgfpathlineto{\pgfqpoint{3.164646in}{3.603895in}}%
\pgfpathlineto{\pgfqpoint{3.153579in}{3.598862in}}%
\pgfpathlineto{\pgfqpoint{3.142508in}{3.593824in}}%
\pgfpathlineto{\pgfqpoint{3.131432in}{3.588810in}}%
\pgfpathlineto{\pgfqpoint{3.137554in}{3.587282in}}%
\pgfpathlineto{\pgfqpoint{3.143682in}{3.585589in}}%
\pgfpathlineto{\pgfqpoint{3.149816in}{3.583708in}}%
\pgfpathlineto{\pgfqpoint{3.155954in}{3.581618in}}%
\pgfpathclose%
\pgfusepath{stroke,fill}%
\end{pgfscope}%
\begin{pgfscope}%
\pgfpathrectangle{\pgfqpoint{0.887500in}{0.275000in}}{\pgfqpoint{4.225000in}{4.225000in}}%
\pgfusepath{clip}%
\pgfsetbuttcap%
\pgfsetroundjoin%
\definecolor{currentfill}{rgb}{0.123444,0.636809,0.528763}%
\pgfsetfillcolor{currentfill}%
\pgfsetfillopacity{0.700000}%
\pgfsetlinewidth{0.501875pt}%
\definecolor{currentstroke}{rgb}{1.000000,1.000000,1.000000}%
\pgfsetstrokecolor{currentstroke}%
\pgfsetstrokeopacity{0.500000}%
\pgfsetdash{}{0pt}%
\pgfpathmoveto{\pgfqpoint{2.323361in}{2.864679in}}%
\pgfpathlineto{\pgfqpoint{2.334509in}{2.875436in}}%
\pgfpathlineto{\pgfqpoint{2.345664in}{2.885720in}}%
\pgfpathlineto{\pgfqpoint{2.356826in}{2.895535in}}%
\pgfpathlineto{\pgfqpoint{2.367994in}{2.904891in}}%
\pgfpathlineto{\pgfqpoint{2.379167in}{2.913796in}}%
\pgfpathlineto{\pgfqpoint{2.373281in}{2.922859in}}%
\pgfpathlineto{\pgfqpoint{2.367401in}{2.931814in}}%
\pgfpathlineto{\pgfqpoint{2.361526in}{2.940654in}}%
\pgfpathlineto{\pgfqpoint{2.355656in}{2.949368in}}%
\pgfpathlineto{\pgfqpoint{2.349793in}{2.957948in}}%
\pgfpathlineto{\pgfqpoint{2.338648in}{2.948222in}}%
\pgfpathlineto{\pgfqpoint{2.327508in}{2.938186in}}%
\pgfpathlineto{\pgfqpoint{2.316372in}{2.927786in}}%
\pgfpathlineto{\pgfqpoint{2.305243in}{2.916970in}}%
\pgfpathlineto{\pgfqpoint{2.294122in}{2.905692in}}%
\pgfpathlineto{\pgfqpoint{2.299962in}{2.897541in}}%
\pgfpathlineto{\pgfqpoint{2.305805in}{2.889397in}}%
\pgfpathlineto{\pgfqpoint{2.311652in}{2.881228in}}%
\pgfpathlineto{\pgfqpoint{2.317504in}{2.872999in}}%
\pgfpathclose%
\pgfusepath{stroke,fill}%
\end{pgfscope}%
\begin{pgfscope}%
\pgfpathrectangle{\pgfqpoint{0.887500in}{0.275000in}}{\pgfqpoint{4.225000in}{4.225000in}}%
\pgfusepath{clip}%
\pgfsetbuttcap%
\pgfsetroundjoin%
\definecolor{currentfill}{rgb}{0.845561,0.887322,0.099702}%
\pgfsetfillcolor{currentfill}%
\pgfsetfillopacity{0.700000}%
\pgfsetlinewidth{0.501875pt}%
\definecolor{currentstroke}{rgb}{1.000000,1.000000,1.000000}%
\pgfsetstrokecolor{currentstroke}%
\pgfsetstrokeopacity{0.500000}%
\pgfsetdash{}{0pt}%
\pgfpathmoveto{\pgfqpoint{3.303740in}{3.597616in}}%
\pgfpathlineto{\pgfqpoint{3.314802in}{3.601709in}}%
\pgfpathlineto{\pgfqpoint{3.325860in}{3.606035in}}%
\pgfpathlineto{\pgfqpoint{3.336915in}{3.610628in}}%
\pgfpathlineto{\pgfqpoint{3.347968in}{3.615497in}}%
\pgfpathlineto{\pgfqpoint{3.359019in}{3.620602in}}%
\pgfpathlineto{\pgfqpoint{3.352816in}{3.626337in}}%
\pgfpathlineto{\pgfqpoint{3.346616in}{3.631779in}}%
\pgfpathlineto{\pgfqpoint{3.340418in}{3.636883in}}%
\pgfpathlineto{\pgfqpoint{3.334223in}{3.641645in}}%
\pgfpathlineto{\pgfqpoint{3.328031in}{3.646079in}}%
\pgfpathlineto{\pgfqpoint{3.316993in}{3.640543in}}%
\pgfpathlineto{\pgfqpoint{3.305953in}{3.635193in}}%
\pgfpathlineto{\pgfqpoint{3.294910in}{3.630064in}}%
\pgfpathlineto{\pgfqpoint{3.283864in}{3.625149in}}%
\pgfpathlineto{\pgfqpoint{3.272815in}{3.620426in}}%
\pgfpathlineto{\pgfqpoint{3.278992in}{3.616300in}}%
\pgfpathlineto{\pgfqpoint{3.285174in}{3.611948in}}%
\pgfpathlineto{\pgfqpoint{3.291359in}{3.607375in}}%
\pgfpathlineto{\pgfqpoint{3.297548in}{3.602590in}}%
\pgfpathclose%
\pgfusepath{stroke,fill}%
\end{pgfscope}%
\begin{pgfscope}%
\pgfpathrectangle{\pgfqpoint{0.887500in}{0.275000in}}{\pgfqpoint{4.225000in}{4.225000in}}%
\pgfusepath{clip}%
\pgfsetbuttcap%
\pgfsetroundjoin%
\definecolor{currentfill}{rgb}{0.876168,0.891125,0.095250}%
\pgfsetfillcolor{currentfill}%
\pgfsetfillopacity{0.700000}%
\pgfsetlinewidth{0.501875pt}%
\definecolor{currentstroke}{rgb}{1.000000,1.000000,1.000000}%
\pgfsetstrokecolor{currentstroke}%
\pgfsetstrokeopacity{0.500000}%
\pgfsetdash{}{0pt}%
\pgfpathmoveto{\pgfqpoint{3.445341in}{3.614780in}}%
\pgfpathlineto{\pgfqpoint{3.456381in}{3.619980in}}%
\pgfpathlineto{\pgfqpoint{3.467417in}{3.625155in}}%
\pgfpathlineto{\pgfqpoint{3.478449in}{3.630312in}}%
\pgfpathlineto{\pgfqpoint{3.489476in}{3.635457in}}%
\pgfpathlineto{\pgfqpoint{3.500500in}{3.640597in}}%
\pgfpathlineto{\pgfqpoint{3.494260in}{3.647790in}}%
\pgfpathlineto{\pgfqpoint{3.488023in}{3.654912in}}%
\pgfpathlineto{\pgfqpoint{3.481790in}{3.661915in}}%
\pgfpathlineto{\pgfqpoint{3.475560in}{3.668751in}}%
\pgfpathlineto{\pgfqpoint{3.469331in}{3.675372in}}%
\pgfpathlineto{\pgfqpoint{3.458318in}{3.669970in}}%
\pgfpathlineto{\pgfqpoint{3.447301in}{3.664532in}}%
\pgfpathlineto{\pgfqpoint{3.436280in}{3.659059in}}%
\pgfpathlineto{\pgfqpoint{3.425254in}{3.653550in}}%
\pgfpathlineto{\pgfqpoint{3.414224in}{3.648008in}}%
\pgfpathlineto{\pgfqpoint{3.420441in}{3.641652in}}%
\pgfpathlineto{\pgfqpoint{3.426661in}{3.635096in}}%
\pgfpathlineto{\pgfqpoint{3.432884in}{3.628396in}}%
\pgfpathlineto{\pgfqpoint{3.439110in}{3.621605in}}%
\pgfpathclose%
\pgfusepath{stroke,fill}%
\end{pgfscope}%
\begin{pgfscope}%
\pgfpathrectangle{\pgfqpoint{0.887500in}{0.275000in}}{\pgfqpoint{4.225000in}{4.225000in}}%
\pgfusepath{clip}%
\pgfsetbuttcap%
\pgfsetroundjoin%
\definecolor{currentfill}{rgb}{0.120638,0.625828,0.533488}%
\pgfsetfillcolor{currentfill}%
\pgfsetfillopacity{0.700000}%
\pgfsetlinewidth{0.501875pt}%
\definecolor{currentstroke}{rgb}{1.000000,1.000000,1.000000}%
\pgfsetstrokecolor{currentstroke}%
\pgfsetstrokeopacity{0.500000}%
\pgfsetdash{}{0pt}%
\pgfpathmoveto{\pgfqpoint{1.811953in}{2.859275in}}%
\pgfpathlineto{\pgfqpoint{1.823360in}{2.862699in}}%
\pgfpathlineto{\pgfqpoint{1.834761in}{2.866135in}}%
\pgfpathlineto{\pgfqpoint{1.846156in}{2.869580in}}%
\pgfpathlineto{\pgfqpoint{1.857546in}{2.873027in}}%
\pgfpathlineto{\pgfqpoint{1.868931in}{2.876470in}}%
\pgfpathlineto{\pgfqpoint{1.863244in}{2.884061in}}%
\pgfpathlineto{\pgfqpoint{1.857561in}{2.891637in}}%
\pgfpathlineto{\pgfqpoint{1.851883in}{2.899199in}}%
\pgfpathlineto{\pgfqpoint{1.846208in}{2.906747in}}%
\pgfpathlineto{\pgfqpoint{1.840538in}{2.914279in}}%
\pgfpathlineto{\pgfqpoint{1.829168in}{2.910797in}}%
\pgfpathlineto{\pgfqpoint{1.817792in}{2.907314in}}%
\pgfpathlineto{\pgfqpoint{1.806410in}{2.903837in}}%
\pgfpathlineto{\pgfqpoint{1.795023in}{2.900373in}}%
\pgfpathlineto{\pgfqpoint{1.783630in}{2.896927in}}%
\pgfpathlineto{\pgfqpoint{1.789287in}{2.889428in}}%
\pgfpathlineto{\pgfqpoint{1.794947in}{2.881914in}}%
\pgfpathlineto{\pgfqpoint{1.800612in}{2.874383in}}%
\pgfpathlineto{\pgfqpoint{1.806280in}{2.866837in}}%
\pgfpathclose%
\pgfusepath{stroke,fill}%
\end{pgfscope}%
\begin{pgfscope}%
\pgfpathrectangle{\pgfqpoint{0.887500in}{0.275000in}}{\pgfqpoint{4.225000in}{4.225000in}}%
\pgfusepath{clip}%
\pgfsetbuttcap%
\pgfsetroundjoin%
\definecolor{currentfill}{rgb}{0.304148,0.764704,0.419943}%
\pgfsetfillcolor{currentfill}%
\pgfsetfillopacity{0.700000}%
\pgfsetlinewidth{0.501875pt}%
\definecolor{currentstroke}{rgb}{1.000000,1.000000,1.000000}%
\pgfsetstrokecolor{currentstroke}%
\pgfsetstrokeopacity{0.500000}%
\pgfsetdash{}{0pt}%
\pgfpathmoveto{\pgfqpoint{2.687467in}{3.108579in}}%
\pgfpathlineto{\pgfqpoint{2.698509in}{3.127958in}}%
\pgfpathlineto{\pgfqpoint{2.709542in}{3.148824in}}%
\pgfpathlineto{\pgfqpoint{2.720565in}{3.171452in}}%
\pgfpathlineto{\pgfqpoint{2.731577in}{3.196117in}}%
\pgfpathlineto{\pgfqpoint{2.742577in}{3.223096in}}%
\pgfpathlineto{\pgfqpoint{2.736504in}{3.241313in}}%
\pgfpathlineto{\pgfqpoint{2.730456in}{3.256498in}}%
\pgfpathlineto{\pgfqpoint{2.724441in}{3.267923in}}%
\pgfpathlineto{\pgfqpoint{2.718463in}{3.275384in}}%
\pgfpathlineto{\pgfqpoint{2.712518in}{3.279322in}}%
\pgfpathlineto{\pgfqpoint{2.701513in}{3.255525in}}%
\pgfpathlineto{\pgfqpoint{2.690521in}{3.231118in}}%
\pgfpathlineto{\pgfqpoint{2.679537in}{3.206497in}}%
\pgfpathlineto{\pgfqpoint{2.668557in}{3.182059in}}%
\pgfpathlineto{\pgfqpoint{2.657579in}{3.158198in}}%
\pgfpathlineto{\pgfqpoint{2.663512in}{3.152158in}}%
\pgfpathlineto{\pgfqpoint{2.669468in}{3.144231in}}%
\pgfpathlineto{\pgfqpoint{2.675447in}{3.134197in}}%
\pgfpathlineto{\pgfqpoint{2.681449in}{3.122179in}}%
\pgfpathclose%
\pgfusepath{stroke,fill}%
\end{pgfscope}%
\begin{pgfscope}%
\pgfpathrectangle{\pgfqpoint{0.887500in}{0.275000in}}{\pgfqpoint{4.225000in}{4.225000in}}%
\pgfusepath{clip}%
\pgfsetbuttcap%
\pgfsetroundjoin%
\definecolor{currentfill}{rgb}{0.132268,0.655014,0.519661}%
\pgfsetfillcolor{currentfill}%
\pgfsetfillopacity{0.700000}%
\pgfsetlinewidth{0.501875pt}%
\definecolor{currentstroke}{rgb}{1.000000,1.000000,1.000000}%
\pgfsetstrokecolor{currentstroke}%
\pgfsetstrokeopacity{0.500000}%
\pgfsetdash{}{0pt}%
\pgfpathmoveto{\pgfqpoint{2.464639in}{2.907365in}}%
\pgfpathlineto{\pgfqpoint{2.475836in}{2.914639in}}%
\pgfpathlineto{\pgfqpoint{2.487034in}{2.921560in}}%
\pgfpathlineto{\pgfqpoint{2.498235in}{2.928081in}}%
\pgfpathlineto{\pgfqpoint{2.509438in}{2.934157in}}%
\pgfpathlineto{\pgfqpoint{2.520644in}{2.939774in}}%
\pgfpathlineto{\pgfqpoint{2.514759in}{2.945609in}}%
\pgfpathlineto{\pgfqpoint{2.508866in}{2.952293in}}%
\pgfpathlineto{\pgfqpoint{2.502963in}{2.960052in}}%
\pgfpathlineto{\pgfqpoint{2.497048in}{2.968924in}}%
\pgfpathlineto{\pgfqpoint{2.491123in}{2.978718in}}%
\pgfpathlineto{\pgfqpoint{2.479915in}{2.974264in}}%
\pgfpathlineto{\pgfqpoint{2.468708in}{2.969355in}}%
\pgfpathlineto{\pgfqpoint{2.457503in}{2.963973in}}%
\pgfpathlineto{\pgfqpoint{2.446302in}{2.958127in}}%
\pgfpathlineto{\pgfqpoint{2.435104in}{2.951826in}}%
\pgfpathlineto{\pgfqpoint{2.441018in}{2.942005in}}%
\pgfpathlineto{\pgfqpoint{2.446929in}{2.932604in}}%
\pgfpathlineto{\pgfqpoint{2.452836in}{2.923731in}}%
\pgfpathlineto{\pgfqpoint{2.458738in}{2.915362in}}%
\pgfpathclose%
\pgfusepath{stroke,fill}%
\end{pgfscope}%
\begin{pgfscope}%
\pgfpathrectangle{\pgfqpoint{0.887500in}{0.275000in}}{\pgfqpoint{4.225000in}{4.225000in}}%
\pgfusepath{clip}%
\pgfsetbuttcap%
\pgfsetroundjoin%
\definecolor{currentfill}{rgb}{0.119483,0.614817,0.537692}%
\pgfsetfillcolor{currentfill}%
\pgfsetfillopacity{0.700000}%
\pgfsetlinewidth{0.501875pt}%
\definecolor{currentstroke}{rgb}{1.000000,1.000000,1.000000}%
\pgfsetstrokecolor{currentstroke}%
\pgfsetstrokeopacity{0.500000}%
\pgfsetdash{}{0pt}%
\pgfpathmoveto{\pgfqpoint{2.267577in}{2.812272in}}%
\pgfpathlineto{\pgfqpoint{2.278760in}{2.821527in}}%
\pgfpathlineto{\pgfqpoint{2.289924in}{2.831709in}}%
\pgfpathlineto{\pgfqpoint{2.301074in}{2.842505in}}%
\pgfpathlineto{\pgfqpoint{2.312218in}{2.853600in}}%
\pgfpathlineto{\pgfqpoint{2.323361in}{2.864679in}}%
\pgfpathlineto{\pgfqpoint{2.317504in}{2.872999in}}%
\pgfpathlineto{\pgfqpoint{2.311652in}{2.881228in}}%
\pgfpathlineto{\pgfqpoint{2.305805in}{2.889397in}}%
\pgfpathlineto{\pgfqpoint{2.299962in}{2.897541in}}%
\pgfpathlineto{\pgfqpoint{2.294122in}{2.905692in}}%
\pgfpathlineto{\pgfqpoint{2.283006in}{2.894088in}}%
\pgfpathlineto{\pgfqpoint{2.271890in}{2.882486in}}%
\pgfpathlineto{\pgfqpoint{2.260766in}{2.871222in}}%
\pgfpathlineto{\pgfqpoint{2.249627in}{2.860629in}}%
\pgfpathlineto{\pgfqpoint{2.238467in}{2.851045in}}%
\pgfpathlineto{\pgfqpoint{2.244287in}{2.843021in}}%
\pgfpathlineto{\pgfqpoint{2.250107in}{2.835184in}}%
\pgfpathlineto{\pgfqpoint{2.255928in}{2.827481in}}%
\pgfpathlineto{\pgfqpoint{2.261751in}{2.819861in}}%
\pgfpathclose%
\pgfusepath{stroke,fill}%
\end{pgfscope}%
\begin{pgfscope}%
\pgfpathrectangle{\pgfqpoint{0.887500in}{0.275000in}}{\pgfqpoint{4.225000in}{4.225000in}}%
\pgfusepath{clip}%
\pgfsetbuttcap%
\pgfsetroundjoin%
\definecolor{currentfill}{rgb}{0.835270,0.886029,0.102646}%
\pgfsetfillcolor{currentfill}%
\pgfsetfillopacity{0.700000}%
\pgfsetlinewidth{0.501875pt}%
\definecolor{currentstroke}{rgb}{1.000000,1.000000,1.000000}%
\pgfsetstrokecolor{currentstroke}%
\pgfsetstrokeopacity{0.500000}%
\pgfsetdash{}{0pt}%
\pgfpathmoveto{\pgfqpoint{3.900643in}{3.573407in}}%
\pgfpathlineto{\pgfqpoint{3.911579in}{3.577715in}}%
\pgfpathlineto{\pgfqpoint{3.922509in}{3.582031in}}%
\pgfpathlineto{\pgfqpoint{3.933436in}{3.586366in}}%
\pgfpathlineto{\pgfqpoint{3.944358in}{3.590730in}}%
\pgfpathlineto{\pgfqpoint{3.938068in}{3.602600in}}%
\pgfpathlineto{\pgfqpoint{3.931777in}{3.614319in}}%
\pgfpathlineto{\pgfqpoint{3.925486in}{3.625863in}}%
\pgfpathlineto{\pgfqpoint{3.919193in}{3.637204in}}%
\pgfpathlineto{\pgfqpoint{3.912899in}{3.648318in}}%
\pgfpathlineto{\pgfqpoint{3.901978in}{3.643760in}}%
\pgfpathlineto{\pgfqpoint{3.891053in}{3.639235in}}%
\pgfpathlineto{\pgfqpoint{3.880125in}{3.634734in}}%
\pgfpathlineto{\pgfqpoint{3.869192in}{3.630246in}}%
\pgfpathlineto{\pgfqpoint{3.875484in}{3.619282in}}%
\pgfpathlineto{\pgfqpoint{3.881775in}{3.608097in}}%
\pgfpathlineto{\pgfqpoint{3.888065in}{3.596710in}}%
\pgfpathlineto{\pgfqpoint{3.894355in}{3.585140in}}%
\pgfpathclose%
\pgfusepath{stroke,fill}%
\end{pgfscope}%
\begin{pgfscope}%
\pgfpathrectangle{\pgfqpoint{0.887500in}{0.275000in}}{\pgfqpoint{4.225000in}{4.225000in}}%
\pgfusepath{clip}%
\pgfsetbuttcap%
\pgfsetroundjoin%
\definecolor{currentfill}{rgb}{0.119738,0.603785,0.541400}%
\pgfsetfillcolor{currentfill}%
\pgfsetfillopacity{0.700000}%
\pgfsetlinewidth{0.501875pt}%
\definecolor{currentstroke}{rgb}{1.000000,1.000000,1.000000}%
\pgfsetstrokecolor{currentstroke}%
\pgfsetstrokeopacity{0.500000}%
\pgfsetdash{}{0pt}%
\pgfpathmoveto{\pgfqpoint{2.125419in}{2.812809in}}%
\pgfpathlineto{\pgfqpoint{2.136767in}{2.815765in}}%
\pgfpathlineto{\pgfqpoint{2.148119in}{2.818276in}}%
\pgfpathlineto{\pgfqpoint{2.159475in}{2.820381in}}%
\pgfpathlineto{\pgfqpoint{2.170827in}{2.822388in}}%
\pgfpathlineto{\pgfqpoint{2.182167in}{2.824637in}}%
\pgfpathlineto{\pgfqpoint{2.176362in}{2.832749in}}%
\pgfpathlineto{\pgfqpoint{2.170556in}{2.841063in}}%
\pgfpathlineto{\pgfqpoint{2.164749in}{2.849609in}}%
\pgfpathlineto{\pgfqpoint{2.158940in}{2.858411in}}%
\pgfpathlineto{\pgfqpoint{2.153128in}{2.867450in}}%
\pgfpathlineto{\pgfqpoint{2.141816in}{2.864424in}}%
\pgfpathlineto{\pgfqpoint{2.130493in}{2.861654in}}%
\pgfpathlineto{\pgfqpoint{2.119164in}{2.858868in}}%
\pgfpathlineto{\pgfqpoint{2.107835in}{2.855819in}}%
\pgfpathlineto{\pgfqpoint{2.096509in}{2.852473in}}%
\pgfpathlineto{\pgfqpoint{2.102284in}{2.844498in}}%
\pgfpathlineto{\pgfqpoint{2.108063in}{2.836544in}}%
\pgfpathlineto{\pgfqpoint{2.113845in}{2.828614in}}%
\pgfpathlineto{\pgfqpoint{2.119630in}{2.820704in}}%
\pgfpathclose%
\pgfusepath{stroke,fill}%
\end{pgfscope}%
\begin{pgfscope}%
\pgfpathrectangle{\pgfqpoint{0.887500in}{0.275000in}}{\pgfqpoint{4.225000in}{4.225000in}}%
\pgfusepath{clip}%
\pgfsetbuttcap%
\pgfsetroundjoin%
\definecolor{currentfill}{rgb}{0.876168,0.891125,0.095250}%
\pgfsetfillcolor{currentfill}%
\pgfsetfillopacity{0.700000}%
\pgfsetlinewidth{0.501875pt}%
\definecolor{currentstroke}{rgb}{1.000000,1.000000,1.000000}%
\pgfsetstrokecolor{currentstroke}%
\pgfsetstrokeopacity{0.500000}%
\pgfsetdash{}{0pt}%
\pgfpathmoveto{\pgfqpoint{3.673224in}{3.608653in}}%
\pgfpathlineto{\pgfqpoint{3.684219in}{3.613441in}}%
\pgfpathlineto{\pgfqpoint{3.695211in}{3.618378in}}%
\pgfpathlineto{\pgfqpoint{3.706201in}{3.623410in}}%
\pgfpathlineto{\pgfqpoint{3.717188in}{3.628485in}}%
\pgfpathlineto{\pgfqpoint{3.728170in}{3.633551in}}%
\pgfpathlineto{\pgfqpoint{3.721884in}{3.642817in}}%
\pgfpathlineto{\pgfqpoint{3.715599in}{3.651871in}}%
\pgfpathlineto{\pgfqpoint{3.709314in}{3.660720in}}%
\pgfpathlineto{\pgfqpoint{3.703030in}{3.669375in}}%
\pgfpathlineto{\pgfqpoint{3.696747in}{3.677843in}}%
\pgfpathlineto{\pgfqpoint{3.685775in}{3.672777in}}%
\pgfpathlineto{\pgfqpoint{3.674799in}{3.667686in}}%
\pgfpathlineto{\pgfqpoint{3.663818in}{3.662590in}}%
\pgfpathlineto{\pgfqpoint{3.652833in}{3.657509in}}%
\pgfpathlineto{\pgfqpoint{3.641845in}{3.652464in}}%
\pgfpathlineto{\pgfqpoint{3.648118in}{3.644064in}}%
\pgfpathlineto{\pgfqpoint{3.654393in}{3.635496in}}%
\pgfpathlineto{\pgfqpoint{3.660669in}{3.626747in}}%
\pgfpathlineto{\pgfqpoint{3.666946in}{3.617803in}}%
\pgfpathclose%
\pgfusepath{stroke,fill}%
\end{pgfscope}%
\begin{pgfscope}%
\pgfpathrectangle{\pgfqpoint{0.887500in}{0.275000in}}{\pgfqpoint{4.225000in}{4.225000in}}%
\pgfusepath{clip}%
\pgfsetbuttcap%
\pgfsetroundjoin%
\definecolor{currentfill}{rgb}{0.180653,0.701402,0.488189}%
\pgfsetfillcolor{currentfill}%
\pgfsetfillopacity{0.700000}%
\pgfsetlinewidth{0.501875pt}%
\definecolor{currentstroke}{rgb}{1.000000,1.000000,1.000000}%
\pgfsetstrokecolor{currentstroke}%
\pgfsetstrokeopacity{0.500000}%
\pgfsetdash{}{0pt}%
\pgfpathmoveto{\pgfqpoint{2.661884in}{2.992115in}}%
\pgfpathlineto{\pgfqpoint{2.673031in}{3.001592in}}%
\pgfpathlineto{\pgfqpoint{2.684183in}{3.010368in}}%
\pgfpathlineto{\pgfqpoint{2.695343in}{3.018063in}}%
\pgfpathlineto{\pgfqpoint{2.706509in}{3.024653in}}%
\pgfpathlineto{\pgfqpoint{2.717672in}{3.031269in}}%
\pgfpathlineto{\pgfqpoint{2.711629in}{3.046603in}}%
\pgfpathlineto{\pgfqpoint{2.705583in}{3.062411in}}%
\pgfpathlineto{\pgfqpoint{2.699539in}{3.078284in}}%
\pgfpathlineto{\pgfqpoint{2.693499in}{3.093811in}}%
\pgfpathlineto{\pgfqpoint{2.687467in}{3.108579in}}%
\pgfpathlineto{\pgfqpoint{2.676417in}{3.090413in}}%
\pgfpathlineto{\pgfqpoint{2.665361in}{3.073186in}}%
\pgfpathlineto{\pgfqpoint{2.654300in}{3.056681in}}%
\pgfpathlineto{\pgfqpoint{2.643234in}{3.040945in}}%
\pgfpathlineto{\pgfqpoint{2.632159in}{3.026108in}}%
\pgfpathlineto{\pgfqpoint{2.638096in}{3.019269in}}%
\pgfpathlineto{\pgfqpoint{2.644037in}{3.012428in}}%
\pgfpathlineto{\pgfqpoint{2.649982in}{3.005608in}}%
\pgfpathlineto{\pgfqpoint{2.655931in}{2.998830in}}%
\pgfpathclose%
\pgfusepath{stroke,fill}%
\end{pgfscope}%
\begin{pgfscope}%
\pgfpathrectangle{\pgfqpoint{0.887500in}{0.275000in}}{\pgfqpoint{4.225000in}{4.225000in}}%
\pgfusepath{clip}%
\pgfsetbuttcap%
\pgfsetroundjoin%
\definecolor{currentfill}{rgb}{0.119699,0.618490,0.536347}%
\pgfsetfillcolor{currentfill}%
\pgfsetfillopacity{0.700000}%
\pgfsetlinewidth{0.501875pt}%
\definecolor{currentstroke}{rgb}{1.000000,1.000000,1.000000}%
\pgfsetstrokecolor{currentstroke}%
\pgfsetstrokeopacity{0.500000}%
\pgfsetdash{}{0pt}%
\pgfpathmoveto{\pgfqpoint{1.897427in}{2.838311in}}%
\pgfpathlineto{\pgfqpoint{1.908820in}{2.841715in}}%
\pgfpathlineto{\pgfqpoint{1.920208in}{2.845102in}}%
\pgfpathlineto{\pgfqpoint{1.931592in}{2.848466in}}%
\pgfpathlineto{\pgfqpoint{1.942971in}{2.851805in}}%
\pgfpathlineto{\pgfqpoint{1.954344in}{2.855137in}}%
\pgfpathlineto{\pgfqpoint{1.948623in}{2.862827in}}%
\pgfpathlineto{\pgfqpoint{1.942906in}{2.870504in}}%
\pgfpathlineto{\pgfqpoint{1.937193in}{2.878169in}}%
\pgfpathlineto{\pgfqpoint{1.931485in}{2.885821in}}%
\pgfpathlineto{\pgfqpoint{1.925780in}{2.893462in}}%
\pgfpathlineto{\pgfqpoint{1.914420in}{2.890094in}}%
\pgfpathlineto{\pgfqpoint{1.903055in}{2.886720in}}%
\pgfpathlineto{\pgfqpoint{1.891685in}{2.883323in}}%
\pgfpathlineto{\pgfqpoint{1.880310in}{2.879905in}}%
\pgfpathlineto{\pgfqpoint{1.868931in}{2.876470in}}%
\pgfpathlineto{\pgfqpoint{1.874622in}{2.868866in}}%
\pgfpathlineto{\pgfqpoint{1.880317in}{2.861247in}}%
\pgfpathlineto{\pgfqpoint{1.886016in}{2.853615in}}%
\pgfpathlineto{\pgfqpoint{1.891719in}{2.845970in}}%
\pgfpathclose%
\pgfusepath{stroke,fill}%
\end{pgfscope}%
\begin{pgfscope}%
\pgfpathrectangle{\pgfqpoint{0.887500in}{0.275000in}}{\pgfqpoint{4.225000in}{4.225000in}}%
\pgfusepath{clip}%
\pgfsetbuttcap%
\pgfsetroundjoin%
\definecolor{currentfill}{rgb}{0.730889,0.871916,0.156029}%
\pgfsetfillcolor{currentfill}%
\pgfsetfillopacity{0.700000}%
\pgfsetlinewidth{0.501875pt}%
\definecolor{currentstroke}{rgb}{1.000000,1.000000,1.000000}%
\pgfsetstrokecolor{currentstroke}%
\pgfsetstrokeopacity{0.500000}%
\pgfsetdash{}{0pt}%
\pgfpathmoveto{\pgfqpoint{2.964766in}{3.524215in}}%
\pgfpathlineto{\pgfqpoint{2.975914in}{3.526591in}}%
\pgfpathlineto{\pgfqpoint{2.987055in}{3.529602in}}%
\pgfpathlineto{\pgfqpoint{2.998189in}{3.533147in}}%
\pgfpathlineto{\pgfqpoint{3.009318in}{3.537124in}}%
\pgfpathlineto{\pgfqpoint{3.020442in}{3.541429in}}%
\pgfpathlineto{\pgfqpoint{3.014367in}{3.542318in}}%
\pgfpathlineto{\pgfqpoint{3.008299in}{3.543103in}}%
\pgfpathlineto{\pgfqpoint{3.002237in}{3.543804in}}%
\pgfpathlineto{\pgfqpoint{2.996181in}{3.544436in}}%
\pgfpathlineto{\pgfqpoint{2.990131in}{3.545015in}}%
\pgfpathlineto{\pgfqpoint{2.979028in}{3.541419in}}%
\pgfpathlineto{\pgfqpoint{2.967919in}{3.537896in}}%
\pgfpathlineto{\pgfqpoint{2.956805in}{3.534352in}}%
\pgfpathlineto{\pgfqpoint{2.945686in}{3.530689in}}%
\pgfpathlineto{\pgfqpoint{2.934563in}{3.526813in}}%
\pgfpathlineto{\pgfqpoint{2.940589in}{3.527249in}}%
\pgfpathlineto{\pgfqpoint{2.946622in}{3.527250in}}%
\pgfpathlineto{\pgfqpoint{2.952663in}{3.526735in}}%
\pgfpathlineto{\pgfqpoint{2.958711in}{3.525700in}}%
\pgfpathclose%
\pgfusepath{stroke,fill}%
\end{pgfscope}%
\begin{pgfscope}%
\pgfpathrectangle{\pgfqpoint{0.887500in}{0.275000in}}{\pgfqpoint{4.225000in}{4.225000in}}%
\pgfusepath{clip}%
\pgfsetbuttcap%
\pgfsetroundjoin%
\definecolor{currentfill}{rgb}{0.146616,0.673050,0.508936}%
\pgfsetfillcolor{currentfill}%
\pgfsetfillopacity{0.700000}%
\pgfsetlinewidth{0.501875pt}%
\definecolor{currentstroke}{rgb}{1.000000,1.000000,1.000000}%
\pgfsetstrokecolor{currentstroke}%
\pgfsetstrokeopacity{0.500000}%
\pgfsetdash{}{0pt}%
\pgfpathmoveto{\pgfqpoint{2.606089in}{2.947309in}}%
\pgfpathlineto{\pgfqpoint{2.617269in}{2.954763in}}%
\pgfpathlineto{\pgfqpoint{2.628434in}{2.963261in}}%
\pgfpathlineto{\pgfqpoint{2.639589in}{2.972571in}}%
\pgfpathlineto{\pgfqpoint{2.650738in}{2.982315in}}%
\pgfpathlineto{\pgfqpoint{2.661884in}{2.992115in}}%
\pgfpathlineto{\pgfqpoint{2.655931in}{2.998830in}}%
\pgfpathlineto{\pgfqpoint{2.649982in}{3.005608in}}%
\pgfpathlineto{\pgfqpoint{2.644037in}{3.012428in}}%
\pgfpathlineto{\pgfqpoint{2.638096in}{3.019269in}}%
\pgfpathlineto{\pgfqpoint{2.632159in}{3.026108in}}%
\pgfpathlineto{\pgfqpoint{2.621074in}{3.012300in}}%
\pgfpathlineto{\pgfqpoint{2.609977in}{2.999649in}}%
\pgfpathlineto{\pgfqpoint{2.598865in}{2.988285in}}%
\pgfpathlineto{\pgfqpoint{2.587736in}{2.978337in}}%
\pgfpathlineto{\pgfqpoint{2.576588in}{2.969849in}}%
\pgfpathlineto{\pgfqpoint{2.582482in}{2.964993in}}%
\pgfpathlineto{\pgfqpoint{2.588378in}{2.960466in}}%
\pgfpathlineto{\pgfqpoint{2.594276in}{2.956114in}}%
\pgfpathlineto{\pgfqpoint{2.600179in}{2.951780in}}%
\pgfpathclose%
\pgfusepath{stroke,fill}%
\end{pgfscope}%
\begin{pgfscope}%
\pgfpathrectangle{\pgfqpoint{0.887500in}{0.275000in}}{\pgfqpoint{4.225000in}{4.225000in}}%
\pgfusepath{clip}%
\pgfsetbuttcap%
\pgfsetroundjoin%
\definecolor{currentfill}{rgb}{0.239374,0.735588,0.455688}%
\pgfsetfillcolor{currentfill}%
\pgfsetfillopacity{0.700000}%
\pgfsetlinewidth{0.501875pt}%
\definecolor{currentstroke}{rgb}{1.000000,1.000000,1.000000}%
\pgfsetstrokecolor{currentstroke}%
\pgfsetstrokeopacity{0.500000}%
\pgfsetdash{}{0pt}%
\pgfpathmoveto{\pgfqpoint{2.717672in}{3.031269in}}%
\pgfpathlineto{\pgfqpoint{2.728820in}{3.039276in}}%
\pgfpathlineto{\pgfqpoint{2.739943in}{3.050039in}}%
\pgfpathlineto{\pgfqpoint{2.751034in}{3.064927in}}%
\pgfpathlineto{\pgfqpoint{2.762085in}{3.085311in}}%
\pgfpathlineto{\pgfqpoint{2.773093in}{3.112566in}}%
\pgfpathlineto{\pgfqpoint{2.766989in}{3.134793in}}%
\pgfpathlineto{\pgfqpoint{2.760880in}{3.157700in}}%
\pgfpathlineto{\pgfqpoint{2.754772in}{3.180547in}}%
\pgfpathlineto{\pgfqpoint{2.748669in}{3.202593in}}%
\pgfpathlineto{\pgfqpoint{2.742577in}{3.223096in}}%
\pgfpathlineto{\pgfqpoint{2.731577in}{3.196117in}}%
\pgfpathlineto{\pgfqpoint{2.720565in}{3.171452in}}%
\pgfpathlineto{\pgfqpoint{2.709542in}{3.148824in}}%
\pgfpathlineto{\pgfqpoint{2.698509in}{3.127958in}}%
\pgfpathlineto{\pgfqpoint{2.687467in}{3.108579in}}%
\pgfpathlineto{\pgfqpoint{2.693499in}{3.093811in}}%
\pgfpathlineto{\pgfqpoint{2.699539in}{3.078284in}}%
\pgfpathlineto{\pgfqpoint{2.705583in}{3.062411in}}%
\pgfpathlineto{\pgfqpoint{2.711629in}{3.046603in}}%
\pgfpathclose%
\pgfusepath{stroke,fill}%
\end{pgfscope}%
\begin{pgfscope}%
\pgfpathrectangle{\pgfqpoint{0.887500in}{0.275000in}}{\pgfqpoint{4.225000in}{4.225000in}}%
\pgfusepath{clip}%
\pgfsetbuttcap%
\pgfsetroundjoin%
\definecolor{currentfill}{rgb}{0.876168,0.891125,0.095250}%
\pgfsetfillcolor{currentfill}%
\pgfsetfillopacity{0.700000}%
\pgfsetlinewidth{0.501875pt}%
\definecolor{currentstroke}{rgb}{1.000000,1.000000,1.000000}%
\pgfsetstrokecolor{currentstroke}%
\pgfsetstrokeopacity{0.500000}%
\pgfsetdash{}{0pt}%
\pgfpathmoveto{\pgfqpoint{3.531771in}{3.605186in}}%
\pgfpathlineto{\pgfqpoint{3.542796in}{3.609789in}}%
\pgfpathlineto{\pgfqpoint{3.553817in}{3.614390in}}%
\pgfpathlineto{\pgfqpoint{3.564834in}{3.618996in}}%
\pgfpathlineto{\pgfqpoint{3.575847in}{3.623620in}}%
\pgfpathlineto{\pgfqpoint{3.586855in}{3.628276in}}%
\pgfpathlineto{\pgfqpoint{3.580588in}{3.635969in}}%
\pgfpathlineto{\pgfqpoint{3.574324in}{3.643632in}}%
\pgfpathlineto{\pgfqpoint{3.568064in}{3.651275in}}%
\pgfpathlineto{\pgfqpoint{3.561808in}{3.658871in}}%
\pgfpathlineto{\pgfqpoint{3.555555in}{3.666393in}}%
\pgfpathlineto{\pgfqpoint{3.544552in}{3.661220in}}%
\pgfpathlineto{\pgfqpoint{3.533545in}{3.656051in}}%
\pgfpathlineto{\pgfqpoint{3.522534in}{3.650890in}}%
\pgfpathlineto{\pgfqpoint{3.511519in}{3.645739in}}%
\pgfpathlineto{\pgfqpoint{3.500500in}{3.640597in}}%
\pgfpathlineto{\pgfqpoint{3.506744in}{3.633383in}}%
\pgfpathlineto{\pgfqpoint{3.512992in}{3.626194in}}%
\pgfpathlineto{\pgfqpoint{3.519246in}{3.619080in}}%
\pgfpathlineto{\pgfqpoint{3.525506in}{3.612084in}}%
\pgfpathclose%
\pgfusepath{stroke,fill}%
\end{pgfscope}%
\begin{pgfscope}%
\pgfpathrectangle{\pgfqpoint{0.887500in}{0.275000in}}{\pgfqpoint{4.225000in}{4.225000in}}%
\pgfusepath{clip}%
\pgfsetbuttcap%
\pgfsetroundjoin%
\definecolor{currentfill}{rgb}{0.120565,0.596422,0.543611}%
\pgfsetfillcolor{currentfill}%
\pgfsetfillopacity{0.700000}%
\pgfsetlinewidth{0.501875pt}%
\definecolor{currentstroke}{rgb}{1.000000,1.000000,1.000000}%
\pgfsetstrokecolor{currentstroke}%
\pgfsetstrokeopacity{0.500000}%
\pgfsetdash{}{0pt}%
\pgfpathmoveto{\pgfqpoint{2.211206in}{2.785960in}}%
\pgfpathlineto{\pgfqpoint{2.222537in}{2.788937in}}%
\pgfpathlineto{\pgfqpoint{2.233844in}{2.792773in}}%
\pgfpathlineto{\pgfqpoint{2.245122in}{2.797778in}}%
\pgfpathlineto{\pgfqpoint{2.256367in}{2.804260in}}%
\pgfpathlineto{\pgfqpoint{2.267577in}{2.812272in}}%
\pgfpathlineto{\pgfqpoint{2.261751in}{2.819861in}}%
\pgfpathlineto{\pgfqpoint{2.255928in}{2.827481in}}%
\pgfpathlineto{\pgfqpoint{2.250107in}{2.835184in}}%
\pgfpathlineto{\pgfqpoint{2.244287in}{2.843021in}}%
\pgfpathlineto{\pgfqpoint{2.238467in}{2.851045in}}%
\pgfpathlineto{\pgfqpoint{2.227277in}{2.842803in}}%
\pgfpathlineto{\pgfqpoint{2.216050in}{2.836212in}}%
\pgfpathlineto{\pgfqpoint{2.204785in}{2.831210in}}%
\pgfpathlineto{\pgfqpoint{2.193489in}{2.827464in}}%
\pgfpathlineto{\pgfqpoint{2.182167in}{2.824637in}}%
\pgfpathlineto{\pgfqpoint{2.187972in}{2.816693in}}%
\pgfpathlineto{\pgfqpoint{2.193777in}{2.808885in}}%
\pgfpathlineto{\pgfqpoint{2.199584in}{2.801182in}}%
\pgfpathlineto{\pgfqpoint{2.205394in}{2.793551in}}%
\pgfpathclose%
\pgfusepath{stroke,fill}%
\end{pgfscope}%
\begin{pgfscope}%
\pgfpathrectangle{\pgfqpoint{0.887500in}{0.275000in}}{\pgfqpoint{4.225000in}{4.225000in}}%
\pgfusepath{clip}%
\pgfsetbuttcap%
\pgfsetroundjoin%
\definecolor{currentfill}{rgb}{0.121380,0.629492,0.531973}%
\pgfsetfillcolor{currentfill}%
\pgfsetfillopacity{0.700000}%
\pgfsetlinewidth{0.501875pt}%
\definecolor{currentstroke}{rgb}{1.000000,1.000000,1.000000}%
\pgfsetstrokecolor{currentstroke}%
\pgfsetstrokeopacity{0.500000}%
\pgfsetdash{}{0pt}%
\pgfpathmoveto{\pgfqpoint{1.669379in}{2.863156in}}%
\pgfpathlineto{\pgfqpoint{1.680830in}{2.866497in}}%
\pgfpathlineto{\pgfqpoint{1.692276in}{2.869842in}}%
\pgfpathlineto{\pgfqpoint{1.703715in}{2.873194in}}%
\pgfpathlineto{\pgfqpoint{1.715149in}{2.876552in}}%
\pgfpathlineto{\pgfqpoint{1.726577in}{2.879919in}}%
\pgfpathlineto{\pgfqpoint{1.720939in}{2.887391in}}%
\pgfpathlineto{\pgfqpoint{1.715305in}{2.894849in}}%
\pgfpathlineto{\pgfqpoint{1.709674in}{2.902293in}}%
\pgfpathlineto{\pgfqpoint{1.704049in}{2.909723in}}%
\pgfpathlineto{\pgfqpoint{1.692631in}{2.906348in}}%
\pgfpathlineto{\pgfqpoint{1.681209in}{2.902977in}}%
\pgfpathlineto{\pgfqpoint{1.669780in}{2.899607in}}%
\pgfpathlineto{\pgfqpoint{1.658347in}{2.896239in}}%
\pgfpathlineto{\pgfqpoint{1.646907in}{2.892872in}}%
\pgfpathlineto{\pgfqpoint{1.652519in}{2.885464in}}%
\pgfpathlineto{\pgfqpoint{1.658135in}{2.878042in}}%
\pgfpathlineto{\pgfqpoint{1.663755in}{2.870606in}}%
\pgfpathclose%
\pgfusepath{stroke,fill}%
\end{pgfscope}%
\begin{pgfscope}%
\pgfpathrectangle{\pgfqpoint{0.887500in}{0.275000in}}{\pgfqpoint{4.225000in}{4.225000in}}%
\pgfusepath{clip}%
\pgfsetbuttcap%
\pgfsetroundjoin%
\definecolor{currentfill}{rgb}{0.855810,0.888601,0.097452}%
\pgfsetfillcolor{currentfill}%
\pgfsetfillopacity{0.700000}%
\pgfsetlinewidth{0.501875pt}%
\definecolor{currentstroke}{rgb}{1.000000,1.000000,1.000000}%
\pgfsetstrokecolor{currentstroke}%
\pgfsetstrokeopacity{0.500000}%
\pgfsetdash{}{0pt}%
\pgfpathmoveto{\pgfqpoint{3.759594in}{3.583733in}}%
\pgfpathlineto{\pgfqpoint{3.770579in}{3.588677in}}%
\pgfpathlineto{\pgfqpoint{3.781557in}{3.593528in}}%
\pgfpathlineto{\pgfqpoint{3.792530in}{3.598296in}}%
\pgfpathlineto{\pgfqpoint{3.803498in}{3.602991in}}%
\pgfpathlineto{\pgfqpoint{3.814460in}{3.607626in}}%
\pgfpathlineto{\pgfqpoint{3.808170in}{3.618188in}}%
\pgfpathlineto{\pgfqpoint{3.801880in}{3.628498in}}%
\pgfpathlineto{\pgfqpoint{3.795588in}{3.638556in}}%
\pgfpathlineto{\pgfqpoint{3.789295in}{3.648369in}}%
\pgfpathlineto{\pgfqpoint{3.783002in}{3.657945in}}%
\pgfpathlineto{\pgfqpoint{3.772046in}{3.653179in}}%
\pgfpathlineto{\pgfqpoint{3.761085in}{3.648364in}}%
\pgfpathlineto{\pgfqpoint{3.750118in}{3.643493in}}%
\pgfpathlineto{\pgfqpoint{3.739147in}{3.638557in}}%
\pgfpathlineto{\pgfqpoint{3.728170in}{3.633551in}}%
\pgfpathlineto{\pgfqpoint{3.734455in}{3.624063in}}%
\pgfpathlineto{\pgfqpoint{3.740741in}{3.614345in}}%
\pgfpathlineto{\pgfqpoint{3.747026in}{3.604389in}}%
\pgfpathlineto{\pgfqpoint{3.753310in}{3.594185in}}%
\pgfpathclose%
\pgfusepath{stroke,fill}%
\end{pgfscope}%
\begin{pgfscope}%
\pgfpathrectangle{\pgfqpoint{0.887500in}{0.275000in}}{\pgfqpoint{4.225000in}{4.225000in}}%
\pgfusepath{clip}%
\pgfsetbuttcap%
\pgfsetroundjoin%
\definecolor{currentfill}{rgb}{0.855810,0.888601,0.097452}%
\pgfsetfillcolor{currentfill}%
\pgfsetfillopacity{0.700000}%
\pgfsetlinewidth{0.501875pt}%
\definecolor{currentstroke}{rgb}{1.000000,1.000000,1.000000}%
\pgfsetstrokecolor{currentstroke}%
\pgfsetstrokeopacity{0.500000}%
\pgfsetdash{}{0pt}%
\pgfpathmoveto{\pgfqpoint{3.390083in}{3.589461in}}%
\pgfpathlineto{\pgfqpoint{3.401140in}{3.594223in}}%
\pgfpathlineto{\pgfqpoint{3.412195in}{3.599205in}}%
\pgfpathlineto{\pgfqpoint{3.423248in}{3.604337in}}%
\pgfpathlineto{\pgfqpoint{3.434296in}{3.609552in}}%
\pgfpathlineto{\pgfqpoint{3.445341in}{3.614780in}}%
\pgfpathlineto{\pgfqpoint{3.439110in}{3.621605in}}%
\pgfpathlineto{\pgfqpoint{3.432884in}{3.628396in}}%
\pgfpathlineto{\pgfqpoint{3.426661in}{3.635096in}}%
\pgfpathlineto{\pgfqpoint{3.420441in}{3.641652in}}%
\pgfpathlineto{\pgfqpoint{3.414224in}{3.648008in}}%
\pgfpathlineto{\pgfqpoint{3.403190in}{3.642434in}}%
\pgfpathlineto{\pgfqpoint{3.392152in}{3.636859in}}%
\pgfpathlineto{\pgfqpoint{3.381111in}{3.631330in}}%
\pgfpathlineto{\pgfqpoint{3.370066in}{3.625895in}}%
\pgfpathlineto{\pgfqpoint{3.359019in}{3.620602in}}%
\pgfpathlineto{\pgfqpoint{3.365224in}{3.614629in}}%
\pgfpathlineto{\pgfqpoint{3.371433in}{3.608473in}}%
\pgfpathlineto{\pgfqpoint{3.377646in}{3.602190in}}%
\pgfpathlineto{\pgfqpoint{3.383862in}{3.595834in}}%
\pgfpathclose%
\pgfusepath{stroke,fill}%
\end{pgfscope}%
\begin{pgfscope}%
\pgfpathrectangle{\pgfqpoint{0.887500in}{0.275000in}}{\pgfqpoint{4.225000in}{4.225000in}}%
\pgfusepath{clip}%
\pgfsetbuttcap%
\pgfsetroundjoin%
\definecolor{currentfill}{rgb}{0.793760,0.880678,0.120005}%
\pgfsetfillcolor{currentfill}%
\pgfsetfillopacity{0.700000}%
\pgfsetlinewidth{0.501875pt}%
\definecolor{currentstroke}{rgb}{1.000000,1.000000,1.000000}%
\pgfsetstrokecolor{currentstroke}%
\pgfsetstrokeopacity{0.500000}%
\pgfsetdash{}{0pt}%
\pgfpathmoveto{\pgfqpoint{3.106563in}{3.560320in}}%
\pgfpathlineto{\pgfqpoint{3.117680in}{3.564354in}}%
\pgfpathlineto{\pgfqpoint{3.128792in}{3.568185in}}%
\pgfpathlineto{\pgfqpoint{3.139899in}{3.571897in}}%
\pgfpathlineto{\pgfqpoint{3.151001in}{3.575576in}}%
\pgfpathlineto{\pgfqpoint{3.162098in}{3.579296in}}%
\pgfpathlineto{\pgfqpoint{3.155954in}{3.581618in}}%
\pgfpathlineto{\pgfqpoint{3.149816in}{3.583708in}}%
\pgfpathlineto{\pgfqpoint{3.143682in}{3.585589in}}%
\pgfpathlineto{\pgfqpoint{3.137554in}{3.587282in}}%
\pgfpathlineto{\pgfqpoint{3.131432in}{3.588810in}}%
\pgfpathlineto{\pgfqpoint{3.120353in}{3.583851in}}%
\pgfpathlineto{\pgfqpoint{3.109269in}{3.578966in}}%
\pgfpathlineto{\pgfqpoint{3.098181in}{3.574148in}}%
\pgfpathlineto{\pgfqpoint{3.087089in}{3.569383in}}%
\pgfpathlineto{\pgfqpoint{3.075992in}{3.564660in}}%
\pgfpathlineto{\pgfqpoint{3.082094in}{3.564163in}}%
\pgfpathlineto{\pgfqpoint{3.088203in}{3.563535in}}%
\pgfpathlineto{\pgfqpoint{3.094317in}{3.562721in}}%
\pgfpathlineto{\pgfqpoint{3.100437in}{3.561668in}}%
\pgfpathclose%
\pgfusepath{stroke,fill}%
\end{pgfscope}%
\begin{pgfscope}%
\pgfpathrectangle{\pgfqpoint{0.887500in}{0.275000in}}{\pgfqpoint{4.225000in}{4.225000in}}%
\pgfusepath{clip}%
\pgfsetbuttcap%
\pgfsetroundjoin%
\definecolor{currentfill}{rgb}{0.126326,0.644107,0.525311}%
\pgfsetfillcolor{currentfill}%
\pgfsetfillopacity{0.700000}%
\pgfsetlinewidth{0.501875pt}%
\definecolor{currentstroke}{rgb}{1.000000,1.000000,1.000000}%
\pgfsetstrokecolor{currentstroke}%
\pgfsetstrokeopacity{0.500000}%
\pgfsetdash{}{0pt}%
\pgfpathmoveto{\pgfqpoint{2.408671in}{2.867098in}}%
\pgfpathlineto{\pgfqpoint{2.419863in}{2.875588in}}%
\pgfpathlineto{\pgfqpoint{2.431056in}{2.883860in}}%
\pgfpathlineto{\pgfqpoint{2.442250in}{2.891930in}}%
\pgfpathlineto{\pgfqpoint{2.453444in}{2.899780in}}%
\pgfpathlineto{\pgfqpoint{2.464639in}{2.907365in}}%
\pgfpathlineto{\pgfqpoint{2.458738in}{2.915362in}}%
\pgfpathlineto{\pgfqpoint{2.452836in}{2.923731in}}%
\pgfpathlineto{\pgfqpoint{2.446929in}{2.932604in}}%
\pgfpathlineto{\pgfqpoint{2.441018in}{2.942005in}}%
\pgfpathlineto{\pgfqpoint{2.435104in}{2.951826in}}%
\pgfpathlineto{\pgfqpoint{2.423909in}{2.945080in}}%
\pgfpathlineto{\pgfqpoint{2.412717in}{2.937899in}}%
\pgfpathlineto{\pgfqpoint{2.401530in}{2.930291in}}%
\pgfpathlineto{\pgfqpoint{2.390346in}{2.922260in}}%
\pgfpathlineto{\pgfqpoint{2.379167in}{2.913796in}}%
\pgfpathlineto{\pgfqpoint{2.385058in}{2.904635in}}%
\pgfpathlineto{\pgfqpoint{2.390954in}{2.895386in}}%
\pgfpathlineto{\pgfqpoint{2.396855in}{2.886055in}}%
\pgfpathlineto{\pgfqpoint{2.402761in}{2.876633in}}%
\pgfpathclose%
\pgfusepath{stroke,fill}%
\end{pgfscope}%
\begin{pgfscope}%
\pgfpathrectangle{\pgfqpoint{0.887500in}{0.275000in}}{\pgfqpoint{4.225000in}{4.225000in}}%
\pgfusepath{clip}%
\pgfsetbuttcap%
\pgfsetroundjoin%
\definecolor{currentfill}{rgb}{0.824940,0.884720,0.106217}%
\pgfsetfillcolor{currentfill}%
\pgfsetfillopacity{0.700000}%
\pgfsetlinewidth{0.501875pt}%
\definecolor{currentstroke}{rgb}{1.000000,1.000000,1.000000}%
\pgfsetstrokecolor{currentstroke}%
\pgfsetstrokeopacity{0.500000}%
\pgfsetdash{}{0pt}%
\pgfpathmoveto{\pgfqpoint{3.248373in}{3.579556in}}%
\pgfpathlineto{\pgfqpoint{3.259456in}{3.582951in}}%
\pgfpathlineto{\pgfqpoint{3.270533in}{3.586425in}}%
\pgfpathlineto{\pgfqpoint{3.281606in}{3.590006in}}%
\pgfpathlineto{\pgfqpoint{3.292675in}{3.593725in}}%
\pgfpathlineto{\pgfqpoint{3.303740in}{3.597616in}}%
\pgfpathlineto{\pgfqpoint{3.297548in}{3.602590in}}%
\pgfpathlineto{\pgfqpoint{3.291359in}{3.607375in}}%
\pgfpathlineto{\pgfqpoint{3.285174in}{3.611948in}}%
\pgfpathlineto{\pgfqpoint{3.278992in}{3.616300in}}%
\pgfpathlineto{\pgfqpoint{3.272815in}{3.620426in}}%
\pgfpathlineto{\pgfqpoint{3.261762in}{3.615870in}}%
\pgfpathlineto{\pgfqpoint{3.250706in}{3.611458in}}%
\pgfpathlineto{\pgfqpoint{3.239645in}{3.607167in}}%
\pgfpathlineto{\pgfqpoint{3.228581in}{3.602973in}}%
\pgfpathlineto{\pgfqpoint{3.217512in}{3.598856in}}%
\pgfpathlineto{\pgfqpoint{3.223675in}{3.595287in}}%
\pgfpathlineto{\pgfqpoint{3.229842in}{3.591562in}}%
\pgfpathlineto{\pgfqpoint{3.236015in}{3.587694in}}%
\pgfpathlineto{\pgfqpoint{3.242192in}{3.583693in}}%
\pgfpathclose%
\pgfusepath{stroke,fill}%
\end{pgfscope}%
\begin{pgfscope}%
\pgfpathrectangle{\pgfqpoint{0.887500in}{0.275000in}}{\pgfqpoint{4.225000in}{4.225000in}}%
\pgfusepath{clip}%
\pgfsetbuttcap%
\pgfsetroundjoin%
\definecolor{currentfill}{rgb}{0.119423,0.611141,0.538982}%
\pgfsetfillcolor{currentfill}%
\pgfsetfillopacity{0.700000}%
\pgfsetlinewidth{0.501875pt}%
\definecolor{currentstroke}{rgb}{1.000000,1.000000,1.000000}%
\pgfsetstrokecolor{currentstroke}%
\pgfsetstrokeopacity{0.500000}%
\pgfsetdash{}{0pt}%
\pgfpathmoveto{\pgfqpoint{1.983010in}{2.816508in}}%
\pgfpathlineto{\pgfqpoint{1.994391in}{2.819827in}}%
\pgfpathlineto{\pgfqpoint{2.005765in}{2.823188in}}%
\pgfpathlineto{\pgfqpoint{2.017133in}{2.826618in}}%
\pgfpathlineto{\pgfqpoint{2.028492in}{2.830140in}}%
\pgfpathlineto{\pgfqpoint{2.039843in}{2.833780in}}%
\pgfpathlineto{\pgfqpoint{2.034089in}{2.841554in}}%
\pgfpathlineto{\pgfqpoint{2.028338in}{2.849316in}}%
\pgfpathlineto{\pgfqpoint{2.022592in}{2.857065in}}%
\pgfpathlineto{\pgfqpoint{2.016850in}{2.864802in}}%
\pgfpathlineto{\pgfqpoint{2.011111in}{2.872527in}}%
\pgfpathlineto{\pgfqpoint{1.999773in}{2.868874in}}%
\pgfpathlineto{\pgfqpoint{1.988427in}{2.865332in}}%
\pgfpathlineto{\pgfqpoint{1.977073in}{2.861877in}}%
\pgfpathlineto{\pgfqpoint{1.965712in}{2.858487in}}%
\pgfpathlineto{\pgfqpoint{1.954344in}{2.855137in}}%
\pgfpathlineto{\pgfqpoint{1.960069in}{2.847436in}}%
\pgfpathlineto{\pgfqpoint{1.965798in}{2.839722in}}%
\pgfpathlineto{\pgfqpoint{1.971531in}{2.831996in}}%
\pgfpathlineto{\pgfqpoint{1.977268in}{2.824258in}}%
\pgfpathclose%
\pgfusepath{stroke,fill}%
\end{pgfscope}%
\begin{pgfscope}%
\pgfpathrectangle{\pgfqpoint{0.887500in}{0.275000in}}{\pgfqpoint{4.225000in}{4.225000in}}%
\pgfusepath{clip}%
\pgfsetbuttcap%
\pgfsetroundjoin%
\definecolor{currentfill}{rgb}{0.824940,0.884720,0.106217}%
\pgfsetfillcolor{currentfill}%
\pgfsetfillopacity{0.700000}%
\pgfsetlinewidth{0.501875pt}%
\definecolor{currentstroke}{rgb}{1.000000,1.000000,1.000000}%
\pgfsetstrokecolor{currentstroke}%
\pgfsetstrokeopacity{0.500000}%
\pgfsetdash{}{0pt}%
\pgfpathmoveto{\pgfqpoint{3.845895in}{3.551633in}}%
\pgfpathlineto{\pgfqpoint{3.856855in}{3.556053in}}%
\pgfpathlineto{\pgfqpoint{3.867810in}{3.560431in}}%
\pgfpathlineto{\pgfqpoint{3.878759in}{3.564775in}}%
\pgfpathlineto{\pgfqpoint{3.889704in}{3.569097in}}%
\pgfpathlineto{\pgfqpoint{3.900643in}{3.573407in}}%
\pgfpathlineto{\pgfqpoint{3.894355in}{3.585140in}}%
\pgfpathlineto{\pgfqpoint{3.888065in}{3.596710in}}%
\pgfpathlineto{\pgfqpoint{3.881775in}{3.608097in}}%
\pgfpathlineto{\pgfqpoint{3.875484in}{3.619282in}}%
\pgfpathlineto{\pgfqpoint{3.869192in}{3.630246in}}%
\pgfpathlineto{\pgfqpoint{3.858255in}{3.625760in}}%
\pgfpathlineto{\pgfqpoint{3.847314in}{3.621266in}}%
\pgfpathlineto{\pgfqpoint{3.836367in}{3.616752in}}%
\pgfpathlineto{\pgfqpoint{3.825416in}{3.612209in}}%
\pgfpathlineto{\pgfqpoint{3.814460in}{3.607626in}}%
\pgfpathlineto{\pgfqpoint{3.820748in}{3.596828in}}%
\pgfpathlineto{\pgfqpoint{3.827035in}{3.585811in}}%
\pgfpathlineto{\pgfqpoint{3.833322in}{3.574595in}}%
\pgfpathlineto{\pgfqpoint{3.839609in}{3.563196in}}%
\pgfpathclose%
\pgfusepath{stroke,fill}%
\end{pgfscope}%
\begin{pgfscope}%
\pgfpathrectangle{\pgfqpoint{0.887500in}{0.275000in}}{\pgfqpoint{4.225000in}{4.225000in}}%
\pgfusepath{clip}%
\pgfsetbuttcap%
\pgfsetroundjoin%
\definecolor{currentfill}{rgb}{0.120638,0.625828,0.533488}%
\pgfsetfillcolor{currentfill}%
\pgfsetfillopacity{0.700000}%
\pgfsetlinewidth{0.501875pt}%
\definecolor{currentstroke}{rgb}{1.000000,1.000000,1.000000}%
\pgfsetstrokecolor{currentstroke}%
\pgfsetstrokeopacity{0.500000}%
\pgfsetdash{}{0pt}%
\pgfpathmoveto{\pgfqpoint{1.754831in}{2.842349in}}%
\pgfpathlineto{\pgfqpoint{1.766267in}{2.845710in}}%
\pgfpathlineto{\pgfqpoint{1.777698in}{2.849083in}}%
\pgfpathlineto{\pgfqpoint{1.789122in}{2.852467in}}%
\pgfpathlineto{\pgfqpoint{1.800540in}{2.855865in}}%
\pgfpathlineto{\pgfqpoint{1.811953in}{2.859275in}}%
\pgfpathlineto{\pgfqpoint{1.806280in}{2.866837in}}%
\pgfpathlineto{\pgfqpoint{1.800612in}{2.874383in}}%
\pgfpathlineto{\pgfqpoint{1.794947in}{2.881914in}}%
\pgfpathlineto{\pgfqpoint{1.789287in}{2.889428in}}%
\pgfpathlineto{\pgfqpoint{1.783630in}{2.896927in}}%
\pgfpathlineto{\pgfqpoint{1.772232in}{2.893497in}}%
\pgfpathlineto{\pgfqpoint{1.760827in}{2.890083in}}%
\pgfpathlineto{\pgfqpoint{1.749416in}{2.886683in}}%
\pgfpathlineto{\pgfqpoint{1.738000in}{2.883296in}}%
\pgfpathlineto{\pgfqpoint{1.726577in}{2.879919in}}%
\pgfpathlineto{\pgfqpoint{1.732220in}{2.872433in}}%
\pgfpathlineto{\pgfqpoint{1.737866in}{2.864933in}}%
\pgfpathlineto{\pgfqpoint{1.743517in}{2.857419in}}%
\pgfpathlineto{\pgfqpoint{1.749172in}{2.849891in}}%
\pgfpathclose%
\pgfusepath{stroke,fill}%
\end{pgfscope}%
\begin{pgfscope}%
\pgfpathrectangle{\pgfqpoint{0.887500in}{0.275000in}}{\pgfqpoint{4.225000in}{4.225000in}}%
\pgfusepath{clip}%
\pgfsetbuttcap%
\pgfsetroundjoin%
\definecolor{currentfill}{rgb}{0.134692,0.658636,0.517649}%
\pgfsetfillcolor{currentfill}%
\pgfsetfillopacity{0.700000}%
\pgfsetlinewidth{0.501875pt}%
\definecolor{currentstroke}{rgb}{1.000000,1.000000,1.000000}%
\pgfsetstrokecolor{currentstroke}%
\pgfsetstrokeopacity{0.500000}%
\pgfsetdash{}{0pt}%
\pgfpathmoveto{\pgfqpoint{2.550078in}{2.915292in}}%
\pgfpathlineto{\pgfqpoint{2.561282in}{2.922124in}}%
\pgfpathlineto{\pgfqpoint{2.572489in}{2.928403in}}%
\pgfpathlineto{\pgfqpoint{2.583696in}{2.934468in}}%
\pgfpathlineto{\pgfqpoint{2.594897in}{2.940657in}}%
\pgfpathlineto{\pgfqpoint{2.606089in}{2.947309in}}%
\pgfpathlineto{\pgfqpoint{2.600179in}{2.951780in}}%
\pgfpathlineto{\pgfqpoint{2.594276in}{2.956114in}}%
\pgfpathlineto{\pgfqpoint{2.588378in}{2.960466in}}%
\pgfpathlineto{\pgfqpoint{2.582482in}{2.964993in}}%
\pgfpathlineto{\pgfqpoint{2.576588in}{2.969849in}}%
\pgfpathlineto{\pgfqpoint{2.565422in}{2.962581in}}%
\pgfpathlineto{\pgfqpoint{2.554242in}{2.956236in}}%
\pgfpathlineto{\pgfqpoint{2.543049in}{2.950517in}}%
\pgfpathlineto{\pgfqpoint{2.531849in}{2.945129in}}%
\pgfpathlineto{\pgfqpoint{2.520644in}{2.939774in}}%
\pgfpathlineto{\pgfqpoint{2.526525in}{2.934559in}}%
\pgfpathlineto{\pgfqpoint{2.532407in}{2.929734in}}%
\pgfpathlineto{\pgfqpoint{2.538290in}{2.925069in}}%
\pgfpathlineto{\pgfqpoint{2.544180in}{2.920332in}}%
\pgfpathclose%
\pgfusepath{stroke,fill}%
\end{pgfscope}%
\begin{pgfscope}%
\pgfpathrectangle{\pgfqpoint{0.887500in}{0.275000in}}{\pgfqpoint{4.225000in}{4.225000in}}%
\pgfusepath{clip}%
\pgfsetbuttcap%
\pgfsetroundjoin%
\definecolor{currentfill}{rgb}{0.180653,0.701402,0.488189}%
\pgfsetfillcolor{currentfill}%
\pgfsetfillopacity{0.700000}%
\pgfsetlinewidth{0.501875pt}%
\definecolor{currentstroke}{rgb}{1.000000,1.000000,1.000000}%
\pgfsetstrokecolor{currentstroke}%
\pgfsetstrokeopacity{0.500000}%
\pgfsetdash{}{0pt}%
\pgfpathmoveto{\pgfqpoint{2.747755in}{2.975051in}}%
\pgfpathlineto{\pgfqpoint{2.758963in}{2.976353in}}%
\pgfpathlineto{\pgfqpoint{2.770140in}{2.981063in}}%
\pgfpathlineto{\pgfqpoint{2.781275in}{2.991184in}}%
\pgfpathlineto{\pgfqpoint{2.792360in}{3.008723in}}%
\pgfpathlineto{\pgfqpoint{2.803388in}{3.035691in}}%
\pgfpathlineto{\pgfqpoint{2.797365in}{3.044940in}}%
\pgfpathlineto{\pgfqpoint{2.791325in}{3.057331in}}%
\pgfpathlineto{\pgfqpoint{2.785265in}{3.073099in}}%
\pgfpathlineto{\pgfqpoint{2.779186in}{3.091756in}}%
\pgfpathlineto{\pgfqpoint{2.773093in}{3.112566in}}%
\pgfpathlineto{\pgfqpoint{2.762085in}{3.085311in}}%
\pgfpathlineto{\pgfqpoint{2.751034in}{3.064927in}}%
\pgfpathlineto{\pgfqpoint{2.739943in}{3.050039in}}%
\pgfpathlineto{\pgfqpoint{2.728820in}{3.039276in}}%
\pgfpathlineto{\pgfqpoint{2.717672in}{3.031269in}}%
\pgfpathlineto{\pgfqpoint{2.723709in}{3.016820in}}%
\pgfpathlineto{\pgfqpoint{2.729738in}{3.003666in}}%
\pgfpathlineto{\pgfqpoint{2.735756in}{2.992216in}}%
\pgfpathlineto{\pgfqpoint{2.741761in}{2.982733in}}%
\pgfpathclose%
\pgfusepath{stroke,fill}%
\end{pgfscope}%
\begin{pgfscope}%
\pgfpathrectangle{\pgfqpoint{0.887500in}{0.275000in}}{\pgfqpoint{4.225000in}{4.225000in}}%
\pgfusepath{clip}%
\pgfsetbuttcap%
\pgfsetroundjoin%
\definecolor{currentfill}{rgb}{0.772852,0.877868,0.131109}%
\pgfsetfillcolor{currentfill}%
\pgfsetfillopacity{0.700000}%
\pgfsetlinewidth{0.501875pt}%
\definecolor{currentstroke}{rgb}{1.000000,1.000000,1.000000}%
\pgfsetstrokecolor{currentstroke}%
\pgfsetstrokeopacity{0.500000}%
\pgfsetdash{}{0pt}%
\pgfpathmoveto{\pgfqpoint{3.932095in}{3.512969in}}%
\pgfpathlineto{\pgfqpoint{3.943034in}{3.517189in}}%
\pgfpathlineto{\pgfqpoint{3.953968in}{3.521434in}}%
\pgfpathlineto{\pgfqpoint{3.964898in}{3.525710in}}%
\pgfpathlineto{\pgfqpoint{3.975825in}{3.530026in}}%
\pgfpathlineto{\pgfqpoint{3.969528in}{3.542262in}}%
\pgfpathlineto{\pgfqpoint{3.963233in}{3.554476in}}%
\pgfpathlineto{\pgfqpoint{3.956941in}{3.566642in}}%
\pgfpathlineto{\pgfqpoint{3.950649in}{3.578735in}}%
\pgfpathlineto{\pgfqpoint{3.944358in}{3.590730in}}%
\pgfpathlineto{\pgfqpoint{3.933436in}{3.586366in}}%
\pgfpathlineto{\pgfqpoint{3.922509in}{3.582031in}}%
\pgfpathlineto{\pgfqpoint{3.911579in}{3.577715in}}%
\pgfpathlineto{\pgfqpoint{3.900643in}{3.573407in}}%
\pgfpathlineto{\pgfqpoint{3.906932in}{3.561530in}}%
\pgfpathlineto{\pgfqpoint{3.913222in}{3.549529in}}%
\pgfpathlineto{\pgfqpoint{3.919512in}{3.537422in}}%
\pgfpathlineto{\pgfqpoint{3.925803in}{3.525229in}}%
\pgfpathclose%
\pgfusepath{stroke,fill}%
\end{pgfscope}%
\begin{pgfscope}%
\pgfpathrectangle{\pgfqpoint{0.887500in}{0.275000in}}{\pgfqpoint{4.225000in}{4.225000in}}%
\pgfusepath{clip}%
\pgfsetbuttcap%
\pgfsetroundjoin%
\definecolor{currentfill}{rgb}{0.866013,0.889868,0.095953}%
\pgfsetfillcolor{currentfill}%
\pgfsetfillopacity{0.700000}%
\pgfsetlinewidth{0.501875pt}%
\definecolor{currentstroke}{rgb}{1.000000,1.000000,1.000000}%
\pgfsetstrokecolor{currentstroke}%
\pgfsetstrokeopacity{0.500000}%
\pgfsetdash{}{0pt}%
\pgfpathmoveto{\pgfqpoint{3.618232in}{3.588061in}}%
\pgfpathlineto{\pgfqpoint{3.629234in}{3.591781in}}%
\pgfpathlineto{\pgfqpoint{3.640233in}{3.595652in}}%
\pgfpathlineto{\pgfqpoint{3.651231in}{3.599729in}}%
\pgfpathlineto{\pgfqpoint{3.662228in}{3.604065in}}%
\pgfpathlineto{\pgfqpoint{3.673224in}{3.608653in}}%
\pgfpathlineto{\pgfqpoint{3.666946in}{3.617803in}}%
\pgfpathlineto{\pgfqpoint{3.660669in}{3.626747in}}%
\pgfpathlineto{\pgfqpoint{3.654393in}{3.635496in}}%
\pgfpathlineto{\pgfqpoint{3.648118in}{3.644064in}}%
\pgfpathlineto{\pgfqpoint{3.641845in}{3.652464in}}%
\pgfpathlineto{\pgfqpoint{3.630854in}{3.647475in}}%
\pgfpathlineto{\pgfqpoint{3.619859in}{3.642564in}}%
\pgfpathlineto{\pgfqpoint{3.608861in}{3.637735in}}%
\pgfpathlineto{\pgfqpoint{3.597860in}{3.632977in}}%
\pgfpathlineto{\pgfqpoint{3.586855in}{3.628276in}}%
\pgfpathlineto{\pgfqpoint{3.593126in}{3.620516in}}%
\pgfpathlineto{\pgfqpoint{3.599400in}{3.612652in}}%
\pgfpathlineto{\pgfqpoint{3.605676in}{3.604647in}}%
\pgfpathlineto{\pgfqpoint{3.611953in}{3.596462in}}%
\pgfpathclose%
\pgfusepath{stroke,fill}%
\end{pgfscope}%
\begin{pgfscope}%
\pgfpathrectangle{\pgfqpoint{0.887500in}{0.275000in}}{\pgfqpoint{4.225000in}{4.225000in}}%
\pgfusepath{clip}%
\pgfsetbuttcap%
\pgfsetroundjoin%
\definecolor{currentfill}{rgb}{0.120638,0.625828,0.533488}%
\pgfsetfillcolor{currentfill}%
\pgfsetfillopacity{0.700000}%
\pgfsetlinewidth{0.501875pt}%
\definecolor{currentstroke}{rgb}{1.000000,1.000000,1.000000}%
\pgfsetstrokecolor{currentstroke}%
\pgfsetstrokeopacity{0.500000}%
\pgfsetdash{}{0pt}%
\pgfpathmoveto{\pgfqpoint{2.352745in}{2.820676in}}%
\pgfpathlineto{\pgfqpoint{2.363924in}{2.830553in}}%
\pgfpathlineto{\pgfqpoint{2.375107in}{2.840117in}}%
\pgfpathlineto{\pgfqpoint{2.386292in}{2.849383in}}%
\pgfpathlineto{\pgfqpoint{2.397481in}{2.858370in}}%
\pgfpathlineto{\pgfqpoint{2.408671in}{2.867098in}}%
\pgfpathlineto{\pgfqpoint{2.402761in}{2.876633in}}%
\pgfpathlineto{\pgfqpoint{2.396855in}{2.886055in}}%
\pgfpathlineto{\pgfqpoint{2.390954in}{2.895386in}}%
\pgfpathlineto{\pgfqpoint{2.385058in}{2.904635in}}%
\pgfpathlineto{\pgfqpoint{2.379167in}{2.913796in}}%
\pgfpathlineto{\pgfqpoint{2.367994in}{2.904891in}}%
\pgfpathlineto{\pgfqpoint{2.356826in}{2.895535in}}%
\pgfpathlineto{\pgfqpoint{2.345664in}{2.885720in}}%
\pgfpathlineto{\pgfqpoint{2.334509in}{2.875436in}}%
\pgfpathlineto{\pgfqpoint{2.323361in}{2.864679in}}%
\pgfpathlineto{\pgfqpoint{2.329224in}{2.856235in}}%
\pgfpathlineto{\pgfqpoint{2.335094in}{2.847634in}}%
\pgfpathlineto{\pgfqpoint{2.340970in}{2.838844in}}%
\pgfpathlineto{\pgfqpoint{2.346854in}{2.829856in}}%
\pgfpathclose%
\pgfusepath{stroke,fill}%
\end{pgfscope}%
\begin{pgfscope}%
\pgfpathrectangle{\pgfqpoint{0.887500in}{0.275000in}}{\pgfqpoint{4.225000in}{4.225000in}}%
\pgfusepath{clip}%
\pgfsetbuttcap%
\pgfsetroundjoin%
\definecolor{currentfill}{rgb}{0.119738,0.603785,0.541400}%
\pgfsetfillcolor{currentfill}%
\pgfsetfillopacity{0.700000}%
\pgfsetlinewidth{0.501875pt}%
\definecolor{currentstroke}{rgb}{1.000000,1.000000,1.000000}%
\pgfsetstrokecolor{currentstroke}%
\pgfsetstrokeopacity{0.500000}%
\pgfsetdash{}{0pt}%
\pgfpathmoveto{\pgfqpoint{2.068676in}{2.794718in}}%
\pgfpathlineto{\pgfqpoint{2.080032in}{2.798455in}}%
\pgfpathlineto{\pgfqpoint{2.091382in}{2.802230in}}%
\pgfpathlineto{\pgfqpoint{2.102729in}{2.805945in}}%
\pgfpathlineto{\pgfqpoint{2.114073in}{2.809503in}}%
\pgfpathlineto{\pgfqpoint{2.125419in}{2.812809in}}%
\pgfpathlineto{\pgfqpoint{2.119630in}{2.820704in}}%
\pgfpathlineto{\pgfqpoint{2.113845in}{2.828614in}}%
\pgfpathlineto{\pgfqpoint{2.108063in}{2.836544in}}%
\pgfpathlineto{\pgfqpoint{2.102284in}{2.844498in}}%
\pgfpathlineto{\pgfqpoint{2.096509in}{2.852473in}}%
\pgfpathlineto{\pgfqpoint{2.085182in}{2.848901in}}%
\pgfpathlineto{\pgfqpoint{2.073854in}{2.845173in}}%
\pgfpathlineto{\pgfqpoint{2.062522in}{2.841363in}}%
\pgfpathlineto{\pgfqpoint{2.051186in}{2.837541in}}%
\pgfpathlineto{\pgfqpoint{2.039843in}{2.833780in}}%
\pgfpathlineto{\pgfqpoint{2.045602in}{2.825993in}}%
\pgfpathlineto{\pgfqpoint{2.051364in}{2.818194in}}%
\pgfpathlineto{\pgfqpoint{2.057131in}{2.810381in}}%
\pgfpathlineto{\pgfqpoint{2.062901in}{2.802556in}}%
\pgfpathclose%
\pgfusepath{stroke,fill}%
\end{pgfscope}%
\begin{pgfscope}%
\pgfpathrectangle{\pgfqpoint{0.887500in}{0.275000in}}{\pgfqpoint{4.225000in}{4.225000in}}%
\pgfusepath{clip}%
\pgfsetbuttcap%
\pgfsetroundjoin%
\definecolor{currentfill}{rgb}{0.468053,0.818921,0.323998}%
\pgfsetfillcolor{currentfill}%
\pgfsetfillopacity{0.700000}%
\pgfsetlinewidth{0.501875pt}%
\definecolor{currentstroke}{rgb}{1.000000,1.000000,1.000000}%
\pgfsetstrokecolor{currentstroke}%
\pgfsetstrokeopacity{0.500000}%
\pgfsetdash{}{0pt}%
\pgfpathmoveto{\pgfqpoint{2.742577in}{3.223096in}}%
\pgfpathlineto{\pgfqpoint{2.753568in}{3.252326in}}%
\pgfpathlineto{\pgfqpoint{2.764558in}{3.282997in}}%
\pgfpathlineto{\pgfqpoint{2.775554in}{3.314196in}}%
\pgfpathlineto{\pgfqpoint{2.786563in}{3.345007in}}%
\pgfpathlineto{\pgfqpoint{2.797590in}{3.374509in}}%
\pgfpathlineto{\pgfqpoint{2.791573in}{3.381627in}}%
\pgfpathlineto{\pgfqpoint{2.785577in}{3.386294in}}%
\pgfpathlineto{\pgfqpoint{2.779605in}{3.388055in}}%
\pgfpathlineto{\pgfqpoint{2.773659in}{3.386843in}}%
\pgfpathlineto{\pgfqpoint{2.767739in}{3.383070in}}%
\pgfpathlineto{\pgfqpoint{2.756669in}{3.364582in}}%
\pgfpathlineto{\pgfqpoint{2.745612in}{3.344899in}}%
\pgfpathlineto{\pgfqpoint{2.734567in}{3.324086in}}%
\pgfpathlineto{\pgfqpoint{2.723536in}{3.302206in}}%
\pgfpathlineto{\pgfqpoint{2.712518in}{3.279322in}}%
\pgfpathlineto{\pgfqpoint{2.718463in}{3.275384in}}%
\pgfpathlineto{\pgfqpoint{2.724441in}{3.267923in}}%
\pgfpathlineto{\pgfqpoint{2.730456in}{3.256498in}}%
\pgfpathlineto{\pgfqpoint{2.736504in}{3.241313in}}%
\pgfpathclose%
\pgfusepath{stroke,fill}%
\end{pgfscope}%
\begin{pgfscope}%
\pgfpathrectangle{\pgfqpoint{0.887500in}{0.275000in}}{\pgfqpoint{4.225000in}{4.225000in}}%
\pgfusepath{clip}%
\pgfsetbuttcap%
\pgfsetroundjoin%
\definecolor{currentfill}{rgb}{0.119699,0.618490,0.536347}%
\pgfsetfillcolor{currentfill}%
\pgfsetfillopacity{0.700000}%
\pgfsetlinewidth{0.501875pt}%
\definecolor{currentstroke}{rgb}{1.000000,1.000000,1.000000}%
\pgfsetstrokecolor{currentstroke}%
\pgfsetstrokeopacity{0.500000}%
\pgfsetdash{}{0pt}%
\pgfpathmoveto{\pgfqpoint{1.840380in}{2.821248in}}%
\pgfpathlineto{\pgfqpoint{1.851800in}{2.824648in}}%
\pgfpathlineto{\pgfqpoint{1.863215in}{2.828059in}}%
\pgfpathlineto{\pgfqpoint{1.874624in}{2.831477in}}%
\pgfpathlineto{\pgfqpoint{1.886028in}{2.834897in}}%
\pgfpathlineto{\pgfqpoint{1.897427in}{2.838311in}}%
\pgfpathlineto{\pgfqpoint{1.891719in}{2.845970in}}%
\pgfpathlineto{\pgfqpoint{1.886016in}{2.853615in}}%
\pgfpathlineto{\pgfqpoint{1.880317in}{2.861247in}}%
\pgfpathlineto{\pgfqpoint{1.874622in}{2.868866in}}%
\pgfpathlineto{\pgfqpoint{1.868931in}{2.876470in}}%
\pgfpathlineto{\pgfqpoint{1.857546in}{2.873027in}}%
\pgfpathlineto{\pgfqpoint{1.846156in}{2.869580in}}%
\pgfpathlineto{\pgfqpoint{1.834761in}{2.866135in}}%
\pgfpathlineto{\pgfqpoint{1.823360in}{2.862699in}}%
\pgfpathlineto{\pgfqpoint{1.811953in}{2.859275in}}%
\pgfpathlineto{\pgfqpoint{1.817630in}{2.851699in}}%
\pgfpathlineto{\pgfqpoint{1.823311in}{2.844107in}}%
\pgfpathlineto{\pgfqpoint{1.828997in}{2.836502in}}%
\pgfpathlineto{\pgfqpoint{1.834686in}{2.828882in}}%
\pgfpathclose%
\pgfusepath{stroke,fill}%
\end{pgfscope}%
\begin{pgfscope}%
\pgfpathrectangle{\pgfqpoint{0.887500in}{0.275000in}}{\pgfqpoint{4.225000in}{4.225000in}}%
\pgfusepath{clip}%
\pgfsetbuttcap%
\pgfsetroundjoin%
\definecolor{currentfill}{rgb}{0.730889,0.871916,0.156029}%
\pgfsetfillcolor{currentfill}%
\pgfsetfillopacity{0.700000}%
\pgfsetlinewidth{0.501875pt}%
\definecolor{currentstroke}{rgb}{1.000000,1.000000,1.000000}%
\pgfsetstrokecolor{currentstroke}%
\pgfsetstrokeopacity{0.500000}%
\pgfsetdash{}{0pt}%
\pgfpathmoveto{\pgfqpoint{2.908935in}{3.514328in}}%
\pgfpathlineto{\pgfqpoint{2.920110in}{3.517127in}}%
\pgfpathlineto{\pgfqpoint{2.931282in}{3.519185in}}%
\pgfpathlineto{\pgfqpoint{2.942450in}{3.520833in}}%
\pgfpathlineto{\pgfqpoint{2.953611in}{3.522399in}}%
\pgfpathlineto{\pgfqpoint{2.964766in}{3.524215in}}%
\pgfpathlineto{\pgfqpoint{2.958711in}{3.525700in}}%
\pgfpathlineto{\pgfqpoint{2.952663in}{3.526735in}}%
\pgfpathlineto{\pgfqpoint{2.946622in}{3.527250in}}%
\pgfpathlineto{\pgfqpoint{2.940589in}{3.527249in}}%
\pgfpathlineto{\pgfqpoint{2.934563in}{3.526813in}}%
\pgfpathlineto{\pgfqpoint{2.923436in}{3.522628in}}%
\pgfpathlineto{\pgfqpoint{2.912306in}{3.518044in}}%
\pgfpathlineto{\pgfqpoint{2.901173in}{3.512968in}}%
\pgfpathlineto{\pgfqpoint{2.890038in}{3.507310in}}%
\pgfpathlineto{\pgfqpoint{2.878903in}{3.500978in}}%
\pgfpathlineto{\pgfqpoint{2.884886in}{3.505727in}}%
\pgfpathlineto{\pgfqpoint{2.890881in}{3.509555in}}%
\pgfpathlineto{\pgfqpoint{2.896888in}{3.512257in}}%
\pgfpathlineto{\pgfqpoint{2.902906in}{3.513807in}}%
\pgfpathclose%
\pgfusepath{stroke,fill}%
\end{pgfscope}%
\begin{pgfscope}%
\pgfpathrectangle{\pgfqpoint{0.887500in}{0.275000in}}{\pgfqpoint{4.225000in}{4.225000in}}%
\pgfusepath{clip}%
\pgfsetbuttcap%
\pgfsetroundjoin%
\definecolor{currentfill}{rgb}{0.855810,0.888601,0.097452}%
\pgfsetfillcolor{currentfill}%
\pgfsetfillopacity{0.700000}%
\pgfsetlinewidth{0.501875pt}%
\definecolor{currentstroke}{rgb}{1.000000,1.000000,1.000000}%
\pgfsetstrokecolor{currentstroke}%
\pgfsetstrokeopacity{0.500000}%
\pgfsetdash{}{0pt}%
\pgfpathmoveto{\pgfqpoint{3.476574in}{3.581954in}}%
\pgfpathlineto{\pgfqpoint{3.487622in}{3.586645in}}%
\pgfpathlineto{\pgfqpoint{3.498666in}{3.591308in}}%
\pgfpathlineto{\pgfqpoint{3.509706in}{3.595950in}}%
\pgfpathlineto{\pgfqpoint{3.520740in}{3.600574in}}%
\pgfpathlineto{\pgfqpoint{3.531771in}{3.605186in}}%
\pgfpathlineto{\pgfqpoint{3.525506in}{3.612084in}}%
\pgfpathlineto{\pgfqpoint{3.519246in}{3.619080in}}%
\pgfpathlineto{\pgfqpoint{3.512992in}{3.626194in}}%
\pgfpathlineto{\pgfqpoint{3.506744in}{3.633383in}}%
\pgfpathlineto{\pgfqpoint{3.500500in}{3.640597in}}%
\pgfpathlineto{\pgfqpoint{3.489476in}{3.635457in}}%
\pgfpathlineto{\pgfqpoint{3.478449in}{3.630312in}}%
\pgfpathlineto{\pgfqpoint{3.467417in}{3.625155in}}%
\pgfpathlineto{\pgfqpoint{3.456381in}{3.619980in}}%
\pgfpathlineto{\pgfqpoint{3.445341in}{3.614780in}}%
\pgfpathlineto{\pgfqpoint{3.451576in}{3.607974in}}%
\pgfpathlineto{\pgfqpoint{3.457816in}{3.601243in}}%
\pgfpathlineto{\pgfqpoint{3.464063in}{3.594643in}}%
\pgfpathlineto{\pgfqpoint{3.470315in}{3.588222in}}%
\pgfpathclose%
\pgfusepath{stroke,fill}%
\end{pgfscope}%
\begin{pgfscope}%
\pgfpathrectangle{\pgfqpoint{0.887500in}{0.275000in}}{\pgfqpoint{4.225000in}{4.225000in}}%
\pgfusepath{clip}%
\pgfsetbuttcap%
\pgfsetroundjoin%
\definecolor{currentfill}{rgb}{0.157851,0.683765,0.501686}%
\pgfsetfillcolor{currentfill}%
\pgfsetfillopacity{0.700000}%
\pgfsetlinewidth{0.501875pt}%
\definecolor{currentstroke}{rgb}{1.000000,1.000000,1.000000}%
\pgfsetstrokecolor{currentstroke}%
\pgfsetstrokeopacity{0.500000}%
\pgfsetdash{}{0pt}%
\pgfpathmoveto{\pgfqpoint{2.691695in}{2.960098in}}%
\pgfpathlineto{\pgfqpoint{2.702885in}{2.966875in}}%
\pgfpathlineto{\pgfqpoint{2.714086in}{2.971900in}}%
\pgfpathlineto{\pgfqpoint{2.725302in}{2.974660in}}%
\pgfpathlineto{\pgfqpoint{2.736529in}{2.975154in}}%
\pgfpathlineto{\pgfqpoint{2.747755in}{2.975051in}}%
\pgfpathlineto{\pgfqpoint{2.741761in}{2.982733in}}%
\pgfpathlineto{\pgfqpoint{2.735756in}{2.992216in}}%
\pgfpathlineto{\pgfqpoint{2.729738in}{3.003666in}}%
\pgfpathlineto{\pgfqpoint{2.723709in}{3.016820in}}%
\pgfpathlineto{\pgfqpoint{2.717672in}{3.031269in}}%
\pgfpathlineto{\pgfqpoint{2.706509in}{3.024653in}}%
\pgfpathlineto{\pgfqpoint{2.695343in}{3.018063in}}%
\pgfpathlineto{\pgfqpoint{2.684183in}{3.010368in}}%
\pgfpathlineto{\pgfqpoint{2.673031in}{3.001592in}}%
\pgfpathlineto{\pgfqpoint{2.661884in}{2.992115in}}%
\pgfpathlineto{\pgfqpoint{2.667840in}{2.985485in}}%
\pgfpathlineto{\pgfqpoint{2.673799in}{2.978960in}}%
\pgfpathlineto{\pgfqpoint{2.679761in}{2.972562in}}%
\pgfpathlineto{\pgfqpoint{2.685726in}{2.966294in}}%
\pgfpathclose%
\pgfusepath{stroke,fill}%
\end{pgfscope}%
\begin{pgfscope}%
\pgfpathrectangle{\pgfqpoint{0.887500in}{0.275000in}}{\pgfqpoint{4.225000in}{4.225000in}}%
\pgfusepath{clip}%
\pgfsetbuttcap%
\pgfsetroundjoin%
\definecolor{currentfill}{rgb}{0.119738,0.603785,0.541400}%
\pgfsetfillcolor{currentfill}%
\pgfsetfillopacity{0.700000}%
\pgfsetlinewidth{0.501875pt}%
\definecolor{currentstroke}{rgb}{1.000000,1.000000,1.000000}%
\pgfsetstrokecolor{currentstroke}%
\pgfsetstrokeopacity{0.500000}%
\pgfsetdash{}{0pt}%
\pgfpathmoveto{\pgfqpoint{2.296788in}{2.773170in}}%
\pgfpathlineto{\pgfqpoint{2.308003in}{2.781594in}}%
\pgfpathlineto{\pgfqpoint{2.319202in}{2.790806in}}%
\pgfpathlineto{\pgfqpoint{2.330388in}{2.800556in}}%
\pgfpathlineto{\pgfqpoint{2.341567in}{2.810596in}}%
\pgfpathlineto{\pgfqpoint{2.352745in}{2.820676in}}%
\pgfpathlineto{\pgfqpoint{2.346854in}{2.829856in}}%
\pgfpathlineto{\pgfqpoint{2.340970in}{2.838844in}}%
\pgfpathlineto{\pgfqpoint{2.335094in}{2.847634in}}%
\pgfpathlineto{\pgfqpoint{2.329224in}{2.856235in}}%
\pgfpathlineto{\pgfqpoint{2.323361in}{2.864679in}}%
\pgfpathlineto{\pgfqpoint{2.312218in}{2.853600in}}%
\pgfpathlineto{\pgfqpoint{2.301074in}{2.842505in}}%
\pgfpathlineto{\pgfqpoint{2.289924in}{2.831709in}}%
\pgfpathlineto{\pgfqpoint{2.278760in}{2.821527in}}%
\pgfpathlineto{\pgfqpoint{2.267577in}{2.812272in}}%
\pgfpathlineto{\pgfqpoint{2.273407in}{2.804663in}}%
\pgfpathlineto{\pgfqpoint{2.279243in}{2.796981in}}%
\pgfpathlineto{\pgfqpoint{2.285085in}{2.789177in}}%
\pgfpathlineto{\pgfqpoint{2.290933in}{2.781236in}}%
\pgfpathclose%
\pgfusepath{stroke,fill}%
\end{pgfscope}%
\begin{pgfscope}%
\pgfpathrectangle{\pgfqpoint{0.887500in}{0.275000in}}{\pgfqpoint{4.225000in}{4.225000in}}%
\pgfusepath{clip}%
\pgfsetbuttcap%
\pgfsetroundjoin%
\definecolor{currentfill}{rgb}{0.835270,0.886029,0.102646}%
\pgfsetfillcolor{currentfill}%
\pgfsetfillopacity{0.700000}%
\pgfsetlinewidth{0.501875pt}%
\definecolor{currentstroke}{rgb}{1.000000,1.000000,1.000000}%
\pgfsetstrokecolor{currentstroke}%
\pgfsetstrokeopacity{0.500000}%
\pgfsetdash{}{0pt}%
\pgfpathmoveto{\pgfqpoint{3.704613in}{3.559373in}}%
\pgfpathlineto{\pgfqpoint{3.715613in}{3.563945in}}%
\pgfpathlineto{\pgfqpoint{3.726612in}{3.568744in}}%
\pgfpathlineto{\pgfqpoint{3.737609in}{3.573692in}}%
\pgfpathlineto{\pgfqpoint{3.748604in}{3.578714in}}%
\pgfpathlineto{\pgfqpoint{3.759594in}{3.583733in}}%
\pgfpathlineto{\pgfqpoint{3.753310in}{3.594185in}}%
\pgfpathlineto{\pgfqpoint{3.747026in}{3.604389in}}%
\pgfpathlineto{\pgfqpoint{3.740741in}{3.614345in}}%
\pgfpathlineto{\pgfqpoint{3.734455in}{3.624063in}}%
\pgfpathlineto{\pgfqpoint{3.728170in}{3.633551in}}%
\pgfpathlineto{\pgfqpoint{3.717188in}{3.628485in}}%
\pgfpathlineto{\pgfqpoint{3.706201in}{3.623410in}}%
\pgfpathlineto{\pgfqpoint{3.695211in}{3.618378in}}%
\pgfpathlineto{\pgfqpoint{3.684219in}{3.613441in}}%
\pgfpathlineto{\pgfqpoint{3.673224in}{3.608653in}}%
\pgfpathlineto{\pgfqpoint{3.679502in}{3.599283in}}%
\pgfpathlineto{\pgfqpoint{3.685780in}{3.589682in}}%
\pgfpathlineto{\pgfqpoint{3.692058in}{3.579836in}}%
\pgfpathlineto{\pgfqpoint{3.698336in}{3.569733in}}%
\pgfpathclose%
\pgfusepath{stroke,fill}%
\end{pgfscope}%
\begin{pgfscope}%
\pgfpathrectangle{\pgfqpoint{0.887500in}{0.275000in}}{\pgfqpoint{4.225000in}{4.225000in}}%
\pgfusepath{clip}%
\pgfsetbuttcap%
\pgfsetroundjoin%
\definecolor{currentfill}{rgb}{0.835270,0.886029,0.102646}%
\pgfsetfillcolor{currentfill}%
\pgfsetfillopacity{0.700000}%
\pgfsetlinewidth{0.501875pt}%
\definecolor{currentstroke}{rgb}{1.000000,1.000000,1.000000}%
\pgfsetstrokecolor{currentstroke}%
\pgfsetstrokeopacity{0.500000}%
\pgfsetdash{}{0pt}%
\pgfpathmoveto{\pgfqpoint{3.334764in}{3.570759in}}%
\pgfpathlineto{\pgfqpoint{3.345833in}{3.573821in}}%
\pgfpathlineto{\pgfqpoint{3.356899in}{3.577162in}}%
\pgfpathlineto{\pgfqpoint{3.367962in}{3.580869in}}%
\pgfpathlineto{\pgfqpoint{3.379023in}{3.584987in}}%
\pgfpathlineto{\pgfqpoint{3.390083in}{3.589461in}}%
\pgfpathlineto{\pgfqpoint{3.383862in}{3.595834in}}%
\pgfpathlineto{\pgfqpoint{3.377646in}{3.602190in}}%
\pgfpathlineto{\pgfqpoint{3.371433in}{3.608473in}}%
\pgfpathlineto{\pgfqpoint{3.365224in}{3.614629in}}%
\pgfpathlineto{\pgfqpoint{3.359019in}{3.620602in}}%
\pgfpathlineto{\pgfqpoint{3.347968in}{3.615497in}}%
\pgfpathlineto{\pgfqpoint{3.336915in}{3.610628in}}%
\pgfpathlineto{\pgfqpoint{3.325860in}{3.606035in}}%
\pgfpathlineto{\pgfqpoint{3.314802in}{3.601709in}}%
\pgfpathlineto{\pgfqpoint{3.303740in}{3.597616in}}%
\pgfpathlineto{\pgfqpoint{3.309937in}{3.592477in}}%
\pgfpathlineto{\pgfqpoint{3.316138in}{3.587197in}}%
\pgfpathlineto{\pgfqpoint{3.322342in}{3.581802in}}%
\pgfpathlineto{\pgfqpoint{3.328551in}{3.576314in}}%
\pgfpathclose%
\pgfusepath{stroke,fill}%
\end{pgfscope}%
\begin{pgfscope}%
\pgfpathrectangle{\pgfqpoint{0.887500in}{0.275000in}}{\pgfqpoint{4.225000in}{4.225000in}}%
\pgfusepath{clip}%
\pgfsetbuttcap%
\pgfsetroundjoin%
\definecolor{currentfill}{rgb}{0.128087,0.647749,0.523491}%
\pgfsetfillcolor{currentfill}%
\pgfsetfillopacity{0.700000}%
\pgfsetlinewidth{0.501875pt}%
\definecolor{currentstroke}{rgb}{1.000000,1.000000,1.000000}%
\pgfsetstrokecolor{currentstroke}%
\pgfsetstrokeopacity{0.500000}%
\pgfsetdash{}{0pt}%
\pgfpathmoveto{\pgfqpoint{2.494188in}{2.868190in}}%
\pgfpathlineto{\pgfqpoint{2.505353in}{2.878854in}}%
\pgfpathlineto{\pgfqpoint{2.516521in}{2.889133in}}%
\pgfpathlineto{\pgfqpoint{2.527697in}{2.898785in}}%
\pgfpathlineto{\pgfqpoint{2.538882in}{2.907570in}}%
\pgfpathlineto{\pgfqpoint{2.550078in}{2.915292in}}%
\pgfpathlineto{\pgfqpoint{2.544180in}{2.920332in}}%
\pgfpathlineto{\pgfqpoint{2.538290in}{2.925069in}}%
\pgfpathlineto{\pgfqpoint{2.532407in}{2.929734in}}%
\pgfpathlineto{\pgfqpoint{2.526525in}{2.934559in}}%
\pgfpathlineto{\pgfqpoint{2.520644in}{2.939774in}}%
\pgfpathlineto{\pgfqpoint{2.509438in}{2.934157in}}%
\pgfpathlineto{\pgfqpoint{2.498235in}{2.928081in}}%
\pgfpathlineto{\pgfqpoint{2.487034in}{2.921560in}}%
\pgfpathlineto{\pgfqpoint{2.475836in}{2.914639in}}%
\pgfpathlineto{\pgfqpoint{2.464639in}{2.907365in}}%
\pgfpathlineto{\pgfqpoint{2.470540in}{2.899603in}}%
\pgfpathlineto{\pgfqpoint{2.476443in}{2.891940in}}%
\pgfpathlineto{\pgfqpoint{2.482351in}{2.884242in}}%
\pgfpathlineto{\pgfqpoint{2.488266in}{2.876370in}}%
\pgfpathclose%
\pgfusepath{stroke,fill}%
\end{pgfscope}%
\begin{pgfscope}%
\pgfpathrectangle{\pgfqpoint{0.887500in}{0.275000in}}{\pgfqpoint{4.225000in}{4.225000in}}%
\pgfusepath{clip}%
\pgfsetbuttcap%
\pgfsetroundjoin%
\definecolor{currentfill}{rgb}{0.783315,0.879285,0.125405}%
\pgfsetfillcolor{currentfill}%
\pgfsetfillopacity{0.700000}%
\pgfsetlinewidth{0.501875pt}%
\definecolor{currentstroke}{rgb}{1.000000,1.000000,1.000000}%
\pgfsetstrokecolor{currentstroke}%
\pgfsetstrokeopacity{0.500000}%
\pgfsetdash{}{0pt}%
\pgfpathmoveto{\pgfqpoint{3.050906in}{3.534804in}}%
\pgfpathlineto{\pgfqpoint{3.062046in}{3.540575in}}%
\pgfpathlineto{\pgfqpoint{3.073182in}{3.546156in}}%
\pgfpathlineto{\pgfqpoint{3.084314in}{3.551306in}}%
\pgfpathlineto{\pgfqpoint{3.095441in}{3.555999in}}%
\pgfpathlineto{\pgfqpoint{3.106563in}{3.560320in}}%
\pgfpathlineto{\pgfqpoint{3.100437in}{3.561668in}}%
\pgfpathlineto{\pgfqpoint{3.094317in}{3.562721in}}%
\pgfpathlineto{\pgfqpoint{3.088203in}{3.563535in}}%
\pgfpathlineto{\pgfqpoint{3.082094in}{3.564163in}}%
\pgfpathlineto{\pgfqpoint{3.075992in}{3.564660in}}%
\pgfpathlineto{\pgfqpoint{3.064891in}{3.559965in}}%
\pgfpathlineto{\pgfqpoint{3.053785in}{3.555286in}}%
\pgfpathlineto{\pgfqpoint{3.042675in}{3.550609in}}%
\pgfpathlineto{\pgfqpoint{3.031561in}{3.545959in}}%
\pgfpathlineto{\pgfqpoint{3.020442in}{3.541429in}}%
\pgfpathlineto{\pgfqpoint{3.026523in}{3.540419in}}%
\pgfpathlineto{\pgfqpoint{3.032609in}{3.539270in}}%
\pgfpathlineto{\pgfqpoint{3.038702in}{3.537963in}}%
\pgfpathlineto{\pgfqpoint{3.044801in}{3.536480in}}%
\pgfpathclose%
\pgfusepath{stroke,fill}%
\end{pgfscope}%
\begin{pgfscope}%
\pgfpathrectangle{\pgfqpoint{0.887500in}{0.275000in}}{\pgfqpoint{4.225000in}{4.225000in}}%
\pgfusepath{clip}%
\pgfsetbuttcap%
\pgfsetroundjoin%
\definecolor{currentfill}{rgb}{0.120565,0.596422,0.543611}%
\pgfsetfillcolor{currentfill}%
\pgfsetfillopacity{0.700000}%
\pgfsetlinewidth{0.501875pt}%
\definecolor{currentstroke}{rgb}{1.000000,1.000000,1.000000}%
\pgfsetstrokecolor{currentstroke}%
\pgfsetstrokeopacity{0.500000}%
\pgfsetdash{}{0pt}%
\pgfpathmoveto{\pgfqpoint{2.154416in}{2.773391in}}%
\pgfpathlineto{\pgfqpoint{2.165774in}{2.776424in}}%
\pgfpathlineto{\pgfqpoint{2.177136in}{2.779059in}}%
\pgfpathlineto{\pgfqpoint{2.188499in}{2.781333in}}%
\pgfpathlineto{\pgfqpoint{2.199858in}{2.783529in}}%
\pgfpathlineto{\pgfqpoint{2.211206in}{2.785960in}}%
\pgfpathlineto{\pgfqpoint{2.205394in}{2.793551in}}%
\pgfpathlineto{\pgfqpoint{2.199584in}{2.801182in}}%
\pgfpathlineto{\pgfqpoint{2.193777in}{2.808885in}}%
\pgfpathlineto{\pgfqpoint{2.187972in}{2.816693in}}%
\pgfpathlineto{\pgfqpoint{2.182167in}{2.824637in}}%
\pgfpathlineto{\pgfqpoint{2.170827in}{2.822388in}}%
\pgfpathlineto{\pgfqpoint{2.159475in}{2.820381in}}%
\pgfpathlineto{\pgfqpoint{2.148119in}{2.818276in}}%
\pgfpathlineto{\pgfqpoint{2.136767in}{2.815765in}}%
\pgfpathlineto{\pgfqpoint{2.125419in}{2.812809in}}%
\pgfpathlineto{\pgfqpoint{2.131211in}{2.804924in}}%
\pgfpathlineto{\pgfqpoint{2.137006in}{2.797045in}}%
\pgfpathlineto{\pgfqpoint{2.142806in}{2.789166in}}%
\pgfpathlineto{\pgfqpoint{2.148609in}{2.781283in}}%
\pgfpathclose%
\pgfusepath{stroke,fill}%
\end{pgfscope}%
\begin{pgfscope}%
\pgfpathrectangle{\pgfqpoint{0.887500in}{0.275000in}}{\pgfqpoint{4.225000in}{4.225000in}}%
\pgfusepath{clip}%
\pgfsetbuttcap%
\pgfsetroundjoin%
\definecolor{currentfill}{rgb}{0.121380,0.629492,0.531973}%
\pgfsetfillcolor{currentfill}%
\pgfsetfillopacity{0.700000}%
\pgfsetlinewidth{0.501875pt}%
\definecolor{currentstroke}{rgb}{1.000000,1.000000,1.000000}%
\pgfsetstrokecolor{currentstroke}%
\pgfsetstrokeopacity{0.500000}%
\pgfsetdash{}{0pt}%
\pgfpathmoveto{\pgfqpoint{1.612041in}{2.846494in}}%
\pgfpathlineto{\pgfqpoint{1.623520in}{2.849824in}}%
\pgfpathlineto{\pgfqpoint{1.634993in}{2.853155in}}%
\pgfpathlineto{\pgfqpoint{1.646461in}{2.856486in}}%
\pgfpathlineto{\pgfqpoint{1.657923in}{2.859820in}}%
\pgfpathlineto{\pgfqpoint{1.669379in}{2.863156in}}%
\pgfpathlineto{\pgfqpoint{1.663755in}{2.870606in}}%
\pgfpathlineto{\pgfqpoint{1.658135in}{2.878042in}}%
\pgfpathlineto{\pgfqpoint{1.652519in}{2.885464in}}%
\pgfpathlineto{\pgfqpoint{1.646907in}{2.892872in}}%
\pgfpathlineto{\pgfqpoint{1.635463in}{2.889505in}}%
\pgfpathlineto{\pgfqpoint{1.624012in}{2.886138in}}%
\pgfpathlineto{\pgfqpoint{1.612557in}{2.882769in}}%
\pgfpathlineto{\pgfqpoint{1.601096in}{2.879399in}}%
\pgfpathlineto{\pgfqpoint{1.589629in}{2.876029in}}%
\pgfpathlineto{\pgfqpoint{1.595226in}{2.868671in}}%
\pgfpathlineto{\pgfqpoint{1.600826in}{2.861296in}}%
\pgfpathlineto{\pgfqpoint{1.606431in}{2.853904in}}%
\pgfpathclose%
\pgfusepath{stroke,fill}%
\end{pgfscope}%
\begin{pgfscope}%
\pgfpathrectangle{\pgfqpoint{0.887500in}{0.275000in}}{\pgfqpoint{4.225000in}{4.225000in}}%
\pgfusepath{clip}%
\pgfsetbuttcap%
\pgfsetroundjoin%
\definecolor{currentfill}{rgb}{0.824940,0.884720,0.106217}%
\pgfsetfillcolor{currentfill}%
\pgfsetfillopacity{0.700000}%
\pgfsetlinewidth{0.501875pt}%
\definecolor{currentstroke}{rgb}{1.000000,1.000000,1.000000}%
\pgfsetstrokecolor{currentstroke}%
\pgfsetstrokeopacity{0.500000}%
\pgfsetdash{}{0pt}%
\pgfpathmoveto{\pgfqpoint{3.192888in}{3.563497in}}%
\pgfpathlineto{\pgfqpoint{3.203995in}{3.566606in}}%
\pgfpathlineto{\pgfqpoint{3.215097in}{3.569761in}}%
\pgfpathlineto{\pgfqpoint{3.226194in}{3.572967in}}%
\pgfpathlineto{\pgfqpoint{3.237286in}{3.576230in}}%
\pgfpathlineto{\pgfqpoint{3.248373in}{3.579556in}}%
\pgfpathlineto{\pgfqpoint{3.242192in}{3.583693in}}%
\pgfpathlineto{\pgfqpoint{3.236015in}{3.587694in}}%
\pgfpathlineto{\pgfqpoint{3.229842in}{3.591562in}}%
\pgfpathlineto{\pgfqpoint{3.223675in}{3.595287in}}%
\pgfpathlineto{\pgfqpoint{3.217512in}{3.598856in}}%
\pgfpathlineto{\pgfqpoint{3.206438in}{3.594810in}}%
\pgfpathlineto{\pgfqpoint{3.195360in}{3.590833in}}%
\pgfpathlineto{\pgfqpoint{3.184277in}{3.586923in}}%
\pgfpathlineto{\pgfqpoint{3.173190in}{3.583078in}}%
\pgfpathlineto{\pgfqpoint{3.162098in}{3.579296in}}%
\pgfpathlineto{\pgfqpoint{3.168246in}{3.576721in}}%
\pgfpathlineto{\pgfqpoint{3.174400in}{3.573871in}}%
\pgfpathlineto{\pgfqpoint{3.180558in}{3.570722in}}%
\pgfpathlineto{\pgfqpoint{3.186721in}{3.567262in}}%
\pgfpathclose%
\pgfusepath{stroke,fill}%
\end{pgfscope}%
\begin{pgfscope}%
\pgfpathrectangle{\pgfqpoint{0.887500in}{0.275000in}}{\pgfqpoint{4.225000in}{4.225000in}}%
\pgfusepath{clip}%
\pgfsetbuttcap%
\pgfsetroundjoin%
\definecolor{currentfill}{rgb}{0.804182,0.882046,0.114965}%
\pgfsetfillcolor{currentfill}%
\pgfsetfillopacity{0.700000}%
\pgfsetlinewidth{0.501875pt}%
\definecolor{currentstroke}{rgb}{1.000000,1.000000,1.000000}%
\pgfsetstrokecolor{currentstroke}%
\pgfsetstrokeopacity{0.500000}%
\pgfsetdash{}{0pt}%
\pgfpathmoveto{\pgfqpoint{3.791009in}{3.528522in}}%
\pgfpathlineto{\pgfqpoint{3.801999in}{3.533313in}}%
\pgfpathlineto{\pgfqpoint{3.812982in}{3.538009in}}%
\pgfpathlineto{\pgfqpoint{3.823959in}{3.542621in}}%
\pgfpathlineto{\pgfqpoint{3.834930in}{3.547159in}}%
\pgfpathlineto{\pgfqpoint{3.845895in}{3.551633in}}%
\pgfpathlineto{\pgfqpoint{3.839609in}{3.563196in}}%
\pgfpathlineto{\pgfqpoint{3.833322in}{3.574595in}}%
\pgfpathlineto{\pgfqpoint{3.827035in}{3.585811in}}%
\pgfpathlineto{\pgfqpoint{3.820748in}{3.596828in}}%
\pgfpathlineto{\pgfqpoint{3.814460in}{3.607626in}}%
\pgfpathlineto{\pgfqpoint{3.803498in}{3.602991in}}%
\pgfpathlineto{\pgfqpoint{3.792530in}{3.598296in}}%
\pgfpathlineto{\pgfqpoint{3.781557in}{3.593528in}}%
\pgfpathlineto{\pgfqpoint{3.770579in}{3.588677in}}%
\pgfpathlineto{\pgfqpoint{3.759594in}{3.583733in}}%
\pgfpathlineto{\pgfqpoint{3.765877in}{3.573053in}}%
\pgfpathlineto{\pgfqpoint{3.772160in}{3.562168in}}%
\pgfpathlineto{\pgfqpoint{3.778442in}{3.551103in}}%
\pgfpathlineto{\pgfqpoint{3.784726in}{3.539880in}}%
\pgfpathclose%
\pgfusepath{stroke,fill}%
\end{pgfscope}%
\begin{pgfscope}%
\pgfpathrectangle{\pgfqpoint{0.887500in}{0.275000in}}{\pgfqpoint{4.225000in}{4.225000in}}%
\pgfusepath{clip}%
\pgfsetbuttcap%
\pgfsetroundjoin%
\definecolor{currentfill}{rgb}{0.119423,0.611141,0.538982}%
\pgfsetfillcolor{currentfill}%
\pgfsetfillopacity{0.700000}%
\pgfsetlinewidth{0.501875pt}%
\definecolor{currentstroke}{rgb}{1.000000,1.000000,1.000000}%
\pgfsetstrokecolor{currentstroke}%
\pgfsetstrokeopacity{0.500000}%
\pgfsetdash{}{0pt}%
\pgfpathmoveto{\pgfqpoint{1.926024in}{2.799829in}}%
\pgfpathlineto{\pgfqpoint{1.937431in}{2.803206in}}%
\pgfpathlineto{\pgfqpoint{1.948833in}{2.806565in}}%
\pgfpathlineto{\pgfqpoint{1.960230in}{2.809899in}}%
\pgfpathlineto{\pgfqpoint{1.971622in}{2.813207in}}%
\pgfpathlineto{\pgfqpoint{1.983010in}{2.816508in}}%
\pgfpathlineto{\pgfqpoint{1.977268in}{2.824258in}}%
\pgfpathlineto{\pgfqpoint{1.971531in}{2.831996in}}%
\pgfpathlineto{\pgfqpoint{1.965798in}{2.839722in}}%
\pgfpathlineto{\pgfqpoint{1.960069in}{2.847436in}}%
\pgfpathlineto{\pgfqpoint{1.954344in}{2.855137in}}%
\pgfpathlineto{\pgfqpoint{1.942971in}{2.851805in}}%
\pgfpathlineto{\pgfqpoint{1.931592in}{2.848466in}}%
\pgfpathlineto{\pgfqpoint{1.920208in}{2.845102in}}%
\pgfpathlineto{\pgfqpoint{1.908820in}{2.841715in}}%
\pgfpathlineto{\pgfqpoint{1.897427in}{2.838311in}}%
\pgfpathlineto{\pgfqpoint{1.903138in}{2.830640in}}%
\pgfpathlineto{\pgfqpoint{1.908853in}{2.822955in}}%
\pgfpathlineto{\pgfqpoint{1.914573in}{2.815259in}}%
\pgfpathlineto{\pgfqpoint{1.920296in}{2.807550in}}%
\pgfpathclose%
\pgfusepath{stroke,fill}%
\end{pgfscope}%
\begin{pgfscope}%
\pgfpathrectangle{\pgfqpoint{0.887500in}{0.275000in}}{\pgfqpoint{4.225000in}{4.225000in}}%
\pgfusepath{clip}%
\pgfsetbuttcap%
\pgfsetroundjoin%
\definecolor{currentfill}{rgb}{0.762373,0.876424,0.137064}%
\pgfsetfillcolor{currentfill}%
\pgfsetfillopacity{0.700000}%
\pgfsetlinewidth{0.501875pt}%
\definecolor{currentstroke}{rgb}{1.000000,1.000000,1.000000}%
\pgfsetstrokecolor{currentstroke}%
\pgfsetstrokeopacity{0.500000}%
\pgfsetdash{}{0pt}%
\pgfpathmoveto{\pgfqpoint{3.877337in}{3.491986in}}%
\pgfpathlineto{\pgfqpoint{3.888298in}{3.496191in}}%
\pgfpathlineto{\pgfqpoint{3.899254in}{3.500385in}}%
\pgfpathlineto{\pgfqpoint{3.910206in}{3.504574in}}%
\pgfpathlineto{\pgfqpoint{3.921153in}{3.508766in}}%
\pgfpathlineto{\pgfqpoint{3.932095in}{3.512969in}}%
\pgfpathlineto{\pgfqpoint{3.925803in}{3.525229in}}%
\pgfpathlineto{\pgfqpoint{3.919512in}{3.537422in}}%
\pgfpathlineto{\pgfqpoint{3.913222in}{3.549529in}}%
\pgfpathlineto{\pgfqpoint{3.906932in}{3.561530in}}%
\pgfpathlineto{\pgfqpoint{3.900643in}{3.573407in}}%
\pgfpathlineto{\pgfqpoint{3.889704in}{3.569097in}}%
\pgfpathlineto{\pgfqpoint{3.878759in}{3.564775in}}%
\pgfpathlineto{\pgfqpoint{3.867810in}{3.560431in}}%
\pgfpathlineto{\pgfqpoint{3.856855in}{3.556053in}}%
\pgfpathlineto{\pgfqpoint{3.845895in}{3.551633in}}%
\pgfpathlineto{\pgfqpoint{3.852182in}{3.539924in}}%
\pgfpathlineto{\pgfqpoint{3.858469in}{3.528087in}}%
\pgfpathlineto{\pgfqpoint{3.864757in}{3.516139in}}%
\pgfpathlineto{\pgfqpoint{3.871046in}{3.504100in}}%
\pgfpathclose%
\pgfusepath{stroke,fill}%
\end{pgfscope}%
\begin{pgfscope}%
\pgfpathrectangle{\pgfqpoint{0.887500in}{0.275000in}}{\pgfqpoint{4.225000in}{4.225000in}}%
\pgfusepath{clip}%
\pgfsetbuttcap%
\pgfsetroundjoin%
\definecolor{currentfill}{rgb}{0.120081,0.622161,0.534946}%
\pgfsetfillcolor{currentfill}%
\pgfsetfillopacity{0.700000}%
\pgfsetlinewidth{0.501875pt}%
\definecolor{currentstroke}{rgb}{1.000000,1.000000,1.000000}%
\pgfsetstrokecolor{currentstroke}%
\pgfsetstrokeopacity{0.500000}%
\pgfsetdash{}{0pt}%
\pgfpathmoveto{\pgfqpoint{1.697563in}{2.825701in}}%
\pgfpathlineto{\pgfqpoint{1.709028in}{2.829014in}}%
\pgfpathlineto{\pgfqpoint{1.720488in}{2.832334in}}%
\pgfpathlineto{\pgfqpoint{1.731941in}{2.835662in}}%
\pgfpathlineto{\pgfqpoint{1.743389in}{2.839000in}}%
\pgfpathlineto{\pgfqpoint{1.754831in}{2.842349in}}%
\pgfpathlineto{\pgfqpoint{1.749172in}{2.849891in}}%
\pgfpathlineto{\pgfqpoint{1.743517in}{2.857419in}}%
\pgfpathlineto{\pgfqpoint{1.737866in}{2.864933in}}%
\pgfpathlineto{\pgfqpoint{1.732220in}{2.872433in}}%
\pgfpathlineto{\pgfqpoint{1.726577in}{2.879919in}}%
\pgfpathlineto{\pgfqpoint{1.715149in}{2.876552in}}%
\pgfpathlineto{\pgfqpoint{1.703715in}{2.873194in}}%
\pgfpathlineto{\pgfqpoint{1.692276in}{2.869842in}}%
\pgfpathlineto{\pgfqpoint{1.680830in}{2.866497in}}%
\pgfpathlineto{\pgfqpoint{1.669379in}{2.863156in}}%
\pgfpathlineto{\pgfqpoint{1.675008in}{2.855693in}}%
\pgfpathlineto{\pgfqpoint{1.680640in}{2.848216in}}%
\pgfpathlineto{\pgfqpoint{1.686277in}{2.840725in}}%
\pgfpathlineto{\pgfqpoint{1.691918in}{2.833220in}}%
\pgfpathclose%
\pgfusepath{stroke,fill}%
\end{pgfscope}%
\begin{pgfscope}%
\pgfpathrectangle{\pgfqpoint{0.887500in}{0.275000in}}{\pgfqpoint{4.225000in}{4.225000in}}%
\pgfusepath{clip}%
\pgfsetbuttcap%
\pgfsetroundjoin%
\definecolor{currentfill}{rgb}{0.855810,0.888601,0.097452}%
\pgfsetfillcolor{currentfill}%
\pgfsetfillopacity{0.700000}%
\pgfsetlinewidth{0.501875pt}%
\definecolor{currentstroke}{rgb}{1.000000,1.000000,1.000000}%
\pgfsetstrokecolor{currentstroke}%
\pgfsetstrokeopacity{0.500000}%
\pgfsetdash{}{0pt}%
\pgfpathmoveto{\pgfqpoint{3.563152in}{3.569840in}}%
\pgfpathlineto{\pgfqpoint{3.574179in}{3.573593in}}%
\pgfpathlineto{\pgfqpoint{3.585200in}{3.577258in}}%
\pgfpathlineto{\pgfqpoint{3.596216in}{3.580855in}}%
\pgfpathlineto{\pgfqpoint{3.607226in}{3.584437in}}%
\pgfpathlineto{\pgfqpoint{3.618232in}{3.588061in}}%
\pgfpathlineto{\pgfqpoint{3.611953in}{3.596462in}}%
\pgfpathlineto{\pgfqpoint{3.605676in}{3.604647in}}%
\pgfpathlineto{\pgfqpoint{3.599400in}{3.612652in}}%
\pgfpathlineto{\pgfqpoint{3.593126in}{3.620516in}}%
\pgfpathlineto{\pgfqpoint{3.586855in}{3.628276in}}%
\pgfpathlineto{\pgfqpoint{3.575847in}{3.623620in}}%
\pgfpathlineto{\pgfqpoint{3.564834in}{3.618996in}}%
\pgfpathlineto{\pgfqpoint{3.553817in}{3.614390in}}%
\pgfpathlineto{\pgfqpoint{3.542796in}{3.609789in}}%
\pgfpathlineto{\pgfqpoint{3.531771in}{3.605186in}}%
\pgfpathlineto{\pgfqpoint{3.538040in}{3.598320in}}%
\pgfpathlineto{\pgfqpoint{3.544314in}{3.591419in}}%
\pgfpathlineto{\pgfqpoint{3.550591in}{3.584416in}}%
\pgfpathlineto{\pgfqpoint{3.556871in}{3.577246in}}%
\pgfpathclose%
\pgfusepath{stroke,fill}%
\end{pgfscope}%
\begin{pgfscope}%
\pgfpathrectangle{\pgfqpoint{0.887500in}{0.275000in}}{\pgfqpoint{4.225000in}{4.225000in}}%
\pgfusepath{clip}%
\pgfsetbuttcap%
\pgfsetroundjoin%
\definecolor{currentfill}{rgb}{0.121380,0.629492,0.531973}%
\pgfsetfillcolor{currentfill}%
\pgfsetfillopacity{0.700000}%
\pgfsetlinewidth{0.501875pt}%
\definecolor{currentstroke}{rgb}{1.000000,1.000000,1.000000}%
\pgfsetstrokecolor{currentstroke}%
\pgfsetstrokeopacity{0.500000}%
\pgfsetdash{}{0pt}%
\pgfpathmoveto{\pgfqpoint{2.438315in}{2.816863in}}%
\pgfpathlineto{\pgfqpoint{2.449504in}{2.826392in}}%
\pgfpathlineto{\pgfqpoint{2.460685in}{2.836307in}}%
\pgfpathlineto{\pgfqpoint{2.471858in}{2.846677in}}%
\pgfpathlineto{\pgfqpoint{2.483024in}{2.857384in}}%
\pgfpathlineto{\pgfqpoint{2.494188in}{2.868190in}}%
\pgfpathlineto{\pgfqpoint{2.488266in}{2.876370in}}%
\pgfpathlineto{\pgfqpoint{2.482351in}{2.884242in}}%
\pgfpathlineto{\pgfqpoint{2.476443in}{2.891940in}}%
\pgfpathlineto{\pgfqpoint{2.470540in}{2.899603in}}%
\pgfpathlineto{\pgfqpoint{2.464639in}{2.907365in}}%
\pgfpathlineto{\pgfqpoint{2.453444in}{2.899780in}}%
\pgfpathlineto{\pgfqpoint{2.442250in}{2.891930in}}%
\pgfpathlineto{\pgfqpoint{2.431056in}{2.883860in}}%
\pgfpathlineto{\pgfqpoint{2.419863in}{2.875588in}}%
\pgfpathlineto{\pgfqpoint{2.408671in}{2.867098in}}%
\pgfpathlineto{\pgfqpoint{2.414588in}{2.857424in}}%
\pgfpathlineto{\pgfqpoint{2.420510in}{2.847588in}}%
\pgfpathlineto{\pgfqpoint{2.426438in}{2.837565in}}%
\pgfpathlineto{\pgfqpoint{2.432373in}{2.827332in}}%
\pgfpathclose%
\pgfusepath{stroke,fill}%
\end{pgfscope}%
\begin{pgfscope}%
\pgfpathrectangle{\pgfqpoint{0.887500in}{0.275000in}}{\pgfqpoint{4.225000in}{4.225000in}}%
\pgfusepath{clip}%
\pgfsetbuttcap%
\pgfsetroundjoin%
\definecolor{currentfill}{rgb}{0.709898,0.868751,0.169257}%
\pgfsetfillcolor{currentfill}%
\pgfsetfillopacity{0.700000}%
\pgfsetlinewidth{0.501875pt}%
\definecolor{currentstroke}{rgb}{1.000000,1.000000,1.000000}%
\pgfsetstrokecolor{currentstroke}%
\pgfsetstrokeopacity{0.500000}%
\pgfsetdash{}{0pt}%
\pgfpathmoveto{\pgfqpoint{3.963587in}{3.451061in}}%
\pgfpathlineto{\pgfqpoint{3.974531in}{3.455294in}}%
\pgfpathlineto{\pgfqpoint{3.985471in}{3.459600in}}%
\pgfpathlineto{\pgfqpoint{3.996409in}{3.463981in}}%
\pgfpathlineto{\pgfqpoint{4.007344in}{3.468441in}}%
\pgfpathlineto{\pgfqpoint{4.001036in}{3.480834in}}%
\pgfpathlineto{\pgfqpoint{3.994730in}{3.493183in}}%
\pgfpathlineto{\pgfqpoint{3.988426in}{3.505493in}}%
\pgfpathlineto{\pgfqpoint{3.982124in}{3.517772in}}%
\pgfpathlineto{\pgfqpoint{3.975825in}{3.530026in}}%
\pgfpathlineto{\pgfqpoint{3.964898in}{3.525710in}}%
\pgfpathlineto{\pgfqpoint{3.953968in}{3.521434in}}%
\pgfpathlineto{\pgfqpoint{3.943034in}{3.517189in}}%
\pgfpathlineto{\pgfqpoint{3.932095in}{3.512969in}}%
\pgfpathlineto{\pgfqpoint{3.938390in}{3.500655in}}%
\pgfpathlineto{\pgfqpoint{3.944686in}{3.488297in}}%
\pgfpathlineto{\pgfqpoint{3.950984in}{3.475905in}}%
\pgfpathlineto{\pgfqpoint{3.957284in}{3.463490in}}%
\pgfpathclose%
\pgfusepath{stroke,fill}%
\end{pgfscope}%
\begin{pgfscope}%
\pgfpathrectangle{\pgfqpoint{0.887500in}{0.275000in}}{\pgfqpoint{4.225000in}{4.225000in}}%
\pgfusepath{clip}%
\pgfsetbuttcap%
\pgfsetroundjoin%
\definecolor{currentfill}{rgb}{0.121831,0.589055,0.545623}%
\pgfsetfillcolor{currentfill}%
\pgfsetfillopacity{0.700000}%
\pgfsetlinewidth{0.501875pt}%
\definecolor{currentstroke}{rgb}{1.000000,1.000000,1.000000}%
\pgfsetstrokecolor{currentstroke}%
\pgfsetstrokeopacity{0.500000}%
\pgfsetdash{}{0pt}%
\pgfpathmoveto{\pgfqpoint{2.240334in}{2.747596in}}%
\pgfpathlineto{\pgfqpoint{2.251672in}{2.750817in}}%
\pgfpathlineto{\pgfqpoint{2.262990in}{2.754754in}}%
\pgfpathlineto{\pgfqpoint{2.274284in}{2.759659in}}%
\pgfpathlineto{\pgfqpoint{2.285550in}{2.765782in}}%
\pgfpathlineto{\pgfqpoint{2.296788in}{2.773170in}}%
\pgfpathlineto{\pgfqpoint{2.290933in}{2.781236in}}%
\pgfpathlineto{\pgfqpoint{2.285085in}{2.789177in}}%
\pgfpathlineto{\pgfqpoint{2.279243in}{2.796981in}}%
\pgfpathlineto{\pgfqpoint{2.273407in}{2.804663in}}%
\pgfpathlineto{\pgfqpoint{2.267577in}{2.812272in}}%
\pgfpathlineto{\pgfqpoint{2.256367in}{2.804260in}}%
\pgfpathlineto{\pgfqpoint{2.245122in}{2.797778in}}%
\pgfpathlineto{\pgfqpoint{2.233844in}{2.792773in}}%
\pgfpathlineto{\pgfqpoint{2.222537in}{2.788937in}}%
\pgfpathlineto{\pgfqpoint{2.211206in}{2.785960in}}%
\pgfpathlineto{\pgfqpoint{2.217022in}{2.778377in}}%
\pgfpathlineto{\pgfqpoint{2.222842in}{2.770770in}}%
\pgfpathlineto{\pgfqpoint{2.228668in}{2.763107in}}%
\pgfpathlineto{\pgfqpoint{2.234498in}{2.755380in}}%
\pgfpathclose%
\pgfusepath{stroke,fill}%
\end{pgfscope}%
\begin{pgfscope}%
\pgfpathrectangle{\pgfqpoint{0.887500in}{0.275000in}}{\pgfqpoint{4.225000in}{4.225000in}}%
\pgfusepath{clip}%
\pgfsetbuttcap%
\pgfsetroundjoin%
\definecolor{currentfill}{rgb}{0.146616,0.673050,0.508936}%
\pgfsetfillcolor{currentfill}%
\pgfsetfillopacity{0.700000}%
\pgfsetlinewidth{0.501875pt}%
\definecolor{currentstroke}{rgb}{1.000000,1.000000,1.000000}%
\pgfsetstrokecolor{currentstroke}%
\pgfsetstrokeopacity{0.500000}%
\pgfsetdash{}{0pt}%
\pgfpathmoveto{\pgfqpoint{2.635786in}{2.917751in}}%
\pgfpathlineto{\pgfqpoint{2.646977in}{2.925749in}}%
\pgfpathlineto{\pgfqpoint{2.658160in}{2.934395in}}%
\pgfpathlineto{\pgfqpoint{2.669337in}{2.943344in}}%
\pgfpathlineto{\pgfqpoint{2.680514in}{2.952083in}}%
\pgfpathlineto{\pgfqpoint{2.691695in}{2.960098in}}%
\pgfpathlineto{\pgfqpoint{2.685726in}{2.966294in}}%
\pgfpathlineto{\pgfqpoint{2.679761in}{2.972562in}}%
\pgfpathlineto{\pgfqpoint{2.673799in}{2.978960in}}%
\pgfpathlineto{\pgfqpoint{2.667840in}{2.985485in}}%
\pgfpathlineto{\pgfqpoint{2.661884in}{2.992115in}}%
\pgfpathlineto{\pgfqpoint{2.650738in}{2.982315in}}%
\pgfpathlineto{\pgfqpoint{2.639589in}{2.972571in}}%
\pgfpathlineto{\pgfqpoint{2.628434in}{2.963261in}}%
\pgfpathlineto{\pgfqpoint{2.617269in}{2.954763in}}%
\pgfpathlineto{\pgfqpoint{2.606089in}{2.947309in}}%
\pgfpathlineto{\pgfqpoint{2.612007in}{2.942544in}}%
\pgfpathlineto{\pgfqpoint{2.617934in}{2.937328in}}%
\pgfpathlineto{\pgfqpoint{2.623873in}{2.931507in}}%
\pgfpathlineto{\pgfqpoint{2.629823in}{2.924970in}}%
\pgfpathclose%
\pgfusepath{stroke,fill}%
\end{pgfscope}%
\begin{pgfscope}%
\pgfpathrectangle{\pgfqpoint{0.887500in}{0.275000in}}{\pgfqpoint{4.225000in}{4.225000in}}%
\pgfusepath{clip}%
\pgfsetbuttcap%
\pgfsetroundjoin%
\definecolor{currentfill}{rgb}{0.709898,0.868751,0.169257}%
\pgfsetfillcolor{currentfill}%
\pgfsetfillopacity{0.700000}%
\pgfsetlinewidth{0.501875pt}%
\definecolor{currentstroke}{rgb}{1.000000,1.000000,1.000000}%
\pgfsetstrokecolor{currentstroke}%
\pgfsetstrokeopacity{0.500000}%
\pgfsetdash{}{0pt}%
\pgfpathmoveto{\pgfqpoint{2.853088in}{3.477697in}}%
\pgfpathlineto{\pgfqpoint{2.864244in}{3.489132in}}%
\pgfpathlineto{\pgfqpoint{2.875409in}{3.498187in}}%
\pgfpathlineto{\pgfqpoint{2.886582in}{3.505187in}}%
\pgfpathlineto{\pgfqpoint{2.897758in}{3.510459in}}%
\pgfpathlineto{\pgfqpoint{2.908935in}{3.514328in}}%
\pgfpathlineto{\pgfqpoint{2.902906in}{3.513807in}}%
\pgfpathlineto{\pgfqpoint{2.896888in}{3.512257in}}%
\pgfpathlineto{\pgfqpoint{2.890881in}{3.509555in}}%
\pgfpathlineto{\pgfqpoint{2.884886in}{3.505727in}}%
\pgfpathlineto{\pgfqpoint{2.878903in}{3.500978in}}%
\pgfpathlineto{\pgfqpoint{2.867767in}{3.493881in}}%
\pgfpathlineto{\pgfqpoint{2.856632in}{3.485929in}}%
\pgfpathlineto{\pgfqpoint{2.845500in}{3.477043in}}%
\pgfpathlineto{\pgfqpoint{2.834371in}{3.467153in}}%
\pgfpathlineto{\pgfqpoint{2.823246in}{3.456192in}}%
\pgfpathlineto{\pgfqpoint{2.829182in}{3.463700in}}%
\pgfpathlineto{\pgfqpoint{2.835133in}{3.469782in}}%
\pgfpathlineto{\pgfqpoint{2.841102in}{3.474128in}}%
\pgfpathlineto{\pgfqpoint{2.847087in}{3.476702in}}%
\pgfpathclose%
\pgfusepath{stroke,fill}%
\end{pgfscope}%
\begin{pgfscope}%
\pgfpathrectangle{\pgfqpoint{0.887500in}{0.275000in}}{\pgfqpoint{4.225000in}{4.225000in}}%
\pgfusepath{clip}%
\pgfsetbuttcap%
\pgfsetroundjoin%
\definecolor{currentfill}{rgb}{0.404001,0.800275,0.362552}%
\pgfsetfillcolor{currentfill}%
\pgfsetfillopacity{0.700000}%
\pgfsetlinewidth{0.501875pt}%
\definecolor{currentstroke}{rgb}{1.000000,1.000000,1.000000}%
\pgfsetstrokecolor{currentstroke}%
\pgfsetstrokeopacity{0.500000}%
\pgfsetdash{}{0pt}%
\pgfpathmoveto{\pgfqpoint{2.773093in}{3.112566in}}%
\pgfpathlineto{\pgfqpoint{2.784059in}{3.147077in}}%
\pgfpathlineto{\pgfqpoint{2.795001in}{3.187042in}}%
\pgfpathlineto{\pgfqpoint{2.805938in}{3.230362in}}%
\pgfpathlineto{\pgfqpoint{2.816885in}{3.274924in}}%
\pgfpathlineto{\pgfqpoint{2.827854in}{3.318601in}}%
\pgfpathlineto{\pgfqpoint{2.821789in}{3.330924in}}%
\pgfpathlineto{\pgfqpoint{2.815727in}{3.343146in}}%
\pgfpathlineto{\pgfqpoint{2.809671in}{3.354798in}}%
\pgfpathlineto{\pgfqpoint{2.803624in}{3.365409in}}%
\pgfpathlineto{\pgfqpoint{2.797590in}{3.374509in}}%
\pgfpathlineto{\pgfqpoint{2.786563in}{3.345007in}}%
\pgfpathlineto{\pgfqpoint{2.775554in}{3.314196in}}%
\pgfpathlineto{\pgfqpoint{2.764558in}{3.282997in}}%
\pgfpathlineto{\pgfqpoint{2.753568in}{3.252326in}}%
\pgfpathlineto{\pgfqpoint{2.742577in}{3.223096in}}%
\pgfpathlineto{\pgfqpoint{2.748669in}{3.202593in}}%
\pgfpathlineto{\pgfqpoint{2.754772in}{3.180547in}}%
\pgfpathlineto{\pgfqpoint{2.760880in}{3.157700in}}%
\pgfpathlineto{\pgfqpoint{2.766989in}{3.134793in}}%
\pgfpathclose%
\pgfusepath{stroke,fill}%
\end{pgfscope}%
\begin{pgfscope}%
\pgfpathrectangle{\pgfqpoint{0.887500in}{0.275000in}}{\pgfqpoint{4.225000in}{4.225000in}}%
\pgfusepath{clip}%
\pgfsetbuttcap%
\pgfsetroundjoin%
\definecolor{currentfill}{rgb}{0.626579,0.854645,0.223353}%
\pgfsetfillcolor{currentfill}%
\pgfsetfillopacity{0.700000}%
\pgfsetlinewidth{0.501875pt}%
\definecolor{currentstroke}{rgb}{1.000000,1.000000,1.000000}%
\pgfsetstrokecolor{currentstroke}%
\pgfsetstrokeopacity{0.500000}%
\pgfsetdash{}{0pt}%
\pgfpathmoveto{\pgfqpoint{2.797590in}{3.374509in}}%
\pgfpathlineto{\pgfqpoint{2.808641in}{3.401776in}}%
\pgfpathlineto{\pgfqpoint{2.819717in}{3.425888in}}%
\pgfpathlineto{\pgfqpoint{2.830820in}{3.446396in}}%
\pgfpathlineto{\pgfqpoint{2.841945in}{3.463559in}}%
\pgfpathlineto{\pgfqpoint{2.853088in}{3.477697in}}%
\pgfpathlineto{\pgfqpoint{2.847087in}{3.476702in}}%
\pgfpathlineto{\pgfqpoint{2.841102in}{3.474128in}}%
\pgfpathlineto{\pgfqpoint{2.835133in}{3.469782in}}%
\pgfpathlineto{\pgfqpoint{2.829182in}{3.463700in}}%
\pgfpathlineto{\pgfqpoint{2.823246in}{3.456192in}}%
\pgfpathlineto{\pgfqpoint{2.812128in}{3.444093in}}%
\pgfpathlineto{\pgfqpoint{2.801016in}{3.430788in}}%
\pgfpathlineto{\pgfqpoint{2.789913in}{3.416211in}}%
\pgfpathlineto{\pgfqpoint{2.778821in}{3.400300in}}%
\pgfpathlineto{\pgfqpoint{2.767739in}{3.383070in}}%
\pgfpathlineto{\pgfqpoint{2.773659in}{3.386843in}}%
\pgfpathlineto{\pgfqpoint{2.779605in}{3.388055in}}%
\pgfpathlineto{\pgfqpoint{2.785577in}{3.386294in}}%
\pgfpathlineto{\pgfqpoint{2.791573in}{3.381627in}}%
\pgfpathclose%
\pgfusepath{stroke,fill}%
\end{pgfscope}%
\begin{pgfscope}%
\pgfpathrectangle{\pgfqpoint{0.887500in}{0.275000in}}{\pgfqpoint{4.225000in}{4.225000in}}%
\pgfusepath{clip}%
\pgfsetbuttcap%
\pgfsetroundjoin%
\definecolor{currentfill}{rgb}{0.119738,0.603785,0.541400}%
\pgfsetfillcolor{currentfill}%
\pgfsetfillopacity{0.700000}%
\pgfsetlinewidth{0.501875pt}%
\definecolor{currentstroke}{rgb}{1.000000,1.000000,1.000000}%
\pgfsetstrokecolor{currentstroke}%
\pgfsetstrokeopacity{0.500000}%
\pgfsetdash{}{0pt}%
\pgfpathmoveto{\pgfqpoint{2.011776in}{2.777577in}}%
\pgfpathlineto{\pgfqpoint{2.023170in}{2.780866in}}%
\pgfpathlineto{\pgfqpoint{2.034558in}{2.784199in}}%
\pgfpathlineto{\pgfqpoint{2.045939in}{2.787601in}}%
\pgfpathlineto{\pgfqpoint{2.057312in}{2.791099in}}%
\pgfpathlineto{\pgfqpoint{2.068676in}{2.794718in}}%
\pgfpathlineto{\pgfqpoint{2.062901in}{2.802556in}}%
\pgfpathlineto{\pgfqpoint{2.057131in}{2.810381in}}%
\pgfpathlineto{\pgfqpoint{2.051364in}{2.818194in}}%
\pgfpathlineto{\pgfqpoint{2.045602in}{2.825993in}}%
\pgfpathlineto{\pgfqpoint{2.039843in}{2.833780in}}%
\pgfpathlineto{\pgfqpoint{2.028492in}{2.830140in}}%
\pgfpathlineto{\pgfqpoint{2.017133in}{2.826618in}}%
\pgfpathlineto{\pgfqpoint{2.005765in}{2.823188in}}%
\pgfpathlineto{\pgfqpoint{1.994391in}{2.819827in}}%
\pgfpathlineto{\pgfqpoint{1.983010in}{2.816508in}}%
\pgfpathlineto{\pgfqpoint{1.988755in}{2.808746in}}%
\pgfpathlineto{\pgfqpoint{1.994504in}{2.800972in}}%
\pgfpathlineto{\pgfqpoint{2.000257in}{2.793186in}}%
\pgfpathlineto{\pgfqpoint{2.006014in}{2.785388in}}%
\pgfpathclose%
\pgfusepath{stroke,fill}%
\end{pgfscope}%
\begin{pgfscope}%
\pgfpathrectangle{\pgfqpoint{0.887500in}{0.275000in}}{\pgfqpoint{4.225000in}{4.225000in}}%
\pgfusepath{clip}%
\pgfsetbuttcap%
\pgfsetroundjoin%
\definecolor{currentfill}{rgb}{0.835270,0.886029,0.102646}%
\pgfsetfillcolor{currentfill}%
\pgfsetfillopacity{0.700000}%
\pgfsetlinewidth{0.501875pt}%
\definecolor{currentstroke}{rgb}{1.000000,1.000000,1.000000}%
\pgfsetstrokecolor{currentstroke}%
\pgfsetstrokeopacity{0.500000}%
\pgfsetdash{}{0pt}%
\pgfpathmoveto{\pgfqpoint{3.421270in}{3.559167in}}%
\pgfpathlineto{\pgfqpoint{3.432337in}{3.563429in}}%
\pgfpathlineto{\pgfqpoint{3.443401in}{3.567906in}}%
\pgfpathlineto{\pgfqpoint{3.454463in}{3.572531in}}%
\pgfpathlineto{\pgfqpoint{3.465520in}{3.577236in}}%
\pgfpathlineto{\pgfqpoint{3.476574in}{3.581954in}}%
\pgfpathlineto{\pgfqpoint{3.470315in}{3.588222in}}%
\pgfpathlineto{\pgfqpoint{3.464063in}{3.594643in}}%
\pgfpathlineto{\pgfqpoint{3.457816in}{3.601243in}}%
\pgfpathlineto{\pgfqpoint{3.451576in}{3.607974in}}%
\pgfpathlineto{\pgfqpoint{3.445341in}{3.614780in}}%
\pgfpathlineto{\pgfqpoint{3.434296in}{3.609552in}}%
\pgfpathlineto{\pgfqpoint{3.423248in}{3.604337in}}%
\pgfpathlineto{\pgfqpoint{3.412195in}{3.599205in}}%
\pgfpathlineto{\pgfqpoint{3.401140in}{3.594223in}}%
\pgfpathlineto{\pgfqpoint{3.390083in}{3.589461in}}%
\pgfpathlineto{\pgfqpoint{3.396309in}{3.583126in}}%
\pgfpathlineto{\pgfqpoint{3.402540in}{3.576884in}}%
\pgfpathlineto{\pgfqpoint{3.408777in}{3.570790in}}%
\pgfpathlineto{\pgfqpoint{3.415020in}{3.564894in}}%
\pgfpathclose%
\pgfusepath{stroke,fill}%
\end{pgfscope}%
\begin{pgfscope}%
\pgfpathrectangle{\pgfqpoint{0.887500in}{0.275000in}}{\pgfqpoint{4.225000in}{4.225000in}}%
\pgfusepath{clip}%
\pgfsetbuttcap%
\pgfsetroundjoin%
\definecolor{currentfill}{rgb}{0.166383,0.690856,0.496502}%
\pgfsetfillcolor{currentfill}%
\pgfsetfillopacity{0.700000}%
\pgfsetlinewidth{0.501875pt}%
\definecolor{currentstroke}{rgb}{1.000000,1.000000,1.000000}%
\pgfsetstrokecolor{currentstroke}%
\pgfsetstrokeopacity{0.500000}%
\pgfsetdash{}{0pt}%
\pgfpathmoveto{\pgfqpoint{2.777659in}{2.955053in}}%
\pgfpathlineto{\pgfqpoint{2.788866in}{2.958813in}}%
\pgfpathlineto{\pgfqpoint{2.800048in}{2.965707in}}%
\pgfpathlineto{\pgfqpoint{2.811195in}{2.977599in}}%
\pgfpathlineto{\pgfqpoint{2.822302in}{2.996355in}}%
\pgfpathlineto{\pgfqpoint{2.833364in}{3.023849in}}%
\pgfpathlineto{\pgfqpoint{2.827377in}{3.022841in}}%
\pgfpathlineto{\pgfqpoint{2.821389in}{3.023158in}}%
\pgfpathlineto{\pgfqpoint{2.815396in}{3.025164in}}%
\pgfpathlineto{\pgfqpoint{2.809397in}{3.029220in}}%
\pgfpathlineto{\pgfqpoint{2.803388in}{3.035691in}}%
\pgfpathlineto{\pgfqpoint{2.792360in}{3.008723in}}%
\pgfpathlineto{\pgfqpoint{2.781275in}{2.991184in}}%
\pgfpathlineto{\pgfqpoint{2.770140in}{2.981063in}}%
\pgfpathlineto{\pgfqpoint{2.758963in}{2.976353in}}%
\pgfpathlineto{\pgfqpoint{2.747755in}{2.975051in}}%
\pgfpathlineto{\pgfqpoint{2.753742in}{2.968923in}}%
\pgfpathlineto{\pgfqpoint{2.759723in}{2.964104in}}%
\pgfpathlineto{\pgfqpoint{2.765702in}{2.960350in}}%
\pgfpathlineto{\pgfqpoint{2.771680in}{2.957415in}}%
\pgfpathclose%
\pgfusepath{stroke,fill}%
\end{pgfscope}%
\begin{pgfscope}%
\pgfpathrectangle{\pgfqpoint{0.887500in}{0.275000in}}{\pgfqpoint{4.225000in}{4.225000in}}%
\pgfusepath{clip}%
\pgfsetbuttcap%
\pgfsetroundjoin%
\definecolor{currentfill}{rgb}{0.119699,0.618490,0.536347}%
\pgfsetfillcolor{currentfill}%
\pgfsetfillopacity{0.700000}%
\pgfsetlinewidth{0.501875pt}%
\definecolor{currentstroke}{rgb}{1.000000,1.000000,1.000000}%
\pgfsetstrokecolor{currentstroke}%
\pgfsetstrokeopacity{0.500000}%
\pgfsetdash{}{0pt}%
\pgfpathmoveto{\pgfqpoint{1.783189in}{2.804432in}}%
\pgfpathlineto{\pgfqpoint{1.794639in}{2.807771in}}%
\pgfpathlineto{\pgfqpoint{1.806083in}{2.811123in}}%
\pgfpathlineto{\pgfqpoint{1.817521in}{2.814486in}}%
\pgfpathlineto{\pgfqpoint{1.828953in}{2.817861in}}%
\pgfpathlineto{\pgfqpoint{1.840380in}{2.821248in}}%
\pgfpathlineto{\pgfqpoint{1.834686in}{2.828882in}}%
\pgfpathlineto{\pgfqpoint{1.828997in}{2.836502in}}%
\pgfpathlineto{\pgfqpoint{1.823311in}{2.844107in}}%
\pgfpathlineto{\pgfqpoint{1.817630in}{2.851699in}}%
\pgfpathlineto{\pgfqpoint{1.811953in}{2.859275in}}%
\pgfpathlineto{\pgfqpoint{1.800540in}{2.855865in}}%
\pgfpathlineto{\pgfqpoint{1.789122in}{2.852467in}}%
\pgfpathlineto{\pgfqpoint{1.777698in}{2.849083in}}%
\pgfpathlineto{\pgfqpoint{1.766267in}{2.845710in}}%
\pgfpathlineto{\pgfqpoint{1.754831in}{2.842349in}}%
\pgfpathlineto{\pgfqpoint{1.760494in}{2.834794in}}%
\pgfpathlineto{\pgfqpoint{1.766162in}{2.827224in}}%
\pgfpathlineto{\pgfqpoint{1.771833in}{2.819641in}}%
\pgfpathlineto{\pgfqpoint{1.777509in}{2.812043in}}%
\pgfpathclose%
\pgfusepath{stroke,fill}%
\end{pgfscope}%
\begin{pgfscope}%
\pgfpathrectangle{\pgfqpoint{0.887500in}{0.275000in}}{\pgfqpoint{4.225000in}{4.225000in}}%
\pgfusepath{clip}%
\pgfsetbuttcap%
\pgfsetroundjoin%
\definecolor{currentfill}{rgb}{0.824940,0.884720,0.106217}%
\pgfsetfillcolor{currentfill}%
\pgfsetfillopacity{0.700000}%
\pgfsetlinewidth{0.501875pt}%
\definecolor{currentstroke}{rgb}{1.000000,1.000000,1.000000}%
\pgfsetstrokecolor{currentstroke}%
\pgfsetstrokeopacity{0.500000}%
\pgfsetdash{}{0pt}%
\pgfpathmoveto{\pgfqpoint{3.649619in}{3.541528in}}%
\pgfpathlineto{\pgfqpoint{3.660619in}{3.544509in}}%
\pgfpathlineto{\pgfqpoint{3.671617in}{3.547710in}}%
\pgfpathlineto{\pgfqpoint{3.682615in}{3.551215in}}%
\pgfpathlineto{\pgfqpoint{3.693613in}{3.555105in}}%
\pgfpathlineto{\pgfqpoint{3.704613in}{3.559373in}}%
\pgfpathlineto{\pgfqpoint{3.698336in}{3.569733in}}%
\pgfpathlineto{\pgfqpoint{3.692058in}{3.579836in}}%
\pgfpathlineto{\pgfqpoint{3.685780in}{3.589682in}}%
\pgfpathlineto{\pgfqpoint{3.679502in}{3.599283in}}%
\pgfpathlineto{\pgfqpoint{3.673224in}{3.608653in}}%
\pgfpathlineto{\pgfqpoint{3.662228in}{3.604065in}}%
\pgfpathlineto{\pgfqpoint{3.651231in}{3.599729in}}%
\pgfpathlineto{\pgfqpoint{3.640233in}{3.595652in}}%
\pgfpathlineto{\pgfqpoint{3.629234in}{3.591781in}}%
\pgfpathlineto{\pgfqpoint{3.618232in}{3.588061in}}%
\pgfpathlineto{\pgfqpoint{3.624511in}{3.579406in}}%
\pgfpathlineto{\pgfqpoint{3.630790in}{3.570461in}}%
\pgfpathlineto{\pgfqpoint{3.637068in}{3.561187in}}%
\pgfpathlineto{\pgfqpoint{3.643345in}{3.551547in}}%
\pgfpathclose%
\pgfusepath{stroke,fill}%
\end{pgfscope}%
\begin{pgfscope}%
\pgfpathrectangle{\pgfqpoint{0.887500in}{0.275000in}}{\pgfqpoint{4.225000in}{4.225000in}}%
\pgfusepath{clip}%
\pgfsetbuttcap%
\pgfsetroundjoin%
\definecolor{currentfill}{rgb}{0.119423,0.611141,0.538982}%
\pgfsetfillcolor{currentfill}%
\pgfsetfillopacity{0.700000}%
\pgfsetlinewidth{0.501875pt}%
\definecolor{currentstroke}{rgb}{1.000000,1.000000,1.000000}%
\pgfsetstrokecolor{currentstroke}%
\pgfsetstrokeopacity{0.500000}%
\pgfsetdash{}{0pt}%
\pgfpathmoveto{\pgfqpoint{2.382293in}{2.772169in}}%
\pgfpathlineto{\pgfqpoint{2.393504in}{2.780985in}}%
\pgfpathlineto{\pgfqpoint{2.404713in}{2.789783in}}%
\pgfpathlineto{\pgfqpoint{2.415919in}{2.798641in}}%
\pgfpathlineto{\pgfqpoint{2.427120in}{2.807641in}}%
\pgfpathlineto{\pgfqpoint{2.438315in}{2.816863in}}%
\pgfpathlineto{\pgfqpoint{2.432373in}{2.827332in}}%
\pgfpathlineto{\pgfqpoint{2.426438in}{2.837565in}}%
\pgfpathlineto{\pgfqpoint{2.420510in}{2.847588in}}%
\pgfpathlineto{\pgfqpoint{2.414588in}{2.857424in}}%
\pgfpathlineto{\pgfqpoint{2.408671in}{2.867098in}}%
\pgfpathlineto{\pgfqpoint{2.397481in}{2.858370in}}%
\pgfpathlineto{\pgfqpoint{2.386292in}{2.849383in}}%
\pgfpathlineto{\pgfqpoint{2.375107in}{2.840117in}}%
\pgfpathlineto{\pgfqpoint{2.363924in}{2.830553in}}%
\pgfpathlineto{\pgfqpoint{2.352745in}{2.820676in}}%
\pgfpathlineto{\pgfqpoint{2.358642in}{2.811312in}}%
\pgfpathlineto{\pgfqpoint{2.364546in}{2.801771in}}%
\pgfpathlineto{\pgfqpoint{2.370455in}{2.792062in}}%
\pgfpathlineto{\pgfqpoint{2.376371in}{2.782192in}}%
\pgfpathclose%
\pgfusepath{stroke,fill}%
\end{pgfscope}%
\begin{pgfscope}%
\pgfpathrectangle{\pgfqpoint{0.887500in}{0.275000in}}{\pgfqpoint{4.225000in}{4.225000in}}%
\pgfusepath{clip}%
\pgfsetbuttcap%
\pgfsetroundjoin%
\definecolor{currentfill}{rgb}{0.762373,0.876424,0.137064}%
\pgfsetfillcolor{currentfill}%
\pgfsetfillopacity{0.700000}%
\pgfsetlinewidth{0.501875pt}%
\definecolor{currentstroke}{rgb}{1.000000,1.000000,1.000000}%
\pgfsetstrokecolor{currentstroke}%
\pgfsetstrokeopacity{0.500000}%
\pgfsetdash{}{0pt}%
\pgfpathmoveto{\pgfqpoint{2.995137in}{3.512429in}}%
\pgfpathlineto{\pgfqpoint{3.006302in}{3.515155in}}%
\pgfpathlineto{\pgfqpoint{3.017460in}{3.519018in}}%
\pgfpathlineto{\pgfqpoint{3.028613in}{3.523759in}}%
\pgfpathlineto{\pgfqpoint{3.039762in}{3.529110in}}%
\pgfpathlineto{\pgfqpoint{3.050906in}{3.534804in}}%
\pgfpathlineto{\pgfqpoint{3.044801in}{3.536480in}}%
\pgfpathlineto{\pgfqpoint{3.038702in}{3.537963in}}%
\pgfpathlineto{\pgfqpoint{3.032609in}{3.539270in}}%
\pgfpathlineto{\pgfqpoint{3.026523in}{3.540419in}}%
\pgfpathlineto{\pgfqpoint{3.020442in}{3.541429in}}%
\pgfpathlineto{\pgfqpoint{3.009318in}{3.537124in}}%
\pgfpathlineto{\pgfqpoint{2.998189in}{3.533147in}}%
\pgfpathlineto{\pgfqpoint{2.987055in}{3.529602in}}%
\pgfpathlineto{\pgfqpoint{2.975914in}{3.526591in}}%
\pgfpathlineto{\pgfqpoint{2.964766in}{3.524215in}}%
\pgfpathlineto{\pgfqpoint{2.970828in}{3.522346in}}%
\pgfpathlineto{\pgfqpoint{2.976896in}{3.520164in}}%
\pgfpathlineto{\pgfqpoint{2.982971in}{3.517738in}}%
\pgfpathlineto{\pgfqpoint{2.989051in}{3.515136in}}%
\pgfpathclose%
\pgfusepath{stroke,fill}%
\end{pgfscope}%
\begin{pgfscope}%
\pgfpathrectangle{\pgfqpoint{0.887500in}{0.275000in}}{\pgfqpoint{4.225000in}{4.225000in}}%
\pgfusepath{clip}%
\pgfsetbuttcap%
\pgfsetroundjoin%
\definecolor{currentfill}{rgb}{0.835270,0.886029,0.102646}%
\pgfsetfillcolor{currentfill}%
\pgfsetfillopacity{0.700000}%
\pgfsetlinewidth{0.501875pt}%
\definecolor{currentstroke}{rgb}{1.000000,1.000000,1.000000}%
\pgfsetstrokecolor{currentstroke}%
\pgfsetstrokeopacity{0.500000}%
\pgfsetdash{}{0pt}%
\pgfpathmoveto{\pgfqpoint{3.279347in}{3.556621in}}%
\pgfpathlineto{\pgfqpoint{3.290442in}{3.559570in}}%
\pgfpathlineto{\pgfqpoint{3.301531in}{3.562382in}}%
\pgfpathlineto{\pgfqpoint{3.312614in}{3.565127in}}%
\pgfpathlineto{\pgfqpoint{3.323692in}{3.567890in}}%
\pgfpathlineto{\pgfqpoint{3.334764in}{3.570759in}}%
\pgfpathlineto{\pgfqpoint{3.328551in}{3.576314in}}%
\pgfpathlineto{\pgfqpoint{3.322342in}{3.581802in}}%
\pgfpathlineto{\pgfqpoint{3.316138in}{3.587197in}}%
\pgfpathlineto{\pgfqpoint{3.309937in}{3.592477in}}%
\pgfpathlineto{\pgfqpoint{3.303740in}{3.597616in}}%
\pgfpathlineto{\pgfqpoint{3.292675in}{3.593725in}}%
\pgfpathlineto{\pgfqpoint{3.281606in}{3.590006in}}%
\pgfpathlineto{\pgfqpoint{3.270533in}{3.586425in}}%
\pgfpathlineto{\pgfqpoint{3.259456in}{3.582951in}}%
\pgfpathlineto{\pgfqpoint{3.248373in}{3.579556in}}%
\pgfpathlineto{\pgfqpoint{3.254560in}{3.575278in}}%
\pgfpathlineto{\pgfqpoint{3.260750in}{3.570851in}}%
\pgfpathlineto{\pgfqpoint{3.266945in}{3.566270in}}%
\pgfpathlineto{\pgfqpoint{3.273144in}{3.561529in}}%
\pgfpathclose%
\pgfusepath{stroke,fill}%
\end{pgfscope}%
\begin{pgfscope}%
\pgfpathrectangle{\pgfqpoint{0.887500in}{0.275000in}}{\pgfqpoint{4.225000in}{4.225000in}}%
\pgfusepath{clip}%
\pgfsetbuttcap%
\pgfsetroundjoin%
\definecolor{currentfill}{rgb}{0.134692,0.658636,0.517649}%
\pgfsetfillcolor{currentfill}%
\pgfsetfillopacity{0.700000}%
\pgfsetlinewidth{0.501875pt}%
\definecolor{currentstroke}{rgb}{1.000000,1.000000,1.000000}%
\pgfsetstrokecolor{currentstroke}%
\pgfsetstrokeopacity{0.500000}%
\pgfsetdash{}{0pt}%
\pgfpathmoveto{\pgfqpoint{2.579782in}{2.878088in}}%
\pgfpathlineto{\pgfqpoint{2.590976in}{2.887031in}}%
\pgfpathlineto{\pgfqpoint{2.602178in}{2.895153in}}%
\pgfpathlineto{\pgfqpoint{2.613383in}{2.902770in}}%
\pgfpathlineto{\pgfqpoint{2.624586in}{2.910197in}}%
\pgfpathlineto{\pgfqpoint{2.635786in}{2.917751in}}%
\pgfpathlineto{\pgfqpoint{2.629823in}{2.924970in}}%
\pgfpathlineto{\pgfqpoint{2.623873in}{2.931507in}}%
\pgfpathlineto{\pgfqpoint{2.617934in}{2.937328in}}%
\pgfpathlineto{\pgfqpoint{2.612007in}{2.942544in}}%
\pgfpathlineto{\pgfqpoint{2.606089in}{2.947309in}}%
\pgfpathlineto{\pgfqpoint{2.594897in}{2.940657in}}%
\pgfpathlineto{\pgfqpoint{2.583696in}{2.934468in}}%
\pgfpathlineto{\pgfqpoint{2.572489in}{2.928403in}}%
\pgfpathlineto{\pgfqpoint{2.561282in}{2.922124in}}%
\pgfpathlineto{\pgfqpoint{2.550078in}{2.915292in}}%
\pgfpathlineto{\pgfqpoint{2.555987in}{2.909720in}}%
\pgfpathlineto{\pgfqpoint{2.561911in}{2.903383in}}%
\pgfpathlineto{\pgfqpoint{2.567851in}{2.896050in}}%
\pgfpathlineto{\pgfqpoint{2.573808in}{2.887578in}}%
\pgfpathclose%
\pgfusepath{stroke,fill}%
\end{pgfscope}%
\begin{pgfscope}%
\pgfpathrectangle{\pgfqpoint{0.887500in}{0.275000in}}{\pgfqpoint{4.225000in}{4.225000in}}%
\pgfusepath{clip}%
\pgfsetbuttcap%
\pgfsetroundjoin%
\definecolor{currentfill}{rgb}{0.793760,0.880678,0.120005}%
\pgfsetfillcolor{currentfill}%
\pgfsetfillopacity{0.700000}%
\pgfsetlinewidth{0.501875pt}%
\definecolor{currentstroke}{rgb}{1.000000,1.000000,1.000000}%
\pgfsetstrokecolor{currentstroke}%
\pgfsetstrokeopacity{0.500000}%
\pgfsetdash{}{0pt}%
\pgfpathmoveto{\pgfqpoint{3.736004in}{3.504866in}}%
\pgfpathlineto{\pgfqpoint{3.747009in}{3.509312in}}%
\pgfpathlineto{\pgfqpoint{3.758013in}{3.513975in}}%
\pgfpathlineto{\pgfqpoint{3.769015in}{3.518781in}}%
\pgfpathlineto{\pgfqpoint{3.780015in}{3.523655in}}%
\pgfpathlineto{\pgfqpoint{3.791009in}{3.528522in}}%
\pgfpathlineto{\pgfqpoint{3.784726in}{3.539880in}}%
\pgfpathlineto{\pgfqpoint{3.778442in}{3.551103in}}%
\pgfpathlineto{\pgfqpoint{3.772160in}{3.562168in}}%
\pgfpathlineto{\pgfqpoint{3.765877in}{3.573053in}}%
\pgfpathlineto{\pgfqpoint{3.759594in}{3.583733in}}%
\pgfpathlineto{\pgfqpoint{3.748604in}{3.578714in}}%
\pgfpathlineto{\pgfqpoint{3.737609in}{3.573692in}}%
\pgfpathlineto{\pgfqpoint{3.726612in}{3.568744in}}%
\pgfpathlineto{\pgfqpoint{3.715613in}{3.563945in}}%
\pgfpathlineto{\pgfqpoint{3.704613in}{3.559373in}}%
\pgfpathlineto{\pgfqpoint{3.710890in}{3.548786in}}%
\pgfpathlineto{\pgfqpoint{3.717167in}{3.538007in}}%
\pgfpathlineto{\pgfqpoint{3.723444in}{3.527070in}}%
\pgfpathlineto{\pgfqpoint{3.729723in}{3.516012in}}%
\pgfpathclose%
\pgfusepath{stroke,fill}%
\end{pgfscope}%
\begin{pgfscope}%
\pgfpathrectangle{\pgfqpoint{0.887500in}{0.275000in}}{\pgfqpoint{4.225000in}{4.225000in}}%
\pgfusepath{clip}%
\pgfsetbuttcap%
\pgfsetroundjoin%
\definecolor{currentfill}{rgb}{0.120565,0.596422,0.543611}%
\pgfsetfillcolor{currentfill}%
\pgfsetfillopacity{0.700000}%
\pgfsetlinewidth{0.501875pt}%
\definecolor{currentstroke}{rgb}{1.000000,1.000000,1.000000}%
\pgfsetstrokecolor{currentstroke}%
\pgfsetstrokeopacity{0.500000}%
\pgfsetdash{}{0pt}%
\pgfpathmoveto{\pgfqpoint{2.097610in}{2.755334in}}%
\pgfpathlineto{\pgfqpoint{2.108979in}{2.759037in}}%
\pgfpathlineto{\pgfqpoint{2.120343in}{2.762783in}}%
\pgfpathlineto{\pgfqpoint{2.131702in}{2.766482in}}%
\pgfpathlineto{\pgfqpoint{2.143059in}{2.770048in}}%
\pgfpathlineto{\pgfqpoint{2.154416in}{2.773391in}}%
\pgfpathlineto{\pgfqpoint{2.148609in}{2.781283in}}%
\pgfpathlineto{\pgfqpoint{2.142806in}{2.789166in}}%
\pgfpathlineto{\pgfqpoint{2.137006in}{2.797045in}}%
\pgfpathlineto{\pgfqpoint{2.131211in}{2.804924in}}%
\pgfpathlineto{\pgfqpoint{2.125419in}{2.812809in}}%
\pgfpathlineto{\pgfqpoint{2.114073in}{2.809503in}}%
\pgfpathlineto{\pgfqpoint{2.102729in}{2.805945in}}%
\pgfpathlineto{\pgfqpoint{2.091382in}{2.802230in}}%
\pgfpathlineto{\pgfqpoint{2.080032in}{2.798455in}}%
\pgfpathlineto{\pgfqpoint{2.068676in}{2.794718in}}%
\pgfpathlineto{\pgfqpoint{2.074455in}{2.786866in}}%
\pgfpathlineto{\pgfqpoint{2.080237in}{2.779002in}}%
\pgfpathlineto{\pgfqpoint{2.086024in}{2.771126in}}%
\pgfpathlineto{\pgfqpoint{2.091815in}{2.763236in}}%
\pgfpathclose%
\pgfusepath{stroke,fill}%
\end{pgfscope}%
\begin{pgfscope}%
\pgfpathrectangle{\pgfqpoint{0.887500in}{0.275000in}}{\pgfqpoint{4.225000in}{4.225000in}}%
\pgfusepath{clip}%
\pgfsetbuttcap%
\pgfsetroundjoin%
\definecolor{currentfill}{rgb}{0.121380,0.629492,0.531973}%
\pgfsetfillcolor{currentfill}%
\pgfsetfillopacity{0.700000}%
\pgfsetlinewidth{0.501875pt}%
\definecolor{currentstroke}{rgb}{1.000000,1.000000,1.000000}%
\pgfsetstrokecolor{currentstroke}%
\pgfsetstrokeopacity{0.500000}%
\pgfsetdash{}{0pt}%
\pgfpathmoveto{\pgfqpoint{1.554563in}{2.829844in}}%
\pgfpathlineto{\pgfqpoint{1.566070in}{2.833175in}}%
\pgfpathlineto{\pgfqpoint{1.577571in}{2.836505in}}%
\pgfpathlineto{\pgfqpoint{1.589066in}{2.839835in}}%
\pgfpathlineto{\pgfqpoint{1.600556in}{2.843164in}}%
\pgfpathlineto{\pgfqpoint{1.612041in}{2.846494in}}%
\pgfpathlineto{\pgfqpoint{1.606431in}{2.853904in}}%
\pgfpathlineto{\pgfqpoint{1.600826in}{2.861296in}}%
\pgfpathlineto{\pgfqpoint{1.595226in}{2.868671in}}%
\pgfpathlineto{\pgfqpoint{1.589629in}{2.876029in}}%
\pgfpathlineto{\pgfqpoint{1.578157in}{2.872660in}}%
\pgfpathlineto{\pgfqpoint{1.566679in}{2.869294in}}%
\pgfpathlineto{\pgfqpoint{1.555196in}{2.865932in}}%
\pgfpathlineto{\pgfqpoint{1.543707in}{2.862576in}}%
\pgfpathlineto{\pgfqpoint{1.532212in}{2.859226in}}%
\pgfpathlineto{\pgfqpoint{1.537793in}{2.851911in}}%
\pgfpathlineto{\pgfqpoint{1.543379in}{2.844576in}}%
\pgfpathlineto{\pgfqpoint{1.548969in}{2.837220in}}%
\pgfpathclose%
\pgfusepath{stroke,fill}%
\end{pgfscope}%
\begin{pgfscope}%
\pgfpathrectangle{\pgfqpoint{0.887500in}{0.275000in}}{\pgfqpoint{4.225000in}{4.225000in}}%
\pgfusepath{clip}%
\pgfsetbuttcap%
\pgfsetroundjoin%
\definecolor{currentfill}{rgb}{0.814576,0.883393,0.110347}%
\pgfsetfillcolor{currentfill}%
\pgfsetfillopacity{0.700000}%
\pgfsetlinewidth{0.501875pt}%
\definecolor{currentstroke}{rgb}{1.000000,1.000000,1.000000}%
\pgfsetstrokecolor{currentstroke}%
\pgfsetstrokeopacity{0.500000}%
\pgfsetdash{}{0pt}%
\pgfpathmoveto{\pgfqpoint{3.137270in}{3.547379in}}%
\pgfpathlineto{\pgfqpoint{3.148405in}{3.550943in}}%
\pgfpathlineto{\pgfqpoint{3.159534in}{3.554241in}}%
\pgfpathlineto{\pgfqpoint{3.170657in}{3.557370in}}%
\pgfpathlineto{\pgfqpoint{3.181775in}{3.560426in}}%
\pgfpathlineto{\pgfqpoint{3.192888in}{3.563497in}}%
\pgfpathlineto{\pgfqpoint{3.186721in}{3.567262in}}%
\pgfpathlineto{\pgfqpoint{3.180558in}{3.570722in}}%
\pgfpathlineto{\pgfqpoint{3.174400in}{3.573871in}}%
\pgfpathlineto{\pgfqpoint{3.168246in}{3.576721in}}%
\pgfpathlineto{\pgfqpoint{3.162098in}{3.579296in}}%
\pgfpathlineto{\pgfqpoint{3.151001in}{3.575576in}}%
\pgfpathlineto{\pgfqpoint{3.139899in}{3.571897in}}%
\pgfpathlineto{\pgfqpoint{3.128792in}{3.568185in}}%
\pgfpathlineto{\pgfqpoint{3.117680in}{3.564354in}}%
\pgfpathlineto{\pgfqpoint{3.106563in}{3.560320in}}%
\pgfpathlineto{\pgfqpoint{3.112694in}{3.558623in}}%
\pgfpathlineto{\pgfqpoint{3.118831in}{3.556522in}}%
\pgfpathlineto{\pgfqpoint{3.124973in}{3.553962in}}%
\pgfpathlineto{\pgfqpoint{3.131119in}{3.550907in}}%
\pgfpathclose%
\pgfusepath{stroke,fill}%
\end{pgfscope}%
\begin{pgfscope}%
\pgfpathrectangle{\pgfqpoint{0.887500in}{0.275000in}}{\pgfqpoint{4.225000in}{4.225000in}}%
\pgfusepath{clip}%
\pgfsetbuttcap%
\pgfsetroundjoin%
\definecolor{currentfill}{rgb}{0.327796,0.773980,0.406640}%
\pgfsetfillcolor{currentfill}%
\pgfsetfillopacity{0.700000}%
\pgfsetlinewidth{0.501875pt}%
\definecolor{currentstroke}{rgb}{1.000000,1.000000,1.000000}%
\pgfsetstrokecolor{currentstroke}%
\pgfsetstrokeopacity{0.500000}%
\pgfsetdash{}{0pt}%
\pgfpathmoveto{\pgfqpoint{2.803388in}{3.035691in}}%
\pgfpathlineto{\pgfqpoint{2.814363in}{3.072747in}}%
\pgfpathlineto{\pgfqpoint{2.825307in}{3.117552in}}%
\pgfpathlineto{\pgfqpoint{2.836241in}{3.167362in}}%
\pgfpathlineto{\pgfqpoint{2.847183in}{3.219413in}}%
\pgfpathlineto{\pgfqpoint{2.858151in}{3.270921in}}%
\pgfpathlineto{\pgfqpoint{2.852099in}{3.277567in}}%
\pgfpathlineto{\pgfqpoint{2.846044in}{3.285736in}}%
\pgfpathlineto{\pgfqpoint{2.839984in}{3.295537in}}%
\pgfpathlineto{\pgfqpoint{2.833920in}{3.306649in}}%
\pgfpathlineto{\pgfqpoint{2.827854in}{3.318601in}}%
\pgfpathlineto{\pgfqpoint{2.816885in}{3.274924in}}%
\pgfpathlineto{\pgfqpoint{2.805938in}{3.230362in}}%
\pgfpathlineto{\pgfqpoint{2.795001in}{3.187042in}}%
\pgfpathlineto{\pgfqpoint{2.784059in}{3.147077in}}%
\pgfpathlineto{\pgfqpoint{2.773093in}{3.112566in}}%
\pgfpathlineto{\pgfqpoint{2.779186in}{3.091756in}}%
\pgfpathlineto{\pgfqpoint{2.785265in}{3.073099in}}%
\pgfpathlineto{\pgfqpoint{2.791325in}{3.057331in}}%
\pgfpathlineto{\pgfqpoint{2.797365in}{3.044940in}}%
\pgfpathclose%
\pgfusepath{stroke,fill}%
\end{pgfscope}%
\begin{pgfscope}%
\pgfpathrectangle{\pgfqpoint{0.887500in}{0.275000in}}{\pgfqpoint{4.225000in}{4.225000in}}%
\pgfusepath{clip}%
\pgfsetbuttcap%
\pgfsetroundjoin%
\definecolor{currentfill}{rgb}{0.751884,0.874951,0.143228}%
\pgfsetfillcolor{currentfill}%
\pgfsetfillopacity{0.700000}%
\pgfsetlinewidth{0.501875pt}%
\definecolor{currentstroke}{rgb}{1.000000,1.000000,1.000000}%
\pgfsetstrokecolor{currentstroke}%
\pgfsetstrokeopacity{0.500000}%
\pgfsetdash{}{0pt}%
\pgfpathmoveto{\pgfqpoint{3.822453in}{3.470538in}}%
\pgfpathlineto{\pgfqpoint{3.833441in}{3.474908in}}%
\pgfpathlineto{\pgfqpoint{3.844423in}{3.479231in}}%
\pgfpathlineto{\pgfqpoint{3.855399in}{3.483513in}}%
\pgfpathlineto{\pgfqpoint{3.866371in}{3.487763in}}%
\pgfpathlineto{\pgfqpoint{3.877337in}{3.491986in}}%
\pgfpathlineto{\pgfqpoint{3.871046in}{3.504100in}}%
\pgfpathlineto{\pgfqpoint{3.864757in}{3.516139in}}%
\pgfpathlineto{\pgfqpoint{3.858469in}{3.528087in}}%
\pgfpathlineto{\pgfqpoint{3.852182in}{3.539924in}}%
\pgfpathlineto{\pgfqpoint{3.845895in}{3.551633in}}%
\pgfpathlineto{\pgfqpoint{3.834930in}{3.547159in}}%
\pgfpathlineto{\pgfqpoint{3.823959in}{3.542621in}}%
\pgfpathlineto{\pgfqpoint{3.812982in}{3.538009in}}%
\pgfpathlineto{\pgfqpoint{3.801999in}{3.533313in}}%
\pgfpathlineto{\pgfqpoint{3.791009in}{3.528522in}}%
\pgfpathlineto{\pgfqpoint{3.797294in}{3.517054in}}%
\pgfpathlineto{\pgfqpoint{3.803581in}{3.505498in}}%
\pgfpathlineto{\pgfqpoint{3.809869in}{3.493878in}}%
\pgfpathlineto{\pgfqpoint{3.816160in}{3.482217in}}%
\pgfpathclose%
\pgfusepath{stroke,fill}%
\end{pgfscope}%
\begin{pgfscope}%
\pgfpathrectangle{\pgfqpoint{0.887500in}{0.275000in}}{\pgfqpoint{4.225000in}{4.225000in}}%
\pgfusepath{clip}%
\pgfsetbuttcap%
\pgfsetroundjoin%
\definecolor{currentfill}{rgb}{0.119423,0.611141,0.538982}%
\pgfsetfillcolor{currentfill}%
\pgfsetfillopacity{0.700000}%
\pgfsetlinewidth{0.501875pt}%
\definecolor{currentstroke}{rgb}{1.000000,1.000000,1.000000}%
\pgfsetstrokecolor{currentstroke}%
\pgfsetstrokeopacity{0.500000}%
\pgfsetdash{}{0pt}%
\pgfpathmoveto{\pgfqpoint{1.868909in}{2.782882in}}%
\pgfpathlineto{\pgfqpoint{1.880343in}{2.786260in}}%
\pgfpathlineto{\pgfqpoint{1.891772in}{2.789648in}}%
\pgfpathlineto{\pgfqpoint{1.903195in}{2.793043in}}%
\pgfpathlineto{\pgfqpoint{1.914612in}{2.796439in}}%
\pgfpathlineto{\pgfqpoint{1.926024in}{2.799829in}}%
\pgfpathlineto{\pgfqpoint{1.920296in}{2.807550in}}%
\pgfpathlineto{\pgfqpoint{1.914573in}{2.815259in}}%
\pgfpathlineto{\pgfqpoint{1.908853in}{2.822955in}}%
\pgfpathlineto{\pgfqpoint{1.903138in}{2.830640in}}%
\pgfpathlineto{\pgfqpoint{1.897427in}{2.838311in}}%
\pgfpathlineto{\pgfqpoint{1.886028in}{2.834897in}}%
\pgfpathlineto{\pgfqpoint{1.874624in}{2.831477in}}%
\pgfpathlineto{\pgfqpoint{1.863215in}{2.828059in}}%
\pgfpathlineto{\pgfqpoint{1.851800in}{2.824648in}}%
\pgfpathlineto{\pgfqpoint{1.840380in}{2.821248in}}%
\pgfpathlineto{\pgfqpoint{1.846077in}{2.813601in}}%
\pgfpathlineto{\pgfqpoint{1.851779in}{2.805940in}}%
\pgfpathlineto{\pgfqpoint{1.857485in}{2.798267in}}%
\pgfpathlineto{\pgfqpoint{1.863195in}{2.790581in}}%
\pgfpathclose%
\pgfusepath{stroke,fill}%
\end{pgfscope}%
\begin{pgfscope}%
\pgfpathrectangle{\pgfqpoint{0.887500in}{0.275000in}}{\pgfqpoint{4.225000in}{4.225000in}}%
\pgfusepath{clip}%
\pgfsetbuttcap%
\pgfsetroundjoin%
\definecolor{currentfill}{rgb}{0.121148,0.592739,0.544641}%
\pgfsetfillcolor{currentfill}%
\pgfsetfillopacity{0.700000}%
\pgfsetlinewidth{0.501875pt}%
\definecolor{currentstroke}{rgb}{1.000000,1.000000,1.000000}%
\pgfsetstrokecolor{currentstroke}%
\pgfsetstrokeopacity{0.500000}%
\pgfsetdash{}{0pt}%
\pgfpathmoveto{\pgfqpoint{2.326142in}{2.731461in}}%
\pgfpathlineto{\pgfqpoint{2.337395in}{2.738629in}}%
\pgfpathlineto{\pgfqpoint{2.348634in}{2.746442in}}%
\pgfpathlineto{\pgfqpoint{2.359861in}{2.754742in}}%
\pgfpathlineto{\pgfqpoint{2.371080in}{2.763371in}}%
\pgfpathlineto{\pgfqpoint{2.382293in}{2.772169in}}%
\pgfpathlineto{\pgfqpoint{2.376371in}{2.782192in}}%
\pgfpathlineto{\pgfqpoint{2.370455in}{2.792062in}}%
\pgfpathlineto{\pgfqpoint{2.364546in}{2.801771in}}%
\pgfpathlineto{\pgfqpoint{2.358642in}{2.811312in}}%
\pgfpathlineto{\pgfqpoint{2.352745in}{2.820676in}}%
\pgfpathlineto{\pgfqpoint{2.341567in}{2.810596in}}%
\pgfpathlineto{\pgfqpoint{2.330388in}{2.800556in}}%
\pgfpathlineto{\pgfqpoint{2.319202in}{2.790806in}}%
\pgfpathlineto{\pgfqpoint{2.308003in}{2.781594in}}%
\pgfpathlineto{\pgfqpoint{2.296788in}{2.773170in}}%
\pgfpathlineto{\pgfqpoint{2.302648in}{2.764994in}}%
\pgfpathlineto{\pgfqpoint{2.308514in}{2.756721in}}%
\pgfpathlineto{\pgfqpoint{2.314385in}{2.748366in}}%
\pgfpathlineto{\pgfqpoint{2.320261in}{2.739941in}}%
\pgfpathclose%
\pgfusepath{stroke,fill}%
\end{pgfscope}%
\begin{pgfscope}%
\pgfpathrectangle{\pgfqpoint{0.887500in}{0.275000in}}{\pgfqpoint{4.225000in}{4.225000in}}%
\pgfusepath{clip}%
\pgfsetbuttcap%
\pgfsetroundjoin%
\definecolor{currentfill}{rgb}{0.699415,0.867117,0.175971}%
\pgfsetfillcolor{currentfill}%
\pgfsetfillopacity{0.700000}%
\pgfsetlinewidth{0.501875pt}%
\definecolor{currentstroke}{rgb}{1.000000,1.000000,1.000000}%
\pgfsetstrokecolor{currentstroke}%
\pgfsetstrokeopacity{0.500000}%
\pgfsetdash{}{0pt}%
\pgfpathmoveto{\pgfqpoint{3.908822in}{3.430891in}}%
\pgfpathlineto{\pgfqpoint{3.919782in}{3.434802in}}%
\pgfpathlineto{\pgfqpoint{3.930738in}{3.438772in}}%
\pgfpathlineto{\pgfqpoint{3.941691in}{3.442803in}}%
\pgfpathlineto{\pgfqpoint{3.952641in}{3.446899in}}%
\pgfpathlineto{\pgfqpoint{3.963587in}{3.451061in}}%
\pgfpathlineto{\pgfqpoint{3.957284in}{3.463490in}}%
\pgfpathlineto{\pgfqpoint{3.950984in}{3.475905in}}%
\pgfpathlineto{\pgfqpoint{3.944686in}{3.488297in}}%
\pgfpathlineto{\pgfqpoint{3.938390in}{3.500655in}}%
\pgfpathlineto{\pgfqpoint{3.932095in}{3.512969in}}%
\pgfpathlineto{\pgfqpoint{3.921153in}{3.508766in}}%
\pgfpathlineto{\pgfqpoint{3.910206in}{3.504574in}}%
\pgfpathlineto{\pgfqpoint{3.899254in}{3.500385in}}%
\pgfpathlineto{\pgfqpoint{3.888298in}{3.496191in}}%
\pgfpathlineto{\pgfqpoint{3.877337in}{3.491986in}}%
\pgfpathlineto{\pgfqpoint{3.883629in}{3.479816in}}%
\pgfpathlineto{\pgfqpoint{3.889923in}{3.467604in}}%
\pgfpathlineto{\pgfqpoint{3.896220in}{3.455369in}}%
\pgfpathlineto{\pgfqpoint{3.902520in}{3.443126in}}%
\pgfpathclose%
\pgfusepath{stroke,fill}%
\end{pgfscope}%
\begin{pgfscope}%
\pgfpathrectangle{\pgfqpoint{0.887500in}{0.275000in}}{\pgfqpoint{4.225000in}{4.225000in}}%
\pgfusepath{clip}%
\pgfsetbuttcap%
\pgfsetroundjoin%
\definecolor{currentfill}{rgb}{0.835270,0.886029,0.102646}%
\pgfsetfillcolor{currentfill}%
\pgfsetfillopacity{0.700000}%
\pgfsetlinewidth{0.501875pt}%
\definecolor{currentstroke}{rgb}{1.000000,1.000000,1.000000}%
\pgfsetstrokecolor{currentstroke}%
\pgfsetstrokeopacity{0.500000}%
\pgfsetdash{}{0pt}%
\pgfpathmoveto{\pgfqpoint{3.507930in}{3.550093in}}%
\pgfpathlineto{\pgfqpoint{3.518985in}{3.554141in}}%
\pgfpathlineto{\pgfqpoint{3.530035in}{3.558149in}}%
\pgfpathlineto{\pgfqpoint{3.541080in}{3.562109in}}%
\pgfpathlineto{\pgfqpoint{3.552119in}{3.566009in}}%
\pgfpathlineto{\pgfqpoint{3.563152in}{3.569840in}}%
\pgfpathlineto{\pgfqpoint{3.556871in}{3.577246in}}%
\pgfpathlineto{\pgfqpoint{3.550591in}{3.584416in}}%
\pgfpathlineto{\pgfqpoint{3.544314in}{3.591419in}}%
\pgfpathlineto{\pgfqpoint{3.538040in}{3.598320in}}%
\pgfpathlineto{\pgfqpoint{3.531771in}{3.605186in}}%
\pgfpathlineto{\pgfqpoint{3.520740in}{3.600574in}}%
\pgfpathlineto{\pgfqpoint{3.509706in}{3.595950in}}%
\pgfpathlineto{\pgfqpoint{3.498666in}{3.591308in}}%
\pgfpathlineto{\pgfqpoint{3.487622in}{3.586645in}}%
\pgfpathlineto{\pgfqpoint{3.476574in}{3.581954in}}%
\pgfpathlineto{\pgfqpoint{3.482838in}{3.575760in}}%
\pgfpathlineto{\pgfqpoint{3.489106in}{3.569557in}}%
\pgfpathlineto{\pgfqpoint{3.495378in}{3.563267in}}%
\pgfpathlineto{\pgfqpoint{3.501653in}{3.556806in}}%
\pgfpathclose%
\pgfusepath{stroke,fill}%
\end{pgfscope}%
\begin{pgfscope}%
\pgfpathrectangle{\pgfqpoint{0.887500in}{0.275000in}}{\pgfqpoint{4.225000in}{4.225000in}}%
\pgfusepath{clip}%
\pgfsetbuttcap%
\pgfsetroundjoin%
\definecolor{currentfill}{rgb}{0.636902,0.856542,0.216620}%
\pgfsetfillcolor{currentfill}%
\pgfsetfillopacity{0.700000}%
\pgfsetlinewidth{0.501875pt}%
\definecolor{currentstroke}{rgb}{1.000000,1.000000,1.000000}%
\pgfsetstrokecolor{currentstroke}%
\pgfsetstrokeopacity{0.500000}%
\pgfsetdash{}{0pt}%
\pgfpathmoveto{\pgfqpoint{3.995149in}{3.389064in}}%
\pgfpathlineto{\pgfqpoint{4.006091in}{3.393058in}}%
\pgfpathlineto{\pgfqpoint{4.017031in}{3.397141in}}%
\pgfpathlineto{\pgfqpoint{4.027968in}{3.401310in}}%
\pgfpathlineto{\pgfqpoint{4.038903in}{3.405566in}}%
\pgfpathlineto{\pgfqpoint{4.032589in}{3.418284in}}%
\pgfpathlineto{\pgfqpoint{4.026276in}{3.430925in}}%
\pgfpathlineto{\pgfqpoint{4.019964in}{3.443493in}}%
\pgfpathlineto{\pgfqpoint{4.013653in}{3.455996in}}%
\pgfpathlineto{\pgfqpoint{4.007344in}{3.468441in}}%
\pgfpathlineto{\pgfqpoint{3.996409in}{3.463981in}}%
\pgfpathlineto{\pgfqpoint{3.985471in}{3.459600in}}%
\pgfpathlineto{\pgfqpoint{3.974531in}{3.455294in}}%
\pgfpathlineto{\pgfqpoint{3.963587in}{3.451061in}}%
\pgfpathlineto{\pgfqpoint{3.969893in}{3.438629in}}%
\pgfpathlineto{\pgfqpoint{3.976202in}{3.426203in}}%
\pgfpathlineto{\pgfqpoint{3.982514in}{3.413794in}}%
\pgfpathlineto{\pgfqpoint{3.988829in}{3.401411in}}%
\pgfpathclose%
\pgfusepath{stroke,fill}%
\end{pgfscope}%
\begin{pgfscope}%
\pgfpathrectangle{\pgfqpoint{0.887500in}{0.275000in}}{\pgfqpoint{4.225000in}{4.225000in}}%
\pgfusepath{clip}%
\pgfsetbuttcap%
\pgfsetroundjoin%
\definecolor{currentfill}{rgb}{0.120081,0.622161,0.534946}%
\pgfsetfillcolor{currentfill}%
\pgfsetfillopacity{0.700000}%
\pgfsetlinewidth{0.501875pt}%
\definecolor{currentstroke}{rgb}{1.000000,1.000000,1.000000}%
\pgfsetstrokecolor{currentstroke}%
\pgfsetstrokeopacity{0.500000}%
\pgfsetdash{}{0pt}%
\pgfpathmoveto{\pgfqpoint{1.640152in}{2.809191in}}%
\pgfpathlineto{\pgfqpoint{1.651645in}{2.812490in}}%
\pgfpathlineto{\pgfqpoint{1.663133in}{2.815789in}}%
\pgfpathlineto{\pgfqpoint{1.674615in}{2.819090in}}%
\pgfpathlineto{\pgfqpoint{1.686092in}{2.822393in}}%
\pgfpathlineto{\pgfqpoint{1.697563in}{2.825701in}}%
\pgfpathlineto{\pgfqpoint{1.691918in}{2.833220in}}%
\pgfpathlineto{\pgfqpoint{1.686277in}{2.840725in}}%
\pgfpathlineto{\pgfqpoint{1.680640in}{2.848216in}}%
\pgfpathlineto{\pgfqpoint{1.675008in}{2.855693in}}%
\pgfpathlineto{\pgfqpoint{1.669379in}{2.863156in}}%
\pgfpathlineto{\pgfqpoint{1.657923in}{2.859820in}}%
\pgfpathlineto{\pgfqpoint{1.646461in}{2.856486in}}%
\pgfpathlineto{\pgfqpoint{1.634993in}{2.853155in}}%
\pgfpathlineto{\pgfqpoint{1.623520in}{2.849824in}}%
\pgfpathlineto{\pgfqpoint{1.612041in}{2.846494in}}%
\pgfpathlineto{\pgfqpoint{1.617654in}{2.839067in}}%
\pgfpathlineto{\pgfqpoint{1.623272in}{2.831623in}}%
\pgfpathlineto{\pgfqpoint{1.628894in}{2.824163in}}%
\pgfpathlineto{\pgfqpoint{1.634521in}{2.816685in}}%
\pgfpathclose%
\pgfusepath{stroke,fill}%
\end{pgfscope}%
\begin{pgfscope}%
\pgfpathrectangle{\pgfqpoint{0.887500in}{0.275000in}}{\pgfqpoint{4.225000in}{4.225000in}}%
\pgfusepath{clip}%
\pgfsetbuttcap%
\pgfsetroundjoin%
\definecolor{currentfill}{rgb}{0.153894,0.680203,0.504172}%
\pgfsetfillcolor{currentfill}%
\pgfsetfillopacity{0.700000}%
\pgfsetlinewidth{0.501875pt}%
\definecolor{currentstroke}{rgb}{1.000000,1.000000,1.000000}%
\pgfsetstrokecolor{currentstroke}%
\pgfsetstrokeopacity{0.500000}%
\pgfsetdash{}{0pt}%
\pgfpathmoveto{\pgfqpoint{2.721618in}{2.927868in}}%
\pgfpathlineto{\pgfqpoint{2.732805in}{2.936915in}}%
\pgfpathlineto{\pgfqpoint{2.744004in}{2.944278in}}%
\pgfpathlineto{\pgfqpoint{2.755215in}{2.949492in}}%
\pgfpathlineto{\pgfqpoint{2.766438in}{2.952567in}}%
\pgfpathlineto{\pgfqpoint{2.777659in}{2.955053in}}%
\pgfpathlineto{\pgfqpoint{2.771680in}{2.957415in}}%
\pgfpathlineto{\pgfqpoint{2.765702in}{2.960350in}}%
\pgfpathlineto{\pgfqpoint{2.759723in}{2.964104in}}%
\pgfpathlineto{\pgfqpoint{2.753742in}{2.968923in}}%
\pgfpathlineto{\pgfqpoint{2.747755in}{2.975051in}}%
\pgfpathlineto{\pgfqpoint{2.736529in}{2.975154in}}%
\pgfpathlineto{\pgfqpoint{2.725302in}{2.974660in}}%
\pgfpathlineto{\pgfqpoint{2.714086in}{2.971900in}}%
\pgfpathlineto{\pgfqpoint{2.702885in}{2.966875in}}%
\pgfpathlineto{\pgfqpoint{2.691695in}{2.960098in}}%
\pgfpathlineto{\pgfqpoint{2.697669in}{2.953908in}}%
\pgfpathlineto{\pgfqpoint{2.703647in}{2.947657in}}%
\pgfpathlineto{\pgfqpoint{2.709631in}{2.941278in}}%
\pgfpathlineto{\pgfqpoint{2.715621in}{2.934704in}}%
\pgfpathclose%
\pgfusepath{stroke,fill}%
\end{pgfscope}%
\begin{pgfscope}%
\pgfpathrectangle{\pgfqpoint{0.887500in}{0.275000in}}{\pgfqpoint{4.225000in}{4.225000in}}%
\pgfusepath{clip}%
\pgfsetbuttcap%
\pgfsetroundjoin%
\definecolor{currentfill}{rgb}{0.121831,0.589055,0.545623}%
\pgfsetfillcolor{currentfill}%
\pgfsetfillopacity{0.700000}%
\pgfsetlinewidth{0.501875pt}%
\definecolor{currentstroke}{rgb}{1.000000,1.000000,1.000000}%
\pgfsetstrokecolor{currentstroke}%
\pgfsetstrokeopacity{0.500000}%
\pgfsetdash{}{0pt}%
\pgfpathmoveto{\pgfqpoint{2.183514in}{2.733642in}}%
\pgfpathlineto{\pgfqpoint{2.194881in}{2.736819in}}%
\pgfpathlineto{\pgfqpoint{2.206249in}{2.739690in}}%
\pgfpathlineto{\pgfqpoint{2.217618in}{2.742286in}}%
\pgfpathlineto{\pgfqpoint{2.228981in}{2.744837in}}%
\pgfpathlineto{\pgfqpoint{2.240334in}{2.747596in}}%
\pgfpathlineto{\pgfqpoint{2.234498in}{2.755380in}}%
\pgfpathlineto{\pgfqpoint{2.228668in}{2.763107in}}%
\pgfpathlineto{\pgfqpoint{2.222842in}{2.770770in}}%
\pgfpathlineto{\pgfqpoint{2.217022in}{2.778377in}}%
\pgfpathlineto{\pgfqpoint{2.211206in}{2.785960in}}%
\pgfpathlineto{\pgfqpoint{2.199858in}{2.783529in}}%
\pgfpathlineto{\pgfqpoint{2.188499in}{2.781333in}}%
\pgfpathlineto{\pgfqpoint{2.177136in}{2.779059in}}%
\pgfpathlineto{\pgfqpoint{2.165774in}{2.776424in}}%
\pgfpathlineto{\pgfqpoint{2.154416in}{2.773391in}}%
\pgfpathlineto{\pgfqpoint{2.160227in}{2.765485in}}%
\pgfpathlineto{\pgfqpoint{2.166043in}{2.757561in}}%
\pgfpathlineto{\pgfqpoint{2.171862in}{2.749612in}}%
\pgfpathlineto{\pgfqpoint{2.177686in}{2.741639in}}%
\pgfpathclose%
\pgfusepath{stroke,fill}%
\end{pgfscope}%
\begin{pgfscope}%
\pgfpathrectangle{\pgfqpoint{0.887500in}{0.275000in}}{\pgfqpoint{4.225000in}{4.225000in}}%
\pgfusepath{clip}%
\pgfsetbuttcap%
\pgfsetroundjoin%
\definecolor{currentfill}{rgb}{0.124780,0.640461,0.527068}%
\pgfsetfillcolor{currentfill}%
\pgfsetfillopacity{0.700000}%
\pgfsetlinewidth{0.501875pt}%
\definecolor{currentstroke}{rgb}{1.000000,1.000000,1.000000}%
\pgfsetstrokecolor{currentstroke}%
\pgfsetstrokeopacity{0.500000}%
\pgfsetdash{}{0pt}%
\pgfpathmoveto{\pgfqpoint{2.523977in}{2.818337in}}%
\pgfpathlineto{\pgfqpoint{2.535123in}{2.831480in}}%
\pgfpathlineto{\pgfqpoint{2.546273in}{2.844376in}}%
\pgfpathlineto{\pgfqpoint{2.557430in}{2.856671in}}%
\pgfpathlineto{\pgfqpoint{2.568599in}{2.868009in}}%
\pgfpathlineto{\pgfqpoint{2.579782in}{2.878088in}}%
\pgfpathlineto{\pgfqpoint{2.573808in}{2.887578in}}%
\pgfpathlineto{\pgfqpoint{2.567851in}{2.896050in}}%
\pgfpathlineto{\pgfqpoint{2.561911in}{2.903383in}}%
\pgfpathlineto{\pgfqpoint{2.555987in}{2.909720in}}%
\pgfpathlineto{\pgfqpoint{2.550078in}{2.915292in}}%
\pgfpathlineto{\pgfqpoint{2.538882in}{2.907570in}}%
\pgfpathlineto{\pgfqpoint{2.527697in}{2.898785in}}%
\pgfpathlineto{\pgfqpoint{2.516521in}{2.889133in}}%
\pgfpathlineto{\pgfqpoint{2.505353in}{2.878854in}}%
\pgfpathlineto{\pgfqpoint{2.494188in}{2.868190in}}%
\pgfpathlineto{\pgfqpoint{2.500120in}{2.859566in}}%
\pgfpathlineto{\pgfqpoint{2.506065in}{2.850360in}}%
\pgfpathlineto{\pgfqpoint{2.512022in}{2.840437in}}%
\pgfpathlineto{\pgfqpoint{2.517993in}{2.829725in}}%
\pgfpathclose%
\pgfusepath{stroke,fill}%
\end{pgfscope}%
\begin{pgfscope}%
\pgfpathrectangle{\pgfqpoint{0.887500in}{0.275000in}}{\pgfqpoint{4.225000in}{4.225000in}}%
\pgfusepath{clip}%
\pgfsetbuttcap%
\pgfsetroundjoin%
\definecolor{currentfill}{rgb}{0.824940,0.884720,0.106217}%
\pgfsetfillcolor{currentfill}%
\pgfsetfillopacity{0.700000}%
\pgfsetlinewidth{0.501875pt}%
\definecolor{currentstroke}{rgb}{1.000000,1.000000,1.000000}%
\pgfsetstrokecolor{currentstroke}%
\pgfsetstrokeopacity{0.500000}%
\pgfsetdash{}{0pt}%
\pgfpathmoveto{\pgfqpoint{3.365901in}{3.542775in}}%
\pgfpathlineto{\pgfqpoint{3.376981in}{3.545430in}}%
\pgfpathlineto{\pgfqpoint{3.388056in}{3.548322in}}%
\pgfpathlineto{\pgfqpoint{3.399129in}{3.551554in}}%
\pgfpathlineto{\pgfqpoint{3.410200in}{3.555186in}}%
\pgfpathlineto{\pgfqpoint{3.421270in}{3.559167in}}%
\pgfpathlineto{\pgfqpoint{3.415020in}{3.564894in}}%
\pgfpathlineto{\pgfqpoint{3.408777in}{3.570790in}}%
\pgfpathlineto{\pgfqpoint{3.402540in}{3.576884in}}%
\pgfpathlineto{\pgfqpoint{3.396309in}{3.583126in}}%
\pgfpathlineto{\pgfqpoint{3.390083in}{3.589461in}}%
\pgfpathlineto{\pgfqpoint{3.379023in}{3.584987in}}%
\pgfpathlineto{\pgfqpoint{3.367962in}{3.580869in}}%
\pgfpathlineto{\pgfqpoint{3.356899in}{3.577162in}}%
\pgfpathlineto{\pgfqpoint{3.345833in}{3.573821in}}%
\pgfpathlineto{\pgfqpoint{3.334764in}{3.570759in}}%
\pgfpathlineto{\pgfqpoint{3.340982in}{3.565162in}}%
\pgfpathlineto{\pgfqpoint{3.347205in}{3.559546in}}%
\pgfpathlineto{\pgfqpoint{3.353432in}{3.553937in}}%
\pgfpathlineto{\pgfqpoint{3.359664in}{3.548355in}}%
\pgfpathclose%
\pgfusepath{stroke,fill}%
\end{pgfscope}%
\begin{pgfscope}%
\pgfpathrectangle{\pgfqpoint{0.887500in}{0.275000in}}{\pgfqpoint{4.225000in}{4.225000in}}%
\pgfusepath{clip}%
\pgfsetbuttcap%
\pgfsetroundjoin%
\definecolor{currentfill}{rgb}{0.119738,0.603785,0.541400}%
\pgfsetfillcolor{currentfill}%
\pgfsetfillopacity{0.700000}%
\pgfsetlinewidth{0.501875pt}%
\definecolor{currentstroke}{rgb}{1.000000,1.000000,1.000000}%
\pgfsetstrokecolor{currentstroke}%
\pgfsetstrokeopacity{0.500000}%
\pgfsetdash{}{0pt}%
\pgfpathmoveto{\pgfqpoint{1.954723in}{2.761046in}}%
\pgfpathlineto{\pgfqpoint{1.966143in}{2.764395in}}%
\pgfpathlineto{\pgfqpoint{1.977559in}{2.767725in}}%
\pgfpathlineto{\pgfqpoint{1.988969in}{2.771030in}}%
\pgfpathlineto{\pgfqpoint{2.000375in}{2.774307in}}%
\pgfpathlineto{\pgfqpoint{2.011776in}{2.777577in}}%
\pgfpathlineto{\pgfqpoint{2.006014in}{2.785388in}}%
\pgfpathlineto{\pgfqpoint{2.000257in}{2.793186in}}%
\pgfpathlineto{\pgfqpoint{1.994504in}{2.800972in}}%
\pgfpathlineto{\pgfqpoint{1.988755in}{2.808746in}}%
\pgfpathlineto{\pgfqpoint{1.983010in}{2.816508in}}%
\pgfpathlineto{\pgfqpoint{1.971622in}{2.813207in}}%
\pgfpathlineto{\pgfqpoint{1.960230in}{2.809899in}}%
\pgfpathlineto{\pgfqpoint{1.948833in}{2.806565in}}%
\pgfpathlineto{\pgfqpoint{1.937431in}{2.803206in}}%
\pgfpathlineto{\pgfqpoint{1.926024in}{2.799829in}}%
\pgfpathlineto{\pgfqpoint{1.931756in}{2.792096in}}%
\pgfpathlineto{\pgfqpoint{1.937491in}{2.784351in}}%
\pgfpathlineto{\pgfqpoint{1.943231in}{2.776595in}}%
\pgfpathlineto{\pgfqpoint{1.948975in}{2.768826in}}%
\pgfpathclose%
\pgfusepath{stroke,fill}%
\end{pgfscope}%
\begin{pgfscope}%
\pgfpathrectangle{\pgfqpoint{0.887500in}{0.275000in}}{\pgfqpoint{4.225000in}{4.225000in}}%
\pgfusepath{clip}%
\pgfsetbuttcap%
\pgfsetroundjoin%
\definecolor{currentfill}{rgb}{0.119483,0.614817,0.537692}%
\pgfsetfillcolor{currentfill}%
\pgfsetfillopacity{0.700000}%
\pgfsetlinewidth{0.501875pt}%
\definecolor{currentstroke}{rgb}{1.000000,1.000000,1.000000}%
\pgfsetstrokecolor{currentstroke}%
\pgfsetstrokeopacity{0.500000}%
\pgfsetdash{}{0pt}%
\pgfpathmoveto{\pgfqpoint{2.468132in}{2.760330in}}%
\pgfpathlineto{\pgfqpoint{2.479328in}{2.770210in}}%
\pgfpathlineto{\pgfqpoint{2.490508in}{2.780979in}}%
\pgfpathlineto{\pgfqpoint{2.501674in}{2.792734in}}%
\pgfpathlineto{\pgfqpoint{2.512829in}{2.805304in}}%
\pgfpathlineto{\pgfqpoint{2.523977in}{2.818337in}}%
\pgfpathlineto{\pgfqpoint{2.517993in}{2.829725in}}%
\pgfpathlineto{\pgfqpoint{2.512022in}{2.840437in}}%
\pgfpathlineto{\pgfqpoint{2.506065in}{2.850360in}}%
\pgfpathlineto{\pgfqpoint{2.500120in}{2.859566in}}%
\pgfpathlineto{\pgfqpoint{2.494188in}{2.868190in}}%
\pgfpathlineto{\pgfqpoint{2.483024in}{2.857384in}}%
\pgfpathlineto{\pgfqpoint{2.471858in}{2.846677in}}%
\pgfpathlineto{\pgfqpoint{2.460685in}{2.836307in}}%
\pgfpathlineto{\pgfqpoint{2.449504in}{2.826392in}}%
\pgfpathlineto{\pgfqpoint{2.438315in}{2.816863in}}%
\pgfpathlineto{\pgfqpoint{2.444264in}{2.806137in}}%
\pgfpathlineto{\pgfqpoint{2.450220in}{2.795127in}}%
\pgfpathlineto{\pgfqpoint{2.456184in}{2.783810in}}%
\pgfpathlineto{\pgfqpoint{2.462155in}{2.772187in}}%
\pgfpathclose%
\pgfusepath{stroke,fill}%
\end{pgfscope}%
\begin{pgfscope}%
\pgfpathrectangle{\pgfqpoint{0.887500in}{0.275000in}}{\pgfqpoint{4.225000in}{4.225000in}}%
\pgfusepath{clip}%
\pgfsetbuttcap%
\pgfsetroundjoin%
\definecolor{currentfill}{rgb}{0.824940,0.884720,0.106217}%
\pgfsetfillcolor{currentfill}%
\pgfsetfillopacity{0.700000}%
\pgfsetlinewidth{0.501875pt}%
\definecolor{currentstroke}{rgb}{1.000000,1.000000,1.000000}%
\pgfsetstrokecolor{currentstroke}%
\pgfsetstrokeopacity{0.500000}%
\pgfsetdash{}{0pt}%
\pgfpathmoveto{\pgfqpoint{3.594544in}{3.526985in}}%
\pgfpathlineto{\pgfqpoint{3.605573in}{3.530089in}}%
\pgfpathlineto{\pgfqpoint{3.616594in}{3.533045in}}%
\pgfpathlineto{\pgfqpoint{3.627608in}{3.535882in}}%
\pgfpathlineto{\pgfqpoint{3.638616in}{3.538681in}}%
\pgfpathlineto{\pgfqpoint{3.649619in}{3.541528in}}%
\pgfpathlineto{\pgfqpoint{3.643345in}{3.551547in}}%
\pgfpathlineto{\pgfqpoint{3.637068in}{3.561187in}}%
\pgfpathlineto{\pgfqpoint{3.630790in}{3.570461in}}%
\pgfpathlineto{\pgfqpoint{3.624511in}{3.579406in}}%
\pgfpathlineto{\pgfqpoint{3.618232in}{3.588061in}}%
\pgfpathlineto{\pgfqpoint{3.607226in}{3.584437in}}%
\pgfpathlineto{\pgfqpoint{3.596216in}{3.580855in}}%
\pgfpathlineto{\pgfqpoint{3.585200in}{3.577258in}}%
\pgfpathlineto{\pgfqpoint{3.574179in}{3.573593in}}%
\pgfpathlineto{\pgfqpoint{3.563152in}{3.569840in}}%
\pgfpathlineto{\pgfqpoint{3.569433in}{3.562133in}}%
\pgfpathlineto{\pgfqpoint{3.575714in}{3.554057in}}%
\pgfpathlineto{\pgfqpoint{3.581994in}{3.545546in}}%
\pgfpathlineto{\pgfqpoint{3.588271in}{3.536532in}}%
\pgfpathclose%
\pgfusepath{stroke,fill}%
\end{pgfscope}%
\begin{pgfscope}%
\pgfpathrectangle{\pgfqpoint{0.887500in}{0.275000in}}{\pgfqpoint{4.225000in}{4.225000in}}%
\pgfusepath{clip}%
\pgfsetbuttcap%
\pgfsetroundjoin%
\definecolor{currentfill}{rgb}{0.119483,0.614817,0.537692}%
\pgfsetfillcolor{currentfill}%
\pgfsetfillopacity{0.700000}%
\pgfsetlinewidth{0.501875pt}%
\definecolor{currentstroke}{rgb}{1.000000,1.000000,1.000000}%
\pgfsetstrokecolor{currentstroke}%
\pgfsetstrokeopacity{0.500000}%
\pgfsetdash{}{0pt}%
\pgfpathmoveto{\pgfqpoint{1.725851in}{2.787892in}}%
\pgfpathlineto{\pgfqpoint{1.737330in}{2.791184in}}%
\pgfpathlineto{\pgfqpoint{1.748803in}{2.794482in}}%
\pgfpathlineto{\pgfqpoint{1.760271in}{2.797789in}}%
\pgfpathlineto{\pgfqpoint{1.771733in}{2.801105in}}%
\pgfpathlineto{\pgfqpoint{1.783189in}{2.804432in}}%
\pgfpathlineto{\pgfqpoint{1.777509in}{2.812043in}}%
\pgfpathlineto{\pgfqpoint{1.771833in}{2.819641in}}%
\pgfpathlineto{\pgfqpoint{1.766162in}{2.827224in}}%
\pgfpathlineto{\pgfqpoint{1.760494in}{2.834794in}}%
\pgfpathlineto{\pgfqpoint{1.754831in}{2.842349in}}%
\pgfpathlineto{\pgfqpoint{1.743389in}{2.839000in}}%
\pgfpathlineto{\pgfqpoint{1.731941in}{2.835662in}}%
\pgfpathlineto{\pgfqpoint{1.720488in}{2.832334in}}%
\pgfpathlineto{\pgfqpoint{1.709028in}{2.829014in}}%
\pgfpathlineto{\pgfqpoint{1.697563in}{2.825701in}}%
\pgfpathlineto{\pgfqpoint{1.703212in}{2.818168in}}%
\pgfpathlineto{\pgfqpoint{1.708866in}{2.810620in}}%
\pgfpathlineto{\pgfqpoint{1.714523in}{2.803059in}}%
\pgfpathlineto{\pgfqpoint{1.720185in}{2.795483in}}%
\pgfpathclose%
\pgfusepath{stroke,fill}%
\end{pgfscope}%
\begin{pgfscope}%
\pgfpathrectangle{\pgfqpoint{0.887500in}{0.275000in}}{\pgfqpoint{4.225000in}{4.225000in}}%
\pgfusepath{clip}%
\pgfsetbuttcap%
\pgfsetroundjoin%
\definecolor{currentfill}{rgb}{0.123463,0.581687,0.547445}%
\pgfsetfillcolor{currentfill}%
\pgfsetfillopacity{0.700000}%
\pgfsetlinewidth{0.501875pt}%
\definecolor{currentstroke}{rgb}{1.000000,1.000000,1.000000}%
\pgfsetstrokecolor{currentstroke}%
\pgfsetstrokeopacity{0.500000}%
\pgfsetdash{}{0pt}%
\pgfpathmoveto{\pgfqpoint{2.269580in}{2.708097in}}%
\pgfpathlineto{\pgfqpoint{2.280929in}{2.711365in}}%
\pgfpathlineto{\pgfqpoint{2.292263in}{2.715164in}}%
\pgfpathlineto{\pgfqpoint{2.303578in}{2.719679in}}%
\pgfpathlineto{\pgfqpoint{2.314871in}{2.725096in}}%
\pgfpathlineto{\pgfqpoint{2.326142in}{2.731461in}}%
\pgfpathlineto{\pgfqpoint{2.320261in}{2.739941in}}%
\pgfpathlineto{\pgfqpoint{2.314385in}{2.748366in}}%
\pgfpathlineto{\pgfqpoint{2.308514in}{2.756721in}}%
\pgfpathlineto{\pgfqpoint{2.302648in}{2.764994in}}%
\pgfpathlineto{\pgfqpoint{2.296788in}{2.773170in}}%
\pgfpathlineto{\pgfqpoint{2.285550in}{2.765782in}}%
\pgfpathlineto{\pgfqpoint{2.274284in}{2.759659in}}%
\pgfpathlineto{\pgfqpoint{2.262990in}{2.754754in}}%
\pgfpathlineto{\pgfqpoint{2.251672in}{2.750817in}}%
\pgfpathlineto{\pgfqpoint{2.240334in}{2.747596in}}%
\pgfpathlineto{\pgfqpoint{2.246174in}{2.739763in}}%
\pgfpathlineto{\pgfqpoint{2.252019in}{2.731888in}}%
\pgfpathlineto{\pgfqpoint{2.257869in}{2.723980in}}%
\pgfpathlineto{\pgfqpoint{2.263722in}{2.716047in}}%
\pgfpathclose%
\pgfusepath{stroke,fill}%
\end{pgfscope}%
\begin{pgfscope}%
\pgfpathrectangle{\pgfqpoint{0.887500in}{0.275000in}}{\pgfqpoint{4.225000in}{4.225000in}}%
\pgfusepath{clip}%
\pgfsetbuttcap%
\pgfsetroundjoin%
\definecolor{currentfill}{rgb}{0.824940,0.884720,0.106217}%
\pgfsetfillcolor{currentfill}%
\pgfsetfillopacity{0.700000}%
\pgfsetlinewidth{0.501875pt}%
\definecolor{currentstroke}{rgb}{1.000000,1.000000,1.000000}%
\pgfsetstrokecolor{currentstroke}%
\pgfsetstrokeopacity{0.500000}%
\pgfsetdash{}{0pt}%
\pgfpathmoveto{\pgfqpoint{3.223785in}{3.540470in}}%
\pgfpathlineto{\pgfqpoint{3.234908in}{3.543795in}}%
\pgfpathlineto{\pgfqpoint{3.246026in}{3.547100in}}%
\pgfpathlineto{\pgfqpoint{3.257139in}{3.550358in}}%
\pgfpathlineto{\pgfqpoint{3.268246in}{3.553541in}}%
\pgfpathlineto{\pgfqpoint{3.279347in}{3.556621in}}%
\pgfpathlineto{\pgfqpoint{3.273144in}{3.561529in}}%
\pgfpathlineto{\pgfqpoint{3.266945in}{3.566270in}}%
\pgfpathlineto{\pgfqpoint{3.260750in}{3.570851in}}%
\pgfpathlineto{\pgfqpoint{3.254560in}{3.575278in}}%
\pgfpathlineto{\pgfqpoint{3.248373in}{3.579556in}}%
\pgfpathlineto{\pgfqpoint{3.237286in}{3.576230in}}%
\pgfpathlineto{\pgfqpoint{3.226194in}{3.572967in}}%
\pgfpathlineto{\pgfqpoint{3.215097in}{3.569761in}}%
\pgfpathlineto{\pgfqpoint{3.203995in}{3.566606in}}%
\pgfpathlineto{\pgfqpoint{3.192888in}{3.563497in}}%
\pgfpathlineto{\pgfqpoint{3.199059in}{3.559436in}}%
\pgfpathlineto{\pgfqpoint{3.205234in}{3.555093in}}%
\pgfpathlineto{\pgfqpoint{3.211414in}{3.550477in}}%
\pgfpathlineto{\pgfqpoint{3.217597in}{3.545599in}}%
\pgfpathclose%
\pgfusepath{stroke,fill}%
\end{pgfscope}%
\begin{pgfscope}%
\pgfpathrectangle{\pgfqpoint{0.887500in}{0.275000in}}{\pgfqpoint{4.225000in}{4.225000in}}%
\pgfusepath{clip}%
\pgfsetbuttcap%
\pgfsetroundjoin%
\definecolor{currentfill}{rgb}{0.762373,0.876424,0.137064}%
\pgfsetfillcolor{currentfill}%
\pgfsetfillopacity{0.700000}%
\pgfsetlinewidth{0.501875pt}%
\definecolor{currentstroke}{rgb}{1.000000,1.000000,1.000000}%
\pgfsetstrokecolor{currentstroke}%
\pgfsetstrokeopacity{0.500000}%
\pgfsetdash{}{0pt}%
\pgfpathmoveto{\pgfqpoint{2.939207in}{3.505941in}}%
\pgfpathlineto{\pgfqpoint{2.950404in}{3.507873in}}%
\pgfpathlineto{\pgfqpoint{2.961597in}{3.509005in}}%
\pgfpathlineto{\pgfqpoint{2.972784in}{3.509821in}}%
\pgfpathlineto{\pgfqpoint{2.983964in}{3.510802in}}%
\pgfpathlineto{\pgfqpoint{2.995137in}{3.512429in}}%
\pgfpathlineto{\pgfqpoint{2.989051in}{3.515136in}}%
\pgfpathlineto{\pgfqpoint{2.982971in}{3.517738in}}%
\pgfpathlineto{\pgfqpoint{2.976896in}{3.520164in}}%
\pgfpathlineto{\pgfqpoint{2.970828in}{3.522346in}}%
\pgfpathlineto{\pgfqpoint{2.964766in}{3.524215in}}%
\pgfpathlineto{\pgfqpoint{2.953611in}{3.522399in}}%
\pgfpathlineto{\pgfqpoint{2.942450in}{3.520833in}}%
\pgfpathlineto{\pgfqpoint{2.931282in}{3.519185in}}%
\pgfpathlineto{\pgfqpoint{2.920110in}{3.517127in}}%
\pgfpathlineto{\pgfqpoint{2.908935in}{3.514328in}}%
\pgfpathlineto{\pgfqpoint{2.914973in}{3.513947in}}%
\pgfpathlineto{\pgfqpoint{2.921020in}{3.512790in}}%
\pgfpathlineto{\pgfqpoint{2.927075in}{3.510985in}}%
\pgfpathlineto{\pgfqpoint{2.933137in}{3.508659in}}%
\pgfpathclose%
\pgfusepath{stroke,fill}%
\end{pgfscope}%
\begin{pgfscope}%
\pgfpathrectangle{\pgfqpoint{0.887500in}{0.275000in}}{\pgfqpoint{4.225000in}{4.225000in}}%
\pgfusepath{clip}%
\pgfsetbuttcap%
\pgfsetroundjoin%
\definecolor{currentfill}{rgb}{0.783315,0.879285,0.125405}%
\pgfsetfillcolor{currentfill}%
\pgfsetfillopacity{0.700000}%
\pgfsetlinewidth{0.501875pt}%
\definecolor{currentstroke}{rgb}{1.000000,1.000000,1.000000}%
\pgfsetstrokecolor{currentstroke}%
\pgfsetstrokeopacity{0.500000}%
\pgfsetdash{}{0pt}%
\pgfpathmoveto{\pgfqpoint{3.680985in}{3.487494in}}%
\pgfpathlineto{\pgfqpoint{3.691990in}{3.490393in}}%
\pgfpathlineto{\pgfqpoint{3.702993in}{3.493510in}}%
\pgfpathlineto{\pgfqpoint{3.713996in}{3.496925in}}%
\pgfpathlineto{\pgfqpoint{3.724999in}{3.500713in}}%
\pgfpathlineto{\pgfqpoint{3.736004in}{3.504866in}}%
\pgfpathlineto{\pgfqpoint{3.729723in}{3.516012in}}%
\pgfpathlineto{\pgfqpoint{3.723444in}{3.527070in}}%
\pgfpathlineto{\pgfqpoint{3.717167in}{3.538007in}}%
\pgfpathlineto{\pgfqpoint{3.710890in}{3.548786in}}%
\pgfpathlineto{\pgfqpoint{3.704613in}{3.559373in}}%
\pgfpathlineto{\pgfqpoint{3.693613in}{3.555105in}}%
\pgfpathlineto{\pgfqpoint{3.682615in}{3.551215in}}%
\pgfpathlineto{\pgfqpoint{3.671617in}{3.547710in}}%
\pgfpathlineto{\pgfqpoint{3.660619in}{3.544509in}}%
\pgfpathlineto{\pgfqpoint{3.649619in}{3.541528in}}%
\pgfpathlineto{\pgfqpoint{3.655893in}{3.531174in}}%
\pgfpathlineto{\pgfqpoint{3.662165in}{3.520540in}}%
\pgfpathlineto{\pgfqpoint{3.668438in}{3.509679in}}%
\pgfpathlineto{\pgfqpoint{3.674711in}{3.498646in}}%
\pgfpathclose%
\pgfusepath{stroke,fill}%
\end{pgfscope}%
\begin{pgfscope}%
\pgfpathrectangle{\pgfqpoint{0.887500in}{0.275000in}}{\pgfqpoint{4.225000in}{4.225000in}}%
\pgfusepath{clip}%
\pgfsetbuttcap%
\pgfsetroundjoin%
\definecolor{currentfill}{rgb}{0.120565,0.596422,0.543611}%
\pgfsetfillcolor{currentfill}%
\pgfsetfillopacity{0.700000}%
\pgfsetlinewidth{0.501875pt}%
\definecolor{currentstroke}{rgb}{1.000000,1.000000,1.000000}%
\pgfsetstrokecolor{currentstroke}%
\pgfsetstrokeopacity{0.500000}%
\pgfsetdash{}{0pt}%
\pgfpathmoveto{\pgfqpoint{2.411982in}{2.720080in}}%
\pgfpathlineto{\pgfqpoint{2.423231in}{2.727305in}}%
\pgfpathlineto{\pgfqpoint{2.434472in}{2.734825in}}%
\pgfpathlineto{\pgfqpoint{2.445703in}{2.742757in}}%
\pgfpathlineto{\pgfqpoint{2.456924in}{2.751218in}}%
\pgfpathlineto{\pgfqpoint{2.468132in}{2.760330in}}%
\pgfpathlineto{\pgfqpoint{2.462155in}{2.772187in}}%
\pgfpathlineto{\pgfqpoint{2.456184in}{2.783810in}}%
\pgfpathlineto{\pgfqpoint{2.450220in}{2.795127in}}%
\pgfpathlineto{\pgfqpoint{2.444264in}{2.806137in}}%
\pgfpathlineto{\pgfqpoint{2.438315in}{2.816863in}}%
\pgfpathlineto{\pgfqpoint{2.427120in}{2.807641in}}%
\pgfpathlineto{\pgfqpoint{2.415919in}{2.798641in}}%
\pgfpathlineto{\pgfqpoint{2.404713in}{2.789783in}}%
\pgfpathlineto{\pgfqpoint{2.393504in}{2.780985in}}%
\pgfpathlineto{\pgfqpoint{2.382293in}{2.772169in}}%
\pgfpathlineto{\pgfqpoint{2.388221in}{2.762000in}}%
\pgfpathlineto{\pgfqpoint{2.394153in}{2.751693in}}%
\pgfpathlineto{\pgfqpoint{2.400092in}{2.741256in}}%
\pgfpathlineto{\pgfqpoint{2.406035in}{2.730705in}}%
\pgfpathclose%
\pgfusepath{stroke,fill}%
\end{pgfscope}%
\begin{pgfscope}%
\pgfpathrectangle{\pgfqpoint{0.887500in}{0.275000in}}{\pgfqpoint{4.225000in}{4.225000in}}%
\pgfusepath{clip}%
\pgfsetbuttcap%
\pgfsetroundjoin%
\definecolor{currentfill}{rgb}{0.140210,0.665859,0.513427}%
\pgfsetfillcolor{currentfill}%
\pgfsetfillopacity{0.700000}%
\pgfsetlinewidth{0.501875pt}%
\definecolor{currentstroke}{rgb}{1.000000,1.000000,1.000000}%
\pgfsetstrokecolor{currentstroke}%
\pgfsetstrokeopacity{0.500000}%
\pgfsetdash{}{0pt}%
\pgfpathmoveto{\pgfqpoint{2.665739in}{2.873484in}}%
\pgfpathlineto{\pgfqpoint{2.676919in}{2.884127in}}%
\pgfpathlineto{\pgfqpoint{2.688094in}{2.895266in}}%
\pgfpathlineto{\pgfqpoint{2.699266in}{2.906578in}}%
\pgfpathlineto{\pgfqpoint{2.710439in}{2.917600in}}%
\pgfpathlineto{\pgfqpoint{2.721618in}{2.927868in}}%
\pgfpathlineto{\pgfqpoint{2.715621in}{2.934704in}}%
\pgfpathlineto{\pgfqpoint{2.709631in}{2.941278in}}%
\pgfpathlineto{\pgfqpoint{2.703647in}{2.947657in}}%
\pgfpathlineto{\pgfqpoint{2.697669in}{2.953908in}}%
\pgfpathlineto{\pgfqpoint{2.691695in}{2.960098in}}%
\pgfpathlineto{\pgfqpoint{2.680514in}{2.952083in}}%
\pgfpathlineto{\pgfqpoint{2.669337in}{2.943344in}}%
\pgfpathlineto{\pgfqpoint{2.658160in}{2.934395in}}%
\pgfpathlineto{\pgfqpoint{2.646977in}{2.925749in}}%
\pgfpathlineto{\pgfqpoint{2.635786in}{2.917751in}}%
\pgfpathlineto{\pgfqpoint{2.641759in}{2.909910in}}%
\pgfpathlineto{\pgfqpoint{2.647741in}{2.901503in}}%
\pgfpathlineto{\pgfqpoint{2.653732in}{2.892591in}}%
\pgfpathlineto{\pgfqpoint{2.659732in}{2.883231in}}%
\pgfpathclose%
\pgfusepath{stroke,fill}%
\end{pgfscope}%
\begin{pgfscope}%
\pgfpathrectangle{\pgfqpoint{0.887500in}{0.275000in}}{\pgfqpoint{4.225000in}{4.225000in}}%
\pgfusepath{clip}%
\pgfsetbuttcap%
\pgfsetroundjoin%
\definecolor{currentfill}{rgb}{0.804182,0.882046,0.114965}%
\pgfsetfillcolor{currentfill}%
\pgfsetfillopacity{0.700000}%
\pgfsetlinewidth{0.501875pt}%
\definecolor{currentstroke}{rgb}{1.000000,1.000000,1.000000}%
\pgfsetstrokecolor{currentstroke}%
\pgfsetstrokeopacity{0.500000}%
\pgfsetdash{}{0pt}%
\pgfpathmoveto{\pgfqpoint{3.081515in}{3.522890in}}%
\pgfpathlineto{\pgfqpoint{3.092676in}{3.528651in}}%
\pgfpathlineto{\pgfqpoint{3.103832in}{3.534127in}}%
\pgfpathlineto{\pgfqpoint{3.114983in}{3.539067in}}%
\pgfpathlineto{\pgfqpoint{3.126129in}{3.543453in}}%
\pgfpathlineto{\pgfqpoint{3.137270in}{3.547379in}}%
\pgfpathlineto{\pgfqpoint{3.131119in}{3.550907in}}%
\pgfpathlineto{\pgfqpoint{3.124973in}{3.553962in}}%
\pgfpathlineto{\pgfqpoint{3.118831in}{3.556522in}}%
\pgfpathlineto{\pgfqpoint{3.112694in}{3.558623in}}%
\pgfpathlineto{\pgfqpoint{3.106563in}{3.560320in}}%
\pgfpathlineto{\pgfqpoint{3.095441in}{3.555999in}}%
\pgfpathlineto{\pgfqpoint{3.084314in}{3.551306in}}%
\pgfpathlineto{\pgfqpoint{3.073182in}{3.546156in}}%
\pgfpathlineto{\pgfqpoint{3.062046in}{3.540575in}}%
\pgfpathlineto{\pgfqpoint{3.050906in}{3.534804in}}%
\pgfpathlineto{\pgfqpoint{3.057016in}{3.532916in}}%
\pgfpathlineto{\pgfqpoint{3.063132in}{3.530797in}}%
\pgfpathlineto{\pgfqpoint{3.069254in}{3.528429in}}%
\pgfpathlineto{\pgfqpoint{3.075382in}{3.525797in}}%
\pgfpathclose%
\pgfusepath{stroke,fill}%
\end{pgfscope}%
\begin{pgfscope}%
\pgfpathrectangle{\pgfqpoint{0.887500in}{0.275000in}}{\pgfqpoint{4.225000in}{4.225000in}}%
\pgfusepath{clip}%
\pgfsetbuttcap%
\pgfsetroundjoin%
\definecolor{currentfill}{rgb}{0.120565,0.596422,0.543611}%
\pgfsetfillcolor{currentfill}%
\pgfsetfillopacity{0.700000}%
\pgfsetlinewidth{0.501875pt}%
\definecolor{currentstroke}{rgb}{1.000000,1.000000,1.000000}%
\pgfsetstrokecolor{currentstroke}%
\pgfsetstrokeopacity{0.500000}%
\pgfsetdash{}{0pt}%
\pgfpathmoveto{\pgfqpoint{2.040643in}{2.738335in}}%
\pgfpathlineto{\pgfqpoint{2.052051in}{2.741599in}}%
\pgfpathlineto{\pgfqpoint{2.063452in}{2.744906in}}%
\pgfpathlineto{\pgfqpoint{2.074846in}{2.748280in}}%
\pgfpathlineto{\pgfqpoint{2.086232in}{2.751748in}}%
\pgfpathlineto{\pgfqpoint{2.097610in}{2.755334in}}%
\pgfpathlineto{\pgfqpoint{2.091815in}{2.763236in}}%
\pgfpathlineto{\pgfqpoint{2.086024in}{2.771126in}}%
\pgfpathlineto{\pgfqpoint{2.080237in}{2.779002in}}%
\pgfpathlineto{\pgfqpoint{2.074455in}{2.786866in}}%
\pgfpathlineto{\pgfqpoint{2.068676in}{2.794718in}}%
\pgfpathlineto{\pgfqpoint{2.057312in}{2.791099in}}%
\pgfpathlineto{\pgfqpoint{2.045939in}{2.787601in}}%
\pgfpathlineto{\pgfqpoint{2.034558in}{2.784199in}}%
\pgfpathlineto{\pgfqpoint{2.023170in}{2.780866in}}%
\pgfpathlineto{\pgfqpoint{2.011776in}{2.777577in}}%
\pgfpathlineto{\pgfqpoint{2.017541in}{2.769754in}}%
\pgfpathlineto{\pgfqpoint{2.023311in}{2.761919in}}%
\pgfpathlineto{\pgfqpoint{2.029084in}{2.754071in}}%
\pgfpathlineto{\pgfqpoint{2.034862in}{2.746209in}}%
\pgfpathclose%
\pgfusepath{stroke,fill}%
\end{pgfscope}%
\begin{pgfscope}%
\pgfpathrectangle{\pgfqpoint{0.887500in}{0.275000in}}{\pgfqpoint{4.225000in}{4.225000in}}%
\pgfusepath{clip}%
\pgfsetbuttcap%
\pgfsetroundjoin%
\definecolor{currentfill}{rgb}{0.730889,0.871916,0.156029}%
\pgfsetfillcolor{currentfill}%
\pgfsetfillopacity{0.700000}%
\pgfsetlinewidth{0.501875pt}%
\definecolor{currentstroke}{rgb}{1.000000,1.000000,1.000000}%
\pgfsetstrokecolor{currentstroke}%
\pgfsetstrokeopacity{0.500000}%
\pgfsetdash{}{0pt}%
\pgfpathmoveto{\pgfqpoint{3.767454in}{3.449056in}}%
\pgfpathlineto{\pgfqpoint{3.778460in}{3.453147in}}%
\pgfpathlineto{\pgfqpoint{3.789463in}{3.457389in}}%
\pgfpathlineto{\pgfqpoint{3.800464in}{3.461733in}}%
\pgfpathlineto{\pgfqpoint{3.811461in}{3.466132in}}%
\pgfpathlineto{\pgfqpoint{3.822453in}{3.470538in}}%
\pgfpathlineto{\pgfqpoint{3.816160in}{3.482217in}}%
\pgfpathlineto{\pgfqpoint{3.809869in}{3.493878in}}%
\pgfpathlineto{\pgfqpoint{3.803581in}{3.505498in}}%
\pgfpathlineto{\pgfqpoint{3.797294in}{3.517054in}}%
\pgfpathlineto{\pgfqpoint{3.791009in}{3.528522in}}%
\pgfpathlineto{\pgfqpoint{3.780015in}{3.523655in}}%
\pgfpathlineto{\pgfqpoint{3.769015in}{3.518781in}}%
\pgfpathlineto{\pgfqpoint{3.758013in}{3.513975in}}%
\pgfpathlineto{\pgfqpoint{3.747009in}{3.509312in}}%
\pgfpathlineto{\pgfqpoint{3.736004in}{3.504866in}}%
\pgfpathlineto{\pgfqpoint{3.742287in}{3.493669in}}%
\pgfpathlineto{\pgfqpoint{3.748573in}{3.482454in}}%
\pgfpathlineto{\pgfqpoint{3.754862in}{3.471258in}}%
\pgfpathlineto{\pgfqpoint{3.761156in}{3.460114in}}%
\pgfpathclose%
\pgfusepath{stroke,fill}%
\end{pgfscope}%
\begin{pgfscope}%
\pgfpathrectangle{\pgfqpoint{0.887500in}{0.275000in}}{\pgfqpoint{4.225000in}{4.225000in}}%
\pgfusepath{clip}%
\pgfsetbuttcap%
\pgfsetroundjoin%
\definecolor{currentfill}{rgb}{0.121380,0.629492,0.531973}%
\pgfsetfillcolor{currentfill}%
\pgfsetfillopacity{0.700000}%
\pgfsetlinewidth{0.501875pt}%
\definecolor{currentstroke}{rgb}{1.000000,1.000000,1.000000}%
\pgfsetstrokecolor{currentstroke}%
\pgfsetstrokeopacity{0.500000}%
\pgfsetdash{}{0pt}%
\pgfpathmoveto{\pgfqpoint{1.496947in}{2.813175in}}%
\pgfpathlineto{\pgfqpoint{1.508482in}{2.816510in}}%
\pgfpathlineto{\pgfqpoint{1.520010in}{2.819845in}}%
\pgfpathlineto{\pgfqpoint{1.531533in}{2.823179in}}%
\pgfpathlineto{\pgfqpoint{1.543051in}{2.826512in}}%
\pgfpathlineto{\pgfqpoint{1.554563in}{2.829844in}}%
\pgfpathlineto{\pgfqpoint{1.548969in}{2.837220in}}%
\pgfpathlineto{\pgfqpoint{1.543379in}{2.844576in}}%
\pgfpathlineto{\pgfqpoint{1.537793in}{2.851911in}}%
\pgfpathlineto{\pgfqpoint{1.532212in}{2.859226in}}%
\pgfpathlineto{\pgfqpoint{1.520712in}{2.855881in}}%
\pgfpathlineto{\pgfqpoint{1.509205in}{2.852541in}}%
\pgfpathlineto{\pgfqpoint{1.497693in}{2.849203in}}%
\pgfpathlineto{\pgfqpoint{1.486176in}{2.845865in}}%
\pgfpathlineto{\pgfqpoint{1.474653in}{2.842526in}}%
\pgfpathlineto{\pgfqpoint{1.480220in}{2.835217in}}%
\pgfpathlineto{\pgfqpoint{1.485791in}{2.827888in}}%
\pgfpathlineto{\pgfqpoint{1.491367in}{2.820541in}}%
\pgfpathclose%
\pgfusepath{stroke,fill}%
\end{pgfscope}%
\begin{pgfscope}%
\pgfpathrectangle{\pgfqpoint{0.887500in}{0.275000in}}{\pgfqpoint{4.225000in}{4.225000in}}%
\pgfusepath{clip}%
\pgfsetbuttcap%
\pgfsetroundjoin%
\definecolor{currentfill}{rgb}{0.688944,0.865448,0.182725}%
\pgfsetfillcolor{currentfill}%
\pgfsetfillopacity{0.700000}%
\pgfsetlinewidth{0.501875pt}%
\definecolor{currentstroke}{rgb}{1.000000,1.000000,1.000000}%
\pgfsetstrokecolor{currentstroke}%
\pgfsetstrokeopacity{0.500000}%
\pgfsetdash{}{0pt}%
\pgfpathmoveto{\pgfqpoint{3.853966in}{3.412120in}}%
\pgfpathlineto{\pgfqpoint{3.864945in}{3.415779in}}%
\pgfpathlineto{\pgfqpoint{3.875920in}{3.419483in}}%
\pgfpathlineto{\pgfqpoint{3.886891in}{3.423235in}}%
\pgfpathlineto{\pgfqpoint{3.897859in}{3.427036in}}%
\pgfpathlineto{\pgfqpoint{3.908822in}{3.430891in}}%
\pgfpathlineto{\pgfqpoint{3.902520in}{3.443126in}}%
\pgfpathlineto{\pgfqpoint{3.896220in}{3.455369in}}%
\pgfpathlineto{\pgfqpoint{3.889923in}{3.467604in}}%
\pgfpathlineto{\pgfqpoint{3.883629in}{3.479816in}}%
\pgfpathlineto{\pgfqpoint{3.877337in}{3.491986in}}%
\pgfpathlineto{\pgfqpoint{3.866371in}{3.487763in}}%
\pgfpathlineto{\pgfqpoint{3.855399in}{3.483513in}}%
\pgfpathlineto{\pgfqpoint{3.844423in}{3.479231in}}%
\pgfpathlineto{\pgfqpoint{3.833441in}{3.474908in}}%
\pgfpathlineto{\pgfqpoint{3.822453in}{3.470538in}}%
\pgfpathlineto{\pgfqpoint{3.828750in}{3.458852in}}%
\pgfpathlineto{\pgfqpoint{3.835049in}{3.447162in}}%
\pgfpathlineto{\pgfqpoint{3.841351in}{3.435474in}}%
\pgfpathlineto{\pgfqpoint{3.847657in}{3.423792in}}%
\pgfpathclose%
\pgfusepath{stroke,fill}%
\end{pgfscope}%
\begin{pgfscope}%
\pgfpathrectangle{\pgfqpoint{0.887500in}{0.275000in}}{\pgfqpoint{4.225000in}{4.225000in}}%
\pgfusepath{clip}%
\pgfsetbuttcap%
\pgfsetroundjoin%
\definecolor{currentfill}{rgb}{0.626579,0.854645,0.223353}%
\pgfsetfillcolor{currentfill}%
\pgfsetfillopacity{0.700000}%
\pgfsetlinewidth{0.501875pt}%
\definecolor{currentstroke}{rgb}{1.000000,1.000000,1.000000}%
\pgfsetstrokecolor{currentstroke}%
\pgfsetstrokeopacity{0.500000}%
\pgfsetdash{}{0pt}%
\pgfpathmoveto{\pgfqpoint{3.940394in}{3.370425in}}%
\pgfpathlineto{\pgfqpoint{3.951351in}{3.373974in}}%
\pgfpathlineto{\pgfqpoint{3.962305in}{3.377612in}}%
\pgfpathlineto{\pgfqpoint{3.973255in}{3.381341in}}%
\pgfpathlineto{\pgfqpoint{3.984203in}{3.385158in}}%
\pgfpathlineto{\pgfqpoint{3.995149in}{3.389064in}}%
\pgfpathlineto{\pgfqpoint{3.988829in}{3.401411in}}%
\pgfpathlineto{\pgfqpoint{3.982514in}{3.413794in}}%
\pgfpathlineto{\pgfqpoint{3.976202in}{3.426203in}}%
\pgfpathlineto{\pgfqpoint{3.969893in}{3.438629in}}%
\pgfpathlineto{\pgfqpoint{3.963587in}{3.451061in}}%
\pgfpathlineto{\pgfqpoint{3.952641in}{3.446899in}}%
\pgfpathlineto{\pgfqpoint{3.941691in}{3.442803in}}%
\pgfpathlineto{\pgfqpoint{3.930738in}{3.438772in}}%
\pgfpathlineto{\pgfqpoint{3.919782in}{3.434802in}}%
\pgfpathlineto{\pgfqpoint{3.908822in}{3.430891in}}%
\pgfpathlineto{\pgfqpoint{3.915128in}{3.418682in}}%
\pgfpathlineto{\pgfqpoint{3.921438in}{3.406514in}}%
\pgfpathlineto{\pgfqpoint{3.927752in}{3.394405in}}%
\pgfpathlineto{\pgfqpoint{3.934070in}{3.382370in}}%
\pgfpathclose%
\pgfusepath{stroke,fill}%
\end{pgfscope}%
\begin{pgfscope}%
\pgfpathrectangle{\pgfqpoint{0.887500in}{0.275000in}}{\pgfqpoint{4.225000in}{4.225000in}}%
\pgfusepath{clip}%
\pgfsetbuttcap%
\pgfsetroundjoin%
\definecolor{currentfill}{rgb}{0.824940,0.884720,0.106217}%
\pgfsetfillcolor{currentfill}%
\pgfsetfillopacity{0.700000}%
\pgfsetlinewidth{0.501875pt}%
\definecolor{currentstroke}{rgb}{1.000000,1.000000,1.000000}%
\pgfsetstrokecolor{currentstroke}%
\pgfsetstrokeopacity{0.500000}%
\pgfsetdash{}{0pt}%
\pgfpathmoveto{\pgfqpoint{3.452585in}{3.530009in}}%
\pgfpathlineto{\pgfqpoint{3.463662in}{3.533902in}}%
\pgfpathlineto{\pgfqpoint{3.474736in}{3.537890in}}%
\pgfpathlineto{\pgfqpoint{3.485805in}{3.541939in}}%
\pgfpathlineto{\pgfqpoint{3.496870in}{3.546018in}}%
\pgfpathlineto{\pgfqpoint{3.507930in}{3.550093in}}%
\pgfpathlineto{\pgfqpoint{3.501653in}{3.556806in}}%
\pgfpathlineto{\pgfqpoint{3.495378in}{3.563267in}}%
\pgfpathlineto{\pgfqpoint{3.489106in}{3.569557in}}%
\pgfpathlineto{\pgfqpoint{3.482838in}{3.575760in}}%
\pgfpathlineto{\pgfqpoint{3.476574in}{3.581954in}}%
\pgfpathlineto{\pgfqpoint{3.465520in}{3.577236in}}%
\pgfpathlineto{\pgfqpoint{3.454463in}{3.572531in}}%
\pgfpathlineto{\pgfqpoint{3.443401in}{3.567906in}}%
\pgfpathlineto{\pgfqpoint{3.432337in}{3.563429in}}%
\pgfpathlineto{\pgfqpoint{3.421270in}{3.559167in}}%
\pgfpathlineto{\pgfqpoint{3.427525in}{3.553521in}}%
\pgfpathlineto{\pgfqpoint{3.433785in}{3.547869in}}%
\pgfpathlineto{\pgfqpoint{3.440049in}{3.542125in}}%
\pgfpathlineto{\pgfqpoint{3.446316in}{3.536201in}}%
\pgfpathclose%
\pgfusepath{stroke,fill}%
\end{pgfscope}%
\begin{pgfscope}%
\pgfpathrectangle{\pgfqpoint{0.887500in}{0.275000in}}{\pgfqpoint{4.225000in}{4.225000in}}%
\pgfusepath{clip}%
\pgfsetbuttcap%
\pgfsetroundjoin%
\definecolor{currentfill}{rgb}{0.119512,0.607464,0.540218}%
\pgfsetfillcolor{currentfill}%
\pgfsetfillopacity{0.700000}%
\pgfsetlinewidth{0.501875pt}%
\definecolor{currentstroke}{rgb}{1.000000,1.000000,1.000000}%
\pgfsetstrokecolor{currentstroke}%
\pgfsetstrokeopacity{0.500000}%
\pgfsetdash{}{0pt}%
\pgfpathmoveto{\pgfqpoint{1.811650in}{2.766171in}}%
\pgfpathlineto{\pgfqpoint{1.823113in}{2.769490in}}%
\pgfpathlineto{\pgfqpoint{1.834571in}{2.772821in}}%
\pgfpathlineto{\pgfqpoint{1.846023in}{2.776163in}}%
\pgfpathlineto{\pgfqpoint{1.857469in}{2.779517in}}%
\pgfpathlineto{\pgfqpoint{1.868909in}{2.782882in}}%
\pgfpathlineto{\pgfqpoint{1.863195in}{2.790581in}}%
\pgfpathlineto{\pgfqpoint{1.857485in}{2.798267in}}%
\pgfpathlineto{\pgfqpoint{1.851779in}{2.805940in}}%
\pgfpathlineto{\pgfqpoint{1.846077in}{2.813601in}}%
\pgfpathlineto{\pgfqpoint{1.840380in}{2.821248in}}%
\pgfpathlineto{\pgfqpoint{1.828953in}{2.817861in}}%
\pgfpathlineto{\pgfqpoint{1.817521in}{2.814486in}}%
\pgfpathlineto{\pgfqpoint{1.806083in}{2.811123in}}%
\pgfpathlineto{\pgfqpoint{1.794639in}{2.807771in}}%
\pgfpathlineto{\pgfqpoint{1.783189in}{2.804432in}}%
\pgfpathlineto{\pgfqpoint{1.788873in}{2.796807in}}%
\pgfpathlineto{\pgfqpoint{1.794561in}{2.789169in}}%
\pgfpathlineto{\pgfqpoint{1.800253in}{2.781516in}}%
\pgfpathlineto{\pgfqpoint{1.805949in}{2.773850in}}%
\pgfpathclose%
\pgfusepath{stroke,fill}%
\end{pgfscope}%
\begin{pgfscope}%
\pgfpathrectangle{\pgfqpoint{0.887500in}{0.275000in}}{\pgfqpoint{4.225000in}{4.225000in}}%
\pgfusepath{clip}%
\pgfsetbuttcap%
\pgfsetroundjoin%
\definecolor{currentfill}{rgb}{0.575563,0.844566,0.256415}%
\pgfsetfillcolor{currentfill}%
\pgfsetfillopacity{0.700000}%
\pgfsetlinewidth{0.501875pt}%
\definecolor{currentstroke}{rgb}{1.000000,1.000000,1.000000}%
\pgfsetstrokecolor{currentstroke}%
\pgfsetstrokeopacity{0.500000}%
\pgfsetdash{}{0pt}%
\pgfpathmoveto{\pgfqpoint{4.026802in}{3.327902in}}%
\pgfpathlineto{\pgfqpoint{4.037732in}{3.331209in}}%
\pgfpathlineto{\pgfqpoint{4.048658in}{3.334541in}}%
\pgfpathlineto{\pgfqpoint{4.059580in}{3.337898in}}%
\pgfpathlineto{\pgfqpoint{4.070497in}{3.341277in}}%
\pgfpathlineto{\pgfqpoint{4.064173in}{3.354150in}}%
\pgfpathlineto{\pgfqpoint{4.057852in}{3.367039in}}%
\pgfpathlineto{\pgfqpoint{4.051534in}{3.379920in}}%
\pgfpathlineto{\pgfqpoint{4.045218in}{3.392770in}}%
\pgfpathlineto{\pgfqpoint{4.038903in}{3.405566in}}%
\pgfpathlineto{\pgfqpoint{4.027968in}{3.401310in}}%
\pgfpathlineto{\pgfqpoint{4.017031in}{3.397141in}}%
\pgfpathlineto{\pgfqpoint{4.006091in}{3.393058in}}%
\pgfpathlineto{\pgfqpoint{3.995149in}{3.389064in}}%
\pgfpathlineto{\pgfqpoint{4.001472in}{3.376760in}}%
\pgfpathlineto{\pgfqpoint{4.007799in}{3.364496in}}%
\pgfpathlineto{\pgfqpoint{4.014129in}{3.352267in}}%
\pgfpathlineto{\pgfqpoint{4.020464in}{3.340071in}}%
\pgfpathclose%
\pgfusepath{stroke,fill}%
\end{pgfscope}%
\begin{pgfscope}%
\pgfpathrectangle{\pgfqpoint{0.887500in}{0.275000in}}{\pgfqpoint{4.225000in}{4.225000in}}%
\pgfusepath{clip}%
\pgfsetbuttcap%
\pgfsetroundjoin%
\definecolor{currentfill}{rgb}{0.128087,0.647749,0.523491}%
\pgfsetfillcolor{currentfill}%
\pgfsetfillopacity{0.700000}%
\pgfsetlinewidth{0.501875pt}%
\definecolor{currentstroke}{rgb}{1.000000,1.000000,1.000000}%
\pgfsetstrokecolor{currentstroke}%
\pgfsetstrokeopacity{0.500000}%
\pgfsetdash{}{0pt}%
\pgfpathmoveto{\pgfqpoint{2.609807in}{2.821192in}}%
\pgfpathlineto{\pgfqpoint{2.620989in}{2.832281in}}%
\pgfpathlineto{\pgfqpoint{2.632176in}{2.842829in}}%
\pgfpathlineto{\pgfqpoint{2.643365in}{2.853061in}}%
\pgfpathlineto{\pgfqpoint{2.654553in}{2.863204in}}%
\pgfpathlineto{\pgfqpoint{2.665739in}{2.873484in}}%
\pgfpathlineto{\pgfqpoint{2.659732in}{2.883231in}}%
\pgfpathlineto{\pgfqpoint{2.653732in}{2.892591in}}%
\pgfpathlineto{\pgfqpoint{2.647741in}{2.901503in}}%
\pgfpathlineto{\pgfqpoint{2.641759in}{2.909910in}}%
\pgfpathlineto{\pgfqpoint{2.635786in}{2.917751in}}%
\pgfpathlineto{\pgfqpoint{2.624586in}{2.910197in}}%
\pgfpathlineto{\pgfqpoint{2.613383in}{2.902770in}}%
\pgfpathlineto{\pgfqpoint{2.602178in}{2.895153in}}%
\pgfpathlineto{\pgfqpoint{2.590976in}{2.887031in}}%
\pgfpathlineto{\pgfqpoint{2.579782in}{2.878088in}}%
\pgfpathlineto{\pgfqpoint{2.585769in}{2.867746in}}%
\pgfpathlineto{\pgfqpoint{2.591767in}{2.856718in}}%
\pgfpathlineto{\pgfqpoint{2.597775in}{2.845171in}}%
\pgfpathlineto{\pgfqpoint{2.603789in}{2.833274in}}%
\pgfpathclose%
\pgfusepath{stroke,fill}%
\end{pgfscope}%
\begin{pgfscope}%
\pgfpathrectangle{\pgfqpoint{0.887500in}{0.275000in}}{\pgfqpoint{4.225000in}{4.225000in}}%
\pgfusepath{clip}%
\pgfsetbuttcap%
\pgfsetroundjoin%
\definecolor{currentfill}{rgb}{0.120081,0.622161,0.534946}%
\pgfsetfillcolor{currentfill}%
\pgfsetfillopacity{0.700000}%
\pgfsetlinewidth{0.501875pt}%
\definecolor{currentstroke}{rgb}{1.000000,1.000000,1.000000}%
\pgfsetstrokecolor{currentstroke}%
\pgfsetstrokeopacity{0.500000}%
\pgfsetdash{}{0pt}%
\pgfpathmoveto{\pgfqpoint{1.582601in}{2.792671in}}%
\pgfpathlineto{\pgfqpoint{1.594122in}{2.795982in}}%
\pgfpathlineto{\pgfqpoint{1.605638in}{2.799289in}}%
\pgfpathlineto{\pgfqpoint{1.617148in}{2.802592in}}%
\pgfpathlineto{\pgfqpoint{1.628653in}{2.805892in}}%
\pgfpathlineto{\pgfqpoint{1.640152in}{2.809191in}}%
\pgfpathlineto{\pgfqpoint{1.634521in}{2.816685in}}%
\pgfpathlineto{\pgfqpoint{1.628894in}{2.824163in}}%
\pgfpathlineto{\pgfqpoint{1.623272in}{2.831623in}}%
\pgfpathlineto{\pgfqpoint{1.617654in}{2.839067in}}%
\pgfpathlineto{\pgfqpoint{1.612041in}{2.846494in}}%
\pgfpathlineto{\pgfqpoint{1.600556in}{2.843164in}}%
\pgfpathlineto{\pgfqpoint{1.589066in}{2.839835in}}%
\pgfpathlineto{\pgfqpoint{1.577571in}{2.836505in}}%
\pgfpathlineto{\pgfqpoint{1.566070in}{2.833175in}}%
\pgfpathlineto{\pgfqpoint{1.554563in}{2.829844in}}%
\pgfpathlineto{\pgfqpoint{1.560162in}{2.822448in}}%
\pgfpathlineto{\pgfqpoint{1.565765in}{2.815032in}}%
\pgfpathlineto{\pgfqpoint{1.571373in}{2.807597in}}%
\pgfpathlineto{\pgfqpoint{1.576985in}{2.800143in}}%
\pgfpathclose%
\pgfusepath{stroke,fill}%
\end{pgfscope}%
\begin{pgfscope}%
\pgfpathrectangle{\pgfqpoint{0.887500in}{0.275000in}}{\pgfqpoint{4.225000in}{4.225000in}}%
\pgfusepath{clip}%
\pgfsetbuttcap%
\pgfsetroundjoin%
\definecolor{currentfill}{rgb}{0.121831,0.589055,0.545623}%
\pgfsetfillcolor{currentfill}%
\pgfsetfillopacity{0.700000}%
\pgfsetlinewidth{0.501875pt}%
\definecolor{currentstroke}{rgb}{1.000000,1.000000,1.000000}%
\pgfsetstrokecolor{currentstroke}%
\pgfsetstrokeopacity{0.500000}%
\pgfsetdash{}{0pt}%
\pgfpathmoveto{\pgfqpoint{2.126644in}{2.715633in}}%
\pgfpathlineto{\pgfqpoint{2.138028in}{2.719281in}}%
\pgfpathlineto{\pgfqpoint{2.149405in}{2.722977in}}%
\pgfpathlineto{\pgfqpoint{2.160778in}{2.726650in}}%
\pgfpathlineto{\pgfqpoint{2.172147in}{2.730228in}}%
\pgfpathlineto{\pgfqpoint{2.183514in}{2.733642in}}%
\pgfpathlineto{\pgfqpoint{2.177686in}{2.741639in}}%
\pgfpathlineto{\pgfqpoint{2.171862in}{2.749612in}}%
\pgfpathlineto{\pgfqpoint{2.166043in}{2.757561in}}%
\pgfpathlineto{\pgfqpoint{2.160227in}{2.765485in}}%
\pgfpathlineto{\pgfqpoint{2.154416in}{2.773391in}}%
\pgfpathlineto{\pgfqpoint{2.143059in}{2.770048in}}%
\pgfpathlineto{\pgfqpoint{2.131702in}{2.766482in}}%
\pgfpathlineto{\pgfqpoint{2.120343in}{2.762783in}}%
\pgfpathlineto{\pgfqpoint{2.108979in}{2.759037in}}%
\pgfpathlineto{\pgfqpoint{2.097610in}{2.755334in}}%
\pgfpathlineto{\pgfqpoint{2.103409in}{2.747419in}}%
\pgfpathlineto{\pgfqpoint{2.109211in}{2.739491in}}%
\pgfpathlineto{\pgfqpoint{2.115018in}{2.731551in}}%
\pgfpathlineto{\pgfqpoint{2.120829in}{2.723598in}}%
\pgfpathclose%
\pgfusepath{stroke,fill}%
\end{pgfscope}%
\begin{pgfscope}%
\pgfpathrectangle{\pgfqpoint{0.887500in}{0.275000in}}{\pgfqpoint{4.225000in}{4.225000in}}%
\pgfusepath{clip}%
\pgfsetbuttcap%
\pgfsetroundjoin%
\definecolor{currentfill}{rgb}{0.124395,0.578002,0.548287}%
\pgfsetfillcolor{currentfill}%
\pgfsetfillopacity{0.700000}%
\pgfsetlinewidth{0.501875pt}%
\definecolor{currentstroke}{rgb}{1.000000,1.000000,1.000000}%
\pgfsetstrokecolor{currentstroke}%
\pgfsetstrokeopacity{0.500000}%
\pgfsetdash{}{0pt}%
\pgfpathmoveto{\pgfqpoint{2.355607in}{2.688706in}}%
\pgfpathlineto{\pgfqpoint{2.366903in}{2.694193in}}%
\pgfpathlineto{\pgfqpoint{2.378188in}{2.700121in}}%
\pgfpathlineto{\pgfqpoint{2.389462in}{2.706442in}}%
\pgfpathlineto{\pgfqpoint{2.400726in}{2.713111in}}%
\pgfpathlineto{\pgfqpoint{2.411982in}{2.720080in}}%
\pgfpathlineto{\pgfqpoint{2.406035in}{2.730705in}}%
\pgfpathlineto{\pgfqpoint{2.400092in}{2.741256in}}%
\pgfpathlineto{\pgfqpoint{2.394153in}{2.751693in}}%
\pgfpathlineto{\pgfqpoint{2.388221in}{2.762000in}}%
\pgfpathlineto{\pgfqpoint{2.382293in}{2.772169in}}%
\pgfpathlineto{\pgfqpoint{2.371080in}{2.763371in}}%
\pgfpathlineto{\pgfqpoint{2.359861in}{2.754742in}}%
\pgfpathlineto{\pgfqpoint{2.348634in}{2.746442in}}%
\pgfpathlineto{\pgfqpoint{2.337395in}{2.738629in}}%
\pgfpathlineto{\pgfqpoint{2.326142in}{2.731461in}}%
\pgfpathlineto{\pgfqpoint{2.332027in}{2.722940in}}%
\pgfpathlineto{\pgfqpoint{2.337916in}{2.714392in}}%
\pgfpathlineto{\pgfqpoint{2.343810in}{2.705830in}}%
\pgfpathlineto{\pgfqpoint{2.349706in}{2.697266in}}%
\pgfpathclose%
\pgfusepath{stroke,fill}%
\end{pgfscope}%
\begin{pgfscope}%
\pgfpathrectangle{\pgfqpoint{0.887500in}{0.275000in}}{\pgfqpoint{4.225000in}{4.225000in}}%
\pgfusepath{clip}%
\pgfsetbuttcap%
\pgfsetroundjoin%
\definecolor{currentfill}{rgb}{0.647257,0.858400,0.209861}%
\pgfsetfillcolor{currentfill}%
\pgfsetfillopacity{0.700000}%
\pgfsetlinewidth{0.501875pt}%
\definecolor{currentstroke}{rgb}{1.000000,1.000000,1.000000}%
\pgfsetstrokecolor{currentstroke}%
\pgfsetstrokeopacity{0.500000}%
\pgfsetdash{}{0pt}%
\pgfpathmoveto{\pgfqpoint{2.827854in}{3.318601in}}%
\pgfpathlineto{\pgfqpoint{2.838856in}{3.359250in}}%
\pgfpathlineto{\pgfqpoint{2.849901in}{3.394733in}}%
\pgfpathlineto{\pgfqpoint{2.860988in}{3.423970in}}%
\pgfpathlineto{\pgfqpoint{2.872110in}{3.447488in}}%
\pgfpathlineto{\pgfqpoint{2.883260in}{3.465950in}}%
\pgfpathlineto{\pgfqpoint{2.877207in}{3.469862in}}%
\pgfpathlineto{\pgfqpoint{2.871162in}{3.473193in}}%
\pgfpathlineto{\pgfqpoint{2.865127in}{3.475742in}}%
\pgfpathlineto{\pgfqpoint{2.859101in}{3.477311in}}%
\pgfpathlineto{\pgfqpoint{2.853088in}{3.477697in}}%
\pgfpathlineto{\pgfqpoint{2.841945in}{3.463559in}}%
\pgfpathlineto{\pgfqpoint{2.830820in}{3.446396in}}%
\pgfpathlineto{\pgfqpoint{2.819717in}{3.425888in}}%
\pgfpathlineto{\pgfqpoint{2.808641in}{3.401776in}}%
\pgfpathlineto{\pgfqpoint{2.797590in}{3.374509in}}%
\pgfpathlineto{\pgfqpoint{2.803624in}{3.365409in}}%
\pgfpathlineto{\pgfqpoint{2.809671in}{3.354798in}}%
\pgfpathlineto{\pgfqpoint{2.815727in}{3.343146in}}%
\pgfpathlineto{\pgfqpoint{2.821789in}{3.330924in}}%
\pgfpathclose%
\pgfusepath{stroke,fill}%
\end{pgfscope}%
\begin{pgfscope}%
\pgfpathrectangle{\pgfqpoint{0.887500in}{0.275000in}}{\pgfqpoint{4.225000in}{4.225000in}}%
\pgfusepath{clip}%
\pgfsetbuttcap%
\pgfsetroundjoin%
\definecolor{currentfill}{rgb}{0.824940,0.884720,0.106217}%
\pgfsetfillcolor{currentfill}%
\pgfsetfillopacity{0.700000}%
\pgfsetlinewidth{0.501875pt}%
\definecolor{currentstroke}{rgb}{1.000000,1.000000,1.000000}%
\pgfsetstrokecolor{currentstroke}%
\pgfsetstrokeopacity{0.500000}%
\pgfsetdash{}{0pt}%
\pgfpathmoveto{\pgfqpoint{3.310421in}{3.529366in}}%
\pgfpathlineto{\pgfqpoint{3.321531in}{3.532401in}}%
\pgfpathlineto{\pgfqpoint{3.332633in}{3.535168in}}%
\pgfpathlineto{\pgfqpoint{3.343728in}{3.537749in}}%
\pgfpathlineto{\pgfqpoint{3.354818in}{3.540249in}}%
\pgfpathlineto{\pgfqpoint{3.365901in}{3.542775in}}%
\pgfpathlineto{\pgfqpoint{3.359664in}{3.548355in}}%
\pgfpathlineto{\pgfqpoint{3.353432in}{3.553937in}}%
\pgfpathlineto{\pgfqpoint{3.347205in}{3.559546in}}%
\pgfpathlineto{\pgfqpoint{3.340982in}{3.565162in}}%
\pgfpathlineto{\pgfqpoint{3.334764in}{3.570759in}}%
\pgfpathlineto{\pgfqpoint{3.323692in}{3.567890in}}%
\pgfpathlineto{\pgfqpoint{3.312614in}{3.565127in}}%
\pgfpathlineto{\pgfqpoint{3.301531in}{3.562382in}}%
\pgfpathlineto{\pgfqpoint{3.290442in}{3.559570in}}%
\pgfpathlineto{\pgfqpoint{3.279347in}{3.556621in}}%
\pgfpathlineto{\pgfqpoint{3.285554in}{3.551541in}}%
\pgfpathlineto{\pgfqpoint{3.291766in}{3.546283in}}%
\pgfpathlineto{\pgfqpoint{3.297980in}{3.540841in}}%
\pgfpathlineto{\pgfqpoint{3.304199in}{3.535207in}}%
\pgfpathclose%
\pgfusepath{stroke,fill}%
\end{pgfscope}%
\begin{pgfscope}%
\pgfpathrectangle{\pgfqpoint{0.887500in}{0.275000in}}{\pgfqpoint{4.225000in}{4.225000in}}%
\pgfusepath{clip}%
\pgfsetbuttcap%
\pgfsetroundjoin%
\definecolor{currentfill}{rgb}{0.311925,0.767822,0.415586}%
\pgfsetfillcolor{currentfill}%
\pgfsetfillopacity{0.700000}%
\pgfsetlinewidth{0.501875pt}%
\definecolor{currentstroke}{rgb}{1.000000,1.000000,1.000000}%
\pgfsetstrokecolor{currentstroke}%
\pgfsetstrokeopacity{0.500000}%
\pgfsetdash{}{0pt}%
\pgfpathmoveto{\pgfqpoint{2.833364in}{3.023849in}}%
\pgfpathlineto{\pgfqpoint{2.844384in}{3.060687in}}%
\pgfpathlineto{\pgfqpoint{2.855382in}{3.104684in}}%
\pgfpathlineto{\pgfqpoint{2.866375in}{3.153273in}}%
\pgfpathlineto{\pgfqpoint{2.877379in}{3.203872in}}%
\pgfpathlineto{\pgfqpoint{2.888407in}{3.253877in}}%
\pgfpathlineto{\pgfqpoint{2.882352in}{3.255759in}}%
\pgfpathlineto{\pgfqpoint{2.876300in}{3.258215in}}%
\pgfpathlineto{\pgfqpoint{2.870250in}{3.261434in}}%
\pgfpathlineto{\pgfqpoint{2.864201in}{3.265606in}}%
\pgfpathlineto{\pgfqpoint{2.858151in}{3.270921in}}%
\pgfpathlineto{\pgfqpoint{2.847183in}{3.219413in}}%
\pgfpathlineto{\pgfqpoint{2.836241in}{3.167362in}}%
\pgfpathlineto{\pgfqpoint{2.825307in}{3.117552in}}%
\pgfpathlineto{\pgfqpoint{2.814363in}{3.072747in}}%
\pgfpathlineto{\pgfqpoint{2.803388in}{3.035691in}}%
\pgfpathlineto{\pgfqpoint{2.809397in}{3.029220in}}%
\pgfpathlineto{\pgfqpoint{2.815396in}{3.025164in}}%
\pgfpathlineto{\pgfqpoint{2.821389in}{3.023158in}}%
\pgfpathlineto{\pgfqpoint{2.827377in}{3.022841in}}%
\pgfpathclose%
\pgfusepath{stroke,fill}%
\end{pgfscope}%
\begin{pgfscope}%
\pgfpathrectangle{\pgfqpoint{0.887500in}{0.275000in}}{\pgfqpoint{4.225000in}{4.225000in}}%
\pgfusepath{clip}%
\pgfsetbuttcap%
\pgfsetroundjoin%
\definecolor{currentfill}{rgb}{0.120638,0.625828,0.533488}%
\pgfsetfillcolor{currentfill}%
\pgfsetfillopacity{0.700000}%
\pgfsetlinewidth{0.501875pt}%
\definecolor{currentstroke}{rgb}{1.000000,1.000000,1.000000}%
\pgfsetstrokecolor{currentstroke}%
\pgfsetstrokeopacity{0.500000}%
\pgfsetdash{}{0pt}%
\pgfpathmoveto{\pgfqpoint{2.553999in}{2.756468in}}%
\pgfpathlineto{\pgfqpoint{2.565154in}{2.769896in}}%
\pgfpathlineto{\pgfqpoint{2.576310in}{2.783394in}}%
\pgfpathlineto{\pgfqpoint{2.587468in}{2.796646in}}%
\pgfpathlineto{\pgfqpoint{2.598633in}{2.809336in}}%
\pgfpathlineto{\pgfqpoint{2.609807in}{2.821192in}}%
\pgfpathlineto{\pgfqpoint{2.603789in}{2.833274in}}%
\pgfpathlineto{\pgfqpoint{2.597775in}{2.845171in}}%
\pgfpathlineto{\pgfqpoint{2.591767in}{2.856718in}}%
\pgfpathlineto{\pgfqpoint{2.585769in}{2.867746in}}%
\pgfpathlineto{\pgfqpoint{2.579782in}{2.878088in}}%
\pgfpathlineto{\pgfqpoint{2.568599in}{2.868009in}}%
\pgfpathlineto{\pgfqpoint{2.557430in}{2.856671in}}%
\pgfpathlineto{\pgfqpoint{2.546273in}{2.844376in}}%
\pgfpathlineto{\pgfqpoint{2.535123in}{2.831480in}}%
\pgfpathlineto{\pgfqpoint{2.523977in}{2.818337in}}%
\pgfpathlineto{\pgfqpoint{2.529970in}{2.806423in}}%
\pgfpathlineto{\pgfqpoint{2.535971in}{2.794131in}}%
\pgfpathlineto{\pgfqpoint{2.541977in}{2.781609in}}%
\pgfpathlineto{\pgfqpoint{2.547987in}{2.769005in}}%
\pgfpathclose%
\pgfusepath{stroke,fill}%
\end{pgfscope}%
\begin{pgfscope}%
\pgfpathrectangle{\pgfqpoint{0.887500in}{0.275000in}}{\pgfqpoint{4.225000in}{4.225000in}}%
\pgfusepath{clip}%
\pgfsetbuttcap%
\pgfsetroundjoin%
\definecolor{currentfill}{rgb}{0.120092,0.600104,0.542530}%
\pgfsetfillcolor{currentfill}%
\pgfsetfillopacity{0.700000}%
\pgfsetlinewidth{0.501875pt}%
\definecolor{currentstroke}{rgb}{1.000000,1.000000,1.000000}%
\pgfsetstrokecolor{currentstroke}%
\pgfsetstrokeopacity{0.500000}%
\pgfsetdash{}{0pt}%
\pgfpathmoveto{\pgfqpoint{1.897540in}{2.744210in}}%
\pgfpathlineto{\pgfqpoint{1.908988in}{2.747570in}}%
\pgfpathlineto{\pgfqpoint{1.920430in}{2.750938in}}%
\pgfpathlineto{\pgfqpoint{1.931866in}{2.754311in}}%
\pgfpathlineto{\pgfqpoint{1.943297in}{2.757682in}}%
\pgfpathlineto{\pgfqpoint{1.954723in}{2.761046in}}%
\pgfpathlineto{\pgfqpoint{1.948975in}{2.768826in}}%
\pgfpathlineto{\pgfqpoint{1.943231in}{2.776595in}}%
\pgfpathlineto{\pgfqpoint{1.937491in}{2.784351in}}%
\pgfpathlineto{\pgfqpoint{1.931756in}{2.792096in}}%
\pgfpathlineto{\pgfqpoint{1.926024in}{2.799829in}}%
\pgfpathlineto{\pgfqpoint{1.914612in}{2.796439in}}%
\pgfpathlineto{\pgfqpoint{1.903195in}{2.793043in}}%
\pgfpathlineto{\pgfqpoint{1.891772in}{2.789648in}}%
\pgfpathlineto{\pgfqpoint{1.880343in}{2.786260in}}%
\pgfpathlineto{\pgfqpoint{1.868909in}{2.782882in}}%
\pgfpathlineto{\pgfqpoint{1.874627in}{2.775172in}}%
\pgfpathlineto{\pgfqpoint{1.880349in}{2.767450in}}%
\pgfpathlineto{\pgfqpoint{1.886076in}{2.759716in}}%
\pgfpathlineto{\pgfqpoint{1.891806in}{2.751969in}}%
\pgfpathclose%
\pgfusepath{stroke,fill}%
\end{pgfscope}%
\begin{pgfscope}%
\pgfpathrectangle{\pgfqpoint{0.887500in}{0.275000in}}{\pgfqpoint{4.225000in}{4.225000in}}%
\pgfusepath{clip}%
\pgfsetbuttcap%
\pgfsetroundjoin%
\definecolor{currentfill}{rgb}{0.120565,0.596422,0.543611}%
\pgfsetfillcolor{currentfill}%
\pgfsetfillopacity{0.700000}%
\pgfsetlinewidth{0.501875pt}%
\definecolor{currentstroke}{rgb}{1.000000,1.000000,1.000000}%
\pgfsetstrokecolor{currentstroke}%
\pgfsetstrokeopacity{0.500000}%
\pgfsetdash{}{0pt}%
\pgfpathmoveto{\pgfqpoint{2.498067in}{2.700531in}}%
\pgfpathlineto{\pgfqpoint{2.509284in}{2.709611in}}%
\pgfpathlineto{\pgfqpoint{2.520485in}{2.719767in}}%
\pgfpathlineto{\pgfqpoint{2.531669in}{2.731090in}}%
\pgfpathlineto{\pgfqpoint{2.542838in}{2.743428in}}%
\pgfpathlineto{\pgfqpoint{2.553999in}{2.756468in}}%
\pgfpathlineto{\pgfqpoint{2.547987in}{2.769005in}}%
\pgfpathlineto{\pgfqpoint{2.541977in}{2.781609in}}%
\pgfpathlineto{\pgfqpoint{2.535971in}{2.794131in}}%
\pgfpathlineto{\pgfqpoint{2.529970in}{2.806423in}}%
\pgfpathlineto{\pgfqpoint{2.523977in}{2.818337in}}%
\pgfpathlineto{\pgfqpoint{2.512829in}{2.805304in}}%
\pgfpathlineto{\pgfqpoint{2.501674in}{2.792734in}}%
\pgfpathlineto{\pgfqpoint{2.490508in}{2.780979in}}%
\pgfpathlineto{\pgfqpoint{2.479328in}{2.770210in}}%
\pgfpathlineto{\pgfqpoint{2.468132in}{2.760330in}}%
\pgfpathlineto{\pgfqpoint{2.474115in}{2.748324in}}%
\pgfpathlineto{\pgfqpoint{2.480100in}{2.736256in}}%
\pgfpathlineto{\pgfqpoint{2.486089in}{2.724211in}}%
\pgfpathlineto{\pgfqpoint{2.492078in}{2.712274in}}%
\pgfpathclose%
\pgfusepath{stroke,fill}%
\end{pgfscope}%
\begin{pgfscope}%
\pgfpathrectangle{\pgfqpoint{0.887500in}{0.275000in}}{\pgfqpoint{4.225000in}{4.225000in}}%
\pgfusepath{clip}%
\pgfsetbuttcap%
\pgfsetroundjoin%
\definecolor{currentfill}{rgb}{0.814576,0.883393,0.110347}%
\pgfsetfillcolor{currentfill}%
\pgfsetfillopacity{0.700000}%
\pgfsetlinewidth{0.501875pt}%
\definecolor{currentstroke}{rgb}{1.000000,1.000000,1.000000}%
\pgfsetstrokecolor{currentstroke}%
\pgfsetstrokeopacity{0.500000}%
\pgfsetdash{}{0pt}%
\pgfpathmoveto{\pgfqpoint{3.539305in}{3.509958in}}%
\pgfpathlineto{\pgfqpoint{3.550364in}{3.513493in}}%
\pgfpathlineto{\pgfqpoint{3.561419in}{3.516985in}}%
\pgfpathlineto{\pgfqpoint{3.572467in}{3.520412in}}%
\pgfpathlineto{\pgfqpoint{3.583509in}{3.523752in}}%
\pgfpathlineto{\pgfqpoint{3.594544in}{3.526985in}}%
\pgfpathlineto{\pgfqpoint{3.588271in}{3.536532in}}%
\pgfpathlineto{\pgfqpoint{3.581994in}{3.545546in}}%
\pgfpathlineto{\pgfqpoint{3.575714in}{3.554057in}}%
\pgfpathlineto{\pgfqpoint{3.569433in}{3.562133in}}%
\pgfpathlineto{\pgfqpoint{3.563152in}{3.569840in}}%
\pgfpathlineto{\pgfqpoint{3.552119in}{3.566009in}}%
\pgfpathlineto{\pgfqpoint{3.541080in}{3.562109in}}%
\pgfpathlineto{\pgfqpoint{3.530035in}{3.558149in}}%
\pgfpathlineto{\pgfqpoint{3.518985in}{3.554141in}}%
\pgfpathlineto{\pgfqpoint{3.507930in}{3.550093in}}%
\pgfpathlineto{\pgfqpoint{3.514208in}{3.543049in}}%
\pgfpathlineto{\pgfqpoint{3.520485in}{3.535590in}}%
\pgfpathlineto{\pgfqpoint{3.526761in}{3.527636in}}%
\pgfpathlineto{\pgfqpoint{3.533035in}{3.519105in}}%
\pgfpathclose%
\pgfusepath{stroke,fill}%
\end{pgfscope}%
\begin{pgfscope}%
\pgfpathrectangle{\pgfqpoint{0.887500in}{0.275000in}}{\pgfqpoint{4.225000in}{4.225000in}}%
\pgfusepath{clip}%
\pgfsetbuttcap%
\pgfsetroundjoin%
\definecolor{currentfill}{rgb}{0.119483,0.614817,0.537692}%
\pgfsetfillcolor{currentfill}%
\pgfsetfillopacity{0.700000}%
\pgfsetlinewidth{0.501875pt}%
\definecolor{currentstroke}{rgb}{1.000000,1.000000,1.000000}%
\pgfsetstrokecolor{currentstroke}%
\pgfsetstrokeopacity{0.500000}%
\pgfsetdash{}{0pt}%
\pgfpathmoveto{\pgfqpoint{1.668369in}{2.771479in}}%
\pgfpathlineto{\pgfqpoint{1.679877in}{2.774760in}}%
\pgfpathlineto{\pgfqpoint{1.691379in}{2.778041in}}%
\pgfpathlineto{\pgfqpoint{1.702875in}{2.781322in}}%
\pgfpathlineto{\pgfqpoint{1.714366in}{2.784606in}}%
\pgfpathlineto{\pgfqpoint{1.725851in}{2.787892in}}%
\pgfpathlineto{\pgfqpoint{1.720185in}{2.795483in}}%
\pgfpathlineto{\pgfqpoint{1.714523in}{2.803059in}}%
\pgfpathlineto{\pgfqpoint{1.708866in}{2.810620in}}%
\pgfpathlineto{\pgfqpoint{1.703212in}{2.818168in}}%
\pgfpathlineto{\pgfqpoint{1.697563in}{2.825701in}}%
\pgfpathlineto{\pgfqpoint{1.686092in}{2.822393in}}%
\pgfpathlineto{\pgfqpoint{1.674615in}{2.819090in}}%
\pgfpathlineto{\pgfqpoint{1.663133in}{2.815789in}}%
\pgfpathlineto{\pgfqpoint{1.651645in}{2.812490in}}%
\pgfpathlineto{\pgfqpoint{1.640152in}{2.809191in}}%
\pgfpathlineto{\pgfqpoint{1.645787in}{2.801681in}}%
\pgfpathlineto{\pgfqpoint{1.651426in}{2.794155in}}%
\pgfpathlineto{\pgfqpoint{1.657069in}{2.786612in}}%
\pgfpathlineto{\pgfqpoint{1.662717in}{2.779054in}}%
\pgfpathclose%
\pgfusepath{stroke,fill}%
\end{pgfscope}%
\begin{pgfscope}%
\pgfpathrectangle{\pgfqpoint{0.887500in}{0.275000in}}{\pgfqpoint{4.225000in}{4.225000in}}%
\pgfusepath{clip}%
\pgfsetbuttcap%
\pgfsetroundjoin%
\definecolor{currentfill}{rgb}{0.124395,0.578002,0.548287}%
\pgfsetfillcolor{currentfill}%
\pgfsetfillopacity{0.700000}%
\pgfsetlinewidth{0.501875pt}%
\definecolor{currentstroke}{rgb}{1.000000,1.000000,1.000000}%
\pgfsetstrokecolor{currentstroke}%
\pgfsetstrokeopacity{0.500000}%
\pgfsetdash{}{0pt}%
\pgfpathmoveto{\pgfqpoint{2.212718in}{2.693338in}}%
\pgfpathlineto{\pgfqpoint{2.224096in}{2.696577in}}%
\pgfpathlineto{\pgfqpoint{2.235473in}{2.699589in}}%
\pgfpathlineto{\pgfqpoint{2.246848in}{2.702397in}}%
\pgfpathlineto{\pgfqpoint{2.258219in}{2.705170in}}%
\pgfpathlineto{\pgfqpoint{2.269580in}{2.708097in}}%
\pgfpathlineto{\pgfqpoint{2.263722in}{2.716047in}}%
\pgfpathlineto{\pgfqpoint{2.257869in}{2.723980in}}%
\pgfpathlineto{\pgfqpoint{2.252019in}{2.731888in}}%
\pgfpathlineto{\pgfqpoint{2.246174in}{2.739763in}}%
\pgfpathlineto{\pgfqpoint{2.240334in}{2.747596in}}%
\pgfpathlineto{\pgfqpoint{2.228981in}{2.744837in}}%
\pgfpathlineto{\pgfqpoint{2.217618in}{2.742286in}}%
\pgfpathlineto{\pgfqpoint{2.206249in}{2.739690in}}%
\pgfpathlineto{\pgfqpoint{2.194881in}{2.736819in}}%
\pgfpathlineto{\pgfqpoint{2.183514in}{2.733642in}}%
\pgfpathlineto{\pgfqpoint{2.189347in}{2.725622in}}%
\pgfpathlineto{\pgfqpoint{2.195183in}{2.717580in}}%
\pgfpathlineto{\pgfqpoint{2.201024in}{2.709518in}}%
\pgfpathlineto{\pgfqpoint{2.206869in}{2.701437in}}%
\pgfpathclose%
\pgfusepath{stroke,fill}%
\end{pgfscope}%
\begin{pgfscope}%
\pgfpathrectangle{\pgfqpoint{0.887500in}{0.275000in}}{\pgfqpoint{4.225000in}{4.225000in}}%
\pgfusepath{clip}%
\pgfsetbuttcap%
\pgfsetroundjoin%
\definecolor{currentfill}{rgb}{0.783315,0.879285,0.125405}%
\pgfsetfillcolor{currentfill}%
\pgfsetfillopacity{0.700000}%
\pgfsetlinewidth{0.501875pt}%
\definecolor{currentstroke}{rgb}{1.000000,1.000000,1.000000}%
\pgfsetstrokecolor{currentstroke}%
\pgfsetstrokeopacity{0.500000}%
\pgfsetdash{}{0pt}%
\pgfpathmoveto{\pgfqpoint{3.025649in}{3.499522in}}%
\pgfpathlineto{\pgfqpoint{3.036832in}{3.502550in}}%
\pgfpathlineto{\pgfqpoint{3.048010in}{3.506674in}}%
\pgfpathlineto{\pgfqpoint{3.059182in}{3.511624in}}%
\pgfpathlineto{\pgfqpoint{3.070351in}{3.517122in}}%
\pgfpathlineto{\pgfqpoint{3.081515in}{3.522890in}}%
\pgfpathlineto{\pgfqpoint{3.075382in}{3.525797in}}%
\pgfpathlineto{\pgfqpoint{3.069254in}{3.528429in}}%
\pgfpathlineto{\pgfqpoint{3.063132in}{3.530797in}}%
\pgfpathlineto{\pgfqpoint{3.057016in}{3.532916in}}%
\pgfpathlineto{\pgfqpoint{3.050906in}{3.534804in}}%
\pgfpathlineto{\pgfqpoint{3.039762in}{3.529110in}}%
\pgfpathlineto{\pgfqpoint{3.028613in}{3.523759in}}%
\pgfpathlineto{\pgfqpoint{3.017460in}{3.519018in}}%
\pgfpathlineto{\pgfqpoint{3.006302in}{3.515155in}}%
\pgfpathlineto{\pgfqpoint{2.995137in}{3.512429in}}%
\pgfpathlineto{\pgfqpoint{3.001228in}{3.509685in}}%
\pgfpathlineto{\pgfqpoint{3.007325in}{3.506975in}}%
\pgfpathlineto{\pgfqpoint{3.013428in}{3.504367in}}%
\pgfpathlineto{\pgfqpoint{3.019536in}{3.501901in}}%
\pgfpathclose%
\pgfusepath{stroke,fill}%
\end{pgfscope}%
\begin{pgfscope}%
\pgfpathrectangle{\pgfqpoint{0.887500in}{0.275000in}}{\pgfqpoint{4.225000in}{4.225000in}}%
\pgfusepath{clip}%
\pgfsetbuttcap%
\pgfsetroundjoin%
\definecolor{currentfill}{rgb}{0.751884,0.874951,0.143228}%
\pgfsetfillcolor{currentfill}%
\pgfsetfillopacity{0.700000}%
\pgfsetlinewidth{0.501875pt}%
\definecolor{currentstroke}{rgb}{1.000000,1.000000,1.000000}%
\pgfsetstrokecolor{currentstroke}%
\pgfsetstrokeopacity{0.500000}%
\pgfsetdash{}{0pt}%
\pgfpathmoveto{\pgfqpoint{2.883260in}{3.465950in}}%
\pgfpathlineto{\pgfqpoint{2.894430in}{3.480030in}}%
\pgfpathlineto{\pgfqpoint{2.905615in}{3.490400in}}%
\pgfpathlineto{\pgfqpoint{2.916809in}{3.497740in}}%
\pgfpathlineto{\pgfqpoint{2.928007in}{3.502726in}}%
\pgfpathlineto{\pgfqpoint{2.939207in}{3.505941in}}%
\pgfpathlineto{\pgfqpoint{2.933137in}{3.508659in}}%
\pgfpathlineto{\pgfqpoint{2.927075in}{3.510985in}}%
\pgfpathlineto{\pgfqpoint{2.921020in}{3.512790in}}%
\pgfpathlineto{\pgfqpoint{2.914973in}{3.513947in}}%
\pgfpathlineto{\pgfqpoint{2.908935in}{3.514328in}}%
\pgfpathlineto{\pgfqpoint{2.897758in}{3.510459in}}%
\pgfpathlineto{\pgfqpoint{2.886582in}{3.505187in}}%
\pgfpathlineto{\pgfqpoint{2.875409in}{3.498187in}}%
\pgfpathlineto{\pgfqpoint{2.864244in}{3.489132in}}%
\pgfpathlineto{\pgfqpoint{2.853088in}{3.477697in}}%
\pgfpathlineto{\pgfqpoint{2.859101in}{3.477311in}}%
\pgfpathlineto{\pgfqpoint{2.865127in}{3.475742in}}%
\pgfpathlineto{\pgfqpoint{2.871162in}{3.473193in}}%
\pgfpathlineto{\pgfqpoint{2.877207in}{3.469862in}}%
\pgfpathclose%
\pgfusepath{stroke,fill}%
\end{pgfscope}%
\begin{pgfscope}%
\pgfpathrectangle{\pgfqpoint{0.887500in}{0.275000in}}{\pgfqpoint{4.225000in}{4.225000in}}%
\pgfusepath{clip}%
\pgfsetbuttcap%
\pgfsetroundjoin%
\definecolor{currentfill}{rgb}{0.180653,0.701402,0.488189}%
\pgfsetfillcolor{currentfill}%
\pgfsetfillopacity{0.700000}%
\pgfsetlinewidth{0.501875pt}%
\definecolor{currentstroke}{rgb}{1.000000,1.000000,1.000000}%
\pgfsetstrokecolor{currentstroke}%
\pgfsetstrokeopacity{0.500000}%
\pgfsetdash{}{0pt}%
\pgfpathmoveto{\pgfqpoint{2.807638in}{2.943467in}}%
\pgfpathlineto{\pgfqpoint{2.818823in}{2.954354in}}%
\pgfpathlineto{\pgfqpoint{2.829991in}{2.967754in}}%
\pgfpathlineto{\pgfqpoint{2.841140in}{2.984987in}}%
\pgfpathlineto{\pgfqpoint{2.852264in}{3.007383in}}%
\pgfpathlineto{\pgfqpoint{2.863363in}{3.036269in}}%
\pgfpathlineto{\pgfqpoint{2.857348in}{3.033857in}}%
\pgfpathlineto{\pgfqpoint{2.851343in}{3.031180in}}%
\pgfpathlineto{\pgfqpoint{2.845344in}{3.028383in}}%
\pgfpathlineto{\pgfqpoint{2.839352in}{3.025818in}}%
\pgfpathlineto{\pgfqpoint{2.833364in}{3.023849in}}%
\pgfpathlineto{\pgfqpoint{2.822302in}{2.996355in}}%
\pgfpathlineto{\pgfqpoint{2.811195in}{2.977599in}}%
\pgfpathlineto{\pgfqpoint{2.800048in}{2.965707in}}%
\pgfpathlineto{\pgfqpoint{2.788866in}{2.958813in}}%
\pgfpathlineto{\pgfqpoint{2.777659in}{2.955053in}}%
\pgfpathlineto{\pgfqpoint{2.783642in}{2.953020in}}%
\pgfpathlineto{\pgfqpoint{2.789629in}{2.951068in}}%
\pgfpathlineto{\pgfqpoint{2.795624in}{2.948950in}}%
\pgfpathlineto{\pgfqpoint{2.801627in}{2.946435in}}%
\pgfpathclose%
\pgfusepath{stroke,fill}%
\end{pgfscope}%
\begin{pgfscope}%
\pgfpathrectangle{\pgfqpoint{0.887500in}{0.275000in}}{\pgfqpoint{4.225000in}{4.225000in}}%
\pgfusepath{clip}%
\pgfsetbuttcap%
\pgfsetroundjoin%
\definecolor{currentfill}{rgb}{0.824940,0.884720,0.106217}%
\pgfsetfillcolor{currentfill}%
\pgfsetfillopacity{0.700000}%
\pgfsetlinewidth{0.501875pt}%
\definecolor{currentstroke}{rgb}{1.000000,1.000000,1.000000}%
\pgfsetstrokecolor{currentstroke}%
\pgfsetstrokeopacity{0.500000}%
\pgfsetdash{}{0pt}%
\pgfpathmoveto{\pgfqpoint{3.168089in}{3.523735in}}%
\pgfpathlineto{\pgfqpoint{3.179239in}{3.527190in}}%
\pgfpathlineto{\pgfqpoint{3.190383in}{3.530551in}}%
\pgfpathlineto{\pgfqpoint{3.201522in}{3.533859in}}%
\pgfpathlineto{\pgfqpoint{3.212656in}{3.537154in}}%
\pgfpathlineto{\pgfqpoint{3.223785in}{3.540470in}}%
\pgfpathlineto{\pgfqpoint{3.217597in}{3.545599in}}%
\pgfpathlineto{\pgfqpoint{3.211414in}{3.550477in}}%
\pgfpathlineto{\pgfqpoint{3.205234in}{3.555093in}}%
\pgfpathlineto{\pgfqpoint{3.199059in}{3.559436in}}%
\pgfpathlineto{\pgfqpoint{3.192888in}{3.563497in}}%
\pgfpathlineto{\pgfqpoint{3.181775in}{3.560426in}}%
\pgfpathlineto{\pgfqpoint{3.170657in}{3.557370in}}%
\pgfpathlineto{\pgfqpoint{3.159534in}{3.554241in}}%
\pgfpathlineto{\pgfqpoint{3.148405in}{3.550943in}}%
\pgfpathlineto{\pgfqpoint{3.137270in}{3.547379in}}%
\pgfpathlineto{\pgfqpoint{3.143425in}{3.543408in}}%
\pgfpathlineto{\pgfqpoint{3.149585in}{3.539027in}}%
\pgfpathlineto{\pgfqpoint{3.155749in}{3.534267in}}%
\pgfpathlineto{\pgfqpoint{3.161917in}{3.529159in}}%
\pgfpathclose%
\pgfusepath{stroke,fill}%
\end{pgfscope}%
\begin{pgfscope}%
\pgfpathrectangle{\pgfqpoint{0.887500in}{0.275000in}}{\pgfqpoint{4.225000in}{4.225000in}}%
\pgfusepath{clip}%
\pgfsetbuttcap%
\pgfsetroundjoin%
\definecolor{currentfill}{rgb}{0.772852,0.877868,0.131109}%
\pgfsetfillcolor{currentfill}%
\pgfsetfillopacity{0.700000}%
\pgfsetlinewidth{0.501875pt}%
\definecolor{currentstroke}{rgb}{1.000000,1.000000,1.000000}%
\pgfsetstrokecolor{currentstroke}%
\pgfsetstrokeopacity{0.500000}%
\pgfsetdash{}{0pt}%
\pgfpathmoveto{\pgfqpoint{3.625887in}{3.473511in}}%
\pgfpathlineto{\pgfqpoint{3.636920in}{3.476468in}}%
\pgfpathlineto{\pgfqpoint{3.647946in}{3.479298in}}%
\pgfpathlineto{\pgfqpoint{3.658964in}{3.482026in}}%
\pgfpathlineto{\pgfqpoint{3.669977in}{3.484732in}}%
\pgfpathlineto{\pgfqpoint{3.680985in}{3.487494in}}%
\pgfpathlineto{\pgfqpoint{3.674711in}{3.498646in}}%
\pgfpathlineto{\pgfqpoint{3.668438in}{3.509679in}}%
\pgfpathlineto{\pgfqpoint{3.662165in}{3.520540in}}%
\pgfpathlineto{\pgfqpoint{3.655893in}{3.531174in}}%
\pgfpathlineto{\pgfqpoint{3.649619in}{3.541528in}}%
\pgfpathlineto{\pgfqpoint{3.638616in}{3.538681in}}%
\pgfpathlineto{\pgfqpoint{3.627608in}{3.535882in}}%
\pgfpathlineto{\pgfqpoint{3.616594in}{3.533045in}}%
\pgfpathlineto{\pgfqpoint{3.605573in}{3.530089in}}%
\pgfpathlineto{\pgfqpoint{3.594544in}{3.526985in}}%
\pgfpathlineto{\pgfqpoint{3.600815in}{3.516960in}}%
\pgfpathlineto{\pgfqpoint{3.607084in}{3.506530in}}%
\pgfpathlineto{\pgfqpoint{3.613352in}{3.495764in}}%
\pgfpathlineto{\pgfqpoint{3.619619in}{3.484734in}}%
\pgfpathclose%
\pgfusepath{stroke,fill}%
\end{pgfscope}%
\begin{pgfscope}%
\pgfpathrectangle{\pgfqpoint{0.887500in}{0.275000in}}{\pgfqpoint{4.225000in}{4.225000in}}%
\pgfusepath{clip}%
\pgfsetbuttcap%
\pgfsetroundjoin%
\definecolor{currentfill}{rgb}{0.720391,0.870350,0.162603}%
\pgfsetfillcolor{currentfill}%
\pgfsetfillopacity{0.700000}%
\pgfsetlinewidth{0.501875pt}%
\definecolor{currentstroke}{rgb}{1.000000,1.000000,1.000000}%
\pgfsetstrokecolor{currentstroke}%
\pgfsetstrokeopacity{0.500000}%
\pgfsetdash{}{0pt}%
\pgfpathmoveto{\pgfqpoint{3.712408in}{3.431844in}}%
\pgfpathlineto{\pgfqpoint{3.723421in}{3.434902in}}%
\pgfpathlineto{\pgfqpoint{3.734431in}{3.438107in}}%
\pgfpathlineto{\pgfqpoint{3.745440in}{3.441510in}}%
\pgfpathlineto{\pgfqpoint{3.756447in}{3.445161in}}%
\pgfpathlineto{\pgfqpoint{3.767454in}{3.449056in}}%
\pgfpathlineto{\pgfqpoint{3.761156in}{3.460114in}}%
\pgfpathlineto{\pgfqpoint{3.754862in}{3.471258in}}%
\pgfpathlineto{\pgfqpoint{3.748573in}{3.482454in}}%
\pgfpathlineto{\pgfqpoint{3.742287in}{3.493669in}}%
\pgfpathlineto{\pgfqpoint{3.736004in}{3.504866in}}%
\pgfpathlineto{\pgfqpoint{3.724999in}{3.500713in}}%
\pgfpathlineto{\pgfqpoint{3.713996in}{3.496925in}}%
\pgfpathlineto{\pgfqpoint{3.702993in}{3.493510in}}%
\pgfpathlineto{\pgfqpoint{3.691990in}{3.490393in}}%
\pgfpathlineto{\pgfqpoint{3.680985in}{3.487494in}}%
\pgfpathlineto{\pgfqpoint{3.687262in}{3.476277in}}%
\pgfpathlineto{\pgfqpoint{3.693542in}{3.465051in}}%
\pgfpathlineto{\pgfqpoint{3.699826in}{3.453868in}}%
\pgfpathlineto{\pgfqpoint{3.706114in}{3.442782in}}%
\pgfpathclose%
\pgfusepath{stroke,fill}%
\end{pgfscope}%
\begin{pgfscope}%
\pgfpathrectangle{\pgfqpoint{0.887500in}{0.275000in}}{\pgfqpoint{4.225000in}{4.225000in}}%
\pgfusepath{clip}%
\pgfsetbuttcap%
\pgfsetroundjoin%
\definecolor{currentfill}{rgb}{0.125394,0.574318,0.549086}%
\pgfsetfillcolor{currentfill}%
\pgfsetfillopacity{0.700000}%
\pgfsetlinewidth{0.501875pt}%
\definecolor{currentstroke}{rgb}{1.000000,1.000000,1.000000}%
\pgfsetstrokecolor{currentstroke}%
\pgfsetstrokeopacity{0.500000}%
\pgfsetdash{}{0pt}%
\pgfpathmoveto{\pgfqpoint{2.441759in}{2.667420in}}%
\pgfpathlineto{\pgfqpoint{2.453047in}{2.672769in}}%
\pgfpathlineto{\pgfqpoint{2.464323in}{2.678645in}}%
\pgfpathlineto{\pgfqpoint{2.475586in}{2.685157in}}%
\pgfpathlineto{\pgfqpoint{2.486834in}{2.692416in}}%
\pgfpathlineto{\pgfqpoint{2.498067in}{2.700531in}}%
\pgfpathlineto{\pgfqpoint{2.492078in}{2.712274in}}%
\pgfpathlineto{\pgfqpoint{2.486089in}{2.724211in}}%
\pgfpathlineto{\pgfqpoint{2.480100in}{2.736256in}}%
\pgfpathlineto{\pgfqpoint{2.474115in}{2.748324in}}%
\pgfpathlineto{\pgfqpoint{2.468132in}{2.760330in}}%
\pgfpathlineto{\pgfqpoint{2.456924in}{2.751218in}}%
\pgfpathlineto{\pgfqpoint{2.445703in}{2.742757in}}%
\pgfpathlineto{\pgfqpoint{2.434472in}{2.734825in}}%
\pgfpathlineto{\pgfqpoint{2.423231in}{2.727305in}}%
\pgfpathlineto{\pgfqpoint{2.411982in}{2.720080in}}%
\pgfpathlineto{\pgfqpoint{2.417933in}{2.709426in}}%
\pgfpathlineto{\pgfqpoint{2.423887in}{2.698788in}}%
\pgfpathlineto{\pgfqpoint{2.429843in}{2.688211in}}%
\pgfpathlineto{\pgfqpoint{2.435800in}{2.677740in}}%
\pgfpathclose%
\pgfusepath{stroke,fill}%
\end{pgfscope}%
\begin{pgfscope}%
\pgfpathrectangle{\pgfqpoint{0.887500in}{0.275000in}}{\pgfqpoint{4.225000in}{4.225000in}}%
\pgfusepath{clip}%
\pgfsetbuttcap%
\pgfsetroundjoin%
\definecolor{currentfill}{rgb}{0.121148,0.592739,0.544641}%
\pgfsetfillcolor{currentfill}%
\pgfsetfillopacity{0.700000}%
\pgfsetlinewidth{0.501875pt}%
\definecolor{currentstroke}{rgb}{1.000000,1.000000,1.000000}%
\pgfsetstrokecolor{currentstroke}%
\pgfsetstrokeopacity{0.500000}%
\pgfsetdash{}{0pt}%
\pgfpathmoveto{\pgfqpoint{1.983524in}{2.721931in}}%
\pgfpathlineto{\pgfqpoint{1.994958in}{2.725255in}}%
\pgfpathlineto{\pgfqpoint{2.006386in}{2.728559in}}%
\pgfpathlineto{\pgfqpoint{2.017810in}{2.731837in}}%
\pgfpathlineto{\pgfqpoint{2.029230in}{2.735090in}}%
\pgfpathlineto{\pgfqpoint{2.040643in}{2.738335in}}%
\pgfpathlineto{\pgfqpoint{2.034862in}{2.746209in}}%
\pgfpathlineto{\pgfqpoint{2.029084in}{2.754071in}}%
\pgfpathlineto{\pgfqpoint{2.023311in}{2.761919in}}%
\pgfpathlineto{\pgfqpoint{2.017541in}{2.769754in}}%
\pgfpathlineto{\pgfqpoint{2.011776in}{2.777577in}}%
\pgfpathlineto{\pgfqpoint{2.000375in}{2.774307in}}%
\pgfpathlineto{\pgfqpoint{1.988969in}{2.771030in}}%
\pgfpathlineto{\pgfqpoint{1.977559in}{2.767725in}}%
\pgfpathlineto{\pgfqpoint{1.966143in}{2.764395in}}%
\pgfpathlineto{\pgfqpoint{1.954723in}{2.761046in}}%
\pgfpathlineto{\pgfqpoint{1.960475in}{2.753252in}}%
\pgfpathlineto{\pgfqpoint{1.966231in}{2.745444in}}%
\pgfpathlineto{\pgfqpoint{1.971991in}{2.737621in}}%
\pgfpathlineto{\pgfqpoint{1.977755in}{2.729784in}}%
\pgfpathclose%
\pgfusepath{stroke,fill}%
\end{pgfscope}%
\begin{pgfscope}%
\pgfpathrectangle{\pgfqpoint{0.887500in}{0.275000in}}{\pgfqpoint{4.225000in}{4.225000in}}%
\pgfusepath{clip}%
\pgfsetbuttcap%
\pgfsetroundjoin%
\definecolor{currentfill}{rgb}{0.121380,0.629492,0.531973}%
\pgfsetfillcolor{currentfill}%
\pgfsetfillopacity{0.700000}%
\pgfsetlinewidth{0.501875pt}%
\definecolor{currentstroke}{rgb}{1.000000,1.000000,1.000000}%
\pgfsetstrokecolor{currentstroke}%
\pgfsetstrokeopacity{0.500000}%
\pgfsetdash{}{0pt}%
\pgfpathmoveto{\pgfqpoint{1.439193in}{2.796477in}}%
\pgfpathlineto{\pgfqpoint{1.450755in}{2.799821in}}%
\pgfpathlineto{\pgfqpoint{1.462311in}{2.803162in}}%
\pgfpathlineto{\pgfqpoint{1.473862in}{2.806500in}}%
\pgfpathlineto{\pgfqpoint{1.485408in}{2.809838in}}%
\pgfpathlineto{\pgfqpoint{1.496947in}{2.813175in}}%
\pgfpathlineto{\pgfqpoint{1.491367in}{2.820541in}}%
\pgfpathlineto{\pgfqpoint{1.485791in}{2.827888in}}%
\pgfpathlineto{\pgfqpoint{1.480220in}{2.835217in}}%
\pgfpathlineto{\pgfqpoint{1.474653in}{2.842526in}}%
\pgfpathlineto{\pgfqpoint{1.463125in}{2.839182in}}%
\pgfpathlineto{\pgfqpoint{1.451591in}{2.835834in}}%
\pgfpathlineto{\pgfqpoint{1.440052in}{2.832478in}}%
\pgfpathlineto{\pgfqpoint{1.428507in}{2.829117in}}%
\pgfpathlineto{\pgfqpoint{1.416958in}{2.825749in}}%
\pgfpathlineto{\pgfqpoint{1.422510in}{2.818460in}}%
\pgfpathlineto{\pgfqpoint{1.428067in}{2.811152in}}%
\pgfpathlineto{\pgfqpoint{1.433628in}{2.803824in}}%
\pgfpathclose%
\pgfusepath{stroke,fill}%
\end{pgfscope}%
\begin{pgfscope}%
\pgfpathrectangle{\pgfqpoint{0.887500in}{0.275000in}}{\pgfqpoint{4.225000in}{4.225000in}}%
\pgfusepath{clip}%
\pgfsetbuttcap%
\pgfsetroundjoin%
\definecolor{currentfill}{rgb}{0.153894,0.680203,0.504172}%
\pgfsetfillcolor{currentfill}%
\pgfsetfillopacity{0.700000}%
\pgfsetlinewidth{0.501875pt}%
\definecolor{currentstroke}{rgb}{1.000000,1.000000,1.000000}%
\pgfsetstrokecolor{currentstroke}%
\pgfsetstrokeopacity{0.500000}%
\pgfsetdash{}{0pt}%
\pgfpathmoveto{\pgfqpoint{2.751712in}{2.887541in}}%
\pgfpathlineto{\pgfqpoint{2.762885in}{2.900658in}}%
\pgfpathlineto{\pgfqpoint{2.774064in}{2.912911in}}%
\pgfpathlineto{\pgfqpoint{2.785251in}{2.923951in}}%
\pgfpathlineto{\pgfqpoint{2.796445in}{2.933772in}}%
\pgfpathlineto{\pgfqpoint{2.807638in}{2.943467in}}%
\pgfpathlineto{\pgfqpoint{2.801627in}{2.946435in}}%
\pgfpathlineto{\pgfqpoint{2.795624in}{2.948950in}}%
\pgfpathlineto{\pgfqpoint{2.789629in}{2.951068in}}%
\pgfpathlineto{\pgfqpoint{2.783642in}{2.953020in}}%
\pgfpathlineto{\pgfqpoint{2.777659in}{2.955053in}}%
\pgfpathlineto{\pgfqpoint{2.766438in}{2.952567in}}%
\pgfpathlineto{\pgfqpoint{2.755215in}{2.949492in}}%
\pgfpathlineto{\pgfqpoint{2.744004in}{2.944278in}}%
\pgfpathlineto{\pgfqpoint{2.732805in}{2.936915in}}%
\pgfpathlineto{\pgfqpoint{2.721618in}{2.927868in}}%
\pgfpathlineto{\pgfqpoint{2.727622in}{2.920702in}}%
\pgfpathlineto{\pgfqpoint{2.733632in}{2.913141in}}%
\pgfpathlineto{\pgfqpoint{2.739651in}{2.905117in}}%
\pgfpathlineto{\pgfqpoint{2.745678in}{2.896569in}}%
\pgfpathclose%
\pgfusepath{stroke,fill}%
\end{pgfscope}%
\begin{pgfscope}%
\pgfpathrectangle{\pgfqpoint{0.887500in}{0.275000in}}{\pgfqpoint{4.225000in}{4.225000in}}%
\pgfusepath{clip}%
\pgfsetbuttcap%
\pgfsetroundjoin%
\definecolor{currentfill}{rgb}{0.678489,0.863742,0.189503}%
\pgfsetfillcolor{currentfill}%
\pgfsetfillopacity{0.700000}%
\pgfsetlinewidth{0.501875pt}%
\definecolor{currentstroke}{rgb}{1.000000,1.000000,1.000000}%
\pgfsetstrokecolor{currentstroke}%
\pgfsetstrokeopacity{0.500000}%
\pgfsetdash{}{0pt}%
\pgfpathmoveto{\pgfqpoint{3.799009in}{3.394573in}}%
\pgfpathlineto{\pgfqpoint{3.810008in}{3.397968in}}%
\pgfpathlineto{\pgfqpoint{3.821004in}{3.401424in}}%
\pgfpathlineto{\pgfqpoint{3.831995in}{3.404938in}}%
\pgfpathlineto{\pgfqpoint{3.842983in}{3.408505in}}%
\pgfpathlineto{\pgfqpoint{3.853966in}{3.412120in}}%
\pgfpathlineto{\pgfqpoint{3.847657in}{3.423792in}}%
\pgfpathlineto{\pgfqpoint{3.841351in}{3.435474in}}%
\pgfpathlineto{\pgfqpoint{3.835049in}{3.447162in}}%
\pgfpathlineto{\pgfqpoint{3.828750in}{3.458852in}}%
\pgfpathlineto{\pgfqpoint{3.822453in}{3.470538in}}%
\pgfpathlineto{\pgfqpoint{3.811461in}{3.466132in}}%
\pgfpathlineto{\pgfqpoint{3.800464in}{3.461733in}}%
\pgfpathlineto{\pgfqpoint{3.789463in}{3.457389in}}%
\pgfpathlineto{\pgfqpoint{3.778460in}{3.453147in}}%
\pgfpathlineto{\pgfqpoint{3.767454in}{3.449056in}}%
\pgfpathlineto{\pgfqpoint{3.773757in}{3.438080in}}%
\pgfpathlineto{\pgfqpoint{3.780064in}{3.427167in}}%
\pgfpathlineto{\pgfqpoint{3.786376in}{3.416293in}}%
\pgfpathlineto{\pgfqpoint{3.792691in}{3.405436in}}%
\pgfpathclose%
\pgfusepath{stroke,fill}%
\end{pgfscope}%
\begin{pgfscope}%
\pgfpathrectangle{\pgfqpoint{0.887500in}{0.275000in}}{\pgfqpoint{4.225000in}{4.225000in}}%
\pgfusepath{clip}%
\pgfsetbuttcap%
\pgfsetroundjoin%
\definecolor{currentfill}{rgb}{0.565498,0.842430,0.262877}%
\pgfsetfillcolor{currentfill}%
\pgfsetfillopacity{0.700000}%
\pgfsetlinewidth{0.501875pt}%
\definecolor{currentstroke}{rgb}{1.000000,1.000000,1.000000}%
\pgfsetstrokecolor{currentstroke}%
\pgfsetstrokeopacity{0.500000}%
\pgfsetdash{}{0pt}%
\pgfpathmoveto{\pgfqpoint{3.972081in}{3.311788in}}%
\pgfpathlineto{\pgfqpoint{3.983034in}{3.314951in}}%
\pgfpathlineto{\pgfqpoint{3.993983in}{3.318145in}}%
\pgfpathlineto{\pgfqpoint{4.004927in}{3.321369in}}%
\pgfpathlineto{\pgfqpoint{4.015867in}{3.324622in}}%
\pgfpathlineto{\pgfqpoint{4.026802in}{3.327902in}}%
\pgfpathlineto{\pgfqpoint{4.020464in}{3.340071in}}%
\pgfpathlineto{\pgfqpoint{4.014129in}{3.352267in}}%
\pgfpathlineto{\pgfqpoint{4.007799in}{3.364496in}}%
\pgfpathlineto{\pgfqpoint{4.001472in}{3.376760in}}%
\pgfpathlineto{\pgfqpoint{3.995149in}{3.389064in}}%
\pgfpathlineto{\pgfqpoint{3.984203in}{3.385158in}}%
\pgfpathlineto{\pgfqpoint{3.973255in}{3.381341in}}%
\pgfpathlineto{\pgfqpoint{3.962305in}{3.377612in}}%
\pgfpathlineto{\pgfqpoint{3.951351in}{3.373974in}}%
\pgfpathlineto{\pgfqpoint{3.940394in}{3.370425in}}%
\pgfpathlineto{\pgfqpoint{3.946722in}{3.358578in}}%
\pgfpathlineto{\pgfqpoint{3.953056in}{3.346810in}}%
\pgfpathlineto{\pgfqpoint{3.959394in}{3.335102in}}%
\pgfpathlineto{\pgfqpoint{3.965736in}{3.323435in}}%
\pgfpathclose%
\pgfusepath{stroke,fill}%
\end{pgfscope}%
\begin{pgfscope}%
\pgfpathrectangle{\pgfqpoint{0.887500in}{0.275000in}}{\pgfqpoint{4.225000in}{4.225000in}}%
\pgfusepath{clip}%
\pgfsetbuttcap%
\pgfsetroundjoin%
\definecolor{currentfill}{rgb}{0.626579,0.854645,0.223353}%
\pgfsetfillcolor{currentfill}%
\pgfsetfillopacity{0.700000}%
\pgfsetlinewidth{0.501875pt}%
\definecolor{currentstroke}{rgb}{1.000000,1.000000,1.000000}%
\pgfsetstrokecolor{currentstroke}%
\pgfsetstrokeopacity{0.500000}%
\pgfsetdash{}{0pt}%
\pgfpathmoveto{\pgfqpoint{3.885562in}{3.354054in}}%
\pgfpathlineto{\pgfqpoint{3.896535in}{3.357144in}}%
\pgfpathlineto{\pgfqpoint{3.907505in}{3.360326in}}%
\pgfpathlineto{\pgfqpoint{3.918471in}{3.363601in}}%
\pgfpathlineto{\pgfqpoint{3.929434in}{3.366968in}}%
\pgfpathlineto{\pgfqpoint{3.940394in}{3.370425in}}%
\pgfpathlineto{\pgfqpoint{3.934070in}{3.382370in}}%
\pgfpathlineto{\pgfqpoint{3.927752in}{3.394405in}}%
\pgfpathlineto{\pgfqpoint{3.921438in}{3.406514in}}%
\pgfpathlineto{\pgfqpoint{3.915128in}{3.418682in}}%
\pgfpathlineto{\pgfqpoint{3.908822in}{3.430891in}}%
\pgfpathlineto{\pgfqpoint{3.897859in}{3.427036in}}%
\pgfpathlineto{\pgfqpoint{3.886891in}{3.423235in}}%
\pgfpathlineto{\pgfqpoint{3.875920in}{3.419483in}}%
\pgfpathlineto{\pgfqpoint{3.864945in}{3.415779in}}%
\pgfpathlineto{\pgfqpoint{3.853966in}{3.412120in}}%
\pgfpathlineto{\pgfqpoint{3.860278in}{3.400462in}}%
\pgfpathlineto{\pgfqpoint{3.866594in}{3.388822in}}%
\pgfpathlineto{\pgfqpoint{3.872913in}{3.377204in}}%
\pgfpathlineto{\pgfqpoint{3.879236in}{3.365614in}}%
\pgfpathclose%
\pgfusepath{stroke,fill}%
\end{pgfscope}%
\begin{pgfscope}%
\pgfpathrectangle{\pgfqpoint{0.887500in}{0.275000in}}{\pgfqpoint{4.225000in}{4.225000in}}%
\pgfusepath{clip}%
\pgfsetbuttcap%
\pgfsetroundjoin%
\definecolor{currentfill}{rgb}{0.824940,0.884720,0.106217}%
\pgfsetfillcolor{currentfill}%
\pgfsetfillopacity{0.700000}%
\pgfsetlinewidth{0.501875pt}%
\definecolor{currentstroke}{rgb}{1.000000,1.000000,1.000000}%
\pgfsetstrokecolor{currentstroke}%
\pgfsetstrokeopacity{0.500000}%
\pgfsetdash{}{0pt}%
\pgfpathmoveto{\pgfqpoint{3.397142in}{3.512772in}}%
\pgfpathlineto{\pgfqpoint{3.408239in}{3.515949in}}%
\pgfpathlineto{\pgfqpoint{3.419331in}{3.519219in}}%
\pgfpathlineto{\pgfqpoint{3.430419in}{3.522639in}}%
\pgfpathlineto{\pgfqpoint{3.441503in}{3.526244in}}%
\pgfpathlineto{\pgfqpoint{3.452585in}{3.530009in}}%
\pgfpathlineto{\pgfqpoint{3.446316in}{3.536201in}}%
\pgfpathlineto{\pgfqpoint{3.440049in}{3.542125in}}%
\pgfpathlineto{\pgfqpoint{3.433785in}{3.547869in}}%
\pgfpathlineto{\pgfqpoint{3.427525in}{3.553521in}}%
\pgfpathlineto{\pgfqpoint{3.421270in}{3.559167in}}%
\pgfpathlineto{\pgfqpoint{3.410200in}{3.555186in}}%
\pgfpathlineto{\pgfqpoint{3.399129in}{3.551554in}}%
\pgfpathlineto{\pgfqpoint{3.388056in}{3.548322in}}%
\pgfpathlineto{\pgfqpoint{3.376981in}{3.545430in}}%
\pgfpathlineto{\pgfqpoint{3.365901in}{3.542775in}}%
\pgfpathlineto{\pgfqpoint{3.372143in}{3.537135in}}%
\pgfpathlineto{\pgfqpoint{3.378388in}{3.531375in}}%
\pgfpathlineto{\pgfqpoint{3.384637in}{3.525436in}}%
\pgfpathlineto{\pgfqpoint{3.390889in}{3.519255in}}%
\pgfpathclose%
\pgfusepath{stroke,fill}%
\end{pgfscope}%
\begin{pgfscope}%
\pgfpathrectangle{\pgfqpoint{0.887500in}{0.275000in}}{\pgfqpoint{4.225000in}{4.225000in}}%
\pgfusepath{clip}%
\pgfsetbuttcap%
\pgfsetroundjoin%
\definecolor{currentfill}{rgb}{0.506271,0.828786,0.300362}%
\pgfsetfillcolor{currentfill}%
\pgfsetfillopacity{0.700000}%
\pgfsetlinewidth{0.501875pt}%
\definecolor{currentstroke}{rgb}{1.000000,1.000000,1.000000}%
\pgfsetstrokecolor{currentstroke}%
\pgfsetstrokeopacity{0.500000}%
\pgfsetdash{}{0pt}%
\pgfpathmoveto{\pgfqpoint{4.058543in}{3.267344in}}%
\pgfpathlineto{\pgfqpoint{4.069463in}{3.270066in}}%
\pgfpathlineto{\pgfqpoint{4.080377in}{3.272749in}}%
\pgfpathlineto{\pgfqpoint{4.091284in}{3.275393in}}%
\pgfpathlineto{\pgfqpoint{4.102185in}{3.277997in}}%
\pgfpathlineto{\pgfqpoint{4.095836in}{3.290429in}}%
\pgfpathlineto{\pgfqpoint{4.089493in}{3.302997in}}%
\pgfpathlineto{\pgfqpoint{4.083156in}{3.315677in}}%
\pgfpathlineto{\pgfqpoint{4.076824in}{3.328445in}}%
\pgfpathlineto{\pgfqpoint{4.070497in}{3.341277in}}%
\pgfpathlineto{\pgfqpoint{4.059580in}{3.337898in}}%
\pgfpathlineto{\pgfqpoint{4.048658in}{3.334541in}}%
\pgfpathlineto{\pgfqpoint{4.037732in}{3.331209in}}%
\pgfpathlineto{\pgfqpoint{4.026802in}{3.327902in}}%
\pgfpathlineto{\pgfqpoint{4.033143in}{3.315758in}}%
\pgfpathlineto{\pgfqpoint{4.039488in}{3.303634in}}%
\pgfpathlineto{\pgfqpoint{4.045837in}{3.291526in}}%
\pgfpathlineto{\pgfqpoint{4.052188in}{3.279431in}}%
\pgfpathclose%
\pgfusepath{stroke,fill}%
\end{pgfscope}%
\begin{pgfscope}%
\pgfpathrectangle{\pgfqpoint{0.887500in}{0.275000in}}{\pgfqpoint{4.225000in}{4.225000in}}%
\pgfusepath{clip}%
\pgfsetbuttcap%
\pgfsetroundjoin%
\definecolor{currentfill}{rgb}{0.119512,0.607464,0.540218}%
\pgfsetfillcolor{currentfill}%
\pgfsetfillopacity{0.700000}%
\pgfsetlinewidth{0.501875pt}%
\definecolor{currentstroke}{rgb}{1.000000,1.000000,1.000000}%
\pgfsetstrokecolor{currentstroke}%
\pgfsetstrokeopacity{0.500000}%
\pgfsetdash{}{0pt}%
\pgfpathmoveto{\pgfqpoint{1.754243in}{2.749721in}}%
\pgfpathlineto{\pgfqpoint{1.765736in}{2.752997in}}%
\pgfpathlineto{\pgfqpoint{1.777223in}{2.756277in}}%
\pgfpathlineto{\pgfqpoint{1.788704in}{2.759566in}}%
\pgfpathlineto{\pgfqpoint{1.800180in}{2.762863in}}%
\pgfpathlineto{\pgfqpoint{1.811650in}{2.766171in}}%
\pgfpathlineto{\pgfqpoint{1.805949in}{2.773850in}}%
\pgfpathlineto{\pgfqpoint{1.800253in}{2.781516in}}%
\pgfpathlineto{\pgfqpoint{1.794561in}{2.789169in}}%
\pgfpathlineto{\pgfqpoint{1.788873in}{2.796807in}}%
\pgfpathlineto{\pgfqpoint{1.783189in}{2.804432in}}%
\pgfpathlineto{\pgfqpoint{1.771733in}{2.801105in}}%
\pgfpathlineto{\pgfqpoint{1.760271in}{2.797789in}}%
\pgfpathlineto{\pgfqpoint{1.748803in}{2.794482in}}%
\pgfpathlineto{\pgfqpoint{1.737330in}{2.791184in}}%
\pgfpathlineto{\pgfqpoint{1.725851in}{2.787892in}}%
\pgfpathlineto{\pgfqpoint{1.731521in}{2.780287in}}%
\pgfpathlineto{\pgfqpoint{1.737195in}{2.772668in}}%
\pgfpathlineto{\pgfqpoint{1.742874in}{2.765034in}}%
\pgfpathlineto{\pgfqpoint{1.748556in}{2.757385in}}%
\pgfpathclose%
\pgfusepath{stroke,fill}%
\end{pgfscope}%
\begin{pgfscope}%
\pgfpathrectangle{\pgfqpoint{0.887500in}{0.275000in}}{\pgfqpoint{4.225000in}{4.225000in}}%
\pgfusepath{clip}%
\pgfsetbuttcap%
\pgfsetroundjoin%
\definecolor{currentfill}{rgb}{0.126453,0.570633,0.549841}%
\pgfsetfillcolor{currentfill}%
\pgfsetfillopacity{0.700000}%
\pgfsetlinewidth{0.501875pt}%
\definecolor{currentstroke}{rgb}{1.000000,1.000000,1.000000}%
\pgfsetstrokecolor{currentstroke}%
\pgfsetstrokeopacity{0.500000}%
\pgfsetdash{}{0pt}%
\pgfpathmoveto{\pgfqpoint{2.298927in}{2.668351in}}%
\pgfpathlineto{\pgfqpoint{2.310287in}{2.671661in}}%
\pgfpathlineto{\pgfqpoint{2.321637in}{2.675255in}}%
\pgfpathlineto{\pgfqpoint{2.332974in}{2.679237in}}%
\pgfpathlineto{\pgfqpoint{2.344298in}{2.683707in}}%
\pgfpathlineto{\pgfqpoint{2.355607in}{2.688706in}}%
\pgfpathlineto{\pgfqpoint{2.349706in}{2.697266in}}%
\pgfpathlineto{\pgfqpoint{2.343810in}{2.705830in}}%
\pgfpathlineto{\pgfqpoint{2.337916in}{2.714392in}}%
\pgfpathlineto{\pgfqpoint{2.332027in}{2.722940in}}%
\pgfpathlineto{\pgfqpoint{2.326142in}{2.731461in}}%
\pgfpathlineto{\pgfqpoint{2.314871in}{2.725096in}}%
\pgfpathlineto{\pgfqpoint{2.303578in}{2.719679in}}%
\pgfpathlineto{\pgfqpoint{2.292263in}{2.715164in}}%
\pgfpathlineto{\pgfqpoint{2.280929in}{2.711365in}}%
\pgfpathlineto{\pgfqpoint{2.269580in}{2.708097in}}%
\pgfpathlineto{\pgfqpoint{2.275442in}{2.700138in}}%
\pgfpathlineto{\pgfqpoint{2.281308in}{2.692178in}}%
\pgfpathlineto{\pgfqpoint{2.287177in}{2.684225in}}%
\pgfpathlineto{\pgfqpoint{2.293050in}{2.676284in}}%
\pgfpathclose%
\pgfusepath{stroke,fill}%
\end{pgfscope}%
\begin{pgfscope}%
\pgfpathrectangle{\pgfqpoint{0.887500in}{0.275000in}}{\pgfqpoint{4.225000in}{4.225000in}}%
\pgfusepath{clip}%
\pgfsetbuttcap%
\pgfsetroundjoin%
\definecolor{currentfill}{rgb}{0.134692,0.658636,0.517649}%
\pgfsetfillcolor{currentfill}%
\pgfsetfillopacity{0.700000}%
\pgfsetlinewidth{0.501875pt}%
\definecolor{currentstroke}{rgb}{1.000000,1.000000,1.000000}%
\pgfsetstrokecolor{currentstroke}%
\pgfsetstrokeopacity{0.500000}%
\pgfsetdash{}{0pt}%
\pgfpathmoveto{\pgfqpoint{2.695857in}{2.820987in}}%
\pgfpathlineto{\pgfqpoint{2.707036in}{2.833353in}}%
\pgfpathlineto{\pgfqpoint{2.718208in}{2.846467in}}%
\pgfpathlineto{\pgfqpoint{2.729376in}{2.860100in}}%
\pgfpathlineto{\pgfqpoint{2.740543in}{2.873906in}}%
\pgfpathlineto{\pgfqpoint{2.751712in}{2.887541in}}%
\pgfpathlineto{\pgfqpoint{2.745678in}{2.896569in}}%
\pgfpathlineto{\pgfqpoint{2.739651in}{2.905117in}}%
\pgfpathlineto{\pgfqpoint{2.733632in}{2.913141in}}%
\pgfpathlineto{\pgfqpoint{2.727622in}{2.920702in}}%
\pgfpathlineto{\pgfqpoint{2.721618in}{2.927868in}}%
\pgfpathlineto{\pgfqpoint{2.710439in}{2.917600in}}%
\pgfpathlineto{\pgfqpoint{2.699266in}{2.906578in}}%
\pgfpathlineto{\pgfqpoint{2.688094in}{2.895266in}}%
\pgfpathlineto{\pgfqpoint{2.676919in}{2.884127in}}%
\pgfpathlineto{\pgfqpoint{2.665739in}{2.873484in}}%
\pgfpathlineto{\pgfqpoint{2.671752in}{2.863408in}}%
\pgfpathlineto{\pgfqpoint{2.677771in}{2.853061in}}%
\pgfpathlineto{\pgfqpoint{2.683795in}{2.842503in}}%
\pgfpathlineto{\pgfqpoint{2.689824in}{2.831793in}}%
\pgfpathclose%
\pgfusepath{stroke,fill}%
\end{pgfscope}%
\begin{pgfscope}%
\pgfpathrectangle{\pgfqpoint{0.887500in}{0.275000in}}{\pgfqpoint{4.225000in}{4.225000in}}%
\pgfusepath{clip}%
\pgfsetbuttcap%
\pgfsetroundjoin%
\definecolor{currentfill}{rgb}{0.120081,0.622161,0.534946}%
\pgfsetfillcolor{currentfill}%
\pgfsetfillopacity{0.700000}%
\pgfsetlinewidth{0.501875pt}%
\definecolor{currentstroke}{rgb}{1.000000,1.000000,1.000000}%
\pgfsetstrokecolor{currentstroke}%
\pgfsetstrokeopacity{0.500000}%
\pgfsetdash{}{0pt}%
\pgfpathmoveto{\pgfqpoint{1.524914in}{2.776069in}}%
\pgfpathlineto{\pgfqpoint{1.536463in}{2.779392in}}%
\pgfpathlineto{\pgfqpoint{1.548006in}{2.782715in}}%
\pgfpathlineto{\pgfqpoint{1.559543in}{2.786036in}}%
\pgfpathlineto{\pgfqpoint{1.571075in}{2.789355in}}%
\pgfpathlineto{\pgfqpoint{1.582601in}{2.792671in}}%
\pgfpathlineto{\pgfqpoint{1.576985in}{2.800143in}}%
\pgfpathlineto{\pgfqpoint{1.571373in}{2.807597in}}%
\pgfpathlineto{\pgfqpoint{1.565765in}{2.815032in}}%
\pgfpathlineto{\pgfqpoint{1.560162in}{2.822448in}}%
\pgfpathlineto{\pgfqpoint{1.554563in}{2.829844in}}%
\pgfpathlineto{\pgfqpoint{1.543051in}{2.826512in}}%
\pgfpathlineto{\pgfqpoint{1.531533in}{2.823179in}}%
\pgfpathlineto{\pgfqpoint{1.520010in}{2.819845in}}%
\pgfpathlineto{\pgfqpoint{1.508482in}{2.816510in}}%
\pgfpathlineto{\pgfqpoint{1.496947in}{2.813175in}}%
\pgfpathlineto{\pgfqpoint{1.502532in}{2.805790in}}%
\pgfpathlineto{\pgfqpoint{1.508121in}{2.798387in}}%
\pgfpathlineto{\pgfqpoint{1.513714in}{2.790965in}}%
\pgfpathlineto{\pgfqpoint{1.519312in}{2.783526in}}%
\pgfpathclose%
\pgfusepath{stroke,fill}%
\end{pgfscope}%
\begin{pgfscope}%
\pgfpathrectangle{\pgfqpoint{0.887500in}{0.275000in}}{\pgfqpoint{4.225000in}{4.225000in}}%
\pgfusepath{clip}%
\pgfsetbuttcap%
\pgfsetroundjoin%
\definecolor{currentfill}{rgb}{0.122606,0.585371,0.546557}%
\pgfsetfillcolor{currentfill}%
\pgfsetfillopacity{0.700000}%
\pgfsetlinewidth{0.501875pt}%
\definecolor{currentstroke}{rgb}{1.000000,1.000000,1.000000}%
\pgfsetstrokecolor{currentstroke}%
\pgfsetstrokeopacity{0.500000}%
\pgfsetdash{}{0pt}%
\pgfpathmoveto{\pgfqpoint{2.069613in}{2.698761in}}%
\pgfpathlineto{\pgfqpoint{2.081033in}{2.702013in}}%
\pgfpathlineto{\pgfqpoint{2.092447in}{2.705304in}}%
\pgfpathlineto{\pgfqpoint{2.103854in}{2.708656in}}%
\pgfpathlineto{\pgfqpoint{2.115253in}{2.712091in}}%
\pgfpathlineto{\pgfqpoint{2.126644in}{2.715633in}}%
\pgfpathlineto{\pgfqpoint{2.120829in}{2.723598in}}%
\pgfpathlineto{\pgfqpoint{2.115018in}{2.731551in}}%
\pgfpathlineto{\pgfqpoint{2.109211in}{2.739491in}}%
\pgfpathlineto{\pgfqpoint{2.103409in}{2.747419in}}%
\pgfpathlineto{\pgfqpoint{2.097610in}{2.755334in}}%
\pgfpathlineto{\pgfqpoint{2.086232in}{2.751748in}}%
\pgfpathlineto{\pgfqpoint{2.074846in}{2.748280in}}%
\pgfpathlineto{\pgfqpoint{2.063452in}{2.744906in}}%
\pgfpathlineto{\pgfqpoint{2.052051in}{2.741599in}}%
\pgfpathlineto{\pgfqpoint{2.040643in}{2.738335in}}%
\pgfpathlineto{\pgfqpoint{2.046429in}{2.730447in}}%
\pgfpathlineto{\pgfqpoint{2.052219in}{2.722546in}}%
\pgfpathlineto{\pgfqpoint{2.058013in}{2.714631in}}%
\pgfpathlineto{\pgfqpoint{2.063810in}{2.706703in}}%
\pgfpathclose%
\pgfusepath{stroke,fill}%
\end{pgfscope}%
\begin{pgfscope}%
\pgfpathrectangle{\pgfqpoint{0.887500in}{0.275000in}}{\pgfqpoint{4.225000in}{4.225000in}}%
\pgfusepath{clip}%
\pgfsetbuttcap%
\pgfsetroundjoin%
\definecolor{currentfill}{rgb}{0.121380,0.629492,0.531973}%
\pgfsetfillcolor{currentfill}%
\pgfsetfillopacity{0.700000}%
\pgfsetlinewidth{0.501875pt}%
\definecolor{currentstroke}{rgb}{1.000000,1.000000,1.000000}%
\pgfsetstrokecolor{currentstroke}%
\pgfsetstrokeopacity{0.500000}%
\pgfsetdash{}{0pt}%
\pgfpathmoveto{\pgfqpoint{2.639912in}{2.763695in}}%
\pgfpathlineto{\pgfqpoint{2.651101in}{2.775292in}}%
\pgfpathlineto{\pgfqpoint{2.662292in}{2.786597in}}%
\pgfpathlineto{\pgfqpoint{2.673484in}{2.797833in}}%
\pgfpathlineto{\pgfqpoint{2.684673in}{2.809221in}}%
\pgfpathlineto{\pgfqpoint{2.695857in}{2.820987in}}%
\pgfpathlineto{\pgfqpoint{2.689824in}{2.831793in}}%
\pgfpathlineto{\pgfqpoint{2.683795in}{2.842503in}}%
\pgfpathlineto{\pgfqpoint{2.677771in}{2.853061in}}%
\pgfpathlineto{\pgfqpoint{2.671752in}{2.863408in}}%
\pgfpathlineto{\pgfqpoint{2.665739in}{2.873484in}}%
\pgfpathlineto{\pgfqpoint{2.654553in}{2.863204in}}%
\pgfpathlineto{\pgfqpoint{2.643365in}{2.853061in}}%
\pgfpathlineto{\pgfqpoint{2.632176in}{2.842829in}}%
\pgfpathlineto{\pgfqpoint{2.620989in}{2.832281in}}%
\pgfpathlineto{\pgfqpoint{2.609807in}{2.821192in}}%
\pgfpathlineto{\pgfqpoint{2.615829in}{2.809092in}}%
\pgfpathlineto{\pgfqpoint{2.621852in}{2.797143in}}%
\pgfpathlineto{\pgfqpoint{2.627874in}{2.785509in}}%
\pgfpathlineto{\pgfqpoint{2.633894in}{2.774350in}}%
\pgfpathclose%
\pgfusepath{stroke,fill}%
\end{pgfscope}%
\begin{pgfscope}%
\pgfpathrectangle{\pgfqpoint{0.887500in}{0.275000in}}{\pgfqpoint{4.225000in}{4.225000in}}%
\pgfusepath{clip}%
\pgfsetbuttcap%
\pgfsetroundjoin%
\definecolor{currentfill}{rgb}{0.119738,0.603785,0.541400}%
\pgfsetfillcolor{currentfill}%
\pgfsetfillopacity{0.700000}%
\pgfsetlinewidth{0.501875pt}%
\definecolor{currentstroke}{rgb}{1.000000,1.000000,1.000000}%
\pgfsetstrokecolor{currentstroke}%
\pgfsetstrokeopacity{0.500000}%
\pgfsetdash{}{0pt}%
\pgfpathmoveto{\pgfqpoint{2.584024in}{2.699804in}}%
\pgfpathlineto{\pgfqpoint{2.595200in}{2.712707in}}%
\pgfpathlineto{\pgfqpoint{2.606374in}{2.725826in}}%
\pgfpathlineto{\pgfqpoint{2.617548in}{2.738879in}}%
\pgfpathlineto{\pgfqpoint{2.628727in}{2.751583in}}%
\pgfpathlineto{\pgfqpoint{2.639912in}{2.763695in}}%
\pgfpathlineto{\pgfqpoint{2.633894in}{2.774350in}}%
\pgfpathlineto{\pgfqpoint{2.627874in}{2.785509in}}%
\pgfpathlineto{\pgfqpoint{2.621852in}{2.797143in}}%
\pgfpathlineto{\pgfqpoint{2.615829in}{2.809092in}}%
\pgfpathlineto{\pgfqpoint{2.609807in}{2.821192in}}%
\pgfpathlineto{\pgfqpoint{2.598633in}{2.809336in}}%
\pgfpathlineto{\pgfqpoint{2.587468in}{2.796646in}}%
\pgfpathlineto{\pgfqpoint{2.576310in}{2.783394in}}%
\pgfpathlineto{\pgfqpoint{2.565154in}{2.769896in}}%
\pgfpathlineto{\pgfqpoint{2.553999in}{2.756468in}}%
\pgfpathlineto{\pgfqpoint{2.560010in}{2.744147in}}%
\pgfpathlineto{\pgfqpoint{2.566020in}{2.732189in}}%
\pgfpathlineto{\pgfqpoint{2.572026in}{2.720742in}}%
\pgfpathlineto{\pgfqpoint{2.578027in}{2.709945in}}%
\pgfpathclose%
\pgfusepath{stroke,fill}%
\end{pgfscope}%
\begin{pgfscope}%
\pgfpathrectangle{\pgfqpoint{0.887500in}{0.275000in}}{\pgfqpoint{4.225000in}{4.225000in}}%
\pgfusepath{clip}%
\pgfsetbuttcap%
\pgfsetroundjoin%
\definecolor{currentfill}{rgb}{0.824940,0.884720,0.106217}%
\pgfsetfillcolor{currentfill}%
\pgfsetfillopacity{0.700000}%
\pgfsetlinewidth{0.501875pt}%
\definecolor{currentstroke}{rgb}{1.000000,1.000000,1.000000}%
\pgfsetstrokecolor{currentstroke}%
\pgfsetstrokeopacity{0.500000}%
\pgfsetdash{}{0pt}%
\pgfpathmoveto{\pgfqpoint{3.254779in}{3.511434in}}%
\pgfpathlineto{\pgfqpoint{3.265919in}{3.515226in}}%
\pgfpathlineto{\pgfqpoint{3.277053in}{3.518964in}}%
\pgfpathlineto{\pgfqpoint{3.288182in}{3.522598in}}%
\pgfpathlineto{\pgfqpoint{3.299305in}{3.526082in}}%
\pgfpathlineto{\pgfqpoint{3.310421in}{3.529366in}}%
\pgfpathlineto{\pgfqpoint{3.304199in}{3.535207in}}%
\pgfpathlineto{\pgfqpoint{3.297980in}{3.540841in}}%
\pgfpathlineto{\pgfqpoint{3.291766in}{3.546283in}}%
\pgfpathlineto{\pgfqpoint{3.285554in}{3.551541in}}%
\pgfpathlineto{\pgfqpoint{3.279347in}{3.556621in}}%
\pgfpathlineto{\pgfqpoint{3.268246in}{3.553541in}}%
\pgfpathlineto{\pgfqpoint{3.257139in}{3.550358in}}%
\pgfpathlineto{\pgfqpoint{3.246026in}{3.547100in}}%
\pgfpathlineto{\pgfqpoint{3.234908in}{3.543795in}}%
\pgfpathlineto{\pgfqpoint{3.223785in}{3.540470in}}%
\pgfpathlineto{\pgfqpoint{3.229976in}{3.535102in}}%
\pgfpathlineto{\pgfqpoint{3.236171in}{3.529505in}}%
\pgfpathlineto{\pgfqpoint{3.242370in}{3.523691in}}%
\pgfpathlineto{\pgfqpoint{3.248573in}{3.517668in}}%
\pgfpathclose%
\pgfusepath{stroke,fill}%
\end{pgfscope}%
\begin{pgfscope}%
\pgfpathrectangle{\pgfqpoint{0.887500in}{0.275000in}}{\pgfqpoint{4.225000in}{4.225000in}}%
\pgfusepath{clip}%
\pgfsetbuttcap%
\pgfsetroundjoin%
\definecolor{currentfill}{rgb}{0.120092,0.600104,0.542530}%
\pgfsetfillcolor{currentfill}%
\pgfsetfillopacity{0.700000}%
\pgfsetlinewidth{0.501875pt}%
\definecolor{currentstroke}{rgb}{1.000000,1.000000,1.000000}%
\pgfsetstrokecolor{currentstroke}%
\pgfsetstrokeopacity{0.500000}%
\pgfsetdash{}{0pt}%
\pgfpathmoveto{\pgfqpoint{1.840214in}{2.727565in}}%
\pgfpathlineto{\pgfqpoint{1.851691in}{2.730873in}}%
\pgfpathlineto{\pgfqpoint{1.863162in}{2.734191in}}%
\pgfpathlineto{\pgfqpoint{1.874627in}{2.737521in}}%
\pgfpathlineto{\pgfqpoint{1.886087in}{2.740860in}}%
\pgfpathlineto{\pgfqpoint{1.897540in}{2.744210in}}%
\pgfpathlineto{\pgfqpoint{1.891806in}{2.751969in}}%
\pgfpathlineto{\pgfqpoint{1.886076in}{2.759716in}}%
\pgfpathlineto{\pgfqpoint{1.880349in}{2.767450in}}%
\pgfpathlineto{\pgfqpoint{1.874627in}{2.775172in}}%
\pgfpathlineto{\pgfqpoint{1.868909in}{2.782882in}}%
\pgfpathlineto{\pgfqpoint{1.857469in}{2.779517in}}%
\pgfpathlineto{\pgfqpoint{1.846023in}{2.776163in}}%
\pgfpathlineto{\pgfqpoint{1.834571in}{2.772821in}}%
\pgfpathlineto{\pgfqpoint{1.823113in}{2.769490in}}%
\pgfpathlineto{\pgfqpoint{1.811650in}{2.766171in}}%
\pgfpathlineto{\pgfqpoint{1.817354in}{2.758477in}}%
\pgfpathlineto{\pgfqpoint{1.823063in}{2.750771in}}%
\pgfpathlineto{\pgfqpoint{1.828776in}{2.743050in}}%
\pgfpathlineto{\pgfqpoint{1.834493in}{2.735315in}}%
\pgfpathclose%
\pgfusepath{stroke,fill}%
\end{pgfscope}%
\begin{pgfscope}%
\pgfpathrectangle{\pgfqpoint{0.887500in}{0.275000in}}{\pgfqpoint{4.225000in}{4.225000in}}%
\pgfusepath{clip}%
\pgfsetbuttcap%
\pgfsetroundjoin%
\definecolor{currentfill}{rgb}{0.125394,0.574318,0.549086}%
\pgfsetfillcolor{currentfill}%
\pgfsetfillopacity{0.700000}%
\pgfsetlinewidth{0.501875pt}%
\definecolor{currentstroke}{rgb}{1.000000,1.000000,1.000000}%
\pgfsetstrokecolor{currentstroke}%
\pgfsetstrokeopacity{0.500000}%
\pgfsetdash{}{0pt}%
\pgfpathmoveto{\pgfqpoint{2.527979in}{2.647602in}}%
\pgfpathlineto{\pgfqpoint{2.539219in}{2.655853in}}%
\pgfpathlineto{\pgfqpoint{2.550442in}{2.665218in}}%
\pgfpathlineto{\pgfqpoint{2.561649in}{2.675779in}}%
\pgfpathlineto{\pgfqpoint{2.572841in}{2.687401in}}%
\pgfpathlineto{\pgfqpoint{2.584024in}{2.699804in}}%
\pgfpathlineto{\pgfqpoint{2.578027in}{2.709945in}}%
\pgfpathlineto{\pgfqpoint{2.572026in}{2.720742in}}%
\pgfpathlineto{\pgfqpoint{2.566020in}{2.732189in}}%
\pgfpathlineto{\pgfqpoint{2.560010in}{2.744147in}}%
\pgfpathlineto{\pgfqpoint{2.553999in}{2.756468in}}%
\pgfpathlineto{\pgfqpoint{2.542838in}{2.743428in}}%
\pgfpathlineto{\pgfqpoint{2.531669in}{2.731090in}}%
\pgfpathlineto{\pgfqpoint{2.520485in}{2.719767in}}%
\pgfpathlineto{\pgfqpoint{2.509284in}{2.709611in}}%
\pgfpathlineto{\pgfqpoint{2.498067in}{2.700531in}}%
\pgfpathlineto{\pgfqpoint{2.504055in}{2.689067in}}%
\pgfpathlineto{\pgfqpoint{2.510042in}{2.677968in}}%
\pgfpathlineto{\pgfqpoint{2.516024in}{2.667319in}}%
\pgfpathlineto{\pgfqpoint{2.522003in}{2.657199in}}%
\pgfpathclose%
\pgfusepath{stroke,fill}%
\end{pgfscope}%
\begin{pgfscope}%
\pgfpathrectangle{\pgfqpoint{0.887500in}{0.275000in}}{\pgfqpoint{4.225000in}{4.225000in}}%
\pgfusepath{clip}%
\pgfsetbuttcap%
\pgfsetroundjoin%
\definecolor{currentfill}{rgb}{0.814576,0.883393,0.110347}%
\pgfsetfillcolor{currentfill}%
\pgfsetfillopacity{0.700000}%
\pgfsetlinewidth{0.501875pt}%
\definecolor{currentstroke}{rgb}{1.000000,1.000000,1.000000}%
\pgfsetstrokecolor{currentstroke}%
\pgfsetstrokeopacity{0.500000}%
\pgfsetdash{}{0pt}%
\pgfpathmoveto{\pgfqpoint{3.483927in}{3.492012in}}%
\pgfpathlineto{\pgfqpoint{3.495013in}{3.495629in}}%
\pgfpathlineto{\pgfqpoint{3.506093in}{3.499233in}}%
\pgfpathlineto{\pgfqpoint{3.517169in}{3.502823in}}%
\pgfpathlineto{\pgfqpoint{3.528239in}{3.506398in}}%
\pgfpathlineto{\pgfqpoint{3.539305in}{3.509958in}}%
\pgfpathlineto{\pgfqpoint{3.533035in}{3.519105in}}%
\pgfpathlineto{\pgfqpoint{3.526761in}{3.527636in}}%
\pgfpathlineto{\pgfqpoint{3.520485in}{3.535590in}}%
\pgfpathlineto{\pgfqpoint{3.514208in}{3.543049in}}%
\pgfpathlineto{\pgfqpoint{3.507930in}{3.550093in}}%
\pgfpathlineto{\pgfqpoint{3.496870in}{3.546018in}}%
\pgfpathlineto{\pgfqpoint{3.485805in}{3.541939in}}%
\pgfpathlineto{\pgfqpoint{3.474736in}{3.537890in}}%
\pgfpathlineto{\pgfqpoint{3.463662in}{3.533902in}}%
\pgfpathlineto{\pgfqpoint{3.452585in}{3.530009in}}%
\pgfpathlineto{\pgfqpoint{3.458855in}{3.523462in}}%
\pgfpathlineto{\pgfqpoint{3.465125in}{3.516473in}}%
\pgfpathlineto{\pgfqpoint{3.471395in}{3.508952in}}%
\pgfpathlineto{\pgfqpoint{3.477662in}{3.500813in}}%
\pgfpathclose%
\pgfusepath{stroke,fill}%
\end{pgfscope}%
\begin{pgfscope}%
\pgfpathrectangle{\pgfqpoint{0.887500in}{0.275000in}}{\pgfqpoint{4.225000in}{4.225000in}}%
\pgfusepath{clip}%
\pgfsetbuttcap%
\pgfsetroundjoin%
\definecolor{currentfill}{rgb}{0.127568,0.566949,0.550556}%
\pgfsetfillcolor{currentfill}%
\pgfsetfillopacity{0.700000}%
\pgfsetlinewidth{0.501875pt}%
\definecolor{currentstroke}{rgb}{1.000000,1.000000,1.000000}%
\pgfsetstrokecolor{currentstroke}%
\pgfsetstrokeopacity{0.500000}%
\pgfsetdash{}{0pt}%
\pgfpathmoveto{\pgfqpoint{2.385161in}{2.646087in}}%
\pgfpathlineto{\pgfqpoint{2.396499in}{2.649835in}}%
\pgfpathlineto{\pgfqpoint{2.407829in}{2.653782in}}%
\pgfpathlineto{\pgfqpoint{2.419149in}{2.657988in}}%
\pgfpathlineto{\pgfqpoint{2.430459in}{2.662514in}}%
\pgfpathlineto{\pgfqpoint{2.441759in}{2.667420in}}%
\pgfpathlineto{\pgfqpoint{2.435800in}{2.677740in}}%
\pgfpathlineto{\pgfqpoint{2.429843in}{2.688211in}}%
\pgfpathlineto{\pgfqpoint{2.423887in}{2.698788in}}%
\pgfpathlineto{\pgfqpoint{2.417933in}{2.709426in}}%
\pgfpathlineto{\pgfqpoint{2.411982in}{2.720080in}}%
\pgfpathlineto{\pgfqpoint{2.400726in}{2.713111in}}%
\pgfpathlineto{\pgfqpoint{2.389462in}{2.706442in}}%
\pgfpathlineto{\pgfqpoint{2.378188in}{2.700121in}}%
\pgfpathlineto{\pgfqpoint{2.366903in}{2.694193in}}%
\pgfpathlineto{\pgfqpoint{2.355607in}{2.688706in}}%
\pgfpathlineto{\pgfqpoint{2.361511in}{2.680152in}}%
\pgfpathlineto{\pgfqpoint{2.367418in}{2.671610in}}%
\pgfpathlineto{\pgfqpoint{2.373329in}{2.663082in}}%
\pgfpathlineto{\pgfqpoint{2.379243in}{2.654574in}}%
\pgfpathclose%
\pgfusepath{stroke,fill}%
\end{pgfscope}%
\begin{pgfscope}%
\pgfpathrectangle{\pgfqpoint{0.887500in}{0.275000in}}{\pgfqpoint{4.225000in}{4.225000in}}%
\pgfusepath{clip}%
\pgfsetbuttcap%
\pgfsetroundjoin%
\definecolor{currentfill}{rgb}{0.119483,0.614817,0.537692}%
\pgfsetfillcolor{currentfill}%
\pgfsetfillopacity{0.700000}%
\pgfsetlinewidth{0.501875pt}%
\definecolor{currentstroke}{rgb}{1.000000,1.000000,1.000000}%
\pgfsetstrokecolor{currentstroke}%
\pgfsetstrokeopacity{0.500000}%
\pgfsetdash{}{0pt}%
\pgfpathmoveto{\pgfqpoint{1.610748in}{2.755041in}}%
\pgfpathlineto{\pgfqpoint{1.622284in}{2.758336in}}%
\pgfpathlineto{\pgfqpoint{1.633813in}{2.761626in}}%
\pgfpathlineto{\pgfqpoint{1.645337in}{2.764913in}}%
\pgfpathlineto{\pgfqpoint{1.656856in}{2.768197in}}%
\pgfpathlineto{\pgfqpoint{1.668369in}{2.771479in}}%
\pgfpathlineto{\pgfqpoint{1.662717in}{2.779054in}}%
\pgfpathlineto{\pgfqpoint{1.657069in}{2.786612in}}%
\pgfpathlineto{\pgfqpoint{1.651426in}{2.794155in}}%
\pgfpathlineto{\pgfqpoint{1.645787in}{2.801681in}}%
\pgfpathlineto{\pgfqpoint{1.640152in}{2.809191in}}%
\pgfpathlineto{\pgfqpoint{1.628653in}{2.805892in}}%
\pgfpathlineto{\pgfqpoint{1.617148in}{2.802592in}}%
\pgfpathlineto{\pgfqpoint{1.605638in}{2.799289in}}%
\pgfpathlineto{\pgfqpoint{1.594122in}{2.795982in}}%
\pgfpathlineto{\pgfqpoint{1.582601in}{2.792671in}}%
\pgfpathlineto{\pgfqpoint{1.588222in}{2.785180in}}%
\pgfpathlineto{\pgfqpoint{1.593847in}{2.777671in}}%
\pgfpathlineto{\pgfqpoint{1.599477in}{2.770145in}}%
\pgfpathlineto{\pgfqpoint{1.605110in}{2.762602in}}%
\pgfpathclose%
\pgfusepath{stroke,fill}%
\end{pgfscope}%
\begin{pgfscope}%
\pgfpathrectangle{\pgfqpoint{0.887500in}{0.275000in}}{\pgfqpoint{4.225000in}{4.225000in}}%
\pgfusepath{clip}%
\pgfsetbuttcap%
\pgfsetroundjoin%
\definecolor{currentfill}{rgb}{0.124395,0.578002,0.548287}%
\pgfsetfillcolor{currentfill}%
\pgfsetfillopacity{0.700000}%
\pgfsetlinewidth{0.501875pt}%
\definecolor{currentstroke}{rgb}{1.000000,1.000000,1.000000}%
\pgfsetstrokecolor{currentstroke}%
\pgfsetstrokeopacity{0.500000}%
\pgfsetdash{}{0pt}%
\pgfpathmoveto{\pgfqpoint{2.155780in}{2.675613in}}%
\pgfpathlineto{\pgfqpoint{2.167178in}{2.679183in}}%
\pgfpathlineto{\pgfqpoint{2.178570in}{2.682795in}}%
\pgfpathlineto{\pgfqpoint{2.189956in}{2.686393in}}%
\pgfpathlineto{\pgfqpoint{2.201339in}{2.689926in}}%
\pgfpathlineto{\pgfqpoint{2.212718in}{2.693338in}}%
\pgfpathlineto{\pgfqpoint{2.206869in}{2.701437in}}%
\pgfpathlineto{\pgfqpoint{2.201024in}{2.709518in}}%
\pgfpathlineto{\pgfqpoint{2.195183in}{2.717580in}}%
\pgfpathlineto{\pgfqpoint{2.189347in}{2.725622in}}%
\pgfpathlineto{\pgfqpoint{2.183514in}{2.733642in}}%
\pgfpathlineto{\pgfqpoint{2.172147in}{2.730228in}}%
\pgfpathlineto{\pgfqpoint{2.160778in}{2.726650in}}%
\pgfpathlineto{\pgfqpoint{2.149405in}{2.722977in}}%
\pgfpathlineto{\pgfqpoint{2.138028in}{2.719281in}}%
\pgfpathlineto{\pgfqpoint{2.126644in}{2.715633in}}%
\pgfpathlineto{\pgfqpoint{2.132464in}{2.707654in}}%
\pgfpathlineto{\pgfqpoint{2.138287in}{2.699663in}}%
\pgfpathlineto{\pgfqpoint{2.144114in}{2.691659in}}%
\pgfpathlineto{\pgfqpoint{2.149945in}{2.683643in}}%
\pgfpathclose%
\pgfusepath{stroke,fill}%
\end{pgfscope}%
\begin{pgfscope}%
\pgfpathrectangle{\pgfqpoint{0.887500in}{0.275000in}}{\pgfqpoint{4.225000in}{4.225000in}}%
\pgfusepath{clip}%
\pgfsetbuttcap%
\pgfsetroundjoin%
\definecolor{currentfill}{rgb}{0.783315,0.879285,0.125405}%
\pgfsetfillcolor{currentfill}%
\pgfsetfillopacity{0.700000}%
\pgfsetlinewidth{0.501875pt}%
\definecolor{currentstroke}{rgb}{1.000000,1.000000,1.000000}%
\pgfsetstrokecolor{currentstroke}%
\pgfsetstrokeopacity{0.500000}%
\pgfsetdash{}{0pt}%
\pgfpathmoveto{\pgfqpoint{2.969639in}{3.490667in}}%
\pgfpathlineto{\pgfqpoint{2.980852in}{3.493231in}}%
\pgfpathlineto{\pgfqpoint{2.992061in}{3.494902in}}%
\pgfpathlineto{\pgfqpoint{3.003264in}{3.496178in}}%
\pgfpathlineto{\pgfqpoint{3.014460in}{3.497553in}}%
\pgfpathlineto{\pgfqpoint{3.025649in}{3.499522in}}%
\pgfpathlineto{\pgfqpoint{3.019536in}{3.501901in}}%
\pgfpathlineto{\pgfqpoint{3.013428in}{3.504367in}}%
\pgfpathlineto{\pgfqpoint{3.007325in}{3.506975in}}%
\pgfpathlineto{\pgfqpoint{3.001228in}{3.509685in}}%
\pgfpathlineto{\pgfqpoint{2.995137in}{3.512429in}}%
\pgfpathlineto{\pgfqpoint{2.983964in}{3.510802in}}%
\pgfpathlineto{\pgfqpoint{2.972784in}{3.509821in}}%
\pgfpathlineto{\pgfqpoint{2.961597in}{3.509005in}}%
\pgfpathlineto{\pgfqpoint{2.950404in}{3.507873in}}%
\pgfpathlineto{\pgfqpoint{2.939207in}{3.505941in}}%
\pgfpathlineto{\pgfqpoint{2.945282in}{3.502958in}}%
\pgfpathlineto{\pgfqpoint{2.951363in}{3.499840in}}%
\pgfpathlineto{\pgfqpoint{2.957450in}{3.496714in}}%
\pgfpathlineto{\pgfqpoint{2.963542in}{3.493667in}}%
\pgfpathclose%
\pgfusepath{stroke,fill}%
\end{pgfscope}%
\begin{pgfscope}%
\pgfpathrectangle{\pgfqpoint{0.887500in}{0.275000in}}{\pgfqpoint{4.225000in}{4.225000in}}%
\pgfusepath{clip}%
\pgfsetbuttcap%
\pgfsetroundjoin%
\definecolor{currentfill}{rgb}{0.814576,0.883393,0.110347}%
\pgfsetfillcolor{currentfill}%
\pgfsetfillopacity{0.700000}%
\pgfsetlinewidth{0.501875pt}%
\definecolor{currentstroke}{rgb}{1.000000,1.000000,1.000000}%
\pgfsetstrokecolor{currentstroke}%
\pgfsetstrokeopacity{0.500000}%
\pgfsetdash{}{0pt}%
\pgfpathmoveto{\pgfqpoint{3.112257in}{3.503936in}}%
\pgfpathlineto{\pgfqpoint{3.123434in}{3.508200in}}%
\pgfpathlineto{\pgfqpoint{3.134606in}{3.512389in}}%
\pgfpathlineto{\pgfqpoint{3.145772in}{3.516375in}}%
\pgfpathlineto{\pgfqpoint{3.156933in}{3.520144in}}%
\pgfpathlineto{\pgfqpoint{3.168089in}{3.523735in}}%
\pgfpathlineto{\pgfqpoint{3.161917in}{3.529159in}}%
\pgfpathlineto{\pgfqpoint{3.155749in}{3.534267in}}%
\pgfpathlineto{\pgfqpoint{3.149585in}{3.539027in}}%
\pgfpathlineto{\pgfqpoint{3.143425in}{3.543408in}}%
\pgfpathlineto{\pgfqpoint{3.137270in}{3.547379in}}%
\pgfpathlineto{\pgfqpoint{3.126129in}{3.543453in}}%
\pgfpathlineto{\pgfqpoint{3.114983in}{3.539067in}}%
\pgfpathlineto{\pgfqpoint{3.103832in}{3.534127in}}%
\pgfpathlineto{\pgfqpoint{3.092676in}{3.528651in}}%
\pgfpathlineto{\pgfqpoint{3.081515in}{3.522890in}}%
\pgfpathlineto{\pgfqpoint{3.087653in}{3.519700in}}%
\pgfpathlineto{\pgfqpoint{3.093797in}{3.516218in}}%
\pgfpathlineto{\pgfqpoint{3.099945in}{3.512436in}}%
\pgfpathlineto{\pgfqpoint{3.106099in}{3.508345in}}%
\pgfpathclose%
\pgfusepath{stroke,fill}%
\end{pgfscope}%
\begin{pgfscope}%
\pgfpathrectangle{\pgfqpoint{0.887500in}{0.275000in}}{\pgfqpoint{4.225000in}{4.225000in}}%
\pgfusepath{clip}%
\pgfsetbuttcap%
\pgfsetroundjoin%
\definecolor{currentfill}{rgb}{0.772852,0.877868,0.131109}%
\pgfsetfillcolor{currentfill}%
\pgfsetfillopacity{0.700000}%
\pgfsetlinewidth{0.501875pt}%
\definecolor{currentstroke}{rgb}{1.000000,1.000000,1.000000}%
\pgfsetstrokecolor{currentstroke}%
\pgfsetstrokeopacity{0.500000}%
\pgfsetdash{}{0pt}%
\pgfpathmoveto{\pgfqpoint{3.570626in}{3.457452in}}%
\pgfpathlineto{\pgfqpoint{3.581690in}{3.460769in}}%
\pgfpathlineto{\pgfqpoint{3.592749in}{3.464053in}}%
\pgfpathlineto{\pgfqpoint{3.603801in}{3.467284in}}%
\pgfpathlineto{\pgfqpoint{3.614848in}{3.470443in}}%
\pgfpathlineto{\pgfqpoint{3.625887in}{3.473511in}}%
\pgfpathlineto{\pgfqpoint{3.619619in}{3.484734in}}%
\pgfpathlineto{\pgfqpoint{3.613352in}{3.495764in}}%
\pgfpathlineto{\pgfqpoint{3.607084in}{3.506530in}}%
\pgfpathlineto{\pgfqpoint{3.600815in}{3.516960in}}%
\pgfpathlineto{\pgfqpoint{3.594544in}{3.526985in}}%
\pgfpathlineto{\pgfqpoint{3.583509in}{3.523752in}}%
\pgfpathlineto{\pgfqpoint{3.572467in}{3.520412in}}%
\pgfpathlineto{\pgfqpoint{3.561419in}{3.516985in}}%
\pgfpathlineto{\pgfqpoint{3.550364in}{3.513493in}}%
\pgfpathlineto{\pgfqpoint{3.539305in}{3.509958in}}%
\pgfpathlineto{\pgfqpoint{3.545572in}{3.500255in}}%
\pgfpathlineto{\pgfqpoint{3.551837in}{3.490075in}}%
\pgfpathlineto{\pgfqpoint{3.558100in}{3.479496in}}%
\pgfpathlineto{\pgfqpoint{3.564363in}{3.468596in}}%
\pgfpathclose%
\pgfusepath{stroke,fill}%
\end{pgfscope}%
\begin{pgfscope}%
\pgfpathrectangle{\pgfqpoint{0.887500in}{0.275000in}}{\pgfqpoint{4.225000in}{4.225000in}}%
\pgfusepath{clip}%
\pgfsetbuttcap%
\pgfsetroundjoin%
\definecolor{currentfill}{rgb}{0.720391,0.870350,0.162603}%
\pgfsetfillcolor{currentfill}%
\pgfsetfillopacity{0.700000}%
\pgfsetlinewidth{0.501875pt}%
\definecolor{currentstroke}{rgb}{1.000000,1.000000,1.000000}%
\pgfsetstrokecolor{currentstroke}%
\pgfsetstrokeopacity{0.500000}%
\pgfsetdash{}{0pt}%
\pgfpathmoveto{\pgfqpoint{3.657268in}{3.416962in}}%
\pgfpathlineto{\pgfqpoint{3.668308in}{3.420034in}}%
\pgfpathlineto{\pgfqpoint{3.679342in}{3.423026in}}%
\pgfpathlineto{\pgfqpoint{3.690369in}{3.425958in}}%
\pgfpathlineto{\pgfqpoint{3.701391in}{3.428880in}}%
\pgfpathlineto{\pgfqpoint{3.712408in}{3.431844in}}%
\pgfpathlineto{\pgfqpoint{3.706114in}{3.442782in}}%
\pgfpathlineto{\pgfqpoint{3.699826in}{3.453868in}}%
\pgfpathlineto{\pgfqpoint{3.693542in}{3.465051in}}%
\pgfpathlineto{\pgfqpoint{3.687262in}{3.476277in}}%
\pgfpathlineto{\pgfqpoint{3.680985in}{3.487494in}}%
\pgfpathlineto{\pgfqpoint{3.669977in}{3.484732in}}%
\pgfpathlineto{\pgfqpoint{3.658964in}{3.482026in}}%
\pgfpathlineto{\pgfqpoint{3.647946in}{3.479298in}}%
\pgfpathlineto{\pgfqpoint{3.636920in}{3.476468in}}%
\pgfpathlineto{\pgfqpoint{3.625887in}{3.473511in}}%
\pgfpathlineto{\pgfqpoint{3.632157in}{3.462164in}}%
\pgfpathlineto{\pgfqpoint{3.638428in}{3.450766in}}%
\pgfpathlineto{\pgfqpoint{3.644704in}{3.439387in}}%
\pgfpathlineto{\pgfqpoint{3.650983in}{3.428097in}}%
\pgfpathclose%
\pgfusepath{stroke,fill}%
\end{pgfscope}%
\begin{pgfscope}%
\pgfpathrectangle{\pgfqpoint{0.887500in}{0.275000in}}{\pgfqpoint{4.225000in}{4.225000in}}%
\pgfusepath{clip}%
\pgfsetbuttcap%
\pgfsetroundjoin%
\definecolor{currentfill}{rgb}{0.626579,0.854645,0.223353}%
\pgfsetfillcolor{currentfill}%
\pgfsetfillopacity{0.700000}%
\pgfsetlinewidth{0.501875pt}%
\definecolor{currentstroke}{rgb}{1.000000,1.000000,1.000000}%
\pgfsetstrokecolor{currentstroke}%
\pgfsetstrokeopacity{0.500000}%
\pgfsetdash{}{0pt}%
\pgfpathmoveto{\pgfqpoint{2.858151in}{3.270921in}}%
\pgfpathlineto{\pgfqpoint{2.869155in}{3.319072in}}%
\pgfpathlineto{\pgfqpoint{2.880206in}{3.361060in}}%
\pgfpathlineto{\pgfqpoint{2.891305in}{3.395457in}}%
\pgfpathlineto{\pgfqpoint{2.902441in}{3.422933in}}%
\pgfpathlineto{\pgfqpoint{2.913607in}{3.444346in}}%
\pgfpathlineto{\pgfqpoint{2.907528in}{3.448473in}}%
\pgfpathlineto{\pgfqpoint{2.901454in}{3.452748in}}%
\pgfpathlineto{\pgfqpoint{2.895384in}{3.457192in}}%
\pgfpathlineto{\pgfqpoint{2.889319in}{3.461660in}}%
\pgfpathlineto{\pgfqpoint{2.883260in}{3.465950in}}%
\pgfpathlineto{\pgfqpoint{2.872110in}{3.447488in}}%
\pgfpathlineto{\pgfqpoint{2.860988in}{3.423970in}}%
\pgfpathlineto{\pgfqpoint{2.849901in}{3.394733in}}%
\pgfpathlineto{\pgfqpoint{2.838856in}{3.359250in}}%
\pgfpathlineto{\pgfqpoint{2.827854in}{3.318601in}}%
\pgfpathlineto{\pgfqpoint{2.833920in}{3.306649in}}%
\pgfpathlineto{\pgfqpoint{2.839984in}{3.295537in}}%
\pgfpathlineto{\pgfqpoint{2.846044in}{3.285736in}}%
\pgfpathlineto{\pgfqpoint{2.852099in}{3.277567in}}%
\pgfpathclose%
\pgfusepath{stroke,fill}%
\end{pgfscope}%
\begin{pgfscope}%
\pgfpathrectangle{\pgfqpoint{0.887500in}{0.275000in}}{\pgfqpoint{4.225000in}{4.225000in}}%
\pgfusepath{clip}%
\pgfsetbuttcap%
\pgfsetroundjoin%
\definecolor{currentfill}{rgb}{0.668054,0.861999,0.196293}%
\pgfsetfillcolor{currentfill}%
\pgfsetfillopacity{0.700000}%
\pgfsetlinewidth{0.501875pt}%
\definecolor{currentstroke}{rgb}{1.000000,1.000000,1.000000}%
\pgfsetstrokecolor{currentstroke}%
\pgfsetstrokeopacity{0.500000}%
\pgfsetdash{}{0pt}%
\pgfpathmoveto{\pgfqpoint{3.743951in}{3.378538in}}%
\pgfpathlineto{\pgfqpoint{3.754971in}{3.381651in}}%
\pgfpathlineto{\pgfqpoint{3.765987in}{3.384794in}}%
\pgfpathlineto{\pgfqpoint{3.776998in}{3.387987in}}%
\pgfpathlineto{\pgfqpoint{3.788006in}{3.391245in}}%
\pgfpathlineto{\pgfqpoint{3.799009in}{3.394573in}}%
\pgfpathlineto{\pgfqpoint{3.792691in}{3.405436in}}%
\pgfpathlineto{\pgfqpoint{3.786376in}{3.416293in}}%
\pgfpathlineto{\pgfqpoint{3.780064in}{3.427167in}}%
\pgfpathlineto{\pgfqpoint{3.773757in}{3.438080in}}%
\pgfpathlineto{\pgfqpoint{3.767454in}{3.449056in}}%
\pgfpathlineto{\pgfqpoint{3.756447in}{3.445161in}}%
\pgfpathlineto{\pgfqpoint{3.745440in}{3.441510in}}%
\pgfpathlineto{\pgfqpoint{3.734431in}{3.438107in}}%
\pgfpathlineto{\pgfqpoint{3.723421in}{3.434902in}}%
\pgfpathlineto{\pgfqpoint{3.712408in}{3.431844in}}%
\pgfpathlineto{\pgfqpoint{3.718708in}{3.421048in}}%
\pgfpathlineto{\pgfqpoint{3.725012in}{3.410357in}}%
\pgfpathlineto{\pgfqpoint{3.731321in}{3.399734in}}%
\pgfpathlineto{\pgfqpoint{3.737634in}{3.389140in}}%
\pgfpathclose%
\pgfusepath{stroke,fill}%
\end{pgfscope}%
\begin{pgfscope}%
\pgfpathrectangle{\pgfqpoint{0.887500in}{0.275000in}}{\pgfqpoint{4.225000in}{4.225000in}}%
\pgfusepath{clip}%
\pgfsetbuttcap%
\pgfsetroundjoin%
\definecolor{currentfill}{rgb}{0.121148,0.592739,0.544641}%
\pgfsetfillcolor{currentfill}%
\pgfsetfillopacity{0.700000}%
\pgfsetlinewidth{0.501875pt}%
\definecolor{currentstroke}{rgb}{1.000000,1.000000,1.000000}%
\pgfsetstrokecolor{currentstroke}%
\pgfsetstrokeopacity{0.500000}%
\pgfsetdash{}{0pt}%
\pgfpathmoveto{\pgfqpoint{1.926275in}{2.705192in}}%
\pgfpathlineto{\pgfqpoint{1.937736in}{2.708539in}}%
\pgfpathlineto{\pgfqpoint{1.949191in}{2.711891in}}%
\pgfpathlineto{\pgfqpoint{1.960641in}{2.715244in}}%
\pgfpathlineto{\pgfqpoint{1.972085in}{2.718593in}}%
\pgfpathlineto{\pgfqpoint{1.983524in}{2.721931in}}%
\pgfpathlineto{\pgfqpoint{1.977755in}{2.729784in}}%
\pgfpathlineto{\pgfqpoint{1.971991in}{2.737621in}}%
\pgfpathlineto{\pgfqpoint{1.966231in}{2.745444in}}%
\pgfpathlineto{\pgfqpoint{1.960475in}{2.753252in}}%
\pgfpathlineto{\pgfqpoint{1.954723in}{2.761046in}}%
\pgfpathlineto{\pgfqpoint{1.943297in}{2.757682in}}%
\pgfpathlineto{\pgfqpoint{1.931866in}{2.754311in}}%
\pgfpathlineto{\pgfqpoint{1.920430in}{2.750938in}}%
\pgfpathlineto{\pgfqpoint{1.908988in}{2.747570in}}%
\pgfpathlineto{\pgfqpoint{1.897540in}{2.744210in}}%
\pgfpathlineto{\pgfqpoint{1.903279in}{2.736437in}}%
\pgfpathlineto{\pgfqpoint{1.909022in}{2.728650in}}%
\pgfpathlineto{\pgfqpoint{1.914768in}{2.720847in}}%
\pgfpathlineto{\pgfqpoint{1.920519in}{2.713028in}}%
\pgfpathclose%
\pgfusepath{stroke,fill}%
\end{pgfscope}%
\begin{pgfscope}%
\pgfpathrectangle{\pgfqpoint{0.887500in}{0.275000in}}{\pgfqpoint{4.225000in}{4.225000in}}%
\pgfusepath{clip}%
\pgfsetbuttcap%
\pgfsetroundjoin%
\definecolor{currentfill}{rgb}{0.120638,0.625828,0.533488}%
\pgfsetfillcolor{currentfill}%
\pgfsetfillopacity{0.700000}%
\pgfsetlinewidth{0.501875pt}%
\definecolor{currentstroke}{rgb}{1.000000,1.000000,1.000000}%
\pgfsetstrokecolor{currentstroke}%
\pgfsetstrokeopacity{0.500000}%
\pgfsetdash{}{0pt}%
\pgfpathmoveto{\pgfqpoint{1.381302in}{2.779698in}}%
\pgfpathlineto{\pgfqpoint{1.392891in}{2.783064in}}%
\pgfpathlineto{\pgfqpoint{1.404475in}{2.786424in}}%
\pgfpathlineto{\pgfqpoint{1.416053in}{2.789780in}}%
\pgfpathlineto{\pgfqpoint{1.427626in}{2.793130in}}%
\pgfpathlineto{\pgfqpoint{1.439193in}{2.796477in}}%
\pgfpathlineto{\pgfqpoint{1.433628in}{2.803824in}}%
\pgfpathlineto{\pgfqpoint{1.428067in}{2.811152in}}%
\pgfpathlineto{\pgfqpoint{1.422510in}{2.818460in}}%
\pgfpathlineto{\pgfqpoint{1.416958in}{2.825749in}}%
\pgfpathlineto{\pgfqpoint{1.405403in}{2.822374in}}%
\pgfpathlineto{\pgfqpoint{1.393843in}{2.818992in}}%
\pgfpathlineto{\pgfqpoint{1.382277in}{2.815604in}}%
\pgfpathlineto{\pgfqpoint{1.370706in}{2.812208in}}%
\pgfpathlineto{\pgfqpoint{1.359130in}{2.808805in}}%
\pgfpathlineto{\pgfqpoint{1.364666in}{2.801563in}}%
\pgfpathlineto{\pgfqpoint{1.370207in}{2.794298in}}%
\pgfpathlineto{\pgfqpoint{1.375752in}{2.787009in}}%
\pgfpathclose%
\pgfusepath{stroke,fill}%
\end{pgfscope}%
\begin{pgfscope}%
\pgfpathrectangle{\pgfqpoint{0.887500in}{0.275000in}}{\pgfqpoint{4.225000in}{4.225000in}}%
\pgfusepath{clip}%
\pgfsetbuttcap%
\pgfsetroundjoin%
\definecolor{currentfill}{rgb}{0.506271,0.828786,0.300362}%
\pgfsetfillcolor{currentfill}%
\pgfsetfillopacity{0.700000}%
\pgfsetlinewidth{0.501875pt}%
\definecolor{currentstroke}{rgb}{1.000000,1.000000,1.000000}%
\pgfsetstrokecolor{currentstroke}%
\pgfsetstrokeopacity{0.500000}%
\pgfsetdash{}{0pt}%
\pgfpathmoveto{\pgfqpoint{4.003846in}{3.253165in}}%
\pgfpathlineto{\pgfqpoint{4.014798in}{3.256075in}}%
\pgfpathlineto{\pgfqpoint{4.025744in}{3.258949in}}%
\pgfpathlineto{\pgfqpoint{4.036683in}{3.261785in}}%
\pgfpathlineto{\pgfqpoint{4.047616in}{3.264584in}}%
\pgfpathlineto{\pgfqpoint{4.058543in}{3.267344in}}%
\pgfpathlineto{\pgfqpoint{4.052188in}{3.279431in}}%
\pgfpathlineto{\pgfqpoint{4.045837in}{3.291526in}}%
\pgfpathlineto{\pgfqpoint{4.039488in}{3.303634in}}%
\pgfpathlineto{\pgfqpoint{4.033143in}{3.315758in}}%
\pgfpathlineto{\pgfqpoint{4.026802in}{3.327902in}}%
\pgfpathlineto{\pgfqpoint{4.015867in}{3.324622in}}%
\pgfpathlineto{\pgfqpoint{4.004927in}{3.321369in}}%
\pgfpathlineto{\pgfqpoint{3.993983in}{3.318145in}}%
\pgfpathlineto{\pgfqpoint{3.983034in}{3.314951in}}%
\pgfpathlineto{\pgfqpoint{3.972081in}{3.311788in}}%
\pgfpathlineto{\pgfqpoint{3.978430in}{3.300141in}}%
\pgfpathlineto{\pgfqpoint{3.984781in}{3.288475in}}%
\pgfpathlineto{\pgfqpoint{3.991135in}{3.276771in}}%
\pgfpathlineto{\pgfqpoint{3.997490in}{3.265007in}}%
\pgfpathclose%
\pgfusepath{stroke,fill}%
\end{pgfscope}%
\begin{pgfscope}%
\pgfpathrectangle{\pgfqpoint{0.887500in}{0.275000in}}{\pgfqpoint{4.225000in}{4.225000in}}%
\pgfusepath{clip}%
\pgfsetbuttcap%
\pgfsetroundjoin%
\definecolor{currentfill}{rgb}{0.449368,0.813768,0.335384}%
\pgfsetfillcolor{currentfill}%
\pgfsetfillopacity{0.700000}%
\pgfsetlinewidth{0.501875pt}%
\definecolor{currentstroke}{rgb}{1.000000,1.000000,1.000000}%
\pgfsetstrokecolor{currentstroke}%
\pgfsetstrokeopacity{0.500000}%
\pgfsetdash{}{0pt}%
\pgfpathmoveto{\pgfqpoint{4.090362in}{3.206890in}}%
\pgfpathlineto{\pgfqpoint{4.101288in}{3.209665in}}%
\pgfpathlineto{\pgfqpoint{4.112207in}{3.212411in}}%
\pgfpathlineto{\pgfqpoint{4.123120in}{3.215127in}}%
\pgfpathlineto{\pgfqpoint{4.134027in}{3.217811in}}%
\pgfpathlineto{\pgfqpoint{4.127648in}{3.229661in}}%
\pgfpathlineto{\pgfqpoint{4.121273in}{3.241579in}}%
\pgfpathlineto{\pgfqpoint{4.114904in}{3.253589in}}%
\pgfpathlineto{\pgfqpoint{4.108541in}{3.265720in}}%
\pgfpathlineto{\pgfqpoint{4.102185in}{3.277997in}}%
\pgfpathlineto{\pgfqpoint{4.091284in}{3.275393in}}%
\pgfpathlineto{\pgfqpoint{4.080377in}{3.272749in}}%
\pgfpathlineto{\pgfqpoint{4.069463in}{3.270066in}}%
\pgfpathlineto{\pgfqpoint{4.058543in}{3.267344in}}%
\pgfpathlineto{\pgfqpoint{4.064901in}{3.255262in}}%
\pgfpathlineto{\pgfqpoint{4.071262in}{3.243181in}}%
\pgfpathlineto{\pgfqpoint{4.077626in}{3.231095in}}%
\pgfpathlineto{\pgfqpoint{4.083993in}{3.218999in}}%
\pgfpathclose%
\pgfusepath{stroke,fill}%
\end{pgfscope}%
\begin{pgfscope}%
\pgfpathrectangle{\pgfqpoint{0.887500in}{0.275000in}}{\pgfqpoint{4.225000in}{4.225000in}}%
\pgfusepath{clip}%
\pgfsetbuttcap%
\pgfsetroundjoin%
\definecolor{currentfill}{rgb}{0.565498,0.842430,0.262877}%
\pgfsetfillcolor{currentfill}%
\pgfsetfillopacity{0.700000}%
\pgfsetlinewidth{0.501875pt}%
\definecolor{currentstroke}{rgb}{1.000000,1.000000,1.000000}%
\pgfsetstrokecolor{currentstroke}%
\pgfsetstrokeopacity{0.500000}%
\pgfsetdash{}{0pt}%
\pgfpathmoveto{\pgfqpoint{3.917247in}{3.296469in}}%
\pgfpathlineto{\pgfqpoint{3.928223in}{3.299463in}}%
\pgfpathlineto{\pgfqpoint{3.939194in}{3.302493in}}%
\pgfpathlineto{\pgfqpoint{3.950161in}{3.305557in}}%
\pgfpathlineto{\pgfqpoint{3.961123in}{3.308656in}}%
\pgfpathlineto{\pgfqpoint{3.972081in}{3.311788in}}%
\pgfpathlineto{\pgfqpoint{3.965736in}{3.323435in}}%
\pgfpathlineto{\pgfqpoint{3.959394in}{3.335102in}}%
\pgfpathlineto{\pgfqpoint{3.953056in}{3.346810in}}%
\pgfpathlineto{\pgfqpoint{3.946722in}{3.358578in}}%
\pgfpathlineto{\pgfqpoint{3.940394in}{3.370425in}}%
\pgfpathlineto{\pgfqpoint{3.929434in}{3.366968in}}%
\pgfpathlineto{\pgfqpoint{3.918471in}{3.363601in}}%
\pgfpathlineto{\pgfqpoint{3.907505in}{3.360326in}}%
\pgfpathlineto{\pgfqpoint{3.896535in}{3.357144in}}%
\pgfpathlineto{\pgfqpoint{3.885562in}{3.354054in}}%
\pgfpathlineto{\pgfqpoint{3.891893in}{3.342525in}}%
\pgfpathlineto{\pgfqpoint{3.898227in}{3.331014in}}%
\pgfpathlineto{\pgfqpoint{3.904564in}{3.319510in}}%
\pgfpathlineto{\pgfqpoint{3.910904in}{3.308000in}}%
\pgfpathclose%
\pgfusepath{stroke,fill}%
\end{pgfscope}%
\begin{pgfscope}%
\pgfpathrectangle{\pgfqpoint{0.887500in}{0.275000in}}{\pgfqpoint{4.225000in}{4.225000in}}%
\pgfusepath{clip}%
\pgfsetbuttcap%
\pgfsetroundjoin%
\definecolor{currentfill}{rgb}{0.616293,0.852709,0.230052}%
\pgfsetfillcolor{currentfill}%
\pgfsetfillopacity{0.700000}%
\pgfsetlinewidth{0.501875pt}%
\definecolor{currentstroke}{rgb}{1.000000,1.000000,1.000000}%
\pgfsetstrokecolor{currentstroke}%
\pgfsetstrokeopacity{0.500000}%
\pgfsetdash{}{0pt}%
\pgfpathmoveto{\pgfqpoint{3.830634in}{3.339404in}}%
\pgfpathlineto{\pgfqpoint{3.841630in}{3.342308in}}%
\pgfpathlineto{\pgfqpoint{3.852620in}{3.345200in}}%
\pgfpathlineto{\pgfqpoint{3.863605in}{3.348106in}}%
\pgfpathlineto{\pgfqpoint{3.874586in}{3.351049in}}%
\pgfpathlineto{\pgfqpoint{3.885562in}{3.354054in}}%
\pgfpathlineto{\pgfqpoint{3.879236in}{3.365614in}}%
\pgfpathlineto{\pgfqpoint{3.872913in}{3.377204in}}%
\pgfpathlineto{\pgfqpoint{3.866594in}{3.388822in}}%
\pgfpathlineto{\pgfqpoint{3.860278in}{3.400462in}}%
\pgfpathlineto{\pgfqpoint{3.853966in}{3.412120in}}%
\pgfpathlineto{\pgfqpoint{3.842983in}{3.408505in}}%
\pgfpathlineto{\pgfqpoint{3.831995in}{3.404938in}}%
\pgfpathlineto{\pgfqpoint{3.821004in}{3.401424in}}%
\pgfpathlineto{\pgfqpoint{3.810008in}{3.397968in}}%
\pgfpathlineto{\pgfqpoint{3.799009in}{3.394573in}}%
\pgfpathlineto{\pgfqpoint{3.805330in}{3.383683in}}%
\pgfpathlineto{\pgfqpoint{3.811653in}{3.372744in}}%
\pgfpathlineto{\pgfqpoint{3.817979in}{3.361732in}}%
\pgfpathlineto{\pgfqpoint{3.824306in}{3.350627in}}%
\pgfpathclose%
\pgfusepath{stroke,fill}%
\end{pgfscope}%
\begin{pgfscope}%
\pgfpathrectangle{\pgfqpoint{0.887500in}{0.275000in}}{\pgfqpoint{4.225000in}{4.225000in}}%
\pgfusepath{clip}%
\pgfsetbuttcap%
\pgfsetroundjoin%
\definecolor{currentfill}{rgb}{0.129933,0.559582,0.551864}%
\pgfsetfillcolor{currentfill}%
\pgfsetfillopacity{0.700000}%
\pgfsetlinewidth{0.501875pt}%
\definecolor{currentstroke}{rgb}{1.000000,1.000000,1.000000}%
\pgfsetstrokecolor{currentstroke}%
\pgfsetstrokeopacity{0.500000}%
\pgfsetdash{}{0pt}%
\pgfpathmoveto{\pgfqpoint{2.471542in}{2.619593in}}%
\pgfpathlineto{\pgfqpoint{2.482858in}{2.623759in}}%
\pgfpathlineto{\pgfqpoint{2.494161in}{2.628544in}}%
\pgfpathlineto{\pgfqpoint{2.505449in}{2.634047in}}%
\pgfpathlineto{\pgfqpoint{2.516722in}{2.640367in}}%
\pgfpathlineto{\pgfqpoint{2.527979in}{2.647602in}}%
\pgfpathlineto{\pgfqpoint{2.522003in}{2.657199in}}%
\pgfpathlineto{\pgfqpoint{2.516024in}{2.667319in}}%
\pgfpathlineto{\pgfqpoint{2.510042in}{2.677968in}}%
\pgfpathlineto{\pgfqpoint{2.504055in}{2.689067in}}%
\pgfpathlineto{\pgfqpoint{2.498067in}{2.700531in}}%
\pgfpathlineto{\pgfqpoint{2.486834in}{2.692416in}}%
\pgfpathlineto{\pgfqpoint{2.475586in}{2.685157in}}%
\pgfpathlineto{\pgfqpoint{2.464323in}{2.678645in}}%
\pgfpathlineto{\pgfqpoint{2.453047in}{2.672769in}}%
\pgfpathlineto{\pgfqpoint{2.441759in}{2.667420in}}%
\pgfpathlineto{\pgfqpoint{2.447717in}{2.657296in}}%
\pgfpathlineto{\pgfqpoint{2.453675in}{2.647413in}}%
\pgfpathlineto{\pgfqpoint{2.459632in}{2.637815in}}%
\pgfpathlineto{\pgfqpoint{2.465587in}{2.628545in}}%
\pgfpathclose%
\pgfusepath{stroke,fill}%
\end{pgfscope}%
\begin{pgfscope}%
\pgfpathrectangle{\pgfqpoint{0.887500in}{0.275000in}}{\pgfqpoint{4.225000in}{4.225000in}}%
\pgfusepath{clip}%
\pgfsetbuttcap%
\pgfsetroundjoin%
\definecolor{currentfill}{rgb}{0.814576,0.883393,0.110347}%
\pgfsetfillcolor{currentfill}%
\pgfsetfillopacity{0.700000}%
\pgfsetlinewidth{0.501875pt}%
\definecolor{currentstroke}{rgb}{1.000000,1.000000,1.000000}%
\pgfsetstrokecolor{currentstroke}%
\pgfsetstrokeopacity{0.500000}%
\pgfsetdash{}{0pt}%
\pgfpathmoveto{\pgfqpoint{3.341576in}{3.496250in}}%
\pgfpathlineto{\pgfqpoint{3.352701in}{3.499824in}}%
\pgfpathlineto{\pgfqpoint{3.363821in}{3.503214in}}%
\pgfpathlineto{\pgfqpoint{3.374934in}{3.506463in}}%
\pgfpathlineto{\pgfqpoint{3.386041in}{3.509629in}}%
\pgfpathlineto{\pgfqpoint{3.397142in}{3.512772in}}%
\pgfpathlineto{\pgfqpoint{3.390889in}{3.519255in}}%
\pgfpathlineto{\pgfqpoint{3.384637in}{3.525436in}}%
\pgfpathlineto{\pgfqpoint{3.378388in}{3.531375in}}%
\pgfpathlineto{\pgfqpoint{3.372143in}{3.537135in}}%
\pgfpathlineto{\pgfqpoint{3.365901in}{3.542775in}}%
\pgfpathlineto{\pgfqpoint{3.354818in}{3.540249in}}%
\pgfpathlineto{\pgfqpoint{3.343728in}{3.537749in}}%
\pgfpathlineto{\pgfqpoint{3.332633in}{3.535168in}}%
\pgfpathlineto{\pgfqpoint{3.321531in}{3.532401in}}%
\pgfpathlineto{\pgfqpoint{3.310421in}{3.529366in}}%
\pgfpathlineto{\pgfqpoint{3.316647in}{3.523294in}}%
\pgfpathlineto{\pgfqpoint{3.322875in}{3.516970in}}%
\pgfpathlineto{\pgfqpoint{3.329106in}{3.510370in}}%
\pgfpathlineto{\pgfqpoint{3.335340in}{3.503471in}}%
\pgfpathclose%
\pgfusepath{stroke,fill}%
\end{pgfscope}%
\begin{pgfscope}%
\pgfpathrectangle{\pgfqpoint{0.887500in}{0.275000in}}{\pgfqpoint{4.225000in}{4.225000in}}%
\pgfusepath{clip}%
\pgfsetbuttcap%
\pgfsetroundjoin%
\definecolor{currentfill}{rgb}{0.126453,0.570633,0.549841}%
\pgfsetfillcolor{currentfill}%
\pgfsetfillopacity{0.700000}%
\pgfsetlinewidth{0.501875pt}%
\definecolor{currentstroke}{rgb}{1.000000,1.000000,1.000000}%
\pgfsetstrokecolor{currentstroke}%
\pgfsetstrokeopacity{0.500000}%
\pgfsetdash{}{0pt}%
\pgfpathmoveto{\pgfqpoint{2.242025in}{2.652613in}}%
\pgfpathlineto{\pgfqpoint{2.253413in}{2.655923in}}%
\pgfpathlineto{\pgfqpoint{2.264799in}{2.659108in}}%
\pgfpathlineto{\pgfqpoint{2.276180in}{2.662177in}}%
\pgfpathlineto{\pgfqpoint{2.287557in}{2.665224in}}%
\pgfpathlineto{\pgfqpoint{2.298927in}{2.668351in}}%
\pgfpathlineto{\pgfqpoint{2.293050in}{2.676284in}}%
\pgfpathlineto{\pgfqpoint{2.287177in}{2.684225in}}%
\pgfpathlineto{\pgfqpoint{2.281308in}{2.692178in}}%
\pgfpathlineto{\pgfqpoint{2.275442in}{2.700138in}}%
\pgfpathlineto{\pgfqpoint{2.269580in}{2.708097in}}%
\pgfpathlineto{\pgfqpoint{2.258219in}{2.705170in}}%
\pgfpathlineto{\pgfqpoint{2.246848in}{2.702397in}}%
\pgfpathlineto{\pgfqpoint{2.235473in}{2.699589in}}%
\pgfpathlineto{\pgfqpoint{2.224096in}{2.696577in}}%
\pgfpathlineto{\pgfqpoint{2.212718in}{2.693338in}}%
\pgfpathlineto{\pgfqpoint{2.218571in}{2.685222in}}%
\pgfpathlineto{\pgfqpoint{2.224429in}{2.677091in}}%
\pgfpathlineto{\pgfqpoint{2.230290in}{2.668945in}}%
\pgfpathlineto{\pgfqpoint{2.236156in}{2.660785in}}%
\pgfpathclose%
\pgfusepath{stroke,fill}%
\end{pgfscope}%
\begin{pgfscope}%
\pgfpathrectangle{\pgfqpoint{0.887500in}{0.275000in}}{\pgfqpoint{4.225000in}{4.225000in}}%
\pgfusepath{clip}%
\pgfsetbuttcap%
\pgfsetroundjoin%
\definecolor{currentfill}{rgb}{0.119512,0.607464,0.540218}%
\pgfsetfillcolor{currentfill}%
\pgfsetfillopacity{0.700000}%
\pgfsetlinewidth{0.501875pt}%
\definecolor{currentstroke}{rgb}{1.000000,1.000000,1.000000}%
\pgfsetstrokecolor{currentstroke}%
\pgfsetstrokeopacity{0.500000}%
\pgfsetdash{}{0pt}%
\pgfpathmoveto{\pgfqpoint{1.696693in}{2.733374in}}%
\pgfpathlineto{\pgfqpoint{1.708214in}{2.736644in}}%
\pgfpathlineto{\pgfqpoint{1.719730in}{2.739913in}}%
\pgfpathlineto{\pgfqpoint{1.731240in}{2.743181in}}%
\pgfpathlineto{\pgfqpoint{1.742744in}{2.746450in}}%
\pgfpathlineto{\pgfqpoint{1.754243in}{2.749721in}}%
\pgfpathlineto{\pgfqpoint{1.748556in}{2.757385in}}%
\pgfpathlineto{\pgfqpoint{1.742874in}{2.765034in}}%
\pgfpathlineto{\pgfqpoint{1.737195in}{2.772668in}}%
\pgfpathlineto{\pgfqpoint{1.731521in}{2.780287in}}%
\pgfpathlineto{\pgfqpoint{1.725851in}{2.787892in}}%
\pgfpathlineto{\pgfqpoint{1.714366in}{2.784606in}}%
\pgfpathlineto{\pgfqpoint{1.702875in}{2.781322in}}%
\pgfpathlineto{\pgfqpoint{1.691379in}{2.778041in}}%
\pgfpathlineto{\pgfqpoint{1.679877in}{2.774760in}}%
\pgfpathlineto{\pgfqpoint{1.668369in}{2.771479in}}%
\pgfpathlineto{\pgfqpoint{1.674026in}{2.763889in}}%
\pgfpathlineto{\pgfqpoint{1.679686in}{2.756283in}}%
\pgfpathlineto{\pgfqpoint{1.685351in}{2.748662in}}%
\pgfpathlineto{\pgfqpoint{1.691020in}{2.741026in}}%
\pgfpathclose%
\pgfusepath{stroke,fill}%
\end{pgfscope}%
\begin{pgfscope}%
\pgfpathrectangle{\pgfqpoint{0.887500in}{0.275000in}}{\pgfqpoint{4.225000in}{4.225000in}}%
\pgfusepath{clip}%
\pgfsetbuttcap%
\pgfsetroundjoin%
\definecolor{currentfill}{rgb}{0.120081,0.622161,0.534946}%
\pgfsetfillcolor{currentfill}%
\pgfsetfillopacity{0.700000}%
\pgfsetlinewidth{0.501875pt}%
\definecolor{currentstroke}{rgb}{1.000000,1.000000,1.000000}%
\pgfsetstrokecolor{currentstroke}%
\pgfsetstrokeopacity{0.500000}%
\pgfsetdash{}{0pt}%
\pgfpathmoveto{\pgfqpoint{1.467087in}{2.759462in}}%
\pgfpathlineto{\pgfqpoint{1.478664in}{2.762783in}}%
\pgfpathlineto{\pgfqpoint{1.490235in}{2.766103in}}%
\pgfpathlineto{\pgfqpoint{1.501800in}{2.769424in}}%
\pgfpathlineto{\pgfqpoint{1.513360in}{2.772746in}}%
\pgfpathlineto{\pgfqpoint{1.524914in}{2.776069in}}%
\pgfpathlineto{\pgfqpoint{1.519312in}{2.783526in}}%
\pgfpathlineto{\pgfqpoint{1.513714in}{2.790965in}}%
\pgfpathlineto{\pgfqpoint{1.508121in}{2.798387in}}%
\pgfpathlineto{\pgfqpoint{1.502532in}{2.805790in}}%
\pgfpathlineto{\pgfqpoint{1.496947in}{2.813175in}}%
\pgfpathlineto{\pgfqpoint{1.485408in}{2.809838in}}%
\pgfpathlineto{\pgfqpoint{1.473862in}{2.806500in}}%
\pgfpathlineto{\pgfqpoint{1.462311in}{2.803162in}}%
\pgfpathlineto{\pgfqpoint{1.450755in}{2.799821in}}%
\pgfpathlineto{\pgfqpoint{1.439193in}{2.796477in}}%
\pgfpathlineto{\pgfqpoint{1.444763in}{2.789111in}}%
\pgfpathlineto{\pgfqpoint{1.450337in}{2.781727in}}%
\pgfpathlineto{\pgfqpoint{1.455916in}{2.774323in}}%
\pgfpathlineto{\pgfqpoint{1.461499in}{2.766902in}}%
\pgfpathclose%
\pgfusepath{stroke,fill}%
\end{pgfscope}%
\begin{pgfscope}%
\pgfpathrectangle{\pgfqpoint{0.887500in}{0.275000in}}{\pgfqpoint{4.225000in}{4.225000in}}%
\pgfusepath{clip}%
\pgfsetbuttcap%
\pgfsetroundjoin%
\definecolor{currentfill}{rgb}{0.122606,0.585371,0.546557}%
\pgfsetfillcolor{currentfill}%
\pgfsetfillopacity{0.700000}%
\pgfsetlinewidth{0.501875pt}%
\definecolor{currentstroke}{rgb}{1.000000,1.000000,1.000000}%
\pgfsetstrokecolor{currentstroke}%
\pgfsetstrokeopacity{0.500000}%
\pgfsetdash{}{0pt}%
\pgfpathmoveto{\pgfqpoint{2.012429in}{2.682418in}}%
\pgfpathlineto{\pgfqpoint{2.023876in}{2.685726in}}%
\pgfpathlineto{\pgfqpoint{2.035318in}{2.689016in}}%
\pgfpathlineto{\pgfqpoint{2.046754in}{2.692283in}}%
\pgfpathlineto{\pgfqpoint{2.058186in}{2.695525in}}%
\pgfpathlineto{\pgfqpoint{2.069613in}{2.698761in}}%
\pgfpathlineto{\pgfqpoint{2.063810in}{2.706703in}}%
\pgfpathlineto{\pgfqpoint{2.058013in}{2.714631in}}%
\pgfpathlineto{\pgfqpoint{2.052219in}{2.722546in}}%
\pgfpathlineto{\pgfqpoint{2.046429in}{2.730447in}}%
\pgfpathlineto{\pgfqpoint{2.040643in}{2.738335in}}%
\pgfpathlineto{\pgfqpoint{2.029230in}{2.735090in}}%
\pgfpathlineto{\pgfqpoint{2.017810in}{2.731837in}}%
\pgfpathlineto{\pgfqpoint{2.006386in}{2.728559in}}%
\pgfpathlineto{\pgfqpoint{1.994958in}{2.725255in}}%
\pgfpathlineto{\pgfqpoint{1.983524in}{2.721931in}}%
\pgfpathlineto{\pgfqpoint{1.989296in}{2.714063in}}%
\pgfpathlineto{\pgfqpoint{1.995073in}{2.706177in}}%
\pgfpathlineto{\pgfqpoint{2.000854in}{2.698274in}}%
\pgfpathlineto{\pgfqpoint{2.006640in}{2.690354in}}%
\pgfpathclose%
\pgfusepath{stroke,fill}%
\end{pgfscope}%
\begin{pgfscope}%
\pgfpathrectangle{\pgfqpoint{0.887500in}{0.275000in}}{\pgfqpoint{4.225000in}{4.225000in}}%
\pgfusepath{clip}%
\pgfsetbuttcap%
\pgfsetroundjoin%
\definecolor{currentfill}{rgb}{0.124780,0.640461,0.527068}%
\pgfsetfillcolor{currentfill}%
\pgfsetfillopacity{0.700000}%
\pgfsetlinewidth{0.501875pt}%
\definecolor{currentstroke}{rgb}{1.000000,1.000000,1.000000}%
\pgfsetstrokecolor{currentstroke}%
\pgfsetstrokeopacity{0.500000}%
\pgfsetdash{}{0pt}%
\pgfpathmoveto{\pgfqpoint{2.726065in}{2.767419in}}%
\pgfpathlineto{\pgfqpoint{2.737260in}{2.779483in}}%
\pgfpathlineto{\pgfqpoint{2.748445in}{2.792861in}}%
\pgfpathlineto{\pgfqpoint{2.759622in}{2.807434in}}%
\pgfpathlineto{\pgfqpoint{2.770792in}{2.822967in}}%
\pgfpathlineto{\pgfqpoint{2.781959in}{2.839226in}}%
\pgfpathlineto{\pgfqpoint{2.775902in}{2.848926in}}%
\pgfpathlineto{\pgfqpoint{2.769848in}{2.858724in}}%
\pgfpathlineto{\pgfqpoint{2.763798in}{2.868503in}}%
\pgfpathlineto{\pgfqpoint{2.757752in}{2.878147in}}%
\pgfpathlineto{\pgfqpoint{2.751712in}{2.887541in}}%
\pgfpathlineto{\pgfqpoint{2.740543in}{2.873906in}}%
\pgfpathlineto{\pgfqpoint{2.729376in}{2.860100in}}%
\pgfpathlineto{\pgfqpoint{2.718208in}{2.846467in}}%
\pgfpathlineto{\pgfqpoint{2.707036in}{2.833353in}}%
\pgfpathlineto{\pgfqpoint{2.695857in}{2.820987in}}%
\pgfpathlineto{\pgfqpoint{2.701894in}{2.810139in}}%
\pgfpathlineto{\pgfqpoint{2.707933in}{2.799303in}}%
\pgfpathlineto{\pgfqpoint{2.713975in}{2.788535in}}%
\pgfpathlineto{\pgfqpoint{2.720019in}{2.777889in}}%
\pgfpathclose%
\pgfusepath{stroke,fill}%
\end{pgfscope}%
\begin{pgfscope}%
\pgfpathrectangle{\pgfqpoint{0.887500in}{0.275000in}}{\pgfqpoint{4.225000in}{4.225000in}}%
\pgfusepath{clip}%
\pgfsetbuttcap%
\pgfsetroundjoin%
\definecolor{currentfill}{rgb}{0.824940,0.884720,0.106217}%
\pgfsetfillcolor{currentfill}%
\pgfsetfillopacity{0.700000}%
\pgfsetlinewidth{0.501875pt}%
\definecolor{currentstroke}{rgb}{1.000000,1.000000,1.000000}%
\pgfsetstrokecolor{currentstroke}%
\pgfsetstrokeopacity{0.500000}%
\pgfsetdash{}{0pt}%
\pgfpathmoveto{\pgfqpoint{3.199008in}{3.492958in}}%
\pgfpathlineto{\pgfqpoint{3.210172in}{3.496521in}}%
\pgfpathlineto{\pgfqpoint{3.221331in}{3.500164in}}%
\pgfpathlineto{\pgfqpoint{3.232485in}{3.503873in}}%
\pgfpathlineto{\pgfqpoint{3.243635in}{3.507636in}}%
\pgfpathlineto{\pgfqpoint{3.254779in}{3.511434in}}%
\pgfpathlineto{\pgfqpoint{3.248573in}{3.517668in}}%
\pgfpathlineto{\pgfqpoint{3.242370in}{3.523691in}}%
\pgfpathlineto{\pgfqpoint{3.236171in}{3.529505in}}%
\pgfpathlineto{\pgfqpoint{3.229976in}{3.535102in}}%
\pgfpathlineto{\pgfqpoint{3.223785in}{3.540470in}}%
\pgfpathlineto{\pgfqpoint{3.212656in}{3.537154in}}%
\pgfpathlineto{\pgfqpoint{3.201522in}{3.533859in}}%
\pgfpathlineto{\pgfqpoint{3.190383in}{3.530551in}}%
\pgfpathlineto{\pgfqpoint{3.179239in}{3.527190in}}%
\pgfpathlineto{\pgfqpoint{3.168089in}{3.523735in}}%
\pgfpathlineto{\pgfqpoint{3.174265in}{3.518027in}}%
\pgfpathlineto{\pgfqpoint{3.180445in}{3.512067in}}%
\pgfpathlineto{\pgfqpoint{3.186628in}{3.505886in}}%
\pgfpathlineto{\pgfqpoint{3.192816in}{3.499515in}}%
\pgfpathclose%
\pgfusepath{stroke,fill}%
\end{pgfscope}%
\begin{pgfscope}%
\pgfpathrectangle{\pgfqpoint{0.887500in}{0.275000in}}{\pgfqpoint{4.225000in}{4.225000in}}%
\pgfusepath{clip}%
\pgfsetbuttcap%
\pgfsetroundjoin%
\definecolor{currentfill}{rgb}{0.119483,0.614817,0.537692}%
\pgfsetfillcolor{currentfill}%
\pgfsetfillopacity{0.700000}%
\pgfsetlinewidth{0.501875pt}%
\definecolor{currentstroke}{rgb}{1.000000,1.000000,1.000000}%
\pgfsetstrokecolor{currentstroke}%
\pgfsetstrokeopacity{0.500000}%
\pgfsetdash{}{0pt}%
\pgfpathmoveto{\pgfqpoint{2.669993in}{2.715907in}}%
\pgfpathlineto{\pgfqpoint{2.681211in}{2.726180in}}%
\pgfpathlineto{\pgfqpoint{2.692430in}{2.736125in}}%
\pgfpathlineto{\pgfqpoint{2.703647in}{2.746083in}}%
\pgfpathlineto{\pgfqpoint{2.714859in}{2.756400in}}%
\pgfpathlineto{\pgfqpoint{2.726065in}{2.767419in}}%
\pgfpathlineto{\pgfqpoint{2.720019in}{2.777889in}}%
\pgfpathlineto{\pgfqpoint{2.713975in}{2.788535in}}%
\pgfpathlineto{\pgfqpoint{2.707933in}{2.799303in}}%
\pgfpathlineto{\pgfqpoint{2.701894in}{2.810139in}}%
\pgfpathlineto{\pgfqpoint{2.695857in}{2.820987in}}%
\pgfpathlineto{\pgfqpoint{2.684673in}{2.809221in}}%
\pgfpathlineto{\pgfqpoint{2.673484in}{2.797833in}}%
\pgfpathlineto{\pgfqpoint{2.662292in}{2.786597in}}%
\pgfpathlineto{\pgfqpoint{2.651101in}{2.775292in}}%
\pgfpathlineto{\pgfqpoint{2.639912in}{2.763695in}}%
\pgfpathlineto{\pgfqpoint{2.645928in}{2.753486in}}%
\pgfpathlineto{\pgfqpoint{2.651943in}{2.743663in}}%
\pgfpathlineto{\pgfqpoint{2.657959in}{2.734166in}}%
\pgfpathlineto{\pgfqpoint{2.663975in}{2.724934in}}%
\pgfpathclose%
\pgfusepath{stroke,fill}%
\end{pgfscope}%
\begin{pgfscope}%
\pgfpathrectangle{\pgfqpoint{0.887500in}{0.275000in}}{\pgfqpoint{4.225000in}{4.225000in}}%
\pgfusepath{clip}%
\pgfsetbuttcap%
\pgfsetroundjoin%
\definecolor{currentfill}{rgb}{0.128729,0.563265,0.551229}%
\pgfsetfillcolor{currentfill}%
\pgfsetfillopacity{0.700000}%
\pgfsetlinewidth{0.501875pt}%
\definecolor{currentstroke}{rgb}{1.000000,1.000000,1.000000}%
\pgfsetstrokecolor{currentstroke}%
\pgfsetstrokeopacity{0.500000}%
\pgfsetdash{}{0pt}%
\pgfpathmoveto{\pgfqpoint{2.328369in}{2.628608in}}%
\pgfpathlineto{\pgfqpoint{2.339739in}{2.632023in}}%
\pgfpathlineto{\pgfqpoint{2.351104in}{2.635466in}}%
\pgfpathlineto{\pgfqpoint{2.362463in}{2.638948in}}%
\pgfpathlineto{\pgfqpoint{2.373815in}{2.642478in}}%
\pgfpathlineto{\pgfqpoint{2.385161in}{2.646087in}}%
\pgfpathlineto{\pgfqpoint{2.379243in}{2.654574in}}%
\pgfpathlineto{\pgfqpoint{2.373329in}{2.663082in}}%
\pgfpathlineto{\pgfqpoint{2.367418in}{2.671610in}}%
\pgfpathlineto{\pgfqpoint{2.361511in}{2.680152in}}%
\pgfpathlineto{\pgfqpoint{2.355607in}{2.688706in}}%
\pgfpathlineto{\pgfqpoint{2.344298in}{2.683707in}}%
\pgfpathlineto{\pgfqpoint{2.332974in}{2.679237in}}%
\pgfpathlineto{\pgfqpoint{2.321637in}{2.675255in}}%
\pgfpathlineto{\pgfqpoint{2.310287in}{2.671661in}}%
\pgfpathlineto{\pgfqpoint{2.298927in}{2.668351in}}%
\pgfpathlineto{\pgfqpoint{2.304807in}{2.660421in}}%
\pgfpathlineto{\pgfqpoint{2.310691in}{2.652487in}}%
\pgfpathlineto{\pgfqpoint{2.316580in}{2.644544in}}%
\pgfpathlineto{\pgfqpoint{2.322472in}{2.636586in}}%
\pgfpathclose%
\pgfusepath{stroke,fill}%
\end{pgfscope}%
\begin{pgfscope}%
\pgfpathrectangle{\pgfqpoint{0.887500in}{0.275000in}}{\pgfqpoint{4.225000in}{4.225000in}}%
\pgfusepath{clip}%
\pgfsetbuttcap%
\pgfsetroundjoin%
\definecolor{currentfill}{rgb}{0.804182,0.882046,0.114965}%
\pgfsetfillcolor{currentfill}%
\pgfsetfillopacity{0.700000}%
\pgfsetlinewidth{0.501875pt}%
\definecolor{currentstroke}{rgb}{1.000000,1.000000,1.000000}%
\pgfsetstrokecolor{currentstroke}%
\pgfsetstrokeopacity{0.500000}%
\pgfsetdash{}{0pt}%
\pgfpathmoveto{\pgfqpoint{3.428419in}{3.473729in}}%
\pgfpathlineto{\pgfqpoint{3.439531in}{3.477424in}}%
\pgfpathlineto{\pgfqpoint{3.450637in}{3.481093in}}%
\pgfpathlineto{\pgfqpoint{3.461739in}{3.484743in}}%
\pgfpathlineto{\pgfqpoint{3.472835in}{3.488383in}}%
\pgfpathlineto{\pgfqpoint{3.483927in}{3.492012in}}%
\pgfpathlineto{\pgfqpoint{3.477662in}{3.500813in}}%
\pgfpathlineto{\pgfqpoint{3.471395in}{3.508952in}}%
\pgfpathlineto{\pgfqpoint{3.465125in}{3.516473in}}%
\pgfpathlineto{\pgfqpoint{3.458855in}{3.523462in}}%
\pgfpathlineto{\pgfqpoint{3.452585in}{3.530009in}}%
\pgfpathlineto{\pgfqpoint{3.441503in}{3.526244in}}%
\pgfpathlineto{\pgfqpoint{3.430419in}{3.522639in}}%
\pgfpathlineto{\pgfqpoint{3.419331in}{3.519219in}}%
\pgfpathlineto{\pgfqpoint{3.408239in}{3.515949in}}%
\pgfpathlineto{\pgfqpoint{3.397142in}{3.512772in}}%
\pgfpathlineto{\pgfqpoint{3.403397in}{3.505926in}}%
\pgfpathlineto{\pgfqpoint{3.409653in}{3.498657in}}%
\pgfpathlineto{\pgfqpoint{3.415909in}{3.490903in}}%
\pgfpathlineto{\pgfqpoint{3.422164in}{3.482603in}}%
\pgfpathclose%
\pgfusepath{stroke,fill}%
\end{pgfscope}%
\begin{pgfscope}%
\pgfpathrectangle{\pgfqpoint{0.887500in}{0.275000in}}{\pgfqpoint{4.225000in}{4.225000in}}%
\pgfusepath{clip}%
\pgfsetbuttcap%
\pgfsetroundjoin%
\definecolor{currentfill}{rgb}{0.120092,0.600104,0.542530}%
\pgfsetfillcolor{currentfill}%
\pgfsetfillopacity{0.700000}%
\pgfsetlinewidth{0.501875pt}%
\definecolor{currentstroke}{rgb}{1.000000,1.000000,1.000000}%
\pgfsetstrokecolor{currentstroke}%
\pgfsetstrokeopacity{0.500000}%
\pgfsetdash{}{0pt}%
\pgfpathmoveto{\pgfqpoint{1.782740in}{2.711178in}}%
\pgfpathlineto{\pgfqpoint{1.794246in}{2.714440in}}%
\pgfpathlineto{\pgfqpoint{1.805747in}{2.717709in}}%
\pgfpathlineto{\pgfqpoint{1.817242in}{2.720984in}}%
\pgfpathlineto{\pgfqpoint{1.828731in}{2.724269in}}%
\pgfpathlineto{\pgfqpoint{1.840214in}{2.727565in}}%
\pgfpathlineto{\pgfqpoint{1.834493in}{2.735315in}}%
\pgfpathlineto{\pgfqpoint{1.828776in}{2.743050in}}%
\pgfpathlineto{\pgfqpoint{1.823063in}{2.750771in}}%
\pgfpathlineto{\pgfqpoint{1.817354in}{2.758477in}}%
\pgfpathlineto{\pgfqpoint{1.811650in}{2.766171in}}%
\pgfpathlineto{\pgfqpoint{1.800180in}{2.762863in}}%
\pgfpathlineto{\pgfqpoint{1.788704in}{2.759566in}}%
\pgfpathlineto{\pgfqpoint{1.777223in}{2.756277in}}%
\pgfpathlineto{\pgfqpoint{1.765736in}{2.752997in}}%
\pgfpathlineto{\pgfqpoint{1.754243in}{2.749721in}}%
\pgfpathlineto{\pgfqpoint{1.759934in}{2.742043in}}%
\pgfpathlineto{\pgfqpoint{1.765629in}{2.734350in}}%
\pgfpathlineto{\pgfqpoint{1.771328in}{2.726641in}}%
\pgfpathlineto{\pgfqpoint{1.777032in}{2.718918in}}%
\pgfpathclose%
\pgfusepath{stroke,fill}%
\end{pgfscope}%
\begin{pgfscope}%
\pgfpathrectangle{\pgfqpoint{0.887500in}{0.275000in}}{\pgfqpoint{4.225000in}{4.225000in}}%
\pgfusepath{clip}%
\pgfsetbuttcap%
\pgfsetroundjoin%
\definecolor{currentfill}{rgb}{0.150148,0.676631,0.506589}%
\pgfsetfillcolor{currentfill}%
\pgfsetfillopacity{0.700000}%
\pgfsetlinewidth{0.501875pt}%
\definecolor{currentstroke}{rgb}{1.000000,1.000000,1.000000}%
\pgfsetstrokecolor{currentstroke}%
\pgfsetstrokeopacity{0.500000}%
\pgfsetdash{}{0pt}%
\pgfpathmoveto{\pgfqpoint{2.781959in}{2.839226in}}%
\pgfpathlineto{\pgfqpoint{2.793124in}{2.855973in}}%
\pgfpathlineto{\pgfqpoint{2.804291in}{2.872972in}}%
\pgfpathlineto{\pgfqpoint{2.815459in}{2.889987in}}%
\pgfpathlineto{\pgfqpoint{2.826630in}{2.906969in}}%
\pgfpathlineto{\pgfqpoint{2.837800in}{2.924487in}}%
\pgfpathlineto{\pgfqpoint{2.831755in}{2.928579in}}%
\pgfpathlineto{\pgfqpoint{2.825716in}{2.932600in}}%
\pgfpathlineto{\pgfqpoint{2.819684in}{2.936473in}}%
\pgfpathlineto{\pgfqpoint{2.813657in}{2.940121in}}%
\pgfpathlineto{\pgfqpoint{2.807638in}{2.943467in}}%
\pgfpathlineto{\pgfqpoint{2.796445in}{2.933772in}}%
\pgfpathlineto{\pgfqpoint{2.785251in}{2.923951in}}%
\pgfpathlineto{\pgfqpoint{2.774064in}{2.912911in}}%
\pgfpathlineto{\pgfqpoint{2.762885in}{2.900658in}}%
\pgfpathlineto{\pgfqpoint{2.751712in}{2.887541in}}%
\pgfpathlineto{\pgfqpoint{2.757752in}{2.878147in}}%
\pgfpathlineto{\pgfqpoint{2.763798in}{2.868503in}}%
\pgfpathlineto{\pgfqpoint{2.769848in}{2.858724in}}%
\pgfpathlineto{\pgfqpoint{2.775902in}{2.848926in}}%
\pgfpathclose%
\pgfusepath{stroke,fill}%
\end{pgfscope}%
\begin{pgfscope}%
\pgfpathrectangle{\pgfqpoint{0.887500in}{0.275000in}}{\pgfqpoint{4.225000in}{4.225000in}}%
\pgfusepath{clip}%
\pgfsetbuttcap%
\pgfsetroundjoin%
\definecolor{currentfill}{rgb}{0.344074,0.780029,0.397381}%
\pgfsetfillcolor{currentfill}%
\pgfsetfillopacity{0.700000}%
\pgfsetlinewidth{0.501875pt}%
\definecolor{currentstroke}{rgb}{1.000000,1.000000,1.000000}%
\pgfsetstrokecolor{currentstroke}%
\pgfsetstrokeopacity{0.500000}%
\pgfsetdash{}{0pt}%
\pgfpathmoveto{\pgfqpoint{2.863363in}{3.036269in}}%
\pgfpathlineto{\pgfqpoint{2.874439in}{3.072028in}}%
\pgfpathlineto{\pgfqpoint{2.885506in}{3.112942in}}%
\pgfpathlineto{\pgfqpoint{2.896576in}{3.157010in}}%
\pgfpathlineto{\pgfqpoint{2.907659in}{3.202219in}}%
\pgfpathlineto{\pgfqpoint{2.918764in}{3.246537in}}%
\pgfpathlineto{\pgfqpoint{2.912680in}{3.248281in}}%
\pgfpathlineto{\pgfqpoint{2.906602in}{3.249767in}}%
\pgfpathlineto{\pgfqpoint{2.900531in}{3.251071in}}%
\pgfpathlineto{\pgfqpoint{2.894467in}{3.252378in}}%
\pgfpathlineto{\pgfqpoint{2.888407in}{3.253877in}}%
\pgfpathlineto{\pgfqpoint{2.877379in}{3.203872in}}%
\pgfpathlineto{\pgfqpoint{2.866375in}{3.153273in}}%
\pgfpathlineto{\pgfqpoint{2.855382in}{3.104684in}}%
\pgfpathlineto{\pgfqpoint{2.844384in}{3.060687in}}%
\pgfpathlineto{\pgfqpoint{2.833364in}{3.023849in}}%
\pgfpathlineto{\pgfqpoint{2.839352in}{3.025818in}}%
\pgfpathlineto{\pgfqpoint{2.845344in}{3.028383in}}%
\pgfpathlineto{\pgfqpoint{2.851343in}{3.031180in}}%
\pgfpathlineto{\pgfqpoint{2.857348in}{3.033857in}}%
\pgfpathclose%
\pgfusepath{stroke,fill}%
\end{pgfscope}%
\begin{pgfscope}%
\pgfpathrectangle{\pgfqpoint{0.887500in}{0.275000in}}{\pgfqpoint{4.225000in}{4.225000in}}%
\pgfusepath{clip}%
\pgfsetbuttcap%
\pgfsetroundjoin%
\definecolor{currentfill}{rgb}{0.122606,0.585371,0.546557}%
\pgfsetfillcolor{currentfill}%
\pgfsetfillopacity{0.700000}%
\pgfsetlinewidth{0.501875pt}%
\definecolor{currentstroke}{rgb}{1.000000,1.000000,1.000000}%
\pgfsetstrokecolor{currentstroke}%
\pgfsetstrokeopacity{0.500000}%
\pgfsetdash{}{0pt}%
\pgfpathmoveto{\pgfqpoint{2.613980in}{2.655846in}}%
\pgfpathlineto{\pgfqpoint{2.625179in}{2.668316in}}%
\pgfpathlineto{\pgfqpoint{2.636377in}{2.680873in}}%
\pgfpathlineto{\pgfqpoint{2.647576in}{2.693196in}}%
\pgfpathlineto{\pgfqpoint{2.658781in}{2.704962in}}%
\pgfpathlineto{\pgfqpoint{2.669993in}{2.715907in}}%
\pgfpathlineto{\pgfqpoint{2.663975in}{2.724934in}}%
\pgfpathlineto{\pgfqpoint{2.657959in}{2.734166in}}%
\pgfpathlineto{\pgfqpoint{2.651943in}{2.743663in}}%
\pgfpathlineto{\pgfqpoint{2.645928in}{2.753486in}}%
\pgfpathlineto{\pgfqpoint{2.639912in}{2.763695in}}%
\pgfpathlineto{\pgfqpoint{2.628727in}{2.751583in}}%
\pgfpathlineto{\pgfqpoint{2.617548in}{2.738879in}}%
\pgfpathlineto{\pgfqpoint{2.606374in}{2.725826in}}%
\pgfpathlineto{\pgfqpoint{2.595200in}{2.712707in}}%
\pgfpathlineto{\pgfqpoint{2.584024in}{2.699804in}}%
\pgfpathlineto{\pgfqpoint{2.590017in}{2.690232in}}%
\pgfpathlineto{\pgfqpoint{2.596008in}{2.681140in}}%
\pgfpathlineto{\pgfqpoint{2.601998in}{2.672437in}}%
\pgfpathlineto{\pgfqpoint{2.607988in}{2.664035in}}%
\pgfpathclose%
\pgfusepath{stroke,fill}%
\end{pgfscope}%
\begin{pgfscope}%
\pgfpathrectangle{\pgfqpoint{0.887500in}{0.275000in}}{\pgfqpoint{4.225000in}{4.225000in}}%
\pgfusepath{clip}%
\pgfsetbuttcap%
\pgfsetroundjoin%
\definecolor{currentfill}{rgb}{0.772852,0.877868,0.131109}%
\pgfsetfillcolor{currentfill}%
\pgfsetfillopacity{0.700000}%
\pgfsetlinewidth{0.501875pt}%
\definecolor{currentstroke}{rgb}{1.000000,1.000000,1.000000}%
\pgfsetstrokecolor{currentstroke}%
\pgfsetstrokeopacity{0.500000}%
\pgfsetdash{}{0pt}%
\pgfpathmoveto{\pgfqpoint{2.913607in}{3.444346in}}%
\pgfpathlineto{\pgfqpoint{2.924795in}{3.460559in}}%
\pgfpathlineto{\pgfqpoint{2.935998in}{3.472443in}}%
\pgfpathlineto{\pgfqpoint{2.947209in}{3.480872in}}%
\pgfpathlineto{\pgfqpoint{2.958424in}{3.486716in}}%
\pgfpathlineto{\pgfqpoint{2.969639in}{3.490667in}}%
\pgfpathlineto{\pgfqpoint{2.963542in}{3.493667in}}%
\pgfpathlineto{\pgfqpoint{2.957450in}{3.496714in}}%
\pgfpathlineto{\pgfqpoint{2.951363in}{3.499840in}}%
\pgfpathlineto{\pgfqpoint{2.945282in}{3.502958in}}%
\pgfpathlineto{\pgfqpoint{2.939207in}{3.505941in}}%
\pgfpathlineto{\pgfqpoint{2.928007in}{3.502726in}}%
\pgfpathlineto{\pgfqpoint{2.916809in}{3.497740in}}%
\pgfpathlineto{\pgfqpoint{2.905615in}{3.490400in}}%
\pgfpathlineto{\pgfqpoint{2.894430in}{3.480030in}}%
\pgfpathlineto{\pgfqpoint{2.883260in}{3.465950in}}%
\pgfpathlineto{\pgfqpoint{2.889319in}{3.461660in}}%
\pgfpathlineto{\pgfqpoint{2.895384in}{3.457192in}}%
\pgfpathlineto{\pgfqpoint{2.901454in}{3.452748in}}%
\pgfpathlineto{\pgfqpoint{2.907528in}{3.448473in}}%
\pgfpathclose%
\pgfusepath{stroke,fill}%
\end{pgfscope}%
\begin{pgfscope}%
\pgfpathrectangle{\pgfqpoint{0.887500in}{0.275000in}}{\pgfqpoint{4.225000in}{4.225000in}}%
\pgfusepath{clip}%
\pgfsetbuttcap%
\pgfsetroundjoin%
\definecolor{currentfill}{rgb}{0.804182,0.882046,0.114965}%
\pgfsetfillcolor{currentfill}%
\pgfsetfillopacity{0.700000}%
\pgfsetlinewidth{0.501875pt}%
\definecolor{currentstroke}{rgb}{1.000000,1.000000,1.000000}%
\pgfsetstrokecolor{currentstroke}%
\pgfsetstrokeopacity{0.500000}%
\pgfsetdash{}{0pt}%
\pgfpathmoveto{\pgfqpoint{3.056303in}{3.486440in}}%
\pgfpathlineto{\pgfqpoint{3.067504in}{3.488962in}}%
\pgfpathlineto{\pgfqpoint{3.078699in}{3.492111in}}%
\pgfpathlineto{\pgfqpoint{3.089890in}{3.495751in}}%
\pgfpathlineto{\pgfqpoint{3.101076in}{3.499740in}}%
\pgfpathlineto{\pgfqpoint{3.112257in}{3.503936in}}%
\pgfpathlineto{\pgfqpoint{3.106099in}{3.508345in}}%
\pgfpathlineto{\pgfqpoint{3.099945in}{3.512436in}}%
\pgfpathlineto{\pgfqpoint{3.093797in}{3.516218in}}%
\pgfpathlineto{\pgfqpoint{3.087653in}{3.519700in}}%
\pgfpathlineto{\pgfqpoint{3.081515in}{3.522890in}}%
\pgfpathlineto{\pgfqpoint{3.070351in}{3.517122in}}%
\pgfpathlineto{\pgfqpoint{3.059182in}{3.511624in}}%
\pgfpathlineto{\pgfqpoint{3.048010in}{3.506674in}}%
\pgfpathlineto{\pgfqpoint{3.036832in}{3.502550in}}%
\pgfpathlineto{\pgfqpoint{3.025649in}{3.499522in}}%
\pgfpathlineto{\pgfqpoint{3.031769in}{3.497159in}}%
\pgfpathlineto{\pgfqpoint{3.037894in}{3.494740in}}%
\pgfpathlineto{\pgfqpoint{3.044024in}{3.492196in}}%
\pgfpathlineto{\pgfqpoint{3.050161in}{3.489453in}}%
\pgfpathclose%
\pgfusepath{stroke,fill}%
\end{pgfscope}%
\begin{pgfscope}%
\pgfpathrectangle{\pgfqpoint{0.887500in}{0.275000in}}{\pgfqpoint{4.225000in}{4.225000in}}%
\pgfusepath{clip}%
\pgfsetbuttcap%
\pgfsetroundjoin%
\definecolor{currentfill}{rgb}{0.119483,0.614817,0.537692}%
\pgfsetfillcolor{currentfill}%
\pgfsetfillopacity{0.700000}%
\pgfsetlinewidth{0.501875pt}%
\definecolor{currentstroke}{rgb}{1.000000,1.000000,1.000000}%
\pgfsetstrokecolor{currentstroke}%
\pgfsetstrokeopacity{0.500000}%
\pgfsetdash{}{0pt}%
\pgfpathmoveto{\pgfqpoint{1.552990in}{2.738522in}}%
\pgfpathlineto{\pgfqpoint{1.564553in}{2.741829in}}%
\pgfpathlineto{\pgfqpoint{1.576110in}{2.745135in}}%
\pgfpathlineto{\pgfqpoint{1.587662in}{2.748440in}}%
\pgfpathlineto{\pgfqpoint{1.599208in}{2.751742in}}%
\pgfpathlineto{\pgfqpoint{1.610748in}{2.755041in}}%
\pgfpathlineto{\pgfqpoint{1.605110in}{2.762602in}}%
\pgfpathlineto{\pgfqpoint{1.599477in}{2.770145in}}%
\pgfpathlineto{\pgfqpoint{1.593847in}{2.777671in}}%
\pgfpathlineto{\pgfqpoint{1.588222in}{2.785180in}}%
\pgfpathlineto{\pgfqpoint{1.582601in}{2.792671in}}%
\pgfpathlineto{\pgfqpoint{1.571075in}{2.789355in}}%
\pgfpathlineto{\pgfqpoint{1.559543in}{2.786036in}}%
\pgfpathlineto{\pgfqpoint{1.548006in}{2.782715in}}%
\pgfpathlineto{\pgfqpoint{1.536463in}{2.779392in}}%
\pgfpathlineto{\pgfqpoint{1.524914in}{2.776069in}}%
\pgfpathlineto{\pgfqpoint{1.530521in}{2.768594in}}%
\pgfpathlineto{\pgfqpoint{1.536131in}{2.761101in}}%
\pgfpathlineto{\pgfqpoint{1.541747in}{2.753592in}}%
\pgfpathlineto{\pgfqpoint{1.547366in}{2.746065in}}%
\pgfpathclose%
\pgfusepath{stroke,fill}%
\end{pgfscope}%
\begin{pgfscope}%
\pgfpathrectangle{\pgfqpoint{0.887500in}{0.275000in}}{\pgfqpoint{4.225000in}{4.225000in}}%
\pgfusepath{clip}%
\pgfsetbuttcap%
\pgfsetroundjoin%
\definecolor{currentfill}{rgb}{0.124395,0.578002,0.548287}%
\pgfsetfillcolor{currentfill}%
\pgfsetfillopacity{0.700000}%
\pgfsetlinewidth{0.501875pt}%
\definecolor{currentstroke}{rgb}{1.000000,1.000000,1.000000}%
\pgfsetstrokecolor{currentstroke}%
\pgfsetstrokeopacity{0.500000}%
\pgfsetdash{}{0pt}%
\pgfpathmoveto{\pgfqpoint{2.098684in}{2.658857in}}%
\pgfpathlineto{\pgfqpoint{2.110116in}{2.662111in}}%
\pgfpathlineto{\pgfqpoint{2.121543in}{2.665396in}}%
\pgfpathlineto{\pgfqpoint{2.132962in}{2.668729in}}%
\pgfpathlineto{\pgfqpoint{2.144375in}{2.672129in}}%
\pgfpathlineto{\pgfqpoint{2.155780in}{2.675613in}}%
\pgfpathlineto{\pgfqpoint{2.149945in}{2.683643in}}%
\pgfpathlineto{\pgfqpoint{2.144114in}{2.691659in}}%
\pgfpathlineto{\pgfqpoint{2.138287in}{2.699663in}}%
\pgfpathlineto{\pgfqpoint{2.132464in}{2.707654in}}%
\pgfpathlineto{\pgfqpoint{2.126644in}{2.715633in}}%
\pgfpathlineto{\pgfqpoint{2.115253in}{2.712091in}}%
\pgfpathlineto{\pgfqpoint{2.103854in}{2.708656in}}%
\pgfpathlineto{\pgfqpoint{2.092447in}{2.705304in}}%
\pgfpathlineto{\pgfqpoint{2.081033in}{2.702013in}}%
\pgfpathlineto{\pgfqpoint{2.069613in}{2.698761in}}%
\pgfpathlineto{\pgfqpoint{2.075419in}{2.690806in}}%
\pgfpathlineto{\pgfqpoint{2.081229in}{2.682838in}}%
\pgfpathlineto{\pgfqpoint{2.087043in}{2.674857in}}%
\pgfpathlineto{\pgfqpoint{2.092861in}{2.666863in}}%
\pgfpathclose%
\pgfusepath{stroke,fill}%
\end{pgfscope}%
\begin{pgfscope}%
\pgfpathrectangle{\pgfqpoint{0.887500in}{0.275000in}}{\pgfqpoint{4.225000in}{4.225000in}}%
\pgfusepath{clip}%
\pgfsetbuttcap%
\pgfsetroundjoin%
\definecolor{currentfill}{rgb}{0.129933,0.559582,0.551864}%
\pgfsetfillcolor{currentfill}%
\pgfsetfillopacity{0.700000}%
\pgfsetlinewidth{0.501875pt}%
\definecolor{currentstroke}{rgb}{1.000000,1.000000,1.000000}%
\pgfsetstrokecolor{currentstroke}%
\pgfsetstrokeopacity{0.500000}%
\pgfsetdash{}{0pt}%
\pgfpathmoveto{\pgfqpoint{2.557837in}{2.605082in}}%
\pgfpathlineto{\pgfqpoint{2.569096in}{2.613062in}}%
\pgfpathlineto{\pgfqpoint{2.580338in}{2.622154in}}%
\pgfpathlineto{\pgfqpoint{2.591564in}{2.632446in}}%
\pgfpathlineto{\pgfqpoint{2.602777in}{2.643782in}}%
\pgfpathlineto{\pgfqpoint{2.613980in}{2.655846in}}%
\pgfpathlineto{\pgfqpoint{2.607988in}{2.664035in}}%
\pgfpathlineto{\pgfqpoint{2.601998in}{2.672437in}}%
\pgfpathlineto{\pgfqpoint{2.596008in}{2.681140in}}%
\pgfpathlineto{\pgfqpoint{2.590017in}{2.690232in}}%
\pgfpathlineto{\pgfqpoint{2.584024in}{2.699804in}}%
\pgfpathlineto{\pgfqpoint{2.572841in}{2.687401in}}%
\pgfpathlineto{\pgfqpoint{2.561649in}{2.675779in}}%
\pgfpathlineto{\pgfqpoint{2.550442in}{2.665218in}}%
\pgfpathlineto{\pgfqpoint{2.539219in}{2.655853in}}%
\pgfpathlineto{\pgfqpoint{2.527979in}{2.647602in}}%
\pgfpathlineto{\pgfqpoint{2.533951in}{2.638461in}}%
\pgfpathlineto{\pgfqpoint{2.539922in}{2.629707in}}%
\pgfpathlineto{\pgfqpoint{2.545893in}{2.621271in}}%
\pgfpathlineto{\pgfqpoint{2.551864in}{2.613086in}}%
\pgfpathclose%
\pgfusepath{stroke,fill}%
\end{pgfscope}%
\begin{pgfscope}%
\pgfpathrectangle{\pgfqpoint{0.887500in}{0.275000in}}{\pgfqpoint{4.225000in}{4.225000in}}%
\pgfusepath{clip}%
\pgfsetbuttcap%
\pgfsetroundjoin%
\definecolor{currentfill}{rgb}{0.772852,0.877868,0.131109}%
\pgfsetfillcolor{currentfill}%
\pgfsetfillopacity{0.700000}%
\pgfsetlinewidth{0.501875pt}%
\definecolor{currentstroke}{rgb}{1.000000,1.000000,1.000000}%
\pgfsetstrokecolor{currentstroke}%
\pgfsetstrokeopacity{0.500000}%
\pgfsetdash{}{0pt}%
\pgfpathmoveto{\pgfqpoint{3.515224in}{3.440588in}}%
\pgfpathlineto{\pgfqpoint{3.526315in}{3.444014in}}%
\pgfpathlineto{\pgfqpoint{3.537401in}{3.447407in}}%
\pgfpathlineto{\pgfqpoint{3.548482in}{3.450773in}}%
\pgfpathlineto{\pgfqpoint{3.559556in}{3.454119in}}%
\pgfpathlineto{\pgfqpoint{3.570626in}{3.457452in}}%
\pgfpathlineto{\pgfqpoint{3.564363in}{3.468596in}}%
\pgfpathlineto{\pgfqpoint{3.558100in}{3.479496in}}%
\pgfpathlineto{\pgfqpoint{3.551837in}{3.490075in}}%
\pgfpathlineto{\pgfqpoint{3.545572in}{3.500255in}}%
\pgfpathlineto{\pgfqpoint{3.539305in}{3.509958in}}%
\pgfpathlineto{\pgfqpoint{3.528239in}{3.506398in}}%
\pgfpathlineto{\pgfqpoint{3.517169in}{3.502823in}}%
\pgfpathlineto{\pgfqpoint{3.506093in}{3.499233in}}%
\pgfpathlineto{\pgfqpoint{3.495013in}{3.495629in}}%
\pgfpathlineto{\pgfqpoint{3.483927in}{3.492012in}}%
\pgfpathlineto{\pgfqpoint{3.490189in}{3.482610in}}%
\pgfpathlineto{\pgfqpoint{3.496449in}{3.472687in}}%
\pgfpathlineto{\pgfqpoint{3.502707in}{3.462323in}}%
\pgfpathlineto{\pgfqpoint{3.508965in}{3.451596in}}%
\pgfpathclose%
\pgfusepath{stroke,fill}%
\end{pgfscope}%
\begin{pgfscope}%
\pgfpathrectangle{\pgfqpoint{0.887500in}{0.275000in}}{\pgfqpoint{4.225000in}{4.225000in}}%
\pgfusepath{clip}%
\pgfsetbuttcap%
\pgfsetroundjoin%
\definecolor{currentfill}{rgb}{0.132444,0.552216,0.553018}%
\pgfsetfillcolor{currentfill}%
\pgfsetfillopacity{0.700000}%
\pgfsetlinewidth{0.501875pt}%
\definecolor{currentstroke}{rgb}{1.000000,1.000000,1.000000}%
\pgfsetstrokecolor{currentstroke}%
\pgfsetstrokeopacity{0.500000}%
\pgfsetdash{}{0pt}%
\pgfpathmoveto{\pgfqpoint{2.414797in}{2.604126in}}%
\pgfpathlineto{\pgfqpoint{2.426162in}{2.606890in}}%
\pgfpathlineto{\pgfqpoint{2.437521in}{2.609704in}}%
\pgfpathlineto{\pgfqpoint{2.448872in}{2.612681in}}%
\pgfpathlineto{\pgfqpoint{2.460212in}{2.615939in}}%
\pgfpathlineto{\pgfqpoint{2.471542in}{2.619593in}}%
\pgfpathlineto{\pgfqpoint{2.465587in}{2.628545in}}%
\pgfpathlineto{\pgfqpoint{2.459632in}{2.637815in}}%
\pgfpathlineto{\pgfqpoint{2.453675in}{2.647413in}}%
\pgfpathlineto{\pgfqpoint{2.447717in}{2.657296in}}%
\pgfpathlineto{\pgfqpoint{2.441759in}{2.667420in}}%
\pgfpathlineto{\pgfqpoint{2.430459in}{2.662514in}}%
\pgfpathlineto{\pgfqpoint{2.419149in}{2.657988in}}%
\pgfpathlineto{\pgfqpoint{2.407829in}{2.653782in}}%
\pgfpathlineto{\pgfqpoint{2.396499in}{2.649835in}}%
\pgfpathlineto{\pgfqpoint{2.385161in}{2.646087in}}%
\pgfpathlineto{\pgfqpoint{2.391082in}{2.637628in}}%
\pgfpathlineto{\pgfqpoint{2.397006in}{2.629199in}}%
\pgfpathlineto{\pgfqpoint{2.402933in}{2.620804in}}%
\pgfpathlineto{\pgfqpoint{2.408863in}{2.612447in}}%
\pgfpathclose%
\pgfusepath{stroke,fill}%
\end{pgfscope}%
\begin{pgfscope}%
\pgfpathrectangle{\pgfqpoint{0.887500in}{0.275000in}}{\pgfqpoint{4.225000in}{4.225000in}}%
\pgfusepath{clip}%
\pgfsetbuttcap%
\pgfsetroundjoin%
\definecolor{currentfill}{rgb}{0.720391,0.870350,0.162603}%
\pgfsetfillcolor{currentfill}%
\pgfsetfillopacity{0.700000}%
\pgfsetlinewidth{0.501875pt}%
\definecolor{currentstroke}{rgb}{1.000000,1.000000,1.000000}%
\pgfsetstrokecolor{currentstroke}%
\pgfsetstrokeopacity{0.500000}%
\pgfsetdash{}{0pt}%
\pgfpathmoveto{\pgfqpoint{3.601977in}{3.400809in}}%
\pgfpathlineto{\pgfqpoint{3.613047in}{3.404108in}}%
\pgfpathlineto{\pgfqpoint{3.624111in}{3.407384in}}%
\pgfpathlineto{\pgfqpoint{3.635169in}{3.410626in}}%
\pgfpathlineto{\pgfqpoint{3.646222in}{3.413822in}}%
\pgfpathlineto{\pgfqpoint{3.657268in}{3.416962in}}%
\pgfpathlineto{\pgfqpoint{3.650983in}{3.428097in}}%
\pgfpathlineto{\pgfqpoint{3.644704in}{3.439387in}}%
\pgfpathlineto{\pgfqpoint{3.638428in}{3.450766in}}%
\pgfpathlineto{\pgfqpoint{3.632157in}{3.462164in}}%
\pgfpathlineto{\pgfqpoint{3.625887in}{3.473511in}}%
\pgfpathlineto{\pgfqpoint{3.614848in}{3.470443in}}%
\pgfpathlineto{\pgfqpoint{3.603801in}{3.467284in}}%
\pgfpathlineto{\pgfqpoint{3.592749in}{3.464053in}}%
\pgfpathlineto{\pgfqpoint{3.581690in}{3.460769in}}%
\pgfpathlineto{\pgfqpoint{3.570626in}{3.457452in}}%
\pgfpathlineto{\pgfqpoint{3.576890in}{3.446143in}}%
\pgfpathlineto{\pgfqpoint{3.583156in}{3.434747in}}%
\pgfpathlineto{\pgfqpoint{3.589425in}{3.423342in}}%
\pgfpathlineto{\pgfqpoint{3.595699in}{3.412005in}}%
\pgfpathclose%
\pgfusepath{stroke,fill}%
\end{pgfscope}%
\begin{pgfscope}%
\pgfpathrectangle{\pgfqpoint{0.887500in}{0.275000in}}{\pgfqpoint{4.225000in}{4.225000in}}%
\pgfusepath{clip}%
\pgfsetbuttcap%
\pgfsetroundjoin%
\definecolor{currentfill}{rgb}{0.458674,0.816363,0.329727}%
\pgfsetfillcolor{currentfill}%
\pgfsetfillopacity{0.700000}%
\pgfsetlinewidth{0.501875pt}%
\definecolor{currentstroke}{rgb}{1.000000,1.000000,1.000000}%
\pgfsetstrokecolor{currentstroke}%
\pgfsetstrokeopacity{0.500000}%
\pgfsetdash{}{0pt}%
\pgfpathmoveto{\pgfqpoint{4.035642in}{3.192577in}}%
\pgfpathlineto{\pgfqpoint{4.046599in}{3.195496in}}%
\pgfpathlineto{\pgfqpoint{4.057549in}{3.198387in}}%
\pgfpathlineto{\pgfqpoint{4.068493in}{3.201250in}}%
\pgfpathlineto{\pgfqpoint{4.079431in}{3.204084in}}%
\pgfpathlineto{\pgfqpoint{4.090362in}{3.206890in}}%
\pgfpathlineto{\pgfqpoint{4.083993in}{3.218999in}}%
\pgfpathlineto{\pgfqpoint{4.077626in}{3.231095in}}%
\pgfpathlineto{\pgfqpoint{4.071262in}{3.243181in}}%
\pgfpathlineto{\pgfqpoint{4.064901in}{3.255262in}}%
\pgfpathlineto{\pgfqpoint{4.058543in}{3.267344in}}%
\pgfpathlineto{\pgfqpoint{4.047616in}{3.264584in}}%
\pgfpathlineto{\pgfqpoint{4.036683in}{3.261785in}}%
\pgfpathlineto{\pgfqpoint{4.025744in}{3.258949in}}%
\pgfpathlineto{\pgfqpoint{4.014798in}{3.256075in}}%
\pgfpathlineto{\pgfqpoint{4.003846in}{3.253165in}}%
\pgfpathlineto{\pgfqpoint{4.010204in}{3.241227in}}%
\pgfpathlineto{\pgfqpoint{4.016562in}{3.229191in}}%
\pgfpathlineto{\pgfqpoint{4.022921in}{3.217065in}}%
\pgfpathlineto{\pgfqpoint{4.029281in}{3.204858in}}%
\pgfpathclose%
\pgfusepath{stroke,fill}%
\end{pgfscope}%
\begin{pgfscope}%
\pgfpathrectangle{\pgfqpoint{0.887500in}{0.275000in}}{\pgfqpoint{4.225000in}{4.225000in}}%
\pgfusepath{clip}%
\pgfsetbuttcap%
\pgfsetroundjoin%
\definecolor{currentfill}{rgb}{0.395174,0.797475,0.367757}%
\pgfsetfillcolor{currentfill}%
\pgfsetfillopacity{0.700000}%
\pgfsetlinewidth{0.501875pt}%
\definecolor{currentstroke}{rgb}{1.000000,1.000000,1.000000}%
\pgfsetstrokecolor{currentstroke}%
\pgfsetstrokeopacity{0.500000}%
\pgfsetdash{}{0pt}%
\pgfpathmoveto{\pgfqpoint{4.122245in}{3.145964in}}%
\pgfpathlineto{\pgfqpoint{4.133185in}{3.149119in}}%
\pgfpathlineto{\pgfqpoint{4.144119in}{3.152285in}}%
\pgfpathlineto{\pgfqpoint{4.155049in}{3.155458in}}%
\pgfpathlineto{\pgfqpoint{4.165973in}{3.158640in}}%
\pgfpathlineto{\pgfqpoint{4.159580in}{3.170551in}}%
\pgfpathlineto{\pgfqpoint{4.153188in}{3.182397in}}%
\pgfpathlineto{\pgfqpoint{4.146798in}{3.194205in}}%
\pgfpathlineto{\pgfqpoint{4.140411in}{3.206001in}}%
\pgfpathlineto{\pgfqpoint{4.134027in}{3.217811in}}%
\pgfpathlineto{\pgfqpoint{4.123120in}{3.215127in}}%
\pgfpathlineto{\pgfqpoint{4.112207in}{3.212411in}}%
\pgfpathlineto{\pgfqpoint{4.101288in}{3.209665in}}%
\pgfpathlineto{\pgfqpoint{4.090362in}{3.206890in}}%
\pgfpathlineto{\pgfqpoint{4.096734in}{3.194761in}}%
\pgfpathlineto{\pgfqpoint{4.103109in}{3.182609in}}%
\pgfpathlineto{\pgfqpoint{4.109486in}{3.170429in}}%
\pgfpathlineto{\pgfqpoint{4.115864in}{3.158215in}}%
\pgfpathclose%
\pgfusepath{stroke,fill}%
\end{pgfscope}%
\begin{pgfscope}%
\pgfpathrectangle{\pgfqpoint{0.887500in}{0.275000in}}{\pgfqpoint{4.225000in}{4.225000in}}%
\pgfusepath{clip}%
\pgfsetbuttcap%
\pgfsetroundjoin%
\definecolor{currentfill}{rgb}{0.668054,0.861999,0.196293}%
\pgfsetfillcolor{currentfill}%
\pgfsetfillopacity{0.700000}%
\pgfsetlinewidth{0.501875pt}%
\definecolor{currentstroke}{rgb}{1.000000,1.000000,1.000000}%
\pgfsetstrokecolor{currentstroke}%
\pgfsetstrokeopacity{0.500000}%
\pgfsetdash{}{0pt}%
\pgfpathmoveto{\pgfqpoint{3.688766in}{3.362842in}}%
\pgfpathlineto{\pgfqpoint{3.699815in}{3.366049in}}%
\pgfpathlineto{\pgfqpoint{3.710858in}{3.369212in}}%
\pgfpathlineto{\pgfqpoint{3.721894in}{3.372336in}}%
\pgfpathlineto{\pgfqpoint{3.732925in}{3.375439in}}%
\pgfpathlineto{\pgfqpoint{3.743951in}{3.378538in}}%
\pgfpathlineto{\pgfqpoint{3.737634in}{3.389140in}}%
\pgfpathlineto{\pgfqpoint{3.731321in}{3.399734in}}%
\pgfpathlineto{\pgfqpoint{3.725012in}{3.410357in}}%
\pgfpathlineto{\pgfqpoint{3.718708in}{3.421048in}}%
\pgfpathlineto{\pgfqpoint{3.712408in}{3.431844in}}%
\pgfpathlineto{\pgfqpoint{3.701391in}{3.428880in}}%
\pgfpathlineto{\pgfqpoint{3.690369in}{3.425958in}}%
\pgfpathlineto{\pgfqpoint{3.679342in}{3.423026in}}%
\pgfpathlineto{\pgfqpoint{3.668308in}{3.420034in}}%
\pgfpathlineto{\pgfqpoint{3.657268in}{3.416962in}}%
\pgfpathlineto{\pgfqpoint{3.663559in}{3.405983in}}%
\pgfpathlineto{\pgfqpoint{3.669855in}{3.395120in}}%
\pgfpathlineto{\pgfqpoint{3.676155in}{3.384335in}}%
\pgfpathlineto{\pgfqpoint{3.682459in}{3.373589in}}%
\pgfpathclose%
\pgfusepath{stroke,fill}%
\end{pgfscope}%
\begin{pgfscope}%
\pgfpathrectangle{\pgfqpoint{0.887500in}{0.275000in}}{\pgfqpoint{4.225000in}{4.225000in}}%
\pgfusepath{clip}%
\pgfsetbuttcap%
\pgfsetroundjoin%
\definecolor{currentfill}{rgb}{0.121148,0.592739,0.544641}%
\pgfsetfillcolor{currentfill}%
\pgfsetfillopacity{0.700000}%
\pgfsetlinewidth{0.501875pt}%
\definecolor{currentstroke}{rgb}{1.000000,1.000000,1.000000}%
\pgfsetstrokecolor{currentstroke}%
\pgfsetstrokeopacity{0.500000}%
\pgfsetdash{}{0pt}%
\pgfpathmoveto{\pgfqpoint{1.868883in}{2.688577in}}%
\pgfpathlineto{\pgfqpoint{1.880373in}{2.691880in}}%
\pgfpathlineto{\pgfqpoint{1.891857in}{2.695194in}}%
\pgfpathlineto{\pgfqpoint{1.903335in}{2.698519in}}%
\pgfpathlineto{\pgfqpoint{1.914808in}{2.701852in}}%
\pgfpathlineto{\pgfqpoint{1.926275in}{2.705192in}}%
\pgfpathlineto{\pgfqpoint{1.920519in}{2.713028in}}%
\pgfpathlineto{\pgfqpoint{1.914768in}{2.720847in}}%
\pgfpathlineto{\pgfqpoint{1.909022in}{2.728650in}}%
\pgfpathlineto{\pgfqpoint{1.903279in}{2.736437in}}%
\pgfpathlineto{\pgfqpoint{1.897540in}{2.744210in}}%
\pgfpathlineto{\pgfqpoint{1.886087in}{2.740860in}}%
\pgfpathlineto{\pgfqpoint{1.874627in}{2.737521in}}%
\pgfpathlineto{\pgfqpoint{1.863162in}{2.734191in}}%
\pgfpathlineto{\pgfqpoint{1.851691in}{2.730873in}}%
\pgfpathlineto{\pgfqpoint{1.840214in}{2.727565in}}%
\pgfpathlineto{\pgfqpoint{1.845939in}{2.719801in}}%
\pgfpathlineto{\pgfqpoint{1.851669in}{2.712020in}}%
\pgfpathlineto{\pgfqpoint{1.857403in}{2.704223in}}%
\pgfpathlineto{\pgfqpoint{1.863141in}{2.696409in}}%
\pgfpathclose%
\pgfusepath{stroke,fill}%
\end{pgfscope}%
\begin{pgfscope}%
\pgfpathrectangle{\pgfqpoint{0.887500in}{0.275000in}}{\pgfqpoint{4.225000in}{4.225000in}}%
\pgfusepath{clip}%
\pgfsetbuttcap%
\pgfsetroundjoin%
\definecolor{currentfill}{rgb}{0.208030,0.718701,0.472873}%
\pgfsetfillcolor{currentfill}%
\pgfsetfillopacity{0.700000}%
\pgfsetlinewidth{0.501875pt}%
\definecolor{currentstroke}{rgb}{1.000000,1.000000,1.000000}%
\pgfsetstrokecolor{currentstroke}%
\pgfsetstrokeopacity{0.500000}%
\pgfsetdash{}{0pt}%
\pgfpathmoveto{\pgfqpoint{2.837800in}{2.924487in}}%
\pgfpathlineto{\pgfqpoint{2.848966in}{2.943236in}}%
\pgfpathlineto{\pgfqpoint{2.860126in}{2.963911in}}%
\pgfpathlineto{\pgfqpoint{2.871279in}{2.987213in}}%
\pgfpathlineto{\pgfqpoint{2.882422in}{3.013842in}}%
\pgfpathlineto{\pgfqpoint{2.893557in}{3.044503in}}%
\pgfpathlineto{\pgfqpoint{2.887502in}{3.043350in}}%
\pgfpathlineto{\pgfqpoint{2.881455in}{3.041955in}}%
\pgfpathlineto{\pgfqpoint{2.875416in}{3.040314in}}%
\pgfpathlineto{\pgfqpoint{2.869386in}{3.038420in}}%
\pgfpathlineto{\pgfqpoint{2.863363in}{3.036269in}}%
\pgfpathlineto{\pgfqpoint{2.852264in}{3.007383in}}%
\pgfpathlineto{\pgfqpoint{2.841140in}{2.984987in}}%
\pgfpathlineto{\pgfqpoint{2.829991in}{2.967754in}}%
\pgfpathlineto{\pgfqpoint{2.818823in}{2.954354in}}%
\pgfpathlineto{\pgfqpoint{2.807638in}{2.943467in}}%
\pgfpathlineto{\pgfqpoint{2.813657in}{2.940121in}}%
\pgfpathlineto{\pgfqpoint{2.819684in}{2.936473in}}%
\pgfpathlineto{\pgfqpoint{2.825716in}{2.932600in}}%
\pgfpathlineto{\pgfqpoint{2.831755in}{2.928579in}}%
\pgfpathclose%
\pgfusepath{stroke,fill}%
\end{pgfscope}%
\begin{pgfscope}%
\pgfpathrectangle{\pgfqpoint{0.887500in}{0.275000in}}{\pgfqpoint{4.225000in}{4.225000in}}%
\pgfusepath{clip}%
\pgfsetbuttcap%
\pgfsetroundjoin%
\definecolor{currentfill}{rgb}{0.120638,0.625828,0.533488}%
\pgfsetfillcolor{currentfill}%
\pgfsetfillopacity{0.700000}%
\pgfsetlinewidth{0.501875pt}%
\definecolor{currentstroke}{rgb}{1.000000,1.000000,1.000000}%
\pgfsetstrokecolor{currentstroke}%
\pgfsetstrokeopacity{0.500000}%
\pgfsetdash{}{0pt}%
\pgfpathmoveto{\pgfqpoint{1.323277in}{2.762757in}}%
\pgfpathlineto{\pgfqpoint{1.334893in}{2.766161in}}%
\pgfpathlineto{\pgfqpoint{1.346503in}{2.769557in}}%
\pgfpathlineto{\pgfqpoint{1.358108in}{2.772945in}}%
\pgfpathlineto{\pgfqpoint{1.369708in}{2.776325in}}%
\pgfpathlineto{\pgfqpoint{1.381302in}{2.779698in}}%
\pgfpathlineto{\pgfqpoint{1.375752in}{2.787009in}}%
\pgfpathlineto{\pgfqpoint{1.370207in}{2.794298in}}%
\pgfpathlineto{\pgfqpoint{1.364666in}{2.801563in}}%
\pgfpathlineto{\pgfqpoint{1.359130in}{2.808805in}}%
\pgfpathlineto{\pgfqpoint{1.347549in}{2.805395in}}%
\pgfpathlineto{\pgfqpoint{1.335962in}{2.801977in}}%
\pgfpathlineto{\pgfqpoint{1.324370in}{2.798551in}}%
\pgfpathlineto{\pgfqpoint{1.312773in}{2.795118in}}%
\pgfpathlineto{\pgfqpoint{1.301171in}{2.791677in}}%
\pgfpathlineto{\pgfqpoint{1.306690in}{2.784489in}}%
\pgfpathlineto{\pgfqpoint{1.312214in}{2.777273in}}%
\pgfpathlineto{\pgfqpoint{1.317743in}{2.770028in}}%
\pgfpathclose%
\pgfusepath{stroke,fill}%
\end{pgfscope}%
\begin{pgfscope}%
\pgfpathrectangle{\pgfqpoint{0.887500in}{0.275000in}}{\pgfqpoint{4.225000in}{4.225000in}}%
\pgfusepath{clip}%
\pgfsetbuttcap%
\pgfsetroundjoin%
\definecolor{currentfill}{rgb}{0.506271,0.828786,0.300362}%
\pgfsetfillcolor{currentfill}%
\pgfsetfillopacity{0.700000}%
\pgfsetlinewidth{0.501875pt}%
\definecolor{currentstroke}{rgb}{1.000000,1.000000,1.000000}%
\pgfsetstrokecolor{currentstroke}%
\pgfsetstrokeopacity{0.500000}%
\pgfsetdash{}{0pt}%
\pgfpathmoveto{\pgfqpoint{3.948994in}{3.238081in}}%
\pgfpathlineto{\pgfqpoint{3.959976in}{3.241167in}}%
\pgfpathlineto{\pgfqpoint{3.970953in}{3.244219in}}%
\pgfpathlineto{\pgfqpoint{3.981924in}{3.247237in}}%
\pgfpathlineto{\pgfqpoint{3.992888in}{3.250219in}}%
\pgfpathlineto{\pgfqpoint{4.003846in}{3.253165in}}%
\pgfpathlineto{\pgfqpoint{3.997490in}{3.265007in}}%
\pgfpathlineto{\pgfqpoint{3.991135in}{3.276771in}}%
\pgfpathlineto{\pgfqpoint{3.984781in}{3.288475in}}%
\pgfpathlineto{\pgfqpoint{3.978430in}{3.300141in}}%
\pgfpathlineto{\pgfqpoint{3.972081in}{3.311788in}}%
\pgfpathlineto{\pgfqpoint{3.961123in}{3.308656in}}%
\pgfpathlineto{\pgfqpoint{3.950161in}{3.305557in}}%
\pgfpathlineto{\pgfqpoint{3.939194in}{3.302493in}}%
\pgfpathlineto{\pgfqpoint{3.928223in}{3.299463in}}%
\pgfpathlineto{\pgfqpoint{3.917247in}{3.296469in}}%
\pgfpathlineto{\pgfqpoint{3.923593in}{3.284907in}}%
\pgfpathlineto{\pgfqpoint{3.929940in}{3.273300in}}%
\pgfpathlineto{\pgfqpoint{3.936290in}{3.261635in}}%
\pgfpathlineto{\pgfqpoint{3.942641in}{3.249899in}}%
\pgfpathclose%
\pgfusepath{stroke,fill}%
\end{pgfscope}%
\begin{pgfscope}%
\pgfpathrectangle{\pgfqpoint{0.887500in}{0.275000in}}{\pgfqpoint{4.225000in}{4.225000in}}%
\pgfusepath{clip}%
\pgfsetbuttcap%
\pgfsetroundjoin%
\definecolor{currentfill}{rgb}{0.616293,0.852709,0.230052}%
\pgfsetfillcolor{currentfill}%
\pgfsetfillopacity{0.700000}%
\pgfsetlinewidth{0.501875pt}%
\definecolor{currentstroke}{rgb}{1.000000,1.000000,1.000000}%
\pgfsetstrokecolor{currentstroke}%
\pgfsetstrokeopacity{0.500000}%
\pgfsetdash{}{0pt}%
\pgfpathmoveto{\pgfqpoint{3.775558in}{3.324084in}}%
\pgfpathlineto{\pgfqpoint{3.786586in}{3.327266in}}%
\pgfpathlineto{\pgfqpoint{3.797608in}{3.330396in}}%
\pgfpathlineto{\pgfqpoint{3.808623in}{3.333465in}}%
\pgfpathlineto{\pgfqpoint{3.819632in}{3.336465in}}%
\pgfpathlineto{\pgfqpoint{3.830634in}{3.339404in}}%
\pgfpathlineto{\pgfqpoint{3.824306in}{3.350627in}}%
\pgfpathlineto{\pgfqpoint{3.817979in}{3.361732in}}%
\pgfpathlineto{\pgfqpoint{3.811653in}{3.372744in}}%
\pgfpathlineto{\pgfqpoint{3.805330in}{3.383683in}}%
\pgfpathlineto{\pgfqpoint{3.799009in}{3.394573in}}%
\pgfpathlineto{\pgfqpoint{3.788006in}{3.391245in}}%
\pgfpathlineto{\pgfqpoint{3.776998in}{3.387987in}}%
\pgfpathlineto{\pgfqpoint{3.765987in}{3.384794in}}%
\pgfpathlineto{\pgfqpoint{3.754971in}{3.381651in}}%
\pgfpathlineto{\pgfqpoint{3.743951in}{3.378538in}}%
\pgfpathlineto{\pgfqpoint{3.750270in}{3.367890in}}%
\pgfpathlineto{\pgfqpoint{3.756591in}{3.357158in}}%
\pgfpathlineto{\pgfqpoint{3.762913in}{3.346306in}}%
\pgfpathlineto{\pgfqpoint{3.769236in}{3.335293in}}%
\pgfpathclose%
\pgfusepath{stroke,fill}%
\end{pgfscope}%
\begin{pgfscope}%
\pgfpathrectangle{\pgfqpoint{0.887500in}{0.275000in}}{\pgfqpoint{4.225000in}{4.225000in}}%
\pgfusepath{clip}%
\pgfsetbuttcap%
\pgfsetroundjoin%
\definecolor{currentfill}{rgb}{0.565498,0.842430,0.262877}%
\pgfsetfillcolor{currentfill}%
\pgfsetfillopacity{0.700000}%
\pgfsetlinewidth{0.501875pt}%
\definecolor{currentstroke}{rgb}{1.000000,1.000000,1.000000}%
\pgfsetstrokecolor{currentstroke}%
\pgfsetstrokeopacity{0.500000}%
\pgfsetdash{}{0pt}%
\pgfpathmoveto{\pgfqpoint{3.862289in}{3.281584in}}%
\pgfpathlineto{\pgfqpoint{3.873292in}{3.284613in}}%
\pgfpathlineto{\pgfqpoint{3.884289in}{3.287597in}}%
\pgfpathlineto{\pgfqpoint{3.895281in}{3.290555in}}%
\pgfpathlineto{\pgfqpoint{3.906267in}{3.293506in}}%
\pgfpathlineto{\pgfqpoint{3.917247in}{3.296469in}}%
\pgfpathlineto{\pgfqpoint{3.910904in}{3.308000in}}%
\pgfpathlineto{\pgfqpoint{3.904564in}{3.319510in}}%
\pgfpathlineto{\pgfqpoint{3.898227in}{3.331014in}}%
\pgfpathlineto{\pgfqpoint{3.891893in}{3.342525in}}%
\pgfpathlineto{\pgfqpoint{3.885562in}{3.354054in}}%
\pgfpathlineto{\pgfqpoint{3.874586in}{3.351049in}}%
\pgfpathlineto{\pgfqpoint{3.863605in}{3.348106in}}%
\pgfpathlineto{\pgfqpoint{3.852620in}{3.345200in}}%
\pgfpathlineto{\pgfqpoint{3.841630in}{3.342308in}}%
\pgfpathlineto{\pgfqpoint{3.830634in}{3.339404in}}%
\pgfpathlineto{\pgfqpoint{3.836963in}{3.328054in}}%
\pgfpathlineto{\pgfqpoint{3.843292in}{3.316584in}}%
\pgfpathlineto{\pgfqpoint{3.849623in}{3.305007in}}%
\pgfpathlineto{\pgfqpoint{3.855955in}{3.293336in}}%
\pgfpathclose%
\pgfusepath{stroke,fill}%
\end{pgfscope}%
\begin{pgfscope}%
\pgfpathrectangle{\pgfqpoint{0.887500in}{0.275000in}}{\pgfqpoint{4.225000in}{4.225000in}}%
\pgfusepath{clip}%
\pgfsetbuttcap%
\pgfsetroundjoin%
\definecolor{currentfill}{rgb}{0.814576,0.883393,0.110347}%
\pgfsetfillcolor{currentfill}%
\pgfsetfillopacity{0.700000}%
\pgfsetlinewidth{0.501875pt}%
\definecolor{currentstroke}{rgb}{1.000000,1.000000,1.000000}%
\pgfsetstrokecolor{currentstroke}%
\pgfsetstrokeopacity{0.500000}%
\pgfsetdash{}{0pt}%
\pgfpathmoveto{\pgfqpoint{3.285860in}{3.476554in}}%
\pgfpathlineto{\pgfqpoint{3.297014in}{3.480622in}}%
\pgfpathlineto{\pgfqpoint{3.308162in}{3.484660in}}%
\pgfpathlineto{\pgfqpoint{3.319306in}{3.488634in}}%
\pgfpathlineto{\pgfqpoint{3.330444in}{3.492509in}}%
\pgfpathlineto{\pgfqpoint{3.341576in}{3.496250in}}%
\pgfpathlineto{\pgfqpoint{3.335340in}{3.503471in}}%
\pgfpathlineto{\pgfqpoint{3.329106in}{3.510370in}}%
\pgfpathlineto{\pgfqpoint{3.322875in}{3.516970in}}%
\pgfpathlineto{\pgfqpoint{3.316647in}{3.523294in}}%
\pgfpathlineto{\pgfqpoint{3.310421in}{3.529366in}}%
\pgfpathlineto{\pgfqpoint{3.299305in}{3.526082in}}%
\pgfpathlineto{\pgfqpoint{3.288182in}{3.522598in}}%
\pgfpathlineto{\pgfqpoint{3.277053in}{3.518964in}}%
\pgfpathlineto{\pgfqpoint{3.265919in}{3.515226in}}%
\pgfpathlineto{\pgfqpoint{3.254779in}{3.511434in}}%
\pgfpathlineto{\pgfqpoint{3.260989in}{3.504972in}}%
\pgfpathlineto{\pgfqpoint{3.267203in}{3.498268in}}%
\pgfpathlineto{\pgfqpoint{3.273419in}{3.491307in}}%
\pgfpathlineto{\pgfqpoint{3.279638in}{3.484074in}}%
\pgfpathclose%
\pgfusepath{stroke,fill}%
\end{pgfscope}%
\begin{pgfscope}%
\pgfpathrectangle{\pgfqpoint{0.887500in}{0.275000in}}{\pgfqpoint{4.225000in}{4.225000in}}%
\pgfusepath{clip}%
\pgfsetbuttcap%
\pgfsetroundjoin%
\definecolor{currentfill}{rgb}{0.126453,0.570633,0.549841}%
\pgfsetfillcolor{currentfill}%
\pgfsetfillopacity{0.700000}%
\pgfsetlinewidth{0.501875pt}%
\definecolor{currentstroke}{rgb}{1.000000,1.000000,1.000000}%
\pgfsetstrokecolor{currentstroke}%
\pgfsetstrokeopacity{0.500000}%
\pgfsetdash{}{0pt}%
\pgfpathmoveto{\pgfqpoint{2.185017in}{2.635271in}}%
\pgfpathlineto{\pgfqpoint{2.196430in}{2.638741in}}%
\pgfpathlineto{\pgfqpoint{2.207836in}{2.642241in}}%
\pgfpathlineto{\pgfqpoint{2.219237in}{2.645741in}}%
\pgfpathlineto{\pgfqpoint{2.230633in}{2.649209in}}%
\pgfpathlineto{\pgfqpoint{2.242025in}{2.652613in}}%
\pgfpathlineto{\pgfqpoint{2.236156in}{2.660785in}}%
\pgfpathlineto{\pgfqpoint{2.230290in}{2.668945in}}%
\pgfpathlineto{\pgfqpoint{2.224429in}{2.677091in}}%
\pgfpathlineto{\pgfqpoint{2.218571in}{2.685222in}}%
\pgfpathlineto{\pgfqpoint{2.212718in}{2.693338in}}%
\pgfpathlineto{\pgfqpoint{2.201339in}{2.689926in}}%
\pgfpathlineto{\pgfqpoint{2.189956in}{2.686393in}}%
\pgfpathlineto{\pgfqpoint{2.178570in}{2.682795in}}%
\pgfpathlineto{\pgfqpoint{2.167178in}{2.679183in}}%
\pgfpathlineto{\pgfqpoint{2.155780in}{2.675613in}}%
\pgfpathlineto{\pgfqpoint{2.161620in}{2.667571in}}%
\pgfpathlineto{\pgfqpoint{2.167463in}{2.659515in}}%
\pgfpathlineto{\pgfqpoint{2.173310in}{2.651447in}}%
\pgfpathlineto{\pgfqpoint{2.179162in}{2.643366in}}%
\pgfpathclose%
\pgfusepath{stroke,fill}%
\end{pgfscope}%
\begin{pgfscope}%
\pgfpathrectangle{\pgfqpoint{0.887500in}{0.275000in}}{\pgfqpoint{4.225000in}{4.225000in}}%
\pgfusepath{clip}%
\pgfsetbuttcap%
\pgfsetroundjoin%
\definecolor{currentfill}{rgb}{0.119512,0.607464,0.540218}%
\pgfsetfillcolor{currentfill}%
\pgfsetfillopacity{0.700000}%
\pgfsetlinewidth{0.501875pt}%
\definecolor{currentstroke}{rgb}{1.000000,1.000000,1.000000}%
\pgfsetstrokecolor{currentstroke}%
\pgfsetstrokeopacity{0.500000}%
\pgfsetdash{}{0pt}%
\pgfpathmoveto{\pgfqpoint{1.639003in}{2.716997in}}%
\pgfpathlineto{\pgfqpoint{1.650552in}{2.720277in}}%
\pgfpathlineto{\pgfqpoint{1.662095in}{2.723555in}}%
\pgfpathlineto{\pgfqpoint{1.673633in}{2.726830in}}%
\pgfpathlineto{\pgfqpoint{1.685166in}{2.730103in}}%
\pgfpathlineto{\pgfqpoint{1.696693in}{2.733374in}}%
\pgfpathlineto{\pgfqpoint{1.691020in}{2.741026in}}%
\pgfpathlineto{\pgfqpoint{1.685351in}{2.748662in}}%
\pgfpathlineto{\pgfqpoint{1.679686in}{2.756283in}}%
\pgfpathlineto{\pgfqpoint{1.674026in}{2.763889in}}%
\pgfpathlineto{\pgfqpoint{1.668369in}{2.771479in}}%
\pgfpathlineto{\pgfqpoint{1.656856in}{2.768197in}}%
\pgfpathlineto{\pgfqpoint{1.645337in}{2.764913in}}%
\pgfpathlineto{\pgfqpoint{1.633813in}{2.761626in}}%
\pgfpathlineto{\pgfqpoint{1.622284in}{2.758336in}}%
\pgfpathlineto{\pgfqpoint{1.610748in}{2.755041in}}%
\pgfpathlineto{\pgfqpoint{1.616391in}{2.747464in}}%
\pgfpathlineto{\pgfqpoint{1.622037in}{2.739871in}}%
\pgfpathlineto{\pgfqpoint{1.627688in}{2.732261in}}%
\pgfpathlineto{\pgfqpoint{1.633343in}{2.724637in}}%
\pgfpathclose%
\pgfusepath{stroke,fill}%
\end{pgfscope}%
\begin{pgfscope}%
\pgfpathrectangle{\pgfqpoint{0.887500in}{0.275000in}}{\pgfqpoint{4.225000in}{4.225000in}}%
\pgfusepath{clip}%
\pgfsetbuttcap%
\pgfsetroundjoin%
\definecolor{currentfill}{rgb}{0.120081,0.622161,0.534946}%
\pgfsetfillcolor{currentfill}%
\pgfsetfillopacity{0.700000}%
\pgfsetlinewidth{0.501875pt}%
\definecolor{currentstroke}{rgb}{1.000000,1.000000,1.000000}%
\pgfsetstrokecolor{currentstroke}%
\pgfsetstrokeopacity{0.500000}%
\pgfsetdash{}{0pt}%
\pgfpathmoveto{\pgfqpoint{1.409120in}{2.742813in}}%
\pgfpathlineto{\pgfqpoint{1.420724in}{2.746152in}}%
\pgfpathlineto{\pgfqpoint{1.432323in}{2.749485in}}%
\pgfpathlineto{\pgfqpoint{1.443917in}{2.752814in}}%
\pgfpathlineto{\pgfqpoint{1.455504in}{2.756139in}}%
\pgfpathlineto{\pgfqpoint{1.467087in}{2.759462in}}%
\pgfpathlineto{\pgfqpoint{1.461499in}{2.766902in}}%
\pgfpathlineto{\pgfqpoint{1.455916in}{2.774323in}}%
\pgfpathlineto{\pgfqpoint{1.450337in}{2.781727in}}%
\pgfpathlineto{\pgfqpoint{1.444763in}{2.789111in}}%
\pgfpathlineto{\pgfqpoint{1.439193in}{2.796477in}}%
\pgfpathlineto{\pgfqpoint{1.427626in}{2.793130in}}%
\pgfpathlineto{\pgfqpoint{1.416053in}{2.789780in}}%
\pgfpathlineto{\pgfqpoint{1.404475in}{2.786424in}}%
\pgfpathlineto{\pgfqpoint{1.392891in}{2.783064in}}%
\pgfpathlineto{\pgfqpoint{1.381302in}{2.779698in}}%
\pgfpathlineto{\pgfqpoint{1.386856in}{2.772364in}}%
\pgfpathlineto{\pgfqpoint{1.392415in}{2.765008in}}%
\pgfpathlineto{\pgfqpoint{1.397979in}{2.757631in}}%
\pgfpathlineto{\pgfqpoint{1.403547in}{2.750232in}}%
\pgfpathclose%
\pgfusepath{stroke,fill}%
\end{pgfscope}%
\begin{pgfscope}%
\pgfpathrectangle{\pgfqpoint{0.887500in}{0.275000in}}{\pgfqpoint{4.225000in}{4.225000in}}%
\pgfusepath{clip}%
\pgfsetbuttcap%
\pgfsetroundjoin%
\definecolor{currentfill}{rgb}{0.135066,0.544853,0.554029}%
\pgfsetfillcolor{currentfill}%
\pgfsetfillopacity{0.700000}%
\pgfsetlinewidth{0.501875pt}%
\definecolor{currentstroke}{rgb}{1.000000,1.000000,1.000000}%
\pgfsetstrokecolor{currentstroke}%
\pgfsetstrokeopacity{0.500000}%
\pgfsetdash{}{0pt}%
\pgfpathmoveto{\pgfqpoint{2.501322in}{2.578103in}}%
\pgfpathlineto{\pgfqpoint{2.512652in}{2.582133in}}%
\pgfpathlineto{\pgfqpoint{2.523970in}{2.586739in}}%
\pgfpathlineto{\pgfqpoint{2.535274in}{2.592027in}}%
\pgfpathlineto{\pgfqpoint{2.546563in}{2.598106in}}%
\pgfpathlineto{\pgfqpoint{2.557837in}{2.605082in}}%
\pgfpathlineto{\pgfqpoint{2.551864in}{2.613086in}}%
\pgfpathlineto{\pgfqpoint{2.545893in}{2.621271in}}%
\pgfpathlineto{\pgfqpoint{2.539922in}{2.629707in}}%
\pgfpathlineto{\pgfqpoint{2.533951in}{2.638461in}}%
\pgfpathlineto{\pgfqpoint{2.527979in}{2.647602in}}%
\pgfpathlineto{\pgfqpoint{2.516722in}{2.640367in}}%
\pgfpathlineto{\pgfqpoint{2.505449in}{2.634047in}}%
\pgfpathlineto{\pgfqpoint{2.494161in}{2.628544in}}%
\pgfpathlineto{\pgfqpoint{2.482858in}{2.623759in}}%
\pgfpathlineto{\pgfqpoint{2.471542in}{2.619593in}}%
\pgfpathlineto{\pgfqpoint{2.477495in}{2.610918in}}%
\pgfpathlineto{\pgfqpoint{2.483450in}{2.602475in}}%
\pgfpathlineto{\pgfqpoint{2.489405in}{2.594221in}}%
\pgfpathlineto{\pgfqpoint{2.495362in}{2.586111in}}%
\pgfpathclose%
\pgfusepath{stroke,fill}%
\end{pgfscope}%
\begin{pgfscope}%
\pgfpathrectangle{\pgfqpoint{0.887500in}{0.275000in}}{\pgfqpoint{4.225000in}{4.225000in}}%
\pgfusepath{clip}%
\pgfsetbuttcap%
\pgfsetroundjoin%
\definecolor{currentfill}{rgb}{0.122606,0.585371,0.546557}%
\pgfsetfillcolor{currentfill}%
\pgfsetfillopacity{0.700000}%
\pgfsetlinewidth{0.501875pt}%
\definecolor{currentstroke}{rgb}{1.000000,1.000000,1.000000}%
\pgfsetstrokecolor{currentstroke}%
\pgfsetstrokeopacity{0.500000}%
\pgfsetdash{}{0pt}%
\pgfpathmoveto{\pgfqpoint{1.955115in}{2.665740in}}%
\pgfpathlineto{\pgfqpoint{1.966588in}{2.669079in}}%
\pgfpathlineto{\pgfqpoint{1.978057in}{2.672420in}}%
\pgfpathlineto{\pgfqpoint{1.989520in}{2.675760in}}%
\pgfpathlineto{\pgfqpoint{2.000977in}{2.679094in}}%
\pgfpathlineto{\pgfqpoint{2.012429in}{2.682418in}}%
\pgfpathlineto{\pgfqpoint{2.006640in}{2.690354in}}%
\pgfpathlineto{\pgfqpoint{2.000854in}{2.698274in}}%
\pgfpathlineto{\pgfqpoint{1.995073in}{2.706177in}}%
\pgfpathlineto{\pgfqpoint{1.989296in}{2.714063in}}%
\pgfpathlineto{\pgfqpoint{1.983524in}{2.721931in}}%
\pgfpathlineto{\pgfqpoint{1.972085in}{2.718593in}}%
\pgfpathlineto{\pgfqpoint{1.960641in}{2.715244in}}%
\pgfpathlineto{\pgfqpoint{1.949191in}{2.711891in}}%
\pgfpathlineto{\pgfqpoint{1.937736in}{2.708539in}}%
\pgfpathlineto{\pgfqpoint{1.926275in}{2.705192in}}%
\pgfpathlineto{\pgfqpoint{1.932034in}{2.697339in}}%
\pgfpathlineto{\pgfqpoint{1.937798in}{2.689467in}}%
\pgfpathlineto{\pgfqpoint{1.943566in}{2.681577in}}%
\pgfpathlineto{\pgfqpoint{1.949338in}{2.673667in}}%
\pgfpathclose%
\pgfusepath{stroke,fill}%
\end{pgfscope}%
\begin{pgfscope}%
\pgfpathrectangle{\pgfqpoint{0.887500in}{0.275000in}}{\pgfqpoint{4.225000in}{4.225000in}}%
\pgfusepath{clip}%
\pgfsetbuttcap%
\pgfsetroundjoin%
\definecolor{currentfill}{rgb}{0.814576,0.883393,0.110347}%
\pgfsetfillcolor{currentfill}%
\pgfsetfillopacity{0.700000}%
\pgfsetlinewidth{0.501875pt}%
\definecolor{currentstroke}{rgb}{1.000000,1.000000,1.000000}%
\pgfsetstrokecolor{currentstroke}%
\pgfsetstrokeopacity{0.500000}%
\pgfsetdash{}{0pt}%
\pgfpathmoveto{\pgfqpoint{3.143115in}{3.476838in}}%
\pgfpathlineto{\pgfqpoint{3.154303in}{3.479789in}}%
\pgfpathlineto{\pgfqpoint{3.165487in}{3.482891in}}%
\pgfpathlineto{\pgfqpoint{3.176666in}{3.486130in}}%
\pgfpathlineto{\pgfqpoint{3.187839in}{3.489490in}}%
\pgfpathlineto{\pgfqpoint{3.199008in}{3.492958in}}%
\pgfpathlineto{\pgfqpoint{3.192816in}{3.499515in}}%
\pgfpathlineto{\pgfqpoint{3.186628in}{3.505886in}}%
\pgfpathlineto{\pgfqpoint{3.180445in}{3.512067in}}%
\pgfpathlineto{\pgfqpoint{3.174265in}{3.518027in}}%
\pgfpathlineto{\pgfqpoint{3.168089in}{3.523735in}}%
\pgfpathlineto{\pgfqpoint{3.156933in}{3.520144in}}%
\pgfpathlineto{\pgfqpoint{3.145772in}{3.516375in}}%
\pgfpathlineto{\pgfqpoint{3.134606in}{3.512389in}}%
\pgfpathlineto{\pgfqpoint{3.123434in}{3.508200in}}%
\pgfpathlineto{\pgfqpoint{3.112257in}{3.503936in}}%
\pgfpathlineto{\pgfqpoint{3.118420in}{3.499201in}}%
\pgfpathlineto{\pgfqpoint{3.124587in}{3.494131in}}%
\pgfpathlineto{\pgfqpoint{3.130759in}{3.488718in}}%
\pgfpathlineto{\pgfqpoint{3.136935in}{3.482953in}}%
\pgfpathclose%
\pgfusepath{stroke,fill}%
\end{pgfscope}%
\begin{pgfscope}%
\pgfpathrectangle{\pgfqpoint{0.887500in}{0.275000in}}{\pgfqpoint{4.225000in}{4.225000in}}%
\pgfusepath{clip}%
\pgfsetbuttcap%
\pgfsetroundjoin%
\definecolor{currentfill}{rgb}{0.129933,0.559582,0.551864}%
\pgfsetfillcolor{currentfill}%
\pgfsetfillopacity{0.700000}%
\pgfsetlinewidth{0.501875pt}%
\definecolor{currentstroke}{rgb}{1.000000,1.000000,1.000000}%
\pgfsetstrokecolor{currentstroke}%
\pgfsetstrokeopacity{0.500000}%
\pgfsetdash{}{0pt}%
\pgfpathmoveto{\pgfqpoint{2.271431in}{2.611608in}}%
\pgfpathlineto{\pgfqpoint{2.282829in}{2.615022in}}%
\pgfpathlineto{\pgfqpoint{2.294222in}{2.618427in}}%
\pgfpathlineto{\pgfqpoint{2.305610in}{2.621820in}}%
\pgfpathlineto{\pgfqpoint{2.316992in}{2.625210in}}%
\pgfpathlineto{\pgfqpoint{2.328369in}{2.628608in}}%
\pgfpathlineto{\pgfqpoint{2.322472in}{2.636586in}}%
\pgfpathlineto{\pgfqpoint{2.316580in}{2.644544in}}%
\pgfpathlineto{\pgfqpoint{2.310691in}{2.652487in}}%
\pgfpathlineto{\pgfqpoint{2.304807in}{2.660421in}}%
\pgfpathlineto{\pgfqpoint{2.298927in}{2.668351in}}%
\pgfpathlineto{\pgfqpoint{2.287557in}{2.665224in}}%
\pgfpathlineto{\pgfqpoint{2.276180in}{2.662177in}}%
\pgfpathlineto{\pgfqpoint{2.264799in}{2.659108in}}%
\pgfpathlineto{\pgfqpoint{2.253413in}{2.655923in}}%
\pgfpathlineto{\pgfqpoint{2.242025in}{2.652613in}}%
\pgfpathlineto{\pgfqpoint{2.247898in}{2.644430in}}%
\pgfpathlineto{\pgfqpoint{2.253776in}{2.636237in}}%
\pgfpathlineto{\pgfqpoint{2.259657in}{2.628035in}}%
\pgfpathlineto{\pgfqpoint{2.265542in}{2.619825in}}%
\pgfpathclose%
\pgfusepath{stroke,fill}%
\end{pgfscope}%
\begin{pgfscope}%
\pgfpathrectangle{\pgfqpoint{0.887500in}{0.275000in}}{\pgfqpoint{4.225000in}{4.225000in}}%
\pgfusepath{clip}%
\pgfsetbuttcap%
\pgfsetroundjoin%
\definecolor{currentfill}{rgb}{0.804182,0.882046,0.114965}%
\pgfsetfillcolor{currentfill}%
\pgfsetfillopacity{0.700000}%
\pgfsetlinewidth{0.501875pt}%
\definecolor{currentstroke}{rgb}{1.000000,1.000000,1.000000}%
\pgfsetstrokecolor{currentstroke}%
\pgfsetstrokeopacity{0.500000}%
\pgfsetdash{}{0pt}%
\pgfpathmoveto{\pgfqpoint{3.372777in}{3.454534in}}%
\pgfpathlineto{\pgfqpoint{3.383916in}{3.458497in}}%
\pgfpathlineto{\pgfqpoint{3.395050in}{3.462391in}}%
\pgfpathlineto{\pgfqpoint{3.406178in}{3.466222in}}%
\pgfpathlineto{\pgfqpoint{3.417301in}{3.469998in}}%
\pgfpathlineto{\pgfqpoint{3.428419in}{3.473729in}}%
\pgfpathlineto{\pgfqpoint{3.422164in}{3.482603in}}%
\pgfpathlineto{\pgfqpoint{3.415909in}{3.490903in}}%
\pgfpathlineto{\pgfqpoint{3.409653in}{3.498657in}}%
\pgfpathlineto{\pgfqpoint{3.403397in}{3.505926in}}%
\pgfpathlineto{\pgfqpoint{3.397142in}{3.512772in}}%
\pgfpathlineto{\pgfqpoint{3.386041in}{3.509629in}}%
\pgfpathlineto{\pgfqpoint{3.374934in}{3.506463in}}%
\pgfpathlineto{\pgfqpoint{3.363821in}{3.503214in}}%
\pgfpathlineto{\pgfqpoint{3.352701in}{3.499824in}}%
\pgfpathlineto{\pgfqpoint{3.341576in}{3.496250in}}%
\pgfpathlineto{\pgfqpoint{3.347813in}{3.488685in}}%
\pgfpathlineto{\pgfqpoint{3.354053in}{3.480752in}}%
\pgfpathlineto{\pgfqpoint{3.360293in}{3.472428in}}%
\pgfpathlineto{\pgfqpoint{3.366535in}{3.463691in}}%
\pgfpathclose%
\pgfusepath{stroke,fill}%
\end{pgfscope}%
\begin{pgfscope}%
\pgfpathrectangle{\pgfqpoint{0.887500in}{0.275000in}}{\pgfqpoint{4.225000in}{4.225000in}}%
\pgfusepath{clip}%
\pgfsetbuttcap%
\pgfsetroundjoin%
\definecolor{currentfill}{rgb}{0.120092,0.600104,0.542530}%
\pgfsetfillcolor{currentfill}%
\pgfsetfillopacity{0.700000}%
\pgfsetlinewidth{0.501875pt}%
\definecolor{currentstroke}{rgb}{1.000000,1.000000,1.000000}%
\pgfsetstrokecolor{currentstroke}%
\pgfsetstrokeopacity{0.500000}%
\pgfsetdash{}{0pt}%
\pgfpathmoveto{\pgfqpoint{1.725122in}{2.694892in}}%
\pgfpathlineto{\pgfqpoint{1.736657in}{2.698150in}}%
\pgfpathlineto{\pgfqpoint{1.748186in}{2.701407in}}%
\pgfpathlineto{\pgfqpoint{1.759710in}{2.704663in}}%
\pgfpathlineto{\pgfqpoint{1.771228in}{2.707919in}}%
\pgfpathlineto{\pgfqpoint{1.782740in}{2.711178in}}%
\pgfpathlineto{\pgfqpoint{1.777032in}{2.718918in}}%
\pgfpathlineto{\pgfqpoint{1.771328in}{2.726641in}}%
\pgfpathlineto{\pgfqpoint{1.765629in}{2.734350in}}%
\pgfpathlineto{\pgfqpoint{1.759934in}{2.742043in}}%
\pgfpathlineto{\pgfqpoint{1.754243in}{2.749721in}}%
\pgfpathlineto{\pgfqpoint{1.742744in}{2.746450in}}%
\pgfpathlineto{\pgfqpoint{1.731240in}{2.743181in}}%
\pgfpathlineto{\pgfqpoint{1.719730in}{2.739913in}}%
\pgfpathlineto{\pgfqpoint{1.708214in}{2.736644in}}%
\pgfpathlineto{\pgfqpoint{1.696693in}{2.733374in}}%
\pgfpathlineto{\pgfqpoint{1.702370in}{2.725708in}}%
\pgfpathlineto{\pgfqpoint{1.708052in}{2.718027in}}%
\pgfpathlineto{\pgfqpoint{1.713737in}{2.710331in}}%
\pgfpathlineto{\pgfqpoint{1.719427in}{2.702619in}}%
\pgfpathclose%
\pgfusepath{stroke,fill}%
\end{pgfscope}%
\begin{pgfscope}%
\pgfpathrectangle{\pgfqpoint{0.887500in}{0.275000in}}{\pgfqpoint{4.225000in}{4.225000in}}%
\pgfusepath{clip}%
\pgfsetbuttcap%
\pgfsetroundjoin%
\definecolor{currentfill}{rgb}{0.626579,0.854645,0.223353}%
\pgfsetfillcolor{currentfill}%
\pgfsetfillopacity{0.700000}%
\pgfsetlinewidth{0.501875pt}%
\definecolor{currentstroke}{rgb}{1.000000,1.000000,1.000000}%
\pgfsetstrokecolor{currentstroke}%
\pgfsetstrokeopacity{0.500000}%
\pgfsetdash{}{0pt}%
\pgfpathmoveto{\pgfqpoint{2.888407in}{3.253877in}}%
\pgfpathlineto{\pgfqpoint{2.899468in}{3.300660in}}%
\pgfpathlineto{\pgfqpoint{2.910569in}{3.341598in}}%
\pgfpathlineto{\pgfqpoint{2.921710in}{3.375364in}}%
\pgfpathlineto{\pgfqpoint{2.932882in}{3.402595in}}%
\pgfpathlineto{\pgfqpoint{2.944078in}{3.424104in}}%
\pgfpathlineto{\pgfqpoint{2.937973in}{3.428273in}}%
\pgfpathlineto{\pgfqpoint{2.931874in}{3.432330in}}%
\pgfpathlineto{\pgfqpoint{2.925780in}{3.436326in}}%
\pgfpathlineto{\pgfqpoint{2.919691in}{3.440314in}}%
\pgfpathlineto{\pgfqpoint{2.913607in}{3.444346in}}%
\pgfpathlineto{\pgfqpoint{2.902441in}{3.422933in}}%
\pgfpathlineto{\pgfqpoint{2.891305in}{3.395457in}}%
\pgfpathlineto{\pgfqpoint{2.880206in}{3.361060in}}%
\pgfpathlineto{\pgfqpoint{2.869155in}{3.319072in}}%
\pgfpathlineto{\pgfqpoint{2.858151in}{3.270921in}}%
\pgfpathlineto{\pgfqpoint{2.864201in}{3.265606in}}%
\pgfpathlineto{\pgfqpoint{2.870250in}{3.261434in}}%
\pgfpathlineto{\pgfqpoint{2.876300in}{3.258215in}}%
\pgfpathlineto{\pgfqpoint{2.882352in}{3.255759in}}%
\pgfpathclose%
\pgfusepath{stroke,fill}%
\end{pgfscope}%
\begin{pgfscope}%
\pgfpathrectangle{\pgfqpoint{0.887500in}{0.275000in}}{\pgfqpoint{4.225000in}{4.225000in}}%
\pgfusepath{clip}%
\pgfsetbuttcap%
\pgfsetroundjoin%
\definecolor{currentfill}{rgb}{0.120638,0.625828,0.533488}%
\pgfsetfillcolor{currentfill}%
\pgfsetfillopacity{0.700000}%
\pgfsetlinewidth{0.501875pt}%
\definecolor{currentstroke}{rgb}{1.000000,1.000000,1.000000}%
\pgfsetstrokecolor{currentstroke}%
\pgfsetstrokeopacity{0.500000}%
\pgfsetdash{}{0pt}%
\pgfpathmoveto{\pgfqpoint{2.756307in}{2.719596in}}%
\pgfpathlineto{\pgfqpoint{2.767519in}{2.731608in}}%
\pgfpathlineto{\pgfqpoint{2.778719in}{2.745343in}}%
\pgfpathlineto{\pgfqpoint{2.789907in}{2.760779in}}%
\pgfpathlineto{\pgfqpoint{2.801087in}{2.777782in}}%
\pgfpathlineto{\pgfqpoint{2.812260in}{2.796220in}}%
\pgfpathlineto{\pgfqpoint{2.806199in}{2.803716in}}%
\pgfpathlineto{\pgfqpoint{2.800138in}{2.811868in}}%
\pgfpathlineto{\pgfqpoint{2.794078in}{2.820580in}}%
\pgfpathlineto{\pgfqpoint{2.788018in}{2.829738in}}%
\pgfpathlineto{\pgfqpoint{2.781959in}{2.839226in}}%
\pgfpathlineto{\pgfqpoint{2.770792in}{2.822967in}}%
\pgfpathlineto{\pgfqpoint{2.759622in}{2.807434in}}%
\pgfpathlineto{\pgfqpoint{2.748445in}{2.792861in}}%
\pgfpathlineto{\pgfqpoint{2.737260in}{2.779483in}}%
\pgfpathlineto{\pgfqpoint{2.726065in}{2.767419in}}%
\pgfpathlineto{\pgfqpoint{2.732112in}{2.757178in}}%
\pgfpathlineto{\pgfqpoint{2.738160in}{2.747223in}}%
\pgfpathlineto{\pgfqpoint{2.744208in}{2.737607in}}%
\pgfpathlineto{\pgfqpoint{2.750258in}{2.728385in}}%
\pgfpathclose%
\pgfusepath{stroke,fill}%
\end{pgfscope}%
\begin{pgfscope}%
\pgfpathrectangle{\pgfqpoint{0.887500in}{0.275000in}}{\pgfqpoint{4.225000in}{4.225000in}}%
\pgfusepath{clip}%
\pgfsetbuttcap%
\pgfsetroundjoin%
\definecolor{currentfill}{rgb}{0.804182,0.882046,0.114965}%
\pgfsetfillcolor{currentfill}%
\pgfsetfillopacity{0.700000}%
\pgfsetlinewidth{0.501875pt}%
\definecolor{currentstroke}{rgb}{1.000000,1.000000,1.000000}%
\pgfsetstrokecolor{currentstroke}%
\pgfsetstrokeopacity{0.500000}%
\pgfsetdash{}{0pt}%
\pgfpathmoveto{\pgfqpoint{3.000210in}{3.474397in}}%
\pgfpathlineto{\pgfqpoint{3.011439in}{3.478032in}}%
\pgfpathlineto{\pgfqpoint{3.022663in}{3.480659in}}%
\pgfpathlineto{\pgfqpoint{3.033882in}{3.482673in}}%
\pgfpathlineto{\pgfqpoint{3.045096in}{3.484468in}}%
\pgfpathlineto{\pgfqpoint{3.056303in}{3.486440in}}%
\pgfpathlineto{\pgfqpoint{3.050161in}{3.489453in}}%
\pgfpathlineto{\pgfqpoint{3.044024in}{3.492196in}}%
\pgfpathlineto{\pgfqpoint{3.037894in}{3.494740in}}%
\pgfpathlineto{\pgfqpoint{3.031769in}{3.497159in}}%
\pgfpathlineto{\pgfqpoint{3.025649in}{3.499522in}}%
\pgfpathlineto{\pgfqpoint{3.014460in}{3.497553in}}%
\pgfpathlineto{\pgfqpoint{3.003264in}{3.496178in}}%
\pgfpathlineto{\pgfqpoint{2.992061in}{3.494902in}}%
\pgfpathlineto{\pgfqpoint{2.980852in}{3.493231in}}%
\pgfpathlineto{\pgfqpoint{2.969639in}{3.490667in}}%
\pgfpathlineto{\pgfqpoint{2.975742in}{3.487657in}}%
\pgfpathlineto{\pgfqpoint{2.981850in}{3.484581in}}%
\pgfpathlineto{\pgfqpoint{2.987965in}{3.481383in}}%
\pgfpathlineto{\pgfqpoint{2.994084in}{3.478007in}}%
\pgfpathclose%
\pgfusepath{stroke,fill}%
\end{pgfscope}%
\begin{pgfscope}%
\pgfpathrectangle{\pgfqpoint{0.887500in}{0.275000in}}{\pgfqpoint{4.225000in}{4.225000in}}%
\pgfusepath{clip}%
\pgfsetbuttcap%
\pgfsetroundjoin%
\definecolor{currentfill}{rgb}{0.120092,0.600104,0.542530}%
\pgfsetfillcolor{currentfill}%
\pgfsetfillopacity{0.700000}%
\pgfsetlinewidth{0.501875pt}%
\definecolor{currentstroke}{rgb}{1.000000,1.000000,1.000000}%
\pgfsetstrokecolor{currentstroke}%
\pgfsetstrokeopacity{0.500000}%
\pgfsetdash{}{0pt}%
\pgfpathmoveto{\pgfqpoint{2.700130in}{2.671733in}}%
\pgfpathlineto{\pgfqpoint{2.711371in}{2.681082in}}%
\pgfpathlineto{\pgfqpoint{2.722613in}{2.690122in}}%
\pgfpathlineto{\pgfqpoint{2.733852in}{2.699274in}}%
\pgfpathlineto{\pgfqpoint{2.745084in}{2.708958in}}%
\pgfpathlineto{\pgfqpoint{2.756307in}{2.719596in}}%
\pgfpathlineto{\pgfqpoint{2.750258in}{2.728385in}}%
\pgfpathlineto{\pgfqpoint{2.744208in}{2.737607in}}%
\pgfpathlineto{\pgfqpoint{2.738160in}{2.747223in}}%
\pgfpathlineto{\pgfqpoint{2.732112in}{2.757178in}}%
\pgfpathlineto{\pgfqpoint{2.726065in}{2.767419in}}%
\pgfpathlineto{\pgfqpoint{2.714859in}{2.756400in}}%
\pgfpathlineto{\pgfqpoint{2.703647in}{2.746083in}}%
\pgfpathlineto{\pgfqpoint{2.692430in}{2.736125in}}%
\pgfpathlineto{\pgfqpoint{2.681211in}{2.726180in}}%
\pgfpathlineto{\pgfqpoint{2.669993in}{2.715907in}}%
\pgfpathlineto{\pgfqpoint{2.676014in}{2.707024in}}%
\pgfpathlineto{\pgfqpoint{2.682037in}{2.698226in}}%
\pgfpathlineto{\pgfqpoint{2.688064in}{2.689451in}}%
\pgfpathlineto{\pgfqpoint{2.694095in}{2.680639in}}%
\pgfpathclose%
\pgfusepath{stroke,fill}%
\end{pgfscope}%
\begin{pgfscope}%
\pgfpathrectangle{\pgfqpoint{0.887500in}{0.275000in}}{\pgfqpoint{4.225000in}{4.225000in}}%
\pgfusepath{clip}%
\pgfsetbuttcap%
\pgfsetroundjoin%
\definecolor{currentfill}{rgb}{0.119483,0.614817,0.537692}%
\pgfsetfillcolor{currentfill}%
\pgfsetfillopacity{0.700000}%
\pgfsetlinewidth{0.501875pt}%
\definecolor{currentstroke}{rgb}{1.000000,1.000000,1.000000}%
\pgfsetstrokecolor{currentstroke}%
\pgfsetstrokeopacity{0.500000}%
\pgfsetdash{}{0pt}%
\pgfpathmoveto{\pgfqpoint{1.495090in}{2.721995in}}%
\pgfpathlineto{\pgfqpoint{1.506682in}{2.725301in}}%
\pgfpathlineto{\pgfqpoint{1.518267in}{2.728605in}}%
\pgfpathlineto{\pgfqpoint{1.529847in}{2.731910in}}%
\pgfpathlineto{\pgfqpoint{1.541421in}{2.735215in}}%
\pgfpathlineto{\pgfqpoint{1.552990in}{2.738522in}}%
\pgfpathlineto{\pgfqpoint{1.547366in}{2.746065in}}%
\pgfpathlineto{\pgfqpoint{1.541747in}{2.753592in}}%
\pgfpathlineto{\pgfqpoint{1.536131in}{2.761101in}}%
\pgfpathlineto{\pgfqpoint{1.530521in}{2.768594in}}%
\pgfpathlineto{\pgfqpoint{1.524914in}{2.776069in}}%
\pgfpathlineto{\pgfqpoint{1.513360in}{2.772746in}}%
\pgfpathlineto{\pgfqpoint{1.501800in}{2.769424in}}%
\pgfpathlineto{\pgfqpoint{1.490235in}{2.766103in}}%
\pgfpathlineto{\pgfqpoint{1.478664in}{2.762783in}}%
\pgfpathlineto{\pgfqpoint{1.467087in}{2.759462in}}%
\pgfpathlineto{\pgfqpoint{1.472679in}{2.752004in}}%
\pgfpathlineto{\pgfqpoint{1.478275in}{2.744528in}}%
\pgfpathlineto{\pgfqpoint{1.483876in}{2.737034in}}%
\pgfpathlineto{\pgfqpoint{1.489481in}{2.729523in}}%
\pgfpathclose%
\pgfusepath{stroke,fill}%
\end{pgfscope}%
\begin{pgfscope}%
\pgfpathrectangle{\pgfqpoint{0.887500in}{0.275000in}}{\pgfqpoint{4.225000in}{4.225000in}}%
\pgfusepath{clip}%
\pgfsetbuttcap%
\pgfsetroundjoin%
\definecolor{currentfill}{rgb}{0.124395,0.578002,0.548287}%
\pgfsetfillcolor{currentfill}%
\pgfsetfillopacity{0.700000}%
\pgfsetlinewidth{0.501875pt}%
\definecolor{currentstroke}{rgb}{1.000000,1.000000,1.000000}%
\pgfsetstrokecolor{currentstroke}%
\pgfsetstrokeopacity{0.500000}%
\pgfsetdash{}{0pt}%
\pgfpathmoveto{\pgfqpoint{2.041438in}{2.642506in}}%
\pgfpathlineto{\pgfqpoint{2.052898in}{2.645810in}}%
\pgfpathlineto{\pgfqpoint{2.064352in}{2.649098in}}%
\pgfpathlineto{\pgfqpoint{2.075801in}{2.652366in}}%
\pgfpathlineto{\pgfqpoint{2.087245in}{2.655614in}}%
\pgfpathlineto{\pgfqpoint{2.098684in}{2.658857in}}%
\pgfpathlineto{\pgfqpoint{2.092861in}{2.666863in}}%
\pgfpathlineto{\pgfqpoint{2.087043in}{2.674857in}}%
\pgfpathlineto{\pgfqpoint{2.081229in}{2.682838in}}%
\pgfpathlineto{\pgfqpoint{2.075419in}{2.690806in}}%
\pgfpathlineto{\pgfqpoint{2.069613in}{2.698761in}}%
\pgfpathlineto{\pgfqpoint{2.058186in}{2.695525in}}%
\pgfpathlineto{\pgfqpoint{2.046754in}{2.692283in}}%
\pgfpathlineto{\pgfqpoint{2.035318in}{2.689016in}}%
\pgfpathlineto{\pgfqpoint{2.023876in}{2.685726in}}%
\pgfpathlineto{\pgfqpoint{2.012429in}{2.682418in}}%
\pgfpathlineto{\pgfqpoint{2.018222in}{2.674465in}}%
\pgfpathlineto{\pgfqpoint{2.024020in}{2.666497in}}%
\pgfpathlineto{\pgfqpoint{2.029822in}{2.658514in}}%
\pgfpathlineto{\pgfqpoint{2.035628in}{2.650517in}}%
\pgfpathclose%
\pgfusepath{stroke,fill}%
\end{pgfscope}%
\begin{pgfscope}%
\pgfpathrectangle{\pgfqpoint{0.887500in}{0.275000in}}{\pgfqpoint{4.225000in}{4.225000in}}%
\pgfusepath{clip}%
\pgfsetbuttcap%
\pgfsetroundjoin%
\definecolor{currentfill}{rgb}{0.772852,0.877868,0.131109}%
\pgfsetfillcolor{currentfill}%
\pgfsetfillopacity{0.700000}%
\pgfsetlinewidth{0.501875pt}%
\definecolor{currentstroke}{rgb}{1.000000,1.000000,1.000000}%
\pgfsetstrokecolor{currentstroke}%
\pgfsetstrokeopacity{0.500000}%
\pgfsetdash{}{0pt}%
\pgfpathmoveto{\pgfqpoint{3.459681in}{3.422752in}}%
\pgfpathlineto{\pgfqpoint{3.470801in}{3.426420in}}%
\pgfpathlineto{\pgfqpoint{3.481915in}{3.430040in}}%
\pgfpathlineto{\pgfqpoint{3.493024in}{3.433607in}}%
\pgfpathlineto{\pgfqpoint{3.504127in}{3.437121in}}%
\pgfpathlineto{\pgfqpoint{3.515224in}{3.440588in}}%
\pgfpathlineto{\pgfqpoint{3.508965in}{3.451596in}}%
\pgfpathlineto{\pgfqpoint{3.502707in}{3.462323in}}%
\pgfpathlineto{\pgfqpoint{3.496449in}{3.472687in}}%
\pgfpathlineto{\pgfqpoint{3.490189in}{3.482610in}}%
\pgfpathlineto{\pgfqpoint{3.483927in}{3.492012in}}%
\pgfpathlineto{\pgfqpoint{3.472835in}{3.488383in}}%
\pgfpathlineto{\pgfqpoint{3.461739in}{3.484743in}}%
\pgfpathlineto{\pgfqpoint{3.450637in}{3.481093in}}%
\pgfpathlineto{\pgfqpoint{3.439531in}{3.477424in}}%
\pgfpathlineto{\pgfqpoint{3.428419in}{3.473729in}}%
\pgfpathlineto{\pgfqpoint{3.434672in}{3.464331in}}%
\pgfpathlineto{\pgfqpoint{3.440924in}{3.454471in}}%
\pgfpathlineto{\pgfqpoint{3.447176in}{3.444212in}}%
\pgfpathlineto{\pgfqpoint{3.453428in}{3.433619in}}%
\pgfpathclose%
\pgfusepath{stroke,fill}%
\end{pgfscope}%
\begin{pgfscope}%
\pgfpathrectangle{\pgfqpoint{0.887500in}{0.275000in}}{\pgfqpoint{4.225000in}{4.225000in}}%
\pgfusepath{clip}%
\pgfsetbuttcap%
\pgfsetroundjoin%
\definecolor{currentfill}{rgb}{0.395174,0.797475,0.367757}%
\pgfsetfillcolor{currentfill}%
\pgfsetfillopacity{0.700000}%
\pgfsetlinewidth{0.501875pt}%
\definecolor{currentstroke}{rgb}{1.000000,1.000000,1.000000}%
\pgfsetstrokecolor{currentstroke}%
\pgfsetstrokeopacity{0.500000}%
\pgfsetdash{}{0pt}%
\pgfpathmoveto{\pgfqpoint{4.067473in}{3.130356in}}%
\pgfpathlineto{\pgfqpoint{4.078438in}{3.133452in}}%
\pgfpathlineto{\pgfqpoint{4.089397in}{3.136562in}}%
\pgfpathlineto{\pgfqpoint{4.100351in}{3.139684in}}%
\pgfpathlineto{\pgfqpoint{4.111301in}{3.142819in}}%
\pgfpathlineto{\pgfqpoint{4.122245in}{3.145964in}}%
\pgfpathlineto{\pgfqpoint{4.115864in}{3.158215in}}%
\pgfpathlineto{\pgfqpoint{4.109486in}{3.170429in}}%
\pgfpathlineto{\pgfqpoint{4.103109in}{3.182609in}}%
\pgfpathlineto{\pgfqpoint{4.096734in}{3.194761in}}%
\pgfpathlineto{\pgfqpoint{4.090362in}{3.206890in}}%
\pgfpathlineto{\pgfqpoint{4.079431in}{3.204084in}}%
\pgfpathlineto{\pgfqpoint{4.068493in}{3.201250in}}%
\pgfpathlineto{\pgfqpoint{4.057549in}{3.198387in}}%
\pgfpathlineto{\pgfqpoint{4.046599in}{3.195496in}}%
\pgfpathlineto{\pgfqpoint{4.035642in}{3.192577in}}%
\pgfpathlineto{\pgfqpoint{4.042005in}{3.180231in}}%
\pgfpathlineto{\pgfqpoint{4.048369in}{3.167828in}}%
\pgfpathlineto{\pgfqpoint{4.054735in}{3.155376in}}%
\pgfpathlineto{\pgfqpoint{4.061103in}{3.142883in}}%
\pgfpathclose%
\pgfusepath{stroke,fill}%
\end{pgfscope}%
\begin{pgfscope}%
\pgfpathrectangle{\pgfqpoint{0.887500in}{0.275000in}}{\pgfqpoint{4.225000in}{4.225000in}}%
\pgfusepath{clip}%
\pgfsetbuttcap%
\pgfsetroundjoin%
\definecolor{currentfill}{rgb}{0.132444,0.552216,0.553018}%
\pgfsetfillcolor{currentfill}%
\pgfsetfillopacity{0.700000}%
\pgfsetlinewidth{0.501875pt}%
\definecolor{currentstroke}{rgb}{1.000000,1.000000,1.000000}%
\pgfsetstrokecolor{currentstroke}%
\pgfsetstrokeopacity{0.500000}%
\pgfsetdash{}{0pt}%
\pgfpathmoveto{\pgfqpoint{2.357919in}{2.588214in}}%
\pgfpathlineto{\pgfqpoint{2.369300in}{2.591701in}}%
\pgfpathlineto{\pgfqpoint{2.380678in}{2.595072in}}%
\pgfpathlineto{\pgfqpoint{2.392054in}{2.598283in}}%
\pgfpathlineto{\pgfqpoint{2.403427in}{2.601293in}}%
\pgfpathlineto{\pgfqpoint{2.414797in}{2.604126in}}%
\pgfpathlineto{\pgfqpoint{2.408863in}{2.612447in}}%
\pgfpathlineto{\pgfqpoint{2.402933in}{2.620804in}}%
\pgfpathlineto{\pgfqpoint{2.397006in}{2.629199in}}%
\pgfpathlineto{\pgfqpoint{2.391082in}{2.637628in}}%
\pgfpathlineto{\pgfqpoint{2.385161in}{2.646087in}}%
\pgfpathlineto{\pgfqpoint{2.373815in}{2.642478in}}%
\pgfpathlineto{\pgfqpoint{2.362463in}{2.638948in}}%
\pgfpathlineto{\pgfqpoint{2.351104in}{2.635466in}}%
\pgfpathlineto{\pgfqpoint{2.339739in}{2.632023in}}%
\pgfpathlineto{\pgfqpoint{2.328369in}{2.628608in}}%
\pgfpathlineto{\pgfqpoint{2.334270in}{2.620603in}}%
\pgfpathlineto{\pgfqpoint{2.340175in}{2.612566in}}%
\pgfpathlineto{\pgfqpoint{2.346085in}{2.604492in}}%
\pgfpathlineto{\pgfqpoint{2.352000in}{2.596374in}}%
\pgfpathclose%
\pgfusepath{stroke,fill}%
\end{pgfscope}%
\begin{pgfscope}%
\pgfpathrectangle{\pgfqpoint{0.887500in}{0.275000in}}{\pgfqpoint{4.225000in}{4.225000in}}%
\pgfusepath{clip}%
\pgfsetbuttcap%
\pgfsetroundjoin%
\definecolor{currentfill}{rgb}{0.344074,0.780029,0.397381}%
\pgfsetfillcolor{currentfill}%
\pgfsetfillopacity{0.700000}%
\pgfsetlinewidth{0.501875pt}%
\definecolor{currentstroke}{rgb}{1.000000,1.000000,1.000000}%
\pgfsetstrokecolor{currentstroke}%
\pgfsetstrokeopacity{0.500000}%
\pgfsetdash{}{0pt}%
\pgfpathmoveto{\pgfqpoint{4.154173in}{3.083903in}}%
\pgfpathlineto{\pgfqpoint{4.165122in}{3.087279in}}%
\pgfpathlineto{\pgfqpoint{4.176067in}{3.090671in}}%
\pgfpathlineto{\pgfqpoint{4.187007in}{3.094077in}}%
\pgfpathlineto{\pgfqpoint{4.197942in}{3.097497in}}%
\pgfpathlineto{\pgfqpoint{4.191549in}{3.109958in}}%
\pgfpathlineto{\pgfqpoint{4.185155in}{3.122300in}}%
\pgfpathlineto{\pgfqpoint{4.178761in}{3.134526in}}%
\pgfpathlineto{\pgfqpoint{4.172367in}{3.146638in}}%
\pgfpathlineto{\pgfqpoint{4.165973in}{3.158640in}}%
\pgfpathlineto{\pgfqpoint{4.155049in}{3.155458in}}%
\pgfpathlineto{\pgfqpoint{4.144119in}{3.152285in}}%
\pgfpathlineto{\pgfqpoint{4.133185in}{3.149119in}}%
\pgfpathlineto{\pgfqpoint{4.122245in}{3.145964in}}%
\pgfpathlineto{\pgfqpoint{4.128628in}{3.133670in}}%
\pgfpathlineto{\pgfqpoint{4.135013in}{3.121327in}}%
\pgfpathlineto{\pgfqpoint{4.141398in}{3.108924in}}%
\pgfpathlineto{\pgfqpoint{4.147785in}{3.096453in}}%
\pgfpathclose%
\pgfusepath{stroke,fill}%
\end{pgfscope}%
\begin{pgfscope}%
\pgfpathrectangle{\pgfqpoint{0.887500in}{0.275000in}}{\pgfqpoint{4.225000in}{4.225000in}}%
\pgfusepath{clip}%
\pgfsetbuttcap%
\pgfsetroundjoin%
\definecolor{currentfill}{rgb}{0.720391,0.870350,0.162603}%
\pgfsetfillcolor{currentfill}%
\pgfsetfillopacity{0.700000}%
\pgfsetlinewidth{0.501875pt}%
\definecolor{currentstroke}{rgb}{1.000000,1.000000,1.000000}%
\pgfsetstrokecolor{currentstroke}%
\pgfsetstrokeopacity{0.500000}%
\pgfsetdash{}{0pt}%
\pgfpathmoveto{\pgfqpoint{3.546546in}{3.384082in}}%
\pgfpathlineto{\pgfqpoint{3.557643in}{3.387475in}}%
\pgfpathlineto{\pgfqpoint{3.568735in}{3.390838in}}%
\pgfpathlineto{\pgfqpoint{3.579821in}{3.394177in}}%
\pgfpathlineto{\pgfqpoint{3.590902in}{3.397499in}}%
\pgfpathlineto{\pgfqpoint{3.601977in}{3.400809in}}%
\pgfpathlineto{\pgfqpoint{3.595699in}{3.412005in}}%
\pgfpathlineto{\pgfqpoint{3.589425in}{3.423342in}}%
\pgfpathlineto{\pgfqpoint{3.583156in}{3.434747in}}%
\pgfpathlineto{\pgfqpoint{3.576890in}{3.446143in}}%
\pgfpathlineto{\pgfqpoint{3.570626in}{3.457452in}}%
\pgfpathlineto{\pgfqpoint{3.559556in}{3.454119in}}%
\pgfpathlineto{\pgfqpoint{3.548482in}{3.450773in}}%
\pgfpathlineto{\pgfqpoint{3.537401in}{3.447407in}}%
\pgfpathlineto{\pgfqpoint{3.526315in}{3.444014in}}%
\pgfpathlineto{\pgfqpoint{3.515224in}{3.440588in}}%
\pgfpathlineto{\pgfqpoint{3.521483in}{3.429376in}}%
\pgfpathlineto{\pgfqpoint{3.527744in}{3.418040in}}%
\pgfpathlineto{\pgfqpoint{3.534007in}{3.406661in}}%
\pgfpathlineto{\pgfqpoint{3.540275in}{3.395317in}}%
\pgfpathclose%
\pgfusepath{stroke,fill}%
\end{pgfscope}%
\begin{pgfscope}%
\pgfpathrectangle{\pgfqpoint{0.887500in}{0.275000in}}{\pgfqpoint{4.225000in}{4.225000in}}%
\pgfusepath{clip}%
\pgfsetbuttcap%
\pgfsetroundjoin%
\definecolor{currentfill}{rgb}{0.125394,0.574318,0.549086}%
\pgfsetfillcolor{currentfill}%
\pgfsetfillopacity{0.700000}%
\pgfsetlinewidth{0.501875pt}%
\definecolor{currentstroke}{rgb}{1.000000,1.000000,1.000000}%
\pgfsetstrokecolor{currentstroke}%
\pgfsetstrokeopacity{0.500000}%
\pgfsetdash{}{0pt}%
\pgfpathmoveto{\pgfqpoint{2.643998in}{2.614962in}}%
\pgfpathlineto{\pgfqpoint{2.655219in}{2.626972in}}%
\pgfpathlineto{\pgfqpoint{2.666440in}{2.638978in}}%
\pgfpathlineto{\pgfqpoint{2.677664in}{2.650649in}}%
\pgfpathlineto{\pgfqpoint{2.688894in}{2.661657in}}%
\pgfpathlineto{\pgfqpoint{2.700130in}{2.671733in}}%
\pgfpathlineto{\pgfqpoint{2.694095in}{2.680639in}}%
\pgfpathlineto{\pgfqpoint{2.688064in}{2.689451in}}%
\pgfpathlineto{\pgfqpoint{2.682037in}{2.698226in}}%
\pgfpathlineto{\pgfqpoint{2.676014in}{2.707024in}}%
\pgfpathlineto{\pgfqpoint{2.669993in}{2.715907in}}%
\pgfpathlineto{\pgfqpoint{2.658781in}{2.704962in}}%
\pgfpathlineto{\pgfqpoint{2.647576in}{2.693196in}}%
\pgfpathlineto{\pgfqpoint{2.636377in}{2.680873in}}%
\pgfpathlineto{\pgfqpoint{2.625179in}{2.668316in}}%
\pgfpathlineto{\pgfqpoint{2.613980in}{2.655846in}}%
\pgfpathlineto{\pgfqpoint{2.619975in}{2.647778in}}%
\pgfpathlineto{\pgfqpoint{2.625973in}{2.639744in}}%
\pgfpathlineto{\pgfqpoint{2.631975in}{2.631653in}}%
\pgfpathlineto{\pgfqpoint{2.637983in}{2.623417in}}%
\pgfpathclose%
\pgfusepath{stroke,fill}%
\end{pgfscope}%
\begin{pgfscope}%
\pgfpathrectangle{\pgfqpoint{0.887500in}{0.275000in}}{\pgfqpoint{4.225000in}{4.225000in}}%
\pgfusepath{clip}%
\pgfsetbuttcap%
\pgfsetroundjoin%
\definecolor{currentfill}{rgb}{0.668054,0.861999,0.196293}%
\pgfsetfillcolor{currentfill}%
\pgfsetfillopacity{0.700000}%
\pgfsetlinewidth{0.501875pt}%
\definecolor{currentstroke}{rgb}{1.000000,1.000000,1.000000}%
\pgfsetstrokecolor{currentstroke}%
\pgfsetstrokeopacity{0.500000}%
\pgfsetdash{}{0pt}%
\pgfpathmoveto{\pgfqpoint{3.633437in}{3.346312in}}%
\pgfpathlineto{\pgfqpoint{3.644514in}{3.349669in}}%
\pgfpathlineto{\pgfqpoint{3.655586in}{3.353005in}}%
\pgfpathlineto{\pgfqpoint{3.666652in}{3.356316in}}%
\pgfpathlineto{\pgfqpoint{3.677712in}{3.359596in}}%
\pgfpathlineto{\pgfqpoint{3.688766in}{3.362842in}}%
\pgfpathlineto{\pgfqpoint{3.682459in}{3.373589in}}%
\pgfpathlineto{\pgfqpoint{3.676155in}{3.384335in}}%
\pgfpathlineto{\pgfqpoint{3.669855in}{3.395120in}}%
\pgfpathlineto{\pgfqpoint{3.663559in}{3.405983in}}%
\pgfpathlineto{\pgfqpoint{3.657268in}{3.416962in}}%
\pgfpathlineto{\pgfqpoint{3.646222in}{3.413822in}}%
\pgfpathlineto{\pgfqpoint{3.635169in}{3.410626in}}%
\pgfpathlineto{\pgfqpoint{3.624111in}{3.407384in}}%
\pgfpathlineto{\pgfqpoint{3.613047in}{3.404108in}}%
\pgfpathlineto{\pgfqpoint{3.601977in}{3.400809in}}%
\pgfpathlineto{\pgfqpoint{3.608260in}{3.389760in}}%
\pgfpathlineto{\pgfqpoint{3.614548in}{3.378821in}}%
\pgfpathlineto{\pgfqpoint{3.620841in}{3.367957in}}%
\pgfpathlineto{\pgfqpoint{3.627137in}{3.357133in}}%
\pgfpathclose%
\pgfusepath{stroke,fill}%
\end{pgfscope}%
\begin{pgfscope}%
\pgfpathrectangle{\pgfqpoint{0.887500in}{0.275000in}}{\pgfqpoint{4.225000in}{4.225000in}}%
\pgfusepath{clip}%
\pgfsetbuttcap%
\pgfsetroundjoin%
\definecolor{currentfill}{rgb}{0.121148,0.592739,0.544641}%
\pgfsetfillcolor{currentfill}%
\pgfsetfillopacity{0.700000}%
\pgfsetlinewidth{0.501875pt}%
\definecolor{currentstroke}{rgb}{1.000000,1.000000,1.000000}%
\pgfsetstrokecolor{currentstroke}%
\pgfsetstrokeopacity{0.500000}%
\pgfsetdash{}{0pt}%
\pgfpathmoveto{\pgfqpoint{1.811343in}{2.672224in}}%
\pgfpathlineto{\pgfqpoint{1.822863in}{2.675478in}}%
\pgfpathlineto{\pgfqpoint{1.834377in}{2.678738in}}%
\pgfpathlineto{\pgfqpoint{1.845885in}{2.682007in}}%
\pgfpathlineto{\pgfqpoint{1.857387in}{2.685286in}}%
\pgfpathlineto{\pgfqpoint{1.868883in}{2.688577in}}%
\pgfpathlineto{\pgfqpoint{1.863141in}{2.696409in}}%
\pgfpathlineto{\pgfqpoint{1.857403in}{2.704223in}}%
\pgfpathlineto{\pgfqpoint{1.851669in}{2.712020in}}%
\pgfpathlineto{\pgfqpoint{1.845939in}{2.719801in}}%
\pgfpathlineto{\pgfqpoint{1.840214in}{2.727565in}}%
\pgfpathlineto{\pgfqpoint{1.828731in}{2.724269in}}%
\pgfpathlineto{\pgfqpoint{1.817242in}{2.720984in}}%
\pgfpathlineto{\pgfqpoint{1.805747in}{2.717709in}}%
\pgfpathlineto{\pgfqpoint{1.794246in}{2.714440in}}%
\pgfpathlineto{\pgfqpoint{1.782740in}{2.711178in}}%
\pgfpathlineto{\pgfqpoint{1.788452in}{2.703422in}}%
\pgfpathlineto{\pgfqpoint{1.794168in}{2.695649in}}%
\pgfpathlineto{\pgfqpoint{1.799889in}{2.687859in}}%
\pgfpathlineto{\pgfqpoint{1.805614in}{2.680051in}}%
\pgfpathclose%
\pgfusepath{stroke,fill}%
\end{pgfscope}%
\begin{pgfscope}%
\pgfpathrectangle{\pgfqpoint{0.887500in}{0.275000in}}{\pgfqpoint{4.225000in}{4.225000in}}%
\pgfusepath{clip}%
\pgfsetbuttcap%
\pgfsetroundjoin%
\definecolor{currentfill}{rgb}{0.804182,0.882046,0.114965}%
\pgfsetfillcolor{currentfill}%
\pgfsetfillopacity{0.700000}%
\pgfsetlinewidth{0.501875pt}%
\definecolor{currentstroke}{rgb}{1.000000,1.000000,1.000000}%
\pgfsetstrokecolor{currentstroke}%
\pgfsetstrokeopacity{0.500000}%
\pgfsetdash{}{0pt}%
\pgfpathmoveto{\pgfqpoint{3.230021in}{3.456930in}}%
\pgfpathlineto{\pgfqpoint{3.241198in}{3.460654in}}%
\pgfpathlineto{\pgfqpoint{3.252370in}{3.464510in}}%
\pgfpathlineto{\pgfqpoint{3.263538in}{3.468466in}}%
\pgfpathlineto{\pgfqpoint{3.274701in}{3.472491in}}%
\pgfpathlineto{\pgfqpoint{3.285860in}{3.476554in}}%
\pgfpathlineto{\pgfqpoint{3.279638in}{3.484074in}}%
\pgfpathlineto{\pgfqpoint{3.273419in}{3.491307in}}%
\pgfpathlineto{\pgfqpoint{3.267203in}{3.498268in}}%
\pgfpathlineto{\pgfqpoint{3.260989in}{3.504972in}}%
\pgfpathlineto{\pgfqpoint{3.254779in}{3.511434in}}%
\pgfpathlineto{\pgfqpoint{3.243635in}{3.507636in}}%
\pgfpathlineto{\pgfqpoint{3.232485in}{3.503873in}}%
\pgfpathlineto{\pgfqpoint{3.221331in}{3.500164in}}%
\pgfpathlineto{\pgfqpoint{3.210172in}{3.496521in}}%
\pgfpathlineto{\pgfqpoint{3.199008in}{3.492958in}}%
\pgfpathlineto{\pgfqpoint{3.205204in}{3.486203in}}%
\pgfpathlineto{\pgfqpoint{3.211403in}{3.479235in}}%
\pgfpathlineto{\pgfqpoint{3.217606in}{3.472043in}}%
\pgfpathlineto{\pgfqpoint{3.223812in}{3.464612in}}%
\pgfpathclose%
\pgfusepath{stroke,fill}%
\end{pgfscope}%
\begin{pgfscope}%
\pgfpathrectangle{\pgfqpoint{0.887500in}{0.275000in}}{\pgfqpoint{4.225000in}{4.225000in}}%
\pgfusepath{clip}%
\pgfsetbuttcap%
\pgfsetroundjoin%
\definecolor{currentfill}{rgb}{0.458674,0.816363,0.329727}%
\pgfsetfillcolor{currentfill}%
\pgfsetfillopacity{0.700000}%
\pgfsetlinewidth{0.501875pt}%
\definecolor{currentstroke}{rgb}{1.000000,1.000000,1.000000}%
\pgfsetstrokecolor{currentstroke}%
\pgfsetstrokeopacity{0.500000}%
\pgfsetdash{}{0pt}%
\pgfpathmoveto{\pgfqpoint{3.980771in}{3.177592in}}%
\pgfpathlineto{\pgfqpoint{3.991757in}{3.180640in}}%
\pgfpathlineto{\pgfqpoint{4.002737in}{3.183663in}}%
\pgfpathlineto{\pgfqpoint{4.013712in}{3.186660in}}%
\pgfpathlineto{\pgfqpoint{4.024680in}{3.189632in}}%
\pgfpathlineto{\pgfqpoint{4.035642in}{3.192577in}}%
\pgfpathlineto{\pgfqpoint{4.029281in}{3.204858in}}%
\pgfpathlineto{\pgfqpoint{4.022921in}{3.217065in}}%
\pgfpathlineto{\pgfqpoint{4.016562in}{3.229191in}}%
\pgfpathlineto{\pgfqpoint{4.010204in}{3.241227in}}%
\pgfpathlineto{\pgfqpoint{4.003846in}{3.253165in}}%
\pgfpathlineto{\pgfqpoint{3.992888in}{3.250219in}}%
\pgfpathlineto{\pgfqpoint{3.981924in}{3.247237in}}%
\pgfpathlineto{\pgfqpoint{3.970953in}{3.244219in}}%
\pgfpathlineto{\pgfqpoint{3.959976in}{3.241167in}}%
\pgfpathlineto{\pgfqpoint{3.948994in}{3.238081in}}%
\pgfpathlineto{\pgfqpoint{3.955347in}{3.226167in}}%
\pgfpathlineto{\pgfqpoint{3.961702in}{3.214158in}}%
\pgfpathlineto{\pgfqpoint{3.968057in}{3.202056in}}%
\pgfpathlineto{\pgfqpoint{3.974413in}{3.189866in}}%
\pgfpathclose%
\pgfusepath{stroke,fill}%
\end{pgfscope}%
\begin{pgfscope}%
\pgfpathrectangle{\pgfqpoint{0.887500in}{0.275000in}}{\pgfqpoint{4.225000in}{4.225000in}}%
\pgfusepath{clip}%
\pgfsetbuttcap%
\pgfsetroundjoin%
\definecolor{currentfill}{rgb}{0.506271,0.828786,0.300362}%
\pgfsetfillcolor{currentfill}%
\pgfsetfillopacity{0.700000}%
\pgfsetlinewidth{0.501875pt}%
\definecolor{currentstroke}{rgb}{1.000000,1.000000,1.000000}%
\pgfsetstrokecolor{currentstroke}%
\pgfsetstrokeopacity{0.500000}%
\pgfsetdash{}{0pt}%
\pgfpathmoveto{\pgfqpoint{3.893987in}{3.222025in}}%
\pgfpathlineto{\pgfqpoint{3.905001in}{3.225334in}}%
\pgfpathlineto{\pgfqpoint{3.916008in}{3.228590in}}%
\pgfpathlineto{\pgfqpoint{3.927010in}{3.231797in}}%
\pgfpathlineto{\pgfqpoint{3.938005in}{3.234959in}}%
\pgfpathlineto{\pgfqpoint{3.948994in}{3.238081in}}%
\pgfpathlineto{\pgfqpoint{3.942641in}{3.249899in}}%
\pgfpathlineto{\pgfqpoint{3.936290in}{3.261635in}}%
\pgfpathlineto{\pgfqpoint{3.929940in}{3.273300in}}%
\pgfpathlineto{\pgfqpoint{3.923593in}{3.284907in}}%
\pgfpathlineto{\pgfqpoint{3.917247in}{3.296469in}}%
\pgfpathlineto{\pgfqpoint{3.906267in}{3.293506in}}%
\pgfpathlineto{\pgfqpoint{3.895281in}{3.290555in}}%
\pgfpathlineto{\pgfqpoint{3.884289in}{3.287597in}}%
\pgfpathlineto{\pgfqpoint{3.873292in}{3.284613in}}%
\pgfpathlineto{\pgfqpoint{3.862289in}{3.281584in}}%
\pgfpathlineto{\pgfqpoint{3.868624in}{3.269761in}}%
\pgfpathlineto{\pgfqpoint{3.874961in}{3.257882in}}%
\pgfpathlineto{\pgfqpoint{3.881300in}{3.245958in}}%
\pgfpathlineto{\pgfqpoint{3.887642in}{3.234001in}}%
\pgfpathclose%
\pgfusepath{stroke,fill}%
\end{pgfscope}%
\begin{pgfscope}%
\pgfpathrectangle{\pgfqpoint{0.887500in}{0.275000in}}{\pgfqpoint{4.225000in}{4.225000in}}%
\pgfusepath{clip}%
\pgfsetbuttcap%
\pgfsetroundjoin%
\definecolor{currentfill}{rgb}{0.616293,0.852709,0.230052}%
\pgfsetfillcolor{currentfill}%
\pgfsetfillopacity{0.700000}%
\pgfsetlinewidth{0.501875pt}%
\definecolor{currentstroke}{rgb}{1.000000,1.000000,1.000000}%
\pgfsetstrokecolor{currentstroke}%
\pgfsetstrokeopacity{0.500000}%
\pgfsetdash{}{0pt}%
\pgfpathmoveto{\pgfqpoint{3.720332in}{3.307728in}}%
\pgfpathlineto{\pgfqpoint{3.731388in}{3.311032in}}%
\pgfpathlineto{\pgfqpoint{3.742439in}{3.314325in}}%
\pgfpathlineto{\pgfqpoint{3.753485in}{3.317604in}}%
\pgfpathlineto{\pgfqpoint{3.764524in}{3.320861in}}%
\pgfpathlineto{\pgfqpoint{3.775558in}{3.324084in}}%
\pgfpathlineto{\pgfqpoint{3.769236in}{3.335293in}}%
\pgfpathlineto{\pgfqpoint{3.762913in}{3.346306in}}%
\pgfpathlineto{\pgfqpoint{3.756591in}{3.357158in}}%
\pgfpathlineto{\pgfqpoint{3.750270in}{3.367890in}}%
\pgfpathlineto{\pgfqpoint{3.743951in}{3.378538in}}%
\pgfpathlineto{\pgfqpoint{3.732925in}{3.375439in}}%
\pgfpathlineto{\pgfqpoint{3.721894in}{3.372336in}}%
\pgfpathlineto{\pgfqpoint{3.710858in}{3.369212in}}%
\pgfpathlineto{\pgfqpoint{3.699815in}{3.366049in}}%
\pgfpathlineto{\pgfqpoint{3.688766in}{3.362842in}}%
\pgfpathlineto{\pgfqpoint{3.695077in}{3.352056in}}%
\pgfpathlineto{\pgfqpoint{3.701389in}{3.341190in}}%
\pgfpathlineto{\pgfqpoint{3.707703in}{3.330207in}}%
\pgfpathlineto{\pgfqpoint{3.714017in}{3.319066in}}%
\pgfpathclose%
\pgfusepath{stroke,fill}%
\end{pgfscope}%
\begin{pgfscope}%
\pgfpathrectangle{\pgfqpoint{0.887500in}{0.275000in}}{\pgfqpoint{4.225000in}{4.225000in}}%
\pgfusepath{clip}%
\pgfsetbuttcap%
\pgfsetroundjoin%
\definecolor{currentfill}{rgb}{0.565498,0.842430,0.262877}%
\pgfsetfillcolor{currentfill}%
\pgfsetfillopacity{0.700000}%
\pgfsetlinewidth{0.501875pt}%
\definecolor{currentstroke}{rgb}{1.000000,1.000000,1.000000}%
\pgfsetstrokecolor{currentstroke}%
\pgfsetstrokeopacity{0.500000}%
\pgfsetdash{}{0pt}%
\pgfpathmoveto{\pgfqpoint{3.807170in}{3.265333in}}%
\pgfpathlineto{\pgfqpoint{3.818207in}{3.268714in}}%
\pgfpathlineto{\pgfqpoint{3.829237in}{3.272046in}}%
\pgfpathlineto{\pgfqpoint{3.840261in}{3.275311in}}%
\pgfpathlineto{\pgfqpoint{3.851279in}{3.278489in}}%
\pgfpathlineto{\pgfqpoint{3.862289in}{3.281584in}}%
\pgfpathlineto{\pgfqpoint{3.855955in}{3.293336in}}%
\pgfpathlineto{\pgfqpoint{3.849623in}{3.305007in}}%
\pgfpathlineto{\pgfqpoint{3.843292in}{3.316584in}}%
\pgfpathlineto{\pgfqpoint{3.836963in}{3.328054in}}%
\pgfpathlineto{\pgfqpoint{3.830634in}{3.339404in}}%
\pgfpathlineto{\pgfqpoint{3.819632in}{3.336465in}}%
\pgfpathlineto{\pgfqpoint{3.808623in}{3.333465in}}%
\pgfpathlineto{\pgfqpoint{3.797608in}{3.330396in}}%
\pgfpathlineto{\pgfqpoint{3.786586in}{3.327266in}}%
\pgfpathlineto{\pgfqpoint{3.775558in}{3.324084in}}%
\pgfpathlineto{\pgfqpoint{3.781880in}{3.312660in}}%
\pgfpathlineto{\pgfqpoint{3.788202in}{3.301044in}}%
\pgfpathlineto{\pgfqpoint{3.794524in}{3.289265in}}%
\pgfpathlineto{\pgfqpoint{3.800846in}{3.277352in}}%
\pgfpathclose%
\pgfusepath{stroke,fill}%
\end{pgfscope}%
\begin{pgfscope}%
\pgfpathrectangle{\pgfqpoint{0.887500in}{0.275000in}}{\pgfqpoint{4.225000in}{4.225000in}}%
\pgfusepath{clip}%
\pgfsetbuttcap%
\pgfsetroundjoin%
\definecolor{currentfill}{rgb}{0.146616,0.673050,0.508936}%
\pgfsetfillcolor{currentfill}%
\pgfsetfillopacity{0.700000}%
\pgfsetlinewidth{0.501875pt}%
\definecolor{currentstroke}{rgb}{1.000000,1.000000,1.000000}%
\pgfsetstrokecolor{currentstroke}%
\pgfsetstrokeopacity{0.500000}%
\pgfsetdash{}{0pt}%
\pgfpathmoveto{\pgfqpoint{2.812260in}{2.796220in}}%
\pgfpathlineto{\pgfqpoint{2.823429in}{2.815957in}}%
\pgfpathlineto{\pgfqpoint{2.834595in}{2.836862in}}%
\pgfpathlineto{\pgfqpoint{2.845761in}{2.858799in}}%
\pgfpathlineto{\pgfqpoint{2.856926in}{2.881692in}}%
\pgfpathlineto{\pgfqpoint{2.868092in}{2.905654in}}%
\pgfpathlineto{\pgfqpoint{2.862025in}{2.908953in}}%
\pgfpathlineto{\pgfqpoint{2.855962in}{2.912557in}}%
\pgfpathlineto{\pgfqpoint{2.849904in}{2.916399in}}%
\pgfpathlineto{\pgfqpoint{2.843849in}{2.920402in}}%
\pgfpathlineto{\pgfqpoint{2.837800in}{2.924487in}}%
\pgfpathlineto{\pgfqpoint{2.826630in}{2.906969in}}%
\pgfpathlineto{\pgfqpoint{2.815459in}{2.889987in}}%
\pgfpathlineto{\pgfqpoint{2.804291in}{2.872972in}}%
\pgfpathlineto{\pgfqpoint{2.793124in}{2.855973in}}%
\pgfpathlineto{\pgfqpoint{2.781959in}{2.839226in}}%
\pgfpathlineto{\pgfqpoint{2.788018in}{2.829738in}}%
\pgfpathlineto{\pgfqpoint{2.794078in}{2.820580in}}%
\pgfpathlineto{\pgfqpoint{2.800138in}{2.811868in}}%
\pgfpathlineto{\pgfqpoint{2.806199in}{2.803716in}}%
\pgfpathclose%
\pgfusepath{stroke,fill}%
\end{pgfscope}%
\begin{pgfscope}%
\pgfpathrectangle{\pgfqpoint{0.887500in}{0.275000in}}{\pgfqpoint{4.225000in}{4.225000in}}%
\pgfusepath{clip}%
\pgfsetbuttcap%
\pgfsetroundjoin%
\definecolor{currentfill}{rgb}{0.132444,0.552216,0.553018}%
\pgfsetfillcolor{currentfill}%
\pgfsetfillopacity{0.700000}%
\pgfsetlinewidth{0.501875pt}%
\definecolor{currentstroke}{rgb}{1.000000,1.000000,1.000000}%
\pgfsetstrokecolor{currentstroke}%
\pgfsetstrokeopacity{0.500000}%
\pgfsetdash{}{0pt}%
\pgfpathmoveto{\pgfqpoint{2.587757in}{2.565406in}}%
\pgfpathlineto{\pgfqpoint{2.599032in}{2.573272in}}%
\pgfpathlineto{\pgfqpoint{2.610293in}{2.582188in}}%
\pgfpathlineto{\pgfqpoint{2.621538in}{2.592243in}}%
\pgfpathlineto{\pgfqpoint{2.632772in}{2.603276in}}%
\pgfpathlineto{\pgfqpoint{2.643998in}{2.614962in}}%
\pgfpathlineto{\pgfqpoint{2.637983in}{2.623417in}}%
\pgfpathlineto{\pgfqpoint{2.631975in}{2.631653in}}%
\pgfpathlineto{\pgfqpoint{2.625973in}{2.639744in}}%
\pgfpathlineto{\pgfqpoint{2.619975in}{2.647778in}}%
\pgfpathlineto{\pgfqpoint{2.613980in}{2.655846in}}%
\pgfpathlineto{\pgfqpoint{2.602777in}{2.643782in}}%
\pgfpathlineto{\pgfqpoint{2.591564in}{2.632446in}}%
\pgfpathlineto{\pgfqpoint{2.580338in}{2.622154in}}%
\pgfpathlineto{\pgfqpoint{2.569096in}{2.613062in}}%
\pgfpathlineto{\pgfqpoint{2.557837in}{2.605082in}}%
\pgfpathlineto{\pgfqpoint{2.563813in}{2.597191in}}%
\pgfpathlineto{\pgfqpoint{2.569792in}{2.589344in}}%
\pgfpathlineto{\pgfqpoint{2.575775in}{2.581473in}}%
\pgfpathlineto{\pgfqpoint{2.581763in}{2.573510in}}%
\pgfpathclose%
\pgfusepath{stroke,fill}%
\end{pgfscope}%
\begin{pgfscope}%
\pgfpathrectangle{\pgfqpoint{0.887500in}{0.275000in}}{\pgfqpoint{4.225000in}{4.225000in}}%
\pgfusepath{clip}%
\pgfsetbuttcap%
\pgfsetroundjoin%
\definecolor{currentfill}{rgb}{0.127568,0.566949,0.550556}%
\pgfsetfillcolor{currentfill}%
\pgfsetfillopacity{0.700000}%
\pgfsetlinewidth{0.501875pt}%
\definecolor{currentstroke}{rgb}{1.000000,1.000000,1.000000}%
\pgfsetstrokecolor{currentstroke}%
\pgfsetstrokeopacity{0.500000}%
\pgfsetdash{}{0pt}%
\pgfpathmoveto{\pgfqpoint{2.127856in}{2.618643in}}%
\pgfpathlineto{\pgfqpoint{2.139301in}{2.621906in}}%
\pgfpathlineto{\pgfqpoint{2.150740in}{2.625188in}}%
\pgfpathlineto{\pgfqpoint{2.162172in}{2.628502in}}%
\pgfpathlineto{\pgfqpoint{2.173598in}{2.631859in}}%
\pgfpathlineto{\pgfqpoint{2.185017in}{2.635271in}}%
\pgfpathlineto{\pgfqpoint{2.179162in}{2.643366in}}%
\pgfpathlineto{\pgfqpoint{2.173310in}{2.651447in}}%
\pgfpathlineto{\pgfqpoint{2.167463in}{2.659515in}}%
\pgfpathlineto{\pgfqpoint{2.161620in}{2.667571in}}%
\pgfpathlineto{\pgfqpoint{2.155780in}{2.675613in}}%
\pgfpathlineto{\pgfqpoint{2.144375in}{2.672129in}}%
\pgfpathlineto{\pgfqpoint{2.132962in}{2.668729in}}%
\pgfpathlineto{\pgfqpoint{2.121543in}{2.665396in}}%
\pgfpathlineto{\pgfqpoint{2.110116in}{2.662111in}}%
\pgfpathlineto{\pgfqpoint{2.098684in}{2.658857in}}%
\pgfpathlineto{\pgfqpoint{2.104510in}{2.650838in}}%
\pgfpathlineto{\pgfqpoint{2.110341in}{2.642807in}}%
\pgfpathlineto{\pgfqpoint{2.116175in}{2.634764in}}%
\pgfpathlineto{\pgfqpoint{2.122014in}{2.626710in}}%
\pgfpathclose%
\pgfusepath{stroke,fill}%
\end{pgfscope}%
\begin{pgfscope}%
\pgfpathrectangle{\pgfqpoint{0.887500in}{0.275000in}}{\pgfqpoint{4.225000in}{4.225000in}}%
\pgfusepath{clip}%
\pgfsetbuttcap%
\pgfsetroundjoin%
\definecolor{currentfill}{rgb}{0.119512,0.607464,0.540218}%
\pgfsetfillcolor{currentfill}%
\pgfsetfillopacity{0.700000}%
\pgfsetlinewidth{0.501875pt}%
\definecolor{currentstroke}{rgb}{1.000000,1.000000,1.000000}%
\pgfsetstrokecolor{currentstroke}%
\pgfsetstrokeopacity{0.500000}%
\pgfsetdash{}{0pt}%
\pgfpathmoveto{\pgfqpoint{1.581173in}{2.700559in}}%
\pgfpathlineto{\pgfqpoint{1.592750in}{2.703850in}}%
\pgfpathlineto{\pgfqpoint{1.604322in}{2.707139in}}%
\pgfpathlineto{\pgfqpoint{1.615888in}{2.710427in}}%
\pgfpathlineto{\pgfqpoint{1.627448in}{2.713713in}}%
\pgfpathlineto{\pgfqpoint{1.639003in}{2.716997in}}%
\pgfpathlineto{\pgfqpoint{1.633343in}{2.724637in}}%
\pgfpathlineto{\pgfqpoint{1.627688in}{2.732261in}}%
\pgfpathlineto{\pgfqpoint{1.622037in}{2.739871in}}%
\pgfpathlineto{\pgfqpoint{1.616391in}{2.747464in}}%
\pgfpathlineto{\pgfqpoint{1.610748in}{2.755041in}}%
\pgfpathlineto{\pgfqpoint{1.599208in}{2.751742in}}%
\pgfpathlineto{\pgfqpoint{1.587662in}{2.748440in}}%
\pgfpathlineto{\pgfqpoint{1.576110in}{2.745135in}}%
\pgfpathlineto{\pgfqpoint{1.564553in}{2.741829in}}%
\pgfpathlineto{\pgfqpoint{1.552990in}{2.738522in}}%
\pgfpathlineto{\pgfqpoint{1.558618in}{2.730962in}}%
\pgfpathlineto{\pgfqpoint{1.564250in}{2.723385in}}%
\pgfpathlineto{\pgfqpoint{1.569887in}{2.715793in}}%
\pgfpathlineto{\pgfqpoint{1.575528in}{2.708184in}}%
\pgfpathclose%
\pgfusepath{stroke,fill}%
\end{pgfscope}%
\begin{pgfscope}%
\pgfpathrectangle{\pgfqpoint{0.887500in}{0.275000in}}{\pgfqpoint{4.225000in}{4.225000in}}%
\pgfusepath{clip}%
\pgfsetbuttcap%
\pgfsetroundjoin%
\definecolor{currentfill}{rgb}{0.136408,0.541173,0.554483}%
\pgfsetfillcolor{currentfill}%
\pgfsetfillopacity{0.700000}%
\pgfsetlinewidth{0.501875pt}%
\definecolor{currentstroke}{rgb}{1.000000,1.000000,1.000000}%
\pgfsetstrokecolor{currentstroke}%
\pgfsetstrokeopacity{0.500000}%
\pgfsetdash{}{0pt}%
\pgfpathmoveto{\pgfqpoint{2.444517in}{2.562818in}}%
\pgfpathlineto{\pgfqpoint{2.455894in}{2.565587in}}%
\pgfpathlineto{\pgfqpoint{2.467264in}{2.568392in}}%
\pgfpathlineto{\pgfqpoint{2.478626in}{2.571341in}}%
\pgfpathlineto{\pgfqpoint{2.489979in}{2.574542in}}%
\pgfpathlineto{\pgfqpoint{2.501322in}{2.578103in}}%
\pgfpathlineto{\pgfqpoint{2.495362in}{2.586111in}}%
\pgfpathlineto{\pgfqpoint{2.489405in}{2.594221in}}%
\pgfpathlineto{\pgfqpoint{2.483450in}{2.602475in}}%
\pgfpathlineto{\pgfqpoint{2.477495in}{2.610918in}}%
\pgfpathlineto{\pgfqpoint{2.471542in}{2.619593in}}%
\pgfpathlineto{\pgfqpoint{2.460212in}{2.615939in}}%
\pgfpathlineto{\pgfqpoint{2.448872in}{2.612681in}}%
\pgfpathlineto{\pgfqpoint{2.437521in}{2.609704in}}%
\pgfpathlineto{\pgfqpoint{2.426162in}{2.606890in}}%
\pgfpathlineto{\pgfqpoint{2.414797in}{2.604126in}}%
\pgfpathlineto{\pgfqpoint{2.420734in}{2.595833in}}%
\pgfpathlineto{\pgfqpoint{2.426674in}{2.587562in}}%
\pgfpathlineto{\pgfqpoint{2.432618in}{2.579308in}}%
\pgfpathlineto{\pgfqpoint{2.438566in}{2.571061in}}%
\pgfpathclose%
\pgfusepath{stroke,fill}%
\end{pgfscope}%
\begin{pgfscope}%
\pgfpathrectangle{\pgfqpoint{0.887500in}{0.275000in}}{\pgfqpoint{4.225000in}{4.225000in}}%
\pgfusepath{clip}%
\pgfsetbuttcap%
\pgfsetroundjoin%
\definecolor{currentfill}{rgb}{0.119699,0.618490,0.536347}%
\pgfsetfillcolor{currentfill}%
\pgfsetfillopacity{0.700000}%
\pgfsetlinewidth{0.501875pt}%
\definecolor{currentstroke}{rgb}{1.000000,1.000000,1.000000}%
\pgfsetstrokecolor{currentstroke}%
\pgfsetstrokeopacity{0.500000}%
\pgfsetdash{}{0pt}%
\pgfpathmoveto{\pgfqpoint{1.351018in}{2.726011in}}%
\pgfpathlineto{\pgfqpoint{1.362649in}{2.729389in}}%
\pgfpathlineto{\pgfqpoint{1.374275in}{2.732757in}}%
\pgfpathlineto{\pgfqpoint{1.385895in}{2.736117in}}%
\pgfpathlineto{\pgfqpoint{1.397510in}{2.739469in}}%
\pgfpathlineto{\pgfqpoint{1.409120in}{2.742813in}}%
\pgfpathlineto{\pgfqpoint{1.403547in}{2.750232in}}%
\pgfpathlineto{\pgfqpoint{1.397979in}{2.757631in}}%
\pgfpathlineto{\pgfqpoint{1.392415in}{2.765008in}}%
\pgfpathlineto{\pgfqpoint{1.386856in}{2.772364in}}%
\pgfpathlineto{\pgfqpoint{1.381302in}{2.779698in}}%
\pgfpathlineto{\pgfqpoint{1.369708in}{2.776325in}}%
\pgfpathlineto{\pgfqpoint{1.358108in}{2.772945in}}%
\pgfpathlineto{\pgfqpoint{1.346503in}{2.769557in}}%
\pgfpathlineto{\pgfqpoint{1.334893in}{2.766161in}}%
\pgfpathlineto{\pgfqpoint{1.323277in}{2.762757in}}%
\pgfpathlineto{\pgfqpoint{1.328816in}{2.755458in}}%
\pgfpathlineto{\pgfqpoint{1.334359in}{2.748134in}}%
\pgfpathlineto{\pgfqpoint{1.339908in}{2.740784in}}%
\pgfpathlineto{\pgfqpoint{1.345460in}{2.733409in}}%
\pgfpathclose%
\pgfusepath{stroke,fill}%
\end{pgfscope}%
\begin{pgfscope}%
\pgfpathrectangle{\pgfqpoint{0.887500in}{0.275000in}}{\pgfqpoint{4.225000in}{4.225000in}}%
\pgfusepath{clip}%
\pgfsetbuttcap%
\pgfsetroundjoin%
\definecolor{currentfill}{rgb}{0.122606,0.585371,0.546557}%
\pgfsetfillcolor{currentfill}%
\pgfsetfillopacity{0.700000}%
\pgfsetlinewidth{0.501875pt}%
\definecolor{currentstroke}{rgb}{1.000000,1.000000,1.000000}%
\pgfsetstrokecolor{currentstroke}%
\pgfsetstrokeopacity{0.500000}%
\pgfsetdash{}{0pt}%
\pgfpathmoveto{\pgfqpoint{1.897658in}{2.649139in}}%
\pgfpathlineto{\pgfqpoint{1.909161in}{2.652441in}}%
\pgfpathlineto{\pgfqpoint{1.920658in}{2.655753in}}%
\pgfpathlineto{\pgfqpoint{1.932150in}{2.659075in}}%
\pgfpathlineto{\pgfqpoint{1.943635in}{2.662405in}}%
\pgfpathlineto{\pgfqpoint{1.955115in}{2.665740in}}%
\pgfpathlineto{\pgfqpoint{1.949338in}{2.673667in}}%
\pgfpathlineto{\pgfqpoint{1.943566in}{2.681577in}}%
\pgfpathlineto{\pgfqpoint{1.937798in}{2.689467in}}%
\pgfpathlineto{\pgfqpoint{1.932034in}{2.697339in}}%
\pgfpathlineto{\pgfqpoint{1.926275in}{2.705192in}}%
\pgfpathlineto{\pgfqpoint{1.914808in}{2.701852in}}%
\pgfpathlineto{\pgfqpoint{1.903335in}{2.698519in}}%
\pgfpathlineto{\pgfqpoint{1.891857in}{2.695194in}}%
\pgfpathlineto{\pgfqpoint{1.880373in}{2.691880in}}%
\pgfpathlineto{\pgfqpoint{1.868883in}{2.688577in}}%
\pgfpathlineto{\pgfqpoint{1.874629in}{2.680727in}}%
\pgfpathlineto{\pgfqpoint{1.880380in}{2.672859in}}%
\pgfpathlineto{\pgfqpoint{1.886135in}{2.664971in}}%
\pgfpathlineto{\pgfqpoint{1.891895in}{2.657064in}}%
\pgfpathclose%
\pgfusepath{stroke,fill}%
\end{pgfscope}%
\begin{pgfscope}%
\pgfpathrectangle{\pgfqpoint{0.887500in}{0.275000in}}{\pgfqpoint{4.225000in}{4.225000in}}%
\pgfusepath{clip}%
\pgfsetbuttcap%
\pgfsetroundjoin%
\definecolor{currentfill}{rgb}{0.814576,0.883393,0.110347}%
\pgfsetfillcolor{currentfill}%
\pgfsetfillopacity{0.700000}%
\pgfsetlinewidth{0.501875pt}%
\definecolor{currentstroke}{rgb}{1.000000,1.000000,1.000000}%
\pgfsetstrokecolor{currentstroke}%
\pgfsetstrokeopacity{0.500000}%
\pgfsetdash{}{0pt}%
\pgfpathmoveto{\pgfqpoint{3.087091in}{3.464903in}}%
\pgfpathlineto{\pgfqpoint{3.098307in}{3.466863in}}%
\pgfpathlineto{\pgfqpoint{3.109517in}{3.469051in}}%
\pgfpathlineto{\pgfqpoint{3.120722in}{3.471453in}}%
\pgfpathlineto{\pgfqpoint{3.131921in}{3.474054in}}%
\pgfpathlineto{\pgfqpoint{3.143115in}{3.476838in}}%
\pgfpathlineto{\pgfqpoint{3.136935in}{3.482953in}}%
\pgfpathlineto{\pgfqpoint{3.130759in}{3.488718in}}%
\pgfpathlineto{\pgfqpoint{3.124587in}{3.494131in}}%
\pgfpathlineto{\pgfqpoint{3.118420in}{3.499201in}}%
\pgfpathlineto{\pgfqpoint{3.112257in}{3.503936in}}%
\pgfpathlineto{\pgfqpoint{3.101076in}{3.499740in}}%
\pgfpathlineto{\pgfqpoint{3.089890in}{3.495751in}}%
\pgfpathlineto{\pgfqpoint{3.078699in}{3.492111in}}%
\pgfpathlineto{\pgfqpoint{3.067504in}{3.488962in}}%
\pgfpathlineto{\pgfqpoint{3.056303in}{3.486440in}}%
\pgfpathlineto{\pgfqpoint{3.062450in}{3.483087in}}%
\pgfpathlineto{\pgfqpoint{3.068603in}{3.479321in}}%
\pgfpathlineto{\pgfqpoint{3.074761in}{3.475071in}}%
\pgfpathlineto{\pgfqpoint{3.080923in}{3.470268in}}%
\pgfpathclose%
\pgfusepath{stroke,fill}%
\end{pgfscope}%
\begin{pgfscope}%
\pgfpathrectangle{\pgfqpoint{0.887500in}{0.275000in}}{\pgfqpoint{4.225000in}{4.225000in}}%
\pgfusepath{clip}%
\pgfsetbuttcap%
\pgfsetroundjoin%
\definecolor{currentfill}{rgb}{0.783315,0.879285,0.125405}%
\pgfsetfillcolor{currentfill}%
\pgfsetfillopacity{0.700000}%
\pgfsetlinewidth{0.501875pt}%
\definecolor{currentstroke}{rgb}{1.000000,1.000000,1.000000}%
\pgfsetstrokecolor{currentstroke}%
\pgfsetstrokeopacity{0.500000}%
\pgfsetdash{}{0pt}%
\pgfpathmoveto{\pgfqpoint{2.944078in}{3.424104in}}%
\pgfpathlineto{\pgfqpoint{2.955291in}{3.440709in}}%
\pgfpathlineto{\pgfqpoint{2.966516in}{3.453235in}}%
\pgfpathlineto{\pgfqpoint{2.977746in}{3.462510in}}%
\pgfpathlineto{\pgfqpoint{2.988978in}{3.469358in}}%
\pgfpathlineto{\pgfqpoint{3.000210in}{3.474397in}}%
\pgfpathlineto{\pgfqpoint{2.994084in}{3.478007in}}%
\pgfpathlineto{\pgfqpoint{2.987965in}{3.481383in}}%
\pgfpathlineto{\pgfqpoint{2.981850in}{3.484581in}}%
\pgfpathlineto{\pgfqpoint{2.975742in}{3.487657in}}%
\pgfpathlineto{\pgfqpoint{2.969639in}{3.490667in}}%
\pgfpathlineto{\pgfqpoint{2.958424in}{3.486716in}}%
\pgfpathlineto{\pgfqpoint{2.947209in}{3.480872in}}%
\pgfpathlineto{\pgfqpoint{2.935998in}{3.472443in}}%
\pgfpathlineto{\pgfqpoint{2.924795in}{3.460559in}}%
\pgfpathlineto{\pgfqpoint{2.913607in}{3.444346in}}%
\pgfpathlineto{\pgfqpoint{2.919691in}{3.440314in}}%
\pgfpathlineto{\pgfqpoint{2.925780in}{3.436326in}}%
\pgfpathlineto{\pgfqpoint{2.931874in}{3.432330in}}%
\pgfpathlineto{\pgfqpoint{2.937973in}{3.428273in}}%
\pgfpathclose%
\pgfusepath{stroke,fill}%
\end{pgfscope}%
\begin{pgfscope}%
\pgfpathrectangle{\pgfqpoint{0.887500in}{0.275000in}}{\pgfqpoint{4.225000in}{4.225000in}}%
\pgfusepath{clip}%
\pgfsetbuttcap%
\pgfsetroundjoin%
\definecolor{currentfill}{rgb}{0.793760,0.880678,0.120005}%
\pgfsetfillcolor{currentfill}%
\pgfsetfillopacity{0.700000}%
\pgfsetlinewidth{0.501875pt}%
\definecolor{currentstroke}{rgb}{1.000000,1.000000,1.000000}%
\pgfsetstrokecolor{currentstroke}%
\pgfsetstrokeopacity{0.500000}%
\pgfsetdash{}{0pt}%
\pgfpathmoveto{\pgfqpoint{3.317003in}{3.434134in}}%
\pgfpathlineto{\pgfqpoint{3.328167in}{3.438242in}}%
\pgfpathlineto{\pgfqpoint{3.339327in}{3.442351in}}%
\pgfpathlineto{\pgfqpoint{3.350482in}{3.446446in}}%
\pgfpathlineto{\pgfqpoint{3.361632in}{3.450512in}}%
\pgfpathlineto{\pgfqpoint{3.372777in}{3.454534in}}%
\pgfpathlineto{\pgfqpoint{3.366535in}{3.463691in}}%
\pgfpathlineto{\pgfqpoint{3.360293in}{3.472428in}}%
\pgfpathlineto{\pgfqpoint{3.354053in}{3.480752in}}%
\pgfpathlineto{\pgfqpoint{3.347813in}{3.488685in}}%
\pgfpathlineto{\pgfqpoint{3.341576in}{3.496250in}}%
\pgfpathlineto{\pgfqpoint{3.330444in}{3.492509in}}%
\pgfpathlineto{\pgfqpoint{3.319306in}{3.488634in}}%
\pgfpathlineto{\pgfqpoint{3.308162in}{3.484660in}}%
\pgfpathlineto{\pgfqpoint{3.297014in}{3.480622in}}%
\pgfpathlineto{\pgfqpoint{3.285860in}{3.476554in}}%
\pgfpathlineto{\pgfqpoint{3.292084in}{3.468732in}}%
\pgfpathlineto{\pgfqpoint{3.298311in}{3.460592in}}%
\pgfpathlineto{\pgfqpoint{3.304540in}{3.452121in}}%
\pgfpathlineto{\pgfqpoint{3.310770in}{3.443302in}}%
\pgfpathclose%
\pgfusepath{stroke,fill}%
\end{pgfscope}%
\begin{pgfscope}%
\pgfpathrectangle{\pgfqpoint{0.887500in}{0.275000in}}{\pgfqpoint{4.225000in}{4.225000in}}%
\pgfusepath{clip}%
\pgfsetbuttcap%
\pgfsetroundjoin%
\definecolor{currentfill}{rgb}{0.129933,0.559582,0.551864}%
\pgfsetfillcolor{currentfill}%
\pgfsetfillopacity{0.700000}%
\pgfsetlinewidth{0.501875pt}%
\definecolor{currentstroke}{rgb}{1.000000,1.000000,1.000000}%
\pgfsetstrokecolor{currentstroke}%
\pgfsetstrokeopacity{0.500000}%
\pgfsetdash{}{0pt}%
\pgfpathmoveto{\pgfqpoint{2.214356in}{2.594610in}}%
\pgfpathlineto{\pgfqpoint{2.225783in}{2.597978in}}%
\pgfpathlineto{\pgfqpoint{2.237203in}{2.601368in}}%
\pgfpathlineto{\pgfqpoint{2.248618in}{2.604774in}}%
\pgfpathlineto{\pgfqpoint{2.260028in}{2.608189in}}%
\pgfpathlineto{\pgfqpoint{2.271431in}{2.611608in}}%
\pgfpathlineto{\pgfqpoint{2.265542in}{2.619825in}}%
\pgfpathlineto{\pgfqpoint{2.259657in}{2.628035in}}%
\pgfpathlineto{\pgfqpoint{2.253776in}{2.636237in}}%
\pgfpathlineto{\pgfqpoint{2.247898in}{2.644430in}}%
\pgfpathlineto{\pgfqpoint{2.242025in}{2.652613in}}%
\pgfpathlineto{\pgfqpoint{2.230633in}{2.649209in}}%
\pgfpathlineto{\pgfqpoint{2.219237in}{2.645741in}}%
\pgfpathlineto{\pgfqpoint{2.207836in}{2.642241in}}%
\pgfpathlineto{\pgfqpoint{2.196430in}{2.638741in}}%
\pgfpathlineto{\pgfqpoint{2.185017in}{2.635271in}}%
\pgfpathlineto{\pgfqpoint{2.190877in}{2.627164in}}%
\pgfpathlineto{\pgfqpoint{2.196741in}{2.619045in}}%
\pgfpathlineto{\pgfqpoint{2.202608in}{2.610912in}}%
\pgfpathlineto{\pgfqpoint{2.208480in}{2.602768in}}%
\pgfpathclose%
\pgfusepath{stroke,fill}%
\end{pgfscope}%
\begin{pgfscope}%
\pgfpathrectangle{\pgfqpoint{0.887500in}{0.275000in}}{\pgfqpoint{4.225000in}{4.225000in}}%
\pgfusepath{clip}%
\pgfsetbuttcap%
\pgfsetroundjoin%
\definecolor{currentfill}{rgb}{0.120092,0.600104,0.542530}%
\pgfsetfillcolor{currentfill}%
\pgfsetfillopacity{0.700000}%
\pgfsetlinewidth{0.501875pt}%
\definecolor{currentstroke}{rgb}{1.000000,1.000000,1.000000}%
\pgfsetstrokecolor{currentstroke}%
\pgfsetstrokeopacity{0.500000}%
\pgfsetdash{}{0pt}%
\pgfpathmoveto{\pgfqpoint{1.667363in}{2.678574in}}%
\pgfpathlineto{\pgfqpoint{1.678926in}{2.681842in}}%
\pgfpathlineto{\pgfqpoint{1.690483in}{2.685108in}}%
\pgfpathlineto{\pgfqpoint{1.702035in}{2.688371in}}%
\pgfpathlineto{\pgfqpoint{1.713581in}{2.691632in}}%
\pgfpathlineto{\pgfqpoint{1.725122in}{2.694892in}}%
\pgfpathlineto{\pgfqpoint{1.719427in}{2.702619in}}%
\pgfpathlineto{\pgfqpoint{1.713737in}{2.710331in}}%
\pgfpathlineto{\pgfqpoint{1.708052in}{2.718027in}}%
\pgfpathlineto{\pgfqpoint{1.702370in}{2.725708in}}%
\pgfpathlineto{\pgfqpoint{1.696693in}{2.733374in}}%
\pgfpathlineto{\pgfqpoint{1.685166in}{2.730103in}}%
\pgfpathlineto{\pgfqpoint{1.673633in}{2.726830in}}%
\pgfpathlineto{\pgfqpoint{1.662095in}{2.723555in}}%
\pgfpathlineto{\pgfqpoint{1.650552in}{2.720277in}}%
\pgfpathlineto{\pgfqpoint{1.639003in}{2.716997in}}%
\pgfpathlineto{\pgfqpoint{1.644666in}{2.709342in}}%
\pgfpathlineto{\pgfqpoint{1.650334in}{2.701672in}}%
\pgfpathlineto{\pgfqpoint{1.656006in}{2.693988in}}%
\pgfpathlineto{\pgfqpoint{1.661682in}{2.686289in}}%
\pgfpathclose%
\pgfusepath{stroke,fill}%
\end{pgfscope}%
\begin{pgfscope}%
\pgfpathrectangle{\pgfqpoint{0.887500in}{0.275000in}}{\pgfqpoint{4.225000in}{4.225000in}}%
\pgfusepath{clip}%
\pgfsetbuttcap%
\pgfsetroundjoin%
\definecolor{currentfill}{rgb}{0.137770,0.537492,0.554906}%
\pgfsetfillcolor{currentfill}%
\pgfsetfillopacity{0.700000}%
\pgfsetlinewidth{0.501875pt}%
\definecolor{currentstroke}{rgb}{1.000000,1.000000,1.000000}%
\pgfsetstrokecolor{currentstroke}%
\pgfsetstrokeopacity{0.500000}%
\pgfsetdash{}{0pt}%
\pgfpathmoveto{\pgfqpoint{2.531178in}{2.538066in}}%
\pgfpathlineto{\pgfqpoint{2.542519in}{2.542294in}}%
\pgfpathlineto{\pgfqpoint{2.553847in}{2.547034in}}%
\pgfpathlineto{\pgfqpoint{2.565164in}{2.552395in}}%
\pgfpathlineto{\pgfqpoint{2.576467in}{2.558483in}}%
\pgfpathlineto{\pgfqpoint{2.587757in}{2.565406in}}%
\pgfpathlineto{\pgfqpoint{2.581763in}{2.573510in}}%
\pgfpathlineto{\pgfqpoint{2.575775in}{2.581473in}}%
\pgfpathlineto{\pgfqpoint{2.569792in}{2.589344in}}%
\pgfpathlineto{\pgfqpoint{2.563813in}{2.597191in}}%
\pgfpathlineto{\pgfqpoint{2.557837in}{2.605082in}}%
\pgfpathlineto{\pgfqpoint{2.546563in}{2.598106in}}%
\pgfpathlineto{\pgfqpoint{2.535274in}{2.592027in}}%
\pgfpathlineto{\pgfqpoint{2.523970in}{2.586739in}}%
\pgfpathlineto{\pgfqpoint{2.512652in}{2.582133in}}%
\pgfpathlineto{\pgfqpoint{2.501322in}{2.578103in}}%
\pgfpathlineto{\pgfqpoint{2.507284in}{2.570153in}}%
\pgfpathlineto{\pgfqpoint{2.513251in}{2.562216in}}%
\pgfpathlineto{\pgfqpoint{2.519221in}{2.554250in}}%
\pgfpathlineto{\pgfqpoint{2.525197in}{2.546210in}}%
\pgfpathclose%
\pgfusepath{stroke,fill}%
\end{pgfscope}%
\begin{pgfscope}%
\pgfpathrectangle{\pgfqpoint{0.887500in}{0.275000in}}{\pgfqpoint{4.225000in}{4.225000in}}%
\pgfusepath{clip}%
\pgfsetbuttcap%
\pgfsetroundjoin%
\definecolor{currentfill}{rgb}{0.119483,0.614817,0.537692}%
\pgfsetfillcolor{currentfill}%
\pgfsetfillopacity{0.700000}%
\pgfsetlinewidth{0.501875pt}%
\definecolor{currentstroke}{rgb}{1.000000,1.000000,1.000000}%
\pgfsetstrokecolor{currentstroke}%
\pgfsetstrokeopacity{0.500000}%
\pgfsetdash{}{0pt}%
\pgfpathmoveto{\pgfqpoint{1.437051in}{2.705420in}}%
\pgfpathlineto{\pgfqpoint{1.448670in}{2.708744in}}%
\pgfpathlineto{\pgfqpoint{1.460283in}{2.712063in}}%
\pgfpathlineto{\pgfqpoint{1.471891in}{2.715377in}}%
\pgfpathlineto{\pgfqpoint{1.483493in}{2.718688in}}%
\pgfpathlineto{\pgfqpoint{1.495090in}{2.721995in}}%
\pgfpathlineto{\pgfqpoint{1.489481in}{2.729523in}}%
\pgfpathlineto{\pgfqpoint{1.483876in}{2.737034in}}%
\pgfpathlineto{\pgfqpoint{1.478275in}{2.744528in}}%
\pgfpathlineto{\pgfqpoint{1.472679in}{2.752004in}}%
\pgfpathlineto{\pgfqpoint{1.467087in}{2.759462in}}%
\pgfpathlineto{\pgfqpoint{1.455504in}{2.756139in}}%
\pgfpathlineto{\pgfqpoint{1.443917in}{2.752814in}}%
\pgfpathlineto{\pgfqpoint{1.432323in}{2.749485in}}%
\pgfpathlineto{\pgfqpoint{1.420724in}{2.746152in}}%
\pgfpathlineto{\pgfqpoint{1.409120in}{2.742813in}}%
\pgfpathlineto{\pgfqpoint{1.414697in}{2.735374in}}%
\pgfpathlineto{\pgfqpoint{1.420279in}{2.727914in}}%
\pgfpathlineto{\pgfqpoint{1.425865in}{2.720435in}}%
\pgfpathlineto{\pgfqpoint{1.431456in}{2.712937in}}%
\pgfpathclose%
\pgfusepath{stroke,fill}%
\end{pgfscope}%
\begin{pgfscope}%
\pgfpathrectangle{\pgfqpoint{0.887500in}{0.275000in}}{\pgfqpoint{4.225000in}{4.225000in}}%
\pgfusepath{clip}%
\pgfsetbuttcap%
\pgfsetroundjoin%
\definecolor{currentfill}{rgb}{0.762373,0.876424,0.137064}%
\pgfsetfillcolor{currentfill}%
\pgfsetfillopacity{0.700000}%
\pgfsetlinewidth{0.501875pt}%
\definecolor{currentstroke}{rgb}{1.000000,1.000000,1.000000}%
\pgfsetstrokecolor{currentstroke}%
\pgfsetstrokeopacity{0.500000}%
\pgfsetdash{}{0pt}%
\pgfpathmoveto{\pgfqpoint{3.404001in}{3.403811in}}%
\pgfpathlineto{\pgfqpoint{3.415148in}{3.407667in}}%
\pgfpathlineto{\pgfqpoint{3.426290in}{3.411493in}}%
\pgfpathlineto{\pgfqpoint{3.437426in}{3.415285in}}%
\pgfpathlineto{\pgfqpoint{3.448556in}{3.419039in}}%
\pgfpathlineto{\pgfqpoint{3.459681in}{3.422752in}}%
\pgfpathlineto{\pgfqpoint{3.453428in}{3.433619in}}%
\pgfpathlineto{\pgfqpoint{3.447176in}{3.444212in}}%
\pgfpathlineto{\pgfqpoint{3.440924in}{3.454471in}}%
\pgfpathlineto{\pgfqpoint{3.434672in}{3.464331in}}%
\pgfpathlineto{\pgfqpoint{3.428419in}{3.473729in}}%
\pgfpathlineto{\pgfqpoint{3.417301in}{3.469998in}}%
\pgfpathlineto{\pgfqpoint{3.406178in}{3.466222in}}%
\pgfpathlineto{\pgfqpoint{3.395050in}{3.462391in}}%
\pgfpathlineto{\pgfqpoint{3.383916in}{3.458497in}}%
\pgfpathlineto{\pgfqpoint{3.372777in}{3.454534in}}%
\pgfpathlineto{\pgfqpoint{3.379019in}{3.444992in}}%
\pgfpathlineto{\pgfqpoint{3.385263in}{3.435107in}}%
\pgfpathlineto{\pgfqpoint{3.391508in}{3.424921in}}%
\pgfpathlineto{\pgfqpoint{3.397754in}{3.414475in}}%
\pgfpathclose%
\pgfusepath{stroke,fill}%
\end{pgfscope}%
\begin{pgfscope}%
\pgfpathrectangle{\pgfqpoint{0.887500in}{0.275000in}}{\pgfqpoint{4.225000in}{4.225000in}}%
\pgfusepath{clip}%
\pgfsetbuttcap%
\pgfsetroundjoin%
\definecolor{currentfill}{rgb}{0.124395,0.578002,0.548287}%
\pgfsetfillcolor{currentfill}%
\pgfsetfillopacity{0.700000}%
\pgfsetlinewidth{0.501875pt}%
\definecolor{currentstroke}{rgb}{1.000000,1.000000,1.000000}%
\pgfsetstrokecolor{currentstroke}%
\pgfsetstrokeopacity{0.500000}%
\pgfsetdash{}{0pt}%
\pgfpathmoveto{\pgfqpoint{1.984060in}{2.625861in}}%
\pgfpathlineto{\pgfqpoint{1.995546in}{2.629196in}}%
\pgfpathlineto{\pgfqpoint{2.007028in}{2.632530in}}%
\pgfpathlineto{\pgfqpoint{2.018503in}{2.635863in}}%
\pgfpathlineto{\pgfqpoint{2.029973in}{2.639189in}}%
\pgfpathlineto{\pgfqpoint{2.041438in}{2.642506in}}%
\pgfpathlineto{\pgfqpoint{2.035628in}{2.650517in}}%
\pgfpathlineto{\pgfqpoint{2.029822in}{2.658514in}}%
\pgfpathlineto{\pgfqpoint{2.024020in}{2.666497in}}%
\pgfpathlineto{\pgfqpoint{2.018222in}{2.674465in}}%
\pgfpathlineto{\pgfqpoint{2.012429in}{2.682418in}}%
\pgfpathlineto{\pgfqpoint{2.000977in}{2.679094in}}%
\pgfpathlineto{\pgfqpoint{1.989520in}{2.675760in}}%
\pgfpathlineto{\pgfqpoint{1.978057in}{2.672420in}}%
\pgfpathlineto{\pgfqpoint{1.966588in}{2.669079in}}%
\pgfpathlineto{\pgfqpoint{1.955115in}{2.665740in}}%
\pgfpathlineto{\pgfqpoint{1.960895in}{2.657795in}}%
\pgfpathlineto{\pgfqpoint{1.966680in}{2.649834in}}%
\pgfpathlineto{\pgfqpoint{1.972469in}{2.641857in}}%
\pgfpathlineto{\pgfqpoint{1.978262in}{2.633866in}}%
\pgfpathclose%
\pgfusepath{stroke,fill}%
\end{pgfscope}%
\begin{pgfscope}%
\pgfpathrectangle{\pgfqpoint{0.887500in}{0.275000in}}{\pgfqpoint{4.225000in}{4.225000in}}%
\pgfusepath{clip}%
\pgfsetbuttcap%
\pgfsetroundjoin%
\definecolor{currentfill}{rgb}{0.344074,0.780029,0.397381}%
\pgfsetfillcolor{currentfill}%
\pgfsetfillopacity{0.700000}%
\pgfsetlinewidth{0.501875pt}%
\definecolor{currentstroke}{rgb}{1.000000,1.000000,1.000000}%
\pgfsetstrokecolor{currentstroke}%
\pgfsetstrokeopacity{0.500000}%
\pgfsetdash{}{0pt}%
\pgfpathmoveto{\pgfqpoint{4.099355in}{3.067254in}}%
\pgfpathlineto{\pgfqpoint{4.110328in}{3.070551in}}%
\pgfpathlineto{\pgfqpoint{4.121296in}{3.073865in}}%
\pgfpathlineto{\pgfqpoint{4.132260in}{3.077195in}}%
\pgfpathlineto{\pgfqpoint{4.143219in}{3.080541in}}%
\pgfpathlineto{\pgfqpoint{4.154173in}{3.083903in}}%
\pgfpathlineto{\pgfqpoint{4.147785in}{3.096453in}}%
\pgfpathlineto{\pgfqpoint{4.141398in}{3.108924in}}%
\pgfpathlineto{\pgfqpoint{4.135013in}{3.121327in}}%
\pgfpathlineto{\pgfqpoint{4.128628in}{3.133670in}}%
\pgfpathlineto{\pgfqpoint{4.122245in}{3.145964in}}%
\pgfpathlineto{\pgfqpoint{4.111301in}{3.142819in}}%
\pgfpathlineto{\pgfqpoint{4.100351in}{3.139684in}}%
\pgfpathlineto{\pgfqpoint{4.089397in}{3.136562in}}%
\pgfpathlineto{\pgfqpoint{4.078438in}{3.133452in}}%
\pgfpathlineto{\pgfqpoint{4.067473in}{3.130356in}}%
\pgfpathlineto{\pgfqpoint{4.073845in}{3.117805in}}%
\pgfpathlineto{\pgfqpoint{4.080220in}{3.105230in}}%
\pgfpathlineto{\pgfqpoint{4.086597in}{3.092621in}}%
\pgfpathlineto{\pgfqpoint{4.092975in}{3.079966in}}%
\pgfpathclose%
\pgfusepath{stroke,fill}%
\end{pgfscope}%
\begin{pgfscope}%
\pgfpathrectangle{\pgfqpoint{0.887500in}{0.275000in}}{\pgfqpoint{4.225000in}{4.225000in}}%
\pgfusepath{clip}%
\pgfsetbuttcap%
\pgfsetroundjoin%
\definecolor{currentfill}{rgb}{0.296479,0.761561,0.424223}%
\pgfsetfillcolor{currentfill}%
\pgfsetfillopacity{0.700000}%
\pgfsetlinewidth{0.501875pt}%
\definecolor{currentstroke}{rgb}{1.000000,1.000000,1.000000}%
\pgfsetstrokecolor{currentstroke}%
\pgfsetstrokeopacity{0.500000}%
\pgfsetdash{}{0pt}%
\pgfpathmoveto{\pgfqpoint{4.186112in}{3.019608in}}%
\pgfpathlineto{\pgfqpoint{4.197066in}{3.023036in}}%
\pgfpathlineto{\pgfqpoint{4.208015in}{3.026462in}}%
\pgfpathlineto{\pgfqpoint{4.218959in}{3.029887in}}%
\pgfpathlineto{\pgfqpoint{4.229898in}{3.033310in}}%
\pgfpathlineto{\pgfqpoint{4.223509in}{3.046407in}}%
\pgfpathlineto{\pgfqpoint{4.217118in}{3.059371in}}%
\pgfpathlineto{\pgfqpoint{4.210727in}{3.072206in}}%
\pgfpathlineto{\pgfqpoint{4.204335in}{3.084913in}}%
\pgfpathlineto{\pgfqpoint{4.197942in}{3.097497in}}%
\pgfpathlineto{\pgfqpoint{4.187007in}{3.094077in}}%
\pgfpathlineto{\pgfqpoint{4.176067in}{3.090671in}}%
\pgfpathlineto{\pgfqpoint{4.165122in}{3.087279in}}%
\pgfpathlineto{\pgfqpoint{4.154173in}{3.083903in}}%
\pgfpathlineto{\pgfqpoint{4.160561in}{3.071263in}}%
\pgfpathlineto{\pgfqpoint{4.166950in}{3.058524in}}%
\pgfpathlineto{\pgfqpoint{4.173338in}{3.045675in}}%
\pgfpathlineto{\pgfqpoint{4.179725in}{3.032706in}}%
\pgfpathclose%
\pgfusepath{stroke,fill}%
\end{pgfscope}%
\begin{pgfscope}%
\pgfpathrectangle{\pgfqpoint{0.887500in}{0.275000in}}{\pgfqpoint{4.225000in}{4.225000in}}%
\pgfusepath{clip}%
\pgfsetbuttcap%
\pgfsetroundjoin%
\definecolor{currentfill}{rgb}{0.377779,0.791781,0.377939}%
\pgfsetfillcolor{currentfill}%
\pgfsetfillopacity{0.700000}%
\pgfsetlinewidth{0.501875pt}%
\definecolor{currentstroke}{rgb}{1.000000,1.000000,1.000000}%
\pgfsetstrokecolor{currentstroke}%
\pgfsetstrokeopacity{0.500000}%
\pgfsetdash{}{0pt}%
\pgfpathmoveto{\pgfqpoint{2.893557in}{3.044503in}}%
\pgfpathlineto{\pgfqpoint{2.904686in}{3.079303in}}%
\pgfpathlineto{\pgfqpoint{2.915818in}{3.117032in}}%
\pgfpathlineto{\pgfqpoint{2.926958in}{3.156301in}}%
\pgfpathlineto{\pgfqpoint{2.938112in}{3.195712in}}%
\pgfpathlineto{\pgfqpoint{2.949284in}{3.233858in}}%
\pgfpathlineto{\pgfqpoint{2.943167in}{3.236925in}}%
\pgfpathlineto{\pgfqpoint{2.937057in}{3.239726in}}%
\pgfpathlineto{\pgfqpoint{2.930953in}{3.242260in}}%
\pgfpathlineto{\pgfqpoint{2.924855in}{3.244530in}}%
\pgfpathlineto{\pgfqpoint{2.918764in}{3.246537in}}%
\pgfpathlineto{\pgfqpoint{2.907659in}{3.202219in}}%
\pgfpathlineto{\pgfqpoint{2.896576in}{3.157010in}}%
\pgfpathlineto{\pgfqpoint{2.885506in}{3.112942in}}%
\pgfpathlineto{\pgfqpoint{2.874439in}{3.072028in}}%
\pgfpathlineto{\pgfqpoint{2.863363in}{3.036269in}}%
\pgfpathlineto{\pgfqpoint{2.869386in}{3.038420in}}%
\pgfpathlineto{\pgfqpoint{2.875416in}{3.040314in}}%
\pgfpathlineto{\pgfqpoint{2.881455in}{3.041955in}}%
\pgfpathlineto{\pgfqpoint{2.887502in}{3.043350in}}%
\pgfpathclose%
\pgfusepath{stroke,fill}%
\end{pgfscope}%
\begin{pgfscope}%
\pgfpathrectangle{\pgfqpoint{0.887500in}{0.275000in}}{\pgfqpoint{4.225000in}{4.225000in}}%
\pgfusepath{clip}%
\pgfsetbuttcap%
\pgfsetroundjoin%
\definecolor{currentfill}{rgb}{0.720391,0.870350,0.162603}%
\pgfsetfillcolor{currentfill}%
\pgfsetfillopacity{0.700000}%
\pgfsetlinewidth{0.501875pt}%
\definecolor{currentstroke}{rgb}{1.000000,1.000000,1.000000}%
\pgfsetstrokecolor{currentstroke}%
\pgfsetstrokeopacity{0.500000}%
\pgfsetdash{}{0pt}%
\pgfpathmoveto{\pgfqpoint{3.490975in}{3.366530in}}%
\pgfpathlineto{\pgfqpoint{3.502101in}{3.370111in}}%
\pgfpathlineto{\pgfqpoint{3.513221in}{3.373666in}}%
\pgfpathlineto{\pgfqpoint{3.524335in}{3.377183in}}%
\pgfpathlineto{\pgfqpoint{3.535443in}{3.380653in}}%
\pgfpathlineto{\pgfqpoint{3.546546in}{3.384082in}}%
\pgfpathlineto{\pgfqpoint{3.540275in}{3.395317in}}%
\pgfpathlineto{\pgfqpoint{3.534007in}{3.406661in}}%
\pgfpathlineto{\pgfqpoint{3.527744in}{3.418040in}}%
\pgfpathlineto{\pgfqpoint{3.521483in}{3.429376in}}%
\pgfpathlineto{\pgfqpoint{3.515224in}{3.440588in}}%
\pgfpathlineto{\pgfqpoint{3.504127in}{3.437121in}}%
\pgfpathlineto{\pgfqpoint{3.493024in}{3.433607in}}%
\pgfpathlineto{\pgfqpoint{3.481915in}{3.430040in}}%
\pgfpathlineto{\pgfqpoint{3.470801in}{3.426420in}}%
\pgfpathlineto{\pgfqpoint{3.459681in}{3.422752in}}%
\pgfpathlineto{\pgfqpoint{3.465936in}{3.411676in}}%
\pgfpathlineto{\pgfqpoint{3.472192in}{3.400453in}}%
\pgfpathlineto{\pgfqpoint{3.478450in}{3.389146in}}%
\pgfpathlineto{\pgfqpoint{3.484711in}{3.377819in}}%
\pgfpathclose%
\pgfusepath{stroke,fill}%
\end{pgfscope}%
\begin{pgfscope}%
\pgfpathrectangle{\pgfqpoint{0.887500in}{0.275000in}}{\pgfqpoint{4.225000in}{4.225000in}}%
\pgfusepath{clip}%
\pgfsetbuttcap%
\pgfsetroundjoin%
\definecolor{currentfill}{rgb}{0.132444,0.552216,0.553018}%
\pgfsetfillcolor{currentfill}%
\pgfsetfillopacity{0.700000}%
\pgfsetlinewidth{0.501875pt}%
\definecolor{currentstroke}{rgb}{1.000000,1.000000,1.000000}%
\pgfsetstrokecolor{currentstroke}%
\pgfsetstrokeopacity{0.500000}%
\pgfsetdash{}{0pt}%
\pgfpathmoveto{\pgfqpoint{2.300935in}{2.570454in}}%
\pgfpathlineto{\pgfqpoint{2.312344in}{2.573934in}}%
\pgfpathlineto{\pgfqpoint{2.323746in}{2.577471in}}%
\pgfpathlineto{\pgfqpoint{2.335142in}{2.581056in}}%
\pgfpathlineto{\pgfqpoint{2.346533in}{2.584651in}}%
\pgfpathlineto{\pgfqpoint{2.357919in}{2.588214in}}%
\pgfpathlineto{\pgfqpoint{2.352000in}{2.596374in}}%
\pgfpathlineto{\pgfqpoint{2.346085in}{2.604492in}}%
\pgfpathlineto{\pgfqpoint{2.340175in}{2.612566in}}%
\pgfpathlineto{\pgfqpoint{2.334270in}{2.620603in}}%
\pgfpathlineto{\pgfqpoint{2.328369in}{2.628608in}}%
\pgfpathlineto{\pgfqpoint{2.316992in}{2.625210in}}%
\pgfpathlineto{\pgfqpoint{2.305610in}{2.621820in}}%
\pgfpathlineto{\pgfqpoint{2.294222in}{2.618427in}}%
\pgfpathlineto{\pgfqpoint{2.282829in}{2.615022in}}%
\pgfpathlineto{\pgfqpoint{2.271431in}{2.611608in}}%
\pgfpathlineto{\pgfqpoint{2.277324in}{2.603385in}}%
\pgfpathlineto{\pgfqpoint{2.283221in}{2.595157in}}%
\pgfpathlineto{\pgfqpoint{2.289122in}{2.586926in}}%
\pgfpathlineto{\pgfqpoint{2.295026in}{2.578692in}}%
\pgfpathclose%
\pgfusepath{stroke,fill}%
\end{pgfscope}%
\begin{pgfscope}%
\pgfpathrectangle{\pgfqpoint{0.887500in}{0.275000in}}{\pgfqpoint{4.225000in}{4.225000in}}%
\pgfusepath{clip}%
\pgfsetbuttcap%
\pgfsetroundjoin%
\definecolor{currentfill}{rgb}{0.395174,0.797475,0.367757}%
\pgfsetfillcolor{currentfill}%
\pgfsetfillopacity{0.700000}%
\pgfsetlinewidth{0.501875pt}%
\definecolor{currentstroke}{rgb}{1.000000,1.000000,1.000000}%
\pgfsetstrokecolor{currentstroke}%
\pgfsetstrokeopacity{0.500000}%
\pgfsetdash{}{0pt}%
\pgfpathmoveto{\pgfqpoint{4.012576in}{3.115107in}}%
\pgfpathlineto{\pgfqpoint{4.023566in}{3.118123in}}%
\pgfpathlineto{\pgfqpoint{4.034550in}{3.121157in}}%
\pgfpathlineto{\pgfqpoint{4.045529in}{3.124208in}}%
\pgfpathlineto{\pgfqpoint{4.056504in}{3.127275in}}%
\pgfpathlineto{\pgfqpoint{4.067473in}{3.130356in}}%
\pgfpathlineto{\pgfqpoint{4.061103in}{3.142883in}}%
\pgfpathlineto{\pgfqpoint{4.054735in}{3.155376in}}%
\pgfpathlineto{\pgfqpoint{4.048369in}{3.167828in}}%
\pgfpathlineto{\pgfqpoint{4.042005in}{3.180231in}}%
\pgfpathlineto{\pgfqpoint{4.035642in}{3.192577in}}%
\pgfpathlineto{\pgfqpoint{4.024680in}{3.189632in}}%
\pgfpathlineto{\pgfqpoint{4.013712in}{3.186660in}}%
\pgfpathlineto{\pgfqpoint{4.002737in}{3.183663in}}%
\pgfpathlineto{\pgfqpoint{3.991757in}{3.180640in}}%
\pgfpathlineto{\pgfqpoint{3.980771in}{3.177592in}}%
\pgfpathlineto{\pgfqpoint{3.987129in}{3.165238in}}%
\pgfpathlineto{\pgfqpoint{3.993489in}{3.152809in}}%
\pgfpathlineto{\pgfqpoint{3.999850in}{3.140308in}}%
\pgfpathlineto{\pgfqpoint{4.006213in}{3.127739in}}%
\pgfpathclose%
\pgfusepath{stroke,fill}%
\end{pgfscope}%
\begin{pgfscope}%
\pgfpathrectangle{\pgfqpoint{0.887500in}{0.275000in}}{\pgfqpoint{4.225000in}{4.225000in}}%
\pgfusepath{clip}%
\pgfsetbuttcap%
\pgfsetroundjoin%
\definecolor{currentfill}{rgb}{0.804182,0.882046,0.114965}%
\pgfsetfillcolor{currentfill}%
\pgfsetfillopacity{0.700000}%
\pgfsetlinewidth{0.501875pt}%
\definecolor{currentstroke}{rgb}{1.000000,1.000000,1.000000}%
\pgfsetstrokecolor{currentstroke}%
\pgfsetstrokeopacity{0.500000}%
\pgfsetdash{}{0pt}%
\pgfpathmoveto{\pgfqpoint{3.174069in}{3.441261in}}%
\pgfpathlineto{\pgfqpoint{3.185269in}{3.443942in}}%
\pgfpathlineto{\pgfqpoint{3.196464in}{3.446849in}}%
\pgfpathlineto{\pgfqpoint{3.207654in}{3.449996in}}%
\pgfpathlineto{\pgfqpoint{3.218840in}{3.453367in}}%
\pgfpathlineto{\pgfqpoint{3.230021in}{3.456930in}}%
\pgfpathlineto{\pgfqpoint{3.223812in}{3.464612in}}%
\pgfpathlineto{\pgfqpoint{3.217606in}{3.472043in}}%
\pgfpathlineto{\pgfqpoint{3.211403in}{3.479235in}}%
\pgfpathlineto{\pgfqpoint{3.205204in}{3.486203in}}%
\pgfpathlineto{\pgfqpoint{3.199008in}{3.492958in}}%
\pgfpathlineto{\pgfqpoint{3.187839in}{3.489490in}}%
\pgfpathlineto{\pgfqpoint{3.176666in}{3.486130in}}%
\pgfpathlineto{\pgfqpoint{3.165487in}{3.482891in}}%
\pgfpathlineto{\pgfqpoint{3.154303in}{3.479789in}}%
\pgfpathlineto{\pgfqpoint{3.143115in}{3.476838in}}%
\pgfpathlineto{\pgfqpoint{3.149298in}{3.470379in}}%
\pgfpathlineto{\pgfqpoint{3.155486in}{3.463585in}}%
\pgfpathlineto{\pgfqpoint{3.161677in}{3.456462in}}%
\pgfpathlineto{\pgfqpoint{3.167871in}{3.449018in}}%
\pgfpathclose%
\pgfusepath{stroke,fill}%
\end{pgfscope}%
\begin{pgfscope}%
\pgfpathrectangle{\pgfqpoint{0.887500in}{0.275000in}}{\pgfqpoint{4.225000in}{4.225000in}}%
\pgfusepath{clip}%
\pgfsetbuttcap%
\pgfsetroundjoin%
\definecolor{currentfill}{rgb}{0.668054,0.861999,0.196293}%
\pgfsetfillcolor{currentfill}%
\pgfsetfillopacity{0.700000}%
\pgfsetlinewidth{0.501875pt}%
\definecolor{currentstroke}{rgb}{1.000000,1.000000,1.000000}%
\pgfsetstrokecolor{currentstroke}%
\pgfsetstrokeopacity{0.500000}%
\pgfsetdash{}{0pt}%
\pgfpathmoveto{\pgfqpoint{3.577968in}{3.329145in}}%
\pgfpathlineto{\pgfqpoint{3.589073in}{3.332658in}}%
\pgfpathlineto{\pgfqpoint{3.600173in}{3.336121in}}%
\pgfpathlineto{\pgfqpoint{3.611266in}{3.339545in}}%
\pgfpathlineto{\pgfqpoint{3.622354in}{3.342939in}}%
\pgfpathlineto{\pgfqpoint{3.633437in}{3.346312in}}%
\pgfpathlineto{\pgfqpoint{3.627137in}{3.357133in}}%
\pgfpathlineto{\pgfqpoint{3.620841in}{3.367957in}}%
\pgfpathlineto{\pgfqpoint{3.614548in}{3.378821in}}%
\pgfpathlineto{\pgfqpoint{3.608260in}{3.389760in}}%
\pgfpathlineto{\pgfqpoint{3.601977in}{3.400809in}}%
\pgfpathlineto{\pgfqpoint{3.590902in}{3.397499in}}%
\pgfpathlineto{\pgfqpoint{3.579821in}{3.394177in}}%
\pgfpathlineto{\pgfqpoint{3.568735in}{3.390838in}}%
\pgfpathlineto{\pgfqpoint{3.557643in}{3.387475in}}%
\pgfpathlineto{\pgfqpoint{3.546546in}{3.384082in}}%
\pgfpathlineto{\pgfqpoint{3.552822in}{3.372967in}}%
\pgfpathlineto{\pgfqpoint{3.559103in}{3.361944in}}%
\pgfpathlineto{\pgfqpoint{3.565388in}{3.350985in}}%
\pgfpathlineto{\pgfqpoint{3.571676in}{3.340062in}}%
\pgfpathclose%
\pgfusepath{stroke,fill}%
\end{pgfscope}%
\begin{pgfscope}%
\pgfpathrectangle{\pgfqpoint{0.887500in}{0.275000in}}{\pgfqpoint{4.225000in}{4.225000in}}%
\pgfusepath{clip}%
\pgfsetbuttcap%
\pgfsetroundjoin%
\definecolor{currentfill}{rgb}{0.121148,0.592739,0.544641}%
\pgfsetfillcolor{currentfill}%
\pgfsetfillopacity{0.700000}%
\pgfsetlinewidth{0.501875pt}%
\definecolor{currentstroke}{rgb}{1.000000,1.000000,1.000000}%
\pgfsetstrokecolor{currentstroke}%
\pgfsetstrokeopacity{0.500000}%
\pgfsetdash{}{0pt}%
\pgfpathmoveto{\pgfqpoint{1.753657in}{2.655994in}}%
\pgfpathlineto{\pgfqpoint{1.765206in}{2.659240in}}%
\pgfpathlineto{\pgfqpoint{1.776748in}{2.662485in}}%
\pgfpathlineto{\pgfqpoint{1.788286in}{2.665730in}}%
\pgfpathlineto{\pgfqpoint{1.799817in}{2.668976in}}%
\pgfpathlineto{\pgfqpoint{1.811343in}{2.672224in}}%
\pgfpathlineto{\pgfqpoint{1.805614in}{2.680051in}}%
\pgfpathlineto{\pgfqpoint{1.799889in}{2.687859in}}%
\pgfpathlineto{\pgfqpoint{1.794168in}{2.695649in}}%
\pgfpathlineto{\pgfqpoint{1.788452in}{2.703422in}}%
\pgfpathlineto{\pgfqpoint{1.782740in}{2.711178in}}%
\pgfpathlineto{\pgfqpoint{1.771228in}{2.707919in}}%
\pgfpathlineto{\pgfqpoint{1.759710in}{2.704663in}}%
\pgfpathlineto{\pgfqpoint{1.748186in}{2.701407in}}%
\pgfpathlineto{\pgfqpoint{1.736657in}{2.698150in}}%
\pgfpathlineto{\pgfqpoint{1.725122in}{2.694892in}}%
\pgfpathlineto{\pgfqpoint{1.730820in}{2.687148in}}%
\pgfpathlineto{\pgfqpoint{1.736523in}{2.679386in}}%
\pgfpathlineto{\pgfqpoint{1.742230in}{2.671607in}}%
\pgfpathlineto{\pgfqpoint{1.747941in}{2.663810in}}%
\pgfpathclose%
\pgfusepath{stroke,fill}%
\end{pgfscope}%
\begin{pgfscope}%
\pgfpathrectangle{\pgfqpoint{0.887500in}{0.275000in}}{\pgfqpoint{4.225000in}{4.225000in}}%
\pgfusepath{clip}%
\pgfsetbuttcap%
\pgfsetroundjoin%
\definecolor{currentfill}{rgb}{0.458674,0.816363,0.329727}%
\pgfsetfillcolor{currentfill}%
\pgfsetfillopacity{0.700000}%
\pgfsetlinewidth{0.501875pt}%
\definecolor{currentstroke}{rgb}{1.000000,1.000000,1.000000}%
\pgfsetstrokecolor{currentstroke}%
\pgfsetstrokeopacity{0.500000}%
\pgfsetdash{}{0pt}%
\pgfpathmoveto{\pgfqpoint{3.925747in}{3.161818in}}%
\pgfpathlineto{\pgfqpoint{3.936765in}{3.165067in}}%
\pgfpathlineto{\pgfqpoint{3.947776in}{3.168262in}}%
\pgfpathlineto{\pgfqpoint{3.958780in}{3.171410in}}%
\pgfpathlineto{\pgfqpoint{3.969778in}{3.174517in}}%
\pgfpathlineto{\pgfqpoint{3.980771in}{3.177592in}}%
\pgfpathlineto{\pgfqpoint{3.974413in}{3.189866in}}%
\pgfpathlineto{\pgfqpoint{3.968057in}{3.202056in}}%
\pgfpathlineto{\pgfqpoint{3.961702in}{3.214158in}}%
\pgfpathlineto{\pgfqpoint{3.955347in}{3.226167in}}%
\pgfpathlineto{\pgfqpoint{3.948994in}{3.238081in}}%
\pgfpathlineto{\pgfqpoint{3.938005in}{3.234959in}}%
\pgfpathlineto{\pgfqpoint{3.927010in}{3.231797in}}%
\pgfpathlineto{\pgfqpoint{3.916008in}{3.228590in}}%
\pgfpathlineto{\pgfqpoint{3.905001in}{3.225334in}}%
\pgfpathlineto{\pgfqpoint{3.893987in}{3.222025in}}%
\pgfpathlineto{\pgfqpoint{3.900334in}{3.210039in}}%
\pgfpathlineto{\pgfqpoint{3.906684in}{3.198038in}}%
\pgfpathlineto{\pgfqpoint{3.913036in}{3.186010in}}%
\pgfpathlineto{\pgfqpoint{3.919391in}{3.173942in}}%
\pgfpathclose%
\pgfusepath{stroke,fill}%
\end{pgfscope}%
\begin{pgfscope}%
\pgfpathrectangle{\pgfqpoint{0.887500in}{0.275000in}}{\pgfqpoint{4.225000in}{4.225000in}}%
\pgfusepath{clip}%
\pgfsetbuttcap%
\pgfsetroundjoin%
\definecolor{currentfill}{rgb}{0.121831,0.589055,0.545623}%
\pgfsetfillcolor{currentfill}%
\pgfsetfillopacity{0.700000}%
\pgfsetlinewidth{0.501875pt}%
\definecolor{currentstroke}{rgb}{1.000000,1.000000,1.000000}%
\pgfsetstrokecolor{currentstroke}%
\pgfsetstrokeopacity{0.500000}%
\pgfsetdash{}{0pt}%
\pgfpathmoveto{\pgfqpoint{2.730390in}{2.623937in}}%
\pgfpathlineto{\pgfqpoint{2.741634in}{2.634666in}}%
\pgfpathlineto{\pgfqpoint{2.752876in}{2.645549in}}%
\pgfpathlineto{\pgfqpoint{2.764113in}{2.656875in}}%
\pgfpathlineto{\pgfqpoint{2.775345in}{2.668934in}}%
\pgfpathlineto{\pgfqpoint{2.786569in}{2.682015in}}%
\pgfpathlineto{\pgfqpoint{2.780514in}{2.688703in}}%
\pgfpathlineto{\pgfqpoint{2.774460in}{2.695798in}}%
\pgfpathlineto{\pgfqpoint{2.768408in}{2.703308in}}%
\pgfpathlineto{\pgfqpoint{2.762357in}{2.711239in}}%
\pgfpathlineto{\pgfqpoint{2.756307in}{2.719596in}}%
\pgfpathlineto{\pgfqpoint{2.745084in}{2.708958in}}%
\pgfpathlineto{\pgfqpoint{2.733852in}{2.699274in}}%
\pgfpathlineto{\pgfqpoint{2.722613in}{2.690122in}}%
\pgfpathlineto{\pgfqpoint{2.711371in}{2.681082in}}%
\pgfpathlineto{\pgfqpoint{2.700130in}{2.671733in}}%
\pgfpathlineto{\pgfqpoint{2.706171in}{2.662679in}}%
\pgfpathlineto{\pgfqpoint{2.712216in}{2.653424in}}%
\pgfpathlineto{\pgfqpoint{2.718268in}{2.643917in}}%
\pgfpathlineto{\pgfqpoint{2.724326in}{2.634106in}}%
\pgfpathclose%
\pgfusepath{stroke,fill}%
\end{pgfscope}%
\begin{pgfscope}%
\pgfpathrectangle{\pgfqpoint{0.887500in}{0.275000in}}{\pgfqpoint{4.225000in}{4.225000in}}%
\pgfusepath{clip}%
\pgfsetbuttcap%
\pgfsetroundjoin%
\definecolor{currentfill}{rgb}{0.119699,0.618490,0.536347}%
\pgfsetfillcolor{currentfill}%
\pgfsetfillopacity{0.700000}%
\pgfsetlinewidth{0.501875pt}%
\definecolor{currentstroke}{rgb}{1.000000,1.000000,1.000000}%
\pgfsetstrokecolor{currentstroke}%
\pgfsetstrokeopacity{0.500000}%
\pgfsetdash{}{0pt}%
\pgfpathmoveto{\pgfqpoint{2.786569in}{2.682015in}}%
\pgfpathlineto{\pgfqpoint{2.797784in}{2.696409in}}%
\pgfpathlineto{\pgfqpoint{2.808990in}{2.712364in}}%
\pgfpathlineto{\pgfqpoint{2.820189in}{2.729918in}}%
\pgfpathlineto{\pgfqpoint{2.831381in}{2.749044in}}%
\pgfpathlineto{\pgfqpoint{2.842569in}{2.769717in}}%
\pgfpathlineto{\pgfqpoint{2.836506in}{2.773456in}}%
\pgfpathlineto{\pgfqpoint{2.830444in}{2.778005in}}%
\pgfpathlineto{\pgfqpoint{2.824382in}{2.783335in}}%
\pgfpathlineto{\pgfqpoint{2.818321in}{2.789416in}}%
\pgfpathlineto{\pgfqpoint{2.812260in}{2.796220in}}%
\pgfpathlineto{\pgfqpoint{2.801087in}{2.777782in}}%
\pgfpathlineto{\pgfqpoint{2.789907in}{2.760779in}}%
\pgfpathlineto{\pgfqpoint{2.778719in}{2.745343in}}%
\pgfpathlineto{\pgfqpoint{2.767519in}{2.731608in}}%
\pgfpathlineto{\pgfqpoint{2.756307in}{2.719596in}}%
\pgfpathlineto{\pgfqpoint{2.762357in}{2.711239in}}%
\pgfpathlineto{\pgfqpoint{2.768408in}{2.703308in}}%
\pgfpathlineto{\pgfqpoint{2.774460in}{2.695798in}}%
\pgfpathlineto{\pgfqpoint{2.780514in}{2.688703in}}%
\pgfpathclose%
\pgfusepath{stroke,fill}%
\end{pgfscope}%
\begin{pgfscope}%
\pgfpathrectangle{\pgfqpoint{0.887500in}{0.275000in}}{\pgfqpoint{4.225000in}{4.225000in}}%
\pgfusepath{clip}%
\pgfsetbuttcap%
\pgfsetroundjoin%
\definecolor{currentfill}{rgb}{0.506271,0.828786,0.300362}%
\pgfsetfillcolor{currentfill}%
\pgfsetfillopacity{0.700000}%
\pgfsetlinewidth{0.501875pt}%
\definecolor{currentstroke}{rgb}{1.000000,1.000000,1.000000}%
\pgfsetstrokecolor{currentstroke}%
\pgfsetstrokeopacity{0.500000}%
\pgfsetdash{}{0pt}%
\pgfpathmoveto{\pgfqpoint{3.838822in}{3.204664in}}%
\pgfpathlineto{\pgfqpoint{3.849867in}{3.208211in}}%
\pgfpathlineto{\pgfqpoint{3.860906in}{3.211737in}}%
\pgfpathlineto{\pgfqpoint{3.871939in}{3.215226in}}%
\pgfpathlineto{\pgfqpoint{3.882966in}{3.218656in}}%
\pgfpathlineto{\pgfqpoint{3.893987in}{3.222025in}}%
\pgfpathlineto{\pgfqpoint{3.887642in}{3.234001in}}%
\pgfpathlineto{\pgfqpoint{3.881300in}{3.245958in}}%
\pgfpathlineto{\pgfqpoint{3.874961in}{3.257882in}}%
\pgfpathlineto{\pgfqpoint{3.868624in}{3.269761in}}%
\pgfpathlineto{\pgfqpoint{3.862289in}{3.281584in}}%
\pgfpathlineto{\pgfqpoint{3.851279in}{3.278489in}}%
\pgfpathlineto{\pgfqpoint{3.840261in}{3.275311in}}%
\pgfpathlineto{\pgfqpoint{3.829237in}{3.272046in}}%
\pgfpathlineto{\pgfqpoint{3.818207in}{3.268714in}}%
\pgfpathlineto{\pgfqpoint{3.807170in}{3.265333in}}%
\pgfpathlineto{\pgfqpoint{3.813495in}{3.253237in}}%
\pgfpathlineto{\pgfqpoint{3.819822in}{3.241094in}}%
\pgfpathlineto{\pgfqpoint{3.826152in}{3.228931in}}%
\pgfpathlineto{\pgfqpoint{3.832485in}{3.216778in}}%
\pgfpathclose%
\pgfusepath{stroke,fill}%
\end{pgfscope}%
\begin{pgfscope}%
\pgfpathrectangle{\pgfqpoint{0.887500in}{0.275000in}}{\pgfqpoint{4.225000in}{4.225000in}}%
\pgfusepath{clip}%
\pgfsetbuttcap%
\pgfsetroundjoin%
\definecolor{currentfill}{rgb}{0.616293,0.852709,0.230052}%
\pgfsetfillcolor{currentfill}%
\pgfsetfillopacity{0.700000}%
\pgfsetlinewidth{0.501875pt}%
\definecolor{currentstroke}{rgb}{1.000000,1.000000,1.000000}%
\pgfsetstrokecolor{currentstroke}%
\pgfsetstrokeopacity{0.500000}%
\pgfsetdash{}{0pt}%
\pgfpathmoveto{\pgfqpoint{3.664968in}{3.291008in}}%
\pgfpathlineto{\pgfqpoint{3.676052in}{3.294384in}}%
\pgfpathlineto{\pgfqpoint{3.687130in}{3.297743in}}%
\pgfpathlineto{\pgfqpoint{3.698203in}{3.301086in}}%
\pgfpathlineto{\pgfqpoint{3.709270in}{3.304414in}}%
\pgfpathlineto{\pgfqpoint{3.720332in}{3.307728in}}%
\pgfpathlineto{\pgfqpoint{3.714017in}{3.319066in}}%
\pgfpathlineto{\pgfqpoint{3.707703in}{3.330207in}}%
\pgfpathlineto{\pgfqpoint{3.701389in}{3.341190in}}%
\pgfpathlineto{\pgfqpoint{3.695077in}{3.352056in}}%
\pgfpathlineto{\pgfqpoint{3.688766in}{3.362842in}}%
\pgfpathlineto{\pgfqpoint{3.677712in}{3.359596in}}%
\pgfpathlineto{\pgfqpoint{3.666652in}{3.356316in}}%
\pgfpathlineto{\pgfqpoint{3.655586in}{3.353005in}}%
\pgfpathlineto{\pgfqpoint{3.644514in}{3.349669in}}%
\pgfpathlineto{\pgfqpoint{3.633437in}{3.346312in}}%
\pgfpathlineto{\pgfqpoint{3.639740in}{3.335459in}}%
\pgfpathlineto{\pgfqpoint{3.646045in}{3.324538in}}%
\pgfpathlineto{\pgfqpoint{3.652351in}{3.313513in}}%
\pgfpathlineto{\pgfqpoint{3.658659in}{3.302348in}}%
\pgfpathclose%
\pgfusepath{stroke,fill}%
\end{pgfscope}%
\begin{pgfscope}%
\pgfpathrectangle{\pgfqpoint{0.887500in}{0.275000in}}{\pgfqpoint{4.225000in}{4.225000in}}%
\pgfusepath{clip}%
\pgfsetbuttcap%
\pgfsetroundjoin%
\definecolor{currentfill}{rgb}{0.565498,0.842430,0.262877}%
\pgfsetfillcolor{currentfill}%
\pgfsetfillopacity{0.700000}%
\pgfsetlinewidth{0.501875pt}%
\definecolor{currentstroke}{rgb}{1.000000,1.000000,1.000000}%
\pgfsetstrokecolor{currentstroke}%
\pgfsetstrokeopacity{0.500000}%
\pgfsetdash{}{0pt}%
\pgfpathmoveto{\pgfqpoint{3.751907in}{3.248366in}}%
\pgfpathlineto{\pgfqpoint{3.762969in}{3.251713in}}%
\pgfpathlineto{\pgfqpoint{3.774027in}{3.255094in}}%
\pgfpathlineto{\pgfqpoint{3.785080in}{3.258503in}}%
\pgfpathlineto{\pgfqpoint{3.796127in}{3.261923in}}%
\pgfpathlineto{\pgfqpoint{3.807170in}{3.265333in}}%
\pgfpathlineto{\pgfqpoint{3.800846in}{3.277352in}}%
\pgfpathlineto{\pgfqpoint{3.794524in}{3.289265in}}%
\pgfpathlineto{\pgfqpoint{3.788202in}{3.301044in}}%
\pgfpathlineto{\pgfqpoint{3.781880in}{3.312660in}}%
\pgfpathlineto{\pgfqpoint{3.775558in}{3.324084in}}%
\pgfpathlineto{\pgfqpoint{3.764524in}{3.320861in}}%
\pgfpathlineto{\pgfqpoint{3.753485in}{3.317604in}}%
\pgfpathlineto{\pgfqpoint{3.742439in}{3.314325in}}%
\pgfpathlineto{\pgfqpoint{3.731388in}{3.311032in}}%
\pgfpathlineto{\pgfqpoint{3.720332in}{3.307728in}}%
\pgfpathlineto{\pgfqpoint{3.726646in}{3.296176in}}%
\pgfpathlineto{\pgfqpoint{3.732960in}{3.284433in}}%
\pgfpathlineto{\pgfqpoint{3.739275in}{3.272530in}}%
\pgfpathlineto{\pgfqpoint{3.745590in}{3.260498in}}%
\pgfpathclose%
\pgfusepath{stroke,fill}%
\end{pgfscope}%
\begin{pgfscope}%
\pgfpathrectangle{\pgfqpoint{0.887500in}{0.275000in}}{\pgfqpoint{4.225000in}{4.225000in}}%
\pgfusepath{clip}%
\pgfsetbuttcap%
\pgfsetroundjoin%
\definecolor{currentfill}{rgb}{0.226397,0.728888,0.462789}%
\pgfsetfillcolor{currentfill}%
\pgfsetfillopacity{0.700000}%
\pgfsetlinewidth{0.501875pt}%
\definecolor{currentstroke}{rgb}{1.000000,1.000000,1.000000}%
\pgfsetstrokecolor{currentstroke}%
\pgfsetstrokeopacity{0.500000}%
\pgfsetdash{}{0pt}%
\pgfpathmoveto{\pgfqpoint{2.868092in}{2.905654in}}%
\pgfpathlineto{\pgfqpoint{2.879259in}{2.930837in}}%
\pgfpathlineto{\pgfqpoint{2.890427in}{2.957392in}}%
\pgfpathlineto{\pgfqpoint{2.901597in}{2.985474in}}%
\pgfpathlineto{\pgfqpoint{2.912769in}{3.015235in}}%
\pgfpathlineto{\pgfqpoint{2.923944in}{3.046830in}}%
\pgfpathlineto{\pgfqpoint{2.917852in}{3.046801in}}%
\pgfpathlineto{\pgfqpoint{2.911767in}{3.046563in}}%
\pgfpathlineto{\pgfqpoint{2.905690in}{3.046105in}}%
\pgfpathlineto{\pgfqpoint{2.899620in}{3.045419in}}%
\pgfpathlineto{\pgfqpoint{2.893557in}{3.044503in}}%
\pgfpathlineto{\pgfqpoint{2.882422in}{3.013842in}}%
\pgfpathlineto{\pgfqpoint{2.871279in}{2.987213in}}%
\pgfpathlineto{\pgfqpoint{2.860126in}{2.963911in}}%
\pgfpathlineto{\pgfqpoint{2.848966in}{2.943236in}}%
\pgfpathlineto{\pgfqpoint{2.837800in}{2.924487in}}%
\pgfpathlineto{\pgfqpoint{2.843849in}{2.920402in}}%
\pgfpathlineto{\pgfqpoint{2.849904in}{2.916399in}}%
\pgfpathlineto{\pgfqpoint{2.855962in}{2.912557in}}%
\pgfpathlineto{\pgfqpoint{2.862025in}{2.908953in}}%
\pgfpathclose%
\pgfusepath{stroke,fill}%
\end{pgfscope}%
\begin{pgfscope}%
\pgfpathrectangle{\pgfqpoint{0.887500in}{0.275000in}}{\pgfqpoint{4.225000in}{4.225000in}}%
\pgfusepath{clip}%
\pgfsetbuttcap%
\pgfsetroundjoin%
\definecolor{currentfill}{rgb}{0.127568,0.566949,0.550556}%
\pgfsetfillcolor{currentfill}%
\pgfsetfillopacity{0.700000}%
\pgfsetlinewidth{0.501875pt}%
\definecolor{currentstroke}{rgb}{1.000000,1.000000,1.000000}%
\pgfsetstrokecolor{currentstroke}%
\pgfsetstrokeopacity{0.500000}%
\pgfsetdash{}{0pt}%
\pgfpathmoveto{\pgfqpoint{2.070549in}{2.602266in}}%
\pgfpathlineto{\pgfqpoint{2.082021in}{2.605566in}}%
\pgfpathlineto{\pgfqpoint{2.093488in}{2.608853in}}%
\pgfpathlineto{\pgfqpoint{2.104950in}{2.612127in}}%
\pgfpathlineto{\pgfqpoint{2.116406in}{2.615387in}}%
\pgfpathlineto{\pgfqpoint{2.127856in}{2.618643in}}%
\pgfpathlineto{\pgfqpoint{2.122014in}{2.626710in}}%
\pgfpathlineto{\pgfqpoint{2.116175in}{2.634764in}}%
\pgfpathlineto{\pgfqpoint{2.110341in}{2.642807in}}%
\pgfpathlineto{\pgfqpoint{2.104510in}{2.650838in}}%
\pgfpathlineto{\pgfqpoint{2.098684in}{2.658857in}}%
\pgfpathlineto{\pgfqpoint{2.087245in}{2.655614in}}%
\pgfpathlineto{\pgfqpoint{2.075801in}{2.652366in}}%
\pgfpathlineto{\pgfqpoint{2.064352in}{2.649098in}}%
\pgfpathlineto{\pgfqpoint{2.052898in}{2.645810in}}%
\pgfpathlineto{\pgfqpoint{2.041438in}{2.642506in}}%
\pgfpathlineto{\pgfqpoint{2.047252in}{2.634482in}}%
\pgfpathlineto{\pgfqpoint{2.053070in}{2.626446in}}%
\pgfpathlineto{\pgfqpoint{2.058893in}{2.618398in}}%
\pgfpathlineto{\pgfqpoint{2.064719in}{2.610338in}}%
\pgfpathclose%
\pgfusepath{stroke,fill}%
\end{pgfscope}%
\begin{pgfscope}%
\pgfpathrectangle{\pgfqpoint{0.887500in}{0.275000in}}{\pgfqpoint{4.225000in}{4.225000in}}%
\pgfusepath{clip}%
\pgfsetbuttcap%
\pgfsetroundjoin%
\definecolor{currentfill}{rgb}{0.119512,0.607464,0.540218}%
\pgfsetfillcolor{currentfill}%
\pgfsetfillopacity{0.700000}%
\pgfsetlinewidth{0.501875pt}%
\definecolor{currentstroke}{rgb}{1.000000,1.000000,1.000000}%
\pgfsetstrokecolor{currentstroke}%
\pgfsetstrokeopacity{0.500000}%
\pgfsetdash{}{0pt}%
\pgfpathmoveto{\pgfqpoint{1.523203in}{2.684100in}}%
\pgfpathlineto{\pgfqpoint{1.534808in}{2.687395in}}%
\pgfpathlineto{\pgfqpoint{1.546408in}{2.690687in}}%
\pgfpathlineto{\pgfqpoint{1.558002in}{2.693978in}}%
\pgfpathlineto{\pgfqpoint{1.569590in}{2.697269in}}%
\pgfpathlineto{\pgfqpoint{1.581173in}{2.700559in}}%
\pgfpathlineto{\pgfqpoint{1.575528in}{2.708184in}}%
\pgfpathlineto{\pgfqpoint{1.569887in}{2.715793in}}%
\pgfpathlineto{\pgfqpoint{1.564250in}{2.723385in}}%
\pgfpathlineto{\pgfqpoint{1.558618in}{2.730962in}}%
\pgfpathlineto{\pgfqpoint{1.552990in}{2.738522in}}%
\pgfpathlineto{\pgfqpoint{1.541421in}{2.735215in}}%
\pgfpathlineto{\pgfqpoint{1.529847in}{2.731910in}}%
\pgfpathlineto{\pgfqpoint{1.518267in}{2.728605in}}%
\pgfpathlineto{\pgfqpoint{1.506682in}{2.725301in}}%
\pgfpathlineto{\pgfqpoint{1.495090in}{2.721995in}}%
\pgfpathlineto{\pgfqpoint{1.500704in}{2.714450in}}%
\pgfpathlineto{\pgfqpoint{1.506322in}{2.706887in}}%
\pgfpathlineto{\pgfqpoint{1.511945in}{2.699308in}}%
\pgfpathlineto{\pgfqpoint{1.517571in}{2.691712in}}%
\pgfpathclose%
\pgfusepath{stroke,fill}%
\end{pgfscope}%
\begin{pgfscope}%
\pgfpathrectangle{\pgfqpoint{0.887500in}{0.275000in}}{\pgfqpoint{4.225000in}{4.225000in}}%
\pgfusepath{clip}%
\pgfsetbuttcap%
\pgfsetroundjoin%
\definecolor{currentfill}{rgb}{0.128729,0.563265,0.551229}%
\pgfsetfillcolor{currentfill}%
\pgfsetfillopacity{0.700000}%
\pgfsetlinewidth{0.501875pt}%
\definecolor{currentstroke}{rgb}{1.000000,1.000000,1.000000}%
\pgfsetstrokecolor{currentstroke}%
\pgfsetstrokeopacity{0.500000}%
\pgfsetdash{}{0pt}%
\pgfpathmoveto{\pgfqpoint{2.674167in}{2.568413in}}%
\pgfpathlineto{\pgfqpoint{2.685415in}{2.579406in}}%
\pgfpathlineto{\pgfqpoint{2.696659in}{2.590640in}}%
\pgfpathlineto{\pgfqpoint{2.707902in}{2.601925in}}%
\pgfpathlineto{\pgfqpoint{2.719145in}{2.613074in}}%
\pgfpathlineto{\pgfqpoint{2.730390in}{2.623937in}}%
\pgfpathlineto{\pgfqpoint{2.724326in}{2.634106in}}%
\pgfpathlineto{\pgfqpoint{2.718268in}{2.643917in}}%
\pgfpathlineto{\pgfqpoint{2.712216in}{2.653424in}}%
\pgfpathlineto{\pgfqpoint{2.706171in}{2.662679in}}%
\pgfpathlineto{\pgfqpoint{2.700130in}{2.671733in}}%
\pgfpathlineto{\pgfqpoint{2.688894in}{2.661657in}}%
\pgfpathlineto{\pgfqpoint{2.677664in}{2.650649in}}%
\pgfpathlineto{\pgfqpoint{2.666440in}{2.638978in}}%
\pgfpathlineto{\pgfqpoint{2.655219in}{2.626972in}}%
\pgfpathlineto{\pgfqpoint{2.643998in}{2.614962in}}%
\pgfpathlineto{\pgfqpoint{2.650018in}{2.606257in}}%
\pgfpathlineto{\pgfqpoint{2.656045in}{2.597275in}}%
\pgfpathlineto{\pgfqpoint{2.662079in}{2.587990in}}%
\pgfpathlineto{\pgfqpoint{2.668120in}{2.578378in}}%
\pgfpathclose%
\pgfusepath{stroke,fill}%
\end{pgfscope}%
\begin{pgfscope}%
\pgfpathrectangle{\pgfqpoint{0.887500in}{0.275000in}}{\pgfqpoint{4.225000in}{4.225000in}}%
\pgfusepath{clip}%
\pgfsetbuttcap%
\pgfsetroundjoin%
\definecolor{currentfill}{rgb}{0.135066,0.544853,0.554029}%
\pgfsetfillcolor{currentfill}%
\pgfsetfillopacity{0.700000}%
\pgfsetlinewidth{0.501875pt}%
\definecolor{currentstroke}{rgb}{1.000000,1.000000,1.000000}%
\pgfsetstrokecolor{currentstroke}%
\pgfsetstrokeopacity{0.500000}%
\pgfsetdash{}{0pt}%
\pgfpathmoveto{\pgfqpoint{2.387581in}{2.546897in}}%
\pgfpathlineto{\pgfqpoint{2.398974in}{2.550379in}}%
\pgfpathlineto{\pgfqpoint{2.410364in}{2.553747in}}%
\pgfpathlineto{\pgfqpoint{2.421751in}{2.556959in}}%
\pgfpathlineto{\pgfqpoint{2.433136in}{2.559976in}}%
\pgfpathlineto{\pgfqpoint{2.444517in}{2.562818in}}%
\pgfpathlineto{\pgfqpoint{2.438566in}{2.571061in}}%
\pgfpathlineto{\pgfqpoint{2.432618in}{2.579308in}}%
\pgfpathlineto{\pgfqpoint{2.426674in}{2.587562in}}%
\pgfpathlineto{\pgfqpoint{2.420734in}{2.595833in}}%
\pgfpathlineto{\pgfqpoint{2.414797in}{2.604126in}}%
\pgfpathlineto{\pgfqpoint{2.403427in}{2.601293in}}%
\pgfpathlineto{\pgfqpoint{2.392054in}{2.598283in}}%
\pgfpathlineto{\pgfqpoint{2.380678in}{2.595072in}}%
\pgfpathlineto{\pgfqpoint{2.369300in}{2.591701in}}%
\pgfpathlineto{\pgfqpoint{2.357919in}{2.588214in}}%
\pgfpathlineto{\pgfqpoint{2.363842in}{2.580014in}}%
\pgfpathlineto{\pgfqpoint{2.369770in}{2.571778in}}%
\pgfpathlineto{\pgfqpoint{2.375703in}{2.563511in}}%
\pgfpathlineto{\pgfqpoint{2.381640in}{2.555216in}}%
\pgfpathclose%
\pgfusepath{stroke,fill}%
\end{pgfscope}%
\begin{pgfscope}%
\pgfpathrectangle{\pgfqpoint{0.887500in}{0.275000in}}{\pgfqpoint{4.225000in}{4.225000in}}%
\pgfusepath{clip}%
\pgfsetbuttcap%
\pgfsetroundjoin%
\definecolor{currentfill}{rgb}{0.783315,0.879285,0.125405}%
\pgfsetfillcolor{currentfill}%
\pgfsetfillopacity{0.700000}%
\pgfsetlinewidth{0.501875pt}%
\definecolor{currentstroke}{rgb}{1.000000,1.000000,1.000000}%
\pgfsetstrokecolor{currentstroke}%
\pgfsetstrokeopacity{0.500000}%
\pgfsetdash{}{0pt}%
\pgfpathmoveto{\pgfqpoint{3.261109in}{3.414291in}}%
\pgfpathlineto{\pgfqpoint{3.272297in}{3.418081in}}%
\pgfpathlineto{\pgfqpoint{3.283480in}{3.421988in}}%
\pgfpathlineto{\pgfqpoint{3.294659in}{3.425985in}}%
\pgfpathlineto{\pgfqpoint{3.305833in}{3.430043in}}%
\pgfpathlineto{\pgfqpoint{3.317003in}{3.434134in}}%
\pgfpathlineto{\pgfqpoint{3.310770in}{3.443302in}}%
\pgfpathlineto{\pgfqpoint{3.304540in}{3.452121in}}%
\pgfpathlineto{\pgfqpoint{3.298311in}{3.460592in}}%
\pgfpathlineto{\pgfqpoint{3.292084in}{3.468732in}}%
\pgfpathlineto{\pgfqpoint{3.285860in}{3.476554in}}%
\pgfpathlineto{\pgfqpoint{3.274701in}{3.472491in}}%
\pgfpathlineto{\pgfqpoint{3.263538in}{3.468466in}}%
\pgfpathlineto{\pgfqpoint{3.252370in}{3.464510in}}%
\pgfpathlineto{\pgfqpoint{3.241198in}{3.460654in}}%
\pgfpathlineto{\pgfqpoint{3.230021in}{3.456930in}}%
\pgfpathlineto{\pgfqpoint{3.236233in}{3.448982in}}%
\pgfpathlineto{\pgfqpoint{3.242448in}{3.440757in}}%
\pgfpathlineto{\pgfqpoint{3.248666in}{3.432239in}}%
\pgfpathlineto{\pgfqpoint{3.254886in}{3.423417in}}%
\pgfpathclose%
\pgfusepath{stroke,fill}%
\end{pgfscope}%
\begin{pgfscope}%
\pgfpathrectangle{\pgfqpoint{0.887500in}{0.275000in}}{\pgfqpoint{4.225000in}{4.225000in}}%
\pgfusepath{clip}%
\pgfsetbuttcap%
\pgfsetroundjoin%
\definecolor{currentfill}{rgb}{0.122606,0.585371,0.546557}%
\pgfsetfillcolor{currentfill}%
\pgfsetfillopacity{0.700000}%
\pgfsetlinewidth{0.501875pt}%
\definecolor{currentstroke}{rgb}{1.000000,1.000000,1.000000}%
\pgfsetstrokecolor{currentstroke}%
\pgfsetstrokeopacity{0.500000}%
\pgfsetdash{}{0pt}%
\pgfpathmoveto{\pgfqpoint{1.840053in}{2.632806in}}%
\pgfpathlineto{\pgfqpoint{1.851586in}{2.636054in}}%
\pgfpathlineto{\pgfqpoint{1.863113in}{2.639310in}}%
\pgfpathlineto{\pgfqpoint{1.874634in}{2.642574in}}%
\pgfpathlineto{\pgfqpoint{1.886149in}{2.645850in}}%
\pgfpathlineto{\pgfqpoint{1.897658in}{2.649139in}}%
\pgfpathlineto{\pgfqpoint{1.891895in}{2.657064in}}%
\pgfpathlineto{\pgfqpoint{1.886135in}{2.664971in}}%
\pgfpathlineto{\pgfqpoint{1.880380in}{2.672859in}}%
\pgfpathlineto{\pgfqpoint{1.874629in}{2.680727in}}%
\pgfpathlineto{\pgfqpoint{1.868883in}{2.688577in}}%
\pgfpathlineto{\pgfqpoint{1.857387in}{2.685286in}}%
\pgfpathlineto{\pgfqpoint{1.845885in}{2.682007in}}%
\pgfpathlineto{\pgfqpoint{1.834377in}{2.678738in}}%
\pgfpathlineto{\pgfqpoint{1.822863in}{2.675478in}}%
\pgfpathlineto{\pgfqpoint{1.811343in}{2.672224in}}%
\pgfpathlineto{\pgfqpoint{1.817076in}{2.664379in}}%
\pgfpathlineto{\pgfqpoint{1.822814in}{2.656515in}}%
\pgfpathlineto{\pgfqpoint{1.828556in}{2.648631in}}%
\pgfpathlineto{\pgfqpoint{1.834302in}{2.640728in}}%
\pgfpathclose%
\pgfusepath{stroke,fill}%
\end{pgfscope}%
\begin{pgfscope}%
\pgfpathrectangle{\pgfqpoint{0.887500in}{0.275000in}}{\pgfqpoint{4.225000in}{4.225000in}}%
\pgfusepath{clip}%
\pgfsetbuttcap%
\pgfsetroundjoin%
\definecolor{currentfill}{rgb}{0.636902,0.856542,0.216620}%
\pgfsetfillcolor{currentfill}%
\pgfsetfillopacity{0.700000}%
\pgfsetlinewidth{0.501875pt}%
\definecolor{currentstroke}{rgb}{1.000000,1.000000,1.000000}%
\pgfsetstrokecolor{currentstroke}%
\pgfsetstrokeopacity{0.500000}%
\pgfsetdash{}{0pt}%
\pgfpathmoveto{\pgfqpoint{2.918764in}{3.246537in}}%
\pgfpathlineto{\pgfqpoint{2.929895in}{3.287920in}}%
\pgfpathlineto{\pgfqpoint{2.941056in}{3.324329in}}%
\pgfpathlineto{\pgfqpoint{2.952246in}{3.354740in}}%
\pgfpathlineto{\pgfqpoint{2.963458in}{3.379673in}}%
\pgfpathlineto{\pgfqpoint{2.974687in}{3.399780in}}%
\pgfpathlineto{\pgfqpoint{2.968554in}{3.405255in}}%
\pgfpathlineto{\pgfqpoint{2.962426in}{3.410396in}}%
\pgfpathlineto{\pgfqpoint{2.956305in}{3.415217in}}%
\pgfpathlineto{\pgfqpoint{2.950188in}{3.419769in}}%
\pgfpathlineto{\pgfqpoint{2.944078in}{3.424104in}}%
\pgfpathlineto{\pgfqpoint{2.932882in}{3.402595in}}%
\pgfpathlineto{\pgfqpoint{2.921710in}{3.375364in}}%
\pgfpathlineto{\pgfqpoint{2.910569in}{3.341598in}}%
\pgfpathlineto{\pgfqpoint{2.899468in}{3.300660in}}%
\pgfpathlineto{\pgfqpoint{2.888407in}{3.253877in}}%
\pgfpathlineto{\pgfqpoint{2.894467in}{3.252378in}}%
\pgfpathlineto{\pgfqpoint{2.900531in}{3.251071in}}%
\pgfpathlineto{\pgfqpoint{2.906602in}{3.249767in}}%
\pgfpathlineto{\pgfqpoint{2.912680in}{3.248281in}}%
\pgfpathclose%
\pgfusepath{stroke,fill}%
\end{pgfscope}%
\begin{pgfscope}%
\pgfpathrectangle{\pgfqpoint{0.887500in}{0.275000in}}{\pgfqpoint{4.225000in}{4.225000in}}%
\pgfusepath{clip}%
\pgfsetbuttcap%
\pgfsetroundjoin%
\definecolor{currentfill}{rgb}{0.814576,0.883393,0.110347}%
\pgfsetfillcolor{currentfill}%
\pgfsetfillopacity{0.700000}%
\pgfsetlinewidth{0.501875pt}%
\definecolor{currentstroke}{rgb}{1.000000,1.000000,1.000000}%
\pgfsetstrokecolor{currentstroke}%
\pgfsetstrokeopacity{0.500000}%
\pgfsetdash{}{0pt}%
\pgfpathmoveto{\pgfqpoint{3.030922in}{3.450907in}}%
\pgfpathlineto{\pgfqpoint{3.042166in}{3.455260in}}%
\pgfpathlineto{\pgfqpoint{3.053406in}{3.458535in}}%
\pgfpathlineto{\pgfqpoint{3.064640in}{3.461033in}}%
\pgfpathlineto{\pgfqpoint{3.075869in}{3.463055in}}%
\pgfpathlineto{\pgfqpoint{3.087091in}{3.464903in}}%
\pgfpathlineto{\pgfqpoint{3.080923in}{3.470268in}}%
\pgfpathlineto{\pgfqpoint{3.074761in}{3.475071in}}%
\pgfpathlineto{\pgfqpoint{3.068603in}{3.479321in}}%
\pgfpathlineto{\pgfqpoint{3.062450in}{3.483087in}}%
\pgfpathlineto{\pgfqpoint{3.056303in}{3.486440in}}%
\pgfpathlineto{\pgfqpoint{3.045096in}{3.484468in}}%
\pgfpathlineto{\pgfqpoint{3.033882in}{3.482673in}}%
\pgfpathlineto{\pgfqpoint{3.022663in}{3.480659in}}%
\pgfpathlineto{\pgfqpoint{3.011439in}{3.478032in}}%
\pgfpathlineto{\pgfqpoint{3.000210in}{3.474397in}}%
\pgfpathlineto{\pgfqpoint{3.006341in}{3.470495in}}%
\pgfpathlineto{\pgfqpoint{3.012478in}{3.466246in}}%
\pgfpathlineto{\pgfqpoint{3.018621in}{3.461594in}}%
\pgfpathlineto{\pgfqpoint{3.024768in}{3.456484in}}%
\pgfpathclose%
\pgfusepath{stroke,fill}%
\end{pgfscope}%
\begin{pgfscope}%
\pgfpathrectangle{\pgfqpoint{0.887500in}{0.275000in}}{\pgfqpoint{4.225000in}{4.225000in}}%
\pgfusepath{clip}%
\pgfsetbuttcap%
\pgfsetroundjoin%
\definecolor{currentfill}{rgb}{0.136408,0.541173,0.554483}%
\pgfsetfillcolor{currentfill}%
\pgfsetfillopacity{0.700000}%
\pgfsetlinewidth{0.501875pt}%
\definecolor{currentstroke}{rgb}{1.000000,1.000000,1.000000}%
\pgfsetstrokecolor{currentstroke}%
\pgfsetstrokeopacity{0.500000}%
\pgfsetdash{}{0pt}%
\pgfpathmoveto{\pgfqpoint{2.617805in}{2.523073in}}%
\pgfpathlineto{\pgfqpoint{2.629100in}{2.530494in}}%
\pgfpathlineto{\pgfqpoint{2.640383in}{2.538751in}}%
\pgfpathlineto{\pgfqpoint{2.651654in}{2.547898in}}%
\pgfpathlineto{\pgfqpoint{2.662914in}{2.557848in}}%
\pgfpathlineto{\pgfqpoint{2.674167in}{2.568413in}}%
\pgfpathlineto{\pgfqpoint{2.668120in}{2.578378in}}%
\pgfpathlineto{\pgfqpoint{2.662079in}{2.587990in}}%
\pgfpathlineto{\pgfqpoint{2.656045in}{2.597275in}}%
\pgfpathlineto{\pgfqpoint{2.650018in}{2.606257in}}%
\pgfpathlineto{\pgfqpoint{2.643998in}{2.614962in}}%
\pgfpathlineto{\pgfqpoint{2.632772in}{2.603276in}}%
\pgfpathlineto{\pgfqpoint{2.621538in}{2.592243in}}%
\pgfpathlineto{\pgfqpoint{2.610293in}{2.582188in}}%
\pgfpathlineto{\pgfqpoint{2.599032in}{2.573272in}}%
\pgfpathlineto{\pgfqpoint{2.587757in}{2.565406in}}%
\pgfpathlineto{\pgfqpoint{2.593756in}{2.557165in}}%
\pgfpathlineto{\pgfqpoint{2.599761in}{2.548799in}}%
\pgfpathlineto{\pgfqpoint{2.605771in}{2.540320in}}%
\pgfpathlineto{\pgfqpoint{2.611786in}{2.531741in}}%
\pgfpathclose%
\pgfusepath{stroke,fill}%
\end{pgfscope}%
\begin{pgfscope}%
\pgfpathrectangle{\pgfqpoint{0.887500in}{0.275000in}}{\pgfqpoint{4.225000in}{4.225000in}}%
\pgfusepath{clip}%
\pgfsetbuttcap%
\pgfsetroundjoin%
\definecolor{currentfill}{rgb}{0.129933,0.559582,0.551864}%
\pgfsetfillcolor{currentfill}%
\pgfsetfillopacity{0.700000}%
\pgfsetlinewidth{0.501875pt}%
\definecolor{currentstroke}{rgb}{1.000000,1.000000,1.000000}%
\pgfsetstrokecolor{currentstroke}%
\pgfsetstrokeopacity{0.500000}%
\pgfsetdash{}{0pt}%
\pgfpathmoveto{\pgfqpoint{2.157131in}{2.578111in}}%
\pgfpathlineto{\pgfqpoint{2.168588in}{2.581382in}}%
\pgfpathlineto{\pgfqpoint{2.180039in}{2.584663in}}%
\pgfpathlineto{\pgfqpoint{2.191484in}{2.587957in}}%
\pgfpathlineto{\pgfqpoint{2.202923in}{2.591271in}}%
\pgfpathlineto{\pgfqpoint{2.214356in}{2.594610in}}%
\pgfpathlineto{\pgfqpoint{2.208480in}{2.602768in}}%
\pgfpathlineto{\pgfqpoint{2.202608in}{2.610912in}}%
\pgfpathlineto{\pgfqpoint{2.196741in}{2.619045in}}%
\pgfpathlineto{\pgfqpoint{2.190877in}{2.627164in}}%
\pgfpathlineto{\pgfqpoint{2.185017in}{2.635271in}}%
\pgfpathlineto{\pgfqpoint{2.173598in}{2.631859in}}%
\pgfpathlineto{\pgfqpoint{2.162172in}{2.628502in}}%
\pgfpathlineto{\pgfqpoint{2.150740in}{2.625188in}}%
\pgfpathlineto{\pgfqpoint{2.139301in}{2.621906in}}%
\pgfpathlineto{\pgfqpoint{2.127856in}{2.618643in}}%
\pgfpathlineto{\pgfqpoint{2.133703in}{2.610563in}}%
\pgfpathlineto{\pgfqpoint{2.139554in}{2.602470in}}%
\pgfpathlineto{\pgfqpoint{2.145409in}{2.594364in}}%
\pgfpathlineto{\pgfqpoint{2.151268in}{2.586245in}}%
\pgfpathclose%
\pgfusepath{stroke,fill}%
\end{pgfscope}%
\begin{pgfscope}%
\pgfpathrectangle{\pgfqpoint{0.887500in}{0.275000in}}{\pgfqpoint{4.225000in}{4.225000in}}%
\pgfusepath{clip}%
\pgfsetbuttcap%
\pgfsetroundjoin%
\definecolor{currentfill}{rgb}{0.120092,0.600104,0.542530}%
\pgfsetfillcolor{currentfill}%
\pgfsetfillopacity{0.700000}%
\pgfsetlinewidth{0.501875pt}%
\definecolor{currentstroke}{rgb}{1.000000,1.000000,1.000000}%
\pgfsetstrokecolor{currentstroke}%
\pgfsetstrokeopacity{0.500000}%
\pgfsetdash{}{0pt}%
\pgfpathmoveto{\pgfqpoint{1.609464in}{2.662200in}}%
\pgfpathlineto{\pgfqpoint{1.621055in}{2.665478in}}%
\pgfpathlineto{\pgfqpoint{1.632640in}{2.668755in}}%
\pgfpathlineto{\pgfqpoint{1.644220in}{2.672030in}}%
\pgfpathlineto{\pgfqpoint{1.655794in}{2.675303in}}%
\pgfpathlineto{\pgfqpoint{1.667363in}{2.678574in}}%
\pgfpathlineto{\pgfqpoint{1.661682in}{2.686289in}}%
\pgfpathlineto{\pgfqpoint{1.656006in}{2.693988in}}%
\pgfpathlineto{\pgfqpoint{1.650334in}{2.701672in}}%
\pgfpathlineto{\pgfqpoint{1.644666in}{2.709342in}}%
\pgfpathlineto{\pgfqpoint{1.639003in}{2.716997in}}%
\pgfpathlineto{\pgfqpoint{1.627448in}{2.713713in}}%
\pgfpathlineto{\pgfqpoint{1.615888in}{2.710427in}}%
\pgfpathlineto{\pgfqpoint{1.604322in}{2.707139in}}%
\pgfpathlineto{\pgfqpoint{1.592750in}{2.703850in}}%
\pgfpathlineto{\pgfqpoint{1.581173in}{2.700559in}}%
\pgfpathlineto{\pgfqpoint{1.586823in}{2.692919in}}%
\pgfpathlineto{\pgfqpoint{1.592477in}{2.685264in}}%
\pgfpathlineto{\pgfqpoint{1.598135in}{2.677593in}}%
\pgfpathlineto{\pgfqpoint{1.603797in}{2.669905in}}%
\pgfpathclose%
\pgfusepath{stroke,fill}%
\end{pgfscope}%
\begin{pgfscope}%
\pgfpathrectangle{\pgfqpoint{0.887500in}{0.275000in}}{\pgfqpoint{4.225000in}{4.225000in}}%
\pgfusepath{clip}%
\pgfsetbuttcap%
\pgfsetroundjoin%
\definecolor{currentfill}{rgb}{0.139147,0.533812,0.555298}%
\pgfsetfillcolor{currentfill}%
\pgfsetfillopacity{0.700000}%
\pgfsetlinewidth{0.501875pt}%
\definecolor{currentstroke}{rgb}{1.000000,1.000000,1.000000}%
\pgfsetstrokecolor{currentstroke}%
\pgfsetstrokeopacity{0.500000}%
\pgfsetdash{}{0pt}%
\pgfpathmoveto{\pgfqpoint{2.474334in}{2.521401in}}%
\pgfpathlineto{\pgfqpoint{2.485719in}{2.524430in}}%
\pgfpathlineto{\pgfqpoint{2.497096in}{2.527523in}}%
\pgfpathlineto{\pgfqpoint{2.508466in}{2.530768in}}%
\pgfpathlineto{\pgfqpoint{2.519827in}{2.534254in}}%
\pgfpathlineto{\pgfqpoint{2.531178in}{2.538066in}}%
\pgfpathlineto{\pgfqpoint{2.525197in}{2.546210in}}%
\pgfpathlineto{\pgfqpoint{2.519221in}{2.554250in}}%
\pgfpathlineto{\pgfqpoint{2.513251in}{2.562216in}}%
\pgfpathlineto{\pgfqpoint{2.507284in}{2.570153in}}%
\pgfpathlineto{\pgfqpoint{2.501322in}{2.578103in}}%
\pgfpathlineto{\pgfqpoint{2.489979in}{2.574542in}}%
\pgfpathlineto{\pgfqpoint{2.478626in}{2.571341in}}%
\pgfpathlineto{\pgfqpoint{2.467264in}{2.568392in}}%
\pgfpathlineto{\pgfqpoint{2.455894in}{2.565587in}}%
\pgfpathlineto{\pgfqpoint{2.444517in}{2.562818in}}%
\pgfpathlineto{\pgfqpoint{2.450472in}{2.554569in}}%
\pgfpathlineto{\pgfqpoint{2.456432in}{2.546310in}}%
\pgfpathlineto{\pgfqpoint{2.462395in}{2.538033in}}%
\pgfpathlineto{\pgfqpoint{2.468363in}{2.529732in}}%
\pgfpathclose%
\pgfusepath{stroke,fill}%
\end{pgfscope}%
\begin{pgfscope}%
\pgfpathrectangle{\pgfqpoint{0.887500in}{0.275000in}}{\pgfqpoint{4.225000in}{4.225000in}}%
\pgfusepath{clip}%
\pgfsetbuttcap%
\pgfsetroundjoin%
\definecolor{currentfill}{rgb}{0.246070,0.738910,0.452024}%
\pgfsetfillcolor{currentfill}%
\pgfsetfillopacity{0.700000}%
\pgfsetlinewidth{0.501875pt}%
\definecolor{currentstroke}{rgb}{1.000000,1.000000,1.000000}%
\pgfsetstrokecolor{currentstroke}%
\pgfsetstrokeopacity{0.500000}%
\pgfsetdash{}{0pt}%
\pgfpathmoveto{\pgfqpoint{4.218028in}{2.952053in}}%
\pgfpathlineto{\pgfqpoint{4.228986in}{2.955518in}}%
\pgfpathlineto{\pgfqpoint{4.239939in}{2.958979in}}%
\pgfpathlineto{\pgfqpoint{4.250886in}{2.962438in}}%
\pgfpathlineto{\pgfqpoint{4.261828in}{2.965894in}}%
\pgfpathlineto{\pgfqpoint{4.255444in}{2.979605in}}%
\pgfpathlineto{\pgfqpoint{4.249059in}{2.993215in}}%
\pgfpathlineto{\pgfqpoint{4.242673in}{3.006711in}}%
\pgfpathlineto{\pgfqpoint{4.236286in}{3.020079in}}%
\pgfpathlineto{\pgfqpoint{4.229898in}{3.033310in}}%
\pgfpathlineto{\pgfqpoint{4.218959in}{3.029887in}}%
\pgfpathlineto{\pgfqpoint{4.208015in}{3.026462in}}%
\pgfpathlineto{\pgfqpoint{4.197066in}{3.023036in}}%
\pgfpathlineto{\pgfqpoint{4.186112in}{3.019608in}}%
\pgfpathlineto{\pgfqpoint{4.192498in}{3.006370in}}%
\pgfpathlineto{\pgfqpoint{4.198882in}{2.992984in}}%
\pgfpathlineto{\pgfqpoint{4.205265in}{2.979458in}}%
\pgfpathlineto{\pgfqpoint{4.211647in}{2.965809in}}%
\pgfpathclose%
\pgfusepath{stroke,fill}%
\end{pgfscope}%
\begin{pgfscope}%
\pgfpathrectangle{\pgfqpoint{0.887500in}{0.275000in}}{\pgfqpoint{4.225000in}{4.225000in}}%
\pgfusepath{clip}%
\pgfsetbuttcap%
\pgfsetroundjoin%
\definecolor{currentfill}{rgb}{0.751884,0.874951,0.143228}%
\pgfsetfillcolor{currentfill}%
\pgfsetfillopacity{0.700000}%
\pgfsetlinewidth{0.501875pt}%
\definecolor{currentstroke}{rgb}{1.000000,1.000000,1.000000}%
\pgfsetstrokecolor{currentstroke}%
\pgfsetstrokeopacity{0.500000}%
\pgfsetdash{}{0pt}%
\pgfpathmoveto{\pgfqpoint{3.348192in}{3.384330in}}%
\pgfpathlineto{\pgfqpoint{3.359364in}{3.388228in}}%
\pgfpathlineto{\pgfqpoint{3.370531in}{3.392133in}}%
\pgfpathlineto{\pgfqpoint{3.381693in}{3.396037in}}%
\pgfpathlineto{\pgfqpoint{3.392850in}{3.399932in}}%
\pgfpathlineto{\pgfqpoint{3.404001in}{3.403811in}}%
\pgfpathlineto{\pgfqpoint{3.397754in}{3.414475in}}%
\pgfpathlineto{\pgfqpoint{3.391508in}{3.424921in}}%
\pgfpathlineto{\pgfqpoint{3.385263in}{3.435107in}}%
\pgfpathlineto{\pgfqpoint{3.379019in}{3.444992in}}%
\pgfpathlineto{\pgfqpoint{3.372777in}{3.454534in}}%
\pgfpathlineto{\pgfqpoint{3.361632in}{3.450512in}}%
\pgfpathlineto{\pgfqpoint{3.350482in}{3.446446in}}%
\pgfpathlineto{\pgfqpoint{3.339327in}{3.442351in}}%
\pgfpathlineto{\pgfqpoint{3.328167in}{3.438242in}}%
\pgfpathlineto{\pgfqpoint{3.317003in}{3.434134in}}%
\pgfpathlineto{\pgfqpoint{3.323236in}{3.424652in}}%
\pgfpathlineto{\pgfqpoint{3.329472in}{3.414891in}}%
\pgfpathlineto{\pgfqpoint{3.335710in}{3.404892in}}%
\pgfpathlineto{\pgfqpoint{3.341950in}{3.394692in}}%
\pgfpathclose%
\pgfusepath{stroke,fill}%
\end{pgfscope}%
\begin{pgfscope}%
\pgfpathrectangle{\pgfqpoint{0.887500in}{0.275000in}}{\pgfqpoint{4.225000in}{4.225000in}}%
\pgfusepath{clip}%
\pgfsetbuttcap%
\pgfsetroundjoin%
\definecolor{currentfill}{rgb}{0.296479,0.761561,0.424223}%
\pgfsetfillcolor{currentfill}%
\pgfsetfillopacity{0.700000}%
\pgfsetlinewidth{0.501875pt}%
\definecolor{currentstroke}{rgb}{1.000000,1.000000,1.000000}%
\pgfsetstrokecolor{currentstroke}%
\pgfsetstrokeopacity{0.500000}%
\pgfsetdash{}{0pt}%
\pgfpathmoveto{\pgfqpoint{4.131262in}{3.002419in}}%
\pgfpathlineto{\pgfqpoint{4.142243in}{3.005865in}}%
\pgfpathlineto{\pgfqpoint{4.153218in}{3.009306in}}%
\pgfpathlineto{\pgfqpoint{4.164188in}{3.012743in}}%
\pgfpathlineto{\pgfqpoint{4.175153in}{3.016177in}}%
\pgfpathlineto{\pgfqpoint{4.186112in}{3.019608in}}%
\pgfpathlineto{\pgfqpoint{4.179725in}{3.032706in}}%
\pgfpathlineto{\pgfqpoint{4.173338in}{3.045675in}}%
\pgfpathlineto{\pgfqpoint{4.166950in}{3.058524in}}%
\pgfpathlineto{\pgfqpoint{4.160561in}{3.071263in}}%
\pgfpathlineto{\pgfqpoint{4.154173in}{3.083903in}}%
\pgfpathlineto{\pgfqpoint{4.143219in}{3.080541in}}%
\pgfpathlineto{\pgfqpoint{4.132260in}{3.077195in}}%
\pgfpathlineto{\pgfqpoint{4.121296in}{3.073865in}}%
\pgfpathlineto{\pgfqpoint{4.110328in}{3.070551in}}%
\pgfpathlineto{\pgfqpoint{4.099355in}{3.067254in}}%
\pgfpathlineto{\pgfqpoint{4.105735in}{3.054473in}}%
\pgfpathlineto{\pgfqpoint{4.112117in}{3.041610in}}%
\pgfpathlineto{\pgfqpoint{4.118499in}{3.028655in}}%
\pgfpathlineto{\pgfqpoint{4.124881in}{3.015595in}}%
\pgfpathclose%
\pgfusepath{stroke,fill}%
\end{pgfscope}%
\begin{pgfscope}%
\pgfpathrectangle{\pgfqpoint{0.887500in}{0.275000in}}{\pgfqpoint{4.225000in}{4.225000in}}%
\pgfusepath{clip}%
\pgfsetbuttcap%
\pgfsetroundjoin%
\definecolor{currentfill}{rgb}{0.119483,0.614817,0.537692}%
\pgfsetfillcolor{currentfill}%
\pgfsetfillopacity{0.700000}%
\pgfsetlinewidth{0.501875pt}%
\definecolor{currentstroke}{rgb}{1.000000,1.000000,1.000000}%
\pgfsetstrokecolor{currentstroke}%
\pgfsetstrokeopacity{0.500000}%
\pgfsetdash{}{0pt}%
\pgfpathmoveto{\pgfqpoint{1.378876in}{2.688681in}}%
\pgfpathlineto{\pgfqpoint{1.390521in}{2.692048in}}%
\pgfpathlineto{\pgfqpoint{1.402162in}{2.695404in}}%
\pgfpathlineto{\pgfqpoint{1.413797in}{2.698750in}}%
\pgfpathlineto{\pgfqpoint{1.425426in}{2.702089in}}%
\pgfpathlineto{\pgfqpoint{1.437051in}{2.705420in}}%
\pgfpathlineto{\pgfqpoint{1.431456in}{2.712937in}}%
\pgfpathlineto{\pgfqpoint{1.425865in}{2.720435in}}%
\pgfpathlineto{\pgfqpoint{1.420279in}{2.727914in}}%
\pgfpathlineto{\pgfqpoint{1.414697in}{2.735374in}}%
\pgfpathlineto{\pgfqpoint{1.409120in}{2.742813in}}%
\pgfpathlineto{\pgfqpoint{1.397510in}{2.739469in}}%
\pgfpathlineto{\pgfqpoint{1.385895in}{2.736117in}}%
\pgfpathlineto{\pgfqpoint{1.374275in}{2.732757in}}%
\pgfpathlineto{\pgfqpoint{1.362649in}{2.729389in}}%
\pgfpathlineto{\pgfqpoint{1.351018in}{2.726011in}}%
\pgfpathlineto{\pgfqpoint{1.356580in}{2.718589in}}%
\pgfpathlineto{\pgfqpoint{1.362147in}{2.711144in}}%
\pgfpathlineto{\pgfqpoint{1.367719in}{2.703678in}}%
\pgfpathlineto{\pgfqpoint{1.373295in}{2.696190in}}%
\pgfpathclose%
\pgfusepath{stroke,fill}%
\end{pgfscope}%
\begin{pgfscope}%
\pgfpathrectangle{\pgfqpoint{0.887500in}{0.275000in}}{\pgfqpoint{4.225000in}{4.225000in}}%
\pgfusepath{clip}%
\pgfsetbuttcap%
\pgfsetroundjoin%
\definecolor{currentfill}{rgb}{0.146616,0.673050,0.508936}%
\pgfsetfillcolor{currentfill}%
\pgfsetfillopacity{0.700000}%
\pgfsetlinewidth{0.501875pt}%
\definecolor{currentstroke}{rgb}{1.000000,1.000000,1.000000}%
\pgfsetstrokecolor{currentstroke}%
\pgfsetstrokeopacity{0.500000}%
\pgfsetdash{}{0pt}%
\pgfpathmoveto{\pgfqpoint{2.842569in}{2.769717in}}%
\pgfpathlineto{\pgfqpoint{2.853753in}{2.791910in}}%
\pgfpathlineto{\pgfqpoint{2.864935in}{2.815598in}}%
\pgfpathlineto{\pgfqpoint{2.876116in}{2.840755in}}%
\pgfpathlineto{\pgfqpoint{2.887299in}{2.867317in}}%
\pgfpathlineto{\pgfqpoint{2.898484in}{2.895100in}}%
\pgfpathlineto{\pgfqpoint{2.892398in}{2.896294in}}%
\pgfpathlineto{\pgfqpoint{2.886316in}{2.897984in}}%
\pgfpathlineto{\pgfqpoint{2.880237in}{2.900132in}}%
\pgfpathlineto{\pgfqpoint{2.874163in}{2.902701in}}%
\pgfpathlineto{\pgfqpoint{2.868092in}{2.905654in}}%
\pgfpathlineto{\pgfqpoint{2.856926in}{2.881692in}}%
\pgfpathlineto{\pgfqpoint{2.845761in}{2.858799in}}%
\pgfpathlineto{\pgfqpoint{2.834595in}{2.836862in}}%
\pgfpathlineto{\pgfqpoint{2.823429in}{2.815957in}}%
\pgfpathlineto{\pgfqpoint{2.812260in}{2.796220in}}%
\pgfpathlineto{\pgfqpoint{2.818321in}{2.789416in}}%
\pgfpathlineto{\pgfqpoint{2.824382in}{2.783335in}}%
\pgfpathlineto{\pgfqpoint{2.830444in}{2.778005in}}%
\pgfpathlineto{\pgfqpoint{2.836506in}{2.773456in}}%
\pgfpathclose%
\pgfusepath{stroke,fill}%
\end{pgfscope}%
\begin{pgfscope}%
\pgfpathrectangle{\pgfqpoint{0.887500in}{0.275000in}}{\pgfqpoint{4.225000in}{4.225000in}}%
\pgfusepath{clip}%
\pgfsetbuttcap%
\pgfsetroundjoin%
\definecolor{currentfill}{rgb}{0.124395,0.578002,0.548287}%
\pgfsetfillcolor{currentfill}%
\pgfsetfillopacity{0.700000}%
\pgfsetlinewidth{0.501875pt}%
\definecolor{currentstroke}{rgb}{1.000000,1.000000,1.000000}%
\pgfsetstrokecolor{currentstroke}%
\pgfsetstrokeopacity{0.500000}%
\pgfsetdash{}{0pt}%
\pgfpathmoveto{\pgfqpoint{1.926539in}{2.609269in}}%
\pgfpathlineto{\pgfqpoint{1.938055in}{2.612570in}}%
\pgfpathlineto{\pgfqpoint{1.949565in}{2.615882in}}%
\pgfpathlineto{\pgfqpoint{1.961069in}{2.619202in}}%
\pgfpathlineto{\pgfqpoint{1.972567in}{2.622529in}}%
\pgfpathlineto{\pgfqpoint{1.984060in}{2.625861in}}%
\pgfpathlineto{\pgfqpoint{1.978262in}{2.633866in}}%
\pgfpathlineto{\pgfqpoint{1.972469in}{2.641857in}}%
\pgfpathlineto{\pgfqpoint{1.966680in}{2.649834in}}%
\pgfpathlineto{\pgfqpoint{1.960895in}{2.657795in}}%
\pgfpathlineto{\pgfqpoint{1.955115in}{2.665740in}}%
\pgfpathlineto{\pgfqpoint{1.943635in}{2.662405in}}%
\pgfpathlineto{\pgfqpoint{1.932150in}{2.659075in}}%
\pgfpathlineto{\pgfqpoint{1.920658in}{2.655753in}}%
\pgfpathlineto{\pgfqpoint{1.909161in}{2.652441in}}%
\pgfpathlineto{\pgfqpoint{1.897658in}{2.649139in}}%
\pgfpathlineto{\pgfqpoint{1.903426in}{2.641197in}}%
\pgfpathlineto{\pgfqpoint{1.909198in}{2.633238in}}%
\pgfpathlineto{\pgfqpoint{1.914974in}{2.625263in}}%
\pgfpathlineto{\pgfqpoint{1.920755in}{2.617273in}}%
\pgfpathclose%
\pgfusepath{stroke,fill}%
\end{pgfscope}%
\begin{pgfscope}%
\pgfpathrectangle{\pgfqpoint{0.887500in}{0.275000in}}{\pgfqpoint{4.225000in}{4.225000in}}%
\pgfusepath{clip}%
\pgfsetbuttcap%
\pgfsetroundjoin%
\definecolor{currentfill}{rgb}{0.344074,0.780029,0.397381}%
\pgfsetfillcolor{currentfill}%
\pgfsetfillopacity{0.700000}%
\pgfsetlinewidth{0.501875pt}%
\definecolor{currentstroke}{rgb}{1.000000,1.000000,1.000000}%
\pgfsetstrokecolor{currentstroke}%
\pgfsetstrokeopacity{0.500000}%
\pgfsetdash{}{0pt}%
\pgfpathmoveto{\pgfqpoint{4.044416in}{3.051031in}}%
\pgfpathlineto{\pgfqpoint{4.055413in}{3.054240in}}%
\pgfpathlineto{\pgfqpoint{4.066406in}{3.057467in}}%
\pgfpathlineto{\pgfqpoint{4.077394in}{3.060712in}}%
\pgfpathlineto{\pgfqpoint{4.088377in}{3.063974in}}%
\pgfpathlineto{\pgfqpoint{4.099355in}{3.067254in}}%
\pgfpathlineto{\pgfqpoint{4.092975in}{3.079966in}}%
\pgfpathlineto{\pgfqpoint{4.086597in}{3.092621in}}%
\pgfpathlineto{\pgfqpoint{4.080220in}{3.105230in}}%
\pgfpathlineto{\pgfqpoint{4.073845in}{3.117805in}}%
\pgfpathlineto{\pgfqpoint{4.067473in}{3.130356in}}%
\pgfpathlineto{\pgfqpoint{4.056504in}{3.127275in}}%
\pgfpathlineto{\pgfqpoint{4.045529in}{3.124208in}}%
\pgfpathlineto{\pgfqpoint{4.034550in}{3.121157in}}%
\pgfpathlineto{\pgfqpoint{4.023566in}{3.118123in}}%
\pgfpathlineto{\pgfqpoint{4.012576in}{3.115107in}}%
\pgfpathlineto{\pgfqpoint{4.018942in}{3.102415in}}%
\pgfpathlineto{\pgfqpoint{4.025308in}{3.089666in}}%
\pgfpathlineto{\pgfqpoint{4.031676in}{3.076855in}}%
\pgfpathlineto{\pgfqpoint{4.038046in}{3.063978in}}%
\pgfpathclose%
\pgfusepath{stroke,fill}%
\end{pgfscope}%
\begin{pgfscope}%
\pgfpathrectangle{\pgfqpoint{0.887500in}{0.275000in}}{\pgfqpoint{4.225000in}{4.225000in}}%
\pgfusepath{clip}%
\pgfsetbuttcap%
\pgfsetroundjoin%
\definecolor{currentfill}{rgb}{0.709898,0.868751,0.169257}%
\pgfsetfillcolor{currentfill}%
\pgfsetfillopacity{0.700000}%
\pgfsetlinewidth{0.501875pt}%
\definecolor{currentstroke}{rgb}{1.000000,1.000000,1.000000}%
\pgfsetstrokecolor{currentstroke}%
\pgfsetstrokeopacity{0.500000}%
\pgfsetdash{}{0pt}%
\pgfpathmoveto{\pgfqpoint{3.435272in}{3.348673in}}%
\pgfpathlineto{\pgfqpoint{3.446423in}{3.352197in}}%
\pgfpathlineto{\pgfqpoint{3.457569in}{3.355756in}}%
\pgfpathlineto{\pgfqpoint{3.468709in}{3.359340in}}%
\pgfpathlineto{\pgfqpoint{3.479845in}{3.362935in}}%
\pgfpathlineto{\pgfqpoint{3.490975in}{3.366530in}}%
\pgfpathlineto{\pgfqpoint{3.484711in}{3.377819in}}%
\pgfpathlineto{\pgfqpoint{3.478450in}{3.389146in}}%
\pgfpathlineto{\pgfqpoint{3.472192in}{3.400453in}}%
\pgfpathlineto{\pgfqpoint{3.465936in}{3.411676in}}%
\pgfpathlineto{\pgfqpoint{3.459681in}{3.422752in}}%
\pgfpathlineto{\pgfqpoint{3.448556in}{3.419039in}}%
\pgfpathlineto{\pgfqpoint{3.437426in}{3.415285in}}%
\pgfpathlineto{\pgfqpoint{3.426290in}{3.411493in}}%
\pgfpathlineto{\pgfqpoint{3.415148in}{3.407667in}}%
\pgfpathlineto{\pgfqpoint{3.404001in}{3.403811in}}%
\pgfpathlineto{\pgfqpoint{3.410251in}{3.392970in}}%
\pgfpathlineto{\pgfqpoint{3.416502in}{3.381994in}}%
\pgfpathlineto{\pgfqpoint{3.422756in}{3.370926in}}%
\pgfpathlineto{\pgfqpoint{3.429013in}{3.359805in}}%
\pgfpathclose%
\pgfusepath{stroke,fill}%
\end{pgfscope}%
\begin{pgfscope}%
\pgfpathrectangle{\pgfqpoint{0.887500in}{0.275000in}}{\pgfqpoint{4.225000in}{4.225000in}}%
\pgfusepath{clip}%
\pgfsetbuttcap%
\pgfsetroundjoin%
\definecolor{currentfill}{rgb}{0.132444,0.552216,0.553018}%
\pgfsetfillcolor{currentfill}%
\pgfsetfillopacity{0.700000}%
\pgfsetlinewidth{0.501875pt}%
\definecolor{currentstroke}{rgb}{1.000000,1.000000,1.000000}%
\pgfsetstrokecolor{currentstroke}%
\pgfsetstrokeopacity{0.500000}%
\pgfsetdash{}{0pt}%
\pgfpathmoveto{\pgfqpoint{2.243795in}{2.553641in}}%
\pgfpathlineto{\pgfqpoint{2.255235in}{2.556951in}}%
\pgfpathlineto{\pgfqpoint{2.266670in}{2.560279in}}%
\pgfpathlineto{\pgfqpoint{2.278098in}{2.563634in}}%
\pgfpathlineto{\pgfqpoint{2.289519in}{2.567024in}}%
\pgfpathlineto{\pgfqpoint{2.300935in}{2.570454in}}%
\pgfpathlineto{\pgfqpoint{2.295026in}{2.578692in}}%
\pgfpathlineto{\pgfqpoint{2.289122in}{2.586926in}}%
\pgfpathlineto{\pgfqpoint{2.283221in}{2.595157in}}%
\pgfpathlineto{\pgfqpoint{2.277324in}{2.603385in}}%
\pgfpathlineto{\pgfqpoint{2.271431in}{2.611608in}}%
\pgfpathlineto{\pgfqpoint{2.260028in}{2.608189in}}%
\pgfpathlineto{\pgfqpoint{2.248618in}{2.604774in}}%
\pgfpathlineto{\pgfqpoint{2.237203in}{2.601368in}}%
\pgfpathlineto{\pgfqpoint{2.225783in}{2.597978in}}%
\pgfpathlineto{\pgfqpoint{2.214356in}{2.594610in}}%
\pgfpathlineto{\pgfqpoint{2.220236in}{2.586441in}}%
\pgfpathlineto{\pgfqpoint{2.226119in}{2.578259in}}%
\pgfpathlineto{\pgfqpoint{2.232007in}{2.570065in}}%
\pgfpathlineto{\pgfqpoint{2.237899in}{2.561859in}}%
\pgfpathclose%
\pgfusepath{stroke,fill}%
\end{pgfscope}%
\begin{pgfscope}%
\pgfpathrectangle{\pgfqpoint{0.887500in}{0.275000in}}{\pgfqpoint{4.225000in}{4.225000in}}%
\pgfusepath{clip}%
\pgfsetbuttcap%
\pgfsetroundjoin%
\definecolor{currentfill}{rgb}{0.657642,0.860219,0.203082}%
\pgfsetfillcolor{currentfill}%
\pgfsetfillopacity{0.700000}%
\pgfsetlinewidth{0.501875pt}%
\definecolor{currentstroke}{rgb}{1.000000,1.000000,1.000000}%
\pgfsetstrokecolor{currentstroke}%
\pgfsetstrokeopacity{0.500000}%
\pgfsetdash{}{0pt}%
\pgfpathmoveto{\pgfqpoint{3.522354in}{3.310706in}}%
\pgfpathlineto{\pgfqpoint{3.533488in}{3.314471in}}%
\pgfpathlineto{\pgfqpoint{3.544616in}{3.318228in}}%
\pgfpathlineto{\pgfqpoint{3.555739in}{3.321937in}}%
\pgfpathlineto{\pgfqpoint{3.566857in}{3.325575in}}%
\pgfpathlineto{\pgfqpoint{3.577968in}{3.329145in}}%
\pgfpathlineto{\pgfqpoint{3.571676in}{3.340062in}}%
\pgfpathlineto{\pgfqpoint{3.565388in}{3.350985in}}%
\pgfpathlineto{\pgfqpoint{3.559103in}{3.361944in}}%
\pgfpathlineto{\pgfqpoint{3.552822in}{3.372967in}}%
\pgfpathlineto{\pgfqpoint{3.546546in}{3.384082in}}%
\pgfpathlineto{\pgfqpoint{3.535443in}{3.380653in}}%
\pgfpathlineto{\pgfqpoint{3.524335in}{3.377183in}}%
\pgfpathlineto{\pgfqpoint{3.513221in}{3.373666in}}%
\pgfpathlineto{\pgfqpoint{3.502101in}{3.370111in}}%
\pgfpathlineto{\pgfqpoint{3.490975in}{3.366530in}}%
\pgfpathlineto{\pgfqpoint{3.497244in}{3.355295in}}%
\pgfpathlineto{\pgfqpoint{3.503516in}{3.344105in}}%
\pgfpathlineto{\pgfqpoint{3.509792in}{3.332951in}}%
\pgfpathlineto{\pgfqpoint{3.516071in}{3.321821in}}%
\pgfpathclose%
\pgfusepath{stroke,fill}%
\end{pgfscope}%
\begin{pgfscope}%
\pgfpathrectangle{\pgfqpoint{0.887500in}{0.275000in}}{\pgfqpoint{4.225000in}{4.225000in}}%
\pgfusepath{clip}%
\pgfsetbuttcap%
\pgfsetroundjoin%
\definecolor{currentfill}{rgb}{0.395174,0.797475,0.367757}%
\pgfsetfillcolor{currentfill}%
\pgfsetfillopacity{0.700000}%
\pgfsetlinewidth{0.501875pt}%
\definecolor{currentstroke}{rgb}{1.000000,1.000000,1.000000}%
\pgfsetstrokecolor{currentstroke}%
\pgfsetstrokeopacity{0.500000}%
\pgfsetdash{}{0pt}%
\pgfpathmoveto{\pgfqpoint{3.957546in}{3.099920in}}%
\pgfpathlineto{\pgfqpoint{3.968565in}{3.103022in}}%
\pgfpathlineto{\pgfqpoint{3.979576in}{3.106076in}}%
\pgfpathlineto{\pgfqpoint{3.990582in}{3.109098in}}%
\pgfpathlineto{\pgfqpoint{4.001582in}{3.112103in}}%
\pgfpathlineto{\pgfqpoint{4.012576in}{3.115107in}}%
\pgfpathlineto{\pgfqpoint{4.006213in}{3.127739in}}%
\pgfpathlineto{\pgfqpoint{3.999850in}{3.140308in}}%
\pgfpathlineto{\pgfqpoint{3.993489in}{3.152809in}}%
\pgfpathlineto{\pgfqpoint{3.987129in}{3.165238in}}%
\pgfpathlineto{\pgfqpoint{3.980771in}{3.177592in}}%
\pgfpathlineto{\pgfqpoint{3.969778in}{3.174517in}}%
\pgfpathlineto{\pgfqpoint{3.958780in}{3.171410in}}%
\pgfpathlineto{\pgfqpoint{3.947776in}{3.168262in}}%
\pgfpathlineto{\pgfqpoint{3.936765in}{3.165067in}}%
\pgfpathlineto{\pgfqpoint{3.925747in}{3.161818in}}%
\pgfpathlineto{\pgfqpoint{3.932105in}{3.149627in}}%
\pgfpathlineto{\pgfqpoint{3.938465in}{3.137355in}}%
\pgfpathlineto{\pgfqpoint{3.944825in}{3.124989in}}%
\pgfpathlineto{\pgfqpoint{3.951185in}{3.112515in}}%
\pgfpathclose%
\pgfusepath{stroke,fill}%
\end{pgfscope}%
\begin{pgfscope}%
\pgfpathrectangle{\pgfqpoint{0.887500in}{0.275000in}}{\pgfqpoint{4.225000in}{4.225000in}}%
\pgfusepath{clip}%
\pgfsetbuttcap%
\pgfsetroundjoin%
\definecolor{currentfill}{rgb}{0.449368,0.813768,0.335384}%
\pgfsetfillcolor{currentfill}%
\pgfsetfillopacity{0.700000}%
\pgfsetlinewidth{0.501875pt}%
\definecolor{currentstroke}{rgb}{1.000000,1.000000,1.000000}%
\pgfsetstrokecolor{currentstroke}%
\pgfsetstrokeopacity{0.500000}%
\pgfsetdash{}{0pt}%
\pgfpathmoveto{\pgfqpoint{3.870563in}{3.144683in}}%
\pgfpathlineto{\pgfqpoint{3.881612in}{3.148203in}}%
\pgfpathlineto{\pgfqpoint{3.892655in}{3.151691in}}%
\pgfpathlineto{\pgfqpoint{3.903693in}{3.155132in}}%
\pgfpathlineto{\pgfqpoint{3.914723in}{3.158509in}}%
\pgfpathlineto{\pgfqpoint{3.925747in}{3.161818in}}%
\pgfpathlineto{\pgfqpoint{3.919391in}{3.173942in}}%
\pgfpathlineto{\pgfqpoint{3.913036in}{3.186010in}}%
\pgfpathlineto{\pgfqpoint{3.906684in}{3.198038in}}%
\pgfpathlineto{\pgfqpoint{3.900334in}{3.210039in}}%
\pgfpathlineto{\pgfqpoint{3.893987in}{3.222025in}}%
\pgfpathlineto{\pgfqpoint{3.882966in}{3.218656in}}%
\pgfpathlineto{\pgfqpoint{3.871939in}{3.215226in}}%
\pgfpathlineto{\pgfqpoint{3.860906in}{3.211737in}}%
\pgfpathlineto{\pgfqpoint{3.849867in}{3.208211in}}%
\pgfpathlineto{\pgfqpoint{3.838822in}{3.204664in}}%
\pgfpathlineto{\pgfqpoint{3.845163in}{3.192613in}}%
\pgfpathlineto{\pgfqpoint{3.851508in}{3.180616in}}%
\pgfpathlineto{\pgfqpoint{3.857857in}{3.168647in}}%
\pgfpathlineto{\pgfqpoint{3.864208in}{3.156679in}}%
\pgfpathclose%
\pgfusepath{stroke,fill}%
\end{pgfscope}%
\begin{pgfscope}%
\pgfpathrectangle{\pgfqpoint{0.887500in}{0.275000in}}{\pgfqpoint{4.225000in}{4.225000in}}%
\pgfusepath{clip}%
\pgfsetbuttcap%
\pgfsetroundjoin%
\definecolor{currentfill}{rgb}{0.121148,0.592739,0.544641}%
\pgfsetfillcolor{currentfill}%
\pgfsetfillopacity{0.700000}%
\pgfsetlinewidth{0.501875pt}%
\definecolor{currentstroke}{rgb}{1.000000,1.000000,1.000000}%
\pgfsetstrokecolor{currentstroke}%
\pgfsetstrokeopacity{0.500000}%
\pgfsetdash{}{0pt}%
\pgfpathmoveto{\pgfqpoint{1.695830in}{2.639729in}}%
\pgfpathlineto{\pgfqpoint{1.707407in}{2.642988in}}%
\pgfpathlineto{\pgfqpoint{1.718978in}{2.646244in}}%
\pgfpathlineto{\pgfqpoint{1.730543in}{2.649497in}}%
\pgfpathlineto{\pgfqpoint{1.742103in}{2.652746in}}%
\pgfpathlineto{\pgfqpoint{1.753657in}{2.655994in}}%
\pgfpathlineto{\pgfqpoint{1.747941in}{2.663810in}}%
\pgfpathlineto{\pgfqpoint{1.742230in}{2.671607in}}%
\pgfpathlineto{\pgfqpoint{1.736523in}{2.679386in}}%
\pgfpathlineto{\pgfqpoint{1.730820in}{2.687148in}}%
\pgfpathlineto{\pgfqpoint{1.725122in}{2.694892in}}%
\pgfpathlineto{\pgfqpoint{1.713581in}{2.691632in}}%
\pgfpathlineto{\pgfqpoint{1.702035in}{2.688371in}}%
\pgfpathlineto{\pgfqpoint{1.690483in}{2.685108in}}%
\pgfpathlineto{\pgfqpoint{1.678926in}{2.681842in}}%
\pgfpathlineto{\pgfqpoint{1.667363in}{2.678574in}}%
\pgfpathlineto{\pgfqpoint{1.673048in}{2.670842in}}%
\pgfpathlineto{\pgfqpoint{1.678737in}{2.663092in}}%
\pgfpathlineto{\pgfqpoint{1.684430in}{2.655324in}}%
\pgfpathlineto{\pgfqpoint{1.690128in}{2.647536in}}%
\pgfpathclose%
\pgfusepath{stroke,fill}%
\end{pgfscope}%
\begin{pgfscope}%
\pgfpathrectangle{\pgfqpoint{0.887500in}{0.275000in}}{\pgfqpoint{4.225000in}{4.225000in}}%
\pgfusepath{clip}%
\pgfsetbuttcap%
\pgfsetroundjoin%
\definecolor{currentfill}{rgb}{0.804182,0.882046,0.114965}%
\pgfsetfillcolor{currentfill}%
\pgfsetfillopacity{0.700000}%
\pgfsetlinewidth{0.501875pt}%
\definecolor{currentstroke}{rgb}{1.000000,1.000000,1.000000}%
\pgfsetstrokecolor{currentstroke}%
\pgfsetstrokeopacity{0.500000}%
\pgfsetdash{}{0pt}%
\pgfpathmoveto{\pgfqpoint{3.117989in}{3.430702in}}%
\pgfpathlineto{\pgfqpoint{3.129217in}{3.432490in}}%
\pgfpathlineto{\pgfqpoint{3.140438in}{3.434422in}}%
\pgfpathlineto{\pgfqpoint{3.151654in}{3.436517in}}%
\pgfpathlineto{\pgfqpoint{3.162864in}{3.438792in}}%
\pgfpathlineto{\pgfqpoint{3.174069in}{3.441261in}}%
\pgfpathlineto{\pgfqpoint{3.167871in}{3.449018in}}%
\pgfpathlineto{\pgfqpoint{3.161677in}{3.456462in}}%
\pgfpathlineto{\pgfqpoint{3.155486in}{3.463585in}}%
\pgfpathlineto{\pgfqpoint{3.149298in}{3.470379in}}%
\pgfpathlineto{\pgfqpoint{3.143115in}{3.476838in}}%
\pgfpathlineto{\pgfqpoint{3.131921in}{3.474054in}}%
\pgfpathlineto{\pgfqpoint{3.120722in}{3.471453in}}%
\pgfpathlineto{\pgfqpoint{3.109517in}{3.469051in}}%
\pgfpathlineto{\pgfqpoint{3.098307in}{3.466863in}}%
\pgfpathlineto{\pgfqpoint{3.087091in}{3.464903in}}%
\pgfpathlineto{\pgfqpoint{3.093262in}{3.459005in}}%
\pgfpathlineto{\pgfqpoint{3.099438in}{3.452605in}}%
\pgfpathlineto{\pgfqpoint{3.105618in}{3.445734in}}%
\pgfpathlineto{\pgfqpoint{3.111802in}{3.438423in}}%
\pgfpathclose%
\pgfusepath{stroke,fill}%
\end{pgfscope}%
\begin{pgfscope}%
\pgfpathrectangle{\pgfqpoint{0.887500in}{0.275000in}}{\pgfqpoint{4.225000in}{4.225000in}}%
\pgfusepath{clip}%
\pgfsetbuttcap%
\pgfsetroundjoin%
\definecolor{currentfill}{rgb}{0.506271,0.828786,0.300362}%
\pgfsetfillcolor{currentfill}%
\pgfsetfillopacity{0.700000}%
\pgfsetlinewidth{0.501875pt}%
\definecolor{currentstroke}{rgb}{1.000000,1.000000,1.000000}%
\pgfsetstrokecolor{currentstroke}%
\pgfsetstrokeopacity{0.500000}%
\pgfsetdash{}{0pt}%
\pgfpathmoveto{\pgfqpoint{3.783528in}{3.187283in}}%
\pgfpathlineto{\pgfqpoint{3.794595in}{3.190661in}}%
\pgfpathlineto{\pgfqpoint{3.805658in}{3.194096in}}%
\pgfpathlineto{\pgfqpoint{3.816717in}{3.197588in}}%
\pgfpathlineto{\pgfqpoint{3.827772in}{3.201116in}}%
\pgfpathlineto{\pgfqpoint{3.838822in}{3.204664in}}%
\pgfpathlineto{\pgfqpoint{3.832485in}{3.216778in}}%
\pgfpathlineto{\pgfqpoint{3.826152in}{3.228931in}}%
\pgfpathlineto{\pgfqpoint{3.819822in}{3.241094in}}%
\pgfpathlineto{\pgfqpoint{3.813495in}{3.253237in}}%
\pgfpathlineto{\pgfqpoint{3.807170in}{3.265333in}}%
\pgfpathlineto{\pgfqpoint{3.796127in}{3.261923in}}%
\pgfpathlineto{\pgfqpoint{3.785080in}{3.258503in}}%
\pgfpathlineto{\pgfqpoint{3.774027in}{3.255094in}}%
\pgfpathlineto{\pgfqpoint{3.762969in}{3.251713in}}%
\pgfpathlineto{\pgfqpoint{3.751907in}{3.248366in}}%
\pgfpathlineto{\pgfqpoint{3.758226in}{3.236166in}}%
\pgfpathlineto{\pgfqpoint{3.764546in}{3.223927in}}%
\pgfpathlineto{\pgfqpoint{3.770870in}{3.211680in}}%
\pgfpathlineto{\pgfqpoint{3.777197in}{3.199455in}}%
\pgfpathclose%
\pgfusepath{stroke,fill}%
\end{pgfscope}%
\begin{pgfscope}%
\pgfpathrectangle{\pgfqpoint{0.887500in}{0.275000in}}{\pgfqpoint{4.225000in}{4.225000in}}%
\pgfusepath{clip}%
\pgfsetbuttcap%
\pgfsetroundjoin%
\definecolor{currentfill}{rgb}{0.616293,0.852709,0.230052}%
\pgfsetfillcolor{currentfill}%
\pgfsetfillopacity{0.700000}%
\pgfsetlinewidth{0.501875pt}%
\definecolor{currentstroke}{rgb}{1.000000,1.000000,1.000000}%
\pgfsetstrokecolor{currentstroke}%
\pgfsetstrokeopacity{0.500000}%
\pgfsetdash{}{0pt}%
\pgfpathmoveto{\pgfqpoint{3.609465in}{3.273675in}}%
\pgfpathlineto{\pgfqpoint{3.620577in}{3.277230in}}%
\pgfpathlineto{\pgfqpoint{3.631683in}{3.280731in}}%
\pgfpathlineto{\pgfqpoint{3.642784in}{3.284189in}}%
\pgfpathlineto{\pgfqpoint{3.653879in}{3.287611in}}%
\pgfpathlineto{\pgfqpoint{3.664968in}{3.291008in}}%
\pgfpathlineto{\pgfqpoint{3.658659in}{3.302348in}}%
\pgfpathlineto{\pgfqpoint{3.652351in}{3.313513in}}%
\pgfpathlineto{\pgfqpoint{3.646045in}{3.324538in}}%
\pgfpathlineto{\pgfqpoint{3.639740in}{3.335459in}}%
\pgfpathlineto{\pgfqpoint{3.633437in}{3.346312in}}%
\pgfpathlineto{\pgfqpoint{3.622354in}{3.342939in}}%
\pgfpathlineto{\pgfqpoint{3.611266in}{3.339545in}}%
\pgfpathlineto{\pgfqpoint{3.600173in}{3.336121in}}%
\pgfpathlineto{\pgfqpoint{3.589073in}{3.332658in}}%
\pgfpathlineto{\pgfqpoint{3.577968in}{3.329145in}}%
\pgfpathlineto{\pgfqpoint{3.584263in}{3.318207in}}%
\pgfpathlineto{\pgfqpoint{3.590560in}{3.307220in}}%
\pgfpathlineto{\pgfqpoint{3.596860in}{3.296154in}}%
\pgfpathlineto{\pgfqpoint{3.603162in}{3.284982in}}%
\pgfpathclose%
\pgfusepath{stroke,fill}%
\end{pgfscope}%
\begin{pgfscope}%
\pgfpathrectangle{\pgfqpoint{0.887500in}{0.275000in}}{\pgfqpoint{4.225000in}{4.225000in}}%
\pgfusepath{clip}%
\pgfsetbuttcap%
\pgfsetroundjoin%
\definecolor{currentfill}{rgb}{0.565498,0.842430,0.262877}%
\pgfsetfillcolor{currentfill}%
\pgfsetfillopacity{0.700000}%
\pgfsetlinewidth{0.501875pt}%
\definecolor{currentstroke}{rgb}{1.000000,1.000000,1.000000}%
\pgfsetstrokecolor{currentstroke}%
\pgfsetstrokeopacity{0.500000}%
\pgfsetdash{}{0pt}%
\pgfpathmoveto{\pgfqpoint{3.696521in}{3.231855in}}%
\pgfpathlineto{\pgfqpoint{3.707609in}{3.235154in}}%
\pgfpathlineto{\pgfqpoint{3.718691in}{3.238447in}}%
\pgfpathlineto{\pgfqpoint{3.729768in}{3.241741in}}%
\pgfpathlineto{\pgfqpoint{3.740840in}{3.245045in}}%
\pgfpathlineto{\pgfqpoint{3.751907in}{3.248366in}}%
\pgfpathlineto{\pgfqpoint{3.745590in}{3.260498in}}%
\pgfpathlineto{\pgfqpoint{3.739275in}{3.272530in}}%
\pgfpathlineto{\pgfqpoint{3.732960in}{3.284433in}}%
\pgfpathlineto{\pgfqpoint{3.726646in}{3.296176in}}%
\pgfpathlineto{\pgfqpoint{3.720332in}{3.307728in}}%
\pgfpathlineto{\pgfqpoint{3.709270in}{3.304414in}}%
\pgfpathlineto{\pgfqpoint{3.698203in}{3.301086in}}%
\pgfpathlineto{\pgfqpoint{3.687130in}{3.297743in}}%
\pgfpathlineto{\pgfqpoint{3.676052in}{3.294384in}}%
\pgfpathlineto{\pgfqpoint{3.664968in}{3.291008in}}%
\pgfpathlineto{\pgfqpoint{3.671277in}{3.279474in}}%
\pgfpathlineto{\pgfqpoint{3.677586in}{3.267767in}}%
\pgfpathlineto{\pgfqpoint{3.683897in}{3.255911in}}%
\pgfpathlineto{\pgfqpoint{3.690208in}{3.243932in}}%
\pgfpathclose%
\pgfusepath{stroke,fill}%
\end{pgfscope}%
\begin{pgfscope}%
\pgfpathrectangle{\pgfqpoint{0.887500in}{0.275000in}}{\pgfqpoint{4.225000in}{4.225000in}}%
\pgfusepath{clip}%
\pgfsetbuttcap%
\pgfsetroundjoin%
\definecolor{currentfill}{rgb}{0.141935,0.526453,0.555991}%
\pgfsetfillcolor{currentfill}%
\pgfsetfillopacity{0.700000}%
\pgfsetlinewidth{0.501875pt}%
\definecolor{currentstroke}{rgb}{1.000000,1.000000,1.000000}%
\pgfsetstrokecolor{currentstroke}%
\pgfsetstrokeopacity{0.500000}%
\pgfsetdash{}{0pt}%
\pgfpathmoveto{\pgfqpoint{2.561157in}{2.496130in}}%
\pgfpathlineto{\pgfqpoint{2.572509in}{2.500389in}}%
\pgfpathlineto{\pgfqpoint{2.583851in}{2.505145in}}%
\pgfpathlineto{\pgfqpoint{2.595180in}{2.510466in}}%
\pgfpathlineto{\pgfqpoint{2.606499in}{2.516420in}}%
\pgfpathlineto{\pgfqpoint{2.617805in}{2.523073in}}%
\pgfpathlineto{\pgfqpoint{2.611786in}{2.531741in}}%
\pgfpathlineto{\pgfqpoint{2.605771in}{2.540320in}}%
\pgfpathlineto{\pgfqpoint{2.599761in}{2.548799in}}%
\pgfpathlineto{\pgfqpoint{2.593756in}{2.557165in}}%
\pgfpathlineto{\pgfqpoint{2.587757in}{2.565406in}}%
\pgfpathlineto{\pgfqpoint{2.576467in}{2.558483in}}%
\pgfpathlineto{\pgfqpoint{2.565164in}{2.552395in}}%
\pgfpathlineto{\pgfqpoint{2.553847in}{2.547034in}}%
\pgfpathlineto{\pgfqpoint{2.542519in}{2.542294in}}%
\pgfpathlineto{\pgfqpoint{2.531178in}{2.538066in}}%
\pgfpathlineto{\pgfqpoint{2.537164in}{2.529826in}}%
\pgfpathlineto{\pgfqpoint{2.543155in}{2.521500in}}%
\pgfpathlineto{\pgfqpoint{2.549151in}{2.513101in}}%
\pgfpathlineto{\pgfqpoint{2.555152in}{2.504640in}}%
\pgfpathclose%
\pgfusepath{stroke,fill}%
\end{pgfscope}%
\begin{pgfscope}%
\pgfpathrectangle{\pgfqpoint{0.887500in}{0.275000in}}{\pgfqpoint{4.225000in}{4.225000in}}%
\pgfusepath{clip}%
\pgfsetbuttcap%
\pgfsetroundjoin%
\definecolor{currentfill}{rgb}{0.127568,0.566949,0.550556}%
\pgfsetfillcolor{currentfill}%
\pgfsetfillopacity{0.700000}%
\pgfsetlinewidth{0.501875pt}%
\definecolor{currentstroke}{rgb}{1.000000,1.000000,1.000000}%
\pgfsetstrokecolor{currentstroke}%
\pgfsetstrokeopacity{0.500000}%
\pgfsetdash{}{0pt}%
\pgfpathmoveto{\pgfqpoint{2.013107in}{2.585657in}}%
\pgfpathlineto{\pgfqpoint{2.024607in}{2.588987in}}%
\pgfpathlineto{\pgfqpoint{2.036101in}{2.592315in}}%
\pgfpathlineto{\pgfqpoint{2.047589in}{2.595639in}}%
\pgfpathlineto{\pgfqpoint{2.059072in}{2.598957in}}%
\pgfpathlineto{\pgfqpoint{2.070549in}{2.602266in}}%
\pgfpathlineto{\pgfqpoint{2.064719in}{2.610338in}}%
\pgfpathlineto{\pgfqpoint{2.058893in}{2.618398in}}%
\pgfpathlineto{\pgfqpoint{2.053070in}{2.626446in}}%
\pgfpathlineto{\pgfqpoint{2.047252in}{2.634482in}}%
\pgfpathlineto{\pgfqpoint{2.041438in}{2.642506in}}%
\pgfpathlineto{\pgfqpoint{2.029973in}{2.639189in}}%
\pgfpathlineto{\pgfqpoint{2.018503in}{2.635863in}}%
\pgfpathlineto{\pgfqpoint{2.007028in}{2.632530in}}%
\pgfpathlineto{\pgfqpoint{1.995546in}{2.629196in}}%
\pgfpathlineto{\pgfqpoint{1.984060in}{2.625861in}}%
\pgfpathlineto{\pgfqpoint{1.989861in}{2.617843in}}%
\pgfpathlineto{\pgfqpoint{1.995667in}{2.609813in}}%
\pgfpathlineto{\pgfqpoint{2.001476in}{2.601772in}}%
\pgfpathlineto{\pgfqpoint{2.007290in}{2.593720in}}%
\pgfpathclose%
\pgfusepath{stroke,fill}%
\end{pgfscope}%
\begin{pgfscope}%
\pgfpathrectangle{\pgfqpoint{0.887500in}{0.275000in}}{\pgfqpoint{4.225000in}{4.225000in}}%
\pgfusepath{clip}%
\pgfsetbuttcap%
\pgfsetroundjoin%
\definecolor{currentfill}{rgb}{0.119512,0.607464,0.540218}%
\pgfsetfillcolor{currentfill}%
\pgfsetfillopacity{0.700000}%
\pgfsetlinewidth{0.501875pt}%
\definecolor{currentstroke}{rgb}{1.000000,1.000000,1.000000}%
\pgfsetstrokecolor{currentstroke}%
\pgfsetstrokeopacity{0.500000}%
\pgfsetdash{}{0pt}%
\pgfpathmoveto{\pgfqpoint{1.465092in}{2.667567in}}%
\pgfpathlineto{\pgfqpoint{1.476725in}{2.670884in}}%
\pgfpathlineto{\pgfqpoint{1.488353in}{2.674196in}}%
\pgfpathlineto{\pgfqpoint{1.499975in}{2.677502in}}%
\pgfpathlineto{\pgfqpoint{1.511592in}{2.680803in}}%
\pgfpathlineto{\pgfqpoint{1.523203in}{2.684100in}}%
\pgfpathlineto{\pgfqpoint{1.517571in}{2.691712in}}%
\pgfpathlineto{\pgfqpoint{1.511945in}{2.699308in}}%
\pgfpathlineto{\pgfqpoint{1.506322in}{2.706887in}}%
\pgfpathlineto{\pgfqpoint{1.500704in}{2.714450in}}%
\pgfpathlineto{\pgfqpoint{1.495090in}{2.721995in}}%
\pgfpathlineto{\pgfqpoint{1.483493in}{2.718688in}}%
\pgfpathlineto{\pgfqpoint{1.471891in}{2.715377in}}%
\pgfpathlineto{\pgfqpoint{1.460283in}{2.712063in}}%
\pgfpathlineto{\pgfqpoint{1.448670in}{2.708744in}}%
\pgfpathlineto{\pgfqpoint{1.437051in}{2.705420in}}%
\pgfpathlineto{\pgfqpoint{1.442650in}{2.697884in}}%
\pgfpathlineto{\pgfqpoint{1.448254in}{2.690331in}}%
\pgfpathlineto{\pgfqpoint{1.453862in}{2.682760in}}%
\pgfpathlineto{\pgfqpoint{1.459475in}{2.675172in}}%
\pgfpathclose%
\pgfusepath{stroke,fill}%
\end{pgfscope}%
\begin{pgfscope}%
\pgfpathrectangle{\pgfqpoint{0.887500in}{0.275000in}}{\pgfqpoint{4.225000in}{4.225000in}}%
\pgfusepath{clip}%
\pgfsetbuttcap%
\pgfsetroundjoin%
\definecolor{currentfill}{rgb}{0.793760,0.880678,0.120005}%
\pgfsetfillcolor{currentfill}%
\pgfsetfillopacity{0.700000}%
\pgfsetlinewidth{0.501875pt}%
\definecolor{currentstroke}{rgb}{1.000000,1.000000,1.000000}%
\pgfsetstrokecolor{currentstroke}%
\pgfsetstrokeopacity{0.500000}%
\pgfsetdash{}{0pt}%
\pgfpathmoveto{\pgfqpoint{2.974687in}{3.399780in}}%
\pgfpathlineto{\pgfqpoint{2.985927in}{3.415720in}}%
\pgfpathlineto{\pgfqpoint{2.997174in}{3.428157in}}%
\pgfpathlineto{\pgfqpoint{3.008424in}{3.437755in}}%
\pgfpathlineto{\pgfqpoint{3.019674in}{3.445175in}}%
\pgfpathlineto{\pgfqpoint{3.030922in}{3.450907in}}%
\pgfpathlineto{\pgfqpoint{3.024768in}{3.456484in}}%
\pgfpathlineto{\pgfqpoint{3.018621in}{3.461594in}}%
\pgfpathlineto{\pgfqpoint{3.012478in}{3.466246in}}%
\pgfpathlineto{\pgfqpoint{3.006341in}{3.470495in}}%
\pgfpathlineto{\pgfqpoint{3.000210in}{3.474397in}}%
\pgfpathlineto{\pgfqpoint{2.988978in}{3.469358in}}%
\pgfpathlineto{\pgfqpoint{2.977746in}{3.462510in}}%
\pgfpathlineto{\pgfqpoint{2.966516in}{3.453235in}}%
\pgfpathlineto{\pgfqpoint{2.955291in}{3.440709in}}%
\pgfpathlineto{\pgfqpoint{2.944078in}{3.424104in}}%
\pgfpathlineto{\pgfqpoint{2.950188in}{3.419769in}}%
\pgfpathlineto{\pgfqpoint{2.956305in}{3.415217in}}%
\pgfpathlineto{\pgfqpoint{2.962426in}{3.410396in}}%
\pgfpathlineto{\pgfqpoint{2.968554in}{3.405255in}}%
\pgfpathclose%
\pgfusepath{stroke,fill}%
\end{pgfscope}%
\begin{pgfscope}%
\pgfpathrectangle{\pgfqpoint{0.887500in}{0.275000in}}{\pgfqpoint{4.225000in}{4.225000in}}%
\pgfusepath{clip}%
\pgfsetbuttcap%
\pgfsetroundjoin%
\definecolor{currentfill}{rgb}{0.136408,0.541173,0.554483}%
\pgfsetfillcolor{currentfill}%
\pgfsetfillopacity{0.700000}%
\pgfsetlinewidth{0.501875pt}%
\definecolor{currentstroke}{rgb}{1.000000,1.000000,1.000000}%
\pgfsetstrokecolor{currentstroke}%
\pgfsetstrokeopacity{0.500000}%
\pgfsetdash{}{0pt}%
\pgfpathmoveto{\pgfqpoint{2.330536in}{2.529167in}}%
\pgfpathlineto{\pgfqpoint{2.341957in}{2.532644in}}%
\pgfpathlineto{\pgfqpoint{2.353372in}{2.536175in}}%
\pgfpathlineto{\pgfqpoint{2.364780in}{2.539754in}}%
\pgfpathlineto{\pgfqpoint{2.376183in}{2.543341in}}%
\pgfpathlineto{\pgfqpoint{2.387581in}{2.546897in}}%
\pgfpathlineto{\pgfqpoint{2.381640in}{2.555216in}}%
\pgfpathlineto{\pgfqpoint{2.375703in}{2.563511in}}%
\pgfpathlineto{\pgfqpoint{2.369770in}{2.571778in}}%
\pgfpathlineto{\pgfqpoint{2.363842in}{2.580014in}}%
\pgfpathlineto{\pgfqpoint{2.357919in}{2.588214in}}%
\pgfpathlineto{\pgfqpoint{2.346533in}{2.584651in}}%
\pgfpathlineto{\pgfqpoint{2.335142in}{2.581056in}}%
\pgfpathlineto{\pgfqpoint{2.323746in}{2.577471in}}%
\pgfpathlineto{\pgfqpoint{2.312344in}{2.573934in}}%
\pgfpathlineto{\pgfqpoint{2.300935in}{2.570454in}}%
\pgfpathlineto{\pgfqpoint{2.306847in}{2.562212in}}%
\pgfpathlineto{\pgfqpoint{2.312763in}{2.553964in}}%
\pgfpathlineto{\pgfqpoint{2.318684in}{2.545708in}}%
\pgfpathlineto{\pgfqpoint{2.324608in}{2.537443in}}%
\pgfpathclose%
\pgfusepath{stroke,fill}%
\end{pgfscope}%
\begin{pgfscope}%
\pgfpathrectangle{\pgfqpoint{0.887500in}{0.275000in}}{\pgfqpoint{4.225000in}{4.225000in}}%
\pgfusepath{clip}%
\pgfsetbuttcap%
\pgfsetroundjoin%
\definecolor{currentfill}{rgb}{0.772852,0.877868,0.131109}%
\pgfsetfillcolor{currentfill}%
\pgfsetfillopacity{0.700000}%
\pgfsetlinewidth{0.501875pt}%
\definecolor{currentstroke}{rgb}{1.000000,1.000000,1.000000}%
\pgfsetstrokecolor{currentstroke}%
\pgfsetstrokeopacity{0.500000}%
\pgfsetdash{}{0pt}%
\pgfpathmoveto{\pgfqpoint{3.205105in}{3.398042in}}%
\pgfpathlineto{\pgfqpoint{3.216315in}{3.400867in}}%
\pgfpathlineto{\pgfqpoint{3.227520in}{3.403912in}}%
\pgfpathlineto{\pgfqpoint{3.238721in}{3.407178in}}%
\pgfpathlineto{\pgfqpoint{3.249917in}{3.410647in}}%
\pgfpathlineto{\pgfqpoint{3.261109in}{3.414291in}}%
\pgfpathlineto{\pgfqpoint{3.254886in}{3.423417in}}%
\pgfpathlineto{\pgfqpoint{3.248666in}{3.432239in}}%
\pgfpathlineto{\pgfqpoint{3.242448in}{3.440757in}}%
\pgfpathlineto{\pgfqpoint{3.236233in}{3.448982in}}%
\pgfpathlineto{\pgfqpoint{3.230021in}{3.456930in}}%
\pgfpathlineto{\pgfqpoint{3.218840in}{3.453367in}}%
\pgfpathlineto{\pgfqpoint{3.207654in}{3.449996in}}%
\pgfpathlineto{\pgfqpoint{3.196464in}{3.446849in}}%
\pgfpathlineto{\pgfqpoint{3.185269in}{3.443942in}}%
\pgfpathlineto{\pgfqpoint{3.174069in}{3.441261in}}%
\pgfpathlineto{\pgfqpoint{3.180270in}{3.433198in}}%
\pgfpathlineto{\pgfqpoint{3.186474in}{3.424836in}}%
\pgfpathlineto{\pgfqpoint{3.192682in}{3.416183in}}%
\pgfpathlineto{\pgfqpoint{3.198892in}{3.407247in}}%
\pgfpathclose%
\pgfusepath{stroke,fill}%
\end{pgfscope}%
\begin{pgfscope}%
\pgfpathrectangle{\pgfqpoint{0.887500in}{0.275000in}}{\pgfqpoint{4.225000in}{4.225000in}}%
\pgfusepath{clip}%
\pgfsetbuttcap%
\pgfsetroundjoin%
\definecolor{currentfill}{rgb}{0.124395,0.578002,0.548287}%
\pgfsetfillcolor{currentfill}%
\pgfsetfillopacity{0.700000}%
\pgfsetlinewidth{0.501875pt}%
\definecolor{currentstroke}{rgb}{1.000000,1.000000,1.000000}%
\pgfsetstrokecolor{currentstroke}%
\pgfsetstrokeopacity{0.500000}%
\pgfsetdash{}{0pt}%
\pgfpathmoveto{\pgfqpoint{2.760815in}{2.565922in}}%
\pgfpathlineto{\pgfqpoint{2.772039in}{2.581127in}}%
\pgfpathlineto{\pgfqpoint{2.783255in}{2.597760in}}%
\pgfpathlineto{\pgfqpoint{2.794464in}{2.615647in}}%
\pgfpathlineto{\pgfqpoint{2.805670in}{2.634613in}}%
\pgfpathlineto{\pgfqpoint{2.816873in}{2.654485in}}%
\pgfpathlineto{\pgfqpoint{2.810808in}{2.659228in}}%
\pgfpathlineto{\pgfqpoint{2.804744in}{2.664343in}}%
\pgfpathlineto{\pgfqpoint{2.798684in}{2.669841in}}%
\pgfpathlineto{\pgfqpoint{2.792625in}{2.675730in}}%
\pgfpathlineto{\pgfqpoint{2.786569in}{2.682015in}}%
\pgfpathlineto{\pgfqpoint{2.775345in}{2.668934in}}%
\pgfpathlineto{\pgfqpoint{2.764113in}{2.656875in}}%
\pgfpathlineto{\pgfqpoint{2.752876in}{2.645549in}}%
\pgfpathlineto{\pgfqpoint{2.741634in}{2.634666in}}%
\pgfpathlineto{\pgfqpoint{2.730390in}{2.623937in}}%
\pgfpathlineto{\pgfqpoint{2.736461in}{2.613359in}}%
\pgfpathlineto{\pgfqpoint{2.742538in}{2.602320in}}%
\pgfpathlineto{\pgfqpoint{2.748623in}{2.590767in}}%
\pgfpathlineto{\pgfqpoint{2.754715in}{2.578648in}}%
\pgfpathclose%
\pgfusepath{stroke,fill}%
\end{pgfscope}%
\begin{pgfscope}%
\pgfpathrectangle{\pgfqpoint{0.887500in}{0.275000in}}{\pgfqpoint{4.225000in}{4.225000in}}%
\pgfusepath{clip}%
\pgfsetbuttcap%
\pgfsetroundjoin%
\definecolor{currentfill}{rgb}{0.133743,0.548535,0.553541}%
\pgfsetfillcolor{currentfill}%
\pgfsetfillopacity{0.700000}%
\pgfsetlinewidth{0.501875pt}%
\definecolor{currentstroke}{rgb}{1.000000,1.000000,1.000000}%
\pgfsetstrokecolor{currentstroke}%
\pgfsetstrokeopacity{0.500000}%
\pgfsetdash{}{0pt}%
\pgfpathmoveto{\pgfqpoint{2.704512in}{2.512396in}}%
\pgfpathlineto{\pgfqpoint{2.715796in}{2.520641in}}%
\pgfpathlineto{\pgfqpoint{2.727069in}{2.529891in}}%
\pgfpathlineto{\pgfqpoint{2.738331in}{2.540373in}}%
\pgfpathlineto{\pgfqpoint{2.749579in}{2.552318in}}%
\pgfpathlineto{\pgfqpoint{2.760815in}{2.565922in}}%
\pgfpathlineto{\pgfqpoint{2.754715in}{2.578648in}}%
\pgfpathlineto{\pgfqpoint{2.748623in}{2.590767in}}%
\pgfpathlineto{\pgfqpoint{2.742538in}{2.602320in}}%
\pgfpathlineto{\pgfqpoint{2.736461in}{2.613359in}}%
\pgfpathlineto{\pgfqpoint{2.730390in}{2.623937in}}%
\pgfpathlineto{\pgfqpoint{2.719145in}{2.613074in}}%
\pgfpathlineto{\pgfqpoint{2.707902in}{2.601925in}}%
\pgfpathlineto{\pgfqpoint{2.696659in}{2.590640in}}%
\pgfpathlineto{\pgfqpoint{2.685415in}{2.579406in}}%
\pgfpathlineto{\pgfqpoint{2.674167in}{2.568413in}}%
\pgfpathlineto{\pgfqpoint{2.680222in}{2.558069in}}%
\pgfpathlineto{\pgfqpoint{2.686284in}{2.547319in}}%
\pgfpathlineto{\pgfqpoint{2.692353in}{2.536140in}}%
\pgfpathlineto{\pgfqpoint{2.698429in}{2.524505in}}%
\pgfpathclose%
\pgfusepath{stroke,fill}%
\end{pgfscope}%
\begin{pgfscope}%
\pgfpathrectangle{\pgfqpoint{0.887500in}{0.275000in}}{\pgfqpoint{4.225000in}{4.225000in}}%
\pgfusepath{clip}%
\pgfsetbuttcap%
\pgfsetroundjoin%
\definecolor{currentfill}{rgb}{0.122606,0.585371,0.546557}%
\pgfsetfillcolor{currentfill}%
\pgfsetfillopacity{0.700000}%
\pgfsetlinewidth{0.501875pt}%
\definecolor{currentstroke}{rgb}{1.000000,1.000000,1.000000}%
\pgfsetstrokecolor{currentstroke}%
\pgfsetstrokeopacity{0.500000}%
\pgfsetdash{}{0pt}%
\pgfpathmoveto{\pgfqpoint{1.782301in}{2.616610in}}%
\pgfpathlineto{\pgfqpoint{1.793863in}{2.619849in}}%
\pgfpathlineto{\pgfqpoint{1.805419in}{2.623087in}}%
\pgfpathlineto{\pgfqpoint{1.816970in}{2.626324in}}%
\pgfpathlineto{\pgfqpoint{1.828514in}{2.629564in}}%
\pgfpathlineto{\pgfqpoint{1.840053in}{2.632806in}}%
\pgfpathlineto{\pgfqpoint{1.834302in}{2.640728in}}%
\pgfpathlineto{\pgfqpoint{1.828556in}{2.648631in}}%
\pgfpathlineto{\pgfqpoint{1.822814in}{2.656515in}}%
\pgfpathlineto{\pgfqpoint{1.817076in}{2.664379in}}%
\pgfpathlineto{\pgfqpoint{1.811343in}{2.672224in}}%
\pgfpathlineto{\pgfqpoint{1.799817in}{2.668976in}}%
\pgfpathlineto{\pgfqpoint{1.788286in}{2.665730in}}%
\pgfpathlineto{\pgfqpoint{1.776748in}{2.662485in}}%
\pgfpathlineto{\pgfqpoint{1.765206in}{2.659240in}}%
\pgfpathlineto{\pgfqpoint{1.753657in}{2.655994in}}%
\pgfpathlineto{\pgfqpoint{1.759377in}{2.648158in}}%
\pgfpathlineto{\pgfqpoint{1.765102in}{2.640302in}}%
\pgfpathlineto{\pgfqpoint{1.770830in}{2.632425in}}%
\pgfpathlineto{\pgfqpoint{1.776564in}{2.624527in}}%
\pgfpathclose%
\pgfusepath{stroke,fill}%
\end{pgfscope}%
\begin{pgfscope}%
\pgfpathrectangle{\pgfqpoint{0.887500in}{0.275000in}}{\pgfqpoint{4.225000in}{4.225000in}}%
\pgfusepath{clip}%
\pgfsetbuttcap%
\pgfsetroundjoin%
\definecolor{currentfill}{rgb}{0.202219,0.715272,0.476084}%
\pgfsetfillcolor{currentfill}%
\pgfsetfillopacity{0.700000}%
\pgfsetlinewidth{0.501875pt}%
\definecolor{currentstroke}{rgb}{1.000000,1.000000,1.000000}%
\pgfsetstrokecolor{currentstroke}%
\pgfsetstrokeopacity{0.500000}%
\pgfsetdash{}{0pt}%
\pgfpathmoveto{\pgfqpoint{4.249943in}{2.882211in}}%
\pgfpathlineto{\pgfqpoint{4.260903in}{2.885716in}}%
\pgfpathlineto{\pgfqpoint{4.271860in}{2.889227in}}%
\pgfpathlineto{\pgfqpoint{4.282811in}{2.892746in}}%
\pgfpathlineto{\pgfqpoint{4.293757in}{2.896275in}}%
\pgfpathlineto{\pgfqpoint{4.287369in}{2.910297in}}%
\pgfpathlineto{\pgfqpoint{4.280983in}{2.924283in}}%
\pgfpathlineto{\pgfqpoint{4.274597in}{2.938220in}}%
\pgfpathlineto{\pgfqpoint{4.268213in}{2.952094in}}%
\pgfpathlineto{\pgfqpoint{4.261828in}{2.965894in}}%
\pgfpathlineto{\pgfqpoint{4.250886in}{2.962438in}}%
\pgfpathlineto{\pgfqpoint{4.239939in}{2.958979in}}%
\pgfpathlineto{\pgfqpoint{4.228986in}{2.955518in}}%
\pgfpathlineto{\pgfqpoint{4.218028in}{2.952053in}}%
\pgfpathlineto{\pgfqpoint{4.224410in}{2.938204in}}%
\pgfpathlineto{\pgfqpoint{4.230791in}{2.924280in}}%
\pgfpathlineto{\pgfqpoint{4.237174in}{2.910296in}}%
\pgfpathlineto{\pgfqpoint{4.243557in}{2.896268in}}%
\pgfpathclose%
\pgfusepath{stroke,fill}%
\end{pgfscope}%
\begin{pgfscope}%
\pgfpathrectangle{\pgfqpoint{0.887500in}{0.275000in}}{\pgfqpoint{4.225000in}{4.225000in}}%
\pgfusepath{clip}%
\pgfsetbuttcap%
\pgfsetroundjoin%
\definecolor{currentfill}{rgb}{0.129933,0.559582,0.551864}%
\pgfsetfillcolor{currentfill}%
\pgfsetfillopacity{0.700000}%
\pgfsetlinewidth{0.501875pt}%
\definecolor{currentstroke}{rgb}{1.000000,1.000000,1.000000}%
\pgfsetstrokecolor{currentstroke}%
\pgfsetstrokeopacity{0.500000}%
\pgfsetdash{}{0pt}%
\pgfpathmoveto{\pgfqpoint{2.099762in}{2.561719in}}%
\pgfpathlineto{\pgfqpoint{2.111247in}{2.565011in}}%
\pgfpathlineto{\pgfqpoint{2.122726in}{2.568295in}}%
\pgfpathlineto{\pgfqpoint{2.134200in}{2.571572in}}%
\pgfpathlineto{\pgfqpoint{2.145668in}{2.574843in}}%
\pgfpathlineto{\pgfqpoint{2.157131in}{2.578111in}}%
\pgfpathlineto{\pgfqpoint{2.151268in}{2.586245in}}%
\pgfpathlineto{\pgfqpoint{2.145409in}{2.594364in}}%
\pgfpathlineto{\pgfqpoint{2.139554in}{2.602470in}}%
\pgfpathlineto{\pgfqpoint{2.133703in}{2.610563in}}%
\pgfpathlineto{\pgfqpoint{2.127856in}{2.618643in}}%
\pgfpathlineto{\pgfqpoint{2.116406in}{2.615387in}}%
\pgfpathlineto{\pgfqpoint{2.104950in}{2.612127in}}%
\pgfpathlineto{\pgfqpoint{2.093488in}{2.608853in}}%
\pgfpathlineto{\pgfqpoint{2.082021in}{2.605566in}}%
\pgfpathlineto{\pgfqpoint{2.070549in}{2.602266in}}%
\pgfpathlineto{\pgfqpoint{2.076384in}{2.594183in}}%
\pgfpathlineto{\pgfqpoint{2.082222in}{2.586086in}}%
\pgfpathlineto{\pgfqpoint{2.088065in}{2.577977in}}%
\pgfpathlineto{\pgfqpoint{2.093911in}{2.569855in}}%
\pgfpathclose%
\pgfusepath{stroke,fill}%
\end{pgfscope}%
\begin{pgfscope}%
\pgfpathrectangle{\pgfqpoint{0.887500in}{0.275000in}}{\pgfqpoint{4.225000in}{4.225000in}}%
\pgfusepath{clip}%
\pgfsetbuttcap%
\pgfsetroundjoin%
\definecolor{currentfill}{rgb}{0.246070,0.738910,0.452024}%
\pgfsetfillcolor{currentfill}%
\pgfsetfillopacity{0.700000}%
\pgfsetlinewidth{0.501875pt}%
\definecolor{currentstroke}{rgb}{1.000000,1.000000,1.000000}%
\pgfsetstrokecolor{currentstroke}%
\pgfsetstrokeopacity{0.500000}%
\pgfsetdash{}{0pt}%
\pgfpathmoveto{\pgfqpoint{4.163159in}{2.934626in}}%
\pgfpathlineto{\pgfqpoint{4.174144in}{2.938128in}}%
\pgfpathlineto{\pgfqpoint{4.185123in}{2.941621in}}%
\pgfpathlineto{\pgfqpoint{4.196097in}{2.945105in}}%
\pgfpathlineto{\pgfqpoint{4.207065in}{2.948582in}}%
\pgfpathlineto{\pgfqpoint{4.218028in}{2.952053in}}%
\pgfpathlineto{\pgfqpoint{4.211647in}{2.965809in}}%
\pgfpathlineto{\pgfqpoint{4.205265in}{2.979458in}}%
\pgfpathlineto{\pgfqpoint{4.198882in}{2.992984in}}%
\pgfpathlineto{\pgfqpoint{4.192498in}{3.006370in}}%
\pgfpathlineto{\pgfqpoint{4.186112in}{3.019608in}}%
\pgfpathlineto{\pgfqpoint{4.175153in}{3.016177in}}%
\pgfpathlineto{\pgfqpoint{4.164188in}{3.012743in}}%
\pgfpathlineto{\pgfqpoint{4.153218in}{3.009306in}}%
\pgfpathlineto{\pgfqpoint{4.142243in}{3.005865in}}%
\pgfpathlineto{\pgfqpoint{4.131262in}{3.002419in}}%
\pgfpathlineto{\pgfqpoint{4.137643in}{2.989114in}}%
\pgfpathlineto{\pgfqpoint{4.144023in}{2.975672in}}%
\pgfpathlineto{\pgfqpoint{4.150402in}{2.962099in}}%
\pgfpathlineto{\pgfqpoint{4.156781in}{2.948413in}}%
\pgfpathclose%
\pgfusepath{stroke,fill}%
\end{pgfscope}%
\begin{pgfscope}%
\pgfpathrectangle{\pgfqpoint{0.887500in}{0.275000in}}{\pgfqpoint{4.225000in}{4.225000in}}%
\pgfusepath{clip}%
\pgfsetbuttcap%
\pgfsetroundjoin%
\definecolor{currentfill}{rgb}{0.120092,0.600104,0.542530}%
\pgfsetfillcolor{currentfill}%
\pgfsetfillopacity{0.700000}%
\pgfsetlinewidth{0.501875pt}%
\definecolor{currentstroke}{rgb}{1.000000,1.000000,1.000000}%
\pgfsetstrokecolor{currentstroke}%
\pgfsetstrokeopacity{0.500000}%
\pgfsetdash{}{0pt}%
\pgfpathmoveto{\pgfqpoint{1.551423in}{2.645794in}}%
\pgfpathlineto{\pgfqpoint{1.563042in}{2.649079in}}%
\pgfpathlineto{\pgfqpoint{1.574656in}{2.652362in}}%
\pgfpathlineto{\pgfqpoint{1.586264in}{2.655642in}}%
\pgfpathlineto{\pgfqpoint{1.597867in}{2.658922in}}%
\pgfpathlineto{\pgfqpoint{1.609464in}{2.662200in}}%
\pgfpathlineto{\pgfqpoint{1.603797in}{2.669905in}}%
\pgfpathlineto{\pgfqpoint{1.598135in}{2.677593in}}%
\pgfpathlineto{\pgfqpoint{1.592477in}{2.685264in}}%
\pgfpathlineto{\pgfqpoint{1.586823in}{2.692919in}}%
\pgfpathlineto{\pgfqpoint{1.581173in}{2.700559in}}%
\pgfpathlineto{\pgfqpoint{1.569590in}{2.697269in}}%
\pgfpathlineto{\pgfqpoint{1.558002in}{2.693978in}}%
\pgfpathlineto{\pgfqpoint{1.546408in}{2.690687in}}%
\pgfpathlineto{\pgfqpoint{1.534808in}{2.687395in}}%
\pgfpathlineto{\pgfqpoint{1.523203in}{2.684100in}}%
\pgfpathlineto{\pgfqpoint{1.528838in}{2.676472in}}%
\pgfpathlineto{\pgfqpoint{1.534478in}{2.668828in}}%
\pgfpathlineto{\pgfqpoint{1.540122in}{2.661167in}}%
\pgfpathlineto{\pgfqpoint{1.545770in}{2.653490in}}%
\pgfpathclose%
\pgfusepath{stroke,fill}%
\end{pgfscope}%
\begin{pgfscope}%
\pgfpathrectangle{\pgfqpoint{0.887500in}{0.275000in}}{\pgfqpoint{4.225000in}{4.225000in}}%
\pgfusepath{clip}%
\pgfsetbuttcap%
\pgfsetroundjoin%
\definecolor{currentfill}{rgb}{0.741388,0.873449,0.149561}%
\pgfsetfillcolor{currentfill}%
\pgfsetfillopacity{0.700000}%
\pgfsetlinewidth{0.501875pt}%
\definecolor{currentstroke}{rgb}{1.000000,1.000000,1.000000}%
\pgfsetstrokecolor{currentstroke}%
\pgfsetstrokeopacity{0.500000}%
\pgfsetdash{}{0pt}%
\pgfpathmoveto{\pgfqpoint{3.292260in}{3.365358in}}%
\pgfpathlineto{\pgfqpoint{3.303456in}{3.369018in}}%
\pgfpathlineto{\pgfqpoint{3.314647in}{3.372768in}}%
\pgfpathlineto{\pgfqpoint{3.325833in}{3.376584in}}%
\pgfpathlineto{\pgfqpoint{3.337015in}{3.380445in}}%
\pgfpathlineto{\pgfqpoint{3.348192in}{3.384330in}}%
\pgfpathlineto{\pgfqpoint{3.341950in}{3.394692in}}%
\pgfpathlineto{\pgfqpoint{3.335710in}{3.404892in}}%
\pgfpathlineto{\pgfqpoint{3.329472in}{3.414891in}}%
\pgfpathlineto{\pgfqpoint{3.323236in}{3.424652in}}%
\pgfpathlineto{\pgfqpoint{3.317003in}{3.434134in}}%
\pgfpathlineto{\pgfqpoint{3.305833in}{3.430043in}}%
\pgfpathlineto{\pgfqpoint{3.294659in}{3.425985in}}%
\pgfpathlineto{\pgfqpoint{3.283480in}{3.421988in}}%
\pgfpathlineto{\pgfqpoint{3.272297in}{3.418081in}}%
\pgfpathlineto{\pgfqpoint{3.261109in}{3.414291in}}%
\pgfpathlineto{\pgfqpoint{3.267334in}{3.404893in}}%
\pgfpathlineto{\pgfqpoint{3.273562in}{3.395262in}}%
\pgfpathlineto{\pgfqpoint{3.279792in}{3.385437in}}%
\pgfpathlineto{\pgfqpoint{3.286024in}{3.375456in}}%
\pgfpathclose%
\pgfusepath{stroke,fill}%
\end{pgfscope}%
\begin{pgfscope}%
\pgfpathrectangle{\pgfqpoint{0.887500in}{0.275000in}}{\pgfqpoint{4.225000in}{4.225000in}}%
\pgfusepath{clip}%
\pgfsetbuttcap%
\pgfsetroundjoin%
\definecolor{currentfill}{rgb}{0.139147,0.533812,0.555298}%
\pgfsetfillcolor{currentfill}%
\pgfsetfillopacity{0.700000}%
\pgfsetlinewidth{0.501875pt}%
\definecolor{currentstroke}{rgb}{1.000000,1.000000,1.000000}%
\pgfsetstrokecolor{currentstroke}%
\pgfsetstrokeopacity{0.500000}%
\pgfsetdash{}{0pt}%
\pgfpathmoveto{\pgfqpoint{2.417346in}{2.505072in}}%
\pgfpathlineto{\pgfqpoint{2.428752in}{2.508529in}}%
\pgfpathlineto{\pgfqpoint{2.440154in}{2.511912in}}%
\pgfpathlineto{\pgfqpoint{2.451551in}{2.515195in}}%
\pgfpathlineto{\pgfqpoint{2.462945in}{2.518351in}}%
\pgfpathlineto{\pgfqpoint{2.474334in}{2.521401in}}%
\pgfpathlineto{\pgfqpoint{2.468363in}{2.529732in}}%
\pgfpathlineto{\pgfqpoint{2.462395in}{2.538033in}}%
\pgfpathlineto{\pgfqpoint{2.456432in}{2.546310in}}%
\pgfpathlineto{\pgfqpoint{2.450472in}{2.554569in}}%
\pgfpathlineto{\pgfqpoint{2.444517in}{2.562818in}}%
\pgfpathlineto{\pgfqpoint{2.433136in}{2.559976in}}%
\pgfpathlineto{\pgfqpoint{2.421751in}{2.556959in}}%
\pgfpathlineto{\pgfqpoint{2.410364in}{2.553747in}}%
\pgfpathlineto{\pgfqpoint{2.398974in}{2.550379in}}%
\pgfpathlineto{\pgfqpoint{2.387581in}{2.546897in}}%
\pgfpathlineto{\pgfqpoint{2.393526in}{2.538557in}}%
\pgfpathlineto{\pgfqpoint{2.399475in}{2.530201in}}%
\pgfpathlineto{\pgfqpoint{2.405428in}{2.521833in}}%
\pgfpathlineto{\pgfqpoint{2.411385in}{2.513456in}}%
\pgfpathclose%
\pgfusepath{stroke,fill}%
\end{pgfscope}%
\begin{pgfscope}%
\pgfpathrectangle{\pgfqpoint{0.887500in}{0.275000in}}{\pgfqpoint{4.225000in}{4.225000in}}%
\pgfusepath{clip}%
\pgfsetbuttcap%
\pgfsetroundjoin%
\definecolor{currentfill}{rgb}{0.288921,0.758394,0.428426}%
\pgfsetfillcolor{currentfill}%
\pgfsetfillopacity{0.700000}%
\pgfsetlinewidth{0.501875pt}%
\definecolor{currentstroke}{rgb}{1.000000,1.000000,1.000000}%
\pgfsetstrokecolor{currentstroke}%
\pgfsetstrokeopacity{0.500000}%
\pgfsetdash{}{0pt}%
\pgfpathmoveto{\pgfqpoint{4.076280in}{2.985104in}}%
\pgfpathlineto{\pgfqpoint{4.087287in}{2.988580in}}%
\pgfpathlineto{\pgfqpoint{4.098289in}{2.992049in}}%
\pgfpathlineto{\pgfqpoint{4.109285in}{2.995511in}}%
\pgfpathlineto{\pgfqpoint{4.120277in}{2.998968in}}%
\pgfpathlineto{\pgfqpoint{4.131262in}{3.002419in}}%
\pgfpathlineto{\pgfqpoint{4.124881in}{3.015595in}}%
\pgfpathlineto{\pgfqpoint{4.118499in}{3.028655in}}%
\pgfpathlineto{\pgfqpoint{4.112117in}{3.041610in}}%
\pgfpathlineto{\pgfqpoint{4.105735in}{3.054473in}}%
\pgfpathlineto{\pgfqpoint{4.099355in}{3.067254in}}%
\pgfpathlineto{\pgfqpoint{4.088377in}{3.063974in}}%
\pgfpathlineto{\pgfqpoint{4.077394in}{3.060712in}}%
\pgfpathlineto{\pgfqpoint{4.066406in}{3.057467in}}%
\pgfpathlineto{\pgfqpoint{4.055413in}{3.054240in}}%
\pgfpathlineto{\pgfqpoint{4.044416in}{3.051031in}}%
\pgfpathlineto{\pgfqpoint{4.050787in}{3.038011in}}%
\pgfpathlineto{\pgfqpoint{4.057160in}{3.024912in}}%
\pgfpathlineto{\pgfqpoint{4.063533in}{3.011730in}}%
\pgfpathlineto{\pgfqpoint{4.069906in}{2.998463in}}%
\pgfpathclose%
\pgfusepath{stroke,fill}%
\end{pgfscope}%
\begin{pgfscope}%
\pgfpathrectangle{\pgfqpoint{0.887500in}{0.275000in}}{\pgfqpoint{4.225000in}{4.225000in}}%
\pgfusepath{clip}%
\pgfsetbuttcap%
\pgfsetroundjoin%
\definecolor{currentfill}{rgb}{0.141935,0.526453,0.555991}%
\pgfsetfillcolor{currentfill}%
\pgfsetfillopacity{0.700000}%
\pgfsetlinewidth{0.501875pt}%
\definecolor{currentstroke}{rgb}{1.000000,1.000000,1.000000}%
\pgfsetstrokecolor{currentstroke}%
\pgfsetstrokeopacity{0.500000}%
\pgfsetdash{}{0pt}%
\pgfpathmoveto{\pgfqpoint{2.647970in}{2.478823in}}%
\pgfpathlineto{\pgfqpoint{2.659291in}{2.484993in}}%
\pgfpathlineto{\pgfqpoint{2.670606in}{2.491398in}}%
\pgfpathlineto{\pgfqpoint{2.681915in}{2.498002in}}%
\pgfpathlineto{\pgfqpoint{2.693218in}{2.504926in}}%
\pgfpathlineto{\pgfqpoint{2.704512in}{2.512396in}}%
\pgfpathlineto{\pgfqpoint{2.698429in}{2.524505in}}%
\pgfpathlineto{\pgfqpoint{2.692353in}{2.536140in}}%
\pgfpathlineto{\pgfqpoint{2.686284in}{2.547319in}}%
\pgfpathlineto{\pgfqpoint{2.680222in}{2.558069in}}%
\pgfpathlineto{\pgfqpoint{2.674167in}{2.568413in}}%
\pgfpathlineto{\pgfqpoint{2.662914in}{2.557848in}}%
\pgfpathlineto{\pgfqpoint{2.651654in}{2.547898in}}%
\pgfpathlineto{\pgfqpoint{2.640383in}{2.538751in}}%
\pgfpathlineto{\pgfqpoint{2.629100in}{2.530494in}}%
\pgfpathlineto{\pgfqpoint{2.617805in}{2.523073in}}%
\pgfpathlineto{\pgfqpoint{2.623830in}{2.514328in}}%
\pgfpathlineto{\pgfqpoint{2.629858in}{2.505519in}}%
\pgfpathlineto{\pgfqpoint{2.635891in}{2.496657in}}%
\pgfpathlineto{\pgfqpoint{2.641929in}{2.487754in}}%
\pgfpathclose%
\pgfusepath{stroke,fill}%
\end{pgfscope}%
\begin{pgfscope}%
\pgfpathrectangle{\pgfqpoint{0.887500in}{0.275000in}}{\pgfqpoint{4.225000in}{4.225000in}}%
\pgfusepath{clip}%
\pgfsetbuttcap%
\pgfsetroundjoin%
\definecolor{currentfill}{rgb}{0.120081,0.622161,0.534946}%
\pgfsetfillcolor{currentfill}%
\pgfsetfillopacity{0.700000}%
\pgfsetlinewidth{0.501875pt}%
\definecolor{currentstroke}{rgb}{1.000000,1.000000,1.000000}%
\pgfsetstrokecolor{currentstroke}%
\pgfsetstrokeopacity{0.500000}%
\pgfsetdash{}{0pt}%
\pgfpathmoveto{\pgfqpoint{2.816873in}{2.654485in}}%
\pgfpathlineto{\pgfqpoint{2.828077in}{2.675088in}}%
\pgfpathlineto{\pgfqpoint{2.839281in}{2.696284in}}%
\pgfpathlineto{\pgfqpoint{2.850486in}{2.718128in}}%
\pgfpathlineto{\pgfqpoint{2.861693in}{2.740731in}}%
\pgfpathlineto{\pgfqpoint{2.872902in}{2.764206in}}%
\pgfpathlineto{\pgfqpoint{2.866832in}{2.763462in}}%
\pgfpathlineto{\pgfqpoint{2.860764in}{2.763661in}}%
\pgfpathlineto{\pgfqpoint{2.854697in}{2.764790in}}%
\pgfpathlineto{\pgfqpoint{2.848632in}{2.766818in}}%
\pgfpathlineto{\pgfqpoint{2.842569in}{2.769717in}}%
\pgfpathlineto{\pgfqpoint{2.831381in}{2.749044in}}%
\pgfpathlineto{\pgfqpoint{2.820189in}{2.729918in}}%
\pgfpathlineto{\pgfqpoint{2.808990in}{2.712364in}}%
\pgfpathlineto{\pgfqpoint{2.797784in}{2.696409in}}%
\pgfpathlineto{\pgfqpoint{2.786569in}{2.682015in}}%
\pgfpathlineto{\pgfqpoint{2.792625in}{2.675730in}}%
\pgfpathlineto{\pgfqpoint{2.798684in}{2.669841in}}%
\pgfpathlineto{\pgfqpoint{2.804744in}{2.664343in}}%
\pgfpathlineto{\pgfqpoint{2.810808in}{2.659228in}}%
\pgfpathclose%
\pgfusepath{stroke,fill}%
\end{pgfscope}%
\begin{pgfscope}%
\pgfpathrectangle{\pgfqpoint{0.887500in}{0.275000in}}{\pgfqpoint{4.225000in}{4.225000in}}%
\pgfusepath{clip}%
\pgfsetbuttcap%
\pgfsetroundjoin%
\definecolor{currentfill}{rgb}{0.125394,0.574318,0.549086}%
\pgfsetfillcolor{currentfill}%
\pgfsetfillopacity{0.700000}%
\pgfsetlinewidth{0.501875pt}%
\definecolor{currentstroke}{rgb}{1.000000,1.000000,1.000000}%
\pgfsetstrokecolor{currentstroke}%
\pgfsetstrokeopacity{0.500000}%
\pgfsetdash{}{0pt}%
\pgfpathmoveto{\pgfqpoint{1.868870in}{2.592941in}}%
\pgfpathlineto{\pgfqpoint{1.880416in}{2.596188in}}%
\pgfpathlineto{\pgfqpoint{1.891956in}{2.599442in}}%
\pgfpathlineto{\pgfqpoint{1.903490in}{2.602705in}}%
\pgfpathlineto{\pgfqpoint{1.915018in}{2.605980in}}%
\pgfpathlineto{\pgfqpoint{1.926539in}{2.609269in}}%
\pgfpathlineto{\pgfqpoint{1.920755in}{2.617273in}}%
\pgfpathlineto{\pgfqpoint{1.914974in}{2.625263in}}%
\pgfpathlineto{\pgfqpoint{1.909198in}{2.633238in}}%
\pgfpathlineto{\pgfqpoint{1.903426in}{2.641197in}}%
\pgfpathlineto{\pgfqpoint{1.897658in}{2.649139in}}%
\pgfpathlineto{\pgfqpoint{1.886149in}{2.645850in}}%
\pgfpathlineto{\pgfqpoint{1.874634in}{2.642574in}}%
\pgfpathlineto{\pgfqpoint{1.863113in}{2.639310in}}%
\pgfpathlineto{\pgfqpoint{1.851586in}{2.636054in}}%
\pgfpathlineto{\pgfqpoint{1.840053in}{2.632806in}}%
\pgfpathlineto{\pgfqpoint{1.845808in}{2.624866in}}%
\pgfpathlineto{\pgfqpoint{1.851567in}{2.616909in}}%
\pgfpathlineto{\pgfqpoint{1.857331in}{2.608935in}}%
\pgfpathlineto{\pgfqpoint{1.863098in}{2.600946in}}%
\pgfpathclose%
\pgfusepath{stroke,fill}%
\end{pgfscope}%
\begin{pgfscope}%
\pgfpathrectangle{\pgfqpoint{0.887500in}{0.275000in}}{\pgfqpoint{4.225000in}{4.225000in}}%
\pgfusepath{clip}%
\pgfsetbuttcap%
\pgfsetroundjoin%
\definecolor{currentfill}{rgb}{0.709898,0.868751,0.169257}%
\pgfsetfillcolor{currentfill}%
\pgfsetfillopacity{0.700000}%
\pgfsetlinewidth{0.501875pt}%
\definecolor{currentstroke}{rgb}{1.000000,1.000000,1.000000}%
\pgfsetstrokecolor{currentstroke}%
\pgfsetstrokeopacity{0.500000}%
\pgfsetdash{}{0pt}%
\pgfpathmoveto{\pgfqpoint{3.379445in}{3.331420in}}%
\pgfpathlineto{\pgfqpoint{3.390621in}{3.334840in}}%
\pgfpathlineto{\pgfqpoint{3.401791in}{3.338270in}}%
\pgfpathlineto{\pgfqpoint{3.412957in}{3.341715in}}%
\pgfpathlineto{\pgfqpoint{3.424117in}{3.345181in}}%
\pgfpathlineto{\pgfqpoint{3.435272in}{3.348673in}}%
\pgfpathlineto{\pgfqpoint{3.429013in}{3.359805in}}%
\pgfpathlineto{\pgfqpoint{3.422756in}{3.370926in}}%
\pgfpathlineto{\pgfqpoint{3.416502in}{3.381994in}}%
\pgfpathlineto{\pgfqpoint{3.410251in}{3.392970in}}%
\pgfpathlineto{\pgfqpoint{3.404001in}{3.403811in}}%
\pgfpathlineto{\pgfqpoint{3.392850in}{3.399932in}}%
\pgfpathlineto{\pgfqpoint{3.381693in}{3.396037in}}%
\pgfpathlineto{\pgfqpoint{3.370531in}{3.392133in}}%
\pgfpathlineto{\pgfqpoint{3.359364in}{3.388228in}}%
\pgfpathlineto{\pgfqpoint{3.348192in}{3.384330in}}%
\pgfpathlineto{\pgfqpoint{3.354436in}{3.373843in}}%
\pgfpathlineto{\pgfqpoint{3.360684in}{3.363271in}}%
\pgfpathlineto{\pgfqpoint{3.366934in}{3.352651in}}%
\pgfpathlineto{\pgfqpoint{3.373188in}{3.342023in}}%
\pgfpathclose%
\pgfusepath{stroke,fill}%
\end{pgfscope}%
\begin{pgfscope}%
\pgfpathrectangle{\pgfqpoint{0.887500in}{0.275000in}}{\pgfqpoint{4.225000in}{4.225000in}}%
\pgfusepath{clip}%
\pgfsetbuttcap%
\pgfsetroundjoin%
\definecolor{currentfill}{rgb}{0.344074,0.780029,0.397381}%
\pgfsetfillcolor{currentfill}%
\pgfsetfillopacity{0.700000}%
\pgfsetlinewidth{0.501875pt}%
\definecolor{currentstroke}{rgb}{1.000000,1.000000,1.000000}%
\pgfsetstrokecolor{currentstroke}%
\pgfsetstrokeopacity{0.500000}%
\pgfsetdash{}{0pt}%
\pgfpathmoveto{\pgfqpoint{3.989349in}{3.034972in}}%
\pgfpathlineto{\pgfqpoint{4.000374in}{3.038226in}}%
\pgfpathlineto{\pgfqpoint{4.011393in}{3.041447in}}%
\pgfpathlineto{\pgfqpoint{4.022406in}{3.044647in}}%
\pgfpathlineto{\pgfqpoint{4.033414in}{3.047838in}}%
\pgfpathlineto{\pgfqpoint{4.044416in}{3.051031in}}%
\pgfpathlineto{\pgfqpoint{4.038046in}{3.063978in}}%
\pgfpathlineto{\pgfqpoint{4.031676in}{3.076855in}}%
\pgfpathlineto{\pgfqpoint{4.025308in}{3.089666in}}%
\pgfpathlineto{\pgfqpoint{4.018942in}{3.102415in}}%
\pgfpathlineto{\pgfqpoint{4.012576in}{3.115107in}}%
\pgfpathlineto{\pgfqpoint{4.001582in}{3.112103in}}%
\pgfpathlineto{\pgfqpoint{3.990582in}{3.109098in}}%
\pgfpathlineto{\pgfqpoint{3.979576in}{3.106076in}}%
\pgfpathlineto{\pgfqpoint{3.968565in}{3.103022in}}%
\pgfpathlineto{\pgfqpoint{3.957546in}{3.099920in}}%
\pgfpathlineto{\pgfqpoint{3.963907in}{3.087190in}}%
\pgfpathlineto{\pgfqpoint{3.970268in}{3.074321in}}%
\pgfpathlineto{\pgfqpoint{3.976628in}{3.061321in}}%
\pgfpathlineto{\pgfqpoint{3.982988in}{3.048201in}}%
\pgfpathclose%
\pgfusepath{stroke,fill}%
\end{pgfscope}%
\begin{pgfscope}%
\pgfpathrectangle{\pgfqpoint{0.887500in}{0.275000in}}{\pgfqpoint{4.225000in}{4.225000in}}%
\pgfusepath{clip}%
\pgfsetbuttcap%
\pgfsetroundjoin%
\definecolor{currentfill}{rgb}{0.657642,0.860219,0.203082}%
\pgfsetfillcolor{currentfill}%
\pgfsetfillopacity{0.700000}%
\pgfsetlinewidth{0.501875pt}%
\definecolor{currentstroke}{rgb}{1.000000,1.000000,1.000000}%
\pgfsetstrokecolor{currentstroke}%
\pgfsetstrokeopacity{0.500000}%
\pgfsetdash{}{0pt}%
\pgfpathmoveto{\pgfqpoint{3.466620in}{3.293087in}}%
\pgfpathlineto{\pgfqpoint{3.477774in}{3.296329in}}%
\pgfpathlineto{\pgfqpoint{3.488925in}{3.299744in}}%
\pgfpathlineto{\pgfqpoint{3.500072in}{3.303304in}}%
\pgfpathlineto{\pgfqpoint{3.511215in}{3.306971in}}%
\pgfpathlineto{\pgfqpoint{3.522354in}{3.310706in}}%
\pgfpathlineto{\pgfqpoint{3.516071in}{3.321821in}}%
\pgfpathlineto{\pgfqpoint{3.509792in}{3.332951in}}%
\pgfpathlineto{\pgfqpoint{3.503516in}{3.344105in}}%
\pgfpathlineto{\pgfqpoint{3.497244in}{3.355295in}}%
\pgfpathlineto{\pgfqpoint{3.490975in}{3.366530in}}%
\pgfpathlineto{\pgfqpoint{3.479845in}{3.362935in}}%
\pgfpathlineto{\pgfqpoint{3.468709in}{3.359340in}}%
\pgfpathlineto{\pgfqpoint{3.457569in}{3.355756in}}%
\pgfpathlineto{\pgfqpoint{3.446423in}{3.352197in}}%
\pgfpathlineto{\pgfqpoint{3.435272in}{3.348673in}}%
\pgfpathlineto{\pgfqpoint{3.441535in}{3.337544in}}%
\pgfpathlineto{\pgfqpoint{3.447802in}{3.326420in}}%
\pgfpathlineto{\pgfqpoint{3.454071in}{3.315302in}}%
\pgfpathlineto{\pgfqpoint{3.460344in}{3.304191in}}%
\pgfpathclose%
\pgfusepath{stroke,fill}%
\end{pgfscope}%
\begin{pgfscope}%
\pgfpathrectangle{\pgfqpoint{0.887500in}{0.275000in}}{\pgfqpoint{4.225000in}{4.225000in}}%
\pgfusepath{clip}%
\pgfsetbuttcap%
\pgfsetroundjoin%
\definecolor{currentfill}{rgb}{0.132444,0.552216,0.553018}%
\pgfsetfillcolor{currentfill}%
\pgfsetfillopacity{0.700000}%
\pgfsetlinewidth{0.501875pt}%
\definecolor{currentstroke}{rgb}{1.000000,1.000000,1.000000}%
\pgfsetstrokecolor{currentstroke}%
\pgfsetstrokeopacity{0.500000}%
\pgfsetdash{}{0pt}%
\pgfpathmoveto{\pgfqpoint{2.186508in}{2.537218in}}%
\pgfpathlineto{\pgfqpoint{2.197977in}{2.540493in}}%
\pgfpathlineto{\pgfqpoint{2.209440in}{2.543771in}}%
\pgfpathlineto{\pgfqpoint{2.220898in}{2.547053in}}%
\pgfpathlineto{\pgfqpoint{2.232349in}{2.550343in}}%
\pgfpathlineto{\pgfqpoint{2.243795in}{2.553641in}}%
\pgfpathlineto{\pgfqpoint{2.237899in}{2.561859in}}%
\pgfpathlineto{\pgfqpoint{2.232007in}{2.570065in}}%
\pgfpathlineto{\pgfqpoint{2.226119in}{2.578259in}}%
\pgfpathlineto{\pgfqpoint{2.220236in}{2.586441in}}%
\pgfpathlineto{\pgfqpoint{2.214356in}{2.594610in}}%
\pgfpathlineto{\pgfqpoint{2.202923in}{2.591271in}}%
\pgfpathlineto{\pgfqpoint{2.191484in}{2.587957in}}%
\pgfpathlineto{\pgfqpoint{2.180039in}{2.584663in}}%
\pgfpathlineto{\pgfqpoint{2.168588in}{2.581382in}}%
\pgfpathlineto{\pgfqpoint{2.157131in}{2.578111in}}%
\pgfpathlineto{\pgfqpoint{2.162998in}{2.569963in}}%
\pgfpathlineto{\pgfqpoint{2.168869in}{2.561800in}}%
\pgfpathlineto{\pgfqpoint{2.174744in}{2.553621in}}%
\pgfpathlineto{\pgfqpoint{2.180624in}{2.545427in}}%
\pgfpathclose%
\pgfusepath{stroke,fill}%
\end{pgfscope}%
\begin{pgfscope}%
\pgfpathrectangle{\pgfqpoint{0.887500in}{0.275000in}}{\pgfqpoint{4.225000in}{4.225000in}}%
\pgfusepath{clip}%
\pgfsetbuttcap%
\pgfsetroundjoin%
\definecolor{currentfill}{rgb}{0.395174,0.797475,0.367757}%
\pgfsetfillcolor{currentfill}%
\pgfsetfillopacity{0.700000}%
\pgfsetlinewidth{0.501875pt}%
\definecolor{currentstroke}{rgb}{1.000000,1.000000,1.000000}%
\pgfsetstrokecolor{currentstroke}%
\pgfsetstrokeopacity{0.500000}%
\pgfsetdash{}{0pt}%
\pgfpathmoveto{\pgfqpoint{3.902352in}{3.083355in}}%
\pgfpathlineto{\pgfqpoint{3.913404in}{3.086797in}}%
\pgfpathlineto{\pgfqpoint{3.924450in}{3.090187in}}%
\pgfpathlineto{\pgfqpoint{3.935489in}{3.093511in}}%
\pgfpathlineto{\pgfqpoint{3.946521in}{3.096755in}}%
\pgfpathlineto{\pgfqpoint{3.957546in}{3.099920in}}%
\pgfpathlineto{\pgfqpoint{3.951185in}{3.112515in}}%
\pgfpathlineto{\pgfqpoint{3.944825in}{3.124989in}}%
\pgfpathlineto{\pgfqpoint{3.938465in}{3.137355in}}%
\pgfpathlineto{\pgfqpoint{3.932105in}{3.149627in}}%
\pgfpathlineto{\pgfqpoint{3.925747in}{3.161818in}}%
\pgfpathlineto{\pgfqpoint{3.914723in}{3.158509in}}%
\pgfpathlineto{\pgfqpoint{3.903693in}{3.155132in}}%
\pgfpathlineto{\pgfqpoint{3.892655in}{3.151691in}}%
\pgfpathlineto{\pgfqpoint{3.881612in}{3.148203in}}%
\pgfpathlineto{\pgfqpoint{3.870563in}{3.144683in}}%
\pgfpathlineto{\pgfqpoint{3.876919in}{3.132634in}}%
\pgfpathlineto{\pgfqpoint{3.883277in}{3.120504in}}%
\pgfpathlineto{\pgfqpoint{3.889635in}{3.108265in}}%
\pgfpathlineto{\pgfqpoint{3.895994in}{3.095891in}}%
\pgfpathclose%
\pgfusepath{stroke,fill}%
\end{pgfscope}%
\begin{pgfscope}%
\pgfpathrectangle{\pgfqpoint{0.887500in}{0.275000in}}{\pgfqpoint{4.225000in}{4.225000in}}%
\pgfusepath{clip}%
\pgfsetbuttcap%
\pgfsetroundjoin%
\definecolor{currentfill}{rgb}{0.449368,0.813768,0.335384}%
\pgfsetfillcolor{currentfill}%
\pgfsetfillopacity{0.700000}%
\pgfsetlinewidth{0.501875pt}%
\definecolor{currentstroke}{rgb}{1.000000,1.000000,1.000000}%
\pgfsetstrokecolor{currentstroke}%
\pgfsetstrokeopacity{0.500000}%
\pgfsetdash{}{0pt}%
\pgfpathmoveto{\pgfqpoint{3.815242in}{3.127210in}}%
\pgfpathlineto{\pgfqpoint{3.826315in}{3.130642in}}%
\pgfpathlineto{\pgfqpoint{3.837384in}{3.134113in}}%
\pgfpathlineto{\pgfqpoint{3.848449in}{3.137622in}}%
\pgfpathlineto{\pgfqpoint{3.859508in}{3.141151in}}%
\pgfpathlineto{\pgfqpoint{3.870563in}{3.144683in}}%
\pgfpathlineto{\pgfqpoint{3.864208in}{3.156679in}}%
\pgfpathlineto{\pgfqpoint{3.857857in}{3.168647in}}%
\pgfpathlineto{\pgfqpoint{3.851508in}{3.180616in}}%
\pgfpathlineto{\pgfqpoint{3.845163in}{3.192613in}}%
\pgfpathlineto{\pgfqpoint{3.838822in}{3.204664in}}%
\pgfpathlineto{\pgfqpoint{3.827772in}{3.201116in}}%
\pgfpathlineto{\pgfqpoint{3.816717in}{3.197588in}}%
\pgfpathlineto{\pgfqpoint{3.805658in}{3.194096in}}%
\pgfpathlineto{\pgfqpoint{3.794595in}{3.190661in}}%
\pgfpathlineto{\pgfqpoint{3.783528in}{3.187283in}}%
\pgfpathlineto{\pgfqpoint{3.789863in}{3.175191in}}%
\pgfpathlineto{\pgfqpoint{3.796203in}{3.163167in}}%
\pgfpathlineto{\pgfqpoint{3.802546in}{3.151182in}}%
\pgfpathlineto{\pgfqpoint{3.808892in}{3.139207in}}%
\pgfpathclose%
\pgfusepath{stroke,fill}%
\end{pgfscope}%
\begin{pgfscope}%
\pgfpathrectangle{\pgfqpoint{0.887500in}{0.275000in}}{\pgfqpoint{4.225000in}{4.225000in}}%
\pgfusepath{clip}%
\pgfsetbuttcap%
\pgfsetroundjoin%
\definecolor{currentfill}{rgb}{0.143343,0.522773,0.556295}%
\pgfsetfillcolor{currentfill}%
\pgfsetfillopacity{0.700000}%
\pgfsetlinewidth{0.501875pt}%
\definecolor{currentstroke}{rgb}{1.000000,1.000000,1.000000}%
\pgfsetstrokecolor{currentstroke}%
\pgfsetstrokeopacity{0.500000}%
\pgfsetdash{}{0pt}%
\pgfpathmoveto{\pgfqpoint{2.504257in}{2.479377in}}%
\pgfpathlineto{\pgfqpoint{2.515652in}{2.482414in}}%
\pgfpathlineto{\pgfqpoint{2.527041in}{2.485520in}}%
\pgfpathlineto{\pgfqpoint{2.538422in}{2.488783in}}%
\pgfpathlineto{\pgfqpoint{2.549794in}{2.492290in}}%
\pgfpathlineto{\pgfqpoint{2.561157in}{2.496130in}}%
\pgfpathlineto{\pgfqpoint{2.555152in}{2.504640in}}%
\pgfpathlineto{\pgfqpoint{2.549151in}{2.513101in}}%
\pgfpathlineto{\pgfqpoint{2.543155in}{2.521500in}}%
\pgfpathlineto{\pgfqpoint{2.537164in}{2.529826in}}%
\pgfpathlineto{\pgfqpoint{2.531178in}{2.538066in}}%
\pgfpathlineto{\pgfqpoint{2.519827in}{2.534254in}}%
\pgfpathlineto{\pgfqpoint{2.508466in}{2.530768in}}%
\pgfpathlineto{\pgfqpoint{2.497096in}{2.527523in}}%
\pgfpathlineto{\pgfqpoint{2.485719in}{2.524430in}}%
\pgfpathlineto{\pgfqpoint{2.474334in}{2.521401in}}%
\pgfpathlineto{\pgfqpoint{2.480311in}{2.513044in}}%
\pgfpathlineto{\pgfqpoint{2.486291in}{2.504660in}}%
\pgfpathlineto{\pgfqpoint{2.492275in}{2.496253in}}%
\pgfpathlineto{\pgfqpoint{2.498264in}{2.487825in}}%
\pgfpathclose%
\pgfusepath{stroke,fill}%
\end{pgfscope}%
\begin{pgfscope}%
\pgfpathrectangle{\pgfqpoint{0.887500in}{0.275000in}}{\pgfqpoint{4.225000in}{4.225000in}}%
\pgfusepath{clip}%
\pgfsetbuttcap%
\pgfsetroundjoin%
\definecolor{currentfill}{rgb}{0.412913,0.803041,0.357269}%
\pgfsetfillcolor{currentfill}%
\pgfsetfillopacity{0.700000}%
\pgfsetlinewidth{0.501875pt}%
\definecolor{currentstroke}{rgb}{1.000000,1.000000,1.000000}%
\pgfsetstrokecolor{currentstroke}%
\pgfsetstrokeopacity{0.500000}%
\pgfsetdash{}{0pt}%
\pgfpathmoveto{\pgfqpoint{2.923944in}{3.046830in}}%
\pgfpathlineto{\pgfqpoint{2.935125in}{3.080130in}}%
\pgfpathlineto{\pgfqpoint{2.946315in}{3.114384in}}%
\pgfpathlineto{\pgfqpoint{2.957516in}{3.148753in}}%
\pgfpathlineto{\pgfqpoint{2.968731in}{3.182398in}}%
\pgfpathlineto{\pgfqpoint{2.979960in}{3.214469in}}%
\pgfpathlineto{\pgfqpoint{2.973813in}{3.218889in}}%
\pgfpathlineto{\pgfqpoint{2.967672in}{3.223039in}}%
\pgfpathlineto{\pgfqpoint{2.961537in}{3.226916in}}%
\pgfpathlineto{\pgfqpoint{2.955408in}{3.230522in}}%
\pgfpathlineto{\pgfqpoint{2.949284in}{3.233858in}}%
\pgfpathlineto{\pgfqpoint{2.938112in}{3.195712in}}%
\pgfpathlineto{\pgfqpoint{2.926958in}{3.156301in}}%
\pgfpathlineto{\pgfqpoint{2.915818in}{3.117032in}}%
\pgfpathlineto{\pgfqpoint{2.904686in}{3.079303in}}%
\pgfpathlineto{\pgfqpoint{2.893557in}{3.044503in}}%
\pgfpathlineto{\pgfqpoint{2.899620in}{3.045419in}}%
\pgfpathlineto{\pgfqpoint{2.905690in}{3.046105in}}%
\pgfpathlineto{\pgfqpoint{2.911767in}{3.046563in}}%
\pgfpathlineto{\pgfqpoint{2.917852in}{3.046801in}}%
\pgfpathclose%
\pgfusepath{stroke,fill}%
\end{pgfscope}%
\begin{pgfscope}%
\pgfpathrectangle{\pgfqpoint{0.887500in}{0.275000in}}{\pgfqpoint{4.225000in}{4.225000in}}%
\pgfusepath{clip}%
\pgfsetbuttcap%
\pgfsetroundjoin%
\definecolor{currentfill}{rgb}{0.606045,0.850733,0.236712}%
\pgfsetfillcolor{currentfill}%
\pgfsetfillopacity{0.700000}%
\pgfsetlinewidth{0.501875pt}%
\definecolor{currentstroke}{rgb}{1.000000,1.000000,1.000000}%
\pgfsetstrokecolor{currentstroke}%
\pgfsetstrokeopacity{0.500000}%
\pgfsetdash{}{0pt}%
\pgfpathmoveto{\pgfqpoint{3.553814in}{3.255001in}}%
\pgfpathlineto{\pgfqpoint{3.564955in}{3.258802in}}%
\pgfpathlineto{\pgfqpoint{3.576091in}{3.262607in}}%
\pgfpathlineto{\pgfqpoint{3.587222in}{3.266370in}}%
\pgfpathlineto{\pgfqpoint{3.598346in}{3.270058in}}%
\pgfpathlineto{\pgfqpoint{3.609465in}{3.273675in}}%
\pgfpathlineto{\pgfqpoint{3.603162in}{3.284982in}}%
\pgfpathlineto{\pgfqpoint{3.596860in}{3.296154in}}%
\pgfpathlineto{\pgfqpoint{3.590560in}{3.307220in}}%
\pgfpathlineto{\pgfqpoint{3.584263in}{3.318207in}}%
\pgfpathlineto{\pgfqpoint{3.577968in}{3.329145in}}%
\pgfpathlineto{\pgfqpoint{3.566857in}{3.325575in}}%
\pgfpathlineto{\pgfqpoint{3.555739in}{3.321937in}}%
\pgfpathlineto{\pgfqpoint{3.544616in}{3.318228in}}%
\pgfpathlineto{\pgfqpoint{3.533488in}{3.314471in}}%
\pgfpathlineto{\pgfqpoint{3.522354in}{3.310706in}}%
\pgfpathlineto{\pgfqpoint{3.528640in}{3.299596in}}%
\pgfpathlineto{\pgfqpoint{3.534929in}{3.288480in}}%
\pgfpathlineto{\pgfqpoint{3.541221in}{3.277349in}}%
\pgfpathlineto{\pgfqpoint{3.547516in}{3.266193in}}%
\pgfpathclose%
\pgfusepath{stroke,fill}%
\end{pgfscope}%
\begin{pgfscope}%
\pgfpathrectangle{\pgfqpoint{0.887500in}{0.275000in}}{\pgfqpoint{4.225000in}{4.225000in}}%
\pgfusepath{clip}%
\pgfsetbuttcap%
\pgfsetroundjoin%
\definecolor{currentfill}{rgb}{0.506271,0.828786,0.300362}%
\pgfsetfillcolor{currentfill}%
\pgfsetfillopacity{0.700000}%
\pgfsetlinewidth{0.501875pt}%
\definecolor{currentstroke}{rgb}{1.000000,1.000000,1.000000}%
\pgfsetstrokecolor{currentstroke}%
\pgfsetstrokeopacity{0.500000}%
\pgfsetdash{}{0pt}%
\pgfpathmoveto{\pgfqpoint{3.728120in}{3.170871in}}%
\pgfpathlineto{\pgfqpoint{3.739212in}{3.174129in}}%
\pgfpathlineto{\pgfqpoint{3.750298in}{3.177388in}}%
\pgfpathlineto{\pgfqpoint{3.761379in}{3.180659in}}%
\pgfpathlineto{\pgfqpoint{3.772456in}{3.183953in}}%
\pgfpathlineto{\pgfqpoint{3.783528in}{3.187283in}}%
\pgfpathlineto{\pgfqpoint{3.777197in}{3.199455in}}%
\pgfpathlineto{\pgfqpoint{3.770870in}{3.211680in}}%
\pgfpathlineto{\pgfqpoint{3.764546in}{3.223927in}}%
\pgfpathlineto{\pgfqpoint{3.758226in}{3.236166in}}%
\pgfpathlineto{\pgfqpoint{3.751907in}{3.248366in}}%
\pgfpathlineto{\pgfqpoint{3.740840in}{3.245045in}}%
\pgfpathlineto{\pgfqpoint{3.729768in}{3.241741in}}%
\pgfpathlineto{\pgfqpoint{3.718691in}{3.238447in}}%
\pgfpathlineto{\pgfqpoint{3.707609in}{3.235154in}}%
\pgfpathlineto{\pgfqpoint{3.696521in}{3.231855in}}%
\pgfpathlineto{\pgfqpoint{3.702835in}{3.219705in}}%
\pgfpathlineto{\pgfqpoint{3.709152in}{3.207506in}}%
\pgfpathlineto{\pgfqpoint{3.715472in}{3.195284in}}%
\pgfpathlineto{\pgfqpoint{3.721794in}{3.183064in}}%
\pgfpathclose%
\pgfusepath{stroke,fill}%
\end{pgfscope}%
\begin{pgfscope}%
\pgfpathrectangle{\pgfqpoint{0.887500in}{0.275000in}}{\pgfqpoint{4.225000in}{4.225000in}}%
\pgfusepath{clip}%
\pgfsetbuttcap%
\pgfsetroundjoin%
\definecolor{currentfill}{rgb}{0.121148,0.592739,0.544641}%
\pgfsetfillcolor{currentfill}%
\pgfsetfillopacity{0.700000}%
\pgfsetlinewidth{0.501875pt}%
\definecolor{currentstroke}{rgb}{1.000000,1.000000,1.000000}%
\pgfsetstrokecolor{currentstroke}%
\pgfsetstrokeopacity{0.500000}%
\pgfsetdash{}{0pt}%
\pgfpathmoveto{\pgfqpoint{1.637863in}{2.623383in}}%
\pgfpathlineto{\pgfqpoint{1.649468in}{2.626656in}}%
\pgfpathlineto{\pgfqpoint{1.661067in}{2.629928in}}%
\pgfpathlineto{\pgfqpoint{1.672660in}{2.633198in}}%
\pgfpathlineto{\pgfqpoint{1.684248in}{2.636465in}}%
\pgfpathlineto{\pgfqpoint{1.695830in}{2.639729in}}%
\pgfpathlineto{\pgfqpoint{1.690128in}{2.647536in}}%
\pgfpathlineto{\pgfqpoint{1.684430in}{2.655324in}}%
\pgfpathlineto{\pgfqpoint{1.678737in}{2.663092in}}%
\pgfpathlineto{\pgfqpoint{1.673048in}{2.670842in}}%
\pgfpathlineto{\pgfqpoint{1.667363in}{2.678574in}}%
\pgfpathlineto{\pgfqpoint{1.655794in}{2.675303in}}%
\pgfpathlineto{\pgfqpoint{1.644220in}{2.672030in}}%
\pgfpathlineto{\pgfqpoint{1.632640in}{2.668755in}}%
\pgfpathlineto{\pgfqpoint{1.621055in}{2.665478in}}%
\pgfpathlineto{\pgfqpoint{1.609464in}{2.662200in}}%
\pgfpathlineto{\pgfqpoint{1.615135in}{2.654478in}}%
\pgfpathlineto{\pgfqpoint{1.620810in}{2.646735in}}%
\pgfpathlineto{\pgfqpoint{1.626490in}{2.638973in}}%
\pgfpathlineto{\pgfqpoint{1.632174in}{2.631189in}}%
\pgfpathclose%
\pgfusepath{stroke,fill}%
\end{pgfscope}%
\begin{pgfscope}%
\pgfpathrectangle{\pgfqpoint{0.887500in}{0.275000in}}{\pgfqpoint{4.225000in}{4.225000in}}%
\pgfusepath{clip}%
\pgfsetbuttcap%
\pgfsetroundjoin%
\definecolor{currentfill}{rgb}{0.555484,0.840254,0.269281}%
\pgfsetfillcolor{currentfill}%
\pgfsetfillopacity{0.700000}%
\pgfsetlinewidth{0.501875pt}%
\definecolor{currentstroke}{rgb}{1.000000,1.000000,1.000000}%
\pgfsetstrokecolor{currentstroke}%
\pgfsetstrokeopacity{0.500000}%
\pgfsetdash{}{0pt}%
\pgfpathmoveto{\pgfqpoint{3.640999in}{3.215137in}}%
\pgfpathlineto{\pgfqpoint{3.652114in}{3.218508in}}%
\pgfpathlineto{\pgfqpoint{3.663224in}{3.221867in}}%
\pgfpathlineto{\pgfqpoint{3.674329in}{3.225212in}}%
\pgfpathlineto{\pgfqpoint{3.685427in}{3.228542in}}%
\pgfpathlineto{\pgfqpoint{3.696521in}{3.231855in}}%
\pgfpathlineto{\pgfqpoint{3.690208in}{3.243932in}}%
\pgfpathlineto{\pgfqpoint{3.683897in}{3.255911in}}%
\pgfpathlineto{\pgfqpoint{3.677586in}{3.267767in}}%
\pgfpathlineto{\pgfqpoint{3.671277in}{3.279474in}}%
\pgfpathlineto{\pgfqpoint{3.664968in}{3.291008in}}%
\pgfpathlineto{\pgfqpoint{3.653879in}{3.287611in}}%
\pgfpathlineto{\pgfqpoint{3.642784in}{3.284189in}}%
\pgfpathlineto{\pgfqpoint{3.631683in}{3.280731in}}%
\pgfpathlineto{\pgfqpoint{3.620577in}{3.277230in}}%
\pgfpathlineto{\pgfqpoint{3.609465in}{3.273675in}}%
\pgfpathlineto{\pgfqpoint{3.615769in}{3.262217in}}%
\pgfpathlineto{\pgfqpoint{3.622074in}{3.250620in}}%
\pgfpathlineto{\pgfqpoint{3.628381in}{3.238898in}}%
\pgfpathlineto{\pgfqpoint{3.634689in}{3.227065in}}%
\pgfpathclose%
\pgfusepath{stroke,fill}%
\end{pgfscope}%
\begin{pgfscope}%
\pgfpathrectangle{\pgfqpoint{0.887500in}{0.275000in}}{\pgfqpoint{4.225000in}{4.225000in}}%
\pgfusepath{clip}%
\pgfsetbuttcap%
\pgfsetroundjoin%
\definecolor{currentfill}{rgb}{0.814576,0.883393,0.110347}%
\pgfsetfillcolor{currentfill}%
\pgfsetfillopacity{0.700000}%
\pgfsetlinewidth{0.501875pt}%
\definecolor{currentstroke}{rgb}{1.000000,1.000000,1.000000}%
\pgfsetstrokecolor{currentstroke}%
\pgfsetstrokeopacity{0.500000}%
\pgfsetdash{}{0pt}%
\pgfpathmoveto{\pgfqpoint{3.061758in}{3.416722in}}%
\pgfpathlineto{\pgfqpoint{3.073016in}{3.421099in}}%
\pgfpathlineto{\pgfqpoint{3.084269in}{3.424411in}}%
\pgfpathlineto{\pgfqpoint{3.095516in}{3.426933in}}%
\pgfpathlineto{\pgfqpoint{3.106756in}{3.428938in}}%
\pgfpathlineto{\pgfqpoint{3.117989in}{3.430702in}}%
\pgfpathlineto{\pgfqpoint{3.111802in}{3.438423in}}%
\pgfpathlineto{\pgfqpoint{3.105618in}{3.445734in}}%
\pgfpathlineto{\pgfqpoint{3.099438in}{3.452605in}}%
\pgfpathlineto{\pgfqpoint{3.093262in}{3.459005in}}%
\pgfpathlineto{\pgfqpoint{3.087091in}{3.464903in}}%
\pgfpathlineto{\pgfqpoint{3.075869in}{3.463055in}}%
\pgfpathlineto{\pgfqpoint{3.064640in}{3.461033in}}%
\pgfpathlineto{\pgfqpoint{3.053406in}{3.458535in}}%
\pgfpathlineto{\pgfqpoint{3.042166in}{3.455260in}}%
\pgfpathlineto{\pgfqpoint{3.030922in}{3.450907in}}%
\pgfpathlineto{\pgfqpoint{3.037080in}{3.444882in}}%
\pgfpathlineto{\pgfqpoint{3.043243in}{3.438431in}}%
\pgfpathlineto{\pgfqpoint{3.049410in}{3.431573in}}%
\pgfpathlineto{\pgfqpoint{3.055582in}{3.424330in}}%
\pgfpathclose%
\pgfusepath{stroke,fill}%
\end{pgfscope}%
\begin{pgfscope}%
\pgfpathrectangle{\pgfqpoint{0.887500in}{0.275000in}}{\pgfqpoint{4.225000in}{4.225000in}}%
\pgfusepath{clip}%
\pgfsetbuttcap%
\pgfsetroundjoin%
\definecolor{currentfill}{rgb}{0.246070,0.738910,0.452024}%
\pgfsetfillcolor{currentfill}%
\pgfsetfillopacity{0.700000}%
\pgfsetlinewidth{0.501875pt}%
\definecolor{currentstroke}{rgb}{1.000000,1.000000,1.000000}%
\pgfsetstrokecolor{currentstroke}%
\pgfsetstrokeopacity{0.500000}%
\pgfsetdash{}{0pt}%
\pgfpathmoveto{\pgfqpoint{2.898484in}{2.895100in}}%
\pgfpathlineto{\pgfqpoint{2.909674in}{2.923890in}}%
\pgfpathlineto{\pgfqpoint{2.920870in}{2.953474in}}%
\pgfpathlineto{\pgfqpoint{2.932074in}{2.983638in}}%
\pgfpathlineto{\pgfqpoint{2.943286in}{3.014168in}}%
\pgfpathlineto{\pgfqpoint{2.954508in}{3.044847in}}%
\pgfpathlineto{\pgfqpoint{2.948382in}{3.045427in}}%
\pgfpathlineto{\pgfqpoint{2.942263in}{3.045945in}}%
\pgfpathlineto{\pgfqpoint{2.936150in}{3.046372in}}%
\pgfpathlineto{\pgfqpoint{2.930044in}{3.046677in}}%
\pgfpathlineto{\pgfqpoint{2.923944in}{3.046830in}}%
\pgfpathlineto{\pgfqpoint{2.912769in}{3.015235in}}%
\pgfpathlineto{\pgfqpoint{2.901597in}{2.985474in}}%
\pgfpathlineto{\pgfqpoint{2.890427in}{2.957392in}}%
\pgfpathlineto{\pgfqpoint{2.879259in}{2.930837in}}%
\pgfpathlineto{\pgfqpoint{2.868092in}{2.905654in}}%
\pgfpathlineto{\pgfqpoint{2.874163in}{2.902701in}}%
\pgfpathlineto{\pgfqpoint{2.880237in}{2.900132in}}%
\pgfpathlineto{\pgfqpoint{2.886316in}{2.897984in}}%
\pgfpathlineto{\pgfqpoint{2.892398in}{2.896294in}}%
\pgfpathclose%
\pgfusepath{stroke,fill}%
\end{pgfscope}%
\begin{pgfscope}%
\pgfpathrectangle{\pgfqpoint{0.887500in}{0.275000in}}{\pgfqpoint{4.225000in}{4.225000in}}%
\pgfusepath{clip}%
\pgfsetbuttcap%
\pgfsetroundjoin%
\definecolor{currentfill}{rgb}{0.127568,0.566949,0.550556}%
\pgfsetfillcolor{currentfill}%
\pgfsetfillopacity{0.700000}%
\pgfsetlinewidth{0.501875pt}%
\definecolor{currentstroke}{rgb}{1.000000,1.000000,1.000000}%
\pgfsetstrokecolor{currentstroke}%
\pgfsetstrokeopacity{0.500000}%
\pgfsetdash{}{0pt}%
\pgfpathmoveto{\pgfqpoint{1.955524in}{2.569057in}}%
\pgfpathlineto{\pgfqpoint{1.967052in}{2.572362in}}%
\pgfpathlineto{\pgfqpoint{1.978575in}{2.575677in}}%
\pgfpathlineto{\pgfqpoint{1.990091in}{2.578999in}}%
\pgfpathlineto{\pgfqpoint{2.001602in}{2.582327in}}%
\pgfpathlineto{\pgfqpoint{2.013107in}{2.585657in}}%
\pgfpathlineto{\pgfqpoint{2.007290in}{2.593720in}}%
\pgfpathlineto{\pgfqpoint{2.001476in}{2.601772in}}%
\pgfpathlineto{\pgfqpoint{1.995667in}{2.609813in}}%
\pgfpathlineto{\pgfqpoint{1.989861in}{2.617843in}}%
\pgfpathlineto{\pgfqpoint{1.984060in}{2.625861in}}%
\pgfpathlineto{\pgfqpoint{1.972567in}{2.622529in}}%
\pgfpathlineto{\pgfqpoint{1.961069in}{2.619202in}}%
\pgfpathlineto{\pgfqpoint{1.949565in}{2.615882in}}%
\pgfpathlineto{\pgfqpoint{1.938055in}{2.612570in}}%
\pgfpathlineto{\pgfqpoint{1.926539in}{2.609269in}}%
\pgfpathlineto{\pgfqpoint{1.932328in}{2.601251in}}%
\pgfpathlineto{\pgfqpoint{1.938121in}{2.593220in}}%
\pgfpathlineto{\pgfqpoint{1.943918in}{2.585178in}}%
\pgfpathlineto{\pgfqpoint{1.949719in}{2.577124in}}%
\pgfpathclose%
\pgfusepath{stroke,fill}%
\end{pgfscope}%
\begin{pgfscope}%
\pgfpathrectangle{\pgfqpoint{0.887500in}{0.275000in}}{\pgfqpoint{4.225000in}{4.225000in}}%
\pgfusepath{clip}%
\pgfsetbuttcap%
\pgfsetroundjoin%
\definecolor{currentfill}{rgb}{0.119512,0.607464,0.540218}%
\pgfsetfillcolor{currentfill}%
\pgfsetfillopacity{0.700000}%
\pgfsetlinewidth{0.501875pt}%
\definecolor{currentstroke}{rgb}{1.000000,1.000000,1.000000}%
\pgfsetstrokecolor{currentstroke}%
\pgfsetstrokeopacity{0.500000}%
\pgfsetdash{}{0pt}%
\pgfpathmoveto{\pgfqpoint{1.406846in}{2.650852in}}%
\pgfpathlineto{\pgfqpoint{1.418506in}{2.654214in}}%
\pgfpathlineto{\pgfqpoint{1.430160in}{2.657566in}}%
\pgfpathlineto{\pgfqpoint{1.441810in}{2.660908in}}%
\pgfpathlineto{\pgfqpoint{1.453454in}{2.664241in}}%
\pgfpathlineto{\pgfqpoint{1.465092in}{2.667567in}}%
\pgfpathlineto{\pgfqpoint{1.459475in}{2.675172in}}%
\pgfpathlineto{\pgfqpoint{1.453862in}{2.682760in}}%
\pgfpathlineto{\pgfqpoint{1.448254in}{2.690331in}}%
\pgfpathlineto{\pgfqpoint{1.442650in}{2.697884in}}%
\pgfpathlineto{\pgfqpoint{1.437051in}{2.705420in}}%
\pgfpathlineto{\pgfqpoint{1.425426in}{2.702089in}}%
\pgfpathlineto{\pgfqpoint{1.413797in}{2.698750in}}%
\pgfpathlineto{\pgfqpoint{1.402162in}{2.695404in}}%
\pgfpathlineto{\pgfqpoint{1.390521in}{2.692048in}}%
\pgfpathlineto{\pgfqpoint{1.378876in}{2.688681in}}%
\pgfpathlineto{\pgfqpoint{1.384461in}{2.681153in}}%
\pgfpathlineto{\pgfqpoint{1.390050in}{2.673605in}}%
\pgfpathlineto{\pgfqpoint{1.395645in}{2.666038in}}%
\pgfpathlineto{\pgfqpoint{1.401243in}{2.658454in}}%
\pgfpathclose%
\pgfusepath{stroke,fill}%
\end{pgfscope}%
\begin{pgfscope}%
\pgfpathrectangle{\pgfqpoint{0.887500in}{0.275000in}}{\pgfqpoint{4.225000in}{4.225000in}}%
\pgfusepath{clip}%
\pgfsetbuttcap%
\pgfsetroundjoin%
\definecolor{currentfill}{rgb}{0.647257,0.858400,0.209861}%
\pgfsetfillcolor{currentfill}%
\pgfsetfillopacity{0.700000}%
\pgfsetlinewidth{0.501875pt}%
\definecolor{currentstroke}{rgb}{1.000000,1.000000,1.000000}%
\pgfsetstrokecolor{currentstroke}%
\pgfsetstrokeopacity{0.500000}%
\pgfsetdash{}{0pt}%
\pgfpathmoveto{\pgfqpoint{2.949284in}{3.233858in}}%
\pgfpathlineto{\pgfqpoint{2.960477in}{3.269322in}}%
\pgfpathlineto{\pgfqpoint{2.971692in}{3.300695in}}%
\pgfpathlineto{\pgfqpoint{2.982925in}{3.327278in}}%
\pgfpathlineto{\pgfqpoint{2.994173in}{3.349456in}}%
\pgfpathlineto{\pgfqpoint{3.005430in}{3.367702in}}%
\pgfpathlineto{\pgfqpoint{2.999272in}{3.374713in}}%
\pgfpathlineto{\pgfqpoint{2.993118in}{3.381435in}}%
\pgfpathlineto{\pgfqpoint{2.986969in}{3.387860in}}%
\pgfpathlineto{\pgfqpoint{2.980825in}{3.393978in}}%
\pgfpathlineto{\pgfqpoint{2.974687in}{3.399780in}}%
\pgfpathlineto{\pgfqpoint{2.963458in}{3.379673in}}%
\pgfpathlineto{\pgfqpoint{2.952246in}{3.354740in}}%
\pgfpathlineto{\pgfqpoint{2.941056in}{3.324329in}}%
\pgfpathlineto{\pgfqpoint{2.929895in}{3.287920in}}%
\pgfpathlineto{\pgfqpoint{2.918764in}{3.246537in}}%
\pgfpathlineto{\pgfqpoint{2.924855in}{3.244530in}}%
\pgfpathlineto{\pgfqpoint{2.930953in}{3.242260in}}%
\pgfpathlineto{\pgfqpoint{2.937057in}{3.239726in}}%
\pgfpathlineto{\pgfqpoint{2.943167in}{3.236925in}}%
\pgfpathclose%
\pgfusepath{stroke,fill}%
\end{pgfscope}%
\begin{pgfscope}%
\pgfpathrectangle{\pgfqpoint{0.887500in}{0.275000in}}{\pgfqpoint{4.225000in}{4.225000in}}%
\pgfusepath{clip}%
\pgfsetbuttcap%
\pgfsetroundjoin%
\definecolor{currentfill}{rgb}{0.162016,0.687316,0.499129}%
\pgfsetfillcolor{currentfill}%
\pgfsetfillopacity{0.700000}%
\pgfsetlinewidth{0.501875pt}%
\definecolor{currentstroke}{rgb}{1.000000,1.000000,1.000000}%
\pgfsetstrokecolor{currentstroke}%
\pgfsetstrokeopacity{0.500000}%
\pgfsetdash{}{0pt}%
\pgfpathmoveto{\pgfqpoint{4.281907in}{2.811940in}}%
\pgfpathlineto{\pgfqpoint{4.292869in}{2.815435in}}%
\pgfpathlineto{\pgfqpoint{4.303827in}{2.818941in}}%
\pgfpathlineto{\pgfqpoint{4.314780in}{2.822458in}}%
\pgfpathlineto{\pgfqpoint{4.325728in}{2.825989in}}%
\pgfpathlineto{\pgfqpoint{4.319329in}{2.840058in}}%
\pgfpathlineto{\pgfqpoint{4.312932in}{2.854120in}}%
\pgfpathlineto{\pgfqpoint{4.306538in}{2.868176in}}%
\pgfpathlineto{\pgfqpoint{4.300147in}{2.882231in}}%
\pgfpathlineto{\pgfqpoint{4.293757in}{2.896275in}}%
\pgfpathlineto{\pgfqpoint{4.282811in}{2.892746in}}%
\pgfpathlineto{\pgfqpoint{4.271860in}{2.889227in}}%
\pgfpathlineto{\pgfqpoint{4.260903in}{2.885716in}}%
\pgfpathlineto{\pgfqpoint{4.249943in}{2.882211in}}%
\pgfpathlineto{\pgfqpoint{4.256330in}{2.868143in}}%
\pgfpathlineto{\pgfqpoint{4.262720in}{2.854078in}}%
\pgfpathlineto{\pgfqpoint{4.269113in}{2.840025in}}%
\pgfpathlineto{\pgfqpoint{4.275508in}{2.825981in}}%
\pgfpathclose%
\pgfusepath{stroke,fill}%
\end{pgfscope}%
\begin{pgfscope}%
\pgfpathrectangle{\pgfqpoint{0.887500in}{0.275000in}}{\pgfqpoint{4.225000in}{4.225000in}}%
\pgfusepath{clip}%
\pgfsetbuttcap%
\pgfsetroundjoin%
\definecolor{currentfill}{rgb}{0.143343,0.522773,0.556295}%
\pgfsetfillcolor{currentfill}%
\pgfsetfillopacity{0.700000}%
\pgfsetlinewidth{0.501875pt}%
\definecolor{currentstroke}{rgb}{1.000000,1.000000,1.000000}%
\pgfsetstrokecolor{currentstroke}%
\pgfsetstrokeopacity{0.500000}%
\pgfsetdash{}{0pt}%
\pgfpathmoveto{\pgfqpoint{2.735000in}{2.448090in}}%
\pgfpathlineto{\pgfqpoint{2.746329in}{2.451729in}}%
\pgfpathlineto{\pgfqpoint{2.757637in}{2.457562in}}%
\pgfpathlineto{\pgfqpoint{2.768917in}{2.466483in}}%
\pgfpathlineto{\pgfqpoint{2.780168in}{2.479386in}}%
\pgfpathlineto{\pgfqpoint{2.791386in}{2.497020in}}%
\pgfpathlineto{\pgfqpoint{2.785266in}{2.511037in}}%
\pgfpathlineto{\pgfqpoint{2.779147in}{2.525076in}}%
\pgfpathlineto{\pgfqpoint{2.773032in}{2.538996in}}%
\pgfpathlineto{\pgfqpoint{2.766921in}{2.552658in}}%
\pgfpathlineto{\pgfqpoint{2.760815in}{2.565922in}}%
\pgfpathlineto{\pgfqpoint{2.749579in}{2.552318in}}%
\pgfpathlineto{\pgfqpoint{2.738331in}{2.540373in}}%
\pgfpathlineto{\pgfqpoint{2.727069in}{2.529891in}}%
\pgfpathlineto{\pgfqpoint{2.715796in}{2.520641in}}%
\pgfpathlineto{\pgfqpoint{2.704512in}{2.512396in}}%
\pgfpathlineto{\pgfqpoint{2.710601in}{2.499879in}}%
\pgfpathlineto{\pgfqpoint{2.716696in}{2.487072in}}%
\pgfpathlineto{\pgfqpoint{2.722795in}{2.474093in}}%
\pgfpathlineto{\pgfqpoint{2.728896in}{2.461059in}}%
\pgfpathclose%
\pgfusepath{stroke,fill}%
\end{pgfscope}%
\begin{pgfscope}%
\pgfpathrectangle{\pgfqpoint{0.887500in}{0.275000in}}{\pgfqpoint{4.225000in}{4.225000in}}%
\pgfusepath{clip}%
\pgfsetbuttcap%
\pgfsetroundjoin%
\definecolor{currentfill}{rgb}{0.136408,0.541173,0.554483}%
\pgfsetfillcolor{currentfill}%
\pgfsetfillopacity{0.700000}%
\pgfsetlinewidth{0.501875pt}%
\definecolor{currentstroke}{rgb}{1.000000,1.000000,1.000000}%
\pgfsetstrokecolor{currentstroke}%
\pgfsetstrokeopacity{0.500000}%
\pgfsetdash{}{0pt}%
\pgfpathmoveto{\pgfqpoint{2.273336in}{2.512356in}}%
\pgfpathlineto{\pgfqpoint{2.284788in}{2.515666in}}%
\pgfpathlineto{\pgfqpoint{2.296234in}{2.518995in}}%
\pgfpathlineto{\pgfqpoint{2.307675in}{2.522350in}}%
\pgfpathlineto{\pgfqpoint{2.319108in}{2.525738in}}%
\pgfpathlineto{\pgfqpoint{2.330536in}{2.529167in}}%
\pgfpathlineto{\pgfqpoint{2.324608in}{2.537443in}}%
\pgfpathlineto{\pgfqpoint{2.318684in}{2.545708in}}%
\pgfpathlineto{\pgfqpoint{2.312763in}{2.553964in}}%
\pgfpathlineto{\pgfqpoint{2.306847in}{2.562212in}}%
\pgfpathlineto{\pgfqpoint{2.300935in}{2.570454in}}%
\pgfpathlineto{\pgfqpoint{2.289519in}{2.567024in}}%
\pgfpathlineto{\pgfqpoint{2.278098in}{2.563634in}}%
\pgfpathlineto{\pgfqpoint{2.266670in}{2.560279in}}%
\pgfpathlineto{\pgfqpoint{2.255235in}{2.556951in}}%
\pgfpathlineto{\pgfqpoint{2.243795in}{2.553641in}}%
\pgfpathlineto{\pgfqpoint{2.249695in}{2.545410in}}%
\pgfpathlineto{\pgfqpoint{2.255599in}{2.537167in}}%
\pgfpathlineto{\pgfqpoint{2.261508in}{2.528910in}}%
\pgfpathlineto{\pgfqpoint{2.267420in}{2.520640in}}%
\pgfpathclose%
\pgfusepath{stroke,fill}%
\end{pgfscope}%
\begin{pgfscope}%
\pgfpathrectangle{\pgfqpoint{0.887500in}{0.275000in}}{\pgfqpoint{4.225000in}{4.225000in}}%
\pgfusepath{clip}%
\pgfsetbuttcap%
\pgfsetroundjoin%
\definecolor{currentfill}{rgb}{0.202219,0.715272,0.476084}%
\pgfsetfillcolor{currentfill}%
\pgfsetfillopacity{0.700000}%
\pgfsetlinewidth{0.501875pt}%
\definecolor{currentstroke}{rgb}{1.000000,1.000000,1.000000}%
\pgfsetstrokecolor{currentstroke}%
\pgfsetstrokeopacity{0.500000}%
\pgfsetdash{}{0pt}%
\pgfpathmoveto{\pgfqpoint{4.195062in}{2.864749in}}%
\pgfpathlineto{\pgfqpoint{4.206048in}{2.868238in}}%
\pgfpathlineto{\pgfqpoint{4.217030in}{2.871727in}}%
\pgfpathlineto{\pgfqpoint{4.228006in}{2.875218in}}%
\pgfpathlineto{\pgfqpoint{4.238977in}{2.878713in}}%
\pgfpathlineto{\pgfqpoint{4.249943in}{2.882211in}}%
\pgfpathlineto{\pgfqpoint{4.243557in}{2.896268in}}%
\pgfpathlineto{\pgfqpoint{4.237174in}{2.910296in}}%
\pgfpathlineto{\pgfqpoint{4.230791in}{2.924280in}}%
\pgfpathlineto{\pgfqpoint{4.224410in}{2.938204in}}%
\pgfpathlineto{\pgfqpoint{4.218028in}{2.952053in}}%
\pgfpathlineto{\pgfqpoint{4.207065in}{2.948582in}}%
\pgfpathlineto{\pgfqpoint{4.196097in}{2.945105in}}%
\pgfpathlineto{\pgfqpoint{4.185123in}{2.941621in}}%
\pgfpathlineto{\pgfqpoint{4.174144in}{2.938128in}}%
\pgfpathlineto{\pgfqpoint{4.163159in}{2.934626in}}%
\pgfpathlineto{\pgfqpoint{4.169537in}{2.920757in}}%
\pgfpathlineto{\pgfqpoint{4.175917in}{2.906819in}}%
\pgfpathlineto{\pgfqpoint{4.182297in}{2.892828in}}%
\pgfpathlineto{\pgfqpoint{4.188679in}{2.878800in}}%
\pgfpathclose%
\pgfusepath{stroke,fill}%
\end{pgfscope}%
\begin{pgfscope}%
\pgfpathrectangle{\pgfqpoint{0.887500in}{0.275000in}}{\pgfqpoint{4.225000in}{4.225000in}}%
\pgfusepath{clip}%
\pgfsetbuttcap%
\pgfsetroundjoin%
\definecolor{currentfill}{rgb}{0.783315,0.879285,0.125405}%
\pgfsetfillcolor{currentfill}%
\pgfsetfillopacity{0.700000}%
\pgfsetlinewidth{0.501875pt}%
\definecolor{currentstroke}{rgb}{1.000000,1.000000,1.000000}%
\pgfsetstrokecolor{currentstroke}%
\pgfsetstrokeopacity{0.500000}%
\pgfsetdash{}{0pt}%
\pgfpathmoveto{\pgfqpoint{3.148977in}{3.387050in}}%
\pgfpathlineto{\pgfqpoint{3.160214in}{3.388847in}}%
\pgfpathlineto{\pgfqpoint{3.171445in}{3.390839in}}%
\pgfpathlineto{\pgfqpoint{3.182670in}{3.393032in}}%
\pgfpathlineto{\pgfqpoint{3.193890in}{3.395432in}}%
\pgfpathlineto{\pgfqpoint{3.205105in}{3.398042in}}%
\pgfpathlineto{\pgfqpoint{3.198892in}{3.407247in}}%
\pgfpathlineto{\pgfqpoint{3.192682in}{3.416183in}}%
\pgfpathlineto{\pgfqpoint{3.186474in}{3.424836in}}%
\pgfpathlineto{\pgfqpoint{3.180270in}{3.433198in}}%
\pgfpathlineto{\pgfqpoint{3.174069in}{3.441261in}}%
\pgfpathlineto{\pgfqpoint{3.162864in}{3.438792in}}%
\pgfpathlineto{\pgfqpoint{3.151654in}{3.436517in}}%
\pgfpathlineto{\pgfqpoint{3.140438in}{3.434422in}}%
\pgfpathlineto{\pgfqpoint{3.129217in}{3.432490in}}%
\pgfpathlineto{\pgfqpoint{3.117989in}{3.430702in}}%
\pgfpathlineto{\pgfqpoint{3.124180in}{3.422604in}}%
\pgfpathlineto{\pgfqpoint{3.130375in}{3.414159in}}%
\pgfpathlineto{\pgfqpoint{3.136572in}{3.405398in}}%
\pgfpathlineto{\pgfqpoint{3.142773in}{3.396352in}}%
\pgfpathclose%
\pgfusepath{stroke,fill}%
\end{pgfscope}%
\begin{pgfscope}%
\pgfpathrectangle{\pgfqpoint{0.887500in}{0.275000in}}{\pgfqpoint{4.225000in}{4.225000in}}%
\pgfusepath{clip}%
\pgfsetbuttcap%
\pgfsetroundjoin%
\definecolor{currentfill}{rgb}{0.144759,0.519093,0.556572}%
\pgfsetfillcolor{currentfill}%
\pgfsetfillopacity{0.700000}%
\pgfsetlinewidth{0.501875pt}%
\definecolor{currentstroke}{rgb}{1.000000,1.000000,1.000000}%
\pgfsetstrokecolor{currentstroke}%
\pgfsetstrokeopacity{0.500000}%
\pgfsetdash{}{0pt}%
\pgfpathmoveto{\pgfqpoint{2.591244in}{2.453241in}}%
\pgfpathlineto{\pgfqpoint{2.602608in}{2.457490in}}%
\pgfpathlineto{\pgfqpoint{2.613962in}{2.462222in}}%
\pgfpathlineto{\pgfqpoint{2.625306in}{2.467387in}}%
\pgfpathlineto{\pgfqpoint{2.636642in}{2.472938in}}%
\pgfpathlineto{\pgfqpoint{2.647970in}{2.478823in}}%
\pgfpathlineto{\pgfqpoint{2.641929in}{2.487754in}}%
\pgfpathlineto{\pgfqpoint{2.635891in}{2.496657in}}%
\pgfpathlineto{\pgfqpoint{2.629858in}{2.505519in}}%
\pgfpathlineto{\pgfqpoint{2.623830in}{2.514328in}}%
\pgfpathlineto{\pgfqpoint{2.617805in}{2.523073in}}%
\pgfpathlineto{\pgfqpoint{2.606499in}{2.516420in}}%
\pgfpathlineto{\pgfqpoint{2.595180in}{2.510466in}}%
\pgfpathlineto{\pgfqpoint{2.583851in}{2.505145in}}%
\pgfpathlineto{\pgfqpoint{2.572509in}{2.500389in}}%
\pgfpathlineto{\pgfqpoint{2.561157in}{2.496130in}}%
\pgfpathlineto{\pgfqpoint{2.567166in}{2.487581in}}%
\pgfpathlineto{\pgfqpoint{2.573180in}{2.479006in}}%
\pgfpathlineto{\pgfqpoint{2.579197in}{2.470416in}}%
\pgfpathlineto{\pgfqpoint{2.585219in}{2.461824in}}%
\pgfpathclose%
\pgfusepath{stroke,fill}%
\end{pgfscope}%
\begin{pgfscope}%
\pgfpathrectangle{\pgfqpoint{0.887500in}{0.275000in}}{\pgfqpoint{4.225000in}{4.225000in}}%
\pgfusepath{clip}%
\pgfsetbuttcap%
\pgfsetroundjoin%
\definecolor{currentfill}{rgb}{0.122606,0.585371,0.546557}%
\pgfsetfillcolor{currentfill}%
\pgfsetfillopacity{0.700000}%
\pgfsetlinewidth{0.501875pt}%
\definecolor{currentstroke}{rgb}{1.000000,1.000000,1.000000}%
\pgfsetstrokecolor{currentstroke}%
\pgfsetstrokeopacity{0.500000}%
\pgfsetdash{}{0pt}%
\pgfpathmoveto{\pgfqpoint{1.724408in}{2.600362in}}%
\pgfpathlineto{\pgfqpoint{1.735998in}{2.603622in}}%
\pgfpathlineto{\pgfqpoint{1.747582in}{2.606876in}}%
\pgfpathlineto{\pgfqpoint{1.759161in}{2.610124in}}%
\pgfpathlineto{\pgfqpoint{1.770734in}{2.613369in}}%
\pgfpathlineto{\pgfqpoint{1.782301in}{2.616610in}}%
\pgfpathlineto{\pgfqpoint{1.776564in}{2.624527in}}%
\pgfpathlineto{\pgfqpoint{1.770830in}{2.632425in}}%
\pgfpathlineto{\pgfqpoint{1.765102in}{2.640302in}}%
\pgfpathlineto{\pgfqpoint{1.759377in}{2.648158in}}%
\pgfpathlineto{\pgfqpoint{1.753657in}{2.655994in}}%
\pgfpathlineto{\pgfqpoint{1.742103in}{2.652746in}}%
\pgfpathlineto{\pgfqpoint{1.730543in}{2.649497in}}%
\pgfpathlineto{\pgfqpoint{1.718978in}{2.646244in}}%
\pgfpathlineto{\pgfqpoint{1.707407in}{2.642988in}}%
\pgfpathlineto{\pgfqpoint{1.695830in}{2.639729in}}%
\pgfpathlineto{\pgfqpoint{1.701537in}{2.631900in}}%
\pgfpathlineto{\pgfqpoint{1.707248in}{2.624049in}}%
\pgfpathlineto{\pgfqpoint{1.712963in}{2.616176in}}%
\pgfpathlineto{\pgfqpoint{1.718684in}{2.608280in}}%
\pgfpathclose%
\pgfusepath{stroke,fill}%
\end{pgfscope}%
\begin{pgfscope}%
\pgfpathrectangle{\pgfqpoint{0.887500in}{0.275000in}}{\pgfqpoint{4.225000in}{4.225000in}}%
\pgfusepath{clip}%
\pgfsetbuttcap%
\pgfsetroundjoin%
\definecolor{currentfill}{rgb}{0.246070,0.738910,0.452024}%
\pgfsetfillcolor{currentfill}%
\pgfsetfillopacity{0.700000}%
\pgfsetlinewidth{0.501875pt}%
\definecolor{currentstroke}{rgb}{1.000000,1.000000,1.000000}%
\pgfsetstrokecolor{currentstroke}%
\pgfsetstrokeopacity{0.500000}%
\pgfsetdash{}{0pt}%
\pgfpathmoveto{\pgfqpoint{4.108153in}{2.916939in}}%
\pgfpathlineto{\pgfqpoint{4.119166in}{2.920504in}}%
\pgfpathlineto{\pgfqpoint{4.130172in}{2.924054in}}%
\pgfpathlineto{\pgfqpoint{4.141173in}{2.927590in}}%
\pgfpathlineto{\pgfqpoint{4.152169in}{2.931114in}}%
\pgfpathlineto{\pgfqpoint{4.163159in}{2.934626in}}%
\pgfpathlineto{\pgfqpoint{4.156781in}{2.948413in}}%
\pgfpathlineto{\pgfqpoint{4.150402in}{2.962099in}}%
\pgfpathlineto{\pgfqpoint{4.144023in}{2.975672in}}%
\pgfpathlineto{\pgfqpoint{4.137643in}{2.989114in}}%
\pgfpathlineto{\pgfqpoint{4.131262in}{3.002419in}}%
\pgfpathlineto{\pgfqpoint{4.120277in}{2.998968in}}%
\pgfpathlineto{\pgfqpoint{4.109285in}{2.995511in}}%
\pgfpathlineto{\pgfqpoint{4.098289in}{2.992049in}}%
\pgfpathlineto{\pgfqpoint{4.087287in}{2.988580in}}%
\pgfpathlineto{\pgfqpoint{4.076280in}{2.985104in}}%
\pgfpathlineto{\pgfqpoint{4.082654in}{2.971651in}}%
\pgfpathlineto{\pgfqpoint{4.089028in}{2.958100in}}%
\pgfpathlineto{\pgfqpoint{4.095403in}{2.944457in}}%
\pgfpathlineto{\pgfqpoint{4.101778in}{2.930733in}}%
\pgfpathclose%
\pgfusepath{stroke,fill}%
\end{pgfscope}%
\begin{pgfscope}%
\pgfpathrectangle{\pgfqpoint{0.887500in}{0.275000in}}{\pgfqpoint{4.225000in}{4.225000in}}%
\pgfusepath{clip}%
\pgfsetbuttcap%
\pgfsetroundjoin%
\definecolor{currentfill}{rgb}{0.129933,0.559582,0.551864}%
\pgfsetfillcolor{currentfill}%
\pgfsetfillopacity{0.700000}%
\pgfsetlinewidth{0.501875pt}%
\definecolor{currentstroke}{rgb}{1.000000,1.000000,1.000000}%
\pgfsetstrokecolor{currentstroke}%
\pgfsetstrokeopacity{0.500000}%
\pgfsetdash{}{0pt}%
\pgfpathmoveto{\pgfqpoint{2.042257in}{2.545150in}}%
\pgfpathlineto{\pgfqpoint{2.053769in}{2.548477in}}%
\pgfpathlineto{\pgfqpoint{2.065275in}{2.551798in}}%
\pgfpathlineto{\pgfqpoint{2.076776in}{2.555112in}}%
\pgfpathlineto{\pgfqpoint{2.088272in}{2.558419in}}%
\pgfpathlineto{\pgfqpoint{2.099762in}{2.561719in}}%
\pgfpathlineto{\pgfqpoint{2.093911in}{2.569855in}}%
\pgfpathlineto{\pgfqpoint{2.088065in}{2.577977in}}%
\pgfpathlineto{\pgfqpoint{2.082222in}{2.586086in}}%
\pgfpathlineto{\pgfqpoint{2.076384in}{2.594183in}}%
\pgfpathlineto{\pgfqpoint{2.070549in}{2.602266in}}%
\pgfpathlineto{\pgfqpoint{2.059072in}{2.598957in}}%
\pgfpathlineto{\pgfqpoint{2.047589in}{2.595639in}}%
\pgfpathlineto{\pgfqpoint{2.036101in}{2.592315in}}%
\pgfpathlineto{\pgfqpoint{2.024607in}{2.588987in}}%
\pgfpathlineto{\pgfqpoint{2.013107in}{2.585657in}}%
\pgfpathlineto{\pgfqpoint{2.018929in}{2.577582in}}%
\pgfpathlineto{\pgfqpoint{2.024755in}{2.569494in}}%
\pgfpathlineto{\pgfqpoint{2.030585in}{2.561394in}}%
\pgfpathlineto{\pgfqpoint{2.036419in}{2.553279in}}%
\pgfpathclose%
\pgfusepath{stroke,fill}%
\end{pgfscope}%
\begin{pgfscope}%
\pgfpathrectangle{\pgfqpoint{0.887500in}{0.275000in}}{\pgfqpoint{4.225000in}{4.225000in}}%
\pgfusepath{clip}%
\pgfsetbuttcap%
\pgfsetroundjoin%
\definecolor{currentfill}{rgb}{0.741388,0.873449,0.149561}%
\pgfsetfillcolor{currentfill}%
\pgfsetfillopacity{0.700000}%
\pgfsetlinewidth{0.501875pt}%
\definecolor{currentstroke}{rgb}{1.000000,1.000000,1.000000}%
\pgfsetstrokecolor{currentstroke}%
\pgfsetstrokeopacity{0.500000}%
\pgfsetdash{}{0pt}%
\pgfpathmoveto{\pgfqpoint{3.236213in}{3.349142in}}%
\pgfpathlineto{\pgfqpoint{3.247432in}{3.352047in}}%
\pgfpathlineto{\pgfqpoint{3.258646in}{3.355134in}}%
\pgfpathlineto{\pgfqpoint{3.269855in}{3.358394in}}%
\pgfpathlineto{\pgfqpoint{3.281060in}{3.361809in}}%
\pgfpathlineto{\pgfqpoint{3.292260in}{3.365358in}}%
\pgfpathlineto{\pgfqpoint{3.286024in}{3.375456in}}%
\pgfpathlineto{\pgfqpoint{3.279792in}{3.385437in}}%
\pgfpathlineto{\pgfqpoint{3.273562in}{3.395262in}}%
\pgfpathlineto{\pgfqpoint{3.267334in}{3.404893in}}%
\pgfpathlineto{\pgfqpoint{3.261109in}{3.414291in}}%
\pgfpathlineto{\pgfqpoint{3.249917in}{3.410647in}}%
\pgfpathlineto{\pgfqpoint{3.238721in}{3.407178in}}%
\pgfpathlineto{\pgfqpoint{3.227520in}{3.403912in}}%
\pgfpathlineto{\pgfqpoint{3.216315in}{3.400867in}}%
\pgfpathlineto{\pgfqpoint{3.205105in}{3.398042in}}%
\pgfpathlineto{\pgfqpoint{3.211320in}{3.388599in}}%
\pgfpathlineto{\pgfqpoint{3.217539in}{3.378954in}}%
\pgfpathlineto{\pgfqpoint{3.223761in}{3.369140in}}%
\pgfpathlineto{\pgfqpoint{3.229985in}{3.359191in}}%
\pgfpathclose%
\pgfusepath{stroke,fill}%
\end{pgfscope}%
\begin{pgfscope}%
\pgfpathrectangle{\pgfqpoint{0.887500in}{0.275000in}}{\pgfqpoint{4.225000in}{4.225000in}}%
\pgfusepath{clip}%
\pgfsetbuttcap%
\pgfsetroundjoin%
\definecolor{currentfill}{rgb}{0.120092,0.600104,0.542530}%
\pgfsetfillcolor{currentfill}%
\pgfsetfillopacity{0.700000}%
\pgfsetlinewidth{0.501875pt}%
\definecolor{currentstroke}{rgb}{1.000000,1.000000,1.000000}%
\pgfsetstrokecolor{currentstroke}%
\pgfsetstrokeopacity{0.500000}%
\pgfsetdash{}{0pt}%
\pgfpathmoveto{\pgfqpoint{1.493242in}{2.629294in}}%
\pgfpathlineto{\pgfqpoint{1.504889in}{2.632607in}}%
\pgfpathlineto{\pgfqpoint{1.516531in}{2.635912in}}%
\pgfpathlineto{\pgfqpoint{1.528167in}{2.639211in}}%
\pgfpathlineto{\pgfqpoint{1.539798in}{2.642505in}}%
\pgfpathlineto{\pgfqpoint{1.551423in}{2.645794in}}%
\pgfpathlineto{\pgfqpoint{1.545770in}{2.653490in}}%
\pgfpathlineto{\pgfqpoint{1.540122in}{2.661167in}}%
\pgfpathlineto{\pgfqpoint{1.534478in}{2.668828in}}%
\pgfpathlineto{\pgfqpoint{1.528838in}{2.676472in}}%
\pgfpathlineto{\pgfqpoint{1.523203in}{2.684100in}}%
\pgfpathlineto{\pgfqpoint{1.511592in}{2.680803in}}%
\pgfpathlineto{\pgfqpoint{1.499975in}{2.677502in}}%
\pgfpathlineto{\pgfqpoint{1.488353in}{2.674196in}}%
\pgfpathlineto{\pgfqpoint{1.476725in}{2.670884in}}%
\pgfpathlineto{\pgfqpoint{1.465092in}{2.667567in}}%
\pgfpathlineto{\pgfqpoint{1.470713in}{2.659945in}}%
\pgfpathlineto{\pgfqpoint{1.476339in}{2.652308in}}%
\pgfpathlineto{\pgfqpoint{1.481969in}{2.644654in}}%
\pgfpathlineto{\pgfqpoint{1.487603in}{2.636984in}}%
\pgfpathclose%
\pgfusepath{stroke,fill}%
\end{pgfscope}%
\begin{pgfscope}%
\pgfpathrectangle{\pgfqpoint{0.887500in}{0.275000in}}{\pgfqpoint{4.225000in}{4.225000in}}%
\pgfusepath{clip}%
\pgfsetbuttcap%
\pgfsetroundjoin%
\definecolor{currentfill}{rgb}{0.127568,0.566949,0.550556}%
\pgfsetfillcolor{currentfill}%
\pgfsetfillopacity{0.700000}%
\pgfsetlinewidth{0.501875pt}%
\definecolor{currentstroke}{rgb}{1.000000,1.000000,1.000000}%
\pgfsetstrokecolor{currentstroke}%
\pgfsetstrokeopacity{0.500000}%
\pgfsetdash{}{0pt}%
\pgfpathmoveto{\pgfqpoint{2.791386in}{2.497020in}}%
\pgfpathlineto{\pgfqpoint{2.802580in}{2.518994in}}%
\pgfpathlineto{\pgfqpoint{2.813757in}{2.544397in}}%
\pgfpathlineto{\pgfqpoint{2.824927in}{2.572314in}}%
\pgfpathlineto{\pgfqpoint{2.836097in}{2.601827in}}%
\pgfpathlineto{\pgfqpoint{2.847272in}{2.632014in}}%
\pgfpathlineto{\pgfqpoint{2.841180in}{2.636827in}}%
\pgfpathlineto{\pgfqpoint{2.835095in}{2.641335in}}%
\pgfpathlineto{\pgfqpoint{2.829016in}{2.645683in}}%
\pgfpathlineto{\pgfqpoint{2.822943in}{2.650018in}}%
\pgfpathlineto{\pgfqpoint{2.816873in}{2.654485in}}%
\pgfpathlineto{\pgfqpoint{2.805670in}{2.634613in}}%
\pgfpathlineto{\pgfqpoint{2.794464in}{2.615647in}}%
\pgfpathlineto{\pgfqpoint{2.783255in}{2.597760in}}%
\pgfpathlineto{\pgfqpoint{2.772039in}{2.581127in}}%
\pgfpathlineto{\pgfqpoint{2.760815in}{2.565922in}}%
\pgfpathlineto{\pgfqpoint{2.766921in}{2.552658in}}%
\pgfpathlineto{\pgfqpoint{2.773032in}{2.538996in}}%
\pgfpathlineto{\pgfqpoint{2.779147in}{2.525076in}}%
\pgfpathlineto{\pgfqpoint{2.785266in}{2.511037in}}%
\pgfpathclose%
\pgfusepath{stroke,fill}%
\end{pgfscope}%
\begin{pgfscope}%
\pgfpathrectangle{\pgfqpoint{0.887500in}{0.275000in}}{\pgfqpoint{4.225000in}{4.225000in}}%
\pgfusepath{clip}%
\pgfsetbuttcap%
\pgfsetroundjoin%
\definecolor{currentfill}{rgb}{0.139147,0.533812,0.555298}%
\pgfsetfillcolor{currentfill}%
\pgfsetfillopacity{0.700000}%
\pgfsetlinewidth{0.501875pt}%
\definecolor{currentstroke}{rgb}{1.000000,1.000000,1.000000}%
\pgfsetstrokecolor{currentstroke}%
\pgfsetstrokeopacity{0.500000}%
\pgfsetdash{}{0pt}%
\pgfpathmoveto{\pgfqpoint{2.360238in}{2.487574in}}%
\pgfpathlineto{\pgfqpoint{2.371671in}{2.491029in}}%
\pgfpathlineto{\pgfqpoint{2.383099in}{2.494521in}}%
\pgfpathlineto{\pgfqpoint{2.394520in}{2.498041in}}%
\pgfpathlineto{\pgfqpoint{2.405936in}{2.501568in}}%
\pgfpathlineto{\pgfqpoint{2.417346in}{2.505072in}}%
\pgfpathlineto{\pgfqpoint{2.411385in}{2.513456in}}%
\pgfpathlineto{\pgfqpoint{2.405428in}{2.521833in}}%
\pgfpathlineto{\pgfqpoint{2.399475in}{2.530201in}}%
\pgfpathlineto{\pgfqpoint{2.393526in}{2.538557in}}%
\pgfpathlineto{\pgfqpoint{2.387581in}{2.546897in}}%
\pgfpathlineto{\pgfqpoint{2.376183in}{2.543341in}}%
\pgfpathlineto{\pgfqpoint{2.364780in}{2.539754in}}%
\pgfpathlineto{\pgfqpoint{2.353372in}{2.536175in}}%
\pgfpathlineto{\pgfqpoint{2.341957in}{2.532644in}}%
\pgfpathlineto{\pgfqpoint{2.330536in}{2.529167in}}%
\pgfpathlineto{\pgfqpoint{2.336468in}{2.520879in}}%
\pgfpathlineto{\pgfqpoint{2.342404in}{2.512577in}}%
\pgfpathlineto{\pgfqpoint{2.348345in}{2.504260in}}%
\pgfpathlineto{\pgfqpoint{2.354289in}{2.495926in}}%
\pgfpathclose%
\pgfusepath{stroke,fill}%
\end{pgfscope}%
\begin{pgfscope}%
\pgfpathrectangle{\pgfqpoint{0.887500in}{0.275000in}}{\pgfqpoint{4.225000in}{4.225000in}}%
\pgfusepath{clip}%
\pgfsetbuttcap%
\pgfsetroundjoin%
\definecolor{currentfill}{rgb}{0.288921,0.758394,0.428426}%
\pgfsetfillcolor{currentfill}%
\pgfsetfillopacity{0.700000}%
\pgfsetlinewidth{0.501875pt}%
\definecolor{currentstroke}{rgb}{1.000000,1.000000,1.000000}%
\pgfsetstrokecolor{currentstroke}%
\pgfsetstrokeopacity{0.500000}%
\pgfsetdash{}{0pt}%
\pgfpathmoveto{\pgfqpoint{4.021162in}{2.967565in}}%
\pgfpathlineto{\pgfqpoint{4.032197in}{2.971101in}}%
\pgfpathlineto{\pgfqpoint{4.043226in}{2.974621in}}%
\pgfpathlineto{\pgfqpoint{4.054249in}{2.978127in}}%
\pgfpathlineto{\pgfqpoint{4.065267in}{2.981620in}}%
\pgfpathlineto{\pgfqpoint{4.076280in}{2.985104in}}%
\pgfpathlineto{\pgfqpoint{4.069906in}{2.998463in}}%
\pgfpathlineto{\pgfqpoint{4.063533in}{3.011730in}}%
\pgfpathlineto{\pgfqpoint{4.057160in}{3.024912in}}%
\pgfpathlineto{\pgfqpoint{4.050787in}{3.038011in}}%
\pgfpathlineto{\pgfqpoint{4.044416in}{3.051031in}}%
\pgfpathlineto{\pgfqpoint{4.033414in}{3.047838in}}%
\pgfpathlineto{\pgfqpoint{4.022406in}{3.044647in}}%
\pgfpathlineto{\pgfqpoint{4.011393in}{3.041447in}}%
\pgfpathlineto{\pgfqpoint{4.000374in}{3.038226in}}%
\pgfpathlineto{\pgfqpoint{3.989349in}{3.034972in}}%
\pgfpathlineto{\pgfqpoint{3.995710in}{3.021644in}}%
\pgfpathlineto{\pgfqpoint{4.002072in}{3.008229in}}%
\pgfpathlineto{\pgfqpoint{4.008434in}{2.994737in}}%
\pgfpathlineto{\pgfqpoint{4.014797in}{2.981179in}}%
\pgfpathclose%
\pgfusepath{stroke,fill}%
\end{pgfscope}%
\begin{pgfscope}%
\pgfpathrectangle{\pgfqpoint{0.887500in}{0.275000in}}{\pgfqpoint{4.225000in}{4.225000in}}%
\pgfusepath{clip}%
\pgfsetbuttcap%
\pgfsetroundjoin%
\definecolor{currentfill}{rgb}{0.699415,0.867117,0.175971}%
\pgfsetfillcolor{currentfill}%
\pgfsetfillopacity{0.700000}%
\pgfsetlinewidth{0.501875pt}%
\definecolor{currentstroke}{rgb}{1.000000,1.000000,1.000000}%
\pgfsetstrokecolor{currentstroke}%
\pgfsetstrokeopacity{0.500000}%
\pgfsetdash{}{0pt}%
\pgfpathmoveto{\pgfqpoint{3.323488in}{3.314456in}}%
\pgfpathlineto{\pgfqpoint{3.334690in}{3.317805in}}%
\pgfpathlineto{\pgfqpoint{3.345886in}{3.321186in}}%
\pgfpathlineto{\pgfqpoint{3.357078in}{3.324589in}}%
\pgfpathlineto{\pgfqpoint{3.368264in}{3.328004in}}%
\pgfpathlineto{\pgfqpoint{3.379445in}{3.331420in}}%
\pgfpathlineto{\pgfqpoint{3.373188in}{3.342023in}}%
\pgfpathlineto{\pgfqpoint{3.366934in}{3.352651in}}%
\pgfpathlineto{\pgfqpoint{3.360684in}{3.363271in}}%
\pgfpathlineto{\pgfqpoint{3.354436in}{3.373843in}}%
\pgfpathlineto{\pgfqpoint{3.348192in}{3.384330in}}%
\pgfpathlineto{\pgfqpoint{3.337015in}{3.380445in}}%
\pgfpathlineto{\pgfqpoint{3.325833in}{3.376584in}}%
\pgfpathlineto{\pgfqpoint{3.314647in}{3.372768in}}%
\pgfpathlineto{\pgfqpoint{3.303456in}{3.369018in}}%
\pgfpathlineto{\pgfqpoint{3.292260in}{3.365358in}}%
\pgfpathlineto{\pgfqpoint{3.298498in}{3.355181in}}%
\pgfpathlineto{\pgfqpoint{3.304740in}{3.344965in}}%
\pgfpathlineto{\pgfqpoint{3.310986in}{3.334747in}}%
\pgfpathlineto{\pgfqpoint{3.317235in}{3.324566in}}%
\pgfpathclose%
\pgfusepath{stroke,fill}%
\end{pgfscope}%
\begin{pgfscope}%
\pgfpathrectangle{\pgfqpoint{0.887500in}{0.275000in}}{\pgfqpoint{4.225000in}{4.225000in}}%
\pgfusepath{clip}%
\pgfsetbuttcap%
\pgfsetroundjoin%
\definecolor{currentfill}{rgb}{0.153894,0.680203,0.504172}%
\pgfsetfillcolor{currentfill}%
\pgfsetfillopacity{0.700000}%
\pgfsetlinewidth{0.501875pt}%
\definecolor{currentstroke}{rgb}{1.000000,1.000000,1.000000}%
\pgfsetstrokecolor{currentstroke}%
\pgfsetstrokeopacity{0.500000}%
\pgfsetdash{}{0pt}%
\pgfpathmoveto{\pgfqpoint{2.872902in}{2.764206in}}%
\pgfpathlineto{\pgfqpoint{2.884112in}{2.788664in}}%
\pgfpathlineto{\pgfqpoint{2.895324in}{2.814220in}}%
\pgfpathlineto{\pgfqpoint{2.906538in}{2.840987in}}%
\pgfpathlineto{\pgfqpoint{2.917756in}{2.868978in}}%
\pgfpathlineto{\pgfqpoint{2.928980in}{2.897870in}}%
\pgfpathlineto{\pgfqpoint{2.922872in}{2.896035in}}%
\pgfpathlineto{\pgfqpoint{2.916768in}{2.894869in}}%
\pgfpathlineto{\pgfqpoint{2.910669in}{2.894349in}}%
\pgfpathlineto{\pgfqpoint{2.904575in}{2.894439in}}%
\pgfpathlineto{\pgfqpoint{2.898484in}{2.895100in}}%
\pgfpathlineto{\pgfqpoint{2.887299in}{2.867317in}}%
\pgfpathlineto{\pgfqpoint{2.876116in}{2.840755in}}%
\pgfpathlineto{\pgfqpoint{2.864935in}{2.815598in}}%
\pgfpathlineto{\pgfqpoint{2.853753in}{2.791910in}}%
\pgfpathlineto{\pgfqpoint{2.842569in}{2.769717in}}%
\pgfpathlineto{\pgfqpoint{2.848632in}{2.766818in}}%
\pgfpathlineto{\pgfqpoint{2.854697in}{2.764790in}}%
\pgfpathlineto{\pgfqpoint{2.860764in}{2.763661in}}%
\pgfpathlineto{\pgfqpoint{2.866832in}{2.763462in}}%
\pgfpathclose%
\pgfusepath{stroke,fill}%
\end{pgfscope}%
\begin{pgfscope}%
\pgfpathrectangle{\pgfqpoint{0.887500in}{0.275000in}}{\pgfqpoint{4.225000in}{4.225000in}}%
\pgfusepath{clip}%
\pgfsetbuttcap%
\pgfsetroundjoin%
\definecolor{currentfill}{rgb}{0.783315,0.879285,0.125405}%
\pgfsetfillcolor{currentfill}%
\pgfsetfillopacity{0.700000}%
\pgfsetlinewidth{0.501875pt}%
\definecolor{currentstroke}{rgb}{1.000000,1.000000,1.000000}%
\pgfsetstrokecolor{currentstroke}%
\pgfsetstrokeopacity{0.500000}%
\pgfsetdash{}{0pt}%
\pgfpathmoveto{\pgfqpoint{3.005430in}{3.367702in}}%
\pgfpathlineto{\pgfqpoint{3.016695in}{3.382497in}}%
\pgfpathlineto{\pgfqpoint{3.027962in}{3.394323in}}%
\pgfpathlineto{\pgfqpoint{3.039230in}{3.403665in}}%
\pgfpathlineto{\pgfqpoint{3.050496in}{3.411003in}}%
\pgfpathlineto{\pgfqpoint{3.061758in}{3.416722in}}%
\pgfpathlineto{\pgfqpoint{3.055582in}{3.424330in}}%
\pgfpathlineto{\pgfqpoint{3.049410in}{3.431573in}}%
\pgfpathlineto{\pgfqpoint{3.043243in}{3.438431in}}%
\pgfpathlineto{\pgfqpoint{3.037080in}{3.444882in}}%
\pgfpathlineto{\pgfqpoint{3.030922in}{3.450907in}}%
\pgfpathlineto{\pgfqpoint{3.019674in}{3.445175in}}%
\pgfpathlineto{\pgfqpoint{3.008424in}{3.437755in}}%
\pgfpathlineto{\pgfqpoint{2.997174in}{3.428157in}}%
\pgfpathlineto{\pgfqpoint{2.985927in}{3.415720in}}%
\pgfpathlineto{\pgfqpoint{2.974687in}{3.399780in}}%
\pgfpathlineto{\pgfqpoint{2.980825in}{3.393978in}}%
\pgfpathlineto{\pgfqpoint{2.986969in}{3.387860in}}%
\pgfpathlineto{\pgfqpoint{2.993118in}{3.381435in}}%
\pgfpathlineto{\pgfqpoint{2.999272in}{3.374713in}}%
\pgfpathclose%
\pgfusepath{stroke,fill}%
\end{pgfscope}%
\begin{pgfscope}%
\pgfpathrectangle{\pgfqpoint{0.887500in}{0.275000in}}{\pgfqpoint{4.225000in}{4.225000in}}%
\pgfusepath{clip}%
\pgfsetbuttcap%
\pgfsetroundjoin%
\definecolor{currentfill}{rgb}{0.125394,0.574318,0.549086}%
\pgfsetfillcolor{currentfill}%
\pgfsetfillopacity{0.700000}%
\pgfsetlinewidth{0.501875pt}%
\definecolor{currentstroke}{rgb}{1.000000,1.000000,1.000000}%
\pgfsetstrokecolor{currentstroke}%
\pgfsetstrokeopacity{0.500000}%
\pgfsetdash{}{0pt}%
\pgfpathmoveto{\pgfqpoint{1.811054in}{2.576748in}}%
\pgfpathlineto{\pgfqpoint{1.822628in}{2.579987in}}%
\pgfpathlineto{\pgfqpoint{1.834197in}{2.583225in}}%
\pgfpathlineto{\pgfqpoint{1.845761in}{2.586461in}}%
\pgfpathlineto{\pgfqpoint{1.857318in}{2.589699in}}%
\pgfpathlineto{\pgfqpoint{1.868870in}{2.592941in}}%
\pgfpathlineto{\pgfqpoint{1.863098in}{2.600946in}}%
\pgfpathlineto{\pgfqpoint{1.857331in}{2.608935in}}%
\pgfpathlineto{\pgfqpoint{1.851567in}{2.616909in}}%
\pgfpathlineto{\pgfqpoint{1.845808in}{2.624866in}}%
\pgfpathlineto{\pgfqpoint{1.840053in}{2.632806in}}%
\pgfpathlineto{\pgfqpoint{1.828514in}{2.629564in}}%
\pgfpathlineto{\pgfqpoint{1.816970in}{2.626324in}}%
\pgfpathlineto{\pgfqpoint{1.805419in}{2.623087in}}%
\pgfpathlineto{\pgfqpoint{1.793863in}{2.619849in}}%
\pgfpathlineto{\pgfqpoint{1.782301in}{2.616610in}}%
\pgfpathlineto{\pgfqpoint{1.788043in}{2.608673in}}%
\pgfpathlineto{\pgfqpoint{1.793789in}{2.600718in}}%
\pgfpathlineto{\pgfqpoint{1.799540in}{2.592745in}}%
\pgfpathlineto{\pgfqpoint{1.805295in}{2.584755in}}%
\pgfpathclose%
\pgfusepath{stroke,fill}%
\end{pgfscope}%
\begin{pgfscope}%
\pgfpathrectangle{\pgfqpoint{0.887500in}{0.275000in}}{\pgfqpoint{4.225000in}{4.225000in}}%
\pgfusepath{clip}%
\pgfsetbuttcap%
\pgfsetroundjoin%
\definecolor{currentfill}{rgb}{0.344074,0.780029,0.397381}%
\pgfsetfillcolor{currentfill}%
\pgfsetfillopacity{0.700000}%
\pgfsetlinewidth{0.501875pt}%
\definecolor{currentstroke}{rgb}{1.000000,1.000000,1.000000}%
\pgfsetstrokecolor{currentstroke}%
\pgfsetstrokeopacity{0.500000}%
\pgfsetdash{}{0pt}%
\pgfpathmoveto{\pgfqpoint{3.934127in}{3.017924in}}%
\pgfpathlineto{\pgfqpoint{3.945184in}{3.021428in}}%
\pgfpathlineto{\pgfqpoint{3.956235in}{3.024895in}}%
\pgfpathlineto{\pgfqpoint{3.967280in}{3.028313in}}%
\pgfpathlineto{\pgfqpoint{3.978318in}{3.031671in}}%
\pgfpathlineto{\pgfqpoint{3.989349in}{3.034972in}}%
\pgfpathlineto{\pgfqpoint{3.982988in}{3.048201in}}%
\pgfpathlineto{\pgfqpoint{3.976628in}{3.061321in}}%
\pgfpathlineto{\pgfqpoint{3.970268in}{3.074321in}}%
\pgfpathlineto{\pgfqpoint{3.963907in}{3.087190in}}%
\pgfpathlineto{\pgfqpoint{3.957546in}{3.099920in}}%
\pgfpathlineto{\pgfqpoint{3.946521in}{3.096755in}}%
\pgfpathlineto{\pgfqpoint{3.935489in}{3.093511in}}%
\pgfpathlineto{\pgfqpoint{3.924450in}{3.090187in}}%
\pgfpathlineto{\pgfqpoint{3.913404in}{3.086797in}}%
\pgfpathlineto{\pgfqpoint{3.902352in}{3.083355in}}%
\pgfpathlineto{\pgfqpoint{3.908709in}{3.070629in}}%
\pgfpathlineto{\pgfqpoint{3.915065in}{3.057702in}}%
\pgfpathlineto{\pgfqpoint{3.921419in}{3.044595in}}%
\pgfpathlineto{\pgfqpoint{3.927773in}{3.031328in}}%
\pgfpathclose%
\pgfusepath{stroke,fill}%
\end{pgfscope}%
\begin{pgfscope}%
\pgfpathrectangle{\pgfqpoint{0.887500in}{0.275000in}}{\pgfqpoint{4.225000in}{4.225000in}}%
\pgfusepath{clip}%
\pgfsetbuttcap%
\pgfsetroundjoin%
\definecolor{currentfill}{rgb}{0.657642,0.860219,0.203082}%
\pgfsetfillcolor{currentfill}%
\pgfsetfillopacity{0.700000}%
\pgfsetlinewidth{0.501875pt}%
\definecolor{currentstroke}{rgb}{1.000000,1.000000,1.000000}%
\pgfsetstrokecolor{currentstroke}%
\pgfsetstrokeopacity{0.500000}%
\pgfsetdash{}{0pt}%
\pgfpathmoveto{\pgfqpoint{3.410783in}{3.278548in}}%
\pgfpathlineto{\pgfqpoint{3.421961in}{3.281340in}}%
\pgfpathlineto{\pgfqpoint{3.433133in}{3.284157in}}%
\pgfpathlineto{\pgfqpoint{3.444300in}{3.287033in}}%
\pgfpathlineto{\pgfqpoint{3.455462in}{3.289999in}}%
\pgfpathlineto{\pgfqpoint{3.466620in}{3.293087in}}%
\pgfpathlineto{\pgfqpoint{3.460344in}{3.304191in}}%
\pgfpathlineto{\pgfqpoint{3.454071in}{3.315302in}}%
\pgfpathlineto{\pgfqpoint{3.447802in}{3.326420in}}%
\pgfpathlineto{\pgfqpoint{3.441535in}{3.337544in}}%
\pgfpathlineto{\pgfqpoint{3.435272in}{3.348673in}}%
\pgfpathlineto{\pgfqpoint{3.424117in}{3.345181in}}%
\pgfpathlineto{\pgfqpoint{3.412957in}{3.341715in}}%
\pgfpathlineto{\pgfqpoint{3.401791in}{3.338270in}}%
\pgfpathlineto{\pgfqpoint{3.390621in}{3.334840in}}%
\pgfpathlineto{\pgfqpoint{3.379445in}{3.331420in}}%
\pgfpathlineto{\pgfqpoint{3.385706in}{3.320847in}}%
\pgfpathlineto{\pgfqpoint{3.391970in}{3.310290in}}%
\pgfpathlineto{\pgfqpoint{3.398238in}{3.299731in}}%
\pgfpathlineto{\pgfqpoint{3.404509in}{3.289156in}}%
\pgfpathclose%
\pgfusepath{stroke,fill}%
\end{pgfscope}%
\begin{pgfscope}%
\pgfpathrectangle{\pgfqpoint{0.887500in}{0.275000in}}{\pgfqpoint{4.225000in}{4.225000in}}%
\pgfusepath{clip}%
\pgfsetbuttcap%
\pgfsetroundjoin%
\definecolor{currentfill}{rgb}{0.147607,0.511733,0.557049}%
\pgfsetfillcolor{currentfill}%
\pgfsetfillopacity{0.700000}%
\pgfsetlinewidth{0.501875pt}%
\definecolor{currentstroke}{rgb}{1.000000,1.000000,1.000000}%
\pgfsetstrokecolor{currentstroke}%
\pgfsetstrokeopacity{0.500000}%
\pgfsetdash{}{0pt}%
\pgfpathmoveto{\pgfqpoint{2.678235in}{2.433946in}}%
\pgfpathlineto{\pgfqpoint{2.689588in}{2.438053in}}%
\pgfpathlineto{\pgfqpoint{2.700942in}{2.441405in}}%
\pgfpathlineto{\pgfqpoint{2.712299in}{2.443818in}}%
\pgfpathlineto{\pgfqpoint{2.723655in}{2.445750in}}%
\pgfpathlineto{\pgfqpoint{2.735000in}{2.448090in}}%
\pgfpathlineto{\pgfqpoint{2.728896in}{2.461059in}}%
\pgfpathlineto{\pgfqpoint{2.722795in}{2.474093in}}%
\pgfpathlineto{\pgfqpoint{2.716696in}{2.487072in}}%
\pgfpathlineto{\pgfqpoint{2.710601in}{2.499879in}}%
\pgfpathlineto{\pgfqpoint{2.704512in}{2.512396in}}%
\pgfpathlineto{\pgfqpoint{2.693218in}{2.504926in}}%
\pgfpathlineto{\pgfqpoint{2.681915in}{2.498002in}}%
\pgfpathlineto{\pgfqpoint{2.670606in}{2.491398in}}%
\pgfpathlineto{\pgfqpoint{2.659291in}{2.484993in}}%
\pgfpathlineto{\pgfqpoint{2.647970in}{2.478823in}}%
\pgfpathlineto{\pgfqpoint{2.654015in}{2.469870in}}%
\pgfpathlineto{\pgfqpoint{2.660064in}{2.460901in}}%
\pgfpathlineto{\pgfqpoint{2.666117in}{2.451920in}}%
\pgfpathlineto{\pgfqpoint{2.672174in}{2.442934in}}%
\pgfpathclose%
\pgfusepath{stroke,fill}%
\end{pgfscope}%
\begin{pgfscope}%
\pgfpathrectangle{\pgfqpoint{0.887500in}{0.275000in}}{\pgfqpoint{4.225000in}{4.225000in}}%
\pgfusepath{clip}%
\pgfsetbuttcap%
\pgfsetroundjoin%
\definecolor{currentfill}{rgb}{0.132444,0.552216,0.553018}%
\pgfsetfillcolor{currentfill}%
\pgfsetfillopacity{0.700000}%
\pgfsetlinewidth{0.501875pt}%
\definecolor{currentstroke}{rgb}{1.000000,1.000000,1.000000}%
\pgfsetstrokecolor{currentstroke}%
\pgfsetstrokeopacity{0.500000}%
\pgfsetdash{}{0pt}%
\pgfpathmoveto{\pgfqpoint{2.129078in}{2.520818in}}%
\pgfpathlineto{\pgfqpoint{2.140575in}{2.524106in}}%
\pgfpathlineto{\pgfqpoint{2.152067in}{2.527389in}}%
\pgfpathlineto{\pgfqpoint{2.163553in}{2.530668in}}%
\pgfpathlineto{\pgfqpoint{2.175033in}{2.533943in}}%
\pgfpathlineto{\pgfqpoint{2.186508in}{2.537218in}}%
\pgfpathlineto{\pgfqpoint{2.180624in}{2.545427in}}%
\pgfpathlineto{\pgfqpoint{2.174744in}{2.553621in}}%
\pgfpathlineto{\pgfqpoint{2.168869in}{2.561800in}}%
\pgfpathlineto{\pgfqpoint{2.162998in}{2.569963in}}%
\pgfpathlineto{\pgfqpoint{2.157131in}{2.578111in}}%
\pgfpathlineto{\pgfqpoint{2.145668in}{2.574843in}}%
\pgfpathlineto{\pgfqpoint{2.134200in}{2.571572in}}%
\pgfpathlineto{\pgfqpoint{2.122726in}{2.568295in}}%
\pgfpathlineto{\pgfqpoint{2.111247in}{2.565011in}}%
\pgfpathlineto{\pgfqpoint{2.099762in}{2.561719in}}%
\pgfpathlineto{\pgfqpoint{2.105617in}{2.553568in}}%
\pgfpathlineto{\pgfqpoint{2.111476in}{2.545403in}}%
\pgfpathlineto{\pgfqpoint{2.117339in}{2.537224in}}%
\pgfpathlineto{\pgfqpoint{2.123207in}{2.529029in}}%
\pgfpathclose%
\pgfusepath{stroke,fill}%
\end{pgfscope}%
\begin{pgfscope}%
\pgfpathrectangle{\pgfqpoint{0.887500in}{0.275000in}}{\pgfqpoint{4.225000in}{4.225000in}}%
\pgfusepath{clip}%
\pgfsetbuttcap%
\pgfsetroundjoin%
\definecolor{currentfill}{rgb}{0.395174,0.797475,0.367757}%
\pgfsetfillcolor{currentfill}%
\pgfsetfillopacity{0.700000}%
\pgfsetlinewidth{0.501875pt}%
\definecolor{currentstroke}{rgb}{1.000000,1.000000,1.000000}%
\pgfsetstrokecolor{currentstroke}%
\pgfsetstrokeopacity{0.500000}%
\pgfsetdash{}{0pt}%
\pgfpathmoveto{\pgfqpoint{3.847007in}{3.065873in}}%
\pgfpathlineto{\pgfqpoint{3.858087in}{3.069364in}}%
\pgfpathlineto{\pgfqpoint{3.869161in}{3.072865in}}%
\pgfpathlineto{\pgfqpoint{3.880230in}{3.076374in}}%
\pgfpathlineto{\pgfqpoint{3.891294in}{3.079875in}}%
\pgfpathlineto{\pgfqpoint{3.902352in}{3.083355in}}%
\pgfpathlineto{\pgfqpoint{3.895994in}{3.095891in}}%
\pgfpathlineto{\pgfqpoint{3.889635in}{3.108265in}}%
\pgfpathlineto{\pgfqpoint{3.883277in}{3.120504in}}%
\pgfpathlineto{\pgfqpoint{3.876919in}{3.132634in}}%
\pgfpathlineto{\pgfqpoint{3.870563in}{3.144683in}}%
\pgfpathlineto{\pgfqpoint{3.859508in}{3.141151in}}%
\pgfpathlineto{\pgfqpoint{3.848449in}{3.137622in}}%
\pgfpathlineto{\pgfqpoint{3.837384in}{3.134113in}}%
\pgfpathlineto{\pgfqpoint{3.826315in}{3.130642in}}%
\pgfpathlineto{\pgfqpoint{3.815242in}{3.127210in}}%
\pgfpathlineto{\pgfqpoint{3.821593in}{3.115164in}}%
\pgfpathlineto{\pgfqpoint{3.827946in}{3.103036in}}%
\pgfpathlineto{\pgfqpoint{3.834300in}{3.090799in}}%
\pgfpathlineto{\pgfqpoint{3.840654in}{3.078421in}}%
\pgfpathclose%
\pgfusepath{stroke,fill}%
\end{pgfscope}%
\begin{pgfscope}%
\pgfpathrectangle{\pgfqpoint{0.887500in}{0.275000in}}{\pgfqpoint{4.225000in}{4.225000in}}%
\pgfusepath{clip}%
\pgfsetbuttcap%
\pgfsetroundjoin%
\definecolor{currentfill}{rgb}{0.606045,0.850733,0.236712}%
\pgfsetfillcolor{currentfill}%
\pgfsetfillopacity{0.700000}%
\pgfsetlinewidth{0.501875pt}%
\definecolor{currentstroke}{rgb}{1.000000,1.000000,1.000000}%
\pgfsetstrokecolor{currentstroke}%
\pgfsetstrokeopacity{0.500000}%
\pgfsetdash{}{0pt}%
\pgfpathmoveto{\pgfqpoint{3.498051in}{3.237716in}}%
\pgfpathlineto{\pgfqpoint{3.509210in}{3.240795in}}%
\pgfpathlineto{\pgfqpoint{3.520366in}{3.244102in}}%
\pgfpathlineto{\pgfqpoint{3.531519in}{3.247604in}}%
\pgfpathlineto{\pgfqpoint{3.542669in}{3.251253in}}%
\pgfpathlineto{\pgfqpoint{3.553814in}{3.255001in}}%
\pgfpathlineto{\pgfqpoint{3.547516in}{3.266193in}}%
\pgfpathlineto{\pgfqpoint{3.541221in}{3.277349in}}%
\pgfpathlineto{\pgfqpoint{3.534929in}{3.288480in}}%
\pgfpathlineto{\pgfqpoint{3.528640in}{3.299596in}}%
\pgfpathlineto{\pgfqpoint{3.522354in}{3.310706in}}%
\pgfpathlineto{\pgfqpoint{3.511215in}{3.306971in}}%
\pgfpathlineto{\pgfqpoint{3.500072in}{3.303304in}}%
\pgfpathlineto{\pgfqpoint{3.488925in}{3.299744in}}%
\pgfpathlineto{\pgfqpoint{3.477774in}{3.296329in}}%
\pgfpathlineto{\pgfqpoint{3.466620in}{3.293087in}}%
\pgfpathlineto{\pgfqpoint{3.472900in}{3.281992in}}%
\pgfpathlineto{\pgfqpoint{3.479183in}{3.270907in}}%
\pgfpathlineto{\pgfqpoint{3.485469in}{3.259831in}}%
\pgfpathlineto{\pgfqpoint{3.491758in}{3.248767in}}%
\pgfpathclose%
\pgfusepath{stroke,fill}%
\end{pgfscope}%
\begin{pgfscope}%
\pgfpathrectangle{\pgfqpoint{0.887500in}{0.275000in}}{\pgfqpoint{4.225000in}{4.225000in}}%
\pgfusepath{clip}%
\pgfsetbuttcap%
\pgfsetroundjoin%
\definecolor{currentfill}{rgb}{0.137339,0.662252,0.515571}%
\pgfsetfillcolor{currentfill}%
\pgfsetfillopacity{0.700000}%
\pgfsetlinewidth{0.501875pt}%
\definecolor{currentstroke}{rgb}{1.000000,1.000000,1.000000}%
\pgfsetstrokecolor{currentstroke}%
\pgfsetstrokeopacity{0.500000}%
\pgfsetdash{}{0pt}%
\pgfpathmoveto{\pgfqpoint{4.313933in}{2.741646in}}%
\pgfpathlineto{\pgfqpoint{4.324895in}{2.745083in}}%
\pgfpathlineto{\pgfqpoint{4.335852in}{2.748525in}}%
\pgfpathlineto{\pgfqpoint{4.346805in}{2.751974in}}%
\pgfpathlineto{\pgfqpoint{4.357752in}{2.755430in}}%
\pgfpathlineto{\pgfqpoint{4.351344in}{2.769580in}}%
\pgfpathlineto{\pgfqpoint{4.344937in}{2.783708in}}%
\pgfpathlineto{\pgfqpoint{4.338532in}{2.797817in}}%
\pgfpathlineto{\pgfqpoint{4.332129in}{2.811910in}}%
\pgfpathlineto{\pgfqpoint{4.325728in}{2.825989in}}%
\pgfpathlineto{\pgfqpoint{4.314780in}{2.822458in}}%
\pgfpathlineto{\pgfqpoint{4.303827in}{2.818941in}}%
\pgfpathlineto{\pgfqpoint{4.292869in}{2.815435in}}%
\pgfpathlineto{\pgfqpoint{4.281907in}{2.811940in}}%
\pgfpathlineto{\pgfqpoint{4.288307in}{2.797900in}}%
\pgfpathlineto{\pgfqpoint{4.294710in}{2.783855in}}%
\pgfpathlineto{\pgfqpoint{4.301116in}{2.769800in}}%
\pgfpathlineto{\pgfqpoint{4.307523in}{2.755732in}}%
\pgfpathclose%
\pgfusepath{stroke,fill}%
\end{pgfscope}%
\begin{pgfscope}%
\pgfpathrectangle{\pgfqpoint{0.887500in}{0.275000in}}{\pgfqpoint{4.225000in}{4.225000in}}%
\pgfusepath{clip}%
\pgfsetbuttcap%
\pgfsetroundjoin%
\definecolor{currentfill}{rgb}{0.449368,0.813768,0.335384}%
\pgfsetfillcolor{currentfill}%
\pgfsetfillopacity{0.700000}%
\pgfsetlinewidth{0.501875pt}%
\definecolor{currentstroke}{rgb}{1.000000,1.000000,1.000000}%
\pgfsetstrokecolor{currentstroke}%
\pgfsetstrokeopacity{0.500000}%
\pgfsetdash{}{0pt}%
\pgfpathmoveto{\pgfqpoint{3.759801in}{3.110313in}}%
\pgfpathlineto{\pgfqpoint{3.770899in}{3.113690in}}%
\pgfpathlineto{\pgfqpoint{3.781993in}{3.117059in}}%
\pgfpathlineto{\pgfqpoint{3.793081in}{3.120429in}}%
\pgfpathlineto{\pgfqpoint{3.804164in}{3.123809in}}%
\pgfpathlineto{\pgfqpoint{3.815242in}{3.127210in}}%
\pgfpathlineto{\pgfqpoint{3.808892in}{3.139207in}}%
\pgfpathlineto{\pgfqpoint{3.802546in}{3.151182in}}%
\pgfpathlineto{\pgfqpoint{3.796203in}{3.163167in}}%
\pgfpathlineto{\pgfqpoint{3.789863in}{3.175191in}}%
\pgfpathlineto{\pgfqpoint{3.783528in}{3.187283in}}%
\pgfpathlineto{\pgfqpoint{3.772456in}{3.183953in}}%
\pgfpathlineto{\pgfqpoint{3.761379in}{3.180659in}}%
\pgfpathlineto{\pgfqpoint{3.750298in}{3.177388in}}%
\pgfpathlineto{\pgfqpoint{3.739212in}{3.174129in}}%
\pgfpathlineto{\pgfqpoint{3.728120in}{3.170871in}}%
\pgfpathlineto{\pgfqpoint{3.734449in}{3.158726in}}%
\pgfpathlineto{\pgfqpoint{3.740783in}{3.146622in}}%
\pgfpathlineto{\pgfqpoint{3.747119in}{3.134534in}}%
\pgfpathlineto{\pgfqpoint{3.753459in}{3.122439in}}%
\pgfpathclose%
\pgfusepath{stroke,fill}%
\end{pgfscope}%
\begin{pgfscope}%
\pgfpathrectangle{\pgfqpoint{0.887500in}{0.275000in}}{\pgfqpoint{4.225000in}{4.225000in}}%
\pgfusepath{clip}%
\pgfsetbuttcap%
\pgfsetroundjoin%
\definecolor{currentfill}{rgb}{0.143343,0.522773,0.556295}%
\pgfsetfillcolor{currentfill}%
\pgfsetfillopacity{0.700000}%
\pgfsetlinewidth{0.501875pt}%
\definecolor{currentstroke}{rgb}{1.000000,1.000000,1.000000}%
\pgfsetstrokecolor{currentstroke}%
\pgfsetstrokeopacity{0.500000}%
\pgfsetdash{}{0pt}%
\pgfpathmoveto{\pgfqpoint{2.447211in}{2.463042in}}%
\pgfpathlineto{\pgfqpoint{2.458629in}{2.466496in}}%
\pgfpathlineto{\pgfqpoint{2.470042in}{2.469878in}}%
\pgfpathlineto{\pgfqpoint{2.481451in}{2.473161in}}%
\pgfpathlineto{\pgfqpoint{2.492856in}{2.476320in}}%
\pgfpathlineto{\pgfqpoint{2.504257in}{2.479377in}}%
\pgfpathlineto{\pgfqpoint{2.498264in}{2.487825in}}%
\pgfpathlineto{\pgfqpoint{2.492275in}{2.496253in}}%
\pgfpathlineto{\pgfqpoint{2.486291in}{2.504660in}}%
\pgfpathlineto{\pgfqpoint{2.480311in}{2.513044in}}%
\pgfpathlineto{\pgfqpoint{2.474334in}{2.521401in}}%
\pgfpathlineto{\pgfqpoint{2.462945in}{2.518351in}}%
\pgfpathlineto{\pgfqpoint{2.451551in}{2.515195in}}%
\pgfpathlineto{\pgfqpoint{2.440154in}{2.511912in}}%
\pgfpathlineto{\pgfqpoint{2.428752in}{2.508529in}}%
\pgfpathlineto{\pgfqpoint{2.417346in}{2.505072in}}%
\pgfpathlineto{\pgfqpoint{2.423311in}{2.496682in}}%
\pgfpathlineto{\pgfqpoint{2.429280in}{2.488285in}}%
\pgfpathlineto{\pgfqpoint{2.435253in}{2.479880in}}%
\pgfpathlineto{\pgfqpoint{2.441230in}{2.471466in}}%
\pgfpathclose%
\pgfusepath{stroke,fill}%
\end{pgfscope}%
\begin{pgfscope}%
\pgfpathrectangle{\pgfqpoint{0.887500in}{0.275000in}}{\pgfqpoint{4.225000in}{4.225000in}}%
\pgfusepath{clip}%
\pgfsetbuttcap%
\pgfsetroundjoin%
\definecolor{currentfill}{rgb}{0.496615,0.826376,0.306377}%
\pgfsetfillcolor{currentfill}%
\pgfsetfillopacity{0.700000}%
\pgfsetlinewidth{0.501875pt}%
\definecolor{currentstroke}{rgb}{1.000000,1.000000,1.000000}%
\pgfsetstrokecolor{currentstroke}%
\pgfsetstrokeopacity{0.500000}%
\pgfsetdash{}{0pt}%
\pgfpathmoveto{\pgfqpoint{3.672580in}{3.154570in}}%
\pgfpathlineto{\pgfqpoint{3.683698in}{3.157799in}}%
\pgfpathlineto{\pgfqpoint{3.694812in}{3.161055in}}%
\pgfpathlineto{\pgfqpoint{3.705920in}{3.164326in}}%
\pgfpathlineto{\pgfqpoint{3.717022in}{3.167602in}}%
\pgfpathlineto{\pgfqpoint{3.728120in}{3.170871in}}%
\pgfpathlineto{\pgfqpoint{3.721794in}{3.183064in}}%
\pgfpathlineto{\pgfqpoint{3.715472in}{3.195284in}}%
\pgfpathlineto{\pgfqpoint{3.709152in}{3.207506in}}%
\pgfpathlineto{\pgfqpoint{3.702835in}{3.219705in}}%
\pgfpathlineto{\pgfqpoint{3.696521in}{3.231855in}}%
\pgfpathlineto{\pgfqpoint{3.685427in}{3.228542in}}%
\pgfpathlineto{\pgfqpoint{3.674329in}{3.225212in}}%
\pgfpathlineto{\pgfqpoint{3.663224in}{3.221867in}}%
\pgfpathlineto{\pgfqpoint{3.652114in}{3.218508in}}%
\pgfpathlineto{\pgfqpoint{3.640999in}{3.215137in}}%
\pgfpathlineto{\pgfqpoint{3.647311in}{3.203127in}}%
\pgfpathlineto{\pgfqpoint{3.653625in}{3.191051in}}%
\pgfpathlineto{\pgfqpoint{3.659941in}{3.178923in}}%
\pgfpathlineto{\pgfqpoint{3.666259in}{3.166758in}}%
\pgfpathclose%
\pgfusepath{stroke,fill}%
\end{pgfscope}%
\begin{pgfscope}%
\pgfpathrectangle{\pgfqpoint{0.887500in}{0.275000in}}{\pgfqpoint{4.225000in}{4.225000in}}%
\pgfusepath{clip}%
\pgfsetbuttcap%
\pgfsetroundjoin%
\definecolor{currentfill}{rgb}{0.555484,0.840254,0.269281}%
\pgfsetfillcolor{currentfill}%
\pgfsetfillopacity{0.700000}%
\pgfsetlinewidth{0.501875pt}%
\definecolor{currentstroke}{rgb}{1.000000,1.000000,1.000000}%
\pgfsetstrokecolor{currentstroke}%
\pgfsetstrokeopacity{0.500000}%
\pgfsetdash{}{0pt}%
\pgfpathmoveto{\pgfqpoint{3.585342in}{3.198183in}}%
\pgfpathlineto{\pgfqpoint{3.596484in}{3.201561in}}%
\pgfpathlineto{\pgfqpoint{3.607620in}{3.204959in}}%
\pgfpathlineto{\pgfqpoint{3.618752in}{3.208360in}}%
\pgfpathlineto{\pgfqpoint{3.629878in}{3.211754in}}%
\pgfpathlineto{\pgfqpoint{3.640999in}{3.215137in}}%
\pgfpathlineto{\pgfqpoint{3.634689in}{3.227065in}}%
\pgfpathlineto{\pgfqpoint{3.628381in}{3.238898in}}%
\pgfpathlineto{\pgfqpoint{3.622074in}{3.250620in}}%
\pgfpathlineto{\pgfqpoint{3.615769in}{3.262217in}}%
\pgfpathlineto{\pgfqpoint{3.609465in}{3.273675in}}%
\pgfpathlineto{\pgfqpoint{3.598346in}{3.270058in}}%
\pgfpathlineto{\pgfqpoint{3.587222in}{3.266370in}}%
\pgfpathlineto{\pgfqpoint{3.576091in}{3.262607in}}%
\pgfpathlineto{\pgfqpoint{3.564955in}{3.258802in}}%
\pgfpathlineto{\pgfqpoint{3.553814in}{3.255001in}}%
\pgfpathlineto{\pgfqpoint{3.560115in}{3.243764in}}%
\pgfpathlineto{\pgfqpoint{3.566418in}{3.232473in}}%
\pgfpathlineto{\pgfqpoint{3.572724in}{3.221118in}}%
\pgfpathlineto{\pgfqpoint{3.579032in}{3.209691in}}%
\pgfpathclose%
\pgfusepath{stroke,fill}%
\end{pgfscope}%
\begin{pgfscope}%
\pgfpathrectangle{\pgfqpoint{0.887500in}{0.275000in}}{\pgfqpoint{4.225000in}{4.225000in}}%
\pgfusepath{clip}%
\pgfsetbuttcap%
\pgfsetroundjoin%
\definecolor{currentfill}{rgb}{0.121148,0.592739,0.544641}%
\pgfsetfillcolor{currentfill}%
\pgfsetfillopacity{0.700000}%
\pgfsetlinewidth{0.501875pt}%
\definecolor{currentstroke}{rgb}{1.000000,1.000000,1.000000}%
\pgfsetstrokecolor{currentstroke}%
\pgfsetstrokeopacity{0.500000}%
\pgfsetdash{}{0pt}%
\pgfpathmoveto{\pgfqpoint{1.579754in}{2.606999in}}%
\pgfpathlineto{\pgfqpoint{1.591387in}{2.610280in}}%
\pgfpathlineto{\pgfqpoint{1.603014in}{2.613557in}}%
\pgfpathlineto{\pgfqpoint{1.614636in}{2.616833in}}%
\pgfpathlineto{\pgfqpoint{1.626253in}{2.620108in}}%
\pgfpathlineto{\pgfqpoint{1.637863in}{2.623383in}}%
\pgfpathlineto{\pgfqpoint{1.632174in}{2.631189in}}%
\pgfpathlineto{\pgfqpoint{1.626490in}{2.638973in}}%
\pgfpathlineto{\pgfqpoint{1.620810in}{2.646735in}}%
\pgfpathlineto{\pgfqpoint{1.615135in}{2.654478in}}%
\pgfpathlineto{\pgfqpoint{1.609464in}{2.662200in}}%
\pgfpathlineto{\pgfqpoint{1.597867in}{2.658922in}}%
\pgfpathlineto{\pgfqpoint{1.586264in}{2.655642in}}%
\pgfpathlineto{\pgfqpoint{1.574656in}{2.652362in}}%
\pgfpathlineto{\pgfqpoint{1.563042in}{2.649079in}}%
\pgfpathlineto{\pgfqpoint{1.551423in}{2.645794in}}%
\pgfpathlineto{\pgfqpoint{1.557080in}{2.638078in}}%
\pgfpathlineto{\pgfqpoint{1.562742in}{2.630343in}}%
\pgfpathlineto{\pgfqpoint{1.568408in}{2.622585in}}%
\pgfpathlineto{\pgfqpoint{1.574078in}{2.614804in}}%
\pgfpathclose%
\pgfusepath{stroke,fill}%
\end{pgfscope}%
\begin{pgfscope}%
\pgfpathrectangle{\pgfqpoint{0.887500in}{0.275000in}}{\pgfqpoint{4.225000in}{4.225000in}}%
\pgfusepath{clip}%
\pgfsetbuttcap%
\pgfsetroundjoin%
\definecolor{currentfill}{rgb}{0.162016,0.687316,0.499129}%
\pgfsetfillcolor{currentfill}%
\pgfsetfillopacity{0.700000}%
\pgfsetlinewidth{0.501875pt}%
\definecolor{currentstroke}{rgb}{1.000000,1.000000,1.000000}%
\pgfsetstrokecolor{currentstroke}%
\pgfsetstrokeopacity{0.500000}%
\pgfsetdash{}{0pt}%
\pgfpathmoveto{\pgfqpoint{4.227020in}{2.794587in}}%
\pgfpathlineto{\pgfqpoint{4.238007in}{2.798046in}}%
\pgfpathlineto{\pgfqpoint{4.248990in}{2.801509in}}%
\pgfpathlineto{\pgfqpoint{4.259967in}{2.804979in}}%
\pgfpathlineto{\pgfqpoint{4.270939in}{2.808455in}}%
\pgfpathlineto{\pgfqpoint{4.281907in}{2.811940in}}%
\pgfpathlineto{\pgfqpoint{4.275508in}{2.825981in}}%
\pgfpathlineto{\pgfqpoint{4.269113in}{2.840025in}}%
\pgfpathlineto{\pgfqpoint{4.262720in}{2.854078in}}%
\pgfpathlineto{\pgfqpoint{4.256330in}{2.868143in}}%
\pgfpathlineto{\pgfqpoint{4.249943in}{2.882211in}}%
\pgfpathlineto{\pgfqpoint{4.238977in}{2.878713in}}%
\pgfpathlineto{\pgfqpoint{4.228006in}{2.875218in}}%
\pgfpathlineto{\pgfqpoint{4.217030in}{2.871727in}}%
\pgfpathlineto{\pgfqpoint{4.206048in}{2.868238in}}%
\pgfpathlineto{\pgfqpoint{4.195062in}{2.864749in}}%
\pgfpathlineto{\pgfqpoint{4.201448in}{2.850693in}}%
\pgfpathlineto{\pgfqpoint{4.207837in}{2.836645in}}%
\pgfpathlineto{\pgfqpoint{4.214228in}{2.822614in}}%
\pgfpathlineto{\pgfqpoint{4.220623in}{2.808597in}}%
\pgfpathclose%
\pgfusepath{stroke,fill}%
\end{pgfscope}%
\begin{pgfscope}%
\pgfpathrectangle{\pgfqpoint{0.887500in}{0.275000in}}{\pgfqpoint{4.225000in}{4.225000in}}%
\pgfusepath{clip}%
\pgfsetbuttcap%
\pgfsetroundjoin%
\definecolor{currentfill}{rgb}{0.153364,0.497000,0.557724}%
\pgfsetfillcolor{currentfill}%
\pgfsetfillopacity{0.700000}%
\pgfsetlinewidth{0.501875pt}%
\definecolor{currentstroke}{rgb}{1.000000,1.000000,1.000000}%
\pgfsetstrokecolor{currentstroke}%
\pgfsetstrokeopacity{0.500000}%
\pgfsetdash{}{0pt}%
\pgfpathmoveto{\pgfqpoint{2.765520in}{2.388329in}}%
\pgfpathlineto{\pgfqpoint{2.776881in}{2.388366in}}%
\pgfpathlineto{\pgfqpoint{2.788214in}{2.391420in}}%
\pgfpathlineto{\pgfqpoint{2.799513in}{2.398858in}}%
\pgfpathlineto{\pgfqpoint{2.810773in}{2.412053in}}%
\pgfpathlineto{\pgfqpoint{2.821992in}{2.432148in}}%
\pgfpathlineto{\pgfqpoint{2.815872in}{2.443962in}}%
\pgfpathlineto{\pgfqpoint{2.809752in}{2.456496in}}%
\pgfpathlineto{\pgfqpoint{2.803630in}{2.469610in}}%
\pgfpathlineto{\pgfqpoint{2.797508in}{2.483165in}}%
\pgfpathlineto{\pgfqpoint{2.791386in}{2.497020in}}%
\pgfpathlineto{\pgfqpoint{2.780168in}{2.479386in}}%
\pgfpathlineto{\pgfqpoint{2.768917in}{2.466483in}}%
\pgfpathlineto{\pgfqpoint{2.757637in}{2.457562in}}%
\pgfpathlineto{\pgfqpoint{2.746329in}{2.451729in}}%
\pgfpathlineto{\pgfqpoint{2.735000in}{2.448090in}}%
\pgfpathlineto{\pgfqpoint{2.741105in}{2.435302in}}%
\pgfpathlineto{\pgfqpoint{2.747210in}{2.422814in}}%
\pgfpathlineto{\pgfqpoint{2.753315in}{2.410744in}}%
\pgfpathlineto{\pgfqpoint{2.759418in}{2.399210in}}%
\pgfpathclose%
\pgfusepath{stroke,fill}%
\end{pgfscope}%
\begin{pgfscope}%
\pgfpathrectangle{\pgfqpoint{0.887500in}{0.275000in}}{\pgfqpoint{4.225000in}{4.225000in}}%
\pgfusepath{clip}%
\pgfsetbuttcap%
\pgfsetroundjoin%
\definecolor{currentfill}{rgb}{0.127568,0.566949,0.550556}%
\pgfsetfillcolor{currentfill}%
\pgfsetfillopacity{0.700000}%
\pgfsetlinewidth{0.501875pt}%
\definecolor{currentstroke}{rgb}{1.000000,1.000000,1.000000}%
\pgfsetstrokecolor{currentstroke}%
\pgfsetstrokeopacity{0.500000}%
\pgfsetdash{}{0pt}%
\pgfpathmoveto{\pgfqpoint{1.897791in}{2.552712in}}%
\pgfpathlineto{\pgfqpoint{1.909350in}{2.555962in}}%
\pgfpathlineto{\pgfqpoint{1.920902in}{2.559220in}}%
\pgfpathlineto{\pgfqpoint{1.932449in}{2.562487in}}%
\pgfpathlineto{\pgfqpoint{1.943989in}{2.565766in}}%
\pgfpathlineto{\pgfqpoint{1.955524in}{2.569057in}}%
\pgfpathlineto{\pgfqpoint{1.949719in}{2.577124in}}%
\pgfpathlineto{\pgfqpoint{1.943918in}{2.585178in}}%
\pgfpathlineto{\pgfqpoint{1.938121in}{2.593220in}}%
\pgfpathlineto{\pgfqpoint{1.932328in}{2.601251in}}%
\pgfpathlineto{\pgfqpoint{1.926539in}{2.609269in}}%
\pgfpathlineto{\pgfqpoint{1.915018in}{2.605980in}}%
\pgfpathlineto{\pgfqpoint{1.903490in}{2.602705in}}%
\pgfpathlineto{\pgfqpoint{1.891956in}{2.599442in}}%
\pgfpathlineto{\pgfqpoint{1.880416in}{2.596188in}}%
\pgfpathlineto{\pgfqpoint{1.868870in}{2.592941in}}%
\pgfpathlineto{\pgfqpoint{1.874646in}{2.584922in}}%
\pgfpathlineto{\pgfqpoint{1.880426in}{2.576889in}}%
\pgfpathlineto{\pgfqpoint{1.886210in}{2.568842in}}%
\pgfpathlineto{\pgfqpoint{1.891999in}{2.560784in}}%
\pgfpathclose%
\pgfusepath{stroke,fill}%
\end{pgfscope}%
\begin{pgfscope}%
\pgfpathrectangle{\pgfqpoint{0.887500in}{0.275000in}}{\pgfqpoint{4.225000in}{4.225000in}}%
\pgfusepath{clip}%
\pgfsetbuttcap%
\pgfsetroundjoin%
\definecolor{currentfill}{rgb}{0.136408,0.541173,0.554483}%
\pgfsetfillcolor{currentfill}%
\pgfsetfillopacity{0.700000}%
\pgfsetlinewidth{0.501875pt}%
\definecolor{currentstroke}{rgb}{1.000000,1.000000,1.000000}%
\pgfsetstrokecolor{currentstroke}%
\pgfsetstrokeopacity{0.500000}%
\pgfsetdash{}{0pt}%
\pgfpathmoveto{\pgfqpoint{2.215988in}{2.495940in}}%
\pgfpathlineto{\pgfqpoint{2.227469in}{2.499214in}}%
\pgfpathlineto{\pgfqpoint{2.238945in}{2.502490in}}%
\pgfpathlineto{\pgfqpoint{2.250414in}{2.505771in}}%
\pgfpathlineto{\pgfqpoint{2.261878in}{2.509059in}}%
\pgfpathlineto{\pgfqpoint{2.273336in}{2.512356in}}%
\pgfpathlineto{\pgfqpoint{2.267420in}{2.520640in}}%
\pgfpathlineto{\pgfqpoint{2.261508in}{2.528910in}}%
\pgfpathlineto{\pgfqpoint{2.255599in}{2.537167in}}%
\pgfpathlineto{\pgfqpoint{2.249695in}{2.545410in}}%
\pgfpathlineto{\pgfqpoint{2.243795in}{2.553641in}}%
\pgfpathlineto{\pgfqpoint{2.232349in}{2.550343in}}%
\pgfpathlineto{\pgfqpoint{2.220898in}{2.547053in}}%
\pgfpathlineto{\pgfqpoint{2.209440in}{2.543771in}}%
\pgfpathlineto{\pgfqpoint{2.197977in}{2.540493in}}%
\pgfpathlineto{\pgfqpoint{2.186508in}{2.537218in}}%
\pgfpathlineto{\pgfqpoint{2.192395in}{2.528992in}}%
\pgfpathlineto{\pgfqpoint{2.198287in}{2.520752in}}%
\pgfpathlineto{\pgfqpoint{2.204183in}{2.512496in}}%
\pgfpathlineto{\pgfqpoint{2.210084in}{2.504226in}}%
\pgfpathclose%
\pgfusepath{stroke,fill}%
\end{pgfscope}%
\begin{pgfscope}%
\pgfpathrectangle{\pgfqpoint{0.887500in}{0.275000in}}{\pgfqpoint{4.225000in}{4.225000in}}%
\pgfusepath{clip}%
\pgfsetbuttcap%
\pgfsetroundjoin%
\definecolor{currentfill}{rgb}{0.196571,0.711827,0.479221}%
\pgfsetfillcolor{currentfill}%
\pgfsetfillopacity{0.700000}%
\pgfsetlinewidth{0.501875pt}%
\definecolor{currentstroke}{rgb}{1.000000,1.000000,1.000000}%
\pgfsetstrokecolor{currentstroke}%
\pgfsetstrokeopacity{0.500000}%
\pgfsetdash{}{0pt}%
\pgfpathmoveto{\pgfqpoint{4.140052in}{2.847275in}}%
\pgfpathlineto{\pgfqpoint{4.151065in}{2.850778in}}%
\pgfpathlineto{\pgfqpoint{4.162072in}{2.854276in}}%
\pgfpathlineto{\pgfqpoint{4.173074in}{2.857770in}}%
\pgfpathlineto{\pgfqpoint{4.184071in}{2.861260in}}%
\pgfpathlineto{\pgfqpoint{4.195062in}{2.864749in}}%
\pgfpathlineto{\pgfqpoint{4.188679in}{2.878800in}}%
\pgfpathlineto{\pgfqpoint{4.182297in}{2.892828in}}%
\pgfpathlineto{\pgfqpoint{4.175917in}{2.906819in}}%
\pgfpathlineto{\pgfqpoint{4.169537in}{2.920757in}}%
\pgfpathlineto{\pgfqpoint{4.163159in}{2.934626in}}%
\pgfpathlineto{\pgfqpoint{4.152169in}{2.931114in}}%
\pgfpathlineto{\pgfqpoint{4.141173in}{2.927590in}}%
\pgfpathlineto{\pgfqpoint{4.130172in}{2.924054in}}%
\pgfpathlineto{\pgfqpoint{4.119166in}{2.920504in}}%
\pgfpathlineto{\pgfqpoint{4.108153in}{2.916939in}}%
\pgfpathlineto{\pgfqpoint{4.114530in}{2.903085in}}%
\pgfpathlineto{\pgfqpoint{4.120908in}{2.889181in}}%
\pgfpathlineto{\pgfqpoint{4.127288in}{2.875237in}}%
\pgfpathlineto{\pgfqpoint{4.133669in}{2.861265in}}%
\pgfpathclose%
\pgfusepath{stroke,fill}%
\end{pgfscope}%
\begin{pgfscope}%
\pgfpathrectangle{\pgfqpoint{0.887500in}{0.275000in}}{\pgfqpoint{4.225000in}{4.225000in}}%
\pgfusepath{clip}%
\pgfsetbuttcap%
\pgfsetroundjoin%
\definecolor{currentfill}{rgb}{0.146180,0.515413,0.556823}%
\pgfsetfillcolor{currentfill}%
\pgfsetfillopacity{0.700000}%
\pgfsetlinewidth{0.501875pt}%
\definecolor{currentstroke}{rgb}{1.000000,1.000000,1.000000}%
\pgfsetstrokecolor{currentstroke}%
\pgfsetstrokeopacity{0.500000}%
\pgfsetdash{}{0pt}%
\pgfpathmoveto{\pgfqpoint{2.534282in}{2.436919in}}%
\pgfpathlineto{\pgfqpoint{2.545690in}{2.439856in}}%
\pgfpathlineto{\pgfqpoint{2.557091in}{2.442857in}}%
\pgfpathlineto{\pgfqpoint{2.568484in}{2.446022in}}%
\pgfpathlineto{\pgfqpoint{2.579869in}{2.449450in}}%
\pgfpathlineto{\pgfqpoint{2.591244in}{2.453241in}}%
\pgfpathlineto{\pgfqpoint{2.585219in}{2.461824in}}%
\pgfpathlineto{\pgfqpoint{2.579197in}{2.470416in}}%
\pgfpathlineto{\pgfqpoint{2.573180in}{2.479006in}}%
\pgfpathlineto{\pgfqpoint{2.567166in}{2.487581in}}%
\pgfpathlineto{\pgfqpoint{2.561157in}{2.496130in}}%
\pgfpathlineto{\pgfqpoint{2.549794in}{2.492290in}}%
\pgfpathlineto{\pgfqpoint{2.538422in}{2.488783in}}%
\pgfpathlineto{\pgfqpoint{2.527041in}{2.485520in}}%
\pgfpathlineto{\pgfqpoint{2.515652in}{2.482414in}}%
\pgfpathlineto{\pgfqpoint{2.504257in}{2.479377in}}%
\pgfpathlineto{\pgfqpoint{2.510254in}{2.470912in}}%
\pgfpathlineto{\pgfqpoint{2.516255in}{2.462431in}}%
\pgfpathlineto{\pgfqpoint{2.522260in}{2.453937in}}%
\pgfpathlineto{\pgfqpoint{2.528269in}{2.445433in}}%
\pgfpathclose%
\pgfusepath{stroke,fill}%
\end{pgfscope}%
\begin{pgfscope}%
\pgfpathrectangle{\pgfqpoint{0.887500in}{0.275000in}}{\pgfqpoint{4.225000in}{4.225000in}}%
\pgfusepath{clip}%
\pgfsetbuttcap%
\pgfsetroundjoin%
\definecolor{currentfill}{rgb}{0.122606,0.585371,0.546557}%
\pgfsetfillcolor{currentfill}%
\pgfsetfillopacity{0.700000}%
\pgfsetlinewidth{0.501875pt}%
\definecolor{currentstroke}{rgb}{1.000000,1.000000,1.000000}%
\pgfsetstrokecolor{currentstroke}%
\pgfsetstrokeopacity{0.500000}%
\pgfsetdash{}{0pt}%
\pgfpathmoveto{\pgfqpoint{1.666375in}{2.583989in}}%
\pgfpathlineto{\pgfqpoint{1.677993in}{2.587270in}}%
\pgfpathlineto{\pgfqpoint{1.689605in}{2.590549in}}%
\pgfpathlineto{\pgfqpoint{1.701212in}{2.593824in}}%
\pgfpathlineto{\pgfqpoint{1.712813in}{2.597096in}}%
\pgfpathlineto{\pgfqpoint{1.724408in}{2.600362in}}%
\pgfpathlineto{\pgfqpoint{1.718684in}{2.608280in}}%
\pgfpathlineto{\pgfqpoint{1.712963in}{2.616176in}}%
\pgfpathlineto{\pgfqpoint{1.707248in}{2.624049in}}%
\pgfpathlineto{\pgfqpoint{1.701537in}{2.631900in}}%
\pgfpathlineto{\pgfqpoint{1.695830in}{2.639729in}}%
\pgfpathlineto{\pgfqpoint{1.684248in}{2.636465in}}%
\pgfpathlineto{\pgfqpoint{1.672660in}{2.633198in}}%
\pgfpathlineto{\pgfqpoint{1.661067in}{2.629928in}}%
\pgfpathlineto{\pgfqpoint{1.649468in}{2.626656in}}%
\pgfpathlineto{\pgfqpoint{1.637863in}{2.623383in}}%
\pgfpathlineto{\pgfqpoint{1.643556in}{2.615554in}}%
\pgfpathlineto{\pgfqpoint{1.649254in}{2.607700in}}%
\pgfpathlineto{\pgfqpoint{1.654957in}{2.599821in}}%
\pgfpathlineto{\pgfqpoint{1.660664in}{2.591917in}}%
\pgfpathclose%
\pgfusepath{stroke,fill}%
\end{pgfscope}%
\begin{pgfscope}%
\pgfpathrectangle{\pgfqpoint{0.887500in}{0.275000in}}{\pgfqpoint{4.225000in}{4.225000in}}%
\pgfusepath{clip}%
\pgfsetbuttcap%
\pgfsetroundjoin%
\definecolor{currentfill}{rgb}{0.783315,0.879285,0.125405}%
\pgfsetfillcolor{currentfill}%
\pgfsetfillopacity{0.700000}%
\pgfsetlinewidth{0.501875pt}%
\definecolor{currentstroke}{rgb}{1.000000,1.000000,1.000000}%
\pgfsetstrokecolor{currentstroke}%
\pgfsetstrokeopacity{0.500000}%
\pgfsetdash{}{0pt}%
\pgfpathmoveto{\pgfqpoint{3.092699in}{3.373932in}}%
\pgfpathlineto{\pgfqpoint{3.103968in}{3.378009in}}%
\pgfpathlineto{\pgfqpoint{3.115230in}{3.381087in}}%
\pgfpathlineto{\pgfqpoint{3.126486in}{3.383438in}}%
\pgfpathlineto{\pgfqpoint{3.137735in}{3.385335in}}%
\pgfpathlineto{\pgfqpoint{3.148977in}{3.387050in}}%
\pgfpathlineto{\pgfqpoint{3.142773in}{3.396352in}}%
\pgfpathlineto{\pgfqpoint{3.136572in}{3.405398in}}%
\pgfpathlineto{\pgfqpoint{3.130375in}{3.414159in}}%
\pgfpathlineto{\pgfqpoint{3.124180in}{3.422604in}}%
\pgfpathlineto{\pgfqpoint{3.117989in}{3.430702in}}%
\pgfpathlineto{\pgfqpoint{3.106756in}{3.428938in}}%
\pgfpathlineto{\pgfqpoint{3.095516in}{3.426933in}}%
\pgfpathlineto{\pgfqpoint{3.084269in}{3.424411in}}%
\pgfpathlineto{\pgfqpoint{3.073016in}{3.421099in}}%
\pgfpathlineto{\pgfqpoint{3.061758in}{3.416722in}}%
\pgfpathlineto{\pgfqpoint{3.067939in}{3.408769in}}%
\pgfpathlineto{\pgfqpoint{3.074123in}{3.400493in}}%
\pgfpathlineto{\pgfqpoint{3.080311in}{3.391915in}}%
\pgfpathlineto{\pgfqpoint{3.086503in}{3.383055in}}%
\pgfpathclose%
\pgfusepath{stroke,fill}%
\end{pgfscope}%
\begin{pgfscope}%
\pgfpathrectangle{\pgfqpoint{0.887500in}{0.275000in}}{\pgfqpoint{4.225000in}{4.225000in}}%
\pgfusepath{clip}%
\pgfsetbuttcap%
\pgfsetroundjoin%
\definecolor{currentfill}{rgb}{0.239374,0.735588,0.455688}%
\pgfsetfillcolor{currentfill}%
\pgfsetfillopacity{0.700000}%
\pgfsetlinewidth{0.501875pt}%
\definecolor{currentstroke}{rgb}{1.000000,1.000000,1.000000}%
\pgfsetstrokecolor{currentstroke}%
\pgfsetstrokeopacity{0.500000}%
\pgfsetdash{}{0pt}%
\pgfpathmoveto{\pgfqpoint{4.053012in}{2.898960in}}%
\pgfpathlineto{\pgfqpoint{4.064051in}{2.902566in}}%
\pgfpathlineto{\pgfqpoint{4.075084in}{2.906171in}}%
\pgfpathlineto{\pgfqpoint{4.086113in}{2.909770in}}%
\pgfpathlineto{\pgfqpoint{4.097136in}{2.913360in}}%
\pgfpathlineto{\pgfqpoint{4.108153in}{2.916939in}}%
\pgfpathlineto{\pgfqpoint{4.101778in}{2.930733in}}%
\pgfpathlineto{\pgfqpoint{4.095403in}{2.944457in}}%
\pgfpathlineto{\pgfqpoint{4.089028in}{2.958100in}}%
\pgfpathlineto{\pgfqpoint{4.082654in}{2.971651in}}%
\pgfpathlineto{\pgfqpoint{4.076280in}{2.985104in}}%
\pgfpathlineto{\pgfqpoint{4.065267in}{2.981620in}}%
\pgfpathlineto{\pgfqpoint{4.054249in}{2.978127in}}%
\pgfpathlineto{\pgfqpoint{4.043226in}{2.974621in}}%
\pgfpathlineto{\pgfqpoint{4.032197in}{2.971101in}}%
\pgfpathlineto{\pgfqpoint{4.021162in}{2.967565in}}%
\pgfpathlineto{\pgfqpoint{4.027528in}{2.953907in}}%
\pgfpathlineto{\pgfqpoint{4.033896in}{2.940214in}}%
\pgfpathlineto{\pgfqpoint{4.040266in}{2.926491in}}%
\pgfpathlineto{\pgfqpoint{4.046638in}{2.912739in}}%
\pgfpathclose%
\pgfusepath{stroke,fill}%
\end{pgfscope}%
\begin{pgfscope}%
\pgfpathrectangle{\pgfqpoint{0.887500in}{0.275000in}}{\pgfqpoint{4.225000in}{4.225000in}}%
\pgfusepath{clip}%
\pgfsetbuttcap%
\pgfsetroundjoin%
\definecolor{currentfill}{rgb}{0.129933,0.559582,0.551864}%
\pgfsetfillcolor{currentfill}%
\pgfsetfillopacity{0.700000}%
\pgfsetlinewidth{0.501875pt}%
\definecolor{currentstroke}{rgb}{1.000000,1.000000,1.000000}%
\pgfsetstrokecolor{currentstroke}%
\pgfsetstrokeopacity{0.500000}%
\pgfsetdash{}{0pt}%
\pgfpathmoveto{\pgfqpoint{1.984611in}{2.528524in}}%
\pgfpathlineto{\pgfqpoint{1.996152in}{2.531837in}}%
\pgfpathlineto{\pgfqpoint{2.007687in}{2.535159in}}%
\pgfpathlineto{\pgfqpoint{2.019216in}{2.538487in}}%
\pgfpathlineto{\pgfqpoint{2.030739in}{2.541819in}}%
\pgfpathlineto{\pgfqpoint{2.042257in}{2.545150in}}%
\pgfpathlineto{\pgfqpoint{2.036419in}{2.553279in}}%
\pgfpathlineto{\pgfqpoint{2.030585in}{2.561394in}}%
\pgfpathlineto{\pgfqpoint{2.024755in}{2.569494in}}%
\pgfpathlineto{\pgfqpoint{2.018929in}{2.577582in}}%
\pgfpathlineto{\pgfqpoint{2.013107in}{2.585657in}}%
\pgfpathlineto{\pgfqpoint{2.001602in}{2.582327in}}%
\pgfpathlineto{\pgfqpoint{1.990091in}{2.578999in}}%
\pgfpathlineto{\pgfqpoint{1.978575in}{2.575677in}}%
\pgfpathlineto{\pgfqpoint{1.967052in}{2.572362in}}%
\pgfpathlineto{\pgfqpoint{1.955524in}{2.569057in}}%
\pgfpathlineto{\pgfqpoint{1.961333in}{2.560979in}}%
\pgfpathlineto{\pgfqpoint{1.967146in}{2.552887in}}%
\pgfpathlineto{\pgfqpoint{1.972964in}{2.544781in}}%
\pgfpathlineto{\pgfqpoint{1.978785in}{2.536660in}}%
\pgfpathclose%
\pgfusepath{stroke,fill}%
\end{pgfscope}%
\begin{pgfscope}%
\pgfpathrectangle{\pgfqpoint{0.887500in}{0.275000in}}{\pgfqpoint{4.225000in}{4.225000in}}%
\pgfusepath{clip}%
\pgfsetbuttcap%
\pgfsetroundjoin%
\definecolor{currentfill}{rgb}{0.741388,0.873449,0.149561}%
\pgfsetfillcolor{currentfill}%
\pgfsetfillopacity{0.700000}%
\pgfsetlinewidth{0.501875pt}%
\definecolor{currentstroke}{rgb}{1.000000,1.000000,1.000000}%
\pgfsetstrokecolor{currentstroke}%
\pgfsetstrokeopacity{0.500000}%
\pgfsetdash{}{0pt}%
\pgfpathmoveto{\pgfqpoint{3.180045in}{3.337568in}}%
\pgfpathlineto{\pgfqpoint{3.191290in}{3.339468in}}%
\pgfpathlineto{\pgfqpoint{3.202528in}{3.341581in}}%
\pgfpathlineto{\pgfqpoint{3.213762in}{3.343902in}}%
\pgfpathlineto{\pgfqpoint{3.224990in}{3.346425in}}%
\pgfpathlineto{\pgfqpoint{3.236213in}{3.349142in}}%
\pgfpathlineto{\pgfqpoint{3.229985in}{3.359191in}}%
\pgfpathlineto{\pgfqpoint{3.223761in}{3.369140in}}%
\pgfpathlineto{\pgfqpoint{3.217539in}{3.378954in}}%
\pgfpathlineto{\pgfqpoint{3.211320in}{3.388599in}}%
\pgfpathlineto{\pgfqpoint{3.205105in}{3.398042in}}%
\pgfpathlineto{\pgfqpoint{3.193890in}{3.395432in}}%
\pgfpathlineto{\pgfqpoint{3.182670in}{3.393032in}}%
\pgfpathlineto{\pgfqpoint{3.171445in}{3.390839in}}%
\pgfpathlineto{\pgfqpoint{3.160214in}{3.388847in}}%
\pgfpathlineto{\pgfqpoint{3.148977in}{3.387050in}}%
\pgfpathlineto{\pgfqpoint{3.155185in}{3.377518in}}%
\pgfpathlineto{\pgfqpoint{3.161395in}{3.367780in}}%
\pgfpathlineto{\pgfqpoint{3.167609in}{3.357859in}}%
\pgfpathlineto{\pgfqpoint{3.173826in}{3.347780in}}%
\pgfpathclose%
\pgfusepath{stroke,fill}%
\end{pgfscope}%
\begin{pgfscope}%
\pgfpathrectangle{\pgfqpoint{0.887500in}{0.275000in}}{\pgfqpoint{4.225000in}{4.225000in}}%
\pgfusepath{clip}%
\pgfsetbuttcap%
\pgfsetroundjoin%
\definecolor{currentfill}{rgb}{0.120092,0.600104,0.542530}%
\pgfsetfillcolor{currentfill}%
\pgfsetfillopacity{0.700000}%
\pgfsetlinewidth{0.501875pt}%
\definecolor{currentstroke}{rgb}{1.000000,1.000000,1.000000}%
\pgfsetstrokecolor{currentstroke}%
\pgfsetstrokeopacity{0.500000}%
\pgfsetdash{}{0pt}%
\pgfpathmoveto{\pgfqpoint{1.434925in}{2.612596in}}%
\pgfpathlineto{\pgfqpoint{1.446599in}{2.615957in}}%
\pgfpathlineto{\pgfqpoint{1.458268in}{2.619306in}}%
\pgfpathlineto{\pgfqpoint{1.469931in}{2.622645in}}%
\pgfpathlineto{\pgfqpoint{1.481589in}{2.625974in}}%
\pgfpathlineto{\pgfqpoint{1.493242in}{2.629294in}}%
\pgfpathlineto{\pgfqpoint{1.487603in}{2.636984in}}%
\pgfpathlineto{\pgfqpoint{1.481969in}{2.644654in}}%
\pgfpathlineto{\pgfqpoint{1.476339in}{2.652308in}}%
\pgfpathlineto{\pgfqpoint{1.470713in}{2.659945in}}%
\pgfpathlineto{\pgfqpoint{1.465092in}{2.667567in}}%
\pgfpathlineto{\pgfqpoint{1.453454in}{2.664241in}}%
\pgfpathlineto{\pgfqpoint{1.441810in}{2.660908in}}%
\pgfpathlineto{\pgfqpoint{1.430160in}{2.657566in}}%
\pgfpathlineto{\pgfqpoint{1.418506in}{2.654214in}}%
\pgfpathlineto{\pgfqpoint{1.406846in}{2.650852in}}%
\pgfpathlineto{\pgfqpoint{1.412453in}{2.643233in}}%
\pgfpathlineto{\pgfqpoint{1.418065in}{2.635599in}}%
\pgfpathlineto{\pgfqpoint{1.423681in}{2.627949in}}%
\pgfpathlineto{\pgfqpoint{1.429301in}{2.620282in}}%
\pgfpathclose%
\pgfusepath{stroke,fill}%
\end{pgfscope}%
\begin{pgfscope}%
\pgfpathrectangle{\pgfqpoint{0.887500in}{0.275000in}}{\pgfqpoint{4.225000in}{4.225000in}}%
\pgfusepath{clip}%
\pgfsetbuttcap%
\pgfsetroundjoin%
\definecolor{currentfill}{rgb}{0.139147,0.533812,0.555298}%
\pgfsetfillcolor{currentfill}%
\pgfsetfillopacity{0.700000}%
\pgfsetlinewidth{0.501875pt}%
\definecolor{currentstroke}{rgb}{1.000000,1.000000,1.000000}%
\pgfsetstrokecolor{currentstroke}%
\pgfsetstrokeopacity{0.500000}%
\pgfsetdash{}{0pt}%
\pgfpathmoveto{\pgfqpoint{2.302979in}{2.470724in}}%
\pgfpathlineto{\pgfqpoint{2.314443in}{2.474047in}}%
\pgfpathlineto{\pgfqpoint{2.325901in}{2.477390in}}%
\pgfpathlineto{\pgfqpoint{2.337353in}{2.480757in}}%
\pgfpathlineto{\pgfqpoint{2.348798in}{2.484151in}}%
\pgfpathlineto{\pgfqpoint{2.360238in}{2.487574in}}%
\pgfpathlineto{\pgfqpoint{2.354289in}{2.495926in}}%
\pgfpathlineto{\pgfqpoint{2.348345in}{2.504260in}}%
\pgfpathlineto{\pgfqpoint{2.342404in}{2.512577in}}%
\pgfpathlineto{\pgfqpoint{2.336468in}{2.520879in}}%
\pgfpathlineto{\pgfqpoint{2.330536in}{2.529167in}}%
\pgfpathlineto{\pgfqpoint{2.319108in}{2.525738in}}%
\pgfpathlineto{\pgfqpoint{2.307675in}{2.522350in}}%
\pgfpathlineto{\pgfqpoint{2.296234in}{2.518995in}}%
\pgfpathlineto{\pgfqpoint{2.284788in}{2.515666in}}%
\pgfpathlineto{\pgfqpoint{2.273336in}{2.512356in}}%
\pgfpathlineto{\pgfqpoint{2.279257in}{2.504059in}}%
\pgfpathlineto{\pgfqpoint{2.285181in}{2.495747in}}%
\pgfpathlineto{\pgfqpoint{2.291110in}{2.487421in}}%
\pgfpathlineto{\pgfqpoint{2.297042in}{2.479080in}}%
\pgfpathclose%
\pgfusepath{stroke,fill}%
\end{pgfscope}%
\begin{pgfscope}%
\pgfpathrectangle{\pgfqpoint{0.887500in}{0.275000in}}{\pgfqpoint{4.225000in}{4.225000in}}%
\pgfusepath{clip}%
\pgfsetbuttcap%
\pgfsetroundjoin%
\definecolor{currentfill}{rgb}{0.288921,0.758394,0.428426}%
\pgfsetfillcolor{currentfill}%
\pgfsetfillopacity{0.700000}%
\pgfsetlinewidth{0.501875pt}%
\definecolor{currentstroke}{rgb}{1.000000,1.000000,1.000000}%
\pgfsetstrokecolor{currentstroke}%
\pgfsetstrokeopacity{0.500000}%
\pgfsetdash{}{0pt}%
\pgfpathmoveto{\pgfqpoint{3.965906in}{2.949634in}}%
\pgfpathlineto{\pgfqpoint{3.976968in}{2.953240in}}%
\pgfpathlineto{\pgfqpoint{3.988025in}{2.956844in}}%
\pgfpathlineto{\pgfqpoint{3.999076in}{2.960437in}}%
\pgfpathlineto{\pgfqpoint{4.010122in}{2.964011in}}%
\pgfpathlineto{\pgfqpoint{4.021162in}{2.967565in}}%
\pgfpathlineto{\pgfqpoint{4.014797in}{2.981179in}}%
\pgfpathlineto{\pgfqpoint{4.008434in}{2.994737in}}%
\pgfpathlineto{\pgfqpoint{4.002072in}{3.008229in}}%
\pgfpathlineto{\pgfqpoint{3.995710in}{3.021644in}}%
\pgfpathlineto{\pgfqpoint{3.989349in}{3.034972in}}%
\pgfpathlineto{\pgfqpoint{3.978318in}{3.031671in}}%
\pgfpathlineto{\pgfqpoint{3.967280in}{3.028313in}}%
\pgfpathlineto{\pgfqpoint{3.956235in}{3.024895in}}%
\pgfpathlineto{\pgfqpoint{3.945184in}{3.021428in}}%
\pgfpathlineto{\pgfqpoint{3.934127in}{3.017924in}}%
\pgfpathlineto{\pgfqpoint{3.940481in}{3.004406in}}%
\pgfpathlineto{\pgfqpoint{3.946835in}{2.990796in}}%
\pgfpathlineto{\pgfqpoint{3.953190in}{2.977116in}}%
\pgfpathlineto{\pgfqpoint{3.959547in}{2.963387in}}%
\pgfpathclose%
\pgfusepath{stroke,fill}%
\end{pgfscope}%
\begin{pgfscope}%
\pgfpathrectangle{\pgfqpoint{0.887500in}{0.275000in}}{\pgfqpoint{4.225000in}{4.225000in}}%
\pgfusepath{clip}%
\pgfsetbuttcap%
\pgfsetroundjoin%
\definecolor{currentfill}{rgb}{0.122312,0.633153,0.530398}%
\pgfsetfillcolor{currentfill}%
\pgfsetfillopacity{0.700000}%
\pgfsetlinewidth{0.501875pt}%
\definecolor{currentstroke}{rgb}{1.000000,1.000000,1.000000}%
\pgfsetstrokecolor{currentstroke}%
\pgfsetstrokeopacity{0.500000}%
\pgfsetdash{}{0pt}%
\pgfpathmoveto{\pgfqpoint{4.346004in}{2.670817in}}%
\pgfpathlineto{\pgfqpoint{4.356966in}{2.674181in}}%
\pgfpathlineto{\pgfqpoint{4.367923in}{2.677546in}}%
\pgfpathlineto{\pgfqpoint{4.378874in}{2.680911in}}%
\pgfpathlineto{\pgfqpoint{4.389820in}{2.684279in}}%
\pgfpathlineto{\pgfqpoint{4.383403in}{2.698560in}}%
\pgfpathlineto{\pgfqpoint{4.376988in}{2.712820in}}%
\pgfpathlineto{\pgfqpoint{4.370575in}{2.727052in}}%
\pgfpathlineto{\pgfqpoint{4.364163in}{2.741255in}}%
\pgfpathlineto{\pgfqpoint{4.357752in}{2.755430in}}%
\pgfpathlineto{\pgfqpoint{4.346805in}{2.751974in}}%
\pgfpathlineto{\pgfqpoint{4.335852in}{2.748525in}}%
\pgfpathlineto{\pgfqpoint{4.324895in}{2.745083in}}%
\pgfpathlineto{\pgfqpoint{4.313933in}{2.741646in}}%
\pgfpathlineto{\pgfqpoint{4.320344in}{2.727537in}}%
\pgfpathlineto{\pgfqpoint{4.326757in}{2.713401in}}%
\pgfpathlineto{\pgfqpoint{4.333171in}{2.699235in}}%
\pgfpathlineto{\pgfqpoint{4.339587in}{2.685039in}}%
\pgfpathclose%
\pgfusepath{stroke,fill}%
\end{pgfscope}%
\begin{pgfscope}%
\pgfpathrectangle{\pgfqpoint{0.887500in}{0.275000in}}{\pgfqpoint{4.225000in}{4.225000in}}%
\pgfusepath{clip}%
\pgfsetbuttcap%
\pgfsetroundjoin%
\definecolor{currentfill}{rgb}{0.149039,0.508051,0.557250}%
\pgfsetfillcolor{currentfill}%
\pgfsetfillopacity{0.700000}%
\pgfsetlinewidth{0.501875pt}%
\definecolor{currentstroke}{rgb}{1.000000,1.000000,1.000000}%
\pgfsetstrokecolor{currentstroke}%
\pgfsetstrokeopacity{0.500000}%
\pgfsetdash{}{0pt}%
\pgfpathmoveto{\pgfqpoint{2.621426in}{2.410414in}}%
\pgfpathlineto{\pgfqpoint{2.632802in}{2.414731in}}%
\pgfpathlineto{\pgfqpoint{2.644168in}{2.419477in}}%
\pgfpathlineto{\pgfqpoint{2.655527in}{2.424420in}}%
\pgfpathlineto{\pgfqpoint{2.666882in}{2.429323in}}%
\pgfpathlineto{\pgfqpoint{2.678235in}{2.433946in}}%
\pgfpathlineto{\pgfqpoint{2.672174in}{2.442934in}}%
\pgfpathlineto{\pgfqpoint{2.666117in}{2.451920in}}%
\pgfpathlineto{\pgfqpoint{2.660064in}{2.460901in}}%
\pgfpathlineto{\pgfqpoint{2.654015in}{2.469870in}}%
\pgfpathlineto{\pgfqpoint{2.647970in}{2.478823in}}%
\pgfpathlineto{\pgfqpoint{2.636642in}{2.472938in}}%
\pgfpathlineto{\pgfqpoint{2.625306in}{2.467387in}}%
\pgfpathlineto{\pgfqpoint{2.613962in}{2.462222in}}%
\pgfpathlineto{\pgfqpoint{2.602608in}{2.457490in}}%
\pgfpathlineto{\pgfqpoint{2.591244in}{2.453241in}}%
\pgfpathlineto{\pgfqpoint{2.597273in}{2.444668in}}%
\pgfpathlineto{\pgfqpoint{2.603305in}{2.436102in}}%
\pgfpathlineto{\pgfqpoint{2.609342in}{2.427540in}}%
\pgfpathlineto{\pgfqpoint{2.615382in}{2.418978in}}%
\pgfpathclose%
\pgfusepath{stroke,fill}%
\end{pgfscope}%
\begin{pgfscope}%
\pgfpathrectangle{\pgfqpoint{0.887500in}{0.275000in}}{\pgfqpoint{4.225000in}{4.225000in}}%
\pgfusepath{clip}%
\pgfsetbuttcap%
\pgfsetroundjoin%
\definecolor{currentfill}{rgb}{0.699415,0.867117,0.175971}%
\pgfsetfillcolor{currentfill}%
\pgfsetfillopacity{0.700000}%
\pgfsetlinewidth{0.501875pt}%
\definecolor{currentstroke}{rgb}{1.000000,1.000000,1.000000}%
\pgfsetstrokecolor{currentstroke}%
\pgfsetstrokeopacity{0.500000}%
\pgfsetdash{}{0pt}%
\pgfpathmoveto{\pgfqpoint{3.267403in}{3.298597in}}%
\pgfpathlineto{\pgfqpoint{3.278630in}{3.301612in}}%
\pgfpathlineto{\pgfqpoint{3.289852in}{3.304718in}}%
\pgfpathlineto{\pgfqpoint{3.301069in}{3.307902in}}%
\pgfpathlineto{\pgfqpoint{3.312281in}{3.311152in}}%
\pgfpathlineto{\pgfqpoint{3.323488in}{3.314456in}}%
\pgfpathlineto{\pgfqpoint{3.317235in}{3.324566in}}%
\pgfpathlineto{\pgfqpoint{3.310986in}{3.334747in}}%
\pgfpathlineto{\pgfqpoint{3.304740in}{3.344965in}}%
\pgfpathlineto{\pgfqpoint{3.298498in}{3.355181in}}%
\pgfpathlineto{\pgfqpoint{3.292260in}{3.365358in}}%
\pgfpathlineto{\pgfqpoint{3.281060in}{3.361809in}}%
\pgfpathlineto{\pgfqpoint{3.269855in}{3.358394in}}%
\pgfpathlineto{\pgfqpoint{3.258646in}{3.355134in}}%
\pgfpathlineto{\pgfqpoint{3.247432in}{3.352047in}}%
\pgfpathlineto{\pgfqpoint{3.236213in}{3.349142in}}%
\pgfpathlineto{\pgfqpoint{3.242444in}{3.339028in}}%
\pgfpathlineto{\pgfqpoint{3.248679in}{3.328882in}}%
\pgfpathlineto{\pgfqpoint{3.254916in}{3.318739in}}%
\pgfpathlineto{\pgfqpoint{3.261158in}{3.308634in}}%
\pgfpathclose%
\pgfusepath{stroke,fill}%
\end{pgfscope}%
\begin{pgfscope}%
\pgfpathrectangle{\pgfqpoint{0.887500in}{0.275000in}}{\pgfqpoint{4.225000in}{4.225000in}}%
\pgfusepath{clip}%
\pgfsetbuttcap%
\pgfsetroundjoin%
\definecolor{currentfill}{rgb}{0.121380,0.629492,0.531973}%
\pgfsetfillcolor{currentfill}%
\pgfsetfillopacity{0.700000}%
\pgfsetlinewidth{0.501875pt}%
\definecolor{currentstroke}{rgb}{1.000000,1.000000,1.000000}%
\pgfsetstrokecolor{currentstroke}%
\pgfsetstrokeopacity{0.500000}%
\pgfsetdash{}{0pt}%
\pgfpathmoveto{\pgfqpoint{2.847272in}{2.632014in}}%
\pgfpathlineto{\pgfqpoint{2.858458in}{2.661950in}}%
\pgfpathlineto{\pgfqpoint{2.869659in}{2.690866in}}%
\pgfpathlineto{\pgfqpoint{2.880874in}{2.718795in}}%
\pgfpathlineto{\pgfqpoint{2.892099in}{2.746015in}}%
\pgfpathlineto{\pgfqpoint{2.903335in}{2.772806in}}%
\pgfpathlineto{\pgfqpoint{2.897232in}{2.771500in}}%
\pgfpathlineto{\pgfqpoint{2.891139in}{2.769667in}}%
\pgfpathlineto{\pgfqpoint{2.885054in}{2.767626in}}%
\pgfpathlineto{\pgfqpoint{2.878975in}{2.765699in}}%
\pgfpathlineto{\pgfqpoint{2.872902in}{2.764206in}}%
\pgfpathlineto{\pgfqpoint{2.861693in}{2.740731in}}%
\pgfpathlineto{\pgfqpoint{2.850486in}{2.718128in}}%
\pgfpathlineto{\pgfqpoint{2.839281in}{2.696284in}}%
\pgfpathlineto{\pgfqpoint{2.828077in}{2.675088in}}%
\pgfpathlineto{\pgfqpoint{2.816873in}{2.654485in}}%
\pgfpathlineto{\pgfqpoint{2.822943in}{2.650018in}}%
\pgfpathlineto{\pgfqpoint{2.829016in}{2.645683in}}%
\pgfpathlineto{\pgfqpoint{2.835095in}{2.641335in}}%
\pgfpathlineto{\pgfqpoint{2.841180in}{2.636827in}}%
\pgfpathclose%
\pgfusepath{stroke,fill}%
\end{pgfscope}%
\begin{pgfscope}%
\pgfpathrectangle{\pgfqpoint{0.887500in}{0.275000in}}{\pgfqpoint{4.225000in}{4.225000in}}%
\pgfusepath{clip}%
\pgfsetbuttcap%
\pgfsetroundjoin%
\definecolor{currentfill}{rgb}{0.125394,0.574318,0.549086}%
\pgfsetfillcolor{currentfill}%
\pgfsetfillopacity{0.700000}%
\pgfsetlinewidth{0.501875pt}%
\definecolor{currentstroke}{rgb}{1.000000,1.000000,1.000000}%
\pgfsetstrokecolor{currentstroke}%
\pgfsetstrokeopacity{0.500000}%
\pgfsetdash{}{0pt}%
\pgfpathmoveto{\pgfqpoint{1.753096in}{2.560475in}}%
\pgfpathlineto{\pgfqpoint{1.764699in}{2.563745in}}%
\pgfpathlineto{\pgfqpoint{1.776296in}{2.567006in}}%
\pgfpathlineto{\pgfqpoint{1.787887in}{2.570259in}}%
\pgfpathlineto{\pgfqpoint{1.799473in}{2.573506in}}%
\pgfpathlineto{\pgfqpoint{1.811054in}{2.576748in}}%
\pgfpathlineto{\pgfqpoint{1.805295in}{2.584755in}}%
\pgfpathlineto{\pgfqpoint{1.799540in}{2.592745in}}%
\pgfpathlineto{\pgfqpoint{1.793789in}{2.600718in}}%
\pgfpathlineto{\pgfqpoint{1.788043in}{2.608673in}}%
\pgfpathlineto{\pgfqpoint{1.782301in}{2.616610in}}%
\pgfpathlineto{\pgfqpoint{1.770734in}{2.613369in}}%
\pgfpathlineto{\pgfqpoint{1.759161in}{2.610124in}}%
\pgfpathlineto{\pgfqpoint{1.747582in}{2.606876in}}%
\pgfpathlineto{\pgfqpoint{1.735998in}{2.603622in}}%
\pgfpathlineto{\pgfqpoint{1.724408in}{2.600362in}}%
\pgfpathlineto{\pgfqpoint{1.730137in}{2.592423in}}%
\pgfpathlineto{\pgfqpoint{1.735870in}{2.584464in}}%
\pgfpathlineto{\pgfqpoint{1.741608in}{2.576486in}}%
\pgfpathlineto{\pgfqpoint{1.747350in}{2.568490in}}%
\pgfpathclose%
\pgfusepath{stroke,fill}%
\end{pgfscope}%
\begin{pgfscope}%
\pgfpathrectangle{\pgfqpoint{0.887500in}{0.275000in}}{\pgfqpoint{4.225000in}{4.225000in}}%
\pgfusepath{clip}%
\pgfsetbuttcap%
\pgfsetroundjoin%
\definecolor{currentfill}{rgb}{0.344074,0.780029,0.397381}%
\pgfsetfillcolor{currentfill}%
\pgfsetfillopacity{0.700000}%
\pgfsetlinewidth{0.501875pt}%
\definecolor{currentstroke}{rgb}{1.000000,1.000000,1.000000}%
\pgfsetstrokecolor{currentstroke}%
\pgfsetstrokeopacity{0.500000}%
\pgfsetdash{}{0pt}%
\pgfpathmoveto{\pgfqpoint{3.878761in}{3.000252in}}%
\pgfpathlineto{\pgfqpoint{3.889844in}{3.003775in}}%
\pgfpathlineto{\pgfqpoint{3.900922in}{3.007309in}}%
\pgfpathlineto{\pgfqpoint{3.911996in}{3.010853in}}%
\pgfpathlineto{\pgfqpoint{3.923064in}{3.014396in}}%
\pgfpathlineto{\pgfqpoint{3.934127in}{3.017924in}}%
\pgfpathlineto{\pgfqpoint{3.927773in}{3.031328in}}%
\pgfpathlineto{\pgfqpoint{3.921419in}{3.044595in}}%
\pgfpathlineto{\pgfqpoint{3.915065in}{3.057702in}}%
\pgfpathlineto{\pgfqpoint{3.908709in}{3.070629in}}%
\pgfpathlineto{\pgfqpoint{3.902352in}{3.083355in}}%
\pgfpathlineto{\pgfqpoint{3.891294in}{3.079875in}}%
\pgfpathlineto{\pgfqpoint{3.880230in}{3.076374in}}%
\pgfpathlineto{\pgfqpoint{3.869161in}{3.072865in}}%
\pgfpathlineto{\pgfqpoint{3.858087in}{3.069364in}}%
\pgfpathlineto{\pgfqpoint{3.847007in}{3.065873in}}%
\pgfpathlineto{\pgfqpoint{3.853360in}{3.053125in}}%
\pgfpathlineto{\pgfqpoint{3.859711in}{3.040166in}}%
\pgfpathlineto{\pgfqpoint{3.866062in}{3.027017in}}%
\pgfpathlineto{\pgfqpoint{3.872411in}{3.013704in}}%
\pgfpathclose%
\pgfusepath{stroke,fill}%
\end{pgfscope}%
\begin{pgfscope}%
\pgfpathrectangle{\pgfqpoint{0.887500in}{0.275000in}}{\pgfqpoint{4.225000in}{4.225000in}}%
\pgfusepath{clip}%
\pgfsetbuttcap%
\pgfsetroundjoin%
\definecolor{currentfill}{rgb}{0.647257,0.858400,0.209861}%
\pgfsetfillcolor{currentfill}%
\pgfsetfillopacity{0.700000}%
\pgfsetlinewidth{0.501875pt}%
\definecolor{currentstroke}{rgb}{1.000000,1.000000,1.000000}%
\pgfsetstrokecolor{currentstroke}%
\pgfsetstrokeopacity{0.500000}%
\pgfsetdash{}{0pt}%
\pgfpathmoveto{\pgfqpoint{2.979960in}{3.214469in}}%
\pgfpathlineto{\pgfqpoint{2.991205in}{3.244118in}}%
\pgfpathlineto{\pgfqpoint{3.002462in}{3.270499in}}%
\pgfpathlineto{\pgfqpoint{3.013732in}{3.293203in}}%
\pgfpathlineto{\pgfqpoint{3.025009in}{3.312486in}}%
\pgfpathlineto{\pgfqpoint{3.036291in}{3.328661in}}%
\pgfpathlineto{\pgfqpoint{3.030110in}{3.336970in}}%
\pgfpathlineto{\pgfqpoint{3.023934in}{3.345037in}}%
\pgfpathlineto{\pgfqpoint{3.017762in}{3.352855in}}%
\pgfpathlineto{\pgfqpoint{3.011594in}{3.360413in}}%
\pgfpathlineto{\pgfqpoint{3.005430in}{3.367702in}}%
\pgfpathlineto{\pgfqpoint{2.994173in}{3.349456in}}%
\pgfpathlineto{\pgfqpoint{2.982925in}{3.327278in}}%
\pgfpathlineto{\pgfqpoint{2.971692in}{3.300695in}}%
\pgfpathlineto{\pgfqpoint{2.960477in}{3.269322in}}%
\pgfpathlineto{\pgfqpoint{2.949284in}{3.233858in}}%
\pgfpathlineto{\pgfqpoint{2.955408in}{3.230522in}}%
\pgfpathlineto{\pgfqpoint{2.961537in}{3.226916in}}%
\pgfpathlineto{\pgfqpoint{2.967672in}{3.223039in}}%
\pgfpathlineto{\pgfqpoint{2.973813in}{3.218889in}}%
\pgfpathclose%
\pgfusepath{stroke,fill}%
\end{pgfscope}%
\begin{pgfscope}%
\pgfpathrectangle{\pgfqpoint{0.887500in}{0.275000in}}{\pgfqpoint{4.225000in}{4.225000in}}%
\pgfusepath{clip}%
\pgfsetbuttcap%
\pgfsetroundjoin%
\definecolor{currentfill}{rgb}{0.137339,0.662252,0.515571}%
\pgfsetfillcolor{currentfill}%
\pgfsetfillopacity{0.700000}%
\pgfsetlinewidth{0.501875pt}%
\definecolor{currentstroke}{rgb}{1.000000,1.000000,1.000000}%
\pgfsetstrokecolor{currentstroke}%
\pgfsetstrokeopacity{0.500000}%
\pgfsetdash{}{0pt}%
\pgfpathmoveto{\pgfqpoint{4.259043in}{2.724504in}}%
\pgfpathlineto{\pgfqpoint{4.270031in}{2.727930in}}%
\pgfpathlineto{\pgfqpoint{4.281014in}{2.731357in}}%
\pgfpathlineto{\pgfqpoint{4.291992in}{2.734784in}}%
\pgfpathlineto{\pgfqpoint{4.302965in}{2.738213in}}%
\pgfpathlineto{\pgfqpoint{4.313933in}{2.741646in}}%
\pgfpathlineto{\pgfqpoint{4.307523in}{2.755732in}}%
\pgfpathlineto{\pgfqpoint{4.301116in}{2.769800in}}%
\pgfpathlineto{\pgfqpoint{4.294710in}{2.783855in}}%
\pgfpathlineto{\pgfqpoint{4.288307in}{2.797900in}}%
\pgfpathlineto{\pgfqpoint{4.281907in}{2.811940in}}%
\pgfpathlineto{\pgfqpoint{4.270939in}{2.808455in}}%
\pgfpathlineto{\pgfqpoint{4.259967in}{2.804979in}}%
\pgfpathlineto{\pgfqpoint{4.248990in}{2.801509in}}%
\pgfpathlineto{\pgfqpoint{4.238007in}{2.798046in}}%
\pgfpathlineto{\pgfqpoint{4.227020in}{2.794587in}}%
\pgfpathlineto{\pgfqpoint{4.233420in}{2.780582in}}%
\pgfpathlineto{\pgfqpoint{4.239822in}{2.766576in}}%
\pgfpathlineto{\pgfqpoint{4.246227in}{2.752564in}}%
\pgfpathlineto{\pgfqpoint{4.252634in}{2.738542in}}%
\pgfpathclose%
\pgfusepath{stroke,fill}%
\end{pgfscope}%
\begin{pgfscope}%
\pgfpathrectangle{\pgfqpoint{0.887500in}{0.275000in}}{\pgfqpoint{4.225000in}{4.225000in}}%
\pgfusepath{clip}%
\pgfsetbuttcap%
\pgfsetroundjoin%
\definecolor{currentfill}{rgb}{0.132444,0.552216,0.553018}%
\pgfsetfillcolor{currentfill}%
\pgfsetfillopacity{0.700000}%
\pgfsetlinewidth{0.501875pt}%
\definecolor{currentstroke}{rgb}{1.000000,1.000000,1.000000}%
\pgfsetstrokecolor{currentstroke}%
\pgfsetstrokeopacity{0.500000}%
\pgfsetdash{}{0pt}%
\pgfpathmoveto{\pgfqpoint{2.821992in}{2.432148in}}%
\pgfpathlineto{\pgfqpoint{2.833179in}{2.458525in}}%
\pgfpathlineto{\pgfqpoint{2.844348in}{2.489759in}}%
\pgfpathlineto{\pgfqpoint{2.855510in}{2.524418in}}%
\pgfpathlineto{\pgfqpoint{2.866675in}{2.561066in}}%
\pgfpathlineto{\pgfqpoint{2.877851in}{2.598257in}}%
\pgfpathlineto{\pgfqpoint{2.871718in}{2.606788in}}%
\pgfpathlineto{\pgfqpoint{2.865593in}{2.614283in}}%
\pgfpathlineto{\pgfqpoint{2.859478in}{2.620888in}}%
\pgfpathlineto{\pgfqpoint{2.853371in}{2.626750in}}%
\pgfpathlineto{\pgfqpoint{2.847272in}{2.632014in}}%
\pgfpathlineto{\pgfqpoint{2.836097in}{2.601827in}}%
\pgfpathlineto{\pgfqpoint{2.824927in}{2.572314in}}%
\pgfpathlineto{\pgfqpoint{2.813757in}{2.544397in}}%
\pgfpathlineto{\pgfqpoint{2.802580in}{2.518994in}}%
\pgfpathlineto{\pgfqpoint{2.791386in}{2.497020in}}%
\pgfpathlineto{\pgfqpoint{2.797508in}{2.483165in}}%
\pgfpathlineto{\pgfqpoint{2.803630in}{2.469610in}}%
\pgfpathlineto{\pgfqpoint{2.809752in}{2.456496in}}%
\pgfpathlineto{\pgfqpoint{2.815872in}{2.443962in}}%
\pgfpathclose%
\pgfusepath{stroke,fill}%
\end{pgfscope}%
\begin{pgfscope}%
\pgfpathrectangle{\pgfqpoint{0.887500in}{0.275000in}}{\pgfqpoint{4.225000in}{4.225000in}}%
\pgfusepath{clip}%
\pgfsetbuttcap%
\pgfsetroundjoin%
\definecolor{currentfill}{rgb}{0.657642,0.860219,0.203082}%
\pgfsetfillcolor{currentfill}%
\pgfsetfillopacity{0.700000}%
\pgfsetlinewidth{0.501875pt}%
\definecolor{currentstroke}{rgb}{1.000000,1.000000,1.000000}%
\pgfsetstrokecolor{currentstroke}%
\pgfsetstrokeopacity{0.500000}%
\pgfsetdash{}{0pt}%
\pgfpathmoveto{\pgfqpoint{3.354808in}{3.264175in}}%
\pgfpathlineto{\pgfqpoint{3.366015in}{3.267138in}}%
\pgfpathlineto{\pgfqpoint{3.377216in}{3.270051in}}%
\pgfpathlineto{\pgfqpoint{3.388411in}{3.272921in}}%
\pgfpathlineto{\pgfqpoint{3.399600in}{3.275750in}}%
\pgfpathlineto{\pgfqpoint{3.410783in}{3.278548in}}%
\pgfpathlineto{\pgfqpoint{3.404509in}{3.289156in}}%
\pgfpathlineto{\pgfqpoint{3.398238in}{3.299731in}}%
\pgfpathlineto{\pgfqpoint{3.391970in}{3.310290in}}%
\pgfpathlineto{\pgfqpoint{3.385706in}{3.320847in}}%
\pgfpathlineto{\pgfqpoint{3.379445in}{3.331420in}}%
\pgfpathlineto{\pgfqpoint{3.368264in}{3.328004in}}%
\pgfpathlineto{\pgfqpoint{3.357078in}{3.324589in}}%
\pgfpathlineto{\pgfqpoint{3.345886in}{3.321186in}}%
\pgfpathlineto{\pgfqpoint{3.334690in}{3.317805in}}%
\pgfpathlineto{\pgfqpoint{3.323488in}{3.314456in}}%
\pgfpathlineto{\pgfqpoint{3.329745in}{3.304411in}}%
\pgfpathlineto{\pgfqpoint{3.336006in}{3.294395in}}%
\pgfpathlineto{\pgfqpoint{3.342270in}{3.284373in}}%
\pgfpathlineto{\pgfqpoint{3.348538in}{3.274312in}}%
\pgfpathclose%
\pgfusepath{stroke,fill}%
\end{pgfscope}%
\begin{pgfscope}%
\pgfpathrectangle{\pgfqpoint{0.887500in}{0.275000in}}{\pgfqpoint{4.225000in}{4.225000in}}%
\pgfusepath{clip}%
\pgfsetbuttcap%
\pgfsetroundjoin%
\definecolor{currentfill}{rgb}{0.395174,0.797475,0.367757}%
\pgfsetfillcolor{currentfill}%
\pgfsetfillopacity{0.700000}%
\pgfsetlinewidth{0.501875pt}%
\definecolor{currentstroke}{rgb}{1.000000,1.000000,1.000000}%
\pgfsetstrokecolor{currentstroke}%
\pgfsetstrokeopacity{0.500000}%
\pgfsetdash{}{0pt}%
\pgfpathmoveto{\pgfqpoint{3.791532in}{3.048371in}}%
\pgfpathlineto{\pgfqpoint{3.802638in}{3.051898in}}%
\pgfpathlineto{\pgfqpoint{3.813739in}{3.055406in}}%
\pgfpathlineto{\pgfqpoint{3.824834in}{3.058900in}}%
\pgfpathlineto{\pgfqpoint{3.835923in}{3.062387in}}%
\pgfpathlineto{\pgfqpoint{3.847007in}{3.065873in}}%
\pgfpathlineto{\pgfqpoint{3.840654in}{3.078421in}}%
\pgfpathlineto{\pgfqpoint{3.834300in}{3.090799in}}%
\pgfpathlineto{\pgfqpoint{3.827946in}{3.103036in}}%
\pgfpathlineto{\pgfqpoint{3.821593in}{3.115164in}}%
\pgfpathlineto{\pgfqpoint{3.815242in}{3.127210in}}%
\pgfpathlineto{\pgfqpoint{3.804164in}{3.123809in}}%
\pgfpathlineto{\pgfqpoint{3.793081in}{3.120429in}}%
\pgfpathlineto{\pgfqpoint{3.781993in}{3.117059in}}%
\pgfpathlineto{\pgfqpoint{3.770899in}{3.113690in}}%
\pgfpathlineto{\pgfqpoint{3.759801in}{3.110313in}}%
\pgfpathlineto{\pgfqpoint{3.766145in}{3.098131in}}%
\pgfpathlineto{\pgfqpoint{3.772490in}{3.085870in}}%
\pgfpathlineto{\pgfqpoint{3.778837in}{3.073506in}}%
\pgfpathlineto{\pgfqpoint{3.785184in}{3.061014in}}%
\pgfpathclose%
\pgfusepath{stroke,fill}%
\end{pgfscope}%
\begin{pgfscope}%
\pgfpathrectangle{\pgfqpoint{0.887500in}{0.275000in}}{\pgfqpoint{4.225000in}{4.225000in}}%
\pgfusepath{clip}%
\pgfsetbuttcap%
\pgfsetroundjoin%
\definecolor{currentfill}{rgb}{0.133743,0.548535,0.553541}%
\pgfsetfillcolor{currentfill}%
\pgfsetfillopacity{0.700000}%
\pgfsetlinewidth{0.501875pt}%
\definecolor{currentstroke}{rgb}{1.000000,1.000000,1.000000}%
\pgfsetstrokecolor{currentstroke}%
\pgfsetstrokeopacity{0.500000}%
\pgfsetdash{}{0pt}%
\pgfpathmoveto{\pgfqpoint{2.071510in}{2.504261in}}%
\pgfpathlineto{\pgfqpoint{2.083035in}{2.507591in}}%
\pgfpathlineto{\pgfqpoint{2.094554in}{2.510911in}}%
\pgfpathlineto{\pgfqpoint{2.106067in}{2.514222in}}%
\pgfpathlineto{\pgfqpoint{2.117575in}{2.517524in}}%
\pgfpathlineto{\pgfqpoint{2.129078in}{2.520818in}}%
\pgfpathlineto{\pgfqpoint{2.123207in}{2.529029in}}%
\pgfpathlineto{\pgfqpoint{2.117339in}{2.537224in}}%
\pgfpathlineto{\pgfqpoint{2.111476in}{2.545403in}}%
\pgfpathlineto{\pgfqpoint{2.105617in}{2.553568in}}%
\pgfpathlineto{\pgfqpoint{2.099762in}{2.561719in}}%
\pgfpathlineto{\pgfqpoint{2.088272in}{2.558419in}}%
\pgfpathlineto{\pgfqpoint{2.076776in}{2.555112in}}%
\pgfpathlineto{\pgfqpoint{2.065275in}{2.551798in}}%
\pgfpathlineto{\pgfqpoint{2.053769in}{2.548477in}}%
\pgfpathlineto{\pgfqpoint{2.042257in}{2.545150in}}%
\pgfpathlineto{\pgfqpoint{2.048099in}{2.537005in}}%
\pgfpathlineto{\pgfqpoint{2.053945in}{2.528845in}}%
\pgfpathlineto{\pgfqpoint{2.059796in}{2.520668in}}%
\pgfpathlineto{\pgfqpoint{2.065651in}{2.512473in}}%
\pgfpathclose%
\pgfusepath{stroke,fill}%
\end{pgfscope}%
\begin{pgfscope}%
\pgfpathrectangle{\pgfqpoint{0.887500in}{0.275000in}}{\pgfqpoint{4.225000in}{4.225000in}}%
\pgfusepath{clip}%
\pgfsetbuttcap%
\pgfsetroundjoin%
\definecolor{currentfill}{rgb}{0.440137,0.811138,0.340967}%
\pgfsetfillcolor{currentfill}%
\pgfsetfillopacity{0.700000}%
\pgfsetlinewidth{0.501875pt}%
\definecolor{currentstroke}{rgb}{1.000000,1.000000,1.000000}%
\pgfsetstrokecolor{currentstroke}%
\pgfsetstrokeopacity{0.500000}%
\pgfsetdash{}{0pt}%
\pgfpathmoveto{\pgfqpoint{3.704226in}{3.093312in}}%
\pgfpathlineto{\pgfqpoint{3.715351in}{3.096696in}}%
\pgfpathlineto{\pgfqpoint{3.726471in}{3.100099in}}%
\pgfpathlineto{\pgfqpoint{3.737586in}{3.103510in}}%
\pgfpathlineto{\pgfqpoint{3.748696in}{3.106918in}}%
\pgfpathlineto{\pgfqpoint{3.759801in}{3.110313in}}%
\pgfpathlineto{\pgfqpoint{3.753459in}{3.122439in}}%
\pgfpathlineto{\pgfqpoint{3.747119in}{3.134534in}}%
\pgfpathlineto{\pgfqpoint{3.740783in}{3.146622in}}%
\pgfpathlineto{\pgfqpoint{3.734449in}{3.158726in}}%
\pgfpathlineto{\pgfqpoint{3.728120in}{3.170871in}}%
\pgfpathlineto{\pgfqpoint{3.717022in}{3.167602in}}%
\pgfpathlineto{\pgfqpoint{3.705920in}{3.164326in}}%
\pgfpathlineto{\pgfqpoint{3.694812in}{3.161055in}}%
\pgfpathlineto{\pgfqpoint{3.683698in}{3.157799in}}%
\pgfpathlineto{\pgfqpoint{3.672580in}{3.154570in}}%
\pgfpathlineto{\pgfqpoint{3.678904in}{3.142371in}}%
\pgfpathlineto{\pgfqpoint{3.685231in}{3.130159in}}%
\pgfpathlineto{\pgfqpoint{3.691560in}{3.117921in}}%
\pgfpathlineto{\pgfqpoint{3.697892in}{3.105643in}}%
\pgfpathclose%
\pgfusepath{stroke,fill}%
\end{pgfscope}%
\begin{pgfscope}%
\pgfpathrectangle{\pgfqpoint{0.887500in}{0.275000in}}{\pgfqpoint{4.225000in}{4.225000in}}%
\pgfusepath{clip}%
\pgfsetbuttcap%
\pgfsetroundjoin%
\definecolor{currentfill}{rgb}{0.606045,0.850733,0.236712}%
\pgfsetfillcolor{currentfill}%
\pgfsetfillopacity{0.700000}%
\pgfsetlinewidth{0.501875pt}%
\definecolor{currentstroke}{rgb}{1.000000,1.000000,1.000000}%
\pgfsetstrokecolor{currentstroke}%
\pgfsetstrokeopacity{0.500000}%
\pgfsetdash{}{0pt}%
\pgfpathmoveto{\pgfqpoint{3.442196in}{3.224470in}}%
\pgfpathlineto{\pgfqpoint{3.453377in}{3.226978in}}%
\pgfpathlineto{\pgfqpoint{3.464553in}{3.229513in}}%
\pgfpathlineto{\pgfqpoint{3.475723in}{3.232118in}}%
\pgfpathlineto{\pgfqpoint{3.486889in}{3.234838in}}%
\pgfpathlineto{\pgfqpoint{3.498051in}{3.237716in}}%
\pgfpathlineto{\pgfqpoint{3.491758in}{3.248767in}}%
\pgfpathlineto{\pgfqpoint{3.485469in}{3.259831in}}%
\pgfpathlineto{\pgfqpoint{3.479183in}{3.270907in}}%
\pgfpathlineto{\pgfqpoint{3.472900in}{3.281992in}}%
\pgfpathlineto{\pgfqpoint{3.466620in}{3.293087in}}%
\pgfpathlineto{\pgfqpoint{3.455462in}{3.289999in}}%
\pgfpathlineto{\pgfqpoint{3.444300in}{3.287033in}}%
\pgfpathlineto{\pgfqpoint{3.433133in}{3.284157in}}%
\pgfpathlineto{\pgfqpoint{3.421961in}{3.281340in}}%
\pgfpathlineto{\pgfqpoint{3.410783in}{3.278548in}}%
\pgfpathlineto{\pgfqpoint{3.417061in}{3.267892in}}%
\pgfpathlineto{\pgfqpoint{3.423341in}{3.257172in}}%
\pgfpathlineto{\pgfqpoint{3.429623in}{3.246372in}}%
\pgfpathlineto{\pgfqpoint{3.435908in}{3.235477in}}%
\pgfpathclose%
\pgfusepath{stroke,fill}%
\end{pgfscope}%
\begin{pgfscope}%
\pgfpathrectangle{\pgfqpoint{0.887500in}{0.275000in}}{\pgfqpoint{4.225000in}{4.225000in}}%
\pgfusepath{clip}%
\pgfsetbuttcap%
\pgfsetroundjoin%
\definecolor{currentfill}{rgb}{0.440137,0.811138,0.340967}%
\pgfsetfillcolor{currentfill}%
\pgfsetfillopacity{0.700000}%
\pgfsetlinewidth{0.501875pt}%
\definecolor{currentstroke}{rgb}{1.000000,1.000000,1.000000}%
\pgfsetstrokecolor{currentstroke}%
\pgfsetstrokeopacity{0.500000}%
\pgfsetdash{}{0pt}%
\pgfpathmoveto{\pgfqpoint{2.954508in}{3.044847in}}%
\pgfpathlineto{\pgfqpoint{2.965740in}{3.075408in}}%
\pgfpathlineto{\pgfqpoint{2.976983in}{3.105468in}}%
\pgfpathlineto{\pgfqpoint{2.988237in}{3.134632in}}%
\pgfpathlineto{\pgfqpoint{2.999502in}{3.162498in}}%
\pgfpathlineto{\pgfqpoint{3.010777in}{3.188668in}}%
\pgfpathlineto{\pgfqpoint{3.004603in}{3.194286in}}%
\pgfpathlineto{\pgfqpoint{2.998435in}{3.199686in}}%
\pgfpathlineto{\pgfqpoint{2.992271in}{3.204857in}}%
\pgfpathlineto{\pgfqpoint{2.986113in}{3.209788in}}%
\pgfpathlineto{\pgfqpoint{2.979960in}{3.214469in}}%
\pgfpathlineto{\pgfqpoint{2.968731in}{3.182398in}}%
\pgfpathlineto{\pgfqpoint{2.957516in}{3.148753in}}%
\pgfpathlineto{\pgfqpoint{2.946315in}{3.114384in}}%
\pgfpathlineto{\pgfqpoint{2.935125in}{3.080130in}}%
\pgfpathlineto{\pgfqpoint{2.923944in}{3.046830in}}%
\pgfpathlineto{\pgfqpoint{2.930044in}{3.046677in}}%
\pgfpathlineto{\pgfqpoint{2.936150in}{3.046372in}}%
\pgfpathlineto{\pgfqpoint{2.942263in}{3.045945in}}%
\pgfpathlineto{\pgfqpoint{2.948382in}{3.045427in}}%
\pgfpathclose%
\pgfusepath{stroke,fill}%
\end{pgfscope}%
\begin{pgfscope}%
\pgfpathrectangle{\pgfqpoint{0.887500in}{0.275000in}}{\pgfqpoint{4.225000in}{4.225000in}}%
\pgfusepath{clip}%
\pgfsetbuttcap%
\pgfsetroundjoin%
\definecolor{currentfill}{rgb}{0.555484,0.840254,0.269281}%
\pgfsetfillcolor{currentfill}%
\pgfsetfillopacity{0.700000}%
\pgfsetlinewidth{0.501875pt}%
\definecolor{currentstroke}{rgb}{1.000000,1.000000,1.000000}%
\pgfsetstrokecolor{currentstroke}%
\pgfsetstrokeopacity{0.500000}%
\pgfsetdash{}{0pt}%
\pgfpathmoveto{\pgfqpoint{3.529562in}{3.182112in}}%
\pgfpathlineto{\pgfqpoint{3.540727in}{3.185169in}}%
\pgfpathlineto{\pgfqpoint{3.551888in}{3.188317in}}%
\pgfpathlineto{\pgfqpoint{3.563044in}{3.191546in}}%
\pgfpathlineto{\pgfqpoint{3.574195in}{3.194839in}}%
\pgfpathlineto{\pgfqpoint{3.585342in}{3.198183in}}%
\pgfpathlineto{\pgfqpoint{3.579032in}{3.209691in}}%
\pgfpathlineto{\pgfqpoint{3.572724in}{3.221118in}}%
\pgfpathlineto{\pgfqpoint{3.566418in}{3.232473in}}%
\pgfpathlineto{\pgfqpoint{3.560115in}{3.243764in}}%
\pgfpathlineto{\pgfqpoint{3.553814in}{3.255001in}}%
\pgfpathlineto{\pgfqpoint{3.542669in}{3.251253in}}%
\pgfpathlineto{\pgfqpoint{3.531519in}{3.247604in}}%
\pgfpathlineto{\pgfqpoint{3.520366in}{3.244102in}}%
\pgfpathlineto{\pgfqpoint{3.509210in}{3.240795in}}%
\pgfpathlineto{\pgfqpoint{3.498051in}{3.237716in}}%
\pgfpathlineto{\pgfqpoint{3.504348in}{3.226670in}}%
\pgfpathlineto{\pgfqpoint{3.510647in}{3.215610in}}%
\pgfpathlineto{\pgfqpoint{3.516950in}{3.204512in}}%
\pgfpathlineto{\pgfqpoint{3.523255in}{3.193354in}}%
\pgfpathclose%
\pgfusepath{stroke,fill}%
\end{pgfscope}%
\begin{pgfscope}%
\pgfpathrectangle{\pgfqpoint{0.887500in}{0.275000in}}{\pgfqpoint{4.225000in}{4.225000in}}%
\pgfusepath{clip}%
\pgfsetbuttcap%
\pgfsetroundjoin%
\definecolor{currentfill}{rgb}{0.143343,0.522773,0.556295}%
\pgfsetfillcolor{currentfill}%
\pgfsetfillopacity{0.700000}%
\pgfsetlinewidth{0.501875pt}%
\definecolor{currentstroke}{rgb}{1.000000,1.000000,1.000000}%
\pgfsetstrokecolor{currentstroke}%
\pgfsetstrokeopacity{0.500000}%
\pgfsetdash{}{0pt}%
\pgfpathmoveto{\pgfqpoint{2.390043in}{2.445560in}}%
\pgfpathlineto{\pgfqpoint{2.401488in}{2.449013in}}%
\pgfpathlineto{\pgfqpoint{2.412928in}{2.452501in}}%
\pgfpathlineto{\pgfqpoint{2.424361in}{2.456018in}}%
\pgfpathlineto{\pgfqpoint{2.435789in}{2.459541in}}%
\pgfpathlineto{\pgfqpoint{2.447211in}{2.463042in}}%
\pgfpathlineto{\pgfqpoint{2.441230in}{2.471466in}}%
\pgfpathlineto{\pgfqpoint{2.435253in}{2.479880in}}%
\pgfpathlineto{\pgfqpoint{2.429280in}{2.488285in}}%
\pgfpathlineto{\pgfqpoint{2.423311in}{2.496682in}}%
\pgfpathlineto{\pgfqpoint{2.417346in}{2.505072in}}%
\pgfpathlineto{\pgfqpoint{2.405936in}{2.501568in}}%
\pgfpathlineto{\pgfqpoint{2.394520in}{2.498041in}}%
\pgfpathlineto{\pgfqpoint{2.383099in}{2.494521in}}%
\pgfpathlineto{\pgfqpoint{2.371671in}{2.491029in}}%
\pgfpathlineto{\pgfqpoint{2.360238in}{2.487574in}}%
\pgfpathlineto{\pgfqpoint{2.366191in}{2.479204in}}%
\pgfpathlineto{\pgfqpoint{2.372148in}{2.470817in}}%
\pgfpathlineto{\pgfqpoint{2.378109in}{2.462414in}}%
\pgfpathlineto{\pgfqpoint{2.384074in}{2.453995in}}%
\pgfpathclose%
\pgfusepath{stroke,fill}%
\end{pgfscope}%
\begin{pgfscope}%
\pgfpathrectangle{\pgfqpoint{0.887500in}{0.275000in}}{\pgfqpoint{4.225000in}{4.225000in}}%
\pgfusepath{clip}%
\pgfsetbuttcap%
\pgfsetroundjoin%
\definecolor{currentfill}{rgb}{0.496615,0.826376,0.306377}%
\pgfsetfillcolor{currentfill}%
\pgfsetfillopacity{0.700000}%
\pgfsetlinewidth{0.501875pt}%
\definecolor{currentstroke}{rgb}{1.000000,1.000000,1.000000}%
\pgfsetstrokecolor{currentstroke}%
\pgfsetstrokeopacity{0.500000}%
\pgfsetdash{}{0pt}%
\pgfpathmoveto{\pgfqpoint{3.616919in}{3.139111in}}%
\pgfpathlineto{\pgfqpoint{3.628061in}{3.142118in}}%
\pgfpathlineto{\pgfqpoint{3.639198in}{3.145154in}}%
\pgfpathlineto{\pgfqpoint{3.650330in}{3.148237in}}%
\pgfpathlineto{\pgfqpoint{3.661457in}{3.151379in}}%
\pgfpathlineto{\pgfqpoint{3.672580in}{3.154570in}}%
\pgfpathlineto{\pgfqpoint{3.666259in}{3.166758in}}%
\pgfpathlineto{\pgfqpoint{3.659941in}{3.178923in}}%
\pgfpathlineto{\pgfqpoint{3.653625in}{3.191051in}}%
\pgfpathlineto{\pgfqpoint{3.647311in}{3.203127in}}%
\pgfpathlineto{\pgfqpoint{3.640999in}{3.215137in}}%
\pgfpathlineto{\pgfqpoint{3.629878in}{3.211754in}}%
\pgfpathlineto{\pgfqpoint{3.618752in}{3.208360in}}%
\pgfpathlineto{\pgfqpoint{3.607620in}{3.204959in}}%
\pgfpathlineto{\pgfqpoint{3.596484in}{3.201561in}}%
\pgfpathlineto{\pgfqpoint{3.585342in}{3.198183in}}%
\pgfpathlineto{\pgfqpoint{3.591654in}{3.186585in}}%
\pgfpathlineto{\pgfqpoint{3.597968in}{3.174888in}}%
\pgfpathlineto{\pgfqpoint{3.604283in}{3.163082in}}%
\pgfpathlineto{\pgfqpoint{3.610600in}{3.151159in}}%
\pgfpathclose%
\pgfusepath{stroke,fill}%
\end{pgfscope}%
\begin{pgfscope}%
\pgfpathrectangle{\pgfqpoint{0.887500in}{0.275000in}}{\pgfqpoint{4.225000in}{4.225000in}}%
\pgfusepath{clip}%
\pgfsetbuttcap%
\pgfsetroundjoin%
\definecolor{currentfill}{rgb}{0.153364,0.497000,0.557724}%
\pgfsetfillcolor{currentfill}%
\pgfsetfillopacity{0.700000}%
\pgfsetlinewidth{0.501875pt}%
\definecolor{currentstroke}{rgb}{1.000000,1.000000,1.000000}%
\pgfsetstrokecolor{currentstroke}%
\pgfsetstrokeopacity{0.500000}%
\pgfsetdash{}{0pt}%
\pgfpathmoveto{\pgfqpoint{2.708593in}{2.389169in}}%
\pgfpathlineto{\pgfqpoint{2.719971in}{2.391646in}}%
\pgfpathlineto{\pgfqpoint{2.731355in}{2.392628in}}%
\pgfpathlineto{\pgfqpoint{2.742747in}{2.391825in}}%
\pgfpathlineto{\pgfqpoint{2.754138in}{2.389939in}}%
\pgfpathlineto{\pgfqpoint{2.765520in}{2.388329in}}%
\pgfpathlineto{\pgfqpoint{2.759418in}{2.399210in}}%
\pgfpathlineto{\pgfqpoint{2.753315in}{2.410744in}}%
\pgfpathlineto{\pgfqpoint{2.747210in}{2.422814in}}%
\pgfpathlineto{\pgfqpoint{2.741105in}{2.435302in}}%
\pgfpathlineto{\pgfqpoint{2.735000in}{2.448090in}}%
\pgfpathlineto{\pgfqpoint{2.723655in}{2.445750in}}%
\pgfpathlineto{\pgfqpoint{2.712299in}{2.443818in}}%
\pgfpathlineto{\pgfqpoint{2.700942in}{2.441405in}}%
\pgfpathlineto{\pgfqpoint{2.689588in}{2.438053in}}%
\pgfpathlineto{\pgfqpoint{2.678235in}{2.433946in}}%
\pgfpathlineto{\pgfqpoint{2.684299in}{2.424962in}}%
\pgfpathlineto{\pgfqpoint{2.690367in}{2.415988in}}%
\pgfpathlineto{\pgfqpoint{2.696439in}{2.407027in}}%
\pgfpathlineto{\pgfqpoint{2.702514in}{2.398086in}}%
\pgfpathclose%
\pgfusepath{stroke,fill}%
\end{pgfscope}%
\begin{pgfscope}%
\pgfpathrectangle{\pgfqpoint{0.887500in}{0.275000in}}{\pgfqpoint{4.225000in}{4.225000in}}%
\pgfusepath{clip}%
\pgfsetbuttcap%
\pgfsetroundjoin%
\definecolor{currentfill}{rgb}{0.121148,0.592739,0.544641}%
\pgfsetfillcolor{currentfill}%
\pgfsetfillopacity{0.700000}%
\pgfsetlinewidth{0.501875pt}%
\definecolor{currentstroke}{rgb}{1.000000,1.000000,1.000000}%
\pgfsetstrokecolor{currentstroke}%
\pgfsetstrokeopacity{0.500000}%
\pgfsetdash{}{0pt}%
\pgfpathmoveto{\pgfqpoint{1.521504in}{2.590520in}}%
\pgfpathlineto{\pgfqpoint{1.533165in}{2.593829in}}%
\pgfpathlineto{\pgfqpoint{1.544820in}{2.597130in}}%
\pgfpathlineto{\pgfqpoint{1.556470in}{2.600425in}}%
\pgfpathlineto{\pgfqpoint{1.568115in}{2.603714in}}%
\pgfpathlineto{\pgfqpoint{1.579754in}{2.606999in}}%
\pgfpathlineto{\pgfqpoint{1.574078in}{2.614804in}}%
\pgfpathlineto{\pgfqpoint{1.568408in}{2.622585in}}%
\pgfpathlineto{\pgfqpoint{1.562742in}{2.630343in}}%
\pgfpathlineto{\pgfqpoint{1.557080in}{2.638078in}}%
\pgfpathlineto{\pgfqpoint{1.551423in}{2.645794in}}%
\pgfpathlineto{\pgfqpoint{1.539798in}{2.642505in}}%
\pgfpathlineto{\pgfqpoint{1.528167in}{2.639211in}}%
\pgfpathlineto{\pgfqpoint{1.516531in}{2.635912in}}%
\pgfpathlineto{\pgfqpoint{1.504889in}{2.632607in}}%
\pgfpathlineto{\pgfqpoint{1.493242in}{2.629294in}}%
\pgfpathlineto{\pgfqpoint{1.498885in}{2.621585in}}%
\pgfpathlineto{\pgfqpoint{1.504533in}{2.613855in}}%
\pgfpathlineto{\pgfqpoint{1.510185in}{2.606101in}}%
\pgfpathlineto{\pgfqpoint{1.515842in}{2.598323in}}%
\pgfpathclose%
\pgfusepath{stroke,fill}%
\end{pgfscope}%
\begin{pgfscope}%
\pgfpathrectangle{\pgfqpoint{0.887500in}{0.275000in}}{\pgfqpoint{4.225000in}{4.225000in}}%
\pgfusepath{clip}%
\pgfsetbuttcap%
\pgfsetroundjoin%
\definecolor{currentfill}{rgb}{0.162016,0.687316,0.499129}%
\pgfsetfillcolor{currentfill}%
\pgfsetfillopacity{0.700000}%
\pgfsetlinewidth{0.501875pt}%
\definecolor{currentstroke}{rgb}{1.000000,1.000000,1.000000}%
\pgfsetstrokecolor{currentstroke}%
\pgfsetstrokeopacity{0.500000}%
\pgfsetdash{}{0pt}%
\pgfpathmoveto{\pgfqpoint{4.172006in}{2.777328in}}%
\pgfpathlineto{\pgfqpoint{4.183019in}{2.780780in}}%
\pgfpathlineto{\pgfqpoint{4.194027in}{2.784230in}}%
\pgfpathlineto{\pgfqpoint{4.205030in}{2.787681in}}%
\pgfpathlineto{\pgfqpoint{4.216027in}{2.791133in}}%
\pgfpathlineto{\pgfqpoint{4.227020in}{2.794587in}}%
\pgfpathlineto{\pgfqpoint{4.220623in}{2.808597in}}%
\pgfpathlineto{\pgfqpoint{4.214228in}{2.822614in}}%
\pgfpathlineto{\pgfqpoint{4.207837in}{2.836645in}}%
\pgfpathlineto{\pgfqpoint{4.201448in}{2.850693in}}%
\pgfpathlineto{\pgfqpoint{4.195062in}{2.864749in}}%
\pgfpathlineto{\pgfqpoint{4.184071in}{2.861260in}}%
\pgfpathlineto{\pgfqpoint{4.173074in}{2.857770in}}%
\pgfpathlineto{\pgfqpoint{4.162072in}{2.854276in}}%
\pgfpathlineto{\pgfqpoint{4.151065in}{2.850778in}}%
\pgfpathlineto{\pgfqpoint{4.140052in}{2.847275in}}%
\pgfpathlineto{\pgfqpoint{4.146437in}{2.833276in}}%
\pgfpathlineto{\pgfqpoint{4.152825in}{2.819280in}}%
\pgfpathlineto{\pgfqpoint{4.159216in}{2.805292in}}%
\pgfpathlineto{\pgfqpoint{4.165610in}{2.791309in}}%
\pgfpathclose%
\pgfusepath{stroke,fill}%
\end{pgfscope}%
\begin{pgfscope}%
\pgfpathrectangle{\pgfqpoint{0.887500in}{0.275000in}}{\pgfqpoint{4.225000in}{4.225000in}}%
\pgfusepath{clip}%
\pgfsetbuttcap%
\pgfsetroundjoin%
\definecolor{currentfill}{rgb}{0.159194,0.482237,0.558073}%
\pgfsetfillcolor{currentfill}%
\pgfsetfillopacity{0.700000}%
\pgfsetlinewidth{0.501875pt}%
\definecolor{currentstroke}{rgb}{1.000000,1.000000,1.000000}%
\pgfsetstrokecolor{currentstroke}%
\pgfsetstrokeopacity{0.500000}%
\pgfsetdash{}{0pt}%
\pgfpathmoveto{\pgfqpoint{2.796017in}{2.342786in}}%
\pgfpathlineto{\pgfqpoint{2.807389in}{2.342432in}}%
\pgfpathlineto{\pgfqpoint{2.818735in}{2.344977in}}%
\pgfpathlineto{\pgfqpoint{2.830051in}{2.351754in}}%
\pgfpathlineto{\pgfqpoint{2.841332in}{2.364101in}}%
\pgfpathlineto{\pgfqpoint{2.852579in}{2.383137in}}%
\pgfpathlineto{\pgfqpoint{2.846460in}{2.391914in}}%
\pgfpathlineto{\pgfqpoint{2.840342in}{2.401105in}}%
\pgfpathlineto{\pgfqpoint{2.834226in}{2.410809in}}%
\pgfpathlineto{\pgfqpoint{2.828109in}{2.421123in}}%
\pgfpathlineto{\pgfqpoint{2.821992in}{2.432148in}}%
\pgfpathlineto{\pgfqpoint{2.810773in}{2.412053in}}%
\pgfpathlineto{\pgfqpoint{2.799513in}{2.398858in}}%
\pgfpathlineto{\pgfqpoint{2.788214in}{2.391420in}}%
\pgfpathlineto{\pgfqpoint{2.776881in}{2.388366in}}%
\pgfpathlineto{\pgfqpoint{2.765520in}{2.388329in}}%
\pgfpathlineto{\pgfqpoint{2.771619in}{2.378156in}}%
\pgfpathlineto{\pgfqpoint{2.777718in}{2.368613in}}%
\pgfpathlineto{\pgfqpoint{2.783817in}{2.359602in}}%
\pgfpathlineto{\pgfqpoint{2.789916in}{2.351025in}}%
\pgfpathclose%
\pgfusepath{stroke,fill}%
\end{pgfscope}%
\begin{pgfscope}%
\pgfpathrectangle{\pgfqpoint{0.887500in}{0.275000in}}{\pgfqpoint{4.225000in}{4.225000in}}%
\pgfusepath{clip}%
\pgfsetbuttcap%
\pgfsetroundjoin%
\definecolor{currentfill}{rgb}{0.127568,0.566949,0.550556}%
\pgfsetfillcolor{currentfill}%
\pgfsetfillopacity{0.700000}%
\pgfsetlinewidth{0.501875pt}%
\definecolor{currentstroke}{rgb}{1.000000,1.000000,1.000000}%
\pgfsetstrokecolor{currentstroke}%
\pgfsetstrokeopacity{0.500000}%
\pgfsetdash{}{0pt}%
\pgfpathmoveto{\pgfqpoint{1.839911in}{2.536494in}}%
\pgfpathlineto{\pgfqpoint{1.851499in}{2.539740in}}%
\pgfpathlineto{\pgfqpoint{1.863081in}{2.542983in}}%
\pgfpathlineto{\pgfqpoint{1.874657in}{2.546224in}}%
\pgfpathlineto{\pgfqpoint{1.886227in}{2.549467in}}%
\pgfpathlineto{\pgfqpoint{1.897791in}{2.552712in}}%
\pgfpathlineto{\pgfqpoint{1.891999in}{2.560784in}}%
\pgfpathlineto{\pgfqpoint{1.886210in}{2.568842in}}%
\pgfpathlineto{\pgfqpoint{1.880426in}{2.576889in}}%
\pgfpathlineto{\pgfqpoint{1.874646in}{2.584922in}}%
\pgfpathlineto{\pgfqpoint{1.868870in}{2.592941in}}%
\pgfpathlineto{\pgfqpoint{1.857318in}{2.589699in}}%
\pgfpathlineto{\pgfqpoint{1.845761in}{2.586461in}}%
\pgfpathlineto{\pgfqpoint{1.834197in}{2.583225in}}%
\pgfpathlineto{\pgfqpoint{1.822628in}{2.579987in}}%
\pgfpathlineto{\pgfqpoint{1.811054in}{2.576748in}}%
\pgfpathlineto{\pgfqpoint{1.816817in}{2.568726in}}%
\pgfpathlineto{\pgfqpoint{1.822584in}{2.560689in}}%
\pgfpathlineto{\pgfqpoint{1.828356in}{2.552638in}}%
\pgfpathlineto{\pgfqpoint{1.834131in}{2.544573in}}%
\pgfpathclose%
\pgfusepath{stroke,fill}%
\end{pgfscope}%
\begin{pgfscope}%
\pgfpathrectangle{\pgfqpoint{0.887500in}{0.275000in}}{\pgfqpoint{4.225000in}{4.225000in}}%
\pgfusepath{clip}%
\pgfsetbuttcap%
\pgfsetroundjoin%
\definecolor{currentfill}{rgb}{0.762373,0.876424,0.137064}%
\pgfsetfillcolor{currentfill}%
\pgfsetfillopacity{0.700000}%
\pgfsetlinewidth{0.501875pt}%
\definecolor{currentstroke}{rgb}{1.000000,1.000000,1.000000}%
\pgfsetstrokecolor{currentstroke}%
\pgfsetstrokeopacity{0.500000}%
\pgfsetdash{}{0pt}%
\pgfpathmoveto{\pgfqpoint{3.036291in}{3.328661in}}%
\pgfpathlineto{\pgfqpoint{3.047576in}{3.342041in}}%
\pgfpathlineto{\pgfqpoint{3.058861in}{3.352943in}}%
\pgfpathlineto{\pgfqpoint{3.070144in}{3.361684in}}%
\pgfpathlineto{\pgfqpoint{3.081424in}{3.368581in}}%
\pgfpathlineto{\pgfqpoint{3.092699in}{3.373932in}}%
\pgfpathlineto{\pgfqpoint{3.086503in}{3.383055in}}%
\pgfpathlineto{\pgfqpoint{3.080311in}{3.391915in}}%
\pgfpathlineto{\pgfqpoint{3.074123in}{3.400493in}}%
\pgfpathlineto{\pgfqpoint{3.067939in}{3.408769in}}%
\pgfpathlineto{\pgfqpoint{3.061758in}{3.416722in}}%
\pgfpathlineto{\pgfqpoint{3.050496in}{3.411003in}}%
\pgfpathlineto{\pgfqpoint{3.039230in}{3.403665in}}%
\pgfpathlineto{\pgfqpoint{3.027962in}{3.394323in}}%
\pgfpathlineto{\pgfqpoint{3.016695in}{3.382497in}}%
\pgfpathlineto{\pgfqpoint{3.005430in}{3.367702in}}%
\pgfpathlineto{\pgfqpoint{3.011594in}{3.360413in}}%
\pgfpathlineto{\pgfqpoint{3.017762in}{3.352855in}}%
\pgfpathlineto{\pgfqpoint{3.023934in}{3.345037in}}%
\pgfpathlineto{\pgfqpoint{3.030110in}{3.336970in}}%
\pgfpathclose%
\pgfusepath{stroke,fill}%
\end{pgfscope}%
\begin{pgfscope}%
\pgfpathrectangle{\pgfqpoint{0.887500in}{0.275000in}}{\pgfqpoint{4.225000in}{4.225000in}}%
\pgfusepath{clip}%
\pgfsetbuttcap%
\pgfsetroundjoin%
\definecolor{currentfill}{rgb}{0.196571,0.711827,0.479221}%
\pgfsetfillcolor{currentfill}%
\pgfsetfillopacity{0.700000}%
\pgfsetlinewidth{0.501875pt}%
\definecolor{currentstroke}{rgb}{1.000000,1.000000,1.000000}%
\pgfsetstrokecolor{currentstroke}%
\pgfsetstrokeopacity{0.500000}%
\pgfsetdash{}{0pt}%
\pgfpathmoveto{\pgfqpoint{4.084910in}{2.829729in}}%
\pgfpathlineto{\pgfqpoint{4.095948in}{2.833233in}}%
\pgfpathlineto{\pgfqpoint{4.106982in}{2.836742in}}%
\pgfpathlineto{\pgfqpoint{4.118011in}{2.840254in}}%
\pgfpathlineto{\pgfqpoint{4.129034in}{2.843766in}}%
\pgfpathlineto{\pgfqpoint{4.140052in}{2.847275in}}%
\pgfpathlineto{\pgfqpoint{4.133669in}{2.861265in}}%
\pgfpathlineto{\pgfqpoint{4.127288in}{2.875237in}}%
\pgfpathlineto{\pgfqpoint{4.120908in}{2.889181in}}%
\pgfpathlineto{\pgfqpoint{4.114530in}{2.903085in}}%
\pgfpathlineto{\pgfqpoint{4.108153in}{2.916939in}}%
\pgfpathlineto{\pgfqpoint{4.097136in}{2.913360in}}%
\pgfpathlineto{\pgfqpoint{4.086113in}{2.909770in}}%
\pgfpathlineto{\pgfqpoint{4.075084in}{2.906171in}}%
\pgfpathlineto{\pgfqpoint{4.064051in}{2.902566in}}%
\pgfpathlineto{\pgfqpoint{4.053012in}{2.898960in}}%
\pgfpathlineto{\pgfqpoint{4.059387in}{2.885157in}}%
\pgfpathlineto{\pgfqpoint{4.065765in}{2.871330in}}%
\pgfpathlineto{\pgfqpoint{4.072144in}{2.857482in}}%
\pgfpathlineto{\pgfqpoint{4.078526in}{2.843614in}}%
\pgfpathclose%
\pgfusepath{stroke,fill}%
\end{pgfscope}%
\begin{pgfscope}%
\pgfpathrectangle{\pgfqpoint{0.887500in}{0.275000in}}{\pgfqpoint{4.225000in}{4.225000in}}%
\pgfusepath{clip}%
\pgfsetbuttcap%
\pgfsetroundjoin%
\definecolor{currentfill}{rgb}{0.136408,0.541173,0.554483}%
\pgfsetfillcolor{currentfill}%
\pgfsetfillopacity{0.700000}%
\pgfsetlinewidth{0.501875pt}%
\definecolor{currentstroke}{rgb}{1.000000,1.000000,1.000000}%
\pgfsetstrokecolor{currentstroke}%
\pgfsetstrokeopacity{0.500000}%
\pgfsetdash{}{0pt}%
\pgfpathmoveto{\pgfqpoint{2.158497in}{2.479533in}}%
\pgfpathlineto{\pgfqpoint{2.170007in}{2.482825in}}%
\pgfpathlineto{\pgfqpoint{2.181510in}{2.486110in}}%
\pgfpathlineto{\pgfqpoint{2.193009in}{2.489390in}}%
\pgfpathlineto{\pgfqpoint{2.204501in}{2.492666in}}%
\pgfpathlineto{\pgfqpoint{2.215988in}{2.495940in}}%
\pgfpathlineto{\pgfqpoint{2.210084in}{2.504226in}}%
\pgfpathlineto{\pgfqpoint{2.204183in}{2.512496in}}%
\pgfpathlineto{\pgfqpoint{2.198287in}{2.520752in}}%
\pgfpathlineto{\pgfqpoint{2.192395in}{2.528992in}}%
\pgfpathlineto{\pgfqpoint{2.186508in}{2.537218in}}%
\pgfpathlineto{\pgfqpoint{2.175033in}{2.533943in}}%
\pgfpathlineto{\pgfqpoint{2.163553in}{2.530668in}}%
\pgfpathlineto{\pgfqpoint{2.152067in}{2.527389in}}%
\pgfpathlineto{\pgfqpoint{2.140575in}{2.524106in}}%
\pgfpathlineto{\pgfqpoint{2.129078in}{2.520818in}}%
\pgfpathlineto{\pgfqpoint{2.134954in}{2.512592in}}%
\pgfpathlineto{\pgfqpoint{2.140833in}{2.504351in}}%
\pgfpathlineto{\pgfqpoint{2.146717in}{2.496094in}}%
\pgfpathlineto{\pgfqpoint{2.152605in}{2.487821in}}%
\pgfpathclose%
\pgfusepath{stroke,fill}%
\end{pgfscope}%
\begin{pgfscope}%
\pgfpathrectangle{\pgfqpoint{0.887500in}{0.275000in}}{\pgfqpoint{4.225000in}{4.225000in}}%
\pgfusepath{clip}%
\pgfsetbuttcap%
\pgfsetroundjoin%
\definecolor{currentfill}{rgb}{0.274149,0.751988,0.436601}%
\pgfsetfillcolor{currentfill}%
\pgfsetfillopacity{0.700000}%
\pgfsetlinewidth{0.501875pt}%
\definecolor{currentstroke}{rgb}{1.000000,1.000000,1.000000}%
\pgfsetstrokecolor{currentstroke}%
\pgfsetstrokeopacity{0.500000}%
\pgfsetdash{}{0pt}%
\pgfpathmoveto{\pgfqpoint{2.928980in}{2.897870in}}%
\pgfpathlineto{\pgfqpoint{2.940211in}{2.927271in}}%
\pgfpathlineto{\pgfqpoint{2.951452in}{2.956788in}}%
\pgfpathlineto{\pgfqpoint{2.962702in}{2.986027in}}%
\pgfpathlineto{\pgfqpoint{2.973962in}{3.014591in}}%
\pgfpathlineto{\pgfqpoint{2.985234in}{3.042082in}}%
\pgfpathlineto{\pgfqpoint{2.979076in}{3.042521in}}%
\pgfpathlineto{\pgfqpoint{2.972925in}{3.043043in}}%
\pgfpathlineto{\pgfqpoint{2.966780in}{3.043625in}}%
\pgfpathlineto{\pgfqpoint{2.960641in}{3.044236in}}%
\pgfpathlineto{\pgfqpoint{2.954508in}{3.044847in}}%
\pgfpathlineto{\pgfqpoint{2.943286in}{3.014168in}}%
\pgfpathlineto{\pgfqpoint{2.932074in}{2.983638in}}%
\pgfpathlineto{\pgfqpoint{2.920870in}{2.953474in}}%
\pgfpathlineto{\pgfqpoint{2.909674in}{2.923890in}}%
\pgfpathlineto{\pgfqpoint{2.898484in}{2.895100in}}%
\pgfpathlineto{\pgfqpoint{2.904575in}{2.894439in}}%
\pgfpathlineto{\pgfqpoint{2.910669in}{2.894349in}}%
\pgfpathlineto{\pgfqpoint{2.916768in}{2.894869in}}%
\pgfpathlineto{\pgfqpoint{2.922872in}{2.896035in}}%
\pgfpathclose%
\pgfusepath{stroke,fill}%
\end{pgfscope}%
\begin{pgfscope}%
\pgfpathrectangle{\pgfqpoint{0.887500in}{0.275000in}}{\pgfqpoint{4.225000in}{4.225000in}}%
\pgfusepath{clip}%
\pgfsetbuttcap%
\pgfsetroundjoin%
\definecolor{currentfill}{rgb}{0.146180,0.515413,0.556823}%
\pgfsetfillcolor{currentfill}%
\pgfsetfillopacity{0.700000}%
\pgfsetlinewidth{0.501875pt}%
\definecolor{currentstroke}{rgb}{1.000000,1.000000,1.000000}%
\pgfsetstrokecolor{currentstroke}%
\pgfsetstrokeopacity{0.500000}%
\pgfsetdash{}{0pt}%
\pgfpathmoveto{\pgfqpoint{2.477176in}{2.420748in}}%
\pgfpathlineto{\pgfqpoint{2.488605in}{2.424213in}}%
\pgfpathlineto{\pgfqpoint{2.500030in}{2.427589in}}%
\pgfpathlineto{\pgfqpoint{2.511451in}{2.430844in}}%
\pgfpathlineto{\pgfqpoint{2.522868in}{2.433946in}}%
\pgfpathlineto{\pgfqpoint{2.534282in}{2.436919in}}%
\pgfpathlineto{\pgfqpoint{2.528269in}{2.445433in}}%
\pgfpathlineto{\pgfqpoint{2.522260in}{2.453937in}}%
\pgfpathlineto{\pgfqpoint{2.516255in}{2.462431in}}%
\pgfpathlineto{\pgfqpoint{2.510254in}{2.470912in}}%
\pgfpathlineto{\pgfqpoint{2.504257in}{2.479377in}}%
\pgfpathlineto{\pgfqpoint{2.492856in}{2.476320in}}%
\pgfpathlineto{\pgfqpoint{2.481451in}{2.473161in}}%
\pgfpathlineto{\pgfqpoint{2.470042in}{2.469878in}}%
\pgfpathlineto{\pgfqpoint{2.458629in}{2.466496in}}%
\pgfpathlineto{\pgfqpoint{2.447211in}{2.463042in}}%
\pgfpathlineto{\pgfqpoint{2.453196in}{2.454607in}}%
\pgfpathlineto{\pgfqpoint{2.459185in}{2.446162in}}%
\pgfpathlineto{\pgfqpoint{2.465178in}{2.437704in}}%
\pgfpathlineto{\pgfqpoint{2.471175in}{2.429233in}}%
\pgfpathclose%
\pgfusepath{stroke,fill}%
\end{pgfscope}%
\begin{pgfscope}%
\pgfpathrectangle{\pgfqpoint{0.887500in}{0.275000in}}{\pgfqpoint{4.225000in}{4.225000in}}%
\pgfusepath{clip}%
\pgfsetbuttcap%
\pgfsetroundjoin%
\definecolor{currentfill}{rgb}{0.122606,0.585371,0.546557}%
\pgfsetfillcolor{currentfill}%
\pgfsetfillopacity{0.700000}%
\pgfsetlinewidth{0.501875pt}%
\definecolor{currentstroke}{rgb}{1.000000,1.000000,1.000000}%
\pgfsetstrokecolor{currentstroke}%
\pgfsetstrokeopacity{0.500000}%
\pgfsetdash{}{0pt}%
\pgfpathmoveto{\pgfqpoint{1.608200in}{2.567575in}}%
\pgfpathlineto{\pgfqpoint{1.619847in}{2.570861in}}%
\pgfpathlineto{\pgfqpoint{1.631487in}{2.574144in}}%
\pgfpathlineto{\pgfqpoint{1.643123in}{2.577426in}}%
\pgfpathlineto{\pgfqpoint{1.654752in}{2.580708in}}%
\pgfpathlineto{\pgfqpoint{1.666375in}{2.583989in}}%
\pgfpathlineto{\pgfqpoint{1.660664in}{2.591917in}}%
\pgfpathlineto{\pgfqpoint{1.654957in}{2.599821in}}%
\pgfpathlineto{\pgfqpoint{1.649254in}{2.607700in}}%
\pgfpathlineto{\pgfqpoint{1.643556in}{2.615554in}}%
\pgfpathlineto{\pgfqpoint{1.637863in}{2.623383in}}%
\pgfpathlineto{\pgfqpoint{1.626253in}{2.620108in}}%
\pgfpathlineto{\pgfqpoint{1.614636in}{2.616833in}}%
\pgfpathlineto{\pgfqpoint{1.603014in}{2.613557in}}%
\pgfpathlineto{\pgfqpoint{1.591387in}{2.610280in}}%
\pgfpathlineto{\pgfqpoint{1.579754in}{2.606999in}}%
\pgfpathlineto{\pgfqpoint{1.585434in}{2.599168in}}%
\pgfpathlineto{\pgfqpoint{1.591118in}{2.591311in}}%
\pgfpathlineto{\pgfqpoint{1.596807in}{2.583426in}}%
\pgfpathlineto{\pgfqpoint{1.602501in}{2.575514in}}%
\pgfpathclose%
\pgfusepath{stroke,fill}%
\end{pgfscope}%
\begin{pgfscope}%
\pgfpathrectangle{\pgfqpoint{0.887500in}{0.275000in}}{\pgfqpoint{4.225000in}{4.225000in}}%
\pgfusepath{clip}%
\pgfsetbuttcap%
\pgfsetroundjoin%
\definecolor{currentfill}{rgb}{0.119512,0.607464,0.540218}%
\pgfsetfillcolor{currentfill}%
\pgfsetfillopacity{0.700000}%
\pgfsetlinewidth{0.501875pt}%
\definecolor{currentstroke}{rgb}{1.000000,1.000000,1.000000}%
\pgfsetstrokecolor{currentstroke}%
\pgfsetstrokeopacity{0.500000}%
\pgfsetdash{}{0pt}%
\pgfpathmoveto{\pgfqpoint{4.378120in}{2.599418in}}%
\pgfpathlineto{\pgfqpoint{4.389082in}{2.602746in}}%
\pgfpathlineto{\pgfqpoint{4.400039in}{2.606075in}}%
\pgfpathlineto{\pgfqpoint{4.410990in}{2.609405in}}%
\pgfpathlineto{\pgfqpoint{4.421936in}{2.612739in}}%
\pgfpathlineto{\pgfqpoint{4.415508in}{2.627048in}}%
\pgfpathlineto{\pgfqpoint{4.409083in}{2.641361in}}%
\pgfpathlineto{\pgfqpoint{4.402659in}{2.655674in}}%
\pgfpathlineto{\pgfqpoint{4.396239in}{2.669982in}}%
\pgfpathlineto{\pgfqpoint{4.389820in}{2.684279in}}%
\pgfpathlineto{\pgfqpoint{4.378874in}{2.680911in}}%
\pgfpathlineto{\pgfqpoint{4.367923in}{2.677546in}}%
\pgfpathlineto{\pgfqpoint{4.356966in}{2.674181in}}%
\pgfpathlineto{\pgfqpoint{4.346004in}{2.670817in}}%
\pgfpathlineto{\pgfqpoint{4.352424in}{2.656571in}}%
\pgfpathlineto{\pgfqpoint{4.358845in}{2.642306in}}%
\pgfpathlineto{\pgfqpoint{4.365268in}{2.628023in}}%
\pgfpathlineto{\pgfqpoint{4.371693in}{2.613726in}}%
\pgfpathclose%
\pgfusepath{stroke,fill}%
\end{pgfscope}%
\begin{pgfscope}%
\pgfpathrectangle{\pgfqpoint{0.887500in}{0.275000in}}{\pgfqpoint{4.225000in}{4.225000in}}%
\pgfusepath{clip}%
\pgfsetbuttcap%
\pgfsetroundjoin%
\definecolor{currentfill}{rgb}{0.239374,0.735588,0.455688}%
\pgfsetfillcolor{currentfill}%
\pgfsetfillopacity{0.700000}%
\pgfsetlinewidth{0.501875pt}%
\definecolor{currentstroke}{rgb}{1.000000,1.000000,1.000000}%
\pgfsetstrokecolor{currentstroke}%
\pgfsetstrokeopacity{0.500000}%
\pgfsetdash{}{0pt}%
\pgfpathmoveto{\pgfqpoint{3.997741in}{2.881000in}}%
\pgfpathlineto{\pgfqpoint{4.008805in}{2.884573in}}%
\pgfpathlineto{\pgfqpoint{4.019864in}{2.888158in}}%
\pgfpathlineto{\pgfqpoint{4.030918in}{2.891753in}}%
\pgfpathlineto{\pgfqpoint{4.041968in}{2.895355in}}%
\pgfpathlineto{\pgfqpoint{4.053012in}{2.898960in}}%
\pgfpathlineto{\pgfqpoint{4.046638in}{2.912739in}}%
\pgfpathlineto{\pgfqpoint{4.040266in}{2.926491in}}%
\pgfpathlineto{\pgfqpoint{4.033896in}{2.940214in}}%
\pgfpathlineto{\pgfqpoint{4.027528in}{2.953907in}}%
\pgfpathlineto{\pgfqpoint{4.021162in}{2.967565in}}%
\pgfpathlineto{\pgfqpoint{4.010122in}{2.964011in}}%
\pgfpathlineto{\pgfqpoint{3.999076in}{2.960437in}}%
\pgfpathlineto{\pgfqpoint{3.988025in}{2.956844in}}%
\pgfpathlineto{\pgfqpoint{3.976968in}{2.953240in}}%
\pgfpathlineto{\pgfqpoint{3.965906in}{2.949634in}}%
\pgfpathlineto{\pgfqpoint{3.972267in}{2.935877in}}%
\pgfpathlineto{\pgfqpoint{3.978631in}{2.922136in}}%
\pgfpathlineto{\pgfqpoint{3.984998in}{2.908414in}}%
\pgfpathlineto{\pgfqpoint{3.991368in}{2.894704in}}%
\pgfpathclose%
\pgfusepath{stroke,fill}%
\end{pgfscope}%
\begin{pgfscope}%
\pgfpathrectangle{\pgfqpoint{0.887500in}{0.275000in}}{\pgfqpoint{4.225000in}{4.225000in}}%
\pgfusepath{clip}%
\pgfsetbuttcap%
\pgfsetroundjoin%
\definecolor{currentfill}{rgb}{0.129933,0.559582,0.551864}%
\pgfsetfillcolor{currentfill}%
\pgfsetfillopacity{0.700000}%
\pgfsetlinewidth{0.501875pt}%
\definecolor{currentstroke}{rgb}{1.000000,1.000000,1.000000}%
\pgfsetstrokecolor{currentstroke}%
\pgfsetstrokeopacity{0.500000}%
\pgfsetdash{}{0pt}%
\pgfpathmoveto{\pgfqpoint{1.926816in}{2.512136in}}%
\pgfpathlineto{\pgfqpoint{1.938387in}{2.515395in}}%
\pgfpathlineto{\pgfqpoint{1.949952in}{2.518661in}}%
\pgfpathlineto{\pgfqpoint{1.961511in}{2.521936in}}%
\pgfpathlineto{\pgfqpoint{1.973064in}{2.525223in}}%
\pgfpathlineto{\pgfqpoint{1.984611in}{2.528524in}}%
\pgfpathlineto{\pgfqpoint{1.978785in}{2.536660in}}%
\pgfpathlineto{\pgfqpoint{1.972964in}{2.544781in}}%
\pgfpathlineto{\pgfqpoint{1.967146in}{2.552887in}}%
\pgfpathlineto{\pgfqpoint{1.961333in}{2.560979in}}%
\pgfpathlineto{\pgfqpoint{1.955524in}{2.569057in}}%
\pgfpathlineto{\pgfqpoint{1.943989in}{2.565766in}}%
\pgfpathlineto{\pgfqpoint{1.932449in}{2.562487in}}%
\pgfpathlineto{\pgfqpoint{1.920902in}{2.559220in}}%
\pgfpathlineto{\pgfqpoint{1.909350in}{2.555962in}}%
\pgfpathlineto{\pgfqpoint{1.897791in}{2.552712in}}%
\pgfpathlineto{\pgfqpoint{1.903588in}{2.544627in}}%
\pgfpathlineto{\pgfqpoint{1.909389in}{2.536527in}}%
\pgfpathlineto{\pgfqpoint{1.915194in}{2.528412in}}%
\pgfpathlineto{\pgfqpoint{1.921003in}{2.520282in}}%
\pgfpathclose%
\pgfusepath{stroke,fill}%
\end{pgfscope}%
\begin{pgfscope}%
\pgfpathrectangle{\pgfqpoint{0.887500in}{0.275000in}}{\pgfqpoint{4.225000in}{4.225000in}}%
\pgfusepath{clip}%
\pgfsetbuttcap%
\pgfsetroundjoin%
\definecolor{currentfill}{rgb}{0.122312,0.633153,0.530398}%
\pgfsetfillcolor{currentfill}%
\pgfsetfillopacity{0.700000}%
\pgfsetlinewidth{0.501875pt}%
\definecolor{currentstroke}{rgb}{1.000000,1.000000,1.000000}%
\pgfsetstrokecolor{currentstroke}%
\pgfsetstrokeopacity{0.500000}%
\pgfsetdash{}{0pt}%
\pgfpathmoveto{\pgfqpoint{4.291115in}{2.653950in}}%
\pgfpathlineto{\pgfqpoint{4.302104in}{2.657333in}}%
\pgfpathlineto{\pgfqpoint{4.313087in}{2.660710in}}%
\pgfpathlineto{\pgfqpoint{4.324065in}{2.664082in}}%
\pgfpathlineto{\pgfqpoint{4.335037in}{2.667451in}}%
\pgfpathlineto{\pgfqpoint{4.346004in}{2.670817in}}%
\pgfpathlineto{\pgfqpoint{4.339587in}{2.685039in}}%
\pgfpathlineto{\pgfqpoint{4.333171in}{2.699235in}}%
\pgfpathlineto{\pgfqpoint{4.326757in}{2.713401in}}%
\pgfpathlineto{\pgfqpoint{4.320344in}{2.727537in}}%
\pgfpathlineto{\pgfqpoint{4.313933in}{2.741646in}}%
\pgfpathlineto{\pgfqpoint{4.302965in}{2.738213in}}%
\pgfpathlineto{\pgfqpoint{4.291992in}{2.734784in}}%
\pgfpathlineto{\pgfqpoint{4.281014in}{2.731357in}}%
\pgfpathlineto{\pgfqpoint{4.270031in}{2.727930in}}%
\pgfpathlineto{\pgfqpoint{4.259043in}{2.724504in}}%
\pgfpathlineto{\pgfqpoint{4.265454in}{2.710447in}}%
\pgfpathlineto{\pgfqpoint{4.271867in}{2.696366in}}%
\pgfpathlineto{\pgfqpoint{4.278281in}{2.682255in}}%
\pgfpathlineto{\pgfqpoint{4.284697in}{2.668116in}}%
\pgfpathclose%
\pgfusepath{stroke,fill}%
\end{pgfscope}%
\begin{pgfscope}%
\pgfpathrectangle{\pgfqpoint{0.887500in}{0.275000in}}{\pgfqpoint{4.225000in}{4.225000in}}%
\pgfusepath{clip}%
\pgfsetbuttcap%
\pgfsetroundjoin%
\definecolor{currentfill}{rgb}{0.139147,0.533812,0.555298}%
\pgfsetfillcolor{currentfill}%
\pgfsetfillopacity{0.700000}%
\pgfsetlinewidth{0.501875pt}%
\definecolor{currentstroke}{rgb}{1.000000,1.000000,1.000000}%
\pgfsetstrokecolor{currentstroke}%
\pgfsetstrokeopacity{0.500000}%
\pgfsetdash{}{0pt}%
\pgfpathmoveto{\pgfqpoint{2.245571in}{2.454295in}}%
\pgfpathlineto{\pgfqpoint{2.257064in}{2.457568in}}%
\pgfpathlineto{\pgfqpoint{2.268552in}{2.460844in}}%
\pgfpathlineto{\pgfqpoint{2.280033in}{2.464126in}}%
\pgfpathlineto{\pgfqpoint{2.291509in}{2.467418in}}%
\pgfpathlineto{\pgfqpoint{2.302979in}{2.470724in}}%
\pgfpathlineto{\pgfqpoint{2.297042in}{2.479080in}}%
\pgfpathlineto{\pgfqpoint{2.291110in}{2.487421in}}%
\pgfpathlineto{\pgfqpoint{2.285181in}{2.495747in}}%
\pgfpathlineto{\pgfqpoint{2.279257in}{2.504059in}}%
\pgfpathlineto{\pgfqpoint{2.273336in}{2.512356in}}%
\pgfpathlineto{\pgfqpoint{2.261878in}{2.509059in}}%
\pgfpathlineto{\pgfqpoint{2.250414in}{2.505771in}}%
\pgfpathlineto{\pgfqpoint{2.238945in}{2.502490in}}%
\pgfpathlineto{\pgfqpoint{2.227469in}{2.499214in}}%
\pgfpathlineto{\pgfqpoint{2.215988in}{2.495940in}}%
\pgfpathlineto{\pgfqpoint{2.221896in}{2.487640in}}%
\pgfpathlineto{\pgfqpoint{2.227809in}{2.479325in}}%
\pgfpathlineto{\pgfqpoint{2.233726in}{2.470996in}}%
\pgfpathlineto{\pgfqpoint{2.239646in}{2.462652in}}%
\pgfpathclose%
\pgfusepath{stroke,fill}%
\end{pgfscope}%
\begin{pgfscope}%
\pgfpathrectangle{\pgfqpoint{0.887500in}{0.275000in}}{\pgfqpoint{4.225000in}{4.225000in}}%
\pgfusepath{clip}%
\pgfsetbuttcap%
\pgfsetroundjoin%
\definecolor{currentfill}{rgb}{0.288921,0.758394,0.428426}%
\pgfsetfillcolor{currentfill}%
\pgfsetfillopacity{0.700000}%
\pgfsetlinewidth{0.501875pt}%
\definecolor{currentstroke}{rgb}{1.000000,1.000000,1.000000}%
\pgfsetstrokecolor{currentstroke}%
\pgfsetstrokeopacity{0.500000}%
\pgfsetdash{}{0pt}%
\pgfpathmoveto{\pgfqpoint{3.910522in}{2.931809in}}%
\pgfpathlineto{\pgfqpoint{3.921608in}{2.935326in}}%
\pgfpathlineto{\pgfqpoint{3.932690in}{2.938870in}}%
\pgfpathlineto{\pgfqpoint{3.943767in}{2.942441in}}%
\pgfpathlineto{\pgfqpoint{3.954839in}{2.946031in}}%
\pgfpathlineto{\pgfqpoint{3.965906in}{2.949634in}}%
\pgfpathlineto{\pgfqpoint{3.959547in}{2.963387in}}%
\pgfpathlineto{\pgfqpoint{3.953190in}{2.977116in}}%
\pgfpathlineto{\pgfqpoint{3.946835in}{2.990796in}}%
\pgfpathlineto{\pgfqpoint{3.940481in}{3.004406in}}%
\pgfpathlineto{\pgfqpoint{3.934127in}{3.017924in}}%
\pgfpathlineto{\pgfqpoint{3.923064in}{3.014396in}}%
\pgfpathlineto{\pgfqpoint{3.911996in}{3.010853in}}%
\pgfpathlineto{\pgfqpoint{3.900922in}{3.007309in}}%
\pgfpathlineto{\pgfqpoint{3.889844in}{3.003775in}}%
\pgfpathlineto{\pgfqpoint{3.878761in}{3.000252in}}%
\pgfpathlineto{\pgfqpoint{3.885110in}{2.986688in}}%
\pgfpathlineto{\pgfqpoint{3.891461in}{2.973036in}}%
\pgfpathlineto{\pgfqpoint{3.897813in}{2.959322in}}%
\pgfpathlineto{\pgfqpoint{3.904166in}{2.945571in}}%
\pgfpathclose%
\pgfusepath{stroke,fill}%
\end{pgfscope}%
\begin{pgfscope}%
\pgfpathrectangle{\pgfqpoint{0.887500in}{0.275000in}}{\pgfqpoint{4.225000in}{4.225000in}}%
\pgfusepath{clip}%
\pgfsetbuttcap%
\pgfsetroundjoin%
\definecolor{currentfill}{rgb}{0.150476,0.504369,0.557430}%
\pgfsetfillcolor{currentfill}%
\pgfsetfillopacity{0.700000}%
\pgfsetlinewidth{0.501875pt}%
\definecolor{currentstroke}{rgb}{1.000000,1.000000,1.000000}%
\pgfsetstrokecolor{currentstroke}%
\pgfsetstrokeopacity{0.500000}%
\pgfsetdash{}{0pt}%
\pgfpathmoveto{\pgfqpoint{2.564406in}{2.394195in}}%
\pgfpathlineto{\pgfqpoint{2.575826in}{2.397082in}}%
\pgfpathlineto{\pgfqpoint{2.587239in}{2.400039in}}%
\pgfpathlineto{\pgfqpoint{2.598644in}{2.403173in}}%
\pgfpathlineto{\pgfqpoint{2.610040in}{2.406595in}}%
\pgfpathlineto{\pgfqpoint{2.621426in}{2.410414in}}%
\pgfpathlineto{\pgfqpoint{2.615382in}{2.418978in}}%
\pgfpathlineto{\pgfqpoint{2.609342in}{2.427540in}}%
\pgfpathlineto{\pgfqpoint{2.603305in}{2.436102in}}%
\pgfpathlineto{\pgfqpoint{2.597273in}{2.444668in}}%
\pgfpathlineto{\pgfqpoint{2.591244in}{2.453241in}}%
\pgfpathlineto{\pgfqpoint{2.579869in}{2.449450in}}%
\pgfpathlineto{\pgfqpoint{2.568484in}{2.446022in}}%
\pgfpathlineto{\pgfqpoint{2.557091in}{2.442857in}}%
\pgfpathlineto{\pgfqpoint{2.545690in}{2.439856in}}%
\pgfpathlineto{\pgfqpoint{2.534282in}{2.436919in}}%
\pgfpathlineto{\pgfqpoint{2.540298in}{2.428395in}}%
\pgfpathlineto{\pgfqpoint{2.546319in}{2.419862in}}%
\pgfpathlineto{\pgfqpoint{2.552344in}{2.411319in}}%
\pgfpathlineto{\pgfqpoint{2.558373in}{2.402763in}}%
\pgfpathclose%
\pgfusepath{stroke,fill}%
\end{pgfscope}%
\begin{pgfscope}%
\pgfpathrectangle{\pgfqpoint{0.887500in}{0.275000in}}{\pgfqpoint{4.225000in}{4.225000in}}%
\pgfusepath{clip}%
\pgfsetbuttcap%
\pgfsetroundjoin%
\definecolor{currentfill}{rgb}{0.751884,0.874951,0.143228}%
\pgfsetfillcolor{currentfill}%
\pgfsetfillopacity{0.700000}%
\pgfsetlinewidth{0.501875pt}%
\definecolor{currentstroke}{rgb}{1.000000,1.000000,1.000000}%
\pgfsetstrokecolor{currentstroke}%
\pgfsetstrokeopacity{0.500000}%
\pgfsetdash{}{0pt}%
\pgfpathmoveto{\pgfqpoint{3.123727in}{3.324746in}}%
\pgfpathlineto{\pgfqpoint{3.135005in}{3.328598in}}%
\pgfpathlineto{\pgfqpoint{3.146275in}{3.331550in}}%
\pgfpathlineto{\pgfqpoint{3.157539in}{3.333859in}}%
\pgfpathlineto{\pgfqpoint{3.168796in}{3.335780in}}%
\pgfpathlineto{\pgfqpoint{3.180045in}{3.337568in}}%
\pgfpathlineto{\pgfqpoint{3.173826in}{3.347780in}}%
\pgfpathlineto{\pgfqpoint{3.167609in}{3.357859in}}%
\pgfpathlineto{\pgfqpoint{3.161395in}{3.367780in}}%
\pgfpathlineto{\pgfqpoint{3.155185in}{3.377518in}}%
\pgfpathlineto{\pgfqpoint{3.148977in}{3.387050in}}%
\pgfpathlineto{\pgfqpoint{3.137735in}{3.385335in}}%
\pgfpathlineto{\pgfqpoint{3.126486in}{3.383438in}}%
\pgfpathlineto{\pgfqpoint{3.115230in}{3.381087in}}%
\pgfpathlineto{\pgfqpoint{3.103968in}{3.378009in}}%
\pgfpathlineto{\pgfqpoint{3.092699in}{3.373932in}}%
\pgfpathlineto{\pgfqpoint{3.098898in}{3.364557in}}%
\pgfpathlineto{\pgfqpoint{3.105100in}{3.354940in}}%
\pgfpathlineto{\pgfqpoint{3.111306in}{3.345092in}}%
\pgfpathlineto{\pgfqpoint{3.117515in}{3.335024in}}%
\pgfpathclose%
\pgfusepath{stroke,fill}%
\end{pgfscope}%
\begin{pgfscope}%
\pgfpathrectangle{\pgfqpoint{0.887500in}{0.275000in}}{\pgfqpoint{4.225000in}{4.225000in}}%
\pgfusepath{clip}%
\pgfsetbuttcap%
\pgfsetroundjoin%
\definecolor{currentfill}{rgb}{0.699415,0.867117,0.175971}%
\pgfsetfillcolor{currentfill}%
\pgfsetfillopacity{0.700000}%
\pgfsetlinewidth{0.501875pt}%
\definecolor{currentstroke}{rgb}{1.000000,1.000000,1.000000}%
\pgfsetstrokecolor{currentstroke}%
\pgfsetstrokeopacity{0.500000}%
\pgfsetdash{}{0pt}%
\pgfpathmoveto{\pgfqpoint{3.211192in}{3.285340in}}%
\pgfpathlineto{\pgfqpoint{3.222445in}{3.287706in}}%
\pgfpathlineto{\pgfqpoint{3.233692in}{3.290226in}}%
\pgfpathlineto{\pgfqpoint{3.244934in}{3.292890in}}%
\pgfpathlineto{\pgfqpoint{3.256171in}{3.295685in}}%
\pgfpathlineto{\pgfqpoint{3.267403in}{3.298597in}}%
\pgfpathlineto{\pgfqpoint{3.261158in}{3.308634in}}%
\pgfpathlineto{\pgfqpoint{3.254916in}{3.318739in}}%
\pgfpathlineto{\pgfqpoint{3.248679in}{3.328882in}}%
\pgfpathlineto{\pgfqpoint{3.242444in}{3.339028in}}%
\pgfpathlineto{\pgfqpoint{3.236213in}{3.349142in}}%
\pgfpathlineto{\pgfqpoint{3.224990in}{3.346425in}}%
\pgfpathlineto{\pgfqpoint{3.213762in}{3.343902in}}%
\pgfpathlineto{\pgfqpoint{3.202528in}{3.341581in}}%
\pgfpathlineto{\pgfqpoint{3.191290in}{3.339468in}}%
\pgfpathlineto{\pgfqpoint{3.180045in}{3.337568in}}%
\pgfpathlineto{\pgfqpoint{3.186268in}{3.327245in}}%
\pgfpathlineto{\pgfqpoint{3.192494in}{3.316837in}}%
\pgfpathlineto{\pgfqpoint{3.198724in}{3.306368in}}%
\pgfpathlineto{\pgfqpoint{3.204956in}{3.295861in}}%
\pgfpathclose%
\pgfusepath{stroke,fill}%
\end{pgfscope}%
\begin{pgfscope}%
\pgfpathrectangle{\pgfqpoint{0.887500in}{0.275000in}}{\pgfqpoint{4.225000in}{4.225000in}}%
\pgfusepath{clip}%
\pgfsetbuttcap%
\pgfsetroundjoin%
\definecolor{currentfill}{rgb}{0.335885,0.777018,0.402049}%
\pgfsetfillcolor{currentfill}%
\pgfsetfillopacity{0.700000}%
\pgfsetlinewidth{0.501875pt}%
\definecolor{currentstroke}{rgb}{1.000000,1.000000,1.000000}%
\pgfsetstrokecolor{currentstroke}%
\pgfsetstrokeopacity{0.500000}%
\pgfsetdash{}{0pt}%
\pgfpathmoveto{\pgfqpoint{3.823265in}{2.982641in}}%
\pgfpathlineto{\pgfqpoint{3.834375in}{2.986181in}}%
\pgfpathlineto{\pgfqpoint{3.845479in}{2.989707in}}%
\pgfpathlineto{\pgfqpoint{3.856578in}{2.993224in}}%
\pgfpathlineto{\pgfqpoint{3.867672in}{2.996737in}}%
\pgfpathlineto{\pgfqpoint{3.878761in}{3.000252in}}%
\pgfpathlineto{\pgfqpoint{3.872411in}{3.013704in}}%
\pgfpathlineto{\pgfqpoint{3.866062in}{3.027017in}}%
\pgfpathlineto{\pgfqpoint{3.859711in}{3.040166in}}%
\pgfpathlineto{\pgfqpoint{3.853360in}{3.053125in}}%
\pgfpathlineto{\pgfqpoint{3.847007in}{3.065873in}}%
\pgfpathlineto{\pgfqpoint{3.835923in}{3.062387in}}%
\pgfpathlineto{\pgfqpoint{3.824834in}{3.058900in}}%
\pgfpathlineto{\pgfqpoint{3.813739in}{3.055406in}}%
\pgfpathlineto{\pgfqpoint{3.802638in}{3.051898in}}%
\pgfpathlineto{\pgfqpoint{3.791532in}{3.048371in}}%
\pgfpathlineto{\pgfqpoint{3.797879in}{3.035553in}}%
\pgfpathlineto{\pgfqpoint{3.804226in}{3.022551in}}%
\pgfpathlineto{\pgfqpoint{3.810572in}{3.009384in}}%
\pgfpathlineto{\pgfqpoint{3.816919in}{2.996073in}}%
\pgfpathclose%
\pgfusepath{stroke,fill}%
\end{pgfscope}%
\begin{pgfscope}%
\pgfpathrectangle{\pgfqpoint{0.887500in}{0.275000in}}{\pgfqpoint{4.225000in}{4.225000in}}%
\pgfusepath{clip}%
\pgfsetbuttcap%
\pgfsetroundjoin%
\definecolor{currentfill}{rgb}{0.125394,0.574318,0.549086}%
\pgfsetfillcolor{currentfill}%
\pgfsetfillopacity{0.700000}%
\pgfsetlinewidth{0.501875pt}%
\definecolor{currentstroke}{rgb}{1.000000,1.000000,1.000000}%
\pgfsetstrokecolor{currentstroke}%
\pgfsetstrokeopacity{0.500000}%
\pgfsetdash{}{0pt}%
\pgfpathmoveto{\pgfqpoint{1.695000in}{2.544029in}}%
\pgfpathlineto{\pgfqpoint{1.706631in}{2.547326in}}%
\pgfpathlineto{\pgfqpoint{1.718255in}{2.550621in}}%
\pgfpathlineto{\pgfqpoint{1.729874in}{2.553912in}}%
\pgfpathlineto{\pgfqpoint{1.741488in}{2.557197in}}%
\pgfpathlineto{\pgfqpoint{1.753096in}{2.560475in}}%
\pgfpathlineto{\pgfqpoint{1.747350in}{2.568490in}}%
\pgfpathlineto{\pgfqpoint{1.741608in}{2.576486in}}%
\pgfpathlineto{\pgfqpoint{1.735870in}{2.584464in}}%
\pgfpathlineto{\pgfqpoint{1.730137in}{2.592423in}}%
\pgfpathlineto{\pgfqpoint{1.724408in}{2.600362in}}%
\pgfpathlineto{\pgfqpoint{1.712813in}{2.597096in}}%
\pgfpathlineto{\pgfqpoint{1.701212in}{2.593824in}}%
\pgfpathlineto{\pgfqpoint{1.689605in}{2.590549in}}%
\pgfpathlineto{\pgfqpoint{1.677993in}{2.587270in}}%
\pgfpathlineto{\pgfqpoint{1.666375in}{2.583989in}}%
\pgfpathlineto{\pgfqpoint{1.672091in}{2.576039in}}%
\pgfpathlineto{\pgfqpoint{1.677812in}{2.568066in}}%
\pgfpathlineto{\pgfqpoint{1.683537in}{2.560073in}}%
\pgfpathlineto{\pgfqpoint{1.689266in}{2.552060in}}%
\pgfpathclose%
\pgfusepath{stroke,fill}%
\end{pgfscope}%
\begin{pgfscope}%
\pgfpathrectangle{\pgfqpoint{0.887500in}{0.275000in}}{\pgfqpoint{4.225000in}{4.225000in}}%
\pgfusepath{clip}%
\pgfsetbuttcap%
\pgfsetroundjoin%
\definecolor{currentfill}{rgb}{0.134692,0.658636,0.517649}%
\pgfsetfillcolor{currentfill}%
\pgfsetfillopacity{0.700000}%
\pgfsetlinewidth{0.501875pt}%
\definecolor{currentstroke}{rgb}{1.000000,1.000000,1.000000}%
\pgfsetstrokecolor{currentstroke}%
\pgfsetstrokeopacity{0.500000}%
\pgfsetdash{}{0pt}%
\pgfpathmoveto{\pgfqpoint{4.204022in}{2.707334in}}%
\pgfpathlineto{\pgfqpoint{4.215037in}{2.710777in}}%
\pgfpathlineto{\pgfqpoint{4.226046in}{2.714214in}}%
\pgfpathlineto{\pgfqpoint{4.237051in}{2.717647in}}%
\pgfpathlineto{\pgfqpoint{4.248049in}{2.721077in}}%
\pgfpathlineto{\pgfqpoint{4.259043in}{2.724504in}}%
\pgfpathlineto{\pgfqpoint{4.252634in}{2.738542in}}%
\pgfpathlineto{\pgfqpoint{4.246227in}{2.752564in}}%
\pgfpathlineto{\pgfqpoint{4.239822in}{2.766576in}}%
\pgfpathlineto{\pgfqpoint{4.233420in}{2.780582in}}%
\pgfpathlineto{\pgfqpoint{4.227020in}{2.794587in}}%
\pgfpathlineto{\pgfqpoint{4.216027in}{2.791133in}}%
\pgfpathlineto{\pgfqpoint{4.205030in}{2.787681in}}%
\pgfpathlineto{\pgfqpoint{4.194027in}{2.784230in}}%
\pgfpathlineto{\pgfqpoint{4.183019in}{2.780780in}}%
\pgfpathlineto{\pgfqpoint{4.172006in}{2.777328in}}%
\pgfpathlineto{\pgfqpoint{4.178404in}{2.763346in}}%
\pgfpathlineto{\pgfqpoint{4.184805in}{2.749358in}}%
\pgfpathlineto{\pgfqpoint{4.191208in}{2.735363in}}%
\pgfpathlineto{\pgfqpoint{4.197614in}{2.721356in}}%
\pgfpathclose%
\pgfusepath{stroke,fill}%
\end{pgfscope}%
\begin{pgfscope}%
\pgfpathrectangle{\pgfqpoint{0.887500in}{0.275000in}}{\pgfqpoint{4.225000in}{4.225000in}}%
\pgfusepath{clip}%
\pgfsetbuttcap%
\pgfsetroundjoin%
\definecolor{currentfill}{rgb}{0.386433,0.794644,0.372886}%
\pgfsetfillcolor{currentfill}%
\pgfsetfillopacity{0.700000}%
\pgfsetlinewidth{0.501875pt}%
\definecolor{currentstroke}{rgb}{1.000000,1.000000,1.000000}%
\pgfsetstrokecolor{currentstroke}%
\pgfsetstrokeopacity{0.500000}%
\pgfsetdash{}{0pt}%
\pgfpathmoveto{\pgfqpoint{3.735918in}{3.030429in}}%
\pgfpathlineto{\pgfqpoint{3.747052in}{3.034044in}}%
\pgfpathlineto{\pgfqpoint{3.758180in}{3.037651in}}%
\pgfpathlineto{\pgfqpoint{3.769303in}{3.041244in}}%
\pgfpathlineto{\pgfqpoint{3.780420in}{3.044820in}}%
\pgfpathlineto{\pgfqpoint{3.791532in}{3.048371in}}%
\pgfpathlineto{\pgfqpoint{3.785184in}{3.061014in}}%
\pgfpathlineto{\pgfqpoint{3.778837in}{3.073506in}}%
\pgfpathlineto{\pgfqpoint{3.772490in}{3.085870in}}%
\pgfpathlineto{\pgfqpoint{3.766145in}{3.098131in}}%
\pgfpathlineto{\pgfqpoint{3.759801in}{3.110313in}}%
\pgfpathlineto{\pgfqpoint{3.748696in}{3.106918in}}%
\pgfpathlineto{\pgfqpoint{3.737586in}{3.103510in}}%
\pgfpathlineto{\pgfqpoint{3.726471in}{3.100099in}}%
\pgfpathlineto{\pgfqpoint{3.715351in}{3.096696in}}%
\pgfpathlineto{\pgfqpoint{3.704226in}{3.093312in}}%
\pgfpathlineto{\pgfqpoint{3.710561in}{3.080917in}}%
\pgfpathlineto{\pgfqpoint{3.716899in}{3.068444in}}%
\pgfpathlineto{\pgfqpoint{3.723237in}{3.055880in}}%
\pgfpathlineto{\pgfqpoint{3.729577in}{3.043212in}}%
\pgfpathclose%
\pgfusepath{stroke,fill}%
\end{pgfscope}%
\begin{pgfscope}%
\pgfpathrectangle{\pgfqpoint{0.887500in}{0.275000in}}{\pgfqpoint{4.225000in}{4.225000in}}%
\pgfusepath{clip}%
\pgfsetbuttcap%
\pgfsetroundjoin%
\definecolor{currentfill}{rgb}{0.647257,0.858400,0.209861}%
\pgfsetfillcolor{currentfill}%
\pgfsetfillopacity{0.700000}%
\pgfsetlinewidth{0.501875pt}%
\definecolor{currentstroke}{rgb}{1.000000,1.000000,1.000000}%
\pgfsetstrokecolor{currentstroke}%
\pgfsetstrokeopacity{0.500000}%
\pgfsetdash{}{0pt}%
\pgfpathmoveto{\pgfqpoint{3.298685in}{3.248507in}}%
\pgfpathlineto{\pgfqpoint{3.309921in}{3.251751in}}%
\pgfpathlineto{\pgfqpoint{3.321152in}{3.254949in}}%
\pgfpathlineto{\pgfqpoint{3.332376in}{3.258085in}}%
\pgfpathlineto{\pgfqpoint{3.343595in}{3.261159in}}%
\pgfpathlineto{\pgfqpoint{3.354808in}{3.264175in}}%
\pgfpathlineto{\pgfqpoint{3.348538in}{3.274312in}}%
\pgfpathlineto{\pgfqpoint{3.342270in}{3.284373in}}%
\pgfpathlineto{\pgfqpoint{3.336006in}{3.294395in}}%
\pgfpathlineto{\pgfqpoint{3.329745in}{3.304411in}}%
\pgfpathlineto{\pgfqpoint{3.323488in}{3.314456in}}%
\pgfpathlineto{\pgfqpoint{3.312281in}{3.311152in}}%
\pgfpathlineto{\pgfqpoint{3.301069in}{3.307902in}}%
\pgfpathlineto{\pgfqpoint{3.289852in}{3.304718in}}%
\pgfpathlineto{\pgfqpoint{3.278630in}{3.301612in}}%
\pgfpathlineto{\pgfqpoint{3.267403in}{3.298597in}}%
\pgfpathlineto{\pgfqpoint{3.273653in}{3.288616in}}%
\pgfpathlineto{\pgfqpoint{3.279906in}{3.278654in}}%
\pgfpathlineto{\pgfqpoint{3.286162in}{3.268674in}}%
\pgfpathlineto{\pgfqpoint{3.292422in}{3.258638in}}%
\pgfpathclose%
\pgfusepath{stroke,fill}%
\end{pgfscope}%
\begin{pgfscope}%
\pgfpathrectangle{\pgfqpoint{0.887500in}{0.275000in}}{\pgfqpoint{4.225000in}{4.225000in}}%
\pgfusepath{clip}%
\pgfsetbuttcap%
\pgfsetroundjoin%
\definecolor{currentfill}{rgb}{0.133743,0.548535,0.553541}%
\pgfsetfillcolor{currentfill}%
\pgfsetfillopacity{0.700000}%
\pgfsetlinewidth{0.501875pt}%
\definecolor{currentstroke}{rgb}{1.000000,1.000000,1.000000}%
\pgfsetstrokecolor{currentstroke}%
\pgfsetstrokeopacity{0.500000}%
\pgfsetdash{}{0pt}%
\pgfpathmoveto{\pgfqpoint{2.013803in}{2.487591in}}%
\pgfpathlineto{\pgfqpoint{2.025356in}{2.490916in}}%
\pgfpathlineto{\pgfqpoint{2.036903in}{2.494249in}}%
\pgfpathlineto{\pgfqpoint{2.048444in}{2.497586in}}%
\pgfpathlineto{\pgfqpoint{2.059980in}{2.500925in}}%
\pgfpathlineto{\pgfqpoint{2.071510in}{2.504261in}}%
\pgfpathlineto{\pgfqpoint{2.065651in}{2.512473in}}%
\pgfpathlineto{\pgfqpoint{2.059796in}{2.520668in}}%
\pgfpathlineto{\pgfqpoint{2.053945in}{2.528845in}}%
\pgfpathlineto{\pgfqpoint{2.048099in}{2.537005in}}%
\pgfpathlineto{\pgfqpoint{2.042257in}{2.545150in}}%
\pgfpathlineto{\pgfqpoint{2.030739in}{2.541819in}}%
\pgfpathlineto{\pgfqpoint{2.019216in}{2.538487in}}%
\pgfpathlineto{\pgfqpoint{2.007687in}{2.535159in}}%
\pgfpathlineto{\pgfqpoint{1.996152in}{2.531837in}}%
\pgfpathlineto{\pgfqpoint{1.984611in}{2.528524in}}%
\pgfpathlineto{\pgfqpoint{1.990441in}{2.520372in}}%
\pgfpathlineto{\pgfqpoint{1.996275in}{2.512203in}}%
\pgfpathlineto{\pgfqpoint{2.002113in}{2.504017in}}%
\pgfpathlineto{\pgfqpoint{2.007956in}{2.495813in}}%
\pgfpathclose%
\pgfusepath{stroke,fill}%
\end{pgfscope}%
\begin{pgfscope}%
\pgfpathrectangle{\pgfqpoint{0.887500in}{0.275000in}}{\pgfqpoint{4.225000in}{4.225000in}}%
\pgfusepath{clip}%
\pgfsetbuttcap%
\pgfsetroundjoin%
\definecolor{currentfill}{rgb}{0.440137,0.811138,0.340967}%
\pgfsetfillcolor{currentfill}%
\pgfsetfillopacity{0.700000}%
\pgfsetlinewidth{0.501875pt}%
\definecolor{currentstroke}{rgb}{1.000000,1.000000,1.000000}%
\pgfsetstrokecolor{currentstroke}%
\pgfsetstrokeopacity{0.500000}%
\pgfsetdash{}{0pt}%
\pgfpathmoveto{\pgfqpoint{3.648528in}{3.076975in}}%
\pgfpathlineto{\pgfqpoint{3.659677in}{3.080167in}}%
\pgfpathlineto{\pgfqpoint{3.670821in}{3.083387in}}%
\pgfpathlineto{\pgfqpoint{3.681961in}{3.086648in}}%
\pgfpathlineto{\pgfqpoint{3.693096in}{3.089959in}}%
\pgfpathlineto{\pgfqpoint{3.704226in}{3.093312in}}%
\pgfpathlineto{\pgfqpoint{3.697892in}{3.105643in}}%
\pgfpathlineto{\pgfqpoint{3.691560in}{3.117921in}}%
\pgfpathlineto{\pgfqpoint{3.685231in}{3.130159in}}%
\pgfpathlineto{\pgfqpoint{3.678904in}{3.142371in}}%
\pgfpathlineto{\pgfqpoint{3.672580in}{3.154570in}}%
\pgfpathlineto{\pgfqpoint{3.661457in}{3.151379in}}%
\pgfpathlineto{\pgfqpoint{3.650330in}{3.148237in}}%
\pgfpathlineto{\pgfqpoint{3.639198in}{3.145154in}}%
\pgfpathlineto{\pgfqpoint{3.628061in}{3.142118in}}%
\pgfpathlineto{\pgfqpoint{3.616919in}{3.139111in}}%
\pgfpathlineto{\pgfqpoint{3.623238in}{3.126929in}}%
\pgfpathlineto{\pgfqpoint{3.629559in}{3.114615in}}%
\pgfpathlineto{\pgfqpoint{3.635880in}{3.102180in}}%
\pgfpathlineto{\pgfqpoint{3.642203in}{3.089630in}}%
\pgfpathclose%
\pgfusepath{stroke,fill}%
\end{pgfscope}%
\begin{pgfscope}%
\pgfpathrectangle{\pgfqpoint{0.887500in}{0.275000in}}{\pgfqpoint{4.225000in}{4.225000in}}%
\pgfusepath{clip}%
\pgfsetbuttcap%
\pgfsetroundjoin%
\definecolor{currentfill}{rgb}{0.143343,0.522773,0.556295}%
\pgfsetfillcolor{currentfill}%
\pgfsetfillopacity{0.700000}%
\pgfsetlinewidth{0.501875pt}%
\definecolor{currentstroke}{rgb}{1.000000,1.000000,1.000000}%
\pgfsetstrokecolor{currentstroke}%
\pgfsetstrokeopacity{0.500000}%
\pgfsetdash{}{0pt}%
\pgfpathmoveto{\pgfqpoint{2.332725in}{2.428717in}}%
\pgfpathlineto{\pgfqpoint{2.344200in}{2.432039in}}%
\pgfpathlineto{\pgfqpoint{2.355670in}{2.435381in}}%
\pgfpathlineto{\pgfqpoint{2.367134in}{2.438747in}}%
\pgfpathlineto{\pgfqpoint{2.378591in}{2.442139in}}%
\pgfpathlineto{\pgfqpoint{2.390043in}{2.445560in}}%
\pgfpathlineto{\pgfqpoint{2.384074in}{2.453995in}}%
\pgfpathlineto{\pgfqpoint{2.378109in}{2.462414in}}%
\pgfpathlineto{\pgfqpoint{2.372148in}{2.470817in}}%
\pgfpathlineto{\pgfqpoint{2.366191in}{2.479204in}}%
\pgfpathlineto{\pgfqpoint{2.360238in}{2.487574in}}%
\pgfpathlineto{\pgfqpoint{2.348798in}{2.484151in}}%
\pgfpathlineto{\pgfqpoint{2.337353in}{2.480757in}}%
\pgfpathlineto{\pgfqpoint{2.325901in}{2.477390in}}%
\pgfpathlineto{\pgfqpoint{2.314443in}{2.474047in}}%
\pgfpathlineto{\pgfqpoint{2.302979in}{2.470724in}}%
\pgfpathlineto{\pgfqpoint{2.308920in}{2.462353in}}%
\pgfpathlineto{\pgfqpoint{2.314865in}{2.453967in}}%
\pgfpathlineto{\pgfqpoint{2.320814in}{2.445565in}}%
\pgfpathlineto{\pgfqpoint{2.326767in}{2.437149in}}%
\pgfpathclose%
\pgfusepath{stroke,fill}%
\end{pgfscope}%
\begin{pgfscope}%
\pgfpathrectangle{\pgfqpoint{0.887500in}{0.275000in}}{\pgfqpoint{4.225000in}{4.225000in}}%
\pgfusepath{clip}%
\pgfsetbuttcap%
\pgfsetroundjoin%
\definecolor{currentfill}{rgb}{0.606045,0.850733,0.236712}%
\pgfsetfillcolor{currentfill}%
\pgfsetfillopacity{0.700000}%
\pgfsetlinewidth{0.501875pt}%
\definecolor{currentstroke}{rgb}{1.000000,1.000000,1.000000}%
\pgfsetstrokecolor{currentstroke}%
\pgfsetstrokeopacity{0.500000}%
\pgfsetdash{}{0pt}%
\pgfpathmoveto{\pgfqpoint{3.386196in}{3.211154in}}%
\pgfpathlineto{\pgfqpoint{3.397409in}{3.213985in}}%
\pgfpathlineto{\pgfqpoint{3.408615in}{3.216718in}}%
\pgfpathlineto{\pgfqpoint{3.419815in}{3.219367in}}%
\pgfpathlineto{\pgfqpoint{3.431008in}{3.221945in}}%
\pgfpathlineto{\pgfqpoint{3.442196in}{3.224470in}}%
\pgfpathlineto{\pgfqpoint{3.435908in}{3.235477in}}%
\pgfpathlineto{\pgfqpoint{3.429623in}{3.246372in}}%
\pgfpathlineto{\pgfqpoint{3.423341in}{3.257172in}}%
\pgfpathlineto{\pgfqpoint{3.417061in}{3.267892in}}%
\pgfpathlineto{\pgfqpoint{3.410783in}{3.278548in}}%
\pgfpathlineto{\pgfqpoint{3.399600in}{3.275750in}}%
\pgfpathlineto{\pgfqpoint{3.388411in}{3.272921in}}%
\pgfpathlineto{\pgfqpoint{3.377216in}{3.270051in}}%
\pgfpathlineto{\pgfqpoint{3.366015in}{3.267138in}}%
\pgfpathlineto{\pgfqpoint{3.354808in}{3.264175in}}%
\pgfpathlineto{\pgfqpoint{3.361082in}{3.253929in}}%
\pgfpathlineto{\pgfqpoint{3.367357in}{3.243539in}}%
\pgfpathlineto{\pgfqpoint{3.373635in}{3.232969in}}%
\pgfpathlineto{\pgfqpoint{3.379915in}{3.222186in}}%
\pgfpathclose%
\pgfusepath{stroke,fill}%
\end{pgfscope}%
\begin{pgfscope}%
\pgfpathrectangle{\pgfqpoint{0.887500in}{0.275000in}}{\pgfqpoint{4.225000in}{4.225000in}}%
\pgfusepath{clip}%
\pgfsetbuttcap%
\pgfsetroundjoin%
\definecolor{currentfill}{rgb}{0.496615,0.826376,0.306377}%
\pgfsetfillcolor{currentfill}%
\pgfsetfillopacity{0.700000}%
\pgfsetlinewidth{0.501875pt}%
\definecolor{currentstroke}{rgb}{1.000000,1.000000,1.000000}%
\pgfsetstrokecolor{currentstroke}%
\pgfsetstrokeopacity{0.500000}%
\pgfsetdash{}{0pt}%
\pgfpathmoveto{\pgfqpoint{3.561122in}{3.123852in}}%
\pgfpathlineto{\pgfqpoint{3.572294in}{3.126994in}}%
\pgfpathlineto{\pgfqpoint{3.583459in}{3.130074in}}%
\pgfpathlineto{\pgfqpoint{3.594618in}{3.133108in}}%
\pgfpathlineto{\pgfqpoint{3.605771in}{3.136114in}}%
\pgfpathlineto{\pgfqpoint{3.616919in}{3.139111in}}%
\pgfpathlineto{\pgfqpoint{3.610600in}{3.151159in}}%
\pgfpathlineto{\pgfqpoint{3.604283in}{3.163082in}}%
\pgfpathlineto{\pgfqpoint{3.597968in}{3.174888in}}%
\pgfpathlineto{\pgfqpoint{3.591654in}{3.186585in}}%
\pgfpathlineto{\pgfqpoint{3.585342in}{3.198183in}}%
\pgfpathlineto{\pgfqpoint{3.574195in}{3.194839in}}%
\pgfpathlineto{\pgfqpoint{3.563044in}{3.191546in}}%
\pgfpathlineto{\pgfqpoint{3.551888in}{3.188317in}}%
\pgfpathlineto{\pgfqpoint{3.540727in}{3.185169in}}%
\pgfpathlineto{\pgfqpoint{3.529562in}{3.182112in}}%
\pgfpathlineto{\pgfqpoint{3.535872in}{3.170764in}}%
\pgfpathlineto{\pgfqpoint{3.542183in}{3.159287in}}%
\pgfpathlineto{\pgfqpoint{3.548495in}{3.147658in}}%
\pgfpathlineto{\pgfqpoint{3.554808in}{3.135854in}}%
\pgfpathclose%
\pgfusepath{stroke,fill}%
\end{pgfscope}%
\begin{pgfscope}%
\pgfpathrectangle{\pgfqpoint{0.887500in}{0.275000in}}{\pgfqpoint{4.225000in}{4.225000in}}%
\pgfusepath{clip}%
\pgfsetbuttcap%
\pgfsetroundjoin%
\definecolor{currentfill}{rgb}{0.555484,0.840254,0.269281}%
\pgfsetfillcolor{currentfill}%
\pgfsetfillopacity{0.700000}%
\pgfsetlinewidth{0.501875pt}%
\definecolor{currentstroke}{rgb}{1.000000,1.000000,1.000000}%
\pgfsetstrokecolor{currentstroke}%
\pgfsetstrokeopacity{0.500000}%
\pgfsetdash{}{0pt}%
\pgfpathmoveto{\pgfqpoint{3.473663in}{3.167550in}}%
\pgfpathlineto{\pgfqpoint{3.484854in}{3.170436in}}%
\pgfpathlineto{\pgfqpoint{3.496039in}{3.173315in}}%
\pgfpathlineto{\pgfqpoint{3.507218in}{3.176206in}}%
\pgfpathlineto{\pgfqpoint{3.518393in}{3.179131in}}%
\pgfpathlineto{\pgfqpoint{3.529562in}{3.182112in}}%
\pgfpathlineto{\pgfqpoint{3.523255in}{3.193354in}}%
\pgfpathlineto{\pgfqpoint{3.516950in}{3.204512in}}%
\pgfpathlineto{\pgfqpoint{3.510647in}{3.215610in}}%
\pgfpathlineto{\pgfqpoint{3.504348in}{3.226670in}}%
\pgfpathlineto{\pgfqpoint{3.498051in}{3.237716in}}%
\pgfpathlineto{\pgfqpoint{3.486889in}{3.234838in}}%
\pgfpathlineto{\pgfqpoint{3.475723in}{3.232118in}}%
\pgfpathlineto{\pgfqpoint{3.464553in}{3.229513in}}%
\pgfpathlineto{\pgfqpoint{3.453377in}{3.226978in}}%
\pgfpathlineto{\pgfqpoint{3.442196in}{3.224470in}}%
\pgfpathlineto{\pgfqpoint{3.448485in}{3.213341in}}%
\pgfpathlineto{\pgfqpoint{3.454777in}{3.202086in}}%
\pgfpathlineto{\pgfqpoint{3.461070in}{3.190704in}}%
\pgfpathlineto{\pgfqpoint{3.467366in}{3.179192in}}%
\pgfpathclose%
\pgfusepath{stroke,fill}%
\end{pgfscope}%
\begin{pgfscope}%
\pgfpathrectangle{\pgfqpoint{0.887500in}{0.275000in}}{\pgfqpoint{4.225000in}{4.225000in}}%
\pgfusepath{clip}%
\pgfsetbuttcap%
\pgfsetroundjoin%
\definecolor{currentfill}{rgb}{0.137770,0.537492,0.554906}%
\pgfsetfillcolor{currentfill}%
\pgfsetfillopacity{0.700000}%
\pgfsetlinewidth{0.501875pt}%
\definecolor{currentstroke}{rgb}{1.000000,1.000000,1.000000}%
\pgfsetstrokecolor{currentstroke}%
\pgfsetstrokeopacity{0.500000}%
\pgfsetdash{}{0pt}%
\pgfpathmoveto{\pgfqpoint{2.852579in}{2.383137in}}%
\pgfpathlineto{\pgfqpoint{2.863800in}{2.408278in}}%
\pgfpathlineto{\pgfqpoint{2.875005in}{2.438167in}}%
\pgfpathlineto{\pgfqpoint{2.886206in}{2.471438in}}%
\pgfpathlineto{\pgfqpoint{2.897410in}{2.506721in}}%
\pgfpathlineto{\pgfqpoint{2.908624in}{2.542637in}}%
\pgfpathlineto{\pgfqpoint{2.902459in}{2.554906in}}%
\pgfpathlineto{\pgfqpoint{2.896297in}{2.566784in}}%
\pgfpathlineto{\pgfqpoint{2.890142in}{2.578088in}}%
\pgfpathlineto{\pgfqpoint{2.883992in}{2.588639in}}%
\pgfpathlineto{\pgfqpoint{2.877851in}{2.598257in}}%
\pgfpathlineto{\pgfqpoint{2.866675in}{2.561066in}}%
\pgfpathlineto{\pgfqpoint{2.855510in}{2.524418in}}%
\pgfpathlineto{\pgfqpoint{2.844348in}{2.489759in}}%
\pgfpathlineto{\pgfqpoint{2.833179in}{2.458525in}}%
\pgfpathlineto{\pgfqpoint{2.821992in}{2.432148in}}%
\pgfpathlineto{\pgfqpoint{2.828109in}{2.421123in}}%
\pgfpathlineto{\pgfqpoint{2.834226in}{2.410809in}}%
\pgfpathlineto{\pgfqpoint{2.840342in}{2.401105in}}%
\pgfpathlineto{\pgfqpoint{2.846460in}{2.391914in}}%
\pgfpathclose%
\pgfusepath{stroke,fill}%
\end{pgfscope}%
\begin{pgfscope}%
\pgfpathrectangle{\pgfqpoint{0.887500in}{0.275000in}}{\pgfqpoint{4.225000in}{4.225000in}}%
\pgfusepath{clip}%
\pgfsetbuttcap%
\pgfsetroundjoin%
\definecolor{currentfill}{rgb}{0.121148,0.592739,0.544641}%
\pgfsetfillcolor{currentfill}%
\pgfsetfillopacity{0.700000}%
\pgfsetlinewidth{0.501875pt}%
\definecolor{currentstroke}{rgb}{1.000000,1.000000,1.000000}%
\pgfsetstrokecolor{currentstroke}%
\pgfsetstrokeopacity{0.500000}%
\pgfsetdash{}{0pt}%
\pgfpathmoveto{\pgfqpoint{1.463118in}{2.573828in}}%
\pgfpathlineto{\pgfqpoint{1.474806in}{2.577189in}}%
\pgfpathlineto{\pgfqpoint{1.486488in}{2.580538in}}%
\pgfpathlineto{\pgfqpoint{1.498165in}{2.583875in}}%
\pgfpathlineto{\pgfqpoint{1.509837in}{2.587202in}}%
\pgfpathlineto{\pgfqpoint{1.521504in}{2.590520in}}%
\pgfpathlineto{\pgfqpoint{1.515842in}{2.598323in}}%
\pgfpathlineto{\pgfqpoint{1.510185in}{2.606101in}}%
\pgfpathlineto{\pgfqpoint{1.504533in}{2.613855in}}%
\pgfpathlineto{\pgfqpoint{1.498885in}{2.621585in}}%
\pgfpathlineto{\pgfqpoint{1.493242in}{2.629294in}}%
\pgfpathlineto{\pgfqpoint{1.481589in}{2.625974in}}%
\pgfpathlineto{\pgfqpoint{1.469931in}{2.622645in}}%
\pgfpathlineto{\pgfqpoint{1.458268in}{2.619306in}}%
\pgfpathlineto{\pgfqpoint{1.446599in}{2.615957in}}%
\pgfpathlineto{\pgfqpoint{1.434925in}{2.612596in}}%
\pgfpathlineto{\pgfqpoint{1.440555in}{2.604890in}}%
\pgfpathlineto{\pgfqpoint{1.446188in}{2.597161in}}%
\pgfpathlineto{\pgfqpoint{1.451827in}{2.589410in}}%
\pgfpathlineto{\pgfqpoint{1.457470in}{2.581632in}}%
\pgfpathclose%
\pgfusepath{stroke,fill}%
\end{pgfscope}%
\begin{pgfscope}%
\pgfpathrectangle{\pgfqpoint{0.887500in}{0.275000in}}{\pgfqpoint{4.225000in}{4.225000in}}%
\pgfusepath{clip}%
\pgfsetbuttcap%
\pgfsetroundjoin%
\definecolor{currentfill}{rgb}{0.157851,0.683765,0.501686}%
\pgfsetfillcolor{currentfill}%
\pgfsetfillopacity{0.700000}%
\pgfsetlinewidth{0.501875pt}%
\definecolor{currentstroke}{rgb}{1.000000,1.000000,1.000000}%
\pgfsetstrokecolor{currentstroke}%
\pgfsetstrokeopacity{0.500000}%
\pgfsetdash{}{0pt}%
\pgfpathmoveto{\pgfqpoint{4.116861in}{2.760085in}}%
\pgfpathlineto{\pgfqpoint{4.127900in}{2.763526in}}%
\pgfpathlineto{\pgfqpoint{4.138934in}{2.766973in}}%
\pgfpathlineto{\pgfqpoint{4.149963in}{2.770423in}}%
\pgfpathlineto{\pgfqpoint{4.160987in}{2.773876in}}%
\pgfpathlineto{\pgfqpoint{4.172006in}{2.777328in}}%
\pgfpathlineto{\pgfqpoint{4.165610in}{2.791309in}}%
\pgfpathlineto{\pgfqpoint{4.159216in}{2.805292in}}%
\pgfpathlineto{\pgfqpoint{4.152825in}{2.819280in}}%
\pgfpathlineto{\pgfqpoint{4.146437in}{2.833276in}}%
\pgfpathlineto{\pgfqpoint{4.140052in}{2.847275in}}%
\pgfpathlineto{\pgfqpoint{4.129034in}{2.843766in}}%
\pgfpathlineto{\pgfqpoint{4.118011in}{2.840254in}}%
\pgfpathlineto{\pgfqpoint{4.106982in}{2.836742in}}%
\pgfpathlineto{\pgfqpoint{4.095948in}{2.833233in}}%
\pgfpathlineto{\pgfqpoint{4.084910in}{2.829729in}}%
\pgfpathlineto{\pgfqpoint{4.091295in}{2.815828in}}%
\pgfpathlineto{\pgfqpoint{4.097684in}{2.801912in}}%
\pgfpathlineto{\pgfqpoint{4.104074in}{2.787983in}}%
\pgfpathlineto{\pgfqpoint{4.110466in}{2.774041in}}%
\pgfpathclose%
\pgfusepath{stroke,fill}%
\end{pgfscope}%
\begin{pgfscope}%
\pgfpathrectangle{\pgfqpoint{0.887500in}{0.275000in}}{\pgfqpoint{4.225000in}{4.225000in}}%
\pgfusepath{clip}%
\pgfsetbuttcap%
\pgfsetroundjoin%
\definecolor{currentfill}{rgb}{0.153364,0.497000,0.557724}%
\pgfsetfillcolor{currentfill}%
\pgfsetfillopacity{0.700000}%
\pgfsetlinewidth{0.501875pt}%
\definecolor{currentstroke}{rgb}{1.000000,1.000000,1.000000}%
\pgfsetstrokecolor{currentstroke}%
\pgfsetstrokeopacity{0.500000}%
\pgfsetdash{}{0pt}%
\pgfpathmoveto{\pgfqpoint{2.651707in}{2.367434in}}%
\pgfpathlineto{\pgfqpoint{2.663094in}{2.371786in}}%
\pgfpathlineto{\pgfqpoint{2.674473in}{2.376505in}}%
\pgfpathlineto{\pgfqpoint{2.685846in}{2.381224in}}%
\pgfpathlineto{\pgfqpoint{2.697219in}{2.385570in}}%
\pgfpathlineto{\pgfqpoint{2.708593in}{2.389169in}}%
\pgfpathlineto{\pgfqpoint{2.702514in}{2.398086in}}%
\pgfpathlineto{\pgfqpoint{2.696439in}{2.407027in}}%
\pgfpathlineto{\pgfqpoint{2.690367in}{2.415988in}}%
\pgfpathlineto{\pgfqpoint{2.684299in}{2.424962in}}%
\pgfpathlineto{\pgfqpoint{2.678235in}{2.433946in}}%
\pgfpathlineto{\pgfqpoint{2.666882in}{2.429323in}}%
\pgfpathlineto{\pgfqpoint{2.655527in}{2.424420in}}%
\pgfpathlineto{\pgfqpoint{2.644168in}{2.419477in}}%
\pgfpathlineto{\pgfqpoint{2.632802in}{2.414731in}}%
\pgfpathlineto{\pgfqpoint{2.621426in}{2.410414in}}%
\pgfpathlineto{\pgfqpoint{2.627474in}{2.401844in}}%
\pgfpathlineto{\pgfqpoint{2.633526in}{2.393264in}}%
\pgfpathlineto{\pgfqpoint{2.639582in}{2.384671in}}%
\pgfpathlineto{\pgfqpoint{2.645643in}{2.376062in}}%
\pgfpathclose%
\pgfusepath{stroke,fill}%
\end{pgfscope}%
\begin{pgfscope}%
\pgfpathrectangle{\pgfqpoint{0.887500in}{0.275000in}}{\pgfqpoint{4.225000in}{4.225000in}}%
\pgfusepath{clip}%
\pgfsetbuttcap%
\pgfsetroundjoin%
\definecolor{currentfill}{rgb}{0.180653,0.701402,0.488189}%
\pgfsetfillcolor{currentfill}%
\pgfsetfillopacity{0.700000}%
\pgfsetlinewidth{0.501875pt}%
\definecolor{currentstroke}{rgb}{1.000000,1.000000,1.000000}%
\pgfsetstrokecolor{currentstroke}%
\pgfsetstrokeopacity{0.500000}%
\pgfsetdash{}{0pt}%
\pgfpathmoveto{\pgfqpoint{2.903335in}{2.772806in}}%
\pgfpathlineto{\pgfqpoint{2.914579in}{2.799448in}}%
\pgfpathlineto{\pgfqpoint{2.925830in}{2.826224in}}%
\pgfpathlineto{\pgfqpoint{2.937087in}{2.853418in}}%
\pgfpathlineto{\pgfqpoint{2.948352in}{2.881168in}}%
\pgfpathlineto{\pgfqpoint{2.959624in}{2.909142in}}%
\pgfpathlineto{\pgfqpoint{2.953479in}{2.907529in}}%
\pgfpathlineto{\pgfqpoint{2.947343in}{2.905318in}}%
\pgfpathlineto{\pgfqpoint{2.941215in}{2.902786in}}%
\pgfpathlineto{\pgfqpoint{2.935094in}{2.900211in}}%
\pgfpathlineto{\pgfqpoint{2.928980in}{2.897870in}}%
\pgfpathlineto{\pgfqpoint{2.917756in}{2.868978in}}%
\pgfpathlineto{\pgfqpoint{2.906538in}{2.840987in}}%
\pgfpathlineto{\pgfqpoint{2.895324in}{2.814220in}}%
\pgfpathlineto{\pgfqpoint{2.884112in}{2.788664in}}%
\pgfpathlineto{\pgfqpoint{2.872902in}{2.764206in}}%
\pgfpathlineto{\pgfqpoint{2.878975in}{2.765699in}}%
\pgfpathlineto{\pgfqpoint{2.885054in}{2.767626in}}%
\pgfpathlineto{\pgfqpoint{2.891139in}{2.769667in}}%
\pgfpathlineto{\pgfqpoint{2.897232in}{2.771500in}}%
\pgfpathclose%
\pgfusepath{stroke,fill}%
\end{pgfscope}%
\begin{pgfscope}%
\pgfpathrectangle{\pgfqpoint{0.887500in}{0.275000in}}{\pgfqpoint{4.225000in}{4.225000in}}%
\pgfusepath{clip}%
\pgfsetbuttcap%
\pgfsetroundjoin%
\definecolor{currentfill}{rgb}{0.124395,0.578002,0.548287}%
\pgfsetfillcolor{currentfill}%
\pgfsetfillopacity{0.700000}%
\pgfsetlinewidth{0.501875pt}%
\definecolor{currentstroke}{rgb}{1.000000,1.000000,1.000000}%
\pgfsetstrokecolor{currentstroke}%
\pgfsetstrokeopacity{0.500000}%
\pgfsetdash{}{0pt}%
\pgfpathmoveto{\pgfqpoint{4.410292in}{2.527829in}}%
\pgfpathlineto{\pgfqpoint{4.421257in}{2.531215in}}%
\pgfpathlineto{\pgfqpoint{4.432217in}{2.534613in}}%
\pgfpathlineto{\pgfqpoint{4.443173in}{2.538023in}}%
\pgfpathlineto{\pgfqpoint{4.454123in}{2.541446in}}%
\pgfpathlineto{\pgfqpoint{4.447679in}{2.555660in}}%
\pgfpathlineto{\pgfqpoint{4.441239in}{2.569897in}}%
\pgfpathlineto{\pgfqpoint{4.434801in}{2.584159in}}%
\pgfpathlineto{\pgfqpoint{4.428367in}{2.598441in}}%
\pgfpathlineto{\pgfqpoint{4.421936in}{2.612739in}}%
\pgfpathlineto{\pgfqpoint{4.410990in}{2.609405in}}%
\pgfpathlineto{\pgfqpoint{4.400039in}{2.606075in}}%
\pgfpathlineto{\pgfqpoint{4.389082in}{2.602746in}}%
\pgfpathlineto{\pgfqpoint{4.378120in}{2.599418in}}%
\pgfpathlineto{\pgfqpoint{4.384549in}{2.585103in}}%
\pgfpathlineto{\pgfqpoint{4.390981in}{2.570783in}}%
\pgfpathlineto{\pgfqpoint{4.397415in}{2.556461in}}%
\pgfpathlineto{\pgfqpoint{4.403852in}{2.542141in}}%
\pgfpathclose%
\pgfusepath{stroke,fill}%
\end{pgfscope}%
\begin{pgfscope}%
\pgfpathrectangle{\pgfqpoint{0.887500in}{0.275000in}}{\pgfqpoint{4.225000in}{4.225000in}}%
\pgfusepath{clip}%
\pgfsetbuttcap%
\pgfsetroundjoin%
\definecolor{currentfill}{rgb}{0.127568,0.566949,0.550556}%
\pgfsetfillcolor{currentfill}%
\pgfsetfillopacity{0.700000}%
\pgfsetlinewidth{0.501875pt}%
\definecolor{currentstroke}{rgb}{1.000000,1.000000,1.000000}%
\pgfsetstrokecolor{currentstroke}%
\pgfsetstrokeopacity{0.500000}%
\pgfsetdash{}{0pt}%
\pgfpathmoveto{\pgfqpoint{1.781891in}{2.520167in}}%
\pgfpathlineto{\pgfqpoint{1.793506in}{2.523451in}}%
\pgfpathlineto{\pgfqpoint{1.805115in}{2.526724in}}%
\pgfpathlineto{\pgfqpoint{1.816720in}{2.529988in}}%
\pgfpathlineto{\pgfqpoint{1.828318in}{2.533244in}}%
\pgfpathlineto{\pgfqpoint{1.839911in}{2.536494in}}%
\pgfpathlineto{\pgfqpoint{1.834131in}{2.544573in}}%
\pgfpathlineto{\pgfqpoint{1.828356in}{2.552638in}}%
\pgfpathlineto{\pgfqpoint{1.822584in}{2.560689in}}%
\pgfpathlineto{\pgfqpoint{1.816817in}{2.568726in}}%
\pgfpathlineto{\pgfqpoint{1.811054in}{2.576748in}}%
\pgfpathlineto{\pgfqpoint{1.799473in}{2.573506in}}%
\pgfpathlineto{\pgfqpoint{1.787887in}{2.570259in}}%
\pgfpathlineto{\pgfqpoint{1.776296in}{2.567006in}}%
\pgfpathlineto{\pgfqpoint{1.764699in}{2.563745in}}%
\pgfpathlineto{\pgfqpoint{1.753096in}{2.560475in}}%
\pgfpathlineto{\pgfqpoint{1.758847in}{2.552444in}}%
\pgfpathlineto{\pgfqpoint{1.764601in}{2.544397in}}%
\pgfpathlineto{\pgfqpoint{1.770360in}{2.536335in}}%
\pgfpathlineto{\pgfqpoint{1.776123in}{2.528259in}}%
\pgfpathclose%
\pgfusepath{stroke,fill}%
\end{pgfscope}%
\begin{pgfscope}%
\pgfpathrectangle{\pgfqpoint{0.887500in}{0.275000in}}{\pgfqpoint{4.225000in}{4.225000in}}%
\pgfusepath{clip}%
\pgfsetbuttcap%
\pgfsetroundjoin%
\definecolor{currentfill}{rgb}{0.159194,0.482237,0.558073}%
\pgfsetfillcolor{currentfill}%
\pgfsetfillopacity{0.700000}%
\pgfsetlinewidth{0.501875pt}%
\definecolor{currentstroke}{rgb}{1.000000,1.000000,1.000000}%
\pgfsetstrokecolor{currentstroke}%
\pgfsetstrokeopacity{0.500000}%
\pgfsetdash{}{0pt}%
\pgfpathmoveto{\pgfqpoint{2.739041in}{2.344892in}}%
\pgfpathlineto{\pgfqpoint{2.750431in}{2.347139in}}%
\pgfpathlineto{\pgfqpoint{2.761826in}{2.347884in}}%
\pgfpathlineto{\pgfqpoint{2.773227in}{2.346844in}}%
\pgfpathlineto{\pgfqpoint{2.784627in}{2.344702in}}%
\pgfpathlineto{\pgfqpoint{2.796017in}{2.342786in}}%
\pgfpathlineto{\pgfqpoint{2.789916in}{2.351025in}}%
\pgfpathlineto{\pgfqpoint{2.783817in}{2.359602in}}%
\pgfpathlineto{\pgfqpoint{2.777718in}{2.368613in}}%
\pgfpathlineto{\pgfqpoint{2.771619in}{2.378156in}}%
\pgfpathlineto{\pgfqpoint{2.765520in}{2.388329in}}%
\pgfpathlineto{\pgfqpoint{2.754138in}{2.389939in}}%
\pgfpathlineto{\pgfqpoint{2.742747in}{2.391825in}}%
\pgfpathlineto{\pgfqpoint{2.731355in}{2.392628in}}%
\pgfpathlineto{\pgfqpoint{2.719971in}{2.391646in}}%
\pgfpathlineto{\pgfqpoint{2.708593in}{2.389169in}}%
\pgfpathlineto{\pgfqpoint{2.714675in}{2.380278in}}%
\pgfpathlineto{\pgfqpoint{2.720761in}{2.371410in}}%
\pgfpathlineto{\pgfqpoint{2.726851in}{2.362559in}}%
\pgfpathlineto{\pgfqpoint{2.732944in}{2.353721in}}%
\pgfpathclose%
\pgfusepath{stroke,fill}%
\end{pgfscope}%
\begin{pgfscope}%
\pgfpathrectangle{\pgfqpoint{0.887500in}{0.275000in}}{\pgfqpoint{4.225000in}{4.225000in}}%
\pgfusepath{clip}%
\pgfsetbuttcap%
\pgfsetroundjoin%
\definecolor{currentfill}{rgb}{0.196571,0.711827,0.479221}%
\pgfsetfillcolor{currentfill}%
\pgfsetfillopacity{0.700000}%
\pgfsetlinewidth{0.501875pt}%
\definecolor{currentstroke}{rgb}{1.000000,1.000000,1.000000}%
\pgfsetstrokecolor{currentstroke}%
\pgfsetstrokeopacity{0.500000}%
\pgfsetdash{}{0pt}%
\pgfpathmoveto{\pgfqpoint{4.029641in}{2.812357in}}%
\pgfpathlineto{\pgfqpoint{4.040705in}{2.815810in}}%
\pgfpathlineto{\pgfqpoint{4.051763in}{2.819274in}}%
\pgfpathlineto{\pgfqpoint{4.062817in}{2.822747in}}%
\pgfpathlineto{\pgfqpoint{4.073866in}{2.826233in}}%
\pgfpathlineto{\pgfqpoint{4.084910in}{2.829729in}}%
\pgfpathlineto{\pgfqpoint{4.078526in}{2.843614in}}%
\pgfpathlineto{\pgfqpoint{4.072144in}{2.857482in}}%
\pgfpathlineto{\pgfqpoint{4.065765in}{2.871330in}}%
\pgfpathlineto{\pgfqpoint{4.059387in}{2.885157in}}%
\pgfpathlineto{\pgfqpoint{4.053012in}{2.898960in}}%
\pgfpathlineto{\pgfqpoint{4.041968in}{2.895355in}}%
\pgfpathlineto{\pgfqpoint{4.030918in}{2.891753in}}%
\pgfpathlineto{\pgfqpoint{4.019864in}{2.888158in}}%
\pgfpathlineto{\pgfqpoint{4.008805in}{2.884573in}}%
\pgfpathlineto{\pgfqpoint{3.997741in}{2.881000in}}%
\pgfpathlineto{\pgfqpoint{4.004117in}{2.867297in}}%
\pgfpathlineto{\pgfqpoint{4.010494in}{2.853587in}}%
\pgfpathlineto{\pgfqpoint{4.016875in}{2.839864in}}%
\pgfpathlineto{\pgfqpoint{4.023257in}{2.826123in}}%
\pgfpathclose%
\pgfusepath{stroke,fill}%
\end{pgfscope}%
\begin{pgfscope}%
\pgfpathrectangle{\pgfqpoint{0.887500in}{0.275000in}}{\pgfqpoint{4.225000in}{4.225000in}}%
\pgfusepath{clip}%
\pgfsetbuttcap%
\pgfsetroundjoin%
\definecolor{currentfill}{rgb}{0.136408,0.541173,0.554483}%
\pgfsetfillcolor{currentfill}%
\pgfsetfillopacity{0.700000}%
\pgfsetlinewidth{0.501875pt}%
\definecolor{currentstroke}{rgb}{1.000000,1.000000,1.000000}%
\pgfsetstrokecolor{currentstroke}%
\pgfsetstrokeopacity{0.500000}%
\pgfsetdash{}{0pt}%
\pgfpathmoveto{\pgfqpoint{2.100869in}{2.462937in}}%
\pgfpathlineto{\pgfqpoint{2.112405in}{2.466277in}}%
\pgfpathlineto{\pgfqpoint{2.123937in}{2.469606in}}%
\pgfpathlineto{\pgfqpoint{2.135462in}{2.472925in}}%
\pgfpathlineto{\pgfqpoint{2.146983in}{2.476233in}}%
\pgfpathlineto{\pgfqpoint{2.158497in}{2.479533in}}%
\pgfpathlineto{\pgfqpoint{2.152605in}{2.487821in}}%
\pgfpathlineto{\pgfqpoint{2.146717in}{2.496094in}}%
\pgfpathlineto{\pgfqpoint{2.140833in}{2.504351in}}%
\pgfpathlineto{\pgfqpoint{2.134954in}{2.512592in}}%
\pgfpathlineto{\pgfqpoint{2.129078in}{2.520818in}}%
\pgfpathlineto{\pgfqpoint{2.117575in}{2.517524in}}%
\pgfpathlineto{\pgfqpoint{2.106067in}{2.514222in}}%
\pgfpathlineto{\pgfqpoint{2.094554in}{2.510911in}}%
\pgfpathlineto{\pgfqpoint{2.083035in}{2.507591in}}%
\pgfpathlineto{\pgfqpoint{2.071510in}{2.504261in}}%
\pgfpathlineto{\pgfqpoint{2.077373in}{2.496032in}}%
\pgfpathlineto{\pgfqpoint{2.083241in}{2.487785in}}%
\pgfpathlineto{\pgfqpoint{2.089113in}{2.479520in}}%
\pgfpathlineto{\pgfqpoint{2.094989in}{2.471237in}}%
\pgfpathclose%
\pgfusepath{stroke,fill}%
\end{pgfscope}%
\begin{pgfscope}%
\pgfpathrectangle{\pgfqpoint{0.887500in}{0.275000in}}{\pgfqpoint{4.225000in}{4.225000in}}%
\pgfusepath{clip}%
\pgfsetbuttcap%
\pgfsetroundjoin%
\definecolor{currentfill}{rgb}{0.626579,0.854645,0.223353}%
\pgfsetfillcolor{currentfill}%
\pgfsetfillopacity{0.700000}%
\pgfsetlinewidth{0.501875pt}%
\definecolor{currentstroke}{rgb}{1.000000,1.000000,1.000000}%
\pgfsetstrokecolor{currentstroke}%
\pgfsetstrokeopacity{0.500000}%
\pgfsetdash{}{0pt}%
\pgfpathmoveto{\pgfqpoint{3.010777in}{3.188668in}}%
\pgfpathlineto{\pgfqpoint{3.022062in}{3.212738in}}%
\pgfpathlineto{\pgfqpoint{3.033354in}{3.234310in}}%
\pgfpathlineto{\pgfqpoint{3.044652in}{3.253203in}}%
\pgfpathlineto{\pgfqpoint{3.055953in}{3.269568in}}%
\pgfpathlineto{\pgfqpoint{3.067255in}{3.283583in}}%
\pgfpathlineto{\pgfqpoint{3.061055in}{3.293061in}}%
\pgfpathlineto{\pgfqpoint{3.054858in}{3.302310in}}%
\pgfpathlineto{\pgfqpoint{3.048665in}{3.311328in}}%
\pgfpathlineto{\pgfqpoint{3.042476in}{3.320113in}}%
\pgfpathlineto{\pgfqpoint{3.036291in}{3.328661in}}%
\pgfpathlineto{\pgfqpoint{3.025009in}{3.312486in}}%
\pgfpathlineto{\pgfqpoint{3.013732in}{3.293203in}}%
\pgfpathlineto{\pgfqpoint{3.002462in}{3.270499in}}%
\pgfpathlineto{\pgfqpoint{2.991205in}{3.244118in}}%
\pgfpathlineto{\pgfqpoint{2.979960in}{3.214469in}}%
\pgfpathlineto{\pgfqpoint{2.986113in}{3.209788in}}%
\pgfpathlineto{\pgfqpoint{2.992271in}{3.204857in}}%
\pgfpathlineto{\pgfqpoint{2.998435in}{3.199686in}}%
\pgfpathlineto{\pgfqpoint{3.004603in}{3.194286in}}%
\pgfpathclose%
\pgfusepath{stroke,fill}%
\end{pgfscope}%
\begin{pgfscope}%
\pgfpathrectangle{\pgfqpoint{0.887500in}{0.275000in}}{\pgfqpoint{4.225000in}{4.225000in}}%
\pgfusepath{clip}%
\pgfsetbuttcap%
\pgfsetroundjoin%
\definecolor{currentfill}{rgb}{0.146180,0.515413,0.556823}%
\pgfsetfillcolor{currentfill}%
\pgfsetfillopacity{0.700000}%
\pgfsetlinewidth{0.501875pt}%
\definecolor{currentstroke}{rgb}{1.000000,1.000000,1.000000}%
\pgfsetstrokecolor{currentstroke}%
\pgfsetstrokeopacity{0.500000}%
\pgfsetdash{}{0pt}%
\pgfpathmoveto{\pgfqpoint{2.419950in}{2.403165in}}%
\pgfpathlineto{\pgfqpoint{2.431407in}{2.406628in}}%
\pgfpathlineto{\pgfqpoint{2.442858in}{2.410135in}}%
\pgfpathlineto{\pgfqpoint{2.454303in}{2.413677in}}%
\pgfpathlineto{\pgfqpoint{2.465742in}{2.417225in}}%
\pgfpathlineto{\pgfqpoint{2.477176in}{2.420748in}}%
\pgfpathlineto{\pgfqpoint{2.471175in}{2.429233in}}%
\pgfpathlineto{\pgfqpoint{2.465178in}{2.437704in}}%
\pgfpathlineto{\pgfqpoint{2.459185in}{2.446162in}}%
\pgfpathlineto{\pgfqpoint{2.453196in}{2.454607in}}%
\pgfpathlineto{\pgfqpoint{2.447211in}{2.463042in}}%
\pgfpathlineto{\pgfqpoint{2.435789in}{2.459541in}}%
\pgfpathlineto{\pgfqpoint{2.424361in}{2.456018in}}%
\pgfpathlineto{\pgfqpoint{2.412928in}{2.452501in}}%
\pgfpathlineto{\pgfqpoint{2.401488in}{2.449013in}}%
\pgfpathlineto{\pgfqpoint{2.390043in}{2.445560in}}%
\pgfpathlineto{\pgfqpoint{2.396016in}{2.437109in}}%
\pgfpathlineto{\pgfqpoint{2.401994in}{2.428644in}}%
\pgfpathlineto{\pgfqpoint{2.407975in}{2.420165in}}%
\pgfpathlineto{\pgfqpoint{2.413961in}{2.411672in}}%
\pgfpathclose%
\pgfusepath{stroke,fill}%
\end{pgfscope}%
\begin{pgfscope}%
\pgfpathrectangle{\pgfqpoint{0.887500in}{0.275000in}}{\pgfqpoint{4.225000in}{4.225000in}}%
\pgfusepath{clip}%
\pgfsetbuttcap%
\pgfsetroundjoin%
\definecolor{currentfill}{rgb}{0.119738,0.603785,0.541400}%
\pgfsetfillcolor{currentfill}%
\pgfsetfillopacity{0.700000}%
\pgfsetlinewidth{0.501875pt}%
\definecolor{currentstroke}{rgb}{1.000000,1.000000,1.000000}%
\pgfsetstrokecolor{currentstroke}%
\pgfsetstrokeopacity{0.500000}%
\pgfsetdash{}{0pt}%
\pgfpathmoveto{\pgfqpoint{4.323230in}{2.582746in}}%
\pgfpathlineto{\pgfqpoint{4.334219in}{2.586089in}}%
\pgfpathlineto{\pgfqpoint{4.345202in}{2.589426in}}%
\pgfpathlineto{\pgfqpoint{4.356180in}{2.592760in}}%
\pgfpathlineto{\pgfqpoint{4.367153in}{2.596090in}}%
\pgfpathlineto{\pgfqpoint{4.378120in}{2.599418in}}%
\pgfpathlineto{\pgfqpoint{4.371693in}{2.613726in}}%
\pgfpathlineto{\pgfqpoint{4.365268in}{2.628023in}}%
\pgfpathlineto{\pgfqpoint{4.358845in}{2.642306in}}%
\pgfpathlineto{\pgfqpoint{4.352424in}{2.656571in}}%
\pgfpathlineto{\pgfqpoint{4.346004in}{2.670817in}}%
\pgfpathlineto{\pgfqpoint{4.335037in}{2.667451in}}%
\pgfpathlineto{\pgfqpoint{4.324065in}{2.664082in}}%
\pgfpathlineto{\pgfqpoint{4.313087in}{2.660710in}}%
\pgfpathlineto{\pgfqpoint{4.302104in}{2.657333in}}%
\pgfpathlineto{\pgfqpoint{4.291115in}{2.653950in}}%
\pgfpathlineto{\pgfqpoint{4.297535in}{2.639757in}}%
\pgfpathlineto{\pgfqpoint{4.303956in}{2.625539in}}%
\pgfpathlineto{\pgfqpoint{4.310379in}{2.611297in}}%
\pgfpathlineto{\pgfqpoint{4.316803in}{2.597032in}}%
\pgfpathclose%
\pgfusepath{stroke,fill}%
\end{pgfscope}%
\begin{pgfscope}%
\pgfpathrectangle{\pgfqpoint{0.887500in}{0.275000in}}{\pgfqpoint{4.225000in}{4.225000in}}%
\pgfusepath{clip}%
\pgfsetbuttcap%
\pgfsetroundjoin%
\definecolor{currentfill}{rgb}{0.232815,0.732247,0.459277}%
\pgfsetfillcolor{currentfill}%
\pgfsetfillopacity{0.700000}%
\pgfsetlinewidth{0.501875pt}%
\definecolor{currentstroke}{rgb}{1.000000,1.000000,1.000000}%
\pgfsetstrokecolor{currentstroke}%
\pgfsetstrokeopacity{0.500000}%
\pgfsetdash{}{0pt}%
\pgfpathmoveto{\pgfqpoint{3.942349in}{2.863405in}}%
\pgfpathlineto{\pgfqpoint{3.953437in}{2.866883in}}%
\pgfpathlineto{\pgfqpoint{3.964520in}{2.870381in}}%
\pgfpathlineto{\pgfqpoint{3.975599in}{2.873901in}}%
\pgfpathlineto{\pgfqpoint{3.986672in}{2.877442in}}%
\pgfpathlineto{\pgfqpoint{3.997741in}{2.881000in}}%
\pgfpathlineto{\pgfqpoint{3.991368in}{2.894704in}}%
\pgfpathlineto{\pgfqpoint{3.984998in}{2.908414in}}%
\pgfpathlineto{\pgfqpoint{3.978631in}{2.922136in}}%
\pgfpathlineto{\pgfqpoint{3.972267in}{2.935877in}}%
\pgfpathlineto{\pgfqpoint{3.965906in}{2.949634in}}%
\pgfpathlineto{\pgfqpoint{3.954839in}{2.946031in}}%
\pgfpathlineto{\pgfqpoint{3.943767in}{2.942441in}}%
\pgfpathlineto{\pgfqpoint{3.932690in}{2.938870in}}%
\pgfpathlineto{\pgfqpoint{3.921608in}{2.935326in}}%
\pgfpathlineto{\pgfqpoint{3.910522in}{2.931809in}}%
\pgfpathlineto{\pgfqpoint{3.916881in}{2.918062in}}%
\pgfpathlineto{\pgfqpoint{3.923243in}{2.904352in}}%
\pgfpathlineto{\pgfqpoint{3.929609in}{2.890679in}}%
\pgfpathlineto{\pgfqpoint{3.935977in}{2.877033in}}%
\pgfpathclose%
\pgfusepath{stroke,fill}%
\end{pgfscope}%
\begin{pgfscope}%
\pgfpathrectangle{\pgfqpoint{0.887500in}{0.275000in}}{\pgfqpoint{4.225000in}{4.225000in}}%
\pgfusepath{clip}%
\pgfsetbuttcap%
\pgfsetroundjoin%
\definecolor{currentfill}{rgb}{0.122606,0.585371,0.546557}%
\pgfsetfillcolor{currentfill}%
\pgfsetfillopacity{0.700000}%
\pgfsetlinewidth{0.501875pt}%
\definecolor{currentstroke}{rgb}{1.000000,1.000000,1.000000}%
\pgfsetstrokecolor{currentstroke}%
\pgfsetstrokeopacity{0.500000}%
\pgfsetdash{}{0pt}%
\pgfpathmoveto{\pgfqpoint{1.549884in}{2.551074in}}%
\pgfpathlineto{\pgfqpoint{1.561558in}{2.554387in}}%
\pgfpathlineto{\pgfqpoint{1.573227in}{2.557693in}}%
\pgfpathlineto{\pgfqpoint{1.584890in}{2.560993in}}%
\pgfpathlineto{\pgfqpoint{1.596548in}{2.564286in}}%
\pgfpathlineto{\pgfqpoint{1.608200in}{2.567575in}}%
\pgfpathlineto{\pgfqpoint{1.602501in}{2.575514in}}%
\pgfpathlineto{\pgfqpoint{1.596807in}{2.583426in}}%
\pgfpathlineto{\pgfqpoint{1.591118in}{2.591311in}}%
\pgfpathlineto{\pgfqpoint{1.585434in}{2.599168in}}%
\pgfpathlineto{\pgfqpoint{1.579754in}{2.606999in}}%
\pgfpathlineto{\pgfqpoint{1.568115in}{2.603714in}}%
\pgfpathlineto{\pgfqpoint{1.556470in}{2.600425in}}%
\pgfpathlineto{\pgfqpoint{1.544820in}{2.597130in}}%
\pgfpathlineto{\pgfqpoint{1.533165in}{2.593829in}}%
\pgfpathlineto{\pgfqpoint{1.521504in}{2.590520in}}%
\pgfpathlineto{\pgfqpoint{1.527170in}{2.582689in}}%
\pgfpathlineto{\pgfqpoint{1.532841in}{2.574830in}}%
\pgfpathlineto{\pgfqpoint{1.538517in}{2.566940in}}%
\pgfpathlineto{\pgfqpoint{1.544198in}{2.559021in}}%
\pgfpathclose%
\pgfusepath{stroke,fill}%
\end{pgfscope}%
\begin{pgfscope}%
\pgfpathrectangle{\pgfqpoint{0.887500in}{0.275000in}}{\pgfqpoint{4.225000in}{4.225000in}}%
\pgfusepath{clip}%
\pgfsetbuttcap%
\pgfsetroundjoin%
\definecolor{currentfill}{rgb}{0.730889,0.871916,0.156029}%
\pgfsetfillcolor{currentfill}%
\pgfsetfillopacity{0.700000}%
\pgfsetlinewidth{0.501875pt}%
\definecolor{currentstroke}{rgb}{1.000000,1.000000,1.000000}%
\pgfsetstrokecolor{currentstroke}%
\pgfsetstrokeopacity{0.500000}%
\pgfsetdash{}{0pt}%
\pgfpathmoveto{\pgfqpoint{3.067255in}{3.283583in}}%
\pgfpathlineto{\pgfqpoint{3.078557in}{3.295427in}}%
\pgfpathlineto{\pgfqpoint{3.089856in}{3.305280in}}%
\pgfpathlineto{\pgfqpoint{3.101152in}{3.313323in}}%
\pgfpathlineto{\pgfqpoint{3.112442in}{3.319739in}}%
\pgfpathlineto{\pgfqpoint{3.123727in}{3.324746in}}%
\pgfpathlineto{\pgfqpoint{3.117515in}{3.335024in}}%
\pgfpathlineto{\pgfqpoint{3.111306in}{3.345092in}}%
\pgfpathlineto{\pgfqpoint{3.105100in}{3.354940in}}%
\pgfpathlineto{\pgfqpoint{3.098898in}{3.364557in}}%
\pgfpathlineto{\pgfqpoint{3.092699in}{3.373932in}}%
\pgfpathlineto{\pgfqpoint{3.081424in}{3.368581in}}%
\pgfpathlineto{\pgfqpoint{3.070144in}{3.361684in}}%
\pgfpathlineto{\pgfqpoint{3.058861in}{3.352943in}}%
\pgfpathlineto{\pgfqpoint{3.047576in}{3.342041in}}%
\pgfpathlineto{\pgfqpoint{3.036291in}{3.328661in}}%
\pgfpathlineto{\pgfqpoint{3.042476in}{3.320113in}}%
\pgfpathlineto{\pgfqpoint{3.048665in}{3.311328in}}%
\pgfpathlineto{\pgfqpoint{3.054858in}{3.302310in}}%
\pgfpathlineto{\pgfqpoint{3.061055in}{3.293061in}}%
\pgfpathclose%
\pgfusepath{stroke,fill}%
\end{pgfscope}%
\begin{pgfscope}%
\pgfpathrectangle{\pgfqpoint{0.887500in}{0.275000in}}{\pgfqpoint{4.225000in}{4.225000in}}%
\pgfusepath{clip}%
\pgfsetbuttcap%
\pgfsetroundjoin%
\definecolor{currentfill}{rgb}{0.163625,0.471133,0.558148}%
\pgfsetfillcolor{currentfill}%
\pgfsetfillopacity{0.700000}%
\pgfsetlinewidth{0.501875pt}%
\definecolor{currentstroke}{rgb}{1.000000,1.000000,1.000000}%
\pgfsetstrokecolor{currentstroke}%
\pgfsetstrokeopacity{0.500000}%
\pgfsetdash{}{0pt}%
\pgfpathmoveto{\pgfqpoint{2.826574in}{2.303252in}}%
\pgfpathlineto{\pgfqpoint{2.837949in}{2.303926in}}%
\pgfpathlineto{\pgfqpoint{2.849304in}{2.307045in}}%
\pgfpathlineto{\pgfqpoint{2.860636in}{2.313717in}}%
\pgfpathlineto{\pgfqpoint{2.871942in}{2.325054in}}%
\pgfpathlineto{\pgfqpoint{2.883222in}{2.341987in}}%
\pgfpathlineto{\pgfqpoint{2.877085in}{2.350182in}}%
\pgfpathlineto{\pgfqpoint{2.870953in}{2.358295in}}%
\pgfpathlineto{\pgfqpoint{2.864825in}{2.366426in}}%
\pgfpathlineto{\pgfqpoint{2.858701in}{2.374674in}}%
\pgfpathlineto{\pgfqpoint{2.852579in}{2.383137in}}%
\pgfpathlineto{\pgfqpoint{2.841332in}{2.364101in}}%
\pgfpathlineto{\pgfqpoint{2.830051in}{2.351754in}}%
\pgfpathlineto{\pgfqpoint{2.818735in}{2.344977in}}%
\pgfpathlineto{\pgfqpoint{2.807389in}{2.342432in}}%
\pgfpathlineto{\pgfqpoint{2.796017in}{2.342786in}}%
\pgfpathlineto{\pgfqpoint{2.802121in}{2.334787in}}%
\pgfpathlineto{\pgfqpoint{2.808228in}{2.326932in}}%
\pgfpathlineto{\pgfqpoint{2.814339in}{2.319123in}}%
\pgfpathlineto{\pgfqpoint{2.820454in}{2.311262in}}%
\pgfpathclose%
\pgfusepath{stroke,fill}%
\end{pgfscope}%
\begin{pgfscope}%
\pgfpathrectangle{\pgfqpoint{0.887500in}{0.275000in}}{\pgfqpoint{4.225000in}{4.225000in}}%
\pgfusepath{clip}%
\pgfsetbuttcap%
\pgfsetroundjoin%
\definecolor{currentfill}{rgb}{0.129933,0.559582,0.551864}%
\pgfsetfillcolor{currentfill}%
\pgfsetfillopacity{0.700000}%
\pgfsetlinewidth{0.501875pt}%
\definecolor{currentstroke}{rgb}{1.000000,1.000000,1.000000}%
\pgfsetstrokecolor{currentstroke}%
\pgfsetstrokeopacity{0.500000}%
\pgfsetdash{}{0pt}%
\pgfpathmoveto{\pgfqpoint{1.868874in}{2.495861in}}%
\pgfpathlineto{\pgfqpoint{1.880474in}{2.499120in}}%
\pgfpathlineto{\pgfqpoint{1.892068in}{2.502375in}}%
\pgfpathlineto{\pgfqpoint{1.903656in}{2.505628in}}%
\pgfpathlineto{\pgfqpoint{1.915239in}{2.508881in}}%
\pgfpathlineto{\pgfqpoint{1.926816in}{2.512136in}}%
\pgfpathlineto{\pgfqpoint{1.921003in}{2.520282in}}%
\pgfpathlineto{\pgfqpoint{1.915194in}{2.528412in}}%
\pgfpathlineto{\pgfqpoint{1.909389in}{2.536527in}}%
\pgfpathlineto{\pgfqpoint{1.903588in}{2.544627in}}%
\pgfpathlineto{\pgfqpoint{1.897791in}{2.552712in}}%
\pgfpathlineto{\pgfqpoint{1.886227in}{2.549467in}}%
\pgfpathlineto{\pgfqpoint{1.874657in}{2.546224in}}%
\pgfpathlineto{\pgfqpoint{1.863081in}{2.542983in}}%
\pgfpathlineto{\pgfqpoint{1.851499in}{2.539740in}}%
\pgfpathlineto{\pgfqpoint{1.839911in}{2.536494in}}%
\pgfpathlineto{\pgfqpoint{1.845695in}{2.528400in}}%
\pgfpathlineto{\pgfqpoint{1.851484in}{2.520290in}}%
\pgfpathlineto{\pgfqpoint{1.857276in}{2.512164in}}%
\pgfpathlineto{\pgfqpoint{1.863073in}{2.504021in}}%
\pgfpathclose%
\pgfusepath{stroke,fill}%
\end{pgfscope}%
\begin{pgfscope}%
\pgfpathrectangle{\pgfqpoint{0.887500in}{0.275000in}}{\pgfqpoint{4.225000in}{4.225000in}}%
\pgfusepath{clip}%
\pgfsetbuttcap%
\pgfsetroundjoin%
\definecolor{currentfill}{rgb}{0.122312,0.633153,0.530398}%
\pgfsetfillcolor{currentfill}%
\pgfsetfillopacity{0.700000}%
\pgfsetlinewidth{0.501875pt}%
\definecolor{currentstroke}{rgb}{1.000000,1.000000,1.000000}%
\pgfsetstrokecolor{currentstroke}%
\pgfsetstrokeopacity{0.500000}%
\pgfsetdash{}{0pt}%
\pgfpathmoveto{\pgfqpoint{4.236088in}{2.636900in}}%
\pgfpathlineto{\pgfqpoint{4.247105in}{2.640332in}}%
\pgfpathlineto{\pgfqpoint{4.258116in}{2.643751in}}%
\pgfpathlineto{\pgfqpoint{4.269121in}{2.647160in}}%
\pgfpathlineto{\pgfqpoint{4.280121in}{2.650559in}}%
\pgfpathlineto{\pgfqpoint{4.291115in}{2.653950in}}%
\pgfpathlineto{\pgfqpoint{4.284697in}{2.668116in}}%
\pgfpathlineto{\pgfqpoint{4.278281in}{2.682255in}}%
\pgfpathlineto{\pgfqpoint{4.271867in}{2.696366in}}%
\pgfpathlineto{\pgfqpoint{4.265454in}{2.710447in}}%
\pgfpathlineto{\pgfqpoint{4.259043in}{2.724504in}}%
\pgfpathlineto{\pgfqpoint{4.248049in}{2.721077in}}%
\pgfpathlineto{\pgfqpoint{4.237051in}{2.717647in}}%
\pgfpathlineto{\pgfqpoint{4.226046in}{2.714214in}}%
\pgfpathlineto{\pgfqpoint{4.215037in}{2.710777in}}%
\pgfpathlineto{\pgfqpoint{4.204022in}{2.707334in}}%
\pgfpathlineto{\pgfqpoint{4.210431in}{2.693293in}}%
\pgfpathlineto{\pgfqpoint{4.216843in}{2.679232in}}%
\pgfpathlineto{\pgfqpoint{4.223256in}{2.665146in}}%
\pgfpathlineto{\pgfqpoint{4.229671in}{2.651035in}}%
\pgfpathclose%
\pgfusepath{stroke,fill}%
\end{pgfscope}%
\begin{pgfscope}%
\pgfpathrectangle{\pgfqpoint{0.887500in}{0.275000in}}{\pgfqpoint{4.225000in}{4.225000in}}%
\pgfusepath{clip}%
\pgfsetbuttcap%
\pgfsetroundjoin%
\definecolor{currentfill}{rgb}{0.281477,0.755203,0.432552}%
\pgfsetfillcolor{currentfill}%
\pgfsetfillopacity{0.700000}%
\pgfsetlinewidth{0.501875pt}%
\definecolor{currentstroke}{rgb}{1.000000,1.000000,1.000000}%
\pgfsetstrokecolor{currentstroke}%
\pgfsetstrokeopacity{0.500000}%
\pgfsetdash{}{0pt}%
\pgfpathmoveto{\pgfqpoint{3.855017in}{2.914422in}}%
\pgfpathlineto{\pgfqpoint{3.866129in}{2.917894in}}%
\pgfpathlineto{\pgfqpoint{3.877235in}{2.921362in}}%
\pgfpathlineto{\pgfqpoint{3.888336in}{2.924833in}}%
\pgfpathlineto{\pgfqpoint{3.899431in}{2.928314in}}%
\pgfpathlineto{\pgfqpoint{3.910522in}{2.931809in}}%
\pgfpathlineto{\pgfqpoint{3.904166in}{2.945571in}}%
\pgfpathlineto{\pgfqpoint{3.897813in}{2.959322in}}%
\pgfpathlineto{\pgfqpoint{3.891461in}{2.973036in}}%
\pgfpathlineto{\pgfqpoint{3.885110in}{2.986688in}}%
\pgfpathlineto{\pgfqpoint{3.878761in}{3.000252in}}%
\pgfpathlineto{\pgfqpoint{3.867672in}{2.996737in}}%
\pgfpathlineto{\pgfqpoint{3.856578in}{2.993224in}}%
\pgfpathlineto{\pgfqpoint{3.845479in}{2.989707in}}%
\pgfpathlineto{\pgfqpoint{3.834375in}{2.986181in}}%
\pgfpathlineto{\pgfqpoint{3.823265in}{2.982641in}}%
\pgfpathlineto{\pgfqpoint{3.829613in}{2.969110in}}%
\pgfpathlineto{\pgfqpoint{3.835961in}{2.955500in}}%
\pgfpathlineto{\pgfqpoint{3.842311in}{2.941834in}}%
\pgfpathlineto{\pgfqpoint{3.848663in}{2.928135in}}%
\pgfpathclose%
\pgfusepath{stroke,fill}%
\end{pgfscope}%
\begin{pgfscope}%
\pgfpathrectangle{\pgfqpoint{0.887500in}{0.275000in}}{\pgfqpoint{4.225000in}{4.225000in}}%
\pgfusepath{clip}%
\pgfsetbuttcap%
\pgfsetroundjoin%
\definecolor{currentfill}{rgb}{0.139147,0.533812,0.555298}%
\pgfsetfillcolor{currentfill}%
\pgfsetfillopacity{0.700000}%
\pgfsetlinewidth{0.501875pt}%
\definecolor{currentstroke}{rgb}{1.000000,1.000000,1.000000}%
\pgfsetstrokecolor{currentstroke}%
\pgfsetstrokeopacity{0.500000}%
\pgfsetdash{}{0pt}%
\pgfpathmoveto{\pgfqpoint{2.188021in}{2.437853in}}%
\pgfpathlineto{\pgfqpoint{2.199542in}{2.441158in}}%
\pgfpathlineto{\pgfqpoint{2.211057in}{2.444454in}}%
\pgfpathlineto{\pgfqpoint{2.222568in}{2.447741in}}%
\pgfpathlineto{\pgfqpoint{2.234072in}{2.451020in}}%
\pgfpathlineto{\pgfqpoint{2.245571in}{2.454295in}}%
\pgfpathlineto{\pgfqpoint{2.239646in}{2.462652in}}%
\pgfpathlineto{\pgfqpoint{2.233726in}{2.470996in}}%
\pgfpathlineto{\pgfqpoint{2.227809in}{2.479325in}}%
\pgfpathlineto{\pgfqpoint{2.221896in}{2.487640in}}%
\pgfpathlineto{\pgfqpoint{2.215988in}{2.495940in}}%
\pgfpathlineto{\pgfqpoint{2.204501in}{2.492666in}}%
\pgfpathlineto{\pgfqpoint{2.193009in}{2.489390in}}%
\pgfpathlineto{\pgfqpoint{2.181510in}{2.486110in}}%
\pgfpathlineto{\pgfqpoint{2.170007in}{2.482825in}}%
\pgfpathlineto{\pgfqpoint{2.158497in}{2.479533in}}%
\pgfpathlineto{\pgfqpoint{2.164394in}{2.471229in}}%
\pgfpathlineto{\pgfqpoint{2.170294in}{2.462909in}}%
\pgfpathlineto{\pgfqpoint{2.176199in}{2.454573in}}%
\pgfpathlineto{\pgfqpoint{2.182108in}{2.446221in}}%
\pgfpathclose%
\pgfusepath{stroke,fill}%
\end{pgfscope}%
\begin{pgfscope}%
\pgfpathrectangle{\pgfqpoint{0.887500in}{0.275000in}}{\pgfqpoint{4.225000in}{4.225000in}}%
\pgfusepath{clip}%
\pgfsetbuttcap%
\pgfsetroundjoin%
\definecolor{currentfill}{rgb}{0.150476,0.504369,0.557430}%
\pgfsetfillcolor{currentfill}%
\pgfsetfillopacity{0.700000}%
\pgfsetlinewidth{0.501875pt}%
\definecolor{currentstroke}{rgb}{1.000000,1.000000,1.000000}%
\pgfsetstrokecolor{currentstroke}%
\pgfsetstrokeopacity{0.500000}%
\pgfsetdash{}{0pt}%
\pgfpathmoveto{\pgfqpoint{2.507242in}{2.378094in}}%
\pgfpathlineto{\pgfqpoint{2.518683in}{2.381572in}}%
\pgfpathlineto{\pgfqpoint{2.530119in}{2.384949in}}%
\pgfpathlineto{\pgfqpoint{2.541552in}{2.388192in}}%
\pgfpathlineto{\pgfqpoint{2.552981in}{2.391266in}}%
\pgfpathlineto{\pgfqpoint{2.564406in}{2.394195in}}%
\pgfpathlineto{\pgfqpoint{2.558373in}{2.402763in}}%
\pgfpathlineto{\pgfqpoint{2.552344in}{2.411319in}}%
\pgfpathlineto{\pgfqpoint{2.546319in}{2.419862in}}%
\pgfpathlineto{\pgfqpoint{2.540298in}{2.428395in}}%
\pgfpathlineto{\pgfqpoint{2.534282in}{2.436919in}}%
\pgfpathlineto{\pgfqpoint{2.522868in}{2.433946in}}%
\pgfpathlineto{\pgfqpoint{2.511451in}{2.430844in}}%
\pgfpathlineto{\pgfqpoint{2.500030in}{2.427589in}}%
\pgfpathlineto{\pgfqpoint{2.488605in}{2.424213in}}%
\pgfpathlineto{\pgfqpoint{2.477176in}{2.420748in}}%
\pgfpathlineto{\pgfqpoint{2.483181in}{2.412249in}}%
\pgfpathlineto{\pgfqpoint{2.489190in}{2.403734in}}%
\pgfpathlineto{\pgfqpoint{2.495203in}{2.395204in}}%
\pgfpathlineto{\pgfqpoint{2.501221in}{2.386658in}}%
\pgfpathclose%
\pgfusepath{stroke,fill}%
\end{pgfscope}%
\begin{pgfscope}%
\pgfpathrectangle{\pgfqpoint{0.887500in}{0.275000in}}{\pgfqpoint{4.225000in}{4.225000in}}%
\pgfusepath{clip}%
\pgfsetbuttcap%
\pgfsetroundjoin%
\definecolor{currentfill}{rgb}{0.124780,0.640461,0.527068}%
\pgfsetfillcolor{currentfill}%
\pgfsetfillopacity{0.700000}%
\pgfsetlinewidth{0.501875pt}%
\definecolor{currentstroke}{rgb}{1.000000,1.000000,1.000000}%
\pgfsetstrokecolor{currentstroke}%
\pgfsetstrokeopacity{0.500000}%
\pgfsetdash{}{0pt}%
\pgfpathmoveto{\pgfqpoint{2.877851in}{2.598257in}}%
\pgfpathlineto{\pgfqpoint{2.889044in}{2.634537in}}%
\pgfpathlineto{\pgfqpoint{2.900259in}{2.668694in}}%
\pgfpathlineto{\pgfqpoint{2.911494in}{2.700744in}}%
\pgfpathlineto{\pgfqpoint{2.922746in}{2.731086in}}%
\pgfpathlineto{\pgfqpoint{2.934011in}{2.760117in}}%
\pgfpathlineto{\pgfqpoint{2.927853in}{2.766294in}}%
\pgfpathlineto{\pgfqpoint{2.921706in}{2.770328in}}%
\pgfpathlineto{\pgfqpoint{2.915571in}{2.772542in}}%
\pgfpathlineto{\pgfqpoint{2.909447in}{2.773261in}}%
\pgfpathlineto{\pgfqpoint{2.903335in}{2.772806in}}%
\pgfpathlineto{\pgfqpoint{2.892099in}{2.746015in}}%
\pgfpathlineto{\pgfqpoint{2.880874in}{2.718795in}}%
\pgfpathlineto{\pgfqpoint{2.869659in}{2.690866in}}%
\pgfpathlineto{\pgfqpoint{2.858458in}{2.661950in}}%
\pgfpathlineto{\pgfqpoint{2.847272in}{2.632014in}}%
\pgfpathlineto{\pgfqpoint{2.853371in}{2.626750in}}%
\pgfpathlineto{\pgfqpoint{2.859478in}{2.620888in}}%
\pgfpathlineto{\pgfqpoint{2.865593in}{2.614283in}}%
\pgfpathlineto{\pgfqpoint{2.871718in}{2.606788in}}%
\pgfpathclose%
\pgfusepath{stroke,fill}%
\end{pgfscope}%
\begin{pgfscope}%
\pgfpathrectangle{\pgfqpoint{0.887500in}{0.275000in}}{\pgfqpoint{4.225000in}{4.225000in}}%
\pgfusepath{clip}%
\pgfsetbuttcap%
\pgfsetroundjoin%
\definecolor{currentfill}{rgb}{0.335885,0.777018,0.402049}%
\pgfsetfillcolor{currentfill}%
\pgfsetfillopacity{0.700000}%
\pgfsetlinewidth{0.501875pt}%
\definecolor{currentstroke}{rgb}{1.000000,1.000000,1.000000}%
\pgfsetstrokecolor{currentstroke}%
\pgfsetstrokeopacity{0.500000}%
\pgfsetdash{}{0pt}%
\pgfpathmoveto{\pgfqpoint{3.767632in}{2.964667in}}%
\pgfpathlineto{\pgfqpoint{3.778770in}{2.968294in}}%
\pgfpathlineto{\pgfqpoint{3.789902in}{2.971906in}}%
\pgfpathlineto{\pgfqpoint{3.801029in}{2.975503in}}%
\pgfpathlineto{\pgfqpoint{3.812150in}{2.979082in}}%
\pgfpathlineto{\pgfqpoint{3.823265in}{2.982641in}}%
\pgfpathlineto{\pgfqpoint{3.816919in}{2.996073in}}%
\pgfpathlineto{\pgfqpoint{3.810572in}{3.009384in}}%
\pgfpathlineto{\pgfqpoint{3.804226in}{3.022551in}}%
\pgfpathlineto{\pgfqpoint{3.797879in}{3.035553in}}%
\pgfpathlineto{\pgfqpoint{3.791532in}{3.048371in}}%
\pgfpathlineto{\pgfqpoint{3.780420in}{3.044820in}}%
\pgfpathlineto{\pgfqpoint{3.769303in}{3.041244in}}%
\pgfpathlineto{\pgfqpoint{3.758180in}{3.037651in}}%
\pgfpathlineto{\pgfqpoint{3.747052in}{3.034044in}}%
\pgfpathlineto{\pgfqpoint{3.735918in}{3.030429in}}%
\pgfpathlineto{\pgfqpoint{3.742260in}{3.017516in}}%
\pgfpathlineto{\pgfqpoint{3.748602in}{3.004471in}}%
\pgfpathlineto{\pgfqpoint{3.754944in}{2.991305in}}%
\pgfpathlineto{\pgfqpoint{3.761288in}{2.978032in}}%
\pgfpathclose%
\pgfusepath{stroke,fill}%
\end{pgfscope}%
\begin{pgfscope}%
\pgfpathrectangle{\pgfqpoint{0.887500in}{0.275000in}}{\pgfqpoint{4.225000in}{4.225000in}}%
\pgfusepath{clip}%
\pgfsetbuttcap%
\pgfsetroundjoin%
\definecolor{currentfill}{rgb}{0.134692,0.658636,0.517649}%
\pgfsetfillcolor{currentfill}%
\pgfsetfillopacity{0.700000}%
\pgfsetlinewidth{0.501875pt}%
\definecolor{currentstroke}{rgb}{1.000000,1.000000,1.000000}%
\pgfsetstrokecolor{currentstroke}%
\pgfsetstrokeopacity{0.500000}%
\pgfsetdash{}{0pt}%
\pgfpathmoveto{\pgfqpoint{4.148866in}{2.690072in}}%
\pgfpathlineto{\pgfqpoint{4.159907in}{2.693524in}}%
\pgfpathlineto{\pgfqpoint{4.170944in}{2.696978in}}%
\pgfpathlineto{\pgfqpoint{4.181975in}{2.700433in}}%
\pgfpathlineto{\pgfqpoint{4.193001in}{2.703885in}}%
\pgfpathlineto{\pgfqpoint{4.204022in}{2.707334in}}%
\pgfpathlineto{\pgfqpoint{4.197614in}{2.721356in}}%
\pgfpathlineto{\pgfqpoint{4.191208in}{2.735363in}}%
\pgfpathlineto{\pgfqpoint{4.184805in}{2.749358in}}%
\pgfpathlineto{\pgfqpoint{4.178404in}{2.763346in}}%
\pgfpathlineto{\pgfqpoint{4.172006in}{2.777328in}}%
\pgfpathlineto{\pgfqpoint{4.160987in}{2.773876in}}%
\pgfpathlineto{\pgfqpoint{4.149963in}{2.770423in}}%
\pgfpathlineto{\pgfqpoint{4.138934in}{2.766973in}}%
\pgfpathlineto{\pgfqpoint{4.127900in}{2.763526in}}%
\pgfpathlineto{\pgfqpoint{4.116861in}{2.760085in}}%
\pgfpathlineto{\pgfqpoint{4.123258in}{2.746113in}}%
\pgfpathlineto{\pgfqpoint{4.129656in}{2.732127in}}%
\pgfpathlineto{\pgfqpoint{4.136057in}{2.718125in}}%
\pgfpathlineto{\pgfqpoint{4.142461in}{2.704107in}}%
\pgfpathclose%
\pgfusepath{stroke,fill}%
\end{pgfscope}%
\begin{pgfscope}%
\pgfpathrectangle{\pgfqpoint{0.887500in}{0.275000in}}{\pgfqpoint{4.225000in}{4.225000in}}%
\pgfusepath{clip}%
\pgfsetbuttcap%
\pgfsetroundjoin%
\definecolor{currentfill}{rgb}{0.125394,0.574318,0.549086}%
\pgfsetfillcolor{currentfill}%
\pgfsetfillopacity{0.700000}%
\pgfsetlinewidth{0.501875pt}%
\definecolor{currentstroke}{rgb}{1.000000,1.000000,1.000000}%
\pgfsetstrokecolor{currentstroke}%
\pgfsetstrokeopacity{0.500000}%
\pgfsetdash{}{0pt}%
\pgfpathmoveto{\pgfqpoint{1.636761in}{2.527538in}}%
\pgfpathlineto{\pgfqpoint{1.648420in}{2.530838in}}%
\pgfpathlineto{\pgfqpoint{1.660074in}{2.534136in}}%
\pgfpathlineto{\pgfqpoint{1.671722in}{2.537433in}}%
\pgfpathlineto{\pgfqpoint{1.683364in}{2.540731in}}%
\pgfpathlineto{\pgfqpoint{1.695000in}{2.544029in}}%
\pgfpathlineto{\pgfqpoint{1.689266in}{2.552060in}}%
\pgfpathlineto{\pgfqpoint{1.683537in}{2.560073in}}%
\pgfpathlineto{\pgfqpoint{1.677812in}{2.568066in}}%
\pgfpathlineto{\pgfqpoint{1.672091in}{2.576039in}}%
\pgfpathlineto{\pgfqpoint{1.666375in}{2.583989in}}%
\pgfpathlineto{\pgfqpoint{1.654752in}{2.580708in}}%
\pgfpathlineto{\pgfqpoint{1.643123in}{2.577426in}}%
\pgfpathlineto{\pgfqpoint{1.631487in}{2.574144in}}%
\pgfpathlineto{\pgfqpoint{1.619847in}{2.570861in}}%
\pgfpathlineto{\pgfqpoint{1.608200in}{2.567575in}}%
\pgfpathlineto{\pgfqpoint{1.613903in}{2.559612in}}%
\pgfpathlineto{\pgfqpoint{1.619611in}{2.551625in}}%
\pgfpathlineto{\pgfqpoint{1.625323in}{2.543617in}}%
\pgfpathlineto{\pgfqpoint{1.631040in}{2.535587in}}%
\pgfpathclose%
\pgfusepath{stroke,fill}%
\end{pgfscope}%
\begin{pgfscope}%
\pgfpathrectangle{\pgfqpoint{0.887500in}{0.275000in}}{\pgfqpoint{4.225000in}{4.225000in}}%
\pgfusepath{clip}%
\pgfsetbuttcap%
\pgfsetroundjoin%
\definecolor{currentfill}{rgb}{0.699415,0.867117,0.175971}%
\pgfsetfillcolor{currentfill}%
\pgfsetfillopacity{0.700000}%
\pgfsetlinewidth{0.501875pt}%
\definecolor{currentstroke}{rgb}{1.000000,1.000000,1.000000}%
\pgfsetstrokecolor{currentstroke}%
\pgfsetstrokeopacity{0.500000}%
\pgfsetdash{}{0pt}%
\pgfpathmoveto{\pgfqpoint{3.154833in}{3.270577in}}%
\pgfpathlineto{\pgfqpoint{3.166119in}{3.274612in}}%
\pgfpathlineto{\pgfqpoint{3.177398in}{3.277896in}}%
\pgfpathlineto{\pgfqpoint{3.188669in}{3.280640in}}%
\pgfpathlineto{\pgfqpoint{3.199934in}{3.283051in}}%
\pgfpathlineto{\pgfqpoint{3.211192in}{3.285340in}}%
\pgfpathlineto{\pgfqpoint{3.204956in}{3.295861in}}%
\pgfpathlineto{\pgfqpoint{3.198724in}{3.306368in}}%
\pgfpathlineto{\pgfqpoint{3.192494in}{3.316837in}}%
\pgfpathlineto{\pgfqpoint{3.186268in}{3.327245in}}%
\pgfpathlineto{\pgfqpoint{3.180045in}{3.337568in}}%
\pgfpathlineto{\pgfqpoint{3.168796in}{3.335780in}}%
\pgfpathlineto{\pgfqpoint{3.157539in}{3.333859in}}%
\pgfpathlineto{\pgfqpoint{3.146275in}{3.331550in}}%
\pgfpathlineto{\pgfqpoint{3.135005in}{3.328598in}}%
\pgfpathlineto{\pgfqpoint{3.123727in}{3.324746in}}%
\pgfpathlineto{\pgfqpoint{3.129942in}{3.314268in}}%
\pgfpathlineto{\pgfqpoint{3.136160in}{3.303602in}}%
\pgfpathlineto{\pgfqpoint{3.142381in}{3.292758in}}%
\pgfpathlineto{\pgfqpoint{3.148606in}{3.281746in}}%
\pgfpathclose%
\pgfusepath{stroke,fill}%
\end{pgfscope}%
\begin{pgfscope}%
\pgfpathrectangle{\pgfqpoint{0.887500in}{0.275000in}}{\pgfqpoint{4.225000in}{4.225000in}}%
\pgfusepath{clip}%
\pgfsetbuttcap%
\pgfsetroundjoin%
\definecolor{currentfill}{rgb}{0.449368,0.813768,0.335384}%
\pgfsetfillcolor{currentfill}%
\pgfsetfillopacity{0.700000}%
\pgfsetlinewidth{0.501875pt}%
\definecolor{currentstroke}{rgb}{1.000000,1.000000,1.000000}%
\pgfsetstrokecolor{currentstroke}%
\pgfsetstrokeopacity{0.500000}%
\pgfsetdash{}{0pt}%
\pgfpathmoveto{\pgfqpoint{2.985234in}{3.042082in}}%
\pgfpathlineto{\pgfqpoint{2.996515in}{3.068200in}}%
\pgfpathlineto{\pgfqpoint{3.007806in}{3.092868in}}%
\pgfpathlineto{\pgfqpoint{3.019105in}{3.116037in}}%
\pgfpathlineto{\pgfqpoint{3.030410in}{3.137658in}}%
\pgfpathlineto{\pgfqpoint{3.041721in}{3.157681in}}%
\pgfpathlineto{\pgfqpoint{3.035523in}{3.164230in}}%
\pgfpathlineto{\pgfqpoint{3.029329in}{3.170612in}}%
\pgfpathlineto{\pgfqpoint{3.023140in}{3.176820in}}%
\pgfpathlineto{\pgfqpoint{3.016956in}{3.182842in}}%
\pgfpathlineto{\pgfqpoint{3.010777in}{3.188668in}}%
\pgfpathlineto{\pgfqpoint{2.999502in}{3.162498in}}%
\pgfpathlineto{\pgfqpoint{2.988237in}{3.134632in}}%
\pgfpathlineto{\pgfqpoint{2.976983in}{3.105468in}}%
\pgfpathlineto{\pgfqpoint{2.965740in}{3.075408in}}%
\pgfpathlineto{\pgfqpoint{2.954508in}{3.044847in}}%
\pgfpathlineto{\pgfqpoint{2.960641in}{3.044236in}}%
\pgfpathlineto{\pgfqpoint{2.966780in}{3.043625in}}%
\pgfpathlineto{\pgfqpoint{2.972925in}{3.043043in}}%
\pgfpathlineto{\pgfqpoint{2.979076in}{3.042521in}}%
\pgfpathclose%
\pgfusepath{stroke,fill}%
\end{pgfscope}%
\begin{pgfscope}%
\pgfpathrectangle{\pgfqpoint{0.887500in}{0.275000in}}{\pgfqpoint{4.225000in}{4.225000in}}%
\pgfusepath{clip}%
\pgfsetbuttcap%
\pgfsetroundjoin%
\definecolor{currentfill}{rgb}{0.386433,0.794644,0.372886}%
\pgfsetfillcolor{currentfill}%
\pgfsetfillopacity{0.700000}%
\pgfsetlinewidth{0.501875pt}%
\definecolor{currentstroke}{rgb}{1.000000,1.000000,1.000000}%
\pgfsetstrokecolor{currentstroke}%
\pgfsetstrokeopacity{0.500000}%
\pgfsetdash{}{0pt}%
\pgfpathmoveto{\pgfqpoint{3.680173in}{3.012421in}}%
\pgfpathlineto{\pgfqpoint{3.691332in}{3.015996in}}%
\pgfpathlineto{\pgfqpoint{3.702486in}{3.019589in}}%
\pgfpathlineto{\pgfqpoint{3.713635in}{3.023196in}}%
\pgfpathlineto{\pgfqpoint{3.724779in}{3.026811in}}%
\pgfpathlineto{\pgfqpoint{3.735918in}{3.030429in}}%
\pgfpathlineto{\pgfqpoint{3.729577in}{3.043212in}}%
\pgfpathlineto{\pgfqpoint{3.723237in}{3.055880in}}%
\pgfpathlineto{\pgfqpoint{3.716899in}{3.068444in}}%
\pgfpathlineto{\pgfqpoint{3.710561in}{3.080917in}}%
\pgfpathlineto{\pgfqpoint{3.704226in}{3.093312in}}%
\pgfpathlineto{\pgfqpoint{3.693096in}{3.089959in}}%
\pgfpathlineto{\pgfqpoint{3.681961in}{3.086648in}}%
\pgfpathlineto{\pgfqpoint{3.670821in}{3.083387in}}%
\pgfpathlineto{\pgfqpoint{3.659677in}{3.080167in}}%
\pgfpathlineto{\pgfqpoint{3.648528in}{3.076975in}}%
\pgfpathlineto{\pgfqpoint{3.654854in}{3.064224in}}%
\pgfpathlineto{\pgfqpoint{3.661181in}{3.051384in}}%
\pgfpathlineto{\pgfqpoint{3.667510in}{3.038464in}}%
\pgfpathlineto{\pgfqpoint{3.673840in}{3.025474in}}%
\pgfpathclose%
\pgfusepath{stroke,fill}%
\end{pgfscope}%
\begin{pgfscope}%
\pgfpathrectangle{\pgfqpoint{0.887500in}{0.275000in}}{\pgfqpoint{4.225000in}{4.225000in}}%
\pgfusepath{clip}%
\pgfsetbuttcap%
\pgfsetroundjoin%
\definecolor{currentfill}{rgb}{0.647257,0.858400,0.209861}%
\pgfsetfillcolor{currentfill}%
\pgfsetfillopacity{0.700000}%
\pgfsetlinewidth{0.501875pt}%
\definecolor{currentstroke}{rgb}{1.000000,1.000000,1.000000}%
\pgfsetstrokecolor{currentstroke}%
\pgfsetstrokeopacity{0.500000}%
\pgfsetdash{}{0pt}%
\pgfpathmoveto{\pgfqpoint{3.242420in}{3.232138in}}%
\pgfpathlineto{\pgfqpoint{3.253684in}{3.235380in}}%
\pgfpathlineto{\pgfqpoint{3.264942in}{3.238653in}}%
\pgfpathlineto{\pgfqpoint{3.276195in}{3.241942in}}%
\pgfpathlineto{\pgfqpoint{3.287443in}{3.245232in}}%
\pgfpathlineto{\pgfqpoint{3.298685in}{3.248507in}}%
\pgfpathlineto{\pgfqpoint{3.292422in}{3.258638in}}%
\pgfpathlineto{\pgfqpoint{3.286162in}{3.268674in}}%
\pgfpathlineto{\pgfqpoint{3.279906in}{3.278654in}}%
\pgfpathlineto{\pgfqpoint{3.273653in}{3.288616in}}%
\pgfpathlineto{\pgfqpoint{3.267403in}{3.298597in}}%
\pgfpathlineto{\pgfqpoint{3.256171in}{3.295685in}}%
\pgfpathlineto{\pgfqpoint{3.244934in}{3.292890in}}%
\pgfpathlineto{\pgfqpoint{3.233692in}{3.290226in}}%
\pgfpathlineto{\pgfqpoint{3.222445in}{3.287706in}}%
\pgfpathlineto{\pgfqpoint{3.211192in}{3.285340in}}%
\pgfpathlineto{\pgfqpoint{3.217431in}{3.274801in}}%
\pgfpathlineto{\pgfqpoint{3.223674in}{3.264228in}}%
\pgfpathlineto{\pgfqpoint{3.229919in}{3.253605in}}%
\pgfpathlineto{\pgfqpoint{3.236168in}{3.242914in}}%
\pgfpathclose%
\pgfusepath{stroke,fill}%
\end{pgfscope}%
\begin{pgfscope}%
\pgfpathrectangle{\pgfqpoint{0.887500in}{0.275000in}}{\pgfqpoint{4.225000in}{4.225000in}}%
\pgfusepath{clip}%
\pgfsetbuttcap%
\pgfsetroundjoin%
\definecolor{currentfill}{rgb}{0.133743,0.548535,0.553541}%
\pgfsetfillcolor{currentfill}%
\pgfsetfillopacity{0.700000}%
\pgfsetlinewidth{0.501875pt}%
\definecolor{currentstroke}{rgb}{1.000000,1.000000,1.000000}%
\pgfsetstrokecolor{currentstroke}%
\pgfsetstrokeopacity{0.500000}%
\pgfsetdash{}{0pt}%
\pgfpathmoveto{\pgfqpoint{4.442537in}{2.456554in}}%
\pgfpathlineto{\pgfqpoint{4.453508in}{2.460067in}}%
\pgfpathlineto{\pgfqpoint{4.464474in}{2.463603in}}%
\pgfpathlineto{\pgfqpoint{4.475437in}{2.467163in}}%
\pgfpathlineto{\pgfqpoint{4.486395in}{2.470748in}}%
\pgfpathlineto{\pgfqpoint{4.479933in}{2.484838in}}%
\pgfpathlineto{\pgfqpoint{4.473476in}{2.498954in}}%
\pgfpathlineto{\pgfqpoint{4.467021in}{2.513093in}}%
\pgfpathlineto{\pgfqpoint{4.460571in}{2.527258in}}%
\pgfpathlineto{\pgfqpoint{4.454123in}{2.541446in}}%
\pgfpathlineto{\pgfqpoint{4.443173in}{2.538023in}}%
\pgfpathlineto{\pgfqpoint{4.432217in}{2.534613in}}%
\pgfpathlineto{\pgfqpoint{4.421257in}{2.531215in}}%
\pgfpathlineto{\pgfqpoint{4.410292in}{2.527829in}}%
\pgfpathlineto{\pgfqpoint{4.416734in}{2.513529in}}%
\pgfpathlineto{\pgfqpoint{4.423179in}{2.499247in}}%
\pgfpathlineto{\pgfqpoint{4.429628in}{2.484987in}}%
\pgfpathlineto{\pgfqpoint{4.436081in}{2.470754in}}%
\pgfpathclose%
\pgfusepath{stroke,fill}%
\end{pgfscope}%
\begin{pgfscope}%
\pgfpathrectangle{\pgfqpoint{0.887500in}{0.275000in}}{\pgfqpoint{4.225000in}{4.225000in}}%
\pgfusepath{clip}%
\pgfsetbuttcap%
\pgfsetroundjoin%
\definecolor{currentfill}{rgb}{0.133743,0.548535,0.553541}%
\pgfsetfillcolor{currentfill}%
\pgfsetfillopacity{0.700000}%
\pgfsetlinewidth{0.501875pt}%
\definecolor{currentstroke}{rgb}{1.000000,1.000000,1.000000}%
\pgfsetstrokecolor{currentstroke}%
\pgfsetstrokeopacity{0.500000}%
\pgfsetdash{}{0pt}%
\pgfpathmoveto{\pgfqpoint{1.955947in}{2.471139in}}%
\pgfpathlineto{\pgfqpoint{1.967530in}{2.474412in}}%
\pgfpathlineto{\pgfqpoint{1.979107in}{2.477692in}}%
\pgfpathlineto{\pgfqpoint{1.990678in}{2.480979in}}%
\pgfpathlineto{\pgfqpoint{2.002244in}{2.484278in}}%
\pgfpathlineto{\pgfqpoint{2.013803in}{2.487591in}}%
\pgfpathlineto{\pgfqpoint{2.007956in}{2.495813in}}%
\pgfpathlineto{\pgfqpoint{2.002113in}{2.504017in}}%
\pgfpathlineto{\pgfqpoint{1.996275in}{2.512203in}}%
\pgfpathlineto{\pgfqpoint{1.990441in}{2.520372in}}%
\pgfpathlineto{\pgfqpoint{1.984611in}{2.528524in}}%
\pgfpathlineto{\pgfqpoint{1.973064in}{2.525223in}}%
\pgfpathlineto{\pgfqpoint{1.961511in}{2.521936in}}%
\pgfpathlineto{\pgfqpoint{1.949952in}{2.518661in}}%
\pgfpathlineto{\pgfqpoint{1.938387in}{2.515395in}}%
\pgfpathlineto{\pgfqpoint{1.926816in}{2.512136in}}%
\pgfpathlineto{\pgfqpoint{1.932634in}{2.503972in}}%
\pgfpathlineto{\pgfqpoint{1.938455in}{2.495792in}}%
\pgfpathlineto{\pgfqpoint{1.944282in}{2.487593in}}%
\pgfpathlineto{\pgfqpoint{1.950112in}{2.479375in}}%
\pgfpathclose%
\pgfusepath{stroke,fill}%
\end{pgfscope}%
\begin{pgfscope}%
\pgfpathrectangle{\pgfqpoint{0.887500in}{0.275000in}}{\pgfqpoint{4.225000in}{4.225000in}}%
\pgfusepath{clip}%
\pgfsetbuttcap%
\pgfsetroundjoin%
\definecolor{currentfill}{rgb}{0.440137,0.811138,0.340967}%
\pgfsetfillcolor{currentfill}%
\pgfsetfillopacity{0.700000}%
\pgfsetlinewidth{0.501875pt}%
\definecolor{currentstroke}{rgb}{1.000000,1.000000,1.000000}%
\pgfsetstrokecolor{currentstroke}%
\pgfsetstrokeopacity{0.500000}%
\pgfsetdash{}{0pt}%
\pgfpathmoveto{\pgfqpoint{3.592698in}{3.060891in}}%
\pgfpathlineto{\pgfqpoint{3.603876in}{3.064173in}}%
\pgfpathlineto{\pgfqpoint{3.615048in}{3.067409in}}%
\pgfpathlineto{\pgfqpoint{3.626213in}{3.070611in}}%
\pgfpathlineto{\pgfqpoint{3.637373in}{3.073795in}}%
\pgfpathlineto{\pgfqpoint{3.648528in}{3.076975in}}%
\pgfpathlineto{\pgfqpoint{3.642203in}{3.089630in}}%
\pgfpathlineto{\pgfqpoint{3.635880in}{3.102180in}}%
\pgfpathlineto{\pgfqpoint{3.629559in}{3.114615in}}%
\pgfpathlineto{\pgfqpoint{3.623238in}{3.126929in}}%
\pgfpathlineto{\pgfqpoint{3.616919in}{3.139111in}}%
\pgfpathlineto{\pgfqpoint{3.605771in}{3.136114in}}%
\pgfpathlineto{\pgfqpoint{3.594618in}{3.133108in}}%
\pgfpathlineto{\pgfqpoint{3.583459in}{3.130074in}}%
\pgfpathlineto{\pgfqpoint{3.572294in}{3.126994in}}%
\pgfpathlineto{\pgfqpoint{3.561122in}{3.123852in}}%
\pgfpathlineto{\pgfqpoint{3.567436in}{3.111632in}}%
\pgfpathlineto{\pgfqpoint{3.573751in}{3.099203in}}%
\pgfpathlineto{\pgfqpoint{3.580066in}{3.086588in}}%
\pgfpathlineto{\pgfqpoint{3.586382in}{3.073809in}}%
\pgfpathclose%
\pgfusepath{stroke,fill}%
\end{pgfscope}%
\begin{pgfscope}%
\pgfpathrectangle{\pgfqpoint{0.887500in}{0.275000in}}{\pgfqpoint{4.225000in}{4.225000in}}%
\pgfusepath{clip}%
\pgfsetbuttcap%
\pgfsetroundjoin%
\definecolor{currentfill}{rgb}{0.143343,0.522773,0.556295}%
\pgfsetfillcolor{currentfill}%
\pgfsetfillopacity{0.700000}%
\pgfsetlinewidth{0.501875pt}%
\definecolor{currentstroke}{rgb}{1.000000,1.000000,1.000000}%
\pgfsetstrokecolor{currentstroke}%
\pgfsetstrokeopacity{0.500000}%
\pgfsetdash{}{0pt}%
\pgfpathmoveto{\pgfqpoint{2.275257in}{2.412283in}}%
\pgfpathlineto{\pgfqpoint{2.286762in}{2.415561in}}%
\pgfpathlineto{\pgfqpoint{2.298261in}{2.418838in}}%
\pgfpathlineto{\pgfqpoint{2.309755in}{2.422121in}}%
\pgfpathlineto{\pgfqpoint{2.321243in}{2.425412in}}%
\pgfpathlineto{\pgfqpoint{2.332725in}{2.428717in}}%
\pgfpathlineto{\pgfqpoint{2.326767in}{2.437149in}}%
\pgfpathlineto{\pgfqpoint{2.320814in}{2.445565in}}%
\pgfpathlineto{\pgfqpoint{2.314865in}{2.453967in}}%
\pgfpathlineto{\pgfqpoint{2.308920in}{2.462353in}}%
\pgfpathlineto{\pgfqpoint{2.302979in}{2.470724in}}%
\pgfpathlineto{\pgfqpoint{2.291509in}{2.467418in}}%
\pgfpathlineto{\pgfqpoint{2.280033in}{2.464126in}}%
\pgfpathlineto{\pgfqpoint{2.268552in}{2.460844in}}%
\pgfpathlineto{\pgfqpoint{2.257064in}{2.457568in}}%
\pgfpathlineto{\pgfqpoint{2.245571in}{2.454295in}}%
\pgfpathlineto{\pgfqpoint{2.251500in}{2.445923in}}%
\pgfpathlineto{\pgfqpoint{2.257433in}{2.437536in}}%
\pgfpathlineto{\pgfqpoint{2.263370in}{2.429134in}}%
\pgfpathlineto{\pgfqpoint{2.269311in}{2.420716in}}%
\pgfpathclose%
\pgfusepath{stroke,fill}%
\end{pgfscope}%
\begin{pgfscope}%
\pgfpathrectangle{\pgfqpoint{0.887500in}{0.275000in}}{\pgfqpoint{4.225000in}{4.225000in}}%
\pgfusepath{clip}%
\pgfsetbuttcap%
\pgfsetroundjoin%
\definecolor{currentfill}{rgb}{0.154815,0.493313,0.557840}%
\pgfsetfillcolor{currentfill}%
\pgfsetfillopacity{0.700000}%
\pgfsetlinewidth{0.501875pt}%
\definecolor{currentstroke}{rgb}{1.000000,1.000000,1.000000}%
\pgfsetstrokecolor{currentstroke}%
\pgfsetstrokeopacity{0.500000}%
\pgfsetdash{}{0pt}%
\pgfpathmoveto{\pgfqpoint{2.594631in}{2.351131in}}%
\pgfpathlineto{\pgfqpoint{2.606062in}{2.354026in}}%
\pgfpathlineto{\pgfqpoint{2.617486in}{2.356993in}}%
\pgfpathlineto{\pgfqpoint{2.628902in}{2.360142in}}%
\pgfpathlineto{\pgfqpoint{2.640309in}{2.363586in}}%
\pgfpathlineto{\pgfqpoint{2.651707in}{2.367434in}}%
\pgfpathlineto{\pgfqpoint{2.645643in}{2.376062in}}%
\pgfpathlineto{\pgfqpoint{2.639582in}{2.384671in}}%
\pgfpathlineto{\pgfqpoint{2.633526in}{2.393264in}}%
\pgfpathlineto{\pgfqpoint{2.627474in}{2.401844in}}%
\pgfpathlineto{\pgfqpoint{2.621426in}{2.410414in}}%
\pgfpathlineto{\pgfqpoint{2.610040in}{2.406595in}}%
\pgfpathlineto{\pgfqpoint{2.598644in}{2.403173in}}%
\pgfpathlineto{\pgfqpoint{2.587239in}{2.400039in}}%
\pgfpathlineto{\pgfqpoint{2.575826in}{2.397082in}}%
\pgfpathlineto{\pgfqpoint{2.564406in}{2.394195in}}%
\pgfpathlineto{\pgfqpoint{2.570443in}{2.385613in}}%
\pgfpathlineto{\pgfqpoint{2.576484in}{2.377016in}}%
\pgfpathlineto{\pgfqpoint{2.582529in}{2.368404in}}%
\pgfpathlineto{\pgfqpoint{2.588578in}{2.359776in}}%
\pgfpathclose%
\pgfusepath{stroke,fill}%
\end{pgfscope}%
\begin{pgfscope}%
\pgfpathrectangle{\pgfqpoint{0.887500in}{0.275000in}}{\pgfqpoint{4.225000in}{4.225000in}}%
\pgfusepath{clip}%
\pgfsetbuttcap%
\pgfsetroundjoin%
\definecolor{currentfill}{rgb}{0.496615,0.826376,0.306377}%
\pgfsetfillcolor{currentfill}%
\pgfsetfillopacity{0.700000}%
\pgfsetlinewidth{0.501875pt}%
\definecolor{currentstroke}{rgb}{1.000000,1.000000,1.000000}%
\pgfsetstrokecolor{currentstroke}%
\pgfsetstrokeopacity{0.500000}%
\pgfsetdash{}{0pt}%
\pgfpathmoveto{\pgfqpoint{3.505173in}{3.107318in}}%
\pgfpathlineto{\pgfqpoint{3.516375in}{3.110719in}}%
\pgfpathlineto{\pgfqpoint{3.527571in}{3.114077in}}%
\pgfpathlineto{\pgfqpoint{3.538761in}{3.117389in}}%
\pgfpathlineto{\pgfqpoint{3.549944in}{3.120648in}}%
\pgfpathlineto{\pgfqpoint{3.561122in}{3.123852in}}%
\pgfpathlineto{\pgfqpoint{3.554808in}{3.135854in}}%
\pgfpathlineto{\pgfqpoint{3.548495in}{3.147658in}}%
\pgfpathlineto{\pgfqpoint{3.542183in}{3.159287in}}%
\pgfpathlineto{\pgfqpoint{3.535872in}{3.170764in}}%
\pgfpathlineto{\pgfqpoint{3.529562in}{3.182112in}}%
\pgfpathlineto{\pgfqpoint{3.518393in}{3.179131in}}%
\pgfpathlineto{\pgfqpoint{3.507218in}{3.176206in}}%
\pgfpathlineto{\pgfqpoint{3.496039in}{3.173315in}}%
\pgfpathlineto{\pgfqpoint{3.484854in}{3.170436in}}%
\pgfpathlineto{\pgfqpoint{3.473663in}{3.167550in}}%
\pgfpathlineto{\pgfqpoint{3.479962in}{3.155776in}}%
\pgfpathlineto{\pgfqpoint{3.486262in}{3.143867in}}%
\pgfpathlineto{\pgfqpoint{3.492565in}{3.131822in}}%
\pgfpathlineto{\pgfqpoint{3.498868in}{3.119640in}}%
\pgfpathclose%
\pgfusepath{stroke,fill}%
\end{pgfscope}%
\begin{pgfscope}%
\pgfpathrectangle{\pgfqpoint{0.887500in}{0.275000in}}{\pgfqpoint{4.225000in}{4.225000in}}%
\pgfusepath{clip}%
\pgfsetbuttcap%
\pgfsetroundjoin%
\definecolor{currentfill}{rgb}{0.606045,0.850733,0.236712}%
\pgfsetfillcolor{currentfill}%
\pgfsetfillopacity{0.700000}%
\pgfsetlinewidth{0.501875pt}%
\definecolor{currentstroke}{rgb}{1.000000,1.000000,1.000000}%
\pgfsetstrokecolor{currentstroke}%
\pgfsetstrokeopacity{0.500000}%
\pgfsetdash{}{0pt}%
\pgfpathmoveto{\pgfqpoint{3.330033in}{3.195129in}}%
\pgfpathlineto{\pgfqpoint{3.341279in}{3.198601in}}%
\pgfpathlineto{\pgfqpoint{3.352518in}{3.201945in}}%
\pgfpathlineto{\pgfqpoint{3.363750in}{3.205148in}}%
\pgfpathlineto{\pgfqpoint{3.374976in}{3.208213in}}%
\pgfpathlineto{\pgfqpoint{3.386196in}{3.211154in}}%
\pgfpathlineto{\pgfqpoint{3.379915in}{3.222186in}}%
\pgfpathlineto{\pgfqpoint{3.373635in}{3.232969in}}%
\pgfpathlineto{\pgfqpoint{3.367357in}{3.243539in}}%
\pgfpathlineto{\pgfqpoint{3.361082in}{3.253929in}}%
\pgfpathlineto{\pgfqpoint{3.354808in}{3.264175in}}%
\pgfpathlineto{\pgfqpoint{3.343595in}{3.261159in}}%
\pgfpathlineto{\pgfqpoint{3.332376in}{3.258085in}}%
\pgfpathlineto{\pgfqpoint{3.321152in}{3.254949in}}%
\pgfpathlineto{\pgfqpoint{3.309921in}{3.251751in}}%
\pgfpathlineto{\pgfqpoint{3.298685in}{3.248507in}}%
\pgfpathlineto{\pgfqpoint{3.304950in}{3.238246in}}%
\pgfpathlineto{\pgfqpoint{3.311218in}{3.227815in}}%
\pgfpathlineto{\pgfqpoint{3.317488in}{3.217177in}}%
\pgfpathlineto{\pgfqpoint{3.323760in}{3.206294in}}%
\pgfpathclose%
\pgfusepath{stroke,fill}%
\end{pgfscope}%
\begin{pgfscope}%
\pgfpathrectangle{\pgfqpoint{0.887500in}{0.275000in}}{\pgfqpoint{4.225000in}{4.225000in}}%
\pgfusepath{clip}%
\pgfsetbuttcap%
\pgfsetroundjoin%
\definecolor{currentfill}{rgb}{0.157851,0.683765,0.501686}%
\pgfsetfillcolor{currentfill}%
\pgfsetfillopacity{0.700000}%
\pgfsetlinewidth{0.501875pt}%
\definecolor{currentstroke}{rgb}{1.000000,1.000000,1.000000}%
\pgfsetstrokecolor{currentstroke}%
\pgfsetstrokeopacity{0.500000}%
\pgfsetdash{}{0pt}%
\pgfpathmoveto{\pgfqpoint{4.061588in}{2.742995in}}%
\pgfpathlineto{\pgfqpoint{4.072653in}{2.746397in}}%
\pgfpathlineto{\pgfqpoint{4.083712in}{2.749807in}}%
\pgfpathlineto{\pgfqpoint{4.094767in}{2.753224in}}%
\pgfpathlineto{\pgfqpoint{4.105816in}{2.756650in}}%
\pgfpathlineto{\pgfqpoint{4.116861in}{2.760085in}}%
\pgfpathlineto{\pgfqpoint{4.110466in}{2.774041in}}%
\pgfpathlineto{\pgfqpoint{4.104074in}{2.787983in}}%
\pgfpathlineto{\pgfqpoint{4.097684in}{2.801912in}}%
\pgfpathlineto{\pgfqpoint{4.091295in}{2.815828in}}%
\pgfpathlineto{\pgfqpoint{4.084910in}{2.829729in}}%
\pgfpathlineto{\pgfqpoint{4.073866in}{2.826233in}}%
\pgfpathlineto{\pgfqpoint{4.062817in}{2.822747in}}%
\pgfpathlineto{\pgfqpoint{4.051763in}{2.819274in}}%
\pgfpathlineto{\pgfqpoint{4.040705in}{2.815810in}}%
\pgfpathlineto{\pgfqpoint{4.029641in}{2.812357in}}%
\pgfpathlineto{\pgfqpoint{4.036027in}{2.798559in}}%
\pgfpathlineto{\pgfqpoint{4.042415in}{2.784725in}}%
\pgfpathlineto{\pgfqpoint{4.048804in}{2.770850in}}%
\pgfpathlineto{\pgfqpoint{4.055195in}{2.756939in}}%
\pgfpathclose%
\pgfusepath{stroke,fill}%
\end{pgfscope}%
\begin{pgfscope}%
\pgfpathrectangle{\pgfqpoint{0.887500in}{0.275000in}}{\pgfqpoint{4.225000in}{4.225000in}}%
\pgfusepath{clip}%
\pgfsetbuttcap%
\pgfsetroundjoin%
\definecolor{currentfill}{rgb}{0.555484,0.840254,0.269281}%
\pgfsetfillcolor{currentfill}%
\pgfsetfillopacity{0.700000}%
\pgfsetlinewidth{0.501875pt}%
\definecolor{currentstroke}{rgb}{1.000000,1.000000,1.000000}%
\pgfsetstrokecolor{currentstroke}%
\pgfsetstrokeopacity{0.500000}%
\pgfsetdash{}{0pt}%
\pgfpathmoveto{\pgfqpoint{3.417618in}{3.152418in}}%
\pgfpathlineto{\pgfqpoint{3.428839in}{3.155572in}}%
\pgfpathlineto{\pgfqpoint{3.440054in}{3.158654in}}%
\pgfpathlineto{\pgfqpoint{3.451263in}{3.161673in}}%
\pgfpathlineto{\pgfqpoint{3.462466in}{3.164635in}}%
\pgfpathlineto{\pgfqpoint{3.473663in}{3.167550in}}%
\pgfpathlineto{\pgfqpoint{3.467366in}{3.179192in}}%
\pgfpathlineto{\pgfqpoint{3.461070in}{3.190704in}}%
\pgfpathlineto{\pgfqpoint{3.454777in}{3.202086in}}%
\pgfpathlineto{\pgfqpoint{3.448485in}{3.213341in}}%
\pgfpathlineto{\pgfqpoint{3.442196in}{3.224470in}}%
\pgfpathlineto{\pgfqpoint{3.431008in}{3.221945in}}%
\pgfpathlineto{\pgfqpoint{3.419815in}{3.219367in}}%
\pgfpathlineto{\pgfqpoint{3.408615in}{3.216718in}}%
\pgfpathlineto{\pgfqpoint{3.397409in}{3.213985in}}%
\pgfpathlineto{\pgfqpoint{3.386196in}{3.211154in}}%
\pgfpathlineto{\pgfqpoint{3.392478in}{3.199856in}}%
\pgfpathlineto{\pgfqpoint{3.398761in}{3.188310in}}%
\pgfpathlineto{\pgfqpoint{3.405045in}{3.176539in}}%
\pgfpathlineto{\pgfqpoint{3.411331in}{3.164568in}}%
\pgfpathclose%
\pgfusepath{stroke,fill}%
\end{pgfscope}%
\begin{pgfscope}%
\pgfpathrectangle{\pgfqpoint{0.887500in}{0.275000in}}{\pgfqpoint{4.225000in}{4.225000in}}%
\pgfusepath{clip}%
\pgfsetbuttcap%
\pgfsetroundjoin%
\definecolor{currentfill}{rgb}{0.124395,0.578002,0.548287}%
\pgfsetfillcolor{currentfill}%
\pgfsetfillopacity{0.700000}%
\pgfsetlinewidth{0.501875pt}%
\definecolor{currentstroke}{rgb}{1.000000,1.000000,1.000000}%
\pgfsetstrokecolor{currentstroke}%
\pgfsetstrokeopacity{0.500000}%
\pgfsetdash{}{0pt}%
\pgfpathmoveto{\pgfqpoint{4.355391in}{2.511047in}}%
\pgfpathlineto{\pgfqpoint{4.366381in}{2.514386in}}%
\pgfpathlineto{\pgfqpoint{4.377366in}{2.517733in}}%
\pgfpathlineto{\pgfqpoint{4.388346in}{2.521089in}}%
\pgfpathlineto{\pgfqpoint{4.399321in}{2.524454in}}%
\pgfpathlineto{\pgfqpoint{4.410292in}{2.527829in}}%
\pgfpathlineto{\pgfqpoint{4.403852in}{2.542141in}}%
\pgfpathlineto{\pgfqpoint{4.397415in}{2.556461in}}%
\pgfpathlineto{\pgfqpoint{4.390981in}{2.570783in}}%
\pgfpathlineto{\pgfqpoint{4.384549in}{2.585103in}}%
\pgfpathlineto{\pgfqpoint{4.378120in}{2.599418in}}%
\pgfpathlineto{\pgfqpoint{4.367153in}{2.596090in}}%
\pgfpathlineto{\pgfqpoint{4.356180in}{2.592760in}}%
\pgfpathlineto{\pgfqpoint{4.345202in}{2.589426in}}%
\pgfpathlineto{\pgfqpoint{4.334219in}{2.586089in}}%
\pgfpathlineto{\pgfqpoint{4.323230in}{2.582746in}}%
\pgfpathlineto{\pgfqpoint{4.329658in}{2.568439in}}%
\pgfpathlineto{\pgfqpoint{4.336088in}{2.554114in}}%
\pgfpathlineto{\pgfqpoint{4.342521in}{2.539771in}}%
\pgfpathlineto{\pgfqpoint{4.348955in}{2.525413in}}%
\pgfpathclose%
\pgfusepath{stroke,fill}%
\end{pgfscope}%
\begin{pgfscope}%
\pgfpathrectangle{\pgfqpoint{0.887500in}{0.275000in}}{\pgfqpoint{4.225000in}{4.225000in}}%
\pgfusepath{clip}%
\pgfsetbuttcap%
\pgfsetroundjoin%
\definecolor{currentfill}{rgb}{0.144759,0.519093,0.556572}%
\pgfsetfillcolor{currentfill}%
\pgfsetfillopacity{0.700000}%
\pgfsetlinewidth{0.501875pt}%
\definecolor{currentstroke}{rgb}{1.000000,1.000000,1.000000}%
\pgfsetstrokecolor{currentstroke}%
\pgfsetstrokeopacity{0.500000}%
\pgfsetdash{}{0pt}%
\pgfpathmoveto{\pgfqpoint{2.883222in}{2.341987in}}%
\pgfpathlineto{\pgfqpoint{2.894483in}{2.364056in}}%
\pgfpathlineto{\pgfqpoint{2.905734in}{2.390167in}}%
\pgfpathlineto{\pgfqpoint{2.916982in}{2.419218in}}%
\pgfpathlineto{\pgfqpoint{2.928234in}{2.450106in}}%
\pgfpathlineto{\pgfqpoint{2.939493in}{2.481724in}}%
\pgfpathlineto{\pgfqpoint{2.933316in}{2.493248in}}%
\pgfpathlineto{\pgfqpoint{2.927140in}{2.505281in}}%
\pgfpathlineto{\pgfqpoint{2.920965in}{2.517644in}}%
\pgfpathlineto{\pgfqpoint{2.914793in}{2.530156in}}%
\pgfpathlineto{\pgfqpoint{2.908624in}{2.542637in}}%
\pgfpathlineto{\pgfqpoint{2.897410in}{2.506721in}}%
\pgfpathlineto{\pgfqpoint{2.886206in}{2.471438in}}%
\pgfpathlineto{\pgfqpoint{2.875005in}{2.438167in}}%
\pgfpathlineto{\pgfqpoint{2.863800in}{2.408278in}}%
\pgfpathlineto{\pgfqpoint{2.852579in}{2.383137in}}%
\pgfpathlineto{\pgfqpoint{2.858701in}{2.374674in}}%
\pgfpathlineto{\pgfqpoint{2.864825in}{2.366426in}}%
\pgfpathlineto{\pgfqpoint{2.870953in}{2.358295in}}%
\pgfpathlineto{\pgfqpoint{2.877085in}{2.350182in}}%
\pgfpathclose%
\pgfusepath{stroke,fill}%
\end{pgfscope}%
\begin{pgfscope}%
\pgfpathrectangle{\pgfqpoint{0.887500in}{0.275000in}}{\pgfqpoint{4.225000in}{4.225000in}}%
\pgfusepath{clip}%
\pgfsetbuttcap%
\pgfsetroundjoin%
\definecolor{currentfill}{rgb}{0.127568,0.566949,0.550556}%
\pgfsetfillcolor{currentfill}%
\pgfsetfillopacity{0.700000}%
\pgfsetlinewidth{0.501875pt}%
\definecolor{currentstroke}{rgb}{1.000000,1.000000,1.000000}%
\pgfsetstrokecolor{currentstroke}%
\pgfsetstrokeopacity{0.500000}%
\pgfsetdash{}{0pt}%
\pgfpathmoveto{\pgfqpoint{1.723732in}{2.503631in}}%
\pgfpathlineto{\pgfqpoint{1.735375in}{2.506947in}}%
\pgfpathlineto{\pgfqpoint{1.747012in}{2.510261in}}%
\pgfpathlineto{\pgfqpoint{1.758644in}{2.513570in}}%
\pgfpathlineto{\pgfqpoint{1.770270in}{2.516873in}}%
\pgfpathlineto{\pgfqpoint{1.781891in}{2.520167in}}%
\pgfpathlineto{\pgfqpoint{1.776123in}{2.528259in}}%
\pgfpathlineto{\pgfqpoint{1.770360in}{2.536335in}}%
\pgfpathlineto{\pgfqpoint{1.764601in}{2.544397in}}%
\pgfpathlineto{\pgfqpoint{1.758847in}{2.552444in}}%
\pgfpathlineto{\pgfqpoint{1.753096in}{2.560475in}}%
\pgfpathlineto{\pgfqpoint{1.741488in}{2.557197in}}%
\pgfpathlineto{\pgfqpoint{1.729874in}{2.553912in}}%
\pgfpathlineto{\pgfqpoint{1.718255in}{2.550621in}}%
\pgfpathlineto{\pgfqpoint{1.706631in}{2.547326in}}%
\pgfpathlineto{\pgfqpoint{1.695000in}{2.544029in}}%
\pgfpathlineto{\pgfqpoint{1.700738in}{2.535980in}}%
\pgfpathlineto{\pgfqpoint{1.706480in}{2.527915in}}%
\pgfpathlineto{\pgfqpoint{1.712227in}{2.519835in}}%
\pgfpathlineto{\pgfqpoint{1.717977in}{2.511741in}}%
\pgfpathclose%
\pgfusepath{stroke,fill}%
\end{pgfscope}%
\begin{pgfscope}%
\pgfpathrectangle{\pgfqpoint{0.887500in}{0.275000in}}{\pgfqpoint{4.225000in}{4.225000in}}%
\pgfusepath{clip}%
\pgfsetbuttcap%
\pgfsetroundjoin%
\definecolor{currentfill}{rgb}{0.191090,0.708366,0.482284}%
\pgfsetfillcolor{currentfill}%
\pgfsetfillopacity{0.700000}%
\pgfsetlinewidth{0.501875pt}%
\definecolor{currentstroke}{rgb}{1.000000,1.000000,1.000000}%
\pgfsetstrokecolor{currentstroke}%
\pgfsetstrokeopacity{0.500000}%
\pgfsetdash{}{0pt}%
\pgfpathmoveto{\pgfqpoint{3.974245in}{2.795187in}}%
\pgfpathlineto{\pgfqpoint{3.985335in}{2.798612in}}%
\pgfpathlineto{\pgfqpoint{3.996419in}{2.802040in}}%
\pgfpathlineto{\pgfqpoint{4.007498in}{2.805473in}}%
\pgfpathlineto{\pgfqpoint{4.018572in}{2.808911in}}%
\pgfpathlineto{\pgfqpoint{4.029641in}{2.812357in}}%
\pgfpathlineto{\pgfqpoint{4.023257in}{2.826123in}}%
\pgfpathlineto{\pgfqpoint{4.016875in}{2.839864in}}%
\pgfpathlineto{\pgfqpoint{4.010494in}{2.853587in}}%
\pgfpathlineto{\pgfqpoint{4.004117in}{2.867297in}}%
\pgfpathlineto{\pgfqpoint{3.997741in}{2.881000in}}%
\pgfpathlineto{\pgfqpoint{3.986672in}{2.877442in}}%
\pgfpathlineto{\pgfqpoint{3.975599in}{2.873901in}}%
\pgfpathlineto{\pgfqpoint{3.964520in}{2.870381in}}%
\pgfpathlineto{\pgfqpoint{3.953437in}{2.866883in}}%
\pgfpathlineto{\pgfqpoint{3.942349in}{2.863405in}}%
\pgfpathlineto{\pgfqpoint{3.948724in}{2.849785in}}%
\pgfpathlineto{\pgfqpoint{3.955101in}{2.836163in}}%
\pgfpathlineto{\pgfqpoint{3.961480in}{2.822530in}}%
\pgfpathlineto{\pgfqpoint{3.967862in}{2.808874in}}%
\pgfpathclose%
\pgfusepath{stroke,fill}%
\end{pgfscope}%
\begin{pgfscope}%
\pgfpathrectangle{\pgfqpoint{0.887500in}{0.275000in}}{\pgfqpoint{4.225000in}{4.225000in}}%
\pgfusepath{clip}%
\pgfsetbuttcap%
\pgfsetroundjoin%
\definecolor{currentfill}{rgb}{0.136408,0.541173,0.554483}%
\pgfsetfillcolor{currentfill}%
\pgfsetfillopacity{0.700000}%
\pgfsetlinewidth{0.501875pt}%
\definecolor{currentstroke}{rgb}{1.000000,1.000000,1.000000}%
\pgfsetstrokecolor{currentstroke}%
\pgfsetstrokeopacity{0.500000}%
\pgfsetdash{}{0pt}%
\pgfpathmoveto{\pgfqpoint{2.043101in}{2.446202in}}%
\pgfpathlineto{\pgfqpoint{2.054666in}{2.449541in}}%
\pgfpathlineto{\pgfqpoint{2.066225in}{2.452888in}}%
\pgfpathlineto{\pgfqpoint{2.077779in}{2.456239in}}%
\pgfpathlineto{\pgfqpoint{2.089327in}{2.459590in}}%
\pgfpathlineto{\pgfqpoint{2.100869in}{2.462937in}}%
\pgfpathlineto{\pgfqpoint{2.094989in}{2.471237in}}%
\pgfpathlineto{\pgfqpoint{2.089113in}{2.479520in}}%
\pgfpathlineto{\pgfqpoint{2.083241in}{2.487785in}}%
\pgfpathlineto{\pgfqpoint{2.077373in}{2.496032in}}%
\pgfpathlineto{\pgfqpoint{2.071510in}{2.504261in}}%
\pgfpathlineto{\pgfqpoint{2.059980in}{2.500925in}}%
\pgfpathlineto{\pgfqpoint{2.048444in}{2.497586in}}%
\pgfpathlineto{\pgfqpoint{2.036903in}{2.494249in}}%
\pgfpathlineto{\pgfqpoint{2.025356in}{2.490916in}}%
\pgfpathlineto{\pgfqpoint{2.013803in}{2.487591in}}%
\pgfpathlineto{\pgfqpoint{2.019654in}{2.479350in}}%
\pgfpathlineto{\pgfqpoint{2.025509in}{2.471091in}}%
\pgfpathlineto{\pgfqpoint{2.031369in}{2.462814in}}%
\pgfpathlineto{\pgfqpoint{2.037233in}{2.454517in}}%
\pgfpathclose%
\pgfusepath{stroke,fill}%
\end{pgfscope}%
\begin{pgfscope}%
\pgfpathrectangle{\pgfqpoint{0.887500in}{0.275000in}}{\pgfqpoint{4.225000in}{4.225000in}}%
\pgfusepath{clip}%
\pgfsetbuttcap%
\pgfsetroundjoin%
\definecolor{currentfill}{rgb}{0.147607,0.511733,0.557049}%
\pgfsetfillcolor{currentfill}%
\pgfsetfillopacity{0.700000}%
\pgfsetlinewidth{0.501875pt}%
\definecolor{currentstroke}{rgb}{1.000000,1.000000,1.000000}%
\pgfsetstrokecolor{currentstroke}%
\pgfsetstrokeopacity{0.500000}%
\pgfsetdash{}{0pt}%
\pgfpathmoveto{\pgfqpoint{2.362573in}{2.386326in}}%
\pgfpathlineto{\pgfqpoint{2.374060in}{2.389646in}}%
\pgfpathlineto{\pgfqpoint{2.385542in}{2.392984in}}%
\pgfpathlineto{\pgfqpoint{2.397018in}{2.396347in}}%
\pgfpathlineto{\pgfqpoint{2.408487in}{2.399739in}}%
\pgfpathlineto{\pgfqpoint{2.419950in}{2.403165in}}%
\pgfpathlineto{\pgfqpoint{2.413961in}{2.411672in}}%
\pgfpathlineto{\pgfqpoint{2.407975in}{2.420165in}}%
\pgfpathlineto{\pgfqpoint{2.401994in}{2.428644in}}%
\pgfpathlineto{\pgfqpoint{2.396016in}{2.437109in}}%
\pgfpathlineto{\pgfqpoint{2.390043in}{2.445560in}}%
\pgfpathlineto{\pgfqpoint{2.378591in}{2.442139in}}%
\pgfpathlineto{\pgfqpoint{2.367134in}{2.438747in}}%
\pgfpathlineto{\pgfqpoint{2.355670in}{2.435381in}}%
\pgfpathlineto{\pgfqpoint{2.344200in}{2.432039in}}%
\pgfpathlineto{\pgfqpoint{2.332725in}{2.428717in}}%
\pgfpathlineto{\pgfqpoint{2.338686in}{2.420270in}}%
\pgfpathlineto{\pgfqpoint{2.344652in}{2.411807in}}%
\pgfpathlineto{\pgfqpoint{2.350621in}{2.403329in}}%
\pgfpathlineto{\pgfqpoint{2.356595in}{2.394835in}}%
\pgfpathclose%
\pgfusepath{stroke,fill}%
\end{pgfscope}%
\begin{pgfscope}%
\pgfpathrectangle{\pgfqpoint{0.887500in}{0.275000in}}{\pgfqpoint{4.225000in}{4.225000in}}%
\pgfusepath{clip}%
\pgfsetbuttcap%
\pgfsetroundjoin%
\definecolor{currentfill}{rgb}{0.119738,0.603785,0.541400}%
\pgfsetfillcolor{currentfill}%
\pgfsetfillopacity{0.700000}%
\pgfsetlinewidth{0.501875pt}%
\definecolor{currentstroke}{rgb}{1.000000,1.000000,1.000000}%
\pgfsetstrokecolor{currentstroke}%
\pgfsetstrokeopacity{0.500000}%
\pgfsetdash{}{0pt}%
\pgfpathmoveto{\pgfqpoint{4.268201in}{2.565903in}}%
\pgfpathlineto{\pgfqpoint{4.279218in}{2.569293in}}%
\pgfpathlineto{\pgfqpoint{4.290230in}{2.572670in}}%
\pgfpathlineto{\pgfqpoint{4.301235in}{2.576038in}}%
\pgfpathlineto{\pgfqpoint{4.312235in}{2.579396in}}%
\pgfpathlineto{\pgfqpoint{4.323230in}{2.582746in}}%
\pgfpathlineto{\pgfqpoint{4.316803in}{2.597032in}}%
\pgfpathlineto{\pgfqpoint{4.310379in}{2.611297in}}%
\pgfpathlineto{\pgfqpoint{4.303956in}{2.625539in}}%
\pgfpathlineto{\pgfqpoint{4.297535in}{2.639757in}}%
\pgfpathlineto{\pgfqpoint{4.291115in}{2.653950in}}%
\pgfpathlineto{\pgfqpoint{4.280121in}{2.650559in}}%
\pgfpathlineto{\pgfqpoint{4.269121in}{2.647160in}}%
\pgfpathlineto{\pgfqpoint{4.258116in}{2.643751in}}%
\pgfpathlineto{\pgfqpoint{4.247105in}{2.640332in}}%
\pgfpathlineto{\pgfqpoint{4.236088in}{2.636900in}}%
\pgfpathlineto{\pgfqpoint{4.242507in}{2.622743in}}%
\pgfpathlineto{\pgfqpoint{4.248928in}{2.608563in}}%
\pgfpathlineto{\pgfqpoint{4.255350in}{2.594363in}}%
\pgfpathlineto{\pgfqpoint{4.261775in}{2.580143in}}%
\pgfpathclose%
\pgfusepath{stroke,fill}%
\end{pgfscope}%
\begin{pgfscope}%
\pgfpathrectangle{\pgfqpoint{0.887500in}{0.275000in}}{\pgfqpoint{4.225000in}{4.225000in}}%
\pgfusepath{clip}%
\pgfsetbuttcap%
\pgfsetroundjoin%
\definecolor{currentfill}{rgb}{0.157729,0.485932,0.558013}%
\pgfsetfillcolor{currentfill}%
\pgfsetfillopacity{0.700000}%
\pgfsetlinewidth{0.501875pt}%
\definecolor{currentstroke}{rgb}{1.000000,1.000000,1.000000}%
\pgfsetstrokecolor{currentstroke}%
\pgfsetstrokeopacity{0.500000}%
\pgfsetdash{}{0pt}%
\pgfpathmoveto{\pgfqpoint{2.682090in}{2.323953in}}%
\pgfpathlineto{\pgfqpoint{2.693490in}{2.328219in}}%
\pgfpathlineto{\pgfqpoint{2.704882in}{2.332807in}}%
\pgfpathlineto{\pgfqpoint{2.716269in}{2.337360in}}%
\pgfpathlineto{\pgfqpoint{2.727655in}{2.341510in}}%
\pgfpathlineto{\pgfqpoint{2.739041in}{2.344892in}}%
\pgfpathlineto{\pgfqpoint{2.732944in}{2.353721in}}%
\pgfpathlineto{\pgfqpoint{2.726851in}{2.362559in}}%
\pgfpathlineto{\pgfqpoint{2.720761in}{2.371410in}}%
\pgfpathlineto{\pgfqpoint{2.714675in}{2.380278in}}%
\pgfpathlineto{\pgfqpoint{2.708593in}{2.389169in}}%
\pgfpathlineto{\pgfqpoint{2.697219in}{2.385570in}}%
\pgfpathlineto{\pgfqpoint{2.685846in}{2.381224in}}%
\pgfpathlineto{\pgfqpoint{2.674473in}{2.376505in}}%
\pgfpathlineto{\pgfqpoint{2.663094in}{2.371786in}}%
\pgfpathlineto{\pgfqpoint{2.651707in}{2.367434in}}%
\pgfpathlineto{\pgfqpoint{2.657775in}{2.358783in}}%
\pgfpathlineto{\pgfqpoint{2.663848in}{2.350110in}}%
\pgfpathlineto{\pgfqpoint{2.669924in}{2.341414in}}%
\pgfpathlineto{\pgfqpoint{2.676005in}{2.332695in}}%
\pgfpathclose%
\pgfusepath{stroke,fill}%
\end{pgfscope}%
\begin{pgfscope}%
\pgfpathrectangle{\pgfqpoint{0.887500in}{0.275000in}}{\pgfqpoint{4.225000in}{4.225000in}}%
\pgfusepath{clip}%
\pgfsetbuttcap%
\pgfsetroundjoin%
\definecolor{currentfill}{rgb}{0.232815,0.732247,0.459277}%
\pgfsetfillcolor{currentfill}%
\pgfsetfillopacity{0.700000}%
\pgfsetlinewidth{0.501875pt}%
\definecolor{currentstroke}{rgb}{1.000000,1.000000,1.000000}%
\pgfsetstrokecolor{currentstroke}%
\pgfsetstrokeopacity{0.500000}%
\pgfsetdash{}{0pt}%
\pgfpathmoveto{\pgfqpoint{3.886833in}{2.846165in}}%
\pgfpathlineto{\pgfqpoint{3.897947in}{2.849610in}}%
\pgfpathlineto{\pgfqpoint{3.909055in}{2.853052in}}%
\pgfpathlineto{\pgfqpoint{3.920158in}{2.856495in}}%
\pgfpathlineto{\pgfqpoint{3.931256in}{2.859944in}}%
\pgfpathlineto{\pgfqpoint{3.942349in}{2.863405in}}%
\pgfpathlineto{\pgfqpoint{3.935977in}{2.877033in}}%
\pgfpathlineto{\pgfqpoint{3.929609in}{2.890679in}}%
\pgfpathlineto{\pgfqpoint{3.923243in}{2.904352in}}%
\pgfpathlineto{\pgfqpoint{3.916881in}{2.918062in}}%
\pgfpathlineto{\pgfqpoint{3.910522in}{2.931809in}}%
\pgfpathlineto{\pgfqpoint{3.899431in}{2.928314in}}%
\pgfpathlineto{\pgfqpoint{3.888336in}{2.924833in}}%
\pgfpathlineto{\pgfqpoint{3.877235in}{2.921362in}}%
\pgfpathlineto{\pgfqpoint{3.866129in}{2.917894in}}%
\pgfpathlineto{\pgfqpoint{3.855017in}{2.914422in}}%
\pgfpathlineto{\pgfqpoint{3.861374in}{2.900720in}}%
\pgfpathlineto{\pgfqpoint{3.867734in}{2.887046in}}%
\pgfpathlineto{\pgfqpoint{3.874098in}{2.873401in}}%
\pgfpathlineto{\pgfqpoint{3.880464in}{2.859777in}}%
\pgfpathclose%
\pgfusepath{stroke,fill}%
\end{pgfscope}%
\begin{pgfscope}%
\pgfpathrectangle{\pgfqpoint{0.887500in}{0.275000in}}{\pgfqpoint{4.225000in}{4.225000in}}%
\pgfusepath{clip}%
\pgfsetbuttcap%
\pgfsetroundjoin%
\definecolor{currentfill}{rgb}{0.123463,0.581687,0.547445}%
\pgfsetfillcolor{currentfill}%
\pgfsetfillopacity{0.700000}%
\pgfsetlinewidth{0.501875pt}%
\definecolor{currentstroke}{rgb}{1.000000,1.000000,1.000000}%
\pgfsetstrokecolor{currentstroke}%
\pgfsetstrokeopacity{0.500000}%
\pgfsetdash{}{0pt}%
\pgfpathmoveto{\pgfqpoint{1.491430in}{2.534355in}}%
\pgfpathlineto{\pgfqpoint{1.503132in}{2.537723in}}%
\pgfpathlineto{\pgfqpoint{1.514828in}{2.541077in}}%
\pgfpathlineto{\pgfqpoint{1.526518in}{2.544420in}}%
\pgfpathlineto{\pgfqpoint{1.538204in}{2.547752in}}%
\pgfpathlineto{\pgfqpoint{1.549884in}{2.551074in}}%
\pgfpathlineto{\pgfqpoint{1.544198in}{2.559021in}}%
\pgfpathlineto{\pgfqpoint{1.538517in}{2.566940in}}%
\pgfpathlineto{\pgfqpoint{1.532841in}{2.574830in}}%
\pgfpathlineto{\pgfqpoint{1.527170in}{2.582689in}}%
\pgfpathlineto{\pgfqpoint{1.521504in}{2.590520in}}%
\pgfpathlineto{\pgfqpoint{1.509837in}{2.587202in}}%
\pgfpathlineto{\pgfqpoint{1.498165in}{2.583875in}}%
\pgfpathlineto{\pgfqpoint{1.486488in}{2.580538in}}%
\pgfpathlineto{\pgfqpoint{1.474806in}{2.577189in}}%
\pgfpathlineto{\pgfqpoint{1.463118in}{2.573828in}}%
\pgfpathlineto{\pgfqpoint{1.468771in}{2.565996in}}%
\pgfpathlineto{\pgfqpoint{1.474428in}{2.558133in}}%
\pgfpathlineto{\pgfqpoint{1.480091in}{2.550238in}}%
\pgfpathlineto{\pgfqpoint{1.485758in}{2.542311in}}%
\pgfpathclose%
\pgfusepath{stroke,fill}%
\end{pgfscope}%
\begin{pgfscope}%
\pgfpathrectangle{\pgfqpoint{0.887500in}{0.275000in}}{\pgfqpoint{4.225000in}{4.225000in}}%
\pgfusepath{clip}%
\pgfsetbuttcap%
\pgfsetroundjoin%
\definecolor{currentfill}{rgb}{0.163625,0.471133,0.558148}%
\pgfsetfillcolor{currentfill}%
\pgfsetfillopacity{0.700000}%
\pgfsetlinewidth{0.501875pt}%
\definecolor{currentstroke}{rgb}{1.000000,1.000000,1.000000}%
\pgfsetstrokecolor{currentstroke}%
\pgfsetstrokeopacity{0.500000}%
\pgfsetdash{}{0pt}%
\pgfpathmoveto{\pgfqpoint{2.769586in}{2.300702in}}%
\pgfpathlineto{\pgfqpoint{2.780983in}{2.303357in}}%
\pgfpathlineto{\pgfqpoint{2.792382in}{2.304801in}}%
\pgfpathlineto{\pgfqpoint{2.803784in}{2.304796in}}%
\pgfpathlineto{\pgfqpoint{2.815184in}{2.303913in}}%
\pgfpathlineto{\pgfqpoint{2.826574in}{2.303252in}}%
\pgfpathlineto{\pgfqpoint{2.820454in}{2.311262in}}%
\pgfpathlineto{\pgfqpoint{2.814339in}{2.319123in}}%
\pgfpathlineto{\pgfqpoint{2.808228in}{2.326932in}}%
\pgfpathlineto{\pgfqpoint{2.802121in}{2.334787in}}%
\pgfpathlineto{\pgfqpoint{2.796017in}{2.342786in}}%
\pgfpathlineto{\pgfqpoint{2.784627in}{2.344702in}}%
\pgfpathlineto{\pgfqpoint{2.773227in}{2.346844in}}%
\pgfpathlineto{\pgfqpoint{2.761826in}{2.347884in}}%
\pgfpathlineto{\pgfqpoint{2.750431in}{2.347139in}}%
\pgfpathlineto{\pgfqpoint{2.739041in}{2.344892in}}%
\pgfpathlineto{\pgfqpoint{2.745142in}{2.336066in}}%
\pgfpathlineto{\pgfqpoint{2.751247in}{2.327239in}}%
\pgfpathlineto{\pgfqpoint{2.757356in}{2.318406in}}%
\pgfpathlineto{\pgfqpoint{2.763469in}{2.309562in}}%
\pgfpathclose%
\pgfusepath{stroke,fill}%
\end{pgfscope}%
\begin{pgfscope}%
\pgfpathrectangle{\pgfqpoint{0.887500in}{0.275000in}}{\pgfqpoint{4.225000in}{4.225000in}}%
\pgfusepath{clip}%
\pgfsetbuttcap%
\pgfsetroundjoin%
\definecolor{currentfill}{rgb}{0.311925,0.767822,0.415586}%
\pgfsetfillcolor{currentfill}%
\pgfsetfillopacity{0.700000}%
\pgfsetlinewidth{0.501875pt}%
\definecolor{currentstroke}{rgb}{1.000000,1.000000,1.000000}%
\pgfsetstrokecolor{currentstroke}%
\pgfsetstrokeopacity{0.500000}%
\pgfsetdash{}{0pt}%
\pgfpathmoveto{\pgfqpoint{2.959624in}{2.909142in}}%
\pgfpathlineto{\pgfqpoint{2.970904in}{2.936913in}}%
\pgfpathlineto{\pgfqpoint{2.982195in}{2.964049in}}%
\pgfpathlineto{\pgfqpoint{2.993495in}{2.990117in}}%
\pgfpathlineto{\pgfqpoint{3.004804in}{3.014683in}}%
\pgfpathlineto{\pgfqpoint{3.016122in}{3.037312in}}%
\pgfpathlineto{\pgfqpoint{3.009931in}{3.039057in}}%
\pgfpathlineto{\pgfqpoint{3.003746in}{3.040272in}}%
\pgfpathlineto{\pgfqpoint{2.997569in}{3.041091in}}%
\pgfpathlineto{\pgfqpoint{2.991398in}{3.041649in}}%
\pgfpathlineto{\pgfqpoint{2.985234in}{3.042082in}}%
\pgfpathlineto{\pgfqpoint{2.973962in}{3.014591in}}%
\pgfpathlineto{\pgfqpoint{2.962702in}{2.986027in}}%
\pgfpathlineto{\pgfqpoint{2.951452in}{2.956788in}}%
\pgfpathlineto{\pgfqpoint{2.940211in}{2.927271in}}%
\pgfpathlineto{\pgfqpoint{2.928980in}{2.897870in}}%
\pgfpathlineto{\pgfqpoint{2.935094in}{2.900211in}}%
\pgfpathlineto{\pgfqpoint{2.941215in}{2.902786in}}%
\pgfpathlineto{\pgfqpoint{2.947343in}{2.905318in}}%
\pgfpathlineto{\pgfqpoint{2.953479in}{2.907529in}}%
\pgfpathclose%
\pgfusepath{stroke,fill}%
\end{pgfscope}%
\begin{pgfscope}%
\pgfpathrectangle{\pgfqpoint{0.887500in}{0.275000in}}{\pgfqpoint{4.225000in}{4.225000in}}%
\pgfusepath{clip}%
\pgfsetbuttcap%
\pgfsetroundjoin%
\definecolor{currentfill}{rgb}{0.122312,0.633153,0.530398}%
\pgfsetfillcolor{currentfill}%
\pgfsetfillopacity{0.700000}%
\pgfsetlinewidth{0.501875pt}%
\definecolor{currentstroke}{rgb}{1.000000,1.000000,1.000000}%
\pgfsetstrokecolor{currentstroke}%
\pgfsetstrokeopacity{0.500000}%
\pgfsetdash{}{0pt}%
\pgfpathmoveto{\pgfqpoint{4.180923in}{2.619634in}}%
\pgfpathlineto{\pgfqpoint{4.191966in}{2.623090in}}%
\pgfpathlineto{\pgfqpoint{4.203005in}{2.626548in}}%
\pgfpathlineto{\pgfqpoint{4.214038in}{2.630005in}}%
\pgfpathlineto{\pgfqpoint{4.225066in}{2.633457in}}%
\pgfpathlineto{\pgfqpoint{4.236088in}{2.636900in}}%
\pgfpathlineto{\pgfqpoint{4.229671in}{2.651035in}}%
\pgfpathlineto{\pgfqpoint{4.223256in}{2.665146in}}%
\pgfpathlineto{\pgfqpoint{4.216843in}{2.679232in}}%
\pgfpathlineto{\pgfqpoint{4.210431in}{2.693293in}}%
\pgfpathlineto{\pgfqpoint{4.204022in}{2.707334in}}%
\pgfpathlineto{\pgfqpoint{4.193001in}{2.703885in}}%
\pgfpathlineto{\pgfqpoint{4.181975in}{2.700433in}}%
\pgfpathlineto{\pgfqpoint{4.170944in}{2.696978in}}%
\pgfpathlineto{\pgfqpoint{4.159907in}{2.693524in}}%
\pgfpathlineto{\pgfqpoint{4.148866in}{2.690072in}}%
\pgfpathlineto{\pgfqpoint{4.155273in}{2.676019in}}%
\pgfpathlineto{\pgfqpoint{4.161682in}{2.661949in}}%
\pgfpathlineto{\pgfqpoint{4.168094in}{2.647860in}}%
\pgfpathlineto{\pgfqpoint{4.174507in}{2.633755in}}%
\pgfpathclose%
\pgfusepath{stroke,fill}%
\end{pgfscope}%
\begin{pgfscope}%
\pgfpathrectangle{\pgfqpoint{0.887500in}{0.275000in}}{\pgfqpoint{4.225000in}{4.225000in}}%
\pgfusepath{clip}%
\pgfsetbuttcap%
\pgfsetroundjoin%
\definecolor{currentfill}{rgb}{0.129933,0.559582,0.551864}%
\pgfsetfillcolor{currentfill}%
\pgfsetfillopacity{0.700000}%
\pgfsetlinewidth{0.501875pt}%
\definecolor{currentstroke}{rgb}{1.000000,1.000000,1.000000}%
\pgfsetstrokecolor{currentstroke}%
\pgfsetstrokeopacity{0.500000}%
\pgfsetdash{}{0pt}%
\pgfpathmoveto{\pgfqpoint{1.810791in}{2.479455in}}%
\pgfpathlineto{\pgfqpoint{1.822419in}{2.482756in}}%
\pgfpathlineto{\pgfqpoint{1.834041in}{2.486046in}}%
\pgfpathlineto{\pgfqpoint{1.845657in}{2.489325in}}%
\pgfpathlineto{\pgfqpoint{1.857268in}{2.492597in}}%
\pgfpathlineto{\pgfqpoint{1.868874in}{2.495861in}}%
\pgfpathlineto{\pgfqpoint{1.863073in}{2.504021in}}%
\pgfpathlineto{\pgfqpoint{1.857276in}{2.512164in}}%
\pgfpathlineto{\pgfqpoint{1.851484in}{2.520290in}}%
\pgfpathlineto{\pgfqpoint{1.845695in}{2.528400in}}%
\pgfpathlineto{\pgfqpoint{1.839911in}{2.536494in}}%
\pgfpathlineto{\pgfqpoint{1.828318in}{2.533244in}}%
\pgfpathlineto{\pgfqpoint{1.816720in}{2.529988in}}%
\pgfpathlineto{\pgfqpoint{1.805115in}{2.526724in}}%
\pgfpathlineto{\pgfqpoint{1.793506in}{2.523451in}}%
\pgfpathlineto{\pgfqpoint{1.781891in}{2.520167in}}%
\pgfpathlineto{\pgfqpoint{1.787662in}{2.512059in}}%
\pgfpathlineto{\pgfqpoint{1.793438in}{2.503935in}}%
\pgfpathlineto{\pgfqpoint{1.799218in}{2.495793in}}%
\pgfpathlineto{\pgfqpoint{1.805002in}{2.487633in}}%
\pgfpathclose%
\pgfusepath{stroke,fill}%
\end{pgfscope}%
\begin{pgfscope}%
\pgfpathrectangle{\pgfqpoint{0.887500in}{0.275000in}}{\pgfqpoint{4.225000in}{4.225000in}}%
\pgfusepath{clip}%
\pgfsetbuttcap%
\pgfsetroundjoin%
\definecolor{currentfill}{rgb}{0.281477,0.755203,0.432552}%
\pgfsetfillcolor{currentfill}%
\pgfsetfillopacity{0.700000}%
\pgfsetlinewidth{0.501875pt}%
\definecolor{currentstroke}{rgb}{1.000000,1.000000,1.000000}%
\pgfsetstrokecolor{currentstroke}%
\pgfsetstrokeopacity{0.500000}%
\pgfsetdash{}{0pt}%
\pgfpathmoveto{\pgfqpoint{3.799378in}{2.896950in}}%
\pgfpathlineto{\pgfqpoint{3.810517in}{2.900455in}}%
\pgfpathlineto{\pgfqpoint{3.821650in}{2.903956in}}%
\pgfpathlineto{\pgfqpoint{3.832778in}{2.907453in}}%
\pgfpathlineto{\pgfqpoint{3.843900in}{2.910942in}}%
\pgfpathlineto{\pgfqpoint{3.855017in}{2.914422in}}%
\pgfpathlineto{\pgfqpoint{3.848663in}{2.928135in}}%
\pgfpathlineto{\pgfqpoint{3.842311in}{2.941834in}}%
\pgfpathlineto{\pgfqpoint{3.835961in}{2.955500in}}%
\pgfpathlineto{\pgfqpoint{3.829613in}{2.969110in}}%
\pgfpathlineto{\pgfqpoint{3.823265in}{2.982641in}}%
\pgfpathlineto{\pgfqpoint{3.812150in}{2.979082in}}%
\pgfpathlineto{\pgfqpoint{3.801029in}{2.975503in}}%
\pgfpathlineto{\pgfqpoint{3.789902in}{2.971906in}}%
\pgfpathlineto{\pgfqpoint{3.778770in}{2.968294in}}%
\pgfpathlineto{\pgfqpoint{3.767632in}{2.964667in}}%
\pgfpathlineto{\pgfqpoint{3.773978in}{2.951224in}}%
\pgfpathlineto{\pgfqpoint{3.780325in}{2.937717in}}%
\pgfpathlineto{\pgfqpoint{3.786674in}{2.924159in}}%
\pgfpathlineto{\pgfqpoint{3.793025in}{2.910566in}}%
\pgfpathclose%
\pgfusepath{stroke,fill}%
\end{pgfscope}%
\begin{pgfscope}%
\pgfpathrectangle{\pgfqpoint{0.887500in}{0.275000in}}{\pgfqpoint{4.225000in}{4.225000in}}%
\pgfusepath{clip}%
\pgfsetbuttcap%
\pgfsetroundjoin%
\definecolor{currentfill}{rgb}{0.688944,0.865448,0.182725}%
\pgfsetfillcolor{currentfill}%
\pgfsetfillopacity{0.700000}%
\pgfsetlinewidth{0.501875pt}%
\definecolor{currentstroke}{rgb}{1.000000,1.000000,1.000000}%
\pgfsetstrokecolor{currentstroke}%
\pgfsetstrokeopacity{0.500000}%
\pgfsetdash{}{0pt}%
\pgfpathmoveto{\pgfqpoint{3.098308in}{3.232857in}}%
\pgfpathlineto{\pgfqpoint{3.109623in}{3.243129in}}%
\pgfpathlineto{\pgfqpoint{3.120934in}{3.251962in}}%
\pgfpathlineto{\pgfqpoint{3.132239in}{3.259423in}}%
\pgfpathlineto{\pgfqpoint{3.143539in}{3.265583in}}%
\pgfpathlineto{\pgfqpoint{3.154833in}{3.270577in}}%
\pgfpathlineto{\pgfqpoint{3.148606in}{3.281746in}}%
\pgfpathlineto{\pgfqpoint{3.142381in}{3.292758in}}%
\pgfpathlineto{\pgfqpoint{3.136160in}{3.303602in}}%
\pgfpathlineto{\pgfqpoint{3.129942in}{3.314268in}}%
\pgfpathlineto{\pgfqpoint{3.123727in}{3.324746in}}%
\pgfpathlineto{\pgfqpoint{3.112442in}{3.319739in}}%
\pgfpathlineto{\pgfqpoint{3.101152in}{3.313323in}}%
\pgfpathlineto{\pgfqpoint{3.089856in}{3.305280in}}%
\pgfpathlineto{\pgfqpoint{3.078557in}{3.295427in}}%
\pgfpathlineto{\pgfqpoint{3.067255in}{3.283583in}}%
\pgfpathlineto{\pgfqpoint{3.073459in}{3.273879in}}%
\pgfpathlineto{\pgfqpoint{3.079666in}{3.263952in}}%
\pgfpathlineto{\pgfqpoint{3.085877in}{3.253804in}}%
\pgfpathlineto{\pgfqpoint{3.092091in}{3.243438in}}%
\pgfpathclose%
\pgfusepath{stroke,fill}%
\end{pgfscope}%
\begin{pgfscope}%
\pgfpathrectangle{\pgfqpoint{0.887500in}{0.275000in}}{\pgfqpoint{4.225000in}{4.225000in}}%
\pgfusepath{clip}%
\pgfsetbuttcap%
\pgfsetroundjoin%
\definecolor{currentfill}{rgb}{0.139147,0.533812,0.555298}%
\pgfsetfillcolor{currentfill}%
\pgfsetfillopacity{0.700000}%
\pgfsetlinewidth{0.501875pt}%
\definecolor{currentstroke}{rgb}{1.000000,1.000000,1.000000}%
\pgfsetstrokecolor{currentstroke}%
\pgfsetstrokeopacity{0.500000}%
\pgfsetdash{}{0pt}%
\pgfpathmoveto{\pgfqpoint{2.130333in}{2.421170in}}%
\pgfpathlineto{\pgfqpoint{2.141881in}{2.424528in}}%
\pgfpathlineto{\pgfqpoint{2.153424in}{2.427876in}}%
\pgfpathlineto{\pgfqpoint{2.164962in}{2.431212in}}%
\pgfpathlineto{\pgfqpoint{2.176494in}{2.434538in}}%
\pgfpathlineto{\pgfqpoint{2.188021in}{2.437853in}}%
\pgfpathlineto{\pgfqpoint{2.182108in}{2.446221in}}%
\pgfpathlineto{\pgfqpoint{2.176199in}{2.454573in}}%
\pgfpathlineto{\pgfqpoint{2.170294in}{2.462909in}}%
\pgfpathlineto{\pgfqpoint{2.164394in}{2.471229in}}%
\pgfpathlineto{\pgfqpoint{2.158497in}{2.479533in}}%
\pgfpathlineto{\pgfqpoint{2.146983in}{2.476233in}}%
\pgfpathlineto{\pgfqpoint{2.135462in}{2.472925in}}%
\pgfpathlineto{\pgfqpoint{2.123937in}{2.469606in}}%
\pgfpathlineto{\pgfqpoint{2.112405in}{2.466277in}}%
\pgfpathlineto{\pgfqpoint{2.100869in}{2.462937in}}%
\pgfpathlineto{\pgfqpoint{2.106753in}{2.454619in}}%
\pgfpathlineto{\pgfqpoint{2.112642in}{2.446284in}}%
\pgfpathlineto{\pgfqpoint{2.118535in}{2.437930in}}%
\pgfpathlineto{\pgfqpoint{2.124432in}{2.429559in}}%
\pgfpathclose%
\pgfusepath{stroke,fill}%
\end{pgfscope}%
\begin{pgfscope}%
\pgfpathrectangle{\pgfqpoint{0.887500in}{0.275000in}}{\pgfqpoint{4.225000in}{4.225000in}}%
\pgfusepath{clip}%
\pgfsetbuttcap%
\pgfsetroundjoin%
\definecolor{currentfill}{rgb}{0.150476,0.504369,0.557430}%
\pgfsetfillcolor{currentfill}%
\pgfsetfillopacity{0.700000}%
\pgfsetlinewidth{0.501875pt}%
\definecolor{currentstroke}{rgb}{1.000000,1.000000,1.000000}%
\pgfsetstrokecolor{currentstroke}%
\pgfsetstrokeopacity{0.500000}%
\pgfsetdash{}{0pt}%
\pgfpathmoveto{\pgfqpoint{2.449959in}{2.360418in}}%
\pgfpathlineto{\pgfqpoint{2.461428in}{2.363895in}}%
\pgfpathlineto{\pgfqpoint{2.472890in}{2.367420in}}%
\pgfpathlineto{\pgfqpoint{2.484346in}{2.370983in}}%
\pgfpathlineto{\pgfqpoint{2.495797in}{2.374553in}}%
\pgfpathlineto{\pgfqpoint{2.507242in}{2.378094in}}%
\pgfpathlineto{\pgfqpoint{2.501221in}{2.386658in}}%
\pgfpathlineto{\pgfqpoint{2.495203in}{2.395204in}}%
\pgfpathlineto{\pgfqpoint{2.489190in}{2.403734in}}%
\pgfpathlineto{\pgfqpoint{2.483181in}{2.412249in}}%
\pgfpathlineto{\pgfqpoint{2.477176in}{2.420748in}}%
\pgfpathlineto{\pgfqpoint{2.465742in}{2.417225in}}%
\pgfpathlineto{\pgfqpoint{2.454303in}{2.413677in}}%
\pgfpathlineto{\pgfqpoint{2.442858in}{2.410135in}}%
\pgfpathlineto{\pgfqpoint{2.431407in}{2.406628in}}%
\pgfpathlineto{\pgfqpoint{2.419950in}{2.403165in}}%
\pgfpathlineto{\pgfqpoint{2.425944in}{2.394645in}}%
\pgfpathlineto{\pgfqpoint{2.431942in}{2.386110in}}%
\pgfpathlineto{\pgfqpoint{2.437943in}{2.377561in}}%
\pgfpathlineto{\pgfqpoint{2.443949in}{2.368997in}}%
\pgfpathclose%
\pgfusepath{stroke,fill}%
\end{pgfscope}%
\begin{pgfscope}%
\pgfpathrectangle{\pgfqpoint{0.887500in}{0.275000in}}{\pgfqpoint{4.225000in}{4.225000in}}%
\pgfusepath{clip}%
\pgfsetbuttcap%
\pgfsetroundjoin%
\definecolor{currentfill}{rgb}{0.143343,0.522773,0.556295}%
\pgfsetfillcolor{currentfill}%
\pgfsetfillopacity{0.700000}%
\pgfsetlinewidth{0.501875pt}%
\definecolor{currentstroke}{rgb}{1.000000,1.000000,1.000000}%
\pgfsetstrokecolor{currentstroke}%
\pgfsetstrokeopacity{0.500000}%
\pgfsetdash{}{0pt}%
\pgfpathmoveto{\pgfqpoint{4.474878in}{2.386199in}}%
\pgfpathlineto{\pgfqpoint{4.485854in}{2.389789in}}%
\pgfpathlineto{\pgfqpoint{4.496825in}{2.393393in}}%
\pgfpathlineto{\pgfqpoint{4.507791in}{2.397015in}}%
\pgfpathlineto{\pgfqpoint{4.518753in}{2.400657in}}%
\pgfpathlineto{\pgfqpoint{4.512275in}{2.414633in}}%
\pgfpathlineto{\pgfqpoint{4.505799in}{2.428625in}}%
\pgfpathlineto{\pgfqpoint{4.499328in}{2.442641in}}%
\pgfpathlineto{\pgfqpoint{4.492859in}{2.456682in}}%
\pgfpathlineto{\pgfqpoint{4.486395in}{2.470748in}}%
\pgfpathlineto{\pgfqpoint{4.475437in}{2.467163in}}%
\pgfpathlineto{\pgfqpoint{4.464474in}{2.463603in}}%
\pgfpathlineto{\pgfqpoint{4.453508in}{2.460067in}}%
\pgfpathlineto{\pgfqpoint{4.442537in}{2.456554in}}%
\pgfpathlineto{\pgfqpoint{4.448996in}{2.442392in}}%
\pgfpathlineto{\pgfqpoint{4.455460in}{2.428272in}}%
\pgfpathlineto{\pgfqpoint{4.461929in}{2.414200in}}%
\pgfpathlineto{\pgfqpoint{4.468401in}{2.400179in}}%
\pgfpathclose%
\pgfusepath{stroke,fill}%
\end{pgfscope}%
\begin{pgfscope}%
\pgfpathrectangle{\pgfqpoint{0.887500in}{0.275000in}}{\pgfqpoint{4.225000in}{4.225000in}}%
\pgfusepath{clip}%
\pgfsetbuttcap%
\pgfsetroundjoin%
\definecolor{currentfill}{rgb}{0.606045,0.850733,0.236712}%
\pgfsetfillcolor{currentfill}%
\pgfsetfillopacity{0.700000}%
\pgfsetlinewidth{0.501875pt}%
\definecolor{currentstroke}{rgb}{1.000000,1.000000,1.000000}%
\pgfsetstrokecolor{currentstroke}%
\pgfsetstrokeopacity{0.500000}%
\pgfsetdash{}{0pt}%
\pgfpathmoveto{\pgfqpoint{3.041721in}{3.157681in}}%
\pgfpathlineto{\pgfqpoint{3.053036in}{3.176058in}}%
\pgfpathlineto{\pgfqpoint{3.064354in}{3.192743in}}%
\pgfpathlineto{\pgfqpoint{3.075672in}{3.207730in}}%
\pgfpathlineto{\pgfqpoint{3.086991in}{3.221079in}}%
\pgfpathlineto{\pgfqpoint{3.098308in}{3.232857in}}%
\pgfpathlineto{\pgfqpoint{3.092091in}{3.243438in}}%
\pgfpathlineto{\pgfqpoint{3.085877in}{3.253804in}}%
\pgfpathlineto{\pgfqpoint{3.079666in}{3.263952in}}%
\pgfpathlineto{\pgfqpoint{3.073459in}{3.273879in}}%
\pgfpathlineto{\pgfqpoint{3.067255in}{3.283583in}}%
\pgfpathlineto{\pgfqpoint{3.055953in}{3.269568in}}%
\pgfpathlineto{\pgfqpoint{3.044652in}{3.253203in}}%
\pgfpathlineto{\pgfqpoint{3.033354in}{3.234310in}}%
\pgfpathlineto{\pgfqpoint{3.022062in}{3.212738in}}%
\pgfpathlineto{\pgfqpoint{3.010777in}{3.188668in}}%
\pgfpathlineto{\pgfqpoint{3.016956in}{3.182842in}}%
\pgfpathlineto{\pgfqpoint{3.023140in}{3.176820in}}%
\pgfpathlineto{\pgfqpoint{3.029329in}{3.170612in}}%
\pgfpathlineto{\pgfqpoint{3.035523in}{3.164230in}}%
\pgfpathclose%
\pgfusepath{stroke,fill}%
\end{pgfscope}%
\begin{pgfscope}%
\pgfpathrectangle{\pgfqpoint{0.887500in}{0.275000in}}{\pgfqpoint{4.225000in}{4.225000in}}%
\pgfusepath{clip}%
\pgfsetbuttcap%
\pgfsetroundjoin%
\definecolor{currentfill}{rgb}{0.335885,0.777018,0.402049}%
\pgfsetfillcolor{currentfill}%
\pgfsetfillopacity{0.700000}%
\pgfsetlinewidth{0.501875pt}%
\definecolor{currentstroke}{rgb}{1.000000,1.000000,1.000000}%
\pgfsetstrokecolor{currentstroke}%
\pgfsetstrokeopacity{0.500000}%
\pgfsetdash{}{0pt}%
\pgfpathmoveto{\pgfqpoint{3.711864in}{2.946433in}}%
\pgfpathlineto{\pgfqpoint{3.723028in}{2.950075in}}%
\pgfpathlineto{\pgfqpoint{3.734187in}{2.953728in}}%
\pgfpathlineto{\pgfqpoint{3.745341in}{2.957382in}}%
\pgfpathlineto{\pgfqpoint{3.756489in}{2.961029in}}%
\pgfpathlineto{\pgfqpoint{3.767632in}{2.964667in}}%
\pgfpathlineto{\pgfqpoint{3.761288in}{2.978032in}}%
\pgfpathlineto{\pgfqpoint{3.754944in}{2.991305in}}%
\pgfpathlineto{\pgfqpoint{3.748602in}{3.004471in}}%
\pgfpathlineto{\pgfqpoint{3.742260in}{3.017516in}}%
\pgfpathlineto{\pgfqpoint{3.735918in}{3.030429in}}%
\pgfpathlineto{\pgfqpoint{3.724779in}{3.026811in}}%
\pgfpathlineto{\pgfqpoint{3.713635in}{3.023196in}}%
\pgfpathlineto{\pgfqpoint{3.702486in}{3.019589in}}%
\pgfpathlineto{\pgfqpoint{3.691332in}{3.015996in}}%
\pgfpathlineto{\pgfqpoint{3.680173in}{3.012421in}}%
\pgfpathlineto{\pgfqpoint{3.686507in}{2.999314in}}%
\pgfpathlineto{\pgfqpoint{3.692843in}{2.986159in}}%
\pgfpathlineto{\pgfqpoint{3.699182in}{2.972959in}}%
\pgfpathlineto{\pgfqpoint{3.705522in}{2.959717in}}%
\pgfpathclose%
\pgfusepath{stroke,fill}%
\end{pgfscope}%
\begin{pgfscope}%
\pgfpathrectangle{\pgfqpoint{0.887500in}{0.275000in}}{\pgfqpoint{4.225000in}{4.225000in}}%
\pgfusepath{clip}%
\pgfsetbuttcap%
\pgfsetroundjoin%
\definecolor{currentfill}{rgb}{0.134692,0.658636,0.517649}%
\pgfsetfillcolor{currentfill}%
\pgfsetfillopacity{0.700000}%
\pgfsetlinewidth{0.501875pt}%
\definecolor{currentstroke}{rgb}{1.000000,1.000000,1.000000}%
\pgfsetstrokecolor{currentstroke}%
\pgfsetstrokeopacity{0.500000}%
\pgfsetdash{}{0pt}%
\pgfpathmoveto{\pgfqpoint{4.093580in}{2.672898in}}%
\pgfpathlineto{\pgfqpoint{4.104648in}{2.676317in}}%
\pgfpathlineto{\pgfqpoint{4.115710in}{2.679745in}}%
\pgfpathlineto{\pgfqpoint{4.126767in}{2.683180in}}%
\pgfpathlineto{\pgfqpoint{4.137819in}{2.686623in}}%
\pgfpathlineto{\pgfqpoint{4.148866in}{2.690072in}}%
\pgfpathlineto{\pgfqpoint{4.142461in}{2.704107in}}%
\pgfpathlineto{\pgfqpoint{4.136057in}{2.718125in}}%
\pgfpathlineto{\pgfqpoint{4.129656in}{2.732127in}}%
\pgfpathlineto{\pgfqpoint{4.123258in}{2.746113in}}%
\pgfpathlineto{\pgfqpoint{4.116861in}{2.760085in}}%
\pgfpathlineto{\pgfqpoint{4.105816in}{2.756650in}}%
\pgfpathlineto{\pgfqpoint{4.094767in}{2.753224in}}%
\pgfpathlineto{\pgfqpoint{4.083712in}{2.749807in}}%
\pgfpathlineto{\pgfqpoint{4.072653in}{2.746397in}}%
\pgfpathlineto{\pgfqpoint{4.061588in}{2.742995in}}%
\pgfpathlineto{\pgfqpoint{4.067983in}{2.729020in}}%
\pgfpathlineto{\pgfqpoint{4.074379in}{2.715020in}}%
\pgfpathlineto{\pgfqpoint{4.080777in}{2.700997in}}%
\pgfpathlineto{\pgfqpoint{4.087178in}{2.686955in}}%
\pgfpathclose%
\pgfusepath{stroke,fill}%
\end{pgfscope}%
\begin{pgfscope}%
\pgfpathrectangle{\pgfqpoint{0.887500in}{0.275000in}}{\pgfqpoint{4.225000in}{4.225000in}}%
\pgfusepath{clip}%
\pgfsetbuttcap%
\pgfsetroundjoin%
\definecolor{currentfill}{rgb}{0.133743,0.548535,0.553541}%
\pgfsetfillcolor{currentfill}%
\pgfsetfillopacity{0.700000}%
\pgfsetlinewidth{0.501875pt}%
\definecolor{currentstroke}{rgb}{1.000000,1.000000,1.000000}%
\pgfsetstrokecolor{currentstroke}%
\pgfsetstrokeopacity{0.500000}%
\pgfsetdash{}{0pt}%
\pgfpathmoveto{\pgfqpoint{4.387614in}{2.439329in}}%
\pgfpathlineto{\pgfqpoint{4.398608in}{2.442731in}}%
\pgfpathlineto{\pgfqpoint{4.409597in}{2.446153in}}%
\pgfpathlineto{\pgfqpoint{4.420581in}{2.449598in}}%
\pgfpathlineto{\pgfqpoint{4.431561in}{2.453065in}}%
\pgfpathlineto{\pgfqpoint{4.442537in}{2.456554in}}%
\pgfpathlineto{\pgfqpoint{4.436081in}{2.470754in}}%
\pgfpathlineto{\pgfqpoint{4.429628in}{2.484987in}}%
\pgfpathlineto{\pgfqpoint{4.423179in}{2.499247in}}%
\pgfpathlineto{\pgfqpoint{4.416734in}{2.513529in}}%
\pgfpathlineto{\pgfqpoint{4.410292in}{2.527829in}}%
\pgfpathlineto{\pgfqpoint{4.399321in}{2.524454in}}%
\pgfpathlineto{\pgfqpoint{4.388346in}{2.521089in}}%
\pgfpathlineto{\pgfqpoint{4.377366in}{2.517733in}}%
\pgfpathlineto{\pgfqpoint{4.366381in}{2.514386in}}%
\pgfpathlineto{\pgfqpoint{4.355391in}{2.511047in}}%
\pgfpathlineto{\pgfqpoint{4.361830in}{2.496681in}}%
\pgfpathlineto{\pgfqpoint{4.368271in}{2.482319in}}%
\pgfpathlineto{\pgfqpoint{4.374716in}{2.467968in}}%
\pgfpathlineto{\pgfqpoint{4.381163in}{2.453636in}}%
\pgfpathclose%
\pgfusepath{stroke,fill}%
\end{pgfscope}%
\begin{pgfscope}%
\pgfpathrectangle{\pgfqpoint{0.887500in}{0.275000in}}{\pgfqpoint{4.225000in}{4.225000in}}%
\pgfusepath{clip}%
\pgfsetbuttcap%
\pgfsetroundjoin%
\definecolor{currentfill}{rgb}{0.125394,0.574318,0.549086}%
\pgfsetfillcolor{currentfill}%
\pgfsetfillopacity{0.700000}%
\pgfsetlinewidth{0.501875pt}%
\definecolor{currentstroke}{rgb}{1.000000,1.000000,1.000000}%
\pgfsetstrokecolor{currentstroke}%
\pgfsetstrokeopacity{0.500000}%
\pgfsetdash{}{0pt}%
\pgfpathmoveto{\pgfqpoint{1.578380in}{2.510973in}}%
\pgfpathlineto{\pgfqpoint{1.590067in}{2.514298in}}%
\pgfpathlineto{\pgfqpoint{1.601749in}{2.517617in}}%
\pgfpathlineto{\pgfqpoint{1.613426in}{2.520929in}}%
\pgfpathlineto{\pgfqpoint{1.625096in}{2.524236in}}%
\pgfpathlineto{\pgfqpoint{1.636761in}{2.527538in}}%
\pgfpathlineto{\pgfqpoint{1.631040in}{2.535587in}}%
\pgfpathlineto{\pgfqpoint{1.625323in}{2.543617in}}%
\pgfpathlineto{\pgfqpoint{1.619611in}{2.551625in}}%
\pgfpathlineto{\pgfqpoint{1.613903in}{2.559612in}}%
\pgfpathlineto{\pgfqpoint{1.608200in}{2.567575in}}%
\pgfpathlineto{\pgfqpoint{1.596548in}{2.564286in}}%
\pgfpathlineto{\pgfqpoint{1.584890in}{2.560993in}}%
\pgfpathlineto{\pgfqpoint{1.573227in}{2.557693in}}%
\pgfpathlineto{\pgfqpoint{1.561558in}{2.554387in}}%
\pgfpathlineto{\pgfqpoint{1.549884in}{2.551074in}}%
\pgfpathlineto{\pgfqpoint{1.555574in}{2.543100in}}%
\pgfpathlineto{\pgfqpoint{1.561269in}{2.535102in}}%
\pgfpathlineto{\pgfqpoint{1.566968in}{2.527080in}}%
\pgfpathlineto{\pgfqpoint{1.572672in}{2.519036in}}%
\pgfpathclose%
\pgfusepath{stroke,fill}%
\end{pgfscope}%
\begin{pgfscope}%
\pgfpathrectangle{\pgfqpoint{0.887500in}{0.275000in}}{\pgfqpoint{4.225000in}{4.225000in}}%
\pgfusepath{clip}%
\pgfsetbuttcap%
\pgfsetroundjoin%
\definecolor{currentfill}{rgb}{0.386433,0.794644,0.372886}%
\pgfsetfillcolor{currentfill}%
\pgfsetfillopacity{0.700000}%
\pgfsetlinewidth{0.501875pt}%
\definecolor{currentstroke}{rgb}{1.000000,1.000000,1.000000}%
\pgfsetstrokecolor{currentstroke}%
\pgfsetstrokeopacity{0.500000}%
\pgfsetdash{}{0pt}%
\pgfpathmoveto{\pgfqpoint{3.624305in}{2.995009in}}%
\pgfpathlineto{\pgfqpoint{3.635488in}{2.998414in}}%
\pgfpathlineto{\pgfqpoint{3.646666in}{3.001861in}}%
\pgfpathlineto{\pgfqpoint{3.657840in}{3.005348in}}%
\pgfpathlineto{\pgfqpoint{3.669009in}{3.008870in}}%
\pgfpathlineto{\pgfqpoint{3.680173in}{3.012421in}}%
\pgfpathlineto{\pgfqpoint{3.673840in}{3.025474in}}%
\pgfpathlineto{\pgfqpoint{3.667510in}{3.038464in}}%
\pgfpathlineto{\pgfqpoint{3.661181in}{3.051384in}}%
\pgfpathlineto{\pgfqpoint{3.654854in}{3.064224in}}%
\pgfpathlineto{\pgfqpoint{3.648528in}{3.076975in}}%
\pgfpathlineto{\pgfqpoint{3.637373in}{3.073795in}}%
\pgfpathlineto{\pgfqpoint{3.626213in}{3.070611in}}%
\pgfpathlineto{\pgfqpoint{3.615048in}{3.067409in}}%
\pgfpathlineto{\pgfqpoint{3.603876in}{3.064173in}}%
\pgfpathlineto{\pgfqpoint{3.592698in}{3.060891in}}%
\pgfpathlineto{\pgfqpoint{3.599016in}{3.047855in}}%
\pgfpathlineto{\pgfqpoint{3.605336in}{3.034726in}}%
\pgfpathlineto{\pgfqpoint{3.611657in}{3.021527in}}%
\pgfpathlineto{\pgfqpoint{3.617980in}{3.008280in}}%
\pgfpathclose%
\pgfusepath{stroke,fill}%
\end{pgfscope}%
\begin{pgfscope}%
\pgfpathrectangle{\pgfqpoint{0.887500in}{0.275000in}}{\pgfqpoint{4.225000in}{4.225000in}}%
\pgfusepath{clip}%
\pgfsetbuttcap%
\pgfsetroundjoin%
\definecolor{currentfill}{rgb}{0.168126,0.459988,0.558082}%
\pgfsetfillcolor{currentfill}%
\pgfsetfillopacity{0.700000}%
\pgfsetlinewidth{0.501875pt}%
\definecolor{currentstroke}{rgb}{1.000000,1.000000,1.000000}%
\pgfsetstrokecolor{currentstroke}%
\pgfsetstrokeopacity{0.500000}%
\pgfsetdash{}{0pt}%
\pgfpathmoveto{\pgfqpoint{2.857255in}{2.259868in}}%
\pgfpathlineto{\pgfqpoint{2.868638in}{2.260825in}}%
\pgfpathlineto{\pgfqpoint{2.880003in}{2.264121in}}%
\pgfpathlineto{\pgfqpoint{2.891348in}{2.270812in}}%
\pgfpathlineto{\pgfqpoint{2.902672in}{2.281954in}}%
\pgfpathlineto{\pgfqpoint{2.913975in}{2.298432in}}%
\pgfpathlineto{\pgfqpoint{2.907816in}{2.307447in}}%
\pgfpathlineto{\pgfqpoint{2.901660in}{2.316330in}}%
\pgfpathlineto{\pgfqpoint{2.895509in}{2.325060in}}%
\pgfpathlineto{\pgfqpoint{2.889363in}{2.333620in}}%
\pgfpathlineto{\pgfqpoint{2.883222in}{2.341987in}}%
\pgfpathlineto{\pgfqpoint{2.871942in}{2.325054in}}%
\pgfpathlineto{\pgfqpoint{2.860636in}{2.313717in}}%
\pgfpathlineto{\pgfqpoint{2.849304in}{2.307045in}}%
\pgfpathlineto{\pgfqpoint{2.837949in}{2.303926in}}%
\pgfpathlineto{\pgfqpoint{2.826574in}{2.303252in}}%
\pgfpathlineto{\pgfqpoint{2.832700in}{2.295005in}}%
\pgfpathlineto{\pgfqpoint{2.838831in}{2.286511in}}%
\pgfpathlineto{\pgfqpoint{2.844968in}{2.277801in}}%
\pgfpathlineto{\pgfqpoint{2.851109in}{2.268909in}}%
\pgfpathclose%
\pgfusepath{stroke,fill}%
\end{pgfscope}%
\begin{pgfscope}%
\pgfpathrectangle{\pgfqpoint{0.887500in}{0.275000in}}{\pgfqpoint{4.225000in}{4.225000in}}%
\pgfusepath{clip}%
\pgfsetbuttcap%
\pgfsetroundjoin%
\definecolor{currentfill}{rgb}{0.647257,0.858400,0.209861}%
\pgfsetfillcolor{currentfill}%
\pgfsetfillopacity{0.700000}%
\pgfsetlinewidth{0.501875pt}%
\definecolor{currentstroke}{rgb}{1.000000,1.000000,1.000000}%
\pgfsetstrokecolor{currentstroke}%
\pgfsetstrokeopacity{0.500000}%
\pgfsetdash{}{0pt}%
\pgfpathmoveto{\pgfqpoint{3.186010in}{3.212971in}}%
\pgfpathlineto{\pgfqpoint{3.197305in}{3.217656in}}%
\pgfpathlineto{\pgfqpoint{3.208594in}{3.221776in}}%
\pgfpathlineto{\pgfqpoint{3.219875in}{3.225471in}}%
\pgfpathlineto{\pgfqpoint{3.231151in}{3.228879in}}%
\pgfpathlineto{\pgfqpoint{3.242420in}{3.232138in}}%
\pgfpathlineto{\pgfqpoint{3.236168in}{3.242914in}}%
\pgfpathlineto{\pgfqpoint{3.229919in}{3.253605in}}%
\pgfpathlineto{\pgfqpoint{3.223674in}{3.264228in}}%
\pgfpathlineto{\pgfqpoint{3.217431in}{3.274801in}}%
\pgfpathlineto{\pgfqpoint{3.211192in}{3.285340in}}%
\pgfpathlineto{\pgfqpoint{3.199934in}{3.283051in}}%
\pgfpathlineto{\pgfqpoint{3.188669in}{3.280640in}}%
\pgfpathlineto{\pgfqpoint{3.177398in}{3.277896in}}%
\pgfpathlineto{\pgfqpoint{3.166119in}{3.274612in}}%
\pgfpathlineto{\pgfqpoint{3.154833in}{3.270577in}}%
\pgfpathlineto{\pgfqpoint{3.161062in}{3.259267in}}%
\pgfpathlineto{\pgfqpoint{3.167295in}{3.247833in}}%
\pgfpathlineto{\pgfqpoint{3.173530in}{3.236294in}}%
\pgfpathlineto{\pgfqpoint{3.179769in}{3.224667in}}%
\pgfpathclose%
\pgfusepath{stroke,fill}%
\end{pgfscope}%
\begin{pgfscope}%
\pgfpathrectangle{\pgfqpoint{0.887500in}{0.275000in}}{\pgfqpoint{4.225000in}{4.225000in}}%
\pgfusepath{clip}%
\pgfsetbuttcap%
\pgfsetroundjoin%
\definecolor{currentfill}{rgb}{0.133743,0.548535,0.553541}%
\pgfsetfillcolor{currentfill}%
\pgfsetfillopacity{0.700000}%
\pgfsetlinewidth{0.501875pt}%
\definecolor{currentstroke}{rgb}{1.000000,1.000000,1.000000}%
\pgfsetstrokecolor{currentstroke}%
\pgfsetstrokeopacity{0.500000}%
\pgfsetdash{}{0pt}%
\pgfpathmoveto{\pgfqpoint{1.897944in}{2.454774in}}%
\pgfpathlineto{\pgfqpoint{1.909556in}{2.458054in}}%
\pgfpathlineto{\pgfqpoint{1.921162in}{2.461329in}}%
\pgfpathlineto{\pgfqpoint{1.932763in}{2.464600in}}%
\pgfpathlineto{\pgfqpoint{1.944357in}{2.467869in}}%
\pgfpathlineto{\pgfqpoint{1.955947in}{2.471139in}}%
\pgfpathlineto{\pgfqpoint{1.950112in}{2.479375in}}%
\pgfpathlineto{\pgfqpoint{1.944282in}{2.487593in}}%
\pgfpathlineto{\pgfqpoint{1.938455in}{2.495792in}}%
\pgfpathlineto{\pgfqpoint{1.932634in}{2.503972in}}%
\pgfpathlineto{\pgfqpoint{1.926816in}{2.512136in}}%
\pgfpathlineto{\pgfqpoint{1.915239in}{2.508881in}}%
\pgfpathlineto{\pgfqpoint{1.903656in}{2.505628in}}%
\pgfpathlineto{\pgfqpoint{1.892068in}{2.502375in}}%
\pgfpathlineto{\pgfqpoint{1.880474in}{2.499120in}}%
\pgfpathlineto{\pgfqpoint{1.868874in}{2.495861in}}%
\pgfpathlineto{\pgfqpoint{1.874679in}{2.487683in}}%
\pgfpathlineto{\pgfqpoint{1.880489in}{2.479485in}}%
\pgfpathlineto{\pgfqpoint{1.886303in}{2.471268in}}%
\pgfpathlineto{\pgfqpoint{1.892121in}{2.463031in}}%
\pgfpathclose%
\pgfusepath{stroke,fill}%
\end{pgfscope}%
\begin{pgfscope}%
\pgfpathrectangle{\pgfqpoint{0.887500in}{0.275000in}}{\pgfqpoint{4.225000in}{4.225000in}}%
\pgfusepath{clip}%
\pgfsetbuttcap%
\pgfsetroundjoin%
\definecolor{currentfill}{rgb}{0.440137,0.811138,0.340967}%
\pgfsetfillcolor{currentfill}%
\pgfsetfillopacity{0.700000}%
\pgfsetlinewidth{0.501875pt}%
\definecolor{currentstroke}{rgb}{1.000000,1.000000,1.000000}%
\pgfsetstrokecolor{currentstroke}%
\pgfsetstrokeopacity{0.500000}%
\pgfsetdash{}{0pt}%
\pgfpathmoveto{\pgfqpoint{3.536720in}{3.043809in}}%
\pgfpathlineto{\pgfqpoint{3.547928in}{3.047308in}}%
\pgfpathlineto{\pgfqpoint{3.559129in}{3.050768in}}%
\pgfpathlineto{\pgfqpoint{3.570325in}{3.054186in}}%
\pgfpathlineto{\pgfqpoint{3.581515in}{3.057561in}}%
\pgfpathlineto{\pgfqpoint{3.592698in}{3.060891in}}%
\pgfpathlineto{\pgfqpoint{3.586382in}{3.073809in}}%
\pgfpathlineto{\pgfqpoint{3.580066in}{3.086588in}}%
\pgfpathlineto{\pgfqpoint{3.573751in}{3.099203in}}%
\pgfpathlineto{\pgfqpoint{3.567436in}{3.111632in}}%
\pgfpathlineto{\pgfqpoint{3.561122in}{3.123852in}}%
\pgfpathlineto{\pgfqpoint{3.549944in}{3.120648in}}%
\pgfpathlineto{\pgfqpoint{3.538761in}{3.117389in}}%
\pgfpathlineto{\pgfqpoint{3.527571in}{3.114077in}}%
\pgfpathlineto{\pgfqpoint{3.516375in}{3.110719in}}%
\pgfpathlineto{\pgfqpoint{3.505173in}{3.107318in}}%
\pgfpathlineto{\pgfqpoint{3.511480in}{3.094856in}}%
\pgfpathlineto{\pgfqpoint{3.517788in}{3.082262in}}%
\pgfpathlineto{\pgfqpoint{3.524097in}{3.069548in}}%
\pgfpathlineto{\pgfqpoint{3.530408in}{3.056726in}}%
\pgfpathclose%
\pgfusepath{stroke,fill}%
\end{pgfscope}%
\begin{pgfscope}%
\pgfpathrectangle{\pgfqpoint{0.887500in}{0.275000in}}{\pgfqpoint{4.225000in}{4.225000in}}%
\pgfusepath{clip}%
\pgfsetbuttcap%
\pgfsetroundjoin%
\definecolor{currentfill}{rgb}{0.143343,0.522773,0.556295}%
\pgfsetfillcolor{currentfill}%
\pgfsetfillopacity{0.700000}%
\pgfsetlinewidth{0.501875pt}%
\definecolor{currentstroke}{rgb}{1.000000,1.000000,1.000000}%
\pgfsetstrokecolor{currentstroke}%
\pgfsetstrokeopacity{0.500000}%
\pgfsetdash{}{0pt}%
\pgfpathmoveto{\pgfqpoint{2.217648in}{2.395766in}}%
\pgfpathlineto{\pgfqpoint{2.229181in}{2.399093in}}%
\pgfpathlineto{\pgfqpoint{2.240708in}{2.402408in}}%
\pgfpathlineto{\pgfqpoint{2.252230in}{2.405711in}}%
\pgfpathlineto{\pgfqpoint{2.263746in}{2.409001in}}%
\pgfpathlineto{\pgfqpoint{2.275257in}{2.412283in}}%
\pgfpathlineto{\pgfqpoint{2.269311in}{2.420716in}}%
\pgfpathlineto{\pgfqpoint{2.263370in}{2.429134in}}%
\pgfpathlineto{\pgfqpoint{2.257433in}{2.437536in}}%
\pgfpathlineto{\pgfqpoint{2.251500in}{2.445923in}}%
\pgfpathlineto{\pgfqpoint{2.245571in}{2.454295in}}%
\pgfpathlineto{\pgfqpoint{2.234072in}{2.451020in}}%
\pgfpathlineto{\pgfqpoint{2.222568in}{2.447741in}}%
\pgfpathlineto{\pgfqpoint{2.211057in}{2.444454in}}%
\pgfpathlineto{\pgfqpoint{2.199542in}{2.441158in}}%
\pgfpathlineto{\pgfqpoint{2.188021in}{2.437853in}}%
\pgfpathlineto{\pgfqpoint{2.193938in}{2.429469in}}%
\pgfpathlineto{\pgfqpoint{2.199859in}{2.421068in}}%
\pgfpathlineto{\pgfqpoint{2.205784in}{2.412651in}}%
\pgfpathlineto{\pgfqpoint{2.211714in}{2.404217in}}%
\pgfpathclose%
\pgfusepath{stroke,fill}%
\end{pgfscope}%
\begin{pgfscope}%
\pgfpathrectangle{\pgfqpoint{0.887500in}{0.275000in}}{\pgfqpoint{4.225000in}{4.225000in}}%
\pgfusepath{clip}%
\pgfsetbuttcap%
\pgfsetroundjoin%
\definecolor{currentfill}{rgb}{0.606045,0.850733,0.236712}%
\pgfsetfillcolor{currentfill}%
\pgfsetfillopacity{0.700000}%
\pgfsetlinewidth{0.501875pt}%
\definecolor{currentstroke}{rgb}{1.000000,1.000000,1.000000}%
\pgfsetstrokecolor{currentstroke}%
\pgfsetstrokeopacity{0.500000}%
\pgfsetdash{}{0pt}%
\pgfpathmoveto{\pgfqpoint{3.273719in}{3.176392in}}%
\pgfpathlineto{\pgfqpoint{3.284993in}{3.180272in}}%
\pgfpathlineto{\pgfqpoint{3.296261in}{3.184100in}}%
\pgfpathlineto{\pgfqpoint{3.307524in}{3.187863in}}%
\pgfpathlineto{\pgfqpoint{3.318782in}{3.191544in}}%
\pgfpathlineto{\pgfqpoint{3.330033in}{3.195129in}}%
\pgfpathlineto{\pgfqpoint{3.323760in}{3.206294in}}%
\pgfpathlineto{\pgfqpoint{3.317488in}{3.217177in}}%
\pgfpathlineto{\pgfqpoint{3.311218in}{3.227815in}}%
\pgfpathlineto{\pgfqpoint{3.304950in}{3.238246in}}%
\pgfpathlineto{\pgfqpoint{3.298685in}{3.248507in}}%
\pgfpathlineto{\pgfqpoint{3.287443in}{3.245232in}}%
\pgfpathlineto{\pgfqpoint{3.276195in}{3.241942in}}%
\pgfpathlineto{\pgfqpoint{3.264942in}{3.238653in}}%
\pgfpathlineto{\pgfqpoint{3.253684in}{3.235380in}}%
\pgfpathlineto{\pgfqpoint{3.242420in}{3.232138in}}%
\pgfpathlineto{\pgfqpoint{3.248675in}{3.221260in}}%
\pgfpathlineto{\pgfqpoint{3.254932in}{3.210264in}}%
\pgfpathlineto{\pgfqpoint{3.261192in}{3.199131in}}%
\pgfpathlineto{\pgfqpoint{3.267454in}{3.187847in}}%
\pgfpathclose%
\pgfusepath{stroke,fill}%
\end{pgfscope}%
\begin{pgfscope}%
\pgfpathrectangle{\pgfqpoint{0.887500in}{0.275000in}}{\pgfqpoint{4.225000in}{4.225000in}}%
\pgfusepath{clip}%
\pgfsetbuttcap%
\pgfsetroundjoin%
\definecolor{currentfill}{rgb}{0.496615,0.826376,0.306377}%
\pgfsetfillcolor{currentfill}%
\pgfsetfillopacity{0.700000}%
\pgfsetlinewidth{0.501875pt}%
\definecolor{currentstroke}{rgb}{1.000000,1.000000,1.000000}%
\pgfsetstrokecolor{currentstroke}%
\pgfsetstrokeopacity{0.500000}%
\pgfsetdash{}{0pt}%
\pgfpathmoveto{\pgfqpoint{3.449080in}{3.089816in}}%
\pgfpathlineto{\pgfqpoint{3.460310in}{3.093375in}}%
\pgfpathlineto{\pgfqpoint{3.471534in}{3.096906in}}%
\pgfpathlineto{\pgfqpoint{3.482753in}{3.100409in}}%
\pgfpathlineto{\pgfqpoint{3.493966in}{3.103880in}}%
\pgfpathlineto{\pgfqpoint{3.505173in}{3.107318in}}%
\pgfpathlineto{\pgfqpoint{3.498868in}{3.119640in}}%
\pgfpathlineto{\pgfqpoint{3.492565in}{3.131822in}}%
\pgfpathlineto{\pgfqpoint{3.486262in}{3.143867in}}%
\pgfpathlineto{\pgfqpoint{3.479962in}{3.155776in}}%
\pgfpathlineto{\pgfqpoint{3.473663in}{3.167550in}}%
\pgfpathlineto{\pgfqpoint{3.462466in}{3.164635in}}%
\pgfpathlineto{\pgfqpoint{3.451263in}{3.161673in}}%
\pgfpathlineto{\pgfqpoint{3.440054in}{3.158654in}}%
\pgfpathlineto{\pgfqpoint{3.428839in}{3.155572in}}%
\pgfpathlineto{\pgfqpoint{3.417618in}{3.152418in}}%
\pgfpathlineto{\pgfqpoint{3.423906in}{3.140113in}}%
\pgfpathlineto{\pgfqpoint{3.430197in}{3.127678in}}%
\pgfpathlineto{\pgfqpoint{3.436489in}{3.115134in}}%
\pgfpathlineto{\pgfqpoint{3.442783in}{3.102506in}}%
\pgfpathclose%
\pgfusepath{stroke,fill}%
\end{pgfscope}%
\begin{pgfscope}%
\pgfpathrectangle{\pgfqpoint{0.887500in}{0.275000in}}{\pgfqpoint{4.225000in}{4.225000in}}%
\pgfusepath{clip}%
\pgfsetbuttcap%
\pgfsetroundjoin%
\definecolor{currentfill}{rgb}{0.154815,0.493313,0.557840}%
\pgfsetfillcolor{currentfill}%
\pgfsetfillopacity{0.700000}%
\pgfsetlinewidth{0.501875pt}%
\definecolor{currentstroke}{rgb}{1.000000,1.000000,1.000000}%
\pgfsetstrokecolor{currentstroke}%
\pgfsetstrokeopacity{0.500000}%
\pgfsetdash{}{0pt}%
\pgfpathmoveto{\pgfqpoint{2.537411in}{2.335001in}}%
\pgfpathlineto{\pgfqpoint{2.548863in}{2.338485in}}%
\pgfpathlineto{\pgfqpoint{2.560311in}{2.341869in}}%
\pgfpathlineto{\pgfqpoint{2.571755in}{2.345117in}}%
\pgfpathlineto{\pgfqpoint{2.583195in}{2.348196in}}%
\pgfpathlineto{\pgfqpoint{2.594631in}{2.351131in}}%
\pgfpathlineto{\pgfqpoint{2.588578in}{2.359776in}}%
\pgfpathlineto{\pgfqpoint{2.582529in}{2.368404in}}%
\pgfpathlineto{\pgfqpoint{2.576484in}{2.377016in}}%
\pgfpathlineto{\pgfqpoint{2.570443in}{2.385613in}}%
\pgfpathlineto{\pgfqpoint{2.564406in}{2.394195in}}%
\pgfpathlineto{\pgfqpoint{2.552981in}{2.391266in}}%
\pgfpathlineto{\pgfqpoint{2.541552in}{2.388192in}}%
\pgfpathlineto{\pgfqpoint{2.530119in}{2.384949in}}%
\pgfpathlineto{\pgfqpoint{2.518683in}{2.381572in}}%
\pgfpathlineto{\pgfqpoint{2.507242in}{2.378094in}}%
\pgfpathlineto{\pgfqpoint{2.513268in}{2.369513in}}%
\pgfpathlineto{\pgfqpoint{2.519297in}{2.360914in}}%
\pgfpathlineto{\pgfqpoint{2.525331in}{2.352296in}}%
\pgfpathlineto{\pgfqpoint{2.531369in}{2.343658in}}%
\pgfpathclose%
\pgfusepath{stroke,fill}%
\end{pgfscope}%
\begin{pgfscope}%
\pgfpathrectangle{\pgfqpoint{0.887500in}{0.275000in}}{\pgfqpoint{4.225000in}{4.225000in}}%
\pgfusepath{clip}%
\pgfsetbuttcap%
\pgfsetroundjoin%
\definecolor{currentfill}{rgb}{0.157851,0.683765,0.501686}%
\pgfsetfillcolor{currentfill}%
\pgfsetfillopacity{0.700000}%
\pgfsetlinewidth{0.501875pt}%
\definecolor{currentstroke}{rgb}{1.000000,1.000000,1.000000}%
\pgfsetstrokecolor{currentstroke}%
\pgfsetstrokeopacity{0.500000}%
\pgfsetdash{}{0pt}%
\pgfpathmoveto{\pgfqpoint{4.006185in}{2.726031in}}%
\pgfpathlineto{\pgfqpoint{4.017277in}{2.729422in}}%
\pgfpathlineto{\pgfqpoint{4.028362in}{2.732812in}}%
\pgfpathlineto{\pgfqpoint{4.039443in}{2.736203in}}%
\pgfpathlineto{\pgfqpoint{4.050518in}{2.739597in}}%
\pgfpathlineto{\pgfqpoint{4.061588in}{2.742995in}}%
\pgfpathlineto{\pgfqpoint{4.055195in}{2.756939in}}%
\pgfpathlineto{\pgfqpoint{4.048804in}{2.770850in}}%
\pgfpathlineto{\pgfqpoint{4.042415in}{2.784725in}}%
\pgfpathlineto{\pgfqpoint{4.036027in}{2.798559in}}%
\pgfpathlineto{\pgfqpoint{4.029641in}{2.812357in}}%
\pgfpathlineto{\pgfqpoint{4.018572in}{2.808911in}}%
\pgfpathlineto{\pgfqpoint{4.007498in}{2.805473in}}%
\pgfpathlineto{\pgfqpoint{3.996419in}{2.802040in}}%
\pgfpathlineto{\pgfqpoint{3.985335in}{2.798612in}}%
\pgfpathlineto{\pgfqpoint{3.974245in}{2.795187in}}%
\pgfpathlineto{\pgfqpoint{3.980631in}{2.781458in}}%
\pgfpathlineto{\pgfqpoint{3.987017in}{2.767679in}}%
\pgfpathlineto{\pgfqpoint{3.993405in}{2.753844in}}%
\pgfpathlineto{\pgfqpoint{3.999795in}{2.739960in}}%
\pgfpathclose%
\pgfusepath{stroke,fill}%
\end{pgfscope}%
\begin{pgfscope}%
\pgfpathrectangle{\pgfqpoint{0.887500in}{0.275000in}}{\pgfqpoint{4.225000in}{4.225000in}}%
\pgfusepath{clip}%
\pgfsetbuttcap%
\pgfsetroundjoin%
\definecolor{currentfill}{rgb}{0.555484,0.840254,0.269281}%
\pgfsetfillcolor{currentfill}%
\pgfsetfillopacity{0.700000}%
\pgfsetlinewidth{0.501875pt}%
\definecolor{currentstroke}{rgb}{1.000000,1.000000,1.000000}%
\pgfsetstrokecolor{currentstroke}%
\pgfsetstrokeopacity{0.500000}%
\pgfsetdash{}{0pt}%
\pgfpathmoveto{\pgfqpoint{3.361419in}{3.135349in}}%
\pgfpathlineto{\pgfqpoint{3.372671in}{3.138942in}}%
\pgfpathlineto{\pgfqpoint{3.383917in}{3.142451in}}%
\pgfpathlineto{\pgfqpoint{3.395157in}{3.145865in}}%
\pgfpathlineto{\pgfqpoint{3.406390in}{3.149185in}}%
\pgfpathlineto{\pgfqpoint{3.417618in}{3.152418in}}%
\pgfpathlineto{\pgfqpoint{3.411331in}{3.164568in}}%
\pgfpathlineto{\pgfqpoint{3.405045in}{3.176539in}}%
\pgfpathlineto{\pgfqpoint{3.398761in}{3.188310in}}%
\pgfpathlineto{\pgfqpoint{3.392478in}{3.199856in}}%
\pgfpathlineto{\pgfqpoint{3.386196in}{3.211154in}}%
\pgfpathlineto{\pgfqpoint{3.374976in}{3.208213in}}%
\pgfpathlineto{\pgfqpoint{3.363750in}{3.205148in}}%
\pgfpathlineto{\pgfqpoint{3.352518in}{3.201945in}}%
\pgfpathlineto{\pgfqpoint{3.341279in}{3.198601in}}%
\pgfpathlineto{\pgfqpoint{3.330033in}{3.195129in}}%
\pgfpathlineto{\pgfqpoint{3.336308in}{3.183664in}}%
\pgfpathlineto{\pgfqpoint{3.342583in}{3.171923in}}%
\pgfpathlineto{\pgfqpoint{3.348860in}{3.159937in}}%
\pgfpathlineto{\pgfqpoint{3.355139in}{3.147736in}}%
\pgfpathclose%
\pgfusepath{stroke,fill}%
\end{pgfscope}%
\begin{pgfscope}%
\pgfpathrectangle{\pgfqpoint{0.887500in}{0.275000in}}{\pgfqpoint{4.225000in}{4.225000in}}%
\pgfusepath{clip}%
\pgfsetbuttcap%
\pgfsetroundjoin%
\definecolor{currentfill}{rgb}{0.124395,0.578002,0.548287}%
\pgfsetfillcolor{currentfill}%
\pgfsetfillopacity{0.700000}%
\pgfsetlinewidth{0.501875pt}%
\definecolor{currentstroke}{rgb}{1.000000,1.000000,1.000000}%
\pgfsetstrokecolor{currentstroke}%
\pgfsetstrokeopacity{0.500000}%
\pgfsetdash{}{0pt}%
\pgfpathmoveto{\pgfqpoint{4.300363in}{2.494448in}}%
\pgfpathlineto{\pgfqpoint{4.311379in}{2.497758in}}%
\pgfpathlineto{\pgfqpoint{4.322390in}{2.501072in}}%
\pgfpathlineto{\pgfqpoint{4.333395in}{2.504391in}}%
\pgfpathlineto{\pgfqpoint{4.344396in}{2.507716in}}%
\pgfpathlineto{\pgfqpoint{4.355391in}{2.511047in}}%
\pgfpathlineto{\pgfqpoint{4.348955in}{2.525413in}}%
\pgfpathlineto{\pgfqpoint{4.342521in}{2.539771in}}%
\pgfpathlineto{\pgfqpoint{4.336088in}{2.554114in}}%
\pgfpathlineto{\pgfqpoint{4.329658in}{2.568439in}}%
\pgfpathlineto{\pgfqpoint{4.323230in}{2.582746in}}%
\pgfpathlineto{\pgfqpoint{4.312235in}{2.579396in}}%
\pgfpathlineto{\pgfqpoint{4.301235in}{2.576038in}}%
\pgfpathlineto{\pgfqpoint{4.290230in}{2.572670in}}%
\pgfpathlineto{\pgfqpoint{4.279218in}{2.569293in}}%
\pgfpathlineto{\pgfqpoint{4.268201in}{2.565903in}}%
\pgfpathlineto{\pgfqpoint{4.274630in}{2.551645in}}%
\pgfpathlineto{\pgfqpoint{4.281060in}{2.537370in}}%
\pgfpathlineto{\pgfqpoint{4.287492in}{2.523078in}}%
\pgfpathlineto{\pgfqpoint{4.293927in}{2.508770in}}%
\pgfpathclose%
\pgfusepath{stroke,fill}%
\end{pgfscope}%
\begin{pgfscope}%
\pgfpathrectangle{\pgfqpoint{0.887500in}{0.275000in}}{\pgfqpoint{4.225000in}{4.225000in}}%
\pgfusepath{clip}%
\pgfsetbuttcap%
\pgfsetroundjoin%
\definecolor{currentfill}{rgb}{0.121380,0.629492,0.531973}%
\pgfsetfillcolor{currentfill}%
\pgfsetfillopacity{0.700000}%
\pgfsetlinewidth{0.501875pt}%
\definecolor{currentstroke}{rgb}{1.000000,1.000000,1.000000}%
\pgfsetstrokecolor{currentstroke}%
\pgfsetstrokeopacity{0.500000}%
\pgfsetdash{}{0pt}%
\pgfpathmoveto{\pgfqpoint{2.908624in}{2.542637in}}%
\pgfpathlineto{\pgfqpoint{2.919854in}{2.577803in}}%
\pgfpathlineto{\pgfqpoint{2.931103in}{2.611067in}}%
\pgfpathlineto{\pgfqpoint{2.942369in}{2.642471in}}%
\pgfpathlineto{\pgfqpoint{2.953649in}{2.672426in}}%
\pgfpathlineto{\pgfqpoint{2.964940in}{2.701346in}}%
\pgfpathlineto{\pgfqpoint{2.958742in}{2.715728in}}%
\pgfpathlineto{\pgfqpoint{2.952549in}{2.729135in}}%
\pgfpathlineto{\pgfqpoint{2.946361in}{2.741228in}}%
\pgfpathlineto{\pgfqpoint{2.940181in}{2.751668in}}%
\pgfpathlineto{\pgfqpoint{2.934011in}{2.760117in}}%
\pgfpathlineto{\pgfqpoint{2.922746in}{2.731086in}}%
\pgfpathlineto{\pgfqpoint{2.911494in}{2.700744in}}%
\pgfpathlineto{\pgfqpoint{2.900259in}{2.668694in}}%
\pgfpathlineto{\pgfqpoint{2.889044in}{2.634537in}}%
\pgfpathlineto{\pgfqpoint{2.877851in}{2.598257in}}%
\pgfpathlineto{\pgfqpoint{2.883992in}{2.588639in}}%
\pgfpathlineto{\pgfqpoint{2.890142in}{2.578088in}}%
\pgfpathlineto{\pgfqpoint{2.896297in}{2.566784in}}%
\pgfpathlineto{\pgfqpoint{2.902459in}{2.554906in}}%
\pgfpathclose%
\pgfusepath{stroke,fill}%
\end{pgfscope}%
\begin{pgfscope}%
\pgfpathrectangle{\pgfqpoint{0.887500in}{0.275000in}}{\pgfqpoint{4.225000in}{4.225000in}}%
\pgfusepath{clip}%
\pgfsetbuttcap%
\pgfsetroundjoin%
\definecolor{currentfill}{rgb}{0.191090,0.708366,0.482284}%
\pgfsetfillcolor{currentfill}%
\pgfsetfillopacity{0.700000}%
\pgfsetlinewidth{0.501875pt}%
\definecolor{currentstroke}{rgb}{1.000000,1.000000,1.000000}%
\pgfsetstrokecolor{currentstroke}%
\pgfsetstrokeopacity{0.500000}%
\pgfsetdash{}{0pt}%
\pgfpathmoveto{\pgfqpoint{3.918717in}{2.778001in}}%
\pgfpathlineto{\pgfqpoint{3.929833in}{2.781456in}}%
\pgfpathlineto{\pgfqpoint{3.940945in}{2.784898in}}%
\pgfpathlineto{\pgfqpoint{3.952050in}{2.788333in}}%
\pgfpathlineto{\pgfqpoint{3.963151in}{2.791761in}}%
\pgfpathlineto{\pgfqpoint{3.974245in}{2.795187in}}%
\pgfpathlineto{\pgfqpoint{3.967862in}{2.808874in}}%
\pgfpathlineto{\pgfqpoint{3.961480in}{2.822530in}}%
\pgfpathlineto{\pgfqpoint{3.955101in}{2.836163in}}%
\pgfpathlineto{\pgfqpoint{3.948724in}{2.849785in}}%
\pgfpathlineto{\pgfqpoint{3.942349in}{2.863405in}}%
\pgfpathlineto{\pgfqpoint{3.931256in}{2.859944in}}%
\pgfpathlineto{\pgfqpoint{3.920158in}{2.856495in}}%
\pgfpathlineto{\pgfqpoint{3.909055in}{2.853052in}}%
\pgfpathlineto{\pgfqpoint{3.897947in}{2.849610in}}%
\pgfpathlineto{\pgfqpoint{3.886833in}{2.846165in}}%
\pgfpathlineto{\pgfqpoint{3.893205in}{2.832557in}}%
\pgfpathlineto{\pgfqpoint{3.899580in}{2.818946in}}%
\pgfpathlineto{\pgfqpoint{3.905957in}{2.805321in}}%
\pgfpathlineto{\pgfqpoint{3.912336in}{2.791676in}}%
\pgfpathclose%
\pgfusepath{stroke,fill}%
\end{pgfscope}%
\begin{pgfscope}%
\pgfpathrectangle{\pgfqpoint{0.887500in}{0.275000in}}{\pgfqpoint{4.225000in}{4.225000in}}%
\pgfusepath{clip}%
\pgfsetbuttcap%
\pgfsetroundjoin%
\definecolor{currentfill}{rgb}{0.127568,0.566949,0.550556}%
\pgfsetfillcolor{currentfill}%
\pgfsetfillopacity{0.700000}%
\pgfsetlinewidth{0.501875pt}%
\definecolor{currentstroke}{rgb}{1.000000,1.000000,1.000000}%
\pgfsetstrokecolor{currentstroke}%
\pgfsetstrokeopacity{0.500000}%
\pgfsetdash{}{0pt}%
\pgfpathmoveto{\pgfqpoint{1.665430in}{2.487051in}}%
\pgfpathlineto{\pgfqpoint{1.677102in}{2.490367in}}%
\pgfpathlineto{\pgfqpoint{1.688769in}{2.493683in}}%
\pgfpathlineto{\pgfqpoint{1.700429in}{2.496998in}}%
\pgfpathlineto{\pgfqpoint{1.712083in}{2.500314in}}%
\pgfpathlineto{\pgfqpoint{1.723732in}{2.503631in}}%
\pgfpathlineto{\pgfqpoint{1.717977in}{2.511741in}}%
\pgfpathlineto{\pgfqpoint{1.712227in}{2.519835in}}%
\pgfpathlineto{\pgfqpoint{1.706480in}{2.527915in}}%
\pgfpathlineto{\pgfqpoint{1.700738in}{2.535980in}}%
\pgfpathlineto{\pgfqpoint{1.695000in}{2.544029in}}%
\pgfpathlineto{\pgfqpoint{1.683364in}{2.540731in}}%
\pgfpathlineto{\pgfqpoint{1.671722in}{2.537433in}}%
\pgfpathlineto{\pgfqpoint{1.660074in}{2.534136in}}%
\pgfpathlineto{\pgfqpoint{1.648420in}{2.530838in}}%
\pgfpathlineto{\pgfqpoint{1.636761in}{2.527538in}}%
\pgfpathlineto{\pgfqpoint{1.642487in}{2.519472in}}%
\pgfpathlineto{\pgfqpoint{1.648216in}{2.511389in}}%
\pgfpathlineto{\pgfqpoint{1.653950in}{2.503291in}}%
\pgfpathlineto{\pgfqpoint{1.659688in}{2.495178in}}%
\pgfpathclose%
\pgfusepath{stroke,fill}%
\end{pgfscope}%
\begin{pgfscope}%
\pgfpathrectangle{\pgfqpoint{0.887500in}{0.275000in}}{\pgfqpoint{4.225000in}{4.225000in}}%
\pgfusepath{clip}%
\pgfsetbuttcap%
\pgfsetroundjoin%
\definecolor{currentfill}{rgb}{0.136408,0.541173,0.554483}%
\pgfsetfillcolor{currentfill}%
\pgfsetfillopacity{0.700000}%
\pgfsetlinewidth{0.501875pt}%
\definecolor{currentstroke}{rgb}{1.000000,1.000000,1.000000}%
\pgfsetstrokecolor{currentstroke}%
\pgfsetstrokeopacity{0.500000}%
\pgfsetdash{}{0pt}%
\pgfpathmoveto{\pgfqpoint{1.985184in}{2.429667in}}%
\pgfpathlineto{\pgfqpoint{1.996779in}{2.432959in}}%
\pgfpathlineto{\pgfqpoint{2.008369in}{2.436255in}}%
\pgfpathlineto{\pgfqpoint{2.019952in}{2.439560in}}%
\pgfpathlineto{\pgfqpoint{2.031530in}{2.442874in}}%
\pgfpathlineto{\pgfqpoint{2.043101in}{2.446202in}}%
\pgfpathlineto{\pgfqpoint{2.037233in}{2.454517in}}%
\pgfpathlineto{\pgfqpoint{2.031369in}{2.462814in}}%
\pgfpathlineto{\pgfqpoint{2.025509in}{2.471091in}}%
\pgfpathlineto{\pgfqpoint{2.019654in}{2.479350in}}%
\pgfpathlineto{\pgfqpoint{2.013803in}{2.487591in}}%
\pgfpathlineto{\pgfqpoint{2.002244in}{2.484278in}}%
\pgfpathlineto{\pgfqpoint{1.990678in}{2.480979in}}%
\pgfpathlineto{\pgfqpoint{1.979107in}{2.477692in}}%
\pgfpathlineto{\pgfqpoint{1.967530in}{2.474412in}}%
\pgfpathlineto{\pgfqpoint{1.955947in}{2.471139in}}%
\pgfpathlineto{\pgfqpoint{1.961785in}{2.462883in}}%
\pgfpathlineto{\pgfqpoint{1.967629in}{2.454609in}}%
\pgfpathlineto{\pgfqpoint{1.973476in}{2.446314in}}%
\pgfpathlineto{\pgfqpoint{1.979328in}{2.438000in}}%
\pgfpathclose%
\pgfusepath{stroke,fill}%
\end{pgfscope}%
\begin{pgfscope}%
\pgfpathrectangle{\pgfqpoint{0.887500in}{0.275000in}}{\pgfqpoint{4.225000in}{4.225000in}}%
\pgfusepath{clip}%
\pgfsetbuttcap%
\pgfsetroundjoin%
\definecolor{currentfill}{rgb}{0.202219,0.715272,0.476084}%
\pgfsetfillcolor{currentfill}%
\pgfsetfillopacity{0.700000}%
\pgfsetlinewidth{0.501875pt}%
\definecolor{currentstroke}{rgb}{1.000000,1.000000,1.000000}%
\pgfsetstrokecolor{currentstroke}%
\pgfsetstrokeopacity{0.500000}%
\pgfsetdash{}{0pt}%
\pgfpathmoveto{\pgfqpoint{2.934011in}{2.760117in}}%
\pgfpathlineto{\pgfqpoint{2.945288in}{2.788241in}}%
\pgfpathlineto{\pgfqpoint{2.956575in}{2.815861in}}%
\pgfpathlineto{\pgfqpoint{2.967871in}{2.843383in}}%
\pgfpathlineto{\pgfqpoint{2.979174in}{2.871035in}}%
\pgfpathlineto{\pgfqpoint{2.990486in}{2.898469in}}%
\pgfpathlineto{\pgfqpoint{2.984295in}{2.904033in}}%
\pgfpathlineto{\pgfqpoint{2.978114in}{2.907604in}}%
\pgfpathlineto{\pgfqpoint{2.971941in}{2.909459in}}%
\pgfpathlineto{\pgfqpoint{2.965778in}{2.909879in}}%
\pgfpathlineto{\pgfqpoint{2.959624in}{2.909142in}}%
\pgfpathlineto{\pgfqpoint{2.948352in}{2.881168in}}%
\pgfpathlineto{\pgfqpoint{2.937087in}{2.853418in}}%
\pgfpathlineto{\pgfqpoint{2.925830in}{2.826224in}}%
\pgfpathlineto{\pgfqpoint{2.914579in}{2.799448in}}%
\pgfpathlineto{\pgfqpoint{2.903335in}{2.772806in}}%
\pgfpathlineto{\pgfqpoint{2.909447in}{2.773261in}}%
\pgfpathlineto{\pgfqpoint{2.915571in}{2.772542in}}%
\pgfpathlineto{\pgfqpoint{2.921706in}{2.770328in}}%
\pgfpathlineto{\pgfqpoint{2.927853in}{2.766294in}}%
\pgfpathclose%
\pgfusepath{stroke,fill}%
\end{pgfscope}%
\begin{pgfscope}%
\pgfpathrectangle{\pgfqpoint{0.887500in}{0.275000in}}{\pgfqpoint{4.225000in}{4.225000in}}%
\pgfusepath{clip}%
\pgfsetbuttcap%
\pgfsetroundjoin%
\definecolor{currentfill}{rgb}{0.159194,0.482237,0.558073}%
\pgfsetfillcolor{currentfill}%
\pgfsetfillopacity{0.700000}%
\pgfsetlinewidth{0.501875pt}%
\definecolor{currentstroke}{rgb}{1.000000,1.000000,1.000000}%
\pgfsetstrokecolor{currentstroke}%
\pgfsetstrokeopacity{0.500000}%
\pgfsetdash{}{0pt}%
\pgfpathmoveto{\pgfqpoint{2.624959in}{2.307614in}}%
\pgfpathlineto{\pgfqpoint{2.636400in}{2.310549in}}%
\pgfpathlineto{\pgfqpoint{2.647835in}{2.313548in}}%
\pgfpathlineto{\pgfqpoint{2.659262in}{2.316714in}}%
\pgfpathlineto{\pgfqpoint{2.670680in}{2.320149in}}%
\pgfpathlineto{\pgfqpoint{2.682090in}{2.323953in}}%
\pgfpathlineto{\pgfqpoint{2.676005in}{2.332695in}}%
\pgfpathlineto{\pgfqpoint{2.669924in}{2.341414in}}%
\pgfpathlineto{\pgfqpoint{2.663848in}{2.350110in}}%
\pgfpathlineto{\pgfqpoint{2.657775in}{2.358783in}}%
\pgfpathlineto{\pgfqpoint{2.651707in}{2.367434in}}%
\pgfpathlineto{\pgfqpoint{2.640309in}{2.363586in}}%
\pgfpathlineto{\pgfqpoint{2.628902in}{2.360142in}}%
\pgfpathlineto{\pgfqpoint{2.617486in}{2.356993in}}%
\pgfpathlineto{\pgfqpoint{2.606062in}{2.354026in}}%
\pgfpathlineto{\pgfqpoint{2.594631in}{2.351131in}}%
\pgfpathlineto{\pgfqpoint{2.600689in}{2.342467in}}%
\pgfpathlineto{\pgfqpoint{2.606750in}{2.333784in}}%
\pgfpathlineto{\pgfqpoint{2.612816in}{2.325082in}}%
\pgfpathlineto{\pgfqpoint{2.618885in}{2.316359in}}%
\pgfpathclose%
\pgfusepath{stroke,fill}%
\end{pgfscope}%
\begin{pgfscope}%
\pgfpathrectangle{\pgfqpoint{0.887500in}{0.275000in}}{\pgfqpoint{4.225000in}{4.225000in}}%
\pgfusepath{clip}%
\pgfsetbuttcap%
\pgfsetroundjoin%
\definecolor{currentfill}{rgb}{0.119738,0.603785,0.541400}%
\pgfsetfillcolor{currentfill}%
\pgfsetfillopacity{0.700000}%
\pgfsetlinewidth{0.501875pt}%
\definecolor{currentstroke}{rgb}{1.000000,1.000000,1.000000}%
\pgfsetstrokecolor{currentstroke}%
\pgfsetstrokeopacity{0.500000}%
\pgfsetdash{}{0pt}%
\pgfpathmoveto{\pgfqpoint{4.213033in}{2.548853in}}%
\pgfpathlineto{\pgfqpoint{4.224077in}{2.552265in}}%
\pgfpathlineto{\pgfqpoint{4.235116in}{2.555679in}}%
\pgfpathlineto{\pgfqpoint{4.246150in}{2.559093in}}%
\pgfpathlineto{\pgfqpoint{4.257178in}{2.562502in}}%
\pgfpathlineto{\pgfqpoint{4.268201in}{2.565903in}}%
\pgfpathlineto{\pgfqpoint{4.261775in}{2.580143in}}%
\pgfpathlineto{\pgfqpoint{4.255350in}{2.594363in}}%
\pgfpathlineto{\pgfqpoint{4.248928in}{2.608563in}}%
\pgfpathlineto{\pgfqpoint{4.242507in}{2.622743in}}%
\pgfpathlineto{\pgfqpoint{4.236088in}{2.636900in}}%
\pgfpathlineto{\pgfqpoint{4.225066in}{2.633457in}}%
\pgfpathlineto{\pgfqpoint{4.214038in}{2.630005in}}%
\pgfpathlineto{\pgfqpoint{4.203005in}{2.626548in}}%
\pgfpathlineto{\pgfqpoint{4.191966in}{2.623090in}}%
\pgfpathlineto{\pgfqpoint{4.180923in}{2.619634in}}%
\pgfpathlineto{\pgfqpoint{4.187340in}{2.605499in}}%
\pgfpathlineto{\pgfqpoint{4.193760in}{2.591352in}}%
\pgfpathlineto{\pgfqpoint{4.200182in}{2.577194in}}%
\pgfpathlineto{\pgfqpoint{4.206606in}{2.563027in}}%
\pgfpathclose%
\pgfusepath{stroke,fill}%
\end{pgfscope}%
\begin{pgfscope}%
\pgfpathrectangle{\pgfqpoint{0.887500in}{0.275000in}}{\pgfqpoint{4.225000in}{4.225000in}}%
\pgfusepath{clip}%
\pgfsetbuttcap%
\pgfsetroundjoin%
\definecolor{currentfill}{rgb}{0.147607,0.511733,0.557049}%
\pgfsetfillcolor{currentfill}%
\pgfsetfillopacity{0.700000}%
\pgfsetlinewidth{0.501875pt}%
\definecolor{currentstroke}{rgb}{1.000000,1.000000,1.000000}%
\pgfsetstrokecolor{currentstroke}%
\pgfsetstrokeopacity{0.500000}%
\pgfsetdash{}{0pt}%
\pgfpathmoveto{\pgfqpoint{2.305046in}{2.369867in}}%
\pgfpathlineto{\pgfqpoint{2.316563in}{2.373156in}}%
\pgfpathlineto{\pgfqpoint{2.328074in}{2.376442in}}%
\pgfpathlineto{\pgfqpoint{2.339580in}{2.379729in}}%
\pgfpathlineto{\pgfqpoint{2.351079in}{2.383022in}}%
\pgfpathlineto{\pgfqpoint{2.362573in}{2.386326in}}%
\pgfpathlineto{\pgfqpoint{2.356595in}{2.394835in}}%
\pgfpathlineto{\pgfqpoint{2.350621in}{2.403329in}}%
\pgfpathlineto{\pgfqpoint{2.344652in}{2.411807in}}%
\pgfpathlineto{\pgfqpoint{2.338686in}{2.420270in}}%
\pgfpathlineto{\pgfqpoint{2.332725in}{2.428717in}}%
\pgfpathlineto{\pgfqpoint{2.321243in}{2.425412in}}%
\pgfpathlineto{\pgfqpoint{2.309755in}{2.422121in}}%
\pgfpathlineto{\pgfqpoint{2.298261in}{2.418838in}}%
\pgfpathlineto{\pgfqpoint{2.286762in}{2.415561in}}%
\pgfpathlineto{\pgfqpoint{2.275257in}{2.412283in}}%
\pgfpathlineto{\pgfqpoint{2.281206in}{2.403833in}}%
\pgfpathlineto{\pgfqpoint{2.287160in}{2.395367in}}%
\pgfpathlineto{\pgfqpoint{2.293118in}{2.386885in}}%
\pgfpathlineto{\pgfqpoint{2.299080in}{2.378385in}}%
\pgfpathclose%
\pgfusepath{stroke,fill}%
\end{pgfscope}%
\begin{pgfscope}%
\pgfpathrectangle{\pgfqpoint{0.887500in}{0.275000in}}{\pgfqpoint{4.225000in}{4.225000in}}%
\pgfusepath{clip}%
\pgfsetbuttcap%
\pgfsetroundjoin%
\definecolor{currentfill}{rgb}{0.149039,0.508051,0.557250}%
\pgfsetfillcolor{currentfill}%
\pgfsetfillopacity{0.700000}%
\pgfsetlinewidth{0.501875pt}%
\definecolor{currentstroke}{rgb}{1.000000,1.000000,1.000000}%
\pgfsetstrokecolor{currentstroke}%
\pgfsetstrokeopacity{0.500000}%
\pgfsetdash{}{0pt}%
\pgfpathmoveto{\pgfqpoint{2.913975in}{2.298432in}}%
\pgfpathlineto{\pgfqpoint{2.925265in}{2.319794in}}%
\pgfpathlineto{\pgfqpoint{2.936547in}{2.344979in}}%
\pgfpathlineto{\pgfqpoint{2.947828in}{2.372919in}}%
\pgfpathlineto{\pgfqpoint{2.959114in}{2.402547in}}%
\pgfpathlineto{\pgfqpoint{2.970408in}{2.432786in}}%
\pgfpathlineto{\pgfqpoint{2.964220in}{2.441742in}}%
\pgfpathlineto{\pgfqpoint{2.958035in}{2.450996in}}%
\pgfpathlineto{\pgfqpoint{2.951852in}{2.460666in}}%
\pgfpathlineto{\pgfqpoint{2.945672in}{2.470870in}}%
\pgfpathlineto{\pgfqpoint{2.939493in}{2.481724in}}%
\pgfpathlineto{\pgfqpoint{2.928234in}{2.450106in}}%
\pgfpathlineto{\pgfqpoint{2.916982in}{2.419218in}}%
\pgfpathlineto{\pgfqpoint{2.905734in}{2.390167in}}%
\pgfpathlineto{\pgfqpoint{2.894483in}{2.364056in}}%
\pgfpathlineto{\pgfqpoint{2.883222in}{2.341987in}}%
\pgfpathlineto{\pgfqpoint{2.889363in}{2.333620in}}%
\pgfpathlineto{\pgfqpoint{2.895509in}{2.325060in}}%
\pgfpathlineto{\pgfqpoint{2.901660in}{2.316330in}}%
\pgfpathlineto{\pgfqpoint{2.907816in}{2.307447in}}%
\pgfpathclose%
\pgfusepath{stroke,fill}%
\end{pgfscope}%
\begin{pgfscope}%
\pgfpathrectangle{\pgfqpoint{0.887500in}{0.275000in}}{\pgfqpoint{4.225000in}{4.225000in}}%
\pgfusepath{clip}%
\pgfsetbuttcap%
\pgfsetroundjoin%
\definecolor{currentfill}{rgb}{0.232815,0.732247,0.459277}%
\pgfsetfillcolor{currentfill}%
\pgfsetfillopacity{0.700000}%
\pgfsetlinewidth{0.501875pt}%
\definecolor{currentstroke}{rgb}{1.000000,1.000000,1.000000}%
\pgfsetstrokecolor{currentstroke}%
\pgfsetstrokeopacity{0.500000}%
\pgfsetdash{}{0pt}%
\pgfpathmoveto{\pgfqpoint{3.831183in}{2.828869in}}%
\pgfpathlineto{\pgfqpoint{3.842324in}{2.832327in}}%
\pgfpathlineto{\pgfqpoint{3.853459in}{2.835789in}}%
\pgfpathlineto{\pgfqpoint{3.864589in}{2.839252in}}%
\pgfpathlineto{\pgfqpoint{3.875714in}{2.842712in}}%
\pgfpathlineto{\pgfqpoint{3.886833in}{2.846165in}}%
\pgfpathlineto{\pgfqpoint{3.880464in}{2.859777in}}%
\pgfpathlineto{\pgfqpoint{3.874098in}{2.873401in}}%
\pgfpathlineto{\pgfqpoint{3.867734in}{2.887046in}}%
\pgfpathlineto{\pgfqpoint{3.861374in}{2.900720in}}%
\pgfpathlineto{\pgfqpoint{3.855017in}{2.914422in}}%
\pgfpathlineto{\pgfqpoint{3.843900in}{2.910942in}}%
\pgfpathlineto{\pgfqpoint{3.832778in}{2.907453in}}%
\pgfpathlineto{\pgfqpoint{3.821650in}{2.903956in}}%
\pgfpathlineto{\pgfqpoint{3.810517in}{2.900455in}}%
\pgfpathlineto{\pgfqpoint{3.799378in}{2.896950in}}%
\pgfpathlineto{\pgfqpoint{3.805734in}{2.883327in}}%
\pgfpathlineto{\pgfqpoint{3.812092in}{2.869708in}}%
\pgfpathlineto{\pgfqpoint{3.818453in}{2.856095in}}%
\pgfpathlineto{\pgfqpoint{3.824817in}{2.842484in}}%
\pgfpathclose%
\pgfusepath{stroke,fill}%
\end{pgfscope}%
\begin{pgfscope}%
\pgfpathrectangle{\pgfqpoint{0.887500in}{0.275000in}}{\pgfqpoint{4.225000in}{4.225000in}}%
\pgfusepath{clip}%
\pgfsetbuttcap%
\pgfsetroundjoin%
\definecolor{currentfill}{rgb}{0.154815,0.493313,0.557840}%
\pgfsetfillcolor{currentfill}%
\pgfsetfillopacity{0.700000}%
\pgfsetlinewidth{0.501875pt}%
\definecolor{currentstroke}{rgb}{1.000000,1.000000,1.000000}%
\pgfsetstrokecolor{currentstroke}%
\pgfsetstrokeopacity{0.500000}%
\pgfsetdash{}{0pt}%
\pgfpathmoveto{\pgfqpoint{4.507309in}{2.316478in}}%
\pgfpathlineto{\pgfqpoint{4.518285in}{2.320044in}}%
\pgfpathlineto{\pgfqpoint{4.529256in}{2.323595in}}%
\pgfpathlineto{\pgfqpoint{4.540222in}{2.327140in}}%
\pgfpathlineto{\pgfqpoint{4.551182in}{2.330683in}}%
\pgfpathlineto{\pgfqpoint{4.544693in}{2.344723in}}%
\pgfpathlineto{\pgfqpoint{4.538205in}{2.358730in}}%
\pgfpathlineto{\pgfqpoint{4.531718in}{2.372715in}}%
\pgfpathlineto{\pgfqpoint{4.525234in}{2.386687in}}%
\pgfpathlineto{\pgfqpoint{4.518753in}{2.400657in}}%
\pgfpathlineto{\pgfqpoint{4.507791in}{2.397015in}}%
\pgfpathlineto{\pgfqpoint{4.496825in}{2.393393in}}%
\pgfpathlineto{\pgfqpoint{4.485854in}{2.389789in}}%
\pgfpathlineto{\pgfqpoint{4.474878in}{2.386199in}}%
\pgfpathlineto{\pgfqpoint{4.481359in}{2.372248in}}%
\pgfpathlineto{\pgfqpoint{4.487842in}{2.358314in}}%
\pgfpathlineto{\pgfqpoint{4.494329in}{2.344382in}}%
\pgfpathlineto{\pgfqpoint{4.500818in}{2.330441in}}%
\pgfpathclose%
\pgfusepath{stroke,fill}%
\end{pgfscope}%
\begin{pgfscope}%
\pgfpathrectangle{\pgfqpoint{0.887500in}{0.275000in}}{\pgfqpoint{4.225000in}{4.225000in}}%
\pgfusepath{clip}%
\pgfsetbuttcap%
\pgfsetroundjoin%
\definecolor{currentfill}{rgb}{0.281477,0.755203,0.432552}%
\pgfsetfillcolor{currentfill}%
\pgfsetfillopacity{0.700000}%
\pgfsetlinewidth{0.501875pt}%
\definecolor{currentstroke}{rgb}{1.000000,1.000000,1.000000}%
\pgfsetstrokecolor{currentstroke}%
\pgfsetstrokeopacity{0.500000}%
\pgfsetdash{}{0pt}%
\pgfpathmoveto{\pgfqpoint{3.743607in}{2.879470in}}%
\pgfpathlineto{\pgfqpoint{3.754772in}{2.882954in}}%
\pgfpathlineto{\pgfqpoint{3.765931in}{2.886445in}}%
\pgfpathlineto{\pgfqpoint{3.777085in}{2.889944in}}%
\pgfpathlineto{\pgfqpoint{3.788235in}{2.893446in}}%
\pgfpathlineto{\pgfqpoint{3.799378in}{2.896950in}}%
\pgfpathlineto{\pgfqpoint{3.793025in}{2.910566in}}%
\pgfpathlineto{\pgfqpoint{3.786674in}{2.924159in}}%
\pgfpathlineto{\pgfqpoint{3.780325in}{2.937717in}}%
\pgfpathlineto{\pgfqpoint{3.773978in}{2.951224in}}%
\pgfpathlineto{\pgfqpoint{3.767632in}{2.964667in}}%
\pgfpathlineto{\pgfqpoint{3.756489in}{2.961029in}}%
\pgfpathlineto{\pgfqpoint{3.745341in}{2.957382in}}%
\pgfpathlineto{\pgfqpoint{3.734187in}{2.953728in}}%
\pgfpathlineto{\pgfqpoint{3.723028in}{2.950075in}}%
\pgfpathlineto{\pgfqpoint{3.711864in}{2.946433in}}%
\pgfpathlineto{\pgfqpoint{3.718209in}{2.933110in}}%
\pgfpathlineto{\pgfqpoint{3.724555in}{2.919750in}}%
\pgfpathlineto{\pgfqpoint{3.730904in}{2.906356in}}%
\pgfpathlineto{\pgfqpoint{3.737254in}{2.892928in}}%
\pgfpathclose%
\pgfusepath{stroke,fill}%
\end{pgfscope}%
\begin{pgfscope}%
\pgfpathrectangle{\pgfqpoint{0.887500in}{0.275000in}}{\pgfqpoint{4.225000in}{4.225000in}}%
\pgfusepath{clip}%
\pgfsetbuttcap%
\pgfsetroundjoin%
\definecolor{currentfill}{rgb}{0.121380,0.629492,0.531973}%
\pgfsetfillcolor{currentfill}%
\pgfsetfillopacity{0.700000}%
\pgfsetlinewidth{0.501875pt}%
\definecolor{currentstroke}{rgb}{1.000000,1.000000,1.000000}%
\pgfsetstrokecolor{currentstroke}%
\pgfsetstrokeopacity{0.500000}%
\pgfsetdash{}{0pt}%
\pgfpathmoveto{\pgfqpoint{4.125628in}{2.602489in}}%
\pgfpathlineto{\pgfqpoint{4.136697in}{2.605895in}}%
\pgfpathlineto{\pgfqpoint{4.147761in}{2.609313in}}%
\pgfpathlineto{\pgfqpoint{4.158820in}{2.612743in}}%
\pgfpathlineto{\pgfqpoint{4.169874in}{2.616184in}}%
\pgfpathlineto{\pgfqpoint{4.180923in}{2.619634in}}%
\pgfpathlineto{\pgfqpoint{4.174507in}{2.633755in}}%
\pgfpathlineto{\pgfqpoint{4.168094in}{2.647860in}}%
\pgfpathlineto{\pgfqpoint{4.161682in}{2.661949in}}%
\pgfpathlineto{\pgfqpoint{4.155273in}{2.676019in}}%
\pgfpathlineto{\pgfqpoint{4.148866in}{2.690072in}}%
\pgfpathlineto{\pgfqpoint{4.137819in}{2.686623in}}%
\pgfpathlineto{\pgfqpoint{4.126767in}{2.683180in}}%
\pgfpathlineto{\pgfqpoint{4.115710in}{2.679745in}}%
\pgfpathlineto{\pgfqpoint{4.104648in}{2.676317in}}%
\pgfpathlineto{\pgfqpoint{4.093580in}{2.672898in}}%
\pgfpathlineto{\pgfqpoint{4.099985in}{2.658829in}}%
\pgfpathlineto{\pgfqpoint{4.106392in}{2.644751in}}%
\pgfpathlineto{\pgfqpoint{4.112802in}{2.630669in}}%
\pgfpathlineto{\pgfqpoint{4.119214in}{2.616582in}}%
\pgfpathclose%
\pgfusepath{stroke,fill}%
\end{pgfscope}%
\begin{pgfscope}%
\pgfpathrectangle{\pgfqpoint{0.887500in}{0.275000in}}{\pgfqpoint{4.225000in}{4.225000in}}%
\pgfusepath{clip}%
\pgfsetbuttcap%
\pgfsetroundjoin%
\definecolor{currentfill}{rgb}{0.143343,0.522773,0.556295}%
\pgfsetfillcolor{currentfill}%
\pgfsetfillopacity{0.700000}%
\pgfsetlinewidth{0.501875pt}%
\definecolor{currentstroke}{rgb}{1.000000,1.000000,1.000000}%
\pgfsetstrokecolor{currentstroke}%
\pgfsetstrokeopacity{0.500000}%
\pgfsetdash{}{0pt}%
\pgfpathmoveto{\pgfqpoint{4.419926in}{2.368373in}}%
\pgfpathlineto{\pgfqpoint{4.430927in}{2.371931in}}%
\pgfpathlineto{\pgfqpoint{4.441922in}{2.375490in}}%
\pgfpathlineto{\pgfqpoint{4.452912in}{2.379052in}}%
\pgfpathlineto{\pgfqpoint{4.463898in}{2.382621in}}%
\pgfpathlineto{\pgfqpoint{4.474878in}{2.386199in}}%
\pgfpathlineto{\pgfqpoint{4.468401in}{2.400179in}}%
\pgfpathlineto{\pgfqpoint{4.461929in}{2.414200in}}%
\pgfpathlineto{\pgfqpoint{4.455460in}{2.428272in}}%
\pgfpathlineto{\pgfqpoint{4.448996in}{2.442392in}}%
\pgfpathlineto{\pgfqpoint{4.442537in}{2.456554in}}%
\pgfpathlineto{\pgfqpoint{4.431561in}{2.453065in}}%
\pgfpathlineto{\pgfqpoint{4.420581in}{2.449598in}}%
\pgfpathlineto{\pgfqpoint{4.409597in}{2.446153in}}%
\pgfpathlineto{\pgfqpoint{4.398608in}{2.442731in}}%
\pgfpathlineto{\pgfqpoint{4.387614in}{2.439329in}}%
\pgfpathlineto{\pgfqpoint{4.394068in}{2.425053in}}%
\pgfpathlineto{\pgfqpoint{4.400527in}{2.410814in}}%
\pgfpathlineto{\pgfqpoint{4.406989in}{2.396620in}}%
\pgfpathlineto{\pgfqpoint{4.413455in}{2.382475in}}%
\pgfpathclose%
\pgfusepath{stroke,fill}%
\end{pgfscope}%
\begin{pgfscope}%
\pgfpathrectangle{\pgfqpoint{0.887500in}{0.275000in}}{\pgfqpoint{4.225000in}{4.225000in}}%
\pgfusepath{clip}%
\pgfsetbuttcap%
\pgfsetroundjoin%
\definecolor{currentfill}{rgb}{0.129933,0.559582,0.551864}%
\pgfsetfillcolor{currentfill}%
\pgfsetfillopacity{0.700000}%
\pgfsetlinewidth{0.501875pt}%
\definecolor{currentstroke}{rgb}{1.000000,1.000000,1.000000}%
\pgfsetstrokecolor{currentstroke}%
\pgfsetstrokeopacity{0.500000}%
\pgfsetdash{}{0pt}%
\pgfpathmoveto{\pgfqpoint{1.752571in}{2.462826in}}%
\pgfpathlineto{\pgfqpoint{1.764226in}{2.466161in}}%
\pgfpathlineto{\pgfqpoint{1.775876in}{2.469494in}}%
\pgfpathlineto{\pgfqpoint{1.787520in}{2.472822in}}%
\pgfpathlineto{\pgfqpoint{1.799158in}{2.476143in}}%
\pgfpathlineto{\pgfqpoint{1.810791in}{2.479455in}}%
\pgfpathlineto{\pgfqpoint{1.805002in}{2.487633in}}%
\pgfpathlineto{\pgfqpoint{1.799218in}{2.495793in}}%
\pgfpathlineto{\pgfqpoint{1.793438in}{2.503935in}}%
\pgfpathlineto{\pgfqpoint{1.787662in}{2.512059in}}%
\pgfpathlineto{\pgfqpoint{1.781891in}{2.520167in}}%
\pgfpathlineto{\pgfqpoint{1.770270in}{2.516873in}}%
\pgfpathlineto{\pgfqpoint{1.758644in}{2.513570in}}%
\pgfpathlineto{\pgfqpoint{1.747012in}{2.510261in}}%
\pgfpathlineto{\pgfqpoint{1.735375in}{2.506947in}}%
\pgfpathlineto{\pgfqpoint{1.723732in}{2.503631in}}%
\pgfpathlineto{\pgfqpoint{1.729491in}{2.495505in}}%
\pgfpathlineto{\pgfqpoint{1.735255in}{2.487362in}}%
\pgfpathlineto{\pgfqpoint{1.741022in}{2.479202in}}%
\pgfpathlineto{\pgfqpoint{1.746794in}{2.471024in}}%
\pgfpathclose%
\pgfusepath{stroke,fill}%
\end{pgfscope}%
\begin{pgfscope}%
\pgfpathrectangle{\pgfqpoint{0.887500in}{0.275000in}}{\pgfqpoint{4.225000in}{4.225000in}}%
\pgfusepath{clip}%
\pgfsetbuttcap%
\pgfsetroundjoin%
\definecolor{currentfill}{rgb}{0.140536,0.530132,0.555659}%
\pgfsetfillcolor{currentfill}%
\pgfsetfillopacity{0.700000}%
\pgfsetlinewidth{0.501875pt}%
\definecolor{currentstroke}{rgb}{1.000000,1.000000,1.000000}%
\pgfsetstrokecolor{currentstroke}%
\pgfsetstrokeopacity{0.500000}%
\pgfsetdash{}{0pt}%
\pgfpathmoveto{\pgfqpoint{2.072505in}{2.404340in}}%
\pgfpathlineto{\pgfqpoint{2.084082in}{2.407699in}}%
\pgfpathlineto{\pgfqpoint{2.095654in}{2.411065in}}%
\pgfpathlineto{\pgfqpoint{2.107219in}{2.414434in}}%
\pgfpathlineto{\pgfqpoint{2.118779in}{2.417804in}}%
\pgfpathlineto{\pgfqpoint{2.130333in}{2.421170in}}%
\pgfpathlineto{\pgfqpoint{2.124432in}{2.429559in}}%
\pgfpathlineto{\pgfqpoint{2.118535in}{2.437930in}}%
\pgfpathlineto{\pgfqpoint{2.112642in}{2.446284in}}%
\pgfpathlineto{\pgfqpoint{2.106753in}{2.454619in}}%
\pgfpathlineto{\pgfqpoint{2.100869in}{2.462937in}}%
\pgfpathlineto{\pgfqpoint{2.089327in}{2.459590in}}%
\pgfpathlineto{\pgfqpoint{2.077779in}{2.456239in}}%
\pgfpathlineto{\pgfqpoint{2.066225in}{2.452888in}}%
\pgfpathlineto{\pgfqpoint{2.054666in}{2.449541in}}%
\pgfpathlineto{\pgfqpoint{2.043101in}{2.446202in}}%
\pgfpathlineto{\pgfqpoint{2.048973in}{2.437868in}}%
\pgfpathlineto{\pgfqpoint{2.054850in}{2.429515in}}%
\pgfpathlineto{\pgfqpoint{2.060731in}{2.421143in}}%
\pgfpathlineto{\pgfqpoint{2.066616in}{2.412751in}}%
\pgfpathclose%
\pgfusepath{stroke,fill}%
\end{pgfscope}%
\begin{pgfscope}%
\pgfpathrectangle{\pgfqpoint{0.887500in}{0.275000in}}{\pgfqpoint{4.225000in}{4.225000in}}%
\pgfusepath{clip}%
\pgfsetbuttcap%
\pgfsetroundjoin%
\definecolor{currentfill}{rgb}{0.468053,0.818921,0.323998}%
\pgfsetfillcolor{currentfill}%
\pgfsetfillopacity{0.700000}%
\pgfsetlinewidth{0.501875pt}%
\definecolor{currentstroke}{rgb}{1.000000,1.000000,1.000000}%
\pgfsetstrokecolor{currentstroke}%
\pgfsetstrokeopacity{0.500000}%
\pgfsetdash{}{0pt}%
\pgfpathmoveto{\pgfqpoint{3.016122in}{3.037312in}}%
\pgfpathlineto{\pgfqpoint{3.027447in}{3.057760in}}%
\pgfpathlineto{\pgfqpoint{3.038777in}{3.076200in}}%
\pgfpathlineto{\pgfqpoint{3.050110in}{3.092861in}}%
\pgfpathlineto{\pgfqpoint{3.061445in}{3.107973in}}%
\pgfpathlineto{\pgfqpoint{3.072781in}{3.121767in}}%
\pgfpathlineto{\pgfqpoint{3.066560in}{3.129456in}}%
\pgfpathlineto{\pgfqpoint{3.060344in}{3.136867in}}%
\pgfpathlineto{\pgfqpoint{3.054132in}{3.144024in}}%
\pgfpathlineto{\pgfqpoint{3.047924in}{3.150954in}}%
\pgfpathlineto{\pgfqpoint{3.041721in}{3.157681in}}%
\pgfpathlineto{\pgfqpoint{3.030410in}{3.137658in}}%
\pgfpathlineto{\pgfqpoint{3.019105in}{3.116037in}}%
\pgfpathlineto{\pgfqpoint{3.007806in}{3.092868in}}%
\pgfpathlineto{\pgfqpoint{2.996515in}{3.068200in}}%
\pgfpathlineto{\pgfqpoint{2.985234in}{3.042082in}}%
\pgfpathlineto{\pgfqpoint{2.991398in}{3.041649in}}%
\pgfpathlineto{\pgfqpoint{2.997569in}{3.041091in}}%
\pgfpathlineto{\pgfqpoint{3.003746in}{3.040272in}}%
\pgfpathlineto{\pgfqpoint{3.009931in}{3.039057in}}%
\pgfpathclose%
\pgfusepath{stroke,fill}%
\end{pgfscope}%
\begin{pgfscope}%
\pgfpathrectangle{\pgfqpoint{0.887500in}{0.275000in}}{\pgfqpoint{4.225000in}{4.225000in}}%
\pgfusepath{clip}%
\pgfsetbuttcap%
\pgfsetroundjoin%
\definecolor{currentfill}{rgb}{0.150476,0.504369,0.557430}%
\pgfsetfillcolor{currentfill}%
\pgfsetfillopacity{0.700000}%
\pgfsetlinewidth{0.501875pt}%
\definecolor{currentstroke}{rgb}{1.000000,1.000000,1.000000}%
\pgfsetstrokecolor{currentstroke}%
\pgfsetstrokeopacity{0.500000}%
\pgfsetdash{}{0pt}%
\pgfpathmoveto{\pgfqpoint{2.392524in}{2.343535in}}%
\pgfpathlineto{\pgfqpoint{2.404023in}{2.346862in}}%
\pgfpathlineto{\pgfqpoint{2.415516in}{2.350209in}}%
\pgfpathlineto{\pgfqpoint{2.427004in}{2.353580in}}%
\pgfpathlineto{\pgfqpoint{2.438485in}{2.356981in}}%
\pgfpathlineto{\pgfqpoint{2.449959in}{2.360418in}}%
\pgfpathlineto{\pgfqpoint{2.443949in}{2.368997in}}%
\pgfpathlineto{\pgfqpoint{2.437943in}{2.377561in}}%
\pgfpathlineto{\pgfqpoint{2.431942in}{2.386110in}}%
\pgfpathlineto{\pgfqpoint{2.425944in}{2.394645in}}%
\pgfpathlineto{\pgfqpoint{2.419950in}{2.403165in}}%
\pgfpathlineto{\pgfqpoint{2.408487in}{2.399739in}}%
\pgfpathlineto{\pgfqpoint{2.397018in}{2.396347in}}%
\pgfpathlineto{\pgfqpoint{2.385542in}{2.392984in}}%
\pgfpathlineto{\pgfqpoint{2.374060in}{2.389646in}}%
\pgfpathlineto{\pgfqpoint{2.362573in}{2.386326in}}%
\pgfpathlineto{\pgfqpoint{2.368555in}{2.377802in}}%
\pgfpathlineto{\pgfqpoint{2.374541in}{2.369260in}}%
\pgfpathlineto{\pgfqpoint{2.380531in}{2.360702in}}%
\pgfpathlineto{\pgfqpoint{2.386525in}{2.352127in}}%
\pgfpathclose%
\pgfusepath{stroke,fill}%
\end{pgfscope}%
\begin{pgfscope}%
\pgfpathrectangle{\pgfqpoint{0.887500in}{0.275000in}}{\pgfqpoint{4.225000in}{4.225000in}}%
\pgfusepath{clip}%
\pgfsetbuttcap%
\pgfsetroundjoin%
\definecolor{currentfill}{rgb}{0.162142,0.474838,0.558140}%
\pgfsetfillcolor{currentfill}%
\pgfsetfillopacity{0.700000}%
\pgfsetlinewidth{0.501875pt}%
\definecolor{currentstroke}{rgb}{1.000000,1.000000,1.000000}%
\pgfsetstrokecolor{currentstroke}%
\pgfsetstrokeopacity{0.500000}%
\pgfsetdash{}{0pt}%
\pgfpathmoveto{\pgfqpoint{2.712575in}{2.279885in}}%
\pgfpathlineto{\pgfqpoint{2.723988in}{2.284045in}}%
\pgfpathlineto{\pgfqpoint{2.735393in}{2.288507in}}%
\pgfpathlineto{\pgfqpoint{2.746793in}{2.292973in}}%
\pgfpathlineto{\pgfqpoint{2.758190in}{2.297140in}}%
\pgfpathlineto{\pgfqpoint{2.769586in}{2.300702in}}%
\pgfpathlineto{\pgfqpoint{2.763469in}{2.309562in}}%
\pgfpathlineto{\pgfqpoint{2.757356in}{2.318406in}}%
\pgfpathlineto{\pgfqpoint{2.751247in}{2.327239in}}%
\pgfpathlineto{\pgfqpoint{2.745142in}{2.336066in}}%
\pgfpathlineto{\pgfqpoint{2.739041in}{2.344892in}}%
\pgfpathlineto{\pgfqpoint{2.727655in}{2.341510in}}%
\pgfpathlineto{\pgfqpoint{2.716269in}{2.337360in}}%
\pgfpathlineto{\pgfqpoint{2.704882in}{2.332807in}}%
\pgfpathlineto{\pgfqpoint{2.693490in}{2.328219in}}%
\pgfpathlineto{\pgfqpoint{2.682090in}{2.323953in}}%
\pgfpathlineto{\pgfqpoint{2.688179in}{2.315187in}}%
\pgfpathlineto{\pgfqpoint{2.694272in}{2.306398in}}%
\pgfpathlineto{\pgfqpoint{2.700369in}{2.297585in}}%
\pgfpathlineto{\pgfqpoint{2.706470in}{2.288747in}}%
\pgfpathclose%
\pgfusepath{stroke,fill}%
\end{pgfscope}%
\begin{pgfscope}%
\pgfpathrectangle{\pgfqpoint{0.887500in}{0.275000in}}{\pgfqpoint{4.225000in}{4.225000in}}%
\pgfusepath{clip}%
\pgfsetbuttcap%
\pgfsetroundjoin%
\definecolor{currentfill}{rgb}{0.327796,0.773980,0.406640}%
\pgfsetfillcolor{currentfill}%
\pgfsetfillopacity{0.700000}%
\pgfsetlinewidth{0.501875pt}%
\definecolor{currentstroke}{rgb}{1.000000,1.000000,1.000000}%
\pgfsetstrokecolor{currentstroke}%
\pgfsetstrokeopacity{0.500000}%
\pgfsetdash{}{0pt}%
\pgfpathmoveto{\pgfqpoint{3.655974in}{2.928729in}}%
\pgfpathlineto{\pgfqpoint{3.667161in}{2.932172in}}%
\pgfpathlineto{\pgfqpoint{3.678344in}{2.935671in}}%
\pgfpathlineto{\pgfqpoint{3.689522in}{2.939221in}}%
\pgfpathlineto{\pgfqpoint{3.700695in}{2.942811in}}%
\pgfpathlineto{\pgfqpoint{3.711864in}{2.946433in}}%
\pgfpathlineto{\pgfqpoint{3.705522in}{2.959717in}}%
\pgfpathlineto{\pgfqpoint{3.699182in}{2.972959in}}%
\pgfpathlineto{\pgfqpoint{3.692843in}{2.986159in}}%
\pgfpathlineto{\pgfqpoint{3.686507in}{2.999314in}}%
\pgfpathlineto{\pgfqpoint{3.680173in}{3.012421in}}%
\pgfpathlineto{\pgfqpoint{3.669009in}{3.008870in}}%
\pgfpathlineto{\pgfqpoint{3.657840in}{3.005348in}}%
\pgfpathlineto{\pgfqpoint{3.646666in}{3.001861in}}%
\pgfpathlineto{\pgfqpoint{3.635488in}{2.998414in}}%
\pgfpathlineto{\pgfqpoint{3.624305in}{2.995009in}}%
\pgfpathlineto{\pgfqpoint{3.630633in}{2.981737in}}%
\pgfpathlineto{\pgfqpoint{3.636964in}{2.968477in}}%
\pgfpathlineto{\pgfqpoint{3.643298in}{2.955226in}}%
\pgfpathlineto{\pgfqpoint{3.649634in}{2.941979in}}%
\pgfpathclose%
\pgfusepath{stroke,fill}%
\end{pgfscope}%
\begin{pgfscope}%
\pgfpathrectangle{\pgfqpoint{0.887500in}{0.275000in}}{\pgfqpoint{4.225000in}{4.225000in}}%
\pgfusepath{clip}%
\pgfsetbuttcap%
\pgfsetroundjoin%
\definecolor{currentfill}{rgb}{0.636902,0.856542,0.216620}%
\pgfsetfillcolor{currentfill}%
\pgfsetfillopacity{0.700000}%
\pgfsetlinewidth{0.501875pt}%
\definecolor{currentstroke}{rgb}{1.000000,1.000000,1.000000}%
\pgfsetstrokecolor{currentstroke}%
\pgfsetstrokeopacity{0.500000}%
\pgfsetdash{}{0pt}%
\pgfpathmoveto{\pgfqpoint{3.129441in}{3.177455in}}%
\pgfpathlineto{\pgfqpoint{3.140765in}{3.186266in}}%
\pgfpathlineto{\pgfqpoint{3.152085in}{3.194243in}}%
\pgfpathlineto{\pgfqpoint{3.163399in}{3.201358in}}%
\pgfpathlineto{\pgfqpoint{3.174708in}{3.207583in}}%
\pgfpathlineto{\pgfqpoint{3.186010in}{3.212971in}}%
\pgfpathlineto{\pgfqpoint{3.179769in}{3.224667in}}%
\pgfpathlineto{\pgfqpoint{3.173530in}{3.236294in}}%
\pgfpathlineto{\pgfqpoint{3.167295in}{3.247833in}}%
\pgfpathlineto{\pgfqpoint{3.161062in}{3.259267in}}%
\pgfpathlineto{\pgfqpoint{3.154833in}{3.270577in}}%
\pgfpathlineto{\pgfqpoint{3.143539in}{3.265583in}}%
\pgfpathlineto{\pgfqpoint{3.132239in}{3.259423in}}%
\pgfpathlineto{\pgfqpoint{3.120934in}{3.251962in}}%
\pgfpathlineto{\pgfqpoint{3.109623in}{3.243129in}}%
\pgfpathlineto{\pgfqpoint{3.098308in}{3.232857in}}%
\pgfpathlineto{\pgfqpoint{3.104529in}{3.222077in}}%
\pgfpathlineto{\pgfqpoint{3.110752in}{3.211124in}}%
\pgfpathlineto{\pgfqpoint{3.116979in}{3.200019in}}%
\pgfpathlineto{\pgfqpoint{3.123209in}{3.188789in}}%
\pgfpathclose%
\pgfusepath{stroke,fill}%
\end{pgfscope}%
\begin{pgfscope}%
\pgfpathrectangle{\pgfqpoint{0.887500in}{0.275000in}}{\pgfqpoint{4.225000in}{4.225000in}}%
\pgfusepath{clip}%
\pgfsetbuttcap%
\pgfsetroundjoin%
\definecolor{currentfill}{rgb}{0.134692,0.658636,0.517649}%
\pgfsetfillcolor{currentfill}%
\pgfsetfillopacity{0.700000}%
\pgfsetlinewidth{0.501875pt}%
\definecolor{currentstroke}{rgb}{1.000000,1.000000,1.000000}%
\pgfsetstrokecolor{currentstroke}%
\pgfsetstrokeopacity{0.500000}%
\pgfsetdash{}{0pt}%
\pgfpathmoveto{\pgfqpoint{4.038167in}{2.655945in}}%
\pgfpathlineto{\pgfqpoint{4.049260in}{2.659316in}}%
\pgfpathlineto{\pgfqpoint{4.060348in}{2.662696in}}%
\pgfpathlineto{\pgfqpoint{4.071430in}{2.666087in}}%
\pgfpathlineto{\pgfqpoint{4.082508in}{2.669488in}}%
\pgfpathlineto{\pgfqpoint{4.093580in}{2.672898in}}%
\pgfpathlineto{\pgfqpoint{4.087178in}{2.686955in}}%
\pgfpathlineto{\pgfqpoint{4.080777in}{2.700997in}}%
\pgfpathlineto{\pgfqpoint{4.074379in}{2.715020in}}%
\pgfpathlineto{\pgfqpoint{4.067983in}{2.729020in}}%
\pgfpathlineto{\pgfqpoint{4.061588in}{2.742995in}}%
\pgfpathlineto{\pgfqpoint{4.050518in}{2.739597in}}%
\pgfpathlineto{\pgfqpoint{4.039443in}{2.736203in}}%
\pgfpathlineto{\pgfqpoint{4.028362in}{2.732812in}}%
\pgfpathlineto{\pgfqpoint{4.017277in}{2.729422in}}%
\pgfpathlineto{\pgfqpoint{4.006185in}{2.726031in}}%
\pgfpathlineto{\pgfqpoint{4.012578in}{2.712065in}}%
\pgfpathlineto{\pgfqpoint{4.018972in}{2.698067in}}%
\pgfpathlineto{\pgfqpoint{4.025368in}{2.684044in}}%
\pgfpathlineto{\pgfqpoint{4.031767in}{2.670002in}}%
\pgfpathclose%
\pgfusepath{stroke,fill}%
\end{pgfscope}%
\begin{pgfscope}%
\pgfpathrectangle{\pgfqpoint{0.887500in}{0.275000in}}{\pgfqpoint{4.225000in}{4.225000in}}%
\pgfusepath{clip}%
\pgfsetbuttcap%
\pgfsetroundjoin%
\definecolor{currentfill}{rgb}{0.386433,0.794644,0.372886}%
\pgfsetfillcolor{currentfill}%
\pgfsetfillopacity{0.700000}%
\pgfsetlinewidth{0.501875pt}%
\definecolor{currentstroke}{rgb}{1.000000,1.000000,1.000000}%
\pgfsetstrokecolor{currentstroke}%
\pgfsetstrokeopacity{0.500000}%
\pgfsetdash{}{0pt}%
\pgfpathmoveto{\pgfqpoint{3.568312in}{2.978219in}}%
\pgfpathlineto{\pgfqpoint{3.579522in}{2.981590in}}%
\pgfpathlineto{\pgfqpoint{3.590726in}{2.984941in}}%
\pgfpathlineto{\pgfqpoint{3.601924in}{2.988286in}}%
\pgfpathlineto{\pgfqpoint{3.613117in}{2.991637in}}%
\pgfpathlineto{\pgfqpoint{3.624305in}{2.995009in}}%
\pgfpathlineto{\pgfqpoint{3.617980in}{3.008280in}}%
\pgfpathlineto{\pgfqpoint{3.611657in}{3.021527in}}%
\pgfpathlineto{\pgfqpoint{3.605336in}{3.034726in}}%
\pgfpathlineto{\pgfqpoint{3.599016in}{3.047855in}}%
\pgfpathlineto{\pgfqpoint{3.592698in}{3.060891in}}%
\pgfpathlineto{\pgfqpoint{3.581515in}{3.057561in}}%
\pgfpathlineto{\pgfqpoint{3.570325in}{3.054186in}}%
\pgfpathlineto{\pgfqpoint{3.559129in}{3.050768in}}%
\pgfpathlineto{\pgfqpoint{3.547928in}{3.047308in}}%
\pgfpathlineto{\pgfqpoint{3.536720in}{3.043809in}}%
\pgfpathlineto{\pgfqpoint{3.543035in}{3.030809in}}%
\pgfpathlineto{\pgfqpoint{3.549351in}{3.017737in}}%
\pgfpathlineto{\pgfqpoint{3.555669in}{3.004607in}}%
\pgfpathlineto{\pgfqpoint{3.561989in}{2.991430in}}%
\pgfpathclose%
\pgfusepath{stroke,fill}%
\end{pgfscope}%
\begin{pgfscope}%
\pgfpathrectangle{\pgfqpoint{0.887500in}{0.275000in}}{\pgfqpoint{4.225000in}{4.225000in}}%
\pgfusepath{clip}%
\pgfsetbuttcap%
\pgfsetroundjoin%
\definecolor{currentfill}{rgb}{0.133743,0.548535,0.553541}%
\pgfsetfillcolor{currentfill}%
\pgfsetfillopacity{0.700000}%
\pgfsetlinewidth{0.501875pt}%
\definecolor{currentstroke}{rgb}{1.000000,1.000000,1.000000}%
\pgfsetstrokecolor{currentstroke}%
\pgfsetstrokeopacity{0.500000}%
\pgfsetdash{}{0pt}%
\pgfpathmoveto{\pgfqpoint{4.332576in}{2.422623in}}%
\pgfpathlineto{\pgfqpoint{4.343593in}{2.425926in}}%
\pgfpathlineto{\pgfqpoint{4.354605in}{2.429247in}}%
\pgfpathlineto{\pgfqpoint{4.365613in}{2.432588in}}%
\pgfpathlineto{\pgfqpoint{4.376616in}{2.435948in}}%
\pgfpathlineto{\pgfqpoint{4.387614in}{2.439329in}}%
\pgfpathlineto{\pgfqpoint{4.381163in}{2.453636in}}%
\pgfpathlineto{\pgfqpoint{4.374716in}{2.467968in}}%
\pgfpathlineto{\pgfqpoint{4.368271in}{2.482319in}}%
\pgfpathlineto{\pgfqpoint{4.361830in}{2.496681in}}%
\pgfpathlineto{\pgfqpoint{4.355391in}{2.511047in}}%
\pgfpathlineto{\pgfqpoint{4.344396in}{2.507716in}}%
\pgfpathlineto{\pgfqpoint{4.333395in}{2.504391in}}%
\pgfpathlineto{\pgfqpoint{4.322390in}{2.501072in}}%
\pgfpathlineto{\pgfqpoint{4.311379in}{2.497758in}}%
\pgfpathlineto{\pgfqpoint{4.300363in}{2.494448in}}%
\pgfpathlineto{\pgfqpoint{4.306801in}{2.480110in}}%
\pgfpathlineto{\pgfqpoint{4.313242in}{2.465759in}}%
\pgfpathlineto{\pgfqpoint{4.319684in}{2.451393in}}%
\pgfpathlineto{\pgfqpoint{4.326129in}{2.437015in}}%
\pgfpathclose%
\pgfusepath{stroke,fill}%
\end{pgfscope}%
\begin{pgfscope}%
\pgfpathrectangle{\pgfqpoint{0.887500in}{0.275000in}}{\pgfqpoint{4.225000in}{4.225000in}}%
\pgfusepath{clip}%
\pgfsetbuttcap%
\pgfsetroundjoin%
\definecolor{currentfill}{rgb}{0.125394,0.574318,0.549086}%
\pgfsetfillcolor{currentfill}%
\pgfsetfillopacity{0.700000}%
\pgfsetlinewidth{0.501875pt}%
\definecolor{currentstroke}{rgb}{1.000000,1.000000,1.000000}%
\pgfsetstrokecolor{currentstroke}%
\pgfsetstrokeopacity{0.500000}%
\pgfsetdash{}{0pt}%
\pgfpathmoveto{\pgfqpoint{1.519862in}{2.494193in}}%
\pgfpathlineto{\pgfqpoint{1.531576in}{2.497572in}}%
\pgfpathlineto{\pgfqpoint{1.543285in}{2.500939in}}%
\pgfpathlineto{\pgfqpoint{1.554989in}{2.504294in}}%
\pgfpathlineto{\pgfqpoint{1.566687in}{2.507638in}}%
\pgfpathlineto{\pgfqpoint{1.578380in}{2.510973in}}%
\pgfpathlineto{\pgfqpoint{1.572672in}{2.519036in}}%
\pgfpathlineto{\pgfqpoint{1.566968in}{2.527080in}}%
\pgfpathlineto{\pgfqpoint{1.561269in}{2.535102in}}%
\pgfpathlineto{\pgfqpoint{1.555574in}{2.543100in}}%
\pgfpathlineto{\pgfqpoint{1.549884in}{2.551074in}}%
\pgfpathlineto{\pgfqpoint{1.538204in}{2.547752in}}%
\pgfpathlineto{\pgfqpoint{1.526518in}{2.544420in}}%
\pgfpathlineto{\pgfqpoint{1.514828in}{2.541077in}}%
\pgfpathlineto{\pgfqpoint{1.503132in}{2.537723in}}%
\pgfpathlineto{\pgfqpoint{1.491430in}{2.534355in}}%
\pgfpathlineto{\pgfqpoint{1.497107in}{2.526371in}}%
\pgfpathlineto{\pgfqpoint{1.502789in}{2.518361in}}%
\pgfpathlineto{\pgfqpoint{1.508475in}{2.510327in}}%
\pgfpathlineto{\pgfqpoint{1.514166in}{2.502270in}}%
\pgfpathclose%
\pgfusepath{stroke,fill}%
\end{pgfscope}%
\begin{pgfscope}%
\pgfpathrectangle{\pgfqpoint{0.887500in}{0.275000in}}{\pgfqpoint{4.225000in}{4.225000in}}%
\pgfusepath{clip}%
\pgfsetbuttcap%
\pgfsetroundjoin%
\definecolor{currentfill}{rgb}{0.168126,0.459988,0.558082}%
\pgfsetfillcolor{currentfill}%
\pgfsetfillopacity{0.700000}%
\pgfsetlinewidth{0.501875pt}%
\definecolor{currentstroke}{rgb}{1.000000,1.000000,1.000000}%
\pgfsetstrokecolor{currentstroke}%
\pgfsetstrokeopacity{0.500000}%
\pgfsetdash{}{0pt}%
\pgfpathmoveto{\pgfqpoint{2.800231in}{2.256064in}}%
\pgfpathlineto{\pgfqpoint{2.811637in}{2.258839in}}%
\pgfpathlineto{\pgfqpoint{2.823045in}{2.260474in}}%
\pgfpathlineto{\pgfqpoint{2.834453in}{2.260745in}}%
\pgfpathlineto{\pgfqpoint{2.845859in}{2.260194in}}%
\pgfpathlineto{\pgfqpoint{2.857255in}{2.259868in}}%
\pgfpathlineto{\pgfqpoint{2.851109in}{2.268909in}}%
\pgfpathlineto{\pgfqpoint{2.844968in}{2.277801in}}%
\pgfpathlineto{\pgfqpoint{2.838831in}{2.286511in}}%
\pgfpathlineto{\pgfqpoint{2.832700in}{2.295005in}}%
\pgfpathlineto{\pgfqpoint{2.826574in}{2.303252in}}%
\pgfpathlineto{\pgfqpoint{2.815184in}{2.303913in}}%
\pgfpathlineto{\pgfqpoint{2.803784in}{2.304796in}}%
\pgfpathlineto{\pgfqpoint{2.792382in}{2.304801in}}%
\pgfpathlineto{\pgfqpoint{2.780983in}{2.303357in}}%
\pgfpathlineto{\pgfqpoint{2.769586in}{2.300702in}}%
\pgfpathlineto{\pgfqpoint{2.775707in}{2.291823in}}%
\pgfpathlineto{\pgfqpoint{2.781832in}{2.282920in}}%
\pgfpathlineto{\pgfqpoint{2.787961in}{2.273994in}}%
\pgfpathlineto{\pgfqpoint{2.794094in}{2.265043in}}%
\pgfpathclose%
\pgfusepath{stroke,fill}%
\end{pgfscope}%
\begin{pgfscope}%
\pgfpathrectangle{\pgfqpoint{0.887500in}{0.275000in}}{\pgfqpoint{4.225000in}{4.225000in}}%
\pgfusepath{clip}%
\pgfsetbuttcap%
\pgfsetroundjoin%
\definecolor{currentfill}{rgb}{0.440137,0.811138,0.340967}%
\pgfsetfillcolor{currentfill}%
\pgfsetfillopacity{0.700000}%
\pgfsetlinewidth{0.501875pt}%
\definecolor{currentstroke}{rgb}{1.000000,1.000000,1.000000}%
\pgfsetstrokecolor{currentstroke}%
\pgfsetstrokeopacity{0.500000}%
\pgfsetdash{}{0pt}%
\pgfpathmoveto{\pgfqpoint{3.480599in}{3.025812in}}%
\pgfpathlineto{\pgfqpoint{3.491834in}{3.029468in}}%
\pgfpathlineto{\pgfqpoint{3.503064in}{3.033099in}}%
\pgfpathlineto{\pgfqpoint{3.514289in}{3.036702in}}%
\pgfpathlineto{\pgfqpoint{3.525507in}{3.040273in}}%
\pgfpathlineto{\pgfqpoint{3.536720in}{3.043809in}}%
\pgfpathlineto{\pgfqpoint{3.530408in}{3.056726in}}%
\pgfpathlineto{\pgfqpoint{3.524097in}{3.069548in}}%
\pgfpathlineto{\pgfqpoint{3.517788in}{3.082262in}}%
\pgfpathlineto{\pgfqpoint{3.511480in}{3.094856in}}%
\pgfpathlineto{\pgfqpoint{3.505173in}{3.107318in}}%
\pgfpathlineto{\pgfqpoint{3.493966in}{3.103880in}}%
\pgfpathlineto{\pgfqpoint{3.482753in}{3.100409in}}%
\pgfpathlineto{\pgfqpoint{3.471534in}{3.096906in}}%
\pgfpathlineto{\pgfqpoint{3.460310in}{3.093375in}}%
\pgfpathlineto{\pgfqpoint{3.449080in}{3.089816in}}%
\pgfpathlineto{\pgfqpoint{3.455379in}{3.077086in}}%
\pgfpathlineto{\pgfqpoint{3.461680in}{3.064322in}}%
\pgfpathlineto{\pgfqpoint{3.467984in}{3.051524in}}%
\pgfpathlineto{\pgfqpoint{3.474290in}{3.038687in}}%
\pgfpathclose%
\pgfusepath{stroke,fill}%
\end{pgfscope}%
\begin{pgfscope}%
\pgfpathrectangle{\pgfqpoint{0.887500in}{0.275000in}}{\pgfqpoint{4.225000in}{4.225000in}}%
\pgfusepath{clip}%
\pgfsetbuttcap%
\pgfsetroundjoin%
\definecolor{currentfill}{rgb}{0.595839,0.848717,0.243329}%
\pgfsetfillcolor{currentfill}%
\pgfsetfillopacity{0.700000}%
\pgfsetlinewidth{0.501875pt}%
\definecolor{currentstroke}{rgb}{1.000000,1.000000,1.000000}%
\pgfsetstrokecolor{currentstroke}%
\pgfsetstrokeopacity{0.500000}%
\pgfsetdash{}{0pt}%
\pgfpathmoveto{\pgfqpoint{3.217260in}{3.154074in}}%
\pgfpathlineto{\pgfqpoint{3.228564in}{3.159223in}}%
\pgfpathlineto{\pgfqpoint{3.239862in}{3.163940in}}%
\pgfpathlineto{\pgfqpoint{3.251154in}{3.168314in}}%
\pgfpathlineto{\pgfqpoint{3.262439in}{3.172435in}}%
\pgfpathlineto{\pgfqpoint{3.273719in}{3.176392in}}%
\pgfpathlineto{\pgfqpoint{3.267454in}{3.187847in}}%
\pgfpathlineto{\pgfqpoint{3.261192in}{3.199131in}}%
\pgfpathlineto{\pgfqpoint{3.254932in}{3.210264in}}%
\pgfpathlineto{\pgfqpoint{3.248675in}{3.221260in}}%
\pgfpathlineto{\pgfqpoint{3.242420in}{3.232138in}}%
\pgfpathlineto{\pgfqpoint{3.231151in}{3.228879in}}%
\pgfpathlineto{\pgfqpoint{3.219875in}{3.225471in}}%
\pgfpathlineto{\pgfqpoint{3.208594in}{3.221776in}}%
\pgfpathlineto{\pgfqpoint{3.197305in}{3.217656in}}%
\pgfpathlineto{\pgfqpoint{3.186010in}{3.212971in}}%
\pgfpathlineto{\pgfqpoint{3.192254in}{3.201223in}}%
\pgfpathlineto{\pgfqpoint{3.198501in}{3.189441in}}%
\pgfpathlineto{\pgfqpoint{3.204751in}{3.177644in}}%
\pgfpathlineto{\pgfqpoint{3.211004in}{3.165849in}}%
\pgfpathclose%
\pgfusepath{stroke,fill}%
\end{pgfscope}%
\begin{pgfscope}%
\pgfpathrectangle{\pgfqpoint{0.887500in}{0.275000in}}{\pgfqpoint{4.225000in}{4.225000in}}%
\pgfusepath{clip}%
\pgfsetbuttcap%
\pgfsetroundjoin%
\definecolor{currentfill}{rgb}{0.133743,0.548535,0.553541}%
\pgfsetfillcolor{currentfill}%
\pgfsetfillopacity{0.700000}%
\pgfsetlinewidth{0.501875pt}%
\definecolor{currentstroke}{rgb}{1.000000,1.000000,1.000000}%
\pgfsetstrokecolor{currentstroke}%
\pgfsetstrokeopacity{0.500000}%
\pgfsetdash{}{0pt}%
\pgfpathmoveto{\pgfqpoint{1.839800in}{2.438255in}}%
\pgfpathlineto{\pgfqpoint{1.851440in}{2.441579in}}%
\pgfpathlineto{\pgfqpoint{1.863074in}{2.444891in}}%
\pgfpathlineto{\pgfqpoint{1.874703in}{2.448194in}}%
\pgfpathlineto{\pgfqpoint{1.886326in}{2.451488in}}%
\pgfpathlineto{\pgfqpoint{1.897944in}{2.454774in}}%
\pgfpathlineto{\pgfqpoint{1.892121in}{2.463031in}}%
\pgfpathlineto{\pgfqpoint{1.886303in}{2.471268in}}%
\pgfpathlineto{\pgfqpoint{1.880489in}{2.479485in}}%
\pgfpathlineto{\pgfqpoint{1.874679in}{2.487683in}}%
\pgfpathlineto{\pgfqpoint{1.868874in}{2.495861in}}%
\pgfpathlineto{\pgfqpoint{1.857268in}{2.492597in}}%
\pgfpathlineto{\pgfqpoint{1.845657in}{2.489325in}}%
\pgfpathlineto{\pgfqpoint{1.834041in}{2.486046in}}%
\pgfpathlineto{\pgfqpoint{1.822419in}{2.482756in}}%
\pgfpathlineto{\pgfqpoint{1.810791in}{2.479455in}}%
\pgfpathlineto{\pgfqpoint{1.816584in}{2.471257in}}%
\pgfpathlineto{\pgfqpoint{1.822382in}{2.463038in}}%
\pgfpathlineto{\pgfqpoint{1.828183in}{2.454799in}}%
\pgfpathlineto{\pgfqpoint{1.833990in}{2.446538in}}%
\pgfpathclose%
\pgfusepath{stroke,fill}%
\end{pgfscope}%
\begin{pgfscope}%
\pgfpathrectangle{\pgfqpoint{0.887500in}{0.275000in}}{\pgfqpoint{4.225000in}{4.225000in}}%
\pgfusepath{clip}%
\pgfsetbuttcap%
\pgfsetroundjoin%
\definecolor{currentfill}{rgb}{0.119512,0.607464,0.540218}%
\pgfsetfillcolor{currentfill}%
\pgfsetfillopacity{0.700000}%
\pgfsetlinewidth{0.501875pt}%
\definecolor{currentstroke}{rgb}{1.000000,1.000000,1.000000}%
\pgfsetstrokecolor{currentstroke}%
\pgfsetstrokeopacity{0.500000}%
\pgfsetdash{}{0pt}%
\pgfpathmoveto{\pgfqpoint{2.939493in}{2.481724in}}%
\pgfpathlineto{\pgfqpoint{2.950764in}{2.512957in}}%
\pgfpathlineto{\pgfqpoint{2.962049in}{2.542885in}}%
\pgfpathlineto{\pgfqpoint{2.973346in}{2.571566in}}%
\pgfpathlineto{\pgfqpoint{2.984653in}{2.599359in}}%
\pgfpathlineto{\pgfqpoint{2.995970in}{2.626624in}}%
\pgfpathlineto{\pgfqpoint{2.989761in}{2.640818in}}%
\pgfpathlineto{\pgfqpoint{2.983553in}{2.655724in}}%
\pgfpathlineto{\pgfqpoint{2.977347in}{2.671005in}}%
\pgfpathlineto{\pgfqpoint{2.971142in}{2.686325in}}%
\pgfpathlineto{\pgfqpoint{2.964940in}{2.701346in}}%
\pgfpathlineto{\pgfqpoint{2.953649in}{2.672426in}}%
\pgfpathlineto{\pgfqpoint{2.942369in}{2.642471in}}%
\pgfpathlineto{\pgfqpoint{2.931103in}{2.611067in}}%
\pgfpathlineto{\pgfqpoint{2.919854in}{2.577803in}}%
\pgfpathlineto{\pgfqpoint{2.908624in}{2.542637in}}%
\pgfpathlineto{\pgfqpoint{2.914793in}{2.530156in}}%
\pgfpathlineto{\pgfqpoint{2.920965in}{2.517644in}}%
\pgfpathlineto{\pgfqpoint{2.927140in}{2.505281in}}%
\pgfpathlineto{\pgfqpoint{2.933316in}{2.493248in}}%
\pgfpathclose%
\pgfusepath{stroke,fill}%
\end{pgfscope}%
\begin{pgfscope}%
\pgfpathrectangle{\pgfqpoint{0.887500in}{0.275000in}}{\pgfqpoint{4.225000in}{4.225000in}}%
\pgfusepath{clip}%
\pgfsetbuttcap%
\pgfsetroundjoin%
\definecolor{currentfill}{rgb}{0.496615,0.826376,0.306377}%
\pgfsetfillcolor{currentfill}%
\pgfsetfillopacity{0.700000}%
\pgfsetlinewidth{0.501875pt}%
\definecolor{currentstroke}{rgb}{1.000000,1.000000,1.000000}%
\pgfsetstrokecolor{currentstroke}%
\pgfsetstrokeopacity{0.500000}%
\pgfsetdash{}{0pt}%
\pgfpathmoveto{\pgfqpoint{3.392847in}{3.071691in}}%
\pgfpathlineto{\pgfqpoint{3.404105in}{3.075347in}}%
\pgfpathlineto{\pgfqpoint{3.415357in}{3.078994in}}%
\pgfpathlineto{\pgfqpoint{3.426603in}{3.082623in}}%
\pgfpathlineto{\pgfqpoint{3.437844in}{3.086231in}}%
\pgfpathlineto{\pgfqpoint{3.449080in}{3.089816in}}%
\pgfpathlineto{\pgfqpoint{3.442783in}{3.102506in}}%
\pgfpathlineto{\pgfqpoint{3.436489in}{3.115134in}}%
\pgfpathlineto{\pgfqpoint{3.430197in}{3.127678in}}%
\pgfpathlineto{\pgfqpoint{3.423906in}{3.140113in}}%
\pgfpathlineto{\pgfqpoint{3.417618in}{3.152418in}}%
\pgfpathlineto{\pgfqpoint{3.406390in}{3.149185in}}%
\pgfpathlineto{\pgfqpoint{3.395157in}{3.145865in}}%
\pgfpathlineto{\pgfqpoint{3.383917in}{3.142451in}}%
\pgfpathlineto{\pgfqpoint{3.372671in}{3.138942in}}%
\pgfpathlineto{\pgfqpoint{3.361419in}{3.135349in}}%
\pgfpathlineto{\pgfqpoint{3.367700in}{3.122807in}}%
\pgfpathlineto{\pgfqpoint{3.373984in}{3.110141in}}%
\pgfpathlineto{\pgfqpoint{3.380269in}{3.097379in}}%
\pgfpathlineto{\pgfqpoint{3.386557in}{3.084552in}}%
\pgfpathclose%
\pgfusepath{stroke,fill}%
\end{pgfscope}%
\begin{pgfscope}%
\pgfpathrectangle{\pgfqpoint{0.887500in}{0.275000in}}{\pgfqpoint{4.225000in}{4.225000in}}%
\pgfusepath{clip}%
\pgfsetbuttcap%
\pgfsetroundjoin%
\definecolor{currentfill}{rgb}{0.575563,0.844566,0.256415}%
\pgfsetfillcolor{currentfill}%
\pgfsetfillopacity{0.700000}%
\pgfsetlinewidth{0.501875pt}%
\definecolor{currentstroke}{rgb}{1.000000,1.000000,1.000000}%
\pgfsetstrokecolor{currentstroke}%
\pgfsetstrokeopacity{0.500000}%
\pgfsetdash{}{0pt}%
\pgfpathmoveto{\pgfqpoint{3.072781in}{3.121767in}}%
\pgfpathlineto{\pgfqpoint{3.084116in}{3.134475in}}%
\pgfpathlineto{\pgfqpoint{3.095450in}{3.146328in}}%
\pgfpathlineto{\pgfqpoint{3.106783in}{3.157456in}}%
\pgfpathlineto{\pgfqpoint{3.118114in}{3.167842in}}%
\pgfpathlineto{\pgfqpoint{3.129441in}{3.177455in}}%
\pgfpathlineto{\pgfqpoint{3.123209in}{3.188789in}}%
\pgfpathlineto{\pgfqpoint{3.116979in}{3.200019in}}%
\pgfpathlineto{\pgfqpoint{3.110752in}{3.211124in}}%
\pgfpathlineto{\pgfqpoint{3.104529in}{3.222077in}}%
\pgfpathlineto{\pgfqpoint{3.098308in}{3.232857in}}%
\pgfpathlineto{\pgfqpoint{3.086991in}{3.221079in}}%
\pgfpathlineto{\pgfqpoint{3.075672in}{3.207730in}}%
\pgfpathlineto{\pgfqpoint{3.064354in}{3.192743in}}%
\pgfpathlineto{\pgfqpoint{3.053036in}{3.176058in}}%
\pgfpathlineto{\pgfqpoint{3.041721in}{3.157681in}}%
\pgfpathlineto{\pgfqpoint{3.047924in}{3.150954in}}%
\pgfpathlineto{\pgfqpoint{3.054132in}{3.144024in}}%
\pgfpathlineto{\pgfqpoint{3.060344in}{3.136867in}}%
\pgfpathlineto{\pgfqpoint{3.066560in}{3.129456in}}%
\pgfpathclose%
\pgfusepath{stroke,fill}%
\end{pgfscope}%
\begin{pgfscope}%
\pgfpathrectangle{\pgfqpoint{0.887500in}{0.275000in}}{\pgfqpoint{4.225000in}{4.225000in}}%
\pgfusepath{clip}%
\pgfsetbuttcap%
\pgfsetroundjoin%
\definecolor{currentfill}{rgb}{0.143343,0.522773,0.556295}%
\pgfsetfillcolor{currentfill}%
\pgfsetfillopacity{0.700000}%
\pgfsetlinewidth{0.501875pt}%
\definecolor{currentstroke}{rgb}{1.000000,1.000000,1.000000}%
\pgfsetstrokecolor{currentstroke}%
\pgfsetstrokeopacity{0.500000}%
\pgfsetdash{}{0pt}%
\pgfpathmoveto{\pgfqpoint{2.159902in}{2.378954in}}%
\pgfpathlineto{\pgfqpoint{2.171462in}{2.382338in}}%
\pgfpathlineto{\pgfqpoint{2.183017in}{2.385713in}}%
\pgfpathlineto{\pgfqpoint{2.194566in}{2.389075in}}%
\pgfpathlineto{\pgfqpoint{2.206110in}{2.392427in}}%
\pgfpathlineto{\pgfqpoint{2.217648in}{2.395766in}}%
\pgfpathlineto{\pgfqpoint{2.211714in}{2.404217in}}%
\pgfpathlineto{\pgfqpoint{2.205784in}{2.412651in}}%
\pgfpathlineto{\pgfqpoint{2.199859in}{2.421068in}}%
\pgfpathlineto{\pgfqpoint{2.193938in}{2.429469in}}%
\pgfpathlineto{\pgfqpoint{2.188021in}{2.437853in}}%
\pgfpathlineto{\pgfqpoint{2.176494in}{2.434538in}}%
\pgfpathlineto{\pgfqpoint{2.164962in}{2.431212in}}%
\pgfpathlineto{\pgfqpoint{2.153424in}{2.427876in}}%
\pgfpathlineto{\pgfqpoint{2.141881in}{2.424528in}}%
\pgfpathlineto{\pgfqpoint{2.130333in}{2.421170in}}%
\pgfpathlineto{\pgfqpoint{2.136238in}{2.412763in}}%
\pgfpathlineto{\pgfqpoint{2.142148in}{2.404338in}}%
\pgfpathlineto{\pgfqpoint{2.148062in}{2.395895in}}%
\pgfpathlineto{\pgfqpoint{2.153980in}{2.387433in}}%
\pgfpathclose%
\pgfusepath{stroke,fill}%
\end{pgfscope}%
\begin{pgfscope}%
\pgfpathrectangle{\pgfqpoint{0.887500in}{0.275000in}}{\pgfqpoint{4.225000in}{4.225000in}}%
\pgfusepath{clip}%
\pgfsetbuttcap%
\pgfsetroundjoin%
\definecolor{currentfill}{rgb}{0.157851,0.683765,0.501686}%
\pgfsetfillcolor{currentfill}%
\pgfsetfillopacity{0.700000}%
\pgfsetlinewidth{0.501875pt}%
\definecolor{currentstroke}{rgb}{1.000000,1.000000,1.000000}%
\pgfsetstrokecolor{currentstroke}%
\pgfsetstrokeopacity{0.500000}%
\pgfsetdash{}{0pt}%
\pgfpathmoveto{\pgfqpoint{3.950647in}{2.708979in}}%
\pgfpathlineto{\pgfqpoint{3.961766in}{2.712411in}}%
\pgfpathlineto{\pgfqpoint{3.972879in}{2.715830in}}%
\pgfpathlineto{\pgfqpoint{3.983987in}{2.719238in}}%
\pgfpathlineto{\pgfqpoint{3.995089in}{2.722637in}}%
\pgfpathlineto{\pgfqpoint{4.006185in}{2.726031in}}%
\pgfpathlineto{\pgfqpoint{3.999795in}{2.739960in}}%
\pgfpathlineto{\pgfqpoint{3.993405in}{2.753844in}}%
\pgfpathlineto{\pgfqpoint{3.987017in}{2.767679in}}%
\pgfpathlineto{\pgfqpoint{3.980631in}{2.781458in}}%
\pgfpathlineto{\pgfqpoint{3.974245in}{2.795187in}}%
\pgfpathlineto{\pgfqpoint{3.963151in}{2.791761in}}%
\pgfpathlineto{\pgfqpoint{3.952050in}{2.788333in}}%
\pgfpathlineto{\pgfqpoint{3.940945in}{2.784898in}}%
\pgfpathlineto{\pgfqpoint{3.929833in}{2.781456in}}%
\pgfpathlineto{\pgfqpoint{3.918717in}{2.778001in}}%
\pgfpathlineto{\pgfqpoint{3.925100in}{2.764289in}}%
\pgfpathlineto{\pgfqpoint{3.931484in}{2.750531in}}%
\pgfpathlineto{\pgfqpoint{3.937870in}{2.736724in}}%
\pgfpathlineto{\pgfqpoint{3.944258in}{2.722871in}}%
\pgfpathclose%
\pgfusepath{stroke,fill}%
\end{pgfscope}%
\begin{pgfscope}%
\pgfpathrectangle{\pgfqpoint{0.887500in}{0.275000in}}{\pgfqpoint{4.225000in}{4.225000in}}%
\pgfusepath{clip}%
\pgfsetbuttcap%
\pgfsetroundjoin%
\definecolor{currentfill}{rgb}{0.154815,0.493313,0.557840}%
\pgfsetfillcolor{currentfill}%
\pgfsetfillopacity{0.700000}%
\pgfsetlinewidth{0.501875pt}%
\definecolor{currentstroke}{rgb}{1.000000,1.000000,1.000000}%
\pgfsetstrokecolor{currentstroke}%
\pgfsetstrokeopacity{0.500000}%
\pgfsetdash{}{0pt}%
\pgfpathmoveto{\pgfqpoint{2.480071in}{2.317268in}}%
\pgfpathlineto{\pgfqpoint{2.491551in}{2.320760in}}%
\pgfpathlineto{\pgfqpoint{2.503025in}{2.324297in}}%
\pgfpathlineto{\pgfqpoint{2.514492in}{2.327871in}}%
\pgfpathlineto{\pgfqpoint{2.525954in}{2.331451in}}%
\pgfpathlineto{\pgfqpoint{2.537411in}{2.335001in}}%
\pgfpathlineto{\pgfqpoint{2.531369in}{2.343658in}}%
\pgfpathlineto{\pgfqpoint{2.525331in}{2.352296in}}%
\pgfpathlineto{\pgfqpoint{2.519297in}{2.360914in}}%
\pgfpathlineto{\pgfqpoint{2.513268in}{2.369513in}}%
\pgfpathlineto{\pgfqpoint{2.507242in}{2.378094in}}%
\pgfpathlineto{\pgfqpoint{2.495797in}{2.374553in}}%
\pgfpathlineto{\pgfqpoint{2.484346in}{2.370983in}}%
\pgfpathlineto{\pgfqpoint{2.472890in}{2.367420in}}%
\pgfpathlineto{\pgfqpoint{2.461428in}{2.363895in}}%
\pgfpathlineto{\pgfqpoint{2.449959in}{2.360418in}}%
\pgfpathlineto{\pgfqpoint{2.455973in}{2.351822in}}%
\pgfpathlineto{\pgfqpoint{2.461992in}{2.343210in}}%
\pgfpathlineto{\pgfqpoint{2.468014in}{2.334580in}}%
\pgfpathlineto{\pgfqpoint{2.474040in}{2.325933in}}%
\pgfpathclose%
\pgfusepath{stroke,fill}%
\end{pgfscope}%
\begin{pgfscope}%
\pgfpathrectangle{\pgfqpoint{0.887500in}{0.275000in}}{\pgfqpoint{4.225000in}{4.225000in}}%
\pgfusepath{clip}%
\pgfsetbuttcap%
\pgfsetroundjoin%
\definecolor{currentfill}{rgb}{0.555484,0.840254,0.269281}%
\pgfsetfillcolor{currentfill}%
\pgfsetfillopacity{0.700000}%
\pgfsetlinewidth{0.501875pt}%
\definecolor{currentstroke}{rgb}{1.000000,1.000000,1.000000}%
\pgfsetstrokecolor{currentstroke}%
\pgfsetstrokeopacity{0.500000}%
\pgfsetdash{}{0pt}%
\pgfpathmoveto{\pgfqpoint{3.305074in}{3.116621in}}%
\pgfpathlineto{\pgfqpoint{3.316354in}{3.120424in}}%
\pgfpathlineto{\pgfqpoint{3.327628in}{3.124211in}}%
\pgfpathlineto{\pgfqpoint{3.338897in}{3.127970in}}%
\pgfpathlineto{\pgfqpoint{3.350161in}{3.131688in}}%
\pgfpathlineto{\pgfqpoint{3.361419in}{3.135349in}}%
\pgfpathlineto{\pgfqpoint{3.355139in}{3.147736in}}%
\pgfpathlineto{\pgfqpoint{3.348860in}{3.159937in}}%
\pgfpathlineto{\pgfqpoint{3.342583in}{3.171923in}}%
\pgfpathlineto{\pgfqpoint{3.336308in}{3.183664in}}%
\pgfpathlineto{\pgfqpoint{3.330033in}{3.195129in}}%
\pgfpathlineto{\pgfqpoint{3.318782in}{3.191544in}}%
\pgfpathlineto{\pgfqpoint{3.307524in}{3.187863in}}%
\pgfpathlineto{\pgfqpoint{3.296261in}{3.184100in}}%
\pgfpathlineto{\pgfqpoint{3.284993in}{3.180272in}}%
\pgfpathlineto{\pgfqpoint{3.273719in}{3.176392in}}%
\pgfpathlineto{\pgfqpoint{3.279986in}{3.164759in}}%
\pgfpathlineto{\pgfqpoint{3.286255in}{3.152956in}}%
\pgfpathlineto{\pgfqpoint{3.292526in}{3.140992in}}%
\pgfpathlineto{\pgfqpoint{3.298799in}{3.128877in}}%
\pgfpathclose%
\pgfusepath{stroke,fill}%
\end{pgfscope}%
\begin{pgfscope}%
\pgfpathrectangle{\pgfqpoint{0.887500in}{0.275000in}}{\pgfqpoint{4.225000in}{4.225000in}}%
\pgfusepath{clip}%
\pgfsetbuttcap%
\pgfsetroundjoin%
\definecolor{currentfill}{rgb}{0.124395,0.578002,0.548287}%
\pgfsetfillcolor{currentfill}%
\pgfsetfillopacity{0.700000}%
\pgfsetlinewidth{0.501875pt}%
\definecolor{currentstroke}{rgb}{1.000000,1.000000,1.000000}%
\pgfsetstrokecolor{currentstroke}%
\pgfsetstrokeopacity{0.500000}%
\pgfsetdash{}{0pt}%
\pgfpathmoveto{\pgfqpoint{4.245202in}{2.477897in}}%
\pgfpathlineto{\pgfqpoint{4.256245in}{2.481216in}}%
\pgfpathlineto{\pgfqpoint{4.267283in}{2.484527in}}%
\pgfpathlineto{\pgfqpoint{4.278315in}{2.487835in}}%
\pgfpathlineto{\pgfqpoint{4.289342in}{2.491141in}}%
\pgfpathlineto{\pgfqpoint{4.300363in}{2.494448in}}%
\pgfpathlineto{\pgfqpoint{4.293927in}{2.508770in}}%
\pgfpathlineto{\pgfqpoint{4.287492in}{2.523078in}}%
\pgfpathlineto{\pgfqpoint{4.281060in}{2.537370in}}%
\pgfpathlineto{\pgfqpoint{4.274630in}{2.551645in}}%
\pgfpathlineto{\pgfqpoint{4.268201in}{2.565903in}}%
\pgfpathlineto{\pgfqpoint{4.257178in}{2.562502in}}%
\pgfpathlineto{\pgfqpoint{4.246150in}{2.559093in}}%
\pgfpathlineto{\pgfqpoint{4.235116in}{2.555679in}}%
\pgfpathlineto{\pgfqpoint{4.224077in}{2.552265in}}%
\pgfpathlineto{\pgfqpoint{4.213033in}{2.548853in}}%
\pgfpathlineto{\pgfqpoint{4.219462in}{2.534673in}}%
\pgfpathlineto{\pgfqpoint{4.225893in}{2.520488in}}%
\pgfpathlineto{\pgfqpoint{4.232327in}{2.506301in}}%
\pgfpathlineto{\pgfqpoint{4.238764in}{2.492107in}}%
\pgfpathclose%
\pgfusepath{stroke,fill}%
\end{pgfscope}%
\begin{pgfscope}%
\pgfpathrectangle{\pgfqpoint{0.887500in}{0.275000in}}{\pgfqpoint{4.225000in}{4.225000in}}%
\pgfusepath{clip}%
\pgfsetbuttcap%
\pgfsetroundjoin%
\definecolor{currentfill}{rgb}{0.172719,0.448791,0.557885}%
\pgfsetfillcolor{currentfill}%
\pgfsetfillopacity{0.700000}%
\pgfsetlinewidth{0.501875pt}%
\definecolor{currentstroke}{rgb}{1.000000,1.000000,1.000000}%
\pgfsetstrokecolor{currentstroke}%
\pgfsetstrokeopacity{0.500000}%
\pgfsetdash{}{0pt}%
\pgfpathmoveto{\pgfqpoint{2.888048in}{2.213564in}}%
\pgfpathlineto{\pgfqpoint{2.899440in}{2.214295in}}%
\pgfpathlineto{\pgfqpoint{2.910816in}{2.217437in}}%
\pgfpathlineto{\pgfqpoint{2.922174in}{2.224082in}}%
\pgfpathlineto{\pgfqpoint{2.933514in}{2.235326in}}%
\pgfpathlineto{\pgfqpoint{2.944836in}{2.252082in}}%
\pgfpathlineto{\pgfqpoint{2.938656in}{2.261456in}}%
\pgfpathlineto{\pgfqpoint{2.932480in}{2.270797in}}%
\pgfpathlineto{\pgfqpoint{2.926308in}{2.280087in}}%
\pgfpathlineto{\pgfqpoint{2.920139in}{2.289305in}}%
\pgfpathlineto{\pgfqpoint{2.913975in}{2.298432in}}%
\pgfpathlineto{\pgfqpoint{2.902672in}{2.281954in}}%
\pgfpathlineto{\pgfqpoint{2.891348in}{2.270812in}}%
\pgfpathlineto{\pgfqpoint{2.880003in}{2.264121in}}%
\pgfpathlineto{\pgfqpoint{2.868638in}{2.260825in}}%
\pgfpathlineto{\pgfqpoint{2.857255in}{2.259868in}}%
\pgfpathlineto{\pgfqpoint{2.863406in}{2.250710in}}%
\pgfpathlineto{\pgfqpoint{2.869561in}{2.241467in}}%
\pgfpathlineto{\pgfqpoint{2.875719in}{2.232174in}}%
\pgfpathlineto{\pgfqpoint{2.881882in}{2.222861in}}%
\pgfpathclose%
\pgfusepath{stroke,fill}%
\end{pgfscope}%
\begin{pgfscope}%
\pgfpathrectangle{\pgfqpoint{0.887500in}{0.275000in}}{\pgfqpoint{4.225000in}{4.225000in}}%
\pgfusepath{clip}%
\pgfsetbuttcap%
\pgfsetroundjoin%
\definecolor{currentfill}{rgb}{0.191090,0.708366,0.482284}%
\pgfsetfillcolor{currentfill}%
\pgfsetfillopacity{0.700000}%
\pgfsetlinewidth{0.501875pt}%
\definecolor{currentstroke}{rgb}{1.000000,1.000000,1.000000}%
\pgfsetstrokecolor{currentstroke}%
\pgfsetstrokeopacity{0.500000}%
\pgfsetdash{}{0pt}%
\pgfpathmoveto{\pgfqpoint{3.863050in}{2.760601in}}%
\pgfpathlineto{\pgfqpoint{3.874194in}{2.764082in}}%
\pgfpathlineto{\pgfqpoint{3.885333in}{2.767568in}}%
\pgfpathlineto{\pgfqpoint{3.896466in}{2.771054in}}%
\pgfpathlineto{\pgfqpoint{3.907594in}{2.774533in}}%
\pgfpathlineto{\pgfqpoint{3.918717in}{2.778001in}}%
\pgfpathlineto{\pgfqpoint{3.912336in}{2.791676in}}%
\pgfpathlineto{\pgfqpoint{3.905957in}{2.805321in}}%
\pgfpathlineto{\pgfqpoint{3.899580in}{2.818946in}}%
\pgfpathlineto{\pgfqpoint{3.893205in}{2.832557in}}%
\pgfpathlineto{\pgfqpoint{3.886833in}{2.846165in}}%
\pgfpathlineto{\pgfqpoint{3.875714in}{2.842712in}}%
\pgfpathlineto{\pgfqpoint{3.864589in}{2.839252in}}%
\pgfpathlineto{\pgfqpoint{3.853459in}{2.835789in}}%
\pgfpathlineto{\pgfqpoint{3.842324in}{2.832327in}}%
\pgfpathlineto{\pgfqpoint{3.831183in}{2.828869in}}%
\pgfpathlineto{\pgfqpoint{3.837552in}{2.815248in}}%
\pgfpathlineto{\pgfqpoint{3.843923in}{2.801614in}}%
\pgfpathlineto{\pgfqpoint{3.850297in}{2.787965in}}%
\pgfpathlineto{\pgfqpoint{3.856672in}{2.774295in}}%
\pgfpathclose%
\pgfusepath{stroke,fill}%
\end{pgfscope}%
\begin{pgfscope}%
\pgfpathrectangle{\pgfqpoint{0.887500in}{0.275000in}}{\pgfqpoint{4.225000in}{4.225000in}}%
\pgfusepath{clip}%
\pgfsetbuttcap%
\pgfsetroundjoin%
\definecolor{currentfill}{rgb}{0.127568,0.566949,0.550556}%
\pgfsetfillcolor{currentfill}%
\pgfsetfillopacity{0.700000}%
\pgfsetlinewidth{0.501875pt}%
\definecolor{currentstroke}{rgb}{1.000000,1.000000,1.000000}%
\pgfsetstrokecolor{currentstroke}%
\pgfsetstrokeopacity{0.500000}%
\pgfsetdash{}{0pt}%
\pgfpathmoveto{\pgfqpoint{1.606986in}{2.470407in}}%
\pgfpathlineto{\pgfqpoint{1.618686in}{2.473747in}}%
\pgfpathlineto{\pgfqpoint{1.630380in}{2.477081in}}%
\pgfpathlineto{\pgfqpoint{1.642069in}{2.480409in}}%
\pgfpathlineto{\pgfqpoint{1.653753in}{2.483731in}}%
\pgfpathlineto{\pgfqpoint{1.665430in}{2.487051in}}%
\pgfpathlineto{\pgfqpoint{1.659688in}{2.495178in}}%
\pgfpathlineto{\pgfqpoint{1.653950in}{2.503291in}}%
\pgfpathlineto{\pgfqpoint{1.648216in}{2.511389in}}%
\pgfpathlineto{\pgfqpoint{1.642487in}{2.519472in}}%
\pgfpathlineto{\pgfqpoint{1.636761in}{2.527538in}}%
\pgfpathlineto{\pgfqpoint{1.625096in}{2.524236in}}%
\pgfpathlineto{\pgfqpoint{1.613426in}{2.520929in}}%
\pgfpathlineto{\pgfqpoint{1.601749in}{2.517617in}}%
\pgfpathlineto{\pgfqpoint{1.590067in}{2.514298in}}%
\pgfpathlineto{\pgfqpoint{1.578380in}{2.510973in}}%
\pgfpathlineto{\pgfqpoint{1.584093in}{2.502890in}}%
\pgfpathlineto{\pgfqpoint{1.589810in}{2.494792in}}%
\pgfpathlineto{\pgfqpoint{1.595531in}{2.486678in}}%
\pgfpathlineto{\pgfqpoint{1.601256in}{2.478550in}}%
\pgfpathclose%
\pgfusepath{stroke,fill}%
\end{pgfscope}%
\begin{pgfscope}%
\pgfpathrectangle{\pgfqpoint{0.887500in}{0.275000in}}{\pgfqpoint{4.225000in}{4.225000in}}%
\pgfusepath{clip}%
\pgfsetbuttcap%
\pgfsetroundjoin%
\definecolor{currentfill}{rgb}{0.165117,0.467423,0.558141}%
\pgfsetfillcolor{currentfill}%
\pgfsetfillopacity{0.700000}%
\pgfsetlinewidth{0.501875pt}%
\definecolor{currentstroke}{rgb}{1.000000,1.000000,1.000000}%
\pgfsetstrokecolor{currentstroke}%
\pgfsetstrokeopacity{0.500000}%
\pgfsetdash{}{0pt}%
\pgfpathmoveto{\pgfqpoint{4.539774in}{2.245885in}}%
\pgfpathlineto{\pgfqpoint{4.550749in}{2.249370in}}%
\pgfpathlineto{\pgfqpoint{4.561718in}{2.252826in}}%
\pgfpathlineto{\pgfqpoint{4.572681in}{2.256261in}}%
\pgfpathlineto{\pgfqpoint{4.583638in}{2.259682in}}%
\pgfpathlineto{\pgfqpoint{4.577147in}{2.274016in}}%
\pgfpathlineto{\pgfqpoint{4.570655in}{2.288279in}}%
\pgfpathlineto{\pgfqpoint{4.564164in}{2.302471in}}%
\pgfpathlineto{\pgfqpoint{4.557673in}{2.316603in}}%
\pgfpathlineto{\pgfqpoint{4.551182in}{2.330683in}}%
\pgfpathlineto{\pgfqpoint{4.540222in}{2.327140in}}%
\pgfpathlineto{\pgfqpoint{4.529256in}{2.323595in}}%
\pgfpathlineto{\pgfqpoint{4.518285in}{2.320044in}}%
\pgfpathlineto{\pgfqpoint{4.507309in}{2.316478in}}%
\pgfpathlineto{\pgfqpoint{4.513801in}{2.302479in}}%
\pgfpathlineto{\pgfqpoint{4.520294in}{2.288432in}}%
\pgfpathlineto{\pgfqpoint{4.526787in}{2.274324in}}%
\pgfpathlineto{\pgfqpoint{4.533281in}{2.260143in}}%
\pgfpathclose%
\pgfusepath{stroke,fill}%
\end{pgfscope}%
\begin{pgfscope}%
\pgfpathrectangle{\pgfqpoint{0.887500in}{0.275000in}}{\pgfqpoint{4.225000in}{4.225000in}}%
\pgfusepath{clip}%
\pgfsetbuttcap%
\pgfsetroundjoin%
\definecolor{currentfill}{rgb}{0.136408,0.541173,0.554483}%
\pgfsetfillcolor{currentfill}%
\pgfsetfillopacity{0.700000}%
\pgfsetlinewidth{0.501875pt}%
\definecolor{currentstroke}{rgb}{1.000000,1.000000,1.000000}%
\pgfsetstrokecolor{currentstroke}%
\pgfsetstrokeopacity{0.500000}%
\pgfsetdash{}{0pt}%
\pgfpathmoveto{\pgfqpoint{1.927121in}{2.413183in}}%
\pgfpathlineto{\pgfqpoint{1.938745in}{2.416490in}}%
\pgfpathlineto{\pgfqpoint{1.950364in}{2.419791in}}%
\pgfpathlineto{\pgfqpoint{1.961976in}{2.423086in}}%
\pgfpathlineto{\pgfqpoint{1.973583in}{2.426377in}}%
\pgfpathlineto{\pgfqpoint{1.985184in}{2.429667in}}%
\pgfpathlineto{\pgfqpoint{1.979328in}{2.438000in}}%
\pgfpathlineto{\pgfqpoint{1.973476in}{2.446314in}}%
\pgfpathlineto{\pgfqpoint{1.967629in}{2.454609in}}%
\pgfpathlineto{\pgfqpoint{1.961785in}{2.462883in}}%
\pgfpathlineto{\pgfqpoint{1.955947in}{2.471139in}}%
\pgfpathlineto{\pgfqpoint{1.944357in}{2.467869in}}%
\pgfpathlineto{\pgfqpoint{1.932763in}{2.464600in}}%
\pgfpathlineto{\pgfqpoint{1.921162in}{2.461329in}}%
\pgfpathlineto{\pgfqpoint{1.909556in}{2.458054in}}%
\pgfpathlineto{\pgfqpoint{1.897944in}{2.454774in}}%
\pgfpathlineto{\pgfqpoint{1.903770in}{2.446497in}}%
\pgfpathlineto{\pgfqpoint{1.909602in}{2.438199in}}%
\pgfpathlineto{\pgfqpoint{1.915437in}{2.429880in}}%
\pgfpathlineto{\pgfqpoint{1.921277in}{2.421542in}}%
\pgfpathclose%
\pgfusepath{stroke,fill}%
\end{pgfscope}%
\begin{pgfscope}%
\pgfpathrectangle{\pgfqpoint{0.887500in}{0.275000in}}{\pgfqpoint{4.225000in}{4.225000in}}%
\pgfusepath{clip}%
\pgfsetbuttcap%
\pgfsetroundjoin%
\definecolor{currentfill}{rgb}{0.119738,0.603785,0.541400}%
\pgfsetfillcolor{currentfill}%
\pgfsetfillopacity{0.700000}%
\pgfsetlinewidth{0.501875pt}%
\definecolor{currentstroke}{rgb}{1.000000,1.000000,1.000000}%
\pgfsetstrokecolor{currentstroke}%
\pgfsetstrokeopacity{0.500000}%
\pgfsetdash{}{0pt}%
\pgfpathmoveto{\pgfqpoint{4.157735in}{2.531934in}}%
\pgfpathlineto{\pgfqpoint{4.168805in}{2.535297in}}%
\pgfpathlineto{\pgfqpoint{4.179870in}{2.538670in}}%
\pgfpathlineto{\pgfqpoint{4.190929in}{2.542053in}}%
\pgfpathlineto{\pgfqpoint{4.201984in}{2.545448in}}%
\pgfpathlineto{\pgfqpoint{4.213033in}{2.548853in}}%
\pgfpathlineto{\pgfqpoint{4.206606in}{2.563027in}}%
\pgfpathlineto{\pgfqpoint{4.200182in}{2.577194in}}%
\pgfpathlineto{\pgfqpoint{4.193760in}{2.591352in}}%
\pgfpathlineto{\pgfqpoint{4.187340in}{2.605499in}}%
\pgfpathlineto{\pgfqpoint{4.180923in}{2.619634in}}%
\pgfpathlineto{\pgfqpoint{4.169874in}{2.616184in}}%
\pgfpathlineto{\pgfqpoint{4.158820in}{2.612743in}}%
\pgfpathlineto{\pgfqpoint{4.147761in}{2.609313in}}%
\pgfpathlineto{\pgfqpoint{4.136697in}{2.605895in}}%
\pgfpathlineto{\pgfqpoint{4.125628in}{2.602489in}}%
\pgfpathlineto{\pgfqpoint{4.132045in}{2.588391in}}%
\pgfpathlineto{\pgfqpoint{4.138464in}{2.574288in}}%
\pgfpathlineto{\pgfqpoint{4.144885in}{2.560177in}}%
\pgfpathlineto{\pgfqpoint{4.151309in}{2.546059in}}%
\pgfpathclose%
\pgfusepath{stroke,fill}%
\end{pgfscope}%
\begin{pgfscope}%
\pgfpathrectangle{\pgfqpoint{0.887500in}{0.275000in}}{\pgfqpoint{4.225000in}{4.225000in}}%
\pgfusepath{clip}%
\pgfsetbuttcap%
\pgfsetroundjoin%
\definecolor{currentfill}{rgb}{0.157729,0.485932,0.558013}%
\pgfsetfillcolor{currentfill}%
\pgfsetfillopacity{0.700000}%
\pgfsetlinewidth{0.501875pt}%
\definecolor{currentstroke}{rgb}{1.000000,1.000000,1.000000}%
\pgfsetstrokecolor{currentstroke}%
\pgfsetstrokeopacity{0.500000}%
\pgfsetdash{}{0pt}%
\pgfpathmoveto{\pgfqpoint{2.567684in}{2.291393in}}%
\pgfpathlineto{\pgfqpoint{2.579147in}{2.294879in}}%
\pgfpathlineto{\pgfqpoint{2.590606in}{2.298270in}}%
\pgfpathlineto{\pgfqpoint{2.602061in}{2.301534in}}%
\pgfpathlineto{\pgfqpoint{2.613512in}{2.304641in}}%
\pgfpathlineto{\pgfqpoint{2.624959in}{2.307614in}}%
\pgfpathlineto{\pgfqpoint{2.618885in}{2.316359in}}%
\pgfpathlineto{\pgfqpoint{2.612816in}{2.325082in}}%
\pgfpathlineto{\pgfqpoint{2.606750in}{2.333784in}}%
\pgfpathlineto{\pgfqpoint{2.600689in}{2.342467in}}%
\pgfpathlineto{\pgfqpoint{2.594631in}{2.351131in}}%
\pgfpathlineto{\pgfqpoint{2.583195in}{2.348196in}}%
\pgfpathlineto{\pgfqpoint{2.571755in}{2.345117in}}%
\pgfpathlineto{\pgfqpoint{2.560311in}{2.341869in}}%
\pgfpathlineto{\pgfqpoint{2.548863in}{2.338485in}}%
\pgfpathlineto{\pgfqpoint{2.537411in}{2.335001in}}%
\pgfpathlineto{\pgfqpoint{2.543457in}{2.326322in}}%
\pgfpathlineto{\pgfqpoint{2.549508in}{2.317623in}}%
\pgfpathlineto{\pgfqpoint{2.555562in}{2.308902in}}%
\pgfpathlineto{\pgfqpoint{2.561621in}{2.300158in}}%
\pgfpathclose%
\pgfusepath{stroke,fill}%
\end{pgfscope}%
\begin{pgfscope}%
\pgfpathrectangle{\pgfqpoint{0.887500in}{0.275000in}}{\pgfqpoint{4.225000in}{4.225000in}}%
\pgfusepath{clip}%
\pgfsetbuttcap%
\pgfsetroundjoin%
\definecolor{currentfill}{rgb}{0.154815,0.493313,0.557840}%
\pgfsetfillcolor{currentfill}%
\pgfsetfillopacity{0.700000}%
\pgfsetlinewidth{0.501875pt}%
\definecolor{currentstroke}{rgb}{1.000000,1.000000,1.000000}%
\pgfsetstrokecolor{currentstroke}%
\pgfsetstrokeopacity{0.500000}%
\pgfsetdash{}{0pt}%
\pgfpathmoveto{\pgfqpoint{4.452330in}{2.298196in}}%
\pgfpathlineto{\pgfqpoint{4.463339in}{2.301935in}}%
\pgfpathlineto{\pgfqpoint{4.474341in}{2.305626in}}%
\pgfpathlineto{\pgfqpoint{4.485337in}{2.309276in}}%
\pgfpathlineto{\pgfqpoint{4.496326in}{2.312891in}}%
\pgfpathlineto{\pgfqpoint{4.507309in}{2.316478in}}%
\pgfpathlineto{\pgfqpoint{4.500818in}{2.330441in}}%
\pgfpathlineto{\pgfqpoint{4.494329in}{2.344382in}}%
\pgfpathlineto{\pgfqpoint{4.487842in}{2.358314in}}%
\pgfpathlineto{\pgfqpoint{4.481359in}{2.372248in}}%
\pgfpathlineto{\pgfqpoint{4.474878in}{2.386199in}}%
\pgfpathlineto{\pgfqpoint{4.463898in}{2.382621in}}%
\pgfpathlineto{\pgfqpoint{4.452912in}{2.379052in}}%
\pgfpathlineto{\pgfqpoint{4.441922in}{2.375490in}}%
\pgfpathlineto{\pgfqpoint{4.430927in}{2.371931in}}%
\pgfpathlineto{\pgfqpoint{4.419926in}{2.368373in}}%
\pgfpathlineto{\pgfqpoint{4.426400in}{2.354306in}}%
\pgfpathlineto{\pgfqpoint{4.432878in}{2.340263in}}%
\pgfpathlineto{\pgfqpoint{4.439360in}{2.326237in}}%
\pgfpathlineto{\pgfqpoint{4.445844in}{2.312217in}}%
\pgfpathclose%
\pgfusepath{stroke,fill}%
\end{pgfscope}%
\begin{pgfscope}%
\pgfpathrectangle{\pgfqpoint{0.887500in}{0.275000in}}{\pgfqpoint{4.225000in}{4.225000in}}%
\pgfusepath{clip}%
\pgfsetbuttcap%
\pgfsetroundjoin%
\definecolor{currentfill}{rgb}{0.147607,0.511733,0.557049}%
\pgfsetfillcolor{currentfill}%
\pgfsetfillopacity{0.700000}%
\pgfsetlinewidth{0.501875pt}%
\definecolor{currentstroke}{rgb}{1.000000,1.000000,1.000000}%
\pgfsetstrokecolor{currentstroke}%
\pgfsetstrokeopacity{0.500000}%
\pgfsetdash{}{0pt}%
\pgfpathmoveto{\pgfqpoint{2.247379in}{2.353246in}}%
\pgfpathlineto{\pgfqpoint{2.258924in}{2.356600in}}%
\pgfpathlineto{\pgfqpoint{2.270462in}{2.359939in}}%
\pgfpathlineto{\pgfqpoint{2.281996in}{2.363263in}}%
\pgfpathlineto{\pgfqpoint{2.293524in}{2.366571in}}%
\pgfpathlineto{\pgfqpoint{2.305046in}{2.369867in}}%
\pgfpathlineto{\pgfqpoint{2.299080in}{2.378385in}}%
\pgfpathlineto{\pgfqpoint{2.293118in}{2.386885in}}%
\pgfpathlineto{\pgfqpoint{2.287160in}{2.395367in}}%
\pgfpathlineto{\pgfqpoint{2.281206in}{2.403833in}}%
\pgfpathlineto{\pgfqpoint{2.275257in}{2.412283in}}%
\pgfpathlineto{\pgfqpoint{2.263746in}{2.409001in}}%
\pgfpathlineto{\pgfqpoint{2.252230in}{2.405711in}}%
\pgfpathlineto{\pgfqpoint{2.240708in}{2.402408in}}%
\pgfpathlineto{\pgfqpoint{2.229181in}{2.399093in}}%
\pgfpathlineto{\pgfqpoint{2.217648in}{2.395766in}}%
\pgfpathlineto{\pgfqpoint{2.223586in}{2.387297in}}%
\pgfpathlineto{\pgfqpoint{2.229528in}{2.378811in}}%
\pgfpathlineto{\pgfqpoint{2.235474in}{2.370308in}}%
\pgfpathlineto{\pgfqpoint{2.241425in}{2.361786in}}%
\pgfpathclose%
\pgfusepath{stroke,fill}%
\end{pgfscope}%
\begin{pgfscope}%
\pgfpathrectangle{\pgfqpoint{0.887500in}{0.275000in}}{\pgfqpoint{4.225000in}{4.225000in}}%
\pgfusepath{clip}%
\pgfsetbuttcap%
\pgfsetroundjoin%
\definecolor{currentfill}{rgb}{0.232815,0.732247,0.459277}%
\pgfsetfillcolor{currentfill}%
\pgfsetfillopacity{0.700000}%
\pgfsetlinewidth{0.501875pt}%
\definecolor{currentstroke}{rgb}{1.000000,1.000000,1.000000}%
\pgfsetstrokecolor{currentstroke}%
\pgfsetstrokeopacity{0.500000}%
\pgfsetdash{}{0pt}%
\pgfpathmoveto{\pgfqpoint{3.775403in}{2.811772in}}%
\pgfpathlineto{\pgfqpoint{3.786570in}{2.815162in}}%
\pgfpathlineto{\pgfqpoint{3.797731in}{2.818566in}}%
\pgfpathlineto{\pgfqpoint{3.808887in}{2.821985in}}%
\pgfpathlineto{\pgfqpoint{3.820038in}{2.825421in}}%
\pgfpathlineto{\pgfqpoint{3.831183in}{2.828869in}}%
\pgfpathlineto{\pgfqpoint{3.824817in}{2.842484in}}%
\pgfpathlineto{\pgfqpoint{3.818453in}{2.856095in}}%
\pgfpathlineto{\pgfqpoint{3.812092in}{2.869708in}}%
\pgfpathlineto{\pgfqpoint{3.805734in}{2.883327in}}%
\pgfpathlineto{\pgfqpoint{3.799378in}{2.896950in}}%
\pgfpathlineto{\pgfqpoint{3.788235in}{2.893446in}}%
\pgfpathlineto{\pgfqpoint{3.777085in}{2.889944in}}%
\pgfpathlineto{\pgfqpoint{3.765931in}{2.886445in}}%
\pgfpathlineto{\pgfqpoint{3.754772in}{2.882954in}}%
\pgfpathlineto{\pgfqpoint{3.743607in}{2.879470in}}%
\pgfpathlineto{\pgfqpoint{3.749962in}{2.865983in}}%
\pgfpathlineto{\pgfqpoint{3.756319in}{2.852470in}}%
\pgfpathlineto{\pgfqpoint{3.762678in}{2.838930in}}%
\pgfpathlineto{\pgfqpoint{3.769040in}{2.825364in}}%
\pgfpathclose%
\pgfusepath{stroke,fill}%
\end{pgfscope}%
\begin{pgfscope}%
\pgfpathrectangle{\pgfqpoint{0.887500in}{0.275000in}}{\pgfqpoint{4.225000in}{4.225000in}}%
\pgfusepath{clip}%
\pgfsetbuttcap%
\pgfsetroundjoin%
\definecolor{currentfill}{rgb}{0.274149,0.751988,0.436601}%
\pgfsetfillcolor{currentfill}%
\pgfsetfillopacity{0.700000}%
\pgfsetlinewidth{0.501875pt}%
\definecolor{currentstroke}{rgb}{1.000000,1.000000,1.000000}%
\pgfsetstrokecolor{currentstroke}%
\pgfsetstrokeopacity{0.500000}%
\pgfsetdash{}{0pt}%
\pgfpathmoveto{\pgfqpoint{3.687706in}{2.862247in}}%
\pgfpathlineto{\pgfqpoint{3.698897in}{2.865660in}}%
\pgfpathlineto{\pgfqpoint{3.710082in}{2.869090in}}%
\pgfpathlineto{\pgfqpoint{3.721262in}{2.872537in}}%
\pgfpathlineto{\pgfqpoint{3.732437in}{2.875997in}}%
\pgfpathlineto{\pgfqpoint{3.743607in}{2.879470in}}%
\pgfpathlineto{\pgfqpoint{3.737254in}{2.892928in}}%
\pgfpathlineto{\pgfqpoint{3.730904in}{2.906356in}}%
\pgfpathlineto{\pgfqpoint{3.724555in}{2.919750in}}%
\pgfpathlineto{\pgfqpoint{3.718209in}{2.933110in}}%
\pgfpathlineto{\pgfqpoint{3.711864in}{2.946433in}}%
\pgfpathlineto{\pgfqpoint{3.700695in}{2.942811in}}%
\pgfpathlineto{\pgfqpoint{3.689522in}{2.939221in}}%
\pgfpathlineto{\pgfqpoint{3.678344in}{2.935671in}}%
\pgfpathlineto{\pgfqpoint{3.667161in}{2.932172in}}%
\pgfpathlineto{\pgfqpoint{3.655974in}{2.928729in}}%
\pgfpathlineto{\pgfqpoint{3.662315in}{2.915471in}}%
\pgfpathlineto{\pgfqpoint{3.668660in}{2.902199in}}%
\pgfpathlineto{\pgfqpoint{3.675006in}{2.888909in}}%
\pgfpathlineto{\pgfqpoint{3.681355in}{2.875593in}}%
\pgfpathclose%
\pgfusepath{stroke,fill}%
\end{pgfscope}%
\begin{pgfscope}%
\pgfpathrectangle{\pgfqpoint{0.887500in}{0.275000in}}{\pgfqpoint{4.225000in}{4.225000in}}%
\pgfusepath{clip}%
\pgfsetbuttcap%
\pgfsetroundjoin%
\definecolor{currentfill}{rgb}{0.121380,0.629492,0.531973}%
\pgfsetfillcolor{currentfill}%
\pgfsetfillopacity{0.700000}%
\pgfsetlinewidth{0.501875pt}%
\definecolor{currentstroke}{rgb}{1.000000,1.000000,1.000000}%
\pgfsetstrokecolor{currentstroke}%
\pgfsetstrokeopacity{0.500000}%
\pgfsetdash{}{0pt}%
\pgfpathmoveto{\pgfqpoint{4.070207in}{2.585640in}}%
\pgfpathlineto{\pgfqpoint{4.081302in}{2.588987in}}%
\pgfpathlineto{\pgfqpoint{4.092391in}{2.592344in}}%
\pgfpathlineto{\pgfqpoint{4.103475in}{2.595713in}}%
\pgfpathlineto{\pgfqpoint{4.114554in}{2.599095in}}%
\pgfpathlineto{\pgfqpoint{4.125628in}{2.602489in}}%
\pgfpathlineto{\pgfqpoint{4.119214in}{2.616582in}}%
\pgfpathlineto{\pgfqpoint{4.112802in}{2.630669in}}%
\pgfpathlineto{\pgfqpoint{4.106392in}{2.644751in}}%
\pgfpathlineto{\pgfqpoint{4.099985in}{2.658829in}}%
\pgfpathlineto{\pgfqpoint{4.093580in}{2.672898in}}%
\pgfpathlineto{\pgfqpoint{4.082508in}{2.669488in}}%
\pgfpathlineto{\pgfqpoint{4.071430in}{2.666087in}}%
\pgfpathlineto{\pgfqpoint{4.060348in}{2.662696in}}%
\pgfpathlineto{\pgfqpoint{4.049260in}{2.659316in}}%
\pgfpathlineto{\pgfqpoint{4.038167in}{2.655945in}}%
\pgfpathlineto{\pgfqpoint{4.044570in}{2.641882in}}%
\pgfpathlineto{\pgfqpoint{4.050975in}{2.627817in}}%
\pgfpathlineto{\pgfqpoint{4.057383in}{2.613756in}}%
\pgfpathlineto{\pgfqpoint{4.063794in}{2.599699in}}%
\pgfpathclose%
\pgfusepath{stroke,fill}%
\end{pgfscope}%
\begin{pgfscope}%
\pgfpathrectangle{\pgfqpoint{0.887500in}{0.275000in}}{\pgfqpoint{4.225000in}{4.225000in}}%
\pgfusepath{clip}%
\pgfsetbuttcap%
\pgfsetroundjoin%
\definecolor{currentfill}{rgb}{0.144759,0.519093,0.556572}%
\pgfsetfillcolor{currentfill}%
\pgfsetfillopacity{0.700000}%
\pgfsetlinewidth{0.501875pt}%
\definecolor{currentstroke}{rgb}{1.000000,1.000000,1.000000}%
\pgfsetstrokecolor{currentstroke}%
\pgfsetstrokeopacity{0.500000}%
\pgfsetdash{}{0pt}%
\pgfpathmoveto{\pgfqpoint{4.364841in}{2.350502in}}%
\pgfpathlineto{\pgfqpoint{4.375869in}{2.354096in}}%
\pgfpathlineto{\pgfqpoint{4.386892in}{2.357678in}}%
\pgfpathlineto{\pgfqpoint{4.397909in}{2.361249in}}%
\pgfpathlineto{\pgfqpoint{4.408920in}{2.364814in}}%
\pgfpathlineto{\pgfqpoint{4.419926in}{2.368373in}}%
\pgfpathlineto{\pgfqpoint{4.413455in}{2.382475in}}%
\pgfpathlineto{\pgfqpoint{4.406989in}{2.396620in}}%
\pgfpathlineto{\pgfqpoint{4.400527in}{2.410814in}}%
\pgfpathlineto{\pgfqpoint{4.394068in}{2.425053in}}%
\pgfpathlineto{\pgfqpoint{4.387614in}{2.439329in}}%
\pgfpathlineto{\pgfqpoint{4.376616in}{2.435948in}}%
\pgfpathlineto{\pgfqpoint{4.365613in}{2.432588in}}%
\pgfpathlineto{\pgfqpoint{4.354605in}{2.429247in}}%
\pgfpathlineto{\pgfqpoint{4.343593in}{2.425926in}}%
\pgfpathlineto{\pgfqpoint{4.332576in}{2.422623in}}%
\pgfpathlineto{\pgfqpoint{4.339024in}{2.408220in}}%
\pgfpathlineto{\pgfqpoint{4.345475in}{2.393805in}}%
\pgfpathlineto{\pgfqpoint{4.351928in}{2.379378in}}%
\pgfpathlineto{\pgfqpoint{4.358384in}{2.364942in}}%
\pgfpathclose%
\pgfusepath{stroke,fill}%
\end{pgfscope}%
\begin{pgfscope}%
\pgfpathrectangle{\pgfqpoint{0.887500in}{0.275000in}}{\pgfqpoint{4.225000in}{4.225000in}}%
\pgfusepath{clip}%
\pgfsetbuttcap%
\pgfsetroundjoin%
\definecolor{currentfill}{rgb}{0.129933,0.559582,0.551864}%
\pgfsetfillcolor{currentfill}%
\pgfsetfillopacity{0.700000}%
\pgfsetlinewidth{0.501875pt}%
\definecolor{currentstroke}{rgb}{1.000000,1.000000,1.000000}%
\pgfsetstrokecolor{currentstroke}%
\pgfsetstrokeopacity{0.500000}%
\pgfsetdash{}{0pt}%
\pgfpathmoveto{\pgfqpoint{1.694206in}{2.446155in}}%
\pgfpathlineto{\pgfqpoint{1.705891in}{2.449489in}}%
\pgfpathlineto{\pgfqpoint{1.717570in}{2.452822in}}%
\pgfpathlineto{\pgfqpoint{1.729242in}{2.456155in}}%
\pgfpathlineto{\pgfqpoint{1.740910in}{2.459490in}}%
\pgfpathlineto{\pgfqpoint{1.752571in}{2.462826in}}%
\pgfpathlineto{\pgfqpoint{1.746794in}{2.471024in}}%
\pgfpathlineto{\pgfqpoint{1.741022in}{2.479202in}}%
\pgfpathlineto{\pgfqpoint{1.735255in}{2.487362in}}%
\pgfpathlineto{\pgfqpoint{1.729491in}{2.495505in}}%
\pgfpathlineto{\pgfqpoint{1.723732in}{2.503631in}}%
\pgfpathlineto{\pgfqpoint{1.712083in}{2.500314in}}%
\pgfpathlineto{\pgfqpoint{1.700429in}{2.496998in}}%
\pgfpathlineto{\pgfqpoint{1.688769in}{2.493683in}}%
\pgfpathlineto{\pgfqpoint{1.677102in}{2.490367in}}%
\pgfpathlineto{\pgfqpoint{1.665430in}{2.487051in}}%
\pgfpathlineto{\pgfqpoint{1.671177in}{2.478907in}}%
\pgfpathlineto{\pgfqpoint{1.676928in}{2.470747in}}%
\pgfpathlineto{\pgfqpoint{1.682683in}{2.462569in}}%
\pgfpathlineto{\pgfqpoint{1.688443in}{2.454372in}}%
\pgfpathclose%
\pgfusepath{stroke,fill}%
\end{pgfscope}%
\begin{pgfscope}%
\pgfpathrectangle{\pgfqpoint{0.887500in}{0.275000in}}{\pgfqpoint{4.225000in}{4.225000in}}%
\pgfusepath{clip}%
\pgfsetbuttcap%
\pgfsetroundjoin%
\definecolor{currentfill}{rgb}{0.162142,0.474838,0.558140}%
\pgfsetfillcolor{currentfill}%
\pgfsetfillopacity{0.700000}%
\pgfsetlinewidth{0.501875pt}%
\definecolor{currentstroke}{rgb}{1.000000,1.000000,1.000000}%
\pgfsetstrokecolor{currentstroke}%
\pgfsetstrokeopacity{0.500000}%
\pgfsetdash{}{0pt}%
\pgfpathmoveto{\pgfqpoint{2.655390in}{2.263536in}}%
\pgfpathlineto{\pgfqpoint{2.666841in}{2.266512in}}%
\pgfpathlineto{\pgfqpoint{2.678286in}{2.269544in}}%
\pgfpathlineto{\pgfqpoint{2.689724in}{2.272722in}}%
\pgfpathlineto{\pgfqpoint{2.701154in}{2.276138in}}%
\pgfpathlineto{\pgfqpoint{2.712575in}{2.279885in}}%
\pgfpathlineto{\pgfqpoint{2.706470in}{2.288747in}}%
\pgfpathlineto{\pgfqpoint{2.700369in}{2.297585in}}%
\pgfpathlineto{\pgfqpoint{2.694272in}{2.306398in}}%
\pgfpathlineto{\pgfqpoint{2.688179in}{2.315187in}}%
\pgfpathlineto{\pgfqpoint{2.682090in}{2.323953in}}%
\pgfpathlineto{\pgfqpoint{2.670680in}{2.320149in}}%
\pgfpathlineto{\pgfqpoint{2.659262in}{2.316714in}}%
\pgfpathlineto{\pgfqpoint{2.647835in}{2.313548in}}%
\pgfpathlineto{\pgfqpoint{2.636400in}{2.310549in}}%
\pgfpathlineto{\pgfqpoint{2.624959in}{2.307614in}}%
\pgfpathlineto{\pgfqpoint{2.631037in}{2.298847in}}%
\pgfpathlineto{\pgfqpoint{2.637119in}{2.290056in}}%
\pgfpathlineto{\pgfqpoint{2.643205in}{2.281242in}}%
\pgfpathlineto{\pgfqpoint{2.649295in}{2.272402in}}%
\pgfpathclose%
\pgfusepath{stroke,fill}%
\end{pgfscope}%
\begin{pgfscope}%
\pgfpathrectangle{\pgfqpoint{0.887500in}{0.275000in}}{\pgfqpoint{4.225000in}{4.225000in}}%
\pgfusepath{clip}%
\pgfsetbuttcap%
\pgfsetroundjoin%
\definecolor{currentfill}{rgb}{0.153364,0.497000,0.557724}%
\pgfsetfillcolor{currentfill}%
\pgfsetfillopacity{0.700000}%
\pgfsetlinewidth{0.501875pt}%
\definecolor{currentstroke}{rgb}{1.000000,1.000000,1.000000}%
\pgfsetstrokecolor{currentstroke}%
\pgfsetstrokeopacity{0.500000}%
\pgfsetdash{}{0pt}%
\pgfpathmoveto{\pgfqpoint{2.944836in}{2.252082in}}%
\pgfpathlineto{\pgfqpoint{2.956149in}{2.273858in}}%
\pgfpathlineto{\pgfqpoint{2.967457in}{2.299518in}}%
\pgfpathlineto{\pgfqpoint{2.978767in}{2.327922in}}%
\pgfpathlineto{\pgfqpoint{2.990083in}{2.357925in}}%
\pgfpathlineto{\pgfqpoint{3.001409in}{2.388379in}}%
\pgfpathlineto{\pgfqpoint{2.995200in}{2.397603in}}%
\pgfpathlineto{\pgfqpoint{2.988996in}{2.406538in}}%
\pgfpathlineto{\pgfqpoint{2.982796in}{2.415302in}}%
\pgfpathlineto{\pgfqpoint{2.976600in}{2.424012in}}%
\pgfpathlineto{\pgfqpoint{2.970408in}{2.432786in}}%
\pgfpathlineto{\pgfqpoint{2.959114in}{2.402547in}}%
\pgfpathlineto{\pgfqpoint{2.947828in}{2.372919in}}%
\pgfpathlineto{\pgfqpoint{2.936547in}{2.344979in}}%
\pgfpathlineto{\pgfqpoint{2.925265in}{2.319794in}}%
\pgfpathlineto{\pgfqpoint{2.913975in}{2.298432in}}%
\pgfpathlineto{\pgfqpoint{2.920139in}{2.289305in}}%
\pgfpathlineto{\pgfqpoint{2.926308in}{2.280087in}}%
\pgfpathlineto{\pgfqpoint{2.932480in}{2.270797in}}%
\pgfpathlineto{\pgfqpoint{2.938656in}{2.261456in}}%
\pgfpathclose%
\pgfusepath{stroke,fill}%
\end{pgfscope}%
\begin{pgfscope}%
\pgfpathrectangle{\pgfqpoint{0.887500in}{0.275000in}}{\pgfqpoint{4.225000in}{4.225000in}}%
\pgfusepath{clip}%
\pgfsetbuttcap%
\pgfsetroundjoin%
\definecolor{currentfill}{rgb}{0.140536,0.530132,0.555659}%
\pgfsetfillcolor{currentfill}%
\pgfsetfillopacity{0.700000}%
\pgfsetlinewidth{0.501875pt}%
\definecolor{currentstroke}{rgb}{1.000000,1.000000,1.000000}%
\pgfsetstrokecolor{currentstroke}%
\pgfsetstrokeopacity{0.500000}%
\pgfsetdash{}{0pt}%
\pgfpathmoveto{\pgfqpoint{2.014529in}{2.387699in}}%
\pgfpathlineto{\pgfqpoint{2.026136in}{2.391013in}}%
\pgfpathlineto{\pgfqpoint{2.037737in}{2.394332in}}%
\pgfpathlineto{\pgfqpoint{2.049333in}{2.397658in}}%
\pgfpathlineto{\pgfqpoint{2.060922in}{2.400993in}}%
\pgfpathlineto{\pgfqpoint{2.072505in}{2.404340in}}%
\pgfpathlineto{\pgfqpoint{2.066616in}{2.412751in}}%
\pgfpathlineto{\pgfqpoint{2.060731in}{2.421143in}}%
\pgfpathlineto{\pgfqpoint{2.054850in}{2.429515in}}%
\pgfpathlineto{\pgfqpoint{2.048973in}{2.437868in}}%
\pgfpathlineto{\pgfqpoint{2.043101in}{2.446202in}}%
\pgfpathlineto{\pgfqpoint{2.031530in}{2.442874in}}%
\pgfpathlineto{\pgfqpoint{2.019952in}{2.439560in}}%
\pgfpathlineto{\pgfqpoint{2.008369in}{2.436255in}}%
\pgfpathlineto{\pgfqpoint{1.996779in}{2.432959in}}%
\pgfpathlineto{\pgfqpoint{1.985184in}{2.429667in}}%
\pgfpathlineto{\pgfqpoint{1.991045in}{2.421313in}}%
\pgfpathlineto{\pgfqpoint{1.996909in}{2.412940in}}%
\pgfpathlineto{\pgfqpoint{2.002778in}{2.404546in}}%
\pgfpathlineto{\pgfqpoint{2.008651in}{2.396133in}}%
\pgfpathclose%
\pgfusepath{stroke,fill}%
\end{pgfscope}%
\begin{pgfscope}%
\pgfpathrectangle{\pgfqpoint{0.887500in}{0.275000in}}{\pgfqpoint{4.225000in}{4.225000in}}%
\pgfusepath{clip}%
\pgfsetbuttcap%
\pgfsetroundjoin%
\definecolor{currentfill}{rgb}{0.344074,0.780029,0.397381}%
\pgfsetfillcolor{currentfill}%
\pgfsetfillopacity{0.700000}%
\pgfsetlinewidth{0.501875pt}%
\definecolor{currentstroke}{rgb}{1.000000,1.000000,1.000000}%
\pgfsetstrokecolor{currentstroke}%
\pgfsetstrokeopacity{0.500000}%
\pgfsetdash{}{0pt}%
\pgfpathmoveto{\pgfqpoint{2.990486in}{2.898469in}}%
\pgfpathlineto{\pgfqpoint{3.001807in}{2.925215in}}%
\pgfpathlineto{\pgfqpoint{3.013137in}{2.950801in}}%
\pgfpathlineto{\pgfqpoint{3.024476in}{2.974756in}}%
\pgfpathlineto{\pgfqpoint{3.035822in}{2.996605in}}%
\pgfpathlineto{\pgfqpoint{3.047173in}{3.015875in}}%
\pgfpathlineto{\pgfqpoint{3.040951in}{3.022309in}}%
\pgfpathlineto{\pgfqpoint{3.034735in}{3.027534in}}%
\pgfpathlineto{\pgfqpoint{3.028524in}{3.031686in}}%
\pgfpathlineto{\pgfqpoint{3.022320in}{3.034900in}}%
\pgfpathlineto{\pgfqpoint{3.016122in}{3.037312in}}%
\pgfpathlineto{\pgfqpoint{3.004804in}{3.014683in}}%
\pgfpathlineto{\pgfqpoint{2.993495in}{2.990117in}}%
\pgfpathlineto{\pgfqpoint{2.982195in}{2.964049in}}%
\pgfpathlineto{\pgfqpoint{2.970904in}{2.936913in}}%
\pgfpathlineto{\pgfqpoint{2.959624in}{2.909142in}}%
\pgfpathlineto{\pgfqpoint{2.965778in}{2.909879in}}%
\pgfpathlineto{\pgfqpoint{2.971941in}{2.909459in}}%
\pgfpathlineto{\pgfqpoint{2.978114in}{2.907604in}}%
\pgfpathlineto{\pgfqpoint{2.984295in}{2.904033in}}%
\pgfpathclose%
\pgfusepath{stroke,fill}%
\end{pgfscope}%
\begin{pgfscope}%
\pgfpathrectangle{\pgfqpoint{0.887500in}{0.275000in}}{\pgfqpoint{4.225000in}{4.225000in}}%
\pgfusepath{clip}%
\pgfsetbuttcap%
\pgfsetroundjoin%
\definecolor{currentfill}{rgb}{0.327796,0.773980,0.406640}%
\pgfsetfillcolor{currentfill}%
\pgfsetfillopacity{0.700000}%
\pgfsetlinewidth{0.501875pt}%
\definecolor{currentstroke}{rgb}{1.000000,1.000000,1.000000}%
\pgfsetstrokecolor{currentstroke}%
\pgfsetstrokeopacity{0.500000}%
\pgfsetdash{}{0pt}%
\pgfpathmoveto{\pgfqpoint{3.599962in}{2.911910in}}%
\pgfpathlineto{\pgfqpoint{3.611175in}{2.915271in}}%
\pgfpathlineto{\pgfqpoint{3.622383in}{2.918619in}}%
\pgfpathlineto{\pgfqpoint{3.633585in}{2.921967in}}%
\pgfpathlineto{\pgfqpoint{3.644782in}{2.925332in}}%
\pgfpathlineto{\pgfqpoint{3.655974in}{2.928729in}}%
\pgfpathlineto{\pgfqpoint{3.649634in}{2.941979in}}%
\pgfpathlineto{\pgfqpoint{3.643298in}{2.955226in}}%
\pgfpathlineto{\pgfqpoint{3.636964in}{2.968477in}}%
\pgfpathlineto{\pgfqpoint{3.630633in}{2.981737in}}%
\pgfpathlineto{\pgfqpoint{3.624305in}{2.995009in}}%
\pgfpathlineto{\pgfqpoint{3.613117in}{2.991637in}}%
\pgfpathlineto{\pgfqpoint{3.601924in}{2.988286in}}%
\pgfpathlineto{\pgfqpoint{3.590726in}{2.984941in}}%
\pgfpathlineto{\pgfqpoint{3.579522in}{2.981590in}}%
\pgfpathlineto{\pgfqpoint{3.568312in}{2.978219in}}%
\pgfpathlineto{\pgfqpoint{3.574637in}{2.964987in}}%
\pgfpathlineto{\pgfqpoint{3.580965in}{2.951740in}}%
\pgfpathlineto{\pgfqpoint{3.587295in}{2.938479in}}%
\pgfpathlineto{\pgfqpoint{3.593627in}{2.925202in}}%
\pgfpathclose%
\pgfusepath{stroke,fill}%
\end{pgfscope}%
\begin{pgfscope}%
\pgfpathrectangle{\pgfqpoint{0.887500in}{0.275000in}}{\pgfqpoint{4.225000in}{4.225000in}}%
\pgfusepath{clip}%
\pgfsetbuttcap%
\pgfsetroundjoin%
\definecolor{currentfill}{rgb}{0.150476,0.504369,0.557430}%
\pgfsetfillcolor{currentfill}%
\pgfsetfillopacity{0.700000}%
\pgfsetlinewidth{0.501875pt}%
\definecolor{currentstroke}{rgb}{1.000000,1.000000,1.000000}%
\pgfsetstrokecolor{currentstroke}%
\pgfsetstrokeopacity{0.500000}%
\pgfsetdash{}{0pt}%
\pgfpathmoveto{\pgfqpoint{2.334940in}{2.327003in}}%
\pgfpathlineto{\pgfqpoint{2.346468in}{2.330311in}}%
\pgfpathlineto{\pgfqpoint{2.357991in}{2.333613in}}%
\pgfpathlineto{\pgfqpoint{2.369508in}{2.336914in}}%
\pgfpathlineto{\pgfqpoint{2.381019in}{2.340220in}}%
\pgfpathlineto{\pgfqpoint{2.392524in}{2.343535in}}%
\pgfpathlineto{\pgfqpoint{2.386525in}{2.352127in}}%
\pgfpathlineto{\pgfqpoint{2.380531in}{2.360702in}}%
\pgfpathlineto{\pgfqpoint{2.374541in}{2.369260in}}%
\pgfpathlineto{\pgfqpoint{2.368555in}{2.377802in}}%
\pgfpathlineto{\pgfqpoint{2.362573in}{2.386326in}}%
\pgfpathlineto{\pgfqpoint{2.351079in}{2.383022in}}%
\pgfpathlineto{\pgfqpoint{2.339580in}{2.379729in}}%
\pgfpathlineto{\pgfqpoint{2.328074in}{2.376442in}}%
\pgfpathlineto{\pgfqpoint{2.316563in}{2.373156in}}%
\pgfpathlineto{\pgfqpoint{2.305046in}{2.369867in}}%
\pgfpathlineto{\pgfqpoint{2.311017in}{2.361332in}}%
\pgfpathlineto{\pgfqpoint{2.316991in}{2.352778in}}%
\pgfpathlineto{\pgfqpoint{2.322970in}{2.344206in}}%
\pgfpathlineto{\pgfqpoint{2.328953in}{2.335614in}}%
\pgfpathclose%
\pgfusepath{stroke,fill}%
\end{pgfscope}%
\begin{pgfscope}%
\pgfpathrectangle{\pgfqpoint{0.887500in}{0.275000in}}{\pgfqpoint{4.225000in}{4.225000in}}%
\pgfusepath{clip}%
\pgfsetbuttcap%
\pgfsetroundjoin%
\definecolor{currentfill}{rgb}{0.575563,0.844566,0.256415}%
\pgfsetfillcolor{currentfill}%
\pgfsetfillopacity{0.700000}%
\pgfsetlinewidth{0.501875pt}%
\definecolor{currentstroke}{rgb}{1.000000,1.000000,1.000000}%
\pgfsetstrokecolor{currentstroke}%
\pgfsetstrokeopacity{0.500000}%
\pgfsetdash{}{0pt}%
\pgfpathmoveto{\pgfqpoint{3.160650in}{3.120088in}}%
\pgfpathlineto{\pgfqpoint{3.171982in}{3.127869in}}%
\pgfpathlineto{\pgfqpoint{3.183310in}{3.135248in}}%
\pgfpathlineto{\pgfqpoint{3.194632in}{3.142125in}}%
\pgfpathlineto{\pgfqpoint{3.205949in}{3.148403in}}%
\pgfpathlineto{\pgfqpoint{3.217260in}{3.154074in}}%
\pgfpathlineto{\pgfqpoint{3.211004in}{3.165849in}}%
\pgfpathlineto{\pgfqpoint{3.204751in}{3.177644in}}%
\pgfpathlineto{\pgfqpoint{3.198501in}{3.189441in}}%
\pgfpathlineto{\pgfqpoint{3.192254in}{3.201223in}}%
\pgfpathlineto{\pgfqpoint{3.186010in}{3.212971in}}%
\pgfpathlineto{\pgfqpoint{3.174708in}{3.207583in}}%
\pgfpathlineto{\pgfqpoint{3.163399in}{3.201358in}}%
\pgfpathlineto{\pgfqpoint{3.152085in}{3.194243in}}%
\pgfpathlineto{\pgfqpoint{3.140765in}{3.186266in}}%
\pgfpathlineto{\pgfqpoint{3.129441in}{3.177455in}}%
\pgfpathlineto{\pgfqpoint{3.135677in}{3.166043in}}%
\pgfpathlineto{\pgfqpoint{3.141915in}{3.154576in}}%
\pgfpathlineto{\pgfqpoint{3.148157in}{3.143079in}}%
\pgfpathlineto{\pgfqpoint{3.154402in}{3.131575in}}%
\pgfpathclose%
\pgfusepath{stroke,fill}%
\end{pgfscope}%
\begin{pgfscope}%
\pgfpathrectangle{\pgfqpoint{0.887500in}{0.275000in}}{\pgfqpoint{4.225000in}{4.225000in}}%
\pgfusepath{clip}%
\pgfsetbuttcap%
\pgfsetroundjoin%
\definecolor{currentfill}{rgb}{0.134692,0.658636,0.517649}%
\pgfsetfillcolor{currentfill}%
\pgfsetfillopacity{0.700000}%
\pgfsetlinewidth{0.501875pt}%
\definecolor{currentstroke}{rgb}{1.000000,1.000000,1.000000}%
\pgfsetstrokecolor{currentstroke}%
\pgfsetstrokeopacity{0.500000}%
\pgfsetdash{}{0pt}%
\pgfpathmoveto{\pgfqpoint{3.982622in}{2.639095in}}%
\pgfpathlineto{\pgfqpoint{3.993742in}{2.642477in}}%
\pgfpathlineto{\pgfqpoint{4.004857in}{2.645849in}}%
\pgfpathlineto{\pgfqpoint{4.015966in}{2.649216in}}%
\pgfpathlineto{\pgfqpoint{4.027069in}{2.652580in}}%
\pgfpathlineto{\pgfqpoint{4.038167in}{2.655945in}}%
\pgfpathlineto{\pgfqpoint{4.031767in}{2.670002in}}%
\pgfpathlineto{\pgfqpoint{4.025368in}{2.684044in}}%
\pgfpathlineto{\pgfqpoint{4.018972in}{2.698067in}}%
\pgfpathlineto{\pgfqpoint{4.012578in}{2.712065in}}%
\pgfpathlineto{\pgfqpoint{4.006185in}{2.726031in}}%
\pgfpathlineto{\pgfqpoint{3.995089in}{2.722637in}}%
\pgfpathlineto{\pgfqpoint{3.983987in}{2.719238in}}%
\pgfpathlineto{\pgfqpoint{3.972879in}{2.715830in}}%
\pgfpathlineto{\pgfqpoint{3.961766in}{2.712411in}}%
\pgfpathlineto{\pgfqpoint{3.950647in}{2.708979in}}%
\pgfpathlineto{\pgfqpoint{3.957038in}{2.695052in}}%
\pgfpathlineto{\pgfqpoint{3.963431in}{2.681095in}}%
\pgfpathlineto{\pgfqpoint{3.969826in}{2.667113in}}%
\pgfpathlineto{\pgfqpoint{3.976223in}{2.653112in}}%
\pgfpathclose%
\pgfusepath{stroke,fill}%
\end{pgfscope}%
\begin{pgfscope}%
\pgfpathrectangle{\pgfqpoint{0.887500in}{0.275000in}}{\pgfqpoint{4.225000in}{4.225000in}}%
\pgfusepath{clip}%
\pgfsetbuttcap%
\pgfsetroundjoin%
\definecolor{currentfill}{rgb}{0.377779,0.791781,0.377939}%
\pgfsetfillcolor{currentfill}%
\pgfsetfillopacity{0.700000}%
\pgfsetlinewidth{0.501875pt}%
\definecolor{currentstroke}{rgb}{1.000000,1.000000,1.000000}%
\pgfsetstrokecolor{currentstroke}%
\pgfsetstrokeopacity{0.500000}%
\pgfsetdash{}{0pt}%
\pgfpathmoveto{\pgfqpoint{3.512176in}{2.960777in}}%
\pgfpathlineto{\pgfqpoint{3.523415in}{2.964351in}}%
\pgfpathlineto{\pgfqpoint{3.534648in}{2.967882in}}%
\pgfpathlineto{\pgfqpoint{3.545875in}{2.971371in}}%
\pgfpathlineto{\pgfqpoint{3.557097in}{2.974817in}}%
\pgfpathlineto{\pgfqpoint{3.568312in}{2.978219in}}%
\pgfpathlineto{\pgfqpoint{3.561989in}{2.991430in}}%
\pgfpathlineto{\pgfqpoint{3.555669in}{3.004607in}}%
\pgfpathlineto{\pgfqpoint{3.549351in}{3.017737in}}%
\pgfpathlineto{\pgfqpoint{3.543035in}{3.030809in}}%
\pgfpathlineto{\pgfqpoint{3.536720in}{3.043809in}}%
\pgfpathlineto{\pgfqpoint{3.525507in}{3.040273in}}%
\pgfpathlineto{\pgfqpoint{3.514289in}{3.036702in}}%
\pgfpathlineto{\pgfqpoint{3.503064in}{3.033099in}}%
\pgfpathlineto{\pgfqpoint{3.491834in}{3.029468in}}%
\pgfpathlineto{\pgfqpoint{3.480599in}{3.025812in}}%
\pgfpathlineto{\pgfqpoint{3.486910in}{3.012895in}}%
\pgfpathlineto{\pgfqpoint{3.493223in}{2.999935in}}%
\pgfpathlineto{\pgfqpoint{3.499538in}{2.986930in}}%
\pgfpathlineto{\pgfqpoint{3.505856in}{2.973878in}}%
\pgfpathclose%
\pgfusepath{stroke,fill}%
\end{pgfscope}%
\begin{pgfscope}%
\pgfpathrectangle{\pgfqpoint{0.887500in}{0.275000in}}{\pgfqpoint{4.225000in}{4.225000in}}%
\pgfusepath{clip}%
\pgfsetbuttcap%
\pgfsetroundjoin%
\definecolor{currentfill}{rgb}{0.196571,0.711827,0.479221}%
\pgfsetfillcolor{currentfill}%
\pgfsetfillopacity{0.700000}%
\pgfsetlinewidth{0.501875pt}%
\definecolor{currentstroke}{rgb}{1.000000,1.000000,1.000000}%
\pgfsetstrokecolor{currentstroke}%
\pgfsetstrokeopacity{0.500000}%
\pgfsetdash{}{0pt}%
\pgfpathmoveto{\pgfqpoint{2.964940in}{2.701346in}}%
\pgfpathlineto{\pgfqpoint{2.976243in}{2.729645in}}%
\pgfpathlineto{\pgfqpoint{2.987554in}{2.757742in}}%
\pgfpathlineto{\pgfqpoint{2.998874in}{2.786055in}}%
\pgfpathlineto{\pgfqpoint{3.010201in}{2.814800in}}%
\pgfpathlineto{\pgfqpoint{3.021538in}{2.843518in}}%
\pgfpathlineto{\pgfqpoint{3.015318in}{2.857307in}}%
\pgfpathlineto{\pgfqpoint{3.009102in}{2.869951in}}%
\pgfpathlineto{\pgfqpoint{3.002891in}{2.881196in}}%
\pgfpathlineto{\pgfqpoint{2.996685in}{2.890787in}}%
\pgfpathlineto{\pgfqpoint{2.990486in}{2.898469in}}%
\pgfpathlineto{\pgfqpoint{2.979174in}{2.871035in}}%
\pgfpathlineto{\pgfqpoint{2.967871in}{2.843383in}}%
\pgfpathlineto{\pgfqpoint{2.956575in}{2.815861in}}%
\pgfpathlineto{\pgfqpoint{2.945288in}{2.788241in}}%
\pgfpathlineto{\pgfqpoint{2.934011in}{2.760117in}}%
\pgfpathlineto{\pgfqpoint{2.940181in}{2.751668in}}%
\pgfpathlineto{\pgfqpoint{2.946361in}{2.741228in}}%
\pgfpathlineto{\pgfqpoint{2.952549in}{2.729135in}}%
\pgfpathlineto{\pgfqpoint{2.958742in}{2.715728in}}%
\pgfpathclose%
\pgfusepath{stroke,fill}%
\end{pgfscope}%
\begin{pgfscope}%
\pgfpathrectangle{\pgfqpoint{0.887500in}{0.275000in}}{\pgfqpoint{4.225000in}{4.225000in}}%
\pgfusepath{clip}%
\pgfsetbuttcap%
\pgfsetroundjoin%
\definecolor{currentfill}{rgb}{0.133743,0.548535,0.553541}%
\pgfsetfillcolor{currentfill}%
\pgfsetfillopacity{0.700000}%
\pgfsetlinewidth{0.501875pt}%
\definecolor{currentstroke}{rgb}{1.000000,1.000000,1.000000}%
\pgfsetstrokecolor{currentstroke}%
\pgfsetstrokeopacity{0.500000}%
\pgfsetdash{}{0pt}%
\pgfpathmoveto{\pgfqpoint{4.277413in}{2.406236in}}%
\pgfpathlineto{\pgfqpoint{4.288456in}{2.409518in}}%
\pgfpathlineto{\pgfqpoint{4.299494in}{2.412791in}}%
\pgfpathlineto{\pgfqpoint{4.310526in}{2.416062in}}%
\pgfpathlineto{\pgfqpoint{4.321554in}{2.419338in}}%
\pgfpathlineto{\pgfqpoint{4.332576in}{2.422623in}}%
\pgfpathlineto{\pgfqpoint{4.326129in}{2.437015in}}%
\pgfpathlineto{\pgfqpoint{4.319684in}{2.451393in}}%
\pgfpathlineto{\pgfqpoint{4.313242in}{2.465759in}}%
\pgfpathlineto{\pgfqpoint{4.306801in}{2.480110in}}%
\pgfpathlineto{\pgfqpoint{4.300363in}{2.494448in}}%
\pgfpathlineto{\pgfqpoint{4.289342in}{2.491141in}}%
\pgfpathlineto{\pgfqpoint{4.278315in}{2.487835in}}%
\pgfpathlineto{\pgfqpoint{4.267283in}{2.484527in}}%
\pgfpathlineto{\pgfqpoint{4.256245in}{2.481216in}}%
\pgfpathlineto{\pgfqpoint{4.245202in}{2.477897in}}%
\pgfpathlineto{\pgfqpoint{4.251642in}{2.463661in}}%
\pgfpathlineto{\pgfqpoint{4.258083in}{2.449387in}}%
\pgfpathlineto{\pgfqpoint{4.264526in}{2.435065in}}%
\pgfpathlineto{\pgfqpoint{4.270969in}{2.420685in}}%
\pgfpathclose%
\pgfusepath{stroke,fill}%
\end{pgfscope}%
\begin{pgfscope}%
\pgfpathrectangle{\pgfqpoint{0.887500in}{0.275000in}}{\pgfqpoint{4.225000in}{4.225000in}}%
\pgfusepath{clip}%
\pgfsetbuttcap%
\pgfsetroundjoin%
\definecolor{currentfill}{rgb}{0.430983,0.808473,0.346476}%
\pgfsetfillcolor{currentfill}%
\pgfsetfillopacity{0.700000}%
\pgfsetlinewidth{0.501875pt}%
\definecolor{currentstroke}{rgb}{1.000000,1.000000,1.000000}%
\pgfsetstrokecolor{currentstroke}%
\pgfsetstrokeopacity{0.500000}%
\pgfsetdash{}{0pt}%
\pgfpathmoveto{\pgfqpoint{3.424339in}{3.007282in}}%
\pgfpathlineto{\pgfqpoint{3.435602in}{3.011012in}}%
\pgfpathlineto{\pgfqpoint{3.446859in}{3.014732in}}%
\pgfpathlineto{\pgfqpoint{3.458111in}{3.018440in}}%
\pgfpathlineto{\pgfqpoint{3.469358in}{3.022135in}}%
\pgfpathlineto{\pgfqpoint{3.480599in}{3.025812in}}%
\pgfpathlineto{\pgfqpoint{3.474290in}{3.038687in}}%
\pgfpathlineto{\pgfqpoint{3.467984in}{3.051524in}}%
\pgfpathlineto{\pgfqpoint{3.461680in}{3.064322in}}%
\pgfpathlineto{\pgfqpoint{3.455379in}{3.077086in}}%
\pgfpathlineto{\pgfqpoint{3.449080in}{3.089816in}}%
\pgfpathlineto{\pgfqpoint{3.437844in}{3.086231in}}%
\pgfpathlineto{\pgfqpoint{3.426603in}{3.082623in}}%
\pgfpathlineto{\pgfqpoint{3.415357in}{3.078994in}}%
\pgfpathlineto{\pgfqpoint{3.404105in}{3.075347in}}%
\pgfpathlineto{\pgfqpoint{3.392847in}{3.071691in}}%
\pgfpathlineto{\pgfqpoint{3.399140in}{3.058822in}}%
\pgfpathlineto{\pgfqpoint{3.405436in}{3.045953in}}%
\pgfpathlineto{\pgfqpoint{3.411735in}{3.033077in}}%
\pgfpathlineto{\pgfqpoint{3.418036in}{3.020188in}}%
\pgfpathclose%
\pgfusepath{stroke,fill}%
\end{pgfscope}%
\begin{pgfscope}%
\pgfpathrectangle{\pgfqpoint{0.887500in}{0.275000in}}{\pgfqpoint{4.225000in}{4.225000in}}%
\pgfusepath{clip}%
\pgfsetbuttcap%
\pgfsetroundjoin%
\definecolor{currentfill}{rgb}{0.545524,0.838039,0.275626}%
\pgfsetfillcolor{currentfill}%
\pgfsetfillopacity{0.700000}%
\pgfsetlinewidth{0.501875pt}%
\definecolor{currentstroke}{rgb}{1.000000,1.000000,1.000000}%
\pgfsetstrokecolor{currentstroke}%
\pgfsetstrokeopacity{0.500000}%
\pgfsetdash{}{0pt}%
\pgfpathmoveto{\pgfqpoint{3.248586in}{3.095072in}}%
\pgfpathlineto{\pgfqpoint{3.259897in}{3.100032in}}%
\pgfpathlineto{\pgfqpoint{3.271201in}{3.104573in}}%
\pgfpathlineto{\pgfqpoint{3.282498in}{3.108788in}}%
\pgfpathlineto{\pgfqpoint{3.293789in}{3.112773in}}%
\pgfpathlineto{\pgfqpoint{3.305074in}{3.116621in}}%
\pgfpathlineto{\pgfqpoint{3.298799in}{3.128877in}}%
\pgfpathlineto{\pgfqpoint{3.292526in}{3.140992in}}%
\pgfpathlineto{\pgfqpoint{3.286255in}{3.152956in}}%
\pgfpathlineto{\pgfqpoint{3.279986in}{3.164759in}}%
\pgfpathlineto{\pgfqpoint{3.273719in}{3.176392in}}%
\pgfpathlineto{\pgfqpoint{3.262439in}{3.172435in}}%
\pgfpathlineto{\pgfqpoint{3.251154in}{3.168314in}}%
\pgfpathlineto{\pgfqpoint{3.239862in}{3.163940in}}%
\pgfpathlineto{\pgfqpoint{3.228564in}{3.159223in}}%
\pgfpathlineto{\pgfqpoint{3.217260in}{3.154074in}}%
\pgfpathlineto{\pgfqpoint{3.223519in}{3.142324in}}%
\pgfpathlineto{\pgfqpoint{3.229782in}{3.130575in}}%
\pgfpathlineto{\pgfqpoint{3.236047in}{3.118802in}}%
\pgfpathlineto{\pgfqpoint{3.242315in}{3.106976in}}%
\pgfpathclose%
\pgfusepath{stroke,fill}%
\end{pgfscope}%
\begin{pgfscope}%
\pgfpathrectangle{\pgfqpoint{0.887500in}{0.275000in}}{\pgfqpoint{4.225000in}{4.225000in}}%
\pgfusepath{clip}%
\pgfsetbuttcap%
\pgfsetroundjoin%
\definecolor{currentfill}{rgb}{0.122606,0.585371,0.546557}%
\pgfsetfillcolor{currentfill}%
\pgfsetfillopacity{0.700000}%
\pgfsetlinewidth{0.501875pt}%
\definecolor{currentstroke}{rgb}{1.000000,1.000000,1.000000}%
\pgfsetstrokecolor{currentstroke}%
\pgfsetstrokeopacity{0.500000}%
\pgfsetdash{}{0pt}%
\pgfpathmoveto{\pgfqpoint{2.970408in}{2.432786in}}%
\pgfpathlineto{\pgfqpoint{2.981713in}{2.462560in}}%
\pgfpathlineto{\pgfqpoint{2.993030in}{2.490970in}}%
\pgfpathlineto{\pgfqpoint{3.004357in}{2.518032in}}%
\pgfpathlineto{\pgfqpoint{3.015693in}{2.544045in}}%
\pgfpathlineto{\pgfqpoint{3.027038in}{2.569310in}}%
\pgfpathlineto{\pgfqpoint{3.020819in}{2.579382in}}%
\pgfpathlineto{\pgfqpoint{3.014604in}{2.589987in}}%
\pgfpathlineto{\pgfqpoint{3.008391in}{2.601289in}}%
\pgfpathlineto{\pgfqpoint{3.002180in}{2.613448in}}%
\pgfpathlineto{\pgfqpoint{2.995970in}{2.626624in}}%
\pgfpathlineto{\pgfqpoint{2.984653in}{2.599359in}}%
\pgfpathlineto{\pgfqpoint{2.973346in}{2.571566in}}%
\pgfpathlineto{\pgfqpoint{2.962049in}{2.542885in}}%
\pgfpathlineto{\pgfqpoint{2.950764in}{2.512957in}}%
\pgfpathlineto{\pgfqpoint{2.939493in}{2.481724in}}%
\pgfpathlineto{\pgfqpoint{2.945672in}{2.470870in}}%
\pgfpathlineto{\pgfqpoint{2.951852in}{2.460666in}}%
\pgfpathlineto{\pgfqpoint{2.958035in}{2.450996in}}%
\pgfpathlineto{\pgfqpoint{2.964220in}{2.441742in}}%
\pgfpathclose%
\pgfusepath{stroke,fill}%
\end{pgfscope}%
\begin{pgfscope}%
\pgfpathrectangle{\pgfqpoint{0.887500in}{0.275000in}}{\pgfqpoint{4.225000in}{4.225000in}}%
\pgfusepath{clip}%
\pgfsetbuttcap%
\pgfsetroundjoin%
\definecolor{currentfill}{rgb}{0.487026,0.823929,0.312321}%
\pgfsetfillcolor{currentfill}%
\pgfsetfillopacity{0.700000}%
\pgfsetlinewidth{0.501875pt}%
\definecolor{currentstroke}{rgb}{1.000000,1.000000,1.000000}%
\pgfsetstrokecolor{currentstroke}%
\pgfsetstrokeopacity{0.500000}%
\pgfsetdash{}{0pt}%
\pgfpathmoveto{\pgfqpoint{3.336483in}{3.053592in}}%
\pgfpathlineto{\pgfqpoint{3.347766in}{3.057158in}}%
\pgfpathlineto{\pgfqpoint{3.359044in}{3.060759in}}%
\pgfpathlineto{\pgfqpoint{3.370317in}{3.064388in}}%
\pgfpathlineto{\pgfqpoint{3.381585in}{3.068035in}}%
\pgfpathlineto{\pgfqpoint{3.392847in}{3.071691in}}%
\pgfpathlineto{\pgfqpoint{3.386557in}{3.084552in}}%
\pgfpathlineto{\pgfqpoint{3.380269in}{3.097379in}}%
\pgfpathlineto{\pgfqpoint{3.373984in}{3.110141in}}%
\pgfpathlineto{\pgfqpoint{3.367700in}{3.122807in}}%
\pgfpathlineto{\pgfqpoint{3.361419in}{3.135349in}}%
\pgfpathlineto{\pgfqpoint{3.350161in}{3.131688in}}%
\pgfpathlineto{\pgfqpoint{3.338897in}{3.127970in}}%
\pgfpathlineto{\pgfqpoint{3.327628in}{3.124211in}}%
\pgfpathlineto{\pgfqpoint{3.316354in}{3.120424in}}%
\pgfpathlineto{\pgfqpoint{3.305074in}{3.116621in}}%
\pgfpathlineto{\pgfqpoint{3.311352in}{3.104236in}}%
\pgfpathlineto{\pgfqpoint{3.317631in}{3.091730in}}%
\pgfpathlineto{\pgfqpoint{3.323913in}{3.079114in}}%
\pgfpathlineto{\pgfqpoint{3.330197in}{3.066398in}}%
\pgfpathclose%
\pgfusepath{stroke,fill}%
\end{pgfscope}%
\begin{pgfscope}%
\pgfpathrectangle{\pgfqpoint{0.887500in}{0.275000in}}{\pgfqpoint{4.225000in}{4.225000in}}%
\pgfusepath{clip}%
\pgfsetbuttcap%
\pgfsetroundjoin%
\definecolor{currentfill}{rgb}{0.133743,0.548535,0.553541}%
\pgfsetfillcolor{currentfill}%
\pgfsetfillopacity{0.700000}%
\pgfsetlinewidth{0.501875pt}%
\definecolor{currentstroke}{rgb}{1.000000,1.000000,1.000000}%
\pgfsetstrokecolor{currentstroke}%
\pgfsetstrokeopacity{0.500000}%
\pgfsetdash{}{0pt}%
\pgfpathmoveto{\pgfqpoint{1.781518in}{2.421518in}}%
\pgfpathlineto{\pgfqpoint{1.793186in}{2.424874in}}%
\pgfpathlineto{\pgfqpoint{1.804848in}{2.428228in}}%
\pgfpathlineto{\pgfqpoint{1.816504in}{2.431578in}}%
\pgfpathlineto{\pgfqpoint{1.828155in}{2.434921in}}%
\pgfpathlineto{\pgfqpoint{1.839800in}{2.438255in}}%
\pgfpathlineto{\pgfqpoint{1.833990in}{2.446538in}}%
\pgfpathlineto{\pgfqpoint{1.828183in}{2.454799in}}%
\pgfpathlineto{\pgfqpoint{1.822382in}{2.463038in}}%
\pgfpathlineto{\pgfqpoint{1.816584in}{2.471257in}}%
\pgfpathlineto{\pgfqpoint{1.810791in}{2.479455in}}%
\pgfpathlineto{\pgfqpoint{1.799158in}{2.476143in}}%
\pgfpathlineto{\pgfqpoint{1.787520in}{2.472822in}}%
\pgfpathlineto{\pgfqpoint{1.775876in}{2.469494in}}%
\pgfpathlineto{\pgfqpoint{1.764226in}{2.466161in}}%
\pgfpathlineto{\pgfqpoint{1.752571in}{2.462826in}}%
\pgfpathlineto{\pgfqpoint{1.758351in}{2.454608in}}%
\pgfpathlineto{\pgfqpoint{1.764137in}{2.446368in}}%
\pgfpathlineto{\pgfqpoint{1.769926in}{2.438108in}}%
\pgfpathlineto{\pgfqpoint{1.775720in}{2.429824in}}%
\pgfpathclose%
\pgfusepath{stroke,fill}%
\end{pgfscope}%
\begin{pgfscope}%
\pgfpathrectangle{\pgfqpoint{0.887500in}{0.275000in}}{\pgfqpoint{4.225000in}{4.225000in}}%
\pgfusepath{clip}%
\pgfsetbuttcap%
\pgfsetroundjoin%
\definecolor{currentfill}{rgb}{0.175841,0.441290,0.557685}%
\pgfsetfillcolor{currentfill}%
\pgfsetfillopacity{0.700000}%
\pgfsetlinewidth{0.501875pt}%
\definecolor{currentstroke}{rgb}{1.000000,1.000000,1.000000}%
\pgfsetstrokecolor{currentstroke}%
\pgfsetstrokeopacity{0.500000}%
\pgfsetdash{}{0pt}%
\pgfpathmoveto{\pgfqpoint{4.572245in}{2.173746in}}%
\pgfpathlineto{\pgfqpoint{4.583218in}{2.177140in}}%
\pgfpathlineto{\pgfqpoint{4.594186in}{2.180518in}}%
\pgfpathlineto{\pgfqpoint{4.605147in}{2.183886in}}%
\pgfpathlineto{\pgfqpoint{4.616103in}{2.187246in}}%
\pgfpathlineto{\pgfqpoint{4.609607in}{2.201806in}}%
\pgfpathlineto{\pgfqpoint{4.603114in}{2.216338in}}%
\pgfpathlineto{\pgfqpoint{4.596621in}{2.230834in}}%
\pgfpathlineto{\pgfqpoint{4.590129in}{2.245285in}}%
\pgfpathlineto{\pgfqpoint{4.583638in}{2.259682in}}%
\pgfpathlineto{\pgfqpoint{4.572681in}{2.256261in}}%
\pgfpathlineto{\pgfqpoint{4.561718in}{2.252826in}}%
\pgfpathlineto{\pgfqpoint{4.550749in}{2.249370in}}%
\pgfpathlineto{\pgfqpoint{4.539774in}{2.245885in}}%
\pgfpathlineto{\pgfqpoint{4.546267in}{2.231558in}}%
\pgfpathlineto{\pgfqpoint{4.552760in}{2.217173in}}%
\pgfpathlineto{\pgfqpoint{4.559254in}{2.202736in}}%
\pgfpathlineto{\pgfqpoint{4.565749in}{2.188258in}}%
\pgfpathclose%
\pgfusepath{stroke,fill}%
\end{pgfscope}%
\begin{pgfscope}%
\pgfpathrectangle{\pgfqpoint{0.887500in}{0.275000in}}{\pgfqpoint{4.225000in}{4.225000in}}%
\pgfusepath{clip}%
\pgfsetbuttcap%
\pgfsetroundjoin%
\definecolor{currentfill}{rgb}{0.166617,0.463708,0.558119}%
\pgfsetfillcolor{currentfill}%
\pgfsetfillopacity{0.700000}%
\pgfsetlinewidth{0.501875pt}%
\definecolor{currentstroke}{rgb}{1.000000,1.000000,1.000000}%
\pgfsetstrokecolor{currentstroke}%
\pgfsetstrokeopacity{0.500000}%
\pgfsetdash{}{0pt}%
\pgfpathmoveto{\pgfqpoint{2.743163in}{2.235199in}}%
\pgfpathlineto{\pgfqpoint{2.754587in}{2.239346in}}%
\pgfpathlineto{\pgfqpoint{2.766004in}{2.243791in}}%
\pgfpathlineto{\pgfqpoint{2.777416in}{2.248251in}}%
\pgfpathlineto{\pgfqpoint{2.788824in}{2.252439in}}%
\pgfpathlineto{\pgfqpoint{2.800231in}{2.256064in}}%
\pgfpathlineto{\pgfqpoint{2.794094in}{2.265043in}}%
\pgfpathlineto{\pgfqpoint{2.787961in}{2.273994in}}%
\pgfpathlineto{\pgfqpoint{2.781832in}{2.282920in}}%
\pgfpathlineto{\pgfqpoint{2.775707in}{2.291823in}}%
\pgfpathlineto{\pgfqpoint{2.769586in}{2.300702in}}%
\pgfpathlineto{\pgfqpoint{2.758190in}{2.297140in}}%
\pgfpathlineto{\pgfqpoint{2.746793in}{2.292973in}}%
\pgfpathlineto{\pgfqpoint{2.735393in}{2.288507in}}%
\pgfpathlineto{\pgfqpoint{2.723988in}{2.284045in}}%
\pgfpathlineto{\pgfqpoint{2.712575in}{2.279885in}}%
\pgfpathlineto{\pgfqpoint{2.718685in}{2.270999in}}%
\pgfpathlineto{\pgfqpoint{2.724798in}{2.262087in}}%
\pgfpathlineto{\pgfqpoint{2.730916in}{2.253150in}}%
\pgfpathlineto{\pgfqpoint{2.737038in}{2.244188in}}%
\pgfpathclose%
\pgfusepath{stroke,fill}%
\end{pgfscope}%
\begin{pgfscope}%
\pgfpathrectangle{\pgfqpoint{0.887500in}{0.275000in}}{\pgfqpoint{4.225000in}{4.225000in}}%
\pgfusepath{clip}%
\pgfsetbuttcap%
\pgfsetroundjoin%
\definecolor{currentfill}{rgb}{0.157851,0.683765,0.501686}%
\pgfsetfillcolor{currentfill}%
\pgfsetfillopacity{0.700000}%
\pgfsetlinewidth{0.501875pt}%
\definecolor{currentstroke}{rgb}{1.000000,1.000000,1.000000}%
\pgfsetstrokecolor{currentstroke}%
\pgfsetstrokeopacity{0.500000}%
\pgfsetdash{}{0pt}%
\pgfpathmoveto{\pgfqpoint{3.894968in}{2.691654in}}%
\pgfpathlineto{\pgfqpoint{3.906115in}{2.695126in}}%
\pgfpathlineto{\pgfqpoint{3.917256in}{2.698600in}}%
\pgfpathlineto{\pgfqpoint{3.928392in}{2.702070in}}%
\pgfpathlineto{\pgfqpoint{3.939522in}{2.705531in}}%
\pgfpathlineto{\pgfqpoint{3.950647in}{2.708979in}}%
\pgfpathlineto{\pgfqpoint{3.944258in}{2.722871in}}%
\pgfpathlineto{\pgfqpoint{3.937870in}{2.736724in}}%
\pgfpathlineto{\pgfqpoint{3.931484in}{2.750531in}}%
\pgfpathlineto{\pgfqpoint{3.925100in}{2.764289in}}%
\pgfpathlineto{\pgfqpoint{3.918717in}{2.778001in}}%
\pgfpathlineto{\pgfqpoint{3.907594in}{2.774533in}}%
\pgfpathlineto{\pgfqpoint{3.896466in}{2.771054in}}%
\pgfpathlineto{\pgfqpoint{3.885333in}{2.767568in}}%
\pgfpathlineto{\pgfqpoint{3.874194in}{2.764082in}}%
\pgfpathlineto{\pgfqpoint{3.863050in}{2.760601in}}%
\pgfpathlineto{\pgfqpoint{3.869430in}{2.746878in}}%
\pgfpathlineto{\pgfqpoint{3.875812in}{2.733122in}}%
\pgfpathlineto{\pgfqpoint{3.882195in}{2.719331in}}%
\pgfpathlineto{\pgfqpoint{3.888581in}{2.705507in}}%
\pgfpathclose%
\pgfusepath{stroke,fill}%
\end{pgfscope}%
\begin{pgfscope}%
\pgfpathrectangle{\pgfqpoint{0.887500in}{0.275000in}}{\pgfqpoint{4.225000in}{4.225000in}}%
\pgfusepath{clip}%
\pgfsetbuttcap%
\pgfsetroundjoin%
\definecolor{currentfill}{rgb}{0.143343,0.522773,0.556295}%
\pgfsetfillcolor{currentfill}%
\pgfsetfillopacity{0.700000}%
\pgfsetlinewidth{0.501875pt}%
\definecolor{currentstroke}{rgb}{1.000000,1.000000,1.000000}%
\pgfsetstrokecolor{currentstroke}%
\pgfsetstrokeopacity{0.500000}%
\pgfsetdash{}{0pt}%
\pgfpathmoveto{\pgfqpoint{2.102016in}{2.361996in}}%
\pgfpathlineto{\pgfqpoint{2.113605in}{2.365381in}}%
\pgfpathlineto{\pgfqpoint{2.125188in}{2.368772in}}%
\pgfpathlineto{\pgfqpoint{2.136765in}{2.372167in}}%
\pgfpathlineto{\pgfqpoint{2.148336in}{2.375562in}}%
\pgfpathlineto{\pgfqpoint{2.159902in}{2.378954in}}%
\pgfpathlineto{\pgfqpoint{2.153980in}{2.387433in}}%
\pgfpathlineto{\pgfqpoint{2.148062in}{2.395895in}}%
\pgfpathlineto{\pgfqpoint{2.142148in}{2.404338in}}%
\pgfpathlineto{\pgfqpoint{2.136238in}{2.412763in}}%
\pgfpathlineto{\pgfqpoint{2.130333in}{2.421170in}}%
\pgfpathlineto{\pgfqpoint{2.118779in}{2.417804in}}%
\pgfpathlineto{\pgfqpoint{2.107219in}{2.414434in}}%
\pgfpathlineto{\pgfqpoint{2.095654in}{2.411065in}}%
\pgfpathlineto{\pgfqpoint{2.084082in}{2.407699in}}%
\pgfpathlineto{\pgfqpoint{2.072505in}{2.404340in}}%
\pgfpathlineto{\pgfqpoint{2.078399in}{2.395910in}}%
\pgfpathlineto{\pgfqpoint{2.084297in}{2.387461in}}%
\pgfpathlineto{\pgfqpoint{2.090199in}{2.378992in}}%
\pgfpathlineto{\pgfqpoint{2.096105in}{2.370504in}}%
\pgfpathclose%
\pgfusepath{stroke,fill}%
\end{pgfscope}%
\begin{pgfscope}%
\pgfpathrectangle{\pgfqpoint{0.887500in}{0.275000in}}{\pgfqpoint{4.225000in}{4.225000in}}%
\pgfusepath{clip}%
\pgfsetbuttcap%
\pgfsetroundjoin%
\definecolor{currentfill}{rgb}{0.154815,0.493313,0.557840}%
\pgfsetfillcolor{currentfill}%
\pgfsetfillopacity{0.700000}%
\pgfsetlinewidth{0.501875pt}%
\definecolor{currentstroke}{rgb}{1.000000,1.000000,1.000000}%
\pgfsetstrokecolor{currentstroke}%
\pgfsetstrokeopacity{0.500000}%
\pgfsetdash{}{0pt}%
\pgfpathmoveto{\pgfqpoint{2.422579in}{2.300291in}}%
\pgfpathlineto{\pgfqpoint{2.434089in}{2.303640in}}%
\pgfpathlineto{\pgfqpoint{2.445594in}{2.307007in}}%
\pgfpathlineto{\pgfqpoint{2.457092in}{2.310397in}}%
\pgfpathlineto{\pgfqpoint{2.468585in}{2.313815in}}%
\pgfpathlineto{\pgfqpoint{2.480071in}{2.317268in}}%
\pgfpathlineto{\pgfqpoint{2.474040in}{2.325933in}}%
\pgfpathlineto{\pgfqpoint{2.468014in}{2.334580in}}%
\pgfpathlineto{\pgfqpoint{2.461992in}{2.343210in}}%
\pgfpathlineto{\pgfqpoint{2.455973in}{2.351822in}}%
\pgfpathlineto{\pgfqpoint{2.449959in}{2.360418in}}%
\pgfpathlineto{\pgfqpoint{2.438485in}{2.356981in}}%
\pgfpathlineto{\pgfqpoint{2.427004in}{2.353580in}}%
\pgfpathlineto{\pgfqpoint{2.415516in}{2.350209in}}%
\pgfpathlineto{\pgfqpoint{2.404023in}{2.346862in}}%
\pgfpathlineto{\pgfqpoint{2.392524in}{2.343535in}}%
\pgfpathlineto{\pgfqpoint{2.398527in}{2.334924in}}%
\pgfpathlineto{\pgfqpoint{2.404533in}{2.326295in}}%
\pgfpathlineto{\pgfqpoint{2.410544in}{2.317646in}}%
\pgfpathlineto{\pgfqpoint{2.416559in}{2.308979in}}%
\pgfpathclose%
\pgfusepath{stroke,fill}%
\end{pgfscope}%
\begin{pgfscope}%
\pgfpathrectangle{\pgfqpoint{0.887500in}{0.275000in}}{\pgfqpoint{4.225000in}{4.225000in}}%
\pgfusepath{clip}%
\pgfsetbuttcap%
\pgfsetroundjoin%
\definecolor{currentfill}{rgb}{0.124395,0.578002,0.548287}%
\pgfsetfillcolor{currentfill}%
\pgfsetfillopacity{0.700000}%
\pgfsetlinewidth{0.501875pt}%
\definecolor{currentstroke}{rgb}{1.000000,1.000000,1.000000}%
\pgfsetstrokecolor{currentstroke}%
\pgfsetstrokeopacity{0.500000}%
\pgfsetdash{}{0pt}%
\pgfpathmoveto{\pgfqpoint{4.189900in}{2.461157in}}%
\pgfpathlineto{\pgfqpoint{4.200972in}{2.464523in}}%
\pgfpathlineto{\pgfqpoint{4.212038in}{2.467882in}}%
\pgfpathlineto{\pgfqpoint{4.223098in}{2.471232in}}%
\pgfpathlineto{\pgfqpoint{4.234153in}{2.474570in}}%
\pgfpathlineto{\pgfqpoint{4.245202in}{2.477897in}}%
\pgfpathlineto{\pgfqpoint{4.238764in}{2.492107in}}%
\pgfpathlineto{\pgfqpoint{4.232327in}{2.506301in}}%
\pgfpathlineto{\pgfqpoint{4.225893in}{2.520488in}}%
\pgfpathlineto{\pgfqpoint{4.219462in}{2.534673in}}%
\pgfpathlineto{\pgfqpoint{4.213033in}{2.548853in}}%
\pgfpathlineto{\pgfqpoint{4.201984in}{2.545448in}}%
\pgfpathlineto{\pgfqpoint{4.190929in}{2.542053in}}%
\pgfpathlineto{\pgfqpoint{4.179870in}{2.538670in}}%
\pgfpathlineto{\pgfqpoint{4.168805in}{2.535297in}}%
\pgfpathlineto{\pgfqpoint{4.157735in}{2.531934in}}%
\pgfpathlineto{\pgfqpoint{4.164164in}{2.517801in}}%
\pgfpathlineto{\pgfqpoint{4.170594in}{2.503659in}}%
\pgfpathlineto{\pgfqpoint{4.177028in}{2.489508in}}%
\pgfpathlineto{\pgfqpoint{4.183463in}{2.475344in}}%
\pgfpathclose%
\pgfusepath{stroke,fill}%
\end{pgfscope}%
\begin{pgfscope}%
\pgfpathrectangle{\pgfqpoint{0.887500in}{0.275000in}}{\pgfqpoint{4.225000in}{4.225000in}}%
\pgfusepath{clip}%
\pgfsetbuttcap%
\pgfsetroundjoin%
\definecolor{currentfill}{rgb}{0.165117,0.467423,0.558141}%
\pgfsetfillcolor{currentfill}%
\pgfsetfillopacity{0.700000}%
\pgfsetlinewidth{0.501875pt}%
\definecolor{currentstroke}{rgb}{1.000000,1.000000,1.000000}%
\pgfsetstrokecolor{currentstroke}%
\pgfsetstrokeopacity{0.500000}%
\pgfsetdash{}{0pt}%
\pgfpathmoveto{\pgfqpoint{4.484791in}{2.227744in}}%
\pgfpathlineto{\pgfqpoint{4.495803in}{2.231494in}}%
\pgfpathlineto{\pgfqpoint{4.506806in}{2.235175in}}%
\pgfpathlineto{\pgfqpoint{4.517803in}{2.238796in}}%
\pgfpathlineto{\pgfqpoint{4.528792in}{2.242363in}}%
\pgfpathlineto{\pgfqpoint{4.539774in}{2.245885in}}%
\pgfpathlineto{\pgfqpoint{4.533281in}{2.260143in}}%
\pgfpathlineto{\pgfqpoint{4.526787in}{2.274324in}}%
\pgfpathlineto{\pgfqpoint{4.520294in}{2.288432in}}%
\pgfpathlineto{\pgfqpoint{4.513801in}{2.302479in}}%
\pgfpathlineto{\pgfqpoint{4.507309in}{2.316478in}}%
\pgfpathlineto{\pgfqpoint{4.496326in}{2.312891in}}%
\pgfpathlineto{\pgfqpoint{4.485337in}{2.309276in}}%
\pgfpathlineto{\pgfqpoint{4.474341in}{2.305626in}}%
\pgfpathlineto{\pgfqpoint{4.463339in}{2.301935in}}%
\pgfpathlineto{\pgfqpoint{4.452330in}{2.298196in}}%
\pgfpathlineto{\pgfqpoint{4.458819in}{2.284163in}}%
\pgfpathlineto{\pgfqpoint{4.465310in}{2.270110in}}%
\pgfpathlineto{\pgfqpoint{4.471803in}{2.256028in}}%
\pgfpathlineto{\pgfqpoint{4.478296in}{2.241907in}}%
\pgfpathclose%
\pgfusepath{stroke,fill}%
\end{pgfscope}%
\begin{pgfscope}%
\pgfpathrectangle{\pgfqpoint{0.887500in}{0.275000in}}{\pgfqpoint{4.225000in}{4.225000in}}%
\pgfusepath{clip}%
\pgfsetbuttcap%
\pgfsetroundjoin%
\definecolor{currentfill}{rgb}{0.191090,0.708366,0.482284}%
\pgfsetfillcolor{currentfill}%
\pgfsetfillopacity{0.700000}%
\pgfsetlinewidth{0.501875pt}%
\definecolor{currentstroke}{rgb}{1.000000,1.000000,1.000000}%
\pgfsetstrokecolor{currentstroke}%
\pgfsetstrokeopacity{0.500000}%
\pgfsetdash{}{0pt}%
\pgfpathmoveto{\pgfqpoint{3.807254in}{2.743426in}}%
\pgfpathlineto{\pgfqpoint{3.818423in}{2.746824in}}%
\pgfpathlineto{\pgfqpoint{3.829588in}{2.750239in}}%
\pgfpathlineto{\pgfqpoint{3.840747in}{2.753674in}}%
\pgfpathlineto{\pgfqpoint{3.851901in}{2.757130in}}%
\pgfpathlineto{\pgfqpoint{3.863050in}{2.760601in}}%
\pgfpathlineto{\pgfqpoint{3.856672in}{2.774295in}}%
\pgfpathlineto{\pgfqpoint{3.850297in}{2.787965in}}%
\pgfpathlineto{\pgfqpoint{3.843923in}{2.801614in}}%
\pgfpathlineto{\pgfqpoint{3.837552in}{2.815248in}}%
\pgfpathlineto{\pgfqpoint{3.831183in}{2.828869in}}%
\pgfpathlineto{\pgfqpoint{3.820038in}{2.825421in}}%
\pgfpathlineto{\pgfqpoint{3.808887in}{2.821985in}}%
\pgfpathlineto{\pgfqpoint{3.797731in}{2.818566in}}%
\pgfpathlineto{\pgfqpoint{3.786570in}{2.815162in}}%
\pgfpathlineto{\pgfqpoint{3.775403in}{2.811772in}}%
\pgfpathlineto{\pgfqpoint{3.781769in}{2.798154in}}%
\pgfpathlineto{\pgfqpoint{3.788137in}{2.784510in}}%
\pgfpathlineto{\pgfqpoint{3.794507in}{2.770841in}}%
\pgfpathlineto{\pgfqpoint{3.800880in}{2.757146in}}%
\pgfpathclose%
\pgfusepath{stroke,fill}%
\end{pgfscope}%
\begin{pgfscope}%
\pgfpathrectangle{\pgfqpoint{0.887500in}{0.275000in}}{\pgfqpoint{4.225000in}{4.225000in}}%
\pgfusepath{clip}%
\pgfsetbuttcap%
\pgfsetroundjoin%
\definecolor{currentfill}{rgb}{0.535621,0.835785,0.281908}%
\pgfsetfillcolor{currentfill}%
\pgfsetfillopacity{0.700000}%
\pgfsetlinewidth{0.501875pt}%
\definecolor{currentstroke}{rgb}{1.000000,1.000000,1.000000}%
\pgfsetstrokecolor{currentstroke}%
\pgfsetstrokeopacity{0.500000}%
\pgfsetdash{}{0pt}%
\pgfpathmoveto{\pgfqpoint{3.103942in}{3.078270in}}%
\pgfpathlineto{\pgfqpoint{3.115290in}{3.086934in}}%
\pgfpathlineto{\pgfqpoint{3.126634in}{3.095341in}}%
\pgfpathlineto{\pgfqpoint{3.137976in}{3.103724in}}%
\pgfpathlineto{\pgfqpoint{3.149315in}{3.112006in}}%
\pgfpathlineto{\pgfqpoint{3.160650in}{3.120088in}}%
\pgfpathlineto{\pgfqpoint{3.154402in}{3.131575in}}%
\pgfpathlineto{\pgfqpoint{3.148157in}{3.143079in}}%
\pgfpathlineto{\pgfqpoint{3.141915in}{3.154576in}}%
\pgfpathlineto{\pgfqpoint{3.135677in}{3.166043in}}%
\pgfpathlineto{\pgfqpoint{3.129441in}{3.177455in}}%
\pgfpathlineto{\pgfqpoint{3.118114in}{3.167842in}}%
\pgfpathlineto{\pgfqpoint{3.106783in}{3.157456in}}%
\pgfpathlineto{\pgfqpoint{3.095450in}{3.146328in}}%
\pgfpathlineto{\pgfqpoint{3.084116in}{3.134475in}}%
\pgfpathlineto{\pgfqpoint{3.072781in}{3.121767in}}%
\pgfpathlineto{\pgfqpoint{3.079005in}{3.113775in}}%
\pgfpathlineto{\pgfqpoint{3.085234in}{3.105454in}}%
\pgfpathlineto{\pgfqpoint{3.091467in}{3.096780in}}%
\pgfpathlineto{\pgfqpoint{3.097703in}{3.087727in}}%
\pgfpathclose%
\pgfusepath{stroke,fill}%
\end{pgfscope}%
\begin{pgfscope}%
\pgfpathrectangle{\pgfqpoint{0.887500in}{0.275000in}}{\pgfqpoint{4.225000in}{4.225000in}}%
\pgfusepath{clip}%
\pgfsetbuttcap%
\pgfsetroundjoin%
\definecolor{currentfill}{rgb}{0.172719,0.448791,0.557885}%
\pgfsetfillcolor{currentfill}%
\pgfsetfillopacity{0.700000}%
\pgfsetlinewidth{0.501875pt}%
\definecolor{currentstroke}{rgb}{1.000000,1.000000,1.000000}%
\pgfsetstrokecolor{currentstroke}%
\pgfsetstrokeopacity{0.500000}%
\pgfsetdash{}{0pt}%
\pgfpathmoveto{\pgfqpoint{2.830977in}{2.210719in}}%
\pgfpathlineto{\pgfqpoint{2.842394in}{2.213402in}}%
\pgfpathlineto{\pgfqpoint{2.853812in}{2.214891in}}%
\pgfpathlineto{\pgfqpoint{2.865229in}{2.214953in}}%
\pgfpathlineto{\pgfqpoint{2.876643in}{2.214148in}}%
\pgfpathlineto{\pgfqpoint{2.888048in}{2.213564in}}%
\pgfpathlineto{\pgfqpoint{2.881882in}{2.222861in}}%
\pgfpathlineto{\pgfqpoint{2.875719in}{2.232174in}}%
\pgfpathlineto{\pgfqpoint{2.869561in}{2.241467in}}%
\pgfpathlineto{\pgfqpoint{2.863406in}{2.250710in}}%
\pgfpathlineto{\pgfqpoint{2.857255in}{2.259868in}}%
\pgfpathlineto{\pgfqpoint{2.845859in}{2.260194in}}%
\pgfpathlineto{\pgfqpoint{2.834453in}{2.260745in}}%
\pgfpathlineto{\pgfqpoint{2.823045in}{2.260474in}}%
\pgfpathlineto{\pgfqpoint{2.811637in}{2.258839in}}%
\pgfpathlineto{\pgfqpoint{2.800231in}{2.256064in}}%
\pgfpathlineto{\pgfqpoint{2.806372in}{2.247058in}}%
\pgfpathlineto{\pgfqpoint{2.812517in}{2.238021in}}%
\pgfpathlineto{\pgfqpoint{2.818666in}{2.228954in}}%
\pgfpathlineto{\pgfqpoint{2.824820in}{2.219853in}}%
\pgfpathclose%
\pgfusepath{stroke,fill}%
\end{pgfscope}%
\begin{pgfscope}%
\pgfpathrectangle{\pgfqpoint{0.887500in}{0.275000in}}{\pgfqpoint{4.225000in}{4.225000in}}%
\pgfusepath{clip}%
\pgfsetbuttcap%
\pgfsetroundjoin%
\definecolor{currentfill}{rgb}{0.127568,0.566949,0.550556}%
\pgfsetfillcolor{currentfill}%
\pgfsetfillopacity{0.700000}%
\pgfsetlinewidth{0.501875pt}%
\definecolor{currentstroke}{rgb}{1.000000,1.000000,1.000000}%
\pgfsetstrokecolor{currentstroke}%
\pgfsetstrokeopacity{0.500000}%
\pgfsetdash{}{0pt}%
\pgfpathmoveto{\pgfqpoint{1.548403in}{2.453555in}}%
\pgfpathlineto{\pgfqpoint{1.560130in}{2.456949in}}%
\pgfpathlineto{\pgfqpoint{1.571852in}{2.460330in}}%
\pgfpathlineto{\pgfqpoint{1.583569in}{2.463699in}}%
\pgfpathlineto{\pgfqpoint{1.595280in}{2.467058in}}%
\pgfpathlineto{\pgfqpoint{1.606986in}{2.470407in}}%
\pgfpathlineto{\pgfqpoint{1.601256in}{2.478550in}}%
\pgfpathlineto{\pgfqpoint{1.595531in}{2.486678in}}%
\pgfpathlineto{\pgfqpoint{1.589810in}{2.494792in}}%
\pgfpathlineto{\pgfqpoint{1.584093in}{2.502890in}}%
\pgfpathlineto{\pgfqpoint{1.578380in}{2.510973in}}%
\pgfpathlineto{\pgfqpoint{1.566687in}{2.507638in}}%
\pgfpathlineto{\pgfqpoint{1.554989in}{2.504294in}}%
\pgfpathlineto{\pgfqpoint{1.543285in}{2.500939in}}%
\pgfpathlineto{\pgfqpoint{1.531576in}{2.497572in}}%
\pgfpathlineto{\pgfqpoint{1.519862in}{2.494193in}}%
\pgfpathlineto{\pgfqpoint{1.525561in}{2.486096in}}%
\pgfpathlineto{\pgfqpoint{1.531265in}{2.477983in}}%
\pgfpathlineto{\pgfqpoint{1.536974in}{2.469855in}}%
\pgfpathlineto{\pgfqpoint{1.542686in}{2.461713in}}%
\pgfpathclose%
\pgfusepath{stroke,fill}%
\end{pgfscope}%
\begin{pgfscope}%
\pgfpathrectangle{\pgfqpoint{0.887500in}{0.275000in}}{\pgfqpoint{4.225000in}{4.225000in}}%
\pgfusepath{clip}%
\pgfsetbuttcap%
\pgfsetroundjoin%
\definecolor{currentfill}{rgb}{0.468053,0.818921,0.323998}%
\pgfsetfillcolor{currentfill}%
\pgfsetfillopacity{0.700000}%
\pgfsetlinewidth{0.501875pt}%
\definecolor{currentstroke}{rgb}{1.000000,1.000000,1.000000}%
\pgfsetstrokecolor{currentstroke}%
\pgfsetstrokeopacity{0.500000}%
\pgfsetdash{}{0pt}%
\pgfpathmoveto{\pgfqpoint{3.047173in}{3.015875in}}%
\pgfpathlineto{\pgfqpoint{3.058528in}{3.032353in}}%
\pgfpathlineto{\pgfqpoint{3.069884in}{3.046395in}}%
\pgfpathlineto{\pgfqpoint{3.081239in}{3.058438in}}%
\pgfpathlineto{\pgfqpoint{3.092592in}{3.068918in}}%
\pgfpathlineto{\pgfqpoint{3.103942in}{3.078270in}}%
\pgfpathlineto{\pgfqpoint{3.097703in}{3.087727in}}%
\pgfpathlineto{\pgfqpoint{3.091467in}{3.096780in}}%
\pgfpathlineto{\pgfqpoint{3.085234in}{3.105454in}}%
\pgfpathlineto{\pgfqpoint{3.079005in}{3.113775in}}%
\pgfpathlineto{\pgfqpoint{3.072781in}{3.121767in}}%
\pgfpathlineto{\pgfqpoint{3.061445in}{3.107973in}}%
\pgfpathlineto{\pgfqpoint{3.050110in}{3.092861in}}%
\pgfpathlineto{\pgfqpoint{3.038777in}{3.076200in}}%
\pgfpathlineto{\pgfqpoint{3.027447in}{3.057760in}}%
\pgfpathlineto{\pgfqpoint{3.016122in}{3.037312in}}%
\pgfpathlineto{\pgfqpoint{3.022320in}{3.034900in}}%
\pgfpathlineto{\pgfqpoint{3.028524in}{3.031686in}}%
\pgfpathlineto{\pgfqpoint{3.034735in}{3.027534in}}%
\pgfpathlineto{\pgfqpoint{3.040951in}{3.022309in}}%
\pgfpathclose%
\pgfusepath{stroke,fill}%
\end{pgfscope}%
\begin{pgfscope}%
\pgfpathrectangle{\pgfqpoint{0.887500in}{0.275000in}}{\pgfqpoint{4.225000in}{4.225000in}}%
\pgfusepath{clip}%
\pgfsetbuttcap%
\pgfsetroundjoin%
\definecolor{currentfill}{rgb}{0.136408,0.541173,0.554483}%
\pgfsetfillcolor{currentfill}%
\pgfsetfillopacity{0.700000}%
\pgfsetlinewidth{0.501875pt}%
\definecolor{currentstroke}{rgb}{1.000000,1.000000,1.000000}%
\pgfsetstrokecolor{currentstroke}%
\pgfsetstrokeopacity{0.500000}%
\pgfsetdash{}{0pt}%
\pgfpathmoveto{\pgfqpoint{1.868919in}{2.396519in}}%
\pgfpathlineto{\pgfqpoint{1.880570in}{2.399871in}}%
\pgfpathlineto{\pgfqpoint{1.892216in}{2.403213in}}%
\pgfpathlineto{\pgfqpoint{1.903857in}{2.406545in}}%
\pgfpathlineto{\pgfqpoint{1.915492in}{2.409868in}}%
\pgfpathlineto{\pgfqpoint{1.927121in}{2.413183in}}%
\pgfpathlineto{\pgfqpoint{1.921277in}{2.421542in}}%
\pgfpathlineto{\pgfqpoint{1.915437in}{2.429880in}}%
\pgfpathlineto{\pgfqpoint{1.909602in}{2.438199in}}%
\pgfpathlineto{\pgfqpoint{1.903770in}{2.446497in}}%
\pgfpathlineto{\pgfqpoint{1.897944in}{2.454774in}}%
\pgfpathlineto{\pgfqpoint{1.886326in}{2.451488in}}%
\pgfpathlineto{\pgfqpoint{1.874703in}{2.448194in}}%
\pgfpathlineto{\pgfqpoint{1.863074in}{2.444891in}}%
\pgfpathlineto{\pgfqpoint{1.851440in}{2.441579in}}%
\pgfpathlineto{\pgfqpoint{1.839800in}{2.438255in}}%
\pgfpathlineto{\pgfqpoint{1.845615in}{2.429950in}}%
\pgfpathlineto{\pgfqpoint{1.851435in}{2.421624in}}%
\pgfpathlineto{\pgfqpoint{1.857258in}{2.413277in}}%
\pgfpathlineto{\pgfqpoint{1.863086in}{2.404908in}}%
\pgfpathclose%
\pgfusepath{stroke,fill}%
\end{pgfscope}%
\begin{pgfscope}%
\pgfpathrectangle{\pgfqpoint{0.887500in}{0.275000in}}{\pgfqpoint{4.225000in}{4.225000in}}%
\pgfusepath{clip}%
\pgfsetbuttcap%
\pgfsetroundjoin%
\definecolor{currentfill}{rgb}{0.154815,0.493313,0.557840}%
\pgfsetfillcolor{currentfill}%
\pgfsetfillopacity{0.700000}%
\pgfsetlinewidth{0.501875pt}%
\definecolor{currentstroke}{rgb}{1.000000,1.000000,1.000000}%
\pgfsetstrokecolor{currentstroke}%
\pgfsetstrokeopacity{0.500000}%
\pgfsetdash{}{0pt}%
\pgfpathmoveto{\pgfqpoint{4.397175in}{2.278535in}}%
\pgfpathlineto{\pgfqpoint{4.408221in}{2.282618in}}%
\pgfpathlineto{\pgfqpoint{4.419260in}{2.286619in}}%
\pgfpathlineto{\pgfqpoint{4.430291in}{2.290545in}}%
\pgfpathlineto{\pgfqpoint{4.441314in}{2.294401in}}%
\pgfpathlineto{\pgfqpoint{4.452330in}{2.298196in}}%
\pgfpathlineto{\pgfqpoint{4.445844in}{2.312217in}}%
\pgfpathlineto{\pgfqpoint{4.439360in}{2.326237in}}%
\pgfpathlineto{\pgfqpoint{4.432878in}{2.340263in}}%
\pgfpathlineto{\pgfqpoint{4.426400in}{2.354306in}}%
\pgfpathlineto{\pgfqpoint{4.419926in}{2.368373in}}%
\pgfpathlineto{\pgfqpoint{4.408920in}{2.364814in}}%
\pgfpathlineto{\pgfqpoint{4.397909in}{2.361249in}}%
\pgfpathlineto{\pgfqpoint{4.386892in}{2.357678in}}%
\pgfpathlineto{\pgfqpoint{4.375869in}{2.354096in}}%
\pgfpathlineto{\pgfqpoint{4.364841in}{2.350502in}}%
\pgfpathlineto{\pgfqpoint{4.371301in}{2.336067in}}%
\pgfpathlineto{\pgfqpoint{4.377765in}{2.321645in}}%
\pgfpathlineto{\pgfqpoint{4.384231in}{2.307243in}}%
\pgfpathlineto{\pgfqpoint{4.390701in}{2.292871in}}%
\pgfpathclose%
\pgfusepath{stroke,fill}%
\end{pgfscope}%
\begin{pgfscope}%
\pgfpathrectangle{\pgfqpoint{0.887500in}{0.275000in}}{\pgfqpoint{4.225000in}{4.225000in}}%
\pgfusepath{clip}%
\pgfsetbuttcap%
\pgfsetroundjoin%
\definecolor{currentfill}{rgb}{0.119738,0.603785,0.541400}%
\pgfsetfillcolor{currentfill}%
\pgfsetfillopacity{0.700000}%
\pgfsetlinewidth{0.501875pt}%
\definecolor{currentstroke}{rgb}{1.000000,1.000000,1.000000}%
\pgfsetstrokecolor{currentstroke}%
\pgfsetstrokeopacity{0.500000}%
\pgfsetdash{}{0pt}%
\pgfpathmoveto{\pgfqpoint{4.102308in}{2.515230in}}%
\pgfpathlineto{\pgfqpoint{4.113404in}{2.518560in}}%
\pgfpathlineto{\pgfqpoint{4.124495in}{2.521894in}}%
\pgfpathlineto{\pgfqpoint{4.135580in}{2.525234in}}%
\pgfpathlineto{\pgfqpoint{4.146660in}{2.528580in}}%
\pgfpathlineto{\pgfqpoint{4.157735in}{2.531934in}}%
\pgfpathlineto{\pgfqpoint{4.151309in}{2.546059in}}%
\pgfpathlineto{\pgfqpoint{4.144885in}{2.560177in}}%
\pgfpathlineto{\pgfqpoint{4.138464in}{2.574288in}}%
\pgfpathlineto{\pgfqpoint{4.132045in}{2.588391in}}%
\pgfpathlineto{\pgfqpoint{4.125628in}{2.602489in}}%
\pgfpathlineto{\pgfqpoint{4.114554in}{2.599095in}}%
\pgfpathlineto{\pgfqpoint{4.103475in}{2.595713in}}%
\pgfpathlineto{\pgfqpoint{4.092391in}{2.592344in}}%
\pgfpathlineto{\pgfqpoint{4.081302in}{2.588987in}}%
\pgfpathlineto{\pgfqpoint{4.070207in}{2.585640in}}%
\pgfpathlineto{\pgfqpoint{4.076623in}{2.571579in}}%
\pgfpathlineto{\pgfqpoint{4.083041in}{2.557510in}}%
\pgfpathlineto{\pgfqpoint{4.089461in}{2.543431in}}%
\pgfpathlineto{\pgfqpoint{4.095884in}{2.529339in}}%
\pgfpathclose%
\pgfusepath{stroke,fill}%
\end{pgfscope}%
\begin{pgfscope}%
\pgfpathrectangle{\pgfqpoint{0.887500in}{0.275000in}}{\pgfqpoint{4.225000in}{4.225000in}}%
\pgfusepath{clip}%
\pgfsetbuttcap%
\pgfsetroundjoin%
\definecolor{currentfill}{rgb}{0.159194,0.482237,0.558073}%
\pgfsetfillcolor{currentfill}%
\pgfsetfillopacity{0.700000}%
\pgfsetlinewidth{0.501875pt}%
\definecolor{currentstroke}{rgb}{1.000000,1.000000,1.000000}%
\pgfsetstrokecolor{currentstroke}%
\pgfsetstrokeopacity{0.500000}%
\pgfsetdash{}{0pt}%
\pgfpathmoveto{\pgfqpoint{2.510286in}{2.273641in}}%
\pgfpathlineto{\pgfqpoint{2.521777in}{2.277147in}}%
\pgfpathlineto{\pgfqpoint{2.533262in}{2.280691in}}%
\pgfpathlineto{\pgfqpoint{2.544742in}{2.284267in}}%
\pgfpathlineto{\pgfqpoint{2.556215in}{2.287845in}}%
\pgfpathlineto{\pgfqpoint{2.567684in}{2.291393in}}%
\pgfpathlineto{\pgfqpoint{2.561621in}{2.300158in}}%
\pgfpathlineto{\pgfqpoint{2.555562in}{2.308902in}}%
\pgfpathlineto{\pgfqpoint{2.549508in}{2.317623in}}%
\pgfpathlineto{\pgfqpoint{2.543457in}{2.326322in}}%
\pgfpathlineto{\pgfqpoint{2.537411in}{2.335001in}}%
\pgfpathlineto{\pgfqpoint{2.525954in}{2.331451in}}%
\pgfpathlineto{\pgfqpoint{2.514492in}{2.327871in}}%
\pgfpathlineto{\pgfqpoint{2.503025in}{2.324297in}}%
\pgfpathlineto{\pgfqpoint{2.491551in}{2.320760in}}%
\pgfpathlineto{\pgfqpoint{2.480071in}{2.317268in}}%
\pgfpathlineto{\pgfqpoint{2.486106in}{2.308584in}}%
\pgfpathlineto{\pgfqpoint{2.492144in}{2.299880in}}%
\pgfpathlineto{\pgfqpoint{2.498187in}{2.291155in}}%
\pgfpathlineto{\pgfqpoint{2.504234in}{2.282409in}}%
\pgfpathclose%
\pgfusepath{stroke,fill}%
\end{pgfscope}%
\begin{pgfscope}%
\pgfpathrectangle{\pgfqpoint{0.887500in}{0.275000in}}{\pgfqpoint{4.225000in}{4.225000in}}%
\pgfusepath{clip}%
\pgfsetbuttcap%
\pgfsetroundjoin%
\definecolor{currentfill}{rgb}{0.147607,0.511733,0.557049}%
\pgfsetfillcolor{currentfill}%
\pgfsetfillopacity{0.700000}%
\pgfsetlinewidth{0.501875pt}%
\definecolor{currentstroke}{rgb}{1.000000,1.000000,1.000000}%
\pgfsetstrokecolor{currentstroke}%
\pgfsetstrokeopacity{0.500000}%
\pgfsetdash{}{0pt}%
\pgfpathmoveto{\pgfqpoint{2.189576in}{2.336281in}}%
\pgfpathlineto{\pgfqpoint{2.201148in}{2.339697in}}%
\pgfpathlineto{\pgfqpoint{2.212714in}{2.343102in}}%
\pgfpathlineto{\pgfqpoint{2.224274in}{2.346496in}}%
\pgfpathlineto{\pgfqpoint{2.235830in}{2.349878in}}%
\pgfpathlineto{\pgfqpoint{2.247379in}{2.353246in}}%
\pgfpathlineto{\pgfqpoint{2.241425in}{2.361786in}}%
\pgfpathlineto{\pgfqpoint{2.235474in}{2.370308in}}%
\pgfpathlineto{\pgfqpoint{2.229528in}{2.378811in}}%
\pgfpathlineto{\pgfqpoint{2.223586in}{2.387297in}}%
\pgfpathlineto{\pgfqpoint{2.217648in}{2.395766in}}%
\pgfpathlineto{\pgfqpoint{2.206110in}{2.392427in}}%
\pgfpathlineto{\pgfqpoint{2.194566in}{2.389075in}}%
\pgfpathlineto{\pgfqpoint{2.183017in}{2.385713in}}%
\pgfpathlineto{\pgfqpoint{2.171462in}{2.382338in}}%
\pgfpathlineto{\pgfqpoint{2.159902in}{2.378954in}}%
\pgfpathlineto{\pgfqpoint{2.165828in}{2.370456in}}%
\pgfpathlineto{\pgfqpoint{2.171759in}{2.361940in}}%
\pgfpathlineto{\pgfqpoint{2.177694in}{2.353405in}}%
\pgfpathlineto{\pgfqpoint{2.183633in}{2.344852in}}%
\pgfpathclose%
\pgfusepath{stroke,fill}%
\end{pgfscope}%
\begin{pgfscope}%
\pgfpathrectangle{\pgfqpoint{0.887500in}{0.275000in}}{\pgfqpoint{4.225000in}{4.225000in}}%
\pgfusepath{clip}%
\pgfsetbuttcap%
\pgfsetroundjoin%
\definecolor{currentfill}{rgb}{0.226397,0.728888,0.462789}%
\pgfsetfillcolor{currentfill}%
\pgfsetfillopacity{0.700000}%
\pgfsetlinewidth{0.501875pt}%
\definecolor{currentstroke}{rgb}{1.000000,1.000000,1.000000}%
\pgfsetstrokecolor{currentstroke}%
\pgfsetstrokeopacity{0.500000}%
\pgfsetdash{}{0pt}%
\pgfpathmoveto{\pgfqpoint{3.719493in}{2.794923in}}%
\pgfpathlineto{\pgfqpoint{3.730686in}{2.798288in}}%
\pgfpathlineto{\pgfqpoint{3.741873in}{2.801652in}}%
\pgfpathlineto{\pgfqpoint{3.753055in}{2.805019in}}%
\pgfpathlineto{\pgfqpoint{3.764232in}{2.808391in}}%
\pgfpathlineto{\pgfqpoint{3.775403in}{2.811772in}}%
\pgfpathlineto{\pgfqpoint{3.769040in}{2.825364in}}%
\pgfpathlineto{\pgfqpoint{3.762678in}{2.838930in}}%
\pgfpathlineto{\pgfqpoint{3.756319in}{2.852470in}}%
\pgfpathlineto{\pgfqpoint{3.749962in}{2.865983in}}%
\pgfpathlineto{\pgfqpoint{3.743607in}{2.879470in}}%
\pgfpathlineto{\pgfqpoint{3.732437in}{2.875997in}}%
\pgfpathlineto{\pgfqpoint{3.721262in}{2.872537in}}%
\pgfpathlineto{\pgfqpoint{3.710082in}{2.869090in}}%
\pgfpathlineto{\pgfqpoint{3.698897in}{2.865660in}}%
\pgfpathlineto{\pgfqpoint{3.687706in}{2.862247in}}%
\pgfpathlineto{\pgfqpoint{3.694060in}{2.848864in}}%
\pgfpathlineto{\pgfqpoint{3.700415in}{2.835441in}}%
\pgfpathlineto{\pgfqpoint{3.706772in}{2.821976in}}%
\pgfpathlineto{\pgfqpoint{3.713132in}{2.808469in}}%
\pgfpathclose%
\pgfusepath{stroke,fill}%
\end{pgfscope}%
\begin{pgfscope}%
\pgfpathrectangle{\pgfqpoint{0.887500in}{0.275000in}}{\pgfqpoint{4.225000in}{4.225000in}}%
\pgfusepath{clip}%
\pgfsetbuttcap%
\pgfsetroundjoin%
\definecolor{currentfill}{rgb}{0.177423,0.437527,0.557565}%
\pgfsetfillcolor{currentfill}%
\pgfsetfillopacity{0.700000}%
\pgfsetlinewidth{0.501875pt}%
\definecolor{currentstroke}{rgb}{1.000000,1.000000,1.000000}%
\pgfsetstrokecolor{currentstroke}%
\pgfsetstrokeopacity{0.500000}%
\pgfsetdash{}{0pt}%
\pgfpathmoveto{\pgfqpoint{2.918935in}{2.167826in}}%
\pgfpathlineto{\pgfqpoint{2.930335in}{2.168625in}}%
\pgfpathlineto{\pgfqpoint{2.941721in}{2.171727in}}%
\pgfpathlineto{\pgfqpoint{2.953093in}{2.178178in}}%
\pgfpathlineto{\pgfqpoint{2.964450in}{2.189028in}}%
\pgfpathlineto{\pgfqpoint{2.975795in}{2.205153in}}%
\pgfpathlineto{\pgfqpoint{2.969595in}{2.214544in}}%
\pgfpathlineto{\pgfqpoint{2.963400in}{2.223930in}}%
\pgfpathlineto{\pgfqpoint{2.957208in}{2.233314in}}%
\pgfpathlineto{\pgfqpoint{2.951020in}{2.242698in}}%
\pgfpathlineto{\pgfqpoint{2.944836in}{2.252082in}}%
\pgfpathlineto{\pgfqpoint{2.933514in}{2.235326in}}%
\pgfpathlineto{\pgfqpoint{2.922174in}{2.224082in}}%
\pgfpathlineto{\pgfqpoint{2.910816in}{2.217437in}}%
\pgfpathlineto{\pgfqpoint{2.899440in}{2.214295in}}%
\pgfpathlineto{\pgfqpoint{2.888048in}{2.213564in}}%
\pgfpathlineto{\pgfqpoint{2.894219in}{2.204313in}}%
\pgfpathlineto{\pgfqpoint{2.900392in}{2.195126in}}%
\pgfpathlineto{\pgfqpoint{2.906569in}{2.185992in}}%
\pgfpathlineto{\pgfqpoint{2.912751in}{2.176897in}}%
\pgfpathclose%
\pgfusepath{stroke,fill}%
\end{pgfscope}%
\begin{pgfscope}%
\pgfpathrectangle{\pgfqpoint{0.887500in}{0.275000in}}{\pgfqpoint{4.225000in}{4.225000in}}%
\pgfusepath{clip}%
\pgfsetbuttcap%
\pgfsetroundjoin%
\definecolor{currentfill}{rgb}{0.274149,0.751988,0.436601}%
\pgfsetfillcolor{currentfill}%
\pgfsetfillopacity{0.700000}%
\pgfsetlinewidth{0.501875pt}%
\definecolor{currentstroke}{rgb}{1.000000,1.000000,1.000000}%
\pgfsetstrokecolor{currentstroke}%
\pgfsetstrokeopacity{0.500000}%
\pgfsetdash{}{0pt}%
\pgfpathmoveto{\pgfqpoint{3.631673in}{2.845164in}}%
\pgfpathlineto{\pgfqpoint{3.642891in}{2.848611in}}%
\pgfpathlineto{\pgfqpoint{3.654103in}{2.852034in}}%
\pgfpathlineto{\pgfqpoint{3.665310in}{2.855442in}}%
\pgfpathlineto{\pgfqpoint{3.676511in}{2.858844in}}%
\pgfpathlineto{\pgfqpoint{3.687706in}{2.862247in}}%
\pgfpathlineto{\pgfqpoint{3.681355in}{2.875593in}}%
\pgfpathlineto{\pgfqpoint{3.675006in}{2.888909in}}%
\pgfpathlineto{\pgfqpoint{3.668660in}{2.902199in}}%
\pgfpathlineto{\pgfqpoint{3.662315in}{2.915471in}}%
\pgfpathlineto{\pgfqpoint{3.655974in}{2.928729in}}%
\pgfpathlineto{\pgfqpoint{3.644782in}{2.925332in}}%
\pgfpathlineto{\pgfqpoint{3.633585in}{2.921967in}}%
\pgfpathlineto{\pgfqpoint{3.622383in}{2.918619in}}%
\pgfpathlineto{\pgfqpoint{3.611175in}{2.915271in}}%
\pgfpathlineto{\pgfqpoint{3.599962in}{2.911910in}}%
\pgfpathlineto{\pgfqpoint{3.606299in}{2.898600in}}%
\pgfpathlineto{\pgfqpoint{3.612639in}{2.885271in}}%
\pgfpathlineto{\pgfqpoint{3.618982in}{2.871923in}}%
\pgfpathlineto{\pgfqpoint{3.625326in}{2.858554in}}%
\pgfpathclose%
\pgfusepath{stroke,fill}%
\end{pgfscope}%
\begin{pgfscope}%
\pgfpathrectangle{\pgfqpoint{0.887500in}{0.275000in}}{\pgfqpoint{4.225000in}{4.225000in}}%
\pgfusepath{clip}%
\pgfsetbuttcap%
\pgfsetroundjoin%
\definecolor{currentfill}{rgb}{0.121380,0.629492,0.531973}%
\pgfsetfillcolor{currentfill}%
\pgfsetfillopacity{0.700000}%
\pgfsetlinewidth{0.501875pt}%
\definecolor{currentstroke}{rgb}{1.000000,1.000000,1.000000}%
\pgfsetstrokecolor{currentstroke}%
\pgfsetstrokeopacity{0.500000}%
\pgfsetdash{}{0pt}%
\pgfpathmoveto{\pgfqpoint{4.014655in}{2.568937in}}%
\pgfpathlineto{\pgfqpoint{4.025777in}{2.572286in}}%
\pgfpathlineto{\pgfqpoint{4.036892in}{2.575628in}}%
\pgfpathlineto{\pgfqpoint{4.048003in}{2.578964in}}%
\pgfpathlineto{\pgfqpoint{4.059108in}{2.582301in}}%
\pgfpathlineto{\pgfqpoint{4.070207in}{2.585640in}}%
\pgfpathlineto{\pgfqpoint{4.063794in}{2.599699in}}%
\pgfpathlineto{\pgfqpoint{4.057383in}{2.613756in}}%
\pgfpathlineto{\pgfqpoint{4.050975in}{2.627817in}}%
\pgfpathlineto{\pgfqpoint{4.044570in}{2.641882in}}%
\pgfpathlineto{\pgfqpoint{4.038167in}{2.655945in}}%
\pgfpathlineto{\pgfqpoint{4.027069in}{2.652580in}}%
\pgfpathlineto{\pgfqpoint{4.015966in}{2.649216in}}%
\pgfpathlineto{\pgfqpoint{4.004857in}{2.645849in}}%
\pgfpathlineto{\pgfqpoint{3.993742in}{2.642477in}}%
\pgfpathlineto{\pgfqpoint{3.982622in}{2.639095in}}%
\pgfpathlineto{\pgfqpoint{3.989024in}{2.625070in}}%
\pgfpathlineto{\pgfqpoint{3.995428in}{2.611039in}}%
\pgfpathlineto{\pgfqpoint{4.001834in}{2.597008in}}%
\pgfpathlineto{\pgfqpoint{4.008244in}{2.582976in}}%
\pgfpathclose%
\pgfusepath{stroke,fill}%
\end{pgfscope}%
\begin{pgfscope}%
\pgfpathrectangle{\pgfqpoint{0.887500in}{0.275000in}}{\pgfqpoint{4.225000in}{4.225000in}}%
\pgfusepath{clip}%
\pgfsetbuttcap%
\pgfsetroundjoin%
\definecolor{currentfill}{rgb}{0.144759,0.519093,0.556572}%
\pgfsetfillcolor{currentfill}%
\pgfsetfillopacity{0.700000}%
\pgfsetlinewidth{0.501875pt}%
\definecolor{currentstroke}{rgb}{1.000000,1.000000,1.000000}%
\pgfsetstrokecolor{currentstroke}%
\pgfsetstrokeopacity{0.500000}%
\pgfsetdash{}{0pt}%
\pgfpathmoveto{\pgfqpoint{4.309625in}{2.332637in}}%
\pgfpathlineto{\pgfqpoint{4.320677in}{2.336153in}}%
\pgfpathlineto{\pgfqpoint{4.331725in}{2.339710in}}%
\pgfpathlineto{\pgfqpoint{4.342769in}{2.343296in}}%
\pgfpathlineto{\pgfqpoint{4.353807in}{2.346898in}}%
\pgfpathlineto{\pgfqpoint{4.364841in}{2.350502in}}%
\pgfpathlineto{\pgfqpoint{4.358384in}{2.364942in}}%
\pgfpathlineto{\pgfqpoint{4.351928in}{2.379378in}}%
\pgfpathlineto{\pgfqpoint{4.345475in}{2.393805in}}%
\pgfpathlineto{\pgfqpoint{4.339024in}{2.408220in}}%
\pgfpathlineto{\pgfqpoint{4.332576in}{2.422623in}}%
\pgfpathlineto{\pgfqpoint{4.321554in}{2.419338in}}%
\pgfpathlineto{\pgfqpoint{4.310526in}{2.416062in}}%
\pgfpathlineto{\pgfqpoint{4.299494in}{2.412791in}}%
\pgfpathlineto{\pgfqpoint{4.288456in}{2.409518in}}%
\pgfpathlineto{\pgfqpoint{4.277413in}{2.406236in}}%
\pgfpathlineto{\pgfqpoint{4.283856in}{2.391707in}}%
\pgfpathlineto{\pgfqpoint{4.290299in}{2.377088in}}%
\pgfpathlineto{\pgfqpoint{4.296742in}{2.362369in}}%
\pgfpathlineto{\pgfqpoint{4.303184in}{2.347544in}}%
\pgfpathclose%
\pgfusepath{stroke,fill}%
\end{pgfscope}%
\begin{pgfscope}%
\pgfpathrectangle{\pgfqpoint{0.887500in}{0.275000in}}{\pgfqpoint{4.225000in}{4.225000in}}%
\pgfusepath{clip}%
\pgfsetbuttcap%
\pgfsetroundjoin%
\definecolor{currentfill}{rgb}{0.129933,0.559582,0.551864}%
\pgfsetfillcolor{currentfill}%
\pgfsetfillopacity{0.700000}%
\pgfsetlinewidth{0.501875pt}%
\definecolor{currentstroke}{rgb}{1.000000,1.000000,1.000000}%
\pgfsetstrokecolor{currentstroke}%
\pgfsetstrokeopacity{0.500000}%
\pgfsetdash{}{0pt}%
\pgfpathmoveto{\pgfqpoint{1.635698in}{2.429429in}}%
\pgfpathlineto{\pgfqpoint{1.647411in}{2.432786in}}%
\pgfpathlineto{\pgfqpoint{1.659119in}{2.436136in}}%
\pgfpathlineto{\pgfqpoint{1.670820in}{2.439480in}}%
\pgfpathlineto{\pgfqpoint{1.682516in}{2.442819in}}%
\pgfpathlineto{\pgfqpoint{1.694206in}{2.446155in}}%
\pgfpathlineto{\pgfqpoint{1.688443in}{2.454372in}}%
\pgfpathlineto{\pgfqpoint{1.682683in}{2.462569in}}%
\pgfpathlineto{\pgfqpoint{1.676928in}{2.470747in}}%
\pgfpathlineto{\pgfqpoint{1.671177in}{2.478907in}}%
\pgfpathlineto{\pgfqpoint{1.665430in}{2.487051in}}%
\pgfpathlineto{\pgfqpoint{1.653753in}{2.483731in}}%
\pgfpathlineto{\pgfqpoint{1.642069in}{2.480409in}}%
\pgfpathlineto{\pgfqpoint{1.630380in}{2.477081in}}%
\pgfpathlineto{\pgfqpoint{1.618686in}{2.473747in}}%
\pgfpathlineto{\pgfqpoint{1.606986in}{2.470407in}}%
\pgfpathlineto{\pgfqpoint{1.612719in}{2.462247in}}%
\pgfpathlineto{\pgfqpoint{1.618458in}{2.454071in}}%
\pgfpathlineto{\pgfqpoint{1.624200in}{2.445877in}}%
\pgfpathlineto{\pgfqpoint{1.629947in}{2.437663in}}%
\pgfpathclose%
\pgfusepath{stroke,fill}%
\end{pgfscope}%
\begin{pgfscope}%
\pgfpathrectangle{\pgfqpoint{0.887500in}{0.275000in}}{\pgfqpoint{4.225000in}{4.225000in}}%
\pgfusepath{clip}%
\pgfsetbuttcap%
\pgfsetroundjoin%
\definecolor{currentfill}{rgb}{0.162016,0.687316,0.499129}%
\pgfsetfillcolor{currentfill}%
\pgfsetfillopacity{0.700000}%
\pgfsetlinewidth{0.501875pt}%
\definecolor{currentstroke}{rgb}{1.000000,1.000000,1.000000}%
\pgfsetstrokecolor{currentstroke}%
\pgfsetstrokeopacity{0.500000}%
\pgfsetdash{}{0pt}%
\pgfpathmoveto{\pgfqpoint{2.995970in}{2.626624in}}%
\pgfpathlineto{\pgfqpoint{3.007295in}{2.653725in}}%
\pgfpathlineto{\pgfqpoint{3.018628in}{2.681025in}}%
\pgfpathlineto{\pgfqpoint{3.029968in}{2.708891in}}%
\pgfpathlineto{\pgfqpoint{3.041317in}{2.737489in}}%
\pgfpathlineto{\pgfqpoint{3.052674in}{2.766339in}}%
\pgfpathlineto{\pgfqpoint{3.046443in}{2.782024in}}%
\pgfpathlineto{\pgfqpoint{3.040214in}{2.797839in}}%
\pgfpathlineto{\pgfqpoint{3.033986in}{2.813529in}}%
\pgfpathlineto{\pgfqpoint{3.027761in}{2.828841in}}%
\pgfpathlineto{\pgfqpoint{3.021538in}{2.843518in}}%
\pgfpathlineto{\pgfqpoint{3.010201in}{2.814800in}}%
\pgfpathlineto{\pgfqpoint{2.998874in}{2.786055in}}%
\pgfpathlineto{\pgfqpoint{2.987554in}{2.757742in}}%
\pgfpathlineto{\pgfqpoint{2.976243in}{2.729645in}}%
\pgfpathlineto{\pgfqpoint{2.964940in}{2.701346in}}%
\pgfpathlineto{\pgfqpoint{2.971142in}{2.686325in}}%
\pgfpathlineto{\pgfqpoint{2.977347in}{2.671005in}}%
\pgfpathlineto{\pgfqpoint{2.983553in}{2.655724in}}%
\pgfpathlineto{\pgfqpoint{2.989761in}{2.640818in}}%
\pgfpathclose%
\pgfusepath{stroke,fill}%
\end{pgfscope}%
\begin{pgfscope}%
\pgfpathrectangle{\pgfqpoint{0.887500in}{0.275000in}}{\pgfqpoint{4.225000in}{4.225000in}}%
\pgfusepath{clip}%
\pgfsetbuttcap%
\pgfsetroundjoin%
\definecolor{currentfill}{rgb}{0.327796,0.773980,0.406640}%
\pgfsetfillcolor{currentfill}%
\pgfsetfillopacity{0.700000}%
\pgfsetlinewidth{0.501875pt}%
\definecolor{currentstroke}{rgb}{1.000000,1.000000,1.000000}%
\pgfsetstrokecolor{currentstroke}%
\pgfsetstrokeopacity{0.500000}%
\pgfsetdash{}{0pt}%
\pgfpathmoveto{\pgfqpoint{3.543808in}{2.894545in}}%
\pgfpathlineto{\pgfqpoint{3.555051in}{2.898102in}}%
\pgfpathlineto{\pgfqpoint{3.566287in}{2.901616in}}%
\pgfpathlineto{\pgfqpoint{3.577518in}{2.905089in}}%
\pgfpathlineto{\pgfqpoint{3.588743in}{2.908520in}}%
\pgfpathlineto{\pgfqpoint{3.599962in}{2.911910in}}%
\pgfpathlineto{\pgfqpoint{3.593627in}{2.925202in}}%
\pgfpathlineto{\pgfqpoint{3.587295in}{2.938479in}}%
\pgfpathlineto{\pgfqpoint{3.580965in}{2.951740in}}%
\pgfpathlineto{\pgfqpoint{3.574637in}{2.964987in}}%
\pgfpathlineto{\pgfqpoint{3.568312in}{2.978219in}}%
\pgfpathlineto{\pgfqpoint{3.557097in}{2.974817in}}%
\pgfpathlineto{\pgfqpoint{3.545875in}{2.971371in}}%
\pgfpathlineto{\pgfqpoint{3.534648in}{2.967882in}}%
\pgfpathlineto{\pgfqpoint{3.523415in}{2.964351in}}%
\pgfpathlineto{\pgfqpoint{3.512176in}{2.960777in}}%
\pgfpathlineto{\pgfqpoint{3.518498in}{2.947625in}}%
\pgfpathlineto{\pgfqpoint{3.524822in}{2.934422in}}%
\pgfpathlineto{\pgfqpoint{3.531149in}{2.921172in}}%
\pgfpathlineto{\pgfqpoint{3.537477in}{2.907878in}}%
\pgfpathclose%
\pgfusepath{stroke,fill}%
\end{pgfscope}%
\begin{pgfscope}%
\pgfpathrectangle{\pgfqpoint{0.887500in}{0.275000in}}{\pgfqpoint{4.225000in}{4.225000in}}%
\pgfusepath{clip}%
\pgfsetbuttcap%
\pgfsetroundjoin%
\definecolor{currentfill}{rgb}{0.162142,0.474838,0.558140}%
\pgfsetfillcolor{currentfill}%
\pgfsetfillopacity{0.700000}%
\pgfsetlinewidth{0.501875pt}%
\definecolor{currentstroke}{rgb}{1.000000,1.000000,1.000000}%
\pgfsetstrokecolor{currentstroke}%
\pgfsetstrokeopacity{0.500000}%
\pgfsetdash{}{0pt}%
\pgfpathmoveto{\pgfqpoint{2.598060in}{2.247214in}}%
\pgfpathlineto{\pgfqpoint{2.609535in}{2.250703in}}%
\pgfpathlineto{\pgfqpoint{2.621005in}{2.254102in}}%
\pgfpathlineto{\pgfqpoint{2.632471in}{2.257385in}}%
\pgfpathlineto{\pgfqpoint{2.643932in}{2.260523in}}%
\pgfpathlineto{\pgfqpoint{2.655390in}{2.263536in}}%
\pgfpathlineto{\pgfqpoint{2.649295in}{2.272402in}}%
\pgfpathlineto{\pgfqpoint{2.643205in}{2.281242in}}%
\pgfpathlineto{\pgfqpoint{2.637119in}{2.290056in}}%
\pgfpathlineto{\pgfqpoint{2.631037in}{2.298847in}}%
\pgfpathlineto{\pgfqpoint{2.624959in}{2.307614in}}%
\pgfpathlineto{\pgfqpoint{2.613512in}{2.304641in}}%
\pgfpathlineto{\pgfqpoint{2.602061in}{2.301534in}}%
\pgfpathlineto{\pgfqpoint{2.590606in}{2.298270in}}%
\pgfpathlineto{\pgfqpoint{2.579147in}{2.294879in}}%
\pgfpathlineto{\pgfqpoint{2.567684in}{2.291393in}}%
\pgfpathlineto{\pgfqpoint{2.573751in}{2.282605in}}%
\pgfpathlineto{\pgfqpoint{2.579822in}{2.273793in}}%
\pgfpathlineto{\pgfqpoint{2.585897in}{2.264958in}}%
\pgfpathlineto{\pgfqpoint{2.591976in}{2.256098in}}%
\pgfpathclose%
\pgfusepath{stroke,fill}%
\end{pgfscope}%
\begin{pgfscope}%
\pgfpathrectangle{\pgfqpoint{0.887500in}{0.275000in}}{\pgfqpoint{4.225000in}{4.225000in}}%
\pgfusepath{clip}%
\pgfsetbuttcap%
\pgfsetroundjoin%
\definecolor{currentfill}{rgb}{0.140536,0.530132,0.555659}%
\pgfsetfillcolor{currentfill}%
\pgfsetfillopacity{0.700000}%
\pgfsetlinewidth{0.501875pt}%
\definecolor{currentstroke}{rgb}{1.000000,1.000000,1.000000}%
\pgfsetstrokecolor{currentstroke}%
\pgfsetstrokeopacity{0.500000}%
\pgfsetdash{}{0pt}%
\pgfpathmoveto{\pgfqpoint{1.956407in}{2.371084in}}%
\pgfpathlineto{\pgfqpoint{1.968043in}{2.374419in}}%
\pgfpathlineto{\pgfqpoint{1.979673in}{2.377747in}}%
\pgfpathlineto{\pgfqpoint{1.991297in}{2.381068in}}%
\pgfpathlineto{\pgfqpoint{2.002916in}{2.384385in}}%
\pgfpathlineto{\pgfqpoint{2.014529in}{2.387699in}}%
\pgfpathlineto{\pgfqpoint{2.008651in}{2.396133in}}%
\pgfpathlineto{\pgfqpoint{2.002778in}{2.404546in}}%
\pgfpathlineto{\pgfqpoint{1.996909in}{2.412940in}}%
\pgfpathlineto{\pgfqpoint{1.991045in}{2.421313in}}%
\pgfpathlineto{\pgfqpoint{1.985184in}{2.429667in}}%
\pgfpathlineto{\pgfqpoint{1.973583in}{2.426377in}}%
\pgfpathlineto{\pgfqpoint{1.961976in}{2.423086in}}%
\pgfpathlineto{\pgfqpoint{1.950364in}{2.419791in}}%
\pgfpathlineto{\pgfqpoint{1.938745in}{2.416490in}}%
\pgfpathlineto{\pgfqpoint{1.927121in}{2.413183in}}%
\pgfpathlineto{\pgfqpoint{1.932970in}{2.404803in}}%
\pgfpathlineto{\pgfqpoint{1.938823in}{2.396404in}}%
\pgfpathlineto{\pgfqpoint{1.944680in}{2.387984in}}%
\pgfpathlineto{\pgfqpoint{1.950542in}{2.379544in}}%
\pgfpathclose%
\pgfusepath{stroke,fill}%
\end{pgfscope}%
\begin{pgfscope}%
\pgfpathrectangle{\pgfqpoint{0.887500in}{0.275000in}}{\pgfqpoint{4.225000in}{4.225000in}}%
\pgfusepath{clip}%
\pgfsetbuttcap%
\pgfsetroundjoin%
\definecolor{currentfill}{rgb}{0.150476,0.504369,0.557430}%
\pgfsetfillcolor{currentfill}%
\pgfsetfillopacity{0.700000}%
\pgfsetlinewidth{0.501875pt}%
\definecolor{currentstroke}{rgb}{1.000000,1.000000,1.000000}%
\pgfsetstrokecolor{currentstroke}%
\pgfsetstrokeopacity{0.500000}%
\pgfsetdash{}{0pt}%
\pgfpathmoveto{\pgfqpoint{2.277216in}{2.310255in}}%
\pgfpathlineto{\pgfqpoint{2.288771in}{2.313639in}}%
\pgfpathlineto{\pgfqpoint{2.300322in}{2.317006in}}%
\pgfpathlineto{\pgfqpoint{2.311866in}{2.320355in}}%
\pgfpathlineto{\pgfqpoint{2.323406in}{2.323686in}}%
\pgfpathlineto{\pgfqpoint{2.334940in}{2.327003in}}%
\pgfpathlineto{\pgfqpoint{2.328953in}{2.335614in}}%
\pgfpathlineto{\pgfqpoint{2.322970in}{2.344206in}}%
\pgfpathlineto{\pgfqpoint{2.316991in}{2.352778in}}%
\pgfpathlineto{\pgfqpoint{2.311017in}{2.361332in}}%
\pgfpathlineto{\pgfqpoint{2.305046in}{2.369867in}}%
\pgfpathlineto{\pgfqpoint{2.293524in}{2.366571in}}%
\pgfpathlineto{\pgfqpoint{2.281996in}{2.363263in}}%
\pgfpathlineto{\pgfqpoint{2.270462in}{2.359939in}}%
\pgfpathlineto{\pgfqpoint{2.258924in}{2.356600in}}%
\pgfpathlineto{\pgfqpoint{2.247379in}{2.353246in}}%
\pgfpathlineto{\pgfqpoint{2.253338in}{2.344687in}}%
\pgfpathlineto{\pgfqpoint{2.259301in}{2.336109in}}%
\pgfpathlineto{\pgfqpoint{2.265269in}{2.327512in}}%
\pgfpathlineto{\pgfqpoint{2.271240in}{2.318894in}}%
\pgfpathclose%
\pgfusepath{stroke,fill}%
\end{pgfscope}%
\begin{pgfscope}%
\pgfpathrectangle{\pgfqpoint{0.887500in}{0.275000in}}{\pgfqpoint{4.225000in}{4.225000in}}%
\pgfusepath{clip}%
\pgfsetbuttcap%
\pgfsetroundjoin%
\definecolor{currentfill}{rgb}{0.525776,0.833491,0.288127}%
\pgfsetfillcolor{currentfill}%
\pgfsetfillopacity{0.700000}%
\pgfsetlinewidth{0.501875pt}%
\definecolor{currentstroke}{rgb}{1.000000,1.000000,1.000000}%
\pgfsetstrokecolor{currentstroke}%
\pgfsetstrokeopacity{0.500000}%
\pgfsetdash{}{0pt}%
\pgfpathmoveto{\pgfqpoint{3.191936in}{3.062301in}}%
\pgfpathlineto{\pgfqpoint{3.203277in}{3.069739in}}%
\pgfpathlineto{\pgfqpoint{3.214613in}{3.076855in}}%
\pgfpathlineto{\pgfqpoint{3.225944in}{3.083517in}}%
\pgfpathlineto{\pgfqpoint{3.237268in}{3.089596in}}%
\pgfpathlineto{\pgfqpoint{3.248586in}{3.095072in}}%
\pgfpathlineto{\pgfqpoint{3.242315in}{3.106976in}}%
\pgfpathlineto{\pgfqpoint{3.236047in}{3.118802in}}%
\pgfpathlineto{\pgfqpoint{3.229782in}{3.130575in}}%
\pgfpathlineto{\pgfqpoint{3.223519in}{3.142324in}}%
\pgfpathlineto{\pgfqpoint{3.217260in}{3.154074in}}%
\pgfpathlineto{\pgfqpoint{3.205949in}{3.148403in}}%
\pgfpathlineto{\pgfqpoint{3.194632in}{3.142125in}}%
\pgfpathlineto{\pgfqpoint{3.183310in}{3.135248in}}%
\pgfpathlineto{\pgfqpoint{3.171982in}{3.127869in}}%
\pgfpathlineto{\pgfqpoint{3.160650in}{3.120088in}}%
\pgfpathlineto{\pgfqpoint{3.166901in}{3.108624in}}%
\pgfpathlineto{\pgfqpoint{3.173156in}{3.097151in}}%
\pgfpathlineto{\pgfqpoint{3.179413in}{3.085631in}}%
\pgfpathlineto{\pgfqpoint{3.185673in}{3.074027in}}%
\pgfpathclose%
\pgfusepath{stroke,fill}%
\end{pgfscope}%
\begin{pgfscope}%
\pgfpathrectangle{\pgfqpoint{0.887500in}{0.275000in}}{\pgfqpoint{4.225000in}{4.225000in}}%
\pgfusepath{clip}%
\pgfsetbuttcap%
\pgfsetroundjoin%
\definecolor{currentfill}{rgb}{0.188923,0.410910,0.556326}%
\pgfsetfillcolor{currentfill}%
\pgfsetfillopacity{0.700000}%
\pgfsetlinewidth{0.501875pt}%
\definecolor{currentstroke}{rgb}{1.000000,1.000000,1.000000}%
\pgfsetstrokecolor{currentstroke}%
\pgfsetstrokeopacity{0.500000}%
\pgfsetdash{}{0pt}%
\pgfpathmoveto{\pgfqpoint{4.604757in}{2.101003in}}%
\pgfpathlineto{\pgfqpoint{4.615729in}{2.104335in}}%
\pgfpathlineto{\pgfqpoint{4.626695in}{2.107670in}}%
\pgfpathlineto{\pgfqpoint{4.637656in}{2.111009in}}%
\pgfpathlineto{\pgfqpoint{4.648612in}{2.114353in}}%
\pgfpathlineto{\pgfqpoint{4.642105in}{2.128916in}}%
\pgfpathlineto{\pgfqpoint{4.635600in}{2.143494in}}%
\pgfpathlineto{\pgfqpoint{4.629099in}{2.158082in}}%
\pgfpathlineto{\pgfqpoint{4.622599in}{2.172668in}}%
\pgfpathlineto{\pgfqpoint{4.616103in}{2.187246in}}%
\pgfpathlineto{\pgfqpoint{4.605147in}{2.183886in}}%
\pgfpathlineto{\pgfqpoint{4.594186in}{2.180518in}}%
\pgfpathlineto{\pgfqpoint{4.583218in}{2.177140in}}%
\pgfpathlineto{\pgfqpoint{4.572245in}{2.173746in}}%
\pgfpathlineto{\pgfqpoint{4.578743in}{2.159211in}}%
\pgfpathlineto{\pgfqpoint{4.585243in}{2.144660in}}%
\pgfpathlineto{\pgfqpoint{4.591745in}{2.130103in}}%
\pgfpathlineto{\pgfqpoint{4.598250in}{2.115548in}}%
\pgfpathclose%
\pgfusepath{stroke,fill}%
\end{pgfscope}%
\begin{pgfscope}%
\pgfpathrectangle{\pgfqpoint{0.887500in}{0.275000in}}{\pgfqpoint{4.225000in}{4.225000in}}%
\pgfusepath{clip}%
\pgfsetbuttcap%
\pgfsetroundjoin%
\definecolor{currentfill}{rgb}{0.377779,0.791781,0.377939}%
\pgfsetfillcolor{currentfill}%
\pgfsetfillopacity{0.700000}%
\pgfsetlinewidth{0.501875pt}%
\definecolor{currentstroke}{rgb}{1.000000,1.000000,1.000000}%
\pgfsetstrokecolor{currentstroke}%
\pgfsetstrokeopacity{0.500000}%
\pgfsetdash{}{0pt}%
\pgfpathmoveto{\pgfqpoint{3.455895in}{2.942289in}}%
\pgfpathlineto{\pgfqpoint{3.467163in}{2.946071in}}%
\pgfpathlineto{\pgfqpoint{3.478425in}{2.949809in}}%
\pgfpathlineto{\pgfqpoint{3.489681in}{2.953505in}}%
\pgfpathlineto{\pgfqpoint{3.500931in}{2.957162in}}%
\pgfpathlineto{\pgfqpoint{3.512176in}{2.960777in}}%
\pgfpathlineto{\pgfqpoint{3.505856in}{2.973878in}}%
\pgfpathlineto{\pgfqpoint{3.499538in}{2.986930in}}%
\pgfpathlineto{\pgfqpoint{3.493223in}{2.999935in}}%
\pgfpathlineto{\pgfqpoint{3.486910in}{3.012895in}}%
\pgfpathlineto{\pgfqpoint{3.480599in}{3.025812in}}%
\pgfpathlineto{\pgfqpoint{3.469358in}{3.022135in}}%
\pgfpathlineto{\pgfqpoint{3.458111in}{3.018440in}}%
\pgfpathlineto{\pgfqpoint{3.446859in}{3.014732in}}%
\pgfpathlineto{\pgfqpoint{3.435602in}{3.011012in}}%
\pgfpathlineto{\pgfqpoint{3.424339in}{3.007282in}}%
\pgfpathlineto{\pgfqpoint{3.430646in}{2.994353in}}%
\pgfpathlineto{\pgfqpoint{3.436955in}{2.981394in}}%
\pgfpathlineto{\pgfqpoint{3.443266in}{2.968401in}}%
\pgfpathlineto{\pgfqpoint{3.449579in}{2.955368in}}%
\pgfpathclose%
\pgfusepath{stroke,fill}%
\end{pgfscope}%
\begin{pgfscope}%
\pgfpathrectangle{\pgfqpoint{0.887500in}{0.275000in}}{\pgfqpoint{4.225000in}{4.225000in}}%
\pgfusepath{clip}%
\pgfsetbuttcap%
\pgfsetroundjoin%
\definecolor{currentfill}{rgb}{0.132268,0.655014,0.519661}%
\pgfsetfillcolor{currentfill}%
\pgfsetfillopacity{0.700000}%
\pgfsetlinewidth{0.501875pt}%
\definecolor{currentstroke}{rgb}{1.000000,1.000000,1.000000}%
\pgfsetstrokecolor{currentstroke}%
\pgfsetstrokeopacity{0.500000}%
\pgfsetdash{}{0pt}%
\pgfpathmoveto{\pgfqpoint{3.926937in}{2.622033in}}%
\pgfpathlineto{\pgfqpoint{3.938085in}{2.625457in}}%
\pgfpathlineto{\pgfqpoint{3.949228in}{2.628878in}}%
\pgfpathlineto{\pgfqpoint{3.960365in}{2.632294in}}%
\pgfpathlineto{\pgfqpoint{3.971496in}{2.635701in}}%
\pgfpathlineto{\pgfqpoint{3.982622in}{2.639095in}}%
\pgfpathlineto{\pgfqpoint{3.976223in}{2.653112in}}%
\pgfpathlineto{\pgfqpoint{3.969826in}{2.667113in}}%
\pgfpathlineto{\pgfqpoint{3.963431in}{2.681095in}}%
\pgfpathlineto{\pgfqpoint{3.957038in}{2.695052in}}%
\pgfpathlineto{\pgfqpoint{3.950647in}{2.708979in}}%
\pgfpathlineto{\pgfqpoint{3.939522in}{2.705531in}}%
\pgfpathlineto{\pgfqpoint{3.928392in}{2.702070in}}%
\pgfpathlineto{\pgfqpoint{3.917256in}{2.698600in}}%
\pgfpathlineto{\pgfqpoint{3.906115in}{2.695126in}}%
\pgfpathlineto{\pgfqpoint{3.894968in}{2.691654in}}%
\pgfpathlineto{\pgfqpoint{3.901358in}{2.677773in}}%
\pgfpathlineto{\pgfqpoint{3.907750in}{2.663868in}}%
\pgfpathlineto{\pgfqpoint{3.914143in}{2.649941in}}%
\pgfpathlineto{\pgfqpoint{3.920539in}{2.635995in}}%
\pgfpathclose%
\pgfusepath{stroke,fill}%
\end{pgfscope}%
\begin{pgfscope}%
\pgfpathrectangle{\pgfqpoint{0.887500in}{0.275000in}}{\pgfqpoint{4.225000in}{4.225000in}}%
\pgfusepath{clip}%
\pgfsetbuttcap%
\pgfsetroundjoin%
\definecolor{currentfill}{rgb}{0.133743,0.548535,0.553541}%
\pgfsetfillcolor{currentfill}%
\pgfsetfillopacity{0.700000}%
\pgfsetlinewidth{0.501875pt}%
\definecolor{currentstroke}{rgb}{1.000000,1.000000,1.000000}%
\pgfsetstrokecolor{currentstroke}%
\pgfsetstrokeopacity{0.500000}%
\pgfsetdash{}{0pt}%
\pgfpathmoveto{\pgfqpoint{4.222105in}{2.389547in}}%
\pgfpathlineto{\pgfqpoint{4.233178in}{2.392920in}}%
\pgfpathlineto{\pgfqpoint{4.244246in}{2.396280in}}%
\pgfpathlineto{\pgfqpoint{4.255307in}{2.399621in}}%
\pgfpathlineto{\pgfqpoint{4.266363in}{2.402939in}}%
\pgfpathlineto{\pgfqpoint{4.277413in}{2.406236in}}%
\pgfpathlineto{\pgfqpoint{4.270969in}{2.420685in}}%
\pgfpathlineto{\pgfqpoint{4.264526in}{2.435065in}}%
\pgfpathlineto{\pgfqpoint{4.258083in}{2.449387in}}%
\pgfpathlineto{\pgfqpoint{4.251642in}{2.463661in}}%
\pgfpathlineto{\pgfqpoint{4.245202in}{2.477897in}}%
\pgfpathlineto{\pgfqpoint{4.234153in}{2.474570in}}%
\pgfpathlineto{\pgfqpoint{4.223098in}{2.471232in}}%
\pgfpathlineto{\pgfqpoint{4.212038in}{2.467882in}}%
\pgfpathlineto{\pgfqpoint{4.200972in}{2.464523in}}%
\pgfpathlineto{\pgfqpoint{4.189900in}{2.461157in}}%
\pgfpathlineto{\pgfqpoint{4.196339in}{2.446938in}}%
\pgfpathlineto{\pgfqpoint{4.202779in}{2.432677in}}%
\pgfpathlineto{\pgfqpoint{4.209220in}{2.418365in}}%
\pgfpathlineto{\pgfqpoint{4.215662in}{2.403991in}}%
\pgfpathclose%
\pgfusepath{stroke,fill}%
\end{pgfscope}%
\begin{pgfscope}%
\pgfpathrectangle{\pgfqpoint{0.887500in}{0.275000in}}{\pgfqpoint{4.225000in}{4.225000in}}%
\pgfusepath{clip}%
\pgfsetbuttcap%
\pgfsetroundjoin%
\definecolor{currentfill}{rgb}{0.430983,0.808473,0.346476}%
\pgfsetfillcolor{currentfill}%
\pgfsetfillopacity{0.700000}%
\pgfsetlinewidth{0.501875pt}%
\definecolor{currentstroke}{rgb}{1.000000,1.000000,1.000000}%
\pgfsetstrokecolor{currentstroke}%
\pgfsetstrokeopacity{0.500000}%
\pgfsetdash{}{0pt}%
\pgfpathmoveto{\pgfqpoint{3.367947in}{2.988537in}}%
\pgfpathlineto{\pgfqpoint{3.379236in}{2.992295in}}%
\pgfpathlineto{\pgfqpoint{3.390520in}{2.996050in}}%
\pgfpathlineto{\pgfqpoint{3.401798in}{2.999800in}}%
\pgfpathlineto{\pgfqpoint{3.413072in}{3.003544in}}%
\pgfpathlineto{\pgfqpoint{3.424339in}{3.007282in}}%
\pgfpathlineto{\pgfqpoint{3.418036in}{3.020188in}}%
\pgfpathlineto{\pgfqpoint{3.411735in}{3.033077in}}%
\pgfpathlineto{\pgfqpoint{3.405436in}{3.045953in}}%
\pgfpathlineto{\pgfqpoint{3.399140in}{3.058822in}}%
\pgfpathlineto{\pgfqpoint{3.392847in}{3.071691in}}%
\pgfpathlineto{\pgfqpoint{3.381585in}{3.068035in}}%
\pgfpathlineto{\pgfqpoint{3.370317in}{3.064388in}}%
\pgfpathlineto{\pgfqpoint{3.359044in}{3.060759in}}%
\pgfpathlineto{\pgfqpoint{3.347766in}{3.057158in}}%
\pgfpathlineto{\pgfqpoint{3.336483in}{3.053592in}}%
\pgfpathlineto{\pgfqpoint{3.342771in}{3.040705in}}%
\pgfpathlineto{\pgfqpoint{3.349062in}{3.027748in}}%
\pgfpathlineto{\pgfqpoint{3.355354in}{3.014729in}}%
\pgfpathlineto{\pgfqpoint{3.361650in}{3.001656in}}%
\pgfpathclose%
\pgfusepath{stroke,fill}%
\end{pgfscope}%
\begin{pgfscope}%
\pgfpathrectangle{\pgfqpoint{0.887500in}{0.275000in}}{\pgfqpoint{4.225000in}{4.225000in}}%
\pgfusepath{clip}%
\pgfsetbuttcap%
\pgfsetroundjoin%
\definecolor{currentfill}{rgb}{0.177423,0.437527,0.557565}%
\pgfsetfillcolor{currentfill}%
\pgfsetfillopacity{0.700000}%
\pgfsetlinewidth{0.501875pt}%
\definecolor{currentstroke}{rgb}{1.000000,1.000000,1.000000}%
\pgfsetstrokecolor{currentstroke}%
\pgfsetstrokeopacity{0.500000}%
\pgfsetdash{}{0pt}%
\pgfpathmoveto{\pgfqpoint{4.517285in}{2.156386in}}%
\pgfpathlineto{\pgfqpoint{4.528290in}{2.159925in}}%
\pgfpathlineto{\pgfqpoint{4.539289in}{2.163426in}}%
\pgfpathlineto{\pgfqpoint{4.550281in}{2.166894in}}%
\pgfpathlineto{\pgfqpoint{4.561266in}{2.170332in}}%
\pgfpathlineto{\pgfqpoint{4.572245in}{2.173746in}}%
\pgfpathlineto{\pgfqpoint{4.565749in}{2.188258in}}%
\pgfpathlineto{\pgfqpoint{4.559254in}{2.202736in}}%
\pgfpathlineto{\pgfqpoint{4.552760in}{2.217173in}}%
\pgfpathlineto{\pgfqpoint{4.546267in}{2.231558in}}%
\pgfpathlineto{\pgfqpoint{4.539774in}{2.245885in}}%
\pgfpathlineto{\pgfqpoint{4.528792in}{2.242363in}}%
\pgfpathlineto{\pgfqpoint{4.517803in}{2.238796in}}%
\pgfpathlineto{\pgfqpoint{4.506806in}{2.235175in}}%
\pgfpathlineto{\pgfqpoint{4.495803in}{2.231494in}}%
\pgfpathlineto{\pgfqpoint{4.484791in}{2.227744in}}%
\pgfpathlineto{\pgfqpoint{4.491287in}{2.213542in}}%
\pgfpathlineto{\pgfqpoint{4.497785in}{2.199302in}}%
\pgfpathlineto{\pgfqpoint{4.504283in}{2.185028in}}%
\pgfpathlineto{\pgfqpoint{4.510783in}{2.170721in}}%
\pgfpathclose%
\pgfusepath{stroke,fill}%
\end{pgfscope}%
\begin{pgfscope}%
\pgfpathrectangle{\pgfqpoint{0.887500in}{0.275000in}}{\pgfqpoint{4.225000in}{4.225000in}}%
\pgfusepath{clip}%
\pgfsetbuttcap%
\pgfsetroundjoin%
\definecolor{currentfill}{rgb}{0.487026,0.823929,0.312321}%
\pgfsetfillcolor{currentfill}%
\pgfsetfillopacity{0.700000}%
\pgfsetlinewidth{0.501875pt}%
\definecolor{currentstroke}{rgb}{1.000000,1.000000,1.000000}%
\pgfsetstrokecolor{currentstroke}%
\pgfsetstrokeopacity{0.500000}%
\pgfsetdash{}{0pt}%
\pgfpathmoveto{\pgfqpoint{3.279974in}{3.033423in}}%
\pgfpathlineto{\pgfqpoint{3.291290in}{3.038147in}}%
\pgfpathlineto{\pgfqpoint{3.302599in}{3.042411in}}%
\pgfpathlineto{\pgfqpoint{3.313900in}{3.046330in}}%
\pgfpathlineto{\pgfqpoint{3.325194in}{3.050019in}}%
\pgfpathlineto{\pgfqpoint{3.336483in}{3.053592in}}%
\pgfpathlineto{\pgfqpoint{3.330197in}{3.066398in}}%
\pgfpathlineto{\pgfqpoint{3.323913in}{3.079114in}}%
\pgfpathlineto{\pgfqpoint{3.317631in}{3.091730in}}%
\pgfpathlineto{\pgfqpoint{3.311352in}{3.104236in}}%
\pgfpathlineto{\pgfqpoint{3.305074in}{3.116621in}}%
\pgfpathlineto{\pgfqpoint{3.293789in}{3.112773in}}%
\pgfpathlineto{\pgfqpoint{3.282498in}{3.108788in}}%
\pgfpathlineto{\pgfqpoint{3.271201in}{3.104573in}}%
\pgfpathlineto{\pgfqpoint{3.259897in}{3.100032in}}%
\pgfpathlineto{\pgfqpoint{3.248586in}{3.095072in}}%
\pgfpathlineto{\pgfqpoint{3.254859in}{3.083061in}}%
\pgfpathlineto{\pgfqpoint{3.261135in}{3.070918in}}%
\pgfpathlineto{\pgfqpoint{3.267413in}{3.058615in}}%
\pgfpathlineto{\pgfqpoint{3.273693in}{3.046126in}}%
\pgfpathclose%
\pgfusepath{stroke,fill}%
\end{pgfscope}%
\begin{pgfscope}%
\pgfpathrectangle{\pgfqpoint{0.887500in}{0.275000in}}{\pgfqpoint{4.225000in}{4.225000in}}%
\pgfusepath{clip}%
\pgfsetbuttcap%
\pgfsetroundjoin%
\definecolor{currentfill}{rgb}{0.159194,0.482237,0.558073}%
\pgfsetfillcolor{currentfill}%
\pgfsetfillopacity{0.700000}%
\pgfsetlinewidth{0.501875pt}%
\definecolor{currentstroke}{rgb}{1.000000,1.000000,1.000000}%
\pgfsetstrokecolor{currentstroke}%
\pgfsetstrokeopacity{0.500000}%
\pgfsetdash{}{0pt}%
\pgfpathmoveto{\pgfqpoint{2.975795in}{2.205153in}}%
\pgfpathlineto{\pgfqpoint{2.987133in}{2.226085in}}%
\pgfpathlineto{\pgfqpoint{2.998470in}{2.250748in}}%
\pgfpathlineto{\pgfqpoint{3.009810in}{2.278056in}}%
\pgfpathlineto{\pgfqpoint{3.021158in}{2.306925in}}%
\pgfpathlineto{\pgfqpoint{3.032514in}{2.336263in}}%
\pgfpathlineto{\pgfqpoint{3.026285in}{2.347343in}}%
\pgfpathlineto{\pgfqpoint{3.020060in}{2.358166in}}%
\pgfpathlineto{\pgfqpoint{3.013839in}{2.368659in}}%
\pgfpathlineto{\pgfqpoint{3.007622in}{2.378750in}}%
\pgfpathlineto{\pgfqpoint{3.001409in}{2.388379in}}%
\pgfpathlineto{\pgfqpoint{2.990083in}{2.357925in}}%
\pgfpathlineto{\pgfqpoint{2.978767in}{2.327922in}}%
\pgfpathlineto{\pgfqpoint{2.967457in}{2.299518in}}%
\pgfpathlineto{\pgfqpoint{2.956149in}{2.273858in}}%
\pgfpathlineto{\pgfqpoint{2.944836in}{2.252082in}}%
\pgfpathlineto{\pgfqpoint{2.951020in}{2.242698in}}%
\pgfpathlineto{\pgfqpoint{2.957208in}{2.233314in}}%
\pgfpathlineto{\pgfqpoint{2.963400in}{2.223930in}}%
\pgfpathlineto{\pgfqpoint{2.969595in}{2.214544in}}%
\pgfpathclose%
\pgfusepath{stroke,fill}%
\end{pgfscope}%
\begin{pgfscope}%
\pgfpathrectangle{\pgfqpoint{0.887500in}{0.275000in}}{\pgfqpoint{4.225000in}{4.225000in}}%
\pgfusepath{clip}%
\pgfsetbuttcap%
\pgfsetroundjoin%
\definecolor{currentfill}{rgb}{0.166617,0.463708,0.558119}%
\pgfsetfillcolor{currentfill}%
\pgfsetfillopacity{0.700000}%
\pgfsetlinewidth{0.501875pt}%
\definecolor{currentstroke}{rgb}{1.000000,1.000000,1.000000}%
\pgfsetstrokecolor{currentstroke}%
\pgfsetstrokeopacity{0.500000}%
\pgfsetdash{}{0pt}%
\pgfpathmoveto{\pgfqpoint{2.685924in}{2.218800in}}%
\pgfpathlineto{\pgfqpoint{2.697386in}{2.221795in}}%
\pgfpathlineto{\pgfqpoint{2.708841in}{2.224843in}}%
\pgfpathlineto{\pgfqpoint{2.720290in}{2.228033in}}%
\pgfpathlineto{\pgfqpoint{2.731731in}{2.231455in}}%
\pgfpathlineto{\pgfqpoint{2.743163in}{2.235199in}}%
\pgfpathlineto{\pgfqpoint{2.737038in}{2.244188in}}%
\pgfpathlineto{\pgfqpoint{2.730916in}{2.253150in}}%
\pgfpathlineto{\pgfqpoint{2.724798in}{2.262087in}}%
\pgfpathlineto{\pgfqpoint{2.718685in}{2.270999in}}%
\pgfpathlineto{\pgfqpoint{2.712575in}{2.279885in}}%
\pgfpathlineto{\pgfqpoint{2.701154in}{2.276138in}}%
\pgfpathlineto{\pgfqpoint{2.689724in}{2.272722in}}%
\pgfpathlineto{\pgfqpoint{2.678286in}{2.269544in}}%
\pgfpathlineto{\pgfqpoint{2.666841in}{2.266512in}}%
\pgfpathlineto{\pgfqpoint{2.655390in}{2.263536in}}%
\pgfpathlineto{\pgfqpoint{2.661488in}{2.254644in}}%
\pgfpathlineto{\pgfqpoint{2.667591in}{2.245724in}}%
\pgfpathlineto{\pgfqpoint{2.673698in}{2.236777in}}%
\pgfpathlineto{\pgfqpoint{2.679808in}{2.227803in}}%
\pgfpathclose%
\pgfusepath{stroke,fill}%
\end{pgfscope}%
\begin{pgfscope}%
\pgfpathrectangle{\pgfqpoint{0.887500in}{0.275000in}}{\pgfqpoint{4.225000in}{4.225000in}}%
\pgfusepath{clip}%
\pgfsetbuttcap%
\pgfsetroundjoin%
\definecolor{currentfill}{rgb}{0.133743,0.548535,0.553541}%
\pgfsetfillcolor{currentfill}%
\pgfsetfillopacity{0.700000}%
\pgfsetlinewidth{0.501875pt}%
\definecolor{currentstroke}{rgb}{1.000000,1.000000,1.000000}%
\pgfsetstrokecolor{currentstroke}%
\pgfsetstrokeopacity{0.500000}%
\pgfsetdash{}{0pt}%
\pgfpathmoveto{\pgfqpoint{1.723093in}{2.404743in}}%
\pgfpathlineto{\pgfqpoint{1.734789in}{2.408098in}}%
\pgfpathlineto{\pgfqpoint{1.746480in}{2.411451in}}%
\pgfpathlineto{\pgfqpoint{1.758166in}{2.414805in}}%
\pgfpathlineto{\pgfqpoint{1.769845in}{2.418161in}}%
\pgfpathlineto{\pgfqpoint{1.781518in}{2.421518in}}%
\pgfpathlineto{\pgfqpoint{1.775720in}{2.429824in}}%
\pgfpathlineto{\pgfqpoint{1.769926in}{2.438108in}}%
\pgfpathlineto{\pgfqpoint{1.764137in}{2.446368in}}%
\pgfpathlineto{\pgfqpoint{1.758351in}{2.454608in}}%
\pgfpathlineto{\pgfqpoint{1.752571in}{2.462826in}}%
\pgfpathlineto{\pgfqpoint{1.740910in}{2.459490in}}%
\pgfpathlineto{\pgfqpoint{1.729242in}{2.456155in}}%
\pgfpathlineto{\pgfqpoint{1.717570in}{2.452822in}}%
\pgfpathlineto{\pgfqpoint{1.705891in}{2.449489in}}%
\pgfpathlineto{\pgfqpoint{1.694206in}{2.446155in}}%
\pgfpathlineto{\pgfqpoint{1.699975in}{2.437918in}}%
\pgfpathlineto{\pgfqpoint{1.705747in}{2.429659in}}%
\pgfpathlineto{\pgfqpoint{1.711525in}{2.421377in}}%
\pgfpathlineto{\pgfqpoint{1.717306in}{2.413072in}}%
\pgfpathclose%
\pgfusepath{stroke,fill}%
\end{pgfscope}%
\begin{pgfscope}%
\pgfpathrectangle{\pgfqpoint{0.887500in}{0.275000in}}{\pgfqpoint{4.225000in}{4.225000in}}%
\pgfusepath{clip}%
\pgfsetbuttcap%
\pgfsetroundjoin%
\definecolor{currentfill}{rgb}{0.153894,0.680203,0.504172}%
\pgfsetfillcolor{currentfill}%
\pgfsetfillopacity{0.700000}%
\pgfsetlinewidth{0.501875pt}%
\definecolor{currentstroke}{rgb}{1.000000,1.000000,1.000000}%
\pgfsetstrokecolor{currentstroke}%
\pgfsetstrokeopacity{0.500000}%
\pgfsetdash{}{0pt}%
\pgfpathmoveto{\pgfqpoint{3.839158in}{2.674456in}}%
\pgfpathlineto{\pgfqpoint{3.850331in}{2.677868in}}%
\pgfpathlineto{\pgfqpoint{3.861498in}{2.681293in}}%
\pgfpathlineto{\pgfqpoint{3.872660in}{2.684733in}}%
\pgfpathlineto{\pgfqpoint{3.883817in}{2.688188in}}%
\pgfpathlineto{\pgfqpoint{3.894968in}{2.691654in}}%
\pgfpathlineto{\pgfqpoint{3.888581in}{2.705507in}}%
\pgfpathlineto{\pgfqpoint{3.882195in}{2.719331in}}%
\pgfpathlineto{\pgfqpoint{3.875812in}{2.733122in}}%
\pgfpathlineto{\pgfqpoint{3.869430in}{2.746878in}}%
\pgfpathlineto{\pgfqpoint{3.863050in}{2.760601in}}%
\pgfpathlineto{\pgfqpoint{3.851901in}{2.757130in}}%
\pgfpathlineto{\pgfqpoint{3.840747in}{2.753674in}}%
\pgfpathlineto{\pgfqpoint{3.829588in}{2.750239in}}%
\pgfpathlineto{\pgfqpoint{3.818423in}{2.746824in}}%
\pgfpathlineto{\pgfqpoint{3.807254in}{2.743426in}}%
\pgfpathlineto{\pgfqpoint{3.813631in}{2.729681in}}%
\pgfpathlineto{\pgfqpoint{3.820009in}{2.715910in}}%
\pgfpathlineto{\pgfqpoint{3.826390in}{2.702115in}}%
\pgfpathlineto{\pgfqpoint{3.832773in}{2.688297in}}%
\pgfpathclose%
\pgfusepath{stroke,fill}%
\end{pgfscope}%
\begin{pgfscope}%
\pgfpathrectangle{\pgfqpoint{0.887500in}{0.275000in}}{\pgfqpoint{4.225000in}{4.225000in}}%
\pgfusepath{clip}%
\pgfsetbuttcap%
\pgfsetroundjoin%
\definecolor{currentfill}{rgb}{0.143343,0.522773,0.556295}%
\pgfsetfillcolor{currentfill}%
\pgfsetfillopacity{0.700000}%
\pgfsetlinewidth{0.501875pt}%
\definecolor{currentstroke}{rgb}{1.000000,1.000000,1.000000}%
\pgfsetstrokecolor{currentstroke}%
\pgfsetstrokeopacity{0.500000}%
\pgfsetdash{}{0pt}%
\pgfpathmoveto{\pgfqpoint{2.043982in}{2.345219in}}%
\pgfpathlineto{\pgfqpoint{2.055600in}{2.348562in}}%
\pgfpathlineto{\pgfqpoint{2.067213in}{2.351908in}}%
\pgfpathlineto{\pgfqpoint{2.078820in}{2.355260in}}%
\pgfpathlineto{\pgfqpoint{2.090421in}{2.358622in}}%
\pgfpathlineto{\pgfqpoint{2.102016in}{2.361996in}}%
\pgfpathlineto{\pgfqpoint{2.096105in}{2.370504in}}%
\pgfpathlineto{\pgfqpoint{2.090199in}{2.378992in}}%
\pgfpathlineto{\pgfqpoint{2.084297in}{2.387461in}}%
\pgfpathlineto{\pgfqpoint{2.078399in}{2.395910in}}%
\pgfpathlineto{\pgfqpoint{2.072505in}{2.404340in}}%
\pgfpathlineto{\pgfqpoint{2.060922in}{2.400993in}}%
\pgfpathlineto{\pgfqpoint{2.049333in}{2.397658in}}%
\pgfpathlineto{\pgfqpoint{2.037737in}{2.394332in}}%
\pgfpathlineto{\pgfqpoint{2.026136in}{2.391013in}}%
\pgfpathlineto{\pgfqpoint{2.014529in}{2.387699in}}%
\pgfpathlineto{\pgfqpoint{2.020411in}{2.379244in}}%
\pgfpathlineto{\pgfqpoint{2.026297in}{2.370769in}}%
\pgfpathlineto{\pgfqpoint{2.032188in}{2.362273in}}%
\pgfpathlineto{\pgfqpoint{2.038082in}{2.353756in}}%
\pgfpathclose%
\pgfusepath{stroke,fill}%
\end{pgfscope}%
\begin{pgfscope}%
\pgfpathrectangle{\pgfqpoint{0.887500in}{0.275000in}}{\pgfqpoint{4.225000in}{4.225000in}}%
\pgfusepath{clip}%
\pgfsetbuttcap%
\pgfsetroundjoin%
\definecolor{currentfill}{rgb}{0.154815,0.493313,0.557840}%
\pgfsetfillcolor{currentfill}%
\pgfsetfillopacity{0.700000}%
\pgfsetlinewidth{0.501875pt}%
\definecolor{currentstroke}{rgb}{1.000000,1.000000,1.000000}%
\pgfsetstrokecolor{currentstroke}%
\pgfsetstrokeopacity{0.500000}%
\pgfsetdash{}{0pt}%
\pgfpathmoveto{\pgfqpoint{2.364938in}{2.283637in}}%
\pgfpathlineto{\pgfqpoint{2.376478in}{2.286972in}}%
\pgfpathlineto{\pgfqpoint{2.388012in}{2.290300in}}%
\pgfpathlineto{\pgfqpoint{2.399540in}{2.293626in}}%
\pgfpathlineto{\pgfqpoint{2.411062in}{2.296955in}}%
\pgfpathlineto{\pgfqpoint{2.422579in}{2.300291in}}%
\pgfpathlineto{\pgfqpoint{2.416559in}{2.308979in}}%
\pgfpathlineto{\pgfqpoint{2.410544in}{2.317646in}}%
\pgfpathlineto{\pgfqpoint{2.404533in}{2.326295in}}%
\pgfpathlineto{\pgfqpoint{2.398527in}{2.334924in}}%
\pgfpathlineto{\pgfqpoint{2.392524in}{2.343535in}}%
\pgfpathlineto{\pgfqpoint{2.381019in}{2.340220in}}%
\pgfpathlineto{\pgfqpoint{2.369508in}{2.336914in}}%
\pgfpathlineto{\pgfqpoint{2.357991in}{2.333613in}}%
\pgfpathlineto{\pgfqpoint{2.346468in}{2.330311in}}%
\pgfpathlineto{\pgfqpoint{2.334940in}{2.327003in}}%
\pgfpathlineto{\pgfqpoint{2.340931in}{2.318372in}}%
\pgfpathlineto{\pgfqpoint{2.346926in}{2.309720in}}%
\pgfpathlineto{\pgfqpoint{2.352926in}{2.301047in}}%
\pgfpathlineto{\pgfqpoint{2.358930in}{2.292353in}}%
\pgfpathclose%
\pgfusepath{stroke,fill}%
\end{pgfscope}%
\begin{pgfscope}%
\pgfpathrectangle{\pgfqpoint{0.887500in}{0.275000in}}{\pgfqpoint{4.225000in}{4.225000in}}%
\pgfusepath{clip}%
\pgfsetbuttcap%
\pgfsetroundjoin%
\definecolor{currentfill}{rgb}{0.166617,0.463708,0.558119}%
\pgfsetfillcolor{currentfill}%
\pgfsetfillopacity{0.700000}%
\pgfsetlinewidth{0.501875pt}%
\definecolor{currentstroke}{rgb}{1.000000,1.000000,1.000000}%
\pgfsetstrokecolor{currentstroke}%
\pgfsetstrokeopacity{0.500000}%
\pgfsetdash{}{0pt}%
\pgfpathmoveto{\pgfqpoint{4.429611in}{2.207682in}}%
\pgfpathlineto{\pgfqpoint{4.440664in}{2.211896in}}%
\pgfpathlineto{\pgfqpoint{4.451708in}{2.216001in}}%
\pgfpathlineto{\pgfqpoint{4.462744in}{2.220006in}}%
\pgfpathlineto{\pgfqpoint{4.473772in}{2.223918in}}%
\pgfpathlineto{\pgfqpoint{4.484791in}{2.227744in}}%
\pgfpathlineto{\pgfqpoint{4.478296in}{2.241907in}}%
\pgfpathlineto{\pgfqpoint{4.471803in}{2.256028in}}%
\pgfpathlineto{\pgfqpoint{4.465310in}{2.270110in}}%
\pgfpathlineto{\pgfqpoint{4.458819in}{2.284163in}}%
\pgfpathlineto{\pgfqpoint{4.452330in}{2.298196in}}%
\pgfpathlineto{\pgfqpoint{4.441314in}{2.294401in}}%
\pgfpathlineto{\pgfqpoint{4.430291in}{2.290545in}}%
\pgfpathlineto{\pgfqpoint{4.419260in}{2.286619in}}%
\pgfpathlineto{\pgfqpoint{4.408221in}{2.282618in}}%
\pgfpathlineto{\pgfqpoint{4.397175in}{2.278535in}}%
\pgfpathlineto{\pgfqpoint{4.403652in}{2.264245in}}%
\pgfpathlineto{\pgfqpoint{4.410135in}{2.250007in}}%
\pgfpathlineto{\pgfqpoint{4.416622in}{2.235831in}}%
\pgfpathlineto{\pgfqpoint{4.423114in}{2.221723in}}%
\pgfpathclose%
\pgfusepath{stroke,fill}%
\end{pgfscope}%
\begin{pgfscope}%
\pgfpathrectangle{\pgfqpoint{0.887500in}{0.275000in}}{\pgfqpoint{4.225000in}{4.225000in}}%
\pgfusepath{clip}%
\pgfsetbuttcap%
\pgfsetroundjoin%
\definecolor{currentfill}{rgb}{0.125394,0.574318,0.549086}%
\pgfsetfillcolor{currentfill}%
\pgfsetfillopacity{0.700000}%
\pgfsetlinewidth{0.501875pt}%
\definecolor{currentstroke}{rgb}{1.000000,1.000000,1.000000}%
\pgfsetstrokecolor{currentstroke}%
\pgfsetstrokeopacity{0.500000}%
\pgfsetdash{}{0pt}%
\pgfpathmoveto{\pgfqpoint{4.134460in}{2.444324in}}%
\pgfpathlineto{\pgfqpoint{4.145559in}{2.447685in}}%
\pgfpathlineto{\pgfqpoint{4.156652in}{2.451050in}}%
\pgfpathlineto{\pgfqpoint{4.167740in}{2.454418in}}%
\pgfpathlineto{\pgfqpoint{4.178823in}{2.457788in}}%
\pgfpathlineto{\pgfqpoint{4.189900in}{2.461157in}}%
\pgfpathlineto{\pgfqpoint{4.183463in}{2.475344in}}%
\pgfpathlineto{\pgfqpoint{4.177028in}{2.489508in}}%
\pgfpathlineto{\pgfqpoint{4.170594in}{2.503659in}}%
\pgfpathlineto{\pgfqpoint{4.164164in}{2.517801in}}%
\pgfpathlineto{\pgfqpoint{4.157735in}{2.531934in}}%
\pgfpathlineto{\pgfqpoint{4.146660in}{2.528580in}}%
\pgfpathlineto{\pgfqpoint{4.135580in}{2.525234in}}%
\pgfpathlineto{\pgfqpoint{4.124495in}{2.521894in}}%
\pgfpathlineto{\pgfqpoint{4.113404in}{2.518560in}}%
\pgfpathlineto{\pgfqpoint{4.102308in}{2.515230in}}%
\pgfpathlineto{\pgfqpoint{4.108735in}{2.501101in}}%
\pgfpathlineto{\pgfqpoint{4.115164in}{2.486949in}}%
\pgfpathlineto{\pgfqpoint{4.121594in}{2.472770in}}%
\pgfpathlineto{\pgfqpoint{4.128026in}{2.458562in}}%
\pgfpathclose%
\pgfusepath{stroke,fill}%
\end{pgfscope}%
\begin{pgfscope}%
\pgfpathrectangle{\pgfqpoint{0.887500in}{0.275000in}}{\pgfqpoint{4.225000in}{4.225000in}}%
\pgfusepath{clip}%
\pgfsetbuttcap%
\pgfsetroundjoin%
\definecolor{currentfill}{rgb}{0.126453,0.570633,0.549841}%
\pgfsetfillcolor{currentfill}%
\pgfsetfillopacity{0.700000}%
\pgfsetlinewidth{0.501875pt}%
\definecolor{currentstroke}{rgb}{1.000000,1.000000,1.000000}%
\pgfsetstrokecolor{currentstroke}%
\pgfsetstrokeopacity{0.500000}%
\pgfsetdash{}{0pt}%
\pgfpathmoveto{\pgfqpoint{3.001409in}{2.388379in}}%
\pgfpathlineto{\pgfqpoint{3.012745in}{2.418131in}}%
\pgfpathlineto{\pgfqpoint{3.024092in}{2.446211in}}%
\pgfpathlineto{\pgfqpoint{3.035449in}{2.472591in}}%
\pgfpathlineto{\pgfqpoint{3.046813in}{2.497533in}}%
\pgfpathlineto{\pgfqpoint{3.058185in}{2.521301in}}%
\pgfpathlineto{\pgfqpoint{3.051948in}{2.531128in}}%
\pgfpathlineto{\pgfqpoint{3.045715in}{2.540681in}}%
\pgfpathlineto{\pgfqpoint{3.039485in}{2.550121in}}%
\pgfpathlineto{\pgfqpoint{3.033260in}{2.559610in}}%
\pgfpathlineto{\pgfqpoint{3.027038in}{2.569310in}}%
\pgfpathlineto{\pgfqpoint{3.015693in}{2.544045in}}%
\pgfpathlineto{\pgfqpoint{3.004357in}{2.518032in}}%
\pgfpathlineto{\pgfqpoint{2.993030in}{2.490970in}}%
\pgfpathlineto{\pgfqpoint{2.981713in}{2.462560in}}%
\pgfpathlineto{\pgfqpoint{2.970408in}{2.432786in}}%
\pgfpathlineto{\pgfqpoint{2.976600in}{2.424012in}}%
\pgfpathlineto{\pgfqpoint{2.982796in}{2.415302in}}%
\pgfpathlineto{\pgfqpoint{2.988996in}{2.406538in}}%
\pgfpathlineto{\pgfqpoint{2.995200in}{2.397603in}}%
\pgfpathclose%
\pgfusepath{stroke,fill}%
\end{pgfscope}%
\begin{pgfscope}%
\pgfpathrectangle{\pgfqpoint{0.887500in}{0.275000in}}{\pgfqpoint{4.225000in}{4.225000in}}%
\pgfusepath{clip}%
\pgfsetbuttcap%
\pgfsetroundjoin%
\definecolor{currentfill}{rgb}{0.327796,0.773980,0.406640}%
\pgfsetfillcolor{currentfill}%
\pgfsetfillopacity{0.700000}%
\pgfsetlinewidth{0.501875pt}%
\definecolor{currentstroke}{rgb}{1.000000,1.000000,1.000000}%
\pgfsetstrokecolor{currentstroke}%
\pgfsetstrokeopacity{0.500000}%
\pgfsetdash{}{0pt}%
\pgfpathmoveto{\pgfqpoint{3.021538in}{2.843518in}}%
\pgfpathlineto{\pgfqpoint{3.032884in}{2.871616in}}%
\pgfpathlineto{\pgfqpoint{3.044239in}{2.898494in}}%
\pgfpathlineto{\pgfqpoint{3.055601in}{2.923552in}}%
\pgfpathlineto{\pgfqpoint{3.066970in}{2.946189in}}%
\pgfpathlineto{\pgfqpoint{3.078343in}{2.965803in}}%
\pgfpathlineto{\pgfqpoint{3.072102in}{2.977962in}}%
\pgfpathlineto{\pgfqpoint{3.065865in}{2.989126in}}%
\pgfpathlineto{\pgfqpoint{3.059630in}{2.999216in}}%
\pgfpathlineto{\pgfqpoint{3.053399in}{3.008158in}}%
\pgfpathlineto{\pgfqpoint{3.047173in}{3.015875in}}%
\pgfpathlineto{\pgfqpoint{3.035822in}{2.996605in}}%
\pgfpathlineto{\pgfqpoint{3.024476in}{2.974756in}}%
\pgfpathlineto{\pgfqpoint{3.013137in}{2.950801in}}%
\pgfpathlineto{\pgfqpoint{3.001807in}{2.925215in}}%
\pgfpathlineto{\pgfqpoint{2.990486in}{2.898469in}}%
\pgfpathlineto{\pgfqpoint{2.996685in}{2.890787in}}%
\pgfpathlineto{\pgfqpoint{3.002891in}{2.881196in}}%
\pgfpathlineto{\pgfqpoint{3.009102in}{2.869951in}}%
\pgfpathlineto{\pgfqpoint{3.015318in}{2.857307in}}%
\pgfpathclose%
\pgfusepath{stroke,fill}%
\end{pgfscope}%
\begin{pgfscope}%
\pgfpathrectangle{\pgfqpoint{0.887500in}{0.275000in}}{\pgfqpoint{4.225000in}{4.225000in}}%
\pgfusepath{clip}%
\pgfsetbuttcap%
\pgfsetroundjoin%
\definecolor{currentfill}{rgb}{0.185783,0.704891,0.485273}%
\pgfsetfillcolor{currentfill}%
\pgfsetfillopacity{0.700000}%
\pgfsetlinewidth{0.501875pt}%
\definecolor{currentstroke}{rgb}{1.000000,1.000000,1.000000}%
\pgfsetstrokecolor{currentstroke}%
\pgfsetstrokeopacity{0.500000}%
\pgfsetdash{}{0pt}%
\pgfpathmoveto{\pgfqpoint{3.751330in}{2.726628in}}%
\pgfpathlineto{\pgfqpoint{3.762525in}{2.729971in}}%
\pgfpathlineto{\pgfqpoint{3.773715in}{2.733319in}}%
\pgfpathlineto{\pgfqpoint{3.784900in}{2.736676in}}%
\pgfpathlineto{\pgfqpoint{3.796080in}{2.740044in}}%
\pgfpathlineto{\pgfqpoint{3.807254in}{2.743426in}}%
\pgfpathlineto{\pgfqpoint{3.800880in}{2.757146in}}%
\pgfpathlineto{\pgfqpoint{3.794507in}{2.770841in}}%
\pgfpathlineto{\pgfqpoint{3.788137in}{2.784510in}}%
\pgfpathlineto{\pgfqpoint{3.781769in}{2.798154in}}%
\pgfpathlineto{\pgfqpoint{3.775403in}{2.811772in}}%
\pgfpathlineto{\pgfqpoint{3.764232in}{2.808391in}}%
\pgfpathlineto{\pgfqpoint{3.753055in}{2.805019in}}%
\pgfpathlineto{\pgfqpoint{3.741873in}{2.801652in}}%
\pgfpathlineto{\pgfqpoint{3.730686in}{2.798288in}}%
\pgfpathlineto{\pgfqpoint{3.719493in}{2.794923in}}%
\pgfpathlineto{\pgfqpoint{3.725856in}{2.781338in}}%
\pgfpathlineto{\pgfqpoint{3.732222in}{2.767714in}}%
\pgfpathlineto{\pgfqpoint{3.738589in}{2.754054in}}%
\pgfpathlineto{\pgfqpoint{3.744959in}{2.740358in}}%
\pgfpathclose%
\pgfusepath{stroke,fill}%
\end{pgfscope}%
\begin{pgfscope}%
\pgfpathrectangle{\pgfqpoint{0.887500in}{0.275000in}}{\pgfqpoint{4.225000in}{4.225000in}}%
\pgfusepath{clip}%
\pgfsetbuttcap%
\pgfsetroundjoin%
\definecolor{currentfill}{rgb}{0.156270,0.489624,0.557936}%
\pgfsetfillcolor{currentfill}%
\pgfsetfillopacity{0.700000}%
\pgfsetlinewidth{0.501875pt}%
\definecolor{currentstroke}{rgb}{1.000000,1.000000,1.000000}%
\pgfsetstrokecolor{currentstroke}%
\pgfsetstrokeopacity{0.500000}%
\pgfsetdash{}{0pt}%
\pgfpathmoveto{\pgfqpoint{4.341865in}{2.258044in}}%
\pgfpathlineto{\pgfqpoint{4.352932in}{2.261986in}}%
\pgfpathlineto{\pgfqpoint{4.363998in}{2.266057in}}%
\pgfpathlineto{\pgfqpoint{4.375061in}{2.270206in}}%
\pgfpathlineto{\pgfqpoint{4.386121in}{2.274382in}}%
\pgfpathlineto{\pgfqpoint{4.397175in}{2.278535in}}%
\pgfpathlineto{\pgfqpoint{4.390701in}{2.292871in}}%
\pgfpathlineto{\pgfqpoint{4.384231in}{2.307243in}}%
\pgfpathlineto{\pgfqpoint{4.377765in}{2.321645in}}%
\pgfpathlineto{\pgfqpoint{4.371301in}{2.336067in}}%
\pgfpathlineto{\pgfqpoint{4.364841in}{2.350502in}}%
\pgfpathlineto{\pgfqpoint{4.353807in}{2.346898in}}%
\pgfpathlineto{\pgfqpoint{4.342769in}{2.343296in}}%
\pgfpathlineto{\pgfqpoint{4.331725in}{2.339710in}}%
\pgfpathlineto{\pgfqpoint{4.320677in}{2.336153in}}%
\pgfpathlineto{\pgfqpoint{4.309625in}{2.332637in}}%
\pgfpathlineto{\pgfqpoint{4.316067in}{2.317682in}}%
\pgfpathlineto{\pgfqpoint{4.322511in}{2.302711in}}%
\pgfpathlineto{\pgfqpoint{4.328958in}{2.287759in}}%
\pgfpathlineto{\pgfqpoint{4.335409in}{2.272859in}}%
\pgfpathclose%
\pgfusepath{stroke,fill}%
\end{pgfscope}%
\begin{pgfscope}%
\pgfpathrectangle{\pgfqpoint{0.887500in}{0.275000in}}{\pgfqpoint{4.225000in}{4.225000in}}%
\pgfusepath{clip}%
\pgfsetbuttcap%
\pgfsetroundjoin%
\definecolor{currentfill}{rgb}{0.171176,0.452530,0.557965}%
\pgfsetfillcolor{currentfill}%
\pgfsetfillopacity{0.700000}%
\pgfsetlinewidth{0.501875pt}%
\definecolor{currentstroke}{rgb}{1.000000,1.000000,1.000000}%
\pgfsetstrokecolor{currentstroke}%
\pgfsetstrokeopacity{0.500000}%
\pgfsetdash{}{0pt}%
\pgfpathmoveto{\pgfqpoint{2.773853in}{2.189855in}}%
\pgfpathlineto{\pgfqpoint{2.785288in}{2.194028in}}%
\pgfpathlineto{\pgfqpoint{2.796716in}{2.198496in}}%
\pgfpathlineto{\pgfqpoint{2.808139in}{2.202967in}}%
\pgfpathlineto{\pgfqpoint{2.819559in}{2.207141in}}%
\pgfpathlineto{\pgfqpoint{2.830977in}{2.210719in}}%
\pgfpathlineto{\pgfqpoint{2.824820in}{2.219853in}}%
\pgfpathlineto{\pgfqpoint{2.818666in}{2.228954in}}%
\pgfpathlineto{\pgfqpoint{2.812517in}{2.238021in}}%
\pgfpathlineto{\pgfqpoint{2.806372in}{2.247058in}}%
\pgfpathlineto{\pgfqpoint{2.800231in}{2.256064in}}%
\pgfpathlineto{\pgfqpoint{2.788824in}{2.252439in}}%
\pgfpathlineto{\pgfqpoint{2.777416in}{2.248251in}}%
\pgfpathlineto{\pgfqpoint{2.766004in}{2.243791in}}%
\pgfpathlineto{\pgfqpoint{2.754587in}{2.239346in}}%
\pgfpathlineto{\pgfqpoint{2.743163in}{2.235199in}}%
\pgfpathlineto{\pgfqpoint{2.749293in}{2.226184in}}%
\pgfpathlineto{\pgfqpoint{2.755427in}{2.217143in}}%
\pgfpathlineto{\pgfqpoint{2.761565in}{2.208075in}}%
\pgfpathlineto{\pgfqpoint{2.767707in}{2.198979in}}%
\pgfpathclose%
\pgfusepath{stroke,fill}%
\end{pgfscope}%
\begin{pgfscope}%
\pgfpathrectangle{\pgfqpoint{0.887500in}{0.275000in}}{\pgfqpoint{4.225000in}{4.225000in}}%
\pgfusepath{clip}%
\pgfsetbuttcap%
\pgfsetroundjoin%
\definecolor{currentfill}{rgb}{0.487026,0.823929,0.312321}%
\pgfsetfillcolor{currentfill}%
\pgfsetfillopacity{0.700000}%
\pgfsetlinewidth{0.501875pt}%
\definecolor{currentstroke}{rgb}{1.000000,1.000000,1.000000}%
\pgfsetstrokecolor{currentstroke}%
\pgfsetstrokeopacity{0.500000}%
\pgfsetdash{}{0pt}%
\pgfpathmoveto{\pgfqpoint{3.135184in}{3.024383in}}%
\pgfpathlineto{\pgfqpoint{3.146541in}{3.031934in}}%
\pgfpathlineto{\pgfqpoint{3.157894in}{3.039373in}}%
\pgfpathlineto{\pgfqpoint{3.169244in}{3.046986in}}%
\pgfpathlineto{\pgfqpoint{3.180592in}{3.054672in}}%
\pgfpathlineto{\pgfqpoint{3.191936in}{3.062301in}}%
\pgfpathlineto{\pgfqpoint{3.185673in}{3.074027in}}%
\pgfpathlineto{\pgfqpoint{3.179413in}{3.085631in}}%
\pgfpathlineto{\pgfqpoint{3.173156in}{3.097151in}}%
\pgfpathlineto{\pgfqpoint{3.166901in}{3.108624in}}%
\pgfpathlineto{\pgfqpoint{3.160650in}{3.120088in}}%
\pgfpathlineto{\pgfqpoint{3.149315in}{3.112006in}}%
\pgfpathlineto{\pgfqpoint{3.137976in}{3.103724in}}%
\pgfpathlineto{\pgfqpoint{3.126634in}{3.095341in}}%
\pgfpathlineto{\pgfqpoint{3.115290in}{3.086934in}}%
\pgfpathlineto{\pgfqpoint{3.103942in}{3.078270in}}%
\pgfpathlineto{\pgfqpoint{3.110185in}{3.068389in}}%
\pgfpathlineto{\pgfqpoint{3.116431in}{3.058071in}}%
\pgfpathlineto{\pgfqpoint{3.122680in}{3.047304in}}%
\pgfpathlineto{\pgfqpoint{3.128931in}{3.036079in}}%
\pgfpathclose%
\pgfusepath{stroke,fill}%
\end{pgfscope}%
\begin{pgfscope}%
\pgfpathrectangle{\pgfqpoint{0.887500in}{0.275000in}}{\pgfqpoint{4.225000in}{4.225000in}}%
\pgfusepath{clip}%
\pgfsetbuttcap%
\pgfsetroundjoin%
\definecolor{currentfill}{rgb}{0.136408,0.541173,0.554483}%
\pgfsetfillcolor{currentfill}%
\pgfsetfillopacity{0.700000}%
\pgfsetlinewidth{0.501875pt}%
\definecolor{currentstroke}{rgb}{1.000000,1.000000,1.000000}%
\pgfsetstrokecolor{currentstroke}%
\pgfsetstrokeopacity{0.500000}%
\pgfsetdash{}{0pt}%
\pgfpathmoveto{\pgfqpoint{1.810577in}{2.379648in}}%
\pgfpathlineto{\pgfqpoint{1.822257in}{2.383030in}}%
\pgfpathlineto{\pgfqpoint{1.833931in}{2.386411in}}%
\pgfpathlineto{\pgfqpoint{1.845599in}{2.389787in}}%
\pgfpathlineto{\pgfqpoint{1.857262in}{2.393157in}}%
\pgfpathlineto{\pgfqpoint{1.868919in}{2.396519in}}%
\pgfpathlineto{\pgfqpoint{1.863086in}{2.404908in}}%
\pgfpathlineto{\pgfqpoint{1.857258in}{2.413277in}}%
\pgfpathlineto{\pgfqpoint{1.851435in}{2.421624in}}%
\pgfpathlineto{\pgfqpoint{1.845615in}{2.429950in}}%
\pgfpathlineto{\pgfqpoint{1.839800in}{2.438255in}}%
\pgfpathlineto{\pgfqpoint{1.828155in}{2.434921in}}%
\pgfpathlineto{\pgfqpoint{1.816504in}{2.431578in}}%
\pgfpathlineto{\pgfqpoint{1.804848in}{2.428228in}}%
\pgfpathlineto{\pgfqpoint{1.793186in}{2.424874in}}%
\pgfpathlineto{\pgfqpoint{1.781518in}{2.421518in}}%
\pgfpathlineto{\pgfqpoint{1.787321in}{2.413188in}}%
\pgfpathlineto{\pgfqpoint{1.793129in}{2.404837in}}%
\pgfpathlineto{\pgfqpoint{1.798940in}{2.396463in}}%
\pgfpathlineto{\pgfqpoint{1.804757in}{2.388066in}}%
\pgfpathclose%
\pgfusepath{stroke,fill}%
\end{pgfscope}%
\begin{pgfscope}%
\pgfpathrectangle{\pgfqpoint{0.887500in}{0.275000in}}{\pgfqpoint{4.225000in}{4.225000in}}%
\pgfusepath{clip}%
\pgfsetbuttcap%
\pgfsetroundjoin%
\definecolor{currentfill}{rgb}{0.119738,0.603785,0.541400}%
\pgfsetfillcolor{currentfill}%
\pgfsetfillopacity{0.700000}%
\pgfsetlinewidth{0.501875pt}%
\definecolor{currentstroke}{rgb}{1.000000,1.000000,1.000000}%
\pgfsetstrokecolor{currentstroke}%
\pgfsetstrokeopacity{0.500000}%
\pgfsetdash{}{0pt}%
\pgfpathmoveto{\pgfqpoint{4.046746in}{2.498529in}}%
\pgfpathlineto{\pgfqpoint{4.057870in}{2.501885in}}%
\pgfpathlineto{\pgfqpoint{4.068988in}{2.505231in}}%
\pgfpathlineto{\pgfqpoint{4.080100in}{2.508568in}}%
\pgfpathlineto{\pgfqpoint{4.091207in}{2.511900in}}%
\pgfpathlineto{\pgfqpoint{4.102308in}{2.515230in}}%
\pgfpathlineto{\pgfqpoint{4.095884in}{2.529339in}}%
\pgfpathlineto{\pgfqpoint{4.089461in}{2.543431in}}%
\pgfpathlineto{\pgfqpoint{4.083041in}{2.557510in}}%
\pgfpathlineto{\pgfqpoint{4.076623in}{2.571579in}}%
\pgfpathlineto{\pgfqpoint{4.070207in}{2.585640in}}%
\pgfpathlineto{\pgfqpoint{4.059108in}{2.582301in}}%
\pgfpathlineto{\pgfqpoint{4.048003in}{2.578964in}}%
\pgfpathlineto{\pgfqpoint{4.036892in}{2.575628in}}%
\pgfpathlineto{\pgfqpoint{4.025777in}{2.572286in}}%
\pgfpathlineto{\pgfqpoint{4.014655in}{2.568937in}}%
\pgfpathlineto{\pgfqpoint{4.021069in}{2.554889in}}%
\pgfpathlineto{\pgfqpoint{4.027485in}{2.540828in}}%
\pgfpathlineto{\pgfqpoint{4.033903in}{2.526750in}}%
\pgfpathlineto{\pgfqpoint{4.040324in}{2.512652in}}%
\pgfpathclose%
\pgfusepath{stroke,fill}%
\end{pgfscope}%
\begin{pgfscope}%
\pgfpathrectangle{\pgfqpoint{0.887500in}{0.275000in}}{\pgfqpoint{4.225000in}{4.225000in}}%
\pgfusepath{clip}%
\pgfsetbuttcap%
\pgfsetroundjoin%
\definecolor{currentfill}{rgb}{0.159194,0.482237,0.558073}%
\pgfsetfillcolor{currentfill}%
\pgfsetfillopacity{0.700000}%
\pgfsetlinewidth{0.501875pt}%
\definecolor{currentstroke}{rgb}{1.000000,1.000000,1.000000}%
\pgfsetstrokecolor{currentstroke}%
\pgfsetstrokeopacity{0.500000}%
\pgfsetdash{}{0pt}%
\pgfpathmoveto{\pgfqpoint{2.452738in}{2.256533in}}%
\pgfpathlineto{\pgfqpoint{2.464259in}{2.259914in}}%
\pgfpathlineto{\pgfqpoint{2.475775in}{2.263310in}}%
\pgfpathlineto{\pgfqpoint{2.487285in}{2.266727in}}%
\pgfpathlineto{\pgfqpoint{2.498788in}{2.270169in}}%
\pgfpathlineto{\pgfqpoint{2.510286in}{2.273641in}}%
\pgfpathlineto{\pgfqpoint{2.504234in}{2.282409in}}%
\pgfpathlineto{\pgfqpoint{2.498187in}{2.291155in}}%
\pgfpathlineto{\pgfqpoint{2.492144in}{2.299880in}}%
\pgfpathlineto{\pgfqpoint{2.486106in}{2.308584in}}%
\pgfpathlineto{\pgfqpoint{2.480071in}{2.317268in}}%
\pgfpathlineto{\pgfqpoint{2.468585in}{2.313815in}}%
\pgfpathlineto{\pgfqpoint{2.457092in}{2.310397in}}%
\pgfpathlineto{\pgfqpoint{2.445594in}{2.307007in}}%
\pgfpathlineto{\pgfqpoint{2.434089in}{2.303640in}}%
\pgfpathlineto{\pgfqpoint{2.422579in}{2.300291in}}%
\pgfpathlineto{\pgfqpoint{2.428602in}{2.291584in}}%
\pgfpathlineto{\pgfqpoint{2.434630in}{2.282855in}}%
\pgfpathlineto{\pgfqpoint{2.440661in}{2.274104in}}%
\pgfpathlineto{\pgfqpoint{2.446697in}{2.265330in}}%
\pgfpathclose%
\pgfusepath{stroke,fill}%
\end{pgfscope}%
\begin{pgfscope}%
\pgfpathrectangle{\pgfqpoint{0.887500in}{0.275000in}}{\pgfqpoint{4.225000in}{4.225000in}}%
\pgfusepath{clip}%
\pgfsetbuttcap%
\pgfsetroundjoin%
\definecolor{currentfill}{rgb}{0.226397,0.728888,0.462789}%
\pgfsetfillcolor{currentfill}%
\pgfsetfillopacity{0.700000}%
\pgfsetlinewidth{0.501875pt}%
\definecolor{currentstroke}{rgb}{1.000000,1.000000,1.000000}%
\pgfsetstrokecolor{currentstroke}%
\pgfsetstrokeopacity{0.500000}%
\pgfsetdash{}{0pt}%
\pgfpathmoveto{\pgfqpoint{3.663443in}{2.777835in}}%
\pgfpathlineto{\pgfqpoint{3.674665in}{2.781314in}}%
\pgfpathlineto{\pgfqpoint{3.685881in}{2.784754in}}%
\pgfpathlineto{\pgfqpoint{3.697091in}{2.788164in}}%
\pgfpathlineto{\pgfqpoint{3.708295in}{2.791551in}}%
\pgfpathlineto{\pgfqpoint{3.719493in}{2.794923in}}%
\pgfpathlineto{\pgfqpoint{3.713132in}{2.808469in}}%
\pgfpathlineto{\pgfqpoint{3.706772in}{2.821976in}}%
\pgfpathlineto{\pgfqpoint{3.700415in}{2.835441in}}%
\pgfpathlineto{\pgfqpoint{3.694060in}{2.848864in}}%
\pgfpathlineto{\pgfqpoint{3.687706in}{2.862247in}}%
\pgfpathlineto{\pgfqpoint{3.676511in}{2.858844in}}%
\pgfpathlineto{\pgfqpoint{3.665310in}{2.855442in}}%
\pgfpathlineto{\pgfqpoint{3.654103in}{2.852034in}}%
\pgfpathlineto{\pgfqpoint{3.642891in}{2.848611in}}%
\pgfpathlineto{\pgfqpoint{3.631673in}{2.845164in}}%
\pgfpathlineto{\pgfqpoint{3.638023in}{2.831752in}}%
\pgfpathlineto{\pgfqpoint{3.644374in}{2.818315in}}%
\pgfpathlineto{\pgfqpoint{3.650728in}{2.804852in}}%
\pgfpathlineto{\pgfqpoint{3.657085in}{2.791359in}}%
\pgfpathclose%
\pgfusepath{stroke,fill}%
\end{pgfscope}%
\begin{pgfscope}%
\pgfpathrectangle{\pgfqpoint{0.887500in}{0.275000in}}{\pgfqpoint{4.225000in}{4.225000in}}%
\pgfusepath{clip}%
\pgfsetbuttcap%
\pgfsetroundjoin%
\definecolor{currentfill}{rgb}{0.147607,0.511733,0.557049}%
\pgfsetfillcolor{currentfill}%
\pgfsetfillopacity{0.700000}%
\pgfsetlinewidth{0.501875pt}%
\definecolor{currentstroke}{rgb}{1.000000,1.000000,1.000000}%
\pgfsetstrokecolor{currentstroke}%
\pgfsetstrokeopacity{0.500000}%
\pgfsetdash{}{0pt}%
\pgfpathmoveto{\pgfqpoint{2.131633in}{2.319164in}}%
\pgfpathlineto{\pgfqpoint{2.143233in}{2.322581in}}%
\pgfpathlineto{\pgfqpoint{2.154827in}{2.326004in}}%
\pgfpathlineto{\pgfqpoint{2.166416in}{2.329431in}}%
\pgfpathlineto{\pgfqpoint{2.177999in}{2.332858in}}%
\pgfpathlineto{\pgfqpoint{2.189576in}{2.336281in}}%
\pgfpathlineto{\pgfqpoint{2.183633in}{2.344852in}}%
\pgfpathlineto{\pgfqpoint{2.177694in}{2.353405in}}%
\pgfpathlineto{\pgfqpoint{2.171759in}{2.361940in}}%
\pgfpathlineto{\pgfqpoint{2.165828in}{2.370456in}}%
\pgfpathlineto{\pgfqpoint{2.159902in}{2.378954in}}%
\pgfpathlineto{\pgfqpoint{2.148336in}{2.375562in}}%
\pgfpathlineto{\pgfqpoint{2.136765in}{2.372167in}}%
\pgfpathlineto{\pgfqpoint{2.125188in}{2.368772in}}%
\pgfpathlineto{\pgfqpoint{2.113605in}{2.365381in}}%
\pgfpathlineto{\pgfqpoint{2.102016in}{2.361996in}}%
\pgfpathlineto{\pgfqpoint{2.107931in}{2.353469in}}%
\pgfpathlineto{\pgfqpoint{2.113850in}{2.344922in}}%
\pgfpathlineto{\pgfqpoint{2.119773in}{2.336356in}}%
\pgfpathlineto{\pgfqpoint{2.125701in}{2.327770in}}%
\pgfpathclose%
\pgfusepath{stroke,fill}%
\end{pgfscope}%
\begin{pgfscope}%
\pgfpathrectangle{\pgfqpoint{0.887500in}{0.275000in}}{\pgfqpoint{4.225000in}{4.225000in}}%
\pgfusepath{clip}%
\pgfsetbuttcap%
\pgfsetroundjoin%
\definecolor{currentfill}{rgb}{0.134692,0.658636,0.517649}%
\pgfsetfillcolor{currentfill}%
\pgfsetfillopacity{0.700000}%
\pgfsetlinewidth{0.501875pt}%
\definecolor{currentstroke}{rgb}{1.000000,1.000000,1.000000}%
\pgfsetstrokecolor{currentstroke}%
\pgfsetstrokeopacity{0.500000}%
\pgfsetdash{}{0pt}%
\pgfpathmoveto{\pgfqpoint{3.027038in}{2.569310in}}%
\pgfpathlineto{\pgfqpoint{3.038389in}{2.594127in}}%
\pgfpathlineto{\pgfqpoint{3.049747in}{2.618800in}}%
\pgfpathlineto{\pgfqpoint{3.061112in}{2.643632in}}%
\pgfpathlineto{\pgfqpoint{3.072484in}{2.668792in}}%
\pgfpathlineto{\pgfqpoint{3.083862in}{2.694002in}}%
\pgfpathlineto{\pgfqpoint{3.077620in}{2.707646in}}%
\pgfpathlineto{\pgfqpoint{3.071380in}{2.721688in}}%
\pgfpathlineto{\pgfqpoint{3.065143in}{2.736141in}}%
\pgfpathlineto{\pgfqpoint{3.058908in}{2.751020in}}%
\pgfpathlineto{\pgfqpoint{3.052674in}{2.766339in}}%
\pgfpathlineto{\pgfqpoint{3.041317in}{2.737489in}}%
\pgfpathlineto{\pgfqpoint{3.029968in}{2.708891in}}%
\pgfpathlineto{\pgfqpoint{3.018628in}{2.681025in}}%
\pgfpathlineto{\pgfqpoint{3.007295in}{2.653725in}}%
\pgfpathlineto{\pgfqpoint{2.995970in}{2.626624in}}%
\pgfpathlineto{\pgfqpoint{3.002180in}{2.613448in}}%
\pgfpathlineto{\pgfqpoint{3.008391in}{2.601289in}}%
\pgfpathlineto{\pgfqpoint{3.014604in}{2.589987in}}%
\pgfpathlineto{\pgfqpoint{3.020819in}{2.579382in}}%
\pgfpathclose%
\pgfusepath{stroke,fill}%
\end{pgfscope}%
\begin{pgfscope}%
\pgfpathrectangle{\pgfqpoint{0.887500in}{0.275000in}}{\pgfqpoint{4.225000in}{4.225000in}}%
\pgfusepath{clip}%
\pgfsetbuttcap%
\pgfsetroundjoin%
\definecolor{currentfill}{rgb}{0.203063,0.379716,0.553925}%
\pgfsetfillcolor{currentfill}%
\pgfsetfillopacity{0.700000}%
\pgfsetlinewidth{0.501875pt}%
\definecolor{currentstroke}{rgb}{1.000000,1.000000,1.000000}%
\pgfsetstrokecolor{currentstroke}%
\pgfsetstrokeopacity{0.500000}%
\pgfsetdash{}{0pt}%
\pgfpathmoveto{\pgfqpoint{4.637337in}{2.028357in}}%
\pgfpathlineto{\pgfqpoint{4.648309in}{2.031676in}}%
\pgfpathlineto{\pgfqpoint{4.659276in}{2.035005in}}%
\pgfpathlineto{\pgfqpoint{4.670238in}{2.038343in}}%
\pgfpathlineto{\pgfqpoint{4.681194in}{2.041691in}}%
\pgfpathlineto{\pgfqpoint{4.674673in}{2.056215in}}%
\pgfpathlineto{\pgfqpoint{4.668153in}{2.070740in}}%
\pgfpathlineto{\pgfqpoint{4.661637in}{2.085269in}}%
\pgfpathlineto{\pgfqpoint{4.655123in}{2.099805in}}%
\pgfpathlineto{\pgfqpoint{4.648612in}{2.114353in}}%
\pgfpathlineto{\pgfqpoint{4.637656in}{2.111009in}}%
\pgfpathlineto{\pgfqpoint{4.626695in}{2.107670in}}%
\pgfpathlineto{\pgfqpoint{4.615729in}{2.104335in}}%
\pgfpathlineto{\pgfqpoint{4.604757in}{2.101003in}}%
\pgfpathlineto{\pgfqpoint{4.611268in}{2.086467in}}%
\pgfpathlineto{\pgfqpoint{4.617781in}{2.071937in}}%
\pgfpathlineto{\pgfqpoint{4.624297in}{2.057411in}}%
\pgfpathlineto{\pgfqpoint{4.630816in}{2.042885in}}%
\pgfpathclose%
\pgfusepath{stroke,fill}%
\end{pgfscope}%
\begin{pgfscope}%
\pgfpathrectangle{\pgfqpoint{0.887500in}{0.275000in}}{\pgfqpoint{4.225000in}{4.225000in}}%
\pgfusepath{clip}%
\pgfsetbuttcap%
\pgfsetroundjoin%
\definecolor{currentfill}{rgb}{0.177423,0.437527,0.557565}%
\pgfsetfillcolor{currentfill}%
\pgfsetfillopacity{0.700000}%
\pgfsetlinewidth{0.501875pt}%
\definecolor{currentstroke}{rgb}{1.000000,1.000000,1.000000}%
\pgfsetstrokecolor{currentstroke}%
\pgfsetstrokeopacity{0.500000}%
\pgfsetdash{}{0pt}%
\pgfpathmoveto{\pgfqpoint{2.861825in}{2.164516in}}%
\pgfpathlineto{\pgfqpoint{2.873252in}{2.167215in}}%
\pgfpathlineto{\pgfqpoint{2.884678in}{2.168765in}}%
\pgfpathlineto{\pgfqpoint{2.896103in}{2.168941in}}%
\pgfpathlineto{\pgfqpoint{2.907524in}{2.168281in}}%
\pgfpathlineto{\pgfqpoint{2.918935in}{2.167826in}}%
\pgfpathlineto{\pgfqpoint{2.912751in}{2.176897in}}%
\pgfpathlineto{\pgfqpoint{2.906569in}{2.185992in}}%
\pgfpathlineto{\pgfqpoint{2.900392in}{2.195126in}}%
\pgfpathlineto{\pgfqpoint{2.894219in}{2.204313in}}%
\pgfpathlineto{\pgfqpoint{2.888048in}{2.213564in}}%
\pgfpathlineto{\pgfqpoint{2.876643in}{2.214148in}}%
\pgfpathlineto{\pgfqpoint{2.865229in}{2.214953in}}%
\pgfpathlineto{\pgfqpoint{2.853812in}{2.214891in}}%
\pgfpathlineto{\pgfqpoint{2.842394in}{2.213402in}}%
\pgfpathlineto{\pgfqpoint{2.830977in}{2.210719in}}%
\pgfpathlineto{\pgfqpoint{2.837139in}{2.201550in}}%
\pgfpathlineto{\pgfqpoint{2.843304in}{2.192344in}}%
\pgfpathlineto{\pgfqpoint{2.849474in}{2.183103in}}%
\pgfpathlineto{\pgfqpoint{2.855647in}{2.173827in}}%
\pgfpathclose%
\pgfusepath{stroke,fill}%
\end{pgfscope}%
\begin{pgfscope}%
\pgfpathrectangle{\pgfqpoint{0.887500in}{0.275000in}}{\pgfqpoint{4.225000in}{4.225000in}}%
\pgfusepath{clip}%
\pgfsetbuttcap%
\pgfsetroundjoin%
\definecolor{currentfill}{rgb}{0.274149,0.751988,0.436601}%
\pgfsetfillcolor{currentfill}%
\pgfsetfillopacity{0.700000}%
\pgfsetlinewidth{0.501875pt}%
\definecolor{currentstroke}{rgb}{1.000000,1.000000,1.000000}%
\pgfsetstrokecolor{currentstroke}%
\pgfsetstrokeopacity{0.500000}%
\pgfsetdash{}{0pt}%
\pgfpathmoveto{\pgfqpoint{3.575496in}{2.827416in}}%
\pgfpathlineto{\pgfqpoint{3.586743in}{2.831031in}}%
\pgfpathlineto{\pgfqpoint{3.597984in}{2.834616in}}%
\pgfpathlineto{\pgfqpoint{3.609220in}{2.838169in}}%
\pgfpathlineto{\pgfqpoint{3.620450in}{2.841686in}}%
\pgfpathlineto{\pgfqpoint{3.631673in}{2.845164in}}%
\pgfpathlineto{\pgfqpoint{3.625326in}{2.858554in}}%
\pgfpathlineto{\pgfqpoint{3.618982in}{2.871923in}}%
\pgfpathlineto{\pgfqpoint{3.612639in}{2.885271in}}%
\pgfpathlineto{\pgfqpoint{3.606299in}{2.898600in}}%
\pgfpathlineto{\pgfqpoint{3.599962in}{2.911910in}}%
\pgfpathlineto{\pgfqpoint{3.588743in}{2.908520in}}%
\pgfpathlineto{\pgfqpoint{3.577518in}{2.905089in}}%
\pgfpathlineto{\pgfqpoint{3.566287in}{2.901616in}}%
\pgfpathlineto{\pgfqpoint{3.555051in}{2.898102in}}%
\pgfpathlineto{\pgfqpoint{3.543808in}{2.894545in}}%
\pgfpathlineto{\pgfqpoint{3.550141in}{2.881175in}}%
\pgfpathlineto{\pgfqpoint{3.556476in}{2.867774in}}%
\pgfpathlineto{\pgfqpoint{3.562814in}{2.854344in}}%
\pgfpathlineto{\pgfqpoint{3.569154in}{2.840890in}}%
\pgfpathclose%
\pgfusepath{stroke,fill}%
\end{pgfscope}%
\begin{pgfscope}%
\pgfpathrectangle{\pgfqpoint{0.887500in}{0.275000in}}{\pgfqpoint{4.225000in}{4.225000in}}%
\pgfusepath{clip}%
\pgfsetbuttcap%
\pgfsetroundjoin%
\definecolor{currentfill}{rgb}{0.121380,0.629492,0.531973}%
\pgfsetfillcolor{currentfill}%
\pgfsetfillopacity{0.700000}%
\pgfsetlinewidth{0.501875pt}%
\definecolor{currentstroke}{rgb}{1.000000,1.000000,1.000000}%
\pgfsetstrokecolor{currentstroke}%
\pgfsetstrokeopacity{0.500000}%
\pgfsetdash{}{0pt}%
\pgfpathmoveto{\pgfqpoint{3.958963in}{2.552045in}}%
\pgfpathlineto{\pgfqpoint{3.970112in}{2.555436in}}%
\pgfpathlineto{\pgfqpoint{3.981256in}{2.558823in}}%
\pgfpathlineto{\pgfqpoint{3.992395in}{2.562204in}}%
\pgfpathlineto{\pgfqpoint{4.003528in}{2.565576in}}%
\pgfpathlineto{\pgfqpoint{4.014655in}{2.568937in}}%
\pgfpathlineto{\pgfqpoint{4.008244in}{2.582976in}}%
\pgfpathlineto{\pgfqpoint{4.001834in}{2.597008in}}%
\pgfpathlineto{\pgfqpoint{3.995428in}{2.611039in}}%
\pgfpathlineto{\pgfqpoint{3.989024in}{2.625070in}}%
\pgfpathlineto{\pgfqpoint{3.982622in}{2.639095in}}%
\pgfpathlineto{\pgfqpoint{3.971496in}{2.635701in}}%
\pgfpathlineto{\pgfqpoint{3.960365in}{2.632294in}}%
\pgfpathlineto{\pgfqpoint{3.949228in}{2.628878in}}%
\pgfpathlineto{\pgfqpoint{3.938085in}{2.625457in}}%
\pgfpathlineto{\pgfqpoint{3.926937in}{2.622033in}}%
\pgfpathlineto{\pgfqpoint{3.933338in}{2.608057in}}%
\pgfpathlineto{\pgfqpoint{3.939740in}{2.594070in}}%
\pgfpathlineto{\pgfqpoint{3.946145in}{2.580074in}}%
\pgfpathlineto{\pgfqpoint{3.952553in}{2.566067in}}%
\pgfpathclose%
\pgfusepath{stroke,fill}%
\end{pgfscope}%
\begin{pgfscope}%
\pgfpathrectangle{\pgfqpoint{0.887500in}{0.275000in}}{\pgfqpoint{4.225000in}{4.225000in}}%
\pgfusepath{clip}%
\pgfsetbuttcap%
\pgfsetroundjoin%
\definecolor{currentfill}{rgb}{0.144759,0.519093,0.556572}%
\pgfsetfillcolor{currentfill}%
\pgfsetfillopacity{0.700000}%
\pgfsetlinewidth{0.501875pt}%
\definecolor{currentstroke}{rgb}{1.000000,1.000000,1.000000}%
\pgfsetstrokecolor{currentstroke}%
\pgfsetstrokeopacity{0.500000}%
\pgfsetdash{}{0pt}%
\pgfpathmoveto{\pgfqpoint{4.254313in}{2.315978in}}%
\pgfpathlineto{\pgfqpoint{4.265383in}{2.319201in}}%
\pgfpathlineto{\pgfqpoint{4.276448in}{2.322466in}}%
\pgfpathlineto{\pgfqpoint{4.287511in}{2.325787in}}%
\pgfpathlineto{\pgfqpoint{4.298569in}{2.329178in}}%
\pgfpathlineto{\pgfqpoint{4.309625in}{2.332637in}}%
\pgfpathlineto{\pgfqpoint{4.303184in}{2.347544in}}%
\pgfpathlineto{\pgfqpoint{4.296742in}{2.362369in}}%
\pgfpathlineto{\pgfqpoint{4.290299in}{2.377088in}}%
\pgfpathlineto{\pgfqpoint{4.283856in}{2.391707in}}%
\pgfpathlineto{\pgfqpoint{4.277413in}{2.406236in}}%
\pgfpathlineto{\pgfqpoint{4.266363in}{2.402939in}}%
\pgfpathlineto{\pgfqpoint{4.255307in}{2.399621in}}%
\pgfpathlineto{\pgfqpoint{4.244246in}{2.396280in}}%
\pgfpathlineto{\pgfqpoint{4.233178in}{2.392920in}}%
\pgfpathlineto{\pgfqpoint{4.222105in}{2.389547in}}%
\pgfpathlineto{\pgfqpoint{4.228547in}{2.375023in}}%
\pgfpathlineto{\pgfqpoint{4.234989in}{2.360409in}}%
\pgfpathlineto{\pgfqpoint{4.241431in}{2.345695in}}%
\pgfpathlineto{\pgfqpoint{4.247872in}{2.330877in}}%
\pgfpathclose%
\pgfusepath{stroke,fill}%
\end{pgfscope}%
\begin{pgfscope}%
\pgfpathrectangle{\pgfqpoint{0.887500in}{0.275000in}}{\pgfqpoint{4.225000in}{4.225000in}}%
\pgfusepath{clip}%
\pgfsetbuttcap%
\pgfsetroundjoin%
\definecolor{currentfill}{rgb}{0.129933,0.559582,0.551864}%
\pgfsetfillcolor{currentfill}%
\pgfsetfillopacity{0.700000}%
\pgfsetlinewidth{0.501875pt}%
\definecolor{currentstroke}{rgb}{1.000000,1.000000,1.000000}%
\pgfsetstrokecolor{currentstroke}%
\pgfsetstrokeopacity{0.500000}%
\pgfsetdash{}{0pt}%
\pgfpathmoveto{\pgfqpoint{1.577051in}{2.412503in}}%
\pgfpathlineto{\pgfqpoint{1.588792in}{2.415911in}}%
\pgfpathlineto{\pgfqpoint{1.600526in}{2.419307in}}%
\pgfpathlineto{\pgfqpoint{1.612256in}{2.422691in}}%
\pgfpathlineto{\pgfqpoint{1.623980in}{2.426065in}}%
\pgfpathlineto{\pgfqpoint{1.635698in}{2.429429in}}%
\pgfpathlineto{\pgfqpoint{1.629947in}{2.437663in}}%
\pgfpathlineto{\pgfqpoint{1.624200in}{2.445877in}}%
\pgfpathlineto{\pgfqpoint{1.618458in}{2.454071in}}%
\pgfpathlineto{\pgfqpoint{1.612719in}{2.462247in}}%
\pgfpathlineto{\pgfqpoint{1.606986in}{2.470407in}}%
\pgfpathlineto{\pgfqpoint{1.595280in}{2.467058in}}%
\pgfpathlineto{\pgfqpoint{1.583569in}{2.463699in}}%
\pgfpathlineto{\pgfqpoint{1.571852in}{2.460330in}}%
\pgfpathlineto{\pgfqpoint{1.560130in}{2.456949in}}%
\pgfpathlineto{\pgfqpoint{1.548403in}{2.453555in}}%
\pgfpathlineto{\pgfqpoint{1.554124in}{2.445382in}}%
\pgfpathlineto{\pgfqpoint{1.559849in}{2.437192in}}%
\pgfpathlineto{\pgfqpoint{1.565579in}{2.428982in}}%
\pgfpathlineto{\pgfqpoint{1.571313in}{2.420753in}}%
\pgfpathclose%
\pgfusepath{stroke,fill}%
\end{pgfscope}%
\begin{pgfscope}%
\pgfpathrectangle{\pgfqpoint{0.887500in}{0.275000in}}{\pgfqpoint{4.225000in}{4.225000in}}%
\pgfusepath{clip}%
\pgfsetbuttcap%
\pgfsetroundjoin%
\definecolor{currentfill}{rgb}{0.188923,0.410910,0.556326}%
\pgfsetfillcolor{currentfill}%
\pgfsetfillopacity{0.700000}%
\pgfsetlinewidth{0.501875pt}%
\definecolor{currentstroke}{rgb}{1.000000,1.000000,1.000000}%
\pgfsetstrokecolor{currentstroke}%
\pgfsetstrokeopacity{0.500000}%
\pgfsetdash{}{0pt}%
\pgfpathmoveto{\pgfqpoint{4.549819in}{2.084357in}}%
\pgfpathlineto{\pgfqpoint{4.560818in}{2.087688in}}%
\pgfpathlineto{\pgfqpoint{4.571811in}{2.091016in}}%
\pgfpathlineto{\pgfqpoint{4.582798in}{2.094345in}}%
\pgfpathlineto{\pgfqpoint{4.593781in}{2.097673in}}%
\pgfpathlineto{\pgfqpoint{4.604757in}{2.101003in}}%
\pgfpathlineto{\pgfqpoint{4.598250in}{2.115548in}}%
\pgfpathlineto{\pgfqpoint{4.591745in}{2.130103in}}%
\pgfpathlineto{\pgfqpoint{4.585243in}{2.144660in}}%
\pgfpathlineto{\pgfqpoint{4.578743in}{2.159211in}}%
\pgfpathlineto{\pgfqpoint{4.572245in}{2.173746in}}%
\pgfpathlineto{\pgfqpoint{4.561266in}{2.170332in}}%
\pgfpathlineto{\pgfqpoint{4.550281in}{2.166894in}}%
\pgfpathlineto{\pgfqpoint{4.539289in}{2.163426in}}%
\pgfpathlineto{\pgfqpoint{4.528290in}{2.159925in}}%
\pgfpathlineto{\pgfqpoint{4.517285in}{2.156386in}}%
\pgfpathlineto{\pgfqpoint{4.523788in}{2.142023in}}%
\pgfpathlineto{\pgfqpoint{4.530293in}{2.127636in}}%
\pgfpathlineto{\pgfqpoint{4.536800in}{2.113228in}}%
\pgfpathlineto{\pgfqpoint{4.543308in}{2.098801in}}%
\pgfpathclose%
\pgfusepath{stroke,fill}%
\end{pgfscope}%
\begin{pgfscope}%
\pgfpathrectangle{\pgfqpoint{0.887500in}{0.275000in}}{\pgfqpoint{4.225000in}{4.225000in}}%
\pgfusepath{clip}%
\pgfsetbuttcap%
\pgfsetroundjoin%
\definecolor{currentfill}{rgb}{0.319809,0.770914,0.411152}%
\pgfsetfillcolor{currentfill}%
\pgfsetfillopacity{0.700000}%
\pgfsetlinewidth{0.501875pt}%
\definecolor{currentstroke}{rgb}{1.000000,1.000000,1.000000}%
\pgfsetstrokecolor{currentstroke}%
\pgfsetstrokeopacity{0.500000}%
\pgfsetdash{}{0pt}%
\pgfpathmoveto{\pgfqpoint{3.487508in}{2.876130in}}%
\pgfpathlineto{\pgfqpoint{3.498779in}{2.879899in}}%
\pgfpathlineto{\pgfqpoint{3.510045in}{2.883624in}}%
\pgfpathlineto{\pgfqpoint{3.521305in}{2.887306in}}%
\pgfpathlineto{\pgfqpoint{3.532559in}{2.890946in}}%
\pgfpathlineto{\pgfqpoint{3.543808in}{2.894545in}}%
\pgfpathlineto{\pgfqpoint{3.537477in}{2.907878in}}%
\pgfpathlineto{\pgfqpoint{3.531149in}{2.921172in}}%
\pgfpathlineto{\pgfqpoint{3.524822in}{2.934422in}}%
\pgfpathlineto{\pgfqpoint{3.518498in}{2.947625in}}%
\pgfpathlineto{\pgfqpoint{3.512176in}{2.960777in}}%
\pgfpathlineto{\pgfqpoint{3.500931in}{2.957162in}}%
\pgfpathlineto{\pgfqpoint{3.489681in}{2.953505in}}%
\pgfpathlineto{\pgfqpoint{3.478425in}{2.949809in}}%
\pgfpathlineto{\pgfqpoint{3.467163in}{2.946071in}}%
\pgfpathlineto{\pgfqpoint{3.455895in}{2.942289in}}%
\pgfpathlineto{\pgfqpoint{3.462213in}{2.929158in}}%
\pgfpathlineto{\pgfqpoint{3.468534in}{2.915973in}}%
\pgfpathlineto{\pgfqpoint{3.474856in}{2.902738in}}%
\pgfpathlineto{\pgfqpoint{3.481181in}{2.889455in}}%
\pgfpathclose%
\pgfusepath{stroke,fill}%
\end{pgfscope}%
\begin{pgfscope}%
\pgfpathrectangle{\pgfqpoint{0.887500in}{0.275000in}}{\pgfqpoint{4.225000in}{4.225000in}}%
\pgfusepath{clip}%
\pgfsetbuttcap%
\pgfsetroundjoin%
\definecolor{currentfill}{rgb}{0.440137,0.811138,0.340967}%
\pgfsetfillcolor{currentfill}%
\pgfsetfillopacity{0.700000}%
\pgfsetlinewidth{0.501875pt}%
\definecolor{currentstroke}{rgb}{1.000000,1.000000,1.000000}%
\pgfsetstrokecolor{currentstroke}%
\pgfsetstrokeopacity{0.500000}%
\pgfsetdash{}{0pt}%
\pgfpathmoveto{\pgfqpoint{3.078343in}{2.965803in}}%
\pgfpathlineto{\pgfqpoint{3.089716in}{2.982121in}}%
\pgfpathlineto{\pgfqpoint{3.101089in}{2.995591in}}%
\pgfpathlineto{\pgfqpoint{3.112458in}{3.006760in}}%
\pgfpathlineto{\pgfqpoint{3.123823in}{3.016174in}}%
\pgfpathlineto{\pgfqpoint{3.135184in}{3.024383in}}%
\pgfpathlineto{\pgfqpoint{3.128931in}{3.036079in}}%
\pgfpathlineto{\pgfqpoint{3.122680in}{3.047304in}}%
\pgfpathlineto{\pgfqpoint{3.116431in}{3.058071in}}%
\pgfpathlineto{\pgfqpoint{3.110185in}{3.068389in}}%
\pgfpathlineto{\pgfqpoint{3.103942in}{3.078270in}}%
\pgfpathlineto{\pgfqpoint{3.092592in}{3.068918in}}%
\pgfpathlineto{\pgfqpoint{3.081239in}{3.058438in}}%
\pgfpathlineto{\pgfqpoint{3.069884in}{3.046395in}}%
\pgfpathlineto{\pgfqpoint{3.058528in}{3.032353in}}%
\pgfpathlineto{\pgfqpoint{3.047173in}{3.015875in}}%
\pgfpathlineto{\pgfqpoint{3.053399in}{3.008158in}}%
\pgfpathlineto{\pgfqpoint{3.059630in}{2.999216in}}%
\pgfpathlineto{\pgfqpoint{3.065865in}{2.989126in}}%
\pgfpathlineto{\pgfqpoint{3.072102in}{2.977962in}}%
\pgfpathclose%
\pgfusepath{stroke,fill}%
\end{pgfscope}%
\begin{pgfscope}%
\pgfpathrectangle{\pgfqpoint{0.887500in}{0.275000in}}{\pgfqpoint{4.225000in}{4.225000in}}%
\pgfusepath{clip}%
\pgfsetbuttcap%
\pgfsetroundjoin%
\definecolor{currentfill}{rgb}{0.162142,0.474838,0.558140}%
\pgfsetfillcolor{currentfill}%
\pgfsetfillopacity{0.700000}%
\pgfsetlinewidth{0.501875pt}%
\definecolor{currentstroke}{rgb}{1.000000,1.000000,1.000000}%
\pgfsetstrokecolor{currentstroke}%
\pgfsetstrokeopacity{0.500000}%
\pgfsetdash{}{0pt}%
\pgfpathmoveto{\pgfqpoint{2.540605in}{2.229431in}}%
\pgfpathlineto{\pgfqpoint{2.552108in}{2.232954in}}%
\pgfpathlineto{\pgfqpoint{2.563604in}{2.236508in}}%
\pgfpathlineto{\pgfqpoint{2.575095in}{2.240088in}}%
\pgfpathlineto{\pgfqpoint{2.586580in}{2.243667in}}%
\pgfpathlineto{\pgfqpoint{2.598060in}{2.247214in}}%
\pgfpathlineto{\pgfqpoint{2.591976in}{2.256098in}}%
\pgfpathlineto{\pgfqpoint{2.585897in}{2.264958in}}%
\pgfpathlineto{\pgfqpoint{2.579822in}{2.273793in}}%
\pgfpathlineto{\pgfqpoint{2.573751in}{2.282605in}}%
\pgfpathlineto{\pgfqpoint{2.567684in}{2.291393in}}%
\pgfpathlineto{\pgfqpoint{2.556215in}{2.287845in}}%
\pgfpathlineto{\pgfqpoint{2.544742in}{2.284267in}}%
\pgfpathlineto{\pgfqpoint{2.533262in}{2.280691in}}%
\pgfpathlineto{\pgfqpoint{2.521777in}{2.277147in}}%
\pgfpathlineto{\pgfqpoint{2.510286in}{2.273641in}}%
\pgfpathlineto{\pgfqpoint{2.516341in}{2.264849in}}%
\pgfpathlineto{\pgfqpoint{2.522401in}{2.256033in}}%
\pgfpathlineto{\pgfqpoint{2.528465in}{2.247192in}}%
\pgfpathlineto{\pgfqpoint{2.534533in}{2.238325in}}%
\pgfpathclose%
\pgfusepath{stroke,fill}%
\end{pgfscope}%
\begin{pgfscope}%
\pgfpathrectangle{\pgfqpoint{0.887500in}{0.275000in}}{\pgfqpoint{4.225000in}{4.225000in}}%
\pgfusepath{clip}%
\pgfsetbuttcap%
\pgfsetroundjoin%
\definecolor{currentfill}{rgb}{0.140536,0.530132,0.555659}%
\pgfsetfillcolor{currentfill}%
\pgfsetfillopacity{0.700000}%
\pgfsetlinewidth{0.501875pt}%
\definecolor{currentstroke}{rgb}{1.000000,1.000000,1.000000}%
\pgfsetstrokecolor{currentstroke}%
\pgfsetstrokeopacity{0.500000}%
\pgfsetdash{}{0pt}%
\pgfpathmoveto{\pgfqpoint{1.898146in}{2.354266in}}%
\pgfpathlineto{\pgfqpoint{1.909809in}{2.357650in}}%
\pgfpathlineto{\pgfqpoint{1.921467in}{2.361024in}}%
\pgfpathlineto{\pgfqpoint{1.933120in}{2.364387in}}%
\pgfpathlineto{\pgfqpoint{1.944766in}{2.367740in}}%
\pgfpathlineto{\pgfqpoint{1.956407in}{2.371084in}}%
\pgfpathlineto{\pgfqpoint{1.950542in}{2.379544in}}%
\pgfpathlineto{\pgfqpoint{1.944680in}{2.387984in}}%
\pgfpathlineto{\pgfqpoint{1.938823in}{2.396404in}}%
\pgfpathlineto{\pgfqpoint{1.932970in}{2.404803in}}%
\pgfpathlineto{\pgfqpoint{1.927121in}{2.413183in}}%
\pgfpathlineto{\pgfqpoint{1.915492in}{2.409868in}}%
\pgfpathlineto{\pgfqpoint{1.903857in}{2.406545in}}%
\pgfpathlineto{\pgfqpoint{1.892216in}{2.403213in}}%
\pgfpathlineto{\pgfqpoint{1.880570in}{2.399871in}}%
\pgfpathlineto{\pgfqpoint{1.868919in}{2.396519in}}%
\pgfpathlineto{\pgfqpoint{1.874756in}{2.388109in}}%
\pgfpathlineto{\pgfqpoint{1.880597in}{2.379678in}}%
\pgfpathlineto{\pgfqpoint{1.886442in}{2.371228in}}%
\pgfpathlineto{\pgfqpoint{1.892292in}{2.362757in}}%
\pgfpathclose%
\pgfusepath{stroke,fill}%
\end{pgfscope}%
\begin{pgfscope}%
\pgfpathrectangle{\pgfqpoint{0.887500in}{0.275000in}}{\pgfqpoint{4.225000in}{4.225000in}}%
\pgfusepath{clip}%
\pgfsetbuttcap%
\pgfsetroundjoin%
\definecolor{currentfill}{rgb}{0.468053,0.818921,0.323998}%
\pgfsetfillcolor{currentfill}%
\pgfsetfillopacity{0.700000}%
\pgfsetlinewidth{0.501875pt}%
\definecolor{currentstroke}{rgb}{1.000000,1.000000,1.000000}%
\pgfsetstrokecolor{currentstroke}%
\pgfsetstrokeopacity{0.500000}%
\pgfsetdash{}{0pt}%
\pgfpathmoveto{\pgfqpoint{3.223285in}{3.000530in}}%
\pgfpathlineto{\pgfqpoint{3.234635in}{3.008236in}}%
\pgfpathlineto{\pgfqpoint{3.245980in}{3.015477in}}%
\pgfpathlineto{\pgfqpoint{3.257319in}{3.022143in}}%
\pgfpathlineto{\pgfqpoint{3.268650in}{3.028125in}}%
\pgfpathlineto{\pgfqpoint{3.279974in}{3.033423in}}%
\pgfpathlineto{\pgfqpoint{3.273693in}{3.046126in}}%
\pgfpathlineto{\pgfqpoint{3.267413in}{3.058615in}}%
\pgfpathlineto{\pgfqpoint{3.261135in}{3.070918in}}%
\pgfpathlineto{\pgfqpoint{3.254859in}{3.083061in}}%
\pgfpathlineto{\pgfqpoint{3.248586in}{3.095072in}}%
\pgfpathlineto{\pgfqpoint{3.237268in}{3.089596in}}%
\pgfpathlineto{\pgfqpoint{3.225944in}{3.083517in}}%
\pgfpathlineto{\pgfqpoint{3.214613in}{3.076855in}}%
\pgfpathlineto{\pgfqpoint{3.203277in}{3.069739in}}%
\pgfpathlineto{\pgfqpoint{3.191936in}{3.062301in}}%
\pgfpathlineto{\pgfqpoint{3.198202in}{3.050415in}}%
\pgfpathlineto{\pgfqpoint{3.204470in}{3.038333in}}%
\pgfpathlineto{\pgfqpoint{3.210740in}{3.026016in}}%
\pgfpathlineto{\pgfqpoint{3.217011in}{3.013428in}}%
\pgfpathclose%
\pgfusepath{stroke,fill}%
\end{pgfscope}%
\begin{pgfscope}%
\pgfpathrectangle{\pgfqpoint{0.887500in}{0.275000in}}{\pgfqpoint{4.225000in}{4.225000in}}%
\pgfusepath{clip}%
\pgfsetbuttcap%
\pgfsetroundjoin%
\definecolor{currentfill}{rgb}{0.150476,0.504369,0.557430}%
\pgfsetfillcolor{currentfill}%
\pgfsetfillopacity{0.700000}%
\pgfsetlinewidth{0.501875pt}%
\definecolor{currentstroke}{rgb}{1.000000,1.000000,1.000000}%
\pgfsetstrokecolor{currentstroke}%
\pgfsetstrokeopacity{0.500000}%
\pgfsetdash{}{0pt}%
\pgfpathmoveto{\pgfqpoint{2.219356in}{2.293123in}}%
\pgfpathlineto{\pgfqpoint{2.230939in}{2.296574in}}%
\pgfpathlineto{\pgfqpoint{2.242516in}{2.300014in}}%
\pgfpathlineto{\pgfqpoint{2.254088in}{2.303442in}}%
\pgfpathlineto{\pgfqpoint{2.265655in}{2.306856in}}%
\pgfpathlineto{\pgfqpoint{2.277216in}{2.310255in}}%
\pgfpathlineto{\pgfqpoint{2.271240in}{2.318894in}}%
\pgfpathlineto{\pgfqpoint{2.265269in}{2.327512in}}%
\pgfpathlineto{\pgfqpoint{2.259301in}{2.336109in}}%
\pgfpathlineto{\pgfqpoint{2.253338in}{2.344687in}}%
\pgfpathlineto{\pgfqpoint{2.247379in}{2.353246in}}%
\pgfpathlineto{\pgfqpoint{2.235830in}{2.349878in}}%
\pgfpathlineto{\pgfqpoint{2.224274in}{2.346496in}}%
\pgfpathlineto{\pgfqpoint{2.212714in}{2.343102in}}%
\pgfpathlineto{\pgfqpoint{2.201148in}{2.339697in}}%
\pgfpathlineto{\pgfqpoint{2.189576in}{2.336281in}}%
\pgfpathlineto{\pgfqpoint{2.195524in}{2.327690in}}%
\pgfpathlineto{\pgfqpoint{2.201475in}{2.319079in}}%
\pgfpathlineto{\pgfqpoint{2.207431in}{2.310448in}}%
\pgfpathlineto{\pgfqpoint{2.213391in}{2.301797in}}%
\pgfpathclose%
\pgfusepath{stroke,fill}%
\end{pgfscope}%
\begin{pgfscope}%
\pgfpathrectangle{\pgfqpoint{0.887500in}{0.275000in}}{\pgfqpoint{4.225000in}{4.225000in}}%
\pgfusepath{clip}%
\pgfsetbuttcap%
\pgfsetroundjoin%
\definecolor{currentfill}{rgb}{0.369214,0.788888,0.382914}%
\pgfsetfillcolor{currentfill}%
\pgfsetfillopacity{0.700000}%
\pgfsetlinewidth{0.501875pt}%
\definecolor{currentstroke}{rgb}{1.000000,1.000000,1.000000}%
\pgfsetstrokecolor{currentstroke}%
\pgfsetstrokeopacity{0.500000}%
\pgfsetdash{}{0pt}%
\pgfpathmoveto{\pgfqpoint{3.399472in}{2.922569in}}%
\pgfpathlineto{\pgfqpoint{3.410768in}{2.926634in}}%
\pgfpathlineto{\pgfqpoint{3.422058in}{2.930634in}}%
\pgfpathlineto{\pgfqpoint{3.433343in}{2.934574in}}%
\pgfpathlineto{\pgfqpoint{3.444622in}{2.938458in}}%
\pgfpathlineto{\pgfqpoint{3.455895in}{2.942289in}}%
\pgfpathlineto{\pgfqpoint{3.449579in}{2.955368in}}%
\pgfpathlineto{\pgfqpoint{3.443266in}{2.968401in}}%
\pgfpathlineto{\pgfqpoint{3.436955in}{2.981394in}}%
\pgfpathlineto{\pgfqpoint{3.430646in}{2.994353in}}%
\pgfpathlineto{\pgfqpoint{3.424339in}{3.007282in}}%
\pgfpathlineto{\pgfqpoint{3.413072in}{3.003544in}}%
\pgfpathlineto{\pgfqpoint{3.401798in}{2.999800in}}%
\pgfpathlineto{\pgfqpoint{3.390520in}{2.996050in}}%
\pgfpathlineto{\pgfqpoint{3.379236in}{2.992295in}}%
\pgfpathlineto{\pgfqpoint{3.367947in}{2.988537in}}%
\pgfpathlineto{\pgfqpoint{3.374247in}{2.975382in}}%
\pgfpathlineto{\pgfqpoint{3.380549in}{2.962200in}}%
\pgfpathlineto{\pgfqpoint{3.386854in}{2.948997in}}%
\pgfpathlineto{\pgfqpoint{3.393162in}{2.935784in}}%
\pgfpathclose%
\pgfusepath{stroke,fill}%
\end{pgfscope}%
\begin{pgfscope}%
\pgfpathrectangle{\pgfqpoint{0.887500in}{0.275000in}}{\pgfqpoint{4.225000in}{4.225000in}}%
\pgfusepath{clip}%
\pgfsetbuttcap%
\pgfsetroundjoin%
\definecolor{currentfill}{rgb}{0.182256,0.426184,0.557120}%
\pgfsetfillcolor{currentfill}%
\pgfsetfillopacity{0.700000}%
\pgfsetlinewidth{0.501875pt}%
\definecolor{currentstroke}{rgb}{1.000000,1.000000,1.000000}%
\pgfsetstrokecolor{currentstroke}%
\pgfsetstrokeopacity{0.500000}%
\pgfsetdash{}{0pt}%
\pgfpathmoveto{\pgfqpoint{2.949919in}{2.122322in}}%
\pgfpathlineto{\pgfqpoint{2.961327in}{2.123421in}}%
\pgfpathlineto{\pgfqpoint{2.972723in}{2.126606in}}%
\pgfpathlineto{\pgfqpoint{2.984108in}{2.132828in}}%
\pgfpathlineto{\pgfqpoint{2.995482in}{2.143036in}}%
\pgfpathlineto{\pgfqpoint{3.006849in}{2.158026in}}%
\pgfpathlineto{\pgfqpoint{3.000630in}{2.167483in}}%
\pgfpathlineto{\pgfqpoint{2.994416in}{2.176921in}}%
\pgfpathlineto{\pgfqpoint{2.988205in}{2.186344in}}%
\pgfpathlineto{\pgfqpoint{2.981998in}{2.195753in}}%
\pgfpathlineto{\pgfqpoint{2.975795in}{2.205153in}}%
\pgfpathlineto{\pgfqpoint{2.964450in}{2.189028in}}%
\pgfpathlineto{\pgfqpoint{2.953093in}{2.178178in}}%
\pgfpathlineto{\pgfqpoint{2.941721in}{2.171727in}}%
\pgfpathlineto{\pgfqpoint{2.930335in}{2.168625in}}%
\pgfpathlineto{\pgfqpoint{2.918935in}{2.167826in}}%
\pgfpathlineto{\pgfqpoint{2.925124in}{2.158764in}}%
\pgfpathlineto{\pgfqpoint{2.931317in}{2.149698in}}%
\pgfpathlineto{\pgfqpoint{2.937514in}{2.140611in}}%
\pgfpathlineto{\pgfqpoint{2.943714in}{2.131491in}}%
\pgfpathclose%
\pgfusepath{stroke,fill}%
\end{pgfscope}%
\begin{pgfscope}%
\pgfpathrectangle{\pgfqpoint{0.887500in}{0.275000in}}{\pgfqpoint{4.225000in}{4.225000in}}%
\pgfusepath{clip}%
\pgfsetbuttcap%
\pgfsetroundjoin%
\definecolor{currentfill}{rgb}{0.132268,0.655014,0.519661}%
\pgfsetfillcolor{currentfill}%
\pgfsetfillopacity{0.700000}%
\pgfsetlinewidth{0.501875pt}%
\definecolor{currentstroke}{rgb}{1.000000,1.000000,1.000000}%
\pgfsetstrokecolor{currentstroke}%
\pgfsetstrokeopacity{0.500000}%
\pgfsetdash{}{0pt}%
\pgfpathmoveto{\pgfqpoint{3.871117in}{2.604966in}}%
\pgfpathlineto{\pgfqpoint{3.882292in}{2.608371in}}%
\pgfpathlineto{\pgfqpoint{3.893461in}{2.611778in}}%
\pgfpathlineto{\pgfqpoint{3.904625in}{2.615191in}}%
\pgfpathlineto{\pgfqpoint{3.915784in}{2.618610in}}%
\pgfpathlineto{\pgfqpoint{3.926937in}{2.622033in}}%
\pgfpathlineto{\pgfqpoint{3.920539in}{2.635995in}}%
\pgfpathlineto{\pgfqpoint{3.914143in}{2.649941in}}%
\pgfpathlineto{\pgfqpoint{3.907750in}{2.663868in}}%
\pgfpathlineto{\pgfqpoint{3.901358in}{2.677773in}}%
\pgfpathlineto{\pgfqpoint{3.894968in}{2.691654in}}%
\pgfpathlineto{\pgfqpoint{3.883817in}{2.688188in}}%
\pgfpathlineto{\pgfqpoint{3.872660in}{2.684733in}}%
\pgfpathlineto{\pgfqpoint{3.861498in}{2.681293in}}%
\pgfpathlineto{\pgfqpoint{3.850331in}{2.677868in}}%
\pgfpathlineto{\pgfqpoint{3.839158in}{2.674456in}}%
\pgfpathlineto{\pgfqpoint{3.845546in}{2.660594in}}%
\pgfpathlineto{\pgfqpoint{3.851935in}{2.646713in}}%
\pgfpathlineto{\pgfqpoint{3.858327in}{2.632814in}}%
\pgfpathlineto{\pgfqpoint{3.864721in}{2.618898in}}%
\pgfpathclose%
\pgfusepath{stroke,fill}%
\end{pgfscope}%
\begin{pgfscope}%
\pgfpathrectangle{\pgfqpoint{0.887500in}{0.275000in}}{\pgfqpoint{4.225000in}{4.225000in}}%
\pgfusepath{clip}%
\pgfsetbuttcap%
\pgfsetroundjoin%
\definecolor{currentfill}{rgb}{0.421908,0.805774,0.351910}%
\pgfsetfillcolor{currentfill}%
\pgfsetfillopacity{0.700000}%
\pgfsetlinewidth{0.501875pt}%
\definecolor{currentstroke}{rgb}{1.000000,1.000000,1.000000}%
\pgfsetstrokecolor{currentstroke}%
\pgfsetstrokeopacity{0.500000}%
\pgfsetdash{}{0pt}%
\pgfpathmoveto{\pgfqpoint{3.311405in}{2.966944in}}%
\pgfpathlineto{\pgfqpoint{3.322728in}{2.971991in}}%
\pgfpathlineto{\pgfqpoint{3.334043in}{2.976567in}}%
\pgfpathlineto{\pgfqpoint{3.345351in}{2.980778in}}%
\pgfpathlineto{\pgfqpoint{3.356652in}{2.984733in}}%
\pgfpathlineto{\pgfqpoint{3.367947in}{2.988537in}}%
\pgfpathlineto{\pgfqpoint{3.361650in}{3.001656in}}%
\pgfpathlineto{\pgfqpoint{3.355354in}{3.014729in}}%
\pgfpathlineto{\pgfqpoint{3.349062in}{3.027748in}}%
\pgfpathlineto{\pgfqpoint{3.342771in}{3.040705in}}%
\pgfpathlineto{\pgfqpoint{3.336483in}{3.053592in}}%
\pgfpathlineto{\pgfqpoint{3.325194in}{3.050019in}}%
\pgfpathlineto{\pgfqpoint{3.313900in}{3.046330in}}%
\pgfpathlineto{\pgfqpoint{3.302599in}{3.042411in}}%
\pgfpathlineto{\pgfqpoint{3.291290in}{3.038147in}}%
\pgfpathlineto{\pgfqpoint{3.279974in}{3.033423in}}%
\pgfpathlineto{\pgfqpoint{3.286257in}{3.020485in}}%
\pgfpathlineto{\pgfqpoint{3.292541in}{3.007328in}}%
\pgfpathlineto{\pgfqpoint{3.298827in}{2.993993in}}%
\pgfpathlineto{\pgfqpoint{3.305115in}{2.980519in}}%
\pgfpathclose%
\pgfusepath{stroke,fill}%
\end{pgfscope}%
\begin{pgfscope}%
\pgfpathrectangle{\pgfqpoint{0.887500in}{0.275000in}}{\pgfqpoint{4.225000in}{4.225000in}}%
\pgfusepath{clip}%
\pgfsetbuttcap%
\pgfsetroundjoin%
\definecolor{currentfill}{rgb}{0.133743,0.548535,0.553541}%
\pgfsetfillcolor{currentfill}%
\pgfsetfillopacity{0.700000}%
\pgfsetlinewidth{0.501875pt}%
\definecolor{currentstroke}{rgb}{1.000000,1.000000,1.000000}%
\pgfsetstrokecolor{currentstroke}%
\pgfsetstrokeopacity{0.500000}%
\pgfsetdash{}{0pt}%
\pgfpathmoveto{\pgfqpoint{4.166654in}{2.372631in}}%
\pgfpathlineto{\pgfqpoint{4.177755in}{2.376009in}}%
\pgfpathlineto{\pgfqpoint{4.188850in}{2.379391in}}%
\pgfpathlineto{\pgfqpoint{4.199941in}{2.382778in}}%
\pgfpathlineto{\pgfqpoint{4.211025in}{2.386165in}}%
\pgfpathlineto{\pgfqpoint{4.222105in}{2.389547in}}%
\pgfpathlineto{\pgfqpoint{4.215662in}{2.403991in}}%
\pgfpathlineto{\pgfqpoint{4.209220in}{2.418365in}}%
\pgfpathlineto{\pgfqpoint{4.202779in}{2.432677in}}%
\pgfpathlineto{\pgfqpoint{4.196339in}{2.446938in}}%
\pgfpathlineto{\pgfqpoint{4.189900in}{2.461157in}}%
\pgfpathlineto{\pgfqpoint{4.178823in}{2.457788in}}%
\pgfpathlineto{\pgfqpoint{4.167740in}{2.454418in}}%
\pgfpathlineto{\pgfqpoint{4.156652in}{2.451050in}}%
\pgfpathlineto{\pgfqpoint{4.145559in}{2.447685in}}%
\pgfpathlineto{\pgfqpoint{4.134460in}{2.444324in}}%
\pgfpathlineto{\pgfqpoint{4.140896in}{2.430054in}}%
\pgfpathlineto{\pgfqpoint{4.147333in}{2.415751in}}%
\pgfpathlineto{\pgfqpoint{4.153772in}{2.401414in}}%
\pgfpathlineto{\pgfqpoint{4.160212in}{2.387041in}}%
\pgfpathclose%
\pgfusepath{stroke,fill}%
\end{pgfscope}%
\begin{pgfscope}%
\pgfpathrectangle{\pgfqpoint{0.887500in}{0.275000in}}{\pgfqpoint{4.225000in}{4.225000in}}%
\pgfusepath{clip}%
\pgfsetbuttcap%
\pgfsetroundjoin%
\definecolor{currentfill}{rgb}{0.177423,0.437527,0.557565}%
\pgfsetfillcolor{currentfill}%
\pgfsetfillopacity{0.700000}%
\pgfsetlinewidth{0.501875pt}%
\definecolor{currentstroke}{rgb}{1.000000,1.000000,1.000000}%
\pgfsetstrokecolor{currentstroke}%
\pgfsetstrokeopacity{0.500000}%
\pgfsetdash{}{0pt}%
\pgfpathmoveto{\pgfqpoint{4.462153in}{2.137959in}}%
\pgfpathlineto{\pgfqpoint{4.473194in}{2.141756in}}%
\pgfpathlineto{\pgfqpoint{4.484227in}{2.145493in}}%
\pgfpathlineto{\pgfqpoint{4.495254in}{2.149174in}}%
\pgfpathlineto{\pgfqpoint{4.506273in}{2.152804in}}%
\pgfpathlineto{\pgfqpoint{4.517285in}{2.156386in}}%
\pgfpathlineto{\pgfqpoint{4.510783in}{2.170721in}}%
\pgfpathlineto{\pgfqpoint{4.504283in}{2.185028in}}%
\pgfpathlineto{\pgfqpoint{4.497785in}{2.199302in}}%
\pgfpathlineto{\pgfqpoint{4.491287in}{2.213542in}}%
\pgfpathlineto{\pgfqpoint{4.484791in}{2.227744in}}%
\pgfpathlineto{\pgfqpoint{4.473772in}{2.223918in}}%
\pgfpathlineto{\pgfqpoint{4.462744in}{2.220006in}}%
\pgfpathlineto{\pgfqpoint{4.451708in}{2.216001in}}%
\pgfpathlineto{\pgfqpoint{4.440664in}{2.211896in}}%
\pgfpathlineto{\pgfqpoint{4.429611in}{2.207682in}}%
\pgfpathlineto{\pgfqpoint{4.436112in}{2.193694in}}%
\pgfpathlineto{\pgfqpoint{4.442617in}{2.179742in}}%
\pgfpathlineto{\pgfqpoint{4.449126in}{2.165813in}}%
\pgfpathlineto{\pgfqpoint{4.455638in}{2.151890in}}%
\pgfpathclose%
\pgfusepath{stroke,fill}%
\end{pgfscope}%
\begin{pgfscope}%
\pgfpathrectangle{\pgfqpoint{0.887500in}{0.275000in}}{\pgfqpoint{4.225000in}{4.225000in}}%
\pgfusepath{clip}%
\pgfsetbuttcap%
\pgfsetroundjoin%
\definecolor{currentfill}{rgb}{0.133743,0.548535,0.553541}%
\pgfsetfillcolor{currentfill}%
\pgfsetfillopacity{0.700000}%
\pgfsetlinewidth{0.501875pt}%
\definecolor{currentstroke}{rgb}{1.000000,1.000000,1.000000}%
\pgfsetstrokecolor{currentstroke}%
\pgfsetstrokeopacity{0.500000}%
\pgfsetdash{}{0pt}%
\pgfpathmoveto{\pgfqpoint{1.664522in}{2.387916in}}%
\pgfpathlineto{\pgfqpoint{1.676248in}{2.391293in}}%
\pgfpathlineto{\pgfqpoint{1.687967in}{2.394662in}}%
\pgfpathlineto{\pgfqpoint{1.699682in}{2.398027in}}%
\pgfpathlineto{\pgfqpoint{1.711390in}{2.401386in}}%
\pgfpathlineto{\pgfqpoint{1.723093in}{2.404743in}}%
\pgfpathlineto{\pgfqpoint{1.717306in}{2.413072in}}%
\pgfpathlineto{\pgfqpoint{1.711525in}{2.421377in}}%
\pgfpathlineto{\pgfqpoint{1.705747in}{2.429659in}}%
\pgfpathlineto{\pgfqpoint{1.699975in}{2.437918in}}%
\pgfpathlineto{\pgfqpoint{1.694206in}{2.446155in}}%
\pgfpathlineto{\pgfqpoint{1.682516in}{2.442819in}}%
\pgfpathlineto{\pgfqpoint{1.670820in}{2.439480in}}%
\pgfpathlineto{\pgfqpoint{1.659119in}{2.436136in}}%
\pgfpathlineto{\pgfqpoint{1.647411in}{2.432786in}}%
\pgfpathlineto{\pgfqpoint{1.635698in}{2.429429in}}%
\pgfpathlineto{\pgfqpoint{1.641454in}{2.421174in}}%
\pgfpathlineto{\pgfqpoint{1.647214in}{2.412896in}}%
\pgfpathlineto{\pgfqpoint{1.652979in}{2.404594in}}%
\pgfpathlineto{\pgfqpoint{1.658748in}{2.396268in}}%
\pgfpathclose%
\pgfusepath{stroke,fill}%
\end{pgfscope}%
\begin{pgfscope}%
\pgfpathrectangle{\pgfqpoint{0.887500in}{0.275000in}}{\pgfqpoint{4.225000in}{4.225000in}}%
\pgfusepath{clip}%
\pgfsetbuttcap%
\pgfsetroundjoin%
\definecolor{currentfill}{rgb}{0.153894,0.680203,0.504172}%
\pgfsetfillcolor{currentfill}%
\pgfsetfillopacity{0.700000}%
\pgfsetlinewidth{0.501875pt}%
\definecolor{currentstroke}{rgb}{1.000000,1.000000,1.000000}%
\pgfsetstrokecolor{currentstroke}%
\pgfsetstrokeopacity{0.500000}%
\pgfsetdash{}{0pt}%
\pgfpathmoveto{\pgfqpoint{3.783217in}{2.657502in}}%
\pgfpathlineto{\pgfqpoint{3.794416in}{2.660887in}}%
\pgfpathlineto{\pgfqpoint{3.805610in}{2.664272in}}%
\pgfpathlineto{\pgfqpoint{3.816798in}{2.667660in}}%
\pgfpathlineto{\pgfqpoint{3.827981in}{2.671054in}}%
\pgfpathlineto{\pgfqpoint{3.839158in}{2.674456in}}%
\pgfpathlineto{\pgfqpoint{3.832773in}{2.688297in}}%
\pgfpathlineto{\pgfqpoint{3.826390in}{2.702115in}}%
\pgfpathlineto{\pgfqpoint{3.820009in}{2.715910in}}%
\pgfpathlineto{\pgfqpoint{3.813631in}{2.729681in}}%
\pgfpathlineto{\pgfqpoint{3.807254in}{2.743426in}}%
\pgfpathlineto{\pgfqpoint{3.796080in}{2.740044in}}%
\pgfpathlineto{\pgfqpoint{3.784900in}{2.736676in}}%
\pgfpathlineto{\pgfqpoint{3.773715in}{2.733319in}}%
\pgfpathlineto{\pgfqpoint{3.762525in}{2.729971in}}%
\pgfpathlineto{\pgfqpoint{3.751330in}{2.726628in}}%
\pgfpathlineto{\pgfqpoint{3.757703in}{2.712863in}}%
\pgfpathlineto{\pgfqpoint{3.764079in}{2.699066in}}%
\pgfpathlineto{\pgfqpoint{3.770456in}{2.685238in}}%
\pgfpathlineto{\pgfqpoint{3.776836in}{2.671382in}}%
\pgfpathclose%
\pgfusepath{stroke,fill}%
\end{pgfscope}%
\begin{pgfscope}%
\pgfpathrectangle{\pgfqpoint{0.887500in}{0.275000in}}{\pgfqpoint{4.225000in}{4.225000in}}%
\pgfusepath{clip}%
\pgfsetbuttcap%
\pgfsetroundjoin%
\definecolor{currentfill}{rgb}{0.166617,0.463708,0.558119}%
\pgfsetfillcolor{currentfill}%
\pgfsetfillopacity{0.700000}%
\pgfsetlinewidth{0.501875pt}%
\definecolor{currentstroke}{rgb}{1.000000,1.000000,1.000000}%
\pgfsetstrokecolor{currentstroke}%
\pgfsetstrokeopacity{0.500000}%
\pgfsetdash{}{0pt}%
\pgfpathmoveto{\pgfqpoint{2.628540in}{2.202403in}}%
\pgfpathlineto{\pgfqpoint{2.640026in}{2.205904in}}%
\pgfpathlineto{\pgfqpoint{2.651507in}{2.209317in}}%
\pgfpathlineto{\pgfqpoint{2.662984in}{2.212614in}}%
\pgfpathlineto{\pgfqpoint{2.674456in}{2.215768in}}%
\pgfpathlineto{\pgfqpoint{2.685924in}{2.218800in}}%
\pgfpathlineto{\pgfqpoint{2.679808in}{2.227803in}}%
\pgfpathlineto{\pgfqpoint{2.673698in}{2.236777in}}%
\pgfpathlineto{\pgfqpoint{2.667591in}{2.245724in}}%
\pgfpathlineto{\pgfqpoint{2.661488in}{2.254644in}}%
\pgfpathlineto{\pgfqpoint{2.655390in}{2.263536in}}%
\pgfpathlineto{\pgfqpoint{2.643932in}{2.260523in}}%
\pgfpathlineto{\pgfqpoint{2.632471in}{2.257385in}}%
\pgfpathlineto{\pgfqpoint{2.621005in}{2.254102in}}%
\pgfpathlineto{\pgfqpoint{2.609535in}{2.250703in}}%
\pgfpathlineto{\pgfqpoint{2.598060in}{2.247214in}}%
\pgfpathlineto{\pgfqpoint{2.604148in}{2.238305in}}%
\pgfpathlineto{\pgfqpoint{2.610239in}{2.229370in}}%
\pgfpathlineto{\pgfqpoint{2.616335in}{2.220409in}}%
\pgfpathlineto{\pgfqpoint{2.622436in}{2.211420in}}%
\pgfpathclose%
\pgfusepath{stroke,fill}%
\end{pgfscope}%
\begin{pgfscope}%
\pgfpathrectangle{\pgfqpoint{0.887500in}{0.275000in}}{\pgfqpoint{4.225000in}{4.225000in}}%
\pgfusepath{clip}%
\pgfsetbuttcap%
\pgfsetroundjoin%
\definecolor{currentfill}{rgb}{0.166617,0.463708,0.558119}%
\pgfsetfillcolor{currentfill}%
\pgfsetfillopacity{0.700000}%
\pgfsetlinewidth{0.501875pt}%
\definecolor{currentstroke}{rgb}{1.000000,1.000000,1.000000}%
\pgfsetstrokecolor{currentstroke}%
\pgfsetstrokeopacity{0.500000}%
\pgfsetdash{}{0pt}%
\pgfpathmoveto{\pgfqpoint{4.374262in}{2.186382in}}%
\pgfpathlineto{\pgfqpoint{4.385336in}{2.190473in}}%
\pgfpathlineto{\pgfqpoint{4.396410in}{2.194710in}}%
\pgfpathlineto{\pgfqpoint{4.407481in}{2.199030in}}%
\pgfpathlineto{\pgfqpoint{4.418549in}{2.203375in}}%
\pgfpathlineto{\pgfqpoint{4.429611in}{2.207682in}}%
\pgfpathlineto{\pgfqpoint{4.423114in}{2.221723in}}%
\pgfpathlineto{\pgfqpoint{4.416622in}{2.235831in}}%
\pgfpathlineto{\pgfqpoint{4.410135in}{2.250007in}}%
\pgfpathlineto{\pgfqpoint{4.403652in}{2.264245in}}%
\pgfpathlineto{\pgfqpoint{4.397175in}{2.278535in}}%
\pgfpathlineto{\pgfqpoint{4.386121in}{2.274382in}}%
\pgfpathlineto{\pgfqpoint{4.375061in}{2.270206in}}%
\pgfpathlineto{\pgfqpoint{4.363998in}{2.266057in}}%
\pgfpathlineto{\pgfqpoint{4.352932in}{2.261986in}}%
\pgfpathlineto{\pgfqpoint{4.341865in}{2.258044in}}%
\pgfpathlineto{\pgfqpoint{4.348327in}{2.243347in}}%
\pgfpathlineto{\pgfqpoint{4.354797in}{2.228803in}}%
\pgfpathlineto{\pgfqpoint{4.361275in}{2.214444in}}%
\pgfpathlineto{\pgfqpoint{4.367764in}{2.200303in}}%
\pgfpathclose%
\pgfusepath{stroke,fill}%
\end{pgfscope}%
\begin{pgfscope}%
\pgfpathrectangle{\pgfqpoint{0.887500in}{0.275000in}}{\pgfqpoint{4.225000in}{4.225000in}}%
\pgfusepath{clip}%
\pgfsetbuttcap%
\pgfsetroundjoin%
\definecolor{currentfill}{rgb}{0.143343,0.522773,0.556295}%
\pgfsetfillcolor{currentfill}%
\pgfsetfillopacity{0.700000}%
\pgfsetlinewidth{0.501875pt}%
\definecolor{currentstroke}{rgb}{1.000000,1.000000,1.000000}%
\pgfsetstrokecolor{currentstroke}%
\pgfsetstrokeopacity{0.500000}%
\pgfsetdash{}{0pt}%
\pgfpathmoveto{\pgfqpoint{1.985802in}{2.328465in}}%
\pgfpathlineto{\pgfqpoint{1.997449in}{2.331829in}}%
\pgfpathlineto{\pgfqpoint{2.009090in}{2.335185in}}%
\pgfpathlineto{\pgfqpoint{2.020727in}{2.338533in}}%
\pgfpathlineto{\pgfqpoint{2.032357in}{2.341877in}}%
\pgfpathlineto{\pgfqpoint{2.043982in}{2.345219in}}%
\pgfpathlineto{\pgfqpoint{2.038082in}{2.353756in}}%
\pgfpathlineto{\pgfqpoint{2.032188in}{2.362273in}}%
\pgfpathlineto{\pgfqpoint{2.026297in}{2.370769in}}%
\pgfpathlineto{\pgfqpoint{2.020411in}{2.379244in}}%
\pgfpathlineto{\pgfqpoint{2.014529in}{2.387699in}}%
\pgfpathlineto{\pgfqpoint{2.002916in}{2.384385in}}%
\pgfpathlineto{\pgfqpoint{1.991297in}{2.381068in}}%
\pgfpathlineto{\pgfqpoint{1.979673in}{2.377747in}}%
\pgfpathlineto{\pgfqpoint{1.968043in}{2.374419in}}%
\pgfpathlineto{\pgfqpoint{1.956407in}{2.371084in}}%
\pgfpathlineto{\pgfqpoint{1.962278in}{2.362603in}}%
\pgfpathlineto{\pgfqpoint{1.968152in}{2.354101in}}%
\pgfpathlineto{\pgfqpoint{1.974031in}{2.345577in}}%
\pgfpathlineto{\pgfqpoint{1.979914in}{2.337032in}}%
\pgfpathclose%
\pgfusepath{stroke,fill}%
\end{pgfscope}%
\begin{pgfscope}%
\pgfpathrectangle{\pgfqpoint{0.887500in}{0.275000in}}{\pgfqpoint{4.225000in}{4.225000in}}%
\pgfusepath{clip}%
\pgfsetbuttcap%
\pgfsetroundjoin%
\definecolor{currentfill}{rgb}{0.154815,0.493313,0.557840}%
\pgfsetfillcolor{currentfill}%
\pgfsetfillopacity{0.700000}%
\pgfsetlinewidth{0.501875pt}%
\definecolor{currentstroke}{rgb}{1.000000,1.000000,1.000000}%
\pgfsetstrokecolor{currentstroke}%
\pgfsetstrokeopacity{0.500000}%
\pgfsetdash{}{0pt}%
\pgfpathmoveto{\pgfqpoint{2.307158in}{2.266730in}}%
\pgfpathlineto{\pgfqpoint{2.318725in}{2.270148in}}%
\pgfpathlineto{\pgfqpoint{2.330286in}{2.273549in}}%
\pgfpathlineto{\pgfqpoint{2.341842in}{2.276930in}}%
\pgfpathlineto{\pgfqpoint{2.353393in}{2.280291in}}%
\pgfpathlineto{\pgfqpoint{2.364938in}{2.283637in}}%
\pgfpathlineto{\pgfqpoint{2.358930in}{2.292353in}}%
\pgfpathlineto{\pgfqpoint{2.352926in}{2.301047in}}%
\pgfpathlineto{\pgfqpoint{2.346926in}{2.309720in}}%
\pgfpathlineto{\pgfqpoint{2.340931in}{2.318372in}}%
\pgfpathlineto{\pgfqpoint{2.334940in}{2.327003in}}%
\pgfpathlineto{\pgfqpoint{2.323406in}{2.323686in}}%
\pgfpathlineto{\pgfqpoint{2.311866in}{2.320355in}}%
\pgfpathlineto{\pgfqpoint{2.300322in}{2.317006in}}%
\pgfpathlineto{\pgfqpoint{2.288771in}{2.313639in}}%
\pgfpathlineto{\pgfqpoint{2.277216in}{2.310255in}}%
\pgfpathlineto{\pgfqpoint{2.283196in}{2.301595in}}%
\pgfpathlineto{\pgfqpoint{2.289180in}{2.292913in}}%
\pgfpathlineto{\pgfqpoint{2.295168in}{2.284209in}}%
\pgfpathlineto{\pgfqpoint{2.301161in}{2.275481in}}%
\pgfpathclose%
\pgfusepath{stroke,fill}%
\end{pgfscope}%
\begin{pgfscope}%
\pgfpathrectangle{\pgfqpoint{0.887500in}{0.275000in}}{\pgfqpoint{4.225000in}{4.225000in}}%
\pgfusepath{clip}%
\pgfsetbuttcap%
\pgfsetroundjoin%
\definecolor{currentfill}{rgb}{0.281477,0.755203,0.432552}%
\pgfsetfillcolor{currentfill}%
\pgfsetfillopacity{0.700000}%
\pgfsetlinewidth{0.501875pt}%
\definecolor{currentstroke}{rgb}{1.000000,1.000000,1.000000}%
\pgfsetstrokecolor{currentstroke}%
\pgfsetstrokeopacity{0.500000}%
\pgfsetdash{}{0pt}%
\pgfpathmoveto{\pgfqpoint{3.052674in}{2.766339in}}%
\pgfpathlineto{\pgfqpoint{3.064041in}{2.794825in}}%
\pgfpathlineto{\pgfqpoint{3.075415in}{2.822330in}}%
\pgfpathlineto{\pgfqpoint{3.086797in}{2.848233in}}%
\pgfpathlineto{\pgfqpoint{3.098184in}{2.871912in}}%
\pgfpathlineto{\pgfqpoint{3.109573in}{2.892743in}}%
\pgfpathlineto{\pgfqpoint{3.103324in}{2.908735in}}%
\pgfpathlineto{\pgfqpoint{3.097076in}{2.924113in}}%
\pgfpathlineto{\pgfqpoint{3.090830in}{2.938802in}}%
\pgfpathlineto{\pgfqpoint{3.084585in}{2.952724in}}%
\pgfpathlineto{\pgfqpoint{3.078343in}{2.965803in}}%
\pgfpathlineto{\pgfqpoint{3.066970in}{2.946189in}}%
\pgfpathlineto{\pgfqpoint{3.055601in}{2.923552in}}%
\pgfpathlineto{\pgfqpoint{3.044239in}{2.898494in}}%
\pgfpathlineto{\pgfqpoint{3.032884in}{2.871616in}}%
\pgfpathlineto{\pgfqpoint{3.021538in}{2.843518in}}%
\pgfpathlineto{\pgfqpoint{3.027761in}{2.828841in}}%
\pgfpathlineto{\pgfqpoint{3.033986in}{2.813529in}}%
\pgfpathlineto{\pgfqpoint{3.040214in}{2.797839in}}%
\pgfpathlineto{\pgfqpoint{3.046443in}{2.782024in}}%
\pgfpathclose%
\pgfusepath{stroke,fill}%
\end{pgfscope}%
\begin{pgfscope}%
\pgfpathrectangle{\pgfqpoint{0.887500in}{0.275000in}}{\pgfqpoint{4.225000in}{4.225000in}}%
\pgfusepath{clip}%
\pgfsetbuttcap%
\pgfsetroundjoin%
\definecolor{currentfill}{rgb}{0.165117,0.467423,0.558141}%
\pgfsetfillcolor{currentfill}%
\pgfsetfillopacity{0.700000}%
\pgfsetlinewidth{0.501875pt}%
\definecolor{currentstroke}{rgb}{1.000000,1.000000,1.000000}%
\pgfsetstrokecolor{currentstroke}%
\pgfsetstrokeopacity{0.500000}%
\pgfsetdash{}{0pt}%
\pgfpathmoveto{\pgfqpoint{3.006849in}{2.158026in}}%
\pgfpathlineto{\pgfqpoint{3.018212in}{2.177392in}}%
\pgfpathlineto{\pgfqpoint{3.029577in}{2.200181in}}%
\pgfpathlineto{\pgfqpoint{3.040946in}{2.225437in}}%
\pgfpathlineto{\pgfqpoint{3.052322in}{2.252201in}}%
\pgfpathlineto{\pgfqpoint{3.063706in}{2.279511in}}%
\pgfpathlineto{\pgfqpoint{3.057461in}{2.290801in}}%
\pgfpathlineto{\pgfqpoint{3.051219in}{2.302193in}}%
\pgfpathlineto{\pgfqpoint{3.044981in}{2.313615in}}%
\pgfpathlineto{\pgfqpoint{3.038746in}{2.324996in}}%
\pgfpathlineto{\pgfqpoint{3.032514in}{2.336263in}}%
\pgfpathlineto{\pgfqpoint{3.021158in}{2.306925in}}%
\pgfpathlineto{\pgfqpoint{3.009810in}{2.278056in}}%
\pgfpathlineto{\pgfqpoint{2.998470in}{2.250748in}}%
\pgfpathlineto{\pgfqpoint{2.987133in}{2.226085in}}%
\pgfpathlineto{\pgfqpoint{2.975795in}{2.205153in}}%
\pgfpathlineto{\pgfqpoint{2.981998in}{2.195753in}}%
\pgfpathlineto{\pgfqpoint{2.988205in}{2.186344in}}%
\pgfpathlineto{\pgfqpoint{2.994416in}{2.176921in}}%
\pgfpathlineto{\pgfqpoint{3.000630in}{2.167483in}}%
\pgfpathclose%
\pgfusepath{stroke,fill}%
\end{pgfscope}%
\begin{pgfscope}%
\pgfpathrectangle{\pgfqpoint{0.887500in}{0.275000in}}{\pgfqpoint{4.225000in}{4.225000in}}%
\pgfusepath{clip}%
\pgfsetbuttcap%
\pgfsetroundjoin%
\definecolor{currentfill}{rgb}{0.125394,0.574318,0.549086}%
\pgfsetfillcolor{currentfill}%
\pgfsetfillopacity{0.700000}%
\pgfsetlinewidth{0.501875pt}%
\definecolor{currentstroke}{rgb}{1.000000,1.000000,1.000000}%
\pgfsetstrokecolor{currentstroke}%
\pgfsetstrokeopacity{0.500000}%
\pgfsetdash{}{0pt}%
\pgfpathmoveto{\pgfqpoint{4.078883in}{2.427436in}}%
\pgfpathlineto{\pgfqpoint{4.090010in}{2.430838in}}%
\pgfpathlineto{\pgfqpoint{4.101131in}{2.434223in}}%
\pgfpathlineto{\pgfqpoint{4.112247in}{2.437597in}}%
\pgfpathlineto{\pgfqpoint{4.123356in}{2.440963in}}%
\pgfpathlineto{\pgfqpoint{4.134460in}{2.444324in}}%
\pgfpathlineto{\pgfqpoint{4.128026in}{2.458562in}}%
\pgfpathlineto{\pgfqpoint{4.121594in}{2.472770in}}%
\pgfpathlineto{\pgfqpoint{4.115164in}{2.486949in}}%
\pgfpathlineto{\pgfqpoint{4.108735in}{2.501101in}}%
\pgfpathlineto{\pgfqpoint{4.102308in}{2.515230in}}%
\pgfpathlineto{\pgfqpoint{4.091207in}{2.511900in}}%
\pgfpathlineto{\pgfqpoint{4.080100in}{2.508568in}}%
\pgfpathlineto{\pgfqpoint{4.068988in}{2.505231in}}%
\pgfpathlineto{\pgfqpoint{4.057870in}{2.501885in}}%
\pgfpathlineto{\pgfqpoint{4.046746in}{2.498529in}}%
\pgfpathlineto{\pgfqpoint{4.053170in}{2.484378in}}%
\pgfpathlineto{\pgfqpoint{4.059596in}{2.470195in}}%
\pgfpathlineto{\pgfqpoint{4.066023in}{2.455977in}}%
\pgfpathlineto{\pgfqpoint{4.072453in}{2.441723in}}%
\pgfpathclose%
\pgfusepath{stroke,fill}%
\end{pgfscope}%
\begin{pgfscope}%
\pgfpathrectangle{\pgfqpoint{0.887500in}{0.275000in}}{\pgfqpoint{4.225000in}{4.225000in}}%
\pgfusepath{clip}%
\pgfsetbuttcap%
\pgfsetroundjoin%
\definecolor{currentfill}{rgb}{0.185783,0.704891,0.485273}%
\pgfsetfillcolor{currentfill}%
\pgfsetfillopacity{0.700000}%
\pgfsetlinewidth{0.501875pt}%
\definecolor{currentstroke}{rgb}{1.000000,1.000000,1.000000}%
\pgfsetstrokecolor{currentstroke}%
\pgfsetstrokeopacity{0.500000}%
\pgfsetdash{}{0pt}%
\pgfpathmoveto{\pgfqpoint{3.695266in}{2.709620in}}%
\pgfpathlineto{\pgfqpoint{3.706491in}{2.713099in}}%
\pgfpathlineto{\pgfqpoint{3.717709in}{2.716528in}}%
\pgfpathlineto{\pgfqpoint{3.728922in}{2.719918in}}%
\pgfpathlineto{\pgfqpoint{3.740129in}{2.723281in}}%
\pgfpathlineto{\pgfqpoint{3.751330in}{2.726628in}}%
\pgfpathlineto{\pgfqpoint{3.744959in}{2.740358in}}%
\pgfpathlineto{\pgfqpoint{3.738589in}{2.754054in}}%
\pgfpathlineto{\pgfqpoint{3.732222in}{2.767714in}}%
\pgfpathlineto{\pgfqpoint{3.725856in}{2.781338in}}%
\pgfpathlineto{\pgfqpoint{3.719493in}{2.794923in}}%
\pgfpathlineto{\pgfqpoint{3.708295in}{2.791551in}}%
\pgfpathlineto{\pgfqpoint{3.697091in}{2.788164in}}%
\pgfpathlineto{\pgfqpoint{3.685881in}{2.784754in}}%
\pgfpathlineto{\pgfqpoint{3.674665in}{2.781314in}}%
\pgfpathlineto{\pgfqpoint{3.663443in}{2.777835in}}%
\pgfpathlineto{\pgfqpoint{3.669804in}{2.764275in}}%
\pgfpathlineto{\pgfqpoint{3.676166in}{2.750676in}}%
\pgfpathlineto{\pgfqpoint{3.682531in}{2.737036in}}%
\pgfpathlineto{\pgfqpoint{3.688897in}{2.723352in}}%
\pgfpathclose%
\pgfusepath{stroke,fill}%
\end{pgfscope}%
\begin{pgfscope}%
\pgfpathrectangle{\pgfqpoint{0.887500in}{0.275000in}}{\pgfqpoint{4.225000in}{4.225000in}}%
\pgfusepath{clip}%
\pgfsetbuttcap%
\pgfsetroundjoin%
\definecolor{currentfill}{rgb}{0.218130,0.347432,0.550038}%
\pgfsetfillcolor{currentfill}%
\pgfsetfillopacity{0.700000}%
\pgfsetlinewidth{0.501875pt}%
\definecolor{currentstroke}{rgb}{1.000000,1.000000,1.000000}%
\pgfsetstrokecolor{currentstroke}%
\pgfsetstrokeopacity{0.500000}%
\pgfsetdash{}{0pt}%
\pgfpathmoveto{\pgfqpoint{4.669976in}{1.955586in}}%
\pgfpathlineto{\pgfqpoint{4.680949in}{1.958917in}}%
\pgfpathlineto{\pgfqpoint{4.691917in}{1.962256in}}%
\pgfpathlineto{\pgfqpoint{4.702880in}{1.965604in}}%
\pgfpathlineto{\pgfqpoint{4.713838in}{1.968962in}}%
\pgfpathlineto{\pgfqpoint{4.707305in}{1.983534in}}%
\pgfpathlineto{\pgfqpoint{4.700774in}{1.998089in}}%
\pgfpathlineto{\pgfqpoint{4.694246in}{2.012632in}}%
\pgfpathlineto{\pgfqpoint{4.687719in}{2.027164in}}%
\pgfpathlineto{\pgfqpoint{4.681194in}{2.041691in}}%
\pgfpathlineto{\pgfqpoint{4.670238in}{2.038343in}}%
\pgfpathlineto{\pgfqpoint{4.659276in}{2.035005in}}%
\pgfpathlineto{\pgfqpoint{4.648309in}{2.031676in}}%
\pgfpathlineto{\pgfqpoint{4.637337in}{2.028357in}}%
\pgfpathlineto{\pgfqpoint{4.643860in}{2.013824in}}%
\pgfpathlineto{\pgfqpoint{4.650386in}{1.999283in}}%
\pgfpathlineto{\pgfqpoint{4.656914in}{1.984732in}}%
\pgfpathlineto{\pgfqpoint{4.663444in}{1.970167in}}%
\pgfpathclose%
\pgfusepath{stroke,fill}%
\end{pgfscope}%
\begin{pgfscope}%
\pgfpathrectangle{\pgfqpoint{0.887500in}{0.275000in}}{\pgfqpoint{4.225000in}{4.225000in}}%
\pgfusepath{clip}%
\pgfsetbuttcap%
\pgfsetroundjoin%
\definecolor{currentfill}{rgb}{0.171176,0.452530,0.557965}%
\pgfsetfillcolor{currentfill}%
\pgfsetfillopacity{0.700000}%
\pgfsetlinewidth{0.501875pt}%
\definecolor{currentstroke}{rgb}{1.000000,1.000000,1.000000}%
\pgfsetstrokecolor{currentstroke}%
\pgfsetstrokeopacity{0.500000}%
\pgfsetdash{}{0pt}%
\pgfpathmoveto{\pgfqpoint{2.716561in}{2.173355in}}%
\pgfpathlineto{\pgfqpoint{2.728033in}{2.176368in}}%
\pgfpathlineto{\pgfqpoint{2.739499in}{2.179434in}}%
\pgfpathlineto{\pgfqpoint{2.750958in}{2.182644in}}%
\pgfpathlineto{\pgfqpoint{2.762410in}{2.186088in}}%
\pgfpathlineto{\pgfqpoint{2.773853in}{2.189855in}}%
\pgfpathlineto{\pgfqpoint{2.767707in}{2.198979in}}%
\pgfpathlineto{\pgfqpoint{2.761565in}{2.208075in}}%
\pgfpathlineto{\pgfqpoint{2.755427in}{2.217143in}}%
\pgfpathlineto{\pgfqpoint{2.749293in}{2.226184in}}%
\pgfpathlineto{\pgfqpoint{2.743163in}{2.235199in}}%
\pgfpathlineto{\pgfqpoint{2.731731in}{2.231455in}}%
\pgfpathlineto{\pgfqpoint{2.720290in}{2.228033in}}%
\pgfpathlineto{\pgfqpoint{2.708841in}{2.224843in}}%
\pgfpathlineto{\pgfqpoint{2.697386in}{2.221795in}}%
\pgfpathlineto{\pgfqpoint{2.685924in}{2.218800in}}%
\pgfpathlineto{\pgfqpoint{2.692043in}{2.209769in}}%
\pgfpathlineto{\pgfqpoint{2.698166in}{2.200709in}}%
\pgfpathlineto{\pgfqpoint{2.704294in}{2.191620in}}%
\pgfpathlineto{\pgfqpoint{2.710425in}{2.182503in}}%
\pgfpathclose%
\pgfusepath{stroke,fill}%
\end{pgfscope}%
\begin{pgfscope}%
\pgfpathrectangle{\pgfqpoint{0.887500in}{0.275000in}}{\pgfqpoint{4.225000in}{4.225000in}}%
\pgfusepath{clip}%
\pgfsetbuttcap%
\pgfsetroundjoin%
\definecolor{currentfill}{rgb}{0.156270,0.489624,0.557936}%
\pgfsetfillcolor{currentfill}%
\pgfsetfillopacity{0.700000}%
\pgfsetlinewidth{0.501875pt}%
\definecolor{currentstroke}{rgb}{1.000000,1.000000,1.000000}%
\pgfsetstrokecolor{currentstroke}%
\pgfsetstrokeopacity{0.500000}%
\pgfsetdash{}{0pt}%
\pgfpathmoveto{\pgfqpoint{4.286549in}{2.241364in}}%
\pgfpathlineto{\pgfqpoint{4.297611in}{2.244335in}}%
\pgfpathlineto{\pgfqpoint{4.308673in}{2.247446in}}%
\pgfpathlineto{\pgfqpoint{4.319735in}{2.250745in}}%
\pgfpathlineto{\pgfqpoint{4.330799in}{2.254280in}}%
\pgfpathlineto{\pgfqpoint{4.341865in}{2.258044in}}%
\pgfpathlineto{\pgfqpoint{4.335409in}{2.272859in}}%
\pgfpathlineto{\pgfqpoint{4.328958in}{2.287759in}}%
\pgfpathlineto{\pgfqpoint{4.322511in}{2.302711in}}%
\pgfpathlineto{\pgfqpoint{4.316067in}{2.317682in}}%
\pgfpathlineto{\pgfqpoint{4.309625in}{2.332637in}}%
\pgfpathlineto{\pgfqpoint{4.298569in}{2.329178in}}%
\pgfpathlineto{\pgfqpoint{4.287511in}{2.325787in}}%
\pgfpathlineto{\pgfqpoint{4.276448in}{2.322466in}}%
\pgfpathlineto{\pgfqpoint{4.265383in}{2.319201in}}%
\pgfpathlineto{\pgfqpoint{4.254313in}{2.315978in}}%
\pgfpathlineto{\pgfqpoint{4.260754in}{2.301029in}}%
\pgfpathlineto{\pgfqpoint{4.267198in}{2.286061in}}%
\pgfpathlineto{\pgfqpoint{4.273644in}{2.271106in}}%
\pgfpathlineto{\pgfqpoint{4.280094in}{2.256197in}}%
\pgfpathclose%
\pgfusepath{stroke,fill}%
\end{pgfscope}%
\begin{pgfscope}%
\pgfpathrectangle{\pgfqpoint{0.887500in}{0.275000in}}{\pgfqpoint{4.225000in}{4.225000in}}%
\pgfusepath{clip}%
\pgfsetbuttcap%
\pgfsetroundjoin%
\definecolor{currentfill}{rgb}{0.430983,0.808473,0.346476}%
\pgfsetfillcolor{currentfill}%
\pgfsetfillopacity{0.700000}%
\pgfsetlinewidth{0.501875pt}%
\definecolor{currentstroke}{rgb}{1.000000,1.000000,1.000000}%
\pgfsetstrokecolor{currentstroke}%
\pgfsetstrokeopacity{0.500000}%
\pgfsetdash{}{0pt}%
\pgfpathmoveto{\pgfqpoint{3.166478in}{2.958462in}}%
\pgfpathlineto{\pgfqpoint{3.177845in}{2.967253in}}%
\pgfpathlineto{\pgfqpoint{3.189210in}{2.975735in}}%
\pgfpathlineto{\pgfqpoint{3.200571in}{2.984169in}}%
\pgfpathlineto{\pgfqpoint{3.211930in}{2.992471in}}%
\pgfpathlineto{\pgfqpoint{3.223285in}{3.000530in}}%
\pgfpathlineto{\pgfqpoint{3.217011in}{3.013428in}}%
\pgfpathlineto{\pgfqpoint{3.210740in}{3.026016in}}%
\pgfpathlineto{\pgfqpoint{3.204470in}{3.038333in}}%
\pgfpathlineto{\pgfqpoint{3.198202in}{3.050415in}}%
\pgfpathlineto{\pgfqpoint{3.191936in}{3.062301in}}%
\pgfpathlineto{\pgfqpoint{3.180592in}{3.054672in}}%
\pgfpathlineto{\pgfqpoint{3.169244in}{3.046986in}}%
\pgfpathlineto{\pgfqpoint{3.157894in}{3.039373in}}%
\pgfpathlineto{\pgfqpoint{3.146541in}{3.031934in}}%
\pgfpathlineto{\pgfqpoint{3.135184in}{3.024383in}}%
\pgfpathlineto{\pgfqpoint{3.141440in}{3.012205in}}%
\pgfpathlineto{\pgfqpoint{3.147697in}{2.999535in}}%
\pgfpathlineto{\pgfqpoint{3.153956in}{2.986362in}}%
\pgfpathlineto{\pgfqpoint{3.160216in}{2.972675in}}%
\pgfpathclose%
\pgfusepath{stroke,fill}%
\end{pgfscope}%
\begin{pgfscope}%
\pgfpathrectangle{\pgfqpoint{0.887500in}{0.275000in}}{\pgfqpoint{4.225000in}{4.225000in}}%
\pgfusepath{clip}%
\pgfsetbuttcap%
\pgfsetroundjoin%
\definecolor{currentfill}{rgb}{0.226397,0.728888,0.462789}%
\pgfsetfillcolor{currentfill}%
\pgfsetfillopacity{0.700000}%
\pgfsetlinewidth{0.501875pt}%
\definecolor{currentstroke}{rgb}{1.000000,1.000000,1.000000}%
\pgfsetstrokecolor{currentstroke}%
\pgfsetstrokeopacity{0.500000}%
\pgfsetdash{}{0pt}%
\pgfpathmoveto{\pgfqpoint{3.607244in}{2.759788in}}%
\pgfpathlineto{\pgfqpoint{3.618495in}{2.763465in}}%
\pgfpathlineto{\pgfqpoint{3.629741in}{2.767117in}}%
\pgfpathlineto{\pgfqpoint{3.640981in}{2.770735in}}%
\pgfpathlineto{\pgfqpoint{3.652215in}{2.774310in}}%
\pgfpathlineto{\pgfqpoint{3.663443in}{2.777835in}}%
\pgfpathlineto{\pgfqpoint{3.657085in}{2.791359in}}%
\pgfpathlineto{\pgfqpoint{3.650728in}{2.804852in}}%
\pgfpathlineto{\pgfqpoint{3.644374in}{2.818315in}}%
\pgfpathlineto{\pgfqpoint{3.638023in}{2.831752in}}%
\pgfpathlineto{\pgfqpoint{3.631673in}{2.845164in}}%
\pgfpathlineto{\pgfqpoint{3.620450in}{2.841686in}}%
\pgfpathlineto{\pgfqpoint{3.609220in}{2.838169in}}%
\pgfpathlineto{\pgfqpoint{3.597984in}{2.834616in}}%
\pgfpathlineto{\pgfqpoint{3.586743in}{2.831031in}}%
\pgfpathlineto{\pgfqpoint{3.575496in}{2.827416in}}%
\pgfpathlineto{\pgfqpoint{3.581841in}{2.813924in}}%
\pgfpathlineto{\pgfqpoint{3.588188in}{2.800419in}}%
\pgfpathlineto{\pgfqpoint{3.594538in}{2.786898in}}%
\pgfpathlineto{\pgfqpoint{3.600890in}{2.773356in}}%
\pgfpathclose%
\pgfusepath{stroke,fill}%
\end{pgfscope}%
\begin{pgfscope}%
\pgfpathrectangle{\pgfqpoint{0.887500in}{0.275000in}}{\pgfqpoint{4.225000in}{4.225000in}}%
\pgfusepath{clip}%
\pgfsetbuttcap%
\pgfsetroundjoin%
\definecolor{currentfill}{rgb}{0.119738,0.603785,0.541400}%
\pgfsetfillcolor{currentfill}%
\pgfsetfillopacity{0.700000}%
\pgfsetlinewidth{0.501875pt}%
\definecolor{currentstroke}{rgb}{1.000000,1.000000,1.000000}%
\pgfsetstrokecolor{currentstroke}%
\pgfsetstrokeopacity{0.500000}%
\pgfsetdash{}{0pt}%
\pgfpathmoveto{\pgfqpoint{3.991041in}{2.481581in}}%
\pgfpathlineto{\pgfqpoint{4.002193in}{2.484984in}}%
\pgfpathlineto{\pgfqpoint{4.013340in}{2.488383in}}%
\pgfpathlineto{\pgfqpoint{4.024481in}{2.491776in}}%
\pgfpathlineto{\pgfqpoint{4.035616in}{2.495159in}}%
\pgfpathlineto{\pgfqpoint{4.046746in}{2.498529in}}%
\pgfpathlineto{\pgfqpoint{4.040324in}{2.512652in}}%
\pgfpathlineto{\pgfqpoint{4.033903in}{2.526750in}}%
\pgfpathlineto{\pgfqpoint{4.027485in}{2.540828in}}%
\pgfpathlineto{\pgfqpoint{4.021069in}{2.554889in}}%
\pgfpathlineto{\pgfqpoint{4.014655in}{2.568937in}}%
\pgfpathlineto{\pgfqpoint{4.003528in}{2.565576in}}%
\pgfpathlineto{\pgfqpoint{3.992395in}{2.562204in}}%
\pgfpathlineto{\pgfqpoint{3.981256in}{2.558823in}}%
\pgfpathlineto{\pgfqpoint{3.970112in}{2.555436in}}%
\pgfpathlineto{\pgfqpoint{3.958963in}{2.552045in}}%
\pgfpathlineto{\pgfqpoint{3.965374in}{2.538004in}}%
\pgfpathlineto{\pgfqpoint{3.971788in}{2.523941in}}%
\pgfpathlineto{\pgfqpoint{3.978204in}{2.509853in}}%
\pgfpathlineto{\pgfqpoint{3.984622in}{2.495734in}}%
\pgfpathclose%
\pgfusepath{stroke,fill}%
\end{pgfscope}%
\begin{pgfscope}%
\pgfpathrectangle{\pgfqpoint{0.887500in}{0.275000in}}{\pgfqpoint{4.225000in}{4.225000in}}%
\pgfusepath{clip}%
\pgfsetbuttcap%
\pgfsetroundjoin%
\definecolor{currentfill}{rgb}{0.136408,0.541173,0.554483}%
\pgfsetfillcolor{currentfill}%
\pgfsetfillopacity{0.700000}%
\pgfsetlinewidth{0.501875pt}%
\definecolor{currentstroke}{rgb}{1.000000,1.000000,1.000000}%
\pgfsetstrokecolor{currentstroke}%
\pgfsetstrokeopacity{0.500000}%
\pgfsetdash{}{0pt}%
\pgfpathmoveto{\pgfqpoint{1.752091in}{2.362741in}}%
\pgfpathlineto{\pgfqpoint{1.763800in}{2.366123in}}%
\pgfpathlineto{\pgfqpoint{1.775503in}{2.369503in}}%
\pgfpathlineto{\pgfqpoint{1.787200in}{2.372883in}}%
\pgfpathlineto{\pgfqpoint{1.798892in}{2.376265in}}%
\pgfpathlineto{\pgfqpoint{1.810577in}{2.379648in}}%
\pgfpathlineto{\pgfqpoint{1.804757in}{2.388066in}}%
\pgfpathlineto{\pgfqpoint{1.798940in}{2.396463in}}%
\pgfpathlineto{\pgfqpoint{1.793129in}{2.404837in}}%
\pgfpathlineto{\pgfqpoint{1.787321in}{2.413188in}}%
\pgfpathlineto{\pgfqpoint{1.781518in}{2.421518in}}%
\pgfpathlineto{\pgfqpoint{1.769845in}{2.418161in}}%
\pgfpathlineto{\pgfqpoint{1.758166in}{2.414805in}}%
\pgfpathlineto{\pgfqpoint{1.746480in}{2.411451in}}%
\pgfpathlineto{\pgfqpoint{1.734789in}{2.408098in}}%
\pgfpathlineto{\pgfqpoint{1.723093in}{2.404743in}}%
\pgfpathlineto{\pgfqpoint{1.728883in}{2.396390in}}%
\pgfpathlineto{\pgfqpoint{1.734678in}{2.388013in}}%
\pgfpathlineto{\pgfqpoint{1.740478in}{2.379613in}}%
\pgfpathlineto{\pgfqpoint{1.746282in}{2.371189in}}%
\pgfpathclose%
\pgfusepath{stroke,fill}%
\end{pgfscope}%
\begin{pgfscope}%
\pgfpathrectangle{\pgfqpoint{0.887500in}{0.275000in}}{\pgfqpoint{4.225000in}{4.225000in}}%
\pgfusepath{clip}%
\pgfsetbuttcap%
\pgfsetroundjoin%
\definecolor{currentfill}{rgb}{0.159194,0.482237,0.558073}%
\pgfsetfillcolor{currentfill}%
\pgfsetfillopacity{0.700000}%
\pgfsetlinewidth{0.501875pt}%
\definecolor{currentstroke}{rgb}{1.000000,1.000000,1.000000}%
\pgfsetstrokecolor{currentstroke}%
\pgfsetstrokeopacity{0.500000}%
\pgfsetdash{}{0pt}%
\pgfpathmoveto{\pgfqpoint{2.395042in}{2.239709in}}%
\pgfpathlineto{\pgfqpoint{2.406593in}{2.243080in}}%
\pgfpathlineto{\pgfqpoint{2.418138in}{2.246443in}}%
\pgfpathlineto{\pgfqpoint{2.429677in}{2.249803in}}%
\pgfpathlineto{\pgfqpoint{2.441210in}{2.253165in}}%
\pgfpathlineto{\pgfqpoint{2.452738in}{2.256533in}}%
\pgfpathlineto{\pgfqpoint{2.446697in}{2.265330in}}%
\pgfpathlineto{\pgfqpoint{2.440661in}{2.274104in}}%
\pgfpathlineto{\pgfqpoint{2.434630in}{2.282855in}}%
\pgfpathlineto{\pgfqpoint{2.428602in}{2.291584in}}%
\pgfpathlineto{\pgfqpoint{2.422579in}{2.300291in}}%
\pgfpathlineto{\pgfqpoint{2.411062in}{2.296955in}}%
\pgfpathlineto{\pgfqpoint{2.399540in}{2.293626in}}%
\pgfpathlineto{\pgfqpoint{2.388012in}{2.290300in}}%
\pgfpathlineto{\pgfqpoint{2.376478in}{2.286972in}}%
\pgfpathlineto{\pgfqpoint{2.364938in}{2.283637in}}%
\pgfpathlineto{\pgfqpoint{2.370950in}{2.274899in}}%
\pgfpathlineto{\pgfqpoint{2.376967in}{2.266137in}}%
\pgfpathlineto{\pgfqpoint{2.382988in}{2.257353in}}%
\pgfpathlineto{\pgfqpoint{2.389013in}{2.248543in}}%
\pgfpathclose%
\pgfusepath{stroke,fill}%
\end{pgfscope}%
\begin{pgfscope}%
\pgfpathrectangle{\pgfqpoint{0.887500in}{0.275000in}}{\pgfqpoint{4.225000in}{4.225000in}}%
\pgfusepath{clip}%
\pgfsetbuttcap%
\pgfsetroundjoin%
\definecolor{currentfill}{rgb}{0.203063,0.379716,0.553925}%
\pgfsetfillcolor{currentfill}%
\pgfsetfillopacity{0.700000}%
\pgfsetlinewidth{0.501875pt}%
\definecolor{currentstroke}{rgb}{1.000000,1.000000,1.000000}%
\pgfsetstrokecolor{currentstroke}%
\pgfsetstrokeopacity{0.500000}%
\pgfsetdash{}{0pt}%
\pgfpathmoveto{\pgfqpoint{4.582401in}{2.011890in}}%
\pgfpathlineto{\pgfqpoint{4.593398in}{2.015167in}}%
\pgfpathlineto{\pgfqpoint{4.604391in}{2.018452in}}%
\pgfpathlineto{\pgfqpoint{4.615378in}{2.021745in}}%
\pgfpathlineto{\pgfqpoint{4.626360in}{2.025047in}}%
\pgfpathlineto{\pgfqpoint{4.637337in}{2.028357in}}%
\pgfpathlineto{\pgfqpoint{4.630816in}{2.042885in}}%
\pgfpathlineto{\pgfqpoint{4.624297in}{2.057411in}}%
\pgfpathlineto{\pgfqpoint{4.617781in}{2.071937in}}%
\pgfpathlineto{\pgfqpoint{4.611268in}{2.086467in}}%
\pgfpathlineto{\pgfqpoint{4.604757in}{2.101003in}}%
\pgfpathlineto{\pgfqpoint{4.593781in}{2.097673in}}%
\pgfpathlineto{\pgfqpoint{4.582798in}{2.094345in}}%
\pgfpathlineto{\pgfqpoint{4.571811in}{2.091016in}}%
\pgfpathlineto{\pgfqpoint{4.560818in}{2.087688in}}%
\pgfpathlineto{\pgfqpoint{4.549819in}{2.084357in}}%
\pgfpathlineto{\pgfqpoint{4.556331in}{2.069897in}}%
\pgfpathlineto{\pgfqpoint{4.562846in}{2.055420in}}%
\pgfpathlineto{\pgfqpoint{4.569362in}{2.040927in}}%
\pgfpathlineto{\pgfqpoint{4.575881in}{2.026417in}}%
\pgfpathclose%
\pgfusepath{stroke,fill}%
\end{pgfscope}%
\begin{pgfscope}%
\pgfpathrectangle{\pgfqpoint{0.887500in}{0.275000in}}{\pgfqpoint{4.225000in}{4.225000in}}%
\pgfusepath{clip}%
\pgfsetbuttcap%
\pgfsetroundjoin%
\definecolor{currentfill}{rgb}{0.147607,0.511733,0.557049}%
\pgfsetfillcolor{currentfill}%
\pgfsetfillopacity{0.700000}%
\pgfsetlinewidth{0.501875pt}%
\definecolor{currentstroke}{rgb}{1.000000,1.000000,1.000000}%
\pgfsetstrokecolor{currentstroke}%
\pgfsetstrokeopacity{0.500000}%
\pgfsetdash{}{0pt}%
\pgfpathmoveto{\pgfqpoint{2.073541in}{2.302218in}}%
\pgfpathlineto{\pgfqpoint{2.085171in}{2.305595in}}%
\pgfpathlineto{\pgfqpoint{2.096796in}{2.308976in}}%
\pgfpathlineto{\pgfqpoint{2.108414in}{2.312362in}}%
\pgfpathlineto{\pgfqpoint{2.120026in}{2.315757in}}%
\pgfpathlineto{\pgfqpoint{2.131633in}{2.319164in}}%
\pgfpathlineto{\pgfqpoint{2.125701in}{2.327770in}}%
\pgfpathlineto{\pgfqpoint{2.119773in}{2.336356in}}%
\pgfpathlineto{\pgfqpoint{2.113850in}{2.344922in}}%
\pgfpathlineto{\pgfqpoint{2.107931in}{2.353469in}}%
\pgfpathlineto{\pgfqpoint{2.102016in}{2.361996in}}%
\pgfpathlineto{\pgfqpoint{2.090421in}{2.358622in}}%
\pgfpathlineto{\pgfqpoint{2.078820in}{2.355260in}}%
\pgfpathlineto{\pgfqpoint{2.067213in}{2.351908in}}%
\pgfpathlineto{\pgfqpoint{2.055600in}{2.348562in}}%
\pgfpathlineto{\pgfqpoint{2.043982in}{2.345219in}}%
\pgfpathlineto{\pgfqpoint{2.049885in}{2.336661in}}%
\pgfpathlineto{\pgfqpoint{2.055793in}{2.328082in}}%
\pgfpathlineto{\pgfqpoint{2.061705in}{2.319482in}}%
\pgfpathlineto{\pgfqpoint{2.067621in}{2.310861in}}%
\pgfpathclose%
\pgfusepath{stroke,fill}%
\end{pgfscope}%
\begin{pgfscope}%
\pgfpathrectangle{\pgfqpoint{0.887500in}{0.275000in}}{\pgfqpoint{4.225000in}{4.225000in}}%
\pgfusepath{clip}%
\pgfsetbuttcap%
\pgfsetroundjoin%
\definecolor{currentfill}{rgb}{0.129933,0.559582,0.551864}%
\pgfsetfillcolor{currentfill}%
\pgfsetfillopacity{0.700000}%
\pgfsetlinewidth{0.501875pt}%
\definecolor{currentstroke}{rgb}{1.000000,1.000000,1.000000}%
\pgfsetstrokecolor{currentstroke}%
\pgfsetstrokeopacity{0.500000}%
\pgfsetdash{}{0pt}%
\pgfpathmoveto{\pgfqpoint{3.032514in}{2.336263in}}%
\pgfpathlineto{\pgfqpoint{3.043879in}{2.364974in}}%
\pgfpathlineto{\pgfqpoint{3.055254in}{2.392142in}}%
\pgfpathlineto{\pgfqpoint{3.066637in}{2.417753in}}%
\pgfpathlineto{\pgfqpoint{3.078026in}{2.442072in}}%
\pgfpathlineto{\pgfqpoint{3.089420in}{2.465366in}}%
\pgfpathlineto{\pgfqpoint{3.083167in}{2.477349in}}%
\pgfpathlineto{\pgfqpoint{3.076917in}{2.489001in}}%
\pgfpathlineto{\pgfqpoint{3.070670in}{2.500254in}}%
\pgfpathlineto{\pgfqpoint{3.064425in}{2.511038in}}%
\pgfpathlineto{\pgfqpoint{3.058185in}{2.521301in}}%
\pgfpathlineto{\pgfqpoint{3.046813in}{2.497533in}}%
\pgfpathlineto{\pgfqpoint{3.035449in}{2.472591in}}%
\pgfpathlineto{\pgfqpoint{3.024092in}{2.446211in}}%
\pgfpathlineto{\pgfqpoint{3.012745in}{2.418131in}}%
\pgfpathlineto{\pgfqpoint{3.001409in}{2.388379in}}%
\pgfpathlineto{\pgfqpoint{3.007622in}{2.378750in}}%
\pgfpathlineto{\pgfqpoint{3.013839in}{2.368659in}}%
\pgfpathlineto{\pgfqpoint{3.020060in}{2.358166in}}%
\pgfpathlineto{\pgfqpoint{3.026285in}{2.347343in}}%
\pgfpathclose%
\pgfusepath{stroke,fill}%
\end{pgfscope}%
\begin{pgfscope}%
\pgfpathrectangle{\pgfqpoint{0.887500in}{0.275000in}}{\pgfqpoint{4.225000in}{4.225000in}}%
\pgfusepath{clip}%
\pgfsetbuttcap%
\pgfsetroundjoin%
\definecolor{currentfill}{rgb}{0.124780,0.640461,0.527068}%
\pgfsetfillcolor{currentfill}%
\pgfsetfillopacity{0.700000}%
\pgfsetlinewidth{0.501875pt}%
\definecolor{currentstroke}{rgb}{1.000000,1.000000,1.000000}%
\pgfsetstrokecolor{currentstroke}%
\pgfsetstrokeopacity{0.500000}%
\pgfsetdash{}{0pt}%
\pgfpathmoveto{\pgfqpoint{3.058185in}{2.521301in}}%
\pgfpathlineto{\pgfqpoint{3.069562in}{2.544159in}}%
\pgfpathlineto{\pgfqpoint{3.080944in}{2.566373in}}%
\pgfpathlineto{\pgfqpoint{3.092332in}{2.588209in}}%
\pgfpathlineto{\pgfqpoint{3.103724in}{2.609855in}}%
\pgfpathlineto{\pgfqpoint{3.115120in}{2.631240in}}%
\pgfpathlineto{\pgfqpoint{3.108862in}{2.643112in}}%
\pgfpathlineto{\pgfqpoint{3.102607in}{2.655310in}}%
\pgfpathlineto{\pgfqpoint{3.096356in}{2.667848in}}%
\pgfpathlineto{\pgfqpoint{3.090108in}{2.680741in}}%
\pgfpathlineto{\pgfqpoint{3.083862in}{2.694002in}}%
\pgfpathlineto{\pgfqpoint{3.072484in}{2.668792in}}%
\pgfpathlineto{\pgfqpoint{3.061112in}{2.643632in}}%
\pgfpathlineto{\pgfqpoint{3.049747in}{2.618800in}}%
\pgfpathlineto{\pgfqpoint{3.038389in}{2.594127in}}%
\pgfpathlineto{\pgfqpoint{3.027038in}{2.569310in}}%
\pgfpathlineto{\pgfqpoint{3.033260in}{2.559610in}}%
\pgfpathlineto{\pgfqpoint{3.039485in}{2.550121in}}%
\pgfpathlineto{\pgfqpoint{3.045715in}{2.540681in}}%
\pgfpathlineto{\pgfqpoint{3.051948in}{2.531128in}}%
\pgfpathclose%
\pgfusepath{stroke,fill}%
\end{pgfscope}%
\begin{pgfscope}%
\pgfpathrectangle{\pgfqpoint{0.887500in}{0.275000in}}{\pgfqpoint{4.225000in}{4.225000in}}%
\pgfusepath{clip}%
\pgfsetbuttcap%
\pgfsetroundjoin%
\definecolor{currentfill}{rgb}{0.274149,0.751988,0.436601}%
\pgfsetfillcolor{currentfill}%
\pgfsetfillopacity{0.700000}%
\pgfsetlinewidth{0.501875pt}%
\definecolor{currentstroke}{rgb}{1.000000,1.000000,1.000000}%
\pgfsetstrokecolor{currentstroke}%
\pgfsetstrokeopacity{0.500000}%
\pgfsetdash{}{0pt}%
\pgfpathmoveto{\pgfqpoint{3.519176in}{2.808991in}}%
\pgfpathlineto{\pgfqpoint{3.530452in}{2.812722in}}%
\pgfpathlineto{\pgfqpoint{3.541721in}{2.816426in}}%
\pgfpathlineto{\pgfqpoint{3.552985in}{2.820110in}}%
\pgfpathlineto{\pgfqpoint{3.564243in}{2.823774in}}%
\pgfpathlineto{\pgfqpoint{3.575496in}{2.827416in}}%
\pgfpathlineto{\pgfqpoint{3.569154in}{2.840890in}}%
\pgfpathlineto{\pgfqpoint{3.562814in}{2.854344in}}%
\pgfpathlineto{\pgfqpoint{3.556476in}{2.867774in}}%
\pgfpathlineto{\pgfqpoint{3.550141in}{2.881175in}}%
\pgfpathlineto{\pgfqpoint{3.543808in}{2.894545in}}%
\pgfpathlineto{\pgfqpoint{3.532559in}{2.890946in}}%
\pgfpathlineto{\pgfqpoint{3.521305in}{2.887306in}}%
\pgfpathlineto{\pgfqpoint{3.510045in}{2.883624in}}%
\pgfpathlineto{\pgfqpoint{3.498779in}{2.879899in}}%
\pgfpathlineto{\pgfqpoint{3.487508in}{2.876130in}}%
\pgfpathlineto{\pgfqpoint{3.493837in}{2.862765in}}%
\pgfpathlineto{\pgfqpoint{3.500168in}{2.849365in}}%
\pgfpathlineto{\pgfqpoint{3.506502in}{2.835934in}}%
\pgfpathlineto{\pgfqpoint{3.512838in}{2.822474in}}%
\pgfpathclose%
\pgfusepath{stroke,fill}%
\end{pgfscope}%
\begin{pgfscope}%
\pgfpathrectangle{\pgfqpoint{0.887500in}{0.275000in}}{\pgfqpoint{4.225000in}{4.225000in}}%
\pgfusepath{clip}%
\pgfsetbuttcap%
\pgfsetroundjoin%
\definecolor{currentfill}{rgb}{0.121380,0.629492,0.531973}%
\pgfsetfillcolor{currentfill}%
\pgfsetfillopacity{0.700000}%
\pgfsetlinewidth{0.501875pt}%
\definecolor{currentstroke}{rgb}{1.000000,1.000000,1.000000}%
\pgfsetstrokecolor{currentstroke}%
\pgfsetstrokeopacity{0.500000}%
\pgfsetdash{}{0pt}%
\pgfpathmoveto{\pgfqpoint{3.903133in}{2.535096in}}%
\pgfpathlineto{\pgfqpoint{3.914310in}{2.538484in}}%
\pgfpathlineto{\pgfqpoint{3.925481in}{2.541871in}}%
\pgfpathlineto{\pgfqpoint{3.936647in}{2.545260in}}%
\pgfpathlineto{\pgfqpoint{3.947807in}{2.548652in}}%
\pgfpathlineto{\pgfqpoint{3.958963in}{2.552045in}}%
\pgfpathlineto{\pgfqpoint{3.952553in}{2.566067in}}%
\pgfpathlineto{\pgfqpoint{3.946145in}{2.580074in}}%
\pgfpathlineto{\pgfqpoint{3.939740in}{2.594070in}}%
\pgfpathlineto{\pgfqpoint{3.933338in}{2.608057in}}%
\pgfpathlineto{\pgfqpoint{3.926937in}{2.622033in}}%
\pgfpathlineto{\pgfqpoint{3.915784in}{2.618610in}}%
\pgfpathlineto{\pgfqpoint{3.904625in}{2.615191in}}%
\pgfpathlineto{\pgfqpoint{3.893461in}{2.611778in}}%
\pgfpathlineto{\pgfqpoint{3.882292in}{2.608371in}}%
\pgfpathlineto{\pgfqpoint{3.871117in}{2.604966in}}%
\pgfpathlineto{\pgfqpoint{3.877516in}{2.591020in}}%
\pgfpathlineto{\pgfqpoint{3.883916in}{2.577061in}}%
\pgfpathlineto{\pgfqpoint{3.890320in}{2.563089in}}%
\pgfpathlineto{\pgfqpoint{3.896725in}{2.549103in}}%
\pgfpathclose%
\pgfusepath{stroke,fill}%
\end{pgfscope}%
\begin{pgfscope}%
\pgfpathrectangle{\pgfqpoint{0.887500in}{0.275000in}}{\pgfqpoint{4.225000in}{4.225000in}}%
\pgfusepath{clip}%
\pgfsetbuttcap%
\pgfsetroundjoin%
\definecolor{currentfill}{rgb}{0.412913,0.803041,0.357269}%
\pgfsetfillcolor{currentfill}%
\pgfsetfillopacity{0.700000}%
\pgfsetlinewidth{0.501875pt}%
\definecolor{currentstroke}{rgb}{1.000000,1.000000,1.000000}%
\pgfsetstrokecolor{currentstroke}%
\pgfsetstrokeopacity{0.500000}%
\pgfsetdash{}{0pt}%
\pgfpathmoveto{\pgfqpoint{3.254672in}{2.931925in}}%
\pgfpathlineto{\pgfqpoint{3.266033in}{2.940275in}}%
\pgfpathlineto{\pgfqpoint{3.277387in}{2.947982in}}%
\pgfpathlineto{\pgfqpoint{3.288734in}{2.955007in}}%
\pgfpathlineto{\pgfqpoint{3.300074in}{2.961317in}}%
\pgfpathlineto{\pgfqpoint{3.311405in}{2.966944in}}%
\pgfpathlineto{\pgfqpoint{3.305115in}{2.980519in}}%
\pgfpathlineto{\pgfqpoint{3.298827in}{2.993993in}}%
\pgfpathlineto{\pgfqpoint{3.292541in}{3.007328in}}%
\pgfpathlineto{\pgfqpoint{3.286257in}{3.020485in}}%
\pgfpathlineto{\pgfqpoint{3.279974in}{3.033423in}}%
\pgfpathlineto{\pgfqpoint{3.268650in}{3.028125in}}%
\pgfpathlineto{\pgfqpoint{3.257319in}{3.022143in}}%
\pgfpathlineto{\pgfqpoint{3.245980in}{3.015477in}}%
\pgfpathlineto{\pgfqpoint{3.234635in}{3.008236in}}%
\pgfpathlineto{\pgfqpoint{3.223285in}{3.000530in}}%
\pgfpathlineto{\pgfqpoint{3.229559in}{2.987293in}}%
\pgfpathlineto{\pgfqpoint{3.235835in}{2.973748in}}%
\pgfpathlineto{\pgfqpoint{3.242112in}{2.959961in}}%
\pgfpathlineto{\pgfqpoint{3.248391in}{2.945998in}}%
\pgfpathclose%
\pgfusepath{stroke,fill}%
\end{pgfscope}%
\begin{pgfscope}%
\pgfpathrectangle{\pgfqpoint{0.887500in}{0.275000in}}{\pgfqpoint{4.225000in}{4.225000in}}%
\pgfusepath{clip}%
\pgfsetbuttcap%
\pgfsetroundjoin%
\definecolor{currentfill}{rgb}{0.319809,0.770914,0.411152}%
\pgfsetfillcolor{currentfill}%
\pgfsetfillopacity{0.700000}%
\pgfsetlinewidth{0.501875pt}%
\definecolor{currentstroke}{rgb}{1.000000,1.000000,1.000000}%
\pgfsetstrokecolor{currentstroke}%
\pgfsetstrokeopacity{0.500000}%
\pgfsetdash{}{0pt}%
\pgfpathmoveto{\pgfqpoint{3.431063in}{2.856515in}}%
\pgfpathlineto{\pgfqpoint{3.442364in}{2.860550in}}%
\pgfpathlineto{\pgfqpoint{3.453658in}{2.864526in}}%
\pgfpathlineto{\pgfqpoint{3.464947in}{2.868446in}}%
\pgfpathlineto{\pgfqpoint{3.476230in}{2.872313in}}%
\pgfpathlineto{\pgfqpoint{3.487508in}{2.876130in}}%
\pgfpathlineto{\pgfqpoint{3.481181in}{2.889455in}}%
\pgfpathlineto{\pgfqpoint{3.474856in}{2.902738in}}%
\pgfpathlineto{\pgfqpoint{3.468534in}{2.915973in}}%
\pgfpathlineto{\pgfqpoint{3.462213in}{2.929158in}}%
\pgfpathlineto{\pgfqpoint{3.455895in}{2.942289in}}%
\pgfpathlineto{\pgfqpoint{3.444622in}{2.938458in}}%
\pgfpathlineto{\pgfqpoint{3.433343in}{2.934574in}}%
\pgfpathlineto{\pgfqpoint{3.422058in}{2.930634in}}%
\pgfpathlineto{\pgfqpoint{3.410768in}{2.926634in}}%
\pgfpathlineto{\pgfqpoint{3.399472in}{2.922569in}}%
\pgfpathlineto{\pgfqpoint{3.405785in}{2.909360in}}%
\pgfpathlineto{\pgfqpoint{3.412100in}{2.896159in}}%
\pgfpathlineto{\pgfqpoint{3.418419in}{2.882958in}}%
\pgfpathlineto{\pgfqpoint{3.424740in}{2.869747in}}%
\pgfpathclose%
\pgfusepath{stroke,fill}%
\end{pgfscope}%
\begin{pgfscope}%
\pgfpathrectangle{\pgfqpoint{0.887500in}{0.275000in}}{\pgfqpoint{4.225000in}{4.225000in}}%
\pgfusepath{clip}%
\pgfsetbuttcap%
\pgfsetroundjoin%
\definecolor{currentfill}{rgb}{0.175841,0.441290,0.557685}%
\pgfsetfillcolor{currentfill}%
\pgfsetfillopacity{0.700000}%
\pgfsetlinewidth{0.501875pt}%
\definecolor{currentstroke}{rgb}{1.000000,1.000000,1.000000}%
\pgfsetstrokecolor{currentstroke}%
\pgfsetstrokeopacity{0.500000}%
\pgfsetdash{}{0pt}%
\pgfpathmoveto{\pgfqpoint{2.804645in}{2.143801in}}%
\pgfpathlineto{\pgfqpoint{2.816091in}{2.147955in}}%
\pgfpathlineto{\pgfqpoint{2.827531in}{2.152386in}}%
\pgfpathlineto{\pgfqpoint{2.838965in}{2.156815in}}%
\pgfpathlineto{\pgfqpoint{2.850396in}{2.160954in}}%
\pgfpathlineto{\pgfqpoint{2.861825in}{2.164516in}}%
\pgfpathlineto{\pgfqpoint{2.855647in}{2.173827in}}%
\pgfpathlineto{\pgfqpoint{2.849474in}{2.183103in}}%
\pgfpathlineto{\pgfqpoint{2.843304in}{2.192344in}}%
\pgfpathlineto{\pgfqpoint{2.837139in}{2.201550in}}%
\pgfpathlineto{\pgfqpoint{2.830977in}{2.210719in}}%
\pgfpathlineto{\pgfqpoint{2.819559in}{2.207141in}}%
\pgfpathlineto{\pgfqpoint{2.808139in}{2.202967in}}%
\pgfpathlineto{\pgfqpoint{2.796716in}{2.198496in}}%
\pgfpathlineto{\pgfqpoint{2.785288in}{2.194028in}}%
\pgfpathlineto{\pgfqpoint{2.773853in}{2.189855in}}%
\pgfpathlineto{\pgfqpoint{2.780003in}{2.180704in}}%
\pgfpathlineto{\pgfqpoint{2.786158in}{2.171524in}}%
\pgfpathlineto{\pgfqpoint{2.792316in}{2.162314in}}%
\pgfpathlineto{\pgfqpoint{2.798478in}{2.153074in}}%
\pgfpathclose%
\pgfusepath{stroke,fill}%
\end{pgfscope}%
\begin{pgfscope}%
\pgfpathrectangle{\pgfqpoint{0.887500in}{0.275000in}}{\pgfqpoint{4.225000in}{4.225000in}}%
\pgfusepath{clip}%
\pgfsetbuttcap%
\pgfsetroundjoin%
\definecolor{currentfill}{rgb}{0.144759,0.519093,0.556572}%
\pgfsetfillcolor{currentfill}%
\pgfsetfillopacity{0.700000}%
\pgfsetlinewidth{0.501875pt}%
\definecolor{currentstroke}{rgb}{1.000000,1.000000,1.000000}%
\pgfsetstrokecolor{currentstroke}%
\pgfsetstrokeopacity{0.500000}%
\pgfsetdash{}{0pt}%
\pgfpathmoveto{\pgfqpoint{4.198884in}{2.299994in}}%
\pgfpathlineto{\pgfqpoint{4.209981in}{2.303216in}}%
\pgfpathlineto{\pgfqpoint{4.221073in}{2.306416in}}%
\pgfpathlineto{\pgfqpoint{4.232158in}{2.309600in}}%
\pgfpathlineto{\pgfqpoint{4.243238in}{2.312783in}}%
\pgfpathlineto{\pgfqpoint{4.254313in}{2.315978in}}%
\pgfpathlineto{\pgfqpoint{4.247872in}{2.330877in}}%
\pgfpathlineto{\pgfqpoint{4.241431in}{2.345695in}}%
\pgfpathlineto{\pgfqpoint{4.234989in}{2.360409in}}%
\pgfpathlineto{\pgfqpoint{4.228547in}{2.375023in}}%
\pgfpathlineto{\pgfqpoint{4.222105in}{2.389547in}}%
\pgfpathlineto{\pgfqpoint{4.211025in}{2.386165in}}%
\pgfpathlineto{\pgfqpoint{4.199941in}{2.382778in}}%
\pgfpathlineto{\pgfqpoint{4.188850in}{2.379391in}}%
\pgfpathlineto{\pgfqpoint{4.177755in}{2.376009in}}%
\pgfpathlineto{\pgfqpoint{4.166654in}{2.372631in}}%
\pgfpathlineto{\pgfqpoint{4.173097in}{2.358182in}}%
\pgfpathlineto{\pgfqpoint{4.179542in}{2.343694in}}%
\pgfpathlineto{\pgfqpoint{4.185988in}{2.329165in}}%
\pgfpathlineto{\pgfqpoint{4.192435in}{2.314596in}}%
\pgfpathclose%
\pgfusepath{stroke,fill}%
\end{pgfscope}%
\begin{pgfscope}%
\pgfpathrectangle{\pgfqpoint{0.887500in}{0.275000in}}{\pgfqpoint{4.225000in}{4.225000in}}%
\pgfusepath{clip}%
\pgfsetbuttcap%
\pgfsetroundjoin%
\definecolor{currentfill}{rgb}{0.386433,0.794644,0.372886}%
\pgfsetfillcolor{currentfill}%
\pgfsetfillopacity{0.700000}%
\pgfsetlinewidth{0.501875pt}%
\definecolor{currentstroke}{rgb}{1.000000,1.000000,1.000000}%
\pgfsetstrokecolor{currentstroke}%
\pgfsetstrokeopacity{0.500000}%
\pgfsetdash{}{0pt}%
\pgfpathmoveto{\pgfqpoint{3.109573in}{2.892743in}}%
\pgfpathlineto{\pgfqpoint{3.120961in}{2.910423in}}%
\pgfpathlineto{\pgfqpoint{3.132346in}{2.925346in}}%
\pgfpathlineto{\pgfqpoint{3.143728in}{2.938000in}}%
\pgfpathlineto{\pgfqpoint{3.155105in}{2.948875in}}%
\pgfpathlineto{\pgfqpoint{3.166478in}{2.958462in}}%
\pgfpathlineto{\pgfqpoint{3.160216in}{2.972675in}}%
\pgfpathlineto{\pgfqpoint{3.153956in}{2.986362in}}%
\pgfpathlineto{\pgfqpoint{3.147697in}{2.999535in}}%
\pgfpathlineto{\pgfqpoint{3.141440in}{3.012205in}}%
\pgfpathlineto{\pgfqpoint{3.135184in}{3.024383in}}%
\pgfpathlineto{\pgfqpoint{3.123823in}{3.016174in}}%
\pgfpathlineto{\pgfqpoint{3.112458in}{3.006760in}}%
\pgfpathlineto{\pgfqpoint{3.101089in}{2.995591in}}%
\pgfpathlineto{\pgfqpoint{3.089716in}{2.982121in}}%
\pgfpathlineto{\pgfqpoint{3.078343in}{2.965803in}}%
\pgfpathlineto{\pgfqpoint{3.084585in}{2.952724in}}%
\pgfpathlineto{\pgfqpoint{3.090830in}{2.938802in}}%
\pgfpathlineto{\pgfqpoint{3.097076in}{2.924113in}}%
\pgfpathlineto{\pgfqpoint{3.103324in}{2.908735in}}%
\pgfpathclose%
\pgfusepath{stroke,fill}%
\end{pgfscope}%
\begin{pgfscope}%
\pgfpathrectangle{\pgfqpoint{0.887500in}{0.275000in}}{\pgfqpoint{4.225000in}{4.225000in}}%
\pgfusepath{clip}%
\pgfsetbuttcap%
\pgfsetroundjoin%
\definecolor{currentfill}{rgb}{0.369214,0.788888,0.382914}%
\pgfsetfillcolor{currentfill}%
\pgfsetfillopacity{0.700000}%
\pgfsetlinewidth{0.501875pt}%
\definecolor{currentstroke}{rgb}{1.000000,1.000000,1.000000}%
\pgfsetstrokecolor{currentstroke}%
\pgfsetstrokeopacity{0.500000}%
\pgfsetdash{}{0pt}%
\pgfpathmoveto{\pgfqpoint{3.342891in}{2.898946in}}%
\pgfpathlineto{\pgfqpoint{3.354222in}{2.904404in}}%
\pgfpathlineto{\pgfqpoint{3.365545in}{2.909406in}}%
\pgfpathlineto{\pgfqpoint{3.376861in}{2.914042in}}%
\pgfpathlineto{\pgfqpoint{3.388170in}{2.918400in}}%
\pgfpathlineto{\pgfqpoint{3.399472in}{2.922569in}}%
\pgfpathlineto{\pgfqpoint{3.393162in}{2.935784in}}%
\pgfpathlineto{\pgfqpoint{3.386854in}{2.948997in}}%
\pgfpathlineto{\pgfqpoint{3.380549in}{2.962200in}}%
\pgfpathlineto{\pgfqpoint{3.374247in}{2.975382in}}%
\pgfpathlineto{\pgfqpoint{3.367947in}{2.988537in}}%
\pgfpathlineto{\pgfqpoint{3.356652in}{2.984733in}}%
\pgfpathlineto{\pgfqpoint{3.345351in}{2.980778in}}%
\pgfpathlineto{\pgfqpoint{3.334043in}{2.976567in}}%
\pgfpathlineto{\pgfqpoint{3.322728in}{2.971991in}}%
\pgfpathlineto{\pgfqpoint{3.311405in}{2.966944in}}%
\pgfpathlineto{\pgfqpoint{3.317697in}{2.953308in}}%
\pgfpathlineto{\pgfqpoint{3.323991in}{2.939652in}}%
\pgfpathlineto{\pgfqpoint{3.330288in}{2.926013in}}%
\pgfpathlineto{\pgfqpoint{3.336588in}{2.912431in}}%
\pgfpathclose%
\pgfusepath{stroke,fill}%
\end{pgfscope}%
\begin{pgfscope}%
\pgfpathrectangle{\pgfqpoint{0.887500in}{0.275000in}}{\pgfqpoint{4.225000in}{4.225000in}}%
\pgfusepath{clip}%
\pgfsetbuttcap%
\pgfsetroundjoin%
\definecolor{currentfill}{rgb}{0.188923,0.410910,0.556326}%
\pgfsetfillcolor{currentfill}%
\pgfsetfillopacity{0.700000}%
\pgfsetlinewidth{0.501875pt}%
\definecolor{currentstroke}{rgb}{1.000000,1.000000,1.000000}%
\pgfsetstrokecolor{currentstroke}%
\pgfsetstrokeopacity{0.500000}%
\pgfsetdash{}{0pt}%
\pgfpathmoveto{\pgfqpoint{4.494742in}{2.067651in}}%
\pgfpathlineto{\pgfqpoint{4.505769in}{2.071002in}}%
\pgfpathlineto{\pgfqpoint{4.516790in}{2.074348in}}%
\pgfpathlineto{\pgfqpoint{4.527805in}{2.077688in}}%
\pgfpathlineto{\pgfqpoint{4.538815in}{2.081024in}}%
\pgfpathlineto{\pgfqpoint{4.549819in}{2.084357in}}%
\pgfpathlineto{\pgfqpoint{4.543308in}{2.098801in}}%
\pgfpathlineto{\pgfqpoint{4.536800in}{2.113228in}}%
\pgfpathlineto{\pgfqpoint{4.530293in}{2.127636in}}%
\pgfpathlineto{\pgfqpoint{4.523788in}{2.142023in}}%
\pgfpathlineto{\pgfqpoint{4.517285in}{2.156386in}}%
\pgfpathlineto{\pgfqpoint{4.506273in}{2.152804in}}%
\pgfpathlineto{\pgfqpoint{4.495254in}{2.149174in}}%
\pgfpathlineto{\pgfqpoint{4.484227in}{2.145493in}}%
\pgfpathlineto{\pgfqpoint{4.473194in}{2.141756in}}%
\pgfpathlineto{\pgfqpoint{4.462153in}{2.137959in}}%
\pgfpathlineto{\pgfqpoint{4.468669in}{2.124004in}}%
\pgfpathlineto{\pgfqpoint{4.475187in}{2.110011in}}%
\pgfpathlineto{\pgfqpoint{4.481705in}{2.095964in}}%
\pgfpathlineto{\pgfqpoint{4.488224in}{2.081848in}}%
\pgfpathclose%
\pgfusepath{stroke,fill}%
\end{pgfscope}%
\begin{pgfscope}%
\pgfpathrectangle{\pgfqpoint{0.887500in}{0.275000in}}{\pgfqpoint{4.225000in}{4.225000in}}%
\pgfusepath{clip}%
\pgfsetbuttcap%
\pgfsetroundjoin%
\definecolor{currentfill}{rgb}{0.220124,0.725509,0.466226}%
\pgfsetfillcolor{currentfill}%
\pgfsetfillopacity{0.700000}%
\pgfsetlinewidth{0.501875pt}%
\definecolor{currentstroke}{rgb}{1.000000,1.000000,1.000000}%
\pgfsetstrokecolor{currentstroke}%
\pgfsetstrokeopacity{0.500000}%
\pgfsetdash{}{0pt}%
\pgfpathmoveto{\pgfqpoint{3.083862in}{2.694002in}}%
\pgfpathlineto{\pgfqpoint{3.095247in}{2.718894in}}%
\pgfpathlineto{\pgfqpoint{3.106638in}{2.743100in}}%
\pgfpathlineto{\pgfqpoint{3.118034in}{2.766247in}}%
\pgfpathlineto{\pgfqpoint{3.129433in}{2.787964in}}%
\pgfpathlineto{\pgfqpoint{3.140833in}{2.807879in}}%
\pgfpathlineto{\pgfqpoint{3.134578in}{2.824969in}}%
\pgfpathlineto{\pgfqpoint{3.128325in}{2.842167in}}%
\pgfpathlineto{\pgfqpoint{3.122073in}{2.859307in}}%
\pgfpathlineto{\pgfqpoint{3.115822in}{2.876221in}}%
\pgfpathlineto{\pgfqpoint{3.109573in}{2.892743in}}%
\pgfpathlineto{\pgfqpoint{3.098184in}{2.871912in}}%
\pgfpathlineto{\pgfqpoint{3.086797in}{2.848233in}}%
\pgfpathlineto{\pgfqpoint{3.075415in}{2.822330in}}%
\pgfpathlineto{\pgfqpoint{3.064041in}{2.794825in}}%
\pgfpathlineto{\pgfqpoint{3.052674in}{2.766339in}}%
\pgfpathlineto{\pgfqpoint{3.058908in}{2.751020in}}%
\pgfpathlineto{\pgfqpoint{3.065143in}{2.736141in}}%
\pgfpathlineto{\pgfqpoint{3.071380in}{2.721688in}}%
\pgfpathlineto{\pgfqpoint{3.077620in}{2.707646in}}%
\pgfpathclose%
\pgfusepath{stroke,fill}%
\end{pgfscope}%
\begin{pgfscope}%
\pgfpathrectangle{\pgfqpoint{0.887500in}{0.275000in}}{\pgfqpoint{4.225000in}{4.225000in}}%
\pgfusepath{clip}%
\pgfsetbuttcap%
\pgfsetroundjoin%
\definecolor{currentfill}{rgb}{0.163625,0.471133,0.558148}%
\pgfsetfillcolor{currentfill}%
\pgfsetfillopacity{0.700000}%
\pgfsetlinewidth{0.501875pt}%
\definecolor{currentstroke}{rgb}{1.000000,1.000000,1.000000}%
\pgfsetstrokecolor{currentstroke}%
\pgfsetstrokeopacity{0.500000}%
\pgfsetdash{}{0pt}%
\pgfpathmoveto{\pgfqpoint{2.483002in}{2.212168in}}%
\pgfpathlineto{\pgfqpoint{2.494535in}{2.215585in}}%
\pgfpathlineto{\pgfqpoint{2.506061in}{2.219016in}}%
\pgfpathlineto{\pgfqpoint{2.517582in}{2.222465in}}%
\pgfpathlineto{\pgfqpoint{2.529096in}{2.225936in}}%
\pgfpathlineto{\pgfqpoint{2.540605in}{2.229431in}}%
\pgfpathlineto{\pgfqpoint{2.534533in}{2.238325in}}%
\pgfpathlineto{\pgfqpoint{2.528465in}{2.247192in}}%
\pgfpathlineto{\pgfqpoint{2.522401in}{2.256033in}}%
\pgfpathlineto{\pgfqpoint{2.516341in}{2.264849in}}%
\pgfpathlineto{\pgfqpoint{2.510286in}{2.273641in}}%
\pgfpathlineto{\pgfqpoint{2.498788in}{2.270169in}}%
\pgfpathlineto{\pgfqpoint{2.487285in}{2.266727in}}%
\pgfpathlineto{\pgfqpoint{2.475775in}{2.263310in}}%
\pgfpathlineto{\pgfqpoint{2.464259in}{2.259914in}}%
\pgfpathlineto{\pgfqpoint{2.452738in}{2.256533in}}%
\pgfpathlineto{\pgfqpoint{2.458782in}{2.247712in}}%
\pgfpathlineto{\pgfqpoint{2.464831in}{2.238866in}}%
\pgfpathlineto{\pgfqpoint{2.470884in}{2.229993in}}%
\pgfpathlineto{\pgfqpoint{2.476941in}{2.221094in}}%
\pgfpathclose%
\pgfusepath{stroke,fill}%
\end{pgfscope}%
\begin{pgfscope}%
\pgfpathrectangle{\pgfqpoint{0.887500in}{0.275000in}}{\pgfqpoint{4.225000in}{4.225000in}}%
\pgfusepath{clip}%
\pgfsetbuttcap%
\pgfsetroundjoin%
\definecolor{currentfill}{rgb}{0.140536,0.530132,0.555659}%
\pgfsetfillcolor{currentfill}%
\pgfsetfillopacity{0.700000}%
\pgfsetlinewidth{0.501875pt}%
\definecolor{currentstroke}{rgb}{1.000000,1.000000,1.000000}%
\pgfsetstrokecolor{currentstroke}%
\pgfsetstrokeopacity{0.500000}%
\pgfsetdash{}{0pt}%
\pgfpathmoveto{\pgfqpoint{1.839746in}{2.337228in}}%
\pgfpathlineto{\pgfqpoint{1.851437in}{2.340645in}}%
\pgfpathlineto{\pgfqpoint{1.863123in}{2.344059in}}%
\pgfpathlineto{\pgfqpoint{1.874803in}{2.347468in}}%
\pgfpathlineto{\pgfqpoint{1.886477in}{2.350871in}}%
\pgfpathlineto{\pgfqpoint{1.898146in}{2.354266in}}%
\pgfpathlineto{\pgfqpoint{1.892292in}{2.362757in}}%
\pgfpathlineto{\pgfqpoint{1.886442in}{2.371228in}}%
\pgfpathlineto{\pgfqpoint{1.880597in}{2.379678in}}%
\pgfpathlineto{\pgfqpoint{1.874756in}{2.388109in}}%
\pgfpathlineto{\pgfqpoint{1.868919in}{2.396519in}}%
\pgfpathlineto{\pgfqpoint{1.857262in}{2.393157in}}%
\pgfpathlineto{\pgfqpoint{1.845599in}{2.389787in}}%
\pgfpathlineto{\pgfqpoint{1.833931in}{2.386411in}}%
\pgfpathlineto{\pgfqpoint{1.822257in}{2.383030in}}%
\pgfpathlineto{\pgfqpoint{1.810577in}{2.379648in}}%
\pgfpathlineto{\pgfqpoint{1.816402in}{2.371207in}}%
\pgfpathlineto{\pgfqpoint{1.822232in}{2.362745in}}%
\pgfpathlineto{\pgfqpoint{1.828065in}{2.354261in}}%
\pgfpathlineto{\pgfqpoint{1.833903in}{2.345755in}}%
\pgfpathclose%
\pgfusepath{stroke,fill}%
\end{pgfscope}%
\begin{pgfscope}%
\pgfpathrectangle{\pgfqpoint{0.887500in}{0.275000in}}{\pgfqpoint{4.225000in}{4.225000in}}%
\pgfusepath{clip}%
\pgfsetbuttcap%
\pgfsetroundjoin%
\definecolor{currentfill}{rgb}{0.150476,0.504369,0.557430}%
\pgfsetfillcolor{currentfill}%
\pgfsetfillopacity{0.700000}%
\pgfsetlinewidth{0.501875pt}%
\definecolor{currentstroke}{rgb}{1.000000,1.000000,1.000000}%
\pgfsetstrokecolor{currentstroke}%
\pgfsetstrokeopacity{0.500000}%
\pgfsetdash{}{0pt}%
\pgfpathmoveto{\pgfqpoint{2.161356in}{2.275820in}}%
\pgfpathlineto{\pgfqpoint{2.172968in}{2.279275in}}%
\pgfpathlineto{\pgfqpoint{2.184573in}{2.282736in}}%
\pgfpathlineto{\pgfqpoint{2.196173in}{2.286201in}}%
\pgfpathlineto{\pgfqpoint{2.207767in}{2.289664in}}%
\pgfpathlineto{\pgfqpoint{2.219356in}{2.293123in}}%
\pgfpathlineto{\pgfqpoint{2.213391in}{2.301797in}}%
\pgfpathlineto{\pgfqpoint{2.207431in}{2.310448in}}%
\pgfpathlineto{\pgfqpoint{2.201475in}{2.319079in}}%
\pgfpathlineto{\pgfqpoint{2.195524in}{2.327690in}}%
\pgfpathlineto{\pgfqpoint{2.189576in}{2.336281in}}%
\pgfpathlineto{\pgfqpoint{2.177999in}{2.332858in}}%
\pgfpathlineto{\pgfqpoint{2.166416in}{2.329431in}}%
\pgfpathlineto{\pgfqpoint{2.154827in}{2.326004in}}%
\pgfpathlineto{\pgfqpoint{2.143233in}{2.322581in}}%
\pgfpathlineto{\pgfqpoint{2.131633in}{2.319164in}}%
\pgfpathlineto{\pgfqpoint{2.137569in}{2.310538in}}%
\pgfpathlineto{\pgfqpoint{2.143509in}{2.301891in}}%
\pgfpathlineto{\pgfqpoint{2.149454in}{2.293223in}}%
\pgfpathlineto{\pgfqpoint{2.155403in}{2.284533in}}%
\pgfpathclose%
\pgfusepath{stroke,fill}%
\end{pgfscope}%
\begin{pgfscope}%
\pgfpathrectangle{\pgfqpoint{0.887500in}{0.275000in}}{\pgfqpoint{4.225000in}{4.225000in}}%
\pgfusepath{clip}%
\pgfsetbuttcap%
\pgfsetroundjoin%
\definecolor{currentfill}{rgb}{0.132268,0.655014,0.519661}%
\pgfsetfillcolor{currentfill}%
\pgfsetfillopacity{0.700000}%
\pgfsetlinewidth{0.501875pt}%
\definecolor{currentstroke}{rgb}{1.000000,1.000000,1.000000}%
\pgfsetstrokecolor{currentstroke}%
\pgfsetstrokeopacity{0.500000}%
\pgfsetdash{}{0pt}%
\pgfpathmoveto{\pgfqpoint{3.815160in}{2.587851in}}%
\pgfpathlineto{\pgfqpoint{3.826363in}{2.591298in}}%
\pgfpathlineto{\pgfqpoint{3.837560in}{2.594729in}}%
\pgfpathlineto{\pgfqpoint{3.848751in}{2.598148in}}%
\pgfpathlineto{\pgfqpoint{3.859937in}{2.601559in}}%
\pgfpathlineto{\pgfqpoint{3.871117in}{2.604966in}}%
\pgfpathlineto{\pgfqpoint{3.864721in}{2.618898in}}%
\pgfpathlineto{\pgfqpoint{3.858327in}{2.632814in}}%
\pgfpathlineto{\pgfqpoint{3.851935in}{2.646713in}}%
\pgfpathlineto{\pgfqpoint{3.845546in}{2.660594in}}%
\pgfpathlineto{\pgfqpoint{3.839158in}{2.674456in}}%
\pgfpathlineto{\pgfqpoint{3.827981in}{2.671054in}}%
\pgfpathlineto{\pgfqpoint{3.816798in}{2.667660in}}%
\pgfpathlineto{\pgfqpoint{3.805610in}{2.664272in}}%
\pgfpathlineto{\pgfqpoint{3.794416in}{2.660887in}}%
\pgfpathlineto{\pgfqpoint{3.783217in}{2.657502in}}%
\pgfpathlineto{\pgfqpoint{3.789601in}{2.643600in}}%
\pgfpathlineto{\pgfqpoint{3.795987in}{2.629681in}}%
\pgfpathlineto{\pgfqpoint{3.802376in}{2.615748in}}%
\pgfpathlineto{\pgfqpoint{3.808767in}{2.601803in}}%
\pgfpathclose%
\pgfusepath{stroke,fill}%
\end{pgfscope}%
\begin{pgfscope}%
\pgfpathrectangle{\pgfqpoint{0.887500in}{0.275000in}}{\pgfqpoint{4.225000in}{4.225000in}}%
\pgfusepath{clip}%
\pgfsetbuttcap%
\pgfsetroundjoin%
\definecolor{currentfill}{rgb}{0.182256,0.426184,0.557120}%
\pgfsetfillcolor{currentfill}%
\pgfsetfillopacity{0.700000}%
\pgfsetlinewidth{0.501875pt}%
\definecolor{currentstroke}{rgb}{1.000000,1.000000,1.000000}%
\pgfsetstrokecolor{currentstroke}%
\pgfsetstrokeopacity{0.500000}%
\pgfsetdash{}{0pt}%
\pgfpathmoveto{\pgfqpoint{2.892773in}{2.117479in}}%
\pgfpathlineto{\pgfqpoint{2.904209in}{2.120289in}}%
\pgfpathlineto{\pgfqpoint{2.915643in}{2.122060in}}%
\pgfpathlineto{\pgfqpoint{2.927075in}{2.122589in}}%
\pgfpathlineto{\pgfqpoint{2.938501in}{2.122362in}}%
\pgfpathlineto{\pgfqpoint{2.949919in}{2.122322in}}%
\pgfpathlineto{\pgfqpoint{2.943714in}{2.131491in}}%
\pgfpathlineto{\pgfqpoint{2.937514in}{2.140611in}}%
\pgfpathlineto{\pgfqpoint{2.931317in}{2.149698in}}%
\pgfpathlineto{\pgfqpoint{2.925124in}{2.158764in}}%
\pgfpathlineto{\pgfqpoint{2.918935in}{2.167826in}}%
\pgfpathlineto{\pgfqpoint{2.907524in}{2.168281in}}%
\pgfpathlineto{\pgfqpoint{2.896103in}{2.168941in}}%
\pgfpathlineto{\pgfqpoint{2.884678in}{2.168765in}}%
\pgfpathlineto{\pgfqpoint{2.873252in}{2.167215in}}%
\pgfpathlineto{\pgfqpoint{2.861825in}{2.164516in}}%
\pgfpathlineto{\pgfqpoint{2.868007in}{2.155173in}}%
\pgfpathlineto{\pgfqpoint{2.874192in}{2.145797in}}%
\pgfpathlineto{\pgfqpoint{2.880382in}{2.136388in}}%
\pgfpathlineto{\pgfqpoint{2.886575in}{2.126949in}}%
\pgfpathclose%
\pgfusepath{stroke,fill}%
\end{pgfscope}%
\begin{pgfscope}%
\pgfpathrectangle{\pgfqpoint{0.887500in}{0.275000in}}{\pgfqpoint{4.225000in}{4.225000in}}%
\pgfusepath{clip}%
\pgfsetbuttcap%
\pgfsetroundjoin%
\definecolor{currentfill}{rgb}{0.177423,0.437527,0.557565}%
\pgfsetfillcolor{currentfill}%
\pgfsetfillopacity{0.700000}%
\pgfsetlinewidth{0.501875pt}%
\definecolor{currentstroke}{rgb}{1.000000,1.000000,1.000000}%
\pgfsetstrokecolor{currentstroke}%
\pgfsetstrokeopacity{0.500000}%
\pgfsetdash{}{0pt}%
\pgfpathmoveto{\pgfqpoint{4.406861in}{2.118717in}}%
\pgfpathlineto{\pgfqpoint{4.417928in}{2.122504in}}%
\pgfpathlineto{\pgfqpoint{4.428991in}{2.126351in}}%
\pgfpathlineto{\pgfqpoint{4.440051in}{2.130228in}}%
\pgfpathlineto{\pgfqpoint{4.451105in}{2.134107in}}%
\pgfpathlineto{\pgfqpoint{4.462153in}{2.137959in}}%
\pgfpathlineto{\pgfqpoint{4.455638in}{2.151890in}}%
\pgfpathlineto{\pgfqpoint{4.449126in}{2.165813in}}%
\pgfpathlineto{\pgfqpoint{4.442617in}{2.179742in}}%
\pgfpathlineto{\pgfqpoint{4.436112in}{2.193694in}}%
\pgfpathlineto{\pgfqpoint{4.429611in}{2.207682in}}%
\pgfpathlineto{\pgfqpoint{4.418549in}{2.203375in}}%
\pgfpathlineto{\pgfqpoint{4.407481in}{2.199030in}}%
\pgfpathlineto{\pgfqpoint{4.396410in}{2.194710in}}%
\pgfpathlineto{\pgfqpoint{4.385336in}{2.190473in}}%
\pgfpathlineto{\pgfqpoint{4.374262in}{2.186382in}}%
\pgfpathlineto{\pgfqpoint{4.380769in}{2.172642in}}%
\pgfpathlineto{\pgfqpoint{4.387284in}{2.159045in}}%
\pgfpathlineto{\pgfqpoint{4.393805in}{2.145551in}}%
\pgfpathlineto{\pgfqpoint{4.400331in}{2.132122in}}%
\pgfpathclose%
\pgfusepath{stroke,fill}%
\end{pgfscope}%
\begin{pgfscope}%
\pgfpathrectangle{\pgfqpoint{0.887500in}{0.275000in}}{\pgfqpoint{4.225000in}{4.225000in}}%
\pgfusepath{clip}%
\pgfsetbuttcap%
\pgfsetroundjoin%
\definecolor{currentfill}{rgb}{0.133743,0.548535,0.553541}%
\pgfsetfillcolor{currentfill}%
\pgfsetfillopacity{0.700000}%
\pgfsetlinewidth{0.501875pt}%
\definecolor{currentstroke}{rgb}{1.000000,1.000000,1.000000}%
\pgfsetstrokecolor{currentstroke}%
\pgfsetstrokeopacity{0.500000}%
\pgfsetdash{}{0pt}%
\pgfpathmoveto{\pgfqpoint{4.111066in}{2.355632in}}%
\pgfpathlineto{\pgfqpoint{4.122195in}{2.359063in}}%
\pgfpathlineto{\pgfqpoint{4.133319in}{2.362473in}}%
\pgfpathlineto{\pgfqpoint{4.144436in}{2.365868in}}%
\pgfpathlineto{\pgfqpoint{4.155548in}{2.369252in}}%
\pgfpathlineto{\pgfqpoint{4.166654in}{2.372631in}}%
\pgfpathlineto{\pgfqpoint{4.160212in}{2.387041in}}%
\pgfpathlineto{\pgfqpoint{4.153772in}{2.401414in}}%
\pgfpathlineto{\pgfqpoint{4.147333in}{2.415751in}}%
\pgfpathlineto{\pgfqpoint{4.140896in}{2.430054in}}%
\pgfpathlineto{\pgfqpoint{4.134460in}{2.444324in}}%
\pgfpathlineto{\pgfqpoint{4.123356in}{2.440963in}}%
\pgfpathlineto{\pgfqpoint{4.112247in}{2.437597in}}%
\pgfpathlineto{\pgfqpoint{4.101131in}{2.434223in}}%
\pgfpathlineto{\pgfqpoint{4.090010in}{2.430838in}}%
\pgfpathlineto{\pgfqpoint{4.078883in}{2.427436in}}%
\pgfpathlineto{\pgfqpoint{4.085316in}{2.413120in}}%
\pgfpathlineto{\pgfqpoint{4.091750in}{2.398778in}}%
\pgfpathlineto{\pgfqpoint{4.098187in}{2.384414in}}%
\pgfpathlineto{\pgfqpoint{4.104625in}{2.370030in}}%
\pgfpathclose%
\pgfusepath{stroke,fill}%
\end{pgfscope}%
\begin{pgfscope}%
\pgfpathrectangle{\pgfqpoint{0.887500in}{0.275000in}}{\pgfqpoint{4.225000in}{4.225000in}}%
\pgfusepath{clip}%
\pgfsetbuttcap%
\pgfsetroundjoin%
\definecolor{currentfill}{rgb}{0.231674,0.318106,0.544834}%
\pgfsetfillcolor{currentfill}%
\pgfsetfillopacity{0.700000}%
\pgfsetlinewidth{0.501875pt}%
\definecolor{currentstroke}{rgb}{1.000000,1.000000,1.000000}%
\pgfsetstrokecolor{currentstroke}%
\pgfsetstrokeopacity{0.500000}%
\pgfsetdash{}{0pt}%
\pgfpathmoveto{\pgfqpoint{4.702659in}{1.882339in}}%
\pgfpathlineto{\pgfqpoint{4.713633in}{1.885677in}}%
\pgfpathlineto{\pgfqpoint{4.724602in}{1.889019in}}%
\pgfpathlineto{\pgfqpoint{4.735566in}{1.892370in}}%
\pgfpathlineto{\pgfqpoint{4.746525in}{1.895731in}}%
\pgfpathlineto{\pgfqpoint{4.739985in}{1.910438in}}%
\pgfpathlineto{\pgfqpoint{4.733446in}{1.925111in}}%
\pgfpathlineto{\pgfqpoint{4.726909in}{1.939754in}}%
\pgfpathlineto{\pgfqpoint{4.720373in}{1.954370in}}%
\pgfpathlineto{\pgfqpoint{4.713838in}{1.968962in}}%
\pgfpathlineto{\pgfqpoint{4.702880in}{1.965604in}}%
\pgfpathlineto{\pgfqpoint{4.691917in}{1.962256in}}%
\pgfpathlineto{\pgfqpoint{4.680949in}{1.958917in}}%
\pgfpathlineto{\pgfqpoint{4.669976in}{1.955586in}}%
\pgfpathlineto{\pgfqpoint{4.676509in}{1.940986in}}%
\pgfpathlineto{\pgfqpoint{4.683044in}{1.926364in}}%
\pgfpathlineto{\pgfqpoint{4.689581in}{1.911717in}}%
\pgfpathlineto{\pgfqpoint{4.696119in}{1.897044in}}%
\pgfpathclose%
\pgfusepath{stroke,fill}%
\end{pgfscope}%
\begin{pgfscope}%
\pgfpathrectangle{\pgfqpoint{0.887500in}{0.275000in}}{\pgfqpoint{4.225000in}{4.225000in}}%
\pgfusepath{clip}%
\pgfsetbuttcap%
\pgfsetroundjoin%
\definecolor{currentfill}{rgb}{0.153894,0.680203,0.504172}%
\pgfsetfillcolor{currentfill}%
\pgfsetfillopacity{0.700000}%
\pgfsetlinewidth{0.501875pt}%
\definecolor{currentstroke}{rgb}{1.000000,1.000000,1.000000}%
\pgfsetstrokecolor{currentstroke}%
\pgfsetstrokeopacity{0.500000}%
\pgfsetdash{}{0pt}%
\pgfpathmoveto{\pgfqpoint{3.727134in}{2.640213in}}%
\pgfpathlineto{\pgfqpoint{3.738363in}{2.643755in}}%
\pgfpathlineto{\pgfqpoint{3.749586in}{2.647245in}}%
\pgfpathlineto{\pgfqpoint{3.760802in}{2.650691in}}%
\pgfpathlineto{\pgfqpoint{3.772013in}{2.654107in}}%
\pgfpathlineto{\pgfqpoint{3.783217in}{2.657502in}}%
\pgfpathlineto{\pgfqpoint{3.776836in}{2.671382in}}%
\pgfpathlineto{\pgfqpoint{3.770456in}{2.685238in}}%
\pgfpathlineto{\pgfqpoint{3.764079in}{2.699066in}}%
\pgfpathlineto{\pgfqpoint{3.757703in}{2.712863in}}%
\pgfpathlineto{\pgfqpoint{3.751330in}{2.726628in}}%
\pgfpathlineto{\pgfqpoint{3.740129in}{2.723281in}}%
\pgfpathlineto{\pgfqpoint{3.728922in}{2.719918in}}%
\pgfpathlineto{\pgfqpoint{3.717709in}{2.716528in}}%
\pgfpathlineto{\pgfqpoint{3.706491in}{2.713099in}}%
\pgfpathlineto{\pgfqpoint{3.695266in}{2.709620in}}%
\pgfpathlineto{\pgfqpoint{3.701636in}{2.695838in}}%
\pgfpathlineto{\pgfqpoint{3.708008in}{2.682002in}}%
\pgfpathlineto{\pgfqpoint{3.714381in}{2.668114in}}%
\pgfpathlineto{\pgfqpoint{3.720757in}{2.654181in}}%
\pgfpathclose%
\pgfusepath{stroke,fill}%
\end{pgfscope}%
\begin{pgfscope}%
\pgfpathrectangle{\pgfqpoint{0.887500in}{0.275000in}}{\pgfqpoint{4.225000in}{4.225000in}}%
\pgfusepath{clip}%
\pgfsetbuttcap%
\pgfsetroundjoin%
\definecolor{currentfill}{rgb}{0.133743,0.548535,0.553541}%
\pgfsetfillcolor{currentfill}%
\pgfsetfillopacity{0.700000}%
\pgfsetlinewidth{0.501875pt}%
\definecolor{currentstroke}{rgb}{1.000000,1.000000,1.000000}%
\pgfsetstrokecolor{currentstroke}%
\pgfsetstrokeopacity{0.500000}%
\pgfsetdash{}{0pt}%
\pgfpathmoveto{\pgfqpoint{1.605813in}{2.370894in}}%
\pgfpathlineto{\pgfqpoint{1.617566in}{2.374321in}}%
\pgfpathlineto{\pgfqpoint{1.629313in}{2.377736in}}%
\pgfpathlineto{\pgfqpoint{1.641055in}{2.381139in}}%
\pgfpathlineto{\pgfqpoint{1.652791in}{2.384532in}}%
\pgfpathlineto{\pgfqpoint{1.664522in}{2.387916in}}%
\pgfpathlineto{\pgfqpoint{1.658748in}{2.396268in}}%
\pgfpathlineto{\pgfqpoint{1.652979in}{2.404594in}}%
\pgfpathlineto{\pgfqpoint{1.647214in}{2.412896in}}%
\pgfpathlineto{\pgfqpoint{1.641454in}{2.421174in}}%
\pgfpathlineto{\pgfqpoint{1.635698in}{2.429429in}}%
\pgfpathlineto{\pgfqpoint{1.623980in}{2.426065in}}%
\pgfpathlineto{\pgfqpoint{1.612256in}{2.422691in}}%
\pgfpathlineto{\pgfqpoint{1.600526in}{2.419307in}}%
\pgfpathlineto{\pgfqpoint{1.588792in}{2.415911in}}%
\pgfpathlineto{\pgfqpoint{1.577051in}{2.412503in}}%
\pgfpathlineto{\pgfqpoint{1.582794in}{2.404231in}}%
\pgfpathlineto{\pgfqpoint{1.588542in}{2.395935in}}%
\pgfpathlineto{\pgfqpoint{1.594294in}{2.387614in}}%
\pgfpathlineto{\pgfqpoint{1.600051in}{2.379267in}}%
\pgfpathclose%
\pgfusepath{stroke,fill}%
\end{pgfscope}%
\begin{pgfscope}%
\pgfpathrectangle{\pgfqpoint{0.887500in}{0.275000in}}{\pgfqpoint{4.225000in}{4.225000in}}%
\pgfusepath{clip}%
\pgfsetbuttcap%
\pgfsetroundjoin%
\definecolor{currentfill}{rgb}{0.168126,0.459988,0.558082}%
\pgfsetfillcolor{currentfill}%
\pgfsetfillopacity{0.700000}%
\pgfsetlinewidth{0.501875pt}%
\definecolor{currentstroke}{rgb}{1.000000,1.000000,1.000000}%
\pgfsetstrokecolor{currentstroke}%
\pgfsetstrokeopacity{0.500000}%
\pgfsetdash{}{0pt}%
\pgfpathmoveto{\pgfqpoint{4.318929in}{2.169415in}}%
\pgfpathlineto{\pgfqpoint{4.329993in}{2.172395in}}%
\pgfpathlineto{\pgfqpoint{4.341057in}{2.175528in}}%
\pgfpathlineto{\pgfqpoint{4.352121in}{2.178876in}}%
\pgfpathlineto{\pgfqpoint{4.363190in}{2.182496in}}%
\pgfpathlineto{\pgfqpoint{4.374262in}{2.186382in}}%
\pgfpathlineto{\pgfqpoint{4.367764in}{2.200303in}}%
\pgfpathlineto{\pgfqpoint{4.361275in}{2.214444in}}%
\pgfpathlineto{\pgfqpoint{4.354797in}{2.228803in}}%
\pgfpathlineto{\pgfqpoint{4.348327in}{2.243347in}}%
\pgfpathlineto{\pgfqpoint{4.341865in}{2.258044in}}%
\pgfpathlineto{\pgfqpoint{4.330799in}{2.254280in}}%
\pgfpathlineto{\pgfqpoint{4.319735in}{2.250745in}}%
\pgfpathlineto{\pgfqpoint{4.308673in}{2.247446in}}%
\pgfpathlineto{\pgfqpoint{4.297611in}{2.244335in}}%
\pgfpathlineto{\pgfqpoint{4.286549in}{2.241364in}}%
\pgfpathlineto{\pgfqpoint{4.293009in}{2.226638in}}%
\pgfpathlineto{\pgfqpoint{4.299476in}{2.212052in}}%
\pgfpathlineto{\pgfqpoint{4.305951in}{2.197638in}}%
\pgfpathlineto{\pgfqpoint{4.312436in}{2.183424in}}%
\pgfpathclose%
\pgfusepath{stroke,fill}%
\end{pgfscope}%
\begin{pgfscope}%
\pgfpathrectangle{\pgfqpoint{0.887500in}{0.275000in}}{\pgfqpoint{4.225000in}{4.225000in}}%
\pgfusepath{clip}%
\pgfsetbuttcap%
\pgfsetroundjoin%
\definecolor{currentfill}{rgb}{0.166617,0.463708,0.558119}%
\pgfsetfillcolor{currentfill}%
\pgfsetfillopacity{0.700000}%
\pgfsetlinewidth{0.501875pt}%
\definecolor{currentstroke}{rgb}{1.000000,1.000000,1.000000}%
\pgfsetstrokecolor{currentstroke}%
\pgfsetstrokeopacity{0.500000}%
\pgfsetdash{}{0pt}%
\pgfpathmoveto{\pgfqpoint{2.571030in}{2.184527in}}%
\pgfpathlineto{\pgfqpoint{2.582543in}{2.188075in}}%
\pgfpathlineto{\pgfqpoint{2.594051in}{2.191651in}}%
\pgfpathlineto{\pgfqpoint{2.605553in}{2.195249in}}%
\pgfpathlineto{\pgfqpoint{2.617049in}{2.198842in}}%
\pgfpathlineto{\pgfqpoint{2.628540in}{2.202403in}}%
\pgfpathlineto{\pgfqpoint{2.622436in}{2.211420in}}%
\pgfpathlineto{\pgfqpoint{2.616335in}{2.220409in}}%
\pgfpathlineto{\pgfqpoint{2.610239in}{2.229370in}}%
\pgfpathlineto{\pgfqpoint{2.604148in}{2.238305in}}%
\pgfpathlineto{\pgfqpoint{2.598060in}{2.247214in}}%
\pgfpathlineto{\pgfqpoint{2.586580in}{2.243667in}}%
\pgfpathlineto{\pgfqpoint{2.575095in}{2.240088in}}%
\pgfpathlineto{\pgfqpoint{2.563604in}{2.236508in}}%
\pgfpathlineto{\pgfqpoint{2.552108in}{2.232954in}}%
\pgfpathlineto{\pgfqpoint{2.540605in}{2.229431in}}%
\pgfpathlineto{\pgfqpoint{2.546682in}{2.220508in}}%
\pgfpathlineto{\pgfqpoint{2.552762in}{2.211557in}}%
\pgfpathlineto{\pgfqpoint{2.558847in}{2.202577in}}%
\pgfpathlineto{\pgfqpoint{2.564936in}{2.193567in}}%
\pgfpathclose%
\pgfusepath{stroke,fill}%
\end{pgfscope}%
\begin{pgfscope}%
\pgfpathrectangle{\pgfqpoint{0.887500in}{0.275000in}}{\pgfqpoint{4.225000in}{4.225000in}}%
\pgfusepath{clip}%
\pgfsetbuttcap%
\pgfsetroundjoin%
\definecolor{currentfill}{rgb}{0.187231,0.414746,0.556547}%
\pgfsetfillcolor{currentfill}%
\pgfsetfillopacity{0.700000}%
\pgfsetlinewidth{0.501875pt}%
\definecolor{currentstroke}{rgb}{1.000000,1.000000,1.000000}%
\pgfsetstrokecolor{currentstroke}%
\pgfsetstrokeopacity{0.500000}%
\pgfsetdash{}{0pt}%
\pgfpathmoveto{\pgfqpoint{2.981004in}{2.075445in}}%
\pgfpathlineto{\pgfqpoint{2.992420in}{2.076725in}}%
\pgfpathlineto{\pgfqpoint{3.003826in}{2.079975in}}%
\pgfpathlineto{\pgfqpoint{3.015223in}{2.086090in}}%
\pgfpathlineto{\pgfqpoint{3.026613in}{2.095966in}}%
\pgfpathlineto{\pgfqpoint{3.038000in}{2.110352in}}%
\pgfpathlineto{\pgfqpoint{3.031762in}{2.119954in}}%
\pgfpathlineto{\pgfqpoint{3.025528in}{2.129517in}}%
\pgfpathlineto{\pgfqpoint{3.019298in}{2.139047in}}%
\pgfpathlineto{\pgfqpoint{3.013071in}{2.148548in}}%
\pgfpathlineto{\pgfqpoint{3.006849in}{2.158026in}}%
\pgfpathlineto{\pgfqpoint{2.995482in}{2.143036in}}%
\pgfpathlineto{\pgfqpoint{2.984108in}{2.132828in}}%
\pgfpathlineto{\pgfqpoint{2.972723in}{2.126606in}}%
\pgfpathlineto{\pgfqpoint{2.961327in}{2.123421in}}%
\pgfpathlineto{\pgfqpoint{2.949919in}{2.122322in}}%
\pgfpathlineto{\pgfqpoint{2.956128in}{2.113091in}}%
\pgfpathlineto{\pgfqpoint{2.962341in}{2.103783in}}%
\pgfpathlineto{\pgfqpoint{2.968558in}{2.094401in}}%
\pgfpathlineto{\pgfqpoint{2.974779in}{2.084953in}}%
\pgfpathclose%
\pgfusepath{stroke,fill}%
\end{pgfscope}%
\begin{pgfscope}%
\pgfpathrectangle{\pgfqpoint{0.887500in}{0.275000in}}{\pgfqpoint{4.225000in}{4.225000in}}%
\pgfusepath{clip}%
\pgfsetbuttcap%
\pgfsetroundjoin%
\definecolor{currentfill}{rgb}{0.143343,0.522773,0.556295}%
\pgfsetfillcolor{currentfill}%
\pgfsetfillopacity{0.700000}%
\pgfsetlinewidth{0.501875pt}%
\definecolor{currentstroke}{rgb}{1.000000,1.000000,1.000000}%
\pgfsetstrokecolor{currentstroke}%
\pgfsetstrokeopacity{0.500000}%
\pgfsetdash{}{0pt}%
\pgfpathmoveto{\pgfqpoint{1.927482in}{2.311489in}}%
\pgfpathlineto{\pgfqpoint{1.939157in}{2.314908in}}%
\pgfpathlineto{\pgfqpoint{1.950826in}{2.318315in}}%
\pgfpathlineto{\pgfqpoint{1.962490in}{2.321709in}}%
\pgfpathlineto{\pgfqpoint{1.974149in}{2.325092in}}%
\pgfpathlineto{\pgfqpoint{1.985802in}{2.328465in}}%
\pgfpathlineto{\pgfqpoint{1.979914in}{2.337032in}}%
\pgfpathlineto{\pgfqpoint{1.974031in}{2.345577in}}%
\pgfpathlineto{\pgfqpoint{1.968152in}{2.354101in}}%
\pgfpathlineto{\pgfqpoint{1.962278in}{2.362603in}}%
\pgfpathlineto{\pgfqpoint{1.956407in}{2.371084in}}%
\pgfpathlineto{\pgfqpoint{1.944766in}{2.367740in}}%
\pgfpathlineto{\pgfqpoint{1.933120in}{2.364387in}}%
\pgfpathlineto{\pgfqpoint{1.921467in}{2.361024in}}%
\pgfpathlineto{\pgfqpoint{1.909809in}{2.357650in}}%
\pgfpathlineto{\pgfqpoint{1.898146in}{2.354266in}}%
\pgfpathlineto{\pgfqpoint{1.904005in}{2.345754in}}%
\pgfpathlineto{\pgfqpoint{1.909867in}{2.337221in}}%
\pgfpathlineto{\pgfqpoint{1.915735in}{2.328667in}}%
\pgfpathlineto{\pgfqpoint{1.921606in}{2.320089in}}%
\pgfpathclose%
\pgfusepath{stroke,fill}%
\end{pgfscope}%
\begin{pgfscope}%
\pgfpathrectangle{\pgfqpoint{0.887500in}{0.275000in}}{\pgfqpoint{4.225000in}{4.225000in}}%
\pgfusepath{clip}%
\pgfsetbuttcap%
\pgfsetroundjoin%
\definecolor{currentfill}{rgb}{0.154815,0.493313,0.557840}%
\pgfsetfillcolor{currentfill}%
\pgfsetfillopacity{0.700000}%
\pgfsetlinewidth{0.501875pt}%
\definecolor{currentstroke}{rgb}{1.000000,1.000000,1.000000}%
\pgfsetstrokecolor{currentstroke}%
\pgfsetstrokeopacity{0.500000}%
\pgfsetdash{}{0pt}%
\pgfpathmoveto{\pgfqpoint{2.249242in}{2.249405in}}%
\pgfpathlineto{\pgfqpoint{2.260837in}{2.252897in}}%
\pgfpathlineto{\pgfqpoint{2.272425in}{2.256377in}}%
\pgfpathlineto{\pgfqpoint{2.284008in}{2.259843in}}%
\pgfpathlineto{\pgfqpoint{2.295586in}{2.263294in}}%
\pgfpathlineto{\pgfqpoint{2.307158in}{2.266730in}}%
\pgfpathlineto{\pgfqpoint{2.301161in}{2.275481in}}%
\pgfpathlineto{\pgfqpoint{2.295168in}{2.284209in}}%
\pgfpathlineto{\pgfqpoint{2.289180in}{2.292913in}}%
\pgfpathlineto{\pgfqpoint{2.283196in}{2.301595in}}%
\pgfpathlineto{\pgfqpoint{2.277216in}{2.310255in}}%
\pgfpathlineto{\pgfqpoint{2.265655in}{2.306856in}}%
\pgfpathlineto{\pgfqpoint{2.254088in}{2.303442in}}%
\pgfpathlineto{\pgfqpoint{2.242516in}{2.300014in}}%
\pgfpathlineto{\pgfqpoint{2.230939in}{2.296574in}}%
\pgfpathlineto{\pgfqpoint{2.219356in}{2.293123in}}%
\pgfpathlineto{\pgfqpoint{2.225325in}{2.284427in}}%
\pgfpathlineto{\pgfqpoint{2.231298in}{2.275708in}}%
\pgfpathlineto{\pgfqpoint{2.237275in}{2.266965in}}%
\pgfpathlineto{\pgfqpoint{2.243257in}{2.258198in}}%
\pgfpathclose%
\pgfusepath{stroke,fill}%
\end{pgfscope}%
\begin{pgfscope}%
\pgfpathrectangle{\pgfqpoint{0.887500in}{0.275000in}}{\pgfqpoint{4.225000in}{4.225000in}}%
\pgfusepath{clip}%
\pgfsetbuttcap%
\pgfsetroundjoin%
\definecolor{currentfill}{rgb}{0.369214,0.788888,0.382914}%
\pgfsetfillcolor{currentfill}%
\pgfsetfillopacity{0.700000}%
\pgfsetlinewidth{0.501875pt}%
\definecolor{currentstroke}{rgb}{1.000000,1.000000,1.000000}%
\pgfsetstrokecolor{currentstroke}%
\pgfsetstrokeopacity{0.500000}%
\pgfsetdash{}{0pt}%
\pgfpathmoveto{\pgfqpoint{3.197799in}{2.881646in}}%
\pgfpathlineto{\pgfqpoint{3.209182in}{2.892838in}}%
\pgfpathlineto{\pgfqpoint{3.220560in}{2.903395in}}%
\pgfpathlineto{\pgfqpoint{3.231935in}{2.913447in}}%
\pgfpathlineto{\pgfqpoint{3.243306in}{2.922969in}}%
\pgfpathlineto{\pgfqpoint{3.254672in}{2.931925in}}%
\pgfpathlineto{\pgfqpoint{3.248391in}{2.945998in}}%
\pgfpathlineto{\pgfqpoint{3.242112in}{2.959961in}}%
\pgfpathlineto{\pgfqpoint{3.235835in}{2.973748in}}%
\pgfpathlineto{\pgfqpoint{3.229559in}{2.987293in}}%
\pgfpathlineto{\pgfqpoint{3.223285in}{3.000530in}}%
\pgfpathlineto{\pgfqpoint{3.211930in}{2.992471in}}%
\pgfpathlineto{\pgfqpoint{3.200571in}{2.984169in}}%
\pgfpathlineto{\pgfqpoint{3.189210in}{2.975735in}}%
\pgfpathlineto{\pgfqpoint{3.177845in}{2.967253in}}%
\pgfpathlineto{\pgfqpoint{3.166478in}{2.958462in}}%
\pgfpathlineto{\pgfqpoint{3.172740in}{2.943723in}}%
\pgfpathlineto{\pgfqpoint{3.179003in}{2.928542in}}%
\pgfpathlineto{\pgfqpoint{3.185267in}{2.913048in}}%
\pgfpathlineto{\pgfqpoint{3.191532in}{2.897373in}}%
\pgfpathclose%
\pgfusepath{stroke,fill}%
\end{pgfscope}%
\begin{pgfscope}%
\pgfpathrectangle{\pgfqpoint{0.887500in}{0.275000in}}{\pgfqpoint{4.225000in}{4.225000in}}%
\pgfusepath{clip}%
\pgfsetbuttcap%
\pgfsetroundjoin%
\definecolor{currentfill}{rgb}{0.125394,0.574318,0.549086}%
\pgfsetfillcolor{currentfill}%
\pgfsetfillopacity{0.700000}%
\pgfsetlinewidth{0.501875pt}%
\definecolor{currentstroke}{rgb}{1.000000,1.000000,1.000000}%
\pgfsetstrokecolor{currentstroke}%
\pgfsetstrokeopacity{0.500000}%
\pgfsetdash{}{0pt}%
\pgfpathmoveto{\pgfqpoint{4.023162in}{2.410209in}}%
\pgfpathlineto{\pgfqpoint{4.034317in}{2.413670in}}%
\pgfpathlineto{\pgfqpoint{4.045467in}{2.417128in}}%
\pgfpathlineto{\pgfqpoint{4.056612in}{2.420578in}}%
\pgfpathlineto{\pgfqpoint{4.067751in}{2.424016in}}%
\pgfpathlineto{\pgfqpoint{4.078883in}{2.427436in}}%
\pgfpathlineto{\pgfqpoint{4.072453in}{2.441723in}}%
\pgfpathlineto{\pgfqpoint{4.066023in}{2.455977in}}%
\pgfpathlineto{\pgfqpoint{4.059596in}{2.470195in}}%
\pgfpathlineto{\pgfqpoint{4.053170in}{2.484378in}}%
\pgfpathlineto{\pgfqpoint{4.046746in}{2.498529in}}%
\pgfpathlineto{\pgfqpoint{4.035616in}{2.495159in}}%
\pgfpathlineto{\pgfqpoint{4.024481in}{2.491776in}}%
\pgfpathlineto{\pgfqpoint{4.013340in}{2.488383in}}%
\pgfpathlineto{\pgfqpoint{4.002193in}{2.484984in}}%
\pgfpathlineto{\pgfqpoint{3.991041in}{2.481581in}}%
\pgfpathlineto{\pgfqpoint{3.997462in}{2.467391in}}%
\pgfpathlineto{\pgfqpoint{4.003885in}{2.453160in}}%
\pgfpathlineto{\pgfqpoint{4.010309in}{2.438884in}}%
\pgfpathlineto{\pgfqpoint{4.016735in}{2.424565in}}%
\pgfpathclose%
\pgfusepath{stroke,fill}%
\end{pgfscope}%
\begin{pgfscope}%
\pgfpathrectangle{\pgfqpoint{0.887500in}{0.275000in}}{\pgfqpoint{4.225000in}{4.225000in}}%
\pgfusepath{clip}%
\pgfsetbuttcap%
\pgfsetroundjoin%
\definecolor{currentfill}{rgb}{0.218130,0.347432,0.550038}%
\pgfsetfillcolor{currentfill}%
\pgfsetfillopacity{0.700000}%
\pgfsetlinewidth{0.501875pt}%
\definecolor{currentstroke}{rgb}{1.000000,1.000000,1.000000}%
\pgfsetstrokecolor{currentstroke}%
\pgfsetstrokeopacity{0.500000}%
\pgfsetdash{}{0pt}%
\pgfpathmoveto{\pgfqpoint{4.615030in}{1.938997in}}%
\pgfpathlineto{\pgfqpoint{4.626030in}{1.942310in}}%
\pgfpathlineto{\pgfqpoint{4.637024in}{1.945625in}}%
\pgfpathlineto{\pgfqpoint{4.648013in}{1.948941in}}%
\pgfpathlineto{\pgfqpoint{4.658997in}{1.952261in}}%
\pgfpathlineto{\pgfqpoint{4.669976in}{1.955586in}}%
\pgfpathlineto{\pgfqpoint{4.663444in}{1.970167in}}%
\pgfpathlineto{\pgfqpoint{4.656914in}{1.984732in}}%
\pgfpathlineto{\pgfqpoint{4.650386in}{1.999283in}}%
\pgfpathlineto{\pgfqpoint{4.643860in}{2.013824in}}%
\pgfpathlineto{\pgfqpoint{4.637337in}{2.028357in}}%
\pgfpathlineto{\pgfqpoint{4.626360in}{2.025047in}}%
\pgfpathlineto{\pgfqpoint{4.615378in}{2.021745in}}%
\pgfpathlineto{\pgfqpoint{4.604391in}{2.018452in}}%
\pgfpathlineto{\pgfqpoint{4.593398in}{2.015167in}}%
\pgfpathlineto{\pgfqpoint{4.582401in}{2.011890in}}%
\pgfpathlineto{\pgfqpoint{4.588923in}{1.997346in}}%
\pgfpathlineto{\pgfqpoint{4.595447in}{1.982785in}}%
\pgfpathlineto{\pgfqpoint{4.601973in}{1.968206in}}%
\pgfpathlineto{\pgfqpoint{4.608500in}{1.953610in}}%
\pgfpathclose%
\pgfusepath{stroke,fill}%
\end{pgfscope}%
\begin{pgfscope}%
\pgfpathrectangle{\pgfqpoint{0.887500in}{0.275000in}}{\pgfqpoint{4.225000in}{4.225000in}}%
\pgfusepath{clip}%
\pgfsetbuttcap%
\pgfsetroundjoin%
\definecolor{currentfill}{rgb}{0.185783,0.704891,0.485273}%
\pgfsetfillcolor{currentfill}%
\pgfsetfillopacity{0.700000}%
\pgfsetlinewidth{0.501875pt}%
\definecolor{currentstroke}{rgb}{1.000000,1.000000,1.000000}%
\pgfsetstrokecolor{currentstroke}%
\pgfsetstrokeopacity{0.500000}%
\pgfsetdash{}{0pt}%
\pgfpathmoveto{\pgfqpoint{3.639047in}{2.691372in}}%
\pgfpathlineto{\pgfqpoint{3.650302in}{2.695104in}}%
\pgfpathlineto{\pgfqpoint{3.661552in}{2.698808in}}%
\pgfpathlineto{\pgfqpoint{3.672796in}{2.702472in}}%
\pgfpathlineto{\pgfqpoint{3.684034in}{2.706080in}}%
\pgfpathlineto{\pgfqpoint{3.695266in}{2.709620in}}%
\pgfpathlineto{\pgfqpoint{3.688897in}{2.723352in}}%
\pgfpathlineto{\pgfqpoint{3.682531in}{2.737036in}}%
\pgfpathlineto{\pgfqpoint{3.676166in}{2.750676in}}%
\pgfpathlineto{\pgfqpoint{3.669804in}{2.764275in}}%
\pgfpathlineto{\pgfqpoint{3.663443in}{2.777835in}}%
\pgfpathlineto{\pgfqpoint{3.652215in}{2.774310in}}%
\pgfpathlineto{\pgfqpoint{3.640981in}{2.770735in}}%
\pgfpathlineto{\pgfqpoint{3.629741in}{2.767117in}}%
\pgfpathlineto{\pgfqpoint{3.618495in}{2.763465in}}%
\pgfpathlineto{\pgfqpoint{3.607244in}{2.759788in}}%
\pgfpathlineto{\pgfqpoint{3.613600in}{2.746188in}}%
\pgfpathlineto{\pgfqpoint{3.619959in}{2.732551in}}%
\pgfpathlineto{\pgfqpoint{3.626320in}{2.718873in}}%
\pgfpathlineto{\pgfqpoint{3.632682in}{2.705149in}}%
\pgfpathclose%
\pgfusepath{stroke,fill}%
\end{pgfscope}%
\begin{pgfscope}%
\pgfpathrectangle{\pgfqpoint{0.887500in}{0.275000in}}{\pgfqpoint{4.225000in}{4.225000in}}%
\pgfusepath{clip}%
\pgfsetbuttcap%
\pgfsetroundjoin%
\definecolor{currentfill}{rgb}{0.171176,0.452530,0.557965}%
\pgfsetfillcolor{currentfill}%
\pgfsetfillopacity{0.700000}%
\pgfsetlinewidth{0.501875pt}%
\definecolor{currentstroke}{rgb}{1.000000,1.000000,1.000000}%
\pgfsetstrokecolor{currentstroke}%
\pgfsetstrokeopacity{0.500000}%
\pgfsetdash{}{0pt}%
\pgfpathmoveto{\pgfqpoint{3.038000in}{2.110352in}}%
\pgfpathlineto{\pgfqpoint{3.049386in}{2.128859in}}%
\pgfpathlineto{\pgfqpoint{3.060775in}{2.150574in}}%
\pgfpathlineto{\pgfqpoint{3.072170in}{2.174586in}}%
\pgfpathlineto{\pgfqpoint{3.083573in}{2.199977in}}%
\pgfpathlineto{\pgfqpoint{3.094983in}{2.225828in}}%
\pgfpathlineto{\pgfqpoint{3.088720in}{2.236301in}}%
\pgfpathlineto{\pgfqpoint{3.082461in}{2.246840in}}%
\pgfpathlineto{\pgfqpoint{3.076206in}{2.257515in}}%
\pgfpathlineto{\pgfqpoint{3.069954in}{2.268396in}}%
\pgfpathlineto{\pgfqpoint{3.063706in}{2.279511in}}%
\pgfpathlineto{\pgfqpoint{3.052322in}{2.252201in}}%
\pgfpathlineto{\pgfqpoint{3.040946in}{2.225437in}}%
\pgfpathlineto{\pgfqpoint{3.029577in}{2.200181in}}%
\pgfpathlineto{\pgfqpoint{3.018212in}{2.177392in}}%
\pgfpathlineto{\pgfqpoint{3.006849in}{2.158026in}}%
\pgfpathlineto{\pgfqpoint{3.013071in}{2.148548in}}%
\pgfpathlineto{\pgfqpoint{3.019298in}{2.139047in}}%
\pgfpathlineto{\pgfqpoint{3.025528in}{2.129517in}}%
\pgfpathlineto{\pgfqpoint{3.031762in}{2.119954in}}%
\pgfpathclose%
\pgfusepath{stroke,fill}%
\end{pgfscope}%
\begin{pgfscope}%
\pgfpathrectangle{\pgfqpoint{0.887500in}{0.275000in}}{\pgfqpoint{4.225000in}{4.225000in}}%
\pgfusepath{clip}%
\pgfsetbuttcap%
\pgfsetroundjoin%
\definecolor{currentfill}{rgb}{0.319809,0.770914,0.411152}%
\pgfsetfillcolor{currentfill}%
\pgfsetfillopacity{0.700000}%
\pgfsetlinewidth{0.501875pt}%
\definecolor{currentstroke}{rgb}{1.000000,1.000000,1.000000}%
\pgfsetstrokecolor{currentstroke}%
\pgfsetstrokeopacity{0.500000}%
\pgfsetdash{}{0pt}%
\pgfpathmoveto{\pgfqpoint{3.140833in}{2.807879in}}%
\pgfpathlineto{\pgfqpoint{3.152232in}{2.825794in}}%
\pgfpathlineto{\pgfqpoint{3.163629in}{2.841893in}}%
\pgfpathlineto{\pgfqpoint{3.175022in}{2.856410in}}%
\pgfpathlineto{\pgfqpoint{3.186413in}{2.869582in}}%
\pgfpathlineto{\pgfqpoint{3.197799in}{2.881646in}}%
\pgfpathlineto{\pgfqpoint{3.191532in}{2.897373in}}%
\pgfpathlineto{\pgfqpoint{3.185267in}{2.913048in}}%
\pgfpathlineto{\pgfqpoint{3.179003in}{2.928542in}}%
\pgfpathlineto{\pgfqpoint{3.172740in}{2.943723in}}%
\pgfpathlineto{\pgfqpoint{3.166478in}{2.958462in}}%
\pgfpathlineto{\pgfqpoint{3.155105in}{2.948875in}}%
\pgfpathlineto{\pgfqpoint{3.143728in}{2.938000in}}%
\pgfpathlineto{\pgfqpoint{3.132346in}{2.925346in}}%
\pgfpathlineto{\pgfqpoint{3.120961in}{2.910423in}}%
\pgfpathlineto{\pgfqpoint{3.109573in}{2.892743in}}%
\pgfpathlineto{\pgfqpoint{3.115822in}{2.876221in}}%
\pgfpathlineto{\pgfqpoint{3.122073in}{2.859307in}}%
\pgfpathlineto{\pgfqpoint{3.128325in}{2.842167in}}%
\pgfpathlineto{\pgfqpoint{3.134578in}{2.824969in}}%
\pgfpathclose%
\pgfusepath{stroke,fill}%
\end{pgfscope}%
\begin{pgfscope}%
\pgfpathrectangle{\pgfqpoint{0.887500in}{0.275000in}}{\pgfqpoint{4.225000in}{4.225000in}}%
\pgfusepath{clip}%
\pgfsetbuttcap%
\pgfsetroundjoin%
\definecolor{currentfill}{rgb}{0.171176,0.452530,0.557965}%
\pgfsetfillcolor{currentfill}%
\pgfsetfillopacity{0.700000}%
\pgfsetlinewidth{0.501875pt}%
\definecolor{currentstroke}{rgb}{1.000000,1.000000,1.000000}%
\pgfsetstrokecolor{currentstroke}%
\pgfsetstrokeopacity{0.500000}%
\pgfsetdash{}{0pt}%
\pgfpathmoveto{\pgfqpoint{2.659124in}{2.156861in}}%
\pgfpathlineto{\pgfqpoint{2.670621in}{2.160383in}}%
\pgfpathlineto{\pgfqpoint{2.682113in}{2.163816in}}%
\pgfpathlineto{\pgfqpoint{2.693600in}{2.167133in}}%
\pgfpathlineto{\pgfqpoint{2.705083in}{2.170305in}}%
\pgfpathlineto{\pgfqpoint{2.716561in}{2.173355in}}%
\pgfpathlineto{\pgfqpoint{2.710425in}{2.182503in}}%
\pgfpathlineto{\pgfqpoint{2.704294in}{2.191620in}}%
\pgfpathlineto{\pgfqpoint{2.698166in}{2.200709in}}%
\pgfpathlineto{\pgfqpoint{2.692043in}{2.209769in}}%
\pgfpathlineto{\pgfqpoint{2.685924in}{2.218800in}}%
\pgfpathlineto{\pgfqpoint{2.674456in}{2.215768in}}%
\pgfpathlineto{\pgfqpoint{2.662984in}{2.212614in}}%
\pgfpathlineto{\pgfqpoint{2.651507in}{2.209317in}}%
\pgfpathlineto{\pgfqpoint{2.640026in}{2.205904in}}%
\pgfpathlineto{\pgfqpoint{2.628540in}{2.202403in}}%
\pgfpathlineto{\pgfqpoint{2.634648in}{2.193356in}}%
\pgfpathlineto{\pgfqpoint{2.640761in}{2.184280in}}%
\pgfpathlineto{\pgfqpoint{2.646878in}{2.175172in}}%
\pgfpathlineto{\pgfqpoint{2.652999in}{2.166033in}}%
\pgfpathclose%
\pgfusepath{stroke,fill}%
\end{pgfscope}%
\begin{pgfscope}%
\pgfpathrectangle{\pgfqpoint{0.887500in}{0.275000in}}{\pgfqpoint{4.225000in}{4.225000in}}%
\pgfusepath{clip}%
\pgfsetbuttcap%
\pgfsetroundjoin%
\definecolor{currentfill}{rgb}{0.156270,0.489624,0.557936}%
\pgfsetfillcolor{currentfill}%
\pgfsetfillopacity{0.700000}%
\pgfsetlinewidth{0.501875pt}%
\definecolor{currentstroke}{rgb}{1.000000,1.000000,1.000000}%
\pgfsetstrokecolor{currentstroke}%
\pgfsetstrokeopacity{0.500000}%
\pgfsetdash{}{0pt}%
\pgfpathmoveto{\pgfqpoint{4.231162in}{2.226894in}}%
\pgfpathlineto{\pgfqpoint{4.242253in}{2.229879in}}%
\pgfpathlineto{\pgfqpoint{4.253337in}{2.232789in}}%
\pgfpathlineto{\pgfqpoint{4.264413in}{2.235640in}}%
\pgfpathlineto{\pgfqpoint{4.275483in}{2.238482in}}%
\pgfpathlineto{\pgfqpoint{4.286549in}{2.241364in}}%
\pgfpathlineto{\pgfqpoint{4.280094in}{2.256197in}}%
\pgfpathlineto{\pgfqpoint{4.273644in}{2.271106in}}%
\pgfpathlineto{\pgfqpoint{4.267198in}{2.286061in}}%
\pgfpathlineto{\pgfqpoint{4.260754in}{2.301029in}}%
\pgfpathlineto{\pgfqpoint{4.254313in}{2.315978in}}%
\pgfpathlineto{\pgfqpoint{4.243238in}{2.312783in}}%
\pgfpathlineto{\pgfqpoint{4.232158in}{2.309600in}}%
\pgfpathlineto{\pgfqpoint{4.221073in}{2.306416in}}%
\pgfpathlineto{\pgfqpoint{4.209981in}{2.303216in}}%
\pgfpathlineto{\pgfqpoint{4.198884in}{2.299994in}}%
\pgfpathlineto{\pgfqpoint{4.205334in}{2.285371in}}%
\pgfpathlineto{\pgfqpoint{4.211787in}{2.270738in}}%
\pgfpathlineto{\pgfqpoint{4.218242in}{2.256106in}}%
\pgfpathlineto{\pgfqpoint{4.224700in}{2.241488in}}%
\pgfpathclose%
\pgfusepath{stroke,fill}%
\end{pgfscope}%
\begin{pgfscope}%
\pgfpathrectangle{\pgfqpoint{0.887500in}{0.275000in}}{\pgfqpoint{4.225000in}{4.225000in}}%
\pgfusepath{clip}%
\pgfsetbuttcap%
\pgfsetroundjoin%
\definecolor{currentfill}{rgb}{0.226397,0.728888,0.462789}%
\pgfsetfillcolor{currentfill}%
\pgfsetfillopacity{0.700000}%
\pgfsetlinewidth{0.501875pt}%
\definecolor{currentstroke}{rgb}{1.000000,1.000000,1.000000}%
\pgfsetstrokecolor{currentstroke}%
\pgfsetstrokeopacity{0.500000}%
\pgfsetdash{}{0pt}%
\pgfpathmoveto{\pgfqpoint{3.550905in}{2.741271in}}%
\pgfpathlineto{\pgfqpoint{3.562184in}{2.745000in}}%
\pgfpathlineto{\pgfqpoint{3.573457in}{2.748699in}}%
\pgfpathlineto{\pgfqpoint{3.584724in}{2.752396in}}%
\pgfpathlineto{\pgfqpoint{3.595987in}{2.756095in}}%
\pgfpathlineto{\pgfqpoint{3.607244in}{2.759788in}}%
\pgfpathlineto{\pgfqpoint{3.600890in}{2.773356in}}%
\pgfpathlineto{\pgfqpoint{3.594538in}{2.786898in}}%
\pgfpathlineto{\pgfqpoint{3.588188in}{2.800419in}}%
\pgfpathlineto{\pgfqpoint{3.581841in}{2.813924in}}%
\pgfpathlineto{\pgfqpoint{3.575496in}{2.827416in}}%
\pgfpathlineto{\pgfqpoint{3.564243in}{2.823774in}}%
\pgfpathlineto{\pgfqpoint{3.552985in}{2.820110in}}%
\pgfpathlineto{\pgfqpoint{3.541721in}{2.816426in}}%
\pgfpathlineto{\pgfqpoint{3.530452in}{2.812722in}}%
\pgfpathlineto{\pgfqpoint{3.519176in}{2.808991in}}%
\pgfpathlineto{\pgfqpoint{3.525517in}{2.795488in}}%
\pgfpathlineto{\pgfqpoint{3.531861in}{2.781968in}}%
\pgfpathlineto{\pgfqpoint{3.538206in}{2.768428in}}%
\pgfpathlineto{\pgfqpoint{3.544554in}{2.754864in}}%
\pgfpathclose%
\pgfusepath{stroke,fill}%
\end{pgfscope}%
\begin{pgfscope}%
\pgfpathrectangle{\pgfqpoint{0.887500in}{0.275000in}}{\pgfqpoint{4.225000in}{4.225000in}}%
\pgfusepath{clip}%
\pgfsetbuttcap%
\pgfsetroundjoin%
\definecolor{currentfill}{rgb}{0.175707,0.697900,0.491033}%
\pgfsetfillcolor{currentfill}%
\pgfsetfillopacity{0.700000}%
\pgfsetlinewidth{0.501875pt}%
\definecolor{currentstroke}{rgb}{1.000000,1.000000,1.000000}%
\pgfsetstrokecolor{currentstroke}%
\pgfsetstrokeopacity{0.500000}%
\pgfsetdash{}{0pt}%
\pgfpathmoveto{\pgfqpoint{3.115120in}{2.631240in}}%
\pgfpathlineto{\pgfqpoint{3.126521in}{2.652239in}}%
\pgfpathlineto{\pgfqpoint{3.137925in}{2.672729in}}%
\pgfpathlineto{\pgfqpoint{3.149332in}{2.692584in}}%
\pgfpathlineto{\pgfqpoint{3.160740in}{2.711679in}}%
\pgfpathlineto{\pgfqpoint{3.172150in}{2.729889in}}%
\pgfpathlineto{\pgfqpoint{3.165878in}{2.743939in}}%
\pgfpathlineto{\pgfqpoint{3.159612in}{2.758928in}}%
\pgfpathlineto{\pgfqpoint{3.153349in}{2.774693in}}%
\pgfpathlineto{\pgfqpoint{3.147090in}{2.791065in}}%
\pgfpathlineto{\pgfqpoint{3.140833in}{2.807879in}}%
\pgfpathlineto{\pgfqpoint{3.129433in}{2.787964in}}%
\pgfpathlineto{\pgfqpoint{3.118034in}{2.766247in}}%
\pgfpathlineto{\pgfqpoint{3.106638in}{2.743100in}}%
\pgfpathlineto{\pgfqpoint{3.095247in}{2.718894in}}%
\pgfpathlineto{\pgfqpoint{3.083862in}{2.694002in}}%
\pgfpathlineto{\pgfqpoint{3.090108in}{2.680741in}}%
\pgfpathlineto{\pgfqpoint{3.096356in}{2.667848in}}%
\pgfpathlineto{\pgfqpoint{3.102607in}{2.655310in}}%
\pgfpathlineto{\pgfqpoint{3.108862in}{2.643112in}}%
\pgfpathclose%
\pgfusepath{stroke,fill}%
\end{pgfscope}%
\begin{pgfscope}%
\pgfpathrectangle{\pgfqpoint{0.887500in}{0.275000in}}{\pgfqpoint{4.225000in}{4.225000in}}%
\pgfusepath{clip}%
\pgfsetbuttcap%
\pgfsetroundjoin%
\definecolor{currentfill}{rgb}{0.119738,0.603785,0.541400}%
\pgfsetfillcolor{currentfill}%
\pgfsetfillopacity{0.700000}%
\pgfsetlinewidth{0.501875pt}%
\definecolor{currentstroke}{rgb}{1.000000,1.000000,1.000000}%
\pgfsetstrokecolor{currentstroke}%
\pgfsetstrokeopacity{0.500000}%
\pgfsetdash{}{0pt}%
\pgfpathmoveto{\pgfqpoint{3.935199in}{2.464608in}}%
\pgfpathlineto{\pgfqpoint{3.946379in}{2.467994in}}%
\pgfpathlineto{\pgfqpoint{3.957552in}{2.471384in}}%
\pgfpathlineto{\pgfqpoint{3.968721in}{2.474778in}}%
\pgfpathlineto{\pgfqpoint{3.979884in}{2.478178in}}%
\pgfpathlineto{\pgfqpoint{3.991041in}{2.481581in}}%
\pgfpathlineto{\pgfqpoint{3.984622in}{2.495734in}}%
\pgfpathlineto{\pgfqpoint{3.978204in}{2.509853in}}%
\pgfpathlineto{\pgfqpoint{3.971788in}{2.523941in}}%
\pgfpathlineto{\pgfqpoint{3.965374in}{2.538004in}}%
\pgfpathlineto{\pgfqpoint{3.958963in}{2.552045in}}%
\pgfpathlineto{\pgfqpoint{3.947807in}{2.548652in}}%
\pgfpathlineto{\pgfqpoint{3.936647in}{2.545260in}}%
\pgfpathlineto{\pgfqpoint{3.925481in}{2.541871in}}%
\pgfpathlineto{\pgfqpoint{3.914310in}{2.538484in}}%
\pgfpathlineto{\pgfqpoint{3.903133in}{2.535096in}}%
\pgfpathlineto{\pgfqpoint{3.909542in}{2.521065in}}%
\pgfpathlineto{\pgfqpoint{3.915954in}{2.507005in}}%
\pgfpathlineto{\pgfqpoint{3.922367in}{2.492912in}}%
\pgfpathlineto{\pgfqpoint{3.928783in}{2.478781in}}%
\pgfpathclose%
\pgfusepath{stroke,fill}%
\end{pgfscope}%
\begin{pgfscope}%
\pgfpathrectangle{\pgfqpoint{0.887500in}{0.275000in}}{\pgfqpoint{4.225000in}{4.225000in}}%
\pgfusepath{clip}%
\pgfsetbuttcap%
\pgfsetroundjoin%
\definecolor{currentfill}{rgb}{0.136408,0.541173,0.554483}%
\pgfsetfillcolor{currentfill}%
\pgfsetfillopacity{0.700000}%
\pgfsetlinewidth{0.501875pt}%
\definecolor{currentstroke}{rgb}{1.000000,1.000000,1.000000}%
\pgfsetstrokecolor{currentstroke}%
\pgfsetstrokeopacity{0.500000}%
\pgfsetdash{}{0pt}%
\pgfpathmoveto{\pgfqpoint{1.693460in}{2.345781in}}%
\pgfpathlineto{\pgfqpoint{1.705198in}{2.349184in}}%
\pgfpathlineto{\pgfqpoint{1.716930in}{2.352581in}}%
\pgfpathlineto{\pgfqpoint{1.728656in}{2.355972in}}%
\pgfpathlineto{\pgfqpoint{1.740376in}{2.359358in}}%
\pgfpathlineto{\pgfqpoint{1.752091in}{2.362741in}}%
\pgfpathlineto{\pgfqpoint{1.746282in}{2.371189in}}%
\pgfpathlineto{\pgfqpoint{1.740478in}{2.379613in}}%
\pgfpathlineto{\pgfqpoint{1.734678in}{2.388013in}}%
\pgfpathlineto{\pgfqpoint{1.728883in}{2.396390in}}%
\pgfpathlineto{\pgfqpoint{1.723093in}{2.404743in}}%
\pgfpathlineto{\pgfqpoint{1.711390in}{2.401386in}}%
\pgfpathlineto{\pgfqpoint{1.699682in}{2.398027in}}%
\pgfpathlineto{\pgfqpoint{1.687967in}{2.394662in}}%
\pgfpathlineto{\pgfqpoint{1.676248in}{2.391293in}}%
\pgfpathlineto{\pgfqpoint{1.664522in}{2.387916in}}%
\pgfpathlineto{\pgfqpoint{1.670301in}{2.379540in}}%
\pgfpathlineto{\pgfqpoint{1.676084in}{2.371138in}}%
\pgfpathlineto{\pgfqpoint{1.681872in}{2.362711in}}%
\pgfpathlineto{\pgfqpoint{1.687664in}{2.354258in}}%
\pgfpathclose%
\pgfusepath{stroke,fill}%
\end{pgfscope}%
\begin{pgfscope}%
\pgfpathrectangle{\pgfqpoint{0.887500in}{0.275000in}}{\pgfqpoint{4.225000in}{4.225000in}}%
\pgfusepath{clip}%
\pgfsetbuttcap%
\pgfsetroundjoin%
\definecolor{currentfill}{rgb}{0.159194,0.482237,0.558073}%
\pgfsetfillcolor{currentfill}%
\pgfsetfillopacity{0.700000}%
\pgfsetlinewidth{0.501875pt}%
\definecolor{currentstroke}{rgb}{1.000000,1.000000,1.000000}%
\pgfsetstrokecolor{currentstroke}%
\pgfsetstrokeopacity{0.500000}%
\pgfsetdash{}{0pt}%
\pgfpathmoveto{\pgfqpoint{2.337207in}{2.222601in}}%
\pgfpathlineto{\pgfqpoint{2.348785in}{2.226062in}}%
\pgfpathlineto{\pgfqpoint{2.360357in}{2.229504in}}%
\pgfpathlineto{\pgfqpoint{2.371924in}{2.232926in}}%
\pgfpathlineto{\pgfqpoint{2.383486in}{2.236326in}}%
\pgfpathlineto{\pgfqpoint{2.395042in}{2.239709in}}%
\pgfpathlineto{\pgfqpoint{2.389013in}{2.248543in}}%
\pgfpathlineto{\pgfqpoint{2.382988in}{2.257353in}}%
\pgfpathlineto{\pgfqpoint{2.376967in}{2.266137in}}%
\pgfpathlineto{\pgfqpoint{2.370950in}{2.274899in}}%
\pgfpathlineto{\pgfqpoint{2.364938in}{2.283637in}}%
\pgfpathlineto{\pgfqpoint{2.353393in}{2.280291in}}%
\pgfpathlineto{\pgfqpoint{2.341842in}{2.276930in}}%
\pgfpathlineto{\pgfqpoint{2.330286in}{2.273549in}}%
\pgfpathlineto{\pgfqpoint{2.318725in}{2.270148in}}%
\pgfpathlineto{\pgfqpoint{2.307158in}{2.266730in}}%
\pgfpathlineto{\pgfqpoint{2.313159in}{2.257954in}}%
\pgfpathlineto{\pgfqpoint{2.319165in}{2.249154in}}%
\pgfpathlineto{\pgfqpoint{2.325175in}{2.240329in}}%
\pgfpathlineto{\pgfqpoint{2.331189in}{2.231478in}}%
\pgfpathclose%
\pgfusepath{stroke,fill}%
\end{pgfscope}%
\begin{pgfscope}%
\pgfpathrectangle{\pgfqpoint{0.887500in}{0.275000in}}{\pgfqpoint{4.225000in}{4.225000in}}%
\pgfusepath{clip}%
\pgfsetbuttcap%
\pgfsetroundjoin%
\definecolor{currentfill}{rgb}{0.147607,0.511733,0.557049}%
\pgfsetfillcolor{currentfill}%
\pgfsetfillopacity{0.700000}%
\pgfsetlinewidth{0.501875pt}%
\definecolor{currentstroke}{rgb}{1.000000,1.000000,1.000000}%
\pgfsetstrokecolor{currentstroke}%
\pgfsetstrokeopacity{0.500000}%
\pgfsetdash{}{0pt}%
\pgfpathmoveto{\pgfqpoint{2.015304in}{2.285286in}}%
\pgfpathlineto{\pgfqpoint{2.026963in}{2.288687in}}%
\pgfpathlineto{\pgfqpoint{2.038616in}{2.292078in}}%
\pgfpathlineto{\pgfqpoint{2.050264in}{2.295463in}}%
\pgfpathlineto{\pgfqpoint{2.061905in}{2.298842in}}%
\pgfpathlineto{\pgfqpoint{2.073541in}{2.302218in}}%
\pgfpathlineto{\pgfqpoint{2.067621in}{2.310861in}}%
\pgfpathlineto{\pgfqpoint{2.061705in}{2.319482in}}%
\pgfpathlineto{\pgfqpoint{2.055793in}{2.328082in}}%
\pgfpathlineto{\pgfqpoint{2.049885in}{2.336661in}}%
\pgfpathlineto{\pgfqpoint{2.043982in}{2.345219in}}%
\pgfpathlineto{\pgfqpoint{2.032357in}{2.341877in}}%
\pgfpathlineto{\pgfqpoint{2.020727in}{2.338533in}}%
\pgfpathlineto{\pgfqpoint{2.009090in}{2.335185in}}%
\pgfpathlineto{\pgfqpoint{1.997449in}{2.331829in}}%
\pgfpathlineto{\pgfqpoint{1.985802in}{2.328465in}}%
\pgfpathlineto{\pgfqpoint{1.991693in}{2.319875in}}%
\pgfpathlineto{\pgfqpoint{1.997590in}{2.311263in}}%
\pgfpathlineto{\pgfqpoint{2.003490in}{2.302628in}}%
\pgfpathlineto{\pgfqpoint{2.009395in}{2.293969in}}%
\pgfpathclose%
\pgfusepath{stroke,fill}%
\end{pgfscope}%
\begin{pgfscope}%
\pgfpathrectangle{\pgfqpoint{0.887500in}{0.275000in}}{\pgfqpoint{4.225000in}{4.225000in}}%
\pgfusepath{clip}%
\pgfsetbuttcap%
\pgfsetroundjoin%
\definecolor{currentfill}{rgb}{0.344074,0.780029,0.397381}%
\pgfsetfillcolor{currentfill}%
\pgfsetfillopacity{0.700000}%
\pgfsetlinewidth{0.501875pt}%
\definecolor{currentstroke}{rgb}{1.000000,1.000000,1.000000}%
\pgfsetstrokecolor{currentstroke}%
\pgfsetstrokeopacity{0.500000}%
\pgfsetdash{}{0pt}%
\pgfpathmoveto{\pgfqpoint{3.286116in}{2.862211in}}%
\pgfpathlineto{\pgfqpoint{3.297486in}{2.870966in}}%
\pgfpathlineto{\pgfqpoint{3.308849in}{2.878992in}}%
\pgfpathlineto{\pgfqpoint{3.320205in}{2.886311in}}%
\pgfpathlineto{\pgfqpoint{3.331552in}{2.892944in}}%
\pgfpathlineto{\pgfqpoint{3.342891in}{2.898946in}}%
\pgfpathlineto{\pgfqpoint{3.336588in}{2.912431in}}%
\pgfpathlineto{\pgfqpoint{3.330288in}{2.926013in}}%
\pgfpathlineto{\pgfqpoint{3.323991in}{2.939652in}}%
\pgfpathlineto{\pgfqpoint{3.317697in}{2.953308in}}%
\pgfpathlineto{\pgfqpoint{3.311405in}{2.966944in}}%
\pgfpathlineto{\pgfqpoint{3.300074in}{2.961317in}}%
\pgfpathlineto{\pgfqpoint{3.288734in}{2.955007in}}%
\pgfpathlineto{\pgfqpoint{3.277387in}{2.947982in}}%
\pgfpathlineto{\pgfqpoint{3.266033in}{2.940275in}}%
\pgfpathlineto{\pgfqpoint{3.254672in}{2.931925in}}%
\pgfpathlineto{\pgfqpoint{3.260955in}{2.917807in}}%
\pgfpathlineto{\pgfqpoint{3.267241in}{2.903711in}}%
\pgfpathlineto{\pgfqpoint{3.273529in}{2.889703in}}%
\pgfpathlineto{\pgfqpoint{3.279821in}{2.875848in}}%
\pgfpathclose%
\pgfusepath{stroke,fill}%
\end{pgfscope}%
\begin{pgfscope}%
\pgfpathrectangle{\pgfqpoint{0.887500in}{0.275000in}}{\pgfqpoint{4.225000in}{4.225000in}}%
\pgfusepath{clip}%
\pgfsetbuttcap%
\pgfsetroundjoin%
\definecolor{currentfill}{rgb}{0.203063,0.379716,0.553925}%
\pgfsetfillcolor{currentfill}%
\pgfsetfillopacity{0.700000}%
\pgfsetlinewidth{0.501875pt}%
\definecolor{currentstroke}{rgb}{1.000000,1.000000,1.000000}%
\pgfsetstrokecolor{currentstroke}%
\pgfsetstrokeopacity{0.500000}%
\pgfsetdash{}{0pt}%
\pgfpathmoveto{\pgfqpoint{4.527335in}{1.995612in}}%
\pgfpathlineto{\pgfqpoint{4.538358in}{1.998854in}}%
\pgfpathlineto{\pgfqpoint{4.549377in}{2.002103in}}%
\pgfpathlineto{\pgfqpoint{4.560390in}{2.005358in}}%
\pgfpathlineto{\pgfqpoint{4.571398in}{2.008621in}}%
\pgfpathlineto{\pgfqpoint{4.582401in}{2.011890in}}%
\pgfpathlineto{\pgfqpoint{4.575881in}{2.026417in}}%
\pgfpathlineto{\pgfqpoint{4.569362in}{2.040927in}}%
\pgfpathlineto{\pgfqpoint{4.562846in}{2.055420in}}%
\pgfpathlineto{\pgfqpoint{4.556331in}{2.069897in}}%
\pgfpathlineto{\pgfqpoint{4.549819in}{2.084357in}}%
\pgfpathlineto{\pgfqpoint{4.538815in}{2.081024in}}%
\pgfpathlineto{\pgfqpoint{4.527805in}{2.077688in}}%
\pgfpathlineto{\pgfqpoint{4.516790in}{2.074348in}}%
\pgfpathlineto{\pgfqpoint{4.505769in}{2.071002in}}%
\pgfpathlineto{\pgfqpoint{4.494742in}{2.067651in}}%
\pgfpathlineto{\pgfqpoint{4.501260in}{2.053375in}}%
\pgfpathlineto{\pgfqpoint{4.507779in}{2.039028in}}%
\pgfpathlineto{\pgfqpoint{4.514297in}{2.024613in}}%
\pgfpathlineto{\pgfqpoint{4.520815in}{2.010140in}}%
\pgfpathclose%
\pgfusepath{stroke,fill}%
\end{pgfscope}%
\begin{pgfscope}%
\pgfpathrectangle{\pgfqpoint{0.887500in}{0.275000in}}{\pgfqpoint{4.225000in}{4.225000in}}%
\pgfusepath{clip}%
\pgfsetbuttcap%
\pgfsetroundjoin%
\definecolor{currentfill}{rgb}{0.137770,0.537492,0.554906}%
\pgfsetfillcolor{currentfill}%
\pgfsetfillopacity{0.700000}%
\pgfsetlinewidth{0.501875pt}%
\definecolor{currentstroke}{rgb}{1.000000,1.000000,1.000000}%
\pgfsetstrokecolor{currentstroke}%
\pgfsetstrokeopacity{0.500000}%
\pgfsetdash{}{0pt}%
\pgfpathmoveto{\pgfqpoint{3.063706in}{2.279511in}}%
\pgfpathlineto{\pgfqpoint{3.075098in}{2.306400in}}%
\pgfpathlineto{\pgfqpoint{3.086497in}{2.332065in}}%
\pgfpathlineto{\pgfqpoint{3.097902in}{2.356524in}}%
\pgfpathlineto{\pgfqpoint{3.109312in}{2.380047in}}%
\pgfpathlineto{\pgfqpoint{3.120728in}{2.402907in}}%
\pgfpathlineto{\pgfqpoint{3.114460in}{2.415508in}}%
\pgfpathlineto{\pgfqpoint{3.108196in}{2.428123in}}%
\pgfpathlineto{\pgfqpoint{3.101935in}{2.440684in}}%
\pgfpathlineto{\pgfqpoint{3.095676in}{2.453121in}}%
\pgfpathlineto{\pgfqpoint{3.089420in}{2.465366in}}%
\pgfpathlineto{\pgfqpoint{3.078026in}{2.442072in}}%
\pgfpathlineto{\pgfqpoint{3.066637in}{2.417753in}}%
\pgfpathlineto{\pgfqpoint{3.055254in}{2.392142in}}%
\pgfpathlineto{\pgfqpoint{3.043879in}{2.364974in}}%
\pgfpathlineto{\pgfqpoint{3.032514in}{2.336263in}}%
\pgfpathlineto{\pgfqpoint{3.038746in}{2.324996in}}%
\pgfpathlineto{\pgfqpoint{3.044981in}{2.313615in}}%
\pgfpathlineto{\pgfqpoint{3.051219in}{2.302193in}}%
\pgfpathlineto{\pgfqpoint{3.057461in}{2.290801in}}%
\pgfpathclose%
\pgfusepath{stroke,fill}%
\end{pgfscope}%
\begin{pgfscope}%
\pgfpathrectangle{\pgfqpoint{0.887500in}{0.275000in}}{\pgfqpoint{4.225000in}{4.225000in}}%
\pgfusepath{clip}%
\pgfsetbuttcap%
\pgfsetroundjoin%
\definecolor{currentfill}{rgb}{0.266941,0.748751,0.440573}%
\pgfsetfillcolor{currentfill}%
\pgfsetfillopacity{0.700000}%
\pgfsetlinewidth{0.501875pt}%
\definecolor{currentstroke}{rgb}{1.000000,1.000000,1.000000}%
\pgfsetstrokecolor{currentstroke}%
\pgfsetstrokeopacity{0.500000}%
\pgfsetdash{}{0pt}%
\pgfpathmoveto{\pgfqpoint{3.462715in}{2.789692in}}%
\pgfpathlineto{\pgfqpoint{3.474019in}{2.793663in}}%
\pgfpathlineto{\pgfqpoint{3.485317in}{2.797570in}}%
\pgfpathlineto{\pgfqpoint{3.496609in}{2.801423in}}%
\pgfpathlineto{\pgfqpoint{3.507896in}{2.805228in}}%
\pgfpathlineto{\pgfqpoint{3.519176in}{2.808991in}}%
\pgfpathlineto{\pgfqpoint{3.512838in}{2.822474in}}%
\pgfpathlineto{\pgfqpoint{3.506502in}{2.835934in}}%
\pgfpathlineto{\pgfqpoint{3.500168in}{2.849365in}}%
\pgfpathlineto{\pgfqpoint{3.493837in}{2.862765in}}%
\pgfpathlineto{\pgfqpoint{3.487508in}{2.876130in}}%
\pgfpathlineto{\pgfqpoint{3.476230in}{2.872313in}}%
\pgfpathlineto{\pgfqpoint{3.464947in}{2.868446in}}%
\pgfpathlineto{\pgfqpoint{3.453658in}{2.864526in}}%
\pgfpathlineto{\pgfqpoint{3.442364in}{2.860550in}}%
\pgfpathlineto{\pgfqpoint{3.431063in}{2.856515in}}%
\pgfpathlineto{\pgfqpoint{3.437389in}{2.843253in}}%
\pgfpathlineto{\pgfqpoint{3.443717in}{2.829949in}}%
\pgfpathlineto{\pgfqpoint{3.450048in}{2.816595in}}%
\pgfpathlineto{\pgfqpoint{3.456380in}{2.803179in}}%
\pgfpathclose%
\pgfusepath{stroke,fill}%
\end{pgfscope}%
\begin{pgfscope}%
\pgfpathrectangle{\pgfqpoint{0.887500in}{0.275000in}}{\pgfqpoint{4.225000in}{4.225000in}}%
\pgfusepath{clip}%
\pgfsetbuttcap%
\pgfsetroundjoin%
\definecolor{currentfill}{rgb}{0.120081,0.622161,0.534946}%
\pgfsetfillcolor{currentfill}%
\pgfsetfillopacity{0.700000}%
\pgfsetlinewidth{0.501875pt}%
\definecolor{currentstroke}{rgb}{1.000000,1.000000,1.000000}%
\pgfsetstrokecolor{currentstroke}%
\pgfsetstrokeopacity{0.500000}%
\pgfsetdash{}{0pt}%
\pgfpathmoveto{\pgfqpoint{3.089420in}{2.465366in}}%
\pgfpathlineto{\pgfqpoint{3.100820in}{2.487901in}}%
\pgfpathlineto{\pgfqpoint{3.112224in}{2.509947in}}%
\pgfpathlineto{\pgfqpoint{3.123633in}{2.531773in}}%
\pgfpathlineto{\pgfqpoint{3.135047in}{2.553555in}}%
\pgfpathlineto{\pgfqpoint{3.146465in}{2.575163in}}%
\pgfpathlineto{\pgfqpoint{3.140189in}{2.586231in}}%
\pgfpathlineto{\pgfqpoint{3.133917in}{2.597272in}}%
\pgfpathlineto{\pgfqpoint{3.127648in}{2.608389in}}%
\pgfpathlineto{\pgfqpoint{3.121382in}{2.619680in}}%
\pgfpathlineto{\pgfqpoint{3.115120in}{2.631240in}}%
\pgfpathlineto{\pgfqpoint{3.103724in}{2.609855in}}%
\pgfpathlineto{\pgfqpoint{3.092332in}{2.588209in}}%
\pgfpathlineto{\pgfqpoint{3.080944in}{2.566373in}}%
\pgfpathlineto{\pgfqpoint{3.069562in}{2.544159in}}%
\pgfpathlineto{\pgfqpoint{3.058185in}{2.521301in}}%
\pgfpathlineto{\pgfqpoint{3.064425in}{2.511038in}}%
\pgfpathlineto{\pgfqpoint{3.070670in}{2.500254in}}%
\pgfpathlineto{\pgfqpoint{3.076917in}{2.489001in}}%
\pgfpathlineto{\pgfqpoint{3.083167in}{2.477349in}}%
\pgfpathclose%
\pgfusepath{stroke,fill}%
\end{pgfscope}%
\begin{pgfscope}%
\pgfpathrectangle{\pgfqpoint{0.887500in}{0.275000in}}{\pgfqpoint{4.225000in}{4.225000in}}%
\pgfusepath{clip}%
\pgfsetbuttcap%
\pgfsetroundjoin%
\definecolor{currentfill}{rgb}{0.311925,0.767822,0.415586}%
\pgfsetfillcolor{currentfill}%
\pgfsetfillopacity{0.700000}%
\pgfsetlinewidth{0.501875pt}%
\definecolor{currentstroke}{rgb}{1.000000,1.000000,1.000000}%
\pgfsetstrokecolor{currentstroke}%
\pgfsetstrokeopacity{0.500000}%
\pgfsetdash{}{0pt}%
\pgfpathmoveto{\pgfqpoint{3.374460in}{2.833241in}}%
\pgfpathlineto{\pgfqpoint{3.385796in}{2.838591in}}%
\pgfpathlineto{\pgfqpoint{3.397124in}{2.843508in}}%
\pgfpathlineto{\pgfqpoint{3.408444in}{2.848078in}}%
\pgfpathlineto{\pgfqpoint{3.419757in}{2.852385in}}%
\pgfpathlineto{\pgfqpoint{3.431063in}{2.856515in}}%
\pgfpathlineto{\pgfqpoint{3.424740in}{2.869747in}}%
\pgfpathlineto{\pgfqpoint{3.418419in}{2.882958in}}%
\pgfpathlineto{\pgfqpoint{3.412100in}{2.896159in}}%
\pgfpathlineto{\pgfqpoint{3.405785in}{2.909360in}}%
\pgfpathlineto{\pgfqpoint{3.399472in}{2.922569in}}%
\pgfpathlineto{\pgfqpoint{3.388170in}{2.918400in}}%
\pgfpathlineto{\pgfqpoint{3.376861in}{2.914042in}}%
\pgfpathlineto{\pgfqpoint{3.365545in}{2.909406in}}%
\pgfpathlineto{\pgfqpoint{3.354222in}{2.904404in}}%
\pgfpathlineto{\pgfqpoint{3.342891in}{2.898946in}}%
\pgfpathlineto{\pgfqpoint{3.349198in}{2.885597in}}%
\pgfpathlineto{\pgfqpoint{3.355508in}{2.872394in}}%
\pgfpathlineto{\pgfqpoint{3.361822in}{2.859298in}}%
\pgfpathlineto{\pgfqpoint{3.368140in}{2.846262in}}%
\pgfpathclose%
\pgfusepath{stroke,fill}%
\end{pgfscope}%
\begin{pgfscope}%
\pgfpathrectangle{\pgfqpoint{0.887500in}{0.275000in}}{\pgfqpoint{4.225000in}{4.225000in}}%
\pgfusepath{clip}%
\pgfsetbuttcap%
\pgfsetroundjoin%
\definecolor{currentfill}{rgb}{0.175841,0.441290,0.557685}%
\pgfsetfillcolor{currentfill}%
\pgfsetfillopacity{0.700000}%
\pgfsetlinewidth{0.501875pt}%
\definecolor{currentstroke}{rgb}{1.000000,1.000000,1.000000}%
\pgfsetstrokecolor{currentstroke}%
\pgfsetstrokeopacity{0.500000}%
\pgfsetdash{}{0pt}%
\pgfpathmoveto{\pgfqpoint{2.747301in}{2.127154in}}%
\pgfpathlineto{\pgfqpoint{2.758783in}{2.130213in}}%
\pgfpathlineto{\pgfqpoint{2.770259in}{2.133322in}}%
\pgfpathlineto{\pgfqpoint{2.781728in}{2.136566in}}%
\pgfpathlineto{\pgfqpoint{2.793190in}{2.140030in}}%
\pgfpathlineto{\pgfqpoint{2.804645in}{2.143801in}}%
\pgfpathlineto{\pgfqpoint{2.798478in}{2.153074in}}%
\pgfpathlineto{\pgfqpoint{2.792316in}{2.162314in}}%
\pgfpathlineto{\pgfqpoint{2.786158in}{2.171524in}}%
\pgfpathlineto{\pgfqpoint{2.780003in}{2.180704in}}%
\pgfpathlineto{\pgfqpoint{2.773853in}{2.189855in}}%
\pgfpathlineto{\pgfqpoint{2.762410in}{2.186088in}}%
\pgfpathlineto{\pgfqpoint{2.750958in}{2.182644in}}%
\pgfpathlineto{\pgfqpoint{2.739499in}{2.179434in}}%
\pgfpathlineto{\pgfqpoint{2.728033in}{2.176368in}}%
\pgfpathlineto{\pgfqpoint{2.716561in}{2.173355in}}%
\pgfpathlineto{\pgfqpoint{2.722701in}{2.164178in}}%
\pgfpathlineto{\pgfqpoint{2.728844in}{2.154971in}}%
\pgfpathlineto{\pgfqpoint{2.734992in}{2.145732in}}%
\pgfpathlineto{\pgfqpoint{2.741144in}{2.136460in}}%
\pgfpathclose%
\pgfusepath{stroke,fill}%
\end{pgfscope}%
\begin{pgfscope}%
\pgfpathrectangle{\pgfqpoint{0.887500in}{0.275000in}}{\pgfqpoint{4.225000in}{4.225000in}}%
\pgfusepath{clip}%
\pgfsetbuttcap%
\pgfsetroundjoin%
\definecolor{currentfill}{rgb}{0.121380,0.629492,0.531973}%
\pgfsetfillcolor{currentfill}%
\pgfsetfillopacity{0.700000}%
\pgfsetlinewidth{0.501875pt}%
\definecolor{currentstroke}{rgb}{1.000000,1.000000,1.000000}%
\pgfsetstrokecolor{currentstroke}%
\pgfsetstrokeopacity{0.500000}%
\pgfsetdash{}{0pt}%
\pgfpathmoveto{\pgfqpoint{3.847164in}{2.518041in}}%
\pgfpathlineto{\pgfqpoint{3.858369in}{2.521477in}}%
\pgfpathlineto{\pgfqpoint{3.869569in}{2.524897in}}%
\pgfpathlineto{\pgfqpoint{3.880762in}{2.528305in}}%
\pgfpathlineto{\pgfqpoint{3.891950in}{2.531704in}}%
\pgfpathlineto{\pgfqpoint{3.903133in}{2.535096in}}%
\pgfpathlineto{\pgfqpoint{3.896725in}{2.549103in}}%
\pgfpathlineto{\pgfqpoint{3.890320in}{2.563089in}}%
\pgfpathlineto{\pgfqpoint{3.883916in}{2.577061in}}%
\pgfpathlineto{\pgfqpoint{3.877516in}{2.591020in}}%
\pgfpathlineto{\pgfqpoint{3.871117in}{2.604966in}}%
\pgfpathlineto{\pgfqpoint{3.859937in}{2.601559in}}%
\pgfpathlineto{\pgfqpoint{3.848751in}{2.598148in}}%
\pgfpathlineto{\pgfqpoint{3.837560in}{2.594729in}}%
\pgfpathlineto{\pgfqpoint{3.826363in}{2.591298in}}%
\pgfpathlineto{\pgfqpoint{3.815160in}{2.587851in}}%
\pgfpathlineto{\pgfqpoint{3.821556in}{2.573894in}}%
\pgfpathlineto{\pgfqpoint{3.827954in}{2.559936in}}%
\pgfpathlineto{\pgfqpoint{3.834355in}{2.545979in}}%
\pgfpathlineto{\pgfqpoint{3.840758in}{2.532017in}}%
\pgfpathclose%
\pgfusepath{stroke,fill}%
\end{pgfscope}%
\begin{pgfscope}%
\pgfpathrectangle{\pgfqpoint{0.887500in}{0.275000in}}{\pgfqpoint{4.225000in}{4.225000in}}%
\pgfusepath{clip}%
\pgfsetbuttcap%
\pgfsetroundjoin%
\definecolor{currentfill}{rgb}{0.144759,0.519093,0.556572}%
\pgfsetfillcolor{currentfill}%
\pgfsetfillopacity{0.700000}%
\pgfsetlinewidth{0.501875pt}%
\definecolor{currentstroke}{rgb}{1.000000,1.000000,1.000000}%
\pgfsetstrokecolor{currentstroke}%
\pgfsetstrokeopacity{0.500000}%
\pgfsetdash{}{0pt}%
\pgfpathmoveto{\pgfqpoint{4.143303in}{2.283528in}}%
\pgfpathlineto{\pgfqpoint{4.154431in}{2.286869in}}%
\pgfpathlineto{\pgfqpoint{4.165554in}{2.290186in}}%
\pgfpathlineto{\pgfqpoint{4.176670in}{2.293479in}}%
\pgfpathlineto{\pgfqpoint{4.187780in}{2.296748in}}%
\pgfpathlineto{\pgfqpoint{4.198884in}{2.299994in}}%
\pgfpathlineto{\pgfqpoint{4.192435in}{2.314596in}}%
\pgfpathlineto{\pgfqpoint{4.185988in}{2.329165in}}%
\pgfpathlineto{\pgfqpoint{4.179542in}{2.343694in}}%
\pgfpathlineto{\pgfqpoint{4.173097in}{2.358182in}}%
\pgfpathlineto{\pgfqpoint{4.166654in}{2.372631in}}%
\pgfpathlineto{\pgfqpoint{4.155548in}{2.369252in}}%
\pgfpathlineto{\pgfqpoint{4.144436in}{2.365868in}}%
\pgfpathlineto{\pgfqpoint{4.133319in}{2.362473in}}%
\pgfpathlineto{\pgfqpoint{4.122195in}{2.359063in}}%
\pgfpathlineto{\pgfqpoint{4.111066in}{2.355632in}}%
\pgfpathlineto{\pgfqpoint{4.117508in}{2.341222in}}%
\pgfpathlineto{\pgfqpoint{4.123953in}{2.326804in}}%
\pgfpathlineto{\pgfqpoint{4.130401in}{2.312382in}}%
\pgfpathlineto{\pgfqpoint{4.136850in}{2.297957in}}%
\pgfpathclose%
\pgfusepath{stroke,fill}%
\end{pgfscope}%
\begin{pgfscope}%
\pgfpathrectangle{\pgfqpoint{0.887500in}{0.275000in}}{\pgfqpoint{4.225000in}{4.225000in}}%
\pgfusepath{clip}%
\pgfsetbuttcap%
\pgfsetroundjoin%
\definecolor{currentfill}{rgb}{0.188923,0.410910,0.556326}%
\pgfsetfillcolor{currentfill}%
\pgfsetfillopacity{0.700000}%
\pgfsetlinewidth{0.501875pt}%
\definecolor{currentstroke}{rgb}{1.000000,1.000000,1.000000}%
\pgfsetstrokecolor{currentstroke}%
\pgfsetstrokeopacity{0.500000}%
\pgfsetdash{}{0pt}%
\pgfpathmoveto{\pgfqpoint{4.439520in}{2.050703in}}%
\pgfpathlineto{\pgfqpoint{4.450577in}{2.054130in}}%
\pgfpathlineto{\pgfqpoint{4.461627in}{2.057535in}}%
\pgfpathlineto{\pgfqpoint{4.472672in}{2.060921in}}%
\pgfpathlineto{\pgfqpoint{4.483710in}{2.064291in}}%
\pgfpathlineto{\pgfqpoint{4.494742in}{2.067651in}}%
\pgfpathlineto{\pgfqpoint{4.488224in}{2.081848in}}%
\pgfpathlineto{\pgfqpoint{4.481705in}{2.095964in}}%
\pgfpathlineto{\pgfqpoint{4.475187in}{2.110011in}}%
\pgfpathlineto{\pgfqpoint{4.468669in}{2.124004in}}%
\pgfpathlineto{\pgfqpoint{4.462153in}{2.137959in}}%
\pgfpathlineto{\pgfqpoint{4.451105in}{2.134107in}}%
\pgfpathlineto{\pgfqpoint{4.440051in}{2.130228in}}%
\pgfpathlineto{\pgfqpoint{4.428991in}{2.126351in}}%
\pgfpathlineto{\pgfqpoint{4.417928in}{2.122504in}}%
\pgfpathlineto{\pgfqpoint{4.406861in}{2.118717in}}%
\pgfpathlineto{\pgfqpoint{4.413394in}{2.105299in}}%
\pgfpathlineto{\pgfqpoint{4.419927in}{2.091826in}}%
\pgfpathlineto{\pgfqpoint{4.426460in}{2.078261in}}%
\pgfpathlineto{\pgfqpoint{4.432992in}{2.064564in}}%
\pgfpathclose%
\pgfusepath{stroke,fill}%
\end{pgfscope}%
\begin{pgfscope}%
\pgfpathrectangle{\pgfqpoint{0.887500in}{0.275000in}}{\pgfqpoint{4.225000in}{4.225000in}}%
\pgfusepath{clip}%
\pgfsetbuttcap%
\pgfsetroundjoin%
\definecolor{currentfill}{rgb}{0.246811,0.283237,0.535941}%
\pgfsetfillcolor{currentfill}%
\pgfsetfillopacity{0.700000}%
\pgfsetlinewidth{0.501875pt}%
\definecolor{currentstroke}{rgb}{1.000000,1.000000,1.000000}%
\pgfsetstrokecolor{currentstroke}%
\pgfsetstrokeopacity{0.500000}%
\pgfsetdash{}{0pt}%
\pgfpathmoveto{\pgfqpoint{4.735370in}{1.808268in}}%
\pgfpathlineto{\pgfqpoint{4.746344in}{1.811575in}}%
\pgfpathlineto{\pgfqpoint{4.757312in}{1.814893in}}%
\pgfpathlineto{\pgfqpoint{4.768276in}{1.818223in}}%
\pgfpathlineto{\pgfqpoint{4.779234in}{1.821566in}}%
\pgfpathlineto{\pgfqpoint{4.772691in}{1.836494in}}%
\pgfpathlineto{\pgfqpoint{4.766149in}{1.851372in}}%
\pgfpathlineto{\pgfqpoint{4.759607in}{1.866201in}}%
\pgfpathlineto{\pgfqpoint{4.753065in}{1.880986in}}%
\pgfpathlineto{\pgfqpoint{4.746525in}{1.895731in}}%
\pgfpathlineto{\pgfqpoint{4.735566in}{1.892370in}}%
\pgfpathlineto{\pgfqpoint{4.724602in}{1.889019in}}%
\pgfpathlineto{\pgfqpoint{4.713633in}{1.885677in}}%
\pgfpathlineto{\pgfqpoint{4.702659in}{1.882339in}}%
\pgfpathlineto{\pgfqpoint{4.709199in}{1.867602in}}%
\pgfpathlineto{\pgfqpoint{4.715741in}{1.852829in}}%
\pgfpathlineto{\pgfqpoint{4.722283in}{1.838018in}}%
\pgfpathlineto{\pgfqpoint{4.728827in}{1.823165in}}%
\pgfpathclose%
\pgfusepath{stroke,fill}%
\end{pgfscope}%
\begin{pgfscope}%
\pgfpathrectangle{\pgfqpoint{0.887500in}{0.275000in}}{\pgfqpoint{4.225000in}{4.225000in}}%
\pgfusepath{clip}%
\pgfsetbuttcap%
\pgfsetroundjoin%
\definecolor{currentfill}{rgb}{0.163625,0.471133,0.558148}%
\pgfsetfillcolor{currentfill}%
\pgfsetfillopacity{0.700000}%
\pgfsetlinewidth{0.501875pt}%
\definecolor{currentstroke}{rgb}{1.000000,1.000000,1.000000}%
\pgfsetstrokecolor{currentstroke}%
\pgfsetstrokeopacity{0.500000}%
\pgfsetdash{}{0pt}%
\pgfpathmoveto{\pgfqpoint{2.425252in}{2.195143in}}%
\pgfpathlineto{\pgfqpoint{2.436814in}{2.198556in}}%
\pgfpathlineto{\pgfqpoint{2.448369in}{2.201960in}}%
\pgfpathlineto{\pgfqpoint{2.459919in}{2.205360in}}%
\pgfpathlineto{\pgfqpoint{2.471464in}{2.208761in}}%
\pgfpathlineto{\pgfqpoint{2.483002in}{2.212168in}}%
\pgfpathlineto{\pgfqpoint{2.476941in}{2.221094in}}%
\pgfpathlineto{\pgfqpoint{2.470884in}{2.229993in}}%
\pgfpathlineto{\pgfqpoint{2.464831in}{2.238866in}}%
\pgfpathlineto{\pgfqpoint{2.458782in}{2.247712in}}%
\pgfpathlineto{\pgfqpoint{2.452738in}{2.256533in}}%
\pgfpathlineto{\pgfqpoint{2.441210in}{2.253165in}}%
\pgfpathlineto{\pgfqpoint{2.429677in}{2.249803in}}%
\pgfpathlineto{\pgfqpoint{2.418138in}{2.246443in}}%
\pgfpathlineto{\pgfqpoint{2.406593in}{2.243080in}}%
\pgfpathlineto{\pgfqpoint{2.395042in}{2.239709in}}%
\pgfpathlineto{\pgfqpoint{2.401076in}{2.230850in}}%
\pgfpathlineto{\pgfqpoint{2.407113in}{2.221964in}}%
\pgfpathlineto{\pgfqpoint{2.413155in}{2.213051in}}%
\pgfpathlineto{\pgfqpoint{2.419202in}{2.204111in}}%
\pgfpathclose%
\pgfusepath{stroke,fill}%
\end{pgfscope}%
\begin{pgfscope}%
\pgfpathrectangle{\pgfqpoint{0.887500in}{0.275000in}}{\pgfqpoint{4.225000in}{4.225000in}}%
\pgfusepath{clip}%
\pgfsetbuttcap%
\pgfsetroundjoin%
\definecolor{currentfill}{rgb}{0.140536,0.530132,0.555659}%
\pgfsetfillcolor{currentfill}%
\pgfsetfillopacity{0.700000}%
\pgfsetlinewidth{0.501875pt}%
\definecolor{currentstroke}{rgb}{1.000000,1.000000,1.000000}%
\pgfsetstrokecolor{currentstroke}%
\pgfsetstrokeopacity{0.500000}%
\pgfsetdash{}{0pt}%
\pgfpathmoveto{\pgfqpoint{1.781201in}{2.320151in}}%
\pgfpathlineto{\pgfqpoint{1.792922in}{2.323567in}}%
\pgfpathlineto{\pgfqpoint{1.804637in}{2.326981in}}%
\pgfpathlineto{\pgfqpoint{1.816346in}{2.330396in}}%
\pgfpathlineto{\pgfqpoint{1.828049in}{2.333811in}}%
\pgfpathlineto{\pgfqpoint{1.839746in}{2.337228in}}%
\pgfpathlineto{\pgfqpoint{1.833903in}{2.345755in}}%
\pgfpathlineto{\pgfqpoint{1.828065in}{2.354261in}}%
\pgfpathlineto{\pgfqpoint{1.822232in}{2.362745in}}%
\pgfpathlineto{\pgfqpoint{1.816402in}{2.371207in}}%
\pgfpathlineto{\pgfqpoint{1.810577in}{2.379648in}}%
\pgfpathlineto{\pgfqpoint{1.798892in}{2.376265in}}%
\pgfpathlineto{\pgfqpoint{1.787200in}{2.372883in}}%
\pgfpathlineto{\pgfqpoint{1.775503in}{2.369503in}}%
\pgfpathlineto{\pgfqpoint{1.763800in}{2.366123in}}%
\pgfpathlineto{\pgfqpoint{1.752091in}{2.362741in}}%
\pgfpathlineto{\pgfqpoint{1.757904in}{2.354270in}}%
\pgfpathlineto{\pgfqpoint{1.763722in}{2.345776in}}%
\pgfpathlineto{\pgfqpoint{1.769544in}{2.337257in}}%
\pgfpathlineto{\pgfqpoint{1.775370in}{2.328716in}}%
\pgfpathclose%
\pgfusepath{stroke,fill}%
\end{pgfscope}%
\begin{pgfscope}%
\pgfpathrectangle{\pgfqpoint{0.887500in}{0.275000in}}{\pgfqpoint{4.225000in}{4.225000in}}%
\pgfusepath{clip}%
\pgfsetbuttcap%
\pgfsetroundjoin%
\definecolor{currentfill}{rgb}{0.151918,0.500685,0.557587}%
\pgfsetfillcolor{currentfill}%
\pgfsetfillopacity{0.700000}%
\pgfsetlinewidth{0.501875pt}%
\definecolor{currentstroke}{rgb}{1.000000,1.000000,1.000000}%
\pgfsetstrokecolor{currentstroke}%
\pgfsetstrokeopacity{0.500000}%
\pgfsetdash{}{0pt}%
\pgfpathmoveto{\pgfqpoint{2.103209in}{2.258671in}}%
\pgfpathlineto{\pgfqpoint{2.114850in}{2.262091in}}%
\pgfpathlineto{\pgfqpoint{2.126486in}{2.265512in}}%
\pgfpathlineto{\pgfqpoint{2.138115in}{2.268939in}}%
\pgfpathlineto{\pgfqpoint{2.149739in}{2.272374in}}%
\pgfpathlineto{\pgfqpoint{2.161356in}{2.275820in}}%
\pgfpathlineto{\pgfqpoint{2.155403in}{2.284533in}}%
\pgfpathlineto{\pgfqpoint{2.149454in}{2.293223in}}%
\pgfpathlineto{\pgfqpoint{2.143509in}{2.301891in}}%
\pgfpathlineto{\pgfqpoint{2.137569in}{2.310538in}}%
\pgfpathlineto{\pgfqpoint{2.131633in}{2.319164in}}%
\pgfpathlineto{\pgfqpoint{2.120026in}{2.315757in}}%
\pgfpathlineto{\pgfqpoint{2.108414in}{2.312362in}}%
\pgfpathlineto{\pgfqpoint{2.096796in}{2.308976in}}%
\pgfpathlineto{\pgfqpoint{2.085171in}{2.305595in}}%
\pgfpathlineto{\pgfqpoint{2.073541in}{2.302218in}}%
\pgfpathlineto{\pgfqpoint{2.079466in}{2.293554in}}%
\pgfpathlineto{\pgfqpoint{2.085395in}{2.284868in}}%
\pgfpathlineto{\pgfqpoint{2.091329in}{2.276159in}}%
\pgfpathlineto{\pgfqpoint{2.097267in}{2.267427in}}%
\pgfpathclose%
\pgfusepath{stroke,fill}%
\end{pgfscope}%
\begin{pgfscope}%
\pgfpathrectangle{\pgfqpoint{0.887500in}{0.275000in}}{\pgfqpoint{4.225000in}{4.225000in}}%
\pgfusepath{clip}%
\pgfsetbuttcap%
\pgfsetroundjoin%
\definecolor{currentfill}{rgb}{0.132268,0.655014,0.519661}%
\pgfsetfillcolor{currentfill}%
\pgfsetfillopacity{0.700000}%
\pgfsetlinewidth{0.501875pt}%
\definecolor{currentstroke}{rgb}{1.000000,1.000000,1.000000}%
\pgfsetstrokecolor{currentstroke}%
\pgfsetstrokeopacity{0.500000}%
\pgfsetdash{}{0pt}%
\pgfpathmoveto{\pgfqpoint{3.759055in}{2.570122in}}%
\pgfpathlineto{\pgfqpoint{3.770289in}{2.573761in}}%
\pgfpathlineto{\pgfqpoint{3.781516in}{2.577345in}}%
\pgfpathlineto{\pgfqpoint{3.792737in}{2.580882in}}%
\pgfpathlineto{\pgfqpoint{3.803951in}{2.584382in}}%
\pgfpathlineto{\pgfqpoint{3.815160in}{2.587851in}}%
\pgfpathlineto{\pgfqpoint{3.808767in}{2.601803in}}%
\pgfpathlineto{\pgfqpoint{3.802376in}{2.615748in}}%
\pgfpathlineto{\pgfqpoint{3.795987in}{2.629681in}}%
\pgfpathlineto{\pgfqpoint{3.789601in}{2.643600in}}%
\pgfpathlineto{\pgfqpoint{3.783217in}{2.657502in}}%
\pgfpathlineto{\pgfqpoint{3.772013in}{2.654107in}}%
\pgfpathlineto{\pgfqpoint{3.760802in}{2.650691in}}%
\pgfpathlineto{\pgfqpoint{3.749586in}{2.647245in}}%
\pgfpathlineto{\pgfqpoint{3.738363in}{2.643755in}}%
\pgfpathlineto{\pgfqpoint{3.727134in}{2.640213in}}%
\pgfpathlineto{\pgfqpoint{3.733514in}{2.626217in}}%
\pgfpathlineto{\pgfqpoint{3.739895in}{2.612201in}}%
\pgfpathlineto{\pgfqpoint{3.746280in}{2.598175in}}%
\pgfpathlineto{\pgfqpoint{3.752666in}{2.584145in}}%
\pgfpathclose%
\pgfusepath{stroke,fill}%
\end{pgfscope}%
\begin{pgfscope}%
\pgfpathrectangle{\pgfqpoint{0.887500in}{0.275000in}}{\pgfqpoint{4.225000in}{4.225000in}}%
\pgfusepath{clip}%
\pgfsetbuttcap%
\pgfsetroundjoin%
\definecolor{currentfill}{rgb}{0.179019,0.433756,0.557430}%
\pgfsetfillcolor{currentfill}%
\pgfsetfillopacity{0.700000}%
\pgfsetlinewidth{0.501875pt}%
\definecolor{currentstroke}{rgb}{1.000000,1.000000,1.000000}%
\pgfsetstrokecolor{currentstroke}%
\pgfsetstrokeopacity{0.500000}%
\pgfsetdash{}{0pt}%
\pgfpathmoveto{\pgfqpoint{4.351498in}{2.101250in}}%
\pgfpathlineto{\pgfqpoint{4.362577in}{2.104589in}}%
\pgfpathlineto{\pgfqpoint{4.373651in}{2.107973in}}%
\pgfpathlineto{\pgfqpoint{4.384722in}{2.111438in}}%
\pgfpathlineto{\pgfqpoint{4.395792in}{2.115019in}}%
\pgfpathlineto{\pgfqpoint{4.406861in}{2.118717in}}%
\pgfpathlineto{\pgfqpoint{4.400331in}{2.132122in}}%
\pgfpathlineto{\pgfqpoint{4.393805in}{2.145551in}}%
\pgfpathlineto{\pgfqpoint{4.387284in}{2.159045in}}%
\pgfpathlineto{\pgfqpoint{4.380769in}{2.172642in}}%
\pgfpathlineto{\pgfqpoint{4.374262in}{2.186382in}}%
\pgfpathlineto{\pgfqpoint{4.363190in}{2.182496in}}%
\pgfpathlineto{\pgfqpoint{4.352121in}{2.178876in}}%
\pgfpathlineto{\pgfqpoint{4.341057in}{2.175528in}}%
\pgfpathlineto{\pgfqpoint{4.329993in}{2.172395in}}%
\pgfpathlineto{\pgfqpoint{4.318929in}{2.169415in}}%
\pgfpathlineto{\pgfqpoint{4.325431in}{2.155577in}}%
\pgfpathlineto{\pgfqpoint{4.331939in}{2.141876in}}%
\pgfpathlineto{\pgfqpoint{4.338454in}{2.128277in}}%
\pgfpathlineto{\pgfqpoint{4.344974in}{2.114747in}}%
\pgfpathclose%
\pgfusepath{stroke,fill}%
\end{pgfscope}%
\begin{pgfscope}%
\pgfpathrectangle{\pgfqpoint{0.887500in}{0.275000in}}{\pgfqpoint{4.225000in}{4.225000in}}%
\pgfusepath{clip}%
\pgfsetbuttcap%
\pgfsetroundjoin%
\definecolor{currentfill}{rgb}{0.311925,0.767822,0.415586}%
\pgfsetfillcolor{currentfill}%
\pgfsetfillopacity{0.700000}%
\pgfsetlinewidth{0.501875pt}%
\definecolor{currentstroke}{rgb}{1.000000,1.000000,1.000000}%
\pgfsetstrokecolor{currentstroke}%
\pgfsetstrokeopacity{0.500000}%
\pgfsetdash{}{0pt}%
\pgfpathmoveto{\pgfqpoint{3.229179in}{2.806790in}}%
\pgfpathlineto{\pgfqpoint{3.240576in}{2.819498in}}%
\pgfpathlineto{\pgfqpoint{3.251969in}{2.831371in}}%
\pgfpathlineto{\pgfqpoint{3.263357in}{2.842434in}}%
\pgfpathlineto{\pgfqpoint{3.274740in}{2.852707in}}%
\pgfpathlineto{\pgfqpoint{3.286116in}{2.862211in}}%
\pgfpathlineto{\pgfqpoint{3.279821in}{2.875848in}}%
\pgfpathlineto{\pgfqpoint{3.273529in}{2.889703in}}%
\pgfpathlineto{\pgfqpoint{3.267241in}{2.903711in}}%
\pgfpathlineto{\pgfqpoint{3.260955in}{2.917807in}}%
\pgfpathlineto{\pgfqpoint{3.254672in}{2.931925in}}%
\pgfpathlineto{\pgfqpoint{3.243306in}{2.922969in}}%
\pgfpathlineto{\pgfqpoint{3.231935in}{2.913447in}}%
\pgfpathlineto{\pgfqpoint{3.220560in}{2.903395in}}%
\pgfpathlineto{\pgfqpoint{3.209182in}{2.892838in}}%
\pgfpathlineto{\pgfqpoint{3.197799in}{2.881646in}}%
\pgfpathlineto{\pgfqpoint{3.204068in}{2.865998in}}%
\pgfpathlineto{\pgfqpoint{3.210340in}{2.850558in}}%
\pgfpathlineto{\pgfqpoint{3.216615in}{2.835457in}}%
\pgfpathlineto{\pgfqpoint{3.222895in}{2.820825in}}%
\pgfpathclose%
\pgfusepath{stroke,fill}%
\end{pgfscope}%
\begin{pgfscope}%
\pgfpathrectangle{\pgfqpoint{0.887500in}{0.275000in}}{\pgfqpoint{4.225000in}{4.225000in}}%
\pgfusepath{clip}%
\pgfsetbuttcap%
\pgfsetroundjoin%
\definecolor{currentfill}{rgb}{0.231674,0.318106,0.544834}%
\pgfsetfillcolor{currentfill}%
\pgfsetfillopacity{0.700000}%
\pgfsetlinewidth{0.501875pt}%
\definecolor{currentstroke}{rgb}{1.000000,1.000000,1.000000}%
\pgfsetstrokecolor{currentstroke}%
\pgfsetstrokeopacity{0.500000}%
\pgfsetdash{}{0pt}%
\pgfpathmoveto{\pgfqpoint{4.647706in}{1.865658in}}%
\pgfpathlineto{\pgfqpoint{4.658708in}{1.869001in}}%
\pgfpathlineto{\pgfqpoint{4.669704in}{1.872339in}}%
\pgfpathlineto{\pgfqpoint{4.680694in}{1.875673in}}%
\pgfpathlineto{\pgfqpoint{4.691679in}{1.879006in}}%
\pgfpathlineto{\pgfqpoint{4.702659in}{1.882339in}}%
\pgfpathlineto{\pgfqpoint{4.696119in}{1.897044in}}%
\pgfpathlineto{\pgfqpoint{4.689581in}{1.911717in}}%
\pgfpathlineto{\pgfqpoint{4.683044in}{1.926364in}}%
\pgfpathlineto{\pgfqpoint{4.676509in}{1.940986in}}%
\pgfpathlineto{\pgfqpoint{4.669976in}{1.955586in}}%
\pgfpathlineto{\pgfqpoint{4.658997in}{1.952261in}}%
\pgfpathlineto{\pgfqpoint{4.648013in}{1.948941in}}%
\pgfpathlineto{\pgfqpoint{4.637024in}{1.945625in}}%
\pgfpathlineto{\pgfqpoint{4.626030in}{1.942310in}}%
\pgfpathlineto{\pgfqpoint{4.615030in}{1.938997in}}%
\pgfpathlineto{\pgfqpoint{4.621562in}{1.924365in}}%
\pgfpathlineto{\pgfqpoint{4.628095in}{1.909716in}}%
\pgfpathlineto{\pgfqpoint{4.634630in}{1.895048in}}%
\pgfpathlineto{\pgfqpoint{4.641167in}{1.880362in}}%
\pgfpathclose%
\pgfusepath{stroke,fill}%
\end{pgfscope}%
\begin{pgfscope}%
\pgfpathrectangle{\pgfqpoint{0.887500in}{0.275000in}}{\pgfqpoint{4.225000in}{4.225000in}}%
\pgfusepath{clip}%
\pgfsetbuttcap%
\pgfsetroundjoin%
\definecolor{currentfill}{rgb}{0.252899,0.742211,0.448284}%
\pgfsetfillcolor{currentfill}%
\pgfsetfillopacity{0.700000}%
\pgfsetlinewidth{0.501875pt}%
\definecolor{currentstroke}{rgb}{1.000000,1.000000,1.000000}%
\pgfsetstrokecolor{currentstroke}%
\pgfsetstrokeopacity{0.500000}%
\pgfsetdash{}{0pt}%
\pgfpathmoveto{\pgfqpoint{3.172150in}{2.729889in}}%
\pgfpathlineto{\pgfqpoint{3.183559in}{2.747133in}}%
\pgfpathlineto{\pgfqpoint{3.194967in}{2.763420in}}%
\pgfpathlineto{\pgfqpoint{3.206374in}{2.778776in}}%
\pgfpathlineto{\pgfqpoint{3.217778in}{2.793225in}}%
\pgfpathlineto{\pgfqpoint{3.229179in}{2.806790in}}%
\pgfpathlineto{\pgfqpoint{3.222895in}{2.820825in}}%
\pgfpathlineto{\pgfqpoint{3.216615in}{2.835457in}}%
\pgfpathlineto{\pgfqpoint{3.210340in}{2.850558in}}%
\pgfpathlineto{\pgfqpoint{3.204068in}{2.865998in}}%
\pgfpathlineto{\pgfqpoint{3.197799in}{2.881646in}}%
\pgfpathlineto{\pgfqpoint{3.186413in}{2.869582in}}%
\pgfpathlineto{\pgfqpoint{3.175022in}{2.856410in}}%
\pgfpathlineto{\pgfqpoint{3.163629in}{2.841893in}}%
\pgfpathlineto{\pgfqpoint{3.152232in}{2.825794in}}%
\pgfpathlineto{\pgfqpoint{3.140833in}{2.807879in}}%
\pgfpathlineto{\pgfqpoint{3.147090in}{2.791065in}}%
\pgfpathlineto{\pgfqpoint{3.153349in}{2.774693in}}%
\pgfpathlineto{\pgfqpoint{3.159612in}{2.758928in}}%
\pgfpathlineto{\pgfqpoint{3.165878in}{2.743939in}}%
\pgfpathclose%
\pgfusepath{stroke,fill}%
\end{pgfscope}%
\begin{pgfscope}%
\pgfpathrectangle{\pgfqpoint{0.887500in}{0.275000in}}{\pgfqpoint{4.225000in}{4.225000in}}%
\pgfusepath{clip}%
\pgfsetbuttcap%
\pgfsetroundjoin%
\definecolor{currentfill}{rgb}{0.133743,0.548535,0.553541}%
\pgfsetfillcolor{currentfill}%
\pgfsetfillopacity{0.700000}%
\pgfsetlinewidth{0.501875pt}%
\definecolor{currentstroke}{rgb}{1.000000,1.000000,1.000000}%
\pgfsetstrokecolor{currentstroke}%
\pgfsetstrokeopacity{0.500000}%
\pgfsetdash{}{0pt}%
\pgfpathmoveto{\pgfqpoint{4.055330in}{2.338199in}}%
\pgfpathlineto{\pgfqpoint{4.066488in}{2.341705in}}%
\pgfpathlineto{\pgfqpoint{4.077641in}{2.345207in}}%
\pgfpathlineto{\pgfqpoint{4.088789in}{2.348700in}}%
\pgfpathlineto{\pgfqpoint{4.099930in}{2.352177in}}%
\pgfpathlineto{\pgfqpoint{4.111066in}{2.355632in}}%
\pgfpathlineto{\pgfqpoint{4.104625in}{2.370030in}}%
\pgfpathlineto{\pgfqpoint{4.098187in}{2.384414in}}%
\pgfpathlineto{\pgfqpoint{4.091750in}{2.398778in}}%
\pgfpathlineto{\pgfqpoint{4.085316in}{2.413120in}}%
\pgfpathlineto{\pgfqpoint{4.078883in}{2.427436in}}%
\pgfpathlineto{\pgfqpoint{4.067751in}{2.424016in}}%
\pgfpathlineto{\pgfqpoint{4.056612in}{2.420578in}}%
\pgfpathlineto{\pgfqpoint{4.045467in}{2.417128in}}%
\pgfpathlineto{\pgfqpoint{4.034317in}{2.413670in}}%
\pgfpathlineto{\pgfqpoint{4.023162in}{2.410209in}}%
\pgfpathlineto{\pgfqpoint{4.029591in}{2.395827in}}%
\pgfpathlineto{\pgfqpoint{4.036022in}{2.381427in}}%
\pgfpathlineto{\pgfqpoint{4.042455in}{2.367016in}}%
\pgfpathlineto{\pgfqpoint{4.048891in}{2.352604in}}%
\pgfpathclose%
\pgfusepath{stroke,fill}%
\end{pgfscope}%
\begin{pgfscope}%
\pgfpathrectangle{\pgfqpoint{0.887500in}{0.275000in}}{\pgfqpoint{4.225000in}{4.225000in}}%
\pgfusepath{clip}%
\pgfsetbuttcap%
\pgfsetroundjoin%
\definecolor{currentfill}{rgb}{0.180629,0.429975,0.557282}%
\pgfsetfillcolor{currentfill}%
\pgfsetfillopacity{0.700000}%
\pgfsetlinewidth{0.501875pt}%
\definecolor{currentstroke}{rgb}{1.000000,1.000000,1.000000}%
\pgfsetstrokecolor{currentstroke}%
\pgfsetstrokeopacity{0.500000}%
\pgfsetdash{}{0pt}%
\pgfpathmoveto{\pgfqpoint{2.835538in}{2.096915in}}%
\pgfpathlineto{\pgfqpoint{2.846995in}{2.101033in}}%
\pgfpathlineto{\pgfqpoint{2.858446in}{2.105405in}}%
\pgfpathlineto{\pgfqpoint{2.869892in}{2.109777in}}%
\pgfpathlineto{\pgfqpoint{2.881334in}{2.113888in}}%
\pgfpathlineto{\pgfqpoint{2.892773in}{2.117479in}}%
\pgfpathlineto{\pgfqpoint{2.886575in}{2.126949in}}%
\pgfpathlineto{\pgfqpoint{2.880382in}{2.136388in}}%
\pgfpathlineto{\pgfqpoint{2.874192in}{2.145797in}}%
\pgfpathlineto{\pgfqpoint{2.868007in}{2.155173in}}%
\pgfpathlineto{\pgfqpoint{2.861825in}{2.164516in}}%
\pgfpathlineto{\pgfqpoint{2.850396in}{2.160954in}}%
\pgfpathlineto{\pgfqpoint{2.838965in}{2.156815in}}%
\pgfpathlineto{\pgfqpoint{2.827531in}{2.152386in}}%
\pgfpathlineto{\pgfqpoint{2.816091in}{2.147955in}}%
\pgfpathlineto{\pgfqpoint{2.804645in}{2.143801in}}%
\pgfpathlineto{\pgfqpoint{2.810815in}{2.134495in}}%
\pgfpathlineto{\pgfqpoint{2.816990in}{2.125155in}}%
\pgfpathlineto{\pgfqpoint{2.823168in}{2.115779in}}%
\pgfpathlineto{\pgfqpoint{2.829351in}{2.106366in}}%
\pgfpathclose%
\pgfusepath{stroke,fill}%
\end{pgfscope}%
\begin{pgfscope}%
\pgfpathrectangle{\pgfqpoint{0.887500in}{0.275000in}}{\pgfqpoint{4.225000in}{4.225000in}}%
\pgfusepath{clip}%
\pgfsetbuttcap%
\pgfsetroundjoin%
\definecolor{currentfill}{rgb}{0.153894,0.680203,0.504172}%
\pgfsetfillcolor{currentfill}%
\pgfsetfillopacity{0.700000}%
\pgfsetlinewidth{0.501875pt}%
\definecolor{currentstroke}{rgb}{1.000000,1.000000,1.000000}%
\pgfsetstrokecolor{currentstroke}%
\pgfsetstrokeopacity{0.500000}%
\pgfsetdash{}{0pt}%
\pgfpathmoveto{\pgfqpoint{3.670895in}{2.621625in}}%
\pgfpathlineto{\pgfqpoint{3.682155in}{2.625424in}}%
\pgfpathlineto{\pgfqpoint{3.693409in}{2.629197in}}%
\pgfpathlineto{\pgfqpoint{3.704657in}{2.632930in}}%
\pgfpathlineto{\pgfqpoint{3.715899in}{2.636606in}}%
\pgfpathlineto{\pgfqpoint{3.727134in}{2.640213in}}%
\pgfpathlineto{\pgfqpoint{3.720757in}{2.654181in}}%
\pgfpathlineto{\pgfqpoint{3.714381in}{2.668114in}}%
\pgfpathlineto{\pgfqpoint{3.708008in}{2.682002in}}%
\pgfpathlineto{\pgfqpoint{3.701636in}{2.695838in}}%
\pgfpathlineto{\pgfqpoint{3.695266in}{2.709620in}}%
\pgfpathlineto{\pgfqpoint{3.684034in}{2.706080in}}%
\pgfpathlineto{\pgfqpoint{3.672796in}{2.702472in}}%
\pgfpathlineto{\pgfqpoint{3.661552in}{2.698808in}}%
\pgfpathlineto{\pgfqpoint{3.650302in}{2.695104in}}%
\pgfpathlineto{\pgfqpoint{3.639047in}{2.691372in}}%
\pgfpathlineto{\pgfqpoint{3.645413in}{2.677539in}}%
\pgfpathlineto{\pgfqpoint{3.651781in}{2.663644in}}%
\pgfpathlineto{\pgfqpoint{3.658150in}{2.649687in}}%
\pgfpathlineto{\pgfqpoint{3.664522in}{2.635678in}}%
\pgfpathclose%
\pgfusepath{stroke,fill}%
\end{pgfscope}%
\begin{pgfscope}%
\pgfpathrectangle{\pgfqpoint{0.887500in}{0.275000in}}{\pgfqpoint{4.225000in}{4.225000in}}%
\pgfusepath{clip}%
\pgfsetbuttcap%
\pgfsetroundjoin%
\definecolor{currentfill}{rgb}{0.166617,0.463708,0.558119}%
\pgfsetfillcolor{currentfill}%
\pgfsetfillopacity{0.700000}%
\pgfsetlinewidth{0.501875pt}%
\definecolor{currentstroke}{rgb}{1.000000,1.000000,1.000000}%
\pgfsetstrokecolor{currentstroke}%
\pgfsetstrokeopacity{0.500000}%
\pgfsetdash{}{0pt}%
\pgfpathmoveto{\pgfqpoint{2.513373in}{2.167096in}}%
\pgfpathlineto{\pgfqpoint{2.524916in}{2.170552in}}%
\pgfpathlineto{\pgfqpoint{2.536453in}{2.174020in}}%
\pgfpathlineto{\pgfqpoint{2.547985in}{2.177503in}}%
\pgfpathlineto{\pgfqpoint{2.559510in}{2.181004in}}%
\pgfpathlineto{\pgfqpoint{2.571030in}{2.184527in}}%
\pgfpathlineto{\pgfqpoint{2.564936in}{2.193567in}}%
\pgfpathlineto{\pgfqpoint{2.558847in}{2.202577in}}%
\pgfpathlineto{\pgfqpoint{2.552762in}{2.211557in}}%
\pgfpathlineto{\pgfqpoint{2.546682in}{2.220508in}}%
\pgfpathlineto{\pgfqpoint{2.540605in}{2.229431in}}%
\pgfpathlineto{\pgfqpoint{2.529096in}{2.225936in}}%
\pgfpathlineto{\pgfqpoint{2.517582in}{2.222465in}}%
\pgfpathlineto{\pgfqpoint{2.506061in}{2.219016in}}%
\pgfpathlineto{\pgfqpoint{2.494535in}{2.215585in}}%
\pgfpathlineto{\pgfqpoint{2.483002in}{2.212168in}}%
\pgfpathlineto{\pgfqpoint{2.489068in}{2.203213in}}%
\pgfpathlineto{\pgfqpoint{2.495138in}{2.194229in}}%
\pgfpathlineto{\pgfqpoint{2.501212in}{2.185215in}}%
\pgfpathlineto{\pgfqpoint{2.507290in}{2.176171in}}%
\pgfpathclose%
\pgfusepath{stroke,fill}%
\end{pgfscope}%
\begin{pgfscope}%
\pgfpathrectangle{\pgfqpoint{0.887500in}{0.275000in}}{\pgfqpoint{4.225000in}{4.225000in}}%
\pgfusepath{clip}%
\pgfsetbuttcap%
\pgfsetroundjoin%
\definecolor{currentfill}{rgb}{0.166617,0.463708,0.558119}%
\pgfsetfillcolor{currentfill}%
\pgfsetfillopacity{0.700000}%
\pgfsetlinewidth{0.501875pt}%
\definecolor{currentstroke}{rgb}{1.000000,1.000000,1.000000}%
\pgfsetstrokecolor{currentstroke}%
\pgfsetstrokeopacity{0.500000}%
\pgfsetdash{}{0pt}%
\pgfpathmoveto{\pgfqpoint{4.263528in}{2.154666in}}%
\pgfpathlineto{\pgfqpoint{4.274624in}{2.157773in}}%
\pgfpathlineto{\pgfqpoint{4.285712in}{2.160765in}}%
\pgfpathlineto{\pgfqpoint{4.296790in}{2.163662in}}%
\pgfpathlineto{\pgfqpoint{4.307862in}{2.166524in}}%
\pgfpathlineto{\pgfqpoint{4.318929in}{2.169415in}}%
\pgfpathlineto{\pgfqpoint{4.312436in}{2.183424in}}%
\pgfpathlineto{\pgfqpoint{4.305951in}{2.197638in}}%
\pgfpathlineto{\pgfqpoint{4.299476in}{2.212052in}}%
\pgfpathlineto{\pgfqpoint{4.293009in}{2.226638in}}%
\pgfpathlineto{\pgfqpoint{4.286549in}{2.241364in}}%
\pgfpathlineto{\pgfqpoint{4.275483in}{2.238482in}}%
\pgfpathlineto{\pgfqpoint{4.264413in}{2.235640in}}%
\pgfpathlineto{\pgfqpoint{4.253337in}{2.232789in}}%
\pgfpathlineto{\pgfqpoint{4.242253in}{2.229879in}}%
\pgfpathlineto{\pgfqpoint{4.231162in}{2.226894in}}%
\pgfpathlineto{\pgfqpoint{4.237626in}{2.212334in}}%
\pgfpathlineto{\pgfqpoint{4.244095in}{2.197822in}}%
\pgfpathlineto{\pgfqpoint{4.250568in}{2.183367in}}%
\pgfpathlineto{\pgfqpoint{4.257046in}{2.168981in}}%
\pgfpathclose%
\pgfusepath{stroke,fill}%
\end{pgfscope}%
\begin{pgfscope}%
\pgfpathrectangle{\pgfqpoint{0.887500in}{0.275000in}}{\pgfqpoint{4.225000in}{4.225000in}}%
\pgfusepath{clip}%
\pgfsetbuttcap%
\pgfsetroundjoin%
\definecolor{currentfill}{rgb}{0.154815,0.493313,0.557840}%
\pgfsetfillcolor{currentfill}%
\pgfsetfillopacity{0.700000}%
\pgfsetlinewidth{0.501875pt}%
\definecolor{currentstroke}{rgb}{1.000000,1.000000,1.000000}%
\pgfsetstrokecolor{currentstroke}%
\pgfsetstrokeopacity{0.500000}%
\pgfsetdash{}{0pt}%
\pgfpathmoveto{\pgfqpoint{2.191187in}{2.231890in}}%
\pgfpathlineto{\pgfqpoint{2.202810in}{2.235389in}}%
\pgfpathlineto{\pgfqpoint{2.214427in}{2.238894in}}%
\pgfpathlineto{\pgfqpoint{2.226038in}{2.242400in}}%
\pgfpathlineto{\pgfqpoint{2.237643in}{2.245905in}}%
\pgfpathlineto{\pgfqpoint{2.249242in}{2.249405in}}%
\pgfpathlineto{\pgfqpoint{2.243257in}{2.258198in}}%
\pgfpathlineto{\pgfqpoint{2.237275in}{2.266965in}}%
\pgfpathlineto{\pgfqpoint{2.231298in}{2.275708in}}%
\pgfpathlineto{\pgfqpoint{2.225325in}{2.284427in}}%
\pgfpathlineto{\pgfqpoint{2.219356in}{2.293123in}}%
\pgfpathlineto{\pgfqpoint{2.207767in}{2.289664in}}%
\pgfpathlineto{\pgfqpoint{2.196173in}{2.286201in}}%
\pgfpathlineto{\pgfqpoint{2.184573in}{2.282736in}}%
\pgfpathlineto{\pgfqpoint{2.172968in}{2.279275in}}%
\pgfpathlineto{\pgfqpoint{2.161356in}{2.275820in}}%
\pgfpathlineto{\pgfqpoint{2.167314in}{2.267083in}}%
\pgfpathlineto{\pgfqpoint{2.173276in}{2.258323in}}%
\pgfpathlineto{\pgfqpoint{2.179242in}{2.249537in}}%
\pgfpathlineto{\pgfqpoint{2.185212in}{2.240727in}}%
\pgfpathclose%
\pgfusepath{stroke,fill}%
\end{pgfscope}%
\begin{pgfscope}%
\pgfpathrectangle{\pgfqpoint{0.887500in}{0.275000in}}{\pgfqpoint{4.225000in}{4.225000in}}%
\pgfusepath{clip}%
\pgfsetbuttcap%
\pgfsetroundjoin%
\definecolor{currentfill}{rgb}{0.187231,0.414746,0.556547}%
\pgfsetfillcolor{currentfill}%
\pgfsetfillopacity{0.700000}%
\pgfsetlinewidth{0.501875pt}%
\definecolor{currentstroke}{rgb}{1.000000,1.000000,1.000000}%
\pgfsetstrokecolor{currentstroke}%
\pgfsetstrokeopacity{0.500000}%
\pgfsetdash{}{0pt}%
\pgfpathmoveto{\pgfqpoint{2.923820in}{2.069661in}}%
\pgfpathlineto{\pgfqpoint{2.935265in}{2.072555in}}%
\pgfpathlineto{\pgfqpoint{2.946708in}{2.074468in}}%
\pgfpathlineto{\pgfqpoint{2.958146in}{2.075210in}}%
\pgfpathlineto{\pgfqpoint{2.969579in}{2.075239in}}%
\pgfpathlineto{\pgfqpoint{2.981004in}{2.075445in}}%
\pgfpathlineto{\pgfqpoint{2.974779in}{2.084953in}}%
\pgfpathlineto{\pgfqpoint{2.968558in}{2.094401in}}%
\pgfpathlineto{\pgfqpoint{2.962341in}{2.103783in}}%
\pgfpathlineto{\pgfqpoint{2.956128in}{2.113091in}}%
\pgfpathlineto{\pgfqpoint{2.949919in}{2.122322in}}%
\pgfpathlineto{\pgfqpoint{2.938501in}{2.122362in}}%
\pgfpathlineto{\pgfqpoint{2.927075in}{2.122589in}}%
\pgfpathlineto{\pgfqpoint{2.915643in}{2.122060in}}%
\pgfpathlineto{\pgfqpoint{2.904209in}{2.120289in}}%
\pgfpathlineto{\pgfqpoint{2.892773in}{2.117479in}}%
\pgfpathlineto{\pgfqpoint{2.898974in}{2.107979in}}%
\pgfpathlineto{\pgfqpoint{2.905180in}{2.098450in}}%
\pgfpathlineto{\pgfqpoint{2.911389in}{2.088889in}}%
\pgfpathlineto{\pgfqpoint{2.917603in}{2.079294in}}%
\pgfpathclose%
\pgfusepath{stroke,fill}%
\end{pgfscope}%
\begin{pgfscope}%
\pgfpathrectangle{\pgfqpoint{0.887500in}{0.275000in}}{\pgfqpoint{4.225000in}{4.225000in}}%
\pgfusepath{clip}%
\pgfsetbuttcap%
\pgfsetroundjoin%
\definecolor{currentfill}{rgb}{0.143343,0.522773,0.556295}%
\pgfsetfillcolor{currentfill}%
\pgfsetfillopacity{0.700000}%
\pgfsetlinewidth{0.501875pt}%
\definecolor{currentstroke}{rgb}{1.000000,1.000000,1.000000}%
\pgfsetstrokecolor{currentstroke}%
\pgfsetstrokeopacity{0.500000}%
\pgfsetdash{}{0pt}%
\pgfpathmoveto{\pgfqpoint{1.869025in}{2.294255in}}%
\pgfpathlineto{\pgfqpoint{1.880727in}{2.297712in}}%
\pgfpathlineto{\pgfqpoint{1.892425in}{2.301167in}}%
\pgfpathlineto{\pgfqpoint{1.904116in}{2.304616in}}%
\pgfpathlineto{\pgfqpoint{1.915802in}{2.308058in}}%
\pgfpathlineto{\pgfqpoint{1.927482in}{2.311489in}}%
\pgfpathlineto{\pgfqpoint{1.921606in}{2.320089in}}%
\pgfpathlineto{\pgfqpoint{1.915735in}{2.328667in}}%
\pgfpathlineto{\pgfqpoint{1.909867in}{2.337221in}}%
\pgfpathlineto{\pgfqpoint{1.904005in}{2.345754in}}%
\pgfpathlineto{\pgfqpoint{1.898146in}{2.354266in}}%
\pgfpathlineto{\pgfqpoint{1.886477in}{2.350871in}}%
\pgfpathlineto{\pgfqpoint{1.874803in}{2.347468in}}%
\pgfpathlineto{\pgfqpoint{1.863123in}{2.344059in}}%
\pgfpathlineto{\pgfqpoint{1.851437in}{2.340645in}}%
\pgfpathlineto{\pgfqpoint{1.839746in}{2.337228in}}%
\pgfpathlineto{\pgfqpoint{1.845593in}{2.328679in}}%
\pgfpathlineto{\pgfqpoint{1.851444in}{2.320108in}}%
\pgfpathlineto{\pgfqpoint{1.857300in}{2.311514in}}%
\pgfpathlineto{\pgfqpoint{1.863160in}{2.302896in}}%
\pgfpathclose%
\pgfusepath{stroke,fill}%
\end{pgfscope}%
\begin{pgfscope}%
\pgfpathrectangle{\pgfqpoint{0.887500in}{0.275000in}}{\pgfqpoint{4.225000in}{4.225000in}}%
\pgfusepath{clip}%
\pgfsetbuttcap%
\pgfsetroundjoin%
\definecolor{currentfill}{rgb}{0.125394,0.574318,0.549086}%
\pgfsetfillcolor{currentfill}%
\pgfsetfillopacity{0.700000}%
\pgfsetlinewidth{0.501875pt}%
\definecolor{currentstroke}{rgb}{1.000000,1.000000,1.000000}%
\pgfsetstrokecolor{currentstroke}%
\pgfsetstrokeopacity{0.500000}%
\pgfsetdash{}{0pt}%
\pgfpathmoveto{\pgfqpoint{3.967304in}{2.392990in}}%
\pgfpathlineto{\pgfqpoint{3.978487in}{2.396423in}}%
\pgfpathlineto{\pgfqpoint{3.989663in}{2.399857in}}%
\pgfpathlineto{\pgfqpoint{4.000835in}{2.403298in}}%
\pgfpathlineto{\pgfqpoint{4.012001in}{2.406751in}}%
\pgfpathlineto{\pgfqpoint{4.023162in}{2.410209in}}%
\pgfpathlineto{\pgfqpoint{4.016735in}{2.424565in}}%
\pgfpathlineto{\pgfqpoint{4.010309in}{2.438884in}}%
\pgfpathlineto{\pgfqpoint{4.003885in}{2.453160in}}%
\pgfpathlineto{\pgfqpoint{3.997462in}{2.467391in}}%
\pgfpathlineto{\pgfqpoint{3.991041in}{2.481581in}}%
\pgfpathlineto{\pgfqpoint{3.979884in}{2.478178in}}%
\pgfpathlineto{\pgfqpoint{3.968721in}{2.474778in}}%
\pgfpathlineto{\pgfqpoint{3.957552in}{2.471384in}}%
\pgfpathlineto{\pgfqpoint{3.946379in}{2.467994in}}%
\pgfpathlineto{\pgfqpoint{3.935199in}{2.464608in}}%
\pgfpathlineto{\pgfqpoint{3.941618in}{2.450388in}}%
\pgfpathlineto{\pgfqpoint{3.948037in}{2.436116in}}%
\pgfpathlineto{\pgfqpoint{3.954458in}{2.421789in}}%
\pgfpathlineto{\pgfqpoint{3.960881in}{2.407410in}}%
\pgfpathclose%
\pgfusepath{stroke,fill}%
\end{pgfscope}%
\begin{pgfscope}%
\pgfpathrectangle{\pgfqpoint{0.887500in}{0.275000in}}{\pgfqpoint{4.225000in}{4.225000in}}%
\pgfusepath{clip}%
\pgfsetbuttcap%
\pgfsetroundjoin%
\definecolor{currentfill}{rgb}{0.185783,0.704891,0.485273}%
\pgfsetfillcolor{currentfill}%
\pgfsetfillopacity{0.700000}%
\pgfsetlinewidth{0.501875pt}%
\definecolor{currentstroke}{rgb}{1.000000,1.000000,1.000000}%
\pgfsetstrokecolor{currentstroke}%
\pgfsetstrokeopacity{0.500000}%
\pgfsetdash{}{0pt}%
\pgfpathmoveto{\pgfqpoint{3.582688in}{2.672684in}}%
\pgfpathlineto{\pgfqpoint{3.593971in}{2.676449in}}%
\pgfpathlineto{\pgfqpoint{3.605248in}{2.680165in}}%
\pgfpathlineto{\pgfqpoint{3.616519in}{2.683889in}}%
\pgfpathlineto{\pgfqpoint{3.627786in}{2.687629in}}%
\pgfpathlineto{\pgfqpoint{3.639047in}{2.691372in}}%
\pgfpathlineto{\pgfqpoint{3.632682in}{2.705149in}}%
\pgfpathlineto{\pgfqpoint{3.626320in}{2.718873in}}%
\pgfpathlineto{\pgfqpoint{3.619959in}{2.732551in}}%
\pgfpathlineto{\pgfqpoint{3.613600in}{2.746188in}}%
\pgfpathlineto{\pgfqpoint{3.607244in}{2.759788in}}%
\pgfpathlineto{\pgfqpoint{3.595987in}{2.756095in}}%
\pgfpathlineto{\pgfqpoint{3.584724in}{2.752396in}}%
\pgfpathlineto{\pgfqpoint{3.573457in}{2.748699in}}%
\pgfpathlineto{\pgfqpoint{3.562184in}{2.745000in}}%
\pgfpathlineto{\pgfqpoint{3.550905in}{2.741271in}}%
\pgfpathlineto{\pgfqpoint{3.557257in}{2.727643in}}%
\pgfpathlineto{\pgfqpoint{3.563612in}{2.713975in}}%
\pgfpathlineto{\pgfqpoint{3.569969in}{2.700263in}}%
\pgfpathlineto{\pgfqpoint{3.576328in}{2.686501in}}%
\pgfpathclose%
\pgfusepath{stroke,fill}%
\end{pgfscope}%
\begin{pgfscope}%
\pgfpathrectangle{\pgfqpoint{0.887500in}{0.275000in}}{\pgfqpoint{4.225000in}{4.225000in}}%
\pgfusepath{clip}%
\pgfsetbuttcap%
\pgfsetroundjoin%
\definecolor{currentfill}{rgb}{0.216210,0.351535,0.550627}%
\pgfsetfillcolor{currentfill}%
\pgfsetfillopacity{0.700000}%
\pgfsetlinewidth{0.501875pt}%
\definecolor{currentstroke}{rgb}{1.000000,1.000000,1.000000}%
\pgfsetstrokecolor{currentstroke}%
\pgfsetstrokeopacity{0.500000}%
\pgfsetdash{}{0pt}%
\pgfpathmoveto{\pgfqpoint{4.559947in}{1.922392in}}%
\pgfpathlineto{\pgfqpoint{4.570975in}{1.925722in}}%
\pgfpathlineto{\pgfqpoint{4.581997in}{1.929047in}}%
\pgfpathlineto{\pgfqpoint{4.593014in}{1.932366in}}%
\pgfpathlineto{\pgfqpoint{4.604025in}{1.935682in}}%
\pgfpathlineto{\pgfqpoint{4.615030in}{1.938997in}}%
\pgfpathlineto{\pgfqpoint{4.608500in}{1.953610in}}%
\pgfpathlineto{\pgfqpoint{4.601973in}{1.968206in}}%
\pgfpathlineto{\pgfqpoint{4.595447in}{1.982785in}}%
\pgfpathlineto{\pgfqpoint{4.588923in}{1.997346in}}%
\pgfpathlineto{\pgfqpoint{4.582401in}{2.011890in}}%
\pgfpathlineto{\pgfqpoint{4.571398in}{2.008621in}}%
\pgfpathlineto{\pgfqpoint{4.560390in}{2.005358in}}%
\pgfpathlineto{\pgfqpoint{4.549377in}{2.002103in}}%
\pgfpathlineto{\pgfqpoint{4.538358in}{1.998854in}}%
\pgfpathlineto{\pgfqpoint{4.527335in}{1.995612in}}%
\pgfpathlineto{\pgfqpoint{4.533855in}{1.981037in}}%
\pgfpathlineto{\pgfqpoint{4.540376in}{1.966422in}}%
\pgfpathlineto{\pgfqpoint{4.546898in}{1.951771in}}%
\pgfpathlineto{\pgfqpoint{4.553422in}{1.937093in}}%
\pgfpathclose%
\pgfusepath{stroke,fill}%
\end{pgfscope}%
\begin{pgfscope}%
\pgfpathrectangle{\pgfqpoint{0.887500in}{0.275000in}}{\pgfqpoint{4.225000in}{4.225000in}}%
\pgfusepath{clip}%
\pgfsetbuttcap%
\pgfsetroundjoin%
\definecolor{currentfill}{rgb}{0.296479,0.761561,0.424223}%
\pgfsetfillcolor{currentfill}%
\pgfsetfillopacity{0.700000}%
\pgfsetlinewidth{0.501875pt}%
\definecolor{currentstroke}{rgb}{1.000000,1.000000,1.000000}%
\pgfsetstrokecolor{currentstroke}%
\pgfsetstrokeopacity{0.500000}%
\pgfsetdash{}{0pt}%
\pgfpathmoveto{\pgfqpoint{3.317655in}{2.797644in}}%
\pgfpathlineto{\pgfqpoint{3.329032in}{2.806040in}}%
\pgfpathlineto{\pgfqpoint{3.340401in}{2.813793in}}%
\pgfpathlineto{\pgfqpoint{3.351762in}{2.820904in}}%
\pgfpathlineto{\pgfqpoint{3.363115in}{2.827373in}}%
\pgfpathlineto{\pgfqpoint{3.374460in}{2.833241in}}%
\pgfpathlineto{\pgfqpoint{3.368140in}{2.846262in}}%
\pgfpathlineto{\pgfqpoint{3.361822in}{2.859298in}}%
\pgfpathlineto{\pgfqpoint{3.355508in}{2.872394in}}%
\pgfpathlineto{\pgfqpoint{3.349198in}{2.885597in}}%
\pgfpathlineto{\pgfqpoint{3.342891in}{2.898946in}}%
\pgfpathlineto{\pgfqpoint{3.331552in}{2.892944in}}%
\pgfpathlineto{\pgfqpoint{3.320205in}{2.886311in}}%
\pgfpathlineto{\pgfqpoint{3.308849in}{2.878992in}}%
\pgfpathlineto{\pgfqpoint{3.297486in}{2.870966in}}%
\pgfpathlineto{\pgfqpoint{3.286116in}{2.862211in}}%
\pgfpathlineto{\pgfqpoint{3.292416in}{2.848858in}}%
\pgfpathlineto{\pgfqpoint{3.298720in}{2.835805in}}%
\pgfpathlineto{\pgfqpoint{3.305028in}{2.822976in}}%
\pgfpathlineto{\pgfqpoint{3.311340in}{2.810285in}}%
\pgfpathclose%
\pgfusepath{stroke,fill}%
\end{pgfscope}%
\begin{pgfscope}%
\pgfpathrectangle{\pgfqpoint{0.887500in}{0.275000in}}{\pgfqpoint{4.225000in}{4.225000in}}%
\pgfusepath{clip}%
\pgfsetbuttcap%
\pgfsetroundjoin%
\definecolor{currentfill}{rgb}{0.150148,0.676631,0.506589}%
\pgfsetfillcolor{currentfill}%
\pgfsetfillopacity{0.700000}%
\pgfsetlinewidth{0.501875pt}%
\definecolor{currentstroke}{rgb}{1.000000,1.000000,1.000000}%
\pgfsetstrokecolor{currentstroke}%
\pgfsetstrokeopacity{0.500000}%
\pgfsetdash{}{0pt}%
\pgfpathmoveto{\pgfqpoint{3.146465in}{2.575163in}}%
\pgfpathlineto{\pgfqpoint{3.157887in}{2.596406in}}%
\pgfpathlineto{\pgfqpoint{3.169313in}{2.617090in}}%
\pgfpathlineto{\pgfqpoint{3.180740in}{2.637022in}}%
\pgfpathlineto{\pgfqpoint{3.192168in}{2.656009in}}%
\pgfpathlineto{\pgfqpoint{3.203595in}{2.673857in}}%
\pgfpathlineto{\pgfqpoint{3.197295in}{2.683860in}}%
\pgfpathlineto{\pgfqpoint{3.191000in}{2.694192in}}%
\pgfpathlineto{\pgfqpoint{3.184710in}{2.705129in}}%
\pgfpathlineto{\pgfqpoint{3.178427in}{2.716946in}}%
\pgfpathlineto{\pgfqpoint{3.172150in}{2.729889in}}%
\pgfpathlineto{\pgfqpoint{3.160740in}{2.711679in}}%
\pgfpathlineto{\pgfqpoint{3.149332in}{2.692584in}}%
\pgfpathlineto{\pgfqpoint{3.137925in}{2.672729in}}%
\pgfpathlineto{\pgfqpoint{3.126521in}{2.652239in}}%
\pgfpathlineto{\pgfqpoint{3.115120in}{2.631240in}}%
\pgfpathlineto{\pgfqpoint{3.121382in}{2.619680in}}%
\pgfpathlineto{\pgfqpoint{3.127648in}{2.608389in}}%
\pgfpathlineto{\pgfqpoint{3.133917in}{2.597272in}}%
\pgfpathlineto{\pgfqpoint{3.140189in}{2.586231in}}%
\pgfpathclose%
\pgfusepath{stroke,fill}%
\end{pgfscope}%
\begin{pgfscope}%
\pgfpathrectangle{\pgfqpoint{0.887500in}{0.275000in}}{\pgfqpoint{4.225000in}{4.225000in}}%
\pgfusepath{clip}%
\pgfsetbuttcap%
\pgfsetroundjoin%
\definecolor{currentfill}{rgb}{0.220124,0.725509,0.466226}%
\pgfsetfillcolor{currentfill}%
\pgfsetfillopacity{0.700000}%
\pgfsetlinewidth{0.501875pt}%
\definecolor{currentstroke}{rgb}{1.000000,1.000000,1.000000}%
\pgfsetstrokecolor{currentstroke}%
\pgfsetstrokeopacity{0.500000}%
\pgfsetdash{}{0pt}%
\pgfpathmoveto{\pgfqpoint{3.494413in}{2.721012in}}%
\pgfpathlineto{\pgfqpoint{3.505726in}{2.725388in}}%
\pgfpathlineto{\pgfqpoint{3.517031in}{2.729569in}}%
\pgfpathlineto{\pgfqpoint{3.528329in}{2.733588in}}%
\pgfpathlineto{\pgfqpoint{3.539620in}{2.737478in}}%
\pgfpathlineto{\pgfqpoint{3.550905in}{2.741271in}}%
\pgfpathlineto{\pgfqpoint{3.544554in}{2.754864in}}%
\pgfpathlineto{\pgfqpoint{3.538206in}{2.768428in}}%
\pgfpathlineto{\pgfqpoint{3.531861in}{2.781968in}}%
\pgfpathlineto{\pgfqpoint{3.525517in}{2.795488in}}%
\pgfpathlineto{\pgfqpoint{3.519176in}{2.808991in}}%
\pgfpathlineto{\pgfqpoint{3.507896in}{2.805228in}}%
\pgfpathlineto{\pgfqpoint{3.496609in}{2.801423in}}%
\pgfpathlineto{\pgfqpoint{3.485317in}{2.797570in}}%
\pgfpathlineto{\pgfqpoint{3.474019in}{2.793663in}}%
\pgfpathlineto{\pgfqpoint{3.462715in}{2.789692in}}%
\pgfpathlineto{\pgfqpoint{3.469051in}{2.776123in}}%
\pgfpathlineto{\pgfqpoint{3.475389in}{2.762464in}}%
\pgfpathlineto{\pgfqpoint{3.481729in}{2.748720in}}%
\pgfpathlineto{\pgfqpoint{3.488070in}{2.734899in}}%
\pgfpathclose%
\pgfusepath{stroke,fill}%
\end{pgfscope}%
\begin{pgfscope}%
\pgfpathrectangle{\pgfqpoint{0.887500in}{0.275000in}}{\pgfqpoint{4.225000in}{4.225000in}}%
\pgfusepath{clip}%
\pgfsetbuttcap%
\pgfsetroundjoin%
\definecolor{currentfill}{rgb}{0.194100,0.399323,0.555565}%
\pgfsetfillcolor{currentfill}%
\pgfsetfillopacity{0.700000}%
\pgfsetlinewidth{0.501875pt}%
\definecolor{currentstroke}{rgb}{1.000000,1.000000,1.000000}%
\pgfsetstrokecolor{currentstroke}%
\pgfsetstrokeopacity{0.500000}%
\pgfsetdash{}{0pt}%
\pgfpathmoveto{\pgfqpoint{3.012188in}{2.027299in}}%
\pgfpathlineto{\pgfqpoint{3.023612in}{2.028706in}}%
\pgfpathlineto{\pgfqpoint{3.035027in}{2.031996in}}%
\pgfpathlineto{\pgfqpoint{3.046437in}{2.038024in}}%
\pgfpathlineto{\pgfqpoint{3.057842in}{2.047649in}}%
\pgfpathlineto{\pgfqpoint{3.069246in}{2.061584in}}%
\pgfpathlineto{\pgfqpoint{3.062989in}{2.071456in}}%
\pgfpathlineto{\pgfqpoint{3.056736in}{2.081264in}}%
\pgfpathlineto{\pgfqpoint{3.050487in}{2.091013in}}%
\pgfpathlineto{\pgfqpoint{3.044241in}{2.100707in}}%
\pgfpathlineto{\pgfqpoint{3.038000in}{2.110352in}}%
\pgfpathlineto{\pgfqpoint{3.026613in}{2.095966in}}%
\pgfpathlineto{\pgfqpoint{3.015223in}{2.086090in}}%
\pgfpathlineto{\pgfqpoint{3.003826in}{2.079975in}}%
\pgfpathlineto{\pgfqpoint{2.992420in}{2.076725in}}%
\pgfpathlineto{\pgfqpoint{2.981004in}{2.075445in}}%
\pgfpathlineto{\pgfqpoint{2.987233in}{2.065887in}}%
\pgfpathlineto{\pgfqpoint{2.993466in}{2.056285in}}%
\pgfpathlineto{\pgfqpoint{2.999703in}{2.046648in}}%
\pgfpathlineto{\pgfqpoint{3.005943in}{2.036984in}}%
\pgfpathclose%
\pgfusepath{stroke,fill}%
\end{pgfscope}%
\begin{pgfscope}%
\pgfpathrectangle{\pgfqpoint{0.887500in}{0.275000in}}{\pgfqpoint{4.225000in}{4.225000in}}%
\pgfusepath{clip}%
\pgfsetbuttcap%
\pgfsetroundjoin%
\definecolor{currentfill}{rgb}{0.171176,0.452530,0.557965}%
\pgfsetfillcolor{currentfill}%
\pgfsetfillopacity{0.700000}%
\pgfsetlinewidth{0.501875pt}%
\definecolor{currentstroke}{rgb}{1.000000,1.000000,1.000000}%
\pgfsetstrokecolor{currentstroke}%
\pgfsetstrokeopacity{0.500000}%
\pgfsetdash{}{0pt}%
\pgfpathmoveto{\pgfqpoint{2.601560in}{2.138850in}}%
\pgfpathlineto{\pgfqpoint{2.613084in}{2.142429in}}%
\pgfpathlineto{\pgfqpoint{2.624603in}{2.146034in}}%
\pgfpathlineto{\pgfqpoint{2.636115in}{2.149658in}}%
\pgfpathlineto{\pgfqpoint{2.647622in}{2.153276in}}%
\pgfpathlineto{\pgfqpoint{2.659124in}{2.156861in}}%
\pgfpathlineto{\pgfqpoint{2.652999in}{2.166033in}}%
\pgfpathlineto{\pgfqpoint{2.646878in}{2.175172in}}%
\pgfpathlineto{\pgfqpoint{2.640761in}{2.184280in}}%
\pgfpathlineto{\pgfqpoint{2.634648in}{2.193356in}}%
\pgfpathlineto{\pgfqpoint{2.628540in}{2.202403in}}%
\pgfpathlineto{\pgfqpoint{2.617049in}{2.198842in}}%
\pgfpathlineto{\pgfqpoint{2.605553in}{2.195249in}}%
\pgfpathlineto{\pgfqpoint{2.594051in}{2.191651in}}%
\pgfpathlineto{\pgfqpoint{2.582543in}{2.188075in}}%
\pgfpathlineto{\pgfqpoint{2.571030in}{2.184527in}}%
\pgfpathlineto{\pgfqpoint{2.577127in}{2.175456in}}%
\pgfpathlineto{\pgfqpoint{2.583229in}{2.166353in}}%
\pgfpathlineto{\pgfqpoint{2.589335in}{2.157218in}}%
\pgfpathlineto{\pgfqpoint{2.595446in}{2.148051in}}%
\pgfpathclose%
\pgfusepath{stroke,fill}%
\end{pgfscope}%
\begin{pgfscope}%
\pgfpathrectangle{\pgfqpoint{0.887500in}{0.275000in}}{\pgfqpoint{4.225000in}{4.225000in}}%
\pgfusepath{clip}%
\pgfsetbuttcap%
\pgfsetroundjoin%
\definecolor{currentfill}{rgb}{0.154815,0.493313,0.557840}%
\pgfsetfillcolor{currentfill}%
\pgfsetfillopacity{0.700000}%
\pgfsetlinewidth{0.501875pt}%
\definecolor{currentstroke}{rgb}{1.000000,1.000000,1.000000}%
\pgfsetstrokecolor{currentstroke}%
\pgfsetstrokeopacity{0.500000}%
\pgfsetdash{}{0pt}%
\pgfpathmoveto{\pgfqpoint{4.175594in}{2.211134in}}%
\pgfpathlineto{\pgfqpoint{4.186721in}{2.214366in}}%
\pgfpathlineto{\pgfqpoint{4.197842in}{2.217567in}}%
\pgfpathlineto{\pgfqpoint{4.208956in}{2.220728in}}%
\pgfpathlineto{\pgfqpoint{4.220062in}{2.223840in}}%
\pgfpathlineto{\pgfqpoint{4.231162in}{2.226894in}}%
\pgfpathlineto{\pgfqpoint{4.224700in}{2.241488in}}%
\pgfpathlineto{\pgfqpoint{4.218242in}{2.256106in}}%
\pgfpathlineto{\pgfqpoint{4.211787in}{2.270738in}}%
\pgfpathlineto{\pgfqpoint{4.205334in}{2.285371in}}%
\pgfpathlineto{\pgfqpoint{4.198884in}{2.299994in}}%
\pgfpathlineto{\pgfqpoint{4.187780in}{2.296748in}}%
\pgfpathlineto{\pgfqpoint{4.176670in}{2.293479in}}%
\pgfpathlineto{\pgfqpoint{4.165554in}{2.290186in}}%
\pgfpathlineto{\pgfqpoint{4.154431in}{2.286869in}}%
\pgfpathlineto{\pgfqpoint{4.143303in}{2.283528in}}%
\pgfpathlineto{\pgfqpoint{4.149757in}{2.269089in}}%
\pgfpathlineto{\pgfqpoint{4.156214in}{2.254634in}}%
\pgfpathlineto{\pgfqpoint{4.162672in}{2.240160in}}%
\pgfpathlineto{\pgfqpoint{4.169132in}{2.225662in}}%
\pgfpathclose%
\pgfusepath{stroke,fill}%
\end{pgfscope}%
\begin{pgfscope}%
\pgfpathrectangle{\pgfqpoint{0.887500in}{0.275000in}}{\pgfqpoint{4.225000in}{4.225000in}}%
\pgfusepath{clip}%
\pgfsetbuttcap%
\pgfsetroundjoin%
\definecolor{currentfill}{rgb}{0.259857,0.745492,0.444467}%
\pgfsetfillcolor{currentfill}%
\pgfsetfillopacity{0.700000}%
\pgfsetlinewidth{0.501875pt}%
\definecolor{currentstroke}{rgb}{1.000000,1.000000,1.000000}%
\pgfsetstrokecolor{currentstroke}%
\pgfsetstrokeopacity{0.500000}%
\pgfsetdash{}{0pt}%
\pgfpathmoveto{\pgfqpoint{3.406090in}{2.766751in}}%
\pgfpathlineto{\pgfqpoint{3.417431in}{2.772017in}}%
\pgfpathlineto{\pgfqpoint{3.428763in}{2.776863in}}%
\pgfpathlineto{\pgfqpoint{3.440088in}{2.781371in}}%
\pgfpathlineto{\pgfqpoint{3.451405in}{2.785620in}}%
\pgfpathlineto{\pgfqpoint{3.462715in}{2.789692in}}%
\pgfpathlineto{\pgfqpoint{3.456380in}{2.803179in}}%
\pgfpathlineto{\pgfqpoint{3.450048in}{2.816595in}}%
\pgfpathlineto{\pgfqpoint{3.443717in}{2.829949in}}%
\pgfpathlineto{\pgfqpoint{3.437389in}{2.843253in}}%
\pgfpathlineto{\pgfqpoint{3.431063in}{2.856515in}}%
\pgfpathlineto{\pgfqpoint{3.419757in}{2.852385in}}%
\pgfpathlineto{\pgfqpoint{3.408444in}{2.848078in}}%
\pgfpathlineto{\pgfqpoint{3.397124in}{2.843508in}}%
\pgfpathlineto{\pgfqpoint{3.385796in}{2.838591in}}%
\pgfpathlineto{\pgfqpoint{3.374460in}{2.833241in}}%
\pgfpathlineto{\pgfqpoint{3.380782in}{2.820189in}}%
\pgfpathlineto{\pgfqpoint{3.387107in}{2.807060in}}%
\pgfpathlineto{\pgfqpoint{3.393434in}{2.793807in}}%
\pgfpathlineto{\pgfqpoint{3.399761in}{2.780386in}}%
\pgfpathclose%
\pgfusepath{stroke,fill}%
\end{pgfscope}%
\begin{pgfscope}%
\pgfpathrectangle{\pgfqpoint{0.887500in}{0.275000in}}{\pgfqpoint{4.225000in}{4.225000in}}%
\pgfusepath{clip}%
\pgfsetbuttcap%
\pgfsetroundjoin%
\definecolor{currentfill}{rgb}{0.120092,0.600104,0.542530}%
\pgfsetfillcolor{currentfill}%
\pgfsetfillopacity{0.700000}%
\pgfsetlinewidth{0.501875pt}%
\definecolor{currentstroke}{rgb}{1.000000,1.000000,1.000000}%
\pgfsetstrokecolor{currentstroke}%
\pgfsetstrokeopacity{0.500000}%
\pgfsetdash{}{0pt}%
\pgfpathmoveto{\pgfqpoint{3.879221in}{2.447646in}}%
\pgfpathlineto{\pgfqpoint{3.890428in}{2.451050in}}%
\pgfpathlineto{\pgfqpoint{3.901629in}{2.454445in}}%
\pgfpathlineto{\pgfqpoint{3.912825in}{2.457835in}}%
\pgfpathlineto{\pgfqpoint{3.924015in}{2.461222in}}%
\pgfpathlineto{\pgfqpoint{3.935199in}{2.464608in}}%
\pgfpathlineto{\pgfqpoint{3.928783in}{2.478781in}}%
\pgfpathlineto{\pgfqpoint{3.922367in}{2.492912in}}%
\pgfpathlineto{\pgfqpoint{3.915954in}{2.507005in}}%
\pgfpathlineto{\pgfqpoint{3.909542in}{2.521065in}}%
\pgfpathlineto{\pgfqpoint{3.903133in}{2.535096in}}%
\pgfpathlineto{\pgfqpoint{3.891950in}{2.531704in}}%
\pgfpathlineto{\pgfqpoint{3.880762in}{2.528305in}}%
\pgfpathlineto{\pgfqpoint{3.869569in}{2.524897in}}%
\pgfpathlineto{\pgfqpoint{3.858369in}{2.521477in}}%
\pgfpathlineto{\pgfqpoint{3.847164in}{2.518041in}}%
\pgfpathlineto{\pgfqpoint{3.853572in}{2.504043in}}%
\pgfpathlineto{\pgfqpoint{3.859982in}{2.490013in}}%
\pgfpathlineto{\pgfqpoint{3.866393in}{2.475943in}}%
\pgfpathlineto{\pgfqpoint{3.872806in}{2.461823in}}%
\pgfpathclose%
\pgfusepath{stroke,fill}%
\end{pgfscope}%
\begin{pgfscope}%
\pgfpathrectangle{\pgfqpoint{0.887500in}{0.275000in}}{\pgfqpoint{4.225000in}{4.225000in}}%
\pgfusepath{clip}%
\pgfsetbuttcap%
\pgfsetroundjoin%
\definecolor{currentfill}{rgb}{0.136408,0.541173,0.554483}%
\pgfsetfillcolor{currentfill}%
\pgfsetfillopacity{0.700000}%
\pgfsetlinewidth{0.501875pt}%
\definecolor{currentstroke}{rgb}{1.000000,1.000000,1.000000}%
\pgfsetstrokecolor{currentstroke}%
\pgfsetstrokeopacity{0.500000}%
\pgfsetdash{}{0pt}%
\pgfpathmoveto{\pgfqpoint{1.634690in}{2.328627in}}%
\pgfpathlineto{\pgfqpoint{1.646455in}{2.332080in}}%
\pgfpathlineto{\pgfqpoint{1.658215in}{2.335521in}}%
\pgfpathlineto{\pgfqpoint{1.669969in}{2.338951in}}%
\pgfpathlineto{\pgfqpoint{1.681717in}{2.342370in}}%
\pgfpathlineto{\pgfqpoint{1.693460in}{2.345781in}}%
\pgfpathlineto{\pgfqpoint{1.687664in}{2.354258in}}%
\pgfpathlineto{\pgfqpoint{1.681872in}{2.362711in}}%
\pgfpathlineto{\pgfqpoint{1.676084in}{2.371138in}}%
\pgfpathlineto{\pgfqpoint{1.670301in}{2.379540in}}%
\pgfpathlineto{\pgfqpoint{1.664522in}{2.387916in}}%
\pgfpathlineto{\pgfqpoint{1.652791in}{2.384532in}}%
\pgfpathlineto{\pgfqpoint{1.641055in}{2.381139in}}%
\pgfpathlineto{\pgfqpoint{1.629313in}{2.377736in}}%
\pgfpathlineto{\pgfqpoint{1.617566in}{2.374321in}}%
\pgfpathlineto{\pgfqpoint{1.605813in}{2.370894in}}%
\pgfpathlineto{\pgfqpoint{1.611579in}{2.362494in}}%
\pgfpathlineto{\pgfqpoint{1.617350in}{2.354068in}}%
\pgfpathlineto{\pgfqpoint{1.623125in}{2.345614in}}%
\pgfpathlineto{\pgfqpoint{1.628905in}{2.337134in}}%
\pgfpathclose%
\pgfusepath{stroke,fill}%
\end{pgfscope}%
\begin{pgfscope}%
\pgfpathrectangle{\pgfqpoint{0.887500in}{0.275000in}}{\pgfqpoint{4.225000in}{4.225000in}}%
\pgfusepath{clip}%
\pgfsetbuttcap%
\pgfsetroundjoin%
\definecolor{currentfill}{rgb}{0.159194,0.482237,0.558073}%
\pgfsetfillcolor{currentfill}%
\pgfsetfillopacity{0.700000}%
\pgfsetlinewidth{0.501875pt}%
\definecolor{currentstroke}{rgb}{1.000000,1.000000,1.000000}%
\pgfsetstrokecolor{currentstroke}%
\pgfsetstrokeopacity{0.500000}%
\pgfsetdash{}{0pt}%
\pgfpathmoveto{\pgfqpoint{2.279237in}{2.205052in}}%
\pgfpathlineto{\pgfqpoint{2.290842in}{2.208589in}}%
\pgfpathlineto{\pgfqpoint{2.302441in}{2.212114in}}%
\pgfpathlineto{\pgfqpoint{2.314035in}{2.215625in}}%
\pgfpathlineto{\pgfqpoint{2.325624in}{2.219121in}}%
\pgfpathlineto{\pgfqpoint{2.337207in}{2.222601in}}%
\pgfpathlineto{\pgfqpoint{2.331189in}{2.231478in}}%
\pgfpathlineto{\pgfqpoint{2.325175in}{2.240329in}}%
\pgfpathlineto{\pgfqpoint{2.319165in}{2.249154in}}%
\pgfpathlineto{\pgfqpoint{2.313159in}{2.257954in}}%
\pgfpathlineto{\pgfqpoint{2.307158in}{2.266730in}}%
\pgfpathlineto{\pgfqpoint{2.295586in}{2.263294in}}%
\pgfpathlineto{\pgfqpoint{2.284008in}{2.259843in}}%
\pgfpathlineto{\pgfqpoint{2.272425in}{2.256377in}}%
\pgfpathlineto{\pgfqpoint{2.260837in}{2.252897in}}%
\pgfpathlineto{\pgfqpoint{2.249242in}{2.249405in}}%
\pgfpathlineto{\pgfqpoint{2.255233in}{2.240587in}}%
\pgfpathlineto{\pgfqpoint{2.261227in}{2.231744in}}%
\pgfpathlineto{\pgfqpoint{2.267226in}{2.222873in}}%
\pgfpathlineto{\pgfqpoint{2.273229in}{2.213976in}}%
\pgfpathclose%
\pgfusepath{stroke,fill}%
\end{pgfscope}%
\begin{pgfscope}%
\pgfpathrectangle{\pgfqpoint{0.887500in}{0.275000in}}{\pgfqpoint{4.225000in}{4.225000in}}%
\pgfusepath{clip}%
\pgfsetbuttcap%
\pgfsetroundjoin%
\definecolor{currentfill}{rgb}{0.147607,0.511733,0.557049}%
\pgfsetfillcolor{currentfill}%
\pgfsetfillopacity{0.700000}%
\pgfsetlinewidth{0.501875pt}%
\definecolor{currentstroke}{rgb}{1.000000,1.000000,1.000000}%
\pgfsetstrokecolor{currentstroke}%
\pgfsetstrokeopacity{0.500000}%
\pgfsetdash{}{0pt}%
\pgfpathmoveto{\pgfqpoint{1.956928in}{2.268119in}}%
\pgfpathlineto{\pgfqpoint{1.968614in}{2.271578in}}%
\pgfpathlineto{\pgfqpoint{1.980295in}{2.275024in}}%
\pgfpathlineto{\pgfqpoint{1.991970in}{2.278456in}}%
\pgfpathlineto{\pgfqpoint{2.003640in}{2.281877in}}%
\pgfpathlineto{\pgfqpoint{2.015304in}{2.285286in}}%
\pgfpathlineto{\pgfqpoint{2.009395in}{2.293969in}}%
\pgfpathlineto{\pgfqpoint{2.003490in}{2.302628in}}%
\pgfpathlineto{\pgfqpoint{1.997590in}{2.311263in}}%
\pgfpathlineto{\pgfqpoint{1.991693in}{2.319875in}}%
\pgfpathlineto{\pgfqpoint{1.985802in}{2.328465in}}%
\pgfpathlineto{\pgfqpoint{1.974149in}{2.325092in}}%
\pgfpathlineto{\pgfqpoint{1.962490in}{2.321709in}}%
\pgfpathlineto{\pgfqpoint{1.950826in}{2.318315in}}%
\pgfpathlineto{\pgfqpoint{1.939157in}{2.314908in}}%
\pgfpathlineto{\pgfqpoint{1.927482in}{2.311489in}}%
\pgfpathlineto{\pgfqpoint{1.933362in}{2.302865in}}%
\pgfpathlineto{\pgfqpoint{1.939247in}{2.294217in}}%
\pgfpathlineto{\pgfqpoint{1.945136in}{2.285543in}}%
\pgfpathlineto{\pgfqpoint{1.951030in}{2.276844in}}%
\pgfpathclose%
\pgfusepath{stroke,fill}%
\end{pgfscope}%
\begin{pgfscope}%
\pgfpathrectangle{\pgfqpoint{0.887500in}{0.275000in}}{\pgfqpoint{4.225000in}{4.225000in}}%
\pgfusepath{clip}%
\pgfsetbuttcap%
\pgfsetroundjoin%
\definecolor{currentfill}{rgb}{0.203063,0.379716,0.553925}%
\pgfsetfillcolor{currentfill}%
\pgfsetfillopacity{0.700000}%
\pgfsetlinewidth{0.501875pt}%
\definecolor{currentstroke}{rgb}{1.000000,1.000000,1.000000}%
\pgfsetstrokecolor{currentstroke}%
\pgfsetstrokeopacity{0.500000}%
\pgfsetdash{}{0pt}%
\pgfpathmoveto{\pgfqpoint{4.472129in}{1.979297in}}%
\pgfpathlineto{\pgfqpoint{4.483183in}{1.982600in}}%
\pgfpathlineto{\pgfqpoint{4.494230in}{1.985875in}}%
\pgfpathlineto{\pgfqpoint{4.505271in}{1.989130in}}%
\pgfpathlineto{\pgfqpoint{4.516306in}{1.992373in}}%
\pgfpathlineto{\pgfqpoint{4.527335in}{1.995612in}}%
\pgfpathlineto{\pgfqpoint{4.520815in}{2.010140in}}%
\pgfpathlineto{\pgfqpoint{4.514297in}{2.024613in}}%
\pgfpathlineto{\pgfqpoint{4.507779in}{2.039028in}}%
\pgfpathlineto{\pgfqpoint{4.501260in}{2.053375in}}%
\pgfpathlineto{\pgfqpoint{4.494742in}{2.067651in}}%
\pgfpathlineto{\pgfqpoint{4.483710in}{2.064291in}}%
\pgfpathlineto{\pgfqpoint{4.472672in}{2.060921in}}%
\pgfpathlineto{\pgfqpoint{4.461627in}{2.057535in}}%
\pgfpathlineto{\pgfqpoint{4.450577in}{2.054130in}}%
\pgfpathlineto{\pgfqpoint{4.439520in}{2.050703in}}%
\pgfpathlineto{\pgfqpoint{4.446046in}{2.036683in}}%
\pgfpathlineto{\pgfqpoint{4.452569in}{2.022517in}}%
\pgfpathlineto{\pgfqpoint{4.459091in}{2.008222in}}%
\pgfpathlineto{\pgfqpoint{4.465610in}{1.993810in}}%
\pgfpathclose%
\pgfusepath{stroke,fill}%
\end{pgfscope}%
\begin{pgfscope}%
\pgfpathrectangle{\pgfqpoint{0.887500in}{0.275000in}}{\pgfqpoint{4.225000in}{4.225000in}}%
\pgfusepath{clip}%
\pgfsetbuttcap%
\pgfsetroundjoin%
\definecolor{currentfill}{rgb}{0.144759,0.519093,0.556572}%
\pgfsetfillcolor{currentfill}%
\pgfsetfillopacity{0.700000}%
\pgfsetlinewidth{0.501875pt}%
\definecolor{currentstroke}{rgb}{1.000000,1.000000,1.000000}%
\pgfsetstrokecolor{currentstroke}%
\pgfsetstrokeopacity{0.500000}%
\pgfsetdash{}{0pt}%
\pgfpathmoveto{\pgfqpoint{3.094983in}{2.225828in}}%
\pgfpathlineto{\pgfqpoint{3.106400in}{2.251218in}}%
\pgfpathlineto{\pgfqpoint{3.117822in}{2.275375in}}%
\pgfpathlineto{\pgfqpoint{3.129248in}{2.298293in}}%
\pgfpathlineto{\pgfqpoint{3.140677in}{2.320204in}}%
\pgfpathlineto{\pgfqpoint{3.152110in}{2.341338in}}%
\pgfpathlineto{\pgfqpoint{3.145827in}{2.353547in}}%
\pgfpathlineto{\pgfqpoint{3.139548in}{2.365751in}}%
\pgfpathlineto{\pgfqpoint{3.133271in}{2.378011in}}%
\pgfpathlineto{\pgfqpoint{3.126998in}{2.390388in}}%
\pgfpathlineto{\pgfqpoint{3.120728in}{2.402907in}}%
\pgfpathlineto{\pgfqpoint{3.109312in}{2.380047in}}%
\pgfpathlineto{\pgfqpoint{3.097902in}{2.356524in}}%
\pgfpathlineto{\pgfqpoint{3.086497in}{2.332065in}}%
\pgfpathlineto{\pgfqpoint{3.075098in}{2.306400in}}%
\pgfpathlineto{\pgfqpoint{3.063706in}{2.279511in}}%
\pgfpathlineto{\pgfqpoint{3.069954in}{2.268396in}}%
\pgfpathlineto{\pgfqpoint{3.076206in}{2.257515in}}%
\pgfpathlineto{\pgfqpoint{3.082461in}{2.246840in}}%
\pgfpathlineto{\pgfqpoint{3.088720in}{2.236301in}}%
\pgfpathclose%
\pgfusepath{stroke,fill}%
\end{pgfscope}%
\begin{pgfscope}%
\pgfpathrectangle{\pgfqpoint{0.887500in}{0.275000in}}{\pgfqpoint{4.225000in}{4.225000in}}%
\pgfusepath{clip}%
\pgfsetbuttcap%
\pgfsetroundjoin%
\definecolor{currentfill}{rgb}{0.177423,0.437527,0.557565}%
\pgfsetfillcolor{currentfill}%
\pgfsetfillopacity{0.700000}%
\pgfsetlinewidth{0.501875pt}%
\definecolor{currentstroke}{rgb}{1.000000,1.000000,1.000000}%
\pgfsetstrokecolor{currentstroke}%
\pgfsetstrokeopacity{0.500000}%
\pgfsetdash{}{0pt}%
\pgfpathmoveto{\pgfqpoint{3.069246in}{2.061584in}}%
\pgfpathlineto{\pgfqpoint{3.080653in}{2.079436in}}%
\pgfpathlineto{\pgfqpoint{3.092066in}{2.100306in}}%
\pgfpathlineto{\pgfqpoint{3.103486in}{2.123292in}}%
\pgfpathlineto{\pgfqpoint{3.114913in}{2.147491in}}%
\pgfpathlineto{\pgfqpoint{3.126347in}{2.171995in}}%
\pgfpathlineto{\pgfqpoint{3.120068in}{2.183191in}}%
\pgfpathlineto{\pgfqpoint{3.113791in}{2.194103in}}%
\pgfpathlineto{\pgfqpoint{3.107519in}{2.204799in}}%
\pgfpathlineto{\pgfqpoint{3.101249in}{2.215351in}}%
\pgfpathlineto{\pgfqpoint{3.094983in}{2.225828in}}%
\pgfpathlineto{\pgfqpoint{3.083573in}{2.199977in}}%
\pgfpathlineto{\pgfqpoint{3.072170in}{2.174586in}}%
\pgfpathlineto{\pgfqpoint{3.060775in}{2.150574in}}%
\pgfpathlineto{\pgfqpoint{3.049386in}{2.128859in}}%
\pgfpathlineto{\pgfqpoint{3.038000in}{2.110352in}}%
\pgfpathlineto{\pgfqpoint{3.044241in}{2.100707in}}%
\pgfpathlineto{\pgfqpoint{3.050487in}{2.091013in}}%
\pgfpathlineto{\pgfqpoint{3.056736in}{2.081264in}}%
\pgfpathlineto{\pgfqpoint{3.062989in}{2.071456in}}%
\pgfpathclose%
\pgfusepath{stroke,fill}%
\end{pgfscope}%
\begin{pgfscope}%
\pgfpathrectangle{\pgfqpoint{0.887500in}{0.275000in}}{\pgfqpoint{4.225000in}{4.225000in}}%
\pgfusepath{clip}%
\pgfsetbuttcap%
\pgfsetroundjoin%
\definecolor{currentfill}{rgb}{0.119738,0.603785,0.541400}%
\pgfsetfillcolor{currentfill}%
\pgfsetfillopacity{0.700000}%
\pgfsetlinewidth{0.501875pt}%
\definecolor{currentstroke}{rgb}{1.000000,1.000000,1.000000}%
\pgfsetstrokecolor{currentstroke}%
\pgfsetstrokeopacity{0.500000}%
\pgfsetdash{}{0pt}%
\pgfpathmoveto{\pgfqpoint{3.120728in}{2.402907in}}%
\pgfpathlineto{\pgfqpoint{3.132148in}{2.425376in}}%
\pgfpathlineto{\pgfqpoint{3.143573in}{2.447729in}}%
\pgfpathlineto{\pgfqpoint{3.155003in}{2.470241in}}%
\pgfpathlineto{\pgfqpoint{3.166440in}{2.493063in}}%
\pgfpathlineto{\pgfqpoint{3.177883in}{2.515936in}}%
\pgfpathlineto{\pgfqpoint{3.171596in}{2.528634in}}%
\pgfpathlineto{\pgfqpoint{3.165310in}{2.540806in}}%
\pgfpathlineto{\pgfqpoint{3.159026in}{2.552551in}}%
\pgfpathlineto{\pgfqpoint{3.152744in}{2.563970in}}%
\pgfpathlineto{\pgfqpoint{3.146465in}{2.575163in}}%
\pgfpathlineto{\pgfqpoint{3.135047in}{2.553555in}}%
\pgfpathlineto{\pgfqpoint{3.123633in}{2.531773in}}%
\pgfpathlineto{\pgfqpoint{3.112224in}{2.509947in}}%
\pgfpathlineto{\pgfqpoint{3.100820in}{2.487901in}}%
\pgfpathlineto{\pgfqpoint{3.089420in}{2.465366in}}%
\pgfpathlineto{\pgfqpoint{3.095676in}{2.453121in}}%
\pgfpathlineto{\pgfqpoint{3.101935in}{2.440684in}}%
\pgfpathlineto{\pgfqpoint{3.108196in}{2.428123in}}%
\pgfpathlineto{\pgfqpoint{3.114460in}{2.415508in}}%
\pgfpathclose%
\pgfusepath{stroke,fill}%
\end{pgfscope}%
\begin{pgfscope}%
\pgfpathrectangle{\pgfqpoint{0.887500in}{0.275000in}}{\pgfqpoint{4.225000in}{4.225000in}}%
\pgfusepath{clip}%
\pgfsetbuttcap%
\pgfsetroundjoin%
\definecolor{currentfill}{rgb}{0.121380,0.629492,0.531973}%
\pgfsetfillcolor{currentfill}%
\pgfsetfillopacity{0.700000}%
\pgfsetlinewidth{0.501875pt}%
\definecolor{currentstroke}{rgb}{1.000000,1.000000,1.000000}%
\pgfsetstrokecolor{currentstroke}%
\pgfsetstrokeopacity{0.500000}%
\pgfsetdash{}{0pt}%
\pgfpathmoveto{\pgfqpoint{3.791045in}{2.500286in}}%
\pgfpathlineto{\pgfqpoint{3.802282in}{2.503948in}}%
\pgfpathlineto{\pgfqpoint{3.813512in}{2.507544in}}%
\pgfpathlineto{\pgfqpoint{3.824736in}{2.511085in}}%
\pgfpathlineto{\pgfqpoint{3.835953in}{2.514581in}}%
\pgfpathlineto{\pgfqpoint{3.847164in}{2.518041in}}%
\pgfpathlineto{\pgfqpoint{3.840758in}{2.532017in}}%
\pgfpathlineto{\pgfqpoint{3.834355in}{2.545979in}}%
\pgfpathlineto{\pgfqpoint{3.827954in}{2.559936in}}%
\pgfpathlineto{\pgfqpoint{3.821556in}{2.573894in}}%
\pgfpathlineto{\pgfqpoint{3.815160in}{2.587851in}}%
\pgfpathlineto{\pgfqpoint{3.803951in}{2.584382in}}%
\pgfpathlineto{\pgfqpoint{3.792737in}{2.580882in}}%
\pgfpathlineto{\pgfqpoint{3.781516in}{2.577345in}}%
\pgfpathlineto{\pgfqpoint{3.770289in}{2.573761in}}%
\pgfpathlineto{\pgfqpoint{3.759055in}{2.570122in}}%
\pgfpathlineto{\pgfqpoint{3.765447in}{2.556113in}}%
\pgfpathlineto{\pgfqpoint{3.771842in}{2.542127in}}%
\pgfpathlineto{\pgfqpoint{3.778241in}{2.528169in}}%
\pgfpathlineto{\pgfqpoint{3.784642in}{2.514228in}}%
\pgfpathclose%
\pgfusepath{stroke,fill}%
\end{pgfscope}%
\begin{pgfscope}%
\pgfpathrectangle{\pgfqpoint{0.887500in}{0.275000in}}{\pgfqpoint{4.225000in}{4.225000in}}%
\pgfusepath{clip}%
\pgfsetbuttcap%
\pgfsetroundjoin%
\definecolor{currentfill}{rgb}{0.246811,0.283237,0.535941}%
\pgfsetfillcolor{currentfill}%
\pgfsetfillopacity{0.700000}%
\pgfsetlinewidth{0.501875pt}%
\definecolor{currentstroke}{rgb}{1.000000,1.000000,1.000000}%
\pgfsetstrokecolor{currentstroke}%
\pgfsetstrokeopacity{0.500000}%
\pgfsetdash{}{0pt}%
\pgfpathmoveto{\pgfqpoint{4.680426in}{1.791855in}}%
\pgfpathlineto{\pgfqpoint{4.691426in}{1.795125in}}%
\pgfpathlineto{\pgfqpoint{4.702420in}{1.798401in}}%
\pgfpathlineto{\pgfqpoint{4.713408in}{1.801682in}}%
\pgfpathlineto{\pgfqpoint{4.724392in}{1.804971in}}%
\pgfpathlineto{\pgfqpoint{4.735370in}{1.808268in}}%
\pgfpathlineto{\pgfqpoint{4.728827in}{1.823165in}}%
\pgfpathlineto{\pgfqpoint{4.722283in}{1.838018in}}%
\pgfpathlineto{\pgfqpoint{4.715741in}{1.852829in}}%
\pgfpathlineto{\pgfqpoint{4.709199in}{1.867602in}}%
\pgfpathlineto{\pgfqpoint{4.702659in}{1.882339in}}%
\pgfpathlineto{\pgfqpoint{4.691679in}{1.879006in}}%
\pgfpathlineto{\pgfqpoint{4.680694in}{1.875673in}}%
\pgfpathlineto{\pgfqpoint{4.669704in}{1.872339in}}%
\pgfpathlineto{\pgfqpoint{4.658708in}{1.869001in}}%
\pgfpathlineto{\pgfqpoint{4.647706in}{1.865658in}}%
\pgfpathlineto{\pgfqpoint{4.654246in}{1.850935in}}%
\pgfpathlineto{\pgfqpoint{4.660789in}{1.836194in}}%
\pgfpathlineto{\pgfqpoint{4.667333in}{1.821433in}}%
\pgfpathlineto{\pgfqpoint{4.673879in}{1.806654in}}%
\pgfpathclose%
\pgfusepath{stroke,fill}%
\end{pgfscope}%
\begin{pgfscope}%
\pgfpathrectangle{\pgfqpoint{0.887500in}{0.275000in}}{\pgfqpoint{4.225000in}{4.225000in}}%
\pgfusepath{clip}%
\pgfsetbuttcap%
\pgfsetroundjoin%
\definecolor{currentfill}{rgb}{0.175841,0.441290,0.557685}%
\pgfsetfillcolor{currentfill}%
\pgfsetfillopacity{0.700000}%
\pgfsetlinewidth{0.501875pt}%
\definecolor{currentstroke}{rgb}{1.000000,1.000000,1.000000}%
\pgfsetstrokecolor{currentstroke}%
\pgfsetstrokeopacity{0.500000}%
\pgfsetdash{}{0pt}%
\pgfpathmoveto{\pgfqpoint{2.689813in}{2.110479in}}%
\pgfpathlineto{\pgfqpoint{2.701320in}{2.114030in}}%
\pgfpathlineto{\pgfqpoint{2.712822in}{2.117493in}}%
\pgfpathlineto{\pgfqpoint{2.724320in}{2.120845in}}%
\pgfpathlineto{\pgfqpoint{2.735813in}{2.124058in}}%
\pgfpathlineto{\pgfqpoint{2.747301in}{2.127154in}}%
\pgfpathlineto{\pgfqpoint{2.741144in}{2.136460in}}%
\pgfpathlineto{\pgfqpoint{2.734992in}{2.145732in}}%
\pgfpathlineto{\pgfqpoint{2.728844in}{2.154971in}}%
\pgfpathlineto{\pgfqpoint{2.722701in}{2.164178in}}%
\pgfpathlineto{\pgfqpoint{2.716561in}{2.173355in}}%
\pgfpathlineto{\pgfqpoint{2.705083in}{2.170305in}}%
\pgfpathlineto{\pgfqpoint{2.693600in}{2.167133in}}%
\pgfpathlineto{\pgfqpoint{2.682113in}{2.163816in}}%
\pgfpathlineto{\pgfqpoint{2.670621in}{2.160383in}}%
\pgfpathlineto{\pgfqpoint{2.659124in}{2.156861in}}%
\pgfpathlineto{\pgfqpoint{2.665253in}{2.147655in}}%
\pgfpathlineto{\pgfqpoint{2.671387in}{2.138415in}}%
\pgfpathlineto{\pgfqpoint{2.677525in}{2.129140in}}%
\pgfpathlineto{\pgfqpoint{2.683667in}{2.119828in}}%
\pgfpathclose%
\pgfusepath{stroke,fill}%
\end{pgfscope}%
\begin{pgfscope}%
\pgfpathrectangle{\pgfqpoint{0.887500in}{0.275000in}}{\pgfqpoint{4.225000in}{4.225000in}}%
\pgfusepath{clip}%
\pgfsetbuttcap%
\pgfsetroundjoin%
\definecolor{currentfill}{rgb}{0.144759,0.519093,0.556572}%
\pgfsetfillcolor{currentfill}%
\pgfsetfillopacity{0.700000}%
\pgfsetlinewidth{0.501875pt}%
\definecolor{currentstroke}{rgb}{1.000000,1.000000,1.000000}%
\pgfsetstrokecolor{currentstroke}%
\pgfsetstrokeopacity{0.500000}%
\pgfsetdash{}{0pt}%
\pgfpathmoveto{\pgfqpoint{4.087570in}{2.266546in}}%
\pgfpathlineto{\pgfqpoint{4.098728in}{2.269961in}}%
\pgfpathlineto{\pgfqpoint{4.109880in}{2.273372in}}%
\pgfpathlineto{\pgfqpoint{4.121027in}{2.276775in}}%
\pgfpathlineto{\pgfqpoint{4.132168in}{2.280162in}}%
\pgfpathlineto{\pgfqpoint{4.143303in}{2.283528in}}%
\pgfpathlineto{\pgfqpoint{4.136850in}{2.297957in}}%
\pgfpathlineto{\pgfqpoint{4.130401in}{2.312382in}}%
\pgfpathlineto{\pgfqpoint{4.123953in}{2.326804in}}%
\pgfpathlineto{\pgfqpoint{4.117508in}{2.341222in}}%
\pgfpathlineto{\pgfqpoint{4.111066in}{2.355632in}}%
\pgfpathlineto{\pgfqpoint{4.099930in}{2.352177in}}%
\pgfpathlineto{\pgfqpoint{4.088789in}{2.348700in}}%
\pgfpathlineto{\pgfqpoint{4.077641in}{2.345207in}}%
\pgfpathlineto{\pgfqpoint{4.066488in}{2.341705in}}%
\pgfpathlineto{\pgfqpoint{4.055330in}{2.338199in}}%
\pgfpathlineto{\pgfqpoint{4.061771in}{2.323810in}}%
\pgfpathlineto{\pgfqpoint{4.068216in}{2.309445in}}%
\pgfpathlineto{\pgfqpoint{4.074664in}{2.295113in}}%
\pgfpathlineto{\pgfqpoint{4.081115in}{2.280819in}}%
\pgfpathclose%
\pgfusepath{stroke,fill}%
\end{pgfscope}%
\begin{pgfscope}%
\pgfpathrectangle{\pgfqpoint{0.887500in}{0.275000in}}{\pgfqpoint{4.225000in}{4.225000in}}%
\pgfusepath{clip}%
\pgfsetbuttcap%
\pgfsetroundjoin%
\definecolor{currentfill}{rgb}{0.190631,0.407061,0.556089}%
\pgfsetfillcolor{currentfill}%
\pgfsetfillopacity{0.700000}%
\pgfsetlinewidth{0.501875pt}%
\definecolor{currentstroke}{rgb}{1.000000,1.000000,1.000000}%
\pgfsetstrokecolor{currentstroke}%
\pgfsetstrokeopacity{0.500000}%
\pgfsetdash{}{0pt}%
\pgfpathmoveto{\pgfqpoint{4.384139in}{2.033067in}}%
\pgfpathlineto{\pgfqpoint{4.395229in}{2.036684in}}%
\pgfpathlineto{\pgfqpoint{4.406312in}{2.040247in}}%
\pgfpathlineto{\pgfqpoint{4.417388in}{2.043766in}}%
\pgfpathlineto{\pgfqpoint{4.428458in}{2.047249in}}%
\pgfpathlineto{\pgfqpoint{4.439520in}{2.050703in}}%
\pgfpathlineto{\pgfqpoint{4.432992in}{2.064564in}}%
\pgfpathlineto{\pgfqpoint{4.426460in}{2.078261in}}%
\pgfpathlineto{\pgfqpoint{4.419927in}{2.091826in}}%
\pgfpathlineto{\pgfqpoint{4.413394in}{2.105299in}}%
\pgfpathlineto{\pgfqpoint{4.406861in}{2.118717in}}%
\pgfpathlineto{\pgfqpoint{4.395792in}{2.115019in}}%
\pgfpathlineto{\pgfqpoint{4.384722in}{2.111438in}}%
\pgfpathlineto{\pgfqpoint{4.373651in}{2.107973in}}%
\pgfpathlineto{\pgfqpoint{4.362577in}{2.104589in}}%
\pgfpathlineto{\pgfqpoint{4.351498in}{2.101250in}}%
\pgfpathlineto{\pgfqpoint{4.358025in}{2.087751in}}%
\pgfpathlineto{\pgfqpoint{4.364554in}{2.074218in}}%
\pgfpathlineto{\pgfqpoint{4.371083in}{2.060614in}}%
\pgfpathlineto{\pgfqpoint{4.377612in}{2.046907in}}%
\pgfpathclose%
\pgfusepath{stroke,fill}%
\end{pgfscope}%
\begin{pgfscope}%
\pgfpathrectangle{\pgfqpoint{0.887500in}{0.275000in}}{\pgfqpoint{4.225000in}{4.225000in}}%
\pgfusepath{clip}%
\pgfsetbuttcap%
\pgfsetroundjoin%
\definecolor{currentfill}{rgb}{0.259857,0.745492,0.444467}%
\pgfsetfillcolor{currentfill}%
\pgfsetfillopacity{0.700000}%
\pgfsetlinewidth{0.501875pt}%
\definecolor{currentstroke}{rgb}{1.000000,1.000000,1.000000}%
\pgfsetstrokecolor{currentstroke}%
\pgfsetstrokeopacity{0.500000}%
\pgfsetdash{}{0pt}%
\pgfpathmoveto{\pgfqpoint{3.260683in}{2.746005in}}%
\pgfpathlineto{\pgfqpoint{3.272088in}{2.757662in}}%
\pgfpathlineto{\pgfqpoint{3.283488in}{2.768618in}}%
\pgfpathlineto{\pgfqpoint{3.294883in}{2.778932in}}%
\pgfpathlineto{\pgfqpoint{3.306272in}{2.788608in}}%
\pgfpathlineto{\pgfqpoint{3.317655in}{2.797644in}}%
\pgfpathlineto{\pgfqpoint{3.311340in}{2.810285in}}%
\pgfpathlineto{\pgfqpoint{3.305028in}{2.822976in}}%
\pgfpathlineto{\pgfqpoint{3.298720in}{2.835805in}}%
\pgfpathlineto{\pgfqpoint{3.292416in}{2.848858in}}%
\pgfpathlineto{\pgfqpoint{3.286116in}{2.862211in}}%
\pgfpathlineto{\pgfqpoint{3.274740in}{2.852707in}}%
\pgfpathlineto{\pgfqpoint{3.263357in}{2.842434in}}%
\pgfpathlineto{\pgfqpoint{3.251969in}{2.831371in}}%
\pgfpathlineto{\pgfqpoint{3.240576in}{2.819498in}}%
\pgfpathlineto{\pgfqpoint{3.229179in}{2.806790in}}%
\pgfpathlineto{\pgfqpoint{3.235468in}{2.793483in}}%
\pgfpathlineto{\pgfqpoint{3.241764in}{2.780927in}}%
\pgfpathlineto{\pgfqpoint{3.248066in}{2.768955in}}%
\pgfpathlineto{\pgfqpoint{3.254372in}{2.757378in}}%
\pgfpathclose%
\pgfusepath{stroke,fill}%
\end{pgfscope}%
\begin{pgfscope}%
\pgfpathrectangle{\pgfqpoint{0.887500in}{0.275000in}}{\pgfqpoint{4.225000in}{4.225000in}}%
\pgfusepath{clip}%
\pgfsetbuttcap%
\pgfsetroundjoin%
\definecolor{currentfill}{rgb}{0.208030,0.718701,0.472873}%
\pgfsetfillcolor{currentfill}%
\pgfsetfillopacity{0.700000}%
\pgfsetlinewidth{0.501875pt}%
\definecolor{currentstroke}{rgb}{1.000000,1.000000,1.000000}%
\pgfsetstrokecolor{currentstroke}%
\pgfsetstrokeopacity{0.500000}%
\pgfsetdash{}{0pt}%
\pgfpathmoveto{\pgfqpoint{3.203595in}{2.673857in}}%
\pgfpathlineto{\pgfqpoint{3.215020in}{2.690454in}}%
\pgfpathlineto{\pgfqpoint{3.226442in}{2.705872in}}%
\pgfpathlineto{\pgfqpoint{3.237860in}{2.720206in}}%
\pgfpathlineto{\pgfqpoint{3.249274in}{2.733552in}}%
\pgfpathlineto{\pgfqpoint{3.260683in}{2.746005in}}%
\pgfpathlineto{\pgfqpoint{3.254372in}{2.757378in}}%
\pgfpathlineto{\pgfqpoint{3.248066in}{2.768955in}}%
\pgfpathlineto{\pgfqpoint{3.241764in}{2.780927in}}%
\pgfpathlineto{\pgfqpoint{3.235468in}{2.793483in}}%
\pgfpathlineto{\pgfqpoint{3.229179in}{2.806790in}}%
\pgfpathlineto{\pgfqpoint{3.217778in}{2.793225in}}%
\pgfpathlineto{\pgfqpoint{3.206374in}{2.778776in}}%
\pgfpathlineto{\pgfqpoint{3.194967in}{2.763420in}}%
\pgfpathlineto{\pgfqpoint{3.183559in}{2.747133in}}%
\pgfpathlineto{\pgfqpoint{3.172150in}{2.729889in}}%
\pgfpathlineto{\pgfqpoint{3.178427in}{2.716946in}}%
\pgfpathlineto{\pgfqpoint{3.184710in}{2.705129in}}%
\pgfpathlineto{\pgfqpoint{3.191000in}{2.694192in}}%
\pgfpathlineto{\pgfqpoint{3.197295in}{2.683860in}}%
\pgfpathclose%
\pgfusepath{stroke,fill}%
\end{pgfscope}%
\begin{pgfscope}%
\pgfpathrectangle{\pgfqpoint{0.887500in}{0.275000in}}{\pgfqpoint{4.225000in}{4.225000in}}%
\pgfusepath{clip}%
\pgfsetbuttcap%
\pgfsetroundjoin%
\definecolor{currentfill}{rgb}{0.163625,0.471133,0.558148}%
\pgfsetfillcolor{currentfill}%
\pgfsetfillopacity{0.700000}%
\pgfsetlinewidth{0.501875pt}%
\definecolor{currentstroke}{rgb}{1.000000,1.000000,1.000000}%
\pgfsetstrokecolor{currentstroke}%
\pgfsetstrokeopacity{0.500000}%
\pgfsetdash{}{0pt}%
\pgfpathmoveto{\pgfqpoint{2.367364in}{2.177805in}}%
\pgfpathlineto{\pgfqpoint{2.378952in}{2.181314in}}%
\pgfpathlineto{\pgfqpoint{2.390535in}{2.184804in}}%
\pgfpathlineto{\pgfqpoint{2.402113in}{2.188271in}}%
\pgfpathlineto{\pgfqpoint{2.413685in}{2.191717in}}%
\pgfpathlineto{\pgfqpoint{2.425252in}{2.195143in}}%
\pgfpathlineto{\pgfqpoint{2.419202in}{2.204111in}}%
\pgfpathlineto{\pgfqpoint{2.413155in}{2.213051in}}%
\pgfpathlineto{\pgfqpoint{2.407113in}{2.221964in}}%
\pgfpathlineto{\pgfqpoint{2.401076in}{2.230850in}}%
\pgfpathlineto{\pgfqpoint{2.395042in}{2.239709in}}%
\pgfpathlineto{\pgfqpoint{2.383486in}{2.236326in}}%
\pgfpathlineto{\pgfqpoint{2.371924in}{2.232926in}}%
\pgfpathlineto{\pgfqpoint{2.360357in}{2.229504in}}%
\pgfpathlineto{\pgfqpoint{2.348785in}{2.226062in}}%
\pgfpathlineto{\pgfqpoint{2.337207in}{2.222601in}}%
\pgfpathlineto{\pgfqpoint{2.343230in}{2.213697in}}%
\pgfpathlineto{\pgfqpoint{2.349257in}{2.204766in}}%
\pgfpathlineto{\pgfqpoint{2.355288in}{2.195807in}}%
\pgfpathlineto{\pgfqpoint{2.361324in}{2.186820in}}%
\pgfpathclose%
\pgfusepath{stroke,fill}%
\end{pgfscope}%
\begin{pgfscope}%
\pgfpathrectangle{\pgfqpoint{0.887500in}{0.275000in}}{\pgfqpoint{4.225000in}{4.225000in}}%
\pgfusepath{clip}%
\pgfsetbuttcap%
\pgfsetroundjoin%
\definecolor{currentfill}{rgb}{0.140536,0.530132,0.555659}%
\pgfsetfillcolor{currentfill}%
\pgfsetfillopacity{0.700000}%
\pgfsetlinewidth{0.501875pt}%
\definecolor{currentstroke}{rgb}{1.000000,1.000000,1.000000}%
\pgfsetstrokecolor{currentstroke}%
\pgfsetstrokeopacity{0.500000}%
\pgfsetdash{}{0pt}%
\pgfpathmoveto{\pgfqpoint{1.722512in}{2.303017in}}%
\pgfpathlineto{\pgfqpoint{1.734261in}{2.306455in}}%
\pgfpathlineto{\pgfqpoint{1.746005in}{2.309887in}}%
\pgfpathlineto{\pgfqpoint{1.757743in}{2.313312in}}%
\pgfpathlineto{\pgfqpoint{1.769475in}{2.316734in}}%
\pgfpathlineto{\pgfqpoint{1.781201in}{2.320151in}}%
\pgfpathlineto{\pgfqpoint{1.775370in}{2.328716in}}%
\pgfpathlineto{\pgfqpoint{1.769544in}{2.337257in}}%
\pgfpathlineto{\pgfqpoint{1.763722in}{2.345776in}}%
\pgfpathlineto{\pgfqpoint{1.757904in}{2.354270in}}%
\pgfpathlineto{\pgfqpoint{1.752091in}{2.362741in}}%
\pgfpathlineto{\pgfqpoint{1.740376in}{2.359358in}}%
\pgfpathlineto{\pgfqpoint{1.728656in}{2.355972in}}%
\pgfpathlineto{\pgfqpoint{1.716930in}{2.352581in}}%
\pgfpathlineto{\pgfqpoint{1.705198in}{2.349184in}}%
\pgfpathlineto{\pgfqpoint{1.693460in}{2.345781in}}%
\pgfpathlineto{\pgfqpoint{1.699262in}{2.337279in}}%
\pgfpathlineto{\pgfqpoint{1.705067in}{2.328751in}}%
\pgfpathlineto{\pgfqpoint{1.710878in}{2.320198in}}%
\pgfpathlineto{\pgfqpoint{1.716693in}{2.311620in}}%
\pgfpathclose%
\pgfusepath{stroke,fill}%
\end{pgfscope}%
\begin{pgfscope}%
\pgfpathrectangle{\pgfqpoint{0.887500in}{0.275000in}}{\pgfqpoint{4.225000in}{4.225000in}}%
\pgfusepath{clip}%
\pgfsetbuttcap%
\pgfsetroundjoin%
\definecolor{currentfill}{rgb}{0.151918,0.500685,0.557587}%
\pgfsetfillcolor{currentfill}%
\pgfsetfillopacity{0.700000}%
\pgfsetlinewidth{0.501875pt}%
\definecolor{currentstroke}{rgb}{1.000000,1.000000,1.000000}%
\pgfsetstrokecolor{currentstroke}%
\pgfsetstrokeopacity{0.500000}%
\pgfsetdash{}{0pt}%
\pgfpathmoveto{\pgfqpoint{2.044917in}{2.241508in}}%
\pgfpathlineto{\pgfqpoint{2.056586in}{2.244957in}}%
\pgfpathlineto{\pgfqpoint{2.068250in}{2.248396in}}%
\pgfpathlineto{\pgfqpoint{2.079909in}{2.251827in}}%
\pgfpathlineto{\pgfqpoint{2.091562in}{2.255251in}}%
\pgfpathlineto{\pgfqpoint{2.103209in}{2.258671in}}%
\pgfpathlineto{\pgfqpoint{2.097267in}{2.267427in}}%
\pgfpathlineto{\pgfqpoint{2.091329in}{2.276159in}}%
\pgfpathlineto{\pgfqpoint{2.085395in}{2.284868in}}%
\pgfpathlineto{\pgfqpoint{2.079466in}{2.293554in}}%
\pgfpathlineto{\pgfqpoint{2.073541in}{2.302218in}}%
\pgfpathlineto{\pgfqpoint{2.061905in}{2.298842in}}%
\pgfpathlineto{\pgfqpoint{2.050264in}{2.295463in}}%
\pgfpathlineto{\pgfqpoint{2.038616in}{2.292078in}}%
\pgfpathlineto{\pgfqpoint{2.026963in}{2.288687in}}%
\pgfpathlineto{\pgfqpoint{2.015304in}{2.285286in}}%
\pgfpathlineto{\pgfqpoint{2.021218in}{2.276580in}}%
\pgfpathlineto{\pgfqpoint{2.027136in}{2.267849in}}%
\pgfpathlineto{\pgfqpoint{2.033059in}{2.259093in}}%
\pgfpathlineto{\pgfqpoint{2.038985in}{2.250313in}}%
\pgfpathclose%
\pgfusepath{stroke,fill}%
\end{pgfscope}%
\begin{pgfscope}%
\pgfpathrectangle{\pgfqpoint{0.887500in}{0.275000in}}{\pgfqpoint{4.225000in}{4.225000in}}%
\pgfusepath{clip}%
\pgfsetbuttcap%
\pgfsetroundjoin%
\definecolor{currentfill}{rgb}{0.132268,0.655014,0.519661}%
\pgfsetfillcolor{currentfill}%
\pgfsetfillopacity{0.700000}%
\pgfsetlinewidth{0.501875pt}%
\definecolor{currentstroke}{rgb}{1.000000,1.000000,1.000000}%
\pgfsetstrokecolor{currentstroke}%
\pgfsetstrokeopacity{0.500000}%
\pgfsetdash{}{0pt}%
\pgfpathmoveto{\pgfqpoint{3.702794in}{2.551054in}}%
\pgfpathlineto{\pgfqpoint{3.714058in}{2.554961in}}%
\pgfpathlineto{\pgfqpoint{3.725317in}{2.558832in}}%
\pgfpathlineto{\pgfqpoint{3.736569in}{2.562656in}}%
\pgfpathlineto{\pgfqpoint{3.747816in}{2.566422in}}%
\pgfpathlineto{\pgfqpoint{3.759055in}{2.570122in}}%
\pgfpathlineto{\pgfqpoint{3.752666in}{2.584145in}}%
\pgfpathlineto{\pgfqpoint{3.746280in}{2.598175in}}%
\pgfpathlineto{\pgfqpoint{3.739895in}{2.612201in}}%
\pgfpathlineto{\pgfqpoint{3.733514in}{2.626217in}}%
\pgfpathlineto{\pgfqpoint{3.727134in}{2.640213in}}%
\pgfpathlineto{\pgfqpoint{3.715899in}{2.636606in}}%
\pgfpathlineto{\pgfqpoint{3.704657in}{2.632930in}}%
\pgfpathlineto{\pgfqpoint{3.693409in}{2.629197in}}%
\pgfpathlineto{\pgfqpoint{3.682155in}{2.625424in}}%
\pgfpathlineto{\pgfqpoint{3.670895in}{2.621625in}}%
\pgfpathlineto{\pgfqpoint{3.677270in}{2.607539in}}%
\pgfpathlineto{\pgfqpoint{3.683647in}{2.593429in}}%
\pgfpathlineto{\pgfqpoint{3.690027in}{2.579306in}}%
\pgfpathlineto{\pgfqpoint{3.696409in}{2.565177in}}%
\pgfpathclose%
\pgfusepath{stroke,fill}%
\end{pgfscope}%
\begin{pgfscope}%
\pgfpathrectangle{\pgfqpoint{0.887500in}{0.275000in}}{\pgfqpoint{4.225000in}{4.225000in}}%
\pgfusepath{clip}%
\pgfsetbuttcap%
\pgfsetroundjoin%
\definecolor{currentfill}{rgb}{0.179019,0.433756,0.557430}%
\pgfsetfillcolor{currentfill}%
\pgfsetfillopacity{0.700000}%
\pgfsetlinewidth{0.501875pt}%
\definecolor{currentstroke}{rgb}{1.000000,1.000000,1.000000}%
\pgfsetstrokecolor{currentstroke}%
\pgfsetstrokeopacity{0.500000}%
\pgfsetdash{}{0pt}%
\pgfpathmoveto{\pgfqpoint{4.296005in}{2.083920in}}%
\pgfpathlineto{\pgfqpoint{4.307120in}{2.087577in}}%
\pgfpathlineto{\pgfqpoint{4.318227in}{2.091117in}}%
\pgfpathlineto{\pgfqpoint{4.329325in}{2.094551in}}%
\pgfpathlineto{\pgfqpoint{4.340415in}{2.097916in}}%
\pgfpathlineto{\pgfqpoint{4.351498in}{2.101250in}}%
\pgfpathlineto{\pgfqpoint{4.344974in}{2.114747in}}%
\pgfpathlineto{\pgfqpoint{4.338454in}{2.128277in}}%
\pgfpathlineto{\pgfqpoint{4.331939in}{2.141876in}}%
\pgfpathlineto{\pgfqpoint{4.325431in}{2.155577in}}%
\pgfpathlineto{\pgfqpoint{4.318929in}{2.169415in}}%
\pgfpathlineto{\pgfqpoint{4.307862in}{2.166524in}}%
\pgfpathlineto{\pgfqpoint{4.296790in}{2.163662in}}%
\pgfpathlineto{\pgfqpoint{4.285712in}{2.160765in}}%
\pgfpathlineto{\pgfqpoint{4.274624in}{2.157773in}}%
\pgfpathlineto{\pgfqpoint{4.263528in}{2.154666in}}%
\pgfpathlineto{\pgfqpoint{4.270015in}{2.140417in}}%
\pgfpathlineto{\pgfqpoint{4.276506in}{2.126225in}}%
\pgfpathlineto{\pgfqpoint{4.283002in}{2.112083in}}%
\pgfpathlineto{\pgfqpoint{4.289501in}{2.097984in}}%
\pgfpathclose%
\pgfusepath{stroke,fill}%
\end{pgfscope}%
\begin{pgfscope}%
\pgfpathrectangle{\pgfqpoint{0.887500in}{0.275000in}}{\pgfqpoint{4.225000in}{4.225000in}}%
\pgfusepath{clip}%
\pgfsetbuttcap%
\pgfsetroundjoin%
\definecolor{currentfill}{rgb}{0.182256,0.426184,0.557120}%
\pgfsetfillcolor{currentfill}%
\pgfsetfillopacity{0.700000}%
\pgfsetlinewidth{0.501875pt}%
\definecolor{currentstroke}{rgb}{1.000000,1.000000,1.000000}%
\pgfsetstrokecolor{currentstroke}%
\pgfsetstrokeopacity{0.500000}%
\pgfsetdash{}{0pt}%
\pgfpathmoveto{\pgfqpoint{2.778143in}{2.080070in}}%
\pgfpathlineto{\pgfqpoint{2.789635in}{2.083196in}}%
\pgfpathlineto{\pgfqpoint{2.801121in}{2.086365in}}%
\pgfpathlineto{\pgfqpoint{2.812600in}{2.089655in}}%
\pgfpathlineto{\pgfqpoint{2.824073in}{2.093146in}}%
\pgfpathlineto{\pgfqpoint{2.835538in}{2.096915in}}%
\pgfpathlineto{\pgfqpoint{2.829351in}{2.106366in}}%
\pgfpathlineto{\pgfqpoint{2.823168in}{2.115779in}}%
\pgfpathlineto{\pgfqpoint{2.816990in}{2.125155in}}%
\pgfpathlineto{\pgfqpoint{2.810815in}{2.134495in}}%
\pgfpathlineto{\pgfqpoint{2.804645in}{2.143801in}}%
\pgfpathlineto{\pgfqpoint{2.793190in}{2.140030in}}%
\pgfpathlineto{\pgfqpoint{2.781728in}{2.136566in}}%
\pgfpathlineto{\pgfqpoint{2.770259in}{2.133322in}}%
\pgfpathlineto{\pgfqpoint{2.758783in}{2.130213in}}%
\pgfpathlineto{\pgfqpoint{2.747301in}{2.127154in}}%
\pgfpathlineto{\pgfqpoint{2.753461in}{2.117813in}}%
\pgfpathlineto{\pgfqpoint{2.759625in}{2.108435in}}%
\pgfpathlineto{\pgfqpoint{2.765794in}{2.099020in}}%
\pgfpathlineto{\pgfqpoint{2.771967in}{2.089565in}}%
\pgfpathclose%
\pgfusepath{stroke,fill}%
\end{pgfscope}%
\begin{pgfscope}%
\pgfpathrectangle{\pgfqpoint{0.887500in}{0.275000in}}{\pgfqpoint{4.225000in}{4.225000in}}%
\pgfusepath{clip}%
\pgfsetbuttcap%
\pgfsetroundjoin%
\definecolor{currentfill}{rgb}{0.133743,0.548535,0.553541}%
\pgfsetfillcolor{currentfill}%
\pgfsetfillopacity{0.700000}%
\pgfsetlinewidth{0.501875pt}%
\definecolor{currentstroke}{rgb}{1.000000,1.000000,1.000000}%
\pgfsetstrokecolor{currentstroke}%
\pgfsetstrokeopacity{0.500000}%
\pgfsetdash{}{0pt}%
\pgfpathmoveto{\pgfqpoint{3.999458in}{2.320766in}}%
\pgfpathlineto{\pgfqpoint{4.010643in}{2.324251in}}%
\pgfpathlineto{\pgfqpoint{4.021823in}{2.327725in}}%
\pgfpathlineto{\pgfqpoint{4.032997in}{2.331204in}}%
\pgfpathlineto{\pgfqpoint{4.044166in}{2.334697in}}%
\pgfpathlineto{\pgfqpoint{4.055330in}{2.338199in}}%
\pgfpathlineto{\pgfqpoint{4.048891in}{2.352604in}}%
\pgfpathlineto{\pgfqpoint{4.042455in}{2.367016in}}%
\pgfpathlineto{\pgfqpoint{4.036022in}{2.381427in}}%
\pgfpathlineto{\pgfqpoint{4.029591in}{2.395827in}}%
\pgfpathlineto{\pgfqpoint{4.023162in}{2.410209in}}%
\pgfpathlineto{\pgfqpoint{4.012001in}{2.406751in}}%
\pgfpathlineto{\pgfqpoint{4.000835in}{2.403298in}}%
\pgfpathlineto{\pgfqpoint{3.989663in}{2.399857in}}%
\pgfpathlineto{\pgfqpoint{3.978487in}{2.396423in}}%
\pgfpathlineto{\pgfqpoint{3.967304in}{2.392990in}}%
\pgfpathlineto{\pgfqpoint{3.973730in}{2.378544in}}%
\pgfpathlineto{\pgfqpoint{3.980158in}{2.364085in}}%
\pgfpathlineto{\pgfqpoint{3.986588in}{2.349627in}}%
\pgfpathlineto{\pgfqpoint{3.993021in}{2.335183in}}%
\pgfpathclose%
\pgfusepath{stroke,fill}%
\end{pgfscope}%
\begin{pgfscope}%
\pgfpathrectangle{\pgfqpoint{0.887500in}{0.275000in}}{\pgfqpoint{4.225000in}{4.225000in}}%
\pgfusepath{clip}%
\pgfsetbuttcap%
\pgfsetroundjoin%
\definecolor{currentfill}{rgb}{0.231674,0.318106,0.544834}%
\pgfsetfillcolor{currentfill}%
\pgfsetfillopacity{0.700000}%
\pgfsetlinewidth{0.501875pt}%
\definecolor{currentstroke}{rgb}{1.000000,1.000000,1.000000}%
\pgfsetstrokecolor{currentstroke}%
\pgfsetstrokeopacity{0.500000}%
\pgfsetdash{}{0pt}%
\pgfpathmoveto{\pgfqpoint{4.592608in}{1.848775in}}%
\pgfpathlineto{\pgfqpoint{4.603640in}{1.852181in}}%
\pgfpathlineto{\pgfqpoint{4.614665in}{1.855570in}}%
\pgfpathlineto{\pgfqpoint{4.625685in}{1.858945in}}%
\pgfpathlineto{\pgfqpoint{4.636698in}{1.862306in}}%
\pgfpathlineto{\pgfqpoint{4.647706in}{1.865658in}}%
\pgfpathlineto{\pgfqpoint{4.641167in}{1.880362in}}%
\pgfpathlineto{\pgfqpoint{4.634630in}{1.895048in}}%
\pgfpathlineto{\pgfqpoint{4.628095in}{1.909716in}}%
\pgfpathlineto{\pgfqpoint{4.621562in}{1.924365in}}%
\pgfpathlineto{\pgfqpoint{4.615030in}{1.938997in}}%
\pgfpathlineto{\pgfqpoint{4.604025in}{1.935682in}}%
\pgfpathlineto{\pgfqpoint{4.593014in}{1.932366in}}%
\pgfpathlineto{\pgfqpoint{4.581997in}{1.929047in}}%
\pgfpathlineto{\pgfqpoint{4.570975in}{1.925722in}}%
\pgfpathlineto{\pgfqpoint{4.559947in}{1.922392in}}%
\pgfpathlineto{\pgfqpoint{4.566474in}{1.907675in}}%
\pgfpathlineto{\pgfqpoint{4.573004in}{1.892949in}}%
\pgfpathlineto{\pgfqpoint{4.579536in}{1.878219in}}%
\pgfpathlineto{\pgfqpoint{4.586070in}{1.863492in}}%
\pgfpathclose%
\pgfusepath{stroke,fill}%
\end{pgfscope}%
\begin{pgfscope}%
\pgfpathrectangle{\pgfqpoint{0.887500in}{0.275000in}}{\pgfqpoint{4.225000in}{4.225000in}}%
\pgfusepath{clip}%
\pgfsetbuttcap%
\pgfsetroundjoin%
\definecolor{currentfill}{rgb}{0.150148,0.676631,0.506589}%
\pgfsetfillcolor{currentfill}%
\pgfsetfillopacity{0.700000}%
\pgfsetlinewidth{0.501875pt}%
\definecolor{currentstroke}{rgb}{1.000000,1.000000,1.000000}%
\pgfsetstrokecolor{currentstroke}%
\pgfsetstrokeopacity{0.500000}%
\pgfsetdash{}{0pt}%
\pgfpathmoveto{\pgfqpoint{3.614517in}{2.602657in}}%
\pgfpathlineto{\pgfqpoint{3.625803in}{2.606472in}}%
\pgfpathlineto{\pgfqpoint{3.637084in}{2.610238in}}%
\pgfpathlineto{\pgfqpoint{3.648359in}{2.614016in}}%
\pgfpathlineto{\pgfqpoint{3.659630in}{2.617817in}}%
\pgfpathlineto{\pgfqpoint{3.670895in}{2.621625in}}%
\pgfpathlineto{\pgfqpoint{3.664522in}{2.635678in}}%
\pgfpathlineto{\pgfqpoint{3.658150in}{2.649687in}}%
\pgfpathlineto{\pgfqpoint{3.651781in}{2.663644in}}%
\pgfpathlineto{\pgfqpoint{3.645413in}{2.677539in}}%
\pgfpathlineto{\pgfqpoint{3.639047in}{2.691372in}}%
\pgfpathlineto{\pgfqpoint{3.627786in}{2.687629in}}%
\pgfpathlineto{\pgfqpoint{3.616519in}{2.683889in}}%
\pgfpathlineto{\pgfqpoint{3.605248in}{2.680165in}}%
\pgfpathlineto{\pgfqpoint{3.593971in}{2.676449in}}%
\pgfpathlineto{\pgfqpoint{3.582688in}{2.672684in}}%
\pgfpathlineto{\pgfqpoint{3.589051in}{2.658808in}}%
\pgfpathlineto{\pgfqpoint{3.595415in}{2.644867in}}%
\pgfpathlineto{\pgfqpoint{3.601780in}{2.630859in}}%
\pgfpathlineto{\pgfqpoint{3.608148in}{2.616788in}}%
\pgfpathclose%
\pgfusepath{stroke,fill}%
\end{pgfscope}%
\begin{pgfscope}%
\pgfpathrectangle{\pgfqpoint{0.887500in}{0.275000in}}{\pgfqpoint{4.225000in}{4.225000in}}%
\pgfusepath{clip}%
\pgfsetbuttcap%
\pgfsetroundjoin%
\definecolor{currentfill}{rgb}{0.168126,0.459988,0.558082}%
\pgfsetfillcolor{currentfill}%
\pgfsetfillopacity{0.700000}%
\pgfsetlinewidth{0.501875pt}%
\definecolor{currentstroke}{rgb}{1.000000,1.000000,1.000000}%
\pgfsetstrokecolor{currentstroke}%
\pgfsetstrokeopacity{0.500000}%
\pgfsetdash{}{0pt}%
\pgfpathmoveto{\pgfqpoint{2.455570in}{2.149857in}}%
\pgfpathlineto{\pgfqpoint{2.467142in}{2.153316in}}%
\pgfpathlineto{\pgfqpoint{2.478708in}{2.156764in}}%
\pgfpathlineto{\pgfqpoint{2.490269in}{2.160207in}}%
\pgfpathlineto{\pgfqpoint{2.501824in}{2.163650in}}%
\pgfpathlineto{\pgfqpoint{2.513373in}{2.167096in}}%
\pgfpathlineto{\pgfqpoint{2.507290in}{2.176171in}}%
\pgfpathlineto{\pgfqpoint{2.501212in}{2.185215in}}%
\pgfpathlineto{\pgfqpoint{2.495138in}{2.194229in}}%
\pgfpathlineto{\pgfqpoint{2.489068in}{2.203213in}}%
\pgfpathlineto{\pgfqpoint{2.483002in}{2.212168in}}%
\pgfpathlineto{\pgfqpoint{2.471464in}{2.208761in}}%
\pgfpathlineto{\pgfqpoint{2.459919in}{2.205360in}}%
\pgfpathlineto{\pgfqpoint{2.448369in}{2.201960in}}%
\pgfpathlineto{\pgfqpoint{2.436814in}{2.198556in}}%
\pgfpathlineto{\pgfqpoint{2.425252in}{2.195143in}}%
\pgfpathlineto{\pgfqpoint{2.431307in}{2.186146in}}%
\pgfpathlineto{\pgfqpoint{2.437366in}{2.177120in}}%
\pgfpathlineto{\pgfqpoint{2.443430in}{2.168064in}}%
\pgfpathlineto{\pgfqpoint{2.449497in}{2.158976in}}%
\pgfpathclose%
\pgfusepath{stroke,fill}%
\end{pgfscope}%
\begin{pgfscope}%
\pgfpathrectangle{\pgfqpoint{0.887500in}{0.275000in}}{\pgfqpoint{4.225000in}{4.225000in}}%
\pgfusepath{clip}%
\pgfsetbuttcap%
\pgfsetroundjoin%
\definecolor{currentfill}{rgb}{0.154815,0.493313,0.557840}%
\pgfsetfillcolor{currentfill}%
\pgfsetfillopacity{0.700000}%
\pgfsetlinewidth{0.501875pt}%
\definecolor{currentstroke}{rgb}{1.000000,1.000000,1.000000}%
\pgfsetstrokecolor{currentstroke}%
\pgfsetstrokeopacity{0.500000}%
\pgfsetdash{}{0pt}%
\pgfpathmoveto{\pgfqpoint{2.132985in}{2.214512in}}%
\pgfpathlineto{\pgfqpoint{2.144637in}{2.217979in}}%
\pgfpathlineto{\pgfqpoint{2.156284in}{2.221447in}}%
\pgfpathlineto{\pgfqpoint{2.167924in}{2.224920in}}%
\pgfpathlineto{\pgfqpoint{2.179559in}{2.228400in}}%
\pgfpathlineto{\pgfqpoint{2.191187in}{2.231890in}}%
\pgfpathlineto{\pgfqpoint{2.185212in}{2.240727in}}%
\pgfpathlineto{\pgfqpoint{2.179242in}{2.249537in}}%
\pgfpathlineto{\pgfqpoint{2.173276in}{2.258323in}}%
\pgfpathlineto{\pgfqpoint{2.167314in}{2.267083in}}%
\pgfpathlineto{\pgfqpoint{2.161356in}{2.275820in}}%
\pgfpathlineto{\pgfqpoint{2.149739in}{2.272374in}}%
\pgfpathlineto{\pgfqpoint{2.138115in}{2.268939in}}%
\pgfpathlineto{\pgfqpoint{2.126486in}{2.265512in}}%
\pgfpathlineto{\pgfqpoint{2.114850in}{2.262091in}}%
\pgfpathlineto{\pgfqpoint{2.103209in}{2.258671in}}%
\pgfpathlineto{\pgfqpoint{2.109155in}{2.249890in}}%
\pgfpathlineto{\pgfqpoint{2.115106in}{2.241085in}}%
\pgfpathlineto{\pgfqpoint{2.121061in}{2.232253in}}%
\pgfpathlineto{\pgfqpoint{2.127021in}{2.223396in}}%
\pgfpathclose%
\pgfusepath{stroke,fill}%
\end{pgfscope}%
\begin{pgfscope}%
\pgfpathrectangle{\pgfqpoint{0.887500in}{0.275000in}}{\pgfqpoint{4.225000in}{4.225000in}}%
\pgfusepath{clip}%
\pgfsetbuttcap%
\pgfsetroundjoin%
\definecolor{currentfill}{rgb}{0.143343,0.522773,0.556295}%
\pgfsetfillcolor{currentfill}%
\pgfsetfillopacity{0.700000}%
\pgfsetlinewidth{0.501875pt}%
\definecolor{currentstroke}{rgb}{1.000000,1.000000,1.000000}%
\pgfsetstrokecolor{currentstroke}%
\pgfsetstrokeopacity{0.500000}%
\pgfsetdash{}{0pt}%
\pgfpathmoveto{\pgfqpoint{1.810423in}{2.276968in}}%
\pgfpathlineto{\pgfqpoint{1.822155in}{2.280426in}}%
\pgfpathlineto{\pgfqpoint{1.833881in}{2.283882in}}%
\pgfpathlineto{\pgfqpoint{1.845602in}{2.287338in}}%
\pgfpathlineto{\pgfqpoint{1.857316in}{2.290796in}}%
\pgfpathlineto{\pgfqpoint{1.869025in}{2.294255in}}%
\pgfpathlineto{\pgfqpoint{1.863160in}{2.302896in}}%
\pgfpathlineto{\pgfqpoint{1.857300in}{2.311514in}}%
\pgfpathlineto{\pgfqpoint{1.851444in}{2.320108in}}%
\pgfpathlineto{\pgfqpoint{1.845593in}{2.328679in}}%
\pgfpathlineto{\pgfqpoint{1.839746in}{2.337228in}}%
\pgfpathlineto{\pgfqpoint{1.828049in}{2.333811in}}%
\pgfpathlineto{\pgfqpoint{1.816346in}{2.330396in}}%
\pgfpathlineto{\pgfqpoint{1.804637in}{2.326981in}}%
\pgfpathlineto{\pgfqpoint{1.792922in}{2.323567in}}%
\pgfpathlineto{\pgfqpoint{1.781201in}{2.320151in}}%
\pgfpathlineto{\pgfqpoint{1.787037in}{2.311563in}}%
\pgfpathlineto{\pgfqpoint{1.792877in}{2.302951in}}%
\pgfpathlineto{\pgfqpoint{1.798721in}{2.294314in}}%
\pgfpathlineto{\pgfqpoint{1.804570in}{2.285654in}}%
\pgfpathclose%
\pgfusepath{stroke,fill}%
\end{pgfscope}%
\begin{pgfscope}%
\pgfpathrectangle{\pgfqpoint{0.887500in}{0.275000in}}{\pgfqpoint{4.225000in}{4.225000in}}%
\pgfusepath{clip}%
\pgfsetbuttcap%
\pgfsetroundjoin%
\definecolor{currentfill}{rgb}{0.180653,0.701402,0.488189}%
\pgfsetfillcolor{currentfill}%
\pgfsetfillopacity{0.700000}%
\pgfsetlinewidth{0.501875pt}%
\definecolor{currentstroke}{rgb}{1.000000,1.000000,1.000000}%
\pgfsetstrokecolor{currentstroke}%
\pgfsetstrokeopacity{0.500000}%
\pgfsetdash{}{0pt}%
\pgfpathmoveto{\pgfqpoint{3.526160in}{2.650935in}}%
\pgfpathlineto{\pgfqpoint{3.537483in}{2.655883in}}%
\pgfpathlineto{\pgfqpoint{3.548797in}{2.660469in}}%
\pgfpathlineto{\pgfqpoint{3.560102in}{2.664757in}}%
\pgfpathlineto{\pgfqpoint{3.571399in}{2.668808in}}%
\pgfpathlineto{\pgfqpoint{3.582688in}{2.672684in}}%
\pgfpathlineto{\pgfqpoint{3.576328in}{2.686501in}}%
\pgfpathlineto{\pgfqpoint{3.569969in}{2.700263in}}%
\pgfpathlineto{\pgfqpoint{3.563612in}{2.713975in}}%
\pgfpathlineto{\pgfqpoint{3.557257in}{2.727643in}}%
\pgfpathlineto{\pgfqpoint{3.550905in}{2.741271in}}%
\pgfpathlineto{\pgfqpoint{3.539620in}{2.737478in}}%
\pgfpathlineto{\pgfqpoint{3.528329in}{2.733588in}}%
\pgfpathlineto{\pgfqpoint{3.517031in}{2.729569in}}%
\pgfpathlineto{\pgfqpoint{3.505726in}{2.725388in}}%
\pgfpathlineto{\pgfqpoint{3.494413in}{2.721012in}}%
\pgfpathlineto{\pgfqpoint{3.500759in}{2.707069in}}%
\pgfpathlineto{\pgfqpoint{3.507106in}{2.693080in}}%
\pgfpathlineto{\pgfqpoint{3.513455in}{2.679055in}}%
\pgfpathlineto{\pgfqpoint{3.519806in}{2.665003in}}%
\pgfpathclose%
\pgfusepath{stroke,fill}%
\end{pgfscope}%
\begin{pgfscope}%
\pgfpathrectangle{\pgfqpoint{0.887500in}{0.275000in}}{\pgfqpoint{4.225000in}{4.225000in}}%
\pgfusepath{clip}%
\pgfsetbuttcap%
\pgfsetroundjoin%
\definecolor{currentfill}{rgb}{0.166617,0.463708,0.558119}%
\pgfsetfillcolor{currentfill}%
\pgfsetfillopacity{0.700000}%
\pgfsetlinewidth{0.501875pt}%
\definecolor{currentstroke}{rgb}{1.000000,1.000000,1.000000}%
\pgfsetstrokecolor{currentstroke}%
\pgfsetstrokeopacity{0.500000}%
\pgfsetdash{}{0pt}%
\pgfpathmoveto{\pgfqpoint{4.207924in}{2.137894in}}%
\pgfpathlineto{\pgfqpoint{4.219059in}{2.141364in}}%
\pgfpathlineto{\pgfqpoint{4.230188in}{2.144792in}}%
\pgfpathlineto{\pgfqpoint{4.241309in}{2.148161in}}%
\pgfpathlineto{\pgfqpoint{4.252423in}{2.151457in}}%
\pgfpathlineto{\pgfqpoint{4.263528in}{2.154666in}}%
\pgfpathlineto{\pgfqpoint{4.257046in}{2.168981in}}%
\pgfpathlineto{\pgfqpoint{4.250568in}{2.183367in}}%
\pgfpathlineto{\pgfqpoint{4.244095in}{2.197822in}}%
\pgfpathlineto{\pgfqpoint{4.237626in}{2.212334in}}%
\pgfpathlineto{\pgfqpoint{4.231162in}{2.226894in}}%
\pgfpathlineto{\pgfqpoint{4.220062in}{2.223840in}}%
\pgfpathlineto{\pgfqpoint{4.208956in}{2.220728in}}%
\pgfpathlineto{\pgfqpoint{4.197842in}{2.217567in}}%
\pgfpathlineto{\pgfqpoint{4.186721in}{2.214366in}}%
\pgfpathlineto{\pgfqpoint{4.175594in}{2.211134in}}%
\pgfpathlineto{\pgfqpoint{4.182058in}{2.196572in}}%
\pgfpathlineto{\pgfqpoint{4.188523in}{2.181970in}}%
\pgfpathlineto{\pgfqpoint{4.194989in}{2.167325in}}%
\pgfpathlineto{\pgfqpoint{4.201456in}{2.152630in}}%
\pgfpathclose%
\pgfusepath{stroke,fill}%
\end{pgfscope}%
\begin{pgfscope}%
\pgfpathrectangle{\pgfqpoint{0.887500in}{0.275000in}}{\pgfqpoint{4.225000in}{4.225000in}}%
\pgfusepath{clip}%
\pgfsetbuttcap%
\pgfsetroundjoin%
\definecolor{currentfill}{rgb}{0.246070,0.738910,0.452024}%
\pgfsetfillcolor{currentfill}%
\pgfsetfillopacity{0.700000}%
\pgfsetlinewidth{0.501875pt}%
\definecolor{currentstroke}{rgb}{1.000000,1.000000,1.000000}%
\pgfsetstrokecolor{currentstroke}%
\pgfsetstrokeopacity{0.500000}%
\pgfsetdash{}{0pt}%
\pgfpathmoveto{\pgfqpoint{3.349258in}{2.732192in}}%
\pgfpathlineto{\pgfqpoint{3.360640in}{2.740205in}}%
\pgfpathlineto{\pgfqpoint{3.372015in}{2.747702in}}%
\pgfpathlineto{\pgfqpoint{3.383382in}{2.754641in}}%
\pgfpathlineto{\pgfqpoint{3.394740in}{2.760985in}}%
\pgfpathlineto{\pgfqpoint{3.406090in}{2.766751in}}%
\pgfpathlineto{\pgfqpoint{3.399761in}{2.780386in}}%
\pgfpathlineto{\pgfqpoint{3.393434in}{2.793807in}}%
\pgfpathlineto{\pgfqpoint{3.387107in}{2.807060in}}%
\pgfpathlineto{\pgfqpoint{3.380782in}{2.820189in}}%
\pgfpathlineto{\pgfqpoint{3.374460in}{2.833241in}}%
\pgfpathlineto{\pgfqpoint{3.363115in}{2.827373in}}%
\pgfpathlineto{\pgfqpoint{3.351762in}{2.820904in}}%
\pgfpathlineto{\pgfqpoint{3.340401in}{2.813793in}}%
\pgfpathlineto{\pgfqpoint{3.329032in}{2.806040in}}%
\pgfpathlineto{\pgfqpoint{3.317655in}{2.797644in}}%
\pgfpathlineto{\pgfqpoint{3.323973in}{2.784968in}}%
\pgfpathlineto{\pgfqpoint{3.330293in}{2.772171in}}%
\pgfpathlineto{\pgfqpoint{3.336614in}{2.759167in}}%
\pgfpathlineto{\pgfqpoint{3.342936in}{2.745869in}}%
\pgfpathclose%
\pgfusepath{stroke,fill}%
\end{pgfscope}%
\begin{pgfscope}%
\pgfpathrectangle{\pgfqpoint{0.887500in}{0.275000in}}{\pgfqpoint{4.225000in}{4.225000in}}%
\pgfusepath{clip}%
\pgfsetbuttcap%
\pgfsetroundjoin%
\definecolor{currentfill}{rgb}{0.125394,0.574318,0.549086}%
\pgfsetfillcolor{currentfill}%
\pgfsetfillopacity{0.700000}%
\pgfsetlinewidth{0.501875pt}%
\definecolor{currentstroke}{rgb}{1.000000,1.000000,1.000000}%
\pgfsetstrokecolor{currentstroke}%
\pgfsetstrokeopacity{0.500000}%
\pgfsetdash{}{0pt}%
\pgfpathmoveto{\pgfqpoint{3.911307in}{2.375642in}}%
\pgfpathlineto{\pgfqpoint{3.922519in}{2.379157in}}%
\pgfpathlineto{\pgfqpoint{3.933724in}{2.382643in}}%
\pgfpathlineto{\pgfqpoint{3.944923in}{2.386106in}}%
\pgfpathlineto{\pgfqpoint{3.956117in}{2.389553in}}%
\pgfpathlineto{\pgfqpoint{3.967304in}{2.392990in}}%
\pgfpathlineto{\pgfqpoint{3.960881in}{2.407410in}}%
\pgfpathlineto{\pgfqpoint{3.954458in}{2.421789in}}%
\pgfpathlineto{\pgfqpoint{3.948037in}{2.436116in}}%
\pgfpathlineto{\pgfqpoint{3.941618in}{2.450388in}}%
\pgfpathlineto{\pgfqpoint{3.935199in}{2.464608in}}%
\pgfpathlineto{\pgfqpoint{3.924015in}{2.461222in}}%
\pgfpathlineto{\pgfqpoint{3.912825in}{2.457835in}}%
\pgfpathlineto{\pgfqpoint{3.901629in}{2.454445in}}%
\pgfpathlineto{\pgfqpoint{3.890428in}{2.451050in}}%
\pgfpathlineto{\pgfqpoint{3.879221in}{2.447646in}}%
\pgfpathlineto{\pgfqpoint{3.885636in}{2.433401in}}%
\pgfpathlineto{\pgfqpoint{3.892053in}{2.419081in}}%
\pgfpathlineto{\pgfqpoint{3.898470in}{2.404677in}}%
\pgfpathlineto{\pgfqpoint{3.904888in}{2.390191in}}%
\pgfpathclose%
\pgfusepath{stroke,fill}%
\end{pgfscope}%
\begin{pgfscope}%
\pgfpathrectangle{\pgfqpoint{0.887500in}{0.275000in}}{\pgfqpoint{4.225000in}{4.225000in}}%
\pgfusepath{clip}%
\pgfsetbuttcap%
\pgfsetroundjoin%
\definecolor{currentfill}{rgb}{0.187231,0.414746,0.556547}%
\pgfsetfillcolor{currentfill}%
\pgfsetfillopacity{0.700000}%
\pgfsetlinewidth{0.501875pt}%
\definecolor{currentstroke}{rgb}{1.000000,1.000000,1.000000}%
\pgfsetstrokecolor{currentstroke}%
\pgfsetstrokeopacity{0.500000}%
\pgfsetdash{}{0pt}%
\pgfpathmoveto{\pgfqpoint{2.866532in}{2.049056in}}%
\pgfpathlineto{\pgfqpoint{2.878000in}{2.053181in}}%
\pgfpathlineto{\pgfqpoint{2.889462in}{2.057547in}}%
\pgfpathlineto{\pgfqpoint{2.900918in}{2.061912in}}%
\pgfpathlineto{\pgfqpoint{2.912371in}{2.066032in}}%
\pgfpathlineto{\pgfqpoint{2.923820in}{2.069661in}}%
\pgfpathlineto{\pgfqpoint{2.917603in}{2.079294in}}%
\pgfpathlineto{\pgfqpoint{2.911389in}{2.088889in}}%
\pgfpathlineto{\pgfqpoint{2.905180in}{2.098450in}}%
\pgfpathlineto{\pgfqpoint{2.898974in}{2.107979in}}%
\pgfpathlineto{\pgfqpoint{2.892773in}{2.117479in}}%
\pgfpathlineto{\pgfqpoint{2.881334in}{2.113888in}}%
\pgfpathlineto{\pgfqpoint{2.869892in}{2.109777in}}%
\pgfpathlineto{\pgfqpoint{2.858446in}{2.105405in}}%
\pgfpathlineto{\pgfqpoint{2.846995in}{2.101033in}}%
\pgfpathlineto{\pgfqpoint{2.835538in}{2.096915in}}%
\pgfpathlineto{\pgfqpoint{2.841728in}{2.087425in}}%
\pgfpathlineto{\pgfqpoint{2.847923in}{2.077895in}}%
\pgfpathlineto{\pgfqpoint{2.854122in}{2.068323in}}%
\pgfpathlineto{\pgfqpoint{2.860325in}{2.058710in}}%
\pgfpathclose%
\pgfusepath{stroke,fill}%
\end{pgfscope}%
\begin{pgfscope}%
\pgfpathrectangle{\pgfqpoint{0.887500in}{0.275000in}}{\pgfqpoint{4.225000in}{4.225000in}}%
\pgfusepath{clip}%
\pgfsetbuttcap%
\pgfsetroundjoin%
\definecolor{currentfill}{rgb}{0.214000,0.722114,0.469588}%
\pgfsetfillcolor{currentfill}%
\pgfsetfillopacity{0.700000}%
\pgfsetlinewidth{0.501875pt}%
\definecolor{currentstroke}{rgb}{1.000000,1.000000,1.000000}%
\pgfsetstrokecolor{currentstroke}%
\pgfsetstrokeopacity{0.500000}%
\pgfsetdash{}{0pt}%
\pgfpathmoveto{\pgfqpoint{3.437735in}{2.694665in}}%
\pgfpathlineto{\pgfqpoint{3.449087in}{2.700691in}}%
\pgfpathlineto{\pgfqpoint{3.460431in}{2.706290in}}%
\pgfpathlineto{\pgfqpoint{3.471766in}{2.711511in}}%
\pgfpathlineto{\pgfqpoint{3.483094in}{2.716403in}}%
\pgfpathlineto{\pgfqpoint{3.494413in}{2.721012in}}%
\pgfpathlineto{\pgfqpoint{3.488070in}{2.734899in}}%
\pgfpathlineto{\pgfqpoint{3.481729in}{2.748720in}}%
\pgfpathlineto{\pgfqpoint{3.475389in}{2.762464in}}%
\pgfpathlineto{\pgfqpoint{3.469051in}{2.776123in}}%
\pgfpathlineto{\pgfqpoint{3.462715in}{2.789692in}}%
\pgfpathlineto{\pgfqpoint{3.451405in}{2.785620in}}%
\pgfpathlineto{\pgfqpoint{3.440088in}{2.781371in}}%
\pgfpathlineto{\pgfqpoint{3.428763in}{2.776863in}}%
\pgfpathlineto{\pgfqpoint{3.417431in}{2.772017in}}%
\pgfpathlineto{\pgfqpoint{3.406090in}{2.766751in}}%
\pgfpathlineto{\pgfqpoint{3.412419in}{2.752855in}}%
\pgfpathlineto{\pgfqpoint{3.418748in}{2.738661in}}%
\pgfpathlineto{\pgfqpoint{3.425076in}{2.724194in}}%
\pgfpathlineto{\pgfqpoint{3.431405in}{2.709511in}}%
\pgfpathclose%
\pgfusepath{stroke,fill}%
\end{pgfscope}%
\begin{pgfscope}%
\pgfpathrectangle{\pgfqpoint{0.887500in}{0.275000in}}{\pgfqpoint{4.225000in}{4.225000in}}%
\pgfusepath{clip}%
\pgfsetbuttcap%
\pgfsetroundjoin%
\definecolor{currentfill}{rgb}{0.216210,0.351535,0.550627}%
\pgfsetfillcolor{currentfill}%
\pgfsetfillopacity{0.700000}%
\pgfsetlinewidth{0.501875pt}%
\definecolor{currentstroke}{rgb}{1.000000,1.000000,1.000000}%
\pgfsetstrokecolor{currentstroke}%
\pgfsetstrokeopacity{0.500000}%
\pgfsetdash{}{0pt}%
\pgfpathmoveto{\pgfqpoint{4.504724in}{1.905716in}}%
\pgfpathlineto{\pgfqpoint{4.515779in}{1.909043in}}%
\pgfpathlineto{\pgfqpoint{4.526829in}{1.912378in}}%
\pgfpathlineto{\pgfqpoint{4.537874in}{1.915716in}}%
\pgfpathlineto{\pgfqpoint{4.548913in}{1.919055in}}%
\pgfpathlineto{\pgfqpoint{4.559947in}{1.922392in}}%
\pgfpathlineto{\pgfqpoint{4.553422in}{1.937093in}}%
\pgfpathlineto{\pgfqpoint{4.546898in}{1.951771in}}%
\pgfpathlineto{\pgfqpoint{4.540376in}{1.966422in}}%
\pgfpathlineto{\pgfqpoint{4.533855in}{1.981037in}}%
\pgfpathlineto{\pgfqpoint{4.527335in}{1.995612in}}%
\pgfpathlineto{\pgfqpoint{4.516306in}{1.992373in}}%
\pgfpathlineto{\pgfqpoint{4.505271in}{1.989130in}}%
\pgfpathlineto{\pgfqpoint{4.494230in}{1.985875in}}%
\pgfpathlineto{\pgfqpoint{4.483183in}{1.982600in}}%
\pgfpathlineto{\pgfqpoint{4.472129in}{1.979297in}}%
\pgfpathlineto{\pgfqpoint{4.478647in}{1.964697in}}%
\pgfpathlineto{\pgfqpoint{4.485165in}{1.950024in}}%
\pgfpathlineto{\pgfqpoint{4.491684in}{1.935294in}}%
\pgfpathlineto{\pgfqpoint{4.498203in}{1.920520in}}%
\pgfpathclose%
\pgfusepath{stroke,fill}%
\end{pgfscope}%
\begin{pgfscope}%
\pgfpathrectangle{\pgfqpoint{0.887500in}{0.275000in}}{\pgfqpoint{4.225000in}{4.225000in}}%
\pgfusepath{clip}%
\pgfsetbuttcap%
\pgfsetroundjoin%
\definecolor{currentfill}{rgb}{0.171176,0.452530,0.557965}%
\pgfsetfillcolor{currentfill}%
\pgfsetfillopacity{0.700000}%
\pgfsetlinewidth{0.501875pt}%
\definecolor{currentstroke}{rgb}{1.000000,1.000000,1.000000}%
\pgfsetstrokecolor{currentstroke}%
\pgfsetstrokeopacity{0.500000}%
\pgfsetdash{}{0pt}%
\pgfpathmoveto{\pgfqpoint{2.543850in}{2.121231in}}%
\pgfpathlineto{\pgfqpoint{2.555404in}{2.124728in}}%
\pgfpathlineto{\pgfqpoint{2.566951in}{2.128236in}}%
\pgfpathlineto{\pgfqpoint{2.578494in}{2.131756in}}%
\pgfpathlineto{\pgfqpoint{2.590030in}{2.135293in}}%
\pgfpathlineto{\pgfqpoint{2.601560in}{2.138850in}}%
\pgfpathlineto{\pgfqpoint{2.595446in}{2.148051in}}%
\pgfpathlineto{\pgfqpoint{2.589335in}{2.157218in}}%
\pgfpathlineto{\pgfqpoint{2.583229in}{2.166353in}}%
\pgfpathlineto{\pgfqpoint{2.577127in}{2.175456in}}%
\pgfpathlineto{\pgfqpoint{2.571030in}{2.184527in}}%
\pgfpathlineto{\pgfqpoint{2.559510in}{2.181004in}}%
\pgfpathlineto{\pgfqpoint{2.547985in}{2.177503in}}%
\pgfpathlineto{\pgfqpoint{2.536453in}{2.174020in}}%
\pgfpathlineto{\pgfqpoint{2.524916in}{2.170552in}}%
\pgfpathlineto{\pgfqpoint{2.513373in}{2.167096in}}%
\pgfpathlineto{\pgfqpoint{2.519460in}{2.157990in}}%
\pgfpathlineto{\pgfqpoint{2.525551in}{2.148850in}}%
\pgfpathlineto{\pgfqpoint{2.531646in}{2.139678in}}%
\pgfpathlineto{\pgfqpoint{2.537746in}{2.130472in}}%
\pgfpathclose%
\pgfusepath{stroke,fill}%
\end{pgfscope}%
\begin{pgfscope}%
\pgfpathrectangle{\pgfqpoint{0.887500in}{0.275000in}}{\pgfqpoint{4.225000in}{4.225000in}}%
\pgfusepath{clip}%
\pgfsetbuttcap%
\pgfsetroundjoin%
\definecolor{currentfill}{rgb}{0.194100,0.399323,0.555565}%
\pgfsetfillcolor{currentfill}%
\pgfsetfillopacity{0.700000}%
\pgfsetlinewidth{0.501875pt}%
\definecolor{currentstroke}{rgb}{1.000000,1.000000,1.000000}%
\pgfsetstrokecolor{currentstroke}%
\pgfsetstrokeopacity{0.500000}%
\pgfsetdash{}{0pt}%
\pgfpathmoveto{\pgfqpoint{2.954965in}{2.020828in}}%
\pgfpathlineto{\pgfqpoint{2.966420in}{2.023787in}}%
\pgfpathlineto{\pgfqpoint{2.977870in}{2.025807in}}%
\pgfpathlineto{\pgfqpoint{2.989316in}{2.026704in}}%
\pgfpathlineto{\pgfqpoint{3.000756in}{2.026918in}}%
\pgfpathlineto{\pgfqpoint{3.012188in}{2.027299in}}%
\pgfpathlineto{\pgfqpoint{3.005943in}{2.036984in}}%
\pgfpathlineto{\pgfqpoint{2.999703in}{2.046648in}}%
\pgfpathlineto{\pgfqpoint{2.993466in}{2.056285in}}%
\pgfpathlineto{\pgfqpoint{2.987233in}{2.065887in}}%
\pgfpathlineto{\pgfqpoint{2.981004in}{2.075445in}}%
\pgfpathlineto{\pgfqpoint{2.969579in}{2.075239in}}%
\pgfpathlineto{\pgfqpoint{2.958146in}{2.075210in}}%
\pgfpathlineto{\pgfqpoint{2.946708in}{2.074468in}}%
\pgfpathlineto{\pgfqpoint{2.935265in}{2.072555in}}%
\pgfpathlineto{\pgfqpoint{2.923820in}{2.069661in}}%
\pgfpathlineto{\pgfqpoint{2.930041in}{2.059988in}}%
\pgfpathlineto{\pgfqpoint{2.936266in}{2.050271in}}%
\pgfpathlineto{\pgfqpoint{2.942495in}{2.040508in}}%
\pgfpathlineto{\pgfqpoint{2.948728in}{2.030694in}}%
\pgfpathclose%
\pgfusepath{stroke,fill}%
\end{pgfscope}%
\begin{pgfscope}%
\pgfpathrectangle{\pgfqpoint{0.887500in}{0.275000in}}{\pgfqpoint{4.225000in}{4.225000in}}%
\pgfusepath{clip}%
\pgfsetbuttcap%
\pgfsetroundjoin%
\definecolor{currentfill}{rgb}{0.134692,0.658636,0.517649}%
\pgfsetfillcolor{currentfill}%
\pgfsetfillopacity{0.700000}%
\pgfsetlinewidth{0.501875pt}%
\definecolor{currentstroke}{rgb}{1.000000,1.000000,1.000000}%
\pgfsetstrokecolor{currentstroke}%
\pgfsetstrokeopacity{0.500000}%
\pgfsetdash{}{0pt}%
\pgfpathmoveto{\pgfqpoint{3.177883in}{2.515936in}}%
\pgfpathlineto{\pgfqpoint{3.189330in}{2.538519in}}%
\pgfpathlineto{\pgfqpoint{3.200781in}{2.560471in}}%
\pgfpathlineto{\pgfqpoint{3.212233in}{2.581448in}}%
\pgfpathlineto{\pgfqpoint{3.223683in}{2.601105in}}%
\pgfpathlineto{\pgfqpoint{3.235130in}{2.619098in}}%
\pgfpathlineto{\pgfqpoint{3.228823in}{2.631602in}}%
\pgfpathlineto{\pgfqpoint{3.222514in}{2.643054in}}%
\pgfpathlineto{\pgfqpoint{3.216206in}{2.653729in}}%
\pgfpathlineto{\pgfqpoint{3.209899in}{2.663905in}}%
\pgfpathlineto{\pgfqpoint{3.203595in}{2.673857in}}%
\pgfpathlineto{\pgfqpoint{3.192168in}{2.656009in}}%
\pgfpathlineto{\pgfqpoint{3.180740in}{2.637022in}}%
\pgfpathlineto{\pgfqpoint{3.169313in}{2.617090in}}%
\pgfpathlineto{\pgfqpoint{3.157887in}{2.596406in}}%
\pgfpathlineto{\pgfqpoint{3.146465in}{2.575163in}}%
\pgfpathlineto{\pgfqpoint{3.152744in}{2.563970in}}%
\pgfpathlineto{\pgfqpoint{3.159026in}{2.552551in}}%
\pgfpathlineto{\pgfqpoint{3.165310in}{2.540806in}}%
\pgfpathlineto{\pgfqpoint{3.171596in}{2.528634in}}%
\pgfpathclose%
\pgfusepath{stroke,fill}%
\end{pgfscope}%
\begin{pgfscope}%
\pgfpathrectangle{\pgfqpoint{0.887500in}{0.275000in}}{\pgfqpoint{4.225000in}{4.225000in}}%
\pgfusepath{clip}%
\pgfsetbuttcap%
\pgfsetroundjoin%
\definecolor{currentfill}{rgb}{0.120092,0.600104,0.542530}%
\pgfsetfillcolor{currentfill}%
\pgfsetfillopacity{0.700000}%
\pgfsetlinewidth{0.501875pt}%
\definecolor{currentstroke}{rgb}{1.000000,1.000000,1.000000}%
\pgfsetstrokecolor{currentstroke}%
\pgfsetstrokeopacity{0.500000}%
\pgfsetdash{}{0pt}%
\pgfpathmoveto{\pgfqpoint{3.823090in}{2.429955in}}%
\pgfpathlineto{\pgfqpoint{3.834330in}{2.433637in}}%
\pgfpathlineto{\pgfqpoint{3.845563in}{2.437230in}}%
\pgfpathlineto{\pgfqpoint{3.856789in}{2.440752in}}%
\pgfpathlineto{\pgfqpoint{3.868008in}{2.444218in}}%
\pgfpathlineto{\pgfqpoint{3.879221in}{2.447646in}}%
\pgfpathlineto{\pgfqpoint{3.872806in}{2.461823in}}%
\pgfpathlineto{\pgfqpoint{3.866393in}{2.475943in}}%
\pgfpathlineto{\pgfqpoint{3.859982in}{2.490013in}}%
\pgfpathlineto{\pgfqpoint{3.853572in}{2.504043in}}%
\pgfpathlineto{\pgfqpoint{3.847164in}{2.518041in}}%
\pgfpathlineto{\pgfqpoint{3.835953in}{2.514581in}}%
\pgfpathlineto{\pgfqpoint{3.824736in}{2.511085in}}%
\pgfpathlineto{\pgfqpoint{3.813512in}{2.507544in}}%
\pgfpathlineto{\pgfqpoint{3.802282in}{2.503948in}}%
\pgfpathlineto{\pgfqpoint{3.791045in}{2.500286in}}%
\pgfpathlineto{\pgfqpoint{3.797451in}{2.486327in}}%
\pgfpathlineto{\pgfqpoint{3.803859in}{2.472332in}}%
\pgfpathlineto{\pgfqpoint{3.810268in}{2.458283in}}%
\pgfpathlineto{\pgfqpoint{3.816679in}{2.444164in}}%
\pgfpathclose%
\pgfusepath{stroke,fill}%
\end{pgfscope}%
\begin{pgfscope}%
\pgfpathrectangle{\pgfqpoint{0.887500in}{0.275000in}}{\pgfqpoint{4.225000in}{4.225000in}}%
\pgfusepath{clip}%
\pgfsetbuttcap%
\pgfsetroundjoin%
\definecolor{currentfill}{rgb}{0.154815,0.493313,0.557840}%
\pgfsetfillcolor{currentfill}%
\pgfsetfillopacity{0.700000}%
\pgfsetlinewidth{0.501875pt}%
\definecolor{currentstroke}{rgb}{1.000000,1.000000,1.000000}%
\pgfsetstrokecolor{currentstroke}%
\pgfsetstrokeopacity{0.500000}%
\pgfsetdash{}{0pt}%
\pgfpathmoveto{\pgfqpoint{4.119869in}{2.194746in}}%
\pgfpathlineto{\pgfqpoint{4.131026in}{2.198040in}}%
\pgfpathlineto{\pgfqpoint{4.142177in}{2.201329in}}%
\pgfpathlineto{\pgfqpoint{4.153322in}{2.204611in}}%
\pgfpathlineto{\pgfqpoint{4.164461in}{2.207880in}}%
\pgfpathlineto{\pgfqpoint{4.175594in}{2.211134in}}%
\pgfpathlineto{\pgfqpoint{4.169132in}{2.225662in}}%
\pgfpathlineto{\pgfqpoint{4.162672in}{2.240160in}}%
\pgfpathlineto{\pgfqpoint{4.156214in}{2.254634in}}%
\pgfpathlineto{\pgfqpoint{4.149757in}{2.269089in}}%
\pgfpathlineto{\pgfqpoint{4.143303in}{2.283528in}}%
\pgfpathlineto{\pgfqpoint{4.132168in}{2.280162in}}%
\pgfpathlineto{\pgfqpoint{4.121027in}{2.276775in}}%
\pgfpathlineto{\pgfqpoint{4.109880in}{2.273372in}}%
\pgfpathlineto{\pgfqpoint{4.098728in}{2.269961in}}%
\pgfpathlineto{\pgfqpoint{4.087570in}{2.266546in}}%
\pgfpathlineto{\pgfqpoint{4.094027in}{2.252274in}}%
\pgfpathlineto{\pgfqpoint{4.100486in}{2.237980in}}%
\pgfpathlineto{\pgfqpoint{4.106946in}{2.223642in}}%
\pgfpathlineto{\pgfqpoint{4.113408in}{2.209238in}}%
\pgfpathclose%
\pgfusepath{stroke,fill}%
\end{pgfscope}%
\begin{pgfscope}%
\pgfpathrectangle{\pgfqpoint{0.887500in}{0.275000in}}{\pgfqpoint{4.225000in}{4.225000in}}%
\pgfusepath{clip}%
\pgfsetbuttcap%
\pgfsetroundjoin%
\definecolor{currentfill}{rgb}{0.159194,0.482237,0.558073}%
\pgfsetfillcolor{currentfill}%
\pgfsetfillopacity{0.700000}%
\pgfsetlinewidth{0.501875pt}%
\definecolor{currentstroke}{rgb}{1.000000,1.000000,1.000000}%
\pgfsetstrokecolor{currentstroke}%
\pgfsetstrokeopacity{0.500000}%
\pgfsetdash{}{0pt}%
\pgfpathmoveto{\pgfqpoint{2.221127in}{2.187301in}}%
\pgfpathlineto{\pgfqpoint{2.232761in}{2.190848in}}%
\pgfpathlineto{\pgfqpoint{2.244388in}{2.194400in}}%
\pgfpathlineto{\pgfqpoint{2.256010in}{2.197954in}}%
\pgfpathlineto{\pgfqpoint{2.267626in}{2.201505in}}%
\pgfpathlineto{\pgfqpoint{2.279237in}{2.205052in}}%
\pgfpathlineto{\pgfqpoint{2.273229in}{2.213976in}}%
\pgfpathlineto{\pgfqpoint{2.267226in}{2.222873in}}%
\pgfpathlineto{\pgfqpoint{2.261227in}{2.231744in}}%
\pgfpathlineto{\pgfqpoint{2.255233in}{2.240587in}}%
\pgfpathlineto{\pgfqpoint{2.249242in}{2.249405in}}%
\pgfpathlineto{\pgfqpoint{2.237643in}{2.245905in}}%
\pgfpathlineto{\pgfqpoint{2.226038in}{2.242400in}}%
\pgfpathlineto{\pgfqpoint{2.214427in}{2.238894in}}%
\pgfpathlineto{\pgfqpoint{2.202810in}{2.235389in}}%
\pgfpathlineto{\pgfqpoint{2.191187in}{2.231890in}}%
\pgfpathlineto{\pgfqpoint{2.197167in}{2.223027in}}%
\pgfpathlineto{\pgfqpoint{2.203150in}{2.214136in}}%
\pgfpathlineto{\pgfqpoint{2.209138in}{2.205219in}}%
\pgfpathlineto{\pgfqpoint{2.215130in}{2.196274in}}%
\pgfpathclose%
\pgfusepath{stroke,fill}%
\end{pgfscope}%
\begin{pgfscope}%
\pgfpathrectangle{\pgfqpoint{0.887500in}{0.275000in}}{\pgfqpoint{4.225000in}{4.225000in}}%
\pgfusepath{clip}%
\pgfsetbuttcap%
\pgfsetroundjoin%
\definecolor{currentfill}{rgb}{0.147607,0.511733,0.557049}%
\pgfsetfillcolor{currentfill}%
\pgfsetfillopacity{0.700000}%
\pgfsetlinewidth{0.501875pt}%
\definecolor{currentstroke}{rgb}{1.000000,1.000000,1.000000}%
\pgfsetstrokecolor{currentstroke}%
\pgfsetstrokeopacity{0.500000}%
\pgfsetdash{}{0pt}%
\pgfpathmoveto{\pgfqpoint{1.898414in}{2.250672in}}%
\pgfpathlineto{\pgfqpoint{1.910128in}{2.254174in}}%
\pgfpathlineto{\pgfqpoint{1.921837in}{2.257671in}}%
\pgfpathlineto{\pgfqpoint{1.933539in}{2.261163in}}%
\pgfpathlineto{\pgfqpoint{1.945237in}{2.264647in}}%
\pgfpathlineto{\pgfqpoint{1.956928in}{2.268119in}}%
\pgfpathlineto{\pgfqpoint{1.951030in}{2.276844in}}%
\pgfpathlineto{\pgfqpoint{1.945136in}{2.285543in}}%
\pgfpathlineto{\pgfqpoint{1.939247in}{2.294217in}}%
\pgfpathlineto{\pgfqpoint{1.933362in}{2.302865in}}%
\pgfpathlineto{\pgfqpoint{1.927482in}{2.311489in}}%
\pgfpathlineto{\pgfqpoint{1.915802in}{2.308058in}}%
\pgfpathlineto{\pgfqpoint{1.904116in}{2.304616in}}%
\pgfpathlineto{\pgfqpoint{1.892425in}{2.301167in}}%
\pgfpathlineto{\pgfqpoint{1.880727in}{2.297712in}}%
\pgfpathlineto{\pgfqpoint{1.869025in}{2.294255in}}%
\pgfpathlineto{\pgfqpoint{1.874894in}{2.285589in}}%
\pgfpathlineto{\pgfqpoint{1.880767in}{2.276898in}}%
\pgfpathlineto{\pgfqpoint{1.886645in}{2.268182in}}%
\pgfpathlineto{\pgfqpoint{1.892527in}{2.259440in}}%
\pgfpathclose%
\pgfusepath{stroke,fill}%
\end{pgfscope}%
\begin{pgfscope}%
\pgfpathrectangle{\pgfqpoint{0.887500in}{0.275000in}}{\pgfqpoint{4.225000in}{4.225000in}}%
\pgfusepath{clip}%
\pgfsetbuttcap%
\pgfsetroundjoin%
\definecolor{currentfill}{rgb}{0.124395,0.578002,0.548287}%
\pgfsetfillcolor{currentfill}%
\pgfsetfillopacity{0.700000}%
\pgfsetlinewidth{0.501875pt}%
\definecolor{currentstroke}{rgb}{1.000000,1.000000,1.000000}%
\pgfsetstrokecolor{currentstroke}%
\pgfsetstrokeopacity{0.500000}%
\pgfsetdash{}{0pt}%
\pgfpathmoveto{\pgfqpoint{3.152110in}{2.341338in}}%
\pgfpathlineto{\pgfqpoint{3.163546in}{2.361928in}}%
\pgfpathlineto{\pgfqpoint{3.174985in}{2.382206in}}%
\pgfpathlineto{\pgfqpoint{3.186428in}{2.402405in}}%
\pgfpathlineto{\pgfqpoint{3.197876in}{2.422673in}}%
\pgfpathlineto{\pgfqpoint{3.209327in}{2.442875in}}%
\pgfpathlineto{\pgfqpoint{3.203038in}{2.458607in}}%
\pgfpathlineto{\pgfqpoint{3.196749in}{2.473875in}}%
\pgfpathlineto{\pgfqpoint{3.190460in}{2.488576in}}%
\pgfpathlineto{\pgfqpoint{3.184171in}{2.502611in}}%
\pgfpathlineto{\pgfqpoint{3.177883in}{2.515936in}}%
\pgfpathlineto{\pgfqpoint{3.166440in}{2.493063in}}%
\pgfpathlineto{\pgfqpoint{3.155003in}{2.470241in}}%
\pgfpathlineto{\pgfqpoint{3.143573in}{2.447729in}}%
\pgfpathlineto{\pgfqpoint{3.132148in}{2.425376in}}%
\pgfpathlineto{\pgfqpoint{3.120728in}{2.402907in}}%
\pgfpathlineto{\pgfqpoint{3.126998in}{2.390388in}}%
\pgfpathlineto{\pgfqpoint{3.133271in}{2.378011in}}%
\pgfpathlineto{\pgfqpoint{3.139548in}{2.365751in}}%
\pgfpathlineto{\pgfqpoint{3.145827in}{2.353547in}}%
\pgfpathclose%
\pgfusepath{stroke,fill}%
\end{pgfscope}%
\begin{pgfscope}%
\pgfpathrectangle{\pgfqpoint{0.887500in}{0.275000in}}{\pgfqpoint{4.225000in}{4.225000in}}%
\pgfusepath{clip}%
\pgfsetbuttcap%
\pgfsetroundjoin%
\definecolor{currentfill}{rgb}{0.203063,0.379716,0.553925}%
\pgfsetfillcolor{currentfill}%
\pgfsetfillopacity{0.700000}%
\pgfsetlinewidth{0.501875pt}%
\definecolor{currentstroke}{rgb}{1.000000,1.000000,1.000000}%
\pgfsetstrokecolor{currentstroke}%
\pgfsetstrokeopacity{0.500000}%
\pgfsetdash{}{0pt}%
\pgfpathmoveto{\pgfqpoint{4.416755in}{1.962125in}}%
\pgfpathlineto{\pgfqpoint{4.427844in}{1.965658in}}%
\pgfpathlineto{\pgfqpoint{4.438926in}{1.969140in}}%
\pgfpathlineto{\pgfqpoint{4.450001in}{1.972573in}}%
\pgfpathlineto{\pgfqpoint{4.461069in}{1.975957in}}%
\pgfpathlineto{\pgfqpoint{4.472129in}{1.979297in}}%
\pgfpathlineto{\pgfqpoint{4.465610in}{1.993810in}}%
\pgfpathlineto{\pgfqpoint{4.459091in}{2.008222in}}%
\pgfpathlineto{\pgfqpoint{4.452569in}{2.022517in}}%
\pgfpathlineto{\pgfqpoint{4.446046in}{2.036683in}}%
\pgfpathlineto{\pgfqpoint{4.439520in}{2.050703in}}%
\pgfpathlineto{\pgfqpoint{4.428458in}{2.047249in}}%
\pgfpathlineto{\pgfqpoint{4.417388in}{2.043766in}}%
\pgfpathlineto{\pgfqpoint{4.406312in}{2.040247in}}%
\pgfpathlineto{\pgfqpoint{4.395229in}{2.036684in}}%
\pgfpathlineto{\pgfqpoint{4.384139in}{2.033067in}}%
\pgfpathlineto{\pgfqpoint{4.390664in}{2.019096in}}%
\pgfpathlineto{\pgfqpoint{4.397188in}{2.005006in}}%
\pgfpathlineto{\pgfqpoint{4.403711in}{1.990807in}}%
\pgfpathlineto{\pgfqpoint{4.410233in}{1.976510in}}%
\pgfpathclose%
\pgfusepath{stroke,fill}%
\end{pgfscope}%
\begin{pgfscope}%
\pgfpathrectangle{\pgfqpoint{0.887500in}{0.275000in}}{\pgfqpoint{4.225000in}{4.225000in}}%
\pgfusepath{clip}%
\pgfsetbuttcap%
\pgfsetroundjoin%
\definecolor{currentfill}{rgb}{0.151918,0.500685,0.557587}%
\pgfsetfillcolor{currentfill}%
\pgfsetfillopacity{0.700000}%
\pgfsetlinewidth{0.501875pt}%
\definecolor{currentstroke}{rgb}{1.000000,1.000000,1.000000}%
\pgfsetstrokecolor{currentstroke}%
\pgfsetstrokeopacity{0.500000}%
\pgfsetdash{}{0pt}%
\pgfpathmoveto{\pgfqpoint{3.126347in}{2.171995in}}%
\pgfpathlineto{\pgfqpoint{3.137786in}{2.195894in}}%
\pgfpathlineto{\pgfqpoint{3.149228in}{2.218419in}}%
\pgfpathlineto{\pgfqpoint{3.160672in}{2.239513in}}%
\pgfpathlineto{\pgfqpoint{3.172117in}{2.259339in}}%
\pgfpathlineto{\pgfqpoint{3.183562in}{2.278061in}}%
\pgfpathlineto{\pgfqpoint{3.177268in}{2.291219in}}%
\pgfpathlineto{\pgfqpoint{3.170975in}{2.304065in}}%
\pgfpathlineto{\pgfqpoint{3.164684in}{2.316658in}}%
\pgfpathlineto{\pgfqpoint{3.158396in}{2.329063in}}%
\pgfpathlineto{\pgfqpoint{3.152110in}{2.341338in}}%
\pgfpathlineto{\pgfqpoint{3.140677in}{2.320204in}}%
\pgfpathlineto{\pgfqpoint{3.129248in}{2.298293in}}%
\pgfpathlineto{\pgfqpoint{3.117822in}{2.275375in}}%
\pgfpathlineto{\pgfqpoint{3.106400in}{2.251218in}}%
\pgfpathlineto{\pgfqpoint{3.094983in}{2.225828in}}%
\pgfpathlineto{\pgfqpoint{3.101249in}{2.215351in}}%
\pgfpathlineto{\pgfqpoint{3.107519in}{2.204799in}}%
\pgfpathlineto{\pgfqpoint{3.113791in}{2.194103in}}%
\pgfpathlineto{\pgfqpoint{3.120068in}{2.183191in}}%
\pgfpathclose%
\pgfusepath{stroke,fill}%
\end{pgfscope}%
\begin{pgfscope}%
\pgfpathrectangle{\pgfqpoint{0.887500in}{0.275000in}}{\pgfqpoint{4.225000in}{4.225000in}}%
\pgfusepath{clip}%
\pgfsetbuttcap%
\pgfsetroundjoin%
\definecolor{currentfill}{rgb}{0.199430,0.387607,0.554642}%
\pgfsetfillcolor{currentfill}%
\pgfsetfillopacity{0.700000}%
\pgfsetlinewidth{0.501875pt}%
\definecolor{currentstroke}{rgb}{1.000000,1.000000,1.000000}%
\pgfsetstrokecolor{currentstroke}%
\pgfsetstrokeopacity{0.500000}%
\pgfsetdash{}{0pt}%
\pgfpathmoveto{\pgfqpoint{3.043468in}{1.978791in}}%
\pgfpathlineto{\pgfqpoint{3.054900in}{1.980522in}}%
\pgfpathlineto{\pgfqpoint{3.066325in}{1.983887in}}%
\pgfpathlineto{\pgfqpoint{3.077746in}{1.989630in}}%
\pgfpathlineto{\pgfqpoint{3.089165in}{1.998497in}}%
\pgfpathlineto{\pgfqpoint{3.100586in}{2.011106in}}%
\pgfpathlineto{\pgfqpoint{3.094311in}{2.021356in}}%
\pgfpathlineto{\pgfqpoint{3.088039in}{2.031529in}}%
\pgfpathlineto{\pgfqpoint{3.081771in}{2.041624in}}%
\pgfpathlineto{\pgfqpoint{3.075507in}{2.051642in}}%
\pgfpathlineto{\pgfqpoint{3.069246in}{2.061584in}}%
\pgfpathlineto{\pgfqpoint{3.057842in}{2.047649in}}%
\pgfpathlineto{\pgfqpoint{3.046437in}{2.038024in}}%
\pgfpathlineto{\pgfqpoint{3.035027in}{2.031996in}}%
\pgfpathlineto{\pgfqpoint{3.023612in}{2.028706in}}%
\pgfpathlineto{\pgfqpoint{3.012188in}{2.027299in}}%
\pgfpathlineto{\pgfqpoint{3.018436in}{2.017603in}}%
\pgfpathlineto{\pgfqpoint{3.024688in}{2.007903in}}%
\pgfpathlineto{\pgfqpoint{3.030944in}{1.998202in}}%
\pgfpathlineto{\pgfqpoint{3.037204in}{1.988499in}}%
\pgfpathclose%
\pgfusepath{stroke,fill}%
\end{pgfscope}%
\begin{pgfscope}%
\pgfpathrectangle{\pgfqpoint{0.887500in}{0.275000in}}{\pgfqpoint{4.225000in}{4.225000in}}%
\pgfusepath{clip}%
\pgfsetbuttcap%
\pgfsetroundjoin%
\definecolor{currentfill}{rgb}{0.120638,0.625828,0.533488}%
\pgfsetfillcolor{currentfill}%
\pgfsetfillopacity{0.700000}%
\pgfsetlinewidth{0.501875pt}%
\definecolor{currentstroke}{rgb}{1.000000,1.000000,1.000000}%
\pgfsetstrokecolor{currentstroke}%
\pgfsetstrokeopacity{0.500000}%
\pgfsetdash{}{0pt}%
\pgfpathmoveto{\pgfqpoint{3.734760in}{2.480734in}}%
\pgfpathlineto{\pgfqpoint{3.746030in}{2.484820in}}%
\pgfpathlineto{\pgfqpoint{3.757294in}{2.488818in}}%
\pgfpathlineto{\pgfqpoint{3.768551in}{2.492726in}}%
\pgfpathlineto{\pgfqpoint{3.779802in}{2.496549in}}%
\pgfpathlineto{\pgfqpoint{3.791045in}{2.500286in}}%
\pgfpathlineto{\pgfqpoint{3.784642in}{2.514228in}}%
\pgfpathlineto{\pgfqpoint{3.778241in}{2.528169in}}%
\pgfpathlineto{\pgfqpoint{3.771842in}{2.542127in}}%
\pgfpathlineto{\pgfqpoint{3.765447in}{2.556113in}}%
\pgfpathlineto{\pgfqpoint{3.759055in}{2.570122in}}%
\pgfpathlineto{\pgfqpoint{3.747816in}{2.566422in}}%
\pgfpathlineto{\pgfqpoint{3.736569in}{2.562656in}}%
\pgfpathlineto{\pgfqpoint{3.725317in}{2.558832in}}%
\pgfpathlineto{\pgfqpoint{3.714058in}{2.554961in}}%
\pgfpathlineto{\pgfqpoint{3.702794in}{2.551054in}}%
\pgfpathlineto{\pgfqpoint{3.709181in}{2.536945in}}%
\pgfpathlineto{\pgfqpoint{3.715571in}{2.522861in}}%
\pgfpathlineto{\pgfqpoint{3.721965in}{2.508808in}}%
\pgfpathlineto{\pgfqpoint{3.728361in}{2.494772in}}%
\pgfpathclose%
\pgfusepath{stroke,fill}%
\end{pgfscope}%
\begin{pgfscope}%
\pgfpathrectangle{\pgfqpoint{0.887500in}{0.275000in}}{\pgfqpoint{4.225000in}{4.225000in}}%
\pgfusepath{clip}%
\pgfsetbuttcap%
\pgfsetroundjoin%
\definecolor{currentfill}{rgb}{0.220124,0.725509,0.466226}%
\pgfsetfillcolor{currentfill}%
\pgfsetfillopacity{0.700000}%
\pgfsetlinewidth{0.501875pt}%
\definecolor{currentstroke}{rgb}{1.000000,1.000000,1.000000}%
\pgfsetstrokecolor{currentstroke}%
\pgfsetstrokeopacity{0.500000}%
\pgfsetdash{}{0pt}%
\pgfpathmoveto{\pgfqpoint{3.292266in}{2.685586in}}%
\pgfpathlineto{\pgfqpoint{3.303674in}{2.695769in}}%
\pgfpathlineto{\pgfqpoint{3.315078in}{2.705456in}}%
\pgfpathlineto{\pgfqpoint{3.326476in}{2.714776in}}%
\pgfpathlineto{\pgfqpoint{3.337870in}{2.723702in}}%
\pgfpathlineto{\pgfqpoint{3.349258in}{2.732192in}}%
\pgfpathlineto{\pgfqpoint{3.342936in}{2.745869in}}%
\pgfpathlineto{\pgfqpoint{3.336614in}{2.759167in}}%
\pgfpathlineto{\pgfqpoint{3.330293in}{2.772171in}}%
\pgfpathlineto{\pgfqpoint{3.323973in}{2.784968in}}%
\pgfpathlineto{\pgfqpoint{3.317655in}{2.797644in}}%
\pgfpathlineto{\pgfqpoint{3.306272in}{2.788608in}}%
\pgfpathlineto{\pgfqpoint{3.294883in}{2.778932in}}%
\pgfpathlineto{\pgfqpoint{3.283488in}{2.768618in}}%
\pgfpathlineto{\pgfqpoint{3.272088in}{2.757662in}}%
\pgfpathlineto{\pgfqpoint{3.260683in}{2.746005in}}%
\pgfpathlineto{\pgfqpoint{3.266998in}{2.734648in}}%
\pgfpathlineto{\pgfqpoint{3.273314in}{2.723116in}}%
\pgfpathlineto{\pgfqpoint{3.279632in}{2.711222in}}%
\pgfpathlineto{\pgfqpoint{3.285950in}{2.698775in}}%
\pgfpathclose%
\pgfusepath{stroke,fill}%
\end{pgfscope}%
\begin{pgfscope}%
\pgfpathrectangle{\pgfqpoint{0.887500in}{0.275000in}}{\pgfqpoint{4.225000in}{4.225000in}}%
\pgfusepath{clip}%
\pgfsetbuttcap%
\pgfsetroundjoin%
\definecolor{currentfill}{rgb}{0.175841,0.441290,0.557685}%
\pgfsetfillcolor{currentfill}%
\pgfsetfillopacity{0.700000}%
\pgfsetlinewidth{0.501875pt}%
\definecolor{currentstroke}{rgb}{1.000000,1.000000,1.000000}%
\pgfsetstrokecolor{currentstroke}%
\pgfsetstrokeopacity{0.500000}%
\pgfsetdash{}{0pt}%
\pgfpathmoveto{\pgfqpoint{2.632196in}{2.092319in}}%
\pgfpathlineto{\pgfqpoint{2.643730in}{2.095934in}}%
\pgfpathlineto{\pgfqpoint{2.655259in}{2.099571in}}%
\pgfpathlineto{\pgfqpoint{2.666783in}{2.103224in}}%
\pgfpathlineto{\pgfqpoint{2.678300in}{2.106869in}}%
\pgfpathlineto{\pgfqpoint{2.689813in}{2.110479in}}%
\pgfpathlineto{\pgfqpoint{2.683667in}{2.119828in}}%
\pgfpathlineto{\pgfqpoint{2.677525in}{2.129140in}}%
\pgfpathlineto{\pgfqpoint{2.671387in}{2.138415in}}%
\pgfpathlineto{\pgfqpoint{2.665253in}{2.147655in}}%
\pgfpathlineto{\pgfqpoint{2.659124in}{2.156861in}}%
\pgfpathlineto{\pgfqpoint{2.647622in}{2.153276in}}%
\pgfpathlineto{\pgfqpoint{2.636115in}{2.149658in}}%
\pgfpathlineto{\pgfqpoint{2.624603in}{2.146034in}}%
\pgfpathlineto{\pgfqpoint{2.613084in}{2.142429in}}%
\pgfpathlineto{\pgfqpoint{2.601560in}{2.138850in}}%
\pgfpathlineto{\pgfqpoint{2.607679in}{2.129615in}}%
\pgfpathlineto{\pgfqpoint{2.613802in}{2.120346in}}%
\pgfpathlineto{\pgfqpoint{2.619929in}{2.111041in}}%
\pgfpathlineto{\pgfqpoint{2.626060in}{2.101699in}}%
\pgfpathclose%
\pgfusepath{stroke,fill}%
\end{pgfscope}%
\begin{pgfscope}%
\pgfpathrectangle{\pgfqpoint{0.887500in}{0.275000in}}{\pgfqpoint{4.225000in}{4.225000in}}%
\pgfusepath{clip}%
\pgfsetbuttcap%
\pgfsetroundjoin%
\definecolor{currentfill}{rgb}{0.246811,0.283237,0.535941}%
\pgfsetfillcolor{currentfill}%
\pgfsetfillopacity{0.700000}%
\pgfsetlinewidth{0.501875pt}%
\definecolor{currentstroke}{rgb}{1.000000,1.000000,1.000000}%
\pgfsetstrokecolor{currentstroke}%
\pgfsetstrokeopacity{0.500000}%
\pgfsetdash{}{0pt}%
\pgfpathmoveto{\pgfqpoint{4.625348in}{1.775542in}}%
\pgfpathlineto{\pgfqpoint{4.636375in}{1.778803in}}%
\pgfpathlineto{\pgfqpoint{4.647396in}{1.782064in}}%
\pgfpathlineto{\pgfqpoint{4.658411in}{1.785326in}}%
\pgfpathlineto{\pgfqpoint{4.669422in}{1.788589in}}%
\pgfpathlineto{\pgfqpoint{4.680426in}{1.791855in}}%
\pgfpathlineto{\pgfqpoint{4.673879in}{1.806654in}}%
\pgfpathlineto{\pgfqpoint{4.667333in}{1.821433in}}%
\pgfpathlineto{\pgfqpoint{4.660789in}{1.836194in}}%
\pgfpathlineto{\pgfqpoint{4.654246in}{1.850935in}}%
\pgfpathlineto{\pgfqpoint{4.647706in}{1.865658in}}%
\pgfpathlineto{\pgfqpoint{4.636698in}{1.862306in}}%
\pgfpathlineto{\pgfqpoint{4.625685in}{1.858945in}}%
\pgfpathlineto{\pgfqpoint{4.614665in}{1.855570in}}%
\pgfpathlineto{\pgfqpoint{4.603640in}{1.852181in}}%
\pgfpathlineto{\pgfqpoint{4.592608in}{1.848775in}}%
\pgfpathlineto{\pgfqpoint{4.599148in}{1.834072in}}%
\pgfpathlineto{\pgfqpoint{4.605693in}{1.819392in}}%
\pgfpathlineto{\pgfqpoint{4.612240in}{1.804739in}}%
\pgfpathlineto{\pgfqpoint{4.618792in}{1.790120in}}%
\pgfpathclose%
\pgfusepath{stroke,fill}%
\end{pgfscope}%
\begin{pgfscope}%
\pgfpathrectangle{\pgfqpoint{0.887500in}{0.275000in}}{\pgfqpoint{4.225000in}{4.225000in}}%
\pgfusepath{clip}%
\pgfsetbuttcap%
\pgfsetroundjoin%
\definecolor{currentfill}{rgb}{0.183898,0.422383,0.556944}%
\pgfsetfillcolor{currentfill}%
\pgfsetfillopacity{0.700000}%
\pgfsetlinewidth{0.501875pt}%
\definecolor{currentstroke}{rgb}{1.000000,1.000000,1.000000}%
\pgfsetstrokecolor{currentstroke}%
\pgfsetstrokeopacity{0.500000}%
\pgfsetdash{}{0pt}%
\pgfpathmoveto{\pgfqpoint{3.100586in}{2.011106in}}%
\pgfpathlineto{\pgfqpoint{3.112011in}{2.027123in}}%
\pgfpathlineto{\pgfqpoint{3.123443in}{2.045777in}}%
\pgfpathlineto{\pgfqpoint{3.134881in}{2.066293in}}%
\pgfpathlineto{\pgfqpoint{3.146326in}{2.087898in}}%
\pgfpathlineto{\pgfqpoint{3.157776in}{2.109812in}}%
\pgfpathlineto{\pgfqpoint{3.151487in}{2.123138in}}%
\pgfpathlineto{\pgfqpoint{3.145199in}{2.136029in}}%
\pgfpathlineto{\pgfqpoint{3.138912in}{2.148468in}}%
\pgfpathlineto{\pgfqpoint{3.132628in}{2.160444in}}%
\pgfpathlineto{\pgfqpoint{3.126347in}{2.171995in}}%
\pgfpathlineto{\pgfqpoint{3.114913in}{2.147491in}}%
\pgfpathlineto{\pgfqpoint{3.103486in}{2.123292in}}%
\pgfpathlineto{\pgfqpoint{3.092066in}{2.100306in}}%
\pgfpathlineto{\pgfqpoint{3.080653in}{2.079436in}}%
\pgfpathlineto{\pgfqpoint{3.069246in}{2.061584in}}%
\pgfpathlineto{\pgfqpoint{3.075507in}{2.051642in}}%
\pgfpathlineto{\pgfqpoint{3.081771in}{2.041624in}}%
\pgfpathlineto{\pgfqpoint{3.088039in}{2.031529in}}%
\pgfpathlineto{\pgfqpoint{3.094311in}{2.021356in}}%
\pgfpathclose%
\pgfusepath{stroke,fill}%
\end{pgfscope}%
\begin{pgfscope}%
\pgfpathrectangle{\pgfqpoint{0.887500in}{0.275000in}}{\pgfqpoint{4.225000in}{4.225000in}}%
\pgfusepath{clip}%
\pgfsetbuttcap%
\pgfsetroundjoin%
\definecolor{currentfill}{rgb}{0.144759,0.519093,0.556572}%
\pgfsetfillcolor{currentfill}%
\pgfsetfillopacity{0.700000}%
\pgfsetlinewidth{0.501875pt}%
\definecolor{currentstroke}{rgb}{1.000000,1.000000,1.000000}%
\pgfsetstrokecolor{currentstroke}%
\pgfsetstrokeopacity{0.500000}%
\pgfsetdash{}{0pt}%
\pgfpathmoveto{\pgfqpoint{4.031698in}{2.249513in}}%
\pgfpathlineto{\pgfqpoint{4.042884in}{2.252941in}}%
\pgfpathlineto{\pgfqpoint{4.054063in}{2.256339in}}%
\pgfpathlineto{\pgfqpoint{4.065237in}{2.259731in}}%
\pgfpathlineto{\pgfqpoint{4.076406in}{2.263134in}}%
\pgfpathlineto{\pgfqpoint{4.087570in}{2.266546in}}%
\pgfpathlineto{\pgfqpoint{4.081115in}{2.280819in}}%
\pgfpathlineto{\pgfqpoint{4.074664in}{2.295113in}}%
\pgfpathlineto{\pgfqpoint{4.068216in}{2.309445in}}%
\pgfpathlineto{\pgfqpoint{4.061771in}{2.323810in}}%
\pgfpathlineto{\pgfqpoint{4.055330in}{2.338199in}}%
\pgfpathlineto{\pgfqpoint{4.044166in}{2.334697in}}%
\pgfpathlineto{\pgfqpoint{4.032997in}{2.331204in}}%
\pgfpathlineto{\pgfqpoint{4.021823in}{2.327725in}}%
\pgfpathlineto{\pgfqpoint{4.010643in}{2.324251in}}%
\pgfpathlineto{\pgfqpoint{3.999458in}{2.320766in}}%
\pgfpathlineto{\pgfqpoint{4.005898in}{2.306391in}}%
\pgfpathlineto{\pgfqpoint{4.012341in}{2.292070in}}%
\pgfpathlineto{\pgfqpoint{4.018789in}{2.277818in}}%
\pgfpathlineto{\pgfqpoint{4.025242in}{2.263641in}}%
\pgfpathclose%
\pgfusepath{stroke,fill}%
\end{pgfscope}%
\begin{pgfscope}%
\pgfpathrectangle{\pgfqpoint{0.887500in}{0.275000in}}{\pgfqpoint{4.225000in}{4.225000in}}%
\pgfusepath{clip}%
\pgfsetbuttcap%
\pgfsetroundjoin%
\definecolor{currentfill}{rgb}{0.190631,0.407061,0.556089}%
\pgfsetfillcolor{currentfill}%
\pgfsetfillopacity{0.700000}%
\pgfsetlinewidth{0.501875pt}%
\definecolor{currentstroke}{rgb}{1.000000,1.000000,1.000000}%
\pgfsetstrokecolor{currentstroke}%
\pgfsetstrokeopacity{0.500000}%
\pgfsetdash{}{0pt}%
\pgfpathmoveto{\pgfqpoint{4.328571in}{2.013871in}}%
\pgfpathlineto{\pgfqpoint{4.339701in}{2.017884in}}%
\pgfpathlineto{\pgfqpoint{4.350822in}{2.021804in}}%
\pgfpathlineto{\pgfqpoint{4.361936in}{2.025636in}}%
\pgfpathlineto{\pgfqpoint{4.373041in}{2.029387in}}%
\pgfpathlineto{\pgfqpoint{4.384139in}{2.033067in}}%
\pgfpathlineto{\pgfqpoint{4.377612in}{2.046907in}}%
\pgfpathlineto{\pgfqpoint{4.371083in}{2.060614in}}%
\pgfpathlineto{\pgfqpoint{4.364554in}{2.074218in}}%
\pgfpathlineto{\pgfqpoint{4.358025in}{2.087751in}}%
\pgfpathlineto{\pgfqpoint{4.351498in}{2.101250in}}%
\pgfpathlineto{\pgfqpoint{4.340415in}{2.097916in}}%
\pgfpathlineto{\pgfqpoint{4.329325in}{2.094551in}}%
\pgfpathlineto{\pgfqpoint{4.318227in}{2.091117in}}%
\pgfpathlineto{\pgfqpoint{4.307120in}{2.087577in}}%
\pgfpathlineto{\pgfqpoint{4.296005in}{2.083920in}}%
\pgfpathlineto{\pgfqpoint{4.302512in}{2.069884in}}%
\pgfpathlineto{\pgfqpoint{4.309022in}{2.055868in}}%
\pgfpathlineto{\pgfqpoint{4.315536in}{2.041866in}}%
\pgfpathlineto{\pgfqpoint{4.322052in}{2.027869in}}%
\pgfpathclose%
\pgfusepath{stroke,fill}%
\end{pgfscope}%
\begin{pgfscope}%
\pgfpathrectangle{\pgfqpoint{0.887500in}{0.275000in}}{\pgfqpoint{4.225000in}{4.225000in}}%
\pgfusepath{clip}%
\pgfsetbuttcap%
\pgfsetroundjoin%
\definecolor{currentfill}{rgb}{0.163625,0.471133,0.558148}%
\pgfsetfillcolor{currentfill}%
\pgfsetfillopacity{0.700000}%
\pgfsetlinewidth{0.501875pt}%
\definecolor{currentstroke}{rgb}{1.000000,1.000000,1.000000}%
\pgfsetstrokecolor{currentstroke}%
\pgfsetstrokeopacity{0.500000}%
\pgfsetdash{}{0pt}%
\pgfpathmoveto{\pgfqpoint{2.309340in}{2.160009in}}%
\pgfpathlineto{\pgfqpoint{2.320955in}{2.163596in}}%
\pgfpathlineto{\pgfqpoint{2.332566in}{2.167170in}}%
\pgfpathlineto{\pgfqpoint{2.344170in}{2.170731in}}%
\pgfpathlineto{\pgfqpoint{2.355770in}{2.174276in}}%
\pgfpathlineto{\pgfqpoint{2.367364in}{2.177805in}}%
\pgfpathlineto{\pgfqpoint{2.361324in}{2.186820in}}%
\pgfpathlineto{\pgfqpoint{2.355288in}{2.195807in}}%
\pgfpathlineto{\pgfqpoint{2.349257in}{2.204766in}}%
\pgfpathlineto{\pgfqpoint{2.343230in}{2.213697in}}%
\pgfpathlineto{\pgfqpoint{2.337207in}{2.222601in}}%
\pgfpathlineto{\pgfqpoint{2.325624in}{2.219121in}}%
\pgfpathlineto{\pgfqpoint{2.314035in}{2.215625in}}%
\pgfpathlineto{\pgfqpoint{2.302441in}{2.212114in}}%
\pgfpathlineto{\pgfqpoint{2.290842in}{2.208589in}}%
\pgfpathlineto{\pgfqpoint{2.279237in}{2.205052in}}%
\pgfpathlineto{\pgfqpoint{2.285249in}{2.196100in}}%
\pgfpathlineto{\pgfqpoint{2.291265in}{2.187120in}}%
\pgfpathlineto{\pgfqpoint{2.297286in}{2.178112in}}%
\pgfpathlineto{\pgfqpoint{2.303310in}{2.169075in}}%
\pgfpathclose%
\pgfusepath{stroke,fill}%
\end{pgfscope}%
\begin{pgfscope}%
\pgfpathrectangle{\pgfqpoint{0.887500in}{0.275000in}}{\pgfqpoint{4.225000in}{4.225000in}}%
\pgfusepath{clip}%
\pgfsetbuttcap%
\pgfsetroundjoin%
\definecolor{currentfill}{rgb}{0.130067,0.651384,0.521608}%
\pgfsetfillcolor{currentfill}%
\pgfsetfillopacity{0.700000}%
\pgfsetlinewidth{0.501875pt}%
\definecolor{currentstroke}{rgb}{1.000000,1.000000,1.000000}%
\pgfsetstrokecolor{currentstroke}%
\pgfsetstrokeopacity{0.500000}%
\pgfsetdash{}{0pt}%
\pgfpathmoveto{\pgfqpoint{3.646388in}{2.531233in}}%
\pgfpathlineto{\pgfqpoint{3.657681in}{2.535255in}}%
\pgfpathlineto{\pgfqpoint{3.668967in}{2.539220in}}%
\pgfpathlineto{\pgfqpoint{3.680248in}{2.543173in}}%
\pgfpathlineto{\pgfqpoint{3.691523in}{2.547121in}}%
\pgfpathlineto{\pgfqpoint{3.702794in}{2.551054in}}%
\pgfpathlineto{\pgfqpoint{3.696409in}{2.565177in}}%
\pgfpathlineto{\pgfqpoint{3.690027in}{2.579306in}}%
\pgfpathlineto{\pgfqpoint{3.683647in}{2.593429in}}%
\pgfpathlineto{\pgfqpoint{3.677270in}{2.607539in}}%
\pgfpathlineto{\pgfqpoint{3.670895in}{2.621625in}}%
\pgfpathlineto{\pgfqpoint{3.659630in}{2.617817in}}%
\pgfpathlineto{\pgfqpoint{3.648359in}{2.614016in}}%
\pgfpathlineto{\pgfqpoint{3.637084in}{2.610238in}}%
\pgfpathlineto{\pgfqpoint{3.625803in}{2.606472in}}%
\pgfpathlineto{\pgfqpoint{3.614517in}{2.602657in}}%
\pgfpathlineto{\pgfqpoint{3.620888in}{2.588470in}}%
\pgfpathlineto{\pgfqpoint{3.627260in}{2.574230in}}%
\pgfpathlineto{\pgfqpoint{3.633634in}{2.559942in}}%
\pgfpathlineto{\pgfqpoint{3.640010in}{2.545608in}}%
\pgfpathclose%
\pgfusepath{stroke,fill}%
\end{pgfscope}%
\begin{pgfscope}%
\pgfpathrectangle{\pgfqpoint{0.887500in}{0.275000in}}{\pgfqpoint{4.225000in}{4.225000in}}%
\pgfusepath{clip}%
\pgfsetbuttcap%
\pgfsetroundjoin%
\definecolor{currentfill}{rgb}{0.140536,0.530132,0.555659}%
\pgfsetfillcolor{currentfill}%
\pgfsetfillopacity{0.700000}%
\pgfsetlinewidth{0.501875pt}%
\definecolor{currentstroke}{rgb}{1.000000,1.000000,1.000000}%
\pgfsetstrokecolor{currentstroke}%
\pgfsetstrokeopacity{0.500000}%
\pgfsetdash{}{0pt}%
\pgfpathmoveto{\pgfqpoint{1.663683in}{2.285684in}}%
\pgfpathlineto{\pgfqpoint{1.675460in}{2.289173in}}%
\pgfpathlineto{\pgfqpoint{1.687231in}{2.292650in}}%
\pgfpathlineto{\pgfqpoint{1.698997in}{2.296116in}}%
\pgfpathlineto{\pgfqpoint{1.710757in}{2.299571in}}%
\pgfpathlineto{\pgfqpoint{1.722512in}{2.303017in}}%
\pgfpathlineto{\pgfqpoint{1.716693in}{2.311620in}}%
\pgfpathlineto{\pgfqpoint{1.710878in}{2.320198in}}%
\pgfpathlineto{\pgfqpoint{1.705067in}{2.328751in}}%
\pgfpathlineto{\pgfqpoint{1.699262in}{2.337279in}}%
\pgfpathlineto{\pgfqpoint{1.693460in}{2.345781in}}%
\pgfpathlineto{\pgfqpoint{1.681717in}{2.342370in}}%
\pgfpathlineto{\pgfqpoint{1.669969in}{2.338951in}}%
\pgfpathlineto{\pgfqpoint{1.658215in}{2.335521in}}%
\pgfpathlineto{\pgfqpoint{1.646455in}{2.332080in}}%
\pgfpathlineto{\pgfqpoint{1.634690in}{2.328627in}}%
\pgfpathlineto{\pgfqpoint{1.640479in}{2.320093in}}%
\pgfpathlineto{\pgfqpoint{1.646273in}{2.311531in}}%
\pgfpathlineto{\pgfqpoint{1.652072in}{2.302943in}}%
\pgfpathlineto{\pgfqpoint{1.657875in}{2.294327in}}%
\pgfpathclose%
\pgfusepath{stroke,fill}%
\end{pgfscope}%
\begin{pgfscope}%
\pgfpathrectangle{\pgfqpoint{0.887500in}{0.275000in}}{\pgfqpoint{4.225000in}{4.225000in}}%
\pgfusepath{clip}%
\pgfsetbuttcap%
\pgfsetroundjoin%
\definecolor{currentfill}{rgb}{0.151918,0.500685,0.557587}%
\pgfsetfillcolor{currentfill}%
\pgfsetfillopacity{0.700000}%
\pgfsetlinewidth{0.501875pt}%
\definecolor{currentstroke}{rgb}{1.000000,1.000000,1.000000}%
\pgfsetstrokecolor{currentstroke}%
\pgfsetstrokeopacity{0.500000}%
\pgfsetdash{}{0pt}%
\pgfpathmoveto{\pgfqpoint{1.986485in}{2.224099in}}%
\pgfpathlineto{\pgfqpoint{1.998183in}{2.227605in}}%
\pgfpathlineto{\pgfqpoint{2.009874in}{2.231098in}}%
\pgfpathlineto{\pgfqpoint{2.021561in}{2.234579in}}%
\pgfpathlineto{\pgfqpoint{2.033241in}{2.238049in}}%
\pgfpathlineto{\pgfqpoint{2.044917in}{2.241508in}}%
\pgfpathlineto{\pgfqpoint{2.038985in}{2.250313in}}%
\pgfpathlineto{\pgfqpoint{2.033059in}{2.259093in}}%
\pgfpathlineto{\pgfqpoint{2.027136in}{2.267849in}}%
\pgfpathlineto{\pgfqpoint{2.021218in}{2.276580in}}%
\pgfpathlineto{\pgfqpoint{2.015304in}{2.285286in}}%
\pgfpathlineto{\pgfqpoint{2.003640in}{2.281877in}}%
\pgfpathlineto{\pgfqpoint{1.991970in}{2.278456in}}%
\pgfpathlineto{\pgfqpoint{1.980295in}{2.275024in}}%
\pgfpathlineto{\pgfqpoint{1.968614in}{2.271578in}}%
\pgfpathlineto{\pgfqpoint{1.956928in}{2.268119in}}%
\pgfpathlineto{\pgfqpoint{1.962831in}{2.259368in}}%
\pgfpathlineto{\pgfqpoint{1.968738in}{2.250590in}}%
\pgfpathlineto{\pgfqpoint{1.974649in}{2.241786in}}%
\pgfpathlineto{\pgfqpoint{1.980565in}{2.232955in}}%
\pgfpathclose%
\pgfusepath{stroke,fill}%
\end{pgfscope}%
\begin{pgfscope}%
\pgfpathrectangle{\pgfqpoint{0.887500in}{0.275000in}}{\pgfqpoint{4.225000in}{4.225000in}}%
\pgfusepath{clip}%
\pgfsetbuttcap%
\pgfsetroundjoin%
\definecolor{currentfill}{rgb}{0.180653,0.701402,0.488189}%
\pgfsetfillcolor{currentfill}%
\pgfsetfillopacity{0.700000}%
\pgfsetlinewidth{0.501875pt}%
\definecolor{currentstroke}{rgb}{1.000000,1.000000,1.000000}%
\pgfsetstrokecolor{currentstroke}%
\pgfsetstrokeopacity{0.500000}%
\pgfsetdash{}{0pt}%
\pgfpathmoveto{\pgfqpoint{3.235130in}{2.619098in}}%
\pgfpathlineto{\pgfqpoint{3.246571in}{2.635249in}}%
\pgfpathlineto{\pgfqpoint{3.258005in}{2.649741in}}%
\pgfpathlineto{\pgfqpoint{3.269432in}{2.662806in}}%
\pgfpathlineto{\pgfqpoint{3.280852in}{2.674677in}}%
\pgfpathlineto{\pgfqpoint{3.292266in}{2.685586in}}%
\pgfpathlineto{\pgfqpoint{3.285950in}{2.698775in}}%
\pgfpathlineto{\pgfqpoint{3.279632in}{2.711222in}}%
\pgfpathlineto{\pgfqpoint{3.273314in}{2.723116in}}%
\pgfpathlineto{\pgfqpoint{3.266998in}{2.734648in}}%
\pgfpathlineto{\pgfqpoint{3.260683in}{2.746005in}}%
\pgfpathlineto{\pgfqpoint{3.249274in}{2.733552in}}%
\pgfpathlineto{\pgfqpoint{3.237860in}{2.720206in}}%
\pgfpathlineto{\pgfqpoint{3.226442in}{2.705872in}}%
\pgfpathlineto{\pgfqpoint{3.215020in}{2.690454in}}%
\pgfpathlineto{\pgfqpoint{3.203595in}{2.673857in}}%
\pgfpathlineto{\pgfqpoint{3.209899in}{2.663905in}}%
\pgfpathlineto{\pgfqpoint{3.216206in}{2.653729in}}%
\pgfpathlineto{\pgfqpoint{3.222514in}{2.643054in}}%
\pgfpathlineto{\pgfqpoint{3.228823in}{2.631602in}}%
\pgfpathclose%
\pgfusepath{stroke,fill}%
\end{pgfscope}%
\begin{pgfscope}%
\pgfpathrectangle{\pgfqpoint{0.887500in}{0.275000in}}{\pgfqpoint{4.225000in}{4.225000in}}%
\pgfusepath{clip}%
\pgfsetbuttcap%
\pgfsetroundjoin%
\definecolor{currentfill}{rgb}{0.179019,0.433756,0.557430}%
\pgfsetfillcolor{currentfill}%
\pgfsetfillopacity{0.700000}%
\pgfsetlinewidth{0.501875pt}%
\definecolor{currentstroke}{rgb}{1.000000,1.000000,1.000000}%
\pgfsetstrokecolor{currentstroke}%
\pgfsetstrokeopacity{0.500000}%
\pgfsetdash{}{0pt}%
\pgfpathmoveto{\pgfqpoint{4.240304in}{2.064261in}}%
\pgfpathlineto{\pgfqpoint{4.251459in}{2.068339in}}%
\pgfpathlineto{\pgfqpoint{4.262607in}{2.072355in}}%
\pgfpathlineto{\pgfqpoint{4.273748in}{2.076298in}}%
\pgfpathlineto{\pgfqpoint{4.284880in}{2.080157in}}%
\pgfpathlineto{\pgfqpoint{4.296005in}{2.083920in}}%
\pgfpathlineto{\pgfqpoint{4.289501in}{2.097984in}}%
\pgfpathlineto{\pgfqpoint{4.283002in}{2.112083in}}%
\pgfpathlineto{\pgfqpoint{4.276506in}{2.126225in}}%
\pgfpathlineto{\pgfqpoint{4.270015in}{2.140417in}}%
\pgfpathlineto{\pgfqpoint{4.263528in}{2.154666in}}%
\pgfpathlineto{\pgfqpoint{4.252423in}{2.151457in}}%
\pgfpathlineto{\pgfqpoint{4.241309in}{2.148161in}}%
\pgfpathlineto{\pgfqpoint{4.230188in}{2.144792in}}%
\pgfpathlineto{\pgfqpoint{4.219059in}{2.141364in}}%
\pgfpathlineto{\pgfqpoint{4.207924in}{2.137894in}}%
\pgfpathlineto{\pgfqpoint{4.214394in}{2.123134in}}%
\pgfpathlineto{\pgfqpoint{4.220867in}{2.108370in}}%
\pgfpathlineto{\pgfqpoint{4.227342in}{2.093624in}}%
\pgfpathlineto{\pgfqpoint{4.233821in}{2.078914in}}%
\pgfpathclose%
\pgfusepath{stroke,fill}%
\end{pgfscope}%
\begin{pgfscope}%
\pgfpathrectangle{\pgfqpoint{0.887500in}{0.275000in}}{\pgfqpoint{4.225000in}{4.225000in}}%
\pgfusepath{clip}%
\pgfsetbuttcap%
\pgfsetroundjoin%
\definecolor{currentfill}{rgb}{0.135066,0.544853,0.554029}%
\pgfsetfillcolor{currentfill}%
\pgfsetfillopacity{0.700000}%
\pgfsetlinewidth{0.501875pt}%
\definecolor{currentstroke}{rgb}{1.000000,1.000000,1.000000}%
\pgfsetstrokecolor{currentstroke}%
\pgfsetstrokeopacity{0.500000}%
\pgfsetdash{}{0pt}%
\pgfpathmoveto{\pgfqpoint{3.943434in}{2.302660in}}%
\pgfpathlineto{\pgfqpoint{3.954653in}{2.306422in}}%
\pgfpathlineto{\pgfqpoint{3.965864in}{2.310099in}}%
\pgfpathlineto{\pgfqpoint{3.977069in}{2.313705in}}%
\pgfpathlineto{\pgfqpoint{3.988267in}{2.317256in}}%
\pgfpathlineto{\pgfqpoint{3.999458in}{2.320766in}}%
\pgfpathlineto{\pgfqpoint{3.993021in}{2.335183in}}%
\pgfpathlineto{\pgfqpoint{3.986588in}{2.349627in}}%
\pgfpathlineto{\pgfqpoint{3.980158in}{2.364085in}}%
\pgfpathlineto{\pgfqpoint{3.973730in}{2.378544in}}%
\pgfpathlineto{\pgfqpoint{3.967304in}{2.392990in}}%
\pgfpathlineto{\pgfqpoint{3.956117in}{2.389553in}}%
\pgfpathlineto{\pgfqpoint{3.944923in}{2.386106in}}%
\pgfpathlineto{\pgfqpoint{3.933724in}{2.382643in}}%
\pgfpathlineto{\pgfqpoint{3.922519in}{2.379157in}}%
\pgfpathlineto{\pgfqpoint{3.911307in}{2.375642in}}%
\pgfpathlineto{\pgfqpoint{3.917728in}{2.361051in}}%
\pgfpathlineto{\pgfqpoint{3.924151in}{2.346437in}}%
\pgfpathlineto{\pgfqpoint{3.930575in}{2.331820in}}%
\pgfpathlineto{\pgfqpoint{3.937003in}{2.317221in}}%
\pgfpathclose%
\pgfusepath{stroke,fill}%
\end{pgfscope}%
\begin{pgfscope}%
\pgfpathrectangle{\pgfqpoint{0.887500in}{0.275000in}}{\pgfqpoint{4.225000in}{4.225000in}}%
\pgfusepath{clip}%
\pgfsetbuttcap%
\pgfsetroundjoin%
\definecolor{currentfill}{rgb}{0.150148,0.676631,0.506589}%
\pgfsetfillcolor{currentfill}%
\pgfsetfillopacity{0.700000}%
\pgfsetlinewidth{0.501875pt}%
\definecolor{currentstroke}{rgb}{1.000000,1.000000,1.000000}%
\pgfsetstrokecolor{currentstroke}%
\pgfsetstrokeopacity{0.500000}%
\pgfsetdash{}{0pt}%
\pgfpathmoveto{\pgfqpoint{3.557966in}{2.580507in}}%
\pgfpathlineto{\pgfqpoint{3.569294in}{2.585570in}}%
\pgfpathlineto{\pgfqpoint{3.580613in}{2.590250in}}%
\pgfpathlineto{\pgfqpoint{3.591922in}{2.594613in}}%
\pgfpathlineto{\pgfqpoint{3.603223in}{2.598726in}}%
\pgfpathlineto{\pgfqpoint{3.614517in}{2.602657in}}%
\pgfpathlineto{\pgfqpoint{3.608148in}{2.616788in}}%
\pgfpathlineto{\pgfqpoint{3.601780in}{2.630859in}}%
\pgfpathlineto{\pgfqpoint{3.595415in}{2.644867in}}%
\pgfpathlineto{\pgfqpoint{3.589051in}{2.658808in}}%
\pgfpathlineto{\pgfqpoint{3.582688in}{2.672684in}}%
\pgfpathlineto{\pgfqpoint{3.571399in}{2.668808in}}%
\pgfpathlineto{\pgfqpoint{3.560102in}{2.664757in}}%
\pgfpathlineto{\pgfqpoint{3.548797in}{2.660469in}}%
\pgfpathlineto{\pgfqpoint{3.537483in}{2.655883in}}%
\pgfpathlineto{\pgfqpoint{3.526160in}{2.650935in}}%
\pgfpathlineto{\pgfqpoint{3.532516in}{2.636860in}}%
\pgfpathlineto{\pgfqpoint{3.538875in}{2.622788in}}%
\pgfpathlineto{\pgfqpoint{3.545236in}{2.608718in}}%
\pgfpathlineto{\pgfqpoint{3.551600in}{2.594631in}}%
\pgfpathclose%
\pgfusepath{stroke,fill}%
\end{pgfscope}%
\begin{pgfscope}%
\pgfpathrectangle{\pgfqpoint{0.887500in}{0.275000in}}{\pgfqpoint{4.225000in}{4.225000in}}%
\pgfusepath{clip}%
\pgfsetbuttcap%
\pgfsetroundjoin%
\definecolor{currentfill}{rgb}{0.182256,0.426184,0.557120}%
\pgfsetfillcolor{currentfill}%
\pgfsetfillopacity{0.700000}%
\pgfsetlinewidth{0.501875pt}%
\definecolor{currentstroke}{rgb}{1.000000,1.000000,1.000000}%
\pgfsetstrokecolor{currentstroke}%
\pgfsetstrokeopacity{0.500000}%
\pgfsetdash{}{0pt}%
\pgfpathmoveto{\pgfqpoint{2.720605in}{2.063149in}}%
\pgfpathlineto{\pgfqpoint{2.732123in}{2.066735in}}%
\pgfpathlineto{\pgfqpoint{2.743635in}{2.070239in}}%
\pgfpathlineto{\pgfqpoint{2.755143in}{2.073638in}}%
\pgfpathlineto{\pgfqpoint{2.766646in}{2.076909in}}%
\pgfpathlineto{\pgfqpoint{2.778143in}{2.080070in}}%
\pgfpathlineto{\pgfqpoint{2.771967in}{2.089565in}}%
\pgfpathlineto{\pgfqpoint{2.765794in}{2.099020in}}%
\pgfpathlineto{\pgfqpoint{2.759625in}{2.108435in}}%
\pgfpathlineto{\pgfqpoint{2.753461in}{2.117813in}}%
\pgfpathlineto{\pgfqpoint{2.747301in}{2.127154in}}%
\pgfpathlineto{\pgfqpoint{2.735813in}{2.124058in}}%
\pgfpathlineto{\pgfqpoint{2.724320in}{2.120845in}}%
\pgfpathlineto{\pgfqpoint{2.712822in}{2.117493in}}%
\pgfpathlineto{\pgfqpoint{2.701320in}{2.114030in}}%
\pgfpathlineto{\pgfqpoint{2.689813in}{2.110479in}}%
\pgfpathlineto{\pgfqpoint{2.695963in}{2.101093in}}%
\pgfpathlineto{\pgfqpoint{2.702117in}{2.091667in}}%
\pgfpathlineto{\pgfqpoint{2.708276in}{2.082202in}}%
\pgfpathlineto{\pgfqpoint{2.714438in}{2.072696in}}%
\pgfpathclose%
\pgfusepath{stroke,fill}%
\end{pgfscope}%
\begin{pgfscope}%
\pgfpathrectangle{\pgfqpoint{0.887500in}{0.275000in}}{\pgfqpoint{4.225000in}{4.225000in}}%
\pgfusepath{clip}%
\pgfsetbuttcap%
\pgfsetroundjoin%
\definecolor{currentfill}{rgb}{0.231674,0.318106,0.544834}%
\pgfsetfillcolor{currentfill}%
\pgfsetfillopacity{0.700000}%
\pgfsetlinewidth{0.501875pt}%
\definecolor{currentstroke}{rgb}{1.000000,1.000000,1.000000}%
\pgfsetstrokecolor{currentstroke}%
\pgfsetstrokeopacity{0.500000}%
\pgfsetdash{}{0pt}%
\pgfpathmoveto{\pgfqpoint{4.537368in}{1.831766in}}%
\pgfpathlineto{\pgfqpoint{4.548425in}{1.835124in}}%
\pgfpathlineto{\pgfqpoint{4.559478in}{1.838516in}}%
\pgfpathlineto{\pgfqpoint{4.570527in}{1.841930in}}%
\pgfpathlineto{\pgfqpoint{4.581570in}{1.845354in}}%
\pgfpathlineto{\pgfqpoint{4.592608in}{1.848775in}}%
\pgfpathlineto{\pgfqpoint{4.586070in}{1.863492in}}%
\pgfpathlineto{\pgfqpoint{4.579536in}{1.878219in}}%
\pgfpathlineto{\pgfqpoint{4.573004in}{1.892949in}}%
\pgfpathlineto{\pgfqpoint{4.566474in}{1.907675in}}%
\pgfpathlineto{\pgfqpoint{4.559947in}{1.922392in}}%
\pgfpathlineto{\pgfqpoint{4.548913in}{1.919055in}}%
\pgfpathlineto{\pgfqpoint{4.537874in}{1.915716in}}%
\pgfpathlineto{\pgfqpoint{4.526829in}{1.912378in}}%
\pgfpathlineto{\pgfqpoint{4.515779in}{1.909043in}}%
\pgfpathlineto{\pgfqpoint{4.504724in}{1.905716in}}%
\pgfpathlineto{\pgfqpoint{4.511247in}{1.890898in}}%
\pgfpathlineto{\pgfqpoint{4.517772in}{1.876080in}}%
\pgfpathlineto{\pgfqpoint{4.524300in}{1.861275in}}%
\pgfpathlineto{\pgfqpoint{4.530832in}{1.846499in}}%
\pgfpathclose%
\pgfusepath{stroke,fill}%
\end{pgfscope}%
\begin{pgfscope}%
\pgfpathrectangle{\pgfqpoint{0.887500in}{0.275000in}}{\pgfqpoint{4.225000in}{4.225000in}}%
\pgfusepath{clip}%
\pgfsetbuttcap%
\pgfsetroundjoin%
\definecolor{currentfill}{rgb}{0.202219,0.715272,0.476084}%
\pgfsetfillcolor{currentfill}%
\pgfsetfillopacity{0.700000}%
\pgfsetlinewidth{0.501875pt}%
\definecolor{currentstroke}{rgb}{1.000000,1.000000,1.000000}%
\pgfsetstrokecolor{currentstroke}%
\pgfsetstrokeopacity{0.500000}%
\pgfsetdash{}{0pt}%
\pgfpathmoveto{\pgfqpoint{3.380850in}{2.656810in}}%
\pgfpathlineto{\pgfqpoint{3.392242in}{2.665463in}}%
\pgfpathlineto{\pgfqpoint{3.403627in}{2.673574in}}%
\pgfpathlineto{\pgfqpoint{3.415005in}{2.681141in}}%
\pgfpathlineto{\pgfqpoint{3.426374in}{2.688164in}}%
\pgfpathlineto{\pgfqpoint{3.437735in}{2.694665in}}%
\pgfpathlineto{\pgfqpoint{3.431405in}{2.709511in}}%
\pgfpathlineto{\pgfqpoint{3.425076in}{2.724194in}}%
\pgfpathlineto{\pgfqpoint{3.418748in}{2.738661in}}%
\pgfpathlineto{\pgfqpoint{3.412419in}{2.752855in}}%
\pgfpathlineto{\pgfqpoint{3.406090in}{2.766751in}}%
\pgfpathlineto{\pgfqpoint{3.394740in}{2.760985in}}%
\pgfpathlineto{\pgfqpoint{3.383382in}{2.754641in}}%
\pgfpathlineto{\pgfqpoint{3.372015in}{2.747702in}}%
\pgfpathlineto{\pgfqpoint{3.360640in}{2.740205in}}%
\pgfpathlineto{\pgfqpoint{3.349258in}{2.732192in}}%
\pgfpathlineto{\pgfqpoint{3.355579in}{2.718048in}}%
\pgfpathlineto{\pgfqpoint{3.361899in}{2.703369in}}%
\pgfpathlineto{\pgfqpoint{3.368216in}{2.688201in}}%
\pgfpathlineto{\pgfqpoint{3.374533in}{2.672648in}}%
\pgfpathclose%
\pgfusepath{stroke,fill}%
\end{pgfscope}%
\begin{pgfscope}%
\pgfpathrectangle{\pgfqpoint{0.887500in}{0.275000in}}{\pgfqpoint{4.225000in}{4.225000in}}%
\pgfusepath{clip}%
\pgfsetbuttcap%
\pgfsetroundjoin%
\definecolor{currentfill}{rgb}{0.175707,0.697900,0.491033}%
\pgfsetfillcolor{currentfill}%
\pgfsetfillopacity{0.700000}%
\pgfsetlinewidth{0.501875pt}%
\definecolor{currentstroke}{rgb}{1.000000,1.000000,1.000000}%
\pgfsetstrokecolor{currentstroke}%
\pgfsetstrokeopacity{0.500000}%
\pgfsetdash{}{0pt}%
\pgfpathmoveto{\pgfqpoint{3.469411in}{2.619920in}}%
\pgfpathlineto{\pgfqpoint{3.480778in}{2.626996in}}%
\pgfpathlineto{\pgfqpoint{3.492137in}{2.633625in}}%
\pgfpathlineto{\pgfqpoint{3.503487in}{2.639818in}}%
\pgfpathlineto{\pgfqpoint{3.514828in}{2.645584in}}%
\pgfpathlineto{\pgfqpoint{3.526160in}{2.650935in}}%
\pgfpathlineto{\pgfqpoint{3.519806in}{2.665003in}}%
\pgfpathlineto{\pgfqpoint{3.513455in}{2.679055in}}%
\pgfpathlineto{\pgfqpoint{3.507106in}{2.693080in}}%
\pgfpathlineto{\pgfqpoint{3.500759in}{2.707069in}}%
\pgfpathlineto{\pgfqpoint{3.494413in}{2.721012in}}%
\pgfpathlineto{\pgfqpoint{3.483094in}{2.716403in}}%
\pgfpathlineto{\pgfqpoint{3.471766in}{2.711511in}}%
\pgfpathlineto{\pgfqpoint{3.460431in}{2.706290in}}%
\pgfpathlineto{\pgfqpoint{3.449087in}{2.700691in}}%
\pgfpathlineto{\pgfqpoint{3.437735in}{2.694665in}}%
\pgfpathlineto{\pgfqpoint{3.444065in}{2.679711in}}%
\pgfpathlineto{\pgfqpoint{3.450398in}{2.664705in}}%
\pgfpathlineto{\pgfqpoint{3.456733in}{2.649702in}}%
\pgfpathlineto{\pgfqpoint{3.463070in}{2.634755in}}%
\pgfpathclose%
\pgfusepath{stroke,fill}%
\end{pgfscope}%
\begin{pgfscope}%
\pgfpathrectangle{\pgfqpoint{0.887500in}{0.275000in}}{\pgfqpoint{4.225000in}{4.225000in}}%
\pgfusepath{clip}%
\pgfsetbuttcap%
\pgfsetroundjoin%
\definecolor{currentfill}{rgb}{0.168126,0.459988,0.558082}%
\pgfsetfillcolor{currentfill}%
\pgfsetfillopacity{0.700000}%
\pgfsetlinewidth{0.501875pt}%
\definecolor{currentstroke}{rgb}{1.000000,1.000000,1.000000}%
\pgfsetstrokecolor{currentstroke}%
\pgfsetstrokeopacity{0.500000}%
\pgfsetdash{}{0pt}%
\pgfpathmoveto{\pgfqpoint{2.397628in}{2.132274in}}%
\pgfpathlineto{\pgfqpoint{2.409227in}{2.135834in}}%
\pgfpathlineto{\pgfqpoint{2.420821in}{2.139374in}}%
\pgfpathlineto{\pgfqpoint{2.432409in}{2.142890in}}%
\pgfpathlineto{\pgfqpoint{2.443992in}{2.146384in}}%
\pgfpathlineto{\pgfqpoint{2.455570in}{2.149857in}}%
\pgfpathlineto{\pgfqpoint{2.449497in}{2.158976in}}%
\pgfpathlineto{\pgfqpoint{2.443430in}{2.168064in}}%
\pgfpathlineto{\pgfqpoint{2.437366in}{2.177120in}}%
\pgfpathlineto{\pgfqpoint{2.431307in}{2.186146in}}%
\pgfpathlineto{\pgfqpoint{2.425252in}{2.195143in}}%
\pgfpathlineto{\pgfqpoint{2.413685in}{2.191717in}}%
\pgfpathlineto{\pgfqpoint{2.402113in}{2.188271in}}%
\pgfpathlineto{\pgfqpoint{2.390535in}{2.184804in}}%
\pgfpathlineto{\pgfqpoint{2.378952in}{2.181314in}}%
\pgfpathlineto{\pgfqpoint{2.367364in}{2.177805in}}%
\pgfpathlineto{\pgfqpoint{2.373408in}{2.168760in}}%
\pgfpathlineto{\pgfqpoint{2.379456in}{2.159685in}}%
\pgfpathlineto{\pgfqpoint{2.385509in}{2.150580in}}%
\pgfpathlineto{\pgfqpoint{2.391566in}{2.141443in}}%
\pgfpathclose%
\pgfusepath{stroke,fill}%
\end{pgfscope}%
\begin{pgfscope}%
\pgfpathrectangle{\pgfqpoint{0.887500in}{0.275000in}}{\pgfqpoint{4.225000in}{4.225000in}}%
\pgfusepath{clip}%
\pgfsetbuttcap%
\pgfsetroundjoin%
\definecolor{currentfill}{rgb}{0.154815,0.493313,0.557840}%
\pgfsetfillcolor{currentfill}%
\pgfsetfillopacity{0.700000}%
\pgfsetlinewidth{0.501875pt}%
\definecolor{currentstroke}{rgb}{1.000000,1.000000,1.000000}%
\pgfsetstrokecolor{currentstroke}%
\pgfsetstrokeopacity{0.500000}%
\pgfsetdash{}{0pt}%
\pgfpathmoveto{\pgfqpoint{2.074639in}{2.197095in}}%
\pgfpathlineto{\pgfqpoint{2.086319in}{2.200596in}}%
\pgfpathlineto{\pgfqpoint{2.097994in}{2.204087in}}%
\pgfpathlineto{\pgfqpoint{2.109663in}{2.207569in}}%
\pgfpathlineto{\pgfqpoint{2.121327in}{2.211043in}}%
\pgfpathlineto{\pgfqpoint{2.132985in}{2.214512in}}%
\pgfpathlineto{\pgfqpoint{2.127021in}{2.223396in}}%
\pgfpathlineto{\pgfqpoint{2.121061in}{2.232253in}}%
\pgfpathlineto{\pgfqpoint{2.115106in}{2.241085in}}%
\pgfpathlineto{\pgfqpoint{2.109155in}{2.249890in}}%
\pgfpathlineto{\pgfqpoint{2.103209in}{2.258671in}}%
\pgfpathlineto{\pgfqpoint{2.091562in}{2.255251in}}%
\pgfpathlineto{\pgfqpoint{2.079909in}{2.251827in}}%
\pgfpathlineto{\pgfqpoint{2.068250in}{2.248396in}}%
\pgfpathlineto{\pgfqpoint{2.056586in}{2.244957in}}%
\pgfpathlineto{\pgfqpoint{2.044917in}{2.241508in}}%
\pgfpathlineto{\pgfqpoint{2.050852in}{2.232677in}}%
\pgfpathlineto{\pgfqpoint{2.056792in}{2.223820in}}%
\pgfpathlineto{\pgfqpoint{2.062737in}{2.214938in}}%
\pgfpathlineto{\pgfqpoint{2.068685in}{2.206030in}}%
\pgfpathclose%
\pgfusepath{stroke,fill}%
\end{pgfscope}%
\begin{pgfscope}%
\pgfpathrectangle{\pgfqpoint{0.887500in}{0.275000in}}{\pgfqpoint{4.225000in}{4.225000in}}%
\pgfusepath{clip}%
\pgfsetbuttcap%
\pgfsetroundjoin%
\definecolor{currentfill}{rgb}{0.144759,0.519093,0.556572}%
\pgfsetfillcolor{currentfill}%
\pgfsetfillopacity{0.700000}%
\pgfsetlinewidth{0.501875pt}%
\definecolor{currentstroke}{rgb}{1.000000,1.000000,1.000000}%
\pgfsetstrokecolor{currentstroke}%
\pgfsetstrokeopacity{0.500000}%
\pgfsetdash{}{0pt}%
\pgfpathmoveto{\pgfqpoint{1.751677in}{2.259621in}}%
\pgfpathlineto{\pgfqpoint{1.763437in}{2.263102in}}%
\pgfpathlineto{\pgfqpoint{1.775192in}{2.266576in}}%
\pgfpathlineto{\pgfqpoint{1.786942in}{2.270045in}}%
\pgfpathlineto{\pgfqpoint{1.798685in}{2.273508in}}%
\pgfpathlineto{\pgfqpoint{1.810423in}{2.276968in}}%
\pgfpathlineto{\pgfqpoint{1.804570in}{2.285654in}}%
\pgfpathlineto{\pgfqpoint{1.798721in}{2.294314in}}%
\pgfpathlineto{\pgfqpoint{1.792877in}{2.302951in}}%
\pgfpathlineto{\pgfqpoint{1.787037in}{2.311563in}}%
\pgfpathlineto{\pgfqpoint{1.781201in}{2.320151in}}%
\pgfpathlineto{\pgfqpoint{1.769475in}{2.316734in}}%
\pgfpathlineto{\pgfqpoint{1.757743in}{2.313312in}}%
\pgfpathlineto{\pgfqpoint{1.746005in}{2.309887in}}%
\pgfpathlineto{\pgfqpoint{1.734261in}{2.306455in}}%
\pgfpathlineto{\pgfqpoint{1.722512in}{2.303017in}}%
\pgfpathlineto{\pgfqpoint{1.728336in}{2.294389in}}%
\pgfpathlineto{\pgfqpoint{1.734164in}{2.285735in}}%
\pgfpathlineto{\pgfqpoint{1.739997in}{2.277056in}}%
\pgfpathlineto{\pgfqpoint{1.745835in}{2.268351in}}%
\pgfpathclose%
\pgfusepath{stroke,fill}%
\end{pgfscope}%
\begin{pgfscope}%
\pgfpathrectangle{\pgfqpoint{0.887500in}{0.275000in}}{\pgfqpoint{4.225000in}{4.225000in}}%
\pgfusepath{clip}%
\pgfsetbuttcap%
\pgfsetroundjoin%
\definecolor{currentfill}{rgb}{0.166617,0.463708,0.558119}%
\pgfsetfillcolor{currentfill}%
\pgfsetfillopacity{0.700000}%
\pgfsetlinewidth{0.501875pt}%
\definecolor{currentstroke}{rgb}{1.000000,1.000000,1.000000}%
\pgfsetstrokecolor{currentstroke}%
\pgfsetstrokeopacity{0.500000}%
\pgfsetdash{}{0pt}%
\pgfpathmoveto{\pgfqpoint{4.152157in}{2.120227in}}%
\pgfpathlineto{\pgfqpoint{4.163322in}{2.123785in}}%
\pgfpathlineto{\pgfqpoint{4.174481in}{2.127335in}}%
\pgfpathlineto{\pgfqpoint{4.185635in}{2.130873in}}%
\pgfpathlineto{\pgfqpoint{4.196783in}{2.134394in}}%
\pgfpathlineto{\pgfqpoint{4.207924in}{2.137894in}}%
\pgfpathlineto{\pgfqpoint{4.201456in}{2.152630in}}%
\pgfpathlineto{\pgfqpoint{4.194989in}{2.167325in}}%
\pgfpathlineto{\pgfqpoint{4.188523in}{2.181970in}}%
\pgfpathlineto{\pgfqpoint{4.182058in}{2.196572in}}%
\pgfpathlineto{\pgfqpoint{4.175594in}{2.211134in}}%
\pgfpathlineto{\pgfqpoint{4.164461in}{2.207880in}}%
\pgfpathlineto{\pgfqpoint{4.153322in}{2.204611in}}%
\pgfpathlineto{\pgfqpoint{4.142177in}{2.201329in}}%
\pgfpathlineto{\pgfqpoint{4.131026in}{2.198040in}}%
\pgfpathlineto{\pgfqpoint{4.119869in}{2.194746in}}%
\pgfpathlineto{\pgfqpoint{4.126330in}{2.180144in}}%
\pgfpathlineto{\pgfqpoint{4.132790in}{2.165409in}}%
\pgfpathlineto{\pgfqpoint{4.139248in}{2.150521in}}%
\pgfpathlineto{\pgfqpoint{4.145704in}{2.135457in}}%
\pgfpathclose%
\pgfusepath{stroke,fill}%
\end{pgfscope}%
\begin{pgfscope}%
\pgfpathrectangle{\pgfqpoint{0.887500in}{0.275000in}}{\pgfqpoint{4.225000in}{4.225000in}}%
\pgfusepath{clip}%
\pgfsetbuttcap%
\pgfsetroundjoin%
\definecolor{currentfill}{rgb}{0.187231,0.414746,0.556547}%
\pgfsetfillcolor{currentfill}%
\pgfsetfillopacity{0.700000}%
\pgfsetlinewidth{0.501875pt}%
\definecolor{currentstroke}{rgb}{1.000000,1.000000,1.000000}%
\pgfsetstrokecolor{currentstroke}%
\pgfsetstrokeopacity{0.500000}%
\pgfsetdash{}{0pt}%
\pgfpathmoveto{\pgfqpoint{2.809088in}{2.031959in}}%
\pgfpathlineto{\pgfqpoint{2.820590in}{2.035150in}}%
\pgfpathlineto{\pgfqpoint{2.832085in}{2.038381in}}%
\pgfpathlineto{\pgfqpoint{2.843574in}{2.041726in}}%
\pgfpathlineto{\pgfqpoint{2.855056in}{2.045259in}}%
\pgfpathlineto{\pgfqpoint{2.866532in}{2.049056in}}%
\pgfpathlineto{\pgfqpoint{2.860325in}{2.058710in}}%
\pgfpathlineto{\pgfqpoint{2.854122in}{2.068323in}}%
\pgfpathlineto{\pgfqpoint{2.847923in}{2.077895in}}%
\pgfpathlineto{\pgfqpoint{2.841728in}{2.087425in}}%
\pgfpathlineto{\pgfqpoint{2.835538in}{2.096915in}}%
\pgfpathlineto{\pgfqpoint{2.824073in}{2.093146in}}%
\pgfpathlineto{\pgfqpoint{2.812600in}{2.089655in}}%
\pgfpathlineto{\pgfqpoint{2.801121in}{2.086365in}}%
\pgfpathlineto{\pgfqpoint{2.789635in}{2.083196in}}%
\pgfpathlineto{\pgfqpoint{2.778143in}{2.080070in}}%
\pgfpathlineto{\pgfqpoint{2.784324in}{2.070534in}}%
\pgfpathlineto{\pgfqpoint{2.790509in}{2.060955in}}%
\pgfpathlineto{\pgfqpoint{2.796698in}{2.051333in}}%
\pgfpathlineto{\pgfqpoint{2.802891in}{2.041667in}}%
\pgfpathclose%
\pgfusepath{stroke,fill}%
\end{pgfscope}%
\begin{pgfscope}%
\pgfpathrectangle{\pgfqpoint{0.887500in}{0.275000in}}{\pgfqpoint{4.225000in}{4.225000in}}%
\pgfusepath{clip}%
\pgfsetbuttcap%
\pgfsetroundjoin%
\definecolor{currentfill}{rgb}{0.125394,0.574318,0.549086}%
\pgfsetfillcolor{currentfill}%
\pgfsetfillopacity{0.700000}%
\pgfsetlinewidth{0.501875pt}%
\definecolor{currentstroke}{rgb}{1.000000,1.000000,1.000000}%
\pgfsetstrokecolor{currentstroke}%
\pgfsetstrokeopacity{0.500000}%
\pgfsetdash{}{0pt}%
\pgfpathmoveto{\pgfqpoint{3.855149in}{2.357070in}}%
\pgfpathlineto{\pgfqpoint{3.866395in}{2.360973in}}%
\pgfpathlineto{\pgfqpoint{3.877634in}{2.364764in}}%
\pgfpathlineto{\pgfqpoint{3.888865in}{2.368462in}}%
\pgfpathlineto{\pgfqpoint{3.900090in}{2.372083in}}%
\pgfpathlineto{\pgfqpoint{3.911307in}{2.375642in}}%
\pgfpathlineto{\pgfqpoint{3.904888in}{2.390191in}}%
\pgfpathlineto{\pgfqpoint{3.898470in}{2.404677in}}%
\pgfpathlineto{\pgfqpoint{3.892053in}{2.419081in}}%
\pgfpathlineto{\pgfqpoint{3.885636in}{2.433401in}}%
\pgfpathlineto{\pgfqpoint{3.879221in}{2.447646in}}%
\pgfpathlineto{\pgfqpoint{3.868008in}{2.444218in}}%
\pgfpathlineto{\pgfqpoint{3.856789in}{2.440752in}}%
\pgfpathlineto{\pgfqpoint{3.845563in}{2.437230in}}%
\pgfpathlineto{\pgfqpoint{3.834330in}{2.433637in}}%
\pgfpathlineto{\pgfqpoint{3.823090in}{2.429955in}}%
\pgfpathlineto{\pgfqpoint{3.829502in}{2.415641in}}%
\pgfpathlineto{\pgfqpoint{3.835914in}{2.401202in}}%
\pgfpathlineto{\pgfqpoint{3.842326in}{2.386622in}}%
\pgfpathlineto{\pgfqpoint{3.848737in}{2.371903in}}%
\pgfpathclose%
\pgfusepath{stroke,fill}%
\end{pgfscope}%
\begin{pgfscope}%
\pgfpathrectangle{\pgfqpoint{0.887500in}{0.275000in}}{\pgfqpoint{4.225000in}{4.225000in}}%
\pgfusepath{clip}%
\pgfsetbuttcap%
\pgfsetroundjoin%
\definecolor{currentfill}{rgb}{0.216210,0.351535,0.550627}%
\pgfsetfillcolor{currentfill}%
\pgfsetfillopacity{0.700000}%
\pgfsetlinewidth{0.501875pt}%
\definecolor{currentstroke}{rgb}{1.000000,1.000000,1.000000}%
\pgfsetstrokecolor{currentstroke}%
\pgfsetstrokeopacity{0.500000}%
\pgfsetdash{}{0pt}%
\pgfpathmoveto{\pgfqpoint{4.449370in}{1.889256in}}%
\pgfpathlineto{\pgfqpoint{4.460451in}{1.892530in}}%
\pgfpathlineto{\pgfqpoint{4.471527in}{1.895809in}}%
\pgfpathlineto{\pgfqpoint{4.482598in}{1.899098in}}%
\pgfpathlineto{\pgfqpoint{4.493663in}{1.902400in}}%
\pgfpathlineto{\pgfqpoint{4.504724in}{1.905716in}}%
\pgfpathlineto{\pgfqpoint{4.498203in}{1.920520in}}%
\pgfpathlineto{\pgfqpoint{4.491684in}{1.935294in}}%
\pgfpathlineto{\pgfqpoint{4.485165in}{1.950024in}}%
\pgfpathlineto{\pgfqpoint{4.478647in}{1.964697in}}%
\pgfpathlineto{\pgfqpoint{4.472129in}{1.979297in}}%
\pgfpathlineto{\pgfqpoint{4.461069in}{1.975957in}}%
\pgfpathlineto{\pgfqpoint{4.450001in}{1.972573in}}%
\pgfpathlineto{\pgfqpoint{4.438926in}{1.969140in}}%
\pgfpathlineto{\pgfqpoint{4.427844in}{1.965658in}}%
\pgfpathlineto{\pgfqpoint{4.416755in}{1.962125in}}%
\pgfpathlineto{\pgfqpoint{4.423277in}{1.947663in}}%
\pgfpathlineto{\pgfqpoint{4.429799in}{1.933134in}}%
\pgfpathlineto{\pgfqpoint{4.436322in}{1.918550in}}%
\pgfpathlineto{\pgfqpoint{4.442845in}{1.903920in}}%
\pgfpathclose%
\pgfusepath{stroke,fill}%
\end{pgfscope}%
\begin{pgfscope}%
\pgfpathrectangle{\pgfqpoint{0.887500in}{0.275000in}}{\pgfqpoint{4.225000in}{4.225000in}}%
\pgfusepath{clip}%
\pgfsetbuttcap%
\pgfsetroundjoin%
\definecolor{currentfill}{rgb}{0.122312,0.633153,0.530398}%
\pgfsetfillcolor{currentfill}%
\pgfsetfillopacity{0.700000}%
\pgfsetlinewidth{0.501875pt}%
\definecolor{currentstroke}{rgb}{1.000000,1.000000,1.000000}%
\pgfsetstrokecolor{currentstroke}%
\pgfsetstrokeopacity{0.500000}%
\pgfsetdash{}{0pt}%
\pgfpathmoveto{\pgfqpoint{3.209327in}{2.442875in}}%
\pgfpathlineto{\pgfqpoint{3.220781in}{2.462818in}}%
\pgfpathlineto{\pgfqpoint{3.232237in}{2.482308in}}%
\pgfpathlineto{\pgfqpoint{3.243693in}{2.501151in}}%
\pgfpathlineto{\pgfqpoint{3.255149in}{2.519154in}}%
\pgfpathlineto{\pgfqpoint{3.266602in}{2.536121in}}%
\pgfpathlineto{\pgfqpoint{3.260316in}{2.555031in}}%
\pgfpathlineto{\pgfqpoint{3.254027in}{2.573041in}}%
\pgfpathlineto{\pgfqpoint{3.247733in}{2.589876in}}%
\pgfpathlineto{\pgfqpoint{3.241434in}{2.605266in}}%
\pgfpathlineto{\pgfqpoint{3.235130in}{2.619098in}}%
\pgfpathlineto{\pgfqpoint{3.223683in}{2.601105in}}%
\pgfpathlineto{\pgfqpoint{3.212233in}{2.581448in}}%
\pgfpathlineto{\pgfqpoint{3.200781in}{2.560471in}}%
\pgfpathlineto{\pgfqpoint{3.189330in}{2.538519in}}%
\pgfpathlineto{\pgfqpoint{3.177883in}{2.515936in}}%
\pgfpathlineto{\pgfqpoint{3.184171in}{2.502611in}}%
\pgfpathlineto{\pgfqpoint{3.190460in}{2.488576in}}%
\pgfpathlineto{\pgfqpoint{3.196749in}{2.473875in}}%
\pgfpathlineto{\pgfqpoint{3.203038in}{2.458607in}}%
\pgfpathclose%
\pgfusepath{stroke,fill}%
\end{pgfscope}%
\begin{pgfscope}%
\pgfpathrectangle{\pgfqpoint{0.887500in}{0.275000in}}{\pgfqpoint{4.225000in}{4.225000in}}%
\pgfusepath{clip}%
\pgfsetbuttcap%
\pgfsetroundjoin%
\definecolor{currentfill}{rgb}{0.132444,0.552216,0.553018}%
\pgfsetfillcolor{currentfill}%
\pgfsetfillopacity{0.700000}%
\pgfsetlinewidth{0.501875pt}%
\definecolor{currentstroke}{rgb}{1.000000,1.000000,1.000000}%
\pgfsetstrokecolor{currentstroke}%
\pgfsetstrokeopacity{0.500000}%
\pgfsetdash{}{0pt}%
\pgfpathmoveto{\pgfqpoint{3.183562in}{2.278061in}}%
\pgfpathlineto{\pgfqpoint{3.195007in}{2.295844in}}%
\pgfpathlineto{\pgfqpoint{3.206452in}{2.312850in}}%
\pgfpathlineto{\pgfqpoint{3.217896in}{2.329247in}}%
\pgfpathlineto{\pgfqpoint{3.229341in}{2.345188in}}%
\pgfpathlineto{\pgfqpoint{3.240785in}{2.360789in}}%
\pgfpathlineto{\pgfqpoint{3.234491in}{2.377326in}}%
\pgfpathlineto{\pgfqpoint{3.228198in}{2.393904in}}%
\pgfpathlineto{\pgfqpoint{3.221906in}{2.410422in}}%
\pgfpathlineto{\pgfqpoint{3.215616in}{2.426780in}}%
\pgfpathlineto{\pgfqpoint{3.209327in}{2.442875in}}%
\pgfpathlineto{\pgfqpoint{3.197876in}{2.422673in}}%
\pgfpathlineto{\pgfqpoint{3.186428in}{2.402405in}}%
\pgfpathlineto{\pgfqpoint{3.174985in}{2.382206in}}%
\pgfpathlineto{\pgfqpoint{3.163546in}{2.361928in}}%
\pgfpathlineto{\pgfqpoint{3.152110in}{2.341338in}}%
\pgfpathlineto{\pgfqpoint{3.158396in}{2.329063in}}%
\pgfpathlineto{\pgfqpoint{3.164684in}{2.316658in}}%
\pgfpathlineto{\pgfqpoint{3.170975in}{2.304065in}}%
\pgfpathlineto{\pgfqpoint{3.177268in}{2.291219in}}%
\pgfpathclose%
\pgfusepath{stroke,fill}%
\end{pgfscope}%
\begin{pgfscope}%
\pgfpathrectangle{\pgfqpoint{0.887500in}{0.275000in}}{\pgfqpoint{4.225000in}{4.225000in}}%
\pgfusepath{clip}%
\pgfsetbuttcap%
\pgfsetroundjoin%
\definecolor{currentfill}{rgb}{0.172719,0.448791,0.557885}%
\pgfsetfillcolor{currentfill}%
\pgfsetfillopacity{0.700000}%
\pgfsetlinewidth{0.501875pt}%
\definecolor{currentstroke}{rgb}{1.000000,1.000000,1.000000}%
\pgfsetstrokecolor{currentstroke}%
\pgfsetstrokeopacity{0.500000}%
\pgfsetdash{}{0pt}%
\pgfpathmoveto{\pgfqpoint{2.485994in}{2.103760in}}%
\pgfpathlineto{\pgfqpoint{2.497577in}{2.107268in}}%
\pgfpathlineto{\pgfqpoint{2.509153in}{2.110764in}}%
\pgfpathlineto{\pgfqpoint{2.520725in}{2.114253in}}%
\pgfpathlineto{\pgfqpoint{2.532290in}{2.117741in}}%
\pgfpathlineto{\pgfqpoint{2.543850in}{2.121231in}}%
\pgfpathlineto{\pgfqpoint{2.537746in}{2.130472in}}%
\pgfpathlineto{\pgfqpoint{2.531646in}{2.139678in}}%
\pgfpathlineto{\pgfqpoint{2.525551in}{2.148850in}}%
\pgfpathlineto{\pgfqpoint{2.519460in}{2.157990in}}%
\pgfpathlineto{\pgfqpoint{2.513373in}{2.167096in}}%
\pgfpathlineto{\pgfqpoint{2.501824in}{2.163650in}}%
\pgfpathlineto{\pgfqpoint{2.490269in}{2.160207in}}%
\pgfpathlineto{\pgfqpoint{2.478708in}{2.156764in}}%
\pgfpathlineto{\pgfqpoint{2.467142in}{2.153316in}}%
\pgfpathlineto{\pgfqpoint{2.455570in}{2.149857in}}%
\pgfpathlineto{\pgfqpoint{2.461646in}{2.140706in}}%
\pgfpathlineto{\pgfqpoint{2.467726in}{2.131521in}}%
\pgfpathlineto{\pgfqpoint{2.473811in}{2.122302in}}%
\pgfpathlineto{\pgfqpoint{2.479901in}{2.113049in}}%
\pgfpathclose%
\pgfusepath{stroke,fill}%
\end{pgfscope}%
\begin{pgfscope}%
\pgfpathrectangle{\pgfqpoint{0.887500in}{0.275000in}}{\pgfqpoint{4.225000in}{4.225000in}}%
\pgfusepath{clip}%
\pgfsetbuttcap%
\pgfsetroundjoin%
\definecolor{currentfill}{rgb}{0.120092,0.600104,0.542530}%
\pgfsetfillcolor{currentfill}%
\pgfsetfillopacity{0.700000}%
\pgfsetlinewidth{0.501875pt}%
\definecolor{currentstroke}{rgb}{1.000000,1.000000,1.000000}%
\pgfsetstrokecolor{currentstroke}%
\pgfsetstrokeopacity{0.500000}%
\pgfsetdash{}{0pt}%
\pgfpathmoveto{\pgfqpoint{3.766779in}{2.409763in}}%
\pgfpathlineto{\pgfqpoint{3.778056in}{2.414069in}}%
\pgfpathlineto{\pgfqpoint{3.789326in}{2.418234in}}%
\pgfpathlineto{\pgfqpoint{3.800588in}{2.422265in}}%
\pgfpathlineto{\pgfqpoint{3.811843in}{2.426170in}}%
\pgfpathlineto{\pgfqpoint{3.823090in}{2.429955in}}%
\pgfpathlineto{\pgfqpoint{3.816679in}{2.444164in}}%
\pgfpathlineto{\pgfqpoint{3.810268in}{2.458283in}}%
\pgfpathlineto{\pgfqpoint{3.803859in}{2.472332in}}%
\pgfpathlineto{\pgfqpoint{3.797451in}{2.486327in}}%
\pgfpathlineto{\pgfqpoint{3.791045in}{2.500286in}}%
\pgfpathlineto{\pgfqpoint{3.779802in}{2.496549in}}%
\pgfpathlineto{\pgfqpoint{3.768551in}{2.492726in}}%
\pgfpathlineto{\pgfqpoint{3.757294in}{2.488818in}}%
\pgfpathlineto{\pgfqpoint{3.746030in}{2.484820in}}%
\pgfpathlineto{\pgfqpoint{3.734760in}{2.480734in}}%
\pgfpathlineto{\pgfqpoint{3.741161in}{2.466671in}}%
\pgfpathlineto{\pgfqpoint{3.747564in}{2.452563in}}%
\pgfpathlineto{\pgfqpoint{3.753968in}{2.438390in}}%
\pgfpathlineto{\pgfqpoint{3.760373in}{2.424130in}}%
\pgfpathclose%
\pgfusepath{stroke,fill}%
\end{pgfscope}%
\begin{pgfscope}%
\pgfpathrectangle{\pgfqpoint{0.887500in}{0.275000in}}{\pgfqpoint{4.225000in}{4.225000in}}%
\pgfusepath{clip}%
\pgfsetbuttcap%
\pgfsetroundjoin%
\definecolor{currentfill}{rgb}{0.175707,0.697900,0.491033}%
\pgfsetfillcolor{currentfill}%
\pgfsetfillopacity{0.700000}%
\pgfsetlinewidth{0.501875pt}%
\definecolor{currentstroke}{rgb}{1.000000,1.000000,1.000000}%
\pgfsetstrokecolor{currentstroke}%
\pgfsetstrokeopacity{0.500000}%
\pgfsetdash{}{0pt}%
\pgfpathmoveto{\pgfqpoint{3.323794in}{2.605422in}}%
\pgfpathlineto{\pgfqpoint{3.335216in}{2.616821in}}%
\pgfpathlineto{\pgfqpoint{3.346633in}{2.627624in}}%
\pgfpathlineto{\pgfqpoint{3.358045in}{2.637889in}}%
\pgfpathlineto{\pgfqpoint{3.369450in}{2.647618in}}%
\pgfpathlineto{\pgfqpoint{3.380850in}{2.656810in}}%
\pgfpathlineto{\pgfqpoint{3.374533in}{2.672648in}}%
\pgfpathlineto{\pgfqpoint{3.368216in}{2.688201in}}%
\pgfpathlineto{\pgfqpoint{3.361899in}{2.703369in}}%
\pgfpathlineto{\pgfqpoint{3.355579in}{2.718048in}}%
\pgfpathlineto{\pgfqpoint{3.349258in}{2.732192in}}%
\pgfpathlineto{\pgfqpoint{3.337870in}{2.723702in}}%
\pgfpathlineto{\pgfqpoint{3.326476in}{2.714776in}}%
\pgfpathlineto{\pgfqpoint{3.315078in}{2.705456in}}%
\pgfpathlineto{\pgfqpoint{3.303674in}{2.695769in}}%
\pgfpathlineto{\pgfqpoint{3.292266in}{2.685586in}}%
\pgfpathlineto{\pgfqpoint{3.298580in}{2.671467in}}%
\pgfpathlineto{\pgfqpoint{3.304889in}{2.656259in}}%
\pgfpathlineto{\pgfqpoint{3.311194in}{2.640050in}}%
\pgfpathlineto{\pgfqpoint{3.317495in}{2.623038in}}%
\pgfpathclose%
\pgfusepath{stroke,fill}%
\end{pgfscope}%
\begin{pgfscope}%
\pgfpathrectangle{\pgfqpoint{0.887500in}{0.275000in}}{\pgfqpoint{4.225000in}{4.225000in}}%
\pgfusepath{clip}%
\pgfsetbuttcap%
\pgfsetroundjoin%
\definecolor{currentfill}{rgb}{0.159194,0.482237,0.558073}%
\pgfsetfillcolor{currentfill}%
\pgfsetfillopacity{0.700000}%
\pgfsetlinewidth{0.501875pt}%
\definecolor{currentstroke}{rgb}{1.000000,1.000000,1.000000}%
\pgfsetstrokecolor{currentstroke}%
\pgfsetstrokeopacity{0.500000}%
\pgfsetdash{}{0pt}%
\pgfpathmoveto{\pgfqpoint{2.162871in}{2.169675in}}%
\pgfpathlineto{\pgfqpoint{2.174534in}{2.173193in}}%
\pgfpathlineto{\pgfqpoint{2.186191in}{2.176711in}}%
\pgfpathlineto{\pgfqpoint{2.197842in}{2.180233in}}%
\pgfpathlineto{\pgfqpoint{2.209488in}{2.183763in}}%
\pgfpathlineto{\pgfqpoint{2.221127in}{2.187301in}}%
\pgfpathlineto{\pgfqpoint{2.215130in}{2.196274in}}%
\pgfpathlineto{\pgfqpoint{2.209138in}{2.205219in}}%
\pgfpathlineto{\pgfqpoint{2.203150in}{2.214136in}}%
\pgfpathlineto{\pgfqpoint{2.197167in}{2.223027in}}%
\pgfpathlineto{\pgfqpoint{2.191187in}{2.231890in}}%
\pgfpathlineto{\pgfqpoint{2.179559in}{2.228400in}}%
\pgfpathlineto{\pgfqpoint{2.167924in}{2.224920in}}%
\pgfpathlineto{\pgfqpoint{2.156284in}{2.221447in}}%
\pgfpathlineto{\pgfqpoint{2.144637in}{2.217979in}}%
\pgfpathlineto{\pgfqpoint{2.132985in}{2.214512in}}%
\pgfpathlineto{\pgfqpoint{2.138953in}{2.205600in}}%
\pgfpathlineto{\pgfqpoint{2.144926in}{2.196661in}}%
\pgfpathlineto{\pgfqpoint{2.150903in}{2.187694in}}%
\pgfpathlineto{\pgfqpoint{2.156885in}{2.178699in}}%
\pgfpathclose%
\pgfusepath{stroke,fill}%
\end{pgfscope}%
\begin{pgfscope}%
\pgfpathrectangle{\pgfqpoint{0.887500in}{0.275000in}}{\pgfqpoint{4.225000in}{4.225000in}}%
\pgfusepath{clip}%
\pgfsetbuttcap%
\pgfsetroundjoin%
\definecolor{currentfill}{rgb}{0.192357,0.403199,0.555836}%
\pgfsetfillcolor{currentfill}%
\pgfsetfillopacity{0.700000}%
\pgfsetlinewidth{0.501875pt}%
\definecolor{currentstroke}{rgb}{1.000000,1.000000,1.000000}%
\pgfsetstrokecolor{currentstroke}%
\pgfsetstrokeopacity{0.500000}%
\pgfsetdash{}{0pt}%
\pgfpathmoveto{\pgfqpoint{2.897625in}{2.000134in}}%
\pgfpathlineto{\pgfqpoint{2.909104in}{2.004281in}}%
\pgfpathlineto{\pgfqpoint{2.920576in}{2.008655in}}%
\pgfpathlineto{\pgfqpoint{2.932044in}{2.013027in}}%
\pgfpathlineto{\pgfqpoint{2.943506in}{2.017163in}}%
\pgfpathlineto{\pgfqpoint{2.954965in}{2.020828in}}%
\pgfpathlineto{\pgfqpoint{2.948728in}{2.030694in}}%
\pgfpathlineto{\pgfqpoint{2.942495in}{2.040508in}}%
\pgfpathlineto{\pgfqpoint{2.936266in}{2.050271in}}%
\pgfpathlineto{\pgfqpoint{2.930041in}{2.059988in}}%
\pgfpathlineto{\pgfqpoint{2.923820in}{2.069661in}}%
\pgfpathlineto{\pgfqpoint{2.912371in}{2.066032in}}%
\pgfpathlineto{\pgfqpoint{2.900918in}{2.061912in}}%
\pgfpathlineto{\pgfqpoint{2.889462in}{2.057547in}}%
\pgfpathlineto{\pgfqpoint{2.878000in}{2.053181in}}%
\pgfpathlineto{\pgfqpoint{2.866532in}{2.049056in}}%
\pgfpathlineto{\pgfqpoint{2.872742in}{2.039358in}}%
\pgfpathlineto{\pgfqpoint{2.878957in}{2.029618in}}%
\pgfpathlineto{\pgfqpoint{2.885176in}{2.019834in}}%
\pgfpathlineto{\pgfqpoint{2.891399in}{2.010006in}}%
\pgfpathclose%
\pgfusepath{stroke,fill}%
\end{pgfscope}%
\begin{pgfscope}%
\pgfpathrectangle{\pgfqpoint{0.887500in}{0.275000in}}{\pgfqpoint{4.225000in}{4.225000in}}%
\pgfusepath{clip}%
\pgfsetbuttcap%
\pgfsetroundjoin%
\definecolor{currentfill}{rgb}{0.154815,0.493313,0.557840}%
\pgfsetfillcolor{currentfill}%
\pgfsetfillopacity{0.700000}%
\pgfsetlinewidth{0.501875pt}%
\definecolor{currentstroke}{rgb}{1.000000,1.000000,1.000000}%
\pgfsetstrokecolor{currentstroke}%
\pgfsetstrokeopacity{0.500000}%
\pgfsetdash{}{0pt}%
\pgfpathmoveto{\pgfqpoint{4.064001in}{2.178228in}}%
\pgfpathlineto{\pgfqpoint{4.075187in}{2.181577in}}%
\pgfpathlineto{\pgfqpoint{4.086366in}{2.184879in}}%
\pgfpathlineto{\pgfqpoint{4.097539in}{2.188163in}}%
\pgfpathlineto{\pgfqpoint{4.108707in}{2.191453in}}%
\pgfpathlineto{\pgfqpoint{4.119869in}{2.194746in}}%
\pgfpathlineto{\pgfqpoint{4.113408in}{2.209238in}}%
\pgfpathlineto{\pgfqpoint{4.106946in}{2.223642in}}%
\pgfpathlineto{\pgfqpoint{4.100486in}{2.237980in}}%
\pgfpathlineto{\pgfqpoint{4.094027in}{2.252274in}}%
\pgfpathlineto{\pgfqpoint{4.087570in}{2.266546in}}%
\pgfpathlineto{\pgfqpoint{4.076406in}{2.263134in}}%
\pgfpathlineto{\pgfqpoint{4.065237in}{2.259731in}}%
\pgfpathlineto{\pgfqpoint{4.054063in}{2.256339in}}%
\pgfpathlineto{\pgfqpoint{4.042884in}{2.252941in}}%
\pgfpathlineto{\pgfqpoint{4.031698in}{2.249513in}}%
\pgfpathlineto{\pgfqpoint{4.038157in}{2.235393in}}%
\pgfpathlineto{\pgfqpoint{4.044617in}{2.221244in}}%
\pgfpathlineto{\pgfqpoint{4.051079in}{2.207026in}}%
\pgfpathlineto{\pgfqpoint{4.057541in}{2.192700in}}%
\pgfpathclose%
\pgfusepath{stroke,fill}%
\end{pgfscope}%
\begin{pgfscope}%
\pgfpathrectangle{\pgfqpoint{0.887500in}{0.275000in}}{\pgfqpoint{4.225000in}{4.225000in}}%
\pgfusepath{clip}%
\pgfsetbuttcap%
\pgfsetroundjoin%
\definecolor{currentfill}{rgb}{0.147607,0.511733,0.557049}%
\pgfsetfillcolor{currentfill}%
\pgfsetfillopacity{0.700000}%
\pgfsetlinewidth{0.501875pt}%
\definecolor{currentstroke}{rgb}{1.000000,1.000000,1.000000}%
\pgfsetstrokecolor{currentstroke}%
\pgfsetstrokeopacity{0.500000}%
\pgfsetdash{}{0pt}%
\pgfpathmoveto{\pgfqpoint{1.839757in}{2.233162in}}%
\pgfpathlineto{\pgfqpoint{1.851500in}{2.236665in}}%
\pgfpathlineto{\pgfqpoint{1.863237in}{2.240166in}}%
\pgfpathlineto{\pgfqpoint{1.874969in}{2.243667in}}%
\pgfpathlineto{\pgfqpoint{1.886694in}{2.247169in}}%
\pgfpathlineto{\pgfqpoint{1.898414in}{2.250672in}}%
\pgfpathlineto{\pgfqpoint{1.892527in}{2.259440in}}%
\pgfpathlineto{\pgfqpoint{1.886645in}{2.268182in}}%
\pgfpathlineto{\pgfqpoint{1.880767in}{2.276898in}}%
\pgfpathlineto{\pgfqpoint{1.874894in}{2.285589in}}%
\pgfpathlineto{\pgfqpoint{1.869025in}{2.294255in}}%
\pgfpathlineto{\pgfqpoint{1.857316in}{2.290796in}}%
\pgfpathlineto{\pgfqpoint{1.845602in}{2.287338in}}%
\pgfpathlineto{\pgfqpoint{1.833881in}{2.283882in}}%
\pgfpathlineto{\pgfqpoint{1.822155in}{2.280426in}}%
\pgfpathlineto{\pgfqpoint{1.810423in}{2.276968in}}%
\pgfpathlineto{\pgfqpoint{1.816281in}{2.268258in}}%
\pgfpathlineto{\pgfqpoint{1.822143in}{2.259522in}}%
\pgfpathlineto{\pgfqpoint{1.828010in}{2.250761in}}%
\pgfpathlineto{\pgfqpoint{1.833881in}{2.241974in}}%
\pgfpathclose%
\pgfusepath{stroke,fill}%
\end{pgfscope}%
\begin{pgfscope}%
\pgfpathrectangle{\pgfqpoint{0.887500in}{0.275000in}}{\pgfqpoint{4.225000in}{4.225000in}}%
\pgfusepath{clip}%
\pgfsetbuttcap%
\pgfsetroundjoin%
\definecolor{currentfill}{rgb}{0.203063,0.379716,0.553925}%
\pgfsetfillcolor{currentfill}%
\pgfsetfillopacity{0.700000}%
\pgfsetlinewidth{0.501875pt}%
\definecolor{currentstroke}{rgb}{1.000000,1.000000,1.000000}%
\pgfsetstrokecolor{currentstroke}%
\pgfsetstrokeopacity{0.500000}%
\pgfsetdash{}{0pt}%
\pgfpathmoveto{\pgfqpoint{4.361202in}{1.943693in}}%
\pgfpathlineto{\pgfqpoint{4.372327in}{1.947485in}}%
\pgfpathlineto{\pgfqpoint{4.383445in}{1.951222in}}%
\pgfpathlineto{\pgfqpoint{4.394555in}{1.954908in}}%
\pgfpathlineto{\pgfqpoint{4.405659in}{1.958542in}}%
\pgfpathlineto{\pgfqpoint{4.416755in}{1.962125in}}%
\pgfpathlineto{\pgfqpoint{4.410233in}{1.976510in}}%
\pgfpathlineto{\pgfqpoint{4.403711in}{1.990807in}}%
\pgfpathlineto{\pgfqpoint{4.397188in}{2.005006in}}%
\pgfpathlineto{\pgfqpoint{4.390664in}{2.019096in}}%
\pgfpathlineto{\pgfqpoint{4.384139in}{2.033067in}}%
\pgfpathlineto{\pgfqpoint{4.373041in}{2.029387in}}%
\pgfpathlineto{\pgfqpoint{4.361936in}{2.025636in}}%
\pgfpathlineto{\pgfqpoint{4.350822in}{2.021804in}}%
\pgfpathlineto{\pgfqpoint{4.339701in}{2.017884in}}%
\pgfpathlineto{\pgfqpoint{4.328571in}{2.013871in}}%
\pgfpathlineto{\pgfqpoint{4.335093in}{1.999867in}}%
\pgfpathlineto{\pgfqpoint{4.341617in}{1.985851in}}%
\pgfpathlineto{\pgfqpoint{4.348143in}{1.971820in}}%
\pgfpathlineto{\pgfqpoint{4.354672in}{1.957769in}}%
\pgfpathclose%
\pgfusepath{stroke,fill}%
\end{pgfscope}%
\begin{pgfscope}%
\pgfpathrectangle{\pgfqpoint{0.887500in}{0.275000in}}{\pgfqpoint{4.225000in}{4.225000in}}%
\pgfusepath{clip}%
\pgfsetbuttcap%
\pgfsetroundjoin%
\definecolor{currentfill}{rgb}{0.160665,0.478540,0.558115}%
\pgfsetfillcolor{currentfill}%
\pgfsetfillopacity{0.700000}%
\pgfsetlinewidth{0.501875pt}%
\definecolor{currentstroke}{rgb}{1.000000,1.000000,1.000000}%
\pgfsetstrokecolor{currentstroke}%
\pgfsetstrokeopacity{0.500000}%
\pgfsetdash{}{0pt}%
\pgfpathmoveto{\pgfqpoint{3.157776in}{2.109812in}}%
\pgfpathlineto{\pgfqpoint{3.169231in}{2.131256in}}%
\pgfpathlineto{\pgfqpoint{3.180686in}{2.151570in}}%
\pgfpathlineto{\pgfqpoint{3.192142in}{2.170687in}}%
\pgfpathlineto{\pgfqpoint{3.203597in}{2.188728in}}%
\pgfpathlineto{\pgfqpoint{3.215051in}{2.205810in}}%
\pgfpathlineto{\pgfqpoint{3.208752in}{2.221204in}}%
\pgfpathlineto{\pgfqpoint{3.202454in}{2.236122in}}%
\pgfpathlineto{\pgfqpoint{3.196156in}{2.250563in}}%
\pgfpathlineto{\pgfqpoint{3.189858in}{2.264530in}}%
\pgfpathlineto{\pgfqpoint{3.183562in}{2.278061in}}%
\pgfpathlineto{\pgfqpoint{3.172117in}{2.259339in}}%
\pgfpathlineto{\pgfqpoint{3.160672in}{2.239513in}}%
\pgfpathlineto{\pgfqpoint{3.149228in}{2.218419in}}%
\pgfpathlineto{\pgfqpoint{3.137786in}{2.195894in}}%
\pgfpathlineto{\pgfqpoint{3.126347in}{2.171995in}}%
\pgfpathlineto{\pgfqpoint{3.132628in}{2.160444in}}%
\pgfpathlineto{\pgfqpoint{3.138912in}{2.148468in}}%
\pgfpathlineto{\pgfqpoint{3.145199in}{2.136029in}}%
\pgfpathlineto{\pgfqpoint{3.151487in}{2.123138in}}%
\pgfpathclose%
\pgfusepath{stroke,fill}%
\end{pgfscope}%
\begin{pgfscope}%
\pgfpathrectangle{\pgfqpoint{0.887500in}{0.275000in}}{\pgfqpoint{4.225000in}{4.225000in}}%
\pgfusepath{clip}%
\pgfsetbuttcap%
\pgfsetroundjoin%
\definecolor{currentfill}{rgb}{0.120638,0.625828,0.533488}%
\pgfsetfillcolor{currentfill}%
\pgfsetfillopacity{0.700000}%
\pgfsetlinewidth{0.501875pt}%
\definecolor{currentstroke}{rgb}{1.000000,1.000000,1.000000}%
\pgfsetstrokecolor{currentstroke}%
\pgfsetstrokeopacity{0.500000}%
\pgfsetdash{}{0pt}%
\pgfpathmoveto{\pgfqpoint{3.678307in}{2.458853in}}%
\pgfpathlineto{\pgfqpoint{3.689611in}{2.463451in}}%
\pgfpathlineto{\pgfqpoint{3.700908in}{2.467920in}}%
\pgfpathlineto{\pgfqpoint{3.712199in}{2.472285in}}%
\pgfpathlineto{\pgfqpoint{3.723482in}{2.476556in}}%
\pgfpathlineto{\pgfqpoint{3.734760in}{2.480734in}}%
\pgfpathlineto{\pgfqpoint{3.728361in}{2.494772in}}%
\pgfpathlineto{\pgfqpoint{3.721965in}{2.508808in}}%
\pgfpathlineto{\pgfqpoint{3.715571in}{2.522861in}}%
\pgfpathlineto{\pgfqpoint{3.709181in}{2.536945in}}%
\pgfpathlineto{\pgfqpoint{3.702794in}{2.551054in}}%
\pgfpathlineto{\pgfqpoint{3.691523in}{2.547121in}}%
\pgfpathlineto{\pgfqpoint{3.680248in}{2.543173in}}%
\pgfpathlineto{\pgfqpoint{3.668967in}{2.539220in}}%
\pgfpathlineto{\pgfqpoint{3.657681in}{2.535255in}}%
\pgfpathlineto{\pgfqpoint{3.646388in}{2.531233in}}%
\pgfpathlineto{\pgfqpoint{3.652768in}{2.516820in}}%
\pgfpathlineto{\pgfqpoint{3.659150in}{2.502373in}}%
\pgfpathlineto{\pgfqpoint{3.665533in}{2.487895in}}%
\pgfpathlineto{\pgfqpoint{3.671919in}{2.473388in}}%
\pgfpathclose%
\pgfusepath{stroke,fill}%
\end{pgfscope}%
\begin{pgfscope}%
\pgfpathrectangle{\pgfqpoint{0.887500in}{0.275000in}}{\pgfqpoint{4.225000in}{4.225000in}}%
\pgfusepath{clip}%
\pgfsetbuttcap%
\pgfsetroundjoin%
\definecolor{currentfill}{rgb}{0.199430,0.387607,0.554642}%
\pgfsetfillcolor{currentfill}%
\pgfsetfillopacity{0.700000}%
\pgfsetlinewidth{0.501875pt}%
\definecolor{currentstroke}{rgb}{1.000000,1.000000,1.000000}%
\pgfsetstrokecolor{currentstroke}%
\pgfsetstrokeopacity{0.500000}%
\pgfsetdash{}{0pt}%
\pgfpathmoveto{\pgfqpoint{2.986207in}{1.970617in}}%
\pgfpathlineto{\pgfqpoint{2.997671in}{1.973699in}}%
\pgfpathlineto{\pgfqpoint{3.009129in}{1.975964in}}%
\pgfpathlineto{\pgfqpoint{3.020582in}{1.977255in}}%
\pgfpathlineto{\pgfqpoint{3.032029in}{1.977951in}}%
\pgfpathlineto{\pgfqpoint{3.043468in}{1.978791in}}%
\pgfpathlineto{\pgfqpoint{3.037204in}{1.988499in}}%
\pgfpathlineto{\pgfqpoint{3.030944in}{1.998202in}}%
\pgfpathlineto{\pgfqpoint{3.024688in}{2.007903in}}%
\pgfpathlineto{\pgfqpoint{3.018436in}{2.017603in}}%
\pgfpathlineto{\pgfqpoint{3.012188in}{2.027299in}}%
\pgfpathlineto{\pgfqpoint{3.000756in}{2.026918in}}%
\pgfpathlineto{\pgfqpoint{2.989316in}{2.026704in}}%
\pgfpathlineto{\pgfqpoint{2.977870in}{2.025807in}}%
\pgfpathlineto{\pgfqpoint{2.966420in}{2.023787in}}%
\pgfpathlineto{\pgfqpoint{2.954965in}{2.020828in}}%
\pgfpathlineto{\pgfqpoint{2.961206in}{2.010905in}}%
\pgfpathlineto{\pgfqpoint{2.967450in}{2.000924in}}%
\pgfpathlineto{\pgfqpoint{2.973699in}{1.990882in}}%
\pgfpathlineto{\pgfqpoint{2.979951in}{1.980779in}}%
\pgfpathclose%
\pgfusepath{stroke,fill}%
\end{pgfscope}%
\begin{pgfscope}%
\pgfpathrectangle{\pgfqpoint{0.887500in}{0.275000in}}{\pgfqpoint{4.225000in}{4.225000in}}%
\pgfusepath{clip}%
\pgfsetbuttcap%
\pgfsetroundjoin%
\definecolor{currentfill}{rgb}{0.146616,0.673050,0.508936}%
\pgfsetfillcolor{currentfill}%
\pgfsetfillopacity{0.700000}%
\pgfsetlinewidth{0.501875pt}%
\definecolor{currentstroke}{rgb}{1.000000,1.000000,1.000000}%
\pgfsetstrokecolor{currentstroke}%
\pgfsetstrokeopacity{0.500000}%
\pgfsetdash{}{0pt}%
\pgfpathmoveto{\pgfqpoint{3.266602in}{2.536121in}}%
\pgfpathlineto{\pgfqpoint{3.278050in}{2.551941in}}%
\pgfpathlineto{\pgfqpoint{3.289494in}{2.566685in}}%
\pgfpathlineto{\pgfqpoint{3.300933in}{2.580450in}}%
\pgfpathlineto{\pgfqpoint{3.312366in}{2.593330in}}%
\pgfpathlineto{\pgfqpoint{3.323794in}{2.605422in}}%
\pgfpathlineto{\pgfqpoint{3.317495in}{2.623038in}}%
\pgfpathlineto{\pgfqpoint{3.311194in}{2.640050in}}%
\pgfpathlineto{\pgfqpoint{3.304889in}{2.656259in}}%
\pgfpathlineto{\pgfqpoint{3.298580in}{2.671467in}}%
\pgfpathlineto{\pgfqpoint{3.292266in}{2.685586in}}%
\pgfpathlineto{\pgfqpoint{3.280852in}{2.674677in}}%
\pgfpathlineto{\pgfqpoint{3.269432in}{2.662806in}}%
\pgfpathlineto{\pgfqpoint{3.258005in}{2.649741in}}%
\pgfpathlineto{\pgfqpoint{3.246571in}{2.635249in}}%
\pgfpathlineto{\pgfqpoint{3.235130in}{2.619098in}}%
\pgfpathlineto{\pgfqpoint{3.241434in}{2.605266in}}%
\pgfpathlineto{\pgfqpoint{3.247733in}{2.589876in}}%
\pgfpathlineto{\pgfqpoint{3.254027in}{2.573041in}}%
\pgfpathlineto{\pgfqpoint{3.260316in}{2.555031in}}%
\pgfpathclose%
\pgfusepath{stroke,fill}%
\end{pgfscope}%
\begin{pgfscope}%
\pgfpathrectangle{\pgfqpoint{0.887500in}{0.275000in}}{\pgfqpoint{4.225000in}{4.225000in}}%
\pgfusepath{clip}%
\pgfsetbuttcap%
\pgfsetroundjoin%
\definecolor{currentfill}{rgb}{0.162016,0.687316,0.499129}%
\pgfsetfillcolor{currentfill}%
\pgfsetfillopacity{0.700000}%
\pgfsetlinewidth{0.501875pt}%
\definecolor{currentstroke}{rgb}{1.000000,1.000000,1.000000}%
\pgfsetstrokecolor{currentstroke}%
\pgfsetstrokeopacity{0.500000}%
\pgfsetdash{}{0pt}%
\pgfpathmoveto{\pgfqpoint{3.412455in}{2.576951in}}%
\pgfpathlineto{\pgfqpoint{3.423862in}{2.586745in}}%
\pgfpathlineto{\pgfqpoint{3.435262in}{2.595861in}}%
\pgfpathlineto{\pgfqpoint{3.446653in}{2.604382in}}%
\pgfpathlineto{\pgfqpoint{3.458036in}{2.612386in}}%
\pgfpathlineto{\pgfqpoint{3.469411in}{2.619920in}}%
\pgfpathlineto{\pgfqpoint{3.463070in}{2.634755in}}%
\pgfpathlineto{\pgfqpoint{3.456733in}{2.649702in}}%
\pgfpathlineto{\pgfqpoint{3.450398in}{2.664705in}}%
\pgfpathlineto{\pgfqpoint{3.444065in}{2.679711in}}%
\pgfpathlineto{\pgfqpoint{3.437735in}{2.694665in}}%
\pgfpathlineto{\pgfqpoint{3.426374in}{2.688164in}}%
\pgfpathlineto{\pgfqpoint{3.415005in}{2.681141in}}%
\pgfpathlineto{\pgfqpoint{3.403627in}{2.673574in}}%
\pgfpathlineto{\pgfqpoint{3.392242in}{2.665463in}}%
\pgfpathlineto{\pgfqpoint{3.380850in}{2.656810in}}%
\pgfpathlineto{\pgfqpoint{3.387167in}{2.640792in}}%
\pgfpathlineto{\pgfqpoint{3.393485in}{2.624695in}}%
\pgfpathlineto{\pgfqpoint{3.399805in}{2.608621in}}%
\pgfpathlineto{\pgfqpoint{3.406128in}{2.592672in}}%
\pgfpathclose%
\pgfusepath{stroke,fill}%
\end{pgfscope}%
\begin{pgfscope}%
\pgfpathrectangle{\pgfqpoint{0.887500in}{0.275000in}}{\pgfqpoint{4.225000in}{4.225000in}}%
\pgfusepath{clip}%
\pgfsetbuttcap%
\pgfsetroundjoin%
\definecolor{currentfill}{rgb}{0.177423,0.437527,0.557565}%
\pgfsetfillcolor{currentfill}%
\pgfsetfillopacity{0.700000}%
\pgfsetlinewidth{0.501875pt}%
\definecolor{currentstroke}{rgb}{1.000000,1.000000,1.000000}%
\pgfsetstrokecolor{currentstroke}%
\pgfsetstrokeopacity{0.500000}%
\pgfsetdash{}{0pt}%
\pgfpathmoveto{\pgfqpoint{2.574433in}{2.074482in}}%
\pgfpathlineto{\pgfqpoint{2.585997in}{2.078027in}}%
\pgfpathlineto{\pgfqpoint{2.597556in}{2.081581in}}%
\pgfpathlineto{\pgfqpoint{2.609108in}{2.085146in}}%
\pgfpathlineto{\pgfqpoint{2.620655in}{2.088724in}}%
\pgfpathlineto{\pgfqpoint{2.632196in}{2.092319in}}%
\pgfpathlineto{\pgfqpoint{2.626060in}{2.101699in}}%
\pgfpathlineto{\pgfqpoint{2.619929in}{2.111041in}}%
\pgfpathlineto{\pgfqpoint{2.613802in}{2.120346in}}%
\pgfpathlineto{\pgfqpoint{2.607679in}{2.129615in}}%
\pgfpathlineto{\pgfqpoint{2.601560in}{2.138850in}}%
\pgfpathlineto{\pgfqpoint{2.590030in}{2.135293in}}%
\pgfpathlineto{\pgfqpoint{2.578494in}{2.131756in}}%
\pgfpathlineto{\pgfqpoint{2.566951in}{2.128236in}}%
\pgfpathlineto{\pgfqpoint{2.555404in}{2.124728in}}%
\pgfpathlineto{\pgfqpoint{2.543850in}{2.121231in}}%
\pgfpathlineto{\pgfqpoint{2.549958in}{2.111956in}}%
\pgfpathlineto{\pgfqpoint{2.556070in}{2.102644in}}%
\pgfpathlineto{\pgfqpoint{2.562187in}{2.093295in}}%
\pgfpathlineto{\pgfqpoint{2.568308in}{2.083908in}}%
\pgfpathclose%
\pgfusepath{stroke,fill}%
\end{pgfscope}%
\begin{pgfscope}%
\pgfpathrectangle{\pgfqpoint{0.887500in}{0.275000in}}{\pgfqpoint{4.225000in}{4.225000in}}%
\pgfusepath{clip}%
\pgfsetbuttcap%
\pgfsetroundjoin%
\definecolor{currentfill}{rgb}{0.190631,0.407061,0.556089}%
\pgfsetfillcolor{currentfill}%
\pgfsetfillopacity{0.700000}%
\pgfsetlinewidth{0.501875pt}%
\definecolor{currentstroke}{rgb}{1.000000,1.000000,1.000000}%
\pgfsetstrokecolor{currentstroke}%
\pgfsetstrokeopacity{0.500000}%
\pgfsetdash{}{0pt}%
\pgfpathmoveto{\pgfqpoint{4.272806in}{1.992541in}}%
\pgfpathlineto{\pgfqpoint{4.283974in}{1.996966in}}%
\pgfpathlineto{\pgfqpoint{4.295135in}{2.001315in}}%
\pgfpathlineto{\pgfqpoint{4.306288in}{2.005584in}}%
\pgfpathlineto{\pgfqpoint{4.317433in}{2.009770in}}%
\pgfpathlineto{\pgfqpoint{4.328571in}{2.013871in}}%
\pgfpathlineto{\pgfqpoint{4.322052in}{2.027869in}}%
\pgfpathlineto{\pgfqpoint{4.315536in}{2.041866in}}%
\pgfpathlineto{\pgfqpoint{4.309022in}{2.055868in}}%
\pgfpathlineto{\pgfqpoint{4.302512in}{2.069884in}}%
\pgfpathlineto{\pgfqpoint{4.296005in}{2.083920in}}%
\pgfpathlineto{\pgfqpoint{4.284880in}{2.080157in}}%
\pgfpathlineto{\pgfqpoint{4.273748in}{2.076298in}}%
\pgfpathlineto{\pgfqpoint{4.262607in}{2.072355in}}%
\pgfpathlineto{\pgfqpoint{4.251459in}{2.068339in}}%
\pgfpathlineto{\pgfqpoint{4.240304in}{2.064261in}}%
\pgfpathlineto{\pgfqpoint{4.246792in}{2.049685in}}%
\pgfpathlineto{\pgfqpoint{4.253286in}{2.035206in}}%
\pgfpathlineto{\pgfqpoint{4.259785in}{2.020843in}}%
\pgfpathlineto{\pgfqpoint{4.266292in}{2.006617in}}%
\pgfpathclose%
\pgfusepath{stroke,fill}%
\end{pgfscope}%
\begin{pgfscope}%
\pgfpathrectangle{\pgfqpoint{0.887500in}{0.275000in}}{\pgfqpoint{4.225000in}{4.225000in}}%
\pgfusepath{clip}%
\pgfsetbuttcap%
\pgfsetroundjoin%
\definecolor{currentfill}{rgb}{0.144759,0.519093,0.556572}%
\pgfsetfillcolor{currentfill}%
\pgfsetfillopacity{0.700000}%
\pgfsetlinewidth{0.501875pt}%
\definecolor{currentstroke}{rgb}{1.000000,1.000000,1.000000}%
\pgfsetstrokecolor{currentstroke}%
\pgfsetstrokeopacity{0.500000}%
\pgfsetdash{}{0pt}%
\pgfpathmoveto{\pgfqpoint{3.975654in}{2.231032in}}%
\pgfpathlineto{\pgfqpoint{3.986879in}{2.234989in}}%
\pgfpathlineto{\pgfqpoint{3.998096in}{2.238792in}}%
\pgfpathlineto{\pgfqpoint{4.009305in}{2.242463in}}%
\pgfpathlineto{\pgfqpoint{4.020505in}{2.246029in}}%
\pgfpathlineto{\pgfqpoint{4.031698in}{2.249513in}}%
\pgfpathlineto{\pgfqpoint{4.025242in}{2.263641in}}%
\pgfpathlineto{\pgfqpoint{4.018789in}{2.277818in}}%
\pgfpathlineto{\pgfqpoint{4.012341in}{2.292070in}}%
\pgfpathlineto{\pgfqpoint{4.005898in}{2.306391in}}%
\pgfpathlineto{\pgfqpoint{3.999458in}{2.320766in}}%
\pgfpathlineto{\pgfqpoint{3.988267in}{2.317256in}}%
\pgfpathlineto{\pgfqpoint{3.977069in}{2.313705in}}%
\pgfpathlineto{\pgfqpoint{3.965864in}{2.310099in}}%
\pgfpathlineto{\pgfqpoint{3.954653in}{2.306422in}}%
\pgfpathlineto{\pgfqpoint{3.943434in}{2.302660in}}%
\pgfpathlineto{\pgfqpoint{3.949869in}{2.288155in}}%
\pgfpathlineto{\pgfqpoint{3.956308in}{2.273728in}}%
\pgfpathlineto{\pgfqpoint{3.962752in}{2.259399in}}%
\pgfpathlineto{\pgfqpoint{3.969201in}{2.245178in}}%
\pgfpathclose%
\pgfusepath{stroke,fill}%
\end{pgfscope}%
\begin{pgfscope}%
\pgfpathrectangle{\pgfqpoint{0.887500in}{0.275000in}}{\pgfqpoint{4.225000in}{4.225000in}}%
\pgfusepath{clip}%
\pgfsetbuttcap%
\pgfsetroundjoin%
\definecolor{currentfill}{rgb}{0.246811,0.283237,0.535941}%
\pgfsetfillcolor{currentfill}%
\pgfsetfillopacity{0.700000}%
\pgfsetlinewidth{0.501875pt}%
\definecolor{currentstroke}{rgb}{1.000000,1.000000,1.000000}%
\pgfsetstrokecolor{currentstroke}%
\pgfsetstrokeopacity{0.500000}%
\pgfsetdash{}{0pt}%
\pgfpathmoveto{\pgfqpoint{4.570131in}{1.759246in}}%
\pgfpathlineto{\pgfqpoint{4.581185in}{1.762499in}}%
\pgfpathlineto{\pgfqpoint{4.592234in}{1.765757in}}%
\pgfpathlineto{\pgfqpoint{4.603278in}{1.769017in}}%
\pgfpathlineto{\pgfqpoint{4.614315in}{1.772279in}}%
\pgfpathlineto{\pgfqpoint{4.625348in}{1.775542in}}%
\pgfpathlineto{\pgfqpoint{4.618792in}{1.790120in}}%
\pgfpathlineto{\pgfqpoint{4.612240in}{1.804739in}}%
\pgfpathlineto{\pgfqpoint{4.605693in}{1.819392in}}%
\pgfpathlineto{\pgfqpoint{4.599148in}{1.834072in}}%
\pgfpathlineto{\pgfqpoint{4.592608in}{1.848775in}}%
\pgfpathlineto{\pgfqpoint{4.581570in}{1.845354in}}%
\pgfpathlineto{\pgfqpoint{4.570527in}{1.841930in}}%
\pgfpathlineto{\pgfqpoint{4.559478in}{1.838516in}}%
\pgfpathlineto{\pgfqpoint{4.548425in}{1.835124in}}%
\pgfpathlineto{\pgfqpoint{4.537368in}{1.831766in}}%
\pgfpathlineto{\pgfqpoint{4.543909in}{1.817090in}}%
\pgfpathlineto{\pgfqpoint{4.550455in}{1.802486in}}%
\pgfpathlineto{\pgfqpoint{4.557007in}{1.787968in}}%
\pgfpathlineto{\pgfqpoint{4.563565in}{1.773550in}}%
\pgfpathclose%
\pgfusepath{stroke,fill}%
\end{pgfscope}%
\begin{pgfscope}%
\pgfpathrectangle{\pgfqpoint{0.887500in}{0.275000in}}{\pgfqpoint{4.225000in}{4.225000in}}%
\pgfusepath{clip}%
\pgfsetbuttcap%
\pgfsetroundjoin%
\definecolor{currentfill}{rgb}{0.128087,0.647749,0.523491}%
\pgfsetfillcolor{currentfill}%
\pgfsetfillopacity{0.700000}%
\pgfsetlinewidth{0.501875pt}%
\definecolor{currentstroke}{rgb}{1.000000,1.000000,1.000000}%
\pgfsetstrokecolor{currentstroke}%
\pgfsetstrokeopacity{0.500000}%
\pgfsetdash{}{0pt}%
\pgfpathmoveto{\pgfqpoint{3.589814in}{2.508608in}}%
\pgfpathlineto{\pgfqpoint{3.601146in}{2.513629in}}%
\pgfpathlineto{\pgfqpoint{3.612468in}{2.518354in}}%
\pgfpathlineto{\pgfqpoint{3.623783in}{2.522831in}}%
\pgfpathlineto{\pgfqpoint{3.635089in}{2.527108in}}%
\pgfpathlineto{\pgfqpoint{3.646388in}{2.531233in}}%
\pgfpathlineto{\pgfqpoint{3.640010in}{2.545608in}}%
\pgfpathlineto{\pgfqpoint{3.633634in}{2.559942in}}%
\pgfpathlineto{\pgfqpoint{3.627260in}{2.574230in}}%
\pgfpathlineto{\pgfqpoint{3.620888in}{2.588470in}}%
\pgfpathlineto{\pgfqpoint{3.614517in}{2.602657in}}%
\pgfpathlineto{\pgfqpoint{3.603223in}{2.598726in}}%
\pgfpathlineto{\pgfqpoint{3.591922in}{2.594613in}}%
\pgfpathlineto{\pgfqpoint{3.580613in}{2.590250in}}%
\pgfpathlineto{\pgfqpoint{3.569294in}{2.585570in}}%
\pgfpathlineto{\pgfqpoint{3.557966in}{2.580507in}}%
\pgfpathlineto{\pgfqpoint{3.564333in}{2.566324in}}%
\pgfpathlineto{\pgfqpoint{3.570702in}{2.552064in}}%
\pgfpathlineto{\pgfqpoint{3.577072in}{2.537705in}}%
\pgfpathlineto{\pgfqpoint{3.583443in}{2.523226in}}%
\pgfpathclose%
\pgfusepath{stroke,fill}%
\end{pgfscope}%
\begin{pgfscope}%
\pgfpathrectangle{\pgfqpoint{0.887500in}{0.275000in}}{\pgfqpoint{4.225000in}{4.225000in}}%
\pgfusepath{clip}%
\pgfsetbuttcap%
\pgfsetroundjoin%
\definecolor{currentfill}{rgb}{0.192357,0.403199,0.555836}%
\pgfsetfillcolor{currentfill}%
\pgfsetfillopacity{0.700000}%
\pgfsetlinewidth{0.501875pt}%
\definecolor{currentstroke}{rgb}{1.000000,1.000000,1.000000}%
\pgfsetstrokecolor{currentstroke}%
\pgfsetstrokeopacity{0.500000}%
\pgfsetdash{}{0pt}%
\pgfpathmoveto{\pgfqpoint{3.132015in}{1.958713in}}%
\pgfpathlineto{\pgfqpoint{3.143452in}{1.971405in}}%
\pgfpathlineto{\pgfqpoint{3.154893in}{1.986107in}}%
\pgfpathlineto{\pgfqpoint{3.166340in}{2.002316in}}%
\pgfpathlineto{\pgfqpoint{3.177791in}{2.019527in}}%
\pgfpathlineto{\pgfqpoint{3.189246in}{2.037235in}}%
\pgfpathlineto{\pgfqpoint{3.182950in}{2.052488in}}%
\pgfpathlineto{\pgfqpoint{3.176655in}{2.067389in}}%
\pgfpathlineto{\pgfqpoint{3.170361in}{2.081921in}}%
\pgfpathlineto{\pgfqpoint{3.164068in}{2.096068in}}%
\pgfpathlineto{\pgfqpoint{3.157776in}{2.109812in}}%
\pgfpathlineto{\pgfqpoint{3.146326in}{2.087898in}}%
\pgfpathlineto{\pgfqpoint{3.134881in}{2.066293in}}%
\pgfpathlineto{\pgfqpoint{3.123443in}{2.045777in}}%
\pgfpathlineto{\pgfqpoint{3.112011in}{2.027123in}}%
\pgfpathlineto{\pgfqpoint{3.100586in}{2.011106in}}%
\pgfpathlineto{\pgfqpoint{3.106865in}{2.000779in}}%
\pgfpathlineto{\pgfqpoint{3.113147in}{1.990375in}}%
\pgfpathlineto{\pgfqpoint{3.119433in}{1.979896in}}%
\pgfpathlineto{\pgfqpoint{3.125723in}{1.969342in}}%
\pgfpathclose%
\pgfusepath{stroke,fill}%
\end{pgfscope}%
\begin{pgfscope}%
\pgfpathrectangle{\pgfqpoint{0.887500in}{0.275000in}}{\pgfqpoint{4.225000in}{4.225000in}}%
\pgfusepath{clip}%
\pgfsetbuttcap%
\pgfsetroundjoin%
\definecolor{currentfill}{rgb}{0.120565,0.596422,0.543611}%
\pgfsetfillcolor{currentfill}%
\pgfsetfillopacity{0.700000}%
\pgfsetlinewidth{0.501875pt}%
\definecolor{currentstroke}{rgb}{1.000000,1.000000,1.000000}%
\pgfsetstrokecolor{currentstroke}%
\pgfsetstrokeopacity{0.500000}%
\pgfsetdash{}{0pt}%
\pgfpathmoveto{\pgfqpoint{3.240785in}{2.360789in}}%
\pgfpathlineto{\pgfqpoint{3.252230in}{2.376161in}}%
\pgfpathlineto{\pgfqpoint{3.263675in}{2.391413in}}%
\pgfpathlineto{\pgfqpoint{3.275121in}{2.406655in}}%
\pgfpathlineto{\pgfqpoint{3.286569in}{2.421997in}}%
\pgfpathlineto{\pgfqpoint{3.298020in}{2.437550in}}%
\pgfpathlineto{\pgfqpoint{3.291733in}{2.456897in}}%
\pgfpathlineto{\pgfqpoint{3.285449in}{2.476698in}}%
\pgfpathlineto{\pgfqpoint{3.279167in}{2.496682in}}%
\pgfpathlineto{\pgfqpoint{3.272885in}{2.516580in}}%
\pgfpathlineto{\pgfqpoint{3.266602in}{2.536121in}}%
\pgfpathlineto{\pgfqpoint{3.255149in}{2.519154in}}%
\pgfpathlineto{\pgfqpoint{3.243693in}{2.501151in}}%
\pgfpathlineto{\pgfqpoint{3.232237in}{2.482308in}}%
\pgfpathlineto{\pgfqpoint{3.220781in}{2.462818in}}%
\pgfpathlineto{\pgfqpoint{3.209327in}{2.442875in}}%
\pgfpathlineto{\pgfqpoint{3.215616in}{2.426780in}}%
\pgfpathlineto{\pgfqpoint{3.221906in}{2.410422in}}%
\pgfpathlineto{\pgfqpoint{3.228198in}{2.393904in}}%
\pgfpathlineto{\pgfqpoint{3.234491in}{2.377326in}}%
\pgfpathclose%
\pgfusepath{stroke,fill}%
\end{pgfscope}%
\begin{pgfscope}%
\pgfpathrectangle{\pgfqpoint{0.887500in}{0.275000in}}{\pgfqpoint{4.225000in}{4.225000in}}%
\pgfusepath{clip}%
\pgfsetbuttcap%
\pgfsetroundjoin%
\definecolor{currentfill}{rgb}{0.143303,0.669459,0.511215}%
\pgfsetfillcolor{currentfill}%
\pgfsetfillopacity{0.700000}%
\pgfsetlinewidth{0.501875pt}%
\definecolor{currentstroke}{rgb}{1.000000,1.000000,1.000000}%
\pgfsetstrokecolor{currentstroke}%
\pgfsetstrokeopacity{0.500000}%
\pgfsetdash{}{0pt}%
\pgfpathmoveto{\pgfqpoint{3.501181in}{2.548577in}}%
\pgfpathlineto{\pgfqpoint{3.512556in}{2.555877in}}%
\pgfpathlineto{\pgfqpoint{3.523922in}{2.562710in}}%
\pgfpathlineto{\pgfqpoint{3.535280in}{2.569087in}}%
\pgfpathlineto{\pgfqpoint{3.546627in}{2.575016in}}%
\pgfpathlineto{\pgfqpoint{3.557966in}{2.580507in}}%
\pgfpathlineto{\pgfqpoint{3.551600in}{2.594631in}}%
\pgfpathlineto{\pgfqpoint{3.545236in}{2.608718in}}%
\pgfpathlineto{\pgfqpoint{3.538875in}{2.622788in}}%
\pgfpathlineto{\pgfqpoint{3.532516in}{2.636860in}}%
\pgfpathlineto{\pgfqpoint{3.526160in}{2.650935in}}%
\pgfpathlineto{\pgfqpoint{3.514828in}{2.645584in}}%
\pgfpathlineto{\pgfqpoint{3.503487in}{2.639818in}}%
\pgfpathlineto{\pgfqpoint{3.492137in}{2.633625in}}%
\pgfpathlineto{\pgfqpoint{3.480778in}{2.626996in}}%
\pgfpathlineto{\pgfqpoint{3.469411in}{2.619920in}}%
\pgfpathlineto{\pgfqpoint{3.475756in}{2.605252in}}%
\pgfpathlineto{\pgfqpoint{3.482105in}{2.590804in}}%
\pgfpathlineto{\pgfqpoint{3.488460in}{2.576586in}}%
\pgfpathlineto{\pgfqpoint{3.494818in}{2.562534in}}%
\pgfpathclose%
\pgfusepath{stroke,fill}%
\end{pgfscope}%
\begin{pgfscope}%
\pgfpathrectangle{\pgfqpoint{0.887500in}{0.275000in}}{\pgfqpoint{4.225000in}{4.225000in}}%
\pgfusepath{clip}%
\pgfsetbuttcap%
\pgfsetroundjoin%
\definecolor{currentfill}{rgb}{0.163625,0.471133,0.558148}%
\pgfsetfillcolor{currentfill}%
\pgfsetfillopacity{0.700000}%
\pgfsetlinewidth{0.501875pt}%
\definecolor{currentstroke}{rgb}{1.000000,1.000000,1.000000}%
\pgfsetstrokecolor{currentstroke}%
\pgfsetstrokeopacity{0.500000}%
\pgfsetdash{}{0pt}%
\pgfpathmoveto{\pgfqpoint{2.251176in}{2.142012in}}%
\pgfpathlineto{\pgfqpoint{2.262820in}{2.145609in}}%
\pgfpathlineto{\pgfqpoint{2.274459in}{2.149211in}}%
\pgfpathlineto{\pgfqpoint{2.286091in}{2.152813in}}%
\pgfpathlineto{\pgfqpoint{2.297718in}{2.156414in}}%
\pgfpathlineto{\pgfqpoint{2.309340in}{2.160009in}}%
\pgfpathlineto{\pgfqpoint{2.303310in}{2.169075in}}%
\pgfpathlineto{\pgfqpoint{2.297286in}{2.178112in}}%
\pgfpathlineto{\pgfqpoint{2.291265in}{2.187120in}}%
\pgfpathlineto{\pgfqpoint{2.285249in}{2.196100in}}%
\pgfpathlineto{\pgfqpoint{2.279237in}{2.205052in}}%
\pgfpathlineto{\pgfqpoint{2.267626in}{2.201505in}}%
\pgfpathlineto{\pgfqpoint{2.256010in}{2.197954in}}%
\pgfpathlineto{\pgfqpoint{2.244388in}{2.194400in}}%
\pgfpathlineto{\pgfqpoint{2.232761in}{2.190848in}}%
\pgfpathlineto{\pgfqpoint{2.221127in}{2.187301in}}%
\pgfpathlineto{\pgfqpoint{2.227128in}{2.178301in}}%
\pgfpathlineto{\pgfqpoint{2.233134in}{2.169272in}}%
\pgfpathlineto{\pgfqpoint{2.239143in}{2.160214in}}%
\pgfpathlineto{\pgfqpoint{2.245158in}{2.151127in}}%
\pgfpathclose%
\pgfusepath{stroke,fill}%
\end{pgfscope}%
\begin{pgfscope}%
\pgfpathrectangle{\pgfqpoint{0.887500in}{0.275000in}}{\pgfqpoint{4.225000in}{4.225000in}}%
\pgfusepath{clip}%
\pgfsetbuttcap%
\pgfsetroundjoin%
\definecolor{currentfill}{rgb}{0.151918,0.500685,0.557587}%
\pgfsetfillcolor{currentfill}%
\pgfsetfillopacity{0.700000}%
\pgfsetlinewidth{0.501875pt}%
\definecolor{currentstroke}{rgb}{1.000000,1.000000,1.000000}%
\pgfsetstrokecolor{currentstroke}%
\pgfsetstrokeopacity{0.500000}%
\pgfsetdash{}{0pt}%
\pgfpathmoveto{\pgfqpoint{1.927916in}{2.206432in}}%
\pgfpathlineto{\pgfqpoint{1.939641in}{2.209977in}}%
\pgfpathlineto{\pgfqpoint{1.951361in}{2.213518in}}%
\pgfpathlineto{\pgfqpoint{1.963074in}{2.217053in}}%
\pgfpathlineto{\pgfqpoint{1.974783in}{2.220581in}}%
\pgfpathlineto{\pgfqpoint{1.986485in}{2.224099in}}%
\pgfpathlineto{\pgfqpoint{1.980565in}{2.232955in}}%
\pgfpathlineto{\pgfqpoint{1.974649in}{2.241786in}}%
\pgfpathlineto{\pgfqpoint{1.968738in}{2.250590in}}%
\pgfpathlineto{\pgfqpoint{1.962831in}{2.259368in}}%
\pgfpathlineto{\pgfqpoint{1.956928in}{2.268119in}}%
\pgfpathlineto{\pgfqpoint{1.945237in}{2.264647in}}%
\pgfpathlineto{\pgfqpoint{1.933539in}{2.261163in}}%
\pgfpathlineto{\pgfqpoint{1.921837in}{2.257671in}}%
\pgfpathlineto{\pgfqpoint{1.910128in}{2.254174in}}%
\pgfpathlineto{\pgfqpoint{1.898414in}{2.250672in}}%
\pgfpathlineto{\pgfqpoint{1.904306in}{2.241877in}}%
\pgfpathlineto{\pgfqpoint{1.910201in}{2.233056in}}%
\pgfpathlineto{\pgfqpoint{1.916102in}{2.224208in}}%
\pgfpathlineto{\pgfqpoint{1.922007in}{2.215333in}}%
\pgfpathclose%
\pgfusepath{stroke,fill}%
\end{pgfscope}%
\begin{pgfscope}%
\pgfpathrectangle{\pgfqpoint{0.887500in}{0.275000in}}{\pgfqpoint{4.225000in}{4.225000in}}%
\pgfusepath{clip}%
\pgfsetbuttcap%
\pgfsetroundjoin%
\definecolor{currentfill}{rgb}{0.206756,0.371758,0.553117}%
\pgfsetfillcolor{currentfill}%
\pgfsetfillopacity{0.700000}%
\pgfsetlinewidth{0.501875pt}%
\definecolor{currentstroke}{rgb}{1.000000,1.000000,1.000000}%
\pgfsetstrokecolor{currentstroke}%
\pgfsetstrokeopacity{0.500000}%
\pgfsetdash{}{0pt}%
\pgfpathmoveto{\pgfqpoint{3.074844in}{1.930077in}}%
\pgfpathlineto{\pgfqpoint{3.086284in}{1.932377in}}%
\pgfpathlineto{\pgfqpoint{3.097720in}{1.935861in}}%
\pgfpathlineto{\pgfqpoint{3.109152in}{1.941066in}}%
\pgfpathlineto{\pgfqpoint{3.120583in}{1.948533in}}%
\pgfpathlineto{\pgfqpoint{3.132015in}{1.958713in}}%
\pgfpathlineto{\pgfqpoint{3.125723in}{1.969342in}}%
\pgfpathlineto{\pgfqpoint{3.119433in}{1.979896in}}%
\pgfpathlineto{\pgfqpoint{3.113147in}{1.990375in}}%
\pgfpathlineto{\pgfqpoint{3.106865in}{2.000779in}}%
\pgfpathlineto{\pgfqpoint{3.100586in}{2.011106in}}%
\pgfpathlineto{\pgfqpoint{3.089165in}{1.998497in}}%
\pgfpathlineto{\pgfqpoint{3.077746in}{1.989630in}}%
\pgfpathlineto{\pgfqpoint{3.066325in}{1.983887in}}%
\pgfpathlineto{\pgfqpoint{3.054900in}{1.980522in}}%
\pgfpathlineto{\pgfqpoint{3.043468in}{1.978791in}}%
\pgfpathlineto{\pgfqpoint{3.049735in}{1.969076in}}%
\pgfpathlineto{\pgfqpoint{3.056007in}{1.959349in}}%
\pgfpathlineto{\pgfqpoint{3.062282in}{1.949609in}}%
\pgfpathlineto{\pgfqpoint{3.068561in}{1.939853in}}%
\pgfpathclose%
\pgfusepath{stroke,fill}%
\end{pgfscope}%
\begin{pgfscope}%
\pgfpathrectangle{\pgfqpoint{0.887500in}{0.275000in}}{\pgfqpoint{4.225000in}{4.225000in}}%
\pgfusepath{clip}%
\pgfsetbuttcap%
\pgfsetroundjoin%
\definecolor{currentfill}{rgb}{0.179019,0.433756,0.557430}%
\pgfsetfillcolor{currentfill}%
\pgfsetfillopacity{0.700000}%
\pgfsetlinewidth{0.501875pt}%
\definecolor{currentstroke}{rgb}{1.000000,1.000000,1.000000}%
\pgfsetstrokecolor{currentstroke}%
\pgfsetstrokeopacity{0.500000}%
\pgfsetdash{}{0pt}%
\pgfpathmoveto{\pgfqpoint{4.184436in}{2.043320in}}%
\pgfpathlineto{\pgfqpoint{4.195621in}{2.047547in}}%
\pgfpathlineto{\pgfqpoint{4.206801in}{2.051765in}}%
\pgfpathlineto{\pgfqpoint{4.217975in}{2.055964in}}%
\pgfpathlineto{\pgfqpoint{4.229143in}{2.060133in}}%
\pgfpathlineto{\pgfqpoint{4.240304in}{2.064261in}}%
\pgfpathlineto{\pgfqpoint{4.233821in}{2.078914in}}%
\pgfpathlineto{\pgfqpoint{4.227342in}{2.093624in}}%
\pgfpathlineto{\pgfqpoint{4.220867in}{2.108370in}}%
\pgfpathlineto{\pgfqpoint{4.214394in}{2.123134in}}%
\pgfpathlineto{\pgfqpoint{4.207924in}{2.137894in}}%
\pgfpathlineto{\pgfqpoint{4.196783in}{2.134394in}}%
\pgfpathlineto{\pgfqpoint{4.185635in}{2.130873in}}%
\pgfpathlineto{\pgfqpoint{4.174481in}{2.127335in}}%
\pgfpathlineto{\pgfqpoint{4.163322in}{2.123785in}}%
\pgfpathlineto{\pgfqpoint{4.152157in}{2.120227in}}%
\pgfpathlineto{\pgfqpoint{4.158609in}{2.104881in}}%
\pgfpathlineto{\pgfqpoint{4.165062in}{2.089466in}}%
\pgfpathlineto{\pgfqpoint{4.171516in}{2.074035in}}%
\pgfpathlineto{\pgfqpoint{4.177974in}{2.058636in}}%
\pgfpathclose%
\pgfusepath{stroke,fill}%
\end{pgfscope}%
\begin{pgfscope}%
\pgfpathrectangle{\pgfqpoint{0.887500in}{0.275000in}}{\pgfqpoint{4.225000in}{4.225000in}}%
\pgfusepath{clip}%
\pgfsetbuttcap%
\pgfsetroundjoin%
\definecolor{currentfill}{rgb}{0.135066,0.544853,0.554029}%
\pgfsetfillcolor{currentfill}%
\pgfsetfillopacity{0.700000}%
\pgfsetlinewidth{0.501875pt}%
\definecolor{currentstroke}{rgb}{1.000000,1.000000,1.000000}%
\pgfsetstrokecolor{currentstroke}%
\pgfsetstrokeopacity{0.500000}%
\pgfsetdash{}{0pt}%
\pgfpathmoveto{\pgfqpoint{3.887226in}{2.282234in}}%
\pgfpathlineto{\pgfqpoint{3.898484in}{2.286566in}}%
\pgfpathlineto{\pgfqpoint{3.909733in}{2.290765in}}%
\pgfpathlineto{\pgfqpoint{3.920974in}{2.294841in}}%
\pgfpathlineto{\pgfqpoint{3.932208in}{2.298802in}}%
\pgfpathlineto{\pgfqpoint{3.943434in}{2.302660in}}%
\pgfpathlineto{\pgfqpoint{3.937003in}{2.317221in}}%
\pgfpathlineto{\pgfqpoint{3.930575in}{2.331820in}}%
\pgfpathlineto{\pgfqpoint{3.924151in}{2.346437in}}%
\pgfpathlineto{\pgfqpoint{3.917728in}{2.361051in}}%
\pgfpathlineto{\pgfqpoint{3.911307in}{2.375642in}}%
\pgfpathlineto{\pgfqpoint{3.900090in}{2.372083in}}%
\pgfpathlineto{\pgfqpoint{3.888865in}{2.368462in}}%
\pgfpathlineto{\pgfqpoint{3.877634in}{2.364764in}}%
\pgfpathlineto{\pgfqpoint{3.866395in}{2.360973in}}%
\pgfpathlineto{\pgfqpoint{3.855149in}{2.357070in}}%
\pgfpathlineto{\pgfqpoint{3.861561in}{2.342154in}}%
\pgfpathlineto{\pgfqpoint{3.867974in}{2.327183in}}%
\pgfpathlineto{\pgfqpoint{3.874389in}{2.312187in}}%
\pgfpathlineto{\pgfqpoint{3.880806in}{2.297194in}}%
\pgfpathclose%
\pgfusepath{stroke,fill}%
\end{pgfscope}%
\begin{pgfscope}%
\pgfpathrectangle{\pgfqpoint{0.887500in}{0.275000in}}{\pgfqpoint{4.225000in}{4.225000in}}%
\pgfusepath{clip}%
\pgfsetbuttcap%
\pgfsetroundjoin%
\definecolor{currentfill}{rgb}{0.182256,0.426184,0.557120}%
\pgfsetfillcolor{currentfill}%
\pgfsetfillopacity{0.700000}%
\pgfsetlinewidth{0.501875pt}%
\definecolor{currentstroke}{rgb}{1.000000,1.000000,1.000000}%
\pgfsetstrokecolor{currentstroke}%
\pgfsetstrokeopacity{0.500000}%
\pgfsetdash{}{0pt}%
\pgfpathmoveto{\pgfqpoint{2.662937in}{2.044814in}}%
\pgfpathlineto{\pgfqpoint{2.674482in}{2.048470in}}%
\pgfpathlineto{\pgfqpoint{2.686021in}{2.052143in}}%
\pgfpathlineto{\pgfqpoint{2.697555in}{2.055830in}}%
\pgfpathlineto{\pgfqpoint{2.709083in}{2.059506in}}%
\pgfpathlineto{\pgfqpoint{2.720605in}{2.063149in}}%
\pgfpathlineto{\pgfqpoint{2.714438in}{2.072696in}}%
\pgfpathlineto{\pgfqpoint{2.708276in}{2.082202in}}%
\pgfpathlineto{\pgfqpoint{2.702117in}{2.091667in}}%
\pgfpathlineto{\pgfqpoint{2.695963in}{2.101093in}}%
\pgfpathlineto{\pgfqpoint{2.689813in}{2.110479in}}%
\pgfpathlineto{\pgfqpoint{2.678300in}{2.106869in}}%
\pgfpathlineto{\pgfqpoint{2.666783in}{2.103224in}}%
\pgfpathlineto{\pgfqpoint{2.655259in}{2.099571in}}%
\pgfpathlineto{\pgfqpoint{2.643730in}{2.095934in}}%
\pgfpathlineto{\pgfqpoint{2.632196in}{2.092319in}}%
\pgfpathlineto{\pgfqpoint{2.638335in}{2.082901in}}%
\pgfpathlineto{\pgfqpoint{2.644479in}{2.073442in}}%
\pgfpathlineto{\pgfqpoint{2.650627in}{2.063942in}}%
\pgfpathlineto{\pgfqpoint{2.656780in}{2.054400in}}%
\pgfpathclose%
\pgfusepath{stroke,fill}%
\end{pgfscope}%
\begin{pgfscope}%
\pgfpathrectangle{\pgfqpoint{0.887500in}{0.275000in}}{\pgfqpoint{4.225000in}{4.225000in}}%
\pgfusepath{clip}%
\pgfsetbuttcap%
\pgfsetroundjoin%
\definecolor{currentfill}{rgb}{0.143343,0.522773,0.556295}%
\pgfsetfillcolor{currentfill}%
\pgfsetfillopacity{0.700000}%
\pgfsetlinewidth{0.501875pt}%
\definecolor{currentstroke}{rgb}{1.000000,1.000000,1.000000}%
\pgfsetstrokecolor{currentstroke}%
\pgfsetstrokeopacity{0.500000}%
\pgfsetdash{}{0pt}%
\pgfpathmoveto{\pgfqpoint{3.215051in}{2.205810in}}%
\pgfpathlineto{\pgfqpoint{3.226504in}{2.222055in}}%
\pgfpathlineto{\pgfqpoint{3.237955in}{2.237581in}}%
\pgfpathlineto{\pgfqpoint{3.249404in}{2.252510in}}%
\pgfpathlineto{\pgfqpoint{3.260853in}{2.266976in}}%
\pgfpathlineto{\pgfqpoint{3.272300in}{2.281158in}}%
\pgfpathlineto{\pgfqpoint{3.265991in}{2.296696in}}%
\pgfpathlineto{\pgfqpoint{3.259685in}{2.312364in}}%
\pgfpathlineto{\pgfqpoint{3.253382in}{2.328240in}}%
\pgfpathlineto{\pgfqpoint{3.247082in}{2.344394in}}%
\pgfpathlineto{\pgfqpoint{3.240785in}{2.360789in}}%
\pgfpathlineto{\pgfqpoint{3.229341in}{2.345188in}}%
\pgfpathlineto{\pgfqpoint{3.217896in}{2.329247in}}%
\pgfpathlineto{\pgfqpoint{3.206452in}{2.312850in}}%
\pgfpathlineto{\pgfqpoint{3.195007in}{2.295844in}}%
\pgfpathlineto{\pgfqpoint{3.183562in}{2.278061in}}%
\pgfpathlineto{\pgfqpoint{3.189858in}{2.264530in}}%
\pgfpathlineto{\pgfqpoint{3.196156in}{2.250563in}}%
\pgfpathlineto{\pgfqpoint{3.202454in}{2.236122in}}%
\pgfpathlineto{\pgfqpoint{3.208752in}{2.221204in}}%
\pgfpathclose%
\pgfusepath{stroke,fill}%
\end{pgfscope}%
\begin{pgfscope}%
\pgfpathrectangle{\pgfqpoint{0.887500in}{0.275000in}}{\pgfqpoint{4.225000in}{4.225000in}}%
\pgfusepath{clip}%
\pgfsetbuttcap%
\pgfsetroundjoin%
\definecolor{currentfill}{rgb}{0.140210,0.665859,0.513427}%
\pgfsetfillcolor{currentfill}%
\pgfsetfillopacity{0.700000}%
\pgfsetlinewidth{0.501875pt}%
\definecolor{currentstroke}{rgb}{1.000000,1.000000,1.000000}%
\pgfsetstrokecolor{currentstroke}%
\pgfsetstrokeopacity{0.500000}%
\pgfsetdash{}{0pt}%
\pgfpathmoveto{\pgfqpoint{3.355292in}{2.515219in}}%
\pgfpathlineto{\pgfqpoint{3.366740in}{2.529410in}}%
\pgfpathlineto{\pgfqpoint{3.378181in}{2.542703in}}%
\pgfpathlineto{\pgfqpoint{3.389614in}{2.555011in}}%
\pgfpathlineto{\pgfqpoint{3.401039in}{2.566400in}}%
\pgfpathlineto{\pgfqpoint{3.412455in}{2.576951in}}%
\pgfpathlineto{\pgfqpoint{3.406128in}{2.592672in}}%
\pgfpathlineto{\pgfqpoint{3.399805in}{2.608621in}}%
\pgfpathlineto{\pgfqpoint{3.393485in}{2.624695in}}%
\pgfpathlineto{\pgfqpoint{3.387167in}{2.640792in}}%
\pgfpathlineto{\pgfqpoint{3.380850in}{2.656810in}}%
\pgfpathlineto{\pgfqpoint{3.369450in}{2.647618in}}%
\pgfpathlineto{\pgfqpoint{3.358045in}{2.637889in}}%
\pgfpathlineto{\pgfqpoint{3.346633in}{2.627624in}}%
\pgfpathlineto{\pgfqpoint{3.335216in}{2.616821in}}%
\pgfpathlineto{\pgfqpoint{3.323794in}{2.605422in}}%
\pgfpathlineto{\pgfqpoint{3.330091in}{2.587399in}}%
\pgfpathlineto{\pgfqpoint{3.336388in}{2.569170in}}%
\pgfpathlineto{\pgfqpoint{3.342687in}{2.550932in}}%
\pgfpathlineto{\pgfqpoint{3.348987in}{2.532882in}}%
\pgfpathclose%
\pgfusepath{stroke,fill}%
\end{pgfscope}%
\begin{pgfscope}%
\pgfpathrectangle{\pgfqpoint{0.887500in}{0.275000in}}{\pgfqpoint{4.225000in}{4.225000in}}%
\pgfusepath{clip}%
\pgfsetbuttcap%
\pgfsetroundjoin%
\definecolor{currentfill}{rgb}{0.231674,0.318106,0.544834}%
\pgfsetfillcolor{currentfill}%
\pgfsetfillopacity{0.700000}%
\pgfsetlinewidth{0.501875pt}%
\definecolor{currentstroke}{rgb}{1.000000,1.000000,1.000000}%
\pgfsetstrokecolor{currentstroke}%
\pgfsetstrokeopacity{0.500000}%
\pgfsetdash{}{0pt}%
\pgfpathmoveto{\pgfqpoint{4.482028in}{1.815786in}}%
\pgfpathlineto{\pgfqpoint{4.493104in}{1.818881in}}%
\pgfpathlineto{\pgfqpoint{4.504175in}{1.822018in}}%
\pgfpathlineto{\pgfqpoint{4.515243in}{1.825206in}}%
\pgfpathlineto{\pgfqpoint{4.526307in}{1.828456in}}%
\pgfpathlineto{\pgfqpoint{4.537368in}{1.831766in}}%
\pgfpathlineto{\pgfqpoint{4.530832in}{1.846499in}}%
\pgfpathlineto{\pgfqpoint{4.524300in}{1.861275in}}%
\pgfpathlineto{\pgfqpoint{4.517772in}{1.876080in}}%
\pgfpathlineto{\pgfqpoint{4.511247in}{1.890898in}}%
\pgfpathlineto{\pgfqpoint{4.504724in}{1.905716in}}%
\pgfpathlineto{\pgfqpoint{4.493663in}{1.902400in}}%
\pgfpathlineto{\pgfqpoint{4.482598in}{1.899098in}}%
\pgfpathlineto{\pgfqpoint{4.471527in}{1.895809in}}%
\pgfpathlineto{\pgfqpoint{4.460451in}{1.892530in}}%
\pgfpathlineto{\pgfqpoint{4.449370in}{1.889256in}}%
\pgfpathlineto{\pgfqpoint{4.455897in}{1.874568in}}%
\pgfpathlineto{\pgfqpoint{4.462426in}{1.859866in}}%
\pgfpathlineto{\pgfqpoint{4.468957in}{1.845161in}}%
\pgfpathlineto{\pgfqpoint{4.475491in}{1.830465in}}%
\pgfpathclose%
\pgfusepath{stroke,fill}%
\end{pgfscope}%
\begin{pgfscope}%
\pgfpathrectangle{\pgfqpoint{0.887500in}{0.275000in}}{\pgfqpoint{4.225000in}{4.225000in}}%
\pgfusepath{clip}%
\pgfsetbuttcap%
\pgfsetroundjoin%
\definecolor{currentfill}{rgb}{0.123444,0.636809,0.528763}%
\pgfsetfillcolor{currentfill}%
\pgfsetfillopacity{0.700000}%
\pgfsetlinewidth{0.501875pt}%
\definecolor{currentstroke}{rgb}{1.000000,1.000000,1.000000}%
\pgfsetstrokecolor{currentstroke}%
\pgfsetstrokeopacity{0.500000}%
\pgfsetdash{}{0pt}%
\pgfpathmoveto{\pgfqpoint{3.298020in}{2.437550in}}%
\pgfpathlineto{\pgfqpoint{3.309473in}{2.453336in}}%
\pgfpathlineto{\pgfqpoint{3.320929in}{2.469189in}}%
\pgfpathlineto{\pgfqpoint{3.332384in}{2.484916in}}%
\pgfpathlineto{\pgfqpoint{3.343840in}{2.500323in}}%
\pgfpathlineto{\pgfqpoint{3.355292in}{2.515219in}}%
\pgfpathlineto{\pgfqpoint{3.348987in}{2.532882in}}%
\pgfpathlineto{\pgfqpoint{3.342687in}{2.550932in}}%
\pgfpathlineto{\pgfqpoint{3.336388in}{2.569170in}}%
\pgfpathlineto{\pgfqpoint{3.330091in}{2.587399in}}%
\pgfpathlineto{\pgfqpoint{3.323794in}{2.605422in}}%
\pgfpathlineto{\pgfqpoint{3.312366in}{2.593330in}}%
\pgfpathlineto{\pgfqpoint{3.300933in}{2.580450in}}%
\pgfpathlineto{\pgfqpoint{3.289494in}{2.566685in}}%
\pgfpathlineto{\pgfqpoint{3.278050in}{2.551941in}}%
\pgfpathlineto{\pgfqpoint{3.266602in}{2.536121in}}%
\pgfpathlineto{\pgfqpoint{3.272885in}{2.516580in}}%
\pgfpathlineto{\pgfqpoint{3.279167in}{2.496682in}}%
\pgfpathlineto{\pgfqpoint{3.285449in}{2.476698in}}%
\pgfpathlineto{\pgfqpoint{3.291733in}{2.456897in}}%
\pgfpathclose%
\pgfusepath{stroke,fill}%
\end{pgfscope}%
\begin{pgfscope}%
\pgfpathrectangle{\pgfqpoint{0.887500in}{0.275000in}}{\pgfqpoint{4.225000in}{4.225000in}}%
\pgfusepath{clip}%
\pgfsetbuttcap%
\pgfsetroundjoin%
\definecolor{currentfill}{rgb}{0.168126,0.459988,0.558082}%
\pgfsetfillcolor{currentfill}%
\pgfsetfillopacity{0.700000}%
\pgfsetlinewidth{0.501875pt}%
\definecolor{currentstroke}{rgb}{1.000000,1.000000,1.000000}%
\pgfsetstrokecolor{currentstroke}%
\pgfsetstrokeopacity{0.500000}%
\pgfsetdash{}{0pt}%
\pgfpathmoveto{\pgfqpoint{2.339551in}{2.114226in}}%
\pgfpathlineto{\pgfqpoint{2.351177in}{2.117862in}}%
\pgfpathlineto{\pgfqpoint{2.362798in}{2.121487in}}%
\pgfpathlineto{\pgfqpoint{2.374413in}{2.125099in}}%
\pgfpathlineto{\pgfqpoint{2.386023in}{2.128695in}}%
\pgfpathlineto{\pgfqpoint{2.397628in}{2.132274in}}%
\pgfpathlineto{\pgfqpoint{2.391566in}{2.141443in}}%
\pgfpathlineto{\pgfqpoint{2.385509in}{2.150580in}}%
\pgfpathlineto{\pgfqpoint{2.379456in}{2.159685in}}%
\pgfpathlineto{\pgfqpoint{2.373408in}{2.168760in}}%
\pgfpathlineto{\pgfqpoint{2.367364in}{2.177805in}}%
\pgfpathlineto{\pgfqpoint{2.355770in}{2.174276in}}%
\pgfpathlineto{\pgfqpoint{2.344170in}{2.170731in}}%
\pgfpathlineto{\pgfqpoint{2.332566in}{2.167170in}}%
\pgfpathlineto{\pgfqpoint{2.320955in}{2.163596in}}%
\pgfpathlineto{\pgfqpoint{2.309340in}{2.160009in}}%
\pgfpathlineto{\pgfqpoint{2.315373in}{2.150914in}}%
\pgfpathlineto{\pgfqpoint{2.321411in}{2.141789in}}%
\pgfpathlineto{\pgfqpoint{2.327453in}{2.132633in}}%
\pgfpathlineto{\pgfqpoint{2.333500in}{2.123445in}}%
\pgfpathclose%
\pgfusepath{stroke,fill}%
\end{pgfscope}%
\begin{pgfscope}%
\pgfpathrectangle{\pgfqpoint{0.887500in}{0.275000in}}{\pgfqpoint{4.225000in}{4.225000in}}%
\pgfusepath{clip}%
\pgfsetbuttcap%
\pgfsetroundjoin%
\definecolor{currentfill}{rgb}{0.154815,0.493313,0.557840}%
\pgfsetfillcolor{currentfill}%
\pgfsetfillopacity{0.700000}%
\pgfsetlinewidth{0.501875pt}%
\definecolor{currentstroke}{rgb}{1.000000,1.000000,1.000000}%
\pgfsetstrokecolor{currentstroke}%
\pgfsetstrokeopacity{0.500000}%
\pgfsetdash{}{0pt}%
\pgfpathmoveto{\pgfqpoint{2.016153in}{2.179425in}}%
\pgfpathlineto{\pgfqpoint{2.027861in}{2.182982in}}%
\pgfpathlineto{\pgfqpoint{2.039564in}{2.186527in}}%
\pgfpathlineto{\pgfqpoint{2.051261in}{2.190060in}}%
\pgfpathlineto{\pgfqpoint{2.062953in}{2.193583in}}%
\pgfpathlineto{\pgfqpoint{2.074639in}{2.197095in}}%
\pgfpathlineto{\pgfqpoint{2.068685in}{2.206030in}}%
\pgfpathlineto{\pgfqpoint{2.062737in}{2.214938in}}%
\pgfpathlineto{\pgfqpoint{2.056792in}{2.223820in}}%
\pgfpathlineto{\pgfqpoint{2.050852in}{2.232677in}}%
\pgfpathlineto{\pgfqpoint{2.044917in}{2.241508in}}%
\pgfpathlineto{\pgfqpoint{2.033241in}{2.238049in}}%
\pgfpathlineto{\pgfqpoint{2.021561in}{2.234579in}}%
\pgfpathlineto{\pgfqpoint{2.009874in}{2.231098in}}%
\pgfpathlineto{\pgfqpoint{1.998183in}{2.227605in}}%
\pgfpathlineto{\pgfqpoint{1.986485in}{2.224099in}}%
\pgfpathlineto{\pgfqpoint{1.992410in}{2.215216in}}%
\pgfpathlineto{\pgfqpoint{1.998339in}{2.206307in}}%
\pgfpathlineto{\pgfqpoint{2.004273in}{2.197373in}}%
\pgfpathlineto{\pgfqpoint{2.010211in}{2.188412in}}%
\pgfpathclose%
\pgfusepath{stroke,fill}%
\end{pgfscope}%
\begin{pgfscope}%
\pgfpathrectangle{\pgfqpoint{0.887500in}{0.275000in}}{\pgfqpoint{4.225000in}{4.225000in}}%
\pgfusepath{clip}%
\pgfsetbuttcap%
\pgfsetroundjoin%
\definecolor{currentfill}{rgb}{0.144759,0.519093,0.556572}%
\pgfsetfillcolor{currentfill}%
\pgfsetfillopacity{0.700000}%
\pgfsetlinewidth{0.501875pt}%
\definecolor{currentstroke}{rgb}{1.000000,1.000000,1.000000}%
\pgfsetstrokecolor{currentstroke}%
\pgfsetstrokeopacity{0.500000}%
\pgfsetdash{}{0pt}%
\pgfpathmoveto{\pgfqpoint{1.692791in}{2.242064in}}%
\pgfpathlineto{\pgfqpoint{1.704579in}{2.245599in}}%
\pgfpathlineto{\pgfqpoint{1.716362in}{2.249121in}}%
\pgfpathlineto{\pgfqpoint{1.728139in}{2.252631in}}%
\pgfpathlineto{\pgfqpoint{1.739911in}{2.256131in}}%
\pgfpathlineto{\pgfqpoint{1.751677in}{2.259621in}}%
\pgfpathlineto{\pgfqpoint{1.745835in}{2.268351in}}%
\pgfpathlineto{\pgfqpoint{1.739997in}{2.277056in}}%
\pgfpathlineto{\pgfqpoint{1.734164in}{2.285735in}}%
\pgfpathlineto{\pgfqpoint{1.728336in}{2.294389in}}%
\pgfpathlineto{\pgfqpoint{1.722512in}{2.303017in}}%
\pgfpathlineto{\pgfqpoint{1.710757in}{2.299571in}}%
\pgfpathlineto{\pgfqpoint{1.698997in}{2.296116in}}%
\pgfpathlineto{\pgfqpoint{1.687231in}{2.292650in}}%
\pgfpathlineto{\pgfqpoint{1.675460in}{2.289173in}}%
\pgfpathlineto{\pgfqpoint{1.663683in}{2.285684in}}%
\pgfpathlineto{\pgfqpoint{1.669495in}{2.277014in}}%
\pgfpathlineto{\pgfqpoint{1.675312in}{2.268317in}}%
\pgfpathlineto{\pgfqpoint{1.681134in}{2.259593in}}%
\pgfpathlineto{\pgfqpoint{1.686960in}{2.250842in}}%
\pgfpathclose%
\pgfusepath{stroke,fill}%
\end{pgfscope}%
\begin{pgfscope}%
\pgfpathrectangle{\pgfqpoint{0.887500in}{0.275000in}}{\pgfqpoint{4.225000in}{4.225000in}}%
\pgfusepath{clip}%
\pgfsetbuttcap%
\pgfsetroundjoin%
\definecolor{currentfill}{rgb}{0.126453,0.570633,0.549841}%
\pgfsetfillcolor{currentfill}%
\pgfsetfillopacity{0.700000}%
\pgfsetlinewidth{0.501875pt}%
\definecolor{currentstroke}{rgb}{1.000000,1.000000,1.000000}%
\pgfsetstrokecolor{currentstroke}%
\pgfsetstrokeopacity{0.500000}%
\pgfsetdash{}{0pt}%
\pgfpathmoveto{\pgfqpoint{3.798803in}{2.335702in}}%
\pgfpathlineto{\pgfqpoint{3.810086in}{2.340188in}}%
\pgfpathlineto{\pgfqpoint{3.821363in}{2.344585in}}%
\pgfpathlineto{\pgfqpoint{3.832632in}{2.348874in}}%
\pgfpathlineto{\pgfqpoint{3.843894in}{2.353040in}}%
\pgfpathlineto{\pgfqpoint{3.855149in}{2.357070in}}%
\pgfpathlineto{\pgfqpoint{3.848737in}{2.371903in}}%
\pgfpathlineto{\pgfqpoint{3.842326in}{2.386622in}}%
\pgfpathlineto{\pgfqpoint{3.835914in}{2.401202in}}%
\pgfpathlineto{\pgfqpoint{3.829502in}{2.415641in}}%
\pgfpathlineto{\pgfqpoint{3.823090in}{2.429955in}}%
\pgfpathlineto{\pgfqpoint{3.811843in}{2.426170in}}%
\pgfpathlineto{\pgfqpoint{3.800588in}{2.422265in}}%
\pgfpathlineto{\pgfqpoint{3.789326in}{2.418234in}}%
\pgfpathlineto{\pgfqpoint{3.778056in}{2.414069in}}%
\pgfpathlineto{\pgfqpoint{3.766779in}{2.409763in}}%
\pgfpathlineto{\pgfqpoint{3.773184in}{2.395268in}}%
\pgfpathlineto{\pgfqpoint{3.779590in}{2.380623in}}%
\pgfpathlineto{\pgfqpoint{3.785995in}{2.365809in}}%
\pgfpathlineto{\pgfqpoint{3.792399in}{2.350826in}}%
\pgfpathclose%
\pgfusepath{stroke,fill}%
\end{pgfscope}%
\begin{pgfscope}%
\pgfpathrectangle{\pgfqpoint{0.887500in}{0.275000in}}{\pgfqpoint{4.225000in}{4.225000in}}%
\pgfusepath{clip}%
\pgfsetbuttcap%
\pgfsetroundjoin%
\definecolor{currentfill}{rgb}{0.129933,0.559582,0.551864}%
\pgfsetfillcolor{currentfill}%
\pgfsetfillopacity{0.700000}%
\pgfsetlinewidth{0.501875pt}%
\definecolor{currentstroke}{rgb}{1.000000,1.000000,1.000000}%
\pgfsetstrokecolor{currentstroke}%
\pgfsetstrokeopacity{0.500000}%
\pgfsetdash{}{0pt}%
\pgfpathmoveto{\pgfqpoint{3.272300in}{2.281158in}}%
\pgfpathlineto{\pgfqpoint{3.283747in}{2.295245in}}%
\pgfpathlineto{\pgfqpoint{3.295195in}{2.309427in}}%
\pgfpathlineto{\pgfqpoint{3.306646in}{2.323892in}}%
\pgfpathlineto{\pgfqpoint{3.318101in}{2.338831in}}%
\pgfpathlineto{\pgfqpoint{3.329561in}{2.354433in}}%
\pgfpathlineto{\pgfqpoint{3.323237in}{2.369200in}}%
\pgfpathlineto{\pgfqpoint{3.316920in}{2.384766in}}%
\pgfpathlineto{\pgfqpoint{3.310612in}{2.401295in}}%
\pgfpathlineto{\pgfqpoint{3.304312in}{2.418927in}}%
\pgfpathlineto{\pgfqpoint{3.298020in}{2.437550in}}%
\pgfpathlineto{\pgfqpoint{3.286569in}{2.421997in}}%
\pgfpathlineto{\pgfqpoint{3.275121in}{2.406655in}}%
\pgfpathlineto{\pgfqpoint{3.263675in}{2.391413in}}%
\pgfpathlineto{\pgfqpoint{3.252230in}{2.376161in}}%
\pgfpathlineto{\pgfqpoint{3.240785in}{2.360789in}}%
\pgfpathlineto{\pgfqpoint{3.247082in}{2.344394in}}%
\pgfpathlineto{\pgfqpoint{3.253382in}{2.328240in}}%
\pgfpathlineto{\pgfqpoint{3.259685in}{2.312364in}}%
\pgfpathlineto{\pgfqpoint{3.265991in}{2.296696in}}%
\pgfpathclose%
\pgfusepath{stroke,fill}%
\end{pgfscope}%
\begin{pgfscope}%
\pgfpathrectangle{\pgfqpoint{0.887500in}{0.275000in}}{\pgfqpoint{4.225000in}{4.225000in}}%
\pgfusepath{clip}%
\pgfsetbuttcap%
\pgfsetroundjoin%
\definecolor{currentfill}{rgb}{0.166617,0.463708,0.558119}%
\pgfsetfillcolor{currentfill}%
\pgfsetfillopacity{0.700000}%
\pgfsetlinewidth{0.501875pt}%
\definecolor{currentstroke}{rgb}{1.000000,1.000000,1.000000}%
\pgfsetstrokecolor{currentstroke}%
\pgfsetstrokeopacity{0.500000}%
\pgfsetdash{}{0pt}%
\pgfpathmoveto{\pgfqpoint{4.096249in}{2.102367in}}%
\pgfpathlineto{\pgfqpoint{4.107442in}{2.105978in}}%
\pgfpathlineto{\pgfqpoint{4.118629in}{2.109550in}}%
\pgfpathlineto{\pgfqpoint{4.129811in}{2.113107in}}%
\pgfpathlineto{\pgfqpoint{4.140987in}{2.116667in}}%
\pgfpathlineto{\pgfqpoint{4.152157in}{2.120227in}}%
\pgfpathlineto{\pgfqpoint{4.145704in}{2.135457in}}%
\pgfpathlineto{\pgfqpoint{4.139248in}{2.150521in}}%
\pgfpathlineto{\pgfqpoint{4.132790in}{2.165409in}}%
\pgfpathlineto{\pgfqpoint{4.126330in}{2.180144in}}%
\pgfpathlineto{\pgfqpoint{4.119869in}{2.194746in}}%
\pgfpathlineto{\pgfqpoint{4.108707in}{2.191453in}}%
\pgfpathlineto{\pgfqpoint{4.097539in}{2.188163in}}%
\pgfpathlineto{\pgfqpoint{4.086366in}{2.184879in}}%
\pgfpathlineto{\pgfqpoint{4.075187in}{2.181577in}}%
\pgfpathlineto{\pgfqpoint{4.064001in}{2.178228in}}%
\pgfpathlineto{\pgfqpoint{4.070459in}{2.163570in}}%
\pgfpathlineto{\pgfqpoint{4.076914in}{2.148688in}}%
\pgfpathlineto{\pgfqpoint{4.083364in}{2.133543in}}%
\pgfpathlineto{\pgfqpoint{4.089809in}{2.118097in}}%
\pgfpathclose%
\pgfusepath{stroke,fill}%
\end{pgfscope}%
\begin{pgfscope}%
\pgfpathrectangle{\pgfqpoint{0.887500in}{0.275000in}}{\pgfqpoint{4.225000in}{4.225000in}}%
\pgfusepath{clip}%
\pgfsetbuttcap%
\pgfsetroundjoin%
\definecolor{currentfill}{rgb}{0.187231,0.414746,0.556547}%
\pgfsetfillcolor{currentfill}%
\pgfsetfillopacity{0.700000}%
\pgfsetlinewidth{0.501875pt}%
\definecolor{currentstroke}{rgb}{1.000000,1.000000,1.000000}%
\pgfsetstrokecolor{currentstroke}%
\pgfsetstrokeopacity{0.500000}%
\pgfsetdash{}{0pt}%
\pgfpathmoveto{\pgfqpoint{2.751501in}{2.014764in}}%
\pgfpathlineto{\pgfqpoint{2.763029in}{2.018396in}}%
\pgfpathlineto{\pgfqpoint{2.774551in}{2.021949in}}%
\pgfpathlineto{\pgfqpoint{2.786069in}{2.025402in}}%
\pgfpathlineto{\pgfqpoint{2.797581in}{2.028733in}}%
\pgfpathlineto{\pgfqpoint{2.809088in}{2.031959in}}%
\pgfpathlineto{\pgfqpoint{2.802891in}{2.041667in}}%
\pgfpathlineto{\pgfqpoint{2.796698in}{2.051333in}}%
\pgfpathlineto{\pgfqpoint{2.790509in}{2.060955in}}%
\pgfpathlineto{\pgfqpoint{2.784324in}{2.070534in}}%
\pgfpathlineto{\pgfqpoint{2.778143in}{2.080070in}}%
\pgfpathlineto{\pgfqpoint{2.766646in}{2.076909in}}%
\pgfpathlineto{\pgfqpoint{2.755143in}{2.073638in}}%
\pgfpathlineto{\pgfqpoint{2.743635in}{2.070239in}}%
\pgfpathlineto{\pgfqpoint{2.732123in}{2.066735in}}%
\pgfpathlineto{\pgfqpoint{2.720605in}{2.063149in}}%
\pgfpathlineto{\pgfqpoint{2.726776in}{2.053560in}}%
\pgfpathlineto{\pgfqpoint{2.732951in}{2.043927in}}%
\pgfpathlineto{\pgfqpoint{2.739131in}{2.034250in}}%
\pgfpathlineto{\pgfqpoint{2.745314in}{2.024529in}}%
\pgfpathclose%
\pgfusepath{stroke,fill}%
\end{pgfscope}%
\begin{pgfscope}%
\pgfpathrectangle{\pgfqpoint{0.887500in}{0.275000in}}{\pgfqpoint{4.225000in}{4.225000in}}%
\pgfusepath{clip}%
\pgfsetbuttcap%
\pgfsetroundjoin%
\definecolor{currentfill}{rgb}{0.172719,0.448791,0.557885}%
\pgfsetfillcolor{currentfill}%
\pgfsetfillopacity{0.700000}%
\pgfsetlinewidth{0.501875pt}%
\definecolor{currentstroke}{rgb}{1.000000,1.000000,1.000000}%
\pgfsetstrokecolor{currentstroke}%
\pgfsetstrokeopacity{0.500000}%
\pgfsetdash{}{0pt}%
\pgfpathmoveto{\pgfqpoint{3.189246in}{2.037235in}}%
\pgfpathlineto{\pgfqpoint{3.200704in}{2.054935in}}%
\pgfpathlineto{\pgfqpoint{3.212163in}{2.072203in}}%
\pgfpathlineto{\pgfqpoint{3.223622in}{2.089017in}}%
\pgfpathlineto{\pgfqpoint{3.235081in}{2.105484in}}%
\pgfpathlineto{\pgfqpoint{3.246542in}{2.121707in}}%
\pgfpathlineto{\pgfqpoint{3.240245in}{2.139477in}}%
\pgfpathlineto{\pgfqpoint{3.233947in}{2.156773in}}%
\pgfpathlineto{\pgfqpoint{3.227649in}{2.173594in}}%
\pgfpathlineto{\pgfqpoint{3.221350in}{2.189940in}}%
\pgfpathlineto{\pgfqpoint{3.215051in}{2.205810in}}%
\pgfpathlineto{\pgfqpoint{3.203597in}{2.188728in}}%
\pgfpathlineto{\pgfqpoint{3.192142in}{2.170687in}}%
\pgfpathlineto{\pgfqpoint{3.180686in}{2.151570in}}%
\pgfpathlineto{\pgfqpoint{3.169231in}{2.131256in}}%
\pgfpathlineto{\pgfqpoint{3.157776in}{2.109812in}}%
\pgfpathlineto{\pgfqpoint{3.164068in}{2.096068in}}%
\pgfpathlineto{\pgfqpoint{3.170361in}{2.081921in}}%
\pgfpathlineto{\pgfqpoint{3.176655in}{2.067389in}}%
\pgfpathlineto{\pgfqpoint{3.182950in}{2.052488in}}%
\pgfpathclose%
\pgfusepath{stroke,fill}%
\end{pgfscope}%
\begin{pgfscope}%
\pgfpathrectangle{\pgfqpoint{0.887500in}{0.275000in}}{\pgfqpoint{4.225000in}{4.225000in}}%
\pgfusepath{clip}%
\pgfsetbuttcap%
\pgfsetroundjoin%
\definecolor{currentfill}{rgb}{0.216210,0.351535,0.550627}%
\pgfsetfillcolor{currentfill}%
\pgfsetfillopacity{0.700000}%
\pgfsetlinewidth{0.501875pt}%
\definecolor{currentstroke}{rgb}{1.000000,1.000000,1.000000}%
\pgfsetstrokecolor{currentstroke}%
\pgfsetstrokeopacity{0.500000}%
\pgfsetdash{}{0pt}%
\pgfpathmoveto{\pgfqpoint{4.393877in}{1.872779in}}%
\pgfpathlineto{\pgfqpoint{4.404988in}{1.876106in}}%
\pgfpathlineto{\pgfqpoint{4.416092in}{1.879411in}}%
\pgfpathlineto{\pgfqpoint{4.427191in}{1.882701in}}%
\pgfpathlineto{\pgfqpoint{4.438283in}{1.885981in}}%
\pgfpathlineto{\pgfqpoint{4.449370in}{1.889256in}}%
\pgfpathlineto{\pgfqpoint{4.442845in}{1.903920in}}%
\pgfpathlineto{\pgfqpoint{4.436322in}{1.918550in}}%
\pgfpathlineto{\pgfqpoint{4.429799in}{1.933134in}}%
\pgfpathlineto{\pgfqpoint{4.423277in}{1.947663in}}%
\pgfpathlineto{\pgfqpoint{4.416755in}{1.962125in}}%
\pgfpathlineto{\pgfqpoint{4.405659in}{1.958542in}}%
\pgfpathlineto{\pgfqpoint{4.394555in}{1.954908in}}%
\pgfpathlineto{\pgfqpoint{4.383445in}{1.951222in}}%
\pgfpathlineto{\pgfqpoint{4.372327in}{1.947485in}}%
\pgfpathlineto{\pgfqpoint{4.361202in}{1.943693in}}%
\pgfpathlineto{\pgfqpoint{4.367734in}{1.929587in}}%
\pgfpathlineto{\pgfqpoint{4.374268in}{1.915447in}}%
\pgfpathlineto{\pgfqpoint{4.380803in}{1.901269in}}%
\pgfpathlineto{\pgfqpoint{4.387339in}{1.887048in}}%
\pgfpathclose%
\pgfusepath{stroke,fill}%
\end{pgfscope}%
\begin{pgfscope}%
\pgfpathrectangle{\pgfqpoint{0.887500in}{0.275000in}}{\pgfqpoint{4.225000in}{4.225000in}}%
\pgfusepath{clip}%
\pgfsetbuttcap%
\pgfsetroundjoin%
\definecolor{currentfill}{rgb}{0.120565,0.596422,0.543611}%
\pgfsetfillcolor{currentfill}%
\pgfsetfillopacity{0.700000}%
\pgfsetlinewidth{0.501875pt}%
\definecolor{currentstroke}{rgb}{1.000000,1.000000,1.000000}%
\pgfsetstrokecolor{currentstroke}%
\pgfsetstrokeopacity{0.500000}%
\pgfsetdash{}{0pt}%
\pgfpathmoveto{\pgfqpoint{3.710275in}{2.385813in}}%
\pgfpathlineto{\pgfqpoint{3.721591in}{2.390979in}}%
\pgfpathlineto{\pgfqpoint{3.732900in}{2.395929in}}%
\pgfpathlineto{\pgfqpoint{3.744200in}{2.400700in}}%
\pgfpathlineto{\pgfqpoint{3.755493in}{2.405309in}}%
\pgfpathlineto{\pgfqpoint{3.766779in}{2.409763in}}%
\pgfpathlineto{\pgfqpoint{3.760373in}{2.424130in}}%
\pgfpathlineto{\pgfqpoint{3.753968in}{2.438390in}}%
\pgfpathlineto{\pgfqpoint{3.747564in}{2.452563in}}%
\pgfpathlineto{\pgfqpoint{3.741161in}{2.466671in}}%
\pgfpathlineto{\pgfqpoint{3.734760in}{2.480734in}}%
\pgfpathlineto{\pgfqpoint{3.723482in}{2.476556in}}%
\pgfpathlineto{\pgfqpoint{3.712199in}{2.472285in}}%
\pgfpathlineto{\pgfqpoint{3.700908in}{2.467920in}}%
\pgfpathlineto{\pgfqpoint{3.689611in}{2.463451in}}%
\pgfpathlineto{\pgfqpoint{3.678307in}{2.458853in}}%
\pgfpathlineto{\pgfqpoint{3.684696in}{2.444292in}}%
\pgfpathlineto{\pgfqpoint{3.691088in}{2.429706in}}%
\pgfpathlineto{\pgfqpoint{3.697481in}{2.415097in}}%
\pgfpathlineto{\pgfqpoint{3.703877in}{2.400466in}}%
\pgfpathclose%
\pgfusepath{stroke,fill}%
\end{pgfscope}%
\begin{pgfscope}%
\pgfpathrectangle{\pgfqpoint{0.887500in}{0.275000in}}{\pgfqpoint{4.225000in}{4.225000in}}%
\pgfusepath{clip}%
\pgfsetbuttcap%
\pgfsetroundjoin%
\definecolor{currentfill}{rgb}{0.132268,0.655014,0.519661}%
\pgfsetfillcolor{currentfill}%
\pgfsetfillopacity{0.700000}%
\pgfsetlinewidth{0.501875pt}%
\definecolor{currentstroke}{rgb}{1.000000,1.000000,1.000000}%
\pgfsetstrokecolor{currentstroke}%
\pgfsetstrokeopacity{0.500000}%
\pgfsetdash{}{0pt}%
\pgfpathmoveto{\pgfqpoint{3.444175in}{2.504011in}}%
\pgfpathlineto{\pgfqpoint{3.455594in}{2.514259in}}%
\pgfpathlineto{\pgfqpoint{3.467004in}{2.523731in}}%
\pgfpathlineto{\pgfqpoint{3.478405in}{2.532541in}}%
\pgfpathlineto{\pgfqpoint{3.489797in}{2.540802in}}%
\pgfpathlineto{\pgfqpoint{3.501181in}{2.548577in}}%
\pgfpathlineto{\pgfqpoint{3.494818in}{2.562534in}}%
\pgfpathlineto{\pgfqpoint{3.488460in}{2.576586in}}%
\pgfpathlineto{\pgfqpoint{3.482105in}{2.590804in}}%
\pgfpathlineto{\pgfqpoint{3.475756in}{2.605252in}}%
\pgfpathlineto{\pgfqpoint{3.469411in}{2.619920in}}%
\pgfpathlineto{\pgfqpoint{3.458036in}{2.612386in}}%
\pgfpathlineto{\pgfqpoint{3.446653in}{2.604382in}}%
\pgfpathlineto{\pgfqpoint{3.435262in}{2.595861in}}%
\pgfpathlineto{\pgfqpoint{3.423862in}{2.586745in}}%
\pgfpathlineto{\pgfqpoint{3.412455in}{2.576951in}}%
\pgfpathlineto{\pgfqpoint{3.418787in}{2.561559in}}%
\pgfpathlineto{\pgfqpoint{3.425125in}{2.546597in}}%
\pgfpathlineto{\pgfqpoint{3.431469in}{2.532089in}}%
\pgfpathlineto{\pgfqpoint{3.437820in}{2.517931in}}%
\pgfpathclose%
\pgfusepath{stroke,fill}%
\end{pgfscope}%
\begin{pgfscope}%
\pgfpathrectangle{\pgfqpoint{0.887500in}{0.275000in}}{\pgfqpoint{4.225000in}{4.225000in}}%
\pgfusepath{clip}%
\pgfsetbuttcap%
\pgfsetroundjoin%
\definecolor{currentfill}{rgb}{0.172719,0.448791,0.557885}%
\pgfsetfillcolor{currentfill}%
\pgfsetfillopacity{0.700000}%
\pgfsetlinewidth{0.501875pt}%
\definecolor{currentstroke}{rgb}{1.000000,1.000000,1.000000}%
\pgfsetstrokecolor{currentstroke}%
\pgfsetstrokeopacity{0.500000}%
\pgfsetdash{}{0pt}%
\pgfpathmoveto{\pgfqpoint{2.428000in}{2.085928in}}%
\pgfpathlineto{\pgfqpoint{2.439610in}{2.089539in}}%
\pgfpathlineto{\pgfqpoint{2.451214in}{2.093128in}}%
\pgfpathlineto{\pgfqpoint{2.462813in}{2.096695in}}%
\pgfpathlineto{\pgfqpoint{2.474406in}{2.100238in}}%
\pgfpathlineto{\pgfqpoint{2.485994in}{2.103760in}}%
\pgfpathlineto{\pgfqpoint{2.479901in}{2.113049in}}%
\pgfpathlineto{\pgfqpoint{2.473811in}{2.122302in}}%
\pgfpathlineto{\pgfqpoint{2.467726in}{2.131521in}}%
\pgfpathlineto{\pgfqpoint{2.461646in}{2.140706in}}%
\pgfpathlineto{\pgfqpoint{2.455570in}{2.149857in}}%
\pgfpathlineto{\pgfqpoint{2.443992in}{2.146384in}}%
\pgfpathlineto{\pgfqpoint{2.432409in}{2.142890in}}%
\pgfpathlineto{\pgfqpoint{2.420821in}{2.139374in}}%
\pgfpathlineto{\pgfqpoint{2.409227in}{2.135834in}}%
\pgfpathlineto{\pgfqpoint{2.397628in}{2.132274in}}%
\pgfpathlineto{\pgfqpoint{2.403694in}{2.123073in}}%
\pgfpathlineto{\pgfqpoint{2.409764in}{2.113839in}}%
\pgfpathlineto{\pgfqpoint{2.415838in}{2.104570in}}%
\pgfpathlineto{\pgfqpoint{2.421917in}{2.095267in}}%
\pgfpathclose%
\pgfusepath{stroke,fill}%
\end{pgfscope}%
\begin{pgfscope}%
\pgfpathrectangle{\pgfqpoint{0.887500in}{0.275000in}}{\pgfqpoint{4.225000in}{4.225000in}}%
\pgfusepath{clip}%
\pgfsetbuttcap%
\pgfsetroundjoin%
\definecolor{currentfill}{rgb}{0.120092,0.600104,0.542530}%
\pgfsetfillcolor{currentfill}%
\pgfsetfillopacity{0.700000}%
\pgfsetlinewidth{0.501875pt}%
\definecolor{currentstroke}{rgb}{1.000000,1.000000,1.000000}%
\pgfsetstrokecolor{currentstroke}%
\pgfsetstrokeopacity{0.500000}%
\pgfsetdash{}{0pt}%
\pgfpathmoveto{\pgfqpoint{3.329561in}{2.354433in}}%
\pgfpathlineto{\pgfqpoint{3.341028in}{2.370747in}}%
\pgfpathlineto{\pgfqpoint{3.352499in}{2.387511in}}%
\pgfpathlineto{\pgfqpoint{3.363974in}{2.404422in}}%
\pgfpathlineto{\pgfqpoint{3.375450in}{2.421178in}}%
\pgfpathlineto{\pgfqpoint{3.386924in}{2.437475in}}%
\pgfpathlineto{\pgfqpoint{3.380582in}{2.451651in}}%
\pgfpathlineto{\pgfqpoint{3.374248in}{2.466388in}}%
\pgfpathlineto{\pgfqpoint{3.367921in}{2.481840in}}%
\pgfpathlineto{\pgfqpoint{3.361603in}{2.498140in}}%
\pgfpathlineto{\pgfqpoint{3.355292in}{2.515219in}}%
\pgfpathlineto{\pgfqpoint{3.343840in}{2.500323in}}%
\pgfpathlineto{\pgfqpoint{3.332384in}{2.484916in}}%
\pgfpathlineto{\pgfqpoint{3.320929in}{2.469189in}}%
\pgfpathlineto{\pgfqpoint{3.309473in}{2.453336in}}%
\pgfpathlineto{\pgfqpoint{3.298020in}{2.437550in}}%
\pgfpathlineto{\pgfqpoint{3.304312in}{2.418927in}}%
\pgfpathlineto{\pgfqpoint{3.310612in}{2.401295in}}%
\pgfpathlineto{\pgfqpoint{3.316920in}{2.384766in}}%
\pgfpathlineto{\pgfqpoint{3.323237in}{2.369200in}}%
\pgfpathclose%
\pgfusepath{stroke,fill}%
\end{pgfscope}%
\begin{pgfscope}%
\pgfpathrectangle{\pgfqpoint{0.887500in}{0.275000in}}{\pgfqpoint{4.225000in}{4.225000in}}%
\pgfusepath{clip}%
\pgfsetbuttcap%
\pgfsetroundjoin%
\definecolor{currentfill}{rgb}{0.192357,0.403199,0.555836}%
\pgfsetfillcolor{currentfill}%
\pgfsetfillopacity{0.700000}%
\pgfsetlinewidth{0.501875pt}%
\definecolor{currentstroke}{rgb}{1.000000,1.000000,1.000000}%
\pgfsetstrokecolor{currentstroke}%
\pgfsetstrokeopacity{0.500000}%
\pgfsetdash{}{0pt}%
\pgfpathmoveto{\pgfqpoint{2.840134in}{1.982757in}}%
\pgfpathlineto{\pgfqpoint{2.851645in}{1.986017in}}%
\pgfpathlineto{\pgfqpoint{2.863150in}{1.989313in}}%
\pgfpathlineto{\pgfqpoint{2.874648in}{1.992718in}}%
\pgfpathlineto{\pgfqpoint{2.886140in}{1.996301in}}%
\pgfpathlineto{\pgfqpoint{2.897625in}{2.000134in}}%
\pgfpathlineto{\pgfqpoint{2.891399in}{2.010006in}}%
\pgfpathlineto{\pgfqpoint{2.885176in}{2.019834in}}%
\pgfpathlineto{\pgfqpoint{2.878957in}{2.029618in}}%
\pgfpathlineto{\pgfqpoint{2.872742in}{2.039358in}}%
\pgfpathlineto{\pgfqpoint{2.866532in}{2.049056in}}%
\pgfpathlineto{\pgfqpoint{2.855056in}{2.045259in}}%
\pgfpathlineto{\pgfqpoint{2.843574in}{2.041726in}}%
\pgfpathlineto{\pgfqpoint{2.832085in}{2.038381in}}%
\pgfpathlineto{\pgfqpoint{2.820590in}{2.035150in}}%
\pgfpathlineto{\pgfqpoint{2.809088in}{2.031959in}}%
\pgfpathlineto{\pgfqpoint{2.815289in}{2.022206in}}%
\pgfpathlineto{\pgfqpoint{2.821495in}{2.012410in}}%
\pgfpathlineto{\pgfqpoint{2.827704in}{2.002570in}}%
\pgfpathlineto{\pgfqpoint{2.833917in}{1.992685in}}%
\pgfpathclose%
\pgfusepath{stroke,fill}%
\end{pgfscope}%
\begin{pgfscope}%
\pgfpathrectangle{\pgfqpoint{0.887500in}{0.275000in}}{\pgfqpoint{4.225000in}{4.225000in}}%
\pgfusepath{clip}%
\pgfsetbuttcap%
\pgfsetroundjoin%
\definecolor{currentfill}{rgb}{0.120081,0.622161,0.534946}%
\pgfsetfillcolor{currentfill}%
\pgfsetfillopacity{0.700000}%
\pgfsetlinewidth{0.501875pt}%
\definecolor{currentstroke}{rgb}{1.000000,1.000000,1.000000}%
\pgfsetstrokecolor{currentstroke}%
\pgfsetstrokeopacity{0.500000}%
\pgfsetdash{}{0pt}%
\pgfpathmoveto{\pgfqpoint{3.621670in}{2.432957in}}%
\pgfpathlineto{\pgfqpoint{3.633014in}{2.438616in}}%
\pgfpathlineto{\pgfqpoint{3.644349in}{2.444007in}}%
\pgfpathlineto{\pgfqpoint{3.655676in}{2.449158in}}%
\pgfpathlineto{\pgfqpoint{3.666995in}{2.454098in}}%
\pgfpathlineto{\pgfqpoint{3.678307in}{2.458853in}}%
\pgfpathlineto{\pgfqpoint{3.671919in}{2.473388in}}%
\pgfpathlineto{\pgfqpoint{3.665533in}{2.487895in}}%
\pgfpathlineto{\pgfqpoint{3.659150in}{2.502373in}}%
\pgfpathlineto{\pgfqpoint{3.652768in}{2.516820in}}%
\pgfpathlineto{\pgfqpoint{3.646388in}{2.531233in}}%
\pgfpathlineto{\pgfqpoint{3.635089in}{2.527108in}}%
\pgfpathlineto{\pgfqpoint{3.623783in}{2.522831in}}%
\pgfpathlineto{\pgfqpoint{3.612468in}{2.518354in}}%
\pgfpathlineto{\pgfqpoint{3.601146in}{2.513629in}}%
\pgfpathlineto{\pgfqpoint{3.589814in}{2.508608in}}%
\pgfpathlineto{\pgfqpoint{3.596186in}{2.493830in}}%
\pgfpathlineto{\pgfqpoint{3.602557in}{2.478872in}}%
\pgfpathlineto{\pgfqpoint{3.608928in}{2.463720in}}%
\pgfpathlineto{\pgfqpoint{3.615299in}{2.448400in}}%
\pgfpathclose%
\pgfusepath{stroke,fill}%
\end{pgfscope}%
\begin{pgfscope}%
\pgfpathrectangle{\pgfqpoint{0.887500in}{0.275000in}}{\pgfqpoint{4.225000in}{4.225000in}}%
\pgfusepath{clip}%
\pgfsetbuttcap%
\pgfsetroundjoin%
\definecolor{currentfill}{rgb}{0.159194,0.482237,0.558073}%
\pgfsetfillcolor{currentfill}%
\pgfsetfillopacity{0.700000}%
\pgfsetlinewidth{0.501875pt}%
\definecolor{currentstroke}{rgb}{1.000000,1.000000,1.000000}%
\pgfsetstrokecolor{currentstroke}%
\pgfsetstrokeopacity{0.500000}%
\pgfsetdash{}{0pt}%
\pgfpathmoveto{\pgfqpoint{2.104471in}{2.152011in}}%
\pgfpathlineto{\pgfqpoint{2.116162in}{2.155561in}}%
\pgfpathlineto{\pgfqpoint{2.127848in}{2.159102in}}%
\pgfpathlineto{\pgfqpoint{2.139528in}{2.162633in}}%
\pgfpathlineto{\pgfqpoint{2.151202in}{2.166157in}}%
\pgfpathlineto{\pgfqpoint{2.162871in}{2.169675in}}%
\pgfpathlineto{\pgfqpoint{2.156885in}{2.178699in}}%
\pgfpathlineto{\pgfqpoint{2.150903in}{2.187694in}}%
\pgfpathlineto{\pgfqpoint{2.144926in}{2.196661in}}%
\pgfpathlineto{\pgfqpoint{2.138953in}{2.205600in}}%
\pgfpathlineto{\pgfqpoint{2.132985in}{2.214512in}}%
\pgfpathlineto{\pgfqpoint{2.121327in}{2.211043in}}%
\pgfpathlineto{\pgfqpoint{2.109663in}{2.207569in}}%
\pgfpathlineto{\pgfqpoint{2.097994in}{2.204087in}}%
\pgfpathlineto{\pgfqpoint{2.086319in}{2.200596in}}%
\pgfpathlineto{\pgfqpoint{2.074639in}{2.197095in}}%
\pgfpathlineto{\pgfqpoint{2.080596in}{2.188133in}}%
\pgfpathlineto{\pgfqpoint{2.086558in}{2.179144in}}%
\pgfpathlineto{\pgfqpoint{2.092525in}{2.170128in}}%
\pgfpathlineto{\pgfqpoint{2.098495in}{2.161083in}}%
\pgfpathclose%
\pgfusepath{stroke,fill}%
\end{pgfscope}%
\begin{pgfscope}%
\pgfpathrectangle{\pgfqpoint{0.887500in}{0.275000in}}{\pgfqpoint{4.225000in}{4.225000in}}%
\pgfusepath{clip}%
\pgfsetbuttcap%
\pgfsetroundjoin%
\definecolor{currentfill}{rgb}{0.156270,0.489624,0.557936}%
\pgfsetfillcolor{currentfill}%
\pgfsetfillopacity{0.700000}%
\pgfsetlinewidth{0.501875pt}%
\definecolor{currentstroke}{rgb}{1.000000,1.000000,1.000000}%
\pgfsetstrokecolor{currentstroke}%
\pgfsetstrokeopacity{0.500000}%
\pgfsetdash{}{0pt}%
\pgfpathmoveto{\pgfqpoint{4.007944in}{2.159681in}}%
\pgfpathlineto{\pgfqpoint{4.019175in}{2.163730in}}%
\pgfpathlineto{\pgfqpoint{4.030395in}{2.167582in}}%
\pgfpathlineto{\pgfqpoint{4.041606in}{2.171261in}}%
\pgfpathlineto{\pgfqpoint{4.052808in}{2.174799in}}%
\pgfpathlineto{\pgfqpoint{4.064001in}{2.178228in}}%
\pgfpathlineto{\pgfqpoint{4.057541in}{2.192700in}}%
\pgfpathlineto{\pgfqpoint{4.051079in}{2.207026in}}%
\pgfpathlineto{\pgfqpoint{4.044617in}{2.221244in}}%
\pgfpathlineto{\pgfqpoint{4.038157in}{2.235393in}}%
\pgfpathlineto{\pgfqpoint{4.031698in}{2.249513in}}%
\pgfpathlineto{\pgfqpoint{4.020505in}{2.246029in}}%
\pgfpathlineto{\pgfqpoint{4.009305in}{2.242463in}}%
\pgfpathlineto{\pgfqpoint{3.998096in}{2.238792in}}%
\pgfpathlineto{\pgfqpoint{3.986879in}{2.234989in}}%
\pgfpathlineto{\pgfqpoint{3.975654in}{2.231032in}}%
\pgfpathlineto{\pgfqpoint{3.982110in}{2.216911in}}%
\pgfpathlineto{\pgfqpoint{3.988569in}{2.202765in}}%
\pgfpathlineto{\pgfqpoint{3.995028in}{2.188544in}}%
\pgfpathlineto{\pgfqpoint{4.001487in}{2.174200in}}%
\pgfpathclose%
\pgfusepath{stroke,fill}%
\end{pgfscope}%
\begin{pgfscope}%
\pgfpathrectangle{\pgfqpoint{0.887500in}{0.275000in}}{\pgfqpoint{4.225000in}{4.225000in}}%
\pgfusepath{clip}%
\pgfsetbuttcap%
\pgfsetroundjoin%
\definecolor{currentfill}{rgb}{0.203063,0.379716,0.553925}%
\pgfsetfillcolor{currentfill}%
\pgfsetfillopacity{0.700000}%
\pgfsetlinewidth{0.501875pt}%
\definecolor{currentstroke}{rgb}{1.000000,1.000000,1.000000}%
\pgfsetstrokecolor{currentstroke}%
\pgfsetstrokeopacity{0.500000}%
\pgfsetdash{}{0pt}%
\pgfpathmoveto{\pgfqpoint{4.305466in}{1.923743in}}%
\pgfpathlineto{\pgfqpoint{4.316628in}{1.927881in}}%
\pgfpathlineto{\pgfqpoint{4.327783in}{1.931940in}}%
\pgfpathlineto{\pgfqpoint{4.338930in}{1.935925in}}%
\pgfpathlineto{\pgfqpoint{4.350070in}{1.939841in}}%
\pgfpathlineto{\pgfqpoint{4.361202in}{1.943693in}}%
\pgfpathlineto{\pgfqpoint{4.354672in}{1.957769in}}%
\pgfpathlineto{\pgfqpoint{4.348143in}{1.971820in}}%
\pgfpathlineto{\pgfqpoint{4.341617in}{1.985851in}}%
\pgfpathlineto{\pgfqpoint{4.335093in}{1.999867in}}%
\pgfpathlineto{\pgfqpoint{4.328571in}{2.013871in}}%
\pgfpathlineto{\pgfqpoint{4.317433in}{2.009770in}}%
\pgfpathlineto{\pgfqpoint{4.306288in}{2.005584in}}%
\pgfpathlineto{\pgfqpoint{4.295135in}{2.001315in}}%
\pgfpathlineto{\pgfqpoint{4.283974in}{1.996966in}}%
\pgfpathlineto{\pgfqpoint{4.272806in}{1.992541in}}%
\pgfpathlineto{\pgfqpoint{4.279326in}{1.978601in}}%
\pgfpathlineto{\pgfqpoint{4.285853in}{1.964774in}}%
\pgfpathlineto{\pgfqpoint{4.292386in}{1.951038in}}%
\pgfpathlineto{\pgfqpoint{4.298924in}{1.937368in}}%
\pgfpathclose%
\pgfusepath{stroke,fill}%
\end{pgfscope}%
\begin{pgfscope}%
\pgfpathrectangle{\pgfqpoint{0.887500in}{0.275000in}}{\pgfqpoint{4.225000in}{4.225000in}}%
\pgfusepath{clip}%
\pgfsetbuttcap%
\pgfsetroundjoin%
\definecolor{currentfill}{rgb}{0.147607,0.511733,0.557049}%
\pgfsetfillcolor{currentfill}%
\pgfsetfillopacity{0.700000}%
\pgfsetlinewidth{0.501875pt}%
\definecolor{currentstroke}{rgb}{1.000000,1.000000,1.000000}%
\pgfsetstrokecolor{currentstroke}%
\pgfsetstrokeopacity{0.500000}%
\pgfsetdash{}{0pt}%
\pgfpathmoveto{\pgfqpoint{1.780955in}{2.215581in}}%
\pgfpathlineto{\pgfqpoint{1.792726in}{2.219110in}}%
\pgfpathlineto{\pgfqpoint{1.804492in}{2.222631in}}%
\pgfpathlineto{\pgfqpoint{1.816253in}{2.226146in}}%
\pgfpathlineto{\pgfqpoint{1.828008in}{2.229656in}}%
\pgfpathlineto{\pgfqpoint{1.839757in}{2.233162in}}%
\pgfpathlineto{\pgfqpoint{1.833881in}{2.241974in}}%
\pgfpathlineto{\pgfqpoint{1.828010in}{2.250761in}}%
\pgfpathlineto{\pgfqpoint{1.822143in}{2.259522in}}%
\pgfpathlineto{\pgfqpoint{1.816281in}{2.268258in}}%
\pgfpathlineto{\pgfqpoint{1.810423in}{2.276968in}}%
\pgfpathlineto{\pgfqpoint{1.798685in}{2.273508in}}%
\pgfpathlineto{\pgfqpoint{1.786942in}{2.270045in}}%
\pgfpathlineto{\pgfqpoint{1.775192in}{2.266576in}}%
\pgfpathlineto{\pgfqpoint{1.763437in}{2.263102in}}%
\pgfpathlineto{\pgfqpoint{1.751677in}{2.259621in}}%
\pgfpathlineto{\pgfqpoint{1.757523in}{2.250865in}}%
\pgfpathlineto{\pgfqpoint{1.763375in}{2.242083in}}%
\pgfpathlineto{\pgfqpoint{1.769230in}{2.233275in}}%
\pgfpathlineto{\pgfqpoint{1.775090in}{2.224441in}}%
\pgfpathclose%
\pgfusepath{stroke,fill}%
\end{pgfscope}%
\begin{pgfscope}%
\pgfpathrectangle{\pgfqpoint{0.887500in}{0.275000in}}{\pgfqpoint{4.225000in}{4.225000in}}%
\pgfusepath{clip}%
\pgfsetbuttcap%
\pgfsetroundjoin%
\definecolor{currentfill}{rgb}{0.126326,0.644107,0.525311}%
\pgfsetfillcolor{currentfill}%
\pgfsetfillopacity{0.700000}%
\pgfsetlinewidth{0.501875pt}%
\definecolor{currentstroke}{rgb}{1.000000,1.000000,1.000000}%
\pgfsetstrokecolor{currentstroke}%
\pgfsetstrokeopacity{0.500000}%
\pgfsetdash{}{0pt}%
\pgfpathmoveto{\pgfqpoint{3.533018in}{2.477768in}}%
\pgfpathlineto{\pgfqpoint{3.544396in}{2.484812in}}%
\pgfpathlineto{\pgfqpoint{3.555765in}{2.491385in}}%
\pgfpathlineto{\pgfqpoint{3.567124in}{2.497520in}}%
\pgfpathlineto{\pgfqpoint{3.578474in}{2.503250in}}%
\pgfpathlineto{\pgfqpoint{3.589814in}{2.508608in}}%
\pgfpathlineto{\pgfqpoint{3.583443in}{2.523226in}}%
\pgfpathlineto{\pgfqpoint{3.577072in}{2.537705in}}%
\pgfpathlineto{\pgfqpoint{3.570702in}{2.552064in}}%
\pgfpathlineto{\pgfqpoint{3.564333in}{2.566324in}}%
\pgfpathlineto{\pgfqpoint{3.557966in}{2.580507in}}%
\pgfpathlineto{\pgfqpoint{3.546627in}{2.575016in}}%
\pgfpathlineto{\pgfqpoint{3.535280in}{2.569087in}}%
\pgfpathlineto{\pgfqpoint{3.523922in}{2.562710in}}%
\pgfpathlineto{\pgfqpoint{3.512556in}{2.555877in}}%
\pgfpathlineto{\pgfqpoint{3.501181in}{2.548577in}}%
\pgfpathlineto{\pgfqpoint{3.507546in}{2.534646in}}%
\pgfpathlineto{\pgfqpoint{3.513913in}{2.520669in}}%
\pgfpathlineto{\pgfqpoint{3.520281in}{2.506578in}}%
\pgfpathlineto{\pgfqpoint{3.526650in}{2.492300in}}%
\pgfpathclose%
\pgfusepath{stroke,fill}%
\end{pgfscope}%
\begin{pgfscope}%
\pgfpathrectangle{\pgfqpoint{0.887500in}{0.275000in}}{\pgfqpoint{4.225000in}{4.225000in}}%
\pgfusepath{clip}%
\pgfsetbuttcap%
\pgfsetroundjoin%
\definecolor{currentfill}{rgb}{0.122312,0.633153,0.530398}%
\pgfsetfillcolor{currentfill}%
\pgfsetfillopacity{0.700000}%
\pgfsetlinewidth{0.501875pt}%
\definecolor{currentstroke}{rgb}{1.000000,1.000000,1.000000}%
\pgfsetstrokecolor{currentstroke}%
\pgfsetstrokeopacity{0.500000}%
\pgfsetdash{}{0pt}%
\pgfpathmoveto{\pgfqpoint{3.386924in}{2.437475in}}%
\pgfpathlineto{\pgfqpoint{3.398393in}{2.453008in}}%
\pgfpathlineto{\pgfqpoint{3.409854in}{2.467475in}}%
\pgfpathlineto{\pgfqpoint{3.421305in}{2.480733in}}%
\pgfpathlineto{\pgfqpoint{3.432745in}{2.492873in}}%
\pgfpathlineto{\pgfqpoint{3.444175in}{2.504011in}}%
\pgfpathlineto{\pgfqpoint{3.437820in}{2.517931in}}%
\pgfpathlineto{\pgfqpoint{3.431469in}{2.532089in}}%
\pgfpathlineto{\pgfqpoint{3.425125in}{2.546597in}}%
\pgfpathlineto{\pgfqpoint{3.418787in}{2.561559in}}%
\pgfpathlineto{\pgfqpoint{3.412455in}{2.576951in}}%
\pgfpathlineto{\pgfqpoint{3.401039in}{2.566400in}}%
\pgfpathlineto{\pgfqpoint{3.389614in}{2.555011in}}%
\pgfpathlineto{\pgfqpoint{3.378181in}{2.542703in}}%
\pgfpathlineto{\pgfqpoint{3.366740in}{2.529410in}}%
\pgfpathlineto{\pgfqpoint{3.355292in}{2.515219in}}%
\pgfpathlineto{\pgfqpoint{3.361603in}{2.498140in}}%
\pgfpathlineto{\pgfqpoint{3.367921in}{2.481840in}}%
\pgfpathlineto{\pgfqpoint{3.374248in}{2.466388in}}%
\pgfpathlineto{\pgfqpoint{3.380582in}{2.451651in}}%
\pgfpathclose%
\pgfusepath{stroke,fill}%
\end{pgfscope}%
\begin{pgfscope}%
\pgfpathrectangle{\pgfqpoint{0.887500in}{0.275000in}}{\pgfqpoint{4.225000in}{4.225000in}}%
\pgfusepath{clip}%
\pgfsetbuttcap%
\pgfsetroundjoin%
\definecolor{currentfill}{rgb}{0.192357,0.403199,0.555836}%
\pgfsetfillcolor{currentfill}%
\pgfsetfillopacity{0.700000}%
\pgfsetlinewidth{0.501875pt}%
\definecolor{currentstroke}{rgb}{1.000000,1.000000,1.000000}%
\pgfsetstrokecolor{currentstroke}%
\pgfsetstrokeopacity{0.500000}%
\pgfsetdash{}{0pt}%
\pgfpathmoveto{\pgfqpoint{4.216869in}{1.969716in}}%
\pgfpathlineto{\pgfqpoint{4.228067in}{1.974304in}}%
\pgfpathlineto{\pgfqpoint{4.239260in}{1.978903in}}%
\pgfpathlineto{\pgfqpoint{4.250448in}{1.983491in}}%
\pgfpathlineto{\pgfqpoint{4.261630in}{1.988044in}}%
\pgfpathlineto{\pgfqpoint{4.272806in}{1.992541in}}%
\pgfpathlineto{\pgfqpoint{4.266292in}{2.006617in}}%
\pgfpathlineto{\pgfqpoint{4.259785in}{2.020843in}}%
\pgfpathlineto{\pgfqpoint{4.253286in}{2.035206in}}%
\pgfpathlineto{\pgfqpoint{4.246792in}{2.049685in}}%
\pgfpathlineto{\pgfqpoint{4.240304in}{2.064261in}}%
\pgfpathlineto{\pgfqpoint{4.229143in}{2.060133in}}%
\pgfpathlineto{\pgfqpoint{4.217975in}{2.055964in}}%
\pgfpathlineto{\pgfqpoint{4.206801in}{2.051765in}}%
\pgfpathlineto{\pgfqpoint{4.195621in}{2.047547in}}%
\pgfpathlineto{\pgfqpoint{4.184436in}{2.043320in}}%
\pgfpathlineto{\pgfqpoint{4.190904in}{2.028137in}}%
\pgfpathlineto{\pgfqpoint{4.197380in}{2.013136in}}%
\pgfpathlineto{\pgfqpoint{4.203865in}{1.998368in}}%
\pgfpathlineto{\pgfqpoint{4.210361in}{1.983883in}}%
\pgfpathclose%
\pgfusepath{stroke,fill}%
\end{pgfscope}%
\begin{pgfscope}%
\pgfpathrectangle{\pgfqpoint{0.887500in}{0.275000in}}{\pgfqpoint{4.225000in}{4.225000in}}%
\pgfusepath{clip}%
\pgfsetbuttcap%
\pgfsetroundjoin%
\definecolor{currentfill}{rgb}{0.154815,0.493313,0.557840}%
\pgfsetfillcolor{currentfill}%
\pgfsetfillopacity{0.700000}%
\pgfsetlinewidth{0.501875pt}%
\definecolor{currentstroke}{rgb}{1.000000,1.000000,1.000000}%
\pgfsetstrokecolor{currentstroke}%
\pgfsetstrokeopacity{0.500000}%
\pgfsetdash{}{0pt}%
\pgfpathmoveto{\pgfqpoint{3.246542in}{2.121707in}}%
\pgfpathlineto{\pgfqpoint{3.258003in}{2.137793in}}%
\pgfpathlineto{\pgfqpoint{3.269466in}{2.153847in}}%
\pgfpathlineto{\pgfqpoint{3.280931in}{2.169974in}}%
\pgfpathlineto{\pgfqpoint{3.292399in}{2.186257in}}%
\pgfpathlineto{\pgfqpoint{3.303869in}{2.202699in}}%
\pgfpathlineto{\pgfqpoint{3.297553in}{2.218753in}}%
\pgfpathlineto{\pgfqpoint{3.291238in}{2.234548in}}%
\pgfpathlineto{\pgfqpoint{3.284924in}{2.250162in}}%
\pgfpathlineto{\pgfqpoint{3.278611in}{2.265673in}}%
\pgfpathlineto{\pgfqpoint{3.272300in}{2.281158in}}%
\pgfpathlineto{\pgfqpoint{3.260853in}{2.266976in}}%
\pgfpathlineto{\pgfqpoint{3.249404in}{2.252510in}}%
\pgfpathlineto{\pgfqpoint{3.237955in}{2.237581in}}%
\pgfpathlineto{\pgfqpoint{3.226504in}{2.222055in}}%
\pgfpathlineto{\pgfqpoint{3.215051in}{2.205810in}}%
\pgfpathlineto{\pgfqpoint{3.221350in}{2.189940in}}%
\pgfpathlineto{\pgfqpoint{3.227649in}{2.173594in}}%
\pgfpathlineto{\pgfqpoint{3.233947in}{2.156773in}}%
\pgfpathlineto{\pgfqpoint{3.240245in}{2.139477in}}%
\pgfpathclose%
\pgfusepath{stroke,fill}%
\end{pgfscope}%
\begin{pgfscope}%
\pgfpathrectangle{\pgfqpoint{0.887500in}{0.275000in}}{\pgfqpoint{4.225000in}{4.225000in}}%
\pgfusepath{clip}%
\pgfsetbuttcap%
\pgfsetroundjoin%
\definecolor{currentfill}{rgb}{0.204903,0.375746,0.553533}%
\pgfsetfillcolor{currentfill}%
\pgfsetfillopacity{0.700000}%
\pgfsetlinewidth{0.501875pt}%
\definecolor{currentstroke}{rgb}{1.000000,1.000000,1.000000}%
\pgfsetstrokecolor{currentstroke}%
\pgfsetstrokeopacity{0.500000}%
\pgfsetdash{}{0pt}%
\pgfpathmoveto{\pgfqpoint{3.163531in}{1.904500in}}%
\pgfpathlineto{\pgfqpoint{3.174970in}{1.912823in}}%
\pgfpathlineto{\pgfqpoint{3.186410in}{1.922341in}}%
\pgfpathlineto{\pgfqpoint{3.197851in}{1.932907in}}%
\pgfpathlineto{\pgfqpoint{3.209295in}{1.944374in}}%
\pgfpathlineto{\pgfqpoint{3.220741in}{1.956594in}}%
\pgfpathlineto{\pgfqpoint{3.214440in}{1.973119in}}%
\pgfpathlineto{\pgfqpoint{3.208140in}{1.989534in}}%
\pgfpathlineto{\pgfqpoint{3.201841in}{2.005737in}}%
\pgfpathlineto{\pgfqpoint{3.195543in}{2.021646in}}%
\pgfpathlineto{\pgfqpoint{3.189246in}{2.037235in}}%
\pgfpathlineto{\pgfqpoint{3.177791in}{2.019527in}}%
\pgfpathlineto{\pgfqpoint{3.166340in}{2.002316in}}%
\pgfpathlineto{\pgfqpoint{3.154893in}{1.986107in}}%
\pgfpathlineto{\pgfqpoint{3.143452in}{1.971405in}}%
\pgfpathlineto{\pgfqpoint{3.132015in}{1.958713in}}%
\pgfpathlineto{\pgfqpoint{3.138312in}{1.948010in}}%
\pgfpathlineto{\pgfqpoint{3.144611in}{1.937234in}}%
\pgfpathlineto{\pgfqpoint{3.150915in}{1.926385in}}%
\pgfpathlineto{\pgfqpoint{3.157221in}{1.915471in}}%
\pgfpathclose%
\pgfusepath{stroke,fill}%
\end{pgfscope}%
\begin{pgfscope}%
\pgfpathrectangle{\pgfqpoint{0.887500in}{0.275000in}}{\pgfqpoint{4.225000in}{4.225000in}}%
\pgfusepath{clip}%
\pgfsetbuttcap%
\pgfsetroundjoin%
\definecolor{currentfill}{rgb}{0.146180,0.515413,0.556823}%
\pgfsetfillcolor{currentfill}%
\pgfsetfillopacity{0.700000}%
\pgfsetlinewidth{0.501875pt}%
\definecolor{currentstroke}{rgb}{1.000000,1.000000,1.000000}%
\pgfsetstrokecolor{currentstroke}%
\pgfsetstrokeopacity{0.500000}%
\pgfsetdash{}{0pt}%
\pgfpathmoveto{\pgfqpoint{3.919394in}{2.208846in}}%
\pgfpathlineto{\pgfqpoint{3.930663in}{2.213607in}}%
\pgfpathlineto{\pgfqpoint{3.941924in}{2.218205in}}%
\pgfpathlineto{\pgfqpoint{3.953176in}{2.222642in}}%
\pgfpathlineto{\pgfqpoint{3.964419in}{2.226917in}}%
\pgfpathlineto{\pgfqpoint{3.975654in}{2.231032in}}%
\pgfpathlineto{\pgfqpoint{3.969201in}{2.245178in}}%
\pgfpathlineto{\pgfqpoint{3.962752in}{2.259399in}}%
\pgfpathlineto{\pgfqpoint{3.956308in}{2.273728in}}%
\pgfpathlineto{\pgfqpoint{3.949869in}{2.288155in}}%
\pgfpathlineto{\pgfqpoint{3.943434in}{2.302660in}}%
\pgfpathlineto{\pgfqpoint{3.932208in}{2.298802in}}%
\pgfpathlineto{\pgfqpoint{3.920974in}{2.294841in}}%
\pgfpathlineto{\pgfqpoint{3.909733in}{2.290765in}}%
\pgfpathlineto{\pgfqpoint{3.898484in}{2.286566in}}%
\pgfpathlineto{\pgfqpoint{3.887226in}{2.282234in}}%
\pgfpathlineto{\pgfqpoint{3.893650in}{2.267336in}}%
\pgfpathlineto{\pgfqpoint{3.900079in}{2.252528in}}%
\pgfpathlineto{\pgfqpoint{3.906512in}{2.237839in}}%
\pgfpathlineto{\pgfqpoint{3.912951in}{2.223290in}}%
\pgfpathclose%
\pgfusepath{stroke,fill}%
\end{pgfscope}%
\begin{pgfscope}%
\pgfpathrectangle{\pgfqpoint{0.887500in}{0.275000in}}{\pgfqpoint{4.225000in}{4.225000in}}%
\pgfusepath{clip}%
\pgfsetbuttcap%
\pgfsetroundjoin%
\definecolor{currentfill}{rgb}{0.177423,0.437527,0.557565}%
\pgfsetfillcolor{currentfill}%
\pgfsetfillopacity{0.700000}%
\pgfsetlinewidth{0.501875pt}%
\definecolor{currentstroke}{rgb}{1.000000,1.000000,1.000000}%
\pgfsetstrokecolor{currentstroke}%
\pgfsetstrokeopacity{0.500000}%
\pgfsetdash{}{0pt}%
\pgfpathmoveto{\pgfqpoint{2.516526in}{2.056757in}}%
\pgfpathlineto{\pgfqpoint{2.528119in}{2.060316in}}%
\pgfpathlineto{\pgfqpoint{2.539706in}{2.063864in}}%
\pgfpathlineto{\pgfqpoint{2.551288in}{2.067405in}}%
\pgfpathlineto{\pgfqpoint{2.562863in}{2.070943in}}%
\pgfpathlineto{\pgfqpoint{2.574433in}{2.074482in}}%
\pgfpathlineto{\pgfqpoint{2.568308in}{2.083908in}}%
\pgfpathlineto{\pgfqpoint{2.562187in}{2.093295in}}%
\pgfpathlineto{\pgfqpoint{2.556070in}{2.102644in}}%
\pgfpathlineto{\pgfqpoint{2.549958in}{2.111956in}}%
\pgfpathlineto{\pgfqpoint{2.543850in}{2.121231in}}%
\pgfpathlineto{\pgfqpoint{2.532290in}{2.117741in}}%
\pgfpathlineto{\pgfqpoint{2.520725in}{2.114253in}}%
\pgfpathlineto{\pgfqpoint{2.509153in}{2.110764in}}%
\pgfpathlineto{\pgfqpoint{2.497577in}{2.107268in}}%
\pgfpathlineto{\pgfqpoint{2.485994in}{2.103760in}}%
\pgfpathlineto{\pgfqpoint{2.492092in}{2.094436in}}%
\pgfpathlineto{\pgfqpoint{2.498194in}{2.085074in}}%
\pgfpathlineto{\pgfqpoint{2.504301in}{2.075674in}}%
\pgfpathlineto{\pgfqpoint{2.510411in}{2.066236in}}%
\pgfpathclose%
\pgfusepath{stroke,fill}%
\end{pgfscope}%
\begin{pgfscope}%
\pgfpathrectangle{\pgfqpoint{0.887500in}{0.275000in}}{\pgfqpoint{4.225000in}{4.225000in}}%
\pgfusepath{clip}%
\pgfsetbuttcap%
\pgfsetroundjoin%
\definecolor{currentfill}{rgb}{0.199430,0.387607,0.554642}%
\pgfsetfillcolor{currentfill}%
\pgfsetfillopacity{0.700000}%
\pgfsetlinewidth{0.501875pt}%
\definecolor{currentstroke}{rgb}{1.000000,1.000000,1.000000}%
\pgfsetstrokecolor{currentstroke}%
\pgfsetstrokeopacity{0.500000}%
\pgfsetdash{}{0pt}%
\pgfpathmoveto{\pgfqpoint{2.928817in}{1.950091in}}%
\pgfpathlineto{\pgfqpoint{2.940306in}{1.954201in}}%
\pgfpathlineto{\pgfqpoint{2.951789in}{1.958510in}}%
\pgfpathlineto{\pgfqpoint{2.963267in}{1.962818in}}%
\pgfpathlineto{\pgfqpoint{2.974739in}{1.966922in}}%
\pgfpathlineto{\pgfqpoint{2.986207in}{1.970617in}}%
\pgfpathlineto{\pgfqpoint{2.979951in}{1.980779in}}%
\pgfpathlineto{\pgfqpoint{2.973699in}{1.990882in}}%
\pgfpathlineto{\pgfqpoint{2.967450in}{2.000924in}}%
\pgfpathlineto{\pgfqpoint{2.961206in}{2.010905in}}%
\pgfpathlineto{\pgfqpoint{2.954965in}{2.020828in}}%
\pgfpathlineto{\pgfqpoint{2.943506in}{2.017163in}}%
\pgfpathlineto{\pgfqpoint{2.932044in}{2.013027in}}%
\pgfpathlineto{\pgfqpoint{2.920576in}{2.008655in}}%
\pgfpathlineto{\pgfqpoint{2.909104in}{2.004281in}}%
\pgfpathlineto{\pgfqpoint{2.897625in}{2.000134in}}%
\pgfpathlineto{\pgfqpoint{2.903856in}{1.990217in}}%
\pgfpathlineto{\pgfqpoint{2.910091in}{1.980255in}}%
\pgfpathlineto{\pgfqpoint{2.916329in}{1.970246in}}%
\pgfpathlineto{\pgfqpoint{2.922571in}{1.960192in}}%
\pgfpathclose%
\pgfusepath{stroke,fill}%
\end{pgfscope}%
\begin{pgfscope}%
\pgfpathrectangle{\pgfqpoint{0.887500in}{0.275000in}}{\pgfqpoint{4.225000in}{4.225000in}}%
\pgfusepath{clip}%
\pgfsetbuttcap%
\pgfsetroundjoin%
\definecolor{currentfill}{rgb}{0.246811,0.283237,0.535941}%
\pgfsetfillcolor{currentfill}%
\pgfsetfillopacity{0.700000}%
\pgfsetlinewidth{0.501875pt}%
\definecolor{currentstroke}{rgb}{1.000000,1.000000,1.000000}%
\pgfsetstrokecolor{currentstroke}%
\pgfsetstrokeopacity{0.500000}%
\pgfsetdash{}{0pt}%
\pgfpathmoveto{\pgfqpoint{4.514777in}{1.743038in}}%
\pgfpathlineto{\pgfqpoint{4.525859in}{1.746284in}}%
\pgfpathlineto{\pgfqpoint{4.536936in}{1.749522in}}%
\pgfpathlineto{\pgfqpoint{4.548006in}{1.752758in}}%
\pgfpathlineto{\pgfqpoint{4.559071in}{1.755998in}}%
\pgfpathlineto{\pgfqpoint{4.570131in}{1.759246in}}%
\pgfpathlineto{\pgfqpoint{4.563565in}{1.773550in}}%
\pgfpathlineto{\pgfqpoint{4.557007in}{1.787968in}}%
\pgfpathlineto{\pgfqpoint{4.550455in}{1.802486in}}%
\pgfpathlineto{\pgfqpoint{4.543909in}{1.817090in}}%
\pgfpathlineto{\pgfqpoint{4.537368in}{1.831766in}}%
\pgfpathlineto{\pgfqpoint{4.526307in}{1.828456in}}%
\pgfpathlineto{\pgfqpoint{4.515243in}{1.825206in}}%
\pgfpathlineto{\pgfqpoint{4.504175in}{1.822018in}}%
\pgfpathlineto{\pgfqpoint{4.493104in}{1.818881in}}%
\pgfpathlineto{\pgfqpoint{4.482028in}{1.815786in}}%
\pgfpathlineto{\pgfqpoint{4.488569in}{1.801137in}}%
\pgfpathlineto{\pgfqpoint{4.495114in}{1.786527in}}%
\pgfpathlineto{\pgfqpoint{4.501663in}{1.771967in}}%
\pgfpathlineto{\pgfqpoint{4.508218in}{1.757467in}}%
\pgfpathclose%
\pgfusepath{stroke,fill}%
\end{pgfscope}%
\begin{pgfscope}%
\pgfpathrectangle{\pgfqpoint{0.887500in}{0.275000in}}{\pgfqpoint{4.225000in}{4.225000in}}%
\pgfusepath{clip}%
\pgfsetbuttcap%
\pgfsetroundjoin%
\definecolor{currentfill}{rgb}{0.187231,0.414746,0.556547}%
\pgfsetfillcolor{currentfill}%
\pgfsetfillopacity{0.700000}%
\pgfsetlinewidth{0.501875pt}%
\definecolor{currentstroke}{rgb}{1.000000,1.000000,1.000000}%
\pgfsetstrokecolor{currentstroke}%
\pgfsetstrokeopacity{0.500000}%
\pgfsetdash{}{0pt}%
\pgfpathmoveto{\pgfqpoint{3.220741in}{1.956594in}}%
\pgfpathlineto{\pgfqpoint{3.232188in}{1.969421in}}%
\pgfpathlineto{\pgfqpoint{3.243638in}{1.982740in}}%
\pgfpathlineto{\pgfqpoint{3.255091in}{1.996601in}}%
\pgfpathlineto{\pgfqpoint{3.266548in}{2.011101in}}%
\pgfpathlineto{\pgfqpoint{3.278009in}{2.026342in}}%
\pgfpathlineto{\pgfqpoint{3.271717in}{2.046054in}}%
\pgfpathlineto{\pgfqpoint{3.265425in}{2.065573in}}%
\pgfpathlineto{\pgfqpoint{3.259132in}{2.084747in}}%
\pgfpathlineto{\pgfqpoint{3.252838in}{2.103464in}}%
\pgfpathlineto{\pgfqpoint{3.246542in}{2.121707in}}%
\pgfpathlineto{\pgfqpoint{3.235081in}{2.105484in}}%
\pgfpathlineto{\pgfqpoint{3.223622in}{2.089017in}}%
\pgfpathlineto{\pgfqpoint{3.212163in}{2.072203in}}%
\pgfpathlineto{\pgfqpoint{3.200704in}{2.054935in}}%
\pgfpathlineto{\pgfqpoint{3.189246in}{2.037235in}}%
\pgfpathlineto{\pgfqpoint{3.195543in}{2.021646in}}%
\pgfpathlineto{\pgfqpoint{3.201841in}{2.005737in}}%
\pgfpathlineto{\pgfqpoint{3.208140in}{1.989534in}}%
\pgfpathlineto{\pgfqpoint{3.214440in}{1.973119in}}%
\pgfpathclose%
\pgfusepath{stroke,fill}%
\end{pgfscope}%
\begin{pgfscope}%
\pgfpathrectangle{\pgfqpoint{0.887500in}{0.275000in}}{\pgfqpoint{4.225000in}{4.225000in}}%
\pgfusepath{clip}%
\pgfsetbuttcap%
\pgfsetroundjoin%
\definecolor{currentfill}{rgb}{0.163625,0.471133,0.558148}%
\pgfsetfillcolor{currentfill}%
\pgfsetfillopacity{0.700000}%
\pgfsetlinewidth{0.501875pt}%
\definecolor{currentstroke}{rgb}{1.000000,1.000000,1.000000}%
\pgfsetstrokecolor{currentstroke}%
\pgfsetstrokeopacity{0.500000}%
\pgfsetdash{}{0pt}%
\pgfpathmoveto{\pgfqpoint{2.192867in}{2.124126in}}%
\pgfpathlineto{\pgfqpoint{2.204540in}{2.127696in}}%
\pgfpathlineto{\pgfqpoint{2.216208in}{2.131267in}}%
\pgfpathlineto{\pgfqpoint{2.227870in}{2.134841in}}%
\pgfpathlineto{\pgfqpoint{2.239526in}{2.138422in}}%
\pgfpathlineto{\pgfqpoint{2.251176in}{2.142012in}}%
\pgfpathlineto{\pgfqpoint{2.245158in}{2.151127in}}%
\pgfpathlineto{\pgfqpoint{2.239143in}{2.160214in}}%
\pgfpathlineto{\pgfqpoint{2.233134in}{2.169272in}}%
\pgfpathlineto{\pgfqpoint{2.227128in}{2.178301in}}%
\pgfpathlineto{\pgfqpoint{2.221127in}{2.187301in}}%
\pgfpathlineto{\pgfqpoint{2.209488in}{2.183763in}}%
\pgfpathlineto{\pgfqpoint{2.197842in}{2.180233in}}%
\pgfpathlineto{\pgfqpoint{2.186191in}{2.176711in}}%
\pgfpathlineto{\pgfqpoint{2.174534in}{2.173193in}}%
\pgfpathlineto{\pgfqpoint{2.162871in}{2.169675in}}%
\pgfpathlineto{\pgfqpoint{2.168861in}{2.160624in}}%
\pgfpathlineto{\pgfqpoint{2.174856in}{2.151543in}}%
\pgfpathlineto{\pgfqpoint{2.180855in}{2.142433in}}%
\pgfpathlineto{\pgfqpoint{2.186859in}{2.133295in}}%
\pgfpathclose%
\pgfusepath{stroke,fill}%
\end{pgfscope}%
\begin{pgfscope}%
\pgfpathrectangle{\pgfqpoint{0.887500in}{0.275000in}}{\pgfqpoint{4.225000in}{4.225000in}}%
\pgfusepath{clip}%
\pgfsetbuttcap%
\pgfsetroundjoin%
\definecolor{currentfill}{rgb}{0.151918,0.500685,0.557587}%
\pgfsetfillcolor{currentfill}%
\pgfsetfillopacity{0.700000}%
\pgfsetlinewidth{0.501875pt}%
\definecolor{currentstroke}{rgb}{1.000000,1.000000,1.000000}%
\pgfsetstrokecolor{currentstroke}%
\pgfsetstrokeopacity{0.500000}%
\pgfsetdash{}{0pt}%
\pgfpathmoveto{\pgfqpoint{1.869203in}{2.188697in}}%
\pgfpathlineto{\pgfqpoint{1.880957in}{2.192246in}}%
\pgfpathlineto{\pgfqpoint{1.892705in}{2.195793in}}%
\pgfpathlineto{\pgfqpoint{1.904448in}{2.199339in}}%
\pgfpathlineto{\pgfqpoint{1.916185in}{2.202885in}}%
\pgfpathlineto{\pgfqpoint{1.927916in}{2.206432in}}%
\pgfpathlineto{\pgfqpoint{1.922007in}{2.215333in}}%
\pgfpathlineto{\pgfqpoint{1.916102in}{2.224208in}}%
\pgfpathlineto{\pgfqpoint{1.910201in}{2.233056in}}%
\pgfpathlineto{\pgfqpoint{1.904306in}{2.241877in}}%
\pgfpathlineto{\pgfqpoint{1.898414in}{2.250672in}}%
\pgfpathlineto{\pgfqpoint{1.886694in}{2.247169in}}%
\pgfpathlineto{\pgfqpoint{1.874969in}{2.243667in}}%
\pgfpathlineto{\pgfqpoint{1.863237in}{2.240166in}}%
\pgfpathlineto{\pgfqpoint{1.851500in}{2.236665in}}%
\pgfpathlineto{\pgfqpoint{1.839757in}{2.233162in}}%
\pgfpathlineto{\pgfqpoint{1.845637in}{2.224322in}}%
\pgfpathlineto{\pgfqpoint{1.851521in}{2.215457in}}%
\pgfpathlineto{\pgfqpoint{1.857411in}{2.206564in}}%
\pgfpathlineto{\pgfqpoint{1.863304in}{2.197644in}}%
\pgfpathclose%
\pgfusepath{stroke,fill}%
\end{pgfscope}%
\begin{pgfscope}%
\pgfpathrectangle{\pgfqpoint{0.887500in}{0.275000in}}{\pgfqpoint{4.225000in}{4.225000in}}%
\pgfusepath{clip}%
\pgfsetbuttcap%
\pgfsetroundjoin%
\definecolor{currentfill}{rgb}{0.206756,0.371758,0.553117}%
\pgfsetfillcolor{currentfill}%
\pgfsetfillopacity{0.700000}%
\pgfsetlinewidth{0.501875pt}%
\definecolor{currentstroke}{rgb}{1.000000,1.000000,1.000000}%
\pgfsetstrokecolor{currentstroke}%
\pgfsetstrokeopacity{0.500000}%
\pgfsetdash{}{0pt}%
\pgfpathmoveto{\pgfqpoint{3.017545in}{1.918928in}}%
\pgfpathlineto{\pgfqpoint{3.029017in}{1.922204in}}%
\pgfpathlineto{\pgfqpoint{3.040483in}{1.924891in}}%
\pgfpathlineto{\pgfqpoint{3.051943in}{1.926871in}}%
\pgfpathlineto{\pgfqpoint{3.063397in}{1.928422in}}%
\pgfpathlineto{\pgfqpoint{3.074844in}{1.930077in}}%
\pgfpathlineto{\pgfqpoint{3.068561in}{1.939853in}}%
\pgfpathlineto{\pgfqpoint{3.062282in}{1.949609in}}%
\pgfpathlineto{\pgfqpoint{3.056007in}{1.959349in}}%
\pgfpathlineto{\pgfqpoint{3.049735in}{1.969076in}}%
\pgfpathlineto{\pgfqpoint{3.043468in}{1.978791in}}%
\pgfpathlineto{\pgfqpoint{3.032029in}{1.977951in}}%
\pgfpathlineto{\pgfqpoint{3.020582in}{1.977255in}}%
\pgfpathlineto{\pgfqpoint{3.009129in}{1.975964in}}%
\pgfpathlineto{\pgfqpoint{2.997671in}{1.973699in}}%
\pgfpathlineto{\pgfqpoint{2.986207in}{1.970617in}}%
\pgfpathlineto{\pgfqpoint{2.992467in}{1.960395in}}%
\pgfpathlineto{\pgfqpoint{2.998731in}{1.950115in}}%
\pgfpathlineto{\pgfqpoint{3.004999in}{1.939776in}}%
\pgfpathlineto{\pgfqpoint{3.011270in}{1.929381in}}%
\pgfpathclose%
\pgfusepath{stroke,fill}%
\end{pgfscope}%
\begin{pgfscope}%
\pgfpathrectangle{\pgfqpoint{0.887500in}{0.275000in}}{\pgfqpoint{4.225000in}{4.225000in}}%
\pgfusepath{clip}%
\pgfsetbuttcap%
\pgfsetroundjoin%
\definecolor{currentfill}{rgb}{0.136408,0.541173,0.554483}%
\pgfsetfillcolor{currentfill}%
\pgfsetfillopacity{0.700000}%
\pgfsetlinewidth{0.501875pt}%
\definecolor{currentstroke}{rgb}{1.000000,1.000000,1.000000}%
\pgfsetstrokecolor{currentstroke}%
\pgfsetstrokeopacity{0.500000}%
\pgfsetdash{}{0pt}%
\pgfpathmoveto{\pgfqpoint{3.830835in}{2.259059in}}%
\pgfpathlineto{\pgfqpoint{3.842124in}{2.263741in}}%
\pgfpathlineto{\pgfqpoint{3.853409in}{2.268458in}}%
\pgfpathlineto{\pgfqpoint{3.864688in}{2.273151in}}%
\pgfpathlineto{\pgfqpoint{3.875961in}{2.277761in}}%
\pgfpathlineto{\pgfqpoint{3.887226in}{2.282234in}}%
\pgfpathlineto{\pgfqpoint{3.880806in}{2.297194in}}%
\pgfpathlineto{\pgfqpoint{3.874389in}{2.312187in}}%
\pgfpathlineto{\pgfqpoint{3.867974in}{2.327183in}}%
\pgfpathlineto{\pgfqpoint{3.861561in}{2.342154in}}%
\pgfpathlineto{\pgfqpoint{3.855149in}{2.357070in}}%
\pgfpathlineto{\pgfqpoint{3.843894in}{2.353040in}}%
\pgfpathlineto{\pgfqpoint{3.832632in}{2.348874in}}%
\pgfpathlineto{\pgfqpoint{3.821363in}{2.344585in}}%
\pgfpathlineto{\pgfqpoint{3.810086in}{2.340188in}}%
\pgfpathlineto{\pgfqpoint{3.798803in}{2.335702in}}%
\pgfpathlineto{\pgfqpoint{3.805207in}{2.320469in}}%
\pgfpathlineto{\pgfqpoint{3.811611in}{2.305157in}}%
\pgfpathlineto{\pgfqpoint{3.818017in}{2.289798in}}%
\pgfpathlineto{\pgfqpoint{3.824425in}{2.274422in}}%
\pgfpathclose%
\pgfusepath{stroke,fill}%
\end{pgfscope}%
\begin{pgfscope}%
\pgfpathrectangle{\pgfqpoint{0.887500in}{0.275000in}}{\pgfqpoint{4.225000in}{4.225000in}}%
\pgfusepath{clip}%
\pgfsetbuttcap%
\pgfsetroundjoin%
\definecolor{currentfill}{rgb}{0.140536,0.530132,0.555659}%
\pgfsetfillcolor{currentfill}%
\pgfsetfillopacity{0.700000}%
\pgfsetlinewidth{0.501875pt}%
\definecolor{currentstroke}{rgb}{1.000000,1.000000,1.000000}%
\pgfsetstrokecolor{currentstroke}%
\pgfsetstrokeopacity{0.500000}%
\pgfsetdash{}{0pt}%
\pgfpathmoveto{\pgfqpoint{3.303869in}{2.202699in}}%
\pgfpathlineto{\pgfqpoint{3.315342in}{2.219291in}}%
\pgfpathlineto{\pgfqpoint{3.326818in}{2.236020in}}%
\pgfpathlineto{\pgfqpoint{3.338297in}{2.252875in}}%
\pgfpathlineto{\pgfqpoint{3.349779in}{2.269846in}}%
\pgfpathlineto{\pgfqpoint{3.361264in}{2.286920in}}%
\pgfpathlineto{\pgfqpoint{3.354916in}{2.300121in}}%
\pgfpathlineto{\pgfqpoint{3.348570in}{2.313310in}}%
\pgfpathlineto{\pgfqpoint{3.342229in}{2.326651in}}%
\pgfpathlineto{\pgfqpoint{3.335892in}{2.340305in}}%
\pgfpathlineto{\pgfqpoint{3.329561in}{2.354433in}}%
\pgfpathlineto{\pgfqpoint{3.318101in}{2.338831in}}%
\pgfpathlineto{\pgfqpoint{3.306646in}{2.323892in}}%
\pgfpathlineto{\pgfqpoint{3.295195in}{2.309427in}}%
\pgfpathlineto{\pgfqpoint{3.283747in}{2.295245in}}%
\pgfpathlineto{\pgfqpoint{3.272300in}{2.281158in}}%
\pgfpathlineto{\pgfqpoint{3.278611in}{2.265673in}}%
\pgfpathlineto{\pgfqpoint{3.284924in}{2.250162in}}%
\pgfpathlineto{\pgfqpoint{3.291238in}{2.234548in}}%
\pgfpathlineto{\pgfqpoint{3.297553in}{2.218753in}}%
\pgfpathclose%
\pgfusepath{stroke,fill}%
\end{pgfscope}%
\begin{pgfscope}%
\pgfpathrectangle{\pgfqpoint{0.887500in}{0.275000in}}{\pgfqpoint{4.225000in}{4.225000in}}%
\pgfusepath{clip}%
\pgfsetbuttcap%
\pgfsetroundjoin%
\definecolor{currentfill}{rgb}{0.180629,0.429975,0.557282}%
\pgfsetfillcolor{currentfill}%
\pgfsetfillopacity{0.700000}%
\pgfsetlinewidth{0.501875pt}%
\definecolor{currentstroke}{rgb}{1.000000,1.000000,1.000000}%
\pgfsetstrokecolor{currentstroke}%
\pgfsetstrokeopacity{0.500000}%
\pgfsetdash{}{0pt}%
\pgfpathmoveto{\pgfqpoint{4.128438in}{2.022302in}}%
\pgfpathlineto{\pgfqpoint{4.139648in}{2.026497in}}%
\pgfpathlineto{\pgfqpoint{4.150852in}{2.030683in}}%
\pgfpathlineto{\pgfqpoint{4.162051in}{2.034879in}}%
\pgfpathlineto{\pgfqpoint{4.173246in}{2.039094in}}%
\pgfpathlineto{\pgfqpoint{4.184436in}{2.043320in}}%
\pgfpathlineto{\pgfqpoint{4.177974in}{2.058636in}}%
\pgfpathlineto{\pgfqpoint{4.171516in}{2.074035in}}%
\pgfpathlineto{\pgfqpoint{4.165062in}{2.089466in}}%
\pgfpathlineto{\pgfqpoint{4.158609in}{2.104881in}}%
\pgfpathlineto{\pgfqpoint{4.152157in}{2.120227in}}%
\pgfpathlineto{\pgfqpoint{4.140987in}{2.116667in}}%
\pgfpathlineto{\pgfqpoint{4.129811in}{2.113107in}}%
\pgfpathlineto{\pgfqpoint{4.118629in}{2.109550in}}%
\pgfpathlineto{\pgfqpoint{4.107442in}{2.105978in}}%
\pgfpathlineto{\pgfqpoint{4.096249in}{2.102367in}}%
\pgfpathlineto{\pgfqpoint{4.102685in}{2.086431in}}%
\pgfpathlineto{\pgfqpoint{4.109119in}{2.070373in}}%
\pgfpathlineto{\pgfqpoint{4.115555in}{2.054276in}}%
\pgfpathlineto{\pgfqpoint{4.121994in}{2.038225in}}%
\pgfpathclose%
\pgfusepath{stroke,fill}%
\end{pgfscope}%
\begin{pgfscope}%
\pgfpathrectangle{\pgfqpoint{0.887500in}{0.275000in}}{\pgfqpoint{4.225000in}{4.225000in}}%
\pgfusepath{clip}%
\pgfsetbuttcap%
\pgfsetroundjoin%
\definecolor{currentfill}{rgb}{0.182256,0.426184,0.557120}%
\pgfsetfillcolor{currentfill}%
\pgfsetfillopacity{0.700000}%
\pgfsetlinewidth{0.501875pt}%
\definecolor{currentstroke}{rgb}{1.000000,1.000000,1.000000}%
\pgfsetstrokecolor{currentstroke}%
\pgfsetstrokeopacity{0.500000}%
\pgfsetdash{}{0pt}%
\pgfpathmoveto{\pgfqpoint{2.605124in}{2.026732in}}%
\pgfpathlineto{\pgfqpoint{2.616698in}{2.030330in}}%
\pgfpathlineto{\pgfqpoint{2.628266in}{2.033935in}}%
\pgfpathlineto{\pgfqpoint{2.639829in}{2.037549in}}%
\pgfpathlineto{\pgfqpoint{2.651386in}{2.041175in}}%
\pgfpathlineto{\pgfqpoint{2.662937in}{2.044814in}}%
\pgfpathlineto{\pgfqpoint{2.656780in}{2.054400in}}%
\pgfpathlineto{\pgfqpoint{2.650627in}{2.063942in}}%
\pgfpathlineto{\pgfqpoint{2.644479in}{2.073442in}}%
\pgfpathlineto{\pgfqpoint{2.638335in}{2.082901in}}%
\pgfpathlineto{\pgfqpoint{2.632196in}{2.092319in}}%
\pgfpathlineto{\pgfqpoint{2.620655in}{2.088724in}}%
\pgfpathlineto{\pgfqpoint{2.609108in}{2.085146in}}%
\pgfpathlineto{\pgfqpoint{2.597556in}{2.081581in}}%
\pgfpathlineto{\pgfqpoint{2.585997in}{2.078027in}}%
\pgfpathlineto{\pgfqpoint{2.574433in}{2.074482in}}%
\pgfpathlineto{\pgfqpoint{2.580563in}{2.065016in}}%
\pgfpathlineto{\pgfqpoint{2.586697in}{2.055509in}}%
\pgfpathlineto{\pgfqpoint{2.592835in}{2.045960in}}%
\pgfpathlineto{\pgfqpoint{2.598977in}{2.036368in}}%
\pgfpathclose%
\pgfusepath{stroke,fill}%
\end{pgfscope}%
\begin{pgfscope}%
\pgfpathrectangle{\pgfqpoint{0.887500in}{0.275000in}}{\pgfqpoint{4.225000in}{4.225000in}}%
\pgfusepath{clip}%
\pgfsetbuttcap%
\pgfsetroundjoin%
\definecolor{currentfill}{rgb}{0.214298,0.355619,0.551184}%
\pgfsetfillcolor{currentfill}%
\pgfsetfillopacity{0.700000}%
\pgfsetlinewidth{0.501875pt}%
\definecolor{currentstroke}{rgb}{1.000000,1.000000,1.000000}%
\pgfsetstrokecolor{currentstroke}%
\pgfsetstrokeopacity{0.500000}%
\pgfsetdash{}{0pt}%
\pgfpathmoveto{\pgfqpoint{3.106316in}{1.880792in}}%
\pgfpathlineto{\pgfqpoint{3.117766in}{1.883815in}}%
\pgfpathlineto{\pgfqpoint{3.129212in}{1.887430in}}%
\pgfpathlineto{\pgfqpoint{3.140653in}{1.891908in}}%
\pgfpathlineto{\pgfqpoint{3.152092in}{1.897519in}}%
\pgfpathlineto{\pgfqpoint{3.163531in}{1.904500in}}%
\pgfpathlineto{\pgfqpoint{3.157221in}{1.915471in}}%
\pgfpathlineto{\pgfqpoint{3.150915in}{1.926385in}}%
\pgfpathlineto{\pgfqpoint{3.144611in}{1.937234in}}%
\pgfpathlineto{\pgfqpoint{3.138312in}{1.948010in}}%
\pgfpathlineto{\pgfqpoint{3.132015in}{1.958713in}}%
\pgfpathlineto{\pgfqpoint{3.120583in}{1.948533in}}%
\pgfpathlineto{\pgfqpoint{3.109152in}{1.941066in}}%
\pgfpathlineto{\pgfqpoint{3.097720in}{1.935861in}}%
\pgfpathlineto{\pgfqpoint{3.086284in}{1.932377in}}%
\pgfpathlineto{\pgfqpoint{3.074844in}{1.930077in}}%
\pgfpathlineto{\pgfqpoint{3.081131in}{1.920280in}}%
\pgfpathlineto{\pgfqpoint{3.087421in}{1.910459in}}%
\pgfpathlineto{\pgfqpoint{3.093716in}{1.900609in}}%
\pgfpathlineto{\pgfqpoint{3.100014in}{1.890724in}}%
\pgfpathclose%
\pgfusepath{stroke,fill}%
\end{pgfscope}%
\begin{pgfscope}%
\pgfpathrectangle{\pgfqpoint{0.887500in}{0.275000in}}{\pgfqpoint{4.225000in}{4.225000in}}%
\pgfusepath{clip}%
\pgfsetbuttcap%
\pgfsetroundjoin%
\definecolor{currentfill}{rgb}{0.231674,0.318106,0.544834}%
\pgfsetfillcolor{currentfill}%
\pgfsetfillopacity{0.700000}%
\pgfsetlinewidth{0.501875pt}%
\definecolor{currentstroke}{rgb}{1.000000,1.000000,1.000000}%
\pgfsetstrokecolor{currentstroke}%
\pgfsetstrokeopacity{0.500000}%
\pgfsetdash{}{0pt}%
\pgfpathmoveto{\pgfqpoint{4.426574in}{1.800556in}}%
\pgfpathlineto{\pgfqpoint{4.437676in}{1.803604in}}%
\pgfpathlineto{\pgfqpoint{4.448772in}{1.806641in}}%
\pgfpathlineto{\pgfqpoint{4.459863in}{1.809677in}}%
\pgfpathlineto{\pgfqpoint{4.470948in}{1.812722in}}%
\pgfpathlineto{\pgfqpoint{4.482028in}{1.815786in}}%
\pgfpathlineto{\pgfqpoint{4.475491in}{1.830465in}}%
\pgfpathlineto{\pgfqpoint{4.468957in}{1.845161in}}%
\pgfpathlineto{\pgfqpoint{4.462426in}{1.859866in}}%
\pgfpathlineto{\pgfqpoint{4.455897in}{1.874568in}}%
\pgfpathlineto{\pgfqpoint{4.449370in}{1.889256in}}%
\pgfpathlineto{\pgfqpoint{4.438283in}{1.885981in}}%
\pgfpathlineto{\pgfqpoint{4.427191in}{1.882701in}}%
\pgfpathlineto{\pgfqpoint{4.416092in}{1.879411in}}%
\pgfpathlineto{\pgfqpoint{4.404988in}{1.876106in}}%
\pgfpathlineto{\pgfqpoint{4.393877in}{1.872779in}}%
\pgfpathlineto{\pgfqpoint{4.400415in}{1.858457in}}%
\pgfpathlineto{\pgfqpoint{4.406954in}{1.844079in}}%
\pgfpathlineto{\pgfqpoint{4.413494in}{1.829639in}}%
\pgfpathlineto{\pgfqpoint{4.420033in}{1.815133in}}%
\pgfpathclose%
\pgfusepath{stroke,fill}%
\end{pgfscope}%
\begin{pgfscope}%
\pgfpathrectangle{\pgfqpoint{0.887500in}{0.275000in}}{\pgfqpoint{4.225000in}{4.225000in}}%
\pgfusepath{clip}%
\pgfsetbuttcap%
\pgfsetroundjoin%
\definecolor{currentfill}{rgb}{0.127568,0.566949,0.550556}%
\pgfsetfillcolor{currentfill}%
\pgfsetfillopacity{0.700000}%
\pgfsetlinewidth{0.501875pt}%
\definecolor{currentstroke}{rgb}{1.000000,1.000000,1.000000}%
\pgfsetstrokecolor{currentstroke}%
\pgfsetstrokeopacity{0.500000}%
\pgfsetdash{}{0pt}%
\pgfpathmoveto{\pgfqpoint{3.742294in}{2.312264in}}%
\pgfpathlineto{\pgfqpoint{3.753609in}{2.317155in}}%
\pgfpathlineto{\pgfqpoint{3.764916in}{2.321874in}}%
\pgfpathlineto{\pgfqpoint{3.776217in}{2.326528in}}%
\pgfpathlineto{\pgfqpoint{3.787513in}{2.331143in}}%
\pgfpathlineto{\pgfqpoint{3.798803in}{2.335702in}}%
\pgfpathlineto{\pgfqpoint{3.792399in}{2.350826in}}%
\pgfpathlineto{\pgfqpoint{3.785995in}{2.365809in}}%
\pgfpathlineto{\pgfqpoint{3.779590in}{2.380623in}}%
\pgfpathlineto{\pgfqpoint{3.773184in}{2.395268in}}%
\pgfpathlineto{\pgfqpoint{3.766779in}{2.409763in}}%
\pgfpathlineto{\pgfqpoint{3.755493in}{2.405309in}}%
\pgfpathlineto{\pgfqpoint{3.744200in}{2.400700in}}%
\pgfpathlineto{\pgfqpoint{3.732900in}{2.395929in}}%
\pgfpathlineto{\pgfqpoint{3.721591in}{2.390979in}}%
\pgfpathlineto{\pgfqpoint{3.710275in}{2.385813in}}%
\pgfpathlineto{\pgfqpoint{3.716674in}{2.371142in}}%
\pgfpathlineto{\pgfqpoint{3.723076in}{2.356452in}}%
\pgfpathlineto{\pgfqpoint{3.729480in}{2.341745in}}%
\pgfpathlineto{\pgfqpoint{3.735886in}{2.327019in}}%
\pgfpathclose%
\pgfusepath{stroke,fill}%
\end{pgfscope}%
\begin{pgfscope}%
\pgfpathrectangle{\pgfqpoint{0.887500in}{0.275000in}}{\pgfqpoint{4.225000in}{4.225000in}}%
\pgfusepath{clip}%
\pgfsetbuttcap%
\pgfsetroundjoin%
\definecolor{currentfill}{rgb}{0.169646,0.456262,0.558030}%
\pgfsetfillcolor{currentfill}%
\pgfsetfillopacity{0.700000}%
\pgfsetlinewidth{0.501875pt}%
\definecolor{currentstroke}{rgb}{1.000000,1.000000,1.000000}%
\pgfsetstrokecolor{currentstroke}%
\pgfsetstrokeopacity{0.500000}%
\pgfsetdash{}{0pt}%
\pgfpathmoveto{\pgfqpoint{3.278009in}{2.026342in}}%
\pgfpathlineto{\pgfqpoint{3.289477in}{2.042423in}}%
\pgfpathlineto{\pgfqpoint{3.300952in}{2.059444in}}%
\pgfpathlineto{\pgfqpoint{3.312436in}{2.077506in}}%
\pgfpathlineto{\pgfqpoint{3.323930in}{2.096620in}}%
\pgfpathlineto{\pgfqpoint{3.335433in}{2.116511in}}%
\pgfpathlineto{\pgfqpoint{3.329123in}{2.134529in}}%
\pgfpathlineto{\pgfqpoint{3.322811in}{2.152227in}}%
\pgfpathlineto{\pgfqpoint{3.316499in}{2.169505in}}%
\pgfpathlineto{\pgfqpoint{3.310184in}{2.186309in}}%
\pgfpathlineto{\pgfqpoint{3.303869in}{2.202699in}}%
\pgfpathlineto{\pgfqpoint{3.292399in}{2.186257in}}%
\pgfpathlineto{\pgfqpoint{3.280931in}{2.169974in}}%
\pgfpathlineto{\pgfqpoint{3.269466in}{2.153847in}}%
\pgfpathlineto{\pgfqpoint{3.258003in}{2.137793in}}%
\pgfpathlineto{\pgfqpoint{3.246542in}{2.121707in}}%
\pgfpathlineto{\pgfqpoint{3.252838in}{2.103464in}}%
\pgfpathlineto{\pgfqpoint{3.259132in}{2.084747in}}%
\pgfpathlineto{\pgfqpoint{3.265425in}{2.065573in}}%
\pgfpathlineto{\pgfqpoint{3.271717in}{2.046054in}}%
\pgfpathclose%
\pgfusepath{stroke,fill}%
\end{pgfscope}%
\begin{pgfscope}%
\pgfpathrectangle{\pgfqpoint{0.887500in}{0.275000in}}{\pgfqpoint{4.225000in}{4.225000in}}%
\pgfusepath{clip}%
\pgfsetbuttcap%
\pgfsetroundjoin%
\definecolor{currentfill}{rgb}{0.168126,0.459988,0.558082}%
\pgfsetfillcolor{currentfill}%
\pgfsetfillopacity{0.700000}%
\pgfsetlinewidth{0.501875pt}%
\definecolor{currentstroke}{rgb}{1.000000,1.000000,1.000000}%
\pgfsetstrokecolor{currentstroke}%
\pgfsetstrokeopacity{0.500000}%
\pgfsetdash{}{0pt}%
\pgfpathmoveto{\pgfqpoint{2.281334in}{2.095976in}}%
\pgfpathlineto{\pgfqpoint{2.292989in}{2.099625in}}%
\pgfpathlineto{\pgfqpoint{2.304638in}{2.103277in}}%
\pgfpathlineto{\pgfqpoint{2.316281in}{2.106930in}}%
\pgfpathlineto{\pgfqpoint{2.327919in}{2.110581in}}%
\pgfpathlineto{\pgfqpoint{2.339551in}{2.114226in}}%
\pgfpathlineto{\pgfqpoint{2.333500in}{2.123445in}}%
\pgfpathlineto{\pgfqpoint{2.327453in}{2.132633in}}%
\pgfpathlineto{\pgfqpoint{2.321411in}{2.141789in}}%
\pgfpathlineto{\pgfqpoint{2.315373in}{2.150914in}}%
\pgfpathlineto{\pgfqpoint{2.309340in}{2.160009in}}%
\pgfpathlineto{\pgfqpoint{2.297718in}{2.156414in}}%
\pgfpathlineto{\pgfqpoint{2.286091in}{2.152813in}}%
\pgfpathlineto{\pgfqpoint{2.274459in}{2.149211in}}%
\pgfpathlineto{\pgfqpoint{2.262820in}{2.145609in}}%
\pgfpathlineto{\pgfqpoint{2.251176in}{2.142012in}}%
\pgfpathlineto{\pgfqpoint{2.257199in}{2.132867in}}%
\pgfpathlineto{\pgfqpoint{2.263226in}{2.123691in}}%
\pgfpathlineto{\pgfqpoint{2.269258in}{2.114485in}}%
\pgfpathlineto{\pgfqpoint{2.275294in}{2.105247in}}%
\pgfpathclose%
\pgfusepath{stroke,fill}%
\end{pgfscope}%
\begin{pgfscope}%
\pgfpathrectangle{\pgfqpoint{0.887500in}{0.275000in}}{\pgfqpoint{4.225000in}{4.225000in}}%
\pgfusepath{clip}%
\pgfsetbuttcap%
\pgfsetroundjoin%
\definecolor{currentfill}{rgb}{0.121380,0.629492,0.531973}%
\pgfsetfillcolor{currentfill}%
\pgfsetfillopacity{0.700000}%
\pgfsetlinewidth{0.501875pt}%
\definecolor{currentstroke}{rgb}{1.000000,1.000000,1.000000}%
\pgfsetstrokecolor{currentstroke}%
\pgfsetstrokeopacity{0.500000}%
\pgfsetdash{}{0pt}%
\pgfpathmoveto{\pgfqpoint{3.475984in}{2.433985in}}%
\pgfpathlineto{\pgfqpoint{3.487411in}{2.444091in}}%
\pgfpathlineto{\pgfqpoint{3.498827in}{2.453447in}}%
\pgfpathlineto{\pgfqpoint{3.510234in}{2.462130in}}%
\pgfpathlineto{\pgfqpoint{3.521631in}{2.470218in}}%
\pgfpathlineto{\pgfqpoint{3.533018in}{2.477768in}}%
\pgfpathlineto{\pgfqpoint{3.526650in}{2.492300in}}%
\pgfpathlineto{\pgfqpoint{3.520281in}{2.506578in}}%
\pgfpathlineto{\pgfqpoint{3.513913in}{2.520669in}}%
\pgfpathlineto{\pgfqpoint{3.507546in}{2.534646in}}%
\pgfpathlineto{\pgfqpoint{3.501181in}{2.548577in}}%
\pgfpathlineto{\pgfqpoint{3.489797in}{2.540802in}}%
\pgfpathlineto{\pgfqpoint{3.478405in}{2.532541in}}%
\pgfpathlineto{\pgfqpoint{3.467004in}{2.523731in}}%
\pgfpathlineto{\pgfqpoint{3.455594in}{2.514259in}}%
\pgfpathlineto{\pgfqpoint{3.444175in}{2.504011in}}%
\pgfpathlineto{\pgfqpoint{3.450534in}{2.490213in}}%
\pgfpathlineto{\pgfqpoint{3.456895in}{2.476426in}}%
\pgfpathlineto{\pgfqpoint{3.463258in}{2.462534in}}%
\pgfpathlineto{\pgfqpoint{3.469622in}{2.448425in}}%
\pgfpathclose%
\pgfusepath{stroke,fill}%
\end{pgfscope}%
\begin{pgfscope}%
\pgfpathrectangle{\pgfqpoint{0.887500in}{0.275000in}}{\pgfqpoint{4.225000in}{4.225000in}}%
\pgfusepath{clip}%
\pgfsetbuttcap%
\pgfsetroundjoin%
\definecolor{currentfill}{rgb}{0.156270,0.489624,0.557936}%
\pgfsetfillcolor{currentfill}%
\pgfsetfillopacity{0.700000}%
\pgfsetlinewidth{0.501875pt}%
\definecolor{currentstroke}{rgb}{1.000000,1.000000,1.000000}%
\pgfsetstrokecolor{currentstroke}%
\pgfsetstrokeopacity{0.500000}%
\pgfsetdash{}{0pt}%
\pgfpathmoveto{\pgfqpoint{1.957529in}{2.161516in}}%
\pgfpathlineto{\pgfqpoint{1.969265in}{2.165109in}}%
\pgfpathlineto{\pgfqpoint{1.980996in}{2.168698in}}%
\pgfpathlineto{\pgfqpoint{1.992721in}{2.172281in}}%
\pgfpathlineto{\pgfqpoint{2.004440in}{2.175858in}}%
\pgfpathlineto{\pgfqpoint{2.016153in}{2.179425in}}%
\pgfpathlineto{\pgfqpoint{2.010211in}{2.188412in}}%
\pgfpathlineto{\pgfqpoint{2.004273in}{2.197373in}}%
\pgfpathlineto{\pgfqpoint{1.998339in}{2.206307in}}%
\pgfpathlineto{\pgfqpoint{1.992410in}{2.215216in}}%
\pgfpathlineto{\pgfqpoint{1.986485in}{2.224099in}}%
\pgfpathlineto{\pgfqpoint{1.974783in}{2.220581in}}%
\pgfpathlineto{\pgfqpoint{1.963074in}{2.217053in}}%
\pgfpathlineto{\pgfqpoint{1.951361in}{2.213518in}}%
\pgfpathlineto{\pgfqpoint{1.939641in}{2.209977in}}%
\pgfpathlineto{\pgfqpoint{1.927916in}{2.206432in}}%
\pgfpathlineto{\pgfqpoint{1.933830in}{2.197503in}}%
\pgfpathlineto{\pgfqpoint{1.939748in}{2.188547in}}%
\pgfpathlineto{\pgfqpoint{1.945670in}{2.179564in}}%
\pgfpathlineto{\pgfqpoint{1.951598in}{2.170554in}}%
\pgfpathclose%
\pgfusepath{stroke,fill}%
\end{pgfscope}%
\begin{pgfscope}%
\pgfpathrectangle{\pgfqpoint{0.887500in}{0.275000in}}{\pgfqpoint{4.225000in}{4.225000in}}%
\pgfusepath{clip}%
\pgfsetbuttcap%
\pgfsetroundjoin%
\definecolor{currentfill}{rgb}{0.121148,0.592739,0.544641}%
\pgfsetfillcolor{currentfill}%
\pgfsetfillopacity{0.700000}%
\pgfsetlinewidth{0.501875pt}%
\definecolor{currentstroke}{rgb}{1.000000,1.000000,1.000000}%
\pgfsetstrokecolor{currentstroke}%
\pgfsetstrokeopacity{0.500000}%
\pgfsetdash{}{0pt}%
\pgfpathmoveto{\pgfqpoint{3.653553in}{2.355426in}}%
\pgfpathlineto{\pgfqpoint{3.664917in}{2.362239in}}%
\pgfpathlineto{\pgfqpoint{3.676271in}{2.368646in}}%
\pgfpathlineto{\pgfqpoint{3.687615in}{2.374685in}}%
\pgfpathlineto{\pgfqpoint{3.698949in}{2.380395in}}%
\pgfpathlineto{\pgfqpoint{3.710275in}{2.385813in}}%
\pgfpathlineto{\pgfqpoint{3.703877in}{2.400466in}}%
\pgfpathlineto{\pgfqpoint{3.697481in}{2.415097in}}%
\pgfpathlineto{\pgfqpoint{3.691088in}{2.429706in}}%
\pgfpathlineto{\pgfqpoint{3.684696in}{2.444292in}}%
\pgfpathlineto{\pgfqpoint{3.678307in}{2.458853in}}%
\pgfpathlineto{\pgfqpoint{3.666995in}{2.454098in}}%
\pgfpathlineto{\pgfqpoint{3.655676in}{2.449158in}}%
\pgfpathlineto{\pgfqpoint{3.644349in}{2.444007in}}%
\pgfpathlineto{\pgfqpoint{3.633014in}{2.438616in}}%
\pgfpathlineto{\pgfqpoint{3.621670in}{2.432957in}}%
\pgfpathlineto{\pgfqpoint{3.628043in}{2.417435in}}%
\pgfpathlineto{\pgfqpoint{3.634416in}{2.401878in}}%
\pgfpathlineto{\pgfqpoint{3.640792in}{2.386328in}}%
\pgfpathlineto{\pgfqpoint{3.647171in}{2.370830in}}%
\pgfpathclose%
\pgfusepath{stroke,fill}%
\end{pgfscope}%
\begin{pgfscope}%
\pgfpathrectangle{\pgfqpoint{0.887500in}{0.275000in}}{\pgfqpoint{4.225000in}{4.225000in}}%
\pgfusepath{clip}%
\pgfsetbuttcap%
\pgfsetroundjoin%
\definecolor{currentfill}{rgb}{0.206756,0.371758,0.553117}%
\pgfsetfillcolor{currentfill}%
\pgfsetfillopacity{0.700000}%
\pgfsetlinewidth{0.501875pt}%
\definecolor{currentstroke}{rgb}{1.000000,1.000000,1.000000}%
\pgfsetstrokecolor{currentstroke}%
\pgfsetstrokeopacity{0.500000}%
\pgfsetdash{}{0pt}%
\pgfpathmoveto{\pgfqpoint{3.252281in}{1.875952in}}%
\pgfpathlineto{\pgfqpoint{3.263711in}{1.884000in}}%
\pgfpathlineto{\pgfqpoint{3.275146in}{1.893346in}}%
\pgfpathlineto{\pgfqpoint{3.286586in}{1.904097in}}%
\pgfpathlineto{\pgfqpoint{3.298035in}{1.916341in}}%
\pgfpathlineto{\pgfqpoint{3.309493in}{1.930165in}}%
\pgfpathlineto{\pgfqpoint{3.303189in}{1.948581in}}%
\pgfpathlineto{\pgfqpoint{3.296890in}{1.967557in}}%
\pgfpathlineto{\pgfqpoint{3.290595in}{1.986943in}}%
\pgfpathlineto{\pgfqpoint{3.284301in}{2.006588in}}%
\pgfpathlineto{\pgfqpoint{3.278009in}{2.026342in}}%
\pgfpathlineto{\pgfqpoint{3.266548in}{2.011101in}}%
\pgfpathlineto{\pgfqpoint{3.255091in}{1.996601in}}%
\pgfpathlineto{\pgfqpoint{3.243638in}{1.982740in}}%
\pgfpathlineto{\pgfqpoint{3.232188in}{1.969421in}}%
\pgfpathlineto{\pgfqpoint{3.220741in}{1.956594in}}%
\pgfpathlineto{\pgfqpoint{3.227043in}{1.940064in}}%
\pgfpathlineto{\pgfqpoint{3.233348in}{1.923631in}}%
\pgfpathlineto{\pgfqpoint{3.239655in}{1.907400in}}%
\pgfpathlineto{\pgfqpoint{3.245966in}{1.891472in}}%
\pgfpathclose%
\pgfusepath{stroke,fill}%
\end{pgfscope}%
\begin{pgfscope}%
\pgfpathrectangle{\pgfqpoint{0.887500in}{0.275000in}}{\pgfqpoint{4.225000in}{4.225000in}}%
\pgfusepath{clip}%
\pgfsetbuttcap%
\pgfsetroundjoin%
\definecolor{currentfill}{rgb}{0.126453,0.570633,0.549841}%
\pgfsetfillcolor{currentfill}%
\pgfsetfillopacity{0.700000}%
\pgfsetlinewidth{0.501875pt}%
\definecolor{currentstroke}{rgb}{1.000000,1.000000,1.000000}%
\pgfsetstrokecolor{currentstroke}%
\pgfsetstrokeopacity{0.500000}%
\pgfsetdash{}{0pt}%
\pgfpathmoveto{\pgfqpoint{3.361264in}{2.286920in}}%
\pgfpathlineto{\pgfqpoint{3.372752in}{2.304040in}}%
\pgfpathlineto{\pgfqpoint{3.384241in}{2.321048in}}%
\pgfpathlineto{\pgfqpoint{3.395729in}{2.337772in}}%
\pgfpathlineto{\pgfqpoint{3.407216in}{2.354041in}}%
\pgfpathlineto{\pgfqpoint{3.418698in}{2.369682in}}%
\pgfpathlineto{\pgfqpoint{3.412339in}{2.383337in}}%
\pgfpathlineto{\pgfqpoint{3.405980in}{2.396792in}}%
\pgfpathlineto{\pgfqpoint{3.399624in}{2.410198in}}%
\pgfpathlineto{\pgfqpoint{3.393272in}{2.423708in}}%
\pgfpathlineto{\pgfqpoint{3.386924in}{2.437475in}}%
\pgfpathlineto{\pgfqpoint{3.375450in}{2.421178in}}%
\pgfpathlineto{\pgfqpoint{3.363974in}{2.404422in}}%
\pgfpathlineto{\pgfqpoint{3.352499in}{2.387511in}}%
\pgfpathlineto{\pgfqpoint{3.341028in}{2.370747in}}%
\pgfpathlineto{\pgfqpoint{3.329561in}{2.354433in}}%
\pgfpathlineto{\pgfqpoint{3.335892in}{2.340305in}}%
\pgfpathlineto{\pgfqpoint{3.342229in}{2.326651in}}%
\pgfpathlineto{\pgfqpoint{3.348570in}{2.313310in}}%
\pgfpathlineto{\pgfqpoint{3.354916in}{2.300121in}}%
\pgfpathclose%
\pgfusepath{stroke,fill}%
\end{pgfscope}%
\begin{pgfscope}%
\pgfpathrectangle{\pgfqpoint{0.887500in}{0.275000in}}{\pgfqpoint{4.225000in}{4.225000in}}%
\pgfusepath{clip}%
\pgfsetbuttcap%
\pgfsetroundjoin%
\definecolor{currentfill}{rgb}{0.166617,0.463708,0.558119}%
\pgfsetfillcolor{currentfill}%
\pgfsetfillopacity{0.700000}%
\pgfsetlinewidth{0.501875pt}%
\definecolor{currentstroke}{rgb}{1.000000,1.000000,1.000000}%
\pgfsetstrokecolor{currentstroke}%
\pgfsetstrokeopacity{0.500000}%
\pgfsetdash{}{0pt}%
\pgfpathmoveto{\pgfqpoint{4.040161in}{2.082813in}}%
\pgfpathlineto{\pgfqpoint{4.051396in}{2.087007in}}%
\pgfpathlineto{\pgfqpoint{4.062622in}{2.091035in}}%
\pgfpathlineto{\pgfqpoint{4.073839in}{2.094920in}}%
\pgfpathlineto{\pgfqpoint{4.085048in}{2.098689in}}%
\pgfpathlineto{\pgfqpoint{4.096249in}{2.102367in}}%
\pgfpathlineto{\pgfqpoint{4.089809in}{2.118097in}}%
\pgfpathlineto{\pgfqpoint{4.083364in}{2.133543in}}%
\pgfpathlineto{\pgfqpoint{4.076914in}{2.148688in}}%
\pgfpathlineto{\pgfqpoint{4.070459in}{2.163570in}}%
\pgfpathlineto{\pgfqpoint{4.064001in}{2.178228in}}%
\pgfpathlineto{\pgfqpoint{4.052808in}{2.174799in}}%
\pgfpathlineto{\pgfqpoint{4.041606in}{2.171261in}}%
\pgfpathlineto{\pgfqpoint{4.030395in}{2.167582in}}%
\pgfpathlineto{\pgfqpoint{4.019175in}{2.163730in}}%
\pgfpathlineto{\pgfqpoint{4.007944in}{2.159681in}}%
\pgfpathlineto{\pgfqpoint{4.014398in}{2.144938in}}%
\pgfpathlineto{\pgfqpoint{4.020848in}{2.129923in}}%
\pgfpathlineto{\pgfqpoint{4.027293in}{2.114585in}}%
\pgfpathlineto{\pgfqpoint{4.033730in}{2.098876in}}%
\pgfpathclose%
\pgfusepath{stroke,fill}%
\end{pgfscope}%
\begin{pgfscope}%
\pgfpathrectangle{\pgfqpoint{0.887500in}{0.275000in}}{\pgfqpoint{4.225000in}{4.225000in}}%
\pgfusepath{clip}%
\pgfsetbuttcap%
\pgfsetroundjoin%
\definecolor{currentfill}{rgb}{0.187231,0.414746,0.556547}%
\pgfsetfillcolor{currentfill}%
\pgfsetfillopacity{0.700000}%
\pgfsetlinewidth{0.501875pt}%
\definecolor{currentstroke}{rgb}{1.000000,1.000000,1.000000}%
\pgfsetstrokecolor{currentstroke}%
\pgfsetstrokeopacity{0.500000}%
\pgfsetdash{}{0pt}%
\pgfpathmoveto{\pgfqpoint{2.693783in}{1.996205in}}%
\pgfpathlineto{\pgfqpoint{2.705338in}{1.999908in}}%
\pgfpathlineto{\pgfqpoint{2.716887in}{2.003628in}}%
\pgfpathlineto{\pgfqpoint{2.728431in}{2.007358in}}%
\pgfpathlineto{\pgfqpoint{2.739969in}{2.011077in}}%
\pgfpathlineto{\pgfqpoint{2.751501in}{2.014764in}}%
\pgfpathlineto{\pgfqpoint{2.745314in}{2.024529in}}%
\pgfpathlineto{\pgfqpoint{2.739131in}{2.034250in}}%
\pgfpathlineto{\pgfqpoint{2.732951in}{2.043927in}}%
\pgfpathlineto{\pgfqpoint{2.726776in}{2.053560in}}%
\pgfpathlineto{\pgfqpoint{2.720605in}{2.063149in}}%
\pgfpathlineto{\pgfqpoint{2.709083in}{2.059506in}}%
\pgfpathlineto{\pgfqpoint{2.697555in}{2.055830in}}%
\pgfpathlineto{\pgfqpoint{2.686021in}{2.052143in}}%
\pgfpathlineto{\pgfqpoint{2.674482in}{2.048470in}}%
\pgfpathlineto{\pgfqpoint{2.662937in}{2.044814in}}%
\pgfpathlineto{\pgfqpoint{2.669097in}{2.035184in}}%
\pgfpathlineto{\pgfqpoint{2.675262in}{2.025508in}}%
\pgfpathlineto{\pgfqpoint{2.681432in}{2.015786in}}%
\pgfpathlineto{\pgfqpoint{2.687605in}{2.006019in}}%
\pgfpathclose%
\pgfusepath{stroke,fill}%
\end{pgfscope}%
\begin{pgfscope}%
\pgfpathrectangle{\pgfqpoint{0.887500in}{0.275000in}}{\pgfqpoint{4.225000in}{4.225000in}}%
\pgfusepath{clip}%
\pgfsetbuttcap%
\pgfsetroundjoin%
\definecolor{currentfill}{rgb}{0.119483,0.614817,0.537692}%
\pgfsetfillcolor{currentfill}%
\pgfsetfillopacity{0.700000}%
\pgfsetlinewidth{0.501875pt}%
\definecolor{currentstroke}{rgb}{1.000000,1.000000,1.000000}%
\pgfsetstrokecolor{currentstroke}%
\pgfsetstrokeopacity{0.500000}%
\pgfsetdash{}{0pt}%
\pgfpathmoveto{\pgfqpoint{3.564821in}{2.399384in}}%
\pgfpathlineto{\pgfqpoint{3.576209in}{2.406934in}}%
\pgfpathlineto{\pgfqpoint{3.587588in}{2.414027in}}%
\pgfpathlineto{\pgfqpoint{3.598958in}{2.420703in}}%
\pgfpathlineto{\pgfqpoint{3.610319in}{2.426999in}}%
\pgfpathlineto{\pgfqpoint{3.621670in}{2.432957in}}%
\pgfpathlineto{\pgfqpoint{3.615299in}{2.448400in}}%
\pgfpathlineto{\pgfqpoint{3.608928in}{2.463720in}}%
\pgfpathlineto{\pgfqpoint{3.602557in}{2.478872in}}%
\pgfpathlineto{\pgfqpoint{3.596186in}{2.493830in}}%
\pgfpathlineto{\pgfqpoint{3.589814in}{2.508608in}}%
\pgfpathlineto{\pgfqpoint{3.578474in}{2.503250in}}%
\pgfpathlineto{\pgfqpoint{3.567124in}{2.497520in}}%
\pgfpathlineto{\pgfqpoint{3.555765in}{2.491385in}}%
\pgfpathlineto{\pgfqpoint{3.544396in}{2.484812in}}%
\pgfpathlineto{\pgfqpoint{3.533018in}{2.477768in}}%
\pgfpathlineto{\pgfqpoint{3.539385in}{2.462909in}}%
\pgfpathlineto{\pgfqpoint{3.545749in}{2.447655in}}%
\pgfpathlineto{\pgfqpoint{3.552109in}{2.431950in}}%
\pgfpathlineto{\pgfqpoint{3.558466in}{2.415834in}}%
\pgfpathclose%
\pgfusepath{stroke,fill}%
\end{pgfscope}%
\begin{pgfscope}%
\pgfpathrectangle{\pgfqpoint{0.887500in}{0.275000in}}{\pgfqpoint{4.225000in}{4.225000in}}%
\pgfusepath{clip}%
\pgfsetbuttcap%
\pgfsetroundjoin%
\definecolor{currentfill}{rgb}{0.216210,0.351535,0.550627}%
\pgfsetfillcolor{currentfill}%
\pgfsetfillopacity{0.700000}%
\pgfsetlinewidth{0.501875pt}%
\definecolor{currentstroke}{rgb}{1.000000,1.000000,1.000000}%
\pgfsetstrokecolor{currentstroke}%
\pgfsetstrokeopacity{0.500000}%
\pgfsetdash{}{0pt}%
\pgfpathmoveto{\pgfqpoint{4.338216in}{1.855484in}}%
\pgfpathlineto{\pgfqpoint{4.349363in}{1.859062in}}%
\pgfpathlineto{\pgfqpoint{4.360502in}{1.862571in}}%
\pgfpathlineto{\pgfqpoint{4.371634in}{1.866020in}}%
\pgfpathlineto{\pgfqpoint{4.382759in}{1.869420in}}%
\pgfpathlineto{\pgfqpoint{4.393877in}{1.872779in}}%
\pgfpathlineto{\pgfqpoint{4.387339in}{1.887048in}}%
\pgfpathlineto{\pgfqpoint{4.380803in}{1.901269in}}%
\pgfpathlineto{\pgfqpoint{4.374268in}{1.915447in}}%
\pgfpathlineto{\pgfqpoint{4.367734in}{1.929587in}}%
\pgfpathlineto{\pgfqpoint{4.361202in}{1.943693in}}%
\pgfpathlineto{\pgfqpoint{4.350070in}{1.939841in}}%
\pgfpathlineto{\pgfqpoint{4.338930in}{1.935925in}}%
\pgfpathlineto{\pgfqpoint{4.327783in}{1.931940in}}%
\pgfpathlineto{\pgfqpoint{4.316628in}{1.927881in}}%
\pgfpathlineto{\pgfqpoint{4.305466in}{1.923743in}}%
\pgfpathlineto{\pgfqpoint{4.312012in}{1.910139in}}%
\pgfpathlineto{\pgfqpoint{4.318560in}{1.896534in}}%
\pgfpathlineto{\pgfqpoint{4.325111in}{1.882905in}}%
\pgfpathlineto{\pgfqpoint{4.331663in}{1.869229in}}%
\pgfpathclose%
\pgfusepath{stroke,fill}%
\end{pgfscope}%
\begin{pgfscope}%
\pgfpathrectangle{\pgfqpoint{0.887500in}{0.275000in}}{\pgfqpoint{4.225000in}{4.225000in}}%
\pgfusepath{clip}%
\pgfsetbuttcap%
\pgfsetroundjoin%
\definecolor{currentfill}{rgb}{0.119512,0.607464,0.540218}%
\pgfsetfillcolor{currentfill}%
\pgfsetfillopacity{0.700000}%
\pgfsetlinewidth{0.501875pt}%
\definecolor{currentstroke}{rgb}{1.000000,1.000000,1.000000}%
\pgfsetstrokecolor{currentstroke}%
\pgfsetstrokeopacity{0.500000}%
\pgfsetdash{}{0pt}%
\pgfpathmoveto{\pgfqpoint{3.418698in}{2.369682in}}%
\pgfpathlineto{\pgfqpoint{3.430174in}{2.384522in}}%
\pgfpathlineto{\pgfqpoint{3.441642in}{2.398392in}}%
\pgfpathlineto{\pgfqpoint{3.453100in}{2.411213in}}%
\pgfpathlineto{\pgfqpoint{3.464547in}{2.423052in}}%
\pgfpathlineto{\pgfqpoint{3.475984in}{2.433985in}}%
\pgfpathlineto{\pgfqpoint{3.469622in}{2.448425in}}%
\pgfpathlineto{\pgfqpoint{3.463258in}{2.462534in}}%
\pgfpathlineto{\pgfqpoint{3.456895in}{2.476426in}}%
\pgfpathlineto{\pgfqpoint{3.450534in}{2.490213in}}%
\pgfpathlineto{\pgfqpoint{3.444175in}{2.504011in}}%
\pgfpathlineto{\pgfqpoint{3.432745in}{2.492873in}}%
\pgfpathlineto{\pgfqpoint{3.421305in}{2.480733in}}%
\pgfpathlineto{\pgfqpoint{3.409854in}{2.467475in}}%
\pgfpathlineto{\pgfqpoint{3.398393in}{2.453008in}}%
\pgfpathlineto{\pgfqpoint{3.386924in}{2.437475in}}%
\pgfpathlineto{\pgfqpoint{3.393272in}{2.423708in}}%
\pgfpathlineto{\pgfqpoint{3.399624in}{2.410198in}}%
\pgfpathlineto{\pgfqpoint{3.405980in}{2.396792in}}%
\pgfpathlineto{\pgfqpoint{3.412339in}{2.383337in}}%
\pgfpathclose%
\pgfusepath{stroke,fill}%
\end{pgfscope}%
\begin{pgfscope}%
\pgfpathrectangle{\pgfqpoint{0.887500in}{0.275000in}}{\pgfqpoint{4.225000in}{4.225000in}}%
\pgfusepath{clip}%
\pgfsetbuttcap%
\pgfsetroundjoin%
\definecolor{currentfill}{rgb}{0.188923,0.410910,0.556326}%
\pgfsetfillcolor{currentfill}%
\pgfsetfillopacity{0.700000}%
\pgfsetlinewidth{0.501875pt}%
\definecolor{currentstroke}{rgb}{1.000000,1.000000,1.000000}%
\pgfsetstrokecolor{currentstroke}%
\pgfsetstrokeopacity{0.500000}%
\pgfsetdash{}{0pt}%
\pgfpathmoveto{\pgfqpoint{3.309493in}{1.930165in}}%
\pgfpathlineto{\pgfqpoint{3.320963in}{1.945657in}}%
\pgfpathlineto{\pgfqpoint{3.332446in}{1.962906in}}%
\pgfpathlineto{\pgfqpoint{3.343945in}{1.982001in}}%
\pgfpathlineto{\pgfqpoint{3.355461in}{2.002892in}}%
\pgfpathlineto{\pgfqpoint{3.366992in}{2.025072in}}%
\pgfpathlineto{\pgfqpoint{3.360677in}{2.043210in}}%
\pgfpathlineto{\pgfqpoint{3.354364in}{2.061522in}}%
\pgfpathlineto{\pgfqpoint{3.348053in}{2.079908in}}%
\pgfpathlineto{\pgfqpoint{3.341743in}{2.098271in}}%
\pgfpathlineto{\pgfqpoint{3.335433in}{2.116511in}}%
\pgfpathlineto{\pgfqpoint{3.323930in}{2.096620in}}%
\pgfpathlineto{\pgfqpoint{3.312436in}{2.077506in}}%
\pgfpathlineto{\pgfqpoint{3.300952in}{2.059444in}}%
\pgfpathlineto{\pgfqpoint{3.289477in}{2.042423in}}%
\pgfpathlineto{\pgfqpoint{3.278009in}{2.026342in}}%
\pgfpathlineto{\pgfqpoint{3.284301in}{2.006588in}}%
\pgfpathlineto{\pgfqpoint{3.290595in}{1.986943in}}%
\pgfpathlineto{\pgfqpoint{3.296890in}{1.967557in}}%
\pgfpathlineto{\pgfqpoint{3.303189in}{1.948581in}}%
\pgfpathclose%
\pgfusepath{stroke,fill}%
\end{pgfscope}%
\begin{pgfscope}%
\pgfpathrectangle{\pgfqpoint{0.887500in}{0.275000in}}{\pgfqpoint{4.225000in}{4.225000in}}%
\pgfusepath{clip}%
\pgfsetbuttcap%
\pgfsetroundjoin%
\definecolor{currentfill}{rgb}{0.172719,0.448791,0.557885}%
\pgfsetfillcolor{currentfill}%
\pgfsetfillopacity{0.700000}%
\pgfsetlinewidth{0.501875pt}%
\definecolor{currentstroke}{rgb}{1.000000,1.000000,1.000000}%
\pgfsetstrokecolor{currentstroke}%
\pgfsetstrokeopacity{0.500000}%
\pgfsetdash{}{0pt}%
\pgfpathmoveto{\pgfqpoint{2.369871in}{2.067623in}}%
\pgfpathlineto{\pgfqpoint{2.381508in}{2.071312in}}%
\pgfpathlineto{\pgfqpoint{2.393139in}{2.074989in}}%
\pgfpathlineto{\pgfqpoint{2.404765in}{2.078651in}}%
\pgfpathlineto{\pgfqpoint{2.416385in}{2.082299in}}%
\pgfpathlineto{\pgfqpoint{2.428000in}{2.085928in}}%
\pgfpathlineto{\pgfqpoint{2.421917in}{2.095267in}}%
\pgfpathlineto{\pgfqpoint{2.415838in}{2.104570in}}%
\pgfpathlineto{\pgfqpoint{2.409764in}{2.113839in}}%
\pgfpathlineto{\pgfqpoint{2.403694in}{2.123073in}}%
\pgfpathlineto{\pgfqpoint{2.397628in}{2.132274in}}%
\pgfpathlineto{\pgfqpoint{2.386023in}{2.128695in}}%
\pgfpathlineto{\pgfqpoint{2.374413in}{2.125099in}}%
\pgfpathlineto{\pgfqpoint{2.362798in}{2.121487in}}%
\pgfpathlineto{\pgfqpoint{2.351177in}{2.117862in}}%
\pgfpathlineto{\pgfqpoint{2.339551in}{2.114226in}}%
\pgfpathlineto{\pgfqpoint{2.345606in}{2.104973in}}%
\pgfpathlineto{\pgfqpoint{2.351666in}{2.095688in}}%
\pgfpathlineto{\pgfqpoint{2.357730in}{2.086368in}}%
\pgfpathlineto{\pgfqpoint{2.363798in}{2.077013in}}%
\pgfpathclose%
\pgfusepath{stroke,fill}%
\end{pgfscope}%
\begin{pgfscope}%
\pgfpathrectangle{\pgfqpoint{0.887500in}{0.275000in}}{\pgfqpoint{4.225000in}{4.225000in}}%
\pgfusepath{clip}%
\pgfsetbuttcap%
\pgfsetroundjoin%
\definecolor{currentfill}{rgb}{0.204903,0.375746,0.553533}%
\pgfsetfillcolor{currentfill}%
\pgfsetfillopacity{0.700000}%
\pgfsetlinewidth{0.501875pt}%
\definecolor{currentstroke}{rgb}{1.000000,1.000000,1.000000}%
\pgfsetstrokecolor{currentstroke}%
\pgfsetstrokeopacity{0.500000}%
\pgfsetdash{}{0pt}%
\pgfpathmoveto{\pgfqpoint{4.249557in}{1.902379in}}%
\pgfpathlineto{\pgfqpoint{4.260748in}{1.906635in}}%
\pgfpathlineto{\pgfqpoint{4.271936in}{1.910935in}}%
\pgfpathlineto{\pgfqpoint{4.283119in}{1.915242in}}%
\pgfpathlineto{\pgfqpoint{4.294296in}{1.919523in}}%
\pgfpathlineto{\pgfqpoint{4.305466in}{1.923743in}}%
\pgfpathlineto{\pgfqpoint{4.298924in}{1.937368in}}%
\pgfpathlineto{\pgfqpoint{4.292386in}{1.951038in}}%
\pgfpathlineto{\pgfqpoint{4.285853in}{1.964774in}}%
\pgfpathlineto{\pgfqpoint{4.279326in}{1.978601in}}%
\pgfpathlineto{\pgfqpoint{4.272806in}{1.992541in}}%
\pgfpathlineto{\pgfqpoint{4.261630in}{1.988044in}}%
\pgfpathlineto{\pgfqpoint{4.250448in}{1.983491in}}%
\pgfpathlineto{\pgfqpoint{4.239260in}{1.978903in}}%
\pgfpathlineto{\pgfqpoint{4.228067in}{1.974304in}}%
\pgfpathlineto{\pgfqpoint{4.216869in}{1.969716in}}%
\pgfpathlineto{\pgfqpoint{4.223389in}{1.955842in}}%
\pgfpathlineto{\pgfqpoint{4.229918in}{1.942216in}}%
\pgfpathlineto{\pgfqpoint{4.236457in}{1.928793in}}%
\pgfpathlineto{\pgfqpoint{4.243004in}{1.915529in}}%
\pgfpathclose%
\pgfusepath{stroke,fill}%
\end{pgfscope}%
\begin{pgfscope}%
\pgfpathrectangle{\pgfqpoint{0.887500in}{0.275000in}}{\pgfqpoint{4.225000in}{4.225000in}}%
\pgfusepath{clip}%
\pgfsetbuttcap%
\pgfsetroundjoin%
\definecolor{currentfill}{rgb}{0.156270,0.489624,0.557936}%
\pgfsetfillcolor{currentfill}%
\pgfsetfillopacity{0.700000}%
\pgfsetlinewidth{0.501875pt}%
\definecolor{currentstroke}{rgb}{1.000000,1.000000,1.000000}%
\pgfsetstrokecolor{currentstroke}%
\pgfsetstrokeopacity{0.500000}%
\pgfsetdash{}{0pt}%
\pgfpathmoveto{\pgfqpoint{3.951645in}{2.136427in}}%
\pgfpathlineto{\pgfqpoint{3.962923in}{2.141473in}}%
\pgfpathlineto{\pgfqpoint{3.974193in}{2.146323in}}%
\pgfpathlineto{\pgfqpoint{3.985453in}{2.150976in}}%
\pgfpathlineto{\pgfqpoint{3.996703in}{2.155429in}}%
\pgfpathlineto{\pgfqpoint{4.007944in}{2.159681in}}%
\pgfpathlineto{\pgfqpoint{4.001487in}{2.174200in}}%
\pgfpathlineto{\pgfqpoint{3.995028in}{2.188544in}}%
\pgfpathlineto{\pgfqpoint{3.988569in}{2.202765in}}%
\pgfpathlineto{\pgfqpoint{3.982110in}{2.216911in}}%
\pgfpathlineto{\pgfqpoint{3.975654in}{2.231032in}}%
\pgfpathlineto{\pgfqpoint{3.964419in}{2.226917in}}%
\pgfpathlineto{\pgfqpoint{3.953176in}{2.222642in}}%
\pgfpathlineto{\pgfqpoint{3.941924in}{2.218205in}}%
\pgfpathlineto{\pgfqpoint{3.930663in}{2.213607in}}%
\pgfpathlineto{\pgfqpoint{3.919394in}{2.208846in}}%
\pgfpathlineto{\pgfqpoint{3.925841in}{2.194457in}}%
\pgfpathlineto{\pgfqpoint{3.932291in}{2.180072in}}%
\pgfpathlineto{\pgfqpoint{3.938742in}{2.165639in}}%
\pgfpathlineto{\pgfqpoint{3.945194in}{2.151108in}}%
\pgfpathclose%
\pgfusepath{stroke,fill}%
\end{pgfscope}%
\begin{pgfscope}%
\pgfpathrectangle{\pgfqpoint{0.887500in}{0.275000in}}{\pgfqpoint{4.225000in}{4.225000in}}%
\pgfusepath{clip}%
\pgfsetbuttcap%
\pgfsetroundjoin%
\definecolor{currentfill}{rgb}{0.192357,0.403199,0.555836}%
\pgfsetfillcolor{currentfill}%
\pgfsetfillopacity{0.700000}%
\pgfsetlinewidth{0.501875pt}%
\definecolor{currentstroke}{rgb}{1.000000,1.000000,1.000000}%
\pgfsetstrokecolor{currentstroke}%
\pgfsetstrokeopacity{0.500000}%
\pgfsetdash{}{0pt}%
\pgfpathmoveto{\pgfqpoint{2.782500in}{1.965260in}}%
\pgfpathlineto{\pgfqpoint{2.794037in}{1.968945in}}%
\pgfpathlineto{\pgfqpoint{2.805569in}{1.972555in}}%
\pgfpathlineto{\pgfqpoint{2.817096in}{1.976068in}}%
\pgfpathlineto{\pgfqpoint{2.828618in}{1.979463in}}%
\pgfpathlineto{\pgfqpoint{2.840134in}{1.982757in}}%
\pgfpathlineto{\pgfqpoint{2.833917in}{1.992685in}}%
\pgfpathlineto{\pgfqpoint{2.827704in}{2.002570in}}%
\pgfpathlineto{\pgfqpoint{2.821495in}{2.012410in}}%
\pgfpathlineto{\pgfqpoint{2.815289in}{2.022206in}}%
\pgfpathlineto{\pgfqpoint{2.809088in}{2.031959in}}%
\pgfpathlineto{\pgfqpoint{2.797581in}{2.028733in}}%
\pgfpathlineto{\pgfqpoint{2.786069in}{2.025402in}}%
\pgfpathlineto{\pgfqpoint{2.774551in}{2.021949in}}%
\pgfpathlineto{\pgfqpoint{2.763029in}{2.018396in}}%
\pgfpathlineto{\pgfqpoint{2.751501in}{2.014764in}}%
\pgfpathlineto{\pgfqpoint{2.757693in}{2.004954in}}%
\pgfpathlineto{\pgfqpoint{2.763889in}{1.995099in}}%
\pgfpathlineto{\pgfqpoint{2.770088in}{1.985199in}}%
\pgfpathlineto{\pgfqpoint{2.776292in}{1.975253in}}%
\pgfpathclose%
\pgfusepath{stroke,fill}%
\end{pgfscope}%
\begin{pgfscope}%
\pgfpathrectangle{\pgfqpoint{0.887500in}{0.275000in}}{\pgfqpoint{4.225000in}{4.225000in}}%
\pgfusepath{clip}%
\pgfsetbuttcap%
\pgfsetroundjoin%
\definecolor{currentfill}{rgb}{0.159194,0.482237,0.558073}%
\pgfsetfillcolor{currentfill}%
\pgfsetfillopacity{0.700000}%
\pgfsetlinewidth{0.501875pt}%
\definecolor{currentstroke}{rgb}{1.000000,1.000000,1.000000}%
\pgfsetstrokecolor{currentstroke}%
\pgfsetstrokeopacity{0.500000}%
\pgfsetdash{}{0pt}%
\pgfpathmoveto{\pgfqpoint{2.045932in}{2.134088in}}%
\pgfpathlineto{\pgfqpoint{2.057650in}{2.137697in}}%
\pgfpathlineto{\pgfqpoint{2.069364in}{2.141293in}}%
\pgfpathlineto{\pgfqpoint{2.081071in}{2.144877in}}%
\pgfpathlineto{\pgfqpoint{2.092774in}{2.148450in}}%
\pgfpathlineto{\pgfqpoint{2.104471in}{2.152011in}}%
\pgfpathlineto{\pgfqpoint{2.098495in}{2.161083in}}%
\pgfpathlineto{\pgfqpoint{2.092525in}{2.170128in}}%
\pgfpathlineto{\pgfqpoint{2.086558in}{2.179144in}}%
\pgfpathlineto{\pgfqpoint{2.080596in}{2.188133in}}%
\pgfpathlineto{\pgfqpoint{2.074639in}{2.197095in}}%
\pgfpathlineto{\pgfqpoint{2.062953in}{2.193583in}}%
\pgfpathlineto{\pgfqpoint{2.051261in}{2.190060in}}%
\pgfpathlineto{\pgfqpoint{2.039564in}{2.186527in}}%
\pgfpathlineto{\pgfqpoint{2.027861in}{2.182982in}}%
\pgfpathlineto{\pgfqpoint{2.016153in}{2.179425in}}%
\pgfpathlineto{\pgfqpoint{2.022100in}{2.170412in}}%
\pgfpathlineto{\pgfqpoint{2.028051in}{2.161373in}}%
\pgfpathlineto{\pgfqpoint{2.034007in}{2.152306in}}%
\pgfpathlineto{\pgfqpoint{2.039967in}{2.143211in}}%
\pgfpathclose%
\pgfusepath{stroke,fill}%
\end{pgfscope}%
\begin{pgfscope}%
\pgfpathrectangle{\pgfqpoint{0.887500in}{0.275000in}}{\pgfqpoint{4.225000in}{4.225000in}}%
\pgfusepath{clip}%
\pgfsetbuttcap%
\pgfsetroundjoin%
\definecolor{currentfill}{rgb}{0.216210,0.351535,0.550627}%
\pgfsetfillcolor{currentfill}%
\pgfsetfillopacity{0.700000}%
\pgfsetlinewidth{0.501875pt}%
\definecolor{currentstroke}{rgb}{1.000000,1.000000,1.000000}%
\pgfsetstrokecolor{currentstroke}%
\pgfsetstrokeopacity{0.500000}%
\pgfsetdash{}{0pt}%
\pgfpathmoveto{\pgfqpoint{3.195131in}{1.849100in}}%
\pgfpathlineto{\pgfqpoint{3.206566in}{1.853285in}}%
\pgfpathlineto{\pgfqpoint{3.217997in}{1.857883in}}%
\pgfpathlineto{\pgfqpoint{3.229425in}{1.863073in}}%
\pgfpathlineto{\pgfqpoint{3.240853in}{1.869037in}}%
\pgfpathlineto{\pgfqpoint{3.252281in}{1.875952in}}%
\pgfpathlineto{\pgfqpoint{3.245966in}{1.891472in}}%
\pgfpathlineto{\pgfqpoint{3.239655in}{1.907400in}}%
\pgfpathlineto{\pgfqpoint{3.233348in}{1.923631in}}%
\pgfpathlineto{\pgfqpoint{3.227043in}{1.940064in}}%
\pgfpathlineto{\pgfqpoint{3.220741in}{1.956594in}}%
\pgfpathlineto{\pgfqpoint{3.209295in}{1.944374in}}%
\pgfpathlineto{\pgfqpoint{3.197851in}{1.932907in}}%
\pgfpathlineto{\pgfqpoint{3.186410in}{1.922341in}}%
\pgfpathlineto{\pgfqpoint{3.174970in}{1.912823in}}%
\pgfpathlineto{\pgfqpoint{3.163531in}{1.904500in}}%
\pgfpathlineto{\pgfqpoint{3.169844in}{1.893481in}}%
\pgfpathlineto{\pgfqpoint{3.176161in}{1.882423in}}%
\pgfpathlineto{\pgfqpoint{3.182481in}{1.871334in}}%
\pgfpathlineto{\pgfqpoint{3.188804in}{1.860224in}}%
\pgfpathclose%
\pgfusepath{stroke,fill}%
\end{pgfscope}%
\begin{pgfscope}%
\pgfpathrectangle{\pgfqpoint{0.887500in}{0.275000in}}{\pgfqpoint{4.225000in}{4.225000in}}%
\pgfusepath{clip}%
\pgfsetbuttcap%
\pgfsetroundjoin%
\definecolor{currentfill}{rgb}{0.147607,0.511733,0.557049}%
\pgfsetfillcolor{currentfill}%
\pgfsetfillopacity{0.700000}%
\pgfsetlinewidth{0.501875pt}%
\definecolor{currentstroke}{rgb}{1.000000,1.000000,1.000000}%
\pgfsetstrokecolor{currentstroke}%
\pgfsetstrokeopacity{0.500000}%
\pgfsetdash{}{0pt}%
\pgfpathmoveto{\pgfqpoint{1.722014in}{2.197781in}}%
\pgfpathlineto{\pgfqpoint{1.733813in}{2.201366in}}%
\pgfpathlineto{\pgfqpoint{1.745606in}{2.204937in}}%
\pgfpathlineto{\pgfqpoint{1.757395in}{2.208496in}}%
\pgfpathlineto{\pgfqpoint{1.769178in}{2.212043in}}%
\pgfpathlineto{\pgfqpoint{1.780955in}{2.215581in}}%
\pgfpathlineto{\pgfqpoint{1.775090in}{2.224441in}}%
\pgfpathlineto{\pgfqpoint{1.769230in}{2.233275in}}%
\pgfpathlineto{\pgfqpoint{1.763375in}{2.242083in}}%
\pgfpathlineto{\pgfqpoint{1.757523in}{2.250865in}}%
\pgfpathlineto{\pgfqpoint{1.751677in}{2.259621in}}%
\pgfpathlineto{\pgfqpoint{1.739911in}{2.256131in}}%
\pgfpathlineto{\pgfqpoint{1.728139in}{2.252631in}}%
\pgfpathlineto{\pgfqpoint{1.716362in}{2.249121in}}%
\pgfpathlineto{\pgfqpoint{1.704579in}{2.245599in}}%
\pgfpathlineto{\pgfqpoint{1.692791in}{2.242064in}}%
\pgfpathlineto{\pgfqpoint{1.698627in}{2.233260in}}%
\pgfpathlineto{\pgfqpoint{1.704466in}{2.224430in}}%
\pgfpathlineto{\pgfqpoint{1.710311in}{2.215573in}}%
\pgfpathlineto{\pgfqpoint{1.716160in}{2.206690in}}%
\pgfpathclose%
\pgfusepath{stroke,fill}%
\end{pgfscope}%
\begin{pgfscope}%
\pgfpathrectangle{\pgfqpoint{0.887500in}{0.275000in}}{\pgfqpoint{4.225000in}{4.225000in}}%
\pgfusepath{clip}%
\pgfsetbuttcap%
\pgfsetroundjoin%
\definecolor{currentfill}{rgb}{0.194100,0.399323,0.555565}%
\pgfsetfillcolor{currentfill}%
\pgfsetfillopacity{0.700000}%
\pgfsetlinewidth{0.501875pt}%
\definecolor{currentstroke}{rgb}{1.000000,1.000000,1.000000}%
\pgfsetstrokecolor{currentstroke}%
\pgfsetstrokeopacity{0.500000}%
\pgfsetdash{}{0pt}%
\pgfpathmoveto{\pgfqpoint{4.160828in}{1.947496in}}%
\pgfpathlineto{\pgfqpoint{4.172044in}{1.951865in}}%
\pgfpathlineto{\pgfqpoint{4.183255in}{1.956240in}}%
\pgfpathlineto{\pgfqpoint{4.194463in}{1.960665in}}%
\pgfpathlineto{\pgfqpoint{4.205668in}{1.965162in}}%
\pgfpathlineto{\pgfqpoint{4.216869in}{1.969716in}}%
\pgfpathlineto{\pgfqpoint{4.210361in}{1.983883in}}%
\pgfpathlineto{\pgfqpoint{4.203865in}{1.998368in}}%
\pgfpathlineto{\pgfqpoint{4.197380in}{2.013136in}}%
\pgfpathlineto{\pgfqpoint{4.190904in}{2.028137in}}%
\pgfpathlineto{\pgfqpoint{4.184436in}{2.043320in}}%
\pgfpathlineto{\pgfqpoint{4.173246in}{2.039094in}}%
\pgfpathlineto{\pgfqpoint{4.162051in}{2.034879in}}%
\pgfpathlineto{\pgfqpoint{4.150852in}{2.030683in}}%
\pgfpathlineto{\pgfqpoint{4.139648in}{2.026497in}}%
\pgfpathlineto{\pgfqpoint{4.128438in}{2.022302in}}%
\pgfpathlineto{\pgfqpoint{4.134891in}{2.006590in}}%
\pgfpathlineto{\pgfqpoint{4.141353in}{1.991173in}}%
\pgfpathlineto{\pgfqpoint{4.147829in}{1.976134in}}%
\pgfpathlineto{\pgfqpoint{4.154320in}{1.961555in}}%
\pgfpathclose%
\pgfusepath{stroke,fill}%
\end{pgfscope}%
\begin{pgfscope}%
\pgfpathrectangle{\pgfqpoint{0.887500in}{0.275000in}}{\pgfqpoint{4.225000in}{4.225000in}}%
\pgfusepath{clip}%
\pgfsetbuttcap%
\pgfsetroundjoin%
\definecolor{currentfill}{rgb}{0.147607,0.511733,0.557049}%
\pgfsetfillcolor{currentfill}%
\pgfsetfillopacity{0.700000}%
\pgfsetlinewidth{0.501875pt}%
\definecolor{currentstroke}{rgb}{1.000000,1.000000,1.000000}%
\pgfsetstrokecolor{currentstroke}%
\pgfsetstrokeopacity{0.500000}%
\pgfsetdash{}{0pt}%
\pgfpathmoveto{\pgfqpoint{3.862944in}{2.183454in}}%
\pgfpathlineto{\pgfqpoint{3.874243in}{2.188538in}}%
\pgfpathlineto{\pgfqpoint{3.885540in}{2.193695in}}%
\pgfpathlineto{\pgfqpoint{3.896831in}{2.198848in}}%
\pgfpathlineto{\pgfqpoint{3.908117in}{2.203921in}}%
\pgfpathlineto{\pgfqpoint{3.919394in}{2.208846in}}%
\pgfpathlineto{\pgfqpoint{3.912951in}{2.223290in}}%
\pgfpathlineto{\pgfqpoint{3.906512in}{2.237839in}}%
\pgfpathlineto{\pgfqpoint{3.900079in}{2.252528in}}%
\pgfpathlineto{\pgfqpoint{3.893650in}{2.267336in}}%
\pgfpathlineto{\pgfqpoint{3.887226in}{2.282234in}}%
\pgfpathlineto{\pgfqpoint{3.875961in}{2.277761in}}%
\pgfpathlineto{\pgfqpoint{3.864688in}{2.273151in}}%
\pgfpathlineto{\pgfqpoint{3.853409in}{2.268458in}}%
\pgfpathlineto{\pgfqpoint{3.842124in}{2.263741in}}%
\pgfpathlineto{\pgfqpoint{3.830835in}{2.259059in}}%
\pgfpathlineto{\pgfqpoint{3.837248in}{2.243742in}}%
\pgfpathlineto{\pgfqpoint{3.843665in}{2.228501in}}%
\pgfpathlineto{\pgfqpoint{3.850087in}{2.213366in}}%
\pgfpathlineto{\pgfqpoint{3.856513in}{2.198360in}}%
\pgfpathclose%
\pgfusepath{stroke,fill}%
\end{pgfscope}%
\begin{pgfscope}%
\pgfpathrectangle{\pgfqpoint{0.887500in}{0.275000in}}{\pgfqpoint{4.225000in}{4.225000in}}%
\pgfusepath{clip}%
\pgfsetbuttcap%
\pgfsetroundjoin%
\definecolor{currentfill}{rgb}{0.151918,0.500685,0.557587}%
\pgfsetfillcolor{currentfill}%
\pgfsetfillopacity{0.700000}%
\pgfsetlinewidth{0.501875pt}%
\definecolor{currentstroke}{rgb}{1.000000,1.000000,1.000000}%
\pgfsetstrokecolor{currentstroke}%
\pgfsetstrokeopacity{0.500000}%
\pgfsetdash{}{0pt}%
\pgfpathmoveto{\pgfqpoint{3.335433in}{2.116511in}}%
\pgfpathlineto{\pgfqpoint{3.346943in}{2.136842in}}%
\pgfpathlineto{\pgfqpoint{3.358458in}{2.157277in}}%
\pgfpathlineto{\pgfqpoint{3.369975in}{2.177477in}}%
\pgfpathlineto{\pgfqpoint{3.381492in}{2.197105in}}%
\pgfpathlineto{\pgfqpoint{3.393005in}{2.215822in}}%
\pgfpathlineto{\pgfqpoint{3.386659in}{2.230969in}}%
\pgfpathlineto{\pgfqpoint{3.380312in}{2.245651in}}%
\pgfpathlineto{\pgfqpoint{3.373964in}{2.259837in}}%
\pgfpathlineto{\pgfqpoint{3.367614in}{2.273546in}}%
\pgfpathlineto{\pgfqpoint{3.361264in}{2.286920in}}%
\pgfpathlineto{\pgfqpoint{3.349779in}{2.269846in}}%
\pgfpathlineto{\pgfqpoint{3.338297in}{2.252875in}}%
\pgfpathlineto{\pgfqpoint{3.326818in}{2.236020in}}%
\pgfpathlineto{\pgfqpoint{3.315342in}{2.219291in}}%
\pgfpathlineto{\pgfqpoint{3.303869in}{2.202699in}}%
\pgfpathlineto{\pgfqpoint{3.310184in}{2.186309in}}%
\pgfpathlineto{\pgfqpoint{3.316499in}{2.169505in}}%
\pgfpathlineto{\pgfqpoint{3.322811in}{2.152227in}}%
\pgfpathlineto{\pgfqpoint{3.329123in}{2.134529in}}%
\pgfpathclose%
\pgfusepath{stroke,fill}%
\end{pgfscope}%
\begin{pgfscope}%
\pgfpathrectangle{\pgfqpoint{0.887500in}{0.275000in}}{\pgfqpoint{4.225000in}{4.225000in}}%
\pgfusepath{clip}%
\pgfsetbuttcap%
\pgfsetroundjoin%
\definecolor{currentfill}{rgb}{0.199430,0.387607,0.554642}%
\pgfsetfillcolor{currentfill}%
\pgfsetfillopacity{0.700000}%
\pgfsetlinewidth{0.501875pt}%
\definecolor{currentstroke}{rgb}{1.000000,1.000000,1.000000}%
\pgfsetstrokecolor{currentstroke}%
\pgfsetstrokeopacity{0.500000}%
\pgfsetdash{}{0pt}%
\pgfpathmoveto{\pgfqpoint{2.871279in}{1.932445in}}%
\pgfpathlineto{\pgfqpoint{2.882799in}{1.935792in}}%
\pgfpathlineto{\pgfqpoint{2.894313in}{1.939167in}}%
\pgfpathlineto{\pgfqpoint{2.905821in}{1.942634in}}%
\pgfpathlineto{\pgfqpoint{2.917322in}{1.946254in}}%
\pgfpathlineto{\pgfqpoint{2.928817in}{1.950091in}}%
\pgfpathlineto{\pgfqpoint{2.922571in}{1.960192in}}%
\pgfpathlineto{\pgfqpoint{2.916329in}{1.970246in}}%
\pgfpathlineto{\pgfqpoint{2.910091in}{1.980255in}}%
\pgfpathlineto{\pgfqpoint{2.903856in}{1.990217in}}%
\pgfpathlineto{\pgfqpoint{2.897625in}{2.000134in}}%
\pgfpathlineto{\pgfqpoint{2.886140in}{1.996301in}}%
\pgfpathlineto{\pgfqpoint{2.874648in}{1.992718in}}%
\pgfpathlineto{\pgfqpoint{2.863150in}{1.989313in}}%
\pgfpathlineto{\pgfqpoint{2.851645in}{1.986017in}}%
\pgfpathlineto{\pgfqpoint{2.840134in}{1.982757in}}%
\pgfpathlineto{\pgfqpoint{2.846355in}{1.972784in}}%
\pgfpathlineto{\pgfqpoint{2.852580in}{1.962767in}}%
\pgfpathlineto{\pgfqpoint{2.858809in}{1.952705in}}%
\pgfpathlineto{\pgfqpoint{2.865042in}{1.942597in}}%
\pgfpathclose%
\pgfusepath{stroke,fill}%
\end{pgfscope}%
\begin{pgfscope}%
\pgfpathrectangle{\pgfqpoint{0.887500in}{0.275000in}}{\pgfqpoint{4.225000in}{4.225000in}}%
\pgfusepath{clip}%
\pgfsetbuttcap%
\pgfsetroundjoin%
\definecolor{currentfill}{rgb}{0.177423,0.437527,0.557565}%
\pgfsetfillcolor{currentfill}%
\pgfsetfillopacity{0.700000}%
\pgfsetlinewidth{0.501875pt}%
\definecolor{currentstroke}{rgb}{1.000000,1.000000,1.000000}%
\pgfsetstrokecolor{currentstroke}%
\pgfsetstrokeopacity{0.500000}%
\pgfsetdash{}{0pt}%
\pgfpathmoveto{\pgfqpoint{2.458481in}{2.038675in}}%
\pgfpathlineto{\pgfqpoint{2.470101in}{2.042335in}}%
\pgfpathlineto{\pgfqpoint{2.481715in}{2.045973in}}%
\pgfpathlineto{\pgfqpoint{2.493325in}{2.049590in}}%
\pgfpathlineto{\pgfqpoint{2.504928in}{2.053183in}}%
\pgfpathlineto{\pgfqpoint{2.516526in}{2.056757in}}%
\pgfpathlineto{\pgfqpoint{2.510411in}{2.066236in}}%
\pgfpathlineto{\pgfqpoint{2.504301in}{2.075674in}}%
\pgfpathlineto{\pgfqpoint{2.498194in}{2.085074in}}%
\pgfpathlineto{\pgfqpoint{2.492092in}{2.094436in}}%
\pgfpathlineto{\pgfqpoint{2.485994in}{2.103760in}}%
\pgfpathlineto{\pgfqpoint{2.474406in}{2.100238in}}%
\pgfpathlineto{\pgfqpoint{2.462813in}{2.096695in}}%
\pgfpathlineto{\pgfqpoint{2.451214in}{2.093128in}}%
\pgfpathlineto{\pgfqpoint{2.439610in}{2.089539in}}%
\pgfpathlineto{\pgfqpoint{2.428000in}{2.085928in}}%
\pgfpathlineto{\pgfqpoint{2.434088in}{2.076554in}}%
\pgfpathlineto{\pgfqpoint{2.440179in}{2.067142in}}%
\pgfpathlineto{\pgfqpoint{2.446276in}{2.057692in}}%
\pgfpathlineto{\pgfqpoint{2.452376in}{2.048204in}}%
\pgfpathclose%
\pgfusepath{stroke,fill}%
\end{pgfscope}%
\begin{pgfscope}%
\pgfpathrectangle{\pgfqpoint{0.887500in}{0.275000in}}{\pgfqpoint{4.225000in}{4.225000in}}%
\pgfusepath{clip}%
\pgfsetbuttcap%
\pgfsetroundjoin%
\definecolor{currentfill}{rgb}{0.223925,0.334994,0.548053}%
\pgfsetfillcolor{currentfill}%
\pgfsetfillopacity{0.700000}%
\pgfsetlinewidth{0.501875pt}%
\definecolor{currentstroke}{rgb}{1.000000,1.000000,1.000000}%
\pgfsetstrokecolor{currentstroke}%
\pgfsetstrokeopacity{0.500000}%
\pgfsetdash{}{0pt}%
\pgfpathmoveto{\pgfqpoint{3.283939in}{1.807623in}}%
\pgfpathlineto{\pgfqpoint{3.295361in}{1.813002in}}%
\pgfpathlineto{\pgfqpoint{3.306788in}{1.819945in}}%
\pgfpathlineto{\pgfqpoint{3.318223in}{1.828565in}}%
\pgfpathlineto{\pgfqpoint{3.329667in}{1.838921in}}%
\pgfpathlineto{\pgfqpoint{3.341122in}{1.851075in}}%
\pgfpathlineto{\pgfqpoint{3.334779in}{1.864970in}}%
\pgfpathlineto{\pgfqpoint{3.328444in}{1.879771in}}%
\pgfpathlineto{\pgfqpoint{3.322119in}{1.895612in}}%
\pgfpathlineto{\pgfqpoint{3.315802in}{1.912458in}}%
\pgfpathlineto{\pgfqpoint{3.309493in}{1.930165in}}%
\pgfpathlineto{\pgfqpoint{3.298035in}{1.916341in}}%
\pgfpathlineto{\pgfqpoint{3.286586in}{1.904097in}}%
\pgfpathlineto{\pgfqpoint{3.275146in}{1.893346in}}%
\pgfpathlineto{\pgfqpoint{3.263711in}{1.884000in}}%
\pgfpathlineto{\pgfqpoint{3.252281in}{1.875952in}}%
\pgfpathlineto{\pgfqpoint{3.258600in}{1.860943in}}%
\pgfpathlineto{\pgfqpoint{3.264925in}{1.846547in}}%
\pgfpathlineto{\pgfqpoint{3.271256in}{1.832865in}}%
\pgfpathlineto{\pgfqpoint{3.277594in}{1.819920in}}%
\pgfpathclose%
\pgfusepath{stroke,fill}%
\end{pgfscope}%
\begin{pgfscope}%
\pgfpathrectangle{\pgfqpoint{0.887500in}{0.275000in}}{\pgfqpoint{4.225000in}{4.225000in}}%
\pgfusepath{clip}%
\pgfsetbuttcap%
\pgfsetroundjoin%
\definecolor{currentfill}{rgb}{0.137770,0.537492,0.554906}%
\pgfsetfillcolor{currentfill}%
\pgfsetfillopacity{0.700000}%
\pgfsetlinewidth{0.501875pt}%
\definecolor{currentstroke}{rgb}{1.000000,1.000000,1.000000}%
\pgfsetstrokecolor{currentstroke}%
\pgfsetstrokeopacity{0.500000}%
\pgfsetdash{}{0pt}%
\pgfpathmoveto{\pgfqpoint{3.774355in}{2.237746in}}%
\pgfpathlineto{\pgfqpoint{3.785658in}{2.241803in}}%
\pgfpathlineto{\pgfqpoint{3.796954in}{2.245832in}}%
\pgfpathlineto{\pgfqpoint{3.808249in}{2.250045in}}%
\pgfpathlineto{\pgfqpoint{3.819543in}{2.254474in}}%
\pgfpathlineto{\pgfqpoint{3.830835in}{2.259059in}}%
\pgfpathlineto{\pgfqpoint{3.824425in}{2.274422in}}%
\pgfpathlineto{\pgfqpoint{3.818017in}{2.289798in}}%
\pgfpathlineto{\pgfqpoint{3.811611in}{2.305157in}}%
\pgfpathlineto{\pgfqpoint{3.805207in}{2.320469in}}%
\pgfpathlineto{\pgfqpoint{3.798803in}{2.335702in}}%
\pgfpathlineto{\pgfqpoint{3.787513in}{2.331143in}}%
\pgfpathlineto{\pgfqpoint{3.776217in}{2.326528in}}%
\pgfpathlineto{\pgfqpoint{3.764916in}{2.321874in}}%
\pgfpathlineto{\pgfqpoint{3.753609in}{2.317155in}}%
\pgfpathlineto{\pgfqpoint{3.742294in}{2.312264in}}%
\pgfpathlineto{\pgfqpoint{3.748704in}{2.297472in}}%
\pgfpathlineto{\pgfqpoint{3.755115in}{2.282633in}}%
\pgfpathlineto{\pgfqpoint{3.761527in}{2.267738in}}%
\pgfpathlineto{\pgfqpoint{3.767941in}{2.252779in}}%
\pgfpathclose%
\pgfusepath{stroke,fill}%
\end{pgfscope}%
\begin{pgfscope}%
\pgfpathrectangle{\pgfqpoint{0.887500in}{0.275000in}}{\pgfqpoint{4.225000in}{4.225000in}}%
\pgfusepath{clip}%
\pgfsetbuttcap%
\pgfsetroundjoin%
\definecolor{currentfill}{rgb}{0.163625,0.471133,0.558148}%
\pgfsetfillcolor{currentfill}%
\pgfsetfillopacity{0.700000}%
\pgfsetlinewidth{0.501875pt}%
\definecolor{currentstroke}{rgb}{1.000000,1.000000,1.000000}%
\pgfsetstrokecolor{currentstroke}%
\pgfsetstrokeopacity{0.500000}%
\pgfsetdash{}{0pt}%
\pgfpathmoveto{\pgfqpoint{2.134413in}{2.106206in}}%
\pgfpathlineto{\pgfqpoint{2.146115in}{2.109807in}}%
\pgfpathlineto{\pgfqpoint{2.157811in}{2.113398in}}%
\pgfpathlineto{\pgfqpoint{2.169502in}{2.116980in}}%
\pgfpathlineto{\pgfqpoint{2.181187in}{2.120555in}}%
\pgfpathlineto{\pgfqpoint{2.192867in}{2.124126in}}%
\pgfpathlineto{\pgfqpoint{2.186859in}{2.133295in}}%
\pgfpathlineto{\pgfqpoint{2.180855in}{2.142433in}}%
\pgfpathlineto{\pgfqpoint{2.174856in}{2.151543in}}%
\pgfpathlineto{\pgfqpoint{2.168861in}{2.160624in}}%
\pgfpathlineto{\pgfqpoint{2.162871in}{2.169675in}}%
\pgfpathlineto{\pgfqpoint{2.151202in}{2.166157in}}%
\pgfpathlineto{\pgfqpoint{2.139528in}{2.162633in}}%
\pgfpathlineto{\pgfqpoint{2.127848in}{2.159102in}}%
\pgfpathlineto{\pgfqpoint{2.116162in}{2.155561in}}%
\pgfpathlineto{\pgfqpoint{2.104471in}{2.152011in}}%
\pgfpathlineto{\pgfqpoint{2.110450in}{2.142909in}}%
\pgfpathlineto{\pgfqpoint{2.116434in}{2.133778in}}%
\pgfpathlineto{\pgfqpoint{2.122423in}{2.124618in}}%
\pgfpathlineto{\pgfqpoint{2.128416in}{2.115427in}}%
\pgfpathclose%
\pgfusepath{stroke,fill}%
\end{pgfscope}%
\begin{pgfscope}%
\pgfpathrectangle{\pgfqpoint{0.887500in}{0.275000in}}{\pgfqpoint{4.225000in}{4.225000in}}%
\pgfusepath{clip}%
\pgfsetbuttcap%
\pgfsetroundjoin%
\definecolor{currentfill}{rgb}{0.244972,0.287675,0.537260}%
\pgfsetfillcolor{currentfill}%
\pgfsetfillopacity{0.700000}%
\pgfsetlinewidth{0.501875pt}%
\definecolor{currentstroke}{rgb}{1.000000,1.000000,1.000000}%
\pgfsetstrokecolor{currentstroke}%
\pgfsetstrokeopacity{0.500000}%
\pgfsetdash{}{0pt}%
\pgfpathmoveto{\pgfqpoint{4.459269in}{1.726455in}}%
\pgfpathlineto{\pgfqpoint{4.470384in}{1.729837in}}%
\pgfpathlineto{\pgfqpoint{4.481493in}{1.733183in}}%
\pgfpathlineto{\pgfqpoint{4.492594in}{1.736495in}}%
\pgfpathlineto{\pgfqpoint{4.503689in}{1.739777in}}%
\pgfpathlineto{\pgfqpoint{4.514777in}{1.743038in}}%
\pgfpathlineto{\pgfqpoint{4.508218in}{1.757467in}}%
\pgfpathlineto{\pgfqpoint{4.501663in}{1.771967in}}%
\pgfpathlineto{\pgfqpoint{4.495114in}{1.786527in}}%
\pgfpathlineto{\pgfqpoint{4.488569in}{1.801137in}}%
\pgfpathlineto{\pgfqpoint{4.482028in}{1.815786in}}%
\pgfpathlineto{\pgfqpoint{4.470948in}{1.812722in}}%
\pgfpathlineto{\pgfqpoint{4.459863in}{1.809677in}}%
\pgfpathlineto{\pgfqpoint{4.448772in}{1.806641in}}%
\pgfpathlineto{\pgfqpoint{4.437676in}{1.803604in}}%
\pgfpathlineto{\pgfqpoint{4.426574in}{1.800556in}}%
\pgfpathlineto{\pgfqpoint{4.433114in}{1.785905in}}%
\pgfpathlineto{\pgfqpoint{4.439653in}{1.771173in}}%
\pgfpathlineto{\pgfqpoint{4.446193in}{1.756357in}}%
\pgfpathlineto{\pgfqpoint{4.452731in}{1.741453in}}%
\pgfpathclose%
\pgfusepath{stroke,fill}%
\end{pgfscope}%
\begin{pgfscope}%
\pgfpathrectangle{\pgfqpoint{0.887500in}{0.275000in}}{\pgfqpoint{4.225000in}{4.225000in}}%
\pgfusepath{clip}%
\pgfsetbuttcap%
\pgfsetroundjoin%
\definecolor{currentfill}{rgb}{0.206756,0.371758,0.553117}%
\pgfsetfillcolor{currentfill}%
\pgfsetfillopacity{0.700000}%
\pgfsetlinewidth{0.501875pt}%
\definecolor{currentstroke}{rgb}{1.000000,1.000000,1.000000}%
\pgfsetstrokecolor{currentstroke}%
\pgfsetstrokeopacity{0.500000}%
\pgfsetdash{}{0pt}%
\pgfpathmoveto{\pgfqpoint{3.341122in}{1.851075in}}%
\pgfpathlineto{\pgfqpoint{3.352591in}{1.865087in}}%
\pgfpathlineto{\pgfqpoint{3.364076in}{1.881019in}}%
\pgfpathlineto{\pgfqpoint{3.375578in}{1.898932in}}%
\pgfpathlineto{\pgfqpoint{3.387099in}{1.918753in}}%
\pgfpathlineto{\pgfqpoint{3.398636in}{1.939963in}}%
\pgfpathlineto{\pgfqpoint{3.392297in}{1.956129in}}%
\pgfpathlineto{\pgfqpoint{3.385963in}{1.972678in}}%
\pgfpathlineto{\pgfqpoint{3.379634in}{1.989708in}}%
\pgfpathlineto{\pgfqpoint{3.373311in}{2.007205in}}%
\pgfpathlineto{\pgfqpoint{3.366992in}{2.025072in}}%
\pgfpathlineto{\pgfqpoint{3.355461in}{2.002892in}}%
\pgfpathlineto{\pgfqpoint{3.343945in}{1.982001in}}%
\pgfpathlineto{\pgfqpoint{3.332446in}{1.962906in}}%
\pgfpathlineto{\pgfqpoint{3.320963in}{1.945657in}}%
\pgfpathlineto{\pgfqpoint{3.309493in}{1.930165in}}%
\pgfpathlineto{\pgfqpoint{3.315802in}{1.912458in}}%
\pgfpathlineto{\pgfqpoint{3.322119in}{1.895612in}}%
\pgfpathlineto{\pgfqpoint{3.328444in}{1.879771in}}%
\pgfpathlineto{\pgfqpoint{3.334779in}{1.864970in}}%
\pgfpathclose%
\pgfusepath{stroke,fill}%
\end{pgfscope}%
\begin{pgfscope}%
\pgfpathrectangle{\pgfqpoint{0.887500in}{0.275000in}}{\pgfqpoint{4.225000in}{4.225000in}}%
\pgfusepath{clip}%
\pgfsetbuttcap%
\pgfsetroundjoin%
\definecolor{currentfill}{rgb}{0.120092,0.600104,0.542530}%
\pgfsetfillcolor{currentfill}%
\pgfsetfillopacity{0.700000}%
\pgfsetlinewidth{0.501875pt}%
\definecolor{currentstroke}{rgb}{1.000000,1.000000,1.000000}%
\pgfsetstrokecolor{currentstroke}%
\pgfsetstrokeopacity{0.500000}%
\pgfsetdash{}{0pt}%
\pgfpathmoveto{\pgfqpoint{3.507740in}{2.353668in}}%
\pgfpathlineto{\pgfqpoint{3.519174in}{2.363921in}}%
\pgfpathlineto{\pgfqpoint{3.530599in}{2.373616in}}%
\pgfpathlineto{\pgfqpoint{3.542016in}{2.382754in}}%
\pgfpathlineto{\pgfqpoint{3.553423in}{2.391337in}}%
\pgfpathlineto{\pgfqpoint{3.564821in}{2.399384in}}%
\pgfpathlineto{\pgfqpoint{3.558466in}{2.415834in}}%
\pgfpathlineto{\pgfqpoint{3.552109in}{2.431950in}}%
\pgfpathlineto{\pgfqpoint{3.545749in}{2.447655in}}%
\pgfpathlineto{\pgfqpoint{3.539385in}{2.462909in}}%
\pgfpathlineto{\pgfqpoint{3.533018in}{2.477768in}}%
\pgfpathlineto{\pgfqpoint{3.521631in}{2.470218in}}%
\pgfpathlineto{\pgfqpoint{3.510234in}{2.462130in}}%
\pgfpathlineto{\pgfqpoint{3.498827in}{2.453447in}}%
\pgfpathlineto{\pgfqpoint{3.487411in}{2.444091in}}%
\pgfpathlineto{\pgfqpoint{3.475984in}{2.433985in}}%
\pgfpathlineto{\pgfqpoint{3.482344in}{2.419101in}}%
\pgfpathlineto{\pgfqpoint{3.488700in}{2.403659in}}%
\pgfpathlineto{\pgfqpoint{3.495052in}{2.387568in}}%
\pgfpathlineto{\pgfqpoint{3.501398in}{2.370872in}}%
\pgfpathclose%
\pgfusepath{stroke,fill}%
\end{pgfscope}%
\begin{pgfscope}%
\pgfpathrectangle{\pgfqpoint{0.887500in}{0.275000in}}{\pgfqpoint{4.225000in}{4.225000in}}%
\pgfusepath{clip}%
\pgfsetbuttcap%
\pgfsetroundjoin%
\definecolor{currentfill}{rgb}{0.180629,0.429975,0.557282}%
\pgfsetfillcolor{currentfill}%
\pgfsetfillopacity{0.700000}%
\pgfsetlinewidth{0.501875pt}%
\definecolor{currentstroke}{rgb}{1.000000,1.000000,1.000000}%
\pgfsetstrokecolor{currentstroke}%
\pgfsetstrokeopacity{0.500000}%
\pgfsetdash{}{0pt}%
\pgfpathmoveto{\pgfqpoint{4.072295in}{2.000602in}}%
\pgfpathlineto{\pgfqpoint{4.083538in}{2.005097in}}%
\pgfpathlineto{\pgfqpoint{4.094773in}{2.009497in}}%
\pgfpathlineto{\pgfqpoint{4.106002in}{2.013819in}}%
\pgfpathlineto{\pgfqpoint{4.117223in}{2.018082in}}%
\pgfpathlineto{\pgfqpoint{4.128438in}{2.022302in}}%
\pgfpathlineto{\pgfqpoint{4.121994in}{2.038225in}}%
\pgfpathlineto{\pgfqpoint{4.115555in}{2.054276in}}%
\pgfpathlineto{\pgfqpoint{4.109119in}{2.070373in}}%
\pgfpathlineto{\pgfqpoint{4.102685in}{2.086431in}}%
\pgfpathlineto{\pgfqpoint{4.096249in}{2.102367in}}%
\pgfpathlineto{\pgfqpoint{4.085048in}{2.098689in}}%
\pgfpathlineto{\pgfqpoint{4.073839in}{2.094920in}}%
\pgfpathlineto{\pgfqpoint{4.062622in}{2.091035in}}%
\pgfpathlineto{\pgfqpoint{4.051396in}{2.087007in}}%
\pgfpathlineto{\pgfqpoint{4.040161in}{2.082813in}}%
\pgfpathlineto{\pgfqpoint{4.046587in}{2.066492in}}%
\pgfpathlineto{\pgfqpoint{4.053011in}{2.050010in}}%
\pgfpathlineto{\pgfqpoint{4.059435in}{2.033469in}}%
\pgfpathlineto{\pgfqpoint{4.065862in}{2.016966in}}%
\pgfpathclose%
\pgfusepath{stroke,fill}%
\end{pgfscope}%
\begin{pgfscope}%
\pgfpathrectangle{\pgfqpoint{0.887500in}{0.275000in}}{\pgfqpoint{4.225000in}{4.225000in}}%
\pgfusepath{clip}%
\pgfsetbuttcap%
\pgfsetroundjoin%
\definecolor{currentfill}{rgb}{0.151918,0.500685,0.557587}%
\pgfsetfillcolor{currentfill}%
\pgfsetfillopacity{0.700000}%
\pgfsetlinewidth{0.501875pt}%
\definecolor{currentstroke}{rgb}{1.000000,1.000000,1.000000}%
\pgfsetstrokecolor{currentstroke}%
\pgfsetstrokeopacity{0.500000}%
\pgfsetdash{}{0pt}%
\pgfpathmoveto{\pgfqpoint{1.810346in}{2.170877in}}%
\pgfpathlineto{\pgfqpoint{1.822128in}{2.174455in}}%
\pgfpathlineto{\pgfqpoint{1.833905in}{2.178024in}}%
\pgfpathlineto{\pgfqpoint{1.845677in}{2.181587in}}%
\pgfpathlineto{\pgfqpoint{1.857443in}{2.185144in}}%
\pgfpathlineto{\pgfqpoint{1.869203in}{2.188697in}}%
\pgfpathlineto{\pgfqpoint{1.863304in}{2.197644in}}%
\pgfpathlineto{\pgfqpoint{1.857411in}{2.206564in}}%
\pgfpathlineto{\pgfqpoint{1.851521in}{2.215457in}}%
\pgfpathlineto{\pgfqpoint{1.845637in}{2.224322in}}%
\pgfpathlineto{\pgfqpoint{1.839757in}{2.233162in}}%
\pgfpathlineto{\pgfqpoint{1.828008in}{2.229656in}}%
\pgfpathlineto{\pgfqpoint{1.816253in}{2.226146in}}%
\pgfpathlineto{\pgfqpoint{1.804492in}{2.222631in}}%
\pgfpathlineto{\pgfqpoint{1.792726in}{2.219110in}}%
\pgfpathlineto{\pgfqpoint{1.780955in}{2.215581in}}%
\pgfpathlineto{\pgfqpoint{1.786824in}{2.206695in}}%
\pgfpathlineto{\pgfqpoint{1.792698in}{2.197782in}}%
\pgfpathlineto{\pgfqpoint{1.798576in}{2.188841in}}%
\pgfpathlineto{\pgfqpoint{1.804458in}{2.179873in}}%
\pgfpathclose%
\pgfusepath{stroke,fill}%
\end{pgfscope}%
\begin{pgfscope}%
\pgfpathrectangle{\pgfqpoint{0.887500in}{0.275000in}}{\pgfqpoint{4.225000in}{4.225000in}}%
\pgfusepath{clip}%
\pgfsetbuttcap%
\pgfsetroundjoin%
\definecolor{currentfill}{rgb}{0.123463,0.581687,0.547445}%
\pgfsetfillcolor{currentfill}%
\pgfsetfillopacity{0.700000}%
\pgfsetlinewidth{0.501875pt}%
\definecolor{currentstroke}{rgb}{1.000000,1.000000,1.000000}%
\pgfsetstrokecolor{currentstroke}%
\pgfsetstrokeopacity{0.500000}%
\pgfsetdash{}{0pt}%
\pgfpathmoveto{\pgfqpoint{3.596590in}{2.314825in}}%
\pgfpathlineto{\pgfqpoint{3.608001in}{2.323832in}}%
\pgfpathlineto{\pgfqpoint{3.619403in}{2.332391in}}%
\pgfpathlineto{\pgfqpoint{3.630796in}{2.340507in}}%
\pgfpathlineto{\pgfqpoint{3.642180in}{2.348184in}}%
\pgfpathlineto{\pgfqpoint{3.653553in}{2.355426in}}%
\pgfpathlineto{\pgfqpoint{3.647171in}{2.370830in}}%
\pgfpathlineto{\pgfqpoint{3.640792in}{2.386328in}}%
\pgfpathlineto{\pgfqpoint{3.634416in}{2.401878in}}%
\pgfpathlineto{\pgfqpoint{3.628043in}{2.417435in}}%
\pgfpathlineto{\pgfqpoint{3.621670in}{2.432957in}}%
\pgfpathlineto{\pgfqpoint{3.610319in}{2.426999in}}%
\pgfpathlineto{\pgfqpoint{3.598958in}{2.420703in}}%
\pgfpathlineto{\pgfqpoint{3.587588in}{2.414027in}}%
\pgfpathlineto{\pgfqpoint{3.576209in}{2.406934in}}%
\pgfpathlineto{\pgfqpoint{3.564821in}{2.399384in}}%
\pgfpathlineto{\pgfqpoint{3.571174in}{2.382677in}}%
\pgfpathlineto{\pgfqpoint{3.577527in}{2.365791in}}%
\pgfpathlineto{\pgfqpoint{3.583880in}{2.348802in}}%
\pgfpathlineto{\pgfqpoint{3.590234in}{2.331788in}}%
\pgfpathclose%
\pgfusepath{stroke,fill}%
\end{pgfscope}%
\begin{pgfscope}%
\pgfpathrectangle{\pgfqpoint{0.887500in}{0.275000in}}{\pgfqpoint{4.225000in}{4.225000in}}%
\pgfusepath{clip}%
\pgfsetbuttcap%
\pgfsetroundjoin%
\definecolor{currentfill}{rgb}{0.128729,0.563265,0.551229}%
\pgfsetfillcolor{currentfill}%
\pgfsetfillopacity{0.700000}%
\pgfsetlinewidth{0.501875pt}%
\definecolor{currentstroke}{rgb}{1.000000,1.000000,1.000000}%
\pgfsetstrokecolor{currentstroke}%
\pgfsetstrokeopacity{0.500000}%
\pgfsetdash{}{0pt}%
\pgfpathmoveto{\pgfqpoint{3.685540in}{2.281117in}}%
\pgfpathlineto{\pgfqpoint{3.696920in}{2.288629in}}%
\pgfpathlineto{\pgfqpoint{3.708284in}{2.295383in}}%
\pgfpathlineto{\pgfqpoint{3.719633in}{2.301496in}}%
\pgfpathlineto{\pgfqpoint{3.730969in}{2.307083in}}%
\pgfpathlineto{\pgfqpoint{3.742294in}{2.312264in}}%
\pgfpathlineto{\pgfqpoint{3.735886in}{2.327019in}}%
\pgfpathlineto{\pgfqpoint{3.729480in}{2.341745in}}%
\pgfpathlineto{\pgfqpoint{3.723076in}{2.356452in}}%
\pgfpathlineto{\pgfqpoint{3.716674in}{2.371142in}}%
\pgfpathlineto{\pgfqpoint{3.710275in}{2.385813in}}%
\pgfpathlineto{\pgfqpoint{3.698949in}{2.380395in}}%
\pgfpathlineto{\pgfqpoint{3.687615in}{2.374685in}}%
\pgfpathlineto{\pgfqpoint{3.676271in}{2.368646in}}%
\pgfpathlineto{\pgfqpoint{3.664917in}{2.362239in}}%
\pgfpathlineto{\pgfqpoint{3.653553in}{2.355426in}}%
\pgfpathlineto{\pgfqpoint{3.659940in}{2.340162in}}%
\pgfpathlineto{\pgfqpoint{3.666332in}{2.325080in}}%
\pgfpathlineto{\pgfqpoint{3.672729in}{2.310222in}}%
\pgfpathlineto{\pgfqpoint{3.679132in}{2.295589in}}%
\pgfpathclose%
\pgfusepath{stroke,fill}%
\end{pgfscope}%
\begin{pgfscope}%
\pgfpathrectangle{\pgfqpoint{0.887500in}{0.275000in}}{\pgfqpoint{4.225000in}{4.225000in}}%
\pgfusepath{clip}%
\pgfsetbuttcap%
\pgfsetroundjoin%
\definecolor{currentfill}{rgb}{0.206756,0.371758,0.553117}%
\pgfsetfillcolor{currentfill}%
\pgfsetfillopacity{0.700000}%
\pgfsetlinewidth{0.501875pt}%
\definecolor{currentstroke}{rgb}{1.000000,1.000000,1.000000}%
\pgfsetstrokecolor{currentstroke}%
\pgfsetstrokeopacity{0.500000}%
\pgfsetdash{}{0pt}%
\pgfpathmoveto{\pgfqpoint{2.960106in}{1.898890in}}%
\pgfpathlineto{\pgfqpoint{2.971605in}{1.902893in}}%
\pgfpathlineto{\pgfqpoint{2.983098in}{1.907045in}}%
\pgfpathlineto{\pgfqpoint{2.994585in}{1.911200in}}%
\pgfpathlineto{\pgfqpoint{3.006068in}{1.915211in}}%
\pgfpathlineto{\pgfqpoint{3.017545in}{1.918928in}}%
\pgfpathlineto{\pgfqpoint{3.011270in}{1.929381in}}%
\pgfpathlineto{\pgfqpoint{3.004999in}{1.939776in}}%
\pgfpathlineto{\pgfqpoint{2.998731in}{1.950115in}}%
\pgfpathlineto{\pgfqpoint{2.992467in}{1.960395in}}%
\pgfpathlineto{\pgfqpoint{2.986207in}{1.970617in}}%
\pgfpathlineto{\pgfqpoint{2.974739in}{1.966922in}}%
\pgfpathlineto{\pgfqpoint{2.963267in}{1.962818in}}%
\pgfpathlineto{\pgfqpoint{2.951789in}{1.958510in}}%
\pgfpathlineto{\pgfqpoint{2.940306in}{1.954201in}}%
\pgfpathlineto{\pgfqpoint{2.928817in}{1.950091in}}%
\pgfpathlineto{\pgfqpoint{2.935068in}{1.939945in}}%
\pgfpathlineto{\pgfqpoint{2.941321in}{1.929751in}}%
\pgfpathlineto{\pgfqpoint{2.947579in}{1.919511in}}%
\pgfpathlineto{\pgfqpoint{2.953841in}{1.909224in}}%
\pgfpathclose%
\pgfusepath{stroke,fill}%
\end{pgfscope}%
\begin{pgfscope}%
\pgfpathrectangle{\pgfqpoint{0.887500in}{0.275000in}}{\pgfqpoint{4.225000in}{4.225000in}}%
\pgfusepath{clip}%
\pgfsetbuttcap%
\pgfsetroundjoin%
\definecolor{currentfill}{rgb}{0.182256,0.426184,0.557120}%
\pgfsetfillcolor{currentfill}%
\pgfsetfillopacity{0.700000}%
\pgfsetlinewidth{0.501875pt}%
\definecolor{currentstroke}{rgb}{1.000000,1.000000,1.000000}%
\pgfsetstrokecolor{currentstroke}%
\pgfsetstrokeopacity{0.500000}%
\pgfsetdash{}{0pt}%
\pgfpathmoveto{\pgfqpoint{2.547167in}{2.008734in}}%
\pgfpathlineto{\pgfqpoint{2.558769in}{2.012347in}}%
\pgfpathlineto{\pgfqpoint{2.570366in}{2.015950in}}%
\pgfpathlineto{\pgfqpoint{2.581958in}{2.019545in}}%
\pgfpathlineto{\pgfqpoint{2.593544in}{2.023138in}}%
\pgfpathlineto{\pgfqpoint{2.605124in}{2.026732in}}%
\pgfpathlineto{\pgfqpoint{2.598977in}{2.036368in}}%
\pgfpathlineto{\pgfqpoint{2.592835in}{2.045960in}}%
\pgfpathlineto{\pgfqpoint{2.586697in}{2.055509in}}%
\pgfpathlineto{\pgfqpoint{2.580563in}{2.065016in}}%
\pgfpathlineto{\pgfqpoint{2.574433in}{2.074482in}}%
\pgfpathlineto{\pgfqpoint{2.562863in}{2.070943in}}%
\pgfpathlineto{\pgfqpoint{2.551288in}{2.067405in}}%
\pgfpathlineto{\pgfqpoint{2.539706in}{2.063864in}}%
\pgfpathlineto{\pgfqpoint{2.528119in}{2.060316in}}%
\pgfpathlineto{\pgfqpoint{2.516526in}{2.056757in}}%
\pgfpathlineto{\pgfqpoint{2.522646in}{2.047238in}}%
\pgfpathlineto{\pgfqpoint{2.528770in}{2.037677in}}%
\pgfpathlineto{\pgfqpoint{2.534898in}{2.028073in}}%
\pgfpathlineto{\pgfqpoint{2.541030in}{2.018426in}}%
\pgfpathclose%
\pgfusepath{stroke,fill}%
\end{pgfscope}%
\begin{pgfscope}%
\pgfpathrectangle{\pgfqpoint{0.887500in}{0.275000in}}{\pgfqpoint{4.225000in}{4.225000in}}%
\pgfusepath{clip}%
\pgfsetbuttcap%
\pgfsetroundjoin%
\definecolor{currentfill}{rgb}{0.135066,0.544853,0.554029}%
\pgfsetfillcolor{currentfill}%
\pgfsetfillopacity{0.700000}%
\pgfsetlinewidth{0.501875pt}%
\definecolor{currentstroke}{rgb}{1.000000,1.000000,1.000000}%
\pgfsetstrokecolor{currentstroke}%
\pgfsetstrokeopacity{0.500000}%
\pgfsetdash{}{0pt}%
\pgfpathmoveto{\pgfqpoint{3.393005in}{2.215822in}}%
\pgfpathlineto{\pgfqpoint{3.404510in}{2.233411in}}%
\pgfpathlineto{\pgfqpoint{3.416009in}{2.249926in}}%
\pgfpathlineto{\pgfqpoint{3.427500in}{2.265458in}}%
\pgfpathlineto{\pgfqpoint{3.438984in}{2.280096in}}%
\pgfpathlineto{\pgfqpoint{3.450460in}{2.293930in}}%
\pgfpathlineto{\pgfqpoint{3.444115in}{2.310247in}}%
\pgfpathlineto{\pgfqpoint{3.437766in}{2.326014in}}%
\pgfpathlineto{\pgfqpoint{3.431413in}{2.341161in}}%
\pgfpathlineto{\pgfqpoint{3.425057in}{2.355674in}}%
\pgfpathlineto{\pgfqpoint{3.418698in}{2.369682in}}%
\pgfpathlineto{\pgfqpoint{3.407216in}{2.354041in}}%
\pgfpathlineto{\pgfqpoint{3.395729in}{2.337772in}}%
\pgfpathlineto{\pgfqpoint{3.384241in}{2.321048in}}%
\pgfpathlineto{\pgfqpoint{3.372752in}{2.304040in}}%
\pgfpathlineto{\pgfqpoint{3.361264in}{2.286920in}}%
\pgfpathlineto{\pgfqpoint{3.367614in}{2.273546in}}%
\pgfpathlineto{\pgfqpoint{3.373964in}{2.259837in}}%
\pgfpathlineto{\pgfqpoint{3.380312in}{2.245651in}}%
\pgfpathlineto{\pgfqpoint{3.386659in}{2.230969in}}%
\pgfpathclose%
\pgfusepath{stroke,fill}%
\end{pgfscope}%
\begin{pgfscope}%
\pgfpathrectangle{\pgfqpoint{0.887500in}{0.275000in}}{\pgfqpoint{4.225000in}{4.225000in}}%
\pgfusepath{clip}%
\pgfsetbuttcap%
\pgfsetroundjoin%
\definecolor{currentfill}{rgb}{0.214298,0.355619,0.551184}%
\pgfsetfillcolor{currentfill}%
\pgfsetfillopacity{0.700000}%
\pgfsetlinewidth{0.501875pt}%
\definecolor{currentstroke}{rgb}{1.000000,1.000000,1.000000}%
\pgfsetstrokecolor{currentstroke}%
\pgfsetstrokeopacity{0.500000}%
\pgfsetdash{}{0pt}%
\pgfpathmoveto{\pgfqpoint{3.048974in}{1.865844in}}%
\pgfpathlineto{\pgfqpoint{3.060455in}{1.869358in}}%
\pgfpathlineto{\pgfqpoint{3.071929in}{1.872578in}}%
\pgfpathlineto{\pgfqpoint{3.083398in}{1.875444in}}%
\pgfpathlineto{\pgfqpoint{3.094860in}{1.878092in}}%
\pgfpathlineto{\pgfqpoint{3.106316in}{1.880792in}}%
\pgfpathlineto{\pgfqpoint{3.100014in}{1.890724in}}%
\pgfpathlineto{\pgfqpoint{3.093716in}{1.900609in}}%
\pgfpathlineto{\pgfqpoint{3.087421in}{1.910459in}}%
\pgfpathlineto{\pgfqpoint{3.081131in}{1.920280in}}%
\pgfpathlineto{\pgfqpoint{3.074844in}{1.930077in}}%
\pgfpathlineto{\pgfqpoint{3.063397in}{1.928422in}}%
\pgfpathlineto{\pgfqpoint{3.051943in}{1.926871in}}%
\pgfpathlineto{\pgfqpoint{3.040483in}{1.924891in}}%
\pgfpathlineto{\pgfqpoint{3.029017in}{1.922204in}}%
\pgfpathlineto{\pgfqpoint{3.017545in}{1.918928in}}%
\pgfpathlineto{\pgfqpoint{3.023823in}{1.908420in}}%
\pgfpathlineto{\pgfqpoint{3.030106in}{1.897855in}}%
\pgfpathlineto{\pgfqpoint{3.036392in}{1.887237in}}%
\pgfpathlineto{\pgfqpoint{3.042681in}{1.876565in}}%
\pgfpathclose%
\pgfusepath{stroke,fill}%
\end{pgfscope}%
\begin{pgfscope}%
\pgfpathrectangle{\pgfqpoint{0.887500in}{0.275000in}}{\pgfqpoint{4.225000in}{4.225000in}}%
\pgfusepath{clip}%
\pgfsetbuttcap%
\pgfsetroundjoin%
\definecolor{currentfill}{rgb}{0.168126,0.459988,0.558082}%
\pgfsetfillcolor{currentfill}%
\pgfsetfillopacity{0.700000}%
\pgfsetlinewidth{0.501875pt}%
\definecolor{currentstroke}{rgb}{1.000000,1.000000,1.000000}%
\pgfsetstrokecolor{currentstroke}%
\pgfsetstrokeopacity{0.500000}%
\pgfsetdash{}{0pt}%
\pgfpathmoveto{\pgfqpoint{2.222972in}{2.077821in}}%
\pgfpathlineto{\pgfqpoint{2.234656in}{2.081446in}}%
\pgfpathlineto{\pgfqpoint{2.246334in}{2.085071in}}%
\pgfpathlineto{\pgfqpoint{2.258007in}{2.088700in}}%
\pgfpathlineto{\pgfqpoint{2.269673in}{2.092334in}}%
\pgfpathlineto{\pgfqpoint{2.281334in}{2.095976in}}%
\pgfpathlineto{\pgfqpoint{2.275294in}{2.105247in}}%
\pgfpathlineto{\pgfqpoint{2.269258in}{2.114485in}}%
\pgfpathlineto{\pgfqpoint{2.263226in}{2.123691in}}%
\pgfpathlineto{\pgfqpoint{2.257199in}{2.132867in}}%
\pgfpathlineto{\pgfqpoint{2.251176in}{2.142012in}}%
\pgfpathlineto{\pgfqpoint{2.239526in}{2.138422in}}%
\pgfpathlineto{\pgfqpoint{2.227870in}{2.134841in}}%
\pgfpathlineto{\pgfqpoint{2.216208in}{2.131267in}}%
\pgfpathlineto{\pgfqpoint{2.204540in}{2.127696in}}%
\pgfpathlineto{\pgfqpoint{2.192867in}{2.124126in}}%
\pgfpathlineto{\pgfqpoint{2.198879in}{2.114928in}}%
\pgfpathlineto{\pgfqpoint{2.204896in}{2.105699in}}%
\pgfpathlineto{\pgfqpoint{2.210917in}{2.096439in}}%
\pgfpathlineto{\pgfqpoint{2.216942in}{2.087147in}}%
\pgfpathclose%
\pgfusepath{stroke,fill}%
\end{pgfscope}%
\begin{pgfscope}%
\pgfpathrectangle{\pgfqpoint{0.887500in}{0.275000in}}{\pgfqpoint{4.225000in}{4.225000in}}%
\pgfusepath{clip}%
\pgfsetbuttcap%
\pgfsetroundjoin%
\definecolor{currentfill}{rgb}{0.124395,0.578002,0.548287}%
\pgfsetfillcolor{currentfill}%
\pgfsetfillopacity{0.700000}%
\pgfsetlinewidth{0.501875pt}%
\definecolor{currentstroke}{rgb}{1.000000,1.000000,1.000000}%
\pgfsetstrokecolor{currentstroke}%
\pgfsetstrokeopacity{0.500000}%
\pgfsetdash{}{0pt}%
\pgfpathmoveto{\pgfqpoint{3.450460in}{2.293930in}}%
\pgfpathlineto{\pgfqpoint{3.461930in}{2.307052in}}%
\pgfpathlineto{\pgfqpoint{3.473392in}{2.319551in}}%
\pgfpathlineto{\pgfqpoint{3.484849in}{2.331484in}}%
\pgfpathlineto{\pgfqpoint{3.496298in}{2.342856in}}%
\pgfpathlineto{\pgfqpoint{3.507740in}{2.353668in}}%
\pgfpathlineto{\pgfqpoint{3.501398in}{2.370872in}}%
\pgfpathlineto{\pgfqpoint{3.495052in}{2.387568in}}%
\pgfpathlineto{\pgfqpoint{3.488700in}{2.403659in}}%
\pgfpathlineto{\pgfqpoint{3.482344in}{2.419101in}}%
\pgfpathlineto{\pgfqpoint{3.475984in}{2.433985in}}%
\pgfpathlineto{\pgfqpoint{3.464547in}{2.423052in}}%
\pgfpathlineto{\pgfqpoint{3.453100in}{2.411213in}}%
\pgfpathlineto{\pgfqpoint{3.441642in}{2.398392in}}%
\pgfpathlineto{\pgfqpoint{3.430174in}{2.384522in}}%
\pgfpathlineto{\pgfqpoint{3.418698in}{2.369682in}}%
\pgfpathlineto{\pgfqpoint{3.425057in}{2.355674in}}%
\pgfpathlineto{\pgfqpoint{3.431413in}{2.341161in}}%
\pgfpathlineto{\pgfqpoint{3.437766in}{2.326014in}}%
\pgfpathlineto{\pgfqpoint{3.444115in}{2.310247in}}%
\pgfpathclose%
\pgfusepath{stroke,fill}%
\end{pgfscope}%
\begin{pgfscope}%
\pgfpathrectangle{\pgfqpoint{0.887500in}{0.275000in}}{\pgfqpoint{4.225000in}{4.225000in}}%
\pgfusepath{clip}%
\pgfsetbuttcap%
\pgfsetroundjoin%
\definecolor{currentfill}{rgb}{0.231674,0.318106,0.544834}%
\pgfsetfillcolor{currentfill}%
\pgfsetfillopacity{0.700000}%
\pgfsetlinewidth{0.501875pt}%
\definecolor{currentstroke}{rgb}{1.000000,1.000000,1.000000}%
\pgfsetstrokecolor{currentstroke}%
\pgfsetstrokeopacity{0.500000}%
\pgfsetdash{}{0pt}%
\pgfpathmoveto{\pgfqpoint{4.370961in}{1.784915in}}%
\pgfpathlineto{\pgfqpoint{4.382098in}{1.788120in}}%
\pgfpathlineto{\pgfqpoint{4.393227in}{1.791280in}}%
\pgfpathlineto{\pgfqpoint{4.404349in}{1.794401in}}%
\pgfpathlineto{\pgfqpoint{4.415465in}{1.797491in}}%
\pgfpathlineto{\pgfqpoint{4.426574in}{1.800556in}}%
\pgfpathlineto{\pgfqpoint{4.420033in}{1.815133in}}%
\pgfpathlineto{\pgfqpoint{4.413494in}{1.829639in}}%
\pgfpathlineto{\pgfqpoint{4.406954in}{1.844079in}}%
\pgfpathlineto{\pgfqpoint{4.400415in}{1.858457in}}%
\pgfpathlineto{\pgfqpoint{4.393877in}{1.872779in}}%
\pgfpathlineto{\pgfqpoint{4.382759in}{1.869420in}}%
\pgfpathlineto{\pgfqpoint{4.371634in}{1.866020in}}%
\pgfpathlineto{\pgfqpoint{4.360502in}{1.862571in}}%
\pgfpathlineto{\pgfqpoint{4.349363in}{1.859062in}}%
\pgfpathlineto{\pgfqpoint{4.338216in}{1.855484in}}%
\pgfpathlineto{\pgfqpoint{4.344769in}{1.841646in}}%
\pgfpathlineto{\pgfqpoint{4.351320in}{1.827693in}}%
\pgfpathlineto{\pgfqpoint{4.357870in}{1.813602in}}%
\pgfpathlineto{\pgfqpoint{4.364417in}{1.799350in}}%
\pgfpathclose%
\pgfusepath{stroke,fill}%
\end{pgfscope}%
\begin{pgfscope}%
\pgfpathrectangle{\pgfqpoint{0.887500in}{0.275000in}}{\pgfqpoint{4.225000in}{4.225000in}}%
\pgfusepath{clip}%
\pgfsetbuttcap%
\pgfsetroundjoin%
\definecolor{currentfill}{rgb}{0.165117,0.467423,0.558141}%
\pgfsetfillcolor{currentfill}%
\pgfsetfillopacity{0.700000}%
\pgfsetlinewidth{0.501875pt}%
\definecolor{currentstroke}{rgb}{1.000000,1.000000,1.000000}%
\pgfsetstrokecolor{currentstroke}%
\pgfsetstrokeopacity{0.500000}%
\pgfsetdash{}{0pt}%
\pgfpathmoveto{\pgfqpoint{3.366992in}{2.025072in}}%
\pgfpathlineto{\pgfqpoint{3.378534in}{2.047939in}}%
\pgfpathlineto{\pgfqpoint{3.390082in}{2.070891in}}%
\pgfpathlineto{\pgfqpoint{3.401632in}{2.093321in}}%
\pgfpathlineto{\pgfqpoint{3.413176in}{2.114623in}}%
\pgfpathlineto{\pgfqpoint{3.424709in}{2.134189in}}%
\pgfpathlineto{\pgfqpoint{3.418370in}{2.151199in}}%
\pgfpathlineto{\pgfqpoint{3.412031in}{2.167898in}}%
\pgfpathlineto{\pgfqpoint{3.405690in}{2.184255in}}%
\pgfpathlineto{\pgfqpoint{3.399348in}{2.200240in}}%
\pgfpathlineto{\pgfqpoint{3.393005in}{2.215822in}}%
\pgfpathlineto{\pgfqpoint{3.381492in}{2.197105in}}%
\pgfpathlineto{\pgfqpoint{3.369975in}{2.177477in}}%
\pgfpathlineto{\pgfqpoint{3.358458in}{2.157277in}}%
\pgfpathlineto{\pgfqpoint{3.346943in}{2.136842in}}%
\pgfpathlineto{\pgfqpoint{3.335433in}{2.116511in}}%
\pgfpathlineto{\pgfqpoint{3.341743in}{2.098271in}}%
\pgfpathlineto{\pgfqpoint{3.348053in}{2.079908in}}%
\pgfpathlineto{\pgfqpoint{3.354364in}{2.061522in}}%
\pgfpathlineto{\pgfqpoint{3.360677in}{2.043210in}}%
\pgfpathclose%
\pgfusepath{stroke,fill}%
\end{pgfscope}%
\begin{pgfscope}%
\pgfpathrectangle{\pgfqpoint{0.887500in}{0.275000in}}{\pgfqpoint{4.225000in}{4.225000in}}%
\pgfusepath{clip}%
\pgfsetbuttcap%
\pgfsetroundjoin%
\definecolor{currentfill}{rgb}{0.168126,0.459988,0.558082}%
\pgfsetfillcolor{currentfill}%
\pgfsetfillopacity{0.700000}%
\pgfsetlinewidth{0.501875pt}%
\definecolor{currentstroke}{rgb}{1.000000,1.000000,1.000000}%
\pgfsetstrokecolor{currentstroke}%
\pgfsetstrokeopacity{0.500000}%
\pgfsetdash{}{0pt}%
\pgfpathmoveto{\pgfqpoint{3.983842in}{2.059067in}}%
\pgfpathlineto{\pgfqpoint{3.995125in}{2.064211in}}%
\pgfpathlineto{\pgfqpoint{4.006398in}{2.069150in}}%
\pgfpathlineto{\pgfqpoint{4.017662in}{2.073892in}}%
\pgfpathlineto{\pgfqpoint{4.028916in}{2.078444in}}%
\pgfpathlineto{\pgfqpoint{4.040161in}{2.082813in}}%
\pgfpathlineto{\pgfqpoint{4.033730in}{2.098876in}}%
\pgfpathlineto{\pgfqpoint{4.027293in}{2.114585in}}%
\pgfpathlineto{\pgfqpoint{4.020848in}{2.129923in}}%
\pgfpathlineto{\pgfqpoint{4.014398in}{2.144938in}}%
\pgfpathlineto{\pgfqpoint{4.007944in}{2.159681in}}%
\pgfpathlineto{\pgfqpoint{3.996703in}{2.155429in}}%
\pgfpathlineto{\pgfqpoint{3.985453in}{2.150976in}}%
\pgfpathlineto{\pgfqpoint{3.974193in}{2.146323in}}%
\pgfpathlineto{\pgfqpoint{3.962923in}{2.141473in}}%
\pgfpathlineto{\pgfqpoint{3.951645in}{2.136427in}}%
\pgfpathlineto{\pgfqpoint{3.958093in}{2.121546in}}%
\pgfpathlineto{\pgfqpoint{3.964538in}{2.106414in}}%
\pgfpathlineto{\pgfqpoint{3.970979in}{2.090980in}}%
\pgfpathlineto{\pgfqpoint{3.977413in}{2.075195in}}%
\pgfpathclose%
\pgfusepath{stroke,fill}%
\end{pgfscope}%
\begin{pgfscope}%
\pgfpathrectangle{\pgfqpoint{0.887500in}{0.275000in}}{\pgfqpoint{4.225000in}{4.225000in}}%
\pgfusepath{clip}%
\pgfsetbuttcap%
\pgfsetroundjoin%
\definecolor{currentfill}{rgb}{0.156270,0.489624,0.557936}%
\pgfsetfillcolor{currentfill}%
\pgfsetfillopacity{0.700000}%
\pgfsetlinewidth{0.501875pt}%
\definecolor{currentstroke}{rgb}{1.000000,1.000000,1.000000}%
\pgfsetstrokecolor{currentstroke}%
\pgfsetstrokeopacity{0.500000}%
\pgfsetdash{}{0pt}%
\pgfpathmoveto{\pgfqpoint{1.898761in}{2.143536in}}%
\pgfpathlineto{\pgfqpoint{1.910527in}{2.147135in}}%
\pgfpathlineto{\pgfqpoint{1.922286in}{2.150732in}}%
\pgfpathlineto{\pgfqpoint{1.934039in}{2.154327in}}%
\pgfpathlineto{\pgfqpoint{1.945787in}{2.157921in}}%
\pgfpathlineto{\pgfqpoint{1.957529in}{2.161516in}}%
\pgfpathlineto{\pgfqpoint{1.951598in}{2.170554in}}%
\pgfpathlineto{\pgfqpoint{1.945670in}{2.179564in}}%
\pgfpathlineto{\pgfqpoint{1.939748in}{2.188547in}}%
\pgfpathlineto{\pgfqpoint{1.933830in}{2.197503in}}%
\pgfpathlineto{\pgfqpoint{1.927916in}{2.206432in}}%
\pgfpathlineto{\pgfqpoint{1.916185in}{2.202885in}}%
\pgfpathlineto{\pgfqpoint{1.904448in}{2.199339in}}%
\pgfpathlineto{\pgfqpoint{1.892705in}{2.195793in}}%
\pgfpathlineto{\pgfqpoint{1.880957in}{2.192246in}}%
\pgfpathlineto{\pgfqpoint{1.869203in}{2.188697in}}%
\pgfpathlineto{\pgfqpoint{1.875105in}{2.179721in}}%
\pgfpathlineto{\pgfqpoint{1.881013in}{2.170718in}}%
\pgfpathlineto{\pgfqpoint{1.886924in}{2.161686in}}%
\pgfpathlineto{\pgfqpoint{1.892841in}{2.152626in}}%
\pgfpathclose%
\pgfusepath{stroke,fill}%
\end{pgfscope}%
\begin{pgfscope}%
\pgfpathrectangle{\pgfqpoint{0.887500in}{0.275000in}}{\pgfqpoint{4.225000in}{4.225000in}}%
\pgfusepath{clip}%
\pgfsetbuttcap%
\pgfsetroundjoin%
\definecolor{currentfill}{rgb}{0.187231,0.414746,0.556547}%
\pgfsetfillcolor{currentfill}%
\pgfsetfillopacity{0.700000}%
\pgfsetlinewidth{0.501875pt}%
\definecolor{currentstroke}{rgb}{1.000000,1.000000,1.000000}%
\pgfsetstrokecolor{currentstroke}%
\pgfsetstrokeopacity{0.500000}%
\pgfsetdash{}{0pt}%
\pgfpathmoveto{\pgfqpoint{2.635920in}{1.977861in}}%
\pgfpathlineto{\pgfqpoint{2.647504in}{1.981514in}}%
\pgfpathlineto{\pgfqpoint{2.659083in}{1.985173in}}%
\pgfpathlineto{\pgfqpoint{2.670655in}{1.988840in}}%
\pgfpathlineto{\pgfqpoint{2.682222in}{1.992517in}}%
\pgfpathlineto{\pgfqpoint{2.693783in}{1.996205in}}%
\pgfpathlineto{\pgfqpoint{2.687605in}{2.006019in}}%
\pgfpathlineto{\pgfqpoint{2.681432in}{2.015786in}}%
\pgfpathlineto{\pgfqpoint{2.675262in}{2.025508in}}%
\pgfpathlineto{\pgfqpoint{2.669097in}{2.035184in}}%
\pgfpathlineto{\pgfqpoint{2.662937in}{2.044814in}}%
\pgfpathlineto{\pgfqpoint{2.651386in}{2.041175in}}%
\pgfpathlineto{\pgfqpoint{2.639829in}{2.037549in}}%
\pgfpathlineto{\pgfqpoint{2.628266in}{2.033935in}}%
\pgfpathlineto{\pgfqpoint{2.616698in}{2.030330in}}%
\pgfpathlineto{\pgfqpoint{2.605124in}{2.026732in}}%
\pgfpathlineto{\pgfqpoint{2.611274in}{2.017050in}}%
\pgfpathlineto{\pgfqpoint{2.617430in}{2.007322in}}%
\pgfpathlineto{\pgfqpoint{2.623589in}{1.997548in}}%
\pgfpathlineto{\pgfqpoint{2.629752in}{1.987728in}}%
\pgfpathclose%
\pgfusepath{stroke,fill}%
\end{pgfscope}%
\begin{pgfscope}%
\pgfpathrectangle{\pgfqpoint{0.887500in}{0.275000in}}{\pgfqpoint{4.225000in}{4.225000in}}%
\pgfusepath{clip}%
\pgfsetbuttcap%
\pgfsetroundjoin%
\definecolor{currentfill}{rgb}{0.221989,0.339161,0.548752}%
\pgfsetfillcolor{currentfill}%
\pgfsetfillopacity{0.700000}%
\pgfsetlinewidth{0.501875pt}%
\definecolor{currentstroke}{rgb}{1.000000,1.000000,1.000000}%
\pgfsetstrokecolor{currentstroke}%
\pgfsetstrokeopacity{0.500000}%
\pgfsetdash{}{0pt}%
\pgfpathmoveto{\pgfqpoint{3.137882in}{1.830100in}}%
\pgfpathlineto{\pgfqpoint{3.149342in}{1.833793in}}%
\pgfpathlineto{\pgfqpoint{3.160798in}{1.837521in}}%
\pgfpathlineto{\pgfqpoint{3.172247in}{1.841300in}}%
\pgfpathlineto{\pgfqpoint{3.183692in}{1.845150in}}%
\pgfpathlineto{\pgfqpoint{3.195131in}{1.849100in}}%
\pgfpathlineto{\pgfqpoint{3.188804in}{1.860224in}}%
\pgfpathlineto{\pgfqpoint{3.182481in}{1.871334in}}%
\pgfpathlineto{\pgfqpoint{3.176161in}{1.882423in}}%
\pgfpathlineto{\pgfqpoint{3.169844in}{1.893481in}}%
\pgfpathlineto{\pgfqpoint{3.163531in}{1.904500in}}%
\pgfpathlineto{\pgfqpoint{3.152092in}{1.897519in}}%
\pgfpathlineto{\pgfqpoint{3.140653in}{1.891908in}}%
\pgfpathlineto{\pgfqpoint{3.129212in}{1.887430in}}%
\pgfpathlineto{\pgfqpoint{3.117766in}{1.883815in}}%
\pgfpathlineto{\pgfqpoint{3.106316in}{1.880792in}}%
\pgfpathlineto{\pgfqpoint{3.112622in}{1.870805in}}%
\pgfpathlineto{\pgfqpoint{3.118931in}{1.860752in}}%
\pgfpathlineto{\pgfqpoint{3.125245in}{1.850624in}}%
\pgfpathlineto{\pgfqpoint{3.131561in}{1.840410in}}%
\pgfpathclose%
\pgfusepath{stroke,fill}%
\end{pgfscope}%
\begin{pgfscope}%
\pgfpathrectangle{\pgfqpoint{0.887500in}{0.275000in}}{\pgfqpoint{4.225000in}{4.225000in}}%
\pgfusepath{clip}%
\pgfsetbuttcap%
\pgfsetroundjoin%
\definecolor{currentfill}{rgb}{0.157729,0.485932,0.558013}%
\pgfsetfillcolor{currentfill}%
\pgfsetfillopacity{0.700000}%
\pgfsetlinewidth{0.501875pt}%
\definecolor{currentstroke}{rgb}{1.000000,1.000000,1.000000}%
\pgfsetstrokecolor{currentstroke}%
\pgfsetstrokeopacity{0.500000}%
\pgfsetdash{}{0pt}%
\pgfpathmoveto{\pgfqpoint{3.895131in}{2.108852in}}%
\pgfpathlineto{\pgfqpoint{3.906446in}{2.114548in}}%
\pgfpathlineto{\pgfqpoint{3.917757in}{2.120201in}}%
\pgfpathlineto{\pgfqpoint{3.929061in}{2.125764in}}%
\pgfpathlineto{\pgfqpoint{3.940357in}{2.131186in}}%
\pgfpathlineto{\pgfqpoint{3.951645in}{2.136427in}}%
\pgfpathlineto{\pgfqpoint{3.945194in}{2.151108in}}%
\pgfpathlineto{\pgfqpoint{3.938742in}{2.165639in}}%
\pgfpathlineto{\pgfqpoint{3.932291in}{2.180072in}}%
\pgfpathlineto{\pgfqpoint{3.925841in}{2.194457in}}%
\pgfpathlineto{\pgfqpoint{3.919394in}{2.208846in}}%
\pgfpathlineto{\pgfqpoint{3.908117in}{2.203921in}}%
\pgfpathlineto{\pgfqpoint{3.896831in}{2.198848in}}%
\pgfpathlineto{\pgfqpoint{3.885540in}{2.193695in}}%
\pgfpathlineto{\pgfqpoint{3.874243in}{2.188538in}}%
\pgfpathlineto{\pgfqpoint{3.862944in}{2.183454in}}%
\pgfpathlineto{\pgfqpoint{3.869378in}{2.168603in}}%
\pgfpathlineto{\pgfqpoint{3.875815in}{2.153762in}}%
\pgfpathlineto{\pgfqpoint{3.882253in}{2.138887in}}%
\pgfpathlineto{\pgfqpoint{3.888692in}{2.123932in}}%
\pgfpathclose%
\pgfusepath{stroke,fill}%
\end{pgfscope}%
\begin{pgfscope}%
\pgfpathrectangle{\pgfqpoint{0.887500in}{0.275000in}}{\pgfqpoint{4.225000in}{4.225000in}}%
\pgfusepath{clip}%
\pgfsetbuttcap%
\pgfsetroundjoin%
\definecolor{currentfill}{rgb}{0.218130,0.347432,0.550038}%
\pgfsetfillcolor{currentfill}%
\pgfsetfillopacity{0.700000}%
\pgfsetlinewidth{0.501875pt}%
\definecolor{currentstroke}{rgb}{1.000000,1.000000,1.000000}%
\pgfsetstrokecolor{currentstroke}%
\pgfsetstrokeopacity{0.500000}%
\pgfsetdash{}{0pt}%
\pgfpathmoveto{\pgfqpoint{4.282373in}{1.836791in}}%
\pgfpathlineto{\pgfqpoint{4.293554in}{1.840569in}}%
\pgfpathlineto{\pgfqpoint{4.304729in}{1.844350in}}%
\pgfpathlineto{\pgfqpoint{4.315898in}{1.848111in}}%
\pgfpathlineto{\pgfqpoint{4.327061in}{1.851830in}}%
\pgfpathlineto{\pgfqpoint{4.338216in}{1.855484in}}%
\pgfpathlineto{\pgfqpoint{4.331663in}{1.869229in}}%
\pgfpathlineto{\pgfqpoint{4.325111in}{1.882905in}}%
\pgfpathlineto{\pgfqpoint{4.318560in}{1.896534in}}%
\pgfpathlineto{\pgfqpoint{4.312012in}{1.910139in}}%
\pgfpathlineto{\pgfqpoint{4.305466in}{1.923743in}}%
\pgfpathlineto{\pgfqpoint{4.294296in}{1.919523in}}%
\pgfpathlineto{\pgfqpoint{4.283119in}{1.915242in}}%
\pgfpathlineto{\pgfqpoint{4.271936in}{1.910935in}}%
\pgfpathlineto{\pgfqpoint{4.260748in}{1.906635in}}%
\pgfpathlineto{\pgfqpoint{4.249557in}{1.902379in}}%
\pgfpathlineto{\pgfqpoint{4.256115in}{1.889300in}}%
\pgfpathlineto{\pgfqpoint{4.262677in}{1.876246in}}%
\pgfpathlineto{\pgfqpoint{4.269242in}{1.863172in}}%
\pgfpathlineto{\pgfqpoint{4.275807in}{1.850036in}}%
\pgfpathclose%
\pgfusepath{stroke,fill}%
\end{pgfscope}%
\begin{pgfscope}%
\pgfpathrectangle{\pgfqpoint{0.887500in}{0.275000in}}{\pgfqpoint{4.225000in}{4.225000in}}%
\pgfusepath{clip}%
\pgfsetbuttcap%
\pgfsetroundjoin%
\definecolor{currentfill}{rgb}{0.206756,0.371758,0.553117}%
\pgfsetfillcolor{currentfill}%
\pgfsetfillopacity{0.700000}%
\pgfsetlinewidth{0.501875pt}%
\definecolor{currentstroke}{rgb}{1.000000,1.000000,1.000000}%
\pgfsetstrokecolor{currentstroke}%
\pgfsetstrokeopacity{0.500000}%
\pgfsetdash{}{0pt}%
\pgfpathmoveto{\pgfqpoint{4.193561in}{1.882441in}}%
\pgfpathlineto{\pgfqpoint{4.204768in}{1.886342in}}%
\pgfpathlineto{\pgfqpoint{4.215969in}{1.890205in}}%
\pgfpathlineto{\pgfqpoint{4.227166in}{1.894136in}}%
\pgfpathlineto{\pgfqpoint{4.238362in}{1.898201in}}%
\pgfpathlineto{\pgfqpoint{4.249557in}{1.902379in}}%
\pgfpathlineto{\pgfqpoint{4.243004in}{1.915529in}}%
\pgfpathlineto{\pgfqpoint{4.236457in}{1.928793in}}%
\pgfpathlineto{\pgfqpoint{4.229918in}{1.942216in}}%
\pgfpathlineto{\pgfqpoint{4.223389in}{1.955842in}}%
\pgfpathlineto{\pgfqpoint{4.216869in}{1.969716in}}%
\pgfpathlineto{\pgfqpoint{4.205668in}{1.965162in}}%
\pgfpathlineto{\pgfqpoint{4.194463in}{1.960665in}}%
\pgfpathlineto{\pgfqpoint{4.183255in}{1.956240in}}%
\pgfpathlineto{\pgfqpoint{4.172044in}{1.951865in}}%
\pgfpathlineto{\pgfqpoint{4.160828in}{1.947496in}}%
\pgfpathlineto{\pgfqpoint{4.167351in}{1.933902in}}%
\pgfpathlineto{\pgfqpoint{4.173889in}{1.920686in}}%
\pgfpathlineto{\pgfqpoint{4.180438in}{1.907762in}}%
\pgfpathlineto{\pgfqpoint{4.186996in}{1.895043in}}%
\pgfpathclose%
\pgfusepath{stroke,fill}%
\end{pgfscope}%
\begin{pgfscope}%
\pgfpathrectangle{\pgfqpoint{0.887500in}{0.275000in}}{\pgfqpoint{4.225000in}{4.225000in}}%
\pgfusepath{clip}%
\pgfsetbuttcap%
\pgfsetroundjoin%
\definecolor{currentfill}{rgb}{0.127568,0.566949,0.550556}%
\pgfsetfillcolor{currentfill}%
\pgfsetfillopacity{0.700000}%
\pgfsetlinewidth{0.501875pt}%
\definecolor{currentstroke}{rgb}{1.000000,1.000000,1.000000}%
\pgfsetstrokecolor{currentstroke}%
\pgfsetstrokeopacity{0.500000}%
\pgfsetdash{}{0pt}%
\pgfpathmoveto{\pgfqpoint{3.539422in}{2.263489in}}%
\pgfpathlineto{\pgfqpoint{3.550868in}{2.274450in}}%
\pgfpathlineto{\pgfqpoint{3.562309in}{2.285130in}}%
\pgfpathlineto{\pgfqpoint{3.573744in}{2.295459in}}%
\pgfpathlineto{\pgfqpoint{3.585171in}{2.305369in}}%
\pgfpathlineto{\pgfqpoint{3.596590in}{2.314825in}}%
\pgfpathlineto{\pgfqpoint{3.590234in}{2.331788in}}%
\pgfpathlineto{\pgfqpoint{3.583880in}{2.348802in}}%
\pgfpathlineto{\pgfqpoint{3.577527in}{2.365791in}}%
\pgfpathlineto{\pgfqpoint{3.571174in}{2.382677in}}%
\pgfpathlineto{\pgfqpoint{3.564821in}{2.399384in}}%
\pgfpathlineto{\pgfqpoint{3.553423in}{2.391337in}}%
\pgfpathlineto{\pgfqpoint{3.542016in}{2.382754in}}%
\pgfpathlineto{\pgfqpoint{3.530599in}{2.373616in}}%
\pgfpathlineto{\pgfqpoint{3.519174in}{2.363921in}}%
\pgfpathlineto{\pgfqpoint{3.507740in}{2.353668in}}%
\pgfpathlineto{\pgfqpoint{3.514079in}{2.336056in}}%
\pgfpathlineto{\pgfqpoint{3.520416in}{2.318134in}}%
\pgfpathlineto{\pgfqpoint{3.526751in}{2.300000in}}%
\pgfpathlineto{\pgfqpoint{3.533086in}{2.281752in}}%
\pgfpathclose%
\pgfusepath{stroke,fill}%
\end{pgfscope}%
\begin{pgfscope}%
\pgfpathrectangle{\pgfqpoint{0.887500in}{0.275000in}}{\pgfqpoint{4.225000in}{4.225000in}}%
\pgfusepath{clip}%
\pgfsetbuttcap%
\pgfsetroundjoin%
\definecolor{currentfill}{rgb}{0.172719,0.448791,0.557885}%
\pgfsetfillcolor{currentfill}%
\pgfsetfillopacity{0.700000}%
\pgfsetlinewidth{0.501875pt}%
\definecolor{currentstroke}{rgb}{1.000000,1.000000,1.000000}%
\pgfsetstrokecolor{currentstroke}%
\pgfsetstrokeopacity{0.500000}%
\pgfsetdash{}{0pt}%
\pgfpathmoveto{\pgfqpoint{2.311602in}{2.049104in}}%
\pgfpathlineto{\pgfqpoint{2.323267in}{2.052808in}}%
\pgfpathlineto{\pgfqpoint{2.334927in}{2.056515in}}%
\pgfpathlineto{\pgfqpoint{2.346580in}{2.060222in}}%
\pgfpathlineto{\pgfqpoint{2.358228in}{2.063925in}}%
\pgfpathlineto{\pgfqpoint{2.369871in}{2.067623in}}%
\pgfpathlineto{\pgfqpoint{2.363798in}{2.077013in}}%
\pgfpathlineto{\pgfqpoint{2.357730in}{2.086368in}}%
\pgfpathlineto{\pgfqpoint{2.351666in}{2.095688in}}%
\pgfpathlineto{\pgfqpoint{2.345606in}{2.104973in}}%
\pgfpathlineto{\pgfqpoint{2.339551in}{2.114226in}}%
\pgfpathlineto{\pgfqpoint{2.327919in}{2.110581in}}%
\pgfpathlineto{\pgfqpoint{2.316281in}{2.106930in}}%
\pgfpathlineto{\pgfqpoint{2.304638in}{2.103277in}}%
\pgfpathlineto{\pgfqpoint{2.292989in}{2.099625in}}%
\pgfpathlineto{\pgfqpoint{2.281334in}{2.095976in}}%
\pgfpathlineto{\pgfqpoint{2.287379in}{2.086672in}}%
\pgfpathlineto{\pgfqpoint{2.293428in}{2.077333in}}%
\pgfpathlineto{\pgfqpoint{2.299482in}{2.067960in}}%
\pgfpathlineto{\pgfqpoint{2.305540in}{2.058550in}}%
\pgfpathclose%
\pgfusepath{stroke,fill}%
\end{pgfscope}%
\begin{pgfscope}%
\pgfpathrectangle{\pgfqpoint{0.887500in}{0.275000in}}{\pgfqpoint{4.225000in}{4.225000in}}%
\pgfusepath{clip}%
\pgfsetbuttcap%
\pgfsetroundjoin%
\definecolor{currentfill}{rgb}{0.229739,0.322361,0.545706}%
\pgfsetfillcolor{currentfill}%
\pgfsetfillopacity{0.700000}%
\pgfsetlinewidth{0.501875pt}%
\definecolor{currentstroke}{rgb}{1.000000,1.000000,1.000000}%
\pgfsetstrokecolor{currentstroke}%
\pgfsetstrokeopacity{0.500000}%
\pgfsetdash{}{0pt}%
\pgfpathmoveto{\pgfqpoint{3.226818in}{1.793552in}}%
\pgfpathlineto{\pgfqpoint{3.238251in}{1.795721in}}%
\pgfpathlineto{\pgfqpoint{3.249678in}{1.797893in}}%
\pgfpathlineto{\pgfqpoint{3.261101in}{1.800388in}}%
\pgfpathlineto{\pgfqpoint{3.272520in}{1.803525in}}%
\pgfpathlineto{\pgfqpoint{3.283939in}{1.807623in}}%
\pgfpathlineto{\pgfqpoint{3.277594in}{1.819920in}}%
\pgfpathlineto{\pgfqpoint{3.271256in}{1.832865in}}%
\pgfpathlineto{\pgfqpoint{3.264925in}{1.846547in}}%
\pgfpathlineto{\pgfqpoint{3.258600in}{1.860943in}}%
\pgfpathlineto{\pgfqpoint{3.252281in}{1.875952in}}%
\pgfpathlineto{\pgfqpoint{3.240853in}{1.869037in}}%
\pgfpathlineto{\pgfqpoint{3.229425in}{1.863073in}}%
\pgfpathlineto{\pgfqpoint{3.217997in}{1.857883in}}%
\pgfpathlineto{\pgfqpoint{3.206566in}{1.853285in}}%
\pgfpathlineto{\pgfqpoint{3.195131in}{1.849100in}}%
\pgfpathlineto{\pgfqpoint{3.201461in}{1.837972in}}%
\pgfpathlineto{\pgfqpoint{3.207795in}{1.826849in}}%
\pgfpathlineto{\pgfqpoint{3.214132in}{1.815739in}}%
\pgfpathlineto{\pgfqpoint{3.220473in}{1.804643in}}%
\pgfpathclose%
\pgfusepath{stroke,fill}%
\end{pgfscope}%
\begin{pgfscope}%
\pgfpathrectangle{\pgfqpoint{0.887500in}{0.275000in}}{\pgfqpoint{4.225000in}{4.225000in}}%
\pgfusepath{clip}%
\pgfsetbuttcap%
\pgfsetroundjoin%
\definecolor{currentfill}{rgb}{0.149039,0.508051,0.557250}%
\pgfsetfillcolor{currentfill}%
\pgfsetfillopacity{0.700000}%
\pgfsetlinewidth{0.501875pt}%
\definecolor{currentstroke}{rgb}{1.000000,1.000000,1.000000}%
\pgfsetstrokecolor{currentstroke}%
\pgfsetstrokeopacity{0.500000}%
\pgfsetdash{}{0pt}%
\pgfpathmoveto{\pgfqpoint{3.806436in}{2.161149in}}%
\pgfpathlineto{\pgfqpoint{3.817741in}{2.165273in}}%
\pgfpathlineto{\pgfqpoint{3.829041in}{2.169409in}}%
\pgfpathlineto{\pgfqpoint{3.840341in}{2.173813in}}%
\pgfpathlineto{\pgfqpoint{3.851643in}{2.178520in}}%
\pgfpathlineto{\pgfqpoint{3.862944in}{2.183454in}}%
\pgfpathlineto{\pgfqpoint{3.856513in}{2.198360in}}%
\pgfpathlineto{\pgfqpoint{3.850087in}{2.213366in}}%
\pgfpathlineto{\pgfqpoint{3.843665in}{2.228501in}}%
\pgfpathlineto{\pgfqpoint{3.837248in}{2.243742in}}%
\pgfpathlineto{\pgfqpoint{3.830835in}{2.259059in}}%
\pgfpathlineto{\pgfqpoint{3.819543in}{2.254474in}}%
\pgfpathlineto{\pgfqpoint{3.808249in}{2.250045in}}%
\pgfpathlineto{\pgfqpoint{3.796954in}{2.245832in}}%
\pgfpathlineto{\pgfqpoint{3.785658in}{2.241803in}}%
\pgfpathlineto{\pgfqpoint{3.774355in}{2.237746in}}%
\pgfpathlineto{\pgfqpoint{3.780771in}{2.222630in}}%
\pgfpathlineto{\pgfqpoint{3.787187in}{2.207423in}}%
\pgfpathlineto{\pgfqpoint{3.793603in}{2.192114in}}%
\pgfpathlineto{\pgfqpoint{3.800020in}{2.176694in}}%
\pgfpathclose%
\pgfusepath{stroke,fill}%
\end{pgfscope}%
\begin{pgfscope}%
\pgfpathrectangle{\pgfqpoint{0.887500in}{0.275000in}}{\pgfqpoint{4.225000in}{4.225000in}}%
\pgfusepath{clip}%
\pgfsetbuttcap%
\pgfsetroundjoin%
\definecolor{currentfill}{rgb}{0.195860,0.395433,0.555276}%
\pgfsetfillcolor{currentfill}%
\pgfsetfillopacity{0.700000}%
\pgfsetlinewidth{0.501875pt}%
\definecolor{currentstroke}{rgb}{1.000000,1.000000,1.000000}%
\pgfsetstrokecolor{currentstroke}%
\pgfsetstrokeopacity{0.500000}%
\pgfsetdash{}{0pt}%
\pgfpathmoveto{\pgfqpoint{4.104634in}{1.924291in}}%
\pgfpathlineto{\pgfqpoint{4.115891in}{1.929249in}}%
\pgfpathlineto{\pgfqpoint{4.127138in}{1.934010in}}%
\pgfpathlineto{\pgfqpoint{4.138376in}{1.938610in}}%
\pgfpathlineto{\pgfqpoint{4.149605in}{1.943092in}}%
\pgfpathlineto{\pgfqpoint{4.160828in}{1.947496in}}%
\pgfpathlineto{\pgfqpoint{4.154320in}{1.961555in}}%
\pgfpathlineto{\pgfqpoint{4.147829in}{1.976134in}}%
\pgfpathlineto{\pgfqpoint{4.141353in}{1.991173in}}%
\pgfpathlineto{\pgfqpoint{4.134891in}{2.006590in}}%
\pgfpathlineto{\pgfqpoint{4.128438in}{2.022302in}}%
\pgfpathlineto{\pgfqpoint{4.117223in}{2.018082in}}%
\pgfpathlineto{\pgfqpoint{4.106002in}{2.013819in}}%
\pgfpathlineto{\pgfqpoint{4.094773in}{2.009497in}}%
\pgfpathlineto{\pgfqpoint{4.083538in}{2.005097in}}%
\pgfpathlineto{\pgfqpoint{4.072295in}{2.000602in}}%
\pgfpathlineto{\pgfqpoint{4.078736in}{1.984475in}}%
\pgfpathlineto{\pgfqpoint{4.085187in}{1.968685in}}%
\pgfpathlineto{\pgfqpoint{4.091652in}{1.953330in}}%
\pgfpathlineto{\pgfqpoint{4.098134in}{1.938509in}}%
\pgfpathclose%
\pgfusepath{stroke,fill}%
\end{pgfscope}%
\begin{pgfscope}%
\pgfpathrectangle{\pgfqpoint{0.887500in}{0.275000in}}{\pgfqpoint{4.225000in}{4.225000in}}%
\pgfusepath{clip}%
\pgfsetbuttcap%
\pgfsetroundjoin%
\definecolor{currentfill}{rgb}{0.133743,0.548535,0.553541}%
\pgfsetfillcolor{currentfill}%
\pgfsetfillopacity{0.700000}%
\pgfsetlinewidth{0.501875pt}%
\definecolor{currentstroke}{rgb}{1.000000,1.000000,1.000000}%
\pgfsetstrokecolor{currentstroke}%
\pgfsetstrokeopacity{0.500000}%
\pgfsetdash{}{0pt}%
\pgfpathmoveto{\pgfqpoint{3.628442in}{2.233145in}}%
\pgfpathlineto{\pgfqpoint{3.639883in}{2.243869in}}%
\pgfpathlineto{\pgfqpoint{3.651315in}{2.254108in}}%
\pgfpathlineto{\pgfqpoint{3.662736in}{2.263781in}}%
\pgfpathlineto{\pgfqpoint{3.674145in}{2.272811in}}%
\pgfpathlineto{\pgfqpoint{3.685540in}{2.281117in}}%
\pgfpathlineto{\pgfqpoint{3.679132in}{2.295589in}}%
\pgfpathlineto{\pgfqpoint{3.672729in}{2.310222in}}%
\pgfpathlineto{\pgfqpoint{3.666332in}{2.325080in}}%
\pgfpathlineto{\pgfqpoint{3.659940in}{2.340162in}}%
\pgfpathlineto{\pgfqpoint{3.653553in}{2.355426in}}%
\pgfpathlineto{\pgfqpoint{3.642180in}{2.348184in}}%
\pgfpathlineto{\pgfqpoint{3.630796in}{2.340507in}}%
\pgfpathlineto{\pgfqpoint{3.619403in}{2.332391in}}%
\pgfpathlineto{\pgfqpoint{3.608001in}{2.323832in}}%
\pgfpathlineto{\pgfqpoint{3.596590in}{2.314825in}}%
\pgfpathlineto{\pgfqpoint{3.602950in}{2.297992in}}%
\pgfpathlineto{\pgfqpoint{3.609315in}{2.281363in}}%
\pgfpathlineto{\pgfqpoint{3.615685in}{2.265015in}}%
\pgfpathlineto{\pgfqpoint{3.622061in}{2.248965in}}%
\pgfpathclose%
\pgfusepath{stroke,fill}%
\end{pgfscope}%
\begin{pgfscope}%
\pgfpathrectangle{\pgfqpoint{0.887500in}{0.275000in}}{\pgfqpoint{4.225000in}{4.225000in}}%
\pgfusepath{clip}%
\pgfsetbuttcap%
\pgfsetroundjoin%
\definecolor{currentfill}{rgb}{0.194100,0.399323,0.555565}%
\pgfsetfillcolor{currentfill}%
\pgfsetfillopacity{0.700000}%
\pgfsetlinewidth{0.501875pt}%
\definecolor{currentstroke}{rgb}{1.000000,1.000000,1.000000}%
\pgfsetstrokecolor{currentstroke}%
\pgfsetstrokeopacity{0.500000}%
\pgfsetdash{}{0pt}%
\pgfpathmoveto{\pgfqpoint{2.724732in}{1.946446in}}%
\pgfpathlineto{\pgfqpoint{2.736297in}{1.950201in}}%
\pgfpathlineto{\pgfqpoint{2.747856in}{1.953971in}}%
\pgfpathlineto{\pgfqpoint{2.759410in}{1.957752in}}%
\pgfpathlineto{\pgfqpoint{2.770957in}{1.961522in}}%
\pgfpathlineto{\pgfqpoint{2.782500in}{1.965260in}}%
\pgfpathlineto{\pgfqpoint{2.776292in}{1.975253in}}%
\pgfpathlineto{\pgfqpoint{2.770088in}{1.985199in}}%
\pgfpathlineto{\pgfqpoint{2.763889in}{1.995099in}}%
\pgfpathlineto{\pgfqpoint{2.757693in}{2.004954in}}%
\pgfpathlineto{\pgfqpoint{2.751501in}{2.014764in}}%
\pgfpathlineto{\pgfqpoint{2.739969in}{2.011077in}}%
\pgfpathlineto{\pgfqpoint{2.728431in}{2.007358in}}%
\pgfpathlineto{\pgfqpoint{2.716887in}{2.003628in}}%
\pgfpathlineto{\pgfqpoint{2.705338in}{1.999908in}}%
\pgfpathlineto{\pgfqpoint{2.693783in}{1.996205in}}%
\pgfpathlineto{\pgfqpoint{2.699964in}{1.986346in}}%
\pgfpathlineto{\pgfqpoint{2.706150in}{1.976440in}}%
\pgfpathlineto{\pgfqpoint{2.712340in}{1.966488in}}%
\pgfpathlineto{\pgfqpoint{2.718534in}{1.956490in}}%
\pgfpathclose%
\pgfusepath{stroke,fill}%
\end{pgfscope}%
\begin{pgfscope}%
\pgfpathrectangle{\pgfqpoint{0.887500in}{0.275000in}}{\pgfqpoint{4.225000in}{4.225000in}}%
\pgfusepath{clip}%
\pgfsetbuttcap%
\pgfsetroundjoin%
\definecolor{currentfill}{rgb}{0.160665,0.478540,0.558115}%
\pgfsetfillcolor{currentfill}%
\pgfsetfillopacity{0.700000}%
\pgfsetlinewidth{0.501875pt}%
\definecolor{currentstroke}{rgb}{1.000000,1.000000,1.000000}%
\pgfsetstrokecolor{currentstroke}%
\pgfsetstrokeopacity{0.500000}%
\pgfsetdash{}{0pt}%
\pgfpathmoveto{\pgfqpoint{1.987254in}{2.115903in}}%
\pgfpathlineto{\pgfqpoint{1.999001in}{2.119552in}}%
\pgfpathlineto{\pgfqpoint{2.010742in}{2.123197in}}%
\pgfpathlineto{\pgfqpoint{2.022477in}{2.126836in}}%
\pgfpathlineto{\pgfqpoint{2.034207in}{2.130467in}}%
\pgfpathlineto{\pgfqpoint{2.045932in}{2.134088in}}%
\pgfpathlineto{\pgfqpoint{2.039967in}{2.143211in}}%
\pgfpathlineto{\pgfqpoint{2.034007in}{2.152306in}}%
\pgfpathlineto{\pgfqpoint{2.028051in}{2.161373in}}%
\pgfpathlineto{\pgfqpoint{2.022100in}{2.170412in}}%
\pgfpathlineto{\pgfqpoint{2.016153in}{2.179425in}}%
\pgfpathlineto{\pgfqpoint{2.004440in}{2.175858in}}%
\pgfpathlineto{\pgfqpoint{1.992721in}{2.172281in}}%
\pgfpathlineto{\pgfqpoint{1.980996in}{2.168698in}}%
\pgfpathlineto{\pgfqpoint{1.969265in}{2.165109in}}%
\pgfpathlineto{\pgfqpoint{1.957529in}{2.161516in}}%
\pgfpathlineto{\pgfqpoint{1.963465in}{2.152450in}}%
\pgfpathlineto{\pgfqpoint{1.969406in}{2.143357in}}%
\pgfpathlineto{\pgfqpoint{1.975351in}{2.134235in}}%
\pgfpathlineto{\pgfqpoint{1.981300in}{2.125084in}}%
\pgfpathclose%
\pgfusepath{stroke,fill}%
\end{pgfscope}%
\begin{pgfscope}%
\pgfpathrectangle{\pgfqpoint{0.887500in}{0.275000in}}{\pgfqpoint{4.225000in}{4.225000in}}%
\pgfusepath{clip}%
\pgfsetbuttcap%
\pgfsetroundjoin%
\definecolor{currentfill}{rgb}{0.221989,0.339161,0.548752}%
\pgfsetfillcolor{currentfill}%
\pgfsetfillopacity{0.700000}%
\pgfsetlinewidth{0.501875pt}%
\definecolor{currentstroke}{rgb}{1.000000,1.000000,1.000000}%
\pgfsetstrokecolor{currentstroke}%
\pgfsetstrokeopacity{0.500000}%
\pgfsetdash{}{0pt}%
\pgfpathmoveto{\pgfqpoint{3.372948in}{1.790186in}}%
\pgfpathlineto{\pgfqpoint{3.384411in}{1.801726in}}%
\pgfpathlineto{\pgfqpoint{3.395885in}{1.814628in}}%
\pgfpathlineto{\pgfqpoint{3.407371in}{1.828904in}}%
\pgfpathlineto{\pgfqpoint{3.418870in}{1.844498in}}%
\pgfpathlineto{\pgfqpoint{3.430380in}{1.861131in}}%
\pgfpathlineto{\pgfqpoint{3.424028in}{1.876988in}}%
\pgfpathlineto{\pgfqpoint{3.417676in}{1.892692in}}%
\pgfpathlineto{\pgfqpoint{3.411327in}{1.908352in}}%
\pgfpathlineto{\pgfqpoint{3.404980in}{1.924073in}}%
\pgfpathlineto{\pgfqpoint{3.398636in}{1.939963in}}%
\pgfpathlineto{\pgfqpoint{3.387099in}{1.918753in}}%
\pgfpathlineto{\pgfqpoint{3.375578in}{1.898932in}}%
\pgfpathlineto{\pgfqpoint{3.364076in}{1.881019in}}%
\pgfpathlineto{\pgfqpoint{3.352591in}{1.865087in}}%
\pgfpathlineto{\pgfqpoint{3.341122in}{1.851075in}}%
\pgfpathlineto{\pgfqpoint{3.347474in}{1.837944in}}%
\pgfpathlineto{\pgfqpoint{3.353833in}{1.825432in}}%
\pgfpathlineto{\pgfqpoint{3.360199in}{1.813398in}}%
\pgfpathlineto{\pgfqpoint{3.366572in}{1.801697in}}%
\pgfpathclose%
\pgfusepath{stroke,fill}%
\end{pgfscope}%
\begin{pgfscope}%
\pgfpathrectangle{\pgfqpoint{0.887500in}{0.275000in}}{\pgfqpoint{4.225000in}{4.225000in}}%
\pgfusepath{clip}%
\pgfsetbuttcap%
\pgfsetroundjoin%
\definecolor{currentfill}{rgb}{0.180629,0.429975,0.557282}%
\pgfsetfillcolor{currentfill}%
\pgfsetfillopacity{0.700000}%
\pgfsetlinewidth{0.501875pt}%
\definecolor{currentstroke}{rgb}{1.000000,1.000000,1.000000}%
\pgfsetstrokecolor{currentstroke}%
\pgfsetstrokeopacity{0.500000}%
\pgfsetdash{}{0pt}%
\pgfpathmoveto{\pgfqpoint{3.398636in}{1.939963in}}%
\pgfpathlineto{\pgfqpoint{3.410185in}{1.961950in}}%
\pgfpathlineto{\pgfqpoint{3.421740in}{1.984101in}}%
\pgfpathlineto{\pgfqpoint{3.433297in}{2.005801in}}%
\pgfpathlineto{\pgfqpoint{3.444847in}{2.026433in}}%
\pgfpathlineto{\pgfqpoint{3.456385in}{2.045381in}}%
\pgfpathlineto{\pgfqpoint{3.450052in}{2.063624in}}%
\pgfpathlineto{\pgfqpoint{3.443717in}{2.081602in}}%
\pgfpathlineto{\pgfqpoint{3.437382in}{2.099360in}}%
\pgfpathlineto{\pgfqpoint{3.431046in}{2.116899in}}%
\pgfpathlineto{\pgfqpoint{3.424709in}{2.134189in}}%
\pgfpathlineto{\pgfqpoint{3.413176in}{2.114623in}}%
\pgfpathlineto{\pgfqpoint{3.401632in}{2.093321in}}%
\pgfpathlineto{\pgfqpoint{3.390082in}{2.070891in}}%
\pgfpathlineto{\pgfqpoint{3.378534in}{2.047939in}}%
\pgfpathlineto{\pgfqpoint{3.366992in}{2.025072in}}%
\pgfpathlineto{\pgfqpoint{3.373311in}{2.007205in}}%
\pgfpathlineto{\pgfqpoint{3.379634in}{1.989708in}}%
\pgfpathlineto{\pgfqpoint{3.385963in}{1.972678in}}%
\pgfpathlineto{\pgfqpoint{3.392297in}{1.956129in}}%
\pgfpathclose%
\pgfusepath{stroke,fill}%
\end{pgfscope}%
\begin{pgfscope}%
\pgfpathrectangle{\pgfqpoint{0.887500in}{0.275000in}}{\pgfqpoint{4.225000in}{4.225000in}}%
\pgfusepath{clip}%
\pgfsetbuttcap%
\pgfsetroundjoin%
\definecolor{currentfill}{rgb}{0.136408,0.541173,0.554483}%
\pgfsetfillcolor{currentfill}%
\pgfsetfillopacity{0.700000}%
\pgfsetlinewidth{0.501875pt}%
\definecolor{currentstroke}{rgb}{1.000000,1.000000,1.000000}%
\pgfsetstrokecolor{currentstroke}%
\pgfsetstrokeopacity{0.500000}%
\pgfsetdash{}{0pt}%
\pgfpathmoveto{\pgfqpoint{3.482152in}{2.206600in}}%
\pgfpathlineto{\pgfqpoint{3.493611in}{2.218206in}}%
\pgfpathlineto{\pgfqpoint{3.505066in}{2.229615in}}%
\pgfpathlineto{\pgfqpoint{3.516521in}{2.241002in}}%
\pgfpathlineto{\pgfqpoint{3.527973in}{2.252316in}}%
\pgfpathlineto{\pgfqpoint{3.539422in}{2.263489in}}%
\pgfpathlineto{\pgfqpoint{3.533086in}{2.281752in}}%
\pgfpathlineto{\pgfqpoint{3.526751in}{2.300000in}}%
\pgfpathlineto{\pgfqpoint{3.520416in}{2.318134in}}%
\pgfpathlineto{\pgfqpoint{3.514079in}{2.336056in}}%
\pgfpathlineto{\pgfqpoint{3.507740in}{2.353668in}}%
\pgfpathlineto{\pgfqpoint{3.496298in}{2.342856in}}%
\pgfpathlineto{\pgfqpoint{3.484849in}{2.331484in}}%
\pgfpathlineto{\pgfqpoint{3.473392in}{2.319551in}}%
\pgfpathlineto{\pgfqpoint{3.461930in}{2.307052in}}%
\pgfpathlineto{\pgfqpoint{3.450460in}{2.293930in}}%
\pgfpathlineto{\pgfqpoint{3.456802in}{2.277135in}}%
\pgfpathlineto{\pgfqpoint{3.463142in}{2.259934in}}%
\pgfpathlineto{\pgfqpoint{3.469480in}{2.242397in}}%
\pgfpathlineto{\pgfqpoint{3.475816in}{2.224595in}}%
\pgfpathclose%
\pgfusepath{stroke,fill}%
\end{pgfscope}%
\begin{pgfscope}%
\pgfpathrectangle{\pgfqpoint{0.887500in}{0.275000in}}{\pgfqpoint{4.225000in}{4.225000in}}%
\pgfusepath{clip}%
\pgfsetbuttcap%
\pgfsetroundjoin%
\definecolor{currentfill}{rgb}{0.139147,0.533812,0.555298}%
\pgfsetfillcolor{currentfill}%
\pgfsetfillopacity{0.700000}%
\pgfsetlinewidth{0.501875pt}%
\definecolor{currentstroke}{rgb}{1.000000,1.000000,1.000000}%
\pgfsetstrokecolor{currentstroke}%
\pgfsetstrokeopacity{0.500000}%
\pgfsetdash{}{0pt}%
\pgfpathmoveto{\pgfqpoint{3.717621in}{2.208870in}}%
\pgfpathlineto{\pgfqpoint{3.729009in}{2.216567in}}%
\pgfpathlineto{\pgfqpoint{3.740373in}{2.223072in}}%
\pgfpathlineto{\pgfqpoint{3.751717in}{2.228615in}}%
\pgfpathlineto{\pgfqpoint{3.763043in}{2.233428in}}%
\pgfpathlineto{\pgfqpoint{3.774355in}{2.237746in}}%
\pgfpathlineto{\pgfqpoint{3.767941in}{2.252779in}}%
\pgfpathlineto{\pgfqpoint{3.761527in}{2.267738in}}%
\pgfpathlineto{\pgfqpoint{3.755115in}{2.282633in}}%
\pgfpathlineto{\pgfqpoint{3.748704in}{2.297472in}}%
\pgfpathlineto{\pgfqpoint{3.742294in}{2.312264in}}%
\pgfpathlineto{\pgfqpoint{3.730969in}{2.307083in}}%
\pgfpathlineto{\pgfqpoint{3.719633in}{2.301496in}}%
\pgfpathlineto{\pgfqpoint{3.708284in}{2.295383in}}%
\pgfpathlineto{\pgfqpoint{3.696920in}{2.288629in}}%
\pgfpathlineto{\pgfqpoint{3.685540in}{2.281117in}}%
\pgfpathlineto{\pgfqpoint{3.691953in}{2.266742in}}%
\pgfpathlineto{\pgfqpoint{3.698368in}{2.252395in}}%
\pgfpathlineto{\pgfqpoint{3.704785in}{2.238012in}}%
\pgfpathlineto{\pgfqpoint{3.711203in}{2.223526in}}%
\pgfpathclose%
\pgfusepath{stroke,fill}%
\end{pgfscope}%
\begin{pgfscope}%
\pgfpathrectangle{\pgfqpoint{0.887500in}{0.275000in}}{\pgfqpoint{4.225000in}{4.225000in}}%
\pgfusepath{clip}%
\pgfsetbuttcap%
\pgfsetroundjoin%
\definecolor{currentfill}{rgb}{0.147607,0.511733,0.557049}%
\pgfsetfillcolor{currentfill}%
\pgfsetfillopacity{0.700000}%
\pgfsetlinewidth{0.501875pt}%
\definecolor{currentstroke}{rgb}{1.000000,1.000000,1.000000}%
\pgfsetstrokecolor{currentstroke}%
\pgfsetstrokeopacity{0.500000}%
\pgfsetdash{}{0pt}%
\pgfpathmoveto{\pgfqpoint{3.424709in}{2.134189in}}%
\pgfpathlineto{\pgfqpoint{3.436225in}{2.151678in}}%
\pgfpathlineto{\pgfqpoint{3.447725in}{2.167338in}}%
\pgfpathlineto{\pgfqpoint{3.459212in}{2.181494in}}%
\pgfpathlineto{\pgfqpoint{3.470686in}{2.194472in}}%
\pgfpathlineto{\pgfqpoint{3.482152in}{2.206600in}}%
\pgfpathlineto{\pgfqpoint{3.475816in}{2.224595in}}%
\pgfpathlineto{\pgfqpoint{3.469480in}{2.242397in}}%
\pgfpathlineto{\pgfqpoint{3.463142in}{2.259934in}}%
\pgfpathlineto{\pgfqpoint{3.456802in}{2.277135in}}%
\pgfpathlineto{\pgfqpoint{3.450460in}{2.293930in}}%
\pgfpathlineto{\pgfqpoint{3.438984in}{2.280096in}}%
\pgfpathlineto{\pgfqpoint{3.427500in}{2.265458in}}%
\pgfpathlineto{\pgfqpoint{3.416009in}{2.249926in}}%
\pgfpathlineto{\pgfqpoint{3.404510in}{2.233411in}}%
\pgfpathlineto{\pgfqpoint{3.393005in}{2.215822in}}%
\pgfpathlineto{\pgfqpoint{3.399348in}{2.200240in}}%
\pgfpathlineto{\pgfqpoint{3.405690in}{2.184255in}}%
\pgfpathlineto{\pgfqpoint{3.412031in}{2.167898in}}%
\pgfpathlineto{\pgfqpoint{3.418370in}{2.151199in}}%
\pgfpathclose%
\pgfusepath{stroke,fill}%
\end{pgfscope}%
\begin{pgfscope}%
\pgfpathrectangle{\pgfqpoint{0.887500in}{0.275000in}}{\pgfqpoint{4.225000in}{4.225000in}}%
\pgfusepath{clip}%
\pgfsetbuttcap%
\pgfsetroundjoin%
\definecolor{currentfill}{rgb}{0.235526,0.309527,0.542944}%
\pgfsetfillcolor{currentfill}%
\pgfsetfillopacity{0.700000}%
\pgfsetlinewidth{0.501875pt}%
\definecolor{currentstroke}{rgb}{1.000000,1.000000,1.000000}%
\pgfsetstrokecolor{currentstroke}%
\pgfsetstrokeopacity{0.500000}%
\pgfsetdash{}{0pt}%
\pgfpathmoveto{\pgfqpoint{3.315752in}{1.752441in}}%
\pgfpathlineto{\pgfqpoint{3.327180in}{1.757401in}}%
\pgfpathlineto{\pgfqpoint{3.338612in}{1.763613in}}%
\pgfpathlineto{\pgfqpoint{3.350050in}{1.771141in}}%
\pgfpathlineto{\pgfqpoint{3.361495in}{1.779994in}}%
\pgfpathlineto{\pgfqpoint{3.372948in}{1.790186in}}%
\pgfpathlineto{\pgfqpoint{3.366572in}{1.801697in}}%
\pgfpathlineto{\pgfqpoint{3.360199in}{1.813398in}}%
\pgfpathlineto{\pgfqpoint{3.353833in}{1.825432in}}%
\pgfpathlineto{\pgfqpoint{3.347474in}{1.837944in}}%
\pgfpathlineto{\pgfqpoint{3.341122in}{1.851075in}}%
\pgfpathlineto{\pgfqpoint{3.329667in}{1.838921in}}%
\pgfpathlineto{\pgfqpoint{3.318223in}{1.828565in}}%
\pgfpathlineto{\pgfqpoint{3.306788in}{1.819945in}}%
\pgfpathlineto{\pgfqpoint{3.295361in}{1.813002in}}%
\pgfpathlineto{\pgfqpoint{3.283939in}{1.807623in}}%
\pgfpathlineto{\pgfqpoint{3.290290in}{1.795876in}}%
\pgfpathlineto{\pgfqpoint{3.296648in}{1.784582in}}%
\pgfpathlineto{\pgfqpoint{3.303011in}{1.773643in}}%
\pgfpathlineto{\pgfqpoint{3.309379in}{1.762962in}}%
\pgfpathclose%
\pgfusepath{stroke,fill}%
\end{pgfscope}%
\begin{pgfscope}%
\pgfpathrectangle{\pgfqpoint{0.887500in}{0.275000in}}{\pgfqpoint{4.225000in}{4.225000in}}%
\pgfusepath{clip}%
\pgfsetbuttcap%
\pgfsetroundjoin%
\definecolor{currentfill}{rgb}{0.177423,0.437527,0.557565}%
\pgfsetfillcolor{currentfill}%
\pgfsetfillopacity{0.700000}%
\pgfsetlinewidth{0.501875pt}%
\definecolor{currentstroke}{rgb}{1.000000,1.000000,1.000000}%
\pgfsetstrokecolor{currentstroke}%
\pgfsetstrokeopacity{0.500000}%
\pgfsetdash{}{0pt}%
\pgfpathmoveto{\pgfqpoint{2.400300in}{2.020110in}}%
\pgfpathlineto{\pgfqpoint{2.411947in}{2.023854in}}%
\pgfpathlineto{\pgfqpoint{2.423589in}{2.027584in}}%
\pgfpathlineto{\pgfqpoint{2.435225in}{2.031299in}}%
\pgfpathlineto{\pgfqpoint{2.446856in}{2.034996in}}%
\pgfpathlineto{\pgfqpoint{2.458481in}{2.038675in}}%
\pgfpathlineto{\pgfqpoint{2.452376in}{2.048204in}}%
\pgfpathlineto{\pgfqpoint{2.446276in}{2.057692in}}%
\pgfpathlineto{\pgfqpoint{2.440179in}{2.067142in}}%
\pgfpathlineto{\pgfqpoint{2.434088in}{2.076554in}}%
\pgfpathlineto{\pgfqpoint{2.428000in}{2.085928in}}%
\pgfpathlineto{\pgfqpoint{2.416385in}{2.082299in}}%
\pgfpathlineto{\pgfqpoint{2.404765in}{2.078651in}}%
\pgfpathlineto{\pgfqpoint{2.393139in}{2.074989in}}%
\pgfpathlineto{\pgfqpoint{2.381508in}{2.071312in}}%
\pgfpathlineto{\pgfqpoint{2.369871in}{2.067623in}}%
\pgfpathlineto{\pgfqpoint{2.375948in}{2.058197in}}%
\pgfpathlineto{\pgfqpoint{2.382029in}{2.048733in}}%
\pgfpathlineto{\pgfqpoint{2.388115in}{2.039232in}}%
\pgfpathlineto{\pgfqpoint{2.394205in}{2.029691in}}%
\pgfpathclose%
\pgfusepath{stroke,fill}%
\end{pgfscope}%
\begin{pgfscope}%
\pgfpathrectangle{\pgfqpoint{0.887500in}{0.275000in}}{\pgfqpoint{4.225000in}{4.225000in}}%
\pgfusepath{clip}%
\pgfsetbuttcap%
\pgfsetroundjoin%
\definecolor{currentfill}{rgb}{0.199430,0.387607,0.554642}%
\pgfsetfillcolor{currentfill}%
\pgfsetfillopacity{0.700000}%
\pgfsetlinewidth{0.501875pt}%
\definecolor{currentstroke}{rgb}{1.000000,1.000000,1.000000}%
\pgfsetstrokecolor{currentstroke}%
\pgfsetstrokeopacity{0.500000}%
\pgfsetdash{}{0pt}%
\pgfpathmoveto{\pgfqpoint{2.813599in}{1.914593in}}%
\pgfpathlineto{\pgfqpoint{2.825146in}{1.918334in}}%
\pgfpathlineto{\pgfqpoint{2.836687in}{1.922006in}}%
\pgfpathlineto{\pgfqpoint{2.848223in}{1.925589in}}%
\pgfpathlineto{\pgfqpoint{2.859754in}{1.929064in}}%
\pgfpathlineto{\pgfqpoint{2.871279in}{1.932445in}}%
\pgfpathlineto{\pgfqpoint{2.865042in}{1.942597in}}%
\pgfpathlineto{\pgfqpoint{2.858809in}{1.952705in}}%
\pgfpathlineto{\pgfqpoint{2.852580in}{1.962767in}}%
\pgfpathlineto{\pgfqpoint{2.846355in}{1.972784in}}%
\pgfpathlineto{\pgfqpoint{2.840134in}{1.982757in}}%
\pgfpathlineto{\pgfqpoint{2.828618in}{1.979463in}}%
\pgfpathlineto{\pgfqpoint{2.817096in}{1.976068in}}%
\pgfpathlineto{\pgfqpoint{2.805569in}{1.972555in}}%
\pgfpathlineto{\pgfqpoint{2.794037in}{1.968945in}}%
\pgfpathlineto{\pgfqpoint{2.782500in}{1.965260in}}%
\pgfpathlineto{\pgfqpoint{2.788712in}{1.955221in}}%
\pgfpathlineto{\pgfqpoint{2.794927in}{1.945136in}}%
\pgfpathlineto{\pgfqpoint{2.801147in}{1.935003in}}%
\pgfpathlineto{\pgfqpoint{2.807371in}{1.924822in}}%
\pgfpathclose%
\pgfusepath{stroke,fill}%
\end{pgfscope}%
\begin{pgfscope}%
\pgfpathrectangle{\pgfqpoint{0.887500in}{0.275000in}}{\pgfqpoint{4.225000in}{4.225000in}}%
\pgfusepath{clip}%
\pgfsetbuttcap%
\pgfsetroundjoin%
\definecolor{currentfill}{rgb}{0.182256,0.426184,0.557120}%
\pgfsetfillcolor{currentfill}%
\pgfsetfillopacity{0.700000}%
\pgfsetlinewidth{0.501875pt}%
\definecolor{currentstroke}{rgb}{1.000000,1.000000,1.000000}%
\pgfsetstrokecolor{currentstroke}%
\pgfsetstrokeopacity{0.500000}%
\pgfsetdash{}{0pt}%
\pgfpathmoveto{\pgfqpoint{4.015954in}{1.976013in}}%
\pgfpathlineto{\pgfqpoint{4.027240in}{1.981279in}}%
\pgfpathlineto{\pgfqpoint{4.038517in}{1.986351in}}%
\pgfpathlineto{\pgfqpoint{4.049785in}{1.991249in}}%
\pgfpathlineto{\pgfqpoint{4.061044in}{1.995993in}}%
\pgfpathlineto{\pgfqpoint{4.072295in}{2.000602in}}%
\pgfpathlineto{\pgfqpoint{4.065862in}{2.016966in}}%
\pgfpathlineto{\pgfqpoint{4.059435in}{2.033469in}}%
\pgfpathlineto{\pgfqpoint{4.053011in}{2.050010in}}%
\pgfpathlineto{\pgfqpoint{4.046587in}{2.066492in}}%
\pgfpathlineto{\pgfqpoint{4.040161in}{2.082813in}}%
\pgfpathlineto{\pgfqpoint{4.028916in}{2.078444in}}%
\pgfpathlineto{\pgfqpoint{4.017662in}{2.073892in}}%
\pgfpathlineto{\pgfqpoint{4.006398in}{2.069150in}}%
\pgfpathlineto{\pgfqpoint{3.995125in}{2.064211in}}%
\pgfpathlineto{\pgfqpoint{3.983842in}{2.059067in}}%
\pgfpathlineto{\pgfqpoint{3.990266in}{2.042670in}}%
\pgfpathlineto{\pgfqpoint{3.996687in}{2.026086in}}%
\pgfpathlineto{\pgfqpoint{4.003108in}{2.009395in}}%
\pgfpathlineto{\pgfqpoint{4.009529in}{1.992677in}}%
\pgfpathclose%
\pgfusepath{stroke,fill}%
\end{pgfscope}%
\begin{pgfscope}%
\pgfpathrectangle{\pgfqpoint{0.887500in}{0.275000in}}{\pgfqpoint{4.225000in}{4.225000in}}%
\pgfusepath{clip}%
\pgfsetbuttcap%
\pgfsetroundjoin%
\definecolor{currentfill}{rgb}{0.163625,0.471133,0.558148}%
\pgfsetfillcolor{currentfill}%
\pgfsetfillopacity{0.700000}%
\pgfsetlinewidth{0.501875pt}%
\definecolor{currentstroke}{rgb}{1.000000,1.000000,1.000000}%
\pgfsetstrokecolor{currentstroke}%
\pgfsetstrokeopacity{0.500000}%
\pgfsetdash{}{0pt}%
\pgfpathmoveto{\pgfqpoint{2.075821in}{2.088021in}}%
\pgfpathlineto{\pgfqpoint{2.087550in}{2.091686in}}%
\pgfpathlineto{\pgfqpoint{2.099274in}{2.095336in}}%
\pgfpathlineto{\pgfqpoint{2.110993in}{2.098972in}}%
\pgfpathlineto{\pgfqpoint{2.122706in}{2.102595in}}%
\pgfpathlineto{\pgfqpoint{2.134413in}{2.106206in}}%
\pgfpathlineto{\pgfqpoint{2.128416in}{2.115427in}}%
\pgfpathlineto{\pgfqpoint{2.122423in}{2.124618in}}%
\pgfpathlineto{\pgfqpoint{2.116434in}{2.133778in}}%
\pgfpathlineto{\pgfqpoint{2.110450in}{2.142909in}}%
\pgfpathlineto{\pgfqpoint{2.104471in}{2.152011in}}%
\pgfpathlineto{\pgfqpoint{2.092774in}{2.148450in}}%
\pgfpathlineto{\pgfqpoint{2.081071in}{2.144877in}}%
\pgfpathlineto{\pgfqpoint{2.069364in}{2.141293in}}%
\pgfpathlineto{\pgfqpoint{2.057650in}{2.137697in}}%
\pgfpathlineto{\pgfqpoint{2.045932in}{2.134088in}}%
\pgfpathlineto{\pgfqpoint{2.051900in}{2.124935in}}%
\pgfpathlineto{\pgfqpoint{2.057874in}{2.115753in}}%
\pgfpathlineto{\pgfqpoint{2.063852in}{2.106540in}}%
\pgfpathlineto{\pgfqpoint{2.069834in}{2.097297in}}%
\pgfpathclose%
\pgfusepath{stroke,fill}%
\end{pgfscope}%
\begin{pgfscope}%
\pgfpathrectangle{\pgfqpoint{0.887500in}{0.275000in}}{\pgfqpoint{4.225000in}{4.225000in}}%
\pgfusepath{clip}%
\pgfsetbuttcap%
\pgfsetroundjoin%
\definecolor{currentfill}{rgb}{0.140536,0.530132,0.555659}%
\pgfsetfillcolor{currentfill}%
\pgfsetfillopacity{0.700000}%
\pgfsetlinewidth{0.501875pt}%
\definecolor{currentstroke}{rgb}{1.000000,1.000000,1.000000}%
\pgfsetstrokecolor{currentstroke}%
\pgfsetstrokeopacity{0.500000}%
\pgfsetdash{}{0pt}%
\pgfpathmoveto{\pgfqpoint{3.571158in}{2.174983in}}%
\pgfpathlineto{\pgfqpoint{3.582622in}{2.186965in}}%
\pgfpathlineto{\pgfqpoint{3.594083in}{2.198847in}}%
\pgfpathlineto{\pgfqpoint{3.605541in}{2.210556in}}%
\pgfpathlineto{\pgfqpoint{3.616994in}{2.222014in}}%
\pgfpathlineto{\pgfqpoint{3.628442in}{2.233145in}}%
\pgfpathlineto{\pgfqpoint{3.622061in}{2.248965in}}%
\pgfpathlineto{\pgfqpoint{3.615685in}{2.265015in}}%
\pgfpathlineto{\pgfqpoint{3.609315in}{2.281363in}}%
\pgfpathlineto{\pgfqpoint{3.602950in}{2.297992in}}%
\pgfpathlineto{\pgfqpoint{3.596590in}{2.314825in}}%
\pgfpathlineto{\pgfqpoint{3.585171in}{2.305369in}}%
\pgfpathlineto{\pgfqpoint{3.573744in}{2.295459in}}%
\pgfpathlineto{\pgfqpoint{3.562309in}{2.285130in}}%
\pgfpathlineto{\pgfqpoint{3.550868in}{2.274450in}}%
\pgfpathlineto{\pgfqpoint{3.539422in}{2.263489in}}%
\pgfpathlineto{\pgfqpoint{3.545761in}{2.245307in}}%
\pgfpathlineto{\pgfqpoint{3.552103in}{2.227305in}}%
\pgfpathlineto{\pgfqpoint{3.558450in}{2.209578in}}%
\pgfpathlineto{\pgfqpoint{3.564802in}{2.192158in}}%
\pgfpathclose%
\pgfusepath{stroke,fill}%
\end{pgfscope}%
\begin{pgfscope}%
\pgfpathrectangle{\pgfqpoint{0.887500in}{0.275000in}}{\pgfqpoint{4.225000in}{4.225000in}}%
\pgfusepath{clip}%
\pgfsetbuttcap%
\pgfsetroundjoin%
\definecolor{currentfill}{rgb}{0.151918,0.500685,0.557587}%
\pgfsetfillcolor{currentfill}%
\pgfsetfillopacity{0.700000}%
\pgfsetlinewidth{0.501875pt}%
\definecolor{currentstroke}{rgb}{1.000000,1.000000,1.000000}%
\pgfsetstrokecolor{currentstroke}%
\pgfsetstrokeopacity{0.500000}%
\pgfsetdash{}{0pt}%
\pgfpathmoveto{\pgfqpoint{1.751350in}{2.152830in}}%
\pgfpathlineto{\pgfqpoint{1.763160in}{2.156465in}}%
\pgfpathlineto{\pgfqpoint{1.774964in}{2.160086in}}%
\pgfpathlineto{\pgfqpoint{1.786764in}{2.163694in}}%
\pgfpathlineto{\pgfqpoint{1.798557in}{2.167291in}}%
\pgfpathlineto{\pgfqpoint{1.810346in}{2.170877in}}%
\pgfpathlineto{\pgfqpoint{1.804458in}{2.179873in}}%
\pgfpathlineto{\pgfqpoint{1.798576in}{2.188841in}}%
\pgfpathlineto{\pgfqpoint{1.792698in}{2.197782in}}%
\pgfpathlineto{\pgfqpoint{1.786824in}{2.206695in}}%
\pgfpathlineto{\pgfqpoint{1.780955in}{2.215581in}}%
\pgfpathlineto{\pgfqpoint{1.769178in}{2.212043in}}%
\pgfpathlineto{\pgfqpoint{1.757395in}{2.208496in}}%
\pgfpathlineto{\pgfqpoint{1.745606in}{2.204937in}}%
\pgfpathlineto{\pgfqpoint{1.733813in}{2.201366in}}%
\pgfpathlineto{\pgfqpoint{1.722014in}{2.197781in}}%
\pgfpathlineto{\pgfqpoint{1.727872in}{2.188845in}}%
\pgfpathlineto{\pgfqpoint{1.733734in}{2.179883in}}%
\pgfpathlineto{\pgfqpoint{1.739601in}{2.170893in}}%
\pgfpathlineto{\pgfqpoint{1.745473in}{2.161876in}}%
\pgfpathclose%
\pgfusepath{stroke,fill}%
\end{pgfscope}%
\begin{pgfscope}%
\pgfpathrectangle{\pgfqpoint{0.887500in}{0.275000in}}{\pgfqpoint{4.225000in}{4.225000in}}%
\pgfusepath{clip}%
\pgfsetbuttcap%
\pgfsetroundjoin%
\definecolor{currentfill}{rgb}{0.244972,0.287675,0.537260}%
\pgfsetfillcolor{currentfill}%
\pgfsetfillopacity{0.700000}%
\pgfsetlinewidth{0.501875pt}%
\definecolor{currentstroke}{rgb}{1.000000,1.000000,1.000000}%
\pgfsetstrokecolor{currentstroke}%
\pgfsetstrokeopacity{0.500000}%
\pgfsetdash{}{0pt}%
\pgfpathmoveto{\pgfqpoint{4.403597in}{1.709198in}}%
\pgfpathlineto{\pgfqpoint{4.414744in}{1.712678in}}%
\pgfpathlineto{\pgfqpoint{4.425884in}{1.716150in}}%
\pgfpathlineto{\pgfqpoint{4.437019in}{1.719606in}}%
\pgfpathlineto{\pgfqpoint{4.448147in}{1.723043in}}%
\pgfpathlineto{\pgfqpoint{4.459269in}{1.726455in}}%
\pgfpathlineto{\pgfqpoint{4.452731in}{1.741453in}}%
\pgfpathlineto{\pgfqpoint{4.446193in}{1.756357in}}%
\pgfpathlineto{\pgfqpoint{4.439653in}{1.771173in}}%
\pgfpathlineto{\pgfqpoint{4.433114in}{1.785905in}}%
\pgfpathlineto{\pgfqpoint{4.426574in}{1.800556in}}%
\pgfpathlineto{\pgfqpoint{4.415465in}{1.797491in}}%
\pgfpathlineto{\pgfqpoint{4.404349in}{1.794401in}}%
\pgfpathlineto{\pgfqpoint{4.393227in}{1.791280in}}%
\pgfpathlineto{\pgfqpoint{4.382098in}{1.788120in}}%
\pgfpathlineto{\pgfqpoint{4.370961in}{1.784915in}}%
\pgfpathlineto{\pgfqpoint{4.377500in}{1.770274in}}%
\pgfpathlineto{\pgfqpoint{4.384035in}{1.755405in}}%
\pgfpathlineto{\pgfqpoint{4.390563in}{1.740284in}}%
\pgfpathlineto{\pgfqpoint{4.397084in}{1.724889in}}%
\pgfpathclose%
\pgfusepath{stroke,fill}%
\end{pgfscope}%
\begin{pgfscope}%
\pgfpathrectangle{\pgfqpoint{0.887500in}{0.275000in}}{\pgfqpoint{4.225000in}{4.225000in}}%
\pgfusepath{clip}%
\pgfsetbuttcap%
\pgfsetroundjoin%
\definecolor{currentfill}{rgb}{0.206756,0.371758,0.553117}%
\pgfsetfillcolor{currentfill}%
\pgfsetfillopacity{0.700000}%
\pgfsetlinewidth{0.501875pt}%
\definecolor{currentstroke}{rgb}{1.000000,1.000000,1.000000}%
\pgfsetstrokecolor{currentstroke}%
\pgfsetstrokeopacity{0.500000}%
\pgfsetdash{}{0pt}%
\pgfpathmoveto{\pgfqpoint{2.902522in}{1.880983in}}%
\pgfpathlineto{\pgfqpoint{2.914051in}{1.884441in}}%
\pgfpathlineto{\pgfqpoint{2.925574in}{1.887913in}}%
\pgfpathlineto{\pgfqpoint{2.937091in}{1.891446in}}%
\pgfpathlineto{\pgfqpoint{2.948601in}{1.895089in}}%
\pgfpathlineto{\pgfqpoint{2.960106in}{1.898890in}}%
\pgfpathlineto{\pgfqpoint{2.953841in}{1.909224in}}%
\pgfpathlineto{\pgfqpoint{2.947579in}{1.919511in}}%
\pgfpathlineto{\pgfqpoint{2.941321in}{1.929751in}}%
\pgfpathlineto{\pgfqpoint{2.935068in}{1.939945in}}%
\pgfpathlineto{\pgfqpoint{2.928817in}{1.950091in}}%
\pgfpathlineto{\pgfqpoint{2.917322in}{1.946254in}}%
\pgfpathlineto{\pgfqpoint{2.905821in}{1.942634in}}%
\pgfpathlineto{\pgfqpoint{2.894313in}{1.939167in}}%
\pgfpathlineto{\pgfqpoint{2.882799in}{1.935792in}}%
\pgfpathlineto{\pgfqpoint{2.871279in}{1.932445in}}%
\pgfpathlineto{\pgfqpoint{2.877520in}{1.922246in}}%
\pgfpathlineto{\pgfqpoint{2.883765in}{1.912001in}}%
\pgfpathlineto{\pgfqpoint{2.890014in}{1.901709in}}%
\pgfpathlineto{\pgfqpoint{2.896266in}{1.891370in}}%
\pgfpathclose%
\pgfusepath{stroke,fill}%
\end{pgfscope}%
\begin{pgfscope}%
\pgfpathrectangle{\pgfqpoint{0.887500in}{0.275000in}}{\pgfqpoint{4.225000in}{4.225000in}}%
\pgfusepath{clip}%
\pgfsetbuttcap%
\pgfsetroundjoin%
\definecolor{currentfill}{rgb}{0.182256,0.426184,0.557120}%
\pgfsetfillcolor{currentfill}%
\pgfsetfillopacity{0.700000}%
\pgfsetlinewidth{0.501875pt}%
\definecolor{currentstroke}{rgb}{1.000000,1.000000,1.000000}%
\pgfsetstrokecolor{currentstroke}%
\pgfsetstrokeopacity{0.500000}%
\pgfsetdash{}{0pt}%
\pgfpathmoveto{\pgfqpoint{2.489071in}{1.990393in}}%
\pgfpathlineto{\pgfqpoint{2.500701in}{1.994103in}}%
\pgfpathlineto{\pgfqpoint{2.512326in}{1.997792in}}%
\pgfpathlineto{\pgfqpoint{2.523945in}{2.001460in}}%
\pgfpathlineto{\pgfqpoint{2.535558in}{2.005106in}}%
\pgfpathlineto{\pgfqpoint{2.547167in}{2.008734in}}%
\pgfpathlineto{\pgfqpoint{2.541030in}{2.018426in}}%
\pgfpathlineto{\pgfqpoint{2.534898in}{2.028073in}}%
\pgfpathlineto{\pgfqpoint{2.528770in}{2.037677in}}%
\pgfpathlineto{\pgfqpoint{2.522646in}{2.047238in}}%
\pgfpathlineto{\pgfqpoint{2.516526in}{2.056757in}}%
\pgfpathlineto{\pgfqpoint{2.504928in}{2.053183in}}%
\pgfpathlineto{\pgfqpoint{2.493325in}{2.049590in}}%
\pgfpathlineto{\pgfqpoint{2.481715in}{2.045973in}}%
\pgfpathlineto{\pgfqpoint{2.470101in}{2.042335in}}%
\pgfpathlineto{\pgfqpoint{2.458481in}{2.038675in}}%
\pgfpathlineto{\pgfqpoint{2.464590in}{2.029106in}}%
\pgfpathlineto{\pgfqpoint{2.470704in}{2.019494in}}%
\pgfpathlineto{\pgfqpoint{2.476822in}{2.009838in}}%
\pgfpathlineto{\pgfqpoint{2.482944in}{2.000139in}}%
\pgfpathclose%
\pgfusepath{stroke,fill}%
\end{pgfscope}%
\begin{pgfscope}%
\pgfpathrectangle{\pgfqpoint{0.887500in}{0.275000in}}{\pgfqpoint{4.225000in}{4.225000in}}%
\pgfusepath{clip}%
\pgfsetbuttcap%
\pgfsetroundjoin%
\definecolor{currentfill}{rgb}{0.171176,0.452530,0.557965}%
\pgfsetfillcolor{currentfill}%
\pgfsetfillopacity{0.700000}%
\pgfsetlinewidth{0.501875pt}%
\definecolor{currentstroke}{rgb}{1.000000,1.000000,1.000000}%
\pgfsetstrokecolor{currentstroke}%
\pgfsetstrokeopacity{0.500000}%
\pgfsetdash{}{0pt}%
\pgfpathmoveto{\pgfqpoint{3.927284in}{2.030060in}}%
\pgfpathlineto{\pgfqpoint{3.938615in}{2.036316in}}%
\pgfpathlineto{\pgfqpoint{3.949935in}{2.042340in}}%
\pgfpathlineto{\pgfqpoint{3.961247in}{2.048137in}}%
\pgfpathlineto{\pgfqpoint{3.972549in}{2.053711in}}%
\pgfpathlineto{\pgfqpoint{3.983842in}{2.059067in}}%
\pgfpathlineto{\pgfqpoint{3.977413in}{2.075195in}}%
\pgfpathlineto{\pgfqpoint{3.970979in}{2.090980in}}%
\pgfpathlineto{\pgfqpoint{3.964538in}{2.106414in}}%
\pgfpathlineto{\pgfqpoint{3.958093in}{2.121546in}}%
\pgfpathlineto{\pgfqpoint{3.951645in}{2.136427in}}%
\pgfpathlineto{\pgfqpoint{3.940357in}{2.131186in}}%
\pgfpathlineto{\pgfqpoint{3.929061in}{2.125764in}}%
\pgfpathlineto{\pgfqpoint{3.917757in}{2.120201in}}%
\pgfpathlineto{\pgfqpoint{3.906446in}{2.114548in}}%
\pgfpathlineto{\pgfqpoint{3.895131in}{2.108852in}}%
\pgfpathlineto{\pgfqpoint{3.901568in}{2.093602in}}%
\pgfpathlineto{\pgfqpoint{3.908003in}{2.078138in}}%
\pgfpathlineto{\pgfqpoint{3.914435in}{2.062415in}}%
\pgfpathlineto{\pgfqpoint{3.920862in}{2.046390in}}%
\pgfpathclose%
\pgfusepath{stroke,fill}%
\end{pgfscope}%
\begin{pgfscope}%
\pgfpathrectangle{\pgfqpoint{0.887500in}{0.275000in}}{\pgfqpoint{4.225000in}{4.225000in}}%
\pgfusepath{clip}%
\pgfsetbuttcap%
\pgfsetroundjoin%
\definecolor{currentfill}{rgb}{0.150476,0.504369,0.557430}%
\pgfsetfillcolor{currentfill}%
\pgfsetfillopacity{0.700000}%
\pgfsetlinewidth{0.501875pt}%
\definecolor{currentstroke}{rgb}{1.000000,1.000000,1.000000}%
\pgfsetstrokecolor{currentstroke}%
\pgfsetstrokeopacity{0.500000}%
\pgfsetdash{}{0pt}%
\pgfpathmoveto{\pgfqpoint{3.513840in}{2.115924in}}%
\pgfpathlineto{\pgfqpoint{3.525304in}{2.127567in}}%
\pgfpathlineto{\pgfqpoint{3.536766in}{2.139206in}}%
\pgfpathlineto{\pgfqpoint{3.548230in}{2.151028in}}%
\pgfpathlineto{\pgfqpoint{3.559694in}{2.162979in}}%
\pgfpathlineto{\pgfqpoint{3.571158in}{2.174983in}}%
\pgfpathlineto{\pgfqpoint{3.564802in}{2.192158in}}%
\pgfpathlineto{\pgfqpoint{3.558450in}{2.209578in}}%
\pgfpathlineto{\pgfqpoint{3.552103in}{2.227305in}}%
\pgfpathlineto{\pgfqpoint{3.545761in}{2.245307in}}%
\pgfpathlineto{\pgfqpoint{3.539422in}{2.263489in}}%
\pgfpathlineto{\pgfqpoint{3.527973in}{2.252316in}}%
\pgfpathlineto{\pgfqpoint{3.516521in}{2.241002in}}%
\pgfpathlineto{\pgfqpoint{3.505066in}{2.229615in}}%
\pgfpathlineto{\pgfqpoint{3.493611in}{2.218206in}}%
\pgfpathlineto{\pgfqpoint{3.482152in}{2.206600in}}%
\pgfpathlineto{\pgfqpoint{3.488487in}{2.188483in}}%
\pgfpathlineto{\pgfqpoint{3.494823in}{2.170313in}}%
\pgfpathlineto{\pgfqpoint{3.501161in}{2.152162in}}%
\pgfpathlineto{\pgfqpoint{3.507500in}{2.134049in}}%
\pgfpathclose%
\pgfusepath{stroke,fill}%
\end{pgfscope}%
\begin{pgfscope}%
\pgfpathrectangle{\pgfqpoint{0.887500in}{0.275000in}}{\pgfqpoint{4.225000in}{4.225000in}}%
\pgfusepath{clip}%
\pgfsetbuttcap%
\pgfsetroundjoin%
\definecolor{currentfill}{rgb}{0.197636,0.391528,0.554969}%
\pgfsetfillcolor{currentfill}%
\pgfsetfillopacity{0.700000}%
\pgfsetlinewidth{0.501875pt}%
\definecolor{currentstroke}{rgb}{1.000000,1.000000,1.000000}%
\pgfsetstrokecolor{currentstroke}%
\pgfsetstrokeopacity{0.500000}%
\pgfsetdash{}{0pt}%
\pgfpathmoveto{\pgfqpoint{3.430380in}{1.861131in}}%
\pgfpathlineto{\pgfqpoint{3.441899in}{1.878479in}}%
\pgfpathlineto{\pgfqpoint{3.453424in}{1.896215in}}%
\pgfpathlineto{\pgfqpoint{3.464953in}{1.914015in}}%
\pgfpathlineto{\pgfqpoint{3.476481in}{1.931551in}}%
\pgfpathlineto{\pgfqpoint{3.488005in}{1.948496in}}%
\pgfpathlineto{\pgfqpoint{3.481690in}{1.968789in}}%
\pgfpathlineto{\pgfqpoint{3.475369in}{1.988576in}}%
\pgfpathlineto{\pgfqpoint{3.469044in}{2.007905in}}%
\pgfpathlineto{\pgfqpoint{3.462716in}{2.026824in}}%
\pgfpathlineto{\pgfqpoint{3.456385in}{2.045381in}}%
\pgfpathlineto{\pgfqpoint{3.444847in}{2.026433in}}%
\pgfpathlineto{\pgfqpoint{3.433297in}{2.005801in}}%
\pgfpathlineto{\pgfqpoint{3.421740in}{1.984101in}}%
\pgfpathlineto{\pgfqpoint{3.410185in}{1.961950in}}%
\pgfpathlineto{\pgfqpoint{3.398636in}{1.939963in}}%
\pgfpathlineto{\pgfqpoint{3.404980in}{1.924073in}}%
\pgfpathlineto{\pgfqpoint{3.411327in}{1.908352in}}%
\pgfpathlineto{\pgfqpoint{3.417676in}{1.892692in}}%
\pgfpathlineto{\pgfqpoint{3.424028in}{1.876988in}}%
\pgfpathclose%
\pgfusepath{stroke,fill}%
\end{pgfscope}%
\begin{pgfscope}%
\pgfpathrectangle{\pgfqpoint{0.887500in}{0.275000in}}{\pgfqpoint{4.225000in}{4.225000in}}%
\pgfusepath{clip}%
\pgfsetbuttcap%
\pgfsetroundjoin%
\definecolor{currentfill}{rgb}{0.162142,0.474838,0.558140}%
\pgfsetfillcolor{currentfill}%
\pgfsetfillopacity{0.700000}%
\pgfsetlinewidth{0.501875pt}%
\definecolor{currentstroke}{rgb}{1.000000,1.000000,1.000000}%
\pgfsetstrokecolor{currentstroke}%
\pgfsetstrokeopacity{0.500000}%
\pgfsetdash{}{0pt}%
\pgfpathmoveto{\pgfqpoint{3.456385in}{2.045381in}}%
\pgfpathlineto{\pgfqpoint{3.467904in}{2.062303in}}%
\pgfpathlineto{\pgfqpoint{3.479407in}{2.077468in}}%
\pgfpathlineto{\pgfqpoint{3.490896in}{2.091226in}}%
\pgfpathlineto{\pgfqpoint{3.502372in}{2.103927in}}%
\pgfpathlineto{\pgfqpoint{3.513840in}{2.115924in}}%
\pgfpathlineto{\pgfqpoint{3.507500in}{2.134049in}}%
\pgfpathlineto{\pgfqpoint{3.501161in}{2.152162in}}%
\pgfpathlineto{\pgfqpoint{3.494823in}{2.170313in}}%
\pgfpathlineto{\pgfqpoint{3.488487in}{2.188483in}}%
\pgfpathlineto{\pgfqpoint{3.482152in}{2.206600in}}%
\pgfpathlineto{\pgfqpoint{3.470686in}{2.194472in}}%
\pgfpathlineto{\pgfqpoint{3.459212in}{2.181494in}}%
\pgfpathlineto{\pgfqpoint{3.447725in}{2.167338in}}%
\pgfpathlineto{\pgfqpoint{3.436225in}{2.151678in}}%
\pgfpathlineto{\pgfqpoint{3.424709in}{2.134189in}}%
\pgfpathlineto{\pgfqpoint{3.431046in}{2.116899in}}%
\pgfpathlineto{\pgfqpoint{3.437382in}{2.099360in}}%
\pgfpathlineto{\pgfqpoint{3.443717in}{2.081602in}}%
\pgfpathlineto{\pgfqpoint{3.450052in}{2.063624in}}%
\pgfpathclose%
\pgfusepath{stroke,fill}%
\end{pgfscope}%
\begin{pgfscope}%
\pgfpathrectangle{\pgfqpoint{0.887500in}{0.275000in}}{\pgfqpoint{4.225000in}{4.225000in}}%
\pgfusepath{clip}%
\pgfsetbuttcap%
\pgfsetroundjoin%
\definecolor{currentfill}{rgb}{0.168126,0.459988,0.558082}%
\pgfsetfillcolor{currentfill}%
\pgfsetfillopacity{0.700000}%
\pgfsetlinewidth{0.501875pt}%
\definecolor{currentstroke}{rgb}{1.000000,1.000000,1.000000}%
\pgfsetstrokecolor{currentstroke}%
\pgfsetstrokeopacity{0.500000}%
\pgfsetdash{}{0pt}%
\pgfpathmoveto{\pgfqpoint{2.164467in}{2.059621in}}%
\pgfpathlineto{\pgfqpoint{2.176179in}{2.063278in}}%
\pgfpathlineto{\pgfqpoint{2.187886in}{2.066925in}}%
\pgfpathlineto{\pgfqpoint{2.199587in}{2.070563in}}%
\pgfpathlineto{\pgfqpoint{2.211282in}{2.074194in}}%
\pgfpathlineto{\pgfqpoint{2.222972in}{2.077821in}}%
\pgfpathlineto{\pgfqpoint{2.216942in}{2.087147in}}%
\pgfpathlineto{\pgfqpoint{2.210917in}{2.096439in}}%
\pgfpathlineto{\pgfqpoint{2.204896in}{2.105699in}}%
\pgfpathlineto{\pgfqpoint{2.198879in}{2.114928in}}%
\pgfpathlineto{\pgfqpoint{2.192867in}{2.124126in}}%
\pgfpathlineto{\pgfqpoint{2.181187in}{2.120555in}}%
\pgfpathlineto{\pgfqpoint{2.169502in}{2.116980in}}%
\pgfpathlineto{\pgfqpoint{2.157811in}{2.113398in}}%
\pgfpathlineto{\pgfqpoint{2.146115in}{2.109807in}}%
\pgfpathlineto{\pgfqpoint{2.134413in}{2.106206in}}%
\pgfpathlineto{\pgfqpoint{2.140415in}{2.096954in}}%
\pgfpathlineto{\pgfqpoint{2.146421in}{2.087670in}}%
\pgfpathlineto{\pgfqpoint{2.152432in}{2.078354in}}%
\pgfpathlineto{\pgfqpoint{2.158447in}{2.069005in}}%
\pgfpathclose%
\pgfusepath{stroke,fill}%
\end{pgfscope}%
\begin{pgfscope}%
\pgfpathrectangle{\pgfqpoint{0.887500in}{0.275000in}}{\pgfqpoint{4.225000in}{4.225000in}}%
\pgfusepath{clip}%
\pgfsetbuttcap%
\pgfsetroundjoin%
\definecolor{currentfill}{rgb}{0.214298,0.355619,0.551184}%
\pgfsetfillcolor{currentfill}%
\pgfsetfillopacity{0.700000}%
\pgfsetlinewidth{0.501875pt}%
\definecolor{currentstroke}{rgb}{1.000000,1.000000,1.000000}%
\pgfsetstrokecolor{currentstroke}%
\pgfsetstrokeopacity{0.500000}%
\pgfsetdash{}{0pt}%
\pgfpathmoveto{\pgfqpoint{2.991489in}{1.846505in}}%
\pgfpathlineto{\pgfqpoint{3.002997in}{1.850356in}}%
\pgfpathlineto{\pgfqpoint{3.014500in}{1.854290in}}%
\pgfpathlineto{\pgfqpoint{3.025997in}{1.858234in}}%
\pgfpathlineto{\pgfqpoint{3.037488in}{1.862110in}}%
\pgfpathlineto{\pgfqpoint{3.048974in}{1.865844in}}%
\pgfpathlineto{\pgfqpoint{3.042681in}{1.876565in}}%
\pgfpathlineto{\pgfqpoint{3.036392in}{1.887237in}}%
\pgfpathlineto{\pgfqpoint{3.030106in}{1.897855in}}%
\pgfpathlineto{\pgfqpoint{3.023823in}{1.908420in}}%
\pgfpathlineto{\pgfqpoint{3.017545in}{1.918928in}}%
\pgfpathlineto{\pgfqpoint{3.006068in}{1.915211in}}%
\pgfpathlineto{\pgfqpoint{2.994585in}{1.911200in}}%
\pgfpathlineto{\pgfqpoint{2.983098in}{1.907045in}}%
\pgfpathlineto{\pgfqpoint{2.971605in}{1.902893in}}%
\pgfpathlineto{\pgfqpoint{2.960106in}{1.898890in}}%
\pgfpathlineto{\pgfqpoint{2.966375in}{1.888509in}}%
\pgfpathlineto{\pgfqpoint{2.972648in}{1.878081in}}%
\pgfpathlineto{\pgfqpoint{2.978925in}{1.867605in}}%
\pgfpathlineto{\pgfqpoint{2.985205in}{1.857080in}}%
\pgfpathclose%
\pgfusepath{stroke,fill}%
\end{pgfscope}%
\begin{pgfscope}%
\pgfpathrectangle{\pgfqpoint{0.887500in}{0.275000in}}{\pgfqpoint{4.225000in}{4.225000in}}%
\pgfusepath{clip}%
\pgfsetbuttcap%
\pgfsetroundjoin%
\definecolor{currentfill}{rgb}{0.160665,0.478540,0.558115}%
\pgfsetfillcolor{currentfill}%
\pgfsetfillopacity{0.700000}%
\pgfsetlinewidth{0.501875pt}%
\definecolor{currentstroke}{rgb}{1.000000,1.000000,1.000000}%
\pgfsetstrokecolor{currentstroke}%
\pgfsetstrokeopacity{0.500000}%
\pgfsetdash{}{0pt}%
\pgfpathmoveto{\pgfqpoint{3.838499in}{2.080917in}}%
\pgfpathlineto{\pgfqpoint{3.849835in}{2.086522in}}%
\pgfpathlineto{\pgfqpoint{3.861163in}{2.091990in}}%
\pgfpathlineto{\pgfqpoint{3.872488in}{2.097525in}}%
\pgfpathlineto{\pgfqpoint{3.883811in}{2.103161in}}%
\pgfpathlineto{\pgfqpoint{3.895131in}{2.108852in}}%
\pgfpathlineto{\pgfqpoint{3.888692in}{2.123932in}}%
\pgfpathlineto{\pgfqpoint{3.882253in}{2.138887in}}%
\pgfpathlineto{\pgfqpoint{3.875815in}{2.153762in}}%
\pgfpathlineto{\pgfqpoint{3.869378in}{2.168603in}}%
\pgfpathlineto{\pgfqpoint{3.862944in}{2.183454in}}%
\pgfpathlineto{\pgfqpoint{3.851643in}{2.178520in}}%
\pgfpathlineto{\pgfqpoint{3.840341in}{2.173813in}}%
\pgfpathlineto{\pgfqpoint{3.829041in}{2.169409in}}%
\pgfpathlineto{\pgfqpoint{3.817741in}{2.165273in}}%
\pgfpathlineto{\pgfqpoint{3.806436in}{2.161149in}}%
\pgfpathlineto{\pgfqpoint{3.812851in}{2.145459in}}%
\pgfpathlineto{\pgfqpoint{3.819266in}{2.129608in}}%
\pgfpathlineto{\pgfqpoint{3.825679in}{2.113579in}}%
\pgfpathlineto{\pgfqpoint{3.832090in}{2.097355in}}%
\pgfpathclose%
\pgfusepath{stroke,fill}%
\end{pgfscope}%
\begin{pgfscope}%
\pgfpathrectangle{\pgfqpoint{0.887500in}{0.275000in}}{\pgfqpoint{4.225000in}{4.225000in}}%
\pgfusepath{clip}%
\pgfsetbuttcap%
\pgfsetroundjoin%
\definecolor{currentfill}{rgb}{0.231674,0.318106,0.544834}%
\pgfsetfillcolor{currentfill}%
\pgfsetfillopacity{0.700000}%
\pgfsetlinewidth{0.501875pt}%
\definecolor{currentstroke}{rgb}{1.000000,1.000000,1.000000}%
\pgfsetstrokecolor{currentstroke}%
\pgfsetstrokeopacity{0.500000}%
\pgfsetdash{}{0pt}%
\pgfpathmoveto{\pgfqpoint{4.315144in}{1.767399in}}%
\pgfpathlineto{\pgfqpoint{4.326329in}{1.771226in}}%
\pgfpathlineto{\pgfqpoint{4.337502in}{1.774851in}}%
\pgfpathlineto{\pgfqpoint{4.348664in}{1.778314in}}%
\pgfpathlineto{\pgfqpoint{4.359817in}{1.781656in}}%
\pgfpathlineto{\pgfqpoint{4.370961in}{1.784915in}}%
\pgfpathlineto{\pgfqpoint{4.364417in}{1.799350in}}%
\pgfpathlineto{\pgfqpoint{4.357870in}{1.813602in}}%
\pgfpathlineto{\pgfqpoint{4.351320in}{1.827693in}}%
\pgfpathlineto{\pgfqpoint{4.344769in}{1.841646in}}%
\pgfpathlineto{\pgfqpoint{4.338216in}{1.855484in}}%
\pgfpathlineto{\pgfqpoint{4.327061in}{1.851830in}}%
\pgfpathlineto{\pgfqpoint{4.315898in}{1.848111in}}%
\pgfpathlineto{\pgfqpoint{4.304729in}{1.844350in}}%
\pgfpathlineto{\pgfqpoint{4.293554in}{1.840569in}}%
\pgfpathlineto{\pgfqpoint{4.282373in}{1.836791in}}%
\pgfpathlineto{\pgfqpoint{4.288937in}{1.823395in}}%
\pgfpathlineto{\pgfqpoint{4.295497in}{1.809801in}}%
\pgfpathlineto{\pgfqpoint{4.302053in}{1.795967in}}%
\pgfpathlineto{\pgfqpoint{4.308603in}{1.781848in}}%
\pgfpathclose%
\pgfusepath{stroke,fill}%
\end{pgfscope}%
\begin{pgfscope}%
\pgfpathrectangle{\pgfqpoint{0.887500in}{0.275000in}}{\pgfqpoint{4.225000in}{4.225000in}}%
\pgfusepath{clip}%
\pgfsetbuttcap%
\pgfsetroundjoin%
\definecolor{currentfill}{rgb}{0.144759,0.519093,0.556572}%
\pgfsetfillcolor{currentfill}%
\pgfsetfillopacity{0.700000}%
\pgfsetlinewidth{0.501875pt}%
\definecolor{currentstroke}{rgb}{1.000000,1.000000,1.000000}%
\pgfsetstrokecolor{currentstroke}%
\pgfsetstrokeopacity{0.500000}%
\pgfsetdash{}{0pt}%
\pgfpathmoveto{\pgfqpoint{3.660394in}{2.154960in}}%
\pgfpathlineto{\pgfqpoint{3.671865in}{2.167210in}}%
\pgfpathlineto{\pgfqpoint{3.683327in}{2.178906in}}%
\pgfpathlineto{\pgfqpoint{3.694776in}{2.189868in}}%
\pgfpathlineto{\pgfqpoint{3.706208in}{2.199916in}}%
\pgfpathlineto{\pgfqpoint{3.717621in}{2.208870in}}%
\pgfpathlineto{\pgfqpoint{3.711203in}{2.223526in}}%
\pgfpathlineto{\pgfqpoint{3.704785in}{2.238012in}}%
\pgfpathlineto{\pgfqpoint{3.698368in}{2.252395in}}%
\pgfpathlineto{\pgfqpoint{3.691953in}{2.266742in}}%
\pgfpathlineto{\pgfqpoint{3.685540in}{2.281117in}}%
\pgfpathlineto{\pgfqpoint{3.674145in}{2.272811in}}%
\pgfpathlineto{\pgfqpoint{3.662736in}{2.263781in}}%
\pgfpathlineto{\pgfqpoint{3.651315in}{2.254108in}}%
\pgfpathlineto{\pgfqpoint{3.639883in}{2.243869in}}%
\pgfpathlineto{\pgfqpoint{3.628442in}{2.233145in}}%
\pgfpathlineto{\pgfqpoint{3.634828in}{2.217483in}}%
\pgfpathlineto{\pgfqpoint{3.641217in}{2.201906in}}%
\pgfpathlineto{\pgfqpoint{3.647609in}{2.186342in}}%
\pgfpathlineto{\pgfqpoint{3.654001in}{2.170717in}}%
\pgfpathclose%
\pgfusepath{stroke,fill}%
\end{pgfscope}%
\begin{pgfscope}%
\pgfpathrectangle{\pgfqpoint{0.887500in}{0.275000in}}{\pgfqpoint{4.225000in}{4.225000in}}%
\pgfusepath{clip}%
\pgfsetbuttcap%
\pgfsetroundjoin%
\definecolor{currentfill}{rgb}{0.156270,0.489624,0.557936}%
\pgfsetfillcolor{currentfill}%
\pgfsetfillopacity{0.700000}%
\pgfsetlinewidth{0.501875pt}%
\definecolor{currentstroke}{rgb}{1.000000,1.000000,1.000000}%
\pgfsetstrokecolor{currentstroke}%
\pgfsetstrokeopacity{0.500000}%
\pgfsetdash{}{0pt}%
\pgfpathmoveto{\pgfqpoint{1.839850in}{2.125463in}}%
\pgfpathlineto{\pgfqpoint{1.851644in}{2.129092in}}%
\pgfpathlineto{\pgfqpoint{1.863432in}{2.132712in}}%
\pgfpathlineto{\pgfqpoint{1.875214in}{2.136326in}}%
\pgfpathlineto{\pgfqpoint{1.886991in}{2.139933in}}%
\pgfpathlineto{\pgfqpoint{1.898761in}{2.143536in}}%
\pgfpathlineto{\pgfqpoint{1.892841in}{2.152626in}}%
\pgfpathlineto{\pgfqpoint{1.886924in}{2.161686in}}%
\pgfpathlineto{\pgfqpoint{1.881013in}{2.170718in}}%
\pgfpathlineto{\pgfqpoint{1.875105in}{2.179721in}}%
\pgfpathlineto{\pgfqpoint{1.869203in}{2.188697in}}%
\pgfpathlineto{\pgfqpoint{1.857443in}{2.185144in}}%
\pgfpathlineto{\pgfqpoint{1.845677in}{2.181587in}}%
\pgfpathlineto{\pgfqpoint{1.833905in}{2.178024in}}%
\pgfpathlineto{\pgfqpoint{1.822128in}{2.174455in}}%
\pgfpathlineto{\pgfqpoint{1.810346in}{2.170877in}}%
\pgfpathlineto{\pgfqpoint{1.816237in}{2.161853in}}%
\pgfpathlineto{\pgfqpoint{1.822134in}{2.152800in}}%
\pgfpathlineto{\pgfqpoint{1.828035in}{2.143718in}}%
\pgfpathlineto{\pgfqpoint{1.833940in}{2.134605in}}%
\pgfpathclose%
\pgfusepath{stroke,fill}%
\end{pgfscope}%
\begin{pgfscope}%
\pgfpathrectangle{\pgfqpoint{0.887500in}{0.275000in}}{\pgfqpoint{4.225000in}{4.225000in}}%
\pgfusepath{clip}%
\pgfsetbuttcap%
\pgfsetroundjoin%
\definecolor{currentfill}{rgb}{0.187231,0.414746,0.556547}%
\pgfsetfillcolor{currentfill}%
\pgfsetfillopacity{0.700000}%
\pgfsetlinewidth{0.501875pt}%
\definecolor{currentstroke}{rgb}{1.000000,1.000000,1.000000}%
\pgfsetstrokecolor{currentstroke}%
\pgfsetstrokeopacity{0.500000}%
\pgfsetdash{}{0pt}%
\pgfpathmoveto{\pgfqpoint{2.577915in}{1.959577in}}%
\pgfpathlineto{\pgfqpoint{2.589527in}{1.963248in}}%
\pgfpathlineto{\pgfqpoint{2.601134in}{1.966908in}}%
\pgfpathlineto{\pgfqpoint{2.612735in}{1.970561in}}%
\pgfpathlineto{\pgfqpoint{2.624330in}{1.974211in}}%
\pgfpathlineto{\pgfqpoint{2.635920in}{1.977861in}}%
\pgfpathlineto{\pgfqpoint{2.629752in}{1.987728in}}%
\pgfpathlineto{\pgfqpoint{2.623589in}{1.997548in}}%
\pgfpathlineto{\pgfqpoint{2.617430in}{2.007322in}}%
\pgfpathlineto{\pgfqpoint{2.611274in}{2.017050in}}%
\pgfpathlineto{\pgfqpoint{2.605124in}{2.026732in}}%
\pgfpathlineto{\pgfqpoint{2.593544in}{2.023138in}}%
\pgfpathlineto{\pgfqpoint{2.581958in}{2.019545in}}%
\pgfpathlineto{\pgfqpoint{2.570366in}{2.015950in}}%
\pgfpathlineto{\pgfqpoint{2.558769in}{2.012347in}}%
\pgfpathlineto{\pgfqpoint{2.547167in}{2.008734in}}%
\pgfpathlineto{\pgfqpoint{2.553308in}{1.998996in}}%
\pgfpathlineto{\pgfqpoint{2.559453in}{1.989211in}}%
\pgfpathlineto{\pgfqpoint{2.565603in}{1.979380in}}%
\pgfpathlineto{\pgfqpoint{2.571756in}{1.969502in}}%
\pgfpathclose%
\pgfusepath{stroke,fill}%
\end{pgfscope}%
\begin{pgfscope}%
\pgfpathrectangle{\pgfqpoint{0.887500in}{0.275000in}}{\pgfqpoint{4.225000in}{4.225000in}}%
\pgfusepath{clip}%
\pgfsetbuttcap%
\pgfsetroundjoin%
\definecolor{currentfill}{rgb}{0.221989,0.339161,0.548752}%
\pgfsetfillcolor{currentfill}%
\pgfsetfillopacity{0.700000}%
\pgfsetlinewidth{0.501875pt}%
\definecolor{currentstroke}{rgb}{1.000000,1.000000,1.000000}%
\pgfsetstrokecolor{currentstroke}%
\pgfsetstrokeopacity{0.500000}%
\pgfsetdash{}{0pt}%
\pgfpathmoveto{\pgfqpoint{3.080494in}{1.811582in}}%
\pgfpathlineto{\pgfqpoint{3.091983in}{1.815326in}}%
\pgfpathlineto{\pgfqpoint{3.103466in}{1.819049in}}%
\pgfpathlineto{\pgfqpoint{3.114944in}{1.822746in}}%
\pgfpathlineto{\pgfqpoint{3.126415in}{1.826424in}}%
\pgfpathlineto{\pgfqpoint{3.137882in}{1.830100in}}%
\pgfpathlineto{\pgfqpoint{3.131561in}{1.840410in}}%
\pgfpathlineto{\pgfqpoint{3.125245in}{1.850624in}}%
\pgfpathlineto{\pgfqpoint{3.118931in}{1.860752in}}%
\pgfpathlineto{\pgfqpoint{3.112622in}{1.870805in}}%
\pgfpathlineto{\pgfqpoint{3.106316in}{1.880792in}}%
\pgfpathlineto{\pgfqpoint{3.094860in}{1.878092in}}%
\pgfpathlineto{\pgfqpoint{3.083398in}{1.875444in}}%
\pgfpathlineto{\pgfqpoint{3.071929in}{1.872578in}}%
\pgfpathlineto{\pgfqpoint{3.060455in}{1.869358in}}%
\pgfpathlineto{\pgfqpoint{3.048974in}{1.865844in}}%
\pgfpathlineto{\pgfqpoint{3.055271in}{1.855075in}}%
\pgfpathlineto{\pgfqpoint{3.061572in}{1.844262in}}%
\pgfpathlineto{\pgfqpoint{3.067875in}{1.833407in}}%
\pgfpathlineto{\pgfqpoint{3.074183in}{1.822513in}}%
\pgfpathclose%
\pgfusepath{stroke,fill}%
\end{pgfscope}%
\begin{pgfscope}%
\pgfpathrectangle{\pgfqpoint{0.887500in}{0.275000in}}{\pgfqpoint{4.225000in}{4.225000in}}%
\pgfusepath{clip}%
\pgfsetbuttcap%
\pgfsetroundjoin%
\definecolor{currentfill}{rgb}{0.218130,0.347432,0.550038}%
\pgfsetfillcolor{currentfill}%
\pgfsetfillopacity{0.700000}%
\pgfsetlinewidth{0.501875pt}%
\definecolor{currentstroke}{rgb}{1.000000,1.000000,1.000000}%
\pgfsetstrokecolor{currentstroke}%
\pgfsetstrokeopacity{0.500000}%
\pgfsetdash{}{0pt}%
\pgfpathmoveto{\pgfqpoint{4.226396in}{1.818157in}}%
\pgfpathlineto{\pgfqpoint{4.237606in}{1.822010in}}%
\pgfpathlineto{\pgfqpoint{4.248806in}{1.825692in}}%
\pgfpathlineto{\pgfqpoint{4.259999in}{1.829337in}}%
\pgfpathlineto{\pgfqpoint{4.271188in}{1.833040in}}%
\pgfpathlineto{\pgfqpoint{4.282373in}{1.836791in}}%
\pgfpathlineto{\pgfqpoint{4.275807in}{1.850036in}}%
\pgfpathlineto{\pgfqpoint{4.269242in}{1.863172in}}%
\pgfpathlineto{\pgfqpoint{4.262677in}{1.876246in}}%
\pgfpathlineto{\pgfqpoint{4.256115in}{1.889300in}}%
\pgfpathlineto{\pgfqpoint{4.249557in}{1.902379in}}%
\pgfpathlineto{\pgfqpoint{4.238362in}{1.898201in}}%
\pgfpathlineto{\pgfqpoint{4.227166in}{1.894136in}}%
\pgfpathlineto{\pgfqpoint{4.215969in}{1.890205in}}%
\pgfpathlineto{\pgfqpoint{4.204768in}{1.886342in}}%
\pgfpathlineto{\pgfqpoint{4.193561in}{1.882441in}}%
\pgfpathlineto{\pgfqpoint{4.200130in}{1.869870in}}%
\pgfpathlineto{\pgfqpoint{4.206701in}{1.857243in}}%
\pgfpathlineto{\pgfqpoint{4.213271in}{1.844473in}}%
\pgfpathlineto{\pgfqpoint{4.219837in}{1.831473in}}%
\pgfpathclose%
\pgfusepath{stroke,fill}%
\end{pgfscope}%
\begin{pgfscope}%
\pgfpathrectangle{\pgfqpoint{0.887500in}{0.275000in}}{\pgfqpoint{4.225000in}{4.225000in}}%
\pgfusepath{clip}%
\pgfsetbuttcap%
\pgfsetroundjoin%
\definecolor{currentfill}{rgb}{0.208623,0.367752,0.552675}%
\pgfsetfillcolor{currentfill}%
\pgfsetfillopacity{0.700000}%
\pgfsetlinewidth{0.501875pt}%
\definecolor{currentstroke}{rgb}{1.000000,1.000000,1.000000}%
\pgfsetstrokecolor{currentstroke}%
\pgfsetstrokeopacity{0.500000}%
\pgfsetdash{}{0pt}%
\pgfpathmoveto{\pgfqpoint{4.137330in}{1.858735in}}%
\pgfpathlineto{\pgfqpoint{4.148611in}{1.864371in}}%
\pgfpathlineto{\pgfqpoint{4.159872in}{1.869468in}}%
\pgfpathlineto{\pgfqpoint{4.171116in}{1.874109in}}%
\pgfpathlineto{\pgfqpoint{4.182345in}{1.878398in}}%
\pgfpathlineto{\pgfqpoint{4.193561in}{1.882441in}}%
\pgfpathlineto{\pgfqpoint{4.186996in}{1.895043in}}%
\pgfpathlineto{\pgfqpoint{4.180438in}{1.907762in}}%
\pgfpathlineto{\pgfqpoint{4.173889in}{1.920686in}}%
\pgfpathlineto{\pgfqpoint{4.167351in}{1.933902in}}%
\pgfpathlineto{\pgfqpoint{4.160828in}{1.947496in}}%
\pgfpathlineto{\pgfqpoint{4.149605in}{1.943092in}}%
\pgfpathlineto{\pgfqpoint{4.138376in}{1.938610in}}%
\pgfpathlineto{\pgfqpoint{4.127138in}{1.934010in}}%
\pgfpathlineto{\pgfqpoint{4.115891in}{1.929249in}}%
\pgfpathlineto{\pgfqpoint{4.104634in}{1.924291in}}%
\pgfpathlineto{\pgfqpoint{4.111152in}{1.910599in}}%
\pgfpathlineto{\pgfqpoint{4.117683in}{1.897315in}}%
\pgfpathlineto{\pgfqpoint{4.124225in}{1.884322in}}%
\pgfpathlineto{\pgfqpoint{4.130775in}{1.871501in}}%
\pgfpathclose%
\pgfusepath{stroke,fill}%
\end{pgfscope}%
\begin{pgfscope}%
\pgfpathrectangle{\pgfqpoint{0.887500in}{0.275000in}}{\pgfqpoint{4.225000in}{4.225000in}}%
\pgfusepath{clip}%
\pgfsetbuttcap%
\pgfsetroundjoin%
\definecolor{currentfill}{rgb}{0.150476,0.504369,0.557430}%
\pgfsetfillcolor{currentfill}%
\pgfsetfillopacity{0.700000}%
\pgfsetlinewidth{0.501875pt}%
\definecolor{currentstroke}{rgb}{1.000000,1.000000,1.000000}%
\pgfsetstrokecolor{currentstroke}%
\pgfsetstrokeopacity{0.500000}%
\pgfsetdash{}{0pt}%
\pgfpathmoveto{\pgfqpoint{3.749661in}{2.130831in}}%
\pgfpathlineto{\pgfqpoint{3.761064in}{2.139125in}}%
\pgfpathlineto{\pgfqpoint{3.772439in}{2.146025in}}%
\pgfpathlineto{\pgfqpoint{3.783790in}{2.151807in}}%
\pgfpathlineto{\pgfqpoint{3.795120in}{2.156754in}}%
\pgfpathlineto{\pgfqpoint{3.806436in}{2.161149in}}%
\pgfpathlineto{\pgfqpoint{3.800020in}{2.176694in}}%
\pgfpathlineto{\pgfqpoint{3.793603in}{2.192114in}}%
\pgfpathlineto{\pgfqpoint{3.787187in}{2.207423in}}%
\pgfpathlineto{\pgfqpoint{3.780771in}{2.222630in}}%
\pgfpathlineto{\pgfqpoint{3.774355in}{2.237746in}}%
\pgfpathlineto{\pgfqpoint{3.763043in}{2.233428in}}%
\pgfpathlineto{\pgfqpoint{3.751717in}{2.228615in}}%
\pgfpathlineto{\pgfqpoint{3.740373in}{2.223072in}}%
\pgfpathlineto{\pgfqpoint{3.729009in}{2.216567in}}%
\pgfpathlineto{\pgfqpoint{3.717621in}{2.208870in}}%
\pgfpathlineto{\pgfqpoint{3.724037in}{2.193978in}}%
\pgfpathlineto{\pgfqpoint{3.730450in}{2.178785in}}%
\pgfpathlineto{\pgfqpoint{3.736859in}{2.163224in}}%
\pgfpathlineto{\pgfqpoint{3.743263in}{2.147239in}}%
\pgfpathclose%
\pgfusepath{stroke,fill}%
\end{pgfscope}%
\begin{pgfscope}%
\pgfpathrectangle{\pgfqpoint{0.887500in}{0.275000in}}{\pgfqpoint{4.225000in}{4.225000in}}%
\pgfusepath{clip}%
\pgfsetbuttcap%
\pgfsetroundjoin%
\definecolor{currentfill}{rgb}{0.197636,0.391528,0.554969}%
\pgfsetfillcolor{currentfill}%
\pgfsetfillopacity{0.700000}%
\pgfsetlinewidth{0.501875pt}%
\definecolor{currentstroke}{rgb}{1.000000,1.000000,1.000000}%
\pgfsetstrokecolor{currentstroke}%
\pgfsetstrokeopacity{0.500000}%
\pgfsetdash{}{0pt}%
\pgfpathmoveto{\pgfqpoint{4.048196in}{1.896256in}}%
\pgfpathlineto{\pgfqpoint{4.059504in}{1.902312in}}%
\pgfpathlineto{\pgfqpoint{4.070802in}{1.908139in}}%
\pgfpathlineto{\pgfqpoint{4.082089in}{1.913741in}}%
\pgfpathlineto{\pgfqpoint{4.093367in}{1.919123in}}%
\pgfpathlineto{\pgfqpoint{4.104634in}{1.924291in}}%
\pgfpathlineto{\pgfqpoint{4.098134in}{1.938509in}}%
\pgfpathlineto{\pgfqpoint{4.091652in}{1.953330in}}%
\pgfpathlineto{\pgfqpoint{4.085187in}{1.968685in}}%
\pgfpathlineto{\pgfqpoint{4.078736in}{1.984475in}}%
\pgfpathlineto{\pgfqpoint{4.072295in}{2.000602in}}%
\pgfpathlineto{\pgfqpoint{4.061044in}{1.995993in}}%
\pgfpathlineto{\pgfqpoint{4.049785in}{1.991249in}}%
\pgfpathlineto{\pgfqpoint{4.038517in}{1.986351in}}%
\pgfpathlineto{\pgfqpoint{4.027240in}{1.981279in}}%
\pgfpathlineto{\pgfqpoint{4.015954in}{1.976013in}}%
\pgfpathlineto{\pgfqpoint{4.022384in}{1.959481in}}%
\pgfpathlineto{\pgfqpoint{4.028821in}{1.943162in}}%
\pgfpathlineto{\pgfqpoint{4.035267in}{1.927136in}}%
\pgfpathlineto{\pgfqpoint{4.041725in}{1.911482in}}%
\pgfpathclose%
\pgfusepath{stroke,fill}%
\end{pgfscope}%
\begin{pgfscope}%
\pgfpathrectangle{\pgfqpoint{0.887500in}{0.275000in}}{\pgfqpoint{4.225000in}{4.225000in}}%
\pgfusepath{clip}%
\pgfsetbuttcap%
\pgfsetroundjoin%
\definecolor{currentfill}{rgb}{0.235526,0.309527,0.542944}%
\pgfsetfillcolor{currentfill}%
\pgfsetfillopacity{0.700000}%
\pgfsetlinewidth{0.501875pt}%
\definecolor{currentstroke}{rgb}{1.000000,1.000000,1.000000}%
\pgfsetstrokecolor{currentstroke}%
\pgfsetstrokeopacity{0.500000}%
\pgfsetdash{}{0pt}%
\pgfpathmoveto{\pgfqpoint{3.404868in}{1.730701in}}%
\pgfpathlineto{\pgfqpoint{3.416313in}{1.738908in}}%
\pgfpathlineto{\pgfqpoint{3.427759in}{1.747592in}}%
\pgfpathlineto{\pgfqpoint{3.439207in}{1.756690in}}%
\pgfpathlineto{\pgfqpoint{3.450657in}{1.766180in}}%
\pgfpathlineto{\pgfqpoint{3.462110in}{1.776166in}}%
\pgfpathlineto{\pgfqpoint{3.455770in}{1.794091in}}%
\pgfpathlineto{\pgfqpoint{3.449427in}{1.811586in}}%
\pgfpathlineto{\pgfqpoint{3.443081in}{1.828537in}}%
\pgfpathlineto{\pgfqpoint{3.436731in}{1.845017in}}%
\pgfpathlineto{\pgfqpoint{3.430380in}{1.861131in}}%
\pgfpathlineto{\pgfqpoint{3.418870in}{1.844498in}}%
\pgfpathlineto{\pgfqpoint{3.407371in}{1.828904in}}%
\pgfpathlineto{\pgfqpoint{3.395885in}{1.814628in}}%
\pgfpathlineto{\pgfqpoint{3.384411in}{1.801726in}}%
\pgfpathlineto{\pgfqpoint{3.372948in}{1.790186in}}%
\pgfpathlineto{\pgfqpoint{3.379329in}{1.778722in}}%
\pgfpathlineto{\pgfqpoint{3.385713in}{1.767162in}}%
\pgfpathlineto{\pgfqpoint{3.392097in}{1.755364in}}%
\pgfpathlineto{\pgfqpoint{3.398482in}{1.743207in}}%
\pgfpathclose%
\pgfusepath{stroke,fill}%
\end{pgfscope}%
\begin{pgfscope}%
\pgfpathrectangle{\pgfqpoint{0.887500in}{0.275000in}}{\pgfqpoint{4.225000in}{4.225000in}}%
\pgfusepath{clip}%
\pgfsetbuttcap%
\pgfsetroundjoin%
\definecolor{currentfill}{rgb}{0.154815,0.493313,0.557840}%
\pgfsetfillcolor{currentfill}%
\pgfsetfillopacity{0.700000}%
\pgfsetlinewidth{0.501875pt}%
\definecolor{currentstroke}{rgb}{1.000000,1.000000,1.000000}%
\pgfsetstrokecolor{currentstroke}%
\pgfsetstrokeopacity{0.500000}%
\pgfsetdash{}{0pt}%
\pgfpathmoveto{\pgfqpoint{3.602983in}{2.090392in}}%
\pgfpathlineto{\pgfqpoint{3.614469in}{2.103534in}}%
\pgfpathlineto{\pgfqpoint{3.625953in}{2.116574in}}%
\pgfpathlineto{\pgfqpoint{3.637436in}{2.129510in}}%
\pgfpathlineto{\pgfqpoint{3.648916in}{2.142335in}}%
\pgfpathlineto{\pgfqpoint{3.660394in}{2.154960in}}%
\pgfpathlineto{\pgfqpoint{3.654001in}{2.170717in}}%
\pgfpathlineto{\pgfqpoint{3.647609in}{2.186342in}}%
\pgfpathlineto{\pgfqpoint{3.641217in}{2.201906in}}%
\pgfpathlineto{\pgfqpoint{3.634828in}{2.217483in}}%
\pgfpathlineto{\pgfqpoint{3.628442in}{2.233145in}}%
\pgfpathlineto{\pgfqpoint{3.616994in}{2.222014in}}%
\pgfpathlineto{\pgfqpoint{3.605541in}{2.210556in}}%
\pgfpathlineto{\pgfqpoint{3.594083in}{2.198847in}}%
\pgfpathlineto{\pgfqpoint{3.582622in}{2.186965in}}%
\pgfpathlineto{\pgfqpoint{3.571158in}{2.174983in}}%
\pgfpathlineto{\pgfqpoint{3.577519in}{2.157985in}}%
\pgfpathlineto{\pgfqpoint{3.583883in}{2.141095in}}%
\pgfpathlineto{\pgfqpoint{3.590248in}{2.124245in}}%
\pgfpathlineto{\pgfqpoint{3.596616in}{2.107367in}}%
\pgfpathclose%
\pgfusepath{stroke,fill}%
\end{pgfscope}%
\begin{pgfscope}%
\pgfpathrectangle{\pgfqpoint{0.887500in}{0.275000in}}{\pgfqpoint{4.225000in}{4.225000in}}%
\pgfusepath{clip}%
\pgfsetbuttcap%
\pgfsetroundjoin%
\definecolor{currentfill}{rgb}{0.218130,0.347432,0.550038}%
\pgfsetfillcolor{currentfill}%
\pgfsetfillopacity{0.700000}%
\pgfsetlinewidth{0.501875pt}%
\definecolor{currentstroke}{rgb}{1.000000,1.000000,1.000000}%
\pgfsetstrokecolor{currentstroke}%
\pgfsetstrokeopacity{0.500000}%
\pgfsetdash{}{0pt}%
\pgfpathmoveto{\pgfqpoint{3.462110in}{1.776166in}}%
\pgfpathlineto{\pgfqpoint{3.473567in}{1.786779in}}%
\pgfpathlineto{\pgfqpoint{3.485031in}{1.798150in}}%
\pgfpathlineto{\pgfqpoint{3.496504in}{1.810412in}}%
\pgfpathlineto{\pgfqpoint{3.507989in}{1.823695in}}%
\pgfpathlineto{\pgfqpoint{3.519487in}{1.838130in}}%
\pgfpathlineto{\pgfqpoint{3.513204in}{1.861345in}}%
\pgfpathlineto{\pgfqpoint{3.506915in}{1.884102in}}%
\pgfpathlineto{\pgfqpoint{3.500619in}{1.906200in}}%
\pgfpathlineto{\pgfqpoint{3.494316in}{1.927649in}}%
\pgfpathlineto{\pgfqpoint{3.488005in}{1.948496in}}%
\pgfpathlineto{\pgfqpoint{3.476481in}{1.931551in}}%
\pgfpathlineto{\pgfqpoint{3.464953in}{1.914015in}}%
\pgfpathlineto{\pgfqpoint{3.453424in}{1.896215in}}%
\pgfpathlineto{\pgfqpoint{3.441899in}{1.878479in}}%
\pgfpathlineto{\pgfqpoint{3.430380in}{1.861131in}}%
\pgfpathlineto{\pgfqpoint{3.436731in}{1.845017in}}%
\pgfpathlineto{\pgfqpoint{3.443081in}{1.828537in}}%
\pgfpathlineto{\pgfqpoint{3.449427in}{1.811586in}}%
\pgfpathlineto{\pgfqpoint{3.455770in}{1.794091in}}%
\pgfpathclose%
\pgfusepath{stroke,fill}%
\end{pgfscope}%
\begin{pgfscope}%
\pgfpathrectangle{\pgfqpoint{0.887500in}{0.275000in}}{\pgfqpoint{4.225000in}{4.225000in}}%
\pgfusepath{clip}%
\pgfsetbuttcap%
\pgfsetroundjoin%
\definecolor{currentfill}{rgb}{0.172719,0.448791,0.557885}%
\pgfsetfillcolor{currentfill}%
\pgfsetfillopacity{0.700000}%
\pgfsetlinewidth{0.501875pt}%
\definecolor{currentstroke}{rgb}{1.000000,1.000000,1.000000}%
\pgfsetstrokecolor{currentstroke}%
\pgfsetstrokeopacity{0.500000}%
\pgfsetdash{}{0pt}%
\pgfpathmoveto{\pgfqpoint{2.253189in}{2.030661in}}%
\pgfpathlineto{\pgfqpoint{2.264883in}{2.034345in}}%
\pgfpathlineto{\pgfqpoint{2.276571in}{2.038029in}}%
\pgfpathlineto{\pgfqpoint{2.288254in}{2.041715in}}%
\pgfpathlineto{\pgfqpoint{2.299931in}{2.045406in}}%
\pgfpathlineto{\pgfqpoint{2.311602in}{2.049104in}}%
\pgfpathlineto{\pgfqpoint{2.305540in}{2.058550in}}%
\pgfpathlineto{\pgfqpoint{2.299482in}{2.067960in}}%
\pgfpathlineto{\pgfqpoint{2.293428in}{2.077333in}}%
\pgfpathlineto{\pgfqpoint{2.287379in}{2.086672in}}%
\pgfpathlineto{\pgfqpoint{2.281334in}{2.095976in}}%
\pgfpathlineto{\pgfqpoint{2.269673in}{2.092334in}}%
\pgfpathlineto{\pgfqpoint{2.258007in}{2.088700in}}%
\pgfpathlineto{\pgfqpoint{2.246334in}{2.085071in}}%
\pgfpathlineto{\pgfqpoint{2.234656in}{2.081446in}}%
\pgfpathlineto{\pgfqpoint{2.222972in}{2.077821in}}%
\pgfpathlineto{\pgfqpoint{2.229006in}{2.068461in}}%
\pgfpathlineto{\pgfqpoint{2.235045in}{2.059066in}}%
\pgfpathlineto{\pgfqpoint{2.241089in}{2.049635in}}%
\pgfpathlineto{\pgfqpoint{2.247136in}{2.040167in}}%
\pgfpathclose%
\pgfusepath{stroke,fill}%
\end{pgfscope}%
\begin{pgfscope}%
\pgfpathrectangle{\pgfqpoint{0.887500in}{0.275000in}}{\pgfqpoint{4.225000in}{4.225000in}}%
\pgfusepath{clip}%
\pgfsetbuttcap%
\pgfsetroundjoin%
\definecolor{currentfill}{rgb}{0.179019,0.433756,0.557430}%
\pgfsetfillcolor{currentfill}%
\pgfsetfillopacity{0.700000}%
\pgfsetlinewidth{0.501875pt}%
\definecolor{currentstroke}{rgb}{1.000000,1.000000,1.000000}%
\pgfsetstrokecolor{currentstroke}%
\pgfsetstrokeopacity{0.500000}%
\pgfsetdash{}{0pt}%
\pgfpathmoveto{\pgfqpoint{3.488005in}{1.948496in}}%
\pgfpathlineto{\pgfqpoint{3.499523in}{1.964650in}}%
\pgfpathlineto{\pgfqpoint{3.511034in}{1.980086in}}%
\pgfpathlineto{\pgfqpoint{3.522539in}{1.994915in}}%
\pgfpathlineto{\pgfqpoint{3.534039in}{2.009249in}}%
\pgfpathlineto{\pgfqpoint{3.545534in}{2.023200in}}%
\pgfpathlineto{\pgfqpoint{3.539199in}{2.042207in}}%
\pgfpathlineto{\pgfqpoint{3.532861in}{2.060928in}}%
\pgfpathlineto{\pgfqpoint{3.526522in}{2.079418in}}%
\pgfpathlineto{\pgfqpoint{3.520181in}{2.097732in}}%
\pgfpathlineto{\pgfqpoint{3.513840in}{2.115924in}}%
\pgfpathlineto{\pgfqpoint{3.502372in}{2.103927in}}%
\pgfpathlineto{\pgfqpoint{3.490896in}{2.091226in}}%
\pgfpathlineto{\pgfqpoint{3.479407in}{2.077468in}}%
\pgfpathlineto{\pgfqpoint{3.467904in}{2.062303in}}%
\pgfpathlineto{\pgfqpoint{3.456385in}{2.045381in}}%
\pgfpathlineto{\pgfqpoint{3.462716in}{2.026824in}}%
\pgfpathlineto{\pgfqpoint{3.469044in}{2.007905in}}%
\pgfpathlineto{\pgfqpoint{3.475369in}{1.988576in}}%
\pgfpathlineto{\pgfqpoint{3.481690in}{1.968789in}}%
\pgfpathclose%
\pgfusepath{stroke,fill}%
\end{pgfscope}%
\begin{pgfscope}%
\pgfpathrectangle{\pgfqpoint{0.887500in}{0.275000in}}{\pgfqpoint{4.225000in}{4.225000in}}%
\pgfusepath{clip}%
\pgfsetbuttcap%
\pgfsetroundjoin%
\definecolor{currentfill}{rgb}{0.165117,0.467423,0.558141}%
\pgfsetfillcolor{currentfill}%
\pgfsetfillopacity{0.700000}%
\pgfsetlinewidth{0.501875pt}%
\definecolor{currentstroke}{rgb}{1.000000,1.000000,1.000000}%
\pgfsetstrokecolor{currentstroke}%
\pgfsetstrokeopacity{0.500000}%
\pgfsetdash{}{0pt}%
\pgfpathmoveto{\pgfqpoint{3.545534in}{2.023200in}}%
\pgfpathlineto{\pgfqpoint{3.557027in}{2.036878in}}%
\pgfpathlineto{\pgfqpoint{3.568518in}{2.050396in}}%
\pgfpathlineto{\pgfqpoint{3.580007in}{2.063820in}}%
\pgfpathlineto{\pgfqpoint{3.591496in}{2.077154in}}%
\pgfpathlineto{\pgfqpoint{3.602983in}{2.090392in}}%
\pgfpathlineto{\pgfqpoint{3.596616in}{2.107367in}}%
\pgfpathlineto{\pgfqpoint{3.590248in}{2.124245in}}%
\pgfpathlineto{\pgfqpoint{3.583883in}{2.141095in}}%
\pgfpathlineto{\pgfqpoint{3.577519in}{2.157985in}}%
\pgfpathlineto{\pgfqpoint{3.571158in}{2.174983in}}%
\pgfpathlineto{\pgfqpoint{3.559694in}{2.162979in}}%
\pgfpathlineto{\pgfqpoint{3.548230in}{2.151028in}}%
\pgfpathlineto{\pgfqpoint{3.536766in}{2.139206in}}%
\pgfpathlineto{\pgfqpoint{3.525304in}{2.127567in}}%
\pgfpathlineto{\pgfqpoint{3.513840in}{2.115924in}}%
\pgfpathlineto{\pgfqpoint{3.520181in}{2.097732in}}%
\pgfpathlineto{\pgfqpoint{3.526522in}{2.079418in}}%
\pgfpathlineto{\pgfqpoint{3.532861in}{2.060928in}}%
\pgfpathlineto{\pgfqpoint{3.539199in}{2.042207in}}%
\pgfpathclose%
\pgfusepath{stroke,fill}%
\end{pgfscope}%
\begin{pgfscope}%
\pgfpathrectangle{\pgfqpoint{0.887500in}{0.275000in}}{\pgfqpoint{4.225000in}{4.225000in}}%
\pgfusepath{clip}%
\pgfsetbuttcap%
\pgfsetroundjoin%
\definecolor{currentfill}{rgb}{0.229739,0.322361,0.545706}%
\pgfsetfillcolor{currentfill}%
\pgfsetfillopacity{0.700000}%
\pgfsetlinewidth{0.501875pt}%
\definecolor{currentstroke}{rgb}{1.000000,1.000000,1.000000}%
\pgfsetstrokecolor{currentstroke}%
\pgfsetstrokeopacity{0.500000}%
\pgfsetdash{}{0pt}%
\pgfpathmoveto{\pgfqpoint{3.169535in}{1.776811in}}%
\pgfpathlineto{\pgfqpoint{3.181006in}{1.780837in}}%
\pgfpathlineto{\pgfqpoint{3.192470in}{1.784624in}}%
\pgfpathlineto{\pgfqpoint{3.203927in}{1.788069in}}%
\pgfpathlineto{\pgfqpoint{3.215376in}{1.791067in}}%
\pgfpathlineto{\pgfqpoint{3.226818in}{1.793552in}}%
\pgfpathlineto{\pgfqpoint{3.220473in}{1.804643in}}%
\pgfpathlineto{\pgfqpoint{3.214132in}{1.815739in}}%
\pgfpathlineto{\pgfqpoint{3.207795in}{1.826849in}}%
\pgfpathlineto{\pgfqpoint{3.201461in}{1.837972in}}%
\pgfpathlineto{\pgfqpoint{3.195131in}{1.849100in}}%
\pgfpathlineto{\pgfqpoint{3.183692in}{1.845150in}}%
\pgfpathlineto{\pgfqpoint{3.172247in}{1.841300in}}%
\pgfpathlineto{\pgfqpoint{3.160798in}{1.837521in}}%
\pgfpathlineto{\pgfqpoint{3.149342in}{1.833793in}}%
\pgfpathlineto{\pgfqpoint{3.137882in}{1.830100in}}%
\pgfpathlineto{\pgfqpoint{3.144206in}{1.819686in}}%
\pgfpathlineto{\pgfqpoint{3.150533in}{1.809156in}}%
\pgfpathlineto{\pgfqpoint{3.156864in}{1.798502in}}%
\pgfpathlineto{\pgfqpoint{3.163198in}{1.787719in}}%
\pgfpathclose%
\pgfusepath{stroke,fill}%
\end{pgfscope}%
\begin{pgfscope}%
\pgfpathrectangle{\pgfqpoint{0.887500in}{0.275000in}}{\pgfqpoint{4.225000in}{4.225000in}}%
\pgfusepath{clip}%
\pgfsetbuttcap%
\pgfsetroundjoin%
\definecolor{currentfill}{rgb}{0.194100,0.399323,0.555565}%
\pgfsetfillcolor{currentfill}%
\pgfsetfillopacity{0.700000}%
\pgfsetlinewidth{0.501875pt}%
\definecolor{currentstroke}{rgb}{1.000000,1.000000,1.000000}%
\pgfsetstrokecolor{currentstroke}%
\pgfsetstrokeopacity{0.500000}%
\pgfsetdash{}{0pt}%
\pgfpathmoveto{\pgfqpoint{2.666822in}{1.927827in}}%
\pgfpathlineto{\pgfqpoint{2.678416in}{1.931537in}}%
\pgfpathlineto{\pgfqpoint{2.690003in}{1.935252in}}%
\pgfpathlineto{\pgfqpoint{2.701586in}{1.938973in}}%
\pgfpathlineto{\pgfqpoint{2.713162in}{1.942704in}}%
\pgfpathlineto{\pgfqpoint{2.724732in}{1.946446in}}%
\pgfpathlineto{\pgfqpoint{2.718534in}{1.956490in}}%
\pgfpathlineto{\pgfqpoint{2.712340in}{1.966488in}}%
\pgfpathlineto{\pgfqpoint{2.706150in}{1.976440in}}%
\pgfpathlineto{\pgfqpoint{2.699964in}{1.986346in}}%
\pgfpathlineto{\pgfqpoint{2.693783in}{1.996205in}}%
\pgfpathlineto{\pgfqpoint{2.682222in}{1.992517in}}%
\pgfpathlineto{\pgfqpoint{2.670655in}{1.988840in}}%
\pgfpathlineto{\pgfqpoint{2.659083in}{1.985173in}}%
\pgfpathlineto{\pgfqpoint{2.647504in}{1.981514in}}%
\pgfpathlineto{\pgfqpoint{2.635920in}{1.977861in}}%
\pgfpathlineto{\pgfqpoint{2.642092in}{1.967947in}}%
\pgfpathlineto{\pgfqpoint{2.648268in}{1.957987in}}%
\pgfpathlineto{\pgfqpoint{2.654449in}{1.947980in}}%
\pgfpathlineto{\pgfqpoint{2.660633in}{1.937927in}}%
\pgfpathclose%
\pgfusepath{stroke,fill}%
\end{pgfscope}%
\begin{pgfscope}%
\pgfpathrectangle{\pgfqpoint{0.887500in}{0.275000in}}{\pgfqpoint{4.225000in}{4.225000in}}%
\pgfusepath{clip}%
\pgfsetbuttcap%
\pgfsetroundjoin%
\definecolor{currentfill}{rgb}{0.160665,0.478540,0.558115}%
\pgfsetfillcolor{currentfill}%
\pgfsetfillopacity{0.700000}%
\pgfsetlinewidth{0.501875pt}%
\definecolor{currentstroke}{rgb}{1.000000,1.000000,1.000000}%
\pgfsetstrokecolor{currentstroke}%
\pgfsetstrokeopacity{0.500000}%
\pgfsetdash{}{0pt}%
\pgfpathmoveto{\pgfqpoint{1.928433in}{2.097643in}}%
\pgfpathlineto{\pgfqpoint{1.940209in}{2.101298in}}%
\pgfpathlineto{\pgfqpoint{1.951979in}{2.104950in}}%
\pgfpathlineto{\pgfqpoint{1.963743in}{2.108601in}}%
\pgfpathlineto{\pgfqpoint{1.975502in}{2.112252in}}%
\pgfpathlineto{\pgfqpoint{1.987254in}{2.115903in}}%
\pgfpathlineto{\pgfqpoint{1.981300in}{2.125084in}}%
\pgfpathlineto{\pgfqpoint{1.975351in}{2.134235in}}%
\pgfpathlineto{\pgfqpoint{1.969406in}{2.143357in}}%
\pgfpathlineto{\pgfqpoint{1.963465in}{2.152450in}}%
\pgfpathlineto{\pgfqpoint{1.957529in}{2.161516in}}%
\pgfpathlineto{\pgfqpoint{1.945787in}{2.157921in}}%
\pgfpathlineto{\pgfqpoint{1.934039in}{2.154327in}}%
\pgfpathlineto{\pgfqpoint{1.922286in}{2.150732in}}%
\pgfpathlineto{\pgfqpoint{1.910527in}{2.147135in}}%
\pgfpathlineto{\pgfqpoint{1.898761in}{2.143536in}}%
\pgfpathlineto{\pgfqpoint{1.904687in}{2.134417in}}%
\pgfpathlineto{\pgfqpoint{1.910617in}{2.125269in}}%
\pgfpathlineto{\pgfqpoint{1.916551in}{2.116091in}}%
\pgfpathlineto{\pgfqpoint{1.922490in}{2.106882in}}%
\pgfpathclose%
\pgfusepath{stroke,fill}%
\end{pgfscope}%
\begin{pgfscope}%
\pgfpathrectangle{\pgfqpoint{0.887500in}{0.275000in}}{\pgfqpoint{4.225000in}{4.225000in}}%
\pgfusepath{clip}%
\pgfsetbuttcap%
\pgfsetroundjoin%
\definecolor{currentfill}{rgb}{0.239346,0.300855,0.540844}%
\pgfsetfillcolor{currentfill}%
\pgfsetfillopacity{0.700000}%
\pgfsetlinewidth{0.501875pt}%
\definecolor{currentstroke}{rgb}{1.000000,1.000000,1.000000}%
\pgfsetstrokecolor{currentstroke}%
\pgfsetstrokeopacity{0.500000}%
\pgfsetdash{}{0pt}%
\pgfpathmoveto{\pgfqpoint{3.258592in}{1.737825in}}%
\pgfpathlineto{\pgfqpoint{3.270033in}{1.740260in}}%
\pgfpathlineto{\pgfqpoint{3.281468in}{1.742678in}}%
\pgfpathlineto{\pgfqpoint{3.292899in}{1.745340in}}%
\pgfpathlineto{\pgfqpoint{3.304326in}{1.748508in}}%
\pgfpathlineto{\pgfqpoint{3.315752in}{1.752441in}}%
\pgfpathlineto{\pgfqpoint{3.309379in}{1.762962in}}%
\pgfpathlineto{\pgfqpoint{3.303011in}{1.773643in}}%
\pgfpathlineto{\pgfqpoint{3.296648in}{1.784582in}}%
\pgfpathlineto{\pgfqpoint{3.290290in}{1.795876in}}%
\pgfpathlineto{\pgfqpoint{3.283939in}{1.807623in}}%
\pgfpathlineto{\pgfqpoint{3.272520in}{1.803525in}}%
\pgfpathlineto{\pgfqpoint{3.261101in}{1.800388in}}%
\pgfpathlineto{\pgfqpoint{3.249678in}{1.797893in}}%
\pgfpathlineto{\pgfqpoint{3.238251in}{1.795721in}}%
\pgfpathlineto{\pgfqpoint{3.226818in}{1.793552in}}%
\pgfpathlineto{\pgfqpoint{3.233166in}{1.782457in}}%
\pgfpathlineto{\pgfqpoint{3.239517in}{1.771346in}}%
\pgfpathlineto{\pgfqpoint{3.245872in}{1.760211in}}%
\pgfpathlineto{\pgfqpoint{3.252230in}{1.749040in}}%
\pgfpathclose%
\pgfusepath{stroke,fill}%
\end{pgfscope}%
\begin{pgfscope}%
\pgfpathrectangle{\pgfqpoint{0.887500in}{0.275000in}}{\pgfqpoint{4.225000in}{4.225000in}}%
\pgfusepath{clip}%
\pgfsetbuttcap%
\pgfsetroundjoin%
\definecolor{currentfill}{rgb}{0.243113,0.292092,0.538516}%
\pgfsetfillcolor{currentfill}%
\pgfsetfillopacity{0.700000}%
\pgfsetlinewidth{0.501875pt}%
\definecolor{currentstroke}{rgb}{1.000000,1.000000,1.000000}%
\pgfsetstrokecolor{currentstroke}%
\pgfsetstrokeopacity{0.500000}%
\pgfsetdash{}{0pt}%
\pgfpathmoveto{\pgfqpoint{3.493824in}{1.686240in}}%
\pgfpathlineto{\pgfqpoint{3.505213in}{1.690622in}}%
\pgfpathlineto{\pgfqpoint{3.516610in}{1.696115in}}%
\pgfpathlineto{\pgfqpoint{3.528022in}{1.703259in}}%
\pgfpathlineto{\pgfqpoint{3.539458in}{1.712596in}}%
\pgfpathlineto{\pgfqpoint{3.550926in}{1.724665in}}%
\pgfpathlineto{\pgfqpoint{3.544623in}{1.746111in}}%
\pgfpathlineto{\pgfqpoint{3.538332in}{1.768449in}}%
\pgfpathlineto{\pgfqpoint{3.532047in}{1.791412in}}%
\pgfpathlineto{\pgfqpoint{3.525767in}{1.814729in}}%
\pgfpathlineto{\pgfqpoint{3.519487in}{1.838130in}}%
\pgfpathlineto{\pgfqpoint{3.507989in}{1.823695in}}%
\pgfpathlineto{\pgfqpoint{3.496504in}{1.810412in}}%
\pgfpathlineto{\pgfqpoint{3.485031in}{1.798150in}}%
\pgfpathlineto{\pgfqpoint{3.473567in}{1.786779in}}%
\pgfpathlineto{\pgfqpoint{3.462110in}{1.776166in}}%
\pgfpathlineto{\pgfqpoint{3.468448in}{1.757986in}}%
\pgfpathlineto{\pgfqpoint{3.474787in}{1.739728in}}%
\pgfpathlineto{\pgfqpoint{3.481127in}{1.721567in}}%
\pgfpathlineto{\pgfqpoint{3.487473in}{1.703679in}}%
\pgfpathclose%
\pgfusepath{stroke,fill}%
\end{pgfscope}%
\begin{pgfscope}%
\pgfpathrectangle{\pgfqpoint{0.887500in}{0.275000in}}{\pgfqpoint{4.225000in}{4.225000in}}%
\pgfusepath{clip}%
\pgfsetbuttcap%
\pgfsetroundjoin%
\definecolor{currentfill}{rgb}{0.185556,0.418570,0.556753}%
\pgfsetfillcolor{currentfill}%
\pgfsetfillopacity{0.700000}%
\pgfsetlinewidth{0.501875pt}%
\definecolor{currentstroke}{rgb}{1.000000,1.000000,1.000000}%
\pgfsetstrokecolor{currentstroke}%
\pgfsetstrokeopacity{0.500000}%
\pgfsetdash{}{0pt}%
\pgfpathmoveto{\pgfqpoint{3.959362in}{1.945694in}}%
\pgfpathlineto{\pgfqpoint{3.970704in}{1.952444in}}%
\pgfpathlineto{\pgfqpoint{3.982034in}{1.958799in}}%
\pgfpathlineto{\pgfqpoint{3.993352in}{1.964811in}}%
\pgfpathlineto{\pgfqpoint{4.004658in}{1.970532in}}%
\pgfpathlineto{\pgfqpoint{4.015954in}{1.976013in}}%
\pgfpathlineto{\pgfqpoint{4.009529in}{1.992677in}}%
\pgfpathlineto{\pgfqpoint{4.003108in}{2.009395in}}%
\pgfpathlineto{\pgfqpoint{3.996687in}{2.026086in}}%
\pgfpathlineto{\pgfqpoint{3.990266in}{2.042670in}}%
\pgfpathlineto{\pgfqpoint{3.983842in}{2.059067in}}%
\pgfpathlineto{\pgfqpoint{3.972549in}{2.053711in}}%
\pgfpathlineto{\pgfqpoint{3.961247in}{2.048137in}}%
\pgfpathlineto{\pgfqpoint{3.949935in}{2.042340in}}%
\pgfpathlineto{\pgfqpoint{3.938615in}{2.036316in}}%
\pgfpathlineto{\pgfqpoint{3.927284in}{2.030060in}}%
\pgfpathlineto{\pgfqpoint{3.933703in}{2.013477in}}%
\pgfpathlineto{\pgfqpoint{3.940119in}{1.996696in}}%
\pgfpathlineto{\pgfqpoint{3.946533in}{1.979769in}}%
\pgfpathlineto{\pgfqpoint{3.952947in}{1.962750in}}%
\pgfpathclose%
\pgfusepath{stroke,fill}%
\end{pgfscope}%
\begin{pgfscope}%
\pgfpathrectangle{\pgfqpoint{0.887500in}{0.275000in}}{\pgfqpoint{4.225000in}{4.225000in}}%
\pgfusepath{clip}%
\pgfsetbuttcap%
\pgfsetroundjoin%
\definecolor{currentfill}{rgb}{0.225863,0.330805,0.547314}%
\pgfsetfillcolor{currentfill}%
\pgfsetfillopacity{0.700000}%
\pgfsetlinewidth{0.501875pt}%
\definecolor{currentstroke}{rgb}{1.000000,1.000000,1.000000}%
\pgfsetstrokecolor{currentstroke}%
\pgfsetstrokeopacity{0.500000}%
\pgfsetdash{}{0pt}%
\pgfpathmoveto{\pgfqpoint{3.550926in}{1.724665in}}%
\pgfpathlineto{\pgfqpoint{3.562432in}{1.739663in}}%
\pgfpathlineto{\pgfqpoint{3.573970in}{1.757027in}}%
\pgfpathlineto{\pgfqpoint{3.585534in}{1.776093in}}%
\pgfpathlineto{\pgfqpoint{3.597116in}{1.796196in}}%
\pgfpathlineto{\pgfqpoint{3.608706in}{1.816672in}}%
\pgfpathlineto{\pgfqpoint{3.602387in}{1.837318in}}%
\pgfpathlineto{\pgfqpoint{3.596074in}{1.858345in}}%
\pgfpathlineto{\pgfqpoint{3.589764in}{1.879602in}}%
\pgfpathlineto{\pgfqpoint{3.583455in}{1.900938in}}%
\pgfpathlineto{\pgfqpoint{3.577146in}{1.922204in}}%
\pgfpathlineto{\pgfqpoint{3.565602in}{1.904763in}}%
\pgfpathlineto{\pgfqpoint{3.554060in}{1.887340in}}%
\pgfpathlineto{\pgfqpoint{3.542525in}{1.870230in}}%
\pgfpathlineto{\pgfqpoint{3.531000in}{1.853728in}}%
\pgfpathlineto{\pgfqpoint{3.519487in}{1.838130in}}%
\pgfpathlineto{\pgfqpoint{3.525767in}{1.814729in}}%
\pgfpathlineto{\pgfqpoint{3.532047in}{1.791412in}}%
\pgfpathlineto{\pgfqpoint{3.538332in}{1.768449in}}%
\pgfpathlineto{\pgfqpoint{3.544623in}{1.746111in}}%
\pgfpathclose%
\pgfusepath{stroke,fill}%
\end{pgfscope}%
\begin{pgfscope}%
\pgfpathrectangle{\pgfqpoint{0.887500in}{0.275000in}}{\pgfqpoint{4.225000in}{4.225000in}}%
\pgfusepath{clip}%
\pgfsetbuttcap%
\pgfsetroundjoin%
\definecolor{currentfill}{rgb}{0.199430,0.387607,0.554642}%
\pgfsetfillcolor{currentfill}%
\pgfsetfillopacity{0.700000}%
\pgfsetlinewidth{0.501875pt}%
\definecolor{currentstroke}{rgb}{1.000000,1.000000,1.000000}%
\pgfsetstrokecolor{currentstroke}%
\pgfsetstrokeopacity{0.500000}%
\pgfsetdash{}{0pt}%
\pgfpathmoveto{\pgfqpoint{3.519487in}{1.838130in}}%
\pgfpathlineto{\pgfqpoint{3.531000in}{1.853728in}}%
\pgfpathlineto{\pgfqpoint{3.542525in}{1.870230in}}%
\pgfpathlineto{\pgfqpoint{3.554060in}{1.887340in}}%
\pgfpathlineto{\pgfqpoint{3.565602in}{1.904763in}}%
\pgfpathlineto{\pgfqpoint{3.577146in}{1.922204in}}%
\pgfpathlineto{\pgfqpoint{3.570833in}{1.943248in}}%
\pgfpathlineto{\pgfqpoint{3.564517in}{1.963918in}}%
\pgfpathlineto{\pgfqpoint{3.558194in}{1.984110in}}%
\pgfpathlineto{\pgfqpoint{3.551866in}{2.003852in}}%
\pgfpathlineto{\pgfqpoint{3.545534in}{2.023200in}}%
\pgfpathlineto{\pgfqpoint{3.534039in}{2.009249in}}%
\pgfpathlineto{\pgfqpoint{3.522539in}{1.994915in}}%
\pgfpathlineto{\pgfqpoint{3.511034in}{1.980086in}}%
\pgfpathlineto{\pgfqpoint{3.499523in}{1.964650in}}%
\pgfpathlineto{\pgfqpoint{3.488005in}{1.948496in}}%
\pgfpathlineto{\pgfqpoint{3.494316in}{1.927649in}}%
\pgfpathlineto{\pgfqpoint{3.500619in}{1.906200in}}%
\pgfpathlineto{\pgfqpoint{3.506915in}{1.884102in}}%
\pgfpathlineto{\pgfqpoint{3.513204in}{1.861345in}}%
\pgfpathclose%
\pgfusepath{stroke,fill}%
\end{pgfscope}%
\begin{pgfscope}%
\pgfpathrectangle{\pgfqpoint{0.887500in}{0.275000in}}{\pgfqpoint{4.225000in}{4.225000in}}%
\pgfusepath{clip}%
\pgfsetbuttcap%
\pgfsetroundjoin%
\definecolor{currentfill}{rgb}{0.177423,0.437527,0.557565}%
\pgfsetfillcolor{currentfill}%
\pgfsetfillopacity{0.700000}%
\pgfsetlinewidth{0.501875pt}%
\definecolor{currentstroke}{rgb}{1.000000,1.000000,1.000000}%
\pgfsetstrokecolor{currentstroke}%
\pgfsetstrokeopacity{0.500000}%
\pgfsetdash{}{0pt}%
\pgfpathmoveto{\pgfqpoint{2.341981in}{2.001294in}}%
\pgfpathlineto{\pgfqpoint{2.353656in}{2.005059in}}%
\pgfpathlineto{\pgfqpoint{2.365325in}{2.008827in}}%
\pgfpathlineto{\pgfqpoint{2.376989in}{2.012594in}}%
\pgfpathlineto{\pgfqpoint{2.388647in}{2.016356in}}%
\pgfpathlineto{\pgfqpoint{2.400300in}{2.020110in}}%
\pgfpathlineto{\pgfqpoint{2.394205in}{2.029691in}}%
\pgfpathlineto{\pgfqpoint{2.388115in}{2.039232in}}%
\pgfpathlineto{\pgfqpoint{2.382029in}{2.048733in}}%
\pgfpathlineto{\pgfqpoint{2.375948in}{2.058197in}}%
\pgfpathlineto{\pgfqpoint{2.369871in}{2.067623in}}%
\pgfpathlineto{\pgfqpoint{2.358228in}{2.063925in}}%
\pgfpathlineto{\pgfqpoint{2.346580in}{2.060222in}}%
\pgfpathlineto{\pgfqpoint{2.334927in}{2.056515in}}%
\pgfpathlineto{\pgfqpoint{2.323267in}{2.052808in}}%
\pgfpathlineto{\pgfqpoint{2.311602in}{2.049104in}}%
\pgfpathlineto{\pgfqpoint{2.317669in}{2.039621in}}%
\pgfpathlineto{\pgfqpoint{2.323740in}{2.030098in}}%
\pgfpathlineto{\pgfqpoint{2.329816in}{2.020537in}}%
\pgfpathlineto{\pgfqpoint{2.335896in}{2.010936in}}%
\pgfpathclose%
\pgfusepath{stroke,fill}%
\end{pgfscope}%
\begin{pgfscope}%
\pgfpathrectangle{\pgfqpoint{0.887500in}{0.275000in}}{\pgfqpoint{4.225000in}{4.225000in}}%
\pgfusepath{clip}%
\pgfsetbuttcap%
\pgfsetroundjoin%
\definecolor{currentfill}{rgb}{0.199430,0.387607,0.554642}%
\pgfsetfillcolor{currentfill}%
\pgfsetfillopacity{0.700000}%
\pgfsetlinewidth{0.501875pt}%
\definecolor{currentstroke}{rgb}{1.000000,1.000000,1.000000}%
\pgfsetstrokecolor{currentstroke}%
\pgfsetstrokeopacity{0.500000}%
\pgfsetdash{}{0pt}%
\pgfpathmoveto{\pgfqpoint{2.755784in}{1.895526in}}%
\pgfpathlineto{\pgfqpoint{2.767358in}{1.899333in}}%
\pgfpathlineto{\pgfqpoint{2.778927in}{1.903154in}}%
\pgfpathlineto{\pgfqpoint{2.790490in}{1.906983in}}%
\pgfpathlineto{\pgfqpoint{2.802047in}{1.910803in}}%
\pgfpathlineto{\pgfqpoint{2.813599in}{1.914593in}}%
\pgfpathlineto{\pgfqpoint{2.807371in}{1.924822in}}%
\pgfpathlineto{\pgfqpoint{2.801147in}{1.935003in}}%
\pgfpathlineto{\pgfqpoint{2.794927in}{1.945136in}}%
\pgfpathlineto{\pgfqpoint{2.788712in}{1.955221in}}%
\pgfpathlineto{\pgfqpoint{2.782500in}{1.965260in}}%
\pgfpathlineto{\pgfqpoint{2.770957in}{1.961522in}}%
\pgfpathlineto{\pgfqpoint{2.759410in}{1.957752in}}%
\pgfpathlineto{\pgfqpoint{2.747856in}{1.953971in}}%
\pgfpathlineto{\pgfqpoint{2.736297in}{1.950201in}}%
\pgfpathlineto{\pgfqpoint{2.724732in}{1.946446in}}%
\pgfpathlineto{\pgfqpoint{2.730935in}{1.936355in}}%
\pgfpathlineto{\pgfqpoint{2.737141in}{1.926218in}}%
\pgfpathlineto{\pgfqpoint{2.743351in}{1.916035in}}%
\pgfpathlineto{\pgfqpoint{2.749566in}{1.905804in}}%
\pgfpathclose%
\pgfusepath{stroke,fill}%
\end{pgfscope}%
\begin{pgfscope}%
\pgfpathrectangle{\pgfqpoint{0.887500in}{0.275000in}}{\pgfqpoint{4.225000in}{4.225000in}}%
\pgfusepath{clip}%
\pgfsetbuttcap%
\pgfsetroundjoin%
\definecolor{currentfill}{rgb}{0.174274,0.445044,0.557792}%
\pgfsetfillcolor{currentfill}%
\pgfsetfillopacity{0.700000}%
\pgfsetlinewidth{0.501875pt}%
\definecolor{currentstroke}{rgb}{1.000000,1.000000,1.000000}%
\pgfsetstrokecolor{currentstroke}%
\pgfsetstrokeopacity{0.500000}%
\pgfsetdash{}{0pt}%
\pgfpathmoveto{\pgfqpoint{3.870494in}{1.994974in}}%
\pgfpathlineto{\pgfqpoint{3.881872in}{2.002606in}}%
\pgfpathlineto{\pgfqpoint{3.893239in}{2.009862in}}%
\pgfpathlineto{\pgfqpoint{3.904596in}{2.016838in}}%
\pgfpathlineto{\pgfqpoint{3.915945in}{2.023569in}}%
\pgfpathlineto{\pgfqpoint{3.927284in}{2.030060in}}%
\pgfpathlineto{\pgfqpoint{3.920862in}{2.046390in}}%
\pgfpathlineto{\pgfqpoint{3.914435in}{2.062415in}}%
\pgfpathlineto{\pgfqpoint{3.908003in}{2.078138in}}%
\pgfpathlineto{\pgfqpoint{3.901568in}{2.093602in}}%
\pgfpathlineto{\pgfqpoint{3.895131in}{2.108852in}}%
\pgfpathlineto{\pgfqpoint{3.883811in}{2.103161in}}%
\pgfpathlineto{\pgfqpoint{3.872488in}{2.097525in}}%
\pgfpathlineto{\pgfqpoint{3.861163in}{2.091990in}}%
\pgfpathlineto{\pgfqpoint{3.849835in}{2.086522in}}%
\pgfpathlineto{\pgfqpoint{3.838499in}{2.080917in}}%
\pgfpathlineto{\pgfqpoint{3.844905in}{2.064250in}}%
\pgfpathlineto{\pgfqpoint{3.851308in}{2.047336in}}%
\pgfpathlineto{\pgfqpoint{3.857708in}{2.030157in}}%
\pgfpathlineto{\pgfqpoint{3.864103in}{2.012698in}}%
\pgfpathclose%
\pgfusepath{stroke,fill}%
\end{pgfscope}%
\begin{pgfscope}%
\pgfpathrectangle{\pgfqpoint{0.887500in}{0.275000in}}{\pgfqpoint{4.225000in}{4.225000in}}%
\pgfusepath{clip}%
\pgfsetbuttcap%
\pgfsetroundjoin%
\definecolor{currentfill}{rgb}{0.244972,0.287675,0.537260}%
\pgfsetfillcolor{currentfill}%
\pgfsetfillopacity{0.700000}%
\pgfsetlinewidth{0.501875pt}%
\definecolor{currentstroke}{rgb}{1.000000,1.000000,1.000000}%
\pgfsetstrokecolor{currentstroke}%
\pgfsetstrokeopacity{0.500000}%
\pgfsetdash{}{0pt}%
\pgfpathmoveto{\pgfqpoint{3.347667in}{1.698957in}}%
\pgfpathlineto{\pgfqpoint{3.359104in}{1.703887in}}%
\pgfpathlineto{\pgfqpoint{3.370543in}{1.709564in}}%
\pgfpathlineto{\pgfqpoint{3.381982in}{1.715967in}}%
\pgfpathlineto{\pgfqpoint{3.393424in}{1.723033in}}%
\pgfpathlineto{\pgfqpoint{3.404868in}{1.730701in}}%
\pgfpathlineto{\pgfqpoint{3.398482in}{1.743207in}}%
\pgfpathlineto{\pgfqpoint{3.392097in}{1.755364in}}%
\pgfpathlineto{\pgfqpoint{3.385713in}{1.767162in}}%
\pgfpathlineto{\pgfqpoint{3.379329in}{1.778722in}}%
\pgfpathlineto{\pgfqpoint{3.372948in}{1.790186in}}%
\pgfpathlineto{\pgfqpoint{3.361495in}{1.779994in}}%
\pgfpathlineto{\pgfqpoint{3.350050in}{1.771141in}}%
\pgfpathlineto{\pgfqpoint{3.338612in}{1.763613in}}%
\pgfpathlineto{\pgfqpoint{3.327180in}{1.757401in}}%
\pgfpathlineto{\pgfqpoint{3.315752in}{1.752441in}}%
\pgfpathlineto{\pgfqpoint{3.322129in}{1.741983in}}%
\pgfpathlineto{\pgfqpoint{3.328510in}{1.731491in}}%
\pgfpathlineto{\pgfqpoint{3.334894in}{1.720867in}}%
\pgfpathlineto{\pgfqpoint{3.341279in}{1.710027in}}%
\pgfpathclose%
\pgfusepath{stroke,fill}%
\end{pgfscope}%
\begin{pgfscope}%
\pgfpathrectangle{\pgfqpoint{0.887500in}{0.275000in}}{\pgfqpoint{4.225000in}{4.225000in}}%
\pgfusepath{clip}%
\pgfsetbuttcap%
\pgfsetroundjoin%
\definecolor{currentfill}{rgb}{0.248629,0.278775,0.534556}%
\pgfsetfillcolor{currentfill}%
\pgfsetfillopacity{0.700000}%
\pgfsetlinewidth{0.501875pt}%
\definecolor{currentstroke}{rgb}{1.000000,1.000000,1.000000}%
\pgfsetstrokecolor{currentstroke}%
\pgfsetstrokeopacity{0.500000}%
\pgfsetdash{}{0pt}%
\pgfpathmoveto{\pgfqpoint{3.582748in}{1.639930in}}%
\pgfpathlineto{\pgfqpoint{3.594237in}{1.653110in}}%
\pgfpathlineto{\pgfqpoint{3.605762in}{1.668790in}}%
\pgfpathlineto{\pgfqpoint{3.617314in}{1.686309in}}%
\pgfpathlineto{\pgfqpoint{3.628887in}{1.705008in}}%
\pgfpathlineto{\pgfqpoint{3.640470in}{1.724223in}}%
\pgfpathlineto{\pgfqpoint{3.634088in}{1.740879in}}%
\pgfpathlineto{\pgfqpoint{3.627722in}{1.758512in}}%
\pgfpathlineto{\pgfqpoint{3.621371in}{1.777119in}}%
\pgfpathlineto{\pgfqpoint{3.615034in}{1.796556in}}%
\pgfpathlineto{\pgfqpoint{3.608706in}{1.816672in}}%
\pgfpathlineto{\pgfqpoint{3.597116in}{1.796196in}}%
\pgfpathlineto{\pgfqpoint{3.585534in}{1.776093in}}%
\pgfpathlineto{\pgfqpoint{3.573970in}{1.757027in}}%
\pgfpathlineto{\pgfqpoint{3.562432in}{1.739663in}}%
\pgfpathlineto{\pgfqpoint{3.550926in}{1.724665in}}%
\pgfpathlineto{\pgfqpoint{3.557244in}{1.704381in}}%
\pgfpathlineto{\pgfqpoint{3.563582in}{1.685527in}}%
\pgfpathlineto{\pgfqpoint{3.569942in}{1.668370in}}%
\pgfpathlineto{\pgfqpoint{3.576331in}{1.653171in}}%
\pgfpathclose%
\pgfusepath{stroke,fill}%
\end{pgfscope}%
\begin{pgfscope}%
\pgfpathrectangle{\pgfqpoint{0.887500in}{0.275000in}}{\pgfqpoint{4.225000in}{4.225000in}}%
\pgfusepath{clip}%
\pgfsetbuttcap%
\pgfsetroundjoin%
\definecolor{currentfill}{rgb}{0.165117,0.467423,0.558141}%
\pgfsetfillcolor{currentfill}%
\pgfsetfillopacity{0.700000}%
\pgfsetlinewidth{0.501875pt}%
\definecolor{currentstroke}{rgb}{1.000000,1.000000,1.000000}%
\pgfsetstrokecolor{currentstroke}%
\pgfsetstrokeopacity{0.500000}%
\pgfsetdash{}{0pt}%
\pgfpathmoveto{\pgfqpoint{2.017091in}{2.069539in}}%
\pgfpathlineto{\pgfqpoint{2.028848in}{2.073249in}}%
\pgfpathlineto{\pgfqpoint{2.040600in}{2.076954in}}%
\pgfpathlineto{\pgfqpoint{2.052346in}{2.080653in}}%
\pgfpathlineto{\pgfqpoint{2.064086in}{2.084343in}}%
\pgfpathlineto{\pgfqpoint{2.075821in}{2.088021in}}%
\pgfpathlineto{\pgfqpoint{2.069834in}{2.097297in}}%
\pgfpathlineto{\pgfqpoint{2.063852in}{2.106540in}}%
\pgfpathlineto{\pgfqpoint{2.057874in}{2.115753in}}%
\pgfpathlineto{\pgfqpoint{2.051900in}{2.124935in}}%
\pgfpathlineto{\pgfqpoint{2.045932in}{2.134088in}}%
\pgfpathlineto{\pgfqpoint{2.034207in}{2.130467in}}%
\pgfpathlineto{\pgfqpoint{2.022477in}{2.126836in}}%
\pgfpathlineto{\pgfqpoint{2.010742in}{2.123197in}}%
\pgfpathlineto{\pgfqpoint{1.999001in}{2.119552in}}%
\pgfpathlineto{\pgfqpoint{1.987254in}{2.115903in}}%
\pgfpathlineto{\pgfqpoint{1.993213in}{2.106692in}}%
\pgfpathlineto{\pgfqpoint{1.999175in}{2.097451in}}%
\pgfpathlineto{\pgfqpoint{2.005143in}{2.088179in}}%
\pgfpathlineto{\pgfqpoint{2.011115in}{2.078875in}}%
\pgfpathclose%
\pgfusepath{stroke,fill}%
\end{pgfscope}%
\begin{pgfscope}%
\pgfpathrectangle{\pgfqpoint{0.887500in}{0.275000in}}{\pgfqpoint{4.225000in}{4.225000in}}%
\pgfusepath{clip}%
\pgfsetbuttcap%
\pgfsetroundjoin%
\definecolor{currentfill}{rgb}{0.182256,0.426184,0.557120}%
\pgfsetfillcolor{currentfill}%
\pgfsetfillopacity{0.700000}%
\pgfsetlinewidth{0.501875pt}%
\definecolor{currentstroke}{rgb}{1.000000,1.000000,1.000000}%
\pgfsetstrokecolor{currentstroke}%
\pgfsetstrokeopacity{0.500000}%
\pgfsetdash{}{0pt}%
\pgfpathmoveto{\pgfqpoint{3.577146in}{1.922204in}}%
\pgfpathlineto{\pgfqpoint{3.588688in}{1.939366in}}%
\pgfpathlineto{\pgfqpoint{3.600225in}{1.955955in}}%
\pgfpathlineto{\pgfqpoint{3.611754in}{1.971831in}}%
\pgfpathlineto{\pgfqpoint{3.623276in}{1.987087in}}%
\pgfpathlineto{\pgfqpoint{3.634791in}{2.001837in}}%
\pgfpathlineto{\pgfqpoint{3.628436in}{2.020193in}}%
\pgfpathlineto{\pgfqpoint{3.622077in}{2.038215in}}%
\pgfpathlineto{\pgfqpoint{3.615715in}{2.055885in}}%
\pgfpathlineto{\pgfqpoint{3.609350in}{2.073254in}}%
\pgfpathlineto{\pgfqpoint{3.602983in}{2.090392in}}%
\pgfpathlineto{\pgfqpoint{3.591496in}{2.077154in}}%
\pgfpathlineto{\pgfqpoint{3.580007in}{2.063820in}}%
\pgfpathlineto{\pgfqpoint{3.568518in}{2.050396in}}%
\pgfpathlineto{\pgfqpoint{3.557027in}{2.036878in}}%
\pgfpathlineto{\pgfqpoint{3.545534in}{2.023200in}}%
\pgfpathlineto{\pgfqpoint{3.551866in}{2.003852in}}%
\pgfpathlineto{\pgfqpoint{3.558194in}{1.984110in}}%
\pgfpathlineto{\pgfqpoint{3.564517in}{1.963918in}}%
\pgfpathlineto{\pgfqpoint{3.570833in}{1.943248in}}%
\pgfpathclose%
\pgfusepath{stroke,fill}%
\end{pgfscope}%
\begin{pgfscope}%
\pgfpathrectangle{\pgfqpoint{0.887500in}{0.275000in}}{\pgfqpoint{4.225000in}{4.225000in}}%
\pgfusepath{clip}%
\pgfsetbuttcap%
\pgfsetroundjoin%
\definecolor{currentfill}{rgb}{0.156270,0.489624,0.557936}%
\pgfsetfillcolor{currentfill}%
\pgfsetfillopacity{0.700000}%
\pgfsetlinewidth{0.501875pt}%
\definecolor{currentstroke}{rgb}{1.000000,1.000000,1.000000}%
\pgfsetstrokecolor{currentstroke}%
\pgfsetstrokeopacity{0.500000}%
\pgfsetdash{}{0pt}%
\pgfpathmoveto{\pgfqpoint{3.692316in}{2.071768in}}%
\pgfpathlineto{\pgfqpoint{3.703813in}{2.085171in}}%
\pgfpathlineto{\pgfqpoint{3.715301in}{2.098009in}}%
\pgfpathlineto{\pgfqpoint{3.726774in}{2.110053in}}%
\pgfpathlineto{\pgfqpoint{3.738229in}{2.121070in}}%
\pgfpathlineto{\pgfqpoint{3.749661in}{2.130831in}}%
\pgfpathlineto{\pgfqpoint{3.743263in}{2.147239in}}%
\pgfpathlineto{\pgfqpoint{3.736859in}{2.163224in}}%
\pgfpathlineto{\pgfqpoint{3.730450in}{2.178785in}}%
\pgfpathlineto{\pgfqpoint{3.724037in}{2.193978in}}%
\pgfpathlineto{\pgfqpoint{3.717621in}{2.208870in}}%
\pgfpathlineto{\pgfqpoint{3.706208in}{2.199916in}}%
\pgfpathlineto{\pgfqpoint{3.694776in}{2.189868in}}%
\pgfpathlineto{\pgfqpoint{3.683327in}{2.178906in}}%
\pgfpathlineto{\pgfqpoint{3.671865in}{2.167210in}}%
\pgfpathlineto{\pgfqpoint{3.660394in}{2.154960in}}%
\pgfpathlineto{\pgfqpoint{3.666785in}{2.138998in}}%
\pgfpathlineto{\pgfqpoint{3.673174in}{2.122759in}}%
\pgfpathlineto{\pgfqpoint{3.679560in}{2.106170in}}%
\pgfpathlineto{\pgfqpoint{3.685941in}{2.089171in}}%
\pgfpathclose%
\pgfusepath{stroke,fill}%
\end{pgfscope}%
\begin{pgfscope}%
\pgfpathrectangle{\pgfqpoint{0.887500in}{0.275000in}}{\pgfqpoint{4.225000in}{4.225000in}}%
\pgfusepath{clip}%
\pgfsetbuttcap%
\pgfsetroundjoin%
\definecolor{currentfill}{rgb}{0.262138,0.242286,0.520837}%
\pgfsetfillcolor{currentfill}%
\pgfsetfillopacity{0.700000}%
\pgfsetlinewidth{0.501875pt}%
\definecolor{currentstroke}{rgb}{1.000000,1.000000,1.000000}%
\pgfsetstrokecolor{currentstroke}%
\pgfsetstrokeopacity{0.500000}%
\pgfsetdash{}{0pt}%
\pgfpathmoveto{\pgfqpoint{3.525758in}{1.611714in}}%
\pgfpathlineto{\pgfqpoint{3.537124in}{1.614049in}}%
\pgfpathlineto{\pgfqpoint{3.548497in}{1.617451in}}%
\pgfpathlineto{\pgfqpoint{3.559886in}{1.622507in}}%
\pgfpathlineto{\pgfqpoint{3.571300in}{1.629805in}}%
\pgfpathlineto{\pgfqpoint{3.582748in}{1.639930in}}%
\pgfpathlineto{\pgfqpoint{3.576331in}{1.653171in}}%
\pgfpathlineto{\pgfqpoint{3.569942in}{1.668370in}}%
\pgfpathlineto{\pgfqpoint{3.563582in}{1.685527in}}%
\pgfpathlineto{\pgfqpoint{3.557244in}{1.704381in}}%
\pgfpathlineto{\pgfqpoint{3.550926in}{1.724665in}}%
\pgfpathlineto{\pgfqpoint{3.539458in}{1.712596in}}%
\pgfpathlineto{\pgfqpoint{3.528022in}{1.703259in}}%
\pgfpathlineto{\pgfqpoint{3.516610in}{1.696115in}}%
\pgfpathlineto{\pgfqpoint{3.505213in}{1.690622in}}%
\pgfpathlineto{\pgfqpoint{3.493824in}{1.686240in}}%
\pgfpathlineto{\pgfqpoint{3.500184in}{1.669424in}}%
\pgfpathlineto{\pgfqpoint{3.506556in}{1.653406in}}%
\pgfpathlineto{\pgfqpoint{3.512940in}{1.638362in}}%
\pgfpathlineto{\pgfqpoint{3.519341in}{1.624461in}}%
\pgfpathclose%
\pgfusepath{stroke,fill}%
\end{pgfscope}%
\begin{pgfscope}%
\pgfpathrectangle{\pgfqpoint{0.887500in}{0.275000in}}{\pgfqpoint{4.225000in}{4.225000in}}%
\pgfusepath{clip}%
\pgfsetbuttcap%
\pgfsetroundjoin%
\definecolor{currentfill}{rgb}{0.168126,0.459988,0.558082}%
\pgfsetfillcolor{currentfill}%
\pgfsetfillopacity{0.700000}%
\pgfsetlinewidth{0.501875pt}%
\definecolor{currentstroke}{rgb}{1.000000,1.000000,1.000000}%
\pgfsetstrokecolor{currentstroke}%
\pgfsetstrokeopacity{0.500000}%
\pgfsetdash{}{0pt}%
\pgfpathmoveto{\pgfqpoint{3.634791in}{2.001837in}}%
\pgfpathlineto{\pgfqpoint{3.646300in}{2.016193in}}%
\pgfpathlineto{\pgfqpoint{3.657807in}{2.030270in}}%
\pgfpathlineto{\pgfqpoint{3.669311in}{2.044180in}}%
\pgfpathlineto{\pgfqpoint{3.680814in}{2.058030in}}%
\pgfpathlineto{\pgfqpoint{3.692316in}{2.071768in}}%
\pgfpathlineto{\pgfqpoint{3.685941in}{2.089171in}}%
\pgfpathlineto{\pgfqpoint{3.679560in}{2.106170in}}%
\pgfpathlineto{\pgfqpoint{3.673174in}{2.122759in}}%
\pgfpathlineto{\pgfqpoint{3.666785in}{2.138998in}}%
\pgfpathlineto{\pgfqpoint{3.660394in}{2.154960in}}%
\pgfpathlineto{\pgfqpoint{3.648916in}{2.142335in}}%
\pgfpathlineto{\pgfqpoint{3.637436in}{2.129510in}}%
\pgfpathlineto{\pgfqpoint{3.625953in}{2.116574in}}%
\pgfpathlineto{\pgfqpoint{3.614469in}{2.103534in}}%
\pgfpathlineto{\pgfqpoint{3.602983in}{2.090392in}}%
\pgfpathlineto{\pgfqpoint{3.609350in}{2.073254in}}%
\pgfpathlineto{\pgfqpoint{3.615715in}{2.055885in}}%
\pgfpathlineto{\pgfqpoint{3.622077in}{2.038215in}}%
\pgfpathlineto{\pgfqpoint{3.628436in}{2.020193in}}%
\pgfpathclose%
\pgfusepath{stroke,fill}%
\end{pgfscope}%
\begin{pgfscope}%
\pgfpathrectangle{\pgfqpoint{0.887500in}{0.275000in}}{\pgfqpoint{4.225000in}{4.225000in}}%
\pgfusepath{clip}%
\pgfsetbuttcap%
\pgfsetroundjoin%
\definecolor{currentfill}{rgb}{0.244972,0.287675,0.537260}%
\pgfsetfillcolor{currentfill}%
\pgfsetfillopacity{0.700000}%
\pgfsetlinewidth{0.501875pt}%
\definecolor{currentstroke}{rgb}{1.000000,1.000000,1.000000}%
\pgfsetstrokecolor{currentstroke}%
\pgfsetstrokeopacity{0.500000}%
\pgfsetdash{}{0pt}%
\pgfpathmoveto{\pgfqpoint{4.347677in}{1.688670in}}%
\pgfpathlineto{\pgfqpoint{4.358904in}{1.693743in}}%
\pgfpathlineto{\pgfqpoint{4.370104in}{1.698154in}}%
\pgfpathlineto{\pgfqpoint{4.381282in}{1.702081in}}%
\pgfpathlineto{\pgfqpoint{4.392444in}{1.705703in}}%
\pgfpathlineto{\pgfqpoint{4.403597in}{1.709198in}}%
\pgfpathlineto{\pgfqpoint{4.397084in}{1.724889in}}%
\pgfpathlineto{\pgfqpoint{4.390563in}{1.740284in}}%
\pgfpathlineto{\pgfqpoint{4.384035in}{1.755405in}}%
\pgfpathlineto{\pgfqpoint{4.377500in}{1.770274in}}%
\pgfpathlineto{\pgfqpoint{4.370961in}{1.784915in}}%
\pgfpathlineto{\pgfqpoint{4.359817in}{1.781656in}}%
\pgfpathlineto{\pgfqpoint{4.348664in}{1.778314in}}%
\pgfpathlineto{\pgfqpoint{4.337502in}{1.774851in}}%
\pgfpathlineto{\pgfqpoint{4.326329in}{1.771226in}}%
\pgfpathlineto{\pgfqpoint{4.315144in}{1.767399in}}%
\pgfpathlineto{\pgfqpoint{4.321676in}{1.752577in}}%
\pgfpathlineto{\pgfqpoint{4.328197in}{1.737337in}}%
\pgfpathlineto{\pgfqpoint{4.334706in}{1.721635in}}%
\pgfpathlineto{\pgfqpoint{4.341199in}{1.705427in}}%
\pgfpathclose%
\pgfusepath{stroke,fill}%
\end{pgfscope}%
\begin{pgfscope}%
\pgfpathrectangle{\pgfqpoint{0.887500in}{0.275000in}}{\pgfqpoint{4.225000in}{4.225000in}}%
\pgfusepath{clip}%
\pgfsetbuttcap%
\pgfsetroundjoin%
\definecolor{currentfill}{rgb}{0.163625,0.471133,0.558148}%
\pgfsetfillcolor{currentfill}%
\pgfsetfillopacity{0.700000}%
\pgfsetlinewidth{0.501875pt}%
\definecolor{currentstroke}{rgb}{1.000000,1.000000,1.000000}%
\pgfsetstrokecolor{currentstroke}%
\pgfsetstrokeopacity{0.500000}%
\pgfsetdash{}{0pt}%
\pgfpathmoveto{\pgfqpoint{3.781574in}{2.043074in}}%
\pgfpathlineto{\pgfqpoint{3.793004in}{2.052689in}}%
\pgfpathlineto{\pgfqpoint{3.804409in}{2.061062in}}%
\pgfpathlineto{\pgfqpoint{3.815790in}{2.068410in}}%
\pgfpathlineto{\pgfqpoint{3.827152in}{2.074954in}}%
\pgfpathlineto{\pgfqpoint{3.838499in}{2.080917in}}%
\pgfpathlineto{\pgfqpoint{3.832090in}{2.097355in}}%
\pgfpathlineto{\pgfqpoint{3.825679in}{2.113579in}}%
\pgfpathlineto{\pgfqpoint{3.819266in}{2.129608in}}%
\pgfpathlineto{\pgfqpoint{3.812851in}{2.145459in}}%
\pgfpathlineto{\pgfqpoint{3.806436in}{2.161149in}}%
\pgfpathlineto{\pgfqpoint{3.795120in}{2.156754in}}%
\pgfpathlineto{\pgfqpoint{3.783790in}{2.151807in}}%
\pgfpathlineto{\pgfqpoint{3.772439in}{2.146025in}}%
\pgfpathlineto{\pgfqpoint{3.761064in}{2.139125in}}%
\pgfpathlineto{\pgfqpoint{3.749661in}{2.130831in}}%
\pgfpathlineto{\pgfqpoint{3.756054in}{2.114017in}}%
\pgfpathlineto{\pgfqpoint{3.762441in}{2.096815in}}%
\pgfpathlineto{\pgfqpoint{3.768824in}{2.079245in}}%
\pgfpathlineto{\pgfqpoint{3.775202in}{2.061325in}}%
\pgfpathclose%
\pgfusepath{stroke,fill}%
\end{pgfscope}%
\begin{pgfscope}%
\pgfpathrectangle{\pgfqpoint{0.887500in}{0.275000in}}{\pgfqpoint{4.225000in}{4.225000in}}%
\pgfusepath{clip}%
\pgfsetbuttcap%
\pgfsetroundjoin%
\definecolor{currentfill}{rgb}{0.206756,0.371758,0.553117}%
\pgfsetfillcolor{currentfill}%
\pgfsetfillopacity{0.700000}%
\pgfsetlinewidth{0.501875pt}%
\definecolor{currentstroke}{rgb}{1.000000,1.000000,1.000000}%
\pgfsetstrokecolor{currentstroke}%
\pgfsetstrokeopacity{0.500000}%
\pgfsetdash{}{0pt}%
\pgfpathmoveto{\pgfqpoint{2.844797in}{1.862710in}}%
\pgfpathlineto{\pgfqpoint{2.856353in}{1.866509in}}%
\pgfpathlineto{\pgfqpoint{2.867903in}{1.870250in}}%
\pgfpathlineto{\pgfqpoint{2.879448in}{1.873915in}}%
\pgfpathlineto{\pgfqpoint{2.890988in}{1.877489in}}%
\pgfpathlineto{\pgfqpoint{2.902522in}{1.880983in}}%
\pgfpathlineto{\pgfqpoint{2.896266in}{1.891370in}}%
\pgfpathlineto{\pgfqpoint{2.890014in}{1.901709in}}%
\pgfpathlineto{\pgfqpoint{2.883765in}{1.912001in}}%
\pgfpathlineto{\pgfqpoint{2.877520in}{1.922246in}}%
\pgfpathlineto{\pgfqpoint{2.871279in}{1.932445in}}%
\pgfpathlineto{\pgfqpoint{2.859754in}{1.929064in}}%
\pgfpathlineto{\pgfqpoint{2.848223in}{1.925589in}}%
\pgfpathlineto{\pgfqpoint{2.836687in}{1.922006in}}%
\pgfpathlineto{\pgfqpoint{2.825146in}{1.918334in}}%
\pgfpathlineto{\pgfqpoint{2.813599in}{1.914593in}}%
\pgfpathlineto{\pgfqpoint{2.819831in}{1.904316in}}%
\pgfpathlineto{\pgfqpoint{2.826066in}{1.893989in}}%
\pgfpathlineto{\pgfqpoint{2.832306in}{1.883613in}}%
\pgfpathlineto{\pgfqpoint{2.838550in}{1.873187in}}%
\pgfpathclose%
\pgfusepath{stroke,fill}%
\end{pgfscope}%
\begin{pgfscope}%
\pgfpathrectangle{\pgfqpoint{0.887500in}{0.275000in}}{\pgfqpoint{4.225000in}{4.225000in}}%
\pgfusepath{clip}%
\pgfsetbuttcap%
\pgfsetroundjoin%
\definecolor{currentfill}{rgb}{0.182256,0.426184,0.557120}%
\pgfsetfillcolor{currentfill}%
\pgfsetfillopacity{0.700000}%
\pgfsetlinewidth{0.501875pt}%
\definecolor{currentstroke}{rgb}{1.000000,1.000000,1.000000}%
\pgfsetstrokecolor{currentstroke}%
\pgfsetstrokeopacity{0.500000}%
\pgfsetdash{}{0pt}%
\pgfpathmoveto{\pgfqpoint{2.430839in}{1.971565in}}%
\pgfpathlineto{\pgfqpoint{2.442497in}{1.975365in}}%
\pgfpathlineto{\pgfqpoint{2.454148in}{1.979149in}}%
\pgfpathlineto{\pgfqpoint{2.465795in}{1.982915in}}%
\pgfpathlineto{\pgfqpoint{2.477435in}{1.986664in}}%
\pgfpathlineto{\pgfqpoint{2.489071in}{1.990393in}}%
\pgfpathlineto{\pgfqpoint{2.482944in}{2.000139in}}%
\pgfpathlineto{\pgfqpoint{2.476822in}{2.009838in}}%
\pgfpathlineto{\pgfqpoint{2.470704in}{2.019494in}}%
\pgfpathlineto{\pgfqpoint{2.464590in}{2.029106in}}%
\pgfpathlineto{\pgfqpoint{2.458481in}{2.038675in}}%
\pgfpathlineto{\pgfqpoint{2.446856in}{2.034996in}}%
\pgfpathlineto{\pgfqpoint{2.435225in}{2.031299in}}%
\pgfpathlineto{\pgfqpoint{2.423589in}{2.027584in}}%
\pgfpathlineto{\pgfqpoint{2.411947in}{2.023854in}}%
\pgfpathlineto{\pgfqpoint{2.400300in}{2.020110in}}%
\pgfpathlineto{\pgfqpoint{2.406399in}{2.010488in}}%
\pgfpathlineto{\pgfqpoint{2.412503in}{2.000824in}}%
\pgfpathlineto{\pgfqpoint{2.418610in}{1.991116in}}%
\pgfpathlineto{\pgfqpoint{2.424723in}{1.981363in}}%
\pgfpathclose%
\pgfusepath{stroke,fill}%
\end{pgfscope}%
\begin{pgfscope}%
\pgfpathrectangle{\pgfqpoint{0.887500in}{0.275000in}}{\pgfqpoint{4.225000in}{4.225000in}}%
\pgfusepath{clip}%
\pgfsetbuttcap%
\pgfsetroundjoin%
\definecolor{currentfill}{rgb}{0.203063,0.379716,0.553925}%
\pgfsetfillcolor{currentfill}%
\pgfsetfillopacity{0.700000}%
\pgfsetlinewidth{0.501875pt}%
\definecolor{currentstroke}{rgb}{1.000000,1.000000,1.000000}%
\pgfsetstrokecolor{currentstroke}%
\pgfsetstrokeopacity{0.500000}%
\pgfsetdash{}{0pt}%
\pgfpathmoveto{\pgfqpoint{3.608706in}{1.816672in}}%
\pgfpathlineto{\pgfqpoint{3.620297in}{1.836851in}}%
\pgfpathlineto{\pgfqpoint{3.631878in}{1.856071in}}%
\pgfpathlineto{\pgfqpoint{3.643442in}{1.873998in}}%
\pgfpathlineto{\pgfqpoint{3.654992in}{1.890804in}}%
\pgfpathlineto{\pgfqpoint{3.666530in}{1.906697in}}%
\pgfpathlineto{\pgfqpoint{3.660185in}{1.926019in}}%
\pgfpathlineto{\pgfqpoint{3.653839in}{1.945240in}}%
\pgfpathlineto{\pgfqpoint{3.647492in}{1.964314in}}%
\pgfpathlineto{\pgfqpoint{3.641142in}{1.983196in}}%
\pgfpathlineto{\pgfqpoint{3.634791in}{2.001837in}}%
\pgfpathlineto{\pgfqpoint{3.623276in}{1.987087in}}%
\pgfpathlineto{\pgfqpoint{3.611754in}{1.971831in}}%
\pgfpathlineto{\pgfqpoint{3.600225in}{1.955955in}}%
\pgfpathlineto{\pgfqpoint{3.588688in}{1.939366in}}%
\pgfpathlineto{\pgfqpoint{3.577146in}{1.922204in}}%
\pgfpathlineto{\pgfqpoint{3.583455in}{1.900938in}}%
\pgfpathlineto{\pgfqpoint{3.589764in}{1.879602in}}%
\pgfpathlineto{\pgfqpoint{3.596074in}{1.858345in}}%
\pgfpathlineto{\pgfqpoint{3.602387in}{1.837318in}}%
\pgfpathclose%
\pgfusepath{stroke,fill}%
\end{pgfscope}%
\begin{pgfscope}%
\pgfpathrectangle{\pgfqpoint{0.887500in}{0.275000in}}{\pgfqpoint{4.225000in}{4.225000in}}%
\pgfusepath{clip}%
\pgfsetbuttcap%
\pgfsetroundjoin%
\definecolor{currentfill}{rgb}{0.250425,0.274290,0.533103}%
\pgfsetfillcolor{currentfill}%
\pgfsetfillopacity{0.700000}%
\pgfsetlinewidth{0.501875pt}%
\definecolor{currentstroke}{rgb}{1.000000,1.000000,1.000000}%
\pgfsetstrokecolor{currentstroke}%
\pgfsetstrokeopacity{0.500000}%
\pgfsetdash{}{0pt}%
\pgfpathmoveto{\pgfqpoint{3.436808in}{1.664875in}}%
\pgfpathlineto{\pgfqpoint{3.448228in}{1.669800in}}%
\pgfpathlineto{\pgfqpoint{3.459641in}{1.674433in}}%
\pgfpathlineto{\pgfqpoint{3.471043in}{1.678641in}}%
\pgfpathlineto{\pgfqpoint{3.482436in}{1.682426in}}%
\pgfpathlineto{\pgfqpoint{3.493824in}{1.686240in}}%
\pgfpathlineto{\pgfqpoint{3.487473in}{1.703679in}}%
\pgfpathlineto{\pgfqpoint{3.481127in}{1.721567in}}%
\pgfpathlineto{\pgfqpoint{3.474787in}{1.739728in}}%
\pgfpathlineto{\pgfqpoint{3.468448in}{1.757986in}}%
\pgfpathlineto{\pgfqpoint{3.462110in}{1.776166in}}%
\pgfpathlineto{\pgfqpoint{3.450657in}{1.766180in}}%
\pgfpathlineto{\pgfqpoint{3.439207in}{1.756690in}}%
\pgfpathlineto{\pgfqpoint{3.427759in}{1.747592in}}%
\pgfpathlineto{\pgfqpoint{3.416313in}{1.738908in}}%
\pgfpathlineto{\pgfqpoint{3.404868in}{1.730701in}}%
\pgfpathlineto{\pgfqpoint{3.411253in}{1.717903in}}%
\pgfpathlineto{\pgfqpoint{3.417639in}{1.704866in}}%
\pgfpathlineto{\pgfqpoint{3.424027in}{1.691646in}}%
\pgfpathlineto{\pgfqpoint{3.430416in}{1.678297in}}%
\pgfpathclose%
\pgfusepath{stroke,fill}%
\end{pgfscope}%
\begin{pgfscope}%
\pgfpathrectangle{\pgfqpoint{0.887500in}{0.275000in}}{\pgfqpoint{4.225000in}{4.225000in}}%
\pgfusepath{clip}%
\pgfsetbuttcap%
\pgfsetroundjoin%
\definecolor{currentfill}{rgb}{0.210503,0.363727,0.552206}%
\pgfsetfillcolor{currentfill}%
\pgfsetfillopacity{0.700000}%
\pgfsetlinewidth{0.501875pt}%
\definecolor{currentstroke}{rgb}{1.000000,1.000000,1.000000}%
\pgfsetstrokecolor{currentstroke}%
\pgfsetstrokeopacity{0.500000}%
\pgfsetdash{}{0pt}%
\pgfpathmoveto{\pgfqpoint{4.080679in}{1.823708in}}%
\pgfpathlineto{\pgfqpoint{4.092036in}{1.831471in}}%
\pgfpathlineto{\pgfqpoint{4.103382in}{1.838901in}}%
\pgfpathlineto{\pgfqpoint{4.114713in}{1.845954in}}%
\pgfpathlineto{\pgfqpoint{4.126030in}{1.852581in}}%
\pgfpathlineto{\pgfqpoint{4.137330in}{1.858735in}}%
\pgfpathlineto{\pgfqpoint{4.130775in}{1.871501in}}%
\pgfpathlineto{\pgfqpoint{4.124225in}{1.884322in}}%
\pgfpathlineto{\pgfqpoint{4.117683in}{1.897315in}}%
\pgfpathlineto{\pgfqpoint{4.111152in}{1.910599in}}%
\pgfpathlineto{\pgfqpoint{4.104634in}{1.924291in}}%
\pgfpathlineto{\pgfqpoint{4.093367in}{1.919123in}}%
\pgfpathlineto{\pgfqpoint{4.082089in}{1.913741in}}%
\pgfpathlineto{\pgfqpoint{4.070802in}{1.908139in}}%
\pgfpathlineto{\pgfqpoint{4.059504in}{1.902312in}}%
\pgfpathlineto{\pgfqpoint{4.048196in}{1.896256in}}%
\pgfpathlineto{\pgfqpoint{4.054678in}{1.881395in}}%
\pgfpathlineto{\pgfqpoint{4.061170in}{1.866805in}}%
\pgfpathlineto{\pgfqpoint{4.067669in}{1.852391in}}%
\pgfpathlineto{\pgfqpoint{4.074173in}{1.838057in}}%
\pgfpathclose%
\pgfusepath{stroke,fill}%
\end{pgfscope}%
\begin{pgfscope}%
\pgfpathrectangle{\pgfqpoint{0.887500in}{0.275000in}}{\pgfqpoint{4.225000in}{4.225000in}}%
\pgfusepath{clip}%
\pgfsetbuttcap%
\pgfsetroundjoin%
\definecolor{currentfill}{rgb}{0.214298,0.355619,0.551184}%
\pgfsetfillcolor{currentfill}%
\pgfsetfillopacity{0.700000}%
\pgfsetlinewidth{0.501875pt}%
\definecolor{currentstroke}{rgb}{1.000000,1.000000,1.000000}%
\pgfsetstrokecolor{currentstroke}%
\pgfsetstrokeopacity{0.500000}%
\pgfsetdash{}{0pt}%
\pgfpathmoveto{\pgfqpoint{2.933861in}{1.828313in}}%
\pgfpathlineto{\pgfqpoint{2.945398in}{1.831906in}}%
\pgfpathlineto{\pgfqpoint{2.956930in}{1.835492in}}%
\pgfpathlineto{\pgfqpoint{2.968455in}{1.839101in}}%
\pgfpathlineto{\pgfqpoint{2.979975in}{1.842763in}}%
\pgfpathlineto{\pgfqpoint{2.991489in}{1.846505in}}%
\pgfpathlineto{\pgfqpoint{2.985205in}{1.857080in}}%
\pgfpathlineto{\pgfqpoint{2.978925in}{1.867605in}}%
\pgfpathlineto{\pgfqpoint{2.972648in}{1.878081in}}%
\pgfpathlineto{\pgfqpoint{2.966375in}{1.888509in}}%
\pgfpathlineto{\pgfqpoint{2.960106in}{1.898890in}}%
\pgfpathlineto{\pgfqpoint{2.948601in}{1.895089in}}%
\pgfpathlineto{\pgfqpoint{2.937091in}{1.891446in}}%
\pgfpathlineto{\pgfqpoint{2.925574in}{1.887913in}}%
\pgfpathlineto{\pgfqpoint{2.914051in}{1.884441in}}%
\pgfpathlineto{\pgfqpoint{2.902522in}{1.880983in}}%
\pgfpathlineto{\pgfqpoint{2.908782in}{1.870548in}}%
\pgfpathlineto{\pgfqpoint{2.915046in}{1.860064in}}%
\pgfpathlineto{\pgfqpoint{2.921314in}{1.849532in}}%
\pgfpathlineto{\pgfqpoint{2.927586in}{1.838949in}}%
\pgfpathclose%
\pgfusepath{stroke,fill}%
\end{pgfscope}%
\begin{pgfscope}%
\pgfpathrectangle{\pgfqpoint{0.887500in}{0.275000in}}{\pgfqpoint{4.225000in}{4.225000in}}%
\pgfusepath{clip}%
\pgfsetbuttcap%
\pgfsetroundjoin%
\definecolor{currentfill}{rgb}{0.169646,0.456262,0.558030}%
\pgfsetfillcolor{currentfill}%
\pgfsetfillopacity{0.700000}%
\pgfsetlinewidth{0.501875pt}%
\definecolor{currentstroke}{rgb}{1.000000,1.000000,1.000000}%
\pgfsetstrokecolor{currentstroke}%
\pgfsetstrokeopacity{0.500000}%
\pgfsetdash{}{0pt}%
\pgfpathmoveto{\pgfqpoint{2.105823in}{2.041146in}}%
\pgfpathlineto{\pgfqpoint{2.117562in}{2.044870in}}%
\pgfpathlineto{\pgfqpoint{2.129297in}{2.048578in}}%
\pgfpathlineto{\pgfqpoint{2.141025in}{2.052272in}}%
\pgfpathlineto{\pgfqpoint{2.152749in}{2.055952in}}%
\pgfpathlineto{\pgfqpoint{2.164467in}{2.059621in}}%
\pgfpathlineto{\pgfqpoint{2.158447in}{2.069005in}}%
\pgfpathlineto{\pgfqpoint{2.152432in}{2.078354in}}%
\pgfpathlineto{\pgfqpoint{2.146421in}{2.087670in}}%
\pgfpathlineto{\pgfqpoint{2.140415in}{2.096954in}}%
\pgfpathlineto{\pgfqpoint{2.134413in}{2.106206in}}%
\pgfpathlineto{\pgfqpoint{2.122706in}{2.102595in}}%
\pgfpathlineto{\pgfqpoint{2.110993in}{2.098972in}}%
\pgfpathlineto{\pgfqpoint{2.099274in}{2.095336in}}%
\pgfpathlineto{\pgfqpoint{2.087550in}{2.091686in}}%
\pgfpathlineto{\pgfqpoint{2.075821in}{2.088021in}}%
\pgfpathlineto{\pgfqpoint{2.081812in}{2.078714in}}%
\pgfpathlineto{\pgfqpoint{2.087808in}{2.069373in}}%
\pgfpathlineto{\pgfqpoint{2.093808in}{2.059999in}}%
\pgfpathlineto{\pgfqpoint{2.099813in}{2.050590in}}%
\pgfpathclose%
\pgfusepath{stroke,fill}%
\end{pgfscope}%
\begin{pgfscope}%
\pgfpathrectangle{\pgfqpoint{0.887500in}{0.275000in}}{\pgfqpoint{4.225000in}{4.225000in}}%
\pgfusepath{clip}%
\pgfsetbuttcap%
\pgfsetroundjoin%
\definecolor{currentfill}{rgb}{0.201239,0.383670,0.554294}%
\pgfsetfillcolor{currentfill}%
\pgfsetfillopacity{0.700000}%
\pgfsetlinewidth{0.501875pt}%
\definecolor{currentstroke}{rgb}{1.000000,1.000000,1.000000}%
\pgfsetstrokecolor{currentstroke}%
\pgfsetstrokeopacity{0.500000}%
\pgfsetdash{}{0pt}%
\pgfpathmoveto{\pgfqpoint{3.991491in}{1.861705in}}%
\pgfpathlineto{\pgfqpoint{4.002857in}{1.869349in}}%
\pgfpathlineto{\pgfqpoint{4.014209in}{1.876567in}}%
\pgfpathlineto{\pgfqpoint{4.025549in}{1.883419in}}%
\pgfpathlineto{\pgfqpoint{4.036878in}{1.889963in}}%
\pgfpathlineto{\pgfqpoint{4.048196in}{1.896256in}}%
\pgfpathlineto{\pgfqpoint{4.041725in}{1.911482in}}%
\pgfpathlineto{\pgfqpoint{4.035267in}{1.927136in}}%
\pgfpathlineto{\pgfqpoint{4.028821in}{1.943162in}}%
\pgfpathlineto{\pgfqpoint{4.022384in}{1.959481in}}%
\pgfpathlineto{\pgfqpoint{4.015954in}{1.976013in}}%
\pgfpathlineto{\pgfqpoint{4.004658in}{1.970532in}}%
\pgfpathlineto{\pgfqpoint{3.993352in}{1.964811in}}%
\pgfpathlineto{\pgfqpoint{3.982034in}{1.958799in}}%
\pgfpathlineto{\pgfqpoint{3.970704in}{1.952444in}}%
\pgfpathlineto{\pgfqpoint{3.959362in}{1.945694in}}%
\pgfpathlineto{\pgfqpoint{3.965778in}{1.928653in}}%
\pgfpathlineto{\pgfqpoint{3.972198in}{1.911682in}}%
\pgfpathlineto{\pgfqpoint{3.978623in}{1.894834in}}%
\pgfpathlineto{\pgfqpoint{3.985054in}{1.878162in}}%
\pgfpathclose%
\pgfusepath{stroke,fill}%
\end{pgfscope}%
\begin{pgfscope}%
\pgfpathrectangle{\pgfqpoint{0.887500in}{0.275000in}}{\pgfqpoint{4.225000in}{4.225000in}}%
\pgfusepath{clip}%
\pgfsetbuttcap%
\pgfsetroundjoin%
\definecolor{currentfill}{rgb}{0.231674,0.318106,0.544834}%
\pgfsetfillcolor{currentfill}%
\pgfsetfillopacity{0.700000}%
\pgfsetlinewidth{0.501875pt}%
\definecolor{currentstroke}{rgb}{1.000000,1.000000,1.000000}%
\pgfsetstrokecolor{currentstroke}%
\pgfsetstrokeopacity{0.500000}%
\pgfsetdash{}{0pt}%
\pgfpathmoveto{\pgfqpoint{4.259000in}{1.743804in}}%
\pgfpathlineto{\pgfqpoint{4.270262in}{1.749272in}}%
\pgfpathlineto{\pgfqpoint{4.281506in}{1.754314in}}%
\pgfpathlineto{\pgfqpoint{4.292734in}{1.758984in}}%
\pgfpathlineto{\pgfqpoint{4.303946in}{1.763332in}}%
\pgfpathlineto{\pgfqpoint{4.315144in}{1.767399in}}%
\pgfpathlineto{\pgfqpoint{4.308603in}{1.781848in}}%
\pgfpathlineto{\pgfqpoint{4.302053in}{1.795967in}}%
\pgfpathlineto{\pgfqpoint{4.295497in}{1.809801in}}%
\pgfpathlineto{\pgfqpoint{4.288937in}{1.823395in}}%
\pgfpathlineto{\pgfqpoint{4.282373in}{1.836791in}}%
\pgfpathlineto{\pgfqpoint{4.271188in}{1.833040in}}%
\pgfpathlineto{\pgfqpoint{4.259999in}{1.829337in}}%
\pgfpathlineto{\pgfqpoint{4.248806in}{1.825692in}}%
\pgfpathlineto{\pgfqpoint{4.237606in}{1.822010in}}%
\pgfpathlineto{\pgfqpoint{4.226396in}{1.818157in}}%
\pgfpathlineto{\pgfqpoint{4.232947in}{1.804438in}}%
\pgfpathlineto{\pgfqpoint{4.239486in}{1.790230in}}%
\pgfpathlineto{\pgfqpoint{4.246009in}{1.775446in}}%
\pgfpathlineto{\pgfqpoint{4.252515in}{1.759999in}}%
\pgfpathclose%
\pgfusepath{stroke,fill}%
\end{pgfscope}%
\begin{pgfscope}%
\pgfpathrectangle{\pgfqpoint{0.887500in}{0.275000in}}{\pgfqpoint{4.225000in}{4.225000in}}%
\pgfusepath{clip}%
\pgfsetbuttcap%
\pgfsetroundjoin%
\definecolor{currentfill}{rgb}{0.156270,0.489624,0.557936}%
\pgfsetfillcolor{currentfill}%
\pgfsetfillopacity{0.700000}%
\pgfsetlinewidth{0.501875pt}%
\definecolor{currentstroke}{rgb}{1.000000,1.000000,1.000000}%
\pgfsetstrokecolor{currentstroke}%
\pgfsetstrokeopacity{0.500000}%
\pgfsetdash{}{0pt}%
\pgfpathmoveto{\pgfqpoint{1.780800in}{2.107159in}}%
\pgfpathlineto{\pgfqpoint{1.792621in}{2.110845in}}%
\pgfpathlineto{\pgfqpoint{1.804437in}{2.114518in}}%
\pgfpathlineto{\pgfqpoint{1.816247in}{2.118177in}}%
\pgfpathlineto{\pgfqpoint{1.828051in}{2.121826in}}%
\pgfpathlineto{\pgfqpoint{1.839850in}{2.125463in}}%
\pgfpathlineto{\pgfqpoint{1.833940in}{2.134605in}}%
\pgfpathlineto{\pgfqpoint{1.828035in}{2.143718in}}%
\pgfpathlineto{\pgfqpoint{1.822134in}{2.152800in}}%
\pgfpathlineto{\pgfqpoint{1.816237in}{2.161853in}}%
\pgfpathlineto{\pgfqpoint{1.810346in}{2.170877in}}%
\pgfpathlineto{\pgfqpoint{1.798557in}{2.167291in}}%
\pgfpathlineto{\pgfqpoint{1.786764in}{2.163694in}}%
\pgfpathlineto{\pgfqpoint{1.774964in}{2.160086in}}%
\pgfpathlineto{\pgfqpoint{1.763160in}{2.156465in}}%
\pgfpathlineto{\pgfqpoint{1.751350in}{2.152830in}}%
\pgfpathlineto{\pgfqpoint{1.757231in}{2.143755in}}%
\pgfpathlineto{\pgfqpoint{1.763116in}{2.134652in}}%
\pgfpathlineto{\pgfqpoint{1.769006in}{2.125518in}}%
\pgfpathlineto{\pgfqpoint{1.774901in}{2.116354in}}%
\pgfpathclose%
\pgfusepath{stroke,fill}%
\end{pgfscope}%
\begin{pgfscope}%
\pgfpathrectangle{\pgfqpoint{0.887500in}{0.275000in}}{\pgfqpoint{4.225000in}{4.225000in}}%
\pgfusepath{clip}%
\pgfsetbuttcap%
\pgfsetroundjoin%
\definecolor{currentfill}{rgb}{0.220057,0.343307,0.549413}%
\pgfsetfillcolor{currentfill}%
\pgfsetfillopacity{0.700000}%
\pgfsetlinewidth{0.501875pt}%
\definecolor{currentstroke}{rgb}{1.000000,1.000000,1.000000}%
\pgfsetstrokecolor{currentstroke}%
\pgfsetstrokeopacity{0.500000}%
\pgfsetdash{}{0pt}%
\pgfpathmoveto{\pgfqpoint{4.170057in}{1.791610in}}%
\pgfpathlineto{\pgfqpoint{4.181375in}{1.798319in}}%
\pgfpathlineto{\pgfqpoint{4.192665in}{1.804214in}}%
\pgfpathlineto{\pgfqpoint{4.203929in}{1.809394in}}%
\pgfpathlineto{\pgfqpoint{4.215172in}{1.813997in}}%
\pgfpathlineto{\pgfqpoint{4.226396in}{1.818157in}}%
\pgfpathlineto{\pgfqpoint{4.219837in}{1.831473in}}%
\pgfpathlineto{\pgfqpoint{4.213271in}{1.844473in}}%
\pgfpathlineto{\pgfqpoint{4.206701in}{1.857243in}}%
\pgfpathlineto{\pgfqpoint{4.200130in}{1.869870in}}%
\pgfpathlineto{\pgfqpoint{4.193561in}{1.882441in}}%
\pgfpathlineto{\pgfqpoint{4.182345in}{1.878398in}}%
\pgfpathlineto{\pgfqpoint{4.171116in}{1.874109in}}%
\pgfpathlineto{\pgfqpoint{4.159872in}{1.869468in}}%
\pgfpathlineto{\pgfqpoint{4.148611in}{1.864371in}}%
\pgfpathlineto{\pgfqpoint{4.137330in}{1.858735in}}%
\pgfpathlineto{\pgfqpoint{4.143886in}{1.845907in}}%
\pgfpathlineto{\pgfqpoint{4.150440in}{1.832897in}}%
\pgfpathlineto{\pgfqpoint{4.156989in}{1.819590in}}%
\pgfpathlineto{\pgfqpoint{4.163530in}{1.805866in}}%
\pgfpathclose%
\pgfusepath{stroke,fill}%
\end{pgfscope}%
\begin{pgfscope}%
\pgfpathrectangle{\pgfqpoint{0.887500in}{0.275000in}}{\pgfqpoint{4.225000in}{4.225000in}}%
\pgfusepath{clip}%
\pgfsetbuttcap%
\pgfsetroundjoin%
\definecolor{currentfill}{rgb}{0.188923,0.410910,0.556326}%
\pgfsetfillcolor{currentfill}%
\pgfsetfillopacity{0.700000}%
\pgfsetlinewidth{0.501875pt}%
\definecolor{currentstroke}{rgb}{1.000000,1.000000,1.000000}%
\pgfsetstrokecolor{currentstroke}%
\pgfsetstrokeopacity{0.500000}%
\pgfsetdash{}{0pt}%
\pgfpathmoveto{\pgfqpoint{2.519770in}{1.940959in}}%
\pgfpathlineto{\pgfqpoint{2.531410in}{1.944723in}}%
\pgfpathlineto{\pgfqpoint{2.543044in}{1.948466in}}%
\pgfpathlineto{\pgfqpoint{2.554673in}{1.952190in}}%
\pgfpathlineto{\pgfqpoint{2.566297in}{1.955892in}}%
\pgfpathlineto{\pgfqpoint{2.577915in}{1.959577in}}%
\pgfpathlineto{\pgfqpoint{2.571756in}{1.969502in}}%
\pgfpathlineto{\pgfqpoint{2.565603in}{1.979380in}}%
\pgfpathlineto{\pgfqpoint{2.559453in}{1.989211in}}%
\pgfpathlineto{\pgfqpoint{2.553308in}{1.998996in}}%
\pgfpathlineto{\pgfqpoint{2.547167in}{2.008734in}}%
\pgfpathlineto{\pgfqpoint{2.535558in}{2.005106in}}%
\pgfpathlineto{\pgfqpoint{2.523945in}{2.001460in}}%
\pgfpathlineto{\pgfqpoint{2.512326in}{1.997792in}}%
\pgfpathlineto{\pgfqpoint{2.500701in}{1.994103in}}%
\pgfpathlineto{\pgfqpoint{2.489071in}{1.990393in}}%
\pgfpathlineto{\pgfqpoint{2.495202in}{1.980602in}}%
\pgfpathlineto{\pgfqpoint{2.501337in}{1.970762in}}%
\pgfpathlineto{\pgfqpoint{2.507477in}{1.960875in}}%
\pgfpathlineto{\pgfqpoint{2.513621in}{1.950941in}}%
\pgfpathclose%
\pgfusepath{stroke,fill}%
\end{pgfscope}%
\begin{pgfscope}%
\pgfpathrectangle{\pgfqpoint{0.887500in}{0.275000in}}{\pgfqpoint{4.225000in}{4.225000in}}%
\pgfusepath{clip}%
\pgfsetbuttcap%
\pgfsetroundjoin%
\definecolor{currentfill}{rgb}{0.188923,0.410910,0.556326}%
\pgfsetfillcolor{currentfill}%
\pgfsetfillopacity{0.700000}%
\pgfsetlinewidth{0.501875pt}%
\definecolor{currentstroke}{rgb}{1.000000,1.000000,1.000000}%
\pgfsetstrokecolor{currentstroke}%
\pgfsetstrokeopacity{0.500000}%
\pgfsetdash{}{0pt}%
\pgfpathmoveto{\pgfqpoint{3.902425in}{1.904270in}}%
\pgfpathlineto{\pgfqpoint{3.913843in}{1.913699in}}%
\pgfpathlineto{\pgfqpoint{3.925246in}{1.922541in}}%
\pgfpathlineto{\pgfqpoint{3.936633in}{1.930795in}}%
\pgfpathlineto{\pgfqpoint{3.948005in}{1.938495in}}%
\pgfpathlineto{\pgfqpoint{3.959362in}{1.945694in}}%
\pgfpathlineto{\pgfqpoint{3.952947in}{1.962750in}}%
\pgfpathlineto{\pgfqpoint{3.946533in}{1.979769in}}%
\pgfpathlineto{\pgfqpoint{3.940119in}{1.996696in}}%
\pgfpathlineto{\pgfqpoint{3.933703in}{2.013477in}}%
\pgfpathlineto{\pgfqpoint{3.927284in}{2.030060in}}%
\pgfpathlineto{\pgfqpoint{3.915945in}{2.023569in}}%
\pgfpathlineto{\pgfqpoint{3.904596in}{2.016838in}}%
\pgfpathlineto{\pgfqpoint{3.893239in}{2.009862in}}%
\pgfpathlineto{\pgfqpoint{3.881872in}{2.002606in}}%
\pgfpathlineto{\pgfqpoint{3.870494in}{1.994974in}}%
\pgfpathlineto{\pgfqpoint{3.876882in}{1.977037in}}%
\pgfpathlineto{\pgfqpoint{3.883268in}{1.958943in}}%
\pgfpathlineto{\pgfqpoint{3.889653in}{1.940747in}}%
\pgfpathlineto{\pgfqpoint{3.896038in}{1.922504in}}%
\pgfpathclose%
\pgfusepath{stroke,fill}%
\end{pgfscope}%
\begin{pgfscope}%
\pgfpathrectangle{\pgfqpoint{0.887500in}{0.275000in}}{\pgfqpoint{4.225000in}{4.225000in}}%
\pgfusepath{clip}%
\pgfsetbuttcap%
\pgfsetroundjoin%
\definecolor{currentfill}{rgb}{0.221989,0.339161,0.548752}%
\pgfsetfillcolor{currentfill}%
\pgfsetfillopacity{0.700000}%
\pgfsetlinewidth{0.501875pt}%
\definecolor{currentstroke}{rgb}{1.000000,1.000000,1.000000}%
\pgfsetstrokecolor{currentstroke}%
\pgfsetstrokeopacity{0.500000}%
\pgfsetdash{}{0pt}%
\pgfpathmoveto{\pgfqpoint{3.022965in}{1.792840in}}%
\pgfpathlineto{\pgfqpoint{3.034482in}{1.796565in}}%
\pgfpathlineto{\pgfqpoint{3.045993in}{1.800310in}}%
\pgfpathlineto{\pgfqpoint{3.057499in}{1.804066in}}%
\pgfpathlineto{\pgfqpoint{3.068999in}{1.807827in}}%
\pgfpathlineto{\pgfqpoint{3.080494in}{1.811582in}}%
\pgfpathlineto{\pgfqpoint{3.074183in}{1.822513in}}%
\pgfpathlineto{\pgfqpoint{3.067875in}{1.833407in}}%
\pgfpathlineto{\pgfqpoint{3.061572in}{1.844262in}}%
\pgfpathlineto{\pgfqpoint{3.055271in}{1.855075in}}%
\pgfpathlineto{\pgfqpoint{3.048974in}{1.865844in}}%
\pgfpathlineto{\pgfqpoint{3.037488in}{1.862110in}}%
\pgfpathlineto{\pgfqpoint{3.025997in}{1.858234in}}%
\pgfpathlineto{\pgfqpoint{3.014500in}{1.854290in}}%
\pgfpathlineto{\pgfqpoint{3.002997in}{1.850356in}}%
\pgfpathlineto{\pgfqpoint{2.991489in}{1.846505in}}%
\pgfpathlineto{\pgfqpoint{2.997777in}{1.835880in}}%
\pgfpathlineto{\pgfqpoint{3.004068in}{1.825201in}}%
\pgfpathlineto{\pgfqpoint{3.010364in}{1.814470in}}%
\pgfpathlineto{\pgfqpoint{3.016662in}{1.803683in}}%
\pgfpathclose%
\pgfusepath{stroke,fill}%
\end{pgfscope}%
\begin{pgfscope}%
\pgfpathrectangle{\pgfqpoint{0.887500in}{0.275000in}}{\pgfqpoint{4.225000in}{4.225000in}}%
\pgfusepath{clip}%
\pgfsetbuttcap%
\pgfsetroundjoin%
\definecolor{currentfill}{rgb}{0.185556,0.418570,0.556753}%
\pgfsetfillcolor{currentfill}%
\pgfsetfillopacity{0.700000}%
\pgfsetlinewidth{0.501875pt}%
\definecolor{currentstroke}{rgb}{1.000000,1.000000,1.000000}%
\pgfsetstrokecolor{currentstroke}%
\pgfsetstrokeopacity{0.500000}%
\pgfsetdash{}{0pt}%
\pgfpathmoveto{\pgfqpoint{3.666530in}{1.906697in}}%
\pgfpathlineto{\pgfqpoint{3.678058in}{1.921890in}}%
\pgfpathlineto{\pgfqpoint{3.689580in}{1.936596in}}%
\pgfpathlineto{\pgfqpoint{3.701098in}{1.951026in}}%
\pgfpathlineto{\pgfqpoint{3.712615in}{1.965385in}}%
\pgfpathlineto{\pgfqpoint{3.724132in}{1.979681in}}%
\pgfpathlineto{\pgfqpoint{3.717776in}{1.998681in}}%
\pgfpathlineto{\pgfqpoint{3.711417in}{2.017417in}}%
\pgfpathlineto{\pgfqpoint{3.705054in}{2.035863in}}%
\pgfpathlineto{\pgfqpoint{3.698688in}{2.053989in}}%
\pgfpathlineto{\pgfqpoint{3.692316in}{2.071768in}}%
\pgfpathlineto{\pgfqpoint{3.680814in}{2.058030in}}%
\pgfpathlineto{\pgfqpoint{3.669311in}{2.044180in}}%
\pgfpathlineto{\pgfqpoint{3.657807in}{2.030270in}}%
\pgfpathlineto{\pgfqpoint{3.646300in}{2.016193in}}%
\pgfpathlineto{\pgfqpoint{3.634791in}{2.001837in}}%
\pgfpathlineto{\pgfqpoint{3.641142in}{1.983196in}}%
\pgfpathlineto{\pgfqpoint{3.647492in}{1.964314in}}%
\pgfpathlineto{\pgfqpoint{3.653839in}{1.945240in}}%
\pgfpathlineto{\pgfqpoint{3.660185in}{1.926019in}}%
\pgfpathclose%
\pgfusepath{stroke,fill}%
\end{pgfscope}%
\begin{pgfscope}%
\pgfpathrectangle{\pgfqpoint{0.887500in}{0.275000in}}{\pgfqpoint{4.225000in}{4.225000in}}%
\pgfusepath{clip}%
\pgfsetbuttcap%
\pgfsetroundjoin%
\definecolor{currentfill}{rgb}{0.225863,0.330805,0.547314}%
\pgfsetfillcolor{currentfill}%
\pgfsetfillopacity{0.700000}%
\pgfsetlinewidth{0.501875pt}%
\definecolor{currentstroke}{rgb}{1.000000,1.000000,1.000000}%
\pgfsetstrokecolor{currentstroke}%
\pgfsetstrokeopacity{0.500000}%
\pgfsetdash{}{0pt}%
\pgfpathmoveto{\pgfqpoint{3.640470in}{1.724223in}}%
\pgfpathlineto{\pgfqpoint{3.652054in}{1.743292in}}%
\pgfpathlineto{\pgfqpoint{3.663630in}{1.761555in}}%
\pgfpathlineto{\pgfqpoint{3.675190in}{1.778677in}}%
\pgfpathlineto{\pgfqpoint{3.686736in}{1.794812in}}%
\pgfpathlineto{\pgfqpoint{3.698271in}{1.810158in}}%
\pgfpathlineto{\pgfqpoint{3.691919in}{1.829325in}}%
\pgfpathlineto{\pgfqpoint{3.685569in}{1.848589in}}%
\pgfpathlineto{\pgfqpoint{3.679222in}{1.867936in}}%
\pgfpathlineto{\pgfqpoint{3.672875in}{1.887321in}}%
\pgfpathlineto{\pgfqpoint{3.666530in}{1.906697in}}%
\pgfpathlineto{\pgfqpoint{3.654992in}{1.890804in}}%
\pgfpathlineto{\pgfqpoint{3.643442in}{1.873998in}}%
\pgfpathlineto{\pgfqpoint{3.631878in}{1.856071in}}%
\pgfpathlineto{\pgfqpoint{3.620297in}{1.836851in}}%
\pgfpathlineto{\pgfqpoint{3.608706in}{1.816672in}}%
\pgfpathlineto{\pgfqpoint{3.615034in}{1.796556in}}%
\pgfpathlineto{\pgfqpoint{3.621371in}{1.777119in}}%
\pgfpathlineto{\pgfqpoint{3.627722in}{1.758512in}}%
\pgfpathlineto{\pgfqpoint{3.634088in}{1.740879in}}%
\pgfpathclose%
\pgfusepath{stroke,fill}%
\end{pgfscope}%
\begin{pgfscope}%
\pgfpathrectangle{\pgfqpoint{0.887500in}{0.275000in}}{\pgfqpoint{4.225000in}{4.225000in}}%
\pgfusepath{clip}%
\pgfsetbuttcap%
\pgfsetroundjoin%
\definecolor{currentfill}{rgb}{0.171176,0.452530,0.557965}%
\pgfsetfillcolor{currentfill}%
\pgfsetfillopacity{0.700000}%
\pgfsetlinewidth{0.501875pt}%
\definecolor{currentstroke}{rgb}{1.000000,1.000000,1.000000}%
\pgfsetstrokecolor{currentstroke}%
\pgfsetstrokeopacity{0.500000}%
\pgfsetdash{}{0pt}%
\pgfpathmoveto{\pgfqpoint{3.724132in}{1.979681in}}%
\pgfpathlineto{\pgfqpoint{3.735644in}{1.993716in}}%
\pgfpathlineto{\pgfqpoint{3.747149in}{2.007282in}}%
\pgfpathlineto{\pgfqpoint{3.758642in}{2.020169in}}%
\pgfpathlineto{\pgfqpoint{3.770119in}{2.032170in}}%
\pgfpathlineto{\pgfqpoint{3.781574in}{2.043074in}}%
\pgfpathlineto{\pgfqpoint{3.775202in}{2.061325in}}%
\pgfpathlineto{\pgfqpoint{3.768824in}{2.079245in}}%
\pgfpathlineto{\pgfqpoint{3.762441in}{2.096815in}}%
\pgfpathlineto{\pgfqpoint{3.756054in}{2.114017in}}%
\pgfpathlineto{\pgfqpoint{3.749661in}{2.130831in}}%
\pgfpathlineto{\pgfqpoint{3.738229in}{2.121070in}}%
\pgfpathlineto{\pgfqpoint{3.726774in}{2.110053in}}%
\pgfpathlineto{\pgfqpoint{3.715301in}{2.098009in}}%
\pgfpathlineto{\pgfqpoint{3.703813in}{2.085171in}}%
\pgfpathlineto{\pgfqpoint{3.692316in}{2.071768in}}%
\pgfpathlineto{\pgfqpoint{3.698688in}{2.053989in}}%
\pgfpathlineto{\pgfqpoint{3.705054in}{2.035863in}}%
\pgfpathlineto{\pgfqpoint{3.711417in}{2.017417in}}%
\pgfpathlineto{\pgfqpoint{3.717776in}{1.998681in}}%
\pgfpathclose%
\pgfusepath{stroke,fill}%
\end{pgfscope}%
\begin{pgfscope}%
\pgfpathrectangle{\pgfqpoint{0.887500in}{0.275000in}}{\pgfqpoint{4.225000in}{4.225000in}}%
\pgfusepath{clip}%
\pgfsetbuttcap%
\pgfsetroundjoin%
\definecolor{currentfill}{rgb}{0.179019,0.433756,0.557430}%
\pgfsetfillcolor{currentfill}%
\pgfsetfillopacity{0.700000}%
\pgfsetlinewidth{0.501875pt}%
\definecolor{currentstroke}{rgb}{1.000000,1.000000,1.000000}%
\pgfsetstrokecolor{currentstroke}%
\pgfsetstrokeopacity{0.500000}%
\pgfsetdash{}{0pt}%
\pgfpathmoveto{\pgfqpoint{3.813374in}{1.947512in}}%
\pgfpathlineto{\pgfqpoint{3.824833in}{1.958589in}}%
\pgfpathlineto{\pgfqpoint{3.836274in}{1.968772in}}%
\pgfpathlineto{\pgfqpoint{3.847696in}{1.978161in}}%
\pgfpathlineto{\pgfqpoint{3.859102in}{1.986860in}}%
\pgfpathlineto{\pgfqpoint{3.870494in}{1.994974in}}%
\pgfpathlineto{\pgfqpoint{3.864103in}{2.012698in}}%
\pgfpathlineto{\pgfqpoint{3.857708in}{2.030157in}}%
\pgfpathlineto{\pgfqpoint{3.851308in}{2.047336in}}%
\pgfpathlineto{\pgfqpoint{3.844905in}{2.064250in}}%
\pgfpathlineto{\pgfqpoint{3.838499in}{2.080917in}}%
\pgfpathlineto{\pgfqpoint{3.827152in}{2.074954in}}%
\pgfpathlineto{\pgfqpoint{3.815790in}{2.068410in}}%
\pgfpathlineto{\pgfqpoint{3.804409in}{2.061062in}}%
\pgfpathlineto{\pgfqpoint{3.793004in}{2.052689in}}%
\pgfpathlineto{\pgfqpoint{3.781574in}{2.043074in}}%
\pgfpathlineto{\pgfqpoint{3.787943in}{2.024509in}}%
\pgfpathlineto{\pgfqpoint{3.794306in}{2.005650in}}%
\pgfpathlineto{\pgfqpoint{3.800666in}{1.986514in}}%
\pgfpathlineto{\pgfqpoint{3.807022in}{1.967122in}}%
\pgfpathclose%
\pgfusepath{stroke,fill}%
\end{pgfscope}%
\begin{pgfscope}%
\pgfpathrectangle{\pgfqpoint{0.887500in}{0.275000in}}{\pgfqpoint{4.225000in}{4.225000in}}%
\pgfusepath{clip}%
\pgfsetbuttcap%
\pgfsetroundjoin%
\definecolor{currentfill}{rgb}{0.172719,0.448791,0.557885}%
\pgfsetfillcolor{currentfill}%
\pgfsetfillopacity{0.700000}%
\pgfsetlinewidth{0.501875pt}%
\definecolor{currentstroke}{rgb}{1.000000,1.000000,1.000000}%
\pgfsetstrokecolor{currentstroke}%
\pgfsetstrokeopacity{0.500000}%
\pgfsetdash{}{0pt}%
\pgfpathmoveto{\pgfqpoint{2.194632in}{2.012155in}}%
\pgfpathlineto{\pgfqpoint{2.206355in}{2.015874in}}%
\pgfpathlineto{\pgfqpoint{2.218072in}{2.019583in}}%
\pgfpathlineto{\pgfqpoint{2.229783in}{2.023283in}}%
\pgfpathlineto{\pgfqpoint{2.241489in}{2.026975in}}%
\pgfpathlineto{\pgfqpoint{2.253189in}{2.030661in}}%
\pgfpathlineto{\pgfqpoint{2.247136in}{2.040167in}}%
\pgfpathlineto{\pgfqpoint{2.241089in}{2.049635in}}%
\pgfpathlineto{\pgfqpoint{2.235045in}{2.059066in}}%
\pgfpathlineto{\pgfqpoint{2.229006in}{2.068461in}}%
\pgfpathlineto{\pgfqpoint{2.222972in}{2.077821in}}%
\pgfpathlineto{\pgfqpoint{2.211282in}{2.074194in}}%
\pgfpathlineto{\pgfqpoint{2.199587in}{2.070563in}}%
\pgfpathlineto{\pgfqpoint{2.187886in}{2.066925in}}%
\pgfpathlineto{\pgfqpoint{2.176179in}{2.063278in}}%
\pgfpathlineto{\pgfqpoint{2.164467in}{2.059621in}}%
\pgfpathlineto{\pgfqpoint{2.170491in}{2.050201in}}%
\pgfpathlineto{\pgfqpoint{2.176519in}{2.040746in}}%
\pgfpathlineto{\pgfqpoint{2.182552in}{2.031254in}}%
\pgfpathlineto{\pgfqpoint{2.188590in}{2.021724in}}%
\pgfpathclose%
\pgfusepath{stroke,fill}%
\end{pgfscope}%
\begin{pgfscope}%
\pgfpathrectangle{\pgfqpoint{0.887500in}{0.275000in}}{\pgfqpoint{4.225000in}{4.225000in}}%
\pgfusepath{clip}%
\pgfsetbuttcap%
\pgfsetroundjoin%
\definecolor{currentfill}{rgb}{0.229739,0.322361,0.545706}%
\pgfsetfillcolor{currentfill}%
\pgfsetfillopacity{0.700000}%
\pgfsetlinewidth{0.501875pt}%
\definecolor{currentstroke}{rgb}{1.000000,1.000000,1.000000}%
\pgfsetstrokecolor{currentstroke}%
\pgfsetstrokeopacity{0.500000}%
\pgfsetdash{}{0pt}%
\pgfpathmoveto{\pgfqpoint{3.112101in}{1.756454in}}%
\pgfpathlineto{\pgfqpoint{3.123599in}{1.760342in}}%
\pgfpathlineto{\pgfqpoint{3.135091in}{1.764337in}}%
\pgfpathlineto{\pgfqpoint{3.146577in}{1.768458in}}%
\pgfpathlineto{\pgfqpoint{3.158059in}{1.772651in}}%
\pgfpathlineto{\pgfqpoint{3.169535in}{1.776811in}}%
\pgfpathlineto{\pgfqpoint{3.163198in}{1.787719in}}%
\pgfpathlineto{\pgfqpoint{3.156864in}{1.798502in}}%
\pgfpathlineto{\pgfqpoint{3.150533in}{1.809156in}}%
\pgfpathlineto{\pgfqpoint{3.144206in}{1.819686in}}%
\pgfpathlineto{\pgfqpoint{3.137882in}{1.830100in}}%
\pgfpathlineto{\pgfqpoint{3.126415in}{1.826424in}}%
\pgfpathlineto{\pgfqpoint{3.114944in}{1.822746in}}%
\pgfpathlineto{\pgfqpoint{3.103466in}{1.819049in}}%
\pgfpathlineto{\pgfqpoint{3.091983in}{1.815326in}}%
\pgfpathlineto{\pgfqpoint{3.080494in}{1.811582in}}%
\pgfpathlineto{\pgfqpoint{3.086808in}{1.800618in}}%
\pgfpathlineto{\pgfqpoint{3.093126in}{1.789622in}}%
\pgfpathlineto{\pgfqpoint{3.099448in}{1.778598in}}%
\pgfpathlineto{\pgfqpoint{3.105773in}{1.767544in}}%
\pgfpathclose%
\pgfusepath{stroke,fill}%
\end{pgfscope}%
\begin{pgfscope}%
\pgfpathrectangle{\pgfqpoint{0.887500in}{0.275000in}}{\pgfqpoint{4.225000in}{4.225000in}}%
\pgfusepath{clip}%
\pgfsetbuttcap%
\pgfsetroundjoin%
\definecolor{currentfill}{rgb}{0.194100,0.399323,0.555565}%
\pgfsetfillcolor{currentfill}%
\pgfsetfillopacity{0.700000}%
\pgfsetlinewidth{0.501875pt}%
\definecolor{currentstroke}{rgb}{1.000000,1.000000,1.000000}%
\pgfsetstrokecolor{currentstroke}%
\pgfsetstrokeopacity{0.500000}%
\pgfsetdash{}{0pt}%
\pgfpathmoveto{\pgfqpoint{2.608769in}{1.909252in}}%
\pgfpathlineto{\pgfqpoint{2.620391in}{1.912982in}}%
\pgfpathlineto{\pgfqpoint{2.632007in}{1.916701in}}%
\pgfpathlineto{\pgfqpoint{2.643618in}{1.920412in}}%
\pgfpathlineto{\pgfqpoint{2.655223in}{1.924120in}}%
\pgfpathlineto{\pgfqpoint{2.666822in}{1.927827in}}%
\pgfpathlineto{\pgfqpoint{2.660633in}{1.937927in}}%
\pgfpathlineto{\pgfqpoint{2.654449in}{1.947980in}}%
\pgfpathlineto{\pgfqpoint{2.648268in}{1.957987in}}%
\pgfpathlineto{\pgfqpoint{2.642092in}{1.967947in}}%
\pgfpathlineto{\pgfqpoint{2.635920in}{1.977861in}}%
\pgfpathlineto{\pgfqpoint{2.624330in}{1.974211in}}%
\pgfpathlineto{\pgfqpoint{2.612735in}{1.970561in}}%
\pgfpathlineto{\pgfqpoint{2.601134in}{1.966908in}}%
\pgfpathlineto{\pgfqpoint{2.589527in}{1.963248in}}%
\pgfpathlineto{\pgfqpoint{2.577915in}{1.959577in}}%
\pgfpathlineto{\pgfqpoint{2.584077in}{1.949605in}}%
\pgfpathlineto{\pgfqpoint{2.590244in}{1.939587in}}%
\pgfpathlineto{\pgfqpoint{2.596414in}{1.929522in}}%
\pgfpathlineto{\pgfqpoint{2.602589in}{1.919410in}}%
\pgfpathclose%
\pgfusepath{stroke,fill}%
\end{pgfscope}%
\begin{pgfscope}%
\pgfpathrectangle{\pgfqpoint{0.887500in}{0.275000in}}{\pgfqpoint{4.225000in}{4.225000in}}%
\pgfusepath{clip}%
\pgfsetbuttcap%
\pgfsetroundjoin%
\definecolor{currentfill}{rgb}{0.160665,0.478540,0.558115}%
\pgfsetfillcolor{currentfill}%
\pgfsetfillopacity{0.700000}%
\pgfsetlinewidth{0.501875pt}%
\definecolor{currentstroke}{rgb}{1.000000,1.000000,1.000000}%
\pgfsetstrokecolor{currentstroke}%
\pgfsetstrokeopacity{0.500000}%
\pgfsetdash{}{0pt}%
\pgfpathmoveto{\pgfqpoint{1.869470in}{2.079294in}}%
\pgfpathlineto{\pgfqpoint{1.881274in}{2.082977in}}%
\pgfpathlineto{\pgfqpoint{1.893072in}{2.086653in}}%
\pgfpathlineto{\pgfqpoint{1.904865in}{2.090322in}}%
\pgfpathlineto{\pgfqpoint{1.916652in}{2.093985in}}%
\pgfpathlineto{\pgfqpoint{1.928433in}{2.097643in}}%
\pgfpathlineto{\pgfqpoint{1.922490in}{2.106882in}}%
\pgfpathlineto{\pgfqpoint{1.916551in}{2.116091in}}%
\pgfpathlineto{\pgfqpoint{1.910617in}{2.125269in}}%
\pgfpathlineto{\pgfqpoint{1.904687in}{2.134417in}}%
\pgfpathlineto{\pgfqpoint{1.898761in}{2.143536in}}%
\pgfpathlineto{\pgfqpoint{1.886991in}{2.139933in}}%
\pgfpathlineto{\pgfqpoint{1.875214in}{2.136326in}}%
\pgfpathlineto{\pgfqpoint{1.863432in}{2.132712in}}%
\pgfpathlineto{\pgfqpoint{1.851644in}{2.129092in}}%
\pgfpathlineto{\pgfqpoint{1.839850in}{2.125463in}}%
\pgfpathlineto{\pgfqpoint{1.845765in}{2.116291in}}%
\pgfpathlineto{\pgfqpoint{1.851684in}{2.107088in}}%
\pgfpathlineto{\pgfqpoint{1.857608in}{2.097854in}}%
\pgfpathlineto{\pgfqpoint{1.863537in}{2.088590in}}%
\pgfpathclose%
\pgfusepath{stroke,fill}%
\end{pgfscope}%
\begin{pgfscope}%
\pgfpathrectangle{\pgfqpoint{0.887500in}{0.275000in}}{\pgfqpoint{4.225000in}{4.225000in}}%
\pgfusepath{clip}%
\pgfsetbuttcap%
\pgfsetroundjoin%
\definecolor{currentfill}{rgb}{0.262138,0.242286,0.520837}%
\pgfsetfillcolor{currentfill}%
\pgfsetfillopacity{0.700000}%
\pgfsetlinewidth{0.501875pt}%
\definecolor{currentstroke}{rgb}{1.000000,1.000000,1.000000}%
\pgfsetstrokecolor{currentstroke}%
\pgfsetstrokeopacity{0.500000}%
\pgfsetdash{}{0pt}%
\pgfpathmoveto{\pgfqpoint{3.615134in}{1.591435in}}%
\pgfpathlineto{\pgfqpoint{3.626592in}{1.601260in}}%
\pgfpathlineto{\pgfqpoint{3.638064in}{1.612195in}}%
\pgfpathlineto{\pgfqpoint{3.649548in}{1.624020in}}%
\pgfpathlineto{\pgfqpoint{3.661042in}{1.636516in}}%
\pgfpathlineto{\pgfqpoint{3.672543in}{1.649462in}}%
\pgfpathlineto{\pgfqpoint{3.666115in}{1.663879in}}%
\pgfpathlineto{\pgfqpoint{3.659691in}{1.678384in}}%
\pgfpathlineto{\pgfqpoint{3.653274in}{1.693156in}}%
\pgfpathlineto{\pgfqpoint{3.646866in}{1.708376in}}%
\pgfpathlineto{\pgfqpoint{3.640470in}{1.724223in}}%
\pgfpathlineto{\pgfqpoint{3.628887in}{1.705008in}}%
\pgfpathlineto{\pgfqpoint{3.617314in}{1.686309in}}%
\pgfpathlineto{\pgfqpoint{3.605762in}{1.668790in}}%
\pgfpathlineto{\pgfqpoint{3.594237in}{1.653110in}}%
\pgfpathlineto{\pgfqpoint{3.582748in}{1.639930in}}%
\pgfpathlineto{\pgfqpoint{3.589191in}{1.628323in}}%
\pgfpathlineto{\pgfqpoint{3.595655in}{1.618011in}}%
\pgfpathlineto{\pgfqpoint{3.602137in}{1.608653in}}%
\pgfpathlineto{\pgfqpoint{3.608631in}{1.599908in}}%
\pgfpathclose%
\pgfusepath{stroke,fill}%
\end{pgfscope}%
\begin{pgfscope}%
\pgfpathrectangle{\pgfqpoint{0.887500in}{0.275000in}}{\pgfqpoint{4.225000in}{4.225000in}}%
\pgfusepath{clip}%
\pgfsetbuttcap%
\pgfsetroundjoin%
\definecolor{currentfill}{rgb}{0.237441,0.305202,0.541921}%
\pgfsetfillcolor{currentfill}%
\pgfsetfillopacity{0.700000}%
\pgfsetlinewidth{0.501875pt}%
\definecolor{currentstroke}{rgb}{1.000000,1.000000,1.000000}%
\pgfsetstrokecolor{currentstroke}%
\pgfsetstrokeopacity{0.500000}%
\pgfsetdash{}{0pt}%
\pgfpathmoveto{\pgfqpoint{3.201269in}{1.720585in}}%
\pgfpathlineto{\pgfqpoint{3.212748in}{1.724613in}}%
\pgfpathlineto{\pgfqpoint{3.224220in}{1.728435in}}%
\pgfpathlineto{\pgfqpoint{3.235685in}{1.731964in}}%
\pgfpathlineto{\pgfqpoint{3.247142in}{1.735113in}}%
\pgfpathlineto{\pgfqpoint{3.258592in}{1.737825in}}%
\pgfpathlineto{\pgfqpoint{3.252230in}{1.749040in}}%
\pgfpathlineto{\pgfqpoint{3.245872in}{1.760211in}}%
\pgfpathlineto{\pgfqpoint{3.239517in}{1.771346in}}%
\pgfpathlineto{\pgfqpoint{3.233166in}{1.782457in}}%
\pgfpathlineto{\pgfqpoint{3.226818in}{1.793552in}}%
\pgfpathlineto{\pgfqpoint{3.215376in}{1.791067in}}%
\pgfpathlineto{\pgfqpoint{3.203927in}{1.788069in}}%
\pgfpathlineto{\pgfqpoint{3.192470in}{1.784624in}}%
\pgfpathlineto{\pgfqpoint{3.181006in}{1.780837in}}%
\pgfpathlineto{\pgfqpoint{3.169535in}{1.776811in}}%
\pgfpathlineto{\pgfqpoint{3.175876in}{1.765785in}}%
\pgfpathlineto{\pgfqpoint{3.182219in}{1.754644in}}%
\pgfpathlineto{\pgfqpoint{3.188566in}{1.743393in}}%
\pgfpathlineto{\pgfqpoint{3.194916in}{1.732039in}}%
\pgfpathclose%
\pgfusepath{stroke,fill}%
\end{pgfscope}%
\begin{pgfscope}%
\pgfpathrectangle{\pgfqpoint{0.887500in}{0.275000in}}{\pgfqpoint{4.225000in}{4.225000in}}%
\pgfusepath{clip}%
\pgfsetbuttcap%
\pgfsetroundjoin%
\definecolor{currentfill}{rgb}{0.204903,0.375746,0.553533}%
\pgfsetfillcolor{currentfill}%
\pgfsetfillopacity{0.700000}%
\pgfsetlinewidth{0.501875pt}%
\definecolor{currentstroke}{rgb}{1.000000,1.000000,1.000000}%
\pgfsetstrokecolor{currentstroke}%
\pgfsetstrokeopacity{0.500000}%
\pgfsetdash{}{0pt}%
\pgfpathmoveto{\pgfqpoint{3.698271in}{1.810158in}}%
\pgfpathlineto{\pgfqpoint{3.709797in}{1.824910in}}%
\pgfpathlineto{\pgfqpoint{3.721317in}{1.839264in}}%
\pgfpathlineto{\pgfqpoint{3.732835in}{1.853419in}}%
\pgfpathlineto{\pgfqpoint{3.744352in}{1.867564in}}%
\pgfpathlineto{\pgfqpoint{3.755871in}{1.881727in}}%
\pgfpathlineto{\pgfqpoint{3.749527in}{1.901617in}}%
\pgfpathlineto{\pgfqpoint{3.743181in}{1.921387in}}%
\pgfpathlineto{\pgfqpoint{3.736834in}{1.941006in}}%
\pgfpathlineto{\pgfqpoint{3.730484in}{1.960447in}}%
\pgfpathlineto{\pgfqpoint{3.724132in}{1.979681in}}%
\pgfpathlineto{\pgfqpoint{3.712615in}{1.965385in}}%
\pgfpathlineto{\pgfqpoint{3.701098in}{1.951026in}}%
\pgfpathlineto{\pgfqpoint{3.689580in}{1.936596in}}%
\pgfpathlineto{\pgfqpoint{3.678058in}{1.921890in}}%
\pgfpathlineto{\pgfqpoint{3.666530in}{1.906697in}}%
\pgfpathlineto{\pgfqpoint{3.672875in}{1.887321in}}%
\pgfpathlineto{\pgfqpoint{3.679222in}{1.867936in}}%
\pgfpathlineto{\pgfqpoint{3.685569in}{1.848589in}}%
\pgfpathlineto{\pgfqpoint{3.691919in}{1.829325in}}%
\pgfpathclose%
\pgfusepath{stroke,fill}%
\end{pgfscope}%
\begin{pgfscope}%
\pgfpathrectangle{\pgfqpoint{0.887500in}{0.275000in}}{\pgfqpoint{4.225000in}{4.225000in}}%
\pgfusepath{clip}%
\pgfsetbuttcap%
\pgfsetroundjoin%
\definecolor{currentfill}{rgb}{0.246811,0.283237,0.535941}%
\pgfsetfillcolor{currentfill}%
\pgfsetfillopacity{0.700000}%
\pgfsetlinewidth{0.501875pt}%
\definecolor{currentstroke}{rgb}{1.000000,1.000000,1.000000}%
\pgfsetstrokecolor{currentstroke}%
\pgfsetstrokeopacity{0.500000}%
\pgfsetdash{}{0pt}%
\pgfpathmoveto{\pgfqpoint{3.290446in}{1.680734in}}%
\pgfpathlineto{\pgfqpoint{3.301899in}{1.684016in}}%
\pgfpathlineto{\pgfqpoint{3.313347in}{1.687330in}}%
\pgfpathlineto{\pgfqpoint{3.324790in}{1.690827in}}%
\pgfpathlineto{\pgfqpoint{3.336229in}{1.694653in}}%
\pgfpathlineto{\pgfqpoint{3.347667in}{1.698957in}}%
\pgfpathlineto{\pgfqpoint{3.341279in}{1.710027in}}%
\pgfpathlineto{\pgfqpoint{3.334894in}{1.720867in}}%
\pgfpathlineto{\pgfqpoint{3.328510in}{1.731491in}}%
\pgfpathlineto{\pgfqpoint{3.322129in}{1.741983in}}%
\pgfpathlineto{\pgfqpoint{3.315752in}{1.752441in}}%
\pgfpathlineto{\pgfqpoint{3.304326in}{1.748508in}}%
\pgfpathlineto{\pgfqpoint{3.292899in}{1.745340in}}%
\pgfpathlineto{\pgfqpoint{3.281468in}{1.742678in}}%
\pgfpathlineto{\pgfqpoint{3.270033in}{1.740260in}}%
\pgfpathlineto{\pgfqpoint{3.258592in}{1.737825in}}%
\pgfpathlineto{\pgfqpoint{3.264956in}{1.726555in}}%
\pgfpathlineto{\pgfqpoint{3.271324in}{1.715220in}}%
\pgfpathlineto{\pgfqpoint{3.277695in}{1.703810in}}%
\pgfpathlineto{\pgfqpoint{3.284069in}{1.692316in}}%
\pgfpathclose%
\pgfusepath{stroke,fill}%
\end{pgfscope}%
\begin{pgfscope}%
\pgfpathrectangle{\pgfqpoint{0.887500in}{0.275000in}}{\pgfqpoint{4.225000in}{4.225000in}}%
\pgfusepath{clip}%
\pgfsetbuttcap%
\pgfsetroundjoin%
\definecolor{currentfill}{rgb}{0.188923,0.410910,0.556326}%
\pgfsetfillcolor{currentfill}%
\pgfsetfillopacity{0.700000}%
\pgfsetlinewidth{0.501875pt}%
\definecolor{currentstroke}{rgb}{1.000000,1.000000,1.000000}%
\pgfsetstrokecolor{currentstroke}%
\pgfsetstrokeopacity{0.500000}%
\pgfsetdash{}{0pt}%
\pgfpathmoveto{\pgfqpoint{3.755871in}{1.881727in}}%
\pgfpathlineto{\pgfqpoint{3.767387in}{1.895760in}}%
\pgfpathlineto{\pgfqpoint{3.778899in}{1.909510in}}%
\pgfpathlineto{\pgfqpoint{3.790403in}{1.922822in}}%
\pgfpathlineto{\pgfqpoint{3.801896in}{1.935541in}}%
\pgfpathlineto{\pgfqpoint{3.813374in}{1.947512in}}%
\pgfpathlineto{\pgfqpoint{3.807022in}{1.967122in}}%
\pgfpathlineto{\pgfqpoint{3.800666in}{1.986514in}}%
\pgfpathlineto{\pgfqpoint{3.794306in}{2.005650in}}%
\pgfpathlineto{\pgfqpoint{3.787943in}{2.024509in}}%
\pgfpathlineto{\pgfqpoint{3.781574in}{2.043074in}}%
\pgfpathlineto{\pgfqpoint{3.770119in}{2.032170in}}%
\pgfpathlineto{\pgfqpoint{3.758642in}{2.020169in}}%
\pgfpathlineto{\pgfqpoint{3.747149in}{2.007282in}}%
\pgfpathlineto{\pgfqpoint{3.735644in}{1.993716in}}%
\pgfpathlineto{\pgfqpoint{3.724132in}{1.979681in}}%
\pgfpathlineto{\pgfqpoint{3.730484in}{1.960447in}}%
\pgfpathlineto{\pgfqpoint{3.736834in}{1.941006in}}%
\pgfpathlineto{\pgfqpoint{3.743181in}{1.921387in}}%
\pgfpathlineto{\pgfqpoint{3.749527in}{1.901617in}}%
\pgfpathclose%
\pgfusepath{stroke,fill}%
\end{pgfscope}%
\begin{pgfscope}%
\pgfpathrectangle{\pgfqpoint{0.887500in}{0.275000in}}{\pgfqpoint{4.225000in}{4.225000in}}%
\pgfusepath{clip}%
\pgfsetbuttcap%
\pgfsetroundjoin%
\definecolor{currentfill}{rgb}{0.179019,0.433756,0.557430}%
\pgfsetfillcolor{currentfill}%
\pgfsetfillopacity{0.700000}%
\pgfsetlinewidth{0.501875pt}%
\definecolor{currentstroke}{rgb}{1.000000,1.000000,1.000000}%
\pgfsetstrokecolor{currentstroke}%
\pgfsetstrokeopacity{0.500000}%
\pgfsetdash{}{0pt}%
\pgfpathmoveto{\pgfqpoint{2.283517in}{1.982536in}}%
\pgfpathlineto{\pgfqpoint{2.295221in}{1.986284in}}%
\pgfpathlineto{\pgfqpoint{2.306920in}{1.990031in}}%
\pgfpathlineto{\pgfqpoint{2.318613in}{1.993780in}}%
\pgfpathlineto{\pgfqpoint{2.330300in}{1.997534in}}%
\pgfpathlineto{\pgfqpoint{2.341981in}{2.001294in}}%
\pgfpathlineto{\pgfqpoint{2.335896in}{2.010936in}}%
\pgfpathlineto{\pgfqpoint{2.329816in}{2.020537in}}%
\pgfpathlineto{\pgfqpoint{2.323740in}{2.030098in}}%
\pgfpathlineto{\pgfqpoint{2.317669in}{2.039621in}}%
\pgfpathlineto{\pgfqpoint{2.311602in}{2.049104in}}%
\pgfpathlineto{\pgfqpoint{2.299931in}{2.045406in}}%
\pgfpathlineto{\pgfqpoint{2.288254in}{2.041715in}}%
\pgfpathlineto{\pgfqpoint{2.276571in}{2.038029in}}%
\pgfpathlineto{\pgfqpoint{2.264883in}{2.034345in}}%
\pgfpathlineto{\pgfqpoint{2.253189in}{2.030661in}}%
\pgfpathlineto{\pgfqpoint{2.259245in}{2.021116in}}%
\pgfpathlineto{\pgfqpoint{2.265307in}{2.011532in}}%
\pgfpathlineto{\pgfqpoint{2.271372in}{2.001907in}}%
\pgfpathlineto{\pgfqpoint{2.277442in}{1.992242in}}%
\pgfpathclose%
\pgfusepath{stroke,fill}%
\end{pgfscope}%
\begin{pgfscope}%
\pgfpathrectangle{\pgfqpoint{0.887500in}{0.275000in}}{\pgfqpoint{4.225000in}{4.225000in}}%
\pgfusepath{clip}%
\pgfsetbuttcap%
\pgfsetroundjoin%
\definecolor{currentfill}{rgb}{0.244972,0.287675,0.537260}%
\pgfsetfillcolor{currentfill}%
\pgfsetfillopacity{0.700000}%
\pgfsetlinewidth{0.501875pt}%
\definecolor{currentstroke}{rgb}{1.000000,1.000000,1.000000}%
\pgfsetstrokecolor{currentstroke}%
\pgfsetstrokeopacity{0.500000}%
\pgfsetdash{}{0pt}%
\pgfpathmoveto{\pgfqpoint{3.672543in}{1.649462in}}%
\pgfpathlineto{\pgfqpoint{3.684047in}{1.662639in}}%
\pgfpathlineto{\pgfqpoint{3.695551in}{1.675828in}}%
\pgfpathlineto{\pgfqpoint{3.707054in}{1.688913in}}%
\pgfpathlineto{\pgfqpoint{3.718555in}{1.701935in}}%
\pgfpathlineto{\pgfqpoint{3.730056in}{1.714946in}}%
\pgfpathlineto{\pgfqpoint{3.723697in}{1.733989in}}%
\pgfpathlineto{\pgfqpoint{3.717339in}{1.753006in}}%
\pgfpathlineto{\pgfqpoint{3.710981in}{1.772024in}}%
\pgfpathlineto{\pgfqpoint{3.704625in}{1.791066in}}%
\pgfpathlineto{\pgfqpoint{3.698271in}{1.810158in}}%
\pgfpathlineto{\pgfqpoint{3.686736in}{1.794812in}}%
\pgfpathlineto{\pgfqpoint{3.675190in}{1.778677in}}%
\pgfpathlineto{\pgfqpoint{3.663630in}{1.761555in}}%
\pgfpathlineto{\pgfqpoint{3.652054in}{1.743292in}}%
\pgfpathlineto{\pgfqpoint{3.640470in}{1.724223in}}%
\pgfpathlineto{\pgfqpoint{3.646866in}{1.708376in}}%
\pgfpathlineto{\pgfqpoint{3.653274in}{1.693156in}}%
\pgfpathlineto{\pgfqpoint{3.659691in}{1.678384in}}%
\pgfpathlineto{\pgfqpoint{3.666115in}{1.663879in}}%
\pgfpathclose%
\pgfusepath{stroke,fill}%
\end{pgfscope}%
\begin{pgfscope}%
\pgfpathrectangle{\pgfqpoint{0.887500in}{0.275000in}}{\pgfqpoint{4.225000in}{4.225000in}}%
\pgfusepath{clip}%
\pgfsetbuttcap%
\pgfsetroundjoin%
\definecolor{currentfill}{rgb}{0.201239,0.383670,0.554294}%
\pgfsetfillcolor{currentfill}%
\pgfsetfillopacity{0.700000}%
\pgfsetlinewidth{0.501875pt}%
\definecolor{currentstroke}{rgb}{1.000000,1.000000,1.000000}%
\pgfsetstrokecolor{currentstroke}%
\pgfsetstrokeopacity{0.500000}%
\pgfsetdash{}{0pt}%
\pgfpathmoveto{\pgfqpoint{2.697827in}{1.876626in}}%
\pgfpathlineto{\pgfqpoint{2.709430in}{1.880394in}}%
\pgfpathlineto{\pgfqpoint{2.721027in}{1.884166in}}%
\pgfpathlineto{\pgfqpoint{2.732618in}{1.887944in}}%
\pgfpathlineto{\pgfqpoint{2.744204in}{1.891730in}}%
\pgfpathlineto{\pgfqpoint{2.755784in}{1.895526in}}%
\pgfpathlineto{\pgfqpoint{2.749566in}{1.905804in}}%
\pgfpathlineto{\pgfqpoint{2.743351in}{1.916035in}}%
\pgfpathlineto{\pgfqpoint{2.737141in}{1.926218in}}%
\pgfpathlineto{\pgfqpoint{2.730935in}{1.936355in}}%
\pgfpathlineto{\pgfqpoint{2.724732in}{1.946446in}}%
\pgfpathlineto{\pgfqpoint{2.713162in}{1.942704in}}%
\pgfpathlineto{\pgfqpoint{2.701586in}{1.938973in}}%
\pgfpathlineto{\pgfqpoint{2.690003in}{1.935252in}}%
\pgfpathlineto{\pgfqpoint{2.678416in}{1.931537in}}%
\pgfpathlineto{\pgfqpoint{2.666822in}{1.927827in}}%
\pgfpathlineto{\pgfqpoint{2.673015in}{1.917681in}}%
\pgfpathlineto{\pgfqpoint{2.679212in}{1.907488in}}%
\pgfpathlineto{\pgfqpoint{2.685413in}{1.897248in}}%
\pgfpathlineto{\pgfqpoint{2.691618in}{1.886961in}}%
\pgfpathclose%
\pgfusepath{stroke,fill}%
\end{pgfscope}%
\begin{pgfscope}%
\pgfpathrectangle{\pgfqpoint{0.887500in}{0.275000in}}{\pgfqpoint{4.225000in}{4.225000in}}%
\pgfusepath{clip}%
\pgfsetbuttcap%
\pgfsetroundjoin%
\definecolor{currentfill}{rgb}{0.197636,0.391528,0.554969}%
\pgfsetfillcolor{currentfill}%
\pgfsetfillopacity{0.700000}%
\pgfsetlinewidth{0.501875pt}%
\definecolor{currentstroke}{rgb}{1.000000,1.000000,1.000000}%
\pgfsetstrokecolor{currentstroke}%
\pgfsetstrokeopacity{0.500000}%
\pgfsetdash{}{0pt}%
\pgfpathmoveto{\pgfqpoint{3.845126in}{1.848479in}}%
\pgfpathlineto{\pgfqpoint{3.856612in}{1.860772in}}%
\pgfpathlineto{\pgfqpoint{3.868085in}{1.872504in}}%
\pgfpathlineto{\pgfqpoint{3.879545in}{1.883667in}}%
\pgfpathlineto{\pgfqpoint{3.890992in}{1.894257in}}%
\pgfpathlineto{\pgfqpoint{3.902425in}{1.904270in}}%
\pgfpathlineto{\pgfqpoint{3.896038in}{1.922504in}}%
\pgfpathlineto{\pgfqpoint{3.889653in}{1.940747in}}%
\pgfpathlineto{\pgfqpoint{3.883268in}{1.958943in}}%
\pgfpathlineto{\pgfqpoint{3.876882in}{1.977037in}}%
\pgfpathlineto{\pgfqpoint{3.870494in}{1.994974in}}%
\pgfpathlineto{\pgfqpoint{3.859102in}{1.986860in}}%
\pgfpathlineto{\pgfqpoint{3.847696in}{1.978161in}}%
\pgfpathlineto{\pgfqpoint{3.836274in}{1.968772in}}%
\pgfpathlineto{\pgfqpoint{3.824833in}{1.958589in}}%
\pgfpathlineto{\pgfqpoint{3.813374in}{1.947512in}}%
\pgfpathlineto{\pgfqpoint{3.819724in}{1.927748in}}%
\pgfpathlineto{\pgfqpoint{3.826073in}{1.907898in}}%
\pgfpathlineto{\pgfqpoint{3.832422in}{1.888025in}}%
\pgfpathlineto{\pgfqpoint{3.838773in}{1.868197in}}%
\pgfpathclose%
\pgfusepath{stroke,fill}%
\end{pgfscope}%
\begin{pgfscope}%
\pgfpathrectangle{\pgfqpoint{0.887500in}{0.275000in}}{\pgfqpoint{4.225000in}{4.225000in}}%
\pgfusepath{clip}%
\pgfsetbuttcap%
\pgfsetroundjoin%
\definecolor{currentfill}{rgb}{0.206756,0.371758,0.553117}%
\pgfsetfillcolor{currentfill}%
\pgfsetfillopacity{0.700000}%
\pgfsetlinewidth{0.501875pt}%
\definecolor{currentstroke}{rgb}{1.000000,1.000000,1.000000}%
\pgfsetstrokecolor{currentstroke}%
\pgfsetstrokeopacity{0.500000}%
\pgfsetdash{}{0pt}%
\pgfpathmoveto{\pgfqpoint{3.934424in}{1.815131in}}%
\pgfpathlineto{\pgfqpoint{3.945870in}{1.825682in}}%
\pgfpathlineto{\pgfqpoint{3.957301in}{1.835609in}}%
\pgfpathlineto{\pgfqpoint{3.968714in}{1.844894in}}%
\pgfpathlineto{\pgfqpoint{3.980110in}{1.853573in}}%
\pgfpathlineto{\pgfqpoint{3.991491in}{1.861705in}}%
\pgfpathlineto{\pgfqpoint{3.985054in}{1.878162in}}%
\pgfpathlineto{\pgfqpoint{3.978623in}{1.894834in}}%
\pgfpathlineto{\pgfqpoint{3.972198in}{1.911682in}}%
\pgfpathlineto{\pgfqpoint{3.965778in}{1.928653in}}%
\pgfpathlineto{\pgfqpoint{3.959362in}{1.945694in}}%
\pgfpathlineto{\pgfqpoint{3.948005in}{1.938495in}}%
\pgfpathlineto{\pgfqpoint{3.936633in}{1.930795in}}%
\pgfpathlineto{\pgfqpoint{3.925246in}{1.922541in}}%
\pgfpathlineto{\pgfqpoint{3.913843in}{1.913699in}}%
\pgfpathlineto{\pgfqpoint{3.902425in}{1.904270in}}%
\pgfpathlineto{\pgfqpoint{3.908815in}{1.886099in}}%
\pgfpathlineto{\pgfqpoint{3.915208in}{1.868047in}}%
\pgfpathlineto{\pgfqpoint{3.921607in}{1.850167in}}%
\pgfpathlineto{\pgfqpoint{3.928012in}{1.832516in}}%
\pgfpathclose%
\pgfusepath{stroke,fill}%
\end{pgfscope}%
\begin{pgfscope}%
\pgfpathrectangle{\pgfqpoint{0.887500in}{0.275000in}}{\pgfqpoint{4.225000in}{4.225000in}}%
\pgfusepath{clip}%
\pgfsetbuttcap%
\pgfsetroundjoin%
\definecolor{currentfill}{rgb}{0.216210,0.351535,0.550627}%
\pgfsetfillcolor{currentfill}%
\pgfsetfillopacity{0.700000}%
\pgfsetlinewidth{0.501875pt}%
\definecolor{currentstroke}{rgb}{1.000000,1.000000,1.000000}%
\pgfsetstrokecolor{currentstroke}%
\pgfsetstrokeopacity{0.500000}%
\pgfsetdash{}{0pt}%
\pgfpathmoveto{\pgfqpoint{4.023744in}{1.780870in}}%
\pgfpathlineto{\pgfqpoint{4.035150in}{1.789981in}}%
\pgfpathlineto{\pgfqpoint{4.046546in}{1.798808in}}%
\pgfpathlineto{\pgfqpoint{4.057933in}{1.807363in}}%
\pgfpathlineto{\pgfqpoint{4.069311in}{1.815661in}}%
\pgfpathlineto{\pgfqpoint{4.080679in}{1.823708in}}%
\pgfpathlineto{\pgfqpoint{4.074173in}{1.838057in}}%
\pgfpathlineto{\pgfqpoint{4.067669in}{1.852391in}}%
\pgfpathlineto{\pgfqpoint{4.061170in}{1.866805in}}%
\pgfpathlineto{\pgfqpoint{4.054678in}{1.881395in}}%
\pgfpathlineto{\pgfqpoint{4.048196in}{1.896256in}}%
\pgfpathlineto{\pgfqpoint{4.036878in}{1.889963in}}%
\pgfpathlineto{\pgfqpoint{4.025549in}{1.883419in}}%
\pgfpathlineto{\pgfqpoint{4.014209in}{1.876567in}}%
\pgfpathlineto{\pgfqpoint{4.002857in}{1.869349in}}%
\pgfpathlineto{\pgfqpoint{3.991491in}{1.861705in}}%
\pgfpathlineto{\pgfqpoint{3.997935in}{1.845423in}}%
\pgfpathlineto{\pgfqpoint{4.004383in}{1.829258in}}%
\pgfpathlineto{\pgfqpoint{4.010835in}{1.813151in}}%
\pgfpathlineto{\pgfqpoint{4.017289in}{1.797041in}}%
\pgfpathclose%
\pgfusepath{stroke,fill}%
\end{pgfscope}%
\begin{pgfscope}%
\pgfpathrectangle{\pgfqpoint{0.887500in}{0.275000in}}{\pgfqpoint{4.225000in}{4.225000in}}%
\pgfusepath{clip}%
\pgfsetbuttcap%
\pgfsetroundjoin%
\definecolor{currentfill}{rgb}{0.246811,0.283237,0.535941}%
\pgfsetfillcolor{currentfill}%
\pgfsetfillopacity{0.700000}%
\pgfsetlinewidth{0.501875pt}%
\definecolor{currentstroke}{rgb}{1.000000,1.000000,1.000000}%
\pgfsetstrokecolor{currentstroke}%
\pgfsetstrokeopacity{0.500000}%
\pgfsetdash{}{0pt}%
\pgfpathmoveto{\pgfqpoint{4.291003in}{1.648595in}}%
\pgfpathlineto{\pgfqpoint{4.302407in}{1.658569in}}%
\pgfpathlineto{\pgfqpoint{4.313782in}{1.667717in}}%
\pgfpathlineto{\pgfqpoint{4.325120in}{1.675816in}}%
\pgfpathlineto{\pgfqpoint{4.336417in}{1.682754in}}%
\pgfpathlineto{\pgfqpoint{4.347677in}{1.688670in}}%
\pgfpathlineto{\pgfqpoint{4.341199in}{1.705427in}}%
\pgfpathlineto{\pgfqpoint{4.334706in}{1.721635in}}%
\pgfpathlineto{\pgfqpoint{4.328197in}{1.737337in}}%
\pgfpathlineto{\pgfqpoint{4.321676in}{1.752577in}}%
\pgfpathlineto{\pgfqpoint{4.315144in}{1.767399in}}%
\pgfpathlineto{\pgfqpoint{4.303946in}{1.763332in}}%
\pgfpathlineto{\pgfqpoint{4.292734in}{1.758984in}}%
\pgfpathlineto{\pgfqpoint{4.281506in}{1.754314in}}%
\pgfpathlineto{\pgfqpoint{4.270262in}{1.749272in}}%
\pgfpathlineto{\pgfqpoint{4.259000in}{1.743804in}}%
\pgfpathlineto{\pgfqpoint{4.265462in}{1.726774in}}%
\pgfpathlineto{\pgfqpoint{4.271896in}{1.708824in}}%
\pgfpathlineto{\pgfqpoint{4.278299in}{1.689867in}}%
\pgfpathlineto{\pgfqpoint{4.284670in}{1.669819in}}%
\pgfpathclose%
\pgfusepath{stroke,fill}%
\end{pgfscope}%
\begin{pgfscope}%
\pgfpathrectangle{\pgfqpoint{0.887500in}{0.275000in}}{\pgfqpoint{4.225000in}{4.225000in}}%
\pgfusepath{clip}%
\pgfsetbuttcap%
\pgfsetroundjoin%
\definecolor{currentfill}{rgb}{0.165117,0.467423,0.558141}%
\pgfsetfillcolor{currentfill}%
\pgfsetfillopacity{0.700000}%
\pgfsetlinewidth{0.501875pt}%
\definecolor{currentstroke}{rgb}{1.000000,1.000000,1.000000}%
\pgfsetstrokecolor{currentstroke}%
\pgfsetstrokeopacity{0.500000}%
\pgfsetdash{}{0pt}%
\pgfpathmoveto{\pgfqpoint{1.958219in}{2.050979in}}%
\pgfpathlineto{\pgfqpoint{1.970005in}{2.054693in}}%
\pgfpathlineto{\pgfqpoint{1.981785in}{2.058405in}}%
\pgfpathlineto{\pgfqpoint{1.993560in}{2.062116in}}%
\pgfpathlineto{\pgfqpoint{2.005328in}{2.065827in}}%
\pgfpathlineto{\pgfqpoint{2.017091in}{2.069539in}}%
\pgfpathlineto{\pgfqpoint{2.011115in}{2.078875in}}%
\pgfpathlineto{\pgfqpoint{2.005143in}{2.088179in}}%
\pgfpathlineto{\pgfqpoint{1.999175in}{2.097451in}}%
\pgfpathlineto{\pgfqpoint{1.993213in}{2.106692in}}%
\pgfpathlineto{\pgfqpoint{1.987254in}{2.115903in}}%
\pgfpathlineto{\pgfqpoint{1.975502in}{2.112252in}}%
\pgfpathlineto{\pgfqpoint{1.963743in}{2.108601in}}%
\pgfpathlineto{\pgfqpoint{1.951979in}{2.104950in}}%
\pgfpathlineto{\pgfqpoint{1.940209in}{2.101298in}}%
\pgfpathlineto{\pgfqpoint{1.928433in}{2.097643in}}%
\pgfpathlineto{\pgfqpoint{1.934381in}{2.088373in}}%
\pgfpathlineto{\pgfqpoint{1.940334in}{2.079072in}}%
\pgfpathlineto{\pgfqpoint{1.946291in}{2.069740in}}%
\pgfpathlineto{\pgfqpoint{1.952253in}{2.060375in}}%
\pgfpathclose%
\pgfusepath{stroke,fill}%
\end{pgfscope}%
\begin{pgfscope}%
\pgfpathrectangle{\pgfqpoint{0.887500in}{0.275000in}}{\pgfqpoint{4.225000in}{4.225000in}}%
\pgfusepath{clip}%
\pgfsetbuttcap%
\pgfsetroundjoin%
\definecolor{currentfill}{rgb}{0.263663,0.237631,0.518762}%
\pgfsetfillcolor{currentfill}%
\pgfsetfillopacity{0.700000}%
\pgfsetlinewidth{0.501875pt}%
\definecolor{currentstroke}{rgb}{1.000000,1.000000,1.000000}%
\pgfsetstrokecolor{currentstroke}%
\pgfsetstrokeopacity{0.500000}%
\pgfsetdash{}{0pt}%
\pgfpathmoveto{\pgfqpoint{3.468815in}{1.598532in}}%
\pgfpathlineto{\pgfqpoint{3.480227in}{1.602172in}}%
\pgfpathlineto{\pgfqpoint{3.491628in}{1.605340in}}%
\pgfpathlineto{\pgfqpoint{3.503016in}{1.607899in}}%
\pgfpathlineto{\pgfqpoint{3.514391in}{1.609860in}}%
\pgfpathlineto{\pgfqpoint{3.525758in}{1.611714in}}%
\pgfpathlineto{\pgfqpoint{3.519341in}{1.624461in}}%
\pgfpathlineto{\pgfqpoint{3.512940in}{1.638362in}}%
\pgfpathlineto{\pgfqpoint{3.506556in}{1.653406in}}%
\pgfpathlineto{\pgfqpoint{3.500184in}{1.669424in}}%
\pgfpathlineto{\pgfqpoint{3.493824in}{1.686240in}}%
\pgfpathlineto{\pgfqpoint{3.482436in}{1.682426in}}%
\pgfpathlineto{\pgfqpoint{3.471043in}{1.678641in}}%
\pgfpathlineto{\pgfqpoint{3.459641in}{1.674433in}}%
\pgfpathlineto{\pgfqpoint{3.448228in}{1.669800in}}%
\pgfpathlineto{\pgfqpoint{3.436808in}{1.664875in}}%
\pgfpathlineto{\pgfqpoint{3.443202in}{1.651433in}}%
\pgfpathlineto{\pgfqpoint{3.449599in}{1.638027in}}%
\pgfpathlineto{\pgfqpoint{3.456000in}{1.624712in}}%
\pgfpathlineto{\pgfqpoint{3.462405in}{1.611541in}}%
\pgfpathclose%
\pgfusepath{stroke,fill}%
\end{pgfscope}%
\begin{pgfscope}%
\pgfpathrectangle{\pgfqpoint{0.887500in}{0.275000in}}{\pgfqpoint{4.225000in}{4.225000in}}%
\pgfusepath{clip}%
\pgfsetbuttcap%
\pgfsetroundjoin%
\definecolor{currentfill}{rgb}{0.253935,0.265254,0.529983}%
\pgfsetfillcolor{currentfill}%
\pgfsetfillopacity{0.700000}%
\pgfsetlinewidth{0.501875pt}%
\definecolor{currentstroke}{rgb}{1.000000,1.000000,1.000000}%
\pgfsetstrokecolor{currentstroke}%
\pgfsetstrokeopacity{0.500000}%
\pgfsetdash{}{0pt}%
\pgfpathmoveto{\pgfqpoint{3.379634in}{1.640539in}}%
\pgfpathlineto{\pgfqpoint{3.391075in}{1.644948in}}%
\pgfpathlineto{\pgfqpoint{3.402513in}{1.649695in}}%
\pgfpathlineto{\pgfqpoint{3.413949in}{1.654688in}}%
\pgfpathlineto{\pgfqpoint{3.425381in}{1.659793in}}%
\pgfpathlineto{\pgfqpoint{3.436808in}{1.664875in}}%
\pgfpathlineto{\pgfqpoint{3.430416in}{1.678297in}}%
\pgfpathlineto{\pgfqpoint{3.424027in}{1.691646in}}%
\pgfpathlineto{\pgfqpoint{3.417639in}{1.704866in}}%
\pgfpathlineto{\pgfqpoint{3.411253in}{1.717903in}}%
\pgfpathlineto{\pgfqpoint{3.404868in}{1.730701in}}%
\pgfpathlineto{\pgfqpoint{3.393424in}{1.723033in}}%
\pgfpathlineto{\pgfqpoint{3.381982in}{1.715967in}}%
\pgfpathlineto{\pgfqpoint{3.370543in}{1.709564in}}%
\pgfpathlineto{\pgfqpoint{3.359104in}{1.703887in}}%
\pgfpathlineto{\pgfqpoint{3.347667in}{1.698957in}}%
\pgfpathlineto{\pgfqpoint{3.354057in}{1.687668in}}%
\pgfpathlineto{\pgfqpoint{3.360448in}{1.676171in}}%
\pgfpathlineto{\pgfqpoint{3.366842in}{1.664476in}}%
\pgfpathlineto{\pgfqpoint{3.373237in}{1.652596in}}%
\pgfpathclose%
\pgfusepath{stroke,fill}%
\end{pgfscope}%
\begin{pgfscope}%
\pgfpathrectangle{\pgfqpoint{0.887500in}{0.275000in}}{\pgfqpoint{4.225000in}{4.225000in}}%
\pgfusepath{clip}%
\pgfsetbuttcap%
\pgfsetroundjoin%
\definecolor{currentfill}{rgb}{0.227802,0.326594,0.546532}%
\pgfsetfillcolor{currentfill}%
\pgfsetfillopacity{0.700000}%
\pgfsetlinewidth{0.501875pt}%
\definecolor{currentstroke}{rgb}{1.000000,1.000000,1.000000}%
\pgfsetstrokecolor{currentstroke}%
\pgfsetstrokeopacity{0.500000}%
\pgfsetdash{}{0pt}%
\pgfpathmoveto{\pgfqpoint{3.730056in}{1.714946in}}%
\pgfpathlineto{\pgfqpoint{3.741558in}{1.727999in}}%
\pgfpathlineto{\pgfqpoint{3.753061in}{1.741147in}}%
\pgfpathlineto{\pgfqpoint{3.764567in}{1.754442in}}%
\pgfpathlineto{\pgfqpoint{3.776076in}{1.767934in}}%
\pgfpathlineto{\pgfqpoint{3.787589in}{1.781595in}}%
\pgfpathlineto{\pgfqpoint{3.781243in}{1.801605in}}%
\pgfpathlineto{\pgfqpoint{3.774900in}{1.821655in}}%
\pgfpathlineto{\pgfqpoint{3.768557in}{1.841714in}}%
\pgfpathlineto{\pgfqpoint{3.762214in}{1.861748in}}%
\pgfpathlineto{\pgfqpoint{3.755871in}{1.881727in}}%
\pgfpathlineto{\pgfqpoint{3.744352in}{1.867564in}}%
\pgfpathlineto{\pgfqpoint{3.732835in}{1.853419in}}%
\pgfpathlineto{\pgfqpoint{3.721317in}{1.839264in}}%
\pgfpathlineto{\pgfqpoint{3.709797in}{1.824910in}}%
\pgfpathlineto{\pgfqpoint{3.698271in}{1.810158in}}%
\pgfpathlineto{\pgfqpoint{3.704625in}{1.791066in}}%
\pgfpathlineto{\pgfqpoint{3.710981in}{1.772024in}}%
\pgfpathlineto{\pgfqpoint{3.717339in}{1.753006in}}%
\pgfpathlineto{\pgfqpoint{3.723697in}{1.733989in}}%
\pgfpathclose%
\pgfusepath{stroke,fill}%
\end{pgfscope}%
\begin{pgfscope}%
\pgfpathrectangle{\pgfqpoint{0.887500in}{0.275000in}}{\pgfqpoint{4.225000in}{4.225000in}}%
\pgfusepath{clip}%
\pgfsetbuttcap%
\pgfsetroundjoin%
\definecolor{currentfill}{rgb}{0.270595,0.214069,0.507052}%
\pgfsetfillcolor{currentfill}%
\pgfsetfillopacity{0.700000}%
\pgfsetlinewidth{0.501875pt}%
\definecolor{currentstroke}{rgb}{1.000000,1.000000,1.000000}%
\pgfsetstrokecolor{currentstroke}%
\pgfsetstrokeopacity{0.500000}%
\pgfsetdash{}{0pt}%
\pgfpathmoveto{\pgfqpoint{3.558027in}{1.558591in}}%
\pgfpathlineto{\pgfqpoint{3.569437in}{1.563633in}}%
\pgfpathlineto{\pgfqpoint{3.580848in}{1.569220in}}%
\pgfpathlineto{\pgfqpoint{3.592266in}{1.575570in}}%
\pgfpathlineto{\pgfqpoint{3.603693in}{1.582902in}}%
\pgfpathlineto{\pgfqpoint{3.615134in}{1.591435in}}%
\pgfpathlineto{\pgfqpoint{3.608631in}{1.599908in}}%
\pgfpathlineto{\pgfqpoint{3.602137in}{1.608653in}}%
\pgfpathlineto{\pgfqpoint{3.595655in}{1.618011in}}%
\pgfpathlineto{\pgfqpoint{3.589191in}{1.628323in}}%
\pgfpathlineto{\pgfqpoint{3.582748in}{1.639930in}}%
\pgfpathlineto{\pgfqpoint{3.571300in}{1.629805in}}%
\pgfpathlineto{\pgfqpoint{3.559886in}{1.622507in}}%
\pgfpathlineto{\pgfqpoint{3.548497in}{1.617451in}}%
\pgfpathlineto{\pgfqpoint{3.537124in}{1.614049in}}%
\pgfpathlineto{\pgfqpoint{3.525758in}{1.611714in}}%
\pgfpathlineto{\pgfqpoint{3.532190in}{1.599937in}}%
\pgfpathlineto{\pgfqpoint{3.538635in}{1.588932in}}%
\pgfpathlineto{\pgfqpoint{3.545091in}{1.578504in}}%
\pgfpathlineto{\pgfqpoint{3.551556in}{1.568456in}}%
\pgfpathclose%
\pgfusepath{stroke,fill}%
\end{pgfscope}%
\begin{pgfscope}%
\pgfpathrectangle{\pgfqpoint{0.887500in}{0.275000in}}{\pgfqpoint{4.225000in}{4.225000in}}%
\pgfusepath{clip}%
\pgfsetbuttcap%
\pgfsetroundjoin%
\definecolor{currentfill}{rgb}{0.210503,0.363727,0.552206}%
\pgfsetfillcolor{currentfill}%
\pgfsetfillopacity{0.700000}%
\pgfsetlinewidth{0.501875pt}%
\definecolor{currentstroke}{rgb}{1.000000,1.000000,1.000000}%
\pgfsetstrokecolor{currentstroke}%
\pgfsetstrokeopacity{0.500000}%
\pgfsetdash{}{0pt}%
\pgfpathmoveto{\pgfqpoint{3.787589in}{1.781595in}}%
\pgfpathlineto{\pgfqpoint{3.799103in}{1.795314in}}%
\pgfpathlineto{\pgfqpoint{3.810616in}{1.808977in}}%
\pgfpathlineto{\pgfqpoint{3.822126in}{1.822468in}}%
\pgfpathlineto{\pgfqpoint{3.833630in}{1.835674in}}%
\pgfpathlineto{\pgfqpoint{3.845126in}{1.848479in}}%
\pgfpathlineto{\pgfqpoint{3.838773in}{1.868197in}}%
\pgfpathlineto{\pgfqpoint{3.832422in}{1.888025in}}%
\pgfpathlineto{\pgfqpoint{3.826073in}{1.907898in}}%
\pgfpathlineto{\pgfqpoint{3.819724in}{1.927748in}}%
\pgfpathlineto{\pgfqpoint{3.813374in}{1.947512in}}%
\pgfpathlineto{\pgfqpoint{3.801896in}{1.935541in}}%
\pgfpathlineto{\pgfqpoint{3.790403in}{1.922822in}}%
\pgfpathlineto{\pgfqpoint{3.778899in}{1.909510in}}%
\pgfpathlineto{\pgfqpoint{3.767387in}{1.895760in}}%
\pgfpathlineto{\pgfqpoint{3.755871in}{1.881727in}}%
\pgfpathlineto{\pgfqpoint{3.762214in}{1.861748in}}%
\pgfpathlineto{\pgfqpoint{3.768557in}{1.841714in}}%
\pgfpathlineto{\pgfqpoint{3.774900in}{1.821655in}}%
\pgfpathlineto{\pgfqpoint{3.781243in}{1.801605in}}%
\pgfpathclose%
\pgfusepath{stroke,fill}%
\end{pgfscope}%
\begin{pgfscope}%
\pgfpathrectangle{\pgfqpoint{0.887500in}{0.275000in}}{\pgfqpoint{4.225000in}{4.225000in}}%
\pgfusepath{clip}%
\pgfsetbuttcap%
\pgfsetroundjoin%
\definecolor{currentfill}{rgb}{0.206756,0.371758,0.553117}%
\pgfsetfillcolor{currentfill}%
\pgfsetfillopacity{0.700000}%
\pgfsetlinewidth{0.501875pt}%
\definecolor{currentstroke}{rgb}{1.000000,1.000000,1.000000}%
\pgfsetstrokecolor{currentstroke}%
\pgfsetstrokeopacity{0.500000}%
\pgfsetdash{}{0pt}%
\pgfpathmoveto{\pgfqpoint{2.786936in}{1.843397in}}%
\pgfpathlineto{\pgfqpoint{2.798520in}{1.847257in}}%
\pgfpathlineto{\pgfqpoint{2.810097in}{1.851127in}}%
\pgfpathlineto{\pgfqpoint{2.821669in}{1.855003in}}%
\pgfpathlineto{\pgfqpoint{2.833236in}{1.858869in}}%
\pgfpathlineto{\pgfqpoint{2.844797in}{1.862710in}}%
\pgfpathlineto{\pgfqpoint{2.838550in}{1.873187in}}%
\pgfpathlineto{\pgfqpoint{2.832306in}{1.883613in}}%
\pgfpathlineto{\pgfqpoint{2.826066in}{1.893989in}}%
\pgfpathlineto{\pgfqpoint{2.819831in}{1.904316in}}%
\pgfpathlineto{\pgfqpoint{2.813599in}{1.914593in}}%
\pgfpathlineto{\pgfqpoint{2.802047in}{1.910803in}}%
\pgfpathlineto{\pgfqpoint{2.790490in}{1.906983in}}%
\pgfpathlineto{\pgfqpoint{2.778927in}{1.903154in}}%
\pgfpathlineto{\pgfqpoint{2.767358in}{1.899333in}}%
\pgfpathlineto{\pgfqpoint{2.755784in}{1.895526in}}%
\pgfpathlineto{\pgfqpoint{2.762007in}{1.885199in}}%
\pgfpathlineto{\pgfqpoint{2.768233in}{1.874823in}}%
\pgfpathlineto{\pgfqpoint{2.774463in}{1.864398in}}%
\pgfpathlineto{\pgfqpoint{2.780698in}{1.853923in}}%
\pgfpathclose%
\pgfusepath{stroke,fill}%
\end{pgfscope}%
\begin{pgfscope}%
\pgfpathrectangle{\pgfqpoint{0.887500in}{0.275000in}}{\pgfqpoint{4.225000in}{4.225000in}}%
\pgfusepath{clip}%
\pgfsetbuttcap%
\pgfsetroundjoin%
\definecolor{currentfill}{rgb}{0.223925,0.334994,0.548053}%
\pgfsetfillcolor{currentfill}%
\pgfsetfillopacity{0.700000}%
\pgfsetlinewidth{0.501875pt}%
\definecolor{currentstroke}{rgb}{1.000000,1.000000,1.000000}%
\pgfsetstrokecolor{currentstroke}%
\pgfsetstrokeopacity{0.500000}%
\pgfsetdash{}{0pt}%
\pgfpathmoveto{\pgfqpoint{4.113152in}{1.748421in}}%
\pgfpathlineto{\pgfqpoint{4.124565in}{1.758038in}}%
\pgfpathlineto{\pgfqpoint{4.135966in}{1.767258in}}%
\pgfpathlineto{\pgfqpoint{4.147350in}{1.775987in}}%
\pgfpathlineto{\pgfqpoint{4.158715in}{1.784135in}}%
\pgfpathlineto{\pgfqpoint{4.170057in}{1.791610in}}%
\pgfpathlineto{\pgfqpoint{4.163530in}{1.805866in}}%
\pgfpathlineto{\pgfqpoint{4.156989in}{1.819590in}}%
\pgfpathlineto{\pgfqpoint{4.150440in}{1.832897in}}%
\pgfpathlineto{\pgfqpoint{4.143886in}{1.845907in}}%
\pgfpathlineto{\pgfqpoint{4.137330in}{1.858735in}}%
\pgfpathlineto{\pgfqpoint{4.126030in}{1.852581in}}%
\pgfpathlineto{\pgfqpoint{4.114713in}{1.845954in}}%
\pgfpathlineto{\pgfqpoint{4.103382in}{1.838901in}}%
\pgfpathlineto{\pgfqpoint{4.092036in}{1.831471in}}%
\pgfpathlineto{\pgfqpoint{4.080679in}{1.823708in}}%
\pgfpathlineto{\pgfqpoint{4.087184in}{1.809250in}}%
\pgfpathlineto{\pgfqpoint{4.093687in}{1.794587in}}%
\pgfpathlineto{\pgfqpoint{4.100185in}{1.779624in}}%
\pgfpathlineto{\pgfqpoint{4.106674in}{1.764267in}}%
\pgfpathclose%
\pgfusepath{stroke,fill}%
\end{pgfscope}%
\begin{pgfscope}%
\pgfpathrectangle{\pgfqpoint{0.887500in}{0.275000in}}{\pgfqpoint{4.225000in}{4.225000in}}%
\pgfusepath{clip}%
\pgfsetbuttcap%
\pgfsetroundjoin%
\definecolor{currentfill}{rgb}{0.183898,0.422383,0.556944}%
\pgfsetfillcolor{currentfill}%
\pgfsetfillopacity{0.700000}%
\pgfsetlinewidth{0.501875pt}%
\definecolor{currentstroke}{rgb}{1.000000,1.000000,1.000000}%
\pgfsetstrokecolor{currentstroke}%
\pgfsetstrokeopacity{0.500000}%
\pgfsetdash{}{0pt}%
\pgfpathmoveto{\pgfqpoint{2.372470in}{1.952444in}}%
\pgfpathlineto{\pgfqpoint{2.384156in}{1.956272in}}%
\pgfpathlineto{\pgfqpoint{2.395835in}{1.960102in}}%
\pgfpathlineto{\pgfqpoint{2.407509in}{1.963930in}}%
\pgfpathlineto{\pgfqpoint{2.419177in}{1.967752in}}%
\pgfpathlineto{\pgfqpoint{2.430839in}{1.971565in}}%
\pgfpathlineto{\pgfqpoint{2.424723in}{1.981363in}}%
\pgfpathlineto{\pgfqpoint{2.418610in}{1.991116in}}%
\pgfpathlineto{\pgfqpoint{2.412503in}{2.000824in}}%
\pgfpathlineto{\pgfqpoint{2.406399in}{2.010488in}}%
\pgfpathlineto{\pgfqpoint{2.400300in}{2.020110in}}%
\pgfpathlineto{\pgfqpoint{2.388647in}{2.016356in}}%
\pgfpathlineto{\pgfqpoint{2.376989in}{2.012594in}}%
\pgfpathlineto{\pgfqpoint{2.365325in}{2.008827in}}%
\pgfpathlineto{\pgfqpoint{2.353656in}{2.005059in}}%
\pgfpathlineto{\pgfqpoint{2.341981in}{2.001294in}}%
\pgfpathlineto{\pgfqpoint{2.348070in}{1.991610in}}%
\pgfpathlineto{\pgfqpoint{2.354163in}{1.981884in}}%
\pgfpathlineto{\pgfqpoint{2.360261in}{1.972115in}}%
\pgfpathlineto{\pgfqpoint{2.366364in}{1.962302in}}%
\pgfpathclose%
\pgfusepath{stroke,fill}%
\end{pgfscope}%
\begin{pgfscope}%
\pgfpathrectangle{\pgfqpoint{0.887500in}{0.275000in}}{\pgfqpoint{4.225000in}{4.225000in}}%
\pgfusepath{clip}%
\pgfsetbuttcap%
\pgfsetroundjoin%
\definecolor{currentfill}{rgb}{0.233603,0.313828,0.543914}%
\pgfsetfillcolor{currentfill}%
\pgfsetfillopacity{0.700000}%
\pgfsetlinewidth{0.501875pt}%
\definecolor{currentstroke}{rgb}{1.000000,1.000000,1.000000}%
\pgfsetstrokecolor{currentstroke}%
\pgfsetstrokeopacity{0.500000}%
\pgfsetdash{}{0pt}%
\pgfpathmoveto{\pgfqpoint{4.202381in}{1.708226in}}%
\pgfpathlineto{\pgfqpoint{4.213749in}{1.716606in}}%
\pgfpathlineto{\pgfqpoint{4.225095in}{1.724312in}}%
\pgfpathlineto{\pgfqpoint{4.236418in}{1.731378in}}%
\pgfpathlineto{\pgfqpoint{4.247719in}{1.737857in}}%
\pgfpathlineto{\pgfqpoint{4.259000in}{1.743804in}}%
\pgfpathlineto{\pgfqpoint{4.252515in}{1.759999in}}%
\pgfpathlineto{\pgfqpoint{4.246009in}{1.775446in}}%
\pgfpathlineto{\pgfqpoint{4.239486in}{1.790230in}}%
\pgfpathlineto{\pgfqpoint{4.232947in}{1.804438in}}%
\pgfpathlineto{\pgfqpoint{4.226396in}{1.818157in}}%
\pgfpathlineto{\pgfqpoint{4.215172in}{1.813997in}}%
\pgfpathlineto{\pgfqpoint{4.203929in}{1.809394in}}%
\pgfpathlineto{\pgfqpoint{4.192665in}{1.804214in}}%
\pgfpathlineto{\pgfqpoint{4.181375in}{1.798319in}}%
\pgfpathlineto{\pgfqpoint{4.170057in}{1.791610in}}%
\pgfpathlineto{\pgfqpoint{4.176569in}{1.776703in}}%
\pgfpathlineto{\pgfqpoint{4.183061in}{1.761027in}}%
\pgfpathlineto{\pgfqpoint{4.189530in}{1.744467in}}%
\pgfpathlineto{\pgfqpoint{4.195971in}{1.726906in}}%
\pgfpathclose%
\pgfusepath{stroke,fill}%
\end{pgfscope}%
\begin{pgfscope}%
\pgfpathrectangle{\pgfqpoint{0.887500in}{0.275000in}}{\pgfqpoint{4.225000in}{4.225000in}}%
\pgfusepath{clip}%
\pgfsetbuttcap%
\pgfsetroundjoin%
\definecolor{currentfill}{rgb}{0.218130,0.347432,0.550038}%
\pgfsetfillcolor{currentfill}%
\pgfsetfillopacity{0.700000}%
\pgfsetlinewidth{0.501875pt}%
\definecolor{currentstroke}{rgb}{1.000000,1.000000,1.000000}%
\pgfsetstrokecolor{currentstroke}%
\pgfsetstrokeopacity{0.500000}%
\pgfsetdash{}{0pt}%
\pgfpathmoveto{\pgfqpoint{3.876989in}{1.753802in}}%
\pgfpathlineto{\pgfqpoint{3.888500in}{1.767135in}}%
\pgfpathlineto{\pgfqpoint{3.900000in}{1.779957in}}%
\pgfpathlineto{\pgfqpoint{3.911488in}{1.792246in}}%
\pgfpathlineto{\pgfqpoint{3.922963in}{1.803978in}}%
\pgfpathlineto{\pgfqpoint{3.934424in}{1.815131in}}%
\pgfpathlineto{\pgfqpoint{3.928012in}{1.832516in}}%
\pgfpathlineto{\pgfqpoint{3.921607in}{1.850167in}}%
\pgfpathlineto{\pgfqpoint{3.915208in}{1.868047in}}%
\pgfpathlineto{\pgfqpoint{3.908815in}{1.886099in}}%
\pgfpathlineto{\pgfqpoint{3.902425in}{1.904270in}}%
\pgfpathlineto{\pgfqpoint{3.890992in}{1.894257in}}%
\pgfpathlineto{\pgfqpoint{3.879545in}{1.883667in}}%
\pgfpathlineto{\pgfqpoint{3.868085in}{1.872504in}}%
\pgfpathlineto{\pgfqpoint{3.856612in}{1.860772in}}%
\pgfpathlineto{\pgfqpoint{3.845126in}{1.848479in}}%
\pgfpathlineto{\pgfqpoint{3.851484in}{1.828936in}}%
\pgfpathlineto{\pgfqpoint{3.857848in}{1.809634in}}%
\pgfpathlineto{\pgfqpoint{3.864219in}{1.790637in}}%
\pgfpathlineto{\pgfqpoint{3.870599in}{1.772011in}}%
\pgfpathclose%
\pgfusepath{stroke,fill}%
\end{pgfscope}%
\begin{pgfscope}%
\pgfpathrectangle{\pgfqpoint{0.887500in}{0.275000in}}{\pgfqpoint{4.225000in}{4.225000in}}%
\pgfusepath{clip}%
\pgfsetbuttcap%
\pgfsetroundjoin%
\definecolor{currentfill}{rgb}{0.250425,0.274290,0.533103}%
\pgfsetfillcolor{currentfill}%
\pgfsetfillopacity{0.700000}%
\pgfsetlinewidth{0.501875pt}%
\definecolor{currentstroke}{rgb}{1.000000,1.000000,1.000000}%
\pgfsetstrokecolor{currentstroke}%
\pgfsetstrokeopacity{0.500000}%
\pgfsetdash{}{0pt}%
\pgfpathmoveto{\pgfqpoint{3.761844in}{1.618513in}}%
\pgfpathlineto{\pgfqpoint{3.773325in}{1.630302in}}%
\pgfpathlineto{\pgfqpoint{3.784819in}{1.642762in}}%
\pgfpathlineto{\pgfqpoint{3.796323in}{1.655805in}}%
\pgfpathlineto{\pgfqpoint{3.807836in}{1.669342in}}%
\pgfpathlineto{\pgfqpoint{3.819358in}{1.683263in}}%
\pgfpathlineto{\pgfqpoint{3.812996in}{1.702596in}}%
\pgfpathlineto{\pgfqpoint{3.806639in}{1.722126in}}%
\pgfpathlineto{\pgfqpoint{3.800286in}{1.741823in}}%
\pgfpathlineto{\pgfqpoint{3.793936in}{1.761657in}}%
\pgfpathlineto{\pgfqpoint{3.787589in}{1.781595in}}%
\pgfpathlineto{\pgfqpoint{3.776076in}{1.767934in}}%
\pgfpathlineto{\pgfqpoint{3.764567in}{1.754442in}}%
\pgfpathlineto{\pgfqpoint{3.753061in}{1.741147in}}%
\pgfpathlineto{\pgfqpoint{3.741558in}{1.727999in}}%
\pgfpathlineto{\pgfqpoint{3.730056in}{1.714946in}}%
\pgfpathlineto{\pgfqpoint{3.736415in}{1.695855in}}%
\pgfpathlineto{\pgfqpoint{3.742774in}{1.676689in}}%
\pgfpathlineto{\pgfqpoint{3.749132in}{1.657424in}}%
\pgfpathlineto{\pgfqpoint{3.755489in}{1.638035in}}%
\pgfpathclose%
\pgfusepath{stroke,fill}%
\end{pgfscope}%
\begin{pgfscope}%
\pgfpathrectangle{\pgfqpoint{0.887500in}{0.275000in}}{\pgfqpoint{4.225000in}{4.225000in}}%
\pgfusepath{clip}%
\pgfsetbuttcap%
\pgfsetroundjoin%
\definecolor{currentfill}{rgb}{0.233603,0.313828,0.543914}%
\pgfsetfillcolor{currentfill}%
\pgfsetfillopacity{0.700000}%
\pgfsetlinewidth{0.501875pt}%
\definecolor{currentstroke}{rgb}{1.000000,1.000000,1.000000}%
\pgfsetstrokecolor{currentstroke}%
\pgfsetstrokeopacity{0.500000}%
\pgfsetdash{}{0pt}%
\pgfpathmoveto{\pgfqpoint{3.819358in}{1.683263in}}%
\pgfpathlineto{\pgfqpoint{3.830884in}{1.697430in}}%
\pgfpathlineto{\pgfqpoint{3.842413in}{1.711705in}}%
\pgfpathlineto{\pgfqpoint{3.853942in}{1.725951in}}%
\pgfpathlineto{\pgfqpoint{3.865469in}{1.740029in}}%
\pgfpathlineto{\pgfqpoint{3.876989in}{1.753802in}}%
\pgfpathlineto{\pgfqpoint{3.870599in}{1.772011in}}%
\pgfpathlineto{\pgfqpoint{3.864219in}{1.790637in}}%
\pgfpathlineto{\pgfqpoint{3.857848in}{1.809634in}}%
\pgfpathlineto{\pgfqpoint{3.851484in}{1.828936in}}%
\pgfpathlineto{\pgfqpoint{3.845126in}{1.848479in}}%
\pgfpathlineto{\pgfqpoint{3.833630in}{1.835674in}}%
\pgfpathlineto{\pgfqpoint{3.822126in}{1.822468in}}%
\pgfpathlineto{\pgfqpoint{3.810616in}{1.808977in}}%
\pgfpathlineto{\pgfqpoint{3.799103in}{1.795314in}}%
\pgfpathlineto{\pgfqpoint{3.787589in}{1.781595in}}%
\pgfpathlineto{\pgfqpoint{3.793936in}{1.761657in}}%
\pgfpathlineto{\pgfqpoint{3.800286in}{1.741823in}}%
\pgfpathlineto{\pgfqpoint{3.806639in}{1.722126in}}%
\pgfpathlineto{\pgfqpoint{3.812996in}{1.702596in}}%
\pgfpathclose%
\pgfusepath{stroke,fill}%
\end{pgfscope}%
\begin{pgfscope}%
\pgfpathrectangle{\pgfqpoint{0.887500in}{0.275000in}}{\pgfqpoint{4.225000in}{4.225000in}}%
\pgfusepath{clip}%
\pgfsetbuttcap%
\pgfsetroundjoin%
\definecolor{currentfill}{rgb}{0.169646,0.456262,0.558030}%
\pgfsetfillcolor{currentfill}%
\pgfsetfillopacity{0.700000}%
\pgfsetlinewidth{0.501875pt}%
\definecolor{currentstroke}{rgb}{1.000000,1.000000,1.000000}%
\pgfsetstrokecolor{currentstroke}%
\pgfsetstrokeopacity{0.500000}%
\pgfsetdash{}{0pt}%
\pgfpathmoveto{\pgfqpoint{2.047042in}{2.022362in}}%
\pgfpathlineto{\pgfqpoint{2.058809in}{2.026132in}}%
\pgfpathlineto{\pgfqpoint{2.070571in}{2.029898in}}%
\pgfpathlineto{\pgfqpoint{2.082327in}{2.033658in}}%
\pgfpathlineto{\pgfqpoint{2.094077in}{2.037408in}}%
\pgfpathlineto{\pgfqpoint{2.105823in}{2.041146in}}%
\pgfpathlineto{\pgfqpoint{2.099813in}{2.050590in}}%
\pgfpathlineto{\pgfqpoint{2.093808in}{2.059999in}}%
\pgfpathlineto{\pgfqpoint{2.087808in}{2.069373in}}%
\pgfpathlineto{\pgfqpoint{2.081812in}{2.078714in}}%
\pgfpathlineto{\pgfqpoint{2.075821in}{2.088021in}}%
\pgfpathlineto{\pgfqpoint{2.064086in}{2.084343in}}%
\pgfpathlineto{\pgfqpoint{2.052346in}{2.080653in}}%
\pgfpathlineto{\pgfqpoint{2.040600in}{2.076954in}}%
\pgfpathlineto{\pgfqpoint{2.028848in}{2.073249in}}%
\pgfpathlineto{\pgfqpoint{2.017091in}{2.069539in}}%
\pgfpathlineto{\pgfqpoint{2.023072in}{2.060171in}}%
\pgfpathlineto{\pgfqpoint{2.029058in}{2.050769in}}%
\pgfpathlineto{\pgfqpoint{2.035048in}{2.041334in}}%
\pgfpathlineto{\pgfqpoint{2.041042in}{2.031865in}}%
\pgfpathclose%
\pgfusepath{stroke,fill}%
\end{pgfscope}%
\begin{pgfscope}%
\pgfpathrectangle{\pgfqpoint{0.887500in}{0.275000in}}{\pgfqpoint{4.225000in}{4.225000in}}%
\pgfusepath{clip}%
\pgfsetbuttcap%
\pgfsetroundjoin%
\definecolor{currentfill}{rgb}{0.214298,0.355619,0.551184}%
\pgfsetfillcolor{currentfill}%
\pgfsetfillopacity{0.700000}%
\pgfsetlinewidth{0.501875pt}%
\definecolor{currentstroke}{rgb}{1.000000,1.000000,1.000000}%
\pgfsetstrokecolor{currentstroke}%
\pgfsetstrokeopacity{0.500000}%
\pgfsetdash{}{0pt}%
\pgfpathmoveto{\pgfqpoint{2.876093in}{1.809553in}}%
\pgfpathlineto{\pgfqpoint{2.887657in}{1.813415in}}%
\pgfpathlineto{\pgfqpoint{2.899216in}{1.817232in}}%
\pgfpathlineto{\pgfqpoint{2.910770in}{1.820992in}}%
\pgfpathlineto{\pgfqpoint{2.922318in}{1.824684in}}%
\pgfpathlineto{\pgfqpoint{2.933861in}{1.828313in}}%
\pgfpathlineto{\pgfqpoint{2.927586in}{1.838949in}}%
\pgfpathlineto{\pgfqpoint{2.921314in}{1.849532in}}%
\pgfpathlineto{\pgfqpoint{2.915046in}{1.860064in}}%
\pgfpathlineto{\pgfqpoint{2.908782in}{1.870548in}}%
\pgfpathlineto{\pgfqpoint{2.902522in}{1.880983in}}%
\pgfpathlineto{\pgfqpoint{2.890988in}{1.877489in}}%
\pgfpathlineto{\pgfqpoint{2.879448in}{1.873915in}}%
\pgfpathlineto{\pgfqpoint{2.867903in}{1.870250in}}%
\pgfpathlineto{\pgfqpoint{2.856353in}{1.866509in}}%
\pgfpathlineto{\pgfqpoint{2.844797in}{1.862710in}}%
\pgfpathlineto{\pgfqpoint{2.851049in}{1.852183in}}%
\pgfpathlineto{\pgfqpoint{2.857304in}{1.841604in}}%
\pgfpathlineto{\pgfqpoint{2.863563in}{1.830973in}}%
\pgfpathlineto{\pgfqpoint{2.869826in}{1.820290in}}%
\pgfpathclose%
\pgfusepath{stroke,fill}%
\end{pgfscope}%
\begin{pgfscope}%
\pgfpathrectangle{\pgfqpoint{0.887500in}{0.275000in}}{\pgfqpoint{4.225000in}{4.225000in}}%
\pgfusepath{clip}%
\pgfsetbuttcap%
\pgfsetroundjoin%
\definecolor{currentfill}{rgb}{0.262138,0.242286,0.520837}%
\pgfsetfillcolor{currentfill}%
\pgfsetfillopacity{0.700000}%
\pgfsetlinewidth{0.501875pt}%
\definecolor{currentstroke}{rgb}{1.000000,1.000000,1.000000}%
\pgfsetstrokecolor{currentstroke}%
\pgfsetstrokeopacity{0.500000}%
\pgfsetdash{}{0pt}%
\pgfpathmoveto{\pgfqpoint{3.704638in}{1.572456in}}%
\pgfpathlineto{\pgfqpoint{3.716052in}{1.579801in}}%
\pgfpathlineto{\pgfqpoint{3.727478in}{1.588046in}}%
\pgfpathlineto{\pgfqpoint{3.738919in}{1.597297in}}%
\pgfpathlineto{\pgfqpoint{3.750375in}{1.607482in}}%
\pgfpathlineto{\pgfqpoint{3.761844in}{1.618513in}}%
\pgfpathlineto{\pgfqpoint{3.755489in}{1.638035in}}%
\pgfpathlineto{\pgfqpoint{3.749132in}{1.657424in}}%
\pgfpathlineto{\pgfqpoint{3.742774in}{1.676689in}}%
\pgfpathlineto{\pgfqpoint{3.736415in}{1.695855in}}%
\pgfpathlineto{\pgfqpoint{3.730056in}{1.714946in}}%
\pgfpathlineto{\pgfqpoint{3.718555in}{1.701935in}}%
\pgfpathlineto{\pgfqpoint{3.707054in}{1.688913in}}%
\pgfpathlineto{\pgfqpoint{3.695551in}{1.675828in}}%
\pgfpathlineto{\pgfqpoint{3.684047in}{1.662639in}}%
\pgfpathlineto{\pgfqpoint{3.672543in}{1.649462in}}%
\pgfpathlineto{\pgfqpoint{3.678972in}{1.634953in}}%
\pgfpathlineto{\pgfqpoint{3.685399in}{1.620173in}}%
\pgfpathlineto{\pgfqpoint{3.691821in}{1.604942in}}%
\pgfpathlineto{\pgfqpoint{3.698234in}{1.589081in}}%
\pgfpathclose%
\pgfusepath{stroke,fill}%
\end{pgfscope}%
\begin{pgfscope}%
\pgfpathrectangle{\pgfqpoint{0.887500in}{0.275000in}}{\pgfqpoint{4.225000in}{4.225000in}}%
\pgfusepath{clip}%
\pgfsetbuttcap%
\pgfsetroundjoin%
\definecolor{currentfill}{rgb}{0.188923,0.410910,0.556326}%
\pgfsetfillcolor{currentfill}%
\pgfsetfillopacity{0.700000}%
\pgfsetlinewidth{0.501875pt}%
\definecolor{currentstroke}{rgb}{1.000000,1.000000,1.000000}%
\pgfsetstrokecolor{currentstroke}%
\pgfsetstrokeopacity{0.500000}%
\pgfsetdash{}{0pt}%
\pgfpathmoveto{\pgfqpoint{2.461489in}{1.921860in}}%
\pgfpathlineto{\pgfqpoint{2.473156in}{1.925715in}}%
\pgfpathlineto{\pgfqpoint{2.484818in}{1.929553in}}%
\pgfpathlineto{\pgfqpoint{2.496474in}{1.933373in}}%
\pgfpathlineto{\pgfqpoint{2.508125in}{1.937175in}}%
\pgfpathlineto{\pgfqpoint{2.519770in}{1.940959in}}%
\pgfpathlineto{\pgfqpoint{2.513621in}{1.950941in}}%
\pgfpathlineto{\pgfqpoint{2.507477in}{1.960875in}}%
\pgfpathlineto{\pgfqpoint{2.501337in}{1.970762in}}%
\pgfpathlineto{\pgfqpoint{2.495202in}{1.980602in}}%
\pgfpathlineto{\pgfqpoint{2.489071in}{1.990393in}}%
\pgfpathlineto{\pgfqpoint{2.477435in}{1.986664in}}%
\pgfpathlineto{\pgfqpoint{2.465795in}{1.982915in}}%
\pgfpathlineto{\pgfqpoint{2.454148in}{1.979149in}}%
\pgfpathlineto{\pgfqpoint{2.442497in}{1.975365in}}%
\pgfpathlineto{\pgfqpoint{2.430839in}{1.971565in}}%
\pgfpathlineto{\pgfqpoint{2.436961in}{1.961719in}}%
\pgfpathlineto{\pgfqpoint{2.443086in}{1.951826in}}%
\pgfpathlineto{\pgfqpoint{2.449216in}{1.941885in}}%
\pgfpathlineto{\pgfqpoint{2.455351in}{1.931897in}}%
\pgfpathclose%
\pgfusepath{stroke,fill}%
\end{pgfscope}%
\begin{pgfscope}%
\pgfpathrectangle{\pgfqpoint{0.887500in}{0.275000in}}{\pgfqpoint{4.225000in}{4.225000in}}%
\pgfusepath{clip}%
\pgfsetbuttcap%
\pgfsetroundjoin%
\definecolor{currentfill}{rgb}{0.225863,0.330805,0.547314}%
\pgfsetfillcolor{currentfill}%
\pgfsetfillopacity{0.700000}%
\pgfsetlinewidth{0.501875pt}%
\definecolor{currentstroke}{rgb}{1.000000,1.000000,1.000000}%
\pgfsetstrokecolor{currentstroke}%
\pgfsetstrokeopacity{0.500000}%
\pgfsetdash{}{0pt}%
\pgfpathmoveto{\pgfqpoint{3.966563in}{1.730441in}}%
\pgfpathlineto{\pgfqpoint{3.978020in}{1.741279in}}%
\pgfpathlineto{\pgfqpoint{3.989466in}{1.751687in}}%
\pgfpathlineto{\pgfqpoint{4.000902in}{1.761736in}}%
\pgfpathlineto{\pgfqpoint{4.012328in}{1.771459in}}%
\pgfpathlineto{\pgfqpoint{4.023744in}{1.780870in}}%
\pgfpathlineto{\pgfqpoint{4.017289in}{1.797041in}}%
\pgfpathlineto{\pgfqpoint{4.010835in}{1.813151in}}%
\pgfpathlineto{\pgfqpoint{4.004383in}{1.829258in}}%
\pgfpathlineto{\pgfqpoint{3.997935in}{1.845423in}}%
\pgfpathlineto{\pgfqpoint{3.991491in}{1.861705in}}%
\pgfpathlineto{\pgfqpoint{3.980110in}{1.853573in}}%
\pgfpathlineto{\pgfqpoint{3.968714in}{1.844894in}}%
\pgfpathlineto{\pgfqpoint{3.957301in}{1.835609in}}%
\pgfpathlineto{\pgfqpoint{3.945870in}{1.825682in}}%
\pgfpathlineto{\pgfqpoint{3.934424in}{1.815131in}}%
\pgfpathlineto{\pgfqpoint{3.940844in}{1.797975in}}%
\pgfpathlineto{\pgfqpoint{3.947268in}{1.780988in}}%
\pgfpathlineto{\pgfqpoint{3.953697in}{1.764110in}}%
\pgfpathlineto{\pgfqpoint{3.960129in}{1.747281in}}%
\pgfpathclose%
\pgfusepath{stroke,fill}%
\end{pgfscope}%
\begin{pgfscope}%
\pgfpathrectangle{\pgfqpoint{0.887500in}{0.275000in}}{\pgfqpoint{4.225000in}{4.225000in}}%
\pgfusepath{clip}%
\pgfsetbuttcap%
\pgfsetroundjoin%
\definecolor{currentfill}{rgb}{0.221989,0.339161,0.548752}%
\pgfsetfillcolor{currentfill}%
\pgfsetfillopacity{0.700000}%
\pgfsetlinewidth{0.501875pt}%
\definecolor{currentstroke}{rgb}{1.000000,1.000000,1.000000}%
\pgfsetstrokecolor{currentstroke}%
\pgfsetstrokeopacity{0.500000}%
\pgfsetdash{}{0pt}%
\pgfpathmoveto{\pgfqpoint{2.965294in}{1.774285in}}%
\pgfpathlineto{\pgfqpoint{2.976839in}{1.778020in}}%
\pgfpathlineto{\pgfqpoint{2.988379in}{1.781733in}}%
\pgfpathlineto{\pgfqpoint{2.999913in}{1.785434in}}%
\pgfpathlineto{\pgfqpoint{3.011442in}{1.789132in}}%
\pgfpathlineto{\pgfqpoint{3.022965in}{1.792840in}}%
\pgfpathlineto{\pgfqpoint{3.016662in}{1.803683in}}%
\pgfpathlineto{\pgfqpoint{3.010364in}{1.814470in}}%
\pgfpathlineto{\pgfqpoint{3.004068in}{1.825201in}}%
\pgfpathlineto{\pgfqpoint{2.997777in}{1.835880in}}%
\pgfpathlineto{\pgfqpoint{2.991489in}{1.846505in}}%
\pgfpathlineto{\pgfqpoint{2.979975in}{1.842763in}}%
\pgfpathlineto{\pgfqpoint{2.968455in}{1.839101in}}%
\pgfpathlineto{\pgfqpoint{2.956930in}{1.835492in}}%
\pgfpathlineto{\pgfqpoint{2.945398in}{1.831906in}}%
\pgfpathlineto{\pgfqpoint{2.933861in}{1.828313in}}%
\pgfpathlineto{\pgfqpoint{2.940140in}{1.817623in}}%
\pgfpathlineto{\pgfqpoint{2.946423in}{1.806878in}}%
\pgfpathlineto{\pgfqpoint{2.952710in}{1.796074in}}%
\pgfpathlineto{\pgfqpoint{2.959000in}{1.785210in}}%
\pgfpathclose%
\pgfusepath{stroke,fill}%
\end{pgfscope}%
\begin{pgfscope}%
\pgfpathrectangle{\pgfqpoint{0.887500in}{0.275000in}}{\pgfqpoint{4.225000in}{4.225000in}}%
\pgfusepath{clip}%
\pgfsetbuttcap%
\pgfsetroundjoin%
\definecolor{currentfill}{rgb}{0.267968,0.223549,0.512008}%
\pgfsetfillcolor{currentfill}%
\pgfsetfillopacity{0.700000}%
\pgfsetlinewidth{0.501875pt}%
\definecolor{currentstroke}{rgb}{1.000000,1.000000,1.000000}%
\pgfsetstrokecolor{currentstroke}%
\pgfsetstrokeopacity{0.500000}%
\pgfsetdash{}{0pt}%
\pgfpathmoveto{\pgfqpoint{3.793637in}{1.521227in}}%
\pgfpathlineto{\pgfqpoint{3.805123in}{1.533112in}}%
\pgfpathlineto{\pgfqpoint{3.816629in}{1.546044in}}%
\pgfpathlineto{\pgfqpoint{3.828152in}{1.559895in}}%
\pgfpathlineto{\pgfqpoint{3.839692in}{1.574532in}}%
\pgfpathlineto{\pgfqpoint{3.851245in}{1.589778in}}%
\pgfpathlineto{\pgfqpoint{3.844857in}{1.608035in}}%
\pgfpathlineto{\pgfqpoint{3.838474in}{1.626516in}}%
\pgfpathlineto{\pgfqpoint{3.832096in}{1.645218in}}%
\pgfpathlineto{\pgfqpoint{3.825724in}{1.664135in}}%
\pgfpathlineto{\pgfqpoint{3.819358in}{1.683263in}}%
\pgfpathlineto{\pgfqpoint{3.807836in}{1.669342in}}%
\pgfpathlineto{\pgfqpoint{3.796323in}{1.655805in}}%
\pgfpathlineto{\pgfqpoint{3.784819in}{1.642762in}}%
\pgfpathlineto{\pgfqpoint{3.773325in}{1.630302in}}%
\pgfpathlineto{\pgfqpoint{3.761844in}{1.618513in}}%
\pgfpathlineto{\pgfqpoint{3.768198in}{1.598912in}}%
\pgfpathlineto{\pgfqpoint{3.774553in}{1.579308in}}%
\pgfpathlineto{\pgfqpoint{3.780910in}{1.559777in}}%
\pgfpathlineto{\pgfqpoint{3.787271in}{1.540391in}}%
\pgfpathclose%
\pgfusepath{stroke,fill}%
\end{pgfscope}%
\begin{pgfscope}%
\pgfpathrectangle{\pgfqpoint{0.887500in}{0.275000in}}{\pgfqpoint{4.225000in}{4.225000in}}%
\pgfusepath{clip}%
\pgfsetbuttcap%
\pgfsetroundjoin%
\definecolor{currentfill}{rgb}{0.253935,0.265254,0.529983}%
\pgfsetfillcolor{currentfill}%
\pgfsetfillopacity{0.700000}%
\pgfsetlinewidth{0.501875pt}%
\definecolor{currentstroke}{rgb}{1.000000,1.000000,1.000000}%
\pgfsetstrokecolor{currentstroke}%
\pgfsetstrokeopacity{0.500000}%
\pgfsetdash{}{0pt}%
\pgfpathmoveto{\pgfqpoint{4.233813in}{1.594011in}}%
\pgfpathlineto{\pgfqpoint{4.245252in}{1.604847in}}%
\pgfpathlineto{\pgfqpoint{4.256696in}{1.615920in}}%
\pgfpathlineto{\pgfqpoint{4.268140in}{1.627050in}}%
\pgfpathlineto{\pgfqpoint{4.279579in}{1.638015in}}%
\pgfpathlineto{\pgfqpoint{4.291003in}{1.648595in}}%
\pgfpathlineto{\pgfqpoint{4.284670in}{1.669819in}}%
\pgfpathlineto{\pgfqpoint{4.278299in}{1.689867in}}%
\pgfpathlineto{\pgfqpoint{4.271896in}{1.708824in}}%
\pgfpathlineto{\pgfqpoint{4.265462in}{1.726774in}}%
\pgfpathlineto{\pgfqpoint{4.259000in}{1.743804in}}%
\pgfpathlineto{\pgfqpoint{4.247719in}{1.737857in}}%
\pgfpathlineto{\pgfqpoint{4.236418in}{1.731378in}}%
\pgfpathlineto{\pgfqpoint{4.225095in}{1.724312in}}%
\pgfpathlineto{\pgfqpoint{4.213749in}{1.716606in}}%
\pgfpathlineto{\pgfqpoint{4.202381in}{1.708226in}}%
\pgfpathlineto{\pgfqpoint{4.208755in}{1.688312in}}%
\pgfpathlineto{\pgfqpoint{4.215089in}{1.667049in}}%
\pgfpathlineto{\pgfqpoint{4.221380in}{1.644320in}}%
\pgfpathlineto{\pgfqpoint{4.227623in}{1.620012in}}%
\pgfpathclose%
\pgfusepath{stroke,fill}%
\end{pgfscope}%
\begin{pgfscope}%
\pgfpathrectangle{\pgfqpoint{0.887500in}{0.275000in}}{\pgfqpoint{4.225000in}{4.225000in}}%
\pgfusepath{clip}%
\pgfsetbuttcap%
\pgfsetroundjoin%
\definecolor{currentfill}{rgb}{0.174274,0.445044,0.557792}%
\pgfsetfillcolor{currentfill}%
\pgfsetfillopacity{0.700000}%
\pgfsetlinewidth{0.501875pt}%
\definecolor{currentstroke}{rgb}{1.000000,1.000000,1.000000}%
\pgfsetstrokecolor{currentstroke}%
\pgfsetstrokeopacity{0.500000}%
\pgfsetdash{}{0pt}%
\pgfpathmoveto{\pgfqpoint{2.135938in}{1.993375in}}%
\pgfpathlineto{\pgfqpoint{2.147687in}{1.997159in}}%
\pgfpathlineto{\pgfqpoint{2.159432in}{2.000928in}}%
\pgfpathlineto{\pgfqpoint{2.171171in}{2.004683in}}%
\pgfpathlineto{\pgfqpoint{2.182904in}{2.008425in}}%
\pgfpathlineto{\pgfqpoint{2.194632in}{2.012155in}}%
\pgfpathlineto{\pgfqpoint{2.188590in}{2.021724in}}%
\pgfpathlineto{\pgfqpoint{2.182552in}{2.031254in}}%
\pgfpathlineto{\pgfqpoint{2.176519in}{2.040746in}}%
\pgfpathlineto{\pgfqpoint{2.170491in}{2.050201in}}%
\pgfpathlineto{\pgfqpoint{2.164467in}{2.059621in}}%
\pgfpathlineto{\pgfqpoint{2.152749in}{2.055952in}}%
\pgfpathlineto{\pgfqpoint{2.141025in}{2.052272in}}%
\pgfpathlineto{\pgfqpoint{2.129297in}{2.048578in}}%
\pgfpathlineto{\pgfqpoint{2.117562in}{2.044870in}}%
\pgfpathlineto{\pgfqpoint{2.105823in}{2.041146in}}%
\pgfpathlineto{\pgfqpoint{2.111837in}{2.031666in}}%
\pgfpathlineto{\pgfqpoint{2.117855in}{2.022150in}}%
\pgfpathlineto{\pgfqpoint{2.123878in}{2.012597in}}%
\pgfpathlineto{\pgfqpoint{2.129906in}{2.003005in}}%
\pgfpathclose%
\pgfusepath{stroke,fill}%
\end{pgfscope}%
\begin{pgfscope}%
\pgfpathrectangle{\pgfqpoint{0.887500in}{0.275000in}}{\pgfqpoint{4.225000in}{4.225000in}}%
\pgfusepath{clip}%
\pgfsetbuttcap%
\pgfsetroundjoin%
\definecolor{currentfill}{rgb}{0.233603,0.313828,0.543914}%
\pgfsetfillcolor{currentfill}%
\pgfsetfillopacity{0.700000}%
\pgfsetlinewidth{0.501875pt}%
\definecolor{currentstroke}{rgb}{1.000000,1.000000,1.000000}%
\pgfsetstrokecolor{currentstroke}%
\pgfsetstrokeopacity{0.500000}%
\pgfsetdash{}{0pt}%
\pgfpathmoveto{\pgfqpoint{4.055969in}{1.697020in}}%
\pgfpathlineto{\pgfqpoint{4.067416in}{1.707568in}}%
\pgfpathlineto{\pgfqpoint{4.078859in}{1.718026in}}%
\pgfpathlineto{\pgfqpoint{4.090297in}{1.728350in}}%
\pgfpathlineto{\pgfqpoint{4.101728in}{1.738498in}}%
\pgfpathlineto{\pgfqpoint{4.113152in}{1.748421in}}%
\pgfpathlineto{\pgfqpoint{4.106674in}{1.764267in}}%
\pgfpathlineto{\pgfqpoint{4.100185in}{1.779624in}}%
\pgfpathlineto{\pgfqpoint{4.093687in}{1.794587in}}%
\pgfpathlineto{\pgfqpoint{4.087184in}{1.809250in}}%
\pgfpathlineto{\pgfqpoint{4.080679in}{1.823708in}}%
\pgfpathlineto{\pgfqpoint{4.069311in}{1.815661in}}%
\pgfpathlineto{\pgfqpoint{4.057933in}{1.807363in}}%
\pgfpathlineto{\pgfqpoint{4.046546in}{1.798808in}}%
\pgfpathlineto{\pgfqpoint{4.035150in}{1.789981in}}%
\pgfpathlineto{\pgfqpoint{4.023744in}{1.780870in}}%
\pgfpathlineto{\pgfqpoint{4.030197in}{1.764578in}}%
\pgfpathlineto{\pgfqpoint{4.036648in}{1.748106in}}%
\pgfpathlineto{\pgfqpoint{4.043094in}{1.731395in}}%
\pgfpathlineto{\pgfqpoint{4.049535in}{1.714386in}}%
\pgfpathclose%
\pgfusepath{stroke,fill}%
\end{pgfscope}%
\begin{pgfscope}%
\pgfpathrectangle{\pgfqpoint{0.887500in}{0.275000in}}{\pgfqpoint{4.225000in}{4.225000in}}%
\pgfusepath{clip}%
\pgfsetbuttcap%
\pgfsetroundjoin%
\definecolor{currentfill}{rgb}{0.253935,0.265254,0.529983}%
\pgfsetfillcolor{currentfill}%
\pgfsetfillopacity{0.700000}%
\pgfsetlinewidth{0.501875pt}%
\definecolor{currentstroke}{rgb}{1.000000,1.000000,1.000000}%
\pgfsetstrokecolor{currentstroke}%
\pgfsetstrokeopacity{0.500000}%
\pgfsetdash{}{0pt}%
\pgfpathmoveto{\pgfqpoint{3.851245in}{1.589778in}}%
\pgfpathlineto{\pgfqpoint{3.862807in}{1.605407in}}%
\pgfpathlineto{\pgfqpoint{3.874374in}{1.621189in}}%
\pgfpathlineto{\pgfqpoint{3.885941in}{1.636896in}}%
\pgfpathlineto{\pgfqpoint{3.897503in}{1.652298in}}%
\pgfpathlineto{\pgfqpoint{3.909053in}{1.667166in}}%
\pgfpathlineto{\pgfqpoint{3.902629in}{1.684118in}}%
\pgfpathlineto{\pgfqpoint{3.896209in}{1.701194in}}%
\pgfpathlineto{\pgfqpoint{3.889795in}{1.718457in}}%
\pgfpathlineto{\pgfqpoint{3.883388in}{1.735972in}}%
\pgfpathlineto{\pgfqpoint{3.876989in}{1.753802in}}%
\pgfpathlineto{\pgfqpoint{3.865469in}{1.740029in}}%
\pgfpathlineto{\pgfqpoint{3.853942in}{1.725951in}}%
\pgfpathlineto{\pgfqpoint{3.842413in}{1.711705in}}%
\pgfpathlineto{\pgfqpoint{3.830884in}{1.697430in}}%
\pgfpathlineto{\pgfqpoint{3.819358in}{1.683263in}}%
\pgfpathlineto{\pgfqpoint{3.825724in}{1.664135in}}%
\pgfpathlineto{\pgfqpoint{3.832096in}{1.645218in}}%
\pgfpathlineto{\pgfqpoint{3.838474in}{1.626516in}}%
\pgfpathlineto{\pgfqpoint{3.844857in}{1.608035in}}%
\pgfpathclose%
\pgfusepath{stroke,fill}%
\end{pgfscope}%
\begin{pgfscope}%
\pgfpathrectangle{\pgfqpoint{0.887500in}{0.275000in}}{\pgfqpoint{4.225000in}{4.225000in}}%
\pgfusepath{clip}%
\pgfsetbuttcap%
\pgfsetroundjoin%
\definecolor{currentfill}{rgb}{0.229739,0.322361,0.545706}%
\pgfsetfillcolor{currentfill}%
\pgfsetfillopacity{0.700000}%
\pgfsetlinewidth{0.501875pt}%
\definecolor{currentstroke}{rgb}{1.000000,1.000000,1.000000}%
\pgfsetstrokecolor{currentstroke}%
\pgfsetstrokeopacity{0.500000}%
\pgfsetdash{}{0pt}%
\pgfpathmoveto{\pgfqpoint{3.054531in}{1.737738in}}%
\pgfpathlineto{\pgfqpoint{3.066056in}{1.741467in}}%
\pgfpathlineto{\pgfqpoint{3.077576in}{1.745179in}}%
\pgfpathlineto{\pgfqpoint{3.089090in}{1.748897in}}%
\pgfpathlineto{\pgfqpoint{3.100598in}{1.752647in}}%
\pgfpathlineto{\pgfqpoint{3.112101in}{1.756454in}}%
\pgfpathlineto{\pgfqpoint{3.105773in}{1.767544in}}%
\pgfpathlineto{\pgfqpoint{3.099448in}{1.778598in}}%
\pgfpathlineto{\pgfqpoint{3.093126in}{1.789622in}}%
\pgfpathlineto{\pgfqpoint{3.086808in}{1.800618in}}%
\pgfpathlineto{\pgfqpoint{3.080494in}{1.811582in}}%
\pgfpathlineto{\pgfqpoint{3.068999in}{1.807827in}}%
\pgfpathlineto{\pgfqpoint{3.057499in}{1.804066in}}%
\pgfpathlineto{\pgfqpoint{3.045993in}{1.800310in}}%
\pgfpathlineto{\pgfqpoint{3.034482in}{1.796565in}}%
\pgfpathlineto{\pgfqpoint{3.022965in}{1.792840in}}%
\pgfpathlineto{\pgfqpoint{3.029271in}{1.781939in}}%
\pgfpathlineto{\pgfqpoint{3.035580in}{1.770980in}}%
\pgfpathlineto{\pgfqpoint{3.041894in}{1.759961in}}%
\pgfpathlineto{\pgfqpoint{3.048210in}{1.748880in}}%
\pgfpathclose%
\pgfusepath{stroke,fill}%
\end{pgfscope}%
\begin{pgfscope}%
\pgfpathrectangle{\pgfqpoint{0.887500in}{0.275000in}}{\pgfqpoint{4.225000in}{4.225000in}}%
\pgfusepath{clip}%
\pgfsetbuttcap%
\pgfsetroundjoin%
\definecolor{currentfill}{rgb}{0.237441,0.305202,0.541921}%
\pgfsetfillcolor{currentfill}%
\pgfsetfillopacity{0.700000}%
\pgfsetlinewidth{0.501875pt}%
\definecolor{currentstroke}{rgb}{1.000000,1.000000,1.000000}%
\pgfsetstrokecolor{currentstroke}%
\pgfsetstrokeopacity{0.500000}%
\pgfsetdash{}{0pt}%
\pgfpathmoveto{\pgfqpoint{3.909053in}{1.667166in}}%
\pgfpathlineto{\pgfqpoint{3.920589in}{1.681281in}}%
\pgfpathlineto{\pgfqpoint{3.932106in}{1.694592in}}%
\pgfpathlineto{\pgfqpoint{3.943607in}{1.707173in}}%
\pgfpathlineto{\pgfqpoint{3.955092in}{1.719098in}}%
\pgfpathlineto{\pgfqpoint{3.966563in}{1.730441in}}%
\pgfpathlineto{\pgfqpoint{3.960129in}{1.747281in}}%
\pgfpathlineto{\pgfqpoint{3.953697in}{1.764110in}}%
\pgfpathlineto{\pgfqpoint{3.947268in}{1.780988in}}%
\pgfpathlineto{\pgfqpoint{3.940844in}{1.797975in}}%
\pgfpathlineto{\pgfqpoint{3.934424in}{1.815131in}}%
\pgfpathlineto{\pgfqpoint{3.922963in}{1.803978in}}%
\pgfpathlineto{\pgfqpoint{3.911488in}{1.792246in}}%
\pgfpathlineto{\pgfqpoint{3.900000in}{1.779957in}}%
\pgfpathlineto{\pgfqpoint{3.888500in}{1.767135in}}%
\pgfpathlineto{\pgfqpoint{3.876989in}{1.753802in}}%
\pgfpathlineto{\pgfqpoint{3.883388in}{1.735972in}}%
\pgfpathlineto{\pgfqpoint{3.889795in}{1.718457in}}%
\pgfpathlineto{\pgfqpoint{3.896209in}{1.701194in}}%
\pgfpathlineto{\pgfqpoint{3.902629in}{1.684118in}}%
\pgfpathclose%
\pgfusepath{stroke,fill}%
\end{pgfscope}%
\begin{pgfscope}%
\pgfpathrectangle{\pgfqpoint{0.887500in}{0.275000in}}{\pgfqpoint{4.225000in}{4.225000in}}%
\pgfusepath{clip}%
\pgfsetbuttcap%
\pgfsetroundjoin%
\definecolor{currentfill}{rgb}{0.194100,0.399323,0.555565}%
\pgfsetfillcolor{currentfill}%
\pgfsetfillopacity{0.700000}%
\pgfsetlinewidth{0.501875pt}%
\definecolor{currentstroke}{rgb}{1.000000,1.000000,1.000000}%
\pgfsetstrokecolor{currentstroke}%
\pgfsetstrokeopacity{0.500000}%
\pgfsetdash{}{0pt}%
\pgfpathmoveto{\pgfqpoint{2.550577in}{1.890341in}}%
\pgfpathlineto{\pgfqpoint{2.562226in}{1.894163in}}%
\pgfpathlineto{\pgfqpoint{2.573870in}{1.897964in}}%
\pgfpathlineto{\pgfqpoint{2.585508in}{1.901746in}}%
\pgfpathlineto{\pgfqpoint{2.597141in}{1.905508in}}%
\pgfpathlineto{\pgfqpoint{2.608769in}{1.909252in}}%
\pgfpathlineto{\pgfqpoint{2.602589in}{1.919410in}}%
\pgfpathlineto{\pgfqpoint{2.596414in}{1.929522in}}%
\pgfpathlineto{\pgfqpoint{2.590244in}{1.939587in}}%
\pgfpathlineto{\pgfqpoint{2.584077in}{1.949605in}}%
\pgfpathlineto{\pgfqpoint{2.577915in}{1.959577in}}%
\pgfpathlineto{\pgfqpoint{2.566297in}{1.955892in}}%
\pgfpathlineto{\pgfqpoint{2.554673in}{1.952190in}}%
\pgfpathlineto{\pgfqpoint{2.543044in}{1.948466in}}%
\pgfpathlineto{\pgfqpoint{2.531410in}{1.944723in}}%
\pgfpathlineto{\pgfqpoint{2.519770in}{1.940959in}}%
\pgfpathlineto{\pgfqpoint{2.525923in}{1.930929in}}%
\pgfpathlineto{\pgfqpoint{2.532080in}{1.920853in}}%
\pgfpathlineto{\pgfqpoint{2.538241in}{1.910729in}}%
\pgfpathlineto{\pgfqpoint{2.544407in}{1.900559in}}%
\pgfpathclose%
\pgfusepath{stroke,fill}%
\end{pgfscope}%
\begin{pgfscope}%
\pgfpathrectangle{\pgfqpoint{0.887500in}{0.275000in}}{\pgfqpoint{4.225000in}{4.225000in}}%
\pgfusepath{clip}%
\pgfsetbuttcap%
\pgfsetroundjoin%
\definecolor{currentfill}{rgb}{0.160665,0.478540,0.558115}%
\pgfsetfillcolor{currentfill}%
\pgfsetfillopacity{0.700000}%
\pgfsetlinewidth{0.501875pt}%
\definecolor{currentstroke}{rgb}{1.000000,1.000000,1.000000}%
\pgfsetstrokecolor{currentstroke}%
\pgfsetstrokeopacity{0.500000}%
\pgfsetdash{}{0pt}%
\pgfpathmoveto{\pgfqpoint{1.810366in}{2.060717in}}%
\pgfpathlineto{\pgfqpoint{1.822198in}{2.064457in}}%
\pgfpathlineto{\pgfqpoint{1.834024in}{2.068184in}}%
\pgfpathlineto{\pgfqpoint{1.845845in}{2.071898in}}%
\pgfpathlineto{\pgfqpoint{1.857660in}{2.075601in}}%
\pgfpathlineto{\pgfqpoint{1.869470in}{2.079294in}}%
\pgfpathlineto{\pgfqpoint{1.863537in}{2.088590in}}%
\pgfpathlineto{\pgfqpoint{1.857608in}{2.097854in}}%
\pgfpathlineto{\pgfqpoint{1.851684in}{2.107088in}}%
\pgfpathlineto{\pgfqpoint{1.845765in}{2.116291in}}%
\pgfpathlineto{\pgfqpoint{1.839850in}{2.125463in}}%
\pgfpathlineto{\pgfqpoint{1.828051in}{2.121826in}}%
\pgfpathlineto{\pgfqpoint{1.816247in}{2.118177in}}%
\pgfpathlineto{\pgfqpoint{1.804437in}{2.114518in}}%
\pgfpathlineto{\pgfqpoint{1.792621in}{2.110845in}}%
\pgfpathlineto{\pgfqpoint{1.780800in}{2.107159in}}%
\pgfpathlineto{\pgfqpoint{1.786704in}{2.097933in}}%
\pgfpathlineto{\pgfqpoint{1.792613in}{2.088676in}}%
\pgfpathlineto{\pgfqpoint{1.798526in}{2.079388in}}%
\pgfpathlineto{\pgfqpoint{1.804444in}{2.070068in}}%
\pgfpathclose%
\pgfusepath{stroke,fill}%
\end{pgfscope}%
\begin{pgfscope}%
\pgfpathrectangle{\pgfqpoint{0.887500in}{0.275000in}}{\pgfqpoint{4.225000in}{4.225000in}}%
\pgfusepath{clip}%
\pgfsetbuttcap%
\pgfsetroundjoin%
\definecolor{currentfill}{rgb}{0.269308,0.218818,0.509577}%
\pgfsetfillcolor{currentfill}%
\pgfsetfillopacity{0.700000}%
\pgfsetlinewidth{0.501875pt}%
\definecolor{currentstroke}{rgb}{1.000000,1.000000,1.000000}%
\pgfsetstrokecolor{currentstroke}%
\pgfsetstrokeopacity{0.500000}%
\pgfsetdash{}{0pt}%
\pgfpathmoveto{\pgfqpoint{3.647599in}{1.541291in}}%
\pgfpathlineto{\pgfqpoint{3.659016in}{1.547542in}}%
\pgfpathlineto{\pgfqpoint{3.670424in}{1.553556in}}%
\pgfpathlineto{\pgfqpoint{3.681828in}{1.559561in}}%
\pgfpathlineto{\pgfqpoint{3.693231in}{1.565785in}}%
\pgfpathlineto{\pgfqpoint{3.704638in}{1.572456in}}%
\pgfpathlineto{\pgfqpoint{3.698234in}{1.589081in}}%
\pgfpathlineto{\pgfqpoint{3.691821in}{1.604942in}}%
\pgfpathlineto{\pgfqpoint{3.685399in}{1.620173in}}%
\pgfpathlineto{\pgfqpoint{3.678972in}{1.634953in}}%
\pgfpathlineto{\pgfqpoint{3.672543in}{1.649462in}}%
\pgfpathlineto{\pgfqpoint{3.661042in}{1.636516in}}%
\pgfpathlineto{\pgfqpoint{3.649548in}{1.624020in}}%
\pgfpathlineto{\pgfqpoint{3.638064in}{1.612195in}}%
\pgfpathlineto{\pgfqpoint{3.626592in}{1.601260in}}%
\pgfpathlineto{\pgfqpoint{3.615134in}{1.591435in}}%
\pgfpathlineto{\pgfqpoint{3.621641in}{1.582893in}}%
\pgfpathlineto{\pgfqpoint{3.628146in}{1.573942in}}%
\pgfpathlineto{\pgfqpoint{3.634644in}{1.564240in}}%
\pgfpathlineto{\pgfqpoint{3.641130in}{1.553447in}}%
\pgfpathclose%
\pgfusepath{stroke,fill}%
\end{pgfscope}%
\begin{pgfscope}%
\pgfpathrectangle{\pgfqpoint{0.887500in}{0.275000in}}{\pgfqpoint{4.225000in}{4.225000in}}%
\pgfusepath{clip}%
\pgfsetbuttcap%
\pgfsetroundjoin%
\definecolor{currentfill}{rgb}{0.276194,0.190074,0.493001}%
\pgfsetfillcolor{currentfill}%
\pgfsetfillopacity{0.700000}%
\pgfsetlinewidth{0.501875pt}%
\definecolor{currentstroke}{rgb}{1.000000,1.000000,1.000000}%
\pgfsetstrokecolor{currentstroke}%
\pgfsetstrokeopacity{0.500000}%
\pgfsetdash{}{0pt}%
\pgfpathmoveto{\pgfqpoint{3.736546in}{1.481707in}}%
\pgfpathlineto{\pgfqpoint{3.747920in}{1.486750in}}%
\pgfpathlineto{\pgfqpoint{3.759313in}{1.493164in}}%
\pgfpathlineto{\pgfqpoint{3.770731in}{1.501123in}}%
\pgfpathlineto{\pgfqpoint{3.782173in}{1.510521in}}%
\pgfpathlineto{\pgfqpoint{3.793637in}{1.521227in}}%
\pgfpathlineto{\pgfqpoint{3.787271in}{1.540391in}}%
\pgfpathlineto{\pgfqpoint{3.780910in}{1.559777in}}%
\pgfpathlineto{\pgfqpoint{3.774553in}{1.579308in}}%
\pgfpathlineto{\pgfqpoint{3.768198in}{1.598912in}}%
\pgfpathlineto{\pgfqpoint{3.761844in}{1.618513in}}%
\pgfpathlineto{\pgfqpoint{3.750375in}{1.607482in}}%
\pgfpathlineto{\pgfqpoint{3.738919in}{1.597297in}}%
\pgfpathlineto{\pgfqpoint{3.727478in}{1.588046in}}%
\pgfpathlineto{\pgfqpoint{3.716052in}{1.579801in}}%
\pgfpathlineto{\pgfqpoint{3.704638in}{1.572456in}}%
\pgfpathlineto{\pgfqpoint{3.711031in}{1.555141in}}%
\pgfpathlineto{\pgfqpoint{3.717416in}{1.537274in}}%
\pgfpathlineto{\pgfqpoint{3.723796in}{1.518988in}}%
\pgfpathlineto{\pgfqpoint{3.730172in}{1.500421in}}%
\pgfpathclose%
\pgfusepath{stroke,fill}%
\end{pgfscope}%
\begin{pgfscope}%
\pgfpathrectangle{\pgfqpoint{0.887500in}{0.275000in}}{\pgfqpoint{4.225000in}{4.225000in}}%
\pgfusepath{clip}%
\pgfsetbuttcap%
\pgfsetroundjoin%
\definecolor{currentfill}{rgb}{0.239346,0.300855,0.540844}%
\pgfsetfillcolor{currentfill}%
\pgfsetfillopacity{0.700000}%
\pgfsetlinewidth{0.501875pt}%
\definecolor{currentstroke}{rgb}{1.000000,1.000000,1.000000}%
\pgfsetstrokecolor{currentstroke}%
\pgfsetstrokeopacity{0.500000}%
\pgfsetdash{}{0pt}%
\pgfpathmoveto{\pgfqpoint{3.143794in}{1.700186in}}%
\pgfpathlineto{\pgfqpoint{3.155300in}{1.704121in}}%
\pgfpathlineto{\pgfqpoint{3.166800in}{1.708141in}}%
\pgfpathlineto{\pgfqpoint{3.178295in}{1.712262in}}%
\pgfpathlineto{\pgfqpoint{3.189785in}{1.716439in}}%
\pgfpathlineto{\pgfqpoint{3.201269in}{1.720585in}}%
\pgfpathlineto{\pgfqpoint{3.194916in}{1.732039in}}%
\pgfpathlineto{\pgfqpoint{3.188566in}{1.743393in}}%
\pgfpathlineto{\pgfqpoint{3.182219in}{1.754644in}}%
\pgfpathlineto{\pgfqpoint{3.175876in}{1.765785in}}%
\pgfpathlineto{\pgfqpoint{3.169535in}{1.776811in}}%
\pgfpathlineto{\pgfqpoint{3.158059in}{1.772651in}}%
\pgfpathlineto{\pgfqpoint{3.146577in}{1.768458in}}%
\pgfpathlineto{\pgfqpoint{3.135091in}{1.764337in}}%
\pgfpathlineto{\pgfqpoint{3.123599in}{1.760342in}}%
\pgfpathlineto{\pgfqpoint{3.112101in}{1.756454in}}%
\pgfpathlineto{\pgfqpoint{3.118433in}{1.745319in}}%
\pgfpathlineto{\pgfqpoint{3.124768in}{1.734132in}}%
\pgfpathlineto{\pgfqpoint{3.131107in}{1.722886in}}%
\pgfpathlineto{\pgfqpoint{3.137449in}{1.711573in}}%
\pgfpathclose%
\pgfusepath{stroke,fill}%
\end{pgfscope}%
\begin{pgfscope}%
\pgfpathrectangle{\pgfqpoint{0.887500in}{0.275000in}}{\pgfqpoint{4.225000in}{4.225000in}}%
\pgfusepath{clip}%
\pgfsetbuttcap%
\pgfsetroundjoin%
\definecolor{currentfill}{rgb}{0.241237,0.296485,0.539709}%
\pgfsetfillcolor{currentfill}%
\pgfsetfillopacity{0.700000}%
\pgfsetlinewidth{0.501875pt}%
\definecolor{currentstroke}{rgb}{1.000000,1.000000,1.000000}%
\pgfsetstrokecolor{currentstroke}%
\pgfsetstrokeopacity{0.500000}%
\pgfsetdash{}{0pt}%
\pgfpathmoveto{\pgfqpoint{4.145277in}{1.658557in}}%
\pgfpathlineto{\pgfqpoint{4.156724in}{1.669270in}}%
\pgfpathlineto{\pgfqpoint{4.168160in}{1.679671in}}%
\pgfpathlineto{\pgfqpoint{4.179583in}{1.689682in}}%
\pgfpathlineto{\pgfqpoint{4.190991in}{1.699226in}}%
\pgfpathlineto{\pgfqpoint{4.202381in}{1.708226in}}%
\pgfpathlineto{\pgfqpoint{4.195971in}{1.726906in}}%
\pgfpathlineto{\pgfqpoint{4.189530in}{1.744467in}}%
\pgfpathlineto{\pgfqpoint{4.183061in}{1.761027in}}%
\pgfpathlineto{\pgfqpoint{4.176569in}{1.776703in}}%
\pgfpathlineto{\pgfqpoint{4.170057in}{1.791610in}}%
\pgfpathlineto{\pgfqpoint{4.158715in}{1.784135in}}%
\pgfpathlineto{\pgfqpoint{4.147350in}{1.775987in}}%
\pgfpathlineto{\pgfqpoint{4.135966in}{1.767258in}}%
\pgfpathlineto{\pgfqpoint{4.124565in}{1.758038in}}%
\pgfpathlineto{\pgfqpoint{4.113152in}{1.748421in}}%
\pgfpathlineto{\pgfqpoint{4.119617in}{1.731992in}}%
\pgfpathlineto{\pgfqpoint{4.126064in}{1.714885in}}%
\pgfpathlineto{\pgfqpoint{4.132492in}{1.697006in}}%
\pgfpathlineto{\pgfqpoint{4.138898in}{1.678261in}}%
\pgfpathclose%
\pgfusepath{stroke,fill}%
\end{pgfscope}%
\begin{pgfscope}%
\pgfpathrectangle{\pgfqpoint{0.887500in}{0.275000in}}{\pgfqpoint{4.225000in}{4.225000in}}%
\pgfusepath{clip}%
\pgfsetbuttcap%
\pgfsetroundjoin%
\definecolor{currentfill}{rgb}{0.246811,0.283237,0.535941}%
\pgfsetfillcolor{currentfill}%
\pgfsetfillopacity{0.700000}%
\pgfsetlinewidth{0.501875pt}%
\definecolor{currentstroke}{rgb}{1.000000,1.000000,1.000000}%
\pgfsetstrokecolor{currentstroke}%
\pgfsetstrokeopacity{0.500000}%
\pgfsetdash{}{0pt}%
\pgfpathmoveto{\pgfqpoint{3.233080in}{1.662005in}}%
\pgfpathlineto{\pgfqpoint{3.244566in}{1.666023in}}%
\pgfpathlineto{\pgfqpoint{3.256046in}{1.669947in}}%
\pgfpathlineto{\pgfqpoint{3.267519in}{1.673733in}}%
\pgfpathlineto{\pgfqpoint{3.278986in}{1.677337in}}%
\pgfpathlineto{\pgfqpoint{3.290446in}{1.680734in}}%
\pgfpathlineto{\pgfqpoint{3.284069in}{1.692316in}}%
\pgfpathlineto{\pgfqpoint{3.277695in}{1.703810in}}%
\pgfpathlineto{\pgfqpoint{3.271324in}{1.715220in}}%
\pgfpathlineto{\pgfqpoint{3.264956in}{1.726555in}}%
\pgfpathlineto{\pgfqpoint{3.258592in}{1.737825in}}%
\pgfpathlineto{\pgfqpoint{3.247142in}{1.735113in}}%
\pgfpathlineto{\pgfqpoint{3.235685in}{1.731964in}}%
\pgfpathlineto{\pgfqpoint{3.224220in}{1.728435in}}%
\pgfpathlineto{\pgfqpoint{3.212748in}{1.724613in}}%
\pgfpathlineto{\pgfqpoint{3.201269in}{1.720585in}}%
\pgfpathlineto{\pgfqpoint{3.207626in}{1.709037in}}%
\pgfpathlineto{\pgfqpoint{3.213985in}{1.697400in}}%
\pgfpathlineto{\pgfqpoint{3.220347in}{1.685679in}}%
\pgfpathlineto{\pgfqpoint{3.226712in}{1.673879in}}%
\pgfpathclose%
\pgfusepath{stroke,fill}%
\end{pgfscope}%
\begin{pgfscope}%
\pgfpathrectangle{\pgfqpoint{0.887500in}{0.275000in}}{\pgfqpoint{4.225000in}{4.225000in}}%
\pgfusepath{clip}%
\pgfsetbuttcap%
\pgfsetroundjoin%
\definecolor{currentfill}{rgb}{0.279574,0.170599,0.479997}%
\pgfsetfillcolor{currentfill}%
\pgfsetfillopacity{0.700000}%
\pgfsetlinewidth{0.501875pt}%
\definecolor{currentstroke}{rgb}{1.000000,1.000000,1.000000}%
\pgfsetstrokecolor{currentstroke}%
\pgfsetstrokeopacity{0.500000}%
\pgfsetdash{}{0pt}%
\pgfpathmoveto{\pgfqpoint{3.825599in}{1.431334in}}%
\pgfpathlineto{\pgfqpoint{3.837095in}{1.443469in}}%
\pgfpathlineto{\pgfqpoint{3.848612in}{1.456718in}}%
\pgfpathlineto{\pgfqpoint{3.860151in}{1.470981in}}%
\pgfpathlineto{\pgfqpoint{3.871709in}{1.486151in}}%
\pgfpathlineto{\pgfqpoint{3.883283in}{1.502034in}}%
\pgfpathlineto{\pgfqpoint{3.876862in}{1.519095in}}%
\pgfpathlineto{\pgfqpoint{3.870448in}{1.536405in}}%
\pgfpathlineto{\pgfqpoint{3.864041in}{1.553958in}}%
\pgfpathlineto{\pgfqpoint{3.857640in}{1.571751in}}%
\pgfpathlineto{\pgfqpoint{3.851245in}{1.589778in}}%
\pgfpathlineto{\pgfqpoint{3.839692in}{1.574532in}}%
\pgfpathlineto{\pgfqpoint{3.828152in}{1.559895in}}%
\pgfpathlineto{\pgfqpoint{3.816629in}{1.546044in}}%
\pgfpathlineto{\pgfqpoint{3.805123in}{1.533112in}}%
\pgfpathlineto{\pgfqpoint{3.793637in}{1.521227in}}%
\pgfpathlineto{\pgfqpoint{3.800010in}{1.502359in}}%
\pgfpathlineto{\pgfqpoint{3.806391in}{1.483862in}}%
\pgfpathlineto{\pgfqpoint{3.812781in}{1.465809in}}%
\pgfpathlineto{\pgfqpoint{3.819184in}{1.448275in}}%
\pgfpathclose%
\pgfusepath{stroke,fill}%
\end{pgfscope}%
\begin{pgfscope}%
\pgfpathrectangle{\pgfqpoint{0.887500in}{0.275000in}}{\pgfqpoint{4.225000in}{4.225000in}}%
\pgfusepath{clip}%
\pgfsetbuttcap%
\pgfsetroundjoin%
\definecolor{currentfill}{rgb}{0.179019,0.433756,0.557430}%
\pgfsetfillcolor{currentfill}%
\pgfsetfillopacity{0.700000}%
\pgfsetlinewidth{0.501875pt}%
\definecolor{currentstroke}{rgb}{1.000000,1.000000,1.000000}%
\pgfsetstrokecolor{currentstroke}%
\pgfsetstrokeopacity{0.500000}%
\pgfsetdash{}{0pt}%
\pgfpathmoveto{\pgfqpoint{2.224911in}{1.963708in}}%
\pgfpathlineto{\pgfqpoint{2.236644in}{1.967492in}}%
\pgfpathlineto{\pgfqpoint{2.248370in}{1.971265in}}%
\pgfpathlineto{\pgfqpoint{2.260092in}{1.975029in}}%
\pgfpathlineto{\pgfqpoint{2.271807in}{1.978785in}}%
\pgfpathlineto{\pgfqpoint{2.283517in}{1.982536in}}%
\pgfpathlineto{\pgfqpoint{2.277442in}{1.992242in}}%
\pgfpathlineto{\pgfqpoint{2.271372in}{2.001907in}}%
\pgfpathlineto{\pgfqpoint{2.265307in}{2.011532in}}%
\pgfpathlineto{\pgfqpoint{2.259245in}{2.021116in}}%
\pgfpathlineto{\pgfqpoint{2.253189in}{2.030661in}}%
\pgfpathlineto{\pgfqpoint{2.241489in}{2.026975in}}%
\pgfpathlineto{\pgfqpoint{2.229783in}{2.023283in}}%
\pgfpathlineto{\pgfqpoint{2.218072in}{2.019583in}}%
\pgfpathlineto{\pgfqpoint{2.206355in}{2.015874in}}%
\pgfpathlineto{\pgfqpoint{2.194632in}{2.012155in}}%
\pgfpathlineto{\pgfqpoint{2.200679in}{2.002547in}}%
\pgfpathlineto{\pgfqpoint{2.206730in}{1.992898in}}%
\pgfpathlineto{\pgfqpoint{2.212786in}{1.983209in}}%
\pgfpathlineto{\pgfqpoint{2.218846in}{1.973479in}}%
\pgfpathclose%
\pgfusepath{stroke,fill}%
\end{pgfscope}%
\begin{pgfscope}%
\pgfpathrectangle{\pgfqpoint{0.887500in}{0.275000in}}{\pgfqpoint{4.225000in}{4.225000in}}%
\pgfusepath{clip}%
\pgfsetbuttcap%
\pgfsetroundjoin%
\definecolor{currentfill}{rgb}{0.201239,0.383670,0.554294}%
\pgfsetfillcolor{currentfill}%
\pgfsetfillopacity{0.700000}%
\pgfsetlinewidth{0.501875pt}%
\definecolor{currentstroke}{rgb}{1.000000,1.000000,1.000000}%
\pgfsetstrokecolor{currentstroke}%
\pgfsetstrokeopacity{0.500000}%
\pgfsetdash{}{0pt}%
\pgfpathmoveto{\pgfqpoint{2.639727in}{1.857758in}}%
\pgfpathlineto{\pgfqpoint{2.651358in}{1.861546in}}%
\pgfpathlineto{\pgfqpoint{2.662984in}{1.865324in}}%
\pgfpathlineto{\pgfqpoint{2.674604in}{1.869094in}}%
\pgfpathlineto{\pgfqpoint{2.686218in}{1.872860in}}%
\pgfpathlineto{\pgfqpoint{2.697827in}{1.876626in}}%
\pgfpathlineto{\pgfqpoint{2.691618in}{1.886961in}}%
\pgfpathlineto{\pgfqpoint{2.685413in}{1.897248in}}%
\pgfpathlineto{\pgfqpoint{2.679212in}{1.907488in}}%
\pgfpathlineto{\pgfqpoint{2.673015in}{1.917681in}}%
\pgfpathlineto{\pgfqpoint{2.666822in}{1.927827in}}%
\pgfpathlineto{\pgfqpoint{2.655223in}{1.924120in}}%
\pgfpathlineto{\pgfqpoint{2.643618in}{1.920412in}}%
\pgfpathlineto{\pgfqpoint{2.632007in}{1.916701in}}%
\pgfpathlineto{\pgfqpoint{2.620391in}{1.912982in}}%
\pgfpathlineto{\pgfqpoint{2.608769in}{1.909252in}}%
\pgfpathlineto{\pgfqpoint{2.614952in}{1.899047in}}%
\pgfpathlineto{\pgfqpoint{2.621140in}{1.888795in}}%
\pgfpathlineto{\pgfqpoint{2.627331in}{1.878497in}}%
\pgfpathlineto{\pgfqpoint{2.633527in}{1.868151in}}%
\pgfpathclose%
\pgfusepath{stroke,fill}%
\end{pgfscope}%
\begin{pgfscope}%
\pgfpathrectangle{\pgfqpoint{0.887500in}{0.275000in}}{\pgfqpoint{4.225000in}{4.225000in}}%
\pgfusepath{clip}%
\pgfsetbuttcap%
\pgfsetroundjoin%
\definecolor{currentfill}{rgb}{0.255645,0.260703,0.528312}%
\pgfsetfillcolor{currentfill}%
\pgfsetfillopacity{0.700000}%
\pgfsetlinewidth{0.501875pt}%
\definecolor{currentstroke}{rgb}{1.000000,1.000000,1.000000}%
\pgfsetstrokecolor{currentstroke}%
\pgfsetstrokeopacity{0.500000}%
\pgfsetdash{}{0pt}%
\pgfpathmoveto{\pgfqpoint{3.322372in}{1.621433in}}%
\pgfpathlineto{\pgfqpoint{3.333835in}{1.625107in}}%
\pgfpathlineto{\pgfqpoint{3.345291in}{1.628781in}}%
\pgfpathlineto{\pgfqpoint{3.356743in}{1.632529in}}%
\pgfpathlineto{\pgfqpoint{3.368190in}{1.636424in}}%
\pgfpathlineto{\pgfqpoint{3.379634in}{1.640539in}}%
\pgfpathlineto{\pgfqpoint{3.373237in}{1.652596in}}%
\pgfpathlineto{\pgfqpoint{3.366842in}{1.664476in}}%
\pgfpathlineto{\pgfqpoint{3.360448in}{1.676171in}}%
\pgfpathlineto{\pgfqpoint{3.354057in}{1.687668in}}%
\pgfpathlineto{\pgfqpoint{3.347667in}{1.698957in}}%
\pgfpathlineto{\pgfqpoint{3.336229in}{1.694653in}}%
\pgfpathlineto{\pgfqpoint{3.324790in}{1.690827in}}%
\pgfpathlineto{\pgfqpoint{3.313347in}{1.687330in}}%
\pgfpathlineto{\pgfqpoint{3.301899in}{1.684016in}}%
\pgfpathlineto{\pgfqpoint{3.290446in}{1.680734in}}%
\pgfpathlineto{\pgfqpoint{3.296826in}{1.669062in}}%
\pgfpathlineto{\pgfqpoint{3.303208in}{1.657298in}}%
\pgfpathlineto{\pgfqpoint{3.309594in}{1.645440in}}%
\pgfpathlineto{\pgfqpoint{3.315982in}{1.633486in}}%
\pgfpathclose%
\pgfusepath{stroke,fill}%
\end{pgfscope}%
\begin{pgfscope}%
\pgfpathrectangle{\pgfqpoint{0.887500in}{0.275000in}}{\pgfqpoint{4.225000in}{4.225000in}}%
\pgfusepath{clip}%
\pgfsetbuttcap%
\pgfsetroundjoin%
\definecolor{currentfill}{rgb}{0.262138,0.242286,0.520837}%
\pgfsetfillcolor{currentfill}%
\pgfsetfillopacity{0.700000}%
\pgfsetlinewidth{0.501875pt}%
\definecolor{currentstroke}{rgb}{1.000000,1.000000,1.000000}%
\pgfsetstrokecolor{currentstroke}%
\pgfsetstrokeopacity{0.500000}%
\pgfsetdash{}{0pt}%
\pgfpathmoveto{\pgfqpoint{4.176675in}{1.542385in}}%
\pgfpathlineto{\pgfqpoint{4.188101in}{1.552534in}}%
\pgfpathlineto{\pgfqpoint{4.199526in}{1.562722in}}%
\pgfpathlineto{\pgfqpoint{4.210952in}{1.572998in}}%
\pgfpathlineto{\pgfqpoint{4.222381in}{1.583411in}}%
\pgfpathlineto{\pgfqpoint{4.233813in}{1.594011in}}%
\pgfpathlineto{\pgfqpoint{4.227623in}{1.620012in}}%
\pgfpathlineto{\pgfqpoint{4.221380in}{1.644320in}}%
\pgfpathlineto{\pgfqpoint{4.215089in}{1.667049in}}%
\pgfpathlineto{\pgfqpoint{4.208755in}{1.688312in}}%
\pgfpathlineto{\pgfqpoint{4.202381in}{1.708226in}}%
\pgfpathlineto{\pgfqpoint{4.190991in}{1.699226in}}%
\pgfpathlineto{\pgfqpoint{4.179583in}{1.689682in}}%
\pgfpathlineto{\pgfqpoint{4.168160in}{1.679671in}}%
\pgfpathlineto{\pgfqpoint{4.156724in}{1.669270in}}%
\pgfpathlineto{\pgfqpoint{4.145277in}{1.658557in}}%
\pgfpathlineto{\pgfqpoint{4.151628in}{1.637799in}}%
\pgfpathlineto{\pgfqpoint{4.157946in}{1.615896in}}%
\pgfpathlineto{\pgfqpoint{4.164229in}{1.592754in}}%
\pgfpathlineto{\pgfqpoint{4.170473in}{1.568281in}}%
\pgfpathclose%
\pgfusepath{stroke,fill}%
\end{pgfscope}%
\begin{pgfscope}%
\pgfpathrectangle{\pgfqpoint{0.887500in}{0.275000in}}{\pgfqpoint{4.225000in}{4.225000in}}%
\pgfusepath{clip}%
\pgfsetbuttcap%
\pgfsetroundjoin%
\definecolor{currentfill}{rgb}{0.271828,0.209303,0.504434}%
\pgfsetfillcolor{currentfill}%
\pgfsetfillopacity{0.700000}%
\pgfsetlinewidth{0.501875pt}%
\definecolor{currentstroke}{rgb}{1.000000,1.000000,1.000000}%
\pgfsetstrokecolor{currentstroke}%
\pgfsetstrokeopacity{0.500000}%
\pgfsetdash{}{0pt}%
\pgfpathmoveto{\pgfqpoint{3.500929in}{1.535058in}}%
\pgfpathlineto{\pgfqpoint{3.512359in}{1.539829in}}%
\pgfpathlineto{\pgfqpoint{3.523785in}{1.544585in}}%
\pgfpathlineto{\pgfqpoint{3.535204in}{1.549267in}}%
\pgfpathlineto{\pgfqpoint{3.546618in}{1.553875in}}%
\pgfpathlineto{\pgfqpoint{3.558027in}{1.558591in}}%
\pgfpathlineto{\pgfqpoint{3.551556in}{1.568456in}}%
\pgfpathlineto{\pgfqpoint{3.545091in}{1.578504in}}%
\pgfpathlineto{\pgfqpoint{3.538635in}{1.588932in}}%
\pgfpathlineto{\pgfqpoint{3.532190in}{1.599937in}}%
\pgfpathlineto{\pgfqpoint{3.525758in}{1.611714in}}%
\pgfpathlineto{\pgfqpoint{3.514391in}{1.609860in}}%
\pgfpathlineto{\pgfqpoint{3.503016in}{1.607899in}}%
\pgfpathlineto{\pgfqpoint{3.491628in}{1.605340in}}%
\pgfpathlineto{\pgfqpoint{3.480227in}{1.602172in}}%
\pgfpathlineto{\pgfqpoint{3.468815in}{1.598532in}}%
\pgfpathlineto{\pgfqpoint{3.475230in}{1.585664in}}%
\pgfpathlineto{\pgfqpoint{3.481649in}{1.572908in}}%
\pgfpathlineto{\pgfqpoint{3.488072in}{1.560240in}}%
\pgfpathlineto{\pgfqpoint{3.494498in}{1.547632in}}%
\pgfpathclose%
\pgfusepath{stroke,fill}%
\end{pgfscope}%
\begin{pgfscope}%
\pgfpathrectangle{\pgfqpoint{0.887500in}{0.275000in}}{\pgfqpoint{4.225000in}{4.225000in}}%
\pgfusepath{clip}%
\pgfsetbuttcap%
\pgfsetroundjoin%
\definecolor{currentfill}{rgb}{0.243113,0.292092,0.538516}%
\pgfsetfillcolor{currentfill}%
\pgfsetfillopacity{0.700000}%
\pgfsetlinewidth{0.501875pt}%
\definecolor{currentstroke}{rgb}{1.000000,1.000000,1.000000}%
\pgfsetstrokecolor{currentstroke}%
\pgfsetstrokeopacity{0.500000}%
\pgfsetdash{}{0pt}%
\pgfpathmoveto{\pgfqpoint{3.998702in}{1.644000in}}%
\pgfpathlineto{\pgfqpoint{4.010161in}{1.654703in}}%
\pgfpathlineto{\pgfqpoint{4.021615in}{1.665257in}}%
\pgfpathlineto{\pgfqpoint{4.033067in}{1.675821in}}%
\pgfpathlineto{\pgfqpoint{4.044519in}{1.686423in}}%
\pgfpathlineto{\pgfqpoint{4.055969in}{1.697020in}}%
\pgfpathlineto{\pgfqpoint{4.049535in}{1.714386in}}%
\pgfpathlineto{\pgfqpoint{4.043094in}{1.731395in}}%
\pgfpathlineto{\pgfqpoint{4.036648in}{1.748106in}}%
\pgfpathlineto{\pgfqpoint{4.030197in}{1.764578in}}%
\pgfpathlineto{\pgfqpoint{4.023744in}{1.780870in}}%
\pgfpathlineto{\pgfqpoint{4.012328in}{1.771459in}}%
\pgfpathlineto{\pgfqpoint{4.000902in}{1.761736in}}%
\pgfpathlineto{\pgfqpoint{3.989466in}{1.751687in}}%
\pgfpathlineto{\pgfqpoint{3.978020in}{1.741279in}}%
\pgfpathlineto{\pgfqpoint{3.966563in}{1.730441in}}%
\pgfpathlineto{\pgfqpoint{3.972996in}{1.713531in}}%
\pgfpathlineto{\pgfqpoint{3.979429in}{1.696492in}}%
\pgfpathlineto{\pgfqpoint{3.985858in}{1.679263in}}%
\pgfpathlineto{\pgfqpoint{3.992283in}{1.661785in}}%
\pgfpathclose%
\pgfusepath{stroke,fill}%
\end{pgfscope}%
\begin{pgfscope}%
\pgfpathrectangle{\pgfqpoint{0.887500in}{0.275000in}}{\pgfqpoint{4.225000in}{4.225000in}}%
\pgfusepath{clip}%
\pgfsetbuttcap%
\pgfsetroundjoin%
\definecolor{currentfill}{rgb}{0.263663,0.237631,0.518762}%
\pgfsetfillcolor{currentfill}%
\pgfsetfillopacity{0.700000}%
\pgfsetlinewidth{0.501875pt}%
\definecolor{currentstroke}{rgb}{1.000000,1.000000,1.000000}%
\pgfsetstrokecolor{currentstroke}%
\pgfsetstrokeopacity{0.500000}%
\pgfsetdash{}{0pt}%
\pgfpathmoveto{\pgfqpoint{3.411649in}{1.578006in}}%
\pgfpathlineto{\pgfqpoint{3.423092in}{1.581989in}}%
\pgfpathlineto{\pgfqpoint{3.434531in}{1.586149in}}%
\pgfpathlineto{\pgfqpoint{3.445966in}{1.590384in}}%
\pgfpathlineto{\pgfqpoint{3.457394in}{1.594557in}}%
\pgfpathlineto{\pgfqpoint{3.468815in}{1.598532in}}%
\pgfpathlineto{\pgfqpoint{3.462405in}{1.611541in}}%
\pgfpathlineto{\pgfqpoint{3.456000in}{1.624712in}}%
\pgfpathlineto{\pgfqpoint{3.449599in}{1.638027in}}%
\pgfpathlineto{\pgfqpoint{3.443202in}{1.651433in}}%
\pgfpathlineto{\pgfqpoint{3.436808in}{1.664875in}}%
\pgfpathlineto{\pgfqpoint{3.425381in}{1.659793in}}%
\pgfpathlineto{\pgfqpoint{3.413949in}{1.654688in}}%
\pgfpathlineto{\pgfqpoint{3.402513in}{1.649695in}}%
\pgfpathlineto{\pgfqpoint{3.391075in}{1.644948in}}%
\pgfpathlineto{\pgfqpoint{3.379634in}{1.640539in}}%
\pgfpathlineto{\pgfqpoint{3.386033in}{1.628318in}}%
\pgfpathlineto{\pgfqpoint{3.392434in}{1.615943in}}%
\pgfpathlineto{\pgfqpoint{3.398837in}{1.603425in}}%
\pgfpathlineto{\pgfqpoint{3.405242in}{1.590775in}}%
\pgfpathclose%
\pgfusepath{stroke,fill}%
\end{pgfscope}%
\begin{pgfscope}%
\pgfpathrectangle{\pgfqpoint{0.887500in}{0.275000in}}{\pgfqpoint{4.225000in}{4.225000in}}%
\pgfusepath{clip}%
\pgfsetbuttcap%
\pgfsetroundjoin%
\definecolor{currentfill}{rgb}{0.165117,0.467423,0.558141}%
\pgfsetfillcolor{currentfill}%
\pgfsetfillopacity{0.700000}%
\pgfsetlinewidth{0.501875pt}%
\definecolor{currentstroke}{rgb}{1.000000,1.000000,1.000000}%
\pgfsetstrokecolor{currentstroke}%
\pgfsetstrokeopacity{0.500000}%
\pgfsetdash{}{0pt}%
\pgfpathmoveto{\pgfqpoint{1.899203in}{2.032344in}}%
\pgfpathlineto{\pgfqpoint{1.911017in}{2.036083in}}%
\pgfpathlineto{\pgfqpoint{1.922826in}{2.039816in}}%
\pgfpathlineto{\pgfqpoint{1.934629in}{2.043542in}}%
\pgfpathlineto{\pgfqpoint{1.946427in}{2.047263in}}%
\pgfpathlineto{\pgfqpoint{1.958219in}{2.050979in}}%
\pgfpathlineto{\pgfqpoint{1.952253in}{2.060375in}}%
\pgfpathlineto{\pgfqpoint{1.946291in}{2.069740in}}%
\pgfpathlineto{\pgfqpoint{1.940334in}{2.079072in}}%
\pgfpathlineto{\pgfqpoint{1.934381in}{2.088373in}}%
\pgfpathlineto{\pgfqpoint{1.928433in}{2.097643in}}%
\pgfpathlineto{\pgfqpoint{1.916652in}{2.093985in}}%
\pgfpathlineto{\pgfqpoint{1.904865in}{2.090322in}}%
\pgfpathlineto{\pgfqpoint{1.893072in}{2.086653in}}%
\pgfpathlineto{\pgfqpoint{1.881274in}{2.082977in}}%
\pgfpathlineto{\pgfqpoint{1.869470in}{2.079294in}}%
\pgfpathlineto{\pgfqpoint{1.875407in}{2.069967in}}%
\pgfpathlineto{\pgfqpoint{1.881349in}{2.060609in}}%
\pgfpathlineto{\pgfqpoint{1.887296in}{2.051219in}}%
\pgfpathlineto{\pgfqpoint{1.893247in}{2.041797in}}%
\pgfpathclose%
\pgfusepath{stroke,fill}%
\end{pgfscope}%
\begin{pgfscope}%
\pgfpathrectangle{\pgfqpoint{0.887500in}{0.275000in}}{\pgfqpoint{4.225000in}{4.225000in}}%
\pgfusepath{clip}%
\pgfsetbuttcap%
\pgfsetroundjoin%
\definecolor{currentfill}{rgb}{0.269308,0.218818,0.509577}%
\pgfsetfillcolor{currentfill}%
\pgfsetfillopacity{0.700000}%
\pgfsetlinewidth{0.501875pt}%
\definecolor{currentstroke}{rgb}{1.000000,1.000000,1.000000}%
\pgfsetstrokecolor{currentstroke}%
\pgfsetstrokeopacity{0.500000}%
\pgfsetdash{}{0pt}%
\pgfpathmoveto{\pgfqpoint{3.883283in}{1.502034in}}%
\pgfpathlineto{\pgfqpoint{3.894869in}{1.518345in}}%
\pgfpathlineto{\pgfqpoint{3.906459in}{1.534794in}}%
\pgfpathlineto{\pgfqpoint{3.918048in}{1.551089in}}%
\pgfpathlineto{\pgfqpoint{3.929628in}{1.566938in}}%
\pgfpathlineto{\pgfqpoint{3.941193in}{1.582049in}}%
\pgfpathlineto{\pgfqpoint{3.934767in}{1.599331in}}%
\pgfpathlineto{\pgfqpoint{3.928339in}{1.616420in}}%
\pgfpathlineto{\pgfqpoint{3.921910in}{1.633380in}}%
\pgfpathlineto{\pgfqpoint{3.915481in}{1.650274in}}%
\pgfpathlineto{\pgfqpoint{3.909053in}{1.667166in}}%
\pgfpathlineto{\pgfqpoint{3.897503in}{1.652298in}}%
\pgfpathlineto{\pgfqpoint{3.885941in}{1.636896in}}%
\pgfpathlineto{\pgfqpoint{3.874374in}{1.621189in}}%
\pgfpathlineto{\pgfqpoint{3.862807in}{1.605407in}}%
\pgfpathlineto{\pgfqpoint{3.851245in}{1.589778in}}%
\pgfpathlineto{\pgfqpoint{3.857640in}{1.571751in}}%
\pgfpathlineto{\pgfqpoint{3.864041in}{1.553958in}}%
\pgfpathlineto{\pgfqpoint{3.870448in}{1.536405in}}%
\pgfpathlineto{\pgfqpoint{3.876862in}{1.519095in}}%
\pgfpathclose%
\pgfusepath{stroke,fill}%
\end{pgfscope}%
\begin{pgfscope}%
\pgfpathrectangle{\pgfqpoint{0.887500in}{0.275000in}}{\pgfqpoint{4.225000in}{4.225000in}}%
\pgfusepath{clip}%
\pgfsetbuttcap%
\pgfsetroundjoin%
\definecolor{currentfill}{rgb}{0.208623,0.367752,0.552675}%
\pgfsetfillcolor{currentfill}%
\pgfsetfillopacity{0.700000}%
\pgfsetlinewidth{0.501875pt}%
\definecolor{currentstroke}{rgb}{1.000000,1.000000,1.000000}%
\pgfsetstrokecolor{currentstroke}%
\pgfsetstrokeopacity{0.500000}%
\pgfsetdash{}{0pt}%
\pgfpathmoveto{\pgfqpoint{2.728933in}{1.824209in}}%
\pgfpathlineto{\pgfqpoint{2.740545in}{1.828036in}}%
\pgfpathlineto{\pgfqpoint{2.752152in}{1.831867in}}%
\pgfpathlineto{\pgfqpoint{2.763752in}{1.835703in}}%
\pgfpathlineto{\pgfqpoint{2.775347in}{1.839546in}}%
\pgfpathlineto{\pgfqpoint{2.786936in}{1.843397in}}%
\pgfpathlineto{\pgfqpoint{2.780698in}{1.853923in}}%
\pgfpathlineto{\pgfqpoint{2.774463in}{1.864398in}}%
\pgfpathlineto{\pgfqpoint{2.768233in}{1.874823in}}%
\pgfpathlineto{\pgfqpoint{2.762007in}{1.885199in}}%
\pgfpathlineto{\pgfqpoint{2.755784in}{1.895526in}}%
\pgfpathlineto{\pgfqpoint{2.744204in}{1.891730in}}%
\pgfpathlineto{\pgfqpoint{2.732618in}{1.887944in}}%
\pgfpathlineto{\pgfqpoint{2.721027in}{1.884166in}}%
\pgfpathlineto{\pgfqpoint{2.709430in}{1.880394in}}%
\pgfpathlineto{\pgfqpoint{2.697827in}{1.876626in}}%
\pgfpathlineto{\pgfqpoint{2.704040in}{1.866242in}}%
\pgfpathlineto{\pgfqpoint{2.710257in}{1.855809in}}%
\pgfpathlineto{\pgfqpoint{2.716479in}{1.845326in}}%
\pgfpathlineto{\pgfqpoint{2.722704in}{1.834793in}}%
\pgfpathclose%
\pgfusepath{stroke,fill}%
\end{pgfscope}%
\begin{pgfscope}%
\pgfpathrectangle{\pgfqpoint{0.887500in}{0.275000in}}{\pgfqpoint{4.225000in}{4.225000in}}%
\pgfusepath{clip}%
\pgfsetbuttcap%
\pgfsetroundjoin%
\definecolor{currentfill}{rgb}{0.252194,0.269783,0.531579}%
\pgfsetfillcolor{currentfill}%
\pgfsetfillopacity{0.700000}%
\pgfsetlinewidth{0.501875pt}%
\definecolor{currentstroke}{rgb}{1.000000,1.000000,1.000000}%
\pgfsetstrokecolor{currentstroke}%
\pgfsetstrokeopacity{0.500000}%
\pgfsetdash{}{0pt}%
\pgfpathmoveto{\pgfqpoint{4.087962in}{1.602778in}}%
\pgfpathlineto{\pgfqpoint{4.099430in}{1.614023in}}%
\pgfpathlineto{\pgfqpoint{4.110897in}{1.625282in}}%
\pgfpathlineto{\pgfqpoint{4.122362in}{1.636496in}}%
\pgfpathlineto{\pgfqpoint{4.133822in}{1.647608in}}%
\pgfpathlineto{\pgfqpoint{4.145277in}{1.658557in}}%
\pgfpathlineto{\pgfqpoint{4.138898in}{1.678261in}}%
\pgfpathlineto{\pgfqpoint{4.132492in}{1.697006in}}%
\pgfpathlineto{\pgfqpoint{4.126064in}{1.714885in}}%
\pgfpathlineto{\pgfqpoint{4.119617in}{1.731992in}}%
\pgfpathlineto{\pgfqpoint{4.113152in}{1.748421in}}%
\pgfpathlineto{\pgfqpoint{4.101728in}{1.738498in}}%
\pgfpathlineto{\pgfqpoint{4.090297in}{1.728350in}}%
\pgfpathlineto{\pgfqpoint{4.078859in}{1.718026in}}%
\pgfpathlineto{\pgfqpoint{4.067416in}{1.707568in}}%
\pgfpathlineto{\pgfqpoint{4.055969in}{1.697020in}}%
\pgfpathlineto{\pgfqpoint{4.062393in}{1.679238in}}%
\pgfpathlineto{\pgfqpoint{4.068806in}{1.660981in}}%
\pgfpathlineto{\pgfqpoint{4.075207in}{1.642191in}}%
\pgfpathlineto{\pgfqpoint{4.081593in}{1.622809in}}%
\pgfpathclose%
\pgfusepath{stroke,fill}%
\end{pgfscope}%
\begin{pgfscope}%
\pgfpathrectangle{\pgfqpoint{0.887500in}{0.275000in}}{\pgfqpoint{4.225000in}{4.225000in}}%
\pgfusepath{clip}%
\pgfsetbuttcap%
\pgfsetroundjoin%
\definecolor{currentfill}{rgb}{0.282884,0.135920,0.453427}%
\pgfsetfillcolor{currentfill}%
\pgfsetfillopacity{0.700000}%
\pgfsetlinewidth{0.501875pt}%
\definecolor{currentstroke}{rgb}{1.000000,1.000000,1.000000}%
\pgfsetstrokecolor{currentstroke}%
\pgfsetstrokeopacity{0.500000}%
\pgfsetdash{}{0pt}%
\pgfpathmoveto{\pgfqpoint{3.768482in}{1.390667in}}%
\pgfpathlineto{\pgfqpoint{3.779858in}{1.395966in}}%
\pgfpathlineto{\pgfqpoint{3.791256in}{1.402638in}}%
\pgfpathlineto{\pgfqpoint{3.802679in}{1.410816in}}%
\pgfpathlineto{\pgfqpoint{3.814127in}{1.420415in}}%
\pgfpathlineto{\pgfqpoint{3.825599in}{1.431334in}}%
\pgfpathlineto{\pgfqpoint{3.819184in}{1.448275in}}%
\pgfpathlineto{\pgfqpoint{3.812781in}{1.465809in}}%
\pgfpathlineto{\pgfqpoint{3.806391in}{1.483862in}}%
\pgfpathlineto{\pgfqpoint{3.800010in}{1.502359in}}%
\pgfpathlineto{\pgfqpoint{3.793637in}{1.521227in}}%
\pgfpathlineto{\pgfqpoint{3.782173in}{1.510521in}}%
\pgfpathlineto{\pgfqpoint{3.770731in}{1.501123in}}%
\pgfpathlineto{\pgfqpoint{3.759313in}{1.493164in}}%
\pgfpathlineto{\pgfqpoint{3.747920in}{1.486750in}}%
\pgfpathlineto{\pgfqpoint{3.736546in}{1.481707in}}%
\pgfpathlineto{\pgfqpoint{3.742922in}{1.462981in}}%
\pgfpathlineto{\pgfqpoint{3.749300in}{1.444380in}}%
\pgfpathlineto{\pgfqpoint{3.755685in}{1.426038in}}%
\pgfpathlineto{\pgfqpoint{3.762078in}{1.408088in}}%
\pgfpathclose%
\pgfusepath{stroke,fill}%
\end{pgfscope}%
\begin{pgfscope}%
\pgfpathrectangle{\pgfqpoint{0.887500in}{0.275000in}}{\pgfqpoint{4.225000in}{4.225000in}}%
\pgfusepath{clip}%
\pgfsetbuttcap%
\pgfsetroundjoin%
\definecolor{currentfill}{rgb}{0.183898,0.422383,0.556944}%
\pgfsetfillcolor{currentfill}%
\pgfsetfillopacity{0.700000}%
\pgfsetlinewidth{0.501875pt}%
\definecolor{currentstroke}{rgb}{1.000000,1.000000,1.000000}%
\pgfsetstrokecolor{currentstroke}%
\pgfsetstrokeopacity{0.500000}%
\pgfsetdash{}{0pt}%
\pgfpathmoveto{\pgfqpoint{2.313958in}{1.933369in}}%
\pgfpathlineto{\pgfqpoint{2.325672in}{1.937180in}}%
\pgfpathlineto{\pgfqpoint{2.337380in}{1.940991in}}%
\pgfpathlineto{\pgfqpoint{2.349083in}{1.944804in}}%
\pgfpathlineto{\pgfqpoint{2.360779in}{1.948621in}}%
\pgfpathlineto{\pgfqpoint{2.372470in}{1.952444in}}%
\pgfpathlineto{\pgfqpoint{2.366364in}{1.962302in}}%
\pgfpathlineto{\pgfqpoint{2.360261in}{1.972115in}}%
\pgfpathlineto{\pgfqpoint{2.354163in}{1.981884in}}%
\pgfpathlineto{\pgfqpoint{2.348070in}{1.991610in}}%
\pgfpathlineto{\pgfqpoint{2.341981in}{2.001294in}}%
\pgfpathlineto{\pgfqpoint{2.330300in}{1.997534in}}%
\pgfpathlineto{\pgfqpoint{2.318613in}{1.993780in}}%
\pgfpathlineto{\pgfqpoint{2.306920in}{1.990031in}}%
\pgfpathlineto{\pgfqpoint{2.295221in}{1.986284in}}%
\pgfpathlineto{\pgfqpoint{2.283517in}{1.982536in}}%
\pgfpathlineto{\pgfqpoint{2.289596in}{1.972788in}}%
\pgfpathlineto{\pgfqpoint{2.295680in}{1.962997in}}%
\pgfpathlineto{\pgfqpoint{2.301768in}{1.953164in}}%
\pgfpathlineto{\pgfqpoint{2.307861in}{1.943288in}}%
\pgfpathclose%
\pgfusepath{stroke,fill}%
\end{pgfscope}%
\begin{pgfscope}%
\pgfpathrectangle{\pgfqpoint{0.887500in}{0.275000in}}{\pgfqpoint{4.225000in}{4.225000in}}%
\pgfusepath{clip}%
\pgfsetbuttcap%
\pgfsetroundjoin%
\definecolor{currentfill}{rgb}{0.255645,0.260703,0.528312}%
\pgfsetfillcolor{currentfill}%
\pgfsetfillopacity{0.700000}%
\pgfsetlinewidth{0.501875pt}%
\definecolor{currentstroke}{rgb}{1.000000,1.000000,1.000000}%
\pgfsetstrokecolor{currentstroke}%
\pgfsetstrokeopacity{0.500000}%
\pgfsetdash{}{0pt}%
\pgfpathmoveto{\pgfqpoint{3.941193in}{1.582049in}}%
\pgfpathlineto{\pgfqpoint{3.952735in}{1.596149in}}%
\pgfpathlineto{\pgfqpoint{3.964254in}{1.609220in}}%
\pgfpathlineto{\pgfqpoint{3.975753in}{1.621435in}}%
\pgfpathlineto{\pgfqpoint{3.987234in}{1.632969in}}%
\pgfpathlineto{\pgfqpoint{3.998702in}{1.644000in}}%
\pgfpathlineto{\pgfqpoint{3.992283in}{1.661785in}}%
\pgfpathlineto{\pgfqpoint{3.985858in}{1.679263in}}%
\pgfpathlineto{\pgfqpoint{3.979429in}{1.696492in}}%
\pgfpathlineto{\pgfqpoint{3.972996in}{1.713531in}}%
\pgfpathlineto{\pgfqpoint{3.966563in}{1.730441in}}%
\pgfpathlineto{\pgfqpoint{3.955092in}{1.719098in}}%
\pgfpathlineto{\pgfqpoint{3.943607in}{1.707173in}}%
\pgfpathlineto{\pgfqpoint{3.932106in}{1.694592in}}%
\pgfpathlineto{\pgfqpoint{3.920589in}{1.681281in}}%
\pgfpathlineto{\pgfqpoint{3.909053in}{1.667166in}}%
\pgfpathlineto{\pgfqpoint{3.915481in}{1.650274in}}%
\pgfpathlineto{\pgfqpoint{3.921910in}{1.633380in}}%
\pgfpathlineto{\pgfqpoint{3.928339in}{1.616420in}}%
\pgfpathlineto{\pgfqpoint{3.934767in}{1.599331in}}%
\pgfpathclose%
\pgfusepath{stroke,fill}%
\end{pgfscope}%
\begin{pgfscope}%
\pgfpathrectangle{\pgfqpoint{0.887500in}{0.275000in}}{\pgfqpoint{4.225000in}{4.225000in}}%
\pgfusepath{clip}%
\pgfsetbuttcap%
\pgfsetroundjoin%
\definecolor{currentfill}{rgb}{0.270595,0.214069,0.507052}%
\pgfsetfillcolor{currentfill}%
\pgfsetfillopacity{0.700000}%
\pgfsetlinewidth{0.501875pt}%
\definecolor{currentstroke}{rgb}{1.000000,1.000000,1.000000}%
\pgfsetstrokecolor{currentstroke}%
\pgfsetstrokeopacity{0.500000}%
\pgfsetdash{}{0pt}%
\pgfpathmoveto{\pgfqpoint{4.119496in}{1.490838in}}%
\pgfpathlineto{\pgfqpoint{4.130941in}{1.501337in}}%
\pgfpathlineto{\pgfqpoint{4.142380in}{1.511722in}}%
\pgfpathlineto{\pgfqpoint{4.153816in}{1.522012in}}%
\pgfpathlineto{\pgfqpoint{4.165247in}{1.532225in}}%
\pgfpathlineto{\pgfqpoint{4.176675in}{1.542385in}}%
\pgfpathlineto{\pgfqpoint{4.170473in}{1.568281in}}%
\pgfpathlineto{\pgfqpoint{4.164229in}{1.592754in}}%
\pgfpathlineto{\pgfqpoint{4.157946in}{1.615896in}}%
\pgfpathlineto{\pgfqpoint{4.151628in}{1.637799in}}%
\pgfpathlineto{\pgfqpoint{4.145277in}{1.658557in}}%
\pgfpathlineto{\pgfqpoint{4.133822in}{1.647608in}}%
\pgfpathlineto{\pgfqpoint{4.122362in}{1.636496in}}%
\pgfpathlineto{\pgfqpoint{4.110897in}{1.625282in}}%
\pgfpathlineto{\pgfqpoint{4.099430in}{1.614023in}}%
\pgfpathlineto{\pgfqpoint{4.087962in}{1.602778in}}%
\pgfpathlineto{\pgfqpoint{4.094313in}{1.582037in}}%
\pgfpathlineto{\pgfqpoint{4.100645in}{1.560531in}}%
\pgfpathlineto{\pgfqpoint{4.106953in}{1.538201in}}%
\pgfpathlineto{\pgfqpoint{4.113238in}{1.514989in}}%
\pgfpathclose%
\pgfusepath{stroke,fill}%
\end{pgfscope}%
\begin{pgfscope}%
\pgfpathrectangle{\pgfqpoint{0.887500in}{0.275000in}}{\pgfqpoint{4.225000in}{4.225000in}}%
\pgfusepath{clip}%
\pgfsetbuttcap%
\pgfsetroundjoin%
\definecolor{currentfill}{rgb}{0.169646,0.456262,0.558030}%
\pgfsetfillcolor{currentfill}%
\pgfsetfillopacity{0.700000}%
\pgfsetlinewidth{0.501875pt}%
\definecolor{currentstroke}{rgb}{1.000000,1.000000,1.000000}%
\pgfsetstrokecolor{currentstroke}%
\pgfsetstrokeopacity{0.500000}%
\pgfsetdash{}{0pt}%
\pgfpathmoveto{\pgfqpoint{1.988118in}{2.003506in}}%
\pgfpathlineto{\pgfqpoint{1.999914in}{2.007278in}}%
\pgfpathlineto{\pgfqpoint{2.011705in}{2.011048in}}%
\pgfpathlineto{\pgfqpoint{2.023490in}{2.014818in}}%
\pgfpathlineto{\pgfqpoint{2.035268in}{2.018590in}}%
\pgfpathlineto{\pgfqpoint{2.047042in}{2.022362in}}%
\pgfpathlineto{\pgfqpoint{2.041042in}{2.031865in}}%
\pgfpathlineto{\pgfqpoint{2.035048in}{2.041334in}}%
\pgfpathlineto{\pgfqpoint{2.029058in}{2.050769in}}%
\pgfpathlineto{\pgfqpoint{2.023072in}{2.060171in}}%
\pgfpathlineto{\pgfqpoint{2.017091in}{2.069539in}}%
\pgfpathlineto{\pgfqpoint{2.005328in}{2.065827in}}%
\pgfpathlineto{\pgfqpoint{1.993560in}{2.062116in}}%
\pgfpathlineto{\pgfqpoint{1.981785in}{2.058405in}}%
\pgfpathlineto{\pgfqpoint{1.970005in}{2.054693in}}%
\pgfpathlineto{\pgfqpoint{1.958219in}{2.050979in}}%
\pgfpathlineto{\pgfqpoint{1.964190in}{2.041551in}}%
\pgfpathlineto{\pgfqpoint{1.970165in}{2.032090in}}%
\pgfpathlineto{\pgfqpoint{1.976145in}{2.022595in}}%
\pgfpathlineto{\pgfqpoint{1.982129in}{2.013068in}}%
\pgfpathclose%
\pgfusepath{stroke,fill}%
\end{pgfscope}%
\begin{pgfscope}%
\pgfpathrectangle{\pgfqpoint{0.887500in}{0.275000in}}{\pgfqpoint{4.225000in}{4.225000in}}%
\pgfusepath{clip}%
\pgfsetbuttcap%
\pgfsetroundjoin%
\definecolor{currentfill}{rgb}{0.274128,0.199721,0.498911}%
\pgfsetfillcolor{currentfill}%
\pgfsetfillopacity{0.700000}%
\pgfsetlinewidth{0.501875pt}%
\definecolor{currentstroke}{rgb}{1.000000,1.000000,1.000000}%
\pgfsetstrokecolor{currentstroke}%
\pgfsetstrokeopacity{0.500000}%
\pgfsetdash{}{0pt}%
\pgfpathmoveto{\pgfqpoint{3.590389in}{1.505172in}}%
\pgfpathlineto{\pgfqpoint{3.601842in}{1.512666in}}%
\pgfpathlineto{\pgfqpoint{3.613292in}{1.520159in}}%
\pgfpathlineto{\pgfqpoint{3.624736in}{1.527518in}}%
\pgfpathlineto{\pgfqpoint{3.636173in}{1.534606in}}%
\pgfpathlineto{\pgfqpoint{3.647599in}{1.541291in}}%
\pgfpathlineto{\pgfqpoint{3.641130in}{1.553447in}}%
\pgfpathlineto{\pgfqpoint{3.634644in}{1.564240in}}%
\pgfpathlineto{\pgfqpoint{3.628146in}{1.573942in}}%
\pgfpathlineto{\pgfqpoint{3.621641in}{1.582893in}}%
\pgfpathlineto{\pgfqpoint{3.615134in}{1.591435in}}%
\pgfpathlineto{\pgfqpoint{3.603693in}{1.582902in}}%
\pgfpathlineto{\pgfqpoint{3.592266in}{1.575570in}}%
\pgfpathlineto{\pgfqpoint{3.580848in}{1.569220in}}%
\pgfpathlineto{\pgfqpoint{3.569437in}{1.563633in}}%
\pgfpathlineto{\pgfqpoint{3.558027in}{1.558591in}}%
\pgfpathlineto{\pgfqpoint{3.564503in}{1.548713in}}%
\pgfpathlineto{\pgfqpoint{3.570979in}{1.538624in}}%
\pgfpathlineto{\pgfqpoint{3.577454in}{1.528130in}}%
\pgfpathlineto{\pgfqpoint{3.583925in}{1.517032in}}%
\pgfpathclose%
\pgfusepath{stroke,fill}%
\end{pgfscope}%
\begin{pgfscope}%
\pgfpathrectangle{\pgfqpoint{0.887500in}{0.275000in}}{\pgfqpoint{4.225000in}{4.225000in}}%
\pgfusepath{clip}%
\pgfsetbuttcap%
\pgfsetroundjoin%
\definecolor{currentfill}{rgb}{0.214298,0.355619,0.551184}%
\pgfsetfillcolor{currentfill}%
\pgfsetfillopacity{0.700000}%
\pgfsetlinewidth{0.501875pt}%
\definecolor{currentstroke}{rgb}{1.000000,1.000000,1.000000}%
\pgfsetstrokecolor{currentstroke}%
\pgfsetstrokeopacity{0.500000}%
\pgfsetdash{}{0pt}%
\pgfpathmoveto{\pgfqpoint{2.818187in}{1.789990in}}%
\pgfpathlineto{\pgfqpoint{2.829779in}{1.793903in}}%
\pgfpathlineto{\pgfqpoint{2.841366in}{1.797823in}}%
\pgfpathlineto{\pgfqpoint{2.852947in}{1.801745in}}%
\pgfpathlineto{\pgfqpoint{2.864522in}{1.805659in}}%
\pgfpathlineto{\pgfqpoint{2.876093in}{1.809553in}}%
\pgfpathlineto{\pgfqpoint{2.869826in}{1.820290in}}%
\pgfpathlineto{\pgfqpoint{2.863563in}{1.830973in}}%
\pgfpathlineto{\pgfqpoint{2.857304in}{1.841604in}}%
\pgfpathlineto{\pgfqpoint{2.851049in}{1.852183in}}%
\pgfpathlineto{\pgfqpoint{2.844797in}{1.862710in}}%
\pgfpathlineto{\pgfqpoint{2.833236in}{1.858869in}}%
\pgfpathlineto{\pgfqpoint{2.821669in}{1.855003in}}%
\pgfpathlineto{\pgfqpoint{2.810097in}{1.851127in}}%
\pgfpathlineto{\pgfqpoint{2.798520in}{1.847257in}}%
\pgfpathlineto{\pgfqpoint{2.786936in}{1.843397in}}%
\pgfpathlineto{\pgfqpoint{2.793178in}{1.832820in}}%
\pgfpathlineto{\pgfqpoint{2.799425in}{1.822191in}}%
\pgfpathlineto{\pgfqpoint{2.805675in}{1.811510in}}%
\pgfpathlineto{\pgfqpoint{2.811929in}{1.800776in}}%
\pgfpathclose%
\pgfusepath{stroke,fill}%
\end{pgfscope}%
\begin{pgfscope}%
\pgfpathrectangle{\pgfqpoint{0.887500in}{0.275000in}}{\pgfqpoint{4.225000in}{4.225000in}}%
\pgfusepath{clip}%
\pgfsetbuttcap%
\pgfsetroundjoin%
\definecolor{currentfill}{rgb}{0.262138,0.242286,0.520837}%
\pgfsetfillcolor{currentfill}%
\pgfsetfillopacity{0.700000}%
\pgfsetlinewidth{0.501875pt}%
\definecolor{currentstroke}{rgb}{1.000000,1.000000,1.000000}%
\pgfsetstrokecolor{currentstroke}%
\pgfsetstrokeopacity{0.500000}%
\pgfsetdash{}{0pt}%
\pgfpathmoveto{\pgfqpoint{4.030651in}{1.548383in}}%
\pgfpathlineto{\pgfqpoint{4.042110in}{1.559020in}}%
\pgfpathlineto{\pgfqpoint{4.053568in}{1.569699in}}%
\pgfpathlineto{\pgfqpoint{4.065030in}{1.580558in}}%
\pgfpathlineto{\pgfqpoint{4.076495in}{1.591603in}}%
\pgfpathlineto{\pgfqpoint{4.087962in}{1.602778in}}%
\pgfpathlineto{\pgfqpoint{4.081593in}{1.622809in}}%
\pgfpathlineto{\pgfqpoint{4.075207in}{1.642191in}}%
\pgfpathlineto{\pgfqpoint{4.068806in}{1.660981in}}%
\pgfpathlineto{\pgfqpoint{4.062393in}{1.679238in}}%
\pgfpathlineto{\pgfqpoint{4.055969in}{1.697020in}}%
\pgfpathlineto{\pgfqpoint{4.044519in}{1.686423in}}%
\pgfpathlineto{\pgfqpoint{4.033067in}{1.675821in}}%
\pgfpathlineto{\pgfqpoint{4.021615in}{1.665257in}}%
\pgfpathlineto{\pgfqpoint{4.010161in}{1.654703in}}%
\pgfpathlineto{\pgfqpoint{3.998702in}{1.644000in}}%
\pgfpathlineto{\pgfqpoint{4.005114in}{1.625847in}}%
\pgfpathlineto{\pgfqpoint{4.011516in}{1.607268in}}%
\pgfpathlineto{\pgfqpoint{4.017908in}{1.588203in}}%
\pgfpathlineto{\pgfqpoint{4.024287in}{1.568594in}}%
\pgfpathclose%
\pgfusepath{stroke,fill}%
\end{pgfscope}%
\begin{pgfscope}%
\pgfpathrectangle{\pgfqpoint{0.887500in}{0.275000in}}{\pgfqpoint{4.225000in}{4.225000in}}%
\pgfusepath{clip}%
\pgfsetbuttcap%
\pgfsetroundjoin%
\definecolor{currentfill}{rgb}{0.188923,0.410910,0.556326}%
\pgfsetfillcolor{currentfill}%
\pgfsetfillopacity{0.700000}%
\pgfsetlinewidth{0.501875pt}%
\definecolor{currentstroke}{rgb}{1.000000,1.000000,1.000000}%
\pgfsetstrokecolor{currentstroke}%
\pgfsetstrokeopacity{0.500000}%
\pgfsetdash{}{0pt}%
\pgfpathmoveto{\pgfqpoint{2.403071in}{1.902463in}}%
\pgfpathlineto{\pgfqpoint{2.414766in}{1.906347in}}%
\pgfpathlineto{\pgfqpoint{2.426455in}{1.910233in}}%
\pgfpathlineto{\pgfqpoint{2.438139in}{1.914116in}}%
\pgfpathlineto{\pgfqpoint{2.449817in}{1.917993in}}%
\pgfpathlineto{\pgfqpoint{2.461489in}{1.921860in}}%
\pgfpathlineto{\pgfqpoint{2.455351in}{1.931897in}}%
\pgfpathlineto{\pgfqpoint{2.449216in}{1.941885in}}%
\pgfpathlineto{\pgfqpoint{2.443086in}{1.951826in}}%
\pgfpathlineto{\pgfqpoint{2.436961in}{1.961719in}}%
\pgfpathlineto{\pgfqpoint{2.430839in}{1.971565in}}%
\pgfpathlineto{\pgfqpoint{2.419177in}{1.967752in}}%
\pgfpathlineto{\pgfqpoint{2.407509in}{1.963930in}}%
\pgfpathlineto{\pgfqpoint{2.395835in}{1.960102in}}%
\pgfpathlineto{\pgfqpoint{2.384156in}{1.956272in}}%
\pgfpathlineto{\pgfqpoint{2.372470in}{1.952444in}}%
\pgfpathlineto{\pgfqpoint{2.378582in}{1.942541in}}%
\pgfpathlineto{\pgfqpoint{2.384697in}{1.932591in}}%
\pgfpathlineto{\pgfqpoint{2.390818in}{1.922594in}}%
\pgfpathlineto{\pgfqpoint{2.396942in}{1.912552in}}%
\pgfpathclose%
\pgfusepath{stroke,fill}%
\end{pgfscope}%
\begin{pgfscope}%
\pgfpathrectangle{\pgfqpoint{0.887500in}{0.275000in}}{\pgfqpoint{4.225000in}{4.225000in}}%
\pgfusepath{clip}%
\pgfsetbuttcap%
\pgfsetroundjoin%
\definecolor{currentfill}{rgb}{0.283187,0.125848,0.444960}%
\pgfsetfillcolor{currentfill}%
\pgfsetfillopacity{0.700000}%
\pgfsetlinewidth{0.501875pt}%
\definecolor{currentstroke}{rgb}{1.000000,1.000000,1.000000}%
\pgfsetstrokecolor{currentstroke}%
\pgfsetstrokeopacity{0.500000}%
\pgfsetdash{}{0pt}%
\pgfpathmoveto{\pgfqpoint{3.857943in}{1.358123in}}%
\pgfpathlineto{\pgfqpoint{3.869425in}{1.369173in}}%
\pgfpathlineto{\pgfqpoint{3.880922in}{1.381004in}}%
\pgfpathlineto{\pgfqpoint{3.892435in}{1.393570in}}%
\pgfpathlineto{\pgfqpoint{3.903962in}{1.406819in}}%
\pgfpathlineto{\pgfqpoint{3.915501in}{1.420623in}}%
\pgfpathlineto{\pgfqpoint{3.909042in}{1.436370in}}%
\pgfpathlineto{\pgfqpoint{3.902591in}{1.452389in}}%
\pgfpathlineto{\pgfqpoint{3.896147in}{1.468676in}}%
\pgfpathlineto{\pgfqpoint{3.889712in}{1.485226in}}%
\pgfpathlineto{\pgfqpoint{3.883283in}{1.502034in}}%
\pgfpathlineto{\pgfqpoint{3.871709in}{1.486151in}}%
\pgfpathlineto{\pgfqpoint{3.860151in}{1.470981in}}%
\pgfpathlineto{\pgfqpoint{3.848612in}{1.456718in}}%
\pgfpathlineto{\pgfqpoint{3.837095in}{1.443469in}}%
\pgfpathlineto{\pgfqpoint{3.825599in}{1.431334in}}%
\pgfpathlineto{\pgfqpoint{3.832030in}{1.415060in}}%
\pgfpathlineto{\pgfqpoint{3.838478in}{1.399528in}}%
\pgfpathlineto{\pgfqpoint{3.844945in}{1.384812in}}%
\pgfpathlineto{\pgfqpoint{3.851432in}{1.370985in}}%
\pgfpathclose%
\pgfusepath{stroke,fill}%
\end{pgfscope}%
\begin{pgfscope}%
\pgfpathrectangle{\pgfqpoint{0.887500in}{0.275000in}}{\pgfqpoint{4.225000in}{4.225000in}}%
\pgfusepath{clip}%
\pgfsetbuttcap%
\pgfsetroundjoin%
\definecolor{currentfill}{rgb}{0.221989,0.339161,0.548752}%
\pgfsetfillcolor{currentfill}%
\pgfsetfillopacity{0.700000}%
\pgfsetlinewidth{0.501875pt}%
\definecolor{currentstroke}{rgb}{1.000000,1.000000,1.000000}%
\pgfsetstrokecolor{currentstroke}%
\pgfsetstrokeopacity{0.500000}%
\pgfsetdash{}{0pt}%
\pgfpathmoveto{\pgfqpoint{2.907483in}{1.755050in}}%
\pgfpathlineto{\pgfqpoint{2.919056in}{1.758966in}}%
\pgfpathlineto{\pgfqpoint{2.930624in}{1.762854in}}%
\pgfpathlineto{\pgfqpoint{2.942186in}{1.766707in}}%
\pgfpathlineto{\pgfqpoint{2.953743in}{1.770517in}}%
\pgfpathlineto{\pgfqpoint{2.965294in}{1.774285in}}%
\pgfpathlineto{\pgfqpoint{2.959000in}{1.785210in}}%
\pgfpathlineto{\pgfqpoint{2.952710in}{1.796074in}}%
\pgfpathlineto{\pgfqpoint{2.946423in}{1.806878in}}%
\pgfpathlineto{\pgfqpoint{2.940140in}{1.817623in}}%
\pgfpathlineto{\pgfqpoint{2.933861in}{1.828313in}}%
\pgfpathlineto{\pgfqpoint{2.922318in}{1.824684in}}%
\pgfpathlineto{\pgfqpoint{2.910770in}{1.820992in}}%
\pgfpathlineto{\pgfqpoint{2.899216in}{1.817232in}}%
\pgfpathlineto{\pgfqpoint{2.887657in}{1.813415in}}%
\pgfpathlineto{\pgfqpoint{2.876093in}{1.809553in}}%
\pgfpathlineto{\pgfqpoint{2.882363in}{1.798763in}}%
\pgfpathlineto{\pgfqpoint{2.888637in}{1.787918in}}%
\pgfpathlineto{\pgfqpoint{2.894916in}{1.777018in}}%
\pgfpathlineto{\pgfqpoint{2.901197in}{1.766062in}}%
\pgfpathclose%
\pgfusepath{stroke,fill}%
\end{pgfscope}%
\begin{pgfscope}%
\pgfpathrectangle{\pgfqpoint{0.887500in}{0.275000in}}{\pgfqpoint{4.225000in}{4.225000in}}%
\pgfusepath{clip}%
\pgfsetbuttcap%
\pgfsetroundjoin%
\definecolor{currentfill}{rgb}{0.278012,0.180367,0.486697}%
\pgfsetfillcolor{currentfill}%
\pgfsetfillopacity{0.700000}%
\pgfsetlinewidth{0.501875pt}%
\definecolor{currentstroke}{rgb}{1.000000,1.000000,1.000000}%
\pgfsetstrokecolor{currentstroke}%
\pgfsetstrokeopacity{0.500000}%
\pgfsetdash{}{0pt}%
\pgfpathmoveto{\pgfqpoint{3.679733in}{1.464266in}}%
\pgfpathlineto{\pgfqpoint{3.691110in}{1.467940in}}%
\pgfpathlineto{\pgfqpoint{3.702474in}{1.471154in}}%
\pgfpathlineto{\pgfqpoint{3.713830in}{1.474275in}}%
\pgfpathlineto{\pgfqpoint{3.725186in}{1.477671in}}%
\pgfpathlineto{\pgfqpoint{3.736546in}{1.481707in}}%
\pgfpathlineto{\pgfqpoint{3.730172in}{1.500421in}}%
\pgfpathlineto{\pgfqpoint{3.723796in}{1.518988in}}%
\pgfpathlineto{\pgfqpoint{3.717416in}{1.537274in}}%
\pgfpathlineto{\pgfqpoint{3.711031in}{1.555141in}}%
\pgfpathlineto{\pgfqpoint{3.704638in}{1.572456in}}%
\pgfpathlineto{\pgfqpoint{3.693231in}{1.565785in}}%
\pgfpathlineto{\pgfqpoint{3.681828in}{1.559561in}}%
\pgfpathlineto{\pgfqpoint{3.670424in}{1.553556in}}%
\pgfpathlineto{\pgfqpoint{3.659016in}{1.547542in}}%
\pgfpathlineto{\pgfqpoint{3.647599in}{1.541291in}}%
\pgfpathlineto{\pgfqpoint{3.654052in}{1.527837in}}%
\pgfpathlineto{\pgfqpoint{3.660489in}{1.513246in}}%
\pgfpathlineto{\pgfqpoint{3.666914in}{1.497680in}}%
\pgfpathlineto{\pgfqpoint{3.673328in}{1.481299in}}%
\pgfpathclose%
\pgfusepath{stroke,fill}%
\end{pgfscope}%
\begin{pgfscope}%
\pgfpathrectangle{\pgfqpoint{0.887500in}{0.275000in}}{\pgfqpoint{4.225000in}{4.225000in}}%
\pgfusepath{clip}%
\pgfsetbuttcap%
\pgfsetroundjoin%
\definecolor{currentfill}{rgb}{0.277134,0.185228,0.489898}%
\pgfsetfillcolor{currentfill}%
\pgfsetfillopacity{0.700000}%
\pgfsetlinewidth{0.501875pt}%
\definecolor{currentstroke}{rgb}{1.000000,1.000000,1.000000}%
\pgfsetstrokecolor{currentstroke}%
\pgfsetstrokeopacity{0.500000}%
\pgfsetdash{}{0pt}%
\pgfpathmoveto{\pgfqpoint{4.062194in}{1.436238in}}%
\pgfpathlineto{\pgfqpoint{4.073663in}{1.447362in}}%
\pgfpathlineto{\pgfqpoint{4.085129in}{1.458463in}}%
\pgfpathlineto{\pgfqpoint{4.096591in}{1.469419in}}%
\pgfpathlineto{\pgfqpoint{4.108046in}{1.480205in}}%
\pgfpathlineto{\pgfqpoint{4.119496in}{1.490838in}}%
\pgfpathlineto{\pgfqpoint{4.113238in}{1.514989in}}%
\pgfpathlineto{\pgfqpoint{4.106953in}{1.538201in}}%
\pgfpathlineto{\pgfqpoint{4.100645in}{1.560531in}}%
\pgfpathlineto{\pgfqpoint{4.094313in}{1.582037in}}%
\pgfpathlineto{\pgfqpoint{4.087962in}{1.602778in}}%
\pgfpathlineto{\pgfqpoint{4.076495in}{1.591603in}}%
\pgfpathlineto{\pgfqpoint{4.065030in}{1.580558in}}%
\pgfpathlineto{\pgfqpoint{4.053568in}{1.569699in}}%
\pgfpathlineto{\pgfqpoint{4.042110in}{1.559020in}}%
\pgfpathlineto{\pgfqpoint{4.030651in}{1.548383in}}%
\pgfpathlineto{\pgfqpoint{4.037000in}{1.527510in}}%
\pgfpathlineto{\pgfqpoint{4.043330in}{1.505917in}}%
\pgfpathlineto{\pgfqpoint{4.049640in}{1.483546in}}%
\pgfpathlineto{\pgfqpoint{4.055929in}{1.460339in}}%
\pgfpathclose%
\pgfusepath{stroke,fill}%
\end{pgfscope}%
\begin{pgfscope}%
\pgfpathrectangle{\pgfqpoint{0.887500in}{0.275000in}}{\pgfqpoint{4.225000in}{4.225000in}}%
\pgfusepath{clip}%
\pgfsetbuttcap%
\pgfsetroundjoin%
\definecolor{currentfill}{rgb}{0.279574,0.170599,0.479997}%
\pgfsetfillcolor{currentfill}%
\pgfsetfillopacity{0.700000}%
\pgfsetlinewidth{0.501875pt}%
\definecolor{currentstroke}{rgb}{1.000000,1.000000,1.000000}%
\pgfsetstrokecolor{currentstroke}%
\pgfsetstrokeopacity{0.500000}%
\pgfsetdash{}{0pt}%
\pgfpathmoveto{\pgfqpoint{3.915501in}{1.420623in}}%
\pgfpathlineto{\pgfqpoint{3.927048in}{1.434772in}}%
\pgfpathlineto{\pgfqpoint{3.938599in}{1.449054in}}%
\pgfpathlineto{\pgfqpoint{3.950149in}{1.463254in}}%
\pgfpathlineto{\pgfqpoint{3.961692in}{1.477158in}}%
\pgfpathlineto{\pgfqpoint{3.973224in}{1.490554in}}%
\pgfpathlineto{\pgfqpoint{3.966836in}{1.509740in}}%
\pgfpathlineto{\pgfqpoint{3.960437in}{1.528421in}}%
\pgfpathlineto{\pgfqpoint{3.954029in}{1.546657in}}%
\pgfpathlineto{\pgfqpoint{3.947614in}{1.564512in}}%
\pgfpathlineto{\pgfqpoint{3.941193in}{1.582049in}}%
\pgfpathlineto{\pgfqpoint{3.929628in}{1.566938in}}%
\pgfpathlineto{\pgfqpoint{3.918048in}{1.551089in}}%
\pgfpathlineto{\pgfqpoint{3.906459in}{1.534794in}}%
\pgfpathlineto{\pgfqpoint{3.894869in}{1.518345in}}%
\pgfpathlineto{\pgfqpoint{3.883283in}{1.502034in}}%
\pgfpathlineto{\pgfqpoint{3.889712in}{1.485226in}}%
\pgfpathlineto{\pgfqpoint{3.896147in}{1.468676in}}%
\pgfpathlineto{\pgfqpoint{3.902591in}{1.452389in}}%
\pgfpathlineto{\pgfqpoint{3.909042in}{1.436370in}}%
\pgfpathclose%
\pgfusepath{stroke,fill}%
\end{pgfscope}%
\begin{pgfscope}%
\pgfpathrectangle{\pgfqpoint{0.887500in}{0.275000in}}{\pgfqpoint{4.225000in}{4.225000in}}%
\pgfusepath{clip}%
\pgfsetbuttcap%
\pgfsetroundjoin%
\definecolor{currentfill}{rgb}{0.270595,0.214069,0.507052}%
\pgfsetfillcolor{currentfill}%
\pgfsetfillopacity{0.700000}%
\pgfsetlinewidth{0.501875pt}%
\definecolor{currentstroke}{rgb}{1.000000,1.000000,1.000000}%
\pgfsetstrokecolor{currentstroke}%
\pgfsetstrokeopacity{0.500000}%
\pgfsetdash{}{0pt}%
\pgfpathmoveto{\pgfqpoint{3.973224in}{1.490554in}}%
\pgfpathlineto{\pgfqpoint{3.984738in}{1.503239in}}%
\pgfpathlineto{\pgfqpoint{3.996235in}{1.515217in}}%
\pgfpathlineto{\pgfqpoint{4.007717in}{1.526632in}}%
\pgfpathlineto{\pgfqpoint{4.019188in}{1.537637in}}%
\pgfpathlineto{\pgfqpoint{4.030651in}{1.548383in}}%
\pgfpathlineto{\pgfqpoint{4.024287in}{1.568594in}}%
\pgfpathlineto{\pgfqpoint{4.017908in}{1.588203in}}%
\pgfpathlineto{\pgfqpoint{4.011516in}{1.607268in}}%
\pgfpathlineto{\pgfqpoint{4.005114in}{1.625847in}}%
\pgfpathlineto{\pgfqpoint{3.998702in}{1.644000in}}%
\pgfpathlineto{\pgfqpoint{3.987234in}{1.632969in}}%
\pgfpathlineto{\pgfqpoint{3.975753in}{1.621435in}}%
\pgfpathlineto{\pgfqpoint{3.964254in}{1.609220in}}%
\pgfpathlineto{\pgfqpoint{3.952735in}{1.596149in}}%
\pgfpathlineto{\pgfqpoint{3.941193in}{1.582049in}}%
\pgfpathlineto{\pgfqpoint{3.947614in}{1.564512in}}%
\pgfpathlineto{\pgfqpoint{3.954029in}{1.546657in}}%
\pgfpathlineto{\pgfqpoint{3.960437in}{1.528421in}}%
\pgfpathlineto{\pgfqpoint{3.966836in}{1.509740in}}%
\pgfpathclose%
\pgfusepath{stroke,fill}%
\end{pgfscope}%
\begin{pgfscope}%
\pgfpathrectangle{\pgfqpoint{0.887500in}{0.275000in}}{\pgfqpoint{4.225000in}{4.225000in}}%
\pgfusepath{clip}%
\pgfsetbuttcap%
\pgfsetroundjoin%
\definecolor{currentfill}{rgb}{0.174274,0.445044,0.557792}%
\pgfsetfillcolor{currentfill}%
\pgfsetfillopacity{0.700000}%
\pgfsetlinewidth{0.501875pt}%
\definecolor{currentstroke}{rgb}{1.000000,1.000000,1.000000}%
\pgfsetstrokecolor{currentstroke}%
\pgfsetstrokeopacity{0.500000}%
\pgfsetdash{}{0pt}%
\pgfpathmoveto{\pgfqpoint{2.077106in}{1.974303in}}%
\pgfpathlineto{\pgfqpoint{2.088883in}{1.978130in}}%
\pgfpathlineto{\pgfqpoint{2.100655in}{1.981953in}}%
\pgfpathlineto{\pgfqpoint{2.112422in}{1.985770in}}%
\pgfpathlineto{\pgfqpoint{2.124182in}{1.989578in}}%
\pgfpathlineto{\pgfqpoint{2.135938in}{1.993375in}}%
\pgfpathlineto{\pgfqpoint{2.129906in}{2.003005in}}%
\pgfpathlineto{\pgfqpoint{2.123878in}{2.012597in}}%
\pgfpathlineto{\pgfqpoint{2.117855in}{2.022150in}}%
\pgfpathlineto{\pgfqpoint{2.111837in}{2.031666in}}%
\pgfpathlineto{\pgfqpoint{2.105823in}{2.041146in}}%
\pgfpathlineto{\pgfqpoint{2.094077in}{2.037408in}}%
\pgfpathlineto{\pgfqpoint{2.082327in}{2.033658in}}%
\pgfpathlineto{\pgfqpoint{2.070571in}{2.029898in}}%
\pgfpathlineto{\pgfqpoint{2.058809in}{2.026132in}}%
\pgfpathlineto{\pgfqpoint{2.047042in}{2.022362in}}%
\pgfpathlineto{\pgfqpoint{2.053045in}{2.012823in}}%
\pgfpathlineto{\pgfqpoint{2.059054in}{2.003248in}}%
\pgfpathlineto{\pgfqpoint{2.065066in}{1.993637in}}%
\pgfpathlineto{\pgfqpoint{2.071084in}{1.983989in}}%
\pgfpathclose%
\pgfusepath{stroke,fill}%
\end{pgfscope}%
\begin{pgfscope}%
\pgfpathrectangle{\pgfqpoint{0.887500in}{0.275000in}}{\pgfqpoint{4.225000in}{4.225000in}}%
\pgfusepath{clip}%
\pgfsetbuttcap%
\pgfsetroundjoin%
\definecolor{currentfill}{rgb}{0.231674,0.318106,0.544834}%
\pgfsetfillcolor{currentfill}%
\pgfsetfillopacity{0.700000}%
\pgfsetlinewidth{0.501875pt}%
\definecolor{currentstroke}{rgb}{1.000000,1.000000,1.000000}%
\pgfsetstrokecolor{currentstroke}%
\pgfsetstrokeopacity{0.500000}%
\pgfsetdash{}{0pt}%
\pgfpathmoveto{\pgfqpoint{2.996819in}{1.718681in}}%
\pgfpathlineto{\pgfqpoint{3.008372in}{1.722541in}}%
\pgfpathlineto{\pgfqpoint{3.019920in}{1.726378in}}%
\pgfpathlineto{\pgfqpoint{3.031463in}{1.730192in}}%
\pgfpathlineto{\pgfqpoint{3.042999in}{1.733979in}}%
\pgfpathlineto{\pgfqpoint{3.054531in}{1.737738in}}%
\pgfpathlineto{\pgfqpoint{3.048210in}{1.748880in}}%
\pgfpathlineto{\pgfqpoint{3.041894in}{1.759961in}}%
\pgfpathlineto{\pgfqpoint{3.035580in}{1.770980in}}%
\pgfpathlineto{\pgfqpoint{3.029271in}{1.781939in}}%
\pgfpathlineto{\pgfqpoint{3.022965in}{1.792840in}}%
\pgfpathlineto{\pgfqpoint{3.011442in}{1.789132in}}%
\pgfpathlineto{\pgfqpoint{2.999913in}{1.785434in}}%
\pgfpathlineto{\pgfqpoint{2.988379in}{1.781733in}}%
\pgfpathlineto{\pgfqpoint{2.976839in}{1.778020in}}%
\pgfpathlineto{\pgfqpoint{2.965294in}{1.774285in}}%
\pgfpathlineto{\pgfqpoint{2.971592in}{1.763296in}}%
\pgfpathlineto{\pgfqpoint{2.977893in}{1.752242in}}%
\pgfpathlineto{\pgfqpoint{2.984198in}{1.741121in}}%
\pgfpathlineto{\pgfqpoint{2.990506in}{1.729933in}}%
\pgfpathclose%
\pgfusepath{stroke,fill}%
\end{pgfscope}%
\begin{pgfscope}%
\pgfpathrectangle{\pgfqpoint{0.887500in}{0.275000in}}{\pgfqpoint{4.225000in}{4.225000in}}%
\pgfusepath{clip}%
\pgfsetbuttcap%
\pgfsetroundjoin%
\definecolor{currentfill}{rgb}{0.195860,0.395433,0.555276}%
\pgfsetfillcolor{currentfill}%
\pgfsetfillopacity{0.700000}%
\pgfsetlinewidth{0.501875pt}%
\definecolor{currentstroke}{rgb}{1.000000,1.000000,1.000000}%
\pgfsetstrokecolor{currentstroke}%
\pgfsetstrokeopacity{0.500000}%
\pgfsetdash{}{0pt}%
\pgfpathmoveto{\pgfqpoint{2.492248in}{1.870972in}}%
\pgfpathlineto{\pgfqpoint{2.503925in}{1.874878in}}%
\pgfpathlineto{\pgfqpoint{2.515596in}{1.878769in}}%
\pgfpathlineto{\pgfqpoint{2.527261in}{1.882644in}}%
\pgfpathlineto{\pgfqpoint{2.538922in}{1.886501in}}%
\pgfpathlineto{\pgfqpoint{2.550577in}{1.890341in}}%
\pgfpathlineto{\pgfqpoint{2.544407in}{1.900559in}}%
\pgfpathlineto{\pgfqpoint{2.538241in}{1.910729in}}%
\pgfpathlineto{\pgfqpoint{2.532080in}{1.920853in}}%
\pgfpathlineto{\pgfqpoint{2.525923in}{1.930929in}}%
\pgfpathlineto{\pgfqpoint{2.519770in}{1.940959in}}%
\pgfpathlineto{\pgfqpoint{2.508125in}{1.937175in}}%
\pgfpathlineto{\pgfqpoint{2.496474in}{1.933373in}}%
\pgfpathlineto{\pgfqpoint{2.484818in}{1.929553in}}%
\pgfpathlineto{\pgfqpoint{2.473156in}{1.925715in}}%
\pgfpathlineto{\pgfqpoint{2.461489in}{1.921860in}}%
\pgfpathlineto{\pgfqpoint{2.467632in}{1.911777in}}%
\pgfpathlineto{\pgfqpoint{2.473780in}{1.901646in}}%
\pgfpathlineto{\pgfqpoint{2.479932in}{1.891468in}}%
\pgfpathlineto{\pgfqpoint{2.486088in}{1.881243in}}%
\pgfpathclose%
\pgfusepath{stroke,fill}%
\end{pgfscope}%
\begin{pgfscope}%
\pgfpathrectangle{\pgfqpoint{0.887500in}{0.275000in}}{\pgfqpoint{4.225000in}{4.225000in}}%
\pgfusepath{clip}%
\pgfsetbuttcap%
\pgfsetroundjoin%
\definecolor{currentfill}{rgb}{0.281924,0.089666,0.412415}%
\pgfsetfillcolor{currentfill}%
\pgfsetfillopacity{0.700000}%
\pgfsetlinewidth{0.501875pt}%
\definecolor{currentstroke}{rgb}{1.000000,1.000000,1.000000}%
\pgfsetstrokecolor{currentstroke}%
\pgfsetstrokeopacity{0.500000}%
\pgfsetdash{}{0pt}%
\pgfpathmoveto{\pgfqpoint{3.800770in}{1.316172in}}%
\pgfpathlineto{\pgfqpoint{3.812174in}{1.322702in}}%
\pgfpathlineto{\pgfqpoint{3.823592in}{1.330148in}}%
\pgfpathlineto{\pgfqpoint{3.835026in}{1.338562in}}%
\pgfpathlineto{\pgfqpoint{3.846477in}{1.347903in}}%
\pgfpathlineto{\pgfqpoint{3.857943in}{1.358123in}}%
\pgfpathlineto{\pgfqpoint{3.851432in}{1.370985in}}%
\pgfpathlineto{\pgfqpoint{3.844945in}{1.384812in}}%
\pgfpathlineto{\pgfqpoint{3.838478in}{1.399528in}}%
\pgfpathlineto{\pgfqpoint{3.832030in}{1.415060in}}%
\pgfpathlineto{\pgfqpoint{3.825599in}{1.431334in}}%
\pgfpathlineto{\pgfqpoint{3.814127in}{1.420415in}}%
\pgfpathlineto{\pgfqpoint{3.802679in}{1.410816in}}%
\pgfpathlineto{\pgfqpoint{3.791256in}{1.402638in}}%
\pgfpathlineto{\pgfqpoint{3.779858in}{1.395966in}}%
\pgfpathlineto{\pgfqpoint{3.768482in}{1.390667in}}%
\pgfpathlineto{\pgfqpoint{3.774900in}{1.373907in}}%
\pgfpathlineto{\pgfqpoint{3.781335in}{1.357944in}}%
\pgfpathlineto{\pgfqpoint{3.787790in}{1.342911in}}%
\pgfpathlineto{\pgfqpoint{3.794267in}{1.328942in}}%
\pgfpathclose%
\pgfusepath{stroke,fill}%
\end{pgfscope}%
\begin{pgfscope}%
\pgfpathrectangle{\pgfqpoint{0.887500in}{0.275000in}}{\pgfqpoint{4.225000in}{4.225000in}}%
\pgfusepath{clip}%
\pgfsetbuttcap%
\pgfsetroundjoin%
\definecolor{currentfill}{rgb}{0.239346,0.300855,0.540844}%
\pgfsetfillcolor{currentfill}%
\pgfsetfillopacity{0.700000}%
\pgfsetlinewidth{0.501875pt}%
\definecolor{currentstroke}{rgb}{1.000000,1.000000,1.000000}%
\pgfsetstrokecolor{currentstroke}%
\pgfsetstrokeopacity{0.500000}%
\pgfsetdash{}{0pt}%
\pgfpathmoveto{\pgfqpoint{3.086184in}{1.681061in}}%
\pgfpathlineto{\pgfqpoint{3.097717in}{1.684883in}}%
\pgfpathlineto{\pgfqpoint{3.109245in}{1.688685in}}%
\pgfpathlineto{\pgfqpoint{3.120767in}{1.692489in}}%
\pgfpathlineto{\pgfqpoint{3.132284in}{1.696316in}}%
\pgfpathlineto{\pgfqpoint{3.143794in}{1.700186in}}%
\pgfpathlineto{\pgfqpoint{3.137449in}{1.711573in}}%
\pgfpathlineto{\pgfqpoint{3.131107in}{1.722886in}}%
\pgfpathlineto{\pgfqpoint{3.124768in}{1.734132in}}%
\pgfpathlineto{\pgfqpoint{3.118433in}{1.745319in}}%
\pgfpathlineto{\pgfqpoint{3.112101in}{1.756454in}}%
\pgfpathlineto{\pgfqpoint{3.100598in}{1.752647in}}%
\pgfpathlineto{\pgfqpoint{3.089090in}{1.748897in}}%
\pgfpathlineto{\pgfqpoint{3.077576in}{1.745179in}}%
\pgfpathlineto{\pgfqpoint{3.066056in}{1.741467in}}%
\pgfpathlineto{\pgfqpoint{3.054531in}{1.737738in}}%
\pgfpathlineto{\pgfqpoint{3.060854in}{1.726532in}}%
\pgfpathlineto{\pgfqpoint{3.067182in}{1.715263in}}%
\pgfpathlineto{\pgfqpoint{3.073512in}{1.703928in}}%
\pgfpathlineto{\pgfqpoint{3.079846in}{1.692528in}}%
\pgfpathclose%
\pgfusepath{stroke,fill}%
\end{pgfscope}%
\begin{pgfscope}%
\pgfpathrectangle{\pgfqpoint{0.887500in}{0.275000in}}{\pgfqpoint{4.225000in}{4.225000in}}%
\pgfusepath{clip}%
\pgfsetbuttcap%
\pgfsetroundjoin%
\definecolor{currentfill}{rgb}{0.281887,0.150881,0.465405}%
\pgfsetfillcolor{currentfill}%
\pgfsetfillopacity{0.700000}%
\pgfsetlinewidth{0.501875pt}%
\definecolor{currentstroke}{rgb}{1.000000,1.000000,1.000000}%
\pgfsetstrokecolor{currentstroke}%
\pgfsetstrokeopacity{0.500000}%
\pgfsetdash{}{0pt}%
\pgfpathmoveto{\pgfqpoint{4.004936in}{1.384843in}}%
\pgfpathlineto{\pgfqpoint{4.016367in}{1.394125in}}%
\pgfpathlineto{\pgfqpoint{4.027810in}{1.404035in}}%
\pgfpathlineto{\pgfqpoint{4.039265in}{1.414444in}}%
\pgfpathlineto{\pgfqpoint{4.050727in}{1.425222in}}%
\pgfpathlineto{\pgfqpoint{4.062194in}{1.436238in}}%
\pgfpathlineto{\pgfqpoint{4.055929in}{1.460339in}}%
\pgfpathlineto{\pgfqpoint{4.049640in}{1.483546in}}%
\pgfpathlineto{\pgfqpoint{4.043330in}{1.505917in}}%
\pgfpathlineto{\pgfqpoint{4.037000in}{1.527510in}}%
\pgfpathlineto{\pgfqpoint{4.030651in}{1.548383in}}%
\pgfpathlineto{\pgfqpoint{4.019188in}{1.537637in}}%
\pgfpathlineto{\pgfqpoint{4.007717in}{1.526632in}}%
\pgfpathlineto{\pgfqpoint{3.996235in}{1.515217in}}%
\pgfpathlineto{\pgfqpoint{3.984738in}{1.503239in}}%
\pgfpathlineto{\pgfqpoint{3.973224in}{1.490554in}}%
\pgfpathlineto{\pgfqpoint{3.979599in}{1.470798in}}%
\pgfpathlineto{\pgfqpoint{3.985959in}{1.450411in}}%
\pgfpathlineto{\pgfqpoint{3.992304in}{1.429331in}}%
\pgfpathlineto{\pgfqpoint{3.998630in}{1.407496in}}%
\pgfpathclose%
\pgfusepath{stroke,fill}%
\end{pgfscope}%
\begin{pgfscope}%
\pgfpathrectangle{\pgfqpoint{0.887500in}{0.275000in}}{\pgfqpoint{4.225000in}{4.225000in}}%
\pgfusepath{clip}%
\pgfsetbuttcap%
\pgfsetroundjoin%
\definecolor{currentfill}{rgb}{0.246811,0.283237,0.535941}%
\pgfsetfillcolor{currentfill}%
\pgfsetfillopacity{0.700000}%
\pgfsetlinewidth{0.501875pt}%
\definecolor{currentstroke}{rgb}{1.000000,1.000000,1.000000}%
\pgfsetstrokecolor{currentstroke}%
\pgfsetstrokeopacity{0.500000}%
\pgfsetdash{}{0pt}%
\pgfpathmoveto{\pgfqpoint{3.175569in}{1.641902in}}%
\pgfpathlineto{\pgfqpoint{3.187082in}{1.645834in}}%
\pgfpathlineto{\pgfqpoint{3.198590in}{1.649818in}}%
\pgfpathlineto{\pgfqpoint{3.210092in}{1.653860in}}%
\pgfpathlineto{\pgfqpoint{3.221589in}{1.657936in}}%
\pgfpathlineto{\pgfqpoint{3.233080in}{1.662005in}}%
\pgfpathlineto{\pgfqpoint{3.226712in}{1.673879in}}%
\pgfpathlineto{\pgfqpoint{3.220347in}{1.685679in}}%
\pgfpathlineto{\pgfqpoint{3.213985in}{1.697400in}}%
\pgfpathlineto{\pgfqpoint{3.207626in}{1.709037in}}%
\pgfpathlineto{\pgfqpoint{3.201269in}{1.720585in}}%
\pgfpathlineto{\pgfqpoint{3.189785in}{1.716439in}}%
\pgfpathlineto{\pgfqpoint{3.178295in}{1.712262in}}%
\pgfpathlineto{\pgfqpoint{3.166800in}{1.708141in}}%
\pgfpathlineto{\pgfqpoint{3.155300in}{1.704121in}}%
\pgfpathlineto{\pgfqpoint{3.143794in}{1.700186in}}%
\pgfpathlineto{\pgfqpoint{3.150143in}{1.688718in}}%
\pgfpathlineto{\pgfqpoint{3.156495in}{1.677161in}}%
\pgfpathlineto{\pgfqpoint{3.162850in}{1.665507in}}%
\pgfpathlineto{\pgfqpoint{3.169208in}{1.653752in}}%
\pgfpathclose%
\pgfusepath{stroke,fill}%
\end{pgfscope}%
\begin{pgfscope}%
\pgfpathrectangle{\pgfqpoint{0.887500in}{0.275000in}}{\pgfqpoint{4.225000in}{4.225000in}}%
\pgfusepath{clip}%
\pgfsetbuttcap%
\pgfsetroundjoin%
\definecolor{currentfill}{rgb}{0.179019,0.433756,0.557430}%
\pgfsetfillcolor{currentfill}%
\pgfsetfillopacity{0.700000}%
\pgfsetlinewidth{0.501875pt}%
\definecolor{currentstroke}{rgb}{1.000000,1.000000,1.000000}%
\pgfsetstrokecolor{currentstroke}%
\pgfsetstrokeopacity{0.500000}%
\pgfsetdash{}{0pt}%
\pgfpathmoveto{\pgfqpoint{2.166167in}{1.944626in}}%
\pgfpathlineto{\pgfqpoint{2.177927in}{1.948467in}}%
\pgfpathlineto{\pgfqpoint{2.189681in}{1.952294in}}%
\pgfpathlineto{\pgfqpoint{2.201430in}{1.956110in}}%
\pgfpathlineto{\pgfqpoint{2.213173in}{1.959915in}}%
\pgfpathlineto{\pgfqpoint{2.224911in}{1.963708in}}%
\pgfpathlineto{\pgfqpoint{2.218846in}{1.973479in}}%
\pgfpathlineto{\pgfqpoint{2.212786in}{1.983209in}}%
\pgfpathlineto{\pgfqpoint{2.206730in}{1.992898in}}%
\pgfpathlineto{\pgfqpoint{2.200679in}{2.002547in}}%
\pgfpathlineto{\pgfqpoint{2.194632in}{2.012155in}}%
\pgfpathlineto{\pgfqpoint{2.182904in}{2.008425in}}%
\pgfpathlineto{\pgfqpoint{2.171171in}{2.004683in}}%
\pgfpathlineto{\pgfqpoint{2.159432in}{2.000928in}}%
\pgfpathlineto{\pgfqpoint{2.147687in}{1.997159in}}%
\pgfpathlineto{\pgfqpoint{2.135938in}{1.993375in}}%
\pgfpathlineto{\pgfqpoint{2.141974in}{1.983706in}}%
\pgfpathlineto{\pgfqpoint{2.148016in}{1.973996in}}%
\pgfpathlineto{\pgfqpoint{2.154061in}{1.964247in}}%
\pgfpathlineto{\pgfqpoint{2.160112in}{1.954457in}}%
\pgfpathclose%
\pgfusepath{stroke,fill}%
\end{pgfscope}%
\begin{pgfscope}%
\pgfpathrectangle{\pgfqpoint{0.887500in}{0.275000in}}{\pgfqpoint{4.225000in}{4.225000in}}%
\pgfusepath{clip}%
\pgfsetbuttcap%
\pgfsetroundjoin%
\definecolor{currentfill}{rgb}{0.201239,0.383670,0.554294}%
\pgfsetfillcolor{currentfill}%
\pgfsetfillopacity{0.700000}%
\pgfsetlinewidth{0.501875pt}%
\definecolor{currentstroke}{rgb}{1.000000,1.000000,1.000000}%
\pgfsetstrokecolor{currentstroke}%
\pgfsetstrokeopacity{0.500000}%
\pgfsetdash{}{0pt}%
\pgfpathmoveto{\pgfqpoint{2.581489in}{1.838554in}}%
\pgfpathlineto{\pgfqpoint{2.593147in}{1.842433in}}%
\pgfpathlineto{\pgfqpoint{2.604800in}{1.846294in}}%
\pgfpathlineto{\pgfqpoint{2.616448in}{1.850134in}}%
\pgfpathlineto{\pgfqpoint{2.628090in}{1.853955in}}%
\pgfpathlineto{\pgfqpoint{2.639727in}{1.857758in}}%
\pgfpathlineto{\pgfqpoint{2.633527in}{1.868151in}}%
\pgfpathlineto{\pgfqpoint{2.627331in}{1.878497in}}%
\pgfpathlineto{\pgfqpoint{2.621140in}{1.888795in}}%
\pgfpathlineto{\pgfqpoint{2.614952in}{1.899047in}}%
\pgfpathlineto{\pgfqpoint{2.608769in}{1.909252in}}%
\pgfpathlineto{\pgfqpoint{2.597141in}{1.905508in}}%
\pgfpathlineto{\pgfqpoint{2.585508in}{1.901746in}}%
\pgfpathlineto{\pgfqpoint{2.573870in}{1.897964in}}%
\pgfpathlineto{\pgfqpoint{2.562226in}{1.894163in}}%
\pgfpathlineto{\pgfqpoint{2.550577in}{1.890341in}}%
\pgfpathlineto{\pgfqpoint{2.556751in}{1.880077in}}%
\pgfpathlineto{\pgfqpoint{2.562929in}{1.869767in}}%
\pgfpathlineto{\pgfqpoint{2.569111in}{1.859410in}}%
\pgfpathlineto{\pgfqpoint{2.575298in}{1.849006in}}%
\pgfpathclose%
\pgfusepath{stroke,fill}%
\end{pgfscope}%
\begin{pgfscope}%
\pgfpathrectangle{\pgfqpoint{0.887500in}{0.275000in}}{\pgfqpoint{4.225000in}{4.225000in}}%
\pgfusepath{clip}%
\pgfsetbuttcap%
\pgfsetroundjoin%
\definecolor{currentfill}{rgb}{0.255645,0.260703,0.528312}%
\pgfsetfillcolor{currentfill}%
\pgfsetfillopacity{0.700000}%
\pgfsetlinewidth{0.501875pt}%
\definecolor{currentstroke}{rgb}{1.000000,1.000000,1.000000}%
\pgfsetstrokecolor{currentstroke}%
\pgfsetstrokeopacity{0.500000}%
\pgfsetdash{}{0pt}%
\pgfpathmoveto{\pgfqpoint{3.264967in}{1.601762in}}%
\pgfpathlineto{\pgfqpoint{3.276460in}{1.605836in}}%
\pgfpathlineto{\pgfqpoint{3.287947in}{1.609866in}}%
\pgfpathlineto{\pgfqpoint{3.299428in}{1.613826in}}%
\pgfpathlineto{\pgfqpoint{3.310903in}{1.617688in}}%
\pgfpathlineto{\pgfqpoint{3.322372in}{1.621433in}}%
\pgfpathlineto{\pgfqpoint{3.315982in}{1.633486in}}%
\pgfpathlineto{\pgfqpoint{3.309594in}{1.645440in}}%
\pgfpathlineto{\pgfqpoint{3.303208in}{1.657298in}}%
\pgfpathlineto{\pgfqpoint{3.296826in}{1.669062in}}%
\pgfpathlineto{\pgfqpoint{3.290446in}{1.680734in}}%
\pgfpathlineto{\pgfqpoint{3.278986in}{1.677337in}}%
\pgfpathlineto{\pgfqpoint{3.267519in}{1.673733in}}%
\pgfpathlineto{\pgfqpoint{3.256046in}{1.669947in}}%
\pgfpathlineto{\pgfqpoint{3.244566in}{1.666023in}}%
\pgfpathlineto{\pgfqpoint{3.233080in}{1.662005in}}%
\pgfpathlineto{\pgfqpoint{3.239452in}{1.650066in}}%
\pgfpathlineto{\pgfqpoint{3.245826in}{1.638065in}}%
\pgfpathlineto{\pgfqpoint{3.252203in}{1.626010in}}%
\pgfpathlineto{\pgfqpoint{3.258584in}{1.613907in}}%
\pgfpathclose%
\pgfusepath{stroke,fill}%
\end{pgfscope}%
\begin{pgfscope}%
\pgfpathrectangle{\pgfqpoint{0.887500in}{0.275000in}}{\pgfqpoint{4.225000in}{4.225000in}}%
\pgfusepath{clip}%
\pgfsetbuttcap%
\pgfsetroundjoin%
\definecolor{currentfill}{rgb}{0.271828,0.209303,0.504434}%
\pgfsetfillcolor{currentfill}%
\pgfsetfillopacity{0.700000}%
\pgfsetlinewidth{0.501875pt}%
\definecolor{currentstroke}{rgb}{1.000000,1.000000,1.000000}%
\pgfsetstrokecolor{currentstroke}%
\pgfsetstrokeopacity{0.500000}%
\pgfsetdash{}{0pt}%
\pgfpathmoveto{\pgfqpoint{3.443718in}{1.512950in}}%
\pgfpathlineto{\pgfqpoint{3.455166in}{1.516962in}}%
\pgfpathlineto{\pgfqpoint{3.466612in}{1.521223in}}%
\pgfpathlineto{\pgfqpoint{3.478054in}{1.525697in}}%
\pgfpathlineto{\pgfqpoint{3.489494in}{1.530328in}}%
\pgfpathlineto{\pgfqpoint{3.500929in}{1.535058in}}%
\pgfpathlineto{\pgfqpoint{3.494498in}{1.547632in}}%
\pgfpathlineto{\pgfqpoint{3.488072in}{1.560240in}}%
\pgfpathlineto{\pgfqpoint{3.481649in}{1.572908in}}%
\pgfpathlineto{\pgfqpoint{3.475230in}{1.585664in}}%
\pgfpathlineto{\pgfqpoint{3.468815in}{1.598532in}}%
\pgfpathlineto{\pgfqpoint{3.457394in}{1.594557in}}%
\pgfpathlineto{\pgfqpoint{3.445966in}{1.590384in}}%
\pgfpathlineto{\pgfqpoint{3.434531in}{1.586149in}}%
\pgfpathlineto{\pgfqpoint{3.423092in}{1.581989in}}%
\pgfpathlineto{\pgfqpoint{3.411649in}{1.578006in}}%
\pgfpathlineto{\pgfqpoint{3.418058in}{1.565134in}}%
\pgfpathlineto{\pgfqpoint{3.424469in}{1.552177in}}%
\pgfpathlineto{\pgfqpoint{3.430883in}{1.539149in}}%
\pgfpathlineto{\pgfqpoint{3.437299in}{1.526068in}}%
\pgfpathclose%
\pgfusepath{stroke,fill}%
\end{pgfscope}%
\begin{pgfscope}%
\pgfpathrectangle{\pgfqpoint{0.887500in}{0.275000in}}{\pgfqpoint{4.225000in}{4.225000in}}%
\pgfusepath{clip}%
\pgfsetbuttcap%
\pgfsetroundjoin%
\definecolor{currentfill}{rgb}{0.283072,0.130895,0.449241}%
\pgfsetfillcolor{currentfill}%
\pgfsetfillopacity{0.700000}%
\pgfsetlinewidth{0.501875pt}%
\definecolor{currentstroke}{rgb}{1.000000,1.000000,1.000000}%
\pgfsetstrokecolor{currentstroke}%
\pgfsetstrokeopacity{0.500000}%
\pgfsetdash{}{0pt}%
\pgfpathmoveto{\pgfqpoint{3.711714in}{1.374945in}}%
\pgfpathlineto{\pgfqpoint{3.723071in}{1.377591in}}%
\pgfpathlineto{\pgfqpoint{3.734421in}{1.380204in}}%
\pgfpathlineto{\pgfqpoint{3.745769in}{1.383067in}}%
\pgfpathlineto{\pgfqpoint{3.757121in}{1.386461in}}%
\pgfpathlineto{\pgfqpoint{3.768482in}{1.390667in}}%
\pgfpathlineto{\pgfqpoint{3.762078in}{1.408088in}}%
\pgfpathlineto{\pgfqpoint{3.755685in}{1.426038in}}%
\pgfpathlineto{\pgfqpoint{3.749300in}{1.444380in}}%
\pgfpathlineto{\pgfqpoint{3.742922in}{1.462981in}}%
\pgfpathlineto{\pgfqpoint{3.736546in}{1.481707in}}%
\pgfpathlineto{\pgfqpoint{3.725186in}{1.477671in}}%
\pgfpathlineto{\pgfqpoint{3.713830in}{1.474275in}}%
\pgfpathlineto{\pgfqpoint{3.702474in}{1.471154in}}%
\pgfpathlineto{\pgfqpoint{3.691110in}{1.467940in}}%
\pgfpathlineto{\pgfqpoint{3.679733in}{1.464266in}}%
\pgfpathlineto{\pgfqpoint{3.686132in}{1.446741in}}%
\pgfpathlineto{\pgfqpoint{3.692527in}{1.428886in}}%
\pgfpathlineto{\pgfqpoint{3.698920in}{1.410861in}}%
\pgfpathlineto{\pgfqpoint{3.705315in}{1.392827in}}%
\pgfpathclose%
\pgfusepath{stroke,fill}%
\end{pgfscope}%
\begin{pgfscope}%
\pgfpathrectangle{\pgfqpoint{0.887500in}{0.275000in}}{\pgfqpoint{4.225000in}{4.225000in}}%
\pgfusepath{clip}%
\pgfsetbuttcap%
\pgfsetroundjoin%
\definecolor{currentfill}{rgb}{0.263663,0.237631,0.518762}%
\pgfsetfillcolor{currentfill}%
\pgfsetfillopacity{0.700000}%
\pgfsetlinewidth{0.501875pt}%
\definecolor{currentstroke}{rgb}{1.000000,1.000000,1.000000}%
\pgfsetstrokecolor{currentstroke}%
\pgfsetstrokeopacity{0.500000}%
\pgfsetdash{}{0pt}%
\pgfpathmoveto{\pgfqpoint{3.354363in}{1.559622in}}%
\pgfpathlineto{\pgfqpoint{3.365831in}{1.563234in}}%
\pgfpathlineto{\pgfqpoint{3.377293in}{1.566836in}}%
\pgfpathlineto{\pgfqpoint{3.388750in}{1.570471in}}%
\pgfpathlineto{\pgfqpoint{3.400202in}{1.574180in}}%
\pgfpathlineto{\pgfqpoint{3.411649in}{1.578006in}}%
\pgfpathlineto{\pgfqpoint{3.405242in}{1.590775in}}%
\pgfpathlineto{\pgfqpoint{3.398837in}{1.603425in}}%
\pgfpathlineto{\pgfqpoint{3.392434in}{1.615943in}}%
\pgfpathlineto{\pgfqpoint{3.386033in}{1.628318in}}%
\pgfpathlineto{\pgfqpoint{3.379634in}{1.640539in}}%
\pgfpathlineto{\pgfqpoint{3.368190in}{1.636424in}}%
\pgfpathlineto{\pgfqpoint{3.356743in}{1.632529in}}%
\pgfpathlineto{\pgfqpoint{3.345291in}{1.628781in}}%
\pgfpathlineto{\pgfqpoint{3.333835in}{1.625107in}}%
\pgfpathlineto{\pgfqpoint{3.322372in}{1.621433in}}%
\pgfpathlineto{\pgfqpoint{3.328765in}{1.609280in}}%
\pgfpathlineto{\pgfqpoint{3.335161in}{1.597024in}}%
\pgfpathlineto{\pgfqpoint{3.341559in}{1.584664in}}%
\pgfpathlineto{\pgfqpoint{3.347960in}{1.572197in}}%
\pgfpathclose%
\pgfusepath{stroke,fill}%
\end{pgfscope}%
\begin{pgfscope}%
\pgfpathrectangle{\pgfqpoint{0.887500in}{0.275000in}}{\pgfqpoint{4.225000in}{4.225000in}}%
\pgfusepath{clip}%
\pgfsetbuttcap%
\pgfsetroundjoin%
\definecolor{currentfill}{rgb}{0.165117,0.467423,0.558141}%
\pgfsetfillcolor{currentfill}%
\pgfsetfillopacity{0.700000}%
\pgfsetlinewidth{0.501875pt}%
\definecolor{currentstroke}{rgb}{1.000000,1.000000,1.000000}%
\pgfsetstrokecolor{currentstroke}%
\pgfsetstrokeopacity{0.500000}%
\pgfsetdash{}{0pt}%
\pgfpathmoveto{\pgfqpoint{1.840048in}{2.013492in}}%
\pgfpathlineto{\pgfqpoint{1.851890in}{2.017286in}}%
\pgfpathlineto{\pgfqpoint{1.863726in}{2.021068in}}%
\pgfpathlineto{\pgfqpoint{1.875557in}{2.024837in}}%
\pgfpathlineto{\pgfqpoint{1.887383in}{2.028595in}}%
\pgfpathlineto{\pgfqpoint{1.899203in}{2.032344in}}%
\pgfpathlineto{\pgfqpoint{1.893247in}{2.041797in}}%
\pgfpathlineto{\pgfqpoint{1.887296in}{2.051219in}}%
\pgfpathlineto{\pgfqpoint{1.881349in}{2.060609in}}%
\pgfpathlineto{\pgfqpoint{1.875407in}{2.069967in}}%
\pgfpathlineto{\pgfqpoint{1.869470in}{2.079294in}}%
\pgfpathlineto{\pgfqpoint{1.857660in}{2.075601in}}%
\pgfpathlineto{\pgfqpoint{1.845845in}{2.071898in}}%
\pgfpathlineto{\pgfqpoint{1.834024in}{2.068184in}}%
\pgfpathlineto{\pgfqpoint{1.822198in}{2.064457in}}%
\pgfpathlineto{\pgfqpoint{1.810366in}{2.060717in}}%
\pgfpathlineto{\pgfqpoint{1.816294in}{2.051335in}}%
\pgfpathlineto{\pgfqpoint{1.822225in}{2.041921in}}%
\pgfpathlineto{\pgfqpoint{1.828161in}{2.032476in}}%
\pgfpathlineto{\pgfqpoint{1.834102in}{2.023000in}}%
\pgfpathclose%
\pgfusepath{stroke,fill}%
\end{pgfscope}%
\begin{pgfscope}%
\pgfpathrectangle{\pgfqpoint{0.887500in}{0.275000in}}{\pgfqpoint{4.225000in}{4.225000in}}%
\pgfusepath{clip}%
\pgfsetbuttcap%
\pgfsetroundjoin%
\definecolor{currentfill}{rgb}{0.283229,0.120777,0.440584}%
\pgfsetfillcolor{currentfill}%
\pgfsetfillopacity{0.700000}%
\pgfsetlinewidth{0.501875pt}%
\definecolor{currentstroke}{rgb}{1.000000,1.000000,1.000000}%
\pgfsetstrokecolor{currentstroke}%
\pgfsetstrokeopacity{0.500000}%
\pgfsetdash{}{0pt}%
\pgfpathmoveto{\pgfqpoint{3.947929in}{1.346141in}}%
\pgfpathlineto{\pgfqpoint{3.959321in}{1.353226in}}%
\pgfpathlineto{\pgfqpoint{3.970715in}{1.360525in}}%
\pgfpathlineto{\pgfqpoint{3.982114in}{1.368152in}}%
\pgfpathlineto{\pgfqpoint{3.993520in}{1.376220in}}%
\pgfpathlineto{\pgfqpoint{4.004936in}{1.384843in}}%
\pgfpathlineto{\pgfqpoint{3.998630in}{1.407496in}}%
\pgfpathlineto{\pgfqpoint{3.992304in}{1.429331in}}%
\pgfpathlineto{\pgfqpoint{3.985959in}{1.450411in}}%
\pgfpathlineto{\pgfqpoint{3.979599in}{1.470798in}}%
\pgfpathlineto{\pgfqpoint{3.973224in}{1.490554in}}%
\pgfpathlineto{\pgfqpoint{3.961692in}{1.477158in}}%
\pgfpathlineto{\pgfqpoint{3.950149in}{1.463254in}}%
\pgfpathlineto{\pgfqpoint{3.938599in}{1.449054in}}%
\pgfpathlineto{\pgfqpoint{3.927048in}{1.434772in}}%
\pgfpathlineto{\pgfqpoint{3.915501in}{1.420623in}}%
\pgfpathlineto{\pgfqpoint{3.921969in}{1.405153in}}%
\pgfpathlineto{\pgfqpoint{3.928445in}{1.389965in}}%
\pgfpathlineto{\pgfqpoint{3.934930in}{1.375064in}}%
\pgfpathlineto{\pgfqpoint{3.941425in}{1.360455in}}%
\pgfpathclose%
\pgfusepath{stroke,fill}%
\end{pgfscope}%
\begin{pgfscope}%
\pgfpathrectangle{\pgfqpoint{0.887500in}{0.275000in}}{\pgfqpoint{4.225000in}{4.225000in}}%
\pgfusepath{clip}%
\pgfsetbuttcap%
\pgfsetroundjoin%
\definecolor{currentfill}{rgb}{0.277134,0.185228,0.489898}%
\pgfsetfillcolor{currentfill}%
\pgfsetfillopacity{0.700000}%
\pgfsetlinewidth{0.501875pt}%
\definecolor{currentstroke}{rgb}{1.000000,1.000000,1.000000}%
\pgfsetstrokecolor{currentstroke}%
\pgfsetstrokeopacity{0.500000}%
\pgfsetdash{}{0pt}%
\pgfpathmoveto{\pgfqpoint{3.533123in}{1.471787in}}%
\pgfpathlineto{\pgfqpoint{3.544575in}{1.477711in}}%
\pgfpathlineto{\pgfqpoint{3.556027in}{1.484018in}}%
\pgfpathlineto{\pgfqpoint{3.567480in}{1.490724in}}%
\pgfpathlineto{\pgfqpoint{3.578935in}{1.497813in}}%
\pgfpathlineto{\pgfqpoint{3.590389in}{1.505172in}}%
\pgfpathlineto{\pgfqpoint{3.583925in}{1.517032in}}%
\pgfpathlineto{\pgfqpoint{3.577454in}{1.528130in}}%
\pgfpathlineto{\pgfqpoint{3.570979in}{1.538624in}}%
\pgfpathlineto{\pgfqpoint{3.564503in}{1.548713in}}%
\pgfpathlineto{\pgfqpoint{3.558027in}{1.558591in}}%
\pgfpathlineto{\pgfqpoint{3.546618in}{1.553875in}}%
\pgfpathlineto{\pgfqpoint{3.535204in}{1.549267in}}%
\pgfpathlineto{\pgfqpoint{3.523785in}{1.544585in}}%
\pgfpathlineto{\pgfqpoint{3.512359in}{1.539829in}}%
\pgfpathlineto{\pgfqpoint{3.500929in}{1.535058in}}%
\pgfpathlineto{\pgfqpoint{3.507362in}{1.522492in}}%
\pgfpathlineto{\pgfqpoint{3.513799in}{1.509908in}}%
\pgfpathlineto{\pgfqpoint{3.520238in}{1.497280in}}%
\pgfpathlineto{\pgfqpoint{3.526680in}{1.484581in}}%
\pgfpathclose%
\pgfusepath{stroke,fill}%
\end{pgfscope}%
\begin{pgfscope}%
\pgfpathrectangle{\pgfqpoint{0.887500in}{0.275000in}}{\pgfqpoint{4.225000in}{4.225000in}}%
\pgfusepath{clip}%
\pgfsetbuttcap%
\pgfsetroundjoin%
\definecolor{currentfill}{rgb}{0.208623,0.367752,0.552675}%
\pgfsetfillcolor{currentfill}%
\pgfsetfillopacity{0.700000}%
\pgfsetlinewidth{0.501875pt}%
\definecolor{currentstroke}{rgb}{1.000000,1.000000,1.000000}%
\pgfsetstrokecolor{currentstroke}%
\pgfsetstrokeopacity{0.500000}%
\pgfsetdash{}{0pt}%
\pgfpathmoveto{\pgfqpoint{2.670788in}{1.805051in}}%
\pgfpathlineto{\pgfqpoint{2.682429in}{1.808897in}}%
\pgfpathlineto{\pgfqpoint{2.694063in}{1.812732in}}%
\pgfpathlineto{\pgfqpoint{2.705692in}{1.816560in}}%
\pgfpathlineto{\pgfqpoint{2.717316in}{1.820385in}}%
\pgfpathlineto{\pgfqpoint{2.728933in}{1.824209in}}%
\pgfpathlineto{\pgfqpoint{2.722704in}{1.834793in}}%
\pgfpathlineto{\pgfqpoint{2.716479in}{1.845326in}}%
\pgfpathlineto{\pgfqpoint{2.710257in}{1.855809in}}%
\pgfpathlineto{\pgfqpoint{2.704040in}{1.866242in}}%
\pgfpathlineto{\pgfqpoint{2.697827in}{1.876626in}}%
\pgfpathlineto{\pgfqpoint{2.686218in}{1.872860in}}%
\pgfpathlineto{\pgfqpoint{2.674604in}{1.869094in}}%
\pgfpathlineto{\pgfqpoint{2.662984in}{1.865324in}}%
\pgfpathlineto{\pgfqpoint{2.651358in}{1.861546in}}%
\pgfpathlineto{\pgfqpoint{2.639727in}{1.857758in}}%
\pgfpathlineto{\pgfqpoint{2.645931in}{1.847315in}}%
\pgfpathlineto{\pgfqpoint{2.652139in}{1.836824in}}%
\pgfpathlineto{\pgfqpoint{2.658352in}{1.826284in}}%
\pgfpathlineto{\pgfqpoint{2.664568in}{1.815693in}}%
\pgfpathclose%
\pgfusepath{stroke,fill}%
\end{pgfscope}%
\begin{pgfscope}%
\pgfpathrectangle{\pgfqpoint{0.887500in}{0.275000in}}{\pgfqpoint{4.225000in}{4.225000in}}%
\pgfusepath{clip}%
\pgfsetbuttcap%
\pgfsetroundjoin%
\definecolor{currentfill}{rgb}{0.183898,0.422383,0.556944}%
\pgfsetfillcolor{currentfill}%
\pgfsetfillopacity{0.700000}%
\pgfsetlinewidth{0.501875pt}%
\definecolor{currentstroke}{rgb}{1.000000,1.000000,1.000000}%
\pgfsetstrokecolor{currentstroke}%
\pgfsetstrokeopacity{0.500000}%
\pgfsetdash{}{0pt}%
\pgfpathmoveto{\pgfqpoint{2.255303in}{1.914225in}}%
\pgfpathlineto{\pgfqpoint{2.267045in}{1.918071in}}%
\pgfpathlineto{\pgfqpoint{2.278782in}{1.921907in}}%
\pgfpathlineto{\pgfqpoint{2.290513in}{1.925735in}}%
\pgfpathlineto{\pgfqpoint{2.302238in}{1.929554in}}%
\pgfpathlineto{\pgfqpoint{2.313958in}{1.933369in}}%
\pgfpathlineto{\pgfqpoint{2.307861in}{1.943288in}}%
\pgfpathlineto{\pgfqpoint{2.301768in}{1.953164in}}%
\pgfpathlineto{\pgfqpoint{2.295680in}{1.962997in}}%
\pgfpathlineto{\pgfqpoint{2.289596in}{1.972788in}}%
\pgfpathlineto{\pgfqpoint{2.283517in}{1.982536in}}%
\pgfpathlineto{\pgfqpoint{2.271807in}{1.978785in}}%
\pgfpathlineto{\pgfqpoint{2.260092in}{1.975029in}}%
\pgfpathlineto{\pgfqpoint{2.248370in}{1.971265in}}%
\pgfpathlineto{\pgfqpoint{2.236644in}{1.967492in}}%
\pgfpathlineto{\pgfqpoint{2.224911in}{1.963708in}}%
\pgfpathlineto{\pgfqpoint{2.230981in}{1.953896in}}%
\pgfpathlineto{\pgfqpoint{2.237054in}{1.944041in}}%
\pgfpathlineto{\pgfqpoint{2.243133in}{1.934145in}}%
\pgfpathlineto{\pgfqpoint{2.249216in}{1.924207in}}%
\pgfpathclose%
\pgfusepath{stroke,fill}%
\end{pgfscope}%
\begin{pgfscope}%
\pgfpathrectangle{\pgfqpoint{0.887500in}{0.275000in}}{\pgfqpoint{4.225000in}{4.225000in}}%
\pgfusepath{clip}%
\pgfsetbuttcap%
\pgfsetroundjoin%
\definecolor{currentfill}{rgb}{0.216210,0.351535,0.550627}%
\pgfsetfillcolor{currentfill}%
\pgfsetfillopacity{0.700000}%
\pgfsetlinewidth{0.501875pt}%
\definecolor{currentstroke}{rgb}{1.000000,1.000000,1.000000}%
\pgfsetstrokecolor{currentstroke}%
\pgfsetstrokeopacity{0.500000}%
\pgfsetdash{}{0pt}%
\pgfpathmoveto{\pgfqpoint{2.760140in}{1.770507in}}%
\pgfpathlineto{\pgfqpoint{2.771761in}{1.774394in}}%
\pgfpathlineto{\pgfqpoint{2.783376in}{1.778285in}}%
\pgfpathlineto{\pgfqpoint{2.794985in}{1.782182in}}%
\pgfpathlineto{\pgfqpoint{2.806589in}{1.786083in}}%
\pgfpathlineto{\pgfqpoint{2.818187in}{1.789990in}}%
\pgfpathlineto{\pgfqpoint{2.811929in}{1.800776in}}%
\pgfpathlineto{\pgfqpoint{2.805675in}{1.811510in}}%
\pgfpathlineto{\pgfqpoint{2.799425in}{1.822191in}}%
\pgfpathlineto{\pgfqpoint{2.793178in}{1.832820in}}%
\pgfpathlineto{\pgfqpoint{2.786936in}{1.843397in}}%
\pgfpathlineto{\pgfqpoint{2.775347in}{1.839546in}}%
\pgfpathlineto{\pgfqpoint{2.763752in}{1.835703in}}%
\pgfpathlineto{\pgfqpoint{2.752152in}{1.831867in}}%
\pgfpathlineto{\pgfqpoint{2.740545in}{1.828036in}}%
\pgfpathlineto{\pgfqpoint{2.728933in}{1.824209in}}%
\pgfpathlineto{\pgfqpoint{2.735167in}{1.813573in}}%
\pgfpathlineto{\pgfqpoint{2.741404in}{1.802885in}}%
\pgfpathlineto{\pgfqpoint{2.747645in}{1.792145in}}%
\pgfpathlineto{\pgfqpoint{2.753891in}{1.781352in}}%
\pgfpathclose%
\pgfusepath{stroke,fill}%
\end{pgfscope}%
\begin{pgfscope}%
\pgfpathrectangle{\pgfqpoint{0.887500in}{0.275000in}}{\pgfqpoint{4.225000in}{4.225000in}}%
\pgfusepath{clip}%
\pgfsetbuttcap%
\pgfsetroundjoin%
\definecolor{currentfill}{rgb}{0.281924,0.089666,0.412415}%
\pgfsetfillcolor{currentfill}%
\pgfsetfillopacity{0.700000}%
\pgfsetlinewidth{0.501875pt}%
\definecolor{currentstroke}{rgb}{1.000000,1.000000,1.000000}%
\pgfsetstrokecolor{currentstroke}%
\pgfsetstrokeopacity{0.500000}%
\pgfsetdash{}{0pt}%
\pgfpathmoveto{\pgfqpoint{3.890914in}{1.310882in}}%
\pgfpathlineto{\pgfqpoint{3.902326in}{1.318018in}}%
\pgfpathlineto{\pgfqpoint{3.913734in}{1.325117in}}%
\pgfpathlineto{\pgfqpoint{3.925137in}{1.332167in}}%
\pgfpathlineto{\pgfqpoint{3.936535in}{1.339159in}}%
\pgfpathlineto{\pgfqpoint{3.947929in}{1.346141in}}%
\pgfpathlineto{\pgfqpoint{3.941425in}{1.360455in}}%
\pgfpathlineto{\pgfqpoint{3.934930in}{1.375064in}}%
\pgfpathlineto{\pgfqpoint{3.928445in}{1.389965in}}%
\pgfpathlineto{\pgfqpoint{3.921969in}{1.405153in}}%
\pgfpathlineto{\pgfqpoint{3.915501in}{1.420623in}}%
\pgfpathlineto{\pgfqpoint{3.903962in}{1.406819in}}%
\pgfpathlineto{\pgfqpoint{3.892435in}{1.393570in}}%
\pgfpathlineto{\pgfqpoint{3.880922in}{1.381004in}}%
\pgfpathlineto{\pgfqpoint{3.869425in}{1.369173in}}%
\pgfpathlineto{\pgfqpoint{3.857943in}{1.358123in}}%
\pgfpathlineto{\pgfqpoint{3.864478in}{1.346299in}}%
\pgfpathlineto{\pgfqpoint{3.871041in}{1.335588in}}%
\pgfpathlineto{\pgfqpoint{3.877633in}{1.326065in}}%
\pgfpathlineto{\pgfqpoint{3.884257in}{1.317804in}}%
\pgfpathclose%
\pgfusepath{stroke,fill}%
\end{pgfscope}%
\begin{pgfscope}%
\pgfpathrectangle{\pgfqpoint{0.887500in}{0.275000in}}{\pgfqpoint{4.225000in}{4.225000in}}%
\pgfusepath{clip}%
\pgfsetbuttcap%
\pgfsetroundjoin%
\definecolor{currentfill}{rgb}{0.169646,0.456262,0.558030}%
\pgfsetfillcolor{currentfill}%
\pgfsetfillopacity{0.700000}%
\pgfsetlinewidth{0.501875pt}%
\definecolor{currentstroke}{rgb}{1.000000,1.000000,1.000000}%
\pgfsetstrokecolor{currentstroke}%
\pgfsetstrokeopacity{0.500000}%
\pgfsetdash{}{0pt}%
\pgfpathmoveto{\pgfqpoint{1.929050in}{1.984588in}}%
\pgfpathlineto{\pgfqpoint{1.940875in}{1.988383in}}%
\pgfpathlineto{\pgfqpoint{1.952694in}{1.992171in}}%
\pgfpathlineto{\pgfqpoint{1.964508in}{1.995954in}}%
\pgfpathlineto{\pgfqpoint{1.976316in}{1.999731in}}%
\pgfpathlineto{\pgfqpoint{1.988118in}{2.003506in}}%
\pgfpathlineto{\pgfqpoint{1.982129in}{2.013068in}}%
\pgfpathlineto{\pgfqpoint{1.976145in}{2.022595in}}%
\pgfpathlineto{\pgfqpoint{1.970165in}{2.032090in}}%
\pgfpathlineto{\pgfqpoint{1.964190in}{2.041551in}}%
\pgfpathlineto{\pgfqpoint{1.958219in}{2.050979in}}%
\pgfpathlineto{\pgfqpoint{1.946427in}{2.047263in}}%
\pgfpathlineto{\pgfqpoint{1.934629in}{2.043542in}}%
\pgfpathlineto{\pgfqpoint{1.922826in}{2.039816in}}%
\pgfpathlineto{\pgfqpoint{1.911017in}{2.036083in}}%
\pgfpathlineto{\pgfqpoint{1.899203in}{2.032344in}}%
\pgfpathlineto{\pgfqpoint{1.905163in}{2.022858in}}%
\pgfpathlineto{\pgfqpoint{1.911128in}{2.013340in}}%
\pgfpathlineto{\pgfqpoint{1.917098in}{2.003789in}}%
\pgfpathlineto{\pgfqpoint{1.923072in}{1.994206in}}%
\pgfpathclose%
\pgfusepath{stroke,fill}%
\end{pgfscope}%
\begin{pgfscope}%
\pgfpathrectangle{\pgfqpoint{0.887500in}{0.275000in}}{\pgfqpoint{4.225000in}{4.225000in}}%
\pgfusepath{clip}%
\pgfsetbuttcap%
\pgfsetroundjoin%
\definecolor{currentfill}{rgb}{0.279574,0.170599,0.479997}%
\pgfsetfillcolor{currentfill}%
\pgfsetfillopacity{0.700000}%
\pgfsetlinewidth{0.501875pt}%
\definecolor{currentstroke}{rgb}{1.000000,1.000000,1.000000}%
\pgfsetstrokecolor{currentstroke}%
\pgfsetstrokeopacity{0.500000}%
\pgfsetdash{}{0pt}%
\pgfpathmoveto{\pgfqpoint{3.622618in}{1.436137in}}%
\pgfpathlineto{\pgfqpoint{3.634062in}{1.442577in}}%
\pgfpathlineto{\pgfqpoint{3.645499in}{1.448788in}}%
\pgfpathlineto{\pgfqpoint{3.656925in}{1.454591in}}%
\pgfpathlineto{\pgfqpoint{3.668338in}{1.459809in}}%
\pgfpathlineto{\pgfqpoint{3.679733in}{1.464266in}}%
\pgfpathlineto{\pgfqpoint{3.673328in}{1.481299in}}%
\pgfpathlineto{\pgfqpoint{3.666914in}{1.497680in}}%
\pgfpathlineto{\pgfqpoint{3.660489in}{1.513246in}}%
\pgfpathlineto{\pgfqpoint{3.654052in}{1.527837in}}%
\pgfpathlineto{\pgfqpoint{3.647599in}{1.541291in}}%
\pgfpathlineto{\pgfqpoint{3.636173in}{1.534606in}}%
\pgfpathlineto{\pgfqpoint{3.624736in}{1.527518in}}%
\pgfpathlineto{\pgfqpoint{3.613292in}{1.520159in}}%
\pgfpathlineto{\pgfqpoint{3.601842in}{1.512666in}}%
\pgfpathlineto{\pgfqpoint{3.590389in}{1.505172in}}%
\pgfpathlineto{\pgfqpoint{3.596846in}{1.492570in}}%
\pgfpathlineto{\pgfqpoint{3.603297in}{1.479294in}}%
\pgfpathlineto{\pgfqpoint{3.609742in}{1.465416in}}%
\pgfpathlineto{\pgfqpoint{3.616182in}{1.451007in}}%
\pgfpathclose%
\pgfusepath{stroke,fill}%
\end{pgfscope}%
\begin{pgfscope}%
\pgfpathrectangle{\pgfqpoint{0.887500in}{0.275000in}}{\pgfqpoint{4.225000in}{4.225000in}}%
\pgfusepath{clip}%
\pgfsetbuttcap%
\pgfsetroundjoin%
\definecolor{currentfill}{rgb}{0.190631,0.407061,0.556089}%
\pgfsetfillcolor{currentfill}%
\pgfsetfillopacity{0.700000}%
\pgfsetlinewidth{0.501875pt}%
\definecolor{currentstroke}{rgb}{1.000000,1.000000,1.000000}%
\pgfsetstrokecolor{currentstroke}%
\pgfsetstrokeopacity{0.500000}%
\pgfsetdash{}{0pt}%
\pgfpathmoveto{\pgfqpoint{2.344510in}{1.883102in}}%
\pgfpathlineto{\pgfqpoint{2.356234in}{1.886971in}}%
\pgfpathlineto{\pgfqpoint{2.367952in}{1.890839in}}%
\pgfpathlineto{\pgfqpoint{2.379664in}{1.894709in}}%
\pgfpathlineto{\pgfqpoint{2.391371in}{1.898583in}}%
\pgfpathlineto{\pgfqpoint{2.403071in}{1.902463in}}%
\pgfpathlineto{\pgfqpoint{2.396942in}{1.912552in}}%
\pgfpathlineto{\pgfqpoint{2.390818in}{1.922594in}}%
\pgfpathlineto{\pgfqpoint{2.384697in}{1.932591in}}%
\pgfpathlineto{\pgfqpoint{2.378582in}{1.942541in}}%
\pgfpathlineto{\pgfqpoint{2.372470in}{1.952444in}}%
\pgfpathlineto{\pgfqpoint{2.360779in}{1.948621in}}%
\pgfpathlineto{\pgfqpoint{2.349083in}{1.944804in}}%
\pgfpathlineto{\pgfqpoint{2.337380in}{1.940991in}}%
\pgfpathlineto{\pgfqpoint{2.325672in}{1.937180in}}%
\pgfpathlineto{\pgfqpoint{2.313958in}{1.933369in}}%
\pgfpathlineto{\pgfqpoint{2.320059in}{1.923405in}}%
\pgfpathlineto{\pgfqpoint{2.326165in}{1.913396in}}%
\pgfpathlineto{\pgfqpoint{2.332276in}{1.903343in}}%
\pgfpathlineto{\pgfqpoint{2.338391in}{1.893245in}}%
\pgfpathclose%
\pgfusepath{stroke,fill}%
\end{pgfscope}%
\begin{pgfscope}%
\pgfpathrectangle{\pgfqpoint{0.887500in}{0.275000in}}{\pgfqpoint{4.225000in}{4.225000in}}%
\pgfusepath{clip}%
\pgfsetbuttcap%
\pgfsetroundjoin%
\definecolor{currentfill}{rgb}{0.280894,0.078907,0.402329}%
\pgfsetfillcolor{currentfill}%
\pgfsetfillopacity{0.700000}%
\pgfsetlinewidth{0.501875pt}%
\definecolor{currentstroke}{rgb}{1.000000,1.000000,1.000000}%
\pgfsetstrokecolor{currentstroke}%
\pgfsetstrokeopacity{0.500000}%
\pgfsetdash{}{0pt}%
\pgfpathmoveto{\pgfqpoint{3.743869in}{1.293406in}}%
\pgfpathlineto{\pgfqpoint{3.755242in}{1.297014in}}%
\pgfpathlineto{\pgfqpoint{3.766617in}{1.300982in}}%
\pgfpathlineto{\pgfqpoint{3.777994in}{1.305424in}}%
\pgfpathlineto{\pgfqpoint{3.789378in}{1.310451in}}%
\pgfpathlineto{\pgfqpoint{3.800770in}{1.316172in}}%
\pgfpathlineto{\pgfqpoint{3.794267in}{1.328942in}}%
\pgfpathlineto{\pgfqpoint{3.787790in}{1.342911in}}%
\pgfpathlineto{\pgfqpoint{3.781335in}{1.357944in}}%
\pgfpathlineto{\pgfqpoint{3.774900in}{1.373907in}}%
\pgfpathlineto{\pgfqpoint{3.768482in}{1.390667in}}%
\pgfpathlineto{\pgfqpoint{3.757121in}{1.386461in}}%
\pgfpathlineto{\pgfqpoint{3.745769in}{1.383067in}}%
\pgfpathlineto{\pgfqpoint{3.734421in}{1.380204in}}%
\pgfpathlineto{\pgfqpoint{3.723071in}{1.377591in}}%
\pgfpathlineto{\pgfqpoint{3.711714in}{1.374945in}}%
\pgfpathlineto{\pgfqpoint{3.718119in}{1.357375in}}%
\pgfpathlineto{\pgfqpoint{3.724534in}{1.340276in}}%
\pgfpathlineto{\pgfqpoint{3.730962in}{1.323808in}}%
\pgfpathlineto{\pgfqpoint{3.737406in}{1.308132in}}%
\pgfpathclose%
\pgfusepath{stroke,fill}%
\end{pgfscope}%
\begin{pgfscope}%
\pgfpathrectangle{\pgfqpoint{0.887500in}{0.275000in}}{\pgfqpoint{4.225000in}{4.225000in}}%
\pgfusepath{clip}%
\pgfsetbuttcap%
\pgfsetroundjoin%
\definecolor{currentfill}{rgb}{0.223925,0.334994,0.548053}%
\pgfsetfillcolor{currentfill}%
\pgfsetfillopacity{0.700000}%
\pgfsetlinewidth{0.501875pt}%
\definecolor{currentstroke}{rgb}{1.000000,1.000000,1.000000}%
\pgfsetstrokecolor{currentstroke}%
\pgfsetstrokeopacity{0.500000}%
\pgfsetdash{}{0pt}%
\pgfpathmoveto{\pgfqpoint{2.849534in}{1.735278in}}%
\pgfpathlineto{\pgfqpoint{2.861135in}{1.739238in}}%
\pgfpathlineto{\pgfqpoint{2.872730in}{1.743199in}}%
\pgfpathlineto{\pgfqpoint{2.884320in}{1.747159in}}%
\pgfpathlineto{\pgfqpoint{2.895904in}{1.751112in}}%
\pgfpathlineto{\pgfqpoint{2.907483in}{1.755050in}}%
\pgfpathlineto{\pgfqpoint{2.901197in}{1.766062in}}%
\pgfpathlineto{\pgfqpoint{2.894916in}{1.777018in}}%
\pgfpathlineto{\pgfqpoint{2.888637in}{1.787918in}}%
\pgfpathlineto{\pgfqpoint{2.882363in}{1.798763in}}%
\pgfpathlineto{\pgfqpoint{2.876093in}{1.809553in}}%
\pgfpathlineto{\pgfqpoint{2.864522in}{1.805659in}}%
\pgfpathlineto{\pgfqpoint{2.852947in}{1.801745in}}%
\pgfpathlineto{\pgfqpoint{2.841366in}{1.797823in}}%
\pgfpathlineto{\pgfqpoint{2.829779in}{1.793903in}}%
\pgfpathlineto{\pgfqpoint{2.818187in}{1.789990in}}%
\pgfpathlineto{\pgfqpoint{2.824448in}{1.779152in}}%
\pgfpathlineto{\pgfqpoint{2.830714in}{1.768261in}}%
\pgfpathlineto{\pgfqpoint{2.836984in}{1.757319in}}%
\pgfpathlineto{\pgfqpoint{2.843257in}{1.746324in}}%
\pgfpathclose%
\pgfusepath{stroke,fill}%
\end{pgfscope}%
\begin{pgfscope}%
\pgfpathrectangle{\pgfqpoint{0.887500in}{0.275000in}}{\pgfqpoint{4.225000in}{4.225000in}}%
\pgfusepath{clip}%
\pgfsetbuttcap%
\pgfsetroundjoin%
\definecolor{currentfill}{rgb}{0.174274,0.445044,0.557792}%
\pgfsetfillcolor{currentfill}%
\pgfsetfillopacity{0.700000}%
\pgfsetlinewidth{0.501875pt}%
\definecolor{currentstroke}{rgb}{1.000000,1.000000,1.000000}%
\pgfsetstrokecolor{currentstroke}%
\pgfsetstrokeopacity{0.500000}%
\pgfsetdash{}{0pt}%
\pgfpathmoveto{\pgfqpoint{2.018131in}{1.955170in}}%
\pgfpathlineto{\pgfqpoint{2.029937in}{1.958997in}}%
\pgfpathlineto{\pgfqpoint{2.041738in}{1.962822in}}%
\pgfpathlineto{\pgfqpoint{2.053533in}{1.966648in}}%
\pgfpathlineto{\pgfqpoint{2.065322in}{1.970475in}}%
\pgfpathlineto{\pgfqpoint{2.077106in}{1.974303in}}%
\pgfpathlineto{\pgfqpoint{2.071084in}{1.983989in}}%
\pgfpathlineto{\pgfqpoint{2.065066in}{1.993637in}}%
\pgfpathlineto{\pgfqpoint{2.059054in}{2.003248in}}%
\pgfpathlineto{\pgfqpoint{2.053045in}{2.012823in}}%
\pgfpathlineto{\pgfqpoint{2.047042in}{2.022362in}}%
\pgfpathlineto{\pgfqpoint{2.035268in}{2.018590in}}%
\pgfpathlineto{\pgfqpoint{2.023490in}{2.014818in}}%
\pgfpathlineto{\pgfqpoint{2.011705in}{2.011048in}}%
\pgfpathlineto{\pgfqpoint{1.999914in}{2.007278in}}%
\pgfpathlineto{\pgfqpoint{1.988118in}{2.003506in}}%
\pgfpathlineto{\pgfqpoint{1.994111in}{1.993909in}}%
\pgfpathlineto{\pgfqpoint{2.000109in}{1.984278in}}%
\pgfpathlineto{\pgfqpoint{2.006112in}{1.974611in}}%
\pgfpathlineto{\pgfqpoint{2.012119in}{1.964909in}}%
\pgfpathclose%
\pgfusepath{stroke,fill}%
\end{pgfscope}%
\begin{pgfscope}%
\pgfpathrectangle{\pgfqpoint{0.887500in}{0.275000in}}{\pgfqpoint{4.225000in}{4.225000in}}%
\pgfusepath{clip}%
\pgfsetbuttcap%
\pgfsetroundjoin%
\definecolor{currentfill}{rgb}{0.231674,0.318106,0.544834}%
\pgfsetfillcolor{currentfill}%
\pgfsetfillopacity{0.700000}%
\pgfsetlinewidth{0.501875pt}%
\definecolor{currentstroke}{rgb}{1.000000,1.000000,1.000000}%
\pgfsetstrokecolor{currentstroke}%
\pgfsetstrokeopacity{0.500000}%
\pgfsetdash{}{0pt}%
\pgfpathmoveto{\pgfqpoint{2.938967in}{1.699120in}}%
\pgfpathlineto{\pgfqpoint{2.950548in}{1.703062in}}%
\pgfpathlineto{\pgfqpoint{2.962124in}{1.706991in}}%
\pgfpathlineto{\pgfqpoint{2.973695in}{1.710904in}}%
\pgfpathlineto{\pgfqpoint{2.985259in}{1.714802in}}%
\pgfpathlineto{\pgfqpoint{2.996819in}{1.718681in}}%
\pgfpathlineto{\pgfqpoint{2.990506in}{1.729933in}}%
\pgfpathlineto{\pgfqpoint{2.984198in}{1.741121in}}%
\pgfpathlineto{\pgfqpoint{2.977893in}{1.752242in}}%
\pgfpathlineto{\pgfqpoint{2.971592in}{1.763296in}}%
\pgfpathlineto{\pgfqpoint{2.965294in}{1.774285in}}%
\pgfpathlineto{\pgfqpoint{2.953743in}{1.770517in}}%
\pgfpathlineto{\pgfqpoint{2.942186in}{1.766707in}}%
\pgfpathlineto{\pgfqpoint{2.930624in}{1.762854in}}%
\pgfpathlineto{\pgfqpoint{2.919056in}{1.758966in}}%
\pgfpathlineto{\pgfqpoint{2.907483in}{1.755050in}}%
\pgfpathlineto{\pgfqpoint{2.913772in}{1.743980in}}%
\pgfpathlineto{\pgfqpoint{2.920065in}{1.732852in}}%
\pgfpathlineto{\pgfqpoint{2.926362in}{1.721666in}}%
\pgfpathlineto{\pgfqpoint{2.932663in}{1.710422in}}%
\pgfpathclose%
\pgfusepath{stroke,fill}%
\end{pgfscope}%
\begin{pgfscope}%
\pgfpathrectangle{\pgfqpoint{0.887500in}{0.275000in}}{\pgfqpoint{4.225000in}{4.225000in}}%
\pgfusepath{clip}%
\pgfsetbuttcap%
\pgfsetroundjoin%
\definecolor{currentfill}{rgb}{0.195860,0.395433,0.555276}%
\pgfsetfillcolor{currentfill}%
\pgfsetfillopacity{0.700000}%
\pgfsetlinewidth{0.501875pt}%
\definecolor{currentstroke}{rgb}{1.000000,1.000000,1.000000}%
\pgfsetstrokecolor{currentstroke}%
\pgfsetstrokeopacity{0.500000}%
\pgfsetdash{}{0pt}%
\pgfpathmoveto{\pgfqpoint{2.433782in}{1.851328in}}%
\pgfpathlineto{\pgfqpoint{2.445486in}{1.855261in}}%
\pgfpathlineto{\pgfqpoint{2.457185in}{1.859195in}}%
\pgfpathlineto{\pgfqpoint{2.468878in}{1.863127in}}%
\pgfpathlineto{\pgfqpoint{2.480566in}{1.867054in}}%
\pgfpathlineto{\pgfqpoint{2.492248in}{1.870972in}}%
\pgfpathlineto{\pgfqpoint{2.486088in}{1.881243in}}%
\pgfpathlineto{\pgfqpoint{2.479932in}{1.891468in}}%
\pgfpathlineto{\pgfqpoint{2.473780in}{1.901646in}}%
\pgfpathlineto{\pgfqpoint{2.467632in}{1.911777in}}%
\pgfpathlineto{\pgfqpoint{2.461489in}{1.921860in}}%
\pgfpathlineto{\pgfqpoint{2.449817in}{1.917993in}}%
\pgfpathlineto{\pgfqpoint{2.438139in}{1.914116in}}%
\pgfpathlineto{\pgfqpoint{2.426455in}{1.910233in}}%
\pgfpathlineto{\pgfqpoint{2.414766in}{1.906347in}}%
\pgfpathlineto{\pgfqpoint{2.403071in}{1.902463in}}%
\pgfpathlineto{\pgfqpoint{2.409205in}{1.892328in}}%
\pgfpathlineto{\pgfqpoint{2.415342in}{1.882146in}}%
\pgfpathlineto{\pgfqpoint{2.421484in}{1.871919in}}%
\pgfpathlineto{\pgfqpoint{2.427631in}{1.861646in}}%
\pgfpathclose%
\pgfusepath{stroke,fill}%
\end{pgfscope}%
\begin{pgfscope}%
\pgfpathrectangle{\pgfqpoint{0.887500in}{0.275000in}}{\pgfqpoint{4.225000in}{4.225000in}}%
\pgfusepath{clip}%
\pgfsetbuttcap%
\pgfsetroundjoin%
\definecolor{currentfill}{rgb}{0.239346,0.300855,0.540844}%
\pgfsetfillcolor{currentfill}%
\pgfsetfillopacity{0.700000}%
\pgfsetlinewidth{0.501875pt}%
\definecolor{currentstroke}{rgb}{1.000000,1.000000,1.000000}%
\pgfsetstrokecolor{currentstroke}%
\pgfsetstrokeopacity{0.500000}%
\pgfsetdash{}{0pt}%
\pgfpathmoveto{\pgfqpoint{3.028432in}{1.661596in}}%
\pgfpathlineto{\pgfqpoint{3.039994in}{1.665526in}}%
\pgfpathlineto{\pgfqpoint{3.051550in}{1.669441in}}%
\pgfpathlineto{\pgfqpoint{3.063100in}{1.673337in}}%
\pgfpathlineto{\pgfqpoint{3.074645in}{1.677212in}}%
\pgfpathlineto{\pgfqpoint{3.086184in}{1.681061in}}%
\pgfpathlineto{\pgfqpoint{3.079846in}{1.692528in}}%
\pgfpathlineto{\pgfqpoint{3.073512in}{1.703928in}}%
\pgfpathlineto{\pgfqpoint{3.067182in}{1.715263in}}%
\pgfpathlineto{\pgfqpoint{3.060854in}{1.726532in}}%
\pgfpathlineto{\pgfqpoint{3.054531in}{1.737738in}}%
\pgfpathlineto{\pgfqpoint{3.042999in}{1.733979in}}%
\pgfpathlineto{\pgfqpoint{3.031463in}{1.730192in}}%
\pgfpathlineto{\pgfqpoint{3.019920in}{1.726378in}}%
\pgfpathlineto{\pgfqpoint{3.008372in}{1.722541in}}%
\pgfpathlineto{\pgfqpoint{2.996819in}{1.718681in}}%
\pgfpathlineto{\pgfqpoint{3.003134in}{1.707370in}}%
\pgfpathlineto{\pgfqpoint{3.009454in}{1.696002in}}%
\pgfpathlineto{\pgfqpoint{3.015776in}{1.684581in}}%
\pgfpathlineto{\pgfqpoint{3.022102in}{1.673112in}}%
\pgfpathclose%
\pgfusepath{stroke,fill}%
\end{pgfscope}%
\begin{pgfscope}%
\pgfpathrectangle{\pgfqpoint{0.887500in}{0.275000in}}{\pgfqpoint{4.225000in}{4.225000in}}%
\pgfusepath{clip}%
\pgfsetbuttcap%
\pgfsetroundjoin%
\definecolor{currentfill}{rgb}{0.278791,0.062145,0.386592}%
\pgfsetfillcolor{currentfill}%
\pgfsetfillopacity{0.700000}%
\pgfsetlinewidth{0.501875pt}%
\definecolor{currentstroke}{rgb}{1.000000,1.000000,1.000000}%
\pgfsetstrokecolor{currentstroke}%
\pgfsetstrokeopacity{0.500000}%
\pgfsetdash{}{0pt}%
\pgfpathmoveto{\pgfqpoint{3.833795in}{1.275020in}}%
\pgfpathlineto{\pgfqpoint{3.845226in}{1.282181in}}%
\pgfpathlineto{\pgfqpoint{3.856653in}{1.289357in}}%
\pgfpathlineto{\pgfqpoint{3.868077in}{1.296540in}}%
\pgfpathlineto{\pgfqpoint{3.879498in}{1.303718in}}%
\pgfpathlineto{\pgfqpoint{3.890914in}{1.310882in}}%
\pgfpathlineto{\pgfqpoint{3.884257in}{1.317804in}}%
\pgfpathlineto{\pgfqpoint{3.877633in}{1.326065in}}%
\pgfpathlineto{\pgfqpoint{3.871041in}{1.335588in}}%
\pgfpathlineto{\pgfqpoint{3.864478in}{1.346299in}}%
\pgfpathlineto{\pgfqpoint{3.857943in}{1.358123in}}%
\pgfpathlineto{\pgfqpoint{3.846477in}{1.347903in}}%
\pgfpathlineto{\pgfqpoint{3.835026in}{1.338562in}}%
\pgfpathlineto{\pgfqpoint{3.823592in}{1.330148in}}%
\pgfpathlineto{\pgfqpoint{3.812174in}{1.322702in}}%
\pgfpathlineto{\pgfqpoint{3.800770in}{1.316172in}}%
\pgfpathlineto{\pgfqpoint{3.807303in}{1.304736in}}%
\pgfpathlineto{\pgfqpoint{3.813868in}{1.294767in}}%
\pgfpathlineto{\pgfqpoint{3.820470in}{1.286401in}}%
\pgfpathlineto{\pgfqpoint{3.827111in}{1.279774in}}%
\pgfpathclose%
\pgfusepath{stroke,fill}%
\end{pgfscope}%
\begin{pgfscope}%
\pgfpathrectangle{\pgfqpoint{0.887500in}{0.275000in}}{\pgfqpoint{4.225000in}{4.225000in}}%
\pgfusepath{clip}%
\pgfsetbuttcap%
\pgfsetroundjoin%
\definecolor{currentfill}{rgb}{0.248629,0.278775,0.534556}%
\pgfsetfillcolor{currentfill}%
\pgfsetfillopacity{0.700000}%
\pgfsetlinewidth{0.501875pt}%
\definecolor{currentstroke}{rgb}{1.000000,1.000000,1.000000}%
\pgfsetstrokecolor{currentstroke}%
\pgfsetstrokeopacity{0.500000}%
\pgfsetdash{}{0pt}%
\pgfpathmoveto{\pgfqpoint{3.117921in}{1.622693in}}%
\pgfpathlineto{\pgfqpoint{3.129462in}{1.626505in}}%
\pgfpathlineto{\pgfqpoint{3.140997in}{1.630324in}}%
\pgfpathlineto{\pgfqpoint{3.152527in}{1.634156in}}%
\pgfpathlineto{\pgfqpoint{3.164051in}{1.638013in}}%
\pgfpathlineto{\pgfqpoint{3.175569in}{1.641902in}}%
\pgfpathlineto{\pgfqpoint{3.169208in}{1.653752in}}%
\pgfpathlineto{\pgfqpoint{3.162850in}{1.665507in}}%
\pgfpathlineto{\pgfqpoint{3.156495in}{1.677161in}}%
\pgfpathlineto{\pgfqpoint{3.150143in}{1.688718in}}%
\pgfpathlineto{\pgfqpoint{3.143794in}{1.700186in}}%
\pgfpathlineto{\pgfqpoint{3.132284in}{1.696316in}}%
\pgfpathlineto{\pgfqpoint{3.120767in}{1.692489in}}%
\pgfpathlineto{\pgfqpoint{3.109245in}{1.688685in}}%
\pgfpathlineto{\pgfqpoint{3.097717in}{1.684883in}}%
\pgfpathlineto{\pgfqpoint{3.086184in}{1.681061in}}%
\pgfpathlineto{\pgfqpoint{3.092525in}{1.669527in}}%
\pgfpathlineto{\pgfqpoint{3.098869in}{1.657923in}}%
\pgfpathlineto{\pgfqpoint{3.105216in}{1.646251in}}%
\pgfpathlineto{\pgfqpoint{3.111567in}{1.634508in}}%
\pgfpathclose%
\pgfusepath{stroke,fill}%
\end{pgfscope}%
\begin{pgfscope}%
\pgfpathrectangle{\pgfqpoint{0.887500in}{0.275000in}}{\pgfqpoint{4.225000in}{4.225000in}}%
\pgfusepath{clip}%
\pgfsetbuttcap%
\pgfsetroundjoin%
\definecolor{currentfill}{rgb}{0.179019,0.433756,0.557430}%
\pgfsetfillcolor{currentfill}%
\pgfsetfillopacity{0.700000}%
\pgfsetlinewidth{0.501875pt}%
\definecolor{currentstroke}{rgb}{1.000000,1.000000,1.000000}%
\pgfsetstrokecolor{currentstroke}%
\pgfsetstrokeopacity{0.500000}%
\pgfsetdash{}{0pt}%
\pgfpathmoveto{\pgfqpoint{2.107284in}{1.925297in}}%
\pgfpathlineto{\pgfqpoint{2.119072in}{1.929173in}}%
\pgfpathlineto{\pgfqpoint{2.130854in}{1.933046in}}%
\pgfpathlineto{\pgfqpoint{2.142630in}{1.936914in}}%
\pgfpathlineto{\pgfqpoint{2.154401in}{1.940775in}}%
\pgfpathlineto{\pgfqpoint{2.166167in}{1.944626in}}%
\pgfpathlineto{\pgfqpoint{2.160112in}{1.954457in}}%
\pgfpathlineto{\pgfqpoint{2.154061in}{1.964247in}}%
\pgfpathlineto{\pgfqpoint{2.148016in}{1.973996in}}%
\pgfpathlineto{\pgfqpoint{2.141974in}{1.983706in}}%
\pgfpathlineto{\pgfqpoint{2.135938in}{1.993375in}}%
\pgfpathlineto{\pgfqpoint{2.124182in}{1.989578in}}%
\pgfpathlineto{\pgfqpoint{2.112422in}{1.985770in}}%
\pgfpathlineto{\pgfqpoint{2.100655in}{1.981953in}}%
\pgfpathlineto{\pgfqpoint{2.088883in}{1.978130in}}%
\pgfpathlineto{\pgfqpoint{2.077106in}{1.974303in}}%
\pgfpathlineto{\pgfqpoint{2.083132in}{1.964580in}}%
\pgfpathlineto{\pgfqpoint{2.089163in}{1.954818in}}%
\pgfpathlineto{\pgfqpoint{2.095199in}{1.945017in}}%
\pgfpathlineto{\pgfqpoint{2.101239in}{1.935177in}}%
\pgfpathclose%
\pgfusepath{stroke,fill}%
\end{pgfscope}%
\begin{pgfscope}%
\pgfpathrectangle{\pgfqpoint{0.887500in}{0.275000in}}{\pgfqpoint{4.225000in}{4.225000in}}%
\pgfusepath{clip}%
\pgfsetbuttcap%
\pgfsetroundjoin%
\definecolor{currentfill}{rgb}{0.201239,0.383670,0.554294}%
\pgfsetfillcolor{currentfill}%
\pgfsetfillopacity{0.700000}%
\pgfsetlinewidth{0.501875pt}%
\definecolor{currentstroke}{rgb}{1.000000,1.000000,1.000000}%
\pgfsetstrokecolor{currentstroke}%
\pgfsetstrokeopacity{0.500000}%
\pgfsetdash{}{0pt}%
\pgfpathmoveto{\pgfqpoint{2.523113in}{1.818919in}}%
\pgfpathlineto{\pgfqpoint{2.534799in}{1.822874in}}%
\pgfpathlineto{\pgfqpoint{2.546480in}{1.826817in}}%
\pgfpathlineto{\pgfqpoint{2.558155in}{1.830745in}}%
\pgfpathlineto{\pgfqpoint{2.569825in}{1.834658in}}%
\pgfpathlineto{\pgfqpoint{2.581489in}{1.838554in}}%
\pgfpathlineto{\pgfqpoint{2.575298in}{1.849006in}}%
\pgfpathlineto{\pgfqpoint{2.569111in}{1.859410in}}%
\pgfpathlineto{\pgfqpoint{2.562929in}{1.869767in}}%
\pgfpathlineto{\pgfqpoint{2.556751in}{1.880077in}}%
\pgfpathlineto{\pgfqpoint{2.550577in}{1.890341in}}%
\pgfpathlineto{\pgfqpoint{2.538922in}{1.886501in}}%
\pgfpathlineto{\pgfqpoint{2.527261in}{1.882644in}}%
\pgfpathlineto{\pgfqpoint{2.515596in}{1.878769in}}%
\pgfpathlineto{\pgfqpoint{2.503925in}{1.874878in}}%
\pgfpathlineto{\pgfqpoint{2.492248in}{1.870972in}}%
\pgfpathlineto{\pgfqpoint{2.498413in}{1.860654in}}%
\pgfpathlineto{\pgfqpoint{2.504581in}{1.850290in}}%
\pgfpathlineto{\pgfqpoint{2.510755in}{1.839880in}}%
\pgfpathlineto{\pgfqpoint{2.516932in}{1.829423in}}%
\pgfpathclose%
\pgfusepath{stroke,fill}%
\end{pgfscope}%
\begin{pgfscope}%
\pgfpathrectangle{\pgfqpoint{0.887500in}{0.275000in}}{\pgfqpoint{4.225000in}{4.225000in}}%
\pgfusepath{clip}%
\pgfsetbuttcap%
\pgfsetroundjoin%
\definecolor{currentfill}{rgb}{0.278012,0.180367,0.486697}%
\pgfsetfillcolor{currentfill}%
\pgfsetfillopacity{0.700000}%
\pgfsetlinewidth{0.501875pt}%
\definecolor{currentstroke}{rgb}{1.000000,1.000000,1.000000}%
\pgfsetstrokecolor{currentstroke}%
\pgfsetstrokeopacity{0.500000}%
\pgfsetdash{}{0pt}%
\pgfpathmoveto{\pgfqpoint{3.475856in}{1.447371in}}%
\pgfpathlineto{\pgfqpoint{3.487313in}{1.451619in}}%
\pgfpathlineto{\pgfqpoint{3.498767in}{1.456163in}}%
\pgfpathlineto{\pgfqpoint{3.510220in}{1.461028in}}%
\pgfpathlineto{\pgfqpoint{3.521672in}{1.466231in}}%
\pgfpathlineto{\pgfqpoint{3.533123in}{1.471787in}}%
\pgfpathlineto{\pgfqpoint{3.526680in}{1.484581in}}%
\pgfpathlineto{\pgfqpoint{3.520238in}{1.497280in}}%
\pgfpathlineto{\pgfqpoint{3.513799in}{1.509908in}}%
\pgfpathlineto{\pgfqpoint{3.507362in}{1.522492in}}%
\pgfpathlineto{\pgfqpoint{3.500929in}{1.535058in}}%
\pgfpathlineto{\pgfqpoint{3.489494in}{1.530328in}}%
\pgfpathlineto{\pgfqpoint{3.478054in}{1.525697in}}%
\pgfpathlineto{\pgfqpoint{3.466612in}{1.521223in}}%
\pgfpathlineto{\pgfqpoint{3.455166in}{1.516962in}}%
\pgfpathlineto{\pgfqpoint{3.443718in}{1.512950in}}%
\pgfpathlineto{\pgfqpoint{3.450139in}{1.499812in}}%
\pgfpathlineto{\pgfqpoint{3.456564in}{1.486669in}}%
\pgfpathlineto{\pgfqpoint{3.462991in}{1.473538in}}%
\pgfpathlineto{\pgfqpoint{3.469422in}{1.460435in}}%
\pgfpathclose%
\pgfusepath{stroke,fill}%
\end{pgfscope}%
\begin{pgfscope}%
\pgfpathrectangle{\pgfqpoint{0.887500in}{0.275000in}}{\pgfqpoint{4.225000in}{4.225000in}}%
\pgfusepath{clip}%
\pgfsetbuttcap%
\pgfsetroundjoin%
\definecolor{currentfill}{rgb}{0.257322,0.256130,0.526563}%
\pgfsetfillcolor{currentfill}%
\pgfsetfillopacity{0.700000}%
\pgfsetlinewidth{0.501875pt}%
\definecolor{currentstroke}{rgb}{1.000000,1.000000,1.000000}%
\pgfsetstrokecolor{currentstroke}%
\pgfsetstrokeopacity{0.500000}%
\pgfsetdash{}{0pt}%
\pgfpathmoveto{\pgfqpoint{3.207421in}{1.581621in}}%
\pgfpathlineto{\pgfqpoint{3.218941in}{1.585561in}}%
\pgfpathlineto{\pgfqpoint{3.230455in}{1.589550in}}%
\pgfpathlineto{\pgfqpoint{3.241965in}{1.593592in}}%
\pgfpathlineto{\pgfqpoint{3.253469in}{1.597672in}}%
\pgfpathlineto{\pgfqpoint{3.264967in}{1.601762in}}%
\pgfpathlineto{\pgfqpoint{3.258584in}{1.613907in}}%
\pgfpathlineto{\pgfqpoint{3.252203in}{1.626010in}}%
\pgfpathlineto{\pgfqpoint{3.245826in}{1.638065in}}%
\pgfpathlineto{\pgfqpoint{3.239452in}{1.650066in}}%
\pgfpathlineto{\pgfqpoint{3.233080in}{1.662005in}}%
\pgfpathlineto{\pgfqpoint{3.221589in}{1.657936in}}%
\pgfpathlineto{\pgfqpoint{3.210092in}{1.653860in}}%
\pgfpathlineto{\pgfqpoint{3.198590in}{1.649818in}}%
\pgfpathlineto{\pgfqpoint{3.187082in}{1.645834in}}%
\pgfpathlineto{\pgfqpoint{3.175569in}{1.641902in}}%
\pgfpathlineto{\pgfqpoint{3.181933in}{1.629969in}}%
\pgfpathlineto{\pgfqpoint{3.188301in}{1.617963in}}%
\pgfpathlineto{\pgfqpoint{3.194671in}{1.605896in}}%
\pgfpathlineto{\pgfqpoint{3.201044in}{1.593778in}}%
\pgfpathclose%
\pgfusepath{stroke,fill}%
\end{pgfscope}%
\begin{pgfscope}%
\pgfpathrectangle{\pgfqpoint{0.887500in}{0.275000in}}{\pgfqpoint{4.225000in}{4.225000in}}%
\pgfusepath{clip}%
\pgfsetbuttcap%
\pgfsetroundjoin%
\definecolor{currentfill}{rgb}{0.271828,0.209303,0.504434}%
\pgfsetfillcolor{currentfill}%
\pgfsetfillopacity{0.700000}%
\pgfsetlinewidth{0.501875pt}%
\definecolor{currentstroke}{rgb}{1.000000,1.000000,1.000000}%
\pgfsetstrokecolor{currentstroke}%
\pgfsetstrokeopacity{0.500000}%
\pgfsetdash{}{0pt}%
\pgfpathmoveto{\pgfqpoint{3.386412in}{1.495192in}}%
\pgfpathlineto{\pgfqpoint{3.397883in}{1.498595in}}%
\pgfpathlineto{\pgfqpoint{3.409349in}{1.502025in}}%
\pgfpathlineto{\pgfqpoint{3.420810in}{1.505529in}}%
\pgfpathlineto{\pgfqpoint{3.432266in}{1.509155in}}%
\pgfpathlineto{\pgfqpoint{3.443718in}{1.512950in}}%
\pgfpathlineto{\pgfqpoint{3.437299in}{1.526068in}}%
\pgfpathlineto{\pgfqpoint{3.430883in}{1.539149in}}%
\pgfpathlineto{\pgfqpoint{3.424469in}{1.552177in}}%
\pgfpathlineto{\pgfqpoint{3.418058in}{1.565134in}}%
\pgfpathlineto{\pgfqpoint{3.411649in}{1.578006in}}%
\pgfpathlineto{\pgfqpoint{3.400202in}{1.574180in}}%
\pgfpathlineto{\pgfqpoint{3.388750in}{1.570471in}}%
\pgfpathlineto{\pgfqpoint{3.377293in}{1.566836in}}%
\pgfpathlineto{\pgfqpoint{3.365831in}{1.563234in}}%
\pgfpathlineto{\pgfqpoint{3.354363in}{1.559622in}}%
\pgfpathlineto{\pgfqpoint{3.360768in}{1.546942in}}%
\pgfpathlineto{\pgfqpoint{3.367176in}{1.534157in}}%
\pgfpathlineto{\pgfqpoint{3.373586in}{1.521269in}}%
\pgfpathlineto{\pgfqpoint{3.379998in}{1.508281in}}%
\pgfpathclose%
\pgfusepath{stroke,fill}%
\end{pgfscope}%
\begin{pgfscope}%
\pgfpathrectangle{\pgfqpoint{0.887500in}{0.275000in}}{\pgfqpoint{4.225000in}{4.225000in}}%
\pgfusepath{clip}%
\pgfsetbuttcap%
\pgfsetroundjoin%
\definecolor{currentfill}{rgb}{0.283072,0.130895,0.449241}%
\pgfsetfillcolor{currentfill}%
\pgfsetfillopacity{0.700000}%
\pgfsetlinewidth{0.501875pt}%
\definecolor{currentstroke}{rgb}{1.000000,1.000000,1.000000}%
\pgfsetstrokecolor{currentstroke}%
\pgfsetstrokeopacity{0.500000}%
\pgfsetdash{}{0pt}%
\pgfpathmoveto{\pgfqpoint{3.654766in}{1.357341in}}%
\pgfpathlineto{\pgfqpoint{3.666175in}{1.361363in}}%
\pgfpathlineto{\pgfqpoint{3.677575in}{1.365178in}}%
\pgfpathlineto{\pgfqpoint{3.688966in}{1.368742in}}%
\pgfpathlineto{\pgfqpoint{3.700345in}{1.372012in}}%
\pgfpathlineto{\pgfqpoint{3.711714in}{1.374945in}}%
\pgfpathlineto{\pgfqpoint{3.705315in}{1.392827in}}%
\pgfpathlineto{\pgfqpoint{3.698920in}{1.410861in}}%
\pgfpathlineto{\pgfqpoint{3.692527in}{1.428886in}}%
\pgfpathlineto{\pgfqpoint{3.686132in}{1.446741in}}%
\pgfpathlineto{\pgfqpoint{3.679733in}{1.464266in}}%
\pgfpathlineto{\pgfqpoint{3.668338in}{1.459809in}}%
\pgfpathlineto{\pgfqpoint{3.656925in}{1.454591in}}%
\pgfpathlineto{\pgfqpoint{3.645499in}{1.448788in}}%
\pgfpathlineto{\pgfqpoint{3.634062in}{1.442577in}}%
\pgfpathlineto{\pgfqpoint{3.622618in}{1.436137in}}%
\pgfpathlineto{\pgfqpoint{3.629051in}{1.420877in}}%
\pgfpathlineto{\pgfqpoint{3.635481in}{1.405296in}}%
\pgfpathlineto{\pgfqpoint{3.641910in}{1.389467in}}%
\pgfpathlineto{\pgfqpoint{3.648338in}{1.373458in}}%
\pgfpathclose%
\pgfusepath{stroke,fill}%
\end{pgfscope}%
\begin{pgfscope}%
\pgfpathrectangle{\pgfqpoint{0.887500in}{0.275000in}}{\pgfqpoint{4.225000in}{4.225000in}}%
\pgfusepath{clip}%
\pgfsetbuttcap%
\pgfsetroundjoin%
\definecolor{currentfill}{rgb}{0.265145,0.232956,0.516599}%
\pgfsetfillcolor{currentfill}%
\pgfsetfillopacity{0.700000}%
\pgfsetlinewidth{0.501875pt}%
\definecolor{currentstroke}{rgb}{1.000000,1.000000,1.000000}%
\pgfsetstrokecolor{currentstroke}%
\pgfsetstrokeopacity{0.500000}%
\pgfsetdash{}{0pt}%
\pgfpathmoveto{\pgfqpoint{3.296930in}{1.540605in}}%
\pgfpathlineto{\pgfqpoint{3.308429in}{1.544528in}}%
\pgfpathlineto{\pgfqpoint{3.319922in}{1.548403in}}%
\pgfpathlineto{\pgfqpoint{3.331408in}{1.552216in}}%
\pgfpathlineto{\pgfqpoint{3.342889in}{1.555958in}}%
\pgfpathlineto{\pgfqpoint{3.354363in}{1.559622in}}%
\pgfpathlineto{\pgfqpoint{3.347960in}{1.572197in}}%
\pgfpathlineto{\pgfqpoint{3.341559in}{1.584664in}}%
\pgfpathlineto{\pgfqpoint{3.335161in}{1.597024in}}%
\pgfpathlineto{\pgfqpoint{3.328765in}{1.609280in}}%
\pgfpathlineto{\pgfqpoint{3.322372in}{1.621433in}}%
\pgfpathlineto{\pgfqpoint{3.310903in}{1.617688in}}%
\pgfpathlineto{\pgfqpoint{3.299428in}{1.613826in}}%
\pgfpathlineto{\pgfqpoint{3.287947in}{1.609866in}}%
\pgfpathlineto{\pgfqpoint{3.276460in}{1.605836in}}%
\pgfpathlineto{\pgfqpoint{3.264967in}{1.601762in}}%
\pgfpathlineto{\pgfqpoint{3.271353in}{1.589581in}}%
\pgfpathlineto{\pgfqpoint{3.277743in}{1.577371in}}%
\pgfpathlineto{\pgfqpoint{3.284136in}{1.565138in}}%
\pgfpathlineto{\pgfqpoint{3.290531in}{1.552886in}}%
\pgfpathclose%
\pgfusepath{stroke,fill}%
\end{pgfscope}%
\begin{pgfscope}%
\pgfpathrectangle{\pgfqpoint{0.887500in}{0.275000in}}{\pgfqpoint{4.225000in}{4.225000in}}%
\pgfusepath{clip}%
\pgfsetbuttcap%
\pgfsetroundjoin%
\definecolor{currentfill}{rgb}{0.280868,0.160771,0.472899}%
\pgfsetfillcolor{currentfill}%
\pgfsetfillopacity{0.700000}%
\pgfsetlinewidth{0.501875pt}%
\definecolor{currentstroke}{rgb}{1.000000,1.000000,1.000000}%
\pgfsetstrokecolor{currentstroke}%
\pgfsetstrokeopacity{0.500000}%
\pgfsetdash{}{0pt}%
\pgfpathmoveto{\pgfqpoint{3.565362in}{1.405887in}}%
\pgfpathlineto{\pgfqpoint{3.576815in}{1.411400in}}%
\pgfpathlineto{\pgfqpoint{3.588267in}{1.417191in}}%
\pgfpathlineto{\pgfqpoint{3.599718in}{1.423280in}}%
\pgfpathlineto{\pgfqpoint{3.611169in}{1.429645in}}%
\pgfpathlineto{\pgfqpoint{3.622618in}{1.436137in}}%
\pgfpathlineto{\pgfqpoint{3.616182in}{1.451007in}}%
\pgfpathlineto{\pgfqpoint{3.609742in}{1.465416in}}%
\pgfpathlineto{\pgfqpoint{3.603297in}{1.479294in}}%
\pgfpathlineto{\pgfqpoint{3.596846in}{1.492570in}}%
\pgfpathlineto{\pgfqpoint{3.590389in}{1.505172in}}%
\pgfpathlineto{\pgfqpoint{3.578935in}{1.497813in}}%
\pgfpathlineto{\pgfqpoint{3.567480in}{1.490724in}}%
\pgfpathlineto{\pgfqpoint{3.556027in}{1.484018in}}%
\pgfpathlineto{\pgfqpoint{3.544575in}{1.477711in}}%
\pgfpathlineto{\pgfqpoint{3.533123in}{1.471787in}}%
\pgfpathlineto{\pgfqpoint{3.539568in}{1.458882in}}%
\pgfpathlineto{\pgfqpoint{3.546015in}{1.445853in}}%
\pgfpathlineto{\pgfqpoint{3.552463in}{1.432686in}}%
\pgfpathlineto{\pgfqpoint{3.558912in}{1.419368in}}%
\pgfpathclose%
\pgfusepath{stroke,fill}%
\end{pgfscope}%
\begin{pgfscope}%
\pgfpathrectangle{\pgfqpoint{0.887500in}{0.275000in}}{\pgfqpoint{4.225000in}{4.225000in}}%
\pgfusepath{clip}%
\pgfsetbuttcap%
\pgfsetroundjoin%
\definecolor{currentfill}{rgb}{0.208623,0.367752,0.552675}%
\pgfsetfillcolor{currentfill}%
\pgfsetfillopacity{0.700000}%
\pgfsetlinewidth{0.501875pt}%
\definecolor{currentstroke}{rgb}{1.000000,1.000000,1.000000}%
\pgfsetstrokecolor{currentstroke}%
\pgfsetstrokeopacity{0.500000}%
\pgfsetdash{}{0pt}%
\pgfpathmoveto{\pgfqpoint{2.612505in}{1.785570in}}%
\pgfpathlineto{\pgfqpoint{2.624172in}{1.789503in}}%
\pgfpathlineto{\pgfqpoint{2.635835in}{1.793418in}}%
\pgfpathlineto{\pgfqpoint{2.647491in}{1.797315in}}%
\pgfpathlineto{\pgfqpoint{2.659143in}{1.801191in}}%
\pgfpathlineto{\pgfqpoint{2.670788in}{1.805051in}}%
\pgfpathlineto{\pgfqpoint{2.664568in}{1.815693in}}%
\pgfpathlineto{\pgfqpoint{2.658352in}{1.826284in}}%
\pgfpathlineto{\pgfqpoint{2.652139in}{1.836824in}}%
\pgfpathlineto{\pgfqpoint{2.645931in}{1.847315in}}%
\pgfpathlineto{\pgfqpoint{2.639727in}{1.857758in}}%
\pgfpathlineto{\pgfqpoint{2.628090in}{1.853955in}}%
\pgfpathlineto{\pgfqpoint{2.616448in}{1.850134in}}%
\pgfpathlineto{\pgfqpoint{2.604800in}{1.846294in}}%
\pgfpathlineto{\pgfqpoint{2.593147in}{1.842433in}}%
\pgfpathlineto{\pgfqpoint{2.581489in}{1.838554in}}%
\pgfpathlineto{\pgfqpoint{2.587684in}{1.828055in}}%
\pgfpathlineto{\pgfqpoint{2.593883in}{1.817508in}}%
\pgfpathlineto{\pgfqpoint{2.600086in}{1.806911in}}%
\pgfpathlineto{\pgfqpoint{2.606293in}{1.796266in}}%
\pgfpathclose%
\pgfusepath{stroke,fill}%
\end{pgfscope}%
\begin{pgfscope}%
\pgfpathrectangle{\pgfqpoint{0.887500in}{0.275000in}}{\pgfqpoint{4.225000in}{4.225000in}}%
\pgfusepath{clip}%
\pgfsetbuttcap%
\pgfsetroundjoin%
\definecolor{currentfill}{rgb}{0.185556,0.418570,0.556753}%
\pgfsetfillcolor{currentfill}%
\pgfsetfillopacity{0.700000}%
\pgfsetlinewidth{0.501875pt}%
\definecolor{currentstroke}{rgb}{1.000000,1.000000,1.000000}%
\pgfsetstrokecolor{currentstroke}%
\pgfsetstrokeopacity{0.500000}%
\pgfsetdash{}{0pt}%
\pgfpathmoveto{\pgfqpoint{2.196509in}{1.894858in}}%
\pgfpathlineto{\pgfqpoint{2.208279in}{1.898750in}}%
\pgfpathlineto{\pgfqpoint{2.220043in}{1.902633in}}%
\pgfpathlineto{\pgfqpoint{2.231802in}{1.906507in}}%
\pgfpathlineto{\pgfqpoint{2.243555in}{1.910371in}}%
\pgfpathlineto{\pgfqpoint{2.255303in}{1.914225in}}%
\pgfpathlineto{\pgfqpoint{2.249216in}{1.924207in}}%
\pgfpathlineto{\pgfqpoint{2.243133in}{1.934145in}}%
\pgfpathlineto{\pgfqpoint{2.237054in}{1.944041in}}%
\pgfpathlineto{\pgfqpoint{2.230981in}{1.953896in}}%
\pgfpathlineto{\pgfqpoint{2.224911in}{1.963708in}}%
\pgfpathlineto{\pgfqpoint{2.213173in}{1.959915in}}%
\pgfpathlineto{\pgfqpoint{2.201430in}{1.956110in}}%
\pgfpathlineto{\pgfqpoint{2.189681in}{1.952294in}}%
\pgfpathlineto{\pgfqpoint{2.177927in}{1.948467in}}%
\pgfpathlineto{\pgfqpoint{2.166167in}{1.944626in}}%
\pgfpathlineto{\pgfqpoint{2.172226in}{1.934755in}}%
\pgfpathlineto{\pgfqpoint{2.178290in}{1.924843in}}%
\pgfpathlineto{\pgfqpoint{2.184359in}{1.914889in}}%
\pgfpathlineto{\pgfqpoint{2.190432in}{1.904895in}}%
\pgfpathclose%
\pgfusepath{stroke,fill}%
\end{pgfscope}%
\begin{pgfscope}%
\pgfpathrectangle{\pgfqpoint{0.887500in}{0.275000in}}{\pgfqpoint{4.225000in}{4.225000in}}%
\pgfusepath{clip}%
\pgfsetbuttcap%
\pgfsetroundjoin%
\definecolor{currentfill}{rgb}{0.216210,0.351535,0.550627}%
\pgfsetfillcolor{currentfill}%
\pgfsetfillopacity{0.700000}%
\pgfsetlinewidth{0.501875pt}%
\definecolor{currentstroke}{rgb}{1.000000,1.000000,1.000000}%
\pgfsetstrokecolor{currentstroke}%
\pgfsetstrokeopacity{0.500000}%
\pgfsetdash{}{0pt}%
\pgfpathmoveto{\pgfqpoint{2.701951in}{1.751066in}}%
\pgfpathlineto{\pgfqpoint{2.713600in}{1.754965in}}%
\pgfpathlineto{\pgfqpoint{2.725243in}{1.758856in}}%
\pgfpathlineto{\pgfqpoint{2.736881in}{1.762741in}}%
\pgfpathlineto{\pgfqpoint{2.748513in}{1.766624in}}%
\pgfpathlineto{\pgfqpoint{2.760140in}{1.770507in}}%
\pgfpathlineto{\pgfqpoint{2.753891in}{1.781352in}}%
\pgfpathlineto{\pgfqpoint{2.747645in}{1.792145in}}%
\pgfpathlineto{\pgfqpoint{2.741404in}{1.802885in}}%
\pgfpathlineto{\pgfqpoint{2.735167in}{1.813573in}}%
\pgfpathlineto{\pgfqpoint{2.728933in}{1.824209in}}%
\pgfpathlineto{\pgfqpoint{2.717316in}{1.820385in}}%
\pgfpathlineto{\pgfqpoint{2.705692in}{1.816560in}}%
\pgfpathlineto{\pgfqpoint{2.694063in}{1.812732in}}%
\pgfpathlineto{\pgfqpoint{2.682429in}{1.808897in}}%
\pgfpathlineto{\pgfqpoint{2.670788in}{1.805051in}}%
\pgfpathlineto{\pgfqpoint{2.677013in}{1.794358in}}%
\pgfpathlineto{\pgfqpoint{2.683241in}{1.783613in}}%
\pgfpathlineto{\pgfqpoint{2.689474in}{1.772815in}}%
\pgfpathlineto{\pgfqpoint{2.695710in}{1.761966in}}%
\pgfpathclose%
\pgfusepath{stroke,fill}%
\end{pgfscope}%
\begin{pgfscope}%
\pgfpathrectangle{\pgfqpoint{0.887500in}{0.275000in}}{\pgfqpoint{4.225000in}{4.225000in}}%
\pgfusepath{clip}%
\pgfsetbuttcap%
\pgfsetroundjoin%
\definecolor{currentfill}{rgb}{0.171176,0.452530,0.557965}%
\pgfsetfillcolor{currentfill}%
\pgfsetfillopacity{0.700000}%
\pgfsetlinewidth{0.501875pt}%
\definecolor{currentstroke}{rgb}{1.000000,1.000000,1.000000}%
\pgfsetstrokecolor{currentstroke}%
\pgfsetstrokeopacity{0.500000}%
\pgfsetdash{}{0pt}%
\pgfpathmoveto{\pgfqpoint{1.869843in}{1.965473in}}%
\pgfpathlineto{\pgfqpoint{1.881696in}{1.969319in}}%
\pgfpathlineto{\pgfqpoint{1.893543in}{1.973152in}}%
\pgfpathlineto{\pgfqpoint{1.905384in}{1.976974in}}%
\pgfpathlineto{\pgfqpoint{1.917220in}{1.980785in}}%
\pgfpathlineto{\pgfqpoint{1.929050in}{1.984588in}}%
\pgfpathlineto{\pgfqpoint{1.923072in}{1.994206in}}%
\pgfpathlineto{\pgfqpoint{1.917098in}{2.003789in}}%
\pgfpathlineto{\pgfqpoint{1.911128in}{2.013340in}}%
\pgfpathlineto{\pgfqpoint{1.905163in}{2.022858in}}%
\pgfpathlineto{\pgfqpoint{1.899203in}{2.032344in}}%
\pgfpathlineto{\pgfqpoint{1.887383in}{2.028595in}}%
\pgfpathlineto{\pgfqpoint{1.875557in}{2.024837in}}%
\pgfpathlineto{\pgfqpoint{1.863726in}{2.021068in}}%
\pgfpathlineto{\pgfqpoint{1.851890in}{2.017286in}}%
\pgfpathlineto{\pgfqpoint{1.840048in}{2.013492in}}%
\pgfpathlineto{\pgfqpoint{1.845998in}{2.003952in}}%
\pgfpathlineto{\pgfqpoint{1.851952in}{1.994381in}}%
\pgfpathlineto{\pgfqpoint{1.857911in}{1.984778in}}%
\pgfpathlineto{\pgfqpoint{1.863875in}{1.975142in}}%
\pgfpathclose%
\pgfusepath{stroke,fill}%
\end{pgfscope}%
\begin{pgfscope}%
\pgfpathrectangle{\pgfqpoint{0.887500in}{0.275000in}}{\pgfqpoint{4.225000in}{4.225000in}}%
\pgfusepath{clip}%
\pgfsetbuttcap%
\pgfsetroundjoin%
\definecolor{currentfill}{rgb}{0.190631,0.407061,0.556089}%
\pgfsetfillcolor{currentfill}%
\pgfsetfillopacity{0.700000}%
\pgfsetlinewidth{0.501875pt}%
\definecolor{currentstroke}{rgb}{1.000000,1.000000,1.000000}%
\pgfsetstrokecolor{currentstroke}%
\pgfsetstrokeopacity{0.500000}%
\pgfsetdash{}{0pt}%
\pgfpathmoveto{\pgfqpoint{2.285807in}{1.863673in}}%
\pgfpathlineto{\pgfqpoint{2.297559in}{1.867575in}}%
\pgfpathlineto{\pgfqpoint{2.309305in}{1.871468in}}%
\pgfpathlineto{\pgfqpoint{2.321046in}{1.875353in}}%
\pgfpathlineto{\pgfqpoint{2.332781in}{1.879230in}}%
\pgfpathlineto{\pgfqpoint{2.344510in}{1.883102in}}%
\pgfpathlineto{\pgfqpoint{2.338391in}{1.893245in}}%
\pgfpathlineto{\pgfqpoint{2.332276in}{1.903343in}}%
\pgfpathlineto{\pgfqpoint{2.326165in}{1.913396in}}%
\pgfpathlineto{\pgfqpoint{2.320059in}{1.923405in}}%
\pgfpathlineto{\pgfqpoint{2.313958in}{1.933369in}}%
\pgfpathlineto{\pgfqpoint{2.302238in}{1.929554in}}%
\pgfpathlineto{\pgfqpoint{2.290513in}{1.925735in}}%
\pgfpathlineto{\pgfqpoint{2.278782in}{1.921907in}}%
\pgfpathlineto{\pgfqpoint{2.267045in}{1.918071in}}%
\pgfpathlineto{\pgfqpoint{2.255303in}{1.914225in}}%
\pgfpathlineto{\pgfqpoint{2.261395in}{1.904201in}}%
\pgfpathlineto{\pgfqpoint{2.267491in}{1.894134in}}%
\pgfpathlineto{\pgfqpoint{2.273592in}{1.884024in}}%
\pgfpathlineto{\pgfqpoint{2.279697in}{1.873870in}}%
\pgfpathclose%
\pgfusepath{stroke,fill}%
\end{pgfscope}%
\begin{pgfscope}%
\pgfpathrectangle{\pgfqpoint{0.887500in}{0.275000in}}{\pgfqpoint{4.225000in}{4.225000in}}%
\pgfusepath{clip}%
\pgfsetbuttcap%
\pgfsetroundjoin%
\definecolor{currentfill}{rgb}{0.277018,0.050344,0.375715}%
\pgfsetfillcolor{currentfill}%
\pgfsetfillopacity{0.700000}%
\pgfsetlinewidth{0.501875pt}%
\definecolor{currentstroke}{rgb}{1.000000,1.000000,1.000000}%
\pgfsetstrokecolor{currentstroke}%
\pgfsetstrokeopacity{0.500000}%
\pgfsetdash{}{0pt}%
\pgfpathmoveto{\pgfqpoint{3.776594in}{1.239628in}}%
\pgfpathlineto{\pgfqpoint{3.788040in}{1.246634in}}%
\pgfpathlineto{\pgfqpoint{3.799484in}{1.253681in}}%
\pgfpathlineto{\pgfqpoint{3.810924in}{1.260765in}}%
\pgfpathlineto{\pgfqpoint{3.822361in}{1.267880in}}%
\pgfpathlineto{\pgfqpoint{3.833795in}{1.275020in}}%
\pgfpathlineto{\pgfqpoint{3.827111in}{1.279774in}}%
\pgfpathlineto{\pgfqpoint{3.820470in}{1.286401in}}%
\pgfpathlineto{\pgfqpoint{3.813868in}{1.294767in}}%
\pgfpathlineto{\pgfqpoint{3.807303in}{1.304736in}}%
\pgfpathlineto{\pgfqpoint{3.800770in}{1.316172in}}%
\pgfpathlineto{\pgfqpoint{3.789378in}{1.310451in}}%
\pgfpathlineto{\pgfqpoint{3.777994in}{1.305424in}}%
\pgfpathlineto{\pgfqpoint{3.766617in}{1.300982in}}%
\pgfpathlineto{\pgfqpoint{3.755242in}{1.297014in}}%
\pgfpathlineto{\pgfqpoint{3.743869in}{1.293406in}}%
\pgfpathlineto{\pgfqpoint{3.750354in}{1.279791in}}%
\pgfpathlineto{\pgfqpoint{3.756866in}{1.267445in}}%
\pgfpathlineto{\pgfqpoint{3.763407in}{1.256529in}}%
\pgfpathlineto{\pgfqpoint{3.769982in}{1.247204in}}%
\pgfpathclose%
\pgfusepath{stroke,fill}%
\end{pgfscope}%
\begin{pgfscope}%
\pgfpathrectangle{\pgfqpoint{0.887500in}{0.275000in}}{\pgfqpoint{4.225000in}{4.225000in}}%
\pgfusepath{clip}%
\pgfsetbuttcap%
\pgfsetroundjoin%
\definecolor{currentfill}{rgb}{0.223925,0.334994,0.548053}%
\pgfsetfillcolor{currentfill}%
\pgfsetfillopacity{0.700000}%
\pgfsetlinewidth{0.501875pt}%
\definecolor{currentstroke}{rgb}{1.000000,1.000000,1.000000}%
\pgfsetstrokecolor{currentstroke}%
\pgfsetstrokeopacity{0.500000}%
\pgfsetdash{}{0pt}%
\pgfpathmoveto{\pgfqpoint{2.791444in}{1.715527in}}%
\pgfpathlineto{\pgfqpoint{2.803073in}{1.719469in}}%
\pgfpathlineto{\pgfqpoint{2.814697in}{1.723416in}}%
\pgfpathlineto{\pgfqpoint{2.826315in}{1.727367in}}%
\pgfpathlineto{\pgfqpoint{2.837927in}{1.731321in}}%
\pgfpathlineto{\pgfqpoint{2.849534in}{1.735278in}}%
\pgfpathlineto{\pgfqpoint{2.843257in}{1.746324in}}%
\pgfpathlineto{\pgfqpoint{2.836984in}{1.757319in}}%
\pgfpathlineto{\pgfqpoint{2.830714in}{1.768261in}}%
\pgfpathlineto{\pgfqpoint{2.824448in}{1.779152in}}%
\pgfpathlineto{\pgfqpoint{2.818187in}{1.789990in}}%
\pgfpathlineto{\pgfqpoint{2.806589in}{1.786083in}}%
\pgfpathlineto{\pgfqpoint{2.794985in}{1.782182in}}%
\pgfpathlineto{\pgfqpoint{2.783376in}{1.778285in}}%
\pgfpathlineto{\pgfqpoint{2.771761in}{1.774394in}}%
\pgfpathlineto{\pgfqpoint{2.760140in}{1.770507in}}%
\pgfpathlineto{\pgfqpoint{2.766393in}{1.759611in}}%
\pgfpathlineto{\pgfqpoint{2.772650in}{1.748664in}}%
\pgfpathlineto{\pgfqpoint{2.778911in}{1.737668in}}%
\pgfpathlineto{\pgfqpoint{2.785175in}{1.726622in}}%
\pgfpathclose%
\pgfusepath{stroke,fill}%
\end{pgfscope}%
\begin{pgfscope}%
\pgfpathrectangle{\pgfqpoint{0.887500in}{0.275000in}}{\pgfqpoint{4.225000in}{4.225000in}}%
\pgfusepath{clip}%
\pgfsetbuttcap%
\pgfsetroundjoin%
\definecolor{currentfill}{rgb}{0.281446,0.084320,0.407414}%
\pgfsetfillcolor{currentfill}%
\pgfsetfillopacity{0.700000}%
\pgfsetlinewidth{0.501875pt}%
\definecolor{currentstroke}{rgb}{1.000000,1.000000,1.000000}%
\pgfsetstrokecolor{currentstroke}%
\pgfsetstrokeopacity{0.500000}%
\pgfsetdash{}{0pt}%
\pgfpathmoveto{\pgfqpoint{3.686952in}{1.277584in}}%
\pgfpathlineto{\pgfqpoint{3.698346in}{1.280707in}}%
\pgfpathlineto{\pgfqpoint{3.709734in}{1.283776in}}%
\pgfpathlineto{\pgfqpoint{3.721116in}{1.286867in}}%
\pgfpathlineto{\pgfqpoint{3.732494in}{1.290052in}}%
\pgfpathlineto{\pgfqpoint{3.743869in}{1.293406in}}%
\pgfpathlineto{\pgfqpoint{3.737406in}{1.308132in}}%
\pgfpathlineto{\pgfqpoint{3.730962in}{1.323808in}}%
\pgfpathlineto{\pgfqpoint{3.724534in}{1.340276in}}%
\pgfpathlineto{\pgfqpoint{3.718119in}{1.357375in}}%
\pgfpathlineto{\pgfqpoint{3.711714in}{1.374945in}}%
\pgfpathlineto{\pgfqpoint{3.700345in}{1.372012in}}%
\pgfpathlineto{\pgfqpoint{3.688966in}{1.368742in}}%
\pgfpathlineto{\pgfqpoint{3.677575in}{1.365178in}}%
\pgfpathlineto{\pgfqpoint{3.666175in}{1.361363in}}%
\pgfpathlineto{\pgfqpoint{3.654766in}{1.357341in}}%
\pgfpathlineto{\pgfqpoint{3.661196in}{1.341186in}}%
\pgfpathlineto{\pgfqpoint{3.667628in}{1.325062in}}%
\pgfpathlineto{\pgfqpoint{3.674064in}{1.309041in}}%
\pgfpathlineto{\pgfqpoint{3.680505in}{1.293191in}}%
\pgfpathclose%
\pgfusepath{stroke,fill}%
\end{pgfscope}%
\begin{pgfscope}%
\pgfpathrectangle{\pgfqpoint{0.887500in}{0.275000in}}{\pgfqpoint{4.225000in}{4.225000in}}%
\pgfusepath{clip}%
\pgfsetbuttcap%
\pgfsetroundjoin%
\definecolor{currentfill}{rgb}{0.175841,0.441290,0.557685}%
\pgfsetfillcolor{currentfill}%
\pgfsetfillopacity{0.700000}%
\pgfsetlinewidth{0.501875pt}%
\definecolor{currentstroke}{rgb}{1.000000,1.000000,1.000000}%
\pgfsetstrokecolor{currentstroke}%
\pgfsetstrokeopacity{0.500000}%
\pgfsetdash{}{0pt}%
\pgfpathmoveto{\pgfqpoint{1.959012in}{1.935987in}}%
\pgfpathlineto{\pgfqpoint{1.970847in}{1.939833in}}%
\pgfpathlineto{\pgfqpoint{1.982677in}{1.943674in}}%
\pgfpathlineto{\pgfqpoint{1.994500in}{1.947510in}}%
\pgfpathlineto{\pgfqpoint{2.006318in}{1.951341in}}%
\pgfpathlineto{\pgfqpoint{2.018131in}{1.955170in}}%
\pgfpathlineto{\pgfqpoint{2.012119in}{1.964909in}}%
\pgfpathlineto{\pgfqpoint{2.006112in}{1.974611in}}%
\pgfpathlineto{\pgfqpoint{2.000109in}{1.984278in}}%
\pgfpathlineto{\pgfqpoint{1.994111in}{1.993909in}}%
\pgfpathlineto{\pgfqpoint{1.988118in}{2.003506in}}%
\pgfpathlineto{\pgfqpoint{1.976316in}{1.999731in}}%
\pgfpathlineto{\pgfqpoint{1.964508in}{1.995954in}}%
\pgfpathlineto{\pgfqpoint{1.952694in}{1.992171in}}%
\pgfpathlineto{\pgfqpoint{1.940875in}{1.988383in}}%
\pgfpathlineto{\pgfqpoint{1.929050in}{1.984588in}}%
\pgfpathlineto{\pgfqpoint{1.935034in}{1.974937in}}%
\pgfpathlineto{\pgfqpoint{1.941021in}{1.965252in}}%
\pgfpathlineto{\pgfqpoint{1.947014in}{1.955532in}}%
\pgfpathlineto{\pgfqpoint{1.953011in}{1.945777in}}%
\pgfpathclose%
\pgfusepath{stroke,fill}%
\end{pgfscope}%
\begin{pgfscope}%
\pgfpathrectangle{\pgfqpoint{0.887500in}{0.275000in}}{\pgfqpoint{4.225000in}{4.225000in}}%
\pgfusepath{clip}%
\pgfsetbuttcap%
\pgfsetroundjoin%
\definecolor{currentfill}{rgb}{0.231674,0.318106,0.544834}%
\pgfsetfillcolor{currentfill}%
\pgfsetfillopacity{0.700000}%
\pgfsetlinewidth{0.501875pt}%
\definecolor{currentstroke}{rgb}{1.000000,1.000000,1.000000}%
\pgfsetstrokecolor{currentstroke}%
\pgfsetstrokeopacity{0.500000}%
\pgfsetdash{}{0pt}%
\pgfpathmoveto{\pgfqpoint{2.880975in}{1.679267in}}%
\pgfpathlineto{\pgfqpoint{2.892584in}{1.683252in}}%
\pgfpathlineto{\pgfqpoint{2.904188in}{1.687231in}}%
\pgfpathlineto{\pgfqpoint{2.915787in}{1.691203in}}%
\pgfpathlineto{\pgfqpoint{2.927379in}{1.695167in}}%
\pgfpathlineto{\pgfqpoint{2.938967in}{1.699120in}}%
\pgfpathlineto{\pgfqpoint{2.932663in}{1.710422in}}%
\pgfpathlineto{\pgfqpoint{2.926362in}{1.721666in}}%
\pgfpathlineto{\pgfqpoint{2.920065in}{1.732852in}}%
\pgfpathlineto{\pgfqpoint{2.913772in}{1.743980in}}%
\pgfpathlineto{\pgfqpoint{2.907483in}{1.755050in}}%
\pgfpathlineto{\pgfqpoint{2.895904in}{1.751112in}}%
\pgfpathlineto{\pgfqpoint{2.884320in}{1.747159in}}%
\pgfpathlineto{\pgfqpoint{2.872730in}{1.743199in}}%
\pgfpathlineto{\pgfqpoint{2.861135in}{1.739238in}}%
\pgfpathlineto{\pgfqpoint{2.849534in}{1.735278in}}%
\pgfpathlineto{\pgfqpoint{2.855815in}{1.724180in}}%
\pgfpathlineto{\pgfqpoint{2.862099in}{1.713030in}}%
\pgfpathlineto{\pgfqpoint{2.868387in}{1.701829in}}%
\pgfpathlineto{\pgfqpoint{2.874679in}{1.690575in}}%
\pgfpathclose%
\pgfusepath{stroke,fill}%
\end{pgfscope}%
\begin{pgfscope}%
\pgfpathrectangle{\pgfqpoint{0.887500in}{0.275000in}}{\pgfqpoint{4.225000in}{4.225000in}}%
\pgfusepath{clip}%
\pgfsetbuttcap%
\pgfsetroundjoin%
\definecolor{currentfill}{rgb}{0.195860,0.395433,0.555276}%
\pgfsetfillcolor{currentfill}%
\pgfsetfillopacity{0.700000}%
\pgfsetlinewidth{0.501875pt}%
\definecolor{currentstroke}{rgb}{1.000000,1.000000,1.000000}%
\pgfsetstrokecolor{currentstroke}%
\pgfsetstrokeopacity{0.500000}%
\pgfsetdash{}{0pt}%
\pgfpathmoveto{\pgfqpoint{2.375172in}{1.831718in}}%
\pgfpathlineto{\pgfqpoint{2.386906in}{1.835637in}}%
\pgfpathlineto{\pgfqpoint{2.398633in}{1.839556in}}%
\pgfpathlineto{\pgfqpoint{2.410355in}{1.843475in}}%
\pgfpathlineto{\pgfqpoint{2.422071in}{1.847399in}}%
\pgfpathlineto{\pgfqpoint{2.433782in}{1.851328in}}%
\pgfpathlineto{\pgfqpoint{2.427631in}{1.861646in}}%
\pgfpathlineto{\pgfqpoint{2.421484in}{1.871919in}}%
\pgfpathlineto{\pgfqpoint{2.415342in}{1.882146in}}%
\pgfpathlineto{\pgfqpoint{2.409205in}{1.892328in}}%
\pgfpathlineto{\pgfqpoint{2.403071in}{1.902463in}}%
\pgfpathlineto{\pgfqpoint{2.391371in}{1.898583in}}%
\pgfpathlineto{\pgfqpoint{2.379664in}{1.894709in}}%
\pgfpathlineto{\pgfqpoint{2.367952in}{1.890839in}}%
\pgfpathlineto{\pgfqpoint{2.356234in}{1.886971in}}%
\pgfpathlineto{\pgfqpoint{2.344510in}{1.883102in}}%
\pgfpathlineto{\pgfqpoint{2.350634in}{1.872914in}}%
\pgfpathlineto{\pgfqpoint{2.356762in}{1.862682in}}%
\pgfpathlineto{\pgfqpoint{2.362894in}{1.852406in}}%
\pgfpathlineto{\pgfqpoint{2.369031in}{1.842084in}}%
\pgfpathclose%
\pgfusepath{stroke,fill}%
\end{pgfscope}%
\begin{pgfscope}%
\pgfpathrectangle{\pgfqpoint{0.887500in}{0.275000in}}{\pgfqpoint{4.225000in}{4.225000in}}%
\pgfusepath{clip}%
\pgfsetbuttcap%
\pgfsetroundjoin%
\definecolor{currentfill}{rgb}{0.241237,0.296485,0.539709}%
\pgfsetfillcolor{currentfill}%
\pgfsetfillopacity{0.700000}%
\pgfsetlinewidth{0.501875pt}%
\definecolor{currentstroke}{rgb}{1.000000,1.000000,1.000000}%
\pgfsetstrokecolor{currentstroke}%
\pgfsetstrokeopacity{0.500000}%
\pgfsetdash{}{0pt}%
\pgfpathmoveto{\pgfqpoint{2.970540in}{1.641813in}}%
\pgfpathlineto{\pgfqpoint{2.982130in}{1.645784in}}%
\pgfpathlineto{\pgfqpoint{2.993714in}{1.649748in}}%
\pgfpathlineto{\pgfqpoint{3.005292in}{1.653705in}}%
\pgfpathlineto{\pgfqpoint{3.016865in}{1.657655in}}%
\pgfpathlineto{\pgfqpoint{3.028432in}{1.661596in}}%
\pgfpathlineto{\pgfqpoint{3.022102in}{1.673112in}}%
\pgfpathlineto{\pgfqpoint{3.015776in}{1.684581in}}%
\pgfpathlineto{\pgfqpoint{3.009454in}{1.696002in}}%
\pgfpathlineto{\pgfqpoint{3.003134in}{1.707370in}}%
\pgfpathlineto{\pgfqpoint{2.996819in}{1.718681in}}%
\pgfpathlineto{\pgfqpoint{2.985259in}{1.714802in}}%
\pgfpathlineto{\pgfqpoint{2.973695in}{1.710904in}}%
\pgfpathlineto{\pgfqpoint{2.962124in}{1.706991in}}%
\pgfpathlineto{\pgfqpoint{2.950548in}{1.703062in}}%
\pgfpathlineto{\pgfqpoint{2.938967in}{1.699120in}}%
\pgfpathlineto{\pgfqpoint{2.945274in}{1.687763in}}%
\pgfpathlineto{\pgfqpoint{2.951585in}{1.676353in}}%
\pgfpathlineto{\pgfqpoint{2.957900in}{1.664890in}}%
\pgfpathlineto{\pgfqpoint{2.964218in}{1.653376in}}%
\pgfpathclose%
\pgfusepath{stroke,fill}%
\end{pgfscope}%
\begin{pgfscope}%
\pgfpathrectangle{\pgfqpoint{0.887500in}{0.275000in}}{\pgfqpoint{4.225000in}{4.225000in}}%
\pgfusepath{clip}%
\pgfsetbuttcap%
\pgfsetroundjoin%
\definecolor{currentfill}{rgb}{0.281887,0.150881,0.465405}%
\pgfsetfillcolor{currentfill}%
\pgfsetfillopacity{0.700000}%
\pgfsetlinewidth{0.501875pt}%
\definecolor{currentstroke}{rgb}{1.000000,1.000000,1.000000}%
\pgfsetstrokecolor{currentstroke}%
\pgfsetstrokeopacity{0.500000}%
\pgfsetdash{}{0pt}%
\pgfpathmoveto{\pgfqpoint{3.508070in}{1.381750in}}%
\pgfpathlineto{\pgfqpoint{3.519534in}{1.386196in}}%
\pgfpathlineto{\pgfqpoint{3.530995in}{1.390804in}}%
\pgfpathlineto{\pgfqpoint{3.542453in}{1.395609in}}%
\pgfpathlineto{\pgfqpoint{3.553908in}{1.400629in}}%
\pgfpathlineto{\pgfqpoint{3.565362in}{1.405887in}}%
\pgfpathlineto{\pgfqpoint{3.558912in}{1.419368in}}%
\pgfpathlineto{\pgfqpoint{3.552463in}{1.432686in}}%
\pgfpathlineto{\pgfqpoint{3.546015in}{1.445853in}}%
\pgfpathlineto{\pgfqpoint{3.539568in}{1.458882in}}%
\pgfpathlineto{\pgfqpoint{3.533123in}{1.471787in}}%
\pgfpathlineto{\pgfqpoint{3.521672in}{1.466231in}}%
\pgfpathlineto{\pgfqpoint{3.510220in}{1.461028in}}%
\pgfpathlineto{\pgfqpoint{3.498767in}{1.456163in}}%
\pgfpathlineto{\pgfqpoint{3.487313in}{1.451619in}}%
\pgfpathlineto{\pgfqpoint{3.475856in}{1.447371in}}%
\pgfpathlineto{\pgfqpoint{3.482294in}{1.434325in}}%
\pgfpathlineto{\pgfqpoint{3.488734in}{1.421269in}}%
\pgfpathlineto{\pgfqpoint{3.495178in}{1.408174in}}%
\pgfpathlineto{\pgfqpoint{3.501623in}{1.395010in}}%
\pgfpathclose%
\pgfusepath{stroke,fill}%
\end{pgfscope}%
\begin{pgfscope}%
\pgfpathrectangle{\pgfqpoint{0.887500in}{0.275000in}}{\pgfqpoint{4.225000in}{4.225000in}}%
\pgfusepath{clip}%
\pgfsetbuttcap%
\pgfsetroundjoin%
\definecolor{currentfill}{rgb}{0.283187,0.125848,0.444960}%
\pgfsetfillcolor{currentfill}%
\pgfsetfillopacity{0.700000}%
\pgfsetlinewidth{0.501875pt}%
\definecolor{currentstroke}{rgb}{1.000000,1.000000,1.000000}%
\pgfsetstrokecolor{currentstroke}%
\pgfsetstrokeopacity{0.500000}%
\pgfsetdash{}{0pt}%
\pgfpathmoveto{\pgfqpoint{3.597615in}{1.335548in}}%
\pgfpathlineto{\pgfqpoint{3.609057in}{1.340031in}}%
\pgfpathlineto{\pgfqpoint{3.620494in}{1.344471in}}%
\pgfpathlineto{\pgfqpoint{3.631925in}{1.348853in}}%
\pgfpathlineto{\pgfqpoint{3.643349in}{1.353157in}}%
\pgfpathlineto{\pgfqpoint{3.654766in}{1.357341in}}%
\pgfpathlineto{\pgfqpoint{3.648338in}{1.373458in}}%
\pgfpathlineto{\pgfqpoint{3.641910in}{1.389467in}}%
\pgfpathlineto{\pgfqpoint{3.635481in}{1.405296in}}%
\pgfpathlineto{\pgfqpoint{3.629051in}{1.420877in}}%
\pgfpathlineto{\pgfqpoint{3.622618in}{1.436137in}}%
\pgfpathlineto{\pgfqpoint{3.611169in}{1.429645in}}%
\pgfpathlineto{\pgfqpoint{3.599718in}{1.423280in}}%
\pgfpathlineto{\pgfqpoint{3.588267in}{1.417191in}}%
\pgfpathlineto{\pgfqpoint{3.576815in}{1.411400in}}%
\pgfpathlineto{\pgfqpoint{3.565362in}{1.405887in}}%
\pgfpathlineto{\pgfqpoint{3.571813in}{1.392227in}}%
\pgfpathlineto{\pgfqpoint{3.578264in}{1.378377in}}%
\pgfpathlineto{\pgfqpoint{3.584714in}{1.364322in}}%
\pgfpathlineto{\pgfqpoint{3.591165in}{1.350050in}}%
\pgfpathclose%
\pgfusepath{stroke,fill}%
\end{pgfscope}%
\begin{pgfscope}%
\pgfpathrectangle{\pgfqpoint{0.887500in}{0.275000in}}{\pgfqpoint{4.225000in}{4.225000in}}%
\pgfusepath{clip}%
\pgfsetbuttcap%
\pgfsetroundjoin%
\definecolor{currentfill}{rgb}{0.248629,0.278775,0.534556}%
\pgfsetfillcolor{currentfill}%
\pgfsetfillopacity{0.700000}%
\pgfsetlinewidth{0.501875pt}%
\definecolor{currentstroke}{rgb}{1.000000,1.000000,1.000000}%
\pgfsetstrokecolor{currentstroke}%
\pgfsetstrokeopacity{0.500000}%
\pgfsetdash{}{0pt}%
\pgfpathmoveto{\pgfqpoint{3.060131in}{1.603433in}}%
\pgfpathlineto{\pgfqpoint{3.071701in}{1.607336in}}%
\pgfpathlineto{\pgfqpoint{3.083264in}{1.611206in}}%
\pgfpathlineto{\pgfqpoint{3.094822in}{1.615050in}}%
\pgfpathlineto{\pgfqpoint{3.106374in}{1.618877in}}%
\pgfpathlineto{\pgfqpoint{3.117921in}{1.622693in}}%
\pgfpathlineto{\pgfqpoint{3.111567in}{1.634508in}}%
\pgfpathlineto{\pgfqpoint{3.105216in}{1.646251in}}%
\pgfpathlineto{\pgfqpoint{3.098869in}{1.657923in}}%
\pgfpathlineto{\pgfqpoint{3.092525in}{1.669527in}}%
\pgfpathlineto{\pgfqpoint{3.086184in}{1.681061in}}%
\pgfpathlineto{\pgfqpoint{3.074645in}{1.677212in}}%
\pgfpathlineto{\pgfqpoint{3.063100in}{1.673337in}}%
\pgfpathlineto{\pgfqpoint{3.051550in}{1.669441in}}%
\pgfpathlineto{\pgfqpoint{3.039994in}{1.665526in}}%
\pgfpathlineto{\pgfqpoint{3.028432in}{1.661596in}}%
\pgfpathlineto{\pgfqpoint{3.034765in}{1.650038in}}%
\pgfpathlineto{\pgfqpoint{3.041102in}{1.638442in}}%
\pgfpathlineto{\pgfqpoint{3.047441in}{1.626811in}}%
\pgfpathlineto{\pgfqpoint{3.053785in}{1.615146in}}%
\pgfpathclose%
\pgfusepath{stroke,fill}%
\end{pgfscope}%
\begin{pgfscope}%
\pgfpathrectangle{\pgfqpoint{0.887500in}{0.275000in}}{\pgfqpoint{4.225000in}{4.225000in}}%
\pgfusepath{clip}%
\pgfsetbuttcap%
\pgfsetroundjoin%
\definecolor{currentfill}{rgb}{0.180629,0.429975,0.557282}%
\pgfsetfillcolor{currentfill}%
\pgfsetfillopacity{0.700000}%
\pgfsetlinewidth{0.501875pt}%
\definecolor{currentstroke}{rgb}{1.000000,1.000000,1.000000}%
\pgfsetstrokecolor{currentstroke}%
\pgfsetstrokeopacity{0.500000}%
\pgfsetdash{}{0pt}%
\pgfpathmoveto{\pgfqpoint{2.048258in}{1.905918in}}%
\pgfpathlineto{\pgfqpoint{2.060075in}{1.909794in}}%
\pgfpathlineto{\pgfqpoint{2.071886in}{1.913669in}}%
\pgfpathlineto{\pgfqpoint{2.083691in}{1.917544in}}%
\pgfpathlineto{\pgfqpoint{2.095490in}{1.921420in}}%
\pgfpathlineto{\pgfqpoint{2.107284in}{1.925297in}}%
\pgfpathlineto{\pgfqpoint{2.101239in}{1.935177in}}%
\pgfpathlineto{\pgfqpoint{2.095199in}{1.945017in}}%
\pgfpathlineto{\pgfqpoint{2.089163in}{1.954818in}}%
\pgfpathlineto{\pgfqpoint{2.083132in}{1.964580in}}%
\pgfpathlineto{\pgfqpoint{2.077106in}{1.974303in}}%
\pgfpathlineto{\pgfqpoint{2.065322in}{1.970475in}}%
\pgfpathlineto{\pgfqpoint{2.053533in}{1.966648in}}%
\pgfpathlineto{\pgfqpoint{2.041738in}{1.962822in}}%
\pgfpathlineto{\pgfqpoint{2.029937in}{1.958997in}}%
\pgfpathlineto{\pgfqpoint{2.018131in}{1.955170in}}%
\pgfpathlineto{\pgfqpoint{2.024147in}{1.945394in}}%
\pgfpathlineto{\pgfqpoint{2.030168in}{1.935582in}}%
\pgfpathlineto{\pgfqpoint{2.036193in}{1.925732in}}%
\pgfpathlineto{\pgfqpoint{2.042223in}{1.915844in}}%
\pgfpathclose%
\pgfusepath{stroke,fill}%
\end{pgfscope}%
\begin{pgfscope}%
\pgfpathrectangle{\pgfqpoint{0.887500in}{0.275000in}}{\pgfqpoint{4.225000in}{4.225000in}}%
\pgfusepath{clip}%
\pgfsetbuttcap%
\pgfsetroundjoin%
\definecolor{currentfill}{rgb}{0.278012,0.180367,0.486697}%
\pgfsetfillcolor{currentfill}%
\pgfsetfillopacity{0.700000}%
\pgfsetlinewidth{0.501875pt}%
\definecolor{currentstroke}{rgb}{1.000000,1.000000,1.000000}%
\pgfsetstrokecolor{currentstroke}%
\pgfsetstrokeopacity{0.500000}%
\pgfsetdash{}{0pt}%
\pgfpathmoveto{\pgfqpoint{3.418515in}{1.428306in}}%
\pgfpathlineto{\pgfqpoint{3.429994in}{1.432055in}}%
\pgfpathlineto{\pgfqpoint{3.441467in}{1.435769in}}%
\pgfpathlineto{\pgfqpoint{3.452934in}{1.439514in}}%
\pgfpathlineto{\pgfqpoint{3.464397in}{1.443359in}}%
\pgfpathlineto{\pgfqpoint{3.475856in}{1.447371in}}%
\pgfpathlineto{\pgfqpoint{3.469422in}{1.460435in}}%
\pgfpathlineto{\pgfqpoint{3.462991in}{1.473538in}}%
\pgfpathlineto{\pgfqpoint{3.456564in}{1.486669in}}%
\pgfpathlineto{\pgfqpoint{3.450139in}{1.499812in}}%
\pgfpathlineto{\pgfqpoint{3.443718in}{1.512950in}}%
\pgfpathlineto{\pgfqpoint{3.432266in}{1.509155in}}%
\pgfpathlineto{\pgfqpoint{3.420810in}{1.505529in}}%
\pgfpathlineto{\pgfqpoint{3.409349in}{1.502025in}}%
\pgfpathlineto{\pgfqpoint{3.397883in}{1.498595in}}%
\pgfpathlineto{\pgfqpoint{3.386412in}{1.495192in}}%
\pgfpathlineto{\pgfqpoint{3.392828in}{1.482005in}}%
\pgfpathlineto{\pgfqpoint{3.399247in}{1.468722in}}%
\pgfpathlineto{\pgfqpoint{3.405667in}{1.455343in}}%
\pgfpathlineto{\pgfqpoint{3.412090in}{1.441871in}}%
\pgfpathclose%
\pgfusepath{stroke,fill}%
\end{pgfscope}%
\begin{pgfscope}%
\pgfpathrectangle{\pgfqpoint{0.887500in}{0.275000in}}{\pgfqpoint{4.225000in}{4.225000in}}%
\pgfusepath{clip}%
\pgfsetbuttcap%
\pgfsetroundjoin%
\definecolor{currentfill}{rgb}{0.203063,0.379716,0.553925}%
\pgfsetfillcolor{currentfill}%
\pgfsetfillopacity{0.700000}%
\pgfsetlinewidth{0.501875pt}%
\definecolor{currentstroke}{rgb}{1.000000,1.000000,1.000000}%
\pgfsetstrokecolor{currentstroke}%
\pgfsetstrokeopacity{0.500000}%
\pgfsetdash{}{0pt}%
\pgfpathmoveto{\pgfqpoint{2.464600in}{1.799050in}}%
\pgfpathlineto{\pgfqpoint{2.476314in}{1.803027in}}%
\pgfpathlineto{\pgfqpoint{2.488022in}{1.807005in}}%
\pgfpathlineto{\pgfqpoint{2.499725in}{1.810982in}}%
\pgfpathlineto{\pgfqpoint{2.511422in}{1.814954in}}%
\pgfpathlineto{\pgfqpoint{2.523113in}{1.818919in}}%
\pgfpathlineto{\pgfqpoint{2.516932in}{1.829423in}}%
\pgfpathlineto{\pgfqpoint{2.510755in}{1.839880in}}%
\pgfpathlineto{\pgfqpoint{2.504581in}{1.850290in}}%
\pgfpathlineto{\pgfqpoint{2.498413in}{1.860654in}}%
\pgfpathlineto{\pgfqpoint{2.492248in}{1.870972in}}%
\pgfpathlineto{\pgfqpoint{2.480566in}{1.867054in}}%
\pgfpathlineto{\pgfqpoint{2.468878in}{1.863127in}}%
\pgfpathlineto{\pgfqpoint{2.457185in}{1.859195in}}%
\pgfpathlineto{\pgfqpoint{2.445486in}{1.855261in}}%
\pgfpathlineto{\pgfqpoint{2.433782in}{1.851328in}}%
\pgfpathlineto{\pgfqpoint{2.439937in}{1.840963in}}%
\pgfpathlineto{\pgfqpoint{2.446096in}{1.830553in}}%
\pgfpathlineto{\pgfqpoint{2.452260in}{1.820098in}}%
\pgfpathlineto{\pgfqpoint{2.458427in}{1.809597in}}%
\pgfpathclose%
\pgfusepath{stroke,fill}%
\end{pgfscope}%
\begin{pgfscope}%
\pgfpathrectangle{\pgfqpoint{0.887500in}{0.275000in}}{\pgfqpoint{4.225000in}{4.225000in}}%
\pgfusepath{clip}%
\pgfsetbuttcap%
\pgfsetroundjoin%
\definecolor{currentfill}{rgb}{0.257322,0.256130,0.526563}%
\pgfsetfillcolor{currentfill}%
\pgfsetfillopacity{0.700000}%
\pgfsetlinewidth{0.501875pt}%
\definecolor{currentstroke}{rgb}{1.000000,1.000000,1.000000}%
\pgfsetstrokecolor{currentstroke}%
\pgfsetstrokeopacity{0.500000}%
\pgfsetdash{}{0pt}%
\pgfpathmoveto{\pgfqpoint{3.149739in}{1.562481in}}%
\pgfpathlineto{\pgfqpoint{3.161286in}{1.566251in}}%
\pgfpathlineto{\pgfqpoint{3.172828in}{1.570045in}}%
\pgfpathlineto{\pgfqpoint{3.184365in}{1.573867in}}%
\pgfpathlineto{\pgfqpoint{3.195896in}{1.577725in}}%
\pgfpathlineto{\pgfqpoint{3.207421in}{1.581621in}}%
\pgfpathlineto{\pgfqpoint{3.201044in}{1.593778in}}%
\pgfpathlineto{\pgfqpoint{3.194671in}{1.605896in}}%
\pgfpathlineto{\pgfqpoint{3.188301in}{1.617963in}}%
\pgfpathlineto{\pgfqpoint{3.181933in}{1.629969in}}%
\pgfpathlineto{\pgfqpoint{3.175569in}{1.641902in}}%
\pgfpathlineto{\pgfqpoint{3.164051in}{1.638013in}}%
\pgfpathlineto{\pgfqpoint{3.152527in}{1.634156in}}%
\pgfpathlineto{\pgfqpoint{3.140997in}{1.630324in}}%
\pgfpathlineto{\pgfqpoint{3.129462in}{1.626505in}}%
\pgfpathlineto{\pgfqpoint{3.117921in}{1.622693in}}%
\pgfpathlineto{\pgfqpoint{3.124278in}{1.610804in}}%
\pgfpathlineto{\pgfqpoint{3.130639in}{1.598840in}}%
\pgfpathlineto{\pgfqpoint{3.137002in}{1.586799in}}%
\pgfpathlineto{\pgfqpoint{3.143369in}{1.574680in}}%
\pgfpathclose%
\pgfusepath{stroke,fill}%
\end{pgfscope}%
\begin{pgfscope}%
\pgfpathrectangle{\pgfqpoint{0.887500in}{0.275000in}}{\pgfqpoint{4.225000in}{4.225000in}}%
\pgfusepath{clip}%
\pgfsetbuttcap%
\pgfsetroundjoin%
\definecolor{currentfill}{rgb}{0.265145,0.232956,0.516599}%
\pgfsetfillcolor{currentfill}%
\pgfsetfillopacity{0.700000}%
\pgfsetlinewidth{0.501875pt}%
\definecolor{currentstroke}{rgb}{1.000000,1.000000,1.000000}%
\pgfsetstrokecolor{currentstroke}%
\pgfsetstrokeopacity{0.500000}%
\pgfsetdash{}{0pt}%
\pgfpathmoveto{\pgfqpoint{3.239351in}{1.520602in}}%
\pgfpathlineto{\pgfqpoint{3.250878in}{1.524630in}}%
\pgfpathlineto{\pgfqpoint{3.262400in}{1.528646in}}%
\pgfpathlineto{\pgfqpoint{3.273916in}{1.532653in}}%
\pgfpathlineto{\pgfqpoint{3.285426in}{1.536642in}}%
\pgfpathlineto{\pgfqpoint{3.296930in}{1.540605in}}%
\pgfpathlineto{\pgfqpoint{3.290531in}{1.552886in}}%
\pgfpathlineto{\pgfqpoint{3.284136in}{1.565138in}}%
\pgfpathlineto{\pgfqpoint{3.277743in}{1.577371in}}%
\pgfpathlineto{\pgfqpoint{3.271353in}{1.589581in}}%
\pgfpathlineto{\pgfqpoint{3.264967in}{1.601762in}}%
\pgfpathlineto{\pgfqpoint{3.253469in}{1.597672in}}%
\pgfpathlineto{\pgfqpoint{3.241965in}{1.593592in}}%
\pgfpathlineto{\pgfqpoint{3.230455in}{1.589550in}}%
\pgfpathlineto{\pgfqpoint{3.218941in}{1.585561in}}%
\pgfpathlineto{\pgfqpoint{3.207421in}{1.581621in}}%
\pgfpathlineto{\pgfqpoint{3.213801in}{1.569435in}}%
\pgfpathlineto{\pgfqpoint{3.220183in}{1.557232in}}%
\pgfpathlineto{\pgfqpoint{3.226569in}{1.545023in}}%
\pgfpathlineto{\pgfqpoint{3.232959in}{1.532818in}}%
\pgfpathclose%
\pgfusepath{stroke,fill}%
\end{pgfscope}%
\begin{pgfscope}%
\pgfpathrectangle{\pgfqpoint{0.887500in}{0.275000in}}{\pgfqpoint{4.225000in}{4.225000in}}%
\pgfusepath{clip}%
\pgfsetbuttcap%
\pgfsetroundjoin%
\definecolor{currentfill}{rgb}{0.271828,0.209303,0.504434}%
\pgfsetfillcolor{currentfill}%
\pgfsetfillopacity{0.700000}%
\pgfsetlinewidth{0.501875pt}%
\definecolor{currentstroke}{rgb}{1.000000,1.000000,1.000000}%
\pgfsetstrokecolor{currentstroke}%
\pgfsetstrokeopacity{0.500000}%
\pgfsetdash{}{0pt}%
\pgfpathmoveto{\pgfqpoint{3.328961in}{1.477450in}}%
\pgfpathlineto{\pgfqpoint{3.340464in}{1.481135in}}%
\pgfpathlineto{\pgfqpoint{3.351960in}{1.484747in}}%
\pgfpathlineto{\pgfqpoint{3.363451in}{1.488290in}}%
\pgfpathlineto{\pgfqpoint{3.374934in}{1.491769in}}%
\pgfpathlineto{\pgfqpoint{3.386412in}{1.495192in}}%
\pgfpathlineto{\pgfqpoint{3.379998in}{1.508281in}}%
\pgfpathlineto{\pgfqpoint{3.373586in}{1.521269in}}%
\pgfpathlineto{\pgfqpoint{3.367176in}{1.534157in}}%
\pgfpathlineto{\pgfqpoint{3.360768in}{1.546942in}}%
\pgfpathlineto{\pgfqpoint{3.354363in}{1.559622in}}%
\pgfpathlineto{\pgfqpoint{3.342889in}{1.555958in}}%
\pgfpathlineto{\pgfqpoint{3.331408in}{1.552216in}}%
\pgfpathlineto{\pgfqpoint{3.319922in}{1.548403in}}%
\pgfpathlineto{\pgfqpoint{3.308429in}{1.544528in}}%
\pgfpathlineto{\pgfqpoint{3.296930in}{1.540605in}}%
\pgfpathlineto{\pgfqpoint{3.303332in}{1.528256in}}%
\pgfpathlineto{\pgfqpoint{3.309736in}{1.515804in}}%
\pgfpathlineto{\pgfqpoint{3.316143in}{1.503210in}}%
\pgfpathlineto{\pgfqpoint{3.322551in}{1.490438in}}%
\pgfpathclose%
\pgfusepath{stroke,fill}%
\end{pgfscope}%
\begin{pgfscope}%
\pgfpathrectangle{\pgfqpoint{0.887500in}{0.275000in}}{\pgfqpoint{4.225000in}{4.225000in}}%
\pgfusepath{clip}%
\pgfsetbuttcap%
\pgfsetroundjoin%
\definecolor{currentfill}{rgb}{0.210503,0.363727,0.552206}%
\pgfsetfillcolor{currentfill}%
\pgfsetfillopacity{0.700000}%
\pgfsetlinewidth{0.501875pt}%
\definecolor{currentstroke}{rgb}{1.000000,1.000000,1.000000}%
\pgfsetstrokecolor{currentstroke}%
\pgfsetstrokeopacity{0.500000}%
\pgfsetdash{}{0pt}%
\pgfpathmoveto{\pgfqpoint{2.554083in}{1.765692in}}%
\pgfpathlineto{\pgfqpoint{2.565779in}{1.769691in}}%
\pgfpathlineto{\pgfqpoint{2.577468in}{1.773680in}}%
\pgfpathlineto{\pgfqpoint{2.589153in}{1.777657in}}%
\pgfpathlineto{\pgfqpoint{2.600831in}{1.781621in}}%
\pgfpathlineto{\pgfqpoint{2.612505in}{1.785570in}}%
\pgfpathlineto{\pgfqpoint{2.606293in}{1.796266in}}%
\pgfpathlineto{\pgfqpoint{2.600086in}{1.806911in}}%
\pgfpathlineto{\pgfqpoint{2.593883in}{1.817508in}}%
\pgfpathlineto{\pgfqpoint{2.587684in}{1.828055in}}%
\pgfpathlineto{\pgfqpoint{2.581489in}{1.838554in}}%
\pgfpathlineto{\pgfqpoint{2.569825in}{1.834658in}}%
\pgfpathlineto{\pgfqpoint{2.558155in}{1.830745in}}%
\pgfpathlineto{\pgfqpoint{2.546480in}{1.826817in}}%
\pgfpathlineto{\pgfqpoint{2.534799in}{1.822874in}}%
\pgfpathlineto{\pgfqpoint{2.523113in}{1.818919in}}%
\pgfpathlineto{\pgfqpoint{2.529299in}{1.808369in}}%
\pgfpathlineto{\pgfqpoint{2.535489in}{1.797771in}}%
\pgfpathlineto{\pgfqpoint{2.541683in}{1.787126in}}%
\pgfpathlineto{\pgfqpoint{2.547881in}{1.776433in}}%
\pgfpathclose%
\pgfusepath{stroke,fill}%
\end{pgfscope}%
\begin{pgfscope}%
\pgfpathrectangle{\pgfqpoint{0.887500in}{0.275000in}}{\pgfqpoint{4.225000in}{4.225000in}}%
\pgfusepath{clip}%
\pgfsetbuttcap%
\pgfsetroundjoin%
\definecolor{currentfill}{rgb}{0.185556,0.418570,0.556753}%
\pgfsetfillcolor{currentfill}%
\pgfsetfillopacity{0.700000}%
\pgfsetlinewidth{0.501875pt}%
\definecolor{currentstroke}{rgb}{1.000000,1.000000,1.000000}%
\pgfsetstrokecolor{currentstroke}%
\pgfsetstrokeopacity{0.500000}%
\pgfsetdash{}{0pt}%
\pgfpathmoveto{\pgfqpoint{2.137576in}{1.875298in}}%
\pgfpathlineto{\pgfqpoint{2.149374in}{1.879218in}}%
\pgfpathlineto{\pgfqpoint{2.161166in}{1.883136in}}%
\pgfpathlineto{\pgfqpoint{2.172953in}{1.887049in}}%
\pgfpathlineto{\pgfqpoint{2.184734in}{1.890957in}}%
\pgfpathlineto{\pgfqpoint{2.196509in}{1.894858in}}%
\pgfpathlineto{\pgfqpoint{2.190432in}{1.904895in}}%
\pgfpathlineto{\pgfqpoint{2.184359in}{1.914889in}}%
\pgfpathlineto{\pgfqpoint{2.178290in}{1.924843in}}%
\pgfpathlineto{\pgfqpoint{2.172226in}{1.934755in}}%
\pgfpathlineto{\pgfqpoint{2.166167in}{1.944626in}}%
\pgfpathlineto{\pgfqpoint{2.154401in}{1.940775in}}%
\pgfpathlineto{\pgfqpoint{2.142630in}{1.936914in}}%
\pgfpathlineto{\pgfqpoint{2.130854in}{1.933046in}}%
\pgfpathlineto{\pgfqpoint{2.119072in}{1.929173in}}%
\pgfpathlineto{\pgfqpoint{2.107284in}{1.925297in}}%
\pgfpathlineto{\pgfqpoint{2.113333in}{1.915378in}}%
\pgfpathlineto{\pgfqpoint{2.119387in}{1.905419in}}%
\pgfpathlineto{\pgfqpoint{2.125445in}{1.895419in}}%
\pgfpathlineto{\pgfqpoint{2.131508in}{1.885379in}}%
\pgfpathclose%
\pgfusepath{stroke,fill}%
\end{pgfscope}%
\begin{pgfscope}%
\pgfpathrectangle{\pgfqpoint{0.887500in}{0.275000in}}{\pgfqpoint{4.225000in}{4.225000in}}%
\pgfusepath{clip}%
\pgfsetbuttcap%
\pgfsetroundjoin%
\definecolor{currentfill}{rgb}{0.216210,0.351535,0.550627}%
\pgfsetfillcolor{currentfill}%
\pgfsetfillopacity{0.700000}%
\pgfsetlinewidth{0.501875pt}%
\definecolor{currentstroke}{rgb}{1.000000,1.000000,1.000000}%
\pgfsetstrokecolor{currentstroke}%
\pgfsetstrokeopacity{0.500000}%
\pgfsetdash{}{0pt}%
\pgfpathmoveto{\pgfqpoint{2.643623in}{1.731339in}}%
\pgfpathlineto{\pgfqpoint{2.655299in}{1.735318in}}%
\pgfpathlineto{\pgfqpoint{2.666971in}{1.739281in}}%
\pgfpathlineto{\pgfqpoint{2.678636in}{1.743226in}}%
\pgfpathlineto{\pgfqpoint{2.690296in}{1.747154in}}%
\pgfpathlineto{\pgfqpoint{2.701951in}{1.751066in}}%
\pgfpathlineto{\pgfqpoint{2.695710in}{1.761966in}}%
\pgfpathlineto{\pgfqpoint{2.689474in}{1.772815in}}%
\pgfpathlineto{\pgfqpoint{2.683241in}{1.783613in}}%
\pgfpathlineto{\pgfqpoint{2.677013in}{1.794358in}}%
\pgfpathlineto{\pgfqpoint{2.670788in}{1.805051in}}%
\pgfpathlineto{\pgfqpoint{2.659143in}{1.801191in}}%
\pgfpathlineto{\pgfqpoint{2.647491in}{1.797315in}}%
\pgfpathlineto{\pgfqpoint{2.635835in}{1.793418in}}%
\pgfpathlineto{\pgfqpoint{2.624172in}{1.789503in}}%
\pgfpathlineto{\pgfqpoint{2.612505in}{1.785570in}}%
\pgfpathlineto{\pgfqpoint{2.618720in}{1.774825in}}%
\pgfpathlineto{\pgfqpoint{2.624940in}{1.764029in}}%
\pgfpathlineto{\pgfqpoint{2.631163in}{1.753182in}}%
\pgfpathlineto{\pgfqpoint{2.637391in}{1.742285in}}%
\pgfpathclose%
\pgfusepath{stroke,fill}%
\end{pgfscope}%
\begin{pgfscope}%
\pgfpathrectangle{\pgfqpoint{0.887500in}{0.275000in}}{\pgfqpoint{4.225000in}{4.225000in}}%
\pgfusepath{clip}%
\pgfsetbuttcap%
\pgfsetroundjoin%
\definecolor{currentfill}{rgb}{0.274952,0.037752,0.364543}%
\pgfsetfillcolor{currentfill}%
\pgfsetfillopacity{0.700000}%
\pgfsetlinewidth{0.501875pt}%
\definecolor{currentstroke}{rgb}{1.000000,1.000000,1.000000}%
\pgfsetstrokecolor{currentstroke}%
\pgfsetstrokeopacity{0.500000}%
\pgfsetdash{}{0pt}%
\pgfpathmoveto{\pgfqpoint{3.719325in}{1.205626in}}%
\pgfpathlineto{\pgfqpoint{3.730783in}{1.212238in}}%
\pgfpathlineto{\pgfqpoint{3.742239in}{1.218960in}}%
\pgfpathlineto{\pgfqpoint{3.753694in}{1.225777in}}%
\pgfpathlineto{\pgfqpoint{3.765145in}{1.232672in}}%
\pgfpathlineto{\pgfqpoint{3.776594in}{1.239628in}}%
\pgfpathlineto{\pgfqpoint{3.769982in}{1.247204in}}%
\pgfpathlineto{\pgfqpoint{3.763407in}{1.256529in}}%
\pgfpathlineto{\pgfqpoint{3.756866in}{1.267445in}}%
\pgfpathlineto{\pgfqpoint{3.750354in}{1.279791in}}%
\pgfpathlineto{\pgfqpoint{3.743869in}{1.293406in}}%
\pgfpathlineto{\pgfqpoint{3.732494in}{1.290052in}}%
\pgfpathlineto{\pgfqpoint{3.721116in}{1.286867in}}%
\pgfpathlineto{\pgfqpoint{3.709734in}{1.283776in}}%
\pgfpathlineto{\pgfqpoint{3.698346in}{1.280707in}}%
\pgfpathlineto{\pgfqpoint{3.686952in}{1.277584in}}%
\pgfpathlineto{\pgfqpoint{3.693406in}{1.262289in}}%
\pgfpathlineto{\pgfqpoint{3.699869in}{1.247376in}}%
\pgfpathlineto{\pgfqpoint{3.706342in}{1.232914in}}%
\pgfpathlineto{\pgfqpoint{3.712827in}{1.218975in}}%
\pgfpathclose%
\pgfusepath{stroke,fill}%
\end{pgfscope}%
\begin{pgfscope}%
\pgfpathrectangle{\pgfqpoint{0.887500in}{0.275000in}}{\pgfqpoint{4.225000in}{4.225000in}}%
\pgfusepath{clip}%
\pgfsetbuttcap%
\pgfsetroundjoin%
\definecolor{currentfill}{rgb}{0.190631,0.407061,0.556089}%
\pgfsetfillcolor{currentfill}%
\pgfsetfillopacity{0.700000}%
\pgfsetlinewidth{0.501875pt}%
\definecolor{currentstroke}{rgb}{1.000000,1.000000,1.000000}%
\pgfsetstrokecolor{currentstroke}%
\pgfsetstrokeopacity{0.500000}%
\pgfsetdash{}{0pt}%
\pgfpathmoveto{\pgfqpoint{2.226964in}{1.844046in}}%
\pgfpathlineto{\pgfqpoint{2.238744in}{1.847987in}}%
\pgfpathlineto{\pgfqpoint{2.250518in}{1.851920in}}%
\pgfpathlineto{\pgfqpoint{2.262287in}{1.855845in}}%
\pgfpathlineto{\pgfqpoint{2.274050in}{1.859763in}}%
\pgfpathlineto{\pgfqpoint{2.285807in}{1.863673in}}%
\pgfpathlineto{\pgfqpoint{2.279697in}{1.873870in}}%
\pgfpathlineto{\pgfqpoint{2.273592in}{1.884024in}}%
\pgfpathlineto{\pgfqpoint{2.267491in}{1.894134in}}%
\pgfpathlineto{\pgfqpoint{2.261395in}{1.904201in}}%
\pgfpathlineto{\pgfqpoint{2.255303in}{1.914225in}}%
\pgfpathlineto{\pgfqpoint{2.243555in}{1.910371in}}%
\pgfpathlineto{\pgfqpoint{2.231802in}{1.906507in}}%
\pgfpathlineto{\pgfqpoint{2.220043in}{1.902633in}}%
\pgfpathlineto{\pgfqpoint{2.208279in}{1.898750in}}%
\pgfpathlineto{\pgfqpoint{2.196509in}{1.894858in}}%
\pgfpathlineto{\pgfqpoint{2.202591in}{1.884780in}}%
\pgfpathlineto{\pgfqpoint{2.208678in}{1.874659in}}%
\pgfpathlineto{\pgfqpoint{2.214769in}{1.864497in}}%
\pgfpathlineto{\pgfqpoint{2.220864in}{1.854293in}}%
\pgfpathclose%
\pgfusepath{stroke,fill}%
\end{pgfscope}%
\begin{pgfscope}%
\pgfpathrectangle{\pgfqpoint{0.887500in}{0.275000in}}{\pgfqpoint{4.225000in}{4.225000in}}%
\pgfusepath{clip}%
\pgfsetbuttcap%
\pgfsetroundjoin%
\definecolor{currentfill}{rgb}{0.223925,0.334994,0.548053}%
\pgfsetfillcolor{currentfill}%
\pgfsetfillopacity{0.700000}%
\pgfsetlinewidth{0.501875pt}%
\definecolor{currentstroke}{rgb}{1.000000,1.000000,1.000000}%
\pgfsetstrokecolor{currentstroke}%
\pgfsetstrokeopacity{0.500000}%
\pgfsetdash{}{0pt}%
\pgfpathmoveto{\pgfqpoint{2.733212in}{1.695832in}}%
\pgfpathlineto{\pgfqpoint{2.744870in}{1.699779in}}%
\pgfpathlineto{\pgfqpoint{2.756522in}{1.703719in}}%
\pgfpathlineto{\pgfqpoint{2.768168in}{1.707654in}}%
\pgfpathlineto{\pgfqpoint{2.779809in}{1.711589in}}%
\pgfpathlineto{\pgfqpoint{2.791444in}{1.715527in}}%
\pgfpathlineto{\pgfqpoint{2.785175in}{1.726622in}}%
\pgfpathlineto{\pgfqpoint{2.778911in}{1.737668in}}%
\pgfpathlineto{\pgfqpoint{2.772650in}{1.748664in}}%
\pgfpathlineto{\pgfqpoint{2.766393in}{1.759611in}}%
\pgfpathlineto{\pgfqpoint{2.760140in}{1.770507in}}%
\pgfpathlineto{\pgfqpoint{2.748513in}{1.766624in}}%
\pgfpathlineto{\pgfqpoint{2.736881in}{1.762741in}}%
\pgfpathlineto{\pgfqpoint{2.725243in}{1.758856in}}%
\pgfpathlineto{\pgfqpoint{2.713600in}{1.754965in}}%
\pgfpathlineto{\pgfqpoint{2.701951in}{1.751066in}}%
\pgfpathlineto{\pgfqpoint{2.708195in}{1.740116in}}%
\pgfpathlineto{\pgfqpoint{2.714444in}{1.729116in}}%
\pgfpathlineto{\pgfqpoint{2.720696in}{1.718068in}}%
\pgfpathlineto{\pgfqpoint{2.726952in}{1.706973in}}%
\pgfpathclose%
\pgfusepath{stroke,fill}%
\end{pgfscope}%
\begin{pgfscope}%
\pgfpathrectangle{\pgfqpoint{0.887500in}{0.275000in}}{\pgfqpoint{4.225000in}{4.225000in}}%
\pgfusepath{clip}%
\pgfsetbuttcap%
\pgfsetroundjoin%
\definecolor{currentfill}{rgb}{0.281446,0.084320,0.407414}%
\pgfsetfillcolor{currentfill}%
\pgfsetfillopacity{0.700000}%
\pgfsetlinewidth{0.501875pt}%
\definecolor{currentstroke}{rgb}{1.000000,1.000000,1.000000}%
\pgfsetstrokecolor{currentstroke}%
\pgfsetstrokeopacity{0.500000}%
\pgfsetdash{}{0pt}%
\pgfpathmoveto{\pgfqpoint{3.629844in}{1.259113in}}%
\pgfpathlineto{\pgfqpoint{3.641284in}{1.263236in}}%
\pgfpathlineto{\pgfqpoint{3.652715in}{1.267177in}}%
\pgfpathlineto{\pgfqpoint{3.664137in}{1.270885in}}%
\pgfpathlineto{\pgfqpoint{3.675549in}{1.274335in}}%
\pgfpathlineto{\pgfqpoint{3.686952in}{1.277584in}}%
\pgfpathlineto{\pgfqpoint{3.680505in}{1.293191in}}%
\pgfpathlineto{\pgfqpoint{3.674064in}{1.309041in}}%
\pgfpathlineto{\pgfqpoint{3.667628in}{1.325062in}}%
\pgfpathlineto{\pgfqpoint{3.661196in}{1.341186in}}%
\pgfpathlineto{\pgfqpoint{3.654766in}{1.357341in}}%
\pgfpathlineto{\pgfqpoint{3.643349in}{1.353157in}}%
\pgfpathlineto{\pgfqpoint{3.631925in}{1.348853in}}%
\pgfpathlineto{\pgfqpoint{3.620494in}{1.344471in}}%
\pgfpathlineto{\pgfqpoint{3.609057in}{1.340031in}}%
\pgfpathlineto{\pgfqpoint{3.597615in}{1.335548in}}%
\pgfpathlineto{\pgfqpoint{3.604064in}{1.320802in}}%
\pgfpathlineto{\pgfqpoint{3.610511in}{1.305798in}}%
\pgfpathlineto{\pgfqpoint{3.616958in}{1.290524in}}%
\pgfpathlineto{\pgfqpoint{3.623402in}{1.274967in}}%
\pgfpathclose%
\pgfusepath{stroke,fill}%
\end{pgfscope}%
\begin{pgfscope}%
\pgfpathrectangle{\pgfqpoint{0.887500in}{0.275000in}}{\pgfqpoint{4.225000in}{4.225000in}}%
\pgfusepath{clip}%
\pgfsetbuttcap%
\pgfsetroundjoin%
\definecolor{currentfill}{rgb}{0.175841,0.441290,0.557685}%
\pgfsetfillcolor{currentfill}%
\pgfsetfillopacity{0.700000}%
\pgfsetlinewidth{0.501875pt}%
\definecolor{currentstroke}{rgb}{1.000000,1.000000,1.000000}%
\pgfsetstrokecolor{currentstroke}%
\pgfsetstrokeopacity{0.500000}%
\pgfsetdash{}{0pt}%
\pgfpathmoveto{\pgfqpoint{1.899753in}{1.916625in}}%
\pgfpathlineto{\pgfqpoint{1.911616in}{1.920518in}}%
\pgfpathlineto{\pgfqpoint{1.923473in}{1.924400in}}%
\pgfpathlineto{\pgfqpoint{1.935325in}{1.928271in}}%
\pgfpathlineto{\pgfqpoint{1.947172in}{1.932133in}}%
\pgfpathlineto{\pgfqpoint{1.959012in}{1.935987in}}%
\pgfpathlineto{\pgfqpoint{1.953011in}{1.945777in}}%
\pgfpathlineto{\pgfqpoint{1.947014in}{1.955532in}}%
\pgfpathlineto{\pgfqpoint{1.941021in}{1.965252in}}%
\pgfpathlineto{\pgfqpoint{1.935034in}{1.974937in}}%
\pgfpathlineto{\pgfqpoint{1.929050in}{1.984588in}}%
\pgfpathlineto{\pgfqpoint{1.917220in}{1.980785in}}%
\pgfpathlineto{\pgfqpoint{1.905384in}{1.976974in}}%
\pgfpathlineto{\pgfqpoint{1.893543in}{1.973152in}}%
\pgfpathlineto{\pgfqpoint{1.881696in}{1.969319in}}%
\pgfpathlineto{\pgfqpoint{1.869843in}{1.965473in}}%
\pgfpathlineto{\pgfqpoint{1.875816in}{1.955771in}}%
\pgfpathlineto{\pgfqpoint{1.881793in}{1.946036in}}%
\pgfpathlineto{\pgfqpoint{1.887775in}{1.936267in}}%
\pgfpathlineto{\pgfqpoint{1.893762in}{1.926463in}}%
\pgfpathclose%
\pgfusepath{stroke,fill}%
\end{pgfscope}%
\begin{pgfscope}%
\pgfpathrectangle{\pgfqpoint{0.887500in}{0.275000in}}{\pgfqpoint{4.225000in}{4.225000in}}%
\pgfusepath{clip}%
\pgfsetbuttcap%
\pgfsetroundjoin%
\definecolor{currentfill}{rgb}{0.233603,0.313828,0.543914}%
\pgfsetfillcolor{currentfill}%
\pgfsetfillopacity{0.700000}%
\pgfsetlinewidth{0.501875pt}%
\definecolor{currentstroke}{rgb}{1.000000,1.000000,1.000000}%
\pgfsetstrokecolor{currentstroke}%
\pgfsetstrokeopacity{0.500000}%
\pgfsetdash{}{0pt}%
\pgfpathmoveto{\pgfqpoint{2.822843in}{1.659331in}}%
\pgfpathlineto{\pgfqpoint{2.834481in}{1.663314in}}%
\pgfpathlineto{\pgfqpoint{2.846113in}{1.667300in}}%
\pgfpathlineto{\pgfqpoint{2.857739in}{1.671289in}}%
\pgfpathlineto{\pgfqpoint{2.869360in}{1.675279in}}%
\pgfpathlineto{\pgfqpoint{2.880975in}{1.679267in}}%
\pgfpathlineto{\pgfqpoint{2.874679in}{1.690575in}}%
\pgfpathlineto{\pgfqpoint{2.868387in}{1.701829in}}%
\pgfpathlineto{\pgfqpoint{2.862099in}{1.713030in}}%
\pgfpathlineto{\pgfqpoint{2.855815in}{1.724180in}}%
\pgfpathlineto{\pgfqpoint{2.849534in}{1.735278in}}%
\pgfpathlineto{\pgfqpoint{2.837927in}{1.731321in}}%
\pgfpathlineto{\pgfqpoint{2.826315in}{1.727367in}}%
\pgfpathlineto{\pgfqpoint{2.814697in}{1.723416in}}%
\pgfpathlineto{\pgfqpoint{2.803073in}{1.719469in}}%
\pgfpathlineto{\pgfqpoint{2.791444in}{1.715527in}}%
\pgfpathlineto{\pgfqpoint{2.797716in}{1.704383in}}%
\pgfpathlineto{\pgfqpoint{2.803992in}{1.693192in}}%
\pgfpathlineto{\pgfqpoint{2.810272in}{1.681954in}}%
\pgfpathlineto{\pgfqpoint{2.816556in}{1.670668in}}%
\pgfpathclose%
\pgfusepath{stroke,fill}%
\end{pgfscope}%
\begin{pgfscope}%
\pgfpathrectangle{\pgfqpoint{0.887500in}{0.275000in}}{\pgfqpoint{4.225000in}{4.225000in}}%
\pgfusepath{clip}%
\pgfsetbuttcap%
\pgfsetroundjoin%
\definecolor{currentfill}{rgb}{0.197636,0.391528,0.554969}%
\pgfsetfillcolor{currentfill}%
\pgfsetfillopacity{0.700000}%
\pgfsetlinewidth{0.501875pt}%
\definecolor{currentstroke}{rgb}{1.000000,1.000000,1.000000}%
\pgfsetstrokecolor{currentstroke}%
\pgfsetstrokeopacity{0.500000}%
\pgfsetdash{}{0pt}%
\pgfpathmoveto{\pgfqpoint{2.316421in}{1.812042in}}%
\pgfpathlineto{\pgfqpoint{2.328183in}{1.815992in}}%
\pgfpathlineto{\pgfqpoint{2.339938in}{1.819935in}}%
\pgfpathlineto{\pgfqpoint{2.351689in}{1.823869in}}%
\pgfpathlineto{\pgfqpoint{2.363433in}{1.827796in}}%
\pgfpathlineto{\pgfqpoint{2.375172in}{1.831718in}}%
\pgfpathlineto{\pgfqpoint{2.369031in}{1.842084in}}%
\pgfpathlineto{\pgfqpoint{2.362894in}{1.852406in}}%
\pgfpathlineto{\pgfqpoint{2.356762in}{1.862682in}}%
\pgfpathlineto{\pgfqpoint{2.350634in}{1.872914in}}%
\pgfpathlineto{\pgfqpoint{2.344510in}{1.883102in}}%
\pgfpathlineto{\pgfqpoint{2.332781in}{1.879230in}}%
\pgfpathlineto{\pgfqpoint{2.321046in}{1.875353in}}%
\pgfpathlineto{\pgfqpoint{2.309305in}{1.871468in}}%
\pgfpathlineto{\pgfqpoint{2.297559in}{1.867575in}}%
\pgfpathlineto{\pgfqpoint{2.285807in}{1.863673in}}%
\pgfpathlineto{\pgfqpoint{2.291921in}{1.853433in}}%
\pgfpathlineto{\pgfqpoint{2.298039in}{1.843150in}}%
\pgfpathlineto{\pgfqpoint{2.304162in}{1.832824in}}%
\pgfpathlineto{\pgfqpoint{2.310290in}{1.822454in}}%
\pgfpathclose%
\pgfusepath{stroke,fill}%
\end{pgfscope}%
\begin{pgfscope}%
\pgfpathrectangle{\pgfqpoint{0.887500in}{0.275000in}}{\pgfqpoint{4.225000in}{4.225000in}}%
\pgfusepath{clip}%
\pgfsetbuttcap%
\pgfsetroundjoin%
\definecolor{currentfill}{rgb}{0.241237,0.296485,0.539709}%
\pgfsetfillcolor{currentfill}%
\pgfsetfillopacity{0.700000}%
\pgfsetlinewidth{0.501875pt}%
\definecolor{currentstroke}{rgb}{1.000000,1.000000,1.000000}%
\pgfsetstrokecolor{currentstroke}%
\pgfsetstrokeopacity{0.500000}%
\pgfsetdash{}{0pt}%
\pgfpathmoveto{\pgfqpoint{2.912508in}{1.621840in}}%
\pgfpathlineto{\pgfqpoint{2.924126in}{1.625849in}}%
\pgfpathlineto{\pgfqpoint{2.935738in}{1.629852in}}%
\pgfpathlineto{\pgfqpoint{2.947344in}{1.633847in}}%
\pgfpathlineto{\pgfqpoint{2.958945in}{1.637834in}}%
\pgfpathlineto{\pgfqpoint{2.970540in}{1.641813in}}%
\pgfpathlineto{\pgfqpoint{2.964218in}{1.653376in}}%
\pgfpathlineto{\pgfqpoint{2.957900in}{1.664890in}}%
\pgfpathlineto{\pgfqpoint{2.951585in}{1.676353in}}%
\pgfpathlineto{\pgfqpoint{2.945274in}{1.687763in}}%
\pgfpathlineto{\pgfqpoint{2.938967in}{1.699120in}}%
\pgfpathlineto{\pgfqpoint{2.927379in}{1.695167in}}%
\pgfpathlineto{\pgfqpoint{2.915787in}{1.691203in}}%
\pgfpathlineto{\pgfqpoint{2.904188in}{1.687231in}}%
\pgfpathlineto{\pgfqpoint{2.892584in}{1.683252in}}%
\pgfpathlineto{\pgfqpoint{2.880975in}{1.679267in}}%
\pgfpathlineto{\pgfqpoint{2.887274in}{1.667902in}}%
\pgfpathlineto{\pgfqpoint{2.893577in}{1.656480in}}%
\pgfpathlineto{\pgfqpoint{2.899884in}{1.644996in}}%
\pgfpathlineto{\pgfqpoint{2.906194in}{1.633450in}}%
\pgfpathclose%
\pgfusepath{stroke,fill}%
\end{pgfscope}%
\begin{pgfscope}%
\pgfpathrectangle{\pgfqpoint{0.887500in}{0.275000in}}{\pgfqpoint{4.225000in}{4.225000in}}%
\pgfusepath{clip}%
\pgfsetbuttcap%
\pgfsetroundjoin%
\definecolor{currentfill}{rgb}{0.283197,0.115680,0.436115}%
\pgfsetfillcolor{currentfill}%
\pgfsetfillopacity{0.700000}%
\pgfsetlinewidth{0.501875pt}%
\definecolor{currentstroke}{rgb}{1.000000,1.000000,1.000000}%
\pgfsetstrokecolor{currentstroke}%
\pgfsetstrokeopacity{0.500000}%
\pgfsetdash{}{0pt}%
\pgfpathmoveto{\pgfqpoint{3.540320in}{1.312991in}}%
\pgfpathlineto{\pgfqpoint{3.551790in}{1.317489in}}%
\pgfpathlineto{\pgfqpoint{3.563254in}{1.321992in}}%
\pgfpathlineto{\pgfqpoint{3.574713in}{1.326514in}}%
\pgfpathlineto{\pgfqpoint{3.586166in}{1.331037in}}%
\pgfpathlineto{\pgfqpoint{3.597615in}{1.335548in}}%
\pgfpathlineto{\pgfqpoint{3.591165in}{1.350050in}}%
\pgfpathlineto{\pgfqpoint{3.584714in}{1.364322in}}%
\pgfpathlineto{\pgfqpoint{3.578264in}{1.378377in}}%
\pgfpathlineto{\pgfqpoint{3.571813in}{1.392227in}}%
\pgfpathlineto{\pgfqpoint{3.565362in}{1.405887in}}%
\pgfpathlineto{\pgfqpoint{3.553908in}{1.400629in}}%
\pgfpathlineto{\pgfqpoint{3.542453in}{1.395609in}}%
\pgfpathlineto{\pgfqpoint{3.530995in}{1.390804in}}%
\pgfpathlineto{\pgfqpoint{3.519534in}{1.386196in}}%
\pgfpathlineto{\pgfqpoint{3.508070in}{1.381750in}}%
\pgfpathlineto{\pgfqpoint{3.514519in}{1.368364in}}%
\pgfpathlineto{\pgfqpoint{3.520969in}{1.354824in}}%
\pgfpathlineto{\pgfqpoint{3.527419in}{1.341100in}}%
\pgfpathlineto{\pgfqpoint{3.533870in}{1.327166in}}%
\pgfpathclose%
\pgfusepath{stroke,fill}%
\end{pgfscope}%
\begin{pgfscope}%
\pgfpathrectangle{\pgfqpoint{0.887500in}{0.275000in}}{\pgfqpoint{4.225000in}{4.225000in}}%
\pgfusepath{clip}%
\pgfsetbuttcap%
\pgfsetroundjoin%
\definecolor{currentfill}{rgb}{0.282290,0.145912,0.461510}%
\pgfsetfillcolor{currentfill}%
\pgfsetfillopacity{0.700000}%
\pgfsetlinewidth{0.501875pt}%
\definecolor{currentstroke}{rgb}{1.000000,1.000000,1.000000}%
\pgfsetstrokecolor{currentstroke}%
\pgfsetstrokeopacity{0.500000}%
\pgfsetdash{}{0pt}%
\pgfpathmoveto{\pgfqpoint{3.450665in}{1.358975in}}%
\pgfpathlineto{\pgfqpoint{3.462161in}{1.363898in}}%
\pgfpathlineto{\pgfqpoint{3.473648in}{1.368548in}}%
\pgfpathlineto{\pgfqpoint{3.485128in}{1.373015in}}%
\pgfpathlineto{\pgfqpoint{3.496602in}{1.377385in}}%
\pgfpathlineto{\pgfqpoint{3.508070in}{1.381750in}}%
\pgfpathlineto{\pgfqpoint{3.501623in}{1.395010in}}%
\pgfpathlineto{\pgfqpoint{3.495178in}{1.408174in}}%
\pgfpathlineto{\pgfqpoint{3.488734in}{1.421269in}}%
\pgfpathlineto{\pgfqpoint{3.482294in}{1.434325in}}%
\pgfpathlineto{\pgfqpoint{3.475856in}{1.447371in}}%
\pgfpathlineto{\pgfqpoint{3.464397in}{1.443359in}}%
\pgfpathlineto{\pgfqpoint{3.452934in}{1.439514in}}%
\pgfpathlineto{\pgfqpoint{3.441467in}{1.435769in}}%
\pgfpathlineto{\pgfqpoint{3.429994in}{1.432055in}}%
\pgfpathlineto{\pgfqpoint{3.418515in}{1.428306in}}%
\pgfpathlineto{\pgfqpoint{3.424941in}{1.414646in}}%
\pgfpathlineto{\pgfqpoint{3.431370in}{1.400886in}}%
\pgfpathlineto{\pgfqpoint{3.437800in}{1.387023in}}%
\pgfpathlineto{\pgfqpoint{3.444232in}{1.373054in}}%
\pgfpathclose%
\pgfusepath{stroke,fill}%
\end{pgfscope}%
\begin{pgfscope}%
\pgfpathrectangle{\pgfqpoint{0.887500in}{0.275000in}}{\pgfqpoint{4.225000in}{4.225000in}}%
\pgfusepath{clip}%
\pgfsetbuttcap%
\pgfsetroundjoin%
\definecolor{currentfill}{rgb}{0.248629,0.278775,0.534556}%
\pgfsetfillcolor{currentfill}%
\pgfsetfillopacity{0.700000}%
\pgfsetlinewidth{0.501875pt}%
\definecolor{currentstroke}{rgb}{1.000000,1.000000,1.000000}%
\pgfsetstrokecolor{currentstroke}%
\pgfsetstrokeopacity{0.500000}%
\pgfsetdash{}{0pt}%
\pgfpathmoveto{\pgfqpoint{3.002201in}{1.583296in}}%
\pgfpathlineto{\pgfqpoint{3.013798in}{1.587392in}}%
\pgfpathlineto{\pgfqpoint{3.025390in}{1.591463in}}%
\pgfpathlineto{\pgfqpoint{3.036976in}{1.595499in}}%
\pgfpathlineto{\pgfqpoint{3.048556in}{1.599490in}}%
\pgfpathlineto{\pgfqpoint{3.060131in}{1.603433in}}%
\pgfpathlineto{\pgfqpoint{3.053785in}{1.615146in}}%
\pgfpathlineto{\pgfqpoint{3.047441in}{1.626811in}}%
\pgfpathlineto{\pgfqpoint{3.041102in}{1.638442in}}%
\pgfpathlineto{\pgfqpoint{3.034765in}{1.650038in}}%
\pgfpathlineto{\pgfqpoint{3.028432in}{1.661596in}}%
\pgfpathlineto{\pgfqpoint{3.016865in}{1.657655in}}%
\pgfpathlineto{\pgfqpoint{3.005292in}{1.653705in}}%
\pgfpathlineto{\pgfqpoint{2.993714in}{1.649748in}}%
\pgfpathlineto{\pgfqpoint{2.982130in}{1.645784in}}%
\pgfpathlineto{\pgfqpoint{2.970540in}{1.641813in}}%
\pgfpathlineto{\pgfqpoint{2.976865in}{1.630202in}}%
\pgfpathlineto{\pgfqpoint{2.983194in}{1.618545in}}%
\pgfpathlineto{\pgfqpoint{2.989526in}{1.606843in}}%
\pgfpathlineto{\pgfqpoint{2.995862in}{1.595096in}}%
\pgfpathclose%
\pgfusepath{stroke,fill}%
\end{pgfscope}%
\begin{pgfscope}%
\pgfpathrectangle{\pgfqpoint{0.887500in}{0.275000in}}{\pgfqpoint{4.225000in}{4.225000in}}%
\pgfusepath{clip}%
\pgfsetbuttcap%
\pgfsetroundjoin%
\definecolor{currentfill}{rgb}{0.180629,0.429975,0.557282}%
\pgfsetfillcolor{currentfill}%
\pgfsetfillopacity{0.700000}%
\pgfsetlinewidth{0.501875pt}%
\definecolor{currentstroke}{rgb}{1.000000,1.000000,1.000000}%
\pgfsetstrokecolor{currentstroke}%
\pgfsetstrokeopacity{0.500000}%
\pgfsetdash{}{0pt}%
\pgfpathmoveto{\pgfqpoint{1.989088in}{1.886492in}}%
\pgfpathlineto{\pgfqpoint{2.000934in}{1.890386in}}%
\pgfpathlineto{\pgfqpoint{2.012773in}{1.894275in}}%
\pgfpathlineto{\pgfqpoint{2.024607in}{1.898159in}}%
\pgfpathlineto{\pgfqpoint{2.036435in}{1.902040in}}%
\pgfpathlineto{\pgfqpoint{2.048258in}{1.905918in}}%
\pgfpathlineto{\pgfqpoint{2.042223in}{1.915844in}}%
\pgfpathlineto{\pgfqpoint{2.036193in}{1.925732in}}%
\pgfpathlineto{\pgfqpoint{2.030168in}{1.935582in}}%
\pgfpathlineto{\pgfqpoint{2.024147in}{1.945394in}}%
\pgfpathlineto{\pgfqpoint{2.018131in}{1.955170in}}%
\pgfpathlineto{\pgfqpoint{2.006318in}{1.951341in}}%
\pgfpathlineto{\pgfqpoint{1.994500in}{1.947510in}}%
\pgfpathlineto{\pgfqpoint{1.982677in}{1.943674in}}%
\pgfpathlineto{\pgfqpoint{1.970847in}{1.939833in}}%
\pgfpathlineto{\pgfqpoint{1.959012in}{1.935987in}}%
\pgfpathlineto{\pgfqpoint{1.965018in}{1.926161in}}%
\pgfpathlineto{\pgfqpoint{1.971029in}{1.916299in}}%
\pgfpathlineto{\pgfqpoint{1.977044in}{1.906400in}}%
\pgfpathlineto{\pgfqpoint{1.983064in}{1.896465in}}%
\pgfpathclose%
\pgfusepath{stroke,fill}%
\end{pgfscope}%
\begin{pgfscope}%
\pgfpathrectangle{\pgfqpoint{0.887500in}{0.275000in}}{\pgfqpoint{4.225000in}{4.225000in}}%
\pgfusepath{clip}%
\pgfsetbuttcap%
\pgfsetroundjoin%
\definecolor{currentfill}{rgb}{0.203063,0.379716,0.553925}%
\pgfsetfillcolor{currentfill}%
\pgfsetfillopacity{0.700000}%
\pgfsetlinewidth{0.501875pt}%
\definecolor{currentstroke}{rgb}{1.000000,1.000000,1.000000}%
\pgfsetstrokecolor{currentstroke}%
\pgfsetstrokeopacity{0.500000}%
\pgfsetdash{}{0pt}%
\pgfpathmoveto{\pgfqpoint{2.405943in}{1.779220in}}%
\pgfpathlineto{\pgfqpoint{2.417686in}{1.783183in}}%
\pgfpathlineto{\pgfqpoint{2.429423in}{1.787146in}}%
\pgfpathlineto{\pgfqpoint{2.441154in}{1.791110in}}%
\pgfpathlineto{\pgfqpoint{2.452880in}{1.795078in}}%
\pgfpathlineto{\pgfqpoint{2.464600in}{1.799050in}}%
\pgfpathlineto{\pgfqpoint{2.458427in}{1.809597in}}%
\pgfpathlineto{\pgfqpoint{2.452260in}{1.820098in}}%
\pgfpathlineto{\pgfqpoint{2.446096in}{1.830553in}}%
\pgfpathlineto{\pgfqpoint{2.439937in}{1.840963in}}%
\pgfpathlineto{\pgfqpoint{2.433782in}{1.851328in}}%
\pgfpathlineto{\pgfqpoint{2.422071in}{1.847399in}}%
\pgfpathlineto{\pgfqpoint{2.410355in}{1.843475in}}%
\pgfpathlineto{\pgfqpoint{2.398633in}{1.839556in}}%
\pgfpathlineto{\pgfqpoint{2.386906in}{1.835637in}}%
\pgfpathlineto{\pgfqpoint{2.375172in}{1.831718in}}%
\pgfpathlineto{\pgfqpoint{2.381318in}{1.821308in}}%
\pgfpathlineto{\pgfqpoint{2.387468in}{1.810853in}}%
\pgfpathlineto{\pgfqpoint{2.393622in}{1.800353in}}%
\pgfpathlineto{\pgfqpoint{2.399780in}{1.789809in}}%
\pgfpathclose%
\pgfusepath{stroke,fill}%
\end{pgfscope}%
\begin{pgfscope}%
\pgfpathrectangle{\pgfqpoint{0.887500in}{0.275000in}}{\pgfqpoint{4.225000in}{4.225000in}}%
\pgfusepath{clip}%
\pgfsetbuttcap%
\pgfsetroundjoin%
\definecolor{currentfill}{rgb}{0.278826,0.175490,0.483397}%
\pgfsetfillcolor{currentfill}%
\pgfsetfillopacity{0.700000}%
\pgfsetlinewidth{0.501875pt}%
\definecolor{currentstroke}{rgb}{1.000000,1.000000,1.000000}%
\pgfsetstrokecolor{currentstroke}%
\pgfsetstrokeopacity{0.500000}%
\pgfsetdash{}{0pt}%
\pgfpathmoveto{\pgfqpoint{3.361021in}{1.407978in}}%
\pgfpathlineto{\pgfqpoint{3.372532in}{1.412183in}}%
\pgfpathlineto{\pgfqpoint{3.384037in}{1.416357in}}%
\pgfpathlineto{\pgfqpoint{3.395536in}{1.420461in}}%
\pgfpathlineto{\pgfqpoint{3.407029in}{1.424454in}}%
\pgfpathlineto{\pgfqpoint{3.418515in}{1.428306in}}%
\pgfpathlineto{\pgfqpoint{3.412090in}{1.441871in}}%
\pgfpathlineto{\pgfqpoint{3.405667in}{1.455343in}}%
\pgfpathlineto{\pgfqpoint{3.399247in}{1.468722in}}%
\pgfpathlineto{\pgfqpoint{3.392828in}{1.482005in}}%
\pgfpathlineto{\pgfqpoint{3.386412in}{1.495192in}}%
\pgfpathlineto{\pgfqpoint{3.374934in}{1.491769in}}%
\pgfpathlineto{\pgfqpoint{3.363451in}{1.488290in}}%
\pgfpathlineto{\pgfqpoint{3.351960in}{1.484747in}}%
\pgfpathlineto{\pgfqpoint{3.340464in}{1.481135in}}%
\pgfpathlineto{\pgfqpoint{3.328961in}{1.477450in}}%
\pgfpathlineto{\pgfqpoint{3.335372in}{1.464208in}}%
\pgfpathlineto{\pgfqpoint{3.341784in}{1.450677in}}%
\pgfpathlineto{\pgfqpoint{3.348197in}{1.436817in}}%
\pgfpathlineto{\pgfqpoint{3.354609in}{1.422593in}}%
\pgfpathclose%
\pgfusepath{stroke,fill}%
\end{pgfscope}%
\begin{pgfscope}%
\pgfpathrectangle{\pgfqpoint{0.887500in}{0.275000in}}{\pgfqpoint{4.225000in}{4.225000in}}%
\pgfusepath{clip}%
\pgfsetbuttcap%
\pgfsetroundjoin%
\definecolor{currentfill}{rgb}{0.257322,0.256130,0.526563}%
\pgfsetfillcolor{currentfill}%
\pgfsetfillopacity{0.700000}%
\pgfsetlinewidth{0.501875pt}%
\definecolor{currentstroke}{rgb}{1.000000,1.000000,1.000000}%
\pgfsetstrokecolor{currentstroke}%
\pgfsetstrokeopacity{0.500000}%
\pgfsetdash{}{0pt}%
\pgfpathmoveto{\pgfqpoint{3.091914in}{1.543507in}}%
\pgfpathlineto{\pgfqpoint{3.103490in}{1.547373in}}%
\pgfpathlineto{\pgfqpoint{3.115061in}{1.551187in}}%
\pgfpathlineto{\pgfqpoint{3.126626in}{1.554966in}}%
\pgfpathlineto{\pgfqpoint{3.138185in}{1.558725in}}%
\pgfpathlineto{\pgfqpoint{3.149739in}{1.562481in}}%
\pgfpathlineto{\pgfqpoint{3.143369in}{1.574680in}}%
\pgfpathlineto{\pgfqpoint{3.137002in}{1.586799in}}%
\pgfpathlineto{\pgfqpoint{3.130639in}{1.598840in}}%
\pgfpathlineto{\pgfqpoint{3.124278in}{1.610804in}}%
\pgfpathlineto{\pgfqpoint{3.117921in}{1.622693in}}%
\pgfpathlineto{\pgfqpoint{3.106374in}{1.618877in}}%
\pgfpathlineto{\pgfqpoint{3.094822in}{1.615050in}}%
\pgfpathlineto{\pgfqpoint{3.083264in}{1.611206in}}%
\pgfpathlineto{\pgfqpoint{3.071701in}{1.607336in}}%
\pgfpathlineto{\pgfqpoint{3.060131in}{1.603433in}}%
\pgfpathlineto{\pgfqpoint{3.066481in}{1.591654in}}%
\pgfpathlineto{\pgfqpoint{3.072835in}{1.579791in}}%
\pgfpathlineto{\pgfqpoint{3.079191in}{1.567824in}}%
\pgfpathlineto{\pgfqpoint{3.085551in}{1.555735in}}%
\pgfpathclose%
\pgfusepath{stroke,fill}%
\end{pgfscope}%
\begin{pgfscope}%
\pgfpathrectangle{\pgfqpoint{0.887500in}{0.275000in}}{\pgfqpoint{4.225000in}{4.225000in}}%
\pgfusepath{clip}%
\pgfsetbuttcap%
\pgfsetroundjoin%
\definecolor{currentfill}{rgb}{0.265145,0.232956,0.516599}%
\pgfsetfillcolor{currentfill}%
\pgfsetfillopacity{0.700000}%
\pgfsetlinewidth{0.501875pt}%
\definecolor{currentstroke}{rgb}{1.000000,1.000000,1.000000}%
\pgfsetstrokecolor{currentstroke}%
\pgfsetstrokeopacity{0.500000}%
\pgfsetdash{}{0pt}%
\pgfpathmoveto{\pgfqpoint{3.181632in}{1.500231in}}%
\pgfpathlineto{\pgfqpoint{3.193187in}{1.504342in}}%
\pgfpathlineto{\pgfqpoint{3.204736in}{1.508433in}}%
\pgfpathlineto{\pgfqpoint{3.216280in}{1.512505in}}%
\pgfpathlineto{\pgfqpoint{3.227819in}{1.516561in}}%
\pgfpathlineto{\pgfqpoint{3.239351in}{1.520602in}}%
\pgfpathlineto{\pgfqpoint{3.232959in}{1.532818in}}%
\pgfpathlineto{\pgfqpoint{3.226569in}{1.545023in}}%
\pgfpathlineto{\pgfqpoint{3.220183in}{1.557232in}}%
\pgfpathlineto{\pgfqpoint{3.213801in}{1.569435in}}%
\pgfpathlineto{\pgfqpoint{3.207421in}{1.581621in}}%
\pgfpathlineto{\pgfqpoint{3.195896in}{1.577725in}}%
\pgfpathlineto{\pgfqpoint{3.184365in}{1.573867in}}%
\pgfpathlineto{\pgfqpoint{3.172828in}{1.570045in}}%
\pgfpathlineto{\pgfqpoint{3.161286in}{1.566251in}}%
\pgfpathlineto{\pgfqpoint{3.149739in}{1.562481in}}%
\pgfpathlineto{\pgfqpoint{3.156111in}{1.550200in}}%
\pgfpathlineto{\pgfqpoint{3.162487in}{1.537836in}}%
\pgfpathlineto{\pgfqpoint{3.168866in}{1.525388in}}%
\pgfpathlineto{\pgfqpoint{3.175247in}{1.512852in}}%
\pgfpathclose%
\pgfusepath{stroke,fill}%
\end{pgfscope}%
\begin{pgfscope}%
\pgfpathrectangle{\pgfqpoint{0.887500in}{0.275000in}}{\pgfqpoint{4.225000in}{4.225000in}}%
\pgfusepath{clip}%
\pgfsetbuttcap%
\pgfsetroundjoin%
\definecolor{currentfill}{rgb}{0.273006,0.204520,0.501721}%
\pgfsetfillcolor{currentfill}%
\pgfsetfillopacity{0.700000}%
\pgfsetlinewidth{0.501875pt}%
\definecolor{currentstroke}{rgb}{1.000000,1.000000,1.000000}%
\pgfsetstrokecolor{currentstroke}%
\pgfsetstrokeopacity{0.500000}%
\pgfsetdash{}{0pt}%
\pgfpathmoveto{\pgfqpoint{3.271354in}{1.457768in}}%
\pgfpathlineto{\pgfqpoint{3.282887in}{1.461882in}}%
\pgfpathlineto{\pgfqpoint{3.294415in}{1.465906in}}%
\pgfpathlineto{\pgfqpoint{3.305937in}{1.469840in}}%
\pgfpathlineto{\pgfqpoint{3.317452in}{1.473686in}}%
\pgfpathlineto{\pgfqpoint{3.328961in}{1.477450in}}%
\pgfpathlineto{\pgfqpoint{3.322551in}{1.490438in}}%
\pgfpathlineto{\pgfqpoint{3.316143in}{1.503210in}}%
\pgfpathlineto{\pgfqpoint{3.309736in}{1.515804in}}%
\pgfpathlineto{\pgfqpoint{3.303332in}{1.528256in}}%
\pgfpathlineto{\pgfqpoint{3.296930in}{1.540605in}}%
\pgfpathlineto{\pgfqpoint{3.285426in}{1.536642in}}%
\pgfpathlineto{\pgfqpoint{3.273916in}{1.532653in}}%
\pgfpathlineto{\pgfqpoint{3.262400in}{1.528646in}}%
\pgfpathlineto{\pgfqpoint{3.250878in}{1.524630in}}%
\pgfpathlineto{\pgfqpoint{3.239351in}{1.520602in}}%
\pgfpathlineto{\pgfqpoint{3.245747in}{1.508331in}}%
\pgfpathlineto{\pgfqpoint{3.252145in}{1.495959in}}%
\pgfpathlineto{\pgfqpoint{3.258546in}{1.483439in}}%
\pgfpathlineto{\pgfqpoint{3.264949in}{1.470724in}}%
\pgfpathclose%
\pgfusepath{stroke,fill}%
\end{pgfscope}%
\begin{pgfscope}%
\pgfpathrectangle{\pgfqpoint{0.887500in}{0.275000in}}{\pgfqpoint{4.225000in}{4.225000in}}%
\pgfusepath{clip}%
\pgfsetbuttcap%
\pgfsetroundjoin%
\definecolor{currentfill}{rgb}{0.210503,0.363727,0.552206}%
\pgfsetfillcolor{currentfill}%
\pgfsetfillopacity{0.700000}%
\pgfsetlinewidth{0.501875pt}%
\definecolor{currentstroke}{rgb}{1.000000,1.000000,1.000000}%
\pgfsetstrokecolor{currentstroke}%
\pgfsetstrokeopacity{0.500000}%
\pgfsetdash{}{0pt}%
\pgfpathmoveto{\pgfqpoint{2.495523in}{1.745625in}}%
\pgfpathlineto{\pgfqpoint{2.507247in}{1.749640in}}%
\pgfpathlineto{\pgfqpoint{2.518964in}{1.753656in}}%
\pgfpathlineto{\pgfqpoint{2.530676in}{1.757672in}}%
\pgfpathlineto{\pgfqpoint{2.542383in}{1.761685in}}%
\pgfpathlineto{\pgfqpoint{2.554083in}{1.765692in}}%
\pgfpathlineto{\pgfqpoint{2.547881in}{1.776433in}}%
\pgfpathlineto{\pgfqpoint{2.541683in}{1.787126in}}%
\pgfpathlineto{\pgfqpoint{2.535489in}{1.797771in}}%
\pgfpathlineto{\pgfqpoint{2.529299in}{1.808369in}}%
\pgfpathlineto{\pgfqpoint{2.523113in}{1.818919in}}%
\pgfpathlineto{\pgfqpoint{2.511422in}{1.814954in}}%
\pgfpathlineto{\pgfqpoint{2.499725in}{1.810982in}}%
\pgfpathlineto{\pgfqpoint{2.488022in}{1.807005in}}%
\pgfpathlineto{\pgfqpoint{2.476314in}{1.803027in}}%
\pgfpathlineto{\pgfqpoint{2.464600in}{1.799050in}}%
\pgfpathlineto{\pgfqpoint{2.470776in}{1.788458in}}%
\pgfpathlineto{\pgfqpoint{2.476956in}{1.777819in}}%
\pgfpathlineto{\pgfqpoint{2.483141in}{1.767134in}}%
\pgfpathlineto{\pgfqpoint{2.489330in}{1.756403in}}%
\pgfpathclose%
\pgfusepath{stroke,fill}%
\end{pgfscope}%
\begin{pgfscope}%
\pgfpathrectangle{\pgfqpoint{0.887500in}{0.275000in}}{\pgfqpoint{4.225000in}{4.225000in}}%
\pgfusepath{clip}%
\pgfsetbuttcap%
\pgfsetroundjoin%
\definecolor{currentfill}{rgb}{0.185556,0.418570,0.556753}%
\pgfsetfillcolor{currentfill}%
\pgfsetfillopacity{0.700000}%
\pgfsetlinewidth{0.501875pt}%
\definecolor{currentstroke}{rgb}{1.000000,1.000000,1.000000}%
\pgfsetstrokecolor{currentstroke}%
\pgfsetstrokeopacity{0.500000}%
\pgfsetdash{}{0pt}%
\pgfpathmoveto{\pgfqpoint{2.078499in}{1.855698in}}%
\pgfpathlineto{\pgfqpoint{2.090326in}{1.859619in}}%
\pgfpathlineto{\pgfqpoint{2.102147in}{1.863538in}}%
\pgfpathlineto{\pgfqpoint{2.113963in}{1.867458in}}%
\pgfpathlineto{\pgfqpoint{2.125772in}{1.871378in}}%
\pgfpathlineto{\pgfqpoint{2.137576in}{1.875298in}}%
\pgfpathlineto{\pgfqpoint{2.131508in}{1.885379in}}%
\pgfpathlineto{\pgfqpoint{2.125445in}{1.895419in}}%
\pgfpathlineto{\pgfqpoint{2.119387in}{1.905419in}}%
\pgfpathlineto{\pgfqpoint{2.113333in}{1.915378in}}%
\pgfpathlineto{\pgfqpoint{2.107284in}{1.925297in}}%
\pgfpathlineto{\pgfqpoint{2.095490in}{1.921420in}}%
\pgfpathlineto{\pgfqpoint{2.083691in}{1.917544in}}%
\pgfpathlineto{\pgfqpoint{2.071886in}{1.913669in}}%
\pgfpathlineto{\pgfqpoint{2.060075in}{1.909794in}}%
\pgfpathlineto{\pgfqpoint{2.048258in}{1.905918in}}%
\pgfpathlineto{\pgfqpoint{2.054297in}{1.895953in}}%
\pgfpathlineto{\pgfqpoint{2.060341in}{1.885949in}}%
\pgfpathlineto{\pgfqpoint{2.066389in}{1.875905in}}%
\pgfpathlineto{\pgfqpoint{2.072442in}{1.865822in}}%
\pgfpathclose%
\pgfusepath{stroke,fill}%
\end{pgfscope}%
\begin{pgfscope}%
\pgfpathrectangle{\pgfqpoint{0.887500in}{0.275000in}}{\pgfqpoint{4.225000in}{4.225000in}}%
\pgfusepath{clip}%
\pgfsetbuttcap%
\pgfsetroundjoin%
\definecolor{currentfill}{rgb}{0.273809,0.031497,0.358853}%
\pgfsetfillcolor{currentfill}%
\pgfsetfillopacity{0.700000}%
\pgfsetlinewidth{0.501875pt}%
\definecolor{currentstroke}{rgb}{1.000000,1.000000,1.000000}%
\pgfsetstrokecolor{currentstroke}%
\pgfsetstrokeopacity{0.500000}%
\pgfsetdash{}{0pt}%
\pgfpathmoveto{\pgfqpoint{3.662010in}{1.174937in}}%
\pgfpathlineto{\pgfqpoint{3.673475in}{1.180655in}}%
\pgfpathlineto{\pgfqpoint{3.684939in}{1.186626in}}%
\pgfpathlineto{\pgfqpoint{3.696403in}{1.192802in}}%
\pgfpathlineto{\pgfqpoint{3.707865in}{1.199142in}}%
\pgfpathlineto{\pgfqpoint{3.719325in}{1.205626in}}%
\pgfpathlineto{\pgfqpoint{3.712827in}{1.218975in}}%
\pgfpathlineto{\pgfqpoint{3.706342in}{1.232914in}}%
\pgfpathlineto{\pgfqpoint{3.699869in}{1.247376in}}%
\pgfpathlineto{\pgfqpoint{3.693406in}{1.262289in}}%
\pgfpathlineto{\pgfqpoint{3.686952in}{1.277584in}}%
\pgfpathlineto{\pgfqpoint{3.675549in}{1.274335in}}%
\pgfpathlineto{\pgfqpoint{3.664137in}{1.270885in}}%
\pgfpathlineto{\pgfqpoint{3.652715in}{1.267177in}}%
\pgfpathlineto{\pgfqpoint{3.641284in}{1.263236in}}%
\pgfpathlineto{\pgfqpoint{3.629844in}{1.259113in}}%
\pgfpathlineto{\pgfqpoint{3.636284in}{1.242950in}}%
\pgfpathlineto{\pgfqpoint{3.642721in}{1.226463in}}%
\pgfpathlineto{\pgfqpoint{3.649155in}{1.209641in}}%
\pgfpathlineto{\pgfqpoint{3.655585in}{1.192470in}}%
\pgfpathclose%
\pgfusepath{stroke,fill}%
\end{pgfscope}%
\begin{pgfscope}%
\pgfpathrectangle{\pgfqpoint{0.887500in}{0.275000in}}{\pgfqpoint{4.225000in}{4.225000in}}%
\pgfusepath{clip}%
\pgfsetbuttcap%
\pgfsetroundjoin%
\definecolor{currentfill}{rgb}{0.218130,0.347432,0.550038}%
\pgfsetfillcolor{currentfill}%
\pgfsetfillopacity{0.700000}%
\pgfsetlinewidth{0.501875pt}%
\definecolor{currentstroke}{rgb}{1.000000,1.000000,1.000000}%
\pgfsetstrokecolor{currentstroke}%
\pgfsetstrokeopacity{0.500000}%
\pgfsetdash{}{0pt}%
\pgfpathmoveto{\pgfqpoint{2.585157in}{1.711261in}}%
\pgfpathlineto{\pgfqpoint{2.596861in}{1.715296in}}%
\pgfpathlineto{\pgfqpoint{2.608560in}{1.719323in}}%
\pgfpathlineto{\pgfqpoint{2.620253in}{1.723340in}}%
\pgfpathlineto{\pgfqpoint{2.631941in}{1.727346in}}%
\pgfpathlineto{\pgfqpoint{2.643623in}{1.731339in}}%
\pgfpathlineto{\pgfqpoint{2.637391in}{1.742285in}}%
\pgfpathlineto{\pgfqpoint{2.631163in}{1.753182in}}%
\pgfpathlineto{\pgfqpoint{2.624940in}{1.764029in}}%
\pgfpathlineto{\pgfqpoint{2.618720in}{1.774825in}}%
\pgfpathlineto{\pgfqpoint{2.612505in}{1.785570in}}%
\pgfpathlineto{\pgfqpoint{2.600831in}{1.781621in}}%
\pgfpathlineto{\pgfqpoint{2.589153in}{1.777657in}}%
\pgfpathlineto{\pgfqpoint{2.577468in}{1.773680in}}%
\pgfpathlineto{\pgfqpoint{2.565779in}{1.769691in}}%
\pgfpathlineto{\pgfqpoint{2.554083in}{1.765692in}}%
\pgfpathlineto{\pgfqpoint{2.560290in}{1.754903in}}%
\pgfpathlineto{\pgfqpoint{2.566501in}{1.744065in}}%
\pgfpathlineto{\pgfqpoint{2.572715in}{1.733178in}}%
\pgfpathlineto{\pgfqpoint{2.578934in}{1.722244in}}%
\pgfpathclose%
\pgfusepath{stroke,fill}%
\end{pgfscope}%
\begin{pgfscope}%
\pgfpathrectangle{\pgfqpoint{0.887500in}{0.275000in}}{\pgfqpoint{4.225000in}{4.225000in}}%
\pgfusepath{clip}%
\pgfsetbuttcap%
\pgfsetroundjoin%
\definecolor{currentfill}{rgb}{0.192357,0.403199,0.555836}%
\pgfsetfillcolor{currentfill}%
\pgfsetfillopacity{0.700000}%
\pgfsetlinewidth{0.501875pt}%
\definecolor{currentstroke}{rgb}{1.000000,1.000000,1.000000}%
\pgfsetstrokecolor{currentstroke}%
\pgfsetstrokeopacity{0.500000}%
\pgfsetdash{}{0pt}%
\pgfpathmoveto{\pgfqpoint{2.167981in}{1.824273in}}%
\pgfpathlineto{\pgfqpoint{2.179789in}{1.828233in}}%
\pgfpathlineto{\pgfqpoint{2.191591in}{1.832192in}}%
\pgfpathlineto{\pgfqpoint{2.203388in}{1.836148in}}%
\pgfpathlineto{\pgfqpoint{2.215179in}{1.840100in}}%
\pgfpathlineto{\pgfqpoint{2.226964in}{1.844046in}}%
\pgfpathlineto{\pgfqpoint{2.220864in}{1.854293in}}%
\pgfpathlineto{\pgfqpoint{2.214769in}{1.864497in}}%
\pgfpathlineto{\pgfqpoint{2.208678in}{1.874659in}}%
\pgfpathlineto{\pgfqpoint{2.202591in}{1.884780in}}%
\pgfpathlineto{\pgfqpoint{2.196509in}{1.894858in}}%
\pgfpathlineto{\pgfqpoint{2.184734in}{1.890957in}}%
\pgfpathlineto{\pgfqpoint{2.172953in}{1.887049in}}%
\pgfpathlineto{\pgfqpoint{2.161166in}{1.883136in}}%
\pgfpathlineto{\pgfqpoint{2.149374in}{1.879218in}}%
\pgfpathlineto{\pgfqpoint{2.137576in}{1.875298in}}%
\pgfpathlineto{\pgfqpoint{2.143648in}{1.865176in}}%
\pgfpathlineto{\pgfqpoint{2.149724in}{1.855012in}}%
\pgfpathlineto{\pgfqpoint{2.155806in}{1.844807in}}%
\pgfpathlineto{\pgfqpoint{2.161891in}{1.834561in}}%
\pgfpathclose%
\pgfusepath{stroke,fill}%
\end{pgfscope}%
\begin{pgfscope}%
\pgfpathrectangle{\pgfqpoint{0.887500in}{0.275000in}}{\pgfqpoint{4.225000in}{4.225000in}}%
\pgfusepath{clip}%
\pgfsetbuttcap%
\pgfsetroundjoin%
\definecolor{currentfill}{rgb}{0.225863,0.330805,0.547314}%
\pgfsetfillcolor{currentfill}%
\pgfsetfillopacity{0.700000}%
\pgfsetlinewidth{0.501875pt}%
\definecolor{currentstroke}{rgb}{1.000000,1.000000,1.000000}%
\pgfsetstrokecolor{currentstroke}%
\pgfsetstrokeopacity{0.500000}%
\pgfsetdash{}{0pt}%
\pgfpathmoveto{\pgfqpoint{2.674841in}{1.675897in}}%
\pgfpathlineto{\pgfqpoint{2.686526in}{1.679913in}}%
\pgfpathlineto{\pgfqpoint{2.698206in}{1.683916in}}%
\pgfpathlineto{\pgfqpoint{2.709880in}{1.687903in}}%
\pgfpathlineto{\pgfqpoint{2.721549in}{1.691874in}}%
\pgfpathlineto{\pgfqpoint{2.733212in}{1.695832in}}%
\pgfpathlineto{\pgfqpoint{2.726952in}{1.706973in}}%
\pgfpathlineto{\pgfqpoint{2.720696in}{1.718068in}}%
\pgfpathlineto{\pgfqpoint{2.714444in}{1.729116in}}%
\pgfpathlineto{\pgfqpoint{2.708195in}{1.740116in}}%
\pgfpathlineto{\pgfqpoint{2.701951in}{1.751066in}}%
\pgfpathlineto{\pgfqpoint{2.690296in}{1.747154in}}%
\pgfpathlineto{\pgfqpoint{2.678636in}{1.743226in}}%
\pgfpathlineto{\pgfqpoint{2.666971in}{1.739281in}}%
\pgfpathlineto{\pgfqpoint{2.655299in}{1.735318in}}%
\pgfpathlineto{\pgfqpoint{2.643623in}{1.731339in}}%
\pgfpathlineto{\pgfqpoint{2.649859in}{1.720344in}}%
\pgfpathlineto{\pgfqpoint{2.656098in}{1.709301in}}%
\pgfpathlineto{\pgfqpoint{2.662342in}{1.698212in}}%
\pgfpathlineto{\pgfqpoint{2.668589in}{1.687077in}}%
\pgfpathclose%
\pgfusepath{stroke,fill}%
\end{pgfscope}%
\begin{pgfscope}%
\pgfpathrectangle{\pgfqpoint{0.887500in}{0.275000in}}{\pgfqpoint{4.225000in}{4.225000in}}%
\pgfusepath{clip}%
\pgfsetbuttcap%
\pgfsetroundjoin%
\definecolor{currentfill}{rgb}{0.280894,0.078907,0.402329}%
\pgfsetfillcolor{currentfill}%
\pgfsetfillopacity{0.700000}%
\pgfsetlinewidth{0.501875pt}%
\definecolor{currentstroke}{rgb}{1.000000,1.000000,1.000000}%
\pgfsetstrokecolor{currentstroke}%
\pgfsetstrokeopacity{0.500000}%
\pgfsetdash{}{0pt}%
\pgfpathmoveto{\pgfqpoint{3.572552in}{1.237510in}}%
\pgfpathlineto{\pgfqpoint{3.584021in}{1.241822in}}%
\pgfpathlineto{\pgfqpoint{3.595485in}{1.246163in}}%
\pgfpathlineto{\pgfqpoint{3.606944in}{1.250526in}}%
\pgfpathlineto{\pgfqpoint{3.618398in}{1.254860in}}%
\pgfpathlineto{\pgfqpoint{3.629844in}{1.259113in}}%
\pgfpathlineto{\pgfqpoint{3.623402in}{1.274967in}}%
\pgfpathlineto{\pgfqpoint{3.616958in}{1.290524in}}%
\pgfpathlineto{\pgfqpoint{3.610511in}{1.305798in}}%
\pgfpathlineto{\pgfqpoint{3.604064in}{1.320802in}}%
\pgfpathlineto{\pgfqpoint{3.597615in}{1.335548in}}%
\pgfpathlineto{\pgfqpoint{3.586166in}{1.331037in}}%
\pgfpathlineto{\pgfqpoint{3.574713in}{1.326514in}}%
\pgfpathlineto{\pgfqpoint{3.563254in}{1.321992in}}%
\pgfpathlineto{\pgfqpoint{3.551790in}{1.317489in}}%
\pgfpathlineto{\pgfqpoint{3.540320in}{1.312991in}}%
\pgfpathlineto{\pgfqpoint{3.546770in}{1.298547in}}%
\pgfpathlineto{\pgfqpoint{3.553219in}{1.283805in}}%
\pgfpathlineto{\pgfqpoint{3.559666in}{1.268738in}}%
\pgfpathlineto{\pgfqpoint{3.566110in}{1.253316in}}%
\pgfpathclose%
\pgfusepath{stroke,fill}%
\end{pgfscope}%
\begin{pgfscope}%
\pgfpathrectangle{\pgfqpoint{0.887500in}{0.275000in}}{\pgfqpoint{4.225000in}{4.225000in}}%
\pgfusepath{clip}%
\pgfsetbuttcap%
\pgfsetroundjoin%
\definecolor{currentfill}{rgb}{0.233603,0.313828,0.543914}%
\pgfsetfillcolor{currentfill}%
\pgfsetfillopacity{0.700000}%
\pgfsetlinewidth{0.501875pt}%
\definecolor{currentstroke}{rgb}{1.000000,1.000000,1.000000}%
\pgfsetstrokecolor{currentstroke}%
\pgfsetstrokeopacity{0.500000}%
\pgfsetdash{}{0pt}%
\pgfpathmoveto{\pgfqpoint{2.764569in}{1.639448in}}%
\pgfpathlineto{\pgfqpoint{2.776235in}{1.643431in}}%
\pgfpathlineto{\pgfqpoint{2.787896in}{1.647407in}}%
\pgfpathlineto{\pgfqpoint{2.799550in}{1.651381in}}%
\pgfpathlineto{\pgfqpoint{2.811200in}{1.655354in}}%
\pgfpathlineto{\pgfqpoint{2.822843in}{1.659331in}}%
\pgfpathlineto{\pgfqpoint{2.816556in}{1.670668in}}%
\pgfpathlineto{\pgfqpoint{2.810272in}{1.681954in}}%
\pgfpathlineto{\pgfqpoint{2.803992in}{1.693192in}}%
\pgfpathlineto{\pgfqpoint{2.797716in}{1.704383in}}%
\pgfpathlineto{\pgfqpoint{2.791444in}{1.715527in}}%
\pgfpathlineto{\pgfqpoint{2.779809in}{1.711589in}}%
\pgfpathlineto{\pgfqpoint{2.768168in}{1.707654in}}%
\pgfpathlineto{\pgfqpoint{2.756522in}{1.703719in}}%
\pgfpathlineto{\pgfqpoint{2.744870in}{1.699779in}}%
\pgfpathlineto{\pgfqpoint{2.733212in}{1.695832in}}%
\pgfpathlineto{\pgfqpoint{2.739476in}{1.684645in}}%
\pgfpathlineto{\pgfqpoint{2.745744in}{1.673413in}}%
\pgfpathlineto{\pgfqpoint{2.752015in}{1.662137in}}%
\pgfpathlineto{\pgfqpoint{2.758290in}{1.650817in}}%
\pgfpathclose%
\pgfusepath{stroke,fill}%
\end{pgfscope}%
\begin{pgfscope}%
\pgfpathrectangle{\pgfqpoint{0.887500in}{0.275000in}}{\pgfqpoint{4.225000in}{4.225000in}}%
\pgfusepath{clip}%
\pgfsetbuttcap%
\pgfsetroundjoin%
\definecolor{currentfill}{rgb}{0.197636,0.391528,0.554969}%
\pgfsetfillcolor{currentfill}%
\pgfsetfillopacity{0.700000}%
\pgfsetlinewidth{0.501875pt}%
\definecolor{currentstroke}{rgb}{1.000000,1.000000,1.000000}%
\pgfsetstrokecolor{currentstroke}%
\pgfsetstrokeopacity{0.500000}%
\pgfsetdash{}{0pt}%
\pgfpathmoveto{\pgfqpoint{2.257530in}{1.792188in}}%
\pgfpathlineto{\pgfqpoint{2.269320in}{1.796171in}}%
\pgfpathlineto{\pgfqpoint{2.281104in}{1.800148in}}%
\pgfpathlineto{\pgfqpoint{2.292882in}{1.804119in}}%
\pgfpathlineto{\pgfqpoint{2.304654in}{1.808084in}}%
\pgfpathlineto{\pgfqpoint{2.316421in}{1.812042in}}%
\pgfpathlineto{\pgfqpoint{2.310290in}{1.822454in}}%
\pgfpathlineto{\pgfqpoint{2.304162in}{1.832824in}}%
\pgfpathlineto{\pgfqpoint{2.298039in}{1.843150in}}%
\pgfpathlineto{\pgfqpoint{2.291921in}{1.853433in}}%
\pgfpathlineto{\pgfqpoint{2.285807in}{1.863673in}}%
\pgfpathlineto{\pgfqpoint{2.274050in}{1.859763in}}%
\pgfpathlineto{\pgfqpoint{2.262287in}{1.855845in}}%
\pgfpathlineto{\pgfqpoint{2.250518in}{1.851920in}}%
\pgfpathlineto{\pgfqpoint{2.238744in}{1.847987in}}%
\pgfpathlineto{\pgfqpoint{2.226964in}{1.844046in}}%
\pgfpathlineto{\pgfqpoint{2.233069in}{1.833758in}}%
\pgfpathlineto{\pgfqpoint{2.239177in}{1.823428in}}%
\pgfpathlineto{\pgfqpoint{2.245291in}{1.813056in}}%
\pgfpathlineto{\pgfqpoint{2.251408in}{1.802643in}}%
\pgfpathclose%
\pgfusepath{stroke,fill}%
\end{pgfscope}%
\begin{pgfscope}%
\pgfpathrectangle{\pgfqpoint{0.887500in}{0.275000in}}{\pgfqpoint{4.225000in}{4.225000in}}%
\pgfusepath{clip}%
\pgfsetbuttcap%
\pgfsetroundjoin%
\definecolor{currentfill}{rgb}{0.283197,0.115680,0.436115}%
\pgfsetfillcolor{currentfill}%
\pgfsetfillopacity{0.700000}%
\pgfsetlinewidth{0.501875pt}%
\definecolor{currentstroke}{rgb}{1.000000,1.000000,1.000000}%
\pgfsetstrokecolor{currentstroke}%
\pgfsetstrokeopacity{0.500000}%
\pgfsetdash{}{0pt}%
\pgfpathmoveto{\pgfqpoint{3.482854in}{1.286791in}}%
\pgfpathlineto{\pgfqpoint{3.494367in}{1.292913in}}%
\pgfpathlineto{\pgfqpoint{3.505869in}{1.298478in}}%
\pgfpathlineto{\pgfqpoint{3.517362in}{1.303601in}}%
\pgfpathlineto{\pgfqpoint{3.528845in}{1.308400in}}%
\pgfpathlineto{\pgfqpoint{3.540320in}{1.312991in}}%
\pgfpathlineto{\pgfqpoint{3.533870in}{1.327166in}}%
\pgfpathlineto{\pgfqpoint{3.527419in}{1.341100in}}%
\pgfpathlineto{\pgfqpoint{3.520969in}{1.354824in}}%
\pgfpathlineto{\pgfqpoint{3.514519in}{1.368364in}}%
\pgfpathlineto{\pgfqpoint{3.508070in}{1.381750in}}%
\pgfpathlineto{\pgfqpoint{3.496602in}{1.377385in}}%
\pgfpathlineto{\pgfqpoint{3.485128in}{1.373015in}}%
\pgfpathlineto{\pgfqpoint{3.473648in}{1.368548in}}%
\pgfpathlineto{\pgfqpoint{3.462161in}{1.363898in}}%
\pgfpathlineto{\pgfqpoint{3.450665in}{1.358975in}}%
\pgfpathlineto{\pgfqpoint{3.457100in}{1.344781in}}%
\pgfpathlineto{\pgfqpoint{3.463537in}{1.330469in}}%
\pgfpathlineto{\pgfqpoint{3.469975in}{1.316036in}}%
\pgfpathlineto{\pgfqpoint{3.476414in}{1.301478in}}%
\pgfpathclose%
\pgfusepath{stroke,fill}%
\end{pgfscope}%
\begin{pgfscope}%
\pgfpathrectangle{\pgfqpoint{0.887500in}{0.275000in}}{\pgfqpoint{4.225000in}{4.225000in}}%
\pgfusepath{clip}%
\pgfsetbuttcap%
\pgfsetroundjoin%
\definecolor{currentfill}{rgb}{0.241237,0.296485,0.539709}%
\pgfsetfillcolor{currentfill}%
\pgfsetfillopacity{0.700000}%
\pgfsetlinewidth{0.501875pt}%
\definecolor{currentstroke}{rgb}{1.000000,1.000000,1.000000}%
\pgfsetstrokecolor{currentstroke}%
\pgfsetstrokeopacity{0.500000}%
\pgfsetdash{}{0pt}%
\pgfpathmoveto{\pgfqpoint{2.854335in}{1.601770in}}%
\pgfpathlineto{\pgfqpoint{2.865981in}{1.605780in}}%
\pgfpathlineto{\pgfqpoint{2.877621in}{1.609794in}}%
\pgfpathlineto{\pgfqpoint{2.889256in}{1.613810in}}%
\pgfpathlineto{\pgfqpoint{2.900885in}{1.617826in}}%
\pgfpathlineto{\pgfqpoint{2.912508in}{1.621840in}}%
\pgfpathlineto{\pgfqpoint{2.906194in}{1.633450in}}%
\pgfpathlineto{\pgfqpoint{2.899884in}{1.644996in}}%
\pgfpathlineto{\pgfqpoint{2.893577in}{1.656480in}}%
\pgfpathlineto{\pgfqpoint{2.887274in}{1.667902in}}%
\pgfpathlineto{\pgfqpoint{2.880975in}{1.679267in}}%
\pgfpathlineto{\pgfqpoint{2.869360in}{1.675279in}}%
\pgfpathlineto{\pgfqpoint{2.857739in}{1.671289in}}%
\pgfpathlineto{\pgfqpoint{2.846113in}{1.667300in}}%
\pgfpathlineto{\pgfqpoint{2.834481in}{1.663314in}}%
\pgfpathlineto{\pgfqpoint{2.822843in}{1.659331in}}%
\pgfpathlineto{\pgfqpoint{2.829134in}{1.647940in}}%
\pgfpathlineto{\pgfqpoint{2.835429in}{1.636491in}}%
\pgfpathlineto{\pgfqpoint{2.841727in}{1.624982in}}%
\pgfpathlineto{\pgfqpoint{2.848030in}{1.613409in}}%
\pgfpathclose%
\pgfusepath{stroke,fill}%
\end{pgfscope}%
\begin{pgfscope}%
\pgfpathrectangle{\pgfqpoint{0.887500in}{0.275000in}}{\pgfqpoint{4.225000in}{4.225000in}}%
\pgfusepath{clip}%
\pgfsetbuttcap%
\pgfsetroundjoin%
\definecolor{currentfill}{rgb}{0.282623,0.140926,0.457517}%
\pgfsetfillcolor{currentfill}%
\pgfsetfillopacity{0.700000}%
\pgfsetlinewidth{0.501875pt}%
\definecolor{currentstroke}{rgb}{1.000000,1.000000,1.000000}%
\pgfsetstrokecolor{currentstroke}%
\pgfsetstrokeopacity{0.500000}%
\pgfsetdash{}{0pt}%
\pgfpathmoveto{\pgfqpoint{3.393078in}{1.330364in}}%
\pgfpathlineto{\pgfqpoint{3.404606in}{1.336196in}}%
\pgfpathlineto{\pgfqpoint{3.416130in}{1.342140in}}%
\pgfpathlineto{\pgfqpoint{3.427649in}{1.348028in}}%
\pgfpathlineto{\pgfqpoint{3.439161in}{1.353689in}}%
\pgfpathlineto{\pgfqpoint{3.450665in}{1.358975in}}%
\pgfpathlineto{\pgfqpoint{3.444232in}{1.373054in}}%
\pgfpathlineto{\pgfqpoint{3.437800in}{1.387023in}}%
\pgfpathlineto{\pgfqpoint{3.431370in}{1.400886in}}%
\pgfpathlineto{\pgfqpoint{3.424941in}{1.414646in}}%
\pgfpathlineto{\pgfqpoint{3.418515in}{1.428306in}}%
\pgfpathlineto{\pgfqpoint{3.407029in}{1.424454in}}%
\pgfpathlineto{\pgfqpoint{3.395536in}{1.420461in}}%
\pgfpathlineto{\pgfqpoint{3.384037in}{1.416357in}}%
\pgfpathlineto{\pgfqpoint{3.372532in}{1.412183in}}%
\pgfpathlineto{\pgfqpoint{3.361021in}{1.407978in}}%
\pgfpathlineto{\pgfqpoint{3.367432in}{1.393002in}}%
\pgfpathlineto{\pgfqpoint{3.373844in}{1.377708in}}%
\pgfpathlineto{\pgfqpoint{3.380255in}{1.362141in}}%
\pgfpathlineto{\pgfqpoint{3.386666in}{1.346345in}}%
\pgfpathclose%
\pgfusepath{stroke,fill}%
\end{pgfscope}%
\begin{pgfscope}%
\pgfpathrectangle{\pgfqpoint{0.887500in}{0.275000in}}{\pgfqpoint{4.225000in}{4.225000in}}%
\pgfusepath{clip}%
\pgfsetbuttcap%
\pgfsetroundjoin%
\definecolor{currentfill}{rgb}{0.250425,0.274290,0.533103}%
\pgfsetfillcolor{currentfill}%
\pgfsetfillopacity{0.700000}%
\pgfsetlinewidth{0.501875pt}%
\definecolor{currentstroke}{rgb}{1.000000,1.000000,1.000000}%
\pgfsetstrokecolor{currentstroke}%
\pgfsetstrokeopacity{0.500000}%
\pgfsetdash{}{0pt}%
\pgfpathmoveto{\pgfqpoint{2.944131in}{1.562757in}}%
\pgfpathlineto{\pgfqpoint{2.955756in}{1.566845in}}%
\pgfpathlineto{\pgfqpoint{2.967376in}{1.570949in}}%
\pgfpathlineto{\pgfqpoint{2.978990in}{1.575064in}}%
\pgfpathlineto{\pgfqpoint{2.990598in}{1.579183in}}%
\pgfpathlineto{\pgfqpoint{3.002201in}{1.583296in}}%
\pgfpathlineto{\pgfqpoint{2.995862in}{1.595096in}}%
\pgfpathlineto{\pgfqpoint{2.989526in}{1.606843in}}%
\pgfpathlineto{\pgfqpoint{2.983194in}{1.618545in}}%
\pgfpathlineto{\pgfqpoint{2.976865in}{1.630202in}}%
\pgfpathlineto{\pgfqpoint{2.970540in}{1.641813in}}%
\pgfpathlineto{\pgfqpoint{2.958945in}{1.637834in}}%
\pgfpathlineto{\pgfqpoint{2.947344in}{1.633847in}}%
\pgfpathlineto{\pgfqpoint{2.935738in}{1.629852in}}%
\pgfpathlineto{\pgfqpoint{2.924126in}{1.625849in}}%
\pgfpathlineto{\pgfqpoint{2.912508in}{1.621840in}}%
\pgfpathlineto{\pgfqpoint{2.918826in}{1.610163in}}%
\pgfpathlineto{\pgfqpoint{2.925147in}{1.598418in}}%
\pgfpathlineto{\pgfqpoint{2.931471in}{1.586602in}}%
\pgfpathlineto{\pgfqpoint{2.937799in}{1.574714in}}%
\pgfpathclose%
\pgfusepath{stroke,fill}%
\end{pgfscope}%
\begin{pgfscope}%
\pgfpathrectangle{\pgfqpoint{0.887500in}{0.275000in}}{\pgfqpoint{4.225000in}{4.225000in}}%
\pgfusepath{clip}%
\pgfsetbuttcap%
\pgfsetroundjoin%
\definecolor{currentfill}{rgb}{0.204903,0.375746,0.553533}%
\pgfsetfillcolor{currentfill}%
\pgfsetfillopacity{0.700000}%
\pgfsetlinewidth{0.501875pt}%
\definecolor{currentstroke}{rgb}{1.000000,1.000000,1.000000}%
\pgfsetstrokecolor{currentstroke}%
\pgfsetstrokeopacity{0.500000}%
\pgfsetdash{}{0pt}%
\pgfpathmoveto{\pgfqpoint{2.347144in}{1.759332in}}%
\pgfpathlineto{\pgfqpoint{2.358915in}{1.763323in}}%
\pgfpathlineto{\pgfqpoint{2.370681in}{1.767307in}}%
\pgfpathlineto{\pgfqpoint{2.382440in}{1.771284in}}%
\pgfpathlineto{\pgfqpoint{2.394194in}{1.775254in}}%
\pgfpathlineto{\pgfqpoint{2.405943in}{1.779220in}}%
\pgfpathlineto{\pgfqpoint{2.399780in}{1.789809in}}%
\pgfpathlineto{\pgfqpoint{2.393622in}{1.800353in}}%
\pgfpathlineto{\pgfqpoint{2.387468in}{1.810853in}}%
\pgfpathlineto{\pgfqpoint{2.381318in}{1.821308in}}%
\pgfpathlineto{\pgfqpoint{2.375172in}{1.831718in}}%
\pgfpathlineto{\pgfqpoint{2.363433in}{1.827796in}}%
\pgfpathlineto{\pgfqpoint{2.351689in}{1.823869in}}%
\pgfpathlineto{\pgfqpoint{2.339938in}{1.819935in}}%
\pgfpathlineto{\pgfqpoint{2.328183in}{1.815992in}}%
\pgfpathlineto{\pgfqpoint{2.316421in}{1.812042in}}%
\pgfpathlineto{\pgfqpoint{2.322557in}{1.801586in}}%
\pgfpathlineto{\pgfqpoint{2.328698in}{1.791087in}}%
\pgfpathlineto{\pgfqpoint{2.334842in}{1.780545in}}%
\pgfpathlineto{\pgfqpoint{2.340991in}{1.769960in}}%
\pgfpathclose%
\pgfusepath{stroke,fill}%
\end{pgfscope}%
\begin{pgfscope}%
\pgfpathrectangle{\pgfqpoint{0.887500in}{0.275000in}}{\pgfqpoint{4.225000in}{4.225000in}}%
\pgfusepath{clip}%
\pgfsetbuttcap%
\pgfsetroundjoin%
\definecolor{currentfill}{rgb}{0.180629,0.429975,0.557282}%
\pgfsetfillcolor{currentfill}%
\pgfsetfillopacity{0.700000}%
\pgfsetlinewidth{0.501875pt}%
\definecolor{currentstroke}{rgb}{1.000000,1.000000,1.000000}%
\pgfsetstrokecolor{currentstroke}%
\pgfsetstrokeopacity{0.500000}%
\pgfsetdash{}{0pt}%
\pgfpathmoveto{\pgfqpoint{1.929778in}{1.866903in}}%
\pgfpathlineto{\pgfqpoint{1.941651in}{1.870839in}}%
\pgfpathlineto{\pgfqpoint{1.953519in}{1.874766in}}%
\pgfpathlineto{\pgfqpoint{1.965381in}{1.878682in}}%
\pgfpathlineto{\pgfqpoint{1.977237in}{1.882591in}}%
\pgfpathlineto{\pgfqpoint{1.989088in}{1.886492in}}%
\pgfpathlineto{\pgfqpoint{1.983064in}{1.896465in}}%
\pgfpathlineto{\pgfqpoint{1.977044in}{1.906400in}}%
\pgfpathlineto{\pgfqpoint{1.971029in}{1.916299in}}%
\pgfpathlineto{\pgfqpoint{1.965018in}{1.926161in}}%
\pgfpathlineto{\pgfqpoint{1.959012in}{1.935987in}}%
\pgfpathlineto{\pgfqpoint{1.947172in}{1.932133in}}%
\pgfpathlineto{\pgfqpoint{1.935325in}{1.928271in}}%
\pgfpathlineto{\pgfqpoint{1.923473in}{1.924400in}}%
\pgfpathlineto{\pgfqpoint{1.911616in}{1.920518in}}%
\pgfpathlineto{\pgfqpoint{1.899753in}{1.916625in}}%
\pgfpathlineto{\pgfqpoint{1.905749in}{1.906752in}}%
\pgfpathlineto{\pgfqpoint{1.911749in}{1.896844in}}%
\pgfpathlineto{\pgfqpoint{1.917754in}{1.886900in}}%
\pgfpathlineto{\pgfqpoint{1.923764in}{1.876920in}}%
\pgfpathclose%
\pgfusepath{stroke,fill}%
\end{pgfscope}%
\begin{pgfscope}%
\pgfpathrectangle{\pgfqpoint{0.887500in}{0.275000in}}{\pgfqpoint{4.225000in}{4.225000in}}%
\pgfusepath{clip}%
\pgfsetbuttcap%
\pgfsetroundjoin%
\definecolor{currentfill}{rgb}{0.278826,0.175490,0.483397}%
\pgfsetfillcolor{currentfill}%
\pgfsetfillopacity{0.700000}%
\pgfsetlinewidth{0.501875pt}%
\definecolor{currentstroke}{rgb}{1.000000,1.000000,1.000000}%
\pgfsetstrokecolor{currentstroke}%
\pgfsetstrokeopacity{0.500000}%
\pgfsetdash{}{0pt}%
\pgfpathmoveto{\pgfqpoint{3.303389in}{1.387762in}}%
\pgfpathlineto{\pgfqpoint{3.314925in}{1.391616in}}%
\pgfpathlineto{\pgfqpoint{3.326457in}{1.395570in}}%
\pgfpathlineto{\pgfqpoint{3.337983in}{1.399632in}}%
\pgfpathlineto{\pgfqpoint{3.349505in}{1.403782in}}%
\pgfpathlineto{\pgfqpoint{3.361021in}{1.407978in}}%
\pgfpathlineto{\pgfqpoint{3.354609in}{1.422593in}}%
\pgfpathlineto{\pgfqpoint{3.348197in}{1.436817in}}%
\pgfpathlineto{\pgfqpoint{3.341784in}{1.450677in}}%
\pgfpathlineto{\pgfqpoint{3.335372in}{1.464208in}}%
\pgfpathlineto{\pgfqpoint{3.328961in}{1.477450in}}%
\pgfpathlineto{\pgfqpoint{3.317452in}{1.473686in}}%
\pgfpathlineto{\pgfqpoint{3.305937in}{1.469840in}}%
\pgfpathlineto{\pgfqpoint{3.294415in}{1.465906in}}%
\pgfpathlineto{\pgfqpoint{3.282887in}{1.461882in}}%
\pgfpathlineto{\pgfqpoint{3.271354in}{1.457768in}}%
\pgfpathlineto{\pgfqpoint{3.277760in}{1.444525in}}%
\pgfpathlineto{\pgfqpoint{3.284167in}{1.430947in}}%
\pgfpathlineto{\pgfqpoint{3.290574in}{1.416989in}}%
\pgfpathlineto{\pgfqpoint{3.296982in}{1.402605in}}%
\pgfpathclose%
\pgfusepath{stroke,fill}%
\end{pgfscope}%
\begin{pgfscope}%
\pgfpathrectangle{\pgfqpoint{0.887500in}{0.275000in}}{\pgfqpoint{4.225000in}{4.225000in}}%
\pgfusepath{clip}%
\pgfsetbuttcap%
\pgfsetroundjoin%
\definecolor{currentfill}{rgb}{0.257322,0.256130,0.526563}%
\pgfsetfillcolor{currentfill}%
\pgfsetfillopacity{0.700000}%
\pgfsetlinewidth{0.501875pt}%
\definecolor{currentstroke}{rgb}{1.000000,1.000000,1.000000}%
\pgfsetstrokecolor{currentstroke}%
\pgfsetstrokeopacity{0.500000}%
\pgfsetdash{}{0pt}%
\pgfpathmoveto{\pgfqpoint{3.033947in}{1.523088in}}%
\pgfpathlineto{\pgfqpoint{3.045551in}{1.527291in}}%
\pgfpathlineto{\pgfqpoint{3.057150in}{1.531452in}}%
\pgfpathlineto{\pgfqpoint{3.068744in}{1.535551in}}%
\pgfpathlineto{\pgfqpoint{3.080332in}{1.539571in}}%
\pgfpathlineto{\pgfqpoint{3.091914in}{1.543507in}}%
\pgfpathlineto{\pgfqpoint{3.085551in}{1.555735in}}%
\pgfpathlineto{\pgfqpoint{3.079191in}{1.567824in}}%
\pgfpathlineto{\pgfqpoint{3.072835in}{1.579791in}}%
\pgfpathlineto{\pgfqpoint{3.066481in}{1.591654in}}%
\pgfpathlineto{\pgfqpoint{3.060131in}{1.603433in}}%
\pgfpathlineto{\pgfqpoint{3.048556in}{1.599490in}}%
\pgfpathlineto{\pgfqpoint{3.036976in}{1.595499in}}%
\pgfpathlineto{\pgfqpoint{3.025390in}{1.591463in}}%
\pgfpathlineto{\pgfqpoint{3.013798in}{1.587392in}}%
\pgfpathlineto{\pgfqpoint{3.002201in}{1.583296in}}%
\pgfpathlineto{\pgfqpoint{3.008543in}{1.571430in}}%
\pgfpathlineto{\pgfqpoint{3.014889in}{1.559488in}}%
\pgfpathlineto{\pgfqpoint{3.021238in}{1.547459in}}%
\pgfpathlineto{\pgfqpoint{3.027591in}{1.535329in}}%
\pgfpathclose%
\pgfusepath{stroke,fill}%
\end{pgfscope}%
\begin{pgfscope}%
\pgfpathrectangle{\pgfqpoint{0.887500in}{0.275000in}}{\pgfqpoint{4.225000in}{4.225000in}}%
\pgfusepath{clip}%
\pgfsetbuttcap%
\pgfsetroundjoin%
\definecolor{currentfill}{rgb}{0.266580,0.228262,0.514349}%
\pgfsetfillcolor{currentfill}%
\pgfsetfillopacity{0.700000}%
\pgfsetlinewidth{0.501875pt}%
\definecolor{currentstroke}{rgb}{1.000000,1.000000,1.000000}%
\pgfsetstrokecolor{currentstroke}%
\pgfsetstrokeopacity{0.500000}%
\pgfsetdash{}{0pt}%
\pgfpathmoveto{\pgfqpoint{3.123773in}{1.479653in}}%
\pgfpathlineto{\pgfqpoint{3.135355in}{1.483737in}}%
\pgfpathlineto{\pgfqpoint{3.146933in}{1.487847in}}%
\pgfpathlineto{\pgfqpoint{3.158505in}{1.491973in}}%
\pgfpathlineto{\pgfqpoint{3.170071in}{1.496104in}}%
\pgfpathlineto{\pgfqpoint{3.181632in}{1.500231in}}%
\pgfpathlineto{\pgfqpoint{3.175247in}{1.512852in}}%
\pgfpathlineto{\pgfqpoint{3.168866in}{1.525388in}}%
\pgfpathlineto{\pgfqpoint{3.162487in}{1.537836in}}%
\pgfpathlineto{\pgfqpoint{3.156111in}{1.550200in}}%
\pgfpathlineto{\pgfqpoint{3.149739in}{1.562481in}}%
\pgfpathlineto{\pgfqpoint{3.138185in}{1.558725in}}%
\pgfpathlineto{\pgfqpoint{3.126626in}{1.554966in}}%
\pgfpathlineto{\pgfqpoint{3.115061in}{1.551187in}}%
\pgfpathlineto{\pgfqpoint{3.103490in}{1.547373in}}%
\pgfpathlineto{\pgfqpoint{3.091914in}{1.543507in}}%
\pgfpathlineto{\pgfqpoint{3.098280in}{1.531119in}}%
\pgfpathlineto{\pgfqpoint{3.104649in}{1.518554in}}%
\pgfpathlineto{\pgfqpoint{3.111021in}{1.505794in}}%
\pgfpathlineto{\pgfqpoint{3.117395in}{1.492820in}}%
\pgfpathclose%
\pgfusepath{stroke,fill}%
\end{pgfscope}%
\begin{pgfscope}%
\pgfpathrectangle{\pgfqpoint{0.887500in}{0.275000in}}{\pgfqpoint{4.225000in}{4.225000in}}%
\pgfusepath{clip}%
\pgfsetbuttcap%
\pgfsetroundjoin%
\definecolor{currentfill}{rgb}{0.273006,0.204520,0.501721}%
\pgfsetfillcolor{currentfill}%
\pgfsetfillopacity{0.700000}%
\pgfsetlinewidth{0.501875pt}%
\definecolor{currentstroke}{rgb}{1.000000,1.000000,1.000000}%
\pgfsetstrokecolor{currentstroke}%
\pgfsetstrokeopacity{0.500000}%
\pgfsetdash{}{0pt}%
\pgfpathmoveto{\pgfqpoint{3.213596in}{1.435897in}}%
\pgfpathlineto{\pgfqpoint{3.225159in}{1.440440in}}%
\pgfpathlineto{\pgfqpoint{3.236716in}{1.444901in}}%
\pgfpathlineto{\pgfqpoint{3.248268in}{1.449277in}}%
\pgfpathlineto{\pgfqpoint{3.259814in}{1.453566in}}%
\pgfpathlineto{\pgfqpoint{3.271354in}{1.457768in}}%
\pgfpathlineto{\pgfqpoint{3.264949in}{1.470724in}}%
\pgfpathlineto{\pgfqpoint{3.258546in}{1.483439in}}%
\pgfpathlineto{\pgfqpoint{3.252145in}{1.495959in}}%
\pgfpathlineto{\pgfqpoint{3.245747in}{1.508331in}}%
\pgfpathlineto{\pgfqpoint{3.239351in}{1.520602in}}%
\pgfpathlineto{\pgfqpoint{3.227819in}{1.516561in}}%
\pgfpathlineto{\pgfqpoint{3.216280in}{1.512505in}}%
\pgfpathlineto{\pgfqpoint{3.204736in}{1.508433in}}%
\pgfpathlineto{\pgfqpoint{3.193187in}{1.504342in}}%
\pgfpathlineto{\pgfqpoint{3.181632in}{1.500231in}}%
\pgfpathlineto{\pgfqpoint{3.188019in}{1.487525in}}%
\pgfpathlineto{\pgfqpoint{3.194409in}{1.474736in}}%
\pgfpathlineto{\pgfqpoint{3.200802in}{1.461867in}}%
\pgfpathlineto{\pgfqpoint{3.207198in}{1.448920in}}%
\pgfpathclose%
\pgfusepath{stroke,fill}%
\end{pgfscope}%
\begin{pgfscope}%
\pgfpathrectangle{\pgfqpoint{0.887500in}{0.275000in}}{\pgfqpoint{4.225000in}{4.225000in}}%
\pgfusepath{clip}%
\pgfsetbuttcap%
\pgfsetroundjoin%
\definecolor{currentfill}{rgb}{0.210503,0.363727,0.552206}%
\pgfsetfillcolor{currentfill}%
\pgfsetfillopacity{0.700000}%
\pgfsetlinewidth{0.501875pt}%
\definecolor{currentstroke}{rgb}{1.000000,1.000000,1.000000}%
\pgfsetstrokecolor{currentstroke}%
\pgfsetstrokeopacity{0.500000}%
\pgfsetdash{}{0pt}%
\pgfpathmoveto{\pgfqpoint{2.436820in}{1.725603in}}%
\pgfpathlineto{\pgfqpoint{2.448572in}{1.729604in}}%
\pgfpathlineto{\pgfqpoint{2.460319in}{1.733605in}}%
\pgfpathlineto{\pgfqpoint{2.472059in}{1.737608in}}%
\pgfpathlineto{\pgfqpoint{2.483794in}{1.741614in}}%
\pgfpathlineto{\pgfqpoint{2.495523in}{1.745625in}}%
\pgfpathlineto{\pgfqpoint{2.489330in}{1.756403in}}%
\pgfpathlineto{\pgfqpoint{2.483141in}{1.767134in}}%
\pgfpathlineto{\pgfqpoint{2.476956in}{1.777819in}}%
\pgfpathlineto{\pgfqpoint{2.470776in}{1.788458in}}%
\pgfpathlineto{\pgfqpoint{2.464600in}{1.799050in}}%
\pgfpathlineto{\pgfqpoint{2.452880in}{1.795078in}}%
\pgfpathlineto{\pgfqpoint{2.441154in}{1.791110in}}%
\pgfpathlineto{\pgfqpoint{2.429423in}{1.787146in}}%
\pgfpathlineto{\pgfqpoint{2.417686in}{1.783183in}}%
\pgfpathlineto{\pgfqpoint{2.405943in}{1.779220in}}%
\pgfpathlineto{\pgfqpoint{2.412110in}{1.768587in}}%
\pgfpathlineto{\pgfqpoint{2.418281in}{1.757908in}}%
\pgfpathlineto{\pgfqpoint{2.424457in}{1.747185in}}%
\pgfpathlineto{\pgfqpoint{2.430636in}{1.736417in}}%
\pgfpathclose%
\pgfusepath{stroke,fill}%
\end{pgfscope}%
\begin{pgfscope}%
\pgfpathrectangle{\pgfqpoint{0.887500in}{0.275000in}}{\pgfqpoint{4.225000in}{4.225000in}}%
\pgfusepath{clip}%
\pgfsetbuttcap%
\pgfsetroundjoin%
\definecolor{currentfill}{rgb}{0.187231,0.414746,0.556547}%
\pgfsetfillcolor{currentfill}%
\pgfsetfillopacity{0.700000}%
\pgfsetlinewidth{0.501875pt}%
\definecolor{currentstroke}{rgb}{1.000000,1.000000,1.000000}%
\pgfsetstrokecolor{currentstroke}%
\pgfsetstrokeopacity{0.500000}%
\pgfsetdash{}{0pt}%
\pgfpathmoveto{\pgfqpoint{2.019279in}{1.836052in}}%
\pgfpathlineto{\pgfqpoint{2.031134in}{1.839990in}}%
\pgfpathlineto{\pgfqpoint{2.042984in}{1.843922in}}%
\pgfpathlineto{\pgfqpoint{2.054828in}{1.847851in}}%
\pgfpathlineto{\pgfqpoint{2.066667in}{1.851776in}}%
\pgfpathlineto{\pgfqpoint{2.078499in}{1.855698in}}%
\pgfpathlineto{\pgfqpoint{2.072442in}{1.865822in}}%
\pgfpathlineto{\pgfqpoint{2.066389in}{1.875905in}}%
\pgfpathlineto{\pgfqpoint{2.060341in}{1.885949in}}%
\pgfpathlineto{\pgfqpoint{2.054297in}{1.895953in}}%
\pgfpathlineto{\pgfqpoint{2.048258in}{1.905918in}}%
\pgfpathlineto{\pgfqpoint{2.036435in}{1.902040in}}%
\pgfpathlineto{\pgfqpoint{2.024607in}{1.898159in}}%
\pgfpathlineto{\pgfqpoint{2.012773in}{1.894275in}}%
\pgfpathlineto{\pgfqpoint{2.000934in}{1.890386in}}%
\pgfpathlineto{\pgfqpoint{1.989088in}{1.886492in}}%
\pgfpathlineto{\pgfqpoint{1.995117in}{1.876481in}}%
\pgfpathlineto{\pgfqpoint{2.001151in}{1.866432in}}%
\pgfpathlineto{\pgfqpoint{2.007189in}{1.856344in}}%
\pgfpathlineto{\pgfqpoint{2.013232in}{1.846218in}}%
\pgfpathclose%
\pgfusepath{stroke,fill}%
\end{pgfscope}%
\begin{pgfscope}%
\pgfpathrectangle{\pgfqpoint{0.887500in}{0.275000in}}{\pgfqpoint{4.225000in}{4.225000in}}%
\pgfusepath{clip}%
\pgfsetbuttcap%
\pgfsetroundjoin%
\definecolor{currentfill}{rgb}{0.272594,0.025563,0.353093}%
\pgfsetfillcolor{currentfill}%
\pgfsetfillopacity{0.700000}%
\pgfsetlinewidth{0.501875pt}%
\definecolor{currentstroke}{rgb}{1.000000,1.000000,1.000000}%
\pgfsetstrokecolor{currentstroke}%
\pgfsetstrokeopacity{0.500000}%
\pgfsetdash{}{0pt}%
\pgfpathmoveto{\pgfqpoint{3.604696in}{1.151745in}}%
\pgfpathlineto{\pgfqpoint{3.616157in}{1.155542in}}%
\pgfpathlineto{\pgfqpoint{3.627619in}{1.159774in}}%
\pgfpathlineto{\pgfqpoint{3.639082in}{1.164449in}}%
\pgfpathlineto{\pgfqpoint{3.650546in}{1.169519in}}%
\pgfpathlineto{\pgfqpoint{3.662010in}{1.174937in}}%
\pgfpathlineto{\pgfqpoint{3.655585in}{1.192470in}}%
\pgfpathlineto{\pgfqpoint{3.649155in}{1.209641in}}%
\pgfpathlineto{\pgfqpoint{3.642721in}{1.226463in}}%
\pgfpathlineto{\pgfqpoint{3.636284in}{1.242950in}}%
\pgfpathlineto{\pgfqpoint{3.629844in}{1.259113in}}%
\pgfpathlineto{\pgfqpoint{3.618398in}{1.254860in}}%
\pgfpathlineto{\pgfqpoint{3.606944in}{1.250526in}}%
\pgfpathlineto{\pgfqpoint{3.595485in}{1.246163in}}%
\pgfpathlineto{\pgfqpoint{3.584021in}{1.241822in}}%
\pgfpathlineto{\pgfqpoint{3.572552in}{1.237510in}}%
\pgfpathlineto{\pgfqpoint{3.578990in}{1.221294in}}%
\pgfpathlineto{\pgfqpoint{3.585424in}{1.204637in}}%
\pgfpathlineto{\pgfqpoint{3.591853in}{1.187512in}}%
\pgfpathlineto{\pgfqpoint{3.598278in}{1.169891in}}%
\pgfpathclose%
\pgfusepath{stroke,fill}%
\end{pgfscope}%
\begin{pgfscope}%
\pgfpathrectangle{\pgfqpoint{0.887500in}{0.275000in}}{\pgfqpoint{4.225000in}{4.225000in}}%
\pgfusepath{clip}%
\pgfsetbuttcap%
\pgfsetroundjoin%
\definecolor{currentfill}{rgb}{0.218130,0.347432,0.550038}%
\pgfsetfillcolor{currentfill}%
\pgfsetfillopacity{0.700000}%
\pgfsetlinewidth{0.501875pt}%
\definecolor{currentstroke}{rgb}{1.000000,1.000000,1.000000}%
\pgfsetstrokecolor{currentstroke}%
\pgfsetstrokeopacity{0.500000}%
\pgfsetdash{}{0pt}%
\pgfpathmoveto{\pgfqpoint{2.526551in}{1.691036in}}%
\pgfpathlineto{\pgfqpoint{2.538283in}{1.695081in}}%
\pgfpathlineto{\pgfqpoint{2.550010in}{1.699128in}}%
\pgfpathlineto{\pgfqpoint{2.561731in}{1.703175in}}%
\pgfpathlineto{\pgfqpoint{2.573447in}{1.707220in}}%
\pgfpathlineto{\pgfqpoint{2.585157in}{1.711261in}}%
\pgfpathlineto{\pgfqpoint{2.578934in}{1.722244in}}%
\pgfpathlineto{\pgfqpoint{2.572715in}{1.733178in}}%
\pgfpathlineto{\pgfqpoint{2.566501in}{1.744065in}}%
\pgfpathlineto{\pgfqpoint{2.560290in}{1.754903in}}%
\pgfpathlineto{\pgfqpoint{2.554083in}{1.765692in}}%
\pgfpathlineto{\pgfqpoint{2.542383in}{1.761685in}}%
\pgfpathlineto{\pgfqpoint{2.530676in}{1.757672in}}%
\pgfpathlineto{\pgfqpoint{2.518964in}{1.753656in}}%
\pgfpathlineto{\pgfqpoint{2.507247in}{1.749640in}}%
\pgfpathlineto{\pgfqpoint{2.495523in}{1.745625in}}%
\pgfpathlineto{\pgfqpoint{2.501720in}{1.734800in}}%
\pgfpathlineto{\pgfqpoint{2.507922in}{1.723929in}}%
\pgfpathlineto{\pgfqpoint{2.514127in}{1.713011in}}%
\pgfpathlineto{\pgfqpoint{2.520337in}{1.702046in}}%
\pgfpathclose%
\pgfusepath{stroke,fill}%
\end{pgfscope}%
\begin{pgfscope}%
\pgfpathrectangle{\pgfqpoint{0.887500in}{0.275000in}}{\pgfqpoint{4.225000in}{4.225000in}}%
\pgfusepath{clip}%
\pgfsetbuttcap%
\pgfsetroundjoin%
\definecolor{currentfill}{rgb}{0.192357,0.403199,0.555836}%
\pgfsetfillcolor{currentfill}%
\pgfsetfillopacity{0.700000}%
\pgfsetlinewidth{0.501875pt}%
\definecolor{currentstroke}{rgb}{1.000000,1.000000,1.000000}%
\pgfsetstrokecolor{currentstroke}%
\pgfsetstrokeopacity{0.500000}%
\pgfsetdash{}{0pt}%
\pgfpathmoveto{\pgfqpoint{2.108855in}{1.804469in}}%
\pgfpathlineto{\pgfqpoint{2.120691in}{1.808431in}}%
\pgfpathlineto{\pgfqpoint{2.132522in}{1.812391in}}%
\pgfpathlineto{\pgfqpoint{2.144348in}{1.816351in}}%
\pgfpathlineto{\pgfqpoint{2.156167in}{1.820312in}}%
\pgfpathlineto{\pgfqpoint{2.167981in}{1.824273in}}%
\pgfpathlineto{\pgfqpoint{2.161891in}{1.834561in}}%
\pgfpathlineto{\pgfqpoint{2.155806in}{1.844807in}}%
\pgfpathlineto{\pgfqpoint{2.149724in}{1.855012in}}%
\pgfpathlineto{\pgfqpoint{2.143648in}{1.865176in}}%
\pgfpathlineto{\pgfqpoint{2.137576in}{1.875298in}}%
\pgfpathlineto{\pgfqpoint{2.125772in}{1.871378in}}%
\pgfpathlineto{\pgfqpoint{2.113963in}{1.867458in}}%
\pgfpathlineto{\pgfqpoint{2.102147in}{1.863538in}}%
\pgfpathlineto{\pgfqpoint{2.090326in}{1.859619in}}%
\pgfpathlineto{\pgfqpoint{2.078499in}{1.855698in}}%
\pgfpathlineto{\pgfqpoint{2.084561in}{1.845534in}}%
\pgfpathlineto{\pgfqpoint{2.090628in}{1.835329in}}%
\pgfpathlineto{\pgfqpoint{2.096699in}{1.825083in}}%
\pgfpathlineto{\pgfqpoint{2.102774in}{1.814796in}}%
\pgfpathclose%
\pgfusepath{stroke,fill}%
\end{pgfscope}%
\begin{pgfscope}%
\pgfpathrectangle{\pgfqpoint{0.887500in}{0.275000in}}{\pgfqpoint{4.225000in}{4.225000in}}%
\pgfusepath{clip}%
\pgfsetbuttcap%
\pgfsetroundjoin%
\definecolor{currentfill}{rgb}{0.225863,0.330805,0.547314}%
\pgfsetfillcolor{currentfill}%
\pgfsetfillopacity{0.700000}%
\pgfsetlinewidth{0.501875pt}%
\definecolor{currentstroke}{rgb}{1.000000,1.000000,1.000000}%
\pgfsetstrokecolor{currentstroke}%
\pgfsetstrokeopacity{0.500000}%
\pgfsetdash{}{0pt}%
\pgfpathmoveto{\pgfqpoint{2.616331in}{1.655663in}}%
\pgfpathlineto{\pgfqpoint{2.628044in}{1.659725in}}%
\pgfpathlineto{\pgfqpoint{2.639752in}{1.663781in}}%
\pgfpathlineto{\pgfqpoint{2.651454in}{1.667829in}}%
\pgfpathlineto{\pgfqpoint{2.663150in}{1.671869in}}%
\pgfpathlineto{\pgfqpoint{2.674841in}{1.675897in}}%
\pgfpathlineto{\pgfqpoint{2.668589in}{1.687077in}}%
\pgfpathlineto{\pgfqpoint{2.662342in}{1.698212in}}%
\pgfpathlineto{\pgfqpoint{2.656098in}{1.709301in}}%
\pgfpathlineto{\pgfqpoint{2.649859in}{1.720344in}}%
\pgfpathlineto{\pgfqpoint{2.643623in}{1.731339in}}%
\pgfpathlineto{\pgfqpoint{2.631941in}{1.727346in}}%
\pgfpathlineto{\pgfqpoint{2.620253in}{1.723340in}}%
\pgfpathlineto{\pgfqpoint{2.608560in}{1.719323in}}%
\pgfpathlineto{\pgfqpoint{2.596861in}{1.715296in}}%
\pgfpathlineto{\pgfqpoint{2.585157in}{1.711261in}}%
\pgfpathlineto{\pgfqpoint{2.591384in}{1.700232in}}%
\pgfpathlineto{\pgfqpoint{2.597614in}{1.689157in}}%
\pgfpathlineto{\pgfqpoint{2.603849in}{1.678037in}}%
\pgfpathlineto{\pgfqpoint{2.610088in}{1.666872in}}%
\pgfpathclose%
\pgfusepath{stroke,fill}%
\end{pgfscope}%
\begin{pgfscope}%
\pgfpathrectangle{\pgfqpoint{0.887500in}{0.275000in}}{\pgfqpoint{4.225000in}{4.225000in}}%
\pgfusepath{clip}%
\pgfsetbuttcap%
\pgfsetroundjoin%
\definecolor{currentfill}{rgb}{0.280267,0.073417,0.397163}%
\pgfsetfillcolor{currentfill}%
\pgfsetfillopacity{0.700000}%
\pgfsetlinewidth{0.501875pt}%
\definecolor{currentstroke}{rgb}{1.000000,1.000000,1.000000}%
\pgfsetstrokecolor{currentstroke}%
\pgfsetstrokeopacity{0.500000}%
\pgfsetdash{}{0pt}%
\pgfpathmoveto{\pgfqpoint{3.515069in}{1.211297in}}%
\pgfpathlineto{\pgfqpoint{3.526589in}{1.217688in}}%
\pgfpathlineto{\pgfqpoint{3.538096in}{1.223346in}}%
\pgfpathlineto{\pgfqpoint{3.549591in}{1.228431in}}%
\pgfpathlineto{\pgfqpoint{3.561076in}{1.233099in}}%
\pgfpathlineto{\pgfqpoint{3.572552in}{1.237510in}}%
\pgfpathlineto{\pgfqpoint{3.566110in}{1.253316in}}%
\pgfpathlineto{\pgfqpoint{3.559666in}{1.268738in}}%
\pgfpathlineto{\pgfqpoint{3.553219in}{1.283805in}}%
\pgfpathlineto{\pgfqpoint{3.546770in}{1.298547in}}%
\pgfpathlineto{\pgfqpoint{3.540320in}{1.312991in}}%
\pgfpathlineto{\pgfqpoint{3.528845in}{1.308400in}}%
\pgfpathlineto{\pgfqpoint{3.517362in}{1.303601in}}%
\pgfpathlineto{\pgfqpoint{3.505869in}{1.298478in}}%
\pgfpathlineto{\pgfqpoint{3.494367in}{1.292913in}}%
\pgfpathlineto{\pgfqpoint{3.482854in}{1.286791in}}%
\pgfpathlineto{\pgfqpoint{3.489295in}{1.271972in}}%
\pgfpathlineto{\pgfqpoint{3.495738in}{1.257017in}}%
\pgfpathlineto{\pgfqpoint{3.502181in}{1.241921in}}%
\pgfpathlineto{\pgfqpoint{3.508624in}{1.226683in}}%
\pgfpathclose%
\pgfusepath{stroke,fill}%
\end{pgfscope}%
\begin{pgfscope}%
\pgfpathrectangle{\pgfqpoint{0.887500in}{0.275000in}}{\pgfqpoint{4.225000in}{4.225000in}}%
\pgfusepath{clip}%
\pgfsetbuttcap%
\pgfsetroundjoin%
\definecolor{currentfill}{rgb}{0.282910,0.105393,0.426902}%
\pgfsetfillcolor{currentfill}%
\pgfsetfillopacity{0.700000}%
\pgfsetlinewidth{0.501875pt}%
\definecolor{currentstroke}{rgb}{1.000000,1.000000,1.000000}%
\pgfsetstrokecolor{currentstroke}%
\pgfsetstrokeopacity{0.500000}%
\pgfsetdash{}{0pt}%
\pgfpathmoveto{\pgfqpoint{3.425158in}{1.249240in}}%
\pgfpathlineto{\pgfqpoint{3.436705in}{1.256838in}}%
\pgfpathlineto{\pgfqpoint{3.448251in}{1.264714in}}%
\pgfpathlineto{\pgfqpoint{3.459794in}{1.272542in}}%
\pgfpathlineto{\pgfqpoint{3.471330in}{1.279997in}}%
\pgfpathlineto{\pgfqpoint{3.482854in}{1.286791in}}%
\pgfpathlineto{\pgfqpoint{3.476414in}{1.301478in}}%
\pgfpathlineto{\pgfqpoint{3.469975in}{1.316036in}}%
\pgfpathlineto{\pgfqpoint{3.463537in}{1.330469in}}%
\pgfpathlineto{\pgfqpoint{3.457100in}{1.344781in}}%
\pgfpathlineto{\pgfqpoint{3.450665in}{1.358975in}}%
\pgfpathlineto{\pgfqpoint{3.439161in}{1.353689in}}%
\pgfpathlineto{\pgfqpoint{3.427649in}{1.348028in}}%
\pgfpathlineto{\pgfqpoint{3.416130in}{1.342140in}}%
\pgfpathlineto{\pgfqpoint{3.404606in}{1.336196in}}%
\pgfpathlineto{\pgfqpoint{3.393078in}{1.330364in}}%
\pgfpathlineto{\pgfqpoint{3.399491in}{1.314243in}}%
\pgfpathlineto{\pgfqpoint{3.405905in}{1.298026in}}%
\pgfpathlineto{\pgfqpoint{3.412320in}{1.281757in}}%
\pgfpathlineto{\pgfqpoint{3.418738in}{1.265480in}}%
\pgfpathclose%
\pgfusepath{stroke,fill}%
\end{pgfscope}%
\begin{pgfscope}%
\pgfpathrectangle{\pgfqpoint{0.887500in}{0.275000in}}{\pgfqpoint{4.225000in}{4.225000in}}%
\pgfusepath{clip}%
\pgfsetbuttcap%
\pgfsetroundjoin%
\definecolor{currentfill}{rgb}{0.233603,0.313828,0.543914}%
\pgfsetfillcolor{currentfill}%
\pgfsetfillopacity{0.700000}%
\pgfsetlinewidth{0.501875pt}%
\definecolor{currentstroke}{rgb}{1.000000,1.000000,1.000000}%
\pgfsetstrokecolor{currentstroke}%
\pgfsetstrokeopacity{0.500000}%
\pgfsetdash{}{0pt}%
\pgfpathmoveto{\pgfqpoint{2.706156in}{1.619346in}}%
\pgfpathlineto{\pgfqpoint{2.717849in}{1.623394in}}%
\pgfpathlineto{\pgfqpoint{2.729538in}{1.627429in}}%
\pgfpathlineto{\pgfqpoint{2.741221in}{1.631450in}}%
\pgfpathlineto{\pgfqpoint{2.752898in}{1.635456in}}%
\pgfpathlineto{\pgfqpoint{2.764569in}{1.639448in}}%
\pgfpathlineto{\pgfqpoint{2.758290in}{1.650817in}}%
\pgfpathlineto{\pgfqpoint{2.752015in}{1.662137in}}%
\pgfpathlineto{\pgfqpoint{2.745744in}{1.673413in}}%
\pgfpathlineto{\pgfqpoint{2.739476in}{1.684645in}}%
\pgfpathlineto{\pgfqpoint{2.733212in}{1.695832in}}%
\pgfpathlineto{\pgfqpoint{2.721549in}{1.691874in}}%
\pgfpathlineto{\pgfqpoint{2.709880in}{1.687903in}}%
\pgfpathlineto{\pgfqpoint{2.698206in}{1.683916in}}%
\pgfpathlineto{\pgfqpoint{2.686526in}{1.679913in}}%
\pgfpathlineto{\pgfqpoint{2.674841in}{1.675897in}}%
\pgfpathlineto{\pgfqpoint{2.681096in}{1.664673in}}%
\pgfpathlineto{\pgfqpoint{2.687355in}{1.653406in}}%
\pgfpathlineto{\pgfqpoint{2.693618in}{1.642097in}}%
\pgfpathlineto{\pgfqpoint{2.699885in}{1.630745in}}%
\pgfpathclose%
\pgfusepath{stroke,fill}%
\end{pgfscope}%
\begin{pgfscope}%
\pgfpathrectangle{\pgfqpoint{0.887500in}{0.275000in}}{\pgfqpoint{4.225000in}{4.225000in}}%
\pgfusepath{clip}%
\pgfsetbuttcap%
\pgfsetroundjoin%
\definecolor{currentfill}{rgb}{0.197636,0.391528,0.554969}%
\pgfsetfillcolor{currentfill}%
\pgfsetfillopacity{0.700000}%
\pgfsetlinewidth{0.501875pt}%
\definecolor{currentstroke}{rgb}{1.000000,1.000000,1.000000}%
\pgfsetstrokecolor{currentstroke}%
\pgfsetstrokeopacity{0.500000}%
\pgfsetdash{}{0pt}%
\pgfpathmoveto{\pgfqpoint{2.198498in}{1.772221in}}%
\pgfpathlineto{\pgfqpoint{2.210316in}{1.776218in}}%
\pgfpathlineto{\pgfqpoint{2.222128in}{1.780215in}}%
\pgfpathlineto{\pgfqpoint{2.233934in}{1.784209in}}%
\pgfpathlineto{\pgfqpoint{2.245735in}{1.788201in}}%
\pgfpathlineto{\pgfqpoint{2.257530in}{1.792188in}}%
\pgfpathlineto{\pgfqpoint{2.251408in}{1.802643in}}%
\pgfpathlineto{\pgfqpoint{2.245291in}{1.813056in}}%
\pgfpathlineto{\pgfqpoint{2.239177in}{1.823428in}}%
\pgfpathlineto{\pgfqpoint{2.233069in}{1.833758in}}%
\pgfpathlineto{\pgfqpoint{2.226964in}{1.844046in}}%
\pgfpathlineto{\pgfqpoint{2.215179in}{1.840100in}}%
\pgfpathlineto{\pgfqpoint{2.203388in}{1.836148in}}%
\pgfpathlineto{\pgfqpoint{2.191591in}{1.832192in}}%
\pgfpathlineto{\pgfqpoint{2.179789in}{1.828233in}}%
\pgfpathlineto{\pgfqpoint{2.167981in}{1.824273in}}%
\pgfpathlineto{\pgfqpoint{2.174076in}{1.813944in}}%
\pgfpathlineto{\pgfqpoint{2.180175in}{1.803574in}}%
\pgfpathlineto{\pgfqpoint{2.186278in}{1.793164in}}%
\pgfpathlineto{\pgfqpoint{2.192386in}{1.782712in}}%
\pgfpathclose%
\pgfusepath{stroke,fill}%
\end{pgfscope}%
\begin{pgfscope}%
\pgfpathrectangle{\pgfqpoint{0.887500in}{0.275000in}}{\pgfqpoint{4.225000in}{4.225000in}}%
\pgfusepath{clip}%
\pgfsetbuttcap%
\pgfsetroundjoin%
\definecolor{currentfill}{rgb}{0.282623,0.140926,0.457517}%
\pgfsetfillcolor{currentfill}%
\pgfsetfillopacity{0.700000}%
\pgfsetlinewidth{0.501875pt}%
\definecolor{currentstroke}{rgb}{1.000000,1.000000,1.000000}%
\pgfsetstrokecolor{currentstroke}%
\pgfsetstrokeopacity{0.500000}%
\pgfsetdash{}{0pt}%
\pgfpathmoveto{\pgfqpoint{3.335425in}{1.308426in}}%
\pgfpathlineto{\pgfqpoint{3.346957in}{1.311498in}}%
\pgfpathlineto{\pgfqpoint{3.358487in}{1.315254in}}%
\pgfpathlineto{\pgfqpoint{3.370018in}{1.319724in}}%
\pgfpathlineto{\pgfqpoint{3.381549in}{1.324817in}}%
\pgfpathlineto{\pgfqpoint{3.393078in}{1.330364in}}%
\pgfpathlineto{\pgfqpoint{3.386666in}{1.346345in}}%
\pgfpathlineto{\pgfqpoint{3.380255in}{1.362141in}}%
\pgfpathlineto{\pgfqpoint{3.373844in}{1.377708in}}%
\pgfpathlineto{\pgfqpoint{3.367432in}{1.393002in}}%
\pgfpathlineto{\pgfqpoint{3.361021in}{1.407978in}}%
\pgfpathlineto{\pgfqpoint{3.349505in}{1.403782in}}%
\pgfpathlineto{\pgfqpoint{3.337983in}{1.399632in}}%
\pgfpathlineto{\pgfqpoint{3.326457in}{1.395570in}}%
\pgfpathlineto{\pgfqpoint{3.314925in}{1.391616in}}%
\pgfpathlineto{\pgfqpoint{3.303389in}{1.387762in}}%
\pgfpathlineto{\pgfqpoint{3.309796in}{1.372500in}}%
\pgfpathlineto{\pgfqpoint{3.316203in}{1.356877in}}%
\pgfpathlineto{\pgfqpoint{3.322610in}{1.340951in}}%
\pgfpathlineto{\pgfqpoint{3.329017in}{1.324781in}}%
\pgfpathclose%
\pgfusepath{stroke,fill}%
\end{pgfscope}%
\begin{pgfscope}%
\pgfpathrectangle{\pgfqpoint{0.887500in}{0.275000in}}{\pgfqpoint{4.225000in}{4.225000in}}%
\pgfusepath{clip}%
\pgfsetbuttcap%
\pgfsetroundjoin%
\definecolor{currentfill}{rgb}{0.243113,0.292092,0.538516}%
\pgfsetfillcolor{currentfill}%
\pgfsetfillopacity{0.700000}%
\pgfsetlinewidth{0.501875pt}%
\definecolor{currentstroke}{rgb}{1.000000,1.000000,1.000000}%
\pgfsetstrokecolor{currentstroke}%
\pgfsetstrokeopacity{0.500000}%
\pgfsetdash{}{0pt}%
\pgfpathmoveto{\pgfqpoint{2.796021in}{1.581756in}}%
\pgfpathlineto{\pgfqpoint{2.807695in}{1.585764in}}%
\pgfpathlineto{\pgfqpoint{2.819364in}{1.589766in}}%
\pgfpathlineto{\pgfqpoint{2.831027in}{1.593765in}}%
\pgfpathlineto{\pgfqpoint{2.842684in}{1.597766in}}%
\pgfpathlineto{\pgfqpoint{2.854335in}{1.601770in}}%
\pgfpathlineto{\pgfqpoint{2.848030in}{1.613409in}}%
\pgfpathlineto{\pgfqpoint{2.841727in}{1.624982in}}%
\pgfpathlineto{\pgfqpoint{2.835429in}{1.636491in}}%
\pgfpathlineto{\pgfqpoint{2.829134in}{1.647940in}}%
\pgfpathlineto{\pgfqpoint{2.822843in}{1.659331in}}%
\pgfpathlineto{\pgfqpoint{2.811200in}{1.655354in}}%
\pgfpathlineto{\pgfqpoint{2.799550in}{1.651381in}}%
\pgfpathlineto{\pgfqpoint{2.787896in}{1.647407in}}%
\pgfpathlineto{\pgfqpoint{2.776235in}{1.643431in}}%
\pgfpathlineto{\pgfqpoint{2.764569in}{1.639448in}}%
\pgfpathlineto{\pgfqpoint{2.770852in}{1.628028in}}%
\pgfpathlineto{\pgfqpoint{2.777139in}{1.616552in}}%
\pgfpathlineto{\pgfqpoint{2.783429in}{1.605017in}}%
\pgfpathlineto{\pgfqpoint{2.789723in}{1.593420in}}%
\pgfpathclose%
\pgfusepath{stroke,fill}%
\end{pgfscope}%
\begin{pgfscope}%
\pgfpathrectangle{\pgfqpoint{0.887500in}{0.275000in}}{\pgfqpoint{4.225000in}{4.225000in}}%
\pgfusepath{clip}%
\pgfsetbuttcap%
\pgfsetroundjoin%
\definecolor{currentfill}{rgb}{0.250425,0.274290,0.533103}%
\pgfsetfillcolor{currentfill}%
\pgfsetfillopacity{0.700000}%
\pgfsetlinewidth{0.501875pt}%
\definecolor{currentstroke}{rgb}{1.000000,1.000000,1.000000}%
\pgfsetstrokecolor{currentstroke}%
\pgfsetstrokeopacity{0.500000}%
\pgfsetdash{}{0pt}%
\pgfpathmoveto{\pgfqpoint{2.885921in}{1.542476in}}%
\pgfpathlineto{\pgfqpoint{2.897574in}{1.546515in}}%
\pgfpathlineto{\pgfqpoint{2.909222in}{1.550561in}}%
\pgfpathlineto{\pgfqpoint{2.920864in}{1.554616in}}%
\pgfpathlineto{\pgfqpoint{2.932500in}{1.558680in}}%
\pgfpathlineto{\pgfqpoint{2.944131in}{1.562757in}}%
\pgfpathlineto{\pgfqpoint{2.937799in}{1.574714in}}%
\pgfpathlineto{\pgfqpoint{2.931471in}{1.586602in}}%
\pgfpathlineto{\pgfqpoint{2.925147in}{1.598418in}}%
\pgfpathlineto{\pgfqpoint{2.918826in}{1.610163in}}%
\pgfpathlineto{\pgfqpoint{2.912508in}{1.621840in}}%
\pgfpathlineto{\pgfqpoint{2.900885in}{1.617826in}}%
\pgfpathlineto{\pgfqpoint{2.889256in}{1.613810in}}%
\pgfpathlineto{\pgfqpoint{2.877621in}{1.609794in}}%
\pgfpathlineto{\pgfqpoint{2.865981in}{1.605780in}}%
\pgfpathlineto{\pgfqpoint{2.854335in}{1.601770in}}%
\pgfpathlineto{\pgfqpoint{2.860645in}{1.590061in}}%
\pgfpathlineto{\pgfqpoint{2.866959in}{1.578279in}}%
\pgfpathlineto{\pgfqpoint{2.873276in}{1.566420in}}%
\pgfpathlineto{\pgfqpoint{2.879596in}{1.554484in}}%
\pgfpathclose%
\pgfusepath{stroke,fill}%
\end{pgfscope}%
\begin{pgfscope}%
\pgfpathrectangle{\pgfqpoint{0.887500in}{0.275000in}}{\pgfqpoint{4.225000in}{4.225000in}}%
\pgfusepath{clip}%
\pgfsetbuttcap%
\pgfsetroundjoin%
\definecolor{currentfill}{rgb}{0.204903,0.375746,0.553533}%
\pgfsetfillcolor{currentfill}%
\pgfsetfillopacity{0.700000}%
\pgfsetlinewidth{0.501875pt}%
\definecolor{currentstroke}{rgb}{1.000000,1.000000,1.000000}%
\pgfsetstrokecolor{currentstroke}%
\pgfsetstrokeopacity{0.500000}%
\pgfsetdash{}{0pt}%
\pgfpathmoveto{\pgfqpoint{2.288205in}{1.739290in}}%
\pgfpathlineto{\pgfqpoint{2.300005in}{1.743308in}}%
\pgfpathlineto{\pgfqpoint{2.311798in}{1.747322in}}%
\pgfpathlineto{\pgfqpoint{2.323586in}{1.751331in}}%
\pgfpathlineto{\pgfqpoint{2.335368in}{1.755334in}}%
\pgfpathlineto{\pgfqpoint{2.347144in}{1.759332in}}%
\pgfpathlineto{\pgfqpoint{2.340991in}{1.769960in}}%
\pgfpathlineto{\pgfqpoint{2.334842in}{1.780545in}}%
\pgfpathlineto{\pgfqpoint{2.328698in}{1.791087in}}%
\pgfpathlineto{\pgfqpoint{2.322557in}{1.801586in}}%
\pgfpathlineto{\pgfqpoint{2.316421in}{1.812042in}}%
\pgfpathlineto{\pgfqpoint{2.304654in}{1.808084in}}%
\pgfpathlineto{\pgfqpoint{2.292882in}{1.804119in}}%
\pgfpathlineto{\pgfqpoint{2.281104in}{1.800148in}}%
\pgfpathlineto{\pgfqpoint{2.269320in}{1.796171in}}%
\pgfpathlineto{\pgfqpoint{2.257530in}{1.792188in}}%
\pgfpathlineto{\pgfqpoint{2.263657in}{1.781692in}}%
\pgfpathlineto{\pgfqpoint{2.269787in}{1.771153in}}%
\pgfpathlineto{\pgfqpoint{2.275922in}{1.760574in}}%
\pgfpathlineto{\pgfqpoint{2.282062in}{1.749953in}}%
\pgfpathclose%
\pgfusepath{stroke,fill}%
\end{pgfscope}%
\begin{pgfscope}%
\pgfpathrectangle{\pgfqpoint{0.887500in}{0.275000in}}{\pgfqpoint{4.225000in}{4.225000in}}%
\pgfusepath{clip}%
\pgfsetbuttcap%
\pgfsetroundjoin%
\definecolor{currentfill}{rgb}{0.278826,0.175490,0.483397}%
\pgfsetfillcolor{currentfill}%
\pgfsetfillopacity{0.700000}%
\pgfsetlinewidth{0.501875pt}%
\definecolor{currentstroke}{rgb}{1.000000,1.000000,1.000000}%
\pgfsetstrokecolor{currentstroke}%
\pgfsetstrokeopacity{0.500000}%
\pgfsetdash{}{0pt}%
\pgfpathmoveto{\pgfqpoint{3.245628in}{1.369706in}}%
\pgfpathlineto{\pgfqpoint{3.257191in}{1.373186in}}%
\pgfpathlineto{\pgfqpoint{3.268749in}{1.376722in}}%
\pgfpathlineto{\pgfqpoint{3.280301in}{1.380325in}}%
\pgfpathlineto{\pgfqpoint{3.291848in}{1.384002in}}%
\pgfpathlineto{\pgfqpoint{3.303389in}{1.387762in}}%
\pgfpathlineto{\pgfqpoint{3.296982in}{1.402605in}}%
\pgfpathlineto{\pgfqpoint{3.290574in}{1.416989in}}%
\pgfpathlineto{\pgfqpoint{3.284167in}{1.430947in}}%
\pgfpathlineto{\pgfqpoint{3.277760in}{1.444525in}}%
\pgfpathlineto{\pgfqpoint{3.271354in}{1.457768in}}%
\pgfpathlineto{\pgfqpoint{3.259814in}{1.453566in}}%
\pgfpathlineto{\pgfqpoint{3.248268in}{1.449277in}}%
\pgfpathlineto{\pgfqpoint{3.236716in}{1.444901in}}%
\pgfpathlineto{\pgfqpoint{3.225159in}{1.440440in}}%
\pgfpathlineto{\pgfqpoint{3.213596in}{1.435897in}}%
\pgfpathlineto{\pgfqpoint{3.219997in}{1.422799in}}%
\pgfpathlineto{\pgfqpoint{3.226401in}{1.409629in}}%
\pgfpathlineto{\pgfqpoint{3.232807in}{1.396388in}}%
\pgfpathlineto{\pgfqpoint{3.239217in}{1.383080in}}%
\pgfpathclose%
\pgfusepath{stroke,fill}%
\end{pgfscope}%
\begin{pgfscope}%
\pgfpathrectangle{\pgfqpoint{0.887500in}{0.275000in}}{\pgfqpoint{4.225000in}{4.225000in}}%
\pgfusepath{clip}%
\pgfsetbuttcap%
\pgfsetroundjoin%
\definecolor{currentfill}{rgb}{0.258965,0.251537,0.524736}%
\pgfsetfillcolor{currentfill}%
\pgfsetfillopacity{0.700000}%
\pgfsetlinewidth{0.501875pt}%
\definecolor{currentstroke}{rgb}{1.000000,1.000000,1.000000}%
\pgfsetstrokecolor{currentstroke}%
\pgfsetstrokeopacity{0.500000}%
\pgfsetdash{}{0pt}%
\pgfpathmoveto{\pgfqpoint{2.975841in}{1.502079in}}%
\pgfpathlineto{\pgfqpoint{2.987473in}{1.506230in}}%
\pgfpathlineto{\pgfqpoint{2.999100in}{1.510414in}}%
\pgfpathlineto{\pgfqpoint{3.010721in}{1.514631in}}%
\pgfpathlineto{\pgfqpoint{3.022337in}{1.518861in}}%
\pgfpathlineto{\pgfqpoint{3.033947in}{1.523088in}}%
\pgfpathlineto{\pgfqpoint{3.027591in}{1.535329in}}%
\pgfpathlineto{\pgfqpoint{3.021238in}{1.547459in}}%
\pgfpathlineto{\pgfqpoint{3.014889in}{1.559488in}}%
\pgfpathlineto{\pgfqpoint{3.008543in}{1.571430in}}%
\pgfpathlineto{\pgfqpoint{3.002201in}{1.583296in}}%
\pgfpathlineto{\pgfqpoint{2.990598in}{1.579183in}}%
\pgfpathlineto{\pgfqpoint{2.978990in}{1.575064in}}%
\pgfpathlineto{\pgfqpoint{2.967376in}{1.570949in}}%
\pgfpathlineto{\pgfqpoint{2.955756in}{1.566845in}}%
\pgfpathlineto{\pgfqpoint{2.944131in}{1.562757in}}%
\pgfpathlineto{\pgfqpoint{2.950466in}{1.550734in}}%
\pgfpathlineto{\pgfqpoint{2.956805in}{1.538652in}}%
\pgfpathlineto{\pgfqpoint{2.963147in}{1.526512in}}%
\pgfpathlineto{\pgfqpoint{2.969492in}{1.514320in}}%
\pgfpathclose%
\pgfusepath{stroke,fill}%
\end{pgfscope}%
\begin{pgfscope}%
\pgfpathrectangle{\pgfqpoint{0.887500in}{0.275000in}}{\pgfqpoint{4.225000in}{4.225000in}}%
\pgfusepath{clip}%
\pgfsetbuttcap%
\pgfsetroundjoin%
\definecolor{currentfill}{rgb}{0.274128,0.199721,0.498911}%
\pgfsetfillcolor{currentfill}%
\pgfsetfillopacity{0.700000}%
\pgfsetlinewidth{0.501875pt}%
\definecolor{currentstroke}{rgb}{1.000000,1.000000,1.000000}%
\pgfsetstrokecolor{currentstroke}%
\pgfsetstrokeopacity{0.500000}%
\pgfsetdash{}{0pt}%
\pgfpathmoveto{\pgfqpoint{3.155702in}{1.413272in}}%
\pgfpathlineto{\pgfqpoint{3.167291in}{1.417621in}}%
\pgfpathlineto{\pgfqpoint{3.178875in}{1.422107in}}%
\pgfpathlineto{\pgfqpoint{3.190454in}{1.426681in}}%
\pgfpathlineto{\pgfqpoint{3.202028in}{1.431294in}}%
\pgfpathlineto{\pgfqpoint{3.213596in}{1.435897in}}%
\pgfpathlineto{\pgfqpoint{3.207198in}{1.448920in}}%
\pgfpathlineto{\pgfqpoint{3.200802in}{1.461867in}}%
\pgfpathlineto{\pgfqpoint{3.194409in}{1.474736in}}%
\pgfpathlineto{\pgfqpoint{3.188019in}{1.487525in}}%
\pgfpathlineto{\pgfqpoint{3.181632in}{1.500231in}}%
\pgfpathlineto{\pgfqpoint{3.170071in}{1.496104in}}%
\pgfpathlineto{\pgfqpoint{3.158505in}{1.491973in}}%
\pgfpathlineto{\pgfqpoint{3.146933in}{1.487847in}}%
\pgfpathlineto{\pgfqpoint{3.135355in}{1.483737in}}%
\pgfpathlineto{\pgfqpoint{3.123773in}{1.479653in}}%
\pgfpathlineto{\pgfqpoint{3.130153in}{1.466359in}}%
\pgfpathlineto{\pgfqpoint{3.136535in}{1.453005in}}%
\pgfpathlineto{\pgfqpoint{3.142921in}{1.439660in}}%
\pgfpathlineto{\pgfqpoint{3.149310in}{1.426393in}}%
\pgfpathclose%
\pgfusepath{stroke,fill}%
\end{pgfscope}%
\begin{pgfscope}%
\pgfpathrectangle{\pgfqpoint{0.887500in}{0.275000in}}{\pgfqpoint{4.225000in}{4.225000in}}%
\pgfusepath{clip}%
\pgfsetbuttcap%
\pgfsetroundjoin%
\definecolor{currentfill}{rgb}{0.266580,0.228262,0.514349}%
\pgfsetfillcolor{currentfill}%
\pgfsetfillopacity{0.700000}%
\pgfsetlinewidth{0.501875pt}%
\definecolor{currentstroke}{rgb}{1.000000,1.000000,1.000000}%
\pgfsetstrokecolor{currentstroke}%
\pgfsetstrokeopacity{0.500000}%
\pgfsetdash{}{0pt}%
\pgfpathmoveto{\pgfqpoint{3.065774in}{1.459836in}}%
\pgfpathlineto{\pgfqpoint{3.077385in}{1.463734in}}%
\pgfpathlineto{\pgfqpoint{3.088991in}{1.467655in}}%
\pgfpathlineto{\pgfqpoint{3.100590in}{1.471609in}}%
\pgfpathlineto{\pgfqpoint{3.112184in}{1.475607in}}%
\pgfpathlineto{\pgfqpoint{3.123773in}{1.479653in}}%
\pgfpathlineto{\pgfqpoint{3.117395in}{1.492820in}}%
\pgfpathlineto{\pgfqpoint{3.111021in}{1.505794in}}%
\pgfpathlineto{\pgfqpoint{3.104649in}{1.518554in}}%
\pgfpathlineto{\pgfqpoint{3.098280in}{1.531119in}}%
\pgfpathlineto{\pgfqpoint{3.091914in}{1.543507in}}%
\pgfpathlineto{\pgfqpoint{3.080332in}{1.539571in}}%
\pgfpathlineto{\pgfqpoint{3.068744in}{1.535551in}}%
\pgfpathlineto{\pgfqpoint{3.057150in}{1.531452in}}%
\pgfpathlineto{\pgfqpoint{3.045551in}{1.527291in}}%
\pgfpathlineto{\pgfqpoint{3.033947in}{1.523088in}}%
\pgfpathlineto{\pgfqpoint{3.040306in}{1.510724in}}%
\pgfpathlineto{\pgfqpoint{3.046668in}{1.498226in}}%
\pgfpathlineto{\pgfqpoint{3.053034in}{1.485582in}}%
\pgfpathlineto{\pgfqpoint{3.059403in}{1.472781in}}%
\pgfpathclose%
\pgfusepath{stroke,fill}%
\end{pgfscope}%
\begin{pgfscope}%
\pgfpathrectangle{\pgfqpoint{0.887500in}{0.275000in}}{\pgfqpoint{4.225000in}{4.225000in}}%
\pgfusepath{clip}%
\pgfsetbuttcap%
\pgfsetroundjoin%
\definecolor{currentfill}{rgb}{0.212395,0.359683,0.551710}%
\pgfsetfillcolor{currentfill}%
\pgfsetfillopacity{0.700000}%
\pgfsetlinewidth{0.501875pt}%
\definecolor{currentstroke}{rgb}{1.000000,1.000000,1.000000}%
\pgfsetstrokecolor{currentstroke}%
\pgfsetstrokeopacity{0.500000}%
\pgfsetdash{}{0pt}%
\pgfpathmoveto{\pgfqpoint{2.377975in}{1.705540in}}%
\pgfpathlineto{\pgfqpoint{2.389755in}{1.709564in}}%
\pgfpathlineto{\pgfqpoint{2.401530in}{1.713582in}}%
\pgfpathlineto{\pgfqpoint{2.413299in}{1.717594in}}%
\pgfpathlineto{\pgfqpoint{2.425062in}{1.721600in}}%
\pgfpathlineto{\pgfqpoint{2.436820in}{1.725603in}}%
\pgfpathlineto{\pgfqpoint{2.430636in}{1.736417in}}%
\pgfpathlineto{\pgfqpoint{2.424457in}{1.747185in}}%
\pgfpathlineto{\pgfqpoint{2.418281in}{1.757908in}}%
\pgfpathlineto{\pgfqpoint{2.412110in}{1.768587in}}%
\pgfpathlineto{\pgfqpoint{2.405943in}{1.779220in}}%
\pgfpathlineto{\pgfqpoint{2.394194in}{1.775254in}}%
\pgfpathlineto{\pgfqpoint{2.382440in}{1.771284in}}%
\pgfpathlineto{\pgfqpoint{2.370681in}{1.767307in}}%
\pgfpathlineto{\pgfqpoint{2.358915in}{1.763323in}}%
\pgfpathlineto{\pgfqpoint{2.347144in}{1.759332in}}%
\pgfpathlineto{\pgfqpoint{2.353302in}{1.748660in}}%
\pgfpathlineto{\pgfqpoint{2.359464in}{1.737945in}}%
\pgfpathlineto{\pgfqpoint{2.365630in}{1.727187in}}%
\pgfpathlineto{\pgfqpoint{2.371800in}{1.716385in}}%
\pgfpathclose%
\pgfusepath{stroke,fill}%
\end{pgfscope}%
\begin{pgfscope}%
\pgfpathrectangle{\pgfqpoint{0.887500in}{0.275000in}}{\pgfqpoint{4.225000in}{4.225000in}}%
\pgfusepath{clip}%
\pgfsetbuttcap%
\pgfsetroundjoin%
\definecolor{currentfill}{rgb}{0.187231,0.414746,0.556547}%
\pgfsetfillcolor{currentfill}%
\pgfsetfillopacity{0.700000}%
\pgfsetlinewidth{0.501875pt}%
\definecolor{currentstroke}{rgb}{1.000000,1.000000,1.000000}%
\pgfsetstrokecolor{currentstroke}%
\pgfsetstrokeopacity{0.500000}%
\pgfsetdash{}{0pt}%
\pgfpathmoveto{\pgfqpoint{1.959918in}{1.816254in}}%
\pgfpathlineto{\pgfqpoint{1.971801in}{1.820231in}}%
\pgfpathlineto{\pgfqpoint{1.983679in}{1.824198in}}%
\pgfpathlineto{\pgfqpoint{1.995551in}{1.828157in}}%
\pgfpathlineto{\pgfqpoint{2.007418in}{1.832107in}}%
\pgfpathlineto{\pgfqpoint{2.019279in}{1.836052in}}%
\pgfpathlineto{\pgfqpoint{2.013232in}{1.846218in}}%
\pgfpathlineto{\pgfqpoint{2.007189in}{1.856344in}}%
\pgfpathlineto{\pgfqpoint{2.001151in}{1.866432in}}%
\pgfpathlineto{\pgfqpoint{1.995117in}{1.876481in}}%
\pgfpathlineto{\pgfqpoint{1.989088in}{1.886492in}}%
\pgfpathlineto{\pgfqpoint{1.977237in}{1.882591in}}%
\pgfpathlineto{\pgfqpoint{1.965381in}{1.878682in}}%
\pgfpathlineto{\pgfqpoint{1.953519in}{1.874766in}}%
\pgfpathlineto{\pgfqpoint{1.941651in}{1.870839in}}%
\pgfpathlineto{\pgfqpoint{1.929778in}{1.866903in}}%
\pgfpathlineto{\pgfqpoint{1.935797in}{1.856849in}}%
\pgfpathlineto{\pgfqpoint{1.941820in}{1.846758in}}%
\pgfpathlineto{\pgfqpoint{1.947848in}{1.836629in}}%
\pgfpathlineto{\pgfqpoint{1.953880in}{1.826461in}}%
\pgfpathclose%
\pgfusepath{stroke,fill}%
\end{pgfscope}%
\begin{pgfscope}%
\pgfpathrectangle{\pgfqpoint{0.887500in}{0.275000in}}{\pgfqpoint{4.225000in}{4.225000in}}%
\pgfusepath{clip}%
\pgfsetbuttcap%
\pgfsetroundjoin%
\definecolor{currentfill}{rgb}{0.272594,0.025563,0.353093}%
\pgfsetfillcolor{currentfill}%
\pgfsetfillopacity{0.700000}%
\pgfsetlinewidth{0.501875pt}%
\definecolor{currentstroke}{rgb}{1.000000,1.000000,1.000000}%
\pgfsetstrokecolor{currentstroke}%
\pgfsetstrokeopacity{0.500000}%
\pgfsetdash{}{0pt}%
\pgfpathmoveto{\pgfqpoint{3.547295in}{1.132033in}}%
\pgfpathlineto{\pgfqpoint{3.558798in}{1.136816in}}%
\pgfpathlineto{\pgfqpoint{3.570286in}{1.140955in}}%
\pgfpathlineto{\pgfqpoint{3.581764in}{1.144674in}}%
\pgfpathlineto{\pgfqpoint{3.593232in}{1.148196in}}%
\pgfpathlineto{\pgfqpoint{3.604696in}{1.151745in}}%
\pgfpathlineto{\pgfqpoint{3.598278in}{1.169891in}}%
\pgfpathlineto{\pgfqpoint{3.591853in}{1.187512in}}%
\pgfpathlineto{\pgfqpoint{3.585424in}{1.204637in}}%
\pgfpathlineto{\pgfqpoint{3.578990in}{1.221294in}}%
\pgfpathlineto{\pgfqpoint{3.572552in}{1.237510in}}%
\pgfpathlineto{\pgfqpoint{3.561076in}{1.233099in}}%
\pgfpathlineto{\pgfqpoint{3.549591in}{1.228431in}}%
\pgfpathlineto{\pgfqpoint{3.538096in}{1.223346in}}%
\pgfpathlineto{\pgfqpoint{3.526589in}{1.217688in}}%
\pgfpathlineto{\pgfqpoint{3.515069in}{1.211297in}}%
\pgfpathlineto{\pgfqpoint{3.521514in}{1.195760in}}%
\pgfpathlineto{\pgfqpoint{3.527959in}{1.180069in}}%
\pgfpathlineto{\pgfqpoint{3.534404in}{1.164220in}}%
\pgfpathlineto{\pgfqpoint{3.540850in}{1.148209in}}%
\pgfpathclose%
\pgfusepath{stroke,fill}%
\end{pgfscope}%
\begin{pgfscope}%
\pgfpathrectangle{\pgfqpoint{0.887500in}{0.275000in}}{\pgfqpoint{4.225000in}{4.225000in}}%
\pgfusepath{clip}%
\pgfsetbuttcap%
\pgfsetroundjoin%
\definecolor{currentfill}{rgb}{0.278791,0.062145,0.386592}%
\pgfsetfillcolor{currentfill}%
\pgfsetfillopacity{0.700000}%
\pgfsetlinewidth{0.501875pt}%
\definecolor{currentstroke}{rgb}{1.000000,1.000000,1.000000}%
\pgfsetstrokecolor{currentstroke}%
\pgfsetstrokeopacity{0.500000}%
\pgfsetdash{}{0pt}%
\pgfpathmoveto{\pgfqpoint{3.457308in}{1.170117in}}%
\pgfpathlineto{\pgfqpoint{3.468867in}{1.178480in}}%
\pgfpathlineto{\pgfqpoint{3.480428in}{1.187204in}}%
\pgfpathlineto{\pgfqpoint{3.491985in}{1.195859in}}%
\pgfpathlineto{\pgfqpoint{3.503534in}{1.204017in}}%
\pgfpathlineto{\pgfqpoint{3.515069in}{1.211297in}}%
\pgfpathlineto{\pgfqpoint{3.508624in}{1.226683in}}%
\pgfpathlineto{\pgfqpoint{3.502181in}{1.241921in}}%
\pgfpathlineto{\pgfqpoint{3.495738in}{1.257017in}}%
\pgfpathlineto{\pgfqpoint{3.489295in}{1.271972in}}%
\pgfpathlineto{\pgfqpoint{3.482854in}{1.286791in}}%
\pgfpathlineto{\pgfqpoint{3.471330in}{1.279997in}}%
\pgfpathlineto{\pgfqpoint{3.459794in}{1.272542in}}%
\pgfpathlineto{\pgfqpoint{3.448251in}{1.264714in}}%
\pgfpathlineto{\pgfqpoint{3.436705in}{1.256838in}}%
\pgfpathlineto{\pgfqpoint{3.425158in}{1.249240in}}%
\pgfpathlineto{\pgfqpoint{3.431580in}{1.233079in}}%
\pgfpathlineto{\pgfqpoint{3.438006in}{1.217042in}}%
\pgfpathlineto{\pgfqpoint{3.444435in}{1.201174in}}%
\pgfpathlineto{\pgfqpoint{3.450869in}{1.185517in}}%
\pgfpathclose%
\pgfusepath{stroke,fill}%
\end{pgfscope}%
\begin{pgfscope}%
\pgfpathrectangle{\pgfqpoint{0.887500in}{0.275000in}}{\pgfqpoint{4.225000in}{4.225000in}}%
\pgfusepath{clip}%
\pgfsetbuttcap%
\pgfsetroundjoin%
\definecolor{currentfill}{rgb}{0.220057,0.343307,0.549413}%
\pgfsetfillcolor{currentfill}%
\pgfsetfillopacity{0.700000}%
\pgfsetlinewidth{0.501875pt}%
\definecolor{currentstroke}{rgb}{1.000000,1.000000,1.000000}%
\pgfsetstrokecolor{currentstroke}%
\pgfsetstrokeopacity{0.500000}%
\pgfsetdash{}{0pt}%
\pgfpathmoveto{\pgfqpoint{2.467802in}{1.670862in}}%
\pgfpathlineto{\pgfqpoint{2.479563in}{1.674893in}}%
\pgfpathlineto{\pgfqpoint{2.491319in}{1.678925in}}%
\pgfpathlineto{\pgfqpoint{2.503069in}{1.682958in}}%
\pgfpathlineto{\pgfqpoint{2.514813in}{1.686995in}}%
\pgfpathlineto{\pgfqpoint{2.526551in}{1.691036in}}%
\pgfpathlineto{\pgfqpoint{2.520337in}{1.702046in}}%
\pgfpathlineto{\pgfqpoint{2.514127in}{1.713011in}}%
\pgfpathlineto{\pgfqpoint{2.507922in}{1.723929in}}%
\pgfpathlineto{\pgfqpoint{2.501720in}{1.734800in}}%
\pgfpathlineto{\pgfqpoint{2.495523in}{1.745625in}}%
\pgfpathlineto{\pgfqpoint{2.483794in}{1.741614in}}%
\pgfpathlineto{\pgfqpoint{2.472059in}{1.737608in}}%
\pgfpathlineto{\pgfqpoint{2.460319in}{1.733605in}}%
\pgfpathlineto{\pgfqpoint{2.448572in}{1.729604in}}%
\pgfpathlineto{\pgfqpoint{2.436820in}{1.725603in}}%
\pgfpathlineto{\pgfqpoint{2.443008in}{1.714745in}}%
\pgfpathlineto{\pgfqpoint{2.449200in}{1.703841in}}%
\pgfpathlineto{\pgfqpoint{2.455397in}{1.692893in}}%
\pgfpathlineto{\pgfqpoint{2.461597in}{1.681899in}}%
\pgfpathclose%
\pgfusepath{stroke,fill}%
\end{pgfscope}%
\begin{pgfscope}%
\pgfpathrectangle{\pgfqpoint{0.887500in}{0.275000in}}{\pgfqpoint{4.225000in}{4.225000in}}%
\pgfusepath{clip}%
\pgfsetbuttcap%
\pgfsetroundjoin%
\definecolor{currentfill}{rgb}{0.282327,0.094955,0.417331}%
\pgfsetfillcolor{currentfill}%
\pgfsetfillopacity{0.700000}%
\pgfsetlinewidth{0.501875pt}%
\definecolor{currentstroke}{rgb}{1.000000,1.000000,1.000000}%
\pgfsetstrokecolor{currentstroke}%
\pgfsetstrokeopacity{0.500000}%
\pgfsetdash{}{0pt}%
\pgfpathmoveto{\pgfqpoint{3.367488in}{1.225916in}}%
\pgfpathlineto{\pgfqpoint{3.379010in}{1.227959in}}%
\pgfpathlineto{\pgfqpoint{3.390538in}{1.231358in}}%
\pgfpathlineto{\pgfqpoint{3.402072in}{1.236176in}}%
\pgfpathlineto{\pgfqpoint{3.413612in}{1.242244in}}%
\pgfpathlineto{\pgfqpoint{3.425158in}{1.249240in}}%
\pgfpathlineto{\pgfqpoint{3.418738in}{1.265480in}}%
\pgfpathlineto{\pgfqpoint{3.412320in}{1.281757in}}%
\pgfpathlineto{\pgfqpoint{3.405905in}{1.298026in}}%
\pgfpathlineto{\pgfqpoint{3.399491in}{1.314243in}}%
\pgfpathlineto{\pgfqpoint{3.393078in}{1.330364in}}%
\pgfpathlineto{\pgfqpoint{3.381549in}{1.324817in}}%
\pgfpathlineto{\pgfqpoint{3.370018in}{1.319724in}}%
\pgfpathlineto{\pgfqpoint{3.358487in}{1.315254in}}%
\pgfpathlineto{\pgfqpoint{3.346957in}{1.311498in}}%
\pgfpathlineto{\pgfqpoint{3.335425in}{1.308426in}}%
\pgfpathlineto{\pgfqpoint{3.341834in}{1.291945in}}%
\pgfpathlineto{\pgfqpoint{3.348244in}{1.275394in}}%
\pgfpathlineto{\pgfqpoint{3.354656in}{1.258834in}}%
\pgfpathlineto{\pgfqpoint{3.361071in}{1.242322in}}%
\pgfpathclose%
\pgfusepath{stroke,fill}%
\end{pgfscope}%
\begin{pgfscope}%
\pgfpathrectangle{\pgfqpoint{0.887500in}{0.275000in}}{\pgfqpoint{4.225000in}{4.225000in}}%
\pgfusepath{clip}%
\pgfsetbuttcap%
\pgfsetroundjoin%
\definecolor{currentfill}{rgb}{0.192357,0.403199,0.555836}%
\pgfsetfillcolor{currentfill}%
\pgfsetfillopacity{0.700000}%
\pgfsetlinewidth{0.501875pt}%
\definecolor{currentstroke}{rgb}{1.000000,1.000000,1.000000}%
\pgfsetstrokecolor{currentstroke}%
\pgfsetstrokeopacity{0.500000}%
\pgfsetdash{}{0pt}%
\pgfpathmoveto{\pgfqpoint{2.049584in}{1.784619in}}%
\pgfpathlineto{\pgfqpoint{2.061450in}{1.788597in}}%
\pgfpathlineto{\pgfqpoint{2.073309in}{1.792570in}}%
\pgfpathlineto{\pgfqpoint{2.085164in}{1.796539in}}%
\pgfpathlineto{\pgfqpoint{2.097012in}{1.800505in}}%
\pgfpathlineto{\pgfqpoint{2.108855in}{1.804469in}}%
\pgfpathlineto{\pgfqpoint{2.102774in}{1.814796in}}%
\pgfpathlineto{\pgfqpoint{2.096699in}{1.825083in}}%
\pgfpathlineto{\pgfqpoint{2.090628in}{1.835329in}}%
\pgfpathlineto{\pgfqpoint{2.084561in}{1.845534in}}%
\pgfpathlineto{\pgfqpoint{2.078499in}{1.855698in}}%
\pgfpathlineto{\pgfqpoint{2.066667in}{1.851776in}}%
\pgfpathlineto{\pgfqpoint{2.054828in}{1.847851in}}%
\pgfpathlineto{\pgfqpoint{2.042984in}{1.843922in}}%
\pgfpathlineto{\pgfqpoint{2.031134in}{1.839990in}}%
\pgfpathlineto{\pgfqpoint{2.019279in}{1.836052in}}%
\pgfpathlineto{\pgfqpoint{2.025331in}{1.825845in}}%
\pgfpathlineto{\pgfqpoint{2.031387in}{1.815599in}}%
\pgfpathlineto{\pgfqpoint{2.037448in}{1.805312in}}%
\pgfpathlineto{\pgfqpoint{2.043514in}{1.794985in}}%
\pgfpathclose%
\pgfusepath{stroke,fill}%
\end{pgfscope}%
\begin{pgfscope}%
\pgfpathrectangle{\pgfqpoint{0.887500in}{0.275000in}}{\pgfqpoint{4.225000in}{4.225000in}}%
\pgfusepath{clip}%
\pgfsetbuttcap%
\pgfsetroundjoin%
\definecolor{currentfill}{rgb}{0.227802,0.326594,0.546532}%
\pgfsetfillcolor{currentfill}%
\pgfsetfillopacity{0.700000}%
\pgfsetlinewidth{0.501875pt}%
\definecolor{currentstroke}{rgb}{1.000000,1.000000,1.000000}%
\pgfsetstrokecolor{currentstroke}%
\pgfsetstrokeopacity{0.500000}%
\pgfsetdash{}{0pt}%
\pgfpathmoveto{\pgfqpoint{2.557680in}{1.635319in}}%
\pgfpathlineto{\pgfqpoint{2.569422in}{1.639387in}}%
\pgfpathlineto{\pgfqpoint{2.581157in}{1.643457in}}%
\pgfpathlineto{\pgfqpoint{2.592887in}{1.647527in}}%
\pgfpathlineto{\pgfqpoint{2.604612in}{1.651596in}}%
\pgfpathlineto{\pgfqpoint{2.616331in}{1.655663in}}%
\pgfpathlineto{\pgfqpoint{2.610088in}{1.666872in}}%
\pgfpathlineto{\pgfqpoint{2.603849in}{1.678037in}}%
\pgfpathlineto{\pgfqpoint{2.597614in}{1.689157in}}%
\pgfpathlineto{\pgfqpoint{2.591384in}{1.700232in}}%
\pgfpathlineto{\pgfqpoint{2.585157in}{1.711261in}}%
\pgfpathlineto{\pgfqpoint{2.573447in}{1.707220in}}%
\pgfpathlineto{\pgfqpoint{2.561731in}{1.703175in}}%
\pgfpathlineto{\pgfqpoint{2.550010in}{1.699128in}}%
\pgfpathlineto{\pgfqpoint{2.538283in}{1.695081in}}%
\pgfpathlineto{\pgfqpoint{2.526551in}{1.691036in}}%
\pgfpathlineto{\pgfqpoint{2.532769in}{1.679980in}}%
\pgfpathlineto{\pgfqpoint{2.538990in}{1.668880in}}%
\pgfpathlineto{\pgfqpoint{2.545216in}{1.657736in}}%
\pgfpathlineto{\pgfqpoint{2.551446in}{1.646549in}}%
\pgfpathclose%
\pgfusepath{stroke,fill}%
\end{pgfscope}%
\begin{pgfscope}%
\pgfpathrectangle{\pgfqpoint{0.887500in}{0.275000in}}{\pgfqpoint{4.225000in}{4.225000in}}%
\pgfusepath{clip}%
\pgfsetbuttcap%
\pgfsetroundjoin%
\definecolor{currentfill}{rgb}{0.235526,0.309527,0.542944}%
\pgfsetfillcolor{currentfill}%
\pgfsetfillopacity{0.700000}%
\pgfsetlinewidth{0.501875pt}%
\definecolor{currentstroke}{rgb}{1.000000,1.000000,1.000000}%
\pgfsetstrokecolor{currentstroke}%
\pgfsetstrokeopacity{0.500000}%
\pgfsetdash{}{0pt}%
\pgfpathmoveto{\pgfqpoint{2.647602in}{1.598980in}}%
\pgfpathlineto{\pgfqpoint{2.659324in}{1.603065in}}%
\pgfpathlineto{\pgfqpoint{2.671041in}{1.607145in}}%
\pgfpathlineto{\pgfqpoint{2.682751in}{1.611220in}}%
\pgfpathlineto{\pgfqpoint{2.694456in}{1.615288in}}%
\pgfpathlineto{\pgfqpoint{2.706156in}{1.619346in}}%
\pgfpathlineto{\pgfqpoint{2.699885in}{1.630745in}}%
\pgfpathlineto{\pgfqpoint{2.693618in}{1.642097in}}%
\pgfpathlineto{\pgfqpoint{2.687355in}{1.653406in}}%
\pgfpathlineto{\pgfqpoint{2.681096in}{1.664673in}}%
\pgfpathlineto{\pgfqpoint{2.674841in}{1.675897in}}%
\pgfpathlineto{\pgfqpoint{2.663150in}{1.671869in}}%
\pgfpathlineto{\pgfqpoint{2.651454in}{1.667829in}}%
\pgfpathlineto{\pgfqpoint{2.639752in}{1.663781in}}%
\pgfpathlineto{\pgfqpoint{2.628044in}{1.659725in}}%
\pgfpathlineto{\pgfqpoint{2.616331in}{1.655663in}}%
\pgfpathlineto{\pgfqpoint{2.622577in}{1.644411in}}%
\pgfpathlineto{\pgfqpoint{2.628828in}{1.633117in}}%
\pgfpathlineto{\pgfqpoint{2.635082in}{1.621781in}}%
\pgfpathlineto{\pgfqpoint{2.641340in}{1.610403in}}%
\pgfpathclose%
\pgfusepath{stroke,fill}%
\end{pgfscope}%
\begin{pgfscope}%
\pgfpathrectangle{\pgfqpoint{0.887500in}{0.275000in}}{\pgfqpoint{4.225000in}{4.225000in}}%
\pgfusepath{clip}%
\pgfsetbuttcap%
\pgfsetroundjoin%
\definecolor{currentfill}{rgb}{0.199430,0.387607,0.554642}%
\pgfsetfillcolor{currentfill}%
\pgfsetfillopacity{0.700000}%
\pgfsetlinewidth{0.501875pt}%
\definecolor{currentstroke}{rgb}{1.000000,1.000000,1.000000}%
\pgfsetstrokecolor{currentstroke}%
\pgfsetstrokeopacity{0.500000}%
\pgfsetdash{}{0pt}%
\pgfpathmoveto{\pgfqpoint{2.139322in}{1.752229in}}%
\pgfpathlineto{\pgfqpoint{2.151169in}{1.756229in}}%
\pgfpathlineto{\pgfqpoint{2.163010in}{1.760227in}}%
\pgfpathlineto{\pgfqpoint{2.174845in}{1.764224in}}%
\pgfpathlineto{\pgfqpoint{2.186674in}{1.768222in}}%
\pgfpathlineto{\pgfqpoint{2.198498in}{1.772221in}}%
\pgfpathlineto{\pgfqpoint{2.192386in}{1.782712in}}%
\pgfpathlineto{\pgfqpoint{2.186278in}{1.793164in}}%
\pgfpathlineto{\pgfqpoint{2.180175in}{1.803574in}}%
\pgfpathlineto{\pgfqpoint{2.174076in}{1.813944in}}%
\pgfpathlineto{\pgfqpoint{2.167981in}{1.824273in}}%
\pgfpathlineto{\pgfqpoint{2.156167in}{1.820312in}}%
\pgfpathlineto{\pgfqpoint{2.144348in}{1.816351in}}%
\pgfpathlineto{\pgfqpoint{2.132522in}{1.812391in}}%
\pgfpathlineto{\pgfqpoint{2.120691in}{1.808431in}}%
\pgfpathlineto{\pgfqpoint{2.108855in}{1.804469in}}%
\pgfpathlineto{\pgfqpoint{2.114939in}{1.794101in}}%
\pgfpathlineto{\pgfqpoint{2.121028in}{1.783693in}}%
\pgfpathlineto{\pgfqpoint{2.127122in}{1.773245in}}%
\pgfpathlineto{\pgfqpoint{2.133220in}{1.762757in}}%
\pgfpathclose%
\pgfusepath{stroke,fill}%
\end{pgfscope}%
\begin{pgfscope}%
\pgfpathrectangle{\pgfqpoint{0.887500in}{0.275000in}}{\pgfqpoint{4.225000in}{4.225000in}}%
\pgfusepath{clip}%
\pgfsetbuttcap%
\pgfsetroundjoin%
\definecolor{currentfill}{rgb}{0.243113,0.292092,0.538516}%
\pgfsetfillcolor{currentfill}%
\pgfsetfillopacity{0.700000}%
\pgfsetlinewidth{0.501875pt}%
\definecolor{currentstroke}{rgb}{1.000000,1.000000,1.000000}%
\pgfsetstrokecolor{currentstroke}%
\pgfsetstrokeopacity{0.500000}%
\pgfsetdash{}{0pt}%
\pgfpathmoveto{\pgfqpoint{2.737566in}{1.561550in}}%
\pgfpathlineto{\pgfqpoint{2.749268in}{1.565615in}}%
\pgfpathlineto{\pgfqpoint{2.760965in}{1.569670in}}%
\pgfpathlineto{\pgfqpoint{2.772656in}{1.573712in}}%
\pgfpathlineto{\pgfqpoint{2.784341in}{1.577740in}}%
\pgfpathlineto{\pgfqpoint{2.796021in}{1.581756in}}%
\pgfpathlineto{\pgfqpoint{2.789723in}{1.593420in}}%
\pgfpathlineto{\pgfqpoint{2.783429in}{1.605017in}}%
\pgfpathlineto{\pgfqpoint{2.777139in}{1.616552in}}%
\pgfpathlineto{\pgfqpoint{2.770852in}{1.628028in}}%
\pgfpathlineto{\pgfqpoint{2.764569in}{1.639448in}}%
\pgfpathlineto{\pgfqpoint{2.752898in}{1.635456in}}%
\pgfpathlineto{\pgfqpoint{2.741221in}{1.631450in}}%
\pgfpathlineto{\pgfqpoint{2.729538in}{1.627429in}}%
\pgfpathlineto{\pgfqpoint{2.717849in}{1.623394in}}%
\pgfpathlineto{\pgfqpoint{2.706156in}{1.619346in}}%
\pgfpathlineto{\pgfqpoint{2.712430in}{1.607899in}}%
\pgfpathlineto{\pgfqpoint{2.718708in}{1.596398in}}%
\pgfpathlineto{\pgfqpoint{2.724990in}{1.584842in}}%
\pgfpathlineto{\pgfqpoint{2.731276in}{1.573227in}}%
\pgfpathclose%
\pgfusepath{stroke,fill}%
\end{pgfscope}%
\begin{pgfscope}%
\pgfpathrectangle{\pgfqpoint{0.887500in}{0.275000in}}{\pgfqpoint{4.225000in}{4.225000in}}%
\pgfusepath{clip}%
\pgfsetbuttcap%
\pgfsetroundjoin%
\definecolor{currentfill}{rgb}{0.282623,0.140926,0.457517}%
\pgfsetfillcolor{currentfill}%
\pgfsetfillopacity{0.700000}%
\pgfsetlinewidth{0.501875pt}%
\definecolor{currentstroke}{rgb}{1.000000,1.000000,1.000000}%
\pgfsetstrokecolor{currentstroke}%
\pgfsetstrokeopacity{0.500000}%
\pgfsetdash{}{0pt}%
\pgfpathmoveto{\pgfqpoint{3.277726in}{1.301980in}}%
\pgfpathlineto{\pgfqpoint{3.289274in}{1.302212in}}%
\pgfpathlineto{\pgfqpoint{3.300817in}{1.302932in}}%
\pgfpathlineto{\pgfqpoint{3.312356in}{1.304182in}}%
\pgfpathlineto{\pgfqpoint{3.323892in}{1.306001in}}%
\pgfpathlineto{\pgfqpoint{3.335425in}{1.308426in}}%
\pgfpathlineto{\pgfqpoint{3.329017in}{1.324781in}}%
\pgfpathlineto{\pgfqpoint{3.322610in}{1.340951in}}%
\pgfpathlineto{\pgfqpoint{3.316203in}{1.356877in}}%
\pgfpathlineto{\pgfqpoint{3.309796in}{1.372500in}}%
\pgfpathlineto{\pgfqpoint{3.303389in}{1.387762in}}%
\pgfpathlineto{\pgfqpoint{3.291848in}{1.384002in}}%
\pgfpathlineto{\pgfqpoint{3.280301in}{1.380325in}}%
\pgfpathlineto{\pgfqpoint{3.268749in}{1.376722in}}%
\pgfpathlineto{\pgfqpoint{3.257191in}{1.373186in}}%
\pgfpathlineto{\pgfqpoint{3.245628in}{1.369706in}}%
\pgfpathlineto{\pgfqpoint{3.252043in}{1.356269in}}%
\pgfpathlineto{\pgfqpoint{3.258459in}{1.342774in}}%
\pgfpathlineto{\pgfqpoint{3.264879in}{1.329225in}}%
\pgfpathlineto{\pgfqpoint{3.271301in}{1.315626in}}%
\pgfpathclose%
\pgfusepath{stroke,fill}%
\end{pgfscope}%
\begin{pgfscope}%
\pgfpathrectangle{\pgfqpoint{0.887500in}{0.275000in}}{\pgfqpoint{4.225000in}{4.225000in}}%
\pgfusepath{clip}%
\pgfsetbuttcap%
\pgfsetroundjoin%
\definecolor{currentfill}{rgb}{0.252194,0.269783,0.531579}%
\pgfsetfillcolor{currentfill}%
\pgfsetfillopacity{0.700000}%
\pgfsetlinewidth{0.501875pt}%
\definecolor{currentstroke}{rgb}{1.000000,1.000000,1.000000}%
\pgfsetstrokecolor{currentstroke}%
\pgfsetstrokeopacity{0.500000}%
\pgfsetdash{}{0pt}%
\pgfpathmoveto{\pgfqpoint{2.827568in}{1.522334in}}%
\pgfpathlineto{\pgfqpoint{2.839250in}{1.526361in}}%
\pgfpathlineto{\pgfqpoint{2.850926in}{1.530388in}}%
\pgfpathlineto{\pgfqpoint{2.862597in}{1.534414in}}%
\pgfpathlineto{\pgfqpoint{2.874262in}{1.538443in}}%
\pgfpathlineto{\pgfqpoint{2.885921in}{1.542476in}}%
\pgfpathlineto{\pgfqpoint{2.879596in}{1.554484in}}%
\pgfpathlineto{\pgfqpoint{2.873276in}{1.566420in}}%
\pgfpathlineto{\pgfqpoint{2.866959in}{1.578279in}}%
\pgfpathlineto{\pgfqpoint{2.860645in}{1.590061in}}%
\pgfpathlineto{\pgfqpoint{2.854335in}{1.601770in}}%
\pgfpathlineto{\pgfqpoint{2.842684in}{1.597766in}}%
\pgfpathlineto{\pgfqpoint{2.831027in}{1.593765in}}%
\pgfpathlineto{\pgfqpoint{2.819364in}{1.589766in}}%
\pgfpathlineto{\pgfqpoint{2.807695in}{1.585764in}}%
\pgfpathlineto{\pgfqpoint{2.796021in}{1.581756in}}%
\pgfpathlineto{\pgfqpoint{2.802323in}{1.570023in}}%
\pgfpathlineto{\pgfqpoint{2.808629in}{1.558215in}}%
\pgfpathlineto{\pgfqpoint{2.814938in}{1.546331in}}%
\pgfpathlineto{\pgfqpoint{2.821251in}{1.534368in}}%
\pgfpathclose%
\pgfusepath{stroke,fill}%
\end{pgfscope}%
\begin{pgfscope}%
\pgfpathrectangle{\pgfqpoint{0.887500in}{0.275000in}}{\pgfqpoint{4.225000in}{4.225000in}}%
\pgfusepath{clip}%
\pgfsetbuttcap%
\pgfsetroundjoin%
\definecolor{currentfill}{rgb}{0.204903,0.375746,0.553533}%
\pgfsetfillcolor{currentfill}%
\pgfsetfillopacity{0.700000}%
\pgfsetlinewidth{0.501875pt}%
\definecolor{currentstroke}{rgb}{1.000000,1.000000,1.000000}%
\pgfsetstrokecolor{currentstroke}%
\pgfsetstrokeopacity{0.500000}%
\pgfsetdash{}{0pt}%
\pgfpathmoveto{\pgfqpoint{2.229125in}{1.719156in}}%
\pgfpathlineto{\pgfqpoint{2.240952in}{1.723186in}}%
\pgfpathlineto{\pgfqpoint{2.252774in}{1.727215in}}%
\pgfpathlineto{\pgfqpoint{2.264590in}{1.731243in}}%
\pgfpathlineto{\pgfqpoint{2.276401in}{1.735268in}}%
\pgfpathlineto{\pgfqpoint{2.288205in}{1.739290in}}%
\pgfpathlineto{\pgfqpoint{2.282062in}{1.749953in}}%
\pgfpathlineto{\pgfqpoint{2.275922in}{1.760574in}}%
\pgfpathlineto{\pgfqpoint{2.269787in}{1.771153in}}%
\pgfpathlineto{\pgfqpoint{2.263657in}{1.781692in}}%
\pgfpathlineto{\pgfqpoint{2.257530in}{1.792188in}}%
\pgfpathlineto{\pgfqpoint{2.245735in}{1.788201in}}%
\pgfpathlineto{\pgfqpoint{2.233934in}{1.784209in}}%
\pgfpathlineto{\pgfqpoint{2.222128in}{1.780215in}}%
\pgfpathlineto{\pgfqpoint{2.210316in}{1.776218in}}%
\pgfpathlineto{\pgfqpoint{2.198498in}{1.772221in}}%
\pgfpathlineto{\pgfqpoint{2.204615in}{1.761688in}}%
\pgfpathlineto{\pgfqpoint{2.210736in}{1.751116in}}%
\pgfpathlineto{\pgfqpoint{2.216861in}{1.740503in}}%
\pgfpathlineto{\pgfqpoint{2.222991in}{1.729849in}}%
\pgfpathclose%
\pgfusepath{stroke,fill}%
\end{pgfscope}%
\begin{pgfscope}%
\pgfpathrectangle{\pgfqpoint{0.887500in}{0.275000in}}{\pgfqpoint{4.225000in}{4.225000in}}%
\pgfusepath{clip}%
\pgfsetbuttcap%
\pgfsetroundjoin%
\definecolor{currentfill}{rgb}{0.258965,0.251537,0.524736}%
\pgfsetfillcolor{currentfill}%
\pgfsetfillopacity{0.700000}%
\pgfsetlinewidth{0.501875pt}%
\definecolor{currentstroke}{rgb}{1.000000,1.000000,1.000000}%
\pgfsetstrokecolor{currentstroke}%
\pgfsetstrokeopacity{0.500000}%
\pgfsetdash{}{0pt}%
\pgfpathmoveto{\pgfqpoint{2.917594in}{1.481675in}}%
\pgfpathlineto{\pgfqpoint{2.929255in}{1.485726in}}%
\pgfpathlineto{\pgfqpoint{2.940910in}{1.489787in}}%
\pgfpathlineto{\pgfqpoint{2.952559in}{1.493863in}}%
\pgfpathlineto{\pgfqpoint{2.964203in}{1.497959in}}%
\pgfpathlineto{\pgfqpoint{2.975841in}{1.502079in}}%
\pgfpathlineto{\pgfqpoint{2.969492in}{1.514320in}}%
\pgfpathlineto{\pgfqpoint{2.963147in}{1.526512in}}%
\pgfpathlineto{\pgfqpoint{2.956805in}{1.538652in}}%
\pgfpathlineto{\pgfqpoint{2.950466in}{1.550734in}}%
\pgfpathlineto{\pgfqpoint{2.944131in}{1.562757in}}%
\pgfpathlineto{\pgfqpoint{2.932500in}{1.558680in}}%
\pgfpathlineto{\pgfqpoint{2.920864in}{1.554616in}}%
\pgfpathlineto{\pgfqpoint{2.909222in}{1.550561in}}%
\pgfpathlineto{\pgfqpoint{2.897574in}{1.546515in}}%
\pgfpathlineto{\pgfqpoint{2.885921in}{1.542476in}}%
\pgfpathlineto{\pgfqpoint{2.892249in}{1.530406in}}%
\pgfpathlineto{\pgfqpoint{2.898580in}{1.518282in}}%
\pgfpathlineto{\pgfqpoint{2.904915in}{1.506113in}}%
\pgfpathlineto{\pgfqpoint{2.911253in}{1.493908in}}%
\pgfpathclose%
\pgfusepath{stroke,fill}%
\end{pgfscope}%
\begin{pgfscope}%
\pgfpathrectangle{\pgfqpoint{0.887500in}{0.275000in}}{\pgfqpoint{4.225000in}{4.225000in}}%
\pgfusepath{clip}%
\pgfsetbuttcap%
\pgfsetroundjoin%
\definecolor{currentfill}{rgb}{0.279574,0.170599,0.479997}%
\pgfsetfillcolor{currentfill}%
\pgfsetfillopacity{0.700000}%
\pgfsetlinewidth{0.501875pt}%
\definecolor{currentstroke}{rgb}{1.000000,1.000000,1.000000}%
\pgfsetstrokecolor{currentstroke}%
\pgfsetstrokeopacity{0.500000}%
\pgfsetdash{}{0pt}%
\pgfpathmoveto{\pgfqpoint{3.187722in}{1.352224in}}%
\pgfpathlineto{\pgfqpoint{3.199316in}{1.355837in}}%
\pgfpathlineto{\pgfqpoint{3.210903in}{1.359360in}}%
\pgfpathlineto{\pgfqpoint{3.222484in}{1.362825in}}%
\pgfpathlineto{\pgfqpoint{3.234059in}{1.366263in}}%
\pgfpathlineto{\pgfqpoint{3.245628in}{1.369706in}}%
\pgfpathlineto{\pgfqpoint{3.239217in}{1.383080in}}%
\pgfpathlineto{\pgfqpoint{3.232807in}{1.396388in}}%
\pgfpathlineto{\pgfqpoint{3.226401in}{1.409629in}}%
\pgfpathlineto{\pgfqpoint{3.219997in}{1.422799in}}%
\pgfpathlineto{\pgfqpoint{3.213596in}{1.435897in}}%
\pgfpathlineto{\pgfqpoint{3.202028in}{1.431294in}}%
\pgfpathlineto{\pgfqpoint{3.190454in}{1.426681in}}%
\pgfpathlineto{\pgfqpoint{3.178875in}{1.422107in}}%
\pgfpathlineto{\pgfqpoint{3.167291in}{1.417621in}}%
\pgfpathlineto{\pgfqpoint{3.155702in}{1.413272in}}%
\pgfpathlineto{\pgfqpoint{3.162097in}{1.400365in}}%
\pgfpathlineto{\pgfqpoint{3.168497in}{1.387740in}}%
\pgfpathlineto{\pgfqpoint{3.174900in}{1.375466in}}%
\pgfpathlineto{\pgfqpoint{3.181309in}{1.363611in}}%
\pgfpathclose%
\pgfusepath{stroke,fill}%
\end{pgfscope}%
\begin{pgfscope}%
\pgfpathrectangle{\pgfqpoint{0.887500in}{0.275000in}}{\pgfqpoint{4.225000in}{4.225000in}}%
\pgfusepath{clip}%
\pgfsetbuttcap%
\pgfsetroundjoin%
\definecolor{currentfill}{rgb}{0.269944,0.014625,0.341379}%
\pgfsetfillcolor{currentfill}%
\pgfsetfillopacity{0.700000}%
\pgfsetlinewidth{0.501875pt}%
\definecolor{currentstroke}{rgb}{1.000000,1.000000,1.000000}%
\pgfsetstrokecolor{currentstroke}%
\pgfsetstrokeopacity{0.500000}%
\pgfsetdash{}{0pt}%
\pgfpathmoveto{\pgfqpoint{3.489596in}{1.098481in}}%
\pgfpathlineto{\pgfqpoint{3.501147in}{1.105471in}}%
\pgfpathlineto{\pgfqpoint{3.512698in}{1.112723in}}%
\pgfpathlineto{\pgfqpoint{3.524243in}{1.119830in}}%
\pgfpathlineto{\pgfqpoint{3.535777in}{1.126385in}}%
\pgfpathlineto{\pgfqpoint{3.547295in}{1.132033in}}%
\pgfpathlineto{\pgfqpoint{3.540850in}{1.148209in}}%
\pgfpathlineto{\pgfqpoint{3.534404in}{1.164220in}}%
\pgfpathlineto{\pgfqpoint{3.527959in}{1.180069in}}%
\pgfpathlineto{\pgfqpoint{3.521514in}{1.195760in}}%
\pgfpathlineto{\pgfqpoint{3.515069in}{1.211297in}}%
\pgfpathlineto{\pgfqpoint{3.503534in}{1.204017in}}%
\pgfpathlineto{\pgfqpoint{3.491985in}{1.195859in}}%
\pgfpathlineto{\pgfqpoint{3.480428in}{1.187204in}}%
\pgfpathlineto{\pgfqpoint{3.468867in}{1.178480in}}%
\pgfpathlineto{\pgfqpoint{3.457308in}{1.170117in}}%
\pgfpathlineto{\pgfqpoint{3.463752in}{1.155015in}}%
\pgfpathlineto{\pgfqpoint{3.470202in}{1.140257in}}%
\pgfpathlineto{\pgfqpoint{3.476659in}{1.125887in}}%
\pgfpathlineto{\pgfqpoint{3.483124in}{1.111947in}}%
\pgfpathclose%
\pgfusepath{stroke,fill}%
\end{pgfscope}%
\begin{pgfscope}%
\pgfpathrectangle{\pgfqpoint{0.887500in}{0.275000in}}{\pgfqpoint{4.225000in}{4.225000in}}%
\pgfusepath{clip}%
\pgfsetbuttcap%
\pgfsetroundjoin%
\definecolor{currentfill}{rgb}{0.274128,0.199721,0.498911}%
\pgfsetfillcolor{currentfill}%
\pgfsetfillopacity{0.700000}%
\pgfsetlinewidth{0.501875pt}%
\definecolor{currentstroke}{rgb}{1.000000,1.000000,1.000000}%
\pgfsetstrokecolor{currentstroke}%
\pgfsetstrokeopacity{0.500000}%
\pgfsetdash{}{0pt}%
\pgfpathmoveto{\pgfqpoint{3.097678in}{1.394498in}}%
\pgfpathlineto{\pgfqpoint{3.109294in}{1.397929in}}%
\pgfpathlineto{\pgfqpoint{3.120904in}{1.401471in}}%
\pgfpathlineto{\pgfqpoint{3.132508in}{1.405181in}}%
\pgfpathlineto{\pgfqpoint{3.144108in}{1.409110in}}%
\pgfpathlineto{\pgfqpoint{3.155702in}{1.413272in}}%
\pgfpathlineto{\pgfqpoint{3.149310in}{1.426393in}}%
\pgfpathlineto{\pgfqpoint{3.142921in}{1.439660in}}%
\pgfpathlineto{\pgfqpoint{3.136535in}{1.453005in}}%
\pgfpathlineto{\pgfqpoint{3.130153in}{1.466359in}}%
\pgfpathlineto{\pgfqpoint{3.123773in}{1.479653in}}%
\pgfpathlineto{\pgfqpoint{3.112184in}{1.475607in}}%
\pgfpathlineto{\pgfqpoint{3.100590in}{1.471609in}}%
\pgfpathlineto{\pgfqpoint{3.088991in}{1.467655in}}%
\pgfpathlineto{\pgfqpoint{3.077385in}{1.463734in}}%
\pgfpathlineto{\pgfqpoint{3.065774in}{1.459836in}}%
\pgfpathlineto{\pgfqpoint{3.072149in}{1.446791in}}%
\pgfpathlineto{\pgfqpoint{3.078527in}{1.433690in}}%
\pgfpathlineto{\pgfqpoint{3.084907in}{1.420578in}}%
\pgfpathlineto{\pgfqpoint{3.091291in}{1.407499in}}%
\pgfpathclose%
\pgfusepath{stroke,fill}%
\end{pgfscope}%
\begin{pgfscope}%
\pgfpathrectangle{\pgfqpoint{0.887500in}{0.275000in}}{\pgfqpoint{4.225000in}{4.225000in}}%
\pgfusepath{clip}%
\pgfsetbuttcap%
\pgfsetroundjoin%
\definecolor{currentfill}{rgb}{0.276022,0.044167,0.370164}%
\pgfsetfillcolor{currentfill}%
\pgfsetfillopacity{0.700000}%
\pgfsetlinewidth{0.501875pt}%
\definecolor{currentstroke}{rgb}{1.000000,1.000000,1.000000}%
\pgfsetstrokecolor{currentstroke}%
\pgfsetstrokeopacity{0.500000}%
\pgfsetdash{}{0pt}%
\pgfpathmoveto{\pgfqpoint{3.399635in}{1.147517in}}%
\pgfpathlineto{\pgfqpoint{3.411149in}{1.148618in}}%
\pgfpathlineto{\pgfqpoint{3.422672in}{1.151474in}}%
\pgfpathlineto{\pgfqpoint{3.434206in}{1.156186in}}%
\pgfpathlineto{\pgfqpoint{3.445753in}{1.162543in}}%
\pgfpathlineto{\pgfqpoint{3.457308in}{1.170117in}}%
\pgfpathlineto{\pgfqpoint{3.450869in}{1.185517in}}%
\pgfpathlineto{\pgfqpoint{3.444435in}{1.201174in}}%
\pgfpathlineto{\pgfqpoint{3.438006in}{1.217042in}}%
\pgfpathlineto{\pgfqpoint{3.431580in}{1.233079in}}%
\pgfpathlineto{\pgfqpoint{3.425158in}{1.249240in}}%
\pgfpathlineto{\pgfqpoint{3.413612in}{1.242244in}}%
\pgfpathlineto{\pgfqpoint{3.402072in}{1.236176in}}%
\pgfpathlineto{\pgfqpoint{3.390538in}{1.231358in}}%
\pgfpathlineto{\pgfqpoint{3.379010in}{1.227959in}}%
\pgfpathlineto{\pgfqpoint{3.367488in}{1.225916in}}%
\pgfpathlineto{\pgfqpoint{3.373908in}{1.209675in}}%
\pgfpathlineto{\pgfqpoint{3.380333in}{1.193656in}}%
\pgfpathlineto{\pgfqpoint{3.386761in}{1.177919in}}%
\pgfpathlineto{\pgfqpoint{3.393195in}{1.162520in}}%
\pgfpathclose%
\pgfusepath{stroke,fill}%
\end{pgfscope}%
\begin{pgfscope}%
\pgfpathrectangle{\pgfqpoint{0.887500in}{0.275000in}}{\pgfqpoint{4.225000in}{4.225000in}}%
\pgfusepath{clip}%
\pgfsetbuttcap%
\pgfsetroundjoin%
\definecolor{currentfill}{rgb}{0.266580,0.228262,0.514349}%
\pgfsetfillcolor{currentfill}%
\pgfsetfillopacity{0.700000}%
\pgfsetlinewidth{0.501875pt}%
\definecolor{currentstroke}{rgb}{1.000000,1.000000,1.000000}%
\pgfsetstrokecolor{currentstroke}%
\pgfsetstrokeopacity{0.500000}%
\pgfsetdash{}{0pt}%
\pgfpathmoveto{\pgfqpoint{3.007634in}{1.440290in}}%
\pgfpathlineto{\pgfqpoint{3.019273in}{1.444236in}}%
\pgfpathlineto{\pgfqpoint{3.030907in}{1.448159in}}%
\pgfpathlineto{\pgfqpoint{3.042535in}{1.452060in}}%
\pgfpathlineto{\pgfqpoint{3.054158in}{1.455948in}}%
\pgfpathlineto{\pgfqpoint{3.065774in}{1.459836in}}%
\pgfpathlineto{\pgfqpoint{3.059403in}{1.472781in}}%
\pgfpathlineto{\pgfqpoint{3.053034in}{1.485582in}}%
\pgfpathlineto{\pgfqpoint{3.046668in}{1.498226in}}%
\pgfpathlineto{\pgfqpoint{3.040306in}{1.510724in}}%
\pgfpathlineto{\pgfqpoint{3.033947in}{1.523088in}}%
\pgfpathlineto{\pgfqpoint{3.022337in}{1.518861in}}%
\pgfpathlineto{\pgfqpoint{3.010721in}{1.514631in}}%
\pgfpathlineto{\pgfqpoint{2.999100in}{1.510414in}}%
\pgfpathlineto{\pgfqpoint{2.987473in}{1.506230in}}%
\pgfpathlineto{\pgfqpoint{2.975841in}{1.502079in}}%
\pgfpathlineto{\pgfqpoint{2.982193in}{1.489795in}}%
\pgfpathlineto{\pgfqpoint{2.988548in}{1.477470in}}%
\pgfpathlineto{\pgfqpoint{2.994907in}{1.465110in}}%
\pgfpathlineto{\pgfqpoint{3.001269in}{1.452717in}}%
\pgfpathclose%
\pgfusepath{stroke,fill}%
\end{pgfscope}%
\begin{pgfscope}%
\pgfpathrectangle{\pgfqpoint{0.887500in}{0.275000in}}{\pgfqpoint{4.225000in}{4.225000in}}%
\pgfusepath{clip}%
\pgfsetbuttcap%
\pgfsetroundjoin%
\definecolor{currentfill}{rgb}{0.212395,0.359683,0.551710}%
\pgfsetfillcolor{currentfill}%
\pgfsetfillopacity{0.700000}%
\pgfsetlinewidth{0.501875pt}%
\definecolor{currentstroke}{rgb}{1.000000,1.000000,1.000000}%
\pgfsetstrokecolor{currentstroke}%
\pgfsetstrokeopacity{0.500000}%
\pgfsetdash{}{0pt}%
\pgfpathmoveto{\pgfqpoint{2.318988in}{1.685348in}}%
\pgfpathlineto{\pgfqpoint{2.330797in}{1.689394in}}%
\pgfpathlineto{\pgfqpoint{2.342600in}{1.693437in}}%
\pgfpathlineto{\pgfqpoint{2.354397in}{1.697476in}}%
\pgfpathlineto{\pgfqpoint{2.366189in}{1.701510in}}%
\pgfpathlineto{\pgfqpoint{2.377975in}{1.705540in}}%
\pgfpathlineto{\pgfqpoint{2.371800in}{1.716385in}}%
\pgfpathlineto{\pgfqpoint{2.365630in}{1.727187in}}%
\pgfpathlineto{\pgfqpoint{2.359464in}{1.737945in}}%
\pgfpathlineto{\pgfqpoint{2.353302in}{1.748660in}}%
\pgfpathlineto{\pgfqpoint{2.347144in}{1.759332in}}%
\pgfpathlineto{\pgfqpoint{2.335368in}{1.755334in}}%
\pgfpathlineto{\pgfqpoint{2.323586in}{1.751331in}}%
\pgfpathlineto{\pgfqpoint{2.311798in}{1.747322in}}%
\pgfpathlineto{\pgfqpoint{2.300005in}{1.743308in}}%
\pgfpathlineto{\pgfqpoint{2.288205in}{1.739290in}}%
\pgfpathlineto{\pgfqpoint{2.294353in}{1.728585in}}%
\pgfpathlineto{\pgfqpoint{2.300506in}{1.717839in}}%
\pgfpathlineto{\pgfqpoint{2.306662in}{1.707051in}}%
\pgfpathlineto{\pgfqpoint{2.312823in}{1.696220in}}%
\pgfpathclose%
\pgfusepath{stroke,fill}%
\end{pgfscope}%
\begin{pgfscope}%
\pgfpathrectangle{\pgfqpoint{0.887500in}{0.275000in}}{\pgfqpoint{4.225000in}{4.225000in}}%
\pgfusepath{clip}%
\pgfsetbuttcap%
\pgfsetroundjoin%
\definecolor{currentfill}{rgb}{0.220057,0.343307,0.549413}%
\pgfsetfillcolor{currentfill}%
\pgfsetfillopacity{0.700000}%
\pgfsetlinewidth{0.501875pt}%
\definecolor{currentstroke}{rgb}{1.000000,1.000000,1.000000}%
\pgfsetstrokecolor{currentstroke}%
\pgfsetstrokeopacity{0.500000}%
\pgfsetdash{}{0pt}%
\pgfpathmoveto{\pgfqpoint{2.408910in}{1.650659in}}%
\pgfpathlineto{\pgfqpoint{2.420700in}{1.654709in}}%
\pgfpathlineto{\pgfqpoint{2.432484in}{1.658754in}}%
\pgfpathlineto{\pgfqpoint{2.444262in}{1.662794in}}%
\pgfpathlineto{\pgfqpoint{2.456035in}{1.666829in}}%
\pgfpathlineto{\pgfqpoint{2.467802in}{1.670862in}}%
\pgfpathlineto{\pgfqpoint{2.461597in}{1.681899in}}%
\pgfpathlineto{\pgfqpoint{2.455397in}{1.692893in}}%
\pgfpathlineto{\pgfqpoint{2.449200in}{1.703841in}}%
\pgfpathlineto{\pgfqpoint{2.443008in}{1.714745in}}%
\pgfpathlineto{\pgfqpoint{2.436820in}{1.725603in}}%
\pgfpathlineto{\pgfqpoint{2.425062in}{1.721600in}}%
\pgfpathlineto{\pgfqpoint{2.413299in}{1.717594in}}%
\pgfpathlineto{\pgfqpoint{2.401530in}{1.713582in}}%
\pgfpathlineto{\pgfqpoint{2.389755in}{1.709564in}}%
\pgfpathlineto{\pgfqpoint{2.377975in}{1.705540in}}%
\pgfpathlineto{\pgfqpoint{2.384153in}{1.694650in}}%
\pgfpathlineto{\pgfqpoint{2.390336in}{1.683718in}}%
\pgfpathlineto{\pgfqpoint{2.396524in}{1.672741in}}%
\pgfpathlineto{\pgfqpoint{2.402715in}{1.661721in}}%
\pgfpathclose%
\pgfusepath{stroke,fill}%
\end{pgfscope}%
\begin{pgfscope}%
\pgfpathrectangle{\pgfqpoint{0.887500in}{0.275000in}}{\pgfqpoint{4.225000in}{4.225000in}}%
\pgfusepath{clip}%
\pgfsetbuttcap%
\pgfsetroundjoin%
\definecolor{currentfill}{rgb}{0.194100,0.399323,0.555565}%
\pgfsetfillcolor{currentfill}%
\pgfsetfillopacity{0.700000}%
\pgfsetlinewidth{0.501875pt}%
\definecolor{currentstroke}{rgb}{1.000000,1.000000,1.000000}%
\pgfsetstrokecolor{currentstroke}%
\pgfsetstrokeopacity{0.500000}%
\pgfsetdash{}{0pt}%
\pgfpathmoveto{\pgfqpoint{1.990172in}{1.764631in}}%
\pgfpathlineto{\pgfqpoint{2.002066in}{1.768644in}}%
\pgfpathlineto{\pgfqpoint{2.013954in}{1.772648in}}%
\pgfpathlineto{\pgfqpoint{2.025836in}{1.776645in}}%
\pgfpathlineto{\pgfqpoint{2.037713in}{1.780635in}}%
\pgfpathlineto{\pgfqpoint{2.049584in}{1.784619in}}%
\pgfpathlineto{\pgfqpoint{2.043514in}{1.794985in}}%
\pgfpathlineto{\pgfqpoint{2.037448in}{1.805312in}}%
\pgfpathlineto{\pgfqpoint{2.031387in}{1.815599in}}%
\pgfpathlineto{\pgfqpoint{2.025331in}{1.825845in}}%
\pgfpathlineto{\pgfqpoint{2.019279in}{1.836052in}}%
\pgfpathlineto{\pgfqpoint{2.007418in}{1.832107in}}%
\pgfpathlineto{\pgfqpoint{1.995551in}{1.828157in}}%
\pgfpathlineto{\pgfqpoint{1.983679in}{1.824198in}}%
\pgfpathlineto{\pgfqpoint{1.971801in}{1.820231in}}%
\pgfpathlineto{\pgfqpoint{1.959918in}{1.816254in}}%
\pgfpathlineto{\pgfqpoint{1.965959in}{1.806009in}}%
\pgfpathlineto{\pgfqpoint{1.972006in}{1.795723in}}%
\pgfpathlineto{\pgfqpoint{1.978057in}{1.785398in}}%
\pgfpathlineto{\pgfqpoint{1.984112in}{1.775034in}}%
\pgfpathclose%
\pgfusepath{stroke,fill}%
\end{pgfscope}%
\begin{pgfscope}%
\pgfpathrectangle{\pgfqpoint{0.887500in}{0.275000in}}{\pgfqpoint{4.225000in}{4.225000in}}%
\pgfusepath{clip}%
\pgfsetbuttcap%
\pgfsetroundjoin%
\definecolor{currentfill}{rgb}{0.227802,0.326594,0.546532}%
\pgfsetfillcolor{currentfill}%
\pgfsetfillopacity{0.700000}%
\pgfsetlinewidth{0.501875pt}%
\definecolor{currentstroke}{rgb}{1.000000,1.000000,1.000000}%
\pgfsetstrokecolor{currentstroke}%
\pgfsetstrokeopacity{0.500000}%
\pgfsetdash{}{0pt}%
\pgfpathmoveto{\pgfqpoint{2.498886in}{1.615033in}}%
\pgfpathlineto{\pgfqpoint{2.510656in}{1.619087in}}%
\pgfpathlineto{\pgfqpoint{2.522421in}{1.623141in}}%
\pgfpathlineto{\pgfqpoint{2.534180in}{1.627197in}}%
\pgfpathlineto{\pgfqpoint{2.545933in}{1.631256in}}%
\pgfpathlineto{\pgfqpoint{2.557680in}{1.635319in}}%
\pgfpathlineto{\pgfqpoint{2.551446in}{1.646549in}}%
\pgfpathlineto{\pgfqpoint{2.545216in}{1.657736in}}%
\pgfpathlineto{\pgfqpoint{2.538990in}{1.668880in}}%
\pgfpathlineto{\pgfqpoint{2.532769in}{1.679980in}}%
\pgfpathlineto{\pgfqpoint{2.526551in}{1.691036in}}%
\pgfpathlineto{\pgfqpoint{2.514813in}{1.686995in}}%
\pgfpathlineto{\pgfqpoint{2.503069in}{1.682958in}}%
\pgfpathlineto{\pgfqpoint{2.491319in}{1.678925in}}%
\pgfpathlineto{\pgfqpoint{2.479563in}{1.674893in}}%
\pgfpathlineto{\pgfqpoint{2.467802in}{1.670862in}}%
\pgfpathlineto{\pgfqpoint{2.474011in}{1.659781in}}%
\pgfpathlineto{\pgfqpoint{2.480223in}{1.648657in}}%
\pgfpathlineto{\pgfqpoint{2.486440in}{1.637490in}}%
\pgfpathlineto{\pgfqpoint{2.492661in}{1.626282in}}%
\pgfpathclose%
\pgfusepath{stroke,fill}%
\end{pgfscope}%
\begin{pgfscope}%
\pgfpathrectangle{\pgfqpoint{0.887500in}{0.275000in}}{\pgfqpoint{4.225000in}{4.225000in}}%
\pgfusepath{clip}%
\pgfsetbuttcap%
\pgfsetroundjoin%
\definecolor{currentfill}{rgb}{0.282656,0.100196,0.422160}%
\pgfsetfillcolor{currentfill}%
\pgfsetfillopacity{0.700000}%
\pgfsetlinewidth{0.501875pt}%
\definecolor{currentstroke}{rgb}{1.000000,1.000000,1.000000}%
\pgfsetstrokecolor{currentstroke}%
\pgfsetstrokeopacity{0.500000}%
\pgfsetdash{}{0pt}%
\pgfpathmoveto{\pgfqpoint{3.309889in}{1.233189in}}%
\pgfpathlineto{\pgfqpoint{3.321412in}{1.229679in}}%
\pgfpathlineto{\pgfqpoint{3.332932in}{1.227110in}}%
\pgfpathlineto{\pgfqpoint{3.344451in}{1.225573in}}%
\pgfpathlineto{\pgfqpoint{3.355969in}{1.225148in}}%
\pgfpathlineto{\pgfqpoint{3.367488in}{1.225916in}}%
\pgfpathlineto{\pgfqpoint{3.361071in}{1.242322in}}%
\pgfpathlineto{\pgfqpoint{3.354656in}{1.258834in}}%
\pgfpathlineto{\pgfqpoint{3.348244in}{1.275394in}}%
\pgfpathlineto{\pgfqpoint{3.341834in}{1.291945in}}%
\pgfpathlineto{\pgfqpoint{3.335425in}{1.308426in}}%
\pgfpathlineto{\pgfqpoint{3.323892in}{1.306001in}}%
\pgfpathlineto{\pgfqpoint{3.312356in}{1.304182in}}%
\pgfpathlineto{\pgfqpoint{3.300817in}{1.302932in}}%
\pgfpathlineto{\pgfqpoint{3.289274in}{1.302212in}}%
\pgfpathlineto{\pgfqpoint{3.277726in}{1.301980in}}%
\pgfpathlineto{\pgfqpoint{3.284153in}{1.288291in}}%
\pgfpathlineto{\pgfqpoint{3.290583in}{1.274564in}}%
\pgfpathlineto{\pgfqpoint{3.297016in}{1.260802in}}%
\pgfpathlineto{\pgfqpoint{3.303451in}{1.247009in}}%
\pgfpathclose%
\pgfusepath{stroke,fill}%
\end{pgfscope}%
\begin{pgfscope}%
\pgfpathrectangle{\pgfqpoint{0.887500in}{0.275000in}}{\pgfqpoint{4.225000in}{4.225000in}}%
\pgfusepath{clip}%
\pgfsetbuttcap%
\pgfsetroundjoin%
\definecolor{currentfill}{rgb}{0.235526,0.309527,0.542944}%
\pgfsetfillcolor{currentfill}%
\pgfsetfillopacity{0.700000}%
\pgfsetlinewidth{0.501875pt}%
\definecolor{currentstroke}{rgb}{1.000000,1.000000,1.000000}%
\pgfsetstrokecolor{currentstroke}%
\pgfsetstrokeopacity{0.500000}%
\pgfsetdash{}{0pt}%
\pgfpathmoveto{\pgfqpoint{2.588908in}{1.578549in}}%
\pgfpathlineto{\pgfqpoint{2.600658in}{1.582632in}}%
\pgfpathlineto{\pgfqpoint{2.612403in}{1.586718in}}%
\pgfpathlineto{\pgfqpoint{2.624142in}{1.590806in}}%
\pgfpathlineto{\pgfqpoint{2.635875in}{1.594894in}}%
\pgfpathlineto{\pgfqpoint{2.647602in}{1.598980in}}%
\pgfpathlineto{\pgfqpoint{2.641340in}{1.610403in}}%
\pgfpathlineto{\pgfqpoint{2.635082in}{1.621781in}}%
\pgfpathlineto{\pgfqpoint{2.628828in}{1.633117in}}%
\pgfpathlineto{\pgfqpoint{2.622577in}{1.644411in}}%
\pgfpathlineto{\pgfqpoint{2.616331in}{1.655663in}}%
\pgfpathlineto{\pgfqpoint{2.604612in}{1.651596in}}%
\pgfpathlineto{\pgfqpoint{2.592887in}{1.647527in}}%
\pgfpathlineto{\pgfqpoint{2.581157in}{1.643457in}}%
\pgfpathlineto{\pgfqpoint{2.569422in}{1.639387in}}%
\pgfpathlineto{\pgfqpoint{2.557680in}{1.635319in}}%
\pgfpathlineto{\pgfqpoint{2.563918in}{1.624048in}}%
\pgfpathlineto{\pgfqpoint{2.570160in}{1.612735in}}%
\pgfpathlineto{\pgfqpoint{2.576405in}{1.601381in}}%
\pgfpathlineto{\pgfqpoint{2.582655in}{1.589986in}}%
\pgfpathclose%
\pgfusepath{stroke,fill}%
\end{pgfscope}%
\begin{pgfscope}%
\pgfpathrectangle{\pgfqpoint{0.887500in}{0.275000in}}{\pgfqpoint{4.225000in}{4.225000in}}%
\pgfusepath{clip}%
\pgfsetbuttcap%
\pgfsetroundjoin%
\definecolor{currentfill}{rgb}{0.199430,0.387607,0.554642}%
\pgfsetfillcolor{currentfill}%
\pgfsetfillopacity{0.700000}%
\pgfsetlinewidth{0.501875pt}%
\definecolor{currentstroke}{rgb}{1.000000,1.000000,1.000000}%
\pgfsetstrokecolor{currentstroke}%
\pgfsetstrokeopacity{0.500000}%
\pgfsetdash{}{0pt}%
\pgfpathmoveto{\pgfqpoint{2.080002in}{1.732195in}}%
\pgfpathlineto{\pgfqpoint{2.091878in}{1.736209in}}%
\pgfpathlineto{\pgfqpoint{2.103747in}{1.740219in}}%
\pgfpathlineto{\pgfqpoint{2.115611in}{1.744225in}}%
\pgfpathlineto{\pgfqpoint{2.127470in}{1.748229in}}%
\pgfpathlineto{\pgfqpoint{2.139322in}{1.752229in}}%
\pgfpathlineto{\pgfqpoint{2.133220in}{1.762757in}}%
\pgfpathlineto{\pgfqpoint{2.127122in}{1.773245in}}%
\pgfpathlineto{\pgfqpoint{2.121028in}{1.783693in}}%
\pgfpathlineto{\pgfqpoint{2.114939in}{1.794101in}}%
\pgfpathlineto{\pgfqpoint{2.108855in}{1.804469in}}%
\pgfpathlineto{\pgfqpoint{2.097012in}{1.800505in}}%
\pgfpathlineto{\pgfqpoint{2.085164in}{1.796539in}}%
\pgfpathlineto{\pgfqpoint{2.073309in}{1.792570in}}%
\pgfpathlineto{\pgfqpoint{2.061450in}{1.788597in}}%
\pgfpathlineto{\pgfqpoint{2.049584in}{1.784619in}}%
\pgfpathlineto{\pgfqpoint{2.055659in}{1.774212in}}%
\pgfpathlineto{\pgfqpoint{2.061738in}{1.763766in}}%
\pgfpathlineto{\pgfqpoint{2.067822in}{1.753281in}}%
\pgfpathlineto{\pgfqpoint{2.073910in}{1.742757in}}%
\pgfpathclose%
\pgfusepath{stroke,fill}%
\end{pgfscope}%
\begin{pgfscope}%
\pgfpathrectangle{\pgfqpoint{0.887500in}{0.275000in}}{\pgfqpoint{4.225000in}{4.225000in}}%
\pgfusepath{clip}%
\pgfsetbuttcap%
\pgfsetroundjoin%
\definecolor{currentfill}{rgb}{0.267004,0.004874,0.329415}%
\pgfsetfillcolor{currentfill}%
\pgfsetfillopacity{0.700000}%
\pgfsetlinewidth{0.501875pt}%
\definecolor{currentstroke}{rgb}{1.000000,1.000000,1.000000}%
\pgfsetstrokecolor{currentstroke}%
\pgfsetstrokeopacity{0.500000}%
\pgfsetdash{}{0pt}%
\pgfpathmoveto{\pgfqpoint{3.431949in}{1.080483in}}%
\pgfpathlineto{\pgfqpoint{3.443460in}{1.081060in}}%
\pgfpathlineto{\pgfqpoint{3.454979in}{1.083157in}}%
\pgfpathlineto{\pgfqpoint{3.466508in}{1.086919in}}%
\pgfpathlineto{\pgfqpoint{3.478049in}{1.092161in}}%
\pgfpathlineto{\pgfqpoint{3.489596in}{1.098481in}}%
\pgfpathlineto{\pgfqpoint{3.483124in}{1.111947in}}%
\pgfpathlineto{\pgfqpoint{3.476659in}{1.125887in}}%
\pgfpathlineto{\pgfqpoint{3.470202in}{1.140257in}}%
\pgfpathlineto{\pgfqpoint{3.463752in}{1.155015in}}%
\pgfpathlineto{\pgfqpoint{3.457308in}{1.170117in}}%
\pgfpathlineto{\pgfqpoint{3.445753in}{1.162543in}}%
\pgfpathlineto{\pgfqpoint{3.434206in}{1.156186in}}%
\pgfpathlineto{\pgfqpoint{3.422672in}{1.151474in}}%
\pgfpathlineto{\pgfqpoint{3.411149in}{1.148618in}}%
\pgfpathlineto{\pgfqpoint{3.399635in}{1.147517in}}%
\pgfpathlineto{\pgfqpoint{3.406082in}{1.132970in}}%
\pgfpathlineto{\pgfqpoint{3.412535in}{1.118934in}}%
\pgfpathlineto{\pgfqpoint{3.418997in}{1.105470in}}%
\pgfpathlineto{\pgfqpoint{3.425468in}{1.092633in}}%
\pgfpathclose%
\pgfusepath{stroke,fill}%
\end{pgfscope}%
\begin{pgfscope}%
\pgfpathrectangle{\pgfqpoint{0.887500in}{0.275000in}}{\pgfqpoint{4.225000in}{4.225000in}}%
\pgfusepath{clip}%
\pgfsetbuttcap%
\pgfsetroundjoin%
\definecolor{currentfill}{rgb}{0.243113,0.292092,0.538516}%
\pgfsetfillcolor{currentfill}%
\pgfsetfillopacity{0.700000}%
\pgfsetlinewidth{0.501875pt}%
\definecolor{currentstroke}{rgb}{1.000000,1.000000,1.000000}%
\pgfsetstrokecolor{currentstroke}%
\pgfsetstrokeopacity{0.500000}%
\pgfsetdash{}{0pt}%
\pgfpathmoveto{\pgfqpoint{2.678971in}{1.541121in}}%
\pgfpathlineto{\pgfqpoint{2.690701in}{1.545214in}}%
\pgfpathlineto{\pgfqpoint{2.702426in}{1.549305in}}%
\pgfpathlineto{\pgfqpoint{2.714145in}{1.553393in}}%
\pgfpathlineto{\pgfqpoint{2.725859in}{1.557475in}}%
\pgfpathlineto{\pgfqpoint{2.737566in}{1.561550in}}%
\pgfpathlineto{\pgfqpoint{2.731276in}{1.573227in}}%
\pgfpathlineto{\pgfqpoint{2.724990in}{1.584842in}}%
\pgfpathlineto{\pgfqpoint{2.718708in}{1.596398in}}%
\pgfpathlineto{\pgfqpoint{2.712430in}{1.607899in}}%
\pgfpathlineto{\pgfqpoint{2.706156in}{1.619346in}}%
\pgfpathlineto{\pgfqpoint{2.694456in}{1.615288in}}%
\pgfpathlineto{\pgfqpoint{2.682751in}{1.611220in}}%
\pgfpathlineto{\pgfqpoint{2.671041in}{1.607145in}}%
\pgfpathlineto{\pgfqpoint{2.659324in}{1.603065in}}%
\pgfpathlineto{\pgfqpoint{2.647602in}{1.598980in}}%
\pgfpathlineto{\pgfqpoint{2.653868in}{1.587511in}}%
\pgfpathlineto{\pgfqpoint{2.660138in}{1.575993in}}%
\pgfpathlineto{\pgfqpoint{2.666412in}{1.564424in}}%
\pgfpathlineto{\pgfqpoint{2.672690in}{1.552801in}}%
\pgfpathclose%
\pgfusepath{stroke,fill}%
\end{pgfscope}%
\begin{pgfscope}%
\pgfpathrectangle{\pgfqpoint{0.887500in}{0.275000in}}{\pgfqpoint{4.225000in}{4.225000in}}%
\pgfusepath{clip}%
\pgfsetbuttcap%
\pgfsetroundjoin%
\definecolor{currentfill}{rgb}{0.252194,0.269783,0.531579}%
\pgfsetfillcolor{currentfill}%
\pgfsetfillopacity{0.700000}%
\pgfsetlinewidth{0.501875pt}%
\definecolor{currentstroke}{rgb}{1.000000,1.000000,1.000000}%
\pgfsetstrokecolor{currentstroke}%
\pgfsetstrokeopacity{0.500000}%
\pgfsetdash{}{0pt}%
\pgfpathmoveto{\pgfqpoint{2.769073in}{1.502129in}}%
\pgfpathlineto{\pgfqpoint{2.780784in}{1.506181in}}%
\pgfpathlineto{\pgfqpoint{2.792488in}{1.510228in}}%
\pgfpathlineto{\pgfqpoint{2.804187in}{1.514269in}}%
\pgfpathlineto{\pgfqpoint{2.815880in}{1.518304in}}%
\pgfpathlineto{\pgfqpoint{2.827568in}{1.522334in}}%
\pgfpathlineto{\pgfqpoint{2.821251in}{1.534368in}}%
\pgfpathlineto{\pgfqpoint{2.814938in}{1.546331in}}%
\pgfpathlineto{\pgfqpoint{2.808629in}{1.558215in}}%
\pgfpathlineto{\pgfqpoint{2.802323in}{1.570023in}}%
\pgfpathlineto{\pgfqpoint{2.796021in}{1.581756in}}%
\pgfpathlineto{\pgfqpoint{2.784341in}{1.577740in}}%
\pgfpathlineto{\pgfqpoint{2.772656in}{1.573712in}}%
\pgfpathlineto{\pgfqpoint{2.760965in}{1.569670in}}%
\pgfpathlineto{\pgfqpoint{2.749268in}{1.565615in}}%
\pgfpathlineto{\pgfqpoint{2.737566in}{1.561550in}}%
\pgfpathlineto{\pgfqpoint{2.743860in}{1.549806in}}%
\pgfpathlineto{\pgfqpoint{2.750158in}{1.537994in}}%
\pgfpathlineto{\pgfqpoint{2.756459in}{1.526110in}}%
\pgfpathlineto{\pgfqpoint{2.762764in}{1.514153in}}%
\pgfpathclose%
\pgfusepath{stroke,fill}%
\end{pgfscope}%
\begin{pgfscope}%
\pgfpathrectangle{\pgfqpoint{0.887500in}{0.275000in}}{\pgfqpoint{4.225000in}{4.225000in}}%
\pgfusepath{clip}%
\pgfsetbuttcap%
\pgfsetroundjoin%
\definecolor{currentfill}{rgb}{0.206756,0.371758,0.553117}%
\pgfsetfillcolor{currentfill}%
\pgfsetfillopacity{0.700000}%
\pgfsetlinewidth{0.501875pt}%
\definecolor{currentstroke}{rgb}{1.000000,1.000000,1.000000}%
\pgfsetstrokecolor{currentstroke}%
\pgfsetstrokeopacity{0.500000}%
\pgfsetdash{}{0pt}%
\pgfpathmoveto{\pgfqpoint{2.169900in}{1.699002in}}%
\pgfpathlineto{\pgfqpoint{2.181756in}{1.703034in}}%
\pgfpathlineto{\pgfqpoint{2.193607in}{1.707065in}}%
\pgfpathlineto{\pgfqpoint{2.205452in}{1.711095in}}%
\pgfpathlineto{\pgfqpoint{2.217291in}{1.715125in}}%
\pgfpathlineto{\pgfqpoint{2.229125in}{1.719156in}}%
\pgfpathlineto{\pgfqpoint{2.222991in}{1.729849in}}%
\pgfpathlineto{\pgfqpoint{2.216861in}{1.740503in}}%
\pgfpathlineto{\pgfqpoint{2.210736in}{1.751116in}}%
\pgfpathlineto{\pgfqpoint{2.204615in}{1.761688in}}%
\pgfpathlineto{\pgfqpoint{2.198498in}{1.772221in}}%
\pgfpathlineto{\pgfqpoint{2.186674in}{1.768222in}}%
\pgfpathlineto{\pgfqpoint{2.174845in}{1.764224in}}%
\pgfpathlineto{\pgfqpoint{2.163010in}{1.760227in}}%
\pgfpathlineto{\pgfqpoint{2.151169in}{1.756229in}}%
\pgfpathlineto{\pgfqpoint{2.139322in}{1.752229in}}%
\pgfpathlineto{\pgfqpoint{2.145429in}{1.741662in}}%
\pgfpathlineto{\pgfqpoint{2.151540in}{1.731056in}}%
\pgfpathlineto{\pgfqpoint{2.157656in}{1.720410in}}%
\pgfpathlineto{\pgfqpoint{2.163776in}{1.709726in}}%
\pgfpathclose%
\pgfusepath{stroke,fill}%
\end{pgfscope}%
\begin{pgfscope}%
\pgfpathrectangle{\pgfqpoint{0.887500in}{0.275000in}}{\pgfqpoint{4.225000in}{4.225000in}}%
\pgfusepath{clip}%
\pgfsetbuttcap%
\pgfsetroundjoin%
\definecolor{currentfill}{rgb}{0.260571,0.246922,0.522828}%
\pgfsetfillcolor{currentfill}%
\pgfsetfillopacity{0.700000}%
\pgfsetlinewidth{0.501875pt}%
\definecolor{currentstroke}{rgb}{1.000000,1.000000,1.000000}%
\pgfsetstrokecolor{currentstroke}%
\pgfsetstrokeopacity{0.500000}%
\pgfsetdash{}{0pt}%
\pgfpathmoveto{\pgfqpoint{2.859205in}{1.461495in}}%
\pgfpathlineto{\pgfqpoint{2.870894in}{1.465523in}}%
\pgfpathlineto{\pgfqpoint{2.882578in}{1.469555in}}%
\pgfpathlineto{\pgfqpoint{2.894255in}{1.473591in}}%
\pgfpathlineto{\pgfqpoint{2.905928in}{1.477631in}}%
\pgfpathlineto{\pgfqpoint{2.917594in}{1.481675in}}%
\pgfpathlineto{\pgfqpoint{2.911253in}{1.493908in}}%
\pgfpathlineto{\pgfqpoint{2.904915in}{1.506113in}}%
\pgfpathlineto{\pgfqpoint{2.898580in}{1.518282in}}%
\pgfpathlineto{\pgfqpoint{2.892249in}{1.530406in}}%
\pgfpathlineto{\pgfqpoint{2.885921in}{1.542476in}}%
\pgfpathlineto{\pgfqpoint{2.874262in}{1.538443in}}%
\pgfpathlineto{\pgfqpoint{2.862597in}{1.534414in}}%
\pgfpathlineto{\pgfqpoint{2.850926in}{1.530388in}}%
\pgfpathlineto{\pgfqpoint{2.839250in}{1.526361in}}%
\pgfpathlineto{\pgfqpoint{2.827568in}{1.522334in}}%
\pgfpathlineto{\pgfqpoint{2.833888in}{1.510240in}}%
\pgfpathlineto{\pgfqpoint{2.840212in}{1.498098in}}%
\pgfpathlineto{\pgfqpoint{2.846540in}{1.485919in}}%
\pgfpathlineto{\pgfqpoint{2.852871in}{1.473714in}}%
\pgfpathclose%
\pgfusepath{stroke,fill}%
\end{pgfscope}%
\begin{pgfscope}%
\pgfpathrectangle{\pgfqpoint{0.887500in}{0.275000in}}{\pgfqpoint{4.225000in}{4.225000in}}%
\pgfusepath{clip}%
\pgfsetbuttcap%
\pgfsetroundjoin%
\definecolor{currentfill}{rgb}{0.279574,0.170599,0.479997}%
\pgfsetfillcolor{currentfill}%
\pgfsetfillopacity{0.700000}%
\pgfsetlinewidth{0.501875pt}%
\definecolor{currentstroke}{rgb}{1.000000,1.000000,1.000000}%
\pgfsetstrokecolor{currentstroke}%
\pgfsetstrokeopacity{0.500000}%
\pgfsetdash{}{0pt}%
\pgfpathmoveto{\pgfqpoint{3.129662in}{1.332224in}}%
\pgfpathlineto{\pgfqpoint{3.141286in}{1.336433in}}%
\pgfpathlineto{\pgfqpoint{3.152904in}{1.340571in}}%
\pgfpathlineto{\pgfqpoint{3.164516in}{1.344601in}}%
\pgfpathlineto{\pgfqpoint{3.176122in}{1.348488in}}%
\pgfpathlineto{\pgfqpoint{3.187722in}{1.352224in}}%
\pgfpathlineto{\pgfqpoint{3.181309in}{1.363611in}}%
\pgfpathlineto{\pgfqpoint{3.174900in}{1.375466in}}%
\pgfpathlineto{\pgfqpoint{3.168497in}{1.387740in}}%
\pgfpathlineto{\pgfqpoint{3.162097in}{1.400365in}}%
\pgfpathlineto{\pgfqpoint{3.155702in}{1.413272in}}%
\pgfpathlineto{\pgfqpoint{3.144108in}{1.409110in}}%
\pgfpathlineto{\pgfqpoint{3.132508in}{1.405181in}}%
\pgfpathlineto{\pgfqpoint{3.120904in}{1.401471in}}%
\pgfpathlineto{\pgfqpoint{3.109294in}{1.397929in}}%
\pgfpathlineto{\pgfqpoint{3.097678in}{1.394498in}}%
\pgfpathlineto{\pgfqpoint{3.104068in}{1.381621in}}%
\pgfpathlineto{\pgfqpoint{3.110461in}{1.368910in}}%
\pgfpathlineto{\pgfqpoint{3.116857in}{1.356412in}}%
\pgfpathlineto{\pgfqpoint{3.123258in}{1.344170in}}%
\pgfpathclose%
\pgfusepath{stroke,fill}%
\end{pgfscope}%
\begin{pgfscope}%
\pgfpathrectangle{\pgfqpoint{0.887500in}{0.275000in}}{\pgfqpoint{4.225000in}{4.225000in}}%
\pgfusepath{clip}%
\pgfsetbuttcap%
\pgfsetroundjoin%
\definecolor{currentfill}{rgb}{0.267968,0.223549,0.512008}%
\pgfsetfillcolor{currentfill}%
\pgfsetfillopacity{0.700000}%
\pgfsetlinewidth{0.501875pt}%
\definecolor{currentstroke}{rgb}{1.000000,1.000000,1.000000}%
\pgfsetstrokecolor{currentstroke}%
\pgfsetstrokeopacity{0.500000}%
\pgfsetdash{}{0pt}%
\pgfpathmoveto{\pgfqpoint{2.949350in}{1.420400in}}%
\pgfpathlineto{\pgfqpoint{2.961018in}{1.424382in}}%
\pgfpathlineto{\pgfqpoint{2.972681in}{1.428367in}}%
\pgfpathlineto{\pgfqpoint{2.984337in}{1.432350in}}%
\pgfpathlineto{\pgfqpoint{2.995988in}{1.436326in}}%
\pgfpathlineto{\pgfqpoint{3.007634in}{1.440290in}}%
\pgfpathlineto{\pgfqpoint{3.001269in}{1.452717in}}%
\pgfpathlineto{\pgfqpoint{2.994907in}{1.465110in}}%
\pgfpathlineto{\pgfqpoint{2.988548in}{1.477470in}}%
\pgfpathlineto{\pgfqpoint{2.982193in}{1.489795in}}%
\pgfpathlineto{\pgfqpoint{2.975841in}{1.502079in}}%
\pgfpathlineto{\pgfqpoint{2.964203in}{1.497959in}}%
\pgfpathlineto{\pgfqpoint{2.952559in}{1.493863in}}%
\pgfpathlineto{\pgfqpoint{2.940910in}{1.489787in}}%
\pgfpathlineto{\pgfqpoint{2.929255in}{1.485726in}}%
\pgfpathlineto{\pgfqpoint{2.917594in}{1.481675in}}%
\pgfpathlineto{\pgfqpoint{2.923939in}{1.469425in}}%
\pgfpathlineto{\pgfqpoint{2.930287in}{1.457165in}}%
\pgfpathlineto{\pgfqpoint{2.936638in}{1.444904in}}%
\pgfpathlineto{\pgfqpoint{2.942992in}{1.432652in}}%
\pgfpathclose%
\pgfusepath{stroke,fill}%
\end{pgfscope}%
\begin{pgfscope}%
\pgfpathrectangle{\pgfqpoint{0.887500in}{0.275000in}}{\pgfqpoint{4.225000in}{4.225000in}}%
\pgfusepath{clip}%
\pgfsetbuttcap%
\pgfsetroundjoin%
\definecolor{currentfill}{rgb}{0.282290,0.145912,0.461510}%
\pgfsetfillcolor{currentfill}%
\pgfsetfillopacity{0.700000}%
\pgfsetlinewidth{0.501875pt}%
\definecolor{currentstroke}{rgb}{1.000000,1.000000,1.000000}%
\pgfsetstrokecolor{currentstroke}%
\pgfsetstrokeopacity{0.500000}%
\pgfsetdash{}{0pt}%
\pgfpathmoveto{\pgfqpoint{3.219864in}{1.300342in}}%
\pgfpathlineto{\pgfqpoint{3.231457in}{1.301651in}}%
\pgfpathlineto{\pgfqpoint{3.243038in}{1.302192in}}%
\pgfpathlineto{\pgfqpoint{3.254610in}{1.302243in}}%
\pgfpathlineto{\pgfqpoint{3.266172in}{1.302080in}}%
\pgfpathlineto{\pgfqpoint{3.277726in}{1.301980in}}%
\pgfpathlineto{\pgfqpoint{3.271301in}{1.315626in}}%
\pgfpathlineto{\pgfqpoint{3.264879in}{1.329225in}}%
\pgfpathlineto{\pgfqpoint{3.258459in}{1.342774in}}%
\pgfpathlineto{\pgfqpoint{3.252043in}{1.356269in}}%
\pgfpathlineto{\pgfqpoint{3.245628in}{1.369706in}}%
\pgfpathlineto{\pgfqpoint{3.234059in}{1.366263in}}%
\pgfpathlineto{\pgfqpoint{3.222484in}{1.362825in}}%
\pgfpathlineto{\pgfqpoint{3.210903in}{1.359360in}}%
\pgfpathlineto{\pgfqpoint{3.199316in}{1.355837in}}%
\pgfpathlineto{\pgfqpoint{3.187722in}{1.352224in}}%
\pgfpathlineto{\pgfqpoint{3.194141in}{1.341264in}}%
\pgfpathlineto{\pgfqpoint{3.200564in}{1.330663in}}%
\pgfpathlineto{\pgfqpoint{3.206993in}{1.320354in}}%
\pgfpathlineto{\pgfqpoint{3.213426in}{1.310269in}}%
\pgfpathclose%
\pgfusepath{stroke,fill}%
\end{pgfscope}%
\begin{pgfscope}%
\pgfpathrectangle{\pgfqpoint{0.887500in}{0.275000in}}{\pgfqpoint{4.225000in}{4.225000in}}%
\pgfusepath{clip}%
\pgfsetbuttcap%
\pgfsetroundjoin%
\definecolor{currentfill}{rgb}{0.274128,0.199721,0.498911}%
\pgfsetfillcolor{currentfill}%
\pgfsetfillopacity{0.700000}%
\pgfsetlinewidth{0.501875pt}%
\definecolor{currentstroke}{rgb}{1.000000,1.000000,1.000000}%
\pgfsetstrokecolor{currentstroke}%
\pgfsetstrokeopacity{0.500000}%
\pgfsetdash{}{0pt}%
\pgfpathmoveto{\pgfqpoint{3.039507in}{1.377190in}}%
\pgfpathlineto{\pgfqpoint{3.051153in}{1.380815in}}%
\pgfpathlineto{\pgfqpoint{3.062793in}{1.384336in}}%
\pgfpathlineto{\pgfqpoint{3.074428in}{1.387757in}}%
\pgfpathlineto{\pgfqpoint{3.086056in}{1.391126in}}%
\pgfpathlineto{\pgfqpoint{3.097678in}{1.394498in}}%
\pgfpathlineto{\pgfqpoint{3.091291in}{1.407499in}}%
\pgfpathlineto{\pgfqpoint{3.084907in}{1.420578in}}%
\pgfpathlineto{\pgfqpoint{3.078527in}{1.433690in}}%
\pgfpathlineto{\pgfqpoint{3.072149in}{1.446791in}}%
\pgfpathlineto{\pgfqpoint{3.065774in}{1.459836in}}%
\pgfpathlineto{\pgfqpoint{3.054158in}{1.455948in}}%
\pgfpathlineto{\pgfqpoint{3.042535in}{1.452060in}}%
\pgfpathlineto{\pgfqpoint{3.030907in}{1.448159in}}%
\pgfpathlineto{\pgfqpoint{3.019273in}{1.444236in}}%
\pgfpathlineto{\pgfqpoint{3.007634in}{1.440290in}}%
\pgfpathlineto{\pgfqpoint{3.014002in}{1.427815in}}%
\pgfpathlineto{\pgfqpoint{3.020373in}{1.415280in}}%
\pgfpathlineto{\pgfqpoint{3.026748in}{1.402673in}}%
\pgfpathlineto{\pgfqpoint{3.033126in}{1.389980in}}%
\pgfpathclose%
\pgfusepath{stroke,fill}%
\end{pgfscope}%
\begin{pgfscope}%
\pgfpathrectangle{\pgfqpoint{0.887500in}{0.275000in}}{\pgfqpoint{4.225000in}{4.225000in}}%
\pgfusepath{clip}%
\pgfsetbuttcap%
\pgfsetroundjoin%
\definecolor{currentfill}{rgb}{0.214298,0.355619,0.551184}%
\pgfsetfillcolor{currentfill}%
\pgfsetfillopacity{0.700000}%
\pgfsetlinewidth{0.501875pt}%
\definecolor{currentstroke}{rgb}{1.000000,1.000000,1.000000}%
\pgfsetstrokecolor{currentstroke}%
\pgfsetstrokeopacity{0.500000}%
\pgfsetdash{}{0pt}%
\pgfpathmoveto{\pgfqpoint{2.259859in}{1.665081in}}%
\pgfpathlineto{\pgfqpoint{2.271697in}{1.669137in}}%
\pgfpathlineto{\pgfqpoint{2.283528in}{1.673192in}}%
\pgfpathlineto{\pgfqpoint{2.295354in}{1.677246in}}%
\pgfpathlineto{\pgfqpoint{2.307174in}{1.681298in}}%
\pgfpathlineto{\pgfqpoint{2.318988in}{1.685348in}}%
\pgfpathlineto{\pgfqpoint{2.312823in}{1.696220in}}%
\pgfpathlineto{\pgfqpoint{2.306662in}{1.707051in}}%
\pgfpathlineto{\pgfqpoint{2.300506in}{1.717839in}}%
\pgfpathlineto{\pgfqpoint{2.294353in}{1.728585in}}%
\pgfpathlineto{\pgfqpoint{2.288205in}{1.739290in}}%
\pgfpathlineto{\pgfqpoint{2.276401in}{1.735268in}}%
\pgfpathlineto{\pgfqpoint{2.264590in}{1.731243in}}%
\pgfpathlineto{\pgfqpoint{2.252774in}{1.727215in}}%
\pgfpathlineto{\pgfqpoint{2.240952in}{1.723186in}}%
\pgfpathlineto{\pgfqpoint{2.229125in}{1.719156in}}%
\pgfpathlineto{\pgfqpoint{2.235263in}{1.708422in}}%
\pgfpathlineto{\pgfqpoint{2.241406in}{1.697647in}}%
\pgfpathlineto{\pgfqpoint{2.247553in}{1.686833in}}%
\pgfpathlineto{\pgfqpoint{2.253704in}{1.675977in}}%
\pgfpathclose%
\pgfusepath{stroke,fill}%
\end{pgfscope}%
\begin{pgfscope}%
\pgfpathrectangle{\pgfqpoint{0.887500in}{0.275000in}}{\pgfqpoint{4.225000in}{4.225000in}}%
\pgfusepath{clip}%
\pgfsetbuttcap%
\pgfsetroundjoin%
\definecolor{currentfill}{rgb}{0.220057,0.343307,0.549413}%
\pgfsetfillcolor{currentfill}%
\pgfsetfillopacity{0.700000}%
\pgfsetlinewidth{0.501875pt}%
\definecolor{currentstroke}{rgb}{1.000000,1.000000,1.000000}%
\pgfsetstrokecolor{currentstroke}%
\pgfsetstrokeopacity{0.500000}%
\pgfsetdash{}{0pt}%
\pgfpathmoveto{\pgfqpoint{2.349877in}{1.630353in}}%
\pgfpathlineto{\pgfqpoint{2.361695in}{1.634420in}}%
\pgfpathlineto{\pgfqpoint{2.373507in}{1.638485in}}%
\pgfpathlineto{\pgfqpoint{2.385314in}{1.642547in}}%
\pgfpathlineto{\pgfqpoint{2.397115in}{1.646605in}}%
\pgfpathlineto{\pgfqpoint{2.408910in}{1.650659in}}%
\pgfpathlineto{\pgfqpoint{2.402715in}{1.661721in}}%
\pgfpathlineto{\pgfqpoint{2.396524in}{1.672741in}}%
\pgfpathlineto{\pgfqpoint{2.390336in}{1.683718in}}%
\pgfpathlineto{\pgfqpoint{2.384153in}{1.694650in}}%
\pgfpathlineto{\pgfqpoint{2.377975in}{1.705540in}}%
\pgfpathlineto{\pgfqpoint{2.366189in}{1.701510in}}%
\pgfpathlineto{\pgfqpoint{2.354397in}{1.697476in}}%
\pgfpathlineto{\pgfqpoint{2.342600in}{1.693437in}}%
\pgfpathlineto{\pgfqpoint{2.330797in}{1.689394in}}%
\pgfpathlineto{\pgfqpoint{2.318988in}{1.685348in}}%
\pgfpathlineto{\pgfqpoint{2.325158in}{1.674433in}}%
\pgfpathlineto{\pgfqpoint{2.331331in}{1.663476in}}%
\pgfpathlineto{\pgfqpoint{2.337509in}{1.652476in}}%
\pgfpathlineto{\pgfqpoint{2.343691in}{1.641435in}}%
\pgfpathclose%
\pgfusepath{stroke,fill}%
\end{pgfscope}%
\begin{pgfscope}%
\pgfpathrectangle{\pgfqpoint{0.887500in}{0.275000in}}{\pgfqpoint{4.225000in}{4.225000in}}%
\pgfusepath{clip}%
\pgfsetbuttcap%
\pgfsetroundjoin%
\definecolor{currentfill}{rgb}{0.277941,0.056324,0.381191}%
\pgfsetfillcolor{currentfill}%
\pgfsetfillopacity{0.700000}%
\pgfsetlinewidth{0.501875pt}%
\definecolor{currentstroke}{rgb}{1.000000,1.000000,1.000000}%
\pgfsetstrokecolor{currentstroke}%
\pgfsetstrokeopacity{0.500000}%
\pgfsetdash{}{0pt}%
\pgfpathmoveto{\pgfqpoint{3.342119in}{1.163827in}}%
\pgfpathlineto{\pgfqpoint{3.353623in}{1.158089in}}%
\pgfpathlineto{\pgfqpoint{3.365124in}{1.153456in}}%
\pgfpathlineto{\pgfqpoint{3.376626in}{1.150065in}}%
\pgfpathlineto{\pgfqpoint{3.388128in}{1.148042in}}%
\pgfpathlineto{\pgfqpoint{3.399635in}{1.147517in}}%
\pgfpathlineto{\pgfqpoint{3.393195in}{1.162520in}}%
\pgfpathlineto{\pgfqpoint{3.386761in}{1.177919in}}%
\pgfpathlineto{\pgfqpoint{3.380333in}{1.193656in}}%
\pgfpathlineto{\pgfqpoint{3.373908in}{1.209675in}}%
\pgfpathlineto{\pgfqpoint{3.367488in}{1.225916in}}%
\pgfpathlineto{\pgfqpoint{3.355969in}{1.225148in}}%
\pgfpathlineto{\pgfqpoint{3.344451in}{1.225573in}}%
\pgfpathlineto{\pgfqpoint{3.332932in}{1.227110in}}%
\pgfpathlineto{\pgfqpoint{3.321412in}{1.229679in}}%
\pgfpathlineto{\pgfqpoint{3.309889in}{1.233189in}}%
\pgfpathlineto{\pgfqpoint{3.316329in}{1.219347in}}%
\pgfpathlineto{\pgfqpoint{3.322773in}{1.205486in}}%
\pgfpathlineto{\pgfqpoint{3.329218in}{1.191609in}}%
\pgfpathlineto{\pgfqpoint{3.335667in}{1.177722in}}%
\pgfpathclose%
\pgfusepath{stroke,fill}%
\end{pgfscope}%
\begin{pgfscope}%
\pgfpathrectangle{\pgfqpoint{0.887500in}{0.275000in}}{\pgfqpoint{4.225000in}{4.225000in}}%
\pgfusepath{clip}%
\pgfsetbuttcap%
\pgfsetroundjoin%
\definecolor{currentfill}{rgb}{0.227802,0.326594,0.546532}%
\pgfsetfillcolor{currentfill}%
\pgfsetfillopacity{0.700000}%
\pgfsetlinewidth{0.501875pt}%
\definecolor{currentstroke}{rgb}{1.000000,1.000000,1.000000}%
\pgfsetstrokecolor{currentstroke}%
\pgfsetstrokeopacity{0.500000}%
\pgfsetdash{}{0pt}%
\pgfpathmoveto{\pgfqpoint{2.439949in}{1.594730in}}%
\pgfpathlineto{\pgfqpoint{2.451748in}{1.598798in}}%
\pgfpathlineto{\pgfqpoint{2.463541in}{1.602862in}}%
\pgfpathlineto{\pgfqpoint{2.475328in}{1.606922in}}%
\pgfpathlineto{\pgfqpoint{2.487110in}{1.610979in}}%
\pgfpathlineto{\pgfqpoint{2.498886in}{1.615033in}}%
\pgfpathlineto{\pgfqpoint{2.492661in}{1.626282in}}%
\pgfpathlineto{\pgfqpoint{2.486440in}{1.637490in}}%
\pgfpathlineto{\pgfqpoint{2.480223in}{1.648657in}}%
\pgfpathlineto{\pgfqpoint{2.474011in}{1.659781in}}%
\pgfpathlineto{\pgfqpoint{2.467802in}{1.670862in}}%
\pgfpathlineto{\pgfqpoint{2.456035in}{1.666829in}}%
\pgfpathlineto{\pgfqpoint{2.444262in}{1.662794in}}%
\pgfpathlineto{\pgfqpoint{2.432484in}{1.658754in}}%
\pgfpathlineto{\pgfqpoint{2.420700in}{1.654709in}}%
\pgfpathlineto{\pgfqpoint{2.408910in}{1.650659in}}%
\pgfpathlineto{\pgfqpoint{2.415110in}{1.639555in}}%
\pgfpathlineto{\pgfqpoint{2.421314in}{1.628409in}}%
\pgfpathlineto{\pgfqpoint{2.427522in}{1.617223in}}%
\pgfpathlineto{\pgfqpoint{2.433733in}{1.605996in}}%
\pgfpathclose%
\pgfusepath{stroke,fill}%
\end{pgfscope}%
\begin{pgfscope}%
\pgfpathrectangle{\pgfqpoint{0.887500in}{0.275000in}}{\pgfqpoint{4.225000in}{4.225000in}}%
\pgfusepath{clip}%
\pgfsetbuttcap%
\pgfsetroundjoin%
\definecolor{currentfill}{rgb}{0.237441,0.305202,0.541921}%
\pgfsetfillcolor{currentfill}%
\pgfsetfillopacity{0.700000}%
\pgfsetlinewidth{0.501875pt}%
\definecolor{currentstroke}{rgb}{1.000000,1.000000,1.000000}%
\pgfsetstrokecolor{currentstroke}%
\pgfsetstrokeopacity{0.500000}%
\pgfsetdash{}{0pt}%
\pgfpathmoveto{\pgfqpoint{2.530070in}{1.558188in}}%
\pgfpathlineto{\pgfqpoint{2.541849in}{1.562256in}}%
\pgfpathlineto{\pgfqpoint{2.553622in}{1.566325in}}%
\pgfpathlineto{\pgfqpoint{2.565390in}{1.570396in}}%
\pgfpathlineto{\pgfqpoint{2.577152in}{1.574470in}}%
\pgfpathlineto{\pgfqpoint{2.588908in}{1.578549in}}%
\pgfpathlineto{\pgfqpoint{2.582655in}{1.589986in}}%
\pgfpathlineto{\pgfqpoint{2.576405in}{1.601381in}}%
\pgfpathlineto{\pgfqpoint{2.570160in}{1.612735in}}%
\pgfpathlineto{\pgfqpoint{2.563918in}{1.624048in}}%
\pgfpathlineto{\pgfqpoint{2.557680in}{1.635319in}}%
\pgfpathlineto{\pgfqpoint{2.545933in}{1.631256in}}%
\pgfpathlineto{\pgfqpoint{2.534180in}{1.627197in}}%
\pgfpathlineto{\pgfqpoint{2.522421in}{1.623141in}}%
\pgfpathlineto{\pgfqpoint{2.510656in}{1.619087in}}%
\pgfpathlineto{\pgfqpoint{2.498886in}{1.615033in}}%
\pgfpathlineto{\pgfqpoint{2.505115in}{1.603743in}}%
\pgfpathlineto{\pgfqpoint{2.511348in}{1.592414in}}%
\pgfpathlineto{\pgfqpoint{2.517585in}{1.581045in}}%
\pgfpathlineto{\pgfqpoint{2.523825in}{1.569637in}}%
\pgfpathclose%
\pgfusepath{stroke,fill}%
\end{pgfscope}%
\begin{pgfscope}%
\pgfpathrectangle{\pgfqpoint{0.887500in}{0.275000in}}{\pgfqpoint{4.225000in}{4.225000in}}%
\pgfusepath{clip}%
\pgfsetbuttcap%
\pgfsetroundjoin%
\definecolor{currentfill}{rgb}{0.201239,0.383670,0.554294}%
\pgfsetfillcolor{currentfill}%
\pgfsetfillopacity{0.700000}%
\pgfsetlinewidth{0.501875pt}%
\definecolor{currentstroke}{rgb}{1.000000,1.000000,1.000000}%
\pgfsetstrokecolor{currentstroke}%
\pgfsetstrokeopacity{0.500000}%
\pgfsetdash{}{0pt}%
\pgfpathmoveto{\pgfqpoint{2.020541in}{1.712035in}}%
\pgfpathlineto{\pgfqpoint{2.032444in}{1.716081in}}%
\pgfpathlineto{\pgfqpoint{2.044342in}{1.720119in}}%
\pgfpathlineto{\pgfqpoint{2.056235in}{1.724150in}}%
\pgfpathlineto{\pgfqpoint{2.068121in}{1.728175in}}%
\pgfpathlineto{\pgfqpoint{2.080002in}{1.732195in}}%
\pgfpathlineto{\pgfqpoint{2.073910in}{1.742757in}}%
\pgfpathlineto{\pgfqpoint{2.067822in}{1.753281in}}%
\pgfpathlineto{\pgfqpoint{2.061738in}{1.763766in}}%
\pgfpathlineto{\pgfqpoint{2.055659in}{1.774212in}}%
\pgfpathlineto{\pgfqpoint{2.049584in}{1.784619in}}%
\pgfpathlineto{\pgfqpoint{2.037713in}{1.780635in}}%
\pgfpathlineto{\pgfqpoint{2.025836in}{1.776645in}}%
\pgfpathlineto{\pgfqpoint{2.013954in}{1.772648in}}%
\pgfpathlineto{\pgfqpoint{2.002066in}{1.768644in}}%
\pgfpathlineto{\pgfqpoint{1.990172in}{1.764631in}}%
\pgfpathlineto{\pgfqpoint{1.996237in}{1.754188in}}%
\pgfpathlineto{\pgfqpoint{2.002306in}{1.743707in}}%
\pgfpathlineto{\pgfqpoint{2.008380in}{1.733188in}}%
\pgfpathlineto{\pgfqpoint{2.014458in}{1.722630in}}%
\pgfpathclose%
\pgfusepath{stroke,fill}%
\end{pgfscope}%
\begin{pgfscope}%
\pgfpathrectangle{\pgfqpoint{0.887500in}{0.275000in}}{\pgfqpoint{4.225000in}{4.225000in}}%
\pgfusepath{clip}%
\pgfsetbuttcap%
\pgfsetroundjoin%
\definecolor{currentfill}{rgb}{0.244972,0.287675,0.537260}%
\pgfsetfillcolor{currentfill}%
\pgfsetfillopacity{0.700000}%
\pgfsetlinewidth{0.501875pt}%
\definecolor{currentstroke}{rgb}{1.000000,1.000000,1.000000}%
\pgfsetstrokecolor{currentstroke}%
\pgfsetstrokeopacity{0.500000}%
\pgfsetdash{}{0pt}%
\pgfpathmoveto{\pgfqpoint{2.620234in}{1.520669in}}%
\pgfpathlineto{\pgfqpoint{2.631993in}{1.524754in}}%
\pgfpathlineto{\pgfqpoint{2.643746in}{1.528843in}}%
\pgfpathlineto{\pgfqpoint{2.655493in}{1.532935in}}%
\pgfpathlineto{\pgfqpoint{2.667235in}{1.537028in}}%
\pgfpathlineto{\pgfqpoint{2.678971in}{1.541121in}}%
\pgfpathlineto{\pgfqpoint{2.672690in}{1.552801in}}%
\pgfpathlineto{\pgfqpoint{2.666412in}{1.564424in}}%
\pgfpathlineto{\pgfqpoint{2.660138in}{1.575993in}}%
\pgfpathlineto{\pgfqpoint{2.653868in}{1.587511in}}%
\pgfpathlineto{\pgfqpoint{2.647602in}{1.598980in}}%
\pgfpathlineto{\pgfqpoint{2.635875in}{1.594894in}}%
\pgfpathlineto{\pgfqpoint{2.624142in}{1.590806in}}%
\pgfpathlineto{\pgfqpoint{2.612403in}{1.586718in}}%
\pgfpathlineto{\pgfqpoint{2.600658in}{1.582632in}}%
\pgfpathlineto{\pgfqpoint{2.588908in}{1.578549in}}%
\pgfpathlineto{\pgfqpoint{2.595165in}{1.567068in}}%
\pgfpathlineto{\pgfqpoint{2.601427in}{1.555541in}}%
\pgfpathlineto{\pgfqpoint{2.607692in}{1.543966in}}%
\pgfpathlineto{\pgfqpoint{2.613961in}{1.532343in}}%
\pgfpathclose%
\pgfusepath{stroke,fill}%
\end{pgfscope}%
\begin{pgfscope}%
\pgfpathrectangle{\pgfqpoint{0.887500in}{0.275000in}}{\pgfqpoint{4.225000in}{4.225000in}}%
\pgfusepath{clip}%
\pgfsetbuttcap%
\pgfsetroundjoin%
\definecolor{currentfill}{rgb}{0.252194,0.269783,0.531579}%
\pgfsetfillcolor{currentfill}%
\pgfsetfillopacity{0.700000}%
\pgfsetlinewidth{0.501875pt}%
\definecolor{currentstroke}{rgb}{1.000000,1.000000,1.000000}%
\pgfsetstrokecolor{currentstroke}%
\pgfsetstrokeopacity{0.500000}%
\pgfsetdash{}{0pt}%
\pgfpathmoveto{\pgfqpoint{2.710437in}{1.481815in}}%
\pgfpathlineto{\pgfqpoint{2.722176in}{1.485883in}}%
\pgfpathlineto{\pgfqpoint{2.733909in}{1.489949in}}%
\pgfpathlineto{\pgfqpoint{2.745636in}{1.494013in}}%
\pgfpathlineto{\pgfqpoint{2.757357in}{1.498073in}}%
\pgfpathlineto{\pgfqpoint{2.769073in}{1.502129in}}%
\pgfpathlineto{\pgfqpoint{2.762764in}{1.514153in}}%
\pgfpathlineto{\pgfqpoint{2.756459in}{1.526110in}}%
\pgfpathlineto{\pgfqpoint{2.750158in}{1.537994in}}%
\pgfpathlineto{\pgfqpoint{2.743860in}{1.549806in}}%
\pgfpathlineto{\pgfqpoint{2.737566in}{1.561550in}}%
\pgfpathlineto{\pgfqpoint{2.725859in}{1.557475in}}%
\pgfpathlineto{\pgfqpoint{2.714145in}{1.553393in}}%
\pgfpathlineto{\pgfqpoint{2.702426in}{1.549305in}}%
\pgfpathlineto{\pgfqpoint{2.690701in}{1.545214in}}%
\pgfpathlineto{\pgfqpoint{2.678971in}{1.541121in}}%
\pgfpathlineto{\pgfqpoint{2.685256in}{1.529384in}}%
\pgfpathlineto{\pgfqpoint{2.691546in}{1.517586in}}%
\pgfpathlineto{\pgfqpoint{2.697839in}{1.505724in}}%
\pgfpathlineto{\pgfqpoint{2.704136in}{1.493799in}}%
\pgfpathclose%
\pgfusepath{stroke,fill}%
\end{pgfscope}%
\begin{pgfscope}%
\pgfpathrectangle{\pgfqpoint{0.887500in}{0.275000in}}{\pgfqpoint{4.225000in}{4.225000in}}%
\pgfusepath{clip}%
\pgfsetbuttcap%
\pgfsetroundjoin%
\definecolor{currentfill}{rgb}{0.206756,0.371758,0.553117}%
\pgfsetfillcolor{currentfill}%
\pgfsetfillopacity{0.700000}%
\pgfsetlinewidth{0.501875pt}%
\definecolor{currentstroke}{rgb}{1.000000,1.000000,1.000000}%
\pgfsetstrokecolor{currentstroke}%
\pgfsetstrokeopacity{0.500000}%
\pgfsetdash{}{0pt}%
\pgfpathmoveto{\pgfqpoint{2.110531in}{1.678808in}}%
\pgfpathlineto{\pgfqpoint{2.122416in}{1.682854in}}%
\pgfpathlineto{\pgfqpoint{2.134296in}{1.686895in}}%
\pgfpathlineto{\pgfqpoint{2.146170in}{1.690934in}}%
\pgfpathlineto{\pgfqpoint{2.158038in}{1.694969in}}%
\pgfpathlineto{\pgfqpoint{2.169900in}{1.699002in}}%
\pgfpathlineto{\pgfqpoint{2.163776in}{1.709726in}}%
\pgfpathlineto{\pgfqpoint{2.157656in}{1.720410in}}%
\pgfpathlineto{\pgfqpoint{2.151540in}{1.731056in}}%
\pgfpathlineto{\pgfqpoint{2.145429in}{1.741662in}}%
\pgfpathlineto{\pgfqpoint{2.139322in}{1.752229in}}%
\pgfpathlineto{\pgfqpoint{2.127470in}{1.748229in}}%
\pgfpathlineto{\pgfqpoint{2.115611in}{1.744225in}}%
\pgfpathlineto{\pgfqpoint{2.103747in}{1.740219in}}%
\pgfpathlineto{\pgfqpoint{2.091878in}{1.736209in}}%
\pgfpathlineto{\pgfqpoint{2.080002in}{1.732195in}}%
\pgfpathlineto{\pgfqpoint{2.086099in}{1.721593in}}%
\pgfpathlineto{\pgfqpoint{2.092201in}{1.710954in}}%
\pgfpathlineto{\pgfqpoint{2.098307in}{1.700276in}}%
\pgfpathlineto{\pgfqpoint{2.104417in}{1.689561in}}%
\pgfpathclose%
\pgfusepath{stroke,fill}%
\end{pgfscope}%
\begin{pgfscope}%
\pgfpathrectangle{\pgfqpoint{0.887500in}{0.275000in}}{\pgfqpoint{4.225000in}{4.225000in}}%
\pgfusepath{clip}%
\pgfsetbuttcap%
\pgfsetroundjoin%
\definecolor{currentfill}{rgb}{0.260571,0.246922,0.522828}%
\pgfsetfillcolor{currentfill}%
\pgfsetfillopacity{0.700000}%
\pgfsetlinewidth{0.501875pt}%
\definecolor{currentstroke}{rgb}{1.000000,1.000000,1.000000}%
\pgfsetstrokecolor{currentstroke}%
\pgfsetstrokeopacity{0.500000}%
\pgfsetdash{}{0pt}%
\pgfpathmoveto{\pgfqpoint{2.800672in}{1.441395in}}%
\pgfpathlineto{\pgfqpoint{2.812390in}{1.445411in}}%
\pgfpathlineto{\pgfqpoint{2.824102in}{1.449429in}}%
\pgfpathlineto{\pgfqpoint{2.835809in}{1.453448in}}%
\pgfpathlineto{\pgfqpoint{2.847510in}{1.457470in}}%
\pgfpathlineto{\pgfqpoint{2.859205in}{1.461495in}}%
\pgfpathlineto{\pgfqpoint{2.852871in}{1.473714in}}%
\pgfpathlineto{\pgfqpoint{2.846540in}{1.485919in}}%
\pgfpathlineto{\pgfqpoint{2.840212in}{1.498098in}}%
\pgfpathlineto{\pgfqpoint{2.833888in}{1.510240in}}%
\pgfpathlineto{\pgfqpoint{2.827568in}{1.522334in}}%
\pgfpathlineto{\pgfqpoint{2.815880in}{1.518304in}}%
\pgfpathlineto{\pgfqpoint{2.804187in}{1.514269in}}%
\pgfpathlineto{\pgfqpoint{2.792488in}{1.510228in}}%
\pgfpathlineto{\pgfqpoint{2.780784in}{1.506181in}}%
\pgfpathlineto{\pgfqpoint{2.769073in}{1.502129in}}%
\pgfpathlineto{\pgfqpoint{2.775386in}{1.490050in}}%
\pgfpathlineto{\pgfqpoint{2.781702in}{1.477926in}}%
\pgfpathlineto{\pgfqpoint{2.788022in}{1.465769in}}%
\pgfpathlineto{\pgfqpoint{2.794345in}{1.453588in}}%
\pgfpathclose%
\pgfusepath{stroke,fill}%
\end{pgfscope}%
\begin{pgfscope}%
\pgfpathrectangle{\pgfqpoint{0.887500in}{0.275000in}}{\pgfqpoint{4.225000in}{4.225000in}}%
\pgfusepath{clip}%
\pgfsetbuttcap%
\pgfsetroundjoin%
\definecolor{currentfill}{rgb}{0.279574,0.170599,0.479997}%
\pgfsetfillcolor{currentfill}%
\pgfsetfillopacity{0.700000}%
\pgfsetlinewidth{0.501875pt}%
\definecolor{currentstroke}{rgb}{1.000000,1.000000,1.000000}%
\pgfsetstrokecolor{currentstroke}%
\pgfsetstrokeopacity{0.500000}%
\pgfsetdash{}{0pt}%
\pgfpathmoveto{\pgfqpoint{3.071458in}{1.311347in}}%
\pgfpathlineto{\pgfqpoint{3.083110in}{1.315404in}}%
\pgfpathlineto{\pgfqpoint{3.094756in}{1.319536in}}%
\pgfpathlineto{\pgfqpoint{3.106397in}{1.323739in}}%
\pgfpathlineto{\pgfqpoint{3.118032in}{1.327981in}}%
\pgfpathlineto{\pgfqpoint{3.129662in}{1.332224in}}%
\pgfpathlineto{\pgfqpoint{3.123258in}{1.344170in}}%
\pgfpathlineto{\pgfqpoint{3.116857in}{1.356412in}}%
\pgfpathlineto{\pgfqpoint{3.110461in}{1.368910in}}%
\pgfpathlineto{\pgfqpoint{3.104068in}{1.381621in}}%
\pgfpathlineto{\pgfqpoint{3.097678in}{1.394498in}}%
\pgfpathlineto{\pgfqpoint{3.086056in}{1.391126in}}%
\pgfpathlineto{\pgfqpoint{3.074428in}{1.387757in}}%
\pgfpathlineto{\pgfqpoint{3.062793in}{1.384336in}}%
\pgfpathlineto{\pgfqpoint{3.051153in}{1.380815in}}%
\pgfpathlineto{\pgfqpoint{3.039507in}{1.377190in}}%
\pgfpathlineto{\pgfqpoint{3.045891in}{1.364289in}}%
\pgfpathlineto{\pgfqpoint{3.052279in}{1.351265in}}%
\pgfpathlineto{\pgfqpoint{3.058669in}{1.338106in}}%
\pgfpathlineto{\pgfqpoint{3.065062in}{1.324798in}}%
\pgfpathclose%
\pgfusepath{stroke,fill}%
\end{pgfscope}%
\begin{pgfscope}%
\pgfpathrectangle{\pgfqpoint{0.887500in}{0.275000in}}{\pgfqpoint{4.225000in}{4.225000in}}%
\pgfusepath{clip}%
\pgfsetbuttcap%
\pgfsetroundjoin%
\definecolor{currentfill}{rgb}{0.267968,0.223549,0.512008}%
\pgfsetfillcolor{currentfill}%
\pgfsetfillopacity{0.700000}%
\pgfsetlinewidth{0.501875pt}%
\definecolor{currentstroke}{rgb}{1.000000,1.000000,1.000000}%
\pgfsetstrokecolor{currentstroke}%
\pgfsetstrokeopacity{0.500000}%
\pgfsetdash{}{0pt}%
\pgfpathmoveto{\pgfqpoint{2.890923in}{1.400557in}}%
\pgfpathlineto{\pgfqpoint{2.902620in}{1.404525in}}%
\pgfpathlineto{\pgfqpoint{2.914311in}{1.408490in}}%
\pgfpathlineto{\pgfqpoint{2.925996in}{1.412456in}}%
\pgfpathlineto{\pgfqpoint{2.937676in}{1.416425in}}%
\pgfpathlineto{\pgfqpoint{2.949350in}{1.420400in}}%
\pgfpathlineto{\pgfqpoint{2.942992in}{1.432652in}}%
\pgfpathlineto{\pgfqpoint{2.936638in}{1.444904in}}%
\pgfpathlineto{\pgfqpoint{2.930287in}{1.457165in}}%
\pgfpathlineto{\pgfqpoint{2.923939in}{1.469425in}}%
\pgfpathlineto{\pgfqpoint{2.917594in}{1.481675in}}%
\pgfpathlineto{\pgfqpoint{2.905928in}{1.477631in}}%
\pgfpathlineto{\pgfqpoint{2.894255in}{1.473591in}}%
\pgfpathlineto{\pgfqpoint{2.882578in}{1.469555in}}%
\pgfpathlineto{\pgfqpoint{2.870894in}{1.465523in}}%
\pgfpathlineto{\pgfqpoint{2.859205in}{1.461495in}}%
\pgfpathlineto{\pgfqpoint{2.865542in}{1.449273in}}%
\pgfpathlineto{\pgfqpoint{2.871882in}{1.437059in}}%
\pgfpathlineto{\pgfqpoint{2.878226in}{1.424864in}}%
\pgfpathlineto{\pgfqpoint{2.884573in}{1.412700in}}%
\pgfpathclose%
\pgfusepath{stroke,fill}%
\end{pgfscope}%
\begin{pgfscope}%
\pgfpathrectangle{\pgfqpoint{0.887500in}{0.275000in}}{\pgfqpoint{4.225000in}{4.225000in}}%
\pgfusepath{clip}%
\pgfsetbuttcap%
\pgfsetroundjoin%
\definecolor{currentfill}{rgb}{0.274128,0.199721,0.498911}%
\pgfsetfillcolor{currentfill}%
\pgfsetfillopacity{0.700000}%
\pgfsetlinewidth{0.501875pt}%
\definecolor{currentstroke}{rgb}{1.000000,1.000000,1.000000}%
\pgfsetstrokecolor{currentstroke}%
\pgfsetstrokeopacity{0.500000}%
\pgfsetdash{}{0pt}%
\pgfpathmoveto{\pgfqpoint{2.981189in}{1.358176in}}%
\pgfpathlineto{\pgfqpoint{2.992864in}{1.362032in}}%
\pgfpathlineto{\pgfqpoint{3.004534in}{1.365881in}}%
\pgfpathlineto{\pgfqpoint{3.016197in}{1.369703in}}%
\pgfpathlineto{\pgfqpoint{3.027855in}{1.373479in}}%
\pgfpathlineto{\pgfqpoint{3.039507in}{1.377190in}}%
\pgfpathlineto{\pgfqpoint{3.033126in}{1.389980in}}%
\pgfpathlineto{\pgfqpoint{3.026748in}{1.402673in}}%
\pgfpathlineto{\pgfqpoint{3.020373in}{1.415280in}}%
\pgfpathlineto{\pgfqpoint{3.014002in}{1.427815in}}%
\pgfpathlineto{\pgfqpoint{3.007634in}{1.440290in}}%
\pgfpathlineto{\pgfqpoint{2.995988in}{1.436326in}}%
\pgfpathlineto{\pgfqpoint{2.984337in}{1.432350in}}%
\pgfpathlineto{\pgfqpoint{2.972681in}{1.428367in}}%
\pgfpathlineto{\pgfqpoint{2.961018in}{1.424382in}}%
\pgfpathlineto{\pgfqpoint{2.949350in}{1.420400in}}%
\pgfpathlineto{\pgfqpoint{2.955711in}{1.408121in}}%
\pgfpathlineto{\pgfqpoint{2.962075in}{1.395787in}}%
\pgfpathlineto{\pgfqpoint{2.968443in}{1.383371in}}%
\pgfpathlineto{\pgfqpoint{2.974814in}{1.370843in}}%
\pgfpathclose%
\pgfusepath{stroke,fill}%
\end{pgfscope}%
\begin{pgfscope}%
\pgfpathrectangle{\pgfqpoint{0.887500in}{0.275000in}}{\pgfqpoint{4.225000in}{4.225000in}}%
\pgfusepath{clip}%
\pgfsetbuttcap%
\pgfsetroundjoin%
\definecolor{currentfill}{rgb}{0.214298,0.355619,0.551184}%
\pgfsetfillcolor{currentfill}%
\pgfsetfillopacity{0.700000}%
\pgfsetlinewidth{0.501875pt}%
\definecolor{currentstroke}{rgb}{1.000000,1.000000,1.000000}%
\pgfsetstrokecolor{currentstroke}%
\pgfsetstrokeopacity{0.500000}%
\pgfsetdash{}{0pt}%
\pgfpathmoveto{\pgfqpoint{2.200586in}{1.644805in}}%
\pgfpathlineto{\pgfqpoint{2.212452in}{1.648860in}}%
\pgfpathlineto{\pgfqpoint{2.224313in}{1.652915in}}%
\pgfpathlineto{\pgfqpoint{2.236167in}{1.656970in}}%
\pgfpathlineto{\pgfqpoint{2.248016in}{1.661025in}}%
\pgfpathlineto{\pgfqpoint{2.259859in}{1.665081in}}%
\pgfpathlineto{\pgfqpoint{2.253704in}{1.675977in}}%
\pgfpathlineto{\pgfqpoint{2.247553in}{1.686833in}}%
\pgfpathlineto{\pgfqpoint{2.241406in}{1.697647in}}%
\pgfpathlineto{\pgfqpoint{2.235263in}{1.708422in}}%
\pgfpathlineto{\pgfqpoint{2.229125in}{1.719156in}}%
\pgfpathlineto{\pgfqpoint{2.217291in}{1.715125in}}%
\pgfpathlineto{\pgfqpoint{2.205452in}{1.711095in}}%
\pgfpathlineto{\pgfqpoint{2.193607in}{1.707065in}}%
\pgfpathlineto{\pgfqpoint{2.181756in}{1.703034in}}%
\pgfpathlineto{\pgfqpoint{2.169900in}{1.699002in}}%
\pgfpathlineto{\pgfqpoint{2.176029in}{1.688240in}}%
\pgfpathlineto{\pgfqpoint{2.182162in}{1.677439in}}%
\pgfpathlineto{\pgfqpoint{2.188299in}{1.666599in}}%
\pgfpathlineto{\pgfqpoint{2.194440in}{1.655721in}}%
\pgfpathclose%
\pgfusepath{stroke,fill}%
\end{pgfscope}%
\begin{pgfscope}%
\pgfpathrectangle{\pgfqpoint{0.887500in}{0.275000in}}{\pgfqpoint{4.225000in}{4.225000in}}%
\pgfusepath{clip}%
\pgfsetbuttcap%
\pgfsetroundjoin%
\definecolor{currentfill}{rgb}{0.283229,0.120777,0.440584}%
\pgfsetfillcolor{currentfill}%
\pgfsetfillopacity{0.700000}%
\pgfsetlinewidth{0.501875pt}%
\definecolor{currentstroke}{rgb}{1.000000,1.000000,1.000000}%
\pgfsetstrokecolor{currentstroke}%
\pgfsetstrokeopacity{0.500000}%
\pgfsetdash{}{0pt}%
\pgfpathmoveto{\pgfqpoint{3.252117in}{1.250701in}}%
\pgfpathlineto{\pgfqpoint{3.263701in}{1.248868in}}%
\pgfpathlineto{\pgfqpoint{3.275268in}{1.245698in}}%
\pgfpathlineto{\pgfqpoint{3.286820in}{1.241694in}}%
\pgfpathlineto{\pgfqpoint{3.298359in}{1.237358in}}%
\pgfpathlineto{\pgfqpoint{3.309889in}{1.233189in}}%
\pgfpathlineto{\pgfqpoint{3.303451in}{1.247009in}}%
\pgfpathlineto{\pgfqpoint{3.297016in}{1.260802in}}%
\pgfpathlineto{\pgfqpoint{3.290583in}{1.274564in}}%
\pgfpathlineto{\pgfqpoint{3.284153in}{1.288291in}}%
\pgfpathlineto{\pgfqpoint{3.277726in}{1.301980in}}%
\pgfpathlineto{\pgfqpoint{3.266172in}{1.302080in}}%
\pgfpathlineto{\pgfqpoint{3.254610in}{1.302243in}}%
\pgfpathlineto{\pgfqpoint{3.243038in}{1.302192in}}%
\pgfpathlineto{\pgfqpoint{3.231457in}{1.301651in}}%
\pgfpathlineto{\pgfqpoint{3.219864in}{1.300342in}}%
\pgfpathlineto{\pgfqpoint{3.226307in}{1.290504in}}%
\pgfpathlineto{\pgfqpoint{3.232754in}{1.280689in}}%
\pgfpathlineto{\pgfqpoint{3.239205in}{1.270828in}}%
\pgfpathlineto{\pgfqpoint{3.245659in}{1.260855in}}%
\pgfpathclose%
\pgfusepath{stroke,fill}%
\end{pgfscope}%
\begin{pgfscope}%
\pgfpathrectangle{\pgfqpoint{0.887500in}{0.275000in}}{\pgfqpoint{4.225000in}{4.225000in}}%
\pgfusepath{clip}%
\pgfsetbuttcap%
\pgfsetroundjoin%
\definecolor{currentfill}{rgb}{0.281887,0.150881,0.465405}%
\pgfsetfillcolor{currentfill}%
\pgfsetfillopacity{0.700000}%
\pgfsetlinewidth{0.501875pt}%
\definecolor{currentstroke}{rgb}{1.000000,1.000000,1.000000}%
\pgfsetstrokecolor{currentstroke}%
\pgfsetstrokeopacity{0.500000}%
\pgfsetdash{}{0pt}%
\pgfpathmoveto{\pgfqpoint{3.161746in}{1.277163in}}%
\pgfpathlineto{\pgfqpoint{3.173386in}{1.283626in}}%
\pgfpathlineto{\pgfqpoint{3.185020in}{1.289451in}}%
\pgfpathlineto{\pgfqpoint{3.196645in}{1.294338in}}%
\pgfpathlineto{\pgfqpoint{3.208260in}{1.297989in}}%
\pgfpathlineto{\pgfqpoint{3.219864in}{1.300342in}}%
\pgfpathlineto{\pgfqpoint{3.213426in}{1.310269in}}%
\pgfpathlineto{\pgfqpoint{3.206993in}{1.320354in}}%
\pgfpathlineto{\pgfqpoint{3.200564in}{1.330663in}}%
\pgfpathlineto{\pgfqpoint{3.194141in}{1.341264in}}%
\pgfpathlineto{\pgfqpoint{3.187722in}{1.352224in}}%
\pgfpathlineto{\pgfqpoint{3.176122in}{1.348488in}}%
\pgfpathlineto{\pgfqpoint{3.164516in}{1.344601in}}%
\pgfpathlineto{\pgfqpoint{3.152904in}{1.340571in}}%
\pgfpathlineto{\pgfqpoint{3.141286in}{1.336433in}}%
\pgfpathlineto{\pgfqpoint{3.129662in}{1.332224in}}%
\pgfpathlineto{\pgfqpoint{3.136070in}{1.320585in}}%
\pgfpathlineto{\pgfqpoint{3.142482in}{1.309255in}}%
\pgfpathlineto{\pgfqpoint{3.148899in}{1.298240in}}%
\pgfpathlineto{\pgfqpoint{3.155320in}{1.287541in}}%
\pgfpathclose%
\pgfusepath{stroke,fill}%
\end{pgfscope}%
\begin{pgfscope}%
\pgfpathrectangle{\pgfqpoint{0.887500in}{0.275000in}}{\pgfqpoint{4.225000in}{4.225000in}}%
\pgfusepath{clip}%
\pgfsetbuttcap%
\pgfsetroundjoin%
\definecolor{currentfill}{rgb}{0.269944,0.014625,0.341379}%
\pgfsetfillcolor{currentfill}%
\pgfsetfillopacity{0.700000}%
\pgfsetlinewidth{0.501875pt}%
\definecolor{currentstroke}{rgb}{1.000000,1.000000,1.000000}%
\pgfsetstrokecolor{currentstroke}%
\pgfsetstrokeopacity{0.500000}%
\pgfsetdash{}{0pt}%
\pgfpathmoveto{\pgfqpoint{3.374419in}{1.094381in}}%
\pgfpathlineto{\pgfqpoint{3.385930in}{1.089937in}}%
\pgfpathlineto{\pgfqpoint{3.397436in}{1.086154in}}%
\pgfpathlineto{\pgfqpoint{3.408940in}{1.083203in}}%
\pgfpathlineto{\pgfqpoint{3.420443in}{1.081255in}}%
\pgfpathlineto{\pgfqpoint{3.431949in}{1.080483in}}%
\pgfpathlineto{\pgfqpoint{3.425468in}{1.092633in}}%
\pgfpathlineto{\pgfqpoint{3.418997in}{1.105470in}}%
\pgfpathlineto{\pgfqpoint{3.412535in}{1.118934in}}%
\pgfpathlineto{\pgfqpoint{3.406082in}{1.132970in}}%
\pgfpathlineto{\pgfqpoint{3.399635in}{1.147517in}}%
\pgfpathlineto{\pgfqpoint{3.388128in}{1.148042in}}%
\pgfpathlineto{\pgfqpoint{3.376626in}{1.150065in}}%
\pgfpathlineto{\pgfqpoint{3.365124in}{1.153456in}}%
\pgfpathlineto{\pgfqpoint{3.353623in}{1.158089in}}%
\pgfpathlineto{\pgfqpoint{3.342119in}{1.163827in}}%
\pgfpathlineto{\pgfqpoint{3.348573in}{1.149929in}}%
\pgfpathlineto{\pgfqpoint{3.355030in}{1.136031in}}%
\pgfpathlineto{\pgfqpoint{3.361490in}{1.122138in}}%
\pgfpathlineto{\pgfqpoint{3.367953in}{1.108254in}}%
\pgfpathclose%
\pgfusepath{stroke,fill}%
\end{pgfscope}%
\begin{pgfscope}%
\pgfpathrectangle{\pgfqpoint{0.887500in}{0.275000in}}{\pgfqpoint{4.225000in}{4.225000in}}%
\pgfusepath{clip}%
\pgfsetbuttcap%
\pgfsetroundjoin%
\definecolor{currentfill}{rgb}{0.221989,0.339161,0.548752}%
\pgfsetfillcolor{currentfill}%
\pgfsetfillopacity{0.700000}%
\pgfsetlinewidth{0.501875pt}%
\definecolor{currentstroke}{rgb}{1.000000,1.000000,1.000000}%
\pgfsetstrokecolor{currentstroke}%
\pgfsetstrokeopacity{0.500000}%
\pgfsetdash{}{0pt}%
\pgfpathmoveto{\pgfqpoint{2.290701in}{1.609995in}}%
\pgfpathlineto{\pgfqpoint{2.302547in}{1.614067in}}%
\pgfpathlineto{\pgfqpoint{2.314388in}{1.618139in}}%
\pgfpathlineto{\pgfqpoint{2.326224in}{1.622212in}}%
\pgfpathlineto{\pgfqpoint{2.338053in}{1.626283in}}%
\pgfpathlineto{\pgfqpoint{2.349877in}{1.630353in}}%
\pgfpathlineto{\pgfqpoint{2.343691in}{1.641435in}}%
\pgfpathlineto{\pgfqpoint{2.337509in}{1.652476in}}%
\pgfpathlineto{\pgfqpoint{2.331331in}{1.663476in}}%
\pgfpathlineto{\pgfqpoint{2.325158in}{1.674433in}}%
\pgfpathlineto{\pgfqpoint{2.318988in}{1.685348in}}%
\pgfpathlineto{\pgfqpoint{2.307174in}{1.681298in}}%
\pgfpathlineto{\pgfqpoint{2.295354in}{1.677246in}}%
\pgfpathlineto{\pgfqpoint{2.283528in}{1.673192in}}%
\pgfpathlineto{\pgfqpoint{2.271697in}{1.669137in}}%
\pgfpathlineto{\pgfqpoint{2.259859in}{1.665081in}}%
\pgfpathlineto{\pgfqpoint{2.266019in}{1.654145in}}%
\pgfpathlineto{\pgfqpoint{2.272183in}{1.643167in}}%
\pgfpathlineto{\pgfqpoint{2.278351in}{1.632149in}}%
\pgfpathlineto{\pgfqpoint{2.284524in}{1.621091in}}%
\pgfpathclose%
\pgfusepath{stroke,fill}%
\end{pgfscope}%
\begin{pgfscope}%
\pgfpathrectangle{\pgfqpoint{0.887500in}{0.275000in}}{\pgfqpoint{4.225000in}{4.225000in}}%
\pgfusepath{clip}%
\pgfsetbuttcap%
\pgfsetroundjoin%
\definecolor{currentfill}{rgb}{0.229739,0.322361,0.545706}%
\pgfsetfillcolor{currentfill}%
\pgfsetfillopacity{0.700000}%
\pgfsetlinewidth{0.501875pt}%
\definecolor{currentstroke}{rgb}{1.000000,1.000000,1.000000}%
\pgfsetstrokecolor{currentstroke}%
\pgfsetstrokeopacity{0.500000}%
\pgfsetdash{}{0pt}%
\pgfpathmoveto{\pgfqpoint{2.380870in}{1.574351in}}%
\pgfpathlineto{\pgfqpoint{2.392697in}{1.578431in}}%
\pgfpathlineto{\pgfqpoint{2.404519in}{1.582509in}}%
\pgfpathlineto{\pgfqpoint{2.416335in}{1.586585in}}%
\pgfpathlineto{\pgfqpoint{2.428145in}{1.590659in}}%
\pgfpathlineto{\pgfqpoint{2.439949in}{1.594730in}}%
\pgfpathlineto{\pgfqpoint{2.433733in}{1.605996in}}%
\pgfpathlineto{\pgfqpoint{2.427522in}{1.617223in}}%
\pgfpathlineto{\pgfqpoint{2.421314in}{1.628409in}}%
\pgfpathlineto{\pgfqpoint{2.415110in}{1.639555in}}%
\pgfpathlineto{\pgfqpoint{2.408910in}{1.650659in}}%
\pgfpathlineto{\pgfqpoint{2.397115in}{1.646605in}}%
\pgfpathlineto{\pgfqpoint{2.385314in}{1.642547in}}%
\pgfpathlineto{\pgfqpoint{2.373507in}{1.638485in}}%
\pgfpathlineto{\pgfqpoint{2.361695in}{1.634420in}}%
\pgfpathlineto{\pgfqpoint{2.349877in}{1.630353in}}%
\pgfpathlineto{\pgfqpoint{2.356067in}{1.619230in}}%
\pgfpathlineto{\pgfqpoint{2.362262in}{1.608068in}}%
\pgfpathlineto{\pgfqpoint{2.368460in}{1.596867in}}%
\pgfpathlineto{\pgfqpoint{2.374663in}{1.585628in}}%
\pgfpathclose%
\pgfusepath{stroke,fill}%
\end{pgfscope}%
\begin{pgfscope}%
\pgfpathrectangle{\pgfqpoint{0.887500in}{0.275000in}}{\pgfqpoint{4.225000in}{4.225000in}}%
\pgfusepath{clip}%
\pgfsetbuttcap%
\pgfsetroundjoin%
\definecolor{currentfill}{rgb}{0.237441,0.305202,0.541921}%
\pgfsetfillcolor{currentfill}%
\pgfsetfillopacity{0.700000}%
\pgfsetlinewidth{0.501875pt}%
\definecolor{currentstroke}{rgb}{1.000000,1.000000,1.000000}%
\pgfsetstrokecolor{currentstroke}%
\pgfsetstrokeopacity{0.500000}%
\pgfsetdash{}{0pt}%
\pgfpathmoveto{\pgfqpoint{2.471088in}{1.537829in}}%
\pgfpathlineto{\pgfqpoint{2.482896in}{1.541906in}}%
\pgfpathlineto{\pgfqpoint{2.494698in}{1.545980in}}%
\pgfpathlineto{\pgfqpoint{2.506494in}{1.550052in}}%
\pgfpathlineto{\pgfqpoint{2.518285in}{1.554121in}}%
\pgfpathlineto{\pgfqpoint{2.530070in}{1.558188in}}%
\pgfpathlineto{\pgfqpoint{2.523825in}{1.569637in}}%
\pgfpathlineto{\pgfqpoint{2.517585in}{1.581045in}}%
\pgfpathlineto{\pgfqpoint{2.511348in}{1.592414in}}%
\pgfpathlineto{\pgfqpoint{2.505115in}{1.603743in}}%
\pgfpathlineto{\pgfqpoint{2.498886in}{1.615033in}}%
\pgfpathlineto{\pgfqpoint{2.487110in}{1.610979in}}%
\pgfpathlineto{\pgfqpoint{2.475328in}{1.606922in}}%
\pgfpathlineto{\pgfqpoint{2.463541in}{1.602862in}}%
\pgfpathlineto{\pgfqpoint{2.451748in}{1.598798in}}%
\pgfpathlineto{\pgfqpoint{2.439949in}{1.594730in}}%
\pgfpathlineto{\pgfqpoint{2.446169in}{1.583426in}}%
\pgfpathlineto{\pgfqpoint{2.452393in}{1.572083in}}%
\pgfpathlineto{\pgfqpoint{2.458621in}{1.560702in}}%
\pgfpathlineto{\pgfqpoint{2.464852in}{1.549285in}}%
\pgfpathclose%
\pgfusepath{stroke,fill}%
\end{pgfscope}%
\begin{pgfscope}%
\pgfpathrectangle{\pgfqpoint{0.887500in}{0.275000in}}{\pgfqpoint{4.225000in}{4.225000in}}%
\pgfusepath{clip}%
\pgfsetbuttcap%
\pgfsetroundjoin%
\definecolor{currentfill}{rgb}{0.244972,0.287675,0.537260}%
\pgfsetfillcolor{currentfill}%
\pgfsetfillopacity{0.700000}%
\pgfsetlinewidth{0.501875pt}%
\definecolor{currentstroke}{rgb}{1.000000,1.000000,1.000000}%
\pgfsetstrokecolor{currentstroke}%
\pgfsetstrokeopacity{0.500000}%
\pgfsetdash{}{0pt}%
\pgfpathmoveto{\pgfqpoint{2.561352in}{1.500305in}}%
\pgfpathlineto{\pgfqpoint{2.573140in}{1.504372in}}%
\pgfpathlineto{\pgfqpoint{2.584922in}{1.508441in}}%
\pgfpathlineto{\pgfqpoint{2.596698in}{1.512512in}}%
\pgfpathlineto{\pgfqpoint{2.608469in}{1.516588in}}%
\pgfpathlineto{\pgfqpoint{2.620234in}{1.520669in}}%
\pgfpathlineto{\pgfqpoint{2.613961in}{1.532343in}}%
\pgfpathlineto{\pgfqpoint{2.607692in}{1.543966in}}%
\pgfpathlineto{\pgfqpoint{2.601427in}{1.555541in}}%
\pgfpathlineto{\pgfqpoint{2.595165in}{1.567068in}}%
\pgfpathlineto{\pgfqpoint{2.588908in}{1.578549in}}%
\pgfpathlineto{\pgfqpoint{2.577152in}{1.574470in}}%
\pgfpathlineto{\pgfqpoint{2.565390in}{1.570396in}}%
\pgfpathlineto{\pgfqpoint{2.553622in}{1.566325in}}%
\pgfpathlineto{\pgfqpoint{2.541849in}{1.562256in}}%
\pgfpathlineto{\pgfqpoint{2.530070in}{1.558188in}}%
\pgfpathlineto{\pgfqpoint{2.536318in}{1.546698in}}%
\pgfpathlineto{\pgfqpoint{2.542571in}{1.535166in}}%
\pgfpathlineto{\pgfqpoint{2.548827in}{1.523590in}}%
\pgfpathlineto{\pgfqpoint{2.555088in}{1.511970in}}%
\pgfpathclose%
\pgfusepath{stroke,fill}%
\end{pgfscope}%
\begin{pgfscope}%
\pgfpathrectangle{\pgfqpoint{0.887500in}{0.275000in}}{\pgfqpoint{4.225000in}{4.225000in}}%
\pgfusepath{clip}%
\pgfsetbuttcap%
\pgfsetroundjoin%
\definecolor{currentfill}{rgb}{0.253935,0.265254,0.529983}%
\pgfsetfillcolor{currentfill}%
\pgfsetfillopacity{0.700000}%
\pgfsetlinewidth{0.501875pt}%
\definecolor{currentstroke}{rgb}{1.000000,1.000000,1.000000}%
\pgfsetstrokecolor{currentstroke}%
\pgfsetstrokeopacity{0.500000}%
\pgfsetdash{}{0pt}%
\pgfpathmoveto{\pgfqpoint{2.651657in}{1.461490in}}%
\pgfpathlineto{\pgfqpoint{2.663424in}{1.465549in}}%
\pgfpathlineto{\pgfqpoint{2.675186in}{1.469613in}}%
\pgfpathlineto{\pgfqpoint{2.686942in}{1.473679in}}%
\pgfpathlineto{\pgfqpoint{2.698692in}{1.477747in}}%
\pgfpathlineto{\pgfqpoint{2.710437in}{1.481815in}}%
\pgfpathlineto{\pgfqpoint{2.704136in}{1.493799in}}%
\pgfpathlineto{\pgfqpoint{2.697839in}{1.505724in}}%
\pgfpathlineto{\pgfqpoint{2.691546in}{1.517586in}}%
\pgfpathlineto{\pgfqpoint{2.685256in}{1.529384in}}%
\pgfpathlineto{\pgfqpoint{2.678971in}{1.541121in}}%
\pgfpathlineto{\pgfqpoint{2.667235in}{1.537028in}}%
\pgfpathlineto{\pgfqpoint{2.655493in}{1.532935in}}%
\pgfpathlineto{\pgfqpoint{2.643746in}{1.528843in}}%
\pgfpathlineto{\pgfqpoint{2.631993in}{1.524754in}}%
\pgfpathlineto{\pgfqpoint{2.620234in}{1.520669in}}%
\pgfpathlineto{\pgfqpoint{2.626510in}{1.508942in}}%
\pgfpathlineto{\pgfqpoint{2.632791in}{1.497162in}}%
\pgfpathlineto{\pgfqpoint{2.639076in}{1.485326in}}%
\pgfpathlineto{\pgfqpoint{2.645365in}{1.473434in}}%
\pgfpathclose%
\pgfusepath{stroke,fill}%
\end{pgfscope}%
\begin{pgfscope}%
\pgfpathrectangle{\pgfqpoint{0.887500in}{0.275000in}}{\pgfqpoint{4.225000in}{4.225000in}}%
\pgfusepath{clip}%
\pgfsetbuttcap%
\pgfsetroundjoin%
\definecolor{currentfill}{rgb}{0.208623,0.367752,0.552675}%
\pgfsetfillcolor{currentfill}%
\pgfsetfillopacity{0.700000}%
\pgfsetlinewidth{0.501875pt}%
\definecolor{currentstroke}{rgb}{1.000000,1.000000,1.000000}%
\pgfsetstrokecolor{currentstroke}%
\pgfsetstrokeopacity{0.500000}%
\pgfsetdash{}{0pt}%
\pgfpathmoveto{\pgfqpoint{2.051020in}{1.658500in}}%
\pgfpathlineto{\pgfqpoint{2.062934in}{1.662574in}}%
\pgfpathlineto{\pgfqpoint{2.074842in}{1.666642in}}%
\pgfpathlineto{\pgfqpoint{2.086744in}{1.670703in}}%
\pgfpathlineto{\pgfqpoint{2.098640in}{1.674758in}}%
\pgfpathlineto{\pgfqpoint{2.110531in}{1.678808in}}%
\pgfpathlineto{\pgfqpoint{2.104417in}{1.689561in}}%
\pgfpathlineto{\pgfqpoint{2.098307in}{1.700276in}}%
\pgfpathlineto{\pgfqpoint{2.092201in}{1.710954in}}%
\pgfpathlineto{\pgfqpoint{2.086099in}{1.721593in}}%
\pgfpathlineto{\pgfqpoint{2.080002in}{1.732195in}}%
\pgfpathlineto{\pgfqpoint{2.068121in}{1.728175in}}%
\pgfpathlineto{\pgfqpoint{2.056235in}{1.724150in}}%
\pgfpathlineto{\pgfqpoint{2.044342in}{1.720119in}}%
\pgfpathlineto{\pgfqpoint{2.032444in}{1.716081in}}%
\pgfpathlineto{\pgfqpoint{2.020541in}{1.712035in}}%
\pgfpathlineto{\pgfqpoint{2.026628in}{1.701402in}}%
\pgfpathlineto{\pgfqpoint{2.032719in}{1.690731in}}%
\pgfpathlineto{\pgfqpoint{2.038815in}{1.680023in}}%
\pgfpathlineto{\pgfqpoint{2.044915in}{1.669279in}}%
\pgfpathclose%
\pgfusepath{stroke,fill}%
\end{pgfscope}%
\begin{pgfscope}%
\pgfpathrectangle{\pgfqpoint{0.887500in}{0.275000in}}{\pgfqpoint{4.225000in}{4.225000in}}%
\pgfusepath{clip}%
\pgfsetbuttcap%
\pgfsetroundjoin%
\definecolor{currentfill}{rgb}{0.262138,0.242286,0.520837}%
\pgfsetfillcolor{currentfill}%
\pgfsetfillopacity{0.700000}%
\pgfsetlinewidth{0.501875pt}%
\definecolor{currentstroke}{rgb}{1.000000,1.000000,1.000000}%
\pgfsetstrokecolor{currentstroke}%
\pgfsetstrokeopacity{0.500000}%
\pgfsetdash{}{0pt}%
\pgfpathmoveto{\pgfqpoint{2.741995in}{1.421309in}}%
\pgfpathlineto{\pgfqpoint{2.753742in}{1.425328in}}%
\pgfpathlineto{\pgfqpoint{2.765483in}{1.429346in}}%
\pgfpathlineto{\pgfqpoint{2.777218in}{1.433363in}}%
\pgfpathlineto{\pgfqpoint{2.788948in}{1.437379in}}%
\pgfpathlineto{\pgfqpoint{2.800672in}{1.441395in}}%
\pgfpathlineto{\pgfqpoint{2.794345in}{1.453588in}}%
\pgfpathlineto{\pgfqpoint{2.788022in}{1.465769in}}%
\pgfpathlineto{\pgfqpoint{2.781702in}{1.477926in}}%
\pgfpathlineto{\pgfqpoint{2.775386in}{1.490050in}}%
\pgfpathlineto{\pgfqpoint{2.769073in}{1.502129in}}%
\pgfpathlineto{\pgfqpoint{2.757357in}{1.498073in}}%
\pgfpathlineto{\pgfqpoint{2.745636in}{1.494013in}}%
\pgfpathlineto{\pgfqpoint{2.733909in}{1.489949in}}%
\pgfpathlineto{\pgfqpoint{2.722176in}{1.485883in}}%
\pgfpathlineto{\pgfqpoint{2.710437in}{1.481815in}}%
\pgfpathlineto{\pgfqpoint{2.716741in}{1.469782in}}%
\pgfpathlineto{\pgfqpoint{2.723050in}{1.457706in}}%
\pgfpathlineto{\pgfqpoint{2.729361in}{1.445597in}}%
\pgfpathlineto{\pgfqpoint{2.735676in}{1.433461in}}%
\pgfpathclose%
\pgfusepath{stroke,fill}%
\end{pgfscope}%
\begin{pgfscope}%
\pgfpathrectangle{\pgfqpoint{0.887500in}{0.275000in}}{\pgfqpoint{4.225000in}{4.225000in}}%
\pgfusepath{clip}%
\pgfsetbuttcap%
\pgfsetroundjoin%
\definecolor{currentfill}{rgb}{0.267968,0.223549,0.512008}%
\pgfsetfillcolor{currentfill}%
\pgfsetfillopacity{0.700000}%
\pgfsetlinewidth{0.501875pt}%
\definecolor{currentstroke}{rgb}{1.000000,1.000000,1.000000}%
\pgfsetstrokecolor{currentstroke}%
\pgfsetstrokeopacity{0.500000}%
\pgfsetdash{}{0pt}%
\pgfpathmoveto{\pgfqpoint{2.832352in}{1.380597in}}%
\pgfpathlineto{\pgfqpoint{2.844077in}{1.384605in}}%
\pgfpathlineto{\pgfqpoint{2.855797in}{1.388607in}}%
\pgfpathlineto{\pgfqpoint{2.867511in}{1.392600in}}%
\pgfpathlineto{\pgfqpoint{2.879220in}{1.396583in}}%
\pgfpathlineto{\pgfqpoint{2.890923in}{1.400557in}}%
\pgfpathlineto{\pgfqpoint{2.884573in}{1.412700in}}%
\pgfpathlineto{\pgfqpoint{2.878226in}{1.424864in}}%
\pgfpathlineto{\pgfqpoint{2.871882in}{1.437059in}}%
\pgfpathlineto{\pgfqpoint{2.865542in}{1.449273in}}%
\pgfpathlineto{\pgfqpoint{2.859205in}{1.461495in}}%
\pgfpathlineto{\pgfqpoint{2.847510in}{1.457470in}}%
\pgfpathlineto{\pgfqpoint{2.835809in}{1.453448in}}%
\pgfpathlineto{\pgfqpoint{2.824102in}{1.449429in}}%
\pgfpathlineto{\pgfqpoint{2.812390in}{1.445411in}}%
\pgfpathlineto{\pgfqpoint{2.800672in}{1.441395in}}%
\pgfpathlineto{\pgfqpoint{2.807001in}{1.429200in}}%
\pgfpathlineto{\pgfqpoint{2.813334in}{1.417014in}}%
\pgfpathlineto{\pgfqpoint{2.819670in}{1.404847in}}%
\pgfpathlineto{\pgfqpoint{2.826009in}{1.392711in}}%
\pgfpathclose%
\pgfusepath{stroke,fill}%
\end{pgfscope}%
\begin{pgfscope}%
\pgfpathrectangle{\pgfqpoint{0.887500in}{0.275000in}}{\pgfqpoint{4.225000in}{4.225000in}}%
\pgfusepath{clip}%
\pgfsetbuttcap%
\pgfsetroundjoin%
\definecolor{currentfill}{rgb}{0.282623,0.140926,0.457517}%
\pgfsetfillcolor{currentfill}%
\pgfsetfillopacity{0.700000}%
\pgfsetlinewidth{0.501875pt}%
\definecolor{currentstroke}{rgb}{1.000000,1.000000,1.000000}%
\pgfsetstrokecolor{currentstroke}%
\pgfsetstrokeopacity{0.500000}%
\pgfsetdash{}{0pt}%
\pgfpathmoveto{\pgfqpoint{3.103484in}{1.245481in}}%
\pgfpathlineto{\pgfqpoint{3.115144in}{1.250940in}}%
\pgfpathlineto{\pgfqpoint{3.126800in}{1.256966in}}%
\pgfpathlineto{\pgfqpoint{3.138452in}{1.263531in}}%
\pgfpathlineto{\pgfqpoint{3.150101in}{1.270364in}}%
\pgfpathlineto{\pgfqpoint{3.161746in}{1.277163in}}%
\pgfpathlineto{\pgfqpoint{3.155320in}{1.287541in}}%
\pgfpathlineto{\pgfqpoint{3.148899in}{1.298240in}}%
\pgfpathlineto{\pgfqpoint{3.142482in}{1.309255in}}%
\pgfpathlineto{\pgfqpoint{3.136070in}{1.320585in}}%
\pgfpathlineto{\pgfqpoint{3.129662in}{1.332224in}}%
\pgfpathlineto{\pgfqpoint{3.118032in}{1.327981in}}%
\pgfpathlineto{\pgfqpoint{3.106397in}{1.323739in}}%
\pgfpathlineto{\pgfqpoint{3.094756in}{1.319536in}}%
\pgfpathlineto{\pgfqpoint{3.083110in}{1.315404in}}%
\pgfpathlineto{\pgfqpoint{3.071458in}{1.311347in}}%
\pgfpathlineto{\pgfqpoint{3.077857in}{1.297840in}}%
\pgfpathlineto{\pgfqpoint{3.084259in}{1.284389in}}%
\pgfpathlineto{\pgfqpoint{3.090664in}{1.271104in}}%
\pgfpathlineto{\pgfqpoint{3.097072in}{1.258098in}}%
\pgfpathclose%
\pgfusepath{stroke,fill}%
\end{pgfscope}%
\begin{pgfscope}%
\pgfpathrectangle{\pgfqpoint{0.887500in}{0.275000in}}{\pgfqpoint{4.225000in}{4.225000in}}%
\pgfusepath{clip}%
\pgfsetbuttcap%
\pgfsetroundjoin%
\definecolor{currentfill}{rgb}{0.279574,0.170599,0.479997}%
\pgfsetfillcolor{currentfill}%
\pgfsetfillopacity{0.700000}%
\pgfsetlinewidth{0.501875pt}%
\definecolor{currentstroke}{rgb}{1.000000,1.000000,1.000000}%
\pgfsetstrokecolor{currentstroke}%
\pgfsetstrokeopacity{0.500000}%
\pgfsetdash{}{0pt}%
\pgfpathmoveto{\pgfqpoint{3.013112in}{1.291785in}}%
\pgfpathlineto{\pgfqpoint{3.024793in}{1.295639in}}%
\pgfpathlineto{\pgfqpoint{3.036468in}{1.299511in}}%
\pgfpathlineto{\pgfqpoint{3.048137in}{1.303412in}}%
\pgfpathlineto{\pgfqpoint{3.059801in}{1.307353in}}%
\pgfpathlineto{\pgfqpoint{3.071458in}{1.311347in}}%
\pgfpathlineto{\pgfqpoint{3.065062in}{1.324798in}}%
\pgfpathlineto{\pgfqpoint{3.058669in}{1.338106in}}%
\pgfpathlineto{\pgfqpoint{3.052279in}{1.351265in}}%
\pgfpathlineto{\pgfqpoint{3.045891in}{1.364289in}}%
\pgfpathlineto{\pgfqpoint{3.039507in}{1.377190in}}%
\pgfpathlineto{\pgfqpoint{3.027855in}{1.373479in}}%
\pgfpathlineto{\pgfqpoint{3.016197in}{1.369703in}}%
\pgfpathlineto{\pgfqpoint{3.004534in}{1.365881in}}%
\pgfpathlineto{\pgfqpoint{2.992864in}{1.362032in}}%
\pgfpathlineto{\pgfqpoint{2.981189in}{1.358176in}}%
\pgfpathlineto{\pgfqpoint{2.987567in}{1.345341in}}%
\pgfpathlineto{\pgfqpoint{2.993948in}{1.332310in}}%
\pgfpathlineto{\pgfqpoint{3.000333in}{1.319056in}}%
\pgfpathlineto{\pgfqpoint{3.006721in}{1.305549in}}%
\pgfpathclose%
\pgfusepath{stroke,fill}%
\end{pgfscope}%
\begin{pgfscope}%
\pgfpathrectangle{\pgfqpoint{0.887500in}{0.275000in}}{\pgfqpoint{4.225000in}{4.225000in}}%
\pgfusepath{clip}%
\pgfsetbuttcap%
\pgfsetroundjoin%
\definecolor{currentfill}{rgb}{0.274128,0.199721,0.498911}%
\pgfsetfillcolor{currentfill}%
\pgfsetfillopacity{0.700000}%
\pgfsetlinewidth{0.501875pt}%
\definecolor{currentstroke}{rgb}{1.000000,1.000000,1.000000}%
\pgfsetstrokecolor{currentstroke}%
\pgfsetstrokeopacity{0.500000}%
\pgfsetdash{}{0pt}%
\pgfpathmoveto{\pgfqpoint{2.922724in}{1.338949in}}%
\pgfpathlineto{\pgfqpoint{2.934429in}{1.342806in}}%
\pgfpathlineto{\pgfqpoint{2.946128in}{1.346651in}}%
\pgfpathlineto{\pgfqpoint{2.957821in}{1.350489in}}%
\pgfpathlineto{\pgfqpoint{2.969508in}{1.354328in}}%
\pgfpathlineto{\pgfqpoint{2.981189in}{1.358176in}}%
\pgfpathlineto{\pgfqpoint{2.974814in}{1.370843in}}%
\pgfpathlineto{\pgfqpoint{2.968443in}{1.383371in}}%
\pgfpathlineto{\pgfqpoint{2.962075in}{1.395787in}}%
\pgfpathlineto{\pgfqpoint{2.955711in}{1.408121in}}%
\pgfpathlineto{\pgfqpoint{2.949350in}{1.420400in}}%
\pgfpathlineto{\pgfqpoint{2.937676in}{1.416425in}}%
\pgfpathlineto{\pgfqpoint{2.925996in}{1.412456in}}%
\pgfpathlineto{\pgfqpoint{2.914311in}{1.408490in}}%
\pgfpathlineto{\pgfqpoint{2.902620in}{1.404525in}}%
\pgfpathlineto{\pgfqpoint{2.890923in}{1.400557in}}%
\pgfpathlineto{\pgfqpoint{2.897276in}{1.388401in}}%
\pgfpathlineto{\pgfqpoint{2.903633in}{1.376198in}}%
\pgfpathlineto{\pgfqpoint{2.909993in}{1.363911in}}%
\pgfpathlineto{\pgfqpoint{2.916357in}{1.351507in}}%
\pgfpathclose%
\pgfusepath{stroke,fill}%
\end{pgfscope}%
\begin{pgfscope}%
\pgfpathrectangle{\pgfqpoint{0.887500in}{0.275000in}}{\pgfqpoint{4.225000in}{4.225000in}}%
\pgfusepath{clip}%
\pgfsetbuttcap%
\pgfsetroundjoin%
\definecolor{currentfill}{rgb}{0.214298,0.355619,0.551184}%
\pgfsetfillcolor{currentfill}%
\pgfsetfillopacity{0.700000}%
\pgfsetlinewidth{0.501875pt}%
\definecolor{currentstroke}{rgb}{1.000000,1.000000,1.000000}%
\pgfsetstrokecolor{currentstroke}%
\pgfsetstrokeopacity{0.500000}%
\pgfsetdash{}{0pt}%
\pgfpathmoveto{\pgfqpoint{2.141169in}{1.624499in}}%
\pgfpathlineto{\pgfqpoint{2.153064in}{1.628566in}}%
\pgfpathlineto{\pgfqpoint{2.164953in}{1.632629in}}%
\pgfpathlineto{\pgfqpoint{2.176836in}{1.636690in}}%
\pgfpathlineto{\pgfqpoint{2.188714in}{1.640748in}}%
\pgfpathlineto{\pgfqpoint{2.200586in}{1.644805in}}%
\pgfpathlineto{\pgfqpoint{2.194440in}{1.655721in}}%
\pgfpathlineto{\pgfqpoint{2.188299in}{1.666599in}}%
\pgfpathlineto{\pgfqpoint{2.182162in}{1.677439in}}%
\pgfpathlineto{\pgfqpoint{2.176029in}{1.688240in}}%
\pgfpathlineto{\pgfqpoint{2.169900in}{1.699002in}}%
\pgfpathlineto{\pgfqpoint{2.158038in}{1.694969in}}%
\pgfpathlineto{\pgfqpoint{2.146170in}{1.690934in}}%
\pgfpathlineto{\pgfqpoint{2.134296in}{1.686895in}}%
\pgfpathlineto{\pgfqpoint{2.122416in}{1.682854in}}%
\pgfpathlineto{\pgfqpoint{2.110531in}{1.678808in}}%
\pgfpathlineto{\pgfqpoint{2.116650in}{1.668019in}}%
\pgfpathlineto{\pgfqpoint{2.122773in}{1.657193in}}%
\pgfpathlineto{\pgfqpoint{2.128901in}{1.646330in}}%
\pgfpathlineto{\pgfqpoint{2.135033in}{1.635433in}}%
\pgfpathclose%
\pgfusepath{stroke,fill}%
\end{pgfscope}%
\begin{pgfscope}%
\pgfpathrectangle{\pgfqpoint{0.887500in}{0.275000in}}{\pgfqpoint{4.225000in}{4.225000in}}%
\pgfusepath{clip}%
\pgfsetbuttcap%
\pgfsetroundjoin%
\definecolor{currentfill}{rgb}{0.282327,0.094955,0.417331}%
\pgfsetfillcolor{currentfill}%
\pgfsetfillopacity{0.700000}%
\pgfsetlinewidth{0.501875pt}%
\definecolor{currentstroke}{rgb}{1.000000,1.000000,1.000000}%
\pgfsetstrokecolor{currentstroke}%
\pgfsetstrokeopacity{0.500000}%
\pgfsetdash{}{0pt}%
\pgfpathmoveto{\pgfqpoint{3.284437in}{1.194865in}}%
\pgfpathlineto{\pgfqpoint{3.296005in}{1.189972in}}%
\pgfpathlineto{\pgfqpoint{3.307555in}{1.183930in}}%
\pgfpathlineto{\pgfqpoint{3.319089in}{1.177232in}}%
\pgfpathlineto{\pgfqpoint{3.330609in}{1.170368in}}%
\pgfpathlineto{\pgfqpoint{3.342119in}{1.163827in}}%
\pgfpathlineto{\pgfqpoint{3.335667in}{1.177722in}}%
\pgfpathlineto{\pgfqpoint{3.329218in}{1.191609in}}%
\pgfpathlineto{\pgfqpoint{3.322773in}{1.205486in}}%
\pgfpathlineto{\pgfqpoint{3.316329in}{1.219347in}}%
\pgfpathlineto{\pgfqpoint{3.309889in}{1.233189in}}%
\pgfpathlineto{\pgfqpoint{3.298359in}{1.237358in}}%
\pgfpathlineto{\pgfqpoint{3.286820in}{1.241694in}}%
\pgfpathlineto{\pgfqpoint{3.275268in}{1.245698in}}%
\pgfpathlineto{\pgfqpoint{3.263701in}{1.248868in}}%
\pgfpathlineto{\pgfqpoint{3.252117in}{1.250701in}}%
\pgfpathlineto{\pgfqpoint{3.258578in}{1.240300in}}%
\pgfpathlineto{\pgfqpoint{3.265041in}{1.229583in}}%
\pgfpathlineto{\pgfqpoint{3.271505in}{1.218483in}}%
\pgfpathlineto{\pgfqpoint{3.277971in}{1.206933in}}%
\pgfpathclose%
\pgfusepath{stroke,fill}%
\end{pgfscope}%
\begin{pgfscope}%
\pgfpathrectangle{\pgfqpoint{0.887500in}{0.275000in}}{\pgfqpoint{4.225000in}{4.225000in}}%
\pgfusepath{clip}%
\pgfsetbuttcap%
\pgfsetroundjoin%
\definecolor{currentfill}{rgb}{0.221989,0.339161,0.548752}%
\pgfsetfillcolor{currentfill}%
\pgfsetfillopacity{0.700000}%
\pgfsetlinewidth{0.501875pt}%
\definecolor{currentstroke}{rgb}{1.000000,1.000000,1.000000}%
\pgfsetstrokecolor{currentstroke}%
\pgfsetstrokeopacity{0.500000}%
\pgfsetdash{}{0pt}%
\pgfpathmoveto{\pgfqpoint{2.231379in}{1.589654in}}%
\pgfpathlineto{\pgfqpoint{2.243255in}{1.593720in}}%
\pgfpathlineto{\pgfqpoint{2.255125in}{1.597786in}}%
\pgfpathlineto{\pgfqpoint{2.266989in}{1.601854in}}%
\pgfpathlineto{\pgfqpoint{2.278848in}{1.605923in}}%
\pgfpathlineto{\pgfqpoint{2.290701in}{1.609995in}}%
\pgfpathlineto{\pgfqpoint{2.284524in}{1.621091in}}%
\pgfpathlineto{\pgfqpoint{2.278351in}{1.632149in}}%
\pgfpathlineto{\pgfqpoint{2.272183in}{1.643167in}}%
\pgfpathlineto{\pgfqpoint{2.266019in}{1.654145in}}%
\pgfpathlineto{\pgfqpoint{2.259859in}{1.665081in}}%
\pgfpathlineto{\pgfqpoint{2.248016in}{1.661025in}}%
\pgfpathlineto{\pgfqpoint{2.236167in}{1.656970in}}%
\pgfpathlineto{\pgfqpoint{2.224313in}{1.652915in}}%
\pgfpathlineto{\pgfqpoint{2.212452in}{1.648860in}}%
\pgfpathlineto{\pgfqpoint{2.200586in}{1.644805in}}%
\pgfpathlineto{\pgfqpoint{2.206736in}{1.633850in}}%
\pgfpathlineto{\pgfqpoint{2.212891in}{1.622857in}}%
\pgfpathlineto{\pgfqpoint{2.219049in}{1.611827in}}%
\pgfpathlineto{\pgfqpoint{2.225212in}{1.600758in}}%
\pgfpathclose%
\pgfusepath{stroke,fill}%
\end{pgfscope}%
\begin{pgfscope}%
\pgfpathrectangle{\pgfqpoint{0.887500in}{0.275000in}}{\pgfqpoint{4.225000in}{4.225000in}}%
\pgfusepath{clip}%
\pgfsetbuttcap%
\pgfsetroundjoin%
\definecolor{currentfill}{rgb}{0.283072,0.130895,0.449241}%
\pgfsetfillcolor{currentfill}%
\pgfsetfillopacity{0.700000}%
\pgfsetlinewidth{0.501875pt}%
\definecolor{currentstroke}{rgb}{1.000000,1.000000,1.000000}%
\pgfsetstrokecolor{currentstroke}%
\pgfsetstrokeopacity{0.500000}%
\pgfsetdash{}{0pt}%
\pgfpathmoveto{\pgfqpoint{3.193953in}{1.230201in}}%
\pgfpathlineto{\pgfqpoint{3.205610in}{1.237636in}}%
\pgfpathlineto{\pgfqpoint{3.217259in}{1.243868in}}%
\pgfpathlineto{\pgfqpoint{3.228895in}{1.248387in}}%
\pgfpathlineto{\pgfqpoint{3.240515in}{1.250695in}}%
\pgfpathlineto{\pgfqpoint{3.252117in}{1.250701in}}%
\pgfpathlineto{\pgfqpoint{3.245659in}{1.260855in}}%
\pgfpathlineto{\pgfqpoint{3.239205in}{1.270828in}}%
\pgfpathlineto{\pgfqpoint{3.232754in}{1.280689in}}%
\pgfpathlineto{\pgfqpoint{3.226307in}{1.290504in}}%
\pgfpathlineto{\pgfqpoint{3.219864in}{1.300342in}}%
\pgfpathlineto{\pgfqpoint{3.208260in}{1.297989in}}%
\pgfpathlineto{\pgfqpoint{3.196645in}{1.294338in}}%
\pgfpathlineto{\pgfqpoint{3.185020in}{1.289451in}}%
\pgfpathlineto{\pgfqpoint{3.173386in}{1.283626in}}%
\pgfpathlineto{\pgfqpoint{3.161746in}{1.277163in}}%
\pgfpathlineto{\pgfqpoint{3.168177in}{1.267109in}}%
\pgfpathlineto{\pgfqpoint{3.174613in}{1.257382in}}%
\pgfpathlineto{\pgfqpoint{3.181054in}{1.247986in}}%
\pgfpathlineto{\pgfqpoint{3.187501in}{1.238924in}}%
\pgfpathclose%
\pgfusepath{stroke,fill}%
\end{pgfscope}%
\begin{pgfscope}%
\pgfpathrectangle{\pgfqpoint{0.887500in}{0.275000in}}{\pgfqpoint{4.225000in}{4.225000in}}%
\pgfusepath{clip}%
\pgfsetbuttcap%
\pgfsetroundjoin%
\definecolor{currentfill}{rgb}{0.229739,0.322361,0.545706}%
\pgfsetfillcolor{currentfill}%
\pgfsetfillopacity{0.700000}%
\pgfsetlinewidth{0.501875pt}%
\definecolor{currentstroke}{rgb}{1.000000,1.000000,1.000000}%
\pgfsetstrokecolor{currentstroke}%
\pgfsetstrokeopacity{0.500000}%
\pgfsetdash{}{0pt}%
\pgfpathmoveto{\pgfqpoint{2.321646in}{1.553953in}}%
\pgfpathlineto{\pgfqpoint{2.333502in}{1.558030in}}%
\pgfpathlineto{\pgfqpoint{2.345353in}{1.562109in}}%
\pgfpathlineto{\pgfqpoint{2.357197in}{1.566190in}}%
\pgfpathlineto{\pgfqpoint{2.369036in}{1.570270in}}%
\pgfpathlineto{\pgfqpoint{2.380870in}{1.574351in}}%
\pgfpathlineto{\pgfqpoint{2.374663in}{1.585628in}}%
\pgfpathlineto{\pgfqpoint{2.368460in}{1.596867in}}%
\pgfpathlineto{\pgfqpoint{2.362262in}{1.608068in}}%
\pgfpathlineto{\pgfqpoint{2.356067in}{1.619230in}}%
\pgfpathlineto{\pgfqpoint{2.349877in}{1.630353in}}%
\pgfpathlineto{\pgfqpoint{2.338053in}{1.626283in}}%
\pgfpathlineto{\pgfqpoint{2.326224in}{1.622212in}}%
\pgfpathlineto{\pgfqpoint{2.314388in}{1.618139in}}%
\pgfpathlineto{\pgfqpoint{2.302547in}{1.614067in}}%
\pgfpathlineto{\pgfqpoint{2.290701in}{1.609995in}}%
\pgfpathlineto{\pgfqpoint{2.296881in}{1.598859in}}%
\pgfpathlineto{\pgfqpoint{2.303066in}{1.587687in}}%
\pgfpathlineto{\pgfqpoint{2.309255in}{1.576478in}}%
\pgfpathlineto{\pgfqpoint{2.315449in}{1.565233in}}%
\pgfpathclose%
\pgfusepath{stroke,fill}%
\end{pgfscope}%
\begin{pgfscope}%
\pgfpathrectangle{\pgfqpoint{0.887500in}{0.275000in}}{\pgfqpoint{4.225000in}{4.225000in}}%
\pgfusepath{clip}%
\pgfsetbuttcap%
\pgfsetroundjoin%
\definecolor{currentfill}{rgb}{0.237441,0.305202,0.541921}%
\pgfsetfillcolor{currentfill}%
\pgfsetfillopacity{0.700000}%
\pgfsetlinewidth{0.501875pt}%
\definecolor{currentstroke}{rgb}{1.000000,1.000000,1.000000}%
\pgfsetstrokecolor{currentstroke}%
\pgfsetstrokeopacity{0.500000}%
\pgfsetdash{}{0pt}%
\pgfpathmoveto{\pgfqpoint{2.411963in}{1.517431in}}%
\pgfpathlineto{\pgfqpoint{2.423799in}{1.521511in}}%
\pgfpathlineto{\pgfqpoint{2.435630in}{1.525591in}}%
\pgfpathlineto{\pgfqpoint{2.447455in}{1.529671in}}%
\pgfpathlineto{\pgfqpoint{2.459275in}{1.533751in}}%
\pgfpathlineto{\pgfqpoint{2.471088in}{1.537829in}}%
\pgfpathlineto{\pgfqpoint{2.464852in}{1.549285in}}%
\pgfpathlineto{\pgfqpoint{2.458621in}{1.560702in}}%
\pgfpathlineto{\pgfqpoint{2.452393in}{1.572083in}}%
\pgfpathlineto{\pgfqpoint{2.446169in}{1.583426in}}%
\pgfpathlineto{\pgfqpoint{2.439949in}{1.594730in}}%
\pgfpathlineto{\pgfqpoint{2.428145in}{1.590659in}}%
\pgfpathlineto{\pgfqpoint{2.416335in}{1.586585in}}%
\pgfpathlineto{\pgfqpoint{2.404519in}{1.582509in}}%
\pgfpathlineto{\pgfqpoint{2.392697in}{1.578431in}}%
\pgfpathlineto{\pgfqpoint{2.380870in}{1.574351in}}%
\pgfpathlineto{\pgfqpoint{2.387080in}{1.563038in}}%
\pgfpathlineto{\pgfqpoint{2.393295in}{1.551689in}}%
\pgfpathlineto{\pgfqpoint{2.399514in}{1.540304in}}%
\pgfpathlineto{\pgfqpoint{2.405736in}{1.528885in}}%
\pgfpathclose%
\pgfusepath{stroke,fill}%
\end{pgfscope}%
\begin{pgfscope}%
\pgfpathrectangle{\pgfqpoint{0.887500in}{0.275000in}}{\pgfqpoint{4.225000in}{4.225000in}}%
\pgfusepath{clip}%
\pgfsetbuttcap%
\pgfsetroundjoin%
\definecolor{currentfill}{rgb}{0.246811,0.283237,0.535941}%
\pgfsetfillcolor{currentfill}%
\pgfsetfillopacity{0.700000}%
\pgfsetlinewidth{0.501875pt}%
\definecolor{currentstroke}{rgb}{1.000000,1.000000,1.000000}%
\pgfsetstrokecolor{currentstroke}%
\pgfsetstrokeopacity{0.500000}%
\pgfsetdash{}{0pt}%
\pgfpathmoveto{\pgfqpoint{2.502326in}{1.479963in}}%
\pgfpathlineto{\pgfqpoint{2.514142in}{1.484034in}}%
\pgfpathlineto{\pgfqpoint{2.525953in}{1.488104in}}%
\pgfpathlineto{\pgfqpoint{2.537759in}{1.492172in}}%
\pgfpathlineto{\pgfqpoint{2.549558in}{1.496238in}}%
\pgfpathlineto{\pgfqpoint{2.561352in}{1.500305in}}%
\pgfpathlineto{\pgfqpoint{2.555088in}{1.511970in}}%
\pgfpathlineto{\pgfqpoint{2.548827in}{1.523590in}}%
\pgfpathlineto{\pgfqpoint{2.542571in}{1.535166in}}%
\pgfpathlineto{\pgfqpoint{2.536318in}{1.546698in}}%
\pgfpathlineto{\pgfqpoint{2.530070in}{1.558188in}}%
\pgfpathlineto{\pgfqpoint{2.518285in}{1.554121in}}%
\pgfpathlineto{\pgfqpoint{2.506494in}{1.550052in}}%
\pgfpathlineto{\pgfqpoint{2.494698in}{1.545980in}}%
\pgfpathlineto{\pgfqpoint{2.482896in}{1.541906in}}%
\pgfpathlineto{\pgfqpoint{2.471088in}{1.537829in}}%
\pgfpathlineto{\pgfqpoint{2.477328in}{1.526335in}}%
\pgfpathlineto{\pgfqpoint{2.483571in}{1.514802in}}%
\pgfpathlineto{\pgfqpoint{2.489819in}{1.503229in}}%
\pgfpathlineto{\pgfqpoint{2.496070in}{1.491617in}}%
\pgfpathclose%
\pgfusepath{stroke,fill}%
\end{pgfscope}%
\begin{pgfscope}%
\pgfpathrectangle{\pgfqpoint{0.887500in}{0.275000in}}{\pgfqpoint{4.225000in}{4.225000in}}%
\pgfusepath{clip}%
\pgfsetbuttcap%
\pgfsetroundjoin%
\definecolor{currentfill}{rgb}{0.253935,0.265254,0.529983}%
\pgfsetfillcolor{currentfill}%
\pgfsetfillopacity{0.700000}%
\pgfsetlinewidth{0.501875pt}%
\definecolor{currentstroke}{rgb}{1.000000,1.000000,1.000000}%
\pgfsetstrokecolor{currentstroke}%
\pgfsetstrokeopacity{0.500000}%
\pgfsetdash{}{0pt}%
\pgfpathmoveto{\pgfqpoint{2.592732in}{1.441265in}}%
\pgfpathlineto{\pgfqpoint{2.604528in}{1.445304in}}%
\pgfpathlineto{\pgfqpoint{2.616319in}{1.449344in}}%
\pgfpathlineto{\pgfqpoint{2.628104in}{1.453388in}}%
\pgfpathlineto{\pgfqpoint{2.639884in}{1.457436in}}%
\pgfpathlineto{\pgfqpoint{2.651657in}{1.461490in}}%
\pgfpathlineto{\pgfqpoint{2.645365in}{1.473434in}}%
\pgfpathlineto{\pgfqpoint{2.639076in}{1.485326in}}%
\pgfpathlineto{\pgfqpoint{2.632791in}{1.497162in}}%
\pgfpathlineto{\pgfqpoint{2.626510in}{1.508942in}}%
\pgfpathlineto{\pgfqpoint{2.620234in}{1.520669in}}%
\pgfpathlineto{\pgfqpoint{2.608469in}{1.516588in}}%
\pgfpathlineto{\pgfqpoint{2.596698in}{1.512512in}}%
\pgfpathlineto{\pgfqpoint{2.584922in}{1.508441in}}%
\pgfpathlineto{\pgfqpoint{2.573140in}{1.504372in}}%
\pgfpathlineto{\pgfqpoint{2.561352in}{1.500305in}}%
\pgfpathlineto{\pgfqpoint{2.567620in}{1.488593in}}%
\pgfpathlineto{\pgfqpoint{2.573892in}{1.476833in}}%
\pgfpathlineto{\pgfqpoint{2.580168in}{1.465025in}}%
\pgfpathlineto{\pgfqpoint{2.586448in}{1.453168in}}%
\pgfpathclose%
\pgfusepath{stroke,fill}%
\end{pgfscope}%
\begin{pgfscope}%
\pgfpathrectangle{\pgfqpoint{0.887500in}{0.275000in}}{\pgfqpoint{4.225000in}{4.225000in}}%
\pgfusepath{clip}%
\pgfsetbuttcap%
\pgfsetroundjoin%
\definecolor{currentfill}{rgb}{0.262138,0.242286,0.520837}%
\pgfsetfillcolor{currentfill}%
\pgfsetfillopacity{0.700000}%
\pgfsetlinewidth{0.501875pt}%
\definecolor{currentstroke}{rgb}{1.000000,1.000000,1.000000}%
\pgfsetstrokecolor{currentstroke}%
\pgfsetstrokeopacity{0.500000}%
\pgfsetdash{}{0pt}%
\pgfpathmoveto{\pgfqpoint{2.683173in}{1.401229in}}%
\pgfpathlineto{\pgfqpoint{2.694949in}{1.405238in}}%
\pgfpathlineto{\pgfqpoint{2.706719in}{1.409252in}}%
\pgfpathlineto{\pgfqpoint{2.718484in}{1.413269in}}%
\pgfpathlineto{\pgfqpoint{2.730242in}{1.417288in}}%
\pgfpathlineto{\pgfqpoint{2.741995in}{1.421309in}}%
\pgfpathlineto{\pgfqpoint{2.735676in}{1.433461in}}%
\pgfpathlineto{\pgfqpoint{2.729361in}{1.445597in}}%
\pgfpathlineto{\pgfqpoint{2.723050in}{1.457706in}}%
\pgfpathlineto{\pgfqpoint{2.716741in}{1.469782in}}%
\pgfpathlineto{\pgfqpoint{2.710437in}{1.481815in}}%
\pgfpathlineto{\pgfqpoint{2.698692in}{1.477747in}}%
\pgfpathlineto{\pgfqpoint{2.686942in}{1.473679in}}%
\pgfpathlineto{\pgfqpoint{2.675186in}{1.469613in}}%
\pgfpathlineto{\pgfqpoint{2.663424in}{1.465549in}}%
\pgfpathlineto{\pgfqpoint{2.651657in}{1.461490in}}%
\pgfpathlineto{\pgfqpoint{2.657953in}{1.449501in}}%
\pgfpathlineto{\pgfqpoint{2.664253in}{1.437474in}}%
\pgfpathlineto{\pgfqpoint{2.670556in}{1.425416in}}%
\pgfpathlineto{\pgfqpoint{2.676863in}{1.413331in}}%
\pgfpathclose%
\pgfusepath{stroke,fill}%
\end{pgfscope}%
\begin{pgfscope}%
\pgfpathrectangle{\pgfqpoint{0.887500in}{0.275000in}}{\pgfqpoint{4.225000in}{4.225000in}}%
\pgfusepath{clip}%
\pgfsetbuttcap%
\pgfsetroundjoin%
\definecolor{currentfill}{rgb}{0.282623,0.140926,0.457517}%
\pgfsetfillcolor{currentfill}%
\pgfsetfillopacity{0.700000}%
\pgfsetlinewidth{0.501875pt}%
\definecolor{currentstroke}{rgb}{1.000000,1.000000,1.000000}%
\pgfsetstrokecolor{currentstroke}%
\pgfsetstrokeopacity{0.500000}%
\pgfsetdash{}{0pt}%
\pgfpathmoveto{\pgfqpoint{3.045111in}{1.223256in}}%
\pgfpathlineto{\pgfqpoint{3.056796in}{1.227353in}}%
\pgfpathlineto{\pgfqpoint{3.068476in}{1.231525in}}%
\pgfpathlineto{\pgfqpoint{3.080151in}{1.235871in}}%
\pgfpathlineto{\pgfqpoint{3.091820in}{1.240491in}}%
\pgfpathlineto{\pgfqpoint{3.103484in}{1.245481in}}%
\pgfpathlineto{\pgfqpoint{3.097072in}{1.258098in}}%
\pgfpathlineto{\pgfqpoint{3.090664in}{1.271104in}}%
\pgfpathlineto{\pgfqpoint{3.084259in}{1.284389in}}%
\pgfpathlineto{\pgfqpoint{3.077857in}{1.297840in}}%
\pgfpathlineto{\pgfqpoint{3.071458in}{1.311347in}}%
\pgfpathlineto{\pgfqpoint{3.059801in}{1.307353in}}%
\pgfpathlineto{\pgfqpoint{3.048137in}{1.303412in}}%
\pgfpathlineto{\pgfqpoint{3.036468in}{1.299511in}}%
\pgfpathlineto{\pgfqpoint{3.024793in}{1.295639in}}%
\pgfpathlineto{\pgfqpoint{3.013112in}{1.291785in}}%
\pgfpathlineto{\pgfqpoint{3.019507in}{1.277864in}}%
\pgfpathlineto{\pgfqpoint{3.025904in}{1.263919in}}%
\pgfpathlineto{\pgfqpoint{3.032303in}{1.250082in}}%
\pgfpathlineto{\pgfqpoint{3.038706in}{1.236483in}}%
\pgfpathclose%
\pgfusepath{stroke,fill}%
\end{pgfscope}%
\begin{pgfscope}%
\pgfpathrectangle{\pgfqpoint{0.887500in}{0.275000in}}{\pgfqpoint{4.225000in}{4.225000in}}%
\pgfusepath{clip}%
\pgfsetbuttcap%
\pgfsetroundjoin%
\definecolor{currentfill}{rgb}{0.269308,0.218818,0.509577}%
\pgfsetfillcolor{currentfill}%
\pgfsetfillopacity{0.700000}%
\pgfsetlinewidth{0.501875pt}%
\definecolor{currentstroke}{rgb}{1.000000,1.000000,1.000000}%
\pgfsetstrokecolor{currentstroke}%
\pgfsetstrokeopacity{0.500000}%
\pgfsetdash{}{0pt}%
\pgfpathmoveto{\pgfqpoint{2.773636in}{1.360554in}}%
\pgfpathlineto{\pgfqpoint{2.785391in}{1.364554in}}%
\pgfpathlineto{\pgfqpoint{2.797140in}{1.368561in}}%
\pgfpathlineto{\pgfqpoint{2.808883in}{1.372572in}}%
\pgfpathlineto{\pgfqpoint{2.820620in}{1.376585in}}%
\pgfpathlineto{\pgfqpoint{2.832352in}{1.380597in}}%
\pgfpathlineto{\pgfqpoint{2.826009in}{1.392711in}}%
\pgfpathlineto{\pgfqpoint{2.819670in}{1.404847in}}%
\pgfpathlineto{\pgfqpoint{2.813334in}{1.417014in}}%
\pgfpathlineto{\pgfqpoint{2.807001in}{1.429200in}}%
\pgfpathlineto{\pgfqpoint{2.800672in}{1.441395in}}%
\pgfpathlineto{\pgfqpoint{2.788948in}{1.437379in}}%
\pgfpathlineto{\pgfqpoint{2.777218in}{1.433363in}}%
\pgfpathlineto{\pgfqpoint{2.765483in}{1.429346in}}%
\pgfpathlineto{\pgfqpoint{2.753742in}{1.425328in}}%
\pgfpathlineto{\pgfqpoint{2.741995in}{1.421309in}}%
\pgfpathlineto{\pgfqpoint{2.748317in}{1.409147in}}%
\pgfpathlineto{\pgfqpoint{2.754642in}{1.396983in}}%
\pgfpathlineto{\pgfqpoint{2.760970in}{1.384827in}}%
\pgfpathlineto{\pgfqpoint{2.767302in}{1.372685in}}%
\pgfpathclose%
\pgfusepath{stroke,fill}%
\end{pgfscope}%
\begin{pgfscope}%
\pgfpathrectangle{\pgfqpoint{0.887500in}{0.275000in}}{\pgfqpoint{4.225000in}{4.225000in}}%
\pgfusepath{clip}%
\pgfsetbuttcap%
\pgfsetroundjoin%
\definecolor{currentfill}{rgb}{0.279574,0.170599,0.479997}%
\pgfsetfillcolor{currentfill}%
\pgfsetfillopacity{0.700000}%
\pgfsetlinewidth{0.501875pt}%
\definecolor{currentstroke}{rgb}{1.000000,1.000000,1.000000}%
\pgfsetstrokecolor{currentstroke}%
\pgfsetstrokeopacity{0.500000}%
\pgfsetdash{}{0pt}%
\pgfpathmoveto{\pgfqpoint{2.954620in}{1.272658in}}%
\pgfpathlineto{\pgfqpoint{2.966330in}{1.276457in}}%
\pgfpathlineto{\pgfqpoint{2.978035in}{1.280272in}}%
\pgfpathlineto{\pgfqpoint{2.989733in}{1.284100in}}%
\pgfpathlineto{\pgfqpoint{3.001426in}{1.287939in}}%
\pgfpathlineto{\pgfqpoint{3.013112in}{1.291785in}}%
\pgfpathlineto{\pgfqpoint{3.006721in}{1.305549in}}%
\pgfpathlineto{\pgfqpoint{3.000333in}{1.319056in}}%
\pgfpathlineto{\pgfqpoint{2.993948in}{1.332310in}}%
\pgfpathlineto{\pgfqpoint{2.987567in}{1.345341in}}%
\pgfpathlineto{\pgfqpoint{2.981189in}{1.358176in}}%
\pgfpathlineto{\pgfqpoint{2.969508in}{1.354328in}}%
\pgfpathlineto{\pgfqpoint{2.957821in}{1.350489in}}%
\pgfpathlineto{\pgfqpoint{2.946128in}{1.346651in}}%
\pgfpathlineto{\pgfqpoint{2.934429in}{1.342806in}}%
\pgfpathlineto{\pgfqpoint{2.922724in}{1.338949in}}%
\pgfpathlineto{\pgfqpoint{2.929096in}{1.326203in}}%
\pgfpathlineto{\pgfqpoint{2.935471in}{1.313234in}}%
\pgfpathlineto{\pgfqpoint{2.941850in}{1.300006in}}%
\pgfpathlineto{\pgfqpoint{2.948233in}{1.286485in}}%
\pgfpathclose%
\pgfusepath{stroke,fill}%
\end{pgfscope}%
\begin{pgfscope}%
\pgfpathrectangle{\pgfqpoint{0.887500in}{0.275000in}}{\pgfqpoint{4.225000in}{4.225000in}}%
\pgfusepath{clip}%
\pgfsetbuttcap%
\pgfsetroundjoin%
\definecolor{currentfill}{rgb}{0.275191,0.194905,0.496005}%
\pgfsetfillcolor{currentfill}%
\pgfsetfillopacity{0.700000}%
\pgfsetlinewidth{0.501875pt}%
\definecolor{currentstroke}{rgb}{1.000000,1.000000,1.000000}%
\pgfsetstrokecolor{currentstroke}%
\pgfsetstrokeopacity{0.500000}%
\pgfsetdash{}{0pt}%
\pgfpathmoveto{\pgfqpoint{2.864115in}{1.319297in}}%
\pgfpathlineto{\pgfqpoint{2.875849in}{1.323275in}}%
\pgfpathlineto{\pgfqpoint{2.887576in}{1.327235in}}%
\pgfpathlineto{\pgfqpoint{2.899298in}{1.331169in}}%
\pgfpathlineto{\pgfqpoint{2.911014in}{1.335073in}}%
\pgfpathlineto{\pgfqpoint{2.922724in}{1.338949in}}%
\pgfpathlineto{\pgfqpoint{2.916357in}{1.351507in}}%
\pgfpathlineto{\pgfqpoint{2.909993in}{1.363911in}}%
\pgfpathlineto{\pgfqpoint{2.903633in}{1.376198in}}%
\pgfpathlineto{\pgfqpoint{2.897276in}{1.388401in}}%
\pgfpathlineto{\pgfqpoint{2.890923in}{1.400557in}}%
\pgfpathlineto{\pgfqpoint{2.879220in}{1.396583in}}%
\pgfpathlineto{\pgfqpoint{2.867511in}{1.392600in}}%
\pgfpathlineto{\pgfqpoint{2.855797in}{1.388607in}}%
\pgfpathlineto{\pgfqpoint{2.844077in}{1.384605in}}%
\pgfpathlineto{\pgfqpoint{2.832352in}{1.380597in}}%
\pgfpathlineto{\pgfqpoint{2.838697in}{1.368475in}}%
\pgfpathlineto{\pgfqpoint{2.845046in}{1.356315in}}%
\pgfpathlineto{\pgfqpoint{2.851399in}{1.344086in}}%
\pgfpathlineto{\pgfqpoint{2.857755in}{1.331757in}}%
\pgfpathclose%
\pgfusepath{stroke,fill}%
\end{pgfscope}%
\begin{pgfscope}%
\pgfpathrectangle{\pgfqpoint{0.887500in}{0.275000in}}{\pgfqpoint{4.225000in}{4.225000in}}%
\pgfusepath{clip}%
\pgfsetbuttcap%
\pgfsetroundjoin%
\definecolor{currentfill}{rgb}{0.216210,0.351535,0.550627}%
\pgfsetfillcolor{currentfill}%
\pgfsetfillopacity{0.700000}%
\pgfsetlinewidth{0.501875pt}%
\definecolor{currentstroke}{rgb}{1.000000,1.000000,1.000000}%
\pgfsetstrokecolor{currentstroke}%
\pgfsetstrokeopacity{0.500000}%
\pgfsetdash{}{0pt}%
\pgfpathmoveto{\pgfqpoint{2.081607in}{1.604099in}}%
\pgfpathlineto{\pgfqpoint{2.093531in}{1.608190in}}%
\pgfpathlineto{\pgfqpoint{2.105449in}{1.612275in}}%
\pgfpathlineto{\pgfqpoint{2.117361in}{1.616354in}}%
\pgfpathlineto{\pgfqpoint{2.129268in}{1.620429in}}%
\pgfpathlineto{\pgfqpoint{2.141169in}{1.624499in}}%
\pgfpathlineto{\pgfqpoint{2.135033in}{1.635433in}}%
\pgfpathlineto{\pgfqpoint{2.128901in}{1.646330in}}%
\pgfpathlineto{\pgfqpoint{2.122773in}{1.657193in}}%
\pgfpathlineto{\pgfqpoint{2.116650in}{1.668019in}}%
\pgfpathlineto{\pgfqpoint{2.110531in}{1.678808in}}%
\pgfpathlineto{\pgfqpoint{2.098640in}{1.674758in}}%
\pgfpathlineto{\pgfqpoint{2.086744in}{1.670703in}}%
\pgfpathlineto{\pgfqpoint{2.074842in}{1.666642in}}%
\pgfpathlineto{\pgfqpoint{2.062934in}{1.662574in}}%
\pgfpathlineto{\pgfqpoint{2.051020in}{1.658500in}}%
\pgfpathlineto{\pgfqpoint{2.057129in}{1.647686in}}%
\pgfpathlineto{\pgfqpoint{2.063242in}{1.636838in}}%
\pgfpathlineto{\pgfqpoint{2.069360in}{1.625957in}}%
\pgfpathlineto{\pgfqpoint{2.075482in}{1.615044in}}%
\pgfpathclose%
\pgfusepath{stroke,fill}%
\end{pgfscope}%
\begin{pgfscope}%
\pgfpathrectangle{\pgfqpoint{0.887500in}{0.275000in}}{\pgfqpoint{4.225000in}{4.225000in}}%
\pgfusepath{clip}%
\pgfsetbuttcap%
\pgfsetroundjoin%
\definecolor{currentfill}{rgb}{0.278791,0.062145,0.386592}%
\pgfsetfillcolor{currentfill}%
\pgfsetfillopacity{0.700000}%
\pgfsetlinewidth{0.501875pt}%
\definecolor{currentstroke}{rgb}{1.000000,1.000000,1.000000}%
\pgfsetstrokecolor{currentstroke}%
\pgfsetstrokeopacity{0.500000}%
\pgfsetdash{}{0pt}%
\pgfpathmoveto{\pgfqpoint{3.316752in}{1.124403in}}%
\pgfpathlineto{\pgfqpoint{3.328302in}{1.117454in}}%
\pgfpathlineto{\pgfqpoint{3.339843in}{1.110950in}}%
\pgfpathlineto{\pgfqpoint{3.351376in}{1.104917in}}%
\pgfpathlineto{\pgfqpoint{3.362901in}{1.099385in}}%
\pgfpathlineto{\pgfqpoint{3.374419in}{1.094381in}}%
\pgfpathlineto{\pgfqpoint{3.367953in}{1.108254in}}%
\pgfpathlineto{\pgfqpoint{3.361490in}{1.122138in}}%
\pgfpathlineto{\pgfqpoint{3.355030in}{1.136031in}}%
\pgfpathlineto{\pgfqpoint{3.348573in}{1.149929in}}%
\pgfpathlineto{\pgfqpoint{3.342119in}{1.163827in}}%
\pgfpathlineto{\pgfqpoint{3.330609in}{1.170368in}}%
\pgfpathlineto{\pgfqpoint{3.319089in}{1.177232in}}%
\pgfpathlineto{\pgfqpoint{3.307555in}{1.183930in}}%
\pgfpathlineto{\pgfqpoint{3.296005in}{1.189972in}}%
\pgfpathlineto{\pgfqpoint{3.284437in}{1.194865in}}%
\pgfpathlineto{\pgfqpoint{3.290903in}{1.182212in}}%
\pgfpathlineto{\pgfqpoint{3.297369in}{1.168907in}}%
\pgfpathlineto{\pgfqpoint{3.303832in}{1.154882in}}%
\pgfpathlineto{\pgfqpoint{3.310294in}{1.140069in}}%
\pgfpathclose%
\pgfusepath{stroke,fill}%
\end{pgfscope}%
\begin{pgfscope}%
\pgfpathrectangle{\pgfqpoint{0.887500in}{0.275000in}}{\pgfqpoint{4.225000in}{4.225000in}}%
\pgfusepath{clip}%
\pgfsetbuttcap%
\pgfsetroundjoin%
\definecolor{currentfill}{rgb}{0.283229,0.120777,0.440584}%
\pgfsetfillcolor{currentfill}%
\pgfsetfillopacity{0.700000}%
\pgfsetlinewidth{0.501875pt}%
\definecolor{currentstroke}{rgb}{1.000000,1.000000,1.000000}%
\pgfsetstrokecolor{currentstroke}%
\pgfsetstrokeopacity{0.500000}%
\pgfsetdash{}{0pt}%
\pgfpathmoveto{\pgfqpoint{3.135611in}{1.192135in}}%
\pgfpathlineto{\pgfqpoint{3.147284in}{1.198525in}}%
\pgfpathlineto{\pgfqpoint{3.158955in}{1.205747in}}%
\pgfpathlineto{\pgfqpoint{3.170624in}{1.213748in}}%
\pgfpathlineto{\pgfqpoint{3.182290in}{1.222068in}}%
\pgfpathlineto{\pgfqpoint{3.193953in}{1.230201in}}%
\pgfpathlineto{\pgfqpoint{3.187501in}{1.238924in}}%
\pgfpathlineto{\pgfqpoint{3.181054in}{1.247986in}}%
\pgfpathlineto{\pgfqpoint{3.174613in}{1.257382in}}%
\pgfpathlineto{\pgfqpoint{3.168177in}{1.267109in}}%
\pgfpathlineto{\pgfqpoint{3.161746in}{1.277163in}}%
\pgfpathlineto{\pgfqpoint{3.150101in}{1.270364in}}%
\pgfpathlineto{\pgfqpoint{3.138452in}{1.263531in}}%
\pgfpathlineto{\pgfqpoint{3.126800in}{1.256966in}}%
\pgfpathlineto{\pgfqpoint{3.115144in}{1.250940in}}%
\pgfpathlineto{\pgfqpoint{3.103484in}{1.245481in}}%
\pgfpathlineto{\pgfqpoint{3.109900in}{1.233365in}}%
\pgfpathlineto{\pgfqpoint{3.116320in}{1.221861in}}%
\pgfpathlineto{\pgfqpoint{3.122745in}{1.211080in}}%
\pgfpathlineto{\pgfqpoint{3.129175in}{1.201135in}}%
\pgfpathclose%
\pgfusepath{stroke,fill}%
\end{pgfscope}%
\begin{pgfscope}%
\pgfpathrectangle{\pgfqpoint{0.887500in}{0.275000in}}{\pgfqpoint{4.225000in}{4.225000in}}%
\pgfusepath{clip}%
\pgfsetbuttcap%
\pgfsetroundjoin%
\definecolor{currentfill}{rgb}{0.223925,0.334994,0.548053}%
\pgfsetfillcolor{currentfill}%
\pgfsetfillopacity{0.700000}%
\pgfsetlinewidth{0.501875pt}%
\definecolor{currentstroke}{rgb}{1.000000,1.000000,1.000000}%
\pgfsetstrokecolor{currentstroke}%
\pgfsetstrokeopacity{0.500000}%
\pgfsetdash{}{0pt}%
\pgfpathmoveto{\pgfqpoint{2.171912in}{1.569318in}}%
\pgfpathlineto{\pgfqpoint{2.183817in}{1.573388in}}%
\pgfpathlineto{\pgfqpoint{2.195716in}{1.577456in}}%
\pgfpathlineto{\pgfqpoint{2.207610in}{1.581522in}}%
\pgfpathlineto{\pgfqpoint{2.219497in}{1.585588in}}%
\pgfpathlineto{\pgfqpoint{2.231379in}{1.589654in}}%
\pgfpathlineto{\pgfqpoint{2.225212in}{1.600758in}}%
\pgfpathlineto{\pgfqpoint{2.219049in}{1.611827in}}%
\pgfpathlineto{\pgfqpoint{2.212891in}{1.622857in}}%
\pgfpathlineto{\pgfqpoint{2.206736in}{1.633850in}}%
\pgfpathlineto{\pgfqpoint{2.200586in}{1.644805in}}%
\pgfpathlineto{\pgfqpoint{2.188714in}{1.640748in}}%
\pgfpathlineto{\pgfqpoint{2.176836in}{1.636690in}}%
\pgfpathlineto{\pgfqpoint{2.164953in}{1.632629in}}%
\pgfpathlineto{\pgfqpoint{2.153064in}{1.628566in}}%
\pgfpathlineto{\pgfqpoint{2.141169in}{1.624499in}}%
\pgfpathlineto{\pgfqpoint{2.147309in}{1.613531in}}%
\pgfpathlineto{\pgfqpoint{2.153453in}{1.602529in}}%
\pgfpathlineto{\pgfqpoint{2.159602in}{1.591492in}}%
\pgfpathlineto{\pgfqpoint{2.165755in}{1.580422in}}%
\pgfpathclose%
\pgfusepath{stroke,fill}%
\end{pgfscope}%
\begin{pgfscope}%
\pgfpathrectangle{\pgfqpoint{0.887500in}{0.275000in}}{\pgfqpoint{4.225000in}{4.225000in}}%
\pgfusepath{clip}%
\pgfsetbuttcap%
\pgfsetroundjoin%
\definecolor{currentfill}{rgb}{0.231674,0.318106,0.544834}%
\pgfsetfillcolor{currentfill}%
\pgfsetfillopacity{0.700000}%
\pgfsetlinewidth{0.501875pt}%
\definecolor{currentstroke}{rgb}{1.000000,1.000000,1.000000}%
\pgfsetstrokecolor{currentstroke}%
\pgfsetstrokeopacity{0.500000}%
\pgfsetdash{}{0pt}%
\pgfpathmoveto{\pgfqpoint{2.262276in}{1.533614in}}%
\pgfpathlineto{\pgfqpoint{2.274161in}{1.537676in}}%
\pgfpathlineto{\pgfqpoint{2.286041in}{1.541740in}}%
\pgfpathlineto{\pgfqpoint{2.297915in}{1.545808in}}%
\pgfpathlineto{\pgfqpoint{2.309784in}{1.549879in}}%
\pgfpathlineto{\pgfqpoint{2.321646in}{1.553953in}}%
\pgfpathlineto{\pgfqpoint{2.315449in}{1.565233in}}%
\pgfpathlineto{\pgfqpoint{2.309255in}{1.576478in}}%
\pgfpathlineto{\pgfqpoint{2.303066in}{1.587687in}}%
\pgfpathlineto{\pgfqpoint{2.296881in}{1.598859in}}%
\pgfpathlineto{\pgfqpoint{2.290701in}{1.609995in}}%
\pgfpathlineto{\pgfqpoint{2.278848in}{1.605923in}}%
\pgfpathlineto{\pgfqpoint{2.266989in}{1.601854in}}%
\pgfpathlineto{\pgfqpoint{2.255125in}{1.597786in}}%
\pgfpathlineto{\pgfqpoint{2.243255in}{1.593720in}}%
\pgfpathlineto{\pgfqpoint{2.231379in}{1.589654in}}%
\pgfpathlineto{\pgfqpoint{2.237550in}{1.578514in}}%
\pgfpathlineto{\pgfqpoint{2.243725in}{1.567339in}}%
\pgfpathlineto{\pgfqpoint{2.249905in}{1.556130in}}%
\pgfpathlineto{\pgfqpoint{2.256088in}{1.544888in}}%
\pgfpathclose%
\pgfusepath{stroke,fill}%
\end{pgfscope}%
\begin{pgfscope}%
\pgfpathrectangle{\pgfqpoint{0.887500in}{0.275000in}}{\pgfqpoint{4.225000in}{4.225000in}}%
\pgfusepath{clip}%
\pgfsetbuttcap%
\pgfsetroundjoin%
\definecolor{currentfill}{rgb}{0.239346,0.300855,0.540844}%
\pgfsetfillcolor{currentfill}%
\pgfsetfillopacity{0.700000}%
\pgfsetlinewidth{0.501875pt}%
\definecolor{currentstroke}{rgb}{1.000000,1.000000,1.000000}%
\pgfsetstrokecolor{currentstroke}%
\pgfsetstrokeopacity{0.500000}%
\pgfsetdash{}{0pt}%
\pgfpathmoveto{\pgfqpoint{2.352692in}{1.497063in}}%
\pgfpathlineto{\pgfqpoint{2.364557in}{1.501131in}}%
\pgfpathlineto{\pgfqpoint{2.376418in}{1.505202in}}%
\pgfpathlineto{\pgfqpoint{2.388272in}{1.509276in}}%
\pgfpathlineto{\pgfqpoint{2.400120in}{1.513353in}}%
\pgfpathlineto{\pgfqpoint{2.411963in}{1.517431in}}%
\pgfpathlineto{\pgfqpoint{2.405736in}{1.528885in}}%
\pgfpathlineto{\pgfqpoint{2.399514in}{1.540304in}}%
\pgfpathlineto{\pgfqpoint{2.393295in}{1.551689in}}%
\pgfpathlineto{\pgfqpoint{2.387080in}{1.563038in}}%
\pgfpathlineto{\pgfqpoint{2.380870in}{1.574351in}}%
\pgfpathlineto{\pgfqpoint{2.369036in}{1.570270in}}%
\pgfpathlineto{\pgfqpoint{2.357197in}{1.566190in}}%
\pgfpathlineto{\pgfqpoint{2.345353in}{1.562109in}}%
\pgfpathlineto{\pgfqpoint{2.333502in}{1.558030in}}%
\pgfpathlineto{\pgfqpoint{2.321646in}{1.553953in}}%
\pgfpathlineto{\pgfqpoint{2.327847in}{1.542640in}}%
\pgfpathlineto{\pgfqpoint{2.334052in}{1.531293in}}%
\pgfpathlineto{\pgfqpoint{2.340261in}{1.519915in}}%
\pgfpathlineto{\pgfqpoint{2.346474in}{1.508505in}}%
\pgfpathclose%
\pgfusepath{stroke,fill}%
\end{pgfscope}%
\begin{pgfscope}%
\pgfpathrectangle{\pgfqpoint{0.887500in}{0.275000in}}{\pgfqpoint{4.225000in}{4.225000in}}%
\pgfusepath{clip}%
\pgfsetbuttcap%
\pgfsetroundjoin%
\definecolor{currentfill}{rgb}{0.246811,0.283237,0.535941}%
\pgfsetfillcolor{currentfill}%
\pgfsetfillopacity{0.700000}%
\pgfsetlinewidth{0.501875pt}%
\definecolor{currentstroke}{rgb}{1.000000,1.000000,1.000000}%
\pgfsetstrokecolor{currentstroke}%
\pgfsetstrokeopacity{0.500000}%
\pgfsetdash{}{0pt}%
\pgfpathmoveto{\pgfqpoint{2.443154in}{1.459631in}}%
\pgfpathlineto{\pgfqpoint{2.455000in}{1.463691in}}%
\pgfpathlineto{\pgfqpoint{2.466840in}{1.467756in}}%
\pgfpathlineto{\pgfqpoint{2.478675in}{1.471823in}}%
\pgfpathlineto{\pgfqpoint{2.490503in}{1.475893in}}%
\pgfpathlineto{\pgfqpoint{2.502326in}{1.479963in}}%
\pgfpathlineto{\pgfqpoint{2.496070in}{1.491617in}}%
\pgfpathlineto{\pgfqpoint{2.489819in}{1.503229in}}%
\pgfpathlineto{\pgfqpoint{2.483571in}{1.514802in}}%
\pgfpathlineto{\pgfqpoint{2.477328in}{1.526335in}}%
\pgfpathlineto{\pgfqpoint{2.471088in}{1.537829in}}%
\pgfpathlineto{\pgfqpoint{2.459275in}{1.533751in}}%
\pgfpathlineto{\pgfqpoint{2.447455in}{1.529671in}}%
\pgfpathlineto{\pgfqpoint{2.435630in}{1.525591in}}%
\pgfpathlineto{\pgfqpoint{2.423799in}{1.521511in}}%
\pgfpathlineto{\pgfqpoint{2.411963in}{1.517431in}}%
\pgfpathlineto{\pgfqpoint{2.418193in}{1.505942in}}%
\pgfpathlineto{\pgfqpoint{2.424427in}{1.494418in}}%
\pgfpathlineto{\pgfqpoint{2.430666in}{1.482858in}}%
\pgfpathlineto{\pgfqpoint{2.436908in}{1.471262in}}%
\pgfpathclose%
\pgfusepath{stroke,fill}%
\end{pgfscope}%
\begin{pgfscope}%
\pgfpathrectangle{\pgfqpoint{0.887500in}{0.275000in}}{\pgfqpoint{4.225000in}{4.225000in}}%
\pgfusepath{clip}%
\pgfsetbuttcap%
\pgfsetroundjoin%
\definecolor{currentfill}{rgb}{0.283197,0.115680,0.436115}%
\pgfsetfillcolor{currentfill}%
\pgfsetfillopacity{0.700000}%
\pgfsetlinewidth{0.501875pt}%
\definecolor{currentstroke}{rgb}{1.000000,1.000000,1.000000}%
\pgfsetstrokecolor{currentstroke}%
\pgfsetstrokeopacity{0.500000}%
\pgfsetdash{}{0pt}%
\pgfpathmoveto{\pgfqpoint{3.226307in}{1.191776in}}%
\pgfpathlineto{\pgfqpoint{3.237964in}{1.195722in}}%
\pgfpathlineto{\pgfqpoint{3.249609in}{1.198349in}}%
\pgfpathlineto{\pgfqpoint{3.261238in}{1.199272in}}%
\pgfpathlineto{\pgfqpoint{3.272848in}{1.198118in}}%
\pgfpathlineto{\pgfqpoint{3.284437in}{1.194865in}}%
\pgfpathlineto{\pgfqpoint{3.277971in}{1.206933in}}%
\pgfpathlineto{\pgfqpoint{3.271505in}{1.218483in}}%
\pgfpathlineto{\pgfqpoint{3.265041in}{1.229583in}}%
\pgfpathlineto{\pgfqpoint{3.258578in}{1.240300in}}%
\pgfpathlineto{\pgfqpoint{3.252117in}{1.250701in}}%
\pgfpathlineto{\pgfqpoint{3.240515in}{1.250695in}}%
\pgfpathlineto{\pgfqpoint{3.228895in}{1.248387in}}%
\pgfpathlineto{\pgfqpoint{3.217259in}{1.243868in}}%
\pgfpathlineto{\pgfqpoint{3.205610in}{1.237636in}}%
\pgfpathlineto{\pgfqpoint{3.193953in}{1.230201in}}%
\pgfpathlineto{\pgfqpoint{3.200412in}{1.221818in}}%
\pgfpathlineto{\pgfqpoint{3.206876in}{1.213781in}}%
\pgfpathlineto{\pgfqpoint{3.213346in}{1.206093in}}%
\pgfpathlineto{\pgfqpoint{3.219823in}{1.198756in}}%
\pgfpathclose%
\pgfusepath{stroke,fill}%
\end{pgfscope}%
\begin{pgfscope}%
\pgfpathrectangle{\pgfqpoint{0.887500in}{0.275000in}}{\pgfqpoint{4.225000in}{4.225000in}}%
\pgfusepath{clip}%
\pgfsetbuttcap%
\pgfsetroundjoin%
\definecolor{currentfill}{rgb}{0.255645,0.260703,0.528312}%
\pgfsetfillcolor{currentfill}%
\pgfsetfillopacity{0.700000}%
\pgfsetlinewidth{0.501875pt}%
\definecolor{currentstroke}{rgb}{1.000000,1.000000,1.000000}%
\pgfsetstrokecolor{currentstroke}%
\pgfsetstrokeopacity{0.500000}%
\pgfsetdash{}{0pt}%
\pgfpathmoveto{\pgfqpoint{2.533661in}{1.421074in}}%
\pgfpathlineto{\pgfqpoint{2.545487in}{1.425113in}}%
\pgfpathlineto{\pgfqpoint{2.557307in}{1.429153in}}%
\pgfpathlineto{\pgfqpoint{2.569121in}{1.433191in}}%
\pgfpathlineto{\pgfqpoint{2.580929in}{1.437228in}}%
\pgfpathlineto{\pgfqpoint{2.592732in}{1.441265in}}%
\pgfpathlineto{\pgfqpoint{2.586448in}{1.453168in}}%
\pgfpathlineto{\pgfqpoint{2.580168in}{1.465025in}}%
\pgfpathlineto{\pgfqpoint{2.573892in}{1.476833in}}%
\pgfpathlineto{\pgfqpoint{2.567620in}{1.488593in}}%
\pgfpathlineto{\pgfqpoint{2.561352in}{1.500305in}}%
\pgfpathlineto{\pgfqpoint{2.549558in}{1.496238in}}%
\pgfpathlineto{\pgfqpoint{2.537759in}{1.492172in}}%
\pgfpathlineto{\pgfqpoint{2.525953in}{1.488104in}}%
\pgfpathlineto{\pgfqpoint{2.514142in}{1.484034in}}%
\pgfpathlineto{\pgfqpoint{2.502326in}{1.479963in}}%
\pgfpathlineto{\pgfqpoint{2.508585in}{1.468269in}}%
\pgfpathlineto{\pgfqpoint{2.514848in}{1.456532in}}%
\pgfpathlineto{\pgfqpoint{2.521115in}{1.444754in}}%
\pgfpathlineto{\pgfqpoint{2.527386in}{1.432933in}}%
\pgfpathclose%
\pgfusepath{stroke,fill}%
\end{pgfscope}%
\begin{pgfscope}%
\pgfpathrectangle{\pgfqpoint{0.887500in}{0.275000in}}{\pgfqpoint{4.225000in}{4.225000in}}%
\pgfusepath{clip}%
\pgfsetbuttcap%
\pgfsetroundjoin%
\definecolor{currentfill}{rgb}{0.262138,0.242286,0.520837}%
\pgfsetfillcolor{currentfill}%
\pgfsetfillopacity{0.700000}%
\pgfsetlinewidth{0.501875pt}%
\definecolor{currentstroke}{rgb}{1.000000,1.000000,1.000000}%
\pgfsetstrokecolor{currentstroke}%
\pgfsetstrokeopacity{0.500000}%
\pgfsetdash{}{0pt}%
\pgfpathmoveto{\pgfqpoint{2.624205in}{1.381266in}}%
\pgfpathlineto{\pgfqpoint{2.636011in}{1.385250in}}%
\pgfpathlineto{\pgfqpoint{2.647810in}{1.389237in}}%
\pgfpathlineto{\pgfqpoint{2.659604in}{1.393229in}}%
\pgfpathlineto{\pgfqpoint{2.671392in}{1.397226in}}%
\pgfpathlineto{\pgfqpoint{2.683173in}{1.401229in}}%
\pgfpathlineto{\pgfqpoint{2.676863in}{1.413331in}}%
\pgfpathlineto{\pgfqpoint{2.670556in}{1.425416in}}%
\pgfpathlineto{\pgfqpoint{2.664253in}{1.437474in}}%
\pgfpathlineto{\pgfqpoint{2.657953in}{1.449501in}}%
\pgfpathlineto{\pgfqpoint{2.651657in}{1.461490in}}%
\pgfpathlineto{\pgfqpoint{2.639884in}{1.457436in}}%
\pgfpathlineto{\pgfqpoint{2.628104in}{1.453388in}}%
\pgfpathlineto{\pgfqpoint{2.616319in}{1.449344in}}%
\pgfpathlineto{\pgfqpoint{2.604528in}{1.445304in}}%
\pgfpathlineto{\pgfqpoint{2.592732in}{1.441265in}}%
\pgfpathlineto{\pgfqpoint{2.599019in}{1.429324in}}%
\pgfpathlineto{\pgfqpoint{2.605310in}{1.417348in}}%
\pgfpathlineto{\pgfqpoint{2.611605in}{1.405342in}}%
\pgfpathlineto{\pgfqpoint{2.617903in}{1.393313in}}%
\pgfpathclose%
\pgfusepath{stroke,fill}%
\end{pgfscope}%
\begin{pgfscope}%
\pgfpathrectangle{\pgfqpoint{0.887500in}{0.275000in}}{\pgfqpoint{4.225000in}{4.225000in}}%
\pgfusepath{clip}%
\pgfsetbuttcap%
\pgfsetroundjoin%
\definecolor{currentfill}{rgb}{0.282884,0.135920,0.453427}%
\pgfsetfillcolor{currentfill}%
\pgfsetfillopacity{0.700000}%
\pgfsetlinewidth{0.501875pt}%
\definecolor{currentstroke}{rgb}{1.000000,1.000000,1.000000}%
\pgfsetstrokecolor{currentstroke}%
\pgfsetstrokeopacity{0.500000}%
\pgfsetdash{}{0pt}%
\pgfpathmoveto{\pgfqpoint{2.986595in}{1.203034in}}%
\pgfpathlineto{\pgfqpoint{2.998310in}{1.206956in}}%
\pgfpathlineto{\pgfqpoint{3.010019in}{1.210971in}}%
\pgfpathlineto{\pgfqpoint{3.021722in}{1.215048in}}%
\pgfpathlineto{\pgfqpoint{3.033419in}{1.219154in}}%
\pgfpathlineto{\pgfqpoint{3.045111in}{1.223256in}}%
\pgfpathlineto{\pgfqpoint{3.038706in}{1.236483in}}%
\pgfpathlineto{\pgfqpoint{3.032303in}{1.250082in}}%
\pgfpathlineto{\pgfqpoint{3.025904in}{1.263919in}}%
\pgfpathlineto{\pgfqpoint{3.019507in}{1.277864in}}%
\pgfpathlineto{\pgfqpoint{3.013112in}{1.291785in}}%
\pgfpathlineto{\pgfqpoint{3.001426in}{1.287939in}}%
\pgfpathlineto{\pgfqpoint{2.989733in}{1.284100in}}%
\pgfpathlineto{\pgfqpoint{2.978035in}{1.280272in}}%
\pgfpathlineto{\pgfqpoint{2.966330in}{1.276457in}}%
\pgfpathlineto{\pgfqpoint{2.954620in}{1.272658in}}%
\pgfpathlineto{\pgfqpoint{2.961010in}{1.258625in}}%
\pgfpathlineto{\pgfqpoint{2.967403in}{1.244516in}}%
\pgfpathlineto{\pgfqpoint{2.973798in}{1.230461in}}%
\pgfpathlineto{\pgfqpoint{2.980196in}{1.216590in}}%
\pgfpathclose%
\pgfusepath{stroke,fill}%
\end{pgfscope}%
\begin{pgfscope}%
\pgfpathrectangle{\pgfqpoint{0.887500in}{0.275000in}}{\pgfqpoint{4.225000in}{4.225000in}}%
\pgfusepath{clip}%
\pgfsetbuttcap%
\pgfsetroundjoin%
\definecolor{currentfill}{rgb}{0.283091,0.110553,0.431554}%
\pgfsetfillcolor{currentfill}%
\pgfsetfillopacity{0.700000}%
\pgfsetlinewidth{0.501875pt}%
\definecolor{currentstroke}{rgb}{1.000000,1.000000,1.000000}%
\pgfsetstrokecolor{currentstroke}%
\pgfsetstrokeopacity{0.500000}%
\pgfsetdash{}{0pt}%
\pgfpathmoveto{\pgfqpoint{3.077186in}{1.167304in}}%
\pgfpathlineto{\pgfqpoint{3.088881in}{1.171833in}}%
\pgfpathlineto{\pgfqpoint{3.100571in}{1.176427in}}%
\pgfpathlineto{\pgfqpoint{3.112256in}{1.181240in}}%
\pgfpathlineto{\pgfqpoint{3.123935in}{1.186424in}}%
\pgfpathlineto{\pgfqpoint{3.135611in}{1.192135in}}%
\pgfpathlineto{\pgfqpoint{3.129175in}{1.201135in}}%
\pgfpathlineto{\pgfqpoint{3.122745in}{1.211080in}}%
\pgfpathlineto{\pgfqpoint{3.116320in}{1.221861in}}%
\pgfpathlineto{\pgfqpoint{3.109900in}{1.233365in}}%
\pgfpathlineto{\pgfqpoint{3.103484in}{1.245481in}}%
\pgfpathlineto{\pgfqpoint{3.091820in}{1.240491in}}%
\pgfpathlineto{\pgfqpoint{3.080151in}{1.235871in}}%
\pgfpathlineto{\pgfqpoint{3.068476in}{1.231525in}}%
\pgfpathlineto{\pgfqpoint{3.056796in}{1.227353in}}%
\pgfpathlineto{\pgfqpoint{3.045111in}{1.223256in}}%
\pgfpathlineto{\pgfqpoint{3.051519in}{1.210533in}}%
\pgfpathlineto{\pgfqpoint{3.057930in}{1.198444in}}%
\pgfpathlineto{\pgfqpoint{3.064345in}{1.187121in}}%
\pgfpathlineto{\pgfqpoint{3.070763in}{1.176697in}}%
\pgfpathclose%
\pgfusepath{stroke,fill}%
\end{pgfscope}%
\begin{pgfscope}%
\pgfpathrectangle{\pgfqpoint{0.887500in}{0.275000in}}{\pgfqpoint{4.225000in}{4.225000in}}%
\pgfusepath{clip}%
\pgfsetbuttcap%
\pgfsetroundjoin%
\definecolor{currentfill}{rgb}{0.269308,0.218818,0.509577}%
\pgfsetfillcolor{currentfill}%
\pgfsetfillopacity{0.700000}%
\pgfsetlinewidth{0.501875pt}%
\definecolor{currentstroke}{rgb}{1.000000,1.000000,1.000000}%
\pgfsetstrokecolor{currentstroke}%
\pgfsetstrokeopacity{0.500000}%
\pgfsetdash{}{0pt}%
\pgfpathmoveto{\pgfqpoint{2.714774in}{1.340654in}}%
\pgfpathlineto{\pgfqpoint{2.726558in}{1.344620in}}%
\pgfpathlineto{\pgfqpoint{2.738337in}{1.348593in}}%
\pgfpathlineto{\pgfqpoint{2.750109in}{1.352573in}}%
\pgfpathlineto{\pgfqpoint{2.761876in}{1.356560in}}%
\pgfpathlineto{\pgfqpoint{2.773636in}{1.360554in}}%
\pgfpathlineto{\pgfqpoint{2.767302in}{1.372685in}}%
\pgfpathlineto{\pgfqpoint{2.760970in}{1.384827in}}%
\pgfpathlineto{\pgfqpoint{2.754642in}{1.396983in}}%
\pgfpathlineto{\pgfqpoint{2.748317in}{1.409147in}}%
\pgfpathlineto{\pgfqpoint{2.741995in}{1.421309in}}%
\pgfpathlineto{\pgfqpoint{2.730242in}{1.417288in}}%
\pgfpathlineto{\pgfqpoint{2.718484in}{1.413269in}}%
\pgfpathlineto{\pgfqpoint{2.706719in}{1.409252in}}%
\pgfpathlineto{\pgfqpoint{2.694949in}{1.405238in}}%
\pgfpathlineto{\pgfqpoint{2.683173in}{1.401229in}}%
\pgfpathlineto{\pgfqpoint{2.689487in}{1.389114in}}%
\pgfpathlineto{\pgfqpoint{2.695804in}{1.376993in}}%
\pgfpathlineto{\pgfqpoint{2.702124in}{1.364874in}}%
\pgfpathlineto{\pgfqpoint{2.708448in}{1.352761in}}%
\pgfpathclose%
\pgfusepath{stroke,fill}%
\end{pgfscope}%
\begin{pgfscope}%
\pgfpathrectangle{\pgfqpoint{0.887500in}{0.275000in}}{\pgfqpoint{4.225000in}{4.225000in}}%
\pgfusepath{clip}%
\pgfsetbuttcap%
\pgfsetroundjoin%
\definecolor{currentfill}{rgb}{0.275191,0.194905,0.496005}%
\pgfsetfillcolor{currentfill}%
\pgfsetfillopacity{0.700000}%
\pgfsetlinewidth{0.501875pt}%
\definecolor{currentstroke}{rgb}{1.000000,1.000000,1.000000}%
\pgfsetstrokecolor{currentstroke}%
\pgfsetstrokeopacity{0.500000}%
\pgfsetdash{}{0pt}%
\pgfpathmoveto{\pgfqpoint{2.805362in}{1.299380in}}%
\pgfpathlineto{\pgfqpoint{2.817124in}{1.303344in}}%
\pgfpathlineto{\pgfqpoint{2.828881in}{1.307323in}}%
\pgfpathlineto{\pgfqpoint{2.840632in}{1.311314in}}%
\pgfpathlineto{\pgfqpoint{2.852376in}{1.315307in}}%
\pgfpathlineto{\pgfqpoint{2.864115in}{1.319297in}}%
\pgfpathlineto{\pgfqpoint{2.857755in}{1.331757in}}%
\pgfpathlineto{\pgfqpoint{2.851399in}{1.344086in}}%
\pgfpathlineto{\pgfqpoint{2.845046in}{1.356315in}}%
\pgfpathlineto{\pgfqpoint{2.838697in}{1.368475in}}%
\pgfpathlineto{\pgfqpoint{2.832352in}{1.380597in}}%
\pgfpathlineto{\pgfqpoint{2.820620in}{1.376585in}}%
\pgfpathlineto{\pgfqpoint{2.808883in}{1.372572in}}%
\pgfpathlineto{\pgfqpoint{2.797140in}{1.368561in}}%
\pgfpathlineto{\pgfqpoint{2.785391in}{1.364554in}}%
\pgfpathlineto{\pgfqpoint{2.773636in}{1.360554in}}%
\pgfpathlineto{\pgfqpoint{2.779974in}{1.348414in}}%
\pgfpathlineto{\pgfqpoint{2.786316in}{1.336247in}}%
\pgfpathlineto{\pgfqpoint{2.792661in}{1.324032in}}%
\pgfpathlineto{\pgfqpoint{2.799009in}{1.311749in}}%
\pgfpathclose%
\pgfusepath{stroke,fill}%
\end{pgfscope}%
\begin{pgfscope}%
\pgfpathrectangle{\pgfqpoint{0.887500in}{0.275000in}}{\pgfqpoint{4.225000in}{4.225000in}}%
\pgfusepath{clip}%
\pgfsetbuttcap%
\pgfsetroundjoin%
\definecolor{currentfill}{rgb}{0.279574,0.170599,0.479997}%
\pgfsetfillcolor{currentfill}%
\pgfsetfillopacity{0.700000}%
\pgfsetlinewidth{0.501875pt}%
\definecolor{currentstroke}{rgb}{1.000000,1.000000,1.000000}%
\pgfsetstrokecolor{currentstroke}%
\pgfsetstrokeopacity{0.500000}%
\pgfsetdash{}{0pt}%
\pgfpathmoveto{\pgfqpoint{2.895978in}{1.253969in}}%
\pgfpathlineto{\pgfqpoint{2.907718in}{1.257669in}}%
\pgfpathlineto{\pgfqpoint{2.919453in}{1.261385in}}%
\pgfpathlineto{\pgfqpoint{2.931181in}{1.265120in}}%
\pgfpathlineto{\pgfqpoint{2.942904in}{1.268878in}}%
\pgfpathlineto{\pgfqpoint{2.954620in}{1.272658in}}%
\pgfpathlineto{\pgfqpoint{2.948233in}{1.286485in}}%
\pgfpathlineto{\pgfqpoint{2.941850in}{1.300006in}}%
\pgfpathlineto{\pgfqpoint{2.935471in}{1.313234in}}%
\pgfpathlineto{\pgfqpoint{2.929096in}{1.326203in}}%
\pgfpathlineto{\pgfqpoint{2.922724in}{1.338949in}}%
\pgfpathlineto{\pgfqpoint{2.911014in}{1.335073in}}%
\pgfpathlineto{\pgfqpoint{2.899298in}{1.331169in}}%
\pgfpathlineto{\pgfqpoint{2.887576in}{1.327235in}}%
\pgfpathlineto{\pgfqpoint{2.875849in}{1.323275in}}%
\pgfpathlineto{\pgfqpoint{2.864115in}{1.319297in}}%
\pgfpathlineto{\pgfqpoint{2.870480in}{1.306675in}}%
\pgfpathlineto{\pgfqpoint{2.876848in}{1.293860in}}%
\pgfpathlineto{\pgfqpoint{2.883221in}{1.280821in}}%
\pgfpathlineto{\pgfqpoint{2.889597in}{1.267528in}}%
\pgfpathclose%
\pgfusepath{stroke,fill}%
\end{pgfscope}%
\begin{pgfscope}%
\pgfpathrectangle{\pgfqpoint{0.887500in}{0.275000in}}{\pgfqpoint{4.225000in}{4.225000in}}%
\pgfusepath{clip}%
\pgfsetbuttcap%
\pgfsetroundjoin%
\definecolor{currentfill}{rgb}{0.223925,0.334994,0.548053}%
\pgfsetfillcolor{currentfill}%
\pgfsetfillopacity{0.700000}%
\pgfsetlinewidth{0.501875pt}%
\definecolor{currentstroke}{rgb}{1.000000,1.000000,1.000000}%
\pgfsetstrokecolor{currentstroke}%
\pgfsetstrokeopacity{0.500000}%
\pgfsetdash{}{0pt}%
\pgfpathmoveto{\pgfqpoint{2.112299in}{1.548930in}}%
\pgfpathlineto{\pgfqpoint{2.124233in}{1.553014in}}%
\pgfpathlineto{\pgfqpoint{2.136162in}{1.557095in}}%
\pgfpathlineto{\pgfqpoint{2.148084in}{1.561172in}}%
\pgfpathlineto{\pgfqpoint{2.160001in}{1.565247in}}%
\pgfpathlineto{\pgfqpoint{2.171912in}{1.569318in}}%
\pgfpathlineto{\pgfqpoint{2.165755in}{1.580422in}}%
\pgfpathlineto{\pgfqpoint{2.159602in}{1.591492in}}%
\pgfpathlineto{\pgfqpoint{2.153453in}{1.602529in}}%
\pgfpathlineto{\pgfqpoint{2.147309in}{1.613531in}}%
\pgfpathlineto{\pgfqpoint{2.141169in}{1.624499in}}%
\pgfpathlineto{\pgfqpoint{2.129268in}{1.620429in}}%
\pgfpathlineto{\pgfqpoint{2.117361in}{1.616354in}}%
\pgfpathlineto{\pgfqpoint{2.105449in}{1.612275in}}%
\pgfpathlineto{\pgfqpoint{2.093531in}{1.608190in}}%
\pgfpathlineto{\pgfqpoint{2.081607in}{1.604099in}}%
\pgfpathlineto{\pgfqpoint{2.087738in}{1.593124in}}%
\pgfpathlineto{\pgfqpoint{2.093872in}{1.582119in}}%
\pgfpathlineto{\pgfqpoint{2.100010in}{1.571084in}}%
\pgfpathlineto{\pgfqpoint{2.106153in}{1.560021in}}%
\pgfpathclose%
\pgfusepath{stroke,fill}%
\end{pgfscope}%
\begin{pgfscope}%
\pgfpathrectangle{\pgfqpoint{0.887500in}{0.275000in}}{\pgfqpoint{4.225000in}{4.225000in}}%
\pgfusepath{clip}%
\pgfsetbuttcap%
\pgfsetroundjoin%
\definecolor{currentfill}{rgb}{0.231674,0.318106,0.544834}%
\pgfsetfillcolor{currentfill}%
\pgfsetfillopacity{0.700000}%
\pgfsetlinewidth{0.501875pt}%
\definecolor{currentstroke}{rgb}{1.000000,1.000000,1.000000}%
\pgfsetstrokecolor{currentstroke}%
\pgfsetstrokeopacity{0.500000}%
\pgfsetdash{}{0pt}%
\pgfpathmoveto{\pgfqpoint{2.202758in}{1.513328in}}%
\pgfpathlineto{\pgfqpoint{2.214673in}{1.517383in}}%
\pgfpathlineto{\pgfqpoint{2.226583in}{1.521439in}}%
\pgfpathlineto{\pgfqpoint{2.238486in}{1.525496in}}%
\pgfpathlineto{\pgfqpoint{2.250384in}{1.529554in}}%
\pgfpathlineto{\pgfqpoint{2.262276in}{1.533614in}}%
\pgfpathlineto{\pgfqpoint{2.256088in}{1.544888in}}%
\pgfpathlineto{\pgfqpoint{2.249905in}{1.556130in}}%
\pgfpathlineto{\pgfqpoint{2.243725in}{1.567339in}}%
\pgfpathlineto{\pgfqpoint{2.237550in}{1.578514in}}%
\pgfpathlineto{\pgfqpoint{2.231379in}{1.589654in}}%
\pgfpathlineto{\pgfqpoint{2.219497in}{1.585588in}}%
\pgfpathlineto{\pgfqpoint{2.207610in}{1.581522in}}%
\pgfpathlineto{\pgfqpoint{2.195716in}{1.577456in}}%
\pgfpathlineto{\pgfqpoint{2.183817in}{1.573388in}}%
\pgfpathlineto{\pgfqpoint{2.171912in}{1.569318in}}%
\pgfpathlineto{\pgfqpoint{2.178073in}{1.558182in}}%
\pgfpathlineto{\pgfqpoint{2.184238in}{1.547015in}}%
\pgfpathlineto{\pgfqpoint{2.190407in}{1.535816in}}%
\pgfpathlineto{\pgfqpoint{2.196581in}{1.524587in}}%
\pgfpathclose%
\pgfusepath{stroke,fill}%
\end{pgfscope}%
\begin{pgfscope}%
\pgfpathrectangle{\pgfqpoint{0.887500in}{0.275000in}}{\pgfqpoint{4.225000in}{4.225000in}}%
\pgfusepath{clip}%
\pgfsetbuttcap%
\pgfsetroundjoin%
\definecolor{currentfill}{rgb}{0.239346,0.300855,0.540844}%
\pgfsetfillcolor{currentfill}%
\pgfsetfillopacity{0.700000}%
\pgfsetlinewidth{0.501875pt}%
\definecolor{currentstroke}{rgb}{1.000000,1.000000,1.000000}%
\pgfsetstrokecolor{currentstroke}%
\pgfsetstrokeopacity{0.500000}%
\pgfsetdash{}{0pt}%
\pgfpathmoveto{\pgfqpoint{2.293273in}{1.476788in}}%
\pgfpathlineto{\pgfqpoint{2.305168in}{1.480835in}}%
\pgfpathlineto{\pgfqpoint{2.317058in}{1.484885in}}%
\pgfpathlineto{\pgfqpoint{2.328942in}{1.488940in}}%
\pgfpathlineto{\pgfqpoint{2.340820in}{1.492999in}}%
\pgfpathlineto{\pgfqpoint{2.352692in}{1.497063in}}%
\pgfpathlineto{\pgfqpoint{2.346474in}{1.508505in}}%
\pgfpathlineto{\pgfqpoint{2.340261in}{1.519915in}}%
\pgfpathlineto{\pgfqpoint{2.334052in}{1.531293in}}%
\pgfpathlineto{\pgfqpoint{2.327847in}{1.542640in}}%
\pgfpathlineto{\pgfqpoint{2.321646in}{1.553953in}}%
\pgfpathlineto{\pgfqpoint{2.309784in}{1.549879in}}%
\pgfpathlineto{\pgfqpoint{2.297915in}{1.545808in}}%
\pgfpathlineto{\pgfqpoint{2.286041in}{1.541740in}}%
\pgfpathlineto{\pgfqpoint{2.274161in}{1.537676in}}%
\pgfpathlineto{\pgfqpoint{2.262276in}{1.533614in}}%
\pgfpathlineto{\pgfqpoint{2.268467in}{1.522308in}}%
\pgfpathlineto{\pgfqpoint{2.274663in}{1.510972in}}%
\pgfpathlineto{\pgfqpoint{2.280862in}{1.499606in}}%
\pgfpathlineto{\pgfqpoint{2.287065in}{1.488212in}}%
\pgfpathclose%
\pgfusepath{stroke,fill}%
\end{pgfscope}%
\begin{pgfscope}%
\pgfpathrectangle{\pgfqpoint{0.887500in}{0.275000in}}{\pgfqpoint{4.225000in}{4.225000in}}%
\pgfusepath{clip}%
\pgfsetbuttcap%
\pgfsetroundjoin%
\definecolor{currentfill}{rgb}{0.283091,0.110553,0.431554}%
\pgfsetfillcolor{currentfill}%
\pgfsetfillopacity{0.700000}%
\pgfsetlinewidth{0.501875pt}%
\definecolor{currentstroke}{rgb}{1.000000,1.000000,1.000000}%
\pgfsetstrokecolor{currentstroke}%
\pgfsetstrokeopacity{0.500000}%
\pgfsetdash{}{0pt}%
\pgfpathmoveto{\pgfqpoint{3.167912in}{1.165252in}}%
\pgfpathlineto{\pgfqpoint{3.179601in}{1.170427in}}%
\pgfpathlineto{\pgfqpoint{3.191285in}{1.175851in}}%
\pgfpathlineto{\pgfqpoint{3.202965in}{1.181457in}}%
\pgfpathlineto{\pgfqpoint{3.214639in}{1.186893in}}%
\pgfpathlineto{\pgfqpoint{3.226307in}{1.191776in}}%
\pgfpathlineto{\pgfqpoint{3.219823in}{1.198756in}}%
\pgfpathlineto{\pgfqpoint{3.213346in}{1.206093in}}%
\pgfpathlineto{\pgfqpoint{3.206876in}{1.213781in}}%
\pgfpathlineto{\pgfqpoint{3.200412in}{1.221818in}}%
\pgfpathlineto{\pgfqpoint{3.193953in}{1.230201in}}%
\pgfpathlineto{\pgfqpoint{3.182290in}{1.222068in}}%
\pgfpathlineto{\pgfqpoint{3.170624in}{1.213748in}}%
\pgfpathlineto{\pgfqpoint{3.158955in}{1.205747in}}%
\pgfpathlineto{\pgfqpoint{3.147284in}{1.198525in}}%
\pgfpathlineto{\pgfqpoint{3.135611in}{1.192135in}}%
\pgfpathlineto{\pgfqpoint{3.142054in}{1.184193in}}%
\pgfpathlineto{\pgfqpoint{3.148505in}{1.177422in}}%
\pgfpathlineto{\pgfqpoint{3.154964in}{1.171933in}}%
\pgfpathlineto{\pgfqpoint{3.161433in}{1.167838in}}%
\pgfpathclose%
\pgfusepath{stroke,fill}%
\end{pgfscope}%
\begin{pgfscope}%
\pgfpathrectangle{\pgfqpoint{0.887500in}{0.275000in}}{\pgfqpoint{4.225000in}{4.225000in}}%
\pgfusepath{clip}%
\pgfsetbuttcap%
\pgfsetroundjoin%
\definecolor{currentfill}{rgb}{0.246811,0.283237,0.535941}%
\pgfsetfillcolor{currentfill}%
\pgfsetfillopacity{0.700000}%
\pgfsetlinewidth{0.501875pt}%
\definecolor{currentstroke}{rgb}{1.000000,1.000000,1.000000}%
\pgfsetstrokecolor{currentstroke}%
\pgfsetstrokeopacity{0.500000}%
\pgfsetdash{}{0pt}%
\pgfpathmoveto{\pgfqpoint{2.383835in}{1.439391in}}%
\pgfpathlineto{\pgfqpoint{2.395711in}{1.443430in}}%
\pgfpathlineto{\pgfqpoint{2.407581in}{1.447474in}}%
\pgfpathlineto{\pgfqpoint{2.419444in}{1.451522in}}%
\pgfpathlineto{\pgfqpoint{2.431302in}{1.455574in}}%
\pgfpathlineto{\pgfqpoint{2.443154in}{1.459631in}}%
\pgfpathlineto{\pgfqpoint{2.436908in}{1.471262in}}%
\pgfpathlineto{\pgfqpoint{2.430666in}{1.482858in}}%
\pgfpathlineto{\pgfqpoint{2.424427in}{1.494418in}}%
\pgfpathlineto{\pgfqpoint{2.418193in}{1.505942in}}%
\pgfpathlineto{\pgfqpoint{2.411963in}{1.517431in}}%
\pgfpathlineto{\pgfqpoint{2.400120in}{1.513353in}}%
\pgfpathlineto{\pgfqpoint{2.388272in}{1.509276in}}%
\pgfpathlineto{\pgfqpoint{2.376418in}{1.505202in}}%
\pgfpathlineto{\pgfqpoint{2.364557in}{1.501131in}}%
\pgfpathlineto{\pgfqpoint{2.352692in}{1.497063in}}%
\pgfpathlineto{\pgfqpoint{2.358912in}{1.485590in}}%
\pgfpathlineto{\pgfqpoint{2.365137in}{1.474086in}}%
\pgfpathlineto{\pgfqpoint{2.371366in}{1.462552in}}%
\pgfpathlineto{\pgfqpoint{2.377599in}{1.450987in}}%
\pgfpathclose%
\pgfusepath{stroke,fill}%
\end{pgfscope}%
\begin{pgfscope}%
\pgfpathrectangle{\pgfqpoint{0.887500in}{0.275000in}}{\pgfqpoint{4.225000in}{4.225000in}}%
\pgfusepath{clip}%
\pgfsetbuttcap%
\pgfsetroundjoin%
\definecolor{currentfill}{rgb}{0.255645,0.260703,0.528312}%
\pgfsetfillcolor{currentfill}%
\pgfsetfillopacity{0.700000}%
\pgfsetlinewidth{0.501875pt}%
\definecolor{currentstroke}{rgb}{1.000000,1.000000,1.000000}%
\pgfsetstrokecolor{currentstroke}%
\pgfsetstrokeopacity{0.500000}%
\pgfsetdash{}{0pt}%
\pgfpathmoveto{\pgfqpoint{2.474443in}{1.400935in}}%
\pgfpathlineto{\pgfqpoint{2.486299in}{1.404950in}}%
\pgfpathlineto{\pgfqpoint{2.498148in}{1.408972in}}%
\pgfpathlineto{\pgfqpoint{2.509992in}{1.413002in}}%
\pgfpathlineto{\pgfqpoint{2.521829in}{1.417036in}}%
\pgfpathlineto{\pgfqpoint{2.533661in}{1.421074in}}%
\pgfpathlineto{\pgfqpoint{2.527386in}{1.432933in}}%
\pgfpathlineto{\pgfqpoint{2.521115in}{1.444754in}}%
\pgfpathlineto{\pgfqpoint{2.514848in}{1.456532in}}%
\pgfpathlineto{\pgfqpoint{2.508585in}{1.468269in}}%
\pgfpathlineto{\pgfqpoint{2.502326in}{1.479963in}}%
\pgfpathlineto{\pgfqpoint{2.490503in}{1.475893in}}%
\pgfpathlineto{\pgfqpoint{2.478675in}{1.471823in}}%
\pgfpathlineto{\pgfqpoint{2.466840in}{1.467756in}}%
\pgfpathlineto{\pgfqpoint{2.455000in}{1.463691in}}%
\pgfpathlineto{\pgfqpoint{2.443154in}{1.459631in}}%
\pgfpathlineto{\pgfqpoint{2.449404in}{1.447963in}}%
\pgfpathlineto{\pgfqpoint{2.455658in}{1.436260in}}%
\pgfpathlineto{\pgfqpoint{2.461916in}{1.424520in}}%
\pgfpathlineto{\pgfqpoint{2.468178in}{1.412744in}}%
\pgfpathclose%
\pgfusepath{stroke,fill}%
\end{pgfscope}%
\begin{pgfscope}%
\pgfpathrectangle{\pgfqpoint{0.887500in}{0.275000in}}{\pgfqpoint{4.225000in}{4.225000in}}%
\pgfusepath{clip}%
\pgfsetbuttcap%
\pgfsetroundjoin%
\definecolor{currentfill}{rgb}{0.263663,0.237631,0.518762}%
\pgfsetfillcolor{currentfill}%
\pgfsetfillopacity{0.700000}%
\pgfsetlinewidth{0.501875pt}%
\definecolor{currentstroke}{rgb}{1.000000,1.000000,1.000000}%
\pgfsetstrokecolor{currentstroke}%
\pgfsetstrokeopacity{0.500000}%
\pgfsetdash{}{0pt}%
\pgfpathmoveto{\pgfqpoint{2.565090in}{1.361351in}}%
\pgfpathlineto{\pgfqpoint{2.576924in}{1.365333in}}%
\pgfpathlineto{\pgfqpoint{2.588753in}{1.369317in}}%
\pgfpathlineto{\pgfqpoint{2.600577in}{1.373300in}}%
\pgfpathlineto{\pgfqpoint{2.612394in}{1.377283in}}%
\pgfpathlineto{\pgfqpoint{2.624205in}{1.381266in}}%
\pgfpathlineto{\pgfqpoint{2.617903in}{1.393313in}}%
\pgfpathlineto{\pgfqpoint{2.611605in}{1.405342in}}%
\pgfpathlineto{\pgfqpoint{2.605310in}{1.417348in}}%
\pgfpathlineto{\pgfqpoint{2.599019in}{1.429324in}}%
\pgfpathlineto{\pgfqpoint{2.592732in}{1.441265in}}%
\pgfpathlineto{\pgfqpoint{2.580929in}{1.437228in}}%
\pgfpathlineto{\pgfqpoint{2.569121in}{1.433191in}}%
\pgfpathlineto{\pgfqpoint{2.557307in}{1.429153in}}%
\pgfpathlineto{\pgfqpoint{2.545487in}{1.425113in}}%
\pgfpathlineto{\pgfqpoint{2.533661in}{1.421074in}}%
\pgfpathlineto{\pgfqpoint{2.539939in}{1.409180in}}%
\pgfpathlineto{\pgfqpoint{2.546222in}{1.397256in}}%
\pgfpathlineto{\pgfqpoint{2.552507in}{1.385307in}}%
\pgfpathlineto{\pgfqpoint{2.558797in}{1.373337in}}%
\pgfpathclose%
\pgfusepath{stroke,fill}%
\end{pgfscope}%
\begin{pgfscope}%
\pgfpathrectangle{\pgfqpoint{0.887500in}{0.275000in}}{\pgfqpoint{4.225000in}{4.225000in}}%
\pgfusepath{clip}%
\pgfsetbuttcap%
\pgfsetroundjoin%
\definecolor{currentfill}{rgb}{0.283091,0.110553,0.431554}%
\pgfsetfillcolor{currentfill}%
\pgfsetfillopacity{0.700000}%
\pgfsetlinewidth{0.501875pt}%
\definecolor{currentstroke}{rgb}{1.000000,1.000000,1.000000}%
\pgfsetstrokecolor{currentstroke}%
\pgfsetstrokeopacity{0.500000}%
\pgfsetdash{}{0pt}%
\pgfpathmoveto{\pgfqpoint{3.018629in}{1.144536in}}%
\pgfpathlineto{\pgfqpoint{3.030352in}{1.148921in}}%
\pgfpathlineto{\pgfqpoint{3.042069in}{1.153456in}}%
\pgfpathlineto{\pgfqpoint{3.053780in}{1.158075in}}%
\pgfpathlineto{\pgfqpoint{3.065486in}{1.162713in}}%
\pgfpathlineto{\pgfqpoint{3.077186in}{1.167304in}}%
\pgfpathlineto{\pgfqpoint{3.070763in}{1.176697in}}%
\pgfpathlineto{\pgfqpoint{3.064345in}{1.187121in}}%
\pgfpathlineto{\pgfqpoint{3.057930in}{1.198444in}}%
\pgfpathlineto{\pgfqpoint{3.051519in}{1.210533in}}%
\pgfpathlineto{\pgfqpoint{3.045111in}{1.223256in}}%
\pgfpathlineto{\pgfqpoint{3.033419in}{1.219154in}}%
\pgfpathlineto{\pgfqpoint{3.021722in}{1.215048in}}%
\pgfpathlineto{\pgfqpoint{3.010019in}{1.210971in}}%
\pgfpathlineto{\pgfqpoint{2.998310in}{1.206956in}}%
\pgfpathlineto{\pgfqpoint{2.986595in}{1.203034in}}%
\pgfpathlineto{\pgfqpoint{2.992998in}{1.189923in}}%
\pgfpathlineto{\pgfqpoint{2.999402in}{1.177388in}}%
\pgfpathlineto{\pgfqpoint{3.005808in}{1.165558in}}%
\pgfpathlineto{\pgfqpoint{3.012217in}{1.154564in}}%
\pgfpathclose%
\pgfusepath{stroke,fill}%
\end{pgfscope}%
\begin{pgfscope}%
\pgfpathrectangle{\pgfqpoint{0.887500in}{0.275000in}}{\pgfqpoint{4.225000in}{4.225000in}}%
\pgfusepath{clip}%
\pgfsetbuttcap%
\pgfsetroundjoin%
\definecolor{currentfill}{rgb}{0.282884,0.135920,0.453427}%
\pgfsetfillcolor{currentfill}%
\pgfsetfillopacity{0.700000}%
\pgfsetlinewidth{0.501875pt}%
\definecolor{currentstroke}{rgb}{1.000000,1.000000,1.000000}%
\pgfsetstrokecolor{currentstroke}%
\pgfsetstrokeopacity{0.500000}%
\pgfsetdash{}{0pt}%
\pgfpathmoveto{\pgfqpoint{2.927926in}{1.185642in}}%
\pgfpathlineto{\pgfqpoint{2.939673in}{1.188836in}}%
\pgfpathlineto{\pgfqpoint{2.951413in}{1.192144in}}%
\pgfpathlineto{\pgfqpoint{2.963147in}{1.195602in}}%
\pgfpathlineto{\pgfqpoint{2.974875in}{1.199239in}}%
\pgfpathlineto{\pgfqpoint{2.986595in}{1.203034in}}%
\pgfpathlineto{\pgfqpoint{2.980196in}{1.216590in}}%
\pgfpathlineto{\pgfqpoint{2.973798in}{1.230461in}}%
\pgfpathlineto{\pgfqpoint{2.967403in}{1.244516in}}%
\pgfpathlineto{\pgfqpoint{2.961010in}{1.258625in}}%
\pgfpathlineto{\pgfqpoint{2.954620in}{1.272658in}}%
\pgfpathlineto{\pgfqpoint{2.942904in}{1.268878in}}%
\pgfpathlineto{\pgfqpoint{2.931181in}{1.265120in}}%
\pgfpathlineto{\pgfqpoint{2.919453in}{1.261385in}}%
\pgfpathlineto{\pgfqpoint{2.907718in}{1.257669in}}%
\pgfpathlineto{\pgfqpoint{2.895978in}{1.253969in}}%
\pgfpathlineto{\pgfqpoint{2.902362in}{1.240228in}}%
\pgfpathlineto{\pgfqpoint{2.908749in}{1.226415in}}%
\pgfpathlineto{\pgfqpoint{2.915139in}{1.212639in}}%
\pgfpathlineto{\pgfqpoint{2.921532in}{1.199012in}}%
\pgfpathclose%
\pgfusepath{stroke,fill}%
\end{pgfscope}%
\begin{pgfscope}%
\pgfpathrectangle{\pgfqpoint{0.887500in}{0.275000in}}{\pgfqpoint{4.225000in}{4.225000in}}%
\pgfusepath{clip}%
\pgfsetbuttcap%
\pgfsetroundjoin%
\definecolor{currentfill}{rgb}{0.269308,0.218818,0.509577}%
\pgfsetfillcolor{currentfill}%
\pgfsetfillopacity{0.700000}%
\pgfsetlinewidth{0.501875pt}%
\definecolor{currentstroke}{rgb}{1.000000,1.000000,1.000000}%
\pgfsetstrokecolor{currentstroke}%
\pgfsetstrokeopacity{0.500000}%
\pgfsetdash{}{0pt}%
\pgfpathmoveto{\pgfqpoint{2.655764in}{1.320929in}}%
\pgfpathlineto{\pgfqpoint{2.667578in}{1.324862in}}%
\pgfpathlineto{\pgfqpoint{2.679386in}{1.328800in}}%
\pgfpathlineto{\pgfqpoint{2.691188in}{1.332745in}}%
\pgfpathlineto{\pgfqpoint{2.702984in}{1.336696in}}%
\pgfpathlineto{\pgfqpoint{2.714774in}{1.340654in}}%
\pgfpathlineto{\pgfqpoint{2.708448in}{1.352761in}}%
\pgfpathlineto{\pgfqpoint{2.702124in}{1.364874in}}%
\pgfpathlineto{\pgfqpoint{2.695804in}{1.376993in}}%
\pgfpathlineto{\pgfqpoint{2.689487in}{1.389114in}}%
\pgfpathlineto{\pgfqpoint{2.683173in}{1.401229in}}%
\pgfpathlineto{\pgfqpoint{2.671392in}{1.397226in}}%
\pgfpathlineto{\pgfqpoint{2.659604in}{1.393229in}}%
\pgfpathlineto{\pgfqpoint{2.647810in}{1.389237in}}%
\pgfpathlineto{\pgfqpoint{2.636011in}{1.385250in}}%
\pgfpathlineto{\pgfqpoint{2.624205in}{1.381266in}}%
\pgfpathlineto{\pgfqpoint{2.630510in}{1.369205in}}%
\pgfpathlineto{\pgfqpoint{2.636819in}{1.357136in}}%
\pgfpathlineto{\pgfqpoint{2.643131in}{1.345065in}}%
\pgfpathlineto{\pgfqpoint{2.649446in}{1.332996in}}%
\pgfpathclose%
\pgfusepath{stroke,fill}%
\end{pgfscope}%
\begin{pgfscope}%
\pgfpathrectangle{\pgfqpoint{0.887500in}{0.275000in}}{\pgfqpoint{4.225000in}{4.225000in}}%
\pgfusepath{clip}%
\pgfsetbuttcap%
\pgfsetroundjoin%
\definecolor{currentfill}{rgb}{0.282656,0.100196,0.422160}%
\pgfsetfillcolor{currentfill}%
\pgfsetfillopacity{0.700000}%
\pgfsetlinewidth{0.501875pt}%
\definecolor{currentstroke}{rgb}{1.000000,1.000000,1.000000}%
\pgfsetstrokecolor{currentstroke}%
\pgfsetstrokeopacity{0.500000}%
\pgfsetdash{}{0pt}%
\pgfpathmoveto{\pgfqpoint{3.258831in}{1.162341in}}%
\pgfpathlineto{\pgfqpoint{3.270443in}{1.155138in}}%
\pgfpathlineto{\pgfqpoint{3.282039in}{1.147413in}}%
\pgfpathlineto{\pgfqpoint{3.293622in}{1.139510in}}%
\pgfpathlineto{\pgfqpoint{3.305192in}{1.131769in}}%
\pgfpathlineto{\pgfqpoint{3.316752in}{1.124403in}}%
\pgfpathlineto{\pgfqpoint{3.310294in}{1.140069in}}%
\pgfpathlineto{\pgfqpoint{3.303832in}{1.154882in}}%
\pgfpathlineto{\pgfqpoint{3.297369in}{1.168907in}}%
\pgfpathlineto{\pgfqpoint{3.290903in}{1.182212in}}%
\pgfpathlineto{\pgfqpoint{3.284437in}{1.194865in}}%
\pgfpathlineto{\pgfqpoint{3.272848in}{1.198118in}}%
\pgfpathlineto{\pgfqpoint{3.261238in}{1.199272in}}%
\pgfpathlineto{\pgfqpoint{3.249609in}{1.198349in}}%
\pgfpathlineto{\pgfqpoint{3.237964in}{1.195722in}}%
\pgfpathlineto{\pgfqpoint{3.226307in}{1.191776in}}%
\pgfpathlineto{\pgfqpoint{3.232797in}{1.185155in}}%
\pgfpathlineto{\pgfqpoint{3.239294in}{1.178897in}}%
\pgfpathlineto{\pgfqpoint{3.245799in}{1.173007in}}%
\pgfpathlineto{\pgfqpoint{3.252311in}{1.167487in}}%
\pgfpathclose%
\pgfusepath{stroke,fill}%
\end{pgfscope}%
\begin{pgfscope}%
\pgfpathrectangle{\pgfqpoint{0.887500in}{0.275000in}}{\pgfqpoint{4.225000in}{4.225000in}}%
\pgfusepath{clip}%
\pgfsetbuttcap%
\pgfsetroundjoin%
\definecolor{currentfill}{rgb}{0.275191,0.194905,0.496005}%
\pgfsetfillcolor{currentfill}%
\pgfsetfillopacity{0.700000}%
\pgfsetlinewidth{0.501875pt}%
\definecolor{currentstroke}{rgb}{1.000000,1.000000,1.000000}%
\pgfsetstrokecolor{currentstroke}%
\pgfsetstrokeopacity{0.500000}%
\pgfsetdash{}{0pt}%
\pgfpathmoveto{\pgfqpoint{2.746460in}{1.279760in}}%
\pgfpathlineto{\pgfqpoint{2.758252in}{1.283662in}}%
\pgfpathlineto{\pgfqpoint{2.770039in}{1.287573in}}%
\pgfpathlineto{\pgfqpoint{2.781819in}{1.291496in}}%
\pgfpathlineto{\pgfqpoint{2.793593in}{1.295431in}}%
\pgfpathlineto{\pgfqpoint{2.805362in}{1.299380in}}%
\pgfpathlineto{\pgfqpoint{2.799009in}{1.311749in}}%
\pgfpathlineto{\pgfqpoint{2.792661in}{1.324032in}}%
\pgfpathlineto{\pgfqpoint{2.786316in}{1.336247in}}%
\pgfpathlineto{\pgfqpoint{2.779974in}{1.348414in}}%
\pgfpathlineto{\pgfqpoint{2.773636in}{1.360554in}}%
\pgfpathlineto{\pgfqpoint{2.761876in}{1.356560in}}%
\pgfpathlineto{\pgfqpoint{2.750109in}{1.352573in}}%
\pgfpathlineto{\pgfqpoint{2.738337in}{1.348593in}}%
\pgfpathlineto{\pgfqpoint{2.726558in}{1.344620in}}%
\pgfpathlineto{\pgfqpoint{2.714774in}{1.340654in}}%
\pgfpathlineto{\pgfqpoint{2.721104in}{1.328540in}}%
\pgfpathlineto{\pgfqpoint{2.727438in}{1.316407in}}%
\pgfpathlineto{\pgfqpoint{2.733775in}{1.304241in}}%
\pgfpathlineto{\pgfqpoint{2.740115in}{1.292029in}}%
\pgfpathclose%
\pgfusepath{stroke,fill}%
\end{pgfscope}%
\begin{pgfscope}%
\pgfpathrectangle{\pgfqpoint{0.887500in}{0.275000in}}{\pgfqpoint{4.225000in}{4.225000in}}%
\pgfusepath{clip}%
\pgfsetbuttcap%
\pgfsetroundjoin%
\definecolor{currentfill}{rgb}{0.280255,0.165693,0.476498}%
\pgfsetfillcolor{currentfill}%
\pgfsetfillopacity{0.700000}%
\pgfsetlinewidth{0.501875pt}%
\definecolor{currentstroke}{rgb}{1.000000,1.000000,1.000000}%
\pgfsetstrokecolor{currentstroke}%
\pgfsetstrokeopacity{0.500000}%
\pgfsetdash{}{0pt}%
\pgfpathmoveto{\pgfqpoint{2.837184in}{1.235551in}}%
\pgfpathlineto{\pgfqpoint{2.848955in}{1.239237in}}%
\pgfpathlineto{\pgfqpoint{2.860720in}{1.242919in}}%
\pgfpathlineto{\pgfqpoint{2.872479in}{1.246598in}}%
\pgfpathlineto{\pgfqpoint{2.884231in}{1.250280in}}%
\pgfpathlineto{\pgfqpoint{2.895978in}{1.253969in}}%
\pgfpathlineto{\pgfqpoint{2.889597in}{1.267528in}}%
\pgfpathlineto{\pgfqpoint{2.883221in}{1.280821in}}%
\pgfpathlineto{\pgfqpoint{2.876848in}{1.293860in}}%
\pgfpathlineto{\pgfqpoint{2.870480in}{1.306675in}}%
\pgfpathlineto{\pgfqpoint{2.864115in}{1.319297in}}%
\pgfpathlineto{\pgfqpoint{2.852376in}{1.315307in}}%
\pgfpathlineto{\pgfqpoint{2.840632in}{1.311314in}}%
\pgfpathlineto{\pgfqpoint{2.828881in}{1.307323in}}%
\pgfpathlineto{\pgfqpoint{2.817124in}{1.303344in}}%
\pgfpathlineto{\pgfqpoint{2.805362in}{1.299380in}}%
\pgfpathlineto{\pgfqpoint{2.811718in}{1.286904in}}%
\pgfpathlineto{\pgfqpoint{2.818079in}{1.274301in}}%
\pgfpathlineto{\pgfqpoint{2.824443in}{1.261552in}}%
\pgfpathlineto{\pgfqpoint{2.830812in}{1.248637in}}%
\pgfpathclose%
\pgfusepath{stroke,fill}%
\end{pgfscope}%
\begin{pgfscope}%
\pgfpathrectangle{\pgfqpoint{0.887500in}{0.275000in}}{\pgfqpoint{4.225000in}{4.225000in}}%
\pgfusepath{clip}%
\pgfsetbuttcap%
\pgfsetroundjoin%
\definecolor{currentfill}{rgb}{0.231674,0.318106,0.544834}%
\pgfsetfillcolor{currentfill}%
\pgfsetfillopacity{0.700000}%
\pgfsetlinewidth{0.501875pt}%
\definecolor{currentstroke}{rgb}{1.000000,1.000000,1.000000}%
\pgfsetstrokecolor{currentstroke}%
\pgfsetstrokeopacity{0.500000}%
\pgfsetdash{}{0pt}%
\pgfpathmoveto{\pgfqpoint{2.143093in}{1.493046in}}%
\pgfpathlineto{\pgfqpoint{2.155038in}{1.497104in}}%
\pgfpathlineto{\pgfqpoint{2.166977in}{1.501161in}}%
\pgfpathlineto{\pgfqpoint{2.178910in}{1.505217in}}%
\pgfpathlineto{\pgfqpoint{2.190837in}{1.509273in}}%
\pgfpathlineto{\pgfqpoint{2.202758in}{1.513328in}}%
\pgfpathlineto{\pgfqpoint{2.196581in}{1.524587in}}%
\pgfpathlineto{\pgfqpoint{2.190407in}{1.535816in}}%
\pgfpathlineto{\pgfqpoint{2.184238in}{1.547015in}}%
\pgfpathlineto{\pgfqpoint{2.178073in}{1.558182in}}%
\pgfpathlineto{\pgfqpoint{2.171912in}{1.569318in}}%
\pgfpathlineto{\pgfqpoint{2.160001in}{1.565247in}}%
\pgfpathlineto{\pgfqpoint{2.148084in}{1.561172in}}%
\pgfpathlineto{\pgfqpoint{2.136162in}{1.557095in}}%
\pgfpathlineto{\pgfqpoint{2.124233in}{1.553014in}}%
\pgfpathlineto{\pgfqpoint{2.112299in}{1.548930in}}%
\pgfpathlineto{\pgfqpoint{2.118450in}{1.537809in}}%
\pgfpathlineto{\pgfqpoint{2.124605in}{1.526661in}}%
\pgfpathlineto{\pgfqpoint{2.130764in}{1.515484in}}%
\pgfpathlineto{\pgfqpoint{2.136926in}{1.504279in}}%
\pgfpathclose%
\pgfusepath{stroke,fill}%
\end{pgfscope}%
\begin{pgfscope}%
\pgfpathrectangle{\pgfqpoint{0.887500in}{0.275000in}}{\pgfqpoint{4.225000in}{4.225000in}}%
\pgfusepath{clip}%
\pgfsetbuttcap%
\pgfsetroundjoin%
\definecolor{currentfill}{rgb}{0.282656,0.100196,0.422160}%
\pgfsetfillcolor{currentfill}%
\pgfsetfillopacity{0.700000}%
\pgfsetlinewidth{0.501875pt}%
\definecolor{currentstroke}{rgb}{1.000000,1.000000,1.000000}%
\pgfsetstrokecolor{currentstroke}%
\pgfsetstrokeopacity{0.500000}%
\pgfsetdash{}{0pt}%
\pgfpathmoveto{\pgfqpoint{3.109395in}{1.140437in}}%
\pgfpathlineto{\pgfqpoint{3.121109in}{1.145513in}}%
\pgfpathlineto{\pgfqpoint{3.132818in}{1.150456in}}%
\pgfpathlineto{\pgfqpoint{3.144521in}{1.155343in}}%
\pgfpathlineto{\pgfqpoint{3.156219in}{1.160249in}}%
\pgfpathlineto{\pgfqpoint{3.167912in}{1.165252in}}%
\pgfpathlineto{\pgfqpoint{3.161433in}{1.167838in}}%
\pgfpathlineto{\pgfqpoint{3.154964in}{1.171933in}}%
\pgfpathlineto{\pgfqpoint{3.148505in}{1.177422in}}%
\pgfpathlineto{\pgfqpoint{3.142054in}{1.184193in}}%
\pgfpathlineto{\pgfqpoint{3.135611in}{1.192135in}}%
\pgfpathlineto{\pgfqpoint{3.123935in}{1.186424in}}%
\pgfpathlineto{\pgfqpoint{3.112256in}{1.181240in}}%
\pgfpathlineto{\pgfqpoint{3.100571in}{1.176427in}}%
\pgfpathlineto{\pgfqpoint{3.088881in}{1.171833in}}%
\pgfpathlineto{\pgfqpoint{3.077186in}{1.167304in}}%
\pgfpathlineto{\pgfqpoint{3.083615in}{1.159074in}}%
\pgfpathlineto{\pgfqpoint{3.090049in}{1.152139in}}%
\pgfpathlineto{\pgfqpoint{3.096490in}{1.146632in}}%
\pgfpathlineto{\pgfqpoint{3.102938in}{1.142687in}}%
\pgfpathclose%
\pgfusepath{stroke,fill}%
\end{pgfscope}%
\begin{pgfscope}%
\pgfpathrectangle{\pgfqpoint{0.887500in}{0.275000in}}{\pgfqpoint{4.225000in}{4.225000in}}%
\pgfusepath{clip}%
\pgfsetbuttcap%
\pgfsetroundjoin%
\definecolor{currentfill}{rgb}{0.241237,0.296485,0.539709}%
\pgfsetfillcolor{currentfill}%
\pgfsetfillopacity{0.700000}%
\pgfsetlinewidth{0.501875pt}%
\definecolor{currentstroke}{rgb}{1.000000,1.000000,1.000000}%
\pgfsetstrokecolor{currentstroke}%
\pgfsetstrokeopacity{0.500000}%
\pgfsetdash{}{0pt}%
\pgfpathmoveto{\pgfqpoint{2.233705in}{1.456603in}}%
\pgfpathlineto{\pgfqpoint{2.245630in}{1.460635in}}%
\pgfpathlineto{\pgfqpoint{2.257550in}{1.464669in}}%
\pgfpathlineto{\pgfqpoint{2.269463in}{1.468706in}}%
\pgfpathlineto{\pgfqpoint{2.281371in}{1.472745in}}%
\pgfpathlineto{\pgfqpoint{2.293273in}{1.476788in}}%
\pgfpathlineto{\pgfqpoint{2.287065in}{1.488212in}}%
\pgfpathlineto{\pgfqpoint{2.280862in}{1.499606in}}%
\pgfpathlineto{\pgfqpoint{2.274663in}{1.510972in}}%
\pgfpathlineto{\pgfqpoint{2.268467in}{1.522308in}}%
\pgfpathlineto{\pgfqpoint{2.262276in}{1.533614in}}%
\pgfpathlineto{\pgfqpoint{2.250384in}{1.529554in}}%
\pgfpathlineto{\pgfqpoint{2.238486in}{1.525496in}}%
\pgfpathlineto{\pgfqpoint{2.226583in}{1.521439in}}%
\pgfpathlineto{\pgfqpoint{2.214673in}{1.517383in}}%
\pgfpathlineto{\pgfqpoint{2.202758in}{1.513328in}}%
\pgfpathlineto{\pgfqpoint{2.208940in}{1.502039in}}%
\pgfpathlineto{\pgfqpoint{2.215125in}{1.490722in}}%
\pgfpathlineto{\pgfqpoint{2.221314in}{1.479376in}}%
\pgfpathlineto{\pgfqpoint{2.227508in}{1.468003in}}%
\pgfpathclose%
\pgfusepath{stroke,fill}%
\end{pgfscope}%
\begin{pgfscope}%
\pgfpathrectangle{\pgfqpoint{0.887500in}{0.275000in}}{\pgfqpoint{4.225000in}{4.225000in}}%
\pgfusepath{clip}%
\pgfsetbuttcap%
\pgfsetroundjoin%
\definecolor{currentfill}{rgb}{0.248629,0.278775,0.534556}%
\pgfsetfillcolor{currentfill}%
\pgfsetfillopacity{0.700000}%
\pgfsetlinewidth{0.501875pt}%
\definecolor{currentstroke}{rgb}{1.000000,1.000000,1.000000}%
\pgfsetstrokecolor{currentstroke}%
\pgfsetstrokeopacity{0.500000}%
\pgfsetdash{}{0pt}%
\pgfpathmoveto{\pgfqpoint{2.324367in}{1.419262in}}%
\pgfpathlineto{\pgfqpoint{2.336273in}{1.423279in}}%
\pgfpathlineto{\pgfqpoint{2.348173in}{1.427301in}}%
\pgfpathlineto{\pgfqpoint{2.360066in}{1.431326in}}%
\pgfpathlineto{\pgfqpoint{2.371954in}{1.435356in}}%
\pgfpathlineto{\pgfqpoint{2.383835in}{1.439391in}}%
\pgfpathlineto{\pgfqpoint{2.377599in}{1.450987in}}%
\pgfpathlineto{\pgfqpoint{2.371366in}{1.462552in}}%
\pgfpathlineto{\pgfqpoint{2.365137in}{1.474086in}}%
\pgfpathlineto{\pgfqpoint{2.358912in}{1.485590in}}%
\pgfpathlineto{\pgfqpoint{2.352692in}{1.497063in}}%
\pgfpathlineto{\pgfqpoint{2.340820in}{1.492999in}}%
\pgfpathlineto{\pgfqpoint{2.328942in}{1.488940in}}%
\pgfpathlineto{\pgfqpoint{2.317058in}{1.484885in}}%
\pgfpathlineto{\pgfqpoint{2.305168in}{1.480835in}}%
\pgfpathlineto{\pgfqpoint{2.293273in}{1.476788in}}%
\pgfpathlineto{\pgfqpoint{2.299484in}{1.465337in}}%
\pgfpathlineto{\pgfqpoint{2.305699in}{1.453858in}}%
\pgfpathlineto{\pgfqpoint{2.311918in}{1.442353in}}%
\pgfpathlineto{\pgfqpoint{2.318141in}{1.430820in}}%
\pgfpathclose%
\pgfusepath{stroke,fill}%
\end{pgfscope}%
\begin{pgfscope}%
\pgfpathrectangle{\pgfqpoint{0.887500in}{0.275000in}}{\pgfqpoint{4.225000in}{4.225000in}}%
\pgfusepath{clip}%
\pgfsetbuttcap%
\pgfsetroundjoin%
\definecolor{currentfill}{rgb}{0.255645,0.260703,0.528312}%
\pgfsetfillcolor{currentfill}%
\pgfsetfillopacity{0.700000}%
\pgfsetlinewidth{0.501875pt}%
\definecolor{currentstroke}{rgb}{1.000000,1.000000,1.000000}%
\pgfsetstrokecolor{currentstroke}%
\pgfsetstrokeopacity{0.500000}%
\pgfsetdash{}{0pt}%
\pgfpathmoveto{\pgfqpoint{2.415076in}{1.380964in}}%
\pgfpathlineto{\pgfqpoint{2.426961in}{1.384947in}}%
\pgfpathlineto{\pgfqpoint{2.438841in}{1.388935in}}%
\pgfpathlineto{\pgfqpoint{2.450714in}{1.392928in}}%
\pgfpathlineto{\pgfqpoint{2.462582in}{1.396928in}}%
\pgfpathlineto{\pgfqpoint{2.474443in}{1.400935in}}%
\pgfpathlineto{\pgfqpoint{2.468178in}{1.412744in}}%
\pgfpathlineto{\pgfqpoint{2.461916in}{1.424520in}}%
\pgfpathlineto{\pgfqpoint{2.455658in}{1.436260in}}%
\pgfpathlineto{\pgfqpoint{2.449404in}{1.447963in}}%
\pgfpathlineto{\pgfqpoint{2.443154in}{1.459631in}}%
\pgfpathlineto{\pgfqpoint{2.431302in}{1.455574in}}%
\pgfpathlineto{\pgfqpoint{2.419444in}{1.451522in}}%
\pgfpathlineto{\pgfqpoint{2.407581in}{1.447474in}}%
\pgfpathlineto{\pgfqpoint{2.395711in}{1.443430in}}%
\pgfpathlineto{\pgfqpoint{2.383835in}{1.439391in}}%
\pgfpathlineto{\pgfqpoint{2.390076in}{1.427765in}}%
\pgfpathlineto{\pgfqpoint{2.396320in}{1.416109in}}%
\pgfpathlineto{\pgfqpoint{2.402568in}{1.404422in}}%
\pgfpathlineto{\pgfqpoint{2.408820in}{1.392707in}}%
\pgfpathclose%
\pgfusepath{stroke,fill}%
\end{pgfscope}%
\begin{pgfscope}%
\pgfpathrectangle{\pgfqpoint{0.887500in}{0.275000in}}{\pgfqpoint{4.225000in}{4.225000in}}%
\pgfusepath{clip}%
\pgfsetbuttcap%
\pgfsetroundjoin%
\definecolor{currentfill}{rgb}{0.263663,0.237631,0.518762}%
\pgfsetfillcolor{currentfill}%
\pgfsetfillopacity{0.700000}%
\pgfsetlinewidth{0.501875pt}%
\definecolor{currentstroke}{rgb}{1.000000,1.000000,1.000000}%
\pgfsetstrokecolor{currentstroke}%
\pgfsetstrokeopacity{0.500000}%
\pgfsetdash{}{0pt}%
\pgfpathmoveto{\pgfqpoint{2.505825in}{1.341533in}}%
\pgfpathlineto{\pgfqpoint{2.517690in}{1.345477in}}%
\pgfpathlineto{\pgfqpoint{2.529549in}{1.349433in}}%
\pgfpathlineto{\pgfqpoint{2.541402in}{1.353399in}}%
\pgfpathlineto{\pgfqpoint{2.553249in}{1.357372in}}%
\pgfpathlineto{\pgfqpoint{2.565090in}{1.361351in}}%
\pgfpathlineto{\pgfqpoint{2.558797in}{1.373337in}}%
\pgfpathlineto{\pgfqpoint{2.552507in}{1.385307in}}%
\pgfpathlineto{\pgfqpoint{2.546222in}{1.397256in}}%
\pgfpathlineto{\pgfqpoint{2.539939in}{1.409180in}}%
\pgfpathlineto{\pgfqpoint{2.533661in}{1.421074in}}%
\pgfpathlineto{\pgfqpoint{2.521829in}{1.417036in}}%
\pgfpathlineto{\pgfqpoint{2.509992in}{1.413002in}}%
\pgfpathlineto{\pgfqpoint{2.498148in}{1.408972in}}%
\pgfpathlineto{\pgfqpoint{2.486299in}{1.404950in}}%
\pgfpathlineto{\pgfqpoint{2.474443in}{1.400935in}}%
\pgfpathlineto{\pgfqpoint{2.480712in}{1.389097in}}%
\pgfpathlineto{\pgfqpoint{2.486985in}{1.377233in}}%
\pgfpathlineto{\pgfqpoint{2.493262in}{1.365349in}}%
\pgfpathlineto{\pgfqpoint{2.499542in}{1.353448in}}%
\pgfpathclose%
\pgfusepath{stroke,fill}%
\end{pgfscope}%
\begin{pgfscope}%
\pgfpathrectangle{\pgfqpoint{0.887500in}{0.275000in}}{\pgfqpoint{4.225000in}{4.225000in}}%
\pgfusepath{clip}%
\pgfsetbuttcap%
\pgfsetroundjoin%
\definecolor{currentfill}{rgb}{0.282910,0.105393,0.426902}%
\pgfsetfillcolor{currentfill}%
\pgfsetfillopacity{0.700000}%
\pgfsetlinewidth{0.501875pt}%
\definecolor{currentstroke}{rgb}{1.000000,1.000000,1.000000}%
\pgfsetstrokecolor{currentstroke}%
\pgfsetstrokeopacity{0.500000}%
\pgfsetdash{}{0pt}%
\pgfpathmoveto{\pgfqpoint{2.959922in}{1.126514in}}%
\pgfpathlineto{\pgfqpoint{2.971677in}{1.129605in}}%
\pgfpathlineto{\pgfqpoint{2.983425in}{1.132902in}}%
\pgfpathlineto{\pgfqpoint{2.995166in}{1.136473in}}%
\pgfpathlineto{\pgfqpoint{3.006901in}{1.140364in}}%
\pgfpathlineto{\pgfqpoint{3.018629in}{1.144536in}}%
\pgfpathlineto{\pgfqpoint{3.012217in}{1.154564in}}%
\pgfpathlineto{\pgfqpoint{3.005808in}{1.165558in}}%
\pgfpathlineto{\pgfqpoint{2.999402in}{1.177388in}}%
\pgfpathlineto{\pgfqpoint{2.992998in}{1.189923in}}%
\pgfpathlineto{\pgfqpoint{2.986595in}{1.203034in}}%
\pgfpathlineto{\pgfqpoint{2.974875in}{1.199239in}}%
\pgfpathlineto{\pgfqpoint{2.963147in}{1.195602in}}%
\pgfpathlineto{\pgfqpoint{2.951413in}{1.192144in}}%
\pgfpathlineto{\pgfqpoint{2.939673in}{1.188836in}}%
\pgfpathlineto{\pgfqpoint{2.927926in}{1.185642in}}%
\pgfpathlineto{\pgfqpoint{2.934322in}{1.172640in}}%
\pgfpathlineto{\pgfqpoint{2.940720in}{1.160117in}}%
\pgfpathlineto{\pgfqpoint{2.947119in}{1.148181in}}%
\pgfpathlineto{\pgfqpoint{2.953520in}{1.136943in}}%
\pgfpathclose%
\pgfusepath{stroke,fill}%
\end{pgfscope}%
\begin{pgfscope}%
\pgfpathrectangle{\pgfqpoint{0.887500in}{0.275000in}}{\pgfqpoint{4.225000in}{4.225000in}}%
\pgfusepath{clip}%
\pgfsetbuttcap%
\pgfsetroundjoin%
\definecolor{currentfill}{rgb}{0.270595,0.214069,0.507052}%
\pgfsetfillcolor{currentfill}%
\pgfsetfillopacity{0.700000}%
\pgfsetlinewidth{0.501875pt}%
\definecolor{currentstroke}{rgb}{1.000000,1.000000,1.000000}%
\pgfsetstrokecolor{currentstroke}%
\pgfsetstrokeopacity{0.500000}%
\pgfsetdash{}{0pt}%
\pgfpathmoveto{\pgfqpoint{2.596604in}{1.301329in}}%
\pgfpathlineto{\pgfqpoint{2.608448in}{1.305240in}}%
\pgfpathlineto{\pgfqpoint{2.620286in}{1.309156in}}%
\pgfpathlineto{\pgfqpoint{2.632118in}{1.313077in}}%
\pgfpathlineto{\pgfqpoint{2.643944in}{1.317001in}}%
\pgfpathlineto{\pgfqpoint{2.655764in}{1.320929in}}%
\pgfpathlineto{\pgfqpoint{2.649446in}{1.332996in}}%
\pgfpathlineto{\pgfqpoint{2.643131in}{1.345065in}}%
\pgfpathlineto{\pgfqpoint{2.636819in}{1.357136in}}%
\pgfpathlineto{\pgfqpoint{2.630510in}{1.369205in}}%
\pgfpathlineto{\pgfqpoint{2.624205in}{1.381266in}}%
\pgfpathlineto{\pgfqpoint{2.612394in}{1.377283in}}%
\pgfpathlineto{\pgfqpoint{2.600577in}{1.373300in}}%
\pgfpathlineto{\pgfqpoint{2.588753in}{1.369317in}}%
\pgfpathlineto{\pgfqpoint{2.576924in}{1.365333in}}%
\pgfpathlineto{\pgfqpoint{2.565090in}{1.361351in}}%
\pgfpathlineto{\pgfqpoint{2.571386in}{1.349353in}}%
\pgfpathlineto{\pgfqpoint{2.577686in}{1.337348in}}%
\pgfpathlineto{\pgfqpoint{2.583989in}{1.325339in}}%
\pgfpathlineto{\pgfqpoint{2.590295in}{1.313333in}}%
\pgfpathclose%
\pgfusepath{stroke,fill}%
\end{pgfscope}%
\begin{pgfscope}%
\pgfpathrectangle{\pgfqpoint{0.887500in}{0.275000in}}{\pgfqpoint{4.225000in}{4.225000in}}%
\pgfusepath{clip}%
\pgfsetbuttcap%
\pgfsetroundjoin%
\definecolor{currentfill}{rgb}{0.282884,0.135920,0.453427}%
\pgfsetfillcolor{currentfill}%
\pgfsetfillopacity{0.700000}%
\pgfsetlinewidth{0.501875pt}%
\definecolor{currentstroke}{rgb}{1.000000,1.000000,1.000000}%
\pgfsetstrokecolor{currentstroke}%
\pgfsetstrokeopacity{0.500000}%
\pgfsetdash{}{0pt}%
\pgfpathmoveto{\pgfqpoint{2.869093in}{1.170061in}}%
\pgfpathlineto{\pgfqpoint{2.880872in}{1.173243in}}%
\pgfpathlineto{\pgfqpoint{2.892645in}{1.176362in}}%
\pgfpathlineto{\pgfqpoint{2.904412in}{1.179443in}}%
\pgfpathlineto{\pgfqpoint{2.916172in}{1.182523in}}%
\pgfpathlineto{\pgfqpoint{2.927926in}{1.185642in}}%
\pgfpathlineto{\pgfqpoint{2.921532in}{1.199012in}}%
\pgfpathlineto{\pgfqpoint{2.915139in}{1.212639in}}%
\pgfpathlineto{\pgfqpoint{2.908749in}{1.226415in}}%
\pgfpathlineto{\pgfqpoint{2.902362in}{1.240228in}}%
\pgfpathlineto{\pgfqpoint{2.895978in}{1.253969in}}%
\pgfpathlineto{\pgfqpoint{2.884231in}{1.250280in}}%
\pgfpathlineto{\pgfqpoint{2.872479in}{1.246598in}}%
\pgfpathlineto{\pgfqpoint{2.860720in}{1.242919in}}%
\pgfpathlineto{\pgfqpoint{2.848955in}{1.239237in}}%
\pgfpathlineto{\pgfqpoint{2.837184in}{1.235551in}}%
\pgfpathlineto{\pgfqpoint{2.843561in}{1.222354in}}%
\pgfpathlineto{\pgfqpoint{2.849940in}{1.209127in}}%
\pgfpathlineto{\pgfqpoint{2.856322in}{1.195949in}}%
\pgfpathlineto{\pgfqpoint{2.862707in}{1.182900in}}%
\pgfpathclose%
\pgfusepath{stroke,fill}%
\end{pgfscope}%
\begin{pgfscope}%
\pgfpathrectangle{\pgfqpoint{0.887500in}{0.275000in}}{\pgfqpoint{4.225000in}{4.225000in}}%
\pgfusepath{clip}%
\pgfsetbuttcap%
\pgfsetroundjoin%
\definecolor{currentfill}{rgb}{0.275191,0.194905,0.496005}%
\pgfsetfillcolor{currentfill}%
\pgfsetfillopacity{0.700000}%
\pgfsetlinewidth{0.501875pt}%
\definecolor{currentstroke}{rgb}{1.000000,1.000000,1.000000}%
\pgfsetstrokecolor{currentstroke}%
\pgfsetstrokeopacity{0.500000}%
\pgfsetdash{}{0pt}%
\pgfpathmoveto{\pgfqpoint{2.687407in}{1.260372in}}%
\pgfpathlineto{\pgfqpoint{2.699230in}{1.264233in}}%
\pgfpathlineto{\pgfqpoint{2.711046in}{1.268102in}}%
\pgfpathlineto{\pgfqpoint{2.722857in}{1.271980in}}%
\pgfpathlineto{\pgfqpoint{2.734661in}{1.275866in}}%
\pgfpathlineto{\pgfqpoint{2.746460in}{1.279760in}}%
\pgfpathlineto{\pgfqpoint{2.740115in}{1.292029in}}%
\pgfpathlineto{\pgfqpoint{2.733775in}{1.304241in}}%
\pgfpathlineto{\pgfqpoint{2.727438in}{1.316407in}}%
\pgfpathlineto{\pgfqpoint{2.721104in}{1.328540in}}%
\pgfpathlineto{\pgfqpoint{2.714774in}{1.340654in}}%
\pgfpathlineto{\pgfqpoint{2.702984in}{1.336696in}}%
\pgfpathlineto{\pgfqpoint{2.691188in}{1.332745in}}%
\pgfpathlineto{\pgfqpoint{2.679386in}{1.328800in}}%
\pgfpathlineto{\pgfqpoint{2.667578in}{1.324862in}}%
\pgfpathlineto{\pgfqpoint{2.655764in}{1.320929in}}%
\pgfpathlineto{\pgfqpoint{2.662086in}{1.308857in}}%
\pgfpathlineto{\pgfqpoint{2.668411in}{1.296773in}}%
\pgfpathlineto{\pgfqpoint{2.674739in}{1.284669in}}%
\pgfpathlineto{\pgfqpoint{2.681071in}{1.272538in}}%
\pgfpathclose%
\pgfusepath{stroke,fill}%
\end{pgfscope}%
\begin{pgfscope}%
\pgfpathrectangle{\pgfqpoint{0.887500in}{0.275000in}}{\pgfqpoint{4.225000in}{4.225000in}}%
\pgfusepath{clip}%
\pgfsetbuttcap%
\pgfsetroundjoin%
\definecolor{currentfill}{rgb}{0.280255,0.165693,0.476498}%
\pgfsetfillcolor{currentfill}%
\pgfsetfillopacity{0.700000}%
\pgfsetlinewidth{0.501875pt}%
\definecolor{currentstroke}{rgb}{1.000000,1.000000,1.000000}%
\pgfsetstrokecolor{currentstroke}%
\pgfsetstrokeopacity{0.500000}%
\pgfsetdash{}{0pt}%
\pgfpathmoveto{\pgfqpoint{2.778240in}{1.217104in}}%
\pgfpathlineto{\pgfqpoint{2.790041in}{1.220789in}}%
\pgfpathlineto{\pgfqpoint{2.801836in}{1.224478in}}%
\pgfpathlineto{\pgfqpoint{2.813625in}{1.228170in}}%
\pgfpathlineto{\pgfqpoint{2.825408in}{1.231861in}}%
\pgfpathlineto{\pgfqpoint{2.837184in}{1.235551in}}%
\pgfpathlineto{\pgfqpoint{2.830812in}{1.248637in}}%
\pgfpathlineto{\pgfqpoint{2.824443in}{1.261552in}}%
\pgfpathlineto{\pgfqpoint{2.818079in}{1.274301in}}%
\pgfpathlineto{\pgfqpoint{2.811718in}{1.286904in}}%
\pgfpathlineto{\pgfqpoint{2.805362in}{1.299380in}}%
\pgfpathlineto{\pgfqpoint{2.793593in}{1.295431in}}%
\pgfpathlineto{\pgfqpoint{2.781819in}{1.291496in}}%
\pgfpathlineto{\pgfqpoint{2.770039in}{1.287573in}}%
\pgfpathlineto{\pgfqpoint{2.758252in}{1.283662in}}%
\pgfpathlineto{\pgfqpoint{2.746460in}{1.279760in}}%
\pgfpathlineto{\pgfqpoint{2.752808in}{1.267420in}}%
\pgfpathlineto{\pgfqpoint{2.759160in}{1.254995in}}%
\pgfpathlineto{\pgfqpoint{2.765516in}{1.242475in}}%
\pgfpathlineto{\pgfqpoint{2.771876in}{1.229845in}}%
\pgfpathclose%
\pgfusepath{stroke,fill}%
\end{pgfscope}%
\begin{pgfscope}%
\pgfpathrectangle{\pgfqpoint{0.887500in}{0.275000in}}{\pgfqpoint{4.225000in}{4.225000in}}%
\pgfusepath{clip}%
\pgfsetbuttcap%
\pgfsetroundjoin%
\definecolor{currentfill}{rgb}{0.282327,0.094955,0.417331}%
\pgfsetfillcolor{currentfill}%
\pgfsetfillopacity{0.700000}%
\pgfsetlinewidth{0.501875pt}%
\definecolor{currentstroke}{rgb}{1.000000,1.000000,1.000000}%
\pgfsetstrokecolor{currentstroke}%
\pgfsetstrokeopacity{0.500000}%
\pgfsetdash{}{0pt}%
\pgfpathmoveto{\pgfqpoint{3.050746in}{1.113464in}}%
\pgfpathlineto{\pgfqpoint{3.062486in}{1.118822in}}%
\pgfpathlineto{\pgfqpoint{3.074221in}{1.124280in}}%
\pgfpathlineto{\pgfqpoint{3.085950in}{1.129756in}}%
\pgfpathlineto{\pgfqpoint{3.097675in}{1.135169in}}%
\pgfpathlineto{\pgfqpoint{3.109395in}{1.140437in}}%
\pgfpathlineto{\pgfqpoint{3.102938in}{1.142687in}}%
\pgfpathlineto{\pgfqpoint{3.096490in}{1.146632in}}%
\pgfpathlineto{\pgfqpoint{3.090049in}{1.152139in}}%
\pgfpathlineto{\pgfqpoint{3.083615in}{1.159074in}}%
\pgfpathlineto{\pgfqpoint{3.077186in}{1.167304in}}%
\pgfpathlineto{\pgfqpoint{3.065486in}{1.162713in}}%
\pgfpathlineto{\pgfqpoint{3.053780in}{1.158075in}}%
\pgfpathlineto{\pgfqpoint{3.042069in}{1.153456in}}%
\pgfpathlineto{\pgfqpoint{3.030352in}{1.148921in}}%
\pgfpathlineto{\pgfqpoint{3.018629in}{1.144536in}}%
\pgfpathlineto{\pgfqpoint{3.025044in}{1.135604in}}%
\pgfpathlineto{\pgfqpoint{3.031463in}{1.127900in}}%
\pgfpathlineto{\pgfqpoint{3.037885in}{1.121554in}}%
\pgfpathlineto{\pgfqpoint{3.044313in}{1.116698in}}%
\pgfpathclose%
\pgfusepath{stroke,fill}%
\end{pgfscope}%
\begin{pgfscope}%
\pgfpathrectangle{\pgfqpoint{0.887500in}{0.275000in}}{\pgfqpoint{4.225000in}{4.225000in}}%
\pgfusepath{clip}%
\pgfsetbuttcap%
\pgfsetroundjoin%
\definecolor{currentfill}{rgb}{0.241237,0.296485,0.539709}%
\pgfsetfillcolor{currentfill}%
\pgfsetfillopacity{0.700000}%
\pgfsetlinewidth{0.501875pt}%
\definecolor{currentstroke}{rgb}{1.000000,1.000000,1.000000}%
\pgfsetstrokecolor{currentstroke}%
\pgfsetstrokeopacity{0.500000}%
\pgfsetdash{}{0pt}%
\pgfpathmoveto{\pgfqpoint{2.173988in}{1.436467in}}%
\pgfpathlineto{\pgfqpoint{2.185943in}{1.440493in}}%
\pgfpathlineto{\pgfqpoint{2.197893in}{1.444518in}}%
\pgfpathlineto{\pgfqpoint{2.209836in}{1.448545in}}%
\pgfpathlineto{\pgfqpoint{2.221774in}{1.452573in}}%
\pgfpathlineto{\pgfqpoint{2.233705in}{1.456603in}}%
\pgfpathlineto{\pgfqpoint{2.227508in}{1.468003in}}%
\pgfpathlineto{\pgfqpoint{2.221314in}{1.479376in}}%
\pgfpathlineto{\pgfqpoint{2.215125in}{1.490722in}}%
\pgfpathlineto{\pgfqpoint{2.208940in}{1.502039in}}%
\pgfpathlineto{\pgfqpoint{2.202758in}{1.513328in}}%
\pgfpathlineto{\pgfqpoint{2.190837in}{1.509273in}}%
\pgfpathlineto{\pgfqpoint{2.178910in}{1.505217in}}%
\pgfpathlineto{\pgfqpoint{2.166977in}{1.501161in}}%
\pgfpathlineto{\pgfqpoint{2.155038in}{1.497104in}}%
\pgfpathlineto{\pgfqpoint{2.143093in}{1.493046in}}%
\pgfpathlineto{\pgfqpoint{2.149264in}{1.481785in}}%
\pgfpathlineto{\pgfqpoint{2.155439in}{1.470497in}}%
\pgfpathlineto{\pgfqpoint{2.161618in}{1.459180in}}%
\pgfpathlineto{\pgfqpoint{2.167801in}{1.447837in}}%
\pgfpathclose%
\pgfusepath{stroke,fill}%
\end{pgfscope}%
\begin{pgfscope}%
\pgfpathrectangle{\pgfqpoint{0.887500in}{0.275000in}}{\pgfqpoint{4.225000in}{4.225000in}}%
\pgfusepath{clip}%
\pgfsetbuttcap%
\pgfsetroundjoin%
\definecolor{currentfill}{rgb}{0.283229,0.120777,0.440584}%
\pgfsetfillcolor{currentfill}%
\pgfsetfillopacity{0.700000}%
\pgfsetlinewidth{0.501875pt}%
\definecolor{currentstroke}{rgb}{1.000000,1.000000,1.000000}%
\pgfsetstrokecolor{currentstroke}%
\pgfsetstrokeopacity{0.500000}%
\pgfsetdash{}{0pt}%
\pgfpathmoveto{\pgfqpoint{3.200508in}{1.178940in}}%
\pgfpathlineto{\pgfqpoint{3.212207in}{1.179056in}}%
\pgfpathlineto{\pgfqpoint{3.223890in}{1.177360in}}%
\pgfpathlineto{\pgfqpoint{3.235555in}{1.173798in}}%
\pgfpathlineto{\pgfqpoint{3.247202in}{1.168676in}}%
\pgfpathlineto{\pgfqpoint{3.258831in}{1.162341in}}%
\pgfpathlineto{\pgfqpoint{3.252311in}{1.167487in}}%
\pgfpathlineto{\pgfqpoint{3.245799in}{1.173007in}}%
\pgfpathlineto{\pgfqpoint{3.239294in}{1.178897in}}%
\pgfpathlineto{\pgfqpoint{3.232797in}{1.185155in}}%
\pgfpathlineto{\pgfqpoint{3.226307in}{1.191776in}}%
\pgfpathlineto{\pgfqpoint{3.214639in}{1.186893in}}%
\pgfpathlineto{\pgfqpoint{3.202965in}{1.181457in}}%
\pgfpathlineto{\pgfqpoint{3.191285in}{1.175851in}}%
\pgfpathlineto{\pgfqpoint{3.179601in}{1.170427in}}%
\pgfpathlineto{\pgfqpoint{3.167912in}{1.165252in}}%
\pgfpathlineto{\pgfqpoint{3.174403in}{1.164287in}}%
\pgfpathlineto{\pgfqpoint{3.180907in}{1.165058in}}%
\pgfpathlineto{\pgfqpoint{3.187425in}{1.167680in}}%
\pgfpathlineto{\pgfqpoint{3.193958in}{1.172268in}}%
\pgfpathclose%
\pgfusepath{stroke,fill}%
\end{pgfscope}%
\begin{pgfscope}%
\pgfpathrectangle{\pgfqpoint{0.887500in}{0.275000in}}{\pgfqpoint{4.225000in}{4.225000in}}%
\pgfusepath{clip}%
\pgfsetbuttcap%
\pgfsetroundjoin%
\definecolor{currentfill}{rgb}{0.248629,0.278775,0.534556}%
\pgfsetfillcolor{currentfill}%
\pgfsetfillopacity{0.700000}%
\pgfsetlinewidth{0.501875pt}%
\definecolor{currentstroke}{rgb}{1.000000,1.000000,1.000000}%
\pgfsetstrokecolor{currentstroke}%
\pgfsetstrokeopacity{0.500000}%
\pgfsetdash{}{0pt}%
\pgfpathmoveto{\pgfqpoint{2.264750in}{1.399226in}}%
\pgfpathlineto{\pgfqpoint{2.276685in}{1.403227in}}%
\pgfpathlineto{\pgfqpoint{2.288615in}{1.407231in}}%
\pgfpathlineto{\pgfqpoint{2.300538in}{1.411238in}}%
\pgfpathlineto{\pgfqpoint{2.312456in}{1.415248in}}%
\pgfpathlineto{\pgfqpoint{2.324367in}{1.419262in}}%
\pgfpathlineto{\pgfqpoint{2.318141in}{1.430820in}}%
\pgfpathlineto{\pgfqpoint{2.311918in}{1.442353in}}%
\pgfpathlineto{\pgfqpoint{2.305699in}{1.453858in}}%
\pgfpathlineto{\pgfqpoint{2.299484in}{1.465337in}}%
\pgfpathlineto{\pgfqpoint{2.293273in}{1.476788in}}%
\pgfpathlineto{\pgfqpoint{2.281371in}{1.472745in}}%
\pgfpathlineto{\pgfqpoint{2.269463in}{1.468706in}}%
\pgfpathlineto{\pgfqpoint{2.257550in}{1.464669in}}%
\pgfpathlineto{\pgfqpoint{2.245630in}{1.460635in}}%
\pgfpathlineto{\pgfqpoint{2.233705in}{1.456603in}}%
\pgfpathlineto{\pgfqpoint{2.239906in}{1.445177in}}%
\pgfpathlineto{\pgfqpoint{2.246111in}{1.433726in}}%
\pgfpathlineto{\pgfqpoint{2.252320in}{1.422250in}}%
\pgfpathlineto{\pgfqpoint{2.258533in}{1.410750in}}%
\pgfpathclose%
\pgfusepath{stroke,fill}%
\end{pgfscope}%
\begin{pgfscope}%
\pgfpathrectangle{\pgfqpoint{0.887500in}{0.275000in}}{\pgfqpoint{4.225000in}{4.225000in}}%
\pgfusepath{clip}%
\pgfsetbuttcap%
\pgfsetroundjoin%
\definecolor{currentfill}{rgb}{0.257322,0.256130,0.526563}%
\pgfsetfillcolor{currentfill}%
\pgfsetfillopacity{0.700000}%
\pgfsetlinewidth{0.501875pt}%
\definecolor{currentstroke}{rgb}{1.000000,1.000000,1.000000}%
\pgfsetstrokecolor{currentstroke}%
\pgfsetstrokeopacity{0.500000}%
\pgfsetdash{}{0pt}%
\pgfpathmoveto{\pgfqpoint{2.355558in}{1.361096in}}%
\pgfpathlineto{\pgfqpoint{2.367473in}{1.365064in}}%
\pgfpathlineto{\pgfqpoint{2.379383in}{1.369035in}}%
\pgfpathlineto{\pgfqpoint{2.391287in}{1.373009in}}%
\pgfpathlineto{\pgfqpoint{2.403184in}{1.376985in}}%
\pgfpathlineto{\pgfqpoint{2.415076in}{1.380964in}}%
\pgfpathlineto{\pgfqpoint{2.408820in}{1.392707in}}%
\pgfpathlineto{\pgfqpoint{2.402568in}{1.404422in}}%
\pgfpathlineto{\pgfqpoint{2.396320in}{1.416109in}}%
\pgfpathlineto{\pgfqpoint{2.390076in}{1.427765in}}%
\pgfpathlineto{\pgfqpoint{2.383835in}{1.439391in}}%
\pgfpathlineto{\pgfqpoint{2.371954in}{1.435356in}}%
\pgfpathlineto{\pgfqpoint{2.360066in}{1.431326in}}%
\pgfpathlineto{\pgfqpoint{2.348173in}{1.427301in}}%
\pgfpathlineto{\pgfqpoint{2.336273in}{1.423279in}}%
\pgfpathlineto{\pgfqpoint{2.324367in}{1.419262in}}%
\pgfpathlineto{\pgfqpoint{2.330598in}{1.407678in}}%
\pgfpathlineto{\pgfqpoint{2.336832in}{1.396068in}}%
\pgfpathlineto{\pgfqpoint{2.343070in}{1.384434in}}%
\pgfpathlineto{\pgfqpoint{2.349312in}{1.372776in}}%
\pgfpathclose%
\pgfusepath{stroke,fill}%
\end{pgfscope}%
\begin{pgfscope}%
\pgfpathrectangle{\pgfqpoint{0.887500in}{0.275000in}}{\pgfqpoint{4.225000in}{4.225000in}}%
\pgfusepath{clip}%
\pgfsetbuttcap%
\pgfsetroundjoin%
\definecolor{currentfill}{rgb}{0.263663,0.237631,0.518762}%
\pgfsetfillcolor{currentfill}%
\pgfsetfillopacity{0.700000}%
\pgfsetlinewidth{0.501875pt}%
\definecolor{currentstroke}{rgb}{1.000000,1.000000,1.000000}%
\pgfsetstrokecolor{currentstroke}%
\pgfsetstrokeopacity{0.500000}%
\pgfsetdash{}{0pt}%
\pgfpathmoveto{\pgfqpoint{2.446409in}{1.321947in}}%
\pgfpathlineto{\pgfqpoint{2.458304in}{1.325852in}}%
\pgfpathlineto{\pgfqpoint{2.470194in}{1.329762in}}%
\pgfpathlineto{\pgfqpoint{2.482077in}{1.333677in}}%
\pgfpathlineto{\pgfqpoint{2.493954in}{1.337600in}}%
\pgfpathlineto{\pgfqpoint{2.505825in}{1.341533in}}%
\pgfpathlineto{\pgfqpoint{2.499542in}{1.353448in}}%
\pgfpathlineto{\pgfqpoint{2.493262in}{1.365349in}}%
\pgfpathlineto{\pgfqpoint{2.486985in}{1.377233in}}%
\pgfpathlineto{\pgfqpoint{2.480712in}{1.389097in}}%
\pgfpathlineto{\pgfqpoint{2.474443in}{1.400935in}}%
\pgfpathlineto{\pgfqpoint{2.462582in}{1.396928in}}%
\pgfpathlineto{\pgfqpoint{2.450714in}{1.392928in}}%
\pgfpathlineto{\pgfqpoint{2.438841in}{1.388935in}}%
\pgfpathlineto{\pgfqpoint{2.426961in}{1.384947in}}%
\pgfpathlineto{\pgfqpoint{2.415076in}{1.380964in}}%
\pgfpathlineto{\pgfqpoint{2.421335in}{1.369197in}}%
\pgfpathlineto{\pgfqpoint{2.427598in}{1.357409in}}%
\pgfpathlineto{\pgfqpoint{2.433865in}{1.345603in}}%
\pgfpathlineto{\pgfqpoint{2.440135in}{1.333781in}}%
\pgfpathclose%
\pgfusepath{stroke,fill}%
\end{pgfscope}%
\begin{pgfscope}%
\pgfpathrectangle{\pgfqpoint{0.887500in}{0.275000in}}{\pgfqpoint{4.225000in}{4.225000in}}%
\pgfusepath{clip}%
\pgfsetbuttcap%
\pgfsetroundjoin%
\definecolor{currentfill}{rgb}{0.270595,0.214069,0.507052}%
\pgfsetfillcolor{currentfill}%
\pgfsetfillopacity{0.700000}%
\pgfsetlinewidth{0.501875pt}%
\definecolor{currentstroke}{rgb}{1.000000,1.000000,1.000000}%
\pgfsetstrokecolor{currentstroke}%
\pgfsetstrokeopacity{0.500000}%
\pgfsetdash{}{0pt}%
\pgfpathmoveto{\pgfqpoint{2.537293in}{1.281896in}}%
\pgfpathlineto{\pgfqpoint{2.549168in}{1.285762in}}%
\pgfpathlineto{\pgfqpoint{2.561036in}{1.289639in}}%
\pgfpathlineto{\pgfqpoint{2.572898in}{1.293527in}}%
\pgfpathlineto{\pgfqpoint{2.584754in}{1.297424in}}%
\pgfpathlineto{\pgfqpoint{2.596604in}{1.301329in}}%
\pgfpathlineto{\pgfqpoint{2.590295in}{1.313333in}}%
\pgfpathlineto{\pgfqpoint{2.583989in}{1.325339in}}%
\pgfpathlineto{\pgfqpoint{2.577686in}{1.337348in}}%
\pgfpathlineto{\pgfqpoint{2.571386in}{1.349353in}}%
\pgfpathlineto{\pgfqpoint{2.565090in}{1.361351in}}%
\pgfpathlineto{\pgfqpoint{2.553249in}{1.357372in}}%
\pgfpathlineto{\pgfqpoint{2.541402in}{1.353399in}}%
\pgfpathlineto{\pgfqpoint{2.529549in}{1.349433in}}%
\pgfpathlineto{\pgfqpoint{2.517690in}{1.345477in}}%
\pgfpathlineto{\pgfqpoint{2.505825in}{1.341533in}}%
\pgfpathlineto{\pgfqpoint{2.512112in}{1.329609in}}%
\pgfpathlineto{\pgfqpoint{2.518403in}{1.317680in}}%
\pgfpathlineto{\pgfqpoint{2.524696in}{1.305749in}}%
\pgfpathlineto{\pgfqpoint{2.530993in}{1.293820in}}%
\pgfpathclose%
\pgfusepath{stroke,fill}%
\end{pgfscope}%
\begin{pgfscope}%
\pgfpathrectangle{\pgfqpoint{0.887500in}{0.275000in}}{\pgfqpoint{4.225000in}{4.225000in}}%
\pgfusepath{clip}%
\pgfsetbuttcap%
\pgfsetroundjoin%
\definecolor{currentfill}{rgb}{0.283091,0.110553,0.431554}%
\pgfsetfillcolor{currentfill}%
\pgfsetfillopacity{0.700000}%
\pgfsetlinewidth{0.501875pt}%
\definecolor{currentstroke}{rgb}{1.000000,1.000000,1.000000}%
\pgfsetstrokecolor{currentstroke}%
\pgfsetstrokeopacity{0.500000}%
\pgfsetdash{}{0pt}%
\pgfpathmoveto{\pgfqpoint{2.901048in}{1.111806in}}%
\pgfpathlineto{\pgfqpoint{2.912836in}{1.114857in}}%
\pgfpathlineto{\pgfqpoint{2.924617in}{1.117800in}}%
\pgfpathlineto{\pgfqpoint{2.936392in}{1.120679in}}%
\pgfpathlineto{\pgfqpoint{2.948160in}{1.123561in}}%
\pgfpathlineto{\pgfqpoint{2.959922in}{1.126514in}}%
\pgfpathlineto{\pgfqpoint{2.953520in}{1.136943in}}%
\pgfpathlineto{\pgfqpoint{2.947119in}{1.148181in}}%
\pgfpathlineto{\pgfqpoint{2.940720in}{1.160117in}}%
\pgfpathlineto{\pgfqpoint{2.934322in}{1.172640in}}%
\pgfpathlineto{\pgfqpoint{2.927926in}{1.185642in}}%
\pgfpathlineto{\pgfqpoint{2.916172in}{1.182523in}}%
\pgfpathlineto{\pgfqpoint{2.904412in}{1.179443in}}%
\pgfpathlineto{\pgfqpoint{2.892645in}{1.176362in}}%
\pgfpathlineto{\pgfqpoint{2.880872in}{1.173243in}}%
\pgfpathlineto{\pgfqpoint{2.869093in}{1.170061in}}%
\pgfpathlineto{\pgfqpoint{2.875481in}{1.157512in}}%
\pgfpathlineto{\pgfqpoint{2.881871in}{1.145331in}}%
\pgfpathlineto{\pgfqpoint{2.888262in}{1.133600in}}%
\pgfpathlineto{\pgfqpoint{2.894655in}{1.122399in}}%
\pgfpathclose%
\pgfusepath{stroke,fill}%
\end{pgfscope}%
\begin{pgfscope}%
\pgfpathrectangle{\pgfqpoint{0.887500in}{0.275000in}}{\pgfqpoint{4.225000in}{4.225000in}}%
\pgfusepath{clip}%
\pgfsetbuttcap%
\pgfsetroundjoin%
\definecolor{currentfill}{rgb}{0.276194,0.190074,0.493001}%
\pgfsetfillcolor{currentfill}%
\pgfsetfillopacity{0.700000}%
\pgfsetlinewidth{0.501875pt}%
\definecolor{currentstroke}{rgb}{1.000000,1.000000,1.000000}%
\pgfsetstrokecolor{currentstroke}%
\pgfsetstrokeopacity{0.500000}%
\pgfsetdash{}{0pt}%
\pgfpathmoveto{\pgfqpoint{2.628203in}{1.241207in}}%
\pgfpathlineto{\pgfqpoint{2.640056in}{1.245021in}}%
\pgfpathlineto{\pgfqpoint{2.651903in}{1.248844in}}%
\pgfpathlineto{\pgfqpoint{2.663744in}{1.252677in}}%
\pgfpathlineto{\pgfqpoint{2.675579in}{1.256520in}}%
\pgfpathlineto{\pgfqpoint{2.687407in}{1.260372in}}%
\pgfpathlineto{\pgfqpoint{2.681071in}{1.272538in}}%
\pgfpathlineto{\pgfqpoint{2.674739in}{1.284669in}}%
\pgfpathlineto{\pgfqpoint{2.668411in}{1.296773in}}%
\pgfpathlineto{\pgfqpoint{2.662086in}{1.308857in}}%
\pgfpathlineto{\pgfqpoint{2.655764in}{1.320929in}}%
\pgfpathlineto{\pgfqpoint{2.643944in}{1.317001in}}%
\pgfpathlineto{\pgfqpoint{2.632118in}{1.313077in}}%
\pgfpathlineto{\pgfqpoint{2.620286in}{1.309156in}}%
\pgfpathlineto{\pgfqpoint{2.608448in}{1.305240in}}%
\pgfpathlineto{\pgfqpoint{2.596604in}{1.301329in}}%
\pgfpathlineto{\pgfqpoint{2.602917in}{1.289323in}}%
\pgfpathlineto{\pgfqpoint{2.609233in}{1.277312in}}%
\pgfpathlineto{\pgfqpoint{2.615553in}{1.265292in}}%
\pgfpathlineto{\pgfqpoint{2.621876in}{1.253258in}}%
\pgfpathclose%
\pgfusepath{stroke,fill}%
\end{pgfscope}%
\begin{pgfscope}%
\pgfpathrectangle{\pgfqpoint{0.887500in}{0.275000in}}{\pgfqpoint{4.225000in}{4.225000in}}%
\pgfusepath{clip}%
\pgfsetbuttcap%
\pgfsetroundjoin%
\definecolor{currentfill}{rgb}{0.282623,0.140926,0.457517}%
\pgfsetfillcolor{currentfill}%
\pgfsetfillopacity{0.700000}%
\pgfsetlinewidth{0.501875pt}%
\definecolor{currentstroke}{rgb}{1.000000,1.000000,1.000000}%
\pgfsetstrokecolor{currentstroke}%
\pgfsetstrokeopacity{0.500000}%
\pgfsetdash{}{0pt}%
\pgfpathmoveto{\pgfqpoint{2.810106in}{1.153533in}}%
\pgfpathlineto{\pgfqpoint{2.821916in}{1.156884in}}%
\pgfpathlineto{\pgfqpoint{2.833719in}{1.160224in}}%
\pgfpathlineto{\pgfqpoint{2.845517in}{1.163541in}}%
\pgfpathlineto{\pgfqpoint{2.857308in}{1.166824in}}%
\pgfpathlineto{\pgfqpoint{2.869093in}{1.170061in}}%
\pgfpathlineto{\pgfqpoint{2.862707in}{1.182900in}}%
\pgfpathlineto{\pgfqpoint{2.856322in}{1.195949in}}%
\pgfpathlineto{\pgfqpoint{2.849940in}{1.209127in}}%
\pgfpathlineto{\pgfqpoint{2.843561in}{1.222354in}}%
\pgfpathlineto{\pgfqpoint{2.837184in}{1.235551in}}%
\pgfpathlineto{\pgfqpoint{2.825408in}{1.231861in}}%
\pgfpathlineto{\pgfqpoint{2.813625in}{1.228170in}}%
\pgfpathlineto{\pgfqpoint{2.801836in}{1.224478in}}%
\pgfpathlineto{\pgfqpoint{2.790041in}{1.220789in}}%
\pgfpathlineto{\pgfqpoint{2.778240in}{1.217104in}}%
\pgfpathlineto{\pgfqpoint{2.784607in}{1.204294in}}%
\pgfpathlineto{\pgfqpoint{2.790978in}{1.191474in}}%
\pgfpathlineto{\pgfqpoint{2.797352in}{1.178702in}}%
\pgfpathlineto{\pgfqpoint{2.803728in}{1.166036in}}%
\pgfpathclose%
\pgfusepath{stroke,fill}%
\end{pgfscope}%
\begin{pgfscope}%
\pgfpathrectangle{\pgfqpoint{0.887500in}{0.275000in}}{\pgfqpoint{4.225000in}{4.225000in}}%
\pgfusepath{clip}%
\pgfsetbuttcap%
\pgfsetroundjoin%
\definecolor{currentfill}{rgb}{0.280255,0.165693,0.476498}%
\pgfsetfillcolor{currentfill}%
\pgfsetfillopacity{0.700000}%
\pgfsetlinewidth{0.501875pt}%
\definecolor{currentstroke}{rgb}{1.000000,1.000000,1.000000}%
\pgfsetstrokecolor{currentstroke}%
\pgfsetstrokeopacity{0.500000}%
\pgfsetdash{}{0pt}%
\pgfpathmoveto{\pgfqpoint{2.719142in}{1.198765in}}%
\pgfpathlineto{\pgfqpoint{2.730974in}{1.202421in}}%
\pgfpathlineto{\pgfqpoint{2.742800in}{1.206082in}}%
\pgfpathlineto{\pgfqpoint{2.754619in}{1.209750in}}%
\pgfpathlineto{\pgfqpoint{2.766433in}{1.213423in}}%
\pgfpathlineto{\pgfqpoint{2.778240in}{1.217104in}}%
\pgfpathlineto{\pgfqpoint{2.771876in}{1.229845in}}%
\pgfpathlineto{\pgfqpoint{2.765516in}{1.242475in}}%
\pgfpathlineto{\pgfqpoint{2.759160in}{1.254995in}}%
\pgfpathlineto{\pgfqpoint{2.752808in}{1.267420in}}%
\pgfpathlineto{\pgfqpoint{2.746460in}{1.279760in}}%
\pgfpathlineto{\pgfqpoint{2.734661in}{1.275866in}}%
\pgfpathlineto{\pgfqpoint{2.722857in}{1.271980in}}%
\pgfpathlineto{\pgfqpoint{2.711046in}{1.268102in}}%
\pgfpathlineto{\pgfqpoint{2.699230in}{1.264233in}}%
\pgfpathlineto{\pgfqpoint{2.687407in}{1.260372in}}%
\pgfpathlineto{\pgfqpoint{2.693747in}{1.248163in}}%
\pgfpathlineto{\pgfqpoint{2.700090in}{1.235905in}}%
\pgfpathlineto{\pgfqpoint{2.706437in}{1.223590in}}%
\pgfpathlineto{\pgfqpoint{2.712788in}{1.211210in}}%
\pgfpathclose%
\pgfusepath{stroke,fill}%
\end{pgfscope}%
\begin{pgfscope}%
\pgfpathrectangle{\pgfqpoint{0.887500in}{0.275000in}}{\pgfqpoint{4.225000in}{4.225000in}}%
\pgfusepath{clip}%
\pgfsetbuttcap%
\pgfsetroundjoin%
\definecolor{currentfill}{rgb}{0.282327,0.094955,0.417331}%
\pgfsetfillcolor{currentfill}%
\pgfsetfillopacity{0.700000}%
\pgfsetlinewidth{0.501875pt}%
\definecolor{currentstroke}{rgb}{1.000000,1.000000,1.000000}%
\pgfsetstrokecolor{currentstroke}%
\pgfsetstrokeopacity{0.500000}%
\pgfsetdash{}{0pt}%
\pgfpathmoveto{\pgfqpoint{2.991966in}{1.090348in}}%
\pgfpathlineto{\pgfqpoint{3.003734in}{1.094456in}}%
\pgfpathlineto{\pgfqpoint{3.015496in}{1.098776in}}%
\pgfpathlineto{\pgfqpoint{3.027251in}{1.103373in}}%
\pgfpathlineto{\pgfqpoint{3.039002in}{1.108287in}}%
\pgfpathlineto{\pgfqpoint{3.050746in}{1.113464in}}%
\pgfpathlineto{\pgfqpoint{3.044313in}{1.116698in}}%
\pgfpathlineto{\pgfqpoint{3.037885in}{1.121554in}}%
\pgfpathlineto{\pgfqpoint{3.031463in}{1.127900in}}%
\pgfpathlineto{\pgfqpoint{3.025044in}{1.135604in}}%
\pgfpathlineto{\pgfqpoint{3.018629in}{1.144536in}}%
\pgfpathlineto{\pgfqpoint{3.006901in}{1.140364in}}%
\pgfpathlineto{\pgfqpoint{2.995166in}{1.136473in}}%
\pgfpathlineto{\pgfqpoint{2.983425in}{1.132902in}}%
\pgfpathlineto{\pgfqpoint{2.971677in}{1.129605in}}%
\pgfpathlineto{\pgfqpoint{2.959922in}{1.126514in}}%
\pgfpathlineto{\pgfqpoint{2.966326in}{1.117002in}}%
\pgfpathlineto{\pgfqpoint{2.972732in}{1.108519in}}%
\pgfpathlineto{\pgfqpoint{2.979141in}{1.101175in}}%
\pgfpathlineto{\pgfqpoint{2.985552in}{1.095081in}}%
\pgfpathclose%
\pgfusepath{stroke,fill}%
\end{pgfscope}%
\begin{pgfscope}%
\pgfpathrectangle{\pgfqpoint{0.887500in}{0.275000in}}{\pgfqpoint{4.225000in}{4.225000in}}%
\pgfusepath{clip}%
\pgfsetbuttcap%
\pgfsetroundjoin%
\definecolor{currentfill}{rgb}{0.283187,0.125848,0.444960}%
\pgfsetfillcolor{currentfill}%
\pgfsetfillopacity{0.700000}%
\pgfsetlinewidth{0.501875pt}%
\definecolor{currentstroke}{rgb}{1.000000,1.000000,1.000000}%
\pgfsetstrokecolor{currentstroke}%
\pgfsetstrokeopacity{0.500000}%
\pgfsetdash{}{0pt}%
\pgfpathmoveto{\pgfqpoint{3.141847in}{1.159359in}}%
\pgfpathlineto{\pgfqpoint{3.153596in}{1.165025in}}%
\pgfpathlineto{\pgfqpoint{3.165338in}{1.170051in}}%
\pgfpathlineto{\pgfqpoint{3.177071in}{1.174203in}}%
\pgfpathlineto{\pgfqpoint{3.188796in}{1.177244in}}%
\pgfpathlineto{\pgfqpoint{3.200508in}{1.178940in}}%
\pgfpathlineto{\pgfqpoint{3.193958in}{1.172268in}}%
\pgfpathlineto{\pgfqpoint{3.187425in}{1.167680in}}%
\pgfpathlineto{\pgfqpoint{3.180907in}{1.165058in}}%
\pgfpathlineto{\pgfqpoint{3.174403in}{1.164287in}}%
\pgfpathlineto{\pgfqpoint{3.167912in}{1.165252in}}%
\pgfpathlineto{\pgfqpoint{3.156219in}{1.160249in}}%
\pgfpathlineto{\pgfqpoint{3.144521in}{1.155343in}}%
\pgfpathlineto{\pgfqpoint{3.132818in}{1.150456in}}%
\pgfpathlineto{\pgfqpoint{3.121109in}{1.145513in}}%
\pgfpathlineto{\pgfqpoint{3.109395in}{1.140437in}}%
\pgfpathlineto{\pgfqpoint{3.115861in}{1.140017in}}%
\pgfpathlineto{\pgfqpoint{3.122338in}{1.141562in}}%
\pgfpathlineto{\pgfqpoint{3.128827in}{1.145209in}}%
\pgfpathlineto{\pgfqpoint{3.135330in}{1.151095in}}%
\pgfpathclose%
\pgfusepath{stroke,fill}%
\end{pgfscope}%
\begin{pgfscope}%
\pgfpathrectangle{\pgfqpoint{0.887500in}{0.275000in}}{\pgfqpoint{4.225000in}{4.225000in}}%
\pgfusepath{clip}%
\pgfsetbuttcap%
\pgfsetroundjoin%
\definecolor{currentfill}{rgb}{0.248629,0.278775,0.534556}%
\pgfsetfillcolor{currentfill}%
\pgfsetfillopacity{0.700000}%
\pgfsetlinewidth{0.501875pt}%
\definecolor{currentstroke}{rgb}{1.000000,1.000000,1.000000}%
\pgfsetstrokecolor{currentstroke}%
\pgfsetstrokeopacity{0.500000}%
\pgfsetdash{}{0pt}%
\pgfpathmoveto{\pgfqpoint{2.204981in}{1.379256in}}%
\pgfpathlineto{\pgfqpoint{2.216947in}{1.383246in}}%
\pgfpathlineto{\pgfqpoint{2.228907in}{1.387238in}}%
\pgfpathlineto{\pgfqpoint{2.240860in}{1.391232in}}%
\pgfpathlineto{\pgfqpoint{2.252808in}{1.395228in}}%
\pgfpathlineto{\pgfqpoint{2.264750in}{1.399226in}}%
\pgfpathlineto{\pgfqpoint{2.258533in}{1.410750in}}%
\pgfpathlineto{\pgfqpoint{2.252320in}{1.422250in}}%
\pgfpathlineto{\pgfqpoint{2.246111in}{1.433726in}}%
\pgfpathlineto{\pgfqpoint{2.239906in}{1.445177in}}%
\pgfpathlineto{\pgfqpoint{2.233705in}{1.456603in}}%
\pgfpathlineto{\pgfqpoint{2.221774in}{1.452573in}}%
\pgfpathlineto{\pgfqpoint{2.209836in}{1.448545in}}%
\pgfpathlineto{\pgfqpoint{2.197893in}{1.444518in}}%
\pgfpathlineto{\pgfqpoint{2.185943in}{1.440493in}}%
\pgfpathlineto{\pgfqpoint{2.173988in}{1.436467in}}%
\pgfpathlineto{\pgfqpoint{2.180179in}{1.425072in}}%
\pgfpathlineto{\pgfqpoint{2.186374in}{1.413653in}}%
\pgfpathlineto{\pgfqpoint{2.192572in}{1.402210in}}%
\pgfpathlineto{\pgfqpoint{2.198775in}{1.390744in}}%
\pgfpathclose%
\pgfusepath{stroke,fill}%
\end{pgfscope}%
\begin{pgfscope}%
\pgfpathrectangle{\pgfqpoint{0.887500in}{0.275000in}}{\pgfqpoint{4.225000in}{4.225000in}}%
\pgfusepath{clip}%
\pgfsetbuttcap%
\pgfsetroundjoin%
\definecolor{currentfill}{rgb}{0.257322,0.256130,0.526563}%
\pgfsetfillcolor{currentfill}%
\pgfsetfillopacity{0.700000}%
\pgfsetlinewidth{0.501875pt}%
\definecolor{currentstroke}{rgb}{1.000000,1.000000,1.000000}%
\pgfsetstrokecolor{currentstroke}%
\pgfsetstrokeopacity{0.500000}%
\pgfsetdash{}{0pt}%
\pgfpathmoveto{\pgfqpoint{2.295890in}{1.341286in}}%
\pgfpathlineto{\pgfqpoint{2.307835in}{1.345244in}}%
\pgfpathlineto{\pgfqpoint{2.319775in}{1.349204in}}%
\pgfpathlineto{\pgfqpoint{2.331709in}{1.353166in}}%
\pgfpathlineto{\pgfqpoint{2.343636in}{1.357130in}}%
\pgfpathlineto{\pgfqpoint{2.355558in}{1.361096in}}%
\pgfpathlineto{\pgfqpoint{2.349312in}{1.372776in}}%
\pgfpathlineto{\pgfqpoint{2.343070in}{1.384434in}}%
\pgfpathlineto{\pgfqpoint{2.336832in}{1.396068in}}%
\pgfpathlineto{\pgfqpoint{2.330598in}{1.407678in}}%
\pgfpathlineto{\pgfqpoint{2.324367in}{1.419262in}}%
\pgfpathlineto{\pgfqpoint{2.312456in}{1.415248in}}%
\pgfpathlineto{\pgfqpoint{2.300538in}{1.411238in}}%
\pgfpathlineto{\pgfqpoint{2.288615in}{1.407231in}}%
\pgfpathlineto{\pgfqpoint{2.276685in}{1.403227in}}%
\pgfpathlineto{\pgfqpoint{2.264750in}{1.399226in}}%
\pgfpathlineto{\pgfqpoint{2.270970in}{1.387680in}}%
\pgfpathlineto{\pgfqpoint{2.277194in}{1.376113in}}%
\pgfpathlineto{\pgfqpoint{2.283422in}{1.364524in}}%
\pgfpathlineto{\pgfqpoint{2.289654in}{1.352914in}}%
\pgfpathclose%
\pgfusepath{stroke,fill}%
\end{pgfscope}%
\begin{pgfscope}%
\pgfpathrectangle{\pgfqpoint{0.887500in}{0.275000in}}{\pgfqpoint{4.225000in}{4.225000in}}%
\pgfusepath{clip}%
\pgfsetbuttcap%
\pgfsetroundjoin%
\definecolor{currentfill}{rgb}{0.263663,0.237631,0.518762}%
\pgfsetfillcolor{currentfill}%
\pgfsetfillopacity{0.700000}%
\pgfsetlinewidth{0.501875pt}%
\definecolor{currentstroke}{rgb}{1.000000,1.000000,1.000000}%
\pgfsetstrokecolor{currentstroke}%
\pgfsetstrokeopacity{0.500000}%
\pgfsetdash{}{0pt}%
\pgfpathmoveto{\pgfqpoint{2.386841in}{1.302436in}}%
\pgfpathlineto{\pgfqpoint{2.398766in}{1.306338in}}%
\pgfpathlineto{\pgfqpoint{2.410686in}{1.310240in}}%
\pgfpathlineto{\pgfqpoint{2.422600in}{1.314143in}}%
\pgfpathlineto{\pgfqpoint{2.434507in}{1.318045in}}%
\pgfpathlineto{\pgfqpoint{2.446409in}{1.321947in}}%
\pgfpathlineto{\pgfqpoint{2.440135in}{1.333781in}}%
\pgfpathlineto{\pgfqpoint{2.433865in}{1.345603in}}%
\pgfpathlineto{\pgfqpoint{2.427598in}{1.357409in}}%
\pgfpathlineto{\pgfqpoint{2.421335in}{1.369197in}}%
\pgfpathlineto{\pgfqpoint{2.415076in}{1.380964in}}%
\pgfpathlineto{\pgfqpoint{2.403184in}{1.376985in}}%
\pgfpathlineto{\pgfqpoint{2.391287in}{1.373009in}}%
\pgfpathlineto{\pgfqpoint{2.379383in}{1.369035in}}%
\pgfpathlineto{\pgfqpoint{2.367473in}{1.365064in}}%
\pgfpathlineto{\pgfqpoint{2.355558in}{1.361096in}}%
\pgfpathlineto{\pgfqpoint{2.361807in}{1.349396in}}%
\pgfpathlineto{\pgfqpoint{2.368060in}{1.337677in}}%
\pgfpathlineto{\pgfqpoint{2.374317in}{1.325943in}}%
\pgfpathlineto{\pgfqpoint{2.380577in}{1.314195in}}%
\pgfpathclose%
\pgfusepath{stroke,fill}%
\end{pgfscope}%
\begin{pgfscope}%
\pgfpathrectangle{\pgfqpoint{0.887500in}{0.275000in}}{\pgfqpoint{4.225000in}{4.225000in}}%
\pgfusepath{clip}%
\pgfsetbuttcap%
\pgfsetroundjoin%
\definecolor{currentfill}{rgb}{0.270595,0.214069,0.507052}%
\pgfsetfillcolor{currentfill}%
\pgfsetfillopacity{0.700000}%
\pgfsetlinewidth{0.501875pt}%
\definecolor{currentstroke}{rgb}{1.000000,1.000000,1.000000}%
\pgfsetstrokecolor{currentstroke}%
\pgfsetstrokeopacity{0.500000}%
\pgfsetdash{}{0pt}%
\pgfpathmoveto{\pgfqpoint{2.477829in}{1.262693in}}%
\pgfpathlineto{\pgfqpoint{2.489734in}{1.266523in}}%
\pgfpathlineto{\pgfqpoint{2.501633in}{1.270356in}}%
\pgfpathlineto{\pgfqpoint{2.513526in}{1.274195in}}%
\pgfpathlineto{\pgfqpoint{2.525413in}{1.278041in}}%
\pgfpathlineto{\pgfqpoint{2.537293in}{1.281896in}}%
\pgfpathlineto{\pgfqpoint{2.530993in}{1.293820in}}%
\pgfpathlineto{\pgfqpoint{2.524696in}{1.305749in}}%
\pgfpathlineto{\pgfqpoint{2.518403in}{1.317680in}}%
\pgfpathlineto{\pgfqpoint{2.512112in}{1.329609in}}%
\pgfpathlineto{\pgfqpoint{2.505825in}{1.341533in}}%
\pgfpathlineto{\pgfqpoint{2.493954in}{1.337600in}}%
\pgfpathlineto{\pgfqpoint{2.482077in}{1.333677in}}%
\pgfpathlineto{\pgfqpoint{2.470194in}{1.329762in}}%
\pgfpathlineto{\pgfqpoint{2.458304in}{1.325852in}}%
\pgfpathlineto{\pgfqpoint{2.446409in}{1.321947in}}%
\pgfpathlineto{\pgfqpoint{2.452686in}{1.310104in}}%
\pgfpathlineto{\pgfqpoint{2.458967in}{1.298254in}}%
\pgfpathlineto{\pgfqpoint{2.465251in}{1.286401in}}%
\pgfpathlineto{\pgfqpoint{2.471538in}{1.274547in}}%
\pgfpathclose%
\pgfusepath{stroke,fill}%
\end{pgfscope}%
\begin{pgfscope}%
\pgfpathrectangle{\pgfqpoint{0.887500in}{0.275000in}}{\pgfqpoint{4.225000in}{4.225000in}}%
\pgfusepath{clip}%
\pgfsetbuttcap%
\pgfsetroundjoin%
\definecolor{currentfill}{rgb}{0.276194,0.190074,0.493001}%
\pgfsetfillcolor{currentfill}%
\pgfsetfillopacity{0.700000}%
\pgfsetlinewidth{0.501875pt}%
\definecolor{currentstroke}{rgb}{1.000000,1.000000,1.000000}%
\pgfsetstrokecolor{currentstroke}%
\pgfsetstrokeopacity{0.500000}%
\pgfsetdash{}{0pt}%
\pgfpathmoveto{\pgfqpoint{2.568845in}{1.222249in}}%
\pgfpathlineto{\pgfqpoint{2.580729in}{1.226028in}}%
\pgfpathlineto{\pgfqpoint{2.592606in}{1.229813in}}%
\pgfpathlineto{\pgfqpoint{2.604478in}{1.233604in}}%
\pgfpathlineto{\pgfqpoint{2.616343in}{1.237402in}}%
\pgfpathlineto{\pgfqpoint{2.628203in}{1.241207in}}%
\pgfpathlineto{\pgfqpoint{2.621876in}{1.253258in}}%
\pgfpathlineto{\pgfqpoint{2.615553in}{1.265292in}}%
\pgfpathlineto{\pgfqpoint{2.609233in}{1.277312in}}%
\pgfpathlineto{\pgfqpoint{2.602917in}{1.289323in}}%
\pgfpathlineto{\pgfqpoint{2.596604in}{1.301329in}}%
\pgfpathlineto{\pgfqpoint{2.584754in}{1.297424in}}%
\pgfpathlineto{\pgfqpoint{2.572898in}{1.293527in}}%
\pgfpathlineto{\pgfqpoint{2.561036in}{1.289639in}}%
\pgfpathlineto{\pgfqpoint{2.549168in}{1.285762in}}%
\pgfpathlineto{\pgfqpoint{2.537293in}{1.281896in}}%
\pgfpathlineto{\pgfqpoint{2.543597in}{1.269973in}}%
\pgfpathlineto{\pgfqpoint{2.549904in}{1.258049in}}%
\pgfpathlineto{\pgfqpoint{2.556214in}{1.246122in}}%
\pgfpathlineto{\pgfqpoint{2.562528in}{1.234189in}}%
\pgfpathclose%
\pgfusepath{stroke,fill}%
\end{pgfscope}%
\begin{pgfscope}%
\pgfpathrectangle{\pgfqpoint{0.887500in}{0.275000in}}{\pgfqpoint{4.225000in}{4.225000in}}%
\pgfusepath{clip}%
\pgfsetbuttcap%
\pgfsetroundjoin%
\definecolor{currentfill}{rgb}{0.283091,0.110553,0.431554}%
\pgfsetfillcolor{currentfill}%
\pgfsetfillopacity{0.700000}%
\pgfsetlinewidth{0.501875pt}%
\definecolor{currentstroke}{rgb}{1.000000,1.000000,1.000000}%
\pgfsetstrokecolor{currentstroke}%
\pgfsetstrokeopacity{0.500000}%
\pgfsetdash{}{0pt}%
\pgfpathmoveto{\pgfqpoint{2.842020in}{1.095489in}}%
\pgfpathlineto{\pgfqpoint{2.853838in}{1.098828in}}%
\pgfpathlineto{\pgfqpoint{2.865650in}{1.102149in}}%
\pgfpathlineto{\pgfqpoint{2.877455in}{1.105433in}}%
\pgfpathlineto{\pgfqpoint{2.889255in}{1.108658in}}%
\pgfpathlineto{\pgfqpoint{2.901048in}{1.111806in}}%
\pgfpathlineto{\pgfqpoint{2.894655in}{1.122399in}}%
\pgfpathlineto{\pgfqpoint{2.888262in}{1.133600in}}%
\pgfpathlineto{\pgfqpoint{2.881871in}{1.145331in}}%
\pgfpathlineto{\pgfqpoint{2.875481in}{1.157512in}}%
\pgfpathlineto{\pgfqpoint{2.869093in}{1.170061in}}%
\pgfpathlineto{\pgfqpoint{2.857308in}{1.166824in}}%
\pgfpathlineto{\pgfqpoint{2.845517in}{1.163541in}}%
\pgfpathlineto{\pgfqpoint{2.833719in}{1.160224in}}%
\pgfpathlineto{\pgfqpoint{2.821916in}{1.156884in}}%
\pgfpathlineto{\pgfqpoint{2.810106in}{1.153533in}}%
\pgfpathlineto{\pgfqpoint{2.816486in}{1.141251in}}%
\pgfpathlineto{\pgfqpoint{2.822867in}{1.129248in}}%
\pgfpathlineto{\pgfqpoint{2.829250in}{1.117581in}}%
\pgfpathlineto{\pgfqpoint{2.835634in}{1.106309in}}%
\pgfpathclose%
\pgfusepath{stroke,fill}%
\end{pgfscope}%
\begin{pgfscope}%
\pgfpathrectangle{\pgfqpoint{0.887500in}{0.275000in}}{\pgfqpoint{4.225000in}{4.225000in}}%
\pgfusepath{clip}%
\pgfsetbuttcap%
\pgfsetroundjoin%
\definecolor{currentfill}{rgb}{0.282623,0.140926,0.457517}%
\pgfsetfillcolor{currentfill}%
\pgfsetfillopacity{0.700000}%
\pgfsetlinewidth{0.501875pt}%
\definecolor{currentstroke}{rgb}{1.000000,1.000000,1.000000}%
\pgfsetstrokecolor{currentstroke}%
\pgfsetstrokeopacity{0.500000}%
\pgfsetdash{}{0pt}%
\pgfpathmoveto{\pgfqpoint{2.750961in}{1.136800in}}%
\pgfpathlineto{\pgfqpoint{2.762803in}{1.140148in}}%
\pgfpathlineto{\pgfqpoint{2.774638in}{1.143491in}}%
\pgfpathlineto{\pgfqpoint{2.786467in}{1.146835in}}%
\pgfpathlineto{\pgfqpoint{2.798290in}{1.150181in}}%
\pgfpathlineto{\pgfqpoint{2.810106in}{1.153533in}}%
\pgfpathlineto{\pgfqpoint{2.803728in}{1.166036in}}%
\pgfpathlineto{\pgfqpoint{2.797352in}{1.178702in}}%
\pgfpathlineto{\pgfqpoint{2.790978in}{1.191474in}}%
\pgfpathlineto{\pgfqpoint{2.784607in}{1.204294in}}%
\pgfpathlineto{\pgfqpoint{2.778240in}{1.217104in}}%
\pgfpathlineto{\pgfqpoint{2.766433in}{1.213423in}}%
\pgfpathlineto{\pgfqpoint{2.754619in}{1.209750in}}%
\pgfpathlineto{\pgfqpoint{2.742800in}{1.206082in}}%
\pgfpathlineto{\pgfqpoint{2.730974in}{1.202421in}}%
\pgfpathlineto{\pgfqpoint{2.719142in}{1.198765in}}%
\pgfpathlineto{\pgfqpoint{2.725500in}{1.186285in}}%
\pgfpathlineto{\pgfqpoint{2.731862in}{1.173809in}}%
\pgfpathlineto{\pgfqpoint{2.738226in}{1.161376in}}%
\pgfpathlineto{\pgfqpoint{2.744592in}{1.149027in}}%
\pgfpathclose%
\pgfusepath{stroke,fill}%
\end{pgfscope}%
\begin{pgfscope}%
\pgfpathrectangle{\pgfqpoint{0.887500in}{0.275000in}}{\pgfqpoint{4.225000in}{4.225000in}}%
\pgfusepath{clip}%
\pgfsetbuttcap%
\pgfsetroundjoin%
\definecolor{currentfill}{rgb}{0.280255,0.165693,0.476498}%
\pgfsetfillcolor{currentfill}%
\pgfsetfillopacity{0.700000}%
\pgfsetlinewidth{0.501875pt}%
\definecolor{currentstroke}{rgb}{1.000000,1.000000,1.000000}%
\pgfsetstrokecolor{currentstroke}%
\pgfsetstrokeopacity{0.500000}%
\pgfsetdash{}{0pt}%
\pgfpathmoveto{\pgfqpoint{2.659890in}{1.180556in}}%
\pgfpathlineto{\pgfqpoint{2.671753in}{1.184191in}}%
\pgfpathlineto{\pgfqpoint{2.683609in}{1.187828in}}%
\pgfpathlineto{\pgfqpoint{2.695460in}{1.191469in}}%
\pgfpathlineto{\pgfqpoint{2.707304in}{1.195115in}}%
\pgfpathlineto{\pgfqpoint{2.719142in}{1.198765in}}%
\pgfpathlineto{\pgfqpoint{2.712788in}{1.211210in}}%
\pgfpathlineto{\pgfqpoint{2.706437in}{1.223590in}}%
\pgfpathlineto{\pgfqpoint{2.700090in}{1.235905in}}%
\pgfpathlineto{\pgfqpoint{2.693747in}{1.248163in}}%
\pgfpathlineto{\pgfqpoint{2.687407in}{1.260372in}}%
\pgfpathlineto{\pgfqpoint{2.675579in}{1.256520in}}%
\pgfpathlineto{\pgfqpoint{2.663744in}{1.252677in}}%
\pgfpathlineto{\pgfqpoint{2.651903in}{1.248844in}}%
\pgfpathlineto{\pgfqpoint{2.640056in}{1.245021in}}%
\pgfpathlineto{\pgfqpoint{2.628203in}{1.241207in}}%
\pgfpathlineto{\pgfqpoint{2.634533in}{1.229135in}}%
\pgfpathlineto{\pgfqpoint{2.640867in}{1.217037in}}%
\pgfpathlineto{\pgfqpoint{2.647204in}{1.204910in}}%
\pgfpathlineto{\pgfqpoint{2.653545in}{1.192749in}}%
\pgfpathclose%
\pgfusepath{stroke,fill}%
\end{pgfscope}%
\begin{pgfscope}%
\pgfpathrectangle{\pgfqpoint{0.887500in}{0.275000in}}{\pgfqpoint{4.225000in}{4.225000in}}%
\pgfusepath{clip}%
\pgfsetbuttcap%
\pgfsetroundjoin%
\definecolor{currentfill}{rgb}{0.282327,0.094955,0.417331}%
\pgfsetfillcolor{currentfill}%
\pgfsetfillopacity{0.700000}%
\pgfsetlinewidth{0.501875pt}%
\definecolor{currentstroke}{rgb}{1.000000,1.000000,1.000000}%
\pgfsetstrokecolor{currentstroke}%
\pgfsetstrokeopacity{0.500000}%
\pgfsetdash{}{0pt}%
\pgfpathmoveto{\pgfqpoint{2.933036in}{1.070788in}}%
\pgfpathlineto{\pgfqpoint{2.944834in}{1.074767in}}%
\pgfpathlineto{\pgfqpoint{2.956626in}{1.078661in}}%
\pgfpathlineto{\pgfqpoint{2.968412in}{1.082514in}}%
\pgfpathlineto{\pgfqpoint{2.980192in}{1.086388in}}%
\pgfpathlineto{\pgfqpoint{2.991966in}{1.090348in}}%
\pgfpathlineto{\pgfqpoint{2.985552in}{1.095081in}}%
\pgfpathlineto{\pgfqpoint{2.979141in}{1.101175in}}%
\pgfpathlineto{\pgfqpoint{2.972732in}{1.108519in}}%
\pgfpathlineto{\pgfqpoint{2.966326in}{1.117002in}}%
\pgfpathlineto{\pgfqpoint{2.959922in}{1.126514in}}%
\pgfpathlineto{\pgfqpoint{2.948160in}{1.123561in}}%
\pgfpathlineto{\pgfqpoint{2.936392in}{1.120679in}}%
\pgfpathlineto{\pgfqpoint{2.924617in}{1.117800in}}%
\pgfpathlineto{\pgfqpoint{2.912836in}{1.114857in}}%
\pgfpathlineto{\pgfqpoint{2.901048in}{1.111806in}}%
\pgfpathlineto{\pgfqpoint{2.907443in}{1.101903in}}%
\pgfpathlineto{\pgfqpoint{2.913839in}{1.092769in}}%
\pgfpathlineto{\pgfqpoint{2.920237in}{1.084485in}}%
\pgfpathlineto{\pgfqpoint{2.926635in}{1.077131in}}%
\pgfpathclose%
\pgfusepath{stroke,fill}%
\end{pgfscope}%
\begin{pgfscope}%
\pgfpathrectangle{\pgfqpoint{0.887500in}{0.275000in}}{\pgfqpoint{4.225000in}{4.225000in}}%
\pgfusepath{clip}%
\pgfsetbuttcap%
\pgfsetroundjoin%
\definecolor{currentfill}{rgb}{0.283197,0.115680,0.436115}%
\pgfsetfillcolor{currentfill}%
\pgfsetfillopacity{0.700000}%
\pgfsetlinewidth{0.501875pt}%
\definecolor{currentstroke}{rgb}{1.000000,1.000000,1.000000}%
\pgfsetstrokecolor{currentstroke}%
\pgfsetstrokeopacity{0.500000}%
\pgfsetdash{}{0pt}%
\pgfpathmoveto{\pgfqpoint{3.083029in}{1.126292in}}%
\pgfpathlineto{\pgfqpoint{3.094802in}{1.133307in}}%
\pgfpathlineto{\pgfqpoint{3.106570in}{1.140182in}}%
\pgfpathlineto{\pgfqpoint{3.118334in}{1.146857in}}%
\pgfpathlineto{\pgfqpoint{3.130093in}{1.153270in}}%
\pgfpathlineto{\pgfqpoint{3.141847in}{1.159359in}}%
\pgfpathlineto{\pgfqpoint{3.135330in}{1.151095in}}%
\pgfpathlineto{\pgfqpoint{3.128827in}{1.145209in}}%
\pgfpathlineto{\pgfqpoint{3.122338in}{1.141562in}}%
\pgfpathlineto{\pgfqpoint{3.115861in}{1.140017in}}%
\pgfpathlineto{\pgfqpoint{3.109395in}{1.140437in}}%
\pgfpathlineto{\pgfqpoint{3.097675in}{1.135169in}}%
\pgfpathlineto{\pgfqpoint{3.085950in}{1.129756in}}%
\pgfpathlineto{\pgfqpoint{3.074221in}{1.124280in}}%
\pgfpathlineto{\pgfqpoint{3.062486in}{1.118822in}}%
\pgfpathlineto{\pgfqpoint{3.050746in}{1.113464in}}%
\pgfpathlineto{\pgfqpoint{3.057186in}{1.111985in}}%
\pgfpathlineto{\pgfqpoint{3.063633in}{1.112393in}}%
\pgfpathlineto{\pgfqpoint{3.070089in}{1.114824in}}%
\pgfpathlineto{\pgfqpoint{3.076553in}{1.119411in}}%
\pgfpathclose%
\pgfusepath{stroke,fill}%
\end{pgfscope}%
\begin{pgfscope}%
\pgfpathrectangle{\pgfqpoint{0.887500in}{0.275000in}}{\pgfqpoint{4.225000in}{4.225000in}}%
\pgfusepath{clip}%
\pgfsetbuttcap%
\pgfsetroundjoin%
\definecolor{currentfill}{rgb}{0.257322,0.256130,0.526563}%
\pgfsetfillcolor{currentfill}%
\pgfsetfillopacity{0.700000}%
\pgfsetlinewidth{0.501875pt}%
\definecolor{currentstroke}{rgb}{1.000000,1.000000,1.000000}%
\pgfsetstrokecolor{currentstroke}%
\pgfsetstrokeopacity{0.500000}%
\pgfsetdash{}{0pt}%
\pgfpathmoveto{\pgfqpoint{2.236070in}{1.321523in}}%
\pgfpathlineto{\pgfqpoint{2.248046in}{1.325473in}}%
\pgfpathlineto{\pgfqpoint{2.260016in}{1.329424in}}%
\pgfpathlineto{\pgfqpoint{2.271980in}{1.333376in}}%
\pgfpathlineto{\pgfqpoint{2.283938in}{1.337331in}}%
\pgfpathlineto{\pgfqpoint{2.295890in}{1.341286in}}%
\pgfpathlineto{\pgfqpoint{2.289654in}{1.352914in}}%
\pgfpathlineto{\pgfqpoint{2.283422in}{1.364524in}}%
\pgfpathlineto{\pgfqpoint{2.277194in}{1.376113in}}%
\pgfpathlineto{\pgfqpoint{2.270970in}{1.387680in}}%
\pgfpathlineto{\pgfqpoint{2.264750in}{1.399226in}}%
\pgfpathlineto{\pgfqpoint{2.252808in}{1.395228in}}%
\pgfpathlineto{\pgfqpoint{2.240860in}{1.391232in}}%
\pgfpathlineto{\pgfqpoint{2.228907in}{1.387238in}}%
\pgfpathlineto{\pgfqpoint{2.216947in}{1.383246in}}%
\pgfpathlineto{\pgfqpoint{2.204981in}{1.379256in}}%
\pgfpathlineto{\pgfqpoint{2.211192in}{1.367748in}}%
\pgfpathlineto{\pgfqpoint{2.217406in}{1.356219in}}%
\pgfpathlineto{\pgfqpoint{2.223623in}{1.344672in}}%
\pgfpathlineto{\pgfqpoint{2.229845in}{1.333106in}}%
\pgfpathclose%
\pgfusepath{stroke,fill}%
\end{pgfscope}%
\begin{pgfscope}%
\pgfpathrectangle{\pgfqpoint{0.887500in}{0.275000in}}{\pgfqpoint{4.225000in}{4.225000in}}%
\pgfusepath{clip}%
\pgfsetbuttcap%
\pgfsetroundjoin%
\definecolor{currentfill}{rgb}{0.265145,0.232956,0.516599}%
\pgfsetfillcolor{currentfill}%
\pgfsetfillopacity{0.700000}%
\pgfsetlinewidth{0.501875pt}%
\definecolor{currentstroke}{rgb}{1.000000,1.000000,1.000000}%
\pgfsetstrokecolor{currentstroke}%
\pgfsetstrokeopacity{0.500000}%
\pgfsetdash{}{0pt}%
\pgfpathmoveto{\pgfqpoint{2.327122in}{1.282927in}}%
\pgfpathlineto{\pgfqpoint{2.339078in}{1.286828in}}%
\pgfpathlineto{\pgfqpoint{2.351027in}{1.290730in}}%
\pgfpathlineto{\pgfqpoint{2.362971in}{1.294632in}}%
\pgfpathlineto{\pgfqpoint{2.374909in}{1.298534in}}%
\pgfpathlineto{\pgfqpoint{2.386841in}{1.302436in}}%
\pgfpathlineto{\pgfqpoint{2.380577in}{1.314195in}}%
\pgfpathlineto{\pgfqpoint{2.374317in}{1.325943in}}%
\pgfpathlineto{\pgfqpoint{2.368060in}{1.337677in}}%
\pgfpathlineto{\pgfqpoint{2.361807in}{1.349396in}}%
\pgfpathlineto{\pgfqpoint{2.355558in}{1.361096in}}%
\pgfpathlineto{\pgfqpoint{2.343636in}{1.357130in}}%
\pgfpathlineto{\pgfqpoint{2.331709in}{1.353166in}}%
\pgfpathlineto{\pgfqpoint{2.319775in}{1.349204in}}%
\pgfpathlineto{\pgfqpoint{2.307835in}{1.345244in}}%
\pgfpathlineto{\pgfqpoint{2.295890in}{1.341286in}}%
\pgfpathlineto{\pgfqpoint{2.302129in}{1.329641in}}%
\pgfpathlineto{\pgfqpoint{2.308372in}{1.317981in}}%
\pgfpathlineto{\pgfqpoint{2.314618in}{1.306307in}}%
\pgfpathlineto{\pgfqpoint{2.320868in}{1.294622in}}%
\pgfpathclose%
\pgfusepath{stroke,fill}%
\end{pgfscope}%
\begin{pgfscope}%
\pgfpathrectangle{\pgfqpoint{0.887500in}{0.275000in}}{\pgfqpoint{4.225000in}{4.225000in}}%
\pgfusepath{clip}%
\pgfsetbuttcap%
\pgfsetroundjoin%
\definecolor{currentfill}{rgb}{0.270595,0.214069,0.507052}%
\pgfsetfillcolor{currentfill}%
\pgfsetfillopacity{0.700000}%
\pgfsetlinewidth{0.501875pt}%
\definecolor{currentstroke}{rgb}{1.000000,1.000000,1.000000}%
\pgfsetstrokecolor{currentstroke}%
\pgfsetstrokeopacity{0.500000}%
\pgfsetdash{}{0pt}%
\pgfpathmoveto{\pgfqpoint{2.418211in}{1.243542in}}%
\pgfpathlineto{\pgfqpoint{2.430147in}{1.247374in}}%
\pgfpathlineto{\pgfqpoint{2.442076in}{1.251205in}}%
\pgfpathlineto{\pgfqpoint{2.454000in}{1.255036in}}%
\pgfpathlineto{\pgfqpoint{2.465917in}{1.258864in}}%
\pgfpathlineto{\pgfqpoint{2.477829in}{1.262693in}}%
\pgfpathlineto{\pgfqpoint{2.471538in}{1.274547in}}%
\pgfpathlineto{\pgfqpoint{2.465251in}{1.286401in}}%
\pgfpathlineto{\pgfqpoint{2.458967in}{1.298254in}}%
\pgfpathlineto{\pgfqpoint{2.452686in}{1.310104in}}%
\pgfpathlineto{\pgfqpoint{2.446409in}{1.321947in}}%
\pgfpathlineto{\pgfqpoint{2.434507in}{1.318045in}}%
\pgfpathlineto{\pgfqpoint{2.422600in}{1.314143in}}%
\pgfpathlineto{\pgfqpoint{2.410686in}{1.310240in}}%
\pgfpathlineto{\pgfqpoint{2.398766in}{1.306338in}}%
\pgfpathlineto{\pgfqpoint{2.386841in}{1.302436in}}%
\pgfpathlineto{\pgfqpoint{2.393108in}{1.290668in}}%
\pgfpathlineto{\pgfqpoint{2.399379in}{1.278892in}}%
\pgfpathlineto{\pgfqpoint{2.405653in}{1.267111in}}%
\pgfpathlineto{\pgfqpoint{2.411930in}{1.255327in}}%
\pgfpathclose%
\pgfusepath{stroke,fill}%
\end{pgfscope}%
\begin{pgfscope}%
\pgfpathrectangle{\pgfqpoint{0.887500in}{0.275000in}}{\pgfqpoint{4.225000in}{4.225000in}}%
\pgfusepath{clip}%
\pgfsetbuttcap%
\pgfsetroundjoin%
\definecolor{currentfill}{rgb}{0.276194,0.190074,0.493001}%
\pgfsetfillcolor{currentfill}%
\pgfsetfillopacity{0.700000}%
\pgfsetlinewidth{0.501875pt}%
\definecolor{currentstroke}{rgb}{1.000000,1.000000,1.000000}%
\pgfsetstrokecolor{currentstroke}%
\pgfsetstrokeopacity{0.500000}%
\pgfsetdash{}{0pt}%
\pgfpathmoveto{\pgfqpoint{2.509333in}{1.203411in}}%
\pgfpathlineto{\pgfqpoint{2.521247in}{1.207174in}}%
\pgfpathlineto{\pgfqpoint{2.533156in}{1.210938in}}%
\pgfpathlineto{\pgfqpoint{2.545058in}{1.214705in}}%
\pgfpathlineto{\pgfqpoint{2.556955in}{1.218475in}}%
\pgfpathlineto{\pgfqpoint{2.568845in}{1.222249in}}%
\pgfpathlineto{\pgfqpoint{2.562528in}{1.234189in}}%
\pgfpathlineto{\pgfqpoint{2.556214in}{1.246122in}}%
\pgfpathlineto{\pgfqpoint{2.549904in}{1.258049in}}%
\pgfpathlineto{\pgfqpoint{2.543597in}{1.269973in}}%
\pgfpathlineto{\pgfqpoint{2.537293in}{1.281896in}}%
\pgfpathlineto{\pgfqpoint{2.525413in}{1.278041in}}%
\pgfpathlineto{\pgfqpoint{2.513526in}{1.274195in}}%
\pgfpathlineto{\pgfqpoint{2.501633in}{1.270356in}}%
\pgfpathlineto{\pgfqpoint{2.489734in}{1.266523in}}%
\pgfpathlineto{\pgfqpoint{2.477829in}{1.262693in}}%
\pgfpathlineto{\pgfqpoint{2.484123in}{1.250840in}}%
\pgfpathlineto{\pgfqpoint{2.490420in}{1.238986in}}%
\pgfpathlineto{\pgfqpoint{2.496721in}{1.227130in}}%
\pgfpathlineto{\pgfqpoint{2.503025in}{1.215272in}}%
\pgfpathclose%
\pgfusepath{stroke,fill}%
\end{pgfscope}%
\begin{pgfscope}%
\pgfpathrectangle{\pgfqpoint{0.887500in}{0.275000in}}{\pgfqpoint{4.225000in}{4.225000in}}%
\pgfusepath{clip}%
\pgfsetbuttcap%
\pgfsetroundjoin%
\definecolor{currentfill}{rgb}{0.280255,0.165693,0.476498}%
\pgfsetfillcolor{currentfill}%
\pgfsetfillopacity{0.700000}%
\pgfsetlinewidth{0.501875pt}%
\definecolor{currentstroke}{rgb}{1.000000,1.000000,1.000000}%
\pgfsetstrokecolor{currentstroke}%
\pgfsetstrokeopacity{0.500000}%
\pgfsetdash{}{0pt}%
\pgfpathmoveto{\pgfqpoint{2.600483in}{1.162354in}}%
\pgfpathlineto{\pgfqpoint{2.612377in}{1.166003in}}%
\pgfpathlineto{\pgfqpoint{2.624264in}{1.169646in}}%
\pgfpathlineto{\pgfqpoint{2.636145in}{1.173285in}}%
\pgfpathlineto{\pgfqpoint{2.648021in}{1.176921in}}%
\pgfpathlineto{\pgfqpoint{2.659890in}{1.180556in}}%
\pgfpathlineto{\pgfqpoint{2.653545in}{1.192749in}}%
\pgfpathlineto{\pgfqpoint{2.647204in}{1.204910in}}%
\pgfpathlineto{\pgfqpoint{2.640867in}{1.217037in}}%
\pgfpathlineto{\pgfqpoint{2.634533in}{1.229135in}}%
\pgfpathlineto{\pgfqpoint{2.628203in}{1.241207in}}%
\pgfpathlineto{\pgfqpoint{2.616343in}{1.237402in}}%
\pgfpathlineto{\pgfqpoint{2.604478in}{1.233604in}}%
\pgfpathlineto{\pgfqpoint{2.592606in}{1.229813in}}%
\pgfpathlineto{\pgfqpoint{2.580729in}{1.226028in}}%
\pgfpathlineto{\pgfqpoint{2.568845in}{1.222249in}}%
\pgfpathlineto{\pgfqpoint{2.575165in}{1.210299in}}%
\pgfpathlineto{\pgfqpoint{2.581489in}{1.198336in}}%
\pgfpathlineto{\pgfqpoint{2.587817in}{1.186359in}}%
\pgfpathlineto{\pgfqpoint{2.594148in}{1.174365in}}%
\pgfpathclose%
\pgfusepath{stroke,fill}%
\end{pgfscope}%
\begin{pgfscope}%
\pgfpathrectangle{\pgfqpoint{0.887500in}{0.275000in}}{\pgfqpoint{4.225000in}{4.225000in}}%
\pgfusepath{clip}%
\pgfsetbuttcap%
\pgfsetroundjoin%
\definecolor{currentfill}{rgb}{0.282623,0.140926,0.457517}%
\pgfsetfillcolor{currentfill}%
\pgfsetfillopacity{0.700000}%
\pgfsetlinewidth{0.501875pt}%
\definecolor{currentstroke}{rgb}{1.000000,1.000000,1.000000}%
\pgfsetstrokecolor{currentstroke}%
\pgfsetstrokeopacity{0.500000}%
\pgfsetdash{}{0pt}%
\pgfpathmoveto{\pgfqpoint{2.691660in}{1.119898in}}%
\pgfpathlineto{\pgfqpoint{2.703533in}{1.123308in}}%
\pgfpathlineto{\pgfqpoint{2.715399in}{1.126702in}}%
\pgfpathlineto{\pgfqpoint{2.727260in}{1.130080in}}%
\pgfpathlineto{\pgfqpoint{2.739114in}{1.133445in}}%
\pgfpathlineto{\pgfqpoint{2.750961in}{1.136800in}}%
\pgfpathlineto{\pgfqpoint{2.744592in}{1.149027in}}%
\pgfpathlineto{\pgfqpoint{2.738226in}{1.161376in}}%
\pgfpathlineto{\pgfqpoint{2.731862in}{1.173809in}}%
\pgfpathlineto{\pgfqpoint{2.725500in}{1.186285in}}%
\pgfpathlineto{\pgfqpoint{2.719142in}{1.198765in}}%
\pgfpathlineto{\pgfqpoint{2.707304in}{1.195115in}}%
\pgfpathlineto{\pgfqpoint{2.695460in}{1.191469in}}%
\pgfpathlineto{\pgfqpoint{2.683609in}{1.187828in}}%
\pgfpathlineto{\pgfqpoint{2.671753in}{1.184191in}}%
\pgfpathlineto{\pgfqpoint{2.659890in}{1.180556in}}%
\pgfpathlineto{\pgfqpoint{2.666238in}{1.168348in}}%
\pgfpathlineto{\pgfqpoint{2.672589in}{1.156153in}}%
\pgfpathlineto{\pgfqpoint{2.678943in}{1.143996in}}%
\pgfpathlineto{\pgfqpoint{2.685300in}{1.131903in}}%
\pgfpathclose%
\pgfusepath{stroke,fill}%
\end{pgfscope}%
\begin{pgfscope}%
\pgfpathrectangle{\pgfqpoint{0.887500in}{0.275000in}}{\pgfqpoint{4.225000in}{4.225000in}}%
\pgfusepath{clip}%
\pgfsetbuttcap%
\pgfsetroundjoin%
\definecolor{currentfill}{rgb}{0.283197,0.115680,0.436115}%
\pgfsetfillcolor{currentfill}%
\pgfsetfillopacity{0.700000}%
\pgfsetlinewidth{0.501875pt}%
\definecolor{currentstroke}{rgb}{1.000000,1.000000,1.000000}%
\pgfsetstrokecolor{currentstroke}%
\pgfsetstrokeopacity{0.500000}%
\pgfsetdash{}{0pt}%
\pgfpathmoveto{\pgfqpoint{2.782833in}{1.078877in}}%
\pgfpathlineto{\pgfqpoint{2.794683in}{1.082194in}}%
\pgfpathlineto{\pgfqpoint{2.806527in}{1.085508in}}%
\pgfpathlineto{\pgfqpoint{2.818364in}{1.088825in}}%
\pgfpathlineto{\pgfqpoint{2.830195in}{1.092151in}}%
\pgfpathlineto{\pgfqpoint{2.842020in}{1.095489in}}%
\pgfpathlineto{\pgfqpoint{2.835634in}{1.106309in}}%
\pgfpathlineto{\pgfqpoint{2.829250in}{1.117581in}}%
\pgfpathlineto{\pgfqpoint{2.822867in}{1.129248in}}%
\pgfpathlineto{\pgfqpoint{2.816486in}{1.141251in}}%
\pgfpathlineto{\pgfqpoint{2.810106in}{1.153533in}}%
\pgfpathlineto{\pgfqpoint{2.798290in}{1.150181in}}%
\pgfpathlineto{\pgfqpoint{2.786467in}{1.146835in}}%
\pgfpathlineto{\pgfqpoint{2.774638in}{1.143491in}}%
\pgfpathlineto{\pgfqpoint{2.762803in}{1.140148in}}%
\pgfpathlineto{\pgfqpoint{2.750961in}{1.136800in}}%
\pgfpathlineto{\pgfqpoint{2.757333in}{1.124735in}}%
\pgfpathlineto{\pgfqpoint{2.763705in}{1.112870in}}%
\pgfpathlineto{\pgfqpoint{2.770080in}{1.101246in}}%
\pgfpathlineto{\pgfqpoint{2.776456in}{1.089902in}}%
\pgfpathclose%
\pgfusepath{stroke,fill}%
\end{pgfscope}%
\begin{pgfscope}%
\pgfpathrectangle{\pgfqpoint{0.887500in}{0.275000in}}{\pgfqpoint{4.225000in}{4.225000in}}%
\pgfusepath{clip}%
\pgfsetbuttcap%
\pgfsetroundjoin%
\definecolor{currentfill}{rgb}{0.282910,0.105393,0.426902}%
\pgfsetfillcolor{currentfill}%
\pgfsetfillopacity{0.700000}%
\pgfsetlinewidth{0.501875pt}%
\definecolor{currentstroke}{rgb}{1.000000,1.000000,1.000000}%
\pgfsetstrokecolor{currentstroke}%
\pgfsetstrokeopacity{0.500000}%
\pgfsetdash{}{0pt}%
\pgfpathmoveto{\pgfqpoint{3.024105in}{1.091025in}}%
\pgfpathlineto{\pgfqpoint{3.035898in}{1.097998in}}%
\pgfpathlineto{\pgfqpoint{3.047687in}{1.105022in}}%
\pgfpathlineto{\pgfqpoint{3.059471in}{1.112094in}}%
\pgfpathlineto{\pgfqpoint{3.071252in}{1.119201in}}%
\pgfpathlineto{\pgfqpoint{3.083029in}{1.126292in}}%
\pgfpathlineto{\pgfqpoint{3.076553in}{1.119411in}}%
\pgfpathlineto{\pgfqpoint{3.070089in}{1.114824in}}%
\pgfpathlineto{\pgfqpoint{3.063633in}{1.112393in}}%
\pgfpathlineto{\pgfqpoint{3.057186in}{1.111985in}}%
\pgfpathlineto{\pgfqpoint{3.050746in}{1.113464in}}%
\pgfpathlineto{\pgfqpoint{3.039002in}{1.108287in}}%
\pgfpathlineto{\pgfqpoint{3.027251in}{1.103373in}}%
\pgfpathlineto{\pgfqpoint{3.015496in}{1.098776in}}%
\pgfpathlineto{\pgfqpoint{3.003734in}{1.094456in}}%
\pgfpathlineto{\pgfqpoint{2.991966in}{1.090348in}}%
\pgfpathlineto{\pgfqpoint{2.998384in}{1.087087in}}%
\pgfpathlineto{\pgfqpoint{3.004806in}{1.085412in}}%
\pgfpathlineto{\pgfqpoint{3.011233in}{1.085434in}}%
\pgfpathlineto{\pgfqpoint{3.017666in}{1.087267in}}%
\pgfpathclose%
\pgfusepath{stroke,fill}%
\end{pgfscope}%
\begin{pgfscope}%
\pgfpathrectangle{\pgfqpoint{0.887500in}{0.275000in}}{\pgfqpoint{4.225000in}{4.225000in}}%
\pgfusepath{clip}%
\pgfsetbuttcap%
\pgfsetroundjoin%
\definecolor{currentfill}{rgb}{0.282327,0.094955,0.417331}%
\pgfsetfillcolor{currentfill}%
\pgfsetfillopacity{0.700000}%
\pgfsetlinewidth{0.501875pt}%
\definecolor{currentstroke}{rgb}{1.000000,1.000000,1.000000}%
\pgfsetstrokecolor{currentstroke}%
\pgfsetstrokeopacity{0.500000}%
\pgfsetdash{}{0pt}%
\pgfpathmoveto{\pgfqpoint{2.873961in}{1.050195in}}%
\pgfpathlineto{\pgfqpoint{2.885788in}{1.054343in}}%
\pgfpathlineto{\pgfqpoint{2.897609in}{1.058495in}}%
\pgfpathlineto{\pgfqpoint{2.909424in}{1.062633in}}%
\pgfpathlineto{\pgfqpoint{2.921233in}{1.066737in}}%
\pgfpathlineto{\pgfqpoint{2.933036in}{1.070788in}}%
\pgfpathlineto{\pgfqpoint{2.926635in}{1.077131in}}%
\pgfpathlineto{\pgfqpoint{2.920237in}{1.084485in}}%
\pgfpathlineto{\pgfqpoint{2.913839in}{1.092769in}}%
\pgfpathlineto{\pgfqpoint{2.907443in}{1.101903in}}%
\pgfpathlineto{\pgfqpoint{2.901048in}{1.111806in}}%
\pgfpathlineto{\pgfqpoint{2.889255in}{1.108658in}}%
\pgfpathlineto{\pgfqpoint{2.877455in}{1.105433in}}%
\pgfpathlineto{\pgfqpoint{2.865650in}{1.102149in}}%
\pgfpathlineto{\pgfqpoint{2.853838in}{1.098828in}}%
\pgfpathlineto{\pgfqpoint{2.842020in}{1.095489in}}%
\pgfpathlineto{\pgfqpoint{2.848406in}{1.085179in}}%
\pgfpathlineto{\pgfqpoint{2.854793in}{1.075436in}}%
\pgfpathlineto{\pgfqpoint{2.861182in}{1.066319in}}%
\pgfpathlineto{\pgfqpoint{2.867571in}{1.057886in}}%
\pgfpathclose%
\pgfusepath{stroke,fill}%
\end{pgfscope}%
\begin{pgfscope}%
\pgfpathrectangle{\pgfqpoint{0.887500in}{0.275000in}}{\pgfqpoint{4.225000in}{4.225000in}}%
\pgfusepath{clip}%
\pgfsetbuttcap%
\pgfsetroundjoin%
\definecolor{currentfill}{rgb}{0.265145,0.232956,0.516599}%
\pgfsetfillcolor{currentfill}%
\pgfsetfillopacity{0.700000}%
\pgfsetlinewidth{0.501875pt}%
\definecolor{currentstroke}{rgb}{1.000000,1.000000,1.000000}%
\pgfsetstrokecolor{currentstroke}%
\pgfsetstrokeopacity{0.500000}%
\pgfsetdash{}{0pt}%
\pgfpathmoveto{\pgfqpoint{2.267251in}{1.263421in}}%
\pgfpathlineto{\pgfqpoint{2.279237in}{1.267321in}}%
\pgfpathlineto{\pgfqpoint{2.291217in}{1.271222in}}%
\pgfpathlineto{\pgfqpoint{2.303191in}{1.275124in}}%
\pgfpathlineto{\pgfqpoint{2.315160in}{1.279025in}}%
\pgfpathlineto{\pgfqpoint{2.327122in}{1.282927in}}%
\pgfpathlineto{\pgfqpoint{2.320868in}{1.294622in}}%
\pgfpathlineto{\pgfqpoint{2.314618in}{1.306307in}}%
\pgfpathlineto{\pgfqpoint{2.308372in}{1.317981in}}%
\pgfpathlineto{\pgfqpoint{2.302129in}{1.329641in}}%
\pgfpathlineto{\pgfqpoint{2.295890in}{1.341286in}}%
\pgfpathlineto{\pgfqpoint{2.283938in}{1.337331in}}%
\pgfpathlineto{\pgfqpoint{2.271980in}{1.333376in}}%
\pgfpathlineto{\pgfqpoint{2.260016in}{1.329424in}}%
\pgfpathlineto{\pgfqpoint{2.248046in}{1.325473in}}%
\pgfpathlineto{\pgfqpoint{2.236070in}{1.321523in}}%
\pgfpathlineto{\pgfqpoint{2.242299in}{1.309926in}}%
\pgfpathlineto{\pgfqpoint{2.248531in}{1.298315in}}%
\pgfpathlineto{\pgfqpoint{2.254768in}{1.286693in}}%
\pgfpathlineto{\pgfqpoint{2.261007in}{1.275061in}}%
\pgfpathclose%
\pgfusepath{stroke,fill}%
\end{pgfscope}%
\begin{pgfscope}%
\pgfpathrectangle{\pgfqpoint{0.887500in}{0.275000in}}{\pgfqpoint{4.225000in}{4.225000in}}%
\pgfusepath{clip}%
\pgfsetbuttcap%
\pgfsetroundjoin%
\definecolor{currentfill}{rgb}{0.270595,0.214069,0.507052}%
\pgfsetfillcolor{currentfill}%
\pgfsetfillopacity{0.700000}%
\pgfsetlinewidth{0.501875pt}%
\definecolor{currentstroke}{rgb}{1.000000,1.000000,1.000000}%
\pgfsetstrokecolor{currentstroke}%
\pgfsetstrokeopacity{0.500000}%
\pgfsetdash{}{0pt}%
\pgfpathmoveto{\pgfqpoint{2.358441in}{1.224364in}}%
\pgfpathlineto{\pgfqpoint{2.370408in}{1.228202in}}%
\pgfpathlineto{\pgfqpoint{2.382368in}{1.232038in}}%
\pgfpathlineto{\pgfqpoint{2.394322in}{1.235874in}}%
\pgfpathlineto{\pgfqpoint{2.406269in}{1.239709in}}%
\pgfpathlineto{\pgfqpoint{2.418211in}{1.243542in}}%
\pgfpathlineto{\pgfqpoint{2.411930in}{1.255327in}}%
\pgfpathlineto{\pgfqpoint{2.405653in}{1.267111in}}%
\pgfpathlineto{\pgfqpoint{2.399379in}{1.278892in}}%
\pgfpathlineto{\pgfqpoint{2.393108in}{1.290668in}}%
\pgfpathlineto{\pgfqpoint{2.386841in}{1.302436in}}%
\pgfpathlineto{\pgfqpoint{2.374909in}{1.298534in}}%
\pgfpathlineto{\pgfqpoint{2.362971in}{1.294632in}}%
\pgfpathlineto{\pgfqpoint{2.351027in}{1.290730in}}%
\pgfpathlineto{\pgfqpoint{2.339078in}{1.286828in}}%
\pgfpathlineto{\pgfqpoint{2.327122in}{1.282927in}}%
\pgfpathlineto{\pgfqpoint{2.333379in}{1.271223in}}%
\pgfpathlineto{\pgfqpoint{2.339639in}{1.259514in}}%
\pgfpathlineto{\pgfqpoint{2.345903in}{1.247799in}}%
\pgfpathlineto{\pgfqpoint{2.352171in}{1.236082in}}%
\pgfpathclose%
\pgfusepath{stroke,fill}%
\end{pgfscope}%
\begin{pgfscope}%
\pgfpathrectangle{\pgfqpoint{0.887500in}{0.275000in}}{\pgfqpoint{4.225000in}{4.225000in}}%
\pgfusepath{clip}%
\pgfsetbuttcap%
\pgfsetroundjoin%
\definecolor{currentfill}{rgb}{0.276194,0.190074,0.493001}%
\pgfsetfillcolor{currentfill}%
\pgfsetfillopacity{0.700000}%
\pgfsetlinewidth{0.501875pt}%
\definecolor{currentstroke}{rgb}{1.000000,1.000000,1.000000}%
\pgfsetstrokecolor{currentstroke}%
\pgfsetstrokeopacity{0.500000}%
\pgfsetdash{}{0pt}%
\pgfpathmoveto{\pgfqpoint{2.449666in}{1.184604in}}%
\pgfpathlineto{\pgfqpoint{2.461612in}{1.188366in}}%
\pgfpathlineto{\pgfqpoint{2.473551in}{1.192127in}}%
\pgfpathlineto{\pgfqpoint{2.485485in}{1.195888in}}%
\pgfpathlineto{\pgfqpoint{2.497412in}{1.199650in}}%
\pgfpathlineto{\pgfqpoint{2.509333in}{1.203411in}}%
\pgfpathlineto{\pgfqpoint{2.503025in}{1.215272in}}%
\pgfpathlineto{\pgfqpoint{2.496721in}{1.227130in}}%
\pgfpathlineto{\pgfqpoint{2.490420in}{1.238986in}}%
\pgfpathlineto{\pgfqpoint{2.484123in}{1.250840in}}%
\pgfpathlineto{\pgfqpoint{2.477829in}{1.262693in}}%
\pgfpathlineto{\pgfqpoint{2.465917in}{1.258864in}}%
\pgfpathlineto{\pgfqpoint{2.454000in}{1.255036in}}%
\pgfpathlineto{\pgfqpoint{2.442076in}{1.251205in}}%
\pgfpathlineto{\pgfqpoint{2.430147in}{1.247374in}}%
\pgfpathlineto{\pgfqpoint{2.418211in}{1.243542in}}%
\pgfpathlineto{\pgfqpoint{2.424495in}{1.231756in}}%
\pgfpathlineto{\pgfqpoint{2.430783in}{1.219968in}}%
\pgfpathlineto{\pgfqpoint{2.437074in}{1.208180in}}%
\pgfpathlineto{\pgfqpoint{2.443369in}{1.196392in}}%
\pgfpathclose%
\pgfusepath{stroke,fill}%
\end{pgfscope}%
\begin{pgfscope}%
\pgfpathrectangle{\pgfqpoint{0.887500in}{0.275000in}}{\pgfqpoint{4.225000in}{4.225000in}}%
\pgfusepath{clip}%
\pgfsetbuttcap%
\pgfsetroundjoin%
\definecolor{currentfill}{rgb}{0.280255,0.165693,0.476498}%
\pgfsetfillcolor{currentfill}%
\pgfsetfillopacity{0.700000}%
\pgfsetlinewidth{0.501875pt}%
\definecolor{currentstroke}{rgb}{1.000000,1.000000,1.000000}%
\pgfsetstrokecolor{currentstroke}%
\pgfsetstrokeopacity{0.500000}%
\pgfsetdash{}{0pt}%
\pgfpathmoveto{\pgfqpoint{2.540922in}{1.144036in}}%
\pgfpathlineto{\pgfqpoint{2.552847in}{1.147706in}}%
\pgfpathlineto{\pgfqpoint{2.564765in}{1.151374in}}%
\pgfpathlineto{\pgfqpoint{2.576677in}{1.155038in}}%
\pgfpathlineto{\pgfqpoint{2.588583in}{1.158699in}}%
\pgfpathlineto{\pgfqpoint{2.600483in}{1.162354in}}%
\pgfpathlineto{\pgfqpoint{2.594148in}{1.174365in}}%
\pgfpathlineto{\pgfqpoint{2.587817in}{1.186359in}}%
\pgfpathlineto{\pgfqpoint{2.581489in}{1.198336in}}%
\pgfpathlineto{\pgfqpoint{2.575165in}{1.210299in}}%
\pgfpathlineto{\pgfqpoint{2.568845in}{1.222249in}}%
\pgfpathlineto{\pgfqpoint{2.556955in}{1.218475in}}%
\pgfpathlineto{\pgfqpoint{2.545058in}{1.214705in}}%
\pgfpathlineto{\pgfqpoint{2.533156in}{1.210938in}}%
\pgfpathlineto{\pgfqpoint{2.521247in}{1.207174in}}%
\pgfpathlineto{\pgfqpoint{2.509333in}{1.203411in}}%
\pgfpathlineto{\pgfqpoint{2.515644in}{1.191546in}}%
\pgfpathlineto{\pgfqpoint{2.521958in}{1.179677in}}%
\pgfpathlineto{\pgfqpoint{2.528276in}{1.167803in}}%
\pgfpathlineto{\pgfqpoint{2.534597in}{1.155922in}}%
\pgfpathclose%
\pgfusepath{stroke,fill}%
\end{pgfscope}%
\begin{pgfscope}%
\pgfpathrectangle{\pgfqpoint{0.887500in}{0.275000in}}{\pgfqpoint{4.225000in}{4.225000in}}%
\pgfusepath{clip}%
\pgfsetbuttcap%
\pgfsetroundjoin%
\definecolor{currentfill}{rgb}{0.282656,0.100196,0.422160}%
\pgfsetfillcolor{currentfill}%
\pgfsetfillopacity{0.700000}%
\pgfsetlinewidth{0.501875pt}%
\definecolor{currentstroke}{rgb}{1.000000,1.000000,1.000000}%
\pgfsetstrokecolor{currentstroke}%
\pgfsetstrokeopacity{0.500000}%
\pgfsetdash{}{0pt}%
\pgfpathmoveto{\pgfqpoint{2.965076in}{1.057075in}}%
\pgfpathlineto{\pgfqpoint{2.976891in}{1.063731in}}%
\pgfpathlineto{\pgfqpoint{2.988701in}{1.070458in}}%
\pgfpathlineto{\pgfqpoint{3.000506in}{1.077251in}}%
\pgfpathlineto{\pgfqpoint{3.012308in}{1.084109in}}%
\pgfpathlineto{\pgfqpoint{3.024105in}{1.091025in}}%
\pgfpathlineto{\pgfqpoint{3.017666in}{1.087267in}}%
\pgfpathlineto{\pgfqpoint{3.011233in}{1.085434in}}%
\pgfpathlineto{\pgfqpoint{3.004806in}{1.085412in}}%
\pgfpathlineto{\pgfqpoint{2.998384in}{1.087087in}}%
\pgfpathlineto{\pgfqpoint{2.991966in}{1.090348in}}%
\pgfpathlineto{\pgfqpoint{2.980192in}{1.086388in}}%
\pgfpathlineto{\pgfqpoint{2.968412in}{1.082514in}}%
\pgfpathlineto{\pgfqpoint{2.956626in}{1.078661in}}%
\pgfpathlineto{\pgfqpoint{2.944834in}{1.074767in}}%
\pgfpathlineto{\pgfqpoint{2.933036in}{1.070788in}}%
\pgfpathlineto{\pgfqpoint{2.939439in}{1.065537in}}%
\pgfpathlineto{\pgfqpoint{2.945844in}{1.061458in}}%
\pgfpathlineto{\pgfqpoint{2.952251in}{1.058634in}}%
\pgfpathlineto{\pgfqpoint{2.958662in}{1.057145in}}%
\pgfpathclose%
\pgfusepath{stroke,fill}%
\end{pgfscope}%
\begin{pgfscope}%
\pgfpathrectangle{\pgfqpoint{0.887500in}{0.275000in}}{\pgfqpoint{4.225000in}{4.225000in}}%
\pgfusepath{clip}%
\pgfsetbuttcap%
\pgfsetroundjoin%
\definecolor{currentfill}{rgb}{0.282623,0.140926,0.457517}%
\pgfsetfillcolor{currentfill}%
\pgfsetfillopacity{0.700000}%
\pgfsetlinewidth{0.501875pt}%
\definecolor{currentstroke}{rgb}{1.000000,1.000000,1.000000}%
\pgfsetstrokecolor{currentstroke}%
\pgfsetstrokeopacity{0.500000}%
\pgfsetdash{}{0pt}%
\pgfpathmoveto{\pgfqpoint{2.632203in}{1.102600in}}%
\pgfpathlineto{\pgfqpoint{2.644107in}{1.106094in}}%
\pgfpathlineto{\pgfqpoint{2.656005in}{1.109570in}}%
\pgfpathlineto{\pgfqpoint{2.667896in}{1.113029in}}%
\pgfpathlineto{\pgfqpoint{2.679781in}{1.116472in}}%
\pgfpathlineto{\pgfqpoint{2.691660in}{1.119898in}}%
\pgfpathlineto{\pgfqpoint{2.685300in}{1.131903in}}%
\pgfpathlineto{\pgfqpoint{2.678943in}{1.143996in}}%
\pgfpathlineto{\pgfqpoint{2.672589in}{1.156153in}}%
\pgfpathlineto{\pgfqpoint{2.666238in}{1.168348in}}%
\pgfpathlineto{\pgfqpoint{2.659890in}{1.180556in}}%
\pgfpathlineto{\pgfqpoint{2.648021in}{1.176921in}}%
\pgfpathlineto{\pgfqpoint{2.636145in}{1.173285in}}%
\pgfpathlineto{\pgfqpoint{2.624264in}{1.169646in}}%
\pgfpathlineto{\pgfqpoint{2.612377in}{1.166003in}}%
\pgfpathlineto{\pgfqpoint{2.600483in}{1.162354in}}%
\pgfpathlineto{\pgfqpoint{2.606821in}{1.150340in}}%
\pgfpathlineto{\pgfqpoint{2.613162in}{1.138341in}}%
\pgfpathlineto{\pgfqpoint{2.619507in}{1.126373in}}%
\pgfpathlineto{\pgfqpoint{2.625854in}{1.114453in}}%
\pgfpathclose%
\pgfusepath{stroke,fill}%
\end{pgfscope}%
\begin{pgfscope}%
\pgfpathrectangle{\pgfqpoint{0.887500in}{0.275000in}}{\pgfqpoint{4.225000in}{4.225000in}}%
\pgfusepath{clip}%
\pgfsetbuttcap%
\pgfsetroundjoin%
\definecolor{currentfill}{rgb}{0.283197,0.115680,0.436115}%
\pgfsetfillcolor{currentfill}%
\pgfsetfillopacity{0.700000}%
\pgfsetlinewidth{0.501875pt}%
\definecolor{currentstroke}{rgb}{1.000000,1.000000,1.000000}%
\pgfsetstrokecolor{currentstroke}%
\pgfsetstrokeopacity{0.500000}%
\pgfsetdash{}{0pt}%
\pgfpathmoveto{\pgfqpoint{2.723489in}{1.062102in}}%
\pgfpathlineto{\pgfqpoint{2.735370in}{1.065492in}}%
\pgfpathlineto{\pgfqpoint{2.747245in}{1.068865in}}%
\pgfpathlineto{\pgfqpoint{2.759114in}{1.072218in}}%
\pgfpathlineto{\pgfqpoint{2.770977in}{1.075554in}}%
\pgfpathlineto{\pgfqpoint{2.782833in}{1.078877in}}%
\pgfpathlineto{\pgfqpoint{2.776456in}{1.089902in}}%
\pgfpathlineto{\pgfqpoint{2.770080in}{1.101246in}}%
\pgfpathlineto{\pgfqpoint{2.763705in}{1.112870in}}%
\pgfpathlineto{\pgfqpoint{2.757333in}{1.124735in}}%
\pgfpathlineto{\pgfqpoint{2.750961in}{1.136800in}}%
\pgfpathlineto{\pgfqpoint{2.739114in}{1.133445in}}%
\pgfpathlineto{\pgfqpoint{2.727260in}{1.130080in}}%
\pgfpathlineto{\pgfqpoint{2.715399in}{1.126702in}}%
\pgfpathlineto{\pgfqpoint{2.703533in}{1.123308in}}%
\pgfpathlineto{\pgfqpoint{2.691660in}{1.119898in}}%
\pgfpathlineto{\pgfqpoint{2.698022in}{1.108007in}}%
\pgfpathlineto{\pgfqpoint{2.704386in}{1.096257in}}%
\pgfpathlineto{\pgfqpoint{2.710751in}{1.084672in}}%
\pgfpathlineto{\pgfqpoint{2.717119in}{1.073279in}}%
\pgfpathclose%
\pgfusepath{stroke,fill}%
\end{pgfscope}%
\begin{pgfscope}%
\pgfpathrectangle{\pgfqpoint{0.887500in}{0.275000in}}{\pgfqpoint{4.225000in}{4.225000in}}%
\pgfusepath{clip}%
\pgfsetbuttcap%
\pgfsetroundjoin%
\definecolor{currentfill}{rgb}{0.282327,0.094955,0.417331}%
\pgfsetfillcolor{currentfill}%
\pgfsetfillopacity{0.700000}%
\pgfsetlinewidth{0.501875pt}%
\definecolor{currentstroke}{rgb}{1.000000,1.000000,1.000000}%
\pgfsetstrokecolor{currentstroke}%
\pgfsetstrokeopacity{0.500000}%
\pgfsetdash{}{0pt}%
\pgfpathmoveto{\pgfqpoint{2.814737in}{1.029918in}}%
\pgfpathlineto{\pgfqpoint{2.826595in}{1.033908in}}%
\pgfpathlineto{\pgfqpoint{2.838445in}{1.037929in}}%
\pgfpathlineto{\pgfqpoint{2.850290in}{1.041983in}}%
\pgfpathlineto{\pgfqpoint{2.862129in}{1.046071in}}%
\pgfpathlineto{\pgfqpoint{2.873961in}{1.050195in}}%
\pgfpathlineto{\pgfqpoint{2.867571in}{1.057886in}}%
\pgfpathlineto{\pgfqpoint{2.861182in}{1.066319in}}%
\pgfpathlineto{\pgfqpoint{2.854793in}{1.075436in}}%
\pgfpathlineto{\pgfqpoint{2.848406in}{1.085179in}}%
\pgfpathlineto{\pgfqpoint{2.842020in}{1.095489in}}%
\pgfpathlineto{\pgfqpoint{2.830195in}{1.092151in}}%
\pgfpathlineto{\pgfqpoint{2.818364in}{1.088825in}}%
\pgfpathlineto{\pgfqpoint{2.806527in}{1.085508in}}%
\pgfpathlineto{\pgfqpoint{2.794683in}{1.082194in}}%
\pgfpathlineto{\pgfqpoint{2.782833in}{1.078877in}}%
\pgfpathlineto{\pgfqpoint{2.789212in}{1.068211in}}%
\pgfpathlineto{\pgfqpoint{2.795592in}{1.057942in}}%
\pgfpathlineto{\pgfqpoint{2.801973in}{1.048111in}}%
\pgfpathlineto{\pgfqpoint{2.808354in}{1.038756in}}%
\pgfpathclose%
\pgfusepath{stroke,fill}%
\end{pgfscope}%
\begin{pgfscope}%
\pgfpathrectangle{\pgfqpoint{0.887500in}{0.275000in}}{\pgfqpoint{4.225000in}{4.225000in}}%
\pgfusepath{clip}%
\pgfsetbuttcap%
\pgfsetroundjoin%
\definecolor{currentfill}{rgb}{0.270595,0.214069,0.507052}%
\pgfsetfillcolor{currentfill}%
\pgfsetfillopacity{0.700000}%
\pgfsetlinewidth{0.501875pt}%
\definecolor{currentstroke}{rgb}{1.000000,1.000000,1.000000}%
\pgfsetstrokecolor{currentstroke}%
\pgfsetstrokeopacity{0.500000}%
\pgfsetdash{}{0pt}%
\pgfpathmoveto{\pgfqpoint{2.298519in}{1.205154in}}%
\pgfpathlineto{\pgfqpoint{2.310515in}{1.208999in}}%
\pgfpathlineto{\pgfqpoint{2.322506in}{1.212842in}}%
\pgfpathlineto{\pgfqpoint{2.334491in}{1.216684in}}%
\pgfpathlineto{\pgfqpoint{2.346469in}{1.220524in}}%
\pgfpathlineto{\pgfqpoint{2.358441in}{1.224364in}}%
\pgfpathlineto{\pgfqpoint{2.352171in}{1.236082in}}%
\pgfpathlineto{\pgfqpoint{2.345903in}{1.247799in}}%
\pgfpathlineto{\pgfqpoint{2.339639in}{1.259514in}}%
\pgfpathlineto{\pgfqpoint{2.333379in}{1.271223in}}%
\pgfpathlineto{\pgfqpoint{2.327122in}{1.282927in}}%
\pgfpathlineto{\pgfqpoint{2.315160in}{1.279025in}}%
\pgfpathlineto{\pgfqpoint{2.303191in}{1.275124in}}%
\pgfpathlineto{\pgfqpoint{2.291217in}{1.271222in}}%
\pgfpathlineto{\pgfqpoint{2.279237in}{1.267321in}}%
\pgfpathlineto{\pgfqpoint{2.267251in}{1.263421in}}%
\pgfpathlineto{\pgfqpoint{2.273497in}{1.251774in}}%
\pgfpathlineto{\pgfqpoint{2.279747in}{1.240122in}}%
\pgfpathlineto{\pgfqpoint{2.286001in}{1.228467in}}%
\pgfpathlineto{\pgfqpoint{2.292258in}{1.216811in}}%
\pgfpathclose%
\pgfusepath{stroke,fill}%
\end{pgfscope}%
\begin{pgfscope}%
\pgfpathrectangle{\pgfqpoint{0.887500in}{0.275000in}}{\pgfqpoint{4.225000in}{4.225000in}}%
\pgfusepath{clip}%
\pgfsetbuttcap%
\pgfsetroundjoin%
\definecolor{currentfill}{rgb}{0.282327,0.094955,0.417331}%
\pgfsetfillcolor{currentfill}%
\pgfsetfillopacity{0.700000}%
\pgfsetlinewidth{0.501875pt}%
\definecolor{currentstroke}{rgb}{1.000000,1.000000,1.000000}%
\pgfsetstrokecolor{currentstroke}%
\pgfsetstrokeopacity{0.500000}%
\pgfsetdash{}{0pt}%
\pgfpathmoveto{\pgfqpoint{2.905935in}{1.024911in}}%
\pgfpathlineto{\pgfqpoint{2.917773in}{1.031190in}}%
\pgfpathlineto{\pgfqpoint{2.929606in}{1.037548in}}%
\pgfpathlineto{\pgfqpoint{2.941434in}{1.043982in}}%
\pgfpathlineto{\pgfqpoint{2.953257in}{1.050492in}}%
\pgfpathlineto{\pgfqpoint{2.965076in}{1.057075in}}%
\pgfpathlineto{\pgfqpoint{2.958662in}{1.057145in}}%
\pgfpathlineto{\pgfqpoint{2.952251in}{1.058634in}}%
\pgfpathlineto{\pgfqpoint{2.945844in}{1.061458in}}%
\pgfpathlineto{\pgfqpoint{2.939439in}{1.065537in}}%
\pgfpathlineto{\pgfqpoint{2.933036in}{1.070788in}}%
\pgfpathlineto{\pgfqpoint{2.921233in}{1.066737in}}%
\pgfpathlineto{\pgfqpoint{2.909424in}{1.062633in}}%
\pgfpathlineto{\pgfqpoint{2.897609in}{1.058495in}}%
\pgfpathlineto{\pgfqpoint{2.885788in}{1.054343in}}%
\pgfpathlineto{\pgfqpoint{2.873961in}{1.050195in}}%
\pgfpathlineto{\pgfqpoint{2.880353in}{1.043304in}}%
\pgfpathlineto{\pgfqpoint{2.886746in}{1.037272in}}%
\pgfpathlineto{\pgfqpoint{2.893140in}{1.032156in}}%
\pgfpathlineto{\pgfqpoint{2.899537in}{1.028016in}}%
\pgfpathclose%
\pgfusepath{stroke,fill}%
\end{pgfscope}%
\begin{pgfscope}%
\pgfpathrectangle{\pgfqpoint{0.887500in}{0.275000in}}{\pgfqpoint{4.225000in}{4.225000in}}%
\pgfusepath{clip}%
\pgfsetbuttcap%
\pgfsetroundjoin%
\definecolor{currentfill}{rgb}{0.276194,0.190074,0.493001}%
\pgfsetfillcolor{currentfill}%
\pgfsetfillopacity{0.700000}%
\pgfsetlinewidth{0.501875pt}%
\definecolor{currentstroke}{rgb}{1.000000,1.000000,1.000000}%
\pgfsetstrokecolor{currentstroke}%
\pgfsetstrokeopacity{0.500000}%
\pgfsetdash{}{0pt}%
\pgfpathmoveto{\pgfqpoint{2.389846in}{1.165777in}}%
\pgfpathlineto{\pgfqpoint{2.401822in}{1.169545in}}%
\pgfpathlineto{\pgfqpoint{2.413793in}{1.173312in}}%
\pgfpathlineto{\pgfqpoint{2.425757in}{1.177077in}}%
\pgfpathlineto{\pgfqpoint{2.437715in}{1.180841in}}%
\pgfpathlineto{\pgfqpoint{2.449666in}{1.184604in}}%
\pgfpathlineto{\pgfqpoint{2.443369in}{1.196392in}}%
\pgfpathlineto{\pgfqpoint{2.437074in}{1.208180in}}%
\pgfpathlineto{\pgfqpoint{2.430783in}{1.219968in}}%
\pgfpathlineto{\pgfqpoint{2.424495in}{1.231756in}}%
\pgfpathlineto{\pgfqpoint{2.418211in}{1.243542in}}%
\pgfpathlineto{\pgfqpoint{2.406269in}{1.239709in}}%
\pgfpathlineto{\pgfqpoint{2.394322in}{1.235874in}}%
\pgfpathlineto{\pgfqpoint{2.382368in}{1.232038in}}%
\pgfpathlineto{\pgfqpoint{2.370408in}{1.228202in}}%
\pgfpathlineto{\pgfqpoint{2.358441in}{1.224364in}}%
\pgfpathlineto{\pgfqpoint{2.364716in}{1.212644in}}%
\pgfpathlineto{\pgfqpoint{2.370993in}{1.200925in}}%
\pgfpathlineto{\pgfqpoint{2.377274in}{1.189207in}}%
\pgfpathlineto{\pgfqpoint{2.383558in}{1.177491in}}%
\pgfpathclose%
\pgfusepath{stroke,fill}%
\end{pgfscope}%
\begin{pgfscope}%
\pgfpathrectangle{\pgfqpoint{0.887500in}{0.275000in}}{\pgfqpoint{4.225000in}{4.225000in}}%
\pgfusepath{clip}%
\pgfsetbuttcap%
\pgfsetroundjoin%
\definecolor{currentfill}{rgb}{0.280255,0.165693,0.476498}%
\pgfsetfillcolor{currentfill}%
\pgfsetfillopacity{0.700000}%
\pgfsetlinewidth{0.501875pt}%
\definecolor{currentstroke}{rgb}{1.000000,1.000000,1.000000}%
\pgfsetstrokecolor{currentstroke}%
\pgfsetstrokeopacity{0.500000}%
\pgfsetdash{}{0pt}%
\pgfpathmoveto{\pgfqpoint{2.481206in}{1.125675in}}%
\pgfpathlineto{\pgfqpoint{2.493162in}{1.129348in}}%
\pgfpathlineto{\pgfqpoint{2.505111in}{1.133021in}}%
\pgfpathlineto{\pgfqpoint{2.517054in}{1.136693in}}%
\pgfpathlineto{\pgfqpoint{2.528991in}{1.140365in}}%
\pgfpathlineto{\pgfqpoint{2.540922in}{1.144036in}}%
\pgfpathlineto{\pgfqpoint{2.534597in}{1.155922in}}%
\pgfpathlineto{\pgfqpoint{2.528276in}{1.167803in}}%
\pgfpathlineto{\pgfqpoint{2.521958in}{1.179677in}}%
\pgfpathlineto{\pgfqpoint{2.515644in}{1.191546in}}%
\pgfpathlineto{\pgfqpoint{2.509333in}{1.203411in}}%
\pgfpathlineto{\pgfqpoint{2.497412in}{1.199650in}}%
\pgfpathlineto{\pgfqpoint{2.485485in}{1.195888in}}%
\pgfpathlineto{\pgfqpoint{2.473551in}{1.192127in}}%
\pgfpathlineto{\pgfqpoint{2.461612in}{1.188366in}}%
\pgfpathlineto{\pgfqpoint{2.449666in}{1.184604in}}%
\pgfpathlineto{\pgfqpoint{2.455968in}{1.172816in}}%
\pgfpathlineto{\pgfqpoint{2.462272in}{1.161028in}}%
\pgfpathlineto{\pgfqpoint{2.468580in}{1.149242in}}%
\pgfpathlineto{\pgfqpoint{2.474891in}{1.137457in}}%
\pgfpathclose%
\pgfusepath{stroke,fill}%
\end{pgfscope}%
\begin{pgfscope}%
\pgfpathrectangle{\pgfqpoint{0.887500in}{0.275000in}}{\pgfqpoint{4.225000in}{4.225000in}}%
\pgfusepath{clip}%
\pgfsetbuttcap%
\pgfsetroundjoin%
\definecolor{currentfill}{rgb}{0.282623,0.140926,0.457517}%
\pgfsetfillcolor{currentfill}%
\pgfsetfillopacity{0.700000}%
\pgfsetlinewidth{0.501875pt}%
\definecolor{currentstroke}{rgb}{1.000000,1.000000,1.000000}%
\pgfsetstrokecolor{currentstroke}%
\pgfsetstrokeopacity{0.500000}%
\pgfsetdash{}{0pt}%
\pgfpathmoveto{\pgfqpoint{2.572593in}{1.084934in}}%
\pgfpathlineto{\pgfqpoint{2.584528in}{1.088487in}}%
\pgfpathlineto{\pgfqpoint{2.596456in}{1.092032in}}%
\pgfpathlineto{\pgfqpoint{2.608378in}{1.095568in}}%
\pgfpathlineto{\pgfqpoint{2.620294in}{1.099091in}}%
\pgfpathlineto{\pgfqpoint{2.632203in}{1.102600in}}%
\pgfpathlineto{\pgfqpoint{2.625854in}{1.114453in}}%
\pgfpathlineto{\pgfqpoint{2.619507in}{1.126373in}}%
\pgfpathlineto{\pgfqpoint{2.613162in}{1.138341in}}%
\pgfpathlineto{\pgfqpoint{2.606821in}{1.150340in}}%
\pgfpathlineto{\pgfqpoint{2.600483in}{1.162354in}}%
\pgfpathlineto{\pgfqpoint{2.588583in}{1.158699in}}%
\pgfpathlineto{\pgfqpoint{2.576677in}{1.155038in}}%
\pgfpathlineto{\pgfqpoint{2.564765in}{1.151374in}}%
\pgfpathlineto{\pgfqpoint{2.552847in}{1.147706in}}%
\pgfpathlineto{\pgfqpoint{2.540922in}{1.144036in}}%
\pgfpathlineto{\pgfqpoint{2.547250in}{1.132156in}}%
\pgfpathlineto{\pgfqpoint{2.553582in}{1.120293in}}%
\pgfpathlineto{\pgfqpoint{2.559916in}{1.108459in}}%
\pgfpathlineto{\pgfqpoint{2.566253in}{1.096669in}}%
\pgfpathclose%
\pgfusepath{stroke,fill}%
\end{pgfscope}%
\begin{pgfscope}%
\pgfpathrectangle{\pgfqpoint{0.887500in}{0.275000in}}{\pgfqpoint{4.225000in}{4.225000in}}%
\pgfusepath{clip}%
\pgfsetbuttcap%
\pgfsetroundjoin%
\definecolor{currentfill}{rgb}{0.283197,0.115680,0.436115}%
\pgfsetfillcolor{currentfill}%
\pgfsetfillopacity{0.700000}%
\pgfsetlinewidth{0.501875pt}%
\definecolor{currentstroke}{rgb}{1.000000,1.000000,1.000000}%
\pgfsetstrokecolor{currentstroke}%
\pgfsetstrokeopacity{0.500000}%
\pgfsetdash{}{0pt}%
\pgfpathmoveto{\pgfqpoint{2.663988in}{1.044933in}}%
\pgfpathlineto{\pgfqpoint{2.675900in}{1.048390in}}%
\pgfpathlineto{\pgfqpoint{2.687807in}{1.051837in}}%
\pgfpathlineto{\pgfqpoint{2.699707in}{1.055272in}}%
\pgfpathlineto{\pgfqpoint{2.711601in}{1.058694in}}%
\pgfpathlineto{\pgfqpoint{2.723489in}{1.062102in}}%
\pgfpathlineto{\pgfqpoint{2.717119in}{1.073279in}}%
\pgfpathlineto{\pgfqpoint{2.710751in}{1.084672in}}%
\pgfpathlineto{\pgfqpoint{2.704386in}{1.096257in}}%
\pgfpathlineto{\pgfqpoint{2.698022in}{1.108007in}}%
\pgfpathlineto{\pgfqpoint{2.691660in}{1.119898in}}%
\pgfpathlineto{\pgfqpoint{2.679781in}{1.116472in}}%
\pgfpathlineto{\pgfqpoint{2.667896in}{1.113029in}}%
\pgfpathlineto{\pgfqpoint{2.656005in}{1.109570in}}%
\pgfpathlineto{\pgfqpoint{2.644107in}{1.106094in}}%
\pgfpathlineto{\pgfqpoint{2.632203in}{1.102600in}}%
\pgfpathlineto{\pgfqpoint{2.638556in}{1.090831in}}%
\pgfpathlineto{\pgfqpoint{2.644911in}{1.079162in}}%
\pgfpathlineto{\pgfqpoint{2.651268in}{1.067611in}}%
\pgfpathlineto{\pgfqpoint{2.657627in}{1.056196in}}%
\pgfpathclose%
\pgfusepath{stroke,fill}%
\end{pgfscope}%
\begin{pgfscope}%
\pgfpathrectangle{\pgfqpoint{0.887500in}{0.275000in}}{\pgfqpoint{4.225000in}{4.225000in}}%
\pgfusepath{clip}%
\pgfsetbuttcap%
\pgfsetroundjoin%
\definecolor{currentfill}{rgb}{0.282327,0.094955,0.417331}%
\pgfsetfillcolor{currentfill}%
\pgfsetfillopacity{0.700000}%
\pgfsetlinewidth{0.501875pt}%
\definecolor{currentstroke}{rgb}{1.000000,1.000000,1.000000}%
\pgfsetstrokecolor{currentstroke}%
\pgfsetstrokeopacity{0.500000}%
\pgfsetdash{}{0pt}%
\pgfpathmoveto{\pgfqpoint{2.755359in}{1.010363in}}%
\pgfpathlineto{\pgfqpoint{2.767247in}{1.014223in}}%
\pgfpathlineto{\pgfqpoint{2.779129in}{1.018110in}}%
\pgfpathlineto{\pgfqpoint{2.791005in}{1.022021in}}%
\pgfpathlineto{\pgfqpoint{2.802874in}{1.025956in}}%
\pgfpathlineto{\pgfqpoint{2.814737in}{1.029918in}}%
\pgfpathlineto{\pgfqpoint{2.808354in}{1.038756in}}%
\pgfpathlineto{\pgfqpoint{2.801973in}{1.048111in}}%
\pgfpathlineto{\pgfqpoint{2.795592in}{1.057942in}}%
\pgfpathlineto{\pgfqpoint{2.789212in}{1.068211in}}%
\pgfpathlineto{\pgfqpoint{2.782833in}{1.078877in}}%
\pgfpathlineto{\pgfqpoint{2.770977in}{1.075554in}}%
\pgfpathlineto{\pgfqpoint{2.759114in}{1.072218in}}%
\pgfpathlineto{\pgfqpoint{2.747245in}{1.068865in}}%
\pgfpathlineto{\pgfqpoint{2.735370in}{1.065492in}}%
\pgfpathlineto{\pgfqpoint{2.723489in}{1.062102in}}%
\pgfpathlineto{\pgfqpoint{2.729860in}{1.051167in}}%
\pgfpathlineto{\pgfqpoint{2.736232in}{1.040500in}}%
\pgfpathlineto{\pgfqpoint{2.742606in}{1.030127in}}%
\pgfpathlineto{\pgfqpoint{2.748982in}{1.020073in}}%
\pgfpathclose%
\pgfusepath{stroke,fill}%
\end{pgfscope}%
\begin{pgfscope}%
\pgfpathrectangle{\pgfqpoint{0.887500in}{0.275000in}}{\pgfqpoint{4.225000in}{4.225000in}}%
\pgfusepath{clip}%
\pgfsetbuttcap%
\pgfsetroundjoin%
\definecolor{currentfill}{rgb}{0.281924,0.089666,0.412415}%
\pgfsetfillcolor{currentfill}%
\pgfsetfillopacity{0.700000}%
\pgfsetlinewidth{0.501875pt}%
\definecolor{currentstroke}{rgb}{1.000000,1.000000,1.000000}%
\pgfsetstrokecolor{currentstroke}%
\pgfsetstrokeopacity{0.500000}%
\pgfsetdash{}{0pt}%
\pgfpathmoveto{\pgfqpoint{2.846669in}{0.994860in}}%
\pgfpathlineto{\pgfqpoint{2.858533in}{1.000662in}}%
\pgfpathlineto{\pgfqpoint{2.870391in}{1.006578in}}%
\pgfpathlineto{\pgfqpoint{2.882244in}{1.012598in}}%
\pgfpathlineto{\pgfqpoint{2.894092in}{1.018712in}}%
\pgfpathlineto{\pgfqpoint{2.905935in}{1.024911in}}%
\pgfpathlineto{\pgfqpoint{2.899537in}{1.028016in}}%
\pgfpathlineto{\pgfqpoint{2.893140in}{1.032156in}}%
\pgfpathlineto{\pgfqpoint{2.886746in}{1.037272in}}%
\pgfpathlineto{\pgfqpoint{2.880353in}{1.043304in}}%
\pgfpathlineto{\pgfqpoint{2.873961in}{1.050195in}}%
\pgfpathlineto{\pgfqpoint{2.862129in}{1.046071in}}%
\pgfpathlineto{\pgfqpoint{2.850290in}{1.041983in}}%
\pgfpathlineto{\pgfqpoint{2.838445in}{1.037929in}}%
\pgfpathlineto{\pgfqpoint{2.826595in}{1.033908in}}%
\pgfpathlineto{\pgfqpoint{2.814737in}{1.029918in}}%
\pgfpathlineto{\pgfqpoint{2.821121in}{1.021636in}}%
\pgfpathlineto{\pgfqpoint{2.827507in}{1.013949in}}%
\pgfpathlineto{\pgfqpoint{2.833893in}{1.006897in}}%
\pgfpathlineto{\pgfqpoint{2.840280in}{1.000521in}}%
\pgfpathclose%
\pgfusepath{stroke,fill}%
\end{pgfscope}%
\begin{pgfscope}%
\pgfpathrectangle{\pgfqpoint{0.887500in}{0.275000in}}{\pgfqpoint{4.225000in}{4.225000in}}%
\pgfusepath{clip}%
\pgfsetbuttcap%
\pgfsetroundjoin%
\definecolor{currentfill}{rgb}{0.276194,0.190074,0.493001}%
\pgfsetfillcolor{currentfill}%
\pgfsetfillopacity{0.700000}%
\pgfsetlinewidth{0.501875pt}%
\definecolor{currentstroke}{rgb}{1.000000,1.000000,1.000000}%
\pgfsetstrokecolor{currentstroke}%
\pgfsetstrokeopacity{0.500000}%
\pgfsetdash{}{0pt}%
\pgfpathmoveto{\pgfqpoint{2.329871in}{1.146903in}}%
\pgfpathlineto{\pgfqpoint{2.341879in}{1.150683in}}%
\pgfpathlineto{\pgfqpoint{2.353880in}{1.154460in}}%
\pgfpathlineto{\pgfqpoint{2.365875in}{1.158235in}}%
\pgfpathlineto{\pgfqpoint{2.377863in}{1.162007in}}%
\pgfpathlineto{\pgfqpoint{2.389846in}{1.165777in}}%
\pgfpathlineto{\pgfqpoint{2.383558in}{1.177491in}}%
\pgfpathlineto{\pgfqpoint{2.377274in}{1.189207in}}%
\pgfpathlineto{\pgfqpoint{2.370993in}{1.200925in}}%
\pgfpathlineto{\pgfqpoint{2.364716in}{1.212644in}}%
\pgfpathlineto{\pgfqpoint{2.358441in}{1.224364in}}%
\pgfpathlineto{\pgfqpoint{2.346469in}{1.220524in}}%
\pgfpathlineto{\pgfqpoint{2.334491in}{1.216684in}}%
\pgfpathlineto{\pgfqpoint{2.322506in}{1.212842in}}%
\pgfpathlineto{\pgfqpoint{2.310515in}{1.208999in}}%
\pgfpathlineto{\pgfqpoint{2.298519in}{1.205154in}}%
\pgfpathlineto{\pgfqpoint{2.304782in}{1.193499in}}%
\pgfpathlineto{\pgfqpoint{2.311050in}{1.181845in}}%
\pgfpathlineto{\pgfqpoint{2.317320in}{1.170194in}}%
\pgfpathlineto{\pgfqpoint{2.323594in}{1.158546in}}%
\pgfpathclose%
\pgfusepath{stroke,fill}%
\end{pgfscope}%
\begin{pgfscope}%
\pgfpathrectangle{\pgfqpoint{0.887500in}{0.275000in}}{\pgfqpoint{4.225000in}{4.225000in}}%
\pgfusepath{clip}%
\pgfsetbuttcap%
\pgfsetroundjoin%
\definecolor{currentfill}{rgb}{0.280255,0.165693,0.476498}%
\pgfsetfillcolor{currentfill}%
\pgfsetfillopacity{0.700000}%
\pgfsetlinewidth{0.501875pt}%
\definecolor{currentstroke}{rgb}{1.000000,1.000000,1.000000}%
\pgfsetstrokecolor{currentstroke}%
\pgfsetstrokeopacity{0.500000}%
\pgfsetdash{}{0pt}%
\pgfpathmoveto{\pgfqpoint{2.421334in}{1.107275in}}%
\pgfpathlineto{\pgfqpoint{2.433321in}{1.110960in}}%
\pgfpathlineto{\pgfqpoint{2.445302in}{1.114642in}}%
\pgfpathlineto{\pgfqpoint{2.457276in}{1.118322in}}%
\pgfpathlineto{\pgfqpoint{2.469244in}{1.121999in}}%
\pgfpathlineto{\pgfqpoint{2.481206in}{1.125675in}}%
\pgfpathlineto{\pgfqpoint{2.474891in}{1.137457in}}%
\pgfpathlineto{\pgfqpoint{2.468580in}{1.149242in}}%
\pgfpathlineto{\pgfqpoint{2.462272in}{1.161028in}}%
\pgfpathlineto{\pgfqpoint{2.455968in}{1.172816in}}%
\pgfpathlineto{\pgfqpoint{2.449666in}{1.184604in}}%
\pgfpathlineto{\pgfqpoint{2.437715in}{1.180841in}}%
\pgfpathlineto{\pgfqpoint{2.425757in}{1.177077in}}%
\pgfpathlineto{\pgfqpoint{2.413793in}{1.173312in}}%
\pgfpathlineto{\pgfqpoint{2.401822in}{1.169545in}}%
\pgfpathlineto{\pgfqpoint{2.389846in}{1.165777in}}%
\pgfpathlineto{\pgfqpoint{2.396137in}{1.154067in}}%
\pgfpathlineto{\pgfqpoint{2.402431in}{1.142360in}}%
\pgfpathlineto{\pgfqpoint{2.408729in}{1.130659in}}%
\pgfpathlineto{\pgfqpoint{2.415030in}{1.118963in}}%
\pgfpathclose%
\pgfusepath{stroke,fill}%
\end{pgfscope}%
\begin{pgfscope}%
\pgfpathrectangle{\pgfqpoint{0.887500in}{0.275000in}}{\pgfqpoint{4.225000in}{4.225000in}}%
\pgfusepath{clip}%
\pgfsetbuttcap%
\pgfsetroundjoin%
\definecolor{currentfill}{rgb}{0.282623,0.140926,0.457517}%
\pgfsetfillcolor{currentfill}%
\pgfsetfillopacity{0.700000}%
\pgfsetlinewidth{0.501875pt}%
\definecolor{currentstroke}{rgb}{1.000000,1.000000,1.000000}%
\pgfsetstrokecolor{currentstroke}%
\pgfsetstrokeopacity{0.500000}%
\pgfsetdash{}{0pt}%
\pgfpathmoveto{\pgfqpoint{2.512826in}{1.067101in}}%
\pgfpathlineto{\pgfqpoint{2.524792in}{1.070675in}}%
\pgfpathlineto{\pgfqpoint{2.536752in}{1.074245in}}%
\pgfpathlineto{\pgfqpoint{2.548705in}{1.077812in}}%
\pgfpathlineto{\pgfqpoint{2.560652in}{1.081375in}}%
\pgfpathlineto{\pgfqpoint{2.572593in}{1.084934in}}%
\pgfpathlineto{\pgfqpoint{2.566253in}{1.096669in}}%
\pgfpathlineto{\pgfqpoint{2.559916in}{1.108459in}}%
\pgfpathlineto{\pgfqpoint{2.553582in}{1.120293in}}%
\pgfpathlineto{\pgfqpoint{2.547250in}{1.132156in}}%
\pgfpathlineto{\pgfqpoint{2.540922in}{1.144036in}}%
\pgfpathlineto{\pgfqpoint{2.528991in}{1.140365in}}%
\pgfpathlineto{\pgfqpoint{2.517054in}{1.136693in}}%
\pgfpathlineto{\pgfqpoint{2.505111in}{1.133021in}}%
\pgfpathlineto{\pgfqpoint{2.493162in}{1.129348in}}%
\pgfpathlineto{\pgfqpoint{2.481206in}{1.125675in}}%
\pgfpathlineto{\pgfqpoint{2.487524in}{1.113903in}}%
\pgfpathlineto{\pgfqpoint{2.493845in}{1.102150in}}%
\pgfpathlineto{\pgfqpoint{2.500169in}{1.090426in}}%
\pgfpathlineto{\pgfqpoint{2.506497in}{1.078740in}}%
\pgfpathclose%
\pgfusepath{stroke,fill}%
\end{pgfscope}%
\begin{pgfscope}%
\pgfpathrectangle{\pgfqpoint{0.887500in}{0.275000in}}{\pgfqpoint{4.225000in}{4.225000in}}%
\pgfusepath{clip}%
\pgfsetbuttcap%
\pgfsetroundjoin%
\definecolor{currentfill}{rgb}{0.283229,0.120777,0.440584}%
\pgfsetfillcolor{currentfill}%
\pgfsetfillopacity{0.700000}%
\pgfsetlinewidth{0.501875pt}%
\definecolor{currentstroke}{rgb}{1.000000,1.000000,1.000000}%
\pgfsetstrokecolor{currentstroke}%
\pgfsetstrokeopacity{0.500000}%
\pgfsetdash{}{0pt}%
\pgfpathmoveto{\pgfqpoint{2.604330in}{1.027521in}}%
\pgfpathlineto{\pgfqpoint{2.616274in}{1.031019in}}%
\pgfpathlineto{\pgfqpoint{2.628212in}{1.034510in}}%
\pgfpathlineto{\pgfqpoint{2.640143in}{1.037992in}}%
\pgfpathlineto{\pgfqpoint{2.652069in}{1.041467in}}%
\pgfpathlineto{\pgfqpoint{2.663988in}{1.044933in}}%
\pgfpathlineto{\pgfqpoint{2.657627in}{1.056196in}}%
\pgfpathlineto{\pgfqpoint{2.651268in}{1.067611in}}%
\pgfpathlineto{\pgfqpoint{2.644911in}{1.079162in}}%
\pgfpathlineto{\pgfqpoint{2.638556in}{1.090831in}}%
\pgfpathlineto{\pgfqpoint{2.632203in}{1.102600in}}%
\pgfpathlineto{\pgfqpoint{2.620294in}{1.099091in}}%
\pgfpathlineto{\pgfqpoint{2.608378in}{1.095568in}}%
\pgfpathlineto{\pgfqpoint{2.596456in}{1.092032in}}%
\pgfpathlineto{\pgfqpoint{2.584528in}{1.088487in}}%
\pgfpathlineto{\pgfqpoint{2.572593in}{1.084934in}}%
\pgfpathlineto{\pgfqpoint{2.578936in}{1.073266in}}%
\pgfpathlineto{\pgfqpoint{2.585281in}{1.061678in}}%
\pgfpathlineto{\pgfqpoint{2.591628in}{1.050183in}}%
\pgfpathlineto{\pgfqpoint{2.597978in}{1.038793in}}%
\pgfpathclose%
\pgfusepath{stroke,fill}%
\end{pgfscope}%
\begin{pgfscope}%
\pgfpathrectangle{\pgfqpoint{0.887500in}{0.275000in}}{\pgfqpoint{4.225000in}{4.225000in}}%
\pgfusepath{clip}%
\pgfsetbuttcap%
\pgfsetroundjoin%
\definecolor{currentfill}{rgb}{0.282327,0.094955,0.417331}%
\pgfsetfillcolor{currentfill}%
\pgfsetfillopacity{0.700000}%
\pgfsetlinewidth{0.501875pt}%
\definecolor{currentstroke}{rgb}{1.000000,1.000000,1.000000}%
\pgfsetstrokecolor{currentstroke}%
\pgfsetstrokeopacity{0.500000}%
\pgfsetdash{}{0pt}%
\pgfpathmoveto{\pgfqpoint{2.695821in}{0.991515in}}%
\pgfpathlineto{\pgfqpoint{2.707742in}{0.995218in}}%
\pgfpathlineto{\pgfqpoint{2.719656in}{0.998956in}}%
\pgfpathlineto{\pgfqpoint{2.731563in}{1.002727in}}%
\pgfpathlineto{\pgfqpoint{2.743464in}{1.006530in}}%
\pgfpathlineto{\pgfqpoint{2.755359in}{1.010363in}}%
\pgfpathlineto{\pgfqpoint{2.748982in}{1.020073in}}%
\pgfpathlineto{\pgfqpoint{2.742606in}{1.030127in}}%
\pgfpathlineto{\pgfqpoint{2.736232in}{1.040500in}}%
\pgfpathlineto{\pgfqpoint{2.729860in}{1.051167in}}%
\pgfpathlineto{\pgfqpoint{2.723489in}{1.062102in}}%
\pgfpathlineto{\pgfqpoint{2.711601in}{1.058694in}}%
\pgfpathlineto{\pgfqpoint{2.699707in}{1.055272in}}%
\pgfpathlineto{\pgfqpoint{2.687807in}{1.051837in}}%
\pgfpathlineto{\pgfqpoint{2.675900in}{1.048390in}}%
\pgfpathlineto{\pgfqpoint{2.663988in}{1.044933in}}%
\pgfpathlineto{\pgfqpoint{2.670351in}{1.033840in}}%
\pgfpathlineto{\pgfqpoint{2.676716in}{1.022935in}}%
\pgfpathlineto{\pgfqpoint{2.683082in}{1.012234in}}%
\pgfpathlineto{\pgfqpoint{2.689451in}{1.001755in}}%
\pgfpathclose%
\pgfusepath{stroke,fill}%
\end{pgfscope}%
\begin{pgfscope}%
\pgfpathrectangle{\pgfqpoint{0.887500in}{0.275000in}}{\pgfqpoint{4.225000in}{4.225000in}}%
\pgfusepath{clip}%
\pgfsetbuttcap%
\pgfsetroundjoin%
\definecolor{currentfill}{rgb}{0.281446,0.084320,0.407414}%
\pgfsetfillcolor{currentfill}%
\pgfsetfillopacity{0.700000}%
\pgfsetlinewidth{0.501875pt}%
\definecolor{currentstroke}{rgb}{1.000000,1.000000,1.000000}%
\pgfsetstrokecolor{currentstroke}%
\pgfsetstrokeopacity{0.500000}%
\pgfsetdash{}{0pt}%
\pgfpathmoveto{\pgfqpoint{2.787263in}{0.967884in}}%
\pgfpathlineto{\pgfqpoint{2.799157in}{0.972988in}}%
\pgfpathlineto{\pgfqpoint{2.811044in}{0.978239in}}%
\pgfpathlineto{\pgfqpoint{2.822925in}{0.983638in}}%
\pgfpathlineto{\pgfqpoint{2.834800in}{0.989182in}}%
\pgfpathlineto{\pgfqpoint{2.846669in}{0.994860in}}%
\pgfpathlineto{\pgfqpoint{2.840280in}{1.000521in}}%
\pgfpathlineto{\pgfqpoint{2.833893in}{1.006897in}}%
\pgfpathlineto{\pgfqpoint{2.827507in}{1.013949in}}%
\pgfpathlineto{\pgfqpoint{2.821121in}{1.021636in}}%
\pgfpathlineto{\pgfqpoint{2.814737in}{1.029918in}}%
\pgfpathlineto{\pgfqpoint{2.802874in}{1.025956in}}%
\pgfpathlineto{\pgfqpoint{2.791005in}{1.022021in}}%
\pgfpathlineto{\pgfqpoint{2.779129in}{1.018110in}}%
\pgfpathlineto{\pgfqpoint{2.767247in}{1.014223in}}%
\pgfpathlineto{\pgfqpoint{2.755359in}{1.010363in}}%
\pgfpathlineto{\pgfqpoint{2.761737in}{1.001024in}}%
\pgfpathlineto{\pgfqpoint{2.768116in}{0.992080in}}%
\pgfpathlineto{\pgfqpoint{2.774497in}{0.983559in}}%
\pgfpathlineto{\pgfqpoint{2.780879in}{0.975485in}}%
\pgfpathclose%
\pgfusepath{stroke,fill}%
\end{pgfscope}%
\begin{pgfscope}%
\pgfpathrectangle{\pgfqpoint{0.887500in}{0.275000in}}{\pgfqpoint{4.225000in}{4.225000in}}%
\pgfusepath{clip}%
\pgfsetbuttcap%
\pgfsetroundjoin%
\definecolor{currentfill}{rgb}{0.280255,0.165693,0.476498}%
\pgfsetfillcolor{currentfill}%
\pgfsetfillopacity{0.700000}%
\pgfsetlinewidth{0.501875pt}%
\definecolor{currentstroke}{rgb}{1.000000,1.000000,1.000000}%
\pgfsetstrokecolor{currentstroke}%
\pgfsetstrokeopacity{0.500000}%
\pgfsetdash{}{0pt}%
\pgfpathmoveto{\pgfqpoint{2.361307in}{1.088789in}}%
\pgfpathlineto{\pgfqpoint{2.373325in}{1.092496in}}%
\pgfpathlineto{\pgfqpoint{2.385337in}{1.096197in}}%
\pgfpathlineto{\pgfqpoint{2.397342in}{1.099894in}}%
\pgfpathlineto{\pgfqpoint{2.409341in}{1.103586in}}%
\pgfpathlineto{\pgfqpoint{2.421334in}{1.107275in}}%
\pgfpathlineto{\pgfqpoint{2.415030in}{1.118963in}}%
\pgfpathlineto{\pgfqpoint{2.408729in}{1.130659in}}%
\pgfpathlineto{\pgfqpoint{2.402431in}{1.142360in}}%
\pgfpathlineto{\pgfqpoint{2.396137in}{1.154067in}}%
\pgfpathlineto{\pgfqpoint{2.389846in}{1.165777in}}%
\pgfpathlineto{\pgfqpoint{2.377863in}{1.162007in}}%
\pgfpathlineto{\pgfqpoint{2.365875in}{1.158235in}}%
\pgfpathlineto{\pgfqpoint{2.353880in}{1.154460in}}%
\pgfpathlineto{\pgfqpoint{2.341879in}{1.150683in}}%
\pgfpathlineto{\pgfqpoint{2.329871in}{1.146903in}}%
\pgfpathlineto{\pgfqpoint{2.336152in}{1.135266in}}%
\pgfpathlineto{\pgfqpoint{2.342436in}{1.123634in}}%
\pgfpathlineto{\pgfqpoint{2.348723in}{1.112011in}}%
\pgfpathlineto{\pgfqpoint{2.355013in}{1.100395in}}%
\pgfpathclose%
\pgfusepath{stroke,fill}%
\end{pgfscope}%
\begin{pgfscope}%
\pgfpathrectangle{\pgfqpoint{0.887500in}{0.275000in}}{\pgfqpoint{4.225000in}{4.225000in}}%
\pgfusepath{clip}%
\pgfsetbuttcap%
\pgfsetroundjoin%
\definecolor{currentfill}{rgb}{0.282290,0.145912,0.461510}%
\pgfsetfillcolor{currentfill}%
\pgfsetfillopacity{0.700000}%
\pgfsetlinewidth{0.501875pt}%
\definecolor{currentstroke}{rgb}{1.000000,1.000000,1.000000}%
\pgfsetstrokecolor{currentstroke}%
\pgfsetstrokeopacity{0.500000}%
\pgfsetdash{}{0pt}%
\pgfpathmoveto{\pgfqpoint{2.452903in}{1.049160in}}%
\pgfpathlineto{\pgfqpoint{2.464900in}{1.052759in}}%
\pgfpathlineto{\pgfqpoint{2.476891in}{1.056352in}}%
\pgfpathlineto{\pgfqpoint{2.488876in}{1.059940in}}%
\pgfpathlineto{\pgfqpoint{2.500854in}{1.063523in}}%
\pgfpathlineto{\pgfqpoint{2.512826in}{1.067101in}}%
\pgfpathlineto{\pgfqpoint{2.506497in}{1.078740in}}%
\pgfpathlineto{\pgfqpoint{2.500169in}{1.090426in}}%
\pgfpathlineto{\pgfqpoint{2.493845in}{1.102150in}}%
\pgfpathlineto{\pgfqpoint{2.487524in}{1.113903in}}%
\pgfpathlineto{\pgfqpoint{2.481206in}{1.125675in}}%
\pgfpathlineto{\pgfqpoint{2.469244in}{1.121999in}}%
\pgfpathlineto{\pgfqpoint{2.457276in}{1.118322in}}%
\pgfpathlineto{\pgfqpoint{2.445302in}{1.114642in}}%
\pgfpathlineto{\pgfqpoint{2.433321in}{1.110960in}}%
\pgfpathlineto{\pgfqpoint{2.421334in}{1.107275in}}%
\pgfpathlineto{\pgfqpoint{2.427642in}{1.095600in}}%
\pgfpathlineto{\pgfqpoint{2.433953in}{1.083944in}}%
\pgfpathlineto{\pgfqpoint{2.440266in}{1.072314in}}%
\pgfpathlineto{\pgfqpoint{2.446583in}{1.060717in}}%
\pgfpathclose%
\pgfusepath{stroke,fill}%
\end{pgfscope}%
\begin{pgfscope}%
\pgfpathrectangle{\pgfqpoint{0.887500in}{0.275000in}}{\pgfqpoint{4.225000in}{4.225000in}}%
\pgfusepath{clip}%
\pgfsetbuttcap%
\pgfsetroundjoin%
\definecolor{currentfill}{rgb}{0.283229,0.120777,0.440584}%
\pgfsetfillcolor{currentfill}%
\pgfsetfillopacity{0.700000}%
\pgfsetlinewidth{0.501875pt}%
\definecolor{currentstroke}{rgb}{1.000000,1.000000,1.000000}%
\pgfsetstrokecolor{currentstroke}%
\pgfsetstrokeopacity{0.500000}%
\pgfsetdash{}{0pt}%
\pgfpathmoveto{\pgfqpoint{2.544515in}{1.009934in}}%
\pgfpathlineto{\pgfqpoint{2.556490in}{1.013464in}}%
\pgfpathlineto{\pgfqpoint{2.568460in}{1.016988in}}%
\pgfpathlineto{\pgfqpoint{2.580423in}{1.020505in}}%
\pgfpathlineto{\pgfqpoint{2.592379in}{1.024016in}}%
\pgfpathlineto{\pgfqpoint{2.604330in}{1.027521in}}%
\pgfpathlineto{\pgfqpoint{2.597978in}{1.038793in}}%
\pgfpathlineto{\pgfqpoint{2.591628in}{1.050183in}}%
\pgfpathlineto{\pgfqpoint{2.585281in}{1.061678in}}%
\pgfpathlineto{\pgfqpoint{2.578936in}{1.073266in}}%
\pgfpathlineto{\pgfqpoint{2.572593in}{1.084934in}}%
\pgfpathlineto{\pgfqpoint{2.560652in}{1.081375in}}%
\pgfpathlineto{\pgfqpoint{2.548705in}{1.077812in}}%
\pgfpathlineto{\pgfqpoint{2.536752in}{1.074245in}}%
\pgfpathlineto{\pgfqpoint{2.524792in}{1.070675in}}%
\pgfpathlineto{\pgfqpoint{2.512826in}{1.067101in}}%
\pgfpathlineto{\pgfqpoint{2.519159in}{1.055519in}}%
\pgfpathlineto{\pgfqpoint{2.525494in}{1.044001in}}%
\pgfpathlineto{\pgfqpoint{2.531832in}{1.032559in}}%
\pgfpathlineto{\pgfqpoint{2.538172in}{1.021200in}}%
\pgfpathclose%
\pgfusepath{stroke,fill}%
\end{pgfscope}%
\begin{pgfscope}%
\pgfpathrectangle{\pgfqpoint{0.887500in}{0.275000in}}{\pgfqpoint{4.225000in}{4.225000in}}%
\pgfusepath{clip}%
\pgfsetbuttcap%
\pgfsetroundjoin%
\definecolor{currentfill}{rgb}{0.282327,0.094955,0.417331}%
\pgfsetfillcolor{currentfill}%
\pgfsetfillopacity{0.700000}%
\pgfsetlinewidth{0.501875pt}%
\definecolor{currentstroke}{rgb}{1.000000,1.000000,1.000000}%
\pgfsetstrokecolor{currentstroke}%
\pgfsetstrokeopacity{0.500000}%
\pgfsetdash{}{0pt}%
\pgfpathmoveto{\pgfqpoint{2.636120in}{0.973364in}}%
\pgfpathlineto{\pgfqpoint{2.648074in}{0.976966in}}%
\pgfpathlineto{\pgfqpoint{2.660020in}{0.980576in}}%
\pgfpathlineto{\pgfqpoint{2.671960in}{0.984200in}}%
\pgfpathlineto{\pgfqpoint{2.683894in}{0.987844in}}%
\pgfpathlineto{\pgfqpoint{2.695821in}{0.991515in}}%
\pgfpathlineto{\pgfqpoint{2.689451in}{1.001755in}}%
\pgfpathlineto{\pgfqpoint{2.683082in}{1.012234in}}%
\pgfpathlineto{\pgfqpoint{2.676716in}{1.022935in}}%
\pgfpathlineto{\pgfqpoint{2.670351in}{1.033840in}}%
\pgfpathlineto{\pgfqpoint{2.663988in}{1.044933in}}%
\pgfpathlineto{\pgfqpoint{2.652069in}{1.041467in}}%
\pgfpathlineto{\pgfqpoint{2.640143in}{1.037992in}}%
\pgfpathlineto{\pgfqpoint{2.628212in}{1.034510in}}%
\pgfpathlineto{\pgfqpoint{2.616274in}{1.031019in}}%
\pgfpathlineto{\pgfqpoint{2.604330in}{1.027521in}}%
\pgfpathlineto{\pgfqpoint{2.610684in}{1.016379in}}%
\pgfpathlineto{\pgfqpoint{2.617040in}{1.005380in}}%
\pgfpathlineto{\pgfqpoint{2.623398in}{0.994536in}}%
\pgfpathlineto{\pgfqpoint{2.629759in}{0.983860in}}%
\pgfpathclose%
\pgfusepath{stroke,fill}%
\end{pgfscope}%
\begin{pgfscope}%
\pgfpathrectangle{\pgfqpoint{0.887500in}{0.275000in}}{\pgfqpoint{4.225000in}{4.225000in}}%
\pgfusepath{clip}%
\pgfsetbuttcap%
\pgfsetroundjoin%
\definecolor{currentfill}{rgb}{0.280894,0.078907,0.402329}%
\pgfsetfillcolor{currentfill}%
\pgfsetfillopacity{0.700000}%
\pgfsetlinewidth{0.501875pt}%
\definecolor{currentstroke}{rgb}{1.000000,1.000000,1.000000}%
\pgfsetstrokecolor{currentstroke}%
\pgfsetstrokeopacity{0.500000}%
\pgfsetdash{}{0pt}%
\pgfpathmoveto{\pgfqpoint{2.727696in}{0.944511in}}%
\pgfpathlineto{\pgfqpoint{2.739623in}{0.948906in}}%
\pgfpathlineto{\pgfqpoint{2.751543in}{0.953439in}}%
\pgfpathlineto{\pgfqpoint{2.763456in}{0.958112in}}%
\pgfpathlineto{\pgfqpoint{2.775363in}{0.962926in}}%
\pgfpathlineto{\pgfqpoint{2.787263in}{0.967884in}}%
\pgfpathlineto{\pgfqpoint{2.780879in}{0.975485in}}%
\pgfpathlineto{\pgfqpoint{2.774497in}{0.983559in}}%
\pgfpathlineto{\pgfqpoint{2.768116in}{0.992080in}}%
\pgfpathlineto{\pgfqpoint{2.761737in}{1.001024in}}%
\pgfpathlineto{\pgfqpoint{2.755359in}{1.010363in}}%
\pgfpathlineto{\pgfqpoint{2.743464in}{1.006530in}}%
\pgfpathlineto{\pgfqpoint{2.731563in}{1.002727in}}%
\pgfpathlineto{\pgfqpoint{2.719656in}{0.998956in}}%
\pgfpathlineto{\pgfqpoint{2.707742in}{0.995218in}}%
\pgfpathlineto{\pgfqpoint{2.695821in}{0.991515in}}%
\pgfpathlineto{\pgfqpoint{2.702193in}{0.981532in}}%
\pgfpathlineto{\pgfqpoint{2.708566in}{0.971822in}}%
\pgfpathlineto{\pgfqpoint{2.714941in}{0.962404in}}%
\pgfpathlineto{\pgfqpoint{2.721318in}{0.953295in}}%
\pgfpathclose%
\pgfusepath{stroke,fill}%
\end{pgfscope}%
\begin{pgfscope}%
\pgfpathrectangle{\pgfqpoint{0.887500in}{0.275000in}}{\pgfqpoint{4.225000in}{4.225000in}}%
\pgfusepath{clip}%
\pgfsetbuttcap%
\pgfsetroundjoin%
\definecolor{currentfill}{rgb}{0.282290,0.145912,0.461510}%
\pgfsetfillcolor{currentfill}%
\pgfsetfillopacity{0.700000}%
\pgfsetlinewidth{0.501875pt}%
\definecolor{currentstroke}{rgb}{1.000000,1.000000,1.000000}%
\pgfsetstrokecolor{currentstroke}%
\pgfsetstrokeopacity{0.500000}%
\pgfsetdash{}{0pt}%
\pgfpathmoveto{\pgfqpoint{2.392822in}{1.031069in}}%
\pgfpathlineto{\pgfqpoint{2.404851in}{1.034701in}}%
\pgfpathlineto{\pgfqpoint{2.416873in}{1.038326in}}%
\pgfpathlineto{\pgfqpoint{2.428889in}{1.041943in}}%
\pgfpathlineto{\pgfqpoint{2.440899in}{1.045555in}}%
\pgfpathlineto{\pgfqpoint{2.452903in}{1.049160in}}%
\pgfpathlineto{\pgfqpoint{2.446583in}{1.060717in}}%
\pgfpathlineto{\pgfqpoint{2.440266in}{1.072314in}}%
\pgfpathlineto{\pgfqpoint{2.433953in}{1.083944in}}%
\pgfpathlineto{\pgfqpoint{2.427642in}{1.095600in}}%
\pgfpathlineto{\pgfqpoint{2.421334in}{1.107275in}}%
\pgfpathlineto{\pgfqpoint{2.409341in}{1.103586in}}%
\pgfpathlineto{\pgfqpoint{2.397342in}{1.099894in}}%
\pgfpathlineto{\pgfqpoint{2.385337in}{1.096197in}}%
\pgfpathlineto{\pgfqpoint{2.373325in}{1.092496in}}%
\pgfpathlineto{\pgfqpoint{2.361307in}{1.088789in}}%
\pgfpathlineto{\pgfqpoint{2.367604in}{1.077198in}}%
\pgfpathlineto{\pgfqpoint{2.373904in}{1.065625in}}%
\pgfpathlineto{\pgfqpoint{2.380207in}{1.054075in}}%
\pgfpathlineto{\pgfqpoint{2.386513in}{1.042555in}}%
\pgfpathclose%
\pgfusepath{stroke,fill}%
\end{pgfscope}%
\begin{pgfscope}%
\pgfpathrectangle{\pgfqpoint{0.887500in}{0.275000in}}{\pgfqpoint{4.225000in}{4.225000in}}%
\pgfusepath{clip}%
\pgfsetbuttcap%
\pgfsetroundjoin%
\definecolor{currentfill}{rgb}{0.283229,0.120777,0.440584}%
\pgfsetfillcolor{currentfill}%
\pgfsetfillopacity{0.700000}%
\pgfsetlinewidth{0.501875pt}%
\definecolor{currentstroke}{rgb}{1.000000,1.000000,1.000000}%
\pgfsetstrokecolor{currentstroke}%
\pgfsetstrokeopacity{0.500000}%
\pgfsetdash{}{0pt}%
\pgfpathmoveto{\pgfqpoint{2.484541in}{0.992196in}}%
\pgfpathlineto{\pgfqpoint{2.496549in}{0.995756in}}%
\pgfpathlineto{\pgfqpoint{2.508550in}{0.999309in}}%
\pgfpathlineto{\pgfqpoint{2.520544in}{1.002857in}}%
\pgfpathlineto{\pgfqpoint{2.532533in}{1.006398in}}%
\pgfpathlineto{\pgfqpoint{2.544515in}{1.009934in}}%
\pgfpathlineto{\pgfqpoint{2.538172in}{1.021200in}}%
\pgfpathlineto{\pgfqpoint{2.531832in}{1.032559in}}%
\pgfpathlineto{\pgfqpoint{2.525494in}{1.044001in}}%
\pgfpathlineto{\pgfqpoint{2.519159in}{1.055519in}}%
\pgfpathlineto{\pgfqpoint{2.512826in}{1.067101in}}%
\pgfpathlineto{\pgfqpoint{2.500854in}{1.063523in}}%
\pgfpathlineto{\pgfqpoint{2.488876in}{1.059940in}}%
\pgfpathlineto{\pgfqpoint{2.476891in}{1.056352in}}%
\pgfpathlineto{\pgfqpoint{2.464900in}{1.052759in}}%
\pgfpathlineto{\pgfqpoint{2.452903in}{1.049160in}}%
\pgfpathlineto{\pgfqpoint{2.459225in}{1.037648in}}%
\pgfpathlineto{\pgfqpoint{2.465550in}{1.026190in}}%
\pgfpathlineto{\pgfqpoint{2.471878in}{1.014790in}}%
\pgfpathlineto{\pgfqpoint{2.478208in}{1.003457in}}%
\pgfpathclose%
\pgfusepath{stroke,fill}%
\end{pgfscope}%
\begin{pgfscope}%
\pgfpathrectangle{\pgfqpoint{0.887500in}{0.275000in}}{\pgfqpoint{4.225000in}{4.225000in}}%
\pgfusepath{clip}%
\pgfsetbuttcap%
\pgfsetroundjoin%
\definecolor{currentfill}{rgb}{0.282656,0.100196,0.422160}%
\pgfsetfillcolor{currentfill}%
\pgfsetfillopacity{0.700000}%
\pgfsetlinewidth{0.501875pt}%
\definecolor{currentstroke}{rgb}{1.000000,1.000000,1.000000}%
\pgfsetstrokecolor{currentstroke}%
\pgfsetstrokeopacity{0.500000}%
\pgfsetdash{}{0pt}%
\pgfpathmoveto{\pgfqpoint{2.576260in}{0.955322in}}%
\pgfpathlineto{\pgfqpoint{2.588245in}{0.958938in}}%
\pgfpathlineto{\pgfqpoint{2.600223in}{0.962552in}}%
\pgfpathlineto{\pgfqpoint{2.612195in}{0.966160in}}%
\pgfpathlineto{\pgfqpoint{2.624161in}{0.969764in}}%
\pgfpathlineto{\pgfqpoint{2.636120in}{0.973364in}}%
\pgfpathlineto{\pgfqpoint{2.629759in}{0.983860in}}%
\pgfpathlineto{\pgfqpoint{2.623398in}{0.994536in}}%
\pgfpathlineto{\pgfqpoint{2.617040in}{1.005380in}}%
\pgfpathlineto{\pgfqpoint{2.610684in}{1.016379in}}%
\pgfpathlineto{\pgfqpoint{2.604330in}{1.027521in}}%
\pgfpathlineto{\pgfqpoint{2.592379in}{1.024016in}}%
\pgfpathlineto{\pgfqpoint{2.580423in}{1.020505in}}%
\pgfpathlineto{\pgfqpoint{2.568460in}{1.016988in}}%
\pgfpathlineto{\pgfqpoint{2.556490in}{1.013464in}}%
\pgfpathlineto{\pgfqpoint{2.544515in}{1.009934in}}%
\pgfpathlineto{\pgfqpoint{2.550859in}{0.998770in}}%
\pgfpathlineto{\pgfqpoint{2.557206in}{0.987718in}}%
\pgfpathlineto{\pgfqpoint{2.563556in}{0.976786in}}%
\pgfpathlineto{\pgfqpoint{2.569907in}{0.965985in}}%
\pgfpathclose%
\pgfusepath{stroke,fill}%
\end{pgfscope}%
\begin{pgfscope}%
\pgfpathrectangle{\pgfqpoint{0.887500in}{0.275000in}}{\pgfqpoint{4.225000in}{4.225000in}}%
\pgfusepath{clip}%
\pgfsetbuttcap%
\pgfsetroundjoin%
\definecolor{currentfill}{rgb}{0.280894,0.078907,0.402329}%
\pgfsetfillcolor{currentfill}%
\pgfsetfillopacity{0.700000}%
\pgfsetlinewidth{0.501875pt}%
\definecolor{currentstroke}{rgb}{1.000000,1.000000,1.000000}%
\pgfsetstrokecolor{currentstroke}%
\pgfsetstrokeopacity{0.500000}%
\pgfsetdash{}{0pt}%
\pgfpathmoveto{\pgfqpoint{2.667957in}{0.924030in}}%
\pgfpathlineto{\pgfqpoint{2.679919in}{0.927993in}}%
\pgfpathlineto{\pgfqpoint{2.691873in}{0.932003in}}%
\pgfpathlineto{\pgfqpoint{2.703821in}{0.936079in}}%
\pgfpathlineto{\pgfqpoint{2.715762in}{0.940242in}}%
\pgfpathlineto{\pgfqpoint{2.727696in}{0.944511in}}%
\pgfpathlineto{\pgfqpoint{2.721318in}{0.953295in}}%
\pgfpathlineto{\pgfqpoint{2.714941in}{0.962404in}}%
\pgfpathlineto{\pgfqpoint{2.708566in}{0.971822in}}%
\pgfpathlineto{\pgfqpoint{2.702193in}{0.981532in}}%
\pgfpathlineto{\pgfqpoint{2.695821in}{0.991515in}}%
\pgfpathlineto{\pgfqpoint{2.683894in}{0.987844in}}%
\pgfpathlineto{\pgfqpoint{2.671960in}{0.984200in}}%
\pgfpathlineto{\pgfqpoint{2.660020in}{0.980576in}}%
\pgfpathlineto{\pgfqpoint{2.648074in}{0.976966in}}%
\pgfpathlineto{\pgfqpoint{2.636120in}{0.973364in}}%
\pgfpathlineto{\pgfqpoint{2.642484in}{0.963061in}}%
\pgfpathlineto{\pgfqpoint{2.648850in}{0.952964in}}%
\pgfpathlineto{\pgfqpoint{2.655217in}{0.943084in}}%
\pgfpathlineto{\pgfqpoint{2.661587in}{0.933436in}}%
\pgfpathclose%
\pgfusepath{stroke,fill}%
\end{pgfscope}%
\begin{pgfscope}%
\pgfpathrectangle{\pgfqpoint{0.887500in}{0.275000in}}{\pgfqpoint{4.225000in}{4.225000in}}%
\pgfusepath{clip}%
\pgfsetbuttcap%
\pgfsetroundjoin%
\definecolor{currentfill}{rgb}{0.283229,0.120777,0.440584}%
\pgfsetfillcolor{currentfill}%
\pgfsetfillopacity{0.700000}%
\pgfsetlinewidth{0.501875pt}%
\definecolor{currentstroke}{rgb}{1.000000,1.000000,1.000000}%
\pgfsetstrokecolor{currentstroke}%
\pgfsetstrokeopacity{0.500000}%
\pgfsetdash{}{0pt}%
\pgfpathmoveto{\pgfqpoint{2.424410in}{0.974317in}}%
\pgfpathlineto{\pgfqpoint{2.436449in}{0.977904in}}%
\pgfpathlineto{\pgfqpoint{2.448481in}{0.981485in}}%
\pgfpathlineto{\pgfqpoint{2.460508in}{0.985061in}}%
\pgfpathlineto{\pgfqpoint{2.472528in}{0.988632in}}%
\pgfpathlineto{\pgfqpoint{2.484541in}{0.992196in}}%
\pgfpathlineto{\pgfqpoint{2.478208in}{1.003457in}}%
\pgfpathlineto{\pgfqpoint{2.471878in}{1.014790in}}%
\pgfpathlineto{\pgfqpoint{2.465550in}{1.026190in}}%
\pgfpathlineto{\pgfqpoint{2.459225in}{1.037648in}}%
\pgfpathlineto{\pgfqpoint{2.452903in}{1.049160in}}%
\pgfpathlineto{\pgfqpoint{2.440899in}{1.045555in}}%
\pgfpathlineto{\pgfqpoint{2.428889in}{1.041943in}}%
\pgfpathlineto{\pgfqpoint{2.416873in}{1.038326in}}%
\pgfpathlineto{\pgfqpoint{2.404851in}{1.034701in}}%
\pgfpathlineto{\pgfqpoint{2.392822in}{1.031069in}}%
\pgfpathlineto{\pgfqpoint{2.399134in}{1.019621in}}%
\pgfpathlineto{\pgfqpoint{2.405449in}{1.008217in}}%
\pgfpathlineto{\pgfqpoint{2.411767in}{0.996862in}}%
\pgfpathlineto{\pgfqpoint{2.418087in}{0.985560in}}%
\pgfpathclose%
\pgfusepath{stroke,fill}%
\end{pgfscope}%
\begin{pgfscope}%
\pgfpathrectangle{\pgfqpoint{0.887500in}{0.275000in}}{\pgfqpoint{4.225000in}{4.225000in}}%
\pgfusepath{clip}%
\pgfsetbuttcap%
\pgfsetroundjoin%
\definecolor{currentfill}{rgb}{0.282656,0.100196,0.422160}%
\pgfsetfillcolor{currentfill}%
\pgfsetfillopacity{0.700000}%
\pgfsetlinewidth{0.501875pt}%
\definecolor{currentstroke}{rgb}{1.000000,1.000000,1.000000}%
\pgfsetstrokecolor{currentstroke}%
\pgfsetstrokeopacity{0.500000}%
\pgfsetdash{}{0pt}%
\pgfpathmoveto{\pgfqpoint{2.516242in}{0.937216in}}%
\pgfpathlineto{\pgfqpoint{2.528258in}{0.940837in}}%
\pgfpathlineto{\pgfqpoint{2.540268in}{0.944459in}}%
\pgfpathlineto{\pgfqpoint{2.552272in}{0.948081in}}%
\pgfpathlineto{\pgfqpoint{2.564269in}{0.951702in}}%
\pgfpathlineto{\pgfqpoint{2.576260in}{0.955322in}}%
\pgfpathlineto{\pgfqpoint{2.569907in}{0.965985in}}%
\pgfpathlineto{\pgfqpoint{2.563556in}{0.976786in}}%
\pgfpathlineto{\pgfqpoint{2.557206in}{0.987718in}}%
\pgfpathlineto{\pgfqpoint{2.550859in}{0.998770in}}%
\pgfpathlineto{\pgfqpoint{2.544515in}{1.009934in}}%
\pgfpathlineto{\pgfqpoint{2.532533in}{1.006398in}}%
\pgfpathlineto{\pgfqpoint{2.520544in}{1.002857in}}%
\pgfpathlineto{\pgfqpoint{2.508550in}{0.999309in}}%
\pgfpathlineto{\pgfqpoint{2.496549in}{0.995756in}}%
\pgfpathlineto{\pgfqpoint{2.484541in}{0.992196in}}%
\pgfpathlineto{\pgfqpoint{2.490877in}{0.981015in}}%
\pgfpathlineto{\pgfqpoint{2.497214in}{0.969920in}}%
\pgfpathlineto{\pgfqpoint{2.503554in}{0.958917in}}%
\pgfpathlineto{\pgfqpoint{2.509897in}{0.948013in}}%
\pgfpathclose%
\pgfusepath{stroke,fill}%
\end{pgfscope}%
\begin{pgfscope}%
\pgfpathrectangle{\pgfqpoint{0.887500in}{0.275000in}}{\pgfqpoint{4.225000in}{4.225000in}}%
\pgfusepath{clip}%
\pgfsetbuttcap%
\pgfsetroundjoin%
\definecolor{currentfill}{rgb}{0.280894,0.078907,0.402329}%
\pgfsetfillcolor{currentfill}%
\pgfsetfillopacity{0.700000}%
\pgfsetlinewidth{0.501875pt}%
\definecolor{currentstroke}{rgb}{1.000000,1.000000,1.000000}%
\pgfsetstrokecolor{currentstroke}%
\pgfsetstrokeopacity{0.500000}%
\pgfsetdash{}{0pt}%
\pgfpathmoveto{\pgfqpoint{2.608057in}{0.904414in}}%
\pgfpathlineto{\pgfqpoint{2.620050in}{0.908324in}}%
\pgfpathlineto{\pgfqpoint{2.632036in}{0.912243in}}%
\pgfpathlineto{\pgfqpoint{2.644016in}{0.916167in}}%
\pgfpathlineto{\pgfqpoint{2.655990in}{0.920094in}}%
\pgfpathlineto{\pgfqpoint{2.667957in}{0.924030in}}%
\pgfpathlineto{\pgfqpoint{2.661587in}{0.933436in}}%
\pgfpathlineto{\pgfqpoint{2.655217in}{0.943084in}}%
\pgfpathlineto{\pgfqpoint{2.648850in}{0.952964in}}%
\pgfpathlineto{\pgfqpoint{2.642484in}{0.963061in}}%
\pgfpathlineto{\pgfqpoint{2.636120in}{0.973364in}}%
\pgfpathlineto{\pgfqpoint{2.624161in}{0.969764in}}%
\pgfpathlineto{\pgfqpoint{2.612195in}{0.966160in}}%
\pgfpathlineto{\pgfqpoint{2.600223in}{0.962552in}}%
\pgfpathlineto{\pgfqpoint{2.588245in}{0.958938in}}%
\pgfpathlineto{\pgfqpoint{2.576260in}{0.955322in}}%
\pgfpathlineto{\pgfqpoint{2.582616in}{0.944807in}}%
\pgfpathlineto{\pgfqpoint{2.588973in}{0.934449in}}%
\pgfpathlineto{\pgfqpoint{2.595333in}{0.924259in}}%
\pgfpathlineto{\pgfqpoint{2.601694in}{0.914244in}}%
\pgfpathclose%
\pgfusepath{stroke,fill}%
\end{pgfscope}%
\begin{pgfscope}%
\pgfpathrectangle{\pgfqpoint{0.887500in}{0.275000in}}{\pgfqpoint{4.225000in}{4.225000in}}%
\pgfusepath{clip}%
\pgfsetbuttcap%
\pgfsetroundjoin%
\definecolor{currentfill}{rgb}{0.282656,0.100196,0.422160}%
\pgfsetfillcolor{currentfill}%
\pgfsetfillopacity{0.700000}%
\pgfsetlinewidth{0.501875pt}%
\definecolor{currentstroke}{rgb}{1.000000,1.000000,1.000000}%
\pgfsetstrokecolor{currentstroke}%
\pgfsetstrokeopacity{0.500000}%
\pgfsetdash{}{0pt}%
\pgfpathmoveto{\pgfqpoint{2.456062in}{0.919153in}}%
\pgfpathlineto{\pgfqpoint{2.468111in}{0.922757in}}%
\pgfpathlineto{\pgfqpoint{2.480153in}{0.926366in}}%
\pgfpathlineto{\pgfqpoint{2.492189in}{0.929980in}}%
\pgfpathlineto{\pgfqpoint{2.504219in}{0.933596in}}%
\pgfpathlineto{\pgfqpoint{2.516242in}{0.937216in}}%
\pgfpathlineto{\pgfqpoint{2.509897in}{0.948013in}}%
\pgfpathlineto{\pgfqpoint{2.503554in}{0.958917in}}%
\pgfpathlineto{\pgfqpoint{2.497214in}{0.969920in}}%
\pgfpathlineto{\pgfqpoint{2.490877in}{0.981015in}}%
\pgfpathlineto{\pgfqpoint{2.484541in}{0.992196in}}%
\pgfpathlineto{\pgfqpoint{2.472528in}{0.988632in}}%
\pgfpathlineto{\pgfqpoint{2.460508in}{0.985061in}}%
\pgfpathlineto{\pgfqpoint{2.448481in}{0.981485in}}%
\pgfpathlineto{\pgfqpoint{2.436449in}{0.977904in}}%
\pgfpathlineto{\pgfqpoint{2.424410in}{0.974317in}}%
\pgfpathlineto{\pgfqpoint{2.430735in}{0.963137in}}%
\pgfpathlineto{\pgfqpoint{2.437063in}{0.952026in}}%
\pgfpathlineto{\pgfqpoint{2.443394in}{0.940988in}}%
\pgfpathlineto{\pgfqpoint{2.449727in}{0.930029in}}%
\pgfpathclose%
\pgfusepath{stroke,fill}%
\end{pgfscope}%
\begin{pgfscope}%
\pgfpathrectangle{\pgfqpoint{0.887500in}{0.275000in}}{\pgfqpoint{4.225000in}{4.225000in}}%
\pgfusepath{clip}%
\pgfsetbuttcap%
\pgfsetroundjoin%
\definecolor{currentfill}{rgb}{0.280894,0.078907,0.402329}%
\pgfsetfillcolor{currentfill}%
\pgfsetfillopacity{0.700000}%
\pgfsetlinewidth{0.501875pt}%
\definecolor{currentstroke}{rgb}{1.000000,1.000000,1.000000}%
\pgfsetstrokecolor{currentstroke}%
\pgfsetstrokeopacity{0.500000}%
\pgfsetdash{}{0pt}%
\pgfpathmoveto{\pgfqpoint{2.547999in}{0.885049in}}%
\pgfpathlineto{\pgfqpoint{2.560023in}{0.888890in}}%
\pgfpathlineto{\pgfqpoint{2.572041in}{0.892749in}}%
\pgfpathlineto{\pgfqpoint{2.584053in}{0.896624in}}%
\pgfpathlineto{\pgfqpoint{2.596058in}{0.900513in}}%
\pgfpathlineto{\pgfqpoint{2.608057in}{0.904414in}}%
\pgfpathlineto{\pgfqpoint{2.601694in}{0.914244in}}%
\pgfpathlineto{\pgfqpoint{2.595333in}{0.924259in}}%
\pgfpathlineto{\pgfqpoint{2.588973in}{0.934449in}}%
\pgfpathlineto{\pgfqpoint{2.582616in}{0.944807in}}%
\pgfpathlineto{\pgfqpoint{2.576260in}{0.955322in}}%
\pgfpathlineto{\pgfqpoint{2.564269in}{0.951702in}}%
\pgfpathlineto{\pgfqpoint{2.552272in}{0.948081in}}%
\pgfpathlineto{\pgfqpoint{2.540268in}{0.944459in}}%
\pgfpathlineto{\pgfqpoint{2.528258in}{0.940837in}}%
\pgfpathlineto{\pgfqpoint{2.516242in}{0.937216in}}%
\pgfpathlineto{\pgfqpoint{2.522589in}{0.926531in}}%
\pgfpathlineto{\pgfqpoint{2.528938in}{0.915965in}}%
\pgfpathlineto{\pgfqpoint{2.535289in}{0.905525in}}%
\pgfpathlineto{\pgfqpoint{2.541643in}{0.895218in}}%
\pgfpathclose%
\pgfusepath{stroke,fill}%
\end{pgfscope}%
\begin{pgfscope}%
\pgfpathrectangle{\pgfqpoint{0.887500in}{0.275000in}}{\pgfqpoint{4.225000in}{4.225000in}}%
\pgfusepath{clip}%
\pgfsetbuttcap%
\pgfsetroundjoin%
\definecolor{currentfill}{rgb}{0.280894,0.078907,0.402329}%
\pgfsetfillcolor{currentfill}%
\pgfsetfillopacity{0.700000}%
\pgfsetlinewidth{0.501875pt}%
\definecolor{currentstroke}{rgb}{1.000000,1.000000,1.000000}%
\pgfsetstrokecolor{currentstroke}%
\pgfsetstrokeopacity{0.500000}%
\pgfsetdash{}{0pt}%
\pgfpathmoveto{\pgfqpoint{2.487775in}{0.866199in}}%
\pgfpathlineto{\pgfqpoint{2.499834in}{0.869915in}}%
\pgfpathlineto{\pgfqpoint{2.511885in}{0.873660in}}%
\pgfpathlineto{\pgfqpoint{2.523929in}{0.877432in}}%
\pgfpathlineto{\pgfqpoint{2.535967in}{0.881229in}}%
\pgfpathlineto{\pgfqpoint{2.547999in}{0.885049in}}%
\pgfpathlineto{\pgfqpoint{2.541643in}{0.895218in}}%
\pgfpathlineto{\pgfqpoint{2.535289in}{0.905525in}}%
\pgfpathlineto{\pgfqpoint{2.528938in}{0.915965in}}%
\pgfpathlineto{\pgfqpoint{2.522589in}{0.926531in}}%
\pgfpathlineto{\pgfqpoint{2.516242in}{0.937216in}}%
\pgfpathlineto{\pgfqpoint{2.504219in}{0.933596in}}%
\pgfpathlineto{\pgfqpoint{2.492189in}{0.929980in}}%
\pgfpathlineto{\pgfqpoint{2.480153in}{0.926366in}}%
\pgfpathlineto{\pgfqpoint{2.468111in}{0.922757in}}%
\pgfpathlineto{\pgfqpoint{2.456062in}{0.919153in}}%
\pgfpathlineto{\pgfqpoint{2.462400in}{0.908366in}}%
\pgfpathlineto{\pgfqpoint{2.468740in}{0.897672in}}%
\pgfpathlineto{\pgfqpoint{2.475083in}{0.887076in}}%
\pgfpathlineto{\pgfqpoint{2.481428in}{0.876583in}}%
\pgfpathclose%
\pgfusepath{stroke,fill}%
\end{pgfscope}%
\begin{pgfscope}%
\pgfpathrectangle{\pgfqpoint{0.887500in}{0.275000in}}{\pgfqpoint{4.225000in}{4.225000in}}%
\pgfusepath{clip}%
\pgfsetbuttcap%
\pgfsetroundjoin%
\definecolor{currentfill}{rgb}{0.000000,0.000000,0.000000}%
\pgfsetfillcolor{currentfill}%
\pgfsetfillopacity{0.800000}%
\pgfsetlinewidth{0.000000pt}%
\definecolor{currentstroke}{rgb}{0.000000,0.000000,0.000000}%
\pgfsetstrokecolor{currentstroke}%
\pgfsetstrokeopacity{0.800000}%
\pgfsetdash{}{0pt}%
\pgfpathmoveto{\pgfqpoint{3.404419in}{3.821112in}}%
\pgfpathcurveto{\pgfqpoint{3.408538in}{3.821112in}}{\pgfqpoint{3.412488in}{3.822748in}}{\pgfqpoint{3.415400in}{3.825660in}}%
\pgfpathcurveto{\pgfqpoint{3.418312in}{3.828572in}}{\pgfqpoint{3.419948in}{3.832522in}}{\pgfqpoint{3.419948in}{3.836640in}}%
\pgfpathcurveto{\pgfqpoint{3.419948in}{3.840758in}}{\pgfqpoint{3.418312in}{3.844708in}}{\pgfqpoint{3.415400in}{3.847620in}}%
\pgfpathcurveto{\pgfqpoint{3.412488in}{3.850532in}}{\pgfqpoint{3.408538in}{3.852168in}}{\pgfqpoint{3.404419in}{3.852168in}}%
\pgfpathcurveto{\pgfqpoint{3.400301in}{3.852168in}}{\pgfqpoint{3.396351in}{3.850532in}}{\pgfqpoint{3.393439in}{3.847620in}}%
\pgfpathcurveto{\pgfqpoint{3.390527in}{3.844708in}}{\pgfqpoint{3.388891in}{3.840758in}}{\pgfqpoint{3.388891in}{3.836640in}}%
\pgfpathcurveto{\pgfqpoint{3.388891in}{3.832522in}}{\pgfqpoint{3.390527in}{3.828572in}}{\pgfqpoint{3.393439in}{3.825660in}}%
\pgfpathcurveto{\pgfqpoint{3.396351in}{3.822748in}}{\pgfqpoint{3.400301in}{3.821112in}}{\pgfqpoint{3.404419in}{3.821112in}}%
\pgfpathclose%
\pgfusepath{fill}%
\end{pgfscope}%
\begin{pgfscope}%
\pgfpathrectangle{\pgfqpoint{0.887500in}{0.275000in}}{\pgfqpoint{4.225000in}{4.225000in}}%
\pgfusepath{clip}%
\pgfsetbuttcap%
\pgfsetroundjoin%
\definecolor{currentfill}{rgb}{0.000000,0.000000,0.000000}%
\pgfsetfillcolor{currentfill}%
\pgfsetfillopacity{0.800000}%
\pgfsetlinewidth{0.000000pt}%
\definecolor{currentstroke}{rgb}{0.000000,0.000000,0.000000}%
\pgfsetstrokecolor{currentstroke}%
\pgfsetstrokeopacity{0.800000}%
\pgfsetdash{}{0pt}%
\pgfpathmoveto{\pgfqpoint{3.575302in}{3.815765in}}%
\pgfpathcurveto{\pgfqpoint{3.579420in}{3.815765in}}{\pgfqpoint{3.583370in}{3.817401in}}{\pgfqpoint{3.586282in}{3.820313in}}%
\pgfpathcurveto{\pgfqpoint{3.589194in}{3.823225in}}{\pgfqpoint{3.590830in}{3.827175in}}{\pgfqpoint{3.590830in}{3.831293in}}%
\pgfpathcurveto{\pgfqpoint{3.590830in}{3.835411in}}{\pgfqpoint{3.589194in}{3.839361in}}{\pgfqpoint{3.586282in}{3.842273in}}%
\pgfpathcurveto{\pgfqpoint{3.583370in}{3.845185in}}{\pgfqpoint{3.579420in}{3.846821in}}{\pgfqpoint{3.575302in}{3.846821in}}%
\pgfpathcurveto{\pgfqpoint{3.571183in}{3.846821in}}{\pgfqpoint{3.567233in}{3.845185in}}{\pgfqpoint{3.564321in}{3.842273in}}%
\pgfpathcurveto{\pgfqpoint{3.561409in}{3.839361in}}{\pgfqpoint{3.559773in}{3.835411in}}{\pgfqpoint{3.559773in}{3.831293in}}%
\pgfpathcurveto{\pgfqpoint{3.559773in}{3.827175in}}{\pgfqpoint{3.561409in}{3.823225in}}{\pgfqpoint{3.564321in}{3.820313in}}%
\pgfpathcurveto{\pgfqpoint{3.567233in}{3.817401in}}{\pgfqpoint{3.571183in}{3.815765in}}{\pgfqpoint{3.575302in}{3.815765in}}%
\pgfpathclose%
\pgfusepath{fill}%
\end{pgfscope}%
\begin{pgfscope}%
\pgfpathrectangle{\pgfqpoint{0.887500in}{0.275000in}}{\pgfqpoint{4.225000in}{4.225000in}}%
\pgfusepath{clip}%
\pgfsetbuttcap%
\pgfsetroundjoin%
\definecolor{currentfill}{rgb}{0.000000,0.000000,0.000000}%
\pgfsetfillcolor{currentfill}%
\pgfsetfillopacity{0.800000}%
\pgfsetlinewidth{0.000000pt}%
\definecolor{currentstroke}{rgb}{0.000000,0.000000,0.000000}%
\pgfsetstrokecolor{currentstroke}%
\pgfsetstrokeopacity{0.800000}%
\pgfsetdash{}{0pt}%
\pgfpathmoveto{\pgfqpoint{3.297346in}{3.794696in}}%
\pgfpathcurveto{\pgfqpoint{3.301465in}{3.794696in}}{\pgfqpoint{3.305415in}{3.796332in}}{\pgfqpoint{3.308327in}{3.799244in}}%
\pgfpathcurveto{\pgfqpoint{3.311239in}{3.802156in}}{\pgfqpoint{3.312875in}{3.806106in}}{\pgfqpoint{3.312875in}{3.810224in}}%
\pgfpathcurveto{\pgfqpoint{3.312875in}{3.814342in}}{\pgfqpoint{3.311239in}{3.818292in}}{\pgfqpoint{3.308327in}{3.821204in}}%
\pgfpathcurveto{\pgfqpoint{3.305415in}{3.824116in}}{\pgfqpoint{3.301465in}{3.825752in}}{\pgfqpoint{3.297346in}{3.825752in}}%
\pgfpathcurveto{\pgfqpoint{3.293228in}{3.825752in}}{\pgfqpoint{3.289278in}{3.824116in}}{\pgfqpoint{3.286366in}{3.821204in}}%
\pgfpathcurveto{\pgfqpoint{3.283454in}{3.818292in}}{\pgfqpoint{3.281818in}{3.814342in}}{\pgfqpoint{3.281818in}{3.810224in}}%
\pgfpathcurveto{\pgfqpoint{3.281818in}{3.806106in}}{\pgfqpoint{3.283454in}{3.802156in}}{\pgfqpoint{3.286366in}{3.799244in}}%
\pgfpathcurveto{\pgfqpoint{3.289278in}{3.796332in}}{\pgfqpoint{3.293228in}{3.794696in}}{\pgfqpoint{3.297346in}{3.794696in}}%
\pgfpathclose%
\pgfusepath{fill}%
\end{pgfscope}%
\begin{pgfscope}%
\pgfpathrectangle{\pgfqpoint{0.887500in}{0.275000in}}{\pgfqpoint{4.225000in}{4.225000in}}%
\pgfusepath{clip}%
\pgfsetbuttcap%
\pgfsetroundjoin%
\definecolor{currentfill}{rgb}{0.000000,0.000000,0.000000}%
\pgfsetfillcolor{currentfill}%
\pgfsetfillopacity{0.800000}%
\pgfsetlinewidth{0.000000pt}%
\definecolor{currentstroke}{rgb}{0.000000,0.000000,0.000000}%
\pgfsetstrokecolor{currentstroke}%
\pgfsetstrokeopacity{0.800000}%
\pgfsetdash{}{0pt}%
\pgfpathmoveto{\pgfqpoint{3.189692in}{3.747568in}}%
\pgfpathcurveto{\pgfqpoint{3.193810in}{3.747568in}}{\pgfqpoint{3.197760in}{3.749204in}}{\pgfqpoint{3.200672in}{3.752116in}}%
\pgfpathcurveto{\pgfqpoint{3.203584in}{3.755028in}}{\pgfqpoint{3.205220in}{3.758978in}}{\pgfqpoint{3.205220in}{3.763096in}}%
\pgfpathcurveto{\pgfqpoint{3.205220in}{3.767214in}}{\pgfqpoint{3.203584in}{3.771164in}}{\pgfqpoint{3.200672in}{3.774076in}}%
\pgfpathcurveto{\pgfqpoint{3.197760in}{3.776988in}}{\pgfqpoint{3.193810in}{3.778624in}}{\pgfqpoint{3.189692in}{3.778624in}}%
\pgfpathcurveto{\pgfqpoint{3.185574in}{3.778624in}}{\pgfqpoint{3.181624in}{3.776988in}}{\pgfqpoint{3.178712in}{3.774076in}}%
\pgfpathcurveto{\pgfqpoint{3.175800in}{3.771164in}}{\pgfqpoint{3.174164in}{3.767214in}}{\pgfqpoint{3.174164in}{3.763096in}}%
\pgfpathcurveto{\pgfqpoint{3.174164in}{3.758978in}}{\pgfqpoint{3.175800in}{3.755028in}}{\pgfqpoint{3.178712in}{3.752116in}}%
\pgfpathcurveto{\pgfqpoint{3.181624in}{3.749204in}}{\pgfqpoint{3.185574in}{3.747568in}}{\pgfqpoint{3.189692in}{3.747568in}}%
\pgfpathclose%
\pgfusepath{fill}%
\end{pgfscope}%
\begin{pgfscope}%
\pgfpathrectangle{\pgfqpoint{0.887500in}{0.275000in}}{\pgfqpoint{4.225000in}{4.225000in}}%
\pgfusepath{clip}%
\pgfsetbuttcap%
\pgfsetroundjoin%
\definecolor{currentfill}{rgb}{0.000000,0.000000,0.000000}%
\pgfsetfillcolor{currentfill}%
\pgfsetfillopacity{0.800000}%
\pgfsetlinewidth{0.000000pt}%
\definecolor{currentstroke}{rgb}{0.000000,0.000000,0.000000}%
\pgfsetstrokecolor{currentstroke}%
\pgfsetstrokeopacity{0.800000}%
\pgfsetdash{}{0pt}%
\pgfpathmoveto{\pgfqpoint{3.468847in}{3.832313in}}%
\pgfpathcurveto{\pgfqpoint{3.472965in}{3.832313in}}{\pgfqpoint{3.476915in}{3.833949in}}{\pgfqpoint{3.479827in}{3.836861in}}%
\pgfpathcurveto{\pgfqpoint{3.482739in}{3.839773in}}{\pgfqpoint{3.484375in}{3.843723in}}{\pgfqpoint{3.484375in}{3.847841in}}%
\pgfpathcurveto{\pgfqpoint{3.484375in}{3.851959in}}{\pgfqpoint{3.482739in}{3.855909in}}{\pgfqpoint{3.479827in}{3.858821in}}%
\pgfpathcurveto{\pgfqpoint{3.476915in}{3.861733in}}{\pgfqpoint{3.472965in}{3.863369in}}{\pgfqpoint{3.468847in}{3.863369in}}%
\pgfpathcurveto{\pgfqpoint{3.464729in}{3.863369in}}{\pgfqpoint{3.460779in}{3.861733in}}{\pgfqpoint{3.457867in}{3.858821in}}%
\pgfpathcurveto{\pgfqpoint{3.454955in}{3.855909in}}{\pgfqpoint{3.453319in}{3.851959in}}{\pgfqpoint{3.453319in}{3.847841in}}%
\pgfpathcurveto{\pgfqpoint{3.453319in}{3.843723in}}{\pgfqpoint{3.454955in}{3.839773in}}{\pgfqpoint{3.457867in}{3.836861in}}%
\pgfpathcurveto{\pgfqpoint{3.460779in}{3.833949in}}{\pgfqpoint{3.464729in}{3.832313in}}{\pgfqpoint{3.468847in}{3.832313in}}%
\pgfpathclose%
\pgfusepath{fill}%
\end{pgfscope}%
\begin{pgfscope}%
\pgfpathrectangle{\pgfqpoint{0.887500in}{0.275000in}}{\pgfqpoint{4.225000in}{4.225000in}}%
\pgfusepath{clip}%
\pgfsetbuttcap%
\pgfsetroundjoin%
\definecolor{currentfill}{rgb}{0.000000,0.000000,0.000000}%
\pgfsetfillcolor{currentfill}%
\pgfsetfillopacity{0.800000}%
\pgfsetlinewidth{0.000000pt}%
\definecolor{currentstroke}{rgb}{0.000000,0.000000,0.000000}%
\pgfsetstrokecolor{currentstroke}%
\pgfsetstrokeopacity{0.800000}%
\pgfsetdash{}{0pt}%
\pgfpathmoveto{\pgfqpoint{3.081623in}{3.691584in}}%
\pgfpathcurveto{\pgfqpoint{3.085741in}{3.691584in}}{\pgfqpoint{3.089691in}{3.693220in}}{\pgfqpoint{3.092603in}{3.696132in}}%
\pgfpathcurveto{\pgfqpoint{3.095515in}{3.699044in}}{\pgfqpoint{3.097151in}{3.702994in}}{\pgfqpoint{3.097151in}{3.707112in}}%
\pgfpathcurveto{\pgfqpoint{3.097151in}{3.711230in}}{\pgfqpoint{3.095515in}{3.715180in}}{\pgfqpoint{3.092603in}{3.718092in}}%
\pgfpathcurveto{\pgfqpoint{3.089691in}{3.721004in}}{\pgfqpoint{3.085741in}{3.722640in}}{\pgfqpoint{3.081623in}{3.722640in}}%
\pgfpathcurveto{\pgfqpoint{3.077505in}{3.722640in}}{\pgfqpoint{3.073555in}{3.721004in}}{\pgfqpoint{3.070643in}{3.718092in}}%
\pgfpathcurveto{\pgfqpoint{3.067731in}{3.715180in}}{\pgfqpoint{3.066094in}{3.711230in}}{\pgfqpoint{3.066094in}{3.707112in}}%
\pgfpathcurveto{\pgfqpoint{3.066094in}{3.702994in}}{\pgfqpoint{3.067731in}{3.699044in}}{\pgfqpoint{3.070643in}{3.696132in}}%
\pgfpathcurveto{\pgfqpoint{3.073555in}{3.693220in}}{\pgfqpoint{3.077505in}{3.691584in}}{\pgfqpoint{3.081623in}{3.691584in}}%
\pgfpathclose%
\pgfusepath{fill}%
\end{pgfscope}%
\begin{pgfscope}%
\pgfpathrectangle{\pgfqpoint{0.887500in}{0.275000in}}{\pgfqpoint{4.225000in}{4.225000in}}%
\pgfusepath{clip}%
\pgfsetbuttcap%
\pgfsetroundjoin%
\definecolor{currentfill}{rgb}{0.000000,0.000000,0.000000}%
\pgfsetfillcolor{currentfill}%
\pgfsetfillopacity{0.800000}%
\pgfsetlinewidth{0.000000pt}%
\definecolor{currentstroke}{rgb}{0.000000,0.000000,0.000000}%
\pgfsetstrokecolor{currentstroke}%
\pgfsetstrokeopacity{0.800000}%
\pgfsetdash{}{0pt}%
\pgfpathmoveto{\pgfqpoint{3.361242in}{3.793561in}}%
\pgfpathcurveto{\pgfqpoint{3.365360in}{3.793561in}}{\pgfqpoint{3.369310in}{3.795197in}}{\pgfqpoint{3.372222in}{3.798109in}}%
\pgfpathcurveto{\pgfqpoint{3.375134in}{3.801021in}}{\pgfqpoint{3.376770in}{3.804971in}}{\pgfqpoint{3.376770in}{3.809089in}}%
\pgfpathcurveto{\pgfqpoint{3.376770in}{3.813207in}}{\pgfqpoint{3.375134in}{3.817157in}}{\pgfqpoint{3.372222in}{3.820069in}}%
\pgfpathcurveto{\pgfqpoint{3.369310in}{3.822981in}}{\pgfqpoint{3.365360in}{3.824617in}}{\pgfqpoint{3.361242in}{3.824617in}}%
\pgfpathcurveto{\pgfqpoint{3.357123in}{3.824617in}}{\pgfqpoint{3.353173in}{3.822981in}}{\pgfqpoint{3.350261in}{3.820069in}}%
\pgfpathcurveto{\pgfqpoint{3.347349in}{3.817157in}}{\pgfqpoint{3.345713in}{3.813207in}}{\pgfqpoint{3.345713in}{3.809089in}}%
\pgfpathcurveto{\pgfqpoint{3.345713in}{3.804971in}}{\pgfqpoint{3.347349in}{3.801021in}}{\pgfqpoint{3.350261in}{3.798109in}}%
\pgfpathcurveto{\pgfqpoint{3.353173in}{3.795197in}}{\pgfqpoint{3.357123in}{3.793561in}}{\pgfqpoint{3.361242in}{3.793561in}}%
\pgfpathclose%
\pgfusepath{fill}%
\end{pgfscope}%
\begin{pgfscope}%
\pgfpathrectangle{\pgfqpoint{0.887500in}{0.275000in}}{\pgfqpoint{4.225000in}{4.225000in}}%
\pgfusepath{clip}%
\pgfsetbuttcap%
\pgfsetroundjoin%
\definecolor{currentfill}{rgb}{0.000000,0.000000,0.000000}%
\pgfsetfillcolor{currentfill}%
\pgfsetfillopacity{0.800000}%
\pgfsetlinewidth{0.000000pt}%
\definecolor{currentstroke}{rgb}{0.000000,0.000000,0.000000}%
\pgfsetstrokecolor{currentstroke}%
\pgfsetstrokeopacity{0.800000}%
\pgfsetdash{}{0pt}%
\pgfpathmoveto{\pgfqpoint{2.973185in}{3.636857in}}%
\pgfpathcurveto{\pgfqpoint{2.977303in}{3.636857in}}{\pgfqpoint{2.981253in}{3.638493in}}{\pgfqpoint{2.984165in}{3.641405in}}%
\pgfpathcurveto{\pgfqpoint{2.987077in}{3.644317in}}{\pgfqpoint{2.988713in}{3.648267in}}{\pgfqpoint{2.988713in}{3.652385in}}%
\pgfpathcurveto{\pgfqpoint{2.988713in}{3.656503in}}{\pgfqpoint{2.987077in}{3.660453in}}{\pgfqpoint{2.984165in}{3.663365in}}%
\pgfpathcurveto{\pgfqpoint{2.981253in}{3.666277in}}{\pgfqpoint{2.977303in}{3.667913in}}{\pgfqpoint{2.973185in}{3.667913in}}%
\pgfpathcurveto{\pgfqpoint{2.969067in}{3.667913in}}{\pgfqpoint{2.965116in}{3.666277in}}{\pgfqpoint{2.962205in}{3.663365in}}%
\pgfpathcurveto{\pgfqpoint{2.959293in}{3.660453in}}{\pgfqpoint{2.957656in}{3.656503in}}{\pgfqpoint{2.957656in}{3.652385in}}%
\pgfpathcurveto{\pgfqpoint{2.957656in}{3.648267in}}{\pgfqpoint{2.959293in}{3.644317in}}{\pgfqpoint{2.962205in}{3.641405in}}%
\pgfpathcurveto{\pgfqpoint{2.965116in}{3.638493in}}{\pgfqpoint{2.969067in}{3.636857in}}{\pgfqpoint{2.973185in}{3.636857in}}%
\pgfpathclose%
\pgfusepath{fill}%
\end{pgfscope}%
\begin{pgfscope}%
\pgfpathrectangle{\pgfqpoint{0.887500in}{0.275000in}}{\pgfqpoint{4.225000in}{4.225000in}}%
\pgfusepath{clip}%
\pgfsetbuttcap%
\pgfsetroundjoin%
\definecolor{currentfill}{rgb}{0.000000,0.000000,0.000000}%
\pgfsetfillcolor{currentfill}%
\pgfsetfillopacity{0.800000}%
\pgfsetlinewidth{0.000000pt}%
\definecolor{currentstroke}{rgb}{0.000000,0.000000,0.000000}%
\pgfsetstrokecolor{currentstroke}%
\pgfsetstrokeopacity{0.800000}%
\pgfsetdash{}{0pt}%
\pgfpathmoveto{\pgfqpoint{3.641122in}{3.848339in}}%
\pgfpathcurveto{\pgfqpoint{3.645240in}{3.848339in}}{\pgfqpoint{3.649190in}{3.849975in}}{\pgfqpoint{3.652102in}{3.852887in}}%
\pgfpathcurveto{\pgfqpoint{3.655014in}{3.855799in}}{\pgfqpoint{3.656650in}{3.859749in}}{\pgfqpoint{3.656650in}{3.863867in}}%
\pgfpathcurveto{\pgfqpoint{3.656650in}{3.867985in}}{\pgfqpoint{3.655014in}{3.871935in}}{\pgfqpoint{3.652102in}{3.874847in}}%
\pgfpathcurveto{\pgfqpoint{3.649190in}{3.877759in}}{\pgfqpoint{3.645240in}{3.879395in}}{\pgfqpoint{3.641122in}{3.879395in}}%
\pgfpathcurveto{\pgfqpoint{3.637003in}{3.879395in}}{\pgfqpoint{3.633053in}{3.877759in}}{\pgfqpoint{3.630141in}{3.874847in}}%
\pgfpathcurveto{\pgfqpoint{3.627229in}{3.871935in}}{\pgfqpoint{3.625593in}{3.867985in}}{\pgfqpoint{3.625593in}{3.863867in}}%
\pgfpathcurveto{\pgfqpoint{3.625593in}{3.859749in}}{\pgfqpoint{3.627229in}{3.855799in}}{\pgfqpoint{3.630141in}{3.852887in}}%
\pgfpathcurveto{\pgfqpoint{3.633053in}{3.849975in}}{\pgfqpoint{3.637003in}{3.848339in}}{\pgfqpoint{3.641122in}{3.848339in}}%
\pgfpathclose%
\pgfusepath{fill}%
\end{pgfscope}%
\begin{pgfscope}%
\pgfpathrectangle{\pgfqpoint{0.887500in}{0.275000in}}{\pgfqpoint{4.225000in}{4.225000in}}%
\pgfusepath{clip}%
\pgfsetbuttcap%
\pgfsetroundjoin%
\definecolor{currentfill}{rgb}{0.000000,0.000000,0.000000}%
\pgfsetfillcolor{currentfill}%
\pgfsetfillopacity{0.800000}%
\pgfsetlinewidth{0.000000pt}%
\definecolor{currentstroke}{rgb}{0.000000,0.000000,0.000000}%
\pgfsetstrokecolor{currentstroke}%
\pgfsetstrokeopacity{0.800000}%
\pgfsetdash{}{0pt}%
\pgfpathmoveto{\pgfqpoint{3.253109in}{3.739773in}}%
\pgfpathcurveto{\pgfqpoint{3.257228in}{3.739773in}}{\pgfqpoint{3.261178in}{3.741409in}}{\pgfqpoint{3.264090in}{3.744321in}}%
\pgfpathcurveto{\pgfqpoint{3.267002in}{3.747233in}}{\pgfqpoint{3.268638in}{3.751183in}}{\pgfqpoint{3.268638in}{3.755301in}}%
\pgfpathcurveto{\pgfqpoint{3.268638in}{3.759419in}}{\pgfqpoint{3.267002in}{3.763369in}}{\pgfqpoint{3.264090in}{3.766281in}}%
\pgfpathcurveto{\pgfqpoint{3.261178in}{3.769193in}}{\pgfqpoint{3.257228in}{3.770829in}}{\pgfqpoint{3.253109in}{3.770829in}}%
\pgfpathcurveto{\pgfqpoint{3.248991in}{3.770829in}}{\pgfqpoint{3.245041in}{3.769193in}}{\pgfqpoint{3.242129in}{3.766281in}}%
\pgfpathcurveto{\pgfqpoint{3.239217in}{3.763369in}}{\pgfqpoint{3.237581in}{3.759419in}}{\pgfqpoint{3.237581in}{3.755301in}}%
\pgfpathcurveto{\pgfqpoint{3.237581in}{3.751183in}}{\pgfqpoint{3.239217in}{3.747233in}}{\pgfqpoint{3.242129in}{3.744321in}}%
\pgfpathcurveto{\pgfqpoint{3.245041in}{3.741409in}}{\pgfqpoint{3.248991in}{3.739773in}}{\pgfqpoint{3.253109in}{3.739773in}}%
\pgfpathclose%
\pgfusepath{fill}%
\end{pgfscope}%
\begin{pgfscope}%
\pgfpathrectangle{\pgfqpoint{0.887500in}{0.275000in}}{\pgfqpoint{4.225000in}{4.225000in}}%
\pgfusepath{clip}%
\pgfsetbuttcap%
\pgfsetroundjoin%
\definecolor{currentfill}{rgb}{0.000000,0.000000,0.000000}%
\pgfsetfillcolor{currentfill}%
\pgfsetfillopacity{0.800000}%
\pgfsetlinewidth{0.000000pt}%
\definecolor{currentstroke}{rgb}{0.000000,0.000000,0.000000}%
\pgfsetstrokecolor{currentstroke}%
\pgfsetstrokeopacity{0.800000}%
\pgfsetdash{}{0pt}%
\pgfpathmoveto{\pgfqpoint{2.864355in}{3.585345in}}%
\pgfpathcurveto{\pgfqpoint{2.868473in}{3.585345in}}{\pgfqpoint{2.872423in}{3.586981in}}{\pgfqpoint{2.875335in}{3.589893in}}%
\pgfpathcurveto{\pgfqpoint{2.878247in}{3.592805in}}{\pgfqpoint{2.879883in}{3.596755in}}{\pgfqpoint{2.879883in}{3.600873in}}%
\pgfpathcurveto{\pgfqpoint{2.879883in}{3.604991in}}{\pgfqpoint{2.878247in}{3.608941in}}{\pgfqpoint{2.875335in}{3.611853in}}%
\pgfpathcurveto{\pgfqpoint{2.872423in}{3.614765in}}{\pgfqpoint{2.868473in}{3.616402in}}{\pgfqpoint{2.864355in}{3.616402in}}%
\pgfpathcurveto{\pgfqpoint{2.860237in}{3.616402in}}{\pgfqpoint{2.856287in}{3.614765in}}{\pgfqpoint{2.853375in}{3.611853in}}%
\pgfpathcurveto{\pgfqpoint{2.850463in}{3.608941in}}{\pgfqpoint{2.848827in}{3.604991in}}{\pgfqpoint{2.848827in}{3.600873in}}%
\pgfpathcurveto{\pgfqpoint{2.848827in}{3.596755in}}{\pgfqpoint{2.850463in}{3.592805in}}{\pgfqpoint{2.853375in}{3.589893in}}%
\pgfpathcurveto{\pgfqpoint{2.856287in}{3.586981in}}{\pgfqpoint{2.860237in}{3.585345in}}{\pgfqpoint{2.864355in}{3.585345in}}%
\pgfpathclose%
\pgfusepath{fill}%
\end{pgfscope}%
\begin{pgfscope}%
\pgfpathrectangle{\pgfqpoint{0.887500in}{0.275000in}}{\pgfqpoint{4.225000in}{4.225000in}}%
\pgfusepath{clip}%
\pgfsetbuttcap%
\pgfsetroundjoin%
\definecolor{currentfill}{rgb}{0.000000,0.000000,0.000000}%
\pgfsetfillcolor{currentfill}%
\pgfsetfillopacity{0.800000}%
\pgfsetlinewidth{0.000000pt}%
\definecolor{currentstroke}{rgb}{0.000000,0.000000,0.000000}%
\pgfsetstrokecolor{currentstroke}%
\pgfsetstrokeopacity{0.800000}%
\pgfsetdash{}{0pt}%
\pgfpathmoveto{\pgfqpoint{3.533908in}{3.836176in}}%
\pgfpathcurveto{\pgfqpoint{3.538026in}{3.836176in}}{\pgfqpoint{3.541976in}{3.837812in}}{\pgfqpoint{3.544888in}{3.840724in}}%
\pgfpathcurveto{\pgfqpoint{3.547800in}{3.843636in}}{\pgfqpoint{3.549436in}{3.847586in}}{\pgfqpoint{3.549436in}{3.851704in}}%
\pgfpathcurveto{\pgfqpoint{3.549436in}{3.855822in}}{\pgfqpoint{3.547800in}{3.859772in}}{\pgfqpoint{3.544888in}{3.862684in}}%
\pgfpathcurveto{\pgfqpoint{3.541976in}{3.865596in}}{\pgfqpoint{3.538026in}{3.867232in}}{\pgfqpoint{3.533908in}{3.867232in}}%
\pgfpathcurveto{\pgfqpoint{3.529790in}{3.867232in}}{\pgfqpoint{3.525840in}{3.865596in}}{\pgfqpoint{3.522928in}{3.862684in}}%
\pgfpathcurveto{\pgfqpoint{3.520016in}{3.859772in}}{\pgfqpoint{3.518379in}{3.855822in}}{\pgfqpoint{3.518379in}{3.851704in}}%
\pgfpathcurveto{\pgfqpoint{3.518379in}{3.847586in}}{\pgfqpoint{3.520016in}{3.843636in}}{\pgfqpoint{3.522928in}{3.840724in}}%
\pgfpathcurveto{\pgfqpoint{3.525840in}{3.837812in}}{\pgfqpoint{3.529790in}{3.836176in}}{\pgfqpoint{3.533908in}{3.836176in}}%
\pgfpathclose%
\pgfusepath{fill}%
\end{pgfscope}%
\begin{pgfscope}%
\pgfpathrectangle{\pgfqpoint{0.887500in}{0.275000in}}{\pgfqpoint{4.225000in}{4.225000in}}%
\pgfusepath{clip}%
\pgfsetbuttcap%
\pgfsetroundjoin%
\definecolor{currentfill}{rgb}{0.000000,0.000000,0.000000}%
\pgfsetfillcolor{currentfill}%
\pgfsetfillopacity{0.800000}%
\pgfsetlinewidth{0.000000pt}%
\definecolor{currentstroke}{rgb}{0.000000,0.000000,0.000000}%
\pgfsetstrokecolor{currentstroke}%
\pgfsetstrokeopacity{0.800000}%
\pgfsetdash{}{0pt}%
\pgfpathmoveto{\pgfqpoint{3.144585in}{3.679642in}}%
\pgfpathcurveto{\pgfqpoint{3.148703in}{3.679642in}}{\pgfqpoint{3.152653in}{3.681278in}}{\pgfqpoint{3.155565in}{3.684190in}}%
\pgfpathcurveto{\pgfqpoint{3.158477in}{3.687102in}}{\pgfqpoint{3.160113in}{3.691052in}}{\pgfqpoint{3.160113in}{3.695171in}}%
\pgfpathcurveto{\pgfqpoint{3.160113in}{3.699289in}}{\pgfqpoint{3.158477in}{3.703239in}}{\pgfqpoint{3.155565in}{3.706151in}}%
\pgfpathcurveto{\pgfqpoint{3.152653in}{3.709063in}}{\pgfqpoint{3.148703in}{3.710699in}}{\pgfqpoint{3.144585in}{3.710699in}}%
\pgfpathcurveto{\pgfqpoint{3.140467in}{3.710699in}}{\pgfqpoint{3.136517in}{3.709063in}}{\pgfqpoint{3.133605in}{3.706151in}}%
\pgfpathcurveto{\pgfqpoint{3.130693in}{3.703239in}}{\pgfqpoint{3.129056in}{3.699289in}}{\pgfqpoint{3.129056in}{3.695171in}}%
\pgfpathcurveto{\pgfqpoint{3.129056in}{3.691052in}}{\pgfqpoint{3.130693in}{3.687102in}}{\pgfqpoint{3.133605in}{3.684190in}}%
\pgfpathcurveto{\pgfqpoint{3.136517in}{3.681278in}}{\pgfqpoint{3.140467in}{3.679642in}}{\pgfqpoint{3.144585in}{3.679642in}}%
\pgfpathclose%
\pgfusepath{fill}%
\end{pgfscope}%
\begin{pgfscope}%
\pgfpathrectangle{\pgfqpoint{0.887500in}{0.275000in}}{\pgfqpoint{4.225000in}{4.225000in}}%
\pgfusepath{clip}%
\pgfsetbuttcap%
\pgfsetroundjoin%
\definecolor{currentfill}{rgb}{0.000000,0.000000,0.000000}%
\pgfsetfillcolor{currentfill}%
\pgfsetfillopacity{0.800000}%
\pgfsetlinewidth{0.000000pt}%
\definecolor{currentstroke}{rgb}{0.000000,0.000000,0.000000}%
\pgfsetstrokecolor{currentstroke}%
\pgfsetstrokeopacity{0.800000}%
\pgfsetdash{}{0pt}%
\pgfpathmoveto{\pgfqpoint{2.755211in}{3.522662in}}%
\pgfpathcurveto{\pgfqpoint{2.759329in}{3.522662in}}{\pgfqpoint{2.763279in}{3.524298in}}{\pgfqpoint{2.766191in}{3.527210in}}%
\pgfpathcurveto{\pgfqpoint{2.769103in}{3.530122in}}{\pgfqpoint{2.770740in}{3.534072in}}{\pgfqpoint{2.770740in}{3.538190in}}%
\pgfpathcurveto{\pgfqpoint{2.770740in}{3.542308in}}{\pgfqpoint{2.769103in}{3.546258in}}{\pgfqpoint{2.766191in}{3.549170in}}%
\pgfpathcurveto{\pgfqpoint{2.763279in}{3.552082in}}{\pgfqpoint{2.759329in}{3.553718in}}{\pgfqpoint{2.755211in}{3.553718in}}%
\pgfpathcurveto{\pgfqpoint{2.751093in}{3.553718in}}{\pgfqpoint{2.747143in}{3.552082in}}{\pgfqpoint{2.744231in}{3.549170in}}%
\pgfpathcurveto{\pgfqpoint{2.741319in}{3.546258in}}{\pgfqpoint{2.739683in}{3.542308in}}{\pgfqpoint{2.739683in}{3.538190in}}%
\pgfpathcurveto{\pgfqpoint{2.739683in}{3.534072in}}{\pgfqpoint{2.741319in}{3.530122in}}{\pgfqpoint{2.744231in}{3.527210in}}%
\pgfpathcurveto{\pgfqpoint{2.747143in}{3.524298in}}{\pgfqpoint{2.751093in}{3.522662in}}{\pgfqpoint{2.755211in}{3.522662in}}%
\pgfpathclose%
\pgfusepath{fill}%
\end{pgfscope}%
\begin{pgfscope}%
\pgfpathrectangle{\pgfqpoint{0.887500in}{0.275000in}}{\pgfqpoint{4.225000in}{4.225000in}}%
\pgfusepath{clip}%
\pgfsetbuttcap%
\pgfsetroundjoin%
\definecolor{currentfill}{rgb}{0.000000,0.000000,0.000000}%
\pgfsetfillcolor{currentfill}%
\pgfsetfillopacity{0.800000}%
\pgfsetlinewidth{0.000000pt}%
\definecolor{currentstroke}{rgb}{0.000000,0.000000,0.000000}%
\pgfsetstrokecolor{currentstroke}%
\pgfsetstrokeopacity{0.800000}%
\pgfsetdash{}{0pt}%
\pgfpathmoveto{\pgfqpoint{2.537228in}{3.294054in}}%
\pgfpathcurveto{\pgfqpoint{2.541346in}{3.294054in}}{\pgfqpoint{2.545296in}{3.295690in}}{\pgfqpoint{2.548208in}{3.298602in}}%
\pgfpathcurveto{\pgfqpoint{2.551120in}{3.301514in}}{\pgfqpoint{2.552756in}{3.305464in}}{\pgfqpoint{2.552756in}{3.309582in}}%
\pgfpathcurveto{\pgfqpoint{2.552756in}{3.313701in}}{\pgfqpoint{2.551120in}{3.317651in}}{\pgfqpoint{2.548208in}{3.320563in}}%
\pgfpathcurveto{\pgfqpoint{2.545296in}{3.323475in}}{\pgfqpoint{2.541346in}{3.325111in}}{\pgfqpoint{2.537228in}{3.325111in}}%
\pgfpathcurveto{\pgfqpoint{2.533109in}{3.325111in}}{\pgfqpoint{2.529159in}{3.323475in}}{\pgfqpoint{2.526247in}{3.320563in}}%
\pgfpathcurveto{\pgfqpoint{2.523335in}{3.317651in}}{\pgfqpoint{2.521699in}{3.313701in}}{\pgfqpoint{2.521699in}{3.309582in}}%
\pgfpathcurveto{\pgfqpoint{2.521699in}{3.305464in}}{\pgfqpoint{2.523335in}{3.301514in}}{\pgfqpoint{2.526247in}{3.298602in}}%
\pgfpathcurveto{\pgfqpoint{2.529159in}{3.295690in}}{\pgfqpoint{2.533109in}{3.294054in}}{\pgfqpoint{2.537228in}{3.294054in}}%
\pgfpathclose%
\pgfusepath{fill}%
\end{pgfscope}%
\begin{pgfscope}%
\pgfpathrectangle{\pgfqpoint{0.887500in}{0.275000in}}{\pgfqpoint{4.225000in}{4.225000in}}%
\pgfusepath{clip}%
\pgfsetbuttcap%
\pgfsetroundjoin%
\definecolor{currentfill}{rgb}{0.000000,0.000000,0.000000}%
\pgfsetfillcolor{currentfill}%
\pgfsetfillopacity{0.800000}%
\pgfsetlinewidth{0.000000pt}%
\definecolor{currentstroke}{rgb}{0.000000,0.000000,0.000000}%
\pgfsetstrokecolor{currentstroke}%
\pgfsetstrokeopacity{0.800000}%
\pgfsetdash{}{0pt}%
\pgfpathmoveto{\pgfqpoint{3.035711in}{3.620686in}}%
\pgfpathcurveto{\pgfqpoint{3.039829in}{3.620686in}}{\pgfqpoint{3.043779in}{3.622322in}}{\pgfqpoint{3.046691in}{3.625234in}}%
\pgfpathcurveto{\pgfqpoint{3.049603in}{3.628146in}}{\pgfqpoint{3.051239in}{3.632096in}}{\pgfqpoint{3.051239in}{3.636215in}}%
\pgfpathcurveto{\pgfqpoint{3.051239in}{3.640333in}}{\pgfqpoint{3.049603in}{3.644283in}}{\pgfqpoint{3.046691in}{3.647195in}}%
\pgfpathcurveto{\pgfqpoint{3.043779in}{3.650107in}}{\pgfqpoint{3.039829in}{3.651743in}}{\pgfqpoint{3.035711in}{3.651743in}}%
\pgfpathcurveto{\pgfqpoint{3.031592in}{3.651743in}}{\pgfqpoint{3.027642in}{3.650107in}}{\pgfqpoint{3.024730in}{3.647195in}}%
\pgfpathcurveto{\pgfqpoint{3.021818in}{3.644283in}}{\pgfqpoint{3.020182in}{3.640333in}}{\pgfqpoint{3.020182in}{3.636215in}}%
\pgfpathcurveto{\pgfqpoint{3.020182in}{3.632096in}}{\pgfqpoint{3.021818in}{3.628146in}}{\pgfqpoint{3.024730in}{3.625234in}}%
\pgfpathcurveto{\pgfqpoint{3.027642in}{3.622322in}}{\pgfqpoint{3.031592in}{3.620686in}}{\pgfqpoint{3.035711in}{3.620686in}}%
\pgfpathclose%
\pgfusepath{fill}%
\end{pgfscope}%
\begin{pgfscope}%
\pgfpathrectangle{\pgfqpoint{0.887500in}{0.275000in}}{\pgfqpoint{4.225000in}{4.225000in}}%
\pgfusepath{clip}%
\pgfsetbuttcap%
\pgfsetroundjoin%
\definecolor{currentfill}{rgb}{0.000000,0.000000,0.000000}%
\pgfsetfillcolor{currentfill}%
\pgfsetfillopacity{0.800000}%
\pgfsetlinewidth{0.000000pt}%
\definecolor{currentstroke}{rgb}{0.000000,0.000000,0.000000}%
\pgfsetstrokecolor{currentstroke}%
\pgfsetstrokeopacity{0.800000}%
\pgfsetdash{}{0pt}%
\pgfpathmoveto{\pgfqpoint{3.425772in}{3.787778in}}%
\pgfpathcurveto{\pgfqpoint{3.429890in}{3.787778in}}{\pgfqpoint{3.433840in}{3.789414in}}{\pgfqpoint{3.436752in}{3.792326in}}%
\pgfpathcurveto{\pgfqpoint{3.439664in}{3.795238in}}{\pgfqpoint{3.441301in}{3.799188in}}{\pgfqpoint{3.441301in}{3.803306in}}%
\pgfpathcurveto{\pgfqpoint{3.441301in}{3.807424in}}{\pgfqpoint{3.439664in}{3.811374in}}{\pgfqpoint{3.436752in}{3.814286in}}%
\pgfpathcurveto{\pgfqpoint{3.433840in}{3.817198in}}{\pgfqpoint{3.429890in}{3.818834in}}{\pgfqpoint{3.425772in}{3.818834in}}%
\pgfpathcurveto{\pgfqpoint{3.421654in}{3.818834in}}{\pgfqpoint{3.417704in}{3.817198in}}{\pgfqpoint{3.414792in}{3.814286in}}%
\pgfpathcurveto{\pgfqpoint{3.411880in}{3.811374in}}{\pgfqpoint{3.410244in}{3.807424in}}{\pgfqpoint{3.410244in}{3.803306in}}%
\pgfpathcurveto{\pgfqpoint{3.410244in}{3.799188in}}{\pgfqpoint{3.411880in}{3.795238in}}{\pgfqpoint{3.414792in}{3.792326in}}%
\pgfpathcurveto{\pgfqpoint{3.417704in}{3.789414in}}{\pgfqpoint{3.421654in}{3.787778in}}{\pgfqpoint{3.425772in}{3.787778in}}%
\pgfpathclose%
\pgfusepath{fill}%
\end{pgfscope}%
\begin{pgfscope}%
\pgfpathrectangle{\pgfqpoint{0.887500in}{0.275000in}}{\pgfqpoint{4.225000in}{4.225000in}}%
\pgfusepath{clip}%
\pgfsetbuttcap%
\pgfsetroundjoin%
\definecolor{currentfill}{rgb}{0.000000,0.000000,0.000000}%
\pgfsetfillcolor{currentfill}%
\pgfsetfillopacity{0.800000}%
\pgfsetlinewidth{0.000000pt}%
\definecolor{currentstroke}{rgb}{0.000000,0.000000,0.000000}%
\pgfsetstrokecolor{currentstroke}%
\pgfsetstrokeopacity{0.800000}%
\pgfsetdash{}{0pt}%
\pgfpathmoveto{\pgfqpoint{2.427166in}{3.242308in}}%
\pgfpathcurveto{\pgfqpoint{2.431284in}{3.242308in}}{\pgfqpoint{2.435234in}{3.243944in}}{\pgfqpoint{2.438146in}{3.246856in}}%
\pgfpathcurveto{\pgfqpoint{2.441058in}{3.249768in}}{\pgfqpoint{2.442694in}{3.253718in}}{\pgfqpoint{2.442694in}{3.257836in}}%
\pgfpathcurveto{\pgfqpoint{2.442694in}{3.261955in}}{\pgfqpoint{2.441058in}{3.265905in}}{\pgfqpoint{2.438146in}{3.268817in}}%
\pgfpathcurveto{\pgfqpoint{2.435234in}{3.271729in}}{\pgfqpoint{2.431284in}{3.273365in}}{\pgfqpoint{2.427166in}{3.273365in}}%
\pgfpathcurveto{\pgfqpoint{2.423048in}{3.273365in}}{\pgfqpoint{2.419098in}{3.271729in}}{\pgfqpoint{2.416186in}{3.268817in}}%
\pgfpathcurveto{\pgfqpoint{2.413274in}{3.265905in}}{\pgfqpoint{2.411638in}{3.261955in}}{\pgfqpoint{2.411638in}{3.257836in}}%
\pgfpathcurveto{\pgfqpoint{2.411638in}{3.253718in}}{\pgfqpoint{2.413274in}{3.249768in}}{\pgfqpoint{2.416186in}{3.246856in}}%
\pgfpathcurveto{\pgfqpoint{2.419098in}{3.243944in}}{\pgfqpoint{2.423048in}{3.242308in}}{\pgfqpoint{2.427166in}{3.242308in}}%
\pgfpathclose%
\pgfusepath{fill}%
\end{pgfscope}%
\begin{pgfscope}%
\pgfpathrectangle{\pgfqpoint{0.887500in}{0.275000in}}{\pgfqpoint{4.225000in}{4.225000in}}%
\pgfusepath{clip}%
\pgfsetbuttcap%
\pgfsetroundjoin%
\definecolor{currentfill}{rgb}{0.000000,0.000000,0.000000}%
\pgfsetfillcolor{currentfill}%
\pgfsetfillopacity{0.800000}%
\pgfsetlinewidth{0.000000pt}%
\definecolor{currentstroke}{rgb}{0.000000,0.000000,0.000000}%
\pgfsetstrokecolor{currentstroke}%
\pgfsetstrokeopacity{0.800000}%
\pgfsetdash{}{0pt}%
\pgfpathmoveto{\pgfqpoint{3.707481in}{3.864433in}}%
\pgfpathcurveto{\pgfqpoint{3.711599in}{3.864433in}}{\pgfqpoint{3.715549in}{3.866069in}}{\pgfqpoint{3.718461in}{3.868981in}}%
\pgfpathcurveto{\pgfqpoint{3.721373in}{3.871893in}}{\pgfqpoint{3.723010in}{3.875843in}}{\pgfqpoint{3.723010in}{3.879961in}}%
\pgfpathcurveto{\pgfqpoint{3.723010in}{3.884079in}}{\pgfqpoint{3.721373in}{3.888029in}}{\pgfqpoint{3.718461in}{3.890941in}}%
\pgfpathcurveto{\pgfqpoint{3.715549in}{3.893853in}}{\pgfqpoint{3.711599in}{3.895489in}}{\pgfqpoint{3.707481in}{3.895489in}}%
\pgfpathcurveto{\pgfqpoint{3.703363in}{3.895489in}}{\pgfqpoint{3.699413in}{3.893853in}}{\pgfqpoint{3.696501in}{3.890941in}}%
\pgfpathcurveto{\pgfqpoint{3.693589in}{3.888029in}}{\pgfqpoint{3.691953in}{3.884079in}}{\pgfqpoint{3.691953in}{3.879961in}}%
\pgfpathcurveto{\pgfqpoint{3.691953in}{3.875843in}}{\pgfqpoint{3.693589in}{3.871893in}}{\pgfqpoint{3.696501in}{3.868981in}}%
\pgfpathcurveto{\pgfqpoint{3.699413in}{3.866069in}}{\pgfqpoint{3.703363in}{3.864433in}}{\pgfqpoint{3.707481in}{3.864433in}}%
\pgfpathclose%
\pgfusepath{fill}%
\end{pgfscope}%
\begin{pgfscope}%
\pgfpathrectangle{\pgfqpoint{0.887500in}{0.275000in}}{\pgfqpoint{4.225000in}{4.225000in}}%
\pgfusepath{clip}%
\pgfsetbuttcap%
\pgfsetroundjoin%
\definecolor{currentfill}{rgb}{0.000000,0.000000,0.000000}%
\pgfsetfillcolor{currentfill}%
\pgfsetfillopacity{0.800000}%
\pgfsetlinewidth{0.000000pt}%
\definecolor{currentstroke}{rgb}{0.000000,0.000000,0.000000}%
\pgfsetstrokecolor{currentstroke}%
\pgfsetstrokeopacity{0.800000}%
\pgfsetdash{}{0pt}%
\pgfpathmoveto{\pgfqpoint{2.926458in}{3.567863in}}%
\pgfpathcurveto{\pgfqpoint{2.930576in}{3.567863in}}{\pgfqpoint{2.934526in}{3.569499in}}{\pgfqpoint{2.937438in}{3.572411in}}%
\pgfpathcurveto{\pgfqpoint{2.940350in}{3.575323in}}{\pgfqpoint{2.941987in}{3.579273in}}{\pgfqpoint{2.941987in}{3.583392in}}%
\pgfpathcurveto{\pgfqpoint{2.941987in}{3.587510in}}{\pgfqpoint{2.940350in}{3.591460in}}{\pgfqpoint{2.937438in}{3.594372in}}%
\pgfpathcurveto{\pgfqpoint{2.934526in}{3.597284in}}{\pgfqpoint{2.930576in}{3.598920in}}{\pgfqpoint{2.926458in}{3.598920in}}%
\pgfpathcurveto{\pgfqpoint{2.922340in}{3.598920in}}{\pgfqpoint{2.918390in}{3.597284in}}{\pgfqpoint{2.915478in}{3.594372in}}%
\pgfpathcurveto{\pgfqpoint{2.912566in}{3.591460in}}{\pgfqpoint{2.910930in}{3.587510in}}{\pgfqpoint{2.910930in}{3.583392in}}%
\pgfpathcurveto{\pgfqpoint{2.910930in}{3.579273in}}{\pgfqpoint{2.912566in}{3.575323in}}{\pgfqpoint{2.915478in}{3.572411in}}%
\pgfpathcurveto{\pgfqpoint{2.918390in}{3.569499in}}{\pgfqpoint{2.922340in}{3.567863in}}{\pgfqpoint{2.926458in}{3.567863in}}%
\pgfpathclose%
\pgfusepath{fill}%
\end{pgfscope}%
\begin{pgfscope}%
\pgfpathrectangle{\pgfqpoint{0.887500in}{0.275000in}}{\pgfqpoint{4.225000in}{4.225000in}}%
\pgfusepath{clip}%
\pgfsetbuttcap%
\pgfsetroundjoin%
\definecolor{currentfill}{rgb}{0.000000,0.000000,0.000000}%
\pgfsetfillcolor{currentfill}%
\pgfsetfillopacity{0.800000}%
\pgfsetlinewidth{0.000000pt}%
\definecolor{currentstroke}{rgb}{0.000000,0.000000,0.000000}%
\pgfsetstrokecolor{currentstroke}%
\pgfsetstrokeopacity{0.800000}%
\pgfsetdash{}{0pt}%
\pgfpathmoveto{\pgfqpoint{3.317155in}{3.727862in}}%
\pgfpathcurveto{\pgfqpoint{3.321273in}{3.727862in}}{\pgfqpoint{3.325223in}{3.729498in}}{\pgfqpoint{3.328135in}{3.732410in}}%
\pgfpathcurveto{\pgfqpoint{3.331047in}{3.735322in}}{\pgfqpoint{3.332683in}{3.739272in}}{\pgfqpoint{3.332683in}{3.743390in}}%
\pgfpathcurveto{\pgfqpoint{3.332683in}{3.747508in}}{\pgfqpoint{3.331047in}{3.751458in}}{\pgfqpoint{3.328135in}{3.754370in}}%
\pgfpathcurveto{\pgfqpoint{3.325223in}{3.757282in}}{\pgfqpoint{3.321273in}{3.758918in}}{\pgfqpoint{3.317155in}{3.758918in}}%
\pgfpathcurveto{\pgfqpoint{3.313037in}{3.758918in}}{\pgfqpoint{3.309087in}{3.757282in}}{\pgfqpoint{3.306175in}{3.754370in}}%
\pgfpathcurveto{\pgfqpoint{3.303263in}{3.751458in}}{\pgfqpoint{3.301626in}{3.747508in}}{\pgfqpoint{3.301626in}{3.743390in}}%
\pgfpathcurveto{\pgfqpoint{3.301626in}{3.739272in}}{\pgfqpoint{3.303263in}{3.735322in}}{\pgfqpoint{3.306175in}{3.732410in}}%
\pgfpathcurveto{\pgfqpoint{3.309087in}{3.729498in}}{\pgfqpoint{3.313037in}{3.727862in}}{\pgfqpoint{3.317155in}{3.727862in}}%
\pgfpathclose%
\pgfusepath{fill}%
\end{pgfscope}%
\begin{pgfscope}%
\pgfpathrectangle{\pgfqpoint{0.887500in}{0.275000in}}{\pgfqpoint{4.225000in}{4.225000in}}%
\pgfusepath{clip}%
\pgfsetbuttcap%
\pgfsetroundjoin%
\definecolor{currentfill}{rgb}{0.000000,0.000000,0.000000}%
\pgfsetfillcolor{currentfill}%
\pgfsetfillopacity{0.800000}%
\pgfsetlinewidth{0.000000pt}%
\definecolor{currentstroke}{rgb}{0.000000,0.000000,0.000000}%
\pgfsetstrokecolor{currentstroke}%
\pgfsetstrokeopacity{0.800000}%
\pgfsetdash{}{0pt}%
\pgfpathmoveto{\pgfqpoint{2.598723in}{3.197825in}}%
\pgfpathcurveto{\pgfqpoint{2.602841in}{3.197825in}}{\pgfqpoint{2.606791in}{3.199461in}}{\pgfqpoint{2.609703in}{3.202373in}}%
\pgfpathcurveto{\pgfqpoint{2.612615in}{3.205285in}}{\pgfqpoint{2.614251in}{3.209235in}}{\pgfqpoint{2.614251in}{3.213353in}}%
\pgfpathcurveto{\pgfqpoint{2.614251in}{3.217471in}}{\pgfqpoint{2.612615in}{3.221421in}}{\pgfqpoint{2.609703in}{3.224333in}}%
\pgfpathcurveto{\pgfqpoint{2.606791in}{3.227245in}}{\pgfqpoint{2.602841in}{3.228881in}}{\pgfqpoint{2.598723in}{3.228881in}}%
\pgfpathcurveto{\pgfqpoint{2.594605in}{3.228881in}}{\pgfqpoint{2.590655in}{3.227245in}}{\pgfqpoint{2.587743in}{3.224333in}}%
\pgfpathcurveto{\pgfqpoint{2.584831in}{3.221421in}}{\pgfqpoint{2.583195in}{3.217471in}}{\pgfqpoint{2.583195in}{3.213353in}}%
\pgfpathcurveto{\pgfqpoint{2.583195in}{3.209235in}}{\pgfqpoint{2.584831in}{3.205285in}}{\pgfqpoint{2.587743in}{3.202373in}}%
\pgfpathcurveto{\pgfqpoint{2.590655in}{3.199461in}}{\pgfqpoint{2.594605in}{3.197825in}}{\pgfqpoint{2.598723in}{3.197825in}}%
\pgfpathclose%
\pgfusepath{fill}%
\end{pgfscope}%
\begin{pgfscope}%
\pgfpathrectangle{\pgfqpoint{0.887500in}{0.275000in}}{\pgfqpoint{4.225000in}{4.225000in}}%
\pgfusepath{clip}%
\pgfsetbuttcap%
\pgfsetroundjoin%
\definecolor{currentfill}{rgb}{0.000000,0.000000,0.000000}%
\pgfsetfillcolor{currentfill}%
\pgfsetfillopacity{0.800000}%
\pgfsetlinewidth{0.000000pt}%
\definecolor{currentstroke}{rgb}{0.000000,0.000000,0.000000}%
\pgfsetstrokecolor{currentstroke}%
\pgfsetstrokeopacity{0.800000}%
\pgfsetdash{}{0pt}%
\pgfpathmoveto{\pgfqpoint{3.599400in}{3.819220in}}%
\pgfpathcurveto{\pgfqpoint{3.603518in}{3.819220in}}{\pgfqpoint{3.607468in}{3.820856in}}{\pgfqpoint{3.610380in}{3.823768in}}%
\pgfpathcurveto{\pgfqpoint{3.613292in}{3.826680in}}{\pgfqpoint{3.614928in}{3.830630in}}{\pgfqpoint{3.614928in}{3.834748in}}%
\pgfpathcurveto{\pgfqpoint{3.614928in}{3.838866in}}{\pgfqpoint{3.613292in}{3.842816in}}{\pgfqpoint{3.610380in}{3.845728in}}%
\pgfpathcurveto{\pgfqpoint{3.607468in}{3.848640in}}{\pgfqpoint{3.603518in}{3.850276in}}{\pgfqpoint{3.599400in}{3.850276in}}%
\pgfpathcurveto{\pgfqpoint{3.595282in}{3.850276in}}{\pgfqpoint{3.591332in}{3.848640in}}{\pgfqpoint{3.588420in}{3.845728in}}%
\pgfpathcurveto{\pgfqpoint{3.585508in}{3.842816in}}{\pgfqpoint{3.583872in}{3.838866in}}{\pgfqpoint{3.583872in}{3.834748in}}%
\pgfpathcurveto{\pgfqpoint{3.583872in}{3.830630in}}{\pgfqpoint{3.585508in}{3.826680in}}{\pgfqpoint{3.588420in}{3.823768in}}%
\pgfpathcurveto{\pgfqpoint{3.591332in}{3.820856in}}{\pgfqpoint{3.595282in}{3.819220in}}{\pgfqpoint{3.599400in}{3.819220in}}%
\pgfpathclose%
\pgfusepath{fill}%
\end{pgfscope}%
\begin{pgfscope}%
\pgfpathrectangle{\pgfqpoint{0.887500in}{0.275000in}}{\pgfqpoint{4.225000in}{4.225000in}}%
\pgfusepath{clip}%
\pgfsetbuttcap%
\pgfsetroundjoin%
\definecolor{currentfill}{rgb}{0.000000,0.000000,0.000000}%
\pgfsetfillcolor{currentfill}%
\pgfsetfillopacity{0.800000}%
\pgfsetlinewidth{0.000000pt}%
\definecolor{currentstroke}{rgb}{0.000000,0.000000,0.000000}%
\pgfsetstrokecolor{currentstroke}%
\pgfsetstrokeopacity{0.800000}%
\pgfsetdash{}{0pt}%
\pgfpathmoveto{\pgfqpoint{2.316250in}{3.214806in}}%
\pgfpathcurveto{\pgfqpoint{2.320369in}{3.214806in}}{\pgfqpoint{2.324319in}{3.216442in}}{\pgfqpoint{2.327231in}{3.219354in}}%
\pgfpathcurveto{\pgfqpoint{2.330143in}{3.222266in}}{\pgfqpoint{2.331779in}{3.226216in}}{\pgfqpoint{2.331779in}{3.230334in}}%
\pgfpathcurveto{\pgfqpoint{2.331779in}{3.234452in}}{\pgfqpoint{2.330143in}{3.238402in}}{\pgfqpoint{2.327231in}{3.241314in}}%
\pgfpathcurveto{\pgfqpoint{2.324319in}{3.244226in}}{\pgfqpoint{2.320369in}{3.245862in}}{\pgfqpoint{2.316250in}{3.245862in}}%
\pgfpathcurveto{\pgfqpoint{2.312132in}{3.245862in}}{\pgfqpoint{2.308182in}{3.244226in}}{\pgfqpoint{2.305270in}{3.241314in}}%
\pgfpathcurveto{\pgfqpoint{2.302358in}{3.238402in}}{\pgfqpoint{2.300722in}{3.234452in}}{\pgfqpoint{2.300722in}{3.230334in}}%
\pgfpathcurveto{\pgfqpoint{2.300722in}{3.226216in}}{\pgfqpoint{2.302358in}{3.222266in}}{\pgfqpoint{2.305270in}{3.219354in}}%
\pgfpathcurveto{\pgfqpoint{2.308182in}{3.216442in}}{\pgfqpoint{2.312132in}{3.214806in}}{\pgfqpoint{2.316250in}{3.214806in}}%
\pgfpathclose%
\pgfusepath{fill}%
\end{pgfscope}%
\begin{pgfscope}%
\pgfpathrectangle{\pgfqpoint{0.887500in}{0.275000in}}{\pgfqpoint{4.225000in}{4.225000in}}%
\pgfusepath{clip}%
\pgfsetbuttcap%
\pgfsetroundjoin%
\definecolor{currentfill}{rgb}{0.000000,0.000000,0.000000}%
\pgfsetfillcolor{currentfill}%
\pgfsetfillopacity{0.800000}%
\pgfsetlinewidth{0.000000pt}%
\definecolor{currentstroke}{rgb}{0.000000,0.000000,0.000000}%
\pgfsetstrokecolor{currentstroke}%
\pgfsetstrokeopacity{0.800000}%
\pgfsetdash{}{0pt}%
\pgfpathmoveto{\pgfqpoint{3.208178in}{3.665691in}}%
\pgfpathcurveto{\pgfqpoint{3.212296in}{3.665691in}}{\pgfqpoint{3.216246in}{3.667327in}}{\pgfqpoint{3.219158in}{3.670239in}}%
\pgfpathcurveto{\pgfqpoint{3.222070in}{3.673151in}}{\pgfqpoint{3.223706in}{3.677101in}}{\pgfqpoint{3.223706in}{3.681219in}}%
\pgfpathcurveto{\pgfqpoint{3.223706in}{3.685337in}}{\pgfqpoint{3.222070in}{3.689287in}}{\pgfqpoint{3.219158in}{3.692199in}}%
\pgfpathcurveto{\pgfqpoint{3.216246in}{3.695111in}}{\pgfqpoint{3.212296in}{3.696747in}}{\pgfqpoint{3.208178in}{3.696747in}}%
\pgfpathcurveto{\pgfqpoint{3.204059in}{3.696747in}}{\pgfqpoint{3.200109in}{3.695111in}}{\pgfqpoint{3.197197in}{3.692199in}}%
\pgfpathcurveto{\pgfqpoint{3.194285in}{3.689287in}}{\pgfqpoint{3.192649in}{3.685337in}}{\pgfqpoint{3.192649in}{3.681219in}}%
\pgfpathcurveto{\pgfqpoint{3.192649in}{3.677101in}}{\pgfqpoint{3.194285in}{3.673151in}}{\pgfqpoint{3.197197in}{3.670239in}}%
\pgfpathcurveto{\pgfqpoint{3.200109in}{3.667327in}}{\pgfqpoint{3.204059in}{3.665691in}}{\pgfqpoint{3.208178in}{3.665691in}}%
\pgfpathclose%
\pgfusepath{fill}%
\end{pgfscope}%
\begin{pgfscope}%
\pgfpathrectangle{\pgfqpoint{0.887500in}{0.275000in}}{\pgfqpoint{4.225000in}{4.225000in}}%
\pgfusepath{clip}%
\pgfsetbuttcap%
\pgfsetroundjoin%
\definecolor{currentfill}{rgb}{0.000000,0.000000,0.000000}%
\pgfsetfillcolor{currentfill}%
\pgfsetfillopacity{0.800000}%
\pgfsetlinewidth{0.000000pt}%
\definecolor{currentstroke}{rgb}{0.000000,0.000000,0.000000}%
\pgfsetstrokecolor{currentstroke}%
\pgfsetstrokeopacity{0.800000}%
\pgfsetdash{}{0pt}%
\pgfpathmoveto{\pgfqpoint{2.816849in}{3.508691in}}%
\pgfpathcurveto{\pgfqpoint{2.820967in}{3.508691in}}{\pgfqpoint{2.824917in}{3.510327in}}{\pgfqpoint{2.827829in}{3.513239in}}%
\pgfpathcurveto{\pgfqpoint{2.830741in}{3.516151in}}{\pgfqpoint{2.832377in}{3.520101in}}{\pgfqpoint{2.832377in}{3.524219in}}%
\pgfpathcurveto{\pgfqpoint{2.832377in}{3.528338in}}{\pgfqpoint{2.830741in}{3.532288in}}{\pgfqpoint{2.827829in}{3.535200in}}%
\pgfpathcurveto{\pgfqpoint{2.824917in}{3.538111in}}{\pgfqpoint{2.820967in}{3.539748in}}{\pgfqpoint{2.816849in}{3.539748in}}%
\pgfpathcurveto{\pgfqpoint{2.812731in}{3.539748in}}{\pgfqpoint{2.808781in}{3.538111in}}{\pgfqpoint{2.805869in}{3.535200in}}%
\pgfpathcurveto{\pgfqpoint{2.802957in}{3.532288in}}{\pgfqpoint{2.801321in}{3.528338in}}{\pgfqpoint{2.801321in}{3.524219in}}%
\pgfpathcurveto{\pgfqpoint{2.801321in}{3.520101in}}{\pgfqpoint{2.802957in}{3.516151in}}{\pgfqpoint{2.805869in}{3.513239in}}%
\pgfpathcurveto{\pgfqpoint{2.808781in}{3.510327in}}{\pgfqpoint{2.812731in}{3.508691in}}{\pgfqpoint{2.816849in}{3.508691in}}%
\pgfpathclose%
\pgfusepath{fill}%
\end{pgfscope}%
\begin{pgfscope}%
\pgfpathrectangle{\pgfqpoint{0.887500in}{0.275000in}}{\pgfqpoint{4.225000in}{4.225000in}}%
\pgfusepath{clip}%
\pgfsetbuttcap%
\pgfsetroundjoin%
\definecolor{currentfill}{rgb}{0.000000,0.000000,0.000000}%
\pgfsetfillcolor{currentfill}%
\pgfsetfillopacity{0.800000}%
\pgfsetlinewidth{0.000000pt}%
\definecolor{currentstroke}{rgb}{0.000000,0.000000,0.000000}%
\pgfsetstrokecolor{currentstroke}%
\pgfsetstrokeopacity{0.800000}%
\pgfsetdash{}{0pt}%
\pgfpathmoveto{\pgfqpoint{2.488382in}{3.149900in}}%
\pgfpathcurveto{\pgfqpoint{2.492500in}{3.149900in}}{\pgfqpoint{2.496451in}{3.151536in}}{\pgfqpoint{2.499362in}{3.154448in}}%
\pgfpathcurveto{\pgfqpoint{2.502274in}{3.157360in}}{\pgfqpoint{2.503911in}{3.161310in}}{\pgfqpoint{2.503911in}{3.165429in}}%
\pgfpathcurveto{\pgfqpoint{2.503911in}{3.169547in}}{\pgfqpoint{2.502274in}{3.173497in}}{\pgfqpoint{2.499362in}{3.176409in}}%
\pgfpathcurveto{\pgfqpoint{2.496451in}{3.179321in}}{\pgfqpoint{2.492500in}{3.180957in}}{\pgfqpoint{2.488382in}{3.180957in}}%
\pgfpathcurveto{\pgfqpoint{2.484264in}{3.180957in}}{\pgfqpoint{2.480314in}{3.179321in}}{\pgfqpoint{2.477402in}{3.176409in}}%
\pgfpathcurveto{\pgfqpoint{2.474490in}{3.173497in}}{\pgfqpoint{2.472854in}{3.169547in}}{\pgfqpoint{2.472854in}{3.165429in}}%
\pgfpathcurveto{\pgfqpoint{2.472854in}{3.161310in}}{\pgfqpoint{2.474490in}{3.157360in}}{\pgfqpoint{2.477402in}{3.154448in}}%
\pgfpathcurveto{\pgfqpoint{2.480314in}{3.151536in}}{\pgfqpoint{2.484264in}{3.149900in}}{\pgfqpoint{2.488382in}{3.149900in}}%
\pgfpathclose%
\pgfusepath{fill}%
\end{pgfscope}%
\begin{pgfscope}%
\pgfpathrectangle{\pgfqpoint{0.887500in}{0.275000in}}{\pgfqpoint{4.225000in}{4.225000in}}%
\pgfusepath{clip}%
\pgfsetbuttcap%
\pgfsetroundjoin%
\definecolor{currentfill}{rgb}{0.000000,0.000000,0.000000}%
\pgfsetfillcolor{currentfill}%
\pgfsetfillopacity{0.800000}%
\pgfsetlinewidth{0.000000pt}%
\definecolor{currentstroke}{rgb}{0.000000,0.000000,0.000000}%
\pgfsetstrokecolor{currentstroke}%
\pgfsetstrokeopacity{0.800000}%
\pgfsetdash{}{0pt}%
\pgfpathmoveto{\pgfqpoint{2.206024in}{3.128829in}}%
\pgfpathcurveto{\pgfqpoint{2.210142in}{3.128829in}}{\pgfqpoint{2.214092in}{3.130465in}}{\pgfqpoint{2.217004in}{3.133377in}}%
\pgfpathcurveto{\pgfqpoint{2.219916in}{3.136289in}}{\pgfqpoint{2.221552in}{3.140239in}}{\pgfqpoint{2.221552in}{3.144358in}}%
\pgfpathcurveto{\pgfqpoint{2.221552in}{3.148476in}}{\pgfqpoint{2.219916in}{3.152426in}}{\pgfqpoint{2.217004in}{3.155338in}}%
\pgfpathcurveto{\pgfqpoint{2.214092in}{3.158250in}}{\pgfqpoint{2.210142in}{3.159886in}}{\pgfqpoint{2.206024in}{3.159886in}}%
\pgfpathcurveto{\pgfqpoint{2.201906in}{3.159886in}}{\pgfqpoint{2.197956in}{3.158250in}}{\pgfqpoint{2.195044in}{3.155338in}}%
\pgfpathcurveto{\pgfqpoint{2.192132in}{3.152426in}}{\pgfqpoint{2.190496in}{3.148476in}}{\pgfqpoint{2.190496in}{3.144358in}}%
\pgfpathcurveto{\pgfqpoint{2.190496in}{3.140239in}}{\pgfqpoint{2.192132in}{3.136289in}}{\pgfqpoint{2.195044in}{3.133377in}}%
\pgfpathcurveto{\pgfqpoint{2.197956in}{3.130465in}}{\pgfqpoint{2.201906in}{3.128829in}}{\pgfqpoint{2.206024in}{3.128829in}}%
\pgfpathclose%
\pgfusepath{fill}%
\end{pgfscope}%
\begin{pgfscope}%
\pgfpathrectangle{\pgfqpoint{0.887500in}{0.275000in}}{\pgfqpoint{4.225000in}{4.225000in}}%
\pgfusepath{clip}%
\pgfsetbuttcap%
\pgfsetroundjoin%
\definecolor{currentfill}{rgb}{0.000000,0.000000,0.000000}%
\pgfsetfillcolor{currentfill}%
\pgfsetfillopacity{0.800000}%
\pgfsetlinewidth{0.000000pt}%
\definecolor{currentstroke}{rgb}{0.000000,0.000000,0.000000}%
\pgfsetstrokecolor{currentstroke}%
\pgfsetstrokeopacity{0.800000}%
\pgfsetdash{}{0pt}%
\pgfpathmoveto{\pgfqpoint{3.490840in}{3.769085in}}%
\pgfpathcurveto{\pgfqpoint{3.494958in}{3.769085in}}{\pgfqpoint{3.498908in}{3.770721in}}{\pgfqpoint{3.501820in}{3.773633in}}%
\pgfpathcurveto{\pgfqpoint{3.504732in}{3.776545in}}{\pgfqpoint{3.506368in}{3.780495in}}{\pgfqpoint{3.506368in}{3.784613in}}%
\pgfpathcurveto{\pgfqpoint{3.506368in}{3.788731in}}{\pgfqpoint{3.504732in}{3.792681in}}{\pgfqpoint{3.501820in}{3.795593in}}%
\pgfpathcurveto{\pgfqpoint{3.498908in}{3.798505in}}{\pgfqpoint{3.494958in}{3.800141in}}{\pgfqpoint{3.490840in}{3.800141in}}%
\pgfpathcurveto{\pgfqpoint{3.486722in}{3.800141in}}{\pgfqpoint{3.482772in}{3.798505in}}{\pgfqpoint{3.479860in}{3.795593in}}%
\pgfpathcurveto{\pgfqpoint{3.476948in}{3.792681in}}{\pgfqpoint{3.475312in}{3.788731in}}{\pgfqpoint{3.475312in}{3.784613in}}%
\pgfpathcurveto{\pgfqpoint{3.475312in}{3.780495in}}{\pgfqpoint{3.476948in}{3.776545in}}{\pgfqpoint{3.479860in}{3.773633in}}%
\pgfpathcurveto{\pgfqpoint{3.482772in}{3.770721in}}{\pgfqpoint{3.486722in}{3.769085in}}{\pgfqpoint{3.490840in}{3.769085in}}%
\pgfpathclose%
\pgfusepath{fill}%
\end{pgfscope}%
\begin{pgfscope}%
\pgfpathrectangle{\pgfqpoint{0.887500in}{0.275000in}}{\pgfqpoint{4.225000in}{4.225000in}}%
\pgfusepath{clip}%
\pgfsetbuttcap%
\pgfsetroundjoin%
\definecolor{currentfill}{rgb}{0.000000,0.000000,0.000000}%
\pgfsetfillcolor{currentfill}%
\pgfsetfillopacity{0.800000}%
\pgfsetlinewidth{0.000000pt}%
\definecolor{currentstroke}{rgb}{0.000000,0.000000,0.000000}%
\pgfsetstrokecolor{currentstroke}%
\pgfsetstrokeopacity{0.800000}%
\pgfsetdash{}{0pt}%
\pgfpathmoveto{\pgfqpoint{3.098863in}{3.608734in}}%
\pgfpathcurveto{\pgfqpoint{3.102981in}{3.608734in}}{\pgfqpoint{3.106931in}{3.610370in}}{\pgfqpoint{3.109843in}{3.613282in}}%
\pgfpathcurveto{\pgfqpoint{3.112755in}{3.616194in}}{\pgfqpoint{3.114391in}{3.620144in}}{\pgfqpoint{3.114391in}{3.624262in}}%
\pgfpathcurveto{\pgfqpoint{3.114391in}{3.628380in}}{\pgfqpoint{3.112755in}{3.632330in}}{\pgfqpoint{3.109843in}{3.635242in}}%
\pgfpathcurveto{\pgfqpoint{3.106931in}{3.638154in}}{\pgfqpoint{3.102981in}{3.639790in}}{\pgfqpoint{3.098863in}{3.639790in}}%
\pgfpathcurveto{\pgfqpoint{3.094745in}{3.639790in}}{\pgfqpoint{3.090795in}{3.638154in}}{\pgfqpoint{3.087883in}{3.635242in}}%
\pgfpathcurveto{\pgfqpoint{3.084971in}{3.632330in}}{\pgfqpoint{3.083335in}{3.628380in}}{\pgfqpoint{3.083335in}{3.624262in}}%
\pgfpathcurveto{\pgfqpoint{3.083335in}{3.620144in}}{\pgfqpoint{3.084971in}{3.616194in}}{\pgfqpoint{3.087883in}{3.613282in}}%
\pgfpathcurveto{\pgfqpoint{3.090795in}{3.610370in}}{\pgfqpoint{3.094745in}{3.608734in}}{\pgfqpoint{3.098863in}{3.608734in}}%
\pgfpathclose%
\pgfusepath{fill}%
\end{pgfscope}%
\begin{pgfscope}%
\pgfpathrectangle{\pgfqpoint{0.887500in}{0.275000in}}{\pgfqpoint{4.225000in}{4.225000in}}%
\pgfusepath{clip}%
\pgfsetbuttcap%
\pgfsetroundjoin%
\definecolor{currentfill}{rgb}{0.000000,0.000000,0.000000}%
\pgfsetfillcolor{currentfill}%
\pgfsetfillopacity{0.800000}%
\pgfsetlinewidth{0.000000pt}%
\definecolor{currentstroke}{rgb}{0.000000,0.000000,0.000000}%
\pgfsetstrokecolor{currentstroke}%
\pgfsetstrokeopacity{0.800000}%
\pgfsetdash{}{0pt}%
\pgfpathmoveto{\pgfqpoint{3.773467in}{3.815682in}}%
\pgfpathcurveto{\pgfqpoint{3.777585in}{3.815682in}}{\pgfqpoint{3.781535in}{3.817318in}}{\pgfqpoint{3.784447in}{3.820230in}}%
\pgfpathcurveto{\pgfqpoint{3.787359in}{3.823142in}}{\pgfqpoint{3.788995in}{3.827092in}}{\pgfqpoint{3.788995in}{3.831210in}}%
\pgfpathcurveto{\pgfqpoint{3.788995in}{3.835329in}}{\pgfqpoint{3.787359in}{3.839279in}}{\pgfqpoint{3.784447in}{3.842191in}}%
\pgfpathcurveto{\pgfqpoint{3.781535in}{3.845102in}}{\pgfqpoint{3.777585in}{3.846739in}}{\pgfqpoint{3.773467in}{3.846739in}}%
\pgfpathcurveto{\pgfqpoint{3.769349in}{3.846739in}}{\pgfqpoint{3.765399in}{3.845102in}}{\pgfqpoint{3.762487in}{3.842191in}}%
\pgfpathcurveto{\pgfqpoint{3.759575in}{3.839279in}}{\pgfqpoint{3.757939in}{3.835329in}}{\pgfqpoint{3.757939in}{3.831210in}}%
\pgfpathcurveto{\pgfqpoint{3.757939in}{3.827092in}}{\pgfqpoint{3.759575in}{3.823142in}}{\pgfqpoint{3.762487in}{3.820230in}}%
\pgfpathcurveto{\pgfqpoint{3.765399in}{3.817318in}}{\pgfqpoint{3.769349in}{3.815682in}}{\pgfqpoint{3.773467in}{3.815682in}}%
\pgfpathclose%
\pgfusepath{fill}%
\end{pgfscope}%
\begin{pgfscope}%
\pgfpathrectangle{\pgfqpoint{0.887500in}{0.275000in}}{\pgfqpoint{4.225000in}{4.225000in}}%
\pgfusepath{clip}%
\pgfsetbuttcap%
\pgfsetroundjoin%
\definecolor{currentfill}{rgb}{0.000000,0.000000,0.000000}%
\pgfsetfillcolor{currentfill}%
\pgfsetfillopacity{0.800000}%
\pgfsetlinewidth{0.000000pt}%
\definecolor{currentstroke}{rgb}{0.000000,0.000000,0.000000}%
\pgfsetstrokecolor{currentstroke}%
\pgfsetstrokeopacity{0.800000}%
\pgfsetdash{}{0pt}%
\pgfpathmoveto{\pgfqpoint{2.094919in}{3.070142in}}%
\pgfpathcurveto{\pgfqpoint{2.099038in}{3.070142in}}{\pgfqpoint{2.102988in}{3.071778in}}{\pgfqpoint{2.105900in}{3.074690in}}%
\pgfpathcurveto{\pgfqpoint{2.108812in}{3.077602in}}{\pgfqpoint{2.110448in}{3.081552in}}{\pgfqpoint{2.110448in}{3.085670in}}%
\pgfpathcurveto{\pgfqpoint{2.110448in}{3.089788in}}{\pgfqpoint{2.108812in}{3.093738in}}{\pgfqpoint{2.105900in}{3.096650in}}%
\pgfpathcurveto{\pgfqpoint{2.102988in}{3.099562in}}{\pgfqpoint{2.099038in}{3.101198in}}{\pgfqpoint{2.094919in}{3.101198in}}%
\pgfpathcurveto{\pgfqpoint{2.090801in}{3.101198in}}{\pgfqpoint{2.086851in}{3.099562in}}{\pgfqpoint{2.083939in}{3.096650in}}%
\pgfpathcurveto{\pgfqpoint{2.081027in}{3.093738in}}{\pgfqpoint{2.079391in}{3.089788in}}{\pgfqpoint{2.079391in}{3.085670in}}%
\pgfpathcurveto{\pgfqpoint{2.079391in}{3.081552in}}{\pgfqpoint{2.081027in}{3.077602in}}{\pgfqpoint{2.083939in}{3.074690in}}%
\pgfpathcurveto{\pgfqpoint{2.086851in}{3.071778in}}{\pgfqpoint{2.090801in}{3.070142in}}{\pgfqpoint{2.094919in}{3.070142in}}%
\pgfpathclose%
\pgfusepath{fill}%
\end{pgfscope}%
\begin{pgfscope}%
\pgfpathrectangle{\pgfqpoint{0.887500in}{0.275000in}}{\pgfqpoint{4.225000in}{4.225000in}}%
\pgfusepath{clip}%
\pgfsetbuttcap%
\pgfsetroundjoin%
\definecolor{currentfill}{rgb}{0.000000,0.000000,0.000000}%
\pgfsetfillcolor{currentfill}%
\pgfsetfillopacity{0.800000}%
\pgfsetlinewidth{0.000000pt}%
\definecolor{currentstroke}{rgb}{0.000000,0.000000,0.000000}%
\pgfsetstrokecolor{currentstroke}%
\pgfsetstrokeopacity{0.800000}%
\pgfsetdash{}{0pt}%
\pgfpathmoveto{\pgfqpoint{2.377329in}{3.118517in}}%
\pgfpathcurveto{\pgfqpoint{2.381447in}{3.118517in}}{\pgfqpoint{2.385397in}{3.120153in}}{\pgfqpoint{2.388309in}{3.123065in}}%
\pgfpathcurveto{\pgfqpoint{2.391221in}{3.125977in}}{\pgfqpoint{2.392857in}{3.129927in}}{\pgfqpoint{2.392857in}{3.134045in}}%
\pgfpathcurveto{\pgfqpoint{2.392857in}{3.138163in}}{\pgfqpoint{2.391221in}{3.142113in}}{\pgfqpoint{2.388309in}{3.145025in}}%
\pgfpathcurveto{\pgfqpoint{2.385397in}{3.147937in}}{\pgfqpoint{2.381447in}{3.149573in}}{\pgfqpoint{2.377329in}{3.149573in}}%
\pgfpathcurveto{\pgfqpoint{2.373211in}{3.149573in}}{\pgfqpoint{2.369261in}{3.147937in}}{\pgfqpoint{2.366349in}{3.145025in}}%
\pgfpathcurveto{\pgfqpoint{2.363437in}{3.142113in}}{\pgfqpoint{2.361801in}{3.138163in}}{\pgfqpoint{2.361801in}{3.134045in}}%
\pgfpathcurveto{\pgfqpoint{2.361801in}{3.129927in}}{\pgfqpoint{2.363437in}{3.125977in}}{\pgfqpoint{2.366349in}{3.123065in}}%
\pgfpathcurveto{\pgfqpoint{2.369261in}{3.120153in}}{\pgfqpoint{2.373211in}{3.118517in}}{\pgfqpoint{2.377329in}{3.118517in}}%
\pgfpathclose%
\pgfusepath{fill}%
\end{pgfscope}%
\begin{pgfscope}%
\pgfpathrectangle{\pgfqpoint{0.887500in}{0.275000in}}{\pgfqpoint{4.225000in}{4.225000in}}%
\pgfusepath{clip}%
\pgfsetbuttcap%
\pgfsetroundjoin%
\definecolor{currentfill}{rgb}{0.000000,0.000000,0.000000}%
\pgfsetfillcolor{currentfill}%
\pgfsetfillopacity{0.800000}%
\pgfsetlinewidth{0.000000pt}%
\definecolor{currentstroke}{rgb}{0.000000,0.000000,0.000000}%
\pgfsetstrokecolor{currentstroke}%
\pgfsetstrokeopacity{0.800000}%
\pgfsetdash{}{0pt}%
\pgfpathmoveto{\pgfqpoint{3.381811in}{3.710828in}}%
\pgfpathcurveto{\pgfqpoint{3.385929in}{3.710828in}}{\pgfqpoint{3.389879in}{3.712464in}}{\pgfqpoint{3.392791in}{3.715376in}}%
\pgfpathcurveto{\pgfqpoint{3.395703in}{3.718288in}}{\pgfqpoint{3.397339in}{3.722238in}}{\pgfqpoint{3.397339in}{3.726356in}}%
\pgfpathcurveto{\pgfqpoint{3.397339in}{3.730474in}}{\pgfqpoint{3.395703in}{3.734424in}}{\pgfqpoint{3.392791in}{3.737336in}}%
\pgfpathcurveto{\pgfqpoint{3.389879in}{3.740248in}}{\pgfqpoint{3.385929in}{3.741884in}}{\pgfqpoint{3.381811in}{3.741884in}}%
\pgfpathcurveto{\pgfqpoint{3.377693in}{3.741884in}}{\pgfqpoint{3.373743in}{3.740248in}}{\pgfqpoint{3.370831in}{3.737336in}}%
\pgfpathcurveto{\pgfqpoint{3.367919in}{3.734424in}}{\pgfqpoint{3.366283in}{3.730474in}}{\pgfqpoint{3.366283in}{3.726356in}}%
\pgfpathcurveto{\pgfqpoint{3.366283in}{3.722238in}}{\pgfqpoint{3.367919in}{3.718288in}}{\pgfqpoint{3.370831in}{3.715376in}}%
\pgfpathcurveto{\pgfqpoint{3.373743in}{3.712464in}}{\pgfqpoint{3.377693in}{3.710828in}}{\pgfqpoint{3.381811in}{3.710828in}}%
\pgfpathclose%
\pgfusepath{fill}%
\end{pgfscope}%
\begin{pgfscope}%
\pgfpathrectangle{\pgfqpoint{0.887500in}{0.275000in}}{\pgfqpoint{4.225000in}{4.225000in}}%
\pgfusepath{clip}%
\pgfsetbuttcap%
\pgfsetroundjoin%
\definecolor{currentfill}{rgb}{0.000000,0.000000,0.000000}%
\pgfsetfillcolor{currentfill}%
\pgfsetfillopacity{0.800000}%
\pgfsetlinewidth{0.000000pt}%
\definecolor{currentstroke}{rgb}{0.000000,0.000000,0.000000}%
\pgfsetstrokecolor{currentstroke}%
\pgfsetstrokeopacity{0.800000}%
\pgfsetdash{}{0pt}%
\pgfpathmoveto{\pgfqpoint{2.989170in}{3.554589in}}%
\pgfpathcurveto{\pgfqpoint{2.993289in}{3.554589in}}{\pgfqpoint{2.997239in}{3.556225in}}{\pgfqpoint{3.000150in}{3.559137in}}%
\pgfpathcurveto{\pgfqpoint{3.003062in}{3.562049in}}{\pgfqpoint{3.004699in}{3.565999in}}{\pgfqpoint{3.004699in}{3.570118in}}%
\pgfpathcurveto{\pgfqpoint{3.004699in}{3.574236in}}{\pgfqpoint{3.003062in}{3.578186in}}{\pgfqpoint{3.000150in}{3.581098in}}%
\pgfpathcurveto{\pgfqpoint{2.997239in}{3.584010in}}{\pgfqpoint{2.993289in}{3.585646in}}{\pgfqpoint{2.989170in}{3.585646in}}%
\pgfpathcurveto{\pgfqpoint{2.985052in}{3.585646in}}{\pgfqpoint{2.981102in}{3.584010in}}{\pgfqpoint{2.978190in}{3.581098in}}%
\pgfpathcurveto{\pgfqpoint{2.975278in}{3.578186in}}{\pgfqpoint{2.973642in}{3.574236in}}{\pgfqpoint{2.973642in}{3.570118in}}%
\pgfpathcurveto{\pgfqpoint{2.973642in}{3.565999in}}{\pgfqpoint{2.975278in}{3.562049in}}{\pgfqpoint{2.978190in}{3.559137in}}%
\pgfpathcurveto{\pgfqpoint{2.981102in}{3.556225in}}{\pgfqpoint{2.985052in}{3.554589in}}{\pgfqpoint{2.989170in}{3.554589in}}%
\pgfpathclose%
\pgfusepath{fill}%
\end{pgfscope}%
\begin{pgfscope}%
\pgfpathrectangle{\pgfqpoint{0.887500in}{0.275000in}}{\pgfqpoint{4.225000in}{4.225000in}}%
\pgfusepath{clip}%
\pgfsetbuttcap%
\pgfsetroundjoin%
\definecolor{currentfill}{rgb}{0.000000,0.000000,0.000000}%
\pgfsetfillcolor{currentfill}%
\pgfsetfillopacity{0.800000}%
\pgfsetlinewidth{0.000000pt}%
\definecolor{currentstroke}{rgb}{0.000000,0.000000,0.000000}%
\pgfsetstrokecolor{currentstroke}%
\pgfsetstrokeopacity{0.800000}%
\pgfsetdash{}{0pt}%
\pgfpathmoveto{\pgfqpoint{1.984253in}{2.980823in}}%
\pgfpathcurveto{\pgfqpoint{1.988371in}{2.980823in}}{\pgfqpoint{1.992321in}{2.982459in}}{\pgfqpoint{1.995233in}{2.985371in}}%
\pgfpathcurveto{\pgfqpoint{1.998145in}{2.988283in}}{\pgfqpoint{1.999781in}{2.992233in}}{\pgfqpoint{1.999781in}{2.996351in}}%
\pgfpathcurveto{\pgfqpoint{1.999781in}{3.000469in}}{\pgfqpoint{1.998145in}{3.004419in}}{\pgfqpoint{1.995233in}{3.007331in}}%
\pgfpathcurveto{\pgfqpoint{1.992321in}{3.010243in}}{\pgfqpoint{1.988371in}{3.011880in}}{\pgfqpoint{1.984253in}{3.011880in}}%
\pgfpathcurveto{\pgfqpoint{1.980135in}{3.011880in}}{\pgfqpoint{1.976185in}{3.010243in}}{\pgfqpoint{1.973273in}{3.007331in}}%
\pgfpathcurveto{\pgfqpoint{1.970361in}{3.004419in}}{\pgfqpoint{1.968725in}{3.000469in}}{\pgfqpoint{1.968725in}{2.996351in}}%
\pgfpathcurveto{\pgfqpoint{1.968725in}{2.992233in}}{\pgfqpoint{1.970361in}{2.988283in}}{\pgfqpoint{1.973273in}{2.985371in}}%
\pgfpathcurveto{\pgfqpoint{1.976185in}{2.982459in}}{\pgfqpoint{1.980135in}{2.980823in}}{\pgfqpoint{1.984253in}{2.980823in}}%
\pgfpathclose%
\pgfusepath{fill}%
\end{pgfscope}%
\begin{pgfscope}%
\pgfpathrectangle{\pgfqpoint{0.887500in}{0.275000in}}{\pgfqpoint{4.225000in}{4.225000in}}%
\pgfusepath{clip}%
\pgfsetbuttcap%
\pgfsetroundjoin%
\definecolor{currentfill}{rgb}{0.000000,0.000000,0.000000}%
\pgfsetfillcolor{currentfill}%
\pgfsetfillopacity{0.800000}%
\pgfsetlinewidth{0.000000pt}%
\definecolor{currentstroke}{rgb}{0.000000,0.000000,0.000000}%
\pgfsetstrokecolor{currentstroke}%
\pgfsetstrokeopacity{0.800000}%
\pgfsetdash{}{0pt}%
\pgfpathmoveto{\pgfqpoint{3.665065in}{3.770097in}}%
\pgfpathcurveto{\pgfqpoint{3.669183in}{3.770097in}}{\pgfqpoint{3.673133in}{3.771734in}}{\pgfqpoint{3.676045in}{3.774645in}}%
\pgfpathcurveto{\pgfqpoint{3.678957in}{3.777557in}}{\pgfqpoint{3.680593in}{3.781507in}}{\pgfqpoint{3.680593in}{3.785626in}}%
\pgfpathcurveto{\pgfqpoint{3.680593in}{3.789744in}}{\pgfqpoint{3.678957in}{3.793694in}}{\pgfqpoint{3.676045in}{3.796606in}}%
\pgfpathcurveto{\pgfqpoint{3.673133in}{3.799518in}}{\pgfqpoint{3.669183in}{3.801154in}}{\pgfqpoint{3.665065in}{3.801154in}}%
\pgfpathcurveto{\pgfqpoint{3.660947in}{3.801154in}}{\pgfqpoint{3.656997in}{3.799518in}}{\pgfqpoint{3.654085in}{3.796606in}}%
\pgfpathcurveto{\pgfqpoint{3.651173in}{3.793694in}}{\pgfqpoint{3.649537in}{3.789744in}}{\pgfqpoint{3.649537in}{3.785626in}}%
\pgfpathcurveto{\pgfqpoint{3.649537in}{3.781507in}}{\pgfqpoint{3.651173in}{3.777557in}}{\pgfqpoint{3.654085in}{3.774645in}}%
\pgfpathcurveto{\pgfqpoint{3.656997in}{3.771734in}}{\pgfqpoint{3.660947in}{3.770097in}}{\pgfqpoint{3.665065in}{3.770097in}}%
\pgfpathclose%
\pgfusepath{fill}%
\end{pgfscope}%
\begin{pgfscope}%
\pgfpathrectangle{\pgfqpoint{0.887500in}{0.275000in}}{\pgfqpoint{4.225000in}{4.225000in}}%
\pgfusepath{clip}%
\pgfsetbuttcap%
\pgfsetroundjoin%
\definecolor{currentfill}{rgb}{0.000000,0.000000,0.000000}%
\pgfsetfillcolor{currentfill}%
\pgfsetfillopacity{0.800000}%
\pgfsetlinewidth{0.000000pt}%
\definecolor{currentstroke}{rgb}{0.000000,0.000000,0.000000}%
\pgfsetstrokecolor{currentstroke}%
\pgfsetstrokeopacity{0.800000}%
\pgfsetdash{}{0pt}%
\pgfpathmoveto{\pgfqpoint{2.549739in}{3.076274in}}%
\pgfpathcurveto{\pgfqpoint{2.553857in}{3.076274in}}{\pgfqpoint{2.557807in}{3.077910in}}{\pgfqpoint{2.560719in}{3.080822in}}%
\pgfpathcurveto{\pgfqpoint{2.563631in}{3.083734in}}{\pgfqpoint{2.565268in}{3.087684in}}{\pgfqpoint{2.565268in}{3.091802in}}%
\pgfpathcurveto{\pgfqpoint{2.565268in}{3.095921in}}{\pgfqpoint{2.563631in}{3.099871in}}{\pgfqpoint{2.560719in}{3.102783in}}%
\pgfpathcurveto{\pgfqpoint{2.557807in}{3.105695in}}{\pgfqpoint{2.553857in}{3.107331in}}{\pgfqpoint{2.549739in}{3.107331in}}%
\pgfpathcurveto{\pgfqpoint{2.545621in}{3.107331in}}{\pgfqpoint{2.541671in}{3.105695in}}{\pgfqpoint{2.538759in}{3.102783in}}%
\pgfpathcurveto{\pgfqpoint{2.535847in}{3.099871in}}{\pgfqpoint{2.534211in}{3.095921in}}{\pgfqpoint{2.534211in}{3.091802in}}%
\pgfpathcurveto{\pgfqpoint{2.534211in}{3.087684in}}{\pgfqpoint{2.535847in}{3.083734in}}{\pgfqpoint{2.538759in}{3.080822in}}%
\pgfpathcurveto{\pgfqpoint{2.541671in}{3.077910in}}{\pgfqpoint{2.545621in}{3.076274in}}{\pgfqpoint{2.549739in}{3.076274in}}%
\pgfpathclose%
\pgfusepath{fill}%
\end{pgfscope}%
\begin{pgfscope}%
\pgfpathrectangle{\pgfqpoint{0.887500in}{0.275000in}}{\pgfqpoint{4.225000in}{4.225000in}}%
\pgfusepath{clip}%
\pgfsetbuttcap%
\pgfsetroundjoin%
\definecolor{currentfill}{rgb}{0.000000,0.000000,0.000000}%
\pgfsetfillcolor{currentfill}%
\pgfsetfillopacity{0.800000}%
\pgfsetlinewidth{0.000000pt}%
\definecolor{currentstroke}{rgb}{0.000000,0.000000,0.000000}%
\pgfsetstrokecolor{currentstroke}%
\pgfsetstrokeopacity{0.800000}%
\pgfsetdash{}{0pt}%
\pgfpathmoveto{\pgfqpoint{2.266565in}{3.045507in}}%
\pgfpathcurveto{\pgfqpoint{2.270683in}{3.045507in}}{\pgfqpoint{2.274633in}{3.047143in}}{\pgfqpoint{2.277545in}{3.050055in}}%
\pgfpathcurveto{\pgfqpoint{2.280457in}{3.052967in}}{\pgfqpoint{2.282094in}{3.056917in}}{\pgfqpoint{2.282094in}{3.061035in}}%
\pgfpathcurveto{\pgfqpoint{2.282094in}{3.065153in}}{\pgfqpoint{2.280457in}{3.069103in}}{\pgfqpoint{2.277545in}{3.072015in}}%
\pgfpathcurveto{\pgfqpoint{2.274633in}{3.074927in}}{\pgfqpoint{2.270683in}{3.076563in}}{\pgfqpoint{2.266565in}{3.076563in}}%
\pgfpathcurveto{\pgfqpoint{2.262447in}{3.076563in}}{\pgfqpoint{2.258497in}{3.074927in}}{\pgfqpoint{2.255585in}{3.072015in}}%
\pgfpathcurveto{\pgfqpoint{2.252673in}{3.069103in}}{\pgfqpoint{2.251037in}{3.065153in}}{\pgfqpoint{2.251037in}{3.061035in}}%
\pgfpathcurveto{\pgfqpoint{2.251037in}{3.056917in}}{\pgfqpoint{2.252673in}{3.052967in}}{\pgfqpoint{2.255585in}{3.050055in}}%
\pgfpathcurveto{\pgfqpoint{2.258497in}{3.047143in}}{\pgfqpoint{2.262447in}{3.045507in}}{\pgfqpoint{2.266565in}{3.045507in}}%
\pgfpathclose%
\pgfusepath{fill}%
\end{pgfscope}%
\begin{pgfscope}%
\pgfpathrectangle{\pgfqpoint{0.887500in}{0.275000in}}{\pgfqpoint{4.225000in}{4.225000in}}%
\pgfusepath{clip}%
\pgfsetbuttcap%
\pgfsetroundjoin%
\definecolor{currentfill}{rgb}{0.000000,0.000000,0.000000}%
\pgfsetfillcolor{currentfill}%
\pgfsetfillopacity{0.800000}%
\pgfsetlinewidth{0.000000pt}%
\definecolor{currentstroke}{rgb}{0.000000,0.000000,0.000000}%
\pgfsetstrokecolor{currentstroke}%
\pgfsetstrokeopacity{0.800000}%
\pgfsetdash{}{0pt}%
\pgfpathmoveto{\pgfqpoint{3.272399in}{3.649097in}}%
\pgfpathcurveto{\pgfqpoint{3.276517in}{3.649097in}}{\pgfqpoint{3.280467in}{3.650734in}}{\pgfqpoint{3.283379in}{3.653646in}}%
\pgfpathcurveto{\pgfqpoint{3.286291in}{3.656558in}}{\pgfqpoint{3.287927in}{3.660508in}}{\pgfqpoint{3.287927in}{3.664626in}}%
\pgfpathcurveto{\pgfqpoint{3.287927in}{3.668744in}}{\pgfqpoint{3.286291in}{3.672694in}}{\pgfqpoint{3.283379in}{3.675606in}}%
\pgfpathcurveto{\pgfqpoint{3.280467in}{3.678518in}}{\pgfqpoint{3.276517in}{3.680154in}}{\pgfqpoint{3.272399in}{3.680154in}}%
\pgfpathcurveto{\pgfqpoint{3.268281in}{3.680154in}}{\pgfqpoint{3.264331in}{3.678518in}}{\pgfqpoint{3.261419in}{3.675606in}}%
\pgfpathcurveto{\pgfqpoint{3.258507in}{3.672694in}}{\pgfqpoint{3.256871in}{3.668744in}}{\pgfqpoint{3.256871in}{3.664626in}}%
\pgfpathcurveto{\pgfqpoint{3.256871in}{3.660508in}}{\pgfqpoint{3.258507in}{3.656558in}}{\pgfqpoint{3.261419in}{3.653646in}}%
\pgfpathcurveto{\pgfqpoint{3.264331in}{3.650734in}}{\pgfqpoint{3.268281in}{3.649097in}}{\pgfqpoint{3.272399in}{3.649097in}}%
\pgfpathclose%
\pgfusepath{fill}%
\end{pgfscope}%
\begin{pgfscope}%
\pgfpathrectangle{\pgfqpoint{0.887500in}{0.275000in}}{\pgfqpoint{4.225000in}{4.225000in}}%
\pgfusepath{clip}%
\pgfsetbuttcap%
\pgfsetroundjoin%
\definecolor{currentfill}{rgb}{0.000000,0.000000,0.000000}%
\pgfsetfillcolor{currentfill}%
\pgfsetfillopacity{0.800000}%
\pgfsetlinewidth{0.000000pt}%
\definecolor{currentstroke}{rgb}{0.000000,0.000000,0.000000}%
\pgfsetstrokecolor{currentstroke}%
\pgfsetstrokeopacity{0.800000}%
\pgfsetdash{}{0pt}%
\pgfpathmoveto{\pgfqpoint{2.879085in}{3.500423in}}%
\pgfpathcurveto{\pgfqpoint{2.883203in}{3.500423in}}{\pgfqpoint{2.887153in}{3.502059in}}{\pgfqpoint{2.890065in}{3.504971in}}%
\pgfpathcurveto{\pgfqpoint{2.892977in}{3.507883in}}{\pgfqpoint{2.894613in}{3.511833in}}{\pgfqpoint{2.894613in}{3.515951in}}%
\pgfpathcurveto{\pgfqpoint{2.894613in}{3.520070in}}{\pgfqpoint{2.892977in}{3.524020in}}{\pgfqpoint{2.890065in}{3.526932in}}%
\pgfpathcurveto{\pgfqpoint{2.887153in}{3.529844in}}{\pgfqpoint{2.883203in}{3.531480in}}{\pgfqpoint{2.879085in}{3.531480in}}%
\pgfpathcurveto{\pgfqpoint{2.874967in}{3.531480in}}{\pgfqpoint{2.871017in}{3.529844in}}{\pgfqpoint{2.868105in}{3.526932in}}%
\pgfpathcurveto{\pgfqpoint{2.865193in}{3.524020in}}{\pgfqpoint{2.863557in}{3.520070in}}{\pgfqpoint{2.863557in}{3.515951in}}%
\pgfpathcurveto{\pgfqpoint{2.863557in}{3.511833in}}{\pgfqpoint{2.865193in}{3.507883in}}{\pgfqpoint{2.868105in}{3.504971in}}%
\pgfpathcurveto{\pgfqpoint{2.871017in}{3.502059in}}{\pgfqpoint{2.874967in}{3.500423in}}{\pgfqpoint{2.879085in}{3.500423in}}%
\pgfpathclose%
\pgfusepath{fill}%
\end{pgfscope}%
\begin{pgfscope}%
\pgfpathrectangle{\pgfqpoint{0.887500in}{0.275000in}}{\pgfqpoint{4.225000in}{4.225000in}}%
\pgfusepath{clip}%
\pgfsetbuttcap%
\pgfsetroundjoin%
\definecolor{currentfill}{rgb}{0.000000,0.000000,0.000000}%
\pgfsetfillcolor{currentfill}%
\pgfsetfillopacity{0.800000}%
\pgfsetlinewidth{0.000000pt}%
\definecolor{currentstroke}{rgb}{0.000000,0.000000,0.000000}%
\pgfsetstrokecolor{currentstroke}%
\pgfsetstrokeopacity{0.800000}%
\pgfsetdash{}{0pt}%
\pgfpathmoveto{\pgfqpoint{2.155627in}{2.966870in}}%
\pgfpathcurveto{\pgfqpoint{2.159745in}{2.966870in}}{\pgfqpoint{2.163695in}{2.968506in}}{\pgfqpoint{2.166607in}{2.971418in}}%
\pgfpathcurveto{\pgfqpoint{2.169519in}{2.974330in}}{\pgfqpoint{2.171155in}{2.978280in}}{\pgfqpoint{2.171155in}{2.982398in}}%
\pgfpathcurveto{\pgfqpoint{2.171155in}{2.986517in}}{\pgfqpoint{2.169519in}{2.990467in}}{\pgfqpoint{2.166607in}{2.993379in}}%
\pgfpathcurveto{\pgfqpoint{2.163695in}{2.996291in}}{\pgfqpoint{2.159745in}{2.997927in}}{\pgfqpoint{2.155627in}{2.997927in}}%
\pgfpathcurveto{\pgfqpoint{2.151509in}{2.997927in}}{\pgfqpoint{2.147559in}{2.996291in}}{\pgfqpoint{2.144647in}{2.993379in}}%
\pgfpathcurveto{\pgfqpoint{2.141735in}{2.990467in}}{\pgfqpoint{2.140099in}{2.986517in}}{\pgfqpoint{2.140099in}{2.982398in}}%
\pgfpathcurveto{\pgfqpoint{2.140099in}{2.978280in}}{\pgfqpoint{2.141735in}{2.974330in}}{\pgfqpoint{2.144647in}{2.971418in}}%
\pgfpathcurveto{\pgfqpoint{2.147559in}{2.968506in}}{\pgfqpoint{2.151509in}{2.966870in}}{\pgfqpoint{2.155627in}{2.966870in}}%
\pgfpathclose%
\pgfusepath{fill}%
\end{pgfscope}%
\begin{pgfscope}%
\pgfpathrectangle{\pgfqpoint{0.887500in}{0.275000in}}{\pgfqpoint{4.225000in}{4.225000in}}%
\pgfusepath{clip}%
\pgfsetbuttcap%
\pgfsetroundjoin%
\definecolor{currentfill}{rgb}{0.000000,0.000000,0.000000}%
\pgfsetfillcolor{currentfill}%
\pgfsetfillopacity{0.800000}%
\pgfsetlinewidth{0.000000pt}%
\definecolor{currentstroke}{rgb}{0.000000,0.000000,0.000000}%
\pgfsetstrokecolor{currentstroke}%
\pgfsetstrokeopacity{0.800000}%
\pgfsetdash{}{0pt}%
\pgfpathmoveto{\pgfqpoint{3.839543in}{3.742465in}}%
\pgfpathcurveto{\pgfqpoint{3.843661in}{3.742465in}}{\pgfqpoint{3.847611in}{3.744101in}}{\pgfqpoint{3.850523in}{3.747013in}}%
\pgfpathcurveto{\pgfqpoint{3.853435in}{3.749925in}}{\pgfqpoint{3.855071in}{3.753875in}}{\pgfqpoint{3.855071in}{3.757993in}}%
\pgfpathcurveto{\pgfqpoint{3.855071in}{3.762111in}}{\pgfqpoint{3.853435in}{3.766061in}}{\pgfqpoint{3.850523in}{3.768973in}}%
\pgfpathcurveto{\pgfqpoint{3.847611in}{3.771885in}}{\pgfqpoint{3.843661in}{3.773522in}}{\pgfqpoint{3.839543in}{3.773522in}}%
\pgfpathcurveto{\pgfqpoint{3.835425in}{3.773522in}}{\pgfqpoint{3.831475in}{3.771885in}}{\pgfqpoint{3.828563in}{3.768973in}}%
\pgfpathcurveto{\pgfqpoint{3.825651in}{3.766061in}}{\pgfqpoint{3.824015in}{3.762111in}}{\pgfqpoint{3.824015in}{3.757993in}}%
\pgfpathcurveto{\pgfqpoint{3.824015in}{3.753875in}}{\pgfqpoint{3.825651in}{3.749925in}}{\pgfqpoint{3.828563in}{3.747013in}}%
\pgfpathcurveto{\pgfqpoint{3.831475in}{3.744101in}}{\pgfqpoint{3.835425in}{3.742465in}}{\pgfqpoint{3.839543in}{3.742465in}}%
\pgfpathclose%
\pgfusepath{fill}%
\end{pgfscope}%
\begin{pgfscope}%
\pgfpathrectangle{\pgfqpoint{0.887500in}{0.275000in}}{\pgfqpoint{4.225000in}{4.225000in}}%
\pgfusepath{clip}%
\pgfsetbuttcap%
\pgfsetroundjoin%
\definecolor{currentfill}{rgb}{0.000000,0.000000,0.000000}%
\pgfsetfillcolor{currentfill}%
\pgfsetfillopacity{0.800000}%
\pgfsetlinewidth{0.000000pt}%
\definecolor{currentstroke}{rgb}{0.000000,0.000000,0.000000}%
\pgfsetstrokecolor{currentstroke}%
\pgfsetstrokeopacity{0.800000}%
\pgfsetdash{}{0pt}%
\pgfpathmoveto{\pgfqpoint{1.871779in}{2.945467in}}%
\pgfpathcurveto{\pgfqpoint{1.875897in}{2.945467in}}{\pgfqpoint{1.879847in}{2.947104in}}{\pgfqpoint{1.882759in}{2.950016in}}%
\pgfpathcurveto{\pgfqpoint{1.885671in}{2.952928in}}{\pgfqpoint{1.887307in}{2.956878in}}{\pgfqpoint{1.887307in}{2.960996in}}%
\pgfpathcurveto{\pgfqpoint{1.887307in}{2.965114in}}{\pgfqpoint{1.885671in}{2.969064in}}{\pgfqpoint{1.882759in}{2.971976in}}%
\pgfpathcurveto{\pgfqpoint{1.879847in}{2.974888in}}{\pgfqpoint{1.875897in}{2.976524in}}{\pgfqpoint{1.871779in}{2.976524in}}%
\pgfpathcurveto{\pgfqpoint{1.867661in}{2.976524in}}{\pgfqpoint{1.863711in}{2.974888in}}{\pgfqpoint{1.860799in}{2.971976in}}%
\pgfpathcurveto{\pgfqpoint{1.857887in}{2.969064in}}{\pgfqpoint{1.856251in}{2.965114in}}{\pgfqpoint{1.856251in}{2.960996in}}%
\pgfpathcurveto{\pgfqpoint{1.856251in}{2.956878in}}{\pgfqpoint{1.857887in}{2.952928in}}{\pgfqpoint{1.860799in}{2.950016in}}%
\pgfpathcurveto{\pgfqpoint{1.863711in}{2.947104in}}{\pgfqpoint{1.867661in}{2.945467in}}{\pgfqpoint{1.871779in}{2.945467in}}%
\pgfpathclose%
\pgfusepath{fill}%
\end{pgfscope}%
\begin{pgfscope}%
\pgfpathrectangle{\pgfqpoint{0.887500in}{0.275000in}}{\pgfqpoint{4.225000in}{4.225000in}}%
\pgfusepath{clip}%
\pgfsetbuttcap%
\pgfsetroundjoin%
\definecolor{currentfill}{rgb}{0.000000,0.000000,0.000000}%
\pgfsetfillcolor{currentfill}%
\pgfsetfillopacity{0.800000}%
\pgfsetlinewidth{0.000000pt}%
\definecolor{currentstroke}{rgb}{0.000000,0.000000,0.000000}%
\pgfsetstrokecolor{currentstroke}%
\pgfsetstrokeopacity{0.800000}%
\pgfsetdash{}{0pt}%
\pgfpathmoveto{\pgfqpoint{3.556189in}{3.720786in}}%
\pgfpathcurveto{\pgfqpoint{3.560307in}{3.720786in}}{\pgfqpoint{3.564257in}{3.722422in}}{\pgfqpoint{3.567169in}{3.725334in}}%
\pgfpathcurveto{\pgfqpoint{3.570081in}{3.728246in}}{\pgfqpoint{3.571717in}{3.732196in}}{\pgfqpoint{3.571717in}{3.736314in}}%
\pgfpathcurveto{\pgfqpoint{3.571717in}{3.740432in}}{\pgfqpoint{3.570081in}{3.744382in}}{\pgfqpoint{3.567169in}{3.747294in}}%
\pgfpathcurveto{\pgfqpoint{3.564257in}{3.750206in}}{\pgfqpoint{3.560307in}{3.751842in}}{\pgfqpoint{3.556189in}{3.751842in}}%
\pgfpathcurveto{\pgfqpoint{3.552070in}{3.751842in}}{\pgfqpoint{3.548120in}{3.750206in}}{\pgfqpoint{3.545208in}{3.747294in}}%
\pgfpathcurveto{\pgfqpoint{3.542296in}{3.744382in}}{\pgfqpoint{3.540660in}{3.740432in}}{\pgfqpoint{3.540660in}{3.736314in}}%
\pgfpathcurveto{\pgfqpoint{3.540660in}{3.732196in}}{\pgfqpoint{3.542296in}{3.728246in}}{\pgfqpoint{3.545208in}{3.725334in}}%
\pgfpathcurveto{\pgfqpoint{3.548120in}{3.722422in}}{\pgfqpoint{3.552070in}{3.720786in}}{\pgfqpoint{3.556189in}{3.720786in}}%
\pgfpathclose%
\pgfusepath{fill}%
\end{pgfscope}%
\begin{pgfscope}%
\pgfpathrectangle{\pgfqpoint{0.887500in}{0.275000in}}{\pgfqpoint{4.225000in}{4.225000in}}%
\pgfusepath{clip}%
\pgfsetbuttcap%
\pgfsetroundjoin%
\definecolor{currentfill}{rgb}{0.000000,0.000000,0.000000}%
\pgfsetfillcolor{currentfill}%
\pgfsetfillopacity{0.800000}%
\pgfsetlinewidth{0.000000pt}%
\definecolor{currentstroke}{rgb}{0.000000,0.000000,0.000000}%
\pgfsetstrokecolor{currentstroke}%
\pgfsetstrokeopacity{0.800000}%
\pgfsetdash{}{0pt}%
\pgfpathmoveto{\pgfqpoint{2.438340in}{3.050304in}}%
\pgfpathcurveto{\pgfqpoint{2.442458in}{3.050304in}}{\pgfqpoint{2.446408in}{3.051940in}}{\pgfqpoint{2.449320in}{3.054852in}}%
\pgfpathcurveto{\pgfqpoint{2.452232in}{3.057764in}}{\pgfqpoint{2.453869in}{3.061714in}}{\pgfqpoint{2.453869in}{3.065832in}}%
\pgfpathcurveto{\pgfqpoint{2.453869in}{3.069950in}}{\pgfqpoint{2.452232in}{3.073900in}}{\pgfqpoint{2.449320in}{3.076812in}}%
\pgfpathcurveto{\pgfqpoint{2.446408in}{3.079724in}}{\pgfqpoint{2.442458in}{3.081361in}}{\pgfqpoint{2.438340in}{3.081361in}}%
\pgfpathcurveto{\pgfqpoint{2.434222in}{3.081361in}}{\pgfqpoint{2.430272in}{3.079724in}}{\pgfqpoint{2.427360in}{3.076812in}}%
\pgfpathcurveto{\pgfqpoint{2.424448in}{3.073900in}}{\pgfqpoint{2.422812in}{3.069950in}}{\pgfqpoint{2.422812in}{3.065832in}}%
\pgfpathcurveto{\pgfqpoint{2.422812in}{3.061714in}}{\pgfqpoint{2.424448in}{3.057764in}}{\pgfqpoint{2.427360in}{3.054852in}}%
\pgfpathcurveto{\pgfqpoint{2.430272in}{3.051940in}}{\pgfqpoint{2.434222in}{3.050304in}}{\pgfqpoint{2.438340in}{3.050304in}}%
\pgfpathclose%
\pgfusepath{fill}%
\end{pgfscope}%
\begin{pgfscope}%
\pgfpathrectangle{\pgfqpoint{0.887500in}{0.275000in}}{\pgfqpoint{4.225000in}{4.225000in}}%
\pgfusepath{clip}%
\pgfsetbuttcap%
\pgfsetroundjoin%
\definecolor{currentfill}{rgb}{0.000000,0.000000,0.000000}%
\pgfsetfillcolor{currentfill}%
\pgfsetfillopacity{0.800000}%
\pgfsetlinewidth{0.000000pt}%
\definecolor{currentstroke}{rgb}{0.000000,0.000000,0.000000}%
\pgfsetstrokecolor{currentstroke}%
\pgfsetstrokeopacity{0.800000}%
\pgfsetdash{}{0pt}%
\pgfpathmoveto{\pgfqpoint{3.162670in}{3.599852in}}%
\pgfpathcurveto{\pgfqpoint{3.166788in}{3.599852in}}{\pgfqpoint{3.170738in}{3.601489in}}{\pgfqpoint{3.173650in}{3.604401in}}%
\pgfpathcurveto{\pgfqpoint{3.176562in}{3.607313in}}{\pgfqpoint{3.178198in}{3.611263in}}{\pgfqpoint{3.178198in}{3.615381in}}%
\pgfpathcurveto{\pgfqpoint{3.178198in}{3.619499in}}{\pgfqpoint{3.176562in}{3.623449in}}{\pgfqpoint{3.173650in}{3.626361in}}%
\pgfpathcurveto{\pgfqpoint{3.170738in}{3.629273in}}{\pgfqpoint{3.166788in}{3.630909in}}{\pgfqpoint{3.162670in}{3.630909in}}%
\pgfpathcurveto{\pgfqpoint{3.158552in}{3.630909in}}{\pgfqpoint{3.154602in}{3.629273in}}{\pgfqpoint{3.151690in}{3.626361in}}%
\pgfpathcurveto{\pgfqpoint{3.148778in}{3.623449in}}{\pgfqpoint{3.147141in}{3.619499in}}{\pgfqpoint{3.147141in}{3.615381in}}%
\pgfpathcurveto{\pgfqpoint{3.147141in}{3.611263in}}{\pgfqpoint{3.148778in}{3.607313in}}{\pgfqpoint{3.151690in}{3.604401in}}%
\pgfpathcurveto{\pgfqpoint{3.154602in}{3.601489in}}{\pgfqpoint{3.158552in}{3.599852in}}{\pgfqpoint{3.162670in}{3.599852in}}%
\pgfpathclose%
\pgfusepath{fill}%
\end{pgfscope}%
\begin{pgfscope}%
\pgfpathrectangle{\pgfqpoint{0.887500in}{0.275000in}}{\pgfqpoint{4.225000in}{4.225000in}}%
\pgfusepath{clip}%
\pgfsetbuttcap%
\pgfsetroundjoin%
\definecolor{currentfill}{rgb}{0.000000,0.000000,0.000000}%
\pgfsetfillcolor{currentfill}%
\pgfsetfillopacity{0.800000}%
\pgfsetlinewidth{0.000000pt}%
\definecolor{currentstroke}{rgb}{0.000000,0.000000,0.000000}%
\pgfsetstrokecolor{currentstroke}%
\pgfsetstrokeopacity{0.800000}%
\pgfsetdash{}{0pt}%
\pgfpathmoveto{\pgfqpoint{2.044090in}{2.901104in}}%
\pgfpathcurveto{\pgfqpoint{2.048209in}{2.901104in}}{\pgfqpoint{2.052159in}{2.902741in}}{\pgfqpoint{2.055071in}{2.905653in}}%
\pgfpathcurveto{\pgfqpoint{2.057983in}{2.908565in}}{\pgfqpoint{2.059619in}{2.912515in}}{\pgfqpoint{2.059619in}{2.916633in}}%
\pgfpathcurveto{\pgfqpoint{2.059619in}{2.920751in}}{\pgfqpoint{2.057983in}{2.924701in}}{\pgfqpoint{2.055071in}{2.927613in}}%
\pgfpathcurveto{\pgfqpoint{2.052159in}{2.930525in}}{\pgfqpoint{2.048209in}{2.932161in}}{\pgfqpoint{2.044090in}{2.932161in}}%
\pgfpathcurveto{\pgfqpoint{2.039972in}{2.932161in}}{\pgfqpoint{2.036022in}{2.930525in}}{\pgfqpoint{2.033110in}{2.927613in}}%
\pgfpathcurveto{\pgfqpoint{2.030198in}{2.924701in}}{\pgfqpoint{2.028562in}{2.920751in}}{\pgfqpoint{2.028562in}{2.916633in}}%
\pgfpathcurveto{\pgfqpoint{2.028562in}{2.912515in}}{\pgfqpoint{2.030198in}{2.908565in}}{\pgfqpoint{2.033110in}{2.905653in}}%
\pgfpathcurveto{\pgfqpoint{2.036022in}{2.902741in}}{\pgfqpoint{2.039972in}{2.901104in}}{\pgfqpoint{2.044090in}{2.901104in}}%
\pgfpathclose%
\pgfusepath{fill}%
\end{pgfscope}%
\begin{pgfscope}%
\pgfpathrectangle{\pgfqpoint{0.887500in}{0.275000in}}{\pgfqpoint{4.225000in}{4.225000in}}%
\pgfusepath{clip}%
\pgfsetbuttcap%
\pgfsetroundjoin%
\definecolor{currentfill}{rgb}{0.000000,0.000000,0.000000}%
\pgfsetfillcolor{currentfill}%
\pgfsetfillopacity{0.800000}%
\pgfsetlinewidth{0.000000pt}%
\definecolor{currentstroke}{rgb}{0.000000,0.000000,0.000000}%
\pgfsetstrokecolor{currentstroke}%
\pgfsetstrokeopacity{0.800000}%
\pgfsetdash{}{0pt}%
\pgfpathmoveto{\pgfqpoint{3.730872in}{3.695878in}}%
\pgfpathcurveto{\pgfqpoint{3.734990in}{3.695878in}}{\pgfqpoint{3.738940in}{3.697514in}}{\pgfqpoint{3.741852in}{3.700426in}}%
\pgfpathcurveto{\pgfqpoint{3.744764in}{3.703338in}}{\pgfqpoint{3.746400in}{3.707288in}}{\pgfqpoint{3.746400in}{3.711406in}}%
\pgfpathcurveto{\pgfqpoint{3.746400in}{3.715525in}}{\pgfqpoint{3.744764in}{3.719475in}}{\pgfqpoint{3.741852in}{3.722387in}}%
\pgfpathcurveto{\pgfqpoint{3.738940in}{3.725299in}}{\pgfqpoint{3.734990in}{3.726935in}}{\pgfqpoint{3.730872in}{3.726935in}}%
\pgfpathcurveto{\pgfqpoint{3.726753in}{3.726935in}}{\pgfqpoint{3.722803in}{3.725299in}}{\pgfqpoint{3.719891in}{3.722387in}}%
\pgfpathcurveto{\pgfqpoint{3.716979in}{3.719475in}}{\pgfqpoint{3.715343in}{3.715525in}}{\pgfqpoint{3.715343in}{3.711406in}}%
\pgfpathcurveto{\pgfqpoint{3.715343in}{3.707288in}}{\pgfqpoint{3.716979in}{3.703338in}}{\pgfqpoint{3.719891in}{3.700426in}}%
\pgfpathcurveto{\pgfqpoint{3.722803in}{3.697514in}}{\pgfqpoint{3.726753in}{3.695878in}}{\pgfqpoint{3.730872in}{3.695878in}}%
\pgfpathclose%
\pgfusepath{fill}%
\end{pgfscope}%
\begin{pgfscope}%
\pgfpathrectangle{\pgfqpoint{0.887500in}{0.275000in}}{\pgfqpoint{4.225000in}{4.225000in}}%
\pgfusepath{clip}%
\pgfsetbuttcap%
\pgfsetroundjoin%
\definecolor{currentfill}{rgb}{0.000000,0.000000,0.000000}%
\pgfsetfillcolor{currentfill}%
\pgfsetfillopacity{0.800000}%
\pgfsetlinewidth{0.000000pt}%
\definecolor{currentstroke}{rgb}{0.000000,0.000000,0.000000}%
\pgfsetstrokecolor{currentstroke}%
\pgfsetstrokeopacity{0.800000}%
\pgfsetdash{}{0pt}%
\pgfpathmoveto{\pgfqpoint{3.446869in}{3.668258in}}%
\pgfpathcurveto{\pgfqpoint{3.450987in}{3.668258in}}{\pgfqpoint{3.454937in}{3.669894in}}{\pgfqpoint{3.457849in}{3.672806in}}%
\pgfpathcurveto{\pgfqpoint{3.460761in}{3.675718in}}{\pgfqpoint{3.462397in}{3.679668in}}{\pgfqpoint{3.462397in}{3.683786in}}%
\pgfpathcurveto{\pgfqpoint{3.462397in}{3.687904in}}{\pgfqpoint{3.460761in}{3.691854in}}{\pgfqpoint{3.457849in}{3.694766in}}%
\pgfpathcurveto{\pgfqpoint{3.454937in}{3.697678in}}{\pgfqpoint{3.450987in}{3.699314in}}{\pgfqpoint{3.446869in}{3.699314in}}%
\pgfpathcurveto{\pgfqpoint{3.442750in}{3.699314in}}{\pgfqpoint{3.438800in}{3.697678in}}{\pgfqpoint{3.435888in}{3.694766in}}%
\pgfpathcurveto{\pgfqpoint{3.432977in}{3.691854in}}{\pgfqpoint{3.431340in}{3.687904in}}{\pgfqpoint{3.431340in}{3.683786in}}%
\pgfpathcurveto{\pgfqpoint{3.431340in}{3.679668in}}{\pgfqpoint{3.432977in}{3.675718in}}{\pgfqpoint{3.435888in}{3.672806in}}%
\pgfpathcurveto{\pgfqpoint{3.438800in}{3.669894in}}{\pgfqpoint{3.442750in}{3.668258in}}{\pgfqpoint{3.446869in}{3.668258in}}%
\pgfpathclose%
\pgfusepath{fill}%
\end{pgfscope}%
\begin{pgfscope}%
\pgfpathrectangle{\pgfqpoint{0.887500in}{0.275000in}}{\pgfqpoint{4.225000in}{4.225000in}}%
\pgfusepath{clip}%
\pgfsetbuttcap%
\pgfsetroundjoin%
\definecolor{currentfill}{rgb}{0.000000,0.000000,0.000000}%
\pgfsetfillcolor{currentfill}%
\pgfsetfillopacity{0.800000}%
\pgfsetlinewidth{0.000000pt}%
\definecolor{currentstroke}{rgb}{0.000000,0.000000,0.000000}%
\pgfsetstrokecolor{currentstroke}%
\pgfsetstrokeopacity{0.800000}%
\pgfsetdash{}{0pt}%
\pgfpathmoveto{\pgfqpoint{2.611567in}{2.999984in}}%
\pgfpathcurveto{\pgfqpoint{2.615685in}{2.999984in}}{\pgfqpoint{2.619635in}{3.001621in}}{\pgfqpoint{2.622547in}{3.004533in}}%
\pgfpathcurveto{\pgfqpoint{2.625459in}{3.007444in}}{\pgfqpoint{2.627095in}{3.011395in}}{\pgfqpoint{2.627095in}{3.015513in}}%
\pgfpathcurveto{\pgfqpoint{2.627095in}{3.019631in}}{\pgfqpoint{2.625459in}{3.023581in}}{\pgfqpoint{2.622547in}{3.026493in}}%
\pgfpathcurveto{\pgfqpoint{2.619635in}{3.029405in}}{\pgfqpoint{2.615685in}{3.031041in}}{\pgfqpoint{2.611567in}{3.031041in}}%
\pgfpathcurveto{\pgfqpoint{2.607448in}{3.031041in}}{\pgfqpoint{2.603498in}{3.029405in}}{\pgfqpoint{2.600586in}{3.026493in}}%
\pgfpathcurveto{\pgfqpoint{2.597674in}{3.023581in}}{\pgfqpoint{2.596038in}{3.019631in}}{\pgfqpoint{2.596038in}{3.015513in}}%
\pgfpathcurveto{\pgfqpoint{2.596038in}{3.011395in}}{\pgfqpoint{2.597674in}{3.007444in}}{\pgfqpoint{2.600586in}{3.004533in}}%
\pgfpathcurveto{\pgfqpoint{2.603498in}{3.001621in}}{\pgfqpoint{2.607448in}{2.999984in}}{\pgfqpoint{2.611567in}{2.999984in}}%
\pgfpathclose%
\pgfusepath{fill}%
\end{pgfscope}%
\begin{pgfscope}%
\pgfpathrectangle{\pgfqpoint{0.887500in}{0.275000in}}{\pgfqpoint{4.225000in}{4.225000in}}%
\pgfusepath{clip}%
\pgfsetbuttcap%
\pgfsetroundjoin%
\definecolor{currentfill}{rgb}{0.000000,0.000000,0.000000}%
\pgfsetfillcolor{currentfill}%
\pgfsetfillopacity{0.800000}%
\pgfsetlinewidth{0.000000pt}%
\definecolor{currentstroke}{rgb}{0.000000,0.000000,0.000000}%
\pgfsetstrokecolor{currentstroke}%
\pgfsetstrokeopacity{0.800000}%
\pgfsetdash{}{0pt}%
\pgfpathmoveto{\pgfqpoint{2.327390in}{2.968675in}}%
\pgfpathcurveto{\pgfqpoint{2.331508in}{2.968675in}}{\pgfqpoint{2.335458in}{2.970311in}}{\pgfqpoint{2.338370in}{2.973223in}}%
\pgfpathcurveto{\pgfqpoint{2.341282in}{2.976135in}}{\pgfqpoint{2.342918in}{2.980085in}}{\pgfqpoint{2.342918in}{2.984203in}}%
\pgfpathcurveto{\pgfqpoint{2.342918in}{2.988321in}}{\pgfqpoint{2.341282in}{2.992271in}}{\pgfqpoint{2.338370in}{2.995183in}}%
\pgfpathcurveto{\pgfqpoint{2.335458in}{2.998095in}}{\pgfqpoint{2.331508in}{2.999731in}}{\pgfqpoint{2.327390in}{2.999731in}}%
\pgfpathcurveto{\pgfqpoint{2.323271in}{2.999731in}}{\pgfqpoint{2.319321in}{2.998095in}}{\pgfqpoint{2.316409in}{2.995183in}}%
\pgfpathcurveto{\pgfqpoint{2.313497in}{2.992271in}}{\pgfqpoint{2.311861in}{2.988321in}}{\pgfqpoint{2.311861in}{2.984203in}}%
\pgfpathcurveto{\pgfqpoint{2.311861in}{2.980085in}}{\pgfqpoint{2.313497in}{2.976135in}}{\pgfqpoint{2.316409in}{2.973223in}}%
\pgfpathcurveto{\pgfqpoint{2.319321in}{2.970311in}}{\pgfqpoint{2.323271in}{2.968675in}}{\pgfqpoint{2.327390in}{2.968675in}}%
\pgfpathclose%
\pgfusepath{fill}%
\end{pgfscope}%
\begin{pgfscope}%
\pgfpathrectangle{\pgfqpoint{0.887500in}{0.275000in}}{\pgfqpoint{4.225000in}{4.225000in}}%
\pgfusepath{clip}%
\pgfsetbuttcap%
\pgfsetroundjoin%
\definecolor{currentfill}{rgb}{0.000000,0.000000,0.000000}%
\pgfsetfillcolor{currentfill}%
\pgfsetfillopacity{0.800000}%
\pgfsetlinewidth{0.000000pt}%
\definecolor{currentstroke}{rgb}{0.000000,0.000000,0.000000}%
\pgfsetstrokecolor{currentstroke}%
\pgfsetstrokeopacity{0.800000}%
\pgfsetdash{}{0pt}%
\pgfpathmoveto{\pgfqpoint{1.758771in}{2.910545in}}%
\pgfpathcurveto{\pgfqpoint{1.762889in}{2.910545in}}{\pgfqpoint{1.766839in}{2.912181in}}{\pgfqpoint{1.769751in}{2.915093in}}%
\pgfpathcurveto{\pgfqpoint{1.772663in}{2.918005in}}{\pgfqpoint{1.774299in}{2.921955in}}{\pgfqpoint{1.774299in}{2.926073in}}%
\pgfpathcurveto{\pgfqpoint{1.774299in}{2.930191in}}{\pgfqpoint{1.772663in}{2.934141in}}{\pgfqpoint{1.769751in}{2.937053in}}%
\pgfpathcurveto{\pgfqpoint{1.766839in}{2.939965in}}{\pgfqpoint{1.762889in}{2.941601in}}{\pgfqpoint{1.758771in}{2.941601in}}%
\pgfpathcurveto{\pgfqpoint{1.754652in}{2.941601in}}{\pgfqpoint{1.750702in}{2.939965in}}{\pgfqpoint{1.747790in}{2.937053in}}%
\pgfpathcurveto{\pgfqpoint{1.744878in}{2.934141in}}{\pgfqpoint{1.743242in}{2.930191in}}{\pgfqpoint{1.743242in}{2.926073in}}%
\pgfpathcurveto{\pgfqpoint{1.743242in}{2.921955in}}{\pgfqpoint{1.744878in}{2.918005in}}{\pgfqpoint{1.747790in}{2.915093in}}%
\pgfpathcurveto{\pgfqpoint{1.750702in}{2.912181in}}{\pgfqpoint{1.754652in}{2.910545in}}{\pgfqpoint{1.758771in}{2.910545in}}%
\pgfpathclose%
\pgfusepath{fill}%
\end{pgfscope}%
\begin{pgfscope}%
\pgfpathrectangle{\pgfqpoint{0.887500in}{0.275000in}}{\pgfqpoint{4.225000in}{4.225000in}}%
\pgfusepath{clip}%
\pgfsetbuttcap%
\pgfsetroundjoin%
\definecolor{currentfill}{rgb}{0.000000,0.000000,0.000000}%
\pgfsetfillcolor{currentfill}%
\pgfsetfillopacity{0.800000}%
\pgfsetlinewidth{0.000000pt}%
\definecolor{currentstroke}{rgb}{0.000000,0.000000,0.000000}%
\pgfsetstrokecolor{currentstroke}%
\pgfsetstrokeopacity{0.800000}%
\pgfsetdash{}{0pt}%
\pgfpathmoveto{\pgfqpoint{2.216864in}{2.854597in}}%
\pgfpathcurveto{\pgfqpoint{2.220982in}{2.854597in}}{\pgfqpoint{2.224933in}{2.856233in}}{\pgfqpoint{2.227844in}{2.859145in}}%
\pgfpathcurveto{\pgfqpoint{2.230756in}{2.862057in}}{\pgfqpoint{2.232393in}{2.866007in}}{\pgfqpoint{2.232393in}{2.870125in}}%
\pgfpathcurveto{\pgfqpoint{2.232393in}{2.874244in}}{\pgfqpoint{2.230756in}{2.878194in}}{\pgfqpoint{2.227844in}{2.881106in}}%
\pgfpathcurveto{\pgfqpoint{2.224933in}{2.884018in}}{\pgfqpoint{2.220982in}{2.885654in}}{\pgfqpoint{2.216864in}{2.885654in}}%
\pgfpathcurveto{\pgfqpoint{2.212746in}{2.885654in}}{\pgfqpoint{2.208796in}{2.884018in}}{\pgfqpoint{2.205884in}{2.881106in}}%
\pgfpathcurveto{\pgfqpoint{2.202972in}{2.878194in}}{\pgfqpoint{2.201336in}{2.874244in}}{\pgfqpoint{2.201336in}{2.870125in}}%
\pgfpathcurveto{\pgfqpoint{2.201336in}{2.866007in}}{\pgfqpoint{2.202972in}{2.862057in}}{\pgfqpoint{2.205884in}{2.859145in}}%
\pgfpathcurveto{\pgfqpoint{2.208796in}{2.856233in}}{\pgfqpoint{2.212746in}{2.854597in}}{\pgfqpoint{2.216864in}{2.854597in}}%
\pgfpathclose%
\pgfusepath{fill}%
\end{pgfscope}%
\begin{pgfscope}%
\pgfpathrectangle{\pgfqpoint{0.887500in}{0.275000in}}{\pgfqpoint{4.225000in}{4.225000in}}%
\pgfusepath{clip}%
\pgfsetbuttcap%
\pgfsetroundjoin%
\definecolor{currentfill}{rgb}{0.000000,0.000000,0.000000}%
\pgfsetfillcolor{currentfill}%
\pgfsetfillopacity{0.800000}%
\pgfsetlinewidth{0.000000pt}%
\definecolor{currentstroke}{rgb}{0.000000,0.000000,0.000000}%
\pgfsetstrokecolor{currentstroke}%
\pgfsetstrokeopacity{0.800000}%
\pgfsetdash{}{0pt}%
\pgfpathmoveto{\pgfqpoint{3.905608in}{3.645339in}}%
\pgfpathcurveto{\pgfqpoint{3.909726in}{3.645339in}}{\pgfqpoint{3.913676in}{3.646976in}}{\pgfqpoint{3.916588in}{3.649888in}}%
\pgfpathcurveto{\pgfqpoint{3.919500in}{3.652800in}}{\pgfqpoint{3.921136in}{3.656750in}}{\pgfqpoint{3.921136in}{3.660868in}}%
\pgfpathcurveto{\pgfqpoint{3.921136in}{3.664986in}}{\pgfqpoint{3.919500in}{3.668936in}}{\pgfqpoint{3.916588in}{3.671848in}}%
\pgfpathcurveto{\pgfqpoint{3.913676in}{3.674760in}}{\pgfqpoint{3.909726in}{3.676396in}}{\pgfqpoint{3.905608in}{3.676396in}}%
\pgfpathcurveto{\pgfqpoint{3.901490in}{3.676396in}}{\pgfqpoint{3.897540in}{3.674760in}}{\pgfqpoint{3.894628in}{3.671848in}}%
\pgfpathcurveto{\pgfqpoint{3.891716in}{3.668936in}}{\pgfqpoint{3.890079in}{3.664986in}}{\pgfqpoint{3.890079in}{3.660868in}}%
\pgfpathcurveto{\pgfqpoint{3.890079in}{3.656750in}}{\pgfqpoint{3.891716in}{3.652800in}}{\pgfqpoint{3.894628in}{3.649888in}}%
\pgfpathcurveto{\pgfqpoint{3.897540in}{3.646976in}}{\pgfqpoint{3.901490in}{3.645339in}}{\pgfqpoint{3.905608in}{3.645339in}}%
\pgfpathclose%
\pgfusepath{fill}%
\end{pgfscope}%
\begin{pgfscope}%
\pgfpathrectangle{\pgfqpoint{0.887500in}{0.275000in}}{\pgfqpoint{4.225000in}{4.225000in}}%
\pgfusepath{clip}%
\pgfsetbuttcap%
\pgfsetroundjoin%
\definecolor{currentfill}{rgb}{0.000000,0.000000,0.000000}%
\pgfsetfillcolor{currentfill}%
\pgfsetfillopacity{0.800000}%
\pgfsetlinewidth{0.000000pt}%
\definecolor{currentstroke}{rgb}{0.000000,0.000000,0.000000}%
\pgfsetstrokecolor{currentstroke}%
\pgfsetstrokeopacity{0.800000}%
\pgfsetdash{}{0pt}%
\pgfpathmoveto{\pgfqpoint{3.052521in}{3.544202in}}%
\pgfpathcurveto{\pgfqpoint{3.056639in}{3.544202in}}{\pgfqpoint{3.060589in}{3.545838in}}{\pgfqpoint{3.063501in}{3.548750in}}%
\pgfpathcurveto{\pgfqpoint{3.066413in}{3.551662in}}{\pgfqpoint{3.068049in}{3.555612in}}{\pgfqpoint{3.068049in}{3.559730in}}%
\pgfpathcurveto{\pgfqpoint{3.068049in}{3.563848in}}{\pgfqpoint{3.066413in}{3.567798in}}{\pgfqpoint{3.063501in}{3.570710in}}%
\pgfpathcurveto{\pgfqpoint{3.060589in}{3.573622in}}{\pgfqpoint{3.056639in}{3.575258in}}{\pgfqpoint{3.052521in}{3.575258in}}%
\pgfpathcurveto{\pgfqpoint{3.048403in}{3.575258in}}{\pgfqpoint{3.044453in}{3.573622in}}{\pgfqpoint{3.041541in}{3.570710in}}%
\pgfpathcurveto{\pgfqpoint{3.038629in}{3.567798in}}{\pgfqpoint{3.036993in}{3.563848in}}{\pgfqpoint{3.036993in}{3.559730in}}%
\pgfpathcurveto{\pgfqpoint{3.036993in}{3.555612in}}{\pgfqpoint{3.038629in}{3.551662in}}{\pgfqpoint{3.041541in}{3.548750in}}%
\pgfpathcurveto{\pgfqpoint{3.044453in}{3.545838in}}{\pgfqpoint{3.048403in}{3.544202in}}{\pgfqpoint{3.052521in}{3.544202in}}%
\pgfpathclose%
\pgfusepath{fill}%
\end{pgfscope}%
\begin{pgfscope}%
\pgfpathrectangle{\pgfqpoint{0.887500in}{0.275000in}}{\pgfqpoint{4.225000in}{4.225000in}}%
\pgfusepath{clip}%
\pgfsetbuttcap%
\pgfsetroundjoin%
\definecolor{currentfill}{rgb}{0.000000,0.000000,0.000000}%
\pgfsetfillcolor{currentfill}%
\pgfsetfillopacity{0.800000}%
\pgfsetlinewidth{0.000000pt}%
\definecolor{currentstroke}{rgb}{0.000000,0.000000,0.000000}%
\pgfsetstrokecolor{currentstroke}%
\pgfsetstrokeopacity{0.800000}%
\pgfsetdash{}{0pt}%
\pgfpathmoveto{\pgfqpoint{1.931337in}{2.866206in}}%
\pgfpathcurveto{\pgfqpoint{1.935455in}{2.866206in}}{\pgfqpoint{1.939405in}{2.867843in}}{\pgfqpoint{1.942317in}{2.870754in}}%
\pgfpathcurveto{\pgfqpoint{1.945229in}{2.873666in}}{\pgfqpoint{1.946865in}{2.877616in}}{\pgfqpoint{1.946865in}{2.881735in}}%
\pgfpathcurveto{\pgfqpoint{1.946865in}{2.885853in}}{\pgfqpoint{1.945229in}{2.889803in}}{\pgfqpoint{1.942317in}{2.892715in}}%
\pgfpathcurveto{\pgfqpoint{1.939405in}{2.895627in}}{\pgfqpoint{1.935455in}{2.897263in}}{\pgfqpoint{1.931337in}{2.897263in}}%
\pgfpathcurveto{\pgfqpoint{1.927219in}{2.897263in}}{\pgfqpoint{1.923269in}{2.895627in}}{\pgfqpoint{1.920357in}{2.892715in}}%
\pgfpathcurveto{\pgfqpoint{1.917445in}{2.889803in}}{\pgfqpoint{1.915809in}{2.885853in}}{\pgfqpoint{1.915809in}{2.881735in}}%
\pgfpathcurveto{\pgfqpoint{1.915809in}{2.877616in}}{\pgfqpoint{1.917445in}{2.873666in}}{\pgfqpoint{1.920357in}{2.870754in}}%
\pgfpathcurveto{\pgfqpoint{1.923269in}{2.867843in}}{\pgfqpoint{1.927219in}{2.866206in}}{\pgfqpoint{1.931337in}{2.866206in}}%
\pgfpathclose%
\pgfusepath{fill}%
\end{pgfscope}%
\begin{pgfscope}%
\pgfpathrectangle{\pgfqpoint{0.887500in}{0.275000in}}{\pgfqpoint{4.225000in}{4.225000in}}%
\pgfusepath{clip}%
\pgfsetbuttcap%
\pgfsetroundjoin%
\definecolor{currentfill}{rgb}{0.000000,0.000000,0.000000}%
\pgfsetfillcolor{currentfill}%
\pgfsetfillopacity{0.800000}%
\pgfsetlinewidth{0.000000pt}%
\definecolor{currentstroke}{rgb}{0.000000,0.000000,0.000000}%
\pgfsetstrokecolor{currentstroke}%
\pgfsetstrokeopacity{0.800000}%
\pgfsetdash{}{0pt}%
\pgfpathmoveto{\pgfqpoint{3.337117in}{3.611036in}}%
\pgfpathcurveto{\pgfqpoint{3.341235in}{3.611036in}}{\pgfqpoint{3.345185in}{3.612672in}}{\pgfqpoint{3.348097in}{3.615584in}}%
\pgfpathcurveto{\pgfqpoint{3.351009in}{3.618496in}}{\pgfqpoint{3.352645in}{3.622446in}}{\pgfqpoint{3.352645in}{3.626564in}}%
\pgfpathcurveto{\pgfqpoint{3.352645in}{3.630682in}}{\pgfqpoint{3.351009in}{3.634632in}}{\pgfqpoint{3.348097in}{3.637544in}}%
\pgfpathcurveto{\pgfqpoint{3.345185in}{3.640456in}}{\pgfqpoint{3.341235in}{3.642092in}}{\pgfqpoint{3.337117in}{3.642092in}}%
\pgfpathcurveto{\pgfqpoint{3.332999in}{3.642092in}}{\pgfqpoint{3.329049in}{3.640456in}}{\pgfqpoint{3.326137in}{3.637544in}}%
\pgfpathcurveto{\pgfqpoint{3.323225in}{3.634632in}}{\pgfqpoint{3.321589in}{3.630682in}}{\pgfqpoint{3.321589in}{3.626564in}}%
\pgfpathcurveto{\pgfqpoint{3.321589in}{3.622446in}}{\pgfqpoint{3.323225in}{3.618496in}}{\pgfqpoint{3.326137in}{3.615584in}}%
\pgfpathcurveto{\pgfqpoint{3.329049in}{3.612672in}}{\pgfqpoint{3.332999in}{3.611036in}}{\pgfqpoint{3.337117in}{3.611036in}}%
\pgfpathclose%
\pgfusepath{fill}%
\end{pgfscope}%
\begin{pgfscope}%
\pgfpathrectangle{\pgfqpoint{0.887500in}{0.275000in}}{\pgfqpoint{4.225000in}{4.225000in}}%
\pgfusepath{clip}%
\pgfsetbuttcap%
\pgfsetroundjoin%
\definecolor{currentfill}{rgb}{0.000000,0.000000,0.000000}%
\pgfsetfillcolor{currentfill}%
\pgfsetfillopacity{0.800000}%
\pgfsetlinewidth{0.000000pt}%
\definecolor{currentstroke}{rgb}{0.000000,0.000000,0.000000}%
\pgfsetstrokecolor{currentstroke}%
\pgfsetstrokeopacity{0.800000}%
\pgfsetdash{}{0pt}%
\pgfpathmoveto{\pgfqpoint{2.500399in}{2.939480in}}%
\pgfpathcurveto{\pgfqpoint{2.504517in}{2.939480in}}{\pgfqpoint{2.508467in}{2.941116in}}{\pgfqpoint{2.511379in}{2.944028in}}%
\pgfpathcurveto{\pgfqpoint{2.514291in}{2.946940in}}{\pgfqpoint{2.515927in}{2.950890in}}{\pgfqpoint{2.515927in}{2.955008in}}%
\pgfpathcurveto{\pgfqpoint{2.515927in}{2.959126in}}{\pgfqpoint{2.514291in}{2.963076in}}{\pgfqpoint{2.511379in}{2.965988in}}%
\pgfpathcurveto{\pgfqpoint{2.508467in}{2.968900in}}{\pgfqpoint{2.504517in}{2.970536in}}{\pgfqpoint{2.500399in}{2.970536in}}%
\pgfpathcurveto{\pgfqpoint{2.496281in}{2.970536in}}{\pgfqpoint{2.492331in}{2.968900in}}{\pgfqpoint{2.489419in}{2.965988in}}%
\pgfpathcurveto{\pgfqpoint{2.486507in}{2.963076in}}{\pgfqpoint{2.484871in}{2.959126in}}{\pgfqpoint{2.484871in}{2.955008in}}%
\pgfpathcurveto{\pgfqpoint{2.484871in}{2.950890in}}{\pgfqpoint{2.486507in}{2.946940in}}{\pgfqpoint{2.489419in}{2.944028in}}%
\pgfpathcurveto{\pgfqpoint{2.492331in}{2.941116in}}{\pgfqpoint{2.496281in}{2.939480in}}{\pgfqpoint{2.500399in}{2.939480in}}%
\pgfpathclose%
\pgfusepath{fill}%
\end{pgfscope}%
\begin{pgfscope}%
\pgfpathrectangle{\pgfqpoint{0.887500in}{0.275000in}}{\pgfqpoint{4.225000in}{4.225000in}}%
\pgfusepath{clip}%
\pgfsetbuttcap%
\pgfsetroundjoin%
\definecolor{currentfill}{rgb}{0.000000,0.000000,0.000000}%
\pgfsetfillcolor{currentfill}%
\pgfsetfillopacity{0.800000}%
\pgfsetlinewidth{0.000000pt}%
\definecolor{currentstroke}{rgb}{0.000000,0.000000,0.000000}%
\pgfsetstrokecolor{currentstroke}%
\pgfsetstrokeopacity{0.800000}%
\pgfsetdash{}{0pt}%
\pgfpathmoveto{\pgfqpoint{3.621721in}{3.645555in}}%
\pgfpathcurveto{\pgfqpoint{3.625839in}{3.645555in}}{\pgfqpoint{3.629789in}{3.647191in}}{\pgfqpoint{3.632701in}{3.650103in}}%
\pgfpathcurveto{\pgfqpoint{3.635613in}{3.653015in}}{\pgfqpoint{3.637249in}{3.656965in}}{\pgfqpoint{3.637249in}{3.661083in}}%
\pgfpathcurveto{\pgfqpoint{3.637249in}{3.665202in}}{\pgfqpoint{3.635613in}{3.669152in}}{\pgfqpoint{3.632701in}{3.672064in}}%
\pgfpathcurveto{\pgfqpoint{3.629789in}{3.674976in}}{\pgfqpoint{3.625839in}{3.676612in}}{\pgfqpoint{3.621721in}{3.676612in}}%
\pgfpathcurveto{\pgfqpoint{3.617603in}{3.676612in}}{\pgfqpoint{3.613653in}{3.674976in}}{\pgfqpoint{3.610741in}{3.672064in}}%
\pgfpathcurveto{\pgfqpoint{3.607829in}{3.669152in}}{\pgfqpoint{3.606193in}{3.665202in}}{\pgfqpoint{3.606193in}{3.661083in}}%
\pgfpathcurveto{\pgfqpoint{3.606193in}{3.656965in}}{\pgfqpoint{3.607829in}{3.653015in}}{\pgfqpoint{3.610741in}{3.650103in}}%
\pgfpathcurveto{\pgfqpoint{3.613653in}{3.647191in}}{\pgfqpoint{3.617603in}{3.645555in}}{\pgfqpoint{3.621721in}{3.645555in}}%
\pgfpathclose%
\pgfusepath{fill}%
\end{pgfscope}%
\begin{pgfscope}%
\pgfpathrectangle{\pgfqpoint{0.887500in}{0.275000in}}{\pgfqpoint{4.225000in}{4.225000in}}%
\pgfusepath{clip}%
\pgfsetbuttcap%
\pgfsetroundjoin%
\definecolor{currentfill}{rgb}{0.000000,0.000000,0.000000}%
\pgfsetfillcolor{currentfill}%
\pgfsetfillopacity{0.800000}%
\pgfsetlinewidth{0.000000pt}%
\definecolor{currentstroke}{rgb}{0.000000,0.000000,0.000000}%
\pgfsetstrokecolor{currentstroke}%
\pgfsetstrokeopacity{0.800000}%
\pgfsetdash{}{0pt}%
\pgfpathmoveto{\pgfqpoint{1.645189in}{2.876906in}}%
\pgfpathcurveto{\pgfqpoint{1.649307in}{2.876906in}}{\pgfqpoint{1.653257in}{2.878543in}}{\pgfqpoint{1.656169in}{2.881454in}}%
\pgfpathcurveto{\pgfqpoint{1.659081in}{2.884366in}}{\pgfqpoint{1.660717in}{2.888316in}}{\pgfqpoint{1.660717in}{2.892435in}}%
\pgfpathcurveto{\pgfqpoint{1.660717in}{2.896553in}}{\pgfqpoint{1.659081in}{2.900503in}}{\pgfqpoint{1.656169in}{2.903415in}}%
\pgfpathcurveto{\pgfqpoint{1.653257in}{2.906327in}}{\pgfqpoint{1.649307in}{2.907963in}}{\pgfqpoint{1.645189in}{2.907963in}}%
\pgfpathcurveto{\pgfqpoint{1.641070in}{2.907963in}}{\pgfqpoint{1.637120in}{2.906327in}}{\pgfqpoint{1.634208in}{2.903415in}}%
\pgfpathcurveto{\pgfqpoint{1.631296in}{2.900503in}}{\pgfqpoint{1.629660in}{2.896553in}}{\pgfqpoint{1.629660in}{2.892435in}}%
\pgfpathcurveto{\pgfqpoint{1.629660in}{2.888316in}}{\pgfqpoint{1.631296in}{2.884366in}}{\pgfqpoint{1.634208in}{2.881454in}}%
\pgfpathcurveto{\pgfqpoint{1.637120in}{2.878543in}}{\pgfqpoint{1.641070in}{2.876906in}}{\pgfqpoint{1.645189in}{2.876906in}}%
\pgfpathclose%
\pgfusepath{fill}%
\end{pgfscope}%
\begin{pgfscope}%
\pgfpathrectangle{\pgfqpoint{0.887500in}{0.275000in}}{\pgfqpoint{4.225000in}{4.225000in}}%
\pgfusepath{clip}%
\pgfsetbuttcap%
\pgfsetroundjoin%
\definecolor{currentfill}{rgb}{0.000000,0.000000,0.000000}%
\pgfsetfillcolor{currentfill}%
\pgfsetfillopacity{0.800000}%
\pgfsetlinewidth{0.000000pt}%
\definecolor{currentstroke}{rgb}{0.000000,0.000000,0.000000}%
\pgfsetstrokecolor{currentstroke}%
\pgfsetstrokeopacity{0.800000}%
\pgfsetdash{}{0pt}%
\pgfpathmoveto{\pgfqpoint{2.104371in}{2.819890in}}%
\pgfpathcurveto{\pgfqpoint{2.108489in}{2.819890in}}{\pgfqpoint{2.112439in}{2.821527in}}{\pgfqpoint{2.115351in}{2.824439in}}%
\pgfpathcurveto{\pgfqpoint{2.118263in}{2.827351in}}{\pgfqpoint{2.119899in}{2.831301in}}{\pgfqpoint{2.119899in}{2.835419in}}%
\pgfpathcurveto{\pgfqpoint{2.119899in}{2.839537in}}{\pgfqpoint{2.118263in}{2.843487in}}{\pgfqpoint{2.115351in}{2.846399in}}%
\pgfpathcurveto{\pgfqpoint{2.112439in}{2.849311in}}{\pgfqpoint{2.108489in}{2.850947in}}{\pgfqpoint{2.104371in}{2.850947in}}%
\pgfpathcurveto{\pgfqpoint{2.100253in}{2.850947in}}{\pgfqpoint{2.096303in}{2.849311in}}{\pgfqpoint{2.093391in}{2.846399in}}%
\pgfpathcurveto{\pgfqpoint{2.090479in}{2.843487in}}{\pgfqpoint{2.088843in}{2.839537in}}{\pgfqpoint{2.088843in}{2.835419in}}%
\pgfpathcurveto{\pgfqpoint{2.088843in}{2.831301in}}{\pgfqpoint{2.090479in}{2.827351in}}{\pgfqpoint{2.093391in}{2.824439in}}%
\pgfpathcurveto{\pgfqpoint{2.096303in}{2.821527in}}{\pgfqpoint{2.100253in}{2.819890in}}{\pgfqpoint{2.104371in}{2.819890in}}%
\pgfpathclose%
\pgfusepath{fill}%
\end{pgfscope}%
\begin{pgfscope}%
\pgfpathrectangle{\pgfqpoint{0.887500in}{0.275000in}}{\pgfqpoint{4.225000in}{4.225000in}}%
\pgfusepath{clip}%
\pgfsetbuttcap%
\pgfsetroundjoin%
\definecolor{currentfill}{rgb}{0.000000,0.000000,0.000000}%
\pgfsetfillcolor{currentfill}%
\pgfsetfillopacity{0.800000}%
\pgfsetlinewidth{0.000000pt}%
\definecolor{currentstroke}{rgb}{0.000000,0.000000,0.000000}%
\pgfsetstrokecolor{currentstroke}%
\pgfsetstrokeopacity{0.800000}%
\pgfsetdash{}{0pt}%
\pgfpathmoveto{\pgfqpoint{3.796763in}{3.599805in}}%
\pgfpathcurveto{\pgfqpoint{3.800881in}{3.599805in}}{\pgfqpoint{3.804831in}{3.601441in}}{\pgfqpoint{3.807743in}{3.604353in}}%
\pgfpathcurveto{\pgfqpoint{3.810655in}{3.607265in}}{\pgfqpoint{3.812292in}{3.611215in}}{\pgfqpoint{3.812292in}{3.615333in}}%
\pgfpathcurveto{\pgfqpoint{3.812292in}{3.619452in}}{\pgfqpoint{3.810655in}{3.623402in}}{\pgfqpoint{3.807743in}{3.626314in}}%
\pgfpathcurveto{\pgfqpoint{3.804831in}{3.629226in}}{\pgfqpoint{3.800881in}{3.630862in}}{\pgfqpoint{3.796763in}{3.630862in}}%
\pgfpathcurveto{\pgfqpoint{3.792645in}{3.630862in}}{\pgfqpoint{3.788695in}{3.629226in}}{\pgfqpoint{3.785783in}{3.626314in}}%
\pgfpathcurveto{\pgfqpoint{3.782871in}{3.623402in}}{\pgfqpoint{3.781235in}{3.619452in}}{\pgfqpoint{3.781235in}{3.615333in}}%
\pgfpathcurveto{\pgfqpoint{3.781235in}{3.611215in}}{\pgfqpoint{3.782871in}{3.607265in}}{\pgfqpoint{3.785783in}{3.604353in}}%
\pgfpathcurveto{\pgfqpoint{3.788695in}{3.601441in}}{\pgfqpoint{3.792645in}{3.599805in}}{\pgfqpoint{3.796763in}{3.599805in}}%
\pgfpathclose%
\pgfusepath{fill}%
\end{pgfscope}%
\begin{pgfscope}%
\pgfpathrectangle{\pgfqpoint{0.887500in}{0.275000in}}{\pgfqpoint{4.225000in}{4.225000in}}%
\pgfusepath{clip}%
\pgfsetbuttcap%
\pgfsetroundjoin%
\definecolor{currentfill}{rgb}{0.000000,0.000000,0.000000}%
\pgfsetfillcolor{currentfill}%
\pgfsetfillopacity{0.800000}%
\pgfsetlinewidth{0.000000pt}%
\definecolor{currentstroke}{rgb}{0.000000,0.000000,0.000000}%
\pgfsetstrokecolor{currentstroke}%
\pgfsetstrokeopacity{0.800000}%
\pgfsetdash{}{0pt}%
\pgfpathmoveto{\pgfqpoint{2.784674in}{3.021589in}}%
\pgfpathcurveto{\pgfqpoint{2.788792in}{3.021589in}}{\pgfqpoint{2.792742in}{3.023225in}}{\pgfqpoint{2.795654in}{3.026137in}}%
\pgfpathcurveto{\pgfqpoint{2.798566in}{3.029049in}}{\pgfqpoint{2.800202in}{3.032999in}}{\pgfqpoint{2.800202in}{3.037117in}}%
\pgfpathcurveto{\pgfqpoint{2.800202in}{3.041236in}}{\pgfqpoint{2.798566in}{3.045186in}}{\pgfqpoint{2.795654in}{3.048098in}}%
\pgfpathcurveto{\pgfqpoint{2.792742in}{3.051010in}}{\pgfqpoint{2.788792in}{3.052646in}}{\pgfqpoint{2.784674in}{3.052646in}}%
\pgfpathcurveto{\pgfqpoint{2.780556in}{3.052646in}}{\pgfqpoint{2.776606in}{3.051010in}}{\pgfqpoint{2.773694in}{3.048098in}}%
\pgfpathcurveto{\pgfqpoint{2.770782in}{3.045186in}}{\pgfqpoint{2.769146in}{3.041236in}}{\pgfqpoint{2.769146in}{3.037117in}}%
\pgfpathcurveto{\pgfqpoint{2.769146in}{3.032999in}}{\pgfqpoint{2.770782in}{3.029049in}}{\pgfqpoint{2.773694in}{3.026137in}}%
\pgfpathcurveto{\pgfqpoint{2.776606in}{3.023225in}}{\pgfqpoint{2.780556in}{3.021589in}}{\pgfqpoint{2.784674in}{3.021589in}}%
\pgfpathclose%
\pgfusepath{fill}%
\end{pgfscope}%
\begin{pgfscope}%
\pgfpathrectangle{\pgfqpoint{0.887500in}{0.275000in}}{\pgfqpoint{4.225000in}{4.225000in}}%
\pgfusepath{clip}%
\pgfsetbuttcap%
\pgfsetroundjoin%
\definecolor{currentfill}{rgb}{0.000000,0.000000,0.000000}%
\pgfsetfillcolor{currentfill}%
\pgfsetfillopacity{0.800000}%
\pgfsetlinewidth{0.000000pt}%
\definecolor{currentstroke}{rgb}{0.000000,0.000000,0.000000}%
\pgfsetstrokecolor{currentstroke}%
\pgfsetstrokeopacity{0.800000}%
\pgfsetdash{}{0pt}%
\pgfpathmoveto{\pgfqpoint{2.941926in}{3.508748in}}%
\pgfpathcurveto{\pgfqpoint{2.946044in}{3.508748in}}{\pgfqpoint{2.949994in}{3.510384in}}{\pgfqpoint{2.952906in}{3.513296in}}%
\pgfpathcurveto{\pgfqpoint{2.955818in}{3.516208in}}{\pgfqpoint{2.957454in}{3.520158in}}{\pgfqpoint{2.957454in}{3.524276in}}%
\pgfpathcurveto{\pgfqpoint{2.957454in}{3.528394in}}{\pgfqpoint{2.955818in}{3.532344in}}{\pgfqpoint{2.952906in}{3.535256in}}%
\pgfpathcurveto{\pgfqpoint{2.949994in}{3.538168in}}{\pgfqpoint{2.946044in}{3.539805in}}{\pgfqpoint{2.941926in}{3.539805in}}%
\pgfpathcurveto{\pgfqpoint{2.937808in}{3.539805in}}{\pgfqpoint{2.933858in}{3.538168in}}{\pgfqpoint{2.930946in}{3.535256in}}%
\pgfpathcurveto{\pgfqpoint{2.928034in}{3.532344in}}{\pgfqpoint{2.926398in}{3.528394in}}{\pgfqpoint{2.926398in}{3.524276in}}%
\pgfpathcurveto{\pgfqpoint{2.926398in}{3.520158in}}{\pgfqpoint{2.928034in}{3.516208in}}{\pgfqpoint{2.930946in}{3.513296in}}%
\pgfpathcurveto{\pgfqpoint{2.933858in}{3.510384in}}{\pgfqpoint{2.937808in}{3.508748in}}{\pgfqpoint{2.941926in}{3.508748in}}%
\pgfpathclose%
\pgfusepath{fill}%
\end{pgfscope}%
\begin{pgfscope}%
\pgfpathrectangle{\pgfqpoint{0.887500in}{0.275000in}}{\pgfqpoint{4.225000in}{4.225000in}}%
\pgfusepath{clip}%
\pgfsetbuttcap%
\pgfsetroundjoin%
\definecolor{currentfill}{rgb}{0.000000,0.000000,0.000000}%
\pgfsetfillcolor{currentfill}%
\pgfsetfillopacity{0.800000}%
\pgfsetlinewidth{0.000000pt}%
\definecolor{currentstroke}{rgb}{0.000000,0.000000,0.000000}%
\pgfsetstrokecolor{currentstroke}%
\pgfsetstrokeopacity{0.800000}%
\pgfsetdash{}{0pt}%
\pgfpathmoveto{\pgfqpoint{3.971516in}{3.522872in}}%
\pgfpathcurveto{\pgfqpoint{3.975634in}{3.522872in}}{\pgfqpoint{3.979584in}{3.524508in}}{\pgfqpoint{3.982496in}{3.527420in}}%
\pgfpathcurveto{\pgfqpoint{3.985408in}{3.530332in}}{\pgfqpoint{3.987045in}{3.534282in}}{\pgfqpoint{3.987045in}{3.538400in}}%
\pgfpathcurveto{\pgfqpoint{3.987045in}{3.542518in}}{\pgfqpoint{3.985408in}{3.546468in}}{\pgfqpoint{3.982496in}{3.549380in}}%
\pgfpathcurveto{\pgfqpoint{3.979584in}{3.552292in}}{\pgfqpoint{3.975634in}{3.553928in}}{\pgfqpoint{3.971516in}{3.553928in}}%
\pgfpathcurveto{\pgfqpoint{3.967398in}{3.553928in}}{\pgfqpoint{3.963448in}{3.552292in}}{\pgfqpoint{3.960536in}{3.549380in}}%
\pgfpathcurveto{\pgfqpoint{3.957624in}{3.546468in}}{\pgfqpoint{3.955988in}{3.542518in}}{\pgfqpoint{3.955988in}{3.538400in}}%
\pgfpathcurveto{\pgfqpoint{3.955988in}{3.534282in}}{\pgfqpoint{3.957624in}{3.530332in}}{\pgfqpoint{3.960536in}{3.527420in}}%
\pgfpathcurveto{\pgfqpoint{3.963448in}{3.524508in}}{\pgfqpoint{3.967398in}{3.522872in}}{\pgfqpoint{3.971516in}{3.522872in}}%
\pgfpathclose%
\pgfusepath{fill}%
\end{pgfscope}%
\begin{pgfscope}%
\pgfpathrectangle{\pgfqpoint{0.887500in}{0.275000in}}{\pgfqpoint{4.225000in}{4.225000in}}%
\pgfusepath{clip}%
\pgfsetbuttcap%
\pgfsetroundjoin%
\definecolor{currentfill}{rgb}{0.000000,0.000000,0.000000}%
\pgfsetfillcolor{currentfill}%
\pgfsetfillopacity{0.800000}%
\pgfsetlinewidth{0.000000pt}%
\definecolor{currentstroke}{rgb}{0.000000,0.000000,0.000000}%
\pgfsetstrokecolor{currentstroke}%
\pgfsetstrokeopacity{0.800000}%
\pgfsetdash{}{0pt}%
\pgfpathmoveto{\pgfqpoint{2.388865in}{2.878433in}}%
\pgfpathcurveto{\pgfqpoint{2.392983in}{2.878433in}}{\pgfqpoint{2.396933in}{2.880069in}}{\pgfqpoint{2.399845in}{2.882981in}}%
\pgfpathcurveto{\pgfqpoint{2.402757in}{2.885893in}}{\pgfqpoint{2.404393in}{2.889843in}}{\pgfqpoint{2.404393in}{2.893961in}}%
\pgfpathcurveto{\pgfqpoint{2.404393in}{2.898079in}}{\pgfqpoint{2.402757in}{2.902029in}}{\pgfqpoint{2.399845in}{2.904941in}}%
\pgfpathcurveto{\pgfqpoint{2.396933in}{2.907853in}}{\pgfqpoint{2.392983in}{2.909489in}}{\pgfqpoint{2.388865in}{2.909489in}}%
\pgfpathcurveto{\pgfqpoint{2.384747in}{2.909489in}}{\pgfqpoint{2.380797in}{2.907853in}}{\pgfqpoint{2.377885in}{2.904941in}}%
\pgfpathcurveto{\pgfqpoint{2.374973in}{2.902029in}}{\pgfqpoint{2.373337in}{2.898079in}}{\pgfqpoint{2.373337in}{2.893961in}}%
\pgfpathcurveto{\pgfqpoint{2.373337in}{2.889843in}}{\pgfqpoint{2.374973in}{2.885893in}}{\pgfqpoint{2.377885in}{2.882981in}}%
\pgfpathcurveto{\pgfqpoint{2.380797in}{2.880069in}}{\pgfqpoint{2.384747in}{2.878433in}}{\pgfqpoint{2.388865in}{2.878433in}}%
\pgfpathclose%
\pgfusepath{fill}%
\end{pgfscope}%
\begin{pgfscope}%
\pgfpathrectangle{\pgfqpoint{0.887500in}{0.275000in}}{\pgfqpoint{4.225000in}{4.225000in}}%
\pgfusepath{clip}%
\pgfsetbuttcap%
\pgfsetroundjoin%
\definecolor{currentfill}{rgb}{0.000000,0.000000,0.000000}%
\pgfsetfillcolor{currentfill}%
\pgfsetfillopacity{0.800000}%
\pgfsetlinewidth{0.000000pt}%
\definecolor{currentstroke}{rgb}{0.000000,0.000000,0.000000}%
\pgfsetstrokecolor{currentstroke}%
\pgfsetstrokeopacity{0.800000}%
\pgfsetdash{}{0pt}%
\pgfpathmoveto{\pgfqpoint{3.512170in}{3.596168in}}%
\pgfpathcurveto{\pgfqpoint{3.516288in}{3.596168in}}{\pgfqpoint{3.520238in}{3.597805in}}{\pgfqpoint{3.523150in}{3.600717in}}%
\pgfpathcurveto{\pgfqpoint{3.526062in}{3.603629in}}{\pgfqpoint{3.527698in}{3.607579in}}{\pgfqpoint{3.527698in}{3.611697in}}%
\pgfpathcurveto{\pgfqpoint{3.527698in}{3.615815in}}{\pgfqpoint{3.526062in}{3.619765in}}{\pgfqpoint{3.523150in}{3.622677in}}%
\pgfpathcurveto{\pgfqpoint{3.520238in}{3.625589in}}{\pgfqpoint{3.516288in}{3.627225in}}{\pgfqpoint{3.512170in}{3.627225in}}%
\pgfpathcurveto{\pgfqpoint{3.508052in}{3.627225in}}{\pgfqpoint{3.504102in}{3.625589in}}{\pgfqpoint{3.501190in}{3.622677in}}%
\pgfpathcurveto{\pgfqpoint{3.498278in}{3.619765in}}{\pgfqpoint{3.496642in}{3.615815in}}{\pgfqpoint{3.496642in}{3.611697in}}%
\pgfpathcurveto{\pgfqpoint{3.496642in}{3.607579in}}{\pgfqpoint{3.498278in}{3.603629in}}{\pgfqpoint{3.501190in}{3.600717in}}%
\pgfpathcurveto{\pgfqpoint{3.504102in}{3.597805in}}{\pgfqpoint{3.508052in}{3.596168in}}{\pgfqpoint{3.512170in}{3.596168in}}%
\pgfpathclose%
\pgfusepath{fill}%
\end{pgfscope}%
\begin{pgfscope}%
\pgfpathrectangle{\pgfqpoint{0.887500in}{0.275000in}}{\pgfqpoint{4.225000in}{4.225000in}}%
\pgfusepath{clip}%
\pgfsetbuttcap%
\pgfsetroundjoin%
\definecolor{currentfill}{rgb}{0.000000,0.000000,0.000000}%
\pgfsetfillcolor{currentfill}%
\pgfsetfillopacity{0.800000}%
\pgfsetlinewidth{0.000000pt}%
\definecolor{currentstroke}{rgb}{0.000000,0.000000,0.000000}%
\pgfsetstrokecolor{currentstroke}%
\pgfsetstrokeopacity{0.800000}%
\pgfsetdash{}{0pt}%
\pgfpathmoveto{\pgfqpoint{3.227048in}{3.568340in}}%
\pgfpathcurveto{\pgfqpoint{3.231166in}{3.568340in}}{\pgfqpoint{3.235116in}{3.569976in}}{\pgfqpoint{3.238028in}{3.572888in}}%
\pgfpathcurveto{\pgfqpoint{3.240940in}{3.575800in}}{\pgfqpoint{3.242576in}{3.579750in}}{\pgfqpoint{3.242576in}{3.583868in}}%
\pgfpathcurveto{\pgfqpoint{3.242576in}{3.587986in}}{\pgfqpoint{3.240940in}{3.591936in}}{\pgfqpoint{3.238028in}{3.594848in}}%
\pgfpathcurveto{\pgfqpoint{3.235116in}{3.597760in}}{\pgfqpoint{3.231166in}{3.599396in}}{\pgfqpoint{3.227048in}{3.599396in}}%
\pgfpathcurveto{\pgfqpoint{3.222930in}{3.599396in}}{\pgfqpoint{3.218980in}{3.597760in}}{\pgfqpoint{3.216068in}{3.594848in}}%
\pgfpathcurveto{\pgfqpoint{3.213156in}{3.591936in}}{\pgfqpoint{3.211519in}{3.587986in}}{\pgfqpoint{3.211519in}{3.583868in}}%
\pgfpathcurveto{\pgfqpoint{3.211519in}{3.579750in}}{\pgfqpoint{3.213156in}{3.575800in}}{\pgfqpoint{3.216068in}{3.572888in}}%
\pgfpathcurveto{\pgfqpoint{3.218980in}{3.569976in}}{\pgfqpoint{3.222930in}{3.568340in}}{\pgfqpoint{3.227048in}{3.568340in}}%
\pgfpathclose%
\pgfusepath{fill}%
\end{pgfscope}%
\begin{pgfscope}%
\pgfpathrectangle{\pgfqpoint{0.887500in}{0.275000in}}{\pgfqpoint{4.225000in}{4.225000in}}%
\pgfusepath{clip}%
\pgfsetbuttcap%
\pgfsetroundjoin%
\definecolor{currentfill}{rgb}{0.000000,0.000000,0.000000}%
\pgfsetfillcolor{currentfill}%
\pgfsetfillopacity{0.800000}%
\pgfsetlinewidth{0.000000pt}%
\definecolor{currentstroke}{rgb}{0.000000,0.000000,0.000000}%
\pgfsetstrokecolor{currentstroke}%
\pgfsetstrokeopacity{0.800000}%
\pgfsetdash{}{0pt}%
\pgfpathmoveto{\pgfqpoint{2.277848in}{2.771681in}}%
\pgfpathcurveto{\pgfqpoint{2.281966in}{2.771681in}}{\pgfqpoint{2.285916in}{2.773317in}}{\pgfqpoint{2.288828in}{2.776229in}}%
\pgfpathcurveto{\pgfqpoint{2.291740in}{2.779141in}}{\pgfqpoint{2.293377in}{2.783091in}}{\pgfqpoint{2.293377in}{2.787209in}}%
\pgfpathcurveto{\pgfqpoint{2.293377in}{2.791328in}}{\pgfqpoint{2.291740in}{2.795278in}}{\pgfqpoint{2.288828in}{2.798190in}}%
\pgfpathcurveto{\pgfqpoint{2.285916in}{2.801102in}}{\pgfqpoint{2.281966in}{2.802738in}}{\pgfqpoint{2.277848in}{2.802738in}}%
\pgfpathcurveto{\pgfqpoint{2.273730in}{2.802738in}}{\pgfqpoint{2.269780in}{2.801102in}}{\pgfqpoint{2.266868in}{2.798190in}}%
\pgfpathcurveto{\pgfqpoint{2.263956in}{2.795278in}}{\pgfqpoint{2.262320in}{2.791328in}}{\pgfqpoint{2.262320in}{2.787209in}}%
\pgfpathcurveto{\pgfqpoint{2.262320in}{2.783091in}}{\pgfqpoint{2.263956in}{2.779141in}}{\pgfqpoint{2.266868in}{2.776229in}}%
\pgfpathcurveto{\pgfqpoint{2.269780in}{2.773317in}}{\pgfqpoint{2.273730in}{2.771681in}}{\pgfqpoint{2.277848in}{2.771681in}}%
\pgfpathclose%
\pgfusepath{fill}%
\end{pgfscope}%
\begin{pgfscope}%
\pgfpathrectangle{\pgfqpoint{0.887500in}{0.275000in}}{\pgfqpoint{4.225000in}{4.225000in}}%
\pgfusepath{clip}%
\pgfsetbuttcap%
\pgfsetroundjoin%
\definecolor{currentfill}{rgb}{0.000000,0.000000,0.000000}%
\pgfsetfillcolor{currentfill}%
\pgfsetfillopacity{0.800000}%
\pgfsetlinewidth{0.000000pt}%
\definecolor{currentstroke}{rgb}{0.000000,0.000000,0.000000}%
\pgfsetstrokecolor{currentstroke}%
\pgfsetstrokeopacity{0.800000}%
\pgfsetdash{}{0pt}%
\pgfpathmoveto{\pgfqpoint{1.818032in}{2.832082in}}%
\pgfpathcurveto{\pgfqpoint{1.822150in}{2.832082in}}{\pgfqpoint{1.826100in}{2.833719in}}{\pgfqpoint{1.829012in}{2.836631in}}%
\pgfpathcurveto{\pgfqpoint{1.831924in}{2.839542in}}{\pgfqpoint{1.833560in}{2.843493in}}{\pgfqpoint{1.833560in}{2.847611in}}%
\pgfpathcurveto{\pgfqpoint{1.833560in}{2.851729in}}{\pgfqpoint{1.831924in}{2.855679in}}{\pgfqpoint{1.829012in}{2.858591in}}%
\pgfpathcurveto{\pgfqpoint{1.826100in}{2.861503in}}{\pgfqpoint{1.822150in}{2.863139in}}{\pgfqpoint{1.818032in}{2.863139in}}%
\pgfpathcurveto{\pgfqpoint{1.813914in}{2.863139in}}{\pgfqpoint{1.809964in}{2.861503in}}{\pgfqpoint{1.807052in}{2.858591in}}%
\pgfpathcurveto{\pgfqpoint{1.804140in}{2.855679in}}{\pgfqpoint{1.802504in}{2.851729in}}{\pgfqpoint{1.802504in}{2.847611in}}%
\pgfpathcurveto{\pgfqpoint{1.802504in}{2.843493in}}{\pgfqpoint{1.804140in}{2.839542in}}{\pgfqpoint{1.807052in}{2.836631in}}%
\pgfpathcurveto{\pgfqpoint{1.809964in}{2.833719in}}{\pgfqpoint{1.813914in}{2.832082in}}{\pgfqpoint{1.818032in}{2.832082in}}%
\pgfpathclose%
\pgfusepath{fill}%
\end{pgfscope}%
\begin{pgfscope}%
\pgfpathrectangle{\pgfqpoint{0.887500in}{0.275000in}}{\pgfqpoint{4.225000in}{4.225000in}}%
\pgfusepath{clip}%
\pgfsetbuttcap%
\pgfsetroundjoin%
\definecolor{currentfill}{rgb}{0.000000,0.000000,0.000000}%
\pgfsetfillcolor{currentfill}%
\pgfsetfillopacity{0.800000}%
\pgfsetlinewidth{0.000000pt}%
\definecolor{currentstroke}{rgb}{0.000000,0.000000,0.000000}%
\pgfsetstrokecolor{currentstroke}%
\pgfsetstrokeopacity{0.800000}%
\pgfsetdash{}{0pt}%
\pgfpathmoveto{\pgfqpoint{2.673586in}{2.949591in}}%
\pgfpathcurveto{\pgfqpoint{2.677704in}{2.949591in}}{\pgfqpoint{2.681654in}{2.951227in}}{\pgfqpoint{2.684566in}{2.954139in}}%
\pgfpathcurveto{\pgfqpoint{2.687478in}{2.957051in}}{\pgfqpoint{2.689114in}{2.961001in}}{\pgfqpoint{2.689114in}{2.965119in}}%
\pgfpathcurveto{\pgfqpoint{2.689114in}{2.969237in}}{\pgfqpoint{2.687478in}{2.973187in}}{\pgfqpoint{2.684566in}{2.976099in}}%
\pgfpathcurveto{\pgfqpoint{2.681654in}{2.979011in}}{\pgfqpoint{2.677704in}{2.980647in}}{\pgfqpoint{2.673586in}{2.980647in}}%
\pgfpathcurveto{\pgfqpoint{2.669468in}{2.980647in}}{\pgfqpoint{2.665518in}{2.979011in}}{\pgfqpoint{2.662606in}{2.976099in}}%
\pgfpathcurveto{\pgfqpoint{2.659694in}{2.973187in}}{\pgfqpoint{2.658058in}{2.969237in}}{\pgfqpoint{2.658058in}{2.965119in}}%
\pgfpathcurveto{\pgfqpoint{2.658058in}{2.961001in}}{\pgfqpoint{2.659694in}{2.957051in}}{\pgfqpoint{2.662606in}{2.954139in}}%
\pgfpathcurveto{\pgfqpoint{2.665518in}{2.951227in}}{\pgfqpoint{2.669468in}{2.949591in}}{\pgfqpoint{2.673586in}{2.949591in}}%
\pgfpathclose%
\pgfusepath{fill}%
\end{pgfscope}%
\begin{pgfscope}%
\pgfpathrectangle{\pgfqpoint{0.887500in}{0.275000in}}{\pgfqpoint{4.225000in}{4.225000in}}%
\pgfusepath{clip}%
\pgfsetbuttcap%
\pgfsetroundjoin%
\definecolor{currentfill}{rgb}{0.000000,0.000000,0.000000}%
\pgfsetfillcolor{currentfill}%
\pgfsetfillopacity{0.800000}%
\pgfsetlinewidth{0.000000pt}%
\definecolor{currentstroke}{rgb}{0.000000,0.000000,0.000000}%
\pgfsetstrokecolor{currentstroke}%
\pgfsetstrokeopacity{0.800000}%
\pgfsetdash{}{0pt}%
\pgfpathmoveto{\pgfqpoint{3.687435in}{3.551226in}}%
\pgfpathcurveto{\pgfqpoint{3.691553in}{3.551226in}}{\pgfqpoint{3.695503in}{3.552862in}}{\pgfqpoint{3.698415in}{3.555774in}}%
\pgfpathcurveto{\pgfqpoint{3.701327in}{3.558686in}}{\pgfqpoint{3.702963in}{3.562636in}}{\pgfqpoint{3.702963in}{3.566754in}}%
\pgfpathcurveto{\pgfqpoint{3.702963in}{3.570873in}}{\pgfqpoint{3.701327in}{3.574823in}}{\pgfqpoint{3.698415in}{3.577735in}}%
\pgfpathcurveto{\pgfqpoint{3.695503in}{3.580646in}}{\pgfqpoint{3.691553in}{3.582283in}}{\pgfqpoint{3.687435in}{3.582283in}}%
\pgfpathcurveto{\pgfqpoint{3.683317in}{3.582283in}}{\pgfqpoint{3.679367in}{3.580646in}}{\pgfqpoint{3.676455in}{3.577735in}}%
\pgfpathcurveto{\pgfqpoint{3.673543in}{3.574823in}}{\pgfqpoint{3.671907in}{3.570873in}}{\pgfqpoint{3.671907in}{3.566754in}}%
\pgfpathcurveto{\pgfqpoint{3.671907in}{3.562636in}}{\pgfqpoint{3.673543in}{3.558686in}}{\pgfqpoint{3.676455in}{3.555774in}}%
\pgfpathcurveto{\pgfqpoint{3.679367in}{3.552862in}}{\pgfqpoint{3.683317in}{3.551226in}}{\pgfqpoint{3.687435in}{3.551226in}}%
\pgfpathclose%
\pgfusepath{fill}%
\end{pgfscope}%
\begin{pgfscope}%
\pgfpathrectangle{\pgfqpoint{0.887500in}{0.275000in}}{\pgfqpoint{4.225000in}{4.225000in}}%
\pgfusepath{clip}%
\pgfsetbuttcap%
\pgfsetroundjoin%
\definecolor{currentfill}{rgb}{0.000000,0.000000,0.000000}%
\pgfsetfillcolor{currentfill}%
\pgfsetfillopacity{0.800000}%
\pgfsetlinewidth{0.000000pt}%
\definecolor{currentstroke}{rgb}{0.000000,0.000000,0.000000}%
\pgfsetstrokecolor{currentstroke}%
\pgfsetstrokeopacity{0.800000}%
\pgfsetdash{}{0pt}%
\pgfpathmoveto{\pgfqpoint{1.531062in}{2.843365in}}%
\pgfpathcurveto{\pgfqpoint{1.535180in}{2.843365in}}{\pgfqpoint{1.539130in}{2.845002in}}{\pgfqpoint{1.542042in}{2.847914in}}%
\pgfpathcurveto{\pgfqpoint{1.544954in}{2.850826in}}{\pgfqpoint{1.546590in}{2.854776in}}{\pgfqpoint{1.546590in}{2.858894in}}%
\pgfpathcurveto{\pgfqpoint{1.546590in}{2.863012in}}{\pgfqpoint{1.544954in}{2.866962in}}{\pgfqpoint{1.542042in}{2.869874in}}%
\pgfpathcurveto{\pgfqpoint{1.539130in}{2.872786in}}{\pgfqpoint{1.535180in}{2.874422in}}{\pgfqpoint{1.531062in}{2.874422in}}%
\pgfpathcurveto{\pgfqpoint{1.526944in}{2.874422in}}{\pgfqpoint{1.522994in}{2.872786in}}{\pgfqpoint{1.520082in}{2.869874in}}%
\pgfpathcurveto{\pgfqpoint{1.517170in}{2.866962in}}{\pgfqpoint{1.515534in}{2.863012in}}{\pgfqpoint{1.515534in}{2.858894in}}%
\pgfpathcurveto{\pgfqpoint{1.515534in}{2.854776in}}{\pgfqpoint{1.517170in}{2.850826in}}{\pgfqpoint{1.520082in}{2.847914in}}%
\pgfpathcurveto{\pgfqpoint{1.522994in}{2.845002in}}{\pgfqpoint{1.526944in}{2.843365in}}{\pgfqpoint{1.531062in}{2.843365in}}%
\pgfpathclose%
\pgfusepath{fill}%
\end{pgfscope}%
\begin{pgfscope}%
\pgfpathrectangle{\pgfqpoint{0.887500in}{0.275000in}}{\pgfqpoint{4.225000in}{4.225000in}}%
\pgfusepath{clip}%
\pgfsetbuttcap%
\pgfsetroundjoin%
\definecolor{currentfill}{rgb}{0.000000,0.000000,0.000000}%
\pgfsetfillcolor{currentfill}%
\pgfsetfillopacity{0.800000}%
\pgfsetlinewidth{0.000000pt}%
\definecolor{currentstroke}{rgb}{0.000000,0.000000,0.000000}%
\pgfsetstrokecolor{currentstroke}%
\pgfsetstrokeopacity{0.800000}%
\pgfsetdash{}{0pt}%
\pgfpathmoveto{\pgfqpoint{1.991343in}{2.785447in}}%
\pgfpathcurveto{\pgfqpoint{1.995461in}{2.785447in}}{\pgfqpoint{1.999411in}{2.787084in}}{\pgfqpoint{2.002323in}{2.789995in}}%
\pgfpathcurveto{\pgfqpoint{2.005235in}{2.792907in}}{\pgfqpoint{2.006871in}{2.796857in}}{\pgfqpoint{2.006871in}{2.800976in}}%
\pgfpathcurveto{\pgfqpoint{2.006871in}{2.805094in}}{\pgfqpoint{2.005235in}{2.809044in}}{\pgfqpoint{2.002323in}{2.811956in}}%
\pgfpathcurveto{\pgfqpoint{1.999411in}{2.814868in}}{\pgfqpoint{1.995461in}{2.816504in}}{\pgfqpoint{1.991343in}{2.816504in}}%
\pgfpathcurveto{\pgfqpoint{1.987224in}{2.816504in}}{\pgfqpoint{1.983274in}{2.814868in}}{\pgfqpoint{1.980362in}{2.811956in}}%
\pgfpathcurveto{\pgfqpoint{1.977450in}{2.809044in}}{\pgfqpoint{1.975814in}{2.805094in}}{\pgfqpoint{1.975814in}{2.800976in}}%
\pgfpathcurveto{\pgfqpoint{1.975814in}{2.796857in}}{\pgfqpoint{1.977450in}{2.792907in}}{\pgfqpoint{1.980362in}{2.789995in}}%
\pgfpathcurveto{\pgfqpoint{1.983274in}{2.787084in}}{\pgfqpoint{1.987224in}{2.785447in}}{\pgfqpoint{1.991343in}{2.785447in}}%
\pgfpathclose%
\pgfusepath{fill}%
\end{pgfscope}%
\begin{pgfscope}%
\pgfpathrectangle{\pgfqpoint{0.887500in}{0.275000in}}{\pgfqpoint{4.225000in}{4.225000in}}%
\pgfusepath{clip}%
\pgfsetbuttcap%
\pgfsetroundjoin%
\definecolor{currentfill}{rgb}{0.000000,0.000000,0.000000}%
\pgfsetfillcolor{currentfill}%
\pgfsetfillopacity{0.800000}%
\pgfsetlinewidth{0.000000pt}%
\definecolor{currentstroke}{rgb}{0.000000,0.000000,0.000000}%
\pgfsetstrokecolor{currentstroke}%
\pgfsetstrokeopacity{0.800000}%
\pgfsetdash{}{0pt}%
\pgfpathmoveto{\pgfqpoint{3.862610in}{3.480443in}}%
\pgfpathcurveto{\pgfqpoint{3.866728in}{3.480443in}}{\pgfqpoint{3.870678in}{3.482079in}}{\pgfqpoint{3.873590in}{3.484991in}}%
\pgfpathcurveto{\pgfqpoint{3.876502in}{3.487903in}}{\pgfqpoint{3.878138in}{3.491853in}}{\pgfqpoint{3.878138in}{3.495971in}}%
\pgfpathcurveto{\pgfqpoint{3.878138in}{3.500089in}}{\pgfqpoint{3.876502in}{3.504039in}}{\pgfqpoint{3.873590in}{3.506951in}}%
\pgfpathcurveto{\pgfqpoint{3.870678in}{3.509863in}}{\pgfqpoint{3.866728in}{3.511499in}}{\pgfqpoint{3.862610in}{3.511499in}}%
\pgfpathcurveto{\pgfqpoint{3.858491in}{3.511499in}}{\pgfqpoint{3.854541in}{3.509863in}}{\pgfqpoint{3.851629in}{3.506951in}}%
\pgfpathcurveto{\pgfqpoint{3.848717in}{3.504039in}}{\pgfqpoint{3.847081in}{3.500089in}}{\pgfqpoint{3.847081in}{3.495971in}}%
\pgfpathcurveto{\pgfqpoint{3.847081in}{3.491853in}}{\pgfqpoint{3.848717in}{3.487903in}}{\pgfqpoint{3.851629in}{3.484991in}}%
\pgfpathcurveto{\pgfqpoint{3.854541in}{3.482079in}}{\pgfqpoint{3.858491in}{3.480443in}}{\pgfqpoint{3.862610in}{3.480443in}}%
\pgfpathclose%
\pgfusepath{fill}%
\end{pgfscope}%
\begin{pgfscope}%
\pgfpathrectangle{\pgfqpoint{0.887500in}{0.275000in}}{\pgfqpoint{4.225000in}{4.225000in}}%
\pgfusepath{clip}%
\pgfsetbuttcap%
\pgfsetroundjoin%
\definecolor{currentfill}{rgb}{0.000000,0.000000,0.000000}%
\pgfsetfillcolor{currentfill}%
\pgfsetfillopacity{0.800000}%
\pgfsetlinewidth{0.000000pt}%
\definecolor{currentstroke}{rgb}{0.000000,0.000000,0.000000}%
\pgfsetstrokecolor{currentstroke}%
\pgfsetstrokeopacity{0.800000}%
\pgfsetdash{}{0pt}%
\pgfpathmoveto{\pgfqpoint{2.830837in}{3.471488in}}%
\pgfpathcurveto{\pgfqpoint{2.834955in}{3.471488in}}{\pgfqpoint{2.838905in}{3.473124in}}{\pgfqpoint{2.841817in}{3.476036in}}%
\pgfpathcurveto{\pgfqpoint{2.844729in}{3.478948in}}{\pgfqpoint{2.846365in}{3.482898in}}{\pgfqpoint{2.846365in}{3.487017in}}%
\pgfpathcurveto{\pgfqpoint{2.846365in}{3.491135in}}{\pgfqpoint{2.844729in}{3.495085in}}{\pgfqpoint{2.841817in}{3.497997in}}%
\pgfpathcurveto{\pgfqpoint{2.838905in}{3.500909in}}{\pgfqpoint{2.834955in}{3.502545in}}{\pgfqpoint{2.830837in}{3.502545in}}%
\pgfpathcurveto{\pgfqpoint{2.826719in}{3.502545in}}{\pgfqpoint{2.822769in}{3.500909in}}{\pgfqpoint{2.819857in}{3.497997in}}%
\pgfpathcurveto{\pgfqpoint{2.816945in}{3.495085in}}{\pgfqpoint{2.815309in}{3.491135in}}{\pgfqpoint{2.815309in}{3.487017in}}%
\pgfpathcurveto{\pgfqpoint{2.815309in}{3.482898in}}{\pgfqpoint{2.816945in}{3.478948in}}{\pgfqpoint{2.819857in}{3.476036in}}%
\pgfpathcurveto{\pgfqpoint{2.822769in}{3.473124in}}{\pgfqpoint{2.826719in}{3.471488in}}{\pgfqpoint{2.830837in}{3.471488in}}%
\pgfpathclose%
\pgfusepath{fill}%
\end{pgfscope}%
\begin{pgfscope}%
\pgfpathrectangle{\pgfqpoint{0.887500in}{0.275000in}}{\pgfqpoint{4.225000in}{4.225000in}}%
\pgfusepath{clip}%
\pgfsetbuttcap%
\pgfsetroundjoin%
\definecolor{currentfill}{rgb}{0.000000,0.000000,0.000000}%
\pgfsetfillcolor{currentfill}%
\pgfsetfillopacity{0.800000}%
\pgfsetlinewidth{0.000000pt}%
\definecolor{currentstroke}{rgb}{0.000000,0.000000,0.000000}%
\pgfsetstrokecolor{currentstroke}%
\pgfsetstrokeopacity{0.800000}%
\pgfsetdash{}{0pt}%
\pgfpathmoveto{\pgfqpoint{4.037573in}{3.392722in}}%
\pgfpathcurveto{\pgfqpoint{4.041692in}{3.392722in}}{\pgfqpoint{4.045642in}{3.394358in}}{\pgfqpoint{4.048554in}{3.397270in}}%
\pgfpathcurveto{\pgfqpoint{4.051466in}{3.400182in}}{\pgfqpoint{4.053102in}{3.404132in}}{\pgfqpoint{4.053102in}{3.408250in}}%
\pgfpathcurveto{\pgfqpoint{4.053102in}{3.412368in}}{\pgfqpoint{4.051466in}{3.416318in}}{\pgfqpoint{4.048554in}{3.419230in}}%
\pgfpathcurveto{\pgfqpoint{4.045642in}{3.422142in}}{\pgfqpoint{4.041692in}{3.423778in}}{\pgfqpoint{4.037573in}{3.423778in}}%
\pgfpathcurveto{\pgfqpoint{4.033455in}{3.423778in}}{\pgfqpoint{4.029505in}{3.422142in}}{\pgfqpoint{4.026593in}{3.419230in}}%
\pgfpathcurveto{\pgfqpoint{4.023681in}{3.416318in}}{\pgfqpoint{4.022045in}{3.412368in}}{\pgfqpoint{4.022045in}{3.408250in}}%
\pgfpathcurveto{\pgfqpoint{4.022045in}{3.404132in}}{\pgfqpoint{4.023681in}{3.400182in}}{\pgfqpoint{4.026593in}{3.397270in}}%
\pgfpathcurveto{\pgfqpoint{4.029505in}{3.394358in}}{\pgfqpoint{4.033455in}{3.392722in}}{\pgfqpoint{4.037573in}{3.392722in}}%
\pgfpathclose%
\pgfusepath{fill}%
\end{pgfscope}%
\begin{pgfscope}%
\pgfpathrectangle{\pgfqpoint{0.887500in}{0.275000in}}{\pgfqpoint{4.225000in}{4.225000in}}%
\pgfusepath{clip}%
\pgfsetbuttcap%
\pgfsetroundjoin%
\definecolor{currentfill}{rgb}{0.000000,0.000000,0.000000}%
\pgfsetfillcolor{currentfill}%
\pgfsetfillopacity{0.800000}%
\pgfsetlinewidth{0.000000pt}%
\definecolor{currentstroke}{rgb}{0.000000,0.000000,0.000000}%
\pgfsetstrokecolor{currentstroke}%
\pgfsetstrokeopacity{0.800000}%
\pgfsetdash{}{0pt}%
\pgfpathmoveto{\pgfqpoint{3.402209in}{3.547979in}}%
\pgfpathcurveto{\pgfqpoint{3.406327in}{3.547979in}}{\pgfqpoint{3.410277in}{3.549616in}}{\pgfqpoint{3.413189in}{3.552528in}}%
\pgfpathcurveto{\pgfqpoint{3.416101in}{3.555440in}}{\pgfqpoint{3.417737in}{3.559390in}}{\pgfqpoint{3.417737in}{3.563508in}}%
\pgfpathcurveto{\pgfqpoint{3.417737in}{3.567626in}}{\pgfqpoint{3.416101in}{3.571576in}}{\pgfqpoint{3.413189in}{3.574488in}}%
\pgfpathcurveto{\pgfqpoint{3.410277in}{3.577400in}}{\pgfqpoint{3.406327in}{3.579036in}}{\pgfqpoint{3.402209in}{3.579036in}}%
\pgfpathcurveto{\pgfqpoint{3.398090in}{3.579036in}}{\pgfqpoint{3.394140in}{3.577400in}}{\pgfqpoint{3.391228in}{3.574488in}}%
\pgfpathcurveto{\pgfqpoint{3.388316in}{3.571576in}}{\pgfqpoint{3.386680in}{3.567626in}}{\pgfqpoint{3.386680in}{3.563508in}}%
\pgfpathcurveto{\pgfqpoint{3.386680in}{3.559390in}}{\pgfqpoint{3.388316in}{3.555440in}}{\pgfqpoint{3.391228in}{3.552528in}}%
\pgfpathcurveto{\pgfqpoint{3.394140in}{3.549616in}}{\pgfqpoint{3.398090in}{3.547979in}}{\pgfqpoint{3.402209in}{3.547979in}}%
\pgfpathclose%
\pgfusepath{fill}%
\end{pgfscope}%
\begin{pgfscope}%
\pgfpathrectangle{\pgfqpoint{0.887500in}{0.275000in}}{\pgfqpoint{4.225000in}{4.225000in}}%
\pgfusepath{clip}%
\pgfsetbuttcap%
\pgfsetroundjoin%
\definecolor{currentfill}{rgb}{0.000000,0.000000,0.000000}%
\pgfsetfillcolor{currentfill}%
\pgfsetfillopacity{0.800000}%
\pgfsetlinewidth{0.000000pt}%
\definecolor{currentstroke}{rgb}{0.000000,0.000000,0.000000}%
\pgfsetstrokecolor{currentstroke}%
\pgfsetstrokeopacity{0.800000}%
\pgfsetdash{}{0pt}%
\pgfpathmoveto{\pgfqpoint{2.562176in}{2.878330in}}%
\pgfpathcurveto{\pgfqpoint{2.566295in}{2.878330in}}{\pgfqpoint{2.570245in}{2.879966in}}{\pgfqpoint{2.573157in}{2.882878in}}%
\pgfpathcurveto{\pgfqpoint{2.576069in}{2.885790in}}{\pgfqpoint{2.577705in}{2.889740in}}{\pgfqpoint{2.577705in}{2.893858in}}%
\pgfpathcurveto{\pgfqpoint{2.577705in}{2.897976in}}{\pgfqpoint{2.576069in}{2.901927in}}{\pgfqpoint{2.573157in}{2.904838in}}%
\pgfpathcurveto{\pgfqpoint{2.570245in}{2.907750in}}{\pgfqpoint{2.566295in}{2.909387in}}{\pgfqpoint{2.562176in}{2.909387in}}%
\pgfpathcurveto{\pgfqpoint{2.558058in}{2.909387in}}{\pgfqpoint{2.554108in}{2.907750in}}{\pgfqpoint{2.551196in}{2.904838in}}%
\pgfpathcurveto{\pgfqpoint{2.548284in}{2.901927in}}{\pgfqpoint{2.546648in}{2.897976in}}{\pgfqpoint{2.546648in}{2.893858in}}%
\pgfpathcurveto{\pgfqpoint{2.546648in}{2.889740in}}{\pgfqpoint{2.548284in}{2.885790in}}{\pgfqpoint{2.551196in}{2.882878in}}%
\pgfpathcurveto{\pgfqpoint{2.554108in}{2.879966in}}{\pgfqpoint{2.558058in}{2.878330in}}{\pgfqpoint{2.562176in}{2.878330in}}%
\pgfpathclose%
\pgfusepath{fill}%
\end{pgfscope}%
\begin{pgfscope}%
\pgfpathrectangle{\pgfqpoint{0.887500in}{0.275000in}}{\pgfqpoint{4.225000in}{4.225000in}}%
\pgfusepath{clip}%
\pgfsetbuttcap%
\pgfsetroundjoin%
\definecolor{currentfill}{rgb}{0.000000,0.000000,0.000000}%
\pgfsetfillcolor{currentfill}%
\pgfsetfillopacity{0.800000}%
\pgfsetlinewidth{0.000000pt}%
\definecolor{currentstroke}{rgb}{0.000000,0.000000,0.000000}%
\pgfsetstrokecolor{currentstroke}%
\pgfsetstrokeopacity{0.800000}%
\pgfsetdash{}{0pt}%
\pgfpathmoveto{\pgfqpoint{2.451123in}{2.762645in}}%
\pgfpathcurveto{\pgfqpoint{2.455241in}{2.762645in}}{\pgfqpoint{2.459191in}{2.764282in}}{\pgfqpoint{2.462103in}{2.767193in}}%
\pgfpathcurveto{\pgfqpoint{2.465015in}{2.770105in}}{\pgfqpoint{2.466651in}{2.774055in}}{\pgfqpoint{2.466651in}{2.778174in}}%
\pgfpathcurveto{\pgfqpoint{2.466651in}{2.782292in}}{\pgfqpoint{2.465015in}{2.786242in}}{\pgfqpoint{2.462103in}{2.789154in}}%
\pgfpathcurveto{\pgfqpoint{2.459191in}{2.792066in}}{\pgfqpoint{2.455241in}{2.793702in}}{\pgfqpoint{2.451123in}{2.793702in}}%
\pgfpathcurveto{\pgfqpoint{2.447005in}{2.793702in}}{\pgfqpoint{2.443055in}{2.792066in}}{\pgfqpoint{2.440143in}{2.789154in}}%
\pgfpathcurveto{\pgfqpoint{2.437231in}{2.786242in}}{\pgfqpoint{2.435595in}{2.782292in}}{\pgfqpoint{2.435595in}{2.778174in}}%
\pgfpathcurveto{\pgfqpoint{2.435595in}{2.774055in}}{\pgfqpoint{2.437231in}{2.770105in}}{\pgfqpoint{2.440143in}{2.767193in}}%
\pgfpathcurveto{\pgfqpoint{2.443055in}{2.764282in}}{\pgfqpoint{2.447005in}{2.762645in}}{\pgfqpoint{2.451123in}{2.762645in}}%
\pgfpathclose%
\pgfusepath{fill}%
\end{pgfscope}%
\begin{pgfscope}%
\pgfpathrectangle{\pgfqpoint{0.887500in}{0.275000in}}{\pgfqpoint{4.225000in}{4.225000in}}%
\pgfusepath{clip}%
\pgfsetbuttcap%
\pgfsetroundjoin%
\definecolor{currentfill}{rgb}{0.000000,0.000000,0.000000}%
\pgfsetfillcolor{currentfill}%
\pgfsetfillopacity{0.800000}%
\pgfsetlinewidth{0.000000pt}%
\definecolor{currentstroke}{rgb}{0.000000,0.000000,0.000000}%
\pgfsetstrokecolor{currentstroke}%
\pgfsetstrokeopacity{0.800000}%
\pgfsetdash{}{0pt}%
\pgfpathmoveto{\pgfqpoint{3.116534in}{3.536396in}}%
\pgfpathcurveto{\pgfqpoint{3.120652in}{3.536396in}}{\pgfqpoint{3.124602in}{3.538032in}}{\pgfqpoint{3.127514in}{3.540944in}}%
\pgfpathcurveto{\pgfqpoint{3.130426in}{3.543856in}}{\pgfqpoint{3.132062in}{3.547806in}}{\pgfqpoint{3.132062in}{3.551924in}}%
\pgfpathcurveto{\pgfqpoint{3.132062in}{3.556042in}}{\pgfqpoint{3.130426in}{3.559992in}}{\pgfqpoint{3.127514in}{3.562904in}}%
\pgfpathcurveto{\pgfqpoint{3.124602in}{3.565816in}}{\pgfqpoint{3.120652in}{3.567452in}}{\pgfqpoint{3.116534in}{3.567452in}}%
\pgfpathcurveto{\pgfqpoint{3.112416in}{3.567452in}}{\pgfqpoint{3.108466in}{3.565816in}}{\pgfqpoint{3.105554in}{3.562904in}}%
\pgfpathcurveto{\pgfqpoint{3.102642in}{3.559992in}}{\pgfqpoint{3.101006in}{3.556042in}}{\pgfqpoint{3.101006in}{3.551924in}}%
\pgfpathcurveto{\pgfqpoint{3.101006in}{3.547806in}}{\pgfqpoint{3.102642in}{3.543856in}}{\pgfqpoint{3.105554in}{3.540944in}}%
\pgfpathcurveto{\pgfqpoint{3.108466in}{3.538032in}}{\pgfqpoint{3.112416in}{3.536396in}}{\pgfqpoint{3.116534in}{3.536396in}}%
\pgfpathclose%
\pgfusepath{fill}%
\end{pgfscope}%
\begin{pgfscope}%
\pgfpathrectangle{\pgfqpoint{0.887500in}{0.275000in}}{\pgfqpoint{4.225000in}{4.225000in}}%
\pgfusepath{clip}%
\pgfsetbuttcap%
\pgfsetroundjoin%
\definecolor{currentfill}{rgb}{0.000000,0.000000,0.000000}%
\pgfsetfillcolor{currentfill}%
\pgfsetfillopacity{0.800000}%
\pgfsetlinewidth{0.000000pt}%
\definecolor{currentstroke}{rgb}{0.000000,0.000000,0.000000}%
\pgfsetstrokecolor{currentstroke}%
\pgfsetstrokeopacity{0.800000}%
\pgfsetdash{}{0pt}%
\pgfpathmoveto{\pgfqpoint{2.165092in}{2.737341in}}%
\pgfpathcurveto{\pgfqpoint{2.169210in}{2.737341in}}{\pgfqpoint{2.173160in}{2.738977in}}{\pgfqpoint{2.176072in}{2.741889in}}%
\pgfpathcurveto{\pgfqpoint{2.178984in}{2.744801in}}{\pgfqpoint{2.180620in}{2.748751in}}{\pgfqpoint{2.180620in}{2.752869in}}%
\pgfpathcurveto{\pgfqpoint{2.180620in}{2.756988in}}{\pgfqpoint{2.178984in}{2.760938in}}{\pgfqpoint{2.176072in}{2.763850in}}%
\pgfpathcurveto{\pgfqpoint{2.173160in}{2.766762in}}{\pgfqpoint{2.169210in}{2.768398in}}{\pgfqpoint{2.165092in}{2.768398in}}%
\pgfpathcurveto{\pgfqpoint{2.160974in}{2.768398in}}{\pgfqpoint{2.157024in}{2.766762in}}{\pgfqpoint{2.154112in}{2.763850in}}%
\pgfpathcurveto{\pgfqpoint{2.151200in}{2.760938in}}{\pgfqpoint{2.149564in}{2.756988in}}{\pgfqpoint{2.149564in}{2.752869in}}%
\pgfpathcurveto{\pgfqpoint{2.149564in}{2.748751in}}{\pgfqpoint{2.151200in}{2.744801in}}{\pgfqpoint{2.154112in}{2.741889in}}%
\pgfpathcurveto{\pgfqpoint{2.157024in}{2.738977in}}{\pgfqpoint{2.160974in}{2.737341in}}{\pgfqpoint{2.165092in}{2.737341in}}%
\pgfpathclose%
\pgfusepath{fill}%
\end{pgfscope}%
\begin{pgfscope}%
\pgfpathrectangle{\pgfqpoint{0.887500in}{0.275000in}}{\pgfqpoint{4.225000in}{4.225000in}}%
\pgfusepath{clip}%
\pgfsetbuttcap%
\pgfsetroundjoin%
\definecolor{currentfill}{rgb}{0.000000,0.000000,0.000000}%
\pgfsetfillcolor{currentfill}%
\pgfsetfillopacity{0.800000}%
\pgfsetlinewidth{0.000000pt}%
\definecolor{currentstroke}{rgb}{0.000000,0.000000,0.000000}%
\pgfsetstrokecolor{currentstroke}%
\pgfsetstrokeopacity{0.800000}%
\pgfsetdash{}{0pt}%
\pgfpathmoveto{\pgfqpoint{1.704160in}{2.798821in}}%
\pgfpathcurveto{\pgfqpoint{1.708278in}{2.798821in}}{\pgfqpoint{1.712228in}{2.800457in}}{\pgfqpoint{1.715140in}{2.803369in}}%
\pgfpathcurveto{\pgfqpoint{1.718052in}{2.806281in}}{\pgfqpoint{1.719689in}{2.810231in}}{\pgfqpoint{1.719689in}{2.814349in}}%
\pgfpathcurveto{\pgfqpoint{1.719689in}{2.818467in}}{\pgfqpoint{1.718052in}{2.822417in}}{\pgfqpoint{1.715140in}{2.825329in}}%
\pgfpathcurveto{\pgfqpoint{1.712228in}{2.828241in}}{\pgfqpoint{1.708278in}{2.829877in}}{\pgfqpoint{1.704160in}{2.829877in}}%
\pgfpathcurveto{\pgfqpoint{1.700042in}{2.829877in}}{\pgfqpoint{1.696092in}{2.828241in}}{\pgfqpoint{1.693180in}{2.825329in}}%
\pgfpathcurveto{\pgfqpoint{1.690268in}{2.822417in}}{\pgfqpoint{1.688632in}{2.818467in}}{\pgfqpoint{1.688632in}{2.814349in}}%
\pgfpathcurveto{\pgfqpoint{1.688632in}{2.810231in}}{\pgfqpoint{1.690268in}{2.806281in}}{\pgfqpoint{1.693180in}{2.803369in}}%
\pgfpathcurveto{\pgfqpoint{1.696092in}{2.800457in}}{\pgfqpoint{1.700042in}{2.798821in}}{\pgfqpoint{1.704160in}{2.798821in}}%
\pgfpathclose%
\pgfusepath{fill}%
\end{pgfscope}%
\begin{pgfscope}%
\pgfpathrectangle{\pgfqpoint{0.887500in}{0.275000in}}{\pgfqpoint{4.225000in}{4.225000in}}%
\pgfusepath{clip}%
\pgfsetbuttcap%
\pgfsetroundjoin%
\definecolor{currentfill}{rgb}{0.000000,0.000000,0.000000}%
\pgfsetfillcolor{currentfill}%
\pgfsetfillopacity{0.800000}%
\pgfsetlinewidth{0.000000pt}%
\definecolor{currentstroke}{rgb}{0.000000,0.000000,0.000000}%
\pgfsetstrokecolor{currentstroke}%
\pgfsetstrokeopacity{0.800000}%
\pgfsetdash{}{0pt}%
\pgfpathmoveto{\pgfqpoint{2.339268in}{2.687509in}}%
\pgfpathcurveto{\pgfqpoint{2.343386in}{2.687509in}}{\pgfqpoint{2.347336in}{2.689145in}}{\pgfqpoint{2.350248in}{2.692057in}}%
\pgfpathcurveto{\pgfqpoint{2.353160in}{2.694969in}}{\pgfqpoint{2.354797in}{2.698919in}}{\pgfqpoint{2.354797in}{2.703037in}}%
\pgfpathcurveto{\pgfqpoint{2.354797in}{2.707155in}}{\pgfqpoint{2.353160in}{2.711105in}}{\pgfqpoint{2.350248in}{2.714017in}}%
\pgfpathcurveto{\pgfqpoint{2.347336in}{2.716929in}}{\pgfqpoint{2.343386in}{2.718565in}}{\pgfqpoint{2.339268in}{2.718565in}}%
\pgfpathcurveto{\pgfqpoint{2.335150in}{2.718565in}}{\pgfqpoint{2.331200in}{2.716929in}}{\pgfqpoint{2.328288in}{2.714017in}}%
\pgfpathcurveto{\pgfqpoint{2.325376in}{2.711105in}}{\pgfqpoint{2.323740in}{2.707155in}}{\pgfqpoint{2.323740in}{2.703037in}}%
\pgfpathcurveto{\pgfqpoint{2.323740in}{2.698919in}}{\pgfqpoint{2.325376in}{2.694969in}}{\pgfqpoint{2.328288in}{2.692057in}}%
\pgfpathcurveto{\pgfqpoint{2.331200in}{2.689145in}}{\pgfqpoint{2.335150in}{2.687509in}}{\pgfqpoint{2.339268in}{2.687509in}}%
\pgfpathclose%
\pgfusepath{fill}%
\end{pgfscope}%
\begin{pgfscope}%
\pgfpathrectangle{\pgfqpoint{0.887500in}{0.275000in}}{\pgfqpoint{4.225000in}{4.225000in}}%
\pgfusepath{clip}%
\pgfsetbuttcap%
\pgfsetroundjoin%
\definecolor{currentfill}{rgb}{0.000000,0.000000,0.000000}%
\pgfsetfillcolor{currentfill}%
\pgfsetfillopacity{0.800000}%
\pgfsetlinewidth{0.000000pt}%
\definecolor{currentstroke}{rgb}{0.000000,0.000000,0.000000}%
\pgfsetstrokecolor{currentstroke}%
\pgfsetstrokeopacity{0.800000}%
\pgfsetdash{}{0pt}%
\pgfpathmoveto{\pgfqpoint{1.416375in}{2.810178in}}%
\pgfpathcurveto{\pgfqpoint{1.420493in}{2.810178in}}{\pgfqpoint{1.424443in}{2.811814in}}{\pgfqpoint{1.427355in}{2.814726in}}%
\pgfpathcurveto{\pgfqpoint{1.430267in}{2.817638in}}{\pgfqpoint{1.431903in}{2.821588in}}{\pgfqpoint{1.431903in}{2.825706in}}%
\pgfpathcurveto{\pgfqpoint{1.431903in}{2.829824in}}{\pgfqpoint{1.430267in}{2.833774in}}{\pgfqpoint{1.427355in}{2.836686in}}%
\pgfpathcurveto{\pgfqpoint{1.424443in}{2.839598in}}{\pgfqpoint{1.420493in}{2.841234in}}{\pgfqpoint{1.416375in}{2.841234in}}%
\pgfpathcurveto{\pgfqpoint{1.412257in}{2.841234in}}{\pgfqpoint{1.408307in}{2.839598in}}{\pgfqpoint{1.405395in}{2.836686in}}%
\pgfpathcurveto{\pgfqpoint{1.402483in}{2.833774in}}{\pgfqpoint{1.400847in}{2.829824in}}{\pgfqpoint{1.400847in}{2.825706in}}%
\pgfpathcurveto{\pgfqpoint{1.400847in}{2.821588in}}{\pgfqpoint{1.402483in}{2.817638in}}{\pgfqpoint{1.405395in}{2.814726in}}%
\pgfpathcurveto{\pgfqpoint{1.408307in}{2.811814in}}{\pgfqpoint{1.412257in}{2.810178in}}{\pgfqpoint{1.416375in}{2.810178in}}%
\pgfpathclose%
\pgfusepath{fill}%
\end{pgfscope}%
\begin{pgfscope}%
\pgfpathrectangle{\pgfqpoint{0.887500in}{0.275000in}}{\pgfqpoint{4.225000in}{4.225000in}}%
\pgfusepath{clip}%
\pgfsetbuttcap%
\pgfsetroundjoin%
\definecolor{currentfill}{rgb}{0.000000,0.000000,0.000000}%
\pgfsetfillcolor{currentfill}%
\pgfsetfillopacity{0.800000}%
\pgfsetlinewidth{0.000000pt}%
\definecolor{currentstroke}{rgb}{0.000000,0.000000,0.000000}%
\pgfsetstrokecolor{currentstroke}%
\pgfsetstrokeopacity{0.800000}%
\pgfsetdash{}{0pt}%
\pgfpathmoveto{\pgfqpoint{1.877753in}{2.751865in}}%
\pgfpathcurveto{\pgfqpoint{1.881871in}{2.751865in}}{\pgfqpoint{1.885821in}{2.753501in}}{\pgfqpoint{1.888733in}{2.756413in}}%
\pgfpathcurveto{\pgfqpoint{1.891645in}{2.759325in}}{\pgfqpoint{1.893282in}{2.763275in}}{\pgfqpoint{1.893282in}{2.767393in}}%
\pgfpathcurveto{\pgfqpoint{1.893282in}{2.771512in}}{\pgfqpoint{1.891645in}{2.775462in}}{\pgfqpoint{1.888733in}{2.778374in}}%
\pgfpathcurveto{\pgfqpoint{1.885821in}{2.781286in}}{\pgfqpoint{1.881871in}{2.782922in}}{\pgfqpoint{1.877753in}{2.782922in}}%
\pgfpathcurveto{\pgfqpoint{1.873635in}{2.782922in}}{\pgfqpoint{1.869685in}{2.781286in}}{\pgfqpoint{1.866773in}{2.778374in}}%
\pgfpathcurveto{\pgfqpoint{1.863861in}{2.775462in}}{\pgfqpoint{1.862225in}{2.771512in}}{\pgfqpoint{1.862225in}{2.767393in}}%
\pgfpathcurveto{\pgfqpoint{1.862225in}{2.763275in}}{\pgfqpoint{1.863861in}{2.759325in}}{\pgfqpoint{1.866773in}{2.756413in}}%
\pgfpathcurveto{\pgfqpoint{1.869685in}{2.753501in}}{\pgfqpoint{1.873635in}{2.751865in}}{\pgfqpoint{1.877753in}{2.751865in}}%
\pgfpathclose%
\pgfusepath{fill}%
\end{pgfscope}%
\begin{pgfscope}%
\pgfpathrectangle{\pgfqpoint{0.887500in}{0.275000in}}{\pgfqpoint{4.225000in}{4.225000in}}%
\pgfusepath{clip}%
\pgfsetbuttcap%
\pgfsetroundjoin%
\definecolor{currentfill}{rgb}{0.000000,0.000000,0.000000}%
\pgfsetfillcolor{currentfill}%
\pgfsetfillopacity{0.800000}%
\pgfsetlinewidth{0.000000pt}%
\definecolor{currentstroke}{rgb}{0.000000,0.000000,0.000000}%
\pgfsetstrokecolor{currentstroke}%
\pgfsetstrokeopacity{0.800000}%
\pgfsetdash{}{0pt}%
\pgfpathmoveto{\pgfqpoint{3.753229in}{3.436968in}}%
\pgfpathcurveto{\pgfqpoint{3.757347in}{3.436968in}}{\pgfqpoint{3.761297in}{3.438604in}}{\pgfqpoint{3.764209in}{3.441516in}}%
\pgfpathcurveto{\pgfqpoint{3.767121in}{3.444428in}}{\pgfqpoint{3.768757in}{3.448378in}}{\pgfqpoint{3.768757in}{3.452497in}}%
\pgfpathcurveto{\pgfqpoint{3.768757in}{3.456615in}}{\pgfqpoint{3.767121in}{3.460565in}}{\pgfqpoint{3.764209in}{3.463477in}}%
\pgfpathcurveto{\pgfqpoint{3.761297in}{3.466389in}}{\pgfqpoint{3.757347in}{3.468025in}}{\pgfqpoint{3.753229in}{3.468025in}}%
\pgfpathcurveto{\pgfqpoint{3.749110in}{3.468025in}}{\pgfqpoint{3.745160in}{3.466389in}}{\pgfqpoint{3.742249in}{3.463477in}}%
\pgfpathcurveto{\pgfqpoint{3.739337in}{3.460565in}}{\pgfqpoint{3.737700in}{3.456615in}}{\pgfqpoint{3.737700in}{3.452497in}}%
\pgfpathcurveto{\pgfqpoint{3.737700in}{3.448378in}}{\pgfqpoint{3.739337in}{3.444428in}}{\pgfqpoint{3.742249in}{3.441516in}}%
\pgfpathcurveto{\pgfqpoint{3.745160in}{3.438604in}}{\pgfqpoint{3.749110in}{3.436968in}}{\pgfqpoint{3.753229in}{3.436968in}}%
\pgfpathclose%
\pgfusepath{fill}%
\end{pgfscope}%
\begin{pgfscope}%
\pgfpathrectangle{\pgfqpoint{0.887500in}{0.275000in}}{\pgfqpoint{4.225000in}{4.225000in}}%
\pgfusepath{clip}%
\pgfsetbuttcap%
\pgfsetroundjoin%
\definecolor{currentfill}{rgb}{0.000000,0.000000,0.000000}%
\pgfsetfillcolor{currentfill}%
\pgfsetfillopacity{0.800000}%
\pgfsetlinewidth{0.000000pt}%
\definecolor{currentstroke}{rgb}{0.000000,0.000000,0.000000}%
\pgfsetstrokecolor{currentstroke}%
\pgfsetstrokeopacity{0.800000}%
\pgfsetdash{}{0pt}%
\pgfpathmoveto{\pgfqpoint{3.928616in}{3.352518in}}%
\pgfpathcurveto{\pgfqpoint{3.932734in}{3.352518in}}{\pgfqpoint{3.936684in}{3.354154in}}{\pgfqpoint{3.939596in}{3.357066in}}%
\pgfpathcurveto{\pgfqpoint{3.942508in}{3.359978in}}{\pgfqpoint{3.944144in}{3.363928in}}{\pgfqpoint{3.944144in}{3.368046in}}%
\pgfpathcurveto{\pgfqpoint{3.944144in}{3.372165in}}{\pgfqpoint{3.942508in}{3.376115in}}{\pgfqpoint{3.939596in}{3.379027in}}%
\pgfpathcurveto{\pgfqpoint{3.936684in}{3.381939in}}{\pgfqpoint{3.932734in}{3.383575in}}{\pgfqpoint{3.928616in}{3.383575in}}%
\pgfpathcurveto{\pgfqpoint{3.924498in}{3.383575in}}{\pgfqpoint{3.920548in}{3.381939in}}{\pgfqpoint{3.917636in}{3.379027in}}%
\pgfpathcurveto{\pgfqpoint{3.914724in}{3.376115in}}{\pgfqpoint{3.913088in}{3.372165in}}{\pgfqpoint{3.913088in}{3.368046in}}%
\pgfpathcurveto{\pgfqpoint{3.913088in}{3.363928in}}{\pgfqpoint{3.914724in}{3.359978in}}{\pgfqpoint{3.917636in}{3.357066in}}%
\pgfpathcurveto{\pgfqpoint{3.920548in}{3.354154in}}{\pgfqpoint{3.924498in}{3.352518in}}{\pgfqpoint{3.928616in}{3.352518in}}%
\pgfpathclose%
\pgfusepath{fill}%
\end{pgfscope}%
\begin{pgfscope}%
\pgfpathrectangle{\pgfqpoint{0.887500in}{0.275000in}}{\pgfqpoint{4.225000in}{4.225000in}}%
\pgfusepath{clip}%
\pgfsetbuttcap%
\pgfsetroundjoin%
\definecolor{currentfill}{rgb}{0.000000,0.000000,0.000000}%
\pgfsetfillcolor{currentfill}%
\pgfsetfillopacity{0.800000}%
\pgfsetlinewidth{0.000000pt}%
\definecolor{currentstroke}{rgb}{0.000000,0.000000,0.000000}%
\pgfsetstrokecolor{currentstroke}%
\pgfsetstrokeopacity{0.800000}%
\pgfsetdash{}{0pt}%
\pgfpathmoveto{\pgfqpoint{4.103857in}{3.259222in}}%
\pgfpathcurveto{\pgfqpoint{4.107975in}{3.259222in}}{\pgfqpoint{4.111925in}{3.260858in}}{\pgfqpoint{4.114837in}{3.263770in}}%
\pgfpathcurveto{\pgfqpoint{4.117749in}{3.266682in}}{\pgfqpoint{4.119385in}{3.270632in}}{\pgfqpoint{4.119385in}{3.274750in}}%
\pgfpathcurveto{\pgfqpoint{4.119385in}{3.278869in}}{\pgfqpoint{4.117749in}{3.282819in}}{\pgfqpoint{4.114837in}{3.285731in}}%
\pgfpathcurveto{\pgfqpoint{4.111925in}{3.288643in}}{\pgfqpoint{4.107975in}{3.290279in}}{\pgfqpoint{4.103857in}{3.290279in}}%
\pgfpathcurveto{\pgfqpoint{4.099739in}{3.290279in}}{\pgfqpoint{4.095789in}{3.288643in}}{\pgfqpoint{4.092877in}{3.285731in}}%
\pgfpathcurveto{\pgfqpoint{4.089965in}{3.282819in}}{\pgfqpoint{4.088329in}{3.278869in}}{\pgfqpoint{4.088329in}{3.274750in}}%
\pgfpathcurveto{\pgfqpoint{4.088329in}{3.270632in}}{\pgfqpoint{4.089965in}{3.266682in}}{\pgfqpoint{4.092877in}{3.263770in}}%
\pgfpathcurveto{\pgfqpoint{4.095789in}{3.260858in}}{\pgfqpoint{4.099739in}{3.259222in}}{\pgfqpoint{4.103857in}{3.259222in}}%
\pgfpathclose%
\pgfusepath{fill}%
\end{pgfscope}%
\begin{pgfscope}%
\pgfpathrectangle{\pgfqpoint{0.887500in}{0.275000in}}{\pgfqpoint{4.225000in}{4.225000in}}%
\pgfusepath{clip}%
\pgfsetbuttcap%
\pgfsetroundjoin%
\definecolor{currentfill}{rgb}{0.000000,0.000000,0.000000}%
\pgfsetfillcolor{currentfill}%
\pgfsetfillopacity{0.800000}%
\pgfsetlinewidth{0.000000pt}%
\definecolor{currentstroke}{rgb}{0.000000,0.000000,0.000000}%
\pgfsetstrokecolor{currentstroke}%
\pgfsetstrokeopacity{0.800000}%
\pgfsetdash{}{0pt}%
\pgfpathmoveto{\pgfqpoint{2.847484in}{2.995569in}}%
\pgfpathcurveto{\pgfqpoint{2.851602in}{2.995569in}}{\pgfqpoint{2.855552in}{2.997206in}}{\pgfqpoint{2.858464in}{3.000117in}}%
\pgfpathcurveto{\pgfqpoint{2.861376in}{3.003029in}}{\pgfqpoint{2.863013in}{3.006979in}}{\pgfqpoint{2.863013in}{3.011098in}}%
\pgfpathcurveto{\pgfqpoint{2.863013in}{3.015216in}}{\pgfqpoint{2.861376in}{3.019166in}}{\pgfqpoint{2.858464in}{3.022078in}}%
\pgfpathcurveto{\pgfqpoint{2.855552in}{3.024990in}}{\pgfqpoint{2.851602in}{3.026626in}}{\pgfqpoint{2.847484in}{3.026626in}}%
\pgfpathcurveto{\pgfqpoint{2.843366in}{3.026626in}}{\pgfqpoint{2.839416in}{3.024990in}}{\pgfqpoint{2.836504in}{3.022078in}}%
\pgfpathcurveto{\pgfqpoint{2.833592in}{3.019166in}}{\pgfqpoint{2.831956in}{3.015216in}}{\pgfqpoint{2.831956in}{3.011098in}}%
\pgfpathcurveto{\pgfqpoint{2.831956in}{3.006979in}}{\pgfqpoint{2.833592in}{3.003029in}}{\pgfqpoint{2.836504in}{3.000117in}}%
\pgfpathcurveto{\pgfqpoint{2.839416in}{2.997206in}}{\pgfqpoint{2.843366in}{2.995569in}}{\pgfqpoint{2.847484in}{2.995569in}}%
\pgfpathclose%
\pgfusepath{fill}%
\end{pgfscope}%
\begin{pgfscope}%
\pgfpathrectangle{\pgfqpoint{0.887500in}{0.275000in}}{\pgfqpoint{4.225000in}{4.225000in}}%
\pgfusepath{clip}%
\pgfsetbuttcap%
\pgfsetroundjoin%
\definecolor{currentfill}{rgb}{0.000000,0.000000,0.000000}%
\pgfsetfillcolor{currentfill}%
\pgfsetfillopacity{0.800000}%
\pgfsetlinewidth{0.000000pt}%
\definecolor{currentstroke}{rgb}{0.000000,0.000000,0.000000}%
\pgfsetstrokecolor{currentstroke}%
\pgfsetstrokeopacity{0.800000}%
\pgfsetdash{}{0pt}%
\pgfpathmoveto{\pgfqpoint{2.736314in}{2.876517in}}%
\pgfpathcurveto{\pgfqpoint{2.740433in}{2.876517in}}{\pgfqpoint{2.744383in}{2.878153in}}{\pgfqpoint{2.747295in}{2.881065in}}%
\pgfpathcurveto{\pgfqpoint{2.750207in}{2.883977in}}{\pgfqpoint{2.751843in}{2.887927in}}{\pgfqpoint{2.751843in}{2.892045in}}%
\pgfpathcurveto{\pgfqpoint{2.751843in}{2.896163in}}{\pgfqpoint{2.750207in}{2.900113in}}{\pgfqpoint{2.747295in}{2.903025in}}%
\pgfpathcurveto{\pgfqpoint{2.744383in}{2.905937in}}{\pgfqpoint{2.740433in}{2.907573in}}{\pgfqpoint{2.736314in}{2.907573in}}%
\pgfpathcurveto{\pgfqpoint{2.732196in}{2.907573in}}{\pgfqpoint{2.728246in}{2.905937in}}{\pgfqpoint{2.725334in}{2.903025in}}%
\pgfpathcurveto{\pgfqpoint{2.722422in}{2.900113in}}{\pgfqpoint{2.720786in}{2.896163in}}{\pgfqpoint{2.720786in}{2.892045in}}%
\pgfpathcurveto{\pgfqpoint{2.720786in}{2.887927in}}{\pgfqpoint{2.722422in}{2.883977in}}{\pgfqpoint{2.725334in}{2.881065in}}%
\pgfpathcurveto{\pgfqpoint{2.728246in}{2.878153in}}{\pgfqpoint{2.732196in}{2.876517in}}{\pgfqpoint{2.736314in}{2.876517in}}%
\pgfpathclose%
\pgfusepath{fill}%
\end{pgfscope}%
\begin{pgfscope}%
\pgfpathrectangle{\pgfqpoint{0.887500in}{0.275000in}}{\pgfqpoint{4.225000in}{4.225000in}}%
\pgfusepath{clip}%
\pgfsetbuttcap%
\pgfsetroundjoin%
\definecolor{currentfill}{rgb}{0.000000,0.000000,0.000000}%
\pgfsetfillcolor{currentfill}%
\pgfsetfillopacity{0.800000}%
\pgfsetlinewidth{0.000000pt}%
\definecolor{currentstroke}{rgb}{0.000000,0.000000,0.000000}%
\pgfsetstrokecolor{currentstroke}%
\pgfsetstrokeopacity{0.800000}%
\pgfsetdash{}{0pt}%
\pgfpathmoveto{\pgfqpoint{3.005522in}{3.484408in}}%
\pgfpathcurveto{\pgfqpoint{3.009640in}{3.484408in}}{\pgfqpoint{3.013590in}{3.486044in}}{\pgfqpoint{3.016502in}{3.488956in}}%
\pgfpathcurveto{\pgfqpoint{3.019414in}{3.491868in}}{\pgfqpoint{3.021050in}{3.495818in}}{\pgfqpoint{3.021050in}{3.499936in}}%
\pgfpathcurveto{\pgfqpoint{3.021050in}{3.504054in}}{\pgfqpoint{3.019414in}{3.508004in}}{\pgfqpoint{3.016502in}{3.510916in}}%
\pgfpathcurveto{\pgfqpoint{3.013590in}{3.513828in}}{\pgfqpoint{3.009640in}{3.515464in}}{\pgfqpoint{3.005522in}{3.515464in}}%
\pgfpathcurveto{\pgfqpoint{3.001404in}{3.515464in}}{\pgfqpoint{2.997454in}{3.513828in}}{\pgfqpoint{2.994542in}{3.510916in}}%
\pgfpathcurveto{\pgfqpoint{2.991630in}{3.508004in}}{\pgfqpoint{2.989994in}{3.504054in}}{\pgfqpoint{2.989994in}{3.499936in}}%
\pgfpathcurveto{\pgfqpoint{2.989994in}{3.495818in}}{\pgfqpoint{2.991630in}{3.491868in}}{\pgfqpoint{2.994542in}{3.488956in}}%
\pgfpathcurveto{\pgfqpoint{2.997454in}{3.486044in}}{\pgfqpoint{3.001404in}{3.484408in}}{\pgfqpoint{3.005522in}{3.484408in}}%
\pgfpathclose%
\pgfusepath{fill}%
\end{pgfscope}%
\begin{pgfscope}%
\pgfpathrectangle{\pgfqpoint{0.887500in}{0.275000in}}{\pgfqpoint{4.225000in}{4.225000in}}%
\pgfusepath{clip}%
\pgfsetbuttcap%
\pgfsetroundjoin%
\definecolor{currentfill}{rgb}{0.000000,0.000000,0.000000}%
\pgfsetfillcolor{currentfill}%
\pgfsetfillopacity{0.800000}%
\pgfsetlinewidth{0.000000pt}%
\definecolor{currentstroke}{rgb}{0.000000,0.000000,0.000000}%
\pgfsetstrokecolor{currentstroke}%
\pgfsetstrokeopacity{0.800000}%
\pgfsetdash{}{0pt}%
\pgfpathmoveto{\pgfqpoint{3.291939in}{3.520970in}}%
\pgfpathcurveto{\pgfqpoint{3.296057in}{3.520970in}}{\pgfqpoint{3.300007in}{3.522606in}}{\pgfqpoint{3.302919in}{3.525518in}}%
\pgfpathcurveto{\pgfqpoint{3.305831in}{3.528430in}}{\pgfqpoint{3.307467in}{3.532380in}}{\pgfqpoint{3.307467in}{3.536498in}}%
\pgfpathcurveto{\pgfqpoint{3.307467in}{3.540616in}}{\pgfqpoint{3.305831in}{3.544566in}}{\pgfqpoint{3.302919in}{3.547478in}}%
\pgfpathcurveto{\pgfqpoint{3.300007in}{3.550390in}}{\pgfqpoint{3.296057in}{3.552026in}}{\pgfqpoint{3.291939in}{3.552026in}}%
\pgfpathcurveto{\pgfqpoint{3.287821in}{3.552026in}}{\pgfqpoint{3.283871in}{3.550390in}}{\pgfqpoint{3.280959in}{3.547478in}}%
\pgfpathcurveto{\pgfqpoint{3.278047in}{3.544566in}}{\pgfqpoint{3.276410in}{3.540616in}}{\pgfqpoint{3.276410in}{3.536498in}}%
\pgfpathcurveto{\pgfqpoint{3.276410in}{3.532380in}}{\pgfqpoint{3.278047in}{3.528430in}}{\pgfqpoint{3.280959in}{3.525518in}}%
\pgfpathcurveto{\pgfqpoint{3.283871in}{3.522606in}}{\pgfqpoint{3.287821in}{3.520970in}}{\pgfqpoint{3.291939in}{3.520970in}}%
\pgfpathclose%
\pgfusepath{fill}%
\end{pgfscope}%
\begin{pgfscope}%
\pgfpathrectangle{\pgfqpoint{0.887500in}{0.275000in}}{\pgfqpoint{4.225000in}{4.225000in}}%
\pgfusepath{clip}%
\pgfsetbuttcap%
\pgfsetroundjoin%
\definecolor{currentfill}{rgb}{0.000000,0.000000,0.000000}%
\pgfsetfillcolor{currentfill}%
\pgfsetfillopacity{0.800000}%
\pgfsetlinewidth{0.000000pt}%
\definecolor{currentstroke}{rgb}{0.000000,0.000000,0.000000}%
\pgfsetstrokecolor{currentstroke}%
\pgfsetstrokeopacity{0.800000}%
\pgfsetdash{}{0pt}%
\pgfpathmoveto{\pgfqpoint{2.625059in}{2.760565in}}%
\pgfpathcurveto{\pgfqpoint{2.629177in}{2.760565in}}{\pgfqpoint{2.633127in}{2.762201in}}{\pgfqpoint{2.636039in}{2.765113in}}%
\pgfpathcurveto{\pgfqpoint{2.638951in}{2.768025in}}{\pgfqpoint{2.640587in}{2.771975in}}{\pgfqpoint{2.640587in}{2.776093in}}%
\pgfpathcurveto{\pgfqpoint{2.640587in}{2.780211in}}{\pgfqpoint{2.638951in}{2.784161in}}{\pgfqpoint{2.636039in}{2.787073in}}%
\pgfpathcurveto{\pgfqpoint{2.633127in}{2.789985in}}{\pgfqpoint{2.629177in}{2.791621in}}{\pgfqpoint{2.625059in}{2.791621in}}%
\pgfpathcurveto{\pgfqpoint{2.620941in}{2.791621in}}{\pgfqpoint{2.616991in}{2.789985in}}{\pgfqpoint{2.614079in}{2.787073in}}%
\pgfpathcurveto{\pgfqpoint{2.611167in}{2.784161in}}{\pgfqpoint{2.609531in}{2.780211in}}{\pgfqpoint{2.609531in}{2.776093in}}%
\pgfpathcurveto{\pgfqpoint{2.609531in}{2.771975in}}{\pgfqpoint{2.611167in}{2.768025in}}{\pgfqpoint{2.614079in}{2.765113in}}%
\pgfpathcurveto{\pgfqpoint{2.616991in}{2.762201in}}{\pgfqpoint{2.620941in}{2.760565in}}{\pgfqpoint{2.625059in}{2.760565in}}%
\pgfpathclose%
\pgfusepath{fill}%
\end{pgfscope}%
\begin{pgfscope}%
\pgfpathrectangle{\pgfqpoint{0.887500in}{0.275000in}}{\pgfqpoint{4.225000in}{4.225000in}}%
\pgfusepath{clip}%
\pgfsetbuttcap%
\pgfsetroundjoin%
\definecolor{currentfill}{rgb}{0.000000,0.000000,0.000000}%
\pgfsetfillcolor{currentfill}%
\pgfsetfillopacity{0.800000}%
\pgfsetlinewidth{0.000000pt}%
\definecolor{currentstroke}{rgb}{0.000000,0.000000,0.000000}%
\pgfsetstrokecolor{currentstroke}%
\pgfsetstrokeopacity{0.800000}%
\pgfsetdash{}{0pt}%
\pgfpathmoveto{\pgfqpoint{2.513852in}{2.635950in}}%
\pgfpathcurveto{\pgfqpoint{2.517970in}{2.635950in}}{\pgfqpoint{2.521920in}{2.637586in}}{\pgfqpoint{2.524832in}{2.640498in}}%
\pgfpathcurveto{\pgfqpoint{2.527744in}{2.643410in}}{\pgfqpoint{2.529380in}{2.647360in}}{\pgfqpoint{2.529380in}{2.651478in}}%
\pgfpathcurveto{\pgfqpoint{2.529380in}{2.655596in}}{\pgfqpoint{2.527744in}{2.659546in}}{\pgfqpoint{2.524832in}{2.662458in}}%
\pgfpathcurveto{\pgfqpoint{2.521920in}{2.665370in}}{\pgfqpoint{2.517970in}{2.667006in}}{\pgfqpoint{2.513852in}{2.667006in}}%
\pgfpathcurveto{\pgfqpoint{2.509734in}{2.667006in}}{\pgfqpoint{2.505784in}{2.665370in}}{\pgfqpoint{2.502872in}{2.662458in}}%
\pgfpathcurveto{\pgfqpoint{2.499960in}{2.659546in}}{\pgfqpoint{2.498324in}{2.655596in}}{\pgfqpoint{2.498324in}{2.651478in}}%
\pgfpathcurveto{\pgfqpoint{2.498324in}{2.647360in}}{\pgfqpoint{2.499960in}{2.643410in}}{\pgfqpoint{2.502872in}{2.640498in}}%
\pgfpathcurveto{\pgfqpoint{2.505784in}{2.637586in}}{\pgfqpoint{2.509734in}{2.635950in}}{\pgfqpoint{2.513852in}{2.635950in}}%
\pgfpathclose%
\pgfusepath{fill}%
\end{pgfscope}%
\begin{pgfscope}%
\pgfpathrectangle{\pgfqpoint{0.887500in}{0.275000in}}{\pgfqpoint{4.225000in}{4.225000in}}%
\pgfusepath{clip}%
\pgfsetbuttcap%
\pgfsetroundjoin%
\definecolor{currentfill}{rgb}{0.000000,0.000000,0.000000}%
\pgfsetfillcolor{currentfill}%
\pgfsetfillopacity{0.800000}%
\pgfsetlinewidth{0.000000pt}%
\definecolor{currentstroke}{rgb}{0.000000,0.000000,0.000000}%
\pgfsetstrokecolor{currentstroke}%
\pgfsetstrokeopacity{0.800000}%
\pgfsetdash{}{0pt}%
\pgfpathmoveto{\pgfqpoint{2.051791in}{2.703351in}}%
\pgfpathcurveto{\pgfqpoint{2.055909in}{2.703351in}}{\pgfqpoint{2.059859in}{2.704987in}}{\pgfqpoint{2.062771in}{2.707899in}}%
\pgfpathcurveto{\pgfqpoint{2.065683in}{2.710811in}}{\pgfqpoint{2.067320in}{2.714761in}}{\pgfqpoint{2.067320in}{2.718879in}}%
\pgfpathcurveto{\pgfqpoint{2.067320in}{2.722997in}}{\pgfqpoint{2.065683in}{2.726947in}}{\pgfqpoint{2.062771in}{2.729859in}}%
\pgfpathcurveto{\pgfqpoint{2.059859in}{2.732771in}}{\pgfqpoint{2.055909in}{2.734407in}}{\pgfqpoint{2.051791in}{2.734407in}}%
\pgfpathcurveto{\pgfqpoint{2.047673in}{2.734407in}}{\pgfqpoint{2.043723in}{2.732771in}}{\pgfqpoint{2.040811in}{2.729859in}}%
\pgfpathcurveto{\pgfqpoint{2.037899in}{2.726947in}}{\pgfqpoint{2.036263in}{2.722997in}}{\pgfqpoint{2.036263in}{2.718879in}}%
\pgfpathcurveto{\pgfqpoint{2.036263in}{2.714761in}}{\pgfqpoint{2.037899in}{2.710811in}}{\pgfqpoint{2.040811in}{2.707899in}}%
\pgfpathcurveto{\pgfqpoint{2.043723in}{2.704987in}}{\pgfqpoint{2.047673in}{2.703351in}}{\pgfqpoint{2.051791in}{2.703351in}}%
\pgfpathclose%
\pgfusepath{fill}%
\end{pgfscope}%
\begin{pgfscope}%
\pgfpathrectangle{\pgfqpoint{0.887500in}{0.275000in}}{\pgfqpoint{4.225000in}{4.225000in}}%
\pgfusepath{clip}%
\pgfsetbuttcap%
\pgfsetroundjoin%
\definecolor{currentfill}{rgb}{0.000000,0.000000,0.000000}%
\pgfsetfillcolor{currentfill}%
\pgfsetfillopacity{0.800000}%
\pgfsetlinewidth{0.000000pt}%
\definecolor{currentstroke}{rgb}{0.000000,0.000000,0.000000}%
\pgfsetstrokecolor{currentstroke}%
\pgfsetstrokeopacity{0.800000}%
\pgfsetdash{}{0pt}%
\pgfpathmoveto{\pgfqpoint{1.589731in}{2.765828in}}%
\pgfpathcurveto{\pgfqpoint{1.593849in}{2.765828in}}{\pgfqpoint{1.597799in}{2.767464in}}{\pgfqpoint{1.600711in}{2.770376in}}%
\pgfpathcurveto{\pgfqpoint{1.603623in}{2.773288in}}{\pgfqpoint{1.605259in}{2.777238in}}{\pgfqpoint{1.605259in}{2.781356in}}%
\pgfpathcurveto{\pgfqpoint{1.605259in}{2.785474in}}{\pgfqpoint{1.603623in}{2.789424in}}{\pgfqpoint{1.600711in}{2.792336in}}%
\pgfpathcurveto{\pgfqpoint{1.597799in}{2.795248in}}{\pgfqpoint{1.593849in}{2.796884in}}{\pgfqpoint{1.589731in}{2.796884in}}%
\pgfpathcurveto{\pgfqpoint{1.585613in}{2.796884in}}{\pgfqpoint{1.581663in}{2.795248in}}{\pgfqpoint{1.578751in}{2.792336in}}%
\pgfpathcurveto{\pgfqpoint{1.575839in}{2.789424in}}{\pgfqpoint{1.574203in}{2.785474in}}{\pgfqpoint{1.574203in}{2.781356in}}%
\pgfpathcurveto{\pgfqpoint{1.574203in}{2.777238in}}{\pgfqpoint{1.575839in}{2.773288in}}{\pgfqpoint{1.578751in}{2.770376in}}%
\pgfpathcurveto{\pgfqpoint{1.581663in}{2.767464in}}{\pgfqpoint{1.585613in}{2.765828in}}{\pgfqpoint{1.589731in}{2.765828in}}%
\pgfpathclose%
\pgfusepath{fill}%
\end{pgfscope}%
\begin{pgfscope}%
\pgfpathrectangle{\pgfqpoint{0.887500in}{0.275000in}}{\pgfqpoint{4.225000in}{4.225000in}}%
\pgfusepath{clip}%
\pgfsetbuttcap%
\pgfsetroundjoin%
\definecolor{currentfill}{rgb}{0.000000,0.000000,0.000000}%
\pgfsetfillcolor{currentfill}%
\pgfsetfillopacity{0.800000}%
\pgfsetlinewidth{0.000000pt}%
\definecolor{currentstroke}{rgb}{0.000000,0.000000,0.000000}%
\pgfsetstrokecolor{currentstroke}%
\pgfsetstrokeopacity{0.800000}%
\pgfsetdash{}{0pt}%
\pgfpathmoveto{\pgfqpoint{3.467799in}{3.483387in}}%
\pgfpathcurveto{\pgfqpoint{3.471917in}{3.483387in}}{\pgfqpoint{3.475867in}{3.485023in}}{\pgfqpoint{3.478779in}{3.487935in}}%
\pgfpathcurveto{\pgfqpoint{3.481691in}{3.490847in}}{\pgfqpoint{3.483327in}{3.494797in}}{\pgfqpoint{3.483327in}{3.498915in}}%
\pgfpathcurveto{\pgfqpoint{3.483327in}{3.503033in}}{\pgfqpoint{3.481691in}{3.506983in}}{\pgfqpoint{3.478779in}{3.509895in}}%
\pgfpathcurveto{\pgfqpoint{3.475867in}{3.512807in}}{\pgfqpoint{3.471917in}{3.514443in}}{\pgfqpoint{3.467799in}{3.514443in}}%
\pgfpathcurveto{\pgfqpoint{3.463681in}{3.514443in}}{\pgfqpoint{3.459730in}{3.512807in}}{\pgfqpoint{3.456819in}{3.509895in}}%
\pgfpathcurveto{\pgfqpoint{3.453907in}{3.506983in}}{\pgfqpoint{3.452270in}{3.503033in}}{\pgfqpoint{3.452270in}{3.498915in}}%
\pgfpathcurveto{\pgfqpoint{3.452270in}{3.494797in}}{\pgfqpoint{3.453907in}{3.490847in}}{\pgfqpoint{3.456819in}{3.487935in}}%
\pgfpathcurveto{\pgfqpoint{3.459730in}{3.485023in}}{\pgfqpoint{3.463681in}{3.483387in}}{\pgfqpoint{3.467799in}{3.483387in}}%
\pgfpathclose%
\pgfusepath{fill}%
\end{pgfscope}%
\begin{pgfscope}%
\pgfpathrectangle{\pgfqpoint{0.887500in}{0.275000in}}{\pgfqpoint{4.225000in}{4.225000in}}%
\pgfusepath{clip}%
\pgfsetbuttcap%
\pgfsetroundjoin%
\definecolor{currentfill}{rgb}{0.000000,0.000000,0.000000}%
\pgfsetfillcolor{currentfill}%
\pgfsetfillopacity{0.800000}%
\pgfsetlinewidth{0.000000pt}%
\definecolor{currentstroke}{rgb}{0.000000,0.000000,0.000000}%
\pgfsetstrokecolor{currentstroke}%
\pgfsetstrokeopacity{0.800000}%
\pgfsetdash{}{0pt}%
\pgfpathmoveto{\pgfqpoint{2.226253in}{2.653447in}}%
\pgfpathcurveto{\pgfqpoint{2.230372in}{2.653447in}}{\pgfqpoint{2.234322in}{2.655084in}}{\pgfqpoint{2.237234in}{2.657996in}}%
\pgfpathcurveto{\pgfqpoint{2.240146in}{2.660908in}}{\pgfqpoint{2.241782in}{2.664858in}}{\pgfqpoint{2.241782in}{2.668976in}}%
\pgfpathcurveto{\pgfqpoint{2.241782in}{2.673094in}}{\pgfqpoint{2.240146in}{2.677044in}}{\pgfqpoint{2.237234in}{2.679956in}}%
\pgfpathcurveto{\pgfqpoint{2.234322in}{2.682868in}}{\pgfqpoint{2.230372in}{2.684504in}}{\pgfqpoint{2.226253in}{2.684504in}}%
\pgfpathcurveto{\pgfqpoint{2.222135in}{2.684504in}}{\pgfqpoint{2.218185in}{2.682868in}}{\pgfqpoint{2.215273in}{2.679956in}}%
\pgfpathcurveto{\pgfqpoint{2.212361in}{2.677044in}}{\pgfqpoint{2.210725in}{2.673094in}}{\pgfqpoint{2.210725in}{2.668976in}}%
\pgfpathcurveto{\pgfqpoint{2.210725in}{2.664858in}}{\pgfqpoint{2.212361in}{2.660908in}}{\pgfqpoint{2.215273in}{2.657996in}}%
\pgfpathcurveto{\pgfqpoint{2.218185in}{2.655084in}}{\pgfqpoint{2.222135in}{2.653447in}}{\pgfqpoint{2.226253in}{2.653447in}}%
\pgfpathclose%
\pgfusepath{fill}%
\end{pgfscope}%
\begin{pgfscope}%
\pgfpathrectangle{\pgfqpoint{0.887500in}{0.275000in}}{\pgfqpoint{4.225000in}{4.225000in}}%
\pgfusepath{clip}%
\pgfsetbuttcap%
\pgfsetroundjoin%
\definecolor{currentfill}{rgb}{0.000000,0.000000,0.000000}%
\pgfsetfillcolor{currentfill}%
\pgfsetfillopacity{0.800000}%
\pgfsetlinewidth{0.000000pt}%
\definecolor{currentstroke}{rgb}{0.000000,0.000000,0.000000}%
\pgfsetstrokecolor{currentstroke}%
\pgfsetstrokeopacity{0.800000}%
\pgfsetdash{}{0pt}%
\pgfpathmoveto{\pgfqpoint{1.301171in}{2.776148in}}%
\pgfpathcurveto{\pgfqpoint{1.305289in}{2.776148in}}{\pgfqpoint{1.309239in}{2.777785in}}{\pgfqpoint{1.312151in}{2.780697in}}%
\pgfpathcurveto{\pgfqpoint{1.315063in}{2.783609in}}{\pgfqpoint{1.316699in}{2.787559in}}{\pgfqpoint{1.316699in}{2.791677in}}%
\pgfpathcurveto{\pgfqpoint{1.316699in}{2.795795in}}{\pgfqpoint{1.315063in}{2.799745in}}{\pgfqpoint{1.312151in}{2.802657in}}%
\pgfpathcurveto{\pgfqpoint{1.309239in}{2.805569in}}{\pgfqpoint{1.305289in}{2.807205in}}{\pgfqpoint{1.301171in}{2.807205in}}%
\pgfpathcurveto{\pgfqpoint{1.297053in}{2.807205in}}{\pgfqpoint{1.293103in}{2.805569in}}{\pgfqpoint{1.290191in}{2.802657in}}%
\pgfpathcurveto{\pgfqpoint{1.287279in}{2.799745in}}{\pgfqpoint{1.285643in}{2.795795in}}{\pgfqpoint{1.285643in}{2.791677in}}%
\pgfpathcurveto{\pgfqpoint{1.285643in}{2.787559in}}{\pgfqpoint{1.287279in}{2.783609in}}{\pgfqpoint{1.290191in}{2.780697in}}%
\pgfpathcurveto{\pgfqpoint{1.293103in}{2.777785in}}{\pgfqpoint{1.297053in}{2.776148in}}{\pgfqpoint{1.301171in}{2.776148in}}%
\pgfpathclose%
\pgfusepath{fill}%
\end{pgfscope}%
\begin{pgfscope}%
\pgfpathrectangle{\pgfqpoint{0.887500in}{0.275000in}}{\pgfqpoint{4.225000in}{4.225000in}}%
\pgfusepath{clip}%
\pgfsetbuttcap%
\pgfsetroundjoin%
\definecolor{currentfill}{rgb}{0.000000,0.000000,0.000000}%
\pgfsetfillcolor{currentfill}%
\pgfsetfillopacity{0.800000}%
\pgfsetlinewidth{0.000000pt}%
\definecolor{currentstroke}{rgb}{0.000000,0.000000,0.000000}%
\pgfsetstrokecolor{currentstroke}%
\pgfsetstrokeopacity{0.800000}%
\pgfsetdash{}{0pt}%
\pgfpathmoveto{\pgfqpoint{1.763596in}{2.718977in}}%
\pgfpathcurveto{\pgfqpoint{1.767714in}{2.718977in}}{\pgfqpoint{1.771664in}{2.720613in}}{\pgfqpoint{1.774576in}{2.723525in}}%
\pgfpathcurveto{\pgfqpoint{1.777488in}{2.726437in}}{\pgfqpoint{1.779124in}{2.730387in}}{\pgfqpoint{1.779124in}{2.734505in}}%
\pgfpathcurveto{\pgfqpoint{1.779124in}{2.738623in}}{\pgfqpoint{1.777488in}{2.742573in}}{\pgfqpoint{1.774576in}{2.745485in}}%
\pgfpathcurveto{\pgfqpoint{1.771664in}{2.748397in}}{\pgfqpoint{1.767714in}{2.750033in}}{\pgfqpoint{1.763596in}{2.750033in}}%
\pgfpathcurveto{\pgfqpoint{1.759478in}{2.750033in}}{\pgfqpoint{1.755528in}{2.748397in}}{\pgfqpoint{1.752616in}{2.745485in}}%
\pgfpathcurveto{\pgfqpoint{1.749704in}{2.742573in}}{\pgfqpoint{1.748068in}{2.738623in}}{\pgfqpoint{1.748068in}{2.734505in}}%
\pgfpathcurveto{\pgfqpoint{1.748068in}{2.730387in}}{\pgfqpoint{1.749704in}{2.726437in}}{\pgfqpoint{1.752616in}{2.723525in}}%
\pgfpathcurveto{\pgfqpoint{1.755528in}{2.720613in}}{\pgfqpoint{1.759478in}{2.718977in}}{\pgfqpoint{1.763596in}{2.718977in}}%
\pgfpathclose%
\pgfusepath{fill}%
\end{pgfscope}%
\begin{pgfscope}%
\pgfpathrectangle{\pgfqpoint{0.887500in}{0.275000in}}{\pgfqpoint{4.225000in}{4.225000in}}%
\pgfusepath{clip}%
\pgfsetbuttcap%
\pgfsetroundjoin%
\definecolor{currentfill}{rgb}{0.000000,0.000000,0.000000}%
\pgfsetfillcolor{currentfill}%
\pgfsetfillopacity{0.800000}%
\pgfsetlinewidth{0.000000pt}%
\definecolor{currentstroke}{rgb}{0.000000,0.000000,0.000000}%
\pgfsetstrokecolor{currentstroke}%
\pgfsetstrokeopacity{0.800000}%
\pgfsetdash{}{0pt}%
\pgfpathmoveto{\pgfqpoint{3.643572in}{3.405897in}}%
\pgfpathcurveto{\pgfqpoint{3.647690in}{3.405897in}}{\pgfqpoint{3.651640in}{3.407533in}}{\pgfqpoint{3.654552in}{3.410445in}}%
\pgfpathcurveto{\pgfqpoint{3.657464in}{3.413357in}}{\pgfqpoint{3.659100in}{3.417307in}}{\pgfqpoint{3.659100in}{3.421425in}}%
\pgfpathcurveto{\pgfqpoint{3.659100in}{3.425543in}}{\pgfqpoint{3.657464in}{3.429493in}}{\pgfqpoint{3.654552in}{3.432405in}}%
\pgfpathcurveto{\pgfqpoint{3.651640in}{3.435317in}}{\pgfqpoint{3.647690in}{3.436953in}}{\pgfqpoint{3.643572in}{3.436953in}}%
\pgfpathcurveto{\pgfqpoint{3.639454in}{3.436953in}}{\pgfqpoint{3.635504in}{3.435317in}}{\pgfqpoint{3.632592in}{3.432405in}}%
\pgfpathcurveto{\pgfqpoint{3.629680in}{3.429493in}}{\pgfqpoint{3.628044in}{3.425543in}}{\pgfqpoint{3.628044in}{3.421425in}}%
\pgfpathcurveto{\pgfqpoint{3.628044in}{3.417307in}}{\pgfqpoint{3.629680in}{3.413357in}}{\pgfqpoint{3.632592in}{3.410445in}}%
\pgfpathcurveto{\pgfqpoint{3.635504in}{3.407533in}}{\pgfqpoint{3.639454in}{3.405897in}}{\pgfqpoint{3.643572in}{3.405897in}}%
\pgfpathclose%
\pgfusepath{fill}%
\end{pgfscope}%
\begin{pgfscope}%
\pgfpathrectangle{\pgfqpoint{0.887500in}{0.275000in}}{\pgfqpoint{4.225000in}{4.225000in}}%
\pgfusepath{clip}%
\pgfsetbuttcap%
\pgfsetroundjoin%
\definecolor{currentfill}{rgb}{0.000000,0.000000,0.000000}%
\pgfsetfillcolor{currentfill}%
\pgfsetfillopacity{0.800000}%
\pgfsetlinewidth{0.000000pt}%
\definecolor{currentstroke}{rgb}{0.000000,0.000000,0.000000}%
\pgfsetstrokecolor{currentstroke}%
\pgfsetstrokeopacity{0.800000}%
\pgfsetdash{}{0pt}%
\pgfpathmoveto{\pgfqpoint{4.170684in}{3.134278in}}%
\pgfpathcurveto{\pgfqpoint{4.174803in}{3.134278in}}{\pgfqpoint{4.178753in}{3.135915in}}{\pgfqpoint{4.181665in}{3.138827in}}%
\pgfpathcurveto{\pgfqpoint{4.184577in}{3.141739in}}{\pgfqpoint{4.186213in}{3.145689in}}{\pgfqpoint{4.186213in}{3.149807in}}%
\pgfpathcurveto{\pgfqpoint{4.186213in}{3.153925in}}{\pgfqpoint{4.184577in}{3.157875in}}{\pgfqpoint{4.181665in}{3.160787in}}%
\pgfpathcurveto{\pgfqpoint{4.178753in}{3.163699in}}{\pgfqpoint{4.174803in}{3.165335in}}{\pgfqpoint{4.170684in}{3.165335in}}%
\pgfpathcurveto{\pgfqpoint{4.166566in}{3.165335in}}{\pgfqpoint{4.162616in}{3.163699in}}{\pgfqpoint{4.159704in}{3.160787in}}%
\pgfpathcurveto{\pgfqpoint{4.156792in}{3.157875in}}{\pgfqpoint{4.155156in}{3.153925in}}{\pgfqpoint{4.155156in}{3.149807in}}%
\pgfpathcurveto{\pgfqpoint{4.155156in}{3.145689in}}{\pgfqpoint{4.156792in}{3.141739in}}{\pgfqpoint{4.159704in}{3.138827in}}%
\pgfpathcurveto{\pgfqpoint{4.162616in}{3.135915in}}{\pgfqpoint{4.166566in}{3.134278in}}{\pgfqpoint{4.170684in}{3.134278in}}%
\pgfpathclose%
\pgfusepath{fill}%
\end{pgfscope}%
\begin{pgfscope}%
\pgfpathrectangle{\pgfqpoint{0.887500in}{0.275000in}}{\pgfqpoint{4.225000in}{4.225000in}}%
\pgfusepath{clip}%
\pgfsetbuttcap%
\pgfsetroundjoin%
\definecolor{currentfill}{rgb}{0.000000,0.000000,0.000000}%
\pgfsetfillcolor{currentfill}%
\pgfsetfillopacity{0.800000}%
\pgfsetlinewidth{0.000000pt}%
\definecolor{currentstroke}{rgb}{0.000000,0.000000,0.000000}%
\pgfsetstrokecolor{currentstroke}%
\pgfsetstrokeopacity{0.800000}%
\pgfsetdash{}{0pt}%
\pgfpathmoveto{\pgfqpoint{3.819401in}{3.323663in}}%
\pgfpathcurveto{\pgfqpoint{3.823519in}{3.323663in}}{\pgfqpoint{3.827469in}{3.325299in}}{\pgfqpoint{3.830381in}{3.328211in}}%
\pgfpathcurveto{\pgfqpoint{3.833293in}{3.331123in}}{\pgfqpoint{3.834930in}{3.335073in}}{\pgfqpoint{3.834930in}{3.339191in}}%
\pgfpathcurveto{\pgfqpoint{3.834930in}{3.343309in}}{\pgfqpoint{3.833293in}{3.347259in}}{\pgfqpoint{3.830381in}{3.350171in}}%
\pgfpathcurveto{\pgfqpoint{3.827469in}{3.353083in}}{\pgfqpoint{3.823519in}{3.354719in}}{\pgfqpoint{3.819401in}{3.354719in}}%
\pgfpathcurveto{\pgfqpoint{3.815283in}{3.354719in}}{\pgfqpoint{3.811333in}{3.353083in}}{\pgfqpoint{3.808421in}{3.350171in}}%
\pgfpathcurveto{\pgfqpoint{3.805509in}{3.347259in}}{\pgfqpoint{3.803873in}{3.343309in}}{\pgfqpoint{3.803873in}{3.339191in}}%
\pgfpathcurveto{\pgfqpoint{3.803873in}{3.335073in}}{\pgfqpoint{3.805509in}{3.331123in}}{\pgfqpoint{3.808421in}{3.328211in}}%
\pgfpathcurveto{\pgfqpoint{3.811333in}{3.325299in}}{\pgfqpoint{3.815283in}{3.323663in}}{\pgfqpoint{3.819401in}{3.323663in}}%
\pgfpathclose%
\pgfusepath{fill}%
\end{pgfscope}%
\begin{pgfscope}%
\pgfpathrectangle{\pgfqpoint{0.887500in}{0.275000in}}{\pgfqpoint{4.225000in}{4.225000in}}%
\pgfusepath{clip}%
\pgfsetbuttcap%
\pgfsetroundjoin%
\definecolor{currentfill}{rgb}{0.000000,0.000000,0.000000}%
\pgfsetfillcolor{currentfill}%
\pgfsetfillopacity{0.800000}%
\pgfsetlinewidth{0.000000pt}%
\definecolor{currentstroke}{rgb}{0.000000,0.000000,0.000000}%
\pgfsetstrokecolor{currentstroke}%
\pgfsetstrokeopacity{0.800000}%
\pgfsetdash{}{0pt}%
\pgfpathmoveto{\pgfqpoint{3.995119in}{3.232126in}}%
\pgfpathcurveto{\pgfqpoint{3.999237in}{3.232126in}}{\pgfqpoint{4.003187in}{3.233762in}}{\pgfqpoint{4.006099in}{3.236674in}}%
\pgfpathcurveto{\pgfqpoint{4.009011in}{3.239586in}}{\pgfqpoint{4.010647in}{3.243536in}}{\pgfqpoint{4.010647in}{3.247654in}}%
\pgfpathcurveto{\pgfqpoint{4.010647in}{3.251772in}}{\pgfqpoint{4.009011in}{3.255722in}}{\pgfqpoint{4.006099in}{3.258634in}}%
\pgfpathcurveto{\pgfqpoint{4.003187in}{3.261546in}}{\pgfqpoint{3.999237in}{3.263182in}}{\pgfqpoint{3.995119in}{3.263182in}}%
\pgfpathcurveto{\pgfqpoint{3.991001in}{3.263182in}}{\pgfqpoint{3.987051in}{3.261546in}}{\pgfqpoint{3.984139in}{3.258634in}}%
\pgfpathcurveto{\pgfqpoint{3.981227in}{3.255722in}}{\pgfqpoint{3.979590in}{3.251772in}}{\pgfqpoint{3.979590in}{3.247654in}}%
\pgfpathcurveto{\pgfqpoint{3.979590in}{3.243536in}}{\pgfqpoint{3.981227in}{3.239586in}}{\pgfqpoint{3.984139in}{3.236674in}}%
\pgfpathcurveto{\pgfqpoint{3.987051in}{3.233762in}}{\pgfqpoint{3.991001in}{3.232126in}}{\pgfqpoint{3.995119in}{3.232126in}}%
\pgfpathclose%
\pgfusepath{fill}%
\end{pgfscope}%
\begin{pgfscope}%
\pgfpathrectangle{\pgfqpoint{0.887500in}{0.275000in}}{\pgfqpoint{4.225000in}{4.225000in}}%
\pgfusepath{clip}%
\pgfsetbuttcap%
\pgfsetroundjoin%
\definecolor{currentfill}{rgb}{0.000000,0.000000,0.000000}%
\pgfsetfillcolor{currentfill}%
\pgfsetfillopacity{0.800000}%
\pgfsetlinewidth{0.000000pt}%
\definecolor{currentstroke}{rgb}{0.000000,0.000000,0.000000}%
\pgfsetstrokecolor{currentstroke}%
\pgfsetstrokeopacity{0.800000}%
\pgfsetdash{}{0pt}%
\pgfpathmoveto{\pgfqpoint{3.181088in}{3.483815in}}%
\pgfpathcurveto{\pgfqpoint{3.185206in}{3.483815in}}{\pgfqpoint{3.189156in}{3.485451in}}{\pgfqpoint{3.192068in}{3.488363in}}%
\pgfpathcurveto{\pgfqpoint{3.194980in}{3.491275in}}{\pgfqpoint{3.196616in}{3.495225in}}{\pgfqpoint{3.196616in}{3.499343in}}%
\pgfpathcurveto{\pgfqpoint{3.196616in}{3.503461in}}{\pgfqpoint{3.194980in}{3.507411in}}{\pgfqpoint{3.192068in}{3.510323in}}%
\pgfpathcurveto{\pgfqpoint{3.189156in}{3.513235in}}{\pgfqpoint{3.185206in}{3.514871in}}{\pgfqpoint{3.181088in}{3.514871in}}%
\pgfpathcurveto{\pgfqpoint{3.176969in}{3.514871in}}{\pgfqpoint{3.173019in}{3.513235in}}{\pgfqpoint{3.170107in}{3.510323in}}%
\pgfpathcurveto{\pgfqpoint{3.167196in}{3.507411in}}{\pgfqpoint{3.165559in}{3.503461in}}{\pgfqpoint{3.165559in}{3.499343in}}%
\pgfpathcurveto{\pgfqpoint{3.165559in}{3.495225in}}{\pgfqpoint{3.167196in}{3.491275in}}{\pgfqpoint{3.170107in}{3.488363in}}%
\pgfpathcurveto{\pgfqpoint{3.173019in}{3.485451in}}{\pgfqpoint{3.176969in}{3.483815in}}{\pgfqpoint{3.181088in}{3.483815in}}%
\pgfpathclose%
\pgfusepath{fill}%
\end{pgfscope}%
\begin{pgfscope}%
\pgfpathrectangle{\pgfqpoint{0.887500in}{0.275000in}}{\pgfqpoint{4.225000in}{4.225000in}}%
\pgfusepath{clip}%
\pgfsetbuttcap%
\pgfsetroundjoin%
\definecolor{currentfill}{rgb}{0.000000,0.000000,0.000000}%
\pgfsetfillcolor{currentfill}%
\pgfsetfillopacity{0.800000}%
\pgfsetlinewidth{0.000000pt}%
\definecolor{currentstroke}{rgb}{0.000000,0.000000,0.000000}%
\pgfsetstrokecolor{currentstroke}%
\pgfsetstrokeopacity{0.800000}%
\pgfsetdash{}{0pt}%
\pgfpathmoveto{\pgfqpoint{2.799777in}{2.761724in}}%
\pgfpathcurveto{\pgfqpoint{2.803895in}{2.761724in}}{\pgfqpoint{2.807845in}{2.763360in}}{\pgfqpoint{2.810757in}{2.766272in}}%
\pgfpathcurveto{\pgfqpoint{2.813669in}{2.769184in}}{\pgfqpoint{2.815306in}{2.773134in}}{\pgfqpoint{2.815306in}{2.777252in}}%
\pgfpathcurveto{\pgfqpoint{2.815306in}{2.781370in}}{\pgfqpoint{2.813669in}{2.785320in}}{\pgfqpoint{2.810757in}{2.788232in}}%
\pgfpathcurveto{\pgfqpoint{2.807845in}{2.791144in}}{\pgfqpoint{2.803895in}{2.792780in}}{\pgfqpoint{2.799777in}{2.792780in}}%
\pgfpathcurveto{\pgfqpoint{2.795659in}{2.792780in}}{\pgfqpoint{2.791709in}{2.791144in}}{\pgfqpoint{2.788797in}{2.788232in}}%
\pgfpathcurveto{\pgfqpoint{2.785885in}{2.785320in}}{\pgfqpoint{2.784249in}{2.781370in}}{\pgfqpoint{2.784249in}{2.777252in}}%
\pgfpathcurveto{\pgfqpoint{2.784249in}{2.773134in}}{\pgfqpoint{2.785885in}{2.769184in}}{\pgfqpoint{2.788797in}{2.766272in}}%
\pgfpathcurveto{\pgfqpoint{2.791709in}{2.763360in}}{\pgfqpoint{2.795659in}{2.761724in}}{\pgfqpoint{2.799777in}{2.761724in}}%
\pgfpathclose%
\pgfusepath{fill}%
\end{pgfscope}%
\begin{pgfscope}%
\pgfpathrectangle{\pgfqpoint{0.887500in}{0.275000in}}{\pgfqpoint{4.225000in}{4.225000in}}%
\pgfusepath{clip}%
\pgfsetbuttcap%
\pgfsetroundjoin%
\definecolor{currentfill}{rgb}{0.000000,0.000000,0.000000}%
\pgfsetfillcolor{currentfill}%
\pgfsetfillopacity{0.800000}%
\pgfsetlinewidth{0.000000pt}%
\definecolor{currentstroke}{rgb}{0.000000,0.000000,0.000000}%
\pgfsetstrokecolor{currentstroke}%
\pgfsetstrokeopacity{0.800000}%
\pgfsetdash{}{0pt}%
\pgfpathmoveto{\pgfqpoint{2.401128in}{2.601900in}}%
\pgfpathcurveto{\pgfqpoint{2.405247in}{2.601900in}}{\pgfqpoint{2.409197in}{2.603536in}}{\pgfqpoint{2.412109in}{2.606448in}}%
\pgfpathcurveto{\pgfqpoint{2.415020in}{2.609360in}}{\pgfqpoint{2.416657in}{2.613310in}}{\pgfqpoint{2.416657in}{2.617428in}}%
\pgfpathcurveto{\pgfqpoint{2.416657in}{2.621547in}}{\pgfqpoint{2.415020in}{2.625497in}}{\pgfqpoint{2.412109in}{2.628409in}}%
\pgfpathcurveto{\pgfqpoint{2.409197in}{2.631320in}}{\pgfqpoint{2.405247in}{2.632957in}}{\pgfqpoint{2.401128in}{2.632957in}}%
\pgfpathcurveto{\pgfqpoint{2.397010in}{2.632957in}}{\pgfqpoint{2.393060in}{2.631320in}}{\pgfqpoint{2.390148in}{2.628409in}}%
\pgfpathcurveto{\pgfqpoint{2.387236in}{2.625497in}}{\pgfqpoint{2.385600in}{2.621547in}}{\pgfqpoint{2.385600in}{2.617428in}}%
\pgfpathcurveto{\pgfqpoint{2.385600in}{2.613310in}}{\pgfqpoint{2.387236in}{2.609360in}}{\pgfqpoint{2.390148in}{2.606448in}}%
\pgfpathcurveto{\pgfqpoint{2.393060in}{2.603536in}}{\pgfqpoint{2.397010in}{2.601900in}}{\pgfqpoint{2.401128in}{2.601900in}}%
\pgfpathclose%
\pgfusepath{fill}%
\end{pgfscope}%
\begin{pgfscope}%
\pgfpathrectangle{\pgfqpoint{0.887500in}{0.275000in}}{\pgfqpoint{4.225000in}{4.225000in}}%
\pgfusepath{clip}%
\pgfsetbuttcap%
\pgfsetroundjoin%
\definecolor{currentfill}{rgb}{0.000000,0.000000,0.000000}%
\pgfsetfillcolor{currentfill}%
\pgfsetfillopacity{0.800000}%
\pgfsetlinewidth{0.000000pt}%
\definecolor{currentstroke}{rgb}{0.000000,0.000000,0.000000}%
\pgfsetstrokecolor{currentstroke}%
\pgfsetstrokeopacity{0.800000}%
\pgfsetdash{}{0pt}%
\pgfpathmoveto{\pgfqpoint{2.893948in}{3.468310in}}%
\pgfpathcurveto{\pgfqpoint{2.898066in}{3.468310in}}{\pgfqpoint{2.902016in}{3.469946in}}{\pgfqpoint{2.904928in}{3.472858in}}%
\pgfpathcurveto{\pgfqpoint{2.907840in}{3.475770in}}{\pgfqpoint{2.909476in}{3.479720in}}{\pgfqpoint{2.909476in}{3.483838in}}%
\pgfpathcurveto{\pgfqpoint{2.909476in}{3.487956in}}{\pgfqpoint{2.907840in}{3.491906in}}{\pgfqpoint{2.904928in}{3.494818in}}%
\pgfpathcurveto{\pgfqpoint{2.902016in}{3.497730in}}{\pgfqpoint{2.898066in}{3.499367in}}{\pgfqpoint{2.893948in}{3.499367in}}%
\pgfpathcurveto{\pgfqpoint{2.889829in}{3.499367in}}{\pgfqpoint{2.885879in}{3.497730in}}{\pgfqpoint{2.882967in}{3.494818in}}%
\pgfpathcurveto{\pgfqpoint{2.880055in}{3.491906in}}{\pgfqpoint{2.878419in}{3.487956in}}{\pgfqpoint{2.878419in}{3.483838in}}%
\pgfpathcurveto{\pgfqpoint{2.878419in}{3.479720in}}{\pgfqpoint{2.880055in}{3.475770in}}{\pgfqpoint{2.882967in}{3.472858in}}%
\pgfpathcurveto{\pgfqpoint{2.885879in}{3.469946in}}{\pgfqpoint{2.889829in}{3.468310in}}{\pgfqpoint{2.893948in}{3.468310in}}%
\pgfpathclose%
\pgfusepath{fill}%
\end{pgfscope}%
\begin{pgfscope}%
\pgfpathrectangle{\pgfqpoint{0.887500in}{0.275000in}}{\pgfqpoint{4.225000in}{4.225000in}}%
\pgfusepath{clip}%
\pgfsetbuttcap%
\pgfsetroundjoin%
\definecolor{currentfill}{rgb}{0.000000,0.000000,0.000000}%
\pgfsetfillcolor{currentfill}%
\pgfsetfillopacity{0.800000}%
\pgfsetlinewidth{0.000000pt}%
\definecolor{currentstroke}{rgb}{0.000000,0.000000,0.000000}%
\pgfsetstrokecolor{currentstroke}%
\pgfsetstrokeopacity{0.800000}%
\pgfsetdash{}{0pt}%
\pgfpathmoveto{\pgfqpoint{1.937924in}{2.670223in}}%
\pgfpathcurveto{\pgfqpoint{1.942042in}{2.670223in}}{\pgfqpoint{1.945992in}{2.671859in}}{\pgfqpoint{1.948904in}{2.674771in}}%
\pgfpathcurveto{\pgfqpoint{1.951816in}{2.677683in}}{\pgfqpoint{1.953452in}{2.681633in}}{\pgfqpoint{1.953452in}{2.685751in}}%
\pgfpathcurveto{\pgfqpoint{1.953452in}{2.689869in}}{\pgfqpoint{1.951816in}{2.693819in}}{\pgfqpoint{1.948904in}{2.696731in}}%
\pgfpathcurveto{\pgfqpoint{1.945992in}{2.699643in}}{\pgfqpoint{1.942042in}{2.701279in}}{\pgfqpoint{1.937924in}{2.701279in}}%
\pgfpathcurveto{\pgfqpoint{1.933806in}{2.701279in}}{\pgfqpoint{1.929856in}{2.699643in}}{\pgfqpoint{1.926944in}{2.696731in}}%
\pgfpathcurveto{\pgfqpoint{1.924032in}{2.693819in}}{\pgfqpoint{1.922396in}{2.689869in}}{\pgfqpoint{1.922396in}{2.685751in}}%
\pgfpathcurveto{\pgfqpoint{1.922396in}{2.681633in}}{\pgfqpoint{1.924032in}{2.677683in}}{\pgfqpoint{1.926944in}{2.674771in}}%
\pgfpathcurveto{\pgfqpoint{1.929856in}{2.671859in}}{\pgfqpoint{1.933806in}{2.670223in}}{\pgfqpoint{1.937924in}{2.670223in}}%
\pgfpathclose%
\pgfusepath{fill}%
\end{pgfscope}%
\begin{pgfscope}%
\pgfpathrectangle{\pgfqpoint{0.887500in}{0.275000in}}{\pgfqpoint{4.225000in}{4.225000in}}%
\pgfusepath{clip}%
\pgfsetbuttcap%
\pgfsetroundjoin%
\definecolor{currentfill}{rgb}{0.000000,0.000000,0.000000}%
\pgfsetfillcolor{currentfill}%
\pgfsetfillopacity{0.800000}%
\pgfsetlinewidth{0.000000pt}%
\definecolor{currentstroke}{rgb}{0.000000,0.000000,0.000000}%
\pgfsetstrokecolor{currentstroke}%
\pgfsetstrokeopacity{0.800000}%
\pgfsetdash{}{0pt}%
\pgfpathmoveto{\pgfqpoint{1.474740in}{2.733018in}}%
\pgfpathcurveto{\pgfqpoint{1.478858in}{2.733018in}}{\pgfqpoint{1.482808in}{2.734654in}}{\pgfqpoint{1.485720in}{2.737566in}}%
\pgfpathcurveto{\pgfqpoint{1.488632in}{2.740478in}}{\pgfqpoint{1.490268in}{2.744428in}}{\pgfqpoint{1.490268in}{2.748546in}}%
\pgfpathcurveto{\pgfqpoint{1.490268in}{2.752665in}}{\pgfqpoint{1.488632in}{2.756615in}}{\pgfqpoint{1.485720in}{2.759527in}}%
\pgfpathcurveto{\pgfqpoint{1.482808in}{2.762439in}}{\pgfqpoint{1.478858in}{2.764075in}}{\pgfqpoint{1.474740in}{2.764075in}}%
\pgfpathcurveto{\pgfqpoint{1.470622in}{2.764075in}}{\pgfqpoint{1.466672in}{2.762439in}}{\pgfqpoint{1.463760in}{2.759527in}}%
\pgfpathcurveto{\pgfqpoint{1.460848in}{2.756615in}}{\pgfqpoint{1.459212in}{2.752665in}}{\pgfqpoint{1.459212in}{2.748546in}}%
\pgfpathcurveto{\pgfqpoint{1.459212in}{2.744428in}}{\pgfqpoint{1.460848in}{2.740478in}}{\pgfqpoint{1.463760in}{2.737566in}}%
\pgfpathcurveto{\pgfqpoint{1.466672in}{2.734654in}}{\pgfqpoint{1.470622in}{2.733018in}}{\pgfqpoint{1.474740in}{2.733018in}}%
\pgfpathclose%
\pgfusepath{fill}%
\end{pgfscope}%
\begin{pgfscope}%
\pgfpathrectangle{\pgfqpoint{0.887500in}{0.275000in}}{\pgfqpoint{4.225000in}{4.225000in}}%
\pgfusepath{clip}%
\pgfsetbuttcap%
\pgfsetroundjoin%
\definecolor{currentfill}{rgb}{0.000000,0.000000,0.000000}%
\pgfsetfillcolor{currentfill}%
\pgfsetfillopacity{0.800000}%
\pgfsetlinewidth{0.000000pt}%
\definecolor{currentstroke}{rgb}{0.000000,0.000000,0.000000}%
\pgfsetstrokecolor{currentstroke}%
\pgfsetstrokeopacity{0.800000}%
\pgfsetdash{}{0pt}%
\pgfpathmoveto{\pgfqpoint{2.688082in}{2.662573in}}%
\pgfpathcurveto{\pgfqpoint{2.692200in}{2.662573in}}{\pgfqpoint{2.696150in}{2.664209in}}{\pgfqpoint{2.699062in}{2.667121in}}%
\pgfpathcurveto{\pgfqpoint{2.701974in}{2.670033in}}{\pgfqpoint{2.703610in}{2.673983in}}{\pgfqpoint{2.703610in}{2.678101in}}%
\pgfpathcurveto{\pgfqpoint{2.703610in}{2.682219in}}{\pgfqpoint{2.701974in}{2.686170in}}{\pgfqpoint{2.699062in}{2.689081in}}%
\pgfpathcurveto{\pgfqpoint{2.696150in}{2.691993in}}{\pgfqpoint{2.692200in}{2.693630in}}{\pgfqpoint{2.688082in}{2.693630in}}%
\pgfpathcurveto{\pgfqpoint{2.683964in}{2.693630in}}{\pgfqpoint{2.680014in}{2.691993in}}{\pgfqpoint{2.677102in}{2.689081in}}%
\pgfpathcurveto{\pgfqpoint{2.674190in}{2.686170in}}{\pgfqpoint{2.672554in}{2.682219in}}{\pgfqpoint{2.672554in}{2.678101in}}%
\pgfpathcurveto{\pgfqpoint{2.672554in}{2.673983in}}{\pgfqpoint{2.674190in}{2.670033in}}{\pgfqpoint{2.677102in}{2.667121in}}%
\pgfpathcurveto{\pgfqpoint{2.680014in}{2.664209in}}{\pgfqpoint{2.683964in}{2.662573in}}{\pgfqpoint{2.688082in}{2.662573in}}%
\pgfpathclose%
\pgfusepath{fill}%
\end{pgfscope}%
\begin{pgfscope}%
\pgfpathrectangle{\pgfqpoint{0.887500in}{0.275000in}}{\pgfqpoint{4.225000in}{4.225000in}}%
\pgfusepath{clip}%
\pgfsetbuttcap%
\pgfsetroundjoin%
\definecolor{currentfill}{rgb}{0.000000,0.000000,0.000000}%
\pgfsetfillcolor{currentfill}%
\pgfsetfillopacity{0.800000}%
\pgfsetlinewidth{0.000000pt}%
\definecolor{currentstroke}{rgb}{0.000000,0.000000,0.000000}%
\pgfsetstrokecolor{currentstroke}%
\pgfsetstrokeopacity{0.800000}%
\pgfsetdash{}{0pt}%
\pgfpathmoveto{\pgfqpoint{3.357193in}{3.446530in}}%
\pgfpathcurveto{\pgfqpoint{3.361311in}{3.446530in}}{\pgfqpoint{3.365261in}{3.448167in}}{\pgfqpoint{3.368173in}{3.451079in}}%
\pgfpathcurveto{\pgfqpoint{3.371085in}{3.453990in}}{\pgfqpoint{3.372721in}{3.457940in}}{\pgfqpoint{3.372721in}{3.462059in}}%
\pgfpathcurveto{\pgfqpoint{3.372721in}{3.466177in}}{\pgfqpoint{3.371085in}{3.470127in}}{\pgfqpoint{3.368173in}{3.473039in}}%
\pgfpathcurveto{\pgfqpoint{3.365261in}{3.475951in}}{\pgfqpoint{3.361311in}{3.477587in}}{\pgfqpoint{3.357193in}{3.477587in}}%
\pgfpathcurveto{\pgfqpoint{3.353074in}{3.477587in}}{\pgfqpoint{3.349124in}{3.475951in}}{\pgfqpoint{3.346213in}{3.473039in}}%
\pgfpathcurveto{\pgfqpoint{3.343301in}{3.470127in}}{\pgfqpoint{3.341664in}{3.466177in}}{\pgfqpoint{3.341664in}{3.462059in}}%
\pgfpathcurveto{\pgfqpoint{3.341664in}{3.457940in}}{\pgfqpoint{3.343301in}{3.453990in}}{\pgfqpoint{3.346213in}{3.451079in}}%
\pgfpathcurveto{\pgfqpoint{3.349124in}{3.448167in}}{\pgfqpoint{3.353074in}{3.446530in}}{\pgfqpoint{3.357193in}{3.446530in}}%
\pgfpathclose%
\pgfusepath{fill}%
\end{pgfscope}%
\begin{pgfscope}%
\pgfpathrectangle{\pgfqpoint{0.887500in}{0.275000in}}{\pgfqpoint{4.225000in}{4.225000in}}%
\pgfusepath{clip}%
\pgfsetbuttcap%
\pgfsetroundjoin%
\definecolor{currentfill}{rgb}{0.000000,0.000000,0.000000}%
\pgfsetfillcolor{currentfill}%
\pgfsetfillopacity{0.800000}%
\pgfsetlinewidth{0.000000pt}%
\definecolor{currentstroke}{rgb}{0.000000,0.000000,0.000000}%
\pgfsetstrokecolor{currentstroke}%
\pgfsetstrokeopacity{0.800000}%
\pgfsetdash{}{0pt}%
\pgfpathmoveto{\pgfqpoint{2.576395in}{2.548880in}}%
\pgfpathcurveto{\pgfqpoint{2.580513in}{2.548880in}}{\pgfqpoint{2.584463in}{2.550516in}}{\pgfqpoint{2.587375in}{2.553428in}}%
\pgfpathcurveto{\pgfqpoint{2.590287in}{2.556340in}}{\pgfqpoint{2.591923in}{2.560290in}}{\pgfqpoint{2.591923in}{2.564408in}}%
\pgfpathcurveto{\pgfqpoint{2.591923in}{2.568526in}}{\pgfqpoint{2.590287in}{2.572476in}}{\pgfqpoint{2.587375in}{2.575388in}}%
\pgfpathcurveto{\pgfqpoint{2.584463in}{2.578300in}}{\pgfqpoint{2.580513in}{2.579936in}}{\pgfqpoint{2.576395in}{2.579936in}}%
\pgfpathcurveto{\pgfqpoint{2.572277in}{2.579936in}}{\pgfqpoint{2.568327in}{2.578300in}}{\pgfqpoint{2.565415in}{2.575388in}}%
\pgfpathcurveto{\pgfqpoint{2.562503in}{2.572476in}}{\pgfqpoint{2.560867in}{2.568526in}}{\pgfqpoint{2.560867in}{2.564408in}}%
\pgfpathcurveto{\pgfqpoint{2.560867in}{2.560290in}}{\pgfqpoint{2.562503in}{2.556340in}}{\pgfqpoint{2.565415in}{2.553428in}}%
\pgfpathcurveto{\pgfqpoint{2.568327in}{2.550516in}}{\pgfqpoint{2.572277in}{2.548880in}}{\pgfqpoint{2.576395in}{2.548880in}}%
\pgfpathclose%
\pgfusepath{fill}%
\end{pgfscope}%
\begin{pgfscope}%
\pgfpathrectangle{\pgfqpoint{0.887500in}{0.275000in}}{\pgfqpoint{4.225000in}{4.225000in}}%
\pgfusepath{clip}%
\pgfsetbuttcap%
\pgfsetroundjoin%
\definecolor{currentfill}{rgb}{0.000000,0.000000,0.000000}%
\pgfsetfillcolor{currentfill}%
\pgfsetfillopacity{0.800000}%
\pgfsetlinewidth{0.000000pt}%
\definecolor{currentstroke}{rgb}{0.000000,0.000000,0.000000}%
\pgfsetstrokecolor{currentstroke}%
\pgfsetstrokeopacity{0.800000}%
\pgfsetdash{}{0pt}%
\pgfpathmoveto{\pgfqpoint{2.112689in}{2.619736in}}%
\pgfpathcurveto{\pgfqpoint{2.116807in}{2.619736in}}{\pgfqpoint{2.120757in}{2.621372in}}{\pgfqpoint{2.123669in}{2.624284in}}%
\pgfpathcurveto{\pgfqpoint{2.126581in}{2.627196in}}{\pgfqpoint{2.128217in}{2.631146in}}{\pgfqpoint{2.128217in}{2.635265in}}%
\pgfpathcurveto{\pgfqpoint{2.128217in}{2.639383in}}{\pgfqpoint{2.126581in}{2.643333in}}{\pgfqpoint{2.123669in}{2.646245in}}%
\pgfpathcurveto{\pgfqpoint{2.120757in}{2.649157in}}{\pgfqpoint{2.116807in}{2.650793in}}{\pgfqpoint{2.112689in}{2.650793in}}%
\pgfpathcurveto{\pgfqpoint{2.108571in}{2.650793in}}{\pgfqpoint{2.104621in}{2.649157in}}{\pgfqpoint{2.101709in}{2.646245in}}%
\pgfpathcurveto{\pgfqpoint{2.098797in}{2.643333in}}{\pgfqpoint{2.097161in}{2.639383in}}{\pgfqpoint{2.097161in}{2.635265in}}%
\pgfpathcurveto{\pgfqpoint{2.097161in}{2.631146in}}{\pgfqpoint{2.098797in}{2.627196in}}{\pgfqpoint{2.101709in}{2.624284in}}%
\pgfpathcurveto{\pgfqpoint{2.104621in}{2.621372in}}{\pgfqpoint{2.108571in}{2.619736in}}{\pgfqpoint{2.112689in}{2.619736in}}%
\pgfpathclose%
\pgfusepath{fill}%
\end{pgfscope}%
\begin{pgfscope}%
\pgfpathrectangle{\pgfqpoint{0.887500in}{0.275000in}}{\pgfqpoint{4.225000in}{4.225000in}}%
\pgfusepath{clip}%
\pgfsetbuttcap%
\pgfsetroundjoin%
\definecolor{currentfill}{rgb}{0.000000,0.000000,0.000000}%
\pgfsetfillcolor{currentfill}%
\pgfsetfillopacity{0.800000}%
\pgfsetlinewidth{0.000000pt}%
\definecolor{currentstroke}{rgb}{0.000000,0.000000,0.000000}%
\pgfsetstrokecolor{currentstroke}%
\pgfsetstrokeopacity{0.800000}%
\pgfsetdash{}{0pt}%
\pgfpathmoveto{\pgfqpoint{3.533371in}{3.373284in}}%
\pgfpathcurveto{\pgfqpoint{3.537489in}{3.373284in}}{\pgfqpoint{3.541439in}{3.374920in}}{\pgfqpoint{3.544351in}{3.377832in}}%
\pgfpathcurveto{\pgfqpoint{3.547263in}{3.380744in}}{\pgfqpoint{3.548899in}{3.384694in}}{\pgfqpoint{3.548899in}{3.388812in}}%
\pgfpathcurveto{\pgfqpoint{3.548899in}{3.392931in}}{\pgfqpoint{3.547263in}{3.396881in}}{\pgfqpoint{3.544351in}{3.399793in}}%
\pgfpathcurveto{\pgfqpoint{3.541439in}{3.402705in}}{\pgfqpoint{3.537489in}{3.404341in}}{\pgfqpoint{3.533371in}{3.404341in}}%
\pgfpathcurveto{\pgfqpoint{3.529253in}{3.404341in}}{\pgfqpoint{3.525303in}{3.402705in}}{\pgfqpoint{3.522391in}{3.399793in}}%
\pgfpathcurveto{\pgfqpoint{3.519479in}{3.396881in}}{\pgfqpoint{3.517843in}{3.392931in}}{\pgfqpoint{3.517843in}{3.388812in}}%
\pgfpathcurveto{\pgfqpoint{3.517843in}{3.384694in}}{\pgfqpoint{3.519479in}{3.380744in}}{\pgfqpoint{3.522391in}{3.377832in}}%
\pgfpathcurveto{\pgfqpoint{3.525303in}{3.374920in}}{\pgfqpoint{3.529253in}{3.373284in}}{\pgfqpoint{3.533371in}{3.373284in}}%
\pgfpathclose%
\pgfusepath{fill}%
\end{pgfscope}%
\begin{pgfscope}%
\pgfpathrectangle{\pgfqpoint{0.887500in}{0.275000in}}{\pgfqpoint{4.225000in}{4.225000in}}%
\pgfusepath{clip}%
\pgfsetbuttcap%
\pgfsetroundjoin%
\definecolor{currentfill}{rgb}{0.000000,0.000000,0.000000}%
\pgfsetfillcolor{currentfill}%
\pgfsetfillopacity{0.800000}%
\pgfsetlinewidth{0.000000pt}%
\definecolor{currentstroke}{rgb}{0.000000,0.000000,0.000000}%
\pgfsetstrokecolor{currentstroke}%
\pgfsetstrokeopacity{0.800000}%
\pgfsetdash{}{0pt}%
\pgfpathmoveto{\pgfqpoint{1.648878in}{2.686272in}}%
\pgfpathcurveto{\pgfqpoint{1.652996in}{2.686272in}}{\pgfqpoint{1.656946in}{2.687908in}}{\pgfqpoint{1.659858in}{2.690820in}}%
\pgfpathcurveto{\pgfqpoint{1.662770in}{2.693732in}}{\pgfqpoint{1.664406in}{2.697682in}}{\pgfqpoint{1.664406in}{2.701801in}}%
\pgfpathcurveto{\pgfqpoint{1.664406in}{2.705919in}}{\pgfqpoint{1.662770in}{2.709869in}}{\pgfqpoint{1.659858in}{2.712781in}}%
\pgfpathcurveto{\pgfqpoint{1.656946in}{2.715693in}}{\pgfqpoint{1.652996in}{2.717329in}}{\pgfqpoint{1.648878in}{2.717329in}}%
\pgfpathcurveto{\pgfqpoint{1.644760in}{2.717329in}}{\pgfqpoint{1.640810in}{2.715693in}}{\pgfqpoint{1.637898in}{2.712781in}}%
\pgfpathcurveto{\pgfqpoint{1.634986in}{2.709869in}}{\pgfqpoint{1.633350in}{2.705919in}}{\pgfqpoint{1.633350in}{2.701801in}}%
\pgfpathcurveto{\pgfqpoint{1.633350in}{2.697682in}}{\pgfqpoint{1.634986in}{2.693732in}}{\pgfqpoint{1.637898in}{2.690820in}}%
\pgfpathcurveto{\pgfqpoint{1.640810in}{2.687908in}}{\pgfqpoint{1.644760in}{2.686272in}}{\pgfqpoint{1.648878in}{2.686272in}}%
\pgfpathclose%
\pgfusepath{fill}%
\end{pgfscope}%
\begin{pgfscope}%
\pgfpathrectangle{\pgfqpoint{0.887500in}{0.275000in}}{\pgfqpoint{4.225000in}{4.225000in}}%
\pgfusepath{clip}%
\pgfsetbuttcap%
\pgfsetroundjoin%
\definecolor{currentfill}{rgb}{0.000000,0.000000,0.000000}%
\pgfsetfillcolor{currentfill}%
\pgfsetfillopacity{0.800000}%
\pgfsetlinewidth{0.000000pt}%
\definecolor{currentstroke}{rgb}{0.000000,0.000000,0.000000}%
\pgfsetstrokecolor{currentstroke}%
\pgfsetstrokeopacity{0.800000}%
\pgfsetdash{}{0pt}%
\pgfpathmoveto{\pgfqpoint{4.237631in}{3.001747in}}%
\pgfpathcurveto{\pgfqpoint{4.241749in}{3.001747in}}{\pgfqpoint{4.245699in}{3.003383in}}{\pgfqpoint{4.248611in}{3.006295in}}%
\pgfpathcurveto{\pgfqpoint{4.251523in}{3.009207in}}{\pgfqpoint{4.253159in}{3.013157in}}{\pgfqpoint{4.253159in}{3.017276in}}%
\pgfpathcurveto{\pgfqpoint{4.253159in}{3.021394in}}{\pgfqpoint{4.251523in}{3.025344in}}{\pgfqpoint{4.248611in}{3.028256in}}%
\pgfpathcurveto{\pgfqpoint{4.245699in}{3.031168in}}{\pgfqpoint{4.241749in}{3.032804in}}{\pgfqpoint{4.237631in}{3.032804in}}%
\pgfpathcurveto{\pgfqpoint{4.233513in}{3.032804in}}{\pgfqpoint{4.229563in}{3.031168in}}{\pgfqpoint{4.226651in}{3.028256in}}%
\pgfpathcurveto{\pgfqpoint{4.223739in}{3.025344in}}{\pgfqpoint{4.222103in}{3.021394in}}{\pgfqpoint{4.222103in}{3.017276in}}%
\pgfpathcurveto{\pgfqpoint{4.222103in}{3.013157in}}{\pgfqpoint{4.223739in}{3.009207in}}{\pgfqpoint{4.226651in}{3.006295in}}%
\pgfpathcurveto{\pgfqpoint{4.229563in}{3.003383in}}{\pgfqpoint{4.233513in}{3.001747in}}{\pgfqpoint{4.237631in}{3.001747in}}%
\pgfpathclose%
\pgfusepath{fill}%
\end{pgfscope}%
\begin{pgfscope}%
\pgfpathrectangle{\pgfqpoint{0.887500in}{0.275000in}}{\pgfqpoint{4.225000in}{4.225000in}}%
\pgfusepath{clip}%
\pgfsetbuttcap%
\pgfsetroundjoin%
\definecolor{currentfill}{rgb}{0.000000,0.000000,0.000000}%
\pgfsetfillcolor{currentfill}%
\pgfsetfillopacity{0.800000}%
\pgfsetlinewidth{0.000000pt}%
\definecolor{currentstroke}{rgb}{0.000000,0.000000,0.000000}%
\pgfsetstrokecolor{currentstroke}%
\pgfsetstrokeopacity{0.800000}%
\pgfsetdash{}{0pt}%
\pgfpathmoveto{\pgfqpoint{4.061736in}{3.102222in}}%
\pgfpathcurveto{\pgfqpoint{4.065854in}{3.102222in}}{\pgfqpoint{4.069804in}{3.103858in}}{\pgfqpoint{4.072716in}{3.106770in}}%
\pgfpathcurveto{\pgfqpoint{4.075628in}{3.109682in}}{\pgfqpoint{4.077264in}{3.113632in}}{\pgfqpoint{4.077264in}{3.117750in}}%
\pgfpathcurveto{\pgfqpoint{4.077264in}{3.121868in}}{\pgfqpoint{4.075628in}{3.125818in}}{\pgfqpoint{4.072716in}{3.128730in}}%
\pgfpathcurveto{\pgfqpoint{4.069804in}{3.131642in}}{\pgfqpoint{4.065854in}{3.133278in}}{\pgfqpoint{4.061736in}{3.133278in}}%
\pgfpathcurveto{\pgfqpoint{4.057618in}{3.133278in}}{\pgfqpoint{4.053668in}{3.131642in}}{\pgfqpoint{4.050756in}{3.128730in}}%
\pgfpathcurveto{\pgfqpoint{4.047844in}{3.125818in}}{\pgfqpoint{4.046208in}{3.121868in}}{\pgfqpoint{4.046208in}{3.117750in}}%
\pgfpathcurveto{\pgfqpoint{4.046208in}{3.113632in}}{\pgfqpoint{4.047844in}{3.109682in}}{\pgfqpoint{4.050756in}{3.106770in}}%
\pgfpathcurveto{\pgfqpoint{4.053668in}{3.103858in}}{\pgfqpoint{4.057618in}{3.102222in}}{\pgfqpoint{4.061736in}{3.102222in}}%
\pgfpathclose%
\pgfusepath{fill}%
\end{pgfscope}%
\begin{pgfscope}%
\pgfpathrectangle{\pgfqpoint{0.887500in}{0.275000in}}{\pgfqpoint{4.225000in}{4.225000in}}%
\pgfusepath{clip}%
\pgfsetbuttcap%
\pgfsetroundjoin%
\definecolor{currentfill}{rgb}{0.000000,0.000000,0.000000}%
\pgfsetfillcolor{currentfill}%
\pgfsetfillopacity{0.800000}%
\pgfsetlinewidth{0.000000pt}%
\definecolor{currentstroke}{rgb}{0.000000,0.000000,0.000000}%
\pgfsetstrokecolor{currentstroke}%
\pgfsetstrokeopacity{0.800000}%
\pgfsetdash{}{0pt}%
\pgfpathmoveto{\pgfqpoint{3.709599in}{3.291714in}}%
\pgfpathcurveto{\pgfqpoint{3.713718in}{3.291714in}}{\pgfqpoint{3.717668in}{3.293350in}}{\pgfqpoint{3.720580in}{3.296262in}}%
\pgfpathcurveto{\pgfqpoint{3.723491in}{3.299174in}}{\pgfqpoint{3.725128in}{3.303124in}}{\pgfqpoint{3.725128in}{3.307242in}}%
\pgfpathcurveto{\pgfqpoint{3.725128in}{3.311360in}}{\pgfqpoint{3.723491in}{3.315310in}}{\pgfqpoint{3.720580in}{3.318222in}}%
\pgfpathcurveto{\pgfqpoint{3.717668in}{3.321134in}}{\pgfqpoint{3.713718in}{3.322770in}}{\pgfqpoint{3.709599in}{3.322770in}}%
\pgfpathcurveto{\pgfqpoint{3.705481in}{3.322770in}}{\pgfqpoint{3.701531in}{3.321134in}}{\pgfqpoint{3.698619in}{3.318222in}}%
\pgfpathcurveto{\pgfqpoint{3.695707in}{3.315310in}}{\pgfqpoint{3.694071in}{3.311360in}}{\pgfqpoint{3.694071in}{3.307242in}}%
\pgfpathcurveto{\pgfqpoint{3.694071in}{3.303124in}}{\pgfqpoint{3.695707in}{3.299174in}}{\pgfqpoint{3.698619in}{3.296262in}}%
\pgfpathcurveto{\pgfqpoint{3.701531in}{3.293350in}}{\pgfqpoint{3.705481in}{3.291714in}}{\pgfqpoint{3.709599in}{3.291714in}}%
\pgfpathclose%
\pgfusepath{fill}%
\end{pgfscope}%
\begin{pgfscope}%
\pgfpathrectangle{\pgfqpoint{0.887500in}{0.275000in}}{\pgfqpoint{4.225000in}{4.225000in}}%
\pgfusepath{clip}%
\pgfsetbuttcap%
\pgfsetroundjoin%
\definecolor{currentfill}{rgb}{0.000000,0.000000,0.000000}%
\pgfsetfillcolor{currentfill}%
\pgfsetfillopacity{0.800000}%
\pgfsetlinewidth{0.000000pt}%
\definecolor{currentstroke}{rgb}{0.000000,0.000000,0.000000}%
\pgfsetstrokecolor{currentstroke}%
\pgfsetstrokeopacity{0.800000}%
\pgfsetdash{}{0pt}%
\pgfpathmoveto{\pgfqpoint{3.885738in}{3.200301in}}%
\pgfpathcurveto{\pgfqpoint{3.889856in}{3.200301in}}{\pgfqpoint{3.893806in}{3.201937in}}{\pgfqpoint{3.896718in}{3.204849in}}%
\pgfpathcurveto{\pgfqpoint{3.899630in}{3.207761in}}{\pgfqpoint{3.901266in}{3.211711in}}{\pgfqpoint{3.901266in}{3.215829in}}%
\pgfpathcurveto{\pgfqpoint{3.901266in}{3.219948in}}{\pgfqpoint{3.899630in}{3.223898in}}{\pgfqpoint{3.896718in}{3.226810in}}%
\pgfpathcurveto{\pgfqpoint{3.893806in}{3.229721in}}{\pgfqpoint{3.889856in}{3.231358in}}{\pgfqpoint{3.885738in}{3.231358in}}%
\pgfpathcurveto{\pgfqpoint{3.881620in}{3.231358in}}{\pgfqpoint{3.877670in}{3.229721in}}{\pgfqpoint{3.874758in}{3.226810in}}%
\pgfpathcurveto{\pgfqpoint{3.871846in}{3.223898in}}{\pgfqpoint{3.870210in}{3.219948in}}{\pgfqpoint{3.870210in}{3.215829in}}%
\pgfpathcurveto{\pgfqpoint{3.870210in}{3.211711in}}{\pgfqpoint{3.871846in}{3.207761in}}{\pgfqpoint{3.874758in}{3.204849in}}%
\pgfpathcurveto{\pgfqpoint{3.877670in}{3.201937in}}{\pgfqpoint{3.881620in}{3.200301in}}{\pgfqpoint{3.885738in}{3.200301in}}%
\pgfpathclose%
\pgfusepath{fill}%
\end{pgfscope}%
\begin{pgfscope}%
\pgfpathrectangle{\pgfqpoint{0.887500in}{0.275000in}}{\pgfqpoint{4.225000in}{4.225000in}}%
\pgfusepath{clip}%
\pgfsetbuttcap%
\pgfsetroundjoin%
\definecolor{currentfill}{rgb}{0.000000,0.000000,0.000000}%
\pgfsetfillcolor{currentfill}%
\pgfsetfillopacity{0.800000}%
\pgfsetlinewidth{0.000000pt}%
\definecolor{currentstroke}{rgb}{0.000000,0.000000,0.000000}%
\pgfsetstrokecolor{currentstroke}%
\pgfsetstrokeopacity{0.800000}%
\pgfsetdash{}{0pt}%
\pgfpathmoveto{\pgfqpoint{2.287859in}{2.568115in}}%
\pgfpathcurveto{\pgfqpoint{2.291977in}{2.568115in}}{\pgfqpoint{2.295927in}{2.569751in}}{\pgfqpoint{2.298839in}{2.572663in}}%
\pgfpathcurveto{\pgfqpoint{2.301751in}{2.575575in}}{\pgfqpoint{2.303387in}{2.579525in}}{\pgfqpoint{2.303387in}{2.583643in}}%
\pgfpathcurveto{\pgfqpoint{2.303387in}{2.587761in}}{\pgfqpoint{2.301751in}{2.591711in}}{\pgfqpoint{2.298839in}{2.594623in}}%
\pgfpathcurveto{\pgfqpoint{2.295927in}{2.597535in}}{\pgfqpoint{2.291977in}{2.599171in}}{\pgfqpoint{2.287859in}{2.599171in}}%
\pgfpathcurveto{\pgfqpoint{2.283741in}{2.599171in}}{\pgfqpoint{2.279791in}{2.597535in}}{\pgfqpoint{2.276879in}{2.594623in}}%
\pgfpathcurveto{\pgfqpoint{2.273967in}{2.591711in}}{\pgfqpoint{2.272330in}{2.587761in}}{\pgfqpoint{2.272330in}{2.583643in}}%
\pgfpathcurveto{\pgfqpoint{2.272330in}{2.579525in}}{\pgfqpoint{2.273967in}{2.575575in}}{\pgfqpoint{2.276879in}{2.572663in}}%
\pgfpathcurveto{\pgfqpoint{2.279791in}{2.569751in}}{\pgfqpoint{2.283741in}{2.568115in}}{\pgfqpoint{2.287859in}{2.568115in}}%
\pgfpathclose%
\pgfusepath{fill}%
\end{pgfscope}%
\begin{pgfscope}%
\pgfpathrectangle{\pgfqpoint{0.887500in}{0.275000in}}{\pgfqpoint{4.225000in}{4.225000in}}%
\pgfusepath{clip}%
\pgfsetbuttcap%
\pgfsetroundjoin%
\definecolor{currentfill}{rgb}{0.000000,0.000000,0.000000}%
\pgfsetfillcolor{currentfill}%
\pgfsetfillopacity{0.800000}%
\pgfsetlinewidth{0.000000pt}%
\definecolor{currentstroke}{rgb}{0.000000,0.000000,0.000000}%
\pgfsetstrokecolor{currentstroke}%
\pgfsetstrokeopacity{0.800000}%
\pgfsetdash{}{0pt}%
\pgfpathmoveto{\pgfqpoint{1.823493in}{2.637449in}}%
\pgfpathcurveto{\pgfqpoint{1.827611in}{2.637449in}}{\pgfqpoint{1.831562in}{2.639085in}}{\pgfqpoint{1.834473in}{2.641997in}}%
\pgfpathcurveto{\pgfqpoint{1.837385in}{2.644909in}}{\pgfqpoint{1.839022in}{2.648859in}}{\pgfqpoint{1.839022in}{2.652977in}}%
\pgfpathcurveto{\pgfqpoint{1.839022in}{2.657095in}}{\pgfqpoint{1.837385in}{2.661045in}}{\pgfqpoint{1.834473in}{2.663957in}}%
\pgfpathcurveto{\pgfqpoint{1.831562in}{2.666869in}}{\pgfqpoint{1.827611in}{2.668505in}}{\pgfqpoint{1.823493in}{2.668505in}}%
\pgfpathcurveto{\pgfqpoint{1.819375in}{2.668505in}}{\pgfqpoint{1.815425in}{2.666869in}}{\pgfqpoint{1.812513in}{2.663957in}}%
\pgfpathcurveto{\pgfqpoint{1.809601in}{2.661045in}}{\pgfqpoint{1.807965in}{2.657095in}}{\pgfqpoint{1.807965in}{2.652977in}}%
\pgfpathcurveto{\pgfqpoint{1.807965in}{2.648859in}}{\pgfqpoint{1.809601in}{2.644909in}}{\pgfqpoint{1.812513in}{2.641997in}}%
\pgfpathcurveto{\pgfqpoint{1.815425in}{2.639085in}}{\pgfqpoint{1.819375in}{2.637449in}}{\pgfqpoint{1.823493in}{2.637449in}}%
\pgfpathclose%
\pgfusepath{fill}%
\end{pgfscope}%
\begin{pgfscope}%
\pgfpathrectangle{\pgfqpoint{0.887500in}{0.275000in}}{\pgfqpoint{4.225000in}{4.225000in}}%
\pgfusepath{clip}%
\pgfsetbuttcap%
\pgfsetroundjoin%
\definecolor{currentfill}{rgb}{0.000000,0.000000,0.000000}%
\pgfsetfillcolor{currentfill}%
\pgfsetfillopacity{0.800000}%
\pgfsetlinewidth{0.000000pt}%
\definecolor{currentstroke}{rgb}{0.000000,0.000000,0.000000}%
\pgfsetstrokecolor{currentstroke}%
\pgfsetstrokeopacity{0.800000}%
\pgfsetdash{}{0pt}%
\pgfpathmoveto{\pgfqpoint{1.359217in}{2.699537in}}%
\pgfpathcurveto{\pgfqpoint{1.363335in}{2.699537in}}{\pgfqpoint{1.367285in}{2.701173in}}{\pgfqpoint{1.370197in}{2.704085in}}%
\pgfpathcurveto{\pgfqpoint{1.373109in}{2.706997in}}{\pgfqpoint{1.374745in}{2.710947in}}{\pgfqpoint{1.374745in}{2.715065in}}%
\pgfpathcurveto{\pgfqpoint{1.374745in}{2.719183in}}{\pgfqpoint{1.373109in}{2.723133in}}{\pgfqpoint{1.370197in}{2.726045in}}%
\pgfpathcurveto{\pgfqpoint{1.367285in}{2.728957in}}{\pgfqpoint{1.363335in}{2.730594in}}{\pgfqpoint{1.359217in}{2.730594in}}%
\pgfpathcurveto{\pgfqpoint{1.355099in}{2.730594in}}{\pgfqpoint{1.351149in}{2.728957in}}{\pgfqpoint{1.348237in}{2.726045in}}%
\pgfpathcurveto{\pgfqpoint{1.345325in}{2.723133in}}{\pgfqpoint{1.343689in}{2.719183in}}{\pgfqpoint{1.343689in}{2.715065in}}%
\pgfpathcurveto{\pgfqpoint{1.343689in}{2.710947in}}{\pgfqpoint{1.345325in}{2.706997in}}{\pgfqpoint{1.348237in}{2.704085in}}%
\pgfpathcurveto{\pgfqpoint{1.351149in}{2.701173in}}{\pgfqpoint{1.355099in}{2.699537in}}{\pgfqpoint{1.359217in}{2.699537in}}%
\pgfpathclose%
\pgfusepath{fill}%
\end{pgfscope}%
\begin{pgfscope}%
\pgfpathrectangle{\pgfqpoint{0.887500in}{0.275000in}}{\pgfqpoint{4.225000in}{4.225000in}}%
\pgfusepath{clip}%
\pgfsetbuttcap%
\pgfsetroundjoin%
\definecolor{currentfill}{rgb}{0.000000,0.000000,0.000000}%
\pgfsetfillcolor{currentfill}%
\pgfsetfillopacity{0.800000}%
\pgfsetlinewidth{0.000000pt}%
\definecolor{currentstroke}{rgb}{0.000000,0.000000,0.000000}%
\pgfsetstrokecolor{currentstroke}%
\pgfsetstrokeopacity{0.800000}%
\pgfsetdash{}{0pt}%
\pgfpathmoveto{\pgfqpoint{2.910752in}{3.013093in}}%
\pgfpathcurveto{\pgfqpoint{2.914870in}{3.013093in}}{\pgfqpoint{2.918820in}{3.014729in}}{\pgfqpoint{2.921732in}{3.017641in}}%
\pgfpathcurveto{\pgfqpoint{2.924644in}{3.020553in}}{\pgfqpoint{2.926280in}{3.024503in}}{\pgfqpoint{2.926280in}{3.028621in}}%
\pgfpathcurveto{\pgfqpoint{2.926280in}{3.032739in}}{\pgfqpoint{2.924644in}{3.036689in}}{\pgfqpoint{2.921732in}{3.039601in}}%
\pgfpathcurveto{\pgfqpoint{2.918820in}{3.042513in}}{\pgfqpoint{2.914870in}{3.044149in}}{\pgfqpoint{2.910752in}{3.044149in}}%
\pgfpathcurveto{\pgfqpoint{2.906633in}{3.044149in}}{\pgfqpoint{2.902683in}{3.042513in}}{\pgfqpoint{2.899771in}{3.039601in}}%
\pgfpathcurveto{\pgfqpoint{2.896859in}{3.036689in}}{\pgfqpoint{2.895223in}{3.032739in}}{\pgfqpoint{2.895223in}{3.028621in}}%
\pgfpathcurveto{\pgfqpoint{2.895223in}{3.024503in}}{\pgfqpoint{2.896859in}{3.020553in}}{\pgfqpoint{2.899771in}{3.017641in}}%
\pgfpathcurveto{\pgfqpoint{2.902683in}{3.014729in}}{\pgfqpoint{2.906633in}{3.013093in}}{\pgfqpoint{2.910752in}{3.013093in}}%
\pgfpathclose%
\pgfusepath{fill}%
\end{pgfscope}%
\begin{pgfscope}%
\pgfpathrectangle{\pgfqpoint{0.887500in}{0.275000in}}{\pgfqpoint{4.225000in}{4.225000in}}%
\pgfusepath{clip}%
\pgfsetbuttcap%
\pgfsetroundjoin%
\definecolor{currentfill}{rgb}{0.000000,0.000000,0.000000}%
\pgfsetfillcolor{currentfill}%
\pgfsetfillopacity{0.800000}%
\pgfsetlinewidth{0.000000pt}%
\definecolor{currentstroke}{rgb}{0.000000,0.000000,0.000000}%
\pgfsetstrokecolor{currentstroke}%
\pgfsetstrokeopacity{0.800000}%
\pgfsetdash{}{0pt}%
\pgfpathmoveto{\pgfqpoint{3.246059in}{3.404015in}}%
\pgfpathcurveto{\pgfqpoint{3.250177in}{3.404015in}}{\pgfqpoint{3.254127in}{3.405651in}}{\pgfqpoint{3.257039in}{3.408563in}}%
\pgfpathcurveto{\pgfqpoint{3.259951in}{3.411475in}}{\pgfqpoint{3.261587in}{3.415425in}}{\pgfqpoint{3.261587in}{3.419543in}}%
\pgfpathcurveto{\pgfqpoint{3.261587in}{3.423661in}}{\pgfqpoint{3.259951in}{3.427611in}}{\pgfqpoint{3.257039in}{3.430523in}}%
\pgfpathcurveto{\pgfqpoint{3.254127in}{3.433435in}}{\pgfqpoint{3.250177in}{3.435072in}}{\pgfqpoint{3.246059in}{3.435072in}}%
\pgfpathcurveto{\pgfqpoint{3.241941in}{3.435072in}}{\pgfqpoint{3.237991in}{3.433435in}}{\pgfqpoint{3.235079in}{3.430523in}}%
\pgfpathcurveto{\pgfqpoint{3.232167in}{3.427611in}}{\pgfqpoint{3.230531in}{3.423661in}}{\pgfqpoint{3.230531in}{3.419543in}}%
\pgfpathcurveto{\pgfqpoint{3.230531in}{3.415425in}}{\pgfqpoint{3.232167in}{3.411475in}}{\pgfqpoint{3.235079in}{3.408563in}}%
\pgfpathcurveto{\pgfqpoint{3.237991in}{3.405651in}}{\pgfqpoint{3.241941in}{3.404015in}}{\pgfqpoint{3.246059in}{3.404015in}}%
\pgfpathclose%
\pgfusepath{fill}%
\end{pgfscope}%
\begin{pgfscope}%
\pgfpathrectangle{\pgfqpoint{0.887500in}{0.275000in}}{\pgfqpoint{4.225000in}{4.225000in}}%
\pgfusepath{clip}%
\pgfsetbuttcap%
\pgfsetroundjoin%
\definecolor{currentfill}{rgb}{0.000000,0.000000,0.000000}%
\pgfsetfillcolor{currentfill}%
\pgfsetfillopacity{0.800000}%
\pgfsetlinewidth{0.000000pt}%
\definecolor{currentstroke}{rgb}{0.000000,0.000000,0.000000}%
\pgfsetstrokecolor{currentstroke}%
\pgfsetstrokeopacity{0.800000}%
\pgfsetdash{}{0pt}%
\pgfpathmoveto{\pgfqpoint{2.463424in}{2.515188in}}%
\pgfpathcurveto{\pgfqpoint{2.467542in}{2.515188in}}{\pgfqpoint{2.471492in}{2.516824in}}{\pgfqpoint{2.474404in}{2.519736in}}%
\pgfpathcurveto{\pgfqpoint{2.477316in}{2.522648in}}{\pgfqpoint{2.478952in}{2.526598in}}{\pgfqpoint{2.478952in}{2.530716in}}%
\pgfpathcurveto{\pgfqpoint{2.478952in}{2.534834in}}{\pgfqpoint{2.477316in}{2.538784in}}{\pgfqpoint{2.474404in}{2.541696in}}%
\pgfpathcurveto{\pgfqpoint{2.471492in}{2.544608in}}{\pgfqpoint{2.467542in}{2.546244in}}{\pgfqpoint{2.463424in}{2.546244in}}%
\pgfpathcurveto{\pgfqpoint{2.459306in}{2.546244in}}{\pgfqpoint{2.455356in}{2.544608in}}{\pgfqpoint{2.452444in}{2.541696in}}%
\pgfpathcurveto{\pgfqpoint{2.449532in}{2.538784in}}{\pgfqpoint{2.447896in}{2.534834in}}{\pgfqpoint{2.447896in}{2.530716in}}%
\pgfpathcurveto{\pgfqpoint{2.447896in}{2.526598in}}{\pgfqpoint{2.449532in}{2.522648in}}{\pgfqpoint{2.452444in}{2.519736in}}%
\pgfpathcurveto{\pgfqpoint{2.455356in}{2.516824in}}{\pgfqpoint{2.459306in}{2.515188in}}{\pgfqpoint{2.463424in}{2.515188in}}%
\pgfpathclose%
\pgfusepath{fill}%
\end{pgfscope}%
\begin{pgfscope}%
\pgfpathrectangle{\pgfqpoint{0.887500in}{0.275000in}}{\pgfqpoint{4.225000in}{4.225000in}}%
\pgfusepath{clip}%
\pgfsetbuttcap%
\pgfsetroundjoin%
\definecolor{currentfill}{rgb}{0.000000,0.000000,0.000000}%
\pgfsetfillcolor{currentfill}%
\pgfsetfillopacity{0.800000}%
\pgfsetlinewidth{0.000000pt}%
\definecolor{currentstroke}{rgb}{0.000000,0.000000,0.000000}%
\pgfsetstrokecolor{currentstroke}%
\pgfsetstrokeopacity{0.800000}%
\pgfsetdash{}{0pt}%
\pgfpathmoveto{\pgfqpoint{2.751594in}{2.551925in}}%
\pgfpathcurveto{\pgfqpoint{2.755712in}{2.551925in}}{\pgfqpoint{2.759662in}{2.553561in}}{\pgfqpoint{2.762574in}{2.556473in}}%
\pgfpathcurveto{\pgfqpoint{2.765486in}{2.559385in}}{\pgfqpoint{2.767122in}{2.563335in}}{\pgfqpoint{2.767122in}{2.567453in}}%
\pgfpathcurveto{\pgfqpoint{2.767122in}{2.571571in}}{\pgfqpoint{2.765486in}{2.575522in}}{\pgfqpoint{2.762574in}{2.578433in}}%
\pgfpathcurveto{\pgfqpoint{2.759662in}{2.581345in}}{\pgfqpoint{2.755712in}{2.582982in}}{\pgfqpoint{2.751594in}{2.582982in}}%
\pgfpathcurveto{\pgfqpoint{2.747476in}{2.582982in}}{\pgfqpoint{2.743526in}{2.581345in}}{\pgfqpoint{2.740614in}{2.578433in}}%
\pgfpathcurveto{\pgfqpoint{2.737702in}{2.575522in}}{\pgfqpoint{2.736066in}{2.571571in}}{\pgfqpoint{2.736066in}{2.567453in}}%
\pgfpathcurveto{\pgfqpoint{2.736066in}{2.563335in}}{\pgfqpoint{2.737702in}{2.559385in}}{\pgfqpoint{2.740614in}{2.556473in}}%
\pgfpathcurveto{\pgfqpoint{2.743526in}{2.553561in}}{\pgfqpoint{2.747476in}{2.551925in}}{\pgfqpoint{2.751594in}{2.551925in}}%
\pgfpathclose%
\pgfusepath{fill}%
\end{pgfscope}%
\begin{pgfscope}%
\pgfpathrectangle{\pgfqpoint{0.887500in}{0.275000in}}{\pgfqpoint{4.225000in}{4.225000in}}%
\pgfusepath{clip}%
\pgfsetbuttcap%
\pgfsetroundjoin%
\definecolor{currentfill}{rgb}{0.000000,0.000000,0.000000}%
\pgfsetfillcolor{currentfill}%
\pgfsetfillopacity{0.800000}%
\pgfsetlinewidth{0.000000pt}%
\definecolor{currentstroke}{rgb}{0.000000,0.000000,0.000000}%
\pgfsetstrokecolor{currentstroke}%
\pgfsetstrokeopacity{0.800000}%
\pgfsetdash{}{0pt}%
\pgfpathmoveto{\pgfqpoint{4.304520in}{2.857087in}}%
\pgfpathcurveto{\pgfqpoint{4.308638in}{2.857087in}}{\pgfqpoint{4.312588in}{2.858723in}}{\pgfqpoint{4.315500in}{2.861635in}}%
\pgfpathcurveto{\pgfqpoint{4.318412in}{2.864547in}}{\pgfqpoint{4.320048in}{2.868497in}}{\pgfqpoint{4.320048in}{2.872615in}}%
\pgfpathcurveto{\pgfqpoint{4.320048in}{2.876733in}}{\pgfqpoint{4.318412in}{2.880683in}}{\pgfqpoint{4.315500in}{2.883595in}}%
\pgfpathcurveto{\pgfqpoint{4.312588in}{2.886507in}}{\pgfqpoint{4.308638in}{2.888143in}}{\pgfqpoint{4.304520in}{2.888143in}}%
\pgfpathcurveto{\pgfqpoint{4.300401in}{2.888143in}}{\pgfqpoint{4.296451in}{2.886507in}}{\pgfqpoint{4.293539in}{2.883595in}}%
\pgfpathcurveto{\pgfqpoint{4.290627in}{2.880683in}}{\pgfqpoint{4.288991in}{2.876733in}}{\pgfqpoint{4.288991in}{2.872615in}}%
\pgfpathcurveto{\pgfqpoint{4.288991in}{2.868497in}}{\pgfqpoint{4.290627in}{2.864547in}}{\pgfqpoint{4.293539in}{2.861635in}}%
\pgfpathcurveto{\pgfqpoint{4.296451in}{2.858723in}}{\pgfqpoint{4.300401in}{2.857087in}}{\pgfqpoint{4.304520in}{2.857087in}}%
\pgfpathclose%
\pgfusepath{fill}%
\end{pgfscope}%
\begin{pgfscope}%
\pgfpathrectangle{\pgfqpoint{0.887500in}{0.275000in}}{\pgfqpoint{4.225000in}{4.225000in}}%
\pgfusepath{clip}%
\pgfsetbuttcap%
\pgfsetroundjoin%
\definecolor{currentfill}{rgb}{0.000000,0.000000,0.000000}%
\pgfsetfillcolor{currentfill}%
\pgfsetfillopacity{0.800000}%
\pgfsetlinewidth{0.000000pt}%
\definecolor{currentstroke}{rgb}{0.000000,0.000000,0.000000}%
\pgfsetstrokecolor{currentstroke}%
\pgfsetstrokeopacity{0.800000}%
\pgfsetdash{}{0pt}%
\pgfpathmoveto{\pgfqpoint{1.998557in}{2.586718in}}%
\pgfpathcurveto{\pgfqpoint{2.002675in}{2.586718in}}{\pgfqpoint{2.006625in}{2.588354in}}{\pgfqpoint{2.009537in}{2.591266in}}%
\pgfpathcurveto{\pgfqpoint{2.012449in}{2.594178in}}{\pgfqpoint{2.014085in}{2.598128in}}{\pgfqpoint{2.014085in}{2.602246in}}%
\pgfpathcurveto{\pgfqpoint{2.014085in}{2.606365in}}{\pgfqpoint{2.012449in}{2.610315in}}{\pgfqpoint{2.009537in}{2.613226in}}%
\pgfpathcurveto{\pgfqpoint{2.006625in}{2.616138in}}{\pgfqpoint{2.002675in}{2.617775in}}{\pgfqpoint{1.998557in}{2.617775in}}%
\pgfpathcurveto{\pgfqpoint{1.994439in}{2.617775in}}{\pgfqpoint{1.990489in}{2.616138in}}{\pgfqpoint{1.987577in}{2.613226in}}%
\pgfpathcurveto{\pgfqpoint{1.984665in}{2.610315in}}{\pgfqpoint{1.983029in}{2.606365in}}{\pgfqpoint{1.983029in}{2.602246in}}%
\pgfpathcurveto{\pgfqpoint{1.983029in}{2.598128in}}{\pgfqpoint{1.984665in}{2.594178in}}{\pgfqpoint{1.987577in}{2.591266in}}%
\pgfpathcurveto{\pgfqpoint{1.990489in}{2.588354in}}{\pgfqpoint{1.994439in}{2.586718in}}{\pgfqpoint{1.998557in}{2.586718in}}%
\pgfpathclose%
\pgfusepath{fill}%
\end{pgfscope}%
\begin{pgfscope}%
\pgfpathrectangle{\pgfqpoint{0.887500in}{0.275000in}}{\pgfqpoint{4.225000in}{4.225000in}}%
\pgfusepath{clip}%
\pgfsetbuttcap%
\pgfsetroundjoin%
\definecolor{currentfill}{rgb}{0.000000,0.000000,0.000000}%
\pgfsetfillcolor{currentfill}%
\pgfsetfillopacity{0.800000}%
\pgfsetlinewidth{0.000000pt}%
\definecolor{currentstroke}{rgb}{0.000000,0.000000,0.000000}%
\pgfsetstrokecolor{currentstroke}%
\pgfsetstrokeopacity{0.800000}%
\pgfsetdash{}{0pt}%
\pgfpathmoveto{\pgfqpoint{3.422618in}{3.337556in}}%
\pgfpathcurveto{\pgfqpoint{3.426737in}{3.337556in}}{\pgfqpoint{3.430687in}{3.339192in}}{\pgfqpoint{3.433599in}{3.342104in}}%
\pgfpathcurveto{\pgfqpoint{3.436511in}{3.345016in}}{\pgfqpoint{3.438147in}{3.348966in}}{\pgfqpoint{3.438147in}{3.353084in}}%
\pgfpathcurveto{\pgfqpoint{3.438147in}{3.357202in}}{\pgfqpoint{3.436511in}{3.361152in}}{\pgfqpoint{3.433599in}{3.364064in}}%
\pgfpathcurveto{\pgfqpoint{3.430687in}{3.366976in}}{\pgfqpoint{3.426737in}{3.368612in}}{\pgfqpoint{3.422618in}{3.368612in}}%
\pgfpathcurveto{\pgfqpoint{3.418500in}{3.368612in}}{\pgfqpoint{3.414550in}{3.366976in}}{\pgfqpoint{3.411638in}{3.364064in}}%
\pgfpathcurveto{\pgfqpoint{3.408726in}{3.361152in}}{\pgfqpoint{3.407090in}{3.357202in}}{\pgfqpoint{3.407090in}{3.353084in}}%
\pgfpathcurveto{\pgfqpoint{3.407090in}{3.348966in}}{\pgfqpoint{3.408726in}{3.345016in}}{\pgfqpoint{3.411638in}{3.342104in}}%
\pgfpathcurveto{\pgfqpoint{3.414550in}{3.339192in}}{\pgfqpoint{3.418500in}{3.337556in}}{\pgfqpoint{3.422618in}{3.337556in}}%
\pgfpathclose%
\pgfusepath{fill}%
\end{pgfscope}%
\begin{pgfscope}%
\pgfpathrectangle{\pgfqpoint{0.887500in}{0.275000in}}{\pgfqpoint{4.225000in}{4.225000in}}%
\pgfusepath{clip}%
\pgfsetbuttcap%
\pgfsetroundjoin%
\definecolor{currentfill}{rgb}{0.000000,0.000000,0.000000}%
\pgfsetfillcolor{currentfill}%
\pgfsetfillopacity{0.800000}%
\pgfsetlinewidth{0.000000pt}%
\definecolor{currentstroke}{rgb}{0.000000,0.000000,0.000000}%
\pgfsetstrokecolor{currentstroke}%
\pgfsetstrokeopacity{0.800000}%
\pgfsetdash{}{0pt}%
\pgfpathmoveto{\pgfqpoint{1.533592in}{2.653752in}}%
\pgfpathcurveto{\pgfqpoint{1.537710in}{2.653752in}}{\pgfqpoint{1.541660in}{2.655388in}}{\pgfqpoint{1.544572in}{2.658300in}}%
\pgfpathcurveto{\pgfqpoint{1.547484in}{2.661212in}}{\pgfqpoint{1.549120in}{2.665162in}}{\pgfqpoint{1.549120in}{2.669280in}}%
\pgfpathcurveto{\pgfqpoint{1.549120in}{2.673398in}}{\pgfqpoint{1.547484in}{2.677348in}}{\pgfqpoint{1.544572in}{2.680260in}}%
\pgfpathcurveto{\pgfqpoint{1.541660in}{2.683172in}}{\pgfqpoint{1.537710in}{2.684808in}}{\pgfqpoint{1.533592in}{2.684808in}}%
\pgfpathcurveto{\pgfqpoint{1.529474in}{2.684808in}}{\pgfqpoint{1.525524in}{2.683172in}}{\pgfqpoint{1.522612in}{2.680260in}}%
\pgfpathcurveto{\pgfqpoint{1.519700in}{2.677348in}}{\pgfqpoint{1.518064in}{2.673398in}}{\pgfqpoint{1.518064in}{2.669280in}}%
\pgfpathcurveto{\pgfqpoint{1.518064in}{2.665162in}}{\pgfqpoint{1.519700in}{2.661212in}}{\pgfqpoint{1.522612in}{2.658300in}}%
\pgfpathcurveto{\pgfqpoint{1.525524in}{2.655388in}}{\pgfqpoint{1.529474in}{2.653752in}}{\pgfqpoint{1.533592in}{2.653752in}}%
\pgfpathclose%
\pgfusepath{fill}%
\end{pgfscope}%
\begin{pgfscope}%
\pgfpathrectangle{\pgfqpoint{0.887500in}{0.275000in}}{\pgfqpoint{4.225000in}{4.225000in}}%
\pgfusepath{clip}%
\pgfsetbuttcap%
\pgfsetroundjoin%
\definecolor{currentfill}{rgb}{0.000000,0.000000,0.000000}%
\pgfsetfillcolor{currentfill}%
\pgfsetfillopacity{0.800000}%
\pgfsetlinewidth{0.000000pt}%
\definecolor{currentstroke}{rgb}{0.000000,0.000000,0.000000}%
\pgfsetstrokecolor{currentstroke}%
\pgfsetstrokeopacity{0.800000}%
\pgfsetdash{}{0pt}%
\pgfpathmoveto{\pgfqpoint{4.128547in}{2.967400in}}%
\pgfpathcurveto{\pgfqpoint{4.132665in}{2.967400in}}{\pgfqpoint{4.136615in}{2.969036in}}{\pgfqpoint{4.139527in}{2.971948in}}%
\pgfpathcurveto{\pgfqpoint{4.142439in}{2.974860in}}{\pgfqpoint{4.144075in}{2.978810in}}{\pgfqpoint{4.144075in}{2.982928in}}%
\pgfpathcurveto{\pgfqpoint{4.144075in}{2.987046in}}{\pgfqpoint{4.142439in}{2.990996in}}{\pgfqpoint{4.139527in}{2.993908in}}%
\pgfpathcurveto{\pgfqpoint{4.136615in}{2.996820in}}{\pgfqpoint{4.132665in}{2.998456in}}{\pgfqpoint{4.128547in}{2.998456in}}%
\pgfpathcurveto{\pgfqpoint{4.124429in}{2.998456in}}{\pgfqpoint{4.120479in}{2.996820in}}{\pgfqpoint{4.117567in}{2.993908in}}%
\pgfpathcurveto{\pgfqpoint{4.114655in}{2.990996in}}{\pgfqpoint{4.113019in}{2.987046in}}{\pgfqpoint{4.113019in}{2.982928in}}%
\pgfpathcurveto{\pgfqpoint{4.113019in}{2.978810in}}{\pgfqpoint{4.114655in}{2.974860in}}{\pgfqpoint{4.117567in}{2.971948in}}%
\pgfpathcurveto{\pgfqpoint{4.120479in}{2.969036in}}{\pgfqpoint{4.124429in}{2.967400in}}{\pgfqpoint{4.128547in}{2.967400in}}%
\pgfpathclose%
\pgfusepath{fill}%
\end{pgfscope}%
\begin{pgfscope}%
\pgfpathrectangle{\pgfqpoint{0.887500in}{0.275000in}}{\pgfqpoint{4.225000in}{4.225000in}}%
\pgfusepath{clip}%
\pgfsetbuttcap%
\pgfsetroundjoin%
\definecolor{currentfill}{rgb}{0.000000,0.000000,0.000000}%
\pgfsetfillcolor{currentfill}%
\pgfsetfillopacity{0.800000}%
\pgfsetlinewidth{0.000000pt}%
\definecolor{currentstroke}{rgb}{0.000000,0.000000,0.000000}%
\pgfsetstrokecolor{currentstroke}%
\pgfsetstrokeopacity{0.800000}%
\pgfsetdash{}{0pt}%
\pgfpathmoveto{\pgfqpoint{3.599238in}{3.257263in}}%
\pgfpathcurveto{\pgfqpoint{3.603356in}{3.257263in}}{\pgfqpoint{3.607306in}{3.258899in}}{\pgfqpoint{3.610218in}{3.261811in}}%
\pgfpathcurveto{\pgfqpoint{3.613130in}{3.264723in}}{\pgfqpoint{3.614767in}{3.268673in}}{\pgfqpoint{3.614767in}{3.272791in}}%
\pgfpathcurveto{\pgfqpoint{3.614767in}{3.276910in}}{\pgfqpoint{3.613130in}{3.280860in}}{\pgfqpoint{3.610218in}{3.283772in}}%
\pgfpathcurveto{\pgfqpoint{3.607306in}{3.286683in}}{\pgfqpoint{3.603356in}{3.288320in}}{\pgfqpoint{3.599238in}{3.288320in}}%
\pgfpathcurveto{\pgfqpoint{3.595120in}{3.288320in}}{\pgfqpoint{3.591170in}{3.286683in}}{\pgfqpoint{3.588258in}{3.283772in}}%
\pgfpathcurveto{\pgfqpoint{3.585346in}{3.280860in}}{\pgfqpoint{3.583710in}{3.276910in}}{\pgfqpoint{3.583710in}{3.272791in}}%
\pgfpathcurveto{\pgfqpoint{3.583710in}{3.268673in}}{\pgfqpoint{3.585346in}{3.264723in}}{\pgfqpoint{3.588258in}{3.261811in}}%
\pgfpathcurveto{\pgfqpoint{3.591170in}{3.258899in}}{\pgfqpoint{3.595120in}{3.257263in}}{\pgfqpoint{3.599238in}{3.257263in}}%
\pgfpathclose%
\pgfusepath{fill}%
\end{pgfscope}%
\begin{pgfscope}%
\pgfpathrectangle{\pgfqpoint{0.887500in}{0.275000in}}{\pgfqpoint{4.225000in}{4.225000in}}%
\pgfusepath{clip}%
\pgfsetbuttcap%
\pgfsetroundjoin%
\definecolor{currentfill}{rgb}{0.000000,0.000000,0.000000}%
\pgfsetfillcolor{currentfill}%
\pgfsetfillopacity{0.800000}%
\pgfsetlinewidth{0.000000pt}%
\definecolor{currentstroke}{rgb}{0.000000,0.000000,0.000000}%
\pgfsetstrokecolor{currentstroke}%
\pgfsetstrokeopacity{0.800000}%
\pgfsetdash{}{0pt}%
\pgfpathmoveto{\pgfqpoint{3.775771in}{3.165027in}}%
\pgfpathcurveto{\pgfqpoint{3.779889in}{3.165027in}}{\pgfqpoint{3.783839in}{3.166663in}}{\pgfqpoint{3.786751in}{3.169575in}}%
\pgfpathcurveto{\pgfqpoint{3.789663in}{3.172487in}}{\pgfqpoint{3.791299in}{3.176437in}}{\pgfqpoint{3.791299in}{3.180555in}}%
\pgfpathcurveto{\pgfqpoint{3.791299in}{3.184673in}}{\pgfqpoint{3.789663in}{3.188623in}}{\pgfqpoint{3.786751in}{3.191535in}}%
\pgfpathcurveto{\pgfqpoint{3.783839in}{3.194447in}}{\pgfqpoint{3.779889in}{3.196083in}}{\pgfqpoint{3.775771in}{3.196083in}}%
\pgfpathcurveto{\pgfqpoint{3.771653in}{3.196083in}}{\pgfqpoint{3.767703in}{3.194447in}}{\pgfqpoint{3.764791in}{3.191535in}}%
\pgfpathcurveto{\pgfqpoint{3.761879in}{3.188623in}}{\pgfqpoint{3.760243in}{3.184673in}}{\pgfqpoint{3.760243in}{3.180555in}}%
\pgfpathcurveto{\pgfqpoint{3.760243in}{3.176437in}}{\pgfqpoint{3.761879in}{3.172487in}}{\pgfqpoint{3.764791in}{3.169575in}}%
\pgfpathcurveto{\pgfqpoint{3.767703in}{3.166663in}}{\pgfqpoint{3.771653in}{3.165027in}}{\pgfqpoint{3.775771in}{3.165027in}}%
\pgfpathclose%
\pgfusepath{fill}%
\end{pgfscope}%
\begin{pgfscope}%
\pgfpathrectangle{\pgfqpoint{0.887500in}{0.275000in}}{\pgfqpoint{4.225000in}{4.225000in}}%
\pgfusepath{clip}%
\pgfsetbuttcap%
\pgfsetroundjoin%
\definecolor{currentfill}{rgb}{0.000000,0.000000,0.000000}%
\pgfsetfillcolor{currentfill}%
\pgfsetfillopacity{0.800000}%
\pgfsetlinewidth{0.000000pt}%
\definecolor{currentstroke}{rgb}{0.000000,0.000000,0.000000}%
\pgfsetstrokecolor{currentstroke}%
\pgfsetstrokeopacity{0.800000}%
\pgfsetdash{}{0pt}%
\pgfpathmoveto{\pgfqpoint{3.952312in}{3.072260in}}%
\pgfpathcurveto{\pgfqpoint{3.956431in}{3.072260in}}{\pgfqpoint{3.960381in}{3.073896in}}{\pgfqpoint{3.963293in}{3.076808in}}%
\pgfpathcurveto{\pgfqpoint{3.966205in}{3.079720in}}{\pgfqpoint{3.967841in}{3.083670in}}{\pgfqpoint{3.967841in}{3.087789in}}%
\pgfpathcurveto{\pgfqpoint{3.967841in}{3.091907in}}{\pgfqpoint{3.966205in}{3.095857in}}{\pgfqpoint{3.963293in}{3.098769in}}%
\pgfpathcurveto{\pgfqpoint{3.960381in}{3.101681in}}{\pgfqpoint{3.956431in}{3.103317in}}{\pgfqpoint{3.952312in}{3.103317in}}%
\pgfpathcurveto{\pgfqpoint{3.948194in}{3.103317in}}{\pgfqpoint{3.944244in}{3.101681in}}{\pgfqpoint{3.941332in}{3.098769in}}%
\pgfpathcurveto{\pgfqpoint{3.938420in}{3.095857in}}{\pgfqpoint{3.936784in}{3.091907in}}{\pgfqpoint{3.936784in}{3.087789in}}%
\pgfpathcurveto{\pgfqpoint{3.936784in}{3.083670in}}{\pgfqpoint{3.938420in}{3.079720in}}{\pgfqpoint{3.941332in}{3.076808in}}%
\pgfpathcurveto{\pgfqpoint{3.944244in}{3.073896in}}{\pgfqpoint{3.948194in}{3.072260in}}{\pgfqpoint{3.952312in}{3.072260in}}%
\pgfpathclose%
\pgfusepath{fill}%
\end{pgfscope}%
\begin{pgfscope}%
\pgfpathrectangle{\pgfqpoint{0.887500in}{0.275000in}}{\pgfqpoint{4.225000in}{4.225000in}}%
\pgfusepath{clip}%
\pgfsetbuttcap%
\pgfsetroundjoin%
\definecolor{currentfill}{rgb}{0.000000,0.000000,0.000000}%
\pgfsetfillcolor{currentfill}%
\pgfsetfillopacity{0.800000}%
\pgfsetlinewidth{0.000000pt}%
\definecolor{currentstroke}{rgb}{0.000000,0.000000,0.000000}%
\pgfsetstrokecolor{currentstroke}%
\pgfsetstrokeopacity{0.800000}%
\pgfsetdash{}{0pt}%
\pgfpathmoveto{\pgfqpoint{2.957855in}{3.420085in}}%
\pgfpathcurveto{\pgfqpoint{2.961973in}{3.420085in}}{\pgfqpoint{2.965923in}{3.421721in}}{\pgfqpoint{2.968835in}{3.424633in}}%
\pgfpathcurveto{\pgfqpoint{2.971747in}{3.427545in}}{\pgfqpoint{2.973383in}{3.431495in}}{\pgfqpoint{2.973383in}{3.435613in}}%
\pgfpathcurveto{\pgfqpoint{2.973383in}{3.439732in}}{\pgfqpoint{2.971747in}{3.443682in}}{\pgfqpoint{2.968835in}{3.446594in}}%
\pgfpathcurveto{\pgfqpoint{2.965923in}{3.449506in}}{\pgfqpoint{2.961973in}{3.451142in}}{\pgfqpoint{2.957855in}{3.451142in}}%
\pgfpathcurveto{\pgfqpoint{2.953737in}{3.451142in}}{\pgfqpoint{2.949787in}{3.449506in}}{\pgfqpoint{2.946875in}{3.446594in}}%
\pgfpathcurveto{\pgfqpoint{2.943963in}{3.443682in}}{\pgfqpoint{2.942327in}{3.439732in}}{\pgfqpoint{2.942327in}{3.435613in}}%
\pgfpathcurveto{\pgfqpoint{2.942327in}{3.431495in}}{\pgfqpoint{2.943963in}{3.427545in}}{\pgfqpoint{2.946875in}{3.424633in}}%
\pgfpathcurveto{\pgfqpoint{2.949787in}{3.421721in}}{\pgfqpoint{2.953737in}{3.420085in}}{\pgfqpoint{2.957855in}{3.420085in}}%
\pgfpathclose%
\pgfusepath{fill}%
\end{pgfscope}%
\begin{pgfscope}%
\pgfpathrectangle{\pgfqpoint{0.887500in}{0.275000in}}{\pgfqpoint{4.225000in}{4.225000in}}%
\pgfusepath{clip}%
\pgfsetbuttcap%
\pgfsetroundjoin%
\definecolor{currentfill}{rgb}{0.000000,0.000000,0.000000}%
\pgfsetfillcolor{currentfill}%
\pgfsetfillopacity{0.800000}%
\pgfsetlinewidth{0.000000pt}%
\definecolor{currentstroke}{rgb}{0.000000,0.000000,0.000000}%
\pgfsetstrokecolor{currentstroke}%
\pgfsetstrokeopacity{0.800000}%
\pgfsetdash{}{0pt}%
\pgfpathmoveto{\pgfqpoint{2.639376in}{2.460445in}}%
\pgfpathcurveto{\pgfqpoint{2.643494in}{2.460445in}}{\pgfqpoint{2.647444in}{2.462081in}}{\pgfqpoint{2.650356in}{2.464993in}}%
\pgfpathcurveto{\pgfqpoint{2.653268in}{2.467905in}}{\pgfqpoint{2.654904in}{2.471855in}}{\pgfqpoint{2.654904in}{2.475973in}}%
\pgfpathcurveto{\pgfqpoint{2.654904in}{2.480091in}}{\pgfqpoint{2.653268in}{2.484041in}}{\pgfqpoint{2.650356in}{2.486953in}}%
\pgfpathcurveto{\pgfqpoint{2.647444in}{2.489865in}}{\pgfqpoint{2.643494in}{2.491501in}}{\pgfqpoint{2.639376in}{2.491501in}}%
\pgfpathcurveto{\pgfqpoint{2.635258in}{2.491501in}}{\pgfqpoint{2.631308in}{2.489865in}}{\pgfqpoint{2.628396in}{2.486953in}}%
\pgfpathcurveto{\pgfqpoint{2.625484in}{2.484041in}}{\pgfqpoint{2.623848in}{2.480091in}}{\pgfqpoint{2.623848in}{2.475973in}}%
\pgfpathcurveto{\pgfqpoint{2.623848in}{2.471855in}}{\pgfqpoint{2.625484in}{2.467905in}}{\pgfqpoint{2.628396in}{2.464993in}}%
\pgfpathcurveto{\pgfqpoint{2.631308in}{2.462081in}}{\pgfqpoint{2.635258in}{2.460445in}}{\pgfqpoint{2.639376in}{2.460445in}}%
\pgfpathclose%
\pgfusepath{fill}%
\end{pgfscope}%
\begin{pgfscope}%
\pgfpathrectangle{\pgfqpoint{0.887500in}{0.275000in}}{\pgfqpoint{4.225000in}{4.225000in}}%
\pgfusepath{clip}%
\pgfsetbuttcap%
\pgfsetroundjoin%
\definecolor{currentfill}{rgb}{0.000000,0.000000,0.000000}%
\pgfsetfillcolor{currentfill}%
\pgfsetfillopacity{0.800000}%
\pgfsetlinewidth{0.000000pt}%
\definecolor{currentstroke}{rgb}{0.000000,0.000000,0.000000}%
\pgfsetstrokecolor{currentstroke}%
\pgfsetstrokeopacity{0.800000}%
\pgfsetdash{}{0pt}%
\pgfpathmoveto{\pgfqpoint{2.863215in}{2.722201in}}%
\pgfpathcurveto{\pgfqpoint{2.867333in}{2.722201in}}{\pgfqpoint{2.871283in}{2.723837in}}{\pgfqpoint{2.874195in}{2.726749in}}%
\pgfpathcurveto{\pgfqpoint{2.877107in}{2.729661in}}{\pgfqpoint{2.878743in}{2.733611in}}{\pgfqpoint{2.878743in}{2.737729in}}%
\pgfpathcurveto{\pgfqpoint{2.878743in}{2.741847in}}{\pgfqpoint{2.877107in}{2.745797in}}{\pgfqpoint{2.874195in}{2.748709in}}%
\pgfpathcurveto{\pgfqpoint{2.871283in}{2.751621in}}{\pgfqpoint{2.867333in}{2.753257in}}{\pgfqpoint{2.863215in}{2.753257in}}%
\pgfpathcurveto{\pgfqpoint{2.859097in}{2.753257in}}{\pgfqpoint{2.855147in}{2.751621in}}{\pgfqpoint{2.852235in}{2.748709in}}%
\pgfpathcurveto{\pgfqpoint{2.849323in}{2.745797in}}{\pgfqpoint{2.847686in}{2.741847in}}{\pgfqpoint{2.847686in}{2.737729in}}%
\pgfpathcurveto{\pgfqpoint{2.847686in}{2.733611in}}{\pgfqpoint{2.849323in}{2.729661in}}{\pgfqpoint{2.852235in}{2.726749in}}%
\pgfpathcurveto{\pgfqpoint{2.855147in}{2.723837in}}{\pgfqpoint{2.859097in}{2.722201in}}{\pgfqpoint{2.863215in}{2.722201in}}%
\pgfpathclose%
\pgfusepath{fill}%
\end{pgfscope}%
\begin{pgfscope}%
\pgfpathrectangle{\pgfqpoint{0.887500in}{0.275000in}}{\pgfqpoint{4.225000in}{4.225000in}}%
\pgfusepath{clip}%
\pgfsetbuttcap%
\pgfsetroundjoin%
\definecolor{currentfill}{rgb}{0.000000,0.000000,0.000000}%
\pgfsetfillcolor{currentfill}%
\pgfsetfillopacity{0.800000}%
\pgfsetlinewidth{0.000000pt}%
\definecolor{currentstroke}{rgb}{0.000000,0.000000,0.000000}%
\pgfsetstrokecolor{currentstroke}%
\pgfsetstrokeopacity{0.800000}%
\pgfsetdash{}{0pt}%
\pgfpathmoveto{\pgfqpoint{2.174034in}{2.534680in}}%
\pgfpathcurveto{\pgfqpoint{2.178152in}{2.534680in}}{\pgfqpoint{2.182102in}{2.536316in}}{\pgfqpoint{2.185014in}{2.539228in}}%
\pgfpathcurveto{\pgfqpoint{2.187926in}{2.542140in}}{\pgfqpoint{2.189562in}{2.546090in}}{\pgfqpoint{2.189562in}{2.550208in}}%
\pgfpathcurveto{\pgfqpoint{2.189562in}{2.554326in}}{\pgfqpoint{2.187926in}{2.558276in}}{\pgfqpoint{2.185014in}{2.561188in}}%
\pgfpathcurveto{\pgfqpoint{2.182102in}{2.564100in}}{\pgfqpoint{2.178152in}{2.565736in}}{\pgfqpoint{2.174034in}{2.565736in}}%
\pgfpathcurveto{\pgfqpoint{2.169915in}{2.565736in}}{\pgfqpoint{2.165965in}{2.564100in}}{\pgfqpoint{2.163053in}{2.561188in}}%
\pgfpathcurveto{\pgfqpoint{2.160141in}{2.558276in}}{\pgfqpoint{2.158505in}{2.554326in}}{\pgfqpoint{2.158505in}{2.550208in}}%
\pgfpathcurveto{\pgfqpoint{2.158505in}{2.546090in}}{\pgfqpoint{2.160141in}{2.542140in}}{\pgfqpoint{2.163053in}{2.539228in}}%
\pgfpathcurveto{\pgfqpoint{2.165965in}{2.536316in}}{\pgfqpoint{2.169915in}{2.534680in}}{\pgfqpoint{2.174034in}{2.534680in}}%
\pgfpathclose%
\pgfusepath{fill}%
\end{pgfscope}%
\begin{pgfscope}%
\pgfpathrectangle{\pgfqpoint{0.887500in}{0.275000in}}{\pgfqpoint{4.225000in}{4.225000in}}%
\pgfusepath{clip}%
\pgfsetbuttcap%
\pgfsetroundjoin%
\definecolor{currentfill}{rgb}{0.000000,0.000000,0.000000}%
\pgfsetfillcolor{currentfill}%
\pgfsetfillopacity{0.800000}%
\pgfsetlinewidth{0.000000pt}%
\definecolor{currentstroke}{rgb}{0.000000,0.000000,0.000000}%
\pgfsetstrokecolor{currentstroke}%
\pgfsetstrokeopacity{0.800000}%
\pgfsetdash{}{0pt}%
\pgfpathmoveto{\pgfqpoint{1.708493in}{2.604943in}}%
\pgfpathcurveto{\pgfqpoint{1.712611in}{2.604943in}}{\pgfqpoint{1.716561in}{2.606580in}}{\pgfqpoint{1.719473in}{2.609492in}}%
\pgfpathcurveto{\pgfqpoint{1.722385in}{2.612404in}}{\pgfqpoint{1.724021in}{2.616354in}}{\pgfqpoint{1.724021in}{2.620472in}}%
\pgfpathcurveto{\pgfqpoint{1.724021in}{2.624590in}}{\pgfqpoint{1.722385in}{2.628540in}}{\pgfqpoint{1.719473in}{2.631452in}}%
\pgfpathcurveto{\pgfqpoint{1.716561in}{2.634364in}}{\pgfqpoint{1.712611in}{2.636000in}}{\pgfqpoint{1.708493in}{2.636000in}}%
\pgfpathcurveto{\pgfqpoint{1.704375in}{2.636000in}}{\pgfqpoint{1.700425in}{2.634364in}}{\pgfqpoint{1.697513in}{2.631452in}}%
\pgfpathcurveto{\pgfqpoint{1.694601in}{2.628540in}}{\pgfqpoint{1.692965in}{2.624590in}}{\pgfqpoint{1.692965in}{2.620472in}}%
\pgfpathcurveto{\pgfqpoint{1.692965in}{2.616354in}}{\pgfqpoint{1.694601in}{2.612404in}}{\pgfqpoint{1.697513in}{2.609492in}}%
\pgfpathcurveto{\pgfqpoint{1.700425in}{2.606580in}}{\pgfqpoint{1.704375in}{2.604943in}}{\pgfqpoint{1.708493in}{2.604943in}}%
\pgfpathclose%
\pgfusepath{fill}%
\end{pgfscope}%
\begin{pgfscope}%
\pgfpathrectangle{\pgfqpoint{0.887500in}{0.275000in}}{\pgfqpoint{4.225000in}{4.225000in}}%
\pgfusepath{clip}%
\pgfsetbuttcap%
\pgfsetroundjoin%
\definecolor{currentfill}{rgb}{0.000000,0.000000,0.000000}%
\pgfsetfillcolor{currentfill}%
\pgfsetfillopacity{0.800000}%
\pgfsetlinewidth{0.000000pt}%
\definecolor{currentstroke}{rgb}{0.000000,0.000000,0.000000}%
\pgfsetstrokecolor{currentstroke}%
\pgfsetstrokeopacity{0.800000}%
\pgfsetdash{}{0pt}%
\pgfpathmoveto{\pgfqpoint{2.815693in}{2.403802in}}%
\pgfpathcurveto{\pgfqpoint{2.819811in}{2.403802in}}{\pgfqpoint{2.823761in}{2.405438in}}{\pgfqpoint{2.826673in}{2.408350in}}%
\pgfpathcurveto{\pgfqpoint{2.829585in}{2.411262in}}{\pgfqpoint{2.831221in}{2.415212in}}{\pgfqpoint{2.831221in}{2.419330in}}%
\pgfpathcurveto{\pgfqpoint{2.831221in}{2.423449in}}{\pgfqpoint{2.829585in}{2.427399in}}{\pgfqpoint{2.826673in}{2.430311in}}%
\pgfpathcurveto{\pgfqpoint{2.823761in}{2.433223in}}{\pgfqpoint{2.819811in}{2.434859in}}{\pgfqpoint{2.815693in}{2.434859in}}%
\pgfpathcurveto{\pgfqpoint{2.811575in}{2.434859in}}{\pgfqpoint{2.807625in}{2.433223in}}{\pgfqpoint{2.804713in}{2.430311in}}%
\pgfpathcurveto{\pgfqpoint{2.801801in}{2.427399in}}{\pgfqpoint{2.800164in}{2.423449in}}{\pgfqpoint{2.800164in}{2.419330in}}%
\pgfpathcurveto{\pgfqpoint{2.800164in}{2.415212in}}{\pgfqpoint{2.801801in}{2.411262in}}{\pgfqpoint{2.804713in}{2.408350in}}%
\pgfpathcurveto{\pgfqpoint{2.807625in}{2.405438in}}{\pgfqpoint{2.811575in}{2.403802in}}{\pgfqpoint{2.815693in}{2.403802in}}%
\pgfpathclose%
\pgfusepath{fill}%
\end{pgfscope}%
\begin{pgfscope}%
\pgfpathrectangle{\pgfqpoint{0.887500in}{0.275000in}}{\pgfqpoint{4.225000in}{4.225000in}}%
\pgfusepath{clip}%
\pgfsetbuttcap%
\pgfsetroundjoin%
\definecolor{currentfill}{rgb}{0.000000,0.000000,0.000000}%
\pgfsetfillcolor{currentfill}%
\pgfsetfillopacity{0.800000}%
\pgfsetlinewidth{0.000000pt}%
\definecolor{currentstroke}{rgb}{0.000000,0.000000,0.000000}%
\pgfsetstrokecolor{currentstroke}%
\pgfsetstrokeopacity{0.800000}%
\pgfsetdash{}{0pt}%
\pgfpathmoveto{\pgfqpoint{4.371587in}{2.709279in}}%
\pgfpathcurveto{\pgfqpoint{4.375705in}{2.709279in}}{\pgfqpoint{4.379655in}{2.710915in}}{\pgfqpoint{4.382567in}{2.713827in}}%
\pgfpathcurveto{\pgfqpoint{4.385479in}{2.716739in}}{\pgfqpoint{4.387116in}{2.720689in}}{\pgfqpoint{4.387116in}{2.724807in}}%
\pgfpathcurveto{\pgfqpoint{4.387116in}{2.728925in}}{\pgfqpoint{4.385479in}{2.732875in}}{\pgfqpoint{4.382567in}{2.735787in}}%
\pgfpathcurveto{\pgfqpoint{4.379655in}{2.738699in}}{\pgfqpoint{4.375705in}{2.740335in}}{\pgfqpoint{4.371587in}{2.740335in}}%
\pgfpathcurveto{\pgfqpoint{4.367469in}{2.740335in}}{\pgfqpoint{4.363519in}{2.738699in}}{\pgfqpoint{4.360607in}{2.735787in}}%
\pgfpathcurveto{\pgfqpoint{4.357695in}{2.732875in}}{\pgfqpoint{4.356059in}{2.728925in}}{\pgfqpoint{4.356059in}{2.724807in}}%
\pgfpathcurveto{\pgfqpoint{4.356059in}{2.720689in}}{\pgfqpoint{4.357695in}{2.716739in}}{\pgfqpoint{4.360607in}{2.713827in}}%
\pgfpathcurveto{\pgfqpoint{4.363519in}{2.710915in}}{\pgfqpoint{4.367469in}{2.709279in}}{\pgfqpoint{4.371587in}{2.709279in}}%
\pgfpathclose%
\pgfusepath{fill}%
\end{pgfscope}%
\begin{pgfscope}%
\pgfpathrectangle{\pgfqpoint{0.887500in}{0.275000in}}{\pgfqpoint{4.225000in}{4.225000in}}%
\pgfusepath{clip}%
\pgfsetbuttcap%
\pgfsetroundjoin%
\definecolor{currentfill}{rgb}{0.000000,0.000000,0.000000}%
\pgfsetfillcolor{currentfill}%
\pgfsetfillopacity{0.800000}%
\pgfsetlinewidth{0.000000pt}%
\definecolor{currentstroke}{rgb}{0.000000,0.000000,0.000000}%
\pgfsetstrokecolor{currentstroke}%
\pgfsetstrokeopacity{0.800000}%
\pgfsetdash{}{0pt}%
\pgfpathmoveto{\pgfqpoint{3.134486in}{3.380087in}}%
\pgfpathcurveto{\pgfqpoint{3.138604in}{3.380087in}}{\pgfqpoint{3.142554in}{3.381723in}}{\pgfqpoint{3.145466in}{3.384635in}}%
\pgfpathcurveto{\pgfqpoint{3.148378in}{3.387547in}}{\pgfqpoint{3.150015in}{3.391497in}}{\pgfqpoint{3.150015in}{3.395615in}}%
\pgfpathcurveto{\pgfqpoint{3.150015in}{3.399733in}}{\pgfqpoint{3.148378in}{3.403683in}}{\pgfqpoint{3.145466in}{3.406595in}}%
\pgfpathcurveto{\pgfqpoint{3.142554in}{3.409507in}}{\pgfqpoint{3.138604in}{3.411143in}}{\pgfqpoint{3.134486in}{3.411143in}}%
\pgfpathcurveto{\pgfqpoint{3.130368in}{3.411143in}}{\pgfqpoint{3.126418in}{3.409507in}}{\pgfqpoint{3.123506in}{3.406595in}}%
\pgfpathcurveto{\pgfqpoint{3.120594in}{3.403683in}}{\pgfqpoint{3.118958in}{3.399733in}}{\pgfqpoint{3.118958in}{3.395615in}}%
\pgfpathcurveto{\pgfqpoint{3.118958in}{3.391497in}}{\pgfqpoint{3.120594in}{3.387547in}}{\pgfqpoint{3.123506in}{3.384635in}}%
\pgfpathcurveto{\pgfqpoint{3.126418in}{3.381723in}}{\pgfqpoint{3.130368in}{3.380087in}}{\pgfqpoint{3.134486in}{3.380087in}}%
\pgfpathclose%
\pgfusepath{fill}%
\end{pgfscope}%
\begin{pgfscope}%
\pgfpathrectangle{\pgfqpoint{0.887500in}{0.275000in}}{\pgfqpoint{4.225000in}{4.225000in}}%
\pgfusepath{clip}%
\pgfsetbuttcap%
\pgfsetroundjoin%
\definecolor{currentfill}{rgb}{0.000000,0.000000,0.000000}%
\pgfsetfillcolor{currentfill}%
\pgfsetfillopacity{0.800000}%
\pgfsetlinewidth{0.000000pt}%
\definecolor{currentstroke}{rgb}{0.000000,0.000000,0.000000}%
\pgfsetstrokecolor{currentstroke}%
\pgfsetstrokeopacity{0.800000}%
\pgfsetdash{}{0pt}%
\pgfpathmoveto{\pgfqpoint{2.349905in}{2.481504in}}%
\pgfpathcurveto{\pgfqpoint{2.354023in}{2.481504in}}{\pgfqpoint{2.357974in}{2.483140in}}{\pgfqpoint{2.360885in}{2.486052in}}%
\pgfpathcurveto{\pgfqpoint{2.363797in}{2.488964in}}{\pgfqpoint{2.365434in}{2.492914in}}{\pgfqpoint{2.365434in}{2.497032in}}%
\pgfpathcurveto{\pgfqpoint{2.365434in}{2.501151in}}{\pgfqpoint{2.363797in}{2.505101in}}{\pgfqpoint{2.360885in}{2.508013in}}%
\pgfpathcurveto{\pgfqpoint{2.357974in}{2.510925in}}{\pgfqpoint{2.354023in}{2.512561in}}{\pgfqpoint{2.349905in}{2.512561in}}%
\pgfpathcurveto{\pgfqpoint{2.345787in}{2.512561in}}{\pgfqpoint{2.341837in}{2.510925in}}{\pgfqpoint{2.338925in}{2.508013in}}%
\pgfpathcurveto{\pgfqpoint{2.336013in}{2.505101in}}{\pgfqpoint{2.334377in}{2.501151in}}{\pgfqpoint{2.334377in}{2.497032in}}%
\pgfpathcurveto{\pgfqpoint{2.334377in}{2.492914in}}{\pgfqpoint{2.336013in}{2.488964in}}{\pgfqpoint{2.338925in}{2.486052in}}%
\pgfpathcurveto{\pgfqpoint{2.341837in}{2.483140in}}{\pgfqpoint{2.345787in}{2.481504in}}{\pgfqpoint{2.349905in}{2.481504in}}%
\pgfpathclose%
\pgfusepath{fill}%
\end{pgfscope}%
\begin{pgfscope}%
\pgfpathrectangle{\pgfqpoint{0.887500in}{0.275000in}}{\pgfqpoint{4.225000in}{4.225000in}}%
\pgfusepath{clip}%
\pgfsetbuttcap%
\pgfsetroundjoin%
\definecolor{currentfill}{rgb}{0.000000,0.000000,0.000000}%
\pgfsetfillcolor{currentfill}%
\pgfsetfillopacity{0.800000}%
\pgfsetlinewidth{0.000000pt}%
\definecolor{currentstroke}{rgb}{0.000000,0.000000,0.000000}%
\pgfsetstrokecolor{currentstroke}%
\pgfsetstrokeopacity{0.800000}%
\pgfsetdash{}{0pt}%
\pgfpathmoveto{\pgfqpoint{4.195369in}{2.821984in}}%
\pgfpathcurveto{\pgfqpoint{4.199487in}{2.821984in}}{\pgfqpoint{4.203437in}{2.823620in}}{\pgfqpoint{4.206349in}{2.826532in}}%
\pgfpathcurveto{\pgfqpoint{4.209261in}{2.829444in}}{\pgfqpoint{4.210897in}{2.833394in}}{\pgfqpoint{4.210897in}{2.837512in}}%
\pgfpathcurveto{\pgfqpoint{4.210897in}{2.841631in}}{\pgfqpoint{4.209261in}{2.845581in}}{\pgfqpoint{4.206349in}{2.848493in}}%
\pgfpathcurveto{\pgfqpoint{4.203437in}{2.851405in}}{\pgfqpoint{4.199487in}{2.853041in}}{\pgfqpoint{4.195369in}{2.853041in}}%
\pgfpathcurveto{\pgfqpoint{4.191251in}{2.853041in}}{\pgfqpoint{4.187301in}{2.851405in}}{\pgfqpoint{4.184389in}{2.848493in}}%
\pgfpathcurveto{\pgfqpoint{4.181477in}{2.845581in}}{\pgfqpoint{4.179841in}{2.841631in}}{\pgfqpoint{4.179841in}{2.837512in}}%
\pgfpathcurveto{\pgfqpoint{4.179841in}{2.833394in}}{\pgfqpoint{4.181477in}{2.829444in}}{\pgfqpoint{4.184389in}{2.826532in}}%
\pgfpathcurveto{\pgfqpoint{4.187301in}{2.823620in}}{\pgfqpoint{4.191251in}{2.821984in}}{\pgfqpoint{4.195369in}{2.821984in}}%
\pgfpathclose%
\pgfusepath{fill}%
\end{pgfscope}%
\begin{pgfscope}%
\pgfpathrectangle{\pgfqpoint{0.887500in}{0.275000in}}{\pgfqpoint{4.225000in}{4.225000in}}%
\pgfusepath{clip}%
\pgfsetbuttcap%
\pgfsetroundjoin%
\definecolor{currentfill}{rgb}{0.000000,0.000000,0.000000}%
\pgfsetfillcolor{currentfill}%
\pgfsetfillopacity{0.800000}%
\pgfsetlinewidth{0.000000pt}%
\definecolor{currentstroke}{rgb}{0.000000,0.000000,0.000000}%
\pgfsetstrokecolor{currentstroke}%
\pgfsetstrokeopacity{0.800000}%
\pgfsetdash{}{0pt}%
\pgfpathmoveto{\pgfqpoint{3.311354in}{3.301662in}}%
\pgfpathcurveto{\pgfqpoint{3.315472in}{3.301662in}}{\pgfqpoint{3.319422in}{3.303298in}}{\pgfqpoint{3.322334in}{3.306210in}}%
\pgfpathcurveto{\pgfqpoint{3.325246in}{3.309122in}}{\pgfqpoint{3.326882in}{3.313072in}}{\pgfqpoint{3.326882in}{3.317190in}}%
\pgfpathcurveto{\pgfqpoint{3.326882in}{3.321308in}}{\pgfqpoint{3.325246in}{3.325258in}}{\pgfqpoint{3.322334in}{3.328170in}}%
\pgfpathcurveto{\pgfqpoint{3.319422in}{3.331082in}}{\pgfqpoint{3.315472in}{3.332718in}}{\pgfqpoint{3.311354in}{3.332718in}}%
\pgfpathcurveto{\pgfqpoint{3.307236in}{3.332718in}}{\pgfqpoint{3.303286in}{3.331082in}}{\pgfqpoint{3.300374in}{3.328170in}}%
\pgfpathcurveto{\pgfqpoint{3.297462in}{3.325258in}}{\pgfqpoint{3.295826in}{3.321308in}}{\pgfqpoint{3.295826in}{3.317190in}}%
\pgfpathcurveto{\pgfqpoint{3.295826in}{3.313072in}}{\pgfqpoint{3.297462in}{3.309122in}}{\pgfqpoint{3.300374in}{3.306210in}}%
\pgfpathcurveto{\pgfqpoint{3.303286in}{3.303298in}}{\pgfqpoint{3.307236in}{3.301662in}}{\pgfqpoint{3.311354in}{3.301662in}}%
\pgfpathclose%
\pgfusepath{fill}%
\end{pgfscope}%
\begin{pgfscope}%
\pgfpathrectangle{\pgfqpoint{0.887500in}{0.275000in}}{\pgfqpoint{4.225000in}{4.225000in}}%
\pgfusepath{clip}%
\pgfsetbuttcap%
\pgfsetroundjoin%
\definecolor{currentfill}{rgb}{0.000000,0.000000,0.000000}%
\pgfsetfillcolor{currentfill}%
\pgfsetfillopacity{0.800000}%
\pgfsetlinewidth{0.000000pt}%
\definecolor{currentstroke}{rgb}{0.000000,0.000000,0.000000}%
\pgfsetstrokecolor{currentstroke}%
\pgfsetstrokeopacity{0.800000}%
\pgfsetdash{}{0pt}%
\pgfpathmoveto{\pgfqpoint{1.883859in}{2.553968in}}%
\pgfpathcurveto{\pgfqpoint{1.887977in}{2.553968in}}{\pgfqpoint{1.891927in}{2.555604in}}{\pgfqpoint{1.894839in}{2.558516in}}%
\pgfpathcurveto{\pgfqpoint{1.897751in}{2.561428in}}{\pgfqpoint{1.899387in}{2.565378in}}{\pgfqpoint{1.899387in}{2.569496in}}%
\pgfpathcurveto{\pgfqpoint{1.899387in}{2.573614in}}{\pgfqpoint{1.897751in}{2.577564in}}{\pgfqpoint{1.894839in}{2.580476in}}%
\pgfpathcurveto{\pgfqpoint{1.891927in}{2.583388in}}{\pgfqpoint{1.887977in}{2.585024in}}{\pgfqpoint{1.883859in}{2.585024in}}%
\pgfpathcurveto{\pgfqpoint{1.879741in}{2.585024in}}{\pgfqpoint{1.875791in}{2.583388in}}{\pgfqpoint{1.872879in}{2.580476in}}%
\pgfpathcurveto{\pgfqpoint{1.869967in}{2.577564in}}{\pgfqpoint{1.868331in}{2.573614in}}{\pgfqpoint{1.868331in}{2.569496in}}%
\pgfpathcurveto{\pgfqpoint{1.868331in}{2.565378in}}{\pgfqpoint{1.869967in}{2.561428in}}{\pgfqpoint{1.872879in}{2.558516in}}%
\pgfpathcurveto{\pgfqpoint{1.875791in}{2.555604in}}{\pgfqpoint{1.879741in}{2.553968in}}{\pgfqpoint{1.883859in}{2.553968in}}%
\pgfpathclose%
\pgfusepath{fill}%
\end{pgfscope}%
\begin{pgfscope}%
\pgfpathrectangle{\pgfqpoint{0.887500in}{0.275000in}}{\pgfqpoint{4.225000in}{4.225000in}}%
\pgfusepath{clip}%
\pgfsetbuttcap%
\pgfsetroundjoin%
\definecolor{currentfill}{rgb}{0.000000,0.000000,0.000000}%
\pgfsetfillcolor{currentfill}%
\pgfsetfillopacity{0.800000}%
\pgfsetlinewidth{0.000000pt}%
\definecolor{currentstroke}{rgb}{0.000000,0.000000,0.000000}%
\pgfsetstrokecolor{currentstroke}%
\pgfsetstrokeopacity{0.800000}%
\pgfsetdash{}{0pt}%
\pgfpathmoveto{\pgfqpoint{4.018930in}{2.932198in}}%
\pgfpathcurveto{\pgfqpoint{4.023048in}{2.932198in}}{\pgfqpoint{4.026998in}{2.933834in}}{\pgfqpoint{4.029910in}{2.936746in}}%
\pgfpathcurveto{\pgfqpoint{4.032822in}{2.939658in}}{\pgfqpoint{4.034458in}{2.943608in}}{\pgfqpoint{4.034458in}{2.947727in}}%
\pgfpathcurveto{\pgfqpoint{4.034458in}{2.951845in}}{\pgfqpoint{4.032822in}{2.955795in}}{\pgfqpoint{4.029910in}{2.958707in}}%
\pgfpathcurveto{\pgfqpoint{4.026998in}{2.961619in}}{\pgfqpoint{4.023048in}{2.963255in}}{\pgfqpoint{4.018930in}{2.963255in}}%
\pgfpathcurveto{\pgfqpoint{4.014812in}{2.963255in}}{\pgfqpoint{4.010862in}{2.961619in}}{\pgfqpoint{4.007950in}{2.958707in}}%
\pgfpathcurveto{\pgfqpoint{4.005038in}{2.955795in}}{\pgfqpoint{4.003402in}{2.951845in}}{\pgfqpoint{4.003402in}{2.947727in}}%
\pgfpathcurveto{\pgfqpoint{4.003402in}{2.943608in}}{\pgfqpoint{4.005038in}{2.939658in}}{\pgfqpoint{4.007950in}{2.936746in}}%
\pgfpathcurveto{\pgfqpoint{4.010862in}{2.933834in}}{\pgfqpoint{4.014812in}{2.932198in}}{\pgfqpoint{4.018930in}{2.932198in}}%
\pgfpathclose%
\pgfusepath{fill}%
\end{pgfscope}%
\begin{pgfscope}%
\pgfpathrectangle{\pgfqpoint{0.887500in}{0.275000in}}{\pgfqpoint{4.225000in}{4.225000in}}%
\pgfusepath{clip}%
\pgfsetbuttcap%
\pgfsetroundjoin%
\definecolor{currentfill}{rgb}{0.000000,0.000000,0.000000}%
\pgfsetfillcolor{currentfill}%
\pgfsetfillopacity{0.800000}%
\pgfsetlinewidth{0.000000pt}%
\definecolor{currentstroke}{rgb}{0.000000,0.000000,0.000000}%
\pgfsetstrokecolor{currentstroke}%
\pgfsetstrokeopacity{0.800000}%
\pgfsetdash{}{0pt}%
\pgfpathmoveto{\pgfqpoint{1.417769in}{2.620473in}}%
\pgfpathcurveto{\pgfqpoint{1.421887in}{2.620473in}}{\pgfqpoint{1.425837in}{2.622109in}}{\pgfqpoint{1.428749in}{2.625021in}}%
\pgfpathcurveto{\pgfqpoint{1.431661in}{2.627933in}}{\pgfqpoint{1.433297in}{2.631883in}}{\pgfqpoint{1.433297in}{2.636001in}}%
\pgfpathcurveto{\pgfqpoint{1.433297in}{2.640119in}}{\pgfqpoint{1.431661in}{2.644069in}}{\pgfqpoint{1.428749in}{2.646981in}}%
\pgfpathcurveto{\pgfqpoint{1.425837in}{2.649893in}}{\pgfqpoint{1.421887in}{2.651529in}}{\pgfqpoint{1.417769in}{2.651529in}}%
\pgfpathcurveto{\pgfqpoint{1.413651in}{2.651529in}}{\pgfqpoint{1.409701in}{2.649893in}}{\pgfqpoint{1.406789in}{2.646981in}}%
\pgfpathcurveto{\pgfqpoint{1.403877in}{2.644069in}}{\pgfqpoint{1.402241in}{2.640119in}}{\pgfqpoint{1.402241in}{2.636001in}}%
\pgfpathcurveto{\pgfqpoint{1.402241in}{2.631883in}}{\pgfqpoint{1.403877in}{2.627933in}}{\pgfqpoint{1.406789in}{2.625021in}}%
\pgfpathcurveto{\pgfqpoint{1.409701in}{2.622109in}}{\pgfqpoint{1.413651in}{2.620473in}}{\pgfqpoint{1.417769in}{2.620473in}}%
\pgfpathclose%
\pgfusepath{fill}%
\end{pgfscope}%
\begin{pgfscope}%
\pgfpathrectangle{\pgfqpoint{0.887500in}{0.275000in}}{\pgfqpoint{4.225000in}{4.225000in}}%
\pgfusepath{clip}%
\pgfsetbuttcap%
\pgfsetroundjoin%
\definecolor{currentfill}{rgb}{0.000000,0.000000,0.000000}%
\pgfsetfillcolor{currentfill}%
\pgfsetfillopacity{0.800000}%
\pgfsetlinewidth{0.000000pt}%
\definecolor{currentstroke}{rgb}{0.000000,0.000000,0.000000}%
\pgfsetstrokecolor{currentstroke}%
\pgfsetstrokeopacity{0.800000}%
\pgfsetdash{}{0pt}%
\pgfpathmoveto{\pgfqpoint{3.488338in}{3.220736in}}%
\pgfpathcurveto{\pgfqpoint{3.492456in}{3.220736in}}{\pgfqpoint{3.496406in}{3.222372in}}{\pgfqpoint{3.499318in}{3.225284in}}%
\pgfpathcurveto{\pgfqpoint{3.502230in}{3.228196in}}{\pgfqpoint{3.503866in}{3.232146in}}{\pgfqpoint{3.503866in}{3.236264in}}%
\pgfpathcurveto{\pgfqpoint{3.503866in}{3.240382in}}{\pgfqpoint{3.502230in}{3.244332in}}{\pgfqpoint{3.499318in}{3.247244in}}%
\pgfpathcurveto{\pgfqpoint{3.496406in}{3.250156in}}{\pgfqpoint{3.492456in}{3.251792in}}{\pgfqpoint{3.488338in}{3.251792in}}%
\pgfpathcurveto{\pgfqpoint{3.484220in}{3.251792in}}{\pgfqpoint{3.480270in}{3.250156in}}{\pgfqpoint{3.477358in}{3.247244in}}%
\pgfpathcurveto{\pgfqpoint{3.474446in}{3.244332in}}{\pgfqpoint{3.472810in}{3.240382in}}{\pgfqpoint{3.472810in}{3.236264in}}%
\pgfpathcurveto{\pgfqpoint{3.472810in}{3.232146in}}{\pgfqpoint{3.474446in}{3.228196in}}{\pgfqpoint{3.477358in}{3.225284in}}%
\pgfpathcurveto{\pgfqpoint{3.480270in}{3.222372in}}{\pgfqpoint{3.484220in}{3.220736in}}{\pgfqpoint{3.488338in}{3.220736in}}%
\pgfpathclose%
\pgfusepath{fill}%
\end{pgfscope}%
\begin{pgfscope}%
\pgfpathrectangle{\pgfqpoint{0.887500in}{0.275000in}}{\pgfqpoint{4.225000in}{4.225000in}}%
\pgfusepath{clip}%
\pgfsetbuttcap%
\pgfsetroundjoin%
\definecolor{currentfill}{rgb}{0.000000,0.000000,0.000000}%
\pgfsetfillcolor{currentfill}%
\pgfsetfillopacity{0.800000}%
\pgfsetlinewidth{0.000000pt}%
\definecolor{currentstroke}{rgb}{0.000000,0.000000,0.000000}%
\pgfsetstrokecolor{currentstroke}%
\pgfsetstrokeopacity{0.800000}%
\pgfsetdash{}{0pt}%
\pgfpathmoveto{\pgfqpoint{3.842263in}{3.037816in}}%
\pgfpathcurveto{\pgfqpoint{3.846381in}{3.037816in}}{\pgfqpoint{3.850331in}{3.039453in}}{\pgfqpoint{3.853243in}{3.042365in}}%
\pgfpathcurveto{\pgfqpoint{3.856155in}{3.045276in}}{\pgfqpoint{3.857791in}{3.049227in}}{\pgfqpoint{3.857791in}{3.053345in}}%
\pgfpathcurveto{\pgfqpoint{3.857791in}{3.057463in}}{\pgfqpoint{3.856155in}{3.061413in}}{\pgfqpoint{3.853243in}{3.064325in}}%
\pgfpathcurveto{\pgfqpoint{3.850331in}{3.067237in}}{\pgfqpoint{3.846381in}{3.068873in}}{\pgfqpoint{3.842263in}{3.068873in}}%
\pgfpathcurveto{\pgfqpoint{3.838145in}{3.068873in}}{\pgfqpoint{3.834195in}{3.067237in}}{\pgfqpoint{3.831283in}{3.064325in}}%
\pgfpathcurveto{\pgfqpoint{3.828371in}{3.061413in}}{\pgfqpoint{3.826735in}{3.057463in}}{\pgfqpoint{3.826735in}{3.053345in}}%
\pgfpathcurveto{\pgfqpoint{3.826735in}{3.049227in}}{\pgfqpoint{3.828371in}{3.045276in}}{\pgfqpoint{3.831283in}{3.042365in}}%
\pgfpathcurveto{\pgfqpoint{3.834195in}{3.039453in}}{\pgfqpoint{3.838145in}{3.037816in}}{\pgfqpoint{3.842263in}{3.037816in}}%
\pgfpathclose%
\pgfusepath{fill}%
\end{pgfscope}%
\begin{pgfscope}%
\pgfpathrectangle{\pgfqpoint{0.887500in}{0.275000in}}{\pgfqpoint{4.225000in}{4.225000in}}%
\pgfusepath{clip}%
\pgfsetbuttcap%
\pgfsetroundjoin%
\definecolor{currentfill}{rgb}{0.000000,0.000000,0.000000}%
\pgfsetfillcolor{currentfill}%
\pgfsetfillopacity{0.800000}%
\pgfsetlinewidth{0.000000pt}%
\definecolor{currentstroke}{rgb}{0.000000,0.000000,0.000000}%
\pgfsetstrokecolor{currentstroke}%
\pgfsetstrokeopacity{0.800000}%
\pgfsetdash{}{0pt}%
\pgfpathmoveto{\pgfqpoint{3.665339in}{3.132802in}}%
\pgfpathcurveto{\pgfqpoint{3.669457in}{3.132802in}}{\pgfqpoint{3.673407in}{3.134438in}}{\pgfqpoint{3.676319in}{3.137350in}}%
\pgfpathcurveto{\pgfqpoint{3.679231in}{3.140262in}}{\pgfqpoint{3.680867in}{3.144212in}}{\pgfqpoint{3.680867in}{3.148330in}}%
\pgfpathcurveto{\pgfqpoint{3.680867in}{3.152448in}}{\pgfqpoint{3.679231in}{3.156398in}}{\pgfqpoint{3.676319in}{3.159310in}}%
\pgfpathcurveto{\pgfqpoint{3.673407in}{3.162222in}}{\pgfqpoint{3.669457in}{3.163859in}}{\pgfqpoint{3.665339in}{3.163859in}}%
\pgfpathcurveto{\pgfqpoint{3.661221in}{3.163859in}}{\pgfqpoint{3.657271in}{3.162222in}}{\pgfqpoint{3.654359in}{3.159310in}}%
\pgfpathcurveto{\pgfqpoint{3.651447in}{3.156398in}}{\pgfqpoint{3.649811in}{3.152448in}}{\pgfqpoint{3.649811in}{3.148330in}}%
\pgfpathcurveto{\pgfqpoint{3.649811in}{3.144212in}}{\pgfqpoint{3.651447in}{3.140262in}}{\pgfqpoint{3.654359in}{3.137350in}}%
\pgfpathcurveto{\pgfqpoint{3.657271in}{3.134438in}}{\pgfqpoint{3.661221in}{3.132802in}}{\pgfqpoint{3.665339in}{3.132802in}}%
\pgfpathclose%
\pgfusepath{fill}%
\end{pgfscope}%
\begin{pgfscope}%
\pgfpathrectangle{\pgfqpoint{0.887500in}{0.275000in}}{\pgfqpoint{4.225000in}{4.225000in}}%
\pgfusepath{clip}%
\pgfsetbuttcap%
\pgfsetroundjoin%
\definecolor{currentfill}{rgb}{0.000000,0.000000,0.000000}%
\pgfsetfillcolor{currentfill}%
\pgfsetfillopacity{0.800000}%
\pgfsetlinewidth{0.000000pt}%
\definecolor{currentstroke}{rgb}{0.000000,0.000000,0.000000}%
\pgfsetstrokecolor{currentstroke}%
\pgfsetstrokeopacity{0.800000}%
\pgfsetdash{}{0pt}%
\pgfpathmoveto{\pgfqpoint{2.526163in}{2.426936in}}%
\pgfpathcurveto{\pgfqpoint{2.530281in}{2.426936in}}{\pgfqpoint{2.534232in}{2.428573in}}{\pgfqpoint{2.537143in}{2.431485in}}%
\pgfpathcurveto{\pgfqpoint{2.540055in}{2.434397in}}{\pgfqpoint{2.541692in}{2.438347in}}{\pgfqpoint{2.541692in}{2.442465in}}%
\pgfpathcurveto{\pgfqpoint{2.541692in}{2.446583in}}{\pgfqpoint{2.540055in}{2.450533in}}{\pgfqpoint{2.537143in}{2.453445in}}%
\pgfpathcurveto{\pgfqpoint{2.534232in}{2.456357in}}{\pgfqpoint{2.530281in}{2.457993in}}{\pgfqpoint{2.526163in}{2.457993in}}%
\pgfpathcurveto{\pgfqpoint{2.522045in}{2.457993in}}{\pgfqpoint{2.518095in}{2.456357in}}{\pgfqpoint{2.515183in}{2.453445in}}%
\pgfpathcurveto{\pgfqpoint{2.512271in}{2.450533in}}{\pgfqpoint{2.510635in}{2.446583in}}{\pgfqpoint{2.510635in}{2.442465in}}%
\pgfpathcurveto{\pgfqpoint{2.510635in}{2.438347in}}{\pgfqpoint{2.512271in}{2.434397in}}{\pgfqpoint{2.515183in}{2.431485in}}%
\pgfpathcurveto{\pgfqpoint{2.518095in}{2.428573in}}{\pgfqpoint{2.522045in}{2.426936in}}{\pgfqpoint{2.526163in}{2.426936in}}%
\pgfpathclose%
\pgfusepath{fill}%
\end{pgfscope}%
\begin{pgfscope}%
\pgfpathrectangle{\pgfqpoint{0.887500in}{0.275000in}}{\pgfqpoint{4.225000in}{4.225000in}}%
\pgfusepath{clip}%
\pgfsetbuttcap%
\pgfsetroundjoin%
\definecolor{currentfill}{rgb}{0.000000,0.000000,0.000000}%
\pgfsetfillcolor{currentfill}%
\pgfsetfillopacity{0.800000}%
\pgfsetlinewidth{0.000000pt}%
\definecolor{currentstroke}{rgb}{0.000000,0.000000,0.000000}%
\pgfsetstrokecolor{currentstroke}%
\pgfsetstrokeopacity{0.800000}%
\pgfsetdash{}{0pt}%
\pgfpathmoveto{\pgfqpoint{2.059640in}{2.501767in}}%
\pgfpathcurveto{\pgfqpoint{2.063758in}{2.501767in}}{\pgfqpoint{2.067708in}{2.503403in}}{\pgfqpoint{2.070620in}{2.506315in}}%
\pgfpathcurveto{\pgfqpoint{2.073532in}{2.509227in}}{\pgfqpoint{2.075168in}{2.513177in}}{\pgfqpoint{2.075168in}{2.517295in}}%
\pgfpathcurveto{\pgfqpoint{2.075168in}{2.521414in}}{\pgfqpoint{2.073532in}{2.525364in}}{\pgfqpoint{2.070620in}{2.528276in}}%
\pgfpathcurveto{\pgfqpoint{2.067708in}{2.531188in}}{\pgfqpoint{2.063758in}{2.532824in}}{\pgfqpoint{2.059640in}{2.532824in}}%
\pgfpathcurveto{\pgfqpoint{2.055522in}{2.532824in}}{\pgfqpoint{2.051572in}{2.531188in}}{\pgfqpoint{2.048660in}{2.528276in}}%
\pgfpathcurveto{\pgfqpoint{2.045748in}{2.525364in}}{\pgfqpoint{2.044112in}{2.521414in}}{\pgfqpoint{2.044112in}{2.517295in}}%
\pgfpathcurveto{\pgfqpoint{2.044112in}{2.513177in}}{\pgfqpoint{2.045748in}{2.509227in}}{\pgfqpoint{2.048660in}{2.506315in}}%
\pgfpathcurveto{\pgfqpoint{2.051572in}{2.503403in}}{\pgfqpoint{2.055522in}{2.501767in}}{\pgfqpoint{2.059640in}{2.501767in}}%
\pgfpathclose%
\pgfusepath{fill}%
\end{pgfscope}%
\begin{pgfscope}%
\pgfpathrectangle{\pgfqpoint{0.887500in}{0.275000in}}{\pgfqpoint{4.225000in}{4.225000in}}%
\pgfusepath{clip}%
\pgfsetbuttcap%
\pgfsetroundjoin%
\definecolor{currentfill}{rgb}{0.000000,0.000000,0.000000}%
\pgfsetfillcolor{currentfill}%
\pgfsetfillopacity{0.800000}%
\pgfsetlinewidth{0.000000pt}%
\definecolor{currentstroke}{rgb}{0.000000,0.000000,0.000000}%
\pgfsetstrokecolor{currentstroke}%
\pgfsetstrokeopacity{0.800000}%
\pgfsetdash{}{0pt}%
\pgfpathmoveto{\pgfqpoint{4.438867in}{2.559620in}}%
\pgfpathcurveto{\pgfqpoint{4.442985in}{2.559620in}}{\pgfqpoint{4.446935in}{2.561256in}}{\pgfqpoint{4.449847in}{2.564168in}}%
\pgfpathcurveto{\pgfqpoint{4.452759in}{2.567080in}}{\pgfqpoint{4.454395in}{2.571030in}}{\pgfqpoint{4.454395in}{2.575149in}}%
\pgfpathcurveto{\pgfqpoint{4.454395in}{2.579267in}}{\pgfqpoint{4.452759in}{2.583217in}}{\pgfqpoint{4.449847in}{2.586129in}}%
\pgfpathcurveto{\pgfqpoint{4.446935in}{2.589041in}}{\pgfqpoint{4.442985in}{2.590677in}}{\pgfqpoint{4.438867in}{2.590677in}}%
\pgfpathcurveto{\pgfqpoint{4.434748in}{2.590677in}}{\pgfqpoint{4.430798in}{2.589041in}}{\pgfqpoint{4.427886in}{2.586129in}}%
\pgfpathcurveto{\pgfqpoint{4.424974in}{2.583217in}}{\pgfqpoint{4.423338in}{2.579267in}}{\pgfqpoint{4.423338in}{2.575149in}}%
\pgfpathcurveto{\pgfqpoint{4.423338in}{2.571030in}}{\pgfqpoint{4.424974in}{2.567080in}}{\pgfqpoint{4.427886in}{2.564168in}}%
\pgfpathcurveto{\pgfqpoint{4.430798in}{2.561256in}}{\pgfqpoint{4.434748in}{2.559620in}}{\pgfqpoint{4.438867in}{2.559620in}}%
\pgfpathclose%
\pgfusepath{fill}%
\end{pgfscope}%
\begin{pgfscope}%
\pgfpathrectangle{\pgfqpoint{0.887500in}{0.275000in}}{\pgfqpoint{4.225000in}{4.225000in}}%
\pgfusepath{clip}%
\pgfsetbuttcap%
\pgfsetroundjoin%
\definecolor{currentfill}{rgb}{0.000000,0.000000,0.000000}%
\pgfsetfillcolor{currentfill}%
\pgfsetfillopacity{0.800000}%
\pgfsetlinewidth{0.000000pt}%
\definecolor{currentstroke}{rgb}{0.000000,0.000000,0.000000}%
\pgfsetstrokecolor{currentstroke}%
\pgfsetstrokeopacity{0.800000}%
\pgfsetdash{}{0pt}%
\pgfpathmoveto{\pgfqpoint{1.592922in}{2.572537in}}%
\pgfpathcurveto{\pgfqpoint{1.597040in}{2.572537in}}{\pgfqpoint{1.600990in}{2.574173in}}{\pgfqpoint{1.603902in}{2.577085in}}%
\pgfpathcurveto{\pgfqpoint{1.606814in}{2.579997in}}{\pgfqpoint{1.608451in}{2.583947in}}{\pgfqpoint{1.608451in}{2.588065in}}%
\pgfpathcurveto{\pgfqpoint{1.608451in}{2.592183in}}{\pgfqpoint{1.606814in}{2.596133in}}{\pgfqpoint{1.603902in}{2.599045in}}%
\pgfpathcurveto{\pgfqpoint{1.600990in}{2.601957in}}{\pgfqpoint{1.597040in}{2.603593in}}{\pgfqpoint{1.592922in}{2.603593in}}%
\pgfpathcurveto{\pgfqpoint{1.588804in}{2.603593in}}{\pgfqpoint{1.584854in}{2.601957in}}{\pgfqpoint{1.581942in}{2.599045in}}%
\pgfpathcurveto{\pgfqpoint{1.579030in}{2.596133in}}{\pgfqpoint{1.577394in}{2.592183in}}{\pgfqpoint{1.577394in}{2.588065in}}%
\pgfpathcurveto{\pgfqpoint{1.577394in}{2.583947in}}{\pgfqpoint{1.579030in}{2.579997in}}{\pgfqpoint{1.581942in}{2.577085in}}%
\pgfpathcurveto{\pgfqpoint{1.584854in}{2.574173in}}{\pgfqpoint{1.588804in}{2.572537in}}{\pgfqpoint{1.592922in}{2.572537in}}%
\pgfpathclose%
\pgfusepath{fill}%
\end{pgfscope}%
\begin{pgfscope}%
\pgfpathrectangle{\pgfqpoint{0.887500in}{0.275000in}}{\pgfqpoint{4.225000in}{4.225000in}}%
\pgfusepath{clip}%
\pgfsetbuttcap%
\pgfsetroundjoin%
\definecolor{currentfill}{rgb}{0.000000,0.000000,0.000000}%
\pgfsetfillcolor{currentfill}%
\pgfsetfillopacity{0.800000}%
\pgfsetlinewidth{0.000000pt}%
\definecolor{currentstroke}{rgb}{0.000000,0.000000,0.000000}%
\pgfsetstrokecolor{currentstroke}%
\pgfsetstrokeopacity{0.800000}%
\pgfsetdash{}{0pt}%
\pgfpathmoveto{\pgfqpoint{2.702795in}{2.370551in}}%
\pgfpathcurveto{\pgfqpoint{2.706914in}{2.370551in}}{\pgfqpoint{2.710864in}{2.372187in}}{\pgfqpoint{2.713776in}{2.375099in}}%
\pgfpathcurveto{\pgfqpoint{2.716688in}{2.378011in}}{\pgfqpoint{2.718324in}{2.381961in}}{\pgfqpoint{2.718324in}{2.386079in}}%
\pgfpathcurveto{\pgfqpoint{2.718324in}{2.390197in}}{\pgfqpoint{2.716688in}{2.394147in}}{\pgfqpoint{2.713776in}{2.397059in}}%
\pgfpathcurveto{\pgfqpoint{2.710864in}{2.399971in}}{\pgfqpoint{2.706914in}{2.401607in}}{\pgfqpoint{2.702795in}{2.401607in}}%
\pgfpathcurveto{\pgfqpoint{2.698677in}{2.401607in}}{\pgfqpoint{2.694727in}{2.399971in}}{\pgfqpoint{2.691815in}{2.397059in}}%
\pgfpathcurveto{\pgfqpoint{2.688903in}{2.394147in}}{\pgfqpoint{2.687267in}{2.390197in}}{\pgfqpoint{2.687267in}{2.386079in}}%
\pgfpathcurveto{\pgfqpoint{2.687267in}{2.381961in}}{\pgfqpoint{2.688903in}{2.378011in}}{\pgfqpoint{2.691815in}{2.375099in}}%
\pgfpathcurveto{\pgfqpoint{2.694727in}{2.372187in}}{\pgfqpoint{2.698677in}{2.370551in}}{\pgfqpoint{2.702795in}{2.370551in}}%
\pgfpathclose%
\pgfusepath{fill}%
\end{pgfscope}%
\begin{pgfscope}%
\pgfpathrectangle{\pgfqpoint{0.887500in}{0.275000in}}{\pgfqpoint{4.225000in}{4.225000in}}%
\pgfusepath{clip}%
\pgfsetbuttcap%
\pgfsetroundjoin%
\definecolor{currentfill}{rgb}{0.000000,0.000000,0.000000}%
\pgfsetfillcolor{currentfill}%
\pgfsetfillopacity{0.800000}%
\pgfsetlinewidth{0.000000pt}%
\definecolor{currentstroke}{rgb}{0.000000,0.000000,0.000000}%
\pgfsetstrokecolor{currentstroke}%
\pgfsetstrokeopacity{0.800000}%
\pgfsetdash{}{0pt}%
\pgfpathmoveto{\pgfqpoint{4.262431in}{2.675222in}}%
\pgfpathcurveto{\pgfqpoint{4.266549in}{2.675222in}}{\pgfqpoint{4.270499in}{2.676858in}}{\pgfqpoint{4.273411in}{2.679770in}}%
\pgfpathcurveto{\pgfqpoint{4.276323in}{2.682682in}}{\pgfqpoint{4.277959in}{2.686632in}}{\pgfqpoint{4.277959in}{2.690750in}}%
\pgfpathcurveto{\pgfqpoint{4.277959in}{2.694868in}}{\pgfqpoint{4.276323in}{2.698818in}}{\pgfqpoint{4.273411in}{2.701730in}}%
\pgfpathcurveto{\pgfqpoint{4.270499in}{2.704642in}}{\pgfqpoint{4.266549in}{2.706279in}}{\pgfqpoint{4.262431in}{2.706279in}}%
\pgfpathcurveto{\pgfqpoint{4.258313in}{2.706279in}}{\pgfqpoint{4.254363in}{2.704642in}}{\pgfqpoint{4.251451in}{2.701730in}}%
\pgfpathcurveto{\pgfqpoint{4.248539in}{2.698818in}}{\pgfqpoint{4.246903in}{2.694868in}}{\pgfqpoint{4.246903in}{2.690750in}}%
\pgfpathcurveto{\pgfqpoint{4.246903in}{2.686632in}}{\pgfqpoint{4.248539in}{2.682682in}}{\pgfqpoint{4.251451in}{2.679770in}}%
\pgfpathcurveto{\pgfqpoint{4.254363in}{2.676858in}}{\pgfqpoint{4.258313in}{2.675222in}}{\pgfqpoint{4.262431in}{2.675222in}}%
\pgfpathclose%
\pgfusepath{fill}%
\end{pgfscope}%
\begin{pgfscope}%
\pgfpathrectangle{\pgfqpoint{0.887500in}{0.275000in}}{\pgfqpoint{4.225000in}{4.225000in}}%
\pgfusepath{clip}%
\pgfsetbuttcap%
\pgfsetroundjoin%
\definecolor{currentfill}{rgb}{0.000000,0.000000,0.000000}%
\pgfsetfillcolor{currentfill}%
\pgfsetfillopacity{0.800000}%
\pgfsetlinewidth{0.000000pt}%
\definecolor{currentstroke}{rgb}{0.000000,0.000000,0.000000}%
\pgfsetstrokecolor{currentstroke}%
\pgfsetstrokeopacity{0.800000}%
\pgfsetdash{}{0pt}%
\pgfpathmoveto{\pgfqpoint{3.022341in}{3.345010in}}%
\pgfpathcurveto{\pgfqpoint{3.026459in}{3.345010in}}{\pgfqpoint{3.030409in}{3.346646in}}{\pgfqpoint{3.033321in}{3.349558in}}%
\pgfpathcurveto{\pgfqpoint{3.036233in}{3.352470in}}{\pgfqpoint{3.037869in}{3.356420in}}{\pgfqpoint{3.037869in}{3.360538in}}%
\pgfpathcurveto{\pgfqpoint{3.037869in}{3.364656in}}{\pgfqpoint{3.036233in}{3.368606in}}{\pgfqpoint{3.033321in}{3.371518in}}%
\pgfpathcurveto{\pgfqpoint{3.030409in}{3.374430in}}{\pgfqpoint{3.026459in}{3.376066in}}{\pgfqpoint{3.022341in}{3.376066in}}%
\pgfpathcurveto{\pgfqpoint{3.018223in}{3.376066in}}{\pgfqpoint{3.014273in}{3.374430in}}{\pgfqpoint{3.011361in}{3.371518in}}%
\pgfpathcurveto{\pgfqpoint{3.008449in}{3.368606in}}{\pgfqpoint{3.006812in}{3.364656in}}{\pgfqpoint{3.006812in}{3.360538in}}%
\pgfpathcurveto{\pgfqpoint{3.006812in}{3.356420in}}{\pgfqpoint{3.008449in}{3.352470in}}{\pgfqpoint{3.011361in}{3.349558in}}%
\pgfpathcurveto{\pgfqpoint{3.014273in}{3.346646in}}{\pgfqpoint{3.018223in}{3.345010in}}{\pgfqpoint{3.022341in}{3.345010in}}%
\pgfpathclose%
\pgfusepath{fill}%
\end{pgfscope}%
\begin{pgfscope}%
\pgfpathrectangle{\pgfqpoint{0.887500in}{0.275000in}}{\pgfqpoint{4.225000in}{4.225000in}}%
\pgfusepath{clip}%
\pgfsetbuttcap%
\pgfsetroundjoin%
\definecolor{currentfill}{rgb}{0.000000,0.000000,0.000000}%
\pgfsetfillcolor{currentfill}%
\pgfsetfillopacity{0.800000}%
\pgfsetlinewidth{0.000000pt}%
\definecolor{currentstroke}{rgb}{0.000000,0.000000,0.000000}%
\pgfsetstrokecolor{currentstroke}%
\pgfsetstrokeopacity{0.800000}%
\pgfsetdash{}{0pt}%
\pgfpathmoveto{\pgfqpoint{2.235828in}{2.448085in}}%
\pgfpathcurveto{\pgfqpoint{2.239946in}{2.448085in}}{\pgfqpoint{2.243896in}{2.449721in}}{\pgfqpoint{2.246808in}{2.452633in}}%
\pgfpathcurveto{\pgfqpoint{2.249720in}{2.455545in}}{\pgfqpoint{2.251356in}{2.459495in}}{\pgfqpoint{2.251356in}{2.463614in}}%
\pgfpathcurveto{\pgfqpoint{2.251356in}{2.467732in}}{\pgfqpoint{2.249720in}{2.471682in}}{\pgfqpoint{2.246808in}{2.474594in}}%
\pgfpathcurveto{\pgfqpoint{2.243896in}{2.477506in}}{\pgfqpoint{2.239946in}{2.479142in}}{\pgfqpoint{2.235828in}{2.479142in}}%
\pgfpathcurveto{\pgfqpoint{2.231710in}{2.479142in}}{\pgfqpoint{2.227760in}{2.477506in}}{\pgfqpoint{2.224848in}{2.474594in}}%
\pgfpathcurveto{\pgfqpoint{2.221936in}{2.471682in}}{\pgfqpoint{2.220300in}{2.467732in}}{\pgfqpoint{2.220300in}{2.463614in}}%
\pgfpathcurveto{\pgfqpoint{2.220300in}{2.459495in}}{\pgfqpoint{2.221936in}{2.455545in}}{\pgfqpoint{2.224848in}{2.452633in}}%
\pgfpathcurveto{\pgfqpoint{2.227760in}{2.449721in}}{\pgfqpoint{2.231710in}{2.448085in}}{\pgfqpoint{2.235828in}{2.448085in}}%
\pgfpathclose%
\pgfusepath{fill}%
\end{pgfscope}%
\begin{pgfscope}%
\pgfpathrectangle{\pgfqpoint{0.887500in}{0.275000in}}{\pgfqpoint{4.225000in}{4.225000in}}%
\pgfusepath{clip}%
\pgfsetbuttcap%
\pgfsetroundjoin%
\definecolor{currentfill}{rgb}{0.000000,0.000000,0.000000}%
\pgfsetfillcolor{currentfill}%
\pgfsetfillopacity{0.800000}%
\pgfsetlinewidth{0.000000pt}%
\definecolor{currentstroke}{rgb}{0.000000,0.000000,0.000000}%
\pgfsetstrokecolor{currentstroke}%
\pgfsetstrokeopacity{0.800000}%
\pgfsetdash{}{0pt}%
\pgfpathmoveto{\pgfqpoint{4.085725in}{2.787579in}}%
\pgfpathcurveto{\pgfqpoint{4.089843in}{2.787579in}}{\pgfqpoint{4.093793in}{2.789215in}}{\pgfqpoint{4.096705in}{2.792127in}}%
\pgfpathcurveto{\pgfqpoint{4.099617in}{2.795039in}}{\pgfqpoint{4.101253in}{2.798989in}}{\pgfqpoint{4.101253in}{2.803107in}}%
\pgfpathcurveto{\pgfqpoint{4.101253in}{2.807225in}}{\pgfqpoint{4.099617in}{2.811175in}}{\pgfqpoint{4.096705in}{2.814087in}}%
\pgfpathcurveto{\pgfqpoint{4.093793in}{2.816999in}}{\pgfqpoint{4.089843in}{2.818635in}}{\pgfqpoint{4.085725in}{2.818635in}}%
\pgfpathcurveto{\pgfqpoint{4.081607in}{2.818635in}}{\pgfqpoint{4.077657in}{2.816999in}}{\pgfqpoint{4.074745in}{2.814087in}}%
\pgfpathcurveto{\pgfqpoint{4.071833in}{2.811175in}}{\pgfqpoint{4.070197in}{2.807225in}}{\pgfqpoint{4.070197in}{2.803107in}}%
\pgfpathcurveto{\pgfqpoint{4.070197in}{2.798989in}}{\pgfqpoint{4.071833in}{2.795039in}}{\pgfqpoint{4.074745in}{2.792127in}}%
\pgfpathcurveto{\pgfqpoint{4.077657in}{2.789215in}}{\pgfqpoint{4.081607in}{2.787579in}}{\pgfqpoint{4.085725in}{2.787579in}}%
\pgfpathclose%
\pgfusepath{fill}%
\end{pgfscope}%
\begin{pgfscope}%
\pgfpathrectangle{\pgfqpoint{0.887500in}{0.275000in}}{\pgfqpoint{4.225000in}{4.225000in}}%
\pgfusepath{clip}%
\pgfsetbuttcap%
\pgfsetroundjoin%
\definecolor{currentfill}{rgb}{0.000000,0.000000,0.000000}%
\pgfsetfillcolor{currentfill}%
\pgfsetfillopacity{0.800000}%
\pgfsetlinewidth{0.000000pt}%
\definecolor{currentstroke}{rgb}{0.000000,0.000000,0.000000}%
\pgfsetstrokecolor{currentstroke}%
\pgfsetstrokeopacity{0.800000}%
\pgfsetdash{}{0pt}%
\pgfpathmoveto{\pgfqpoint{2.974820in}{3.013288in}}%
\pgfpathcurveto{\pgfqpoint{2.978938in}{3.013288in}}{\pgfqpoint{2.982888in}{3.014924in}}{\pgfqpoint{2.985800in}{3.017836in}}%
\pgfpathcurveto{\pgfqpoint{2.988712in}{3.020748in}}{\pgfqpoint{2.990348in}{3.024698in}}{\pgfqpoint{2.990348in}{3.028816in}}%
\pgfpathcurveto{\pgfqpoint{2.990348in}{3.032934in}}{\pgfqpoint{2.988712in}{3.036884in}}{\pgfqpoint{2.985800in}{3.039796in}}%
\pgfpathcurveto{\pgfqpoint{2.982888in}{3.042708in}}{\pgfqpoint{2.978938in}{3.044344in}}{\pgfqpoint{2.974820in}{3.044344in}}%
\pgfpathcurveto{\pgfqpoint{2.970701in}{3.044344in}}{\pgfqpoint{2.966751in}{3.042708in}}{\pgfqpoint{2.963839in}{3.039796in}}%
\pgfpathcurveto{\pgfqpoint{2.960928in}{3.036884in}}{\pgfqpoint{2.959291in}{3.032934in}}{\pgfqpoint{2.959291in}{3.028816in}}%
\pgfpathcurveto{\pgfqpoint{2.959291in}{3.024698in}}{\pgfqpoint{2.960928in}{3.020748in}}{\pgfqpoint{2.963839in}{3.017836in}}%
\pgfpathcurveto{\pgfqpoint{2.966751in}{3.014924in}}{\pgfqpoint{2.970701in}{3.013288in}}{\pgfqpoint{2.974820in}{3.013288in}}%
\pgfpathclose%
\pgfusepath{fill}%
\end{pgfscope}%
\begin{pgfscope}%
\pgfpathrectangle{\pgfqpoint{0.887500in}{0.275000in}}{\pgfqpoint{4.225000in}{4.225000in}}%
\pgfusepath{clip}%
\pgfsetbuttcap%
\pgfsetroundjoin%
\definecolor{currentfill}{rgb}{0.000000,0.000000,0.000000}%
\pgfsetfillcolor{currentfill}%
\pgfsetfillopacity{0.800000}%
\pgfsetlinewidth{0.000000pt}%
\definecolor{currentstroke}{rgb}{0.000000,0.000000,0.000000}%
\pgfsetstrokecolor{currentstroke}%
\pgfsetstrokeopacity{0.800000}%
\pgfsetdash{}{0pt}%
\pgfpathmoveto{\pgfqpoint{1.768590in}{2.521401in}}%
\pgfpathcurveto{\pgfqpoint{1.772708in}{2.521401in}}{\pgfqpoint{1.776658in}{2.523037in}}{\pgfqpoint{1.779570in}{2.525949in}}%
\pgfpathcurveto{\pgfqpoint{1.782482in}{2.528861in}}{\pgfqpoint{1.784118in}{2.532811in}}{\pgfqpoint{1.784118in}{2.536929in}}%
\pgfpathcurveto{\pgfqpoint{1.784118in}{2.541047in}}{\pgfqpoint{1.782482in}{2.544997in}}{\pgfqpoint{1.779570in}{2.547909in}}%
\pgfpathcurveto{\pgfqpoint{1.776658in}{2.550821in}}{\pgfqpoint{1.772708in}{2.552457in}}{\pgfqpoint{1.768590in}{2.552457in}}%
\pgfpathcurveto{\pgfqpoint{1.764472in}{2.552457in}}{\pgfqpoint{1.760522in}{2.550821in}}{\pgfqpoint{1.757610in}{2.547909in}}%
\pgfpathcurveto{\pgfqpoint{1.754698in}{2.544997in}}{\pgfqpoint{1.753062in}{2.541047in}}{\pgfqpoint{1.753062in}{2.536929in}}%
\pgfpathcurveto{\pgfqpoint{1.753062in}{2.532811in}}{\pgfqpoint{1.754698in}{2.528861in}}{\pgfqpoint{1.757610in}{2.525949in}}%
\pgfpathcurveto{\pgfqpoint{1.760522in}{2.523037in}}{\pgfqpoint{1.764472in}{2.521401in}}{\pgfqpoint{1.768590in}{2.521401in}}%
\pgfpathclose%
\pgfusepath{fill}%
\end{pgfscope}%
\begin{pgfscope}%
\pgfpathrectangle{\pgfqpoint{0.887500in}{0.275000in}}{\pgfqpoint{4.225000in}{4.225000in}}%
\pgfusepath{clip}%
\pgfsetbuttcap%
\pgfsetroundjoin%
\definecolor{currentfill}{rgb}{0.000000,0.000000,0.000000}%
\pgfsetfillcolor{currentfill}%
\pgfsetfillopacity{0.800000}%
\pgfsetlinewidth{0.000000pt}%
\definecolor{currentstroke}{rgb}{0.000000,0.000000,0.000000}%
\pgfsetstrokecolor{currentstroke}%
\pgfsetstrokeopacity{0.800000}%
\pgfsetdash{}{0pt}%
\pgfpathmoveto{\pgfqpoint{3.908781in}{2.896138in}}%
\pgfpathcurveto{\pgfqpoint{3.912899in}{2.896138in}}{\pgfqpoint{3.916849in}{2.897775in}}{\pgfqpoint{3.919761in}{2.900687in}}%
\pgfpathcurveto{\pgfqpoint{3.922673in}{2.903599in}}{\pgfqpoint{3.924310in}{2.907549in}}{\pgfqpoint{3.924310in}{2.911667in}}%
\pgfpathcurveto{\pgfqpoint{3.924310in}{2.915785in}}{\pgfqpoint{3.922673in}{2.919735in}}{\pgfqpoint{3.919761in}{2.922647in}}%
\pgfpathcurveto{\pgfqpoint{3.916849in}{2.925559in}}{\pgfqpoint{3.912899in}{2.927195in}}{\pgfqpoint{3.908781in}{2.927195in}}%
\pgfpathcurveto{\pgfqpoint{3.904663in}{2.927195in}}{\pgfqpoint{3.900713in}{2.925559in}}{\pgfqpoint{3.897801in}{2.922647in}}%
\pgfpathcurveto{\pgfqpoint{3.894889in}{2.919735in}}{\pgfqpoint{3.893253in}{2.915785in}}{\pgfqpoint{3.893253in}{2.911667in}}%
\pgfpathcurveto{\pgfqpoint{3.893253in}{2.907549in}}{\pgfqpoint{3.894889in}{2.903599in}}{\pgfqpoint{3.897801in}{2.900687in}}%
\pgfpathcurveto{\pgfqpoint{3.900713in}{2.897775in}}{\pgfqpoint{3.904663in}{2.896138in}}{\pgfqpoint{3.908781in}{2.896138in}}%
\pgfpathclose%
\pgfusepath{fill}%
\end{pgfscope}%
\begin{pgfscope}%
\pgfpathrectangle{\pgfqpoint{0.887500in}{0.275000in}}{\pgfqpoint{4.225000in}{4.225000in}}%
\pgfusepath{clip}%
\pgfsetbuttcap%
\pgfsetroundjoin%
\definecolor{currentfill}{rgb}{0.000000,0.000000,0.000000}%
\pgfsetfillcolor{currentfill}%
\pgfsetfillopacity{0.800000}%
\pgfsetlinewidth{0.000000pt}%
\definecolor{currentstroke}{rgb}{0.000000,0.000000,0.000000}%
\pgfsetstrokecolor{currentstroke}%
\pgfsetstrokeopacity{0.800000}%
\pgfsetdash{}{0pt}%
\pgfpathmoveto{\pgfqpoint{2.879783in}{2.311666in}}%
\pgfpathcurveto{\pgfqpoint{2.883901in}{2.311666in}}{\pgfqpoint{2.887851in}{2.313303in}}{\pgfqpoint{2.890763in}{2.316215in}}%
\pgfpathcurveto{\pgfqpoint{2.893675in}{2.319127in}}{\pgfqpoint{2.895311in}{2.323077in}}{\pgfqpoint{2.895311in}{2.327195in}}%
\pgfpathcurveto{\pgfqpoint{2.895311in}{2.331313in}}{\pgfqpoint{2.893675in}{2.335263in}}{\pgfqpoint{2.890763in}{2.338175in}}%
\pgfpathcurveto{\pgfqpoint{2.887851in}{2.341087in}}{\pgfqpoint{2.883901in}{2.342723in}}{\pgfqpoint{2.879783in}{2.342723in}}%
\pgfpathcurveto{\pgfqpoint{2.875665in}{2.342723in}}{\pgfqpoint{2.871715in}{2.341087in}}{\pgfqpoint{2.868803in}{2.338175in}}%
\pgfpathcurveto{\pgfqpoint{2.865891in}{2.335263in}}{\pgfqpoint{2.864255in}{2.331313in}}{\pgfqpoint{2.864255in}{2.327195in}}%
\pgfpathcurveto{\pgfqpoint{2.864255in}{2.323077in}}{\pgfqpoint{2.865891in}{2.319127in}}{\pgfqpoint{2.868803in}{2.316215in}}%
\pgfpathcurveto{\pgfqpoint{2.871715in}{2.313303in}}{\pgfqpoint{2.875665in}{2.311666in}}{\pgfqpoint{2.879783in}{2.311666in}}%
\pgfpathclose%
\pgfusepath{fill}%
\end{pgfscope}%
\begin{pgfscope}%
\pgfpathrectangle{\pgfqpoint{0.887500in}{0.275000in}}{\pgfqpoint{4.225000in}{4.225000in}}%
\pgfusepath{clip}%
\pgfsetbuttcap%
\pgfsetroundjoin%
\definecolor{currentfill}{rgb}{0.000000,0.000000,0.000000}%
\pgfsetfillcolor{currentfill}%
\pgfsetfillopacity{0.800000}%
\pgfsetlinewidth{0.000000pt}%
\definecolor{currentstroke}{rgb}{0.000000,0.000000,0.000000}%
\pgfsetstrokecolor{currentstroke}%
\pgfsetstrokeopacity{0.800000}%
\pgfsetdash{}{0pt}%
\pgfpathmoveto{\pgfqpoint{3.731672in}{3.002081in}}%
\pgfpathcurveto{\pgfqpoint{3.735791in}{3.002081in}}{\pgfqpoint{3.739741in}{3.003717in}}{\pgfqpoint{3.742653in}{3.006629in}}%
\pgfpathcurveto{\pgfqpoint{3.745565in}{3.009541in}}{\pgfqpoint{3.747201in}{3.013491in}}{\pgfqpoint{3.747201in}{3.017609in}}%
\pgfpathcurveto{\pgfqpoint{3.747201in}{3.021727in}}{\pgfqpoint{3.745565in}{3.025677in}}{\pgfqpoint{3.742653in}{3.028589in}}%
\pgfpathcurveto{\pgfqpoint{3.739741in}{3.031501in}}{\pgfqpoint{3.735791in}{3.033137in}}{\pgfqpoint{3.731672in}{3.033137in}}%
\pgfpathcurveto{\pgfqpoint{3.727554in}{3.033137in}}{\pgfqpoint{3.723604in}{3.031501in}}{\pgfqpoint{3.720692in}{3.028589in}}%
\pgfpathcurveto{\pgfqpoint{3.717780in}{3.025677in}}{\pgfqpoint{3.716144in}{3.021727in}}{\pgfqpoint{3.716144in}{3.017609in}}%
\pgfpathcurveto{\pgfqpoint{3.716144in}{3.013491in}}{\pgfqpoint{3.717780in}{3.009541in}}{\pgfqpoint{3.720692in}{3.006629in}}%
\pgfpathcurveto{\pgfqpoint{3.723604in}{3.003717in}}{\pgfqpoint{3.727554in}{3.002081in}}{\pgfqpoint{3.731672in}{3.002081in}}%
\pgfpathclose%
\pgfusepath{fill}%
\end{pgfscope}%
\begin{pgfscope}%
\pgfpathrectangle{\pgfqpoint{0.887500in}{0.275000in}}{\pgfqpoint{4.225000in}{4.225000in}}%
\pgfusepath{clip}%
\pgfsetbuttcap%
\pgfsetroundjoin%
\definecolor{currentfill}{rgb}{0.000000,0.000000,0.000000}%
\pgfsetfillcolor{currentfill}%
\pgfsetfillopacity{0.800000}%
\pgfsetlinewidth{0.000000pt}%
\definecolor{currentstroke}{rgb}{0.000000,0.000000,0.000000}%
\pgfsetstrokecolor{currentstroke}%
\pgfsetstrokeopacity{0.800000}%
\pgfsetdash{}{0pt}%
\pgfpathmoveto{\pgfqpoint{3.377023in}{3.196248in}}%
\pgfpathcurveto{\pgfqpoint{3.381141in}{3.196248in}}{\pgfqpoint{3.385091in}{3.197884in}}{\pgfqpoint{3.388003in}{3.200796in}}%
\pgfpathcurveto{\pgfqpoint{3.390915in}{3.203708in}}{\pgfqpoint{3.392551in}{3.207658in}}{\pgfqpoint{3.392551in}{3.211776in}}%
\pgfpathcurveto{\pgfqpoint{3.392551in}{3.215895in}}{\pgfqpoint{3.390915in}{3.219845in}}{\pgfqpoint{3.388003in}{3.222757in}}%
\pgfpathcurveto{\pgfqpoint{3.385091in}{3.225668in}}{\pgfqpoint{3.381141in}{3.227305in}}{\pgfqpoint{3.377023in}{3.227305in}}%
\pgfpathcurveto{\pgfqpoint{3.372905in}{3.227305in}}{\pgfqpoint{3.368955in}{3.225668in}}{\pgfqpoint{3.366043in}{3.222757in}}%
\pgfpathcurveto{\pgfqpoint{3.363131in}{3.219845in}}{\pgfqpoint{3.361495in}{3.215895in}}{\pgfqpoint{3.361495in}{3.211776in}}%
\pgfpathcurveto{\pgfqpoint{3.361495in}{3.207658in}}{\pgfqpoint{3.363131in}{3.203708in}}{\pgfqpoint{3.366043in}{3.200796in}}%
\pgfpathcurveto{\pgfqpoint{3.368955in}{3.197884in}}{\pgfqpoint{3.372905in}{3.196248in}}{\pgfqpoint{3.377023in}{3.196248in}}%
\pgfpathclose%
\pgfusepath{fill}%
\end{pgfscope}%
\begin{pgfscope}%
\pgfpathrectangle{\pgfqpoint{0.887500in}{0.275000in}}{\pgfqpoint{4.225000in}{4.225000in}}%
\pgfusepath{clip}%
\pgfsetbuttcap%
\pgfsetroundjoin%
\definecolor{currentfill}{rgb}{0.000000,0.000000,0.000000}%
\pgfsetfillcolor{currentfill}%
\pgfsetfillopacity{0.800000}%
\pgfsetlinewidth{0.000000pt}%
\definecolor{currentstroke}{rgb}{0.000000,0.000000,0.000000}%
\pgfsetstrokecolor{currentstroke}%
\pgfsetstrokeopacity{0.800000}%
\pgfsetdash{}{0pt}%
\pgfpathmoveto{\pgfqpoint{3.554398in}{3.102510in}}%
\pgfpathcurveto{\pgfqpoint{3.558516in}{3.102510in}}{\pgfqpoint{3.562466in}{3.104146in}}{\pgfqpoint{3.565378in}{3.107058in}}%
\pgfpathcurveto{\pgfqpoint{3.568290in}{3.109970in}}{\pgfqpoint{3.569926in}{3.113920in}}{\pgfqpoint{3.569926in}{3.118038in}}%
\pgfpathcurveto{\pgfqpoint{3.569926in}{3.122156in}}{\pgfqpoint{3.568290in}{3.126106in}}{\pgfqpoint{3.565378in}{3.129018in}}%
\pgfpathcurveto{\pgfqpoint{3.562466in}{3.131930in}}{\pgfqpoint{3.558516in}{3.133566in}}{\pgfqpoint{3.554398in}{3.133566in}}%
\pgfpathcurveto{\pgfqpoint{3.550280in}{3.133566in}}{\pgfqpoint{3.546330in}{3.131930in}}{\pgfqpoint{3.543418in}{3.129018in}}%
\pgfpathcurveto{\pgfqpoint{3.540506in}{3.126106in}}{\pgfqpoint{3.538870in}{3.122156in}}{\pgfqpoint{3.538870in}{3.118038in}}%
\pgfpathcurveto{\pgfqpoint{3.538870in}{3.113920in}}{\pgfqpoint{3.540506in}{3.109970in}}{\pgfqpoint{3.543418in}{3.107058in}}%
\pgfpathcurveto{\pgfqpoint{3.546330in}{3.104146in}}{\pgfqpoint{3.550280in}{3.102510in}}{\pgfqpoint{3.554398in}{3.102510in}}%
\pgfpathclose%
\pgfusepath{fill}%
\end{pgfscope}%
\begin{pgfscope}%
\pgfpathrectangle{\pgfqpoint{0.887500in}{0.275000in}}{\pgfqpoint{4.225000in}{4.225000in}}%
\pgfusepath{clip}%
\pgfsetbuttcap%
\pgfsetroundjoin%
\definecolor{currentfill}{rgb}{0.000000,0.000000,0.000000}%
\pgfsetfillcolor{currentfill}%
\pgfsetfillopacity{0.800000}%
\pgfsetlinewidth{0.000000pt}%
\definecolor{currentstroke}{rgb}{0.000000,0.000000,0.000000}%
\pgfsetstrokecolor{currentstroke}%
\pgfsetstrokeopacity{0.800000}%
\pgfsetdash{}{0pt}%
\pgfpathmoveto{\pgfqpoint{4.506481in}{2.411623in}}%
\pgfpathcurveto{\pgfqpoint{4.510599in}{2.411623in}}{\pgfqpoint{4.514549in}{2.413259in}}{\pgfqpoint{4.517461in}{2.416171in}}%
\pgfpathcurveto{\pgfqpoint{4.520373in}{2.419083in}}{\pgfqpoint{4.522009in}{2.423033in}}{\pgfqpoint{4.522009in}{2.427151in}}%
\pgfpathcurveto{\pgfqpoint{4.522009in}{2.431269in}}{\pgfqpoint{4.520373in}{2.435219in}}{\pgfqpoint{4.517461in}{2.438131in}}%
\pgfpathcurveto{\pgfqpoint{4.514549in}{2.441043in}}{\pgfqpoint{4.510599in}{2.442679in}}{\pgfqpoint{4.506481in}{2.442679in}}%
\pgfpathcurveto{\pgfqpoint{4.502363in}{2.442679in}}{\pgfqpoint{4.498413in}{2.441043in}}{\pgfqpoint{4.495501in}{2.438131in}}%
\pgfpathcurveto{\pgfqpoint{4.492589in}{2.435219in}}{\pgfqpoint{4.490952in}{2.431269in}}{\pgfqpoint{4.490952in}{2.427151in}}%
\pgfpathcurveto{\pgfqpoint{4.490952in}{2.423033in}}{\pgfqpoint{4.492589in}{2.419083in}}{\pgfqpoint{4.495501in}{2.416171in}}%
\pgfpathcurveto{\pgfqpoint{4.498413in}{2.413259in}}{\pgfqpoint{4.502363in}{2.411623in}}{\pgfqpoint{4.506481in}{2.411623in}}%
\pgfpathclose%
\pgfusepath{fill}%
\end{pgfscope}%
\begin{pgfscope}%
\pgfpathrectangle{\pgfqpoint{0.887500in}{0.275000in}}{\pgfqpoint{4.225000in}{4.225000in}}%
\pgfusepath{clip}%
\pgfsetbuttcap%
\pgfsetroundjoin%
\definecolor{currentfill}{rgb}{0.000000,0.000000,0.000000}%
\pgfsetfillcolor{currentfill}%
\pgfsetfillopacity{0.800000}%
\pgfsetlinewidth{0.000000pt}%
\definecolor{currentstroke}{rgb}{0.000000,0.000000,0.000000}%
\pgfsetstrokecolor{currentstroke}%
\pgfsetstrokeopacity{0.800000}%
\pgfsetdash{}{0pt}%
\pgfpathmoveto{\pgfqpoint{2.412400in}{2.393265in}}%
\pgfpathcurveto{\pgfqpoint{2.416518in}{2.393265in}}{\pgfqpoint{2.420468in}{2.394901in}}{\pgfqpoint{2.423380in}{2.397813in}}%
\pgfpathcurveto{\pgfqpoint{2.426292in}{2.400725in}}{\pgfqpoint{2.427928in}{2.404675in}}{\pgfqpoint{2.427928in}{2.408793in}}%
\pgfpathcurveto{\pgfqpoint{2.427928in}{2.412912in}}{\pgfqpoint{2.426292in}{2.416862in}}{\pgfqpoint{2.423380in}{2.419773in}}%
\pgfpathcurveto{\pgfqpoint{2.420468in}{2.422685in}}{\pgfqpoint{2.416518in}{2.424322in}}{\pgfqpoint{2.412400in}{2.424322in}}%
\pgfpathcurveto{\pgfqpoint{2.408282in}{2.424322in}}{\pgfqpoint{2.404332in}{2.422685in}}{\pgfqpoint{2.401420in}{2.419773in}}%
\pgfpathcurveto{\pgfqpoint{2.398508in}{2.416862in}}{\pgfqpoint{2.396872in}{2.412912in}}{\pgfqpoint{2.396872in}{2.408793in}}%
\pgfpathcurveto{\pgfqpoint{2.396872in}{2.404675in}}{\pgfqpoint{2.398508in}{2.400725in}}{\pgfqpoint{2.401420in}{2.397813in}}%
\pgfpathcurveto{\pgfqpoint{2.404332in}{2.394901in}}{\pgfqpoint{2.408282in}{2.393265in}}{\pgfqpoint{2.412400in}{2.393265in}}%
\pgfpathclose%
\pgfusepath{fill}%
\end{pgfscope}%
\begin{pgfscope}%
\pgfpathrectangle{\pgfqpoint{0.887500in}{0.275000in}}{\pgfqpoint{4.225000in}{4.225000in}}%
\pgfusepath{clip}%
\pgfsetbuttcap%
\pgfsetroundjoin%
\definecolor{currentfill}{rgb}{0.000000,0.000000,0.000000}%
\pgfsetfillcolor{currentfill}%
\pgfsetfillopacity{0.800000}%
\pgfsetlinewidth{0.000000pt}%
\definecolor{currentstroke}{rgb}{0.000000,0.000000,0.000000}%
\pgfsetstrokecolor{currentstroke}%
\pgfsetstrokeopacity{0.800000}%
\pgfsetdash{}{0pt}%
\pgfpathmoveto{\pgfqpoint{1.944683in}{2.468865in}}%
\pgfpathcurveto{\pgfqpoint{1.948801in}{2.468865in}}{\pgfqpoint{1.952751in}{2.470502in}}{\pgfqpoint{1.955663in}{2.473413in}}%
\pgfpathcurveto{\pgfqpoint{1.958575in}{2.476325in}}{\pgfqpoint{1.960211in}{2.480275in}}{\pgfqpoint{1.960211in}{2.484394in}}%
\pgfpathcurveto{\pgfqpoint{1.960211in}{2.488512in}}{\pgfqpoint{1.958575in}{2.492462in}}{\pgfqpoint{1.955663in}{2.495374in}}%
\pgfpathcurveto{\pgfqpoint{1.952751in}{2.498286in}}{\pgfqpoint{1.948801in}{2.499922in}}{\pgfqpoint{1.944683in}{2.499922in}}%
\pgfpathcurveto{\pgfqpoint{1.940565in}{2.499922in}}{\pgfqpoint{1.936615in}{2.498286in}}{\pgfqpoint{1.933703in}{2.495374in}}%
\pgfpathcurveto{\pgfqpoint{1.930791in}{2.492462in}}{\pgfqpoint{1.929155in}{2.488512in}}{\pgfqpoint{1.929155in}{2.484394in}}%
\pgfpathcurveto{\pgfqpoint{1.929155in}{2.480275in}}{\pgfqpoint{1.930791in}{2.476325in}}{\pgfqpoint{1.933703in}{2.473413in}}%
\pgfpathcurveto{\pgfqpoint{1.936615in}{2.470502in}}{\pgfqpoint{1.940565in}{2.468865in}}{\pgfqpoint{1.944683in}{2.468865in}}%
\pgfpathclose%
\pgfusepath{fill}%
\end{pgfscope}%
\begin{pgfscope}%
\pgfpathrectangle{\pgfqpoint{0.887500in}{0.275000in}}{\pgfqpoint{4.225000in}{4.225000in}}%
\pgfusepath{clip}%
\pgfsetbuttcap%
\pgfsetroundjoin%
\definecolor{currentfill}{rgb}{0.000000,0.000000,0.000000}%
\pgfsetfillcolor{currentfill}%
\pgfsetfillopacity{0.800000}%
\pgfsetlinewidth{0.000000pt}%
\definecolor{currentstroke}{rgb}{0.000000,0.000000,0.000000}%
\pgfsetstrokecolor{currentstroke}%
\pgfsetstrokeopacity{0.800000}%
\pgfsetdash{}{0pt}%
\pgfpathmoveto{\pgfqpoint{4.329700in}{2.526170in}}%
\pgfpathcurveto{\pgfqpoint{4.333818in}{2.526170in}}{\pgfqpoint{4.337768in}{2.527806in}}{\pgfqpoint{4.340680in}{2.530718in}}%
\pgfpathcurveto{\pgfqpoint{4.343592in}{2.533630in}}{\pgfqpoint{4.345228in}{2.537580in}}{\pgfqpoint{4.345228in}{2.541698in}}%
\pgfpathcurveto{\pgfqpoint{4.345228in}{2.545816in}}{\pgfqpoint{4.343592in}{2.549766in}}{\pgfqpoint{4.340680in}{2.552678in}}%
\pgfpathcurveto{\pgfqpoint{4.337768in}{2.555590in}}{\pgfqpoint{4.333818in}{2.557226in}}{\pgfqpoint{4.329700in}{2.557226in}}%
\pgfpathcurveto{\pgfqpoint{4.325581in}{2.557226in}}{\pgfqpoint{4.321631in}{2.555590in}}{\pgfqpoint{4.318719in}{2.552678in}}%
\pgfpathcurveto{\pgfqpoint{4.315807in}{2.549766in}}{\pgfqpoint{4.314171in}{2.545816in}}{\pgfqpoint{4.314171in}{2.541698in}}%
\pgfpathcurveto{\pgfqpoint{4.314171in}{2.537580in}}{\pgfqpoint{4.315807in}{2.533630in}}{\pgfqpoint{4.318719in}{2.530718in}}%
\pgfpathcurveto{\pgfqpoint{4.321631in}{2.527806in}}{\pgfqpoint{4.325581in}{2.526170in}}{\pgfqpoint{4.329700in}{2.526170in}}%
\pgfpathclose%
\pgfusepath{fill}%
\end{pgfscope}%
\begin{pgfscope}%
\pgfpathrectangle{\pgfqpoint{0.887500in}{0.275000in}}{\pgfqpoint{4.225000in}{4.225000in}}%
\pgfusepath{clip}%
\pgfsetbuttcap%
\pgfsetroundjoin%
\definecolor{currentfill}{rgb}{0.000000,0.000000,0.000000}%
\pgfsetfillcolor{currentfill}%
\pgfsetfillopacity{0.800000}%
\pgfsetlinewidth{0.000000pt}%
\definecolor{currentstroke}{rgb}{0.000000,0.000000,0.000000}%
\pgfsetstrokecolor{currentstroke}%
\pgfsetstrokeopacity{0.800000}%
\pgfsetdash{}{0pt}%
\pgfpathmoveto{\pgfqpoint{1.476812in}{2.539284in}}%
\pgfpathcurveto{\pgfqpoint{1.480930in}{2.539284in}}{\pgfqpoint{1.484880in}{2.540920in}}{\pgfqpoint{1.487792in}{2.543832in}}%
\pgfpathcurveto{\pgfqpoint{1.490704in}{2.546744in}}{\pgfqpoint{1.492340in}{2.550694in}}{\pgfqpoint{1.492340in}{2.554812in}}%
\pgfpathcurveto{\pgfqpoint{1.492340in}{2.558931in}}{\pgfqpoint{1.490704in}{2.562881in}}{\pgfqpoint{1.487792in}{2.565793in}}%
\pgfpathcurveto{\pgfqpoint{1.484880in}{2.568705in}}{\pgfqpoint{1.480930in}{2.570341in}}{\pgfqpoint{1.476812in}{2.570341in}}%
\pgfpathcurveto{\pgfqpoint{1.472694in}{2.570341in}}{\pgfqpoint{1.468744in}{2.568705in}}{\pgfqpoint{1.465832in}{2.565793in}}%
\pgfpathcurveto{\pgfqpoint{1.462920in}{2.562881in}}{\pgfqpoint{1.461283in}{2.558931in}}{\pgfqpoint{1.461283in}{2.554812in}}%
\pgfpathcurveto{\pgfqpoint{1.461283in}{2.550694in}}{\pgfqpoint{1.462920in}{2.546744in}}{\pgfqpoint{1.465832in}{2.543832in}}%
\pgfpathcurveto{\pgfqpoint{1.468744in}{2.540920in}}{\pgfqpoint{1.472694in}{2.539284in}}{\pgfqpoint{1.476812in}{2.539284in}}%
\pgfpathclose%
\pgfusepath{fill}%
\end{pgfscope}%
\begin{pgfscope}%
\pgfpathrectangle{\pgfqpoint{0.887500in}{0.275000in}}{\pgfqpoint{4.225000in}{4.225000in}}%
\pgfusepath{clip}%
\pgfsetbuttcap%
\pgfsetroundjoin%
\definecolor{currentfill}{rgb}{0.000000,0.000000,0.000000}%
\pgfsetfillcolor{currentfill}%
\pgfsetfillopacity{0.800000}%
\pgfsetlinewidth{0.000000pt}%
\definecolor{currentstroke}{rgb}{0.000000,0.000000,0.000000}%
\pgfsetstrokecolor{currentstroke}%
\pgfsetstrokeopacity{0.800000}%
\pgfsetdash{}{0pt}%
\pgfpathmoveto{\pgfqpoint{4.152749in}{2.641001in}}%
\pgfpathcurveto{\pgfqpoint{4.156867in}{2.641001in}}{\pgfqpoint{4.160817in}{2.642638in}}{\pgfqpoint{4.163729in}{2.645549in}}%
\pgfpathcurveto{\pgfqpoint{4.166641in}{2.648461in}}{\pgfqpoint{4.168278in}{2.652411in}}{\pgfqpoint{4.168278in}{2.656530in}}%
\pgfpathcurveto{\pgfqpoint{4.168278in}{2.660648in}}{\pgfqpoint{4.166641in}{2.664598in}}{\pgfqpoint{4.163729in}{2.667510in}}%
\pgfpathcurveto{\pgfqpoint{4.160817in}{2.670422in}}{\pgfqpoint{4.156867in}{2.672058in}}{\pgfqpoint{4.152749in}{2.672058in}}%
\pgfpathcurveto{\pgfqpoint{4.148631in}{2.672058in}}{\pgfqpoint{4.144681in}{2.670422in}}{\pgfqpoint{4.141769in}{2.667510in}}%
\pgfpathcurveto{\pgfqpoint{4.138857in}{2.664598in}}{\pgfqpoint{4.137221in}{2.660648in}}{\pgfqpoint{4.137221in}{2.656530in}}%
\pgfpathcurveto{\pgfqpoint{4.137221in}{2.652411in}}{\pgfqpoint{4.138857in}{2.648461in}}{\pgfqpoint{4.141769in}{2.645549in}}%
\pgfpathcurveto{\pgfqpoint{4.144681in}{2.642638in}}{\pgfqpoint{4.148631in}{2.641001in}}{\pgfqpoint{4.152749in}{2.641001in}}%
\pgfpathclose%
\pgfusepath{fill}%
\end{pgfscope}%
\begin{pgfscope}%
\pgfpathrectangle{\pgfqpoint{0.887500in}{0.275000in}}{\pgfqpoint{4.225000in}{4.225000in}}%
\pgfusepath{clip}%
\pgfsetbuttcap%
\pgfsetroundjoin%
\definecolor{currentfill}{rgb}{0.000000,0.000000,0.000000}%
\pgfsetfillcolor{currentfill}%
\pgfsetfillopacity{0.800000}%
\pgfsetlinewidth{0.000000pt}%
\definecolor{currentstroke}{rgb}{0.000000,0.000000,0.000000}%
\pgfsetstrokecolor{currentstroke}%
\pgfsetstrokeopacity{0.800000}%
\pgfsetdash{}{0pt}%
\pgfpathmoveto{\pgfqpoint{2.589345in}{2.337222in}}%
\pgfpathcurveto{\pgfqpoint{2.593463in}{2.337222in}}{\pgfqpoint{2.597413in}{2.338858in}}{\pgfqpoint{2.600325in}{2.341770in}}%
\pgfpathcurveto{\pgfqpoint{2.603237in}{2.344682in}}{\pgfqpoint{2.604873in}{2.348632in}}{\pgfqpoint{2.604873in}{2.352750in}}%
\pgfpathcurveto{\pgfqpoint{2.604873in}{2.356868in}}{\pgfqpoint{2.603237in}{2.360818in}}{\pgfqpoint{2.600325in}{2.363730in}}%
\pgfpathcurveto{\pgfqpoint{2.597413in}{2.366642in}}{\pgfqpoint{2.593463in}{2.368279in}}{\pgfqpoint{2.589345in}{2.368279in}}%
\pgfpathcurveto{\pgfqpoint{2.585227in}{2.368279in}}{\pgfqpoint{2.581277in}{2.366642in}}{\pgfqpoint{2.578365in}{2.363730in}}%
\pgfpathcurveto{\pgfqpoint{2.575453in}{2.360818in}}{\pgfqpoint{2.573817in}{2.356868in}}{\pgfqpoint{2.573817in}{2.352750in}}%
\pgfpathcurveto{\pgfqpoint{2.573817in}{2.348632in}}{\pgfqpoint{2.575453in}{2.344682in}}{\pgfqpoint{2.578365in}{2.341770in}}%
\pgfpathcurveto{\pgfqpoint{2.581277in}{2.338858in}}{\pgfqpoint{2.585227in}{2.337222in}}{\pgfqpoint{2.589345in}{2.337222in}}%
\pgfpathclose%
\pgfusepath{fill}%
\end{pgfscope}%
\begin{pgfscope}%
\pgfpathrectangle{\pgfqpoint{0.887500in}{0.275000in}}{\pgfqpoint{4.225000in}{4.225000in}}%
\pgfusepath{clip}%
\pgfsetbuttcap%
\pgfsetroundjoin%
\definecolor{currentfill}{rgb}{0.000000,0.000000,0.000000}%
\pgfsetfillcolor{currentfill}%
\pgfsetfillopacity{0.800000}%
\pgfsetlinewidth{0.000000pt}%
\definecolor{currentstroke}{rgb}{0.000000,0.000000,0.000000}%
\pgfsetstrokecolor{currentstroke}%
\pgfsetstrokeopacity{0.800000}%
\pgfsetdash{}{0pt}%
\pgfpathmoveto{\pgfqpoint{2.927053in}{2.739931in}}%
\pgfpathcurveto{\pgfqpoint{2.931171in}{2.739931in}}{\pgfqpoint{2.935121in}{2.741567in}}{\pgfqpoint{2.938033in}{2.744479in}}%
\pgfpathcurveto{\pgfqpoint{2.940945in}{2.747391in}}{\pgfqpoint{2.942582in}{2.751341in}}{\pgfqpoint{2.942582in}{2.755459in}}%
\pgfpathcurveto{\pgfqpoint{2.942582in}{2.759577in}}{\pgfqpoint{2.940945in}{2.763527in}}{\pgfqpoint{2.938033in}{2.766439in}}%
\pgfpathcurveto{\pgfqpoint{2.935121in}{2.769351in}}{\pgfqpoint{2.931171in}{2.770987in}}{\pgfqpoint{2.927053in}{2.770987in}}%
\pgfpathcurveto{\pgfqpoint{2.922935in}{2.770987in}}{\pgfqpoint{2.918985in}{2.769351in}}{\pgfqpoint{2.916073in}{2.766439in}}%
\pgfpathcurveto{\pgfqpoint{2.913161in}{2.763527in}}{\pgfqpoint{2.911525in}{2.759577in}}{\pgfqpoint{2.911525in}{2.755459in}}%
\pgfpathcurveto{\pgfqpoint{2.911525in}{2.751341in}}{\pgfqpoint{2.913161in}{2.747391in}}{\pgfqpoint{2.916073in}{2.744479in}}%
\pgfpathcurveto{\pgfqpoint{2.918985in}{2.741567in}}{\pgfqpoint{2.922935in}{2.739931in}}{\pgfqpoint{2.927053in}{2.739931in}}%
\pgfpathclose%
\pgfusepath{fill}%
\end{pgfscope}%
\begin{pgfscope}%
\pgfpathrectangle{\pgfqpoint{0.887500in}{0.275000in}}{\pgfqpoint{4.225000in}{4.225000in}}%
\pgfusepath{clip}%
\pgfsetbuttcap%
\pgfsetroundjoin%
\definecolor{currentfill}{rgb}{0.000000,0.000000,0.000000}%
\pgfsetfillcolor{currentfill}%
\pgfsetfillopacity{0.800000}%
\pgfsetlinewidth{0.000000pt}%
\definecolor{currentstroke}{rgb}{0.000000,0.000000,0.000000}%
\pgfsetstrokecolor{currentstroke}%
\pgfsetstrokeopacity{0.800000}%
\pgfsetdash{}{0pt}%
\pgfpathmoveto{\pgfqpoint{3.975562in}{2.753268in}}%
\pgfpathcurveto{\pgfqpoint{3.979680in}{2.753268in}}{\pgfqpoint{3.983630in}{2.754904in}}{\pgfqpoint{3.986542in}{2.757816in}}%
\pgfpathcurveto{\pgfqpoint{3.989454in}{2.760728in}}{\pgfqpoint{3.991090in}{2.764678in}}{\pgfqpoint{3.991090in}{2.768796in}}%
\pgfpathcurveto{\pgfqpoint{3.991090in}{2.772914in}}{\pgfqpoint{3.989454in}{2.776864in}}{\pgfqpoint{3.986542in}{2.779776in}}%
\pgfpathcurveto{\pgfqpoint{3.983630in}{2.782688in}}{\pgfqpoint{3.979680in}{2.784324in}}{\pgfqpoint{3.975562in}{2.784324in}}%
\pgfpathcurveto{\pgfqpoint{3.971444in}{2.784324in}}{\pgfqpoint{3.967494in}{2.782688in}}{\pgfqpoint{3.964582in}{2.779776in}}%
\pgfpathcurveto{\pgfqpoint{3.961670in}{2.776864in}}{\pgfqpoint{3.960033in}{2.772914in}}{\pgfqpoint{3.960033in}{2.768796in}}%
\pgfpathcurveto{\pgfqpoint{3.960033in}{2.764678in}}{\pgfqpoint{3.961670in}{2.760728in}}{\pgfqpoint{3.964582in}{2.757816in}}%
\pgfpathcurveto{\pgfqpoint{3.967494in}{2.754904in}}{\pgfqpoint{3.971444in}{2.753268in}}{\pgfqpoint{3.975562in}{2.753268in}}%
\pgfpathclose%
\pgfusepath{fill}%
\end{pgfscope}%
\begin{pgfscope}%
\pgfpathrectangle{\pgfqpoint{0.887500in}{0.275000in}}{\pgfqpoint{4.225000in}{4.225000in}}%
\pgfusepath{clip}%
\pgfsetbuttcap%
\pgfsetroundjoin%
\definecolor{currentfill}{rgb}{0.000000,0.000000,0.000000}%
\pgfsetfillcolor{currentfill}%
\pgfsetfillopacity{0.800000}%
\pgfsetlinewidth{0.000000pt}%
\definecolor{currentstroke}{rgb}{0.000000,0.000000,0.000000}%
\pgfsetstrokecolor{currentstroke}%
\pgfsetstrokeopacity{0.800000}%
\pgfsetdash{}{0pt}%
\pgfpathmoveto{\pgfqpoint{2.121182in}{2.415018in}}%
\pgfpathcurveto{\pgfqpoint{2.125301in}{2.415018in}}{\pgfqpoint{2.129251in}{2.416654in}}{\pgfqpoint{2.132162in}{2.419566in}}%
\pgfpathcurveto{\pgfqpoint{2.135074in}{2.422478in}}{\pgfqpoint{2.136711in}{2.426428in}}{\pgfqpoint{2.136711in}{2.430546in}}%
\pgfpathcurveto{\pgfqpoint{2.136711in}{2.434664in}}{\pgfqpoint{2.135074in}{2.438614in}}{\pgfqpoint{2.132162in}{2.441526in}}%
\pgfpathcurveto{\pgfqpoint{2.129251in}{2.444438in}}{\pgfqpoint{2.125301in}{2.446074in}}{\pgfqpoint{2.121182in}{2.446074in}}%
\pgfpathcurveto{\pgfqpoint{2.117064in}{2.446074in}}{\pgfqpoint{2.113114in}{2.444438in}}{\pgfqpoint{2.110202in}{2.441526in}}%
\pgfpathcurveto{\pgfqpoint{2.107290in}{2.438614in}}{\pgfqpoint{2.105654in}{2.434664in}}{\pgfqpoint{2.105654in}{2.430546in}}%
\pgfpathcurveto{\pgfqpoint{2.105654in}{2.426428in}}{\pgfqpoint{2.107290in}{2.422478in}}{\pgfqpoint{2.110202in}{2.419566in}}%
\pgfpathcurveto{\pgfqpoint{2.113114in}{2.416654in}}{\pgfqpoint{2.117064in}{2.415018in}}{\pgfqpoint{2.121182in}{2.415018in}}%
\pgfpathclose%
\pgfusepath{fill}%
\end{pgfscope}%
\begin{pgfscope}%
\pgfpathrectangle{\pgfqpoint{0.887500in}{0.275000in}}{\pgfqpoint{4.225000in}{4.225000in}}%
\pgfusepath{clip}%
\pgfsetbuttcap%
\pgfsetroundjoin%
\definecolor{currentfill}{rgb}{0.000000,0.000000,0.000000}%
\pgfsetfillcolor{currentfill}%
\pgfsetfillopacity{0.800000}%
\pgfsetlinewidth{0.000000pt}%
\definecolor{currentstroke}{rgb}{0.000000,0.000000,0.000000}%
\pgfsetstrokecolor{currentstroke}%
\pgfsetstrokeopacity{0.800000}%
\pgfsetdash{}{0pt}%
\pgfpathmoveto{\pgfqpoint{3.087281in}{3.239433in}}%
\pgfpathcurveto{\pgfqpoint{3.091400in}{3.239433in}}{\pgfqpoint{3.095350in}{3.241069in}}{\pgfqpoint{3.098262in}{3.243981in}}%
\pgfpathcurveto{\pgfqpoint{3.101173in}{3.246893in}}{\pgfqpoint{3.102810in}{3.250843in}}{\pgfqpoint{3.102810in}{3.254961in}}%
\pgfpathcurveto{\pgfqpoint{3.102810in}{3.259079in}}{\pgfqpoint{3.101173in}{3.263029in}}{\pgfqpoint{3.098262in}{3.265941in}}%
\pgfpathcurveto{\pgfqpoint{3.095350in}{3.268853in}}{\pgfqpoint{3.091400in}{3.270489in}}{\pgfqpoint{3.087281in}{3.270489in}}%
\pgfpathcurveto{\pgfqpoint{3.083163in}{3.270489in}}{\pgfqpoint{3.079213in}{3.268853in}}{\pgfqpoint{3.076301in}{3.265941in}}%
\pgfpathcurveto{\pgfqpoint{3.073389in}{3.263029in}}{\pgfqpoint{3.071753in}{3.259079in}}{\pgfqpoint{3.071753in}{3.254961in}}%
\pgfpathcurveto{\pgfqpoint{3.071753in}{3.250843in}}{\pgfqpoint{3.073389in}{3.246893in}}{\pgfqpoint{3.076301in}{3.243981in}}%
\pgfpathcurveto{\pgfqpoint{3.079213in}{3.241069in}}{\pgfqpoint{3.083163in}{3.239433in}}{\pgfqpoint{3.087281in}{3.239433in}}%
\pgfpathclose%
\pgfusepath{fill}%
\end{pgfscope}%
\begin{pgfscope}%
\pgfpathrectangle{\pgfqpoint{0.887500in}{0.275000in}}{\pgfqpoint{4.225000in}{4.225000in}}%
\pgfusepath{clip}%
\pgfsetbuttcap%
\pgfsetroundjoin%
\definecolor{currentfill}{rgb}{0.000000,0.000000,0.000000}%
\pgfsetfillcolor{currentfill}%
\pgfsetfillopacity{0.800000}%
\pgfsetlinewidth{0.000000pt}%
\definecolor{currentstroke}{rgb}{0.000000,0.000000,0.000000}%
\pgfsetstrokecolor{currentstroke}%
\pgfsetstrokeopacity{0.800000}%
\pgfsetdash{}{0pt}%
\pgfpathmoveto{\pgfqpoint{3.798151in}{2.861816in}}%
\pgfpathcurveto{\pgfqpoint{3.802270in}{2.861816in}}{\pgfqpoint{3.806220in}{2.863452in}}{\pgfqpoint{3.809132in}{2.866364in}}%
\pgfpathcurveto{\pgfqpoint{3.812043in}{2.869276in}}{\pgfqpoint{3.813680in}{2.873226in}}{\pgfqpoint{3.813680in}{2.877345in}}%
\pgfpathcurveto{\pgfqpoint{3.813680in}{2.881463in}}{\pgfqpoint{3.812043in}{2.885413in}}{\pgfqpoint{3.809132in}{2.888325in}}%
\pgfpathcurveto{\pgfqpoint{3.806220in}{2.891237in}}{\pgfqpoint{3.802270in}{2.892873in}}{\pgfqpoint{3.798151in}{2.892873in}}%
\pgfpathcurveto{\pgfqpoint{3.794033in}{2.892873in}}{\pgfqpoint{3.790083in}{2.891237in}}{\pgfqpoint{3.787171in}{2.888325in}}%
\pgfpathcurveto{\pgfqpoint{3.784259in}{2.885413in}}{\pgfqpoint{3.782623in}{2.881463in}}{\pgfqpoint{3.782623in}{2.877345in}}%
\pgfpathcurveto{\pgfqpoint{3.782623in}{2.873226in}}{\pgfqpoint{3.784259in}{2.869276in}}{\pgfqpoint{3.787171in}{2.866364in}}%
\pgfpathcurveto{\pgfqpoint{3.790083in}{2.863452in}}{\pgfqpoint{3.794033in}{2.861816in}}{\pgfqpoint{3.798151in}{2.861816in}}%
\pgfpathclose%
\pgfusepath{fill}%
\end{pgfscope}%
\begin{pgfscope}%
\pgfpathrectangle{\pgfqpoint{0.887500in}{0.275000in}}{\pgfqpoint{4.225000in}{4.225000in}}%
\pgfusepath{clip}%
\pgfsetbuttcap%
\pgfsetroundjoin%
\definecolor{currentfill}{rgb}{0.000000,0.000000,0.000000}%
\pgfsetfillcolor{currentfill}%
\pgfsetfillopacity{0.800000}%
\pgfsetlinewidth{0.000000pt}%
\definecolor{currentstroke}{rgb}{0.000000,0.000000,0.000000}%
\pgfsetstrokecolor{currentstroke}%
\pgfsetstrokeopacity{0.800000}%
\pgfsetdash{}{0pt}%
\pgfpathmoveto{\pgfqpoint{1.652755in}{2.488673in}}%
\pgfpathcurveto{\pgfqpoint{1.656873in}{2.488673in}}{\pgfqpoint{1.660823in}{2.490310in}}{\pgfqpoint{1.663735in}{2.493221in}}%
\pgfpathcurveto{\pgfqpoint{1.666647in}{2.496133in}}{\pgfqpoint{1.668283in}{2.500083in}}{\pgfqpoint{1.668283in}{2.504202in}}%
\pgfpathcurveto{\pgfqpoint{1.668283in}{2.508320in}}{\pgfqpoint{1.666647in}{2.512270in}}{\pgfqpoint{1.663735in}{2.515182in}}%
\pgfpathcurveto{\pgfqpoint{1.660823in}{2.518094in}}{\pgfqpoint{1.656873in}{2.519730in}}{\pgfqpoint{1.652755in}{2.519730in}}%
\pgfpathcurveto{\pgfqpoint{1.648637in}{2.519730in}}{\pgfqpoint{1.644687in}{2.518094in}}{\pgfqpoint{1.641775in}{2.515182in}}%
\pgfpathcurveto{\pgfqpoint{1.638863in}{2.512270in}}{\pgfqpoint{1.637227in}{2.508320in}}{\pgfqpoint{1.637227in}{2.504202in}}%
\pgfpathcurveto{\pgfqpoint{1.637227in}{2.500083in}}{\pgfqpoint{1.638863in}{2.496133in}}{\pgfqpoint{1.641775in}{2.493221in}}%
\pgfpathcurveto{\pgfqpoint{1.644687in}{2.490310in}}{\pgfqpoint{1.648637in}{2.488673in}}{\pgfqpoint{1.652755in}{2.488673in}}%
\pgfpathclose%
\pgfusepath{fill}%
\end{pgfscope}%
\begin{pgfscope}%
\pgfpathrectangle{\pgfqpoint{0.887500in}{0.275000in}}{\pgfqpoint{4.225000in}{4.225000in}}%
\pgfusepath{clip}%
\pgfsetbuttcap%
\pgfsetroundjoin%
\definecolor{currentfill}{rgb}{0.000000,0.000000,0.000000}%
\pgfsetfillcolor{currentfill}%
\pgfsetfillopacity{0.800000}%
\pgfsetlinewidth{0.000000pt}%
\definecolor{currentstroke}{rgb}{0.000000,0.000000,0.000000}%
\pgfsetstrokecolor{currentstroke}%
\pgfsetstrokeopacity{0.800000}%
\pgfsetdash{}{0pt}%
\pgfpathmoveto{\pgfqpoint{2.766658in}{2.278676in}}%
\pgfpathcurveto{\pgfqpoint{2.770777in}{2.278676in}}{\pgfqpoint{2.774727in}{2.280312in}}{\pgfqpoint{2.777638in}{2.283224in}}%
\pgfpathcurveto{\pgfqpoint{2.780550in}{2.286136in}}{\pgfqpoint{2.782187in}{2.290086in}}{\pgfqpoint{2.782187in}{2.294204in}}%
\pgfpathcurveto{\pgfqpoint{2.782187in}{2.298322in}}{\pgfqpoint{2.780550in}{2.302272in}}{\pgfqpoint{2.777638in}{2.305184in}}%
\pgfpathcurveto{\pgfqpoint{2.774727in}{2.308096in}}{\pgfqpoint{2.770777in}{2.309732in}}{\pgfqpoint{2.766658in}{2.309732in}}%
\pgfpathcurveto{\pgfqpoint{2.762540in}{2.309732in}}{\pgfqpoint{2.758590in}{2.308096in}}{\pgfqpoint{2.755678in}{2.305184in}}%
\pgfpathcurveto{\pgfqpoint{2.752766in}{2.302272in}}{\pgfqpoint{2.751130in}{2.298322in}}{\pgfqpoint{2.751130in}{2.294204in}}%
\pgfpathcurveto{\pgfqpoint{2.751130in}{2.290086in}}{\pgfqpoint{2.752766in}{2.286136in}}{\pgfqpoint{2.755678in}{2.283224in}}%
\pgfpathcurveto{\pgfqpoint{2.758590in}{2.280312in}}{\pgfqpoint{2.762540in}{2.278676in}}{\pgfqpoint{2.766658in}{2.278676in}}%
\pgfpathclose%
\pgfusepath{fill}%
\end{pgfscope}%
\begin{pgfscope}%
\pgfpathrectangle{\pgfqpoint{0.887500in}{0.275000in}}{\pgfqpoint{4.225000in}{4.225000in}}%
\pgfusepath{clip}%
\pgfsetbuttcap%
\pgfsetroundjoin%
\definecolor{currentfill}{rgb}{0.000000,0.000000,0.000000}%
\pgfsetfillcolor{currentfill}%
\pgfsetfillopacity{0.800000}%
\pgfsetlinewidth{0.000000pt}%
\definecolor{currentstroke}{rgb}{0.000000,0.000000,0.000000}%
\pgfsetstrokecolor{currentstroke}%
\pgfsetstrokeopacity{0.800000}%
\pgfsetdash{}{0pt}%
\pgfpathmoveto{\pgfqpoint{4.574413in}{2.264502in}}%
\pgfpathcurveto{\pgfqpoint{4.578532in}{2.264502in}}{\pgfqpoint{4.582482in}{2.266138in}}{\pgfqpoint{4.585394in}{2.269050in}}%
\pgfpathcurveto{\pgfqpoint{4.588306in}{2.271962in}}{\pgfqpoint{4.589942in}{2.275912in}}{\pgfqpoint{4.589942in}{2.280030in}}%
\pgfpathcurveto{\pgfqpoint{4.589942in}{2.284149in}}{\pgfqpoint{4.588306in}{2.288099in}}{\pgfqpoint{4.585394in}{2.291011in}}%
\pgfpathcurveto{\pgfqpoint{4.582482in}{2.293922in}}{\pgfqpoint{4.578532in}{2.295559in}}{\pgfqpoint{4.574413in}{2.295559in}}%
\pgfpathcurveto{\pgfqpoint{4.570295in}{2.295559in}}{\pgfqpoint{4.566345in}{2.293922in}}{\pgfqpoint{4.563433in}{2.291011in}}%
\pgfpathcurveto{\pgfqpoint{4.560521in}{2.288099in}}{\pgfqpoint{4.558885in}{2.284149in}}{\pgfqpoint{4.558885in}{2.280030in}}%
\pgfpathcurveto{\pgfqpoint{4.558885in}{2.275912in}}{\pgfqpoint{4.560521in}{2.271962in}}{\pgfqpoint{4.563433in}{2.269050in}}%
\pgfpathcurveto{\pgfqpoint{4.566345in}{2.266138in}}{\pgfqpoint{4.570295in}{2.264502in}}{\pgfqpoint{4.574413in}{2.264502in}}%
\pgfpathclose%
\pgfusepath{fill}%
\end{pgfscope}%
\begin{pgfscope}%
\pgfpathrectangle{\pgfqpoint{0.887500in}{0.275000in}}{\pgfqpoint{4.225000in}{4.225000in}}%
\pgfusepath{clip}%
\pgfsetbuttcap%
\pgfsetroundjoin%
\definecolor{currentfill}{rgb}{0.000000,0.000000,0.000000}%
\pgfsetfillcolor{currentfill}%
\pgfsetfillopacity{0.800000}%
\pgfsetlinewidth{0.000000pt}%
\definecolor{currentstroke}{rgb}{0.000000,0.000000,0.000000}%
\pgfsetstrokecolor{currentstroke}%
\pgfsetstrokeopacity{0.800000}%
\pgfsetdash{}{0pt}%
\pgfpathmoveto{\pgfqpoint{3.620566in}{2.966438in}}%
\pgfpathcurveto{\pgfqpoint{3.624685in}{2.966438in}}{\pgfqpoint{3.628635in}{2.968074in}}{\pgfqpoint{3.631547in}{2.970986in}}%
\pgfpathcurveto{\pgfqpoint{3.634458in}{2.973898in}}{\pgfqpoint{3.636095in}{2.977848in}}{\pgfqpoint{3.636095in}{2.981966in}}%
\pgfpathcurveto{\pgfqpoint{3.636095in}{2.986084in}}{\pgfqpoint{3.634458in}{2.990034in}}{\pgfqpoint{3.631547in}{2.992946in}}%
\pgfpathcurveto{\pgfqpoint{3.628635in}{2.995858in}}{\pgfqpoint{3.624685in}{2.997495in}}{\pgfqpoint{3.620566in}{2.997495in}}%
\pgfpathcurveto{\pgfqpoint{3.616448in}{2.997495in}}{\pgfqpoint{3.612498in}{2.995858in}}{\pgfqpoint{3.609586in}{2.992946in}}%
\pgfpathcurveto{\pgfqpoint{3.606674in}{2.990034in}}{\pgfqpoint{3.605038in}{2.986084in}}{\pgfqpoint{3.605038in}{2.981966in}}%
\pgfpathcurveto{\pgfqpoint{3.605038in}{2.977848in}}{\pgfqpoint{3.606674in}{2.973898in}}{\pgfqpoint{3.609586in}{2.970986in}}%
\pgfpathcurveto{\pgfqpoint{3.612498in}{2.968074in}}{\pgfqpoint{3.616448in}{2.966438in}}{\pgfqpoint{3.620566in}{2.966438in}}%
\pgfpathclose%
\pgfusepath{fill}%
\end{pgfscope}%
\begin{pgfscope}%
\pgfpathrectangle{\pgfqpoint{0.887500in}{0.275000in}}{\pgfqpoint{4.225000in}{4.225000in}}%
\pgfusepath{clip}%
\pgfsetbuttcap%
\pgfsetroundjoin%
\definecolor{currentfill}{rgb}{0.000000,0.000000,0.000000}%
\pgfsetfillcolor{currentfill}%
\pgfsetfillopacity{0.800000}%
\pgfsetlinewidth{0.000000pt}%
\definecolor{currentstroke}{rgb}{0.000000,0.000000,0.000000}%
\pgfsetstrokecolor{currentstroke}%
\pgfsetstrokeopacity{0.800000}%
\pgfsetdash{}{0pt}%
\pgfpathmoveto{\pgfqpoint{3.265069in}{3.160710in}}%
\pgfpathcurveto{\pgfqpoint{3.269187in}{3.160710in}}{\pgfqpoint{3.273137in}{3.162346in}}{\pgfqpoint{3.276049in}{3.165258in}}%
\pgfpathcurveto{\pgfqpoint{3.278961in}{3.168170in}}{\pgfqpoint{3.280597in}{3.172120in}}{\pgfqpoint{3.280597in}{3.176238in}}%
\pgfpathcurveto{\pgfqpoint{3.280597in}{3.180356in}}{\pgfqpoint{3.278961in}{3.184306in}}{\pgfqpoint{3.276049in}{3.187218in}}%
\pgfpathcurveto{\pgfqpoint{3.273137in}{3.190130in}}{\pgfqpoint{3.269187in}{3.191766in}}{\pgfqpoint{3.265069in}{3.191766in}}%
\pgfpathcurveto{\pgfqpoint{3.260951in}{3.191766in}}{\pgfqpoint{3.257001in}{3.190130in}}{\pgfqpoint{3.254089in}{3.187218in}}%
\pgfpathcurveto{\pgfqpoint{3.251177in}{3.184306in}}{\pgfqpoint{3.249541in}{3.180356in}}{\pgfqpoint{3.249541in}{3.176238in}}%
\pgfpathcurveto{\pgfqpoint{3.249541in}{3.172120in}}{\pgfqpoint{3.251177in}{3.168170in}}{\pgfqpoint{3.254089in}{3.165258in}}%
\pgfpathcurveto{\pgfqpoint{3.257001in}{3.162346in}}{\pgfqpoint{3.260951in}{3.160710in}}{\pgfqpoint{3.265069in}{3.160710in}}%
\pgfpathclose%
\pgfusepath{fill}%
\end{pgfscope}%
\begin{pgfscope}%
\pgfpathrectangle{\pgfqpoint{0.887500in}{0.275000in}}{\pgfqpoint{4.225000in}{4.225000in}}%
\pgfusepath{clip}%
\pgfsetbuttcap%
\pgfsetroundjoin%
\definecolor{currentfill}{rgb}{0.000000,0.000000,0.000000}%
\pgfsetfillcolor{currentfill}%
\pgfsetfillopacity{0.800000}%
\pgfsetlinewidth{0.000000pt}%
\definecolor{currentstroke}{rgb}{0.000000,0.000000,0.000000}%
\pgfsetstrokecolor{currentstroke}%
\pgfsetstrokeopacity{0.800000}%
\pgfsetdash{}{0pt}%
\pgfpathmoveto{\pgfqpoint{3.442866in}{3.067709in}}%
\pgfpathcurveto{\pgfqpoint{3.446984in}{3.067709in}}{\pgfqpoint{3.450934in}{3.069345in}}{\pgfqpoint{3.453846in}{3.072257in}}%
\pgfpathcurveto{\pgfqpoint{3.456758in}{3.075169in}}{\pgfqpoint{3.458394in}{3.079119in}}{\pgfqpoint{3.458394in}{3.083237in}}%
\pgfpathcurveto{\pgfqpoint{3.458394in}{3.087356in}}{\pgfqpoint{3.456758in}{3.091306in}}{\pgfqpoint{3.453846in}{3.094218in}}%
\pgfpathcurveto{\pgfqpoint{3.450934in}{3.097129in}}{\pgfqpoint{3.446984in}{3.098766in}}{\pgfqpoint{3.442866in}{3.098766in}}%
\pgfpathcurveto{\pgfqpoint{3.438747in}{3.098766in}}{\pgfqpoint{3.434797in}{3.097129in}}{\pgfqpoint{3.431885in}{3.094218in}}%
\pgfpathcurveto{\pgfqpoint{3.428974in}{3.091306in}}{\pgfqpoint{3.427337in}{3.087356in}}{\pgfqpoint{3.427337in}{3.083237in}}%
\pgfpathcurveto{\pgfqpoint{3.427337in}{3.079119in}}{\pgfqpoint{3.428974in}{3.075169in}}{\pgfqpoint{3.431885in}{3.072257in}}%
\pgfpathcurveto{\pgfqpoint{3.434797in}{3.069345in}}{\pgfqpoint{3.438747in}{3.067709in}}{\pgfqpoint{3.442866in}{3.067709in}}%
\pgfpathclose%
\pgfusepath{fill}%
\end{pgfscope}%
\begin{pgfscope}%
\pgfpathrectangle{\pgfqpoint{0.887500in}{0.275000in}}{\pgfqpoint{4.225000in}{4.225000in}}%
\pgfusepath{clip}%
\pgfsetbuttcap%
\pgfsetroundjoin%
\definecolor{currentfill}{rgb}{0.000000,0.000000,0.000000}%
\pgfsetfillcolor{currentfill}%
\pgfsetfillopacity{0.800000}%
\pgfsetlinewidth{0.000000pt}%
\definecolor{currentstroke}{rgb}{0.000000,0.000000,0.000000}%
\pgfsetstrokecolor{currentstroke}%
\pgfsetstrokeopacity{0.800000}%
\pgfsetdash{}{0pt}%
\pgfpathmoveto{\pgfqpoint{4.397202in}{2.375861in}}%
\pgfpathcurveto{\pgfqpoint{4.401320in}{2.375861in}}{\pgfqpoint{4.405270in}{2.377498in}}{\pgfqpoint{4.408182in}{2.380410in}}%
\pgfpathcurveto{\pgfqpoint{4.411094in}{2.383322in}}{\pgfqpoint{4.412731in}{2.387272in}}{\pgfqpoint{4.412731in}{2.391390in}}%
\pgfpathcurveto{\pgfqpoint{4.412731in}{2.395508in}}{\pgfqpoint{4.411094in}{2.399458in}}{\pgfqpoint{4.408182in}{2.402370in}}%
\pgfpathcurveto{\pgfqpoint{4.405270in}{2.405282in}}{\pgfqpoint{4.401320in}{2.406918in}}{\pgfqpoint{4.397202in}{2.406918in}}%
\pgfpathcurveto{\pgfqpoint{4.393084in}{2.406918in}}{\pgfqpoint{4.389134in}{2.405282in}}{\pgfqpoint{4.386222in}{2.402370in}}%
\pgfpathcurveto{\pgfqpoint{4.383310in}{2.399458in}}{\pgfqpoint{4.381674in}{2.395508in}}{\pgfqpoint{4.381674in}{2.391390in}}%
\pgfpathcurveto{\pgfqpoint{4.381674in}{2.387272in}}{\pgfqpoint{4.383310in}{2.383322in}}{\pgfqpoint{4.386222in}{2.380410in}}%
\pgfpathcurveto{\pgfqpoint{4.389134in}{2.377498in}}{\pgfqpoint{4.393084in}{2.375861in}}{\pgfqpoint{4.397202in}{2.375861in}}%
\pgfpathclose%
\pgfusepath{fill}%
\end{pgfscope}%
\begin{pgfscope}%
\pgfpathrectangle{\pgfqpoint{0.887500in}{0.275000in}}{\pgfqpoint{4.225000in}{4.225000in}}%
\pgfusepath{clip}%
\pgfsetbuttcap%
\pgfsetroundjoin%
\definecolor{currentfill}{rgb}{0.000000,0.000000,0.000000}%
\pgfsetfillcolor{currentfill}%
\pgfsetfillopacity{0.800000}%
\pgfsetlinewidth{0.000000pt}%
\definecolor{currentstroke}{rgb}{0.000000,0.000000,0.000000}%
\pgfsetstrokecolor{currentstroke}%
\pgfsetstrokeopacity{0.800000}%
\pgfsetdash{}{0pt}%
\pgfpathmoveto{\pgfqpoint{2.298074in}{2.359858in}}%
\pgfpathcurveto{\pgfqpoint{2.302192in}{2.359858in}}{\pgfqpoint{2.306142in}{2.361494in}}{\pgfqpoint{2.309054in}{2.364406in}}%
\pgfpathcurveto{\pgfqpoint{2.311966in}{2.367318in}}{\pgfqpoint{2.313602in}{2.371268in}}{\pgfqpoint{2.313602in}{2.375386in}}%
\pgfpathcurveto{\pgfqpoint{2.313602in}{2.379504in}}{\pgfqpoint{2.311966in}{2.383454in}}{\pgfqpoint{2.309054in}{2.386366in}}%
\pgfpathcurveto{\pgfqpoint{2.306142in}{2.389278in}}{\pgfqpoint{2.302192in}{2.390914in}}{\pgfqpoint{2.298074in}{2.390914in}}%
\pgfpathcurveto{\pgfqpoint{2.293956in}{2.390914in}}{\pgfqpoint{2.290006in}{2.389278in}}{\pgfqpoint{2.287094in}{2.386366in}}%
\pgfpathcurveto{\pgfqpoint{2.284182in}{2.383454in}}{\pgfqpoint{2.282546in}{2.379504in}}{\pgfqpoint{2.282546in}{2.375386in}}%
\pgfpathcurveto{\pgfqpoint{2.282546in}{2.371268in}}{\pgfqpoint{2.284182in}{2.367318in}}{\pgfqpoint{2.287094in}{2.364406in}}%
\pgfpathcurveto{\pgfqpoint{2.290006in}{2.361494in}}{\pgfqpoint{2.293956in}{2.359858in}}{\pgfqpoint{2.298074in}{2.359858in}}%
\pgfpathclose%
\pgfusepath{fill}%
\end{pgfscope}%
\begin{pgfscope}%
\pgfpathrectangle{\pgfqpoint{0.887500in}{0.275000in}}{\pgfqpoint{4.225000in}{4.225000in}}%
\pgfusepath{clip}%
\pgfsetbuttcap%
\pgfsetroundjoin%
\definecolor{currentfill}{rgb}{0.000000,0.000000,0.000000}%
\pgfsetfillcolor{currentfill}%
\pgfsetfillopacity{0.800000}%
\pgfsetlinewidth{0.000000pt}%
\definecolor{currentstroke}{rgb}{0.000000,0.000000,0.000000}%
\pgfsetstrokecolor{currentstroke}%
\pgfsetstrokeopacity{0.800000}%
\pgfsetdash{}{0pt}%
\pgfpathmoveto{\pgfqpoint{2.944310in}{2.217117in}}%
\pgfpathcurveto{\pgfqpoint{2.948428in}{2.217117in}}{\pgfqpoint{2.952378in}{2.218753in}}{\pgfqpoint{2.955290in}{2.221665in}}%
\pgfpathcurveto{\pgfqpoint{2.958202in}{2.224577in}}{\pgfqpoint{2.959838in}{2.228527in}}{\pgfqpoint{2.959838in}{2.232645in}}%
\pgfpathcurveto{\pgfqpoint{2.959838in}{2.236764in}}{\pgfqpoint{2.958202in}{2.240714in}}{\pgfqpoint{2.955290in}{2.243626in}}%
\pgfpathcurveto{\pgfqpoint{2.952378in}{2.246538in}}{\pgfqpoint{2.948428in}{2.248174in}}{\pgfqpoint{2.944310in}{2.248174in}}%
\pgfpathcurveto{\pgfqpoint{2.940192in}{2.248174in}}{\pgfqpoint{2.936242in}{2.246538in}}{\pgfqpoint{2.933330in}{2.243626in}}%
\pgfpathcurveto{\pgfqpoint{2.930418in}{2.240714in}}{\pgfqpoint{2.928782in}{2.236764in}}{\pgfqpoint{2.928782in}{2.232645in}}%
\pgfpathcurveto{\pgfqpoint{2.928782in}{2.228527in}}{\pgfqpoint{2.930418in}{2.224577in}}{\pgfqpoint{2.933330in}{2.221665in}}%
\pgfpathcurveto{\pgfqpoint{2.936242in}{2.218753in}}{\pgfqpoint{2.940192in}{2.217117in}}{\pgfqpoint{2.944310in}{2.217117in}}%
\pgfpathclose%
\pgfusepath{fill}%
\end{pgfscope}%
\begin{pgfscope}%
\pgfpathrectangle{\pgfqpoint{0.887500in}{0.275000in}}{\pgfqpoint{4.225000in}{4.225000in}}%
\pgfusepath{clip}%
\pgfsetbuttcap%
\pgfsetroundjoin%
\definecolor{currentfill}{rgb}{0.000000,0.000000,0.000000}%
\pgfsetfillcolor{currentfill}%
\pgfsetfillopacity{0.800000}%
\pgfsetlinewidth{0.000000pt}%
\definecolor{currentstroke}{rgb}{0.000000,0.000000,0.000000}%
\pgfsetstrokecolor{currentstroke}%
\pgfsetstrokeopacity{0.800000}%
\pgfsetdash{}{0pt}%
\pgfpathmoveto{\pgfqpoint{1.829156in}{2.435974in}}%
\pgfpathcurveto{\pgfqpoint{1.833274in}{2.435974in}}{\pgfqpoint{1.837224in}{2.437610in}}{\pgfqpoint{1.840136in}{2.440522in}}%
\pgfpathcurveto{\pgfqpoint{1.843048in}{2.443434in}}{\pgfqpoint{1.844684in}{2.447384in}}{\pgfqpoint{1.844684in}{2.451502in}}%
\pgfpathcurveto{\pgfqpoint{1.844684in}{2.455620in}}{\pgfqpoint{1.843048in}{2.459570in}}{\pgfqpoint{1.840136in}{2.462482in}}%
\pgfpathcurveto{\pgfqpoint{1.837224in}{2.465394in}}{\pgfqpoint{1.833274in}{2.467030in}}{\pgfqpoint{1.829156in}{2.467030in}}%
\pgfpathcurveto{\pgfqpoint{1.825038in}{2.467030in}}{\pgfqpoint{1.821088in}{2.465394in}}{\pgfqpoint{1.818176in}{2.462482in}}%
\pgfpathcurveto{\pgfqpoint{1.815264in}{2.459570in}}{\pgfqpoint{1.813628in}{2.455620in}}{\pgfqpoint{1.813628in}{2.451502in}}%
\pgfpathcurveto{\pgfqpoint{1.813628in}{2.447384in}}{\pgfqpoint{1.815264in}{2.443434in}}{\pgfqpoint{1.818176in}{2.440522in}}%
\pgfpathcurveto{\pgfqpoint{1.821088in}{2.437610in}}{\pgfqpoint{1.825038in}{2.435974in}}{\pgfqpoint{1.829156in}{2.435974in}}%
\pgfpathclose%
\pgfusepath{fill}%
\end{pgfscope}%
\begin{pgfscope}%
\pgfpathrectangle{\pgfqpoint{0.887500in}{0.275000in}}{\pgfqpoint{4.225000in}{4.225000in}}%
\pgfusepath{clip}%
\pgfsetbuttcap%
\pgfsetroundjoin%
\definecolor{currentfill}{rgb}{0.000000,0.000000,0.000000}%
\pgfsetfillcolor{currentfill}%
\pgfsetfillopacity{0.800000}%
\pgfsetlinewidth{0.000000pt}%
\definecolor{currentstroke}{rgb}{0.000000,0.000000,0.000000}%
\pgfsetstrokecolor{currentstroke}%
\pgfsetstrokeopacity{0.800000}%
\pgfsetdash{}{0pt}%
\pgfpathmoveto{\pgfqpoint{4.220011in}{2.492900in}}%
\pgfpathcurveto{\pgfqpoint{4.224129in}{2.492900in}}{\pgfqpoint{4.228079in}{2.494536in}}{\pgfqpoint{4.230991in}{2.497448in}}%
\pgfpathcurveto{\pgfqpoint{4.233903in}{2.500360in}}{\pgfqpoint{4.235539in}{2.504310in}}{\pgfqpoint{4.235539in}{2.508428in}}%
\pgfpathcurveto{\pgfqpoint{4.235539in}{2.512546in}}{\pgfqpoint{4.233903in}{2.516496in}}{\pgfqpoint{4.230991in}{2.519408in}}%
\pgfpathcurveto{\pgfqpoint{4.228079in}{2.522320in}}{\pgfqpoint{4.224129in}{2.523956in}}{\pgfqpoint{4.220011in}{2.523956in}}%
\pgfpathcurveto{\pgfqpoint{4.215892in}{2.523956in}}{\pgfqpoint{4.211942in}{2.522320in}}{\pgfqpoint{4.209030in}{2.519408in}}%
\pgfpathcurveto{\pgfqpoint{4.206118in}{2.516496in}}{\pgfqpoint{4.204482in}{2.512546in}}{\pgfqpoint{4.204482in}{2.508428in}}%
\pgfpathcurveto{\pgfqpoint{4.204482in}{2.504310in}}{\pgfqpoint{4.206118in}{2.500360in}}{\pgfqpoint{4.209030in}{2.497448in}}%
\pgfpathcurveto{\pgfqpoint{4.211942in}{2.494536in}}{\pgfqpoint{4.215892in}{2.492900in}}{\pgfqpoint{4.220011in}{2.492900in}}%
\pgfpathclose%
\pgfusepath{fill}%
\end{pgfscope}%
\begin{pgfscope}%
\pgfpathrectangle{\pgfqpoint{0.887500in}{0.275000in}}{\pgfqpoint{4.225000in}{4.225000in}}%
\pgfusepath{clip}%
\pgfsetbuttcap%
\pgfsetroundjoin%
\definecolor{currentfill}{rgb}{0.000000,0.000000,0.000000}%
\pgfsetfillcolor{currentfill}%
\pgfsetfillopacity{0.800000}%
\pgfsetlinewidth{0.000000pt}%
\definecolor{currentstroke}{rgb}{0.000000,0.000000,0.000000}%
\pgfsetstrokecolor{currentstroke}%
\pgfsetstrokeopacity{0.800000}%
\pgfsetdash{}{0pt}%
\pgfpathmoveto{\pgfqpoint{2.991943in}{2.594560in}}%
\pgfpathcurveto{\pgfqpoint{2.996061in}{2.594560in}}{\pgfqpoint{3.000011in}{2.596196in}}{\pgfqpoint{3.002923in}{2.599108in}}%
\pgfpathcurveto{\pgfqpoint{3.005835in}{2.602020in}}{\pgfqpoint{3.007471in}{2.605970in}}{\pgfqpoint{3.007471in}{2.610088in}}%
\pgfpathcurveto{\pgfqpoint{3.007471in}{2.614207in}}{\pgfqpoint{3.005835in}{2.618157in}}{\pgfqpoint{3.002923in}{2.621069in}}%
\pgfpathcurveto{\pgfqpoint{3.000011in}{2.623981in}}{\pgfqpoint{2.996061in}{2.625617in}}{\pgfqpoint{2.991943in}{2.625617in}}%
\pgfpathcurveto{\pgfqpoint{2.987825in}{2.625617in}}{\pgfqpoint{2.983875in}{2.623981in}}{\pgfqpoint{2.980963in}{2.621069in}}%
\pgfpathcurveto{\pgfqpoint{2.978051in}{2.618157in}}{\pgfqpoint{2.976415in}{2.614207in}}{\pgfqpoint{2.976415in}{2.610088in}}%
\pgfpathcurveto{\pgfqpoint{2.976415in}{2.605970in}}{\pgfqpoint{2.978051in}{2.602020in}}{\pgfqpoint{2.980963in}{2.599108in}}%
\pgfpathcurveto{\pgfqpoint{2.983875in}{2.596196in}}{\pgfqpoint{2.987825in}{2.594560in}}{\pgfqpoint{2.991943in}{2.594560in}}%
\pgfpathclose%
\pgfusepath{fill}%
\end{pgfscope}%
\begin{pgfscope}%
\pgfpathrectangle{\pgfqpoint{0.887500in}{0.275000in}}{\pgfqpoint{4.225000in}{4.225000in}}%
\pgfusepath{clip}%
\pgfsetbuttcap%
\pgfsetroundjoin%
\definecolor{currentfill}{rgb}{0.000000,0.000000,0.000000}%
\pgfsetfillcolor{currentfill}%
\pgfsetfillopacity{0.800000}%
\pgfsetlinewidth{0.000000pt}%
\definecolor{currentstroke}{rgb}{0.000000,0.000000,0.000000}%
\pgfsetstrokecolor{currentstroke}%
\pgfsetstrokeopacity{0.800000}%
\pgfsetdash{}{0pt}%
\pgfpathmoveto{\pgfqpoint{4.042543in}{2.606788in}}%
\pgfpathcurveto{\pgfqpoint{4.046661in}{2.606788in}}{\pgfqpoint{4.050611in}{2.608424in}}{\pgfqpoint{4.053523in}{2.611336in}}%
\pgfpathcurveto{\pgfqpoint{4.056435in}{2.614248in}}{\pgfqpoint{4.058071in}{2.618198in}}{\pgfqpoint{4.058071in}{2.622316in}}%
\pgfpathcurveto{\pgfqpoint{4.058071in}{2.626435in}}{\pgfqpoint{4.056435in}{2.630385in}}{\pgfqpoint{4.053523in}{2.633297in}}%
\pgfpathcurveto{\pgfqpoint{4.050611in}{2.636208in}}{\pgfqpoint{4.046661in}{2.637845in}}{\pgfqpoint{4.042543in}{2.637845in}}%
\pgfpathcurveto{\pgfqpoint{4.038425in}{2.637845in}}{\pgfqpoint{4.034475in}{2.636208in}}{\pgfqpoint{4.031563in}{2.633297in}}%
\pgfpathcurveto{\pgfqpoint{4.028651in}{2.630385in}}{\pgfqpoint{4.027015in}{2.626435in}}{\pgfqpoint{4.027015in}{2.622316in}}%
\pgfpathcurveto{\pgfqpoint{4.027015in}{2.618198in}}{\pgfqpoint{4.028651in}{2.614248in}}{\pgfqpoint{4.031563in}{2.611336in}}%
\pgfpathcurveto{\pgfqpoint{4.034475in}{2.608424in}}{\pgfqpoint{4.038425in}{2.606788in}}{\pgfqpoint{4.042543in}{2.606788in}}%
\pgfpathclose%
\pgfusepath{fill}%
\end{pgfscope}%
\begin{pgfscope}%
\pgfpathrectangle{\pgfqpoint{0.887500in}{0.275000in}}{\pgfqpoint{4.225000in}{4.225000in}}%
\pgfusepath{clip}%
\pgfsetbuttcap%
\pgfsetroundjoin%
\definecolor{currentfill}{rgb}{0.000000,0.000000,0.000000}%
\pgfsetfillcolor{currentfill}%
\pgfsetfillopacity{0.800000}%
\pgfsetlinewidth{0.000000pt}%
\definecolor{currentstroke}{rgb}{0.000000,0.000000,0.000000}%
\pgfsetstrokecolor{currentstroke}%
\pgfsetstrokeopacity{0.800000}%
\pgfsetdash{}{0pt}%
\pgfpathmoveto{\pgfqpoint{2.475343in}{2.303387in}}%
\pgfpathcurveto{\pgfqpoint{2.479461in}{2.303387in}}{\pgfqpoint{2.483411in}{2.305024in}}{\pgfqpoint{2.486323in}{2.307936in}}%
\pgfpathcurveto{\pgfqpoint{2.489235in}{2.310847in}}{\pgfqpoint{2.490871in}{2.314798in}}{\pgfqpoint{2.490871in}{2.318916in}}%
\pgfpathcurveto{\pgfqpoint{2.490871in}{2.323034in}}{\pgfqpoint{2.489235in}{2.326984in}}{\pgfqpoint{2.486323in}{2.329896in}}%
\pgfpathcurveto{\pgfqpoint{2.483411in}{2.332808in}}{\pgfqpoint{2.479461in}{2.334444in}}{\pgfqpoint{2.475343in}{2.334444in}}%
\pgfpathcurveto{\pgfqpoint{2.471225in}{2.334444in}}{\pgfqpoint{2.467275in}{2.332808in}}{\pgfqpoint{2.464363in}{2.329896in}}%
\pgfpathcurveto{\pgfqpoint{2.461451in}{2.326984in}}{\pgfqpoint{2.459815in}{2.323034in}}{\pgfqpoint{2.459815in}{2.318916in}}%
\pgfpathcurveto{\pgfqpoint{2.459815in}{2.314798in}}{\pgfqpoint{2.461451in}{2.310847in}}{\pgfqpoint{2.464363in}{2.307936in}}%
\pgfpathcurveto{\pgfqpoint{2.467275in}{2.305024in}}{\pgfqpoint{2.471225in}{2.303387in}}{\pgfqpoint{2.475343in}{2.303387in}}%
\pgfpathclose%
\pgfusepath{fill}%
\end{pgfscope}%
\begin{pgfscope}%
\pgfpathrectangle{\pgfqpoint{0.887500in}{0.275000in}}{\pgfqpoint{4.225000in}{4.225000in}}%
\pgfusepath{clip}%
\pgfsetbuttcap%
\pgfsetroundjoin%
\definecolor{currentfill}{rgb}{0.000000,0.000000,0.000000}%
\pgfsetfillcolor{currentfill}%
\pgfsetfillopacity{0.800000}%
\pgfsetlinewidth{0.000000pt}%
\definecolor{currentstroke}{rgb}{0.000000,0.000000,0.000000}%
\pgfsetstrokecolor{currentstroke}%
\pgfsetstrokeopacity{0.800000}%
\pgfsetdash{}{0pt}%
\pgfpathmoveto{\pgfqpoint{3.864869in}{2.718792in}}%
\pgfpathcurveto{\pgfqpoint{3.868987in}{2.718792in}}{\pgfqpoint{3.872937in}{2.720428in}}{\pgfqpoint{3.875849in}{2.723340in}}%
\pgfpathcurveto{\pgfqpoint{3.878761in}{2.726252in}}{\pgfqpoint{3.880397in}{2.730202in}}{\pgfqpoint{3.880397in}{2.734320in}}%
\pgfpathcurveto{\pgfqpoint{3.880397in}{2.738438in}}{\pgfqpoint{3.878761in}{2.742388in}}{\pgfqpoint{3.875849in}{2.745300in}}%
\pgfpathcurveto{\pgfqpoint{3.872937in}{2.748212in}}{\pgfqpoint{3.868987in}{2.749849in}}{\pgfqpoint{3.864869in}{2.749849in}}%
\pgfpathcurveto{\pgfqpoint{3.860751in}{2.749849in}}{\pgfqpoint{3.856801in}{2.748212in}}{\pgfqpoint{3.853889in}{2.745300in}}%
\pgfpathcurveto{\pgfqpoint{3.850977in}{2.742388in}}{\pgfqpoint{3.849341in}{2.738438in}}{\pgfqpoint{3.849341in}{2.734320in}}%
\pgfpathcurveto{\pgfqpoint{3.849341in}{2.730202in}}{\pgfqpoint{3.850977in}{2.726252in}}{\pgfqpoint{3.853889in}{2.723340in}}%
\pgfpathcurveto{\pgfqpoint{3.856801in}{2.720428in}}{\pgfqpoint{3.860751in}{2.718792in}}{\pgfqpoint{3.864869in}{2.718792in}}%
\pgfpathclose%
\pgfusepath{fill}%
\end{pgfscope}%
\begin{pgfscope}%
\pgfpathrectangle{\pgfqpoint{0.887500in}{0.275000in}}{\pgfqpoint{4.225000in}{4.225000in}}%
\pgfusepath{clip}%
\pgfsetbuttcap%
\pgfsetroundjoin%
\definecolor{currentfill}{rgb}{0.000000,0.000000,0.000000}%
\pgfsetfillcolor{currentfill}%
\pgfsetfillopacity{0.800000}%
\pgfsetlinewidth{0.000000pt}%
\definecolor{currentstroke}{rgb}{0.000000,0.000000,0.000000}%
\pgfsetstrokecolor{currentstroke}%
\pgfsetstrokeopacity{0.800000}%
\pgfsetdash{}{0pt}%
\pgfpathmoveto{\pgfqpoint{4.642447in}{2.112621in}}%
\pgfpathcurveto{\pgfqpoint{4.646565in}{2.112621in}}{\pgfqpoint{4.650515in}{2.114257in}}{\pgfqpoint{4.653427in}{2.117169in}}%
\pgfpathcurveto{\pgfqpoint{4.656339in}{2.120081in}}{\pgfqpoint{4.657975in}{2.124031in}}{\pgfqpoint{4.657975in}{2.128149in}}%
\pgfpathcurveto{\pgfqpoint{4.657975in}{2.132267in}}{\pgfqpoint{4.656339in}{2.136217in}}{\pgfqpoint{4.653427in}{2.139129in}}%
\pgfpathcurveto{\pgfqpoint{4.650515in}{2.142041in}}{\pgfqpoint{4.646565in}{2.143677in}}{\pgfqpoint{4.642447in}{2.143677in}}%
\pgfpathcurveto{\pgfqpoint{4.638329in}{2.143677in}}{\pgfqpoint{4.634379in}{2.142041in}}{\pgfqpoint{4.631467in}{2.139129in}}%
\pgfpathcurveto{\pgfqpoint{4.628555in}{2.136217in}}{\pgfqpoint{4.626919in}{2.132267in}}{\pgfqpoint{4.626919in}{2.128149in}}%
\pgfpathcurveto{\pgfqpoint{4.626919in}{2.124031in}}{\pgfqpoint{4.628555in}{2.120081in}}{\pgfqpoint{4.631467in}{2.117169in}}%
\pgfpathcurveto{\pgfqpoint{4.634379in}{2.114257in}}{\pgfqpoint{4.638329in}{2.112621in}}{\pgfqpoint{4.642447in}{2.112621in}}%
\pgfpathclose%
\pgfusepath{fill}%
\end{pgfscope}%
\begin{pgfscope}%
\pgfpathrectangle{\pgfqpoint{0.887500in}{0.275000in}}{\pgfqpoint{4.225000in}{4.225000in}}%
\pgfusepath{clip}%
\pgfsetbuttcap%
\pgfsetroundjoin%
\definecolor{currentfill}{rgb}{0.000000,0.000000,0.000000}%
\pgfsetfillcolor{currentfill}%
\pgfsetfillopacity{0.800000}%
\pgfsetlinewidth{0.000000pt}%
\definecolor{currentstroke}{rgb}{0.000000,0.000000,0.000000}%
\pgfsetstrokecolor{currentstroke}%
\pgfsetstrokeopacity{0.800000}%
\pgfsetdash{}{0pt}%
\pgfpathmoveto{\pgfqpoint{2.005973in}{2.381787in}}%
\pgfpathcurveto{\pgfqpoint{2.010092in}{2.381787in}}{\pgfqpoint{2.014042in}{2.383423in}}{\pgfqpoint{2.016954in}{2.386335in}}%
\pgfpathcurveto{\pgfqpoint{2.019865in}{2.389247in}}{\pgfqpoint{2.021502in}{2.393197in}}{\pgfqpoint{2.021502in}{2.397316in}}%
\pgfpathcurveto{\pgfqpoint{2.021502in}{2.401434in}}{\pgfqpoint{2.019865in}{2.405384in}}{\pgfqpoint{2.016954in}{2.408296in}}%
\pgfpathcurveto{\pgfqpoint{2.014042in}{2.411208in}}{\pgfqpoint{2.010092in}{2.412844in}}{\pgfqpoint{2.005973in}{2.412844in}}%
\pgfpathcurveto{\pgfqpoint{2.001855in}{2.412844in}}{\pgfqpoint{1.997905in}{2.411208in}}{\pgfqpoint{1.994993in}{2.408296in}}%
\pgfpathcurveto{\pgfqpoint{1.992081in}{2.405384in}}{\pgfqpoint{1.990445in}{2.401434in}}{\pgfqpoint{1.990445in}{2.397316in}}%
\pgfpathcurveto{\pgfqpoint{1.990445in}{2.393197in}}{\pgfqpoint{1.992081in}{2.389247in}}{\pgfqpoint{1.994993in}{2.386335in}}%
\pgfpathcurveto{\pgfqpoint{1.997905in}{2.383423in}}{\pgfqpoint{2.001855in}{2.381787in}}{\pgfqpoint{2.005973in}{2.381787in}}%
\pgfpathclose%
\pgfusepath{fill}%
\end{pgfscope}%
\begin{pgfscope}%
\pgfpathrectangle{\pgfqpoint{0.887500in}{0.275000in}}{\pgfqpoint{4.225000in}{4.225000in}}%
\pgfusepath{clip}%
\pgfsetbuttcap%
\pgfsetroundjoin%
\definecolor{currentfill}{rgb}{0.000000,0.000000,0.000000}%
\pgfsetfillcolor{currentfill}%
\pgfsetfillopacity{0.800000}%
\pgfsetlinewidth{0.000000pt}%
\definecolor{currentstroke}{rgb}{0.000000,0.000000,0.000000}%
\pgfsetstrokecolor{currentstroke}%
\pgfsetstrokeopacity{0.800000}%
\pgfsetdash{}{0pt}%
\pgfpathmoveto{\pgfqpoint{3.152555in}{3.110923in}}%
\pgfpathcurveto{\pgfqpoint{3.156673in}{3.110923in}}{\pgfqpoint{3.160623in}{3.112559in}}{\pgfqpoint{3.163535in}{3.115471in}}%
\pgfpathcurveto{\pgfqpoint{3.166447in}{3.118383in}}{\pgfqpoint{3.168083in}{3.122333in}}{\pgfqpoint{3.168083in}{3.126452in}}%
\pgfpathcurveto{\pgfqpoint{3.168083in}{3.130570in}}{\pgfqpoint{3.166447in}{3.134520in}}{\pgfqpoint{3.163535in}{3.137432in}}%
\pgfpathcurveto{\pgfqpoint{3.160623in}{3.140344in}}{\pgfqpoint{3.156673in}{3.141980in}}{\pgfqpoint{3.152555in}{3.141980in}}%
\pgfpathcurveto{\pgfqpoint{3.148436in}{3.141980in}}{\pgfqpoint{3.144486in}{3.140344in}}{\pgfqpoint{3.141574in}{3.137432in}}%
\pgfpathcurveto{\pgfqpoint{3.138662in}{3.134520in}}{\pgfqpoint{3.137026in}{3.130570in}}{\pgfqpoint{3.137026in}{3.126452in}}%
\pgfpathcurveto{\pgfqpoint{3.137026in}{3.122333in}}{\pgfqpoint{3.138662in}{3.118383in}}{\pgfqpoint{3.141574in}{3.115471in}}%
\pgfpathcurveto{\pgfqpoint{3.144486in}{3.112559in}}{\pgfqpoint{3.148436in}{3.110923in}}{\pgfqpoint{3.152555in}{3.110923in}}%
\pgfpathclose%
\pgfusepath{fill}%
\end{pgfscope}%
\begin{pgfscope}%
\pgfpathrectangle{\pgfqpoint{0.887500in}{0.275000in}}{\pgfqpoint{4.225000in}{4.225000in}}%
\pgfusepath{clip}%
\pgfsetbuttcap%
\pgfsetroundjoin%
\definecolor{currentfill}{rgb}{0.000000,0.000000,0.000000}%
\pgfsetfillcolor{currentfill}%
\pgfsetfillopacity{0.800000}%
\pgfsetlinewidth{0.000000pt}%
\definecolor{currentstroke}{rgb}{0.000000,0.000000,0.000000}%
\pgfsetstrokecolor{currentstroke}%
\pgfsetstrokeopacity{0.800000}%
\pgfsetdash{}{0pt}%
\pgfpathmoveto{\pgfqpoint{3.686993in}{2.827503in}}%
\pgfpathcurveto{\pgfqpoint{3.691112in}{2.827503in}}{\pgfqpoint{3.695062in}{2.829139in}}{\pgfqpoint{3.697974in}{2.832051in}}%
\pgfpathcurveto{\pgfqpoint{3.700886in}{2.834963in}}{\pgfqpoint{3.702522in}{2.838913in}}{\pgfqpoint{3.702522in}{2.843031in}}%
\pgfpathcurveto{\pgfqpoint{3.702522in}{2.847150in}}{\pgfqpoint{3.700886in}{2.851100in}}{\pgfqpoint{3.697974in}{2.854012in}}%
\pgfpathcurveto{\pgfqpoint{3.695062in}{2.856924in}}{\pgfqpoint{3.691112in}{2.858560in}}{\pgfqpoint{3.686993in}{2.858560in}}%
\pgfpathcurveto{\pgfqpoint{3.682875in}{2.858560in}}{\pgfqpoint{3.678925in}{2.856924in}}{\pgfqpoint{3.676013in}{2.854012in}}%
\pgfpathcurveto{\pgfqpoint{3.673101in}{2.851100in}}{\pgfqpoint{3.671465in}{2.847150in}}{\pgfqpoint{3.671465in}{2.843031in}}%
\pgfpathcurveto{\pgfqpoint{3.671465in}{2.838913in}}{\pgfqpoint{3.673101in}{2.834963in}}{\pgfqpoint{3.676013in}{2.832051in}}%
\pgfpathcurveto{\pgfqpoint{3.678925in}{2.829139in}}{\pgfqpoint{3.682875in}{2.827503in}}{\pgfqpoint{3.686993in}{2.827503in}}%
\pgfpathclose%
\pgfusepath{fill}%
\end{pgfscope}%
\begin{pgfscope}%
\pgfpathrectangle{\pgfqpoint{0.887500in}{0.275000in}}{\pgfqpoint{4.225000in}{4.225000in}}%
\pgfusepath{clip}%
\pgfsetbuttcap%
\pgfsetroundjoin%
\definecolor{currentfill}{rgb}{0.000000,0.000000,0.000000}%
\pgfsetfillcolor{currentfill}%
\pgfsetfillopacity{0.800000}%
\pgfsetlinewidth{0.000000pt}%
\definecolor{currentstroke}{rgb}{0.000000,0.000000,0.000000}%
\pgfsetstrokecolor{currentstroke}%
\pgfsetstrokeopacity{0.800000}%
\pgfsetdash{}{0pt}%
\pgfpathmoveto{\pgfqpoint{1.536372in}{2.455183in}}%
\pgfpathcurveto{\pgfqpoint{1.540491in}{2.455183in}}{\pgfqpoint{1.544441in}{2.456819in}}{\pgfqpoint{1.547353in}{2.459731in}}%
\pgfpathcurveto{\pgfqpoint{1.550265in}{2.462643in}}{\pgfqpoint{1.551901in}{2.466593in}}{\pgfqpoint{1.551901in}{2.470711in}}%
\pgfpathcurveto{\pgfqpoint{1.551901in}{2.474829in}}{\pgfqpoint{1.550265in}{2.478779in}}{\pgfqpoint{1.547353in}{2.481691in}}%
\pgfpathcurveto{\pgfqpoint{1.544441in}{2.484603in}}{\pgfqpoint{1.540491in}{2.486239in}}{\pgfqpoint{1.536372in}{2.486239in}}%
\pgfpathcurveto{\pgfqpoint{1.532254in}{2.486239in}}{\pgfqpoint{1.528304in}{2.484603in}}{\pgfqpoint{1.525392in}{2.481691in}}%
\pgfpathcurveto{\pgfqpoint{1.522480in}{2.478779in}}{\pgfqpoint{1.520844in}{2.474829in}}{\pgfqpoint{1.520844in}{2.470711in}}%
\pgfpathcurveto{\pgfqpoint{1.520844in}{2.466593in}}{\pgfqpoint{1.522480in}{2.462643in}}{\pgfqpoint{1.525392in}{2.459731in}}%
\pgfpathcurveto{\pgfqpoint{1.528304in}{2.456819in}}{\pgfqpoint{1.532254in}{2.455183in}}{\pgfqpoint{1.536372in}{2.455183in}}%
\pgfpathclose%
\pgfusepath{fill}%
\end{pgfscope}%
\begin{pgfscope}%
\pgfpathrectangle{\pgfqpoint{0.887500in}{0.275000in}}{\pgfqpoint{4.225000in}{4.225000in}}%
\pgfusepath{clip}%
\pgfsetbuttcap%
\pgfsetroundjoin%
\definecolor{currentfill}{rgb}{0.000000,0.000000,0.000000}%
\pgfsetfillcolor{currentfill}%
\pgfsetfillopacity{0.800000}%
\pgfsetlinewidth{0.000000pt}%
\definecolor{currentstroke}{rgb}{0.000000,0.000000,0.000000}%
\pgfsetstrokecolor{currentstroke}%
\pgfsetstrokeopacity{0.800000}%
\pgfsetdash{}{0pt}%
\pgfpathmoveto{\pgfqpoint{2.652977in}{2.245351in}}%
\pgfpathcurveto{\pgfqpoint{2.657095in}{2.245351in}}{\pgfqpoint{2.661045in}{2.246987in}}{\pgfqpoint{2.663957in}{2.249899in}}%
\pgfpathcurveto{\pgfqpoint{2.666869in}{2.252811in}}{\pgfqpoint{2.668505in}{2.256761in}}{\pgfqpoint{2.668505in}{2.260879in}}%
\pgfpathcurveto{\pgfqpoint{2.668505in}{2.264997in}}{\pgfqpoint{2.666869in}{2.268947in}}{\pgfqpoint{2.663957in}{2.271859in}}%
\pgfpathcurveto{\pgfqpoint{2.661045in}{2.274771in}}{\pgfqpoint{2.657095in}{2.276407in}}{\pgfqpoint{2.652977in}{2.276407in}}%
\pgfpathcurveto{\pgfqpoint{2.648859in}{2.276407in}}{\pgfqpoint{2.644909in}{2.274771in}}{\pgfqpoint{2.641997in}{2.271859in}}%
\pgfpathcurveto{\pgfqpoint{2.639085in}{2.268947in}}{\pgfqpoint{2.637449in}{2.264997in}}{\pgfqpoint{2.637449in}{2.260879in}}%
\pgfpathcurveto{\pgfqpoint{2.637449in}{2.256761in}}{\pgfqpoint{2.639085in}{2.252811in}}{\pgfqpoint{2.641997in}{2.249899in}}%
\pgfpathcurveto{\pgfqpoint{2.644909in}{2.246987in}}{\pgfqpoint{2.648859in}{2.245351in}}{\pgfqpoint{2.652977in}{2.245351in}}%
\pgfpathclose%
\pgfusepath{fill}%
\end{pgfscope}%
\begin{pgfscope}%
\pgfpathrectangle{\pgfqpoint{0.887500in}{0.275000in}}{\pgfqpoint{4.225000in}{4.225000in}}%
\pgfusepath{clip}%
\pgfsetbuttcap%
\pgfsetroundjoin%
\definecolor{currentfill}{rgb}{0.000000,0.000000,0.000000}%
\pgfsetfillcolor{currentfill}%
\pgfsetfillopacity{0.800000}%
\pgfsetlinewidth{0.000000pt}%
\definecolor{currentstroke}{rgb}{0.000000,0.000000,0.000000}%
\pgfsetstrokecolor{currentstroke}%
\pgfsetstrokeopacity{0.800000}%
\pgfsetdash{}{0pt}%
\pgfpathmoveto{\pgfqpoint{4.465114in}{2.229259in}}%
\pgfpathcurveto{\pgfqpoint{4.469233in}{2.229259in}}{\pgfqpoint{4.473183in}{2.230895in}}{\pgfqpoint{4.476095in}{2.233807in}}%
\pgfpathcurveto{\pgfqpoint{4.479006in}{2.236719in}}{\pgfqpoint{4.480643in}{2.240669in}}{\pgfqpoint{4.480643in}{2.244787in}}%
\pgfpathcurveto{\pgfqpoint{4.480643in}{2.248905in}}{\pgfqpoint{4.479006in}{2.252855in}}{\pgfqpoint{4.476095in}{2.255767in}}%
\pgfpathcurveto{\pgfqpoint{4.473183in}{2.258679in}}{\pgfqpoint{4.469233in}{2.260315in}}{\pgfqpoint{4.465114in}{2.260315in}}%
\pgfpathcurveto{\pgfqpoint{4.460996in}{2.260315in}}{\pgfqpoint{4.457046in}{2.258679in}}{\pgfqpoint{4.454134in}{2.255767in}}%
\pgfpathcurveto{\pgfqpoint{4.451222in}{2.252855in}}{\pgfqpoint{4.449586in}{2.248905in}}{\pgfqpoint{4.449586in}{2.244787in}}%
\pgfpathcurveto{\pgfqpoint{4.449586in}{2.240669in}}{\pgfqpoint{4.451222in}{2.236719in}}{\pgfqpoint{4.454134in}{2.233807in}}%
\pgfpathcurveto{\pgfqpoint{4.457046in}{2.230895in}}{\pgfqpoint{4.460996in}{2.229259in}}{\pgfqpoint{4.465114in}{2.229259in}}%
\pgfpathclose%
\pgfusepath{fill}%
\end{pgfscope}%
\begin{pgfscope}%
\pgfpathrectangle{\pgfqpoint{0.887500in}{0.275000in}}{\pgfqpoint{4.225000in}{4.225000in}}%
\pgfusepath{clip}%
\pgfsetbuttcap%
\pgfsetroundjoin%
\definecolor{currentfill}{rgb}{0.000000,0.000000,0.000000}%
\pgfsetfillcolor{currentfill}%
\pgfsetfillopacity{0.800000}%
\pgfsetlinewidth{0.000000pt}%
\definecolor{currentstroke}{rgb}{0.000000,0.000000,0.000000}%
\pgfsetstrokecolor{currentstroke}%
\pgfsetstrokeopacity{0.800000}%
\pgfsetdash{}{0pt}%
\pgfpathmoveto{\pgfqpoint{3.508960in}{2.932804in}}%
\pgfpathcurveto{\pgfqpoint{3.513079in}{2.932804in}}{\pgfqpoint{3.517029in}{2.934440in}}{\pgfqpoint{3.519941in}{2.937352in}}%
\pgfpathcurveto{\pgfqpoint{3.522853in}{2.940264in}}{\pgfqpoint{3.524489in}{2.944214in}}{\pgfqpoint{3.524489in}{2.948332in}}%
\pgfpathcurveto{\pgfqpoint{3.524489in}{2.952450in}}{\pgfqpoint{3.522853in}{2.956400in}}{\pgfqpoint{3.519941in}{2.959312in}}%
\pgfpathcurveto{\pgfqpoint{3.517029in}{2.962224in}}{\pgfqpoint{3.513079in}{2.963860in}}{\pgfqpoint{3.508960in}{2.963860in}}%
\pgfpathcurveto{\pgfqpoint{3.504842in}{2.963860in}}{\pgfqpoint{3.500892in}{2.962224in}}{\pgfqpoint{3.497980in}{2.959312in}}%
\pgfpathcurveto{\pgfqpoint{3.495068in}{2.956400in}}{\pgfqpoint{3.493432in}{2.952450in}}{\pgfqpoint{3.493432in}{2.948332in}}%
\pgfpathcurveto{\pgfqpoint{3.493432in}{2.944214in}}{\pgfqpoint{3.495068in}{2.940264in}}{\pgfqpoint{3.497980in}{2.937352in}}%
\pgfpathcurveto{\pgfqpoint{3.500892in}{2.934440in}}{\pgfqpoint{3.504842in}{2.932804in}}{\pgfqpoint{3.508960in}{2.932804in}}%
\pgfpathclose%
\pgfusepath{fill}%
\end{pgfscope}%
\begin{pgfscope}%
\pgfpathrectangle{\pgfqpoint{0.887500in}{0.275000in}}{\pgfqpoint{4.225000in}{4.225000in}}%
\pgfusepath{clip}%
\pgfsetbuttcap%
\pgfsetroundjoin%
\definecolor{currentfill}{rgb}{0.000000,0.000000,0.000000}%
\pgfsetfillcolor{currentfill}%
\pgfsetfillopacity{0.800000}%
\pgfsetlinewidth{0.000000pt}%
\definecolor{currentstroke}{rgb}{0.000000,0.000000,0.000000}%
\pgfsetstrokecolor{currentstroke}%
\pgfsetstrokeopacity{0.800000}%
\pgfsetdash{}{0pt}%
\pgfpathmoveto{\pgfqpoint{3.039618in}{2.993942in}}%
\pgfpathcurveto{\pgfqpoint{3.043737in}{2.993942in}}{\pgfqpoint{3.047687in}{2.995578in}}{\pgfqpoint{3.050598in}{2.998490in}}%
\pgfpathcurveto{\pgfqpoint{3.053510in}{3.001402in}}{\pgfqpoint{3.055147in}{3.005352in}}{\pgfqpoint{3.055147in}{3.009470in}}%
\pgfpathcurveto{\pgfqpoint{3.055147in}{3.013588in}}{\pgfqpoint{3.053510in}{3.017539in}}{\pgfqpoint{3.050598in}{3.020450in}}%
\pgfpathcurveto{\pgfqpoint{3.047687in}{3.023362in}}{\pgfqpoint{3.043737in}{3.024999in}}{\pgfqpoint{3.039618in}{3.024999in}}%
\pgfpathcurveto{\pgfqpoint{3.035500in}{3.024999in}}{\pgfqpoint{3.031550in}{3.023362in}}{\pgfqpoint{3.028638in}{3.020450in}}%
\pgfpathcurveto{\pgfqpoint{3.025726in}{3.017539in}}{\pgfqpoint{3.024090in}{3.013588in}}{\pgfqpoint{3.024090in}{3.009470in}}%
\pgfpathcurveto{\pgfqpoint{3.024090in}{3.005352in}}{\pgfqpoint{3.025726in}{3.001402in}}{\pgfqpoint{3.028638in}{2.998490in}}%
\pgfpathcurveto{\pgfqpoint{3.031550in}{2.995578in}}{\pgfqpoint{3.035500in}{2.993942in}}{\pgfqpoint{3.039618in}{2.993942in}}%
\pgfpathclose%
\pgfusepath{fill}%
\end{pgfscope}%
\begin{pgfscope}%
\pgfpathrectangle{\pgfqpoint{0.887500in}{0.275000in}}{\pgfqpoint{4.225000in}{4.225000in}}%
\pgfusepath{clip}%
\pgfsetbuttcap%
\pgfsetroundjoin%
\definecolor{currentfill}{rgb}{0.000000,0.000000,0.000000}%
\pgfsetfillcolor{currentfill}%
\pgfsetfillopacity{0.800000}%
\pgfsetlinewidth{0.000000pt}%
\definecolor{currentstroke}{rgb}{0.000000,0.000000,0.000000}%
\pgfsetstrokecolor{currentstroke}%
\pgfsetstrokeopacity{0.800000}%
\pgfsetdash{}{0pt}%
\pgfpathmoveto{\pgfqpoint{3.330798in}{3.031957in}}%
\pgfpathcurveto{\pgfqpoint{3.334916in}{3.031957in}}{\pgfqpoint{3.338866in}{3.033593in}}{\pgfqpoint{3.341778in}{3.036505in}}%
\pgfpathcurveto{\pgfqpoint{3.344690in}{3.039417in}}{\pgfqpoint{3.346326in}{3.043367in}}{\pgfqpoint{3.346326in}{3.047486in}}%
\pgfpathcurveto{\pgfqpoint{3.346326in}{3.051604in}}{\pgfqpoint{3.344690in}{3.055554in}}{\pgfqpoint{3.341778in}{3.058466in}}%
\pgfpathcurveto{\pgfqpoint{3.338866in}{3.061378in}}{\pgfqpoint{3.334916in}{3.063014in}}{\pgfqpoint{3.330798in}{3.063014in}}%
\pgfpathcurveto{\pgfqpoint{3.326680in}{3.063014in}}{\pgfqpoint{3.322730in}{3.061378in}}{\pgfqpoint{3.319818in}{3.058466in}}%
\pgfpathcurveto{\pgfqpoint{3.316906in}{3.055554in}}{\pgfqpoint{3.315269in}{3.051604in}}{\pgfqpoint{3.315269in}{3.047486in}}%
\pgfpathcurveto{\pgfqpoint{3.315269in}{3.043367in}}{\pgfqpoint{3.316906in}{3.039417in}}{\pgfqpoint{3.319818in}{3.036505in}}%
\pgfpathcurveto{\pgfqpoint{3.322730in}{3.033593in}}{\pgfqpoint{3.326680in}{3.031957in}}{\pgfqpoint{3.330798in}{3.031957in}}%
\pgfpathclose%
\pgfusepath{fill}%
\end{pgfscope}%
\begin{pgfscope}%
\pgfpathrectangle{\pgfqpoint{0.887500in}{0.275000in}}{\pgfqpoint{4.225000in}{4.225000in}}%
\pgfusepath{clip}%
\pgfsetbuttcap%
\pgfsetroundjoin%
\definecolor{currentfill}{rgb}{0.000000,0.000000,0.000000}%
\pgfsetfillcolor{currentfill}%
\pgfsetfillopacity{0.800000}%
\pgfsetlinewidth{0.000000pt}%
\definecolor{currentstroke}{rgb}{0.000000,0.000000,0.000000}%
\pgfsetstrokecolor{currentstroke}%
\pgfsetstrokeopacity{0.800000}%
\pgfsetdash{}{0pt}%
\pgfpathmoveto{\pgfqpoint{4.287470in}{2.342160in}}%
\pgfpathcurveto{\pgfqpoint{4.291588in}{2.342160in}}{\pgfqpoint{4.295538in}{2.343796in}}{\pgfqpoint{4.298450in}{2.346708in}}%
\pgfpathcurveto{\pgfqpoint{4.301362in}{2.349620in}}{\pgfqpoint{4.302998in}{2.353570in}}{\pgfqpoint{4.302998in}{2.357688in}}%
\pgfpathcurveto{\pgfqpoint{4.302998in}{2.361806in}}{\pgfqpoint{4.301362in}{2.365756in}}{\pgfqpoint{4.298450in}{2.368668in}}%
\pgfpathcurveto{\pgfqpoint{4.295538in}{2.371580in}}{\pgfqpoint{4.291588in}{2.373217in}}{\pgfqpoint{4.287470in}{2.373217in}}%
\pgfpathcurveto{\pgfqpoint{4.283352in}{2.373217in}}{\pgfqpoint{4.279402in}{2.371580in}}{\pgfqpoint{4.276490in}{2.368668in}}%
\pgfpathcurveto{\pgfqpoint{4.273578in}{2.365756in}}{\pgfqpoint{4.271942in}{2.361806in}}{\pgfqpoint{4.271942in}{2.357688in}}%
\pgfpathcurveto{\pgfqpoint{4.271942in}{2.353570in}}{\pgfqpoint{4.273578in}{2.349620in}}{\pgfqpoint{4.276490in}{2.346708in}}%
\pgfpathcurveto{\pgfqpoint{4.279402in}{2.343796in}}{\pgfqpoint{4.283352in}{2.342160in}}{\pgfqpoint{4.287470in}{2.342160in}}%
\pgfpathclose%
\pgfusepath{fill}%
\end{pgfscope}%
\begin{pgfscope}%
\pgfpathrectangle{\pgfqpoint{0.887500in}{0.275000in}}{\pgfqpoint{4.225000in}{4.225000in}}%
\pgfusepath{clip}%
\pgfsetbuttcap%
\pgfsetroundjoin%
\definecolor{currentfill}{rgb}{0.000000,0.000000,0.000000}%
\pgfsetfillcolor{currentfill}%
\pgfsetfillopacity{0.800000}%
\pgfsetlinewidth{0.000000pt}%
\definecolor{currentstroke}{rgb}{0.000000,0.000000,0.000000}%
\pgfsetstrokecolor{currentstroke}%
\pgfsetstrokeopacity{0.800000}%
\pgfsetdash{}{0pt}%
\pgfpathmoveto{\pgfqpoint{2.183187in}{2.326287in}}%
\pgfpathcurveto{\pgfqpoint{2.187305in}{2.326287in}}{\pgfqpoint{2.191255in}{2.327923in}}{\pgfqpoint{2.194167in}{2.330835in}}%
\pgfpathcurveto{\pgfqpoint{2.197079in}{2.333747in}}{\pgfqpoint{2.198715in}{2.337697in}}{\pgfqpoint{2.198715in}{2.341815in}}%
\pgfpathcurveto{\pgfqpoint{2.198715in}{2.345933in}}{\pgfqpoint{2.197079in}{2.349883in}}{\pgfqpoint{2.194167in}{2.352795in}}%
\pgfpathcurveto{\pgfqpoint{2.191255in}{2.355707in}}{\pgfqpoint{2.187305in}{2.357343in}}{\pgfqpoint{2.183187in}{2.357343in}}%
\pgfpathcurveto{\pgfqpoint{2.179069in}{2.357343in}}{\pgfqpoint{2.175119in}{2.355707in}}{\pgfqpoint{2.172207in}{2.352795in}}%
\pgfpathcurveto{\pgfqpoint{2.169295in}{2.349883in}}{\pgfqpoint{2.167658in}{2.345933in}}{\pgfqpoint{2.167658in}{2.341815in}}%
\pgfpathcurveto{\pgfqpoint{2.167658in}{2.337697in}}{\pgfqpoint{2.169295in}{2.333747in}}{\pgfqpoint{2.172207in}{2.330835in}}%
\pgfpathcurveto{\pgfqpoint{2.175119in}{2.327923in}}{\pgfqpoint{2.179069in}{2.326287in}}{\pgfqpoint{2.183187in}{2.326287in}}%
\pgfpathclose%
\pgfusepath{fill}%
\end{pgfscope}%
\begin{pgfscope}%
\pgfpathrectangle{\pgfqpoint{0.887500in}{0.275000in}}{\pgfqpoint{4.225000in}{4.225000in}}%
\pgfusepath{clip}%
\pgfsetbuttcap%
\pgfsetroundjoin%
\definecolor{currentfill}{rgb}{0.000000,0.000000,0.000000}%
\pgfsetfillcolor{currentfill}%
\pgfsetfillopacity{0.800000}%
\pgfsetlinewidth{0.000000pt}%
\definecolor{currentstroke}{rgb}{0.000000,0.000000,0.000000}%
\pgfsetstrokecolor{currentstroke}%
\pgfsetstrokeopacity{0.800000}%
\pgfsetdash{}{0pt}%
\pgfpathmoveto{\pgfqpoint{2.830965in}{2.184296in}}%
\pgfpathcurveto{\pgfqpoint{2.835083in}{2.184296in}}{\pgfqpoint{2.839033in}{2.185932in}}{\pgfqpoint{2.841945in}{2.188844in}}%
\pgfpathcurveto{\pgfqpoint{2.844857in}{2.191756in}}{\pgfqpoint{2.846493in}{2.195706in}}{\pgfqpoint{2.846493in}{2.199824in}}%
\pgfpathcurveto{\pgfqpoint{2.846493in}{2.203942in}}{\pgfqpoint{2.844857in}{2.207892in}}{\pgfqpoint{2.841945in}{2.210804in}}%
\pgfpathcurveto{\pgfqpoint{2.839033in}{2.213716in}}{\pgfqpoint{2.835083in}{2.215352in}}{\pgfqpoint{2.830965in}{2.215352in}}%
\pgfpathcurveto{\pgfqpoint{2.826847in}{2.215352in}}{\pgfqpoint{2.822897in}{2.213716in}}{\pgfqpoint{2.819985in}{2.210804in}}%
\pgfpathcurveto{\pgfqpoint{2.817073in}{2.207892in}}{\pgfqpoint{2.815437in}{2.203942in}}{\pgfqpoint{2.815437in}{2.199824in}}%
\pgfpathcurveto{\pgfqpoint{2.815437in}{2.195706in}}{\pgfqpoint{2.817073in}{2.191756in}}{\pgfqpoint{2.819985in}{2.188844in}}%
\pgfpathcurveto{\pgfqpoint{2.822897in}{2.185932in}}{\pgfqpoint{2.826847in}{2.184296in}}{\pgfqpoint{2.830965in}{2.184296in}}%
\pgfpathclose%
\pgfusepath{fill}%
\end{pgfscope}%
\begin{pgfscope}%
\pgfpathrectangle{\pgfqpoint{0.887500in}{0.275000in}}{\pgfqpoint{4.225000in}{4.225000in}}%
\pgfusepath{clip}%
\pgfsetbuttcap%
\pgfsetroundjoin%
\definecolor{currentfill}{rgb}{0.000000,0.000000,0.000000}%
\pgfsetfillcolor{currentfill}%
\pgfsetfillopacity{0.800000}%
\pgfsetlinewidth{0.000000pt}%
\definecolor{currentstroke}{rgb}{0.000000,0.000000,0.000000}%
\pgfsetstrokecolor{currentstroke}%
\pgfsetstrokeopacity{0.800000}%
\pgfsetdash{}{0pt}%
\pgfpathmoveto{\pgfqpoint{1.713062in}{2.402834in}}%
\pgfpathcurveto{\pgfqpoint{1.717180in}{2.402834in}}{\pgfqpoint{1.721130in}{2.404470in}}{\pgfqpoint{1.724042in}{2.407382in}}%
\pgfpathcurveto{\pgfqpoint{1.726954in}{2.410294in}}{\pgfqpoint{1.728591in}{2.414244in}}{\pgfqpoint{1.728591in}{2.418362in}}%
\pgfpathcurveto{\pgfqpoint{1.728591in}{2.422480in}}{\pgfqpoint{1.726954in}{2.426430in}}{\pgfqpoint{1.724042in}{2.429342in}}%
\pgfpathcurveto{\pgfqpoint{1.721130in}{2.432254in}}{\pgfqpoint{1.717180in}{2.433891in}}{\pgfqpoint{1.713062in}{2.433891in}}%
\pgfpathcurveto{\pgfqpoint{1.708944in}{2.433891in}}{\pgfqpoint{1.704994in}{2.432254in}}{\pgfqpoint{1.702082in}{2.429342in}}%
\pgfpathcurveto{\pgfqpoint{1.699170in}{2.426430in}}{\pgfqpoint{1.697534in}{2.422480in}}{\pgfqpoint{1.697534in}{2.418362in}}%
\pgfpathcurveto{\pgfqpoint{1.697534in}{2.414244in}}{\pgfqpoint{1.699170in}{2.410294in}}{\pgfqpoint{1.702082in}{2.407382in}}%
\pgfpathcurveto{\pgfqpoint{1.704994in}{2.404470in}}{\pgfqpoint{1.708944in}{2.402834in}}{\pgfqpoint{1.713062in}{2.402834in}}%
\pgfpathclose%
\pgfusepath{fill}%
\end{pgfscope}%
\begin{pgfscope}%
\pgfpathrectangle{\pgfqpoint{0.887500in}{0.275000in}}{\pgfqpoint{4.225000in}{4.225000in}}%
\pgfusepath{clip}%
\pgfsetbuttcap%
\pgfsetroundjoin%
\definecolor{currentfill}{rgb}{0.000000,0.000000,0.000000}%
\pgfsetfillcolor{currentfill}%
\pgfsetfillopacity{0.800000}%
\pgfsetlinewidth{0.000000pt}%
\definecolor{currentstroke}{rgb}{0.000000,0.000000,0.000000}%
\pgfsetstrokecolor{currentstroke}%
\pgfsetstrokeopacity{0.800000}%
\pgfsetdash{}{0pt}%
\pgfpathmoveto{\pgfqpoint{4.109779in}{2.459295in}}%
\pgfpathcurveto{\pgfqpoint{4.113897in}{2.459295in}}{\pgfqpoint{4.117847in}{2.460931in}}{\pgfqpoint{4.120759in}{2.463843in}}%
\pgfpathcurveto{\pgfqpoint{4.123671in}{2.466755in}}{\pgfqpoint{4.125307in}{2.470705in}}{\pgfqpoint{4.125307in}{2.474823in}}%
\pgfpathcurveto{\pgfqpoint{4.125307in}{2.478941in}}{\pgfqpoint{4.123671in}{2.482891in}}{\pgfqpoint{4.120759in}{2.485803in}}%
\pgfpathcurveto{\pgfqpoint{4.117847in}{2.488715in}}{\pgfqpoint{4.113897in}{2.490351in}}{\pgfqpoint{4.109779in}{2.490351in}}%
\pgfpathcurveto{\pgfqpoint{4.105661in}{2.490351in}}{\pgfqpoint{4.101711in}{2.488715in}}{\pgfqpoint{4.098799in}{2.485803in}}%
\pgfpathcurveto{\pgfqpoint{4.095887in}{2.482891in}}{\pgfqpoint{4.094251in}{2.478941in}}{\pgfqpoint{4.094251in}{2.474823in}}%
\pgfpathcurveto{\pgfqpoint{4.094251in}{2.470705in}}{\pgfqpoint{4.095887in}{2.466755in}}{\pgfqpoint{4.098799in}{2.463843in}}%
\pgfpathcurveto{\pgfqpoint{4.101711in}{2.460931in}}{\pgfqpoint{4.105661in}{2.459295in}}{\pgfqpoint{4.109779in}{2.459295in}}%
\pgfpathclose%
\pgfusepath{fill}%
\end{pgfscope}%
\begin{pgfscope}%
\pgfpathrectangle{\pgfqpoint{0.887500in}{0.275000in}}{\pgfqpoint{4.225000in}{4.225000in}}%
\pgfusepath{clip}%
\pgfsetbuttcap%
\pgfsetroundjoin%
\definecolor{currentfill}{rgb}{0.000000,0.000000,0.000000}%
\pgfsetfillcolor{currentfill}%
\pgfsetfillopacity{0.800000}%
\pgfsetlinewidth{0.000000pt}%
\definecolor{currentstroke}{rgb}{0.000000,0.000000,0.000000}%
\pgfsetstrokecolor{currentstroke}%
\pgfsetstrokeopacity{0.800000}%
\pgfsetdash{}{0pt}%
\pgfpathmoveto{\pgfqpoint{4.710743in}{1.960338in}}%
\pgfpathcurveto{\pgfqpoint{4.714862in}{1.960338in}}{\pgfqpoint{4.718812in}{1.961974in}}{\pgfqpoint{4.721724in}{1.964886in}}%
\pgfpathcurveto{\pgfqpoint{4.724636in}{1.967798in}}{\pgfqpoint{4.726272in}{1.971748in}}{\pgfqpoint{4.726272in}{1.975867in}}%
\pgfpathcurveto{\pgfqpoint{4.726272in}{1.979985in}}{\pgfqpoint{4.724636in}{1.983935in}}{\pgfqpoint{4.721724in}{1.986847in}}%
\pgfpathcurveto{\pgfqpoint{4.718812in}{1.989759in}}{\pgfqpoint{4.714862in}{1.991395in}}{\pgfqpoint{4.710743in}{1.991395in}}%
\pgfpathcurveto{\pgfqpoint{4.706625in}{1.991395in}}{\pgfqpoint{4.702675in}{1.989759in}}{\pgfqpoint{4.699763in}{1.986847in}}%
\pgfpathcurveto{\pgfqpoint{4.696851in}{1.983935in}}{\pgfqpoint{4.695215in}{1.979985in}}{\pgfqpoint{4.695215in}{1.975867in}}%
\pgfpathcurveto{\pgfqpoint{4.695215in}{1.971748in}}{\pgfqpoint{4.696851in}{1.967798in}}{\pgfqpoint{4.699763in}{1.964886in}}%
\pgfpathcurveto{\pgfqpoint{4.702675in}{1.961974in}}{\pgfqpoint{4.706625in}{1.960338in}}{\pgfqpoint{4.710743in}{1.960338in}}%
\pgfpathclose%
\pgfusepath{fill}%
\end{pgfscope}%
\begin{pgfscope}%
\pgfpathrectangle{\pgfqpoint{0.887500in}{0.275000in}}{\pgfqpoint{4.225000in}{4.225000in}}%
\pgfusepath{clip}%
\pgfsetbuttcap%
\pgfsetroundjoin%
\definecolor{currentfill}{rgb}{0.000000,0.000000,0.000000}%
\pgfsetfillcolor{currentfill}%
\pgfsetfillopacity{0.800000}%
\pgfsetlinewidth{0.000000pt}%
\definecolor{currentstroke}{rgb}{0.000000,0.000000,0.000000}%
\pgfsetstrokecolor{currentstroke}%
\pgfsetstrokeopacity{0.800000}%
\pgfsetdash{}{0pt}%
\pgfpathmoveto{\pgfqpoint{3.931823in}{2.573360in}}%
\pgfpathcurveto{\pgfqpoint{3.935941in}{2.573360in}}{\pgfqpoint{3.939891in}{2.574996in}}{\pgfqpoint{3.942803in}{2.577908in}}%
\pgfpathcurveto{\pgfqpoint{3.945715in}{2.580820in}}{\pgfqpoint{3.947352in}{2.584770in}}{\pgfqpoint{3.947352in}{2.588888in}}%
\pgfpathcurveto{\pgfqpoint{3.947352in}{2.593006in}}{\pgfqpoint{3.945715in}{2.596956in}}{\pgfqpoint{3.942803in}{2.599868in}}%
\pgfpathcurveto{\pgfqpoint{3.939891in}{2.602780in}}{\pgfqpoint{3.935941in}{2.604416in}}{\pgfqpoint{3.931823in}{2.604416in}}%
\pgfpathcurveto{\pgfqpoint{3.927705in}{2.604416in}}{\pgfqpoint{3.923755in}{2.602780in}}{\pgfqpoint{3.920843in}{2.599868in}}%
\pgfpathcurveto{\pgfqpoint{3.917931in}{2.596956in}}{\pgfqpoint{3.916295in}{2.593006in}}{\pgfqpoint{3.916295in}{2.588888in}}%
\pgfpathcurveto{\pgfqpoint{3.916295in}{2.584770in}}{\pgfqpoint{3.917931in}{2.580820in}}{\pgfqpoint{3.920843in}{2.577908in}}%
\pgfpathcurveto{\pgfqpoint{3.923755in}{2.574996in}}{\pgfqpoint{3.927705in}{2.573360in}}{\pgfqpoint{3.931823in}{2.573360in}}%
\pgfpathclose%
\pgfusepath{fill}%
\end{pgfscope}%
\begin{pgfscope}%
\pgfpathrectangle{\pgfqpoint{0.887500in}{0.275000in}}{\pgfqpoint{4.225000in}{4.225000in}}%
\pgfusepath{clip}%
\pgfsetbuttcap%
\pgfsetroundjoin%
\definecolor{currentfill}{rgb}{0.000000,0.000000,0.000000}%
\pgfsetfillcolor{currentfill}%
\pgfsetfillopacity{0.800000}%
\pgfsetlinewidth{0.000000pt}%
\definecolor{currentstroke}{rgb}{0.000000,0.000000,0.000000}%
\pgfsetstrokecolor{currentstroke}%
\pgfsetstrokeopacity{0.800000}%
\pgfsetdash{}{0pt}%
\pgfpathmoveto{\pgfqpoint{2.360778in}{2.269644in}}%
\pgfpathcurveto{\pgfqpoint{2.364896in}{2.269644in}}{\pgfqpoint{2.368846in}{2.271280in}}{\pgfqpoint{2.371758in}{2.274192in}}%
\pgfpathcurveto{\pgfqpoint{2.374670in}{2.277104in}}{\pgfqpoint{2.376306in}{2.281054in}}{\pgfqpoint{2.376306in}{2.285172in}}%
\pgfpathcurveto{\pgfqpoint{2.376306in}{2.289291in}}{\pgfqpoint{2.374670in}{2.293241in}}{\pgfqpoint{2.371758in}{2.296153in}}%
\pgfpathcurveto{\pgfqpoint{2.368846in}{2.299065in}}{\pgfqpoint{2.364896in}{2.300701in}}{\pgfqpoint{2.360778in}{2.300701in}}%
\pgfpathcurveto{\pgfqpoint{2.356660in}{2.300701in}}{\pgfqpoint{2.352710in}{2.299065in}}{\pgfqpoint{2.349798in}{2.296153in}}%
\pgfpathcurveto{\pgfqpoint{2.346886in}{2.293241in}}{\pgfqpoint{2.345250in}{2.289291in}}{\pgfqpoint{2.345250in}{2.285172in}}%
\pgfpathcurveto{\pgfqpoint{2.345250in}{2.281054in}}{\pgfqpoint{2.346886in}{2.277104in}}{\pgfqpoint{2.349798in}{2.274192in}}%
\pgfpathcurveto{\pgfqpoint{2.352710in}{2.271280in}}{\pgfqpoint{2.356660in}{2.269644in}}{\pgfqpoint{2.360778in}{2.269644in}}%
\pgfpathclose%
\pgfusepath{fill}%
\end{pgfscope}%
\begin{pgfscope}%
\pgfpathrectangle{\pgfqpoint{0.887500in}{0.275000in}}{\pgfqpoint{4.225000in}{4.225000in}}%
\pgfusepath{clip}%
\pgfsetbuttcap%
\pgfsetroundjoin%
\definecolor{currentfill}{rgb}{0.000000,0.000000,0.000000}%
\pgfsetfillcolor{currentfill}%
\pgfsetfillopacity{0.800000}%
\pgfsetlinewidth{0.000000pt}%
\definecolor{currentstroke}{rgb}{0.000000,0.000000,0.000000}%
\pgfsetstrokecolor{currentstroke}%
\pgfsetstrokeopacity{0.800000}%
\pgfsetdash{}{0pt}%
\pgfpathmoveto{\pgfqpoint{3.009270in}{2.120143in}}%
\pgfpathcurveto{\pgfqpoint{3.013388in}{2.120143in}}{\pgfqpoint{3.017338in}{2.121779in}}{\pgfqpoint{3.020250in}{2.124691in}}%
\pgfpathcurveto{\pgfqpoint{3.023162in}{2.127603in}}{\pgfqpoint{3.024798in}{2.131553in}}{\pgfqpoint{3.024798in}{2.135671in}}%
\pgfpathcurveto{\pgfqpoint{3.024798in}{2.139790in}}{\pgfqpoint{3.023162in}{2.143740in}}{\pgfqpoint{3.020250in}{2.146652in}}%
\pgfpathcurveto{\pgfqpoint{3.017338in}{2.149564in}}{\pgfqpoint{3.013388in}{2.151200in}}{\pgfqpoint{3.009270in}{2.151200in}}%
\pgfpathcurveto{\pgfqpoint{3.005151in}{2.151200in}}{\pgfqpoint{3.001201in}{2.149564in}}{\pgfqpoint{2.998289in}{2.146652in}}%
\pgfpathcurveto{\pgfqpoint{2.995377in}{2.143740in}}{\pgfqpoint{2.993741in}{2.139790in}}{\pgfqpoint{2.993741in}{2.135671in}}%
\pgfpathcurveto{\pgfqpoint{2.993741in}{2.131553in}}{\pgfqpoint{2.995377in}{2.127603in}}{\pgfqpoint{2.998289in}{2.124691in}}%
\pgfpathcurveto{\pgfqpoint{3.001201in}{2.121779in}}{\pgfqpoint{3.005151in}{2.120143in}}{\pgfqpoint{3.009270in}{2.120143in}}%
\pgfpathclose%
\pgfusepath{fill}%
\end{pgfscope}%
\begin{pgfscope}%
\pgfpathrectangle{\pgfqpoint{0.887500in}{0.275000in}}{\pgfqpoint{4.225000in}{4.225000in}}%
\pgfusepath{clip}%
\pgfsetbuttcap%
\pgfsetroundjoin%
\definecolor{currentfill}{rgb}{0.000000,0.000000,0.000000}%
\pgfsetfillcolor{currentfill}%
\pgfsetfillopacity{0.800000}%
\pgfsetlinewidth{0.000000pt}%
\definecolor{currentstroke}{rgb}{0.000000,0.000000,0.000000}%
\pgfsetstrokecolor{currentstroke}%
\pgfsetstrokeopacity{0.800000}%
\pgfsetdash{}{0pt}%
\pgfpathmoveto{\pgfqpoint{3.753656in}{2.684930in}}%
\pgfpathcurveto{\pgfqpoint{3.757774in}{2.684930in}}{\pgfqpoint{3.761724in}{2.686566in}}{\pgfqpoint{3.764636in}{2.689478in}}%
\pgfpathcurveto{\pgfqpoint{3.767548in}{2.692390in}}{\pgfqpoint{3.769184in}{2.696340in}}{\pgfqpoint{3.769184in}{2.700458in}}%
\pgfpathcurveto{\pgfqpoint{3.769184in}{2.704577in}}{\pgfqpoint{3.767548in}{2.708527in}}{\pgfqpoint{3.764636in}{2.711439in}}%
\pgfpathcurveto{\pgfqpoint{3.761724in}{2.714351in}}{\pgfqpoint{3.757774in}{2.715987in}}{\pgfqpoint{3.753656in}{2.715987in}}%
\pgfpathcurveto{\pgfqpoint{3.749538in}{2.715987in}}{\pgfqpoint{3.745588in}{2.714351in}}{\pgfqpoint{3.742676in}{2.711439in}}%
\pgfpathcurveto{\pgfqpoint{3.739764in}{2.708527in}}{\pgfqpoint{3.738128in}{2.704577in}}{\pgfqpoint{3.738128in}{2.700458in}}%
\pgfpathcurveto{\pgfqpoint{3.738128in}{2.696340in}}{\pgfqpoint{3.739764in}{2.692390in}}{\pgfqpoint{3.742676in}{2.689478in}}%
\pgfpathcurveto{\pgfqpoint{3.745588in}{2.686566in}}{\pgfqpoint{3.749538in}{2.684930in}}{\pgfqpoint{3.753656in}{2.684930in}}%
\pgfpathclose%
\pgfusepath{fill}%
\end{pgfscope}%
\begin{pgfscope}%
\pgfpathrectangle{\pgfqpoint{0.887500in}{0.275000in}}{\pgfqpoint{4.225000in}{4.225000in}}%
\pgfusepath{clip}%
\pgfsetbuttcap%
\pgfsetroundjoin%
\definecolor{currentfill}{rgb}{0.000000,0.000000,0.000000}%
\pgfsetfillcolor{currentfill}%
\pgfsetfillopacity{0.800000}%
\pgfsetlinewidth{0.000000pt}%
\definecolor{currentstroke}{rgb}{0.000000,0.000000,0.000000}%
\pgfsetstrokecolor{currentstroke}%
\pgfsetstrokeopacity{0.800000}%
\pgfsetdash{}{0pt}%
\pgfpathmoveto{\pgfqpoint{4.533191in}{2.079163in}}%
\pgfpathcurveto{\pgfqpoint{4.537309in}{2.079163in}}{\pgfqpoint{4.541259in}{2.080799in}}{\pgfqpoint{4.544171in}{2.083711in}}%
\pgfpathcurveto{\pgfqpoint{4.547083in}{2.086623in}}{\pgfqpoint{4.548719in}{2.090573in}}{\pgfqpoint{4.548719in}{2.094691in}}%
\pgfpathcurveto{\pgfqpoint{4.548719in}{2.098809in}}{\pgfqpoint{4.547083in}{2.102759in}}{\pgfqpoint{4.544171in}{2.105671in}}%
\pgfpathcurveto{\pgfqpoint{4.541259in}{2.108583in}}{\pgfqpoint{4.537309in}{2.110219in}}{\pgfqpoint{4.533191in}{2.110219in}}%
\pgfpathcurveto{\pgfqpoint{4.529072in}{2.110219in}}{\pgfqpoint{4.525122in}{2.108583in}}{\pgfqpoint{4.522210in}{2.105671in}}%
\pgfpathcurveto{\pgfqpoint{4.519298in}{2.102759in}}{\pgfqpoint{4.517662in}{2.098809in}}{\pgfqpoint{4.517662in}{2.094691in}}%
\pgfpathcurveto{\pgfqpoint{4.517662in}{2.090573in}}{\pgfqpoint{4.519298in}{2.086623in}}{\pgfqpoint{4.522210in}{2.083711in}}%
\pgfpathcurveto{\pgfqpoint{4.525122in}{2.080799in}}{\pgfqpoint{4.529072in}{2.079163in}}{\pgfqpoint{4.533191in}{2.079163in}}%
\pgfpathclose%
\pgfusepath{fill}%
\end{pgfscope}%
\begin{pgfscope}%
\pgfpathrectangle{\pgfqpoint{0.887500in}{0.275000in}}{\pgfqpoint{4.225000in}{4.225000in}}%
\pgfusepath{clip}%
\pgfsetbuttcap%
\pgfsetroundjoin%
\definecolor{currentfill}{rgb}{0.000000,0.000000,0.000000}%
\pgfsetfillcolor{currentfill}%
\pgfsetfillopacity{0.800000}%
\pgfsetlinewidth{0.000000pt}%
\definecolor{currentstroke}{rgb}{0.000000,0.000000,0.000000}%
\pgfsetstrokecolor{currentstroke}%
\pgfsetstrokeopacity{0.800000}%
\pgfsetdash{}{0pt}%
\pgfpathmoveto{\pgfqpoint{1.890199in}{2.348307in}}%
\pgfpathcurveto{\pgfqpoint{1.894318in}{2.348307in}}{\pgfqpoint{1.898268in}{2.349944in}}{\pgfqpoint{1.901180in}{2.352856in}}%
\pgfpathcurveto{\pgfqpoint{1.904092in}{2.355767in}}{\pgfqpoint{1.905728in}{2.359718in}}{\pgfqpoint{1.905728in}{2.363836in}}%
\pgfpathcurveto{\pgfqpoint{1.905728in}{2.367954in}}{\pgfqpoint{1.904092in}{2.371904in}}{\pgfqpoint{1.901180in}{2.374816in}}%
\pgfpathcurveto{\pgfqpoint{1.898268in}{2.377728in}}{\pgfqpoint{1.894318in}{2.379364in}}{\pgfqpoint{1.890199in}{2.379364in}}%
\pgfpathcurveto{\pgfqpoint{1.886081in}{2.379364in}}{\pgfqpoint{1.882131in}{2.377728in}}{\pgfqpoint{1.879219in}{2.374816in}}%
\pgfpathcurveto{\pgfqpoint{1.876307in}{2.371904in}}{\pgfqpoint{1.874671in}{2.367954in}}{\pgfqpoint{1.874671in}{2.363836in}}%
\pgfpathcurveto{\pgfqpoint{1.874671in}{2.359718in}}{\pgfqpoint{1.876307in}{2.355767in}}{\pgfqpoint{1.879219in}{2.352856in}}%
\pgfpathcurveto{\pgfqpoint{1.882131in}{2.349944in}}{\pgfqpoint{1.886081in}{2.348307in}}{\pgfqpoint{1.890199in}{2.348307in}}%
\pgfpathclose%
\pgfusepath{fill}%
\end{pgfscope}%
\begin{pgfscope}%
\pgfpathrectangle{\pgfqpoint{0.887500in}{0.275000in}}{\pgfqpoint{4.225000in}{4.225000in}}%
\pgfusepath{clip}%
\pgfsetbuttcap%
\pgfsetroundjoin%
\definecolor{currentfill}{rgb}{0.000000,0.000000,0.000000}%
\pgfsetfillcolor{currentfill}%
\pgfsetfillopacity{0.800000}%
\pgfsetlinewidth{0.000000pt}%
\definecolor{currentstroke}{rgb}{0.000000,0.000000,0.000000}%
\pgfsetstrokecolor{currentstroke}%
\pgfsetstrokeopacity{0.800000}%
\pgfsetdash{}{0pt}%
\pgfpathmoveto{\pgfqpoint{3.057095in}{2.493251in}}%
\pgfpathcurveto{\pgfqpoint{3.061213in}{2.493251in}}{\pgfqpoint{3.065163in}{2.494887in}}{\pgfqpoint{3.068075in}{2.497799in}}%
\pgfpathcurveto{\pgfqpoint{3.070987in}{2.500711in}}{\pgfqpoint{3.072623in}{2.504661in}}{\pgfqpoint{3.072623in}{2.508779in}}%
\pgfpathcurveto{\pgfqpoint{3.072623in}{2.512898in}}{\pgfqpoint{3.070987in}{2.516848in}}{\pgfqpoint{3.068075in}{2.519760in}}%
\pgfpathcurveto{\pgfqpoint{3.065163in}{2.522672in}}{\pgfqpoint{3.061213in}{2.524308in}}{\pgfqpoint{3.057095in}{2.524308in}}%
\pgfpathcurveto{\pgfqpoint{3.052976in}{2.524308in}}{\pgfqpoint{3.049026in}{2.522672in}}{\pgfqpoint{3.046114in}{2.519760in}}%
\pgfpathcurveto{\pgfqpoint{3.043202in}{2.516848in}}{\pgfqpoint{3.041566in}{2.512898in}}{\pgfqpoint{3.041566in}{2.508779in}}%
\pgfpathcurveto{\pgfqpoint{3.041566in}{2.504661in}}{\pgfqpoint{3.043202in}{2.500711in}}{\pgfqpoint{3.046114in}{2.497799in}}%
\pgfpathcurveto{\pgfqpoint{3.049026in}{2.494887in}}{\pgfqpoint{3.052976in}{2.493251in}}{\pgfqpoint{3.057095in}{2.493251in}}%
\pgfpathclose%
\pgfusepath{fill}%
\end{pgfscope}%
\begin{pgfscope}%
\pgfpathrectangle{\pgfqpoint{0.887500in}{0.275000in}}{\pgfqpoint{4.225000in}{4.225000in}}%
\pgfusepath{clip}%
\pgfsetbuttcap%
\pgfsetroundjoin%
\definecolor{currentfill}{rgb}{0.000000,0.000000,0.000000}%
\pgfsetfillcolor{currentfill}%
\pgfsetfillopacity{0.800000}%
\pgfsetlinewidth{0.000000pt}%
\definecolor{currentstroke}{rgb}{0.000000,0.000000,0.000000}%
\pgfsetstrokecolor{currentstroke}%
\pgfsetstrokeopacity{0.800000}%
\pgfsetdash{}{0pt}%
\pgfpathmoveto{\pgfqpoint{3.575296in}{2.792677in}}%
\pgfpathcurveto{\pgfqpoint{3.579414in}{2.792677in}}{\pgfqpoint{3.583364in}{2.794313in}}{\pgfqpoint{3.586276in}{2.797225in}}%
\pgfpathcurveto{\pgfqpoint{3.589188in}{2.800137in}}{\pgfqpoint{3.590824in}{2.804087in}}{\pgfqpoint{3.590824in}{2.808205in}}%
\pgfpathcurveto{\pgfqpoint{3.590824in}{2.812323in}}{\pgfqpoint{3.589188in}{2.816273in}}{\pgfqpoint{3.586276in}{2.819185in}}%
\pgfpathcurveto{\pgfqpoint{3.583364in}{2.822097in}}{\pgfqpoint{3.579414in}{2.823733in}}{\pgfqpoint{3.575296in}{2.823733in}}%
\pgfpathcurveto{\pgfqpoint{3.571177in}{2.823733in}}{\pgfqpoint{3.567227in}{2.822097in}}{\pgfqpoint{3.564315in}{2.819185in}}%
\pgfpathcurveto{\pgfqpoint{3.561403in}{2.816273in}}{\pgfqpoint{3.559767in}{2.812323in}}{\pgfqpoint{3.559767in}{2.808205in}}%
\pgfpathcurveto{\pgfqpoint{3.559767in}{2.804087in}}{\pgfqpoint{3.561403in}{2.800137in}}{\pgfqpoint{3.564315in}{2.797225in}}%
\pgfpathcurveto{\pgfqpoint{3.567227in}{2.794313in}}{\pgfqpoint{3.571177in}{2.792677in}}{\pgfqpoint{3.575296in}{2.792677in}}%
\pgfpathclose%
\pgfusepath{fill}%
\end{pgfscope}%
\begin{pgfscope}%
\pgfpathrectangle{\pgfqpoint{0.887500in}{0.275000in}}{\pgfqpoint{4.225000in}{4.225000in}}%
\pgfusepath{clip}%
\pgfsetbuttcap%
\pgfsetroundjoin%
\definecolor{currentfill}{rgb}{0.000000,0.000000,0.000000}%
\pgfsetfillcolor{currentfill}%
\pgfsetfillopacity{0.800000}%
\pgfsetlinewidth{0.000000pt}%
\definecolor{currentstroke}{rgb}{0.000000,0.000000,0.000000}%
\pgfsetstrokecolor{currentstroke}%
\pgfsetstrokeopacity{0.800000}%
\pgfsetdash{}{0pt}%
\pgfpathmoveto{\pgfqpoint{3.218164in}{2.986860in}}%
\pgfpathcurveto{\pgfqpoint{3.222282in}{2.986860in}}{\pgfqpoint{3.226232in}{2.988496in}}{\pgfqpoint{3.229144in}{2.991408in}}%
\pgfpathcurveto{\pgfqpoint{3.232056in}{2.994320in}}{\pgfqpoint{3.233692in}{2.998270in}}{\pgfqpoint{3.233692in}{3.002388in}}%
\pgfpathcurveto{\pgfqpoint{3.233692in}{3.006506in}}{\pgfqpoint{3.232056in}{3.010456in}}{\pgfqpoint{3.229144in}{3.013368in}}%
\pgfpathcurveto{\pgfqpoint{3.226232in}{3.016280in}}{\pgfqpoint{3.222282in}{3.017916in}}{\pgfqpoint{3.218164in}{3.017916in}}%
\pgfpathcurveto{\pgfqpoint{3.214046in}{3.017916in}}{\pgfqpoint{3.210096in}{3.016280in}}{\pgfqpoint{3.207184in}{3.013368in}}%
\pgfpathcurveto{\pgfqpoint{3.204272in}{3.010456in}}{\pgfqpoint{3.202635in}{3.006506in}}{\pgfqpoint{3.202635in}{3.002388in}}%
\pgfpathcurveto{\pgfqpoint{3.202635in}{2.998270in}}{\pgfqpoint{3.204272in}{2.994320in}}{\pgfqpoint{3.207184in}{2.991408in}}%
\pgfpathcurveto{\pgfqpoint{3.210096in}{2.988496in}}{\pgfqpoint{3.214046in}{2.986860in}}{\pgfqpoint{3.218164in}{2.986860in}}%
\pgfpathclose%
\pgfusepath{fill}%
\end{pgfscope}%
\begin{pgfscope}%
\pgfpathrectangle{\pgfqpoint{0.887500in}{0.275000in}}{\pgfqpoint{4.225000in}{4.225000in}}%
\pgfusepath{clip}%
\pgfsetbuttcap%
\pgfsetroundjoin%
\definecolor{currentfill}{rgb}{0.000000,0.000000,0.000000}%
\pgfsetfillcolor{currentfill}%
\pgfsetfillopacity{0.800000}%
\pgfsetlinewidth{0.000000pt}%
\definecolor{currentstroke}{rgb}{0.000000,0.000000,0.000000}%
\pgfsetstrokecolor{currentstroke}%
\pgfsetstrokeopacity{0.800000}%
\pgfsetdash{}{0pt}%
\pgfpathmoveto{\pgfqpoint{4.355048in}{2.186650in}}%
\pgfpathcurveto{\pgfqpoint{4.359166in}{2.186650in}}{\pgfqpoint{4.363116in}{2.188286in}}{\pgfqpoint{4.366028in}{2.191198in}}%
\pgfpathcurveto{\pgfqpoint{4.368940in}{2.194110in}}{\pgfqpoint{4.370576in}{2.198060in}}{\pgfqpoint{4.370576in}{2.202178in}}%
\pgfpathcurveto{\pgfqpoint{4.370576in}{2.206296in}}{\pgfqpoint{4.368940in}{2.210246in}}{\pgfqpoint{4.366028in}{2.213158in}}%
\pgfpathcurveto{\pgfqpoint{4.363116in}{2.216070in}}{\pgfqpoint{4.359166in}{2.217706in}}{\pgfqpoint{4.355048in}{2.217706in}}%
\pgfpathcurveto{\pgfqpoint{4.350930in}{2.217706in}}{\pgfqpoint{4.346980in}{2.216070in}}{\pgfqpoint{4.344068in}{2.213158in}}%
\pgfpathcurveto{\pgfqpoint{4.341156in}{2.210246in}}{\pgfqpoint{4.339520in}{2.206296in}}{\pgfqpoint{4.339520in}{2.202178in}}%
\pgfpathcurveto{\pgfqpoint{4.339520in}{2.198060in}}{\pgfqpoint{4.341156in}{2.194110in}}{\pgfqpoint{4.344068in}{2.191198in}}%
\pgfpathcurveto{\pgfqpoint{4.346980in}{2.188286in}}{\pgfqpoint{4.350930in}{2.186650in}}{\pgfqpoint{4.355048in}{2.186650in}}%
\pgfpathclose%
\pgfusepath{fill}%
\end{pgfscope}%
\begin{pgfscope}%
\pgfpathrectangle{\pgfqpoint{0.887500in}{0.275000in}}{\pgfqpoint{4.225000in}{4.225000in}}%
\pgfusepath{clip}%
\pgfsetbuttcap%
\pgfsetroundjoin%
\definecolor{currentfill}{rgb}{0.000000,0.000000,0.000000}%
\pgfsetfillcolor{currentfill}%
\pgfsetfillopacity{0.800000}%
\pgfsetlinewidth{0.000000pt}%
\definecolor{currentstroke}{rgb}{0.000000,0.000000,0.000000}%
\pgfsetstrokecolor{currentstroke}%
\pgfsetstrokeopacity{0.800000}%
\pgfsetdash{}{0pt}%
\pgfpathmoveto{\pgfqpoint{3.396777in}{2.894560in}}%
\pgfpathcurveto{\pgfqpoint{3.400895in}{2.894560in}}{\pgfqpoint{3.404845in}{2.896196in}}{\pgfqpoint{3.407757in}{2.899108in}}%
\pgfpathcurveto{\pgfqpoint{3.410669in}{2.902020in}}{\pgfqpoint{3.412305in}{2.905970in}}{\pgfqpoint{3.412305in}{2.910088in}}%
\pgfpathcurveto{\pgfqpoint{3.412305in}{2.914206in}}{\pgfqpoint{3.410669in}{2.918156in}}{\pgfqpoint{3.407757in}{2.921068in}}%
\pgfpathcurveto{\pgfqpoint{3.404845in}{2.923980in}}{\pgfqpoint{3.400895in}{2.925616in}}{\pgfqpoint{3.396777in}{2.925616in}}%
\pgfpathcurveto{\pgfqpoint{3.392658in}{2.925616in}}{\pgfqpoint{3.388708in}{2.923980in}}{\pgfqpoint{3.385796in}{2.921068in}}%
\pgfpathcurveto{\pgfqpoint{3.382884in}{2.918156in}}{\pgfqpoint{3.381248in}{2.914206in}}{\pgfqpoint{3.381248in}{2.910088in}}%
\pgfpathcurveto{\pgfqpoint{3.381248in}{2.905970in}}{\pgfqpoint{3.382884in}{2.902020in}}{\pgfqpoint{3.385796in}{2.899108in}}%
\pgfpathcurveto{\pgfqpoint{3.388708in}{2.896196in}}{\pgfqpoint{3.392658in}{2.894560in}}{\pgfqpoint{3.396777in}{2.894560in}}%
\pgfpathclose%
\pgfusepath{fill}%
\end{pgfscope}%
\begin{pgfscope}%
\pgfpathrectangle{\pgfqpoint{0.887500in}{0.275000in}}{\pgfqpoint{4.225000in}{4.225000in}}%
\pgfusepath{clip}%
\pgfsetbuttcap%
\pgfsetroundjoin%
\definecolor{currentfill}{rgb}{0.000000,0.000000,0.000000}%
\pgfsetfillcolor{currentfill}%
\pgfsetfillopacity{0.800000}%
\pgfsetlinewidth{0.000000pt}%
\definecolor{currentstroke}{rgb}{0.000000,0.000000,0.000000}%
\pgfsetstrokecolor{currentstroke}%
\pgfsetstrokeopacity{0.800000}%
\pgfsetdash{}{0pt}%
\pgfpathmoveto{\pgfqpoint{3.104956in}{2.864572in}}%
\pgfpathcurveto{\pgfqpoint{3.109074in}{2.864572in}}{\pgfqpoint{3.113024in}{2.866208in}}{\pgfqpoint{3.115936in}{2.869120in}}%
\pgfpathcurveto{\pgfqpoint{3.118848in}{2.872032in}}{\pgfqpoint{3.120484in}{2.875982in}}{\pgfqpoint{3.120484in}{2.880100in}}%
\pgfpathcurveto{\pgfqpoint{3.120484in}{2.884219in}}{\pgfqpoint{3.118848in}{2.888169in}}{\pgfqpoint{3.115936in}{2.891081in}}%
\pgfpathcurveto{\pgfqpoint{3.113024in}{2.893992in}}{\pgfqpoint{3.109074in}{2.895629in}}{\pgfqpoint{3.104956in}{2.895629in}}%
\pgfpathcurveto{\pgfqpoint{3.100838in}{2.895629in}}{\pgfqpoint{3.096888in}{2.893992in}}{\pgfqpoint{3.093976in}{2.891081in}}%
\pgfpathcurveto{\pgfqpoint{3.091064in}{2.888169in}}{\pgfqpoint{3.089428in}{2.884219in}}{\pgfqpoint{3.089428in}{2.880100in}}%
\pgfpathcurveto{\pgfqpoint{3.089428in}{2.875982in}}{\pgfqpoint{3.091064in}{2.872032in}}{\pgfqpoint{3.093976in}{2.869120in}}%
\pgfpathcurveto{\pgfqpoint{3.096888in}{2.866208in}}{\pgfqpoint{3.100838in}{2.864572in}}{\pgfqpoint{3.104956in}{2.864572in}}%
\pgfpathclose%
\pgfusepath{fill}%
\end{pgfscope}%
\begin{pgfscope}%
\pgfpathrectangle{\pgfqpoint{0.887500in}{0.275000in}}{\pgfqpoint{4.225000in}{4.225000in}}%
\pgfusepath{clip}%
\pgfsetbuttcap%
\pgfsetroundjoin%
\definecolor{currentfill}{rgb}{0.000000,0.000000,0.000000}%
\pgfsetfillcolor{currentfill}%
\pgfsetfillopacity{0.800000}%
\pgfsetlinewidth{0.000000pt}%
\definecolor{currentstroke}{rgb}{0.000000,0.000000,0.000000}%
\pgfsetstrokecolor{currentstroke}%
\pgfsetstrokeopacity{0.800000}%
\pgfsetdash{}{0pt}%
\pgfpathmoveto{\pgfqpoint{2.538741in}{2.211347in}}%
\pgfpathcurveto{\pgfqpoint{2.542859in}{2.211347in}}{\pgfqpoint{2.546809in}{2.212983in}}{\pgfqpoint{2.549721in}{2.215895in}}%
\pgfpathcurveto{\pgfqpoint{2.552633in}{2.218807in}}{\pgfqpoint{2.554269in}{2.222757in}}{\pgfqpoint{2.554269in}{2.226875in}}%
\pgfpathcurveto{\pgfqpoint{2.554269in}{2.230993in}}{\pgfqpoint{2.552633in}{2.234943in}}{\pgfqpoint{2.549721in}{2.237855in}}%
\pgfpathcurveto{\pgfqpoint{2.546809in}{2.240767in}}{\pgfqpoint{2.542859in}{2.242403in}}{\pgfqpoint{2.538741in}{2.242403in}}%
\pgfpathcurveto{\pgfqpoint{2.534622in}{2.242403in}}{\pgfqpoint{2.530672in}{2.240767in}}{\pgfqpoint{2.527760in}{2.237855in}}%
\pgfpathcurveto{\pgfqpoint{2.524848in}{2.234943in}}{\pgfqpoint{2.523212in}{2.230993in}}{\pgfqpoint{2.523212in}{2.226875in}}%
\pgfpathcurveto{\pgfqpoint{2.523212in}{2.222757in}}{\pgfqpoint{2.524848in}{2.218807in}}{\pgfqpoint{2.527760in}{2.215895in}}%
\pgfpathcurveto{\pgfqpoint{2.530672in}{2.212983in}}{\pgfqpoint{2.534622in}{2.211347in}}{\pgfqpoint{2.538741in}{2.211347in}}%
\pgfpathclose%
\pgfusepath{fill}%
\end{pgfscope}%
\begin{pgfscope}%
\pgfpathrectangle{\pgfqpoint{0.887500in}{0.275000in}}{\pgfqpoint{4.225000in}{4.225000in}}%
\pgfusepath{clip}%
\pgfsetbuttcap%
\pgfsetroundjoin%
\definecolor{currentfill}{rgb}{0.000000,0.000000,0.000000}%
\pgfsetfillcolor{currentfill}%
\pgfsetfillopacity{0.800000}%
\pgfsetlinewidth{0.000000pt}%
\definecolor{currentstroke}{rgb}{0.000000,0.000000,0.000000}%
\pgfsetstrokecolor{currentstroke}%
\pgfsetstrokeopacity{0.800000}%
\pgfsetdash{}{0pt}%
\pgfpathmoveto{\pgfqpoint{4.177219in}{2.308981in}}%
\pgfpathcurveto{\pgfqpoint{4.181337in}{2.308981in}}{\pgfqpoint{4.185287in}{2.310617in}}{\pgfqpoint{4.188199in}{2.313529in}}%
\pgfpathcurveto{\pgfqpoint{4.191111in}{2.316441in}}{\pgfqpoint{4.192747in}{2.320391in}}{\pgfqpoint{4.192747in}{2.324509in}}%
\pgfpathcurveto{\pgfqpoint{4.192747in}{2.328627in}}{\pgfqpoint{4.191111in}{2.332577in}}{\pgfqpoint{4.188199in}{2.335489in}}%
\pgfpathcurveto{\pgfqpoint{4.185287in}{2.338401in}}{\pgfqpoint{4.181337in}{2.340037in}}{\pgfqpoint{4.177219in}{2.340037in}}%
\pgfpathcurveto{\pgfqpoint{4.173100in}{2.340037in}}{\pgfqpoint{4.169150in}{2.338401in}}{\pgfqpoint{4.166238in}{2.335489in}}%
\pgfpathcurveto{\pgfqpoint{4.163326in}{2.332577in}}{\pgfqpoint{4.161690in}{2.328627in}}{\pgfqpoint{4.161690in}{2.324509in}}%
\pgfpathcurveto{\pgfqpoint{4.161690in}{2.320391in}}{\pgfqpoint{4.163326in}{2.316441in}}{\pgfqpoint{4.166238in}{2.313529in}}%
\pgfpathcurveto{\pgfqpoint{4.169150in}{2.310617in}}{\pgfqpoint{4.173100in}{2.308981in}}{\pgfqpoint{4.177219in}{2.308981in}}%
\pgfpathclose%
\pgfusepath{fill}%
\end{pgfscope}%
\begin{pgfscope}%
\pgfpathrectangle{\pgfqpoint{0.887500in}{0.275000in}}{\pgfqpoint{4.225000in}{4.225000in}}%
\pgfusepath{clip}%
\pgfsetbuttcap%
\pgfsetroundjoin%
\definecolor{currentfill}{rgb}{0.000000,0.000000,0.000000}%
\pgfsetfillcolor{currentfill}%
\pgfsetfillopacity{0.800000}%
\pgfsetlinewidth{0.000000pt}%
\definecolor{currentstroke}{rgb}{0.000000,0.000000,0.000000}%
\pgfsetstrokecolor{currentstroke}%
\pgfsetstrokeopacity{0.800000}%
\pgfsetdash{}{0pt}%
\pgfpathmoveto{\pgfqpoint{2.067736in}{2.292464in}}%
\pgfpathcurveto{\pgfqpoint{2.071854in}{2.292464in}}{\pgfqpoint{2.075804in}{2.294100in}}{\pgfqpoint{2.078716in}{2.297012in}}%
\pgfpathcurveto{\pgfqpoint{2.081628in}{2.299924in}}{\pgfqpoint{2.083264in}{2.303874in}}{\pgfqpoint{2.083264in}{2.307992in}}%
\pgfpathcurveto{\pgfqpoint{2.083264in}{2.312111in}}{\pgfqpoint{2.081628in}{2.316061in}}{\pgfqpoint{2.078716in}{2.318972in}}%
\pgfpathcurveto{\pgfqpoint{2.075804in}{2.321884in}}{\pgfqpoint{2.071854in}{2.323521in}}{\pgfqpoint{2.067736in}{2.323521in}}%
\pgfpathcurveto{\pgfqpoint{2.063618in}{2.323521in}}{\pgfqpoint{2.059668in}{2.321884in}}{\pgfqpoint{2.056756in}{2.318972in}}%
\pgfpathcurveto{\pgfqpoint{2.053844in}{2.316061in}}{\pgfqpoint{2.052208in}{2.312111in}}{\pgfqpoint{2.052208in}{2.307992in}}%
\pgfpathcurveto{\pgfqpoint{2.052208in}{2.303874in}}{\pgfqpoint{2.053844in}{2.299924in}}{\pgfqpoint{2.056756in}{2.297012in}}%
\pgfpathcurveto{\pgfqpoint{2.059668in}{2.294100in}}{\pgfqpoint{2.063618in}{2.292464in}}{\pgfqpoint{2.067736in}{2.292464in}}%
\pgfpathclose%
\pgfusepath{fill}%
\end{pgfscope}%
\begin{pgfscope}%
\pgfpathrectangle{\pgfqpoint{0.887500in}{0.275000in}}{\pgfqpoint{4.225000in}{4.225000in}}%
\pgfusepath{clip}%
\pgfsetbuttcap%
\pgfsetroundjoin%
\definecolor{currentfill}{rgb}{0.000000,0.000000,0.000000}%
\pgfsetfillcolor{currentfill}%
\pgfsetfillopacity{0.800000}%
\pgfsetlinewidth{0.000000pt}%
\definecolor{currentstroke}{rgb}{0.000000,0.000000,0.000000}%
\pgfsetstrokecolor{currentstroke}%
\pgfsetstrokeopacity{0.800000}%
\pgfsetdash{}{0pt}%
\pgfpathmoveto{\pgfqpoint{4.779234in}{1.806038in}}%
\pgfpathcurveto{\pgfqpoint{4.783352in}{1.806038in}}{\pgfqpoint{4.787302in}{1.807674in}}{\pgfqpoint{4.790214in}{1.810586in}}%
\pgfpathcurveto{\pgfqpoint{4.793126in}{1.813498in}}{\pgfqpoint{4.794762in}{1.817448in}}{\pgfqpoint{4.794762in}{1.821566in}}%
\pgfpathcurveto{\pgfqpoint{4.794762in}{1.825684in}}{\pgfqpoint{4.793126in}{1.829634in}}{\pgfqpoint{4.790214in}{1.832546in}}%
\pgfpathcurveto{\pgfqpoint{4.787302in}{1.835458in}}{\pgfqpoint{4.783352in}{1.837094in}}{\pgfqpoint{4.779234in}{1.837094in}}%
\pgfpathcurveto{\pgfqpoint{4.775116in}{1.837094in}}{\pgfqpoint{4.771166in}{1.835458in}}{\pgfqpoint{4.768254in}{1.832546in}}%
\pgfpathcurveto{\pgfqpoint{4.765342in}{1.829634in}}{\pgfqpoint{4.763706in}{1.825684in}}{\pgfqpoint{4.763706in}{1.821566in}}%
\pgfpathcurveto{\pgfqpoint{4.763706in}{1.817448in}}{\pgfqpoint{4.765342in}{1.813498in}}{\pgfqpoint{4.768254in}{1.810586in}}%
\pgfpathcurveto{\pgfqpoint{4.771166in}{1.807674in}}{\pgfqpoint{4.775116in}{1.806038in}}{\pgfqpoint{4.779234in}{1.806038in}}%
\pgfpathclose%
\pgfusepath{fill}%
\end{pgfscope}%
\begin{pgfscope}%
\pgfpathrectangle{\pgfqpoint{0.887500in}{0.275000in}}{\pgfqpoint{4.225000in}{4.225000in}}%
\pgfusepath{clip}%
\pgfsetbuttcap%
\pgfsetroundjoin%
\definecolor{currentfill}{rgb}{0.000000,0.000000,0.000000}%
\pgfsetfillcolor{currentfill}%
\pgfsetfillopacity{0.800000}%
\pgfsetlinewidth{0.000000pt}%
\definecolor{currentstroke}{rgb}{0.000000,0.000000,0.000000}%
\pgfsetstrokecolor{currentstroke}%
\pgfsetstrokeopacity{0.800000}%
\pgfsetdash{}{0pt}%
\pgfpathmoveto{\pgfqpoint{2.717061in}{2.150712in}}%
\pgfpathcurveto{\pgfqpoint{2.721179in}{2.150712in}}{\pgfqpoint{2.725129in}{2.152348in}}{\pgfqpoint{2.728041in}{2.155260in}}%
\pgfpathcurveto{\pgfqpoint{2.730953in}{2.158172in}}{\pgfqpoint{2.732589in}{2.162122in}}{\pgfqpoint{2.732589in}{2.166240in}}%
\pgfpathcurveto{\pgfqpoint{2.732589in}{2.170358in}}{\pgfqpoint{2.730953in}{2.174308in}}{\pgfqpoint{2.728041in}{2.177220in}}%
\pgfpathcurveto{\pgfqpoint{2.725129in}{2.180132in}}{\pgfqpoint{2.721179in}{2.181768in}}{\pgfqpoint{2.717061in}{2.181768in}}%
\pgfpathcurveto{\pgfqpoint{2.712943in}{2.181768in}}{\pgfqpoint{2.708993in}{2.180132in}}{\pgfqpoint{2.706081in}{2.177220in}}%
\pgfpathcurveto{\pgfqpoint{2.703169in}{2.174308in}}{\pgfqpoint{2.701533in}{2.170358in}}{\pgfqpoint{2.701533in}{2.166240in}}%
\pgfpathcurveto{\pgfqpoint{2.701533in}{2.162122in}}{\pgfqpoint{2.703169in}{2.158172in}}{\pgfqpoint{2.706081in}{2.155260in}}%
\pgfpathcurveto{\pgfqpoint{2.708993in}{2.152348in}}{\pgfqpoint{2.712943in}{2.150712in}}{\pgfqpoint{2.717061in}{2.150712in}}%
\pgfpathclose%
\pgfusepath{fill}%
\end{pgfscope}%
\begin{pgfscope}%
\pgfpathrectangle{\pgfqpoint{0.887500in}{0.275000in}}{\pgfqpoint{4.225000in}{4.225000in}}%
\pgfusepath{clip}%
\pgfsetbuttcap%
\pgfsetroundjoin%
\definecolor{currentfill}{rgb}{0.000000,0.000000,0.000000}%
\pgfsetfillcolor{currentfill}%
\pgfsetfillopacity{0.800000}%
\pgfsetlinewidth{0.000000pt}%
\definecolor{currentstroke}{rgb}{0.000000,0.000000,0.000000}%
\pgfsetstrokecolor{currentstroke}%
\pgfsetstrokeopacity{0.800000}%
\pgfsetdash{}{0pt}%
\pgfpathmoveto{\pgfqpoint{1.596415in}{2.369014in}}%
\pgfpathcurveto{\pgfqpoint{1.600533in}{2.369014in}}{\pgfqpoint{1.604483in}{2.370650in}}{\pgfqpoint{1.607395in}{2.373562in}}%
\pgfpathcurveto{\pgfqpoint{1.610307in}{2.376474in}}{\pgfqpoint{1.611943in}{2.380424in}}{\pgfqpoint{1.611943in}{2.384542in}}%
\pgfpathcurveto{\pgfqpoint{1.611943in}{2.388660in}}{\pgfqpoint{1.610307in}{2.392610in}}{\pgfqpoint{1.607395in}{2.395522in}}%
\pgfpathcurveto{\pgfqpoint{1.604483in}{2.398434in}}{\pgfqpoint{1.600533in}{2.400070in}}{\pgfqpoint{1.596415in}{2.400070in}}%
\pgfpathcurveto{\pgfqpoint{1.592297in}{2.400070in}}{\pgfqpoint{1.588347in}{2.398434in}}{\pgfqpoint{1.585435in}{2.395522in}}%
\pgfpathcurveto{\pgfqpoint{1.582523in}{2.392610in}}{\pgfqpoint{1.580887in}{2.388660in}}{\pgfqpoint{1.580887in}{2.384542in}}%
\pgfpathcurveto{\pgfqpoint{1.580887in}{2.380424in}}{\pgfqpoint{1.582523in}{2.376474in}}{\pgfqpoint{1.585435in}{2.373562in}}%
\pgfpathcurveto{\pgfqpoint{1.588347in}{2.370650in}}{\pgfqpoint{1.592297in}{2.369014in}}{\pgfqpoint{1.596415in}{2.369014in}}%
\pgfpathclose%
\pgfusepath{fill}%
\end{pgfscope}%
\begin{pgfscope}%
\pgfpathrectangle{\pgfqpoint{0.887500in}{0.275000in}}{\pgfqpoint{4.225000in}{4.225000in}}%
\pgfusepath{clip}%
\pgfsetbuttcap%
\pgfsetroundjoin%
\definecolor{currentfill}{rgb}{0.000000,0.000000,0.000000}%
\pgfsetfillcolor{currentfill}%
\pgfsetfillopacity{0.800000}%
\pgfsetlinewidth{0.000000pt}%
\definecolor{currentstroke}{rgb}{0.000000,0.000000,0.000000}%
\pgfsetstrokecolor{currentstroke}%
\pgfsetstrokeopacity{0.800000}%
\pgfsetdash{}{0pt}%
\pgfpathmoveto{\pgfqpoint{3.999009in}{2.425611in}}%
\pgfpathcurveto{\pgfqpoint{4.003127in}{2.425611in}}{\pgfqpoint{4.007077in}{2.427247in}}{\pgfqpoint{4.009989in}{2.430159in}}%
\pgfpathcurveto{\pgfqpoint{4.012901in}{2.433071in}}{\pgfqpoint{4.014537in}{2.437021in}}{\pgfqpoint{4.014537in}{2.441139in}}%
\pgfpathcurveto{\pgfqpoint{4.014537in}{2.445257in}}{\pgfqpoint{4.012901in}{2.449207in}}{\pgfqpoint{4.009989in}{2.452119in}}%
\pgfpathcurveto{\pgfqpoint{4.007077in}{2.455031in}}{\pgfqpoint{4.003127in}{2.456667in}}{\pgfqpoint{3.999009in}{2.456667in}}%
\pgfpathcurveto{\pgfqpoint{3.994891in}{2.456667in}}{\pgfqpoint{3.990941in}{2.455031in}}{\pgfqpoint{3.988029in}{2.452119in}}%
\pgfpathcurveto{\pgfqpoint{3.985117in}{2.449207in}}{\pgfqpoint{3.983480in}{2.445257in}}{\pgfqpoint{3.983480in}{2.441139in}}%
\pgfpathcurveto{\pgfqpoint{3.983480in}{2.437021in}}{\pgfqpoint{3.985117in}{2.433071in}}{\pgfqpoint{3.988029in}{2.430159in}}%
\pgfpathcurveto{\pgfqpoint{3.990941in}{2.427247in}}{\pgfqpoint{3.994891in}{2.425611in}}{\pgfqpoint{3.999009in}{2.425611in}}%
\pgfpathclose%
\pgfusepath{fill}%
\end{pgfscope}%
\begin{pgfscope}%
\pgfpathrectangle{\pgfqpoint{0.887500in}{0.275000in}}{\pgfqpoint{4.225000in}{4.225000in}}%
\pgfusepath{clip}%
\pgfsetbuttcap%
\pgfsetroundjoin%
\definecolor{currentfill}{rgb}{0.000000,0.000000,0.000000}%
\pgfsetfillcolor{currentfill}%
\pgfsetfillopacity{0.800000}%
\pgfsetlinewidth{0.000000pt}%
\definecolor{currentstroke}{rgb}{0.000000,0.000000,0.000000}%
\pgfsetstrokecolor{currentstroke}%
\pgfsetstrokeopacity{0.800000}%
\pgfsetdash{}{0pt}%
\pgfpathmoveto{\pgfqpoint{4.601474in}{1.927036in}}%
\pgfpathcurveto{\pgfqpoint{4.605592in}{1.927036in}}{\pgfqpoint{4.609542in}{1.928672in}}{\pgfqpoint{4.612454in}{1.931584in}}%
\pgfpathcurveto{\pgfqpoint{4.615366in}{1.934496in}}{\pgfqpoint{4.617002in}{1.938446in}}{\pgfqpoint{4.617002in}{1.942564in}}%
\pgfpathcurveto{\pgfqpoint{4.617002in}{1.946682in}}{\pgfqpoint{4.615366in}{1.950632in}}{\pgfqpoint{4.612454in}{1.953544in}}%
\pgfpathcurveto{\pgfqpoint{4.609542in}{1.956456in}}{\pgfqpoint{4.605592in}{1.958092in}}{\pgfqpoint{4.601474in}{1.958092in}}%
\pgfpathcurveto{\pgfqpoint{4.597356in}{1.958092in}}{\pgfqpoint{4.593406in}{1.956456in}}{\pgfqpoint{4.590494in}{1.953544in}}%
\pgfpathcurveto{\pgfqpoint{4.587582in}{1.950632in}}{\pgfqpoint{4.585946in}{1.946682in}}{\pgfqpoint{4.585946in}{1.942564in}}%
\pgfpathcurveto{\pgfqpoint{4.585946in}{1.938446in}}{\pgfqpoint{4.587582in}{1.934496in}}{\pgfqpoint{4.590494in}{1.931584in}}%
\pgfpathcurveto{\pgfqpoint{4.593406in}{1.928672in}}{\pgfqpoint{4.597356in}{1.927036in}}{\pgfqpoint{4.601474in}{1.927036in}}%
\pgfpathclose%
\pgfusepath{fill}%
\end{pgfscope}%
\begin{pgfscope}%
\pgfpathrectangle{\pgfqpoint{0.887500in}{0.275000in}}{\pgfqpoint{4.225000in}{4.225000in}}%
\pgfusepath{clip}%
\pgfsetbuttcap%
\pgfsetroundjoin%
\definecolor{currentfill}{rgb}{0.000000,0.000000,0.000000}%
\pgfsetfillcolor{currentfill}%
\pgfsetfillopacity{0.800000}%
\pgfsetlinewidth{0.000000pt}%
\definecolor{currentstroke}{rgb}{0.000000,0.000000,0.000000}%
\pgfsetstrokecolor{currentstroke}%
\pgfsetstrokeopacity{0.800000}%
\pgfsetdash{}{0pt}%
\pgfpathmoveto{\pgfqpoint{3.170501in}{2.695958in}}%
\pgfpathcurveto{\pgfqpoint{3.174619in}{2.695958in}}{\pgfqpoint{3.178569in}{2.697594in}}{\pgfqpoint{3.181481in}{2.700506in}}%
\pgfpathcurveto{\pgfqpoint{3.184393in}{2.703418in}}{\pgfqpoint{3.186029in}{2.707368in}}{\pgfqpoint{3.186029in}{2.711486in}}%
\pgfpathcurveto{\pgfqpoint{3.186029in}{2.715604in}}{\pgfqpoint{3.184393in}{2.719554in}}{\pgfqpoint{3.181481in}{2.722466in}}%
\pgfpathcurveto{\pgfqpoint{3.178569in}{2.725378in}}{\pgfqpoint{3.174619in}{2.727014in}}{\pgfqpoint{3.170501in}{2.727014in}}%
\pgfpathcurveto{\pgfqpoint{3.166383in}{2.727014in}}{\pgfqpoint{3.162433in}{2.725378in}}{\pgfqpoint{3.159521in}{2.722466in}}%
\pgfpathcurveto{\pgfqpoint{3.156609in}{2.719554in}}{\pgfqpoint{3.154973in}{2.715604in}}{\pgfqpoint{3.154973in}{2.711486in}}%
\pgfpathcurveto{\pgfqpoint{3.154973in}{2.707368in}}{\pgfqpoint{3.156609in}{2.703418in}}{\pgfqpoint{3.159521in}{2.700506in}}%
\pgfpathcurveto{\pgfqpoint{3.162433in}{2.697594in}}{\pgfqpoint{3.166383in}{2.695958in}}{\pgfqpoint{3.170501in}{2.695958in}}%
\pgfpathclose%
\pgfusepath{fill}%
\end{pgfscope}%
\begin{pgfscope}%
\pgfpathrectangle{\pgfqpoint{0.887500in}{0.275000in}}{\pgfqpoint{4.225000in}{4.225000in}}%
\pgfusepath{clip}%
\pgfsetbuttcap%
\pgfsetroundjoin%
\definecolor{currentfill}{rgb}{0.000000,0.000000,0.000000}%
\pgfsetfillcolor{currentfill}%
\pgfsetfillopacity{0.800000}%
\pgfsetlinewidth{0.000000pt}%
\definecolor{currentstroke}{rgb}{0.000000,0.000000,0.000000}%
\pgfsetstrokecolor{currentstroke}%
\pgfsetstrokeopacity{0.800000}%
\pgfsetdash{}{0pt}%
\pgfpathmoveto{\pgfqpoint{3.820552in}{2.539334in}}%
\pgfpathcurveto{\pgfqpoint{3.824671in}{2.539334in}}{\pgfqpoint{3.828621in}{2.540971in}}{\pgfqpoint{3.831533in}{2.543883in}}%
\pgfpathcurveto{\pgfqpoint{3.834445in}{2.546795in}}{\pgfqpoint{3.836081in}{2.550745in}}{\pgfqpoint{3.836081in}{2.554863in}}%
\pgfpathcurveto{\pgfqpoint{3.836081in}{2.558981in}}{\pgfqpoint{3.834445in}{2.562931in}}{\pgfqpoint{3.831533in}{2.565843in}}%
\pgfpathcurveto{\pgfqpoint{3.828621in}{2.568755in}}{\pgfqpoint{3.824671in}{2.570391in}}{\pgfqpoint{3.820552in}{2.570391in}}%
\pgfpathcurveto{\pgfqpoint{3.816434in}{2.570391in}}{\pgfqpoint{3.812484in}{2.568755in}}{\pgfqpoint{3.809572in}{2.565843in}}%
\pgfpathcurveto{\pgfqpoint{3.806660in}{2.562931in}}{\pgfqpoint{3.805024in}{2.558981in}}{\pgfqpoint{3.805024in}{2.554863in}}%
\pgfpathcurveto{\pgfqpoint{3.805024in}{2.550745in}}{\pgfqpoint{3.806660in}{2.546795in}}{\pgfqpoint{3.809572in}{2.543883in}}%
\pgfpathcurveto{\pgfqpoint{3.812484in}{2.540971in}}{\pgfqpoint{3.816434in}{2.539334in}}{\pgfqpoint{3.820552in}{2.539334in}}%
\pgfpathclose%
\pgfusepath{fill}%
\end{pgfscope}%
\begin{pgfscope}%
\pgfpathrectangle{\pgfqpoint{0.887500in}{0.275000in}}{\pgfqpoint{4.225000in}{4.225000in}}%
\pgfusepath{clip}%
\pgfsetbuttcap%
\pgfsetroundjoin%
\definecolor{currentfill}{rgb}{0.000000,0.000000,0.000000}%
\pgfsetfillcolor{currentfill}%
\pgfsetfillopacity{0.800000}%
\pgfsetlinewidth{0.000000pt}%
\definecolor{currentstroke}{rgb}{0.000000,0.000000,0.000000}%
\pgfsetstrokecolor{currentstroke}%
\pgfsetstrokeopacity{0.800000}%
\pgfsetdash{}{0pt}%
\pgfpathmoveto{\pgfqpoint{2.245655in}{2.235391in}}%
\pgfpathcurveto{\pgfqpoint{2.249773in}{2.235391in}}{\pgfqpoint{2.253723in}{2.237027in}}{\pgfqpoint{2.256635in}{2.239939in}}%
\pgfpathcurveto{\pgfqpoint{2.259547in}{2.242851in}}{\pgfqpoint{2.261183in}{2.246801in}}{\pgfqpoint{2.261183in}{2.250919in}}%
\pgfpathcurveto{\pgfqpoint{2.261183in}{2.255038in}}{\pgfqpoint{2.259547in}{2.258988in}}{\pgfqpoint{2.256635in}{2.261899in}}%
\pgfpathcurveto{\pgfqpoint{2.253723in}{2.264811in}}{\pgfqpoint{2.249773in}{2.266448in}}{\pgfqpoint{2.245655in}{2.266448in}}%
\pgfpathcurveto{\pgfqpoint{2.241537in}{2.266448in}}{\pgfqpoint{2.237587in}{2.264811in}}{\pgfqpoint{2.234675in}{2.261899in}}%
\pgfpathcurveto{\pgfqpoint{2.231763in}{2.258988in}}{\pgfqpoint{2.230127in}{2.255038in}}{\pgfqpoint{2.230127in}{2.250919in}}%
\pgfpathcurveto{\pgfqpoint{2.230127in}{2.246801in}}{\pgfqpoint{2.231763in}{2.242851in}}{\pgfqpoint{2.234675in}{2.239939in}}%
\pgfpathcurveto{\pgfqpoint{2.237587in}{2.237027in}}{\pgfqpoint{2.241537in}{2.235391in}}{\pgfqpoint{2.245655in}{2.235391in}}%
\pgfpathclose%
\pgfusepath{fill}%
\end{pgfscope}%
\begin{pgfscope}%
\pgfpathrectangle{\pgfqpoint{0.887500in}{0.275000in}}{\pgfqpoint{4.225000in}{4.225000in}}%
\pgfusepath{clip}%
\pgfsetbuttcap%
\pgfsetroundjoin%
\definecolor{currentfill}{rgb}{0.000000,0.000000,0.000000}%
\pgfsetfillcolor{currentfill}%
\pgfsetfillopacity{0.800000}%
\pgfsetlinewidth{0.000000pt}%
\definecolor{currentstroke}{rgb}{0.000000,0.000000,0.000000}%
\pgfsetstrokecolor{currentstroke}%
\pgfsetstrokeopacity{0.800000}%
\pgfsetdash{}{0pt}%
\pgfpathmoveto{\pgfqpoint{2.895714in}{2.086801in}}%
\pgfpathcurveto{\pgfqpoint{2.899832in}{2.086801in}}{\pgfqpoint{2.903782in}{2.088437in}}{\pgfqpoint{2.906694in}{2.091349in}}%
\pgfpathcurveto{\pgfqpoint{2.909606in}{2.094261in}}{\pgfqpoint{2.911242in}{2.098211in}}{\pgfqpoint{2.911242in}{2.102329in}}%
\pgfpathcurveto{\pgfqpoint{2.911242in}{2.106447in}}{\pgfqpoint{2.909606in}{2.110397in}}{\pgfqpoint{2.906694in}{2.113309in}}%
\pgfpathcurveto{\pgfqpoint{2.903782in}{2.116221in}}{\pgfqpoint{2.899832in}{2.117857in}}{\pgfqpoint{2.895714in}{2.117857in}}%
\pgfpathcurveto{\pgfqpoint{2.891596in}{2.117857in}}{\pgfqpoint{2.887646in}{2.116221in}}{\pgfqpoint{2.884734in}{2.113309in}}%
\pgfpathcurveto{\pgfqpoint{2.881822in}{2.110397in}}{\pgfqpoint{2.880186in}{2.106447in}}{\pgfqpoint{2.880186in}{2.102329in}}%
\pgfpathcurveto{\pgfqpoint{2.880186in}{2.098211in}}{\pgfqpoint{2.881822in}{2.094261in}}{\pgfqpoint{2.884734in}{2.091349in}}%
\pgfpathcurveto{\pgfqpoint{2.887646in}{2.088437in}}{\pgfqpoint{2.891596in}{2.086801in}}{\pgfqpoint{2.895714in}{2.086801in}}%
\pgfpathclose%
\pgfusepath{fill}%
\end{pgfscope}%
\begin{pgfscope}%
\pgfpathrectangle{\pgfqpoint{0.887500in}{0.275000in}}{\pgfqpoint{4.225000in}{4.225000in}}%
\pgfusepath{clip}%
\pgfsetbuttcap%
\pgfsetroundjoin%
\definecolor{currentfill}{rgb}{0.000000,0.000000,0.000000}%
\pgfsetfillcolor{currentfill}%
\pgfsetfillopacity{0.800000}%
\pgfsetlinewidth{0.000000pt}%
\definecolor{currentstroke}{rgb}{0.000000,0.000000,0.000000}%
\pgfsetstrokecolor{currentstroke}%
\pgfsetstrokeopacity{0.800000}%
\pgfsetdash{}{0pt}%
\pgfpathmoveto{\pgfqpoint{3.283992in}{2.841739in}}%
\pgfpathcurveto{\pgfqpoint{3.288110in}{2.841739in}}{\pgfqpoint{3.292060in}{2.843375in}}{\pgfqpoint{3.294972in}{2.846287in}}%
\pgfpathcurveto{\pgfqpoint{3.297884in}{2.849199in}}{\pgfqpoint{3.299520in}{2.853149in}}{\pgfqpoint{3.299520in}{2.857267in}}%
\pgfpathcurveto{\pgfqpoint{3.299520in}{2.861385in}}{\pgfqpoint{3.297884in}{2.865335in}}{\pgfqpoint{3.294972in}{2.868247in}}%
\pgfpathcurveto{\pgfqpoint{3.292060in}{2.871159in}}{\pgfqpoint{3.288110in}{2.872795in}}{\pgfqpoint{3.283992in}{2.872795in}}%
\pgfpathcurveto{\pgfqpoint{3.279873in}{2.872795in}}{\pgfqpoint{3.275923in}{2.871159in}}{\pgfqpoint{3.273011in}{2.868247in}}%
\pgfpathcurveto{\pgfqpoint{3.270099in}{2.865335in}}{\pgfqpoint{3.268463in}{2.861385in}}{\pgfqpoint{3.268463in}{2.857267in}}%
\pgfpathcurveto{\pgfqpoint{3.268463in}{2.853149in}}{\pgfqpoint{3.270099in}{2.849199in}}{\pgfqpoint{3.273011in}{2.846287in}}%
\pgfpathcurveto{\pgfqpoint{3.275923in}{2.843375in}}{\pgfqpoint{3.279873in}{2.841739in}}{\pgfqpoint{3.283992in}{2.841739in}}%
\pgfpathclose%
\pgfusepath{fill}%
\end{pgfscope}%
\begin{pgfscope}%
\pgfpathrectangle{\pgfqpoint{0.887500in}{0.275000in}}{\pgfqpoint{4.225000in}{4.225000in}}%
\pgfusepath{clip}%
\pgfsetbuttcap%
\pgfsetroundjoin%
\definecolor{currentfill}{rgb}{0.000000,0.000000,0.000000}%
\pgfsetfillcolor{currentfill}%
\pgfsetfillopacity{0.800000}%
\pgfsetlinewidth{0.000000pt}%
\definecolor{currentstroke}{rgb}{0.000000,0.000000,0.000000}%
\pgfsetstrokecolor{currentstroke}%
\pgfsetstrokeopacity{0.800000}%
\pgfsetdash{}{0pt}%
\pgfpathmoveto{\pgfqpoint{3.641890in}{2.650122in}}%
\pgfpathcurveto{\pgfqpoint{3.646008in}{2.650122in}}{\pgfqpoint{3.649958in}{2.651758in}}{\pgfqpoint{3.652870in}{2.654670in}}%
\pgfpathcurveto{\pgfqpoint{3.655782in}{2.657582in}}{\pgfqpoint{3.657418in}{2.661532in}}{\pgfqpoint{3.657418in}{2.665650in}}%
\pgfpathcurveto{\pgfqpoint{3.657418in}{2.669768in}}{\pgfqpoint{3.655782in}{2.673718in}}{\pgfqpoint{3.652870in}{2.676630in}}%
\pgfpathcurveto{\pgfqpoint{3.649958in}{2.679542in}}{\pgfqpoint{3.646008in}{2.681178in}}{\pgfqpoint{3.641890in}{2.681178in}}%
\pgfpathcurveto{\pgfqpoint{3.637772in}{2.681178in}}{\pgfqpoint{3.633822in}{2.679542in}}{\pgfqpoint{3.630910in}{2.676630in}}%
\pgfpathcurveto{\pgfqpoint{3.627998in}{2.673718in}}{\pgfqpoint{3.626362in}{2.669768in}}{\pgfqpoint{3.626362in}{2.665650in}}%
\pgfpathcurveto{\pgfqpoint{3.626362in}{2.661532in}}{\pgfqpoint{3.627998in}{2.657582in}}{\pgfqpoint{3.630910in}{2.654670in}}%
\pgfpathcurveto{\pgfqpoint{3.633822in}{2.651758in}}{\pgfqpoint{3.637772in}{2.650122in}}{\pgfqpoint{3.641890in}{2.650122in}}%
\pgfpathclose%
\pgfusepath{fill}%
\end{pgfscope}%
\begin{pgfscope}%
\pgfpathrectangle{\pgfqpoint{0.887500in}{0.275000in}}{\pgfqpoint{4.225000in}{4.225000in}}%
\pgfusepath{clip}%
\pgfsetbuttcap%
\pgfsetroundjoin%
\definecolor{currentfill}{rgb}{0.000000,0.000000,0.000000}%
\pgfsetfillcolor{currentfill}%
\pgfsetfillopacity{0.800000}%
\pgfsetlinewidth{0.000000pt}%
\definecolor{currentstroke}{rgb}{0.000000,0.000000,0.000000}%
\pgfsetstrokecolor{currentstroke}%
\pgfsetstrokeopacity{0.800000}%
\pgfsetdash{}{0pt}%
\pgfpathmoveto{\pgfqpoint{4.423379in}{2.045197in}}%
\pgfpathcurveto{\pgfqpoint{4.427498in}{2.045197in}}{\pgfqpoint{4.431448in}{2.046833in}}{\pgfqpoint{4.434360in}{2.049745in}}%
\pgfpathcurveto{\pgfqpoint{4.437271in}{2.052657in}}{\pgfqpoint{4.438908in}{2.056607in}}{\pgfqpoint{4.438908in}{2.060725in}}%
\pgfpathcurveto{\pgfqpoint{4.438908in}{2.064843in}}{\pgfqpoint{4.437271in}{2.068793in}}{\pgfqpoint{4.434360in}{2.071705in}}%
\pgfpathcurveto{\pgfqpoint{4.431448in}{2.074617in}}{\pgfqpoint{4.427498in}{2.076253in}}{\pgfqpoint{4.423379in}{2.076253in}}%
\pgfpathcurveto{\pgfqpoint{4.419261in}{2.076253in}}{\pgfqpoint{4.415311in}{2.074617in}}{\pgfqpoint{4.412399in}{2.071705in}}%
\pgfpathcurveto{\pgfqpoint{4.409487in}{2.068793in}}{\pgfqpoint{4.407851in}{2.064843in}}{\pgfqpoint{4.407851in}{2.060725in}}%
\pgfpathcurveto{\pgfqpoint{4.407851in}{2.056607in}}{\pgfqpoint{4.409487in}{2.052657in}}{\pgfqpoint{4.412399in}{2.049745in}}%
\pgfpathcurveto{\pgfqpoint{4.415311in}{2.046833in}}{\pgfqpoint{4.419261in}{2.045197in}}{\pgfqpoint{4.423379in}{2.045197in}}%
\pgfpathclose%
\pgfusepath{fill}%
\end{pgfscope}%
\begin{pgfscope}%
\pgfpathrectangle{\pgfqpoint{0.887500in}{0.275000in}}{\pgfqpoint{4.225000in}{4.225000in}}%
\pgfusepath{clip}%
\pgfsetbuttcap%
\pgfsetroundjoin%
\definecolor{currentfill}{rgb}{0.000000,0.000000,0.000000}%
\pgfsetfillcolor{currentfill}%
\pgfsetfillopacity{0.800000}%
\pgfsetlinewidth{0.000000pt}%
\definecolor{currentstroke}{rgb}{0.000000,0.000000,0.000000}%
\pgfsetstrokecolor{currentstroke}%
\pgfsetstrokeopacity{0.800000}%
\pgfsetdash{}{0pt}%
\pgfpathmoveto{\pgfqpoint{1.773857in}{2.314576in}}%
\pgfpathcurveto{\pgfqpoint{1.777975in}{2.314576in}}{\pgfqpoint{1.781925in}{2.316212in}}{\pgfqpoint{1.784837in}{2.319124in}}%
\pgfpathcurveto{\pgfqpoint{1.787749in}{2.322036in}}{\pgfqpoint{1.789385in}{2.325986in}}{\pgfqpoint{1.789385in}{2.330105in}}%
\pgfpathcurveto{\pgfqpoint{1.789385in}{2.334223in}}{\pgfqpoint{1.787749in}{2.338173in}}{\pgfqpoint{1.784837in}{2.341085in}}%
\pgfpathcurveto{\pgfqpoint{1.781925in}{2.343997in}}{\pgfqpoint{1.777975in}{2.345633in}}{\pgfqpoint{1.773857in}{2.345633in}}%
\pgfpathcurveto{\pgfqpoint{1.769739in}{2.345633in}}{\pgfqpoint{1.765789in}{2.343997in}}{\pgfqpoint{1.762877in}{2.341085in}}%
\pgfpathcurveto{\pgfqpoint{1.759965in}{2.338173in}}{\pgfqpoint{1.758329in}{2.334223in}}{\pgfqpoint{1.758329in}{2.330105in}}%
\pgfpathcurveto{\pgfqpoint{1.758329in}{2.325986in}}{\pgfqpoint{1.759965in}{2.322036in}}{\pgfqpoint{1.762877in}{2.319124in}}%
\pgfpathcurveto{\pgfqpoint{1.765789in}{2.316212in}}{\pgfqpoint{1.769739in}{2.314576in}}{\pgfqpoint{1.773857in}{2.314576in}}%
\pgfpathclose%
\pgfusepath{fill}%
\end{pgfscope}%
\begin{pgfscope}%
\pgfpathrectangle{\pgfqpoint{0.887500in}{0.275000in}}{\pgfqpoint{4.225000in}{4.225000in}}%
\pgfusepath{clip}%
\pgfsetbuttcap%
\pgfsetroundjoin%
\definecolor{currentfill}{rgb}{0.000000,0.000000,0.000000}%
\pgfsetfillcolor{currentfill}%
\pgfsetfillopacity{0.800000}%
\pgfsetlinewidth{0.000000pt}%
\definecolor{currentstroke}{rgb}{0.000000,0.000000,0.000000}%
\pgfsetstrokecolor{currentstroke}%
\pgfsetstrokeopacity{0.800000}%
\pgfsetdash{}{0pt}%
\pgfpathmoveto{\pgfqpoint{3.463034in}{2.755071in}}%
\pgfpathcurveto{\pgfqpoint{3.467152in}{2.755071in}}{\pgfqpoint{3.471102in}{2.756707in}}{\pgfqpoint{3.474014in}{2.759619in}}%
\pgfpathcurveto{\pgfqpoint{3.476926in}{2.762531in}}{\pgfqpoint{3.478562in}{2.766481in}}{\pgfqpoint{3.478562in}{2.770599in}}%
\pgfpathcurveto{\pgfqpoint{3.478562in}{2.774717in}}{\pgfqpoint{3.476926in}{2.778667in}}{\pgfqpoint{3.474014in}{2.781579in}}%
\pgfpathcurveto{\pgfqpoint{3.471102in}{2.784491in}}{\pgfqpoint{3.467152in}{2.786127in}}{\pgfqpoint{3.463034in}{2.786127in}}%
\pgfpathcurveto{\pgfqpoint{3.458915in}{2.786127in}}{\pgfqpoint{3.454965in}{2.784491in}}{\pgfqpoint{3.452053in}{2.781579in}}%
\pgfpathcurveto{\pgfqpoint{3.449141in}{2.778667in}}{\pgfqpoint{3.447505in}{2.774717in}}{\pgfqpoint{3.447505in}{2.770599in}}%
\pgfpathcurveto{\pgfqpoint{3.447505in}{2.766481in}}{\pgfqpoint{3.449141in}{2.762531in}}{\pgfqpoint{3.452053in}{2.759619in}}%
\pgfpathcurveto{\pgfqpoint{3.454965in}{2.756707in}}{\pgfqpoint{3.458915in}{2.755071in}}{\pgfqpoint{3.463034in}{2.755071in}}%
\pgfpathclose%
\pgfusepath{fill}%
\end{pgfscope}%
\begin{pgfscope}%
\pgfpathrectangle{\pgfqpoint{0.887500in}{0.275000in}}{\pgfqpoint{4.225000in}{4.225000in}}%
\pgfusepath{clip}%
\pgfsetbuttcap%
\pgfsetroundjoin%
\definecolor{currentfill}{rgb}{0.000000,0.000000,0.000000}%
\pgfsetfillcolor{currentfill}%
\pgfsetfillopacity{0.800000}%
\pgfsetlinewidth{0.000000pt}%
\definecolor{currentstroke}{rgb}{0.000000,0.000000,0.000000}%
\pgfsetstrokecolor{currentstroke}%
\pgfsetstrokeopacity{0.800000}%
\pgfsetdash{}{0pt}%
\pgfpathmoveto{\pgfqpoint{3.122611in}{2.363089in}}%
\pgfpathcurveto{\pgfqpoint{3.126729in}{2.363089in}}{\pgfqpoint{3.130679in}{2.364725in}}{\pgfqpoint{3.133591in}{2.367637in}}%
\pgfpathcurveto{\pgfqpoint{3.136503in}{2.370549in}}{\pgfqpoint{3.138139in}{2.374499in}}{\pgfqpoint{3.138139in}{2.378617in}}%
\pgfpathcurveto{\pgfqpoint{3.138139in}{2.382735in}}{\pgfqpoint{3.136503in}{2.386685in}}{\pgfqpoint{3.133591in}{2.389597in}}%
\pgfpathcurveto{\pgfqpoint{3.130679in}{2.392509in}}{\pgfqpoint{3.126729in}{2.394145in}}{\pgfqpoint{3.122611in}{2.394145in}}%
\pgfpathcurveto{\pgfqpoint{3.118493in}{2.394145in}}{\pgfqpoint{3.114543in}{2.392509in}}{\pgfqpoint{3.111631in}{2.389597in}}%
\pgfpathcurveto{\pgfqpoint{3.108719in}{2.386685in}}{\pgfqpoint{3.107083in}{2.382735in}}{\pgfqpoint{3.107083in}{2.378617in}}%
\pgfpathcurveto{\pgfqpoint{3.107083in}{2.374499in}}{\pgfqpoint{3.108719in}{2.370549in}}{\pgfqpoint{3.111631in}{2.367637in}}%
\pgfpathcurveto{\pgfqpoint{3.114543in}{2.364725in}}{\pgfqpoint{3.118493in}{2.363089in}}{\pgfqpoint{3.122611in}{2.363089in}}%
\pgfpathclose%
\pgfusepath{fill}%
\end{pgfscope}%
\begin{pgfscope}%
\pgfpathrectangle{\pgfqpoint{0.887500in}{0.275000in}}{\pgfqpoint{4.225000in}{4.225000in}}%
\pgfusepath{clip}%
\pgfsetbuttcap%
\pgfsetroundjoin%
\definecolor{currentfill}{rgb}{0.000000,0.000000,0.000000}%
\pgfsetfillcolor{currentfill}%
\pgfsetfillopacity{0.800000}%
\pgfsetlinewidth{0.000000pt}%
\definecolor{currentstroke}{rgb}{0.000000,0.000000,0.000000}%
\pgfsetstrokecolor{currentstroke}%
\pgfsetstrokeopacity{0.800000}%
\pgfsetdash{}{0pt}%
\pgfpathmoveto{\pgfqpoint{3.074656in}{2.019366in}}%
\pgfpathcurveto{\pgfqpoint{3.078774in}{2.019366in}}{\pgfqpoint{3.082724in}{2.021002in}}{\pgfqpoint{3.085636in}{2.023914in}}%
\pgfpathcurveto{\pgfqpoint{3.088548in}{2.026826in}}{\pgfqpoint{3.090184in}{2.030776in}}{\pgfqpoint{3.090184in}{2.034894in}}%
\pgfpathcurveto{\pgfqpoint{3.090184in}{2.039012in}}{\pgfqpoint{3.088548in}{2.042962in}}{\pgfqpoint{3.085636in}{2.045874in}}%
\pgfpathcurveto{\pgfqpoint{3.082724in}{2.048786in}}{\pgfqpoint{3.078774in}{2.050422in}}{\pgfqpoint{3.074656in}{2.050422in}}%
\pgfpathcurveto{\pgfqpoint{3.070538in}{2.050422in}}{\pgfqpoint{3.066588in}{2.048786in}}{\pgfqpoint{3.063676in}{2.045874in}}%
\pgfpathcurveto{\pgfqpoint{3.060764in}{2.042962in}}{\pgfqpoint{3.059128in}{2.039012in}}{\pgfqpoint{3.059128in}{2.034894in}}%
\pgfpathcurveto{\pgfqpoint{3.059128in}{2.030776in}}{\pgfqpoint{3.060764in}{2.026826in}}{\pgfqpoint{3.063676in}{2.023914in}}%
\pgfpathcurveto{\pgfqpoint{3.066588in}{2.021002in}}{\pgfqpoint{3.070538in}{2.019366in}}{\pgfqpoint{3.074656in}{2.019366in}}%
\pgfpathclose%
\pgfusepath{fill}%
\end{pgfscope}%
\begin{pgfscope}%
\pgfpathrectangle{\pgfqpoint{0.887500in}{0.275000in}}{\pgfqpoint{4.225000in}{4.225000in}}%
\pgfusepath{clip}%
\pgfsetbuttcap%
\pgfsetroundjoin%
\definecolor{currentfill}{rgb}{0.000000,0.000000,0.000000}%
\pgfsetfillcolor{currentfill}%
\pgfsetfillopacity{0.800000}%
\pgfsetlinewidth{0.000000pt}%
\definecolor{currentstroke}{rgb}{0.000000,0.000000,0.000000}%
\pgfsetstrokecolor{currentstroke}%
\pgfsetstrokeopacity{0.800000}%
\pgfsetdash{}{0pt}%
\pgfpathmoveto{\pgfqpoint{2.423945in}{2.176918in}}%
\pgfpathcurveto{\pgfqpoint{2.428064in}{2.176918in}}{\pgfqpoint{2.432014in}{2.178554in}}{\pgfqpoint{2.434926in}{2.181466in}}%
\pgfpathcurveto{\pgfqpoint{2.437838in}{2.184378in}}{\pgfqpoint{2.439474in}{2.188328in}}{\pgfqpoint{2.439474in}{2.192446in}}%
\pgfpathcurveto{\pgfqpoint{2.439474in}{2.196564in}}{\pgfqpoint{2.437838in}{2.200514in}}{\pgfqpoint{2.434926in}{2.203426in}}%
\pgfpathcurveto{\pgfqpoint{2.432014in}{2.206338in}}{\pgfqpoint{2.428064in}{2.207974in}}{\pgfqpoint{2.423945in}{2.207974in}}%
\pgfpathcurveto{\pgfqpoint{2.419827in}{2.207974in}}{\pgfqpoint{2.415877in}{2.206338in}}{\pgfqpoint{2.412965in}{2.203426in}}%
\pgfpathcurveto{\pgfqpoint{2.410053in}{2.200514in}}{\pgfqpoint{2.408417in}{2.196564in}}{\pgfqpoint{2.408417in}{2.192446in}}%
\pgfpathcurveto{\pgfqpoint{2.408417in}{2.188328in}}{\pgfqpoint{2.410053in}{2.184378in}}{\pgfqpoint{2.412965in}{2.181466in}}%
\pgfpathcurveto{\pgfqpoint{2.415877in}{2.178554in}}{\pgfqpoint{2.419827in}{2.176918in}}{\pgfqpoint{2.423945in}{2.176918in}}%
\pgfpathclose%
\pgfusepath{fill}%
\end{pgfscope}%
\begin{pgfscope}%
\pgfpathrectangle{\pgfqpoint{0.887500in}{0.275000in}}{\pgfqpoint{4.225000in}{4.225000in}}%
\pgfusepath{clip}%
\pgfsetbuttcap%
\pgfsetroundjoin%
\definecolor{currentfill}{rgb}{0.000000,0.000000,0.000000}%
\pgfsetfillcolor{currentfill}%
\pgfsetfillopacity{0.800000}%
\pgfsetlinewidth{0.000000pt}%
\definecolor{currentstroke}{rgb}{0.000000,0.000000,0.000000}%
\pgfsetstrokecolor{currentstroke}%
\pgfsetstrokeopacity{0.800000}%
\pgfsetdash{}{0pt}%
\pgfpathmoveto{\pgfqpoint{4.066409in}{2.275207in}}%
\pgfpathcurveto{\pgfqpoint{4.070527in}{2.275207in}}{\pgfqpoint{4.074477in}{2.276843in}}{\pgfqpoint{4.077389in}{2.279755in}}%
\pgfpathcurveto{\pgfqpoint{4.080301in}{2.282667in}}{\pgfqpoint{4.081937in}{2.286617in}}{\pgfqpoint{4.081937in}{2.290735in}}%
\pgfpathcurveto{\pgfqpoint{4.081937in}{2.294853in}}{\pgfqpoint{4.080301in}{2.298803in}}{\pgfqpoint{4.077389in}{2.301715in}}%
\pgfpathcurveto{\pgfqpoint{4.074477in}{2.304627in}}{\pgfqpoint{4.070527in}{2.306263in}}{\pgfqpoint{4.066409in}{2.306263in}}%
\pgfpathcurveto{\pgfqpoint{4.062291in}{2.306263in}}{\pgfqpoint{4.058341in}{2.304627in}}{\pgfqpoint{4.055429in}{2.301715in}}%
\pgfpathcurveto{\pgfqpoint{4.052517in}{2.298803in}}{\pgfqpoint{4.050881in}{2.294853in}}{\pgfqpoint{4.050881in}{2.290735in}}%
\pgfpathcurveto{\pgfqpoint{4.050881in}{2.286617in}}{\pgfqpoint{4.052517in}{2.282667in}}{\pgfqpoint{4.055429in}{2.279755in}}%
\pgfpathcurveto{\pgfqpoint{4.058341in}{2.276843in}}{\pgfqpoint{4.062291in}{2.275207in}}{\pgfqpoint{4.066409in}{2.275207in}}%
\pgfpathclose%
\pgfusepath{fill}%
\end{pgfscope}%
\begin{pgfscope}%
\pgfpathrectangle{\pgfqpoint{0.887500in}{0.275000in}}{\pgfqpoint{4.225000in}{4.225000in}}%
\pgfusepath{clip}%
\pgfsetbuttcap%
\pgfsetroundjoin%
\definecolor{currentfill}{rgb}{0.000000,0.000000,0.000000}%
\pgfsetfillcolor{currentfill}%
\pgfsetfillopacity{0.800000}%
\pgfsetlinewidth{0.000000pt}%
\definecolor{currentstroke}{rgb}{0.000000,0.000000,0.000000}%
\pgfsetstrokecolor{currentstroke}%
\pgfsetstrokeopacity{0.800000}%
\pgfsetdash{}{0pt}%
\pgfpathmoveto{\pgfqpoint{1.951721in}{2.258303in}}%
\pgfpathcurveto{\pgfqpoint{1.955839in}{2.258303in}}{\pgfqpoint{1.959789in}{2.259939in}}{\pgfqpoint{1.962701in}{2.262851in}}%
\pgfpathcurveto{\pgfqpoint{1.965613in}{2.265763in}}{\pgfqpoint{1.967249in}{2.269713in}}{\pgfqpoint{1.967249in}{2.273831in}}%
\pgfpathcurveto{\pgfqpoint{1.967249in}{2.277949in}}{\pgfqpoint{1.965613in}{2.281899in}}{\pgfqpoint{1.962701in}{2.284811in}}%
\pgfpathcurveto{\pgfqpoint{1.959789in}{2.287723in}}{\pgfqpoint{1.955839in}{2.289359in}}{\pgfqpoint{1.951721in}{2.289359in}}%
\pgfpathcurveto{\pgfqpoint{1.947603in}{2.289359in}}{\pgfqpoint{1.943653in}{2.287723in}}{\pgfqpoint{1.940741in}{2.284811in}}%
\pgfpathcurveto{\pgfqpoint{1.937829in}{2.281899in}}{\pgfqpoint{1.936193in}{2.277949in}}{\pgfqpoint{1.936193in}{2.273831in}}%
\pgfpathcurveto{\pgfqpoint{1.936193in}{2.269713in}}{\pgfqpoint{1.937829in}{2.265763in}}{\pgfqpoint{1.940741in}{2.262851in}}%
\pgfpathcurveto{\pgfqpoint{1.943653in}{2.259939in}}{\pgfqpoint{1.947603in}{2.258303in}}{\pgfqpoint{1.951721in}{2.258303in}}%
\pgfpathclose%
\pgfusepath{fill}%
\end{pgfscope}%
\begin{pgfscope}%
\pgfpathrectangle{\pgfqpoint{0.887500in}{0.275000in}}{\pgfqpoint{4.225000in}{4.225000in}}%
\pgfusepath{clip}%
\pgfsetbuttcap%
\pgfsetroundjoin%
\definecolor{currentfill}{rgb}{0.000000,0.000000,0.000000}%
\pgfsetfillcolor{currentfill}%
\pgfsetfillopacity{0.800000}%
\pgfsetlinewidth{0.000000pt}%
\definecolor{currentstroke}{rgb}{0.000000,0.000000,0.000000}%
\pgfsetstrokecolor{currentstroke}%
\pgfsetstrokeopacity{0.800000}%
\pgfsetdash{}{0pt}%
\pgfpathmoveto{\pgfqpoint{4.669958in}{1.772882in}}%
\pgfpathcurveto{\pgfqpoint{4.674076in}{1.772882in}}{\pgfqpoint{4.678026in}{1.774518in}}{\pgfqpoint{4.680938in}{1.777430in}}%
\pgfpathcurveto{\pgfqpoint{4.683850in}{1.780342in}}{\pgfqpoint{4.685486in}{1.784292in}}{\pgfqpoint{4.685486in}{1.788411in}}%
\pgfpathcurveto{\pgfqpoint{4.685486in}{1.792529in}}{\pgfqpoint{4.683850in}{1.796479in}}{\pgfqpoint{4.680938in}{1.799391in}}%
\pgfpathcurveto{\pgfqpoint{4.678026in}{1.802303in}}{\pgfqpoint{4.674076in}{1.803939in}}{\pgfqpoint{4.669958in}{1.803939in}}%
\pgfpathcurveto{\pgfqpoint{4.665840in}{1.803939in}}{\pgfqpoint{4.661890in}{1.802303in}}{\pgfqpoint{4.658978in}{1.799391in}}%
\pgfpathcurveto{\pgfqpoint{4.656066in}{1.796479in}}{\pgfqpoint{4.654429in}{1.792529in}}{\pgfqpoint{4.654429in}{1.788411in}}%
\pgfpathcurveto{\pgfqpoint{4.654429in}{1.784292in}}{\pgfqpoint{4.656066in}{1.780342in}}{\pgfqpoint{4.658978in}{1.777430in}}%
\pgfpathcurveto{\pgfqpoint{4.661890in}{1.774518in}}{\pgfqpoint{4.665840in}{1.772882in}}{\pgfqpoint{4.669958in}{1.772882in}}%
\pgfpathclose%
\pgfusepath{fill}%
\end{pgfscope}%
\begin{pgfscope}%
\pgfpathrectangle{\pgfqpoint{0.887500in}{0.275000in}}{\pgfqpoint{4.225000in}{4.225000in}}%
\pgfusepath{clip}%
\pgfsetbuttcap%
\pgfsetroundjoin%
\definecolor{currentfill}{rgb}{0.000000,0.000000,0.000000}%
\pgfsetfillcolor{currentfill}%
\pgfsetfillopacity{0.800000}%
\pgfsetlinewidth{0.000000pt}%
\definecolor{currentstroke}{rgb}{0.000000,0.000000,0.000000}%
\pgfsetstrokecolor{currentstroke}%
\pgfsetstrokeopacity{0.800000}%
\pgfsetdash{}{0pt}%
\pgfpathmoveto{\pgfqpoint{2.602600in}{2.116274in}}%
\pgfpathcurveto{\pgfqpoint{2.606718in}{2.116274in}}{\pgfqpoint{2.610668in}{2.117910in}}{\pgfqpoint{2.613580in}{2.120822in}}%
\pgfpathcurveto{\pgfqpoint{2.616492in}{2.123734in}}{\pgfqpoint{2.618128in}{2.127684in}}{\pgfqpoint{2.618128in}{2.131803in}}%
\pgfpathcurveto{\pgfqpoint{2.618128in}{2.135921in}}{\pgfqpoint{2.616492in}{2.139871in}}{\pgfqpoint{2.613580in}{2.142783in}}%
\pgfpathcurveto{\pgfqpoint{2.610668in}{2.145695in}}{\pgfqpoint{2.606718in}{2.147331in}}{\pgfqpoint{2.602600in}{2.147331in}}%
\pgfpathcurveto{\pgfqpoint{2.598481in}{2.147331in}}{\pgfqpoint{2.594531in}{2.145695in}}{\pgfqpoint{2.591620in}{2.142783in}}%
\pgfpathcurveto{\pgfqpoint{2.588708in}{2.139871in}}{\pgfqpoint{2.587071in}{2.135921in}}{\pgfqpoint{2.587071in}{2.131803in}}%
\pgfpathcurveto{\pgfqpoint{2.587071in}{2.127684in}}{\pgfqpoint{2.588708in}{2.123734in}}{\pgfqpoint{2.591620in}{2.120822in}}%
\pgfpathcurveto{\pgfqpoint{2.594531in}{2.117910in}}{\pgfqpoint{2.598481in}{2.116274in}}{\pgfqpoint{2.602600in}{2.116274in}}%
\pgfpathclose%
\pgfusepath{fill}%
\end{pgfscope}%
\begin{pgfscope}%
\pgfpathrectangle{\pgfqpoint{0.887500in}{0.275000in}}{\pgfqpoint{4.225000in}{4.225000in}}%
\pgfusepath{clip}%
\pgfsetbuttcap%
\pgfsetroundjoin%
\definecolor{currentfill}{rgb}{0.000000,0.000000,0.000000}%
\pgfsetfillcolor{currentfill}%
\pgfsetfillopacity{0.800000}%
\pgfsetlinewidth{0.000000pt}%
\definecolor{currentstroke}{rgb}{0.000000,0.000000,0.000000}%
\pgfsetstrokecolor{currentstroke}%
\pgfsetstrokeopacity{0.800000}%
\pgfsetdash{}{0pt}%
\pgfpathmoveto{\pgfqpoint{3.887689in}{2.391588in}}%
\pgfpathcurveto{\pgfqpoint{3.891808in}{2.391588in}}{\pgfqpoint{3.895758in}{2.393224in}}{\pgfqpoint{3.898670in}{2.396136in}}%
\pgfpathcurveto{\pgfqpoint{3.901582in}{2.399048in}}{\pgfqpoint{3.903218in}{2.402998in}}{\pgfqpoint{3.903218in}{2.407117in}}%
\pgfpathcurveto{\pgfqpoint{3.903218in}{2.411235in}}{\pgfqpoint{3.901582in}{2.415185in}}{\pgfqpoint{3.898670in}{2.418097in}}%
\pgfpathcurveto{\pgfqpoint{3.895758in}{2.421009in}}{\pgfqpoint{3.891808in}{2.422645in}}{\pgfqpoint{3.887689in}{2.422645in}}%
\pgfpathcurveto{\pgfqpoint{3.883571in}{2.422645in}}{\pgfqpoint{3.879621in}{2.421009in}}{\pgfqpoint{3.876709in}{2.418097in}}%
\pgfpathcurveto{\pgfqpoint{3.873797in}{2.415185in}}{\pgfqpoint{3.872161in}{2.411235in}}{\pgfqpoint{3.872161in}{2.407117in}}%
\pgfpathcurveto{\pgfqpoint{3.872161in}{2.402998in}}{\pgfqpoint{3.873797in}{2.399048in}}{\pgfqpoint{3.876709in}{2.396136in}}%
\pgfpathcurveto{\pgfqpoint{3.879621in}{2.393224in}}{\pgfqpoint{3.883571in}{2.391588in}}{\pgfqpoint{3.887689in}{2.391588in}}%
\pgfpathclose%
\pgfusepath{fill}%
\end{pgfscope}%
\begin{pgfscope}%
\pgfpathrectangle{\pgfqpoint{0.887500in}{0.275000in}}{\pgfqpoint{4.225000in}{4.225000in}}%
\pgfusepath{clip}%
\pgfsetbuttcap%
\pgfsetroundjoin%
\definecolor{currentfill}{rgb}{0.000000,0.000000,0.000000}%
\pgfsetfillcolor{currentfill}%
\pgfsetfillopacity{0.800000}%
\pgfsetlinewidth{0.000000pt}%
\definecolor{currentstroke}{rgb}{0.000000,0.000000,0.000000}%
\pgfsetstrokecolor{currentstroke}%
\pgfsetstrokeopacity{0.800000}%
\pgfsetdash{}{0pt}%
\pgfpathmoveto{\pgfqpoint{3.708697in}{2.502716in}}%
\pgfpathcurveto{\pgfqpoint{3.712815in}{2.502716in}}{\pgfqpoint{3.716765in}{2.504352in}}{\pgfqpoint{3.719677in}{2.507264in}}%
\pgfpathcurveto{\pgfqpoint{3.722589in}{2.510176in}}{\pgfqpoint{3.724225in}{2.514126in}}{\pgfqpoint{3.724225in}{2.518244in}}%
\pgfpathcurveto{\pgfqpoint{3.724225in}{2.522362in}}{\pgfqpoint{3.722589in}{2.526312in}}{\pgfqpoint{3.719677in}{2.529224in}}%
\pgfpathcurveto{\pgfqpoint{3.716765in}{2.532136in}}{\pgfqpoint{3.712815in}{2.533772in}}{\pgfqpoint{3.708697in}{2.533772in}}%
\pgfpathcurveto{\pgfqpoint{3.704579in}{2.533772in}}{\pgfqpoint{3.700629in}{2.532136in}}{\pgfqpoint{3.697717in}{2.529224in}}%
\pgfpathcurveto{\pgfqpoint{3.694805in}{2.526312in}}{\pgfqpoint{3.693169in}{2.522362in}}{\pgfqpoint{3.693169in}{2.518244in}}%
\pgfpathcurveto{\pgfqpoint{3.693169in}{2.514126in}}{\pgfqpoint{3.694805in}{2.510176in}}{\pgfqpoint{3.697717in}{2.507264in}}%
\pgfpathcurveto{\pgfqpoint{3.700629in}{2.504352in}}{\pgfqpoint{3.704579in}{2.502716in}}{\pgfqpoint{3.708697in}{2.502716in}}%
\pgfpathclose%
\pgfusepath{fill}%
\end{pgfscope}%
\begin{pgfscope}%
\pgfpathrectangle{\pgfqpoint{0.887500in}{0.275000in}}{\pgfqpoint{4.225000in}{4.225000in}}%
\pgfusepath{clip}%
\pgfsetbuttcap%
\pgfsetroundjoin%
\definecolor{currentfill}{rgb}{0.000000,0.000000,0.000000}%
\pgfsetfillcolor{currentfill}%
\pgfsetfillopacity{0.800000}%
\pgfsetlinewidth{0.000000pt}%
\definecolor{currentstroke}{rgb}{0.000000,0.000000,0.000000}%
\pgfsetstrokecolor{currentstroke}%
\pgfsetstrokeopacity{0.800000}%
\pgfsetdash{}{0pt}%
\pgfpathmoveto{\pgfqpoint{4.491679in}{1.894164in}}%
\pgfpathcurveto{\pgfqpoint{4.495797in}{1.894164in}}{\pgfqpoint{4.499747in}{1.895800in}}{\pgfqpoint{4.502659in}{1.898712in}}%
\pgfpathcurveto{\pgfqpoint{4.505571in}{1.901624in}}{\pgfqpoint{4.507207in}{1.905574in}}{\pgfqpoint{4.507207in}{1.909692in}}%
\pgfpathcurveto{\pgfqpoint{4.507207in}{1.913810in}}{\pgfqpoint{4.505571in}{1.917760in}}{\pgfqpoint{4.502659in}{1.920672in}}%
\pgfpathcurveto{\pgfqpoint{4.499747in}{1.923584in}}{\pgfqpoint{4.495797in}{1.925220in}}{\pgfqpoint{4.491679in}{1.925220in}}%
\pgfpathcurveto{\pgfqpoint{4.487561in}{1.925220in}}{\pgfqpoint{4.483611in}{1.923584in}}{\pgfqpoint{4.480699in}{1.920672in}}%
\pgfpathcurveto{\pgfqpoint{4.477787in}{1.917760in}}{\pgfqpoint{4.476151in}{1.913810in}}{\pgfqpoint{4.476151in}{1.909692in}}%
\pgfpathcurveto{\pgfqpoint{4.476151in}{1.905574in}}{\pgfqpoint{4.477787in}{1.901624in}}{\pgfqpoint{4.480699in}{1.898712in}}%
\pgfpathcurveto{\pgfqpoint{4.483611in}{1.895800in}}{\pgfqpoint{4.487561in}{1.894164in}}{\pgfqpoint{4.491679in}{1.894164in}}%
\pgfpathclose%
\pgfusepath{fill}%
\end{pgfscope}%
\begin{pgfscope}%
\pgfpathrectangle{\pgfqpoint{0.887500in}{0.275000in}}{\pgfqpoint{4.225000in}{4.225000in}}%
\pgfusepath{clip}%
\pgfsetbuttcap%
\pgfsetroundjoin%
\definecolor{currentfill}{rgb}{0.000000,0.000000,0.000000}%
\pgfsetfillcolor{currentfill}%
\pgfsetfillopacity{0.800000}%
\pgfsetlinewidth{0.000000pt}%
\definecolor{currentstroke}{rgb}{0.000000,0.000000,0.000000}%
\pgfsetstrokecolor{currentstroke}%
\pgfsetstrokeopacity{0.800000}%
\pgfsetdash{}{0pt}%
\pgfpathmoveto{\pgfqpoint{2.129972in}{2.200711in}}%
\pgfpathcurveto{\pgfqpoint{2.134090in}{2.200711in}}{\pgfqpoint{2.138040in}{2.202348in}}{\pgfqpoint{2.140952in}{2.205260in}}%
\pgfpathcurveto{\pgfqpoint{2.143864in}{2.208172in}}{\pgfqpoint{2.145500in}{2.212122in}}{\pgfqpoint{2.145500in}{2.216240in}}%
\pgfpathcurveto{\pgfqpoint{2.145500in}{2.220358in}}{\pgfqpoint{2.143864in}{2.224308in}}{\pgfqpoint{2.140952in}{2.227220in}}%
\pgfpathcurveto{\pgfqpoint{2.138040in}{2.230132in}}{\pgfqpoint{2.134090in}{2.231768in}}{\pgfqpoint{2.129972in}{2.231768in}}%
\pgfpathcurveto{\pgfqpoint{2.125854in}{2.231768in}}{\pgfqpoint{2.121904in}{2.230132in}}{\pgfqpoint{2.118992in}{2.227220in}}%
\pgfpathcurveto{\pgfqpoint{2.116080in}{2.224308in}}{\pgfqpoint{2.114444in}{2.220358in}}{\pgfqpoint{2.114444in}{2.216240in}}%
\pgfpathcurveto{\pgfqpoint{2.114444in}{2.212122in}}{\pgfqpoint{2.116080in}{2.208172in}}{\pgfqpoint{2.118992in}{2.205260in}}%
\pgfpathcurveto{\pgfqpoint{2.121904in}{2.202348in}}{\pgfqpoint{2.125854in}{2.200711in}}{\pgfqpoint{2.129972in}{2.200711in}}%
\pgfpathclose%
\pgfusepath{fill}%
\end{pgfscope}%
\begin{pgfscope}%
\pgfpathrectangle{\pgfqpoint{0.887500in}{0.275000in}}{\pgfqpoint{4.225000in}{4.225000in}}%
\pgfusepath{clip}%
\pgfsetbuttcap%
\pgfsetroundjoin%
\definecolor{currentfill}{rgb}{0.000000,0.000000,0.000000}%
\pgfsetfillcolor{currentfill}%
\pgfsetfillopacity{0.800000}%
\pgfsetlinewidth{0.000000pt}%
\definecolor{currentstroke}{rgb}{0.000000,0.000000,0.000000}%
\pgfsetstrokecolor{currentstroke}%
\pgfsetstrokeopacity{0.800000}%
\pgfsetdash{}{0pt}%
\pgfpathmoveto{\pgfqpoint{3.350143in}{2.703430in}}%
\pgfpathcurveto{\pgfqpoint{3.354261in}{2.703430in}}{\pgfqpoint{3.358211in}{2.705066in}}{\pgfqpoint{3.361123in}{2.707978in}}%
\pgfpathcurveto{\pgfqpoint{3.364035in}{2.710890in}}{\pgfqpoint{3.365671in}{2.714840in}}{\pgfqpoint{3.365671in}{2.718958in}}%
\pgfpathcurveto{\pgfqpoint{3.365671in}{2.723076in}}{\pgfqpoint{3.364035in}{2.727026in}}{\pgfqpoint{3.361123in}{2.729938in}}%
\pgfpathcurveto{\pgfqpoint{3.358211in}{2.732850in}}{\pgfqpoint{3.354261in}{2.734486in}}{\pgfqpoint{3.350143in}{2.734486in}}%
\pgfpathcurveto{\pgfqpoint{3.346025in}{2.734486in}}{\pgfqpoint{3.342075in}{2.732850in}}{\pgfqpoint{3.339163in}{2.729938in}}%
\pgfpathcurveto{\pgfqpoint{3.336251in}{2.727026in}}{\pgfqpoint{3.334615in}{2.723076in}}{\pgfqpoint{3.334615in}{2.718958in}}%
\pgfpathcurveto{\pgfqpoint{3.334615in}{2.714840in}}{\pgfqpoint{3.336251in}{2.710890in}}{\pgfqpoint{3.339163in}{2.707978in}}%
\pgfpathcurveto{\pgfqpoint{3.342075in}{2.705066in}}{\pgfqpoint{3.346025in}{2.703430in}}{\pgfqpoint{3.350143in}{2.703430in}}%
\pgfpathclose%
\pgfusepath{fill}%
\end{pgfscope}%
\begin{pgfscope}%
\pgfpathrectangle{\pgfqpoint{0.887500in}{0.275000in}}{\pgfqpoint{4.225000in}{4.225000in}}%
\pgfusepath{clip}%
\pgfsetbuttcap%
\pgfsetroundjoin%
\definecolor{currentfill}{rgb}{0.000000,0.000000,0.000000}%
\pgfsetfillcolor{currentfill}%
\pgfsetfillopacity{0.800000}%
\pgfsetlinewidth{0.000000pt}%
\definecolor{currentstroke}{rgb}{0.000000,0.000000,0.000000}%
\pgfsetstrokecolor{currentstroke}%
\pgfsetstrokeopacity{0.800000}%
\pgfsetdash{}{0pt}%
\pgfpathmoveto{\pgfqpoint{3.529511in}{2.609146in}}%
\pgfpathcurveto{\pgfqpoint{3.533630in}{2.609146in}}{\pgfqpoint{3.537580in}{2.610782in}}{\pgfqpoint{3.540492in}{2.613694in}}%
\pgfpathcurveto{\pgfqpoint{3.543404in}{2.616606in}}{\pgfqpoint{3.545040in}{2.620556in}}{\pgfqpoint{3.545040in}{2.624674in}}%
\pgfpathcurveto{\pgfqpoint{3.545040in}{2.628792in}}{\pgfqpoint{3.543404in}{2.632742in}}{\pgfqpoint{3.540492in}{2.635654in}}%
\pgfpathcurveto{\pgfqpoint{3.537580in}{2.638566in}}{\pgfqpoint{3.533630in}{2.640203in}}{\pgfqpoint{3.529511in}{2.640203in}}%
\pgfpathcurveto{\pgfqpoint{3.525393in}{2.640203in}}{\pgfqpoint{3.521443in}{2.638566in}}{\pgfqpoint{3.518531in}{2.635654in}}%
\pgfpathcurveto{\pgfqpoint{3.515619in}{2.632742in}}{\pgfqpoint{3.513983in}{2.628792in}}{\pgfqpoint{3.513983in}{2.624674in}}%
\pgfpathcurveto{\pgfqpoint{3.513983in}{2.620556in}}{\pgfqpoint{3.515619in}{2.616606in}}{\pgfqpoint{3.518531in}{2.613694in}}%
\pgfpathcurveto{\pgfqpoint{3.521443in}{2.610782in}}{\pgfqpoint{3.525393in}{2.609146in}}{\pgfqpoint{3.529511in}{2.609146in}}%
\pgfpathclose%
\pgfusepath{fill}%
\end{pgfscope}%
\begin{pgfscope}%
\pgfpathrectangle{\pgfqpoint{0.887500in}{0.275000in}}{\pgfqpoint{4.225000in}{4.225000in}}%
\pgfusepath{clip}%
\pgfsetbuttcap%
\pgfsetroundjoin%
\definecolor{currentfill}{rgb}{0.000000,0.000000,0.000000}%
\pgfsetfillcolor{currentfill}%
\pgfsetfillopacity{0.800000}%
\pgfsetlinewidth{0.000000pt}%
\definecolor{currentstroke}{rgb}{0.000000,0.000000,0.000000}%
\pgfsetstrokecolor{currentstroke}%
\pgfsetstrokeopacity{0.800000}%
\pgfsetdash{}{0pt}%
\pgfpathmoveto{\pgfqpoint{2.781597in}{2.052521in}}%
\pgfpathcurveto{\pgfqpoint{2.785715in}{2.052521in}}{\pgfqpoint{2.789665in}{2.054157in}}{\pgfqpoint{2.792577in}{2.057069in}}%
\pgfpathcurveto{\pgfqpoint{2.795489in}{2.059981in}}{\pgfqpoint{2.797125in}{2.063931in}}{\pgfqpoint{2.797125in}{2.068049in}}%
\pgfpathcurveto{\pgfqpoint{2.797125in}{2.072167in}}{\pgfqpoint{2.795489in}{2.076117in}}{\pgfqpoint{2.792577in}{2.079029in}}%
\pgfpathcurveto{\pgfqpoint{2.789665in}{2.081941in}}{\pgfqpoint{2.785715in}{2.083577in}}{\pgfqpoint{2.781597in}{2.083577in}}%
\pgfpathcurveto{\pgfqpoint{2.777479in}{2.083577in}}{\pgfqpoint{2.773529in}{2.081941in}}{\pgfqpoint{2.770617in}{2.079029in}}%
\pgfpathcurveto{\pgfqpoint{2.767705in}{2.076117in}}{\pgfqpoint{2.766069in}{2.072167in}}{\pgfqpoint{2.766069in}{2.068049in}}%
\pgfpathcurveto{\pgfqpoint{2.766069in}{2.063931in}}{\pgfqpoint{2.767705in}{2.059981in}}{\pgfqpoint{2.770617in}{2.057069in}}%
\pgfpathcurveto{\pgfqpoint{2.773529in}{2.054157in}}{\pgfqpoint{2.777479in}{2.052521in}}{\pgfqpoint{2.781597in}{2.052521in}}%
\pgfpathclose%
\pgfusepath{fill}%
\end{pgfscope}%
\begin{pgfscope}%
\pgfpathrectangle{\pgfqpoint{0.887500in}{0.275000in}}{\pgfqpoint{4.225000in}{4.225000in}}%
\pgfusepath{clip}%
\pgfsetbuttcap%
\pgfsetroundjoin%
\definecolor{currentfill}{rgb}{0.000000,0.000000,0.000000}%
\pgfsetfillcolor{currentfill}%
\pgfsetfillopacity{0.800000}%
\pgfsetlinewidth{0.000000pt}%
\definecolor{currentstroke}{rgb}{0.000000,0.000000,0.000000}%
\pgfsetstrokecolor{currentstroke}%
\pgfsetstrokeopacity{0.800000}%
\pgfsetdash{}{0pt}%
\pgfpathmoveto{\pgfqpoint{3.236471in}{2.578735in}}%
\pgfpathcurveto{\pgfqpoint{3.240589in}{2.578735in}}{\pgfqpoint{3.244539in}{2.580371in}}{\pgfqpoint{3.247451in}{2.583283in}}%
\pgfpathcurveto{\pgfqpoint{3.250363in}{2.586195in}}{\pgfqpoint{3.252000in}{2.590145in}}{\pgfqpoint{3.252000in}{2.594263in}}%
\pgfpathcurveto{\pgfqpoint{3.252000in}{2.598381in}}{\pgfqpoint{3.250363in}{2.602331in}}{\pgfqpoint{3.247451in}{2.605243in}}%
\pgfpathcurveto{\pgfqpoint{3.244539in}{2.608155in}}{\pgfqpoint{3.240589in}{2.609791in}}{\pgfqpoint{3.236471in}{2.609791in}}%
\pgfpathcurveto{\pgfqpoint{3.232353in}{2.609791in}}{\pgfqpoint{3.228403in}{2.608155in}}{\pgfqpoint{3.225491in}{2.605243in}}%
\pgfpathcurveto{\pgfqpoint{3.222579in}{2.602331in}}{\pgfqpoint{3.220943in}{2.598381in}}{\pgfqpoint{3.220943in}{2.594263in}}%
\pgfpathcurveto{\pgfqpoint{3.220943in}{2.590145in}}{\pgfqpoint{3.222579in}{2.586195in}}{\pgfqpoint{3.225491in}{2.583283in}}%
\pgfpathcurveto{\pgfqpoint{3.228403in}{2.580371in}}{\pgfqpoint{3.232353in}{2.578735in}}{\pgfqpoint{3.236471in}{2.578735in}}%
\pgfpathclose%
\pgfusepath{fill}%
\end{pgfscope}%
\begin{pgfscope}%
\pgfpathrectangle{\pgfqpoint{0.887500in}{0.275000in}}{\pgfqpoint{4.225000in}{4.225000in}}%
\pgfusepath{clip}%
\pgfsetbuttcap%
\pgfsetroundjoin%
\definecolor{currentfill}{rgb}{0.000000,0.000000,0.000000}%
\pgfsetfillcolor{currentfill}%
\pgfsetfillopacity{0.800000}%
\pgfsetlinewidth{0.000000pt}%
\definecolor{currentstroke}{rgb}{0.000000,0.000000,0.000000}%
\pgfsetstrokecolor{currentstroke}%
\pgfsetstrokeopacity{0.800000}%
\pgfsetdash{}{0pt}%
\pgfpathmoveto{\pgfqpoint{4.312946in}{2.008664in}}%
\pgfpathcurveto{\pgfqpoint{4.317064in}{2.008664in}}{\pgfqpoint{4.321014in}{2.010300in}}{\pgfqpoint{4.323926in}{2.013212in}}%
\pgfpathcurveto{\pgfqpoint{4.326838in}{2.016124in}}{\pgfqpoint{4.328474in}{2.020074in}}{\pgfqpoint{4.328474in}{2.024192in}}%
\pgfpathcurveto{\pgfqpoint{4.328474in}{2.028310in}}{\pgfqpoint{4.326838in}{2.032260in}}{\pgfqpoint{4.323926in}{2.035172in}}%
\pgfpathcurveto{\pgfqpoint{4.321014in}{2.038084in}}{\pgfqpoint{4.317064in}{2.039721in}}{\pgfqpoint{4.312946in}{2.039721in}}%
\pgfpathcurveto{\pgfqpoint{4.308828in}{2.039721in}}{\pgfqpoint{4.304878in}{2.038084in}}{\pgfqpoint{4.301966in}{2.035172in}}%
\pgfpathcurveto{\pgfqpoint{4.299054in}{2.032260in}}{\pgfqpoint{4.297418in}{2.028310in}}{\pgfqpoint{4.297418in}{2.024192in}}%
\pgfpathcurveto{\pgfqpoint{4.297418in}{2.020074in}}{\pgfqpoint{4.299054in}{2.016124in}}{\pgfqpoint{4.301966in}{2.013212in}}%
\pgfpathcurveto{\pgfqpoint{4.304878in}{2.010300in}}{\pgfqpoint{4.308828in}{2.008664in}}{\pgfqpoint{4.312946in}{2.008664in}}%
\pgfpathclose%
\pgfusepath{fill}%
\end{pgfscope}%
\begin{pgfscope}%
\pgfpathrectangle{\pgfqpoint{0.887500in}{0.275000in}}{\pgfqpoint{4.225000in}{4.225000in}}%
\pgfusepath{clip}%
\pgfsetbuttcap%
\pgfsetroundjoin%
\definecolor{currentfill}{rgb}{0.000000,0.000000,0.000000}%
\pgfsetfillcolor{currentfill}%
\pgfsetfillopacity{0.800000}%
\pgfsetlinewidth{0.000000pt}%
\definecolor{currentstroke}{rgb}{0.000000,0.000000,0.000000}%
\pgfsetstrokecolor{currentstroke}%
\pgfsetstrokeopacity{0.800000}%
\pgfsetdash{}{0pt}%
\pgfpathmoveto{\pgfqpoint{1.656958in}{2.280161in}}%
\pgfpathcurveto{\pgfqpoint{1.661077in}{2.280161in}}{\pgfqpoint{1.665027in}{2.281797in}}{\pgfqpoint{1.667939in}{2.284709in}}%
\pgfpathcurveto{\pgfqpoint{1.670851in}{2.287621in}}{\pgfqpoint{1.672487in}{2.291571in}}{\pgfqpoint{1.672487in}{2.295689in}}%
\pgfpathcurveto{\pgfqpoint{1.672487in}{2.299807in}}{\pgfqpoint{1.670851in}{2.303757in}}{\pgfqpoint{1.667939in}{2.306669in}}%
\pgfpathcurveto{\pgfqpoint{1.665027in}{2.309581in}}{\pgfqpoint{1.661077in}{2.311218in}}{\pgfqpoint{1.656958in}{2.311218in}}%
\pgfpathcurveto{\pgfqpoint{1.652840in}{2.311218in}}{\pgfqpoint{1.648890in}{2.309581in}}{\pgfqpoint{1.645978in}{2.306669in}}%
\pgfpathcurveto{\pgfqpoint{1.643066in}{2.303757in}}{\pgfqpoint{1.641430in}{2.299807in}}{\pgfqpoint{1.641430in}{2.295689in}}%
\pgfpathcurveto{\pgfqpoint{1.641430in}{2.291571in}}{\pgfqpoint{1.643066in}{2.287621in}}{\pgfqpoint{1.645978in}{2.284709in}}%
\pgfpathcurveto{\pgfqpoint{1.648890in}{2.281797in}}{\pgfqpoint{1.652840in}{2.280161in}}{\pgfqpoint{1.656958in}{2.280161in}}%
\pgfpathclose%
\pgfusepath{fill}%
\end{pgfscope}%
\begin{pgfscope}%
\pgfpathrectangle{\pgfqpoint{0.887500in}{0.275000in}}{\pgfqpoint{4.225000in}{4.225000in}}%
\pgfusepath{clip}%
\pgfsetbuttcap%
\pgfsetroundjoin%
\definecolor{currentfill}{rgb}{0.000000,0.000000,0.000000}%
\pgfsetfillcolor{currentfill}%
\pgfsetfillopacity{0.800000}%
\pgfsetlinewidth{0.000000pt}%
\definecolor{currentstroke}{rgb}{0.000000,0.000000,0.000000}%
\pgfsetstrokecolor{currentstroke}%
\pgfsetstrokeopacity{0.800000}%
\pgfsetdash{}{0pt}%
\pgfpathmoveto{\pgfqpoint{3.188464in}{2.237027in}}%
\pgfpathcurveto{\pgfqpoint{3.192582in}{2.237027in}}{\pgfqpoint{3.196532in}{2.238663in}}{\pgfqpoint{3.199444in}{2.241575in}}%
\pgfpathcurveto{\pgfqpoint{3.202356in}{2.244487in}}{\pgfqpoint{3.203993in}{2.248437in}}{\pgfqpoint{3.203993in}{2.252556in}}%
\pgfpathcurveto{\pgfqpoint{3.203993in}{2.256674in}}{\pgfqpoint{3.202356in}{2.260624in}}{\pgfqpoint{3.199444in}{2.263536in}}%
\pgfpathcurveto{\pgfqpoint{3.196532in}{2.266448in}}{\pgfqpoint{3.192582in}{2.268084in}}{\pgfqpoint{3.188464in}{2.268084in}}%
\pgfpathcurveto{\pgfqpoint{3.184346in}{2.268084in}}{\pgfqpoint{3.180396in}{2.266448in}}{\pgfqpoint{3.177484in}{2.263536in}}%
\pgfpathcurveto{\pgfqpoint{3.174572in}{2.260624in}}{\pgfqpoint{3.172936in}{2.256674in}}{\pgfqpoint{3.172936in}{2.252556in}}%
\pgfpathcurveto{\pgfqpoint{3.172936in}{2.248437in}}{\pgfqpoint{3.174572in}{2.244487in}}{\pgfqpoint{3.177484in}{2.241575in}}%
\pgfpathcurveto{\pgfqpoint{3.180396in}{2.238663in}}{\pgfqpoint{3.184346in}{2.237027in}}{\pgfqpoint{3.188464in}{2.237027in}}%
\pgfpathclose%
\pgfusepath{fill}%
\end{pgfscope}%
\begin{pgfscope}%
\pgfpathrectangle{\pgfqpoint{0.887500in}{0.275000in}}{\pgfqpoint{4.225000in}{4.225000in}}%
\pgfusepath{clip}%
\pgfsetbuttcap%
\pgfsetroundjoin%
\definecolor{currentfill}{rgb}{0.000000,0.000000,0.000000}%
\pgfsetfillcolor{currentfill}%
\pgfsetfillopacity{0.800000}%
\pgfsetlinewidth{0.000000pt}%
\definecolor{currentstroke}{rgb}{0.000000,0.000000,0.000000}%
\pgfsetstrokecolor{currentstroke}%
\pgfsetstrokeopacity{0.800000}%
\pgfsetdash{}{0pt}%
\pgfpathmoveto{\pgfqpoint{4.134055in}{2.123616in}}%
\pgfpathcurveto{\pgfqpoint{4.138173in}{2.123616in}}{\pgfqpoint{4.142123in}{2.125252in}}{\pgfqpoint{4.145035in}{2.128164in}}%
\pgfpathcurveto{\pgfqpoint{4.147947in}{2.131076in}}{\pgfqpoint{4.149583in}{2.135026in}}{\pgfqpoint{4.149583in}{2.139144in}}%
\pgfpathcurveto{\pgfqpoint{4.149583in}{2.143262in}}{\pgfqpoint{4.147947in}{2.147212in}}{\pgfqpoint{4.145035in}{2.150124in}}%
\pgfpathcurveto{\pgfqpoint{4.142123in}{2.153036in}}{\pgfqpoint{4.138173in}{2.154672in}}{\pgfqpoint{4.134055in}{2.154672in}}%
\pgfpathcurveto{\pgfqpoint{4.129937in}{2.154672in}}{\pgfqpoint{4.125987in}{2.153036in}}{\pgfqpoint{4.123075in}{2.150124in}}%
\pgfpathcurveto{\pgfqpoint{4.120163in}{2.147212in}}{\pgfqpoint{4.118527in}{2.143262in}}{\pgfqpoint{4.118527in}{2.139144in}}%
\pgfpathcurveto{\pgfqpoint{4.118527in}{2.135026in}}{\pgfqpoint{4.120163in}{2.131076in}}{\pgfqpoint{4.123075in}{2.128164in}}%
\pgfpathcurveto{\pgfqpoint{4.125987in}{2.125252in}}{\pgfqpoint{4.129937in}{2.123616in}}{\pgfqpoint{4.134055in}{2.123616in}}%
\pgfpathclose%
\pgfusepath{fill}%
\end{pgfscope}%
\begin{pgfscope}%
\pgfpathrectangle{\pgfqpoint{0.887500in}{0.275000in}}{\pgfqpoint{4.225000in}{4.225000in}}%
\pgfusepath{clip}%
\pgfsetbuttcap%
\pgfsetroundjoin%
\definecolor{currentfill}{rgb}{0.000000,0.000000,0.000000}%
\pgfsetfillcolor{currentfill}%
\pgfsetfillopacity{0.800000}%
\pgfsetlinewidth{0.000000pt}%
\definecolor{currentstroke}{rgb}{0.000000,0.000000,0.000000}%
\pgfsetstrokecolor{currentstroke}%
\pgfsetstrokeopacity{0.800000}%
\pgfsetdash{}{0pt}%
\pgfpathmoveto{\pgfqpoint{2.308596in}{2.141718in}}%
\pgfpathcurveto{\pgfqpoint{2.312715in}{2.141718in}}{\pgfqpoint{2.316665in}{2.143354in}}{\pgfqpoint{2.319577in}{2.146266in}}%
\pgfpathcurveto{\pgfqpoint{2.322488in}{2.149178in}}{\pgfqpoint{2.324125in}{2.153128in}}{\pgfqpoint{2.324125in}{2.157246in}}%
\pgfpathcurveto{\pgfqpoint{2.324125in}{2.161364in}}{\pgfqpoint{2.322488in}{2.165314in}}{\pgfqpoint{2.319577in}{2.168226in}}%
\pgfpathcurveto{\pgfqpoint{2.316665in}{2.171138in}}{\pgfqpoint{2.312715in}{2.172774in}}{\pgfqpoint{2.308596in}{2.172774in}}%
\pgfpathcurveto{\pgfqpoint{2.304478in}{2.172774in}}{\pgfqpoint{2.300528in}{2.171138in}}{\pgfqpoint{2.297616in}{2.168226in}}%
\pgfpathcurveto{\pgfqpoint{2.294704in}{2.165314in}}{\pgfqpoint{2.293068in}{2.161364in}}{\pgfqpoint{2.293068in}{2.157246in}}%
\pgfpathcurveto{\pgfqpoint{2.293068in}{2.153128in}}{\pgfqpoint{2.294704in}{2.149178in}}{\pgfqpoint{2.297616in}{2.146266in}}%
\pgfpathcurveto{\pgfqpoint{2.300528in}{2.143354in}}{\pgfqpoint{2.304478in}{2.141718in}}{\pgfqpoint{2.308596in}{2.141718in}}%
\pgfpathclose%
\pgfusepath{fill}%
\end{pgfscope}%
\begin{pgfscope}%
\pgfpathrectangle{\pgfqpoint{0.887500in}{0.275000in}}{\pgfqpoint{4.225000in}{4.225000in}}%
\pgfusepath{clip}%
\pgfsetbuttcap%
\pgfsetroundjoin%
\definecolor{currentfill}{rgb}{0.000000,0.000000,0.000000}%
\pgfsetfillcolor{currentfill}%
\pgfsetfillopacity{0.800000}%
\pgfsetlinewidth{0.000000pt}%
\definecolor{currentstroke}{rgb}{0.000000,0.000000,0.000000}%
\pgfsetstrokecolor{currentstroke}%
\pgfsetstrokeopacity{0.800000}%
\pgfsetdash{}{0pt}%
\pgfpathmoveto{\pgfqpoint{2.960899in}{1.985151in}}%
\pgfpathcurveto{\pgfqpoint{2.965017in}{1.985151in}}{\pgfqpoint{2.968967in}{1.986787in}}{\pgfqpoint{2.971879in}{1.989699in}}%
\pgfpathcurveto{\pgfqpoint{2.974791in}{1.992611in}}{\pgfqpoint{2.976427in}{1.996561in}}{\pgfqpoint{2.976427in}{2.000679in}}%
\pgfpathcurveto{\pgfqpoint{2.976427in}{2.004797in}}{\pgfqpoint{2.974791in}{2.008747in}}{\pgfqpoint{2.971879in}{2.011659in}}%
\pgfpathcurveto{\pgfqpoint{2.968967in}{2.014571in}}{\pgfqpoint{2.965017in}{2.016207in}}{\pgfqpoint{2.960899in}{2.016207in}}%
\pgfpathcurveto{\pgfqpoint{2.956780in}{2.016207in}}{\pgfqpoint{2.952830in}{2.014571in}}{\pgfqpoint{2.949918in}{2.011659in}}%
\pgfpathcurveto{\pgfqpoint{2.947006in}{2.008747in}}{\pgfqpoint{2.945370in}{2.004797in}}{\pgfqpoint{2.945370in}{2.000679in}}%
\pgfpathcurveto{\pgfqpoint{2.945370in}{1.996561in}}{\pgfqpoint{2.947006in}{1.992611in}}{\pgfqpoint{2.949918in}{1.989699in}}%
\pgfpathcurveto{\pgfqpoint{2.952830in}{1.986787in}}{\pgfqpoint{2.956780in}{1.985151in}}{\pgfqpoint{2.960899in}{1.985151in}}%
\pgfpathclose%
\pgfusepath{fill}%
\end{pgfscope}%
\begin{pgfscope}%
\pgfpathrectangle{\pgfqpoint{0.887500in}{0.275000in}}{\pgfqpoint{4.225000in}{4.225000in}}%
\pgfusepath{clip}%
\pgfsetbuttcap%
\pgfsetroundjoin%
\definecolor{currentfill}{rgb}{0.000000,0.000000,0.000000}%
\pgfsetfillcolor{currentfill}%
\pgfsetfillopacity{0.800000}%
\pgfsetlinewidth{0.000000pt}%
\definecolor{currentstroke}{rgb}{0.000000,0.000000,0.000000}%
\pgfsetstrokecolor{currentstroke}%
\pgfsetstrokeopacity{0.800000}%
\pgfsetdash{}{0pt}%
\pgfpathmoveto{\pgfqpoint{3.302322in}{2.382585in}}%
\pgfpathcurveto{\pgfqpoint{3.306440in}{2.382585in}}{\pgfqpoint{3.310390in}{2.384221in}}{\pgfqpoint{3.313302in}{2.387133in}}%
\pgfpathcurveto{\pgfqpoint{3.316214in}{2.390045in}}{\pgfqpoint{3.317850in}{2.393995in}}{\pgfqpoint{3.317850in}{2.398113in}}%
\pgfpathcurveto{\pgfqpoint{3.317850in}{2.402231in}}{\pgfqpoint{3.316214in}{2.406181in}}{\pgfqpoint{3.313302in}{2.409093in}}%
\pgfpathcurveto{\pgfqpoint{3.310390in}{2.412005in}}{\pgfqpoint{3.306440in}{2.413642in}}{\pgfqpoint{3.302322in}{2.413642in}}%
\pgfpathcurveto{\pgfqpoint{3.298204in}{2.413642in}}{\pgfqpoint{3.294254in}{2.412005in}}{\pgfqpoint{3.291342in}{2.409093in}}%
\pgfpathcurveto{\pgfqpoint{3.288430in}{2.406181in}}{\pgfqpoint{3.286794in}{2.402231in}}{\pgfqpoint{3.286794in}{2.398113in}}%
\pgfpathcurveto{\pgfqpoint{3.286794in}{2.393995in}}{\pgfqpoint{3.288430in}{2.390045in}}{\pgfqpoint{3.291342in}{2.387133in}}%
\pgfpathcurveto{\pgfqpoint{3.294254in}{2.384221in}}{\pgfqpoint{3.298204in}{2.382585in}}{\pgfqpoint{3.302322in}{2.382585in}}%
\pgfpathclose%
\pgfusepath{fill}%
\end{pgfscope}%
\begin{pgfscope}%
\pgfpathrectangle{\pgfqpoint{0.887500in}{0.275000in}}{\pgfqpoint{4.225000in}{4.225000in}}%
\pgfusepath{clip}%
\pgfsetbuttcap%
\pgfsetroundjoin%
\definecolor{currentfill}{rgb}{0.000000,0.000000,0.000000}%
\pgfsetfillcolor{currentfill}%
\pgfsetfillopacity{0.800000}%
\pgfsetlinewidth{0.000000pt}%
\definecolor{currentstroke}{rgb}{0.000000,0.000000,0.000000}%
\pgfsetstrokecolor{currentstroke}%
\pgfsetstrokeopacity{0.800000}%
\pgfsetdash{}{0pt}%
\pgfpathmoveto{\pgfqpoint{3.955016in}{2.239713in}}%
\pgfpathcurveto{\pgfqpoint{3.959134in}{2.239713in}}{\pgfqpoint{3.963084in}{2.241349in}}{\pgfqpoint{3.965996in}{2.244261in}}%
\pgfpathcurveto{\pgfqpoint{3.968908in}{2.247173in}}{\pgfqpoint{3.970544in}{2.251123in}}{\pgfqpoint{3.970544in}{2.255241in}}%
\pgfpathcurveto{\pgfqpoint{3.970544in}{2.259359in}}{\pgfqpoint{3.968908in}{2.263309in}}{\pgfqpoint{3.965996in}{2.266221in}}%
\pgfpathcurveto{\pgfqpoint{3.963084in}{2.269133in}}{\pgfqpoint{3.959134in}{2.270769in}}{\pgfqpoint{3.955016in}{2.270769in}}%
\pgfpathcurveto{\pgfqpoint{3.950897in}{2.270769in}}{\pgfqpoint{3.946947in}{2.269133in}}{\pgfqpoint{3.944035in}{2.266221in}}%
\pgfpathcurveto{\pgfqpoint{3.941123in}{2.263309in}}{\pgfqpoint{3.939487in}{2.259359in}}{\pgfqpoint{3.939487in}{2.255241in}}%
\pgfpathcurveto{\pgfqpoint{3.939487in}{2.251123in}}{\pgfqpoint{3.941123in}{2.247173in}}{\pgfqpoint{3.944035in}{2.244261in}}%
\pgfpathcurveto{\pgfqpoint{3.946947in}{2.241349in}}{\pgfqpoint{3.950897in}{2.239713in}}{\pgfqpoint{3.955016in}{2.239713in}}%
\pgfpathclose%
\pgfusepath{fill}%
\end{pgfscope}%
\begin{pgfscope}%
\pgfpathrectangle{\pgfqpoint{0.887500in}{0.275000in}}{\pgfqpoint{4.225000in}{4.225000in}}%
\pgfusepath{clip}%
\pgfsetbuttcap%
\pgfsetroundjoin%
\definecolor{currentfill}{rgb}{0.000000,0.000000,0.000000}%
\pgfsetfillcolor{currentfill}%
\pgfsetfillopacity{0.800000}%
\pgfsetlinewidth{0.000000pt}%
\definecolor{currentstroke}{rgb}{0.000000,0.000000,0.000000}%
\pgfsetstrokecolor{currentstroke}%
\pgfsetstrokeopacity{0.800000}%
\pgfsetdash{}{0pt}%
\pgfpathmoveto{\pgfqpoint{1.835142in}{2.223714in}}%
\pgfpathcurveto{\pgfqpoint{1.839260in}{2.223714in}}{\pgfqpoint{1.843210in}{2.225350in}}{\pgfqpoint{1.846122in}{2.228262in}}%
\pgfpathcurveto{\pgfqpoint{1.849034in}{2.231174in}}{\pgfqpoint{1.850670in}{2.235124in}}{\pgfqpoint{1.850670in}{2.239243in}}%
\pgfpathcurveto{\pgfqpoint{1.850670in}{2.243361in}}{\pgfqpoint{1.849034in}{2.247311in}}{\pgfqpoint{1.846122in}{2.250223in}}%
\pgfpathcurveto{\pgfqpoint{1.843210in}{2.253135in}}{\pgfqpoint{1.839260in}{2.254771in}}{\pgfqpoint{1.835142in}{2.254771in}}%
\pgfpathcurveto{\pgfqpoint{1.831024in}{2.254771in}}{\pgfqpoint{1.827074in}{2.253135in}}{\pgfqpoint{1.824162in}{2.250223in}}%
\pgfpathcurveto{\pgfqpoint{1.821250in}{2.247311in}}{\pgfqpoint{1.819613in}{2.243361in}}{\pgfqpoint{1.819613in}{2.239243in}}%
\pgfpathcurveto{\pgfqpoint{1.819613in}{2.235124in}}{\pgfqpoint{1.821250in}{2.231174in}}{\pgfqpoint{1.824162in}{2.228262in}}%
\pgfpathcurveto{\pgfqpoint{1.827074in}{2.225350in}}{\pgfqpoint{1.831024in}{2.223714in}}{\pgfqpoint{1.835142in}{2.223714in}}%
\pgfpathclose%
\pgfusepath{fill}%
\end{pgfscope}%
\begin{pgfscope}%
\pgfpathrectangle{\pgfqpoint{0.887500in}{0.275000in}}{\pgfqpoint{4.225000in}{4.225000in}}%
\pgfusepath{clip}%
\pgfsetbuttcap%
\pgfsetroundjoin%
\definecolor{currentfill}{rgb}{0.000000,0.000000,0.000000}%
\pgfsetfillcolor{currentfill}%
\pgfsetfillopacity{0.800000}%
\pgfsetlinewidth{0.000000pt}%
\definecolor{currentstroke}{rgb}{0.000000,0.000000,0.000000}%
\pgfsetstrokecolor{currentstroke}%
\pgfsetstrokeopacity{0.800000}%
\pgfsetdash{}{0pt}%
\pgfpathmoveto{\pgfqpoint{3.140456in}{1.914089in}}%
\pgfpathcurveto{\pgfqpoint{3.144575in}{1.914089in}}{\pgfqpoint{3.148525in}{1.915725in}}{\pgfqpoint{3.151437in}{1.918637in}}%
\pgfpathcurveto{\pgfqpoint{3.154349in}{1.921549in}}{\pgfqpoint{3.155985in}{1.925499in}}{\pgfqpoint{3.155985in}{1.929617in}}%
\pgfpathcurveto{\pgfqpoint{3.155985in}{1.933735in}}{\pgfqpoint{3.154349in}{1.937685in}}{\pgfqpoint{3.151437in}{1.940597in}}%
\pgfpathcurveto{\pgfqpoint{3.148525in}{1.943509in}}{\pgfqpoint{3.144575in}{1.945145in}}{\pgfqpoint{3.140456in}{1.945145in}}%
\pgfpathcurveto{\pgfqpoint{3.136338in}{1.945145in}}{\pgfqpoint{3.132388in}{1.943509in}}{\pgfqpoint{3.129476in}{1.940597in}}%
\pgfpathcurveto{\pgfqpoint{3.126564in}{1.937685in}}{\pgfqpoint{3.124928in}{1.933735in}}{\pgfqpoint{3.124928in}{1.929617in}}%
\pgfpathcurveto{\pgfqpoint{3.124928in}{1.925499in}}{\pgfqpoint{3.126564in}{1.921549in}}{\pgfqpoint{3.129476in}{1.918637in}}%
\pgfpathcurveto{\pgfqpoint{3.132388in}{1.915725in}}{\pgfqpoint{3.136338in}{1.914089in}}{\pgfqpoint{3.140456in}{1.914089in}}%
\pgfpathclose%
\pgfusepath{fill}%
\end{pgfscope}%
\begin{pgfscope}%
\pgfpathrectangle{\pgfqpoint{0.887500in}{0.275000in}}{\pgfqpoint{4.225000in}{4.225000in}}%
\pgfusepath{clip}%
\pgfsetbuttcap%
\pgfsetroundjoin%
\definecolor{currentfill}{rgb}{0.000000,0.000000,0.000000}%
\pgfsetfillcolor{currentfill}%
\pgfsetfillopacity{0.800000}%
\pgfsetlinewidth{0.000000pt}%
\definecolor{currentstroke}{rgb}{0.000000,0.000000,0.000000}%
\pgfsetstrokecolor{currentstroke}%
\pgfsetstrokeopacity{0.800000}%
\pgfsetdash{}{0pt}%
\pgfpathmoveto{\pgfqpoint{3.416351in}{2.538152in}}%
\pgfpathcurveto{\pgfqpoint{3.420469in}{2.538152in}}{\pgfqpoint{3.424419in}{2.539788in}}{\pgfqpoint{3.427331in}{2.542700in}}%
\pgfpathcurveto{\pgfqpoint{3.430243in}{2.545612in}}{\pgfqpoint{3.431879in}{2.549562in}}{\pgfqpoint{3.431879in}{2.553680in}}%
\pgfpathcurveto{\pgfqpoint{3.431879in}{2.557799in}}{\pgfqpoint{3.430243in}{2.561749in}}{\pgfqpoint{3.427331in}{2.564661in}}%
\pgfpathcurveto{\pgfqpoint{3.424419in}{2.567573in}}{\pgfqpoint{3.420469in}{2.569209in}}{\pgfqpoint{3.416351in}{2.569209in}}%
\pgfpathcurveto{\pgfqpoint{3.412233in}{2.569209in}}{\pgfqpoint{3.408283in}{2.567573in}}{\pgfqpoint{3.405371in}{2.564661in}}%
\pgfpathcurveto{\pgfqpoint{3.402459in}{2.561749in}}{\pgfqpoint{3.400823in}{2.557799in}}{\pgfqpoint{3.400823in}{2.553680in}}%
\pgfpathcurveto{\pgfqpoint{3.400823in}{2.549562in}}{\pgfqpoint{3.402459in}{2.545612in}}{\pgfqpoint{3.405371in}{2.542700in}}%
\pgfpathcurveto{\pgfqpoint{3.408283in}{2.539788in}}{\pgfqpoint{3.412233in}{2.538152in}}{\pgfqpoint{3.416351in}{2.538152in}}%
\pgfpathclose%
\pgfusepath{fill}%
\end{pgfscope}%
\begin{pgfscope}%
\pgfpathrectangle{\pgfqpoint{0.887500in}{0.275000in}}{\pgfqpoint{4.225000in}{4.225000in}}%
\pgfusepath{clip}%
\pgfsetbuttcap%
\pgfsetroundjoin%
\definecolor{currentfill}{rgb}{0.000000,0.000000,0.000000}%
\pgfsetfillcolor{currentfill}%
\pgfsetfillopacity{0.800000}%
\pgfsetlinewidth{0.000000pt}%
\definecolor{currentstroke}{rgb}{0.000000,0.000000,0.000000}%
\pgfsetstrokecolor{currentstroke}%
\pgfsetstrokeopacity{0.800000}%
\pgfsetdash{}{0pt}%
\pgfpathmoveto{\pgfqpoint{3.254366in}{2.052623in}}%
\pgfpathcurveto{\pgfqpoint{3.258485in}{2.052623in}}{\pgfqpoint{3.262435in}{2.054259in}}{\pgfqpoint{3.265347in}{2.057171in}}%
\pgfpathcurveto{\pgfqpoint{3.268258in}{2.060083in}}{\pgfqpoint{3.269895in}{2.064033in}}{\pgfqpoint{3.269895in}{2.068151in}}%
\pgfpathcurveto{\pgfqpoint{3.269895in}{2.072270in}}{\pgfqpoint{3.268258in}{2.076220in}}{\pgfqpoint{3.265347in}{2.079132in}}%
\pgfpathcurveto{\pgfqpoint{3.262435in}{2.082044in}}{\pgfqpoint{3.258485in}{2.083680in}}{\pgfqpoint{3.254366in}{2.083680in}}%
\pgfpathcurveto{\pgfqpoint{3.250248in}{2.083680in}}{\pgfqpoint{3.246298in}{2.082044in}}{\pgfqpoint{3.243386in}{2.079132in}}%
\pgfpathcurveto{\pgfqpoint{3.240474in}{2.076220in}}{\pgfqpoint{3.238838in}{2.072270in}}{\pgfqpoint{3.238838in}{2.068151in}}%
\pgfpathcurveto{\pgfqpoint{3.238838in}{2.064033in}}{\pgfqpoint{3.240474in}{2.060083in}}{\pgfqpoint{3.243386in}{2.057171in}}%
\pgfpathcurveto{\pgfqpoint{3.246298in}{2.054259in}}{\pgfqpoint{3.250248in}{2.052623in}}{\pgfqpoint{3.254366in}{2.052623in}}%
\pgfpathclose%
\pgfusepath{fill}%
\end{pgfscope}%
\begin{pgfscope}%
\pgfpathrectangle{\pgfqpoint{0.887500in}{0.275000in}}{\pgfqpoint{4.225000in}{4.225000in}}%
\pgfusepath{clip}%
\pgfsetbuttcap%
\pgfsetroundjoin%
\definecolor{currentfill}{rgb}{0.000000,0.000000,0.000000}%
\pgfsetfillcolor{currentfill}%
\pgfsetfillopacity{0.800000}%
\pgfsetlinewidth{0.000000pt}%
\definecolor{currentstroke}{rgb}{0.000000,0.000000,0.000000}%
\pgfsetstrokecolor{currentstroke}%
\pgfsetstrokeopacity{0.800000}%
\pgfsetdash{}{0pt}%
\pgfpathmoveto{\pgfqpoint{2.487583in}{2.080980in}}%
\pgfpathcurveto{\pgfqpoint{2.491701in}{2.080980in}}{\pgfqpoint{2.495651in}{2.082616in}}{\pgfqpoint{2.498563in}{2.085528in}}%
\pgfpathcurveto{\pgfqpoint{2.501475in}{2.088440in}}{\pgfqpoint{2.503111in}{2.092390in}}{\pgfqpoint{2.503111in}{2.096508in}}%
\pgfpathcurveto{\pgfqpoint{2.503111in}{2.100626in}}{\pgfqpoint{2.501475in}{2.104576in}}{\pgfqpoint{2.498563in}{2.107488in}}%
\pgfpathcurveto{\pgfqpoint{2.495651in}{2.110400in}}{\pgfqpoint{2.491701in}{2.112036in}}{\pgfqpoint{2.487583in}{2.112036in}}%
\pgfpathcurveto{\pgfqpoint{2.483465in}{2.112036in}}{\pgfqpoint{2.479515in}{2.110400in}}{\pgfqpoint{2.476603in}{2.107488in}}%
\pgfpathcurveto{\pgfqpoint{2.473691in}{2.104576in}}{\pgfqpoint{2.472055in}{2.100626in}}{\pgfqpoint{2.472055in}{2.096508in}}%
\pgfpathcurveto{\pgfqpoint{2.472055in}{2.092390in}}{\pgfqpoint{2.473691in}{2.088440in}}{\pgfqpoint{2.476603in}{2.085528in}}%
\pgfpathcurveto{\pgfqpoint{2.479515in}{2.082616in}}{\pgfqpoint{2.483465in}{2.080980in}}{\pgfqpoint{2.487583in}{2.080980in}}%
\pgfpathclose%
\pgfusepath{fill}%
\end{pgfscope}%
\begin{pgfscope}%
\pgfpathrectangle{\pgfqpoint{0.887500in}{0.275000in}}{\pgfqpoint{4.225000in}{4.225000in}}%
\pgfusepath{clip}%
\pgfsetbuttcap%
\pgfsetroundjoin%
\definecolor{currentfill}{rgb}{0.000000,0.000000,0.000000}%
\pgfsetfillcolor{currentfill}%
\pgfsetfillopacity{0.800000}%
\pgfsetlinewidth{0.000000pt}%
\definecolor{currentstroke}{rgb}{0.000000,0.000000,0.000000}%
\pgfsetstrokecolor{currentstroke}%
\pgfsetstrokeopacity{0.800000}%
\pgfsetdash{}{0pt}%
\pgfpathmoveto{\pgfqpoint{4.560176in}{1.740752in}}%
\pgfpathcurveto{\pgfqpoint{4.564294in}{1.740752in}}{\pgfqpoint{4.568244in}{1.742388in}}{\pgfqpoint{4.571156in}{1.745300in}}%
\pgfpathcurveto{\pgfqpoint{4.574068in}{1.748212in}}{\pgfqpoint{4.575704in}{1.752162in}}{\pgfqpoint{4.575704in}{1.756280in}}%
\pgfpathcurveto{\pgfqpoint{4.575704in}{1.760398in}}{\pgfqpoint{4.574068in}{1.764348in}}{\pgfqpoint{4.571156in}{1.767260in}}%
\pgfpathcurveto{\pgfqpoint{4.568244in}{1.770172in}}{\pgfqpoint{4.564294in}{1.771808in}}{\pgfqpoint{4.560176in}{1.771808in}}%
\pgfpathcurveto{\pgfqpoint{4.556058in}{1.771808in}}{\pgfqpoint{4.552108in}{1.770172in}}{\pgfqpoint{4.549196in}{1.767260in}}%
\pgfpathcurveto{\pgfqpoint{4.546284in}{1.764348in}}{\pgfqpoint{4.544647in}{1.760398in}}{\pgfqpoint{4.544647in}{1.756280in}}%
\pgfpathcurveto{\pgfqpoint{4.544647in}{1.752162in}}{\pgfqpoint{4.546284in}{1.748212in}}{\pgfqpoint{4.549196in}{1.745300in}}%
\pgfpathcurveto{\pgfqpoint{4.552108in}{1.742388in}}{\pgfqpoint{4.556058in}{1.740752in}}{\pgfqpoint{4.560176in}{1.740752in}}%
\pgfpathclose%
\pgfusepath{fill}%
\end{pgfscope}%
\begin{pgfscope}%
\pgfpathrectangle{\pgfqpoint{0.887500in}{0.275000in}}{\pgfqpoint{4.225000in}{4.225000in}}%
\pgfusepath{clip}%
\pgfsetbuttcap%
\pgfsetroundjoin%
\definecolor{currentfill}{rgb}{0.000000,0.000000,0.000000}%
\pgfsetfillcolor{currentfill}%
\pgfsetfillopacity{0.800000}%
\pgfsetlinewidth{0.000000pt}%
\definecolor{currentstroke}{rgb}{0.000000,0.000000,0.000000}%
\pgfsetstrokecolor{currentstroke}%
\pgfsetstrokeopacity{0.800000}%
\pgfsetdash{}{0pt}%
\pgfpathmoveto{\pgfqpoint{3.775768in}{2.354456in}}%
\pgfpathcurveto{\pgfqpoint{3.779886in}{2.354456in}}{\pgfqpoint{3.783836in}{2.356092in}}{\pgfqpoint{3.786748in}{2.359004in}}%
\pgfpathcurveto{\pgfqpoint{3.789660in}{2.361916in}}{\pgfqpoint{3.791296in}{2.365866in}}{\pgfqpoint{3.791296in}{2.369984in}}%
\pgfpathcurveto{\pgfqpoint{3.791296in}{2.374103in}}{\pgfqpoint{3.789660in}{2.378053in}}{\pgfqpoint{3.786748in}{2.380965in}}%
\pgfpathcurveto{\pgfqpoint{3.783836in}{2.383877in}}{\pgfqpoint{3.779886in}{2.385513in}}{\pgfqpoint{3.775768in}{2.385513in}}%
\pgfpathcurveto{\pgfqpoint{3.771650in}{2.385513in}}{\pgfqpoint{3.767700in}{2.383877in}}{\pgfqpoint{3.764788in}{2.380965in}}%
\pgfpathcurveto{\pgfqpoint{3.761876in}{2.378053in}}{\pgfqpoint{3.760240in}{2.374103in}}{\pgfqpoint{3.760240in}{2.369984in}}%
\pgfpathcurveto{\pgfqpoint{3.760240in}{2.365866in}}{\pgfqpoint{3.761876in}{2.361916in}}{\pgfqpoint{3.764788in}{2.359004in}}%
\pgfpathcurveto{\pgfqpoint{3.767700in}{2.356092in}}{\pgfqpoint{3.771650in}{2.354456in}}{\pgfqpoint{3.775768in}{2.354456in}}%
\pgfpathclose%
\pgfusepath{fill}%
\end{pgfscope}%
\begin{pgfscope}%
\pgfpathrectangle{\pgfqpoint{0.887500in}{0.275000in}}{\pgfqpoint{4.225000in}{4.225000in}}%
\pgfusepath{clip}%
\pgfsetbuttcap%
\pgfsetroundjoin%
\definecolor{currentfill}{rgb}{0.000000,0.000000,0.000000}%
\pgfsetfillcolor{currentfill}%
\pgfsetfillopacity{0.800000}%
\pgfsetlinewidth{0.000000pt}%
\definecolor{currentstroke}{rgb}{0.000000,0.000000,0.000000}%
\pgfsetstrokecolor{currentstroke}%
\pgfsetstrokeopacity{0.800000}%
\pgfsetdash{}{0pt}%
\pgfpathmoveto{\pgfqpoint{3.596200in}{2.458111in}}%
\pgfpathcurveto{\pgfqpoint{3.600318in}{2.458111in}}{\pgfqpoint{3.604268in}{2.459747in}}{\pgfqpoint{3.607180in}{2.462659in}}%
\pgfpathcurveto{\pgfqpoint{3.610092in}{2.465571in}}{\pgfqpoint{3.611728in}{2.469521in}}{\pgfqpoint{3.611728in}{2.473640in}}%
\pgfpathcurveto{\pgfqpoint{3.611728in}{2.477758in}}{\pgfqpoint{3.610092in}{2.481708in}}{\pgfqpoint{3.607180in}{2.484620in}}%
\pgfpathcurveto{\pgfqpoint{3.604268in}{2.487532in}}{\pgfqpoint{3.600318in}{2.489168in}}{\pgfqpoint{3.596200in}{2.489168in}}%
\pgfpathcurveto{\pgfqpoint{3.592082in}{2.489168in}}{\pgfqpoint{3.588132in}{2.487532in}}{\pgfqpoint{3.585220in}{2.484620in}}%
\pgfpathcurveto{\pgfqpoint{3.582308in}{2.481708in}}{\pgfqpoint{3.580672in}{2.477758in}}{\pgfqpoint{3.580672in}{2.473640in}}%
\pgfpathcurveto{\pgfqpoint{3.580672in}{2.469521in}}{\pgfqpoint{3.582308in}{2.465571in}}{\pgfqpoint{3.585220in}{2.462659in}}%
\pgfpathcurveto{\pgfqpoint{3.588132in}{2.459747in}}{\pgfqpoint{3.592082in}{2.458111in}}{\pgfqpoint{3.596200in}{2.458111in}}%
\pgfpathclose%
\pgfusepath{fill}%
\end{pgfscope}%
\begin{pgfscope}%
\pgfpathrectangle{\pgfqpoint{0.887500in}{0.275000in}}{\pgfqpoint{4.225000in}{4.225000in}}%
\pgfusepath{clip}%
\pgfsetbuttcap%
\pgfsetroundjoin%
\definecolor{currentfill}{rgb}{0.000000,0.000000,0.000000}%
\pgfsetfillcolor{currentfill}%
\pgfsetfillopacity{0.800000}%
\pgfsetlinewidth{0.000000pt}%
\definecolor{currentstroke}{rgb}{0.000000,0.000000,0.000000}%
\pgfsetstrokecolor{currentstroke}%
\pgfsetstrokeopacity{0.800000}%
\pgfsetdash{}{0pt}%
\pgfpathmoveto{\pgfqpoint{3.320247in}{1.843106in}}%
\pgfpathcurveto{\pgfqpoint{3.324365in}{1.843106in}}{\pgfqpoint{3.328315in}{1.844742in}}{\pgfqpoint{3.331227in}{1.847654in}}%
\pgfpathcurveto{\pgfqpoint{3.334139in}{1.850566in}}{\pgfqpoint{3.335775in}{1.854516in}}{\pgfqpoint{3.335775in}{1.858634in}}%
\pgfpathcurveto{\pgfqpoint{3.335775in}{1.862752in}}{\pgfqpoint{3.334139in}{1.866702in}}{\pgfqpoint{3.331227in}{1.869614in}}%
\pgfpathcurveto{\pgfqpoint{3.328315in}{1.872526in}}{\pgfqpoint{3.324365in}{1.874162in}}{\pgfqpoint{3.320247in}{1.874162in}}%
\pgfpathcurveto{\pgfqpoint{3.316129in}{1.874162in}}{\pgfqpoint{3.312179in}{1.872526in}}{\pgfqpoint{3.309267in}{1.869614in}}%
\pgfpathcurveto{\pgfqpoint{3.306355in}{1.866702in}}{\pgfqpoint{3.304719in}{1.862752in}}{\pgfqpoint{3.304719in}{1.858634in}}%
\pgfpathcurveto{\pgfqpoint{3.304719in}{1.854516in}}{\pgfqpoint{3.306355in}{1.850566in}}{\pgfqpoint{3.309267in}{1.847654in}}%
\pgfpathcurveto{\pgfqpoint{3.312179in}{1.844742in}}{\pgfqpoint{3.316129in}{1.843106in}}{\pgfqpoint{3.320247in}{1.843106in}}%
\pgfpathclose%
\pgfusepath{fill}%
\end{pgfscope}%
\begin{pgfscope}%
\pgfpathrectangle{\pgfqpoint{0.887500in}{0.275000in}}{\pgfqpoint{4.225000in}{4.225000in}}%
\pgfusepath{clip}%
\pgfsetbuttcap%
\pgfsetroundjoin%
\definecolor{currentfill}{rgb}{0.000000,0.000000,0.000000}%
\pgfsetfillcolor{currentfill}%
\pgfsetfillopacity{0.800000}%
\pgfsetlinewidth{0.000000pt}%
\definecolor{currentstroke}{rgb}{0.000000,0.000000,0.000000}%
\pgfsetstrokecolor{currentstroke}%
\pgfsetstrokeopacity{0.800000}%
\pgfsetdash{}{0pt}%
\pgfpathmoveto{\pgfqpoint{2.013729in}{2.165516in}}%
\pgfpathcurveto{\pgfqpoint{2.017847in}{2.165516in}}{\pgfqpoint{2.021797in}{2.167153in}}{\pgfqpoint{2.024709in}{2.170065in}}%
\pgfpathcurveto{\pgfqpoint{2.027621in}{2.172977in}}{\pgfqpoint{2.029257in}{2.176927in}}{\pgfqpoint{2.029257in}{2.181045in}}%
\pgfpathcurveto{\pgfqpoint{2.029257in}{2.185163in}}{\pgfqpoint{2.027621in}{2.189113in}}{\pgfqpoint{2.024709in}{2.192025in}}%
\pgfpathcurveto{\pgfqpoint{2.021797in}{2.194937in}}{\pgfqpoint{2.017847in}{2.196573in}}{\pgfqpoint{2.013729in}{2.196573in}}%
\pgfpathcurveto{\pgfqpoint{2.009611in}{2.196573in}}{\pgfqpoint{2.005661in}{2.194937in}}{\pgfqpoint{2.002749in}{2.192025in}}%
\pgfpathcurveto{\pgfqpoint{1.999837in}{2.189113in}}{\pgfqpoint{1.998200in}{2.185163in}}{\pgfqpoint{1.998200in}{2.181045in}}%
\pgfpathcurveto{\pgfqpoint{1.998200in}{2.176927in}}{\pgfqpoint{1.999837in}{2.172977in}}{\pgfqpoint{2.002749in}{2.170065in}}%
\pgfpathcurveto{\pgfqpoint{2.005661in}{2.167153in}}{\pgfqpoint{2.009611in}{2.165516in}}{\pgfqpoint{2.013729in}{2.165516in}}%
\pgfpathclose%
\pgfusepath{fill}%
\end{pgfscope}%
\begin{pgfscope}%
\pgfpathrectangle{\pgfqpoint{0.887500in}{0.275000in}}{\pgfqpoint{4.225000in}{4.225000in}}%
\pgfusepath{clip}%
\pgfsetbuttcap%
\pgfsetroundjoin%
\definecolor{currentfill}{rgb}{0.000000,0.000000,0.000000}%
\pgfsetfillcolor{currentfill}%
\pgfsetfillopacity{0.800000}%
\pgfsetlinewidth{0.000000pt}%
\definecolor{currentstroke}{rgb}{0.000000,0.000000,0.000000}%
\pgfsetstrokecolor{currentstroke}%
\pgfsetstrokeopacity{0.800000}%
\pgfsetdash{}{0pt}%
\pgfpathmoveto{\pgfqpoint{4.381326in}{1.861040in}}%
\pgfpathcurveto{\pgfqpoint{4.385444in}{1.861040in}}{\pgfqpoint{4.389394in}{1.862676in}}{\pgfqpoint{4.392306in}{1.865588in}}%
\pgfpathcurveto{\pgfqpoint{4.395218in}{1.868500in}}{\pgfqpoint{4.396854in}{1.872450in}}{\pgfqpoint{4.396854in}{1.876568in}}%
\pgfpathcurveto{\pgfqpoint{4.396854in}{1.880686in}}{\pgfqpoint{4.395218in}{1.884636in}}{\pgfqpoint{4.392306in}{1.887548in}}%
\pgfpathcurveto{\pgfqpoint{4.389394in}{1.890460in}}{\pgfqpoint{4.385444in}{1.892097in}}{\pgfqpoint{4.381326in}{1.892097in}}%
\pgfpathcurveto{\pgfqpoint{4.377208in}{1.892097in}}{\pgfqpoint{4.373258in}{1.890460in}}{\pgfqpoint{4.370346in}{1.887548in}}%
\pgfpathcurveto{\pgfqpoint{4.367434in}{1.884636in}}{\pgfqpoint{4.365798in}{1.880686in}}{\pgfqpoint{4.365798in}{1.876568in}}%
\pgfpathcurveto{\pgfqpoint{4.365798in}{1.872450in}}{\pgfqpoint{4.367434in}{1.868500in}}{\pgfqpoint{4.370346in}{1.865588in}}%
\pgfpathcurveto{\pgfqpoint{4.373258in}{1.862676in}}{\pgfqpoint{4.377208in}{1.861040in}}{\pgfqpoint{4.381326in}{1.861040in}}%
\pgfpathclose%
\pgfusepath{fill}%
\end{pgfscope}%
\begin{pgfscope}%
\pgfpathrectangle{\pgfqpoint{0.887500in}{0.275000in}}{\pgfqpoint{4.225000in}{4.225000in}}%
\pgfusepath{clip}%
\pgfsetbuttcap%
\pgfsetroundjoin%
\definecolor{currentfill}{rgb}{0.000000,0.000000,0.000000}%
\pgfsetfillcolor{currentfill}%
\pgfsetfillopacity{0.800000}%
\pgfsetlinewidth{0.000000pt}%
\definecolor{currentstroke}{rgb}{0.000000,0.000000,0.000000}%
\pgfsetstrokecolor{currentstroke}%
\pgfsetstrokeopacity{0.800000}%
\pgfsetdash{}{0pt}%
\pgfpathmoveto{\pgfqpoint{4.201736in}{1.963722in}}%
\pgfpathcurveto{\pgfqpoint{4.205854in}{1.963722in}}{\pgfqpoint{4.209804in}{1.965358in}}{\pgfqpoint{4.212716in}{1.968270in}}%
\pgfpathcurveto{\pgfqpoint{4.215628in}{1.971182in}}{\pgfqpoint{4.217264in}{1.975132in}}{\pgfqpoint{4.217264in}{1.979250in}}%
\pgfpathcurveto{\pgfqpoint{4.217264in}{1.983368in}}{\pgfqpoint{4.215628in}{1.987318in}}{\pgfqpoint{4.212716in}{1.990230in}}%
\pgfpathcurveto{\pgfqpoint{4.209804in}{1.993142in}}{\pgfqpoint{4.205854in}{1.994778in}}{\pgfqpoint{4.201736in}{1.994778in}}%
\pgfpathcurveto{\pgfqpoint{4.197618in}{1.994778in}}{\pgfqpoint{4.193668in}{1.993142in}}{\pgfqpoint{4.190756in}{1.990230in}}%
\pgfpathcurveto{\pgfqpoint{4.187844in}{1.987318in}}{\pgfqpoint{4.186208in}{1.983368in}}{\pgfqpoint{4.186208in}{1.979250in}}%
\pgfpathcurveto{\pgfqpoint{4.186208in}{1.975132in}}{\pgfqpoint{4.187844in}{1.971182in}}{\pgfqpoint{4.190756in}{1.968270in}}%
\pgfpathcurveto{\pgfqpoint{4.193668in}{1.965358in}}{\pgfqpoint{4.197618in}{1.963722in}}{\pgfqpoint{4.201736in}{1.963722in}}%
\pgfpathclose%
\pgfusepath{fill}%
\end{pgfscope}%
\begin{pgfscope}%
\pgfpathrectangle{\pgfqpoint{0.887500in}{0.275000in}}{\pgfqpoint{4.225000in}{4.225000in}}%
\pgfusepath{clip}%
\pgfsetbuttcap%
\pgfsetroundjoin%
\definecolor{currentfill}{rgb}{0.000000,0.000000,0.000000}%
\pgfsetfillcolor{currentfill}%
\pgfsetfillopacity{0.800000}%
\pgfsetlinewidth{0.000000pt}%
\definecolor{currentstroke}{rgb}{0.000000,0.000000,0.000000}%
\pgfsetstrokecolor{currentstroke}%
\pgfsetstrokeopacity{0.800000}%
\pgfsetdash{}{0pt}%
\pgfpathmoveto{\pgfqpoint{2.666922in}{2.017298in}}%
\pgfpathcurveto{\pgfqpoint{2.671040in}{2.017298in}}{\pgfqpoint{2.674990in}{2.018934in}}{\pgfqpoint{2.677902in}{2.021846in}}%
\pgfpathcurveto{\pgfqpoint{2.680814in}{2.024758in}}{\pgfqpoint{2.682450in}{2.028708in}}{\pgfqpoint{2.682450in}{2.032826in}}%
\pgfpathcurveto{\pgfqpoint{2.682450in}{2.036944in}}{\pgfqpoint{2.680814in}{2.040894in}}{\pgfqpoint{2.677902in}{2.043806in}}%
\pgfpathcurveto{\pgfqpoint{2.674990in}{2.046718in}}{\pgfqpoint{2.671040in}{2.048354in}}{\pgfqpoint{2.666922in}{2.048354in}}%
\pgfpathcurveto{\pgfqpoint{2.662804in}{2.048354in}}{\pgfqpoint{2.658854in}{2.046718in}}{\pgfqpoint{2.655942in}{2.043806in}}%
\pgfpathcurveto{\pgfqpoint{2.653030in}{2.040894in}}{\pgfqpoint{2.651394in}{2.036944in}}{\pgfqpoint{2.651394in}{2.032826in}}%
\pgfpathcurveto{\pgfqpoint{2.651394in}{2.028708in}}{\pgfqpoint{2.653030in}{2.024758in}}{\pgfqpoint{2.655942in}{2.021846in}}%
\pgfpathcurveto{\pgfqpoint{2.658854in}{2.018934in}}{\pgfqpoint{2.662804in}{2.017298in}}{\pgfqpoint{2.666922in}{2.017298in}}%
\pgfpathclose%
\pgfusepath{fill}%
\end{pgfscope}%
\begin{pgfscope}%
\pgfpathrectangle{\pgfqpoint{0.887500in}{0.275000in}}{\pgfqpoint{4.225000in}{4.225000in}}%
\pgfusepath{clip}%
\pgfsetbuttcap%
\pgfsetroundjoin%
\definecolor{currentfill}{rgb}{0.000000,0.000000,0.000000}%
\pgfsetfillcolor{currentfill}%
\pgfsetfillopacity{0.800000}%
\pgfsetlinewidth{0.000000pt}%
\definecolor{currentstroke}{rgb}{0.000000,0.000000,0.000000}%
\pgfsetstrokecolor{currentstroke}%
\pgfsetstrokeopacity{0.800000}%
\pgfsetdash{}{0pt}%
\pgfpathmoveto{\pgfqpoint{3.368648in}{2.233046in}}%
\pgfpathcurveto{\pgfqpoint{3.372767in}{2.233046in}}{\pgfqpoint{3.376717in}{2.234682in}}{\pgfqpoint{3.379629in}{2.237594in}}%
\pgfpathcurveto{\pgfqpoint{3.382541in}{2.240506in}}{\pgfqpoint{3.384177in}{2.244456in}}{\pgfqpoint{3.384177in}{2.248574in}}%
\pgfpathcurveto{\pgfqpoint{3.384177in}{2.252693in}}{\pgfqpoint{3.382541in}{2.256643in}}{\pgfqpoint{3.379629in}{2.259555in}}%
\pgfpathcurveto{\pgfqpoint{3.376717in}{2.262467in}}{\pgfqpoint{3.372767in}{2.264103in}}{\pgfqpoint{3.368648in}{2.264103in}}%
\pgfpathcurveto{\pgfqpoint{3.364530in}{2.264103in}}{\pgfqpoint{3.360580in}{2.262467in}}{\pgfqpoint{3.357668in}{2.259555in}}%
\pgfpathcurveto{\pgfqpoint{3.354756in}{2.256643in}}{\pgfqpoint{3.353120in}{2.252693in}}{\pgfqpoint{3.353120in}{2.248574in}}%
\pgfpathcurveto{\pgfqpoint{3.353120in}{2.244456in}}{\pgfqpoint{3.354756in}{2.240506in}}{\pgfqpoint{3.357668in}{2.237594in}}%
\pgfpathcurveto{\pgfqpoint{3.360580in}{2.234682in}}{\pgfqpoint{3.364530in}{2.233046in}}{\pgfqpoint{3.368648in}{2.233046in}}%
\pgfpathclose%
\pgfusepath{fill}%
\end{pgfscope}%
\begin{pgfscope}%
\pgfpathrectangle{\pgfqpoint{0.887500in}{0.275000in}}{\pgfqpoint{4.225000in}{4.225000in}}%
\pgfusepath{clip}%
\pgfsetbuttcap%
\pgfsetroundjoin%
\definecolor{currentfill}{rgb}{0.000000,0.000000,0.000000}%
\pgfsetfillcolor{currentfill}%
\pgfsetfillopacity{0.800000}%
\pgfsetlinewidth{0.000000pt}%
\definecolor{currentstroke}{rgb}{0.000000,0.000000,0.000000}%
\pgfsetstrokecolor{currentstroke}%
\pgfsetstrokeopacity{0.800000}%
\pgfsetdash{}{0pt}%
\pgfpathmoveto{\pgfqpoint{4.022607in}{2.087511in}}%
\pgfpathcurveto{\pgfqpoint{4.026725in}{2.087511in}}{\pgfqpoint{4.030675in}{2.089148in}}{\pgfqpoint{4.033587in}{2.092060in}}%
\pgfpathcurveto{\pgfqpoint{4.036499in}{2.094972in}}{\pgfqpoint{4.038135in}{2.098922in}}{\pgfqpoint{4.038135in}{2.103040in}}%
\pgfpathcurveto{\pgfqpoint{4.038135in}{2.107158in}}{\pgfqpoint{4.036499in}{2.111108in}}{\pgfqpoint{4.033587in}{2.114020in}}%
\pgfpathcurveto{\pgfqpoint{4.030675in}{2.116932in}}{\pgfqpoint{4.026725in}{2.118568in}}{\pgfqpoint{4.022607in}{2.118568in}}%
\pgfpathcurveto{\pgfqpoint{4.018489in}{2.118568in}}{\pgfqpoint{4.014539in}{2.116932in}}{\pgfqpoint{4.011627in}{2.114020in}}%
\pgfpathcurveto{\pgfqpoint{4.008715in}{2.111108in}}{\pgfqpoint{4.007079in}{2.107158in}}{\pgfqpoint{4.007079in}{2.103040in}}%
\pgfpathcurveto{\pgfqpoint{4.007079in}{2.098922in}}{\pgfqpoint{4.008715in}{2.094972in}}{\pgfqpoint{4.011627in}{2.092060in}}%
\pgfpathcurveto{\pgfqpoint{4.014539in}{2.089148in}}{\pgfqpoint{4.018489in}{2.087511in}}{\pgfqpoint{4.022607in}{2.087511in}}%
\pgfpathclose%
\pgfusepath{fill}%
\end{pgfscope}%
\begin{pgfscope}%
\pgfpathrectangle{\pgfqpoint{0.887500in}{0.275000in}}{\pgfqpoint{4.225000in}{4.225000in}}%
\pgfusepath{clip}%
\pgfsetbuttcap%
\pgfsetroundjoin%
\definecolor{currentfill}{rgb}{0.000000,0.000000,0.000000}%
\pgfsetfillcolor{currentfill}%
\pgfsetfillopacity{0.800000}%
\pgfsetlinewidth{0.000000pt}%
\definecolor{currentstroke}{rgb}{0.000000,0.000000,0.000000}%
\pgfsetstrokecolor{currentstroke}%
\pgfsetstrokeopacity{0.800000}%
\pgfsetdash{}{0pt}%
\pgfpathmoveto{\pgfqpoint{2.192689in}{2.106000in}}%
\pgfpathcurveto{\pgfqpoint{2.196807in}{2.106000in}}{\pgfqpoint{2.200757in}{2.107636in}}{\pgfqpoint{2.203669in}{2.110548in}}%
\pgfpathcurveto{\pgfqpoint{2.206581in}{2.113460in}}{\pgfqpoint{2.208217in}{2.117410in}}{\pgfqpoint{2.208217in}{2.121528in}}%
\pgfpathcurveto{\pgfqpoint{2.208217in}{2.125646in}}{\pgfqpoint{2.206581in}{2.129597in}}{\pgfqpoint{2.203669in}{2.132508in}}%
\pgfpathcurveto{\pgfqpoint{2.200757in}{2.135420in}}{\pgfqpoint{2.196807in}{2.137057in}}{\pgfqpoint{2.192689in}{2.137057in}}%
\pgfpathcurveto{\pgfqpoint{2.188571in}{2.137057in}}{\pgfqpoint{2.184621in}{2.135420in}}{\pgfqpoint{2.181709in}{2.132508in}}%
\pgfpathcurveto{\pgfqpoint{2.178797in}{2.129597in}}{\pgfqpoint{2.177161in}{2.125646in}}{\pgfqpoint{2.177161in}{2.121528in}}%
\pgfpathcurveto{\pgfqpoint{2.177161in}{2.117410in}}{\pgfqpoint{2.178797in}{2.113460in}}{\pgfqpoint{2.181709in}{2.110548in}}%
\pgfpathcurveto{\pgfqpoint{2.184621in}{2.107636in}}{\pgfqpoint{2.188571in}{2.106000in}}{\pgfqpoint{2.192689in}{2.106000in}}%
\pgfpathclose%
\pgfusepath{fill}%
\end{pgfscope}%
\begin{pgfscope}%
\pgfpathrectangle{\pgfqpoint{0.887500in}{0.275000in}}{\pgfqpoint{4.225000in}{4.225000in}}%
\pgfusepath{clip}%
\pgfsetbuttcap%
\pgfsetroundjoin%
\definecolor{currentfill}{rgb}{0.000000,0.000000,0.000000}%
\pgfsetfillcolor{currentfill}%
\pgfsetfillopacity{0.800000}%
\pgfsetlinewidth{0.000000pt}%
\definecolor{currentstroke}{rgb}{0.000000,0.000000,0.000000}%
\pgfsetstrokecolor{currentstroke}%
\pgfsetstrokeopacity{0.800000}%
\pgfsetdash{}{0pt}%
\pgfpathmoveto{\pgfqpoint{2.846580in}{1.949822in}}%
\pgfpathcurveto{\pgfqpoint{2.850698in}{1.949822in}}{\pgfqpoint{2.854648in}{1.951458in}}{\pgfqpoint{2.857560in}{1.954370in}}%
\pgfpathcurveto{\pgfqpoint{2.860472in}{1.957282in}}{\pgfqpoint{2.862108in}{1.961232in}}{\pgfqpoint{2.862108in}{1.965350in}}%
\pgfpathcurveto{\pgfqpoint{2.862108in}{1.969468in}}{\pgfqpoint{2.860472in}{1.973418in}}{\pgfqpoint{2.857560in}{1.976330in}}%
\pgfpathcurveto{\pgfqpoint{2.854648in}{1.979242in}}{\pgfqpoint{2.850698in}{1.980878in}}{\pgfqpoint{2.846580in}{1.980878in}}%
\pgfpathcurveto{\pgfqpoint{2.842462in}{1.980878in}}{\pgfqpoint{2.838512in}{1.979242in}}{\pgfqpoint{2.835600in}{1.976330in}}%
\pgfpathcurveto{\pgfqpoint{2.832688in}{1.973418in}}{\pgfqpoint{2.831052in}{1.969468in}}{\pgfqpoint{2.831052in}{1.965350in}}%
\pgfpathcurveto{\pgfqpoint{2.831052in}{1.961232in}}{\pgfqpoint{2.832688in}{1.957282in}}{\pgfqpoint{2.835600in}{1.954370in}}%
\pgfpathcurveto{\pgfqpoint{2.838512in}{1.951458in}}{\pgfqpoint{2.842462in}{1.949822in}}{\pgfqpoint{2.846580in}{1.949822in}}%
\pgfpathclose%
\pgfusepath{fill}%
\end{pgfscope}%
\begin{pgfscope}%
\pgfpathrectangle{\pgfqpoint{0.887500in}{0.275000in}}{\pgfqpoint{4.225000in}{4.225000in}}%
\pgfusepath{clip}%
\pgfsetbuttcap%
\pgfsetroundjoin%
\definecolor{currentfill}{rgb}{0.000000,0.000000,0.000000}%
\pgfsetfillcolor{currentfill}%
\pgfsetfillopacity{0.800000}%
\pgfsetlinewidth{0.000000pt}%
\definecolor{currentstroke}{rgb}{0.000000,0.000000,0.000000}%
\pgfsetstrokecolor{currentstroke}%
\pgfsetstrokeopacity{0.800000}%
\pgfsetdash{}{0pt}%
\pgfpathmoveto{\pgfqpoint{3.482949in}{2.389179in}}%
\pgfpathcurveto{\pgfqpoint{3.487067in}{2.389179in}}{\pgfqpoint{3.491017in}{2.390815in}}{\pgfqpoint{3.493929in}{2.393727in}}%
\pgfpathcurveto{\pgfqpoint{3.496841in}{2.396639in}}{\pgfqpoint{3.498477in}{2.400589in}}{\pgfqpoint{3.498477in}{2.404707in}}%
\pgfpathcurveto{\pgfqpoint{3.498477in}{2.408825in}}{\pgfqpoint{3.496841in}{2.412775in}}{\pgfqpoint{3.493929in}{2.415687in}}%
\pgfpathcurveto{\pgfqpoint{3.491017in}{2.418599in}}{\pgfqpoint{3.487067in}{2.420235in}}{\pgfqpoint{3.482949in}{2.420235in}}%
\pgfpathcurveto{\pgfqpoint{3.478830in}{2.420235in}}{\pgfqpoint{3.474880in}{2.418599in}}{\pgfqpoint{3.471968in}{2.415687in}}%
\pgfpathcurveto{\pgfqpoint{3.469056in}{2.412775in}}{\pgfqpoint{3.467420in}{2.408825in}}{\pgfqpoint{3.467420in}{2.404707in}}%
\pgfpathcurveto{\pgfqpoint{3.467420in}{2.400589in}}{\pgfqpoint{3.469056in}{2.396639in}}{\pgfqpoint{3.471968in}{2.393727in}}%
\pgfpathcurveto{\pgfqpoint{3.474880in}{2.390815in}}{\pgfqpoint{3.478830in}{2.389179in}}{\pgfqpoint{3.482949in}{2.389179in}}%
\pgfpathclose%
\pgfusepath{fill}%
\end{pgfscope}%
\begin{pgfscope}%
\pgfpathrectangle{\pgfqpoint{0.887500in}{0.275000in}}{\pgfqpoint{4.225000in}{4.225000in}}%
\pgfusepath{clip}%
\pgfsetbuttcap%
\pgfsetroundjoin%
\definecolor{currentfill}{rgb}{0.000000,0.000000,0.000000}%
\pgfsetfillcolor{currentfill}%
\pgfsetfillopacity{0.800000}%
\pgfsetlinewidth{0.000000pt}%
\definecolor{currentstroke}{rgb}{0.000000,0.000000,0.000000}%
\pgfsetstrokecolor{currentstroke}%
\pgfsetstrokeopacity{0.800000}%
\pgfsetdash{}{0pt}%
\pgfpathmoveto{\pgfqpoint{3.662998in}{2.296794in}}%
\pgfpathcurveto{\pgfqpoint{3.667116in}{2.296794in}}{\pgfqpoint{3.671066in}{2.298431in}}{\pgfqpoint{3.673978in}{2.301343in}}%
\pgfpathcurveto{\pgfqpoint{3.676890in}{2.304254in}}{\pgfqpoint{3.678526in}{2.308205in}}{\pgfqpoint{3.678526in}{2.312323in}}%
\pgfpathcurveto{\pgfqpoint{3.678526in}{2.316441in}}{\pgfqpoint{3.676890in}{2.320391in}}{\pgfqpoint{3.673978in}{2.323303in}}%
\pgfpathcurveto{\pgfqpoint{3.671066in}{2.326215in}}{\pgfqpoint{3.667116in}{2.327851in}}{\pgfqpoint{3.662998in}{2.327851in}}%
\pgfpathcurveto{\pgfqpoint{3.658880in}{2.327851in}}{\pgfqpoint{3.654930in}{2.326215in}}{\pgfqpoint{3.652018in}{2.323303in}}%
\pgfpathcurveto{\pgfqpoint{3.649106in}{2.320391in}}{\pgfqpoint{3.647470in}{2.316441in}}{\pgfqpoint{3.647470in}{2.312323in}}%
\pgfpathcurveto{\pgfqpoint{3.647470in}{2.308205in}}{\pgfqpoint{3.649106in}{2.304254in}}{\pgfqpoint{3.652018in}{2.301343in}}%
\pgfpathcurveto{\pgfqpoint{3.654930in}{2.298431in}}{\pgfqpoint{3.658880in}{2.296794in}}{\pgfqpoint{3.662998in}{2.296794in}}%
\pgfpathclose%
\pgfusepath{fill}%
\end{pgfscope}%
\begin{pgfscope}%
\pgfpathrectangle{\pgfqpoint{0.887500in}{0.275000in}}{\pgfqpoint{4.225000in}{4.225000in}}%
\pgfusepath{clip}%
\pgfsetbuttcap%
\pgfsetroundjoin%
\definecolor{currentfill}{rgb}{0.000000,0.000000,0.000000}%
\pgfsetfillcolor{currentfill}%
\pgfsetfillopacity{0.800000}%
\pgfsetlinewidth{0.000000pt}%
\definecolor{currentstroke}{rgb}{0.000000,0.000000,0.000000}%
\pgfsetstrokecolor{currentstroke}%
\pgfsetstrokeopacity{0.800000}%
\pgfsetdash{}{0pt}%
\pgfpathmoveto{\pgfqpoint{1.718008in}{2.188351in}}%
\pgfpathcurveto{\pgfqpoint{1.722126in}{2.188351in}}{\pgfqpoint{1.726076in}{2.189987in}}{\pgfqpoint{1.728988in}{2.192899in}}%
\pgfpathcurveto{\pgfqpoint{1.731900in}{2.195811in}}{\pgfqpoint{1.733536in}{2.199761in}}{\pgfqpoint{1.733536in}{2.203879in}}%
\pgfpathcurveto{\pgfqpoint{1.733536in}{2.207997in}}{\pgfqpoint{1.731900in}{2.211947in}}{\pgfqpoint{1.728988in}{2.214859in}}%
\pgfpathcurveto{\pgfqpoint{1.726076in}{2.217771in}}{\pgfqpoint{1.722126in}{2.219407in}}{\pgfqpoint{1.718008in}{2.219407in}}%
\pgfpathcurveto{\pgfqpoint{1.713890in}{2.219407in}}{\pgfqpoint{1.709940in}{2.217771in}}{\pgfqpoint{1.707028in}{2.214859in}}%
\pgfpathcurveto{\pgfqpoint{1.704116in}{2.211947in}}{\pgfqpoint{1.702480in}{2.207997in}}{\pgfqpoint{1.702480in}{2.203879in}}%
\pgfpathcurveto{\pgfqpoint{1.702480in}{2.199761in}}{\pgfqpoint{1.704116in}{2.195811in}}{\pgfqpoint{1.707028in}{2.192899in}}%
\pgfpathcurveto{\pgfqpoint{1.709940in}{2.189987in}}{\pgfqpoint{1.713890in}{2.188351in}}{\pgfqpoint{1.718008in}{2.188351in}}%
\pgfpathclose%
\pgfusepath{fill}%
\end{pgfscope}%
\begin{pgfscope}%
\pgfpathrectangle{\pgfqpoint{0.887500in}{0.275000in}}{\pgfqpoint{4.225000in}{4.225000in}}%
\pgfusepath{clip}%
\pgfsetbuttcap%
\pgfsetroundjoin%
\definecolor{currentfill}{rgb}{0.000000,0.000000,0.000000}%
\pgfsetfillcolor{currentfill}%
\pgfsetfillopacity{0.800000}%
\pgfsetlinewidth{0.000000pt}%
\definecolor{currentstroke}{rgb}{0.000000,0.000000,0.000000}%
\pgfsetstrokecolor{currentstroke}%
\pgfsetstrokeopacity{0.800000}%
\pgfsetdash{}{0pt}%
\pgfpathmoveto{\pgfqpoint{3.026507in}{1.878476in}}%
\pgfpathcurveto{\pgfqpoint{3.030625in}{1.878476in}}{\pgfqpoint{3.034575in}{1.880112in}}{\pgfqpoint{3.037487in}{1.883024in}}%
\pgfpathcurveto{\pgfqpoint{3.040399in}{1.885936in}}{\pgfqpoint{3.042035in}{1.889886in}}{\pgfqpoint{3.042035in}{1.894004in}}%
\pgfpathcurveto{\pgfqpoint{3.042035in}{1.898122in}}{\pgfqpoint{3.040399in}{1.902072in}}{\pgfqpoint{3.037487in}{1.904984in}}%
\pgfpathcurveto{\pgfqpoint{3.034575in}{1.907896in}}{\pgfqpoint{3.030625in}{1.909532in}}{\pgfqpoint{3.026507in}{1.909532in}}%
\pgfpathcurveto{\pgfqpoint{3.022389in}{1.909532in}}{\pgfqpoint{3.018439in}{1.907896in}}{\pgfqpoint{3.015527in}{1.904984in}}%
\pgfpathcurveto{\pgfqpoint{3.012615in}{1.902072in}}{\pgfqpoint{3.010979in}{1.898122in}}{\pgfqpoint{3.010979in}{1.894004in}}%
\pgfpathcurveto{\pgfqpoint{3.010979in}{1.889886in}}{\pgfqpoint{3.012615in}{1.885936in}}{\pgfqpoint{3.015527in}{1.883024in}}%
\pgfpathcurveto{\pgfqpoint{3.018439in}{1.880112in}}{\pgfqpoint{3.022389in}{1.878476in}}{\pgfqpoint{3.026507in}{1.878476in}}%
\pgfpathclose%
\pgfusepath{fill}%
\end{pgfscope}%
\begin{pgfscope}%
\pgfpathrectangle{\pgfqpoint{0.887500in}{0.275000in}}{\pgfqpoint{4.225000in}{4.225000in}}%
\pgfusepath{clip}%
\pgfsetbuttcap%
\pgfsetroundjoin%
\definecolor{currentfill}{rgb}{0.000000,0.000000,0.000000}%
\pgfsetfillcolor{currentfill}%
\pgfsetfillopacity{0.800000}%
\pgfsetlinewidth{0.000000pt}%
\definecolor{currentstroke}{rgb}{0.000000,0.000000,0.000000}%
\pgfsetstrokecolor{currentstroke}%
\pgfsetstrokeopacity{0.800000}%
\pgfsetdash{}{0pt}%
\pgfpathmoveto{\pgfqpoint{2.372014in}{2.044737in}}%
\pgfpathcurveto{\pgfqpoint{2.376132in}{2.044737in}}{\pgfqpoint{2.380082in}{2.046373in}}{\pgfqpoint{2.382994in}{2.049285in}}%
\pgfpathcurveto{\pgfqpoint{2.385906in}{2.052197in}}{\pgfqpoint{2.387542in}{2.056147in}}{\pgfqpoint{2.387542in}{2.060265in}}%
\pgfpathcurveto{\pgfqpoint{2.387542in}{2.064383in}}{\pgfqpoint{2.385906in}{2.068333in}}{\pgfqpoint{2.382994in}{2.071245in}}%
\pgfpathcurveto{\pgfqpoint{2.380082in}{2.074157in}}{\pgfqpoint{2.376132in}{2.075793in}}{\pgfqpoint{2.372014in}{2.075793in}}%
\pgfpathcurveto{\pgfqpoint{2.367896in}{2.075793in}}{\pgfqpoint{2.363946in}{2.074157in}}{\pgfqpoint{2.361034in}{2.071245in}}%
\pgfpathcurveto{\pgfqpoint{2.358122in}{2.068333in}}{\pgfqpoint{2.356486in}{2.064383in}}{\pgfqpoint{2.356486in}{2.060265in}}%
\pgfpathcurveto{\pgfqpoint{2.356486in}{2.056147in}}{\pgfqpoint{2.358122in}{2.052197in}}{\pgfqpoint{2.361034in}{2.049285in}}%
\pgfpathcurveto{\pgfqpoint{2.363946in}{2.046373in}}{\pgfqpoint{2.367896in}{2.044737in}}{\pgfqpoint{2.372014in}{2.044737in}}%
\pgfpathclose%
\pgfusepath{fill}%
\end{pgfscope}%
\begin{pgfscope}%
\pgfpathrectangle{\pgfqpoint{0.887500in}{0.275000in}}{\pgfqpoint{4.225000in}{4.225000in}}%
\pgfusepath{clip}%
\pgfsetbuttcap%
\pgfsetroundjoin%
\definecolor{currentfill}{rgb}{0.000000,0.000000,0.000000}%
\pgfsetfillcolor{currentfill}%
\pgfsetfillopacity{0.800000}%
\pgfsetlinewidth{0.000000pt}%
\definecolor{currentstroke}{rgb}{0.000000,0.000000,0.000000}%
\pgfsetstrokecolor{currentstroke}%
\pgfsetstrokeopacity{0.800000}%
\pgfsetdash{}{0pt}%
\pgfpathmoveto{\pgfqpoint{3.435058in}{2.064129in}}%
\pgfpathcurveto{\pgfqpoint{3.439176in}{2.064129in}}{\pgfqpoint{3.443126in}{2.065765in}}{\pgfqpoint{3.446038in}{2.068677in}}%
\pgfpathcurveto{\pgfqpoint{3.448950in}{2.071589in}}{\pgfqpoint{3.450586in}{2.075539in}}{\pgfqpoint{3.450586in}{2.079658in}}%
\pgfpathcurveto{\pgfqpoint{3.450586in}{2.083776in}}{\pgfqpoint{3.448950in}{2.087726in}}{\pgfqpoint{3.446038in}{2.090638in}}%
\pgfpathcurveto{\pgfqpoint{3.443126in}{2.093550in}}{\pgfqpoint{3.439176in}{2.095186in}}{\pgfqpoint{3.435058in}{2.095186in}}%
\pgfpathcurveto{\pgfqpoint{3.430940in}{2.095186in}}{\pgfqpoint{3.426990in}{2.093550in}}{\pgfqpoint{3.424078in}{2.090638in}}%
\pgfpathcurveto{\pgfqpoint{3.421166in}{2.087726in}}{\pgfqpoint{3.419530in}{2.083776in}}{\pgfqpoint{3.419530in}{2.079658in}}%
\pgfpathcurveto{\pgfqpoint{3.419530in}{2.075539in}}{\pgfqpoint{3.421166in}{2.071589in}}{\pgfqpoint{3.424078in}{2.068677in}}%
\pgfpathcurveto{\pgfqpoint{3.426990in}{2.065765in}}{\pgfqpoint{3.430940in}{2.064129in}}{\pgfqpoint{3.435058in}{2.064129in}}%
\pgfpathclose%
\pgfusepath{fill}%
\end{pgfscope}%
\begin{pgfscope}%
\pgfpathrectangle{\pgfqpoint{0.887500in}{0.275000in}}{\pgfqpoint{4.225000in}{4.225000in}}%
\pgfusepath{clip}%
\pgfsetbuttcap%
\pgfsetroundjoin%
\definecolor{currentfill}{rgb}{0.000000,0.000000,0.000000}%
\pgfsetfillcolor{currentfill}%
\pgfsetfillopacity{0.800000}%
\pgfsetlinewidth{0.000000pt}%
\definecolor{currentstroke}{rgb}{0.000000,0.000000,0.000000}%
\pgfsetstrokecolor{currentstroke}%
\pgfsetstrokeopacity{0.800000}%
\pgfsetdash{}{0pt}%
\pgfpathmoveto{\pgfqpoint{3.206651in}{1.803359in}}%
\pgfpathcurveto{\pgfqpoint{3.210770in}{1.803359in}}{\pgfqpoint{3.214720in}{1.804995in}}{\pgfqpoint{3.217632in}{1.807907in}}%
\pgfpathcurveto{\pgfqpoint{3.220544in}{1.810819in}}{\pgfqpoint{3.222180in}{1.814769in}}{\pgfqpoint{3.222180in}{1.818887in}}%
\pgfpathcurveto{\pgfqpoint{3.222180in}{1.823006in}}{\pgfqpoint{3.220544in}{1.826956in}}{\pgfqpoint{3.217632in}{1.829868in}}%
\pgfpathcurveto{\pgfqpoint{3.214720in}{1.832779in}}{\pgfqpoint{3.210770in}{1.834416in}}{\pgfqpoint{3.206651in}{1.834416in}}%
\pgfpathcurveto{\pgfqpoint{3.202533in}{1.834416in}}{\pgfqpoint{3.198583in}{1.832779in}}{\pgfqpoint{3.195671in}{1.829868in}}%
\pgfpathcurveto{\pgfqpoint{3.192759in}{1.826956in}}{\pgfqpoint{3.191123in}{1.823006in}}{\pgfqpoint{3.191123in}{1.818887in}}%
\pgfpathcurveto{\pgfqpoint{3.191123in}{1.814769in}}{\pgfqpoint{3.192759in}{1.810819in}}{\pgfqpoint{3.195671in}{1.807907in}}%
\pgfpathcurveto{\pgfqpoint{3.198583in}{1.804995in}}{\pgfqpoint{3.202533in}{1.803359in}}{\pgfqpoint{3.206651in}{1.803359in}}%
\pgfpathclose%
\pgfusepath{fill}%
\end{pgfscope}%
\begin{pgfscope}%
\pgfpathrectangle{\pgfqpoint{0.887500in}{0.275000in}}{\pgfqpoint{4.225000in}{4.225000in}}%
\pgfusepath{clip}%
\pgfsetbuttcap%
\pgfsetroundjoin%
\definecolor{currentfill}{rgb}{0.000000,0.000000,0.000000}%
\pgfsetfillcolor{currentfill}%
\pgfsetfillopacity{0.800000}%
\pgfsetlinewidth{0.000000pt}%
\definecolor{currentstroke}{rgb}{0.000000,0.000000,0.000000}%
\pgfsetstrokecolor{currentstroke}%
\pgfsetstrokeopacity{0.800000}%
\pgfsetdash{}{0pt}%
\pgfpathmoveto{\pgfqpoint{3.549307in}{2.199209in}}%
\pgfpathcurveto{\pgfqpoint{3.553425in}{2.199209in}}{\pgfqpoint{3.557375in}{2.200845in}}{\pgfqpoint{3.560287in}{2.203757in}}%
\pgfpathcurveto{\pgfqpoint{3.563199in}{2.206669in}}{\pgfqpoint{3.564835in}{2.210619in}}{\pgfqpoint{3.564835in}{2.214737in}}%
\pgfpathcurveto{\pgfqpoint{3.564835in}{2.218855in}}{\pgfqpoint{3.563199in}{2.222805in}}{\pgfqpoint{3.560287in}{2.225717in}}%
\pgfpathcurveto{\pgfqpoint{3.557375in}{2.228629in}}{\pgfqpoint{3.553425in}{2.230265in}}{\pgfqpoint{3.549307in}{2.230265in}}%
\pgfpathcurveto{\pgfqpoint{3.545189in}{2.230265in}}{\pgfqpoint{3.541239in}{2.228629in}}{\pgfqpoint{3.538327in}{2.225717in}}%
\pgfpathcurveto{\pgfqpoint{3.535415in}{2.222805in}}{\pgfqpoint{3.533779in}{2.218855in}}{\pgfqpoint{3.533779in}{2.214737in}}%
\pgfpathcurveto{\pgfqpoint{3.533779in}{2.210619in}}{\pgfqpoint{3.535415in}{2.206669in}}{\pgfqpoint{3.538327in}{2.203757in}}%
\pgfpathcurveto{\pgfqpoint{3.541239in}{2.200845in}}{\pgfqpoint{3.545189in}{2.199209in}}{\pgfqpoint{3.549307in}{2.199209in}}%
\pgfpathclose%
\pgfusepath{fill}%
\end{pgfscope}%
\begin{pgfscope}%
\pgfpathrectangle{\pgfqpoint{0.887500in}{0.275000in}}{\pgfqpoint{4.225000in}{4.225000in}}%
\pgfusepath{clip}%
\pgfsetbuttcap%
\pgfsetroundjoin%
\definecolor{currentfill}{rgb}{0.000000,0.000000,0.000000}%
\pgfsetfillcolor{currentfill}%
\pgfsetfillopacity{0.800000}%
\pgfsetlinewidth{0.000000pt}%
\definecolor{currentstroke}{rgb}{0.000000,0.000000,0.000000}%
\pgfsetstrokecolor{currentstroke}%
\pgfsetstrokeopacity{0.800000}%
\pgfsetdash{}{0pt}%
\pgfpathmoveto{\pgfqpoint{4.090042in}{1.918924in}}%
\pgfpathcurveto{\pgfqpoint{4.094160in}{1.918924in}}{\pgfqpoint{4.098110in}{1.920560in}}{\pgfqpoint{4.101022in}{1.923472in}}%
\pgfpathcurveto{\pgfqpoint{4.103934in}{1.926384in}}{\pgfqpoint{4.105570in}{1.930334in}}{\pgfqpoint{4.105570in}{1.934452in}}%
\pgfpathcurveto{\pgfqpoint{4.105570in}{1.938570in}}{\pgfqpoint{4.103934in}{1.942520in}}{\pgfqpoint{4.101022in}{1.945432in}}%
\pgfpathcurveto{\pgfqpoint{4.098110in}{1.948344in}}{\pgfqpoint{4.094160in}{1.949980in}}{\pgfqpoint{4.090042in}{1.949980in}}%
\pgfpathcurveto{\pgfqpoint{4.085923in}{1.949980in}}{\pgfqpoint{4.081973in}{1.948344in}}{\pgfqpoint{4.079061in}{1.945432in}}%
\pgfpathcurveto{\pgfqpoint{4.076149in}{1.942520in}}{\pgfqpoint{4.074513in}{1.938570in}}{\pgfqpoint{4.074513in}{1.934452in}}%
\pgfpathcurveto{\pgfqpoint{4.074513in}{1.930334in}}{\pgfqpoint{4.076149in}{1.926384in}}{\pgfqpoint{4.079061in}{1.923472in}}%
\pgfpathcurveto{\pgfqpoint{4.081973in}{1.920560in}}{\pgfqpoint{4.085923in}{1.918924in}}{\pgfqpoint{4.090042in}{1.918924in}}%
\pgfpathclose%
\pgfusepath{fill}%
\end{pgfscope}%
\begin{pgfscope}%
\pgfpathrectangle{\pgfqpoint{0.887500in}{0.275000in}}{\pgfqpoint{4.225000in}{4.225000in}}%
\pgfusepath{clip}%
\pgfsetbuttcap%
\pgfsetroundjoin%
\definecolor{currentfill}{rgb}{0.000000,0.000000,0.000000}%
\pgfsetfillcolor{currentfill}%
\pgfsetfillopacity{0.800000}%
\pgfsetlinewidth{0.000000pt}%
\definecolor{currentstroke}{rgb}{0.000000,0.000000,0.000000}%
\pgfsetstrokecolor{currentstroke}%
\pgfsetstrokeopacity{0.800000}%
\pgfsetdash{}{0pt}%
\pgfpathmoveto{\pgfqpoint{1.896919in}{2.129976in}}%
\pgfpathcurveto{\pgfqpoint{1.901037in}{2.129976in}}{\pgfqpoint{1.904987in}{2.131612in}}{\pgfqpoint{1.907899in}{2.134524in}}%
\pgfpathcurveto{\pgfqpoint{1.910811in}{2.137436in}}{\pgfqpoint{1.912448in}{2.141386in}}{\pgfqpoint{1.912448in}{2.145504in}}%
\pgfpathcurveto{\pgfqpoint{1.912448in}{2.149622in}}{\pgfqpoint{1.910811in}{2.153572in}}{\pgfqpoint{1.907899in}{2.156484in}}%
\pgfpathcurveto{\pgfqpoint{1.904987in}{2.159396in}}{\pgfqpoint{1.901037in}{2.161032in}}{\pgfqpoint{1.896919in}{2.161032in}}%
\pgfpathcurveto{\pgfqpoint{1.892801in}{2.161032in}}{\pgfqpoint{1.888851in}{2.159396in}}{\pgfqpoint{1.885939in}{2.156484in}}%
\pgfpathcurveto{\pgfqpoint{1.883027in}{2.153572in}}{\pgfqpoint{1.881391in}{2.149622in}}{\pgfqpoint{1.881391in}{2.145504in}}%
\pgfpathcurveto{\pgfqpoint{1.881391in}{2.141386in}}{\pgfqpoint{1.883027in}{2.137436in}}{\pgfqpoint{1.885939in}{2.134524in}}%
\pgfpathcurveto{\pgfqpoint{1.888851in}{2.131612in}}{\pgfqpoint{1.892801in}{2.129976in}}{\pgfqpoint{1.896919in}{2.129976in}}%
\pgfpathclose%
\pgfusepath{fill}%
\end{pgfscope}%
\begin{pgfscope}%
\pgfpathrectangle{\pgfqpoint{0.887500in}{0.275000in}}{\pgfqpoint{4.225000in}{4.225000in}}%
\pgfusepath{clip}%
\pgfsetbuttcap%
\pgfsetroundjoin%
\definecolor{currentfill}{rgb}{0.000000,0.000000,0.000000}%
\pgfsetfillcolor{currentfill}%
\pgfsetfillopacity{0.800000}%
\pgfsetlinewidth{0.000000pt}%
\definecolor{currentstroke}{rgb}{0.000000,0.000000,0.000000}%
\pgfsetstrokecolor{currentstroke}%
\pgfsetstrokeopacity{0.800000}%
\pgfsetdash{}{0pt}%
\pgfpathmoveto{\pgfqpoint{4.270340in}{1.825351in}}%
\pgfpathcurveto{\pgfqpoint{4.274459in}{1.825351in}}{\pgfqpoint{4.278409in}{1.826987in}}{\pgfqpoint{4.281321in}{1.829899in}}%
\pgfpathcurveto{\pgfqpoint{4.284233in}{1.832811in}}{\pgfqpoint{4.285869in}{1.836761in}}{\pgfqpoint{4.285869in}{1.840879in}}%
\pgfpathcurveto{\pgfqpoint{4.285869in}{1.844998in}}{\pgfqpoint{4.284233in}{1.848948in}}{\pgfqpoint{4.281321in}{1.851860in}}%
\pgfpathcurveto{\pgfqpoint{4.278409in}{1.854772in}}{\pgfqpoint{4.274459in}{1.856408in}}{\pgfqpoint{4.270340in}{1.856408in}}%
\pgfpathcurveto{\pgfqpoint{4.266222in}{1.856408in}}{\pgfqpoint{4.262272in}{1.854772in}}{\pgfqpoint{4.259360in}{1.851860in}}%
\pgfpathcurveto{\pgfqpoint{4.256448in}{1.848948in}}{\pgfqpoint{4.254812in}{1.844998in}}{\pgfqpoint{4.254812in}{1.840879in}}%
\pgfpathcurveto{\pgfqpoint{4.254812in}{1.836761in}}{\pgfqpoint{4.256448in}{1.832811in}}{\pgfqpoint{4.259360in}{1.829899in}}%
\pgfpathcurveto{\pgfqpoint{4.262272in}{1.826987in}}{\pgfqpoint{4.266222in}{1.825351in}}{\pgfqpoint{4.270340in}{1.825351in}}%
\pgfpathclose%
\pgfusepath{fill}%
\end{pgfscope}%
\begin{pgfscope}%
\pgfpathrectangle{\pgfqpoint{0.887500in}{0.275000in}}{\pgfqpoint{4.225000in}{4.225000in}}%
\pgfusepath{clip}%
\pgfsetbuttcap%
\pgfsetroundjoin%
\definecolor{currentfill}{rgb}{0.000000,0.000000,0.000000}%
\pgfsetfillcolor{currentfill}%
\pgfsetfillopacity{0.800000}%
\pgfsetlinewidth{0.000000pt}%
\definecolor{currentstroke}{rgb}{0.000000,0.000000,0.000000}%
\pgfsetstrokecolor{currentstroke}%
\pgfsetstrokeopacity{0.800000}%
\pgfsetdash{}{0pt}%
\pgfpathmoveto{\pgfqpoint{3.386948in}{1.722693in}}%
\pgfpathcurveto{\pgfqpoint{3.391066in}{1.722693in}}{\pgfqpoint{3.395016in}{1.724329in}}{\pgfqpoint{3.397928in}{1.727241in}}%
\pgfpathcurveto{\pgfqpoint{3.400840in}{1.730153in}}{\pgfqpoint{3.402476in}{1.734103in}}{\pgfqpoint{3.402476in}{1.738221in}}%
\pgfpathcurveto{\pgfqpoint{3.402476in}{1.742339in}}{\pgfqpoint{3.400840in}{1.746289in}}{\pgfqpoint{3.397928in}{1.749201in}}%
\pgfpathcurveto{\pgfqpoint{3.395016in}{1.752113in}}{\pgfqpoint{3.391066in}{1.753749in}}{\pgfqpoint{3.386948in}{1.753749in}}%
\pgfpathcurveto{\pgfqpoint{3.382830in}{1.753749in}}{\pgfqpoint{3.378880in}{1.752113in}}{\pgfqpoint{3.375968in}{1.749201in}}%
\pgfpathcurveto{\pgfqpoint{3.373056in}{1.746289in}}{\pgfqpoint{3.371419in}{1.742339in}}{\pgfqpoint{3.371419in}{1.738221in}}%
\pgfpathcurveto{\pgfqpoint{3.371419in}{1.734103in}}{\pgfqpoint{3.373056in}{1.730153in}}{\pgfqpoint{3.375968in}{1.727241in}}%
\pgfpathcurveto{\pgfqpoint{3.378880in}{1.724329in}}{\pgfqpoint{3.382830in}{1.722693in}}{\pgfqpoint{3.386948in}{1.722693in}}%
\pgfpathclose%
\pgfusepath{fill}%
\end{pgfscope}%
\begin{pgfscope}%
\pgfpathrectangle{\pgfqpoint{0.887500in}{0.275000in}}{\pgfqpoint{4.225000in}{4.225000in}}%
\pgfusepath{clip}%
\pgfsetbuttcap%
\pgfsetroundjoin%
\definecolor{currentfill}{rgb}{0.000000,0.000000,0.000000}%
\pgfsetfillcolor{currentfill}%
\pgfsetfillopacity{0.800000}%
\pgfsetlinewidth{0.000000pt}%
\definecolor{currentstroke}{rgb}{0.000000,0.000000,0.000000}%
\pgfsetstrokecolor{currentstroke}%
\pgfsetstrokeopacity{0.800000}%
\pgfsetdash{}{0pt}%
\pgfpathmoveto{\pgfqpoint{3.501245in}{1.857553in}}%
\pgfpathcurveto{\pgfqpoint{3.505364in}{1.857553in}}{\pgfqpoint{3.509314in}{1.859189in}}{\pgfqpoint{3.512226in}{1.862101in}}%
\pgfpathcurveto{\pgfqpoint{3.515138in}{1.865013in}}{\pgfqpoint{3.516774in}{1.868963in}}{\pgfqpoint{3.516774in}{1.873081in}}%
\pgfpathcurveto{\pgfqpoint{3.516774in}{1.877199in}}{\pgfqpoint{3.515138in}{1.881149in}}{\pgfqpoint{3.512226in}{1.884061in}}%
\pgfpathcurveto{\pgfqpoint{3.509314in}{1.886973in}}{\pgfqpoint{3.505364in}{1.888610in}}{\pgfqpoint{3.501245in}{1.888610in}}%
\pgfpathcurveto{\pgfqpoint{3.497127in}{1.888610in}}{\pgfqpoint{3.493177in}{1.886973in}}{\pgfqpoint{3.490265in}{1.884061in}}%
\pgfpathcurveto{\pgfqpoint{3.487353in}{1.881149in}}{\pgfqpoint{3.485717in}{1.877199in}}{\pgfqpoint{3.485717in}{1.873081in}}%
\pgfpathcurveto{\pgfqpoint{3.485717in}{1.868963in}}{\pgfqpoint{3.487353in}{1.865013in}}{\pgfqpoint{3.490265in}{1.862101in}}%
\pgfpathcurveto{\pgfqpoint{3.493177in}{1.859189in}}{\pgfqpoint{3.497127in}{1.857553in}}{\pgfqpoint{3.501245in}{1.857553in}}%
\pgfpathclose%
\pgfusepath{fill}%
\end{pgfscope}%
\begin{pgfscope}%
\pgfpathrectangle{\pgfqpoint{0.887500in}{0.275000in}}{\pgfqpoint{4.225000in}{4.225000in}}%
\pgfusepath{clip}%
\pgfsetbuttcap%
\pgfsetroundjoin%
\definecolor{currentfill}{rgb}{0.000000,0.000000,0.000000}%
\pgfsetfillcolor{currentfill}%
\pgfsetfillopacity{0.800000}%
\pgfsetlinewidth{0.000000pt}%
\definecolor{currentstroke}{rgb}{0.000000,0.000000,0.000000}%
\pgfsetstrokecolor{currentstroke}%
\pgfsetstrokeopacity{0.800000}%
\pgfsetdash{}{0pt}%
\pgfpathmoveto{\pgfqpoint{2.551693in}{1.980955in}}%
\pgfpathcurveto{\pgfqpoint{2.555811in}{1.980955in}}{\pgfqpoint{2.559761in}{1.982592in}}{\pgfqpoint{2.562673in}{1.985504in}}%
\pgfpathcurveto{\pgfqpoint{2.565585in}{1.988416in}}{\pgfqpoint{2.567221in}{1.992366in}}{\pgfqpoint{2.567221in}{1.996484in}}%
\pgfpathcurveto{\pgfqpoint{2.567221in}{2.000602in}}{\pgfqpoint{2.565585in}{2.004552in}}{\pgfqpoint{2.562673in}{2.007464in}}%
\pgfpathcurveto{\pgfqpoint{2.559761in}{2.010376in}}{\pgfqpoint{2.555811in}{2.012012in}}{\pgfqpoint{2.551693in}{2.012012in}}%
\pgfpathcurveto{\pgfqpoint{2.547575in}{2.012012in}}{\pgfqpoint{2.543625in}{2.010376in}}{\pgfqpoint{2.540713in}{2.007464in}}%
\pgfpathcurveto{\pgfqpoint{2.537801in}{2.004552in}}{\pgfqpoint{2.536165in}{2.000602in}}{\pgfqpoint{2.536165in}{1.996484in}}%
\pgfpathcurveto{\pgfqpoint{2.536165in}{1.992366in}}{\pgfqpoint{2.537801in}{1.988416in}}{\pgfqpoint{2.540713in}{1.985504in}}%
\pgfpathcurveto{\pgfqpoint{2.543625in}{1.982592in}}{\pgfqpoint{2.547575in}{1.980955in}}{\pgfqpoint{2.551693in}{1.980955in}}%
\pgfpathclose%
\pgfusepath{fill}%
\end{pgfscope}%
\begin{pgfscope}%
\pgfpathrectangle{\pgfqpoint{0.887500in}{0.275000in}}{\pgfqpoint{4.225000in}{4.225000in}}%
\pgfusepath{clip}%
\pgfsetbuttcap%
\pgfsetroundjoin%
\definecolor{currentfill}{rgb}{0.000000,0.000000,0.000000}%
\pgfsetfillcolor{currentfill}%
\pgfsetfillopacity{0.800000}%
\pgfsetlinewidth{0.000000pt}%
\definecolor{currentstroke}{rgb}{0.000000,0.000000,0.000000}%
\pgfsetstrokecolor{currentstroke}%
\pgfsetstrokeopacity{0.800000}%
\pgfsetdash{}{0pt}%
\pgfpathmoveto{\pgfqpoint{3.567302in}{1.635141in}}%
\pgfpathcurveto{\pgfqpoint{3.571420in}{1.635141in}}{\pgfqpoint{3.575370in}{1.636777in}}{\pgfqpoint{3.578282in}{1.639689in}}%
\pgfpathcurveto{\pgfqpoint{3.581194in}{1.642601in}}{\pgfqpoint{3.582830in}{1.646551in}}{\pgfqpoint{3.582830in}{1.650669in}}%
\pgfpathcurveto{\pgfqpoint{3.582830in}{1.654787in}}{\pgfqpoint{3.581194in}{1.658737in}}{\pgfqpoint{3.578282in}{1.661649in}}%
\pgfpathcurveto{\pgfqpoint{3.575370in}{1.664561in}}{\pgfqpoint{3.571420in}{1.666197in}}{\pgfqpoint{3.567302in}{1.666197in}}%
\pgfpathcurveto{\pgfqpoint{3.563184in}{1.666197in}}{\pgfqpoint{3.559234in}{1.664561in}}{\pgfqpoint{3.556322in}{1.661649in}}%
\pgfpathcurveto{\pgfqpoint{3.553410in}{1.658737in}}{\pgfqpoint{3.551774in}{1.654787in}}{\pgfqpoint{3.551774in}{1.650669in}}%
\pgfpathcurveto{\pgfqpoint{3.551774in}{1.646551in}}{\pgfqpoint{3.553410in}{1.642601in}}{\pgfqpoint{3.556322in}{1.639689in}}%
\pgfpathcurveto{\pgfqpoint{3.559234in}{1.636777in}}{\pgfqpoint{3.563184in}{1.635141in}}{\pgfqpoint{3.567302in}{1.635141in}}%
\pgfpathclose%
\pgfusepath{fill}%
\end{pgfscope}%
\begin{pgfscope}%
\pgfpathrectangle{\pgfqpoint{0.887500in}{0.275000in}}{\pgfqpoint{4.225000in}{4.225000in}}%
\pgfusepath{clip}%
\pgfsetbuttcap%
\pgfsetroundjoin%
\definecolor{currentfill}{rgb}{0.000000,0.000000,0.000000}%
\pgfsetfillcolor{currentfill}%
\pgfsetfillopacity{0.800000}%
\pgfsetlinewidth{0.000000pt}%
\definecolor{currentstroke}{rgb}{0.000000,0.000000,0.000000}%
\pgfsetstrokecolor{currentstroke}%
\pgfsetstrokeopacity{0.800000}%
\pgfsetdash{}{0pt}%
\pgfpathmoveto{\pgfqpoint{3.910294in}{2.036497in}}%
\pgfpathcurveto{\pgfqpoint{3.914412in}{2.036497in}}{\pgfqpoint{3.918362in}{2.038133in}}{\pgfqpoint{3.921274in}{2.041045in}}%
\pgfpathcurveto{\pgfqpoint{3.924186in}{2.043957in}}{\pgfqpoint{3.925822in}{2.047907in}}{\pgfqpoint{3.925822in}{2.052025in}}%
\pgfpathcurveto{\pgfqpoint{3.925822in}{2.056143in}}{\pgfqpoint{3.924186in}{2.060093in}}{\pgfqpoint{3.921274in}{2.063005in}}%
\pgfpathcurveto{\pgfqpoint{3.918362in}{2.065917in}}{\pgfqpoint{3.914412in}{2.067553in}}{\pgfqpoint{3.910294in}{2.067553in}}%
\pgfpathcurveto{\pgfqpoint{3.906175in}{2.067553in}}{\pgfqpoint{3.902225in}{2.065917in}}{\pgfqpoint{3.899314in}{2.063005in}}%
\pgfpathcurveto{\pgfqpoint{3.896402in}{2.060093in}}{\pgfqpoint{3.894765in}{2.056143in}}{\pgfqpoint{3.894765in}{2.052025in}}%
\pgfpathcurveto{\pgfqpoint{3.894765in}{2.047907in}}{\pgfqpoint{3.896402in}{2.043957in}}{\pgfqpoint{3.899314in}{2.041045in}}%
\pgfpathcurveto{\pgfqpoint{3.902225in}{2.038133in}}{\pgfqpoint{3.906175in}{2.036497in}}{\pgfqpoint{3.910294in}{2.036497in}}%
\pgfpathclose%
\pgfusepath{fill}%
\end{pgfscope}%
\begin{pgfscope}%
\pgfpathrectangle{\pgfqpoint{0.887500in}{0.275000in}}{\pgfqpoint{4.225000in}{4.225000in}}%
\pgfusepath{clip}%
\pgfsetbuttcap%
\pgfsetroundjoin%
\definecolor{currentfill}{rgb}{0.000000,0.000000,0.000000}%
\pgfsetfillcolor{currentfill}%
\pgfsetfillopacity{0.800000}%
\pgfsetlinewidth{0.000000pt}%
\definecolor{currentstroke}{rgb}{0.000000,0.000000,0.000000}%
\pgfsetstrokecolor{currentstroke}%
\pgfsetstrokeopacity{0.800000}%
\pgfsetdash{}{0pt}%
\pgfpathmoveto{\pgfqpoint{3.730176in}{2.143914in}}%
\pgfpathcurveto{\pgfqpoint{3.734295in}{2.143914in}}{\pgfqpoint{3.738245in}{2.145550in}}{\pgfqpoint{3.741157in}{2.148462in}}%
\pgfpathcurveto{\pgfqpoint{3.744069in}{2.151374in}}{\pgfqpoint{3.745705in}{2.155324in}}{\pgfqpoint{3.745705in}{2.159442in}}%
\pgfpathcurveto{\pgfqpoint{3.745705in}{2.163560in}}{\pgfqpoint{3.744069in}{2.167510in}}{\pgfqpoint{3.741157in}{2.170422in}}%
\pgfpathcurveto{\pgfqpoint{3.738245in}{2.173334in}}{\pgfqpoint{3.734295in}{2.174970in}}{\pgfqpoint{3.730176in}{2.174970in}}%
\pgfpathcurveto{\pgfqpoint{3.726058in}{2.174970in}}{\pgfqpoint{3.722108in}{2.173334in}}{\pgfqpoint{3.719196in}{2.170422in}}%
\pgfpathcurveto{\pgfqpoint{3.716284in}{2.167510in}}{\pgfqpoint{3.714648in}{2.163560in}}{\pgfqpoint{3.714648in}{2.159442in}}%
\pgfpathcurveto{\pgfqpoint{3.714648in}{2.155324in}}{\pgfqpoint{3.716284in}{2.151374in}}{\pgfqpoint{3.719196in}{2.148462in}}%
\pgfpathcurveto{\pgfqpoint{3.722108in}{2.145550in}}{\pgfqpoint{3.726058in}{2.143914in}}{\pgfqpoint{3.730176in}{2.143914in}}%
\pgfpathclose%
\pgfusepath{fill}%
\end{pgfscope}%
\begin{pgfscope}%
\pgfpathrectangle{\pgfqpoint{0.887500in}{0.275000in}}{\pgfqpoint{4.225000in}{4.225000in}}%
\pgfusepath{clip}%
\pgfsetbuttcap%
\pgfsetroundjoin%
\definecolor{currentfill}{rgb}{0.000000,0.000000,0.000000}%
\pgfsetfillcolor{currentfill}%
\pgfsetfillopacity{0.800000}%
\pgfsetlinewidth{0.000000pt}%
\definecolor{currentstroke}{rgb}{0.000000,0.000000,0.000000}%
\pgfsetstrokecolor{currentstroke}%
\pgfsetstrokeopacity{0.800000}%
\pgfsetdash{}{0pt}%
\pgfpathmoveto{\pgfqpoint{3.615893in}{2.015757in}}%
\pgfpathcurveto{\pgfqpoint{3.620011in}{2.015757in}}{\pgfqpoint{3.623961in}{2.017393in}}{\pgfqpoint{3.626873in}{2.020305in}}%
\pgfpathcurveto{\pgfqpoint{3.629785in}{2.023217in}}{\pgfqpoint{3.631422in}{2.027167in}}{\pgfqpoint{3.631422in}{2.031285in}}%
\pgfpathcurveto{\pgfqpoint{3.631422in}{2.035403in}}{\pgfqpoint{3.629785in}{2.039353in}}{\pgfqpoint{3.626873in}{2.042265in}}%
\pgfpathcurveto{\pgfqpoint{3.623961in}{2.045177in}}{\pgfqpoint{3.620011in}{2.046813in}}{\pgfqpoint{3.615893in}{2.046813in}}%
\pgfpathcurveto{\pgfqpoint{3.611775in}{2.046813in}}{\pgfqpoint{3.607825in}{2.045177in}}{\pgfqpoint{3.604913in}{2.042265in}}%
\pgfpathcurveto{\pgfqpoint{3.602001in}{2.039353in}}{\pgfqpoint{3.600365in}{2.035403in}}{\pgfqpoint{3.600365in}{2.031285in}}%
\pgfpathcurveto{\pgfqpoint{3.600365in}{2.027167in}}{\pgfqpoint{3.602001in}{2.023217in}}{\pgfqpoint{3.604913in}{2.020305in}}%
\pgfpathcurveto{\pgfqpoint{3.607825in}{2.017393in}}{\pgfqpoint{3.611775in}{2.015757in}}{\pgfqpoint{3.615893in}{2.015757in}}%
\pgfpathclose%
\pgfusepath{fill}%
\end{pgfscope}%
\begin{pgfscope}%
\pgfpathrectangle{\pgfqpoint{0.887500in}{0.275000in}}{\pgfqpoint{4.225000in}{4.225000in}}%
\pgfusepath{clip}%
\pgfsetbuttcap%
\pgfsetroundjoin%
\definecolor{currentfill}{rgb}{0.000000,0.000000,0.000000}%
\pgfsetfillcolor{currentfill}%
\pgfsetfillopacity{0.800000}%
\pgfsetlinewidth{0.000000pt}%
\definecolor{currentstroke}{rgb}{0.000000,0.000000,0.000000}%
\pgfsetstrokecolor{currentstroke}%
\pgfsetstrokeopacity{0.800000}%
\pgfsetdash{}{0pt}%
\pgfpathmoveto{\pgfqpoint{2.076222in}{2.069763in}}%
\pgfpathcurveto{\pgfqpoint{2.080340in}{2.069763in}}{\pgfqpoint{2.084290in}{2.071399in}}{\pgfqpoint{2.087202in}{2.074311in}}%
\pgfpathcurveto{\pgfqpoint{2.090114in}{2.077223in}}{\pgfqpoint{2.091750in}{2.081173in}}{\pgfqpoint{2.091750in}{2.085291in}}%
\pgfpathcurveto{\pgfqpoint{2.091750in}{2.089409in}}{\pgfqpoint{2.090114in}{2.093359in}}{\pgfqpoint{2.087202in}{2.096271in}}%
\pgfpathcurveto{\pgfqpoint{2.084290in}{2.099183in}}{\pgfqpoint{2.080340in}{2.100819in}}{\pgfqpoint{2.076222in}{2.100819in}}%
\pgfpathcurveto{\pgfqpoint{2.072104in}{2.100819in}}{\pgfqpoint{2.068154in}{2.099183in}}{\pgfqpoint{2.065242in}{2.096271in}}%
\pgfpathcurveto{\pgfqpoint{2.062330in}{2.093359in}}{\pgfqpoint{2.060693in}{2.089409in}}{\pgfqpoint{2.060693in}{2.085291in}}%
\pgfpathcurveto{\pgfqpoint{2.060693in}{2.081173in}}{\pgfqpoint{2.062330in}{2.077223in}}{\pgfqpoint{2.065242in}{2.074311in}}%
\pgfpathcurveto{\pgfqpoint{2.068154in}{2.071399in}}{\pgfqpoint{2.072104in}{2.069763in}}{\pgfqpoint{2.076222in}{2.069763in}}%
\pgfpathclose%
\pgfusepath{fill}%
\end{pgfscope}%
\begin{pgfscope}%
\pgfpathrectangle{\pgfqpoint{0.887500in}{0.275000in}}{\pgfqpoint{4.225000in}{4.225000in}}%
\pgfusepath{clip}%
\pgfsetbuttcap%
\pgfsetroundjoin%
\definecolor{currentfill}{rgb}{0.000000,0.000000,0.000000}%
\pgfsetfillcolor{currentfill}%
\pgfsetfillopacity{0.800000}%
\pgfsetlinewidth{0.000000pt}%
\definecolor{currentstroke}{rgb}{0.000000,0.000000,0.000000}%
\pgfsetstrokecolor{currentstroke}%
\pgfsetstrokeopacity{0.800000}%
\pgfsetdash{}{0pt}%
\pgfpathmoveto{\pgfqpoint{2.731703in}{1.913459in}}%
\pgfpathcurveto{\pgfqpoint{2.735821in}{1.913459in}}{\pgfqpoint{2.739772in}{1.915095in}}{\pgfqpoint{2.742683in}{1.918007in}}%
\pgfpathcurveto{\pgfqpoint{2.745595in}{1.920919in}}{\pgfqpoint{2.747232in}{1.924869in}}{\pgfqpoint{2.747232in}{1.928987in}}%
\pgfpathcurveto{\pgfqpoint{2.747232in}{1.933106in}}{\pgfqpoint{2.745595in}{1.937056in}}{\pgfqpoint{2.742683in}{1.939968in}}%
\pgfpathcurveto{\pgfqpoint{2.739772in}{1.942880in}}{\pgfqpoint{2.735821in}{1.944516in}}{\pgfqpoint{2.731703in}{1.944516in}}%
\pgfpathcurveto{\pgfqpoint{2.727585in}{1.944516in}}{\pgfqpoint{2.723635in}{1.942880in}}{\pgfqpoint{2.720723in}{1.939968in}}%
\pgfpathcurveto{\pgfqpoint{2.717811in}{1.937056in}}{\pgfqpoint{2.716175in}{1.933106in}}{\pgfqpoint{2.716175in}{1.928987in}}%
\pgfpathcurveto{\pgfqpoint{2.716175in}{1.924869in}}{\pgfqpoint{2.717811in}{1.920919in}}{\pgfqpoint{2.720723in}{1.918007in}}%
\pgfpathcurveto{\pgfqpoint{2.723635in}{1.915095in}}{\pgfqpoint{2.727585in}{1.913459in}}{\pgfqpoint{2.731703in}{1.913459in}}%
\pgfpathclose%
\pgfusepath{fill}%
\end{pgfscope}%
\begin{pgfscope}%
\pgfpathrectangle{\pgfqpoint{0.887500in}{0.275000in}}{\pgfqpoint{4.225000in}{4.225000in}}%
\pgfusepath{clip}%
\pgfsetbuttcap%
\pgfsetroundjoin%
\definecolor{currentfill}{rgb}{0.000000,0.000000,0.000000}%
\pgfsetfillcolor{currentfill}%
\pgfsetfillopacity{0.800000}%
\pgfsetlinewidth{0.000000pt}%
\definecolor{currentstroke}{rgb}{0.000000,0.000000,0.000000}%
\pgfsetstrokecolor{currentstroke}%
\pgfsetstrokeopacity{0.800000}%
\pgfsetdash{}{0pt}%
\pgfpathmoveto{\pgfqpoint{3.682363in}{1.815071in}}%
\pgfpathcurveto{\pgfqpoint{3.686481in}{1.815071in}}{\pgfqpoint{3.690431in}{1.816707in}}{\pgfqpoint{3.693343in}{1.819619in}}%
\pgfpathcurveto{\pgfqpoint{3.696255in}{1.822531in}}{\pgfqpoint{3.697891in}{1.826481in}}{\pgfqpoint{3.697891in}{1.830599in}}%
\pgfpathcurveto{\pgfqpoint{3.697891in}{1.834717in}}{\pgfqpoint{3.696255in}{1.838667in}}{\pgfqpoint{3.693343in}{1.841579in}}%
\pgfpathcurveto{\pgfqpoint{3.690431in}{1.844491in}}{\pgfqpoint{3.686481in}{1.846127in}}{\pgfqpoint{3.682363in}{1.846127in}}%
\pgfpathcurveto{\pgfqpoint{3.678245in}{1.846127in}}{\pgfqpoint{3.674295in}{1.844491in}}{\pgfqpoint{3.671383in}{1.841579in}}%
\pgfpathcurveto{\pgfqpoint{3.668471in}{1.838667in}}{\pgfqpoint{3.666835in}{1.834717in}}{\pgfqpoint{3.666835in}{1.830599in}}%
\pgfpathcurveto{\pgfqpoint{3.666835in}{1.826481in}}{\pgfqpoint{3.668471in}{1.822531in}}{\pgfqpoint{3.671383in}{1.819619in}}%
\pgfpathcurveto{\pgfqpoint{3.674295in}{1.816707in}}{\pgfqpoint{3.678245in}{1.815071in}}{\pgfqpoint{3.682363in}{1.815071in}}%
\pgfpathclose%
\pgfusepath{fill}%
\end{pgfscope}%
\begin{pgfscope}%
\pgfpathrectangle{\pgfqpoint{0.887500in}{0.275000in}}{\pgfqpoint{4.225000in}{4.225000in}}%
\pgfusepath{clip}%
\pgfsetbuttcap%
\pgfsetroundjoin%
\definecolor{currentfill}{rgb}{0.000000,0.000000,0.000000}%
\pgfsetfillcolor{currentfill}%
\pgfsetfillopacity{0.800000}%
\pgfsetlinewidth{0.000000pt}%
\definecolor{currentstroke}{rgb}{0.000000,0.000000,0.000000}%
\pgfsetstrokecolor{currentstroke}%
\pgfsetstrokeopacity{0.800000}%
\pgfsetdash{}{0pt}%
\pgfpathmoveto{\pgfqpoint{2.911996in}{1.842087in}}%
\pgfpathcurveto{\pgfqpoint{2.916115in}{1.842087in}}{\pgfqpoint{2.920065in}{1.843723in}}{\pgfqpoint{2.922977in}{1.846635in}}%
\pgfpathcurveto{\pgfqpoint{2.925889in}{1.849547in}}{\pgfqpoint{2.927525in}{1.853497in}}{\pgfqpoint{2.927525in}{1.857615in}}%
\pgfpathcurveto{\pgfqpoint{2.927525in}{1.861733in}}{\pgfqpoint{2.925889in}{1.865683in}}{\pgfqpoint{2.922977in}{1.868595in}}%
\pgfpathcurveto{\pgfqpoint{2.920065in}{1.871507in}}{\pgfqpoint{2.916115in}{1.873143in}}{\pgfqpoint{2.911996in}{1.873143in}}%
\pgfpathcurveto{\pgfqpoint{2.907878in}{1.873143in}}{\pgfqpoint{2.903928in}{1.871507in}}{\pgfqpoint{2.901016in}{1.868595in}}%
\pgfpathcurveto{\pgfqpoint{2.898104in}{1.865683in}}{\pgfqpoint{2.896468in}{1.861733in}}{\pgfqpoint{2.896468in}{1.857615in}}%
\pgfpathcurveto{\pgfqpoint{2.896468in}{1.853497in}}{\pgfqpoint{2.898104in}{1.849547in}}{\pgfqpoint{2.901016in}{1.846635in}}%
\pgfpathcurveto{\pgfqpoint{2.903928in}{1.843723in}}{\pgfqpoint{2.907878in}{1.842087in}}{\pgfqpoint{2.911996in}{1.842087in}}%
\pgfpathclose%
\pgfusepath{fill}%
\end{pgfscope}%
\begin{pgfscope}%
\pgfpathrectangle{\pgfqpoint{0.887500in}{0.275000in}}{\pgfqpoint{4.225000in}{4.225000in}}%
\pgfusepath{clip}%
\pgfsetbuttcap%
\pgfsetroundjoin%
\definecolor{currentfill}{rgb}{0.000000,0.000000,0.000000}%
\pgfsetfillcolor{currentfill}%
\pgfsetfillopacity{0.800000}%
\pgfsetlinewidth{0.000000pt}%
\definecolor{currentstroke}{rgb}{0.000000,0.000000,0.000000}%
\pgfsetstrokecolor{currentstroke}%
\pgfsetstrokeopacity{0.800000}%
\pgfsetdash{}{0pt}%
\pgfpathmoveto{\pgfqpoint{3.796951in}{1.955471in}}%
\pgfpathcurveto{\pgfqpoint{3.801070in}{1.955471in}}{\pgfqpoint{3.805020in}{1.957107in}}{\pgfqpoint{3.807932in}{1.960019in}}%
\pgfpathcurveto{\pgfqpoint{3.810844in}{1.962931in}}{\pgfqpoint{3.812480in}{1.966881in}}{\pgfqpoint{3.812480in}{1.970999in}}%
\pgfpathcurveto{\pgfqpoint{3.812480in}{1.975117in}}{\pgfqpoint{3.810844in}{1.979067in}}{\pgfqpoint{3.807932in}{1.981979in}}%
\pgfpathcurveto{\pgfqpoint{3.805020in}{1.984891in}}{\pgfqpoint{3.801070in}{1.986527in}}{\pgfqpoint{3.796951in}{1.986527in}}%
\pgfpathcurveto{\pgfqpoint{3.792833in}{1.986527in}}{\pgfqpoint{3.788883in}{1.984891in}}{\pgfqpoint{3.785971in}{1.981979in}}%
\pgfpathcurveto{\pgfqpoint{3.783059in}{1.979067in}}{\pgfqpoint{3.781423in}{1.975117in}}{\pgfqpoint{3.781423in}{1.970999in}}%
\pgfpathcurveto{\pgfqpoint{3.781423in}{1.966881in}}{\pgfqpoint{3.783059in}{1.962931in}}{\pgfqpoint{3.785971in}{1.960019in}}%
\pgfpathcurveto{\pgfqpoint{3.788883in}{1.957107in}}{\pgfqpoint{3.792833in}{1.955471in}}{\pgfqpoint{3.796951in}{1.955471in}}%
\pgfpathclose%
\pgfusepath{fill}%
\end{pgfscope}%
\begin{pgfscope}%
\pgfpathrectangle{\pgfqpoint{0.887500in}{0.275000in}}{\pgfqpoint{4.225000in}{4.225000in}}%
\pgfusepath{clip}%
\pgfsetbuttcap%
\pgfsetroundjoin%
\definecolor{currentfill}{rgb}{0.000000,0.000000,0.000000}%
\pgfsetfillcolor{currentfill}%
\pgfsetfillopacity{0.800000}%
\pgfsetlinewidth{0.000000pt}%
\definecolor{currentstroke}{rgb}{0.000000,0.000000,0.000000}%
\pgfsetstrokecolor{currentstroke}%
\pgfsetstrokeopacity{0.800000}%
\pgfsetdash{}{0pt}%
\pgfpathmoveto{\pgfqpoint{2.255889in}{2.007886in}}%
\pgfpathcurveto{\pgfqpoint{2.260007in}{2.007886in}}{\pgfqpoint{2.263957in}{2.009522in}}{\pgfqpoint{2.266869in}{2.012434in}}%
\pgfpathcurveto{\pgfqpoint{2.269781in}{2.015346in}}{\pgfqpoint{2.271417in}{2.019296in}}{\pgfqpoint{2.271417in}{2.023414in}}%
\pgfpathcurveto{\pgfqpoint{2.271417in}{2.027532in}}{\pgfqpoint{2.269781in}{2.031482in}}{\pgfqpoint{2.266869in}{2.034394in}}%
\pgfpathcurveto{\pgfqpoint{2.263957in}{2.037306in}}{\pgfqpoint{2.260007in}{2.038942in}}{\pgfqpoint{2.255889in}{2.038942in}}%
\pgfpathcurveto{\pgfqpoint{2.251771in}{2.038942in}}{\pgfqpoint{2.247821in}{2.037306in}}{\pgfqpoint{2.244909in}{2.034394in}}%
\pgfpathcurveto{\pgfqpoint{2.241997in}{2.031482in}}{\pgfqpoint{2.240361in}{2.027532in}}{\pgfqpoint{2.240361in}{2.023414in}}%
\pgfpathcurveto{\pgfqpoint{2.240361in}{2.019296in}}{\pgfqpoint{2.241997in}{2.015346in}}{\pgfqpoint{2.244909in}{2.012434in}}%
\pgfpathcurveto{\pgfqpoint{2.247821in}{2.009522in}}{\pgfqpoint{2.251771in}{2.007886in}}{\pgfqpoint{2.255889in}{2.007886in}}%
\pgfpathclose%
\pgfusepath{fill}%
\end{pgfscope}%
\begin{pgfscope}%
\pgfpathrectangle{\pgfqpoint{0.887500in}{0.275000in}}{\pgfqpoint{4.225000in}{4.225000in}}%
\pgfusepath{clip}%
\pgfsetbuttcap%
\pgfsetroundjoin%
\definecolor{currentfill}{rgb}{0.000000,0.000000,0.000000}%
\pgfsetfillcolor{currentfill}%
\pgfsetfillopacity{0.800000}%
\pgfsetlinewidth{0.000000pt}%
\definecolor{currentstroke}{rgb}{0.000000,0.000000,0.000000}%
\pgfsetstrokecolor{currentstroke}%
\pgfsetstrokeopacity{0.800000}%
\pgfsetdash{}{0pt}%
\pgfpathmoveto{\pgfqpoint{3.977560in}{1.861729in}}%
\pgfpathcurveto{\pgfqpoint{3.981678in}{1.861729in}}{\pgfqpoint{3.985628in}{1.863365in}}{\pgfqpoint{3.988540in}{1.866277in}}%
\pgfpathcurveto{\pgfqpoint{3.991452in}{1.869189in}}{\pgfqpoint{3.993088in}{1.873139in}}{\pgfqpoint{3.993088in}{1.877257in}}%
\pgfpathcurveto{\pgfqpoint{3.993088in}{1.881375in}}{\pgfqpoint{3.991452in}{1.885325in}}{\pgfqpoint{3.988540in}{1.888237in}}%
\pgfpathcurveto{\pgfqpoint{3.985628in}{1.891149in}}{\pgfqpoint{3.981678in}{1.892785in}}{\pgfqpoint{3.977560in}{1.892785in}}%
\pgfpathcurveto{\pgfqpoint{3.973442in}{1.892785in}}{\pgfqpoint{3.969492in}{1.891149in}}{\pgfqpoint{3.966580in}{1.888237in}}%
\pgfpathcurveto{\pgfqpoint{3.963668in}{1.885325in}}{\pgfqpoint{3.962032in}{1.881375in}}{\pgfqpoint{3.962032in}{1.877257in}}%
\pgfpathcurveto{\pgfqpoint{3.962032in}{1.873139in}}{\pgfqpoint{3.963668in}{1.869189in}}{\pgfqpoint{3.966580in}{1.866277in}}%
\pgfpathcurveto{\pgfqpoint{3.969492in}{1.863365in}}{\pgfqpoint{3.973442in}{1.861729in}}{\pgfqpoint{3.977560in}{1.861729in}}%
\pgfpathclose%
\pgfusepath{fill}%
\end{pgfscope}%
\begin{pgfscope}%
\pgfpathrectangle{\pgfqpoint{0.887500in}{0.275000in}}{\pgfqpoint{4.225000in}{4.225000in}}%
\pgfusepath{clip}%
\pgfsetbuttcap%
\pgfsetroundjoin%
\definecolor{currentfill}{rgb}{0.000000,0.000000,0.000000}%
\pgfsetfillcolor{currentfill}%
\pgfsetfillopacity{0.800000}%
\pgfsetlinewidth{0.000000pt}%
\definecolor{currentstroke}{rgb}{0.000000,0.000000,0.000000}%
\pgfsetstrokecolor{currentstroke}%
\pgfsetstrokeopacity{0.800000}%
\pgfsetdash{}{0pt}%
\pgfpathmoveto{\pgfqpoint{4.338816in}{1.672834in}}%
\pgfpathcurveto{\pgfqpoint{4.342934in}{1.672834in}}{\pgfqpoint{4.346884in}{1.674470in}}{\pgfqpoint{4.349796in}{1.677382in}}%
\pgfpathcurveto{\pgfqpoint{4.352708in}{1.680294in}}{\pgfqpoint{4.354344in}{1.684244in}}{\pgfqpoint{4.354344in}{1.688362in}}%
\pgfpathcurveto{\pgfqpoint{4.354344in}{1.692480in}}{\pgfqpoint{4.352708in}{1.696430in}}{\pgfqpoint{4.349796in}{1.699342in}}%
\pgfpathcurveto{\pgfqpoint{4.346884in}{1.702254in}}{\pgfqpoint{4.342934in}{1.703890in}}{\pgfqpoint{4.338816in}{1.703890in}}%
\pgfpathcurveto{\pgfqpoint{4.334698in}{1.703890in}}{\pgfqpoint{4.330747in}{1.702254in}}{\pgfqpoint{4.327836in}{1.699342in}}%
\pgfpathcurveto{\pgfqpoint{4.324924in}{1.696430in}}{\pgfqpoint{4.323287in}{1.692480in}}{\pgfqpoint{4.323287in}{1.688362in}}%
\pgfpathcurveto{\pgfqpoint{4.323287in}{1.684244in}}{\pgfqpoint{4.324924in}{1.680294in}}{\pgfqpoint{4.327836in}{1.677382in}}%
\pgfpathcurveto{\pgfqpoint{4.330747in}{1.674470in}}{\pgfqpoint{4.334698in}{1.672834in}}{\pgfqpoint{4.338816in}{1.672834in}}%
\pgfpathclose%
\pgfusepath{fill}%
\end{pgfscope}%
\begin{pgfscope}%
\pgfpathrectangle{\pgfqpoint{0.887500in}{0.275000in}}{\pgfqpoint{4.225000in}{4.225000in}}%
\pgfusepath{clip}%
\pgfsetbuttcap%
\pgfsetroundjoin%
\definecolor{currentfill}{rgb}{0.000000,0.000000,0.000000}%
\pgfsetfillcolor{currentfill}%
\pgfsetfillopacity{0.800000}%
\pgfsetlinewidth{0.000000pt}%
\definecolor{currentstroke}{rgb}{0.000000,0.000000,0.000000}%
\pgfsetstrokecolor{currentstroke}%
\pgfsetstrokeopacity{0.800000}%
\pgfsetdash{}{0pt}%
\pgfpathmoveto{\pgfqpoint{3.748806in}{1.607268in}}%
\pgfpathcurveto{\pgfqpoint{3.752925in}{1.607268in}}{\pgfqpoint{3.756875in}{1.608904in}}{\pgfqpoint{3.759787in}{1.611816in}}%
\pgfpathcurveto{\pgfqpoint{3.762698in}{1.614728in}}{\pgfqpoint{3.764335in}{1.618678in}}{\pgfqpoint{3.764335in}{1.622797in}}%
\pgfpathcurveto{\pgfqpoint{3.764335in}{1.626915in}}{\pgfqpoint{3.762698in}{1.630865in}}{\pgfqpoint{3.759787in}{1.633777in}}%
\pgfpathcurveto{\pgfqpoint{3.756875in}{1.636689in}}{\pgfqpoint{3.752925in}{1.638325in}}{\pgfqpoint{3.748806in}{1.638325in}}%
\pgfpathcurveto{\pgfqpoint{3.744688in}{1.638325in}}{\pgfqpoint{3.740738in}{1.636689in}}{\pgfqpoint{3.737826in}{1.633777in}}%
\pgfpathcurveto{\pgfqpoint{3.734914in}{1.630865in}}{\pgfqpoint{3.733278in}{1.626915in}}{\pgfqpoint{3.733278in}{1.622797in}}%
\pgfpathcurveto{\pgfqpoint{3.733278in}{1.618678in}}{\pgfqpoint{3.734914in}{1.614728in}}{\pgfqpoint{3.737826in}{1.611816in}}%
\pgfpathcurveto{\pgfqpoint{3.740738in}{1.608904in}}{\pgfqpoint{3.744688in}{1.607268in}}{\pgfqpoint{3.748806in}{1.607268in}}%
\pgfpathclose%
\pgfusepath{fill}%
\end{pgfscope}%
\begin{pgfscope}%
\pgfpathrectangle{\pgfqpoint{0.887500in}{0.275000in}}{\pgfqpoint{4.225000in}{4.225000in}}%
\pgfusepath{clip}%
\pgfsetbuttcap%
\pgfsetroundjoin%
\definecolor{currentfill}{rgb}{0.000000,0.000000,0.000000}%
\pgfsetfillcolor{currentfill}%
\pgfsetfillopacity{0.800000}%
\pgfsetlinewidth{0.000000pt}%
\definecolor{currentstroke}{rgb}{0.000000,0.000000,0.000000}%
\pgfsetstrokecolor{currentstroke}%
\pgfsetstrokeopacity{0.800000}%
\pgfsetdash{}{0pt}%
\pgfpathmoveto{\pgfqpoint{3.092520in}{1.766508in}}%
\pgfpathcurveto{\pgfqpoint{3.096638in}{1.766508in}}{\pgfqpoint{3.100588in}{1.768144in}}{\pgfqpoint{3.103500in}{1.771056in}}%
\pgfpathcurveto{\pgfqpoint{3.106412in}{1.773968in}}{\pgfqpoint{3.108049in}{1.777918in}}{\pgfqpoint{3.108049in}{1.782037in}}%
\pgfpathcurveto{\pgfqpoint{3.108049in}{1.786155in}}{\pgfqpoint{3.106412in}{1.790105in}}{\pgfqpoint{3.103500in}{1.793017in}}%
\pgfpathcurveto{\pgfqpoint{3.100588in}{1.795929in}}{\pgfqpoint{3.096638in}{1.797565in}}{\pgfqpoint{3.092520in}{1.797565in}}%
\pgfpathcurveto{\pgfqpoint{3.088402in}{1.797565in}}{\pgfqpoint{3.084452in}{1.795929in}}{\pgfqpoint{3.081540in}{1.793017in}}%
\pgfpathcurveto{\pgfqpoint{3.078628in}{1.790105in}}{\pgfqpoint{3.076992in}{1.786155in}}{\pgfqpoint{3.076992in}{1.782037in}}%
\pgfpathcurveto{\pgfqpoint{3.076992in}{1.777918in}}{\pgfqpoint{3.078628in}{1.773968in}}{\pgfqpoint{3.081540in}{1.771056in}}%
\pgfpathcurveto{\pgfqpoint{3.084452in}{1.768144in}}{\pgfqpoint{3.088402in}{1.766508in}}{\pgfqpoint{3.092520in}{1.766508in}}%
\pgfpathclose%
\pgfusepath{fill}%
\end{pgfscope}%
\begin{pgfscope}%
\pgfpathrectangle{\pgfqpoint{0.887500in}{0.275000in}}{\pgfqpoint{4.225000in}{4.225000in}}%
\pgfusepath{clip}%
\pgfsetbuttcap%
\pgfsetroundjoin%
\definecolor{currentfill}{rgb}{0.000000,0.000000,0.000000}%
\pgfsetfillcolor{currentfill}%
\pgfsetfillopacity{0.800000}%
\pgfsetlinewidth{0.000000pt}%
\definecolor{currentstroke}{rgb}{0.000000,0.000000,0.000000}%
\pgfsetstrokecolor{currentstroke}%
\pgfsetstrokeopacity{0.800000}%
\pgfsetdash{}{0pt}%
\pgfpathmoveto{\pgfqpoint{4.158582in}{1.781597in}}%
\pgfpathcurveto{\pgfqpoint{4.162700in}{1.781597in}}{\pgfqpoint{4.166650in}{1.783234in}}{\pgfqpoint{4.169562in}{1.786146in}}%
\pgfpathcurveto{\pgfqpoint{4.172474in}{1.789058in}}{\pgfqpoint{4.174110in}{1.793008in}}{\pgfqpoint{4.174110in}{1.797126in}}%
\pgfpathcurveto{\pgfqpoint{4.174110in}{1.801244in}}{\pgfqpoint{4.172474in}{1.805194in}}{\pgfqpoint{4.169562in}{1.808106in}}%
\pgfpathcurveto{\pgfqpoint{4.166650in}{1.811018in}}{\pgfqpoint{4.162700in}{1.812654in}}{\pgfqpoint{4.158582in}{1.812654in}}%
\pgfpathcurveto{\pgfqpoint{4.154464in}{1.812654in}}{\pgfqpoint{4.150514in}{1.811018in}}{\pgfqpoint{4.147602in}{1.808106in}}%
\pgfpathcurveto{\pgfqpoint{4.144690in}{1.805194in}}{\pgfqpoint{4.143054in}{1.801244in}}{\pgfqpoint{4.143054in}{1.797126in}}%
\pgfpathcurveto{\pgfqpoint{4.143054in}{1.793008in}}{\pgfqpoint{4.144690in}{1.789058in}}{\pgfqpoint{4.147602in}{1.786146in}}%
\pgfpathcurveto{\pgfqpoint{4.150514in}{1.783234in}}{\pgfqpoint{4.154464in}{1.781597in}}{\pgfqpoint{4.158582in}{1.781597in}}%
\pgfpathclose%
\pgfusepath{fill}%
\end{pgfscope}%
\begin{pgfscope}%
\pgfpathrectangle{\pgfqpoint{0.887500in}{0.275000in}}{\pgfqpoint{4.225000in}{4.225000in}}%
\pgfusepath{clip}%
\pgfsetbuttcap%
\pgfsetroundjoin%
\definecolor{currentfill}{rgb}{0.000000,0.000000,0.000000}%
\pgfsetfillcolor{currentfill}%
\pgfsetfillopacity{0.800000}%
\pgfsetlinewidth{0.000000pt}%
\definecolor{currentstroke}{rgb}{0.000000,0.000000,0.000000}%
\pgfsetstrokecolor{currentstroke}%
\pgfsetstrokeopacity{0.800000}%
\pgfsetdash{}{0pt}%
\pgfpathmoveto{\pgfqpoint{3.863494in}{1.748814in}}%
\pgfpathcurveto{\pgfqpoint{3.867613in}{1.748814in}}{\pgfqpoint{3.871563in}{1.750450in}}{\pgfqpoint{3.874475in}{1.753362in}}%
\pgfpathcurveto{\pgfqpoint{3.877387in}{1.756274in}}{\pgfqpoint{3.879023in}{1.760224in}}{\pgfqpoint{3.879023in}{1.764343in}}%
\pgfpathcurveto{\pgfqpoint{3.879023in}{1.768461in}}{\pgfqpoint{3.877387in}{1.772411in}}{\pgfqpoint{3.874475in}{1.775323in}}%
\pgfpathcurveto{\pgfqpoint{3.871563in}{1.778235in}}{\pgfqpoint{3.867613in}{1.779871in}}{\pgfqpoint{3.863494in}{1.779871in}}%
\pgfpathcurveto{\pgfqpoint{3.859376in}{1.779871in}}{\pgfqpoint{3.855426in}{1.778235in}}{\pgfqpoint{3.852514in}{1.775323in}}%
\pgfpathcurveto{\pgfqpoint{3.849602in}{1.772411in}}{\pgfqpoint{3.847966in}{1.768461in}}{\pgfqpoint{3.847966in}{1.764343in}}%
\pgfpathcurveto{\pgfqpoint{3.847966in}{1.760224in}}{\pgfqpoint{3.849602in}{1.756274in}}{\pgfqpoint{3.852514in}{1.753362in}}%
\pgfpathcurveto{\pgfqpoint{3.855426in}{1.750450in}}{\pgfqpoint{3.859376in}{1.748814in}}{\pgfqpoint{3.863494in}{1.748814in}}%
\pgfpathclose%
\pgfusepath{fill}%
\end{pgfscope}%
\begin{pgfscope}%
\pgfpathrectangle{\pgfqpoint{0.887500in}{0.275000in}}{\pgfqpoint{4.225000in}{4.225000in}}%
\pgfusepath{clip}%
\pgfsetbuttcap%
\pgfsetroundjoin%
\definecolor{currentfill}{rgb}{0.000000,0.000000,0.000000}%
\pgfsetfillcolor{currentfill}%
\pgfsetfillopacity{0.800000}%
\pgfsetlinewidth{0.000000pt}%
\definecolor{currentstroke}{rgb}{0.000000,0.000000,0.000000}%
\pgfsetstrokecolor{currentstroke}%
\pgfsetstrokeopacity{0.800000}%
\pgfsetdash{}{0pt}%
\pgfpathmoveto{\pgfqpoint{1.779558in}{2.093569in}}%
\pgfpathcurveto{\pgfqpoint{1.783676in}{2.093569in}}{\pgfqpoint{1.787626in}{2.095205in}}{\pgfqpoint{1.790538in}{2.098117in}}%
\pgfpathcurveto{\pgfqpoint{1.793450in}{2.101029in}}{\pgfqpoint{1.795086in}{2.104979in}}{\pgfqpoint{1.795086in}{2.109097in}}%
\pgfpathcurveto{\pgfqpoint{1.795086in}{2.113216in}}{\pgfqpoint{1.793450in}{2.117166in}}{\pgfqpoint{1.790538in}{2.120078in}}%
\pgfpathcurveto{\pgfqpoint{1.787626in}{2.122990in}}{\pgfqpoint{1.783676in}{2.124626in}}{\pgfqpoint{1.779558in}{2.124626in}}%
\pgfpathcurveto{\pgfqpoint{1.775440in}{2.124626in}}{\pgfqpoint{1.771490in}{2.122990in}}{\pgfqpoint{1.768578in}{2.120078in}}%
\pgfpathcurveto{\pgfqpoint{1.765666in}{2.117166in}}{\pgfqpoint{1.764030in}{2.113216in}}{\pgfqpoint{1.764030in}{2.109097in}}%
\pgfpathcurveto{\pgfqpoint{1.764030in}{2.104979in}}{\pgfqpoint{1.765666in}{2.101029in}}{\pgfqpoint{1.768578in}{2.098117in}}%
\pgfpathcurveto{\pgfqpoint{1.771490in}{2.095205in}}{\pgfqpoint{1.775440in}{2.093569in}}{\pgfqpoint{1.779558in}{2.093569in}}%
\pgfpathclose%
\pgfusepath{fill}%
\end{pgfscope}%
\begin{pgfscope}%
\pgfpathrectangle{\pgfqpoint{0.887500in}{0.275000in}}{\pgfqpoint{4.225000in}{4.225000in}}%
\pgfusepath{clip}%
\pgfsetbuttcap%
\pgfsetroundjoin%
\definecolor{currentfill}{rgb}{0.000000,0.000000,0.000000}%
\pgfsetfillcolor{currentfill}%
\pgfsetfillopacity{0.800000}%
\pgfsetlinewidth{0.000000pt}%
\definecolor{currentstroke}{rgb}{0.000000,0.000000,0.000000}%
\pgfsetstrokecolor{currentstroke}%
\pgfsetstrokeopacity{0.800000}%
\pgfsetdash{}{0pt}%
\pgfpathmoveto{\pgfqpoint{3.273213in}{1.685971in}}%
\pgfpathcurveto{\pgfqpoint{3.277332in}{1.685971in}}{\pgfqpoint{3.281282in}{1.687608in}}{\pgfqpoint{3.284194in}{1.690520in}}%
\pgfpathcurveto{\pgfqpoint{3.287106in}{1.693432in}}{\pgfqpoint{3.288742in}{1.697382in}}{\pgfqpoint{3.288742in}{1.701500in}}%
\pgfpathcurveto{\pgfqpoint{3.288742in}{1.705618in}}{\pgfqpoint{3.287106in}{1.709568in}}{\pgfqpoint{3.284194in}{1.712480in}}%
\pgfpathcurveto{\pgfqpoint{3.281282in}{1.715392in}}{\pgfqpoint{3.277332in}{1.717028in}}{\pgfqpoint{3.273213in}{1.717028in}}%
\pgfpathcurveto{\pgfqpoint{3.269095in}{1.717028in}}{\pgfqpoint{3.265145in}{1.715392in}}{\pgfqpoint{3.262233in}{1.712480in}}%
\pgfpathcurveto{\pgfqpoint{3.259321in}{1.709568in}}{\pgfqpoint{3.257685in}{1.705618in}}{\pgfqpoint{3.257685in}{1.701500in}}%
\pgfpathcurveto{\pgfqpoint{3.257685in}{1.697382in}}{\pgfqpoint{3.259321in}{1.693432in}}{\pgfqpoint{3.262233in}{1.690520in}}%
\pgfpathcurveto{\pgfqpoint{3.265145in}{1.687608in}}{\pgfqpoint{3.269095in}{1.685971in}}{\pgfqpoint{3.273213in}{1.685971in}}%
\pgfpathclose%
\pgfusepath{fill}%
\end{pgfscope}%
\begin{pgfscope}%
\pgfpathrectangle{\pgfqpoint{0.887500in}{0.275000in}}{\pgfqpoint{4.225000in}{4.225000in}}%
\pgfusepath{clip}%
\pgfsetbuttcap%
\pgfsetroundjoin%
\definecolor{currentfill}{rgb}{0.000000,0.000000,0.000000}%
\pgfsetfillcolor{currentfill}%
\pgfsetfillopacity{0.800000}%
\pgfsetlinewidth{0.000000pt}%
\definecolor{currentstroke}{rgb}{0.000000,0.000000,0.000000}%
\pgfsetstrokecolor{currentstroke}%
\pgfsetstrokeopacity{0.800000}%
\pgfsetdash{}{0pt}%
\pgfpathmoveto{\pgfqpoint{2.435912in}{1.943660in}}%
\pgfpathcurveto{\pgfqpoint{2.440030in}{1.943660in}}{\pgfqpoint{2.443981in}{1.945296in}}{\pgfqpoint{2.446892in}{1.948208in}}%
\pgfpathcurveto{\pgfqpoint{2.449804in}{1.951120in}}{\pgfqpoint{2.451441in}{1.955070in}}{\pgfqpoint{2.451441in}{1.959188in}}%
\pgfpathcurveto{\pgfqpoint{2.451441in}{1.963306in}}{\pgfqpoint{2.449804in}{1.967256in}}{\pgfqpoint{2.446892in}{1.970168in}}%
\pgfpathcurveto{\pgfqpoint{2.443981in}{1.973080in}}{\pgfqpoint{2.440030in}{1.974717in}}{\pgfqpoint{2.435912in}{1.974717in}}%
\pgfpathcurveto{\pgfqpoint{2.431794in}{1.974717in}}{\pgfqpoint{2.427844in}{1.973080in}}{\pgfqpoint{2.424932in}{1.970168in}}%
\pgfpathcurveto{\pgfqpoint{2.422020in}{1.967256in}}{\pgfqpoint{2.420384in}{1.963306in}}{\pgfqpoint{2.420384in}{1.959188in}}%
\pgfpathcurveto{\pgfqpoint{2.420384in}{1.955070in}}{\pgfqpoint{2.422020in}{1.951120in}}{\pgfqpoint{2.424932in}{1.948208in}}%
\pgfpathcurveto{\pgfqpoint{2.427844in}{1.945296in}}{\pgfqpoint{2.431794in}{1.943660in}}{\pgfqpoint{2.435912in}{1.943660in}}%
\pgfpathclose%
\pgfusepath{fill}%
\end{pgfscope}%
\begin{pgfscope}%
\pgfpathrectangle{\pgfqpoint{0.887500in}{0.275000in}}{\pgfqpoint{4.225000in}{4.225000in}}%
\pgfusepath{clip}%
\pgfsetbuttcap%
\pgfsetroundjoin%
\definecolor{currentfill}{rgb}{0.000000,0.000000,0.000000}%
\pgfsetfillcolor{currentfill}%
\pgfsetfillopacity{0.800000}%
\pgfsetlinewidth{0.000000pt}%
\definecolor{currentstroke}{rgb}{0.000000,0.000000,0.000000}%
\pgfsetstrokecolor{currentstroke}%
\pgfsetstrokeopacity{0.800000}%
\pgfsetdash{}{0pt}%
\pgfpathmoveto{\pgfqpoint{3.815402in}{1.403316in}}%
\pgfpathcurveto{\pgfqpoint{3.819521in}{1.403316in}}{\pgfqpoint{3.823471in}{1.404952in}}{\pgfqpoint{3.826383in}{1.407864in}}%
\pgfpathcurveto{\pgfqpoint{3.829295in}{1.410776in}}{\pgfqpoint{3.830931in}{1.414726in}}{\pgfqpoint{3.830931in}{1.418844in}}%
\pgfpathcurveto{\pgfqpoint{3.830931in}{1.422962in}}{\pgfqpoint{3.829295in}{1.426912in}}{\pgfqpoint{3.826383in}{1.429824in}}%
\pgfpathcurveto{\pgfqpoint{3.823471in}{1.432736in}}{\pgfqpoint{3.819521in}{1.434372in}}{\pgfqpoint{3.815402in}{1.434372in}}%
\pgfpathcurveto{\pgfqpoint{3.811284in}{1.434372in}}{\pgfqpoint{3.807334in}{1.432736in}}{\pgfqpoint{3.804422in}{1.429824in}}%
\pgfpathcurveto{\pgfqpoint{3.801510in}{1.426912in}}{\pgfqpoint{3.799874in}{1.422962in}}{\pgfqpoint{3.799874in}{1.418844in}}%
\pgfpathcurveto{\pgfqpoint{3.799874in}{1.414726in}}{\pgfqpoint{3.801510in}{1.410776in}}{\pgfqpoint{3.804422in}{1.407864in}}%
\pgfpathcurveto{\pgfqpoint{3.807334in}{1.404952in}}{\pgfqpoint{3.811284in}{1.403316in}}{\pgfqpoint{3.815402in}{1.403316in}}%
\pgfpathclose%
\pgfusepath{fill}%
\end{pgfscope}%
\begin{pgfscope}%
\pgfpathrectangle{\pgfqpoint{0.887500in}{0.275000in}}{\pgfqpoint{4.225000in}{4.225000in}}%
\pgfusepath{clip}%
\pgfsetbuttcap%
\pgfsetroundjoin%
\definecolor{currentfill}{rgb}{0.000000,0.000000,0.000000}%
\pgfsetfillcolor{currentfill}%
\pgfsetfillopacity{0.800000}%
\pgfsetlinewidth{0.000000pt}%
\definecolor{currentstroke}{rgb}{0.000000,0.000000,0.000000}%
\pgfsetstrokecolor{currentstroke}%
\pgfsetstrokeopacity{0.800000}%
\pgfsetdash{}{0pt}%
\pgfpathmoveto{\pgfqpoint{3.453987in}{1.598451in}}%
\pgfpathcurveto{\pgfqpoint{3.458105in}{1.598451in}}{\pgfqpoint{3.462055in}{1.600088in}}{\pgfqpoint{3.464967in}{1.603000in}}%
\pgfpathcurveto{\pgfqpoint{3.467879in}{1.605912in}}{\pgfqpoint{3.469515in}{1.609862in}}{\pgfqpoint{3.469515in}{1.613980in}}%
\pgfpathcurveto{\pgfqpoint{3.469515in}{1.618098in}}{\pgfqpoint{3.467879in}{1.622048in}}{\pgfqpoint{3.464967in}{1.624960in}}%
\pgfpathcurveto{\pgfqpoint{3.462055in}{1.627872in}}{\pgfqpoint{3.458105in}{1.629508in}}{\pgfqpoint{3.453987in}{1.629508in}}%
\pgfpathcurveto{\pgfqpoint{3.449869in}{1.629508in}}{\pgfqpoint{3.445919in}{1.627872in}}{\pgfqpoint{3.443007in}{1.624960in}}%
\pgfpathcurveto{\pgfqpoint{3.440095in}{1.622048in}}{\pgfqpoint{3.438459in}{1.618098in}}{\pgfqpoint{3.438459in}{1.613980in}}%
\pgfpathcurveto{\pgfqpoint{3.438459in}{1.609862in}}{\pgfqpoint{3.440095in}{1.605912in}}{\pgfqpoint{3.443007in}{1.603000in}}%
\pgfpathcurveto{\pgfqpoint{3.445919in}{1.600088in}}{\pgfqpoint{3.449869in}{1.598451in}}{\pgfqpoint{3.453987in}{1.598451in}}%
\pgfpathclose%
\pgfusepath{fill}%
\end{pgfscope}%
\begin{pgfscope}%
\pgfpathrectangle{\pgfqpoint{0.887500in}{0.275000in}}{\pgfqpoint{4.225000in}{4.225000in}}%
\pgfusepath{clip}%
\pgfsetbuttcap%
\pgfsetroundjoin%
\definecolor{currentfill}{rgb}{0.000000,0.000000,0.000000}%
\pgfsetfillcolor{currentfill}%
\pgfsetfillopacity{0.800000}%
\pgfsetlinewidth{0.000000pt}%
\definecolor{currentstroke}{rgb}{0.000000,0.000000,0.000000}%
\pgfsetstrokecolor{currentstroke}%
\pgfsetstrokeopacity{0.800000}%
\pgfsetdash{}{0pt}%
\pgfpathmoveto{\pgfqpoint{4.044973in}{1.684936in}}%
\pgfpathcurveto{\pgfqpoint{4.049092in}{1.684936in}}{\pgfqpoint{4.053042in}{1.686572in}}{\pgfqpoint{4.055954in}{1.689484in}}%
\pgfpathcurveto{\pgfqpoint{4.058866in}{1.692396in}}{\pgfqpoint{4.060502in}{1.696346in}}{\pgfqpoint{4.060502in}{1.700464in}}%
\pgfpathcurveto{\pgfqpoint{4.060502in}{1.704582in}}{\pgfqpoint{4.058866in}{1.708532in}}{\pgfqpoint{4.055954in}{1.711444in}}%
\pgfpathcurveto{\pgfqpoint{4.053042in}{1.714356in}}{\pgfqpoint{4.049092in}{1.715992in}}{\pgfqpoint{4.044973in}{1.715992in}}%
\pgfpathcurveto{\pgfqpoint{4.040855in}{1.715992in}}{\pgfqpoint{4.036905in}{1.714356in}}{\pgfqpoint{4.033993in}{1.711444in}}%
\pgfpathcurveto{\pgfqpoint{4.031081in}{1.708532in}}{\pgfqpoint{4.029445in}{1.704582in}}{\pgfqpoint{4.029445in}{1.700464in}}%
\pgfpathcurveto{\pgfqpoint{4.029445in}{1.696346in}}{\pgfqpoint{4.031081in}{1.692396in}}{\pgfqpoint{4.033993in}{1.689484in}}%
\pgfpathcurveto{\pgfqpoint{4.036905in}{1.686572in}}{\pgfqpoint{4.040855in}{1.684936in}}{\pgfqpoint{4.044973in}{1.684936in}}%
\pgfpathclose%
\pgfusepath{fill}%
\end{pgfscope}%
\begin{pgfscope}%
\pgfpathrectangle{\pgfqpoint{0.887500in}{0.275000in}}{\pgfqpoint{4.225000in}{4.225000in}}%
\pgfusepath{clip}%
\pgfsetbuttcap%
\pgfsetroundjoin%
\definecolor{currentfill}{rgb}{0.000000,0.000000,0.000000}%
\pgfsetfillcolor{currentfill}%
\pgfsetfillopacity{0.800000}%
\pgfsetlinewidth{0.000000pt}%
\definecolor{currentstroke}{rgb}{0.000000,0.000000,0.000000}%
\pgfsetstrokecolor{currentstroke}%
\pgfsetstrokeopacity{0.800000}%
\pgfsetdash{}{0pt}%
\pgfpathmoveto{\pgfqpoint{1.959194in}{2.033002in}}%
\pgfpathcurveto{\pgfqpoint{1.963312in}{2.033002in}}{\pgfqpoint{1.967262in}{2.034638in}}{\pgfqpoint{1.970174in}{2.037550in}}%
\pgfpathcurveto{\pgfqpoint{1.973086in}{2.040462in}}{\pgfqpoint{1.974722in}{2.044412in}}{\pgfqpoint{1.974722in}{2.048530in}}%
\pgfpathcurveto{\pgfqpoint{1.974722in}{2.052648in}}{\pgfqpoint{1.973086in}{2.056598in}}{\pgfqpoint{1.970174in}{2.059510in}}%
\pgfpathcurveto{\pgfqpoint{1.967262in}{2.062422in}}{\pgfqpoint{1.963312in}{2.064059in}}{\pgfqpoint{1.959194in}{2.064059in}}%
\pgfpathcurveto{\pgfqpoint{1.955076in}{2.064059in}}{\pgfqpoint{1.951126in}{2.062422in}}{\pgfqpoint{1.948214in}{2.059510in}}%
\pgfpathcurveto{\pgfqpoint{1.945302in}{2.056598in}}{\pgfqpoint{1.943666in}{2.052648in}}{\pgfqpoint{1.943666in}{2.048530in}}%
\pgfpathcurveto{\pgfqpoint{1.943666in}{2.044412in}}{\pgfqpoint{1.945302in}{2.040462in}}{\pgfqpoint{1.948214in}{2.037550in}}%
\pgfpathcurveto{\pgfqpoint{1.951126in}{2.034638in}}{\pgfqpoint{1.955076in}{2.033002in}}{\pgfqpoint{1.959194in}{2.033002in}}%
\pgfpathclose%
\pgfusepath{fill}%
\end{pgfscope}%
\begin{pgfscope}%
\pgfpathrectangle{\pgfqpoint{0.887500in}{0.275000in}}{\pgfqpoint{4.225000in}{4.225000in}}%
\pgfusepath{clip}%
\pgfsetbuttcap%
\pgfsetroundjoin%
\definecolor{currentfill}{rgb}{0.000000,0.000000,0.000000}%
\pgfsetfillcolor{currentfill}%
\pgfsetfillopacity{0.800000}%
\pgfsetlinewidth{0.000000pt}%
\definecolor{currentstroke}{rgb}{0.000000,0.000000,0.000000}%
\pgfsetstrokecolor{currentstroke}%
\pgfsetstrokeopacity{0.800000}%
\pgfsetdash{}{0pt}%
\pgfpathmoveto{\pgfqpoint{2.616274in}{1.875972in}}%
\pgfpathcurveto{\pgfqpoint{2.620392in}{1.875972in}}{\pgfqpoint{2.624342in}{1.877608in}}{\pgfqpoint{2.627254in}{1.880520in}}%
\pgfpathcurveto{\pgfqpoint{2.630166in}{1.883432in}}{\pgfqpoint{2.631802in}{1.887382in}}{\pgfqpoint{2.631802in}{1.891500in}}%
\pgfpathcurveto{\pgfqpoint{2.631802in}{1.895618in}}{\pgfqpoint{2.630166in}{1.899568in}}{\pgfqpoint{2.627254in}{1.902480in}}%
\pgfpathcurveto{\pgfqpoint{2.624342in}{1.905392in}}{\pgfqpoint{2.620392in}{1.907028in}}{\pgfqpoint{2.616274in}{1.907028in}}%
\pgfpathcurveto{\pgfqpoint{2.612155in}{1.907028in}}{\pgfqpoint{2.608205in}{1.905392in}}{\pgfqpoint{2.605293in}{1.902480in}}%
\pgfpathcurveto{\pgfqpoint{2.602381in}{1.899568in}}{\pgfqpoint{2.600745in}{1.895618in}}{\pgfqpoint{2.600745in}{1.891500in}}%
\pgfpathcurveto{\pgfqpoint{2.600745in}{1.887382in}}{\pgfqpoint{2.602381in}{1.883432in}}{\pgfqpoint{2.605293in}{1.880520in}}%
\pgfpathcurveto{\pgfqpoint{2.608205in}{1.877608in}}{\pgfqpoint{2.612155in}{1.875972in}}{\pgfqpoint{2.616274in}{1.875972in}}%
\pgfpathclose%
\pgfusepath{fill}%
\end{pgfscope}%
\begin{pgfscope}%
\pgfpathrectangle{\pgfqpoint{0.887500in}{0.275000in}}{\pgfqpoint{4.225000in}{4.225000in}}%
\pgfusepath{clip}%
\pgfsetbuttcap%
\pgfsetroundjoin%
\definecolor{currentfill}{rgb}{0.000000,0.000000,0.000000}%
\pgfsetfillcolor{currentfill}%
\pgfsetfillopacity{0.800000}%
\pgfsetlinewidth{0.000000pt}%
\definecolor{currentstroke}{rgb}{0.000000,0.000000,0.000000}%
\pgfsetstrokecolor{currentstroke}%
\pgfsetstrokeopacity{0.800000}%
\pgfsetdash{}{0pt}%
\pgfpathmoveto{\pgfqpoint{3.930707in}{1.568196in}}%
\pgfpathcurveto{\pgfqpoint{3.934825in}{1.568196in}}{\pgfqpoint{3.938775in}{1.569832in}}{\pgfqpoint{3.941687in}{1.572744in}}%
\pgfpathcurveto{\pgfqpoint{3.944599in}{1.575656in}}{\pgfqpoint{3.946235in}{1.579606in}}{\pgfqpoint{3.946235in}{1.583724in}}%
\pgfpathcurveto{\pgfqpoint{3.946235in}{1.587842in}}{\pgfqpoint{3.944599in}{1.591792in}}{\pgfqpoint{3.941687in}{1.594704in}}%
\pgfpathcurveto{\pgfqpoint{3.938775in}{1.597616in}}{\pgfqpoint{3.934825in}{1.599252in}}{\pgfqpoint{3.930707in}{1.599252in}}%
\pgfpathcurveto{\pgfqpoint{3.926589in}{1.599252in}}{\pgfqpoint{3.922639in}{1.597616in}}{\pgfqpoint{3.919727in}{1.594704in}}%
\pgfpathcurveto{\pgfqpoint{3.916815in}{1.591792in}}{\pgfqpoint{3.915179in}{1.587842in}}{\pgfqpoint{3.915179in}{1.583724in}}%
\pgfpathcurveto{\pgfqpoint{3.915179in}{1.579606in}}{\pgfqpoint{3.916815in}{1.575656in}}{\pgfqpoint{3.919727in}{1.572744in}}%
\pgfpathcurveto{\pgfqpoint{3.922639in}{1.569832in}}{\pgfqpoint{3.926589in}{1.568196in}}{\pgfqpoint{3.930707in}{1.568196in}}%
\pgfpathclose%
\pgfusepath{fill}%
\end{pgfscope}%
\begin{pgfscope}%
\pgfpathrectangle{\pgfqpoint{0.887500in}{0.275000in}}{\pgfqpoint{4.225000in}{4.225000in}}%
\pgfusepath{clip}%
\pgfsetbuttcap%
\pgfsetroundjoin%
\definecolor{currentfill}{rgb}{0.000000,0.000000,0.000000}%
\pgfsetfillcolor{currentfill}%
\pgfsetfillopacity{0.800000}%
\pgfsetlinewidth{0.000000pt}%
\definecolor{currentstroke}{rgb}{0.000000,0.000000,0.000000}%
\pgfsetstrokecolor{currentstroke}%
\pgfsetstrokeopacity{0.800000}%
\pgfsetdash{}{0pt}%
\pgfpathmoveto{\pgfqpoint{2.796930in}{1.804401in}}%
\pgfpathcurveto{\pgfqpoint{2.801048in}{1.804401in}}{\pgfqpoint{2.804998in}{1.806037in}}{\pgfqpoint{2.807910in}{1.808949in}}%
\pgfpathcurveto{\pgfqpoint{2.810822in}{1.811861in}}{\pgfqpoint{2.812458in}{1.815811in}}{\pgfqpoint{2.812458in}{1.819929in}}%
\pgfpathcurveto{\pgfqpoint{2.812458in}{1.824048in}}{\pgfqpoint{2.810822in}{1.827998in}}{\pgfqpoint{2.807910in}{1.830910in}}%
\pgfpathcurveto{\pgfqpoint{2.804998in}{1.833822in}}{\pgfqpoint{2.801048in}{1.835458in}}{\pgfqpoint{2.796930in}{1.835458in}}%
\pgfpathcurveto{\pgfqpoint{2.792812in}{1.835458in}}{\pgfqpoint{2.788862in}{1.833822in}}{\pgfqpoint{2.785950in}{1.830910in}}%
\pgfpathcurveto{\pgfqpoint{2.783038in}{1.827998in}}{\pgfqpoint{2.781402in}{1.824048in}}{\pgfqpoint{2.781402in}{1.819929in}}%
\pgfpathcurveto{\pgfqpoint{2.781402in}{1.815811in}}{\pgfqpoint{2.783038in}{1.811861in}}{\pgfqpoint{2.785950in}{1.808949in}}%
\pgfpathcurveto{\pgfqpoint{2.788862in}{1.806037in}}{\pgfqpoint{2.792812in}{1.804401in}}{\pgfqpoint{2.796930in}{1.804401in}}%
\pgfpathclose%
\pgfusepath{fill}%
\end{pgfscope}%
\begin{pgfscope}%
\pgfpathrectangle{\pgfqpoint{0.887500in}{0.275000in}}{\pgfqpoint{4.225000in}{4.225000in}}%
\pgfusepath{clip}%
\pgfsetbuttcap%
\pgfsetroundjoin%
\definecolor{currentfill}{rgb}{0.000000,0.000000,0.000000}%
\pgfsetfillcolor{currentfill}%
\pgfsetfillopacity{0.800000}%
\pgfsetlinewidth{0.000000pt}%
\definecolor{currentstroke}{rgb}{0.000000,0.000000,0.000000}%
\pgfsetstrokecolor{currentstroke}%
\pgfsetstrokeopacity{0.800000}%
\pgfsetdash{}{0pt}%
\pgfpathmoveto{\pgfqpoint{3.997469in}{1.361789in}}%
\pgfpathcurveto{\pgfqpoint{4.001587in}{1.361789in}}{\pgfqpoint{4.005537in}{1.363425in}}{\pgfqpoint{4.008449in}{1.366337in}}%
\pgfpathcurveto{\pgfqpoint{4.011361in}{1.369249in}}{\pgfqpoint{4.012997in}{1.373199in}}{\pgfqpoint{4.012997in}{1.377317in}}%
\pgfpathcurveto{\pgfqpoint{4.012997in}{1.381435in}}{\pgfqpoint{4.011361in}{1.385385in}}{\pgfqpoint{4.008449in}{1.388297in}}%
\pgfpathcurveto{\pgfqpoint{4.005537in}{1.391209in}}{\pgfqpoint{4.001587in}{1.392846in}}{\pgfqpoint{3.997469in}{1.392846in}}%
\pgfpathcurveto{\pgfqpoint{3.993351in}{1.392846in}}{\pgfqpoint{3.989401in}{1.391209in}}{\pgfqpoint{3.986489in}{1.388297in}}%
\pgfpathcurveto{\pgfqpoint{3.983577in}{1.385385in}}{\pgfqpoint{3.981941in}{1.381435in}}{\pgfqpoint{3.981941in}{1.377317in}}%
\pgfpathcurveto{\pgfqpoint{3.981941in}{1.373199in}}{\pgfqpoint{3.983577in}{1.369249in}}{\pgfqpoint{3.986489in}{1.366337in}}%
\pgfpathcurveto{\pgfqpoint{3.989401in}{1.363425in}}{\pgfqpoint{3.993351in}{1.361789in}}{\pgfqpoint{3.997469in}{1.361789in}}%
\pgfpathclose%
\pgfusepath{fill}%
\end{pgfscope}%
\begin{pgfscope}%
\pgfpathrectangle{\pgfqpoint{0.887500in}{0.275000in}}{\pgfqpoint{4.225000in}{4.225000in}}%
\pgfusepath{clip}%
\pgfsetbuttcap%
\pgfsetroundjoin%
\definecolor{currentfill}{rgb}{0.000000,0.000000,0.000000}%
\pgfsetfillcolor{currentfill}%
\pgfsetfillopacity{0.800000}%
\pgfsetlinewidth{0.000000pt}%
\definecolor{currentstroke}{rgb}{0.000000,0.000000,0.000000}%
\pgfsetstrokecolor{currentstroke}%
\pgfsetstrokeopacity{0.800000}%
\pgfsetdash{}{0pt}%
\pgfpathmoveto{\pgfqpoint{2.139208in}{1.970424in}}%
\pgfpathcurveto{\pgfqpoint{2.143326in}{1.970424in}}{\pgfqpoint{2.147276in}{1.972060in}}{\pgfqpoint{2.150188in}{1.974972in}}%
\pgfpathcurveto{\pgfqpoint{2.153100in}{1.977884in}}{\pgfqpoint{2.154736in}{1.981834in}}{\pgfqpoint{2.154736in}{1.985952in}}%
\pgfpathcurveto{\pgfqpoint{2.154736in}{1.990070in}}{\pgfqpoint{2.153100in}{1.994020in}}{\pgfqpoint{2.150188in}{1.996932in}}%
\pgfpathcurveto{\pgfqpoint{2.147276in}{1.999844in}}{\pgfqpoint{2.143326in}{2.001480in}}{\pgfqpoint{2.139208in}{2.001480in}}%
\pgfpathcurveto{\pgfqpoint{2.135090in}{2.001480in}}{\pgfqpoint{2.131140in}{1.999844in}}{\pgfqpoint{2.128228in}{1.996932in}}%
\pgfpathcurveto{\pgfqpoint{2.125316in}{1.994020in}}{\pgfqpoint{2.123680in}{1.990070in}}{\pgfqpoint{2.123680in}{1.985952in}}%
\pgfpathcurveto{\pgfqpoint{2.123680in}{1.981834in}}{\pgfqpoint{2.125316in}{1.977884in}}{\pgfqpoint{2.128228in}{1.974972in}}%
\pgfpathcurveto{\pgfqpoint{2.131140in}{1.972060in}}{\pgfqpoint{2.135090in}{1.970424in}}{\pgfqpoint{2.139208in}{1.970424in}}%
\pgfpathclose%
\pgfusepath{fill}%
\end{pgfscope}%
\begin{pgfscope}%
\pgfpathrectangle{\pgfqpoint{0.887500in}{0.275000in}}{\pgfqpoint{4.225000in}{4.225000in}}%
\pgfusepath{clip}%
\pgfsetbuttcap%
\pgfsetroundjoin%
\definecolor{currentfill}{rgb}{0.000000,0.000000,0.000000}%
\pgfsetfillcolor{currentfill}%
\pgfsetfillopacity{0.800000}%
\pgfsetlinewidth{0.000000pt}%
\definecolor{currentstroke}{rgb}{0.000000,0.000000,0.000000}%
\pgfsetstrokecolor{currentstroke}%
\pgfsetstrokeopacity{0.800000}%
\pgfsetdash{}{0pt}%
\pgfpathmoveto{\pgfqpoint{2.977830in}{1.728618in}}%
\pgfpathcurveto{\pgfqpoint{2.981948in}{1.728618in}}{\pgfqpoint{2.985898in}{1.730255in}}{\pgfqpoint{2.988810in}{1.733167in}}%
\pgfpathcurveto{\pgfqpoint{2.991722in}{1.736079in}}{\pgfqpoint{2.993359in}{1.740029in}}{\pgfqpoint{2.993359in}{1.744147in}}%
\pgfpathcurveto{\pgfqpoint{2.993359in}{1.748265in}}{\pgfqpoint{2.991722in}{1.752215in}}{\pgfqpoint{2.988810in}{1.755127in}}%
\pgfpathcurveto{\pgfqpoint{2.985898in}{1.758039in}}{\pgfqpoint{2.981948in}{1.759675in}}{\pgfqpoint{2.977830in}{1.759675in}}%
\pgfpathcurveto{\pgfqpoint{2.973712in}{1.759675in}}{\pgfqpoint{2.969762in}{1.758039in}}{\pgfqpoint{2.966850in}{1.755127in}}%
\pgfpathcurveto{\pgfqpoint{2.963938in}{1.752215in}}{\pgfqpoint{2.962302in}{1.748265in}}{\pgfqpoint{2.962302in}{1.744147in}}%
\pgfpathcurveto{\pgfqpoint{2.962302in}{1.740029in}}{\pgfqpoint{2.963938in}{1.736079in}}{\pgfqpoint{2.966850in}{1.733167in}}%
\pgfpathcurveto{\pgfqpoint{2.969762in}{1.730255in}}{\pgfqpoint{2.973712in}{1.728618in}}{\pgfqpoint{2.977830in}{1.728618in}}%
\pgfpathclose%
\pgfusepath{fill}%
\end{pgfscope}%
\begin{pgfscope}%
\pgfpathrectangle{\pgfqpoint{0.887500in}{0.275000in}}{\pgfqpoint{4.225000in}{4.225000in}}%
\pgfusepath{clip}%
\pgfsetbuttcap%
\pgfsetroundjoin%
\definecolor{currentfill}{rgb}{0.000000,0.000000,0.000000}%
\pgfsetfillcolor{currentfill}%
\pgfsetfillopacity{0.800000}%
\pgfsetlinewidth{0.000000pt}%
\definecolor{currentstroke}{rgb}{0.000000,0.000000,0.000000}%
\pgfsetstrokecolor{currentstroke}%
\pgfsetstrokeopacity{0.800000}%
\pgfsetdash{}{0pt}%
\pgfpathmoveto{\pgfqpoint{3.158916in}{1.648041in}}%
\pgfpathcurveto{\pgfqpoint{3.163034in}{1.648041in}}{\pgfqpoint{3.166984in}{1.649677in}}{\pgfqpoint{3.169896in}{1.652589in}}%
\pgfpathcurveto{\pgfqpoint{3.172808in}{1.655501in}}{\pgfqpoint{3.174444in}{1.659451in}}{\pgfqpoint{3.174444in}{1.663569in}}%
\pgfpathcurveto{\pgfqpoint{3.174444in}{1.667687in}}{\pgfqpoint{3.172808in}{1.671637in}}{\pgfqpoint{3.169896in}{1.674549in}}%
\pgfpathcurveto{\pgfqpoint{3.166984in}{1.677461in}}{\pgfqpoint{3.163034in}{1.679097in}}{\pgfqpoint{3.158916in}{1.679097in}}%
\pgfpathcurveto{\pgfqpoint{3.154798in}{1.679097in}}{\pgfqpoint{3.150848in}{1.677461in}}{\pgfqpoint{3.147936in}{1.674549in}}%
\pgfpathcurveto{\pgfqpoint{3.145024in}{1.671637in}}{\pgfqpoint{3.143387in}{1.667687in}}{\pgfqpoint{3.143387in}{1.663569in}}%
\pgfpathcurveto{\pgfqpoint{3.143387in}{1.659451in}}{\pgfqpoint{3.145024in}{1.655501in}}{\pgfqpoint{3.147936in}{1.652589in}}%
\pgfpathcurveto{\pgfqpoint{3.150848in}{1.649677in}}{\pgfqpoint{3.154798in}{1.648041in}}{\pgfqpoint{3.158916in}{1.648041in}}%
\pgfpathclose%
\pgfusepath{fill}%
\end{pgfscope}%
\begin{pgfscope}%
\pgfpathrectangle{\pgfqpoint{0.887500in}{0.275000in}}{\pgfqpoint{4.225000in}{4.225000in}}%
\pgfusepath{clip}%
\pgfsetbuttcap%
\pgfsetroundjoin%
\definecolor{currentfill}{rgb}{0.000000,0.000000,0.000000}%
\pgfsetfillcolor{currentfill}%
\pgfsetfillopacity{0.800000}%
\pgfsetlinewidth{0.000000pt}%
\definecolor{currentstroke}{rgb}{0.000000,0.000000,0.000000}%
\pgfsetstrokecolor{currentstroke}%
\pgfsetstrokeopacity{0.800000}%
\pgfsetdash{}{0pt}%
\pgfpathmoveto{\pgfqpoint{2.319581in}{1.905493in}}%
\pgfpathcurveto{\pgfqpoint{2.323699in}{1.905493in}}{\pgfqpoint{2.327649in}{1.907129in}}{\pgfqpoint{2.330561in}{1.910041in}}%
\pgfpathcurveto{\pgfqpoint{2.333473in}{1.912953in}}{\pgfqpoint{2.335109in}{1.916903in}}{\pgfqpoint{2.335109in}{1.921021in}}%
\pgfpathcurveto{\pgfqpoint{2.335109in}{1.925139in}}{\pgfqpoint{2.333473in}{1.929090in}}{\pgfqpoint{2.330561in}{1.932001in}}%
\pgfpathcurveto{\pgfqpoint{2.327649in}{1.934913in}}{\pgfqpoint{2.323699in}{1.936550in}}{\pgfqpoint{2.319581in}{1.936550in}}%
\pgfpathcurveto{\pgfqpoint{2.315463in}{1.936550in}}{\pgfqpoint{2.311513in}{1.934913in}}{\pgfqpoint{2.308601in}{1.932001in}}%
\pgfpathcurveto{\pgfqpoint{2.305689in}{1.929090in}}{\pgfqpoint{2.304053in}{1.925139in}}{\pgfqpoint{2.304053in}{1.921021in}}%
\pgfpathcurveto{\pgfqpoint{2.304053in}{1.916903in}}{\pgfqpoint{2.305689in}{1.912953in}}{\pgfqpoint{2.308601in}{1.910041in}}%
\pgfpathcurveto{\pgfqpoint{2.311513in}{1.907129in}}{\pgfqpoint{2.315463in}{1.905493in}}{\pgfqpoint{2.319581in}{1.905493in}}%
\pgfpathclose%
\pgfusepath{fill}%
\end{pgfscope}%
\begin{pgfscope}%
\pgfpathrectangle{\pgfqpoint{0.887500in}{0.275000in}}{\pgfqpoint{4.225000in}{4.225000in}}%
\pgfusepath{clip}%
\pgfsetbuttcap%
\pgfsetroundjoin%
\definecolor{currentfill}{rgb}{0.000000,0.000000,0.000000}%
\pgfsetfillcolor{currentfill}%
\pgfsetfillopacity{0.800000}%
\pgfsetlinewidth{0.000000pt}%
\definecolor{currentstroke}{rgb}{0.000000,0.000000,0.000000}%
\pgfsetstrokecolor{currentstroke}%
\pgfsetstrokeopacity{0.800000}%
\pgfsetdash{}{0pt}%
\pgfpathmoveto{\pgfqpoint{3.521255in}{1.462364in}}%
\pgfpathcurveto{\pgfqpoint{3.525373in}{1.462364in}}{\pgfqpoint{3.529323in}{1.464000in}}{\pgfqpoint{3.532235in}{1.466912in}}%
\pgfpathcurveto{\pgfqpoint{3.535147in}{1.469824in}}{\pgfqpoint{3.536784in}{1.473774in}}{\pgfqpoint{3.536784in}{1.477892in}}%
\pgfpathcurveto{\pgfqpoint{3.536784in}{1.482010in}}{\pgfqpoint{3.535147in}{1.485960in}}{\pgfqpoint{3.532235in}{1.488872in}}%
\pgfpathcurveto{\pgfqpoint{3.529323in}{1.491784in}}{\pgfqpoint{3.525373in}{1.493420in}}{\pgfqpoint{3.521255in}{1.493420in}}%
\pgfpathcurveto{\pgfqpoint{3.517137in}{1.493420in}}{\pgfqpoint{3.513187in}{1.491784in}}{\pgfqpoint{3.510275in}{1.488872in}}%
\pgfpathcurveto{\pgfqpoint{3.507363in}{1.485960in}}{\pgfqpoint{3.505727in}{1.482010in}}{\pgfqpoint{3.505727in}{1.477892in}}%
\pgfpathcurveto{\pgfqpoint{3.505727in}{1.473774in}}{\pgfqpoint{3.507363in}{1.469824in}}{\pgfqpoint{3.510275in}{1.466912in}}%
\pgfpathcurveto{\pgfqpoint{3.513187in}{1.464000in}}{\pgfqpoint{3.517137in}{1.462364in}}{\pgfqpoint{3.521255in}{1.462364in}}%
\pgfpathclose%
\pgfusepath{fill}%
\end{pgfscope}%
\begin{pgfscope}%
\pgfpathrectangle{\pgfqpoint{0.887500in}{0.275000in}}{\pgfqpoint{4.225000in}{4.225000in}}%
\pgfusepath{clip}%
\pgfsetbuttcap%
\pgfsetroundjoin%
\definecolor{currentfill}{rgb}{0.000000,0.000000,0.000000}%
\pgfsetfillcolor{currentfill}%
\pgfsetfillopacity{0.800000}%
\pgfsetlinewidth{0.000000pt}%
\definecolor{currentstroke}{rgb}{0.000000,0.000000,0.000000}%
\pgfsetstrokecolor{currentstroke}%
\pgfsetstrokeopacity{0.800000}%
\pgfsetdash{}{0pt}%
\pgfpathmoveto{\pgfqpoint{3.702427in}{1.366049in}}%
\pgfpathcurveto{\pgfqpoint{3.706545in}{1.366049in}}{\pgfqpoint{3.710495in}{1.367685in}}{\pgfqpoint{3.713407in}{1.370597in}}%
\pgfpathcurveto{\pgfqpoint{3.716319in}{1.373509in}}{\pgfqpoint{3.717955in}{1.377459in}}{\pgfqpoint{3.717955in}{1.381577in}}%
\pgfpathcurveto{\pgfqpoint{3.717955in}{1.385696in}}{\pgfqpoint{3.716319in}{1.389646in}}{\pgfqpoint{3.713407in}{1.392558in}}%
\pgfpathcurveto{\pgfqpoint{3.710495in}{1.395469in}}{\pgfqpoint{3.706545in}{1.397106in}}{\pgfqpoint{3.702427in}{1.397106in}}%
\pgfpathcurveto{\pgfqpoint{3.698309in}{1.397106in}}{\pgfqpoint{3.694359in}{1.395469in}}{\pgfqpoint{3.691447in}{1.392558in}}%
\pgfpathcurveto{\pgfqpoint{3.688535in}{1.389646in}}{\pgfqpoint{3.686899in}{1.385696in}}{\pgfqpoint{3.686899in}{1.381577in}}%
\pgfpathcurveto{\pgfqpoint{3.686899in}{1.377459in}}{\pgfqpoint{3.688535in}{1.373509in}}{\pgfqpoint{3.691447in}{1.370597in}}%
\pgfpathcurveto{\pgfqpoint{3.694359in}{1.367685in}}{\pgfqpoint{3.698309in}{1.366049in}}{\pgfqpoint{3.702427in}{1.366049in}}%
\pgfpathclose%
\pgfusepath{fill}%
\end{pgfscope}%
\begin{pgfscope}%
\pgfpathrectangle{\pgfqpoint{0.887500in}{0.275000in}}{\pgfqpoint{4.225000in}{4.225000in}}%
\pgfusepath{clip}%
\pgfsetbuttcap%
\pgfsetroundjoin%
\definecolor{currentfill}{rgb}{0.000000,0.000000,0.000000}%
\pgfsetfillcolor{currentfill}%
\pgfsetfillopacity{0.800000}%
\pgfsetlinewidth{0.000000pt}%
\definecolor{currentstroke}{rgb}{0.000000,0.000000,0.000000}%
\pgfsetstrokecolor{currentstroke}%
\pgfsetstrokeopacity{0.800000}%
\pgfsetdash{}{0pt}%
\pgfpathmoveto{\pgfqpoint{3.340103in}{1.560384in}}%
\pgfpathcurveto{\pgfqpoint{3.344222in}{1.560384in}}{\pgfqpoint{3.348172in}{1.562020in}}{\pgfqpoint{3.351084in}{1.564932in}}%
\pgfpathcurveto{\pgfqpoint{3.353996in}{1.567844in}}{\pgfqpoint{3.355632in}{1.571794in}}{\pgfqpoint{3.355632in}{1.575912in}}%
\pgfpathcurveto{\pgfqpoint{3.355632in}{1.580030in}}{\pgfqpoint{3.353996in}{1.583980in}}{\pgfqpoint{3.351084in}{1.586892in}}%
\pgfpathcurveto{\pgfqpoint{3.348172in}{1.589804in}}{\pgfqpoint{3.344222in}{1.591440in}}{\pgfqpoint{3.340103in}{1.591440in}}%
\pgfpathcurveto{\pgfqpoint{3.335985in}{1.591440in}}{\pgfqpoint{3.332035in}{1.589804in}}{\pgfqpoint{3.329123in}{1.586892in}}%
\pgfpathcurveto{\pgfqpoint{3.326211in}{1.583980in}}{\pgfqpoint{3.324575in}{1.580030in}}{\pgfqpoint{3.324575in}{1.575912in}}%
\pgfpathcurveto{\pgfqpoint{3.324575in}{1.571794in}}{\pgfqpoint{3.326211in}{1.567844in}}{\pgfqpoint{3.329123in}{1.564932in}}%
\pgfpathcurveto{\pgfqpoint{3.332035in}{1.562020in}}{\pgfqpoint{3.335985in}{1.560384in}}{\pgfqpoint{3.340103in}{1.560384in}}%
\pgfpathclose%
\pgfusepath{fill}%
\end{pgfscope}%
\begin{pgfscope}%
\pgfpathrectangle{\pgfqpoint{0.887500in}{0.275000in}}{\pgfqpoint{4.225000in}{4.225000in}}%
\pgfusepath{clip}%
\pgfsetbuttcap%
\pgfsetroundjoin%
\definecolor{currentfill}{rgb}{0.000000,0.000000,0.000000}%
\pgfsetfillcolor{currentfill}%
\pgfsetfillopacity{0.800000}%
\pgfsetlinewidth{0.000000pt}%
\definecolor{currentstroke}{rgb}{0.000000,0.000000,0.000000}%
\pgfsetstrokecolor{currentstroke}%
\pgfsetstrokeopacity{0.800000}%
\pgfsetdash{}{0pt}%
\pgfpathmoveto{\pgfqpoint{1.841613in}{1.995456in}}%
\pgfpathcurveto{\pgfqpoint{1.845731in}{1.995456in}}{\pgfqpoint{1.849681in}{1.997092in}}{\pgfqpoint{1.852593in}{2.000004in}}%
\pgfpathcurveto{\pgfqpoint{1.855505in}{2.002916in}}{\pgfqpoint{1.857141in}{2.006866in}}{\pgfqpoint{1.857141in}{2.010985in}}%
\pgfpathcurveto{\pgfqpoint{1.857141in}{2.015103in}}{\pgfqpoint{1.855505in}{2.019053in}}{\pgfqpoint{1.852593in}{2.021965in}}%
\pgfpathcurveto{\pgfqpoint{1.849681in}{2.024877in}}{\pgfqpoint{1.845731in}{2.026513in}}{\pgfqpoint{1.841613in}{2.026513in}}%
\pgfpathcurveto{\pgfqpoint{1.837495in}{2.026513in}}{\pgfqpoint{1.833545in}{2.024877in}}{\pgfqpoint{1.830633in}{2.021965in}}%
\pgfpathcurveto{\pgfqpoint{1.827721in}{2.019053in}}{\pgfqpoint{1.826085in}{2.015103in}}{\pgfqpoint{1.826085in}{2.010985in}}%
\pgfpathcurveto{\pgfqpoint{1.826085in}{2.006866in}}{\pgfqpoint{1.827721in}{2.002916in}}{\pgfqpoint{1.830633in}{2.000004in}}%
\pgfpathcurveto{\pgfqpoint{1.833545in}{1.997092in}}{\pgfqpoint{1.837495in}{1.995456in}}{\pgfqpoint{1.841613in}{1.995456in}}%
\pgfpathclose%
\pgfusepath{fill}%
\end{pgfscope}%
\begin{pgfscope}%
\pgfpathrectangle{\pgfqpoint{0.887500in}{0.275000in}}{\pgfqpoint{4.225000in}{4.225000in}}%
\pgfusepath{clip}%
\pgfsetbuttcap%
\pgfsetroundjoin%
\definecolor{currentfill}{rgb}{0.000000,0.000000,0.000000}%
\pgfsetfillcolor{currentfill}%
\pgfsetfillopacity{0.800000}%
\pgfsetlinewidth{0.000000pt}%
\definecolor{currentstroke}{rgb}{0.000000,0.000000,0.000000}%
\pgfsetstrokecolor{currentstroke}%
\pgfsetstrokeopacity{0.800000}%
\pgfsetdash{}{0pt}%
\pgfpathmoveto{\pgfqpoint{2.500292in}{1.837526in}}%
\pgfpathcurveto{\pgfqpoint{2.504411in}{1.837526in}}{\pgfqpoint{2.508361in}{1.839163in}}{\pgfqpoint{2.511273in}{1.842074in}}%
\pgfpathcurveto{\pgfqpoint{2.514185in}{1.844986in}}{\pgfqpoint{2.515821in}{1.848936in}}{\pgfqpoint{2.515821in}{1.853055in}}%
\pgfpathcurveto{\pgfqpoint{2.515821in}{1.857173in}}{\pgfqpoint{2.514185in}{1.861123in}}{\pgfqpoint{2.511273in}{1.864035in}}%
\pgfpathcurveto{\pgfqpoint{2.508361in}{1.866947in}}{\pgfqpoint{2.504411in}{1.868583in}}{\pgfqpoint{2.500292in}{1.868583in}}%
\pgfpathcurveto{\pgfqpoint{2.496174in}{1.868583in}}{\pgfqpoint{2.492224in}{1.866947in}}{\pgfqpoint{2.489312in}{1.864035in}}%
\pgfpathcurveto{\pgfqpoint{2.486400in}{1.861123in}}{\pgfqpoint{2.484764in}{1.857173in}}{\pgfqpoint{2.484764in}{1.853055in}}%
\pgfpathcurveto{\pgfqpoint{2.484764in}{1.848936in}}{\pgfqpoint{2.486400in}{1.844986in}}{\pgfqpoint{2.489312in}{1.842074in}}%
\pgfpathcurveto{\pgfqpoint{2.492224in}{1.839163in}}{\pgfqpoint{2.496174in}{1.837526in}}{\pgfqpoint{2.500292in}{1.837526in}}%
\pgfpathclose%
\pgfusepath{fill}%
\end{pgfscope}%
\begin{pgfscope}%
\pgfpathrectangle{\pgfqpoint{0.887500in}{0.275000in}}{\pgfqpoint{4.225000in}{4.225000in}}%
\pgfusepath{clip}%
\pgfsetbuttcap%
\pgfsetroundjoin%
\definecolor{currentfill}{rgb}{0.000000,0.000000,0.000000}%
\pgfsetfillcolor{currentfill}%
\pgfsetfillopacity{0.800000}%
\pgfsetlinewidth{0.000000pt}%
\definecolor{currentstroke}{rgb}{0.000000,0.000000,0.000000}%
\pgfsetstrokecolor{currentstroke}%
\pgfsetstrokeopacity{0.800000}%
\pgfsetdash{}{0pt}%
\pgfpathmoveto{\pgfqpoint{2.681310in}{1.765758in}}%
\pgfpathcurveto{\pgfqpoint{2.685428in}{1.765758in}}{\pgfqpoint{2.689378in}{1.767394in}}{\pgfqpoint{2.692290in}{1.770306in}}%
\pgfpathcurveto{\pgfqpoint{2.695202in}{1.773218in}}{\pgfqpoint{2.696838in}{1.777168in}}{\pgfqpoint{2.696838in}{1.781286in}}%
\pgfpathcurveto{\pgfqpoint{2.696838in}{1.785404in}}{\pgfqpoint{2.695202in}{1.789354in}}{\pgfqpoint{2.692290in}{1.792266in}}%
\pgfpathcurveto{\pgfqpoint{2.689378in}{1.795178in}}{\pgfqpoint{2.685428in}{1.796814in}}{\pgfqpoint{2.681310in}{1.796814in}}%
\pgfpathcurveto{\pgfqpoint{2.677192in}{1.796814in}}{\pgfqpoint{2.673242in}{1.795178in}}{\pgfqpoint{2.670330in}{1.792266in}}%
\pgfpathcurveto{\pgfqpoint{2.667418in}{1.789354in}}{\pgfqpoint{2.665782in}{1.785404in}}{\pgfqpoint{2.665782in}{1.781286in}}%
\pgfpathcurveto{\pgfqpoint{2.665782in}{1.777168in}}{\pgfqpoint{2.667418in}{1.773218in}}{\pgfqpoint{2.670330in}{1.770306in}}%
\pgfpathcurveto{\pgfqpoint{2.673242in}{1.767394in}}{\pgfqpoint{2.677192in}{1.765758in}}{\pgfqpoint{2.681310in}{1.765758in}}%
\pgfpathclose%
\pgfusepath{fill}%
\end{pgfscope}%
\begin{pgfscope}%
\pgfpathrectangle{\pgfqpoint{0.887500in}{0.275000in}}{\pgfqpoint{4.225000in}{4.225000in}}%
\pgfusepath{clip}%
\pgfsetbuttcap%
\pgfsetroundjoin%
\definecolor{currentfill}{rgb}{0.000000,0.000000,0.000000}%
\pgfsetfillcolor{currentfill}%
\pgfsetfillopacity{0.800000}%
\pgfsetlinewidth{0.000000pt}%
\definecolor{currentstroke}{rgb}{0.000000,0.000000,0.000000}%
\pgfsetstrokecolor{currentstroke}%
\pgfsetstrokeopacity{0.800000}%
\pgfsetdash{}{0pt}%
\pgfpathmoveto{\pgfqpoint{2.021968in}{1.932433in}}%
\pgfpathcurveto{\pgfqpoint{2.026086in}{1.932433in}}{\pgfqpoint{2.030036in}{1.934070in}}{\pgfqpoint{2.032948in}{1.936982in}}%
\pgfpathcurveto{\pgfqpoint{2.035860in}{1.939894in}}{\pgfqpoint{2.037496in}{1.943844in}}{\pgfqpoint{2.037496in}{1.947962in}}%
\pgfpathcurveto{\pgfqpoint{2.037496in}{1.952080in}}{\pgfqpoint{2.035860in}{1.956030in}}{\pgfqpoint{2.032948in}{1.958942in}}%
\pgfpathcurveto{\pgfqpoint{2.030036in}{1.961854in}}{\pgfqpoint{2.026086in}{1.963490in}}{\pgfqpoint{2.021968in}{1.963490in}}%
\pgfpathcurveto{\pgfqpoint{2.017850in}{1.963490in}}{\pgfqpoint{2.013900in}{1.961854in}}{\pgfqpoint{2.010988in}{1.958942in}}%
\pgfpathcurveto{\pgfqpoint{2.008076in}{1.956030in}}{\pgfqpoint{2.006440in}{1.952080in}}{\pgfqpoint{2.006440in}{1.947962in}}%
\pgfpathcurveto{\pgfqpoint{2.006440in}{1.943844in}}{\pgfqpoint{2.008076in}{1.939894in}}{\pgfqpoint{2.010988in}{1.936982in}}%
\pgfpathcurveto{\pgfqpoint{2.013900in}{1.934070in}}{\pgfqpoint{2.017850in}{1.932433in}}{\pgfqpoint{2.021968in}{1.932433in}}%
\pgfpathclose%
\pgfusepath{fill}%
\end{pgfscope}%
\begin{pgfscope}%
\pgfpathrectangle{\pgfqpoint{0.887500in}{0.275000in}}{\pgfqpoint{4.225000in}{4.225000in}}%
\pgfusepath{clip}%
\pgfsetbuttcap%
\pgfsetroundjoin%
\definecolor{currentfill}{rgb}{0.000000,0.000000,0.000000}%
\pgfsetfillcolor{currentfill}%
\pgfsetfillopacity{0.800000}%
\pgfsetlinewidth{0.000000pt}%
\definecolor{currentstroke}{rgb}{0.000000,0.000000,0.000000}%
\pgfsetstrokecolor{currentstroke}%
\pgfsetstrokeopacity{0.800000}%
\pgfsetdash{}{0pt}%
\pgfpathmoveto{\pgfqpoint{2.862585in}{1.689684in}}%
\pgfpathcurveto{\pgfqpoint{2.866703in}{1.689684in}}{\pgfqpoint{2.870653in}{1.691321in}}{\pgfqpoint{2.873565in}{1.694233in}}%
\pgfpathcurveto{\pgfqpoint{2.876477in}{1.697145in}}{\pgfqpoint{2.878113in}{1.701095in}}{\pgfqpoint{2.878113in}{1.705213in}}%
\pgfpathcurveto{\pgfqpoint{2.878113in}{1.709331in}}{\pgfqpoint{2.876477in}{1.713281in}}{\pgfqpoint{2.873565in}{1.716193in}}%
\pgfpathcurveto{\pgfqpoint{2.870653in}{1.719105in}}{\pgfqpoint{2.866703in}{1.720741in}}{\pgfqpoint{2.862585in}{1.720741in}}%
\pgfpathcurveto{\pgfqpoint{2.858467in}{1.720741in}}{\pgfqpoint{2.854517in}{1.719105in}}{\pgfqpoint{2.851605in}{1.716193in}}%
\pgfpathcurveto{\pgfqpoint{2.848693in}{1.713281in}}{\pgfqpoint{2.847057in}{1.709331in}}{\pgfqpoint{2.847057in}{1.705213in}}%
\pgfpathcurveto{\pgfqpoint{2.847057in}{1.701095in}}{\pgfqpoint{2.848693in}{1.697145in}}{\pgfqpoint{2.851605in}{1.694233in}}%
\pgfpathcurveto{\pgfqpoint{2.854517in}{1.691321in}}{\pgfqpoint{2.858467in}{1.689684in}}{\pgfqpoint{2.862585in}{1.689684in}}%
\pgfpathclose%
\pgfusepath{fill}%
\end{pgfscope}%
\begin{pgfscope}%
\pgfpathrectangle{\pgfqpoint{0.887500in}{0.275000in}}{\pgfqpoint{4.225000in}{4.225000in}}%
\pgfusepath{clip}%
\pgfsetbuttcap%
\pgfsetroundjoin%
\definecolor{currentfill}{rgb}{0.000000,0.000000,0.000000}%
\pgfsetfillcolor{currentfill}%
\pgfsetfillopacity{0.800000}%
\pgfsetlinewidth{0.000000pt}%
\definecolor{currentstroke}{rgb}{0.000000,0.000000,0.000000}%
\pgfsetstrokecolor{currentstroke}%
\pgfsetstrokeopacity{0.800000}%
\pgfsetdash{}{0pt}%
\pgfpathmoveto{\pgfqpoint{3.044059in}{1.608896in}}%
\pgfpathcurveto{\pgfqpoint{3.048178in}{1.608896in}}{\pgfqpoint{3.052128in}{1.610532in}}{\pgfqpoint{3.055040in}{1.613444in}}%
\pgfpathcurveto{\pgfqpoint{3.057951in}{1.616356in}}{\pgfqpoint{3.059588in}{1.620306in}}{\pgfqpoint{3.059588in}{1.624424in}}%
\pgfpathcurveto{\pgfqpoint{3.059588in}{1.628542in}}{\pgfqpoint{3.057951in}{1.632492in}}{\pgfqpoint{3.055040in}{1.635404in}}%
\pgfpathcurveto{\pgfqpoint{3.052128in}{1.638316in}}{\pgfqpoint{3.048178in}{1.639952in}}{\pgfqpoint{3.044059in}{1.639952in}}%
\pgfpathcurveto{\pgfqpoint{3.039941in}{1.639952in}}{\pgfqpoint{3.035991in}{1.638316in}}{\pgfqpoint{3.033079in}{1.635404in}}%
\pgfpathcurveto{\pgfqpoint{3.030167in}{1.632492in}}{\pgfqpoint{3.028531in}{1.628542in}}{\pgfqpoint{3.028531in}{1.624424in}}%
\pgfpathcurveto{\pgfqpoint{3.028531in}{1.620306in}}{\pgfqpoint{3.030167in}{1.616356in}}{\pgfqpoint{3.033079in}{1.613444in}}%
\pgfpathcurveto{\pgfqpoint{3.035991in}{1.610532in}}{\pgfqpoint{3.039941in}{1.608896in}}{\pgfqpoint{3.044059in}{1.608896in}}%
\pgfpathclose%
\pgfusepath{fill}%
\end{pgfscope}%
\begin{pgfscope}%
\pgfpathrectangle{\pgfqpoint{0.887500in}{0.275000in}}{\pgfqpoint{4.225000in}{4.225000in}}%
\pgfusepath{clip}%
\pgfsetbuttcap%
\pgfsetroundjoin%
\definecolor{currentfill}{rgb}{0.000000,0.000000,0.000000}%
\pgfsetfillcolor{currentfill}%
\pgfsetfillopacity{0.800000}%
\pgfsetlinewidth{0.000000pt}%
\definecolor{currentstroke}{rgb}{0.000000,0.000000,0.000000}%
\pgfsetstrokecolor{currentstroke}%
\pgfsetstrokeopacity{0.800000}%
\pgfsetdash{}{0pt}%
\pgfpathmoveto{\pgfqpoint{3.770297in}{1.220232in}}%
\pgfpathcurveto{\pgfqpoint{3.774415in}{1.220232in}}{\pgfqpoint{3.778365in}{1.221868in}}{\pgfqpoint{3.781277in}{1.224780in}}%
\pgfpathcurveto{\pgfqpoint{3.784189in}{1.227692in}}{\pgfqpoint{3.785825in}{1.231642in}}{\pgfqpoint{3.785825in}{1.235760in}}%
\pgfpathcurveto{\pgfqpoint{3.785825in}{1.239878in}}{\pgfqpoint{3.784189in}{1.243828in}}{\pgfqpoint{3.781277in}{1.246740in}}%
\pgfpathcurveto{\pgfqpoint{3.778365in}{1.249652in}}{\pgfqpoint{3.774415in}{1.251288in}}{\pgfqpoint{3.770297in}{1.251288in}}%
\pgfpathcurveto{\pgfqpoint{3.766179in}{1.251288in}}{\pgfqpoint{3.762229in}{1.249652in}}{\pgfqpoint{3.759317in}{1.246740in}}%
\pgfpathcurveto{\pgfqpoint{3.756405in}{1.243828in}}{\pgfqpoint{3.754769in}{1.239878in}}{\pgfqpoint{3.754769in}{1.235760in}}%
\pgfpathcurveto{\pgfqpoint{3.754769in}{1.231642in}}{\pgfqpoint{3.756405in}{1.227692in}}{\pgfqpoint{3.759317in}{1.224780in}}%
\pgfpathcurveto{\pgfqpoint{3.762229in}{1.221868in}}{\pgfqpoint{3.766179in}{1.220232in}}{\pgfqpoint{3.770297in}{1.220232in}}%
\pgfpathclose%
\pgfusepath{fill}%
\end{pgfscope}%
\begin{pgfscope}%
\pgfpathrectangle{\pgfqpoint{0.887500in}{0.275000in}}{\pgfqpoint{4.225000in}{4.225000in}}%
\pgfusepath{clip}%
\pgfsetbuttcap%
\pgfsetroundjoin%
\definecolor{currentfill}{rgb}{0.000000,0.000000,0.000000}%
\pgfsetfillcolor{currentfill}%
\pgfsetfillopacity{0.800000}%
\pgfsetlinewidth{0.000000pt}%
\definecolor{currentstroke}{rgb}{0.000000,0.000000,0.000000}%
\pgfsetstrokecolor{currentstroke}%
\pgfsetstrokeopacity{0.800000}%
\pgfsetdash{}{0pt}%
\pgfpathmoveto{\pgfqpoint{2.202693in}{1.866796in}}%
\pgfpathcurveto{\pgfqpoint{2.206812in}{1.866796in}}{\pgfqpoint{2.210762in}{1.868432in}}{\pgfqpoint{2.213674in}{1.871344in}}%
\pgfpathcurveto{\pgfqpoint{2.216586in}{1.874256in}}{\pgfqpoint{2.218222in}{1.878206in}}{\pgfqpoint{2.218222in}{1.882324in}}%
\pgfpathcurveto{\pgfqpoint{2.218222in}{1.886442in}}{\pgfqpoint{2.216586in}{1.890392in}}{\pgfqpoint{2.213674in}{1.893304in}}%
\pgfpathcurveto{\pgfqpoint{2.210762in}{1.896216in}}{\pgfqpoint{2.206812in}{1.897852in}}{\pgfqpoint{2.202693in}{1.897852in}}%
\pgfpathcurveto{\pgfqpoint{2.198575in}{1.897852in}}{\pgfqpoint{2.194625in}{1.896216in}}{\pgfqpoint{2.191713in}{1.893304in}}%
\pgfpathcurveto{\pgfqpoint{2.188801in}{1.890392in}}{\pgfqpoint{2.187165in}{1.886442in}}{\pgfqpoint{2.187165in}{1.882324in}}%
\pgfpathcurveto{\pgfqpoint{2.187165in}{1.878206in}}{\pgfqpoint{2.188801in}{1.874256in}}{\pgfqpoint{2.191713in}{1.871344in}}%
\pgfpathcurveto{\pgfqpoint{2.194625in}{1.868432in}}{\pgfqpoint{2.198575in}{1.866796in}}{\pgfqpoint{2.202693in}{1.866796in}}%
\pgfpathclose%
\pgfusepath{fill}%
\end{pgfscope}%
\begin{pgfscope}%
\pgfpathrectangle{\pgfqpoint{0.887500in}{0.275000in}}{\pgfqpoint{4.225000in}{4.225000in}}%
\pgfusepath{clip}%
\pgfsetbuttcap%
\pgfsetroundjoin%
\definecolor{currentfill}{rgb}{0.000000,0.000000,0.000000}%
\pgfsetfillcolor{currentfill}%
\pgfsetfillopacity{0.800000}%
\pgfsetlinewidth{0.000000pt}%
\definecolor{currentstroke}{rgb}{0.000000,0.000000,0.000000}%
\pgfsetstrokecolor{currentstroke}%
\pgfsetstrokeopacity{0.800000}%
\pgfsetdash{}{0pt}%
\pgfpathmoveto{\pgfqpoint{3.225659in}{1.521445in}}%
\pgfpathcurveto{\pgfqpoint{3.229777in}{1.521445in}}{\pgfqpoint{3.233727in}{1.523081in}}{\pgfqpoint{3.236639in}{1.525993in}}%
\pgfpathcurveto{\pgfqpoint{3.239551in}{1.528905in}}{\pgfqpoint{3.241188in}{1.532855in}}{\pgfqpoint{3.241188in}{1.536973in}}%
\pgfpathcurveto{\pgfqpoint{3.241188in}{1.541091in}}{\pgfqpoint{3.239551in}{1.545041in}}{\pgfqpoint{3.236639in}{1.547953in}}%
\pgfpathcurveto{\pgfqpoint{3.233727in}{1.550865in}}{\pgfqpoint{3.229777in}{1.552501in}}{\pgfqpoint{3.225659in}{1.552501in}}%
\pgfpathcurveto{\pgfqpoint{3.221541in}{1.552501in}}{\pgfqpoint{3.217591in}{1.550865in}}{\pgfqpoint{3.214679in}{1.547953in}}%
\pgfpathcurveto{\pgfqpoint{3.211767in}{1.545041in}}{\pgfqpoint{3.210131in}{1.541091in}}{\pgfqpoint{3.210131in}{1.536973in}}%
\pgfpathcurveto{\pgfqpoint{3.210131in}{1.532855in}}{\pgfqpoint{3.211767in}{1.528905in}}{\pgfqpoint{3.214679in}{1.525993in}}%
\pgfpathcurveto{\pgfqpoint{3.217591in}{1.523081in}}{\pgfqpoint{3.221541in}{1.521445in}}{\pgfqpoint{3.225659in}{1.521445in}}%
\pgfpathclose%
\pgfusepath{fill}%
\end{pgfscope}%
\begin{pgfscope}%
\pgfpathrectangle{\pgfqpoint{0.887500in}{0.275000in}}{\pgfqpoint{4.225000in}{4.225000in}}%
\pgfusepath{clip}%
\pgfsetbuttcap%
\pgfsetroundjoin%
\definecolor{currentfill}{rgb}{0.000000,0.000000,0.000000}%
\pgfsetfillcolor{currentfill}%
\pgfsetfillopacity{0.800000}%
\pgfsetlinewidth{0.000000pt}%
\definecolor{currentstroke}{rgb}{0.000000,0.000000,0.000000}%
\pgfsetstrokecolor{currentstroke}%
\pgfsetstrokeopacity{0.800000}%
\pgfsetdash{}{0pt}%
\pgfpathmoveto{\pgfqpoint{3.407263in}{1.424127in}}%
\pgfpathcurveto{\pgfqpoint{3.411381in}{1.424127in}}{\pgfqpoint{3.415331in}{1.425763in}}{\pgfqpoint{3.418243in}{1.428675in}}%
\pgfpathcurveto{\pgfqpoint{3.421155in}{1.431587in}}{\pgfqpoint{3.422791in}{1.435537in}}{\pgfqpoint{3.422791in}{1.439655in}}%
\pgfpathcurveto{\pgfqpoint{3.422791in}{1.443773in}}{\pgfqpoint{3.421155in}{1.447723in}}{\pgfqpoint{3.418243in}{1.450635in}}%
\pgfpathcurveto{\pgfqpoint{3.415331in}{1.453547in}}{\pgfqpoint{3.411381in}{1.455183in}}{\pgfqpoint{3.407263in}{1.455183in}}%
\pgfpathcurveto{\pgfqpoint{3.403144in}{1.455183in}}{\pgfqpoint{3.399194in}{1.453547in}}{\pgfqpoint{3.396282in}{1.450635in}}%
\pgfpathcurveto{\pgfqpoint{3.393370in}{1.447723in}}{\pgfqpoint{3.391734in}{1.443773in}}{\pgfqpoint{3.391734in}{1.439655in}}%
\pgfpathcurveto{\pgfqpoint{3.391734in}{1.435537in}}{\pgfqpoint{3.393370in}{1.431587in}}{\pgfqpoint{3.396282in}{1.428675in}}%
\pgfpathcurveto{\pgfqpoint{3.399194in}{1.425763in}}{\pgfqpoint{3.403144in}{1.424127in}}{\pgfqpoint{3.407263in}{1.424127in}}%
\pgfpathclose%
\pgfusepath{fill}%
\end{pgfscope}%
\begin{pgfscope}%
\pgfpathrectangle{\pgfqpoint{0.887500in}{0.275000in}}{\pgfqpoint{4.225000in}{4.225000in}}%
\pgfusepath{clip}%
\pgfsetbuttcap%
\pgfsetroundjoin%
\definecolor{currentfill}{rgb}{0.000000,0.000000,0.000000}%
\pgfsetfillcolor{currentfill}%
\pgfsetfillopacity{0.800000}%
\pgfsetlinewidth{0.000000pt}%
\definecolor{currentstroke}{rgb}{0.000000,0.000000,0.000000}%
\pgfsetstrokecolor{currentstroke}%
\pgfsetstrokeopacity{0.800000}%
\pgfsetdash{}{0pt}%
\pgfpathmoveto{\pgfqpoint{3.588875in}{1.327490in}}%
\pgfpathcurveto{\pgfqpoint{3.592993in}{1.327490in}}{\pgfqpoint{3.596943in}{1.329126in}}{\pgfqpoint{3.599855in}{1.332038in}}%
\pgfpathcurveto{\pgfqpoint{3.602767in}{1.334950in}}{\pgfqpoint{3.604404in}{1.338900in}}{\pgfqpoint{3.604404in}{1.343018in}}%
\pgfpathcurveto{\pgfqpoint{3.604404in}{1.347136in}}{\pgfqpoint{3.602767in}{1.351086in}}{\pgfqpoint{3.599855in}{1.353998in}}%
\pgfpathcurveto{\pgfqpoint{3.596943in}{1.356910in}}{\pgfqpoint{3.592993in}{1.358547in}}{\pgfqpoint{3.588875in}{1.358547in}}%
\pgfpathcurveto{\pgfqpoint{3.584757in}{1.358547in}}{\pgfqpoint{3.580807in}{1.356910in}}{\pgfqpoint{3.577895in}{1.353998in}}%
\pgfpathcurveto{\pgfqpoint{3.574983in}{1.351086in}}{\pgfqpoint{3.573347in}{1.347136in}}{\pgfqpoint{3.573347in}{1.343018in}}%
\pgfpathcurveto{\pgfqpoint{3.573347in}{1.338900in}}{\pgfqpoint{3.574983in}{1.334950in}}{\pgfqpoint{3.577895in}{1.332038in}}%
\pgfpathcurveto{\pgfqpoint{3.580807in}{1.329126in}}{\pgfqpoint{3.584757in}{1.327490in}}{\pgfqpoint{3.588875in}{1.327490in}}%
\pgfpathclose%
\pgfusepath{fill}%
\end{pgfscope}%
\begin{pgfscope}%
\pgfpathrectangle{\pgfqpoint{0.887500in}{0.275000in}}{\pgfqpoint{4.225000in}{4.225000in}}%
\pgfusepath{clip}%
\pgfsetbuttcap%
\pgfsetroundjoin%
\definecolor{currentfill}{rgb}{0.000000,0.000000,0.000000}%
\pgfsetfillcolor{currentfill}%
\pgfsetfillopacity{0.800000}%
\pgfsetlinewidth{0.000000pt}%
\definecolor{currentstroke}{rgb}{0.000000,0.000000,0.000000}%
\pgfsetstrokecolor{currentstroke}%
\pgfsetstrokeopacity{0.800000}%
\pgfsetdash{}{0pt}%
\pgfpathmoveto{\pgfqpoint{2.383760in}{1.798291in}}%
\pgfpathcurveto{\pgfqpoint{2.387878in}{1.798291in}}{\pgfqpoint{2.391828in}{1.799927in}}{\pgfqpoint{2.394740in}{1.802839in}}%
\pgfpathcurveto{\pgfqpoint{2.397652in}{1.805751in}}{\pgfqpoint{2.399288in}{1.809701in}}{\pgfqpoint{2.399288in}{1.813819in}}%
\pgfpathcurveto{\pgfqpoint{2.399288in}{1.817937in}}{\pgfqpoint{2.397652in}{1.821887in}}{\pgfqpoint{2.394740in}{1.824799in}}%
\pgfpathcurveto{\pgfqpoint{2.391828in}{1.827711in}}{\pgfqpoint{2.387878in}{1.829347in}}{\pgfqpoint{2.383760in}{1.829347in}}%
\pgfpathcurveto{\pgfqpoint{2.379642in}{1.829347in}}{\pgfqpoint{2.375692in}{1.827711in}}{\pgfqpoint{2.372780in}{1.824799in}}%
\pgfpathcurveto{\pgfqpoint{2.369868in}{1.821887in}}{\pgfqpoint{2.368232in}{1.817937in}}{\pgfqpoint{2.368232in}{1.813819in}}%
\pgfpathcurveto{\pgfqpoint{2.368232in}{1.809701in}}{\pgfqpoint{2.369868in}{1.805751in}}{\pgfqpoint{2.372780in}{1.802839in}}%
\pgfpathcurveto{\pgfqpoint{2.375692in}{1.799927in}}{\pgfqpoint{2.379642in}{1.798291in}}{\pgfqpoint{2.383760in}{1.798291in}}%
\pgfpathclose%
\pgfusepath{fill}%
\end{pgfscope}%
\begin{pgfscope}%
\pgfpathrectangle{\pgfqpoint{0.887500in}{0.275000in}}{\pgfqpoint{4.225000in}{4.225000in}}%
\pgfusepath{clip}%
\pgfsetbuttcap%
\pgfsetroundjoin%
\definecolor{currentfill}{rgb}{0.000000,0.000000,0.000000}%
\pgfsetfillcolor{currentfill}%
\pgfsetfillopacity{0.800000}%
\pgfsetlinewidth{0.000000pt}%
\definecolor{currentstroke}{rgb}{0.000000,0.000000,0.000000}%
\pgfsetstrokecolor{currentstroke}%
\pgfsetstrokeopacity{0.800000}%
\pgfsetdash{}{0pt}%
\pgfpathmoveto{\pgfqpoint{2.565138in}{1.726237in}}%
\pgfpathcurveto{\pgfqpoint{2.569256in}{1.726237in}}{\pgfqpoint{2.573206in}{1.727874in}}{\pgfqpoint{2.576118in}{1.730786in}}%
\pgfpathcurveto{\pgfqpoint{2.579030in}{1.733698in}}{\pgfqpoint{2.580666in}{1.737648in}}{\pgfqpoint{2.580666in}{1.741766in}}%
\pgfpathcurveto{\pgfqpoint{2.580666in}{1.745884in}}{\pgfqpoint{2.579030in}{1.749834in}}{\pgfqpoint{2.576118in}{1.752746in}}%
\pgfpathcurveto{\pgfqpoint{2.573206in}{1.755658in}}{\pgfqpoint{2.569256in}{1.757294in}}{\pgfqpoint{2.565138in}{1.757294in}}%
\pgfpathcurveto{\pgfqpoint{2.561020in}{1.757294in}}{\pgfqpoint{2.557070in}{1.755658in}}{\pgfqpoint{2.554158in}{1.752746in}}%
\pgfpathcurveto{\pgfqpoint{2.551246in}{1.749834in}}{\pgfqpoint{2.549610in}{1.745884in}}{\pgfqpoint{2.549610in}{1.741766in}}%
\pgfpathcurveto{\pgfqpoint{2.549610in}{1.737648in}}{\pgfqpoint{2.551246in}{1.733698in}}{\pgfqpoint{2.554158in}{1.730786in}}%
\pgfpathcurveto{\pgfqpoint{2.557070in}{1.727874in}}{\pgfqpoint{2.561020in}{1.726237in}}{\pgfqpoint{2.565138in}{1.726237in}}%
\pgfpathclose%
\pgfusepath{fill}%
\end{pgfscope}%
\begin{pgfscope}%
\pgfpathrectangle{\pgfqpoint{0.887500in}{0.275000in}}{\pgfqpoint{4.225000in}{4.225000in}}%
\pgfusepath{clip}%
\pgfsetbuttcap%
\pgfsetroundjoin%
\definecolor{currentfill}{rgb}{0.000000,0.000000,0.000000}%
\pgfsetfillcolor{currentfill}%
\pgfsetfillopacity{0.800000}%
\pgfsetlinewidth{0.000000pt}%
\definecolor{currentstroke}{rgb}{0.000000,0.000000,0.000000}%
\pgfsetstrokecolor{currentstroke}%
\pgfsetstrokeopacity{0.800000}%
\pgfsetdash{}{0pt}%
\pgfpathmoveto{\pgfqpoint{1.904171in}{1.893826in}}%
\pgfpathcurveto{\pgfqpoint{1.908289in}{1.893826in}}{\pgfqpoint{1.912239in}{1.895462in}}{\pgfqpoint{1.915151in}{1.898374in}}%
\pgfpathcurveto{\pgfqpoint{1.918063in}{1.901286in}}{\pgfqpoint{1.919699in}{1.905236in}}{\pgfqpoint{1.919699in}{1.909354in}}%
\pgfpathcurveto{\pgfqpoint{1.919699in}{1.913472in}}{\pgfqpoint{1.918063in}{1.917422in}}{\pgfqpoint{1.915151in}{1.920334in}}%
\pgfpathcurveto{\pgfqpoint{1.912239in}{1.923246in}}{\pgfqpoint{1.908289in}{1.924882in}}{\pgfqpoint{1.904171in}{1.924882in}}%
\pgfpathcurveto{\pgfqpoint{1.900053in}{1.924882in}}{\pgfqpoint{1.896103in}{1.923246in}}{\pgfqpoint{1.893191in}{1.920334in}}%
\pgfpathcurveto{\pgfqpoint{1.890279in}{1.917422in}}{\pgfqpoint{1.888642in}{1.913472in}}{\pgfqpoint{1.888642in}{1.909354in}}%
\pgfpathcurveto{\pgfqpoint{1.888642in}{1.905236in}}{\pgfqpoint{1.890279in}{1.901286in}}{\pgfqpoint{1.893191in}{1.898374in}}%
\pgfpathcurveto{\pgfqpoint{1.896103in}{1.895462in}}{\pgfqpoint{1.900053in}{1.893826in}}{\pgfqpoint{1.904171in}{1.893826in}}%
\pgfpathclose%
\pgfusepath{fill}%
\end{pgfscope}%
\begin{pgfscope}%
\pgfpathrectangle{\pgfqpoint{0.887500in}{0.275000in}}{\pgfqpoint{4.225000in}{4.225000in}}%
\pgfusepath{clip}%
\pgfsetbuttcap%
\pgfsetroundjoin%
\definecolor{currentfill}{rgb}{0.000000,0.000000,0.000000}%
\pgfsetfillcolor{currentfill}%
\pgfsetfillopacity{0.800000}%
\pgfsetlinewidth{0.000000pt}%
\definecolor{currentstroke}{rgb}{0.000000,0.000000,0.000000}%
\pgfsetstrokecolor{currentstroke}%
\pgfsetstrokeopacity{0.800000}%
\pgfsetdash{}{0pt}%
\pgfpathmoveto{\pgfqpoint{3.656240in}{1.154204in}}%
\pgfpathcurveto{\pgfqpoint{3.660358in}{1.154204in}}{\pgfqpoint{3.664308in}{1.155841in}}{\pgfqpoint{3.667220in}{1.158753in}}%
\pgfpathcurveto{\pgfqpoint{3.670132in}{1.161664in}}{\pgfqpoint{3.671768in}{1.165615in}}{\pgfqpoint{3.671768in}{1.169733in}}%
\pgfpathcurveto{\pgfqpoint{3.671768in}{1.173851in}}{\pgfqpoint{3.670132in}{1.177801in}}{\pgfqpoint{3.667220in}{1.180713in}}%
\pgfpathcurveto{\pgfqpoint{3.664308in}{1.183625in}}{\pgfqpoint{3.660358in}{1.185261in}}{\pgfqpoint{3.656240in}{1.185261in}}%
\pgfpathcurveto{\pgfqpoint{3.652122in}{1.185261in}}{\pgfqpoint{3.648172in}{1.183625in}}{\pgfqpoint{3.645260in}{1.180713in}}%
\pgfpathcurveto{\pgfqpoint{3.642348in}{1.177801in}}{\pgfqpoint{3.640711in}{1.173851in}}{\pgfqpoint{3.640711in}{1.169733in}}%
\pgfpathcurveto{\pgfqpoint{3.640711in}{1.165615in}}{\pgfqpoint{3.642348in}{1.161664in}}{\pgfqpoint{3.645260in}{1.158753in}}%
\pgfpathcurveto{\pgfqpoint{3.648172in}{1.155841in}}{\pgfqpoint{3.652122in}{1.154204in}}{\pgfqpoint{3.656240in}{1.154204in}}%
\pgfpathclose%
\pgfusepath{fill}%
\end{pgfscope}%
\begin{pgfscope}%
\pgfpathrectangle{\pgfqpoint{0.887500in}{0.275000in}}{\pgfqpoint{4.225000in}{4.225000in}}%
\pgfusepath{clip}%
\pgfsetbuttcap%
\pgfsetroundjoin%
\definecolor{currentfill}{rgb}{0.000000,0.000000,0.000000}%
\pgfsetfillcolor{currentfill}%
\pgfsetfillopacity{0.800000}%
\pgfsetlinewidth{0.000000pt}%
\definecolor{currentstroke}{rgb}{0.000000,0.000000,0.000000}%
\pgfsetstrokecolor{currentstroke}%
\pgfsetstrokeopacity{0.800000}%
\pgfsetdash{}{0pt}%
\pgfpathmoveto{\pgfqpoint{2.746786in}{1.650046in}}%
\pgfpathcurveto{\pgfqpoint{2.750904in}{1.650046in}}{\pgfqpoint{2.754854in}{1.651682in}}{\pgfqpoint{2.757766in}{1.654594in}}%
\pgfpathcurveto{\pgfqpoint{2.760678in}{1.657506in}}{\pgfqpoint{2.762314in}{1.661456in}}{\pgfqpoint{2.762314in}{1.665574in}}%
\pgfpathcurveto{\pgfqpoint{2.762314in}{1.669692in}}{\pgfqpoint{2.760678in}{1.673642in}}{\pgfqpoint{2.757766in}{1.676554in}}%
\pgfpathcurveto{\pgfqpoint{2.754854in}{1.679466in}}{\pgfqpoint{2.750904in}{1.681102in}}{\pgfqpoint{2.746786in}{1.681102in}}%
\pgfpathcurveto{\pgfqpoint{2.742667in}{1.681102in}}{\pgfqpoint{2.738717in}{1.679466in}}{\pgfqpoint{2.735806in}{1.676554in}}%
\pgfpathcurveto{\pgfqpoint{2.732894in}{1.673642in}}{\pgfqpoint{2.731257in}{1.669692in}}{\pgfqpoint{2.731257in}{1.665574in}}%
\pgfpathcurveto{\pgfqpoint{2.731257in}{1.661456in}}{\pgfqpoint{2.732894in}{1.657506in}}{\pgfqpoint{2.735806in}{1.654594in}}%
\pgfpathcurveto{\pgfqpoint{2.738717in}{1.651682in}}{\pgfqpoint{2.742667in}{1.650046in}}{\pgfqpoint{2.746786in}{1.650046in}}%
\pgfpathclose%
\pgfusepath{fill}%
\end{pgfscope}%
\begin{pgfscope}%
\pgfpathrectangle{\pgfqpoint{0.887500in}{0.275000in}}{\pgfqpoint{4.225000in}{4.225000in}}%
\pgfusepath{clip}%
\pgfsetbuttcap%
\pgfsetroundjoin%
\definecolor{currentfill}{rgb}{0.000000,0.000000,0.000000}%
\pgfsetfillcolor{currentfill}%
\pgfsetfillopacity{0.800000}%
\pgfsetlinewidth{0.000000pt}%
\definecolor{currentstroke}{rgb}{0.000000,0.000000,0.000000}%
\pgfsetstrokecolor{currentstroke}%
\pgfsetstrokeopacity{0.800000}%
\pgfsetdash{}{0pt}%
\pgfpathmoveto{\pgfqpoint{3.474621in}{1.275438in}}%
\pgfpathcurveto{\pgfqpoint{3.478739in}{1.275438in}}{\pgfqpoint{3.482689in}{1.277074in}}{\pgfqpoint{3.485601in}{1.279986in}}%
\pgfpathcurveto{\pgfqpoint{3.488513in}{1.282898in}}{\pgfqpoint{3.490149in}{1.286848in}}{\pgfqpoint{3.490149in}{1.290967in}}%
\pgfpathcurveto{\pgfqpoint{3.490149in}{1.295085in}}{\pgfqpoint{3.488513in}{1.299035in}}{\pgfqpoint{3.485601in}{1.301947in}}%
\pgfpathcurveto{\pgfqpoint{3.482689in}{1.304859in}}{\pgfqpoint{3.478739in}{1.306495in}}{\pgfqpoint{3.474621in}{1.306495in}}%
\pgfpathcurveto{\pgfqpoint{3.470503in}{1.306495in}}{\pgfqpoint{3.466553in}{1.304859in}}{\pgfqpoint{3.463641in}{1.301947in}}%
\pgfpathcurveto{\pgfqpoint{3.460729in}{1.299035in}}{\pgfqpoint{3.459093in}{1.295085in}}{\pgfqpoint{3.459093in}{1.290967in}}%
\pgfpathcurveto{\pgfqpoint{3.459093in}{1.286848in}}{\pgfqpoint{3.460729in}{1.282898in}}{\pgfqpoint{3.463641in}{1.279986in}}%
\pgfpathcurveto{\pgfqpoint{3.466553in}{1.277074in}}{\pgfqpoint{3.470503in}{1.275438in}}{\pgfqpoint{3.474621in}{1.275438in}}%
\pgfpathclose%
\pgfusepath{fill}%
\end{pgfscope}%
\begin{pgfscope}%
\pgfpathrectangle{\pgfqpoint{0.887500in}{0.275000in}}{\pgfqpoint{4.225000in}{4.225000in}}%
\pgfusepath{clip}%
\pgfsetbuttcap%
\pgfsetroundjoin%
\definecolor{currentfill}{rgb}{0.000000,0.000000,0.000000}%
\pgfsetfillcolor{currentfill}%
\pgfsetfillopacity{0.800000}%
\pgfsetlinewidth{0.000000pt}%
\definecolor{currentstroke}{rgb}{0.000000,0.000000,0.000000}%
\pgfsetstrokecolor{currentstroke}%
\pgfsetstrokeopacity{0.800000}%
\pgfsetdash{}{0pt}%
\pgfpathmoveto{\pgfqpoint{2.928648in}{1.568959in}}%
\pgfpathcurveto{\pgfqpoint{2.932766in}{1.568959in}}{\pgfqpoint{2.936716in}{1.570595in}}{\pgfqpoint{2.939628in}{1.573507in}}%
\pgfpathcurveto{\pgfqpoint{2.942540in}{1.576419in}}{\pgfqpoint{2.944177in}{1.580369in}}{\pgfqpoint{2.944177in}{1.584487in}}%
\pgfpathcurveto{\pgfqpoint{2.944177in}{1.588606in}}{\pgfqpoint{2.942540in}{1.592556in}}{\pgfqpoint{2.939628in}{1.595468in}}%
\pgfpathcurveto{\pgfqpoint{2.936716in}{1.598380in}}{\pgfqpoint{2.932766in}{1.600016in}}{\pgfqpoint{2.928648in}{1.600016in}}%
\pgfpathcurveto{\pgfqpoint{2.924530in}{1.600016in}}{\pgfqpoint{2.920580in}{1.598380in}}{\pgfqpoint{2.917668in}{1.595468in}}%
\pgfpathcurveto{\pgfqpoint{2.914756in}{1.592556in}}{\pgfqpoint{2.913120in}{1.588606in}}{\pgfqpoint{2.913120in}{1.584487in}}%
\pgfpathcurveto{\pgfqpoint{2.913120in}{1.580369in}}{\pgfqpoint{2.914756in}{1.576419in}}{\pgfqpoint{2.917668in}{1.573507in}}%
\pgfpathcurveto{\pgfqpoint{2.920580in}{1.570595in}}{\pgfqpoint{2.924530in}{1.568959in}}{\pgfqpoint{2.928648in}{1.568959in}}%
\pgfpathclose%
\pgfusepath{fill}%
\end{pgfscope}%
\begin{pgfscope}%
\pgfpathrectangle{\pgfqpoint{0.887500in}{0.275000in}}{\pgfqpoint{4.225000in}{4.225000in}}%
\pgfusepath{clip}%
\pgfsetbuttcap%
\pgfsetroundjoin%
\definecolor{currentfill}{rgb}{0.000000,0.000000,0.000000}%
\pgfsetfillcolor{currentfill}%
\pgfsetfillopacity{0.800000}%
\pgfsetlinewidth{0.000000pt}%
\definecolor{currentstroke}{rgb}{0.000000,0.000000,0.000000}%
\pgfsetstrokecolor{currentstroke}%
\pgfsetstrokeopacity{0.800000}%
\pgfsetdash{}{0pt}%
\pgfpathmoveto{\pgfqpoint{2.085242in}{1.827825in}}%
\pgfpathcurveto{\pgfqpoint{2.089361in}{1.827825in}}{\pgfqpoint{2.093311in}{1.829461in}}{\pgfqpoint{2.096223in}{1.832373in}}%
\pgfpathcurveto{\pgfqpoint{2.099135in}{1.835285in}}{\pgfqpoint{2.100771in}{1.839235in}}{\pgfqpoint{2.100771in}{1.843353in}}%
\pgfpathcurveto{\pgfqpoint{2.100771in}{1.847471in}}{\pgfqpoint{2.099135in}{1.851421in}}{\pgfqpoint{2.096223in}{1.854333in}}%
\pgfpathcurveto{\pgfqpoint{2.093311in}{1.857245in}}{\pgfqpoint{2.089361in}{1.858881in}}{\pgfqpoint{2.085242in}{1.858881in}}%
\pgfpathcurveto{\pgfqpoint{2.081124in}{1.858881in}}{\pgfqpoint{2.077174in}{1.857245in}}{\pgfqpoint{2.074262in}{1.854333in}}%
\pgfpathcurveto{\pgfqpoint{2.071350in}{1.851421in}}{\pgfqpoint{2.069714in}{1.847471in}}{\pgfqpoint{2.069714in}{1.843353in}}%
\pgfpathcurveto{\pgfqpoint{2.069714in}{1.839235in}}{\pgfqpoint{2.071350in}{1.835285in}}{\pgfqpoint{2.074262in}{1.832373in}}%
\pgfpathcurveto{\pgfqpoint{2.077174in}{1.829461in}}{\pgfqpoint{2.081124in}{1.827825in}}{\pgfqpoint{2.085242in}{1.827825in}}%
\pgfpathclose%
\pgfusepath{fill}%
\end{pgfscope}%
\begin{pgfscope}%
\pgfpathrectangle{\pgfqpoint{0.887500in}{0.275000in}}{\pgfqpoint{4.225000in}{4.225000in}}%
\pgfusepath{clip}%
\pgfsetbuttcap%
\pgfsetroundjoin%
\definecolor{currentfill}{rgb}{0.000000,0.000000,0.000000}%
\pgfsetfillcolor{currentfill}%
\pgfsetfillopacity{0.800000}%
\pgfsetlinewidth{0.000000pt}%
\definecolor{currentstroke}{rgb}{0.000000,0.000000,0.000000}%
\pgfsetstrokecolor{currentstroke}%
\pgfsetstrokeopacity{0.800000}%
\pgfsetdash{}{0pt}%
\pgfpathmoveto{\pgfqpoint{3.110657in}{1.481545in}}%
\pgfpathcurveto{\pgfqpoint{3.114775in}{1.481545in}}{\pgfqpoint{3.118725in}{1.483181in}}{\pgfqpoint{3.121637in}{1.486093in}}%
\pgfpathcurveto{\pgfqpoint{3.124549in}{1.489005in}}{\pgfqpoint{3.126185in}{1.492955in}}{\pgfqpoint{3.126185in}{1.497073in}}%
\pgfpathcurveto{\pgfqpoint{3.126185in}{1.501191in}}{\pgfqpoint{3.124549in}{1.505141in}}{\pgfqpoint{3.121637in}{1.508053in}}%
\pgfpathcurveto{\pgfqpoint{3.118725in}{1.510965in}}{\pgfqpoint{3.114775in}{1.512601in}}{\pgfqpoint{3.110657in}{1.512601in}}%
\pgfpathcurveto{\pgfqpoint{3.106538in}{1.512601in}}{\pgfqpoint{3.102588in}{1.510965in}}{\pgfqpoint{3.099676in}{1.508053in}}%
\pgfpathcurveto{\pgfqpoint{3.096764in}{1.505141in}}{\pgfqpoint{3.095128in}{1.501191in}}{\pgfqpoint{3.095128in}{1.497073in}}%
\pgfpathcurveto{\pgfqpoint{3.095128in}{1.492955in}}{\pgfqpoint{3.096764in}{1.489005in}}{\pgfqpoint{3.099676in}{1.486093in}}%
\pgfpathcurveto{\pgfqpoint{3.102588in}{1.483181in}}{\pgfqpoint{3.106538in}{1.481545in}}{\pgfqpoint{3.110657in}{1.481545in}}%
\pgfpathclose%
\pgfusepath{fill}%
\end{pgfscope}%
\begin{pgfscope}%
\pgfpathrectangle{\pgfqpoint{0.887500in}{0.275000in}}{\pgfqpoint{4.225000in}{4.225000in}}%
\pgfusepath{clip}%
\pgfsetbuttcap%
\pgfsetroundjoin%
\definecolor{currentfill}{rgb}{0.000000,0.000000,0.000000}%
\pgfsetfillcolor{currentfill}%
\pgfsetfillopacity{0.800000}%
\pgfsetlinewidth{0.000000pt}%
\definecolor{currentstroke}{rgb}{0.000000,0.000000,0.000000}%
\pgfsetstrokecolor{currentstroke}%
\pgfsetstrokeopacity{0.800000}%
\pgfsetdash{}{0pt}%
\pgfpathmoveto{\pgfqpoint{3.292706in}{1.385274in}}%
\pgfpathcurveto{\pgfqpoint{3.296824in}{1.385274in}}{\pgfqpoint{3.300775in}{1.386910in}}{\pgfqpoint{3.303686in}{1.389822in}}%
\pgfpathcurveto{\pgfqpoint{3.306598in}{1.392734in}}{\pgfqpoint{3.308235in}{1.396684in}}{\pgfqpoint{3.308235in}{1.400802in}}%
\pgfpathcurveto{\pgfqpoint{3.308235in}{1.404921in}}{\pgfqpoint{3.306598in}{1.408871in}}{\pgfqpoint{3.303686in}{1.411783in}}%
\pgfpathcurveto{\pgfqpoint{3.300775in}{1.414695in}}{\pgfqpoint{3.296824in}{1.416331in}}{\pgfqpoint{3.292706in}{1.416331in}}%
\pgfpathcurveto{\pgfqpoint{3.288588in}{1.416331in}}{\pgfqpoint{3.284638in}{1.414695in}}{\pgfqpoint{3.281726in}{1.411783in}}%
\pgfpathcurveto{\pgfqpoint{3.278814in}{1.408871in}}{\pgfqpoint{3.277178in}{1.404921in}}{\pgfqpoint{3.277178in}{1.400802in}}%
\pgfpathcurveto{\pgfqpoint{3.277178in}{1.396684in}}{\pgfqpoint{3.278814in}{1.392734in}}{\pgfqpoint{3.281726in}{1.389822in}}%
\pgfpathcurveto{\pgfqpoint{3.284638in}{1.386910in}}{\pgfqpoint{3.288588in}{1.385274in}}{\pgfqpoint{3.292706in}{1.385274in}}%
\pgfpathclose%
\pgfusepath{fill}%
\end{pgfscope}%
\begin{pgfscope}%
\pgfpathrectangle{\pgfqpoint{0.887500in}{0.275000in}}{\pgfqpoint{4.225000in}{4.225000in}}%
\pgfusepath{clip}%
\pgfsetbuttcap%
\pgfsetroundjoin%
\definecolor{currentfill}{rgb}{0.000000,0.000000,0.000000}%
\pgfsetfillcolor{currentfill}%
\pgfsetfillopacity{0.800000}%
\pgfsetlinewidth{0.000000pt}%
\definecolor{currentstroke}{rgb}{0.000000,0.000000,0.000000}%
\pgfsetstrokecolor{currentstroke}%
\pgfsetstrokeopacity{0.800000}%
\pgfsetdash{}{0pt}%
\pgfpathmoveto{\pgfqpoint{2.266670in}{1.758607in}}%
\pgfpathcurveto{\pgfqpoint{2.270788in}{1.758607in}}{\pgfqpoint{2.274738in}{1.760243in}}{\pgfqpoint{2.277650in}{1.763155in}}%
\pgfpathcurveto{\pgfqpoint{2.280562in}{1.766067in}}{\pgfqpoint{2.282199in}{1.770017in}}{\pgfqpoint{2.282199in}{1.774135in}}%
\pgfpathcurveto{\pgfqpoint{2.282199in}{1.778253in}}{\pgfqpoint{2.280562in}{1.782203in}}{\pgfqpoint{2.277650in}{1.785115in}}%
\pgfpathcurveto{\pgfqpoint{2.274738in}{1.788027in}}{\pgfqpoint{2.270788in}{1.789663in}}{\pgfqpoint{2.266670in}{1.789663in}}%
\pgfpathcurveto{\pgfqpoint{2.262552in}{1.789663in}}{\pgfqpoint{2.258602in}{1.788027in}}{\pgfqpoint{2.255690in}{1.785115in}}%
\pgfpathcurveto{\pgfqpoint{2.252778in}{1.782203in}}{\pgfqpoint{2.251142in}{1.778253in}}{\pgfqpoint{2.251142in}{1.774135in}}%
\pgfpathcurveto{\pgfqpoint{2.251142in}{1.770017in}}{\pgfqpoint{2.252778in}{1.766067in}}{\pgfqpoint{2.255690in}{1.763155in}}%
\pgfpathcurveto{\pgfqpoint{2.258602in}{1.760243in}}{\pgfqpoint{2.262552in}{1.758607in}}{\pgfqpoint{2.266670in}{1.758607in}}%
\pgfpathclose%
\pgfusepath{fill}%
\end{pgfscope}%
\begin{pgfscope}%
\pgfpathrectangle{\pgfqpoint{0.887500in}{0.275000in}}{\pgfqpoint{4.225000in}{4.225000in}}%
\pgfusepath{clip}%
\pgfsetbuttcap%
\pgfsetroundjoin%
\definecolor{currentfill}{rgb}{0.000000,0.000000,0.000000}%
\pgfsetfillcolor{currentfill}%
\pgfsetfillopacity{0.800000}%
\pgfsetlinewidth{0.000000pt}%
\definecolor{currentstroke}{rgb}{0.000000,0.000000,0.000000}%
\pgfsetstrokecolor{currentstroke}%
\pgfsetstrokeopacity{0.800000}%
\pgfsetdash{}{0pt}%
\pgfpathmoveto{\pgfqpoint{2.448411in}{1.686181in}}%
\pgfpathcurveto{\pgfqpoint{2.452530in}{1.686181in}}{\pgfqpoint{2.456480in}{1.687818in}}{\pgfqpoint{2.459392in}{1.690730in}}%
\pgfpathcurveto{\pgfqpoint{2.462304in}{1.693642in}}{\pgfqpoint{2.463940in}{1.697592in}}{\pgfqpoint{2.463940in}{1.701710in}}%
\pgfpathcurveto{\pgfqpoint{2.463940in}{1.705828in}}{\pgfqpoint{2.462304in}{1.709778in}}{\pgfqpoint{2.459392in}{1.712690in}}%
\pgfpathcurveto{\pgfqpoint{2.456480in}{1.715602in}}{\pgfqpoint{2.452530in}{1.717238in}}{\pgfqpoint{2.448411in}{1.717238in}}%
\pgfpathcurveto{\pgfqpoint{2.444293in}{1.717238in}}{\pgfqpoint{2.440343in}{1.715602in}}{\pgfqpoint{2.437431in}{1.712690in}}%
\pgfpathcurveto{\pgfqpoint{2.434519in}{1.709778in}}{\pgfqpoint{2.432883in}{1.705828in}}{\pgfqpoint{2.432883in}{1.701710in}}%
\pgfpathcurveto{\pgfqpoint{2.432883in}{1.697592in}}{\pgfqpoint{2.434519in}{1.693642in}}{\pgfqpoint{2.437431in}{1.690730in}}%
\pgfpathcurveto{\pgfqpoint{2.440343in}{1.687818in}}{\pgfqpoint{2.444293in}{1.686181in}}{\pgfqpoint{2.448411in}{1.686181in}}%
\pgfpathclose%
\pgfusepath{fill}%
\end{pgfscope}%
\begin{pgfscope}%
\pgfpathrectangle{\pgfqpoint{0.887500in}{0.275000in}}{\pgfqpoint{4.225000in}{4.225000in}}%
\pgfusepath{clip}%
\pgfsetbuttcap%
\pgfsetroundjoin%
\definecolor{currentfill}{rgb}{0.000000,0.000000,0.000000}%
\pgfsetfillcolor{currentfill}%
\pgfsetfillopacity{0.800000}%
\pgfsetlinewidth{0.000000pt}%
\definecolor{currentstroke}{rgb}{0.000000,0.000000,0.000000}%
\pgfsetstrokecolor{currentstroke}%
\pgfsetstrokeopacity{0.800000}%
\pgfsetdash{}{0pt}%
\pgfpathmoveto{\pgfqpoint{3.359752in}{1.206623in}}%
\pgfpathcurveto{\pgfqpoint{3.363871in}{1.206623in}}{\pgfqpoint{3.367821in}{1.208260in}}{\pgfqpoint{3.370733in}{1.211172in}}%
\pgfpathcurveto{\pgfqpoint{3.373645in}{1.214083in}}{\pgfqpoint{3.375281in}{1.218034in}}{\pgfqpoint{3.375281in}{1.222152in}}%
\pgfpathcurveto{\pgfqpoint{3.375281in}{1.226270in}}{\pgfqpoint{3.373645in}{1.230220in}}{\pgfqpoint{3.370733in}{1.233132in}}%
\pgfpathcurveto{\pgfqpoint{3.367821in}{1.236044in}}{\pgfqpoint{3.363871in}{1.237680in}}{\pgfqpoint{3.359752in}{1.237680in}}%
\pgfpathcurveto{\pgfqpoint{3.355634in}{1.237680in}}{\pgfqpoint{3.351684in}{1.236044in}}{\pgfqpoint{3.348772in}{1.233132in}}%
\pgfpathcurveto{\pgfqpoint{3.345860in}{1.230220in}}{\pgfqpoint{3.344224in}{1.226270in}}{\pgfqpoint{3.344224in}{1.222152in}}%
\pgfpathcurveto{\pgfqpoint{3.344224in}{1.218034in}}{\pgfqpoint{3.345860in}{1.214083in}}{\pgfqpoint{3.348772in}{1.211172in}}%
\pgfpathcurveto{\pgfqpoint{3.351684in}{1.208260in}}{\pgfqpoint{3.355634in}{1.206623in}}{\pgfqpoint{3.359752in}{1.206623in}}%
\pgfpathclose%
\pgfusepath{fill}%
\end{pgfscope}%
\begin{pgfscope}%
\pgfpathrectangle{\pgfqpoint{0.887500in}{0.275000in}}{\pgfqpoint{4.225000in}{4.225000in}}%
\pgfusepath{clip}%
\pgfsetbuttcap%
\pgfsetroundjoin%
\definecolor{currentfill}{rgb}{0.000000,0.000000,0.000000}%
\pgfsetfillcolor{currentfill}%
\pgfsetfillopacity{0.800000}%
\pgfsetlinewidth{0.000000pt}%
\definecolor{currentstroke}{rgb}{0.000000,0.000000,0.000000}%
\pgfsetstrokecolor{currentstroke}%
\pgfsetstrokeopacity{0.800000}%
\pgfsetdash{}{0pt}%
\pgfpathmoveto{\pgfqpoint{3.542117in}{1.114291in}}%
\pgfpathcurveto{\pgfqpoint{3.546235in}{1.114291in}}{\pgfqpoint{3.550185in}{1.115927in}}{\pgfqpoint{3.553097in}{1.118839in}}%
\pgfpathcurveto{\pgfqpoint{3.556009in}{1.121751in}}{\pgfqpoint{3.557645in}{1.125701in}}{\pgfqpoint{3.557645in}{1.129819in}}%
\pgfpathcurveto{\pgfqpoint{3.557645in}{1.133937in}}{\pgfqpoint{3.556009in}{1.137887in}}{\pgfqpoint{3.553097in}{1.140799in}}%
\pgfpathcurveto{\pgfqpoint{3.550185in}{1.143711in}}{\pgfqpoint{3.546235in}{1.145347in}}{\pgfqpoint{3.542117in}{1.145347in}}%
\pgfpathcurveto{\pgfqpoint{3.537998in}{1.145347in}}{\pgfqpoint{3.534048in}{1.143711in}}{\pgfqpoint{3.531136in}{1.140799in}}%
\pgfpathcurveto{\pgfqpoint{3.528224in}{1.137887in}}{\pgfqpoint{3.526588in}{1.133937in}}{\pgfqpoint{3.526588in}{1.129819in}}%
\pgfpathcurveto{\pgfqpoint{3.526588in}{1.125701in}}{\pgfqpoint{3.528224in}{1.121751in}}{\pgfqpoint{3.531136in}{1.118839in}}%
\pgfpathcurveto{\pgfqpoint{3.534048in}{1.115927in}}{\pgfqpoint{3.537998in}{1.114291in}}{\pgfqpoint{3.542117in}{1.114291in}}%
\pgfpathclose%
\pgfusepath{fill}%
\end{pgfscope}%
\begin{pgfscope}%
\pgfpathrectangle{\pgfqpoint{0.887500in}{0.275000in}}{\pgfqpoint{4.225000in}{4.225000in}}%
\pgfusepath{clip}%
\pgfsetbuttcap%
\pgfsetroundjoin%
\definecolor{currentfill}{rgb}{0.000000,0.000000,0.000000}%
\pgfsetfillcolor{currentfill}%
\pgfsetfillopacity{0.800000}%
\pgfsetlinewidth{0.000000pt}%
\definecolor{currentstroke}{rgb}{0.000000,0.000000,0.000000}%
\pgfsetstrokecolor{currentstroke}%
\pgfsetstrokeopacity{0.800000}%
\pgfsetdash{}{0pt}%
\pgfpathmoveto{\pgfqpoint{2.630431in}{1.609784in}}%
\pgfpathcurveto{\pgfqpoint{2.634550in}{1.609784in}}{\pgfqpoint{2.638500in}{1.611421in}}{\pgfqpoint{2.641412in}{1.614333in}}%
\pgfpathcurveto{\pgfqpoint{2.644324in}{1.617244in}}{\pgfqpoint{2.645960in}{1.621195in}}{\pgfqpoint{2.645960in}{1.625313in}}%
\pgfpathcurveto{\pgfqpoint{2.645960in}{1.629431in}}{\pgfqpoint{2.644324in}{1.633381in}}{\pgfqpoint{2.641412in}{1.636293in}}%
\pgfpathcurveto{\pgfqpoint{2.638500in}{1.639205in}}{\pgfqpoint{2.634550in}{1.640841in}}{\pgfqpoint{2.630431in}{1.640841in}}%
\pgfpathcurveto{\pgfqpoint{2.626313in}{1.640841in}}{\pgfqpoint{2.622363in}{1.639205in}}{\pgfqpoint{2.619451in}{1.636293in}}%
\pgfpathcurveto{\pgfqpoint{2.616539in}{1.633381in}}{\pgfqpoint{2.614903in}{1.629431in}}{\pgfqpoint{2.614903in}{1.625313in}}%
\pgfpathcurveto{\pgfqpoint{2.614903in}{1.621195in}}{\pgfqpoint{2.616539in}{1.617244in}}{\pgfqpoint{2.619451in}{1.614333in}}%
\pgfpathcurveto{\pgfqpoint{2.622363in}{1.611421in}}{\pgfqpoint{2.626313in}{1.609784in}}{\pgfqpoint{2.630431in}{1.609784in}}%
\pgfpathclose%
\pgfusepath{fill}%
\end{pgfscope}%
\begin{pgfscope}%
\pgfpathrectangle{\pgfqpoint{0.887500in}{0.275000in}}{\pgfqpoint{4.225000in}{4.225000in}}%
\pgfusepath{clip}%
\pgfsetbuttcap%
\pgfsetroundjoin%
\definecolor{currentfill}{rgb}{0.000000,0.000000,0.000000}%
\pgfsetfillcolor{currentfill}%
\pgfsetfillopacity{0.800000}%
\pgfsetlinewidth{0.000000pt}%
\definecolor{currentstroke}{rgb}{0.000000,0.000000,0.000000}%
\pgfsetstrokecolor{currentstroke}%
\pgfsetstrokeopacity{0.800000}%
\pgfsetdash{}{0pt}%
\pgfpathmoveto{\pgfqpoint{2.812680in}{1.528744in}}%
\pgfpathcurveto{\pgfqpoint{2.816799in}{1.528744in}}{\pgfqpoint{2.820749in}{1.530380in}}{\pgfqpoint{2.823661in}{1.533292in}}%
\pgfpathcurveto{\pgfqpoint{2.826573in}{1.536204in}}{\pgfqpoint{2.828209in}{1.540154in}}{\pgfqpoint{2.828209in}{1.544272in}}%
\pgfpathcurveto{\pgfqpoint{2.828209in}{1.548390in}}{\pgfqpoint{2.826573in}{1.552340in}}{\pgfqpoint{2.823661in}{1.555252in}}%
\pgfpathcurveto{\pgfqpoint{2.820749in}{1.558164in}}{\pgfqpoint{2.816799in}{1.559800in}}{\pgfqpoint{2.812680in}{1.559800in}}%
\pgfpathcurveto{\pgfqpoint{2.808562in}{1.559800in}}{\pgfqpoint{2.804612in}{1.558164in}}{\pgfqpoint{2.801700in}{1.555252in}}%
\pgfpathcurveto{\pgfqpoint{2.798788in}{1.552340in}}{\pgfqpoint{2.797152in}{1.548390in}}{\pgfqpoint{2.797152in}{1.544272in}}%
\pgfpathcurveto{\pgfqpoint{2.797152in}{1.540154in}}{\pgfqpoint{2.798788in}{1.536204in}}{\pgfqpoint{2.801700in}{1.533292in}}%
\pgfpathcurveto{\pgfqpoint{2.804612in}{1.530380in}}{\pgfqpoint{2.808562in}{1.528744in}}{\pgfqpoint{2.812680in}{1.528744in}}%
\pgfpathclose%
\pgfusepath{fill}%
\end{pgfscope}%
\begin{pgfscope}%
\pgfpathrectangle{\pgfqpoint{0.887500in}{0.275000in}}{\pgfqpoint{4.225000in}{4.225000in}}%
\pgfusepath{clip}%
\pgfsetbuttcap%
\pgfsetroundjoin%
\definecolor{currentfill}{rgb}{0.000000,0.000000,0.000000}%
\pgfsetfillcolor{currentfill}%
\pgfsetfillopacity{0.800000}%
\pgfsetlinewidth{0.000000pt}%
\definecolor{currentstroke}{rgb}{0.000000,0.000000,0.000000}%
\pgfsetstrokecolor{currentstroke}%
\pgfsetstrokeopacity{0.800000}%
\pgfsetdash{}{0pt}%
\pgfpathmoveto{\pgfqpoint{1.967232in}{1.788318in}}%
\pgfpathcurveto{\pgfqpoint{1.971350in}{1.788318in}}{\pgfqpoint{1.975300in}{1.789955in}}{\pgfqpoint{1.978212in}{1.792867in}}%
\pgfpathcurveto{\pgfqpoint{1.981124in}{1.795778in}}{\pgfqpoint{1.982760in}{1.799729in}}{\pgfqpoint{1.982760in}{1.803847in}}%
\pgfpathcurveto{\pgfqpoint{1.982760in}{1.807965in}}{\pgfqpoint{1.981124in}{1.811915in}}{\pgfqpoint{1.978212in}{1.814827in}}%
\pgfpathcurveto{\pgfqpoint{1.975300in}{1.817739in}}{\pgfqpoint{1.971350in}{1.819375in}}{\pgfqpoint{1.967232in}{1.819375in}}%
\pgfpathcurveto{\pgfqpoint{1.963114in}{1.819375in}}{\pgfqpoint{1.959164in}{1.817739in}}{\pgfqpoint{1.956252in}{1.814827in}}%
\pgfpathcurveto{\pgfqpoint{1.953340in}{1.811915in}}{\pgfqpoint{1.951704in}{1.807965in}}{\pgfqpoint{1.951704in}{1.803847in}}%
\pgfpathcurveto{\pgfqpoint{1.951704in}{1.799729in}}{\pgfqpoint{1.953340in}{1.795778in}}{\pgfqpoint{1.956252in}{1.792867in}}%
\pgfpathcurveto{\pgfqpoint{1.959164in}{1.789955in}}{\pgfqpoint{1.963114in}{1.788318in}}{\pgfqpoint{1.967232in}{1.788318in}}%
\pgfpathclose%
\pgfusepath{fill}%
\end{pgfscope}%
\begin{pgfscope}%
\pgfpathrectangle{\pgfqpoint{0.887500in}{0.275000in}}{\pgfqpoint{4.225000in}{4.225000in}}%
\pgfusepath{clip}%
\pgfsetbuttcap%
\pgfsetroundjoin%
\definecolor{currentfill}{rgb}{0.000000,0.000000,0.000000}%
\pgfsetfillcolor{currentfill}%
\pgfsetfillopacity{0.800000}%
\pgfsetlinewidth{0.000000pt}%
\definecolor{currentstroke}{rgb}{0.000000,0.000000,0.000000}%
\pgfsetstrokecolor{currentstroke}%
\pgfsetstrokeopacity{0.800000}%
\pgfsetdash{}{0pt}%
\pgfpathmoveto{\pgfqpoint{2.995097in}{1.441451in}}%
\pgfpathcurveto{\pgfqpoint{2.999215in}{1.441451in}}{\pgfqpoint{3.003165in}{1.443088in}}{\pgfqpoint{3.006077in}{1.446000in}}%
\pgfpathcurveto{\pgfqpoint{3.008989in}{1.448911in}}{\pgfqpoint{3.010625in}{1.452861in}}{\pgfqpoint{3.010625in}{1.456980in}}%
\pgfpathcurveto{\pgfqpoint{3.010625in}{1.461098in}}{\pgfqpoint{3.008989in}{1.465048in}}{\pgfqpoint{3.006077in}{1.467960in}}%
\pgfpathcurveto{\pgfqpoint{3.003165in}{1.470872in}}{\pgfqpoint{2.999215in}{1.472508in}}{\pgfqpoint{2.995097in}{1.472508in}}%
\pgfpathcurveto{\pgfqpoint{2.990979in}{1.472508in}}{\pgfqpoint{2.987029in}{1.470872in}}{\pgfqpoint{2.984117in}{1.467960in}}%
\pgfpathcurveto{\pgfqpoint{2.981205in}{1.465048in}}{\pgfqpoint{2.979569in}{1.461098in}}{\pgfqpoint{2.979569in}{1.456980in}}%
\pgfpathcurveto{\pgfqpoint{2.979569in}{1.452861in}}{\pgfqpoint{2.981205in}{1.448911in}}{\pgfqpoint{2.984117in}{1.446000in}}%
\pgfpathcurveto{\pgfqpoint{2.987029in}{1.443088in}}{\pgfqpoint{2.990979in}{1.441451in}}{\pgfqpoint{2.995097in}{1.441451in}}%
\pgfpathclose%
\pgfusepath{fill}%
\end{pgfscope}%
\begin{pgfscope}%
\pgfpathrectangle{\pgfqpoint{0.887500in}{0.275000in}}{\pgfqpoint{4.225000in}{4.225000in}}%
\pgfusepath{clip}%
\pgfsetbuttcap%
\pgfsetroundjoin%
\definecolor{currentfill}{rgb}{0.000000,0.000000,0.000000}%
\pgfsetfillcolor{currentfill}%
\pgfsetfillopacity{0.800000}%
\pgfsetlinewidth{0.000000pt}%
\definecolor{currentstroke}{rgb}{0.000000,0.000000,0.000000}%
\pgfsetstrokecolor{currentstroke}%
\pgfsetstrokeopacity{0.800000}%
\pgfsetdash{}{0pt}%
\pgfpathmoveto{\pgfqpoint{3.177588in}{1.346231in}}%
\pgfpathcurveto{\pgfqpoint{3.181706in}{1.346231in}}{\pgfqpoint{3.185656in}{1.347867in}}{\pgfqpoint{3.188568in}{1.350779in}}%
\pgfpathcurveto{\pgfqpoint{3.191480in}{1.353691in}}{\pgfqpoint{3.193116in}{1.357641in}}{\pgfqpoint{3.193116in}{1.361759in}}%
\pgfpathcurveto{\pgfqpoint{3.193116in}{1.365877in}}{\pgfqpoint{3.191480in}{1.369827in}}{\pgfqpoint{3.188568in}{1.372739in}}%
\pgfpathcurveto{\pgfqpoint{3.185656in}{1.375651in}}{\pgfqpoint{3.181706in}{1.377287in}}{\pgfqpoint{3.177588in}{1.377287in}}%
\pgfpathcurveto{\pgfqpoint{3.173469in}{1.377287in}}{\pgfqpoint{3.169519in}{1.375651in}}{\pgfqpoint{3.166607in}{1.372739in}}%
\pgfpathcurveto{\pgfqpoint{3.163696in}{1.369827in}}{\pgfqpoint{3.162059in}{1.365877in}}{\pgfqpoint{3.162059in}{1.361759in}}%
\pgfpathcurveto{\pgfqpoint{3.162059in}{1.357641in}}{\pgfqpoint{3.163696in}{1.353691in}}{\pgfqpoint{3.166607in}{1.350779in}}%
\pgfpathcurveto{\pgfqpoint{3.169519in}{1.347867in}}{\pgfqpoint{3.173469in}{1.346231in}}{\pgfqpoint{3.177588in}{1.346231in}}%
\pgfpathclose%
\pgfusepath{fill}%
\end{pgfscope}%
\begin{pgfscope}%
\pgfpathrectangle{\pgfqpoint{0.887500in}{0.275000in}}{\pgfqpoint{4.225000in}{4.225000in}}%
\pgfusepath{clip}%
\pgfsetbuttcap%
\pgfsetroundjoin%
\definecolor{currentfill}{rgb}{0.000000,0.000000,0.000000}%
\pgfsetfillcolor{currentfill}%
\pgfsetfillopacity{0.800000}%
\pgfsetlinewidth{0.000000pt}%
\definecolor{currentstroke}{rgb}{0.000000,0.000000,0.000000}%
\pgfsetstrokecolor{currentstroke}%
\pgfsetstrokeopacity{0.800000}%
\pgfsetdash{}{0pt}%
\pgfpathmoveto{\pgfqpoint{2.149014in}{1.718818in}}%
\pgfpathcurveto{\pgfqpoint{2.153132in}{1.718818in}}{\pgfqpoint{2.157082in}{1.720454in}}{\pgfqpoint{2.159994in}{1.723366in}}%
\pgfpathcurveto{\pgfqpoint{2.162906in}{1.726278in}}{\pgfqpoint{2.164542in}{1.730228in}}{\pgfqpoint{2.164542in}{1.734346in}}%
\pgfpathcurveto{\pgfqpoint{2.164542in}{1.738464in}}{\pgfqpoint{2.162906in}{1.742414in}}{\pgfqpoint{2.159994in}{1.745326in}}%
\pgfpathcurveto{\pgfqpoint{2.157082in}{1.748238in}}{\pgfqpoint{2.153132in}{1.749874in}}{\pgfqpoint{2.149014in}{1.749874in}}%
\pgfpathcurveto{\pgfqpoint{2.144896in}{1.749874in}}{\pgfqpoint{2.140946in}{1.748238in}}{\pgfqpoint{2.138034in}{1.745326in}}%
\pgfpathcurveto{\pgfqpoint{2.135122in}{1.742414in}}{\pgfqpoint{2.133486in}{1.738464in}}{\pgfqpoint{2.133486in}{1.734346in}}%
\pgfpathcurveto{\pgfqpoint{2.133486in}{1.730228in}}{\pgfqpoint{2.135122in}{1.726278in}}{\pgfqpoint{2.138034in}{1.723366in}}%
\pgfpathcurveto{\pgfqpoint{2.140946in}{1.720454in}}{\pgfqpoint{2.144896in}{1.718818in}}{\pgfqpoint{2.149014in}{1.718818in}}%
\pgfpathclose%
\pgfusepath{fill}%
\end{pgfscope}%
\begin{pgfscope}%
\pgfpathrectangle{\pgfqpoint{0.887500in}{0.275000in}}{\pgfqpoint{4.225000in}{4.225000in}}%
\pgfusepath{clip}%
\pgfsetbuttcap%
\pgfsetroundjoin%
\definecolor{currentfill}{rgb}{0.000000,0.000000,0.000000}%
\pgfsetfillcolor{currentfill}%
\pgfsetfillopacity{0.800000}%
\pgfsetlinewidth{0.000000pt}%
\definecolor{currentstroke}{rgb}{0.000000,0.000000,0.000000}%
\pgfsetstrokecolor{currentstroke}%
\pgfsetstrokeopacity{0.800000}%
\pgfsetdash{}{0pt}%
\pgfpathmoveto{\pgfqpoint{3.427215in}{1.051542in}}%
\pgfpathcurveto{\pgfqpoint{3.431333in}{1.051542in}}{\pgfqpoint{3.435283in}{1.053178in}}{\pgfqpoint{3.438195in}{1.056090in}}%
\pgfpathcurveto{\pgfqpoint{3.441107in}{1.059002in}}{\pgfqpoint{3.442743in}{1.062952in}}{\pgfqpoint{3.442743in}{1.067070in}}%
\pgfpathcurveto{\pgfqpoint{3.442743in}{1.071188in}}{\pgfqpoint{3.441107in}{1.075138in}}{\pgfqpoint{3.438195in}{1.078050in}}%
\pgfpathcurveto{\pgfqpoint{3.435283in}{1.080962in}}{\pgfqpoint{3.431333in}{1.082599in}}{\pgfqpoint{3.427215in}{1.082599in}}%
\pgfpathcurveto{\pgfqpoint{3.423097in}{1.082599in}}{\pgfqpoint{3.419147in}{1.080962in}}{\pgfqpoint{3.416235in}{1.078050in}}%
\pgfpathcurveto{\pgfqpoint{3.413323in}{1.075138in}}{\pgfqpoint{3.411687in}{1.071188in}}{\pgfqpoint{3.411687in}{1.067070in}}%
\pgfpathcurveto{\pgfqpoint{3.411687in}{1.062952in}}{\pgfqpoint{3.413323in}{1.059002in}}{\pgfqpoint{3.416235in}{1.056090in}}%
\pgfpathcurveto{\pgfqpoint{3.419147in}{1.053178in}}{\pgfqpoint{3.423097in}{1.051542in}}{\pgfqpoint{3.427215in}{1.051542in}}%
\pgfpathclose%
\pgfusepath{fill}%
\end{pgfscope}%
\begin{pgfscope}%
\pgfpathrectangle{\pgfqpoint{0.887500in}{0.275000in}}{\pgfqpoint{4.225000in}{4.225000in}}%
\pgfusepath{clip}%
\pgfsetbuttcap%
\pgfsetroundjoin%
\definecolor{currentfill}{rgb}{0.000000,0.000000,0.000000}%
\pgfsetfillcolor{currentfill}%
\pgfsetfillopacity{0.800000}%
\pgfsetlinewidth{0.000000pt}%
\definecolor{currentstroke}{rgb}{0.000000,0.000000,0.000000}%
\pgfsetstrokecolor{currentstroke}%
\pgfsetstrokeopacity{0.800000}%
\pgfsetdash{}{0pt}%
\pgfpathmoveto{\pgfqpoint{2.331124in}{1.645846in}}%
\pgfpathcurveto{\pgfqpoint{2.335242in}{1.645846in}}{\pgfqpoint{2.339192in}{1.647482in}}{\pgfqpoint{2.342104in}{1.650394in}}%
\pgfpathcurveto{\pgfqpoint{2.345016in}{1.653306in}}{\pgfqpoint{2.346652in}{1.657256in}}{\pgfqpoint{2.346652in}{1.661374in}}%
\pgfpathcurveto{\pgfqpoint{2.346652in}{1.665492in}}{\pgfqpoint{2.345016in}{1.669442in}}{\pgfqpoint{2.342104in}{1.672354in}}%
\pgfpathcurveto{\pgfqpoint{2.339192in}{1.675266in}}{\pgfqpoint{2.335242in}{1.676903in}}{\pgfqpoint{2.331124in}{1.676903in}}%
\pgfpathcurveto{\pgfqpoint{2.327006in}{1.676903in}}{\pgfqpoint{2.323056in}{1.675266in}}{\pgfqpoint{2.320144in}{1.672354in}}%
\pgfpathcurveto{\pgfqpoint{2.317232in}{1.669442in}}{\pgfqpoint{2.315596in}{1.665492in}}{\pgfqpoint{2.315596in}{1.661374in}}%
\pgfpathcurveto{\pgfqpoint{2.315596in}{1.657256in}}{\pgfqpoint{2.317232in}{1.653306in}}{\pgfqpoint{2.320144in}{1.650394in}}%
\pgfpathcurveto{\pgfqpoint{2.323056in}{1.647482in}}{\pgfqpoint{2.327006in}{1.645846in}}{\pgfqpoint{2.331124in}{1.645846in}}%
\pgfpathclose%
\pgfusepath{fill}%
\end{pgfscope}%
\begin{pgfscope}%
\pgfpathrectangle{\pgfqpoint{0.887500in}{0.275000in}}{\pgfqpoint{4.225000in}{4.225000in}}%
\pgfusepath{clip}%
\pgfsetbuttcap%
\pgfsetroundjoin%
\definecolor{currentfill}{rgb}{0.000000,0.000000,0.000000}%
\pgfsetfillcolor{currentfill}%
\pgfsetfillopacity{0.800000}%
\pgfsetlinewidth{0.000000pt}%
\definecolor{currentstroke}{rgb}{0.000000,0.000000,0.000000}%
\pgfsetstrokecolor{currentstroke}%
\pgfsetstrokeopacity{0.800000}%
\pgfsetdash{}{0pt}%
\pgfpathmoveto{\pgfqpoint{2.513518in}{1.569243in}}%
\pgfpathcurveto{\pgfqpoint{2.517636in}{1.569243in}}{\pgfqpoint{2.521586in}{1.570879in}}{\pgfqpoint{2.524498in}{1.573791in}}%
\pgfpathcurveto{\pgfqpoint{2.527410in}{1.576703in}}{\pgfqpoint{2.529046in}{1.580653in}}{\pgfqpoint{2.529046in}{1.584771in}}%
\pgfpathcurveto{\pgfqpoint{2.529046in}{1.588889in}}{\pgfqpoint{2.527410in}{1.592839in}}{\pgfqpoint{2.524498in}{1.595751in}}%
\pgfpathcurveto{\pgfqpoint{2.521586in}{1.598663in}}{\pgfqpoint{2.517636in}{1.600299in}}{\pgfqpoint{2.513518in}{1.600299in}}%
\pgfpathcurveto{\pgfqpoint{2.509400in}{1.600299in}}{\pgfqpoint{2.505449in}{1.598663in}}{\pgfqpoint{2.502538in}{1.595751in}}%
\pgfpathcurveto{\pgfqpoint{2.499626in}{1.592839in}}{\pgfqpoint{2.497989in}{1.588889in}}{\pgfqpoint{2.497989in}{1.584771in}}%
\pgfpathcurveto{\pgfqpoint{2.497989in}{1.580653in}}{\pgfqpoint{2.499626in}{1.576703in}}{\pgfqpoint{2.502538in}{1.573791in}}%
\pgfpathcurveto{\pgfqpoint{2.505449in}{1.570879in}}{\pgfqpoint{2.509400in}{1.569243in}}{\pgfqpoint{2.513518in}{1.569243in}}%
\pgfpathclose%
\pgfusepath{fill}%
\end{pgfscope}%
\begin{pgfscope}%
\pgfpathrectangle{\pgfqpoint{0.887500in}{0.275000in}}{\pgfqpoint{4.225000in}{4.225000in}}%
\pgfusepath{clip}%
\pgfsetbuttcap%
\pgfsetroundjoin%
\definecolor{currentfill}{rgb}{0.000000,0.000000,0.000000}%
\pgfsetfillcolor{currentfill}%
\pgfsetfillopacity{0.800000}%
\pgfsetlinewidth{0.000000pt}%
\definecolor{currentstroke}{rgb}{0.000000,0.000000,0.000000}%
\pgfsetstrokecolor{currentstroke}%
\pgfsetstrokeopacity{0.800000}%
\pgfsetdash{}{0pt}%
\pgfpathmoveto{\pgfqpoint{2.696153in}{1.488248in}}%
\pgfpathcurveto{\pgfqpoint{2.700271in}{1.488248in}}{\pgfqpoint{2.704221in}{1.489884in}}{\pgfqpoint{2.707133in}{1.492796in}}%
\pgfpathcurveto{\pgfqpoint{2.710045in}{1.495708in}}{\pgfqpoint{2.711681in}{1.499658in}}{\pgfqpoint{2.711681in}{1.503776in}}%
\pgfpathcurveto{\pgfqpoint{2.711681in}{1.507894in}}{\pgfqpoint{2.710045in}{1.511844in}}{\pgfqpoint{2.707133in}{1.514756in}}%
\pgfpathcurveto{\pgfqpoint{2.704221in}{1.517668in}}{\pgfqpoint{2.700271in}{1.519304in}}{\pgfqpoint{2.696153in}{1.519304in}}%
\pgfpathcurveto{\pgfqpoint{2.692035in}{1.519304in}}{\pgfqpoint{2.688084in}{1.517668in}}{\pgfqpoint{2.685173in}{1.514756in}}%
\pgfpathcurveto{\pgfqpoint{2.682261in}{1.511844in}}{\pgfqpoint{2.680624in}{1.507894in}}{\pgfqpoint{2.680624in}{1.503776in}}%
\pgfpathcurveto{\pgfqpoint{2.680624in}{1.499658in}}{\pgfqpoint{2.682261in}{1.495708in}}{\pgfqpoint{2.685173in}{1.492796in}}%
\pgfpathcurveto{\pgfqpoint{2.688084in}{1.489884in}}{\pgfqpoint{2.692035in}{1.488248in}}{\pgfqpoint{2.696153in}{1.488248in}}%
\pgfpathclose%
\pgfusepath{fill}%
\end{pgfscope}%
\begin{pgfscope}%
\pgfpathrectangle{\pgfqpoint{0.887500in}{0.275000in}}{\pgfqpoint{4.225000in}{4.225000in}}%
\pgfusepath{clip}%
\pgfsetbuttcap%
\pgfsetroundjoin%
\definecolor{currentfill}{rgb}{0.000000,0.000000,0.000000}%
\pgfsetfillcolor{currentfill}%
\pgfsetfillopacity{0.800000}%
\pgfsetlinewidth{0.000000pt}%
\definecolor{currentstroke}{rgb}{0.000000,0.000000,0.000000}%
\pgfsetstrokecolor{currentstroke}%
\pgfsetstrokeopacity{0.800000}%
\pgfsetdash{}{0pt}%
\pgfpathmoveto{\pgfqpoint{2.878975in}{1.401421in}}%
\pgfpathcurveto{\pgfqpoint{2.883093in}{1.401421in}}{\pgfqpoint{2.887043in}{1.403057in}}{\pgfqpoint{2.889955in}{1.405969in}}%
\pgfpathcurveto{\pgfqpoint{2.892867in}{1.408881in}}{\pgfqpoint{2.894503in}{1.412831in}}{\pgfqpoint{2.894503in}{1.416950in}}%
\pgfpathcurveto{\pgfqpoint{2.894503in}{1.421068in}}{\pgfqpoint{2.892867in}{1.425018in}}{\pgfqpoint{2.889955in}{1.427930in}}%
\pgfpathcurveto{\pgfqpoint{2.887043in}{1.430842in}}{\pgfqpoint{2.883093in}{1.432478in}}{\pgfqpoint{2.878975in}{1.432478in}}%
\pgfpathcurveto{\pgfqpoint{2.874857in}{1.432478in}}{\pgfqpoint{2.870907in}{1.430842in}}{\pgfqpoint{2.867995in}{1.427930in}}%
\pgfpathcurveto{\pgfqpoint{2.865083in}{1.425018in}}{\pgfqpoint{2.863447in}{1.421068in}}{\pgfqpoint{2.863447in}{1.416950in}}%
\pgfpathcurveto{\pgfqpoint{2.863447in}{1.412831in}}{\pgfqpoint{2.865083in}{1.408881in}}{\pgfqpoint{2.867995in}{1.405969in}}%
\pgfpathcurveto{\pgfqpoint{2.870907in}{1.403057in}}{\pgfqpoint{2.874857in}{1.401421in}}{\pgfqpoint{2.878975in}{1.401421in}}%
\pgfpathclose%
\pgfusepath{fill}%
\end{pgfscope}%
\begin{pgfscope}%
\pgfpathrectangle{\pgfqpoint{0.887500in}{0.275000in}}{\pgfqpoint{4.225000in}{4.225000in}}%
\pgfusepath{clip}%
\pgfsetbuttcap%
\pgfsetroundjoin%
\definecolor{currentfill}{rgb}{0.000000,0.000000,0.000000}%
\pgfsetfillcolor{currentfill}%
\pgfsetfillopacity{0.800000}%
\pgfsetlinewidth{0.000000pt}%
\definecolor{currentstroke}{rgb}{0.000000,0.000000,0.000000}%
\pgfsetstrokecolor{currentstroke}%
\pgfsetstrokeopacity{0.800000}%
\pgfsetdash{}{0pt}%
\pgfpathmoveto{\pgfqpoint{3.061902in}{1.307596in}}%
\pgfpathcurveto{\pgfqpoint{3.066020in}{1.307596in}}{\pgfqpoint{3.069971in}{1.309232in}}{\pgfqpoint{3.072882in}{1.312144in}}%
\pgfpathcurveto{\pgfqpoint{3.075794in}{1.315056in}}{\pgfqpoint{3.077431in}{1.319006in}}{\pgfqpoint{3.077431in}{1.323124in}}%
\pgfpathcurveto{\pgfqpoint{3.077431in}{1.327242in}}{\pgfqpoint{3.075794in}{1.331192in}}{\pgfqpoint{3.072882in}{1.334104in}}%
\pgfpathcurveto{\pgfqpoint{3.069971in}{1.337016in}}{\pgfqpoint{3.066020in}{1.338652in}}{\pgfqpoint{3.061902in}{1.338652in}}%
\pgfpathcurveto{\pgfqpoint{3.057784in}{1.338652in}}{\pgfqpoint{3.053834in}{1.337016in}}{\pgfqpoint{3.050922in}{1.334104in}}%
\pgfpathcurveto{\pgfqpoint{3.048010in}{1.331192in}}{\pgfqpoint{3.046374in}{1.327242in}}{\pgfqpoint{3.046374in}{1.323124in}}%
\pgfpathcurveto{\pgfqpoint{3.046374in}{1.319006in}}{\pgfqpoint{3.048010in}{1.315056in}}{\pgfqpoint{3.050922in}{1.312144in}}%
\pgfpathcurveto{\pgfqpoint{3.053834in}{1.309232in}}{\pgfqpoint{3.057784in}{1.307596in}}{\pgfqpoint{3.061902in}{1.307596in}}%
\pgfpathclose%
\pgfusepath{fill}%
\end{pgfscope}%
\begin{pgfscope}%
\pgfpathrectangle{\pgfqpoint{0.887500in}{0.275000in}}{\pgfqpoint{4.225000in}{4.225000in}}%
\pgfusepath{clip}%
\pgfsetbuttcap%
\pgfsetroundjoin%
\definecolor{currentfill}{rgb}{0.000000,0.000000,0.000000}%
\pgfsetfillcolor{currentfill}%
\pgfsetfillopacity{0.800000}%
\pgfsetlinewidth{0.000000pt}%
\definecolor{currentstroke}{rgb}{0.000000,0.000000,0.000000}%
\pgfsetstrokecolor{currentstroke}%
\pgfsetstrokeopacity{0.800000}%
\pgfsetdash{}{0pt}%
\pgfpathmoveto{\pgfqpoint{2.030795in}{1.678576in}}%
\pgfpathcurveto{\pgfqpoint{2.034913in}{1.678576in}}{\pgfqpoint{2.038863in}{1.680213in}}{\pgfqpoint{2.041775in}{1.683125in}}%
\pgfpathcurveto{\pgfqpoint{2.044687in}{1.686037in}}{\pgfqpoint{2.046323in}{1.689987in}}{\pgfqpoint{2.046323in}{1.694105in}}%
\pgfpathcurveto{\pgfqpoint{2.046323in}{1.698223in}}{\pgfqpoint{2.044687in}{1.702173in}}{\pgfqpoint{2.041775in}{1.705085in}}%
\pgfpathcurveto{\pgfqpoint{2.038863in}{1.707997in}}{\pgfqpoint{2.034913in}{1.709633in}}{\pgfqpoint{2.030795in}{1.709633in}}%
\pgfpathcurveto{\pgfqpoint{2.026677in}{1.709633in}}{\pgfqpoint{2.022727in}{1.707997in}}{\pgfqpoint{2.019815in}{1.705085in}}%
\pgfpathcurveto{\pgfqpoint{2.016903in}{1.702173in}}{\pgfqpoint{2.015267in}{1.698223in}}{\pgfqpoint{2.015267in}{1.694105in}}%
\pgfpathcurveto{\pgfqpoint{2.015267in}{1.689987in}}{\pgfqpoint{2.016903in}{1.686037in}}{\pgfqpoint{2.019815in}{1.683125in}}%
\pgfpathcurveto{\pgfqpoint{2.022727in}{1.680213in}}{\pgfqpoint{2.026677in}{1.678576in}}{\pgfqpoint{2.030795in}{1.678576in}}%
\pgfpathclose%
\pgfusepath{fill}%
\end{pgfscope}%
\begin{pgfscope}%
\pgfpathrectangle{\pgfqpoint{0.887500in}{0.275000in}}{\pgfqpoint{4.225000in}{4.225000in}}%
\pgfusepath{clip}%
\pgfsetbuttcap%
\pgfsetroundjoin%
\definecolor{currentfill}{rgb}{0.000000,0.000000,0.000000}%
\pgfsetfillcolor{currentfill}%
\pgfsetfillopacity{0.800000}%
\pgfsetlinewidth{0.000000pt}%
\definecolor{currentstroke}{rgb}{0.000000,0.000000,0.000000}%
\pgfsetstrokecolor{currentstroke}%
\pgfsetstrokeopacity{0.800000}%
\pgfsetdash{}{0pt}%
\pgfpathmoveto{\pgfqpoint{3.245028in}{1.246121in}}%
\pgfpathcurveto{\pgfqpoint{3.249147in}{1.246121in}}{\pgfqpoint{3.253097in}{1.247757in}}{\pgfqpoint{3.256009in}{1.250669in}}%
\pgfpathcurveto{\pgfqpoint{3.258920in}{1.253581in}}{\pgfqpoint{3.260557in}{1.257531in}}{\pgfqpoint{3.260557in}{1.261649in}}%
\pgfpathcurveto{\pgfqpoint{3.260557in}{1.265768in}}{\pgfqpoint{3.258920in}{1.269718in}}{\pgfqpoint{3.256009in}{1.272630in}}%
\pgfpathcurveto{\pgfqpoint{3.253097in}{1.275542in}}{\pgfqpoint{3.249147in}{1.277178in}}{\pgfqpoint{3.245028in}{1.277178in}}%
\pgfpathcurveto{\pgfqpoint{3.240910in}{1.277178in}}{\pgfqpoint{3.236960in}{1.275542in}}{\pgfqpoint{3.234048in}{1.272630in}}%
\pgfpathcurveto{\pgfqpoint{3.231136in}{1.269718in}}{\pgfqpoint{3.229500in}{1.265768in}}{\pgfqpoint{3.229500in}{1.261649in}}%
\pgfpathcurveto{\pgfqpoint{3.229500in}{1.257531in}}{\pgfqpoint{3.231136in}{1.253581in}}{\pgfqpoint{3.234048in}{1.250669in}}%
\pgfpathcurveto{\pgfqpoint{3.236960in}{1.247757in}}{\pgfqpoint{3.240910in}{1.246121in}}{\pgfqpoint{3.245028in}{1.246121in}}%
\pgfpathclose%
\pgfusepath{fill}%
\end{pgfscope}%
\begin{pgfscope}%
\pgfpathrectangle{\pgfqpoint{0.887500in}{0.275000in}}{\pgfqpoint{4.225000in}{4.225000in}}%
\pgfusepath{clip}%
\pgfsetbuttcap%
\pgfsetroundjoin%
\definecolor{currentfill}{rgb}{0.000000,0.000000,0.000000}%
\pgfsetfillcolor{currentfill}%
\pgfsetfillopacity{0.800000}%
\pgfsetlinewidth{0.000000pt}%
\definecolor{currentstroke}{rgb}{0.000000,0.000000,0.000000}%
\pgfsetstrokecolor{currentstroke}%
\pgfsetstrokeopacity{0.800000}%
\pgfsetdash{}{0pt}%
\pgfpathmoveto{\pgfqpoint{2.213267in}{1.605489in}}%
\pgfpathcurveto{\pgfqpoint{2.217385in}{1.605489in}}{\pgfqpoint{2.221335in}{1.607125in}}{\pgfqpoint{2.224247in}{1.610037in}}%
\pgfpathcurveto{\pgfqpoint{2.227159in}{1.612949in}}{\pgfqpoint{2.228795in}{1.616899in}}{\pgfqpoint{2.228795in}{1.621017in}}%
\pgfpathcurveto{\pgfqpoint{2.228795in}{1.625136in}}{\pgfqpoint{2.227159in}{1.629086in}}{\pgfqpoint{2.224247in}{1.631998in}}%
\pgfpathcurveto{\pgfqpoint{2.221335in}{1.634909in}}{\pgfqpoint{2.217385in}{1.636546in}}{\pgfqpoint{2.213267in}{1.636546in}}%
\pgfpathcurveto{\pgfqpoint{2.209149in}{1.636546in}}{\pgfqpoint{2.205199in}{1.634909in}}{\pgfqpoint{2.202287in}{1.631998in}}%
\pgfpathcurveto{\pgfqpoint{2.199375in}{1.629086in}}{\pgfqpoint{2.197739in}{1.625136in}}{\pgfqpoint{2.197739in}{1.621017in}}%
\pgfpathcurveto{\pgfqpoint{2.197739in}{1.616899in}}{\pgfqpoint{2.199375in}{1.612949in}}{\pgfqpoint{2.202287in}{1.610037in}}%
\pgfpathcurveto{\pgfqpoint{2.205199in}{1.607125in}}{\pgfqpoint{2.209149in}{1.605489in}}{\pgfqpoint{2.213267in}{1.605489in}}%
\pgfpathclose%
\pgfusepath{fill}%
\end{pgfscope}%
\begin{pgfscope}%
\pgfpathrectangle{\pgfqpoint{0.887500in}{0.275000in}}{\pgfqpoint{4.225000in}{4.225000in}}%
\pgfusepath{clip}%
\pgfsetbuttcap%
\pgfsetroundjoin%
\definecolor{currentfill}{rgb}{0.000000,0.000000,0.000000}%
\pgfsetfillcolor{currentfill}%
\pgfsetfillopacity{0.800000}%
\pgfsetlinewidth{0.000000pt}%
\definecolor{currentstroke}{rgb}{0.000000,0.000000,0.000000}%
\pgfsetstrokecolor{currentstroke}%
\pgfsetstrokeopacity{0.800000}%
\pgfsetdash{}{0pt}%
\pgfpathmoveto{\pgfqpoint{2.396038in}{1.528592in}}%
\pgfpathcurveto{\pgfqpoint{2.400156in}{1.528592in}}{\pgfqpoint{2.404106in}{1.530228in}}{\pgfqpoint{2.407018in}{1.533140in}}%
\pgfpathcurveto{\pgfqpoint{2.409930in}{1.536052in}}{\pgfqpoint{2.411566in}{1.540002in}}{\pgfqpoint{2.411566in}{1.544120in}}%
\pgfpathcurveto{\pgfqpoint{2.411566in}{1.548238in}}{\pgfqpoint{2.409930in}{1.552188in}}{\pgfqpoint{2.407018in}{1.555100in}}%
\pgfpathcurveto{\pgfqpoint{2.404106in}{1.558012in}}{\pgfqpoint{2.400156in}{1.559648in}}{\pgfqpoint{2.396038in}{1.559648in}}%
\pgfpathcurveto{\pgfqpoint{2.391920in}{1.559648in}}{\pgfqpoint{2.387970in}{1.558012in}}{\pgfqpoint{2.385058in}{1.555100in}}%
\pgfpathcurveto{\pgfqpoint{2.382146in}{1.552188in}}{\pgfqpoint{2.380510in}{1.548238in}}{\pgfqpoint{2.380510in}{1.544120in}}%
\pgfpathcurveto{\pgfqpoint{2.380510in}{1.540002in}}{\pgfqpoint{2.382146in}{1.536052in}}{\pgfqpoint{2.385058in}{1.533140in}}%
\pgfpathcurveto{\pgfqpoint{2.387970in}{1.530228in}}{\pgfqpoint{2.391920in}{1.528592in}}{\pgfqpoint{2.396038in}{1.528592in}}%
\pgfpathclose%
\pgfusepath{fill}%
\end{pgfscope}%
\begin{pgfscope}%
\pgfpathrectangle{\pgfqpoint{0.887500in}{0.275000in}}{\pgfqpoint{4.225000in}{4.225000in}}%
\pgfusepath{clip}%
\pgfsetbuttcap%
\pgfsetroundjoin%
\definecolor{currentfill}{rgb}{0.000000,0.000000,0.000000}%
\pgfsetfillcolor{currentfill}%
\pgfsetfillopacity{0.800000}%
\pgfsetlinewidth{0.000000pt}%
\definecolor{currentstroke}{rgb}{0.000000,0.000000,0.000000}%
\pgfsetstrokecolor{currentstroke}%
\pgfsetstrokeopacity{0.800000}%
\pgfsetdash{}{0pt}%
\pgfpathmoveto{\pgfqpoint{2.579058in}{1.447728in}}%
\pgfpathcurveto{\pgfqpoint{2.583177in}{1.447728in}}{\pgfqpoint{2.587127in}{1.449365in}}{\pgfqpoint{2.590039in}{1.452277in}}%
\pgfpathcurveto{\pgfqpoint{2.592950in}{1.455189in}}{\pgfqpoint{2.594587in}{1.459139in}}{\pgfqpoint{2.594587in}{1.463257in}}%
\pgfpathcurveto{\pgfqpoint{2.594587in}{1.467375in}}{\pgfqpoint{2.592950in}{1.471325in}}{\pgfqpoint{2.590039in}{1.474237in}}%
\pgfpathcurveto{\pgfqpoint{2.587127in}{1.477149in}}{\pgfqpoint{2.583177in}{1.478785in}}{\pgfqpoint{2.579058in}{1.478785in}}%
\pgfpathcurveto{\pgfqpoint{2.574940in}{1.478785in}}{\pgfqpoint{2.570990in}{1.477149in}}{\pgfqpoint{2.568078in}{1.474237in}}%
\pgfpathcurveto{\pgfqpoint{2.565166in}{1.471325in}}{\pgfqpoint{2.563530in}{1.467375in}}{\pgfqpoint{2.563530in}{1.463257in}}%
\pgfpathcurveto{\pgfqpoint{2.563530in}{1.459139in}}{\pgfqpoint{2.565166in}{1.455189in}}{\pgfqpoint{2.568078in}{1.452277in}}%
\pgfpathcurveto{\pgfqpoint{2.570990in}{1.449365in}}{\pgfqpoint{2.574940in}{1.447728in}}{\pgfqpoint{2.579058in}{1.447728in}}%
\pgfpathclose%
\pgfusepath{fill}%
\end{pgfscope}%
\begin{pgfscope}%
\pgfpathrectangle{\pgfqpoint{0.887500in}{0.275000in}}{\pgfqpoint{4.225000in}{4.225000in}}%
\pgfusepath{clip}%
\pgfsetbuttcap%
\pgfsetroundjoin%
\definecolor{currentfill}{rgb}{0.000000,0.000000,0.000000}%
\pgfsetfillcolor{currentfill}%
\pgfsetfillopacity{0.800000}%
\pgfsetlinewidth{0.000000pt}%
\definecolor{currentstroke}{rgb}{0.000000,0.000000,0.000000}%
\pgfsetstrokecolor{currentstroke}%
\pgfsetstrokeopacity{0.800000}%
\pgfsetdash{}{0pt}%
\pgfpathmoveto{\pgfqpoint{2.762284in}{1.361540in}}%
\pgfpathcurveto{\pgfqpoint{2.766402in}{1.361540in}}{\pgfqpoint{2.770352in}{1.363176in}}{\pgfqpoint{2.773264in}{1.366088in}}%
\pgfpathcurveto{\pgfqpoint{2.776176in}{1.369000in}}{\pgfqpoint{2.777812in}{1.372950in}}{\pgfqpoint{2.777812in}{1.377068in}}%
\pgfpathcurveto{\pgfqpoint{2.777812in}{1.381187in}}{\pgfqpoint{2.776176in}{1.385137in}}{\pgfqpoint{2.773264in}{1.388048in}}%
\pgfpathcurveto{\pgfqpoint{2.770352in}{1.390960in}}{\pgfqpoint{2.766402in}{1.392597in}}{\pgfqpoint{2.762284in}{1.392597in}}%
\pgfpathcurveto{\pgfqpoint{2.758165in}{1.392597in}}{\pgfqpoint{2.754215in}{1.390960in}}{\pgfqpoint{2.751303in}{1.388048in}}%
\pgfpathcurveto{\pgfqpoint{2.748391in}{1.385137in}}{\pgfqpoint{2.746755in}{1.381187in}}{\pgfqpoint{2.746755in}{1.377068in}}%
\pgfpathcurveto{\pgfqpoint{2.746755in}{1.372950in}}{\pgfqpoint{2.748391in}{1.369000in}}{\pgfqpoint{2.751303in}{1.366088in}}%
\pgfpathcurveto{\pgfqpoint{2.754215in}{1.363176in}}{\pgfqpoint{2.758165in}{1.361540in}}{\pgfqpoint{2.762284in}{1.361540in}}%
\pgfpathclose%
\pgfusepath{fill}%
\end{pgfscope}%
\begin{pgfscope}%
\pgfpathrectangle{\pgfqpoint{0.887500in}{0.275000in}}{\pgfqpoint{4.225000in}{4.225000in}}%
\pgfusepath{clip}%
\pgfsetbuttcap%
\pgfsetroundjoin%
\definecolor{currentfill}{rgb}{0.000000,0.000000,0.000000}%
\pgfsetfillcolor{currentfill}%
\pgfsetfillopacity{0.800000}%
\pgfsetlinewidth{0.000000pt}%
\definecolor{currentstroke}{rgb}{0.000000,0.000000,0.000000}%
\pgfsetstrokecolor{currentstroke}%
\pgfsetstrokeopacity{0.800000}%
\pgfsetdash{}{0pt}%
\pgfpathmoveto{\pgfqpoint{2.945641in}{1.269370in}}%
\pgfpathcurveto{\pgfqpoint{2.949759in}{1.269370in}}{\pgfqpoint{2.953709in}{1.271006in}}{\pgfqpoint{2.956621in}{1.273918in}}%
\pgfpathcurveto{\pgfqpoint{2.959533in}{1.276830in}}{\pgfqpoint{2.961169in}{1.280780in}}{\pgfqpoint{2.961169in}{1.284898in}}%
\pgfpathcurveto{\pgfqpoint{2.961169in}{1.289016in}}{\pgfqpoint{2.959533in}{1.292967in}}{\pgfqpoint{2.956621in}{1.295878in}}%
\pgfpathcurveto{\pgfqpoint{2.953709in}{1.298790in}}{\pgfqpoint{2.949759in}{1.300427in}}{\pgfqpoint{2.945641in}{1.300427in}}%
\pgfpathcurveto{\pgfqpoint{2.941523in}{1.300427in}}{\pgfqpoint{2.937573in}{1.298790in}}{\pgfqpoint{2.934661in}{1.295878in}}%
\pgfpathcurveto{\pgfqpoint{2.931749in}{1.292967in}}{\pgfqpoint{2.930113in}{1.289016in}}{\pgfqpoint{2.930113in}{1.284898in}}%
\pgfpathcurveto{\pgfqpoint{2.930113in}{1.280780in}}{\pgfqpoint{2.931749in}{1.276830in}}{\pgfqpoint{2.934661in}{1.273918in}}%
\pgfpathcurveto{\pgfqpoint{2.937573in}{1.271006in}}{\pgfqpoint{2.941523in}{1.269370in}}{\pgfqpoint{2.945641in}{1.269370in}}%
\pgfpathclose%
\pgfusepath{fill}%
\end{pgfscope}%
\begin{pgfscope}%
\pgfpathrectangle{\pgfqpoint{0.887500in}{0.275000in}}{\pgfqpoint{4.225000in}{4.225000in}}%
\pgfusepath{clip}%
\pgfsetbuttcap%
\pgfsetroundjoin%
\definecolor{currentfill}{rgb}{0.000000,0.000000,0.000000}%
\pgfsetfillcolor{currentfill}%
\pgfsetfillopacity{0.800000}%
\pgfsetlinewidth{0.000000pt}%
\definecolor{currentstroke}{rgb}{0.000000,0.000000,0.000000}%
\pgfsetstrokecolor{currentstroke}%
\pgfsetstrokeopacity{0.800000}%
\pgfsetdash{}{0pt}%
\pgfpathmoveto{\pgfqpoint{3.129059in}{1.178628in}}%
\pgfpathcurveto{\pgfqpoint{3.133177in}{1.178628in}}{\pgfqpoint{3.137127in}{1.180264in}}{\pgfqpoint{3.140039in}{1.183176in}}%
\pgfpathcurveto{\pgfqpoint{3.142951in}{1.186088in}}{\pgfqpoint{3.144587in}{1.190038in}}{\pgfqpoint{3.144587in}{1.194156in}}%
\pgfpathcurveto{\pgfqpoint{3.144587in}{1.198274in}}{\pgfqpoint{3.142951in}{1.202224in}}{\pgfqpoint{3.140039in}{1.205136in}}%
\pgfpathcurveto{\pgfqpoint{3.137127in}{1.208048in}}{\pgfqpoint{3.133177in}{1.209684in}}{\pgfqpoint{3.129059in}{1.209684in}}%
\pgfpathcurveto{\pgfqpoint{3.124941in}{1.209684in}}{\pgfqpoint{3.120991in}{1.208048in}}{\pgfqpoint{3.118079in}{1.205136in}}%
\pgfpathcurveto{\pgfqpoint{3.115167in}{1.202224in}}{\pgfqpoint{3.113531in}{1.198274in}}{\pgfqpoint{3.113531in}{1.194156in}}%
\pgfpathcurveto{\pgfqpoint{3.113531in}{1.190038in}}{\pgfqpoint{3.115167in}{1.186088in}}{\pgfqpoint{3.118079in}{1.183176in}}%
\pgfpathcurveto{\pgfqpoint{3.120991in}{1.180264in}}{\pgfqpoint{3.124941in}{1.178628in}}{\pgfqpoint{3.129059in}{1.178628in}}%
\pgfpathclose%
\pgfusepath{fill}%
\end{pgfscope}%
\begin{pgfscope}%
\pgfpathrectangle{\pgfqpoint{0.887500in}{0.275000in}}{\pgfqpoint{4.225000in}{4.225000in}}%
\pgfusepath{clip}%
\pgfsetbuttcap%
\pgfsetroundjoin%
\definecolor{currentfill}{rgb}{0.000000,0.000000,0.000000}%
\pgfsetfillcolor{currentfill}%
\pgfsetfillopacity{0.800000}%
\pgfsetlinewidth{0.000000pt}%
\definecolor{currentstroke}{rgb}{0.000000,0.000000,0.000000}%
\pgfsetstrokecolor{currentstroke}%
\pgfsetstrokeopacity{0.800000}%
\pgfsetdash{}{0pt}%
\pgfpathmoveto{\pgfqpoint{2.094841in}{1.564850in}}%
\pgfpathcurveto{\pgfqpoint{2.098959in}{1.564850in}}{\pgfqpoint{2.102909in}{1.566486in}}{\pgfqpoint{2.105821in}{1.569398in}}%
\pgfpathcurveto{\pgfqpoint{2.108733in}{1.572310in}}{\pgfqpoint{2.110369in}{1.576260in}}{\pgfqpoint{2.110369in}{1.580378in}}%
\pgfpathcurveto{\pgfqpoint{2.110369in}{1.584496in}}{\pgfqpoint{2.108733in}{1.588446in}}{\pgfqpoint{2.105821in}{1.591358in}}%
\pgfpathcurveto{\pgfqpoint{2.102909in}{1.594270in}}{\pgfqpoint{2.098959in}{1.595907in}}{\pgfqpoint{2.094841in}{1.595907in}}%
\pgfpathcurveto{\pgfqpoint{2.090723in}{1.595907in}}{\pgfqpoint{2.086773in}{1.594270in}}{\pgfqpoint{2.083861in}{1.591358in}}%
\pgfpathcurveto{\pgfqpoint{2.080949in}{1.588446in}}{\pgfqpoint{2.079313in}{1.584496in}}{\pgfqpoint{2.079313in}{1.580378in}}%
\pgfpathcurveto{\pgfqpoint{2.079313in}{1.576260in}}{\pgfqpoint{2.080949in}{1.572310in}}{\pgfqpoint{2.083861in}{1.569398in}}%
\pgfpathcurveto{\pgfqpoint{2.086773in}{1.566486in}}{\pgfqpoint{2.090723in}{1.564850in}}{\pgfqpoint{2.094841in}{1.564850in}}%
\pgfpathclose%
\pgfusepath{fill}%
\end{pgfscope}%
\begin{pgfscope}%
\pgfpathrectangle{\pgfqpoint{0.887500in}{0.275000in}}{\pgfqpoint{4.225000in}{4.225000in}}%
\pgfusepath{clip}%
\pgfsetbuttcap%
\pgfsetroundjoin%
\definecolor{currentfill}{rgb}{0.000000,0.000000,0.000000}%
\pgfsetfillcolor{currentfill}%
\pgfsetfillopacity{0.800000}%
\pgfsetlinewidth{0.000000pt}%
\definecolor{currentstroke}{rgb}{0.000000,0.000000,0.000000}%
\pgfsetstrokecolor{currentstroke}%
\pgfsetstrokeopacity{0.800000}%
\pgfsetdash{}{0pt}%
\pgfpathmoveto{\pgfqpoint{2.277982in}{1.488090in}}%
\pgfpathcurveto{\pgfqpoint{2.282100in}{1.488090in}}{\pgfqpoint{2.286051in}{1.489726in}}{\pgfqpoint{2.288962in}{1.492638in}}%
\pgfpathcurveto{\pgfqpoint{2.291874in}{1.495550in}}{\pgfqpoint{2.293511in}{1.499500in}}{\pgfqpoint{2.293511in}{1.503619in}}%
\pgfpathcurveto{\pgfqpoint{2.293511in}{1.507737in}}{\pgfqpoint{2.291874in}{1.511687in}}{\pgfqpoint{2.288962in}{1.514599in}}%
\pgfpathcurveto{\pgfqpoint{2.286051in}{1.517511in}}{\pgfqpoint{2.282100in}{1.519147in}}{\pgfqpoint{2.277982in}{1.519147in}}%
\pgfpathcurveto{\pgfqpoint{2.273864in}{1.519147in}}{\pgfqpoint{2.269914in}{1.517511in}}{\pgfqpoint{2.267002in}{1.514599in}}%
\pgfpathcurveto{\pgfqpoint{2.264090in}{1.511687in}}{\pgfqpoint{2.262454in}{1.507737in}}{\pgfqpoint{2.262454in}{1.503619in}}%
\pgfpathcurveto{\pgfqpoint{2.262454in}{1.499500in}}{\pgfqpoint{2.264090in}{1.495550in}}{\pgfqpoint{2.267002in}{1.492638in}}%
\pgfpathcurveto{\pgfqpoint{2.269914in}{1.489726in}}{\pgfqpoint{2.273864in}{1.488090in}}{\pgfqpoint{2.277982in}{1.488090in}}%
\pgfpathclose%
\pgfusepath{fill}%
\end{pgfscope}%
\begin{pgfscope}%
\pgfpathrectangle{\pgfqpoint{0.887500in}{0.275000in}}{\pgfqpoint{4.225000in}{4.225000in}}%
\pgfusepath{clip}%
\pgfsetbuttcap%
\pgfsetroundjoin%
\definecolor{currentfill}{rgb}{0.000000,0.000000,0.000000}%
\pgfsetfillcolor{currentfill}%
\pgfsetfillopacity{0.800000}%
\pgfsetlinewidth{0.000000pt}%
\definecolor{currentstroke}{rgb}{0.000000,0.000000,0.000000}%
\pgfsetstrokecolor{currentstroke}%
\pgfsetstrokeopacity{0.800000}%
\pgfsetdash{}{0pt}%
\pgfpathmoveto{\pgfqpoint{2.461389in}{1.407357in}}%
\pgfpathcurveto{\pgfqpoint{2.465507in}{1.407357in}}{\pgfqpoint{2.469457in}{1.408994in}}{\pgfqpoint{2.472369in}{1.411905in}}%
\pgfpathcurveto{\pgfqpoint{2.475281in}{1.414817in}}{\pgfqpoint{2.476918in}{1.418767in}}{\pgfqpoint{2.476918in}{1.422886in}}%
\pgfpathcurveto{\pgfqpoint{2.476918in}{1.427004in}}{\pgfqpoint{2.475281in}{1.430954in}}{\pgfqpoint{2.472369in}{1.433866in}}%
\pgfpathcurveto{\pgfqpoint{2.469457in}{1.436778in}}{\pgfqpoint{2.465507in}{1.438414in}}{\pgfqpoint{2.461389in}{1.438414in}}%
\pgfpathcurveto{\pgfqpoint{2.457271in}{1.438414in}}{\pgfqpoint{2.453321in}{1.436778in}}{\pgfqpoint{2.450409in}{1.433866in}}%
\pgfpathcurveto{\pgfqpoint{2.447497in}{1.430954in}}{\pgfqpoint{2.445861in}{1.427004in}}{\pgfqpoint{2.445861in}{1.422886in}}%
\pgfpathcurveto{\pgfqpoint{2.445861in}{1.418767in}}{\pgfqpoint{2.447497in}{1.414817in}}{\pgfqpoint{2.450409in}{1.411905in}}%
\pgfpathcurveto{\pgfqpoint{2.453321in}{1.408994in}}{\pgfqpoint{2.457271in}{1.407357in}}{\pgfqpoint{2.461389in}{1.407357in}}%
\pgfpathclose%
\pgfusepath{fill}%
\end{pgfscope}%
\begin{pgfscope}%
\pgfpathrectangle{\pgfqpoint{0.887500in}{0.275000in}}{\pgfqpoint{4.225000in}{4.225000in}}%
\pgfusepath{clip}%
\pgfsetbuttcap%
\pgfsetroundjoin%
\definecolor{currentfill}{rgb}{0.000000,0.000000,0.000000}%
\pgfsetfillcolor{currentfill}%
\pgfsetfillopacity{0.800000}%
\pgfsetlinewidth{0.000000pt}%
\definecolor{currentstroke}{rgb}{0.000000,0.000000,0.000000}%
\pgfsetstrokecolor{currentstroke}%
\pgfsetstrokeopacity{0.800000}%
\pgfsetdash{}{0pt}%
\pgfpathmoveto{\pgfqpoint{2.645014in}{1.321980in}}%
\pgfpathcurveto{\pgfqpoint{2.649132in}{1.321980in}}{\pgfqpoint{2.653082in}{1.323617in}}{\pgfqpoint{2.655994in}{1.326529in}}%
\pgfpathcurveto{\pgfqpoint{2.658906in}{1.329441in}}{\pgfqpoint{2.660542in}{1.333391in}}{\pgfqpoint{2.660542in}{1.337509in}}%
\pgfpathcurveto{\pgfqpoint{2.660542in}{1.341627in}}{\pgfqpoint{2.658906in}{1.345577in}}{\pgfqpoint{2.655994in}{1.348489in}}%
\pgfpathcurveto{\pgfqpoint{2.653082in}{1.351401in}}{\pgfqpoint{2.649132in}{1.353037in}}{\pgfqpoint{2.645014in}{1.353037in}}%
\pgfpathcurveto{\pgfqpoint{2.640896in}{1.353037in}}{\pgfqpoint{2.636946in}{1.351401in}}{\pgfqpoint{2.634034in}{1.348489in}}%
\pgfpathcurveto{\pgfqpoint{2.631122in}{1.345577in}}{\pgfqpoint{2.629486in}{1.341627in}}{\pgfqpoint{2.629486in}{1.337509in}}%
\pgfpathcurveto{\pgfqpoint{2.629486in}{1.333391in}}{\pgfqpoint{2.631122in}{1.329441in}}{\pgfqpoint{2.634034in}{1.326529in}}%
\pgfpathcurveto{\pgfqpoint{2.636946in}{1.323617in}}{\pgfqpoint{2.640896in}{1.321980in}}{\pgfqpoint{2.645014in}{1.321980in}}%
\pgfpathclose%
\pgfusepath{fill}%
\end{pgfscope}%
\begin{pgfscope}%
\pgfpathrectangle{\pgfqpoint{0.887500in}{0.275000in}}{\pgfqpoint{4.225000in}{4.225000in}}%
\pgfusepath{clip}%
\pgfsetbuttcap%
\pgfsetroundjoin%
\definecolor{currentfill}{rgb}{0.000000,0.000000,0.000000}%
\pgfsetfillcolor{currentfill}%
\pgfsetfillopacity{0.800000}%
\pgfsetlinewidth{0.000000pt}%
\definecolor{currentstroke}{rgb}{0.000000,0.000000,0.000000}%
\pgfsetstrokecolor{currentstroke}%
\pgfsetstrokeopacity{0.800000}%
\pgfsetdash{}{0pt}%
\pgfpathmoveto{\pgfqpoint{3.012662in}{1.131051in}}%
\pgfpathcurveto{\pgfqpoint{3.016780in}{1.131051in}}{\pgfqpoint{3.020730in}{1.132687in}}{\pgfqpoint{3.023642in}{1.135599in}}%
\pgfpathcurveto{\pgfqpoint{3.026554in}{1.138511in}}{\pgfqpoint{3.028190in}{1.142461in}}{\pgfqpoint{3.028190in}{1.146579in}}%
\pgfpathcurveto{\pgfqpoint{3.028190in}{1.150697in}}{\pgfqpoint{3.026554in}{1.154647in}}{\pgfqpoint{3.023642in}{1.157559in}}%
\pgfpathcurveto{\pgfqpoint{3.020730in}{1.160471in}}{\pgfqpoint{3.016780in}{1.162107in}}{\pgfqpoint{3.012662in}{1.162107in}}%
\pgfpathcurveto{\pgfqpoint{3.008544in}{1.162107in}}{\pgfqpoint{3.004594in}{1.160471in}}{\pgfqpoint{3.001682in}{1.157559in}}%
\pgfpathcurveto{\pgfqpoint{2.998770in}{1.154647in}}{\pgfqpoint{2.997134in}{1.150697in}}{\pgfqpoint{2.997134in}{1.146579in}}%
\pgfpathcurveto{\pgfqpoint{2.997134in}{1.142461in}}{\pgfqpoint{2.998770in}{1.138511in}}{\pgfqpoint{3.001682in}{1.135599in}}%
\pgfpathcurveto{\pgfqpoint{3.004594in}{1.132687in}}{\pgfqpoint{3.008544in}{1.131051in}}{\pgfqpoint{3.012662in}{1.131051in}}%
\pgfpathclose%
\pgfusepath{fill}%
\end{pgfscope}%
\begin{pgfscope}%
\pgfpathrectangle{\pgfqpoint{0.887500in}{0.275000in}}{\pgfqpoint{4.225000in}{4.225000in}}%
\pgfusepath{clip}%
\pgfsetbuttcap%
\pgfsetroundjoin%
\definecolor{currentfill}{rgb}{0.000000,0.000000,0.000000}%
\pgfsetfillcolor{currentfill}%
\pgfsetfillopacity{0.800000}%
\pgfsetlinewidth{0.000000pt}%
\definecolor{currentstroke}{rgb}{0.000000,0.000000,0.000000}%
\pgfsetstrokecolor{currentstroke}%
\pgfsetstrokeopacity{0.800000}%
\pgfsetdash{}{0pt}%
\pgfpathmoveto{\pgfqpoint{2.828792in}{1.231727in}}%
\pgfpathcurveto{\pgfqpoint{2.832910in}{1.231727in}}{\pgfqpoint{2.836860in}{1.233363in}}{\pgfqpoint{2.839772in}{1.236275in}}%
\pgfpathcurveto{\pgfqpoint{2.842684in}{1.239187in}}{\pgfqpoint{2.844321in}{1.243137in}}{\pgfqpoint{2.844321in}{1.247255in}}%
\pgfpathcurveto{\pgfqpoint{2.844321in}{1.251373in}}{\pgfqpoint{2.842684in}{1.255324in}}{\pgfqpoint{2.839772in}{1.258235in}}%
\pgfpathcurveto{\pgfqpoint{2.836860in}{1.261147in}}{\pgfqpoint{2.832910in}{1.262784in}}{\pgfqpoint{2.828792in}{1.262784in}}%
\pgfpathcurveto{\pgfqpoint{2.824674in}{1.262784in}}{\pgfqpoint{2.820724in}{1.261147in}}{\pgfqpoint{2.817812in}{1.258235in}}%
\pgfpathcurveto{\pgfqpoint{2.814900in}{1.255324in}}{\pgfqpoint{2.813264in}{1.251373in}}{\pgfqpoint{2.813264in}{1.247255in}}%
\pgfpathcurveto{\pgfqpoint{2.813264in}{1.243137in}}{\pgfqpoint{2.814900in}{1.239187in}}{\pgfqpoint{2.817812in}{1.236275in}}%
\pgfpathcurveto{\pgfqpoint{2.820724in}{1.233363in}}{\pgfqpoint{2.824674in}{1.231727in}}{\pgfqpoint{2.828792in}{1.231727in}}%
\pgfpathclose%
\pgfusepath{fill}%
\end{pgfscope}%
\begin{pgfscope}%
\pgfpathrectangle{\pgfqpoint{0.887500in}{0.275000in}}{\pgfqpoint{4.225000in}{4.225000in}}%
\pgfusepath{clip}%
\pgfsetbuttcap%
\pgfsetroundjoin%
\definecolor{currentfill}{rgb}{0.000000,0.000000,0.000000}%
\pgfsetfillcolor{currentfill}%
\pgfsetfillopacity{0.800000}%
\pgfsetlinewidth{0.000000pt}%
\definecolor{currentstroke}{rgb}{0.000000,0.000000,0.000000}%
\pgfsetstrokecolor{currentstroke}%
\pgfsetstrokeopacity{0.800000}%
\pgfsetdash{}{0pt}%
\pgfpathmoveto{\pgfqpoint{2.159341in}{1.447824in}}%
\pgfpathcurveto{\pgfqpoint{2.163459in}{1.447824in}}{\pgfqpoint{2.167409in}{1.449461in}}{\pgfqpoint{2.170321in}{1.452373in}}%
\pgfpathcurveto{\pgfqpoint{2.173233in}{1.455285in}}{\pgfqpoint{2.174870in}{1.459235in}}{\pgfqpoint{2.174870in}{1.463353in}}%
\pgfpathcurveto{\pgfqpoint{2.174870in}{1.467471in}}{\pgfqpoint{2.173233in}{1.471421in}}{\pgfqpoint{2.170321in}{1.474333in}}%
\pgfpathcurveto{\pgfqpoint{2.167409in}{1.477245in}}{\pgfqpoint{2.163459in}{1.478881in}}{\pgfqpoint{2.159341in}{1.478881in}}%
\pgfpathcurveto{\pgfqpoint{2.155223in}{1.478881in}}{\pgfqpoint{2.151273in}{1.477245in}}{\pgfqpoint{2.148361in}{1.474333in}}%
\pgfpathcurveto{\pgfqpoint{2.145449in}{1.471421in}}{\pgfqpoint{2.143813in}{1.467471in}}{\pgfqpoint{2.143813in}{1.463353in}}%
\pgfpathcurveto{\pgfqpoint{2.143813in}{1.459235in}}{\pgfqpoint{2.145449in}{1.455285in}}{\pgfqpoint{2.148361in}{1.452373in}}%
\pgfpathcurveto{\pgfqpoint{2.151273in}{1.449461in}}{\pgfqpoint{2.155223in}{1.447824in}}{\pgfqpoint{2.159341in}{1.447824in}}%
\pgfpathclose%
\pgfusepath{fill}%
\end{pgfscope}%
\begin{pgfscope}%
\pgfpathrectangle{\pgfqpoint{0.887500in}{0.275000in}}{\pgfqpoint{4.225000in}{4.225000in}}%
\pgfusepath{clip}%
\pgfsetbuttcap%
\pgfsetroundjoin%
\definecolor{currentfill}{rgb}{0.000000,0.000000,0.000000}%
\pgfsetfillcolor{currentfill}%
\pgfsetfillopacity{0.800000}%
\pgfsetlinewidth{0.000000pt}%
\definecolor{currentstroke}{rgb}{0.000000,0.000000,0.000000}%
\pgfsetstrokecolor{currentstroke}%
\pgfsetstrokeopacity{0.800000}%
\pgfsetdash{}{0pt}%
\pgfpathmoveto{\pgfqpoint{2.343131in}{1.367481in}}%
\pgfpathcurveto{\pgfqpoint{2.347250in}{1.367481in}}{\pgfqpoint{2.351200in}{1.369117in}}{\pgfqpoint{2.354112in}{1.372029in}}%
\pgfpathcurveto{\pgfqpoint{2.357024in}{1.374941in}}{\pgfqpoint{2.358660in}{1.378891in}}{\pgfqpoint{2.358660in}{1.383009in}}%
\pgfpathcurveto{\pgfqpoint{2.358660in}{1.387127in}}{\pgfqpoint{2.357024in}{1.391077in}}{\pgfqpoint{2.354112in}{1.393989in}}%
\pgfpathcurveto{\pgfqpoint{2.351200in}{1.396901in}}{\pgfqpoint{2.347250in}{1.398537in}}{\pgfqpoint{2.343131in}{1.398537in}}%
\pgfpathcurveto{\pgfqpoint{2.339013in}{1.398537in}}{\pgfqpoint{2.335063in}{1.396901in}}{\pgfqpoint{2.332151in}{1.393989in}}%
\pgfpathcurveto{\pgfqpoint{2.329239in}{1.391077in}}{\pgfqpoint{2.327603in}{1.387127in}}{\pgfqpoint{2.327603in}{1.383009in}}%
\pgfpathcurveto{\pgfqpoint{2.327603in}{1.378891in}}{\pgfqpoint{2.329239in}{1.374941in}}{\pgfqpoint{2.332151in}{1.372029in}}%
\pgfpathcurveto{\pgfqpoint{2.335063in}{1.369117in}}{\pgfqpoint{2.339013in}{1.367481in}}{\pgfqpoint{2.343131in}{1.367481in}}%
\pgfpathclose%
\pgfusepath{fill}%
\end{pgfscope}%
\begin{pgfscope}%
\pgfpathrectangle{\pgfqpoint{0.887500in}{0.275000in}}{\pgfqpoint{4.225000in}{4.225000in}}%
\pgfusepath{clip}%
\pgfsetbuttcap%
\pgfsetroundjoin%
\definecolor{currentfill}{rgb}{0.000000,0.000000,0.000000}%
\pgfsetfillcolor{currentfill}%
\pgfsetfillopacity{0.800000}%
\pgfsetlinewidth{0.000000pt}%
\definecolor{currentstroke}{rgb}{0.000000,0.000000,0.000000}%
\pgfsetstrokecolor{currentstroke}%
\pgfsetstrokeopacity{0.800000}%
\pgfsetdash{}{0pt}%
\pgfpathmoveto{\pgfqpoint{2.527154in}{1.282916in}}%
\pgfpathcurveto{\pgfqpoint{2.531272in}{1.282916in}}{\pgfqpoint{2.535222in}{1.284552in}}{\pgfqpoint{2.538134in}{1.287464in}}%
\pgfpathcurveto{\pgfqpoint{2.541046in}{1.290376in}}{\pgfqpoint{2.542682in}{1.294326in}}{\pgfqpoint{2.542682in}{1.298444in}}%
\pgfpathcurveto{\pgfqpoint{2.542682in}{1.302562in}}{\pgfqpoint{2.541046in}{1.306512in}}{\pgfqpoint{2.538134in}{1.309424in}}%
\pgfpathcurveto{\pgfqpoint{2.535222in}{1.312336in}}{\pgfqpoint{2.531272in}{1.313972in}}{\pgfqpoint{2.527154in}{1.313972in}}%
\pgfpathcurveto{\pgfqpoint{2.523036in}{1.313972in}}{\pgfqpoint{2.519086in}{1.312336in}}{\pgfqpoint{2.516174in}{1.309424in}}%
\pgfpathcurveto{\pgfqpoint{2.513262in}{1.306512in}}{\pgfqpoint{2.511626in}{1.302562in}}{\pgfqpoint{2.511626in}{1.298444in}}%
\pgfpathcurveto{\pgfqpoint{2.511626in}{1.294326in}}{\pgfqpoint{2.513262in}{1.290376in}}{\pgfqpoint{2.516174in}{1.287464in}}%
\pgfpathcurveto{\pgfqpoint{2.519086in}{1.284552in}}{\pgfqpoint{2.523036in}{1.282916in}}{\pgfqpoint{2.527154in}{1.282916in}}%
\pgfpathclose%
\pgfusepath{fill}%
\end{pgfscope}%
\begin{pgfscope}%
\pgfpathrectangle{\pgfqpoint{0.887500in}{0.275000in}}{\pgfqpoint{4.225000in}{4.225000in}}%
\pgfusepath{clip}%
\pgfsetbuttcap%
\pgfsetroundjoin%
\definecolor{currentfill}{rgb}{0.000000,0.000000,0.000000}%
\pgfsetfillcolor{currentfill}%
\pgfsetfillopacity{0.800000}%
\pgfsetlinewidth{0.000000pt}%
\definecolor{currentstroke}{rgb}{0.000000,0.000000,0.000000}%
\pgfsetstrokecolor{currentstroke}%
\pgfsetstrokeopacity{0.800000}%
\pgfsetdash{}{0pt}%
\pgfpathmoveto{\pgfqpoint{2.711347in}{1.194497in}}%
\pgfpathcurveto{\pgfqpoint{2.715465in}{1.194497in}}{\pgfqpoint{2.719415in}{1.196133in}}{\pgfqpoint{2.722327in}{1.199045in}}%
\pgfpathcurveto{\pgfqpoint{2.725239in}{1.201957in}}{\pgfqpoint{2.726875in}{1.205907in}}{\pgfqpoint{2.726875in}{1.210025in}}%
\pgfpathcurveto{\pgfqpoint{2.726875in}{1.214143in}}{\pgfqpoint{2.725239in}{1.218093in}}{\pgfqpoint{2.722327in}{1.221005in}}%
\pgfpathcurveto{\pgfqpoint{2.719415in}{1.223917in}}{\pgfqpoint{2.715465in}{1.225553in}}{\pgfqpoint{2.711347in}{1.225553in}}%
\pgfpathcurveto{\pgfqpoint{2.707228in}{1.225553in}}{\pgfqpoint{2.703278in}{1.223917in}}{\pgfqpoint{2.700366in}{1.221005in}}%
\pgfpathcurveto{\pgfqpoint{2.697455in}{1.218093in}}{\pgfqpoint{2.695818in}{1.214143in}}{\pgfqpoint{2.695818in}{1.210025in}}%
\pgfpathcurveto{\pgfqpoint{2.695818in}{1.205907in}}{\pgfqpoint{2.697455in}{1.201957in}}{\pgfqpoint{2.700366in}{1.199045in}}%
\pgfpathcurveto{\pgfqpoint{2.703278in}{1.196133in}}{\pgfqpoint{2.707228in}{1.194497in}}{\pgfqpoint{2.711347in}{1.194497in}}%
\pgfpathclose%
\pgfusepath{fill}%
\end{pgfscope}%
\begin{pgfscope}%
\pgfpathrectangle{\pgfqpoint{0.887500in}{0.275000in}}{\pgfqpoint{4.225000in}{4.225000in}}%
\pgfusepath{clip}%
\pgfsetbuttcap%
\pgfsetroundjoin%
\definecolor{currentfill}{rgb}{0.000000,0.000000,0.000000}%
\pgfsetfillcolor{currentfill}%
\pgfsetfillopacity{0.800000}%
\pgfsetlinewidth{0.000000pt}%
\definecolor{currentstroke}{rgb}{0.000000,0.000000,0.000000}%
\pgfsetstrokecolor{currentstroke}%
\pgfsetstrokeopacity{0.800000}%
\pgfsetdash{}{0pt}%
\pgfpathmoveto{\pgfqpoint{2.895662in}{1.100528in}}%
\pgfpathcurveto{\pgfqpoint{2.899780in}{1.100528in}}{\pgfqpoint{2.903730in}{1.102164in}}{\pgfqpoint{2.906642in}{1.105076in}}%
\pgfpathcurveto{\pgfqpoint{2.909554in}{1.107988in}}{\pgfqpoint{2.911190in}{1.111938in}}{\pgfqpoint{2.911190in}{1.116056in}}%
\pgfpathcurveto{\pgfqpoint{2.911190in}{1.120174in}}{\pgfqpoint{2.909554in}{1.124124in}}{\pgfqpoint{2.906642in}{1.127036in}}%
\pgfpathcurveto{\pgfqpoint{2.903730in}{1.129948in}}{\pgfqpoint{2.899780in}{1.131584in}}{\pgfqpoint{2.895662in}{1.131584in}}%
\pgfpathcurveto{\pgfqpoint{2.891544in}{1.131584in}}{\pgfqpoint{2.887594in}{1.129948in}}{\pgfqpoint{2.884682in}{1.127036in}}%
\pgfpathcurveto{\pgfqpoint{2.881770in}{1.124124in}}{\pgfqpoint{2.880134in}{1.120174in}}{\pgfqpoint{2.880134in}{1.116056in}}%
\pgfpathcurveto{\pgfqpoint{2.880134in}{1.111938in}}{\pgfqpoint{2.881770in}{1.107988in}}{\pgfqpoint{2.884682in}{1.105076in}}%
\pgfpathcurveto{\pgfqpoint{2.887594in}{1.102164in}}{\pgfqpoint{2.891544in}{1.100528in}}{\pgfqpoint{2.895662in}{1.100528in}}%
\pgfpathclose%
\pgfusepath{fill}%
\end{pgfscope}%
\begin{pgfscope}%
\pgfpathrectangle{\pgfqpoint{0.887500in}{0.275000in}}{\pgfqpoint{4.225000in}{4.225000in}}%
\pgfusepath{clip}%
\pgfsetbuttcap%
\pgfsetroundjoin%
\definecolor{currentfill}{rgb}{0.000000,0.000000,0.000000}%
\pgfsetfillcolor{currentfill}%
\pgfsetfillopacity{0.800000}%
\pgfsetlinewidth{0.000000pt}%
\definecolor{currentstroke}{rgb}{0.000000,0.000000,0.000000}%
\pgfsetstrokecolor{currentstroke}%
\pgfsetstrokeopacity{0.800000}%
\pgfsetdash{}{0pt}%
\pgfpathmoveto{\pgfqpoint{3.197041in}{1.175543in}}%
\pgfpathcurveto{\pgfqpoint{3.201160in}{1.175543in}}{\pgfqpoint{3.205110in}{1.177179in}}{\pgfqpoint{3.208022in}{1.180091in}}%
\pgfpathcurveto{\pgfqpoint{3.210933in}{1.183003in}}{\pgfqpoint{3.212570in}{1.186953in}}{\pgfqpoint{3.212570in}{1.191071in}}%
\pgfpathcurveto{\pgfqpoint{3.212570in}{1.195189in}}{\pgfqpoint{3.210933in}{1.199139in}}{\pgfqpoint{3.208022in}{1.202051in}}%
\pgfpathcurveto{\pgfqpoint{3.205110in}{1.204963in}}{\pgfqpoint{3.201160in}{1.206599in}}{\pgfqpoint{3.197041in}{1.206599in}}%
\pgfpathcurveto{\pgfqpoint{3.192923in}{1.206599in}}{\pgfqpoint{3.188973in}{1.204963in}}{\pgfqpoint{3.186061in}{1.202051in}}%
\pgfpathcurveto{\pgfqpoint{3.183149in}{1.199139in}}{\pgfqpoint{3.181513in}{1.195189in}}{\pgfqpoint{3.181513in}{1.191071in}}%
\pgfpathcurveto{\pgfqpoint{3.181513in}{1.186953in}}{\pgfqpoint{3.183149in}{1.183003in}}{\pgfqpoint{3.186061in}{1.180091in}}%
\pgfpathcurveto{\pgfqpoint{3.188973in}{1.177179in}}{\pgfqpoint{3.192923in}{1.175543in}}{\pgfqpoint{3.197041in}{1.175543in}}%
\pgfpathclose%
\pgfusepath{fill}%
\end{pgfscope}%
\begin{pgfscope}%
\pgfpathrectangle{\pgfqpoint{0.887500in}{0.275000in}}{\pgfqpoint{4.225000in}{4.225000in}}%
\pgfusepath{clip}%
\pgfsetbuttcap%
\pgfsetroundjoin%
\definecolor{currentfill}{rgb}{0.000000,0.000000,0.000000}%
\pgfsetfillcolor{currentfill}%
\pgfsetfillopacity{0.800000}%
\pgfsetlinewidth{0.000000pt}%
\definecolor{currentstroke}{rgb}{0.000000,0.000000,0.000000}%
\pgfsetstrokecolor{currentstroke}%
\pgfsetstrokeopacity{0.800000}%
\pgfsetdash{}{0pt}%
\pgfpathmoveto{\pgfqpoint{2.224278in}{1.327927in}}%
\pgfpathcurveto{\pgfqpoint{2.228396in}{1.327927in}}{\pgfqpoint{2.232346in}{1.329563in}}{\pgfqpoint{2.235258in}{1.332475in}}%
\pgfpathcurveto{\pgfqpoint{2.238170in}{1.335387in}}{\pgfqpoint{2.239806in}{1.339337in}}{\pgfqpoint{2.239806in}{1.343455in}}%
\pgfpathcurveto{\pgfqpoint{2.239806in}{1.347573in}}{\pgfqpoint{2.238170in}{1.351523in}}{\pgfqpoint{2.235258in}{1.354435in}}%
\pgfpathcurveto{\pgfqpoint{2.232346in}{1.357347in}}{\pgfqpoint{2.228396in}{1.358983in}}{\pgfqpoint{2.224278in}{1.358983in}}%
\pgfpathcurveto{\pgfqpoint{2.220160in}{1.358983in}}{\pgfqpoint{2.216210in}{1.357347in}}{\pgfqpoint{2.213298in}{1.354435in}}%
\pgfpathcurveto{\pgfqpoint{2.210386in}{1.351523in}}{\pgfqpoint{2.208750in}{1.347573in}}{\pgfqpoint{2.208750in}{1.343455in}}%
\pgfpathcurveto{\pgfqpoint{2.208750in}{1.339337in}}{\pgfqpoint{2.210386in}{1.335387in}}{\pgfqpoint{2.213298in}{1.332475in}}%
\pgfpathcurveto{\pgfqpoint{2.216210in}{1.329563in}}{\pgfqpoint{2.220160in}{1.327927in}}{\pgfqpoint{2.224278in}{1.327927in}}%
\pgfpathclose%
\pgfusepath{fill}%
\end{pgfscope}%
\begin{pgfscope}%
\pgfpathrectangle{\pgfqpoint{0.887500in}{0.275000in}}{\pgfqpoint{4.225000in}{4.225000in}}%
\pgfusepath{clip}%
\pgfsetbuttcap%
\pgfsetroundjoin%
\definecolor{currentfill}{rgb}{0.000000,0.000000,0.000000}%
\pgfsetfillcolor{currentfill}%
\pgfsetfillopacity{0.800000}%
\pgfsetlinewidth{0.000000pt}%
\definecolor{currentstroke}{rgb}{0.000000,0.000000,0.000000}%
\pgfsetstrokecolor{currentstroke}%
\pgfsetstrokeopacity{0.800000}%
\pgfsetdash{}{0pt}%
\pgfpathmoveto{\pgfqpoint{2.408689in}{1.244607in}}%
\pgfpathcurveto{\pgfqpoint{2.412807in}{1.244607in}}{\pgfqpoint{2.416757in}{1.246243in}}{\pgfqpoint{2.419669in}{1.249155in}}%
\pgfpathcurveto{\pgfqpoint{2.422581in}{1.252067in}}{\pgfqpoint{2.424217in}{1.256017in}}{\pgfqpoint{2.424217in}{1.260135in}}%
\pgfpathcurveto{\pgfqpoint{2.424217in}{1.264253in}}{\pgfqpoint{2.422581in}{1.268203in}}{\pgfqpoint{2.419669in}{1.271115in}}%
\pgfpathcurveto{\pgfqpoint{2.416757in}{1.274027in}}{\pgfqpoint{2.412807in}{1.275663in}}{\pgfqpoint{2.408689in}{1.275663in}}%
\pgfpathcurveto{\pgfqpoint{2.404571in}{1.275663in}}{\pgfqpoint{2.400621in}{1.274027in}}{\pgfqpoint{2.397709in}{1.271115in}}%
\pgfpathcurveto{\pgfqpoint{2.394797in}{1.268203in}}{\pgfqpoint{2.393161in}{1.264253in}}{\pgfqpoint{2.393161in}{1.260135in}}%
\pgfpathcurveto{\pgfqpoint{2.393161in}{1.256017in}}{\pgfqpoint{2.394797in}{1.252067in}}{\pgfqpoint{2.397709in}{1.249155in}}%
\pgfpathcurveto{\pgfqpoint{2.400621in}{1.246243in}}{\pgfqpoint{2.404571in}{1.244607in}}{\pgfqpoint{2.408689in}{1.244607in}}%
\pgfpathclose%
\pgfusepath{fill}%
\end{pgfscope}%
\begin{pgfscope}%
\pgfpathrectangle{\pgfqpoint{0.887500in}{0.275000in}}{\pgfqpoint{4.225000in}{4.225000in}}%
\pgfusepath{clip}%
\pgfsetbuttcap%
\pgfsetroundjoin%
\definecolor{currentfill}{rgb}{0.000000,0.000000,0.000000}%
\pgfsetfillcolor{currentfill}%
\pgfsetfillopacity{0.800000}%
\pgfsetlinewidth{0.000000pt}%
\definecolor{currentstroke}{rgb}{0.000000,0.000000,0.000000}%
\pgfsetstrokecolor{currentstroke}%
\pgfsetstrokeopacity{0.800000}%
\pgfsetdash{}{0pt}%
\pgfpathmoveto{\pgfqpoint{2.593292in}{1.157853in}}%
\pgfpathcurveto{\pgfqpoint{2.597410in}{1.157853in}}{\pgfqpoint{2.601360in}{1.159489in}}{\pgfqpoint{2.604272in}{1.162401in}}%
\pgfpathcurveto{\pgfqpoint{2.607184in}{1.165313in}}{\pgfqpoint{2.608820in}{1.169263in}}{\pgfqpoint{2.608820in}{1.173381in}}%
\pgfpathcurveto{\pgfqpoint{2.608820in}{1.177499in}}{\pgfqpoint{2.607184in}{1.181449in}}{\pgfqpoint{2.604272in}{1.184361in}}%
\pgfpathcurveto{\pgfqpoint{2.601360in}{1.187273in}}{\pgfqpoint{2.597410in}{1.188909in}}{\pgfqpoint{2.593292in}{1.188909in}}%
\pgfpathcurveto{\pgfqpoint{2.589174in}{1.188909in}}{\pgfqpoint{2.585224in}{1.187273in}}{\pgfqpoint{2.582312in}{1.184361in}}%
\pgfpathcurveto{\pgfqpoint{2.579400in}{1.181449in}}{\pgfqpoint{2.577763in}{1.177499in}}{\pgfqpoint{2.577763in}{1.173381in}}%
\pgfpathcurveto{\pgfqpoint{2.577763in}{1.169263in}}{\pgfqpoint{2.579400in}{1.165313in}}{\pgfqpoint{2.582312in}{1.162401in}}%
\pgfpathcurveto{\pgfqpoint{2.585224in}{1.159489in}}{\pgfqpoint{2.589174in}{1.157853in}}{\pgfqpoint{2.593292in}{1.157853in}}%
\pgfpathclose%
\pgfusepath{fill}%
\end{pgfscope}%
\begin{pgfscope}%
\pgfpathrectangle{\pgfqpoint{0.887500in}{0.275000in}}{\pgfqpoint{4.225000in}{4.225000in}}%
\pgfusepath{clip}%
\pgfsetbuttcap%
\pgfsetroundjoin%
\definecolor{currentfill}{rgb}{0.000000,0.000000,0.000000}%
\pgfsetfillcolor{currentfill}%
\pgfsetfillopacity{0.800000}%
\pgfsetlinewidth{0.000000pt}%
\definecolor{currentstroke}{rgb}{0.000000,0.000000,0.000000}%
\pgfsetstrokecolor{currentstroke}%
\pgfsetstrokeopacity{0.800000}%
\pgfsetdash{}{0pt}%
\pgfpathmoveto{\pgfqpoint{2.778033in}{1.068201in}}%
\pgfpathcurveto{\pgfqpoint{2.782151in}{1.068201in}}{\pgfqpoint{2.786101in}{1.069837in}}{\pgfqpoint{2.789013in}{1.072749in}}%
\pgfpathcurveto{\pgfqpoint{2.791925in}{1.075661in}}{\pgfqpoint{2.793561in}{1.079611in}}{\pgfqpoint{2.793561in}{1.083729in}}%
\pgfpathcurveto{\pgfqpoint{2.793561in}{1.087847in}}{\pgfqpoint{2.791925in}{1.091797in}}{\pgfqpoint{2.789013in}{1.094709in}}%
\pgfpathcurveto{\pgfqpoint{2.786101in}{1.097621in}}{\pgfqpoint{2.782151in}{1.099257in}}{\pgfqpoint{2.778033in}{1.099257in}}%
\pgfpathcurveto{\pgfqpoint{2.773915in}{1.099257in}}{\pgfqpoint{2.769965in}{1.097621in}}{\pgfqpoint{2.767053in}{1.094709in}}%
\pgfpathcurveto{\pgfqpoint{2.764141in}{1.091797in}}{\pgfqpoint{2.762505in}{1.087847in}}{\pgfqpoint{2.762505in}{1.083729in}}%
\pgfpathcurveto{\pgfqpoint{2.762505in}{1.079611in}}{\pgfqpoint{2.764141in}{1.075661in}}{\pgfqpoint{2.767053in}{1.072749in}}%
\pgfpathcurveto{\pgfqpoint{2.769965in}{1.069837in}}{\pgfqpoint{2.773915in}{1.068201in}}{\pgfqpoint{2.778033in}{1.068201in}}%
\pgfpathclose%
\pgfusepath{fill}%
\end{pgfscope}%
\begin{pgfscope}%
\pgfpathrectangle{\pgfqpoint{0.887500in}{0.275000in}}{\pgfqpoint{4.225000in}{4.225000in}}%
\pgfusepath{clip}%
\pgfsetbuttcap%
\pgfsetroundjoin%
\definecolor{currentfill}{rgb}{0.000000,0.000000,0.000000}%
\pgfsetfillcolor{currentfill}%
\pgfsetfillopacity{0.800000}%
\pgfsetlinewidth{0.000000pt}%
\definecolor{currentstroke}{rgb}{0.000000,0.000000,0.000000}%
\pgfsetstrokecolor{currentstroke}%
\pgfsetstrokeopacity{0.800000}%
\pgfsetdash{}{0pt}%
\pgfpathmoveto{\pgfqpoint{3.080085in}{1.108693in}}%
\pgfpathcurveto{\pgfqpoint{3.084203in}{1.108693in}}{\pgfqpoint{3.088153in}{1.110329in}}{\pgfqpoint{3.091065in}{1.113241in}}%
\pgfpathcurveto{\pgfqpoint{3.093977in}{1.116153in}}{\pgfqpoint{3.095613in}{1.120103in}}{\pgfqpoint{3.095613in}{1.124221in}}%
\pgfpathcurveto{\pgfqpoint{3.095613in}{1.128339in}}{\pgfqpoint{3.093977in}{1.132289in}}{\pgfqpoint{3.091065in}{1.135201in}}%
\pgfpathcurveto{\pgfqpoint{3.088153in}{1.138113in}}{\pgfqpoint{3.084203in}{1.139749in}}{\pgfqpoint{3.080085in}{1.139749in}}%
\pgfpathcurveto{\pgfqpoint{3.075967in}{1.139749in}}{\pgfqpoint{3.072017in}{1.138113in}}{\pgfqpoint{3.069105in}{1.135201in}}%
\pgfpathcurveto{\pgfqpoint{3.066193in}{1.132289in}}{\pgfqpoint{3.064557in}{1.128339in}}{\pgfqpoint{3.064557in}{1.124221in}}%
\pgfpathcurveto{\pgfqpoint{3.064557in}{1.120103in}}{\pgfqpoint{3.066193in}{1.116153in}}{\pgfqpoint{3.069105in}{1.113241in}}%
\pgfpathcurveto{\pgfqpoint{3.072017in}{1.110329in}}{\pgfqpoint{3.075967in}{1.108693in}}{\pgfqpoint{3.080085in}{1.108693in}}%
\pgfpathclose%
\pgfusepath{fill}%
\end{pgfscope}%
\begin{pgfscope}%
\pgfpathrectangle{\pgfqpoint{0.887500in}{0.275000in}}{\pgfqpoint{4.225000in}{4.225000in}}%
\pgfusepath{clip}%
\pgfsetbuttcap%
\pgfsetroundjoin%
\definecolor{currentfill}{rgb}{0.000000,0.000000,0.000000}%
\pgfsetfillcolor{currentfill}%
\pgfsetfillopacity{0.800000}%
\pgfsetlinewidth{0.000000pt}%
\definecolor{currentstroke}{rgb}{0.000000,0.000000,0.000000}%
\pgfsetstrokecolor{currentstroke}%
\pgfsetstrokeopacity{0.800000}%
\pgfsetdash{}{0pt}%
\pgfpathmoveto{\pgfqpoint{2.962714in}{1.039879in}}%
\pgfpathcurveto{\pgfqpoint{2.966832in}{1.039879in}}{\pgfqpoint{2.970782in}{1.041515in}}{\pgfqpoint{2.973694in}{1.044427in}}%
\pgfpathcurveto{\pgfqpoint{2.976606in}{1.047339in}}{\pgfqpoint{2.978242in}{1.051289in}}{\pgfqpoint{2.978242in}{1.055407in}}%
\pgfpathcurveto{\pgfqpoint{2.978242in}{1.059526in}}{\pgfqpoint{2.976606in}{1.063476in}}{\pgfqpoint{2.973694in}{1.066388in}}%
\pgfpathcurveto{\pgfqpoint{2.970782in}{1.069299in}}{\pgfqpoint{2.966832in}{1.070936in}}{\pgfqpoint{2.962714in}{1.070936in}}%
\pgfpathcurveto{\pgfqpoint{2.958596in}{1.070936in}}{\pgfqpoint{2.954646in}{1.069299in}}{\pgfqpoint{2.951734in}{1.066388in}}%
\pgfpathcurveto{\pgfqpoint{2.948822in}{1.063476in}}{\pgfqpoint{2.947186in}{1.059526in}}{\pgfqpoint{2.947186in}{1.055407in}}%
\pgfpathcurveto{\pgfqpoint{2.947186in}{1.051289in}}{\pgfqpoint{2.948822in}{1.047339in}}{\pgfqpoint{2.951734in}{1.044427in}}%
\pgfpathcurveto{\pgfqpoint{2.954646in}{1.041515in}}{\pgfqpoint{2.958596in}{1.039879in}}{\pgfqpoint{2.962714in}{1.039879in}}%
\pgfpathclose%
\pgfusepath{fill}%
\end{pgfscope}%
\begin{pgfscope}%
\pgfpathrectangle{\pgfqpoint{0.887500in}{0.275000in}}{\pgfqpoint{4.225000in}{4.225000in}}%
\pgfusepath{clip}%
\pgfsetbuttcap%
\pgfsetroundjoin%
\definecolor{currentfill}{rgb}{0.000000,0.000000,0.000000}%
\pgfsetfillcolor{currentfill}%
\pgfsetfillopacity{0.800000}%
\pgfsetlinewidth{0.000000pt}%
\definecolor{currentstroke}{rgb}{0.000000,0.000000,0.000000}%
\pgfsetstrokecolor{currentstroke}%
\pgfsetstrokeopacity{0.800000}%
\pgfsetdash{}{0pt}%
\pgfpathmoveto{\pgfqpoint{2.289623in}{1.206191in}}%
\pgfpathcurveto{\pgfqpoint{2.293741in}{1.206191in}}{\pgfqpoint{2.297691in}{1.207827in}}{\pgfqpoint{2.300603in}{1.210739in}}%
\pgfpathcurveto{\pgfqpoint{2.303515in}{1.213651in}}{\pgfqpoint{2.305151in}{1.217601in}}{\pgfqpoint{2.305151in}{1.221719in}}%
\pgfpathcurveto{\pgfqpoint{2.305151in}{1.225837in}}{\pgfqpoint{2.303515in}{1.229787in}}{\pgfqpoint{2.300603in}{1.232699in}}%
\pgfpathcurveto{\pgfqpoint{2.297691in}{1.235611in}}{\pgfqpoint{2.293741in}{1.237247in}}{\pgfqpoint{2.289623in}{1.237247in}}%
\pgfpathcurveto{\pgfqpoint{2.285505in}{1.237247in}}{\pgfqpoint{2.281555in}{1.235611in}}{\pgfqpoint{2.278643in}{1.232699in}}%
\pgfpathcurveto{\pgfqpoint{2.275731in}{1.229787in}}{\pgfqpoint{2.274095in}{1.225837in}}{\pgfqpoint{2.274095in}{1.221719in}}%
\pgfpathcurveto{\pgfqpoint{2.274095in}{1.217601in}}{\pgfqpoint{2.275731in}{1.213651in}}{\pgfqpoint{2.278643in}{1.210739in}}%
\pgfpathcurveto{\pgfqpoint{2.281555in}{1.207827in}}{\pgfqpoint{2.285505in}{1.206191in}}{\pgfqpoint{2.289623in}{1.206191in}}%
\pgfpathclose%
\pgfusepath{fill}%
\end{pgfscope}%
\begin{pgfscope}%
\pgfpathrectangle{\pgfqpoint{0.887500in}{0.275000in}}{\pgfqpoint{4.225000in}{4.225000in}}%
\pgfusepath{clip}%
\pgfsetbuttcap%
\pgfsetroundjoin%
\definecolor{currentfill}{rgb}{0.000000,0.000000,0.000000}%
\pgfsetfillcolor{currentfill}%
\pgfsetfillopacity{0.800000}%
\pgfsetlinewidth{0.000000pt}%
\definecolor{currentstroke}{rgb}{0.000000,0.000000,0.000000}%
\pgfsetstrokecolor{currentstroke}%
\pgfsetstrokeopacity{0.800000}%
\pgfsetdash{}{0pt}%
\pgfpathmoveto{\pgfqpoint{2.474625in}{1.121193in}}%
\pgfpathcurveto{\pgfqpoint{2.478743in}{1.121193in}}{\pgfqpoint{2.482693in}{1.122829in}}{\pgfqpoint{2.485605in}{1.125741in}}%
\pgfpathcurveto{\pgfqpoint{2.488517in}{1.128653in}}{\pgfqpoint{2.490153in}{1.132603in}}{\pgfqpoint{2.490153in}{1.136721in}}%
\pgfpathcurveto{\pgfqpoint{2.490153in}{1.140839in}}{\pgfqpoint{2.488517in}{1.144789in}}{\pgfqpoint{2.485605in}{1.147701in}}%
\pgfpathcurveto{\pgfqpoint{2.482693in}{1.150613in}}{\pgfqpoint{2.478743in}{1.152249in}}{\pgfqpoint{2.474625in}{1.152249in}}%
\pgfpathcurveto{\pgfqpoint{2.470507in}{1.152249in}}{\pgfqpoint{2.466557in}{1.150613in}}{\pgfqpoint{2.463645in}{1.147701in}}%
\pgfpathcurveto{\pgfqpoint{2.460733in}{1.144789in}}{\pgfqpoint{2.459097in}{1.140839in}}{\pgfqpoint{2.459097in}{1.136721in}}%
\pgfpathcurveto{\pgfqpoint{2.459097in}{1.132603in}}{\pgfqpoint{2.460733in}{1.128653in}}{\pgfqpoint{2.463645in}{1.125741in}}%
\pgfpathcurveto{\pgfqpoint{2.466557in}{1.122829in}}{\pgfqpoint{2.470507in}{1.121193in}}{\pgfqpoint{2.474625in}{1.121193in}}%
\pgfpathclose%
\pgfusepath{fill}%
\end{pgfscope}%
\begin{pgfscope}%
\pgfpathrectangle{\pgfqpoint{0.887500in}{0.275000in}}{\pgfqpoint{4.225000in}{4.225000in}}%
\pgfusepath{clip}%
\pgfsetbuttcap%
\pgfsetroundjoin%
\definecolor{currentfill}{rgb}{0.000000,0.000000,0.000000}%
\pgfsetfillcolor{currentfill}%
\pgfsetfillopacity{0.800000}%
\pgfsetlinewidth{0.000000pt}%
\definecolor{currentstroke}{rgb}{0.000000,0.000000,0.000000}%
\pgfsetstrokecolor{currentstroke}%
\pgfsetstrokeopacity{0.800000}%
\pgfsetdash{}{0pt}%
\pgfpathmoveto{\pgfqpoint{2.659782in}{1.034406in}}%
\pgfpathcurveto{\pgfqpoint{2.663901in}{1.034406in}}{\pgfqpoint{2.667851in}{1.036042in}}{\pgfqpoint{2.670763in}{1.038954in}}%
\pgfpathcurveto{\pgfqpoint{2.673675in}{1.041866in}}{\pgfqpoint{2.675311in}{1.045816in}}{\pgfqpoint{2.675311in}{1.049934in}}%
\pgfpathcurveto{\pgfqpoint{2.675311in}{1.054052in}}{\pgfqpoint{2.673675in}{1.058002in}}{\pgfqpoint{2.670763in}{1.060914in}}%
\pgfpathcurveto{\pgfqpoint{2.667851in}{1.063826in}}{\pgfqpoint{2.663901in}{1.065462in}}{\pgfqpoint{2.659782in}{1.065462in}}%
\pgfpathcurveto{\pgfqpoint{2.655664in}{1.065462in}}{\pgfqpoint{2.651714in}{1.063826in}}{\pgfqpoint{2.648802in}{1.060914in}}%
\pgfpathcurveto{\pgfqpoint{2.645890in}{1.058002in}}{\pgfqpoint{2.644254in}{1.054052in}}{\pgfqpoint{2.644254in}{1.049934in}}%
\pgfpathcurveto{\pgfqpoint{2.644254in}{1.045816in}}{\pgfqpoint{2.645890in}{1.041866in}}{\pgfqpoint{2.648802in}{1.038954in}}%
\pgfpathcurveto{\pgfqpoint{2.651714in}{1.036042in}}{\pgfqpoint{2.655664in}{1.034406in}}{\pgfqpoint{2.659782in}{1.034406in}}%
\pgfpathclose%
\pgfusepath{fill}%
\end{pgfscope}%
\begin{pgfscope}%
\pgfpathrectangle{\pgfqpoint{0.887500in}{0.275000in}}{\pgfqpoint{4.225000in}{4.225000in}}%
\pgfusepath{clip}%
\pgfsetbuttcap%
\pgfsetroundjoin%
\definecolor{currentfill}{rgb}{0.000000,0.000000,0.000000}%
\pgfsetfillcolor{currentfill}%
\pgfsetfillopacity{0.800000}%
\pgfsetlinewidth{0.000000pt}%
\definecolor{currentstroke}{rgb}{0.000000,0.000000,0.000000}%
\pgfsetstrokecolor{currentstroke}%
\pgfsetstrokeopacity{0.800000}%
\pgfsetdash{}{0pt}%
\pgfpathmoveto{\pgfqpoint{2.844898in}{0.976860in}}%
\pgfpathcurveto{\pgfqpoint{2.849017in}{0.976860in}}{\pgfqpoint{2.852967in}{0.978497in}}{\pgfqpoint{2.855879in}{0.981408in}}%
\pgfpathcurveto{\pgfqpoint{2.858791in}{0.984320in}}{\pgfqpoint{2.860427in}{0.988270in}}{\pgfqpoint{2.860427in}{0.992389in}}%
\pgfpathcurveto{\pgfqpoint{2.860427in}{0.996507in}}{\pgfqpoint{2.858791in}{1.000457in}}{\pgfqpoint{2.855879in}{1.003369in}}%
\pgfpathcurveto{\pgfqpoint{2.852967in}{1.006281in}}{\pgfqpoint{2.849017in}{1.007917in}}{\pgfqpoint{2.844898in}{1.007917in}}%
\pgfpathcurveto{\pgfqpoint{2.840780in}{1.007917in}}{\pgfqpoint{2.836830in}{1.006281in}}{\pgfqpoint{2.833918in}{1.003369in}}%
\pgfpathcurveto{\pgfqpoint{2.831006in}{1.000457in}}{\pgfqpoint{2.829370in}{0.996507in}}{\pgfqpoint{2.829370in}{0.992389in}}%
\pgfpathcurveto{\pgfqpoint{2.829370in}{0.988270in}}{\pgfqpoint{2.831006in}{0.984320in}}{\pgfqpoint{2.833918in}{0.981408in}}%
\pgfpathcurveto{\pgfqpoint{2.836830in}{0.978497in}}{\pgfqpoint{2.840780in}{0.976860in}}{\pgfqpoint{2.844898in}{0.976860in}}%
\pgfpathclose%
\pgfusepath{fill}%
\end{pgfscope}%
\begin{pgfscope}%
\pgfpathrectangle{\pgfqpoint{0.887500in}{0.275000in}}{\pgfqpoint{4.225000in}{4.225000in}}%
\pgfusepath{clip}%
\pgfsetbuttcap%
\pgfsetroundjoin%
\definecolor{currentfill}{rgb}{0.000000,0.000000,0.000000}%
\pgfsetfillcolor{currentfill}%
\pgfsetfillopacity{0.800000}%
\pgfsetlinewidth{0.000000pt}%
\definecolor{currentstroke}{rgb}{0.000000,0.000000,0.000000}%
\pgfsetstrokecolor{currentstroke}%
\pgfsetstrokeopacity{0.800000}%
\pgfsetdash{}{0pt}%
\pgfpathmoveto{\pgfqpoint{2.355344in}{1.084256in}}%
\pgfpathcurveto{\pgfqpoint{2.359462in}{1.084256in}}{\pgfqpoint{2.363413in}{1.085892in}}{\pgfqpoint{2.366324in}{1.088804in}}%
\pgfpathcurveto{\pgfqpoint{2.369236in}{1.091716in}}{\pgfqpoint{2.370873in}{1.095666in}}{\pgfqpoint{2.370873in}{1.099784in}}%
\pgfpathcurveto{\pgfqpoint{2.370873in}{1.103902in}}{\pgfqpoint{2.369236in}{1.107852in}}{\pgfqpoint{2.366324in}{1.110764in}}%
\pgfpathcurveto{\pgfqpoint{2.363413in}{1.113676in}}{\pgfqpoint{2.359462in}{1.115312in}}{\pgfqpoint{2.355344in}{1.115312in}}%
\pgfpathcurveto{\pgfqpoint{2.351226in}{1.115312in}}{\pgfqpoint{2.347276in}{1.113676in}}{\pgfqpoint{2.344364in}{1.110764in}}%
\pgfpathcurveto{\pgfqpoint{2.341452in}{1.107852in}}{\pgfqpoint{2.339816in}{1.103902in}}{\pgfqpoint{2.339816in}{1.099784in}}%
\pgfpathcurveto{\pgfqpoint{2.339816in}{1.095666in}}{\pgfqpoint{2.341452in}{1.091716in}}{\pgfqpoint{2.344364in}{1.088804in}}%
\pgfpathcurveto{\pgfqpoint{2.347276in}{1.085892in}}{\pgfqpoint{2.351226in}{1.084256in}}{\pgfqpoint{2.355344in}{1.084256in}}%
\pgfpathclose%
\pgfusepath{fill}%
\end{pgfscope}%
\begin{pgfscope}%
\pgfpathrectangle{\pgfqpoint{0.887500in}{0.275000in}}{\pgfqpoint{4.225000in}{4.225000in}}%
\pgfusepath{clip}%
\pgfsetbuttcap%
\pgfsetroundjoin%
\definecolor{currentfill}{rgb}{0.000000,0.000000,0.000000}%
\pgfsetfillcolor{currentfill}%
\pgfsetfillopacity{0.800000}%
\pgfsetlinewidth{0.000000pt}%
\definecolor{currentstroke}{rgb}{0.000000,0.000000,0.000000}%
\pgfsetstrokecolor{currentstroke}%
\pgfsetstrokeopacity{0.800000}%
\pgfsetdash{}{0pt}%
\pgfpathmoveto{\pgfqpoint{2.540910in}{0.999652in}}%
\pgfpathcurveto{\pgfqpoint{2.545028in}{0.999652in}}{\pgfqpoint{2.548978in}{1.001288in}}{\pgfqpoint{2.551890in}{1.004200in}}%
\pgfpathcurveto{\pgfqpoint{2.554802in}{1.007112in}}{\pgfqpoint{2.556438in}{1.011062in}}{\pgfqpoint{2.556438in}{1.015180in}}%
\pgfpathcurveto{\pgfqpoint{2.556438in}{1.019298in}}{\pgfqpoint{2.554802in}{1.023248in}}{\pgfqpoint{2.551890in}{1.026160in}}%
\pgfpathcurveto{\pgfqpoint{2.548978in}{1.029072in}}{\pgfqpoint{2.545028in}{1.030708in}}{\pgfqpoint{2.540910in}{1.030708in}}%
\pgfpathcurveto{\pgfqpoint{2.536792in}{1.030708in}}{\pgfqpoint{2.532842in}{1.029072in}}{\pgfqpoint{2.529930in}{1.026160in}}%
\pgfpathcurveto{\pgfqpoint{2.527018in}{1.023248in}}{\pgfqpoint{2.525381in}{1.019298in}}{\pgfqpoint{2.525381in}{1.015180in}}%
\pgfpathcurveto{\pgfqpoint{2.525381in}{1.011062in}}{\pgfqpoint{2.527018in}{1.007112in}}{\pgfqpoint{2.529930in}{1.004200in}}%
\pgfpathcurveto{\pgfqpoint{2.532842in}{1.001288in}}{\pgfqpoint{2.536792in}{0.999652in}}{\pgfqpoint{2.540910in}{0.999652in}}%
\pgfpathclose%
\pgfusepath{fill}%
\end{pgfscope}%
\begin{pgfscope}%
\pgfpathrectangle{\pgfqpoint{0.887500in}{0.275000in}}{\pgfqpoint{4.225000in}{4.225000in}}%
\pgfusepath{clip}%
\pgfsetbuttcap%
\pgfsetroundjoin%
\definecolor{currentfill}{rgb}{0.000000,0.000000,0.000000}%
\pgfsetfillcolor{currentfill}%
\pgfsetfillopacity{0.800000}%
\pgfsetlinewidth{0.000000pt}%
\definecolor{currentstroke}{rgb}{0.000000,0.000000,0.000000}%
\pgfsetstrokecolor{currentstroke}%
\pgfsetstrokeopacity{0.800000}%
\pgfsetdash{}{0pt}%
\pgfpathmoveto{\pgfqpoint{2.726506in}{0.928184in}}%
\pgfpathcurveto{\pgfqpoint{2.730624in}{0.928184in}}{\pgfqpoint{2.734574in}{0.929820in}}{\pgfqpoint{2.737486in}{0.932732in}}%
\pgfpathcurveto{\pgfqpoint{2.740398in}{0.935644in}}{\pgfqpoint{2.742034in}{0.939594in}}{\pgfqpoint{2.742034in}{0.943713in}}%
\pgfpathcurveto{\pgfqpoint{2.742034in}{0.947831in}}{\pgfqpoint{2.740398in}{0.951781in}}{\pgfqpoint{2.737486in}{0.954693in}}%
\pgfpathcurveto{\pgfqpoint{2.734574in}{0.957605in}}{\pgfqpoint{2.730624in}{0.959241in}}{\pgfqpoint{2.726506in}{0.959241in}}%
\pgfpathcurveto{\pgfqpoint{2.722388in}{0.959241in}}{\pgfqpoint{2.718438in}{0.957605in}}{\pgfqpoint{2.715526in}{0.954693in}}%
\pgfpathcurveto{\pgfqpoint{2.712614in}{0.951781in}}{\pgfqpoint{2.710978in}{0.947831in}}{\pgfqpoint{2.710978in}{0.943713in}}%
\pgfpathcurveto{\pgfqpoint{2.710978in}{0.939594in}}{\pgfqpoint{2.712614in}{0.935644in}}{\pgfqpoint{2.715526in}{0.932732in}}%
\pgfpathcurveto{\pgfqpoint{2.718438in}{0.929820in}}{\pgfqpoint{2.722388in}{0.928184in}}{\pgfqpoint{2.726506in}{0.928184in}}%
\pgfpathclose%
\pgfusepath{fill}%
\end{pgfscope}%
\begin{pgfscope}%
\pgfpathrectangle{\pgfqpoint{0.887500in}{0.275000in}}{\pgfqpoint{4.225000in}{4.225000in}}%
\pgfusepath{clip}%
\pgfsetbuttcap%
\pgfsetroundjoin%
\definecolor{currentfill}{rgb}{0.000000,0.000000,0.000000}%
\pgfsetfillcolor{currentfill}%
\pgfsetfillopacity{0.800000}%
\pgfsetlinewidth{0.000000pt}%
\definecolor{currentstroke}{rgb}{0.000000,0.000000,0.000000}%
\pgfsetstrokecolor{currentstroke}%
\pgfsetstrokeopacity{0.800000}%
\pgfsetdash{}{0pt}%
\pgfpathmoveto{\pgfqpoint{2.421414in}{0.964107in}}%
\pgfpathcurveto{\pgfqpoint{2.425532in}{0.964107in}}{\pgfqpoint{2.429482in}{0.965743in}}{\pgfqpoint{2.432394in}{0.968655in}}%
\pgfpathcurveto{\pgfqpoint{2.435306in}{0.971567in}}{\pgfqpoint{2.436943in}{0.975517in}}{\pgfqpoint{2.436943in}{0.979635in}}%
\pgfpathcurveto{\pgfqpoint{2.436943in}{0.983753in}}{\pgfqpoint{2.435306in}{0.987703in}}{\pgfqpoint{2.432394in}{0.990615in}}%
\pgfpathcurveto{\pgfqpoint{2.429482in}{0.993527in}}{\pgfqpoint{2.425532in}{0.995163in}}{\pgfqpoint{2.421414in}{0.995163in}}%
\pgfpathcurveto{\pgfqpoint{2.417296in}{0.995163in}}{\pgfqpoint{2.413346in}{0.993527in}}{\pgfqpoint{2.410434in}{0.990615in}}%
\pgfpathcurveto{\pgfqpoint{2.407522in}{0.987703in}}{\pgfqpoint{2.405886in}{0.983753in}}{\pgfqpoint{2.405886in}{0.979635in}}%
\pgfpathcurveto{\pgfqpoint{2.405886in}{0.975517in}}{\pgfqpoint{2.407522in}{0.971567in}}{\pgfqpoint{2.410434in}{0.968655in}}%
\pgfpathcurveto{\pgfqpoint{2.413346in}{0.965743in}}{\pgfqpoint{2.417296in}{0.964107in}}{\pgfqpoint{2.421414in}{0.964107in}}%
\pgfpathclose%
\pgfusepath{fill}%
\end{pgfscope}%
\begin{pgfscope}%
\pgfpathrectangle{\pgfqpoint{0.887500in}{0.275000in}}{\pgfqpoint{4.225000in}{4.225000in}}%
\pgfusepath{clip}%
\pgfsetbuttcap%
\pgfsetroundjoin%
\definecolor{currentfill}{rgb}{0.000000,0.000000,0.000000}%
\pgfsetfillcolor{currentfill}%
\pgfsetfillopacity{0.800000}%
\pgfsetlinewidth{0.000000pt}%
\definecolor{currentstroke}{rgb}{0.000000,0.000000,0.000000}%
\pgfsetstrokecolor{currentstroke}%
\pgfsetstrokeopacity{0.800000}%
\pgfsetdash{}{0pt}%
\pgfpathmoveto{\pgfqpoint{2.607461in}{0.888360in}}%
\pgfpathcurveto{\pgfqpoint{2.611580in}{0.888360in}}{\pgfqpoint{2.615530in}{0.889996in}}{\pgfqpoint{2.618441in}{0.892908in}}%
\pgfpathcurveto{\pgfqpoint{2.621353in}{0.895820in}}{\pgfqpoint{2.622990in}{0.899770in}}{\pgfqpoint{2.622990in}{0.903888in}}%
\pgfpathcurveto{\pgfqpoint{2.622990in}{0.908006in}}{\pgfqpoint{2.621353in}{0.911956in}}{\pgfqpoint{2.618441in}{0.914868in}}%
\pgfpathcurveto{\pgfqpoint{2.615530in}{0.917780in}}{\pgfqpoint{2.611580in}{0.919416in}}{\pgfqpoint{2.607461in}{0.919416in}}%
\pgfpathcurveto{\pgfqpoint{2.603343in}{0.919416in}}{\pgfqpoint{2.599393in}{0.917780in}}{\pgfqpoint{2.596481in}{0.914868in}}%
\pgfpathcurveto{\pgfqpoint{2.593569in}{0.911956in}}{\pgfqpoint{2.591933in}{0.908006in}}{\pgfqpoint{2.591933in}{0.903888in}}%
\pgfpathcurveto{\pgfqpoint{2.591933in}{0.899770in}}{\pgfqpoint{2.593569in}{0.895820in}}{\pgfqpoint{2.596481in}{0.892908in}}%
\pgfpathcurveto{\pgfqpoint{2.599393in}{0.889996in}}{\pgfqpoint{2.603343in}{0.888360in}}{\pgfqpoint{2.607461in}{0.888360in}}%
\pgfpathclose%
\pgfusepath{fill}%
\end{pgfscope}%
\begin{pgfscope}%
\pgfpathrectangle{\pgfqpoint{0.887500in}{0.275000in}}{\pgfqpoint{4.225000in}{4.225000in}}%
\pgfusepath{clip}%
\pgfsetbuttcap%
\pgfsetroundjoin%
\definecolor{currentfill}{rgb}{0.000000,0.000000,0.000000}%
\pgfsetfillcolor{currentfill}%
\pgfsetfillopacity{0.800000}%
\pgfsetlinewidth{0.000000pt}%
\definecolor{currentstroke}{rgb}{0.000000,0.000000,0.000000}%
\pgfsetstrokecolor{currentstroke}%
\pgfsetstrokeopacity{0.800000}%
\pgfsetdash{}{0pt}%
\pgfpathmoveto{\pgfqpoint{2.487775in}{0.850671in}}%
\pgfpathcurveto{\pgfqpoint{2.491894in}{0.850671in}}{\pgfqpoint{2.495844in}{0.852307in}}{\pgfqpoint{2.498756in}{0.855219in}}%
\pgfpathcurveto{\pgfqpoint{2.501667in}{0.858131in}}{\pgfqpoint{2.503304in}{0.862081in}}{\pgfqpoint{2.503304in}{0.866199in}}%
\pgfpathcurveto{\pgfqpoint{2.503304in}{0.870317in}}{\pgfqpoint{2.501667in}{0.874267in}}{\pgfqpoint{2.498756in}{0.877179in}}%
\pgfpathcurveto{\pgfqpoint{2.495844in}{0.880091in}}{\pgfqpoint{2.491894in}{0.881727in}}{\pgfqpoint{2.487775in}{0.881727in}}%
\pgfpathcurveto{\pgfqpoint{2.483657in}{0.881727in}}{\pgfqpoint{2.479707in}{0.880091in}}{\pgfqpoint{2.476795in}{0.877179in}}%
\pgfpathcurveto{\pgfqpoint{2.473883in}{0.874267in}}{\pgfqpoint{2.472247in}{0.870317in}}{\pgfqpoint{2.472247in}{0.866199in}}%
\pgfpathcurveto{\pgfqpoint{2.472247in}{0.862081in}}{\pgfqpoint{2.473883in}{0.858131in}}{\pgfqpoint{2.476795in}{0.855219in}}%
\pgfpathcurveto{\pgfqpoint{2.479707in}{0.852307in}}{\pgfqpoint{2.483657in}{0.850671in}}{\pgfqpoint{2.487775in}{0.850671in}}%
\pgfpathclose%
\pgfusepath{fill}%
\end{pgfscope}%
\begin{pgfscope}%
\definecolor{textcolor}{rgb}{0.150000,0.150000,0.150000}%
\pgfsetstrokecolor{textcolor}%
\pgfsetfillcolor{textcolor}%
\pgftext[x=3.000000in,y=4.583333in,,base]{\color{textcolor}\sffamily\fontsize{12.000000}{14.400000}\selectfont Interpolated Model of \(\displaystyle C_L\) for SD7032 Airfoil}%
\end{pgfscope}%
\begin{pgfscope}%
\pgfsetbuttcap%
\pgfsetmiterjoin%
\definecolor{currentfill}{rgb}{0.917647,0.917647,0.949020}%
\pgfsetfillcolor{currentfill}%
\pgfsetfillopacity{0.800000}%
\pgfsetlinewidth{1.003750pt}%
\definecolor{currentstroke}{rgb}{0.800000,0.800000,0.800000}%
\pgfsetstrokecolor{currentstroke}%
\pgfsetstrokeopacity{0.800000}%
\pgfsetdash{}{0pt}%
\pgfpathmoveto{\pgfqpoint{3.837111in}{4.164806in}}%
\pgfpathlineto{\pgfqpoint{5.005556in}{4.164806in}}%
\pgfpathquadraticcurveto{\pgfqpoint{5.036111in}{4.164806in}}{\pgfqpoint{5.036111in}{4.195361in}}%
\pgfpathlineto{\pgfqpoint{5.036111in}{4.393056in}}%
\pgfpathquadraticcurveto{\pgfqpoint{5.036111in}{4.423611in}}{\pgfqpoint{5.005556in}{4.423611in}}%
\pgfpathlineto{\pgfqpoint{3.837111in}{4.423611in}}%
\pgfpathquadraticcurveto{\pgfqpoint{3.806556in}{4.423611in}}{\pgfqpoint{3.806556in}{4.393056in}}%
\pgfpathlineto{\pgfqpoint{3.806556in}{4.195361in}}%
\pgfpathquadraticcurveto{\pgfqpoint{3.806556in}{4.164806in}}{\pgfqpoint{3.837111in}{4.164806in}}%
\pgfpathclose%
\pgfusepath{stroke,fill}%
\end{pgfscope}%
\begin{pgfscope}%
\pgfsetbuttcap%
\pgfsetroundjoin%
\definecolor{currentfill}{rgb}{0.000000,0.000000,0.000000}%
\pgfsetfillcolor{currentfill}%
\pgfsetfillopacity{0.800000}%
\pgfsetlinewidth{0.000000pt}%
\definecolor{currentstroke}{rgb}{0.000000,0.000000,0.000000}%
\pgfsetstrokecolor{currentstroke}%
\pgfsetstrokeopacity{0.800000}%
\pgfsetdash{}{0pt}%
\pgfsys@defobject{currentmarker}{\pgfqpoint{-0.015528in}{-0.015528in}}{\pgfqpoint{0.015528in}{0.015528in}}{%
\pgfpathmoveto{\pgfqpoint{0.000000in}{-0.015528in}}%
\pgfpathcurveto{\pgfqpoint{0.004118in}{-0.015528in}}{\pgfqpoint{0.008068in}{-0.013892in}}{\pgfqpoint{0.010980in}{-0.010980in}}%
\pgfpathcurveto{\pgfqpoint{0.013892in}{-0.008068in}}{\pgfqpoint{0.015528in}{-0.004118in}}{\pgfqpoint{0.015528in}{0.000000in}}%
\pgfpathcurveto{\pgfqpoint{0.015528in}{0.004118in}}{\pgfqpoint{0.013892in}{0.008068in}}{\pgfqpoint{0.010980in}{0.010980in}}%
\pgfpathcurveto{\pgfqpoint{0.008068in}{0.013892in}}{\pgfqpoint{0.004118in}{0.015528in}}{\pgfqpoint{0.000000in}{0.015528in}}%
\pgfpathcurveto{\pgfqpoint{-0.004118in}{0.015528in}}{\pgfqpoint{-0.008068in}{0.013892in}}{\pgfqpoint{-0.010980in}{0.010980in}}%
\pgfpathcurveto{\pgfqpoint{-0.013892in}{0.008068in}}{\pgfqpoint{-0.015528in}{0.004118in}}{\pgfqpoint{-0.015528in}{0.000000in}}%
\pgfpathcurveto{\pgfqpoint{-0.015528in}{-0.004118in}}{\pgfqpoint{-0.013892in}{-0.008068in}}{\pgfqpoint{-0.010980in}{-0.010980in}}%
\pgfpathcurveto{\pgfqpoint{-0.008068in}{-0.013892in}}{\pgfqpoint{-0.004118in}{-0.015528in}}{\pgfqpoint{0.000000in}{-0.015528in}}%
\pgfpathclose%
\pgfusepath{fill}%
}%
\begin{pgfscope}%
\pgfsys@transformshift{4.020444in}{4.295660in}%
\pgfsys@useobject{currentmarker}{}%
\end{pgfscope}%
\end{pgfscope}%
\begin{pgfscope}%
\definecolor{textcolor}{rgb}{0.150000,0.150000,0.150000}%
\pgfsetstrokecolor{textcolor}%
\pgfsetfillcolor{textcolor}%
\pgftext[x=4.295444in,y=4.255556in,left,base]{\color{textcolor}\sffamily\fontsize{11.000000}{13.200000}\selectfont XFoil Data}%
\end{pgfscope}%
\end{pgfpicture}%
\makeatother%
\endgroup%

    \ifdraft{}{%% Creator: Matplotlib, PGF backend
%%
%% To include the figure in your LaTeX document, write
%%   \input{<filename>.pgf}
%%
%% Make sure the required packages are loaded in your preamble
%%   \usepackage{pgf}
%%
%% Figures using additional raster images can only be included by \input if
%% they are in the same directory as the main LaTeX file. For loading figures
%% from other directories you can use the `import` package
%%   \usepackage{import}
%%
%% and then include the figures with
%%   \import{<path to file>}{<filename>.pgf}
%%
%% Matplotlib used the following preamble
%%   \usepackage{fontspec}
%%
\begingroup%
\makeatletter%
\begin{pgfpicture}%
\pgfpathrectangle{\pgfpointorigin}{\pgfqpoint{6.000000in}{5.000000in}}%
\pgfusepath{use as bounding box, clip}%
\begin{pgfscope}%
\pgfsetbuttcap%
\pgfsetmiterjoin%
\definecolor{currentfill}{rgb}{1.000000,1.000000,1.000000}%
\pgfsetfillcolor{currentfill}%
\pgfsetlinewidth{0.000000pt}%
\definecolor{currentstroke}{rgb}{1.000000,1.000000,1.000000}%
\pgfsetstrokecolor{currentstroke}%
\pgfsetstrokeopacity{0.000000}%
\pgfsetdash{}{0pt}%
\pgfpathmoveto{\pgfqpoint{0.000000in}{0.000000in}}%
\pgfpathlineto{\pgfqpoint{6.000000in}{0.000000in}}%
\pgfpathlineto{\pgfqpoint{6.000000in}{5.000000in}}%
\pgfpathlineto{\pgfqpoint{0.000000in}{5.000000in}}%
\pgfpathclose%
\pgfusepath{fill}%
\end{pgfscope}%
\begin{pgfscope}%
\pgfsetbuttcap%
\pgfsetmiterjoin%
\definecolor{currentfill}{rgb}{0.917647,0.917647,0.949020}%
\pgfsetfillcolor{currentfill}%
\pgfsetlinewidth{0.000000pt}%
\definecolor{currentstroke}{rgb}{0.000000,0.000000,0.000000}%
\pgfsetstrokecolor{currentstroke}%
\pgfsetstrokeopacity{0.000000}%
\pgfsetdash{}{0pt}%
\pgfpathmoveto{\pgfqpoint{0.887500in}{0.275000in}}%
\pgfpathlineto{\pgfqpoint{5.112500in}{0.275000in}}%
\pgfpathlineto{\pgfqpoint{5.112500in}{4.500000in}}%
\pgfpathlineto{\pgfqpoint{0.887500in}{4.500000in}}%
\pgfpathclose%
\pgfusepath{fill}%
\end{pgfscope}%
\begin{pgfscope}%
\pgfsetbuttcap%
\pgfsetmiterjoin%
\definecolor{currentfill}{rgb}{0.950000,0.950000,0.950000}%
\pgfsetfillcolor{currentfill}%
\pgfsetfillopacity{0.500000}%
\pgfsetlinewidth{1.003750pt}%
\definecolor{currentstroke}{rgb}{0.950000,0.950000,0.950000}%
\pgfsetstrokecolor{currentstroke}%
\pgfsetstrokeopacity{0.500000}%
\pgfsetdash{}{0pt}%
\pgfpathmoveto{\pgfqpoint{4.907673in}{1.316750in}}%
\pgfpathlineto{\pgfqpoint{3.512440in}{2.486262in}}%
\pgfpathlineto{\pgfqpoint{3.531835in}{4.172907in}}%
\pgfpathlineto{\pgfqpoint{4.993837in}{3.106004in}}%
\pgfusepath{stroke,fill}%
\end{pgfscope}%
\begin{pgfscope}%
\pgfsetbuttcap%
\pgfsetmiterjoin%
\definecolor{currentfill}{rgb}{0.900000,0.900000,0.900000}%
\pgfsetfillcolor{currentfill}%
\pgfsetfillopacity{0.500000}%
\pgfsetlinewidth{1.003750pt}%
\definecolor{currentstroke}{rgb}{0.900000,0.900000,0.900000}%
\pgfsetstrokecolor{currentstroke}%
\pgfsetstrokeopacity{0.500000}%
\pgfsetdash{}{0pt}%
\pgfpathmoveto{\pgfqpoint{1.273588in}{1.835516in}}%
\pgfpathlineto{\pgfqpoint{3.512440in}{2.486262in}}%
\pgfpathlineto{\pgfqpoint{3.531835in}{4.172907in}}%
\pgfpathlineto{\pgfqpoint{1.193692in}{3.580256in}}%
\pgfusepath{stroke,fill}%
\end{pgfscope}%
\begin{pgfscope}%
\pgfsetbuttcap%
\pgfsetmiterjoin%
\definecolor{currentfill}{rgb}{0.925000,0.925000,0.925000}%
\pgfsetfillcolor{currentfill}%
\pgfsetfillopacity{0.500000}%
\pgfsetlinewidth{1.003750pt}%
\definecolor{currentstroke}{rgb}{0.925000,0.925000,0.925000}%
\pgfsetstrokecolor{currentstroke}%
\pgfsetstrokeopacity{0.500000}%
\pgfsetdash{}{0pt}%
\pgfpathmoveto{\pgfqpoint{2.534381in}{0.541633in}}%
\pgfpathlineto{\pgfqpoint{4.907673in}{1.316750in}}%
\pgfpathlineto{\pgfqpoint{3.512440in}{2.486262in}}%
\pgfpathlineto{\pgfqpoint{1.273588in}{1.835516in}}%
\pgfusepath{stroke,fill}%
\end{pgfscope}%
\begin{pgfscope}%
\pgfsetroundcap%
\pgfsetroundjoin%
\pgfsetlinewidth{1.254687pt}%
\definecolor{currentstroke}{rgb}{1.000000,1.000000,1.000000}%
\pgfsetstrokecolor{currentstroke}%
\pgfsetdash{}{0pt}%
\pgfpathmoveto{\pgfqpoint{4.907673in}{1.316750in}}%
\pgfpathlineto{\pgfqpoint{2.534381in}{0.541633in}}%
\pgfusepath{stroke}%
\end{pgfscope}%
\begin{pgfscope}%
\definecolor{textcolor}{rgb}{0.150000,0.150000,0.150000}%
\pgfsetstrokecolor{textcolor}%
\pgfsetfillcolor{textcolor}%
\pgftext[x=3.420909in, y=0.225013in, left, base,rotate=18.087038]{\color{textcolor}\sffamily\fontsize{12.000000}{14.400000}\selectfont Reynolds Number}%
\end{pgfscope}%
\begin{pgfscope}%
\pgfsetbuttcap%
\pgfsetroundjoin%
\pgfsetlinewidth{1.003750pt}%
\definecolor{currentstroke}{rgb}{1.000000,1.000000,1.000000}%
\pgfsetstrokecolor{currentstroke}%
\pgfsetdash{}{0pt}%
\pgfpathmoveto{\pgfqpoint{2.584363in}{0.557957in}}%
\pgfpathlineto{\pgfqpoint{1.320536in}{1.849162in}}%
\pgfpathlineto{\pgfqpoint{1.242823in}{3.592709in}}%
\pgfusepath{stroke}%
\end{pgfscope}%
\begin{pgfscope}%
\pgfsetbuttcap%
\pgfsetroundjoin%
\pgfsetlinewidth{1.003750pt}%
\definecolor{currentstroke}{rgb}{1.000000,1.000000,1.000000}%
\pgfsetstrokecolor{currentstroke}%
\pgfsetdash{}{0pt}%
\pgfpathmoveto{\pgfqpoint{3.369619in}{0.814421in}}%
\pgfpathlineto{\pgfqpoint{2.059260in}{2.063880in}}%
\pgfpathlineto{\pgfqpoint{2.015332in}{3.788518in}}%
\pgfusepath{stroke}%
\end{pgfscope}%
\begin{pgfscope}%
\pgfsetbuttcap%
\pgfsetroundjoin%
\pgfsetlinewidth{1.003750pt}%
\definecolor{currentstroke}{rgb}{1.000000,1.000000,1.000000}%
\pgfsetstrokecolor{currentstroke}%
\pgfsetdash{}{0pt}%
\pgfpathmoveto{\pgfqpoint{4.128599in}{1.062304in}}%
\pgfpathlineto{\pgfqpoint{2.775317in}{2.272009in}}%
\pgfpathlineto{\pgfqpoint{2.763111in}{3.978058in}}%
\pgfusepath{stroke}%
\end{pgfscope}%
\begin{pgfscope}%
\pgfsetbuttcap%
\pgfsetroundjoin%
\pgfsetlinewidth{1.003750pt}%
\definecolor{currentstroke}{rgb}{1.000000,1.000000,1.000000}%
\pgfsetstrokecolor{currentstroke}%
\pgfsetdash{}{0pt}%
\pgfpathmoveto{\pgfqpoint{4.862599in}{1.302029in}}%
\pgfpathlineto{\pgfqpoint{3.469735in}{2.473849in}}%
\pgfpathlineto{\pgfqpoint{3.487328in}{4.161626in}}%
\pgfusepath{stroke}%
\end{pgfscope}%
\begin{pgfscope}%
\pgfsetbuttcap%
\pgfsetroundjoin%
\pgfsetlinewidth{1.003750pt}%
\definecolor{currentstroke}{rgb}{1.000000,1.000000,1.000000}%
\pgfsetstrokecolor{currentstroke}%
\pgfsetdash{}{0pt}%
\pgfpathmoveto{\pgfqpoint{2.823594in}{0.636090in}}%
\pgfpathlineto{\pgfqpoint{1.545362in}{1.914510in}}%
\pgfpathlineto{\pgfqpoint{1.478046in}{3.652331in}}%
\pgfusepath{stroke}%
\end{pgfscope}%
\begin{pgfscope}%
\pgfsetbuttcap%
\pgfsetroundjoin%
\pgfsetlinewidth{1.003750pt}%
\definecolor{currentstroke}{rgb}{1.000000,1.000000,1.000000}%
\pgfsetstrokecolor{currentstroke}%
\pgfsetdash{}{0pt}%
\pgfpathmoveto{\pgfqpoint{3.600798in}{0.889924in}}%
\pgfpathlineto{\pgfqpoint{2.277151in}{2.127212in}}%
\pgfpathlineto{\pgfqpoint{2.242983in}{3.846220in}}%
\pgfusepath{stroke}%
\end{pgfscope}%
\begin{pgfscope}%
\pgfsetbuttcap%
\pgfsetroundjoin%
\pgfsetlinewidth{1.003750pt}%
\definecolor{currentstroke}{rgb}{1.000000,1.000000,1.000000}%
\pgfsetstrokecolor{currentstroke}%
\pgfsetdash{}{0pt}%
\pgfpathmoveto{\pgfqpoint{4.352126in}{1.135308in}}%
\pgfpathlineto{\pgfqpoint{2.986589in}{2.333417in}}%
\pgfpathlineto{\pgfqpoint{2.983549in}{4.033932in}}%
\pgfusepath{stroke}%
\end{pgfscope}%
\begin{pgfscope}%
\pgfsetbuttcap%
\pgfsetroundjoin%
\pgfsetlinewidth{1.003750pt}%
\definecolor{currentstroke}{rgb}{1.000000,1.000000,1.000000}%
\pgfsetstrokecolor{currentstroke}%
\pgfsetdash{}{0pt}%
\pgfpathmoveto{\pgfqpoint{2.962389in}{0.681420in}}%
\pgfpathlineto{\pgfqpoint{1.675892in}{1.952450in}}%
\pgfpathlineto{\pgfqpoint{1.614565in}{3.686935in}}%
\pgfusepath{stroke}%
\end{pgfscope}%
\begin{pgfscope}%
\pgfsetbuttcap%
\pgfsetroundjoin%
\pgfsetlinewidth{1.003750pt}%
\definecolor{currentstroke}{rgb}{1.000000,1.000000,1.000000}%
\pgfsetstrokecolor{currentstroke}%
\pgfsetdash{}{0pt}%
\pgfpathmoveto{\pgfqpoint{3.734940in}{0.933735in}}%
\pgfpathlineto{\pgfqpoint{2.403669in}{2.163986in}}%
\pgfpathlineto{\pgfqpoint{2.375124in}{3.879714in}}%
\pgfusepath{stroke}%
\end{pgfscope}%
\begin{pgfscope}%
\pgfsetbuttcap%
\pgfsetroundjoin%
\pgfsetlinewidth{1.003750pt}%
\definecolor{currentstroke}{rgb}{1.000000,1.000000,1.000000}%
\pgfsetstrokecolor{currentstroke}%
\pgfsetdash{}{0pt}%
\pgfpathmoveto{\pgfqpoint{4.481845in}{1.177674in}}%
\pgfpathlineto{\pgfqpoint{3.109277in}{2.369078in}}%
\pgfpathlineto{\pgfqpoint{3.111520in}{4.066369in}}%
\pgfusepath{stroke}%
\end{pgfscope}%
\begin{pgfscope}%
\pgfsetbuttcap%
\pgfsetroundjoin%
\pgfsetlinewidth{1.003750pt}%
\definecolor{currentstroke}{rgb}{1.000000,1.000000,1.000000}%
\pgfsetstrokecolor{currentstroke}%
\pgfsetdash{}{0pt}%
\pgfpathmoveto{\pgfqpoint{3.060357in}{0.713417in}}%
\pgfpathlineto{\pgfqpoint{1.768067in}{1.979241in}}%
\pgfpathlineto{\pgfqpoint{1.710950in}{3.711366in}}%
\pgfusepath{stroke}%
\end{pgfscope}%
\begin{pgfscope}%
\pgfsetbuttcap%
\pgfsetroundjoin%
\pgfsetlinewidth{1.003750pt}%
\definecolor{currentstroke}{rgb}{1.000000,1.000000,1.000000}%
\pgfsetstrokecolor{currentstroke}%
\pgfsetdash{}{0pt}%
\pgfpathmoveto{\pgfqpoint{3.829633in}{0.964662in}}%
\pgfpathlineto{\pgfqpoint{2.493017in}{2.189956in}}%
\pgfpathlineto{\pgfqpoint{2.468425in}{3.903364in}}%
\pgfusepath{stroke}%
\end{pgfscope}%
\begin{pgfscope}%
\pgfsetbuttcap%
\pgfsetroundjoin%
\pgfsetlinewidth{1.003750pt}%
\definecolor{currentstroke}{rgb}{1.000000,1.000000,1.000000}%
\pgfsetstrokecolor{currentstroke}%
\pgfsetdash{}{0pt}%
\pgfpathmoveto{\pgfqpoint{4.573424in}{1.207584in}}%
\pgfpathlineto{\pgfqpoint{3.195927in}{2.394264in}}%
\pgfpathlineto{\pgfqpoint{3.201884in}{4.089274in}}%
\pgfusepath{stroke}%
\end{pgfscope}%
\begin{pgfscope}%
\pgfsetbuttcap%
\pgfsetroundjoin%
\pgfsetlinewidth{1.003750pt}%
\definecolor{currentstroke}{rgb}{1.000000,1.000000,1.000000}%
\pgfsetstrokecolor{currentstroke}%
\pgfsetdash{}{0pt}%
\pgfpathmoveto{\pgfqpoint{3.136060in}{0.738141in}}%
\pgfpathlineto{\pgfqpoint{1.839315in}{1.999950in}}%
\pgfpathlineto{\pgfqpoint{1.785440in}{3.730247in}}%
\pgfusepath{stroke}%
\end{pgfscope}%
\begin{pgfscope}%
\pgfsetbuttcap%
\pgfsetroundjoin%
\pgfsetlinewidth{1.003750pt}%
\definecolor{currentstroke}{rgb}{1.000000,1.000000,1.000000}%
\pgfsetstrokecolor{currentstroke}%
\pgfsetdash{}{0pt}%
\pgfpathmoveto{\pgfqpoint{3.902809in}{0.988561in}}%
\pgfpathlineto{\pgfqpoint{2.562084in}{2.210031in}}%
\pgfpathlineto{\pgfqpoint{2.540537in}{3.921642in}}%
\pgfusepath{stroke}%
\end{pgfscope}%
\begin{pgfscope}%
\pgfsetbuttcap%
\pgfsetroundjoin%
\pgfsetlinewidth{1.003750pt}%
\definecolor{currentstroke}{rgb}{1.000000,1.000000,1.000000}%
\pgfsetstrokecolor{currentstroke}%
\pgfsetdash{}{0pt}%
\pgfpathmoveto{\pgfqpoint{4.644197in}{1.230698in}}%
\pgfpathlineto{\pgfqpoint{3.262912in}{2.413734in}}%
\pgfpathlineto{\pgfqpoint{3.271729in}{4.106978in}}%
\pgfusepath{stroke}%
\end{pgfscope}%
\begin{pgfscope}%
\pgfsetbuttcap%
\pgfsetroundjoin%
\pgfsetlinewidth{1.003750pt}%
\definecolor{currentstroke}{rgb}{1.000000,1.000000,1.000000}%
\pgfsetstrokecolor{currentstroke}%
\pgfsetdash{}{0pt}%
\pgfpathmoveto{\pgfqpoint{3.197727in}{0.758281in}}%
\pgfpathlineto{\pgfqpoint{1.897369in}{2.016824in}}%
\pgfpathlineto{\pgfqpoint{1.846129in}{3.745630in}}%
\pgfusepath{stroke}%
\end{pgfscope}%
\begin{pgfscope}%
\pgfsetbuttcap%
\pgfsetroundjoin%
\pgfsetlinewidth{1.003750pt}%
\definecolor{currentstroke}{rgb}{1.000000,1.000000,1.000000}%
\pgfsetstrokecolor{currentstroke}%
\pgfsetdash{}{0pt}%
\pgfpathmoveto{\pgfqpoint{3.962420in}{1.008030in}}%
\pgfpathlineto{\pgfqpoint{2.618363in}{2.226389in}}%
\pgfpathlineto{\pgfqpoint{2.599290in}{3.936534in}}%
\pgfusepath{stroke}%
\end{pgfscope}%
\begin{pgfscope}%
\pgfsetbuttcap%
\pgfsetroundjoin%
\pgfsetlinewidth{1.003750pt}%
\definecolor{currentstroke}{rgb}{1.000000,1.000000,1.000000}%
\pgfsetstrokecolor{currentstroke}%
\pgfsetdash{}{0pt}%
\pgfpathmoveto{\pgfqpoint{4.701855in}{1.249529in}}%
\pgfpathlineto{\pgfqpoint{3.317496in}{2.429599in}}%
\pgfpathlineto{\pgfqpoint{3.328638in}{4.121402in}}%
\pgfusepath{stroke}%
\end{pgfscope}%
\begin{pgfscope}%
\pgfsetbuttcap%
\pgfsetroundjoin%
\pgfsetlinewidth{1.003750pt}%
\definecolor{currentstroke}{rgb}{1.000000,1.000000,1.000000}%
\pgfsetstrokecolor{currentstroke}%
\pgfsetdash{}{0pt}%
\pgfpathmoveto{\pgfqpoint{3.249735in}{0.775267in}}%
\pgfpathlineto{\pgfqpoint{1.946340in}{2.031058in}}%
\pgfpathlineto{\pgfqpoint{1.897318in}{3.758604in}}%
\pgfusepath{stroke}%
\end{pgfscope}%
\begin{pgfscope}%
\pgfsetbuttcap%
\pgfsetroundjoin%
\pgfsetlinewidth{1.003750pt}%
\definecolor{currentstroke}{rgb}{1.000000,1.000000,1.000000}%
\pgfsetstrokecolor{currentstroke}%
\pgfsetdash{}{0pt}%
\pgfpathmoveto{\pgfqpoint{4.012698in}{1.024451in}}%
\pgfpathlineto{\pgfqpoint{2.665840in}{2.240188in}}%
\pgfpathlineto{\pgfqpoint{2.648849in}{3.949096in}}%
\pgfusepath{stroke}%
\end{pgfscope}%
\begin{pgfscope}%
\pgfsetbuttcap%
\pgfsetroundjoin%
\pgfsetlinewidth{1.003750pt}%
\definecolor{currentstroke}{rgb}{1.000000,1.000000,1.000000}%
\pgfsetstrokecolor{currentstroke}%
\pgfsetdash{}{0pt}%
\pgfpathmoveto{\pgfqpoint{4.750486in}{1.265412in}}%
\pgfpathlineto{\pgfqpoint{3.363545in}{2.442984in}}%
\pgfpathlineto{\pgfqpoint{3.376643in}{4.133570in}}%
\pgfusepath{stroke}%
\end{pgfscope}%
\begin{pgfscope}%
\pgfsetbuttcap%
\pgfsetroundjoin%
\pgfsetlinewidth{1.003750pt}%
\definecolor{currentstroke}{rgb}{1.000000,1.000000,1.000000}%
\pgfsetstrokecolor{currentstroke}%
\pgfsetdash{}{0pt}%
\pgfpathmoveto{\pgfqpoint{3.294691in}{0.789950in}}%
\pgfpathlineto{\pgfqpoint{1.988679in}{2.043364in}}%
\pgfpathlineto{\pgfqpoint{1.941569in}{3.769821in}}%
\pgfusepath{stroke}%
\end{pgfscope}%
\begin{pgfscope}%
\pgfsetbuttcap%
\pgfsetroundjoin%
\pgfsetlinewidth{1.003750pt}%
\definecolor{currentstroke}{rgb}{1.000000,1.000000,1.000000}%
\pgfsetstrokecolor{currentstroke}%
\pgfsetdash{}{0pt}%
\pgfpathmoveto{\pgfqpoint{4.056159in}{1.038645in}}%
\pgfpathlineto{\pgfqpoint{2.706887in}{2.252119in}}%
\pgfpathlineto{\pgfqpoint{2.691692in}{3.959955in}}%
\pgfusepath{stroke}%
\end{pgfscope}%
\begin{pgfscope}%
\pgfsetbuttcap%
\pgfsetroundjoin%
\pgfsetlinewidth{1.003750pt}%
\definecolor{currentstroke}{rgb}{1.000000,1.000000,1.000000}%
\pgfsetstrokecolor{currentstroke}%
\pgfsetdash{}{0pt}%
\pgfpathmoveto{\pgfqpoint{4.792526in}{1.279143in}}%
\pgfpathlineto{\pgfqpoint{3.403358in}{2.454556in}}%
\pgfpathlineto{\pgfqpoint{3.418144in}{4.144090in}}%
\pgfusepath{stroke}%
\end{pgfscope}%
\begin{pgfscope}%
\pgfsetbuttcap%
\pgfsetroundjoin%
\pgfsetlinewidth{1.003750pt}%
\definecolor{currentstroke}{rgb}{1.000000,1.000000,1.000000}%
\pgfsetstrokecolor{currentstroke}%
\pgfsetdash{}{0pt}%
\pgfpathmoveto{\pgfqpoint{3.334272in}{0.802877in}}%
\pgfpathlineto{\pgfqpoint{2.025960in}{2.054201in}}%
\pgfpathlineto{\pgfqpoint{1.980533in}{3.779697in}}%
\pgfusepath{stroke}%
\end{pgfscope}%
\begin{pgfscope}%
\pgfsetbuttcap%
\pgfsetroundjoin%
\pgfsetlinewidth{1.003750pt}%
\definecolor{currentstroke}{rgb}{1.000000,1.000000,1.000000}%
\pgfsetstrokecolor{currentstroke}%
\pgfsetdash{}{0pt}%
\pgfpathmoveto{\pgfqpoint{4.094425in}{1.051143in}}%
\pgfpathlineto{\pgfqpoint{2.743032in}{2.262625in}}%
\pgfpathlineto{\pgfqpoint{2.729417in}{3.969517in}}%
\pgfusepath{stroke}%
\end{pgfscope}%
\begin{pgfscope}%
\pgfsetbuttcap%
\pgfsetroundjoin%
\pgfsetlinewidth{1.003750pt}%
\definecolor{currentstroke}{rgb}{1.000000,1.000000,1.000000}%
\pgfsetstrokecolor{currentstroke}%
\pgfsetdash{}{0pt}%
\pgfpathmoveto{\pgfqpoint{4.829541in}{1.291232in}}%
\pgfpathlineto{\pgfqpoint{3.438418in}{2.464746in}}%
\pgfpathlineto{\pgfqpoint{3.454688in}{4.153352in}}%
\pgfusepath{stroke}%
\end{pgfscope}%
\begin{pgfscope}%
\pgfpathrectangle{\pgfqpoint{0.887500in}{0.275000in}}{\pgfqpoint{4.225000in}{4.225000in}}%
\pgfusepath{clip}%
\pgfsetroundcap%
\pgfsetroundjoin%
\pgfsetlinewidth{1.003750pt}%
\definecolor{currentstroke}{rgb}{1.000000,1.000000,1.000000}%
\pgfsetstrokecolor{currentstroke}%
\pgfsetdash{}{0pt}%
\pgfpathmoveto{\pgfqpoint{3.057095in}{2.444595in}}%
\pgfusepath{stroke}%
\end{pgfscope}%
\begin{pgfscope}%
\pgfsetroundcap%
\pgfsetroundjoin%
\pgfsetlinewidth{1.254687pt}%
\definecolor{currentstroke}{rgb}{0.150000,0.150000,0.150000}%
\pgfsetstrokecolor{currentstroke}%
\pgfsetdash{}{0pt}%
\pgfpathmoveto{\pgfqpoint{2.573314in}{0.569246in}}%
\pgfpathlineto{\pgfqpoint{2.606512in}{0.535328in}}%
\pgfusepath{stroke}%
\end{pgfscope}%
\begin{pgfscope}%
\definecolor{textcolor}{rgb}{0.150000,0.150000,0.150000}%
\pgfsetstrokecolor{textcolor}%
\pgfsetfillcolor{textcolor}%
\pgftext[x=2.686647in,y=0.326577in,,top]{\color{textcolor}\sffamily\fontsize{11.000000}{13.200000}\selectfont \(\displaystyle 10^3\)}%
\end{pgfscope}%
\begin{pgfscope}%
\pgfpathrectangle{\pgfqpoint{0.887500in}{0.275000in}}{\pgfqpoint{4.225000in}{4.225000in}}%
\pgfusepath{clip}%
\pgfsetroundcap%
\pgfsetroundjoin%
\pgfsetlinewidth{1.003750pt}%
\definecolor{currentstroke}{rgb}{1.000000,1.000000,1.000000}%
\pgfsetstrokecolor{currentstroke}%
\pgfsetdash{}{0pt}%
\pgfpathmoveto{\pgfqpoint{3.057095in}{2.444595in}}%
\pgfusepath{stroke}%
\end{pgfscope}%
\begin{pgfscope}%
\pgfsetroundcap%
\pgfsetroundjoin%
\pgfsetlinewidth{1.254687pt}%
\definecolor{currentstroke}{rgb}{0.150000,0.150000,0.150000}%
\pgfsetstrokecolor{currentstroke}%
\pgfsetdash{}{0pt}%
\pgfpathmoveto{\pgfqpoint{3.358179in}{0.825330in}}%
\pgfpathlineto{\pgfqpoint{3.392550in}{0.792557in}}%
\pgfusepath{stroke}%
\end{pgfscope}%
\begin{pgfscope}%
\definecolor{textcolor}{rgb}{0.150000,0.150000,0.150000}%
\pgfsetstrokecolor{textcolor}%
\pgfsetfillcolor{textcolor}%
\pgftext[x=3.472790in,y=0.587852in,,top]{\color{textcolor}\sffamily\fontsize{11.000000}{13.200000}\selectfont \(\displaystyle 10^4\)}%
\end{pgfscope}%
\begin{pgfscope}%
\pgfpathrectangle{\pgfqpoint{0.887500in}{0.275000in}}{\pgfqpoint{4.225000in}{4.225000in}}%
\pgfusepath{clip}%
\pgfsetroundcap%
\pgfsetroundjoin%
\pgfsetlinewidth{1.003750pt}%
\definecolor{currentstroke}{rgb}{1.000000,1.000000,1.000000}%
\pgfsetstrokecolor{currentstroke}%
\pgfsetdash{}{0pt}%
\pgfpathmoveto{\pgfqpoint{3.057095in}{2.444595in}}%
\pgfusepath{stroke}%
\end{pgfscope}%
\begin{pgfscope}%
\pgfsetroundcap%
\pgfsetroundjoin%
\pgfsetlinewidth{1.254687pt}%
\definecolor{currentstroke}{rgb}{0.150000,0.150000,0.150000}%
\pgfsetstrokecolor{currentstroke}%
\pgfsetdash{}{0pt}%
\pgfpathmoveto{\pgfqpoint{4.116801in}{1.072850in}}%
\pgfpathlineto{\pgfqpoint{4.152246in}{1.041165in}}%
\pgfusepath{stroke}%
\end{pgfscope}%
\begin{pgfscope}%
\definecolor{textcolor}{rgb}{0.150000,0.150000,0.150000}%
\pgfsetstrokecolor{textcolor}%
\pgfsetfillcolor{textcolor}%
\pgftext[x=4.232540in,y=0.840355in,,top]{\color{textcolor}\sffamily\fontsize{11.000000}{13.200000}\selectfont \(\displaystyle 10^5\)}%
\end{pgfscope}%
\begin{pgfscope}%
\pgfpathrectangle{\pgfqpoint{0.887500in}{0.275000in}}{\pgfqpoint{4.225000in}{4.225000in}}%
\pgfusepath{clip}%
\pgfsetroundcap%
\pgfsetroundjoin%
\pgfsetlinewidth{1.003750pt}%
\definecolor{currentstroke}{rgb}{1.000000,1.000000,1.000000}%
\pgfsetstrokecolor{currentstroke}%
\pgfsetdash{}{0pt}%
\pgfpathmoveto{\pgfqpoint{3.057095in}{2.444595in}}%
\pgfusepath{stroke}%
\end{pgfscope}%
\begin{pgfscope}%
\pgfsetroundcap%
\pgfsetroundjoin%
\pgfsetlinewidth{1.254687pt}%
\definecolor{currentstroke}{rgb}{0.150000,0.150000,0.150000}%
\pgfsetstrokecolor{currentstroke}%
\pgfsetdash{}{0pt}%
\pgfpathmoveto{\pgfqpoint{4.850472in}{1.312231in}}%
\pgfpathlineto{\pgfqpoint{4.886904in}{1.281580in}}%
\pgfusepath{stroke}%
\end{pgfscope}%
\begin{pgfscope}%
\definecolor{textcolor}{rgb}{0.150000,0.150000,0.150000}%
\pgfsetstrokecolor{textcolor}%
\pgfsetfillcolor{textcolor}%
\pgftext[x=4.967205in,y=1.084521in,,top]{\color{textcolor}\sffamily\fontsize{11.000000}{13.200000}\selectfont \(\displaystyle 10^6\)}%
\end{pgfscope}%
\begin{pgfscope}%
\pgfsetroundcap%
\pgfsetroundjoin%
\pgfsetlinewidth{1.003750pt}%
\definecolor{currentstroke}{rgb}{0.000000,0.000000,0.000000}%
\pgfsetstrokecolor{currentstroke}%
\pgfsetstrokeopacity{0.300000}%
\pgfsetdash{}{0pt}%
\pgfpathmoveto{\pgfqpoint{2.812423in}{0.647262in}}%
\pgfpathlineto{\pgfqpoint{2.845985in}{0.613695in}}%
\pgfusepath{stroke}%
\end{pgfscope}%
\begin{pgfscope}%
\pgfsetroundcap%
\pgfsetroundjoin%
\pgfsetlinewidth{1.003750pt}%
\definecolor{currentstroke}{rgb}{0.000000,0.000000,0.000000}%
\pgfsetstrokecolor{currentstroke}%
\pgfsetstrokeopacity{0.300000}%
\pgfsetdash{}{0pt}%
\pgfpathmoveto{\pgfqpoint{3.589247in}{0.900722in}}%
\pgfpathlineto{\pgfqpoint{3.623951in}{0.868282in}}%
\pgfusepath{stroke}%
\end{pgfscope}%
\begin{pgfscope}%
\pgfsetroundcap%
\pgfsetroundjoin%
\pgfsetlinewidth{1.003750pt}%
\definecolor{currentstroke}{rgb}{0.000000,0.000000,0.000000}%
\pgfsetstrokecolor{currentstroke}%
\pgfsetstrokeopacity{0.300000}%
\pgfsetdash{}{0pt}%
\pgfpathmoveto{\pgfqpoint{4.340226in}{1.145749in}}%
\pgfpathlineto{\pgfqpoint{4.375977in}{1.114381in}}%
\pgfusepath{stroke}%
\end{pgfscope}%
\begin{pgfscope}%
\pgfsetroundcap%
\pgfsetroundjoin%
\pgfsetlinewidth{1.003750pt}%
\definecolor{currentstroke}{rgb}{0.000000,0.000000,0.000000}%
\pgfsetstrokecolor{currentstroke}%
\pgfsetstrokeopacity{0.300000}%
\pgfsetdash{}{0pt}%
\pgfpathmoveto{\pgfqpoint{2.951149in}{0.692525in}}%
\pgfpathlineto{\pgfqpoint{2.984919in}{0.659161in}}%
\pgfusepath{stroke}%
\end{pgfscope}%
\begin{pgfscope}%
\pgfsetroundcap%
\pgfsetroundjoin%
\pgfsetlinewidth{1.003750pt}%
\definecolor{currentstroke}{rgb}{0.000000,0.000000,0.000000}%
\pgfsetstrokecolor{currentstroke}%
\pgfsetstrokeopacity{0.300000}%
\pgfsetdash{}{0pt}%
\pgfpathmoveto{\pgfqpoint{3.723326in}{0.944469in}}%
\pgfpathlineto{\pgfqpoint{3.758221in}{0.912222in}}%
\pgfusepath{stroke}%
\end{pgfscope}%
\begin{pgfscope}%
\pgfsetroundcap%
\pgfsetroundjoin%
\pgfsetlinewidth{1.003750pt}%
\definecolor{currentstroke}{rgb}{0.000000,0.000000,0.000000}%
\pgfsetstrokecolor{currentstroke}%
\pgfsetstrokeopacity{0.300000}%
\pgfsetdash{}{0pt}%
\pgfpathmoveto{\pgfqpoint{4.469887in}{1.188054in}}%
\pgfpathlineto{\pgfqpoint{4.505814in}{1.156869in}}%
\pgfusepath{stroke}%
\end{pgfscope}%
\begin{pgfscope}%
\pgfsetroundcap%
\pgfsetroundjoin%
\pgfsetlinewidth{1.003750pt}%
\definecolor{currentstroke}{rgb}{0.000000,0.000000,0.000000}%
\pgfsetstrokecolor{currentstroke}%
\pgfsetstrokeopacity{0.300000}%
\pgfsetdash{}{0pt}%
\pgfpathmoveto{\pgfqpoint{3.049069in}{0.724474in}}%
\pgfpathlineto{\pgfqpoint{3.082985in}{0.691252in}}%
\pgfusepath{stroke}%
\end{pgfscope}%
\begin{pgfscope}%
\pgfsetroundcap%
\pgfsetroundjoin%
\pgfsetlinewidth{1.003750pt}%
\definecolor{currentstroke}{rgb}{0.000000,0.000000,0.000000}%
\pgfsetstrokecolor{currentstroke}%
\pgfsetstrokeopacity{0.300000}%
\pgfsetdash{}{0pt}%
\pgfpathmoveto{\pgfqpoint{3.817974in}{0.975350in}}%
\pgfpathlineto{\pgfqpoint{3.853003in}{0.943239in}}%
\pgfusepath{stroke}%
\end{pgfscope}%
\begin{pgfscope}%
\pgfsetroundcap%
\pgfsetroundjoin%
\pgfsetlinewidth{1.003750pt}%
\definecolor{currentstroke}{rgb}{0.000000,0.000000,0.000000}%
\pgfsetstrokecolor{currentstroke}%
\pgfsetstrokeopacity{0.300000}%
\pgfsetdash{}{0pt}%
\pgfpathmoveto{\pgfqpoint{4.561424in}{1.217921in}}%
\pgfpathlineto{\pgfqpoint{4.597474in}{1.186865in}}%
\pgfusepath{stroke}%
\end{pgfscope}%
\begin{pgfscope}%
\pgfsetroundcap%
\pgfsetroundjoin%
\pgfsetlinewidth{1.003750pt}%
\definecolor{currentstroke}{rgb}{0.000000,0.000000,0.000000}%
\pgfsetstrokecolor{currentstroke}%
\pgfsetstrokeopacity{0.300000}%
\pgfsetdash{}{0pt}%
\pgfpathmoveto{\pgfqpoint{3.124734in}{0.749162in}}%
\pgfpathlineto{\pgfqpoint{3.158762in}{0.716050in}}%
\pgfusepath{stroke}%
\end{pgfscope}%
\begin{pgfscope}%
\pgfsetroundcap%
\pgfsetroundjoin%
\pgfsetlinewidth{1.003750pt}%
\definecolor{currentstroke}{rgb}{0.000000,0.000000,0.000000}%
\pgfsetstrokecolor{currentstroke}%
\pgfsetstrokeopacity{0.300000}%
\pgfsetdash{}{0pt}%
\pgfpathmoveto{\pgfqpoint{3.891115in}{0.999214in}}%
\pgfpathlineto{\pgfqpoint{3.926247in}{0.967208in}}%
\pgfusepath{stroke}%
\end{pgfscope}%
\begin{pgfscope}%
\pgfsetroundcap%
\pgfsetroundjoin%
\pgfsetlinewidth{1.003750pt}%
\definecolor{currentstroke}{rgb}{0.000000,0.000000,0.000000}%
\pgfsetstrokecolor{currentstroke}%
\pgfsetstrokeopacity{0.300000}%
\pgfsetdash{}{0pt}%
\pgfpathmoveto{\pgfqpoint{4.632166in}{1.241003in}}%
\pgfpathlineto{\pgfqpoint{4.668310in}{1.210046in}}%
\pgfusepath{stroke}%
\end{pgfscope}%
\begin{pgfscope}%
\pgfsetroundcap%
\pgfsetroundjoin%
\pgfsetlinewidth{1.003750pt}%
\definecolor{currentstroke}{rgb}{0.000000,0.000000,0.000000}%
\pgfsetstrokecolor{currentstroke}%
\pgfsetstrokeopacity{0.300000}%
\pgfsetdash{}{0pt}%
\pgfpathmoveto{\pgfqpoint{3.186370in}{0.769272in}}%
\pgfpathlineto{\pgfqpoint{3.220489in}{0.736250in}}%
\pgfusepath{stroke}%
\end{pgfscope}%
\begin{pgfscope}%
\pgfsetroundcap%
\pgfsetroundjoin%
\pgfsetlinewidth{1.003750pt}%
\definecolor{currentstroke}{rgb}{0.000000,0.000000,0.000000}%
\pgfsetstrokecolor{currentstroke}%
\pgfsetstrokeopacity{0.300000}%
\pgfsetdash{}{0pt}%
\pgfpathmoveto{\pgfqpoint{3.950699in}{1.018655in}}%
\pgfpathlineto{\pgfqpoint{3.985914in}{0.986734in}}%
\pgfusepath{stroke}%
\end{pgfscope}%
\begin{pgfscope}%
\pgfsetroundcap%
\pgfsetroundjoin%
\pgfsetlinewidth{1.003750pt}%
\definecolor{currentstroke}{rgb}{0.000000,0.000000,0.000000}%
\pgfsetstrokecolor{currentstroke}%
\pgfsetstrokeopacity{0.300000}%
\pgfsetdash{}{0pt}%
\pgfpathmoveto{\pgfqpoint{4.689798in}{1.259807in}}%
\pgfpathlineto{\pgfqpoint{4.726019in}{1.228931in}}%
\pgfusepath{stroke}%
\end{pgfscope}%
\begin{pgfscope}%
\pgfsetroundcap%
\pgfsetroundjoin%
\pgfsetlinewidth{1.003750pt}%
\definecolor{currentstroke}{rgb}{0.000000,0.000000,0.000000}%
\pgfsetstrokecolor{currentstroke}%
\pgfsetstrokeopacity{0.300000}%
\pgfsetdash{}{0pt}%
\pgfpathmoveto{\pgfqpoint{3.238354in}{0.786233in}}%
\pgfpathlineto{\pgfqpoint{3.272549in}{0.753287in}}%
\pgfusepath{stroke}%
\end{pgfscope}%
\begin{pgfscope}%
\pgfsetroundcap%
\pgfsetroundjoin%
\pgfsetlinewidth{1.003750pt}%
\definecolor{currentstroke}{rgb}{0.000000,0.000000,0.000000}%
\pgfsetstrokecolor{currentstroke}%
\pgfsetstrokeopacity{0.300000}%
\pgfsetdash{}{0pt}%
\pgfpathmoveto{\pgfqpoint{4.000953in}{1.035052in}}%
\pgfpathlineto{\pgfqpoint{4.036238in}{1.003202in}}%
\pgfusepath{stroke}%
\end{pgfscope}%
\begin{pgfscope}%
\pgfsetroundcap%
\pgfsetroundjoin%
\pgfsetlinewidth{1.003750pt}%
\definecolor{currentstroke}{rgb}{0.000000,0.000000,0.000000}%
\pgfsetstrokecolor{currentstroke}%
\pgfsetstrokeopacity{0.300000}%
\pgfsetdash{}{0pt}%
\pgfpathmoveto{\pgfqpoint{4.738408in}{1.275667in}}%
\pgfpathlineto{\pgfqpoint{4.774693in}{1.244859in}}%
\pgfusepath{stroke}%
\end{pgfscope}%
\begin{pgfscope}%
\pgfsetroundcap%
\pgfsetroundjoin%
\pgfsetlinewidth{1.003750pt}%
\definecolor{currentstroke}{rgb}{0.000000,0.000000,0.000000}%
\pgfsetstrokecolor{currentstroke}%
\pgfsetstrokeopacity{0.300000}%
\pgfsetdash{}{0pt}%
\pgfpathmoveto{\pgfqpoint{3.283288in}{0.800894in}}%
\pgfpathlineto{\pgfqpoint{3.317549in}{0.768013in}}%
\pgfusepath{stroke}%
\end{pgfscope}%
\begin{pgfscope}%
\pgfsetroundcap%
\pgfsetroundjoin%
\pgfsetlinewidth{1.003750pt}%
\definecolor{currentstroke}{rgb}{0.000000,0.000000,0.000000}%
\pgfsetstrokecolor{currentstroke}%
\pgfsetstrokeopacity{0.300000}%
\pgfsetdash{}{0pt}%
\pgfpathmoveto{\pgfqpoint{4.044394in}{1.049226in}}%
\pgfpathlineto{\pgfqpoint{4.079740in}{1.017438in}}%
\pgfusepath{stroke}%
\end{pgfscope}%
\begin{pgfscope}%
\pgfsetroundcap%
\pgfsetroundjoin%
\pgfsetlinewidth{1.003750pt}%
\definecolor{currentstroke}{rgb}{0.000000,0.000000,0.000000}%
\pgfsetstrokecolor{currentstroke}%
\pgfsetstrokeopacity{0.300000}%
\pgfsetdash{}{0pt}%
\pgfpathmoveto{\pgfqpoint{4.780429in}{1.289378in}}%
\pgfpathlineto{\pgfqpoint{4.816770in}{1.258629in}}%
\pgfusepath{stroke}%
\end{pgfscope}%
\begin{pgfscope}%
\pgfsetroundcap%
\pgfsetroundjoin%
\pgfsetlinewidth{1.003750pt}%
\definecolor{currentstroke}{rgb}{0.000000,0.000000,0.000000}%
\pgfsetstrokecolor{currentstroke}%
\pgfsetstrokeopacity{0.300000}%
\pgfsetdash{}{0pt}%
\pgfpathmoveto{\pgfqpoint{3.322849in}{0.813802in}}%
\pgfpathlineto{\pgfqpoint{3.357168in}{0.780978in}}%
\pgfusepath{stroke}%
\end{pgfscope}%
\begin{pgfscope}%
\pgfsetroundcap%
\pgfsetroundjoin%
\pgfsetlinewidth{1.003750pt}%
\definecolor{currentstroke}{rgb}{0.000000,0.000000,0.000000}%
\pgfsetstrokecolor{currentstroke}%
\pgfsetstrokeopacity{0.300000}%
\pgfsetdash{}{0pt}%
\pgfpathmoveto{\pgfqpoint{4.082642in}{1.061705in}}%
\pgfpathlineto{\pgfqpoint{4.118041in}{1.029972in}}%
\pgfusepath{stroke}%
\end{pgfscope}%
\begin{pgfscope}%
\pgfsetroundcap%
\pgfsetroundjoin%
\pgfsetlinewidth{1.003750pt}%
\definecolor{currentstroke}{rgb}{0.000000,0.000000,0.000000}%
\pgfsetstrokecolor{currentstroke}%
\pgfsetstrokeopacity{0.300000}%
\pgfsetdash{}{0pt}%
\pgfpathmoveto{\pgfqpoint{4.817428in}{1.301449in}}%
\pgfpathlineto{\pgfqpoint{4.853817in}{1.270753in}}%
\pgfusepath{stroke}%
\end{pgfscope}%
\begin{pgfscope}%
\pgfsetroundcap%
\pgfsetroundjoin%
\pgfsetlinewidth{1.254687pt}%
\definecolor{currentstroke}{rgb}{1.000000,1.000000,1.000000}%
\pgfsetstrokecolor{currentstroke}%
\pgfsetdash{}{0pt}%
\pgfpathmoveto{\pgfqpoint{1.273588in}{1.835516in}}%
\pgfpathlineto{\pgfqpoint{2.534381in}{0.541633in}}%
\pgfusepath{stroke}%
\end{pgfscope}%
\begin{pgfscope}%
\definecolor{textcolor}{rgb}{0.150000,0.150000,0.150000}%
\pgfsetstrokecolor{textcolor}%
\pgfsetfillcolor{textcolor}%
\pgftext[x=0.945471in, y=1.339199in, left, base,rotate=314.257888]{\color{textcolor}\sffamily\fontsize{12.000000}{14.400000}\selectfont Angle of Attack [deg.]}%
\end{pgfscope}%
\begin{pgfscope}%
\pgfsetbuttcap%
\pgfsetroundjoin%
\pgfsetlinewidth{1.003750pt}%
\definecolor{currentstroke}{rgb}{1.000000,1.000000,1.000000}%
\pgfsetstrokecolor{currentstroke}%
\pgfsetdash{}{0pt}%
\pgfpathmoveto{\pgfqpoint{4.961933in}{3.129286in}}%
\pgfpathlineto{\pgfqpoint{4.877339in}{1.342176in}}%
\pgfpathlineto{\pgfqpoint{2.506852in}{0.569885in}}%
\pgfusepath{stroke}%
\end{pgfscope}%
\begin{pgfscope}%
\pgfsetbuttcap%
\pgfsetroundjoin%
\pgfsetlinewidth{1.003750pt}%
\definecolor{currentstroke}{rgb}{1.000000,1.000000,1.000000}%
\pgfsetstrokecolor{currentstroke}%
\pgfsetdash{}{0pt}%
\pgfpathmoveto{\pgfqpoint{4.661589in}{3.348464in}}%
\pgfpathlineto{\pgfqpoint{4.591528in}{1.581748in}}%
\pgfpathlineto{\pgfqpoint{2.247722in}{0.835815in}}%
\pgfusepath{stroke}%
\end{pgfscope}%
\begin{pgfscope}%
\pgfsetbuttcap%
\pgfsetroundjoin%
\pgfsetlinewidth{1.003750pt}%
\definecolor{currentstroke}{rgb}{1.000000,1.000000,1.000000}%
\pgfsetstrokecolor{currentstroke}%
\pgfsetdash{}{0pt}%
\pgfpathmoveto{\pgfqpoint{4.371595in}{3.560088in}}%
\pgfpathlineto{\pgfqpoint{4.315143in}{1.813419in}}%
\pgfpathlineto{\pgfqpoint{1.997586in}{1.092517in}}%
\pgfusepath{stroke}%
\end{pgfscope}%
\begin{pgfscope}%
\pgfsetbuttcap%
\pgfsetroundjoin%
\pgfsetlinewidth{1.003750pt}%
\definecolor{currentstroke}{rgb}{1.000000,1.000000,1.000000}%
\pgfsetstrokecolor{currentstroke}%
\pgfsetdash{}{0pt}%
\pgfpathmoveto{\pgfqpoint{4.091426in}{3.764543in}}%
\pgfpathlineto{\pgfqpoint{4.047727in}{2.037573in}}%
\pgfpathlineto{\pgfqpoint{1.755981in}{1.340462in}}%
\pgfusepath{stroke}%
\end{pgfscope}%
\begin{pgfscope}%
\pgfsetbuttcap%
\pgfsetroundjoin%
\pgfsetlinewidth{1.003750pt}%
\definecolor{currentstroke}{rgb}{1.000000,1.000000,1.000000}%
\pgfsetstrokecolor{currentstroke}%
\pgfsetdash{}{0pt}%
\pgfpathmoveto{\pgfqpoint{3.820590in}{3.962186in}}%
\pgfpathlineto{\pgfqpoint{3.788849in}{2.254570in}}%
\pgfpathlineto{\pgfqpoint{1.522480in}{1.580092in}}%
\pgfusepath{stroke}%
\end{pgfscope}%
\begin{pgfscope}%
\pgfsetbuttcap%
\pgfsetroundjoin%
\pgfsetlinewidth{1.003750pt}%
\definecolor{currentstroke}{rgb}{1.000000,1.000000,1.000000}%
\pgfsetstrokecolor{currentstroke}%
\pgfsetdash{}{0pt}%
\pgfpathmoveto{\pgfqpoint{3.558629in}{4.153353in}}%
\pgfpathlineto{\pgfqpoint{3.538106in}{2.464747in}}%
\pgfpathlineto{\pgfqpoint{1.296682in}{1.811817in}}%
\pgfusepath{stroke}%
\end{pgfscope}%
\begin{pgfscope}%
\pgfsetbuttcap%
\pgfsetroundjoin%
\pgfsetlinewidth{1.003750pt}%
\definecolor{currentstroke}{rgb}{1.000000,1.000000,1.000000}%
\pgfsetstrokecolor{currentstroke}%
\pgfsetdash{}{0pt}%
\pgfpathmoveto{\pgfqpoint{4.810433in}{3.239844in}}%
\pgfpathlineto{\pgfqpoint{4.733225in}{1.462975in}}%
\pgfpathlineto{\pgfqpoint{2.376133in}{0.704034in}}%
\pgfusepath{stroke}%
\end{pgfscope}%
\begin{pgfscope}%
\pgfsetbuttcap%
\pgfsetroundjoin%
\pgfsetlinewidth{1.003750pt}%
\definecolor{currentstroke}{rgb}{1.000000,1.000000,1.000000}%
\pgfsetstrokecolor{currentstroke}%
\pgfsetdash{}{0pt}%
\pgfpathmoveto{\pgfqpoint{4.515332in}{3.455195in}}%
\pgfpathlineto{\pgfqpoint{4.452187in}{1.698547in}}%
\pgfpathlineto{\pgfqpoint{2.121559in}{0.965289in}}%
\pgfusepath{stroke}%
\end{pgfscope}%
\begin{pgfscope}%
\pgfsetbuttcap%
\pgfsetroundjoin%
\pgfsetlinewidth{1.003750pt}%
\definecolor{currentstroke}{rgb}{1.000000,1.000000,1.000000}%
\pgfsetstrokecolor{currentstroke}%
\pgfsetdash{}{0pt}%
\pgfpathmoveto{\pgfqpoint{4.230314in}{3.663188in}}%
\pgfpathlineto{\pgfqpoint{4.180342in}{1.926413in}}%
\pgfpathlineto{\pgfqpoint{1.875745in}{1.217556in}}%
\pgfusepath{stroke}%
\end{pgfscope}%
\begin{pgfscope}%
\pgfsetbuttcap%
\pgfsetroundjoin%
\pgfsetlinewidth{1.003750pt}%
\definecolor{currentstroke}{rgb}{1.000000,1.000000,1.000000}%
\pgfsetstrokecolor{currentstroke}%
\pgfsetdash{}{0pt}%
\pgfpathmoveto{\pgfqpoint{3.954871in}{3.864194in}}%
\pgfpathlineto{\pgfqpoint{3.917246in}{2.146945in}}%
\pgfpathlineto{\pgfqpoint{1.638244in}{1.461290in}}%
\pgfusepath{stroke}%
\end{pgfscope}%
\begin{pgfscope}%
\pgfsetbuttcap%
\pgfsetroundjoin%
\pgfsetlinewidth{1.003750pt}%
\definecolor{currentstroke}{rgb}{1.000000,1.000000,1.000000}%
\pgfsetstrokecolor{currentstroke}%
\pgfsetdash{}{0pt}%
\pgfpathmoveto{\pgfqpoint{3.688528in}{4.058559in}}%
\pgfpathlineto{\pgfqpoint{3.662485in}{2.360491in}}%
\pgfpathlineto{\pgfqpoint{1.408642in}{1.696918in}}%
\pgfusepath{stroke}%
\end{pgfscope}%
\begin{pgfscope}%
\pgfpathrectangle{\pgfqpoint{0.887500in}{0.275000in}}{\pgfqpoint{4.225000in}{4.225000in}}%
\pgfusepath{clip}%
\pgfsetroundcap%
\pgfsetroundjoin%
\pgfsetlinewidth{1.003750pt}%
\definecolor{currentstroke}{rgb}{1.000000,1.000000,1.000000}%
\pgfsetstrokecolor{currentstroke}%
\pgfsetdash{}{0pt}%
\pgfpathmoveto{\pgfqpoint{3.057095in}{2.444595in}}%
\pgfusepath{stroke}%
\end{pgfscope}%
\begin{pgfscope}%
\pgfsetroundcap%
\pgfsetroundjoin%
\pgfsetlinewidth{1.254687pt}%
\definecolor{currentstroke}{rgb}{0.150000,0.150000,0.150000}%
\pgfsetstrokecolor{currentstroke}%
\pgfsetdash{}{0pt}%
\pgfpathmoveto{\pgfqpoint{2.526832in}{0.576394in}}%
\pgfpathlineto{\pgfqpoint{2.466839in}{0.556849in}}%
\pgfusepath{stroke}%
\end{pgfscope}%
\begin{pgfscope}%
\definecolor{textcolor}{rgb}{0.150000,0.150000,0.150000}%
\pgfsetstrokecolor{textcolor}%
\pgfsetfillcolor{textcolor}%
\pgftext[x=2.327268in,y=0.382670in,right,bottom]{\color{textcolor}\sffamily\fontsize{11.000000}{13.200000}\selectfont \ensuremath{-}4}%
\end{pgfscope}%
\begin{pgfscope}%
\pgfpathrectangle{\pgfqpoint{0.887500in}{0.275000in}}{\pgfqpoint{4.225000in}{4.225000in}}%
\pgfusepath{clip}%
\pgfsetroundcap%
\pgfsetroundjoin%
\pgfsetlinewidth{1.003750pt}%
\definecolor{currentstroke}{rgb}{1.000000,1.000000,1.000000}%
\pgfsetstrokecolor{currentstroke}%
\pgfsetdash{}{0pt}%
\pgfpathmoveto{\pgfqpoint{3.057095in}{2.444595in}}%
\pgfusepath{stroke}%
\end{pgfscope}%
\begin{pgfscope}%
\pgfsetroundcap%
\pgfsetroundjoin%
\pgfsetlinewidth{1.254687pt}%
\definecolor{currentstroke}{rgb}{0.150000,0.150000,0.150000}%
\pgfsetstrokecolor{currentstroke}%
\pgfsetdash{}{0pt}%
\pgfpathmoveto{\pgfqpoint{2.267460in}{0.842097in}}%
\pgfpathlineto{\pgfqpoint{2.208197in}{0.823236in}}%
\pgfusepath{stroke}%
\end{pgfscope}%
\begin{pgfscope}%
\definecolor{textcolor}{rgb}{0.150000,0.150000,0.150000}%
\pgfsetstrokecolor{textcolor}%
\pgfsetfillcolor{textcolor}%
\pgftext[x=2.071177in,y=0.652045in,right,bottom]{\color{textcolor}\sffamily\fontsize{11.000000}{13.200000}\selectfont 0}%
\end{pgfscope}%
\begin{pgfscope}%
\pgfpathrectangle{\pgfqpoint{0.887500in}{0.275000in}}{\pgfqpoint{4.225000in}{4.225000in}}%
\pgfusepath{clip}%
\pgfsetroundcap%
\pgfsetroundjoin%
\pgfsetlinewidth{1.003750pt}%
\definecolor{currentstroke}{rgb}{1.000000,1.000000,1.000000}%
\pgfsetstrokecolor{currentstroke}%
\pgfsetdash{}{0pt}%
\pgfpathmoveto{\pgfqpoint{3.057095in}{2.444595in}}%
\pgfusepath{stroke}%
\end{pgfscope}%
\begin{pgfscope}%
\pgfsetroundcap%
\pgfsetroundjoin%
\pgfsetlinewidth{1.254687pt}%
\definecolor{currentstroke}{rgb}{0.150000,0.150000,0.150000}%
\pgfsetstrokecolor{currentstroke}%
\pgfsetdash{}{0pt}%
\pgfpathmoveto{\pgfqpoint{2.017085in}{1.098582in}}%
\pgfpathlineto{\pgfqpoint{1.958537in}{1.080370in}}%
\pgfusepath{stroke}%
\end{pgfscope}%
\begin{pgfscope}%
\definecolor{textcolor}{rgb}{0.150000,0.150000,0.150000}%
\pgfsetstrokecolor{textcolor}%
\pgfsetfillcolor{textcolor}%
\pgftext[x=1.823978in,y=0.912067in,right,bottom]{\color{textcolor}\sffamily\fontsize{11.000000}{13.200000}\selectfont 4}%
\end{pgfscope}%
\begin{pgfscope}%
\pgfpathrectangle{\pgfqpoint{0.887500in}{0.275000in}}{\pgfqpoint{4.225000in}{4.225000in}}%
\pgfusepath{clip}%
\pgfsetroundcap%
\pgfsetroundjoin%
\pgfsetlinewidth{1.003750pt}%
\definecolor{currentstroke}{rgb}{1.000000,1.000000,1.000000}%
\pgfsetstrokecolor{currentstroke}%
\pgfsetdash{}{0pt}%
\pgfpathmoveto{\pgfqpoint{3.057095in}{2.444595in}}%
\pgfusepath{stroke}%
\end{pgfscope}%
\begin{pgfscope}%
\pgfsetroundcap%
\pgfsetroundjoin%
\pgfsetlinewidth{1.254687pt}%
\definecolor{currentstroke}{rgb}{0.150000,0.150000,0.150000}%
\pgfsetstrokecolor{currentstroke}%
\pgfsetdash{}{0pt}%
\pgfpathmoveto{\pgfqpoint{1.775248in}{1.346323in}}%
\pgfpathlineto{\pgfqpoint{1.717401in}{1.328727in}}%
\pgfusepath{stroke}%
\end{pgfscope}%
\begin{pgfscope}%
\definecolor{textcolor}{rgb}{0.150000,0.150000,0.150000}%
\pgfsetstrokecolor{textcolor}%
\pgfsetfillcolor{textcolor}%
\pgftext[x=1.585216in,y=1.163216in,right,bottom]{\color{textcolor}\sffamily\fontsize{11.000000}{13.200000}\selectfont 8}%
\end{pgfscope}%
\begin{pgfscope}%
\pgfpathrectangle{\pgfqpoint{0.887500in}{0.275000in}}{\pgfqpoint{4.225000in}{4.225000in}}%
\pgfusepath{clip}%
\pgfsetroundcap%
\pgfsetroundjoin%
\pgfsetlinewidth{1.003750pt}%
\definecolor{currentstroke}{rgb}{1.000000,1.000000,1.000000}%
\pgfsetstrokecolor{currentstroke}%
\pgfsetdash{}{0pt}%
\pgfpathmoveto{\pgfqpoint{3.057095in}{2.444595in}}%
\pgfusepath{stroke}%
\end{pgfscope}%
\begin{pgfscope}%
\pgfsetroundcap%
\pgfsetroundjoin%
\pgfsetlinewidth{1.254687pt}%
\definecolor{currentstroke}{rgb}{0.150000,0.150000,0.150000}%
\pgfsetstrokecolor{currentstroke}%
\pgfsetdash{}{0pt}%
\pgfpathmoveto{\pgfqpoint{1.541518in}{1.585757in}}%
\pgfpathlineto{\pgfqpoint{1.484359in}{1.568747in}}%
\pgfusepath{stroke}%
\end{pgfscope}%
\begin{pgfscope}%
\definecolor{textcolor}{rgb}{0.150000,0.150000,0.150000}%
\pgfsetstrokecolor{textcolor}%
\pgfsetfillcolor{textcolor}%
\pgftext[x=1.354465in,y=1.405937in,right,bottom]{\color{textcolor}\sffamily\fontsize{11.000000}{13.200000}\selectfont 12}%
\end{pgfscope}%
\begin{pgfscope}%
\pgfpathrectangle{\pgfqpoint{0.887500in}{0.275000in}}{\pgfqpoint{4.225000in}{4.225000in}}%
\pgfusepath{clip}%
\pgfsetroundcap%
\pgfsetroundjoin%
\pgfsetlinewidth{1.003750pt}%
\definecolor{currentstroke}{rgb}{1.000000,1.000000,1.000000}%
\pgfsetstrokecolor{currentstroke}%
\pgfsetdash{}{0pt}%
\pgfpathmoveto{\pgfqpoint{3.057095in}{2.444595in}}%
\pgfusepath{stroke}%
\end{pgfscope}%
\begin{pgfscope}%
\pgfsetroundcap%
\pgfsetroundjoin%
\pgfsetlinewidth{1.254687pt}%
\definecolor{currentstroke}{rgb}{0.150000,0.150000,0.150000}%
\pgfsetstrokecolor{currentstroke}%
\pgfsetdash{}{0pt}%
\pgfpathmoveto{\pgfqpoint{1.315495in}{1.817297in}}%
\pgfpathlineto{\pgfqpoint{1.259010in}{1.800843in}}%
\pgfusepath{stroke}%
\end{pgfscope}%
\begin{pgfscope}%
\definecolor{textcolor}{rgb}{0.150000,0.150000,0.150000}%
\pgfsetstrokecolor{textcolor}%
\pgfsetfillcolor{textcolor}%
\pgftext[x=1.131330in,y=1.640647in,right,bottom]{\color{textcolor}\sffamily\fontsize{11.000000}{13.200000}\selectfont 16}%
\end{pgfscope}%
\begin{pgfscope}%
\pgfsetroundcap%
\pgfsetroundjoin%
\pgfsetlinewidth{1.003750pt}%
\definecolor{currentstroke}{rgb}{0.000000,0.000000,0.000000}%
\pgfsetstrokecolor{currentstroke}%
\pgfsetstrokeopacity{0.300000}%
\pgfsetdash{}{0pt}%
\pgfpathmoveto{\pgfqpoint{2.395991in}{0.710428in}}%
\pgfpathlineto{\pgfqpoint{2.336365in}{0.691230in}}%
\pgfusepath{stroke}%
\end{pgfscope}%
\begin{pgfscope}%
\pgfsetroundcap%
\pgfsetroundjoin%
\pgfsetlinewidth{1.003750pt}%
\definecolor{currentstroke}{rgb}{0.000000,0.000000,0.000000}%
\pgfsetstrokecolor{currentstroke}%
\pgfsetstrokeopacity{0.300000}%
\pgfsetdash{}{0pt}%
\pgfpathmoveto{\pgfqpoint{2.141178in}{0.971461in}}%
\pgfpathlineto{\pgfqpoint{2.082274in}{0.952929in}}%
\pgfusepath{stroke}%
\end{pgfscope}%
\begin{pgfscope}%
\pgfsetroundcap%
\pgfsetroundjoin%
\pgfsetlinewidth{1.003750pt}%
\definecolor{currentstroke}{rgb}{0.000000,0.000000,0.000000}%
\pgfsetstrokecolor{currentstroke}%
\pgfsetstrokeopacity{0.300000}%
\pgfsetdash{}{0pt}%
\pgfpathmoveto{\pgfqpoint{1.895127in}{1.223517in}}%
\pgfpathlineto{\pgfqpoint{1.836931in}{1.205617in}}%
\pgfusepath{stroke}%
\end{pgfscope}%
\begin{pgfscope}%
\pgfsetroundcap%
\pgfsetroundjoin%
\pgfsetlinewidth{1.003750pt}%
\definecolor{currentstroke}{rgb}{0.000000,0.000000,0.000000}%
\pgfsetstrokecolor{currentstroke}%
\pgfsetstrokeopacity{0.300000}%
\pgfsetdash{}{0pt}%
\pgfpathmoveto{\pgfqpoint{1.657395in}{1.467052in}}%
\pgfpathlineto{\pgfqpoint{1.599894in}{1.449752in}}%
\pgfusepath{stroke}%
\end{pgfscope}%
\begin{pgfscope}%
\pgfsetroundcap%
\pgfsetroundjoin%
\pgfsetlinewidth{1.003750pt}%
\definecolor{currentstroke}{rgb}{0.000000,0.000000,0.000000}%
\pgfsetstrokecolor{currentstroke}%
\pgfsetstrokeopacity{0.300000}%
\pgfsetdash{}{0pt}%
\pgfpathmoveto{\pgfqpoint{1.427567in}{1.702489in}}%
\pgfpathlineto{\pgfqpoint{1.370747in}{1.685761in}}%
\pgfusepath{stroke}%
\end{pgfscope}%
\begin{pgfscope}%
\pgfsetroundcap%
\pgfsetroundjoin%
\pgfsetlinewidth{1.254687pt}%
\definecolor{currentstroke}{rgb}{1.000000,1.000000,1.000000}%
\pgfsetstrokecolor{currentstroke}%
\pgfsetdash{}{0pt}%
\pgfpathmoveto{\pgfqpoint{1.273588in}{1.835516in}}%
\pgfpathlineto{\pgfqpoint{1.193692in}{3.580256in}}%
\pgfusepath{stroke}%
\end{pgfscope}%
\begin{pgfscope}%
\definecolor{textcolor}{rgb}{0.150000,0.150000,0.150000}%
\pgfsetstrokecolor{textcolor}%
\pgfsetfillcolor{textcolor}%
\pgftext[x=0.609474in, y=3.246352in, left, base,rotate=272.621908]{\color{textcolor}\sffamily\fontsize{12.000000}{14.400000}\selectfont Lift Coeff. \(\displaystyle C_L\)}%
\end{pgfscope}%
\begin{pgfscope}%
\pgfsetbuttcap%
\pgfsetroundjoin%
\pgfsetlinewidth{1.003750pt}%
\definecolor{currentstroke}{rgb}{1.000000,1.000000,1.000000}%
\pgfsetstrokecolor{currentstroke}%
\pgfsetdash{}{0pt}%
\pgfpathmoveto{\pgfqpoint{1.266388in}{1.992764in}}%
\pgfpathlineto{\pgfqpoint{3.514191in}{2.638566in}}%
\pgfpathlineto{\pgfqpoint{4.915426in}{1.477763in}}%
\pgfusepath{stroke}%
\end{pgfscope}%
\begin{pgfscope}%
\pgfsetbuttcap%
\pgfsetroundjoin%
\pgfsetlinewidth{1.003750pt}%
\definecolor{currentstroke}{rgb}{1.000000,1.000000,1.000000}%
\pgfsetstrokecolor{currentstroke}%
\pgfsetdash{}{0pt}%
\pgfpathmoveto{\pgfqpoint{1.258846in}{2.157445in}}%
\pgfpathlineto{\pgfqpoint{3.516024in}{2.798008in}}%
\pgfpathlineto{\pgfqpoint{4.923549in}{1.646439in}}%
\pgfusepath{stroke}%
\end{pgfscope}%
\begin{pgfscope}%
\pgfsetbuttcap%
\pgfsetroundjoin%
\pgfsetlinewidth{1.003750pt}%
\definecolor{currentstroke}{rgb}{1.000000,1.000000,1.000000}%
\pgfsetstrokecolor{currentstroke}%
\pgfsetdash{}{0pt}%
\pgfpathmoveto{\pgfqpoint{1.251241in}{2.323519in}}%
\pgfpathlineto{\pgfqpoint{3.517873in}{2.958734in}}%
\pgfpathlineto{\pgfqpoint{4.931743in}{1.816596in}}%
\pgfusepath{stroke}%
\end{pgfscope}%
\begin{pgfscope}%
\pgfsetbuttcap%
\pgfsetroundjoin%
\pgfsetlinewidth{1.003750pt}%
\definecolor{currentstroke}{rgb}{1.000000,1.000000,1.000000}%
\pgfsetstrokecolor{currentstroke}%
\pgfsetdash{}{0pt}%
\pgfpathmoveto{\pgfqpoint{1.243572in}{2.491004in}}%
\pgfpathlineto{\pgfqpoint{3.519736in}{3.120759in}}%
\pgfpathlineto{\pgfqpoint{4.940010in}{1.988254in}}%
\pgfusepath{stroke}%
\end{pgfscope}%
\begin{pgfscope}%
\pgfsetbuttcap%
\pgfsetroundjoin%
\pgfsetlinewidth{1.003750pt}%
\definecolor{currentstroke}{rgb}{1.000000,1.000000,1.000000}%
\pgfsetstrokecolor{currentstroke}%
\pgfsetdash{}{0pt}%
\pgfpathmoveto{\pgfqpoint{1.235837in}{2.659917in}}%
\pgfpathlineto{\pgfqpoint{3.521614in}{3.284100in}}%
\pgfpathlineto{\pgfqpoint{4.948349in}{2.161433in}}%
\pgfusepath{stroke}%
\end{pgfscope}%
\begin{pgfscope}%
\pgfsetbuttcap%
\pgfsetroundjoin%
\pgfsetlinewidth{1.003750pt}%
\definecolor{currentstroke}{rgb}{1.000000,1.000000,1.000000}%
\pgfsetstrokecolor{currentstroke}%
\pgfsetdash{}{0pt}%
\pgfpathmoveto{\pgfqpoint{1.228035in}{2.830278in}}%
\pgfpathlineto{\pgfqpoint{3.523508in}{3.448773in}}%
\pgfpathlineto{\pgfqpoint{4.956763in}{2.336152in}}%
\pgfusepath{stroke}%
\end{pgfscope}%
\begin{pgfscope}%
\pgfsetbuttcap%
\pgfsetroundjoin%
\pgfsetlinewidth{1.003750pt}%
\definecolor{currentstroke}{rgb}{1.000000,1.000000,1.000000}%
\pgfsetstrokecolor{currentstroke}%
\pgfsetdash{}{0pt}%
\pgfpathmoveto{\pgfqpoint{1.220167in}{3.002104in}}%
\pgfpathlineto{\pgfqpoint{3.525417in}{3.614794in}}%
\pgfpathlineto{\pgfqpoint{4.965252in}{2.512433in}}%
\pgfusepath{stroke}%
\end{pgfscope}%
\begin{pgfscope}%
\pgfsetbuttcap%
\pgfsetroundjoin%
\pgfsetlinewidth{1.003750pt}%
\definecolor{currentstroke}{rgb}{1.000000,1.000000,1.000000}%
\pgfsetstrokecolor{currentstroke}%
\pgfsetdash{}{0pt}%
\pgfpathmoveto{\pgfqpoint{1.212231in}{3.175415in}}%
\pgfpathlineto{\pgfqpoint{3.527342in}{3.782180in}}%
\pgfpathlineto{\pgfqpoint{4.973818in}{2.690297in}}%
\pgfusepath{stroke}%
\end{pgfscope}%
\begin{pgfscope}%
\pgfsetbuttcap%
\pgfsetroundjoin%
\pgfsetlinewidth{1.003750pt}%
\definecolor{currentstroke}{rgb}{1.000000,1.000000,1.000000}%
\pgfsetstrokecolor{currentstroke}%
\pgfsetdash{}{0pt}%
\pgfpathmoveto{\pgfqpoint{1.204225in}{3.350230in}}%
\pgfpathlineto{\pgfqpoint{3.529282in}{3.950947in}}%
\pgfpathlineto{\pgfqpoint{4.982460in}{2.869764in}}%
\pgfusepath{stroke}%
\end{pgfscope}%
\begin{pgfscope}%
\pgfsetbuttcap%
\pgfsetroundjoin%
\pgfsetlinewidth{1.003750pt}%
\definecolor{currentstroke}{rgb}{1.000000,1.000000,1.000000}%
\pgfsetstrokecolor{currentstroke}%
\pgfsetdash{}{0pt}%
\pgfpathmoveto{\pgfqpoint{1.196150in}{3.526569in}}%
\pgfpathlineto{\pgfqpoint{3.531239in}{4.121113in}}%
\pgfpathlineto{\pgfqpoint{4.991181in}{3.050857in}}%
\pgfusepath{stroke}%
\end{pgfscope}%
\begin{pgfscope}%
\pgfpathrectangle{\pgfqpoint{0.887500in}{0.275000in}}{\pgfqpoint{4.225000in}{4.225000in}}%
\pgfusepath{clip}%
\pgfsetroundcap%
\pgfsetroundjoin%
\pgfsetlinewidth{1.003750pt}%
\definecolor{currentstroke}{rgb}{1.000000,1.000000,1.000000}%
\pgfsetstrokecolor{currentstroke}%
\pgfsetdash{}{0pt}%
\pgfpathmoveto{\pgfqpoint{3.057095in}{2.444595in}}%
\pgfusepath{stroke}%
\end{pgfscope}%
\begin{pgfscope}%
\pgfsetroundcap%
\pgfsetroundjoin%
\pgfsetlinewidth{1.254687pt}%
\definecolor{currentstroke}{rgb}{0.150000,0.150000,0.150000}%
\pgfsetstrokecolor{currentstroke}%
\pgfsetdash{}{0pt}%
\pgfpathmoveto{\pgfqpoint{1.285256in}{1.998185in}}%
\pgfpathlineto{\pgfqpoint{1.228605in}{1.981909in}}%
\pgfusepath{stroke}%
\end{pgfscope}%
\begin{pgfscope}%
\definecolor{textcolor}{rgb}{0.150000,0.150000,0.150000}%
\pgfsetstrokecolor{textcolor}%
\pgfsetfillcolor{textcolor}%
\pgftext[x=1.011837in,y=2.028690in,,top]{\color{textcolor}\sffamily\fontsize{11.000000}{13.200000}\selectfont \ensuremath{-}0.2}%
\end{pgfscope}%
\begin{pgfscope}%
\pgfpathrectangle{\pgfqpoint{0.887500in}{0.275000in}}{\pgfqpoint{4.225000in}{4.225000in}}%
\pgfusepath{clip}%
\pgfsetroundcap%
\pgfsetroundjoin%
\pgfsetlinewidth{1.003750pt}%
\definecolor{currentstroke}{rgb}{1.000000,1.000000,1.000000}%
\pgfsetstrokecolor{currentstroke}%
\pgfsetdash{}{0pt}%
\pgfpathmoveto{\pgfqpoint{3.057095in}{2.444595in}}%
\pgfusepath{stroke}%
\end{pgfscope}%
\begin{pgfscope}%
\pgfsetroundcap%
\pgfsetroundjoin%
\pgfsetlinewidth{1.254687pt}%
\definecolor{currentstroke}{rgb}{0.150000,0.150000,0.150000}%
\pgfsetstrokecolor{currentstroke}%
\pgfsetdash{}{0pt}%
\pgfpathmoveto{\pgfqpoint{1.277798in}{2.162824in}}%
\pgfpathlineto{\pgfqpoint{1.220898in}{2.146676in}}%
\pgfusepath{stroke}%
\end{pgfscope}%
\begin{pgfscope}%
\definecolor{textcolor}{rgb}{0.150000,0.150000,0.150000}%
\pgfsetstrokecolor{textcolor}%
\pgfsetfillcolor{textcolor}%
\pgftext[x=1.003247in,y=2.193086in,,top]{\color{textcolor}\sffamily\fontsize{11.000000}{13.200000}\selectfont 0.0}%
\end{pgfscope}%
\begin{pgfscope}%
\pgfpathrectangle{\pgfqpoint{0.887500in}{0.275000in}}{\pgfqpoint{4.225000in}{4.225000in}}%
\pgfusepath{clip}%
\pgfsetroundcap%
\pgfsetroundjoin%
\pgfsetlinewidth{1.003750pt}%
\definecolor{currentstroke}{rgb}{1.000000,1.000000,1.000000}%
\pgfsetstrokecolor{currentstroke}%
\pgfsetdash{}{0pt}%
\pgfpathmoveto{\pgfqpoint{3.057095in}{2.444595in}}%
\pgfusepath{stroke}%
\end{pgfscope}%
\begin{pgfscope}%
\pgfsetroundcap%
\pgfsetroundjoin%
\pgfsetlinewidth{1.254687pt}%
\definecolor{currentstroke}{rgb}{0.150000,0.150000,0.150000}%
\pgfsetstrokecolor{currentstroke}%
\pgfsetdash{}{0pt}%
\pgfpathmoveto{\pgfqpoint{1.270276in}{2.328854in}}%
\pgfpathlineto{\pgfqpoint{1.213127in}{2.312838in}}%
\pgfusepath{stroke}%
\end{pgfscope}%
\begin{pgfscope}%
\definecolor{textcolor}{rgb}{0.150000,0.150000,0.150000}%
\pgfsetstrokecolor{textcolor}%
\pgfsetfillcolor{textcolor}%
\pgftext[x=0.994584in,y=2.358869in,,top]{\color{textcolor}\sffamily\fontsize{11.000000}{13.200000}\selectfont 0.2}%
\end{pgfscope}%
\begin{pgfscope}%
\pgfpathrectangle{\pgfqpoint{0.887500in}{0.275000in}}{\pgfqpoint{4.225000in}{4.225000in}}%
\pgfusepath{clip}%
\pgfsetroundcap%
\pgfsetroundjoin%
\pgfsetlinewidth{1.003750pt}%
\definecolor{currentstroke}{rgb}{1.000000,1.000000,1.000000}%
\pgfsetstrokecolor{currentstroke}%
\pgfsetdash{}{0pt}%
\pgfpathmoveto{\pgfqpoint{3.057095in}{2.444595in}}%
\pgfusepath{stroke}%
\end{pgfscope}%
\begin{pgfscope}%
\pgfsetroundcap%
\pgfsetroundjoin%
\pgfsetlinewidth{1.254687pt}%
\definecolor{currentstroke}{rgb}{0.150000,0.150000,0.150000}%
\pgfsetstrokecolor{currentstroke}%
\pgfsetdash{}{0pt}%
\pgfpathmoveto{\pgfqpoint{1.262690in}{2.496294in}}%
\pgfpathlineto{\pgfqpoint{1.205289in}{2.480412in}}%
\pgfusepath{stroke}%
\end{pgfscope}%
\begin{pgfscope}%
\definecolor{textcolor}{rgb}{0.150000,0.150000,0.150000}%
\pgfsetstrokecolor{textcolor}%
\pgfsetfillcolor{textcolor}%
\pgftext[x=0.985847in,y=2.526057in,,top]{\color{textcolor}\sffamily\fontsize{11.000000}{13.200000}\selectfont 0.4}%
\end{pgfscope}%
\begin{pgfscope}%
\pgfpathrectangle{\pgfqpoint{0.887500in}{0.275000in}}{\pgfqpoint{4.225000in}{4.225000in}}%
\pgfusepath{clip}%
\pgfsetroundcap%
\pgfsetroundjoin%
\pgfsetlinewidth{1.003750pt}%
\definecolor{currentstroke}{rgb}{1.000000,1.000000,1.000000}%
\pgfsetstrokecolor{currentstroke}%
\pgfsetdash{}{0pt}%
\pgfpathmoveto{\pgfqpoint{3.057095in}{2.444595in}}%
\pgfusepath{stroke}%
\end{pgfscope}%
\begin{pgfscope}%
\pgfsetroundcap%
\pgfsetroundjoin%
\pgfsetlinewidth{1.254687pt}%
\definecolor{currentstroke}{rgb}{0.150000,0.150000,0.150000}%
\pgfsetstrokecolor{currentstroke}%
\pgfsetdash{}{0pt}%
\pgfpathmoveto{\pgfqpoint{1.255040in}{2.665161in}}%
\pgfpathlineto{\pgfqpoint{1.197384in}{2.649417in}}%
\pgfusepath{stroke}%
\end{pgfscope}%
\begin{pgfscope}%
\definecolor{textcolor}{rgb}{0.150000,0.150000,0.150000}%
\pgfsetstrokecolor{textcolor}%
\pgfsetfillcolor{textcolor}%
\pgftext[x=0.977037in,y=2.694667in,,top]{\color{textcolor}\sffamily\fontsize{11.000000}{13.200000}\selectfont 0.6}%
\end{pgfscope}%
\begin{pgfscope}%
\pgfpathrectangle{\pgfqpoint{0.887500in}{0.275000in}}{\pgfqpoint{4.225000in}{4.225000in}}%
\pgfusepath{clip}%
\pgfsetroundcap%
\pgfsetroundjoin%
\pgfsetlinewidth{1.003750pt}%
\definecolor{currentstroke}{rgb}{1.000000,1.000000,1.000000}%
\pgfsetstrokecolor{currentstroke}%
\pgfsetdash{}{0pt}%
\pgfpathmoveto{\pgfqpoint{3.057095in}{2.444595in}}%
\pgfusepath{stroke}%
\end{pgfscope}%
\begin{pgfscope}%
\pgfsetroundcap%
\pgfsetroundjoin%
\pgfsetlinewidth{1.254687pt}%
\definecolor{currentstroke}{rgb}{0.150000,0.150000,0.150000}%
\pgfsetstrokecolor{currentstroke}%
\pgfsetdash{}{0pt}%
\pgfpathmoveto{\pgfqpoint{1.247324in}{2.835475in}}%
\pgfpathlineto{\pgfqpoint{1.189412in}{2.819871in}}%
\pgfusepath{stroke}%
\end{pgfscope}%
\begin{pgfscope}%
\definecolor{textcolor}{rgb}{0.150000,0.150000,0.150000}%
\pgfsetstrokecolor{textcolor}%
\pgfsetfillcolor{textcolor}%
\pgftext[x=0.968151in,y=2.864717in,,top]{\color{textcolor}\sffamily\fontsize{11.000000}{13.200000}\selectfont 0.8}%
\end{pgfscope}%
\begin{pgfscope}%
\pgfpathrectangle{\pgfqpoint{0.887500in}{0.275000in}}{\pgfqpoint{4.225000in}{4.225000in}}%
\pgfusepath{clip}%
\pgfsetroundcap%
\pgfsetroundjoin%
\pgfsetlinewidth{1.003750pt}%
\definecolor{currentstroke}{rgb}{1.000000,1.000000,1.000000}%
\pgfsetstrokecolor{currentstroke}%
\pgfsetdash{}{0pt}%
\pgfpathmoveto{\pgfqpoint{3.057095in}{2.444595in}}%
\pgfusepath{stroke}%
\end{pgfscope}%
\begin{pgfscope}%
\pgfsetroundcap%
\pgfsetroundjoin%
\pgfsetlinewidth{1.254687pt}%
\definecolor{currentstroke}{rgb}{0.150000,0.150000,0.150000}%
\pgfsetstrokecolor{currentstroke}%
\pgfsetdash{}{0pt}%
\pgfpathmoveto{\pgfqpoint{1.239542in}{3.007253in}}%
\pgfpathlineto{\pgfqpoint{1.181371in}{2.991793in}}%
\pgfusepath{stroke}%
\end{pgfscope}%
\begin{pgfscope}%
\definecolor{textcolor}{rgb}{0.150000,0.150000,0.150000}%
\pgfsetstrokecolor{textcolor}%
\pgfsetfillcolor{textcolor}%
\pgftext[x=0.959189in,y=3.036227in,,top]{\color{textcolor}\sffamily\fontsize{11.000000}{13.200000}\selectfont 1.0}%
\end{pgfscope}%
\begin{pgfscope}%
\pgfpathrectangle{\pgfqpoint{0.887500in}{0.275000in}}{\pgfqpoint{4.225000in}{4.225000in}}%
\pgfusepath{clip}%
\pgfsetroundcap%
\pgfsetroundjoin%
\pgfsetlinewidth{1.003750pt}%
\definecolor{currentstroke}{rgb}{1.000000,1.000000,1.000000}%
\pgfsetstrokecolor{currentstroke}%
\pgfsetdash{}{0pt}%
\pgfpathmoveto{\pgfqpoint{3.057095in}{2.444595in}}%
\pgfusepath{stroke}%
\end{pgfscope}%
\begin{pgfscope}%
\pgfsetroundcap%
\pgfsetroundjoin%
\pgfsetlinewidth{1.254687pt}%
\definecolor{currentstroke}{rgb}{0.150000,0.150000,0.150000}%
\pgfsetstrokecolor{currentstroke}%
\pgfsetdash{}{0pt}%
\pgfpathmoveto{\pgfqpoint{1.231692in}{3.180516in}}%
\pgfpathlineto{\pgfqpoint{1.173260in}{3.165201in}}%
\pgfusepath{stroke}%
\end{pgfscope}%
\begin{pgfscope}%
\definecolor{textcolor}{rgb}{0.150000,0.150000,0.150000}%
\pgfsetstrokecolor{textcolor}%
\pgfsetfillcolor{textcolor}%
\pgftext[x=0.950149in,y=3.209215in,,top]{\color{textcolor}\sffamily\fontsize{11.000000}{13.200000}\selectfont 1.2}%
\end{pgfscope}%
\begin{pgfscope}%
\pgfpathrectangle{\pgfqpoint{0.887500in}{0.275000in}}{\pgfqpoint{4.225000in}{4.225000in}}%
\pgfusepath{clip}%
\pgfsetroundcap%
\pgfsetroundjoin%
\pgfsetlinewidth{1.003750pt}%
\definecolor{currentstroke}{rgb}{1.000000,1.000000,1.000000}%
\pgfsetstrokecolor{currentstroke}%
\pgfsetdash{}{0pt}%
\pgfpathmoveto{\pgfqpoint{3.057095in}{2.444595in}}%
\pgfusepath{stroke}%
\end{pgfscope}%
\begin{pgfscope}%
\pgfsetroundcap%
\pgfsetroundjoin%
\pgfsetlinewidth{1.254687pt}%
\definecolor{currentstroke}{rgb}{0.150000,0.150000,0.150000}%
\pgfsetstrokecolor{currentstroke}%
\pgfsetdash{}{0pt}%
\pgfpathmoveto{\pgfqpoint{1.223774in}{3.355281in}}%
\pgfpathlineto{\pgfqpoint{1.165079in}{3.340116in}}%
\pgfusepath{stroke}%
\end{pgfscope}%
\begin{pgfscope}%
\definecolor{textcolor}{rgb}{0.150000,0.150000,0.150000}%
\pgfsetstrokecolor{textcolor}%
\pgfsetfillcolor{textcolor}%
\pgftext[x=0.941032in,y=3.383699in,,top]{\color{textcolor}\sffamily\fontsize{11.000000}{13.200000}\selectfont 1.4}%
\end{pgfscope}%
\begin{pgfscope}%
\pgfpathrectangle{\pgfqpoint{0.887500in}{0.275000in}}{\pgfqpoint{4.225000in}{4.225000in}}%
\pgfusepath{clip}%
\pgfsetroundcap%
\pgfsetroundjoin%
\pgfsetlinewidth{1.003750pt}%
\definecolor{currentstroke}{rgb}{1.000000,1.000000,1.000000}%
\pgfsetstrokecolor{currentstroke}%
\pgfsetdash{}{0pt}%
\pgfpathmoveto{\pgfqpoint{3.057095in}{2.444595in}}%
\pgfusepath{stroke}%
\end{pgfscope}%
\begin{pgfscope}%
\pgfsetroundcap%
\pgfsetroundjoin%
\pgfsetlinewidth{1.254687pt}%
\definecolor{currentstroke}{rgb}{0.150000,0.150000,0.150000}%
\pgfsetstrokecolor{currentstroke}%
\pgfsetdash{}{0pt}%
\pgfpathmoveto{\pgfqpoint{1.215788in}{3.531569in}}%
\pgfpathlineto{\pgfqpoint{1.156826in}{3.516556in}}%
\pgfusepath{stroke}%
\end{pgfscope}%
\begin{pgfscope}%
\definecolor{textcolor}{rgb}{0.150000,0.150000,0.150000}%
\pgfsetstrokecolor{textcolor}%
\pgfsetfillcolor{textcolor}%
\pgftext[x=0.931835in,y=3.559701in,,top]{\color{textcolor}\sffamily\fontsize{11.000000}{13.200000}\selectfont 1.6}%
\end{pgfscope}%
\begin{pgfscope}%
\pgfpathrectangle{\pgfqpoint{0.887500in}{0.275000in}}{\pgfqpoint{4.225000in}{4.225000in}}%
\pgfusepath{clip}%
\pgfsetbuttcap%
\pgfsetroundjoin%
\definecolor{currentfill}{rgb}{0.730889,0.871916,0.156029}%
\pgfsetfillcolor{currentfill}%
\pgfsetfillopacity{0.700000}%
\pgfsetlinewidth{0.501875pt}%
\definecolor{currentstroke}{rgb}{1.000000,1.000000,1.000000}%
\pgfsetstrokecolor{currentstroke}%
\pgfsetstrokeopacity{0.500000}%
\pgfsetdash{}{0pt}%
\pgfpathmoveto{\pgfqpoint{3.492699in}{3.844824in}}%
\pgfpathlineto{\pgfqpoint{3.503390in}{3.845218in}}%
\pgfpathlineto{\pgfqpoint{3.514072in}{3.845374in}}%
\pgfpathlineto{\pgfqpoint{3.524745in}{3.845288in}}%
\pgfpathlineto{\pgfqpoint{3.535409in}{3.844954in}}%
\pgfpathlineto{\pgfqpoint{3.529317in}{3.851359in}}%
\pgfpathlineto{\pgfqpoint{3.523245in}{3.859164in}}%
\pgfpathlineto{\pgfqpoint{3.517194in}{3.868459in}}%
\pgfpathlineto{\pgfqpoint{3.511163in}{3.879334in}}%
\pgfpathlineto{\pgfqpoint{3.500452in}{3.874791in}}%
\pgfpathlineto{\pgfqpoint{3.489737in}{3.870327in}}%
\pgfpathlineto{\pgfqpoint{3.479018in}{3.865950in}}%
\pgfpathlineto{\pgfqpoint{3.468297in}{3.861669in}}%
\pgfpathlineto{\pgfqpoint{3.474377in}{3.856178in}}%
\pgfpathlineto{\pgfqpoint{3.480471in}{3.851577in}}%
\pgfpathlineto{\pgfqpoint{3.486578in}{3.847810in}}%
\pgfpathclose%
\pgfusepath{stroke,fill}%
\end{pgfscope}%
\begin{pgfscope}%
\pgfpathrectangle{\pgfqpoint{0.887500in}{0.275000in}}{\pgfqpoint{4.225000in}{4.225000in}}%
\pgfusepath{clip}%
\pgfsetbuttcap%
\pgfsetroundjoin%
\definecolor{currentfill}{rgb}{0.730889,0.871916,0.156029}%
\pgfsetfillcolor{currentfill}%
\pgfsetfillopacity{0.700000}%
\pgfsetlinewidth{0.501875pt}%
\definecolor{currentstroke}{rgb}{1.000000,1.000000,1.000000}%
\pgfsetstrokecolor{currentstroke}%
\pgfsetstrokeopacity{0.500000}%
\pgfsetdash{}{0pt}%
\pgfpathmoveto{\pgfqpoint{3.439116in}{3.839446in}}%
\pgfpathlineto{\pgfqpoint{3.449848in}{3.840960in}}%
\pgfpathlineto{\pgfqpoint{3.460573in}{3.842260in}}%
\pgfpathlineto{\pgfqpoint{3.471290in}{3.843340in}}%
\pgfpathlineto{\pgfqpoint{3.481999in}{3.844196in}}%
\pgfpathlineto{\pgfqpoint{3.492699in}{3.844824in}}%
\pgfpathlineto{\pgfqpoint{3.486578in}{3.847810in}}%
\pgfpathlineto{\pgfqpoint{3.480471in}{3.851577in}}%
\pgfpathlineto{\pgfqpoint{3.474377in}{3.856178in}}%
\pgfpathlineto{\pgfqpoint{3.468297in}{3.861669in}}%
\pgfpathlineto{\pgfqpoint{3.457572in}{3.857492in}}%
\pgfpathlineto{\pgfqpoint{3.446844in}{3.853429in}}%
\pgfpathlineto{\pgfqpoint{3.436113in}{3.849488in}}%
\pgfpathlineto{\pgfqpoint{3.425378in}{3.845678in}}%
\pgfpathlineto{\pgfqpoint{3.414641in}{3.842008in}}%
\pgfpathlineto{\pgfqpoint{3.420747in}{3.840948in}}%
\pgfpathlineto{\pgfqpoint{3.426861in}{3.840184in}}%
\pgfpathlineto{\pgfqpoint{3.432984in}{3.839691in}}%
\pgfpathclose%
\pgfusepath{stroke,fill}%
\end{pgfscope}%
\begin{pgfscope}%
\pgfpathrectangle{\pgfqpoint{0.887500in}{0.275000in}}{\pgfqpoint{4.225000in}{4.225000in}}%
\pgfusepath{clip}%
\pgfsetbuttcap%
\pgfsetroundjoin%
\definecolor{currentfill}{rgb}{0.741388,0.873449,0.149561}%
\pgfsetfillcolor{currentfill}%
\pgfsetfillopacity{0.700000}%
\pgfsetlinewidth{0.501875pt}%
\definecolor{currentstroke}{rgb}{1.000000,1.000000,1.000000}%
\pgfsetstrokecolor{currentstroke}%
\pgfsetstrokeopacity{0.500000}%
\pgfsetdash{}{0pt}%
\pgfpathmoveto{\pgfqpoint{3.523493in}{3.839711in}}%
\pgfpathlineto{\pgfqpoint{3.534177in}{3.838141in}}%
\pgfpathlineto{\pgfqpoint{3.544850in}{3.836138in}}%
\pgfpathlineto{\pgfqpoint{3.555509in}{3.833699in}}%
\pgfpathlineto{\pgfqpoint{3.566155in}{3.830824in}}%
\pgfpathlineto{\pgfqpoint{3.559969in}{3.831561in}}%
\pgfpathlineto{\pgfqpoint{3.553801in}{3.833253in}}%
\pgfpathlineto{\pgfqpoint{3.547652in}{3.835990in}}%
\pgfpathlineto{\pgfqpoint{3.541520in}{3.839861in}}%
\pgfpathlineto{\pgfqpoint{3.535409in}{3.844954in}}%
\pgfpathlineto{\pgfqpoint{3.524745in}{3.845288in}}%
\pgfpathlineto{\pgfqpoint{3.514072in}{3.845374in}}%
\pgfpathlineto{\pgfqpoint{3.503390in}{3.845218in}}%
\pgfpathlineto{\pgfqpoint{3.492699in}{3.844824in}}%
\pgfpathlineto{\pgfqpoint{3.498832in}{3.842565in}}%
\pgfpathlineto{\pgfqpoint{3.504979in}{3.840977in}}%
\pgfpathlineto{\pgfqpoint{3.511138in}{3.840009in}}%
\pgfpathlineto{\pgfqpoint{3.517309in}{3.839605in}}%
\pgfpathclose%
\pgfusepath{stroke,fill}%
\end{pgfscope}%
\begin{pgfscope}%
\pgfpathrectangle{\pgfqpoint{0.887500in}{0.275000in}}{\pgfqpoint{4.225000in}{4.225000in}}%
\pgfusepath{clip}%
\pgfsetbuttcap%
\pgfsetroundjoin%
\definecolor{currentfill}{rgb}{0.741388,0.873449,0.149561}%
\pgfsetfillcolor{currentfill}%
\pgfsetfillopacity{0.700000}%
\pgfsetlinewidth{0.501875pt}%
\definecolor{currentstroke}{rgb}{1.000000,1.000000,1.000000}%
\pgfsetstrokecolor{currentstroke}%
\pgfsetstrokeopacity{0.500000}%
\pgfsetdash{}{0pt}%
\pgfpathmoveto{\pgfqpoint{3.385341in}{3.828820in}}%
\pgfpathlineto{\pgfqpoint{3.396110in}{3.831336in}}%
\pgfpathlineto{\pgfqpoint{3.406872in}{3.833662in}}%
\pgfpathlineto{\pgfqpoint{3.417627in}{3.835792in}}%
\pgfpathlineto{\pgfqpoint{3.428375in}{3.837721in}}%
\pgfpathlineto{\pgfqpoint{3.439116in}{3.839446in}}%
\pgfpathlineto{\pgfqpoint{3.432984in}{3.839691in}}%
\pgfpathlineto{\pgfqpoint{3.426861in}{3.840184in}}%
\pgfpathlineto{\pgfqpoint{3.420747in}{3.840948in}}%
\pgfpathlineto{\pgfqpoint{3.414641in}{3.842008in}}%
\pgfpathlineto{\pgfqpoint{3.403899in}{3.838485in}}%
\pgfpathlineto{\pgfqpoint{3.393155in}{3.835119in}}%
\pgfpathlineto{\pgfqpoint{3.382407in}{3.831919in}}%
\pgfpathlineto{\pgfqpoint{3.371655in}{3.828893in}}%
\pgfpathlineto{\pgfqpoint{3.360900in}{3.826050in}}%
\pgfpathlineto{\pgfqpoint{3.367001in}{3.826733in}}%
\pgfpathlineto{\pgfqpoint{3.373108in}{3.827427in}}%
\pgfpathlineto{\pgfqpoint{3.379221in}{3.828126in}}%
\pgfpathclose%
\pgfusepath{stroke,fill}%
\end{pgfscope}%
\begin{pgfscope}%
\pgfpathrectangle{\pgfqpoint{0.887500in}{0.275000in}}{\pgfqpoint{4.225000in}{4.225000in}}%
\pgfusepath{clip}%
\pgfsetbuttcap%
\pgfsetroundjoin%
\definecolor{currentfill}{rgb}{0.762373,0.876424,0.137064}%
\pgfsetfillcolor{currentfill}%
\pgfsetfillopacity{0.700000}%
\pgfsetlinewidth{0.501875pt}%
\definecolor{currentstroke}{rgb}{1.000000,1.000000,1.000000}%
\pgfsetstrokecolor{currentstroke}%
\pgfsetstrokeopacity{0.500000}%
\pgfsetdash{}{0pt}%
\pgfpathmoveto{\pgfqpoint{3.469893in}{3.841118in}}%
\pgfpathlineto{\pgfqpoint{3.480635in}{3.841690in}}%
\pgfpathlineto{\pgfqpoint{3.491367in}{3.841836in}}%
\pgfpathlineto{\pgfqpoint{3.502087in}{3.841557in}}%
\pgfpathlineto{\pgfqpoint{3.512796in}{3.840849in}}%
\pgfpathlineto{\pgfqpoint{3.523493in}{3.839711in}}%
\pgfpathlineto{\pgfqpoint{3.517309in}{3.839605in}}%
\pgfpathlineto{\pgfqpoint{3.511138in}{3.840009in}}%
\pgfpathlineto{\pgfqpoint{3.504979in}{3.840977in}}%
\pgfpathlineto{\pgfqpoint{3.498832in}{3.842565in}}%
\pgfpathlineto{\pgfqpoint{3.492699in}{3.844824in}}%
\pgfpathlineto{\pgfqpoint{3.481999in}{3.844196in}}%
\pgfpathlineto{\pgfqpoint{3.471290in}{3.843340in}}%
\pgfpathlineto{\pgfqpoint{3.460573in}{3.842260in}}%
\pgfpathlineto{\pgfqpoint{3.449848in}{3.840960in}}%
\pgfpathlineto{\pgfqpoint{3.439116in}{3.839446in}}%
\pgfpathlineto{\pgfqpoint{3.445255in}{3.839425in}}%
\pgfpathlineto{\pgfqpoint{3.451403in}{3.839606in}}%
\pgfpathlineto{\pgfqpoint{3.457559in}{3.839964in}}%
\pgfpathlineto{\pgfqpoint{3.463722in}{3.840476in}}%
\pgfpathclose%
\pgfusepath{stroke,fill}%
\end{pgfscope}%
\begin{pgfscope}%
\pgfpathrectangle{\pgfqpoint{0.887500in}{0.275000in}}{\pgfqpoint{4.225000in}{4.225000in}}%
\pgfusepath{clip}%
\pgfsetbuttcap%
\pgfsetroundjoin%
\definecolor{currentfill}{rgb}{0.741388,0.873449,0.149561}%
\pgfsetfillcolor{currentfill}%
\pgfsetfillopacity{0.700000}%
\pgfsetlinewidth{0.501875pt}%
\definecolor{currentstroke}{rgb}{1.000000,1.000000,1.000000}%
\pgfsetstrokecolor{currentstroke}%
\pgfsetstrokeopacity{0.500000}%
\pgfsetdash{}{0pt}%
\pgfpathmoveto{\pgfqpoint{3.331395in}{3.813049in}}%
\pgfpathlineto{\pgfqpoint{3.342198in}{3.816678in}}%
\pgfpathlineto{\pgfqpoint{3.352994in}{3.820055in}}%
\pgfpathlineto{\pgfqpoint{3.363783in}{3.823193in}}%
\pgfpathlineto{\pgfqpoint{3.374565in}{3.826110in}}%
\pgfpathlineto{\pgfqpoint{3.385341in}{3.828820in}}%
\pgfpathlineto{\pgfqpoint{3.379221in}{3.828126in}}%
\pgfpathlineto{\pgfqpoint{3.373108in}{3.827427in}}%
\pgfpathlineto{\pgfqpoint{3.367001in}{3.826733in}}%
\pgfpathlineto{\pgfqpoint{3.360900in}{3.826050in}}%
\pgfpathlineto{\pgfqpoint{3.350141in}{3.823363in}}%
\pgfpathlineto{\pgfqpoint{3.339377in}{3.820748in}}%
\pgfpathlineto{\pgfqpoint{3.328608in}{3.818110in}}%
\pgfpathlineto{\pgfqpoint{3.317833in}{3.815357in}}%
\pgfpathlineto{\pgfqpoint{3.307051in}{3.812396in}}%
\pgfpathlineto{\pgfqpoint{3.313128in}{3.812583in}}%
\pgfpathlineto{\pgfqpoint{3.319211in}{3.812757in}}%
\pgfpathlineto{\pgfqpoint{3.325300in}{3.812914in}}%
\pgfpathclose%
\pgfusepath{stroke,fill}%
\end{pgfscope}%
\begin{pgfscope}%
\pgfpathrectangle{\pgfqpoint{0.887500in}{0.275000in}}{\pgfqpoint{4.225000in}{4.225000in}}%
\pgfusepath{clip}%
\pgfsetbuttcap%
\pgfsetroundjoin%
\definecolor{currentfill}{rgb}{0.772852,0.877868,0.131109}%
\pgfsetfillcolor{currentfill}%
\pgfsetfillopacity{0.700000}%
\pgfsetlinewidth{0.501875pt}%
\definecolor{currentstroke}{rgb}{1.000000,1.000000,1.000000}%
\pgfsetstrokecolor{currentstroke}%
\pgfsetstrokeopacity{0.500000}%
\pgfsetdash{}{0pt}%
\pgfpathmoveto{\pgfqpoint{3.554573in}{3.845992in}}%
\pgfpathlineto{\pgfqpoint{3.565280in}{3.844758in}}%
\pgfpathlineto{\pgfqpoint{3.575974in}{3.843076in}}%
\pgfpathlineto{\pgfqpoint{3.586655in}{3.840945in}}%
\pgfpathlineto{\pgfqpoint{3.597322in}{3.838363in}}%
\pgfpathlineto{\pgfqpoint{3.591060in}{3.835656in}}%
\pgfpathlineto{\pgfqpoint{3.584811in}{3.833459in}}%
\pgfpathlineto{\pgfqpoint{3.578577in}{3.831862in}}%
\pgfpathlineto{\pgfqpoint{3.572358in}{3.830954in}}%
\pgfpathlineto{\pgfqpoint{3.566155in}{3.830824in}}%
\pgfpathlineto{\pgfqpoint{3.555509in}{3.833699in}}%
\pgfpathlineto{\pgfqpoint{3.544850in}{3.836138in}}%
\pgfpathlineto{\pgfqpoint{3.534177in}{3.838141in}}%
\pgfpathlineto{\pgfqpoint{3.523493in}{3.839711in}}%
\pgfpathlineto{\pgfqpoint{3.529688in}{3.840273in}}%
\pgfpathlineto{\pgfqpoint{3.535894in}{3.841236in}}%
\pgfpathlineto{\pgfqpoint{3.542110in}{3.842547in}}%
\pgfpathlineto{\pgfqpoint{3.548337in}{3.844150in}}%
\pgfpathclose%
\pgfusepath{stroke,fill}%
\end{pgfscope}%
\begin{pgfscope}%
\pgfpathrectangle{\pgfqpoint{0.887500in}{0.275000in}}{\pgfqpoint{4.225000in}{4.225000in}}%
\pgfusepath{clip}%
\pgfsetbuttcap%
\pgfsetroundjoin%
\definecolor{currentfill}{rgb}{0.730889,0.871916,0.156029}%
\pgfsetfillcolor{currentfill}%
\pgfsetfillopacity{0.700000}%
\pgfsetlinewidth{0.501875pt}%
\definecolor{currentstroke}{rgb}{1.000000,1.000000,1.000000}%
\pgfsetstrokecolor{currentstroke}%
\pgfsetstrokeopacity{0.500000}%
\pgfsetdash{}{0pt}%
\pgfpathmoveto{\pgfqpoint{3.277292in}{3.790926in}}%
\pgfpathlineto{\pgfqpoint{3.288124in}{3.795822in}}%
\pgfpathlineto{\pgfqpoint{3.298950in}{3.800517in}}%
\pgfpathlineto{\pgfqpoint{3.309772in}{3.804972in}}%
\pgfpathlineto{\pgfqpoint{3.320587in}{3.809152in}}%
\pgfpathlineto{\pgfqpoint{3.331395in}{3.813049in}}%
\pgfpathlineto{\pgfqpoint{3.325300in}{3.812914in}}%
\pgfpathlineto{\pgfqpoint{3.319211in}{3.812757in}}%
\pgfpathlineto{\pgfqpoint{3.313128in}{3.812583in}}%
\pgfpathlineto{\pgfqpoint{3.307051in}{3.812396in}}%
\pgfpathlineto{\pgfqpoint{3.296263in}{3.809132in}}%
\pgfpathlineto{\pgfqpoint{3.285468in}{3.805475in}}%
\pgfpathlineto{\pgfqpoint{3.274665in}{3.801401in}}%
\pgfpathlineto{\pgfqpoint{3.263857in}{3.796971in}}%
\pgfpathlineto{\pgfqpoint{3.253043in}{3.792250in}}%
\pgfpathlineto{\pgfqpoint{3.259096in}{3.791960in}}%
\pgfpathlineto{\pgfqpoint{3.265155in}{3.791643in}}%
\pgfpathlineto{\pgfqpoint{3.271221in}{3.791299in}}%
\pgfpathclose%
\pgfusepath{stroke,fill}%
\end{pgfscope}%
\begin{pgfscope}%
\pgfpathrectangle{\pgfqpoint{0.887500in}{0.275000in}}{\pgfqpoint{4.225000in}{4.225000in}}%
\pgfusepath{clip}%
\pgfsetbuttcap%
\pgfsetroundjoin%
\definecolor{currentfill}{rgb}{0.772852,0.877868,0.131109}%
\pgfsetfillcolor{currentfill}%
\pgfsetfillopacity{0.700000}%
\pgfsetlinewidth{0.501875pt}%
\definecolor{currentstroke}{rgb}{1.000000,1.000000,1.000000}%
\pgfsetstrokecolor{currentstroke}%
\pgfsetstrokeopacity{0.500000}%
\pgfsetdash{}{0pt}%
\pgfpathmoveto{\pgfqpoint{3.416033in}{3.831960in}}%
\pgfpathlineto{\pgfqpoint{3.426824in}{3.834626in}}%
\pgfpathlineto{\pgfqpoint{3.437606in}{3.836876in}}%
\pgfpathlineto{\pgfqpoint{3.448378in}{3.838710in}}%
\pgfpathlineto{\pgfqpoint{3.459141in}{3.840124in}}%
\pgfpathlineto{\pgfqpoint{3.469893in}{3.841118in}}%
\pgfpathlineto{\pgfqpoint{3.463722in}{3.840476in}}%
\pgfpathlineto{\pgfqpoint{3.457559in}{3.839964in}}%
\pgfpathlineto{\pgfqpoint{3.451403in}{3.839606in}}%
\pgfpathlineto{\pgfqpoint{3.445255in}{3.839425in}}%
\pgfpathlineto{\pgfqpoint{3.439116in}{3.839446in}}%
\pgfpathlineto{\pgfqpoint{3.428375in}{3.837721in}}%
\pgfpathlineto{\pgfqpoint{3.417627in}{3.835792in}}%
\pgfpathlineto{\pgfqpoint{3.406872in}{3.833662in}}%
\pgfpathlineto{\pgfqpoint{3.396110in}{3.831336in}}%
\pgfpathlineto{\pgfqpoint{3.385341in}{3.828820in}}%
\pgfpathlineto{\pgfqpoint{3.391467in}{3.829502in}}%
\pgfpathlineto{\pgfqpoint{3.397600in}{3.830165in}}%
\pgfpathlineto{\pgfqpoint{3.403738in}{3.830801in}}%
\pgfpathlineto{\pgfqpoint{3.409883in}{3.831402in}}%
\pgfpathclose%
\pgfusepath{stroke,fill}%
\end{pgfscope}%
\begin{pgfscope}%
\pgfpathrectangle{\pgfqpoint{0.887500in}{0.275000in}}{\pgfqpoint{4.225000in}{4.225000in}}%
\pgfusepath{clip}%
\pgfsetbuttcap%
\pgfsetroundjoin%
\definecolor{currentfill}{rgb}{0.709898,0.868751,0.169257}%
\pgfsetfillcolor{currentfill}%
\pgfsetfillopacity{0.700000}%
\pgfsetlinewidth{0.501875pt}%
\definecolor{currentstroke}{rgb}{1.000000,1.000000,1.000000}%
\pgfsetstrokecolor{currentstroke}%
\pgfsetstrokeopacity{0.500000}%
\pgfsetdash{}{0pt}%
\pgfpathmoveto{\pgfqpoint{3.223064in}{3.764696in}}%
\pgfpathlineto{\pgfqpoint{3.233918in}{3.770067in}}%
\pgfpathlineto{\pgfqpoint{3.244768in}{3.775398in}}%
\pgfpathlineto{\pgfqpoint{3.255613in}{3.780676in}}%
\pgfpathlineto{\pgfqpoint{3.266455in}{3.785864in}}%
\pgfpathlineto{\pgfqpoint{3.277292in}{3.790926in}}%
\pgfpathlineto{\pgfqpoint{3.271221in}{3.791299in}}%
\pgfpathlineto{\pgfqpoint{3.265155in}{3.791643in}}%
\pgfpathlineto{\pgfqpoint{3.259096in}{3.791960in}}%
\pgfpathlineto{\pgfqpoint{3.253043in}{3.792250in}}%
\pgfpathlineto{\pgfqpoint{3.242223in}{3.787306in}}%
\pgfpathlineto{\pgfqpoint{3.231399in}{3.782204in}}%
\pgfpathlineto{\pgfqpoint{3.220571in}{3.777011in}}%
\pgfpathlineto{\pgfqpoint{3.209739in}{3.771791in}}%
\pgfpathlineto{\pgfqpoint{3.198903in}{3.766567in}}%
\pgfpathlineto{\pgfqpoint{3.204934in}{3.766195in}}%
\pgfpathlineto{\pgfqpoint{3.210972in}{3.765758in}}%
\pgfpathlineto{\pgfqpoint{3.217015in}{3.765257in}}%
\pgfpathclose%
\pgfusepath{stroke,fill}%
\end{pgfscope}%
\begin{pgfscope}%
\pgfpathrectangle{\pgfqpoint{0.887500in}{0.275000in}}{\pgfqpoint{4.225000in}{4.225000in}}%
\pgfusepath{clip}%
\pgfsetbuttcap%
\pgfsetroundjoin%
\definecolor{currentfill}{rgb}{0.804182,0.882046,0.114965}%
\pgfsetfillcolor{currentfill}%
\pgfsetfillopacity{0.700000}%
\pgfsetlinewidth{0.501875pt}%
\definecolor{currentstroke}{rgb}{1.000000,1.000000,1.000000}%
\pgfsetstrokecolor{currentstroke}%
\pgfsetstrokeopacity{0.500000}%
\pgfsetdash{}{0pt}%
\pgfpathmoveto{\pgfqpoint{3.500858in}{3.845459in}}%
\pgfpathlineto{\pgfqpoint{3.511624in}{3.846456in}}%
\pgfpathlineto{\pgfqpoint{3.522379in}{3.847009in}}%
\pgfpathlineto{\pgfqpoint{3.533122in}{3.847116in}}%
\pgfpathlineto{\pgfqpoint{3.543854in}{3.846777in}}%
\pgfpathlineto{\pgfqpoint{3.554573in}{3.845992in}}%
\pgfpathlineto{\pgfqpoint{3.548337in}{3.844150in}}%
\pgfpathlineto{\pgfqpoint{3.542110in}{3.842547in}}%
\pgfpathlineto{\pgfqpoint{3.535894in}{3.841236in}}%
\pgfpathlineto{\pgfqpoint{3.529688in}{3.840273in}}%
\pgfpathlineto{\pgfqpoint{3.523493in}{3.839711in}}%
\pgfpathlineto{\pgfqpoint{3.512796in}{3.840849in}}%
\pgfpathlineto{\pgfqpoint{3.502087in}{3.841557in}}%
\pgfpathlineto{\pgfqpoint{3.491367in}{3.841836in}}%
\pgfpathlineto{\pgfqpoint{3.480635in}{3.841690in}}%
\pgfpathlineto{\pgfqpoint{3.469893in}{3.841118in}}%
\pgfpathlineto{\pgfqpoint{3.476072in}{3.841868in}}%
\pgfpathlineto{\pgfqpoint{3.482258in}{3.842700in}}%
\pgfpathlineto{\pgfqpoint{3.488451in}{3.843592in}}%
\pgfpathlineto{\pgfqpoint{3.494651in}{3.844519in}}%
\pgfpathclose%
\pgfusepath{stroke,fill}%
\end{pgfscope}%
\begin{pgfscope}%
\pgfpathrectangle{\pgfqpoint{0.887500in}{0.275000in}}{\pgfqpoint{4.225000in}{4.225000in}}%
\pgfusepath{clip}%
\pgfsetbuttcap%
\pgfsetroundjoin%
\definecolor{currentfill}{rgb}{0.783315,0.879285,0.125405}%
\pgfsetfillcolor{currentfill}%
\pgfsetfillopacity{0.700000}%
\pgfsetlinewidth{0.501875pt}%
\definecolor{currentstroke}{rgb}{1.000000,1.000000,1.000000}%
\pgfsetstrokecolor{currentstroke}%
\pgfsetstrokeopacity{0.500000}%
\pgfsetdash{}{0pt}%
\pgfpathmoveto{\pgfqpoint{3.361961in}{3.813258in}}%
\pgfpathlineto{\pgfqpoint{3.372789in}{3.817604in}}%
\pgfpathlineto{\pgfqpoint{3.383611in}{3.821681in}}%
\pgfpathlineto{\pgfqpoint{3.394426in}{3.825455in}}%
\pgfpathlineto{\pgfqpoint{3.405234in}{3.828893in}}%
\pgfpathlineto{\pgfqpoint{3.416033in}{3.831960in}}%
\pgfpathlineto{\pgfqpoint{3.409883in}{3.831402in}}%
\pgfpathlineto{\pgfqpoint{3.403738in}{3.830801in}}%
\pgfpathlineto{\pgfqpoint{3.397600in}{3.830165in}}%
\pgfpathlineto{\pgfqpoint{3.391467in}{3.829502in}}%
\pgfpathlineto{\pgfqpoint{3.385341in}{3.828820in}}%
\pgfpathlineto{\pgfqpoint{3.374565in}{3.826110in}}%
\pgfpathlineto{\pgfqpoint{3.363783in}{3.823193in}}%
\pgfpathlineto{\pgfqpoint{3.352994in}{3.820055in}}%
\pgfpathlineto{\pgfqpoint{3.342198in}{3.816678in}}%
\pgfpathlineto{\pgfqpoint{3.331395in}{3.813049in}}%
\pgfpathlineto{\pgfqpoint{3.337497in}{3.813158in}}%
\pgfpathlineto{\pgfqpoint{3.343604in}{3.813238in}}%
\pgfpathlineto{\pgfqpoint{3.349717in}{3.813284in}}%
\pgfpathlineto{\pgfqpoint{3.355836in}{3.813292in}}%
\pgfpathclose%
\pgfusepath{stroke,fill}%
\end{pgfscope}%
\begin{pgfscope}%
\pgfpathrectangle{\pgfqpoint{0.887500in}{0.275000in}}{\pgfqpoint{4.225000in}{4.225000in}}%
\pgfusepath{clip}%
\pgfsetbuttcap%
\pgfsetroundjoin%
\definecolor{currentfill}{rgb}{0.688944,0.865448,0.182725}%
\pgfsetfillcolor{currentfill}%
\pgfsetfillopacity{0.700000}%
\pgfsetlinewidth{0.501875pt}%
\definecolor{currentstroke}{rgb}{1.000000,1.000000,1.000000}%
\pgfsetstrokecolor{currentstroke}%
\pgfsetstrokeopacity{0.500000}%
\pgfsetdash{}{0pt}%
\pgfpathmoveto{\pgfqpoint{3.168736in}{3.737108in}}%
\pgfpathlineto{\pgfqpoint{3.179609in}{3.742740in}}%
\pgfpathlineto{\pgfqpoint{3.190479in}{3.748310in}}%
\pgfpathlineto{\pgfqpoint{3.201344in}{3.753823in}}%
\pgfpathlineto{\pgfqpoint{3.212206in}{3.759283in}}%
\pgfpathlineto{\pgfqpoint{3.223064in}{3.764696in}}%
\pgfpathlineto{\pgfqpoint{3.217015in}{3.765257in}}%
\pgfpathlineto{\pgfqpoint{3.210972in}{3.765758in}}%
\pgfpathlineto{\pgfqpoint{3.204934in}{3.766195in}}%
\pgfpathlineto{\pgfqpoint{3.198903in}{3.766567in}}%
\pgfpathlineto{\pgfqpoint{3.188063in}{3.761329in}}%
\pgfpathlineto{\pgfqpoint{3.177219in}{3.756065in}}%
\pgfpathlineto{\pgfqpoint{3.166371in}{3.750762in}}%
\pgfpathlineto{\pgfqpoint{3.155519in}{3.745409in}}%
\pgfpathlineto{\pgfqpoint{3.144663in}{3.739992in}}%
\pgfpathlineto{\pgfqpoint{3.150672in}{3.739367in}}%
\pgfpathlineto{\pgfqpoint{3.156688in}{3.738676in}}%
\pgfpathlineto{\pgfqpoint{3.162709in}{3.737921in}}%
\pgfpathclose%
\pgfusepath{stroke,fill}%
\end{pgfscope}%
\begin{pgfscope}%
\pgfpathrectangle{\pgfqpoint{0.887500in}{0.275000in}}{\pgfqpoint{4.225000in}{4.225000in}}%
\pgfusepath{clip}%
\pgfsetbuttcap%
\pgfsetroundjoin%
\definecolor{currentfill}{rgb}{0.772852,0.877868,0.131109}%
\pgfsetfillcolor{currentfill}%
\pgfsetfillopacity{0.700000}%
\pgfsetlinewidth{0.501875pt}%
\definecolor{currentstroke}{rgb}{1.000000,1.000000,1.000000}%
\pgfsetstrokecolor{currentstroke}%
\pgfsetstrokeopacity{0.500000}%
\pgfsetdash{}{0pt}%
\pgfpathmoveto{\pgfqpoint{3.307735in}{3.788592in}}%
\pgfpathlineto{\pgfqpoint{3.318590in}{3.793833in}}%
\pgfpathlineto{\pgfqpoint{3.329441in}{3.798941in}}%
\pgfpathlineto{\pgfqpoint{3.340286in}{3.803896in}}%
\pgfpathlineto{\pgfqpoint{3.351126in}{3.808677in}}%
\pgfpathlineto{\pgfqpoint{3.361961in}{3.813258in}}%
\pgfpathlineto{\pgfqpoint{3.355836in}{3.813292in}}%
\pgfpathlineto{\pgfqpoint{3.349717in}{3.813284in}}%
\pgfpathlineto{\pgfqpoint{3.343604in}{3.813238in}}%
\pgfpathlineto{\pgfqpoint{3.337497in}{3.813158in}}%
\pgfpathlineto{\pgfqpoint{3.331395in}{3.813049in}}%
\pgfpathlineto{\pgfqpoint{3.320587in}{3.809152in}}%
\pgfpathlineto{\pgfqpoint{3.309772in}{3.804972in}}%
\pgfpathlineto{\pgfqpoint{3.298950in}{3.800517in}}%
\pgfpathlineto{\pgfqpoint{3.288124in}{3.795822in}}%
\pgfpathlineto{\pgfqpoint{3.277292in}{3.790926in}}%
\pgfpathlineto{\pgfqpoint{3.283369in}{3.790523in}}%
\pgfpathlineto{\pgfqpoint{3.289452in}{3.790089in}}%
\pgfpathlineto{\pgfqpoint{3.295540in}{3.789623in}}%
\pgfpathlineto{\pgfqpoint{3.301635in}{3.789125in}}%
\pgfpathclose%
\pgfusepath{stroke,fill}%
\end{pgfscope}%
\begin{pgfscope}%
\pgfpathrectangle{\pgfqpoint{0.887500in}{0.275000in}}{\pgfqpoint{4.225000in}{4.225000in}}%
\pgfusepath{clip}%
\pgfsetbuttcap%
\pgfsetroundjoin%
\definecolor{currentfill}{rgb}{0.835270,0.886029,0.102646}%
\pgfsetfillcolor{currentfill}%
\pgfsetfillopacity{0.700000}%
\pgfsetlinewidth{0.501875pt}%
\definecolor{currentstroke}{rgb}{1.000000,1.000000,1.000000}%
\pgfsetstrokecolor{currentstroke}%
\pgfsetstrokeopacity{0.500000}%
\pgfsetdash{}{0pt}%
\pgfpathmoveto{\pgfqpoint{3.585879in}{3.856850in}}%
\pgfpathlineto{\pgfqpoint{3.596627in}{3.857272in}}%
\pgfpathlineto{\pgfqpoint{3.607364in}{3.857341in}}%
\pgfpathlineto{\pgfqpoint{3.618090in}{3.857056in}}%
\pgfpathlineto{\pgfqpoint{3.628804in}{3.856416in}}%
\pgfpathlineto{\pgfqpoint{3.622489in}{3.852504in}}%
\pgfpathlineto{\pgfqpoint{3.616182in}{3.848652in}}%
\pgfpathlineto{\pgfqpoint{3.609884in}{3.844951in}}%
\pgfpathlineto{\pgfqpoint{3.603597in}{3.841492in}}%
\pgfpathlineto{\pgfqpoint{3.597322in}{3.838363in}}%
\pgfpathlineto{\pgfqpoint{3.586655in}{3.840945in}}%
\pgfpathlineto{\pgfqpoint{3.575974in}{3.843076in}}%
\pgfpathlineto{\pgfqpoint{3.565280in}{3.844758in}}%
\pgfpathlineto{\pgfqpoint{3.554573in}{3.845992in}}%
\pgfpathlineto{\pgfqpoint{3.560819in}{3.848017in}}%
\pgfpathlineto{\pgfqpoint{3.567073in}{3.850170in}}%
\pgfpathlineto{\pgfqpoint{3.573335in}{3.852397in}}%
\pgfpathlineto{\pgfqpoint{3.579604in}{3.854642in}}%
\pgfpathclose%
\pgfusepath{stroke,fill}%
\end{pgfscope}%
\begin{pgfscope}%
\pgfpathrectangle{\pgfqpoint{0.887500in}{0.275000in}}{\pgfqpoint{4.225000in}{4.225000in}}%
\pgfusepath{clip}%
\pgfsetbuttcap%
\pgfsetroundjoin%
\definecolor{currentfill}{rgb}{0.824940,0.884720,0.106217}%
\pgfsetfillcolor{currentfill}%
\pgfsetfillopacity{0.700000}%
\pgfsetlinewidth{0.501875pt}%
\definecolor{currentstroke}{rgb}{1.000000,1.000000,1.000000}%
\pgfsetstrokecolor{currentstroke}%
\pgfsetstrokeopacity{0.500000}%
\pgfsetdash{}{0pt}%
\pgfpathmoveto{\pgfqpoint{3.446873in}{3.833842in}}%
\pgfpathlineto{\pgfqpoint{3.457689in}{3.837046in}}%
\pgfpathlineto{\pgfqpoint{3.468496in}{3.839811in}}%
\pgfpathlineto{\pgfqpoint{3.479294in}{3.842135in}}%
\pgfpathlineto{\pgfqpoint{3.490081in}{3.844018in}}%
\pgfpathlineto{\pgfqpoint{3.500858in}{3.845459in}}%
\pgfpathlineto{\pgfqpoint{3.494651in}{3.844519in}}%
\pgfpathlineto{\pgfqpoint{3.488451in}{3.843592in}}%
\pgfpathlineto{\pgfqpoint{3.482258in}{3.842700in}}%
\pgfpathlineto{\pgfqpoint{3.476072in}{3.841868in}}%
\pgfpathlineto{\pgfqpoint{3.469893in}{3.841118in}}%
\pgfpathlineto{\pgfqpoint{3.459141in}{3.840124in}}%
\pgfpathlineto{\pgfqpoint{3.448378in}{3.838710in}}%
\pgfpathlineto{\pgfqpoint{3.437606in}{3.836876in}}%
\pgfpathlineto{\pgfqpoint{3.426824in}{3.834626in}}%
\pgfpathlineto{\pgfqpoint{3.416033in}{3.831960in}}%
\pgfpathlineto{\pgfqpoint{3.422190in}{3.832468in}}%
\pgfpathlineto{\pgfqpoint{3.428352in}{3.832919in}}%
\pgfpathlineto{\pgfqpoint{3.434520in}{3.833303in}}%
\pgfpathlineto{\pgfqpoint{3.440694in}{3.833613in}}%
\pgfpathclose%
\pgfusepath{stroke,fill}%
\end{pgfscope}%
\begin{pgfscope}%
\pgfpathrectangle{\pgfqpoint{0.887500in}{0.275000in}}{\pgfqpoint{4.225000in}{4.225000in}}%
\pgfusepath{clip}%
\pgfsetbuttcap%
\pgfsetroundjoin%
\definecolor{currentfill}{rgb}{0.668054,0.861999,0.196293}%
\pgfsetfillcolor{currentfill}%
\pgfsetfillopacity{0.700000}%
\pgfsetlinewidth{0.501875pt}%
\definecolor{currentstroke}{rgb}{1.000000,1.000000,1.000000}%
\pgfsetstrokecolor{currentstroke}%
\pgfsetstrokeopacity{0.500000}%
\pgfsetdash{}{0pt}%
\pgfpathmoveto{\pgfqpoint{3.114311in}{3.708151in}}%
\pgfpathlineto{\pgfqpoint{3.125203in}{3.714011in}}%
\pgfpathlineto{\pgfqpoint{3.136092in}{3.719850in}}%
\pgfpathlineto{\pgfqpoint{3.146977in}{3.725654in}}%
\pgfpathlineto{\pgfqpoint{3.157858in}{3.731410in}}%
\pgfpathlineto{\pgfqpoint{3.168736in}{3.737108in}}%
\pgfpathlineto{\pgfqpoint{3.162709in}{3.737921in}}%
\pgfpathlineto{\pgfqpoint{3.156688in}{3.738676in}}%
\pgfpathlineto{\pgfqpoint{3.150672in}{3.739367in}}%
\pgfpathlineto{\pgfqpoint{3.144663in}{3.739992in}}%
\pgfpathlineto{\pgfqpoint{3.133803in}{3.734502in}}%
\pgfpathlineto{\pgfqpoint{3.122939in}{3.728945in}}%
\pgfpathlineto{\pgfqpoint{3.112072in}{3.723334in}}%
\pgfpathlineto{\pgfqpoint{3.101200in}{3.717688in}}%
\pgfpathlineto{\pgfqpoint{3.090326in}{3.712019in}}%
\pgfpathlineto{\pgfqpoint{3.096313in}{3.711130in}}%
\pgfpathlineto{\pgfqpoint{3.102307in}{3.710187in}}%
\pgfpathlineto{\pgfqpoint{3.108306in}{3.709193in}}%
\pgfpathclose%
\pgfusepath{stroke,fill}%
\end{pgfscope}%
\begin{pgfscope}%
\pgfpathrectangle{\pgfqpoint{0.887500in}{0.275000in}}{\pgfqpoint{4.225000in}{4.225000in}}%
\pgfusepath{clip}%
\pgfsetbuttcap%
\pgfsetroundjoin%
\definecolor{currentfill}{rgb}{0.751884,0.874951,0.143228}%
\pgfsetfillcolor{currentfill}%
\pgfsetfillopacity{0.700000}%
\pgfsetlinewidth{0.501875pt}%
\definecolor{currentstroke}{rgb}{1.000000,1.000000,1.000000}%
\pgfsetstrokecolor{currentstroke}%
\pgfsetstrokeopacity{0.500000}%
\pgfsetdash{}{0pt}%
\pgfpathmoveto{\pgfqpoint{3.253394in}{3.761084in}}%
\pgfpathlineto{\pgfqpoint{3.264271in}{3.766701in}}%
\pgfpathlineto{\pgfqpoint{3.275143in}{3.772274in}}%
\pgfpathlineto{\pgfqpoint{3.286011in}{3.777793in}}%
\pgfpathlineto{\pgfqpoint{3.296875in}{3.783239in}}%
\pgfpathlineto{\pgfqpoint{3.307735in}{3.788592in}}%
\pgfpathlineto{\pgfqpoint{3.301635in}{3.789125in}}%
\pgfpathlineto{\pgfqpoint{3.295540in}{3.789623in}}%
\pgfpathlineto{\pgfqpoint{3.289452in}{3.790089in}}%
\pgfpathlineto{\pgfqpoint{3.283369in}{3.790523in}}%
\pgfpathlineto{\pgfqpoint{3.277292in}{3.790926in}}%
\pgfpathlineto{\pgfqpoint{3.266455in}{3.785864in}}%
\pgfpathlineto{\pgfqpoint{3.255613in}{3.780676in}}%
\pgfpathlineto{\pgfqpoint{3.244768in}{3.775398in}}%
\pgfpathlineto{\pgfqpoint{3.233918in}{3.770067in}}%
\pgfpathlineto{\pgfqpoint{3.223064in}{3.764696in}}%
\pgfpathlineto{\pgfqpoint{3.229119in}{3.764078in}}%
\pgfpathlineto{\pgfqpoint{3.235179in}{3.763405in}}%
\pgfpathlineto{\pgfqpoint{3.241245in}{3.762680in}}%
\pgfpathlineto{\pgfqpoint{3.247317in}{3.761905in}}%
\pgfpathclose%
\pgfusepath{stroke,fill}%
\end{pgfscope}%
\begin{pgfscope}%
\pgfpathrectangle{\pgfqpoint{0.887500in}{0.275000in}}{\pgfqpoint{4.225000in}{4.225000in}}%
\pgfusepath{clip}%
\pgfsetbuttcap%
\pgfsetroundjoin%
\definecolor{currentfill}{rgb}{0.636902,0.856542,0.216620}%
\pgfsetfillcolor{currentfill}%
\pgfsetfillopacity{0.700000}%
\pgfsetlinewidth{0.501875pt}%
\definecolor{currentstroke}{rgb}{1.000000,1.000000,1.000000}%
\pgfsetstrokecolor{currentstroke}%
\pgfsetstrokeopacity{0.500000}%
\pgfsetdash{}{0pt}%
\pgfpathmoveto{\pgfqpoint{3.059796in}{3.678957in}}%
\pgfpathlineto{\pgfqpoint{3.070706in}{3.684742in}}%
\pgfpathlineto{\pgfqpoint{3.081613in}{3.690565in}}%
\pgfpathlineto{\pgfqpoint{3.092516in}{3.696416in}}%
\pgfpathlineto{\pgfqpoint{3.103415in}{3.702282in}}%
\pgfpathlineto{\pgfqpoint{3.114311in}{3.708151in}}%
\pgfpathlineto{\pgfqpoint{3.108306in}{3.709193in}}%
\pgfpathlineto{\pgfqpoint{3.102307in}{3.710187in}}%
\pgfpathlineto{\pgfqpoint{3.096313in}{3.711130in}}%
\pgfpathlineto{\pgfqpoint{3.090326in}{3.712019in}}%
\pgfpathlineto{\pgfqpoint{3.079447in}{3.706346in}}%
\pgfpathlineto{\pgfqpoint{3.068565in}{3.700683in}}%
\pgfpathlineto{\pgfqpoint{3.057679in}{3.695044in}}%
\pgfpathlineto{\pgfqpoint{3.046789in}{3.689440in}}%
\pgfpathlineto{\pgfqpoint{3.035896in}{3.683877in}}%
\pgfpathlineto{\pgfqpoint{3.041862in}{3.682711in}}%
\pgfpathlineto{\pgfqpoint{3.047835in}{3.681499in}}%
\pgfpathlineto{\pgfqpoint{3.053813in}{3.680245in}}%
\pgfpathclose%
\pgfusepath{stroke,fill}%
\end{pgfscope}%
\begin{pgfscope}%
\pgfpathrectangle{\pgfqpoint{0.887500in}{0.275000in}}{\pgfqpoint{4.225000in}{4.225000in}}%
\pgfusepath{clip}%
\pgfsetbuttcap%
\pgfsetroundjoin%
\definecolor{currentfill}{rgb}{0.824940,0.884720,0.106217}%
\pgfsetfillcolor{currentfill}%
\pgfsetfillopacity{0.700000}%
\pgfsetlinewidth{0.501875pt}%
\definecolor{currentstroke}{rgb}{1.000000,1.000000,1.000000}%
\pgfsetstrokecolor{currentstroke}%
\pgfsetstrokeopacity{0.500000}%
\pgfsetdash{}{0pt}%
\pgfpathmoveto{\pgfqpoint{3.392672in}{3.812310in}}%
\pgfpathlineto{\pgfqpoint{3.403526in}{3.817208in}}%
\pgfpathlineto{\pgfqpoint{3.414373in}{3.821854in}}%
\pgfpathlineto{\pgfqpoint{3.425214in}{3.826204in}}%
\pgfpathlineto{\pgfqpoint{3.436048in}{3.830215in}}%
\pgfpathlineto{\pgfqpoint{3.446873in}{3.833842in}}%
\pgfpathlineto{\pgfqpoint{3.440694in}{3.833613in}}%
\pgfpathlineto{\pgfqpoint{3.434520in}{3.833303in}}%
\pgfpathlineto{\pgfqpoint{3.428352in}{3.832919in}}%
\pgfpathlineto{\pgfqpoint{3.422190in}{3.832468in}}%
\pgfpathlineto{\pgfqpoint{3.416033in}{3.831960in}}%
\pgfpathlineto{\pgfqpoint{3.405234in}{3.828893in}}%
\pgfpathlineto{\pgfqpoint{3.394426in}{3.825455in}}%
\pgfpathlineto{\pgfqpoint{3.383611in}{3.821681in}}%
\pgfpathlineto{\pgfqpoint{3.372789in}{3.817604in}}%
\pgfpathlineto{\pgfqpoint{3.361961in}{3.813258in}}%
\pgfpathlineto{\pgfqpoint{3.368091in}{3.813178in}}%
\pgfpathlineto{\pgfqpoint{3.374228in}{3.813047in}}%
\pgfpathlineto{\pgfqpoint{3.380370in}{3.812861in}}%
\pgfpathlineto{\pgfqpoint{3.386518in}{3.812617in}}%
\pgfpathclose%
\pgfusepath{stroke,fill}%
\end{pgfscope}%
\begin{pgfscope}%
\pgfpathrectangle{\pgfqpoint{0.887500in}{0.275000in}}{\pgfqpoint{4.225000in}{4.225000in}}%
\pgfusepath{clip}%
\pgfsetbuttcap%
\pgfsetroundjoin%
\definecolor{currentfill}{rgb}{0.855810,0.888601,0.097452}%
\pgfsetfillcolor{currentfill}%
\pgfsetfillopacity{0.700000}%
\pgfsetlinewidth{0.501875pt}%
\definecolor{currentstroke}{rgb}{1.000000,1.000000,1.000000}%
\pgfsetstrokecolor{currentstroke}%
\pgfsetstrokeopacity{0.500000}%
\pgfsetdash{}{0pt}%
\pgfpathmoveto{\pgfqpoint{3.531982in}{3.849493in}}%
\pgfpathlineto{\pgfqpoint{3.542781in}{3.851661in}}%
\pgfpathlineto{\pgfqpoint{3.553571in}{3.853481in}}%
\pgfpathlineto{\pgfqpoint{3.564351in}{3.854954in}}%
\pgfpathlineto{\pgfqpoint{3.575120in}{3.856077in}}%
\pgfpathlineto{\pgfqpoint{3.585879in}{3.856850in}}%
\pgfpathlineto{\pgfqpoint{3.579604in}{3.854642in}}%
\pgfpathlineto{\pgfqpoint{3.573335in}{3.852397in}}%
\pgfpathlineto{\pgfqpoint{3.567073in}{3.850170in}}%
\pgfpathlineto{\pgfqpoint{3.560819in}{3.848017in}}%
\pgfpathlineto{\pgfqpoint{3.554573in}{3.845992in}}%
\pgfpathlineto{\pgfqpoint{3.543854in}{3.846777in}}%
\pgfpathlineto{\pgfqpoint{3.533122in}{3.847116in}}%
\pgfpathlineto{\pgfqpoint{3.522379in}{3.847009in}}%
\pgfpathlineto{\pgfqpoint{3.511624in}{3.846456in}}%
\pgfpathlineto{\pgfqpoint{3.500858in}{3.845459in}}%
\pgfpathlineto{\pgfqpoint{3.507071in}{3.846386in}}%
\pgfpathlineto{\pgfqpoint{3.513290in}{3.847277in}}%
\pgfpathlineto{\pgfqpoint{3.519515in}{3.848108in}}%
\pgfpathlineto{\pgfqpoint{3.525746in}{3.848854in}}%
\pgfpathclose%
\pgfusepath{stroke,fill}%
\end{pgfscope}%
\begin{pgfscope}%
\pgfpathrectangle{\pgfqpoint{0.887500in}{0.275000in}}{\pgfqpoint{4.225000in}{4.225000in}}%
\pgfusepath{clip}%
\pgfsetbuttcap%
\pgfsetroundjoin%
\definecolor{currentfill}{rgb}{0.720391,0.870350,0.162603}%
\pgfsetfillcolor{currentfill}%
\pgfsetfillopacity{0.700000}%
\pgfsetlinewidth{0.501875pt}%
\definecolor{currentstroke}{rgb}{1.000000,1.000000,1.000000}%
\pgfsetstrokecolor{currentstroke}%
\pgfsetstrokeopacity{0.500000}%
\pgfsetdash{}{0pt}%
\pgfpathmoveto{\pgfqpoint{3.198955in}{3.732289in}}%
\pgfpathlineto{\pgfqpoint{3.209851in}{3.738146in}}%
\pgfpathlineto{\pgfqpoint{3.220742in}{3.743954in}}%
\pgfpathlineto{\pgfqpoint{3.231630in}{3.749711in}}%
\pgfpathlineto{\pgfqpoint{3.242514in}{3.755421in}}%
\pgfpathlineto{\pgfqpoint{3.253394in}{3.761084in}}%
\pgfpathlineto{\pgfqpoint{3.247317in}{3.761905in}}%
\pgfpathlineto{\pgfqpoint{3.241245in}{3.762680in}}%
\pgfpathlineto{\pgfqpoint{3.235179in}{3.763405in}}%
\pgfpathlineto{\pgfqpoint{3.229119in}{3.764078in}}%
\pgfpathlineto{\pgfqpoint{3.223064in}{3.764696in}}%
\pgfpathlineto{\pgfqpoint{3.212206in}{3.759283in}}%
\pgfpathlineto{\pgfqpoint{3.201344in}{3.753823in}}%
\pgfpathlineto{\pgfqpoint{3.190479in}{3.748310in}}%
\pgfpathlineto{\pgfqpoint{3.179609in}{3.742740in}}%
\pgfpathlineto{\pgfqpoint{3.168736in}{3.737108in}}%
\pgfpathlineto{\pgfqpoint{3.174768in}{3.736240in}}%
\pgfpathlineto{\pgfqpoint{3.180806in}{3.735320in}}%
\pgfpathlineto{\pgfqpoint{3.186850in}{3.734352in}}%
\pgfpathlineto{\pgfqpoint{3.192900in}{3.733340in}}%
\pgfpathclose%
\pgfusepath{stroke,fill}%
\end{pgfscope}%
\begin{pgfscope}%
\pgfpathrectangle{\pgfqpoint{0.887500in}{0.275000in}}{\pgfqpoint{4.225000in}{4.225000in}}%
\pgfusepath{clip}%
\pgfsetbuttcap%
\pgfsetroundjoin%
\definecolor{currentfill}{rgb}{0.616293,0.852709,0.230052}%
\pgfsetfillcolor{currentfill}%
\pgfsetfillopacity{0.700000}%
\pgfsetlinewidth{0.501875pt}%
\definecolor{currentstroke}{rgb}{1.000000,1.000000,1.000000}%
\pgfsetstrokecolor{currentstroke}%
\pgfsetstrokeopacity{0.500000}%
\pgfsetdash{}{0pt}%
\pgfpathmoveto{\pgfqpoint{3.005189in}{3.650995in}}%
\pgfpathlineto{\pgfqpoint{3.016118in}{3.656426in}}%
\pgfpathlineto{\pgfqpoint{3.027044in}{3.661945in}}%
\pgfpathlineto{\pgfqpoint{3.037965in}{3.667547in}}%
\pgfpathlineto{\pgfqpoint{3.048882in}{3.673222in}}%
\pgfpathlineto{\pgfqpoint{3.059796in}{3.678957in}}%
\pgfpathlineto{\pgfqpoint{3.053813in}{3.680245in}}%
\pgfpathlineto{\pgfqpoint{3.047835in}{3.681499in}}%
\pgfpathlineto{\pgfqpoint{3.041862in}{3.682711in}}%
\pgfpathlineto{\pgfqpoint{3.035896in}{3.683877in}}%
\pgfpathlineto{\pgfqpoint{3.024998in}{3.678362in}}%
\pgfpathlineto{\pgfqpoint{3.014097in}{3.672902in}}%
\pgfpathlineto{\pgfqpoint{3.003192in}{3.667504in}}%
\pgfpathlineto{\pgfqpoint{2.992284in}{3.662175in}}%
\pgfpathlineto{\pgfqpoint{2.981371in}{3.656917in}}%
\pgfpathlineto{\pgfqpoint{2.987317in}{3.655467in}}%
\pgfpathlineto{\pgfqpoint{2.993269in}{3.653993in}}%
\pgfpathlineto{\pgfqpoint{2.999226in}{3.652500in}}%
\pgfpathclose%
\pgfusepath{stroke,fill}%
\end{pgfscope}%
\begin{pgfscope}%
\pgfpathrectangle{\pgfqpoint{0.887500in}{0.275000in}}{\pgfqpoint{4.225000in}{4.225000in}}%
\pgfusepath{clip}%
\pgfsetbuttcap%
\pgfsetroundjoin%
\definecolor{currentfill}{rgb}{0.804182,0.882046,0.114965}%
\pgfsetfillcolor{currentfill}%
\pgfsetfillopacity{0.700000}%
\pgfsetlinewidth{0.501875pt}%
\definecolor{currentstroke}{rgb}{1.000000,1.000000,1.000000}%
\pgfsetstrokecolor{currentstroke}%
\pgfsetstrokeopacity{0.500000}%
\pgfsetdash{}{0pt}%
\pgfpathmoveto{\pgfqpoint{3.338323in}{3.785387in}}%
\pgfpathlineto{\pgfqpoint{3.349202in}{3.791006in}}%
\pgfpathlineto{\pgfqpoint{3.360077in}{3.796527in}}%
\pgfpathlineto{\pgfqpoint{3.370947in}{3.801932in}}%
\pgfpathlineto{\pgfqpoint{3.381812in}{3.807203in}}%
\pgfpathlineto{\pgfqpoint{3.392672in}{3.812310in}}%
\pgfpathlineto{\pgfqpoint{3.386518in}{3.812617in}}%
\pgfpathlineto{\pgfqpoint{3.380370in}{3.812861in}}%
\pgfpathlineto{\pgfqpoint{3.374228in}{3.813047in}}%
\pgfpathlineto{\pgfqpoint{3.368091in}{3.813178in}}%
\pgfpathlineto{\pgfqpoint{3.361961in}{3.813258in}}%
\pgfpathlineto{\pgfqpoint{3.351126in}{3.808677in}}%
\pgfpathlineto{\pgfqpoint{3.340286in}{3.803896in}}%
\pgfpathlineto{\pgfqpoint{3.329441in}{3.798941in}}%
\pgfpathlineto{\pgfqpoint{3.318590in}{3.793833in}}%
\pgfpathlineto{\pgfqpoint{3.307735in}{3.788592in}}%
\pgfpathlineto{\pgfqpoint{3.313841in}{3.788025in}}%
\pgfpathlineto{\pgfqpoint{3.319953in}{3.787422in}}%
\pgfpathlineto{\pgfqpoint{3.326071in}{3.786782in}}%
\pgfpathlineto{\pgfqpoint{3.332194in}{3.786104in}}%
\pgfpathclose%
\pgfusepath{stroke,fill}%
\end{pgfscope}%
\begin{pgfscope}%
\pgfpathrectangle{\pgfqpoint{0.887500in}{0.275000in}}{\pgfqpoint{4.225000in}{4.225000in}}%
\pgfusepath{clip}%
\pgfsetbuttcap%
\pgfsetroundjoin%
\definecolor{currentfill}{rgb}{0.699415,0.867117,0.175971}%
\pgfsetfillcolor{currentfill}%
\pgfsetfillopacity{0.700000}%
\pgfsetlinewidth{0.501875pt}%
\definecolor{currentstroke}{rgb}{1.000000,1.000000,1.000000}%
\pgfsetstrokecolor{currentstroke}%
\pgfsetstrokeopacity{0.500000}%
\pgfsetdash{}{0pt}%
\pgfpathmoveto{\pgfqpoint{3.144422in}{3.702367in}}%
\pgfpathlineto{\pgfqpoint{3.155336in}{3.708410in}}%
\pgfpathlineto{\pgfqpoint{3.166246in}{3.714432in}}%
\pgfpathlineto{\pgfqpoint{3.177153in}{3.720425in}}%
\pgfpathlineto{\pgfqpoint{3.188056in}{3.726380in}}%
\pgfpathlineto{\pgfqpoint{3.198955in}{3.732289in}}%
\pgfpathlineto{\pgfqpoint{3.192900in}{3.733340in}}%
\pgfpathlineto{\pgfqpoint{3.186850in}{3.734352in}}%
\pgfpathlineto{\pgfqpoint{3.180806in}{3.735320in}}%
\pgfpathlineto{\pgfqpoint{3.174768in}{3.736240in}}%
\pgfpathlineto{\pgfqpoint{3.168736in}{3.737108in}}%
\pgfpathlineto{\pgfqpoint{3.157858in}{3.731410in}}%
\pgfpathlineto{\pgfqpoint{3.146977in}{3.725654in}}%
\pgfpathlineto{\pgfqpoint{3.136092in}{3.719850in}}%
\pgfpathlineto{\pgfqpoint{3.125203in}{3.714011in}}%
\pgfpathlineto{\pgfqpoint{3.114311in}{3.708151in}}%
\pgfpathlineto{\pgfqpoint{3.120322in}{3.707067in}}%
\pgfpathlineto{\pgfqpoint{3.126338in}{3.705942in}}%
\pgfpathlineto{\pgfqpoint{3.132361in}{3.704782in}}%
\pgfpathlineto{\pgfqpoint{3.138389in}{3.703589in}}%
\pgfpathclose%
\pgfusepath{stroke,fill}%
\end{pgfscope}%
\begin{pgfscope}%
\pgfpathrectangle{\pgfqpoint{0.887500in}{0.275000in}}{\pgfqpoint{4.225000in}{4.225000in}}%
\pgfusepath{clip}%
\pgfsetbuttcap%
\pgfsetroundjoin%
\definecolor{currentfill}{rgb}{0.866013,0.889868,0.095953}%
\pgfsetfillcolor{currentfill}%
\pgfsetfillopacity{0.700000}%
\pgfsetlinewidth{0.501875pt}%
\definecolor{currentstroke}{rgb}{1.000000,1.000000,1.000000}%
\pgfsetstrokecolor{currentstroke}%
\pgfsetstrokeopacity{0.500000}%
\pgfsetdash{}{0pt}%
\pgfpathmoveto{\pgfqpoint{3.477849in}{3.833491in}}%
\pgfpathlineto{\pgfqpoint{3.488693in}{3.837376in}}%
\pgfpathlineto{\pgfqpoint{3.499528in}{3.840919in}}%
\pgfpathlineto{\pgfqpoint{3.510355in}{3.844121in}}%
\pgfpathlineto{\pgfqpoint{3.521173in}{3.846979in}}%
\pgfpathlineto{\pgfqpoint{3.531982in}{3.849493in}}%
\pgfpathlineto{\pgfqpoint{3.525746in}{3.848854in}}%
\pgfpathlineto{\pgfqpoint{3.519515in}{3.848108in}}%
\pgfpathlineto{\pgfqpoint{3.513290in}{3.847277in}}%
\pgfpathlineto{\pgfqpoint{3.507071in}{3.846386in}}%
\pgfpathlineto{\pgfqpoint{3.500858in}{3.845459in}}%
\pgfpathlineto{\pgfqpoint{3.490081in}{3.844018in}}%
\pgfpathlineto{\pgfqpoint{3.479294in}{3.842135in}}%
\pgfpathlineto{\pgfqpoint{3.468496in}{3.839811in}}%
\pgfpathlineto{\pgfqpoint{3.457689in}{3.837046in}}%
\pgfpathlineto{\pgfqpoint{3.446873in}{3.833842in}}%
\pgfpathlineto{\pgfqpoint{3.453058in}{3.833982in}}%
\pgfpathlineto{\pgfqpoint{3.459248in}{3.834025in}}%
\pgfpathlineto{\pgfqpoint{3.465443in}{3.833962in}}%
\pgfpathlineto{\pgfqpoint{3.471644in}{3.833787in}}%
\pgfpathclose%
\pgfusepath{stroke,fill}%
\end{pgfscope}%
\begin{pgfscope}%
\pgfpathrectangle{\pgfqpoint{0.887500in}{0.275000in}}{\pgfqpoint{4.225000in}{4.225000in}}%
\pgfusepath{clip}%
\pgfsetbuttcap%
\pgfsetroundjoin%
\definecolor{currentfill}{rgb}{0.595839,0.848717,0.243329}%
\pgfsetfillcolor{currentfill}%
\pgfsetfillopacity{0.700000}%
\pgfsetlinewidth{0.501875pt}%
\definecolor{currentstroke}{rgb}{1.000000,1.000000,1.000000}%
\pgfsetstrokecolor{currentstroke}%
\pgfsetstrokeopacity{0.500000}%
\pgfsetdash{}{0pt}%
\pgfpathmoveto{\pgfqpoint{2.950485in}{3.624146in}}%
\pgfpathlineto{\pgfqpoint{2.961434in}{3.629583in}}%
\pgfpathlineto{\pgfqpoint{2.972378in}{3.634954in}}%
\pgfpathlineto{\pgfqpoint{2.983319in}{3.640291in}}%
\pgfpathlineto{\pgfqpoint{2.994256in}{3.645627in}}%
\pgfpathlineto{\pgfqpoint{3.005189in}{3.650995in}}%
\pgfpathlineto{\pgfqpoint{2.999226in}{3.652500in}}%
\pgfpathlineto{\pgfqpoint{2.993269in}{3.653993in}}%
\pgfpathlineto{\pgfqpoint{2.987317in}{3.655467in}}%
\pgfpathlineto{\pgfqpoint{2.981371in}{3.656917in}}%
\pgfpathlineto{\pgfqpoint{2.970454in}{3.651708in}}%
\pgfpathlineto{\pgfqpoint{2.959533in}{3.646521in}}%
\pgfpathlineto{\pgfqpoint{2.948608in}{3.641329in}}%
\pgfpathlineto{\pgfqpoint{2.937679in}{3.636105in}}%
\pgfpathlineto{\pgfqpoint{2.926747in}{3.630821in}}%
\pgfpathlineto{\pgfqpoint{2.932673in}{3.629113in}}%
\pgfpathlineto{\pgfqpoint{2.938605in}{3.627430in}}%
\pgfpathlineto{\pgfqpoint{2.944542in}{3.625774in}}%
\pgfpathclose%
\pgfusepath{stroke,fill}%
\end{pgfscope}%
\begin{pgfscope}%
\pgfpathrectangle{\pgfqpoint{0.887500in}{0.275000in}}{\pgfqpoint{4.225000in}{4.225000in}}%
\pgfusepath{clip}%
\pgfsetbuttcap%
\pgfsetroundjoin%
\definecolor{currentfill}{rgb}{0.906311,0.894855,0.098125}%
\pgfsetfillcolor{currentfill}%
\pgfsetfillopacity{0.700000}%
\pgfsetlinewidth{0.501875pt}%
\definecolor{currentstroke}{rgb}{1.000000,1.000000,1.000000}%
\pgfsetstrokecolor{currentstroke}%
\pgfsetstrokeopacity{0.500000}%
\pgfsetdash{}{0pt}%
\pgfpathmoveto{\pgfqpoint{3.617330in}{3.865476in}}%
\pgfpathlineto{\pgfqpoint{3.628126in}{3.867897in}}%
\pgfpathlineto{\pgfqpoint{3.638913in}{3.870103in}}%
\pgfpathlineto{\pgfqpoint{3.649693in}{3.872094in}}%
\pgfpathlineto{\pgfqpoint{3.660463in}{3.873867in}}%
\pgfpathlineto{\pgfqpoint{3.654124in}{3.870905in}}%
\pgfpathlineto{\pgfqpoint{3.647788in}{3.867616in}}%
\pgfpathlineto{\pgfqpoint{3.641455in}{3.864060in}}%
\pgfpathlineto{\pgfqpoint{3.635126in}{3.860299in}}%
\pgfpathlineto{\pgfqpoint{3.628804in}{3.856416in}}%
\pgfpathlineto{\pgfqpoint{3.618090in}{3.857056in}}%
\pgfpathlineto{\pgfqpoint{3.607364in}{3.857341in}}%
\pgfpathlineto{\pgfqpoint{3.596627in}{3.857272in}}%
\pgfpathlineto{\pgfqpoint{3.585879in}{3.856850in}}%
\pgfpathlineto{\pgfqpoint{3.592161in}{3.858966in}}%
\pgfpathlineto{\pgfqpoint{3.598447in}{3.860934in}}%
\pgfpathlineto{\pgfqpoint{3.604738in}{3.862702in}}%
\pgfpathlineto{\pgfqpoint{3.611032in}{3.864229in}}%
\pgfpathclose%
\pgfusepath{stroke,fill}%
\end{pgfscope}%
\begin{pgfscope}%
\pgfpathrectangle{\pgfqpoint{0.887500in}{0.275000in}}{\pgfqpoint{4.225000in}{4.225000in}}%
\pgfusepath{clip}%
\pgfsetbuttcap%
\pgfsetroundjoin%
\definecolor{currentfill}{rgb}{0.783315,0.879285,0.125405}%
\pgfsetfillcolor{currentfill}%
\pgfsetfillopacity{0.700000}%
\pgfsetlinewidth{0.501875pt}%
\definecolor{currentstroke}{rgb}{1.000000,1.000000,1.000000}%
\pgfsetstrokecolor{currentstroke}%
\pgfsetstrokeopacity{0.500000}%
\pgfsetdash{}{0pt}%
\pgfpathmoveto{\pgfqpoint{3.283868in}{3.756383in}}%
\pgfpathlineto{\pgfqpoint{3.294766in}{3.762256in}}%
\pgfpathlineto{\pgfqpoint{3.305662in}{3.768104in}}%
\pgfpathlineto{\pgfqpoint{3.316553in}{3.773919in}}%
\pgfpathlineto{\pgfqpoint{3.327440in}{3.779686in}}%
\pgfpathlineto{\pgfqpoint{3.338323in}{3.785387in}}%
\pgfpathlineto{\pgfqpoint{3.332194in}{3.786104in}}%
\pgfpathlineto{\pgfqpoint{3.326071in}{3.786782in}}%
\pgfpathlineto{\pgfqpoint{3.319953in}{3.787422in}}%
\pgfpathlineto{\pgfqpoint{3.313841in}{3.788025in}}%
\pgfpathlineto{\pgfqpoint{3.307735in}{3.788592in}}%
\pgfpathlineto{\pgfqpoint{3.296875in}{3.783239in}}%
\pgfpathlineto{\pgfqpoint{3.286011in}{3.777793in}}%
\pgfpathlineto{\pgfqpoint{3.275143in}{3.772274in}}%
\pgfpathlineto{\pgfqpoint{3.264271in}{3.766701in}}%
\pgfpathlineto{\pgfqpoint{3.253394in}{3.761084in}}%
\pgfpathlineto{\pgfqpoint{3.259478in}{3.760220in}}%
\pgfpathlineto{\pgfqpoint{3.265567in}{3.759315in}}%
\pgfpathlineto{\pgfqpoint{3.271661in}{3.758372in}}%
\pgfpathlineto{\pgfqpoint{3.277762in}{3.757393in}}%
\pgfpathclose%
\pgfusepath{stroke,fill}%
\end{pgfscope}%
\begin{pgfscope}%
\pgfpathrectangle{\pgfqpoint{0.887500in}{0.275000in}}{\pgfqpoint{4.225000in}{4.225000in}}%
\pgfusepath{clip}%
\pgfsetbuttcap%
\pgfsetroundjoin%
\definecolor{currentfill}{rgb}{0.668054,0.861999,0.196293}%
\pgfsetfillcolor{currentfill}%
\pgfsetfillopacity{0.700000}%
\pgfsetlinewidth{0.501875pt}%
\definecolor{currentstroke}{rgb}{1.000000,1.000000,1.000000}%
\pgfsetstrokecolor{currentstroke}%
\pgfsetstrokeopacity{0.500000}%
\pgfsetdash{}{0pt}%
\pgfpathmoveto{\pgfqpoint{3.089800in}{3.672175in}}%
\pgfpathlineto{\pgfqpoint{3.100732in}{3.678167in}}%
\pgfpathlineto{\pgfqpoint{3.111660in}{3.684199in}}%
\pgfpathlineto{\pgfqpoint{3.122584in}{3.690253in}}%
\pgfpathlineto{\pgfqpoint{3.133505in}{3.696312in}}%
\pgfpathlineto{\pgfqpoint{3.144422in}{3.702367in}}%
\pgfpathlineto{\pgfqpoint{3.138389in}{3.703589in}}%
\pgfpathlineto{\pgfqpoint{3.132361in}{3.704782in}}%
\pgfpathlineto{\pgfqpoint{3.126338in}{3.705942in}}%
\pgfpathlineto{\pgfqpoint{3.120322in}{3.707067in}}%
\pgfpathlineto{\pgfqpoint{3.114311in}{3.708151in}}%
\pgfpathlineto{\pgfqpoint{3.103415in}{3.702282in}}%
\pgfpathlineto{\pgfqpoint{3.092516in}{3.696416in}}%
\pgfpathlineto{\pgfqpoint{3.081613in}{3.690565in}}%
\pgfpathlineto{\pgfqpoint{3.070706in}{3.684742in}}%
\pgfpathlineto{\pgfqpoint{3.059796in}{3.678957in}}%
\pgfpathlineto{\pgfqpoint{3.065786in}{3.677638in}}%
\pgfpathlineto{\pgfqpoint{3.071781in}{3.676296in}}%
\pgfpathlineto{\pgfqpoint{3.077782in}{3.674934in}}%
\pgfpathlineto{\pgfqpoint{3.083788in}{3.673558in}}%
\pgfpathclose%
\pgfusepath{stroke,fill}%
\end{pgfscope}%
\begin{pgfscope}%
\pgfpathrectangle{\pgfqpoint{0.887500in}{0.275000in}}{\pgfqpoint{4.225000in}{4.225000in}}%
\pgfusepath{clip}%
\pgfsetbuttcap%
\pgfsetroundjoin%
\definecolor{currentfill}{rgb}{0.575563,0.844566,0.256415}%
\pgfsetfillcolor{currentfill}%
\pgfsetfillopacity{0.700000}%
\pgfsetlinewidth{0.501875pt}%
\definecolor{currentstroke}{rgb}{1.000000,1.000000,1.000000}%
\pgfsetstrokecolor{currentstroke}%
\pgfsetstrokeopacity{0.500000}%
\pgfsetdash{}{0pt}%
\pgfpathmoveto{\pgfqpoint{2.895689in}{3.595933in}}%
\pgfpathlineto{\pgfqpoint{2.906656in}{3.601564in}}%
\pgfpathlineto{\pgfqpoint{2.917619in}{3.607262in}}%
\pgfpathlineto{\pgfqpoint{2.928578in}{3.612965in}}%
\pgfpathlineto{\pgfqpoint{2.939533in}{3.618611in}}%
\pgfpathlineto{\pgfqpoint{2.950485in}{3.624146in}}%
\pgfpathlineto{\pgfqpoint{2.944542in}{3.625774in}}%
\pgfpathlineto{\pgfqpoint{2.938605in}{3.627430in}}%
\pgfpathlineto{\pgfqpoint{2.932673in}{3.629113in}}%
\pgfpathlineto{\pgfqpoint{2.926747in}{3.630821in}}%
\pgfpathlineto{\pgfqpoint{2.915810in}{3.625450in}}%
\pgfpathlineto{\pgfqpoint{2.904871in}{3.619978in}}%
\pgfpathlineto{\pgfqpoint{2.893927in}{3.614431in}}%
\pgfpathlineto{\pgfqpoint{2.882981in}{3.608843in}}%
\pgfpathlineto{\pgfqpoint{2.872030in}{3.603247in}}%
\pgfpathlineto{\pgfqpoint{2.877937in}{3.601342in}}%
\pgfpathlineto{\pgfqpoint{2.883849in}{3.599488in}}%
\pgfpathlineto{\pgfqpoint{2.889766in}{3.597686in}}%
\pgfpathclose%
\pgfusepath{stroke,fill}%
\end{pgfscope}%
\begin{pgfscope}%
\pgfpathrectangle{\pgfqpoint{0.887500in}{0.275000in}}{\pgfqpoint{4.225000in}{4.225000in}}%
\pgfusepath{clip}%
\pgfsetbuttcap%
\pgfsetroundjoin%
\definecolor{currentfill}{rgb}{0.866013,0.889868,0.095953}%
\pgfsetfillcolor{currentfill}%
\pgfsetfillopacity{0.700000}%
\pgfsetlinewidth{0.501875pt}%
\definecolor{currentstroke}{rgb}{1.000000,1.000000,1.000000}%
\pgfsetstrokecolor{currentstroke}%
\pgfsetstrokeopacity{0.500000}%
\pgfsetdash{}{0pt}%
\pgfpathmoveto{\pgfqpoint{3.423522in}{3.809683in}}%
\pgfpathlineto{\pgfqpoint{3.434400in}{3.814933in}}%
\pgfpathlineto{\pgfqpoint{3.445273in}{3.819968in}}%
\pgfpathlineto{\pgfqpoint{3.456139in}{3.824759in}}%
\pgfpathlineto{\pgfqpoint{3.466998in}{3.829276in}}%
\pgfpathlineto{\pgfqpoint{3.477849in}{3.833491in}}%
\pgfpathlineto{\pgfqpoint{3.471644in}{3.833787in}}%
\pgfpathlineto{\pgfqpoint{3.465443in}{3.833962in}}%
\pgfpathlineto{\pgfqpoint{3.459248in}{3.834025in}}%
\pgfpathlineto{\pgfqpoint{3.453058in}{3.833982in}}%
\pgfpathlineto{\pgfqpoint{3.446873in}{3.833842in}}%
\pgfpathlineto{\pgfqpoint{3.436048in}{3.830215in}}%
\pgfpathlineto{\pgfqpoint{3.425214in}{3.826204in}}%
\pgfpathlineto{\pgfqpoint{3.414373in}{3.821854in}}%
\pgfpathlineto{\pgfqpoint{3.403526in}{3.817208in}}%
\pgfpathlineto{\pgfqpoint{3.392672in}{3.812310in}}%
\pgfpathlineto{\pgfqpoint{3.398831in}{3.811935in}}%
\pgfpathlineto{\pgfqpoint{3.404995in}{3.811490in}}%
\pgfpathlineto{\pgfqpoint{3.411166in}{3.810969in}}%
\pgfpathlineto{\pgfqpoint{3.417341in}{3.810368in}}%
\pgfpathclose%
\pgfusepath{stroke,fill}%
\end{pgfscope}%
\begin{pgfscope}%
\pgfpathrectangle{\pgfqpoint{0.887500in}{0.275000in}}{\pgfqpoint{4.225000in}{4.225000in}}%
\pgfusepath{clip}%
\pgfsetbuttcap%
\pgfsetroundjoin%
\definecolor{currentfill}{rgb}{0.751884,0.874951,0.143228}%
\pgfsetfillcolor{currentfill}%
\pgfsetfillopacity{0.700000}%
\pgfsetlinewidth{0.501875pt}%
\definecolor{currentstroke}{rgb}{1.000000,1.000000,1.000000}%
\pgfsetstrokecolor{currentstroke}%
\pgfsetstrokeopacity{0.500000}%
\pgfsetdash{}{0pt}%
\pgfpathmoveto{\pgfqpoint{3.229318in}{3.726560in}}%
\pgfpathlineto{\pgfqpoint{3.240235in}{3.732592in}}%
\pgfpathlineto{\pgfqpoint{3.251149in}{3.738588in}}%
\pgfpathlineto{\pgfqpoint{3.262059in}{3.744551in}}%
\pgfpathlineto{\pgfqpoint{3.272965in}{3.750482in}}%
\pgfpathlineto{\pgfqpoint{3.283868in}{3.756383in}}%
\pgfpathlineto{\pgfqpoint{3.277762in}{3.757393in}}%
\pgfpathlineto{\pgfqpoint{3.271661in}{3.758372in}}%
\pgfpathlineto{\pgfqpoint{3.265567in}{3.759315in}}%
\pgfpathlineto{\pgfqpoint{3.259478in}{3.760220in}}%
\pgfpathlineto{\pgfqpoint{3.253394in}{3.761084in}}%
\pgfpathlineto{\pgfqpoint{3.242514in}{3.755421in}}%
\pgfpathlineto{\pgfqpoint{3.231630in}{3.749711in}}%
\pgfpathlineto{\pgfqpoint{3.220742in}{3.743954in}}%
\pgfpathlineto{\pgfqpoint{3.209851in}{3.738146in}}%
\pgfpathlineto{\pgfqpoint{3.198955in}{3.732289in}}%
\pgfpathlineto{\pgfqpoint{3.205016in}{3.731201in}}%
\pgfpathlineto{\pgfqpoint{3.211083in}{3.730080in}}%
\pgfpathlineto{\pgfqpoint{3.217155in}{3.728930in}}%
\pgfpathlineto{\pgfqpoint{3.223234in}{3.727756in}}%
\pgfpathclose%
\pgfusepath{stroke,fill}%
\end{pgfscope}%
\begin{pgfscope}%
\pgfpathrectangle{\pgfqpoint{0.887500in}{0.275000in}}{\pgfqpoint{4.225000in}{4.225000in}}%
\pgfusepath{clip}%
\pgfsetbuttcap%
\pgfsetroundjoin%
\definecolor{currentfill}{rgb}{0.647257,0.858400,0.209861}%
\pgfsetfillcolor{currentfill}%
\pgfsetfillopacity{0.700000}%
\pgfsetlinewidth{0.501875pt}%
\definecolor{currentstroke}{rgb}{1.000000,1.000000,1.000000}%
\pgfsetstrokecolor{currentstroke}%
\pgfsetstrokeopacity{0.500000}%
\pgfsetdash{}{0pt}%
\pgfpathmoveto{\pgfqpoint{3.035090in}{3.643480in}}%
\pgfpathlineto{\pgfqpoint{3.046039in}{3.648994in}}%
\pgfpathlineto{\pgfqpoint{3.056985in}{3.654633in}}%
\pgfpathlineto{\pgfqpoint{3.067927in}{3.660389in}}%
\pgfpathlineto{\pgfqpoint{3.078866in}{3.666243in}}%
\pgfpathlineto{\pgfqpoint{3.089800in}{3.672175in}}%
\pgfpathlineto{\pgfqpoint{3.083788in}{3.673558in}}%
\pgfpathlineto{\pgfqpoint{3.077782in}{3.674934in}}%
\pgfpathlineto{\pgfqpoint{3.071781in}{3.676296in}}%
\pgfpathlineto{\pgfqpoint{3.065786in}{3.677638in}}%
\pgfpathlineto{\pgfqpoint{3.059796in}{3.678957in}}%
\pgfpathlineto{\pgfqpoint{3.048882in}{3.673222in}}%
\pgfpathlineto{\pgfqpoint{3.037965in}{3.667547in}}%
\pgfpathlineto{\pgfqpoint{3.027044in}{3.661945in}}%
\pgfpathlineto{\pgfqpoint{3.016118in}{3.656426in}}%
\pgfpathlineto{\pgfqpoint{3.005189in}{3.650995in}}%
\pgfpathlineto{\pgfqpoint{3.011158in}{3.649482in}}%
\pgfpathlineto{\pgfqpoint{3.017133in}{3.647969in}}%
\pgfpathlineto{\pgfqpoint{3.023113in}{3.646460in}}%
\pgfpathlineto{\pgfqpoint{3.029098in}{3.644962in}}%
\pgfpathclose%
\pgfusepath{stroke,fill}%
\end{pgfscope}%
\begin{pgfscope}%
\pgfpathrectangle{\pgfqpoint{0.887500in}{0.275000in}}{\pgfqpoint{4.225000in}{4.225000in}}%
\pgfusepath{clip}%
\pgfsetbuttcap%
\pgfsetroundjoin%
\definecolor{currentfill}{rgb}{0.906311,0.894855,0.098125}%
\pgfsetfillcolor{currentfill}%
\pgfsetfillopacity{0.700000}%
\pgfsetlinewidth{0.501875pt}%
\definecolor{currentstroke}{rgb}{1.000000,1.000000,1.000000}%
\pgfsetstrokecolor{currentstroke}%
\pgfsetstrokeopacity{0.500000}%
\pgfsetdash{}{0pt}%
\pgfpathmoveto{\pgfqpoint{3.563229in}{3.850222in}}%
\pgfpathlineto{\pgfqpoint{3.574065in}{3.853687in}}%
\pgfpathlineto{\pgfqpoint{3.584893in}{3.856947in}}%
\pgfpathlineto{\pgfqpoint{3.595713in}{3.859999in}}%
\pgfpathlineto{\pgfqpoint{3.606525in}{3.862843in}}%
\pgfpathlineto{\pgfqpoint{3.617330in}{3.865476in}}%
\pgfpathlineto{\pgfqpoint{3.611032in}{3.864229in}}%
\pgfpathlineto{\pgfqpoint{3.604738in}{3.862702in}}%
\pgfpathlineto{\pgfqpoint{3.598447in}{3.860934in}}%
\pgfpathlineto{\pgfqpoint{3.592161in}{3.858966in}}%
\pgfpathlineto{\pgfqpoint{3.585879in}{3.856850in}}%
\pgfpathlineto{\pgfqpoint{3.575120in}{3.856077in}}%
\pgfpathlineto{\pgfqpoint{3.564351in}{3.854954in}}%
\pgfpathlineto{\pgfqpoint{3.553571in}{3.853481in}}%
\pgfpathlineto{\pgfqpoint{3.542781in}{3.851661in}}%
\pgfpathlineto{\pgfqpoint{3.531982in}{3.849493in}}%
\pgfpathlineto{\pgfqpoint{3.538223in}{3.849999in}}%
\pgfpathlineto{\pgfqpoint{3.544469in}{3.850348in}}%
\pgfpathlineto{\pgfqpoint{3.550718in}{3.850517in}}%
\pgfpathlineto{\pgfqpoint{3.556972in}{3.850483in}}%
\pgfpathclose%
\pgfusepath{stroke,fill}%
\end{pgfscope}%
\begin{pgfscope}%
\pgfpathrectangle{\pgfqpoint{0.887500in}{0.275000in}}{\pgfqpoint{4.225000in}{4.225000in}}%
\pgfusepath{clip}%
\pgfsetbuttcap%
\pgfsetroundjoin%
\definecolor{currentfill}{rgb}{0.555484,0.840254,0.269281}%
\pgfsetfillcolor{currentfill}%
\pgfsetfillopacity{0.700000}%
\pgfsetlinewidth{0.501875pt}%
\definecolor{currentstroke}{rgb}{1.000000,1.000000,1.000000}%
\pgfsetstrokecolor{currentstroke}%
\pgfsetstrokeopacity{0.500000}%
\pgfsetdash{}{0pt}%
\pgfpathmoveto{\pgfqpoint{2.840784in}{3.570406in}}%
\pgfpathlineto{\pgfqpoint{2.851775in}{3.575249in}}%
\pgfpathlineto{\pgfqpoint{2.862761in}{3.580079in}}%
\pgfpathlineto{\pgfqpoint{2.873742in}{3.585128in}}%
\pgfpathlineto{\pgfqpoint{2.884718in}{3.590434in}}%
\pgfpathlineto{\pgfqpoint{2.895689in}{3.595933in}}%
\pgfpathlineto{\pgfqpoint{2.889766in}{3.597686in}}%
\pgfpathlineto{\pgfqpoint{2.883849in}{3.599488in}}%
\pgfpathlineto{\pgfqpoint{2.877937in}{3.601342in}}%
\pgfpathlineto{\pgfqpoint{2.872030in}{3.603247in}}%
\pgfpathlineto{\pgfqpoint{2.861076in}{3.597676in}}%
\pgfpathlineto{\pgfqpoint{2.850117in}{3.592163in}}%
\pgfpathlineto{\pgfqpoint{2.839154in}{3.586742in}}%
\pgfpathlineto{\pgfqpoint{2.828187in}{3.581383in}}%
\pgfpathlineto{\pgfqpoint{2.817216in}{3.575933in}}%
\pgfpathlineto{\pgfqpoint{2.823100in}{3.574585in}}%
\pgfpathlineto{\pgfqpoint{2.828989in}{3.573211in}}%
\pgfpathlineto{\pgfqpoint{2.834884in}{3.571816in}}%
\pgfpathclose%
\pgfusepath{stroke,fill}%
\end{pgfscope}%
\begin{pgfscope}%
\pgfpathrectangle{\pgfqpoint{0.887500in}{0.275000in}}{\pgfqpoint{4.225000in}{4.225000in}}%
\pgfusepath{clip}%
\pgfsetbuttcap%
\pgfsetroundjoin%
\definecolor{currentfill}{rgb}{0.845561,0.887322,0.099702}%
\pgfsetfillcolor{currentfill}%
\pgfsetfillopacity{0.700000}%
\pgfsetlinewidth{0.501875pt}%
\definecolor{currentstroke}{rgb}{1.000000,1.000000,1.000000}%
\pgfsetstrokecolor{currentstroke}%
\pgfsetstrokeopacity{0.500000}%
\pgfsetdash{}{0pt}%
\pgfpathmoveto{\pgfqpoint{3.369055in}{3.781180in}}%
\pgfpathlineto{\pgfqpoint{3.379958in}{3.787102in}}%
\pgfpathlineto{\pgfqpoint{3.390856in}{3.792934in}}%
\pgfpathlineto{\pgfqpoint{3.401750in}{3.798656in}}%
\pgfpathlineto{\pgfqpoint{3.412638in}{3.804248in}}%
\pgfpathlineto{\pgfqpoint{3.423522in}{3.809683in}}%
\pgfpathlineto{\pgfqpoint{3.417341in}{3.810368in}}%
\pgfpathlineto{\pgfqpoint{3.411166in}{3.810969in}}%
\pgfpathlineto{\pgfqpoint{3.404995in}{3.811490in}}%
\pgfpathlineto{\pgfqpoint{3.398831in}{3.811935in}}%
\pgfpathlineto{\pgfqpoint{3.392672in}{3.812310in}}%
\pgfpathlineto{\pgfqpoint{3.381812in}{3.807203in}}%
\pgfpathlineto{\pgfqpoint{3.370947in}{3.801932in}}%
\pgfpathlineto{\pgfqpoint{3.360077in}{3.796527in}}%
\pgfpathlineto{\pgfqpoint{3.349202in}{3.791006in}}%
\pgfpathlineto{\pgfqpoint{3.338323in}{3.785387in}}%
\pgfpathlineto{\pgfqpoint{3.344458in}{3.784630in}}%
\pgfpathlineto{\pgfqpoint{3.350599in}{3.783832in}}%
\pgfpathlineto{\pgfqpoint{3.356745in}{3.782992in}}%
\pgfpathlineto{\pgfqpoint{3.362897in}{3.782108in}}%
\pgfpathclose%
\pgfusepath{stroke,fill}%
\end{pgfscope}%
\begin{pgfscope}%
\pgfpathrectangle{\pgfqpoint{0.887500in}{0.275000in}}{\pgfqpoint{4.225000in}{4.225000in}}%
\pgfusepath{clip}%
\pgfsetbuttcap%
\pgfsetroundjoin%
\definecolor{currentfill}{rgb}{0.730889,0.871916,0.156029}%
\pgfsetfillcolor{currentfill}%
\pgfsetfillopacity{0.700000}%
\pgfsetlinewidth{0.501875pt}%
\definecolor{currentstroke}{rgb}{1.000000,1.000000,1.000000}%
\pgfsetstrokecolor{currentstroke}%
\pgfsetstrokeopacity{0.500000}%
\pgfsetdash{}{0pt}%
\pgfpathmoveto{\pgfqpoint{3.174677in}{3.695957in}}%
\pgfpathlineto{\pgfqpoint{3.185612in}{3.702114in}}%
\pgfpathlineto{\pgfqpoint{3.196544in}{3.708260in}}%
\pgfpathlineto{\pgfqpoint{3.207472in}{3.714388in}}%
\pgfpathlineto{\pgfqpoint{3.218397in}{3.720491in}}%
\pgfpathlineto{\pgfqpoint{3.229318in}{3.726560in}}%
\pgfpathlineto{\pgfqpoint{3.223234in}{3.727756in}}%
\pgfpathlineto{\pgfqpoint{3.217155in}{3.728930in}}%
\pgfpathlineto{\pgfqpoint{3.211083in}{3.730080in}}%
\pgfpathlineto{\pgfqpoint{3.205016in}{3.731201in}}%
\pgfpathlineto{\pgfqpoint{3.198955in}{3.732289in}}%
\pgfpathlineto{\pgfqpoint{3.188056in}{3.726380in}}%
\pgfpathlineto{\pgfqpoint{3.177153in}{3.720425in}}%
\pgfpathlineto{\pgfqpoint{3.166246in}{3.714432in}}%
\pgfpathlineto{\pgfqpoint{3.155336in}{3.708410in}}%
\pgfpathlineto{\pgfqpoint{3.144422in}{3.702367in}}%
\pgfpathlineto{\pgfqpoint{3.150462in}{3.701121in}}%
\pgfpathlineto{\pgfqpoint{3.156507in}{3.699853in}}%
\pgfpathlineto{\pgfqpoint{3.162558in}{3.698567in}}%
\pgfpathlineto{\pgfqpoint{3.168614in}{3.697267in}}%
\pgfpathclose%
\pgfusepath{stroke,fill}%
\end{pgfscope}%
\begin{pgfscope}%
\pgfpathrectangle{\pgfqpoint{0.887500in}{0.275000in}}{\pgfqpoint{4.225000in}{4.225000in}}%
\pgfusepath{clip}%
\pgfsetbuttcap%
\pgfsetroundjoin%
\definecolor{currentfill}{rgb}{0.616293,0.852709,0.230052}%
\pgfsetfillcolor{currentfill}%
\pgfsetfillopacity{0.700000}%
\pgfsetlinewidth{0.501875pt}%
\definecolor{currentstroke}{rgb}{1.000000,1.000000,1.000000}%
\pgfsetstrokecolor{currentstroke}%
\pgfsetstrokeopacity{0.500000}%
\pgfsetdash{}{0pt}%
\pgfpathmoveto{\pgfqpoint{2.980283in}{3.616468in}}%
\pgfpathlineto{\pgfqpoint{2.991252in}{3.621941in}}%
\pgfpathlineto{\pgfqpoint{3.002217in}{3.627335in}}%
\pgfpathlineto{\pgfqpoint{3.013179in}{3.632694in}}%
\pgfpathlineto{\pgfqpoint{3.024136in}{3.638062in}}%
\pgfpathlineto{\pgfqpoint{3.035090in}{3.643480in}}%
\pgfpathlineto{\pgfqpoint{3.029098in}{3.644962in}}%
\pgfpathlineto{\pgfqpoint{3.023113in}{3.646460in}}%
\pgfpathlineto{\pgfqpoint{3.017133in}{3.647969in}}%
\pgfpathlineto{\pgfqpoint{3.011158in}{3.649482in}}%
\pgfpathlineto{\pgfqpoint{3.005189in}{3.650995in}}%
\pgfpathlineto{\pgfqpoint{2.994256in}{3.645627in}}%
\pgfpathlineto{\pgfqpoint{2.983319in}{3.640291in}}%
\pgfpathlineto{\pgfqpoint{2.972378in}{3.634954in}}%
\pgfpathlineto{\pgfqpoint{2.961434in}{3.629583in}}%
\pgfpathlineto{\pgfqpoint{2.950485in}{3.624146in}}%
\pgfpathlineto{\pgfqpoint{2.956434in}{3.622548in}}%
\pgfpathlineto{\pgfqpoint{2.962388in}{3.620979in}}%
\pgfpathlineto{\pgfqpoint{2.968347in}{3.619443in}}%
\pgfpathlineto{\pgfqpoint{2.974312in}{3.617939in}}%
\pgfpathclose%
\pgfusepath{stroke,fill}%
\end{pgfscope}%
\begin{pgfscope}%
\pgfpathrectangle{\pgfqpoint{0.887500in}{0.275000in}}{\pgfqpoint{4.225000in}{4.225000in}}%
\pgfusepath{clip}%
\pgfsetbuttcap%
\pgfsetroundjoin%
\definecolor{currentfill}{rgb}{0.906311,0.894855,0.098125}%
\pgfsetfillcolor{currentfill}%
\pgfsetfillopacity{0.700000}%
\pgfsetlinewidth{0.501875pt}%
\definecolor{currentstroke}{rgb}{1.000000,1.000000,1.000000}%
\pgfsetstrokecolor{currentstroke}%
\pgfsetstrokeopacity{0.500000}%
\pgfsetdash{}{0pt}%
\pgfpathmoveto{\pgfqpoint{3.508947in}{3.829877in}}%
\pgfpathlineto{\pgfqpoint{3.519817in}{3.834342in}}%
\pgfpathlineto{\pgfqpoint{3.530680in}{3.838611in}}%
\pgfpathlineto{\pgfqpoint{3.541537in}{3.842682in}}%
\pgfpathlineto{\pgfqpoint{3.552387in}{3.846553in}}%
\pgfpathlineto{\pgfqpoint{3.563229in}{3.850222in}}%
\pgfpathlineto{\pgfqpoint{3.556972in}{3.850483in}}%
\pgfpathlineto{\pgfqpoint{3.550718in}{3.850517in}}%
\pgfpathlineto{\pgfqpoint{3.544469in}{3.850348in}}%
\pgfpathlineto{\pgfqpoint{3.538223in}{3.849999in}}%
\pgfpathlineto{\pgfqpoint{3.531982in}{3.849493in}}%
\pgfpathlineto{\pgfqpoint{3.521173in}{3.846979in}}%
\pgfpathlineto{\pgfqpoint{3.510355in}{3.844121in}}%
\pgfpathlineto{\pgfqpoint{3.499528in}{3.840919in}}%
\pgfpathlineto{\pgfqpoint{3.488693in}{3.837376in}}%
\pgfpathlineto{\pgfqpoint{3.477849in}{3.833491in}}%
\pgfpathlineto{\pgfqpoint{3.484060in}{3.833066in}}%
\pgfpathlineto{\pgfqpoint{3.490275in}{3.832505in}}%
\pgfpathlineto{\pgfqpoint{3.496494in}{3.831797in}}%
\pgfpathlineto{\pgfqpoint{3.502718in}{3.830927in}}%
\pgfpathclose%
\pgfusepath{stroke,fill}%
\end{pgfscope}%
\begin{pgfscope}%
\pgfpathrectangle{\pgfqpoint{0.887500in}{0.275000in}}{\pgfqpoint{4.225000in}{4.225000in}}%
\pgfusepath{clip}%
\pgfsetbuttcap%
\pgfsetroundjoin%
\definecolor{currentfill}{rgb}{0.525776,0.833491,0.288127}%
\pgfsetfillcolor{currentfill}%
\pgfsetfillopacity{0.700000}%
\pgfsetlinewidth{0.501875pt}%
\definecolor{currentstroke}{rgb}{1.000000,1.000000,1.000000}%
\pgfsetstrokecolor{currentstroke}%
\pgfsetstrokeopacity{0.500000}%
\pgfsetdash{}{0pt}%
\pgfpathmoveto{\pgfqpoint{2.785829in}{3.537028in}}%
\pgfpathlineto{\pgfqpoint{2.796814in}{3.545777in}}%
\pgfpathlineto{\pgfqpoint{2.807804in}{3.553236in}}%
\pgfpathlineto{\pgfqpoint{2.818797in}{3.559656in}}%
\pgfpathlineto{\pgfqpoint{2.829791in}{3.565294in}}%
\pgfpathlineto{\pgfqpoint{2.840784in}{3.570406in}}%
\pgfpathlineto{\pgfqpoint{2.834884in}{3.571816in}}%
\pgfpathlineto{\pgfqpoint{2.828989in}{3.573211in}}%
\pgfpathlineto{\pgfqpoint{2.823100in}{3.574585in}}%
\pgfpathlineto{\pgfqpoint{2.817216in}{3.575933in}}%
\pgfpathlineto{\pgfqpoint{2.806243in}{3.570224in}}%
\pgfpathlineto{\pgfqpoint{2.795270in}{3.564086in}}%
\pgfpathlineto{\pgfqpoint{2.784296in}{3.557353in}}%
\pgfpathlineto{\pgfqpoint{2.773325in}{3.549855in}}%
\pgfpathlineto{\pgfqpoint{2.762358in}{3.541429in}}%
\pgfpathlineto{\pgfqpoint{2.768215in}{3.540610in}}%
\pgfpathlineto{\pgfqpoint{2.774080in}{3.539585in}}%
\pgfpathlineto{\pgfqpoint{2.779951in}{3.538382in}}%
\pgfpathclose%
\pgfusepath{stroke,fill}%
\end{pgfscope}%
\begin{pgfscope}%
\pgfpathrectangle{\pgfqpoint{0.887500in}{0.275000in}}{\pgfqpoint{4.225000in}{4.225000in}}%
\pgfusepath{clip}%
\pgfsetbuttcap%
\pgfsetroundjoin%
\definecolor{currentfill}{rgb}{0.814576,0.883393,0.110347}%
\pgfsetfillcolor{currentfill}%
\pgfsetfillopacity{0.700000}%
\pgfsetlinewidth{0.501875pt}%
\definecolor{currentstroke}{rgb}{1.000000,1.000000,1.000000}%
\pgfsetstrokecolor{currentstroke}%
\pgfsetstrokeopacity{0.500000}%
\pgfsetdash{}{0pt}%
\pgfpathmoveto{\pgfqpoint{3.314485in}{3.750945in}}%
\pgfpathlineto{\pgfqpoint{3.325406in}{3.757017in}}%
\pgfpathlineto{\pgfqpoint{3.336324in}{3.763089in}}%
\pgfpathlineto{\pgfqpoint{3.347238in}{3.769153in}}%
\pgfpathlineto{\pgfqpoint{3.358148in}{3.775191in}}%
\pgfpathlineto{\pgfqpoint{3.369055in}{3.781180in}}%
\pgfpathlineto{\pgfqpoint{3.362897in}{3.782108in}}%
\pgfpathlineto{\pgfqpoint{3.356745in}{3.782992in}}%
\pgfpathlineto{\pgfqpoint{3.350599in}{3.783832in}}%
\pgfpathlineto{\pgfqpoint{3.344458in}{3.784630in}}%
\pgfpathlineto{\pgfqpoint{3.338323in}{3.785387in}}%
\pgfpathlineto{\pgfqpoint{3.327440in}{3.779686in}}%
\pgfpathlineto{\pgfqpoint{3.316553in}{3.773919in}}%
\pgfpathlineto{\pgfqpoint{3.305662in}{3.768104in}}%
\pgfpathlineto{\pgfqpoint{3.294766in}{3.762256in}}%
\pgfpathlineto{\pgfqpoint{3.283868in}{3.756383in}}%
\pgfpathlineto{\pgfqpoint{3.289980in}{3.755343in}}%
\pgfpathlineto{\pgfqpoint{3.296097in}{3.754276in}}%
\pgfpathlineto{\pgfqpoint{3.302221in}{3.753186in}}%
\pgfpathlineto{\pgfqpoint{3.308350in}{3.752074in}}%
\pgfpathclose%
\pgfusepath{stroke,fill}%
\end{pgfscope}%
\begin{pgfscope}%
\pgfpathrectangle{\pgfqpoint{0.887500in}{0.275000in}}{\pgfqpoint{4.225000in}{4.225000in}}%
\pgfusepath{clip}%
\pgfsetbuttcap%
\pgfsetroundjoin%
\definecolor{currentfill}{rgb}{0.699415,0.867117,0.175971}%
\pgfsetfillcolor{currentfill}%
\pgfsetfillopacity{0.700000}%
\pgfsetlinewidth{0.501875pt}%
\definecolor{currentstroke}{rgb}{1.000000,1.000000,1.000000}%
\pgfsetstrokecolor{currentstroke}%
\pgfsetstrokeopacity{0.500000}%
\pgfsetdash{}{0pt}%
\pgfpathmoveto{\pgfqpoint{3.119948in}{3.665333in}}%
\pgfpathlineto{\pgfqpoint{3.130901in}{3.671392in}}%
\pgfpathlineto{\pgfqpoint{3.141850in}{3.677502in}}%
\pgfpathlineto{\pgfqpoint{3.152795in}{3.683643in}}%
\pgfpathlineto{\pgfqpoint{3.163738in}{3.689797in}}%
\pgfpathlineto{\pgfqpoint{3.174677in}{3.695957in}}%
\pgfpathlineto{\pgfqpoint{3.168614in}{3.697267in}}%
\pgfpathlineto{\pgfqpoint{3.162558in}{3.698567in}}%
\pgfpathlineto{\pgfqpoint{3.156507in}{3.699853in}}%
\pgfpathlineto{\pgfqpoint{3.150462in}{3.701121in}}%
\pgfpathlineto{\pgfqpoint{3.144422in}{3.702367in}}%
\pgfpathlineto{\pgfqpoint{3.133505in}{3.696312in}}%
\pgfpathlineto{\pgfqpoint{3.122584in}{3.690253in}}%
\pgfpathlineto{\pgfqpoint{3.111660in}{3.684199in}}%
\pgfpathlineto{\pgfqpoint{3.100732in}{3.678167in}}%
\pgfpathlineto{\pgfqpoint{3.089800in}{3.672175in}}%
\pgfpathlineto{\pgfqpoint{3.095818in}{3.670789in}}%
\pgfpathlineto{\pgfqpoint{3.101842in}{3.669407in}}%
\pgfpathlineto{\pgfqpoint{3.107872in}{3.668033in}}%
\pgfpathlineto{\pgfqpoint{3.113907in}{3.666673in}}%
\pgfpathclose%
\pgfusepath{stroke,fill}%
\end{pgfscope}%
\begin{pgfscope}%
\pgfpathrectangle{\pgfqpoint{0.887500in}{0.275000in}}{\pgfqpoint{4.225000in}{4.225000in}}%
\pgfusepath{clip}%
\pgfsetbuttcap%
\pgfsetroundjoin%
\definecolor{currentfill}{rgb}{0.955300,0.901065,0.118128}%
\pgfsetfillcolor{currentfill}%
\pgfsetfillopacity{0.700000}%
\pgfsetlinewidth{0.501875pt}%
\definecolor{currentstroke}{rgb}{1.000000,1.000000,1.000000}%
\pgfsetstrokecolor{currentstroke}%
\pgfsetstrokeopacity{0.500000}%
\pgfsetdash{}{0pt}%
\pgfpathmoveto{\pgfqpoint{3.648838in}{3.866170in}}%
\pgfpathlineto{\pgfqpoint{3.659677in}{3.870204in}}%
\pgfpathlineto{\pgfqpoint{3.670511in}{3.874148in}}%
\pgfpathlineto{\pgfqpoint{3.681339in}{3.877999in}}%
\pgfpathlineto{\pgfqpoint{3.692160in}{3.881757in}}%
\pgfpathlineto{\pgfqpoint{3.685824in}{3.881296in}}%
\pgfpathlineto{\pgfqpoint{3.679485in}{3.880218in}}%
\pgfpathlineto{\pgfqpoint{3.673145in}{3.878582in}}%
\pgfpathlineto{\pgfqpoint{3.666804in}{3.876445in}}%
\pgfpathlineto{\pgfqpoint{3.660463in}{3.873867in}}%
\pgfpathlineto{\pgfqpoint{3.649693in}{3.872094in}}%
\pgfpathlineto{\pgfqpoint{3.638913in}{3.870103in}}%
\pgfpathlineto{\pgfqpoint{3.628126in}{3.867897in}}%
\pgfpathlineto{\pgfqpoint{3.617330in}{3.865476in}}%
\pgfpathlineto{\pgfqpoint{3.623629in}{3.866405in}}%
\pgfpathlineto{\pgfqpoint{3.629931in}{3.866978in}}%
\pgfpathlineto{\pgfqpoint{3.636233in}{3.867156in}}%
\pgfpathlineto{\pgfqpoint{3.642536in}{3.866899in}}%
\pgfpathclose%
\pgfusepath{stroke,fill}%
\end{pgfscope}%
\begin{pgfscope}%
\pgfpathrectangle{\pgfqpoint{0.887500in}{0.275000in}}{\pgfqpoint{4.225000in}{4.225000in}}%
\pgfusepath{clip}%
\pgfsetbuttcap%
\pgfsetroundjoin%
\definecolor{currentfill}{rgb}{0.259857,0.745492,0.444467}%
\pgfsetfillcolor{currentfill}%
\pgfsetfillopacity{0.700000}%
\pgfsetlinewidth{0.501875pt}%
\definecolor{currentstroke}{rgb}{1.000000,1.000000,1.000000}%
\pgfsetstrokecolor{currentstroke}%
\pgfsetstrokeopacity{0.500000}%
\pgfsetdash{}{0pt}%
\pgfpathmoveto{\pgfqpoint{2.566888in}{3.264073in}}%
\pgfpathlineto{\pgfqpoint{2.577855in}{3.275621in}}%
\pgfpathlineto{\pgfqpoint{2.588798in}{3.289177in}}%
\pgfpathlineto{\pgfqpoint{2.599724in}{3.304223in}}%
\pgfpathlineto{\pgfqpoint{2.610641in}{3.320240in}}%
\pgfpathlineto{\pgfqpoint{2.621556in}{3.336709in}}%
\pgfpathlineto{\pgfqpoint{2.615684in}{3.344903in}}%
\pgfpathlineto{\pgfqpoint{2.609811in}{3.353519in}}%
\pgfpathlineto{\pgfqpoint{2.603938in}{3.362505in}}%
\pgfpathlineto{\pgfqpoint{2.598064in}{3.371808in}}%
\pgfpathlineto{\pgfqpoint{2.587119in}{3.359346in}}%
\pgfpathlineto{\pgfqpoint{2.576170in}{3.347197in}}%
\pgfpathlineto{\pgfqpoint{2.565213in}{3.335632in}}%
\pgfpathlineto{\pgfqpoint{2.554246in}{3.324922in}}%
\pgfpathlineto{\pgfqpoint{2.543263in}{3.315338in}}%
\pgfpathlineto{\pgfqpoint{2.549189in}{3.300644in}}%
\pgfpathlineto{\pgfqpoint{2.555102in}{3.287219in}}%
\pgfpathlineto{\pgfqpoint{2.561001in}{3.275036in}}%
\pgfpathclose%
\pgfusepath{stroke,fill}%
\end{pgfscope}%
\begin{pgfscope}%
\pgfpathrectangle{\pgfqpoint{0.887500in}{0.275000in}}{\pgfqpoint{4.225000in}{4.225000in}}%
\pgfusepath{clip}%
\pgfsetbuttcap%
\pgfsetroundjoin%
\definecolor{currentfill}{rgb}{0.595839,0.848717,0.243329}%
\pgfsetfillcolor{currentfill}%
\pgfsetfillopacity{0.700000}%
\pgfsetlinewidth{0.501875pt}%
\definecolor{currentstroke}{rgb}{1.000000,1.000000,1.000000}%
\pgfsetstrokecolor{currentstroke}%
\pgfsetstrokeopacity{0.500000}%
\pgfsetdash{}{0pt}%
\pgfpathmoveto{\pgfqpoint{2.925383in}{3.587907in}}%
\pgfpathlineto{\pgfqpoint{2.936371in}{3.593576in}}%
\pgfpathlineto{\pgfqpoint{2.947354in}{3.599351in}}%
\pgfpathlineto{\pgfqpoint{2.958334in}{3.605146in}}%
\pgfpathlineto{\pgfqpoint{2.969310in}{3.610875in}}%
\pgfpathlineto{\pgfqpoint{2.980283in}{3.616468in}}%
\pgfpathlineto{\pgfqpoint{2.974312in}{3.617939in}}%
\pgfpathlineto{\pgfqpoint{2.968347in}{3.619443in}}%
\pgfpathlineto{\pgfqpoint{2.962388in}{3.620979in}}%
\pgfpathlineto{\pgfqpoint{2.956434in}{3.622548in}}%
\pgfpathlineto{\pgfqpoint{2.950485in}{3.624146in}}%
\pgfpathlineto{\pgfqpoint{2.939533in}{3.618611in}}%
\pgfpathlineto{\pgfqpoint{2.928578in}{3.612965in}}%
\pgfpathlineto{\pgfqpoint{2.917619in}{3.607262in}}%
\pgfpathlineto{\pgfqpoint{2.906656in}{3.601564in}}%
\pgfpathlineto{\pgfqpoint{2.895689in}{3.595933in}}%
\pgfpathlineto{\pgfqpoint{2.901617in}{3.594231in}}%
\pgfpathlineto{\pgfqpoint{2.907550in}{3.592578in}}%
\pgfpathlineto{\pgfqpoint{2.913489in}{3.590973in}}%
\pgfpathlineto{\pgfqpoint{2.919434in}{3.589416in}}%
\pgfpathclose%
\pgfusepath{stroke,fill}%
\end{pgfscope}%
\begin{pgfscope}%
\pgfpathrectangle{\pgfqpoint{0.887500in}{0.275000in}}{\pgfqpoint{4.225000in}{4.225000in}}%
\pgfusepath{clip}%
\pgfsetbuttcap%
\pgfsetroundjoin%
\definecolor{currentfill}{rgb}{0.477504,0.821444,0.318195}%
\pgfsetfillcolor{currentfill}%
\pgfsetfillopacity{0.700000}%
\pgfsetlinewidth{0.501875pt}%
\definecolor{currentstroke}{rgb}{1.000000,1.000000,1.000000}%
\pgfsetstrokecolor{currentstroke}%
\pgfsetstrokeopacity{0.500000}%
\pgfsetdash{}{0pt}%
\pgfpathmoveto{\pgfqpoint{2.731001in}{3.477810in}}%
\pgfpathlineto{\pgfqpoint{2.741958in}{3.490923in}}%
\pgfpathlineto{\pgfqpoint{2.752918in}{3.503644in}}%
\pgfpathlineto{\pgfqpoint{2.763882in}{3.515729in}}%
\pgfpathlineto{\pgfqpoint{2.774853in}{3.526938in}}%
\pgfpathlineto{\pgfqpoint{2.785829in}{3.537028in}}%
\pgfpathlineto{\pgfqpoint{2.779951in}{3.538382in}}%
\pgfpathlineto{\pgfqpoint{2.774080in}{3.539585in}}%
\pgfpathlineto{\pgfqpoint{2.768215in}{3.540610in}}%
\pgfpathlineto{\pgfqpoint{2.762358in}{3.541429in}}%
\pgfpathlineto{\pgfqpoint{2.751395in}{3.532048in}}%
\pgfpathlineto{\pgfqpoint{2.740438in}{3.521876in}}%
\pgfpathlineto{\pgfqpoint{2.729483in}{3.511093in}}%
\pgfpathlineto{\pgfqpoint{2.718531in}{3.499877in}}%
\pgfpathlineto{\pgfqpoint{2.707581in}{3.488405in}}%
\pgfpathlineto{\pgfqpoint{2.713424in}{3.486255in}}%
\pgfpathlineto{\pgfqpoint{2.719276in}{3.483726in}}%
\pgfpathlineto{\pgfqpoint{2.725135in}{3.480889in}}%
\pgfpathclose%
\pgfusepath{stroke,fill}%
\end{pgfscope}%
\begin{pgfscope}%
\pgfpathrectangle{\pgfqpoint{0.887500in}{0.275000in}}{\pgfqpoint{4.225000in}{4.225000in}}%
\pgfusepath{clip}%
\pgfsetbuttcap%
\pgfsetroundjoin%
\definecolor{currentfill}{rgb}{0.335885,0.777018,0.402049}%
\pgfsetfillcolor{currentfill}%
\pgfsetfillopacity{0.700000}%
\pgfsetlinewidth{0.501875pt}%
\definecolor{currentstroke}{rgb}{1.000000,1.000000,1.000000}%
\pgfsetstrokecolor{currentstroke}%
\pgfsetstrokeopacity{0.500000}%
\pgfsetdash{}{0pt}%
\pgfpathmoveto{\pgfqpoint{2.621556in}{3.336709in}}%
\pgfpathlineto{\pgfqpoint{2.632472in}{3.353107in}}%
\pgfpathlineto{\pgfqpoint{2.643398in}{3.368918in}}%
\pgfpathlineto{\pgfqpoint{2.654333in}{3.383885in}}%
\pgfpathlineto{\pgfqpoint{2.665276in}{3.398140in}}%
\pgfpathlineto{\pgfqpoint{2.676225in}{3.411849in}}%
\pgfpathlineto{\pgfqpoint{2.670362in}{3.417132in}}%
\pgfpathlineto{\pgfqpoint{2.664506in}{3.422235in}}%
\pgfpathlineto{\pgfqpoint{2.658656in}{3.427065in}}%
\pgfpathlineto{\pgfqpoint{2.652815in}{3.431530in}}%
\pgfpathlineto{\pgfqpoint{2.641860in}{3.420116in}}%
\pgfpathlineto{\pgfqpoint{2.630907in}{3.408497in}}%
\pgfpathlineto{\pgfqpoint{2.619956in}{3.396585in}}%
\pgfpathlineto{\pgfqpoint{2.609009in}{3.384311in}}%
\pgfpathlineto{\pgfqpoint{2.598064in}{3.371808in}}%
\pgfpathlineto{\pgfqpoint{2.603938in}{3.362505in}}%
\pgfpathlineto{\pgfqpoint{2.609811in}{3.353519in}}%
\pgfpathlineto{\pgfqpoint{2.615684in}{3.344903in}}%
\pgfpathclose%
\pgfusepath{stroke,fill}%
\end{pgfscope}%
\begin{pgfscope}%
\pgfpathrectangle{\pgfqpoint{0.887500in}{0.275000in}}{\pgfqpoint{4.225000in}{4.225000in}}%
\pgfusepath{clip}%
\pgfsetbuttcap%
\pgfsetroundjoin%
\definecolor{currentfill}{rgb}{0.404001,0.800275,0.362552}%
\pgfsetfillcolor{currentfill}%
\pgfsetfillopacity{0.700000}%
\pgfsetlinewidth{0.501875pt}%
\definecolor{currentstroke}{rgb}{1.000000,1.000000,1.000000}%
\pgfsetstrokecolor{currentstroke}%
\pgfsetstrokeopacity{0.500000}%
\pgfsetdash{}{0pt}%
\pgfpathmoveto{\pgfqpoint{2.676225in}{3.411849in}}%
\pgfpathlineto{\pgfqpoint{2.687179in}{3.425178in}}%
\pgfpathlineto{\pgfqpoint{2.698134in}{3.438294in}}%
\pgfpathlineto{\pgfqpoint{2.709090in}{3.451364in}}%
\pgfpathlineto{\pgfqpoint{2.720046in}{3.464547in}}%
\pgfpathlineto{\pgfqpoint{2.731001in}{3.477810in}}%
\pgfpathlineto{\pgfqpoint{2.725135in}{3.480889in}}%
\pgfpathlineto{\pgfqpoint{2.719276in}{3.483726in}}%
\pgfpathlineto{\pgfqpoint{2.713424in}{3.486255in}}%
\pgfpathlineto{\pgfqpoint{2.707581in}{3.488405in}}%
\pgfpathlineto{\pgfqpoint{2.696631in}{3.476857in}}%
\pgfpathlineto{\pgfqpoint{2.685679in}{3.465403in}}%
\pgfpathlineto{\pgfqpoint{2.674725in}{3.454087in}}%
\pgfpathlineto{\pgfqpoint{2.663771in}{3.442824in}}%
\pgfpathlineto{\pgfqpoint{2.652815in}{3.431530in}}%
\pgfpathlineto{\pgfqpoint{2.658656in}{3.427065in}}%
\pgfpathlineto{\pgfqpoint{2.664506in}{3.422235in}}%
\pgfpathlineto{\pgfqpoint{2.670362in}{3.417132in}}%
\pgfpathclose%
\pgfusepath{stroke,fill}%
\end{pgfscope}%
\begin{pgfscope}%
\pgfpathrectangle{\pgfqpoint{0.887500in}{0.275000in}}{\pgfqpoint{4.225000in}{4.225000in}}%
\pgfusepath{clip}%
\pgfsetbuttcap%
\pgfsetroundjoin%
\definecolor{currentfill}{rgb}{0.226397,0.728888,0.462789}%
\pgfsetfillcolor{currentfill}%
\pgfsetfillopacity{0.700000}%
\pgfsetlinewidth{0.501875pt}%
\definecolor{currentstroke}{rgb}{1.000000,1.000000,1.000000}%
\pgfsetstrokecolor{currentstroke}%
\pgfsetstrokeopacity{0.500000}%
\pgfsetdash{}{0pt}%
\pgfpathmoveto{\pgfqpoint{2.511638in}{3.236443in}}%
\pgfpathlineto{\pgfqpoint{2.522731in}{3.239334in}}%
\pgfpathlineto{\pgfqpoint{2.533807in}{3.243071in}}%
\pgfpathlineto{\pgfqpoint{2.544862in}{3.248127in}}%
\pgfpathlineto{\pgfqpoint{2.555890in}{3.254971in}}%
\pgfpathlineto{\pgfqpoint{2.566888in}{3.264073in}}%
\pgfpathlineto{\pgfqpoint{2.561001in}{3.275036in}}%
\pgfpathlineto{\pgfqpoint{2.555102in}{3.287219in}}%
\pgfpathlineto{\pgfqpoint{2.549189in}{3.300644in}}%
\pgfpathlineto{\pgfqpoint{2.543263in}{3.315338in}}%
\pgfpathlineto{\pgfqpoint{2.532260in}{3.307107in}}%
\pgfpathlineto{\pgfqpoint{2.521239in}{3.300139in}}%
\pgfpathlineto{\pgfqpoint{2.510200in}{3.294194in}}%
\pgfpathlineto{\pgfqpoint{2.499147in}{3.289033in}}%
\pgfpathlineto{\pgfqpoint{2.488083in}{3.284418in}}%
\pgfpathlineto{\pgfqpoint{2.493989in}{3.270925in}}%
\pgfpathlineto{\pgfqpoint{2.499883in}{3.258456in}}%
\pgfpathlineto{\pgfqpoint{2.505765in}{3.246974in}}%
\pgfpathclose%
\pgfusepath{stroke,fill}%
\end{pgfscope}%
\begin{pgfscope}%
\pgfpathrectangle{\pgfqpoint{0.887500in}{0.275000in}}{\pgfqpoint{4.225000in}{4.225000in}}%
\pgfusepath{clip}%
\pgfsetbuttcap%
\pgfsetroundjoin%
\definecolor{currentfill}{rgb}{0.896320,0.893616,0.096335}%
\pgfsetfillcolor{currentfill}%
\pgfsetfillopacity{0.700000}%
\pgfsetlinewidth{0.501875pt}%
\definecolor{currentstroke}{rgb}{1.000000,1.000000,1.000000}%
\pgfsetstrokecolor{currentstroke}%
\pgfsetstrokeopacity{0.500000}%
\pgfsetdash{}{0pt}%
\pgfpathmoveto{\pgfqpoint{3.454504in}{3.804773in}}%
\pgfpathlineto{\pgfqpoint{3.465404in}{3.810143in}}%
\pgfpathlineto{\pgfqpoint{3.476299in}{3.815345in}}%
\pgfpathlineto{\pgfqpoint{3.487187in}{3.820372in}}%
\pgfpathlineto{\pgfqpoint{3.498070in}{3.825218in}}%
\pgfpathlineto{\pgfqpoint{3.508947in}{3.829877in}}%
\pgfpathlineto{\pgfqpoint{3.502718in}{3.830927in}}%
\pgfpathlineto{\pgfqpoint{3.496494in}{3.831797in}}%
\pgfpathlineto{\pgfqpoint{3.490275in}{3.832505in}}%
\pgfpathlineto{\pgfqpoint{3.484060in}{3.833066in}}%
\pgfpathlineto{\pgfqpoint{3.477849in}{3.833491in}}%
\pgfpathlineto{\pgfqpoint{3.466998in}{3.829276in}}%
\pgfpathlineto{\pgfqpoint{3.456139in}{3.824759in}}%
\pgfpathlineto{\pgfqpoint{3.445273in}{3.819968in}}%
\pgfpathlineto{\pgfqpoint{3.434400in}{3.814933in}}%
\pgfpathlineto{\pgfqpoint{3.423522in}{3.809683in}}%
\pgfpathlineto{\pgfqpoint{3.429708in}{3.808910in}}%
\pgfpathlineto{\pgfqpoint{3.435900in}{3.808045in}}%
\pgfpathlineto{\pgfqpoint{3.442096in}{3.807079in}}%
\pgfpathlineto{\pgfqpoint{3.448298in}{3.805996in}}%
\pgfpathclose%
\pgfusepath{stroke,fill}%
\end{pgfscope}%
\begin{pgfscope}%
\pgfpathrectangle{\pgfqpoint{0.887500in}{0.275000in}}{\pgfqpoint{4.225000in}{4.225000in}}%
\pgfusepath{clip}%
\pgfsetbuttcap%
\pgfsetroundjoin%
\definecolor{currentfill}{rgb}{0.793760,0.880678,0.120005}%
\pgfsetfillcolor{currentfill}%
\pgfsetfillopacity{0.700000}%
\pgfsetlinewidth{0.501875pt}%
\definecolor{currentstroke}{rgb}{1.000000,1.000000,1.000000}%
\pgfsetstrokecolor{currentstroke}%
\pgfsetstrokeopacity{0.500000}%
\pgfsetdash{}{0pt}%
\pgfpathmoveto{\pgfqpoint{3.259825in}{3.720397in}}%
\pgfpathlineto{\pgfqpoint{3.270764in}{3.726549in}}%
\pgfpathlineto{\pgfqpoint{3.281700in}{3.732675in}}%
\pgfpathlineto{\pgfqpoint{3.292632in}{3.738779in}}%
\pgfpathlineto{\pgfqpoint{3.303560in}{3.744867in}}%
\pgfpathlineto{\pgfqpoint{3.314485in}{3.750945in}}%
\pgfpathlineto{\pgfqpoint{3.308350in}{3.752074in}}%
\pgfpathlineto{\pgfqpoint{3.302221in}{3.753186in}}%
\pgfpathlineto{\pgfqpoint{3.296097in}{3.754276in}}%
\pgfpathlineto{\pgfqpoint{3.289980in}{3.755343in}}%
\pgfpathlineto{\pgfqpoint{3.283868in}{3.756383in}}%
\pgfpathlineto{\pgfqpoint{3.272965in}{3.750482in}}%
\pgfpathlineto{\pgfqpoint{3.262059in}{3.744551in}}%
\pgfpathlineto{\pgfqpoint{3.251149in}{3.738588in}}%
\pgfpathlineto{\pgfqpoint{3.240235in}{3.732592in}}%
\pgfpathlineto{\pgfqpoint{3.229318in}{3.726560in}}%
\pgfpathlineto{\pgfqpoint{3.235408in}{3.725347in}}%
\pgfpathlineto{\pgfqpoint{3.241503in}{3.724120in}}%
\pgfpathlineto{\pgfqpoint{3.247604in}{3.722884in}}%
\pgfpathlineto{\pgfqpoint{3.253712in}{3.721642in}}%
\pgfpathclose%
\pgfusepath{stroke,fill}%
\end{pgfscope}%
\begin{pgfscope}%
\pgfpathrectangle{\pgfqpoint{0.887500in}{0.275000in}}{\pgfqpoint{4.225000in}{4.225000in}}%
\pgfusepath{clip}%
\pgfsetbuttcap%
\pgfsetroundjoin%
\definecolor{currentfill}{rgb}{0.668054,0.861999,0.196293}%
\pgfsetfillcolor{currentfill}%
\pgfsetfillopacity{0.700000}%
\pgfsetlinewidth{0.501875pt}%
\definecolor{currentstroke}{rgb}{1.000000,1.000000,1.000000}%
\pgfsetstrokecolor{currentstroke}%
\pgfsetstrokeopacity{0.500000}%
\pgfsetdash{}{0pt}%
\pgfpathmoveto{\pgfqpoint{3.065132in}{3.636525in}}%
\pgfpathlineto{\pgfqpoint{3.076102in}{3.642027in}}%
\pgfpathlineto{\pgfqpoint{3.087069in}{3.647673in}}%
\pgfpathlineto{\pgfqpoint{3.098032in}{3.653453in}}%
\pgfpathlineto{\pgfqpoint{3.108992in}{3.659347in}}%
\pgfpathlineto{\pgfqpoint{3.119948in}{3.665333in}}%
\pgfpathlineto{\pgfqpoint{3.113907in}{3.666673in}}%
\pgfpathlineto{\pgfqpoint{3.107872in}{3.668033in}}%
\pgfpathlineto{\pgfqpoint{3.101842in}{3.669407in}}%
\pgfpathlineto{\pgfqpoint{3.095818in}{3.670789in}}%
\pgfpathlineto{\pgfqpoint{3.089800in}{3.672175in}}%
\pgfpathlineto{\pgfqpoint{3.078866in}{3.666243in}}%
\pgfpathlineto{\pgfqpoint{3.067927in}{3.660389in}}%
\pgfpathlineto{\pgfqpoint{3.056985in}{3.654633in}}%
\pgfpathlineto{\pgfqpoint{3.046039in}{3.648994in}}%
\pgfpathlineto{\pgfqpoint{3.035090in}{3.643480in}}%
\pgfpathlineto{\pgfqpoint{3.041087in}{3.642021in}}%
\pgfpathlineto{\pgfqpoint{3.047089in}{3.640590in}}%
\pgfpathlineto{\pgfqpoint{3.053098in}{3.639193in}}%
\pgfpathlineto{\pgfqpoint{3.059112in}{3.637836in}}%
\pgfpathclose%
\pgfusepath{stroke,fill}%
\end{pgfscope}%
\begin{pgfscope}%
\pgfpathrectangle{\pgfqpoint{0.887500in}{0.275000in}}{\pgfqpoint{4.225000in}{4.225000in}}%
\pgfusepath{clip}%
\pgfsetbuttcap%
\pgfsetroundjoin%
\definecolor{currentfill}{rgb}{0.955300,0.901065,0.118128}%
\pgfsetfillcolor{currentfill}%
\pgfsetfillopacity{0.700000}%
\pgfsetlinewidth{0.501875pt}%
\definecolor{currentstroke}{rgb}{1.000000,1.000000,1.000000}%
\pgfsetstrokecolor{currentstroke}%
\pgfsetstrokeopacity{0.500000}%
\pgfsetdash{}{0pt}%
\pgfpathmoveto{\pgfqpoint{3.594555in}{3.844716in}}%
\pgfpathlineto{\pgfqpoint{3.605423in}{3.849171in}}%
\pgfpathlineto{\pgfqpoint{3.616285in}{3.853546in}}%
\pgfpathlineto{\pgfqpoint{3.627141in}{3.857839in}}%
\pgfpathlineto{\pgfqpoint{3.637993in}{3.862048in}}%
\pgfpathlineto{\pgfqpoint{3.648838in}{3.866170in}}%
\pgfpathlineto{\pgfqpoint{3.642536in}{3.866899in}}%
\pgfpathlineto{\pgfqpoint{3.636233in}{3.867156in}}%
\pgfpathlineto{\pgfqpoint{3.629931in}{3.866978in}}%
\pgfpathlineto{\pgfqpoint{3.623629in}{3.866405in}}%
\pgfpathlineto{\pgfqpoint{3.617330in}{3.865476in}}%
\pgfpathlineto{\pgfqpoint{3.606525in}{3.862843in}}%
\pgfpathlineto{\pgfqpoint{3.595713in}{3.859999in}}%
\pgfpathlineto{\pgfqpoint{3.584893in}{3.856947in}}%
\pgfpathlineto{\pgfqpoint{3.574065in}{3.853687in}}%
\pgfpathlineto{\pgfqpoint{3.563229in}{3.850222in}}%
\pgfpathlineto{\pgfqpoint{3.569490in}{3.849711in}}%
\pgfpathlineto{\pgfqpoint{3.575753in}{3.848928in}}%
\pgfpathlineto{\pgfqpoint{3.582019in}{3.847850in}}%
\pgfpathlineto{\pgfqpoint{3.588286in}{3.846453in}}%
\pgfpathclose%
\pgfusepath{stroke,fill}%
\end{pgfscope}%
\begin{pgfscope}%
\pgfpathrectangle{\pgfqpoint{0.887500in}{0.275000in}}{\pgfqpoint{4.225000in}{4.225000in}}%
\pgfusepath{clip}%
\pgfsetbuttcap%
\pgfsetroundjoin%
\definecolor{currentfill}{rgb}{0.575563,0.844566,0.256415}%
\pgfsetfillcolor{currentfill}%
\pgfsetfillopacity{0.700000}%
\pgfsetlinewidth{0.501875pt}%
\definecolor{currentstroke}{rgb}{1.000000,1.000000,1.000000}%
\pgfsetstrokecolor{currentstroke}%
\pgfsetstrokeopacity{0.500000}%
\pgfsetdash{}{0pt}%
\pgfpathmoveto{\pgfqpoint{2.870371in}{3.563330in}}%
\pgfpathlineto{\pgfqpoint{2.881384in}{3.567852in}}%
\pgfpathlineto{\pgfqpoint{2.892392in}{3.572381in}}%
\pgfpathlineto{\pgfqpoint{2.903394in}{3.577224in}}%
\pgfpathlineto{\pgfqpoint{2.914391in}{3.582427in}}%
\pgfpathlineto{\pgfqpoint{2.925383in}{3.587907in}}%
\pgfpathlineto{\pgfqpoint{2.919434in}{3.589416in}}%
\pgfpathlineto{\pgfqpoint{2.913489in}{3.590973in}}%
\pgfpathlineto{\pgfqpoint{2.907550in}{3.592578in}}%
\pgfpathlineto{\pgfqpoint{2.901617in}{3.594231in}}%
\pgfpathlineto{\pgfqpoint{2.895689in}{3.595933in}}%
\pgfpathlineto{\pgfqpoint{2.884718in}{3.590434in}}%
\pgfpathlineto{\pgfqpoint{2.873742in}{3.585128in}}%
\pgfpathlineto{\pgfqpoint{2.862761in}{3.580079in}}%
\pgfpathlineto{\pgfqpoint{2.851775in}{3.575249in}}%
\pgfpathlineto{\pgfqpoint{2.840784in}{3.570406in}}%
\pgfpathlineto{\pgfqpoint{2.846690in}{3.568987in}}%
\pgfpathlineto{\pgfqpoint{2.852602in}{3.567565in}}%
\pgfpathlineto{\pgfqpoint{2.858520in}{3.566144in}}%
\pgfpathlineto{\pgfqpoint{2.864443in}{3.564730in}}%
\pgfpathclose%
\pgfusepath{stroke,fill}%
\end{pgfscope}%
\begin{pgfscope}%
\pgfpathrectangle{\pgfqpoint{0.887500in}{0.275000in}}{\pgfqpoint{4.225000in}{4.225000in}}%
\pgfusepath{clip}%
\pgfsetbuttcap%
\pgfsetroundjoin%
\definecolor{currentfill}{rgb}{0.239374,0.735588,0.455688}%
\pgfsetfillcolor{currentfill}%
\pgfsetfillopacity{0.700000}%
\pgfsetlinewidth{0.501875pt}%
\definecolor{currentstroke}{rgb}{1.000000,1.000000,1.000000}%
\pgfsetstrokecolor{currentstroke}%
\pgfsetstrokeopacity{0.500000}%
\pgfsetdash{}{0pt}%
\pgfpathmoveto{\pgfqpoint{2.596166in}{3.226712in}}%
\pgfpathlineto{\pgfqpoint{2.607142in}{3.238982in}}%
\pgfpathlineto{\pgfqpoint{2.618095in}{3.253358in}}%
\pgfpathlineto{\pgfqpoint{2.629032in}{3.269320in}}%
\pgfpathlineto{\pgfqpoint{2.639961in}{3.286348in}}%
\pgfpathlineto{\pgfqpoint{2.650886in}{3.303919in}}%
\pgfpathlineto{\pgfqpoint{2.645024in}{3.309210in}}%
\pgfpathlineto{\pgfqpoint{2.639160in}{3.315187in}}%
\pgfpathlineto{\pgfqpoint{2.633294in}{3.321798in}}%
\pgfpathlineto{\pgfqpoint{2.627426in}{3.328990in}}%
\pgfpathlineto{\pgfqpoint{2.621556in}{3.336709in}}%
\pgfpathlineto{\pgfqpoint{2.610641in}{3.320240in}}%
\pgfpathlineto{\pgfqpoint{2.599724in}{3.304223in}}%
\pgfpathlineto{\pgfqpoint{2.588798in}{3.289177in}}%
\pgfpathlineto{\pgfqpoint{2.577855in}{3.275621in}}%
\pgfpathlineto{\pgfqpoint{2.566888in}{3.264073in}}%
\pgfpathlineto{\pgfqpoint{2.572763in}{3.254305in}}%
\pgfpathlineto{\pgfqpoint{2.578628in}{3.245708in}}%
\pgfpathlineto{\pgfqpoint{2.584483in}{3.238259in}}%
\pgfpathlineto{\pgfqpoint{2.590328in}{3.231935in}}%
\pgfpathclose%
\pgfusepath{stroke,fill}%
\end{pgfscope}%
\begin{pgfscope}%
\pgfpathrectangle{\pgfqpoint{0.887500in}{0.275000in}}{\pgfqpoint{4.225000in}{4.225000in}}%
\pgfusepath{clip}%
\pgfsetbuttcap%
\pgfsetroundjoin%
\definecolor{currentfill}{rgb}{0.876168,0.891125,0.095250}%
\pgfsetfillcolor{currentfill}%
\pgfsetfillopacity{0.700000}%
\pgfsetlinewidth{0.501875pt}%
\definecolor{currentstroke}{rgb}{1.000000,1.000000,1.000000}%
\pgfsetstrokecolor{currentstroke}%
\pgfsetstrokeopacity{0.500000}%
\pgfsetdash{}{0pt}%
\pgfpathmoveto{\pgfqpoint{3.399927in}{3.775735in}}%
\pgfpathlineto{\pgfqpoint{3.410852in}{3.781775in}}%
\pgfpathlineto{\pgfqpoint{3.421772in}{3.787722in}}%
\pgfpathlineto{\pgfqpoint{3.432688in}{3.793552in}}%
\pgfpathlineto{\pgfqpoint{3.443599in}{3.799240in}}%
\pgfpathlineto{\pgfqpoint{3.454504in}{3.804773in}}%
\pgfpathlineto{\pgfqpoint{3.448298in}{3.805996in}}%
\pgfpathlineto{\pgfqpoint{3.442096in}{3.807079in}}%
\pgfpathlineto{\pgfqpoint{3.435900in}{3.808045in}}%
\pgfpathlineto{\pgfqpoint{3.429708in}{3.808910in}}%
\pgfpathlineto{\pgfqpoint{3.423522in}{3.809683in}}%
\pgfpathlineto{\pgfqpoint{3.412638in}{3.804248in}}%
\pgfpathlineto{\pgfqpoint{3.401750in}{3.798656in}}%
\pgfpathlineto{\pgfqpoint{3.390856in}{3.792934in}}%
\pgfpathlineto{\pgfqpoint{3.379958in}{3.787102in}}%
\pgfpathlineto{\pgfqpoint{3.369055in}{3.781180in}}%
\pgfpathlineto{\pgfqpoint{3.375218in}{3.780207in}}%
\pgfpathlineto{\pgfqpoint{3.381387in}{3.779188in}}%
\pgfpathlineto{\pgfqpoint{3.387562in}{3.778117in}}%
\pgfpathlineto{\pgfqpoint{3.393742in}{3.776974in}}%
\pgfpathclose%
\pgfusepath{stroke,fill}%
\end{pgfscope}%
\begin{pgfscope}%
\pgfpathrectangle{\pgfqpoint{0.887500in}{0.275000in}}{\pgfqpoint{4.225000in}{4.225000in}}%
\pgfusepath{clip}%
\pgfsetbuttcap%
\pgfsetroundjoin%
\definecolor{currentfill}{rgb}{0.762373,0.876424,0.137064}%
\pgfsetfillcolor{currentfill}%
\pgfsetfillopacity{0.700000}%
\pgfsetlinewidth{0.501875pt}%
\definecolor{currentstroke}{rgb}{1.000000,1.000000,1.000000}%
\pgfsetstrokecolor{currentstroke}%
\pgfsetstrokeopacity{0.500000}%
\pgfsetdash{}{0pt}%
\pgfpathmoveto{\pgfqpoint{3.205075in}{3.689383in}}%
\pgfpathlineto{\pgfqpoint{3.216032in}{3.695578in}}%
\pgfpathlineto{\pgfqpoint{3.226986in}{3.701791in}}%
\pgfpathlineto{\pgfqpoint{3.237936in}{3.708008in}}%
\pgfpathlineto{\pgfqpoint{3.248882in}{3.714215in}}%
\pgfpathlineto{\pgfqpoint{3.259825in}{3.720397in}}%
\pgfpathlineto{\pgfqpoint{3.253712in}{3.721642in}}%
\pgfpathlineto{\pgfqpoint{3.247604in}{3.722884in}}%
\pgfpathlineto{\pgfqpoint{3.241503in}{3.724120in}}%
\pgfpathlineto{\pgfqpoint{3.235408in}{3.725347in}}%
\pgfpathlineto{\pgfqpoint{3.229318in}{3.726560in}}%
\pgfpathlineto{\pgfqpoint{3.218397in}{3.720491in}}%
\pgfpathlineto{\pgfqpoint{3.207472in}{3.714388in}}%
\pgfpathlineto{\pgfqpoint{3.196544in}{3.708260in}}%
\pgfpathlineto{\pgfqpoint{3.185612in}{3.702114in}}%
\pgfpathlineto{\pgfqpoint{3.174677in}{3.695957in}}%
\pgfpathlineto{\pgfqpoint{3.180745in}{3.694640in}}%
\pgfpathlineto{\pgfqpoint{3.186819in}{3.693321in}}%
\pgfpathlineto{\pgfqpoint{3.192898in}{3.692002in}}%
\pgfpathlineto{\pgfqpoint{3.198984in}{3.690688in}}%
\pgfpathclose%
\pgfusepath{stroke,fill}%
\end{pgfscope}%
\begin{pgfscope}%
\pgfpathrectangle{\pgfqpoint{0.887500in}{0.275000in}}{\pgfqpoint{4.225000in}{4.225000in}}%
\pgfusepath{clip}%
\pgfsetbuttcap%
\pgfsetroundjoin%
\definecolor{currentfill}{rgb}{0.214000,0.722114,0.469588}%
\pgfsetfillcolor{currentfill}%
\pgfsetfillopacity{0.700000}%
\pgfsetlinewidth{0.501875pt}%
\definecolor{currentstroke}{rgb}{1.000000,1.000000,1.000000}%
\pgfsetstrokecolor{currentstroke}%
\pgfsetstrokeopacity{0.500000}%
\pgfsetdash{}{0pt}%
\pgfpathmoveto{\pgfqpoint{2.456111in}{3.220779in}}%
\pgfpathlineto{\pgfqpoint{2.467216in}{3.224662in}}%
\pgfpathlineto{\pgfqpoint{2.478321in}{3.228223in}}%
\pgfpathlineto{\pgfqpoint{2.489427in}{3.231318in}}%
\pgfpathlineto{\pgfqpoint{2.500534in}{3.233928in}}%
\pgfpathlineto{\pgfqpoint{2.511638in}{3.236443in}}%
\pgfpathlineto{\pgfqpoint{2.505765in}{3.246974in}}%
\pgfpathlineto{\pgfqpoint{2.499883in}{3.258456in}}%
\pgfpathlineto{\pgfqpoint{2.493989in}{3.270925in}}%
\pgfpathlineto{\pgfqpoint{2.488083in}{3.284418in}}%
\pgfpathlineto{\pgfqpoint{2.477009in}{3.280110in}}%
\pgfpathlineto{\pgfqpoint{2.465930in}{3.275869in}}%
\pgfpathlineto{\pgfqpoint{2.454848in}{3.271505in}}%
\pgfpathlineto{\pgfqpoint{2.443762in}{3.267066in}}%
\pgfpathlineto{\pgfqpoint{2.432672in}{3.262674in}}%
\pgfpathlineto{\pgfqpoint{2.438541in}{3.251266in}}%
\pgfpathlineto{\pgfqpoint{2.444404in}{3.240506in}}%
\pgfpathlineto{\pgfqpoint{2.450260in}{3.230356in}}%
\pgfpathclose%
\pgfusepath{stroke,fill}%
\end{pgfscope}%
\begin{pgfscope}%
\pgfpathrectangle{\pgfqpoint{0.887500in}{0.275000in}}{\pgfqpoint{4.225000in}{4.225000in}}%
\pgfusepath{clip}%
\pgfsetbuttcap%
\pgfsetroundjoin%
\definecolor{currentfill}{rgb}{0.647257,0.858400,0.209861}%
\pgfsetfillcolor{currentfill}%
\pgfsetfillopacity{0.700000}%
\pgfsetlinewidth{0.501875pt}%
\definecolor{currentstroke}{rgb}{1.000000,1.000000,1.000000}%
\pgfsetstrokecolor{currentstroke}%
\pgfsetstrokeopacity{0.500000}%
\pgfsetdash{}{0pt}%
\pgfpathmoveto{\pgfqpoint{3.010221in}{3.609664in}}%
\pgfpathlineto{\pgfqpoint{3.021211in}{3.615113in}}%
\pgfpathlineto{\pgfqpoint{3.032197in}{3.620474in}}%
\pgfpathlineto{\pgfqpoint{3.043179in}{3.625798in}}%
\pgfpathlineto{\pgfqpoint{3.054158in}{3.631132in}}%
\pgfpathlineto{\pgfqpoint{3.065132in}{3.636525in}}%
\pgfpathlineto{\pgfqpoint{3.059112in}{3.637836in}}%
\pgfpathlineto{\pgfqpoint{3.053098in}{3.639193in}}%
\pgfpathlineto{\pgfqpoint{3.047089in}{3.640590in}}%
\pgfpathlineto{\pgfqpoint{3.041087in}{3.642021in}}%
\pgfpathlineto{\pgfqpoint{3.035090in}{3.643480in}}%
\pgfpathlineto{\pgfqpoint{3.024136in}{3.638062in}}%
\pgfpathlineto{\pgfqpoint{3.013179in}{3.632694in}}%
\pgfpathlineto{\pgfqpoint{3.002217in}{3.627335in}}%
\pgfpathlineto{\pgfqpoint{2.991252in}{3.621941in}}%
\pgfpathlineto{\pgfqpoint{2.980283in}{3.616468in}}%
\pgfpathlineto{\pgfqpoint{2.986259in}{3.615033in}}%
\pgfpathlineto{\pgfqpoint{2.992241in}{3.613634in}}%
\pgfpathlineto{\pgfqpoint{2.998228in}{3.612272in}}%
\pgfpathlineto{\pgfqpoint{3.004222in}{3.610948in}}%
\pgfpathclose%
\pgfusepath{stroke,fill}%
\end{pgfscope}%
\begin{pgfscope}%
\pgfpathrectangle{\pgfqpoint{0.887500in}{0.275000in}}{\pgfqpoint{4.225000in}{4.225000in}}%
\pgfusepath{clip}%
\pgfsetbuttcap%
\pgfsetroundjoin%
\definecolor{currentfill}{rgb}{0.202219,0.715272,0.476084}%
\pgfsetfillcolor{currentfill}%
\pgfsetfillopacity{0.700000}%
\pgfsetlinewidth{0.501875pt}%
\definecolor{currentstroke}{rgb}{1.000000,1.000000,1.000000}%
\pgfsetstrokecolor{currentstroke}%
\pgfsetstrokeopacity{0.500000}%
\pgfsetdash{}{0pt}%
\pgfpathmoveto{\pgfqpoint{2.540878in}{3.196772in}}%
\pgfpathlineto{\pgfqpoint{2.551978in}{3.199982in}}%
\pgfpathlineto{\pgfqpoint{2.563061in}{3.204100in}}%
\pgfpathlineto{\pgfqpoint{2.574123in}{3.209608in}}%
\pgfpathlineto{\pgfqpoint{2.585159in}{3.216985in}}%
\pgfpathlineto{\pgfqpoint{2.596166in}{3.226712in}}%
\pgfpathlineto{\pgfqpoint{2.590328in}{3.231935in}}%
\pgfpathlineto{\pgfqpoint{2.584483in}{3.238259in}}%
\pgfpathlineto{\pgfqpoint{2.578628in}{3.245708in}}%
\pgfpathlineto{\pgfqpoint{2.572763in}{3.254305in}}%
\pgfpathlineto{\pgfqpoint{2.566888in}{3.264073in}}%
\pgfpathlineto{\pgfqpoint{2.555890in}{3.254971in}}%
\pgfpathlineto{\pgfqpoint{2.544862in}{3.248127in}}%
\pgfpathlineto{\pgfqpoint{2.533807in}{3.243071in}}%
\pgfpathlineto{\pgfqpoint{2.522731in}{3.239334in}}%
\pgfpathlineto{\pgfqpoint{2.511638in}{3.236443in}}%
\pgfpathlineto{\pgfqpoint{2.517501in}{3.226827in}}%
\pgfpathlineto{\pgfqpoint{2.523356in}{3.218088in}}%
\pgfpathlineto{\pgfqpoint{2.529203in}{3.210190in}}%
\pgfpathlineto{\pgfqpoint{2.535044in}{3.203096in}}%
\pgfpathclose%
\pgfusepath{stroke,fill}%
\end{pgfscope}%
\begin{pgfscope}%
\pgfpathrectangle{\pgfqpoint{0.887500in}{0.275000in}}{\pgfqpoint{4.225000in}{4.225000in}}%
\pgfusepath{clip}%
\pgfsetbuttcap%
\pgfsetroundjoin%
\definecolor{currentfill}{rgb}{0.319809,0.770914,0.411152}%
\pgfsetfillcolor{currentfill}%
\pgfsetfillopacity{0.700000}%
\pgfsetlinewidth{0.501875pt}%
\definecolor{currentstroke}{rgb}{1.000000,1.000000,1.000000}%
\pgfsetstrokecolor{currentstroke}%
\pgfsetstrokeopacity{0.500000}%
\pgfsetdash{}{0pt}%
\pgfpathmoveto{\pgfqpoint{2.650886in}{3.303919in}}%
\pgfpathlineto{\pgfqpoint{2.661813in}{3.321511in}}%
\pgfpathlineto{\pgfqpoint{2.672748in}{3.338605in}}%
\pgfpathlineto{\pgfqpoint{2.683693in}{3.354944in}}%
\pgfpathlineto{\pgfqpoint{2.694645in}{3.370663in}}%
\pgfpathlineto{\pgfqpoint{2.705602in}{3.385933in}}%
\pgfpathlineto{\pgfqpoint{2.699721in}{3.390742in}}%
\pgfpathlineto{\pgfqpoint{2.693842in}{3.395830in}}%
\pgfpathlineto{\pgfqpoint{2.687966in}{3.401105in}}%
\pgfpathlineto{\pgfqpoint{2.682093in}{3.406475in}}%
\pgfpathlineto{\pgfqpoint{2.676225in}{3.411849in}}%
\pgfpathlineto{\pgfqpoint{2.665276in}{3.398140in}}%
\pgfpathlineto{\pgfqpoint{2.654333in}{3.383885in}}%
\pgfpathlineto{\pgfqpoint{2.643398in}{3.368918in}}%
\pgfpathlineto{\pgfqpoint{2.632472in}{3.353107in}}%
\pgfpathlineto{\pgfqpoint{2.621556in}{3.336709in}}%
\pgfpathlineto{\pgfqpoint{2.627426in}{3.328990in}}%
\pgfpathlineto{\pgfqpoint{2.633294in}{3.321798in}}%
\pgfpathlineto{\pgfqpoint{2.639160in}{3.315187in}}%
\pgfpathlineto{\pgfqpoint{2.645024in}{3.309210in}}%
\pgfpathclose%
\pgfusepath{stroke,fill}%
\end{pgfscope}%
\begin{pgfscope}%
\pgfpathrectangle{\pgfqpoint{0.887500in}{0.275000in}}{\pgfqpoint{4.225000in}{4.225000in}}%
\pgfusepath{clip}%
\pgfsetbuttcap%
\pgfsetroundjoin%
\definecolor{currentfill}{rgb}{0.555484,0.840254,0.269281}%
\pgfsetfillcolor{currentfill}%
\pgfsetfillopacity{0.700000}%
\pgfsetlinewidth{0.501875pt}%
\definecolor{currentstroke}{rgb}{1.000000,1.000000,1.000000}%
\pgfsetstrokecolor{currentstroke}%
\pgfsetstrokeopacity{0.500000}%
\pgfsetdash{}{0pt}%
\pgfpathmoveto{\pgfqpoint{2.815312in}{3.528982in}}%
\pgfpathlineto{\pgfqpoint{2.826315in}{3.538540in}}%
\pgfpathlineto{\pgfqpoint{2.837326in}{3.546419in}}%
\pgfpathlineto{\pgfqpoint{2.848340in}{3.552952in}}%
\pgfpathlineto{\pgfqpoint{2.859356in}{3.558476in}}%
\pgfpathlineto{\pgfqpoint{2.870371in}{3.563330in}}%
\pgfpathlineto{\pgfqpoint{2.864443in}{3.564730in}}%
\pgfpathlineto{\pgfqpoint{2.858520in}{3.566144in}}%
\pgfpathlineto{\pgfqpoint{2.852602in}{3.567565in}}%
\pgfpathlineto{\pgfqpoint{2.846690in}{3.568987in}}%
\pgfpathlineto{\pgfqpoint{2.840784in}{3.570406in}}%
\pgfpathlineto{\pgfqpoint{2.829791in}{3.565294in}}%
\pgfpathlineto{\pgfqpoint{2.818797in}{3.559656in}}%
\pgfpathlineto{\pgfqpoint{2.807804in}{3.553236in}}%
\pgfpathlineto{\pgfqpoint{2.796814in}{3.545777in}}%
\pgfpathlineto{\pgfqpoint{2.785829in}{3.537028in}}%
\pgfpathlineto{\pgfqpoint{2.791714in}{3.535552in}}%
\pgfpathlineto{\pgfqpoint{2.797605in}{3.533981in}}%
\pgfpathlineto{\pgfqpoint{2.803502in}{3.532344in}}%
\pgfpathlineto{\pgfqpoint{2.809404in}{3.530668in}}%
\pgfpathclose%
\pgfusepath{stroke,fill}%
\end{pgfscope}%
\begin{pgfscope}%
\pgfpathrectangle{\pgfqpoint{0.887500in}{0.275000in}}{\pgfqpoint{4.225000in}{4.225000in}}%
\pgfusepath{clip}%
\pgfsetbuttcap%
\pgfsetroundjoin%
\definecolor{currentfill}{rgb}{0.945636,0.899815,0.112838}%
\pgfsetfillcolor{currentfill}%
\pgfsetfillopacity{0.700000}%
\pgfsetlinewidth{0.501875pt}%
\definecolor{currentstroke}{rgb}{1.000000,1.000000,1.000000}%
\pgfsetstrokecolor{currentstroke}%
\pgfsetstrokeopacity{0.500000}%
\pgfsetdash{}{0pt}%
\pgfpathmoveto{\pgfqpoint{3.540139in}{3.821301in}}%
\pgfpathlineto{\pgfqpoint{3.551032in}{3.826129in}}%
\pgfpathlineto{\pgfqpoint{3.561921in}{3.830886in}}%
\pgfpathlineto{\pgfqpoint{3.572804in}{3.835571in}}%
\pgfpathlineto{\pgfqpoint{3.583682in}{3.840182in}}%
\pgfpathlineto{\pgfqpoint{3.594555in}{3.844716in}}%
\pgfpathlineto{\pgfqpoint{3.588286in}{3.846453in}}%
\pgfpathlineto{\pgfqpoint{3.582019in}{3.847850in}}%
\pgfpathlineto{\pgfqpoint{3.575753in}{3.848928in}}%
\pgfpathlineto{\pgfqpoint{3.569490in}{3.849711in}}%
\pgfpathlineto{\pgfqpoint{3.563229in}{3.850222in}}%
\pgfpathlineto{\pgfqpoint{3.552387in}{3.846553in}}%
\pgfpathlineto{\pgfqpoint{3.541537in}{3.842682in}}%
\pgfpathlineto{\pgfqpoint{3.530680in}{3.838611in}}%
\pgfpathlineto{\pgfqpoint{3.519817in}{3.834342in}}%
\pgfpathlineto{\pgfqpoint{3.508947in}{3.829877in}}%
\pgfpathlineto{\pgfqpoint{3.515179in}{3.828628in}}%
\pgfpathlineto{\pgfqpoint{3.521414in}{3.827164in}}%
\pgfpathlineto{\pgfqpoint{3.527653in}{3.825467in}}%
\pgfpathlineto{\pgfqpoint{3.533894in}{3.823518in}}%
\pgfpathclose%
\pgfusepath{stroke,fill}%
\end{pgfscope}%
\begin{pgfscope}%
\pgfpathrectangle{\pgfqpoint{0.887500in}{0.275000in}}{\pgfqpoint{4.225000in}{4.225000in}}%
\pgfusepath{clip}%
\pgfsetbuttcap%
\pgfsetroundjoin%
\definecolor{currentfill}{rgb}{0.412913,0.803041,0.357269}%
\pgfsetfillcolor{currentfill}%
\pgfsetfillopacity{0.700000}%
\pgfsetlinewidth{0.501875pt}%
\definecolor{currentstroke}{rgb}{1.000000,1.000000,1.000000}%
\pgfsetstrokecolor{currentstroke}%
\pgfsetstrokeopacity{0.500000}%
\pgfsetdash{}{0pt}%
\pgfpathmoveto{\pgfqpoint{2.705602in}{3.385933in}}%
\pgfpathlineto{\pgfqpoint{2.716563in}{3.400923in}}%
\pgfpathlineto{\pgfqpoint{2.727527in}{3.415802in}}%
\pgfpathlineto{\pgfqpoint{2.738491in}{3.430742in}}%
\pgfpathlineto{\pgfqpoint{2.749455in}{3.445903in}}%
\pgfpathlineto{\pgfqpoint{2.760419in}{3.461218in}}%
\pgfpathlineto{\pgfqpoint{2.754526in}{3.464467in}}%
\pgfpathlineto{\pgfqpoint{2.748637in}{3.467820in}}%
\pgfpathlineto{\pgfqpoint{2.742753in}{3.471207in}}%
\pgfpathlineto{\pgfqpoint{2.736874in}{3.474560in}}%
\pgfpathlineto{\pgfqpoint{2.731001in}{3.477810in}}%
\pgfpathlineto{\pgfqpoint{2.720046in}{3.464547in}}%
\pgfpathlineto{\pgfqpoint{2.709090in}{3.451364in}}%
\pgfpathlineto{\pgfqpoint{2.698134in}{3.438294in}}%
\pgfpathlineto{\pgfqpoint{2.687179in}{3.425178in}}%
\pgfpathlineto{\pgfqpoint{2.676225in}{3.411849in}}%
\pgfpathlineto{\pgfqpoint{2.682093in}{3.406475in}}%
\pgfpathlineto{\pgfqpoint{2.687966in}{3.401105in}}%
\pgfpathlineto{\pgfqpoint{2.693842in}{3.395830in}}%
\pgfpathlineto{\pgfqpoint{2.699721in}{3.390742in}}%
\pgfpathclose%
\pgfusepath{stroke,fill}%
\end{pgfscope}%
\begin{pgfscope}%
\pgfpathrectangle{\pgfqpoint{0.887500in}{0.275000in}}{\pgfqpoint{4.225000in}{4.225000in}}%
\pgfusepath{clip}%
\pgfsetbuttcap%
\pgfsetroundjoin%
\definecolor{currentfill}{rgb}{0.496615,0.826376,0.306377}%
\pgfsetfillcolor{currentfill}%
\pgfsetfillopacity{0.700000}%
\pgfsetlinewidth{0.501875pt}%
\definecolor{currentstroke}{rgb}{1.000000,1.000000,1.000000}%
\pgfsetstrokecolor{currentstroke}%
\pgfsetstrokeopacity{0.500000}%
\pgfsetdash{}{0pt}%
\pgfpathmoveto{\pgfqpoint{2.760419in}{3.461218in}}%
\pgfpathlineto{\pgfqpoint{2.771385in}{3.476375in}}%
\pgfpathlineto{\pgfqpoint{2.782356in}{3.491053in}}%
\pgfpathlineto{\pgfqpoint{2.793334in}{3.504929in}}%
\pgfpathlineto{\pgfqpoint{2.804319in}{3.517679in}}%
\pgfpathlineto{\pgfqpoint{2.815312in}{3.528982in}}%
\pgfpathlineto{\pgfqpoint{2.809404in}{3.530668in}}%
\pgfpathlineto{\pgfqpoint{2.803502in}{3.532344in}}%
\pgfpathlineto{\pgfqpoint{2.797605in}{3.533981in}}%
\pgfpathlineto{\pgfqpoint{2.791714in}{3.535552in}}%
\pgfpathlineto{\pgfqpoint{2.785829in}{3.537028in}}%
\pgfpathlineto{\pgfqpoint{2.774853in}{3.526938in}}%
\pgfpathlineto{\pgfqpoint{2.763882in}{3.515729in}}%
\pgfpathlineto{\pgfqpoint{2.752918in}{3.503644in}}%
\pgfpathlineto{\pgfqpoint{2.741958in}{3.490923in}}%
\pgfpathlineto{\pgfqpoint{2.731001in}{3.477810in}}%
\pgfpathlineto{\pgfqpoint{2.736874in}{3.474560in}}%
\pgfpathlineto{\pgfqpoint{2.742753in}{3.471207in}}%
\pgfpathlineto{\pgfqpoint{2.748637in}{3.467820in}}%
\pgfpathlineto{\pgfqpoint{2.754526in}{3.464467in}}%
\pgfpathclose%
\pgfusepath{stroke,fill}%
\end{pgfscope}%
\begin{pgfscope}%
\pgfpathrectangle{\pgfqpoint{0.887500in}{0.275000in}}{\pgfqpoint{4.225000in}{4.225000in}}%
\pgfusepath{clip}%
\pgfsetbuttcap%
\pgfsetroundjoin%
\definecolor{currentfill}{rgb}{0.730889,0.871916,0.156029}%
\pgfsetfillcolor{currentfill}%
\pgfsetfillopacity{0.700000}%
\pgfsetlinewidth{0.501875pt}%
\definecolor{currentstroke}{rgb}{1.000000,1.000000,1.000000}%
\pgfsetstrokecolor{currentstroke}%
\pgfsetstrokeopacity{0.500000}%
\pgfsetdash{}{0pt}%
\pgfpathmoveto{\pgfqpoint{3.150241in}{3.659125in}}%
\pgfpathlineto{\pgfqpoint{3.161215in}{3.665045in}}%
\pgfpathlineto{\pgfqpoint{3.172185in}{3.671039in}}%
\pgfpathlineto{\pgfqpoint{3.183152in}{3.677099in}}%
\pgfpathlineto{\pgfqpoint{3.194115in}{3.683219in}}%
\pgfpathlineto{\pgfqpoint{3.205075in}{3.689383in}}%
\pgfpathlineto{\pgfqpoint{3.198984in}{3.690688in}}%
\pgfpathlineto{\pgfqpoint{3.192898in}{3.692002in}}%
\pgfpathlineto{\pgfqpoint{3.186819in}{3.693321in}}%
\pgfpathlineto{\pgfqpoint{3.180745in}{3.694640in}}%
\pgfpathlineto{\pgfqpoint{3.174677in}{3.695957in}}%
\pgfpathlineto{\pgfqpoint{3.163738in}{3.689797in}}%
\pgfpathlineto{\pgfqpoint{3.152795in}{3.683643in}}%
\pgfpathlineto{\pgfqpoint{3.141850in}{3.677502in}}%
\pgfpathlineto{\pgfqpoint{3.130901in}{3.671392in}}%
\pgfpathlineto{\pgfqpoint{3.119948in}{3.665333in}}%
\pgfpathlineto{\pgfqpoint{3.125995in}{3.664018in}}%
\pgfpathlineto{\pgfqpoint{3.132047in}{3.662735in}}%
\pgfpathlineto{\pgfqpoint{3.138106in}{3.661487in}}%
\pgfpathlineto{\pgfqpoint{3.144171in}{3.660282in}}%
\pgfpathclose%
\pgfusepath{stroke,fill}%
\end{pgfscope}%
\begin{pgfscope}%
\pgfpathrectangle{\pgfqpoint{0.887500in}{0.275000in}}{\pgfqpoint{4.225000in}{4.225000in}}%
\pgfusepath{clip}%
\pgfsetbuttcap%
\pgfsetroundjoin%
\definecolor{currentfill}{rgb}{0.845561,0.887322,0.099702}%
\pgfsetfillcolor{currentfill}%
\pgfsetfillopacity{0.700000}%
\pgfsetlinewidth{0.501875pt}%
\definecolor{currentstroke}{rgb}{1.000000,1.000000,1.000000}%
\pgfsetstrokecolor{currentstroke}%
\pgfsetstrokeopacity{0.500000}%
\pgfsetdash{}{0pt}%
\pgfpathmoveto{\pgfqpoint{3.345247in}{3.745001in}}%
\pgfpathlineto{\pgfqpoint{3.356190in}{3.751150in}}%
\pgfpathlineto{\pgfqpoint{3.367129in}{3.757311in}}%
\pgfpathlineto{\pgfqpoint{3.378066in}{3.763479in}}%
\pgfpathlineto{\pgfqpoint{3.388998in}{3.769628in}}%
\pgfpathlineto{\pgfqpoint{3.399927in}{3.775735in}}%
\pgfpathlineto{\pgfqpoint{3.393742in}{3.776974in}}%
\pgfpathlineto{\pgfqpoint{3.387562in}{3.778117in}}%
\pgfpathlineto{\pgfqpoint{3.381387in}{3.779188in}}%
\pgfpathlineto{\pgfqpoint{3.375218in}{3.780207in}}%
\pgfpathlineto{\pgfqpoint{3.369055in}{3.781180in}}%
\pgfpathlineto{\pgfqpoint{3.358148in}{3.775191in}}%
\pgfpathlineto{\pgfqpoint{3.347238in}{3.769153in}}%
\pgfpathlineto{\pgfqpoint{3.336324in}{3.763089in}}%
\pgfpathlineto{\pgfqpoint{3.325406in}{3.757017in}}%
\pgfpathlineto{\pgfqpoint{3.314485in}{3.750945in}}%
\pgfpathlineto{\pgfqpoint{3.320626in}{3.749800in}}%
\pgfpathlineto{\pgfqpoint{3.326772in}{3.748643in}}%
\pgfpathlineto{\pgfqpoint{3.332925in}{3.747472in}}%
\pgfpathlineto{\pgfqpoint{3.339083in}{3.746265in}}%
\pgfpathclose%
\pgfusepath{stroke,fill}%
\end{pgfscope}%
\begin{pgfscope}%
\pgfpathrectangle{\pgfqpoint{0.887500in}{0.275000in}}{\pgfqpoint{4.225000in}{4.225000in}}%
\pgfusepath{clip}%
\pgfsetbuttcap%
\pgfsetroundjoin%
\definecolor{currentfill}{rgb}{0.993248,0.906157,0.143936}%
\pgfsetfillcolor{currentfill}%
\pgfsetfillopacity{0.700000}%
\pgfsetlinewidth{0.501875pt}%
\definecolor{currentstroke}{rgb}{1.000000,1.000000,1.000000}%
\pgfsetstrokecolor{currentstroke}%
\pgfsetstrokeopacity{0.500000}%
\pgfsetdash{}{0pt}%
\pgfpathmoveto{\pgfqpoint{3.680317in}{3.854314in}}%
\pgfpathlineto{\pgfqpoint{3.691185in}{3.859059in}}%
\pgfpathlineto{\pgfqpoint{3.702049in}{3.863778in}}%
\pgfpathlineto{\pgfqpoint{3.712908in}{3.868468in}}%
\pgfpathlineto{\pgfqpoint{3.723762in}{3.873129in}}%
\pgfpathlineto{\pgfqpoint{3.717454in}{3.876346in}}%
\pgfpathlineto{\pgfqpoint{3.711140in}{3.878857in}}%
\pgfpathlineto{\pgfqpoint{3.704819in}{3.880594in}}%
\pgfpathlineto{\pgfqpoint{3.698492in}{3.881543in}}%
\pgfpathlineto{\pgfqpoint{3.692160in}{3.881757in}}%
\pgfpathlineto{\pgfqpoint{3.681339in}{3.877999in}}%
\pgfpathlineto{\pgfqpoint{3.670511in}{3.874148in}}%
\pgfpathlineto{\pgfqpoint{3.659677in}{3.870204in}}%
\pgfpathlineto{\pgfqpoint{3.648838in}{3.866170in}}%
\pgfpathlineto{\pgfqpoint{3.655139in}{3.864929in}}%
\pgfpathlineto{\pgfqpoint{3.661438in}{3.863138in}}%
\pgfpathlineto{\pgfqpoint{3.667733in}{3.860761in}}%
\pgfpathlineto{\pgfqpoint{3.674026in}{3.857805in}}%
\pgfpathclose%
\pgfusepath{stroke,fill}%
\end{pgfscope}%
\begin{pgfscope}%
\pgfpathrectangle{\pgfqpoint{0.887500in}{0.275000in}}{\pgfqpoint{4.225000in}{4.225000in}}%
\pgfusepath{clip}%
\pgfsetbuttcap%
\pgfsetroundjoin%
\definecolor{currentfill}{rgb}{0.626579,0.854645,0.223353}%
\pgfsetfillcolor{currentfill}%
\pgfsetfillopacity{0.700000}%
\pgfsetlinewidth{0.501875pt}%
\definecolor{currentstroke}{rgb}{1.000000,1.000000,1.000000}%
\pgfsetstrokecolor{currentstroke}%
\pgfsetstrokeopacity{0.500000}%
\pgfsetdash{}{0pt}%
\pgfpathmoveto{\pgfqpoint{2.955215in}{3.581047in}}%
\pgfpathlineto{\pgfqpoint{2.966224in}{3.586731in}}%
\pgfpathlineto{\pgfqpoint{2.977229in}{3.592527in}}%
\pgfpathlineto{\pgfqpoint{2.988230in}{3.598341in}}%
\pgfpathlineto{\pgfqpoint{2.999227in}{3.604079in}}%
\pgfpathlineto{\pgfqpoint{3.010221in}{3.609664in}}%
\pgfpathlineto{\pgfqpoint{3.004222in}{3.610948in}}%
\pgfpathlineto{\pgfqpoint{2.998228in}{3.612272in}}%
\pgfpathlineto{\pgfqpoint{2.992241in}{3.613634in}}%
\pgfpathlineto{\pgfqpoint{2.986259in}{3.615033in}}%
\pgfpathlineto{\pgfqpoint{2.980283in}{3.616468in}}%
\pgfpathlineto{\pgfqpoint{2.969310in}{3.610875in}}%
\pgfpathlineto{\pgfqpoint{2.958334in}{3.605146in}}%
\pgfpathlineto{\pgfqpoint{2.947354in}{3.599351in}}%
\pgfpathlineto{\pgfqpoint{2.936371in}{3.593576in}}%
\pgfpathlineto{\pgfqpoint{2.925383in}{3.587907in}}%
\pgfpathlineto{\pgfqpoint{2.931338in}{3.586444in}}%
\pgfpathlineto{\pgfqpoint{2.937299in}{3.585027in}}%
\pgfpathlineto{\pgfqpoint{2.943266in}{3.583656in}}%
\pgfpathlineto{\pgfqpoint{2.949238in}{3.582329in}}%
\pgfpathclose%
\pgfusepath{stroke,fill}%
\end{pgfscope}%
\begin{pgfscope}%
\pgfpathrectangle{\pgfqpoint{0.887500in}{0.275000in}}{\pgfqpoint{4.225000in}{4.225000in}}%
\pgfusepath{clip}%
\pgfsetbuttcap%
\pgfsetroundjoin%
\definecolor{currentfill}{rgb}{0.214000,0.722114,0.469588}%
\pgfsetfillcolor{currentfill}%
\pgfsetfillopacity{0.700000}%
\pgfsetlinewidth{0.501875pt}%
\definecolor{currentstroke}{rgb}{1.000000,1.000000,1.000000}%
\pgfsetstrokecolor{currentstroke}%
\pgfsetstrokeopacity{0.500000}%
\pgfsetdash{}{0pt}%
\pgfpathmoveto{\pgfqpoint{2.400511in}{3.201515in}}%
\pgfpathlineto{\pgfqpoint{2.411648in}{3.204913in}}%
\pgfpathlineto{\pgfqpoint{2.422775in}{3.208641in}}%
\pgfpathlineto{\pgfqpoint{2.433892in}{3.212624in}}%
\pgfpathlineto{\pgfqpoint{2.445004in}{3.216718in}}%
\pgfpathlineto{\pgfqpoint{2.456111in}{3.220779in}}%
\pgfpathlineto{\pgfqpoint{2.450260in}{3.230356in}}%
\pgfpathlineto{\pgfqpoint{2.444404in}{3.240506in}}%
\pgfpathlineto{\pgfqpoint{2.438541in}{3.251266in}}%
\pgfpathlineto{\pgfqpoint{2.432672in}{3.262674in}}%
\pgfpathlineto{\pgfqpoint{2.421574in}{3.258450in}}%
\pgfpathlineto{\pgfqpoint{2.410467in}{3.254516in}}%
\pgfpathlineto{\pgfqpoint{2.399348in}{3.250993in}}%
\pgfpathlineto{\pgfqpoint{2.388215in}{3.248004in}}%
\pgfpathlineto{\pgfqpoint{2.377067in}{3.245580in}}%
\pgfpathlineto{\pgfqpoint{2.382945in}{3.233297in}}%
\pgfpathlineto{\pgfqpoint{2.388811in}{3.221907in}}%
\pgfpathlineto{\pgfqpoint{2.394665in}{3.211337in}}%
\pgfpathclose%
\pgfusepath{stroke,fill}%
\end{pgfscope}%
\begin{pgfscope}%
\pgfpathrectangle{\pgfqpoint{0.887500in}{0.275000in}}{\pgfqpoint{4.225000in}{4.225000in}}%
\pgfusepath{clip}%
\pgfsetbuttcap%
\pgfsetroundjoin%
\definecolor{currentfill}{rgb}{0.196571,0.711827,0.479221}%
\pgfsetfillcolor{currentfill}%
\pgfsetfillopacity{0.700000}%
\pgfsetlinewidth{0.501875pt}%
\definecolor{currentstroke}{rgb}{1.000000,1.000000,1.000000}%
\pgfsetstrokecolor{currentstroke}%
\pgfsetstrokeopacity{0.500000}%
\pgfsetdash{}{0pt}%
\pgfpathmoveto{\pgfqpoint{2.485310in}{3.180141in}}%
\pgfpathlineto{\pgfqpoint{2.496424in}{3.184142in}}%
\pgfpathlineto{\pgfqpoint{2.507538in}{3.187861in}}%
\pgfpathlineto{\pgfqpoint{2.518653in}{3.191150in}}%
\pgfpathlineto{\pgfqpoint{2.529768in}{3.193988in}}%
\pgfpathlineto{\pgfqpoint{2.540878in}{3.196772in}}%
\pgfpathlineto{\pgfqpoint{2.535044in}{3.203096in}}%
\pgfpathlineto{\pgfqpoint{2.529203in}{3.210190in}}%
\pgfpathlineto{\pgfqpoint{2.523356in}{3.218088in}}%
\pgfpathlineto{\pgfqpoint{2.517501in}{3.226827in}}%
\pgfpathlineto{\pgfqpoint{2.511638in}{3.236443in}}%
\pgfpathlineto{\pgfqpoint{2.500534in}{3.233928in}}%
\pgfpathlineto{\pgfqpoint{2.489427in}{3.231318in}}%
\pgfpathlineto{\pgfqpoint{2.478321in}{3.228223in}}%
\pgfpathlineto{\pgfqpoint{2.467216in}{3.224662in}}%
\pgfpathlineto{\pgfqpoint{2.456111in}{3.220779in}}%
\pgfpathlineto{\pgfqpoint{2.461957in}{3.211735in}}%
\pgfpathlineto{\pgfqpoint{2.467799in}{3.203188in}}%
\pgfpathlineto{\pgfqpoint{2.473638in}{3.195098in}}%
\pgfpathlineto{\pgfqpoint{2.479475in}{3.187429in}}%
\pgfpathclose%
\pgfusepath{stroke,fill}%
\end{pgfscope}%
\begin{pgfscope}%
\pgfpathrectangle{\pgfqpoint{0.887500in}{0.275000in}}{\pgfqpoint{4.225000in}{4.225000in}}%
\pgfusepath{clip}%
\pgfsetbuttcap%
\pgfsetroundjoin%
\definecolor{currentfill}{rgb}{0.926106,0.897330,0.104071}%
\pgfsetfillcolor{currentfill}%
\pgfsetfillopacity{0.700000}%
\pgfsetlinewidth{0.501875pt}%
\definecolor{currentstroke}{rgb}{1.000000,1.000000,1.000000}%
\pgfsetstrokecolor{currentstroke}%
\pgfsetstrokeopacity{0.500000}%
\pgfsetdash{}{0pt}%
\pgfpathmoveto{\pgfqpoint{3.485595in}{3.795830in}}%
\pgfpathlineto{\pgfqpoint{3.496514in}{3.801141in}}%
\pgfpathlineto{\pgfqpoint{3.507428in}{3.806332in}}%
\pgfpathlineto{\pgfqpoint{3.518337in}{3.811414in}}%
\pgfpathlineto{\pgfqpoint{3.529240in}{3.816400in}}%
\pgfpathlineto{\pgfqpoint{3.540139in}{3.821301in}}%
\pgfpathlineto{\pgfqpoint{3.533894in}{3.823518in}}%
\pgfpathlineto{\pgfqpoint{3.527653in}{3.825467in}}%
\pgfpathlineto{\pgfqpoint{3.521414in}{3.827164in}}%
\pgfpathlineto{\pgfqpoint{3.515179in}{3.828628in}}%
\pgfpathlineto{\pgfqpoint{3.508947in}{3.829877in}}%
\pgfpathlineto{\pgfqpoint{3.498070in}{3.825218in}}%
\pgfpathlineto{\pgfqpoint{3.487187in}{3.820372in}}%
\pgfpathlineto{\pgfqpoint{3.476299in}{3.815345in}}%
\pgfpathlineto{\pgfqpoint{3.465404in}{3.810143in}}%
\pgfpathlineto{\pgfqpoint{3.454504in}{3.804773in}}%
\pgfpathlineto{\pgfqpoint{3.460715in}{3.803389in}}%
\pgfpathlineto{\pgfqpoint{3.466929in}{3.801824in}}%
\pgfpathlineto{\pgfqpoint{3.473148in}{3.800057in}}%
\pgfpathlineto{\pgfqpoint{3.479369in}{3.798066in}}%
\pgfpathclose%
\pgfusepath{stroke,fill}%
\end{pgfscope}%
\begin{pgfscope}%
\pgfpathrectangle{\pgfqpoint{0.887500in}{0.275000in}}{\pgfqpoint{4.225000in}{4.225000in}}%
\pgfusepath{clip}%
\pgfsetbuttcap%
\pgfsetroundjoin%
\definecolor{currentfill}{rgb}{0.824940,0.884720,0.106217}%
\pgfsetfillcolor{currentfill}%
\pgfsetfillopacity{0.700000}%
\pgfsetlinewidth{0.501875pt}%
\definecolor{currentstroke}{rgb}{1.000000,1.000000,1.000000}%
\pgfsetstrokecolor{currentstroke}%
\pgfsetstrokeopacity{0.500000}%
\pgfsetdash{}{0pt}%
\pgfpathmoveto{\pgfqpoint{3.290478in}{3.714151in}}%
\pgfpathlineto{\pgfqpoint{3.301439in}{3.720365in}}%
\pgfpathlineto{\pgfqpoint{3.312397in}{3.726547in}}%
\pgfpathlineto{\pgfqpoint{3.323350in}{3.732708in}}%
\pgfpathlineto{\pgfqpoint{3.334300in}{3.738857in}}%
\pgfpathlineto{\pgfqpoint{3.345247in}{3.745001in}}%
\pgfpathlineto{\pgfqpoint{3.339083in}{3.746265in}}%
\pgfpathlineto{\pgfqpoint{3.332925in}{3.747472in}}%
\pgfpathlineto{\pgfqpoint{3.326772in}{3.748643in}}%
\pgfpathlineto{\pgfqpoint{3.320626in}{3.749800in}}%
\pgfpathlineto{\pgfqpoint{3.314485in}{3.750945in}}%
\pgfpathlineto{\pgfqpoint{3.303560in}{3.744867in}}%
\pgfpathlineto{\pgfqpoint{3.292632in}{3.738779in}}%
\pgfpathlineto{\pgfqpoint{3.281700in}{3.732675in}}%
\pgfpathlineto{\pgfqpoint{3.270764in}{3.726549in}}%
\pgfpathlineto{\pgfqpoint{3.259825in}{3.720397in}}%
\pgfpathlineto{\pgfqpoint{3.265944in}{3.719155in}}%
\pgfpathlineto{\pgfqpoint{3.272069in}{3.717918in}}%
\pgfpathlineto{\pgfqpoint{3.278199in}{3.716685in}}%
\pgfpathlineto{\pgfqpoint{3.284336in}{3.715437in}}%
\pgfpathclose%
\pgfusepath{stroke,fill}%
\end{pgfscope}%
\begin{pgfscope}%
\pgfpathrectangle{\pgfqpoint{0.887500in}{0.275000in}}{\pgfqpoint{4.225000in}{4.225000in}}%
\pgfusepath{clip}%
\pgfsetbuttcap%
\pgfsetroundjoin%
\definecolor{currentfill}{rgb}{0.699415,0.867117,0.175971}%
\pgfsetfillcolor{currentfill}%
\pgfsetfillopacity{0.700000}%
\pgfsetlinewidth{0.501875pt}%
\definecolor{currentstroke}{rgb}{1.000000,1.000000,1.000000}%
\pgfsetstrokecolor{currentstroke}%
\pgfsetstrokeopacity{0.500000}%
\pgfsetdash{}{0pt}%
\pgfpathmoveto{\pgfqpoint{3.095319in}{3.630867in}}%
\pgfpathlineto{\pgfqpoint{3.106311in}{3.636321in}}%
\pgfpathlineto{\pgfqpoint{3.117299in}{3.641876in}}%
\pgfpathlineto{\pgfqpoint{3.128283in}{3.647534in}}%
\pgfpathlineto{\pgfqpoint{3.139264in}{3.653285in}}%
\pgfpathlineto{\pgfqpoint{3.150241in}{3.659125in}}%
\pgfpathlineto{\pgfqpoint{3.144171in}{3.660282in}}%
\pgfpathlineto{\pgfqpoint{3.138106in}{3.661487in}}%
\pgfpathlineto{\pgfqpoint{3.132047in}{3.662735in}}%
\pgfpathlineto{\pgfqpoint{3.125995in}{3.664018in}}%
\pgfpathlineto{\pgfqpoint{3.119948in}{3.665333in}}%
\pgfpathlineto{\pgfqpoint{3.108992in}{3.659347in}}%
\pgfpathlineto{\pgfqpoint{3.098032in}{3.653453in}}%
\pgfpathlineto{\pgfqpoint{3.087069in}{3.647673in}}%
\pgfpathlineto{\pgfqpoint{3.076102in}{3.642027in}}%
\pgfpathlineto{\pgfqpoint{3.065132in}{3.636525in}}%
\pgfpathlineto{\pgfqpoint{3.071158in}{3.635266in}}%
\pgfpathlineto{\pgfqpoint{3.077189in}{3.634064in}}%
\pgfpathlineto{\pgfqpoint{3.083227in}{3.632927in}}%
\pgfpathlineto{\pgfqpoint{3.089270in}{3.631859in}}%
\pgfpathclose%
\pgfusepath{stroke,fill}%
\end{pgfscope}%
\begin{pgfscope}%
\pgfpathrectangle{\pgfqpoint{0.887500in}{0.275000in}}{\pgfqpoint{4.225000in}{4.225000in}}%
\pgfusepath{clip}%
\pgfsetbuttcap%
\pgfsetroundjoin%
\definecolor{currentfill}{rgb}{0.974417,0.903590,0.130215}%
\pgfsetfillcolor{currentfill}%
\pgfsetfillopacity{0.700000}%
\pgfsetlinewidth{0.501875pt}%
\definecolor{currentstroke}{rgb}{1.000000,1.000000,1.000000}%
\pgfsetstrokecolor{currentstroke}%
\pgfsetstrokeopacity{0.500000}%
\pgfsetdash{}{0pt}%
\pgfpathmoveto{\pgfqpoint{3.625908in}{3.830225in}}%
\pgfpathlineto{\pgfqpoint{3.636799in}{3.835087in}}%
\pgfpathlineto{\pgfqpoint{3.647685in}{3.839929in}}%
\pgfpathlineto{\pgfqpoint{3.658567in}{3.844748in}}%
\pgfpathlineto{\pgfqpoint{3.669444in}{3.849543in}}%
\pgfpathlineto{\pgfqpoint{3.680317in}{3.854314in}}%
\pgfpathlineto{\pgfqpoint{3.674026in}{3.857805in}}%
\pgfpathlineto{\pgfqpoint{3.667733in}{3.860761in}}%
\pgfpathlineto{\pgfqpoint{3.661438in}{3.863138in}}%
\pgfpathlineto{\pgfqpoint{3.655139in}{3.864929in}}%
\pgfpathlineto{\pgfqpoint{3.648838in}{3.866170in}}%
\pgfpathlineto{\pgfqpoint{3.637993in}{3.862048in}}%
\pgfpathlineto{\pgfqpoint{3.627141in}{3.857839in}}%
\pgfpathlineto{\pgfqpoint{3.616285in}{3.853546in}}%
\pgfpathlineto{\pgfqpoint{3.605423in}{3.849171in}}%
\pgfpathlineto{\pgfqpoint{3.594555in}{3.844716in}}%
\pgfpathlineto{\pgfqpoint{3.600825in}{3.842613in}}%
\pgfpathlineto{\pgfqpoint{3.607096in}{3.840123in}}%
\pgfpathlineto{\pgfqpoint{3.613366in}{3.837224in}}%
\pgfpathlineto{\pgfqpoint{3.619637in}{3.833918in}}%
\pgfpathclose%
\pgfusepath{stroke,fill}%
\end{pgfscope}%
\begin{pgfscope}%
\pgfpathrectangle{\pgfqpoint{0.887500in}{0.275000in}}{\pgfqpoint{4.225000in}{4.225000in}}%
\pgfusepath{clip}%
\pgfsetbuttcap%
\pgfsetroundjoin%
\definecolor{currentfill}{rgb}{0.606045,0.850733,0.236712}%
\pgfsetfillcolor{currentfill}%
\pgfsetfillopacity{0.700000}%
\pgfsetlinewidth{0.501875pt}%
\definecolor{currentstroke}{rgb}{1.000000,1.000000,1.000000}%
\pgfsetstrokecolor{currentstroke}%
\pgfsetstrokeopacity{0.500000}%
\pgfsetdash{}{0pt}%
\pgfpathmoveto{\pgfqpoint{2.900099in}{3.556724in}}%
\pgfpathlineto{\pgfqpoint{2.911132in}{3.561166in}}%
\pgfpathlineto{\pgfqpoint{2.922161in}{3.565608in}}%
\pgfpathlineto{\pgfqpoint{2.933184in}{3.570392in}}%
\pgfpathlineto{\pgfqpoint{2.944202in}{3.575569in}}%
\pgfpathlineto{\pgfqpoint{2.955215in}{3.581047in}}%
\pgfpathlineto{\pgfqpoint{2.949238in}{3.582329in}}%
\pgfpathlineto{\pgfqpoint{2.943266in}{3.583656in}}%
\pgfpathlineto{\pgfqpoint{2.937299in}{3.585027in}}%
\pgfpathlineto{\pgfqpoint{2.931338in}{3.586444in}}%
\pgfpathlineto{\pgfqpoint{2.925383in}{3.587907in}}%
\pgfpathlineto{\pgfqpoint{2.914391in}{3.582427in}}%
\pgfpathlineto{\pgfqpoint{2.903394in}{3.577224in}}%
\pgfpathlineto{\pgfqpoint{2.892392in}{3.572381in}}%
\pgfpathlineto{\pgfqpoint{2.881384in}{3.567852in}}%
\pgfpathlineto{\pgfqpoint{2.870371in}{3.563330in}}%
\pgfpathlineto{\pgfqpoint{2.876306in}{3.561948in}}%
\pgfpathlineto{\pgfqpoint{2.882246in}{3.560591in}}%
\pgfpathlineto{\pgfqpoint{2.888191in}{3.559265in}}%
\pgfpathlineto{\pgfqpoint{2.894142in}{3.557973in}}%
\pgfpathclose%
\pgfusepath{stroke,fill}%
\end{pgfscope}%
\begin{pgfscope}%
\pgfpathrectangle{\pgfqpoint{0.887500in}{0.275000in}}{\pgfqpoint{4.225000in}{4.225000in}}%
\pgfusepath{clip}%
\pgfsetbuttcap%
\pgfsetroundjoin%
\definecolor{currentfill}{rgb}{0.239374,0.735588,0.455688}%
\pgfsetfillcolor{currentfill}%
\pgfsetfillopacity{0.700000}%
\pgfsetlinewidth{0.501875pt}%
\definecolor{currentstroke}{rgb}{1.000000,1.000000,1.000000}%
\pgfsetstrokecolor{currentstroke}%
\pgfsetstrokeopacity{0.500000}%
\pgfsetdash{}{0pt}%
\pgfpathmoveto{\pgfqpoint{2.625250in}{3.216314in}}%
\pgfpathlineto{\pgfqpoint{2.636250in}{3.228779in}}%
\pgfpathlineto{\pgfqpoint{2.647235in}{3.242658in}}%
\pgfpathlineto{\pgfqpoint{2.658211in}{3.257637in}}%
\pgfpathlineto{\pgfqpoint{2.669180in}{3.273398in}}%
\pgfpathlineto{\pgfqpoint{2.680146in}{3.289626in}}%
\pgfpathlineto{\pgfqpoint{2.674301in}{3.290684in}}%
\pgfpathlineto{\pgfqpoint{2.668452in}{3.292697in}}%
\pgfpathlineto{\pgfqpoint{2.662599in}{3.295609in}}%
\pgfpathlineto{\pgfqpoint{2.656744in}{3.299367in}}%
\pgfpathlineto{\pgfqpoint{2.650886in}{3.303919in}}%
\pgfpathlineto{\pgfqpoint{2.639961in}{3.286348in}}%
\pgfpathlineto{\pgfqpoint{2.629032in}{3.269320in}}%
\pgfpathlineto{\pgfqpoint{2.618095in}{3.253358in}}%
\pgfpathlineto{\pgfqpoint{2.607142in}{3.238982in}}%
\pgfpathlineto{\pgfqpoint{2.596166in}{3.226712in}}%
\pgfpathlineto{\pgfqpoint{2.601995in}{3.222567in}}%
\pgfpathlineto{\pgfqpoint{2.607817in}{3.219477in}}%
\pgfpathlineto{\pgfqpoint{2.613634in}{3.217420in}}%
\pgfpathlineto{\pgfqpoint{2.619444in}{3.216374in}}%
\pgfpathclose%
\pgfusepath{stroke,fill}%
\end{pgfscope}%
\begin{pgfscope}%
\pgfpathrectangle{\pgfqpoint{0.887500in}{0.275000in}}{\pgfqpoint{4.225000in}{4.225000in}}%
\pgfusepath{clip}%
\pgfsetbuttcap%
\pgfsetroundjoin%
\definecolor{currentfill}{rgb}{0.196571,0.711827,0.479221}%
\pgfsetfillcolor{currentfill}%
\pgfsetfillopacity{0.700000}%
\pgfsetlinewidth{0.501875pt}%
\definecolor{currentstroke}{rgb}{1.000000,1.000000,1.000000}%
\pgfsetstrokecolor{currentstroke}%
\pgfsetstrokeopacity{0.500000}%
\pgfsetdash{}{0pt}%
\pgfpathmoveto{\pgfqpoint{2.569990in}{3.175410in}}%
\pgfpathlineto{\pgfqpoint{2.581071in}{3.181538in}}%
\pgfpathlineto{\pgfqpoint{2.592140in}{3.188418in}}%
\pgfpathlineto{\pgfqpoint{2.603194in}{3.196324in}}%
\pgfpathlineto{\pgfqpoint{2.614232in}{3.205531in}}%
\pgfpathlineto{\pgfqpoint{2.625250in}{3.216314in}}%
\pgfpathlineto{\pgfqpoint{2.619444in}{3.216374in}}%
\pgfpathlineto{\pgfqpoint{2.613634in}{3.217420in}}%
\pgfpathlineto{\pgfqpoint{2.607817in}{3.219477in}}%
\pgfpathlineto{\pgfqpoint{2.601995in}{3.222567in}}%
\pgfpathlineto{\pgfqpoint{2.596166in}{3.226712in}}%
\pgfpathlineto{\pgfqpoint{2.585159in}{3.216985in}}%
\pgfpathlineto{\pgfqpoint{2.574123in}{3.209608in}}%
\pgfpathlineto{\pgfqpoint{2.563061in}{3.204100in}}%
\pgfpathlineto{\pgfqpoint{2.551978in}{3.199982in}}%
\pgfpathlineto{\pgfqpoint{2.540878in}{3.196772in}}%
\pgfpathlineto{\pgfqpoint{2.546708in}{3.191179in}}%
\pgfpathlineto{\pgfqpoint{2.552533in}{3.186283in}}%
\pgfpathlineto{\pgfqpoint{2.558354in}{3.182047in}}%
\pgfpathlineto{\pgfqpoint{2.564173in}{3.178434in}}%
\pgfpathclose%
\pgfusepath{stroke,fill}%
\end{pgfscope}%
\begin{pgfscope}%
\pgfpathrectangle{\pgfqpoint{0.887500in}{0.275000in}}{\pgfqpoint{4.225000in}{4.225000in}}%
\pgfusepath{clip}%
\pgfsetbuttcap%
\pgfsetroundjoin%
\definecolor{currentfill}{rgb}{0.208030,0.718701,0.472873}%
\pgfsetfillcolor{currentfill}%
\pgfsetfillopacity{0.700000}%
\pgfsetlinewidth{0.501875pt}%
\definecolor{currentstroke}{rgb}{1.000000,1.000000,1.000000}%
\pgfsetstrokecolor{currentstroke}%
\pgfsetstrokeopacity{0.500000}%
\pgfsetdash{}{0pt}%
\pgfpathmoveto{\pgfqpoint{2.344768in}{3.183387in}}%
\pgfpathlineto{\pgfqpoint{2.355912in}{3.187812in}}%
\pgfpathlineto{\pgfqpoint{2.367063in}{3.191643in}}%
\pgfpathlineto{\pgfqpoint{2.378215in}{3.195074in}}%
\pgfpathlineto{\pgfqpoint{2.389365in}{3.198301in}}%
\pgfpathlineto{\pgfqpoint{2.400511in}{3.201515in}}%
\pgfpathlineto{\pgfqpoint{2.394665in}{3.211337in}}%
\pgfpathlineto{\pgfqpoint{2.388811in}{3.221907in}}%
\pgfpathlineto{\pgfqpoint{2.382945in}{3.233297in}}%
\pgfpathlineto{\pgfqpoint{2.377067in}{3.245580in}}%
\pgfpathlineto{\pgfqpoint{2.365909in}{3.243458in}}%
\pgfpathlineto{\pgfqpoint{2.354745in}{3.241317in}}%
\pgfpathlineto{\pgfqpoint{2.343581in}{3.238832in}}%
\pgfpathlineto{\pgfqpoint{2.332424in}{3.235682in}}%
\pgfpathlineto{\pgfqpoint{2.321279in}{3.231542in}}%
\pgfpathlineto{\pgfqpoint{2.327181in}{3.217656in}}%
\pgfpathlineto{\pgfqpoint{2.333061in}{3.205076in}}%
\pgfpathlineto{\pgfqpoint{2.338923in}{3.193691in}}%
\pgfpathclose%
\pgfusepath{stroke,fill}%
\end{pgfscope}%
\begin{pgfscope}%
\pgfpathrectangle{\pgfqpoint{0.887500in}{0.275000in}}{\pgfqpoint{4.225000in}{4.225000in}}%
\pgfusepath{clip}%
\pgfsetbuttcap%
\pgfsetroundjoin%
\definecolor{currentfill}{rgb}{0.327796,0.773980,0.406640}%
\pgfsetfillcolor{currentfill}%
\pgfsetfillopacity{0.700000}%
\pgfsetlinewidth{0.501875pt}%
\definecolor{currentstroke}{rgb}{1.000000,1.000000,1.000000}%
\pgfsetstrokecolor{currentstroke}%
\pgfsetstrokeopacity{0.500000}%
\pgfsetdash{}{0pt}%
\pgfpathmoveto{\pgfqpoint{2.680146in}{3.289626in}}%
\pgfpathlineto{\pgfqpoint{2.691113in}{3.306003in}}%
\pgfpathlineto{\pgfqpoint{2.702083in}{3.322214in}}%
\pgfpathlineto{\pgfqpoint{2.713058in}{3.338107in}}%
\pgfpathlineto{\pgfqpoint{2.724037in}{3.353772in}}%
\pgfpathlineto{\pgfqpoint{2.735018in}{3.369320in}}%
\pgfpathlineto{\pgfqpoint{2.729135in}{3.371343in}}%
\pgfpathlineto{\pgfqpoint{2.723252in}{3.374108in}}%
\pgfpathlineto{\pgfqpoint{2.717368in}{3.377524in}}%
\pgfpathlineto{\pgfqpoint{2.711485in}{3.381496in}}%
\pgfpathlineto{\pgfqpoint{2.705602in}{3.385933in}}%
\pgfpathlineto{\pgfqpoint{2.694645in}{3.370663in}}%
\pgfpathlineto{\pgfqpoint{2.683693in}{3.354944in}}%
\pgfpathlineto{\pgfqpoint{2.672748in}{3.338605in}}%
\pgfpathlineto{\pgfqpoint{2.661813in}{3.321511in}}%
\pgfpathlineto{\pgfqpoint{2.650886in}{3.303919in}}%
\pgfpathlineto{\pgfqpoint{2.656744in}{3.299367in}}%
\pgfpathlineto{\pgfqpoint{2.662599in}{3.295609in}}%
\pgfpathlineto{\pgfqpoint{2.668452in}{3.292697in}}%
\pgfpathlineto{\pgfqpoint{2.674301in}{3.290684in}}%
\pgfpathclose%
\pgfusepath{stroke,fill}%
\end{pgfscope}%
\begin{pgfscope}%
\pgfpathrectangle{\pgfqpoint{0.887500in}{0.275000in}}{\pgfqpoint{4.225000in}{4.225000in}}%
\pgfusepath{clip}%
\pgfsetbuttcap%
\pgfsetroundjoin%
\definecolor{currentfill}{rgb}{0.906311,0.894855,0.098125}%
\pgfsetfillcolor{currentfill}%
\pgfsetfillopacity{0.700000}%
\pgfsetlinewidth{0.501875pt}%
\definecolor{currentstroke}{rgb}{1.000000,1.000000,1.000000}%
\pgfsetstrokecolor{currentstroke}%
\pgfsetstrokeopacity{0.500000}%
\pgfsetdash{}{0pt}%
\pgfpathmoveto{\pgfqpoint{3.430921in}{3.767283in}}%
\pgfpathlineto{\pgfqpoint{3.441866in}{3.773224in}}%
\pgfpathlineto{\pgfqpoint{3.452805in}{3.779070in}}%
\pgfpathlineto{\pgfqpoint{3.463740in}{3.784799in}}%
\pgfpathlineto{\pgfqpoint{3.474670in}{3.790387in}}%
\pgfpathlineto{\pgfqpoint{3.485595in}{3.795830in}}%
\pgfpathlineto{\pgfqpoint{3.479369in}{3.798066in}}%
\pgfpathlineto{\pgfqpoint{3.473148in}{3.800057in}}%
\pgfpathlineto{\pgfqpoint{3.466929in}{3.801824in}}%
\pgfpathlineto{\pgfqpoint{3.460715in}{3.803389in}}%
\pgfpathlineto{\pgfqpoint{3.454504in}{3.804773in}}%
\pgfpathlineto{\pgfqpoint{3.443599in}{3.799240in}}%
\pgfpathlineto{\pgfqpoint{3.432688in}{3.793552in}}%
\pgfpathlineto{\pgfqpoint{3.421772in}{3.787722in}}%
\pgfpathlineto{\pgfqpoint{3.410852in}{3.781775in}}%
\pgfpathlineto{\pgfqpoint{3.399927in}{3.775735in}}%
\pgfpathlineto{\pgfqpoint{3.406117in}{3.774377in}}%
\pgfpathlineto{\pgfqpoint{3.412312in}{3.772877in}}%
\pgfpathlineto{\pgfqpoint{3.418511in}{3.771210in}}%
\pgfpathlineto{\pgfqpoint{3.424714in}{3.769353in}}%
\pgfpathclose%
\pgfusepath{stroke,fill}%
\end{pgfscope}%
\begin{pgfscope}%
\pgfpathrectangle{\pgfqpoint{0.887500in}{0.275000in}}{\pgfqpoint{4.225000in}{4.225000in}}%
\pgfusepath{clip}%
\pgfsetbuttcap%
\pgfsetroundjoin%
\definecolor{currentfill}{rgb}{0.793760,0.880678,0.120005}%
\pgfsetfillcolor{currentfill}%
\pgfsetfillopacity{0.700000}%
\pgfsetlinewidth{0.501875pt}%
\definecolor{currentstroke}{rgb}{1.000000,1.000000,1.000000}%
\pgfsetstrokecolor{currentstroke}%
\pgfsetstrokeopacity{0.500000}%
\pgfsetdash{}{0pt}%
\pgfpathmoveto{\pgfqpoint{3.235621in}{3.682990in}}%
\pgfpathlineto{\pgfqpoint{3.246599in}{3.689145in}}%
\pgfpathlineto{\pgfqpoint{3.257574in}{3.695368in}}%
\pgfpathlineto{\pgfqpoint{3.268546in}{3.701630in}}%
\pgfpathlineto{\pgfqpoint{3.279514in}{3.707901in}}%
\pgfpathlineto{\pgfqpoint{3.290478in}{3.714151in}}%
\pgfpathlineto{\pgfqpoint{3.284336in}{3.715437in}}%
\pgfpathlineto{\pgfqpoint{3.278199in}{3.716685in}}%
\pgfpathlineto{\pgfqpoint{3.272069in}{3.717918in}}%
\pgfpathlineto{\pgfqpoint{3.265944in}{3.719155in}}%
\pgfpathlineto{\pgfqpoint{3.259825in}{3.720397in}}%
\pgfpathlineto{\pgfqpoint{3.248882in}{3.714215in}}%
\pgfpathlineto{\pgfqpoint{3.237936in}{3.708008in}}%
\pgfpathlineto{\pgfqpoint{3.226986in}{3.701791in}}%
\pgfpathlineto{\pgfqpoint{3.216032in}{3.695578in}}%
\pgfpathlineto{\pgfqpoint{3.205075in}{3.689383in}}%
\pgfpathlineto{\pgfqpoint{3.211173in}{3.688089in}}%
\pgfpathlineto{\pgfqpoint{3.217276in}{3.686812in}}%
\pgfpathlineto{\pgfqpoint{3.223385in}{3.685549in}}%
\pgfpathlineto{\pgfqpoint{3.229500in}{3.684282in}}%
\pgfpathclose%
\pgfusepath{stroke,fill}%
\end{pgfscope}%
\begin{pgfscope}%
\pgfpathrectangle{\pgfqpoint{0.887500in}{0.275000in}}{\pgfqpoint{4.225000in}{4.225000in}}%
\pgfusepath{clip}%
\pgfsetbuttcap%
\pgfsetroundjoin%
\definecolor{currentfill}{rgb}{0.191090,0.708366,0.482284}%
\pgfsetfillcolor{currentfill}%
\pgfsetfillopacity{0.700000}%
\pgfsetlinewidth{0.501875pt}%
\definecolor{currentstroke}{rgb}{1.000000,1.000000,1.000000}%
\pgfsetstrokecolor{currentstroke}%
\pgfsetstrokeopacity{0.500000}%
\pgfsetdash{}{0pt}%
\pgfpathmoveto{\pgfqpoint{2.429649in}{3.161065in}}%
\pgfpathlineto{\pgfqpoint{2.440800in}{3.164307in}}%
\pgfpathlineto{\pgfqpoint{2.451939in}{3.167942in}}%
\pgfpathlineto{\pgfqpoint{2.463069in}{3.171892in}}%
\pgfpathlineto{\pgfqpoint{2.474192in}{3.176008in}}%
\pgfpathlineto{\pgfqpoint{2.485310in}{3.180141in}}%
\pgfpathlineto{\pgfqpoint{2.479475in}{3.187429in}}%
\pgfpathlineto{\pgfqpoint{2.473638in}{3.195098in}}%
\pgfpathlineto{\pgfqpoint{2.467799in}{3.203188in}}%
\pgfpathlineto{\pgfqpoint{2.461957in}{3.211735in}}%
\pgfpathlineto{\pgfqpoint{2.456111in}{3.220779in}}%
\pgfpathlineto{\pgfqpoint{2.445004in}{3.216718in}}%
\pgfpathlineto{\pgfqpoint{2.433892in}{3.212624in}}%
\pgfpathlineto{\pgfqpoint{2.422775in}{3.208641in}}%
\pgfpathlineto{\pgfqpoint{2.411648in}{3.204913in}}%
\pgfpathlineto{\pgfqpoint{2.400511in}{3.201515in}}%
\pgfpathlineto{\pgfqpoint{2.406348in}{3.192368in}}%
\pgfpathlineto{\pgfqpoint{2.412179in}{3.183822in}}%
\pgfpathlineto{\pgfqpoint{2.418006in}{3.175804in}}%
\pgfpathlineto{\pgfqpoint{2.423829in}{3.168243in}}%
\pgfpathclose%
\pgfusepath{stroke,fill}%
\end{pgfscope}%
\begin{pgfscope}%
\pgfpathrectangle{\pgfqpoint{0.887500in}{0.275000in}}{\pgfqpoint{4.225000in}{4.225000in}}%
\pgfusepath{clip}%
\pgfsetbuttcap%
\pgfsetroundjoin%
\definecolor{currentfill}{rgb}{0.678489,0.863742,0.189503}%
\pgfsetfillcolor{currentfill}%
\pgfsetfillopacity{0.700000}%
\pgfsetlinewidth{0.501875pt}%
\definecolor{currentstroke}{rgb}{1.000000,1.000000,1.000000}%
\pgfsetstrokecolor{currentstroke}%
\pgfsetstrokeopacity{0.500000}%
\pgfsetdash{}{0pt}%
\pgfpathmoveto{\pgfqpoint{3.040302in}{3.603877in}}%
\pgfpathlineto{\pgfqpoint{3.051313in}{3.609367in}}%
\pgfpathlineto{\pgfqpoint{3.062321in}{3.614773in}}%
\pgfpathlineto{\pgfqpoint{3.073324in}{3.620132in}}%
\pgfpathlineto{\pgfqpoint{3.084324in}{3.625483in}}%
\pgfpathlineto{\pgfqpoint{3.095319in}{3.630867in}}%
\pgfpathlineto{\pgfqpoint{3.089270in}{3.631859in}}%
\pgfpathlineto{\pgfqpoint{3.083227in}{3.632927in}}%
\pgfpathlineto{\pgfqpoint{3.077189in}{3.634064in}}%
\pgfpathlineto{\pgfqpoint{3.071158in}{3.635266in}}%
\pgfpathlineto{\pgfqpoint{3.065132in}{3.636525in}}%
\pgfpathlineto{\pgfqpoint{3.054158in}{3.631132in}}%
\pgfpathlineto{\pgfqpoint{3.043179in}{3.625798in}}%
\pgfpathlineto{\pgfqpoint{3.032197in}{3.620474in}}%
\pgfpathlineto{\pgfqpoint{3.021211in}{3.615113in}}%
\pgfpathlineto{\pgfqpoint{3.010221in}{3.609664in}}%
\pgfpathlineto{\pgfqpoint{3.016225in}{3.608420in}}%
\pgfpathlineto{\pgfqpoint{3.022236in}{3.607219in}}%
\pgfpathlineto{\pgfqpoint{3.028252in}{3.606060in}}%
\pgfpathlineto{\pgfqpoint{3.034274in}{3.604946in}}%
\pgfpathclose%
\pgfusepath{stroke,fill}%
\end{pgfscope}%
\begin{pgfscope}%
\pgfpathrectangle{\pgfqpoint{0.887500in}{0.275000in}}{\pgfqpoint{4.225000in}{4.225000in}}%
\pgfusepath{clip}%
\pgfsetbuttcap%
\pgfsetroundjoin%
\definecolor{currentfill}{rgb}{0.421908,0.805774,0.351910}%
\pgfsetfillcolor{currentfill}%
\pgfsetfillopacity{0.700000}%
\pgfsetlinewidth{0.501875pt}%
\definecolor{currentstroke}{rgb}{1.000000,1.000000,1.000000}%
\pgfsetstrokecolor{currentstroke}%
\pgfsetstrokeopacity{0.500000}%
\pgfsetdash{}{0pt}%
\pgfpathmoveto{\pgfqpoint{2.735018in}{3.369320in}}%
\pgfpathlineto{\pgfqpoint{2.746001in}{3.384863in}}%
\pgfpathlineto{\pgfqpoint{2.756984in}{3.400512in}}%
\pgfpathlineto{\pgfqpoint{2.767968in}{3.416378in}}%
\pgfpathlineto{\pgfqpoint{2.778951in}{3.432566in}}%
\pgfpathlineto{\pgfqpoint{2.789935in}{3.448955in}}%
\pgfpathlineto{\pgfqpoint{2.784026in}{3.450645in}}%
\pgfpathlineto{\pgfqpoint{2.778120in}{3.452786in}}%
\pgfpathlineto{\pgfqpoint{2.772216in}{3.455308in}}%
\pgfpathlineto{\pgfqpoint{2.766316in}{3.458142in}}%
\pgfpathlineto{\pgfqpoint{2.760419in}{3.461218in}}%
\pgfpathlineto{\pgfqpoint{2.749455in}{3.445903in}}%
\pgfpathlineto{\pgfqpoint{2.738491in}{3.430742in}}%
\pgfpathlineto{\pgfqpoint{2.727527in}{3.415802in}}%
\pgfpathlineto{\pgfqpoint{2.716563in}{3.400923in}}%
\pgfpathlineto{\pgfqpoint{2.705602in}{3.385933in}}%
\pgfpathlineto{\pgfqpoint{2.711485in}{3.381496in}}%
\pgfpathlineto{\pgfqpoint{2.717368in}{3.377524in}}%
\pgfpathlineto{\pgfqpoint{2.723252in}{3.374108in}}%
\pgfpathlineto{\pgfqpoint{2.729135in}{3.371343in}}%
\pgfpathclose%
\pgfusepath{stroke,fill}%
\end{pgfscope}%
\begin{pgfscope}%
\pgfpathrectangle{\pgfqpoint{0.887500in}{0.275000in}}{\pgfqpoint{4.225000in}{4.225000in}}%
\pgfusepath{clip}%
\pgfsetbuttcap%
\pgfsetroundjoin%
\definecolor{currentfill}{rgb}{0.575563,0.844566,0.256415}%
\pgfsetfillcolor{currentfill}%
\pgfsetfillopacity{0.700000}%
\pgfsetlinewidth{0.501875pt}%
\definecolor{currentstroke}{rgb}{1.000000,1.000000,1.000000}%
\pgfsetstrokecolor{currentstroke}%
\pgfsetstrokeopacity{0.500000}%
\pgfsetdash{}{0pt}%
\pgfpathmoveto{\pgfqpoint{2.844933in}{3.521387in}}%
\pgfpathlineto{\pgfqpoint{2.855957in}{3.531451in}}%
\pgfpathlineto{\pgfqpoint{2.866989in}{3.539646in}}%
\pgfpathlineto{\pgfqpoint{2.878025in}{3.546339in}}%
\pgfpathlineto{\pgfqpoint{2.889062in}{3.551907in}}%
\pgfpathlineto{\pgfqpoint{2.900099in}{3.556724in}}%
\pgfpathlineto{\pgfqpoint{2.894142in}{3.557973in}}%
\pgfpathlineto{\pgfqpoint{2.888191in}{3.559265in}}%
\pgfpathlineto{\pgfqpoint{2.882246in}{3.560591in}}%
\pgfpathlineto{\pgfqpoint{2.876306in}{3.561948in}}%
\pgfpathlineto{\pgfqpoint{2.870371in}{3.563330in}}%
\pgfpathlineto{\pgfqpoint{2.859356in}{3.558476in}}%
\pgfpathlineto{\pgfqpoint{2.848340in}{3.552952in}}%
\pgfpathlineto{\pgfqpoint{2.837326in}{3.546419in}}%
\pgfpathlineto{\pgfqpoint{2.826315in}{3.538540in}}%
\pgfpathlineto{\pgfqpoint{2.815312in}{3.528982in}}%
\pgfpathlineto{\pgfqpoint{2.821226in}{3.527314in}}%
\pgfpathlineto{\pgfqpoint{2.827145in}{3.525692in}}%
\pgfpathlineto{\pgfqpoint{2.833069in}{3.524145in}}%
\pgfpathlineto{\pgfqpoint{2.838999in}{3.522700in}}%
\pgfpathclose%
\pgfusepath{stroke,fill}%
\end{pgfscope}%
\begin{pgfscope}%
\pgfpathrectangle{\pgfqpoint{0.887500in}{0.275000in}}{\pgfqpoint{4.225000in}{4.225000in}}%
\pgfusepath{clip}%
\pgfsetbuttcap%
\pgfsetroundjoin%
\definecolor{currentfill}{rgb}{0.964894,0.902323,0.123941}%
\pgfsetfillcolor{currentfill}%
\pgfsetfillopacity{0.700000}%
\pgfsetlinewidth{0.501875pt}%
\definecolor{currentstroke}{rgb}{1.000000,1.000000,1.000000}%
\pgfsetstrokecolor{currentstroke}%
\pgfsetstrokeopacity{0.500000}%
\pgfsetdash{}{0pt}%
\pgfpathmoveto{\pgfqpoint{3.571388in}{3.805637in}}%
\pgfpathlineto{\pgfqpoint{3.582301in}{3.810587in}}%
\pgfpathlineto{\pgfqpoint{3.593209in}{3.815523in}}%
\pgfpathlineto{\pgfqpoint{3.604113in}{3.820442in}}%
\pgfpathlineto{\pgfqpoint{3.615013in}{3.825343in}}%
\pgfpathlineto{\pgfqpoint{3.625908in}{3.830225in}}%
\pgfpathlineto{\pgfqpoint{3.619637in}{3.833918in}}%
\pgfpathlineto{\pgfqpoint{3.613366in}{3.837224in}}%
\pgfpathlineto{\pgfqpoint{3.607096in}{3.840123in}}%
\pgfpathlineto{\pgfqpoint{3.600825in}{3.842613in}}%
\pgfpathlineto{\pgfqpoint{3.594555in}{3.844716in}}%
\pgfpathlineto{\pgfqpoint{3.583682in}{3.840182in}}%
\pgfpathlineto{\pgfqpoint{3.572804in}{3.835571in}}%
\pgfpathlineto{\pgfqpoint{3.561921in}{3.830886in}}%
\pgfpathlineto{\pgfqpoint{3.551032in}{3.826129in}}%
\pgfpathlineto{\pgfqpoint{3.540139in}{3.821301in}}%
\pgfpathlineto{\pgfqpoint{3.546385in}{3.818797in}}%
\pgfpathlineto{\pgfqpoint{3.552634in}{3.815989in}}%
\pgfpathlineto{\pgfqpoint{3.558884in}{3.812859in}}%
\pgfpathlineto{\pgfqpoint{3.565135in}{3.809405in}}%
\pgfpathclose%
\pgfusepath{stroke,fill}%
\end{pgfscope}%
\begin{pgfscope}%
\pgfpathrectangle{\pgfqpoint{0.887500in}{0.275000in}}{\pgfqpoint{4.225000in}{4.225000in}}%
\pgfusepath{clip}%
\pgfsetbuttcap%
\pgfsetroundjoin%
\definecolor{currentfill}{rgb}{0.515992,0.831158,0.294279}%
\pgfsetfillcolor{currentfill}%
\pgfsetfillopacity{0.700000}%
\pgfsetlinewidth{0.501875pt}%
\definecolor{currentstroke}{rgb}{1.000000,1.000000,1.000000}%
\pgfsetstrokecolor{currentstroke}%
\pgfsetstrokeopacity{0.500000}%
\pgfsetdash{}{0pt}%
\pgfpathmoveto{\pgfqpoint{2.789935in}{3.448955in}}%
\pgfpathlineto{\pgfqpoint{2.800921in}{3.465195in}}%
\pgfpathlineto{\pgfqpoint{2.811913in}{3.480922in}}%
\pgfpathlineto{\pgfqpoint{2.822911in}{3.495773in}}%
\pgfpathlineto{\pgfqpoint{2.833918in}{3.509383in}}%
\pgfpathlineto{\pgfqpoint{2.844933in}{3.521387in}}%
\pgfpathlineto{\pgfqpoint{2.838999in}{3.522700in}}%
\pgfpathlineto{\pgfqpoint{2.833069in}{3.524145in}}%
\pgfpathlineto{\pgfqpoint{2.827145in}{3.525692in}}%
\pgfpathlineto{\pgfqpoint{2.821226in}{3.527314in}}%
\pgfpathlineto{\pgfqpoint{2.815312in}{3.528982in}}%
\pgfpathlineto{\pgfqpoint{2.804319in}{3.517679in}}%
\pgfpathlineto{\pgfqpoint{2.793334in}{3.504929in}}%
\pgfpathlineto{\pgfqpoint{2.782356in}{3.491053in}}%
\pgfpathlineto{\pgfqpoint{2.771385in}{3.476375in}}%
\pgfpathlineto{\pgfqpoint{2.760419in}{3.461218in}}%
\pgfpathlineto{\pgfqpoint{2.766316in}{3.458142in}}%
\pgfpathlineto{\pgfqpoint{2.772216in}{3.455308in}}%
\pgfpathlineto{\pgfqpoint{2.778120in}{3.452786in}}%
\pgfpathlineto{\pgfqpoint{2.784026in}{3.450645in}}%
\pgfpathclose%
\pgfusepath{stroke,fill}%
\end{pgfscope}%
\begin{pgfscope}%
\pgfpathrectangle{\pgfqpoint{0.887500in}{0.275000in}}{\pgfqpoint{4.225000in}{4.225000in}}%
\pgfusepath{clip}%
\pgfsetbuttcap%
\pgfsetroundjoin%
\definecolor{currentfill}{rgb}{0.993248,0.906157,0.143936}%
\pgfsetfillcolor{currentfill}%
\pgfsetfillopacity{0.700000}%
\pgfsetlinewidth{0.501875pt}%
\definecolor{currentstroke}{rgb}{1.000000,1.000000,1.000000}%
\pgfsetstrokecolor{currentstroke}%
\pgfsetstrokeopacity{0.500000}%
\pgfsetdash{}{0pt}%
\pgfpathmoveto{\pgfqpoint{3.711747in}{3.830425in}}%
\pgfpathlineto{\pgfqpoint{3.722629in}{3.835092in}}%
\pgfpathlineto{\pgfqpoint{3.733506in}{3.839735in}}%
\pgfpathlineto{\pgfqpoint{3.744379in}{3.844355in}}%
\pgfpathlineto{\pgfqpoint{3.755247in}{3.848952in}}%
\pgfpathlineto{\pgfqpoint{3.748954in}{3.854629in}}%
\pgfpathlineto{\pgfqpoint{3.742660in}{3.859956in}}%
\pgfpathlineto{\pgfqpoint{3.736364in}{3.864863in}}%
\pgfpathlineto{\pgfqpoint{3.730065in}{3.869277in}}%
\pgfpathlineto{\pgfqpoint{3.723762in}{3.873129in}}%
\pgfpathlineto{\pgfqpoint{3.712908in}{3.868468in}}%
\pgfpathlineto{\pgfqpoint{3.702049in}{3.863778in}}%
\pgfpathlineto{\pgfqpoint{3.691185in}{3.859059in}}%
\pgfpathlineto{\pgfqpoint{3.680317in}{3.854314in}}%
\pgfpathlineto{\pgfqpoint{3.686605in}{3.850334in}}%
\pgfpathlineto{\pgfqpoint{3.692892in}{3.845910in}}%
\pgfpathlineto{\pgfqpoint{3.699177in}{3.841087in}}%
\pgfpathlineto{\pgfqpoint{3.705462in}{3.835910in}}%
\pgfpathclose%
\pgfusepath{stroke,fill}%
\end{pgfscope}%
\begin{pgfscope}%
\pgfpathrectangle{\pgfqpoint{0.887500in}{0.275000in}}{\pgfqpoint{4.225000in}{4.225000in}}%
\pgfusepath{clip}%
\pgfsetbuttcap%
\pgfsetroundjoin%
\definecolor{currentfill}{rgb}{0.185783,0.704891,0.485273}%
\pgfsetfillcolor{currentfill}%
\pgfsetfillopacity{0.700000}%
\pgfsetlinewidth{0.501875pt}%
\definecolor{currentstroke}{rgb}{1.000000,1.000000,1.000000}%
\pgfsetstrokecolor{currentstroke}%
\pgfsetstrokeopacity{0.500000}%
\pgfsetdash{}{0pt}%
\pgfpathmoveto{\pgfqpoint{2.514483in}{3.148104in}}%
\pgfpathlineto{\pgfqpoint{2.525595in}{3.153393in}}%
\pgfpathlineto{\pgfqpoint{2.536700in}{3.158837in}}%
\pgfpathlineto{\pgfqpoint{2.547801in}{3.164307in}}%
\pgfpathlineto{\pgfqpoint{2.558899in}{3.169758in}}%
\pgfpathlineto{\pgfqpoint{2.569990in}{3.175410in}}%
\pgfpathlineto{\pgfqpoint{2.564173in}{3.178434in}}%
\pgfpathlineto{\pgfqpoint{2.558354in}{3.182047in}}%
\pgfpathlineto{\pgfqpoint{2.552533in}{3.186283in}}%
\pgfpathlineto{\pgfqpoint{2.546708in}{3.191179in}}%
\pgfpathlineto{\pgfqpoint{2.540878in}{3.196772in}}%
\pgfpathlineto{\pgfqpoint{2.529768in}{3.193988in}}%
\pgfpathlineto{\pgfqpoint{2.518653in}{3.191150in}}%
\pgfpathlineto{\pgfqpoint{2.507538in}{3.187861in}}%
\pgfpathlineto{\pgfqpoint{2.496424in}{3.184142in}}%
\pgfpathlineto{\pgfqpoint{2.485310in}{3.180141in}}%
\pgfpathlineto{\pgfqpoint{2.491144in}{3.173198in}}%
\pgfpathlineto{\pgfqpoint{2.496977in}{3.166560in}}%
\pgfpathlineto{\pgfqpoint{2.502811in}{3.160191in}}%
\pgfpathlineto{\pgfqpoint{2.508646in}{3.154052in}}%
\pgfpathclose%
\pgfusepath{stroke,fill}%
\end{pgfscope}%
\begin{pgfscope}%
\pgfpathrectangle{\pgfqpoint{0.887500in}{0.275000in}}{\pgfqpoint{4.225000in}{4.225000in}}%
\pgfusepath{clip}%
\pgfsetbuttcap%
\pgfsetroundjoin%
\definecolor{currentfill}{rgb}{0.876168,0.891125,0.095250}%
\pgfsetfillcolor{currentfill}%
\pgfsetfillopacity{0.700000}%
\pgfsetlinewidth{0.501875pt}%
\definecolor{currentstroke}{rgb}{1.000000,1.000000,1.000000}%
\pgfsetstrokecolor{currentstroke}%
\pgfsetstrokeopacity{0.500000}%
\pgfsetdash{}{0pt}%
\pgfpathmoveto{\pgfqpoint{3.376141in}{3.736970in}}%
\pgfpathlineto{\pgfqpoint{3.387104in}{3.743048in}}%
\pgfpathlineto{\pgfqpoint{3.398064in}{3.749130in}}%
\pgfpathlineto{\pgfqpoint{3.409020in}{3.755212in}}%
\pgfpathlineto{\pgfqpoint{3.419973in}{3.761271in}}%
\pgfpathlineto{\pgfqpoint{3.430921in}{3.767283in}}%
\pgfpathlineto{\pgfqpoint{3.424714in}{3.769353in}}%
\pgfpathlineto{\pgfqpoint{3.418511in}{3.771210in}}%
\pgfpathlineto{\pgfqpoint{3.412312in}{3.772877in}}%
\pgfpathlineto{\pgfqpoint{3.406117in}{3.774377in}}%
\pgfpathlineto{\pgfqpoint{3.399927in}{3.775735in}}%
\pgfpathlineto{\pgfqpoint{3.388998in}{3.769628in}}%
\pgfpathlineto{\pgfqpoint{3.378066in}{3.763479in}}%
\pgfpathlineto{\pgfqpoint{3.367129in}{3.757311in}}%
\pgfpathlineto{\pgfqpoint{3.356190in}{3.751150in}}%
\pgfpathlineto{\pgfqpoint{3.345247in}{3.745001in}}%
\pgfpathlineto{\pgfqpoint{3.351416in}{3.743654in}}%
\pgfpathlineto{\pgfqpoint{3.357590in}{3.742202in}}%
\pgfpathlineto{\pgfqpoint{3.363769in}{3.740620in}}%
\pgfpathlineto{\pgfqpoint{3.369953in}{3.738884in}}%
\pgfpathclose%
\pgfusepath{stroke,fill}%
\end{pgfscope}%
\begin{pgfscope}%
\pgfpathrectangle{\pgfqpoint{0.887500in}{0.275000in}}{\pgfqpoint{4.225000in}{4.225000in}}%
\pgfusepath{clip}%
\pgfsetbuttcap%
\pgfsetroundjoin%
\definecolor{currentfill}{rgb}{0.762373,0.876424,0.137064}%
\pgfsetfillcolor{currentfill}%
\pgfsetfillopacity{0.700000}%
\pgfsetlinewidth{0.501875pt}%
\definecolor{currentstroke}{rgb}{1.000000,1.000000,1.000000}%
\pgfsetstrokecolor{currentstroke}%
\pgfsetstrokeopacity{0.500000}%
\pgfsetdash{}{0pt}%
\pgfpathmoveto{\pgfqpoint{3.180684in}{3.654071in}}%
\pgfpathlineto{\pgfqpoint{3.191678in}{3.659597in}}%
\pgfpathlineto{\pgfqpoint{3.202669in}{3.665232in}}%
\pgfpathlineto{\pgfqpoint{3.213656in}{3.671006in}}%
\pgfpathlineto{\pgfqpoint{3.224640in}{3.676934in}}%
\pgfpathlineto{\pgfqpoint{3.235621in}{3.682990in}}%
\pgfpathlineto{\pgfqpoint{3.229500in}{3.684282in}}%
\pgfpathlineto{\pgfqpoint{3.223385in}{3.685549in}}%
\pgfpathlineto{\pgfqpoint{3.217276in}{3.686812in}}%
\pgfpathlineto{\pgfqpoint{3.211173in}{3.688089in}}%
\pgfpathlineto{\pgfqpoint{3.205075in}{3.689383in}}%
\pgfpathlineto{\pgfqpoint{3.194115in}{3.683219in}}%
\pgfpathlineto{\pgfqpoint{3.183152in}{3.677099in}}%
\pgfpathlineto{\pgfqpoint{3.172185in}{3.671039in}}%
\pgfpathlineto{\pgfqpoint{3.161215in}{3.665045in}}%
\pgfpathlineto{\pgfqpoint{3.150241in}{3.659125in}}%
\pgfpathlineto{\pgfqpoint{3.156318in}{3.658020in}}%
\pgfpathlineto{\pgfqpoint{3.162400in}{3.656975in}}%
\pgfpathlineto{\pgfqpoint{3.168489in}{3.655987in}}%
\pgfpathlineto{\pgfqpoint{3.174583in}{3.655030in}}%
\pgfpathclose%
\pgfusepath{stroke,fill}%
\end{pgfscope}%
\begin{pgfscope}%
\pgfpathrectangle{\pgfqpoint{0.887500in}{0.275000in}}{\pgfqpoint{4.225000in}{4.225000in}}%
\pgfusepath{clip}%
\pgfsetbuttcap%
\pgfsetroundjoin%
\definecolor{currentfill}{rgb}{0.202219,0.715272,0.476084}%
\pgfsetfillcolor{currentfill}%
\pgfsetfillopacity{0.700000}%
\pgfsetlinewidth{0.501875pt}%
\definecolor{currentstroke}{rgb}{1.000000,1.000000,1.000000}%
\pgfsetstrokecolor{currentstroke}%
\pgfsetstrokeopacity{0.500000}%
\pgfsetdash{}{0pt}%
\pgfpathmoveto{\pgfqpoint{2.289183in}{3.150339in}}%
\pgfpathlineto{\pgfqpoint{2.300287in}{3.157984in}}%
\pgfpathlineto{\pgfqpoint{2.311395in}{3.165299in}}%
\pgfpathlineto{\pgfqpoint{2.322509in}{3.172093in}}%
\pgfpathlineto{\pgfqpoint{2.333632in}{3.178176in}}%
\pgfpathlineto{\pgfqpoint{2.344768in}{3.183387in}}%
\pgfpathlineto{\pgfqpoint{2.338923in}{3.193691in}}%
\pgfpathlineto{\pgfqpoint{2.333061in}{3.205076in}}%
\pgfpathlineto{\pgfqpoint{2.327181in}{3.217656in}}%
\pgfpathlineto{\pgfqpoint{2.321279in}{3.231542in}}%
\pgfpathlineto{\pgfqpoint{2.310154in}{3.226092in}}%
\pgfpathlineto{\pgfqpoint{2.299052in}{3.219195in}}%
\pgfpathlineto{\pgfqpoint{2.287969in}{3.211131in}}%
\pgfpathlineto{\pgfqpoint{2.276899in}{3.202231in}}%
\pgfpathlineto{\pgfqpoint{2.265838in}{3.192827in}}%
\pgfpathlineto{\pgfqpoint{2.271695in}{3.180909in}}%
\pgfpathlineto{\pgfqpoint{2.277537in}{3.169908in}}%
\pgfpathlineto{\pgfqpoint{2.283365in}{3.159743in}}%
\pgfpathclose%
\pgfusepath{stroke,fill}%
\end{pgfscope}%
\begin{pgfscope}%
\pgfpathrectangle{\pgfqpoint{0.887500in}{0.275000in}}{\pgfqpoint{4.225000in}{4.225000in}}%
\pgfusepath{clip}%
\pgfsetbuttcap%
\pgfsetroundjoin%
\definecolor{currentfill}{rgb}{0.657642,0.860219,0.203082}%
\pgfsetfillcolor{currentfill}%
\pgfsetfillopacity{0.700000}%
\pgfsetlinewidth{0.501875pt}%
\definecolor{currentstroke}{rgb}{1.000000,1.000000,1.000000}%
\pgfsetstrokecolor{currentstroke}%
\pgfsetstrokeopacity{0.500000}%
\pgfsetdash{}{0pt}%
\pgfpathmoveto{\pgfqpoint{2.985188in}{3.575278in}}%
\pgfpathlineto{\pgfqpoint{2.996219in}{3.580927in}}%
\pgfpathlineto{\pgfqpoint{3.007245in}{3.586702in}}%
\pgfpathlineto{\pgfqpoint{3.018268in}{3.592511in}}%
\pgfpathlineto{\pgfqpoint{3.029287in}{3.598261in}}%
\pgfpathlineto{\pgfqpoint{3.040302in}{3.603877in}}%
\pgfpathlineto{\pgfqpoint{3.034274in}{3.604946in}}%
\pgfpathlineto{\pgfqpoint{3.028252in}{3.606060in}}%
\pgfpathlineto{\pgfqpoint{3.022236in}{3.607219in}}%
\pgfpathlineto{\pgfqpoint{3.016225in}{3.608420in}}%
\pgfpathlineto{\pgfqpoint{3.010221in}{3.609664in}}%
\pgfpathlineto{\pgfqpoint{2.999227in}{3.604079in}}%
\pgfpathlineto{\pgfqpoint{2.988230in}{3.598341in}}%
\pgfpathlineto{\pgfqpoint{2.977229in}{3.592527in}}%
\pgfpathlineto{\pgfqpoint{2.966224in}{3.586731in}}%
\pgfpathlineto{\pgfqpoint{2.955215in}{3.581047in}}%
\pgfpathlineto{\pgfqpoint{2.961198in}{3.579808in}}%
\pgfpathlineto{\pgfqpoint{2.967187in}{3.578613in}}%
\pgfpathlineto{\pgfqpoint{2.973182in}{3.577459in}}%
\pgfpathlineto{\pgfqpoint{2.979182in}{3.576348in}}%
\pgfpathclose%
\pgfusepath{stroke,fill}%
\end{pgfscope}%
\begin{pgfscope}%
\pgfpathrectangle{\pgfqpoint{0.887500in}{0.275000in}}{\pgfqpoint{4.225000in}{4.225000in}}%
\pgfusepath{clip}%
\pgfsetbuttcap%
\pgfsetroundjoin%
\definecolor{currentfill}{rgb}{0.191090,0.708366,0.482284}%
\pgfsetfillcolor{currentfill}%
\pgfsetfillopacity{0.700000}%
\pgfsetlinewidth{0.501875pt}%
\definecolor{currentstroke}{rgb}{1.000000,1.000000,1.000000}%
\pgfsetstrokecolor{currentstroke}%
\pgfsetstrokeopacity{0.500000}%
\pgfsetdash{}{0pt}%
\pgfpathmoveto{\pgfqpoint{2.373827in}{3.144208in}}%
\pgfpathlineto{\pgfqpoint{2.384989in}{3.148356in}}%
\pgfpathlineto{\pgfqpoint{2.396156in}{3.151909in}}%
\pgfpathlineto{\pgfqpoint{2.407324in}{3.155073in}}%
\pgfpathlineto{\pgfqpoint{2.418489in}{3.158055in}}%
\pgfpathlineto{\pgfqpoint{2.429649in}{3.161065in}}%
\pgfpathlineto{\pgfqpoint{2.423829in}{3.168243in}}%
\pgfpathlineto{\pgfqpoint{2.418006in}{3.175804in}}%
\pgfpathlineto{\pgfqpoint{2.412179in}{3.183822in}}%
\pgfpathlineto{\pgfqpoint{2.406348in}{3.192368in}}%
\pgfpathlineto{\pgfqpoint{2.400511in}{3.201515in}}%
\pgfpathlineto{\pgfqpoint{2.389365in}{3.198301in}}%
\pgfpathlineto{\pgfqpoint{2.378215in}{3.195074in}}%
\pgfpathlineto{\pgfqpoint{2.367063in}{3.191643in}}%
\pgfpathlineto{\pgfqpoint{2.355912in}{3.187812in}}%
\pgfpathlineto{\pgfqpoint{2.344768in}{3.183387in}}%
\pgfpathlineto{\pgfqpoint{2.350598in}{3.174055in}}%
\pgfpathlineto{\pgfqpoint{2.356418in}{3.165583in}}%
\pgfpathlineto{\pgfqpoint{2.362227in}{3.157858in}}%
\pgfpathlineto{\pgfqpoint{2.368030in}{3.150770in}}%
\pgfpathclose%
\pgfusepath{stroke,fill}%
\end{pgfscope}%
\begin{pgfscope}%
\pgfpathrectangle{\pgfqpoint{0.887500in}{0.275000in}}{\pgfqpoint{4.225000in}{4.225000in}}%
\pgfusepath{clip}%
\pgfsetbuttcap%
\pgfsetroundjoin%
\definecolor{currentfill}{rgb}{0.945636,0.899815,0.112838}%
\pgfsetfillcolor{currentfill}%
\pgfsetfillopacity{0.700000}%
\pgfsetlinewidth{0.501875pt}%
\definecolor{currentstroke}{rgb}{1.000000,1.000000,1.000000}%
\pgfsetstrokecolor{currentstroke}%
\pgfsetstrokeopacity{0.500000}%
\pgfsetdash{}{0pt}%
\pgfpathmoveto{\pgfqpoint{3.516757in}{3.780320in}}%
\pgfpathlineto{\pgfqpoint{3.527693in}{3.785505in}}%
\pgfpathlineto{\pgfqpoint{3.538624in}{3.790615in}}%
\pgfpathlineto{\pgfqpoint{3.549550in}{3.795664in}}%
\pgfpathlineto{\pgfqpoint{3.560472in}{3.800667in}}%
\pgfpathlineto{\pgfqpoint{3.571388in}{3.805637in}}%
\pgfpathlineto{\pgfqpoint{3.565135in}{3.809405in}}%
\pgfpathlineto{\pgfqpoint{3.558884in}{3.812859in}}%
\pgfpathlineto{\pgfqpoint{3.552634in}{3.815989in}}%
\pgfpathlineto{\pgfqpoint{3.546385in}{3.818797in}}%
\pgfpathlineto{\pgfqpoint{3.540139in}{3.821301in}}%
\pgfpathlineto{\pgfqpoint{3.529240in}{3.816400in}}%
\pgfpathlineto{\pgfqpoint{3.518337in}{3.811414in}}%
\pgfpathlineto{\pgfqpoint{3.507428in}{3.806332in}}%
\pgfpathlineto{\pgfqpoint{3.496514in}{3.801141in}}%
\pgfpathlineto{\pgfqpoint{3.485595in}{3.795830in}}%
\pgfpathlineto{\pgfqpoint{3.491823in}{3.793329in}}%
\pgfpathlineto{\pgfqpoint{3.498053in}{3.790541in}}%
\pgfpathlineto{\pgfqpoint{3.504286in}{3.787445in}}%
\pgfpathlineto{\pgfqpoint{3.510521in}{3.784036in}}%
\pgfpathclose%
\pgfusepath{stroke,fill}%
\end{pgfscope}%
\begin{pgfscope}%
\pgfpathrectangle{\pgfqpoint{0.887500in}{0.275000in}}{\pgfqpoint{4.225000in}{4.225000in}}%
\pgfusepath{clip}%
\pgfsetbuttcap%
\pgfsetroundjoin%
\definecolor{currentfill}{rgb}{0.175707,0.697900,0.491033}%
\pgfsetfillcolor{currentfill}%
\pgfsetfillopacity{0.700000}%
\pgfsetlinewidth{0.501875pt}%
\definecolor{currentstroke}{rgb}{1.000000,1.000000,1.000000}%
\pgfsetstrokecolor{currentstroke}%
\pgfsetstrokeopacity{0.500000}%
\pgfsetdash{}{0pt}%
\pgfpathmoveto{\pgfqpoint{2.233593in}{3.113343in}}%
\pgfpathlineto{\pgfqpoint{2.244728in}{3.120174in}}%
\pgfpathlineto{\pgfqpoint{2.255854in}{3.127325in}}%
\pgfpathlineto{\pgfqpoint{2.266970in}{3.134819in}}%
\pgfpathlineto{\pgfqpoint{2.278078in}{3.142553in}}%
\pgfpathlineto{\pgfqpoint{2.289183in}{3.150339in}}%
\pgfpathlineto{\pgfqpoint{2.283365in}{3.159743in}}%
\pgfpathlineto{\pgfqpoint{2.277537in}{3.169908in}}%
\pgfpathlineto{\pgfqpoint{2.271695in}{3.180909in}}%
\pgfpathlineto{\pgfqpoint{2.265838in}{3.192827in}}%
\pgfpathlineto{\pgfqpoint{2.254778in}{3.183250in}}%
\pgfpathlineto{\pgfqpoint{2.243713in}{3.173833in}}%
\pgfpathlineto{\pgfqpoint{2.232636in}{3.164904in}}%
\pgfpathlineto{\pgfqpoint{2.221543in}{3.156637in}}%
\pgfpathlineto{\pgfqpoint{2.210435in}{3.148976in}}%
\pgfpathlineto{\pgfqpoint{2.216228in}{3.139665in}}%
\pgfpathlineto{\pgfqpoint{2.222018in}{3.130637in}}%
\pgfpathlineto{\pgfqpoint{2.227807in}{3.121871in}}%
\pgfpathclose%
\pgfusepath{stroke,fill}%
\end{pgfscope}%
\begin{pgfscope}%
\pgfpathrectangle{\pgfqpoint{0.887500in}{0.275000in}}{\pgfqpoint{4.225000in}{4.225000in}}%
\pgfusepath{clip}%
\pgfsetbuttcap%
\pgfsetroundjoin%
\definecolor{currentfill}{rgb}{0.983868,0.904867,0.136897}%
\pgfsetfillcolor{currentfill}%
\pgfsetfillopacity{0.700000}%
\pgfsetlinewidth{0.501875pt}%
\definecolor{currentstroke}{rgb}{1.000000,1.000000,1.000000}%
\pgfsetstrokecolor{currentstroke}%
\pgfsetstrokeopacity{0.500000}%
\pgfsetdash{}{0pt}%
\pgfpathmoveto{\pgfqpoint{3.657268in}{3.806727in}}%
\pgfpathlineto{\pgfqpoint{3.668173in}{3.811516in}}%
\pgfpathlineto{\pgfqpoint{3.679073in}{3.816279in}}%
\pgfpathlineto{\pgfqpoint{3.689969in}{3.821019in}}%
\pgfpathlineto{\pgfqpoint{3.700860in}{3.825734in}}%
\pgfpathlineto{\pgfqpoint{3.711747in}{3.830425in}}%
\pgfpathlineto{\pgfqpoint{3.705462in}{3.835910in}}%
\pgfpathlineto{\pgfqpoint{3.699177in}{3.841087in}}%
\pgfpathlineto{\pgfqpoint{3.692892in}{3.845910in}}%
\pgfpathlineto{\pgfqpoint{3.686605in}{3.850334in}}%
\pgfpathlineto{\pgfqpoint{3.680317in}{3.854314in}}%
\pgfpathlineto{\pgfqpoint{3.669444in}{3.849543in}}%
\pgfpathlineto{\pgfqpoint{3.658567in}{3.844748in}}%
\pgfpathlineto{\pgfqpoint{3.647685in}{3.839929in}}%
\pgfpathlineto{\pgfqpoint{3.636799in}{3.835087in}}%
\pgfpathlineto{\pgfqpoint{3.625908in}{3.830225in}}%
\pgfpathlineto{\pgfqpoint{3.632179in}{3.826168in}}%
\pgfpathlineto{\pgfqpoint{3.638450in}{3.821767in}}%
\pgfpathlineto{\pgfqpoint{3.644722in}{3.817046in}}%
\pgfpathlineto{\pgfqpoint{3.650994in}{3.812025in}}%
\pgfpathclose%
\pgfusepath{stroke,fill}%
\end{pgfscope}%
\begin{pgfscope}%
\pgfpathrectangle{\pgfqpoint{0.887500in}{0.275000in}}{\pgfqpoint{4.225000in}{4.225000in}}%
\pgfusepath{clip}%
\pgfsetbuttcap%
\pgfsetroundjoin%
\definecolor{currentfill}{rgb}{0.845561,0.887322,0.099702}%
\pgfsetfillcolor{currentfill}%
\pgfsetfillopacity{0.700000}%
\pgfsetlinewidth{0.501875pt}%
\definecolor{currentstroke}{rgb}{1.000000,1.000000,1.000000}%
\pgfsetstrokecolor{currentstroke}%
\pgfsetstrokeopacity{0.500000}%
\pgfsetdash{}{0pt}%
\pgfpathmoveto{\pgfqpoint{3.321270in}{3.706332in}}%
\pgfpathlineto{\pgfqpoint{3.332252in}{3.712525in}}%
\pgfpathlineto{\pgfqpoint{3.343230in}{3.718676in}}%
\pgfpathlineto{\pgfqpoint{3.354204in}{3.724794in}}%
\pgfpathlineto{\pgfqpoint{3.365174in}{3.730889in}}%
\pgfpathlineto{\pgfqpoint{3.376141in}{3.736970in}}%
\pgfpathlineto{\pgfqpoint{3.369953in}{3.738884in}}%
\pgfpathlineto{\pgfqpoint{3.363769in}{3.740620in}}%
\pgfpathlineto{\pgfqpoint{3.357590in}{3.742202in}}%
\pgfpathlineto{\pgfqpoint{3.351416in}{3.743654in}}%
\pgfpathlineto{\pgfqpoint{3.345247in}{3.745001in}}%
\pgfpathlineto{\pgfqpoint{3.334300in}{3.738857in}}%
\pgfpathlineto{\pgfqpoint{3.323350in}{3.732708in}}%
\pgfpathlineto{\pgfqpoint{3.312397in}{3.726547in}}%
\pgfpathlineto{\pgfqpoint{3.301439in}{3.720365in}}%
\pgfpathlineto{\pgfqpoint{3.290478in}{3.714151in}}%
\pgfpathlineto{\pgfqpoint{3.296626in}{3.712804in}}%
\pgfpathlineto{\pgfqpoint{3.302780in}{3.711371in}}%
\pgfpathlineto{\pgfqpoint{3.308938in}{3.709831in}}%
\pgfpathlineto{\pgfqpoint{3.315102in}{3.708159in}}%
\pgfpathclose%
\pgfusepath{stroke,fill}%
\end{pgfscope}%
\begin{pgfscope}%
\pgfpathrectangle{\pgfqpoint{0.887500in}{0.275000in}}{\pgfqpoint{4.225000in}{4.225000in}}%
\pgfusepath{clip}%
\pgfsetbuttcap%
\pgfsetroundjoin%
\definecolor{currentfill}{rgb}{0.741388,0.873449,0.149561}%
\pgfsetfillcolor{currentfill}%
\pgfsetfillopacity{0.700000}%
\pgfsetlinewidth{0.501875pt}%
\definecolor{currentstroke}{rgb}{1.000000,1.000000,1.000000}%
\pgfsetstrokecolor{currentstroke}%
\pgfsetstrokeopacity{0.500000}%
\pgfsetdash{}{0pt}%
\pgfpathmoveto{\pgfqpoint{3.125657in}{3.627053in}}%
\pgfpathlineto{\pgfqpoint{3.136670in}{3.632472in}}%
\pgfpathlineto{\pgfqpoint{3.147679in}{3.637855in}}%
\pgfpathlineto{\pgfqpoint{3.158685in}{3.643230in}}%
\pgfpathlineto{\pgfqpoint{3.169686in}{3.648625in}}%
\pgfpathlineto{\pgfqpoint{3.180684in}{3.654071in}}%
\pgfpathlineto{\pgfqpoint{3.174583in}{3.655030in}}%
\pgfpathlineto{\pgfqpoint{3.168489in}{3.655987in}}%
\pgfpathlineto{\pgfqpoint{3.162400in}{3.656975in}}%
\pgfpathlineto{\pgfqpoint{3.156318in}{3.658020in}}%
\pgfpathlineto{\pgfqpoint{3.150241in}{3.659125in}}%
\pgfpathlineto{\pgfqpoint{3.139264in}{3.653285in}}%
\pgfpathlineto{\pgfqpoint{3.128283in}{3.647534in}}%
\pgfpathlineto{\pgfqpoint{3.117299in}{3.641876in}}%
\pgfpathlineto{\pgfqpoint{3.106311in}{3.636321in}}%
\pgfpathlineto{\pgfqpoint{3.095319in}{3.630867in}}%
\pgfpathlineto{\pgfqpoint{3.101375in}{3.629956in}}%
\pgfpathlineto{\pgfqpoint{3.107436in}{3.629133in}}%
\pgfpathlineto{\pgfqpoint{3.113503in}{3.628396in}}%
\pgfpathlineto{\pgfqpoint{3.119577in}{3.627714in}}%
\pgfpathclose%
\pgfusepath{stroke,fill}%
\end{pgfscope}%
\begin{pgfscope}%
\pgfpathrectangle{\pgfqpoint{0.887500in}{0.275000in}}{\pgfqpoint{4.225000in}{4.225000in}}%
\pgfusepath{clip}%
\pgfsetbuttcap%
\pgfsetroundjoin%
\definecolor{currentfill}{rgb}{0.180653,0.701402,0.488189}%
\pgfsetfillcolor{currentfill}%
\pgfsetfillopacity{0.700000}%
\pgfsetlinewidth{0.501875pt}%
\definecolor{currentstroke}{rgb}{1.000000,1.000000,1.000000}%
\pgfsetstrokecolor{currentstroke}%
\pgfsetstrokeopacity{0.500000}%
\pgfsetdash{}{0pt}%
\pgfpathmoveto{\pgfqpoint{2.458763in}{3.128367in}}%
\pgfpathlineto{\pgfqpoint{2.469936in}{3.131053in}}%
\pgfpathlineto{\pgfqpoint{2.481092in}{3.134444in}}%
\pgfpathlineto{\pgfqpoint{2.492234in}{3.138502in}}%
\pgfpathlineto{\pgfqpoint{2.503364in}{3.143098in}}%
\pgfpathlineto{\pgfqpoint{2.514483in}{3.148104in}}%
\pgfpathlineto{\pgfqpoint{2.508646in}{3.154052in}}%
\pgfpathlineto{\pgfqpoint{2.502811in}{3.160191in}}%
\pgfpathlineto{\pgfqpoint{2.496977in}{3.166560in}}%
\pgfpathlineto{\pgfqpoint{2.491144in}{3.173198in}}%
\pgfpathlineto{\pgfqpoint{2.485310in}{3.180141in}}%
\pgfpathlineto{\pgfqpoint{2.474192in}{3.176008in}}%
\pgfpathlineto{\pgfqpoint{2.463069in}{3.171892in}}%
\pgfpathlineto{\pgfqpoint{2.451939in}{3.167942in}}%
\pgfpathlineto{\pgfqpoint{2.440800in}{3.164307in}}%
\pgfpathlineto{\pgfqpoint{2.429649in}{3.161065in}}%
\pgfpathlineto{\pgfqpoint{2.435469in}{3.154196in}}%
\pgfpathlineto{\pgfqpoint{2.441289in}{3.147565in}}%
\pgfpathlineto{\pgfqpoint{2.447111in}{3.141099in}}%
\pgfpathlineto{\pgfqpoint{2.452935in}{3.134724in}}%
\pgfpathclose%
\pgfusepath{stroke,fill}%
\end{pgfscope}%
\begin{pgfscope}%
\pgfpathrectangle{\pgfqpoint{0.887500in}{0.275000in}}{\pgfqpoint{4.225000in}{4.225000in}}%
\pgfusepath{clip}%
\pgfsetbuttcap%
\pgfsetroundjoin%
\definecolor{currentfill}{rgb}{0.636902,0.856542,0.216620}%
\pgfsetfillcolor{currentfill}%
\pgfsetfillopacity{0.700000}%
\pgfsetlinewidth{0.501875pt}%
\definecolor{currentstroke}{rgb}{1.000000,1.000000,1.000000}%
\pgfsetstrokecolor{currentstroke}%
\pgfsetstrokeopacity{0.500000}%
\pgfsetdash{}{0pt}%
\pgfpathmoveto{\pgfqpoint{2.929965in}{3.551297in}}%
\pgfpathlineto{\pgfqpoint{2.941020in}{3.555632in}}%
\pgfpathlineto{\pgfqpoint{2.952070in}{3.560001in}}%
\pgfpathlineto{\pgfqpoint{2.963114in}{3.564723in}}%
\pgfpathlineto{\pgfqpoint{2.974154in}{3.569846in}}%
\pgfpathlineto{\pgfqpoint{2.985188in}{3.575278in}}%
\pgfpathlineto{\pgfqpoint{2.979182in}{3.576348in}}%
\pgfpathlineto{\pgfqpoint{2.973182in}{3.577459in}}%
\pgfpathlineto{\pgfqpoint{2.967187in}{3.578613in}}%
\pgfpathlineto{\pgfqpoint{2.961198in}{3.579808in}}%
\pgfpathlineto{\pgfqpoint{2.955215in}{3.581047in}}%
\pgfpathlineto{\pgfqpoint{2.944202in}{3.575569in}}%
\pgfpathlineto{\pgfqpoint{2.933184in}{3.570392in}}%
\pgfpathlineto{\pgfqpoint{2.922161in}{3.565608in}}%
\pgfpathlineto{\pgfqpoint{2.911132in}{3.561166in}}%
\pgfpathlineto{\pgfqpoint{2.900099in}{3.556724in}}%
\pgfpathlineto{\pgfqpoint{2.906061in}{3.555521in}}%
\pgfpathlineto{\pgfqpoint{2.912028in}{3.554372in}}%
\pgfpathlineto{\pgfqpoint{2.918002in}{3.553281in}}%
\pgfpathlineto{\pgfqpoint{2.923981in}{3.552254in}}%
\pgfpathclose%
\pgfusepath{stroke,fill}%
\end{pgfscope}%
\begin{pgfscope}%
\pgfpathrectangle{\pgfqpoint{0.887500in}{0.275000in}}{\pgfqpoint{4.225000in}{4.225000in}}%
\pgfusepath{clip}%
\pgfsetbuttcap%
\pgfsetroundjoin%
\definecolor{currentfill}{rgb}{0.180653,0.701402,0.488189}%
\pgfsetfillcolor{currentfill}%
\pgfsetfillopacity{0.700000}%
\pgfsetlinewidth{0.501875pt}%
\definecolor{currentstroke}{rgb}{1.000000,1.000000,1.000000}%
\pgfsetstrokecolor{currentstroke}%
\pgfsetstrokeopacity{0.500000}%
\pgfsetdash{}{0pt}%
\pgfpathmoveto{\pgfqpoint{2.318157in}{3.111999in}}%
\pgfpathlineto{\pgfqpoint{2.329277in}{3.119602in}}%
\pgfpathlineto{\pgfqpoint{2.340402in}{3.126795in}}%
\pgfpathlineto{\pgfqpoint{2.351533in}{3.133405in}}%
\pgfpathlineto{\pgfqpoint{2.362675in}{3.139257in}}%
\pgfpathlineto{\pgfqpoint{2.373827in}{3.144208in}}%
\pgfpathlineto{\pgfqpoint{2.368030in}{3.150770in}}%
\pgfpathlineto{\pgfqpoint{2.362227in}{3.157858in}}%
\pgfpathlineto{\pgfqpoint{2.356418in}{3.165583in}}%
\pgfpathlineto{\pgfqpoint{2.350598in}{3.174055in}}%
\pgfpathlineto{\pgfqpoint{2.344768in}{3.183387in}}%
\pgfpathlineto{\pgfqpoint{2.333632in}{3.178176in}}%
\pgfpathlineto{\pgfqpoint{2.322509in}{3.172093in}}%
\pgfpathlineto{\pgfqpoint{2.311395in}{3.165299in}}%
\pgfpathlineto{\pgfqpoint{2.300287in}{3.157984in}}%
\pgfpathlineto{\pgfqpoint{2.289183in}{3.150339in}}%
\pgfpathlineto{\pgfqpoint{2.294990in}{3.141616in}}%
\pgfpathlineto{\pgfqpoint{2.300789in}{3.133499in}}%
\pgfpathlineto{\pgfqpoint{2.306583in}{3.125908in}}%
\pgfpathlineto{\pgfqpoint{2.312371in}{3.118768in}}%
\pgfpathclose%
\pgfusepath{stroke,fill}%
\end{pgfscope}%
\begin{pgfscope}%
\pgfpathrectangle{\pgfqpoint{0.887500in}{0.275000in}}{\pgfqpoint{4.225000in}{4.225000in}}%
\pgfusepath{clip}%
\pgfsetbuttcap%
\pgfsetroundjoin%
\definecolor{currentfill}{rgb}{0.162016,0.687316,0.499129}%
\pgfsetfillcolor{currentfill}%
\pgfsetfillopacity{0.700000}%
\pgfsetlinewidth{0.501875pt}%
\definecolor{currentstroke}{rgb}{1.000000,1.000000,1.000000}%
\pgfsetstrokecolor{currentstroke}%
\pgfsetstrokeopacity{0.500000}%
\pgfsetdash{}{0pt}%
\pgfpathmoveto{\pgfqpoint{2.177789in}{3.082751in}}%
\pgfpathlineto{\pgfqpoint{2.188965in}{3.088511in}}%
\pgfpathlineto{\pgfqpoint{2.200134in}{3.094415in}}%
\pgfpathlineto{\pgfqpoint{2.211295in}{3.100498in}}%
\pgfpathlineto{\pgfqpoint{2.222449in}{3.106796in}}%
\pgfpathlineto{\pgfqpoint{2.233593in}{3.113343in}}%
\pgfpathlineto{\pgfqpoint{2.227807in}{3.121871in}}%
\pgfpathlineto{\pgfqpoint{2.222018in}{3.130637in}}%
\pgfpathlineto{\pgfqpoint{2.216228in}{3.139665in}}%
\pgfpathlineto{\pgfqpoint{2.210435in}{3.148976in}}%
\pgfpathlineto{\pgfqpoint{2.199313in}{3.141845in}}%
\pgfpathlineto{\pgfqpoint{2.188178in}{3.135169in}}%
\pgfpathlineto{\pgfqpoint{2.177032in}{3.128870in}}%
\pgfpathlineto{\pgfqpoint{2.165875in}{3.122873in}}%
\pgfpathlineto{\pgfqpoint{2.154709in}{3.117102in}}%
\pgfpathlineto{\pgfqpoint{2.160470in}{3.108705in}}%
\pgfpathlineto{\pgfqpoint{2.166237in}{3.100171in}}%
\pgfpathlineto{\pgfqpoint{2.172010in}{3.091515in}}%
\pgfpathclose%
\pgfusepath{stroke,fill}%
\end{pgfscope}%
\begin{pgfscope}%
\pgfpathrectangle{\pgfqpoint{0.887500in}{0.275000in}}{\pgfqpoint{4.225000in}{4.225000in}}%
\pgfusepath{clip}%
\pgfsetbuttcap%
\pgfsetroundjoin%
\definecolor{currentfill}{rgb}{0.274149,0.751988,0.436601}%
\pgfsetfillcolor{currentfill}%
\pgfsetfillopacity{0.700000}%
\pgfsetlinewidth{0.501875pt}%
\definecolor{currentstroke}{rgb}{1.000000,1.000000,1.000000}%
\pgfsetstrokecolor{currentstroke}%
\pgfsetstrokeopacity{0.500000}%
\pgfsetdash{}{0pt}%
\pgfpathmoveto{\pgfqpoint{2.654229in}{3.230049in}}%
\pgfpathlineto{\pgfqpoint{2.665250in}{3.243680in}}%
\pgfpathlineto{\pgfqpoint{2.676269in}{3.257595in}}%
\pgfpathlineto{\pgfqpoint{2.687287in}{3.271744in}}%
\pgfpathlineto{\pgfqpoint{2.698303in}{3.286076in}}%
\pgfpathlineto{\pgfqpoint{2.709320in}{3.300541in}}%
\pgfpathlineto{\pgfqpoint{2.703492in}{3.296014in}}%
\pgfpathlineto{\pgfqpoint{2.697661in}{3.292714in}}%
\pgfpathlineto{\pgfqpoint{2.691826in}{3.290586in}}%
\pgfpathlineto{\pgfqpoint{2.685988in}{3.289575in}}%
\pgfpathlineto{\pgfqpoint{2.680146in}{3.289626in}}%
\pgfpathlineto{\pgfqpoint{2.669180in}{3.273398in}}%
\pgfpathlineto{\pgfqpoint{2.658211in}{3.257637in}}%
\pgfpathlineto{\pgfqpoint{2.647235in}{3.242658in}}%
\pgfpathlineto{\pgfqpoint{2.636250in}{3.228779in}}%
\pgfpathlineto{\pgfqpoint{2.625250in}{3.216314in}}%
\pgfpathlineto{\pgfqpoint{2.631052in}{3.217220in}}%
\pgfpathlineto{\pgfqpoint{2.636850in}{3.219069in}}%
\pgfpathlineto{\pgfqpoint{2.642645in}{3.221838in}}%
\pgfpathlineto{\pgfqpoint{2.648438in}{3.225505in}}%
\pgfpathclose%
\pgfusepath{stroke,fill}%
\end{pgfscope}%
\begin{pgfscope}%
\pgfpathrectangle{\pgfqpoint{0.887500in}{0.275000in}}{\pgfqpoint{4.225000in}{4.225000in}}%
\pgfusepath{clip}%
\pgfsetbuttcap%
\pgfsetroundjoin%
\definecolor{currentfill}{rgb}{0.220124,0.725509,0.466226}%
\pgfsetfillcolor{currentfill}%
\pgfsetfillopacity{0.700000}%
\pgfsetlinewidth{0.501875pt}%
\definecolor{currentstroke}{rgb}{1.000000,1.000000,1.000000}%
\pgfsetstrokecolor{currentstroke}%
\pgfsetstrokeopacity{0.500000}%
\pgfsetdash{}{0pt}%
\pgfpathmoveto{\pgfqpoint{2.599064in}{3.167841in}}%
\pgfpathlineto{\pgfqpoint{2.610108in}{3.179336in}}%
\pgfpathlineto{\pgfqpoint{2.621145in}{3.191350in}}%
\pgfpathlineto{\pgfqpoint{2.632177in}{3.203837in}}%
\pgfpathlineto{\pgfqpoint{2.643205in}{3.216752in}}%
\pgfpathlineto{\pgfqpoint{2.654229in}{3.230049in}}%
\pgfpathlineto{\pgfqpoint{2.648438in}{3.225505in}}%
\pgfpathlineto{\pgfqpoint{2.642645in}{3.221838in}}%
\pgfpathlineto{\pgfqpoint{2.636850in}{3.219069in}}%
\pgfpathlineto{\pgfqpoint{2.631052in}{3.217220in}}%
\pgfpathlineto{\pgfqpoint{2.625250in}{3.216314in}}%
\pgfpathlineto{\pgfqpoint{2.614232in}{3.205531in}}%
\pgfpathlineto{\pgfqpoint{2.603194in}{3.196324in}}%
\pgfpathlineto{\pgfqpoint{2.592140in}{3.188418in}}%
\pgfpathlineto{\pgfqpoint{2.581071in}{3.181538in}}%
\pgfpathlineto{\pgfqpoint{2.569990in}{3.175410in}}%
\pgfpathlineto{\pgfqpoint{2.575805in}{3.172937in}}%
\pgfpathlineto{\pgfqpoint{2.581619in}{3.170980in}}%
\pgfpathlineto{\pgfqpoint{2.587433in}{3.169502in}}%
\pgfpathlineto{\pgfqpoint{2.593248in}{3.168468in}}%
\pgfpathclose%
\pgfusepath{stroke,fill}%
\end{pgfscope}%
\begin{pgfscope}%
\pgfpathrectangle{\pgfqpoint{0.887500in}{0.275000in}}{\pgfqpoint{4.225000in}{4.225000in}}%
\pgfusepath{clip}%
\pgfsetbuttcap%
\pgfsetroundjoin%
\definecolor{currentfill}{rgb}{0.926106,0.897330,0.104071}%
\pgfsetfillcolor{currentfill}%
\pgfsetfillopacity{0.700000}%
\pgfsetlinewidth{0.501875pt}%
\definecolor{currentstroke}{rgb}{1.000000,1.000000,1.000000}%
\pgfsetstrokecolor{currentstroke}%
\pgfsetstrokeopacity{0.500000}%
\pgfsetdash{}{0pt}%
\pgfpathmoveto{\pgfqpoint{3.462004in}{3.752956in}}%
\pgfpathlineto{\pgfqpoint{3.472964in}{3.758611in}}%
\pgfpathlineto{\pgfqpoint{3.483920in}{3.764186in}}%
\pgfpathlineto{\pgfqpoint{3.494871in}{3.769669in}}%
\pgfpathlineto{\pgfqpoint{3.505817in}{3.775046in}}%
\pgfpathlineto{\pgfqpoint{3.516757in}{3.780320in}}%
\pgfpathlineto{\pgfqpoint{3.510521in}{3.784036in}}%
\pgfpathlineto{\pgfqpoint{3.504286in}{3.787445in}}%
\pgfpathlineto{\pgfqpoint{3.498053in}{3.790541in}}%
\pgfpathlineto{\pgfqpoint{3.491823in}{3.793329in}}%
\pgfpathlineto{\pgfqpoint{3.485595in}{3.795830in}}%
\pgfpathlineto{\pgfqpoint{3.474670in}{3.790387in}}%
\pgfpathlineto{\pgfqpoint{3.463740in}{3.784799in}}%
\pgfpathlineto{\pgfqpoint{3.452805in}{3.779070in}}%
\pgfpathlineto{\pgfqpoint{3.441866in}{3.773224in}}%
\pgfpathlineto{\pgfqpoint{3.430921in}{3.767283in}}%
\pgfpathlineto{\pgfqpoint{3.437132in}{3.764975in}}%
\pgfpathlineto{\pgfqpoint{3.443346in}{3.762406in}}%
\pgfpathlineto{\pgfqpoint{3.449563in}{3.759552in}}%
\pgfpathlineto{\pgfqpoint{3.455782in}{3.756403in}}%
\pgfpathclose%
\pgfusepath{stroke,fill}%
\end{pgfscope}%
\begin{pgfscope}%
\pgfpathrectangle{\pgfqpoint{0.887500in}{0.275000in}}{\pgfqpoint{4.225000in}{4.225000in}}%
\pgfusepath{clip}%
\pgfsetbuttcap%
\pgfsetroundjoin%
\definecolor{currentfill}{rgb}{0.824940,0.884720,0.106217}%
\pgfsetfillcolor{currentfill}%
\pgfsetfillopacity{0.700000}%
\pgfsetlinewidth{0.501875pt}%
\definecolor{currentstroke}{rgb}{1.000000,1.000000,1.000000}%
\pgfsetstrokecolor{currentstroke}%
\pgfsetstrokeopacity{0.500000}%
\pgfsetdash{}{0pt}%
\pgfpathmoveto{\pgfqpoint{3.266308in}{3.675368in}}%
\pgfpathlineto{\pgfqpoint{3.277307in}{3.681417in}}%
\pgfpathlineto{\pgfqpoint{3.288303in}{3.687585in}}%
\pgfpathlineto{\pgfqpoint{3.299295in}{3.693825in}}%
\pgfpathlineto{\pgfqpoint{3.310284in}{3.700089in}}%
\pgfpathlineto{\pgfqpoint{3.321270in}{3.706332in}}%
\pgfpathlineto{\pgfqpoint{3.315102in}{3.708159in}}%
\pgfpathlineto{\pgfqpoint{3.308938in}{3.709831in}}%
\pgfpathlineto{\pgfqpoint{3.302780in}{3.711371in}}%
\pgfpathlineto{\pgfqpoint{3.296626in}{3.712804in}}%
\pgfpathlineto{\pgfqpoint{3.290478in}{3.714151in}}%
\pgfpathlineto{\pgfqpoint{3.279514in}{3.707901in}}%
\pgfpathlineto{\pgfqpoint{3.268546in}{3.701630in}}%
\pgfpathlineto{\pgfqpoint{3.257574in}{3.695368in}}%
\pgfpathlineto{\pgfqpoint{3.246599in}{3.689145in}}%
\pgfpathlineto{\pgfqpoint{3.235621in}{3.682990in}}%
\pgfpathlineto{\pgfqpoint{3.241748in}{3.681650in}}%
\pgfpathlineto{\pgfqpoint{3.247880in}{3.680239in}}%
\pgfpathlineto{\pgfqpoint{3.254018in}{3.678737in}}%
\pgfpathlineto{\pgfqpoint{3.260160in}{3.677121in}}%
\pgfpathclose%
\pgfusepath{stroke,fill}%
\end{pgfscope}%
\begin{pgfscope}%
\pgfpathrectangle{\pgfqpoint{0.887500in}{0.275000in}}{\pgfqpoint{4.225000in}{4.225000in}}%
\pgfusepath{clip}%
\pgfsetbuttcap%
\pgfsetroundjoin%
\definecolor{currentfill}{rgb}{0.352360,0.783011,0.392636}%
\pgfsetfillcolor{currentfill}%
\pgfsetfillopacity{0.700000}%
\pgfsetlinewidth{0.501875pt}%
\definecolor{currentstroke}{rgb}{1.000000,1.000000,1.000000}%
\pgfsetstrokecolor{currentstroke}%
\pgfsetstrokeopacity{0.500000}%
\pgfsetdash{}{0pt}%
\pgfpathmoveto{\pgfqpoint{2.709320in}{3.300541in}}%
\pgfpathlineto{\pgfqpoint{2.720336in}{3.315087in}}%
\pgfpathlineto{\pgfqpoint{2.731354in}{3.329665in}}%
\pgfpathlineto{\pgfqpoint{2.742372in}{3.344258in}}%
\pgfpathlineto{\pgfqpoint{2.753391in}{3.358900in}}%
\pgfpathlineto{\pgfqpoint{2.764410in}{3.373628in}}%
\pgfpathlineto{\pgfqpoint{2.758535in}{3.370530in}}%
\pgfpathlineto{\pgfqpoint{2.752659in}{3.368645in}}%
\pgfpathlineto{\pgfqpoint{2.746780in}{3.367878in}}%
\pgfpathlineto{\pgfqpoint{2.740900in}{3.368134in}}%
\pgfpathlineto{\pgfqpoint{2.735018in}{3.369320in}}%
\pgfpathlineto{\pgfqpoint{2.724037in}{3.353772in}}%
\pgfpathlineto{\pgfqpoint{2.713058in}{3.338107in}}%
\pgfpathlineto{\pgfqpoint{2.702083in}{3.322214in}}%
\pgfpathlineto{\pgfqpoint{2.691113in}{3.306003in}}%
\pgfpathlineto{\pgfqpoint{2.680146in}{3.289626in}}%
\pgfpathlineto{\pgfqpoint{2.685988in}{3.289575in}}%
\pgfpathlineto{\pgfqpoint{2.691826in}{3.290586in}}%
\pgfpathlineto{\pgfqpoint{2.697661in}{3.292714in}}%
\pgfpathlineto{\pgfqpoint{2.703492in}{3.296014in}}%
\pgfpathclose%
\pgfusepath{stroke,fill}%
\end{pgfscope}%
\begin{pgfscope}%
\pgfpathrectangle{\pgfqpoint{0.887500in}{0.275000in}}{\pgfqpoint{4.225000in}{4.225000in}}%
\pgfusepath{clip}%
\pgfsetbuttcap%
\pgfsetroundjoin%
\definecolor{currentfill}{rgb}{0.964894,0.902323,0.123941}%
\pgfsetfillcolor{currentfill}%
\pgfsetfillopacity{0.700000}%
\pgfsetlinewidth{0.501875pt}%
\definecolor{currentstroke}{rgb}{1.000000,1.000000,1.000000}%
\pgfsetstrokecolor{currentstroke}%
\pgfsetstrokeopacity{0.500000}%
\pgfsetdash{}{0pt}%
\pgfpathmoveto{\pgfqpoint{3.602674in}{3.782402in}}%
\pgfpathlineto{\pgfqpoint{3.613602in}{3.787319in}}%
\pgfpathlineto{\pgfqpoint{3.624525in}{3.792209in}}%
\pgfpathlineto{\pgfqpoint{3.635444in}{3.797074in}}%
\pgfpathlineto{\pgfqpoint{3.646358in}{3.801913in}}%
\pgfpathlineto{\pgfqpoint{3.657268in}{3.806727in}}%
\pgfpathlineto{\pgfqpoint{3.650994in}{3.812025in}}%
\pgfpathlineto{\pgfqpoint{3.644722in}{3.817046in}}%
\pgfpathlineto{\pgfqpoint{3.638450in}{3.821767in}}%
\pgfpathlineto{\pgfqpoint{3.632179in}{3.826168in}}%
\pgfpathlineto{\pgfqpoint{3.625908in}{3.830225in}}%
\pgfpathlineto{\pgfqpoint{3.615013in}{3.825343in}}%
\pgfpathlineto{\pgfqpoint{3.604113in}{3.820442in}}%
\pgfpathlineto{\pgfqpoint{3.593209in}{3.815523in}}%
\pgfpathlineto{\pgfqpoint{3.582301in}{3.810587in}}%
\pgfpathlineto{\pgfqpoint{3.571388in}{3.805637in}}%
\pgfpathlineto{\pgfqpoint{3.577643in}{3.801563in}}%
\pgfpathlineto{\pgfqpoint{3.583899in}{3.797193in}}%
\pgfpathlineto{\pgfqpoint{3.590156in}{3.792536in}}%
\pgfpathlineto{\pgfqpoint{3.596414in}{3.787603in}}%
\pgfpathclose%
\pgfusepath{stroke,fill}%
\end{pgfscope}%
\begin{pgfscope}%
\pgfpathrectangle{\pgfqpoint{0.887500in}{0.275000in}}{\pgfqpoint{4.225000in}{4.225000in}}%
\pgfusepath{clip}%
\pgfsetbuttcap%
\pgfsetroundjoin%
\definecolor{currentfill}{rgb}{0.993248,0.906157,0.143936}%
\pgfsetfillcolor{currentfill}%
\pgfsetfillopacity{0.700000}%
\pgfsetlinewidth{0.501875pt}%
\definecolor{currentstroke}{rgb}{1.000000,1.000000,1.000000}%
\pgfsetstrokecolor{currentstroke}%
\pgfsetstrokeopacity{0.500000}%
\pgfsetdash{}{0pt}%
\pgfpathmoveto{\pgfqpoint{3.743191in}{3.799888in}}%
\pgfpathlineto{\pgfqpoint{3.754083in}{3.804394in}}%
\pgfpathlineto{\pgfqpoint{3.764971in}{3.808864in}}%
\pgfpathlineto{\pgfqpoint{3.775854in}{3.813301in}}%
\pgfpathlineto{\pgfqpoint{3.786731in}{3.817703in}}%
\pgfpathlineto{\pgfqpoint{3.780429in}{3.824167in}}%
\pgfpathlineto{\pgfqpoint{3.774130in}{3.830544in}}%
\pgfpathlineto{\pgfqpoint{3.767834in}{3.836838in}}%
\pgfpathlineto{\pgfqpoint{3.761540in}{3.842998in}}%
\pgfpathlineto{\pgfqpoint{3.755247in}{3.848952in}}%
\pgfpathlineto{\pgfqpoint{3.744379in}{3.844355in}}%
\pgfpathlineto{\pgfqpoint{3.733506in}{3.839735in}}%
\pgfpathlineto{\pgfqpoint{3.722629in}{3.835092in}}%
\pgfpathlineto{\pgfqpoint{3.711747in}{3.830425in}}%
\pgfpathlineto{\pgfqpoint{3.718032in}{3.824677in}}%
\pgfpathlineto{\pgfqpoint{3.724319in}{3.818712in}}%
\pgfpathlineto{\pgfqpoint{3.730607in}{3.812576in}}%
\pgfpathlineto{\pgfqpoint{3.736898in}{3.806302in}}%
\pgfpathclose%
\pgfusepath{stroke,fill}%
\end{pgfscope}%
\begin{pgfscope}%
\pgfpathrectangle{\pgfqpoint{0.887500in}{0.275000in}}{\pgfqpoint{4.225000in}{4.225000in}}%
\pgfusepath{clip}%
\pgfsetbuttcap%
\pgfsetroundjoin%
\definecolor{currentfill}{rgb}{0.709898,0.868751,0.169257}%
\pgfsetfillcolor{currentfill}%
\pgfsetfillopacity{0.700000}%
\pgfsetlinewidth{0.501875pt}%
\definecolor{currentstroke}{rgb}{1.000000,1.000000,1.000000}%
\pgfsetstrokecolor{currentstroke}%
\pgfsetstrokeopacity{0.500000}%
\pgfsetdash{}{0pt}%
\pgfpathmoveto{\pgfqpoint{3.070530in}{3.599185in}}%
\pgfpathlineto{\pgfqpoint{3.081563in}{3.604879in}}%
\pgfpathlineto{\pgfqpoint{3.092592in}{3.610508in}}%
\pgfpathlineto{\pgfqpoint{3.103618in}{3.616077in}}%
\pgfpathlineto{\pgfqpoint{3.114639in}{3.621590in}}%
\pgfpathlineto{\pgfqpoint{3.125657in}{3.627053in}}%
\pgfpathlineto{\pgfqpoint{3.119577in}{3.627714in}}%
\pgfpathlineto{\pgfqpoint{3.113503in}{3.628396in}}%
\pgfpathlineto{\pgfqpoint{3.107436in}{3.629133in}}%
\pgfpathlineto{\pgfqpoint{3.101375in}{3.629956in}}%
\pgfpathlineto{\pgfqpoint{3.095319in}{3.630867in}}%
\pgfpathlineto{\pgfqpoint{3.084324in}{3.625483in}}%
\pgfpathlineto{\pgfqpoint{3.073324in}{3.620132in}}%
\pgfpathlineto{\pgfqpoint{3.062321in}{3.614773in}}%
\pgfpathlineto{\pgfqpoint{3.051313in}{3.609367in}}%
\pgfpathlineto{\pgfqpoint{3.040302in}{3.603877in}}%
\pgfpathlineto{\pgfqpoint{3.046336in}{3.602854in}}%
\pgfpathlineto{\pgfqpoint{3.052375in}{3.601879in}}%
\pgfpathlineto{\pgfqpoint{3.058421in}{3.600950in}}%
\pgfpathlineto{\pgfqpoint{3.064472in}{3.600056in}}%
\pgfpathclose%
\pgfusepath{stroke,fill}%
\end{pgfscope}%
\begin{pgfscope}%
\pgfpathrectangle{\pgfqpoint{0.887500in}{0.275000in}}{\pgfqpoint{4.225000in}{4.225000in}}%
\pgfusepath{clip}%
\pgfsetbuttcap%
\pgfsetroundjoin%
\definecolor{currentfill}{rgb}{0.191090,0.708366,0.482284}%
\pgfsetfillcolor{currentfill}%
\pgfsetfillopacity{0.700000}%
\pgfsetlinewidth{0.501875pt}%
\definecolor{currentstroke}{rgb}{1.000000,1.000000,1.000000}%
\pgfsetstrokecolor{currentstroke}%
\pgfsetstrokeopacity{0.500000}%
\pgfsetdash{}{0pt}%
\pgfpathmoveto{\pgfqpoint{2.543712in}{3.119914in}}%
\pgfpathlineto{\pgfqpoint{2.554804in}{3.128020in}}%
\pgfpathlineto{\pgfqpoint{2.565884in}{3.136940in}}%
\pgfpathlineto{\pgfqpoint{2.576953in}{3.146595in}}%
\pgfpathlineto{\pgfqpoint{2.588012in}{3.156913in}}%
\pgfpathlineto{\pgfqpoint{2.599064in}{3.167841in}}%
\pgfpathlineto{\pgfqpoint{2.593248in}{3.168468in}}%
\pgfpathlineto{\pgfqpoint{2.587433in}{3.169502in}}%
\pgfpathlineto{\pgfqpoint{2.581619in}{3.170980in}}%
\pgfpathlineto{\pgfqpoint{2.575805in}{3.172937in}}%
\pgfpathlineto{\pgfqpoint{2.569990in}{3.175410in}}%
\pgfpathlineto{\pgfqpoint{2.558899in}{3.169758in}}%
\pgfpathlineto{\pgfqpoint{2.547801in}{3.164307in}}%
\pgfpathlineto{\pgfqpoint{2.536700in}{3.158837in}}%
\pgfpathlineto{\pgfqpoint{2.525595in}{3.153393in}}%
\pgfpathlineto{\pgfqpoint{2.514483in}{3.148104in}}%
\pgfpathlineto{\pgfqpoint{2.520322in}{3.142311in}}%
\pgfpathlineto{\pgfqpoint{2.526164in}{3.136633in}}%
\pgfpathlineto{\pgfqpoint{2.532009in}{3.131033in}}%
\pgfpathlineto{\pgfqpoint{2.537858in}{3.125473in}}%
\pgfpathclose%
\pgfusepath{stroke,fill}%
\end{pgfscope}%
\begin{pgfscope}%
\pgfpathrectangle{\pgfqpoint{0.887500in}{0.275000in}}{\pgfqpoint{4.225000in}{4.225000in}}%
\pgfusepath{clip}%
\pgfsetbuttcap%
\pgfsetroundjoin%
\definecolor{currentfill}{rgb}{0.449368,0.813768,0.335384}%
\pgfsetfillcolor{currentfill}%
\pgfsetfillopacity{0.700000}%
\pgfsetlinewidth{0.501875pt}%
\definecolor{currentstroke}{rgb}{1.000000,1.000000,1.000000}%
\pgfsetstrokecolor{currentstroke}%
\pgfsetstrokeopacity{0.500000}%
\pgfsetdash{}{0pt}%
\pgfpathmoveto{\pgfqpoint{2.764410in}{3.373628in}}%
\pgfpathlineto{\pgfqpoint{2.775430in}{3.388482in}}%
\pgfpathlineto{\pgfqpoint{2.786449in}{3.403499in}}%
\pgfpathlineto{\pgfqpoint{2.797469in}{3.418717in}}%
\pgfpathlineto{\pgfqpoint{2.808488in}{3.434168in}}%
\pgfpathlineto{\pgfqpoint{2.819509in}{3.449716in}}%
\pgfpathlineto{\pgfqpoint{2.813590in}{3.448102in}}%
\pgfpathlineto{\pgfqpoint{2.807674in}{3.447289in}}%
\pgfpathlineto{\pgfqpoint{2.801759in}{3.447207in}}%
\pgfpathlineto{\pgfqpoint{2.795846in}{3.447786in}}%
\pgfpathlineto{\pgfqpoint{2.789935in}{3.448955in}}%
\pgfpathlineto{\pgfqpoint{2.778951in}{3.432566in}}%
\pgfpathlineto{\pgfqpoint{2.767968in}{3.416378in}}%
\pgfpathlineto{\pgfqpoint{2.756984in}{3.400512in}}%
\pgfpathlineto{\pgfqpoint{2.746001in}{3.384863in}}%
\pgfpathlineto{\pgfqpoint{2.735018in}{3.369320in}}%
\pgfpathlineto{\pgfqpoint{2.740900in}{3.368134in}}%
\pgfpathlineto{\pgfqpoint{2.746780in}{3.367878in}}%
\pgfpathlineto{\pgfqpoint{2.752659in}{3.368645in}}%
\pgfpathlineto{\pgfqpoint{2.758535in}{3.370530in}}%
\pgfpathclose%
\pgfusepath{stroke,fill}%
\end{pgfscope}%
\begin{pgfscope}%
\pgfpathrectangle{\pgfqpoint{0.887500in}{0.275000in}}{\pgfqpoint{4.225000in}{4.225000in}}%
\pgfusepath{clip}%
\pgfsetbuttcap%
\pgfsetroundjoin%
\definecolor{currentfill}{rgb}{0.616293,0.852709,0.230052}%
\pgfsetfillcolor{currentfill}%
\pgfsetfillopacity{0.700000}%
\pgfsetlinewidth{0.501875pt}%
\definecolor{currentstroke}{rgb}{1.000000,1.000000,1.000000}%
\pgfsetstrokecolor{currentstroke}%
\pgfsetstrokeopacity{0.500000}%
\pgfsetdash{}{0pt}%
\pgfpathmoveto{\pgfqpoint{2.874677in}{3.517786in}}%
\pgfpathlineto{\pgfqpoint{2.885730in}{3.527238in}}%
\pgfpathlineto{\pgfqpoint{2.896787in}{3.534967in}}%
\pgfpathlineto{\pgfqpoint{2.907847in}{3.541317in}}%
\pgfpathlineto{\pgfqpoint{2.918907in}{3.546643in}}%
\pgfpathlineto{\pgfqpoint{2.929965in}{3.551297in}}%
\pgfpathlineto{\pgfqpoint{2.923981in}{3.552254in}}%
\pgfpathlineto{\pgfqpoint{2.918002in}{3.553281in}}%
\pgfpathlineto{\pgfqpoint{2.912028in}{3.554372in}}%
\pgfpathlineto{\pgfqpoint{2.906061in}{3.555521in}}%
\pgfpathlineto{\pgfqpoint{2.900099in}{3.556724in}}%
\pgfpathlineto{\pgfqpoint{2.889062in}{3.551907in}}%
\pgfpathlineto{\pgfqpoint{2.878025in}{3.546339in}}%
\pgfpathlineto{\pgfqpoint{2.866989in}{3.539646in}}%
\pgfpathlineto{\pgfqpoint{2.855957in}{3.531451in}}%
\pgfpathlineto{\pgfqpoint{2.844933in}{3.521387in}}%
\pgfpathlineto{\pgfqpoint{2.850872in}{3.520233in}}%
\pgfpathlineto{\pgfqpoint{2.856816in}{3.519268in}}%
\pgfpathlineto{\pgfqpoint{2.862765in}{3.518519in}}%
\pgfpathlineto{\pgfqpoint{2.868719in}{3.518016in}}%
\pgfpathclose%
\pgfusepath{stroke,fill}%
\end{pgfscope}%
\begin{pgfscope}%
\pgfpathrectangle{\pgfqpoint{0.887500in}{0.275000in}}{\pgfqpoint{4.225000in}{4.225000in}}%
\pgfusepath{clip}%
\pgfsetbuttcap%
\pgfsetroundjoin%
\definecolor{currentfill}{rgb}{0.166383,0.690856,0.496502}%
\pgfsetfillcolor{currentfill}%
\pgfsetfillopacity{0.700000}%
\pgfsetlinewidth{0.501875pt}%
\definecolor{currentstroke}{rgb}{1.000000,1.000000,1.000000}%
\pgfsetstrokecolor{currentstroke}%
\pgfsetstrokeopacity{0.500000}%
\pgfsetdash{}{0pt}%
\pgfpathmoveto{\pgfqpoint{2.262523in}{3.073481in}}%
\pgfpathlineto{\pgfqpoint{2.273662in}{3.080889in}}%
\pgfpathlineto{\pgfqpoint{2.284793in}{3.088473in}}%
\pgfpathlineto{\pgfqpoint{2.295918in}{3.096261in}}%
\pgfpathlineto{\pgfqpoint{2.307039in}{3.104161in}}%
\pgfpathlineto{\pgfqpoint{2.318157in}{3.111999in}}%
\pgfpathlineto{\pgfqpoint{2.312371in}{3.118768in}}%
\pgfpathlineto{\pgfqpoint{2.306583in}{3.125908in}}%
\pgfpathlineto{\pgfqpoint{2.300789in}{3.133499in}}%
\pgfpathlineto{\pgfqpoint{2.294990in}{3.141616in}}%
\pgfpathlineto{\pgfqpoint{2.289183in}{3.150339in}}%
\pgfpathlineto{\pgfqpoint{2.278078in}{3.142553in}}%
\pgfpathlineto{\pgfqpoint{2.266970in}{3.134819in}}%
\pgfpathlineto{\pgfqpoint{2.255854in}{3.127325in}}%
\pgfpathlineto{\pgfqpoint{2.244728in}{3.120174in}}%
\pgfpathlineto{\pgfqpoint{2.233593in}{3.113343in}}%
\pgfpathlineto{\pgfqpoint{2.239379in}{3.105030in}}%
\pgfpathlineto{\pgfqpoint{2.245164in}{3.096910in}}%
\pgfpathlineto{\pgfqpoint{2.250950in}{3.088961in}}%
\pgfpathlineto{\pgfqpoint{2.256736in}{3.081159in}}%
\pgfpathclose%
\pgfusepath{stroke,fill}%
\end{pgfscope}%
\begin{pgfscope}%
\pgfpathrectangle{\pgfqpoint{0.887500in}{0.275000in}}{\pgfqpoint{4.225000in}{4.225000in}}%
\pgfusepath{clip}%
\pgfsetbuttcap%
\pgfsetroundjoin%
\definecolor{currentfill}{rgb}{0.180653,0.701402,0.488189}%
\pgfsetfillcolor{currentfill}%
\pgfsetfillopacity{0.700000}%
\pgfsetlinewidth{0.501875pt}%
\definecolor{currentstroke}{rgb}{1.000000,1.000000,1.000000}%
\pgfsetstrokecolor{currentstroke}%
\pgfsetstrokeopacity{0.500000}%
\pgfsetdash{}{0pt}%
\pgfpathmoveto{\pgfqpoint{2.402811in}{3.115378in}}%
\pgfpathlineto{\pgfqpoint{2.413997in}{3.118968in}}%
\pgfpathlineto{\pgfqpoint{2.425191in}{3.121746in}}%
\pgfpathlineto{\pgfqpoint{2.436386in}{3.124028in}}%
\pgfpathlineto{\pgfqpoint{2.447579in}{3.126130in}}%
\pgfpathlineto{\pgfqpoint{2.458763in}{3.128367in}}%
\pgfpathlineto{\pgfqpoint{2.452935in}{3.134724in}}%
\pgfpathlineto{\pgfqpoint{2.447111in}{3.141099in}}%
\pgfpathlineto{\pgfqpoint{2.441289in}{3.147565in}}%
\pgfpathlineto{\pgfqpoint{2.435469in}{3.154196in}}%
\pgfpathlineto{\pgfqpoint{2.429649in}{3.161065in}}%
\pgfpathlineto{\pgfqpoint{2.418489in}{3.158055in}}%
\pgfpathlineto{\pgfqpoint{2.407324in}{3.155073in}}%
\pgfpathlineto{\pgfqpoint{2.396156in}{3.151909in}}%
\pgfpathlineto{\pgfqpoint{2.384989in}{3.148356in}}%
\pgfpathlineto{\pgfqpoint{2.373827in}{3.144208in}}%
\pgfpathlineto{\pgfqpoint{2.379622in}{3.138060in}}%
\pgfpathlineto{\pgfqpoint{2.385416in}{3.132214in}}%
\pgfpathlineto{\pgfqpoint{2.391211in}{3.126560in}}%
\pgfpathlineto{\pgfqpoint{2.397009in}{3.120985in}}%
\pgfpathclose%
\pgfusepath{stroke,fill}%
\end{pgfscope}%
\begin{pgfscope}%
\pgfpathrectangle{\pgfqpoint{0.887500in}{0.275000in}}{\pgfqpoint{4.225000in}{4.225000in}}%
\pgfusepath{clip}%
\pgfsetbuttcap%
\pgfsetroundjoin%
\definecolor{currentfill}{rgb}{0.150148,0.676631,0.506589}%
\pgfsetfillcolor{currentfill}%
\pgfsetfillopacity{0.700000}%
\pgfsetlinewidth{0.501875pt}%
\definecolor{currentstroke}{rgb}{1.000000,1.000000,1.000000}%
\pgfsetstrokecolor{currentstroke}%
\pgfsetstrokeopacity{0.500000}%
\pgfsetdash{}{0pt}%
\pgfpathmoveto{\pgfqpoint{2.121887in}{3.052355in}}%
\pgfpathlineto{\pgfqpoint{2.133061in}{3.059071in}}%
\pgfpathlineto{\pgfqpoint{2.144241in}{3.065341in}}%
\pgfpathlineto{\pgfqpoint{2.155424in}{3.071292in}}%
\pgfpathlineto{\pgfqpoint{2.166608in}{3.077053in}}%
\pgfpathlineto{\pgfqpoint{2.177789in}{3.082751in}}%
\pgfpathlineto{\pgfqpoint{2.172010in}{3.091515in}}%
\pgfpathlineto{\pgfqpoint{2.166237in}{3.100171in}}%
\pgfpathlineto{\pgfqpoint{2.160470in}{3.108705in}}%
\pgfpathlineto{\pgfqpoint{2.154709in}{3.117102in}}%
\pgfpathlineto{\pgfqpoint{2.143537in}{3.111442in}}%
\pgfpathlineto{\pgfqpoint{2.132362in}{3.105738in}}%
\pgfpathlineto{\pgfqpoint{2.121187in}{3.099835in}}%
\pgfpathlineto{\pgfqpoint{2.110017in}{3.093578in}}%
\pgfpathlineto{\pgfqpoint{2.098856in}{3.086810in}}%
\pgfpathlineto{\pgfqpoint{2.104593in}{3.078884in}}%
\pgfpathlineto{\pgfqpoint{2.110344in}{3.070470in}}%
\pgfpathlineto{\pgfqpoint{2.116109in}{3.061612in}}%
\pgfpathclose%
\pgfusepath{stroke,fill}%
\end{pgfscope}%
\begin{pgfscope}%
\pgfpathrectangle{\pgfqpoint{0.887500in}{0.275000in}}{\pgfqpoint{4.225000in}{4.225000in}}%
\pgfusepath{clip}%
\pgfsetbuttcap%
\pgfsetroundjoin%
\definecolor{currentfill}{rgb}{0.545524,0.838039,0.275626}%
\pgfsetfillcolor{currentfill}%
\pgfsetfillopacity{0.700000}%
\pgfsetlinewidth{0.501875pt}%
\definecolor{currentstroke}{rgb}{1.000000,1.000000,1.000000}%
\pgfsetstrokecolor{currentstroke}%
\pgfsetstrokeopacity{0.500000}%
\pgfsetdash{}{0pt}%
\pgfpathmoveto{\pgfqpoint{2.819509in}{3.449716in}}%
\pgfpathlineto{\pgfqpoint{2.830532in}{3.465049in}}%
\pgfpathlineto{\pgfqpoint{2.841559in}{3.479844in}}%
\pgfpathlineto{\pgfqpoint{2.852592in}{3.493780in}}%
\pgfpathlineto{\pgfqpoint{2.863631in}{3.506535in}}%
\pgfpathlineto{\pgfqpoint{2.874677in}{3.517786in}}%
\pgfpathlineto{\pgfqpoint{2.868719in}{3.518016in}}%
\pgfpathlineto{\pgfqpoint{2.862765in}{3.518519in}}%
\pgfpathlineto{\pgfqpoint{2.856816in}{3.519268in}}%
\pgfpathlineto{\pgfqpoint{2.850872in}{3.520233in}}%
\pgfpathlineto{\pgfqpoint{2.844933in}{3.521387in}}%
\pgfpathlineto{\pgfqpoint{2.833918in}{3.509383in}}%
\pgfpathlineto{\pgfqpoint{2.822911in}{3.495773in}}%
\pgfpathlineto{\pgfqpoint{2.811913in}{3.480922in}}%
\pgfpathlineto{\pgfqpoint{2.800921in}{3.465195in}}%
\pgfpathlineto{\pgfqpoint{2.789935in}{3.448955in}}%
\pgfpathlineto{\pgfqpoint{2.795846in}{3.447786in}}%
\pgfpathlineto{\pgfqpoint{2.801759in}{3.447207in}}%
\pgfpathlineto{\pgfqpoint{2.807674in}{3.447289in}}%
\pgfpathlineto{\pgfqpoint{2.813590in}{3.448102in}}%
\pgfpathclose%
\pgfusepath{stroke,fill}%
\end{pgfscope}%
\begin{pgfscope}%
\pgfpathrectangle{\pgfqpoint{0.887500in}{0.275000in}}{\pgfqpoint{4.225000in}{4.225000in}}%
\pgfusepath{clip}%
\pgfsetbuttcap%
\pgfsetroundjoin%
\definecolor{currentfill}{rgb}{0.896320,0.893616,0.096335}%
\pgfsetfillcolor{currentfill}%
\pgfsetfillopacity{0.700000}%
\pgfsetlinewidth{0.501875pt}%
\definecolor{currentstroke}{rgb}{1.000000,1.000000,1.000000}%
\pgfsetstrokecolor{currentstroke}%
\pgfsetstrokeopacity{0.500000}%
\pgfsetdash{}{0pt}%
\pgfpathmoveto{\pgfqpoint{3.407140in}{3.723939in}}%
\pgfpathlineto{\pgfqpoint{3.418121in}{3.729805in}}%
\pgfpathlineto{\pgfqpoint{3.429098in}{3.735646in}}%
\pgfpathlineto{\pgfqpoint{3.440071in}{3.741461in}}%
\pgfpathlineto{\pgfqpoint{3.451039in}{3.747235in}}%
\pgfpathlineto{\pgfqpoint{3.462004in}{3.752956in}}%
\pgfpathlineto{\pgfqpoint{3.455782in}{3.756403in}}%
\pgfpathlineto{\pgfqpoint{3.449563in}{3.759552in}}%
\pgfpathlineto{\pgfqpoint{3.443346in}{3.762406in}}%
\pgfpathlineto{\pgfqpoint{3.437132in}{3.764975in}}%
\pgfpathlineto{\pgfqpoint{3.430921in}{3.767283in}}%
\pgfpathlineto{\pgfqpoint{3.419973in}{3.761271in}}%
\pgfpathlineto{\pgfqpoint{3.409020in}{3.755212in}}%
\pgfpathlineto{\pgfqpoint{3.398064in}{3.749130in}}%
\pgfpathlineto{\pgfqpoint{3.387104in}{3.743048in}}%
\pgfpathlineto{\pgfqpoint{3.376141in}{3.736970in}}%
\pgfpathlineto{\pgfqpoint{3.382334in}{3.734855in}}%
\pgfpathlineto{\pgfqpoint{3.388530in}{3.732515in}}%
\pgfpathlineto{\pgfqpoint{3.394730in}{3.729925in}}%
\pgfpathlineto{\pgfqpoint{3.400933in}{3.727070in}}%
\pgfpathclose%
\pgfusepath{stroke,fill}%
\end{pgfscope}%
\begin{pgfscope}%
\pgfpathrectangle{\pgfqpoint{0.887500in}{0.275000in}}{\pgfqpoint{4.225000in}{4.225000in}}%
\pgfusepath{clip}%
\pgfsetbuttcap%
\pgfsetroundjoin%
\definecolor{currentfill}{rgb}{0.793760,0.880678,0.120005}%
\pgfsetfillcolor{currentfill}%
\pgfsetfillopacity{0.700000}%
\pgfsetlinewidth{0.501875pt}%
\definecolor{currentstroke}{rgb}{1.000000,1.000000,1.000000}%
\pgfsetstrokecolor{currentstroke}%
\pgfsetstrokeopacity{0.500000}%
\pgfsetdash{}{0pt}%
\pgfpathmoveto{\pgfqpoint{3.211273in}{3.648141in}}%
\pgfpathlineto{\pgfqpoint{3.222286in}{3.653195in}}%
\pgfpathlineto{\pgfqpoint{3.233296in}{3.658399in}}%
\pgfpathlineto{\pgfqpoint{3.244303in}{3.663817in}}%
\pgfpathlineto{\pgfqpoint{3.255307in}{3.669486in}}%
\pgfpathlineto{\pgfqpoint{3.266308in}{3.675368in}}%
\pgfpathlineto{\pgfqpoint{3.260160in}{3.677121in}}%
\pgfpathlineto{\pgfqpoint{3.254018in}{3.678737in}}%
\pgfpathlineto{\pgfqpoint{3.247880in}{3.680239in}}%
\pgfpathlineto{\pgfqpoint{3.241748in}{3.681650in}}%
\pgfpathlineto{\pgfqpoint{3.235621in}{3.682990in}}%
\pgfpathlineto{\pgfqpoint{3.224640in}{3.676934in}}%
\pgfpathlineto{\pgfqpoint{3.213656in}{3.671006in}}%
\pgfpathlineto{\pgfqpoint{3.202669in}{3.665232in}}%
\pgfpathlineto{\pgfqpoint{3.191678in}{3.659597in}}%
\pgfpathlineto{\pgfqpoint{3.180684in}{3.654071in}}%
\pgfpathlineto{\pgfqpoint{3.186791in}{3.653079in}}%
\pgfpathlineto{\pgfqpoint{3.192903in}{3.652023in}}%
\pgfpathlineto{\pgfqpoint{3.199021in}{3.650869in}}%
\pgfpathlineto{\pgfqpoint{3.205144in}{3.649585in}}%
\pgfpathclose%
\pgfusepath{stroke,fill}%
\end{pgfscope}%
\begin{pgfscope}%
\pgfpathrectangle{\pgfqpoint{0.887500in}{0.275000in}}{\pgfqpoint{4.225000in}{4.225000in}}%
\pgfusepath{clip}%
\pgfsetbuttcap%
\pgfsetroundjoin%
\definecolor{currentfill}{rgb}{0.974417,0.903590,0.130215}%
\pgfsetfillcolor{currentfill}%
\pgfsetfillopacity{0.700000}%
\pgfsetlinewidth{0.501875pt}%
\definecolor{currentstroke}{rgb}{1.000000,1.000000,1.000000}%
\pgfsetstrokecolor{currentstroke}%
\pgfsetstrokeopacity{0.500000}%
\pgfsetdash{}{0pt}%
\pgfpathmoveto{\pgfqpoint{3.688654in}{3.776803in}}%
\pgfpathlineto{\pgfqpoint{3.699571in}{3.781497in}}%
\pgfpathlineto{\pgfqpoint{3.710484in}{3.786151in}}%
\pgfpathlineto{\pgfqpoint{3.721391in}{3.790768in}}%
\pgfpathlineto{\pgfqpoint{3.732293in}{3.795346in}}%
\pgfpathlineto{\pgfqpoint{3.743191in}{3.799888in}}%
\pgfpathlineto{\pgfqpoint{3.736898in}{3.806302in}}%
\pgfpathlineto{\pgfqpoint{3.730607in}{3.812576in}}%
\pgfpathlineto{\pgfqpoint{3.724319in}{3.818712in}}%
\pgfpathlineto{\pgfqpoint{3.718032in}{3.824677in}}%
\pgfpathlineto{\pgfqpoint{3.711747in}{3.830425in}}%
\pgfpathlineto{\pgfqpoint{3.700860in}{3.825734in}}%
\pgfpathlineto{\pgfqpoint{3.689969in}{3.821019in}}%
\pgfpathlineto{\pgfqpoint{3.679073in}{3.816279in}}%
\pgfpathlineto{\pgfqpoint{3.668173in}{3.811516in}}%
\pgfpathlineto{\pgfqpoint{3.657268in}{3.806727in}}%
\pgfpathlineto{\pgfqpoint{3.663542in}{3.801173in}}%
\pgfpathlineto{\pgfqpoint{3.669818in}{3.795385in}}%
\pgfpathlineto{\pgfqpoint{3.676095in}{3.789385in}}%
\pgfpathlineto{\pgfqpoint{3.682374in}{3.783190in}}%
\pgfpathclose%
\pgfusepath{stroke,fill}%
\end{pgfscope}%
\begin{pgfscope}%
\pgfpathrectangle{\pgfqpoint{0.887500in}{0.275000in}}{\pgfqpoint{4.225000in}{4.225000in}}%
\pgfusepath{clip}%
\pgfsetbuttcap%
\pgfsetroundjoin%
\definecolor{currentfill}{rgb}{0.955300,0.901065,0.118128}%
\pgfsetfillcolor{currentfill}%
\pgfsetfillopacity{0.700000}%
\pgfsetlinewidth{0.501875pt}%
\definecolor{currentstroke}{rgb}{1.000000,1.000000,1.000000}%
\pgfsetstrokecolor{currentstroke}%
\pgfsetstrokeopacity{0.500000}%
\pgfsetdash{}{0pt}%
\pgfpathmoveto{\pgfqpoint{3.547968in}{3.757345in}}%
\pgfpathlineto{\pgfqpoint{3.558918in}{3.762431in}}%
\pgfpathlineto{\pgfqpoint{3.569864in}{3.767476in}}%
\pgfpathlineto{\pgfqpoint{3.580805in}{3.772484in}}%
\pgfpathlineto{\pgfqpoint{3.591742in}{3.777458in}}%
\pgfpathlineto{\pgfqpoint{3.602674in}{3.782402in}}%
\pgfpathlineto{\pgfqpoint{3.596414in}{3.787603in}}%
\pgfpathlineto{\pgfqpoint{3.590156in}{3.792536in}}%
\pgfpathlineto{\pgfqpoint{3.583899in}{3.797193in}}%
\pgfpathlineto{\pgfqpoint{3.577643in}{3.801563in}}%
\pgfpathlineto{\pgfqpoint{3.571388in}{3.805637in}}%
\pgfpathlineto{\pgfqpoint{3.560472in}{3.800667in}}%
\pgfpathlineto{\pgfqpoint{3.549550in}{3.795664in}}%
\pgfpathlineto{\pgfqpoint{3.538624in}{3.790615in}}%
\pgfpathlineto{\pgfqpoint{3.527693in}{3.785505in}}%
\pgfpathlineto{\pgfqpoint{3.516757in}{3.780320in}}%
\pgfpathlineto{\pgfqpoint{3.522996in}{3.776303in}}%
\pgfpathlineto{\pgfqpoint{3.529236in}{3.771990in}}%
\pgfpathlineto{\pgfqpoint{3.535478in}{3.767389in}}%
\pgfpathlineto{\pgfqpoint{3.541722in}{3.762505in}}%
\pgfpathclose%
\pgfusepath{stroke,fill}%
\end{pgfscope}%
\begin{pgfscope}%
\pgfpathrectangle{\pgfqpoint{0.887500in}{0.275000in}}{\pgfqpoint{4.225000in}{4.225000in}}%
\pgfusepath{clip}%
\pgfsetbuttcap%
\pgfsetroundjoin%
\definecolor{currentfill}{rgb}{0.688944,0.865448,0.182725}%
\pgfsetfillcolor{currentfill}%
\pgfsetfillopacity{0.700000}%
\pgfsetlinewidth{0.501875pt}%
\definecolor{currentstroke}{rgb}{1.000000,1.000000,1.000000}%
\pgfsetstrokecolor{currentstroke}%
\pgfsetstrokeopacity{0.500000}%
\pgfsetdash{}{0pt}%
\pgfpathmoveto{\pgfqpoint{3.015307in}{3.570535in}}%
\pgfpathlineto{\pgfqpoint{3.026359in}{3.576074in}}%
\pgfpathlineto{\pgfqpoint{3.037407in}{3.581788in}}%
\pgfpathlineto{\pgfqpoint{3.048452in}{3.587597in}}%
\pgfpathlineto{\pgfqpoint{3.059493in}{3.593419in}}%
\pgfpathlineto{\pgfqpoint{3.070530in}{3.599185in}}%
\pgfpathlineto{\pgfqpoint{3.064472in}{3.600056in}}%
\pgfpathlineto{\pgfqpoint{3.058421in}{3.600950in}}%
\pgfpathlineto{\pgfqpoint{3.052375in}{3.601879in}}%
\pgfpathlineto{\pgfqpoint{3.046336in}{3.602854in}}%
\pgfpathlineto{\pgfqpoint{3.040302in}{3.603877in}}%
\pgfpathlineto{\pgfqpoint{3.029287in}{3.598261in}}%
\pgfpathlineto{\pgfqpoint{3.018268in}{3.592511in}}%
\pgfpathlineto{\pgfqpoint{3.007245in}{3.586702in}}%
\pgfpathlineto{\pgfqpoint{2.996219in}{3.580927in}}%
\pgfpathlineto{\pgfqpoint{2.985188in}{3.575278in}}%
\pgfpathlineto{\pgfqpoint{2.991200in}{3.574248in}}%
\pgfpathlineto{\pgfqpoint{2.997218in}{3.573258in}}%
\pgfpathlineto{\pgfqpoint{3.003242in}{3.572308in}}%
\pgfpathlineto{\pgfqpoint{3.009271in}{3.571400in}}%
\pgfpathclose%
\pgfusepath{stroke,fill}%
\end{pgfscope}%
\begin{pgfscope}%
\pgfpathrectangle{\pgfqpoint{0.887500in}{0.275000in}}{\pgfqpoint{4.225000in}{4.225000in}}%
\pgfusepath{clip}%
\pgfsetbuttcap%
\pgfsetroundjoin%
\definecolor{currentfill}{rgb}{0.150148,0.676631,0.506589}%
\pgfsetfillcolor{currentfill}%
\pgfsetfillopacity{0.700000}%
\pgfsetlinewidth{0.501875pt}%
\definecolor{currentstroke}{rgb}{1.000000,1.000000,1.000000}%
\pgfsetstrokecolor{currentstroke}%
\pgfsetstrokeopacity{0.500000}%
\pgfsetdash{}{0pt}%
\pgfpathmoveto{\pgfqpoint{2.206759in}{3.037782in}}%
\pgfpathlineto{\pgfqpoint{2.217920in}{3.044868in}}%
\pgfpathlineto{\pgfqpoint{2.229077in}{3.051944in}}%
\pgfpathlineto{\pgfqpoint{2.240230in}{3.059046in}}%
\pgfpathlineto{\pgfqpoint{2.251380in}{3.066213in}}%
\pgfpathlineto{\pgfqpoint{2.262523in}{3.073481in}}%
\pgfpathlineto{\pgfqpoint{2.256736in}{3.081159in}}%
\pgfpathlineto{\pgfqpoint{2.250950in}{3.088961in}}%
\pgfpathlineto{\pgfqpoint{2.245164in}{3.096910in}}%
\pgfpathlineto{\pgfqpoint{2.239379in}{3.105030in}}%
\pgfpathlineto{\pgfqpoint{2.233593in}{3.113343in}}%
\pgfpathlineto{\pgfqpoint{2.222449in}{3.106796in}}%
\pgfpathlineto{\pgfqpoint{2.211295in}{3.100498in}}%
\pgfpathlineto{\pgfqpoint{2.200134in}{3.094415in}}%
\pgfpathlineto{\pgfqpoint{2.188965in}{3.088511in}}%
\pgfpathlineto{\pgfqpoint{2.177789in}{3.082751in}}%
\pgfpathlineto{\pgfqpoint{2.183574in}{3.073891in}}%
\pgfpathlineto{\pgfqpoint{2.189363in}{3.064951in}}%
\pgfpathlineto{\pgfqpoint{2.195158in}{3.055944in}}%
\pgfpathlineto{\pgfqpoint{2.200956in}{3.046883in}}%
\pgfpathclose%
\pgfusepath{stroke,fill}%
\end{pgfscope}%
\begin{pgfscope}%
\pgfpathrectangle{\pgfqpoint{0.887500in}{0.275000in}}{\pgfqpoint{4.225000in}{4.225000in}}%
\pgfusepath{clip}%
\pgfsetbuttcap%
\pgfsetroundjoin%
\definecolor{currentfill}{rgb}{0.175707,0.697900,0.491033}%
\pgfsetfillcolor{currentfill}%
\pgfsetfillopacity{0.700000}%
\pgfsetlinewidth{0.501875pt}%
\definecolor{currentstroke}{rgb}{1.000000,1.000000,1.000000}%
\pgfsetstrokecolor{currentstroke}%
\pgfsetstrokeopacity{0.500000}%
\pgfsetdash{}{0pt}%
\pgfpathmoveto{\pgfqpoint{2.488000in}{3.094297in}}%
\pgfpathlineto{\pgfqpoint{2.499181in}{3.097232in}}%
\pgfpathlineto{\pgfqpoint{2.510342in}{3.101281in}}%
\pgfpathlineto{\pgfqpoint{2.521482in}{3.106464in}}%
\pgfpathlineto{\pgfqpoint{2.532605in}{3.112702in}}%
\pgfpathlineto{\pgfqpoint{2.543712in}{3.119914in}}%
\pgfpathlineto{\pgfqpoint{2.537858in}{3.125473in}}%
\pgfpathlineto{\pgfqpoint{2.532009in}{3.131033in}}%
\pgfpathlineto{\pgfqpoint{2.526164in}{3.136633in}}%
\pgfpathlineto{\pgfqpoint{2.520322in}{3.142311in}}%
\pgfpathlineto{\pgfqpoint{2.514483in}{3.148104in}}%
\pgfpathlineto{\pgfqpoint{2.503364in}{3.143098in}}%
\pgfpathlineto{\pgfqpoint{2.492234in}{3.138502in}}%
\pgfpathlineto{\pgfqpoint{2.481092in}{3.134444in}}%
\pgfpathlineto{\pgfqpoint{2.469936in}{3.131053in}}%
\pgfpathlineto{\pgfqpoint{2.458763in}{3.128367in}}%
\pgfpathlineto{\pgfqpoint{2.464597in}{3.121955in}}%
\pgfpathlineto{\pgfqpoint{2.470436in}{3.115416in}}%
\pgfpathlineto{\pgfqpoint{2.476282in}{3.108675in}}%
\pgfpathlineto{\pgfqpoint{2.482137in}{3.101660in}}%
\pgfpathclose%
\pgfusepath{stroke,fill}%
\end{pgfscope}%
\begin{pgfscope}%
\pgfpathrectangle{\pgfqpoint{0.887500in}{0.275000in}}{\pgfqpoint{4.225000in}{4.225000in}}%
\pgfusepath{clip}%
\pgfsetbuttcap%
\pgfsetroundjoin%
\definecolor{currentfill}{rgb}{0.140210,0.665859,0.513427}%
\pgfsetfillcolor{currentfill}%
\pgfsetfillopacity{0.700000}%
\pgfsetlinewidth{0.501875pt}%
\definecolor{currentstroke}{rgb}{1.000000,1.000000,1.000000}%
\pgfsetstrokecolor{currentstroke}%
\pgfsetstrokeopacity{0.500000}%
\pgfsetdash{}{0pt}%
\pgfpathmoveto{\pgfqpoint{2.066174in}{3.010553in}}%
\pgfpathlineto{\pgfqpoint{2.077305in}{3.019590in}}%
\pgfpathlineto{\pgfqpoint{2.088437in}{3.028516in}}%
\pgfpathlineto{\pgfqpoint{2.099575in}{3.037088in}}%
\pgfpathlineto{\pgfqpoint{2.110724in}{3.045065in}}%
\pgfpathlineto{\pgfqpoint{2.121887in}{3.052355in}}%
\pgfpathlineto{\pgfqpoint{2.116109in}{3.061612in}}%
\pgfpathlineto{\pgfqpoint{2.110344in}{3.070470in}}%
\pgfpathlineto{\pgfqpoint{2.104593in}{3.078884in}}%
\pgfpathlineto{\pgfqpoint{2.098856in}{3.086810in}}%
\pgfpathlineto{\pgfqpoint{2.087707in}{3.079376in}}%
\pgfpathlineto{\pgfqpoint{2.076575in}{3.071139in}}%
\pgfpathlineto{\pgfqpoint{2.065458in}{3.062197in}}%
\pgfpathlineto{\pgfqpoint{2.054350in}{3.052824in}}%
\pgfpathlineto{\pgfqpoint{2.043243in}{3.043298in}}%
\pgfpathlineto{\pgfqpoint{2.048958in}{3.035602in}}%
\pgfpathlineto{\pgfqpoint{2.054685in}{3.027559in}}%
\pgfpathlineto{\pgfqpoint{2.060424in}{3.019199in}}%
\pgfpathclose%
\pgfusepath{stroke,fill}%
\end{pgfscope}%
\begin{pgfscope}%
\pgfpathrectangle{\pgfqpoint{0.887500in}{0.275000in}}{\pgfqpoint{4.225000in}{4.225000in}}%
\pgfusepath{clip}%
\pgfsetbuttcap%
\pgfsetroundjoin%
\definecolor{currentfill}{rgb}{0.175707,0.697900,0.491033}%
\pgfsetfillcolor{currentfill}%
\pgfsetfillopacity{0.700000}%
\pgfsetlinewidth{0.501875pt}%
\definecolor{currentstroke}{rgb}{1.000000,1.000000,1.000000}%
\pgfsetstrokecolor{currentstroke}%
\pgfsetstrokeopacity{0.500000}%
\pgfsetdash{}{0pt}%
\pgfpathmoveto{\pgfqpoint{2.347095in}{3.081018in}}%
\pgfpathlineto{\pgfqpoint{2.358215in}{3.089606in}}%
\pgfpathlineto{\pgfqpoint{2.369344in}{3.097565in}}%
\pgfpathlineto{\pgfqpoint{2.380484in}{3.104661in}}%
\pgfpathlineto{\pgfqpoint{2.391639in}{3.110661in}}%
\pgfpathlineto{\pgfqpoint{2.402811in}{3.115378in}}%
\pgfpathlineto{\pgfqpoint{2.397009in}{3.120985in}}%
\pgfpathlineto{\pgfqpoint{2.391211in}{3.126560in}}%
\pgfpathlineto{\pgfqpoint{2.385416in}{3.132214in}}%
\pgfpathlineto{\pgfqpoint{2.379622in}{3.138060in}}%
\pgfpathlineto{\pgfqpoint{2.373827in}{3.144208in}}%
\pgfpathlineto{\pgfqpoint{2.362675in}{3.139257in}}%
\pgfpathlineto{\pgfqpoint{2.351533in}{3.133405in}}%
\pgfpathlineto{\pgfqpoint{2.340402in}{3.126795in}}%
\pgfpathlineto{\pgfqpoint{2.329277in}{3.119602in}}%
\pgfpathlineto{\pgfqpoint{2.318157in}{3.111999in}}%
\pgfpathlineto{\pgfqpoint{2.323941in}{3.105525in}}%
\pgfpathlineto{\pgfqpoint{2.329726in}{3.099267in}}%
\pgfpathlineto{\pgfqpoint{2.335512in}{3.093149in}}%
\pgfpathlineto{\pgfqpoint{2.341301in}{3.087092in}}%
\pgfpathclose%
\pgfusepath{stroke,fill}%
\end{pgfscope}%
\begin{pgfscope}%
\pgfpathrectangle{\pgfqpoint{0.887500in}{0.275000in}}{\pgfqpoint{4.225000in}{4.225000in}}%
\pgfusepath{clip}%
\pgfsetbuttcap%
\pgfsetroundjoin%
\definecolor{currentfill}{rgb}{0.876168,0.891125,0.095250}%
\pgfsetfillcolor{currentfill}%
\pgfsetfillopacity{0.700000}%
\pgfsetlinewidth{0.501875pt}%
\definecolor{currentstroke}{rgb}{1.000000,1.000000,1.000000}%
\pgfsetstrokecolor{currentstroke}%
\pgfsetstrokeopacity{0.500000}%
\pgfsetdash{}{0pt}%
\pgfpathmoveto{\pgfqpoint{3.352176in}{3.694082in}}%
\pgfpathlineto{\pgfqpoint{3.363177in}{3.700141in}}%
\pgfpathlineto{\pgfqpoint{3.374173in}{3.706151in}}%
\pgfpathlineto{\pgfqpoint{3.385166in}{3.712118in}}%
\pgfpathlineto{\pgfqpoint{3.396155in}{3.718045in}}%
\pgfpathlineto{\pgfqpoint{3.407140in}{3.723939in}}%
\pgfpathlineto{\pgfqpoint{3.400933in}{3.727070in}}%
\pgfpathlineto{\pgfqpoint{3.394730in}{3.729925in}}%
\pgfpathlineto{\pgfqpoint{3.388530in}{3.732515in}}%
\pgfpathlineto{\pgfqpoint{3.382334in}{3.734855in}}%
\pgfpathlineto{\pgfqpoint{3.376141in}{3.736970in}}%
\pgfpathlineto{\pgfqpoint{3.365174in}{3.730889in}}%
\pgfpathlineto{\pgfqpoint{3.354204in}{3.724794in}}%
\pgfpathlineto{\pgfqpoint{3.343230in}{3.718676in}}%
\pgfpathlineto{\pgfqpoint{3.332252in}{3.712525in}}%
\pgfpathlineto{\pgfqpoint{3.321270in}{3.706332in}}%
\pgfpathlineto{\pgfqpoint{3.327443in}{3.704327in}}%
\pgfpathlineto{\pgfqpoint{3.333620in}{3.702119in}}%
\pgfpathlineto{\pgfqpoint{3.339802in}{3.699687in}}%
\pgfpathlineto{\pgfqpoint{3.345987in}{3.697012in}}%
\pgfpathclose%
\pgfusepath{stroke,fill}%
\end{pgfscope}%
\begin{pgfscope}%
\pgfpathrectangle{\pgfqpoint{0.887500in}{0.275000in}}{\pgfqpoint{4.225000in}{4.225000in}}%
\pgfusepath{clip}%
\pgfsetbuttcap%
\pgfsetroundjoin%
\definecolor{currentfill}{rgb}{0.772852,0.877868,0.131109}%
\pgfsetfillcolor{currentfill}%
\pgfsetfillopacity{0.700000}%
\pgfsetlinewidth{0.501875pt}%
\definecolor{currentstroke}{rgb}{1.000000,1.000000,1.000000}%
\pgfsetstrokecolor{currentstroke}%
\pgfsetstrokeopacity{0.500000}%
\pgfsetdash{}{0pt}%
\pgfpathmoveto{\pgfqpoint{3.156144in}{3.622803in}}%
\pgfpathlineto{\pgfqpoint{3.167178in}{3.628089in}}%
\pgfpathlineto{\pgfqpoint{3.178209in}{3.633208in}}%
\pgfpathlineto{\pgfqpoint{3.189234in}{3.638213in}}%
\pgfpathlineto{\pgfqpoint{3.200256in}{3.643168in}}%
\pgfpathlineto{\pgfqpoint{3.211273in}{3.648141in}}%
\pgfpathlineto{\pgfqpoint{3.205144in}{3.649585in}}%
\pgfpathlineto{\pgfqpoint{3.199021in}{3.650869in}}%
\pgfpathlineto{\pgfqpoint{3.192903in}{3.652023in}}%
\pgfpathlineto{\pgfqpoint{3.186791in}{3.653079in}}%
\pgfpathlineto{\pgfqpoint{3.180684in}{3.654071in}}%
\pgfpathlineto{\pgfqpoint{3.169686in}{3.648625in}}%
\pgfpathlineto{\pgfqpoint{3.158685in}{3.643230in}}%
\pgfpathlineto{\pgfqpoint{3.147679in}{3.637855in}}%
\pgfpathlineto{\pgfqpoint{3.136670in}{3.632472in}}%
\pgfpathlineto{\pgfqpoint{3.125657in}{3.627053in}}%
\pgfpathlineto{\pgfqpoint{3.131742in}{3.626377in}}%
\pgfpathlineto{\pgfqpoint{3.137834in}{3.625650in}}%
\pgfpathlineto{\pgfqpoint{3.143931in}{3.624835in}}%
\pgfpathlineto{\pgfqpoint{3.150035in}{3.623898in}}%
\pgfpathclose%
\pgfusepath{stroke,fill}%
\end{pgfscope}%
\begin{pgfscope}%
\pgfpathrectangle{\pgfqpoint{0.887500in}{0.275000in}}{\pgfqpoint{4.225000in}{4.225000in}}%
\pgfusepath{clip}%
\pgfsetbuttcap%
\pgfsetroundjoin%
\definecolor{currentfill}{rgb}{0.128087,0.647749,0.523491}%
\pgfsetfillcolor{currentfill}%
\pgfsetfillopacity{0.700000}%
\pgfsetlinewidth{0.501875pt}%
\definecolor{currentstroke}{rgb}{1.000000,1.000000,1.000000}%
\pgfsetstrokecolor{currentstroke}%
\pgfsetstrokeopacity{0.500000}%
\pgfsetdash{}{0pt}%
\pgfpathmoveto{\pgfqpoint{2.010315in}{2.971915in}}%
\pgfpathlineto{\pgfqpoint{2.021531in}{2.978150in}}%
\pgfpathlineto{\pgfqpoint{2.032722in}{2.985217in}}%
\pgfpathlineto{\pgfqpoint{2.043889in}{2.993123in}}%
\pgfpathlineto{\pgfqpoint{2.055037in}{3.001650in}}%
\pgfpathlineto{\pgfqpoint{2.066174in}{3.010553in}}%
\pgfpathlineto{\pgfqpoint{2.060424in}{3.019199in}}%
\pgfpathlineto{\pgfqpoint{2.054685in}{3.027559in}}%
\pgfpathlineto{\pgfqpoint{2.048958in}{3.035602in}}%
\pgfpathlineto{\pgfqpoint{2.043243in}{3.043298in}}%
\pgfpathlineto{\pgfqpoint{2.032131in}{3.033896in}}%
\pgfpathlineto{\pgfqpoint{2.021008in}{3.024894in}}%
\pgfpathlineto{\pgfqpoint{2.009864in}{3.016569in}}%
\pgfpathlineto{\pgfqpoint{1.998694in}{3.009169in}}%
\pgfpathlineto{\pgfqpoint{1.987496in}{3.002688in}}%
\pgfpathlineto{\pgfqpoint{1.993193in}{2.995073in}}%
\pgfpathlineto{\pgfqpoint{1.998895in}{2.987404in}}%
\pgfpathlineto{\pgfqpoint{2.004603in}{2.979683in}}%
\pgfpathclose%
\pgfusepath{stroke,fill}%
\end{pgfscope}%
\begin{pgfscope}%
\pgfpathrectangle{\pgfqpoint{0.887500in}{0.275000in}}{\pgfqpoint{4.225000in}{4.225000in}}%
\pgfusepath{clip}%
\pgfsetbuttcap%
\pgfsetroundjoin%
\definecolor{currentfill}{rgb}{0.974417,0.903590,0.130215}%
\pgfsetfillcolor{currentfill}%
\pgfsetfillopacity{0.700000}%
\pgfsetlinewidth{0.501875pt}%
\definecolor{currentstroke}{rgb}{1.000000,1.000000,1.000000}%
\pgfsetstrokecolor{currentstroke}%
\pgfsetstrokeopacity{0.500000}%
\pgfsetdash{}{0pt}%
\pgfpathmoveto{\pgfqpoint{3.774682in}{3.765418in}}%
\pgfpathlineto{\pgfqpoint{3.785588in}{3.769990in}}%
\pgfpathlineto{\pgfqpoint{3.796489in}{3.774529in}}%
\pgfpathlineto{\pgfqpoint{3.807385in}{3.779033in}}%
\pgfpathlineto{\pgfqpoint{3.818276in}{3.783505in}}%
\pgfpathlineto{\pgfqpoint{3.811964in}{3.790649in}}%
\pgfpathlineto{\pgfqpoint{3.805653in}{3.797624in}}%
\pgfpathlineto{\pgfqpoint{3.799344in}{3.804448in}}%
\pgfpathlineto{\pgfqpoint{3.793036in}{3.811136in}}%
\pgfpathlineto{\pgfqpoint{3.786731in}{3.817703in}}%
\pgfpathlineto{\pgfqpoint{3.775854in}{3.813301in}}%
\pgfpathlineto{\pgfqpoint{3.764971in}{3.808864in}}%
\pgfpathlineto{\pgfqpoint{3.754083in}{3.804394in}}%
\pgfpathlineto{\pgfqpoint{3.743191in}{3.799888in}}%
\pgfpathlineto{\pgfqpoint{3.749486in}{3.793325in}}%
\pgfpathlineto{\pgfqpoint{3.755783in}{3.786605in}}%
\pgfpathlineto{\pgfqpoint{3.762081in}{3.779720in}}%
\pgfpathlineto{\pgfqpoint{3.768381in}{3.772660in}}%
\pgfpathclose%
\pgfusepath{stroke,fill}%
\end{pgfscope}%
\begin{pgfscope}%
\pgfpathrectangle{\pgfqpoint{0.887500in}{0.275000in}}{\pgfqpoint{4.225000in}{4.225000in}}%
\pgfusepath{clip}%
\pgfsetbuttcap%
\pgfsetroundjoin%
\definecolor{currentfill}{rgb}{0.140210,0.665859,0.513427}%
\pgfsetfillcolor{currentfill}%
\pgfsetfillopacity{0.700000}%
\pgfsetlinewidth{0.501875pt}%
\definecolor{currentstroke}{rgb}{1.000000,1.000000,1.000000}%
\pgfsetstrokecolor{currentstroke}%
\pgfsetstrokeopacity{0.500000}%
\pgfsetdash{}{0pt}%
\pgfpathmoveto{\pgfqpoint{2.150927in}{3.001621in}}%
\pgfpathlineto{\pgfqpoint{2.162097in}{3.008986in}}%
\pgfpathlineto{\pgfqpoint{2.173265in}{3.016274in}}%
\pgfpathlineto{\pgfqpoint{2.184432in}{3.023496in}}%
\pgfpathlineto{\pgfqpoint{2.195597in}{3.030662in}}%
\pgfpathlineto{\pgfqpoint{2.206759in}{3.037782in}}%
\pgfpathlineto{\pgfqpoint{2.200956in}{3.046883in}}%
\pgfpathlineto{\pgfqpoint{2.195158in}{3.055944in}}%
\pgfpathlineto{\pgfqpoint{2.189363in}{3.064951in}}%
\pgfpathlineto{\pgfqpoint{2.183574in}{3.073891in}}%
\pgfpathlineto{\pgfqpoint{2.177789in}{3.082751in}}%
\pgfpathlineto{\pgfqpoint{2.166608in}{3.077053in}}%
\pgfpathlineto{\pgfqpoint{2.155424in}{3.071292in}}%
\pgfpathlineto{\pgfqpoint{2.144241in}{3.065341in}}%
\pgfpathlineto{\pgfqpoint{2.133061in}{3.059071in}}%
\pgfpathlineto{\pgfqpoint{2.121887in}{3.052355in}}%
\pgfpathlineto{\pgfqpoint{2.127677in}{3.042742in}}%
\pgfpathlineto{\pgfqpoint{2.133477in}{3.032817in}}%
\pgfpathlineto{\pgfqpoint{2.139286in}{3.022626in}}%
\pgfpathlineto{\pgfqpoint{2.145103in}{3.012213in}}%
\pgfpathclose%
\pgfusepath{stroke,fill}%
\end{pgfscope}%
\begin{pgfscope}%
\pgfpathrectangle{\pgfqpoint{0.887500in}{0.275000in}}{\pgfqpoint{4.225000in}{4.225000in}}%
\pgfusepath{clip}%
\pgfsetbuttcap%
\pgfsetroundjoin%
\definecolor{currentfill}{rgb}{0.935904,0.898570,0.108131}%
\pgfsetfillcolor{currentfill}%
\pgfsetfillopacity{0.700000}%
\pgfsetlinewidth{0.501875pt}%
\definecolor{currentstroke}{rgb}{1.000000,1.000000,1.000000}%
\pgfsetstrokecolor{currentstroke}%
\pgfsetstrokeopacity{0.500000}%
\pgfsetdash{}{0pt}%
\pgfpathmoveto{\pgfqpoint{3.493146in}{3.731210in}}%
\pgfpathlineto{\pgfqpoint{3.504120in}{3.736537in}}%
\pgfpathlineto{\pgfqpoint{3.515088in}{3.741813in}}%
\pgfpathlineto{\pgfqpoint{3.526053in}{3.747039in}}%
\pgfpathlineto{\pgfqpoint{3.537012in}{3.752216in}}%
\pgfpathlineto{\pgfqpoint{3.547968in}{3.757345in}}%
\pgfpathlineto{\pgfqpoint{3.541722in}{3.762505in}}%
\pgfpathlineto{\pgfqpoint{3.535478in}{3.767389in}}%
\pgfpathlineto{\pgfqpoint{3.529236in}{3.771990in}}%
\pgfpathlineto{\pgfqpoint{3.522996in}{3.776303in}}%
\pgfpathlineto{\pgfqpoint{3.516757in}{3.780320in}}%
\pgfpathlineto{\pgfqpoint{3.505817in}{3.775046in}}%
\pgfpathlineto{\pgfqpoint{3.494871in}{3.769669in}}%
\pgfpathlineto{\pgfqpoint{3.483920in}{3.764186in}}%
\pgfpathlineto{\pgfqpoint{3.472964in}{3.758611in}}%
\pgfpathlineto{\pgfqpoint{3.462004in}{3.752956in}}%
\pgfpathlineto{\pgfqpoint{3.468228in}{3.749211in}}%
\pgfpathlineto{\pgfqpoint{3.474455in}{3.745165in}}%
\pgfpathlineto{\pgfqpoint{3.480683in}{3.740817in}}%
\pgfpathlineto{\pgfqpoint{3.486914in}{3.736166in}}%
\pgfpathclose%
\pgfusepath{stroke,fill}%
\end{pgfscope}%
\begin{pgfscope}%
\pgfpathrectangle{\pgfqpoint{0.887500in}{0.275000in}}{\pgfqpoint{4.225000in}{4.225000in}}%
\pgfusepath{clip}%
\pgfsetbuttcap%
\pgfsetroundjoin%
\definecolor{currentfill}{rgb}{0.668054,0.861999,0.196293}%
\pgfsetfillcolor{currentfill}%
\pgfsetfillopacity{0.700000}%
\pgfsetlinewidth{0.501875pt}%
\definecolor{currentstroke}{rgb}{1.000000,1.000000,1.000000}%
\pgfsetstrokecolor{currentstroke}%
\pgfsetstrokeopacity{0.500000}%
\pgfsetdash{}{0pt}%
\pgfpathmoveto{\pgfqpoint{2.959974in}{3.547667in}}%
\pgfpathlineto{\pgfqpoint{2.971051in}{3.551642in}}%
\pgfpathlineto{\pgfqpoint{2.982122in}{3.555778in}}%
\pgfpathlineto{\pgfqpoint{2.993189in}{3.560307in}}%
\pgfpathlineto{\pgfqpoint{3.004250in}{3.565253in}}%
\pgfpathlineto{\pgfqpoint{3.015307in}{3.570535in}}%
\pgfpathlineto{\pgfqpoint{3.009271in}{3.571400in}}%
\pgfpathlineto{\pgfqpoint{3.003242in}{3.572308in}}%
\pgfpathlineto{\pgfqpoint{2.997218in}{3.573258in}}%
\pgfpathlineto{\pgfqpoint{2.991200in}{3.574248in}}%
\pgfpathlineto{\pgfqpoint{2.985188in}{3.575278in}}%
\pgfpathlineto{\pgfqpoint{2.974154in}{3.569846in}}%
\pgfpathlineto{\pgfqpoint{2.963114in}{3.564723in}}%
\pgfpathlineto{\pgfqpoint{2.952070in}{3.560001in}}%
\pgfpathlineto{\pgfqpoint{2.941020in}{3.555632in}}%
\pgfpathlineto{\pgfqpoint{2.929965in}{3.551297in}}%
\pgfpathlineto{\pgfqpoint{2.935956in}{3.550417in}}%
\pgfpathlineto{\pgfqpoint{2.941952in}{3.549617in}}%
\pgfpathlineto{\pgfqpoint{2.947953in}{3.548901in}}%
\pgfpathlineto{\pgfqpoint{2.953961in}{3.548256in}}%
\pgfpathclose%
\pgfusepath{stroke,fill}%
\end{pgfscope}%
\begin{pgfscope}%
\pgfpathrectangle{\pgfqpoint{0.887500in}{0.275000in}}{\pgfqpoint{4.225000in}{4.225000in}}%
\pgfusepath{clip}%
\pgfsetbuttcap%
\pgfsetroundjoin%
\definecolor{currentfill}{rgb}{0.964894,0.902323,0.123941}%
\pgfsetfillcolor{currentfill}%
\pgfsetfillopacity{0.700000}%
\pgfsetlinewidth{0.501875pt}%
\definecolor{currentstroke}{rgb}{1.000000,1.000000,1.000000}%
\pgfsetstrokecolor{currentstroke}%
\pgfsetstrokeopacity{0.500000}%
\pgfsetdash{}{0pt}%
\pgfpathmoveto{\pgfqpoint{3.633995in}{3.752716in}}%
\pgfpathlineto{\pgfqpoint{3.644937in}{3.757619in}}%
\pgfpathlineto{\pgfqpoint{3.655873in}{3.762478in}}%
\pgfpathlineto{\pgfqpoint{3.666805in}{3.767294in}}%
\pgfpathlineto{\pgfqpoint{3.677732in}{3.772069in}}%
\pgfpathlineto{\pgfqpoint{3.688654in}{3.776803in}}%
\pgfpathlineto{\pgfqpoint{3.682374in}{3.783190in}}%
\pgfpathlineto{\pgfqpoint{3.676095in}{3.789385in}}%
\pgfpathlineto{\pgfqpoint{3.669818in}{3.795385in}}%
\pgfpathlineto{\pgfqpoint{3.663542in}{3.801173in}}%
\pgfpathlineto{\pgfqpoint{3.657268in}{3.806727in}}%
\pgfpathlineto{\pgfqpoint{3.646358in}{3.801913in}}%
\pgfpathlineto{\pgfqpoint{3.635444in}{3.797074in}}%
\pgfpathlineto{\pgfqpoint{3.624525in}{3.792209in}}%
\pgfpathlineto{\pgfqpoint{3.613602in}{3.787319in}}%
\pgfpathlineto{\pgfqpoint{3.602674in}{3.782402in}}%
\pgfpathlineto{\pgfqpoint{3.608936in}{3.776943in}}%
\pgfpathlineto{\pgfqpoint{3.615198in}{3.771236in}}%
\pgfpathlineto{\pgfqpoint{3.621463in}{3.765291in}}%
\pgfpathlineto{\pgfqpoint{3.627728in}{3.759115in}}%
\pgfpathclose%
\pgfusepath{stroke,fill}%
\end{pgfscope}%
\begin{pgfscope}%
\pgfpathrectangle{\pgfqpoint{0.887500in}{0.275000in}}{\pgfqpoint{4.225000in}{4.225000in}}%
\pgfusepath{clip}%
\pgfsetbuttcap%
\pgfsetroundjoin%
\definecolor{currentfill}{rgb}{0.157851,0.683765,0.501686}%
\pgfsetfillcolor{currentfill}%
\pgfsetfillopacity{0.700000}%
\pgfsetlinewidth{0.501875pt}%
\definecolor{currentstroke}{rgb}{1.000000,1.000000,1.000000}%
\pgfsetstrokecolor{currentstroke}%
\pgfsetstrokeopacity{0.500000}%
\pgfsetdash{}{0pt}%
\pgfpathmoveto{\pgfqpoint{2.291496in}{3.036170in}}%
\pgfpathlineto{\pgfqpoint{2.302624in}{3.044923in}}%
\pgfpathlineto{\pgfqpoint{2.313748in}{3.053812in}}%
\pgfpathlineto{\pgfqpoint{2.324865in}{3.062889in}}%
\pgfpathlineto{\pgfqpoint{2.335980in}{3.072035in}}%
\pgfpathlineto{\pgfqpoint{2.347095in}{3.081018in}}%
\pgfpathlineto{\pgfqpoint{2.341301in}{3.087092in}}%
\pgfpathlineto{\pgfqpoint{2.335512in}{3.093149in}}%
\pgfpathlineto{\pgfqpoint{2.329726in}{3.099267in}}%
\pgfpathlineto{\pgfqpoint{2.323941in}{3.105525in}}%
\pgfpathlineto{\pgfqpoint{2.318157in}{3.111999in}}%
\pgfpathlineto{\pgfqpoint{2.307039in}{3.104161in}}%
\pgfpathlineto{\pgfqpoint{2.295918in}{3.096261in}}%
\pgfpathlineto{\pgfqpoint{2.284793in}{3.088473in}}%
\pgfpathlineto{\pgfqpoint{2.273662in}{3.080889in}}%
\pgfpathlineto{\pgfqpoint{2.262523in}{3.073481in}}%
\pgfpathlineto{\pgfqpoint{2.268313in}{3.065906in}}%
\pgfpathlineto{\pgfqpoint{2.274104in}{3.058410in}}%
\pgfpathlineto{\pgfqpoint{2.279898in}{3.050971in}}%
\pgfpathlineto{\pgfqpoint{2.285695in}{3.043565in}}%
\pgfpathclose%
\pgfusepath{stroke,fill}%
\end{pgfscope}%
\begin{pgfscope}%
\pgfpathrectangle{\pgfqpoint{0.887500in}{0.275000in}}{\pgfqpoint{4.225000in}{4.225000in}}%
\pgfusepath{clip}%
\pgfsetbuttcap%
\pgfsetroundjoin%
\definecolor{currentfill}{rgb}{0.175707,0.697900,0.491033}%
\pgfsetfillcolor{currentfill}%
\pgfsetfillopacity{0.700000}%
\pgfsetlinewidth{0.501875pt}%
\definecolor{currentstroke}{rgb}{1.000000,1.000000,1.000000}%
\pgfsetstrokecolor{currentstroke}%
\pgfsetstrokeopacity{0.500000}%
\pgfsetdash{}{0pt}%
\pgfpathmoveto{\pgfqpoint{2.431960in}{3.082942in}}%
\pgfpathlineto{\pgfqpoint{2.443165in}{3.086143in}}%
\pgfpathlineto{\pgfqpoint{2.454379in}{3.088467in}}%
\pgfpathlineto{\pgfqpoint{2.465593in}{3.090326in}}%
\pgfpathlineto{\pgfqpoint{2.476802in}{3.092132in}}%
\pgfpathlineto{\pgfqpoint{2.488000in}{3.094297in}}%
\pgfpathlineto{\pgfqpoint{2.482137in}{3.101660in}}%
\pgfpathlineto{\pgfqpoint{2.476282in}{3.108675in}}%
\pgfpathlineto{\pgfqpoint{2.470436in}{3.115416in}}%
\pgfpathlineto{\pgfqpoint{2.464597in}{3.121955in}}%
\pgfpathlineto{\pgfqpoint{2.458763in}{3.128367in}}%
\pgfpathlineto{\pgfqpoint{2.447579in}{3.126130in}}%
\pgfpathlineto{\pgfqpoint{2.436386in}{3.124028in}}%
\pgfpathlineto{\pgfqpoint{2.425191in}{3.121746in}}%
\pgfpathlineto{\pgfqpoint{2.413997in}{3.118968in}}%
\pgfpathlineto{\pgfqpoint{2.402811in}{3.115378in}}%
\pgfpathlineto{\pgfqpoint{2.408621in}{3.109627in}}%
\pgfpathlineto{\pgfqpoint{2.414439in}{3.103619in}}%
\pgfpathlineto{\pgfqpoint{2.420267in}{3.097244in}}%
\pgfpathlineto{\pgfqpoint{2.426107in}{3.090389in}}%
\pgfpathclose%
\pgfusepath{stroke,fill}%
\end{pgfscope}%
\begin{pgfscope}%
\pgfpathrectangle{\pgfqpoint{0.887500in}{0.275000in}}{\pgfqpoint{4.225000in}{4.225000in}}%
\pgfusepath{clip}%
\pgfsetbuttcap%
\pgfsetroundjoin%
\definecolor{currentfill}{rgb}{0.845561,0.887322,0.099702}%
\pgfsetfillcolor{currentfill}%
\pgfsetfillopacity{0.700000}%
\pgfsetlinewidth{0.501875pt}%
\definecolor{currentstroke}{rgb}{1.000000,1.000000,1.000000}%
\pgfsetstrokecolor{currentstroke}%
\pgfsetstrokeopacity{0.500000}%
\pgfsetdash{}{0pt}%
\pgfpathmoveto{\pgfqpoint{3.297119in}{3.663804in}}%
\pgfpathlineto{\pgfqpoint{3.308137in}{3.669707in}}%
\pgfpathlineto{\pgfqpoint{3.319152in}{3.675736in}}%
\pgfpathlineto{\pgfqpoint{3.330163in}{3.681840in}}%
\pgfpathlineto{\pgfqpoint{3.341172in}{3.687972in}}%
\pgfpathlineto{\pgfqpoint{3.352176in}{3.694082in}}%
\pgfpathlineto{\pgfqpoint{3.345987in}{3.697012in}}%
\pgfpathlineto{\pgfqpoint{3.339802in}{3.699687in}}%
\pgfpathlineto{\pgfqpoint{3.333620in}{3.702119in}}%
\pgfpathlineto{\pgfqpoint{3.327443in}{3.704327in}}%
\pgfpathlineto{\pgfqpoint{3.321270in}{3.706332in}}%
\pgfpathlineto{\pgfqpoint{3.310284in}{3.700089in}}%
\pgfpathlineto{\pgfqpoint{3.299295in}{3.693825in}}%
\pgfpathlineto{\pgfqpoint{3.288303in}{3.687585in}}%
\pgfpathlineto{\pgfqpoint{3.277307in}{3.681417in}}%
\pgfpathlineto{\pgfqpoint{3.266308in}{3.675368in}}%
\pgfpathlineto{\pgfqpoint{3.272461in}{3.673457in}}%
\pgfpathlineto{\pgfqpoint{3.278619in}{3.671364in}}%
\pgfpathlineto{\pgfqpoint{3.284781in}{3.669069in}}%
\pgfpathlineto{\pgfqpoint{3.290948in}{3.666553in}}%
\pgfpathclose%
\pgfusepath{stroke,fill}%
\end{pgfscope}%
\begin{pgfscope}%
\pgfpathrectangle{\pgfqpoint{0.887500in}{0.275000in}}{\pgfqpoint{4.225000in}{4.225000in}}%
\pgfusepath{clip}%
\pgfsetbuttcap%
\pgfsetroundjoin%
\definecolor{currentfill}{rgb}{0.196571,0.711827,0.479221}%
\pgfsetfillcolor{currentfill}%
\pgfsetfillopacity{0.700000}%
\pgfsetlinewidth{0.501875pt}%
\definecolor{currentstroke}{rgb}{1.000000,1.000000,1.000000}%
\pgfsetstrokecolor{currentstroke}%
\pgfsetstrokeopacity{0.500000}%
\pgfsetdash{}{0pt}%
\pgfpathmoveto{\pgfqpoint{2.573059in}{3.090995in}}%
\pgfpathlineto{\pgfqpoint{2.584116in}{3.103619in}}%
\pgfpathlineto{\pgfqpoint{2.595156in}{3.117750in}}%
\pgfpathlineto{\pgfqpoint{2.606179in}{3.133380in}}%
\pgfpathlineto{\pgfqpoint{2.617189in}{3.150430in}}%
\pgfpathlineto{\pgfqpoint{2.628189in}{3.168580in}}%
\pgfpathlineto{\pgfqpoint{2.622352in}{3.168387in}}%
\pgfpathlineto{\pgfqpoint{2.616524in}{3.168002in}}%
\pgfpathlineto{\pgfqpoint{2.610701in}{3.167664in}}%
\pgfpathlineto{\pgfqpoint{2.604881in}{3.167585in}}%
\pgfpathlineto{\pgfqpoint{2.599064in}{3.167841in}}%
\pgfpathlineto{\pgfqpoint{2.588012in}{3.156913in}}%
\pgfpathlineto{\pgfqpoint{2.576953in}{3.146595in}}%
\pgfpathlineto{\pgfqpoint{2.565884in}{3.136940in}}%
\pgfpathlineto{\pgfqpoint{2.554804in}{3.128020in}}%
\pgfpathlineto{\pgfqpoint{2.543712in}{3.119914in}}%
\pgfpathlineto{\pgfqpoint{2.549570in}{3.114318in}}%
\pgfpathlineto{\pgfqpoint{2.555434in}{3.108646in}}%
\pgfpathlineto{\pgfqpoint{2.561303in}{3.102869in}}%
\pgfpathlineto{\pgfqpoint{2.567178in}{3.096985in}}%
\pgfpathclose%
\pgfusepath{stroke,fill}%
\end{pgfscope}%
\begin{pgfscope}%
\pgfpathrectangle{\pgfqpoint{0.887500in}{0.275000in}}{\pgfqpoint{4.225000in}{4.225000in}}%
\pgfusepath{clip}%
\pgfsetbuttcap%
\pgfsetroundjoin%
\definecolor{currentfill}{rgb}{0.132268,0.655014,0.519661}%
\pgfsetfillcolor{currentfill}%
\pgfsetfillopacity{0.700000}%
\pgfsetlinewidth{0.501875pt}%
\definecolor{currentstroke}{rgb}{1.000000,1.000000,1.000000}%
\pgfsetstrokecolor{currentstroke}%
\pgfsetstrokeopacity{0.500000}%
\pgfsetdash{}{0pt}%
\pgfpathmoveto{\pgfqpoint{2.095049in}{2.964109in}}%
\pgfpathlineto{\pgfqpoint{2.106233in}{2.971499in}}%
\pgfpathlineto{\pgfqpoint{2.117411in}{2.979035in}}%
\pgfpathlineto{\pgfqpoint{2.128584in}{2.986623in}}%
\pgfpathlineto{\pgfqpoint{2.139756in}{2.994168in}}%
\pgfpathlineto{\pgfqpoint{2.150927in}{3.001621in}}%
\pgfpathlineto{\pgfqpoint{2.145103in}{3.012213in}}%
\pgfpathlineto{\pgfqpoint{2.139286in}{3.022626in}}%
\pgfpathlineto{\pgfqpoint{2.133477in}{3.032817in}}%
\pgfpathlineto{\pgfqpoint{2.127677in}{3.042742in}}%
\pgfpathlineto{\pgfqpoint{2.121887in}{3.052355in}}%
\pgfpathlineto{\pgfqpoint{2.110724in}{3.045065in}}%
\pgfpathlineto{\pgfqpoint{2.099575in}{3.037088in}}%
\pgfpathlineto{\pgfqpoint{2.088437in}{3.028516in}}%
\pgfpathlineto{\pgfqpoint{2.077305in}{3.019590in}}%
\pgfpathlineto{\pgfqpoint{2.066174in}{3.010553in}}%
\pgfpathlineto{\pgfqpoint{2.071933in}{3.001652in}}%
\pgfpathlineto{\pgfqpoint{2.077700in}{2.992526in}}%
\pgfpathlineto{\pgfqpoint{2.083476in}{2.983206in}}%
\pgfpathlineto{\pgfqpoint{2.089259in}{2.973724in}}%
\pgfpathclose%
\pgfusepath{stroke,fill}%
\end{pgfscope}%
\begin{pgfscope}%
\pgfpathrectangle{\pgfqpoint{0.887500in}{0.275000in}}{\pgfqpoint{4.225000in}{4.225000in}}%
\pgfusepath{clip}%
\pgfsetbuttcap%
\pgfsetroundjoin%
\definecolor{currentfill}{rgb}{0.123444,0.636809,0.528763}%
\pgfsetfillcolor{currentfill}%
\pgfsetfillopacity{0.700000}%
\pgfsetlinewidth{0.501875pt}%
\definecolor{currentstroke}{rgb}{1.000000,1.000000,1.000000}%
\pgfsetstrokecolor{currentstroke}%
\pgfsetstrokeopacity{0.500000}%
\pgfsetdash{}{0pt}%
\pgfpathmoveto{\pgfqpoint{1.953948in}{2.949139in}}%
\pgfpathlineto{\pgfqpoint{1.965251in}{2.952966in}}%
\pgfpathlineto{\pgfqpoint{1.976541in}{2.957039in}}%
\pgfpathlineto{\pgfqpoint{1.987818in}{2.961477in}}%
\pgfpathlineto{\pgfqpoint{1.999076in}{2.966397in}}%
\pgfpathlineto{\pgfqpoint{2.010315in}{2.971915in}}%
\pgfpathlineto{\pgfqpoint{2.004603in}{2.979683in}}%
\pgfpathlineto{\pgfqpoint{1.998895in}{2.987404in}}%
\pgfpathlineto{\pgfqpoint{1.993193in}{2.995073in}}%
\pgfpathlineto{\pgfqpoint{1.987496in}{3.002688in}}%
\pgfpathlineto{\pgfqpoint{1.976273in}{2.996996in}}%
\pgfpathlineto{\pgfqpoint{1.965028in}{2.991961in}}%
\pgfpathlineto{\pgfqpoint{1.953765in}{2.987451in}}%
\pgfpathlineto{\pgfqpoint{1.942485in}{2.983334in}}%
\pgfpathlineto{\pgfqpoint{1.931194in}{2.979478in}}%
\pgfpathlineto{\pgfqpoint{1.936877in}{2.971909in}}%
\pgfpathlineto{\pgfqpoint{1.942563in}{2.964329in}}%
\pgfpathlineto{\pgfqpoint{1.948254in}{2.956739in}}%
\pgfpathclose%
\pgfusepath{stroke,fill}%
\end{pgfscope}%
\begin{pgfscope}%
\pgfpathrectangle{\pgfqpoint{0.887500in}{0.275000in}}{\pgfqpoint{4.225000in}{4.225000in}}%
\pgfusepath{clip}%
\pgfsetbuttcap%
\pgfsetroundjoin%
\definecolor{currentfill}{rgb}{0.751884,0.874951,0.143228}%
\pgfsetfillcolor{currentfill}%
\pgfsetfillopacity{0.700000}%
\pgfsetlinewidth{0.501875pt}%
\definecolor{currentstroke}{rgb}{1.000000,1.000000,1.000000}%
\pgfsetstrokecolor{currentstroke}%
\pgfsetstrokeopacity{0.500000}%
\pgfsetdash{}{0pt}%
\pgfpathmoveto{\pgfqpoint{3.100907in}{3.594708in}}%
\pgfpathlineto{\pgfqpoint{3.111962in}{3.600444in}}%
\pgfpathlineto{\pgfqpoint{3.123014in}{3.606153in}}%
\pgfpathlineto{\pgfqpoint{3.134061in}{3.611803in}}%
\pgfpathlineto{\pgfqpoint{3.145104in}{3.617364in}}%
\pgfpathlineto{\pgfqpoint{3.156144in}{3.622803in}}%
\pgfpathlineto{\pgfqpoint{3.150035in}{3.623898in}}%
\pgfpathlineto{\pgfqpoint{3.143931in}{3.624835in}}%
\pgfpathlineto{\pgfqpoint{3.137834in}{3.625650in}}%
\pgfpathlineto{\pgfqpoint{3.131742in}{3.626377in}}%
\pgfpathlineto{\pgfqpoint{3.125657in}{3.627053in}}%
\pgfpathlineto{\pgfqpoint{3.114639in}{3.621590in}}%
\pgfpathlineto{\pgfqpoint{3.103618in}{3.616077in}}%
\pgfpathlineto{\pgfqpoint{3.092592in}{3.610508in}}%
\pgfpathlineto{\pgfqpoint{3.081563in}{3.604879in}}%
\pgfpathlineto{\pgfqpoint{3.070530in}{3.599185in}}%
\pgfpathlineto{\pgfqpoint{3.076593in}{3.598323in}}%
\pgfpathlineto{\pgfqpoint{3.082663in}{3.597457in}}%
\pgfpathlineto{\pgfqpoint{3.088738in}{3.596574in}}%
\pgfpathlineto{\pgfqpoint{3.094820in}{3.595662in}}%
\pgfpathclose%
\pgfusepath{stroke,fill}%
\end{pgfscope}%
\begin{pgfscope}%
\pgfpathrectangle{\pgfqpoint{0.887500in}{0.275000in}}{\pgfqpoint{4.225000in}{4.225000in}}%
\pgfusepath{clip}%
\pgfsetbuttcap%
\pgfsetroundjoin%
\definecolor{currentfill}{rgb}{0.259857,0.745492,0.444467}%
\pgfsetfillcolor{currentfill}%
\pgfsetfillopacity{0.700000}%
\pgfsetlinewidth{0.501875pt}%
\definecolor{currentstroke}{rgb}{1.000000,1.000000,1.000000}%
\pgfsetstrokecolor{currentstroke}%
\pgfsetstrokeopacity{0.500000}%
\pgfsetdash{}{0pt}%
\pgfpathmoveto{\pgfqpoint{2.628189in}{3.168580in}}%
\pgfpathlineto{\pgfqpoint{2.639185in}{3.187465in}}%
\pgfpathlineto{\pgfqpoint{2.650180in}{3.206718in}}%
\pgfpathlineto{\pgfqpoint{2.661180in}{3.225970in}}%
\pgfpathlineto{\pgfqpoint{2.672186in}{3.244853in}}%
\pgfpathlineto{\pgfqpoint{2.683202in}{3.262998in}}%
\pgfpathlineto{\pgfqpoint{2.677396in}{3.255895in}}%
\pgfpathlineto{\pgfqpoint{2.671600in}{3.248630in}}%
\pgfpathlineto{\pgfqpoint{2.665809in}{3.241677in}}%
\pgfpathlineto{\pgfqpoint{2.660020in}{3.235447in}}%
\pgfpathlineto{\pgfqpoint{2.654229in}{3.230049in}}%
\pgfpathlineto{\pgfqpoint{2.643205in}{3.216752in}}%
\pgfpathlineto{\pgfqpoint{2.632177in}{3.203837in}}%
\pgfpathlineto{\pgfqpoint{2.621145in}{3.191350in}}%
\pgfpathlineto{\pgfqpoint{2.610108in}{3.179336in}}%
\pgfpathlineto{\pgfqpoint{2.599064in}{3.167841in}}%
\pgfpathlineto{\pgfqpoint{2.604881in}{3.167585in}}%
\pgfpathlineto{\pgfqpoint{2.610701in}{3.167664in}}%
\pgfpathlineto{\pgfqpoint{2.616524in}{3.168002in}}%
\pgfpathlineto{\pgfqpoint{2.622352in}{3.168387in}}%
\pgfpathclose%
\pgfusepath{stroke,fill}%
\end{pgfscope}%
\begin{pgfscope}%
\pgfpathrectangle{\pgfqpoint{0.887500in}{0.275000in}}{\pgfqpoint{4.225000in}{4.225000in}}%
\pgfusepath{clip}%
\pgfsetbuttcap%
\pgfsetroundjoin%
\definecolor{currentfill}{rgb}{0.140210,0.665859,0.513427}%
\pgfsetfillcolor{currentfill}%
\pgfsetfillopacity{0.700000}%
\pgfsetlinewidth{0.501875pt}%
\definecolor{currentstroke}{rgb}{1.000000,1.000000,1.000000}%
\pgfsetstrokecolor{currentstroke}%
\pgfsetstrokeopacity{0.500000}%
\pgfsetdash{}{0pt}%
\pgfpathmoveto{\pgfqpoint{2.235827in}{2.992168in}}%
\pgfpathlineto{\pgfqpoint{2.246960in}{3.001215in}}%
\pgfpathlineto{\pgfqpoint{2.258095in}{3.010075in}}%
\pgfpathlineto{\pgfqpoint{2.269230in}{3.018811in}}%
\pgfpathlineto{\pgfqpoint{2.280365in}{3.027488in}}%
\pgfpathlineto{\pgfqpoint{2.291496in}{3.036170in}}%
\pgfpathlineto{\pgfqpoint{2.285695in}{3.043565in}}%
\pgfpathlineto{\pgfqpoint{2.279898in}{3.050971in}}%
\pgfpathlineto{\pgfqpoint{2.274104in}{3.058410in}}%
\pgfpathlineto{\pgfqpoint{2.268313in}{3.065906in}}%
\pgfpathlineto{\pgfqpoint{2.262523in}{3.073481in}}%
\pgfpathlineto{\pgfqpoint{2.251380in}{3.066213in}}%
\pgfpathlineto{\pgfqpoint{2.240230in}{3.059046in}}%
\pgfpathlineto{\pgfqpoint{2.229077in}{3.051944in}}%
\pgfpathlineto{\pgfqpoint{2.217920in}{3.044868in}}%
\pgfpathlineto{\pgfqpoint{2.206759in}{3.037782in}}%
\pgfpathlineto{\pgfqpoint{2.212566in}{3.028656in}}%
\pgfpathlineto{\pgfqpoint{2.218377in}{3.019518in}}%
\pgfpathlineto{\pgfqpoint{2.224191in}{3.010381in}}%
\pgfpathlineto{\pgfqpoint{2.230007in}{3.001260in}}%
\pgfpathclose%
\pgfusepath{stroke,fill}%
\end{pgfscope}%
\begin{pgfscope}%
\pgfpathrectangle{\pgfqpoint{0.887500in}{0.275000in}}{\pgfqpoint{4.225000in}{4.225000in}}%
\pgfusepath{clip}%
\pgfsetbuttcap%
\pgfsetroundjoin%
\definecolor{currentfill}{rgb}{0.964894,0.902323,0.123941}%
\pgfsetfillcolor{currentfill}%
\pgfsetfillopacity{0.700000}%
\pgfsetlinewidth{0.501875pt}%
\definecolor{currentstroke}{rgb}{1.000000,1.000000,1.000000}%
\pgfsetstrokecolor{currentstroke}%
\pgfsetstrokeopacity{0.500000}%
\pgfsetdash{}{0pt}%
\pgfpathmoveto{\pgfqpoint{3.720079in}{3.742007in}}%
\pgfpathlineto{\pgfqpoint{3.731009in}{3.746766in}}%
\pgfpathlineto{\pgfqpoint{3.741935in}{3.751485in}}%
\pgfpathlineto{\pgfqpoint{3.752856in}{3.756166in}}%
\pgfpathlineto{\pgfqpoint{3.763771in}{3.760810in}}%
\pgfpathlineto{\pgfqpoint{3.774682in}{3.765418in}}%
\pgfpathlineto{\pgfqpoint{3.768381in}{3.772660in}}%
\pgfpathlineto{\pgfqpoint{3.762081in}{3.779720in}}%
\pgfpathlineto{\pgfqpoint{3.755783in}{3.786605in}}%
\pgfpathlineto{\pgfqpoint{3.749486in}{3.793325in}}%
\pgfpathlineto{\pgfqpoint{3.743191in}{3.799888in}}%
\pgfpathlineto{\pgfqpoint{3.732293in}{3.795346in}}%
\pgfpathlineto{\pgfqpoint{3.721391in}{3.790768in}}%
\pgfpathlineto{\pgfqpoint{3.710484in}{3.786151in}}%
\pgfpathlineto{\pgfqpoint{3.699571in}{3.781497in}}%
\pgfpathlineto{\pgfqpoint{3.688654in}{3.776803in}}%
\pgfpathlineto{\pgfqpoint{3.694936in}{3.770225in}}%
\pgfpathlineto{\pgfqpoint{3.701220in}{3.763456in}}%
\pgfpathlineto{\pgfqpoint{3.707505in}{3.756496in}}%
\pgfpathlineto{\pgfqpoint{3.713791in}{3.749346in}}%
\pgfpathclose%
\pgfusepath{stroke,fill}%
\end{pgfscope}%
\begin{pgfscope}%
\pgfpathrectangle{\pgfqpoint{0.887500in}{0.275000in}}{\pgfqpoint{4.225000in}{4.225000in}}%
\pgfusepath{clip}%
\pgfsetbuttcap%
\pgfsetroundjoin%
\definecolor{currentfill}{rgb}{0.906311,0.894855,0.098125}%
\pgfsetfillcolor{currentfill}%
\pgfsetfillopacity{0.700000}%
\pgfsetlinewidth{0.501875pt}%
\definecolor{currentstroke}{rgb}{1.000000,1.000000,1.000000}%
\pgfsetstrokecolor{currentstroke}%
\pgfsetstrokeopacity{0.500000}%
\pgfsetdash{}{0pt}%
\pgfpathmoveto{\pgfqpoint{3.438212in}{3.703804in}}%
\pgfpathlineto{\pgfqpoint{3.449208in}{3.709389in}}%
\pgfpathlineto{\pgfqpoint{3.460199in}{3.714923in}}%
\pgfpathlineto{\pgfqpoint{3.471186in}{3.720404in}}%
\pgfpathlineto{\pgfqpoint{3.482168in}{3.725833in}}%
\pgfpathlineto{\pgfqpoint{3.493146in}{3.731210in}}%
\pgfpathlineto{\pgfqpoint{3.486914in}{3.736166in}}%
\pgfpathlineto{\pgfqpoint{3.480683in}{3.740817in}}%
\pgfpathlineto{\pgfqpoint{3.474455in}{3.745165in}}%
\pgfpathlineto{\pgfqpoint{3.468228in}{3.749211in}}%
\pgfpathlineto{\pgfqpoint{3.462004in}{3.752956in}}%
\pgfpathlineto{\pgfqpoint{3.451039in}{3.747235in}}%
\pgfpathlineto{\pgfqpoint{3.440071in}{3.741461in}}%
\pgfpathlineto{\pgfqpoint{3.429098in}{3.735646in}}%
\pgfpathlineto{\pgfqpoint{3.418121in}{3.729805in}}%
\pgfpathlineto{\pgfqpoint{3.407140in}{3.723939in}}%
\pgfpathlineto{\pgfqpoint{3.413349in}{3.720523in}}%
\pgfpathlineto{\pgfqpoint{3.419561in}{3.716812in}}%
\pgfpathlineto{\pgfqpoint{3.425776in}{3.712795in}}%
\pgfpathlineto{\pgfqpoint{3.431993in}{3.708462in}}%
\pgfpathclose%
\pgfusepath{stroke,fill}%
\end{pgfscope}%
\begin{pgfscope}%
\pgfpathrectangle{\pgfqpoint{0.887500in}{0.275000in}}{\pgfqpoint{4.225000in}{4.225000in}}%
\pgfusepath{clip}%
\pgfsetbuttcap%
\pgfsetroundjoin%
\definecolor{currentfill}{rgb}{0.170948,0.694384,0.493803}%
\pgfsetfillcolor{currentfill}%
\pgfsetfillopacity{0.700000}%
\pgfsetlinewidth{0.501875pt}%
\definecolor{currentstroke}{rgb}{1.000000,1.000000,1.000000}%
\pgfsetstrokecolor{currentstroke}%
\pgfsetstrokeopacity{0.500000}%
\pgfsetdash{}{0pt}%
\pgfpathmoveto{\pgfqpoint{2.517466in}{3.050702in}}%
\pgfpathlineto{\pgfqpoint{2.528629in}{3.055742in}}%
\pgfpathlineto{\pgfqpoint{2.539770in}{3.062251in}}%
\pgfpathlineto{\pgfqpoint{2.550887in}{3.070302in}}%
\pgfpathlineto{\pgfqpoint{2.561983in}{3.079886in}}%
\pgfpathlineto{\pgfqpoint{2.573059in}{3.090995in}}%
\pgfpathlineto{\pgfqpoint{2.567178in}{3.096985in}}%
\pgfpathlineto{\pgfqpoint{2.561303in}{3.102869in}}%
\pgfpathlineto{\pgfqpoint{2.555434in}{3.108646in}}%
\pgfpathlineto{\pgfqpoint{2.549570in}{3.114318in}}%
\pgfpathlineto{\pgfqpoint{2.543712in}{3.119914in}}%
\pgfpathlineto{\pgfqpoint{2.532605in}{3.112702in}}%
\pgfpathlineto{\pgfqpoint{2.521482in}{3.106464in}}%
\pgfpathlineto{\pgfqpoint{2.510342in}{3.101281in}}%
\pgfpathlineto{\pgfqpoint{2.499181in}{3.097232in}}%
\pgfpathlineto{\pgfqpoint{2.488000in}{3.094297in}}%
\pgfpathlineto{\pgfqpoint{2.493873in}{3.086513in}}%
\pgfpathlineto{\pgfqpoint{2.499756in}{3.078235in}}%
\pgfpathlineto{\pgfqpoint{2.505651in}{3.069430in}}%
\pgfpathlineto{\pgfqpoint{2.511555in}{3.060206in}}%
\pgfpathclose%
\pgfusepath{stroke,fill}%
\end{pgfscope}%
\begin{pgfscope}%
\pgfpathrectangle{\pgfqpoint{0.887500in}{0.275000in}}{\pgfqpoint{4.225000in}{4.225000in}}%
\pgfusepath{clip}%
\pgfsetbuttcap%
\pgfsetroundjoin%
\definecolor{currentfill}{rgb}{0.647257,0.858400,0.209861}%
\pgfsetfillcolor{currentfill}%
\pgfsetfillopacity{0.700000}%
\pgfsetlinewidth{0.501875pt}%
\definecolor{currentstroke}{rgb}{1.000000,1.000000,1.000000}%
\pgfsetstrokecolor{currentstroke}%
\pgfsetstrokeopacity{0.500000}%
\pgfsetdash{}{0pt}%
\pgfpathmoveto{\pgfqpoint{2.904543in}{3.521209in}}%
\pgfpathlineto{\pgfqpoint{2.915632in}{3.528233in}}%
\pgfpathlineto{\pgfqpoint{2.926720in}{3.534135in}}%
\pgfpathlineto{\pgfqpoint{2.937808in}{3.539171in}}%
\pgfpathlineto{\pgfqpoint{2.948893in}{3.543596in}}%
\pgfpathlineto{\pgfqpoint{2.959974in}{3.547667in}}%
\pgfpathlineto{\pgfqpoint{2.953961in}{3.548256in}}%
\pgfpathlineto{\pgfqpoint{2.947953in}{3.548901in}}%
\pgfpathlineto{\pgfqpoint{2.941952in}{3.549617in}}%
\pgfpathlineto{\pgfqpoint{2.935956in}{3.550417in}}%
\pgfpathlineto{\pgfqpoint{2.929965in}{3.551297in}}%
\pgfpathlineto{\pgfqpoint{2.918907in}{3.546643in}}%
\pgfpathlineto{\pgfqpoint{2.907847in}{3.541317in}}%
\pgfpathlineto{\pgfqpoint{2.896787in}{3.534967in}}%
\pgfpathlineto{\pgfqpoint{2.885730in}{3.527238in}}%
\pgfpathlineto{\pgfqpoint{2.874677in}{3.517786in}}%
\pgfpathlineto{\pgfqpoint{2.880640in}{3.517860in}}%
\pgfpathlineto{\pgfqpoint{2.886608in}{3.518265in}}%
\pgfpathlineto{\pgfqpoint{2.892581in}{3.519007in}}%
\pgfpathlineto{\pgfqpoint{2.898559in}{3.520018in}}%
\pgfpathclose%
\pgfusepath{stroke,fill}%
\end{pgfscope}%
\begin{pgfscope}%
\pgfpathrectangle{\pgfqpoint{0.887500in}{0.275000in}}{\pgfqpoint{4.225000in}{4.225000in}}%
\pgfusepath{clip}%
\pgfsetbuttcap%
\pgfsetroundjoin%
\definecolor{currentfill}{rgb}{0.945636,0.899815,0.112838}%
\pgfsetfillcolor{currentfill}%
\pgfsetfillopacity{0.700000}%
\pgfsetlinewidth{0.501875pt}%
\definecolor{currentstroke}{rgb}{1.000000,1.000000,1.000000}%
\pgfsetstrokecolor{currentstroke}%
\pgfsetstrokeopacity{0.500000}%
\pgfsetdash{}{0pt}%
\pgfpathmoveto{\pgfqpoint{3.579218in}{3.727629in}}%
\pgfpathlineto{\pgfqpoint{3.590183in}{3.732712in}}%
\pgfpathlineto{\pgfqpoint{3.601143in}{3.737765in}}%
\pgfpathlineto{\pgfqpoint{3.612098in}{3.742786in}}%
\pgfpathlineto{\pgfqpoint{3.623049in}{3.747771in}}%
\pgfpathlineto{\pgfqpoint{3.633995in}{3.752716in}}%
\pgfpathlineto{\pgfqpoint{3.627728in}{3.759115in}}%
\pgfpathlineto{\pgfqpoint{3.621463in}{3.765291in}}%
\pgfpathlineto{\pgfqpoint{3.615198in}{3.771236in}}%
\pgfpathlineto{\pgfqpoint{3.608936in}{3.776943in}}%
\pgfpathlineto{\pgfqpoint{3.602674in}{3.782402in}}%
\pgfpathlineto{\pgfqpoint{3.591742in}{3.777458in}}%
\pgfpathlineto{\pgfqpoint{3.580805in}{3.772484in}}%
\pgfpathlineto{\pgfqpoint{3.569864in}{3.767476in}}%
\pgfpathlineto{\pgfqpoint{3.558918in}{3.762431in}}%
\pgfpathlineto{\pgfqpoint{3.547968in}{3.757345in}}%
\pgfpathlineto{\pgfqpoint{3.554215in}{3.751915in}}%
\pgfpathlineto{\pgfqpoint{3.560463in}{3.746222in}}%
\pgfpathlineto{\pgfqpoint{3.566713in}{3.740271in}}%
\pgfpathlineto{\pgfqpoint{3.572965in}{3.734069in}}%
\pgfpathclose%
\pgfusepath{stroke,fill}%
\end{pgfscope}%
\begin{pgfscope}%
\pgfpathrectangle{\pgfqpoint{0.887500in}{0.275000in}}{\pgfqpoint{4.225000in}{4.225000in}}%
\pgfusepath{clip}%
\pgfsetbuttcap%
\pgfsetroundjoin%
\definecolor{currentfill}{rgb}{0.335885,0.777018,0.402049}%
\pgfsetfillcolor{currentfill}%
\pgfsetfillopacity{0.700000}%
\pgfsetlinewidth{0.501875pt}%
\definecolor{currentstroke}{rgb}{1.000000,1.000000,1.000000}%
\pgfsetstrokecolor{currentstroke}%
\pgfsetstrokeopacity{0.500000}%
\pgfsetdash{}{0pt}%
\pgfpathmoveto{\pgfqpoint{2.683202in}{3.262998in}}%
\pgfpathlineto{\pgfqpoint{2.694230in}{3.280175in}}%
\pgfpathlineto{\pgfqpoint{2.705268in}{3.296466in}}%
\pgfpathlineto{\pgfqpoint{2.716315in}{3.311990in}}%
\pgfpathlineto{\pgfqpoint{2.727368in}{3.326870in}}%
\pgfpathlineto{\pgfqpoint{2.738427in}{3.341228in}}%
\pgfpathlineto{\pgfqpoint{2.732603in}{3.331333in}}%
\pgfpathlineto{\pgfqpoint{2.726783in}{3.321946in}}%
\pgfpathlineto{\pgfqpoint{2.720965in}{3.313498in}}%
\pgfpathlineto{\pgfqpoint{2.715144in}{3.306350in}}%
\pgfpathlineto{\pgfqpoint{2.709320in}{3.300541in}}%
\pgfpathlineto{\pgfqpoint{2.698303in}{3.286076in}}%
\pgfpathlineto{\pgfqpoint{2.687287in}{3.271744in}}%
\pgfpathlineto{\pgfqpoint{2.676269in}{3.257595in}}%
\pgfpathlineto{\pgfqpoint{2.665250in}{3.243680in}}%
\pgfpathlineto{\pgfqpoint{2.654229in}{3.230049in}}%
\pgfpathlineto{\pgfqpoint{2.660020in}{3.235447in}}%
\pgfpathlineto{\pgfqpoint{2.665809in}{3.241677in}}%
\pgfpathlineto{\pgfqpoint{2.671600in}{3.248630in}}%
\pgfpathlineto{\pgfqpoint{2.677396in}{3.255895in}}%
\pgfpathclose%
\pgfusepath{stroke,fill}%
\end{pgfscope}%
\begin{pgfscope}%
\pgfpathrectangle{\pgfqpoint{0.887500in}{0.275000in}}{\pgfqpoint{4.225000in}{4.225000in}}%
\pgfusepath{clip}%
\pgfsetbuttcap%
\pgfsetroundjoin%
\definecolor{currentfill}{rgb}{0.421908,0.805774,0.351910}%
\pgfsetfillcolor{currentfill}%
\pgfsetfillopacity{0.700000}%
\pgfsetlinewidth{0.501875pt}%
\definecolor{currentstroke}{rgb}{1.000000,1.000000,1.000000}%
\pgfsetstrokecolor{currentstroke}%
\pgfsetstrokeopacity{0.500000}%
\pgfsetdash{}{0pt}%
\pgfpathmoveto{\pgfqpoint{2.738427in}{3.341228in}}%
\pgfpathlineto{\pgfqpoint{2.749490in}{3.355185in}}%
\pgfpathlineto{\pgfqpoint{2.760555in}{3.368862in}}%
\pgfpathlineto{\pgfqpoint{2.771622in}{3.382336in}}%
\pgfpathlineto{\pgfqpoint{2.782691in}{3.395611in}}%
\pgfpathlineto{\pgfqpoint{2.793761in}{3.408686in}}%
\pgfpathlineto{\pgfqpoint{2.787889in}{3.399458in}}%
\pgfpathlineto{\pgfqpoint{2.782021in}{3.391071in}}%
\pgfpathlineto{\pgfqpoint{2.776152in}{3.383843in}}%
\pgfpathlineto{\pgfqpoint{2.770282in}{3.378034in}}%
\pgfpathlineto{\pgfqpoint{2.764410in}{3.373628in}}%
\pgfpathlineto{\pgfqpoint{2.753391in}{3.358900in}}%
\pgfpathlineto{\pgfqpoint{2.742372in}{3.344258in}}%
\pgfpathlineto{\pgfqpoint{2.731354in}{3.329665in}}%
\pgfpathlineto{\pgfqpoint{2.720336in}{3.315087in}}%
\pgfpathlineto{\pgfqpoint{2.709320in}{3.300541in}}%
\pgfpathlineto{\pgfqpoint{2.715144in}{3.306350in}}%
\pgfpathlineto{\pgfqpoint{2.720965in}{3.313498in}}%
\pgfpathlineto{\pgfqpoint{2.726783in}{3.321946in}}%
\pgfpathlineto{\pgfqpoint{2.732603in}{3.331333in}}%
\pgfpathclose%
\pgfusepath{stroke,fill}%
\end{pgfscope}%
\begin{pgfscope}%
\pgfpathrectangle{\pgfqpoint{0.887500in}{0.275000in}}{\pgfqpoint{4.225000in}{4.225000in}}%
\pgfusepath{clip}%
\pgfsetbuttcap%
\pgfsetroundjoin%
\definecolor{currentfill}{rgb}{0.814576,0.883393,0.110347}%
\pgfsetfillcolor{currentfill}%
\pgfsetfillopacity{0.700000}%
\pgfsetlinewidth{0.501875pt}%
\definecolor{currentstroke}{rgb}{1.000000,1.000000,1.000000}%
\pgfsetstrokecolor{currentstroke}%
\pgfsetstrokeopacity{0.500000}%
\pgfsetdash{}{0pt}%
\pgfpathmoveto{\pgfqpoint{3.241989in}{3.637470in}}%
\pgfpathlineto{\pgfqpoint{3.253021in}{3.642310in}}%
\pgfpathlineto{\pgfqpoint{3.264050in}{3.647323in}}%
\pgfpathlineto{\pgfqpoint{3.275076in}{3.652568in}}%
\pgfpathlineto{\pgfqpoint{3.286099in}{3.658074in}}%
\pgfpathlineto{\pgfqpoint{3.297119in}{3.663804in}}%
\pgfpathlineto{\pgfqpoint{3.290948in}{3.666553in}}%
\pgfpathlineto{\pgfqpoint{3.284781in}{3.669069in}}%
\pgfpathlineto{\pgfqpoint{3.278619in}{3.671364in}}%
\pgfpathlineto{\pgfqpoint{3.272461in}{3.673457in}}%
\pgfpathlineto{\pgfqpoint{3.266308in}{3.675368in}}%
\pgfpathlineto{\pgfqpoint{3.255307in}{3.669486in}}%
\pgfpathlineto{\pgfqpoint{3.244303in}{3.663817in}}%
\pgfpathlineto{\pgfqpoint{3.233296in}{3.658399in}}%
\pgfpathlineto{\pgfqpoint{3.222286in}{3.653195in}}%
\pgfpathlineto{\pgfqpoint{3.211273in}{3.648141in}}%
\pgfpathlineto{\pgfqpoint{3.217407in}{3.646502in}}%
\pgfpathlineto{\pgfqpoint{3.223546in}{3.644637in}}%
\pgfpathlineto{\pgfqpoint{3.229689in}{3.642515in}}%
\pgfpathlineto{\pgfqpoint{3.235837in}{3.640124in}}%
\pgfpathclose%
\pgfusepath{stroke,fill}%
\end{pgfscope}%
\begin{pgfscope}%
\pgfpathrectangle{\pgfqpoint{0.887500in}{0.275000in}}{\pgfqpoint{4.225000in}{4.225000in}}%
\pgfusepath{clip}%
\pgfsetbuttcap%
\pgfsetroundjoin%
\definecolor{currentfill}{rgb}{0.124780,0.640461,0.527068}%
\pgfsetfillcolor{currentfill}%
\pgfsetfillopacity{0.700000}%
\pgfsetlinewidth{0.501875pt}%
\definecolor{currentstroke}{rgb}{1.000000,1.000000,1.000000}%
\pgfsetstrokecolor{currentstroke}%
\pgfsetstrokeopacity{0.500000}%
\pgfsetdash{}{0pt}%
\pgfpathmoveto{\pgfqpoint{2.038945in}{2.932509in}}%
\pgfpathlineto{\pgfqpoint{2.050197in}{2.937884in}}%
\pgfpathlineto{\pgfqpoint{2.061432in}{2.943758in}}%
\pgfpathlineto{\pgfqpoint{2.072651in}{2.950144in}}%
\pgfpathlineto{\pgfqpoint{2.083855in}{2.956959in}}%
\pgfpathlineto{\pgfqpoint{2.095049in}{2.964109in}}%
\pgfpathlineto{\pgfqpoint{2.089259in}{2.973724in}}%
\pgfpathlineto{\pgfqpoint{2.083476in}{2.983206in}}%
\pgfpathlineto{\pgfqpoint{2.077700in}{2.992526in}}%
\pgfpathlineto{\pgfqpoint{2.071933in}{3.001652in}}%
\pgfpathlineto{\pgfqpoint{2.066174in}{3.010553in}}%
\pgfpathlineto{\pgfqpoint{2.055037in}{3.001650in}}%
\pgfpathlineto{\pgfqpoint{2.043889in}{2.993123in}}%
\pgfpathlineto{\pgfqpoint{2.032722in}{2.985217in}}%
\pgfpathlineto{\pgfqpoint{2.021531in}{2.978150in}}%
\pgfpathlineto{\pgfqpoint{2.010315in}{2.971915in}}%
\pgfpathlineto{\pgfqpoint{2.016032in}{2.964104in}}%
\pgfpathlineto{\pgfqpoint{2.021754in}{2.956254in}}%
\pgfpathlineto{\pgfqpoint{2.027480in}{2.948369in}}%
\pgfpathlineto{\pgfqpoint{2.033210in}{2.940452in}}%
\pgfpathclose%
\pgfusepath{stroke,fill}%
\end{pgfscope}%
\begin{pgfscope}%
\pgfpathrectangle{\pgfqpoint{0.887500in}{0.275000in}}{\pgfqpoint{4.225000in}{4.225000in}}%
\pgfusepath{clip}%
\pgfsetbuttcap%
\pgfsetroundjoin%
\definecolor{currentfill}{rgb}{0.130067,0.651384,0.521608}%
\pgfsetfillcolor{currentfill}%
\pgfsetfillopacity{0.700000}%
\pgfsetlinewidth{0.501875pt}%
\definecolor{currentstroke}{rgb}{1.000000,1.000000,1.000000}%
\pgfsetstrokecolor{currentstroke}%
\pgfsetstrokeopacity{0.500000}%
\pgfsetdash{}{0pt}%
\pgfpathmoveto{\pgfqpoint{2.180115in}{2.947539in}}%
\pgfpathlineto{\pgfqpoint{2.191274in}{2.955908in}}%
\pgfpathlineto{\pgfqpoint{2.202421in}{2.964699in}}%
\pgfpathlineto{\pgfqpoint{2.213560in}{2.973769in}}%
\pgfpathlineto{\pgfqpoint{2.224694in}{2.982973in}}%
\pgfpathlineto{\pgfqpoint{2.235827in}{2.992168in}}%
\pgfpathlineto{\pgfqpoint{2.230007in}{3.001260in}}%
\pgfpathlineto{\pgfqpoint{2.224191in}{3.010381in}}%
\pgfpathlineto{\pgfqpoint{2.218377in}{3.019518in}}%
\pgfpathlineto{\pgfqpoint{2.212566in}{3.028656in}}%
\pgfpathlineto{\pgfqpoint{2.206759in}{3.037782in}}%
\pgfpathlineto{\pgfqpoint{2.195597in}{3.030662in}}%
\pgfpathlineto{\pgfqpoint{2.184432in}{3.023496in}}%
\pgfpathlineto{\pgfqpoint{2.173265in}{3.016274in}}%
\pgfpathlineto{\pgfqpoint{2.162097in}{3.008986in}}%
\pgfpathlineto{\pgfqpoint{2.150927in}{3.001621in}}%
\pgfpathlineto{\pgfqpoint{2.156757in}{2.990895in}}%
\pgfpathlineto{\pgfqpoint{2.162592in}{2.980079in}}%
\pgfpathlineto{\pgfqpoint{2.168430in}{2.969219in}}%
\pgfpathlineto{\pgfqpoint{2.174272in}{2.958357in}}%
\pgfpathclose%
\pgfusepath{stroke,fill}%
\end{pgfscope}%
\begin{pgfscope}%
\pgfpathrectangle{\pgfqpoint{0.887500in}{0.275000in}}{\pgfqpoint{4.225000in}{4.225000in}}%
\pgfusepath{clip}%
\pgfsetbuttcap%
\pgfsetroundjoin%
\definecolor{currentfill}{rgb}{0.506271,0.828786,0.300362}%
\pgfsetfillcolor{currentfill}%
\pgfsetfillopacity{0.700000}%
\pgfsetlinewidth{0.501875pt}%
\definecolor{currentstroke}{rgb}{1.000000,1.000000,1.000000}%
\pgfsetstrokecolor{currentstroke}%
\pgfsetstrokeopacity{0.500000}%
\pgfsetdash{}{0pt}%
\pgfpathmoveto{\pgfqpoint{2.793761in}{3.408686in}}%
\pgfpathlineto{\pgfqpoint{2.804833in}{3.421561in}}%
\pgfpathlineto{\pgfqpoint{2.815906in}{3.434236in}}%
\pgfpathlineto{\pgfqpoint{2.826981in}{3.446709in}}%
\pgfpathlineto{\pgfqpoint{2.838056in}{3.458977in}}%
\pgfpathlineto{\pgfqpoint{2.849132in}{3.470964in}}%
\pgfpathlineto{\pgfqpoint{2.843200in}{3.465206in}}%
\pgfpathlineto{\pgfqpoint{2.837274in}{3.460025in}}%
\pgfpathlineto{\pgfqpoint{2.831350in}{3.455634in}}%
\pgfpathlineto{\pgfqpoint{2.825429in}{3.452203in}}%
\pgfpathlineto{\pgfqpoint{2.819509in}{3.449716in}}%
\pgfpathlineto{\pgfqpoint{2.808488in}{3.434168in}}%
\pgfpathlineto{\pgfqpoint{2.797469in}{3.418717in}}%
\pgfpathlineto{\pgfqpoint{2.786449in}{3.403499in}}%
\pgfpathlineto{\pgfqpoint{2.775430in}{3.388482in}}%
\pgfpathlineto{\pgfqpoint{2.764410in}{3.373628in}}%
\pgfpathlineto{\pgfqpoint{2.770282in}{3.378034in}}%
\pgfpathlineto{\pgfqpoint{2.776152in}{3.383843in}}%
\pgfpathlineto{\pgfqpoint{2.782021in}{3.391071in}}%
\pgfpathlineto{\pgfqpoint{2.787889in}{3.399458in}}%
\pgfpathclose%
\pgfusepath{stroke,fill}%
\end{pgfscope}%
\begin{pgfscope}%
\pgfpathrectangle{\pgfqpoint{0.887500in}{0.275000in}}{\pgfqpoint{4.225000in}{4.225000in}}%
\pgfusepath{clip}%
\pgfsetbuttcap%
\pgfsetroundjoin%
\definecolor{currentfill}{rgb}{0.170948,0.694384,0.493803}%
\pgfsetfillcolor{currentfill}%
\pgfsetfillopacity{0.700000}%
\pgfsetlinewidth{0.501875pt}%
\definecolor{currentstroke}{rgb}{1.000000,1.000000,1.000000}%
\pgfsetstrokecolor{currentstroke}%
\pgfsetstrokeopacity{0.500000}%
\pgfsetdash{}{0pt}%
\pgfpathmoveto{\pgfqpoint{2.376178in}{3.047676in}}%
\pgfpathlineto{\pgfqpoint{2.387306in}{3.056808in}}%
\pgfpathlineto{\pgfqpoint{2.398445in}{3.065154in}}%
\pgfpathlineto{\pgfqpoint{2.409598in}{3.072455in}}%
\pgfpathlineto{\pgfqpoint{2.420769in}{3.078452in}}%
\pgfpathlineto{\pgfqpoint{2.431960in}{3.082942in}}%
\pgfpathlineto{\pgfqpoint{2.426107in}{3.090389in}}%
\pgfpathlineto{\pgfqpoint{2.420267in}{3.097244in}}%
\pgfpathlineto{\pgfqpoint{2.414439in}{3.103619in}}%
\pgfpathlineto{\pgfqpoint{2.408621in}{3.109627in}}%
\pgfpathlineto{\pgfqpoint{2.402811in}{3.115378in}}%
\pgfpathlineto{\pgfqpoint{2.391639in}{3.110661in}}%
\pgfpathlineto{\pgfqpoint{2.380484in}{3.104661in}}%
\pgfpathlineto{\pgfqpoint{2.369344in}{3.097565in}}%
\pgfpathlineto{\pgfqpoint{2.358215in}{3.089606in}}%
\pgfpathlineto{\pgfqpoint{2.347095in}{3.081018in}}%
\pgfpathlineto{\pgfqpoint{2.352895in}{3.074850in}}%
\pgfpathlineto{\pgfqpoint{2.358702in}{3.068510in}}%
\pgfpathlineto{\pgfqpoint{2.364517in}{3.061920in}}%
\pgfpathlineto{\pgfqpoint{2.370342in}{3.055001in}}%
\pgfpathclose%
\pgfusepath{stroke,fill}%
\end{pgfscope}%
\begin{pgfscope}%
\pgfpathrectangle{\pgfqpoint{0.887500in}{0.275000in}}{\pgfqpoint{4.225000in}{4.225000in}}%
\pgfusepath{clip}%
\pgfsetbuttcap%
\pgfsetroundjoin%
\definecolor{currentfill}{rgb}{0.595839,0.848717,0.243329}%
\pgfsetfillcolor{currentfill}%
\pgfsetfillopacity{0.700000}%
\pgfsetlinewidth{0.501875pt}%
\definecolor{currentstroke}{rgb}{1.000000,1.000000,1.000000}%
\pgfsetstrokecolor{currentstroke}%
\pgfsetstrokeopacity{0.500000}%
\pgfsetdash{}{0pt}%
\pgfpathmoveto{\pgfqpoint{2.849132in}{3.470964in}}%
\pgfpathlineto{\pgfqpoint{2.860210in}{3.482515in}}%
\pgfpathlineto{\pgfqpoint{2.871290in}{3.493473in}}%
\pgfpathlineto{\pgfqpoint{2.882372in}{3.503680in}}%
\pgfpathlineto{\pgfqpoint{2.893457in}{3.512978in}}%
\pgfpathlineto{\pgfqpoint{2.904543in}{3.521209in}}%
\pgfpathlineto{\pgfqpoint{2.898559in}{3.520018in}}%
\pgfpathlineto{\pgfqpoint{2.892581in}{3.519007in}}%
\pgfpathlineto{\pgfqpoint{2.886608in}{3.518265in}}%
\pgfpathlineto{\pgfqpoint{2.880640in}{3.517860in}}%
\pgfpathlineto{\pgfqpoint{2.874677in}{3.517786in}}%
\pgfpathlineto{\pgfqpoint{2.863631in}{3.506535in}}%
\pgfpathlineto{\pgfqpoint{2.852592in}{3.493780in}}%
\pgfpathlineto{\pgfqpoint{2.841559in}{3.479844in}}%
\pgfpathlineto{\pgfqpoint{2.830532in}{3.465049in}}%
\pgfpathlineto{\pgfqpoint{2.819509in}{3.449716in}}%
\pgfpathlineto{\pgfqpoint{2.825429in}{3.452203in}}%
\pgfpathlineto{\pgfqpoint{2.831350in}{3.455634in}}%
\pgfpathlineto{\pgfqpoint{2.837274in}{3.460025in}}%
\pgfpathlineto{\pgfqpoint{2.843200in}{3.465206in}}%
\pgfpathclose%
\pgfusepath{stroke,fill}%
\end{pgfscope}%
\begin{pgfscope}%
\pgfpathrectangle{\pgfqpoint{0.887500in}{0.275000in}}{\pgfqpoint{4.225000in}{4.225000in}}%
\pgfusepath{clip}%
\pgfsetbuttcap%
\pgfsetroundjoin%
\definecolor{currentfill}{rgb}{0.720391,0.870350,0.162603}%
\pgfsetfillcolor{currentfill}%
\pgfsetfillopacity{0.700000}%
\pgfsetlinewidth{0.501875pt}%
\definecolor{currentstroke}{rgb}{1.000000,1.000000,1.000000}%
\pgfsetstrokecolor{currentstroke}%
\pgfsetstrokeopacity{0.500000}%
\pgfsetdash{}{0pt}%
\pgfpathmoveto{\pgfqpoint{3.045573in}{3.566955in}}%
\pgfpathlineto{\pgfqpoint{3.056648in}{3.572223in}}%
\pgfpathlineto{\pgfqpoint{3.067718in}{3.577684in}}%
\pgfpathlineto{\pgfqpoint{3.078785in}{3.583285in}}%
\pgfpathlineto{\pgfqpoint{3.089848in}{3.588976in}}%
\pgfpathlineto{\pgfqpoint{3.100907in}{3.594708in}}%
\pgfpathlineto{\pgfqpoint{3.094820in}{3.595662in}}%
\pgfpathlineto{\pgfqpoint{3.088738in}{3.596574in}}%
\pgfpathlineto{\pgfqpoint{3.082663in}{3.597457in}}%
\pgfpathlineto{\pgfqpoint{3.076593in}{3.598323in}}%
\pgfpathlineto{\pgfqpoint{3.070530in}{3.599185in}}%
\pgfpathlineto{\pgfqpoint{3.059493in}{3.593419in}}%
\pgfpathlineto{\pgfqpoint{3.048452in}{3.587597in}}%
\pgfpathlineto{\pgfqpoint{3.037407in}{3.581788in}}%
\pgfpathlineto{\pgfqpoint{3.026359in}{3.576074in}}%
\pgfpathlineto{\pgfqpoint{3.015307in}{3.570535in}}%
\pgfpathlineto{\pgfqpoint{3.021348in}{3.569717in}}%
\pgfpathlineto{\pgfqpoint{3.027395in}{3.568948in}}%
\pgfpathlineto{\pgfqpoint{3.033448in}{3.568230in}}%
\pgfpathlineto{\pgfqpoint{3.039508in}{3.567564in}}%
\pgfpathclose%
\pgfusepath{stroke,fill}%
\end{pgfscope}%
\begin{pgfscope}%
\pgfpathrectangle{\pgfqpoint{0.887500in}{0.275000in}}{\pgfqpoint{4.225000in}{4.225000in}}%
\pgfusepath{clip}%
\pgfsetbuttcap%
\pgfsetroundjoin%
\definecolor{currentfill}{rgb}{0.955300,0.901065,0.118128}%
\pgfsetfillcolor{currentfill}%
\pgfsetfillopacity{0.700000}%
\pgfsetlinewidth{0.501875pt}%
\definecolor{currentstroke}{rgb}{1.000000,1.000000,1.000000}%
\pgfsetstrokecolor{currentstroke}%
\pgfsetstrokeopacity{0.500000}%
\pgfsetdash{}{0pt}%
\pgfpathmoveto{\pgfqpoint{3.806197in}{3.726174in}}%
\pgfpathlineto{\pgfqpoint{3.817116in}{3.730834in}}%
\pgfpathlineto{\pgfqpoint{3.828029in}{3.735474in}}%
\pgfpathlineto{\pgfqpoint{3.838939in}{3.740099in}}%
\pgfpathlineto{\pgfqpoint{3.849843in}{3.744710in}}%
\pgfpathlineto{\pgfqpoint{3.843530in}{3.752928in}}%
\pgfpathlineto{\pgfqpoint{3.837217in}{3.760904in}}%
\pgfpathlineto{\pgfqpoint{3.830903in}{3.768649in}}%
\pgfpathlineto{\pgfqpoint{3.824589in}{3.776177in}}%
\pgfpathlineto{\pgfqpoint{3.818276in}{3.783505in}}%
\pgfpathlineto{\pgfqpoint{3.807385in}{3.779033in}}%
\pgfpathlineto{\pgfqpoint{3.796489in}{3.774529in}}%
\pgfpathlineto{\pgfqpoint{3.785588in}{3.769990in}}%
\pgfpathlineto{\pgfqpoint{3.774682in}{3.765418in}}%
\pgfpathlineto{\pgfqpoint{3.780984in}{3.757985in}}%
\pgfpathlineto{\pgfqpoint{3.787287in}{3.750352in}}%
\pgfpathlineto{\pgfqpoint{3.793590in}{3.742511in}}%
\pgfpathlineto{\pgfqpoint{3.799894in}{3.734454in}}%
\pgfpathclose%
\pgfusepath{stroke,fill}%
\end{pgfscope}%
\begin{pgfscope}%
\pgfpathrectangle{\pgfqpoint{0.887500in}{0.275000in}}{\pgfqpoint{4.225000in}{4.225000in}}%
\pgfusepath{clip}%
\pgfsetbuttcap%
\pgfsetroundjoin%
\definecolor{currentfill}{rgb}{0.123444,0.636809,0.528763}%
\pgfsetfillcolor{currentfill}%
\pgfsetfillopacity{0.700000}%
\pgfsetlinewidth{0.501875pt}%
\definecolor{currentstroke}{rgb}{1.000000,1.000000,1.000000}%
\pgfsetstrokecolor{currentstroke}%
\pgfsetstrokeopacity{0.500000}%
\pgfsetdash{}{0pt}%
\pgfpathmoveto{\pgfqpoint{1.897316in}{2.931476in}}%
\pgfpathlineto{\pgfqpoint{1.908656in}{2.934897in}}%
\pgfpathlineto{\pgfqpoint{1.919990in}{2.938353in}}%
\pgfpathlineto{\pgfqpoint{1.931317in}{2.941864in}}%
\pgfpathlineto{\pgfqpoint{1.942637in}{2.945452in}}%
\pgfpathlineto{\pgfqpoint{1.953948in}{2.949139in}}%
\pgfpathlineto{\pgfqpoint{1.948254in}{2.956739in}}%
\pgfpathlineto{\pgfqpoint{1.942563in}{2.964329in}}%
\pgfpathlineto{\pgfqpoint{1.936877in}{2.971909in}}%
\pgfpathlineto{\pgfqpoint{1.931194in}{2.979478in}}%
\pgfpathlineto{\pgfqpoint{1.919894in}{2.975764in}}%
\pgfpathlineto{\pgfqpoint{1.908585in}{2.972147in}}%
\pgfpathlineto{\pgfqpoint{1.897269in}{2.968605in}}%
\pgfpathlineto{\pgfqpoint{1.885947in}{2.965117in}}%
\pgfpathlineto{\pgfqpoint{1.874618in}{2.961661in}}%
\pgfpathlineto{\pgfqpoint{1.880287in}{2.954134in}}%
\pgfpathlineto{\pgfqpoint{1.885959in}{2.946594in}}%
\pgfpathlineto{\pgfqpoint{1.891636in}{2.939041in}}%
\pgfpathclose%
\pgfusepath{stroke,fill}%
\end{pgfscope}%
\begin{pgfscope}%
\pgfpathrectangle{\pgfqpoint{0.887500in}{0.275000in}}{\pgfqpoint{4.225000in}{4.225000in}}%
\pgfusepath{clip}%
\pgfsetbuttcap%
\pgfsetroundjoin%
\definecolor{currentfill}{rgb}{0.886271,0.892374,0.095374}%
\pgfsetfillcolor{currentfill}%
\pgfsetfillopacity{0.700000}%
\pgfsetlinewidth{0.501875pt}%
\definecolor{currentstroke}{rgb}{1.000000,1.000000,1.000000}%
\pgfsetstrokecolor{currentstroke}%
\pgfsetstrokeopacity{0.500000}%
\pgfsetdash{}{0pt}%
\pgfpathmoveto{\pgfqpoint{3.383170in}{3.675149in}}%
\pgfpathlineto{\pgfqpoint{3.394187in}{3.680972in}}%
\pgfpathlineto{\pgfqpoint{3.405200in}{3.686751in}}%
\pgfpathlineto{\pgfqpoint{3.416208in}{3.692484in}}%
\pgfpathlineto{\pgfqpoint{3.427212in}{3.698168in}}%
\pgfpathlineto{\pgfqpoint{3.438212in}{3.703804in}}%
\pgfpathlineto{\pgfqpoint{3.431993in}{3.708462in}}%
\pgfpathlineto{\pgfqpoint{3.425776in}{3.712795in}}%
\pgfpathlineto{\pgfqpoint{3.419561in}{3.716812in}}%
\pgfpathlineto{\pgfqpoint{3.413349in}{3.720523in}}%
\pgfpathlineto{\pgfqpoint{3.407140in}{3.723939in}}%
\pgfpathlineto{\pgfqpoint{3.396155in}{3.718045in}}%
\pgfpathlineto{\pgfqpoint{3.385166in}{3.712118in}}%
\pgfpathlineto{\pgfqpoint{3.374173in}{3.706151in}}%
\pgfpathlineto{\pgfqpoint{3.363177in}{3.700141in}}%
\pgfpathlineto{\pgfqpoint{3.352176in}{3.694082in}}%
\pgfpathlineto{\pgfqpoint{3.358369in}{3.690883in}}%
\pgfpathlineto{\pgfqpoint{3.364565in}{3.687404in}}%
\pgfpathlineto{\pgfqpoint{3.370764in}{3.683630in}}%
\pgfpathlineto{\pgfqpoint{3.376966in}{3.679549in}}%
\pgfpathclose%
\pgfusepath{stroke,fill}%
\end{pgfscope}%
\begin{pgfscope}%
\pgfpathrectangle{\pgfqpoint{0.887500in}{0.275000in}}{\pgfqpoint{4.225000in}{4.225000in}}%
\pgfusepath{clip}%
\pgfsetbuttcap%
\pgfsetroundjoin%
\definecolor{currentfill}{rgb}{0.945636,0.899815,0.112838}%
\pgfsetfillcolor{currentfill}%
\pgfsetfillopacity{0.700000}%
\pgfsetlinewidth{0.501875pt}%
\definecolor{currentstroke}{rgb}{1.000000,1.000000,1.000000}%
\pgfsetstrokecolor{currentstroke}%
\pgfsetstrokeopacity{0.500000}%
\pgfsetdash{}{0pt}%
\pgfpathmoveto{\pgfqpoint{3.665353in}{3.717597in}}%
\pgfpathlineto{\pgfqpoint{3.676308in}{3.722564in}}%
\pgfpathlineto{\pgfqpoint{3.687258in}{3.727488in}}%
\pgfpathlineto{\pgfqpoint{3.698203in}{3.732370in}}%
\pgfpathlineto{\pgfqpoint{3.709143in}{3.737209in}}%
\pgfpathlineto{\pgfqpoint{3.720079in}{3.742007in}}%
\pgfpathlineto{\pgfqpoint{3.713791in}{3.749346in}}%
\pgfpathlineto{\pgfqpoint{3.707505in}{3.756496in}}%
\pgfpathlineto{\pgfqpoint{3.701220in}{3.763456in}}%
\pgfpathlineto{\pgfqpoint{3.694936in}{3.770225in}}%
\pgfpathlineto{\pgfqpoint{3.688654in}{3.776803in}}%
\pgfpathlineto{\pgfqpoint{3.677732in}{3.772069in}}%
\pgfpathlineto{\pgfqpoint{3.666805in}{3.767294in}}%
\pgfpathlineto{\pgfqpoint{3.655873in}{3.762478in}}%
\pgfpathlineto{\pgfqpoint{3.644937in}{3.757619in}}%
\pgfpathlineto{\pgfqpoint{3.633995in}{3.752716in}}%
\pgfpathlineto{\pgfqpoint{3.640264in}{3.746100in}}%
\pgfpathlineto{\pgfqpoint{3.646534in}{3.739274in}}%
\pgfpathlineto{\pgfqpoint{3.652805in}{3.732244in}}%
\pgfpathlineto{\pgfqpoint{3.659078in}{3.725016in}}%
\pgfpathclose%
\pgfusepath{stroke,fill}%
\end{pgfscope}%
\begin{pgfscope}%
\pgfpathrectangle{\pgfqpoint{0.887500in}{0.275000in}}{\pgfqpoint{4.225000in}{4.225000in}}%
\pgfusepath{clip}%
\pgfsetbuttcap%
\pgfsetroundjoin%
\definecolor{currentfill}{rgb}{0.150148,0.676631,0.506589}%
\pgfsetfillcolor{currentfill}%
\pgfsetfillopacity{0.700000}%
\pgfsetlinewidth{0.501875pt}%
\definecolor{currentstroke}{rgb}{1.000000,1.000000,1.000000}%
\pgfsetstrokecolor{currentstroke}%
\pgfsetstrokeopacity{0.500000}%
\pgfsetdash{}{0pt}%
\pgfpathmoveto{\pgfqpoint{2.320570in}{2.998560in}}%
\pgfpathlineto{\pgfqpoint{2.331696in}{3.008330in}}%
\pgfpathlineto{\pgfqpoint{2.342819in}{3.018152in}}%
\pgfpathlineto{\pgfqpoint{2.353939in}{3.028090in}}%
\pgfpathlineto{\pgfqpoint{2.365057in}{3.038017in}}%
\pgfpathlineto{\pgfqpoint{2.376178in}{3.047676in}}%
\pgfpathlineto{\pgfqpoint{2.370342in}{3.055001in}}%
\pgfpathlineto{\pgfqpoint{2.364517in}{3.061920in}}%
\pgfpathlineto{\pgfqpoint{2.358702in}{3.068510in}}%
\pgfpathlineto{\pgfqpoint{2.352895in}{3.074850in}}%
\pgfpathlineto{\pgfqpoint{2.347095in}{3.081018in}}%
\pgfpathlineto{\pgfqpoint{2.335980in}{3.072035in}}%
\pgfpathlineto{\pgfqpoint{2.324865in}{3.062889in}}%
\pgfpathlineto{\pgfqpoint{2.313748in}{3.053812in}}%
\pgfpathlineto{\pgfqpoint{2.302624in}{3.044923in}}%
\pgfpathlineto{\pgfqpoint{2.291496in}{3.036170in}}%
\pgfpathlineto{\pgfqpoint{2.297301in}{3.028763in}}%
\pgfpathlineto{\pgfqpoint{2.303111in}{3.021321in}}%
\pgfpathlineto{\pgfqpoint{2.308925in}{3.013822in}}%
\pgfpathlineto{\pgfqpoint{2.314745in}{3.006243in}}%
\pgfpathclose%
\pgfusepath{stroke,fill}%
\end{pgfscope}%
\begin{pgfscope}%
\pgfpathrectangle{\pgfqpoint{0.887500in}{0.275000in}}{\pgfqpoint{4.225000in}{4.225000in}}%
\pgfusepath{clip}%
\pgfsetbuttcap%
\pgfsetroundjoin%
\definecolor{currentfill}{rgb}{0.926106,0.897330,0.104071}%
\pgfsetfillcolor{currentfill}%
\pgfsetfillopacity{0.700000}%
\pgfsetlinewidth{0.501875pt}%
\definecolor{currentstroke}{rgb}{1.000000,1.000000,1.000000}%
\pgfsetstrokecolor{currentstroke}%
\pgfsetstrokeopacity{0.500000}%
\pgfsetdash{}{0pt}%
\pgfpathmoveto{\pgfqpoint{3.524329in}{3.701857in}}%
\pgfpathlineto{\pgfqpoint{3.535316in}{3.707059in}}%
\pgfpathlineto{\pgfqpoint{3.546298in}{3.712236in}}%
\pgfpathlineto{\pgfqpoint{3.557276in}{3.717389in}}%
\pgfpathlineto{\pgfqpoint{3.568249in}{3.722520in}}%
\pgfpathlineto{\pgfqpoint{3.579218in}{3.727629in}}%
\pgfpathlineto{\pgfqpoint{3.572965in}{3.734069in}}%
\pgfpathlineto{\pgfqpoint{3.566713in}{3.740271in}}%
\pgfpathlineto{\pgfqpoint{3.560463in}{3.746222in}}%
\pgfpathlineto{\pgfqpoint{3.554215in}{3.751915in}}%
\pgfpathlineto{\pgfqpoint{3.547968in}{3.757345in}}%
\pgfpathlineto{\pgfqpoint{3.537012in}{3.752216in}}%
\pgfpathlineto{\pgfqpoint{3.526053in}{3.747039in}}%
\pgfpathlineto{\pgfqpoint{3.515088in}{3.741813in}}%
\pgfpathlineto{\pgfqpoint{3.504120in}{3.736537in}}%
\pgfpathlineto{\pgfqpoint{3.493146in}{3.731210in}}%
\pgfpathlineto{\pgfqpoint{3.499380in}{3.725948in}}%
\pgfpathlineto{\pgfqpoint{3.505615in}{3.720379in}}%
\pgfpathlineto{\pgfqpoint{3.511852in}{3.714500in}}%
\pgfpathlineto{\pgfqpoint{3.518090in}{3.708318in}}%
\pgfpathclose%
\pgfusepath{stroke,fill}%
\end{pgfscope}%
\begin{pgfscope}%
\pgfpathrectangle{\pgfqpoint{0.887500in}{0.275000in}}{\pgfqpoint{4.225000in}{4.225000in}}%
\pgfusepath{clip}%
\pgfsetbuttcap%
\pgfsetroundjoin%
\definecolor{currentfill}{rgb}{0.123444,0.636809,0.528763}%
\pgfsetfillcolor{currentfill}%
\pgfsetfillopacity{0.700000}%
\pgfsetlinewidth{0.501875pt}%
\definecolor{currentstroke}{rgb}{1.000000,1.000000,1.000000}%
\pgfsetstrokecolor{currentstroke}%
\pgfsetstrokeopacity{0.500000}%
\pgfsetdash{}{0pt}%
\pgfpathmoveto{\pgfqpoint{2.124064in}{2.915125in}}%
\pgfpathlineto{\pgfqpoint{2.135308in}{2.920432in}}%
\pgfpathlineto{\pgfqpoint{2.146537in}{2.926235in}}%
\pgfpathlineto{\pgfqpoint{2.157749in}{2.932637in}}%
\pgfpathlineto{\pgfqpoint{2.168941in}{2.939738in}}%
\pgfpathlineto{\pgfqpoint{2.180115in}{2.947539in}}%
\pgfpathlineto{\pgfqpoint{2.174272in}{2.958357in}}%
\pgfpathlineto{\pgfqpoint{2.168430in}{2.969219in}}%
\pgfpathlineto{\pgfqpoint{2.162592in}{2.980079in}}%
\pgfpathlineto{\pgfqpoint{2.156757in}{2.990895in}}%
\pgfpathlineto{\pgfqpoint{2.150927in}{3.001621in}}%
\pgfpathlineto{\pgfqpoint{2.139756in}{2.994168in}}%
\pgfpathlineto{\pgfqpoint{2.128584in}{2.986623in}}%
\pgfpathlineto{\pgfqpoint{2.117411in}{2.979035in}}%
\pgfpathlineto{\pgfqpoint{2.106233in}{2.971499in}}%
\pgfpathlineto{\pgfqpoint{2.095049in}{2.964109in}}%
\pgfpathlineto{\pgfqpoint{2.100843in}{2.954392in}}%
\pgfpathlineto{\pgfqpoint{2.106643in}{2.944604in}}%
\pgfpathlineto{\pgfqpoint{2.112447in}{2.934777in}}%
\pgfpathlineto{\pgfqpoint{2.118254in}{2.924940in}}%
\pgfpathclose%
\pgfusepath{stroke,fill}%
\end{pgfscope}%
\begin{pgfscope}%
\pgfpathrectangle{\pgfqpoint{0.887500in}{0.275000in}}{\pgfqpoint{4.225000in}{4.225000in}}%
\pgfusepath{clip}%
\pgfsetbuttcap%
\pgfsetroundjoin%
\definecolor{currentfill}{rgb}{0.162016,0.687316,0.499129}%
\pgfsetfillcolor{currentfill}%
\pgfsetfillopacity{0.700000}%
\pgfsetlinewidth{0.501875pt}%
\definecolor{currentstroke}{rgb}{1.000000,1.000000,1.000000}%
\pgfsetstrokecolor{currentstroke}%
\pgfsetstrokeopacity{0.500000}%
\pgfsetdash{}{0pt}%
\pgfpathmoveto{\pgfqpoint{2.461451in}{3.034392in}}%
\pgfpathlineto{\pgfqpoint{2.472662in}{3.037800in}}%
\pgfpathlineto{\pgfqpoint{2.483875in}{3.040735in}}%
\pgfpathlineto{\pgfqpoint{2.495084in}{3.043598in}}%
\pgfpathlineto{\pgfqpoint{2.506282in}{3.046787in}}%
\pgfpathlineto{\pgfqpoint{2.517466in}{3.050702in}}%
\pgfpathlineto{\pgfqpoint{2.511555in}{3.060206in}}%
\pgfpathlineto{\pgfqpoint{2.505651in}{3.069430in}}%
\pgfpathlineto{\pgfqpoint{2.499756in}{3.078235in}}%
\pgfpathlineto{\pgfqpoint{2.493873in}{3.086513in}}%
\pgfpathlineto{\pgfqpoint{2.488000in}{3.094297in}}%
\pgfpathlineto{\pgfqpoint{2.476802in}{3.092132in}}%
\pgfpathlineto{\pgfqpoint{2.465593in}{3.090326in}}%
\pgfpathlineto{\pgfqpoint{2.454379in}{3.088467in}}%
\pgfpathlineto{\pgfqpoint{2.443165in}{3.086143in}}%
\pgfpathlineto{\pgfqpoint{2.431960in}{3.082942in}}%
\pgfpathlineto{\pgfqpoint{2.437827in}{3.074790in}}%
\pgfpathlineto{\pgfqpoint{2.443711in}{3.065821in}}%
\pgfpathlineto{\pgfqpoint{2.449612in}{3.055985in}}%
\pgfpathlineto{\pgfqpoint{2.455526in}{3.045441in}}%
\pgfpathclose%
\pgfusepath{stroke,fill}%
\end{pgfscope}%
\begin{pgfscope}%
\pgfpathrectangle{\pgfqpoint{0.887500in}{0.275000in}}{\pgfqpoint{4.225000in}{4.225000in}}%
\pgfusepath{clip}%
\pgfsetbuttcap%
\pgfsetroundjoin%
\definecolor{currentfill}{rgb}{0.121380,0.629492,0.531973}%
\pgfsetfillcolor{currentfill}%
\pgfsetfillopacity{0.700000}%
\pgfsetlinewidth{0.501875pt}%
\definecolor{currentstroke}{rgb}{1.000000,1.000000,1.000000}%
\pgfsetstrokecolor{currentstroke}%
\pgfsetstrokeopacity{0.500000}%
\pgfsetdash{}{0pt}%
\pgfpathmoveto{\pgfqpoint{1.982480in}{2.910974in}}%
\pgfpathlineto{\pgfqpoint{1.993796in}{2.914775in}}%
\pgfpathlineto{\pgfqpoint{2.005102in}{2.918767in}}%
\pgfpathlineto{\pgfqpoint{2.016397in}{2.923012in}}%
\pgfpathlineto{\pgfqpoint{2.027678in}{2.927573in}}%
\pgfpathlineto{\pgfqpoint{2.038945in}{2.932509in}}%
\pgfpathlineto{\pgfqpoint{2.033210in}{2.940452in}}%
\pgfpathlineto{\pgfqpoint{2.027480in}{2.948369in}}%
\pgfpathlineto{\pgfqpoint{2.021754in}{2.956254in}}%
\pgfpathlineto{\pgfqpoint{2.016032in}{2.964104in}}%
\pgfpathlineto{\pgfqpoint{2.010315in}{2.971915in}}%
\pgfpathlineto{\pgfqpoint{1.999076in}{2.966397in}}%
\pgfpathlineto{\pgfqpoint{1.987818in}{2.961477in}}%
\pgfpathlineto{\pgfqpoint{1.976541in}{2.957039in}}%
\pgfpathlineto{\pgfqpoint{1.965251in}{2.952966in}}%
\pgfpathlineto{\pgfqpoint{1.953948in}{2.949139in}}%
\pgfpathlineto{\pgfqpoint{1.959646in}{2.941527in}}%
\pgfpathlineto{\pgfqpoint{1.965349in}{2.933906in}}%
\pgfpathlineto{\pgfqpoint{1.971055in}{2.926273in}}%
\pgfpathlineto{\pgfqpoint{1.976766in}{2.918629in}}%
\pgfpathclose%
\pgfusepath{stroke,fill}%
\end{pgfscope}%
\begin{pgfscope}%
\pgfpathrectangle{\pgfqpoint{0.887500in}{0.275000in}}{\pgfqpoint{4.225000in}{4.225000in}}%
\pgfusepath{clip}%
\pgfsetbuttcap%
\pgfsetroundjoin%
\definecolor{currentfill}{rgb}{0.804182,0.882046,0.114965}%
\pgfsetfillcolor{currentfill}%
\pgfsetfillopacity{0.700000}%
\pgfsetlinewidth{0.501875pt}%
\definecolor{currentstroke}{rgb}{1.000000,1.000000,1.000000}%
\pgfsetstrokecolor{currentstroke}%
\pgfsetstrokeopacity{0.500000}%
\pgfsetdash{}{0pt}%
\pgfpathmoveto{\pgfqpoint{3.186767in}{3.613828in}}%
\pgfpathlineto{\pgfqpoint{3.197820in}{3.618669in}}%
\pgfpathlineto{\pgfqpoint{3.208869in}{3.623402in}}%
\pgfpathlineto{\pgfqpoint{3.219914in}{3.628075in}}%
\pgfpathlineto{\pgfqpoint{3.230954in}{3.632744in}}%
\pgfpathlineto{\pgfqpoint{3.241989in}{3.637470in}}%
\pgfpathlineto{\pgfqpoint{3.235837in}{3.640124in}}%
\pgfpathlineto{\pgfqpoint{3.229689in}{3.642515in}}%
\pgfpathlineto{\pgfqpoint{3.223546in}{3.644637in}}%
\pgfpathlineto{\pgfqpoint{3.217407in}{3.646502in}}%
\pgfpathlineto{\pgfqpoint{3.211273in}{3.648141in}}%
\pgfpathlineto{\pgfqpoint{3.200256in}{3.643168in}}%
\pgfpathlineto{\pgfqpoint{3.189234in}{3.638213in}}%
\pgfpathlineto{\pgfqpoint{3.178209in}{3.633208in}}%
\pgfpathlineto{\pgfqpoint{3.167178in}{3.628089in}}%
\pgfpathlineto{\pgfqpoint{3.156144in}{3.622803in}}%
\pgfpathlineto{\pgfqpoint{3.162258in}{3.621512in}}%
\pgfpathlineto{\pgfqpoint{3.168378in}{3.619992in}}%
\pgfpathlineto{\pgfqpoint{3.174502in}{3.618205in}}%
\pgfpathlineto{\pgfqpoint{3.180632in}{3.616146in}}%
\pgfpathclose%
\pgfusepath{stroke,fill}%
\end{pgfscope}%
\begin{pgfscope}%
\pgfpathrectangle{\pgfqpoint{0.887500in}{0.275000in}}{\pgfqpoint{4.225000in}{4.225000in}}%
\pgfusepath{clip}%
\pgfsetbuttcap%
\pgfsetroundjoin%
\definecolor{currentfill}{rgb}{0.134692,0.658636,0.517649}%
\pgfsetfillcolor{currentfill}%
\pgfsetfillopacity{0.700000}%
\pgfsetlinewidth{0.501875pt}%
\definecolor{currentstroke}{rgb}{1.000000,1.000000,1.000000}%
\pgfsetstrokecolor{currentstroke}%
\pgfsetstrokeopacity{0.500000}%
\pgfsetdash{}{0pt}%
\pgfpathmoveto{\pgfqpoint{2.264956in}{2.947635in}}%
\pgfpathlineto{\pgfqpoint{2.276071in}{2.958363in}}%
\pgfpathlineto{\pgfqpoint{2.287192in}{2.968742in}}%
\pgfpathlineto{\pgfqpoint{2.298317in}{2.978847in}}%
\pgfpathlineto{\pgfqpoint{2.309443in}{2.988760in}}%
\pgfpathlineto{\pgfqpoint{2.320570in}{2.998560in}}%
\pgfpathlineto{\pgfqpoint{2.314745in}{3.006243in}}%
\pgfpathlineto{\pgfqpoint{2.308925in}{3.013822in}}%
\pgfpathlineto{\pgfqpoint{2.303111in}{3.021321in}}%
\pgfpathlineto{\pgfqpoint{2.297301in}{3.028763in}}%
\pgfpathlineto{\pgfqpoint{2.291496in}{3.036170in}}%
\pgfpathlineto{\pgfqpoint{2.280365in}{3.027488in}}%
\pgfpathlineto{\pgfqpoint{2.269230in}{3.018811in}}%
\pgfpathlineto{\pgfqpoint{2.258095in}{3.010075in}}%
\pgfpathlineto{\pgfqpoint{2.246960in}{3.001215in}}%
\pgfpathlineto{\pgfqpoint{2.235827in}{2.992168in}}%
\pgfpathlineto{\pgfqpoint{2.241649in}{2.983120in}}%
\pgfpathlineto{\pgfqpoint{2.247473in}{2.974129in}}%
\pgfpathlineto{\pgfqpoint{2.253299in}{2.965208in}}%
\pgfpathlineto{\pgfqpoint{2.259127in}{2.956373in}}%
\pgfpathclose%
\pgfusepath{stroke,fill}%
\end{pgfscope}%
\begin{pgfscope}%
\pgfpathrectangle{\pgfqpoint{0.887500in}{0.275000in}}{\pgfqpoint{4.225000in}{4.225000in}}%
\pgfusepath{clip}%
\pgfsetbuttcap%
\pgfsetroundjoin%
\definecolor{currentfill}{rgb}{0.945636,0.899815,0.112838}%
\pgfsetfillcolor{currentfill}%
\pgfsetfillopacity{0.700000}%
\pgfsetlinewidth{0.501875pt}%
\definecolor{currentstroke}{rgb}{1.000000,1.000000,1.000000}%
\pgfsetstrokecolor{currentstroke}%
\pgfsetstrokeopacity{0.500000}%
\pgfsetdash{}{0pt}%
\pgfpathmoveto{\pgfqpoint{3.751533in}{3.702478in}}%
\pgfpathlineto{\pgfqpoint{3.762475in}{3.707281in}}%
\pgfpathlineto{\pgfqpoint{3.773413in}{3.712048in}}%
\pgfpathlineto{\pgfqpoint{3.784346in}{3.716784in}}%
\pgfpathlineto{\pgfqpoint{3.795274in}{3.721492in}}%
\pgfpathlineto{\pgfqpoint{3.806197in}{3.726174in}}%
\pgfpathlineto{\pgfqpoint{3.799894in}{3.734454in}}%
\pgfpathlineto{\pgfqpoint{3.793590in}{3.742511in}}%
\pgfpathlineto{\pgfqpoint{3.787287in}{3.750352in}}%
\pgfpathlineto{\pgfqpoint{3.780984in}{3.757985in}}%
\pgfpathlineto{\pgfqpoint{3.774682in}{3.765418in}}%
\pgfpathlineto{\pgfqpoint{3.763771in}{3.760810in}}%
\pgfpathlineto{\pgfqpoint{3.752856in}{3.756166in}}%
\pgfpathlineto{\pgfqpoint{3.741935in}{3.751485in}}%
\pgfpathlineto{\pgfqpoint{3.731009in}{3.746766in}}%
\pgfpathlineto{\pgfqpoint{3.720079in}{3.742007in}}%
\pgfpathlineto{\pgfqpoint{3.726367in}{3.734479in}}%
\pgfpathlineto{\pgfqpoint{3.732657in}{3.726762in}}%
\pgfpathlineto{\pgfqpoint{3.738948in}{3.718857in}}%
\pgfpathlineto{\pgfqpoint{3.745240in}{3.710763in}}%
\pgfpathclose%
\pgfusepath{stroke,fill}%
\end{pgfscope}%
\begin{pgfscope}%
\pgfpathrectangle{\pgfqpoint{0.887500in}{0.275000in}}{\pgfqpoint{4.225000in}{4.225000in}}%
\pgfusepath{clip}%
\pgfsetbuttcap%
\pgfsetroundjoin%
\definecolor{currentfill}{rgb}{0.709898,0.868751,0.169257}%
\pgfsetfillcolor{currentfill}%
\pgfsetfillopacity{0.700000}%
\pgfsetlinewidth{0.501875pt}%
\definecolor{currentstroke}{rgb}{1.000000,1.000000,1.000000}%
\pgfsetstrokecolor{currentstroke}%
\pgfsetstrokeopacity{0.500000}%
\pgfsetdash{}{0pt}%
\pgfpathmoveto{\pgfqpoint{2.990131in}{3.545015in}}%
\pgfpathlineto{\pgfqpoint{3.001230in}{3.548782in}}%
\pgfpathlineto{\pgfqpoint{3.012323in}{3.552815in}}%
\pgfpathlineto{\pgfqpoint{3.023411in}{3.557200in}}%
\pgfpathlineto{\pgfqpoint{3.034494in}{3.561930in}}%
\pgfpathlineto{\pgfqpoint{3.045573in}{3.566955in}}%
\pgfpathlineto{\pgfqpoint{3.039508in}{3.567564in}}%
\pgfpathlineto{\pgfqpoint{3.033448in}{3.568230in}}%
\pgfpathlineto{\pgfqpoint{3.027395in}{3.568948in}}%
\pgfpathlineto{\pgfqpoint{3.021348in}{3.569717in}}%
\pgfpathlineto{\pgfqpoint{3.015307in}{3.570535in}}%
\pgfpathlineto{\pgfqpoint{3.004250in}{3.565253in}}%
\pgfpathlineto{\pgfqpoint{2.993189in}{3.560307in}}%
\pgfpathlineto{\pgfqpoint{2.982122in}{3.555778in}}%
\pgfpathlineto{\pgfqpoint{2.971051in}{3.551642in}}%
\pgfpathlineto{\pgfqpoint{2.959974in}{3.547667in}}%
\pgfpathlineto{\pgfqpoint{2.965993in}{3.547118in}}%
\pgfpathlineto{\pgfqpoint{2.972019in}{3.546594in}}%
\pgfpathlineto{\pgfqpoint{2.978050in}{3.546079in}}%
\pgfpathlineto{\pgfqpoint{2.984088in}{3.545558in}}%
\pgfpathclose%
\pgfusepath{stroke,fill}%
\end{pgfscope}%
\begin{pgfscope}%
\pgfpathrectangle{\pgfqpoint{0.887500in}{0.275000in}}{\pgfqpoint{4.225000in}{4.225000in}}%
\pgfusepath{clip}%
\pgfsetbuttcap%
\pgfsetroundjoin%
\definecolor{currentfill}{rgb}{0.157851,0.683765,0.501686}%
\pgfsetfillcolor{currentfill}%
\pgfsetfillopacity{0.700000}%
\pgfsetlinewidth{0.501875pt}%
\definecolor{currentstroke}{rgb}{1.000000,1.000000,1.000000}%
\pgfsetstrokecolor{currentstroke}%
\pgfsetstrokeopacity{0.500000}%
\pgfsetdash{}{0pt}%
\pgfpathmoveto{\pgfqpoint{2.547062in}{3.003820in}}%
\pgfpathlineto{\pgfqpoint{2.558202in}{3.011623in}}%
\pgfpathlineto{\pgfqpoint{2.569319in}{3.021052in}}%
\pgfpathlineto{\pgfqpoint{2.580415in}{3.032181in}}%
\pgfpathlineto{\pgfqpoint{2.591489in}{3.045022in}}%
\pgfpathlineto{\pgfqpoint{2.602544in}{3.059589in}}%
\pgfpathlineto{\pgfqpoint{2.596636in}{3.066054in}}%
\pgfpathlineto{\pgfqpoint{2.590734in}{3.072430in}}%
\pgfpathlineto{\pgfqpoint{2.584837in}{3.078714in}}%
\pgfpathlineto{\pgfqpoint{2.578945in}{3.084904in}}%
\pgfpathlineto{\pgfqpoint{2.573059in}{3.090995in}}%
\pgfpathlineto{\pgfqpoint{2.561983in}{3.079886in}}%
\pgfpathlineto{\pgfqpoint{2.550887in}{3.070302in}}%
\pgfpathlineto{\pgfqpoint{2.539770in}{3.062251in}}%
\pgfpathlineto{\pgfqpoint{2.528629in}{3.055742in}}%
\pgfpathlineto{\pgfqpoint{2.517466in}{3.050702in}}%
\pgfpathlineto{\pgfqpoint{2.523382in}{3.041057in}}%
\pgfpathlineto{\pgfqpoint{2.529301in}{3.031407in}}%
\pgfpathlineto{\pgfqpoint{2.535222in}{3.021892in}}%
\pgfpathlineto{\pgfqpoint{2.541143in}{3.012650in}}%
\pgfpathclose%
\pgfusepath{stroke,fill}%
\end{pgfscope}%
\begin{pgfscope}%
\pgfpathrectangle{\pgfqpoint{0.887500in}{0.275000in}}{\pgfqpoint{4.225000in}{4.225000in}}%
\pgfusepath{clip}%
\pgfsetbuttcap%
\pgfsetroundjoin%
\definecolor{currentfill}{rgb}{0.855810,0.888601,0.097452}%
\pgfsetfillcolor{currentfill}%
\pgfsetfillopacity{0.700000}%
\pgfsetlinewidth{0.501875pt}%
\definecolor{currentstroke}{rgb}{1.000000,1.000000,1.000000}%
\pgfsetstrokecolor{currentstroke}%
\pgfsetstrokeopacity{0.500000}%
\pgfsetdash{}{0pt}%
\pgfpathmoveto{\pgfqpoint{3.328031in}{3.646079in}}%
\pgfpathlineto{\pgfqpoint{3.339065in}{3.651762in}}%
\pgfpathlineto{\pgfqpoint{3.350097in}{3.657551in}}%
\pgfpathlineto{\pgfqpoint{3.361125in}{3.663405in}}%
\pgfpathlineto{\pgfqpoint{3.372149in}{3.669284in}}%
\pgfpathlineto{\pgfqpoint{3.383170in}{3.675149in}}%
\pgfpathlineto{\pgfqpoint{3.376966in}{3.679549in}}%
\pgfpathlineto{\pgfqpoint{3.370764in}{3.683630in}}%
\pgfpathlineto{\pgfqpoint{3.364565in}{3.687404in}}%
\pgfpathlineto{\pgfqpoint{3.358369in}{3.690883in}}%
\pgfpathlineto{\pgfqpoint{3.352176in}{3.694082in}}%
\pgfpathlineto{\pgfqpoint{3.341172in}{3.687972in}}%
\pgfpathlineto{\pgfqpoint{3.330163in}{3.681840in}}%
\pgfpathlineto{\pgfqpoint{3.319152in}{3.675736in}}%
\pgfpathlineto{\pgfqpoint{3.308137in}{3.669707in}}%
\pgfpathlineto{\pgfqpoint{3.297119in}{3.663804in}}%
\pgfpathlineto{\pgfqpoint{3.303294in}{3.660807in}}%
\pgfpathlineto{\pgfqpoint{3.309473in}{3.657549in}}%
\pgfpathlineto{\pgfqpoint{3.315655in}{3.654018in}}%
\pgfpathlineto{\pgfqpoint{3.321842in}{3.650199in}}%
\pgfpathclose%
\pgfusepath{stroke,fill}%
\end{pgfscope}%
\begin{pgfscope}%
\pgfpathrectangle{\pgfqpoint{0.887500in}{0.275000in}}{\pgfqpoint{4.225000in}{4.225000in}}%
\pgfusepath{clip}%
\pgfsetbuttcap%
\pgfsetroundjoin%
\definecolor{currentfill}{rgb}{0.202219,0.715272,0.476084}%
\pgfsetfillcolor{currentfill}%
\pgfsetfillopacity{0.700000}%
\pgfsetlinewidth{0.501875pt}%
\definecolor{currentstroke}{rgb}{1.000000,1.000000,1.000000}%
\pgfsetstrokecolor{currentstroke}%
\pgfsetstrokeopacity{0.500000}%
\pgfsetdash{}{0pt}%
\pgfpathmoveto{\pgfqpoint{2.602544in}{3.059589in}}%
\pgfpathlineto{\pgfqpoint{2.613581in}{3.075895in}}%
\pgfpathlineto{\pgfqpoint{2.624601in}{3.093955in}}%
\pgfpathlineto{\pgfqpoint{2.635605in}{3.113781in}}%
\pgfpathlineto{\pgfqpoint{2.646596in}{3.135308in}}%
\pgfpathlineto{\pgfqpoint{2.657579in}{3.158198in}}%
\pgfpathlineto{\pgfqpoint{2.651667in}{3.162594in}}%
\pgfpathlineto{\pgfqpoint{2.645773in}{3.165587in}}%
\pgfpathlineto{\pgfqpoint{2.639898in}{3.167421in}}%
\pgfpathlineto{\pgfqpoint{2.634037in}{3.168338in}}%
\pgfpathlineto{\pgfqpoint{2.628189in}{3.168580in}}%
\pgfpathlineto{\pgfqpoint{2.617189in}{3.150430in}}%
\pgfpathlineto{\pgfqpoint{2.606179in}{3.133380in}}%
\pgfpathlineto{\pgfqpoint{2.595156in}{3.117750in}}%
\pgfpathlineto{\pgfqpoint{2.584116in}{3.103619in}}%
\pgfpathlineto{\pgfqpoint{2.573059in}{3.090995in}}%
\pgfpathlineto{\pgfqpoint{2.578945in}{3.084904in}}%
\pgfpathlineto{\pgfqpoint{2.584837in}{3.078714in}}%
\pgfpathlineto{\pgfqpoint{2.590734in}{3.072430in}}%
\pgfpathlineto{\pgfqpoint{2.596636in}{3.066054in}}%
\pgfpathclose%
\pgfusepath{stroke,fill}%
\end{pgfscope}%
\begin{pgfscope}%
\pgfpathrectangle{\pgfqpoint{0.887500in}{0.275000in}}{\pgfqpoint{4.225000in}{4.225000in}}%
\pgfusepath{clip}%
\pgfsetbuttcap%
\pgfsetroundjoin%
\definecolor{currentfill}{rgb}{0.123444,0.636809,0.528763}%
\pgfsetfillcolor{currentfill}%
\pgfsetfillopacity{0.700000}%
\pgfsetlinewidth{0.501875pt}%
\definecolor{currentstroke}{rgb}{1.000000,1.000000,1.000000}%
\pgfsetstrokecolor{currentstroke}%
\pgfsetstrokeopacity{0.500000}%
\pgfsetdash{}{0pt}%
\pgfpathmoveto{\pgfqpoint{2.209328in}{2.895652in}}%
\pgfpathlineto{\pgfqpoint{2.220480in}{2.904908in}}%
\pgfpathlineto{\pgfqpoint{2.231613in}{2.915012in}}%
\pgfpathlineto{\pgfqpoint{2.242733in}{2.925686in}}%
\pgfpathlineto{\pgfqpoint{2.253845in}{2.936653in}}%
\pgfpathlineto{\pgfqpoint{2.264956in}{2.947635in}}%
\pgfpathlineto{\pgfqpoint{2.259127in}{2.956373in}}%
\pgfpathlineto{\pgfqpoint{2.253299in}{2.965208in}}%
\pgfpathlineto{\pgfqpoint{2.247473in}{2.974129in}}%
\pgfpathlineto{\pgfqpoint{2.241649in}{2.983120in}}%
\pgfpathlineto{\pgfqpoint{2.235827in}{2.992168in}}%
\pgfpathlineto{\pgfqpoint{2.224694in}{2.982973in}}%
\pgfpathlineto{\pgfqpoint{2.213560in}{2.973769in}}%
\pgfpathlineto{\pgfqpoint{2.202421in}{2.964699in}}%
\pgfpathlineto{\pgfqpoint{2.191274in}{2.955908in}}%
\pgfpathlineto{\pgfqpoint{2.180115in}{2.947539in}}%
\pgfpathlineto{\pgfqpoint{2.185959in}{2.936809in}}%
\pgfpathlineto{\pgfqpoint{2.191804in}{2.926211in}}%
\pgfpathlineto{\pgfqpoint{2.197647in}{2.915789in}}%
\pgfpathlineto{\pgfqpoint{2.203489in}{2.905588in}}%
\pgfpathclose%
\pgfusepath{stroke,fill}%
\end{pgfscope}%
\begin{pgfscope}%
\pgfpathrectangle{\pgfqpoint{0.887500in}{0.275000in}}{\pgfqpoint{4.225000in}{4.225000in}}%
\pgfusepath{clip}%
\pgfsetbuttcap%
\pgfsetroundjoin%
\definecolor{currentfill}{rgb}{0.935904,0.898570,0.108131}%
\pgfsetfillcolor{currentfill}%
\pgfsetfillopacity{0.700000}%
\pgfsetlinewidth{0.501875pt}%
\definecolor{currentstroke}{rgb}{1.000000,1.000000,1.000000}%
\pgfsetstrokecolor{currentstroke}%
\pgfsetstrokeopacity{0.500000}%
\pgfsetdash{}{0pt}%
\pgfpathmoveto{\pgfqpoint{3.610508in}{3.692218in}}%
\pgfpathlineto{\pgfqpoint{3.621486in}{3.697349in}}%
\pgfpathlineto{\pgfqpoint{3.632460in}{3.702458in}}%
\pgfpathlineto{\pgfqpoint{3.643429in}{3.707539in}}%
\pgfpathlineto{\pgfqpoint{3.654393in}{3.712587in}}%
\pgfpathlineto{\pgfqpoint{3.665353in}{3.717597in}}%
\pgfpathlineto{\pgfqpoint{3.659078in}{3.725016in}}%
\pgfpathlineto{\pgfqpoint{3.652805in}{3.732244in}}%
\pgfpathlineto{\pgfqpoint{3.646534in}{3.739274in}}%
\pgfpathlineto{\pgfqpoint{3.640264in}{3.746100in}}%
\pgfpathlineto{\pgfqpoint{3.633995in}{3.752716in}}%
\pgfpathlineto{\pgfqpoint{3.623049in}{3.747771in}}%
\pgfpathlineto{\pgfqpoint{3.612098in}{3.742786in}}%
\pgfpathlineto{\pgfqpoint{3.601143in}{3.737765in}}%
\pgfpathlineto{\pgfqpoint{3.590183in}{3.732712in}}%
\pgfpathlineto{\pgfqpoint{3.579218in}{3.727629in}}%
\pgfpathlineto{\pgfqpoint{3.585473in}{3.720959in}}%
\pgfpathlineto{\pgfqpoint{3.591729in}{3.714073in}}%
\pgfpathlineto{\pgfqpoint{3.597987in}{3.706980in}}%
\pgfpathlineto{\pgfqpoint{3.604247in}{3.699691in}}%
\pgfpathclose%
\pgfusepath{stroke,fill}%
\end{pgfscope}%
\begin{pgfscope}%
\pgfpathrectangle{\pgfqpoint{0.887500in}{0.275000in}}{\pgfqpoint{4.225000in}{4.225000in}}%
\pgfusepath{clip}%
\pgfsetbuttcap%
\pgfsetroundjoin%
\definecolor{currentfill}{rgb}{0.120638,0.625828,0.533488}%
\pgfsetfillcolor{currentfill}%
\pgfsetfillopacity{0.700000}%
\pgfsetlinewidth{0.501875pt}%
\definecolor{currentstroke}{rgb}{1.000000,1.000000,1.000000}%
\pgfsetstrokecolor{currentstroke}%
\pgfsetstrokeopacity{0.500000}%
\pgfsetdash{}{0pt}%
\pgfpathmoveto{\pgfqpoint{2.067681in}{2.892528in}}%
\pgfpathlineto{\pgfqpoint{2.078973in}{2.896804in}}%
\pgfpathlineto{\pgfqpoint{2.090258in}{2.901149in}}%
\pgfpathlineto{\pgfqpoint{2.101537in}{2.905586in}}%
\pgfpathlineto{\pgfqpoint{2.112806in}{2.910211in}}%
\pgfpathlineto{\pgfqpoint{2.124064in}{2.915125in}}%
\pgfpathlineto{\pgfqpoint{2.118254in}{2.924940in}}%
\pgfpathlineto{\pgfqpoint{2.112447in}{2.934777in}}%
\pgfpathlineto{\pgfqpoint{2.106643in}{2.944604in}}%
\pgfpathlineto{\pgfqpoint{2.100843in}{2.954392in}}%
\pgfpathlineto{\pgfqpoint{2.095049in}{2.964109in}}%
\pgfpathlineto{\pgfqpoint{2.083855in}{2.956959in}}%
\pgfpathlineto{\pgfqpoint{2.072651in}{2.950144in}}%
\pgfpathlineto{\pgfqpoint{2.061432in}{2.943758in}}%
\pgfpathlineto{\pgfqpoint{2.050197in}{2.937884in}}%
\pgfpathlineto{\pgfqpoint{2.038945in}{2.932509in}}%
\pgfpathlineto{\pgfqpoint{2.044684in}{2.924543in}}%
\pgfpathlineto{\pgfqpoint{2.050428in}{2.916557in}}%
\pgfpathlineto{\pgfqpoint{2.056175in}{2.908557in}}%
\pgfpathlineto{\pgfqpoint{2.061926in}{2.900546in}}%
\pgfpathclose%
\pgfusepath{stroke,fill}%
\end{pgfscope}%
\begin{pgfscope}%
\pgfpathrectangle{\pgfqpoint{0.887500in}{0.275000in}}{\pgfqpoint{4.225000in}{4.225000in}}%
\pgfusepath{clip}%
\pgfsetbuttcap%
\pgfsetroundjoin%
\definecolor{currentfill}{rgb}{0.906311,0.894855,0.098125}%
\pgfsetfillcolor{currentfill}%
\pgfsetfillopacity{0.700000}%
\pgfsetlinewidth{0.501875pt}%
\definecolor{currentstroke}{rgb}{1.000000,1.000000,1.000000}%
\pgfsetstrokecolor{currentstroke}%
\pgfsetstrokeopacity{0.500000}%
\pgfsetdash{}{0pt}%
\pgfpathmoveto{\pgfqpoint{3.469331in}{3.675372in}}%
\pgfpathlineto{\pgfqpoint{3.480339in}{3.680739in}}%
\pgfpathlineto{\pgfqpoint{3.491344in}{3.686069in}}%
\pgfpathlineto{\pgfqpoint{3.502343in}{3.691364in}}%
\pgfpathlineto{\pgfqpoint{3.513338in}{3.696626in}}%
\pgfpathlineto{\pgfqpoint{3.524329in}{3.701857in}}%
\pgfpathlineto{\pgfqpoint{3.518090in}{3.708318in}}%
\pgfpathlineto{\pgfqpoint{3.511852in}{3.714500in}}%
\pgfpathlineto{\pgfqpoint{3.505615in}{3.720379in}}%
\pgfpathlineto{\pgfqpoint{3.499380in}{3.725948in}}%
\pgfpathlineto{\pgfqpoint{3.493146in}{3.731210in}}%
\pgfpathlineto{\pgfqpoint{3.482168in}{3.725833in}}%
\pgfpathlineto{\pgfqpoint{3.471186in}{3.720404in}}%
\pgfpathlineto{\pgfqpoint{3.460199in}{3.714923in}}%
\pgfpathlineto{\pgfqpoint{3.449208in}{3.709389in}}%
\pgfpathlineto{\pgfqpoint{3.438212in}{3.703804in}}%
\pgfpathlineto{\pgfqpoint{3.444433in}{3.698811in}}%
\pgfpathlineto{\pgfqpoint{3.450656in}{3.693472in}}%
\pgfpathlineto{\pgfqpoint{3.456879in}{3.687777in}}%
\pgfpathlineto{\pgfqpoint{3.463105in}{3.681731in}}%
\pgfpathclose%
\pgfusepath{stroke,fill}%
\end{pgfscope}%
\begin{pgfscope}%
\pgfpathrectangle{\pgfqpoint{0.887500in}{0.275000in}}{\pgfqpoint{4.225000in}{4.225000in}}%
\pgfusepath{clip}%
\pgfsetbuttcap%
\pgfsetroundjoin%
\definecolor{currentfill}{rgb}{0.122312,0.633153,0.530398}%
\pgfsetfillcolor{currentfill}%
\pgfsetfillopacity{0.700000}%
\pgfsetlinewidth{0.501875pt}%
\definecolor{currentstroke}{rgb}{1.000000,1.000000,1.000000}%
\pgfsetstrokecolor{currentstroke}%
\pgfsetstrokeopacity{0.500000}%
\pgfsetdash{}{0pt}%
\pgfpathmoveto{\pgfqpoint{1.840538in}{2.914279in}}%
\pgfpathlineto{\pgfqpoint{1.851903in}{2.917754in}}%
\pgfpathlineto{\pgfqpoint{1.863264in}{2.921214in}}%
\pgfpathlineto{\pgfqpoint{1.874619in}{2.924653in}}%
\pgfpathlineto{\pgfqpoint{1.885970in}{2.928068in}}%
\pgfpathlineto{\pgfqpoint{1.897316in}{2.931476in}}%
\pgfpathlineto{\pgfqpoint{1.891636in}{2.939041in}}%
\pgfpathlineto{\pgfqpoint{1.885959in}{2.946594in}}%
\pgfpathlineto{\pgfqpoint{1.880287in}{2.954134in}}%
\pgfpathlineto{\pgfqpoint{1.874618in}{2.961661in}}%
\pgfpathlineto{\pgfqpoint{1.863284in}{2.958216in}}%
\pgfpathlineto{\pgfqpoint{1.851944in}{2.954760in}}%
\pgfpathlineto{\pgfqpoint{1.840600in}{2.951278in}}%
\pgfpathlineto{\pgfqpoint{1.829251in}{2.947774in}}%
\pgfpathlineto{\pgfqpoint{1.817898in}{2.944255in}}%
\pgfpathlineto{\pgfqpoint{1.823552in}{2.936785in}}%
\pgfpathlineto{\pgfqpoint{1.829210in}{2.929298in}}%
\pgfpathlineto{\pgfqpoint{1.834872in}{2.921796in}}%
\pgfpathclose%
\pgfusepath{stroke,fill}%
\end{pgfscope}%
\begin{pgfscope}%
\pgfpathrectangle{\pgfqpoint{0.887500in}{0.275000in}}{\pgfqpoint{4.225000in}{4.225000in}}%
\pgfusepath{clip}%
\pgfsetbuttcap%
\pgfsetroundjoin%
\definecolor{currentfill}{rgb}{0.157851,0.683765,0.501686}%
\pgfsetfillcolor{currentfill}%
\pgfsetfillopacity{0.700000}%
\pgfsetlinewidth{0.501875pt}%
\definecolor{currentstroke}{rgb}{1.000000,1.000000,1.000000}%
\pgfsetstrokecolor{currentstroke}%
\pgfsetstrokeopacity{0.500000}%
\pgfsetdash{}{0pt}%
\pgfpathmoveto{\pgfqpoint{2.405545in}{3.003140in}}%
\pgfpathlineto{\pgfqpoint{2.416706in}{3.011081in}}%
\pgfpathlineto{\pgfqpoint{2.427876in}{3.018326in}}%
\pgfpathlineto{\pgfqpoint{2.439056in}{3.024721in}}%
\pgfpathlineto{\pgfqpoint{2.450247in}{3.030112in}}%
\pgfpathlineto{\pgfqpoint{2.461451in}{3.034392in}}%
\pgfpathlineto{\pgfqpoint{2.455526in}{3.045441in}}%
\pgfpathlineto{\pgfqpoint{2.449612in}{3.055985in}}%
\pgfpathlineto{\pgfqpoint{2.443711in}{3.065821in}}%
\pgfpathlineto{\pgfqpoint{2.437827in}{3.074790in}}%
\pgfpathlineto{\pgfqpoint{2.431960in}{3.082942in}}%
\pgfpathlineto{\pgfqpoint{2.420769in}{3.078452in}}%
\pgfpathlineto{\pgfqpoint{2.409598in}{3.072455in}}%
\pgfpathlineto{\pgfqpoint{2.398445in}{3.065154in}}%
\pgfpathlineto{\pgfqpoint{2.387306in}{3.056808in}}%
\pgfpathlineto{\pgfqpoint{2.376178in}{3.047676in}}%
\pgfpathlineto{\pgfqpoint{2.382026in}{3.039866in}}%
\pgfpathlineto{\pgfqpoint{2.387887in}{3.031493in}}%
\pgfpathlineto{\pgfqpoint{2.393762in}{3.022517in}}%
\pgfpathlineto{\pgfqpoint{2.399648in}{3.013027in}}%
\pgfpathclose%
\pgfusepath{stroke,fill}%
\end{pgfscope}%
\begin{pgfscope}%
\pgfpathrectangle{\pgfqpoint{0.887500in}{0.275000in}}{\pgfqpoint{4.225000in}{4.225000in}}%
\pgfusepath{clip}%
\pgfsetbuttcap%
\pgfsetroundjoin%
\definecolor{currentfill}{rgb}{0.783315,0.879285,0.125405}%
\pgfsetfillcolor{currentfill}%
\pgfsetfillopacity{0.700000}%
\pgfsetlinewidth{0.501875pt}%
\definecolor{currentstroke}{rgb}{1.000000,1.000000,1.000000}%
\pgfsetstrokecolor{currentstroke}%
\pgfsetstrokeopacity{0.500000}%
\pgfsetdash{}{0pt}%
\pgfpathmoveto{\pgfqpoint{3.131432in}{3.588810in}}%
\pgfpathlineto{\pgfqpoint{3.142508in}{3.593824in}}%
\pgfpathlineto{\pgfqpoint{3.153579in}{3.598862in}}%
\pgfpathlineto{\pgfqpoint{3.164646in}{3.603895in}}%
\pgfpathlineto{\pgfqpoint{3.175708in}{3.608893in}}%
\pgfpathlineto{\pgfqpoint{3.186767in}{3.613828in}}%
\pgfpathlineto{\pgfqpoint{3.180632in}{3.616146in}}%
\pgfpathlineto{\pgfqpoint{3.174502in}{3.618205in}}%
\pgfpathlineto{\pgfqpoint{3.168378in}{3.619992in}}%
\pgfpathlineto{\pgfqpoint{3.162258in}{3.621512in}}%
\pgfpathlineto{\pgfqpoint{3.156144in}{3.622803in}}%
\pgfpathlineto{\pgfqpoint{3.145104in}{3.617364in}}%
\pgfpathlineto{\pgfqpoint{3.134061in}{3.611803in}}%
\pgfpathlineto{\pgfqpoint{3.123014in}{3.606153in}}%
\pgfpathlineto{\pgfqpoint{3.111962in}{3.600444in}}%
\pgfpathlineto{\pgfqpoint{3.100907in}{3.594708in}}%
\pgfpathlineto{\pgfqpoint{3.107000in}{3.593697in}}%
\pgfpathlineto{\pgfqpoint{3.113100in}{3.592618in}}%
\pgfpathlineto{\pgfqpoint{3.119205in}{3.591456in}}%
\pgfpathlineto{\pgfqpoint{3.125316in}{3.590194in}}%
\pgfpathclose%
\pgfusepath{stroke,fill}%
\end{pgfscope}%
\begin{pgfscope}%
\pgfpathrectangle{\pgfqpoint{0.887500in}{0.275000in}}{\pgfqpoint{4.225000in}{4.225000in}}%
\pgfusepath{clip}%
\pgfsetbuttcap%
\pgfsetroundjoin%
\definecolor{currentfill}{rgb}{0.926106,0.897330,0.104071}%
\pgfsetfillcolor{currentfill}%
\pgfsetfillopacity{0.700000}%
\pgfsetlinewidth{0.501875pt}%
\definecolor{currentstroke}{rgb}{1.000000,1.000000,1.000000}%
\pgfsetstrokecolor{currentstroke}%
\pgfsetstrokeopacity{0.500000}%
\pgfsetdash{}{0pt}%
\pgfpathmoveto{\pgfqpoint{3.837708in}{3.681295in}}%
\pgfpathlineto{\pgfqpoint{3.848636in}{3.685922in}}%
\pgfpathlineto{\pgfqpoint{3.859559in}{3.690552in}}%
\pgfpathlineto{\pgfqpoint{3.870478in}{3.695191in}}%
\pgfpathlineto{\pgfqpoint{3.881393in}{3.699849in}}%
\pgfpathlineto{\pgfqpoint{3.875086in}{3.709338in}}%
\pgfpathlineto{\pgfqpoint{3.868777in}{3.718565in}}%
\pgfpathlineto{\pgfqpoint{3.862467in}{3.727532in}}%
\pgfpathlineto{\pgfqpoint{3.856155in}{3.736246in}}%
\pgfpathlineto{\pgfqpoint{3.849843in}{3.744710in}}%
\pgfpathlineto{\pgfqpoint{3.838939in}{3.740099in}}%
\pgfpathlineto{\pgfqpoint{3.828029in}{3.735474in}}%
\pgfpathlineto{\pgfqpoint{3.817116in}{3.730834in}}%
\pgfpathlineto{\pgfqpoint{3.806197in}{3.726174in}}%
\pgfpathlineto{\pgfqpoint{3.812500in}{3.717668in}}%
\pgfpathlineto{\pgfqpoint{3.818803in}{3.708931in}}%
\pgfpathlineto{\pgfqpoint{3.825106in}{3.699959in}}%
\pgfpathlineto{\pgfqpoint{3.831407in}{3.690749in}}%
\pgfpathclose%
\pgfusepath{stroke,fill}%
\end{pgfscope}%
\begin{pgfscope}%
\pgfpathrectangle{\pgfqpoint{0.887500in}{0.275000in}}{\pgfqpoint{4.225000in}{4.225000in}}%
\pgfusepath{clip}%
\pgfsetbuttcap%
\pgfsetroundjoin%
\definecolor{currentfill}{rgb}{0.121380,0.629492,0.531973}%
\pgfsetfillcolor{currentfill}%
\pgfsetfillopacity{0.700000}%
\pgfsetlinewidth{0.501875pt}%
\definecolor{currentstroke}{rgb}{1.000000,1.000000,1.000000}%
\pgfsetstrokecolor{currentstroke}%
\pgfsetstrokeopacity{0.500000}%
\pgfsetdash{}{0pt}%
\pgfpathmoveto{\pgfqpoint{1.925780in}{2.893462in}}%
\pgfpathlineto{\pgfqpoint{1.937134in}{2.896845in}}%
\pgfpathlineto{\pgfqpoint{1.948481in}{2.900265in}}%
\pgfpathlineto{\pgfqpoint{1.959822in}{2.903746in}}%
\pgfpathlineto{\pgfqpoint{1.971155in}{2.907308in}}%
\pgfpathlineto{\pgfqpoint{1.982480in}{2.910974in}}%
\pgfpathlineto{\pgfqpoint{1.976766in}{2.918629in}}%
\pgfpathlineto{\pgfqpoint{1.971055in}{2.926273in}}%
\pgfpathlineto{\pgfqpoint{1.965349in}{2.933906in}}%
\pgfpathlineto{\pgfqpoint{1.959646in}{2.941527in}}%
\pgfpathlineto{\pgfqpoint{1.953948in}{2.949139in}}%
\pgfpathlineto{\pgfqpoint{1.942637in}{2.945452in}}%
\pgfpathlineto{\pgfqpoint{1.931317in}{2.941864in}}%
\pgfpathlineto{\pgfqpoint{1.919990in}{2.938353in}}%
\pgfpathlineto{\pgfqpoint{1.908656in}{2.934897in}}%
\pgfpathlineto{\pgfqpoint{1.897316in}{2.931476in}}%
\pgfpathlineto{\pgfqpoint{1.903001in}{2.923898in}}%
\pgfpathlineto{\pgfqpoint{1.908689in}{2.916308in}}%
\pgfpathlineto{\pgfqpoint{1.914382in}{2.908705in}}%
\pgfpathlineto{\pgfqpoint{1.920079in}{2.901089in}}%
\pgfpathclose%
\pgfusepath{stroke,fill}%
\end{pgfscope}%
\begin{pgfscope}%
\pgfpathrectangle{\pgfqpoint{0.887500in}{0.275000in}}{\pgfqpoint{4.225000in}{4.225000in}}%
\pgfusepath{clip}%
\pgfsetbuttcap%
\pgfsetroundjoin%
\definecolor{currentfill}{rgb}{0.120081,0.622161,0.534946}%
\pgfsetfillcolor{currentfill}%
\pgfsetfillopacity{0.700000}%
\pgfsetlinewidth{0.501875pt}%
\definecolor{currentstroke}{rgb}{1.000000,1.000000,1.000000}%
\pgfsetstrokecolor{currentstroke}%
\pgfsetstrokeopacity{0.500000}%
\pgfsetdash{}{0pt}%
\pgfpathmoveto{\pgfqpoint{2.153128in}{2.867450in}}%
\pgfpathlineto{\pgfqpoint{2.164423in}{2.871002in}}%
\pgfpathlineto{\pgfqpoint{2.175695in}{2.875351in}}%
\pgfpathlineto{\pgfqpoint{2.186939in}{2.880768in}}%
\pgfpathlineto{\pgfqpoint{2.198150in}{2.887520in}}%
\pgfpathlineto{\pgfqpoint{2.209328in}{2.895652in}}%
\pgfpathlineto{\pgfqpoint{2.203489in}{2.905588in}}%
\pgfpathlineto{\pgfqpoint{2.197647in}{2.915789in}}%
\pgfpathlineto{\pgfqpoint{2.191804in}{2.926211in}}%
\pgfpathlineto{\pgfqpoint{2.185959in}{2.936809in}}%
\pgfpathlineto{\pgfqpoint{2.180115in}{2.947539in}}%
\pgfpathlineto{\pgfqpoint{2.168941in}{2.939738in}}%
\pgfpathlineto{\pgfqpoint{2.157749in}{2.932637in}}%
\pgfpathlineto{\pgfqpoint{2.146537in}{2.926235in}}%
\pgfpathlineto{\pgfqpoint{2.135308in}{2.920432in}}%
\pgfpathlineto{\pgfqpoint{2.124064in}{2.915125in}}%
\pgfpathlineto{\pgfqpoint{2.129876in}{2.905362in}}%
\pgfpathlineto{\pgfqpoint{2.135689in}{2.895682in}}%
\pgfpathlineto{\pgfqpoint{2.141502in}{2.886117in}}%
\pgfpathlineto{\pgfqpoint{2.147316in}{2.876695in}}%
\pgfpathclose%
\pgfusepath{stroke,fill}%
\end{pgfscope}%
\begin{pgfscope}%
\pgfpathrectangle{\pgfqpoint{0.887500in}{0.275000in}}{\pgfqpoint{4.225000in}{4.225000in}}%
\pgfusepath{clip}%
\pgfsetbuttcap%
\pgfsetroundjoin%
\definecolor{currentfill}{rgb}{0.146616,0.673050,0.508936}%
\pgfsetfillcolor{currentfill}%
\pgfsetfillopacity{0.700000}%
\pgfsetlinewidth{0.501875pt}%
\definecolor{currentstroke}{rgb}{1.000000,1.000000,1.000000}%
\pgfsetstrokecolor{currentstroke}%
\pgfsetstrokeopacity{0.500000}%
\pgfsetdash{}{0pt}%
\pgfpathmoveto{\pgfqpoint{2.491123in}{2.978718in}}%
\pgfpathlineto{\pgfqpoint{2.502331in}{2.982938in}}%
\pgfpathlineto{\pgfqpoint{2.513532in}{2.987242in}}%
\pgfpathlineto{\pgfqpoint{2.524724in}{2.991946in}}%
\pgfpathlineto{\pgfqpoint{2.535901in}{2.997366in}}%
\pgfpathlineto{\pgfqpoint{2.547062in}{3.003820in}}%
\pgfpathlineto{\pgfqpoint{2.541143in}{3.012650in}}%
\pgfpathlineto{\pgfqpoint{2.535222in}{3.021892in}}%
\pgfpathlineto{\pgfqpoint{2.529301in}{3.031407in}}%
\pgfpathlineto{\pgfqpoint{2.523382in}{3.041057in}}%
\pgfpathlineto{\pgfqpoint{2.517466in}{3.050702in}}%
\pgfpathlineto{\pgfqpoint{2.506282in}{3.046787in}}%
\pgfpathlineto{\pgfqpoint{2.495084in}{3.043598in}}%
\pgfpathlineto{\pgfqpoint{2.483875in}{3.040735in}}%
\pgfpathlineto{\pgfqpoint{2.472662in}{3.037800in}}%
\pgfpathlineto{\pgfqpoint{2.461451in}{3.034392in}}%
\pgfpathlineto{\pgfqpoint{2.467383in}{3.023043in}}%
\pgfpathlineto{\pgfqpoint{2.473319in}{3.011597in}}%
\pgfpathlineto{\pgfqpoint{2.479256in}{3.000258in}}%
\pgfpathlineto{\pgfqpoint{2.485192in}{2.989230in}}%
\pgfpathclose%
\pgfusepath{stroke,fill}%
\end{pgfscope}%
\begin{pgfscope}%
\pgfpathrectangle{\pgfqpoint{0.887500in}{0.275000in}}{\pgfqpoint{4.225000in}{4.225000in}}%
\pgfusepath{clip}%
\pgfsetbuttcap%
\pgfsetroundjoin%
\definecolor{currentfill}{rgb}{0.835270,0.886029,0.102646}%
\pgfsetfillcolor{currentfill}%
\pgfsetfillopacity{0.700000}%
\pgfsetlinewidth{0.501875pt}%
\definecolor{currentstroke}{rgb}{1.000000,1.000000,1.000000}%
\pgfsetstrokecolor{currentstroke}%
\pgfsetstrokeopacity{0.500000}%
\pgfsetdash{}{0pt}%
\pgfpathmoveto{\pgfqpoint{3.272815in}{3.620426in}}%
\pgfpathlineto{\pgfqpoint{3.283864in}{3.625149in}}%
\pgfpathlineto{\pgfqpoint{3.294910in}{3.630064in}}%
\pgfpathlineto{\pgfqpoint{3.305953in}{3.635193in}}%
\pgfpathlineto{\pgfqpoint{3.316993in}{3.640543in}}%
\pgfpathlineto{\pgfqpoint{3.328031in}{3.646079in}}%
\pgfpathlineto{\pgfqpoint{3.321842in}{3.650199in}}%
\pgfpathlineto{\pgfqpoint{3.315655in}{3.654018in}}%
\pgfpathlineto{\pgfqpoint{3.309473in}{3.657549in}}%
\pgfpathlineto{\pgfqpoint{3.303294in}{3.660807in}}%
\pgfpathlineto{\pgfqpoint{3.297119in}{3.663804in}}%
\pgfpathlineto{\pgfqpoint{3.286099in}{3.658074in}}%
\pgfpathlineto{\pgfqpoint{3.275076in}{3.652568in}}%
\pgfpathlineto{\pgfqpoint{3.264050in}{3.647323in}}%
\pgfpathlineto{\pgfqpoint{3.253021in}{3.642310in}}%
\pgfpathlineto{\pgfqpoint{3.241989in}{3.637470in}}%
\pgfpathlineto{\pgfqpoint{3.248146in}{3.634557in}}%
\pgfpathlineto{\pgfqpoint{3.254307in}{3.631390in}}%
\pgfpathlineto{\pgfqpoint{3.260472in}{3.627976in}}%
\pgfpathlineto{\pgfqpoint{3.266642in}{3.624320in}}%
\pgfpathclose%
\pgfusepath{stroke,fill}%
\end{pgfscope}%
\begin{pgfscope}%
\pgfpathrectangle{\pgfqpoint{0.887500in}{0.275000in}}{\pgfqpoint{4.225000in}{4.225000in}}%
\pgfusepath{clip}%
\pgfsetbuttcap%
\pgfsetroundjoin%
\definecolor{currentfill}{rgb}{0.926106,0.897330,0.104071}%
\pgfsetfillcolor{currentfill}%
\pgfsetfillopacity{0.700000}%
\pgfsetlinewidth{0.501875pt}%
\definecolor{currentstroke}{rgb}{1.000000,1.000000,1.000000}%
\pgfsetstrokecolor{currentstroke}%
\pgfsetstrokeopacity{0.500000}%
\pgfsetdash{}{0pt}%
\pgfpathmoveto{\pgfqpoint{3.696747in}{3.677843in}}%
\pgfpathlineto{\pgfqpoint{3.707714in}{3.682863in}}%
\pgfpathlineto{\pgfqpoint{3.718676in}{3.687834in}}%
\pgfpathlineto{\pgfqpoint{3.729633in}{3.692758in}}%
\pgfpathlineto{\pgfqpoint{3.740586in}{3.697639in}}%
\pgfpathlineto{\pgfqpoint{3.751533in}{3.702478in}}%
\pgfpathlineto{\pgfqpoint{3.745240in}{3.710763in}}%
\pgfpathlineto{\pgfqpoint{3.738948in}{3.718857in}}%
\pgfpathlineto{\pgfqpoint{3.732657in}{3.726762in}}%
\pgfpathlineto{\pgfqpoint{3.726367in}{3.734479in}}%
\pgfpathlineto{\pgfqpoint{3.720079in}{3.742007in}}%
\pgfpathlineto{\pgfqpoint{3.709143in}{3.737209in}}%
\pgfpathlineto{\pgfqpoint{3.698203in}{3.732370in}}%
\pgfpathlineto{\pgfqpoint{3.687258in}{3.727488in}}%
\pgfpathlineto{\pgfqpoint{3.676308in}{3.722564in}}%
\pgfpathlineto{\pgfqpoint{3.665353in}{3.717597in}}%
\pgfpathlineto{\pgfqpoint{3.671629in}{3.709993in}}%
\pgfpathlineto{\pgfqpoint{3.677906in}{3.702210in}}%
\pgfpathlineto{\pgfqpoint{3.684185in}{3.694255in}}%
\pgfpathlineto{\pgfqpoint{3.690465in}{3.686133in}}%
\pgfpathclose%
\pgfusepath{stroke,fill}%
\end{pgfscope}%
\begin{pgfscope}%
\pgfpathrectangle{\pgfqpoint{0.887500in}{0.275000in}}{\pgfqpoint{4.225000in}{4.225000in}}%
\pgfusepath{clip}%
\pgfsetbuttcap%
\pgfsetroundjoin%
\definecolor{currentfill}{rgb}{0.688944,0.865448,0.182725}%
\pgfsetfillcolor{currentfill}%
\pgfsetfillopacity{0.700000}%
\pgfsetlinewidth{0.501875pt}%
\definecolor{currentstroke}{rgb}{1.000000,1.000000,1.000000}%
\pgfsetstrokecolor{currentstroke}%
\pgfsetstrokeopacity{0.500000}%
\pgfsetdash{}{0pt}%
\pgfpathmoveto{\pgfqpoint{2.934563in}{3.526813in}}%
\pgfpathlineto{\pgfqpoint{2.945686in}{3.530689in}}%
\pgfpathlineto{\pgfqpoint{2.956805in}{3.534352in}}%
\pgfpathlineto{\pgfqpoint{2.967919in}{3.537896in}}%
\pgfpathlineto{\pgfqpoint{2.979028in}{3.541419in}}%
\pgfpathlineto{\pgfqpoint{2.990131in}{3.545015in}}%
\pgfpathlineto{\pgfqpoint{2.984088in}{3.545558in}}%
\pgfpathlineto{\pgfqpoint{2.978050in}{3.546079in}}%
\pgfpathlineto{\pgfqpoint{2.972019in}{3.546594in}}%
\pgfpathlineto{\pgfqpoint{2.965993in}{3.547118in}}%
\pgfpathlineto{\pgfqpoint{2.959974in}{3.547667in}}%
\pgfpathlineto{\pgfqpoint{2.948893in}{3.543596in}}%
\pgfpathlineto{\pgfqpoint{2.937808in}{3.539171in}}%
\pgfpathlineto{\pgfqpoint{2.926720in}{3.534135in}}%
\pgfpathlineto{\pgfqpoint{2.915632in}{3.528233in}}%
\pgfpathlineto{\pgfqpoint{2.904543in}{3.521209in}}%
\pgfpathlineto{\pgfqpoint{2.910534in}{3.522493in}}%
\pgfpathlineto{\pgfqpoint{2.916530in}{3.523784in}}%
\pgfpathlineto{\pgfqpoint{2.922534in}{3.524992in}}%
\pgfpathlineto{\pgfqpoint{2.928545in}{3.526031in}}%
\pgfpathclose%
\pgfusepath{stroke,fill}%
\end{pgfscope}%
\begin{pgfscope}%
\pgfpathrectangle{\pgfqpoint{0.887500in}{0.275000in}}{\pgfqpoint{4.225000in}{4.225000in}}%
\pgfusepath{clip}%
\pgfsetbuttcap%
\pgfsetroundjoin%
\definecolor{currentfill}{rgb}{0.143303,0.669459,0.511215}%
\pgfsetfillcolor{currentfill}%
\pgfsetfillopacity{0.700000}%
\pgfsetlinewidth{0.501875pt}%
\definecolor{currentstroke}{rgb}{1.000000,1.000000,1.000000}%
\pgfsetstrokecolor{currentstroke}%
\pgfsetstrokeopacity{0.500000}%
\pgfsetdash{}{0pt}%
\pgfpathmoveto{\pgfqpoint{2.349793in}{2.957948in}}%
\pgfpathlineto{\pgfqpoint{2.360940in}{2.967417in}}%
\pgfpathlineto{\pgfqpoint{2.372089in}{2.976682in}}%
\pgfpathlineto{\pgfqpoint{2.383238in}{2.985787in}}%
\pgfpathlineto{\pgfqpoint{2.394389in}{2.994657in}}%
\pgfpathlineto{\pgfqpoint{2.405545in}{3.003140in}}%
\pgfpathlineto{\pgfqpoint{2.399648in}{3.013027in}}%
\pgfpathlineto{\pgfqpoint{2.393762in}{3.022517in}}%
\pgfpathlineto{\pgfqpoint{2.387887in}{3.031493in}}%
\pgfpathlineto{\pgfqpoint{2.382026in}{3.039866in}}%
\pgfpathlineto{\pgfqpoint{2.376178in}{3.047676in}}%
\pgfpathlineto{\pgfqpoint{2.365057in}{3.038017in}}%
\pgfpathlineto{\pgfqpoint{2.353939in}{3.028090in}}%
\pgfpathlineto{\pgfqpoint{2.342819in}{3.018152in}}%
\pgfpathlineto{\pgfqpoint{2.331696in}{3.008330in}}%
\pgfpathlineto{\pgfqpoint{2.320570in}{2.998560in}}%
\pgfpathlineto{\pgfqpoint{2.326402in}{2.990751in}}%
\pgfpathlineto{\pgfqpoint{2.332239in}{2.982793in}}%
\pgfpathlineto{\pgfqpoint{2.338084in}{2.974670in}}%
\pgfpathlineto{\pgfqpoint{2.343935in}{2.966385in}}%
\pgfpathclose%
\pgfusepath{stroke,fill}%
\end{pgfscope}%
\begin{pgfscope}%
\pgfpathrectangle{\pgfqpoint{0.887500in}{0.275000in}}{\pgfqpoint{4.225000in}{4.225000in}}%
\pgfusepath{clip}%
\pgfsetbuttcap%
\pgfsetroundjoin%
\definecolor{currentfill}{rgb}{0.304148,0.764704,0.419943}%
\pgfsetfillcolor{currentfill}%
\pgfsetfillopacity{0.700000}%
\pgfsetlinewidth{0.501875pt}%
\definecolor{currentstroke}{rgb}{1.000000,1.000000,1.000000}%
\pgfsetstrokecolor{currentstroke}%
\pgfsetstrokeopacity{0.500000}%
\pgfsetdash{}{0pt}%
\pgfpathmoveto{\pgfqpoint{2.657579in}{3.158198in}}%
\pgfpathlineto{\pgfqpoint{2.668557in}{3.182059in}}%
\pgfpathlineto{\pgfqpoint{2.679537in}{3.206497in}}%
\pgfpathlineto{\pgfqpoint{2.690521in}{3.231118in}}%
\pgfpathlineto{\pgfqpoint{2.701513in}{3.255525in}}%
\pgfpathlineto{\pgfqpoint{2.712518in}{3.279322in}}%
\pgfpathlineto{\pgfqpoint{2.706604in}{3.280217in}}%
\pgfpathlineto{\pgfqpoint{2.700719in}{3.278550in}}%
\pgfpathlineto{\pgfqpoint{2.694860in}{3.274805in}}%
\pgfpathlineto{\pgfqpoint{2.689022in}{3.269460in}}%
\pgfpathlineto{\pgfqpoint{2.683202in}{3.262998in}}%
\pgfpathlineto{\pgfqpoint{2.672186in}{3.244853in}}%
\pgfpathlineto{\pgfqpoint{2.661180in}{3.225970in}}%
\pgfpathlineto{\pgfqpoint{2.650180in}{3.206718in}}%
\pgfpathlineto{\pgfqpoint{2.639185in}{3.187465in}}%
\pgfpathlineto{\pgfqpoint{2.628189in}{3.168580in}}%
\pgfpathlineto{\pgfqpoint{2.634037in}{3.168338in}}%
\pgfpathlineto{\pgfqpoint{2.639898in}{3.167421in}}%
\pgfpathlineto{\pgfqpoint{2.645773in}{3.165587in}}%
\pgfpathlineto{\pgfqpoint{2.651667in}{3.162594in}}%
\pgfpathclose%
\pgfusepath{stroke,fill}%
\end{pgfscope}%
\begin{pgfscope}%
\pgfpathrectangle{\pgfqpoint{0.887500in}{0.275000in}}{\pgfqpoint{4.225000in}{4.225000in}}%
\pgfusepath{clip}%
\pgfsetbuttcap%
\pgfsetroundjoin%
\definecolor{currentfill}{rgb}{0.916242,0.896091,0.100717}%
\pgfsetfillcolor{currentfill}%
\pgfsetfillopacity{0.700000}%
\pgfsetlinewidth{0.501875pt}%
\definecolor{currentstroke}{rgb}{1.000000,1.000000,1.000000}%
\pgfsetstrokecolor{currentstroke}%
\pgfsetstrokeopacity{0.500000}%
\pgfsetdash{}{0pt}%
\pgfpathmoveto{\pgfqpoint{3.555555in}{3.666393in}}%
\pgfpathlineto{\pgfqpoint{3.566554in}{3.671568in}}%
\pgfpathlineto{\pgfqpoint{3.577549in}{3.676741in}}%
\pgfpathlineto{\pgfqpoint{3.588540in}{3.681910in}}%
\pgfpathlineto{\pgfqpoint{3.599526in}{3.687070in}}%
\pgfpathlineto{\pgfqpoint{3.610508in}{3.692218in}}%
\pgfpathlineto{\pgfqpoint{3.604247in}{3.699691in}}%
\pgfpathlineto{\pgfqpoint{3.597987in}{3.706980in}}%
\pgfpathlineto{\pgfqpoint{3.591729in}{3.714073in}}%
\pgfpathlineto{\pgfqpoint{3.585473in}{3.720959in}}%
\pgfpathlineto{\pgfqpoint{3.579218in}{3.727629in}}%
\pgfpathlineto{\pgfqpoint{3.568249in}{3.722520in}}%
\pgfpathlineto{\pgfqpoint{3.557276in}{3.717389in}}%
\pgfpathlineto{\pgfqpoint{3.546298in}{3.712236in}}%
\pgfpathlineto{\pgfqpoint{3.535316in}{3.707059in}}%
\pgfpathlineto{\pgfqpoint{3.524329in}{3.701857in}}%
\pgfpathlineto{\pgfqpoint{3.530570in}{3.695147in}}%
\pgfpathlineto{\pgfqpoint{3.536813in}{3.688217in}}%
\pgfpathlineto{\pgfqpoint{3.543058in}{3.681095in}}%
\pgfpathlineto{\pgfqpoint{3.549305in}{3.673811in}}%
\pgfpathclose%
\pgfusepath{stroke,fill}%
\end{pgfscope}%
\begin{pgfscope}%
\pgfpathrectangle{\pgfqpoint{0.887500in}{0.275000in}}{\pgfqpoint{4.225000in}{4.225000in}}%
\pgfusepath{clip}%
\pgfsetbuttcap%
\pgfsetroundjoin%
\definecolor{currentfill}{rgb}{0.886271,0.892374,0.095374}%
\pgfsetfillcolor{currentfill}%
\pgfsetfillopacity{0.700000}%
\pgfsetlinewidth{0.501875pt}%
\definecolor{currentstroke}{rgb}{1.000000,1.000000,1.000000}%
\pgfsetstrokecolor{currentstroke}%
\pgfsetstrokeopacity{0.500000}%
\pgfsetdash{}{0pt}%
\pgfpathmoveto{\pgfqpoint{3.414224in}{3.648008in}}%
\pgfpathlineto{\pgfqpoint{3.425254in}{3.653550in}}%
\pgfpathlineto{\pgfqpoint{3.436280in}{3.659059in}}%
\pgfpathlineto{\pgfqpoint{3.447301in}{3.664532in}}%
\pgfpathlineto{\pgfqpoint{3.458318in}{3.669970in}}%
\pgfpathlineto{\pgfqpoint{3.469331in}{3.675372in}}%
\pgfpathlineto{\pgfqpoint{3.463105in}{3.681731in}}%
\pgfpathlineto{\pgfqpoint{3.456879in}{3.687777in}}%
\pgfpathlineto{\pgfqpoint{3.450656in}{3.693472in}}%
\pgfpathlineto{\pgfqpoint{3.444433in}{3.698811in}}%
\pgfpathlineto{\pgfqpoint{3.438212in}{3.703804in}}%
\pgfpathlineto{\pgfqpoint{3.427212in}{3.698168in}}%
\pgfpathlineto{\pgfqpoint{3.416208in}{3.692484in}}%
\pgfpathlineto{\pgfqpoint{3.405200in}{3.686751in}}%
\pgfpathlineto{\pgfqpoint{3.394187in}{3.680972in}}%
\pgfpathlineto{\pgfqpoint{3.383170in}{3.675149in}}%
\pgfpathlineto{\pgfqpoint{3.389377in}{3.670416in}}%
\pgfpathlineto{\pgfqpoint{3.395586in}{3.665338in}}%
\pgfpathlineto{\pgfqpoint{3.401797in}{3.659902in}}%
\pgfpathlineto{\pgfqpoint{3.408010in}{3.654109in}}%
\pgfpathclose%
\pgfusepath{stroke,fill}%
\end{pgfscope}%
\begin{pgfscope}%
\pgfpathrectangle{\pgfqpoint{0.887500in}{0.275000in}}{\pgfqpoint{4.225000in}{4.225000in}}%
\pgfusepath{clip}%
\pgfsetbuttcap%
\pgfsetroundjoin%
\definecolor{currentfill}{rgb}{0.120081,0.622161,0.534946}%
\pgfsetfillcolor{currentfill}%
\pgfsetfillopacity{0.700000}%
\pgfsetlinewidth{0.501875pt}%
\definecolor{currentstroke}{rgb}{1.000000,1.000000,1.000000}%
\pgfsetstrokecolor{currentstroke}%
\pgfsetstrokeopacity{0.500000}%
\pgfsetdash{}{0pt}%
\pgfpathmoveto{\pgfqpoint{2.011111in}{2.872527in}}%
\pgfpathlineto{\pgfqpoint{2.022441in}{2.876307in}}%
\pgfpathlineto{\pgfqpoint{2.033762in}{2.880208in}}%
\pgfpathlineto{\pgfqpoint{2.045076in}{2.884219in}}%
\pgfpathlineto{\pgfqpoint{2.056382in}{2.888329in}}%
\pgfpathlineto{\pgfqpoint{2.067681in}{2.892528in}}%
\pgfpathlineto{\pgfqpoint{2.061926in}{2.900546in}}%
\pgfpathlineto{\pgfqpoint{2.056175in}{2.908557in}}%
\pgfpathlineto{\pgfqpoint{2.050428in}{2.916557in}}%
\pgfpathlineto{\pgfqpoint{2.044684in}{2.924543in}}%
\pgfpathlineto{\pgfqpoint{2.038945in}{2.932509in}}%
\pgfpathlineto{\pgfqpoint{2.027678in}{2.927573in}}%
\pgfpathlineto{\pgfqpoint{2.016397in}{2.923012in}}%
\pgfpathlineto{\pgfqpoint{2.005102in}{2.918767in}}%
\pgfpathlineto{\pgfqpoint{1.993796in}{2.914775in}}%
\pgfpathlineto{\pgfqpoint{1.982480in}{2.910974in}}%
\pgfpathlineto{\pgfqpoint{1.988198in}{2.903308in}}%
\pgfpathlineto{\pgfqpoint{1.993920in}{2.895630in}}%
\pgfpathlineto{\pgfqpoint{1.999647in}{2.887941in}}%
\pgfpathlineto{\pgfqpoint{2.005377in}{2.880240in}}%
\pgfpathclose%
\pgfusepath{stroke,fill}%
\end{pgfscope}%
\begin{pgfscope}%
\pgfpathrectangle{\pgfqpoint{0.887500in}{0.275000in}}{\pgfqpoint{4.225000in}{4.225000in}}%
\pgfusepath{clip}%
\pgfsetbuttcap%
\pgfsetroundjoin%
\definecolor{currentfill}{rgb}{0.128087,0.647749,0.523491}%
\pgfsetfillcolor{currentfill}%
\pgfsetfillopacity{0.700000}%
\pgfsetlinewidth{0.501875pt}%
\definecolor{currentstroke}{rgb}{1.000000,1.000000,1.000000}%
\pgfsetstrokecolor{currentstroke}%
\pgfsetstrokeopacity{0.500000}%
\pgfsetdash{}{0pt}%
\pgfpathmoveto{\pgfqpoint{2.294122in}{2.905692in}}%
\pgfpathlineto{\pgfqpoint{2.305243in}{2.916970in}}%
\pgfpathlineto{\pgfqpoint{2.316372in}{2.927786in}}%
\pgfpathlineto{\pgfqpoint{2.327508in}{2.938186in}}%
\pgfpathlineto{\pgfqpoint{2.338648in}{2.948222in}}%
\pgfpathlineto{\pgfqpoint{2.349793in}{2.957948in}}%
\pgfpathlineto{\pgfqpoint{2.343935in}{2.966385in}}%
\pgfpathlineto{\pgfqpoint{2.338084in}{2.974670in}}%
\pgfpathlineto{\pgfqpoint{2.332239in}{2.982793in}}%
\pgfpathlineto{\pgfqpoint{2.326402in}{2.990751in}}%
\pgfpathlineto{\pgfqpoint{2.320570in}{2.998560in}}%
\pgfpathlineto{\pgfqpoint{2.309443in}{2.988760in}}%
\pgfpathlineto{\pgfqpoint{2.298317in}{2.978847in}}%
\pgfpathlineto{\pgfqpoint{2.287192in}{2.968742in}}%
\pgfpathlineto{\pgfqpoint{2.276071in}{2.958363in}}%
\pgfpathlineto{\pgfqpoint{2.264956in}{2.947635in}}%
\pgfpathlineto{\pgfqpoint{2.270787in}{2.939011in}}%
\pgfpathlineto{\pgfqpoint{2.276619in}{2.930512in}}%
\pgfpathlineto{\pgfqpoint{2.282451in}{2.922145in}}%
\pgfpathlineto{\pgfqpoint{2.288286in}{2.913882in}}%
\pgfpathclose%
\pgfusepath{stroke,fill}%
\end{pgfscope}%
\begin{pgfscope}%
\pgfpathrectangle{\pgfqpoint{0.887500in}{0.275000in}}{\pgfqpoint{4.225000in}{4.225000in}}%
\pgfusepath{clip}%
\pgfsetbuttcap%
\pgfsetroundjoin%
\definecolor{currentfill}{rgb}{0.762373,0.876424,0.137064}%
\pgfsetfillcolor{currentfill}%
\pgfsetfillopacity{0.700000}%
\pgfsetlinewidth{0.501875pt}%
\definecolor{currentstroke}{rgb}{1.000000,1.000000,1.000000}%
\pgfsetstrokecolor{currentstroke}%
\pgfsetstrokeopacity{0.500000}%
\pgfsetdash{}{0pt}%
\pgfpathmoveto{\pgfqpoint{3.075992in}{3.564660in}}%
\pgfpathlineto{\pgfqpoint{3.087089in}{3.569383in}}%
\pgfpathlineto{\pgfqpoint{3.098181in}{3.574148in}}%
\pgfpathlineto{\pgfqpoint{3.109269in}{3.578966in}}%
\pgfpathlineto{\pgfqpoint{3.120353in}{3.583851in}}%
\pgfpathlineto{\pgfqpoint{3.131432in}{3.588810in}}%
\pgfpathlineto{\pgfqpoint{3.125316in}{3.590194in}}%
\pgfpathlineto{\pgfqpoint{3.119205in}{3.591456in}}%
\pgfpathlineto{\pgfqpoint{3.113100in}{3.592618in}}%
\pgfpathlineto{\pgfqpoint{3.107000in}{3.593697in}}%
\pgfpathlineto{\pgfqpoint{3.100907in}{3.594708in}}%
\pgfpathlineto{\pgfqpoint{3.089848in}{3.588976in}}%
\pgfpathlineto{\pgfqpoint{3.078785in}{3.583285in}}%
\pgfpathlineto{\pgfqpoint{3.067718in}{3.577684in}}%
\pgfpathlineto{\pgfqpoint{3.056648in}{3.572223in}}%
\pgfpathlineto{\pgfqpoint{3.045573in}{3.566955in}}%
\pgfpathlineto{\pgfqpoint{3.051645in}{3.566403in}}%
\pgfpathlineto{\pgfqpoint{3.057722in}{3.565912in}}%
\pgfpathlineto{\pgfqpoint{3.063806in}{3.565481in}}%
\pgfpathlineto{\pgfqpoint{3.069896in}{3.565082in}}%
\pgfpathclose%
\pgfusepath{stroke,fill}%
\end{pgfscope}%
\begin{pgfscope}%
\pgfpathrectangle{\pgfqpoint{0.887500in}{0.275000in}}{\pgfqpoint{4.225000in}{4.225000in}}%
\pgfusepath{clip}%
\pgfsetbuttcap%
\pgfsetroundjoin%
\definecolor{currentfill}{rgb}{0.916242,0.896091,0.100717}%
\pgfsetfillcolor{currentfill}%
\pgfsetfillopacity{0.700000}%
\pgfsetlinewidth{0.501875pt}%
\definecolor{currentstroke}{rgb}{1.000000,1.000000,1.000000}%
\pgfsetstrokecolor{currentstroke}%
\pgfsetstrokeopacity{0.500000}%
\pgfsetdash{}{0pt}%
\pgfpathmoveto{\pgfqpoint{3.783002in}{3.657945in}}%
\pgfpathlineto{\pgfqpoint{3.793953in}{3.662668in}}%
\pgfpathlineto{\pgfqpoint{3.804899in}{3.667358in}}%
\pgfpathlineto{\pgfqpoint{3.815840in}{3.672021in}}%
\pgfpathlineto{\pgfqpoint{3.826776in}{3.676664in}}%
\pgfpathlineto{\pgfqpoint{3.837708in}{3.681295in}}%
\pgfpathlineto{\pgfqpoint{3.831407in}{3.690749in}}%
\pgfpathlineto{\pgfqpoint{3.825106in}{3.699959in}}%
\pgfpathlineto{\pgfqpoint{3.818803in}{3.708931in}}%
\pgfpathlineto{\pgfqpoint{3.812500in}{3.717668in}}%
\pgfpathlineto{\pgfqpoint{3.806197in}{3.726174in}}%
\pgfpathlineto{\pgfqpoint{3.795274in}{3.721492in}}%
\pgfpathlineto{\pgfqpoint{3.784346in}{3.716784in}}%
\pgfpathlineto{\pgfqpoint{3.773413in}{3.712048in}}%
\pgfpathlineto{\pgfqpoint{3.762475in}{3.707281in}}%
\pgfpathlineto{\pgfqpoint{3.751533in}{3.702478in}}%
\pgfpathlineto{\pgfqpoint{3.757826in}{3.693995in}}%
\pgfpathlineto{\pgfqpoint{3.764120in}{3.685307in}}%
\pgfpathlineto{\pgfqpoint{3.770414in}{3.676407in}}%
\pgfpathlineto{\pgfqpoint{3.776708in}{3.667288in}}%
\pgfpathclose%
\pgfusepath{stroke,fill}%
\end{pgfscope}%
\begin{pgfscope}%
\pgfpathrectangle{\pgfqpoint{0.887500in}{0.275000in}}{\pgfqpoint{4.225000in}{4.225000in}}%
\pgfusepath{clip}%
\pgfsetbuttcap%
\pgfsetroundjoin%
\definecolor{currentfill}{rgb}{0.122312,0.633153,0.530398}%
\pgfsetfillcolor{currentfill}%
\pgfsetfillopacity{0.700000}%
\pgfsetlinewidth{0.501875pt}%
\definecolor{currentstroke}{rgb}{1.000000,1.000000,1.000000}%
\pgfsetstrokecolor{currentstroke}%
\pgfsetstrokeopacity{0.500000}%
\pgfsetdash{}{0pt}%
\pgfpathmoveto{\pgfqpoint{1.783630in}{2.896927in}}%
\pgfpathlineto{\pgfqpoint{1.795023in}{2.900373in}}%
\pgfpathlineto{\pgfqpoint{1.806410in}{2.903837in}}%
\pgfpathlineto{\pgfqpoint{1.817792in}{2.907314in}}%
\pgfpathlineto{\pgfqpoint{1.829168in}{2.910797in}}%
\pgfpathlineto{\pgfqpoint{1.840538in}{2.914279in}}%
\pgfpathlineto{\pgfqpoint{1.834872in}{2.921796in}}%
\pgfpathlineto{\pgfqpoint{1.829210in}{2.929298in}}%
\pgfpathlineto{\pgfqpoint{1.823552in}{2.936785in}}%
\pgfpathlineto{\pgfqpoint{1.817898in}{2.944255in}}%
\pgfpathlineto{\pgfqpoint{1.806539in}{2.940732in}}%
\pgfpathlineto{\pgfqpoint{1.795175in}{2.937212in}}%
\pgfpathlineto{\pgfqpoint{1.783806in}{2.933703in}}%
\pgfpathlineto{\pgfqpoint{1.772430in}{2.930215in}}%
\pgfpathlineto{\pgfqpoint{1.761048in}{2.926753in}}%
\pgfpathlineto{\pgfqpoint{1.766687in}{2.919322in}}%
\pgfpathlineto{\pgfqpoint{1.772331in}{2.911874in}}%
\pgfpathlineto{\pgfqpoint{1.777979in}{2.904409in}}%
\pgfpathclose%
\pgfusepath{stroke,fill}%
\end{pgfscope}%
\begin{pgfscope}%
\pgfpathrectangle{\pgfqpoint{0.887500in}{0.275000in}}{\pgfqpoint{4.225000in}{4.225000in}}%
\pgfusepath{clip}%
\pgfsetbuttcap%
\pgfsetroundjoin%
\definecolor{currentfill}{rgb}{0.120081,0.622161,0.534946}%
\pgfsetfillcolor{currentfill}%
\pgfsetfillopacity{0.700000}%
\pgfsetlinewidth{0.501875pt}%
\definecolor{currentstroke}{rgb}{1.000000,1.000000,1.000000}%
\pgfsetstrokecolor{currentstroke}%
\pgfsetstrokeopacity{0.500000}%
\pgfsetdash{}{0pt}%
\pgfpathmoveto{\pgfqpoint{2.238467in}{2.851045in}}%
\pgfpathlineto{\pgfqpoint{2.249627in}{2.860629in}}%
\pgfpathlineto{\pgfqpoint{2.260766in}{2.871222in}}%
\pgfpathlineto{\pgfqpoint{2.271890in}{2.882486in}}%
\pgfpathlineto{\pgfqpoint{2.283006in}{2.894088in}}%
\pgfpathlineto{\pgfqpoint{2.294122in}{2.905692in}}%
\pgfpathlineto{\pgfqpoint{2.288286in}{2.913882in}}%
\pgfpathlineto{\pgfqpoint{2.282451in}{2.922145in}}%
\pgfpathlineto{\pgfqpoint{2.276619in}{2.930512in}}%
\pgfpathlineto{\pgfqpoint{2.270787in}{2.939011in}}%
\pgfpathlineto{\pgfqpoint{2.264956in}{2.947635in}}%
\pgfpathlineto{\pgfqpoint{2.253845in}{2.936653in}}%
\pgfpathlineto{\pgfqpoint{2.242733in}{2.925686in}}%
\pgfpathlineto{\pgfqpoint{2.231613in}{2.915012in}}%
\pgfpathlineto{\pgfqpoint{2.220480in}{2.904908in}}%
\pgfpathlineto{\pgfqpoint{2.209328in}{2.895652in}}%
\pgfpathlineto{\pgfqpoint{2.215164in}{2.886025in}}%
\pgfpathlineto{\pgfqpoint{2.220996in}{2.876752in}}%
\pgfpathlineto{\pgfqpoint{2.226823in}{2.867859in}}%
\pgfpathlineto{\pgfqpoint{2.232646in}{2.859307in}}%
\pgfpathclose%
\pgfusepath{stroke,fill}%
\end{pgfscope}%
\begin{pgfscope}%
\pgfpathrectangle{\pgfqpoint{0.887500in}{0.275000in}}{\pgfqpoint{4.225000in}{4.225000in}}%
\pgfusepath{clip}%
\pgfsetbuttcap%
\pgfsetroundjoin%
\definecolor{currentfill}{rgb}{0.150148,0.676631,0.506589}%
\pgfsetfillcolor{currentfill}%
\pgfsetfillopacity{0.700000}%
\pgfsetlinewidth{0.501875pt}%
\definecolor{currentstroke}{rgb}{1.000000,1.000000,1.000000}%
\pgfsetstrokecolor{currentstroke}%
\pgfsetstrokeopacity{0.500000}%
\pgfsetdash{}{0pt}%
\pgfpathmoveto{\pgfqpoint{2.576588in}{2.969849in}}%
\pgfpathlineto{\pgfqpoint{2.587736in}{2.978337in}}%
\pgfpathlineto{\pgfqpoint{2.598865in}{2.988285in}}%
\pgfpathlineto{\pgfqpoint{2.609977in}{2.999649in}}%
\pgfpathlineto{\pgfqpoint{2.621074in}{3.012300in}}%
\pgfpathlineto{\pgfqpoint{2.632159in}{3.026108in}}%
\pgfpathlineto{\pgfqpoint{2.626226in}{3.032925in}}%
\pgfpathlineto{\pgfqpoint{2.620298in}{3.039699in}}%
\pgfpathlineto{\pgfqpoint{2.614375in}{3.046408in}}%
\pgfpathlineto{\pgfqpoint{2.608457in}{3.053040in}}%
\pgfpathlineto{\pgfqpoint{2.602544in}{3.059589in}}%
\pgfpathlineto{\pgfqpoint{2.591489in}{3.045022in}}%
\pgfpathlineto{\pgfqpoint{2.580415in}{3.032181in}}%
\pgfpathlineto{\pgfqpoint{2.569319in}{3.021052in}}%
\pgfpathlineto{\pgfqpoint{2.558202in}{3.011623in}}%
\pgfpathlineto{\pgfqpoint{2.547062in}{3.003820in}}%
\pgfpathlineto{\pgfqpoint{2.552977in}{2.995539in}}%
\pgfpathlineto{\pgfqpoint{2.558888in}{2.987947in}}%
\pgfpathlineto{\pgfqpoint{2.564793in}{2.981173in}}%
\pgfpathlineto{\pgfqpoint{2.570692in}{2.975191in}}%
\pgfpathclose%
\pgfusepath{stroke,fill}%
\end{pgfscope}%
\begin{pgfscope}%
\pgfpathrectangle{\pgfqpoint{0.887500in}{0.275000in}}{\pgfqpoint{4.225000in}{4.225000in}}%
\pgfusepath{clip}%
\pgfsetbuttcap%
\pgfsetroundjoin%
\definecolor{currentfill}{rgb}{0.119483,0.614817,0.537692}%
\pgfsetfillcolor{currentfill}%
\pgfsetfillopacity{0.700000}%
\pgfsetlinewidth{0.501875pt}%
\definecolor{currentstroke}{rgb}{1.000000,1.000000,1.000000}%
\pgfsetstrokecolor{currentstroke}%
\pgfsetstrokeopacity{0.500000}%
\pgfsetdash{}{0pt}%
\pgfpathmoveto{\pgfqpoint{2.096509in}{2.852473in}}%
\pgfpathlineto{\pgfqpoint{2.107835in}{2.855819in}}%
\pgfpathlineto{\pgfqpoint{2.119164in}{2.858868in}}%
\pgfpathlineto{\pgfqpoint{2.130493in}{2.861654in}}%
\pgfpathlineto{\pgfqpoint{2.141816in}{2.864424in}}%
\pgfpathlineto{\pgfqpoint{2.153128in}{2.867450in}}%
\pgfpathlineto{\pgfqpoint{2.147316in}{2.876695in}}%
\pgfpathlineto{\pgfqpoint{2.141502in}{2.886117in}}%
\pgfpathlineto{\pgfqpoint{2.135689in}{2.895682in}}%
\pgfpathlineto{\pgfqpoint{2.129876in}{2.905362in}}%
\pgfpathlineto{\pgfqpoint{2.124064in}{2.915125in}}%
\pgfpathlineto{\pgfqpoint{2.112806in}{2.910211in}}%
\pgfpathlineto{\pgfqpoint{2.101537in}{2.905586in}}%
\pgfpathlineto{\pgfqpoint{2.090258in}{2.901149in}}%
\pgfpathlineto{\pgfqpoint{2.078973in}{2.896804in}}%
\pgfpathlineto{\pgfqpoint{2.067681in}{2.892528in}}%
\pgfpathlineto{\pgfqpoint{2.073439in}{2.884506in}}%
\pgfpathlineto{\pgfqpoint{2.079201in}{2.876486in}}%
\pgfpathlineto{\pgfqpoint{2.084967in}{2.868472in}}%
\pgfpathlineto{\pgfqpoint{2.090736in}{2.860466in}}%
\pgfpathclose%
\pgfusepath{stroke,fill}%
\end{pgfscope}%
\begin{pgfscope}%
\pgfpathrectangle{\pgfqpoint{0.887500in}{0.275000in}}{\pgfqpoint{4.225000in}{4.225000in}}%
\pgfusepath{clip}%
\pgfsetbuttcap%
\pgfsetroundjoin%
\definecolor{currentfill}{rgb}{0.143303,0.669459,0.511215}%
\pgfsetfillcolor{currentfill}%
\pgfsetfillopacity{0.700000}%
\pgfsetlinewidth{0.501875pt}%
\definecolor{currentstroke}{rgb}{1.000000,1.000000,1.000000}%
\pgfsetstrokecolor{currentstroke}%
\pgfsetstrokeopacity{0.500000}%
\pgfsetdash{}{0pt}%
\pgfpathmoveto{\pgfqpoint{2.435104in}{2.951826in}}%
\pgfpathlineto{\pgfqpoint{2.446302in}{2.958127in}}%
\pgfpathlineto{\pgfqpoint{2.457503in}{2.963973in}}%
\pgfpathlineto{\pgfqpoint{2.468708in}{2.969355in}}%
\pgfpathlineto{\pgfqpoint{2.479915in}{2.974264in}}%
\pgfpathlineto{\pgfqpoint{2.491123in}{2.978718in}}%
\pgfpathlineto{\pgfqpoint{2.485192in}{2.989230in}}%
\pgfpathlineto{\pgfqpoint{2.479256in}{3.000258in}}%
\pgfpathlineto{\pgfqpoint{2.473319in}{3.011597in}}%
\pgfpathlineto{\pgfqpoint{2.467383in}{3.023043in}}%
\pgfpathlineto{\pgfqpoint{2.461451in}{3.034392in}}%
\pgfpathlineto{\pgfqpoint{2.450247in}{3.030112in}}%
\pgfpathlineto{\pgfqpoint{2.439056in}{3.024721in}}%
\pgfpathlineto{\pgfqpoint{2.427876in}{3.018326in}}%
\pgfpathlineto{\pgfqpoint{2.416706in}{3.011081in}}%
\pgfpathlineto{\pgfqpoint{2.405545in}{3.003140in}}%
\pgfpathlineto{\pgfqpoint{2.411449in}{2.992971in}}%
\pgfpathlineto{\pgfqpoint{2.417359in}{2.982639in}}%
\pgfpathlineto{\pgfqpoint{2.423272in}{2.972260in}}%
\pgfpathlineto{\pgfqpoint{2.429188in}{2.961950in}}%
\pgfpathclose%
\pgfusepath{stroke,fill}%
\end{pgfscope}%
\begin{pgfscope}%
\pgfpathrectangle{\pgfqpoint{0.887500in}{0.275000in}}{\pgfqpoint{4.225000in}{4.225000in}}%
\pgfusepath{clip}%
\pgfsetbuttcap%
\pgfsetroundjoin%
\definecolor{currentfill}{rgb}{0.668054,0.861999,0.196293}%
\pgfsetfillcolor{currentfill}%
\pgfsetfillopacity{0.700000}%
\pgfsetlinewidth{0.501875pt}%
\definecolor{currentstroke}{rgb}{1.000000,1.000000,1.000000}%
\pgfsetstrokecolor{currentstroke}%
\pgfsetstrokeopacity{0.500000}%
\pgfsetdash{}{0pt}%
\pgfpathmoveto{\pgfqpoint{2.878903in}{3.500978in}}%
\pgfpathlineto{\pgfqpoint{2.890038in}{3.507310in}}%
\pgfpathlineto{\pgfqpoint{2.901173in}{3.512968in}}%
\pgfpathlineto{\pgfqpoint{2.912306in}{3.518044in}}%
\pgfpathlineto{\pgfqpoint{2.923436in}{3.522628in}}%
\pgfpathlineto{\pgfqpoint{2.934563in}{3.526813in}}%
\pgfpathlineto{\pgfqpoint{2.928545in}{3.526031in}}%
\pgfpathlineto{\pgfqpoint{2.922534in}{3.524992in}}%
\pgfpathlineto{\pgfqpoint{2.916530in}{3.523784in}}%
\pgfpathlineto{\pgfqpoint{2.910534in}{3.522493in}}%
\pgfpathlineto{\pgfqpoint{2.904543in}{3.521209in}}%
\pgfpathlineto{\pgfqpoint{2.893457in}{3.512978in}}%
\pgfpathlineto{\pgfqpoint{2.882372in}{3.503680in}}%
\pgfpathlineto{\pgfqpoint{2.871290in}{3.493473in}}%
\pgfpathlineto{\pgfqpoint{2.860210in}{3.482515in}}%
\pgfpathlineto{\pgfqpoint{2.849132in}{3.470964in}}%
\pgfpathlineto{\pgfqpoint{2.855070in}{3.477088in}}%
\pgfpathlineto{\pgfqpoint{2.861014in}{3.483365in}}%
\pgfpathlineto{\pgfqpoint{2.866968in}{3.489582in}}%
\pgfpathlineto{\pgfqpoint{2.872930in}{3.495525in}}%
\pgfpathclose%
\pgfusepath{stroke,fill}%
\end{pgfscope}%
\begin{pgfscope}%
\pgfpathrectangle{\pgfqpoint{0.887500in}{0.275000in}}{\pgfqpoint{4.225000in}{4.225000in}}%
\pgfusepath{clip}%
\pgfsetbuttcap%
\pgfsetroundjoin%
\definecolor{currentfill}{rgb}{0.120638,0.625828,0.533488}%
\pgfsetfillcolor{currentfill}%
\pgfsetfillopacity{0.700000}%
\pgfsetlinewidth{0.501875pt}%
\definecolor{currentstroke}{rgb}{1.000000,1.000000,1.000000}%
\pgfsetstrokecolor{currentstroke}%
\pgfsetstrokeopacity{0.500000}%
\pgfsetdash{}{0pt}%
\pgfpathmoveto{\pgfqpoint{1.868931in}{2.876470in}}%
\pgfpathlineto{\pgfqpoint{1.880310in}{2.879905in}}%
\pgfpathlineto{\pgfqpoint{1.891685in}{2.883323in}}%
\pgfpathlineto{\pgfqpoint{1.903055in}{2.886720in}}%
\pgfpathlineto{\pgfqpoint{1.914420in}{2.890094in}}%
\pgfpathlineto{\pgfqpoint{1.925780in}{2.893462in}}%
\pgfpathlineto{\pgfqpoint{1.920079in}{2.901089in}}%
\pgfpathlineto{\pgfqpoint{1.914382in}{2.908705in}}%
\pgfpathlineto{\pgfqpoint{1.908689in}{2.916308in}}%
\pgfpathlineto{\pgfqpoint{1.903001in}{2.923898in}}%
\pgfpathlineto{\pgfqpoint{1.897316in}{2.931476in}}%
\pgfpathlineto{\pgfqpoint{1.885970in}{2.928068in}}%
\pgfpathlineto{\pgfqpoint{1.874619in}{2.924653in}}%
\pgfpathlineto{\pgfqpoint{1.863264in}{2.921214in}}%
\pgfpathlineto{\pgfqpoint{1.851903in}{2.917754in}}%
\pgfpathlineto{\pgfqpoint{1.840538in}{2.914279in}}%
\pgfpathlineto{\pgfqpoint{1.846208in}{2.906747in}}%
\pgfpathlineto{\pgfqpoint{1.851883in}{2.899199in}}%
\pgfpathlineto{\pgfqpoint{1.857561in}{2.891637in}}%
\pgfpathlineto{\pgfqpoint{1.863244in}{2.884061in}}%
\pgfpathclose%
\pgfusepath{stroke,fill}%
\end{pgfscope}%
\begin{pgfscope}%
\pgfpathrectangle{\pgfqpoint{0.887500in}{0.275000in}}{\pgfqpoint{4.225000in}{4.225000in}}%
\pgfusepath{clip}%
\pgfsetbuttcap%
\pgfsetroundjoin%
\definecolor{currentfill}{rgb}{0.906311,0.894855,0.098125}%
\pgfsetfillcolor{currentfill}%
\pgfsetfillopacity{0.700000}%
\pgfsetlinewidth{0.501875pt}%
\definecolor{currentstroke}{rgb}{1.000000,1.000000,1.000000}%
\pgfsetstrokecolor{currentstroke}%
\pgfsetstrokeopacity{0.500000}%
\pgfsetdash{}{0pt}%
\pgfpathmoveto{\pgfqpoint{3.641845in}{3.652464in}}%
\pgfpathlineto{\pgfqpoint{3.652833in}{3.657509in}}%
\pgfpathlineto{\pgfqpoint{3.663818in}{3.662590in}}%
\pgfpathlineto{\pgfqpoint{3.674799in}{3.667686in}}%
\pgfpathlineto{\pgfqpoint{3.685775in}{3.672777in}}%
\pgfpathlineto{\pgfqpoint{3.696747in}{3.677843in}}%
\pgfpathlineto{\pgfqpoint{3.690465in}{3.686133in}}%
\pgfpathlineto{\pgfqpoint{3.684185in}{3.694255in}}%
\pgfpathlineto{\pgfqpoint{3.677906in}{3.702210in}}%
\pgfpathlineto{\pgfqpoint{3.671629in}{3.709993in}}%
\pgfpathlineto{\pgfqpoint{3.665353in}{3.717597in}}%
\pgfpathlineto{\pgfqpoint{3.654393in}{3.712587in}}%
\pgfpathlineto{\pgfqpoint{3.643429in}{3.707539in}}%
\pgfpathlineto{\pgfqpoint{3.632460in}{3.702458in}}%
\pgfpathlineto{\pgfqpoint{3.621486in}{3.697349in}}%
\pgfpathlineto{\pgfqpoint{3.610508in}{3.692218in}}%
\pgfpathlineto{\pgfqpoint{3.616772in}{3.684573in}}%
\pgfpathlineto{\pgfqpoint{3.623037in}{3.676765in}}%
\pgfpathlineto{\pgfqpoint{3.629304in}{3.668806in}}%
\pgfpathlineto{\pgfqpoint{3.635574in}{3.660707in}}%
\pgfpathclose%
\pgfusepath{stroke,fill}%
\end{pgfscope}%
\begin{pgfscope}%
\pgfpathrectangle{\pgfqpoint{0.887500in}{0.275000in}}{\pgfqpoint{4.225000in}{4.225000in}}%
\pgfusepath{clip}%
\pgfsetbuttcap%
\pgfsetroundjoin%
\definecolor{currentfill}{rgb}{0.824940,0.884720,0.106217}%
\pgfsetfillcolor{currentfill}%
\pgfsetfillopacity{0.700000}%
\pgfsetlinewidth{0.501875pt}%
\definecolor{currentstroke}{rgb}{1.000000,1.000000,1.000000}%
\pgfsetstrokecolor{currentstroke}%
\pgfsetstrokeopacity{0.500000}%
\pgfsetdash{}{0pt}%
\pgfpathmoveto{\pgfqpoint{3.217512in}{3.598856in}}%
\pgfpathlineto{\pgfqpoint{3.228581in}{3.602973in}}%
\pgfpathlineto{\pgfqpoint{3.239645in}{3.607167in}}%
\pgfpathlineto{\pgfqpoint{3.250706in}{3.611458in}}%
\pgfpathlineto{\pgfqpoint{3.261762in}{3.615870in}}%
\pgfpathlineto{\pgfqpoint{3.272815in}{3.620426in}}%
\pgfpathlineto{\pgfqpoint{3.266642in}{3.624320in}}%
\pgfpathlineto{\pgfqpoint{3.260472in}{3.627976in}}%
\pgfpathlineto{\pgfqpoint{3.254307in}{3.631390in}}%
\pgfpathlineto{\pgfqpoint{3.248146in}{3.634557in}}%
\pgfpathlineto{\pgfqpoint{3.241989in}{3.637470in}}%
\pgfpathlineto{\pgfqpoint{3.230954in}{3.632744in}}%
\pgfpathlineto{\pgfqpoint{3.219914in}{3.628075in}}%
\pgfpathlineto{\pgfqpoint{3.208869in}{3.623402in}}%
\pgfpathlineto{\pgfqpoint{3.197820in}{3.618669in}}%
\pgfpathlineto{\pgfqpoint{3.186767in}{3.613828in}}%
\pgfpathlineto{\pgfqpoint{3.192906in}{3.611265in}}%
\pgfpathlineto{\pgfqpoint{3.199050in}{3.608471in}}%
\pgfpathlineto{\pgfqpoint{3.205199in}{3.605462in}}%
\pgfpathlineto{\pgfqpoint{3.211353in}{3.602252in}}%
\pgfpathclose%
\pgfusepath{stroke,fill}%
\end{pgfscope}%
\begin{pgfscope}%
\pgfpathrectangle{\pgfqpoint{0.887500in}{0.275000in}}{\pgfqpoint{4.225000in}{4.225000in}}%
\pgfusepath{clip}%
\pgfsetbuttcap%
\pgfsetroundjoin%
\definecolor{currentfill}{rgb}{0.430983,0.808473,0.346476}%
\pgfsetfillcolor{currentfill}%
\pgfsetfillopacity{0.700000}%
\pgfsetlinewidth{0.501875pt}%
\definecolor{currentstroke}{rgb}{1.000000,1.000000,1.000000}%
\pgfsetstrokecolor{currentstroke}%
\pgfsetstrokeopacity{0.500000}%
\pgfsetdash{}{0pt}%
\pgfpathmoveto{\pgfqpoint{2.712518in}{3.279322in}}%
\pgfpathlineto{\pgfqpoint{2.723536in}{3.302206in}}%
\pgfpathlineto{\pgfqpoint{2.734567in}{3.324086in}}%
\pgfpathlineto{\pgfqpoint{2.745612in}{3.344899in}}%
\pgfpathlineto{\pgfqpoint{2.756669in}{3.364582in}}%
\pgfpathlineto{\pgfqpoint{2.767739in}{3.383070in}}%
\pgfpathlineto{\pgfqpoint{2.761841in}{3.377178in}}%
\pgfpathlineto{\pgfqpoint{2.755964in}{3.369609in}}%
\pgfpathlineto{\pgfqpoint{2.750104in}{3.360802in}}%
\pgfpathlineto{\pgfqpoint{2.744260in}{3.351196in}}%
\pgfpathlineto{\pgfqpoint{2.738427in}{3.341228in}}%
\pgfpathlineto{\pgfqpoint{2.727368in}{3.326870in}}%
\pgfpathlineto{\pgfqpoint{2.716315in}{3.311990in}}%
\pgfpathlineto{\pgfqpoint{2.705268in}{3.296466in}}%
\pgfpathlineto{\pgfqpoint{2.694230in}{3.280175in}}%
\pgfpathlineto{\pgfqpoint{2.683202in}{3.262998in}}%
\pgfpathlineto{\pgfqpoint{2.689022in}{3.269460in}}%
\pgfpathlineto{\pgfqpoint{2.694860in}{3.274805in}}%
\pgfpathlineto{\pgfqpoint{2.700719in}{3.278550in}}%
\pgfpathlineto{\pgfqpoint{2.706604in}{3.280217in}}%
\pgfpathclose%
\pgfusepath{stroke,fill}%
\end{pgfscope}%
\begin{pgfscope}%
\pgfpathrectangle{\pgfqpoint{0.887500in}{0.275000in}}{\pgfqpoint{4.225000in}{4.225000in}}%
\pgfusepath{clip}%
\pgfsetbuttcap%
\pgfsetroundjoin%
\definecolor{currentfill}{rgb}{0.886271,0.892374,0.095374}%
\pgfsetfillcolor{currentfill}%
\pgfsetfillopacity{0.700000}%
\pgfsetlinewidth{0.501875pt}%
\definecolor{currentstroke}{rgb}{1.000000,1.000000,1.000000}%
\pgfsetstrokecolor{currentstroke}%
\pgfsetstrokeopacity{0.500000}%
\pgfsetdash{}{0pt}%
\pgfpathmoveto{\pgfqpoint{3.869192in}{3.630246in}}%
\pgfpathlineto{\pgfqpoint{3.880125in}{3.634734in}}%
\pgfpathlineto{\pgfqpoint{3.891053in}{3.639235in}}%
\pgfpathlineto{\pgfqpoint{3.901978in}{3.643760in}}%
\pgfpathlineto{\pgfqpoint{3.912899in}{3.648318in}}%
\pgfpathlineto{\pgfqpoint{3.906602in}{3.659178in}}%
\pgfpathlineto{\pgfqpoint{3.900303in}{3.669761in}}%
\pgfpathlineto{\pgfqpoint{3.894002in}{3.680065in}}%
\pgfpathlineto{\pgfqpoint{3.887698in}{3.690093in}}%
\pgfpathlineto{\pgfqpoint{3.881393in}{3.699849in}}%
\pgfpathlineto{\pgfqpoint{3.870478in}{3.695191in}}%
\pgfpathlineto{\pgfqpoint{3.859559in}{3.690552in}}%
\pgfpathlineto{\pgfqpoint{3.848636in}{3.685922in}}%
\pgfpathlineto{\pgfqpoint{3.837708in}{3.681295in}}%
\pgfpathlineto{\pgfqpoint{3.844008in}{3.671595in}}%
\pgfpathlineto{\pgfqpoint{3.850306in}{3.661643in}}%
\pgfpathlineto{\pgfqpoint{3.856603in}{3.651436in}}%
\pgfpathlineto{\pgfqpoint{3.862899in}{3.640970in}}%
\pgfpathclose%
\pgfusepath{stroke,fill}%
\end{pgfscope}%
\begin{pgfscope}%
\pgfpathrectangle{\pgfqpoint{0.887500in}{0.275000in}}{\pgfqpoint{4.225000in}{4.225000in}}%
\pgfusepath{clip}%
\pgfsetbuttcap%
\pgfsetroundjoin%
\definecolor{currentfill}{rgb}{0.202219,0.715272,0.476084}%
\pgfsetfillcolor{currentfill}%
\pgfsetfillopacity{0.700000}%
\pgfsetlinewidth{0.501875pt}%
\definecolor{currentstroke}{rgb}{1.000000,1.000000,1.000000}%
\pgfsetstrokecolor{currentstroke}%
\pgfsetstrokeopacity{0.500000}%
\pgfsetdash{}{0pt}%
\pgfpathmoveto{\pgfqpoint{2.632159in}{3.026108in}}%
\pgfpathlineto{\pgfqpoint{2.643234in}{3.040945in}}%
\pgfpathlineto{\pgfqpoint{2.654300in}{3.056681in}}%
\pgfpathlineto{\pgfqpoint{2.665361in}{3.073186in}}%
\pgfpathlineto{\pgfqpoint{2.676417in}{3.090413in}}%
\pgfpathlineto{\pgfqpoint{2.687467in}{3.108579in}}%
\pgfpathlineto{\pgfqpoint{2.681449in}{3.122179in}}%
\pgfpathlineto{\pgfqpoint{2.675447in}{3.134197in}}%
\pgfpathlineto{\pgfqpoint{2.669468in}{3.144231in}}%
\pgfpathlineto{\pgfqpoint{2.663512in}{3.152158in}}%
\pgfpathlineto{\pgfqpoint{2.657579in}{3.158198in}}%
\pgfpathlineto{\pgfqpoint{2.646596in}{3.135308in}}%
\pgfpathlineto{\pgfqpoint{2.635605in}{3.113781in}}%
\pgfpathlineto{\pgfqpoint{2.624601in}{3.093955in}}%
\pgfpathlineto{\pgfqpoint{2.613581in}{3.075895in}}%
\pgfpathlineto{\pgfqpoint{2.602544in}{3.059589in}}%
\pgfpathlineto{\pgfqpoint{2.608457in}{3.053040in}}%
\pgfpathlineto{\pgfqpoint{2.614375in}{3.046408in}}%
\pgfpathlineto{\pgfqpoint{2.620298in}{3.039699in}}%
\pgfpathlineto{\pgfqpoint{2.626226in}{3.032925in}}%
\pgfpathclose%
\pgfusepath{stroke,fill}%
\end{pgfscope}%
\begin{pgfscope}%
\pgfpathrectangle{\pgfqpoint{0.887500in}{0.275000in}}{\pgfqpoint{4.225000in}{4.225000in}}%
\pgfusepath{clip}%
\pgfsetbuttcap%
\pgfsetroundjoin%
\definecolor{currentfill}{rgb}{0.866013,0.889868,0.095953}%
\pgfsetfillcolor{currentfill}%
\pgfsetfillopacity{0.700000}%
\pgfsetlinewidth{0.501875pt}%
\definecolor{currentstroke}{rgb}{1.000000,1.000000,1.000000}%
\pgfsetstrokecolor{currentstroke}%
\pgfsetstrokeopacity{0.500000}%
\pgfsetdash{}{0pt}%
\pgfpathmoveto{\pgfqpoint{3.359019in}{3.620602in}}%
\pgfpathlineto{\pgfqpoint{3.370066in}{3.625895in}}%
\pgfpathlineto{\pgfqpoint{3.381111in}{3.631330in}}%
\pgfpathlineto{\pgfqpoint{3.392152in}{3.636859in}}%
\pgfpathlineto{\pgfqpoint{3.403190in}{3.642434in}}%
\pgfpathlineto{\pgfqpoint{3.414224in}{3.648008in}}%
\pgfpathlineto{\pgfqpoint{3.408010in}{3.654109in}}%
\pgfpathlineto{\pgfqpoint{3.401797in}{3.659902in}}%
\pgfpathlineto{\pgfqpoint{3.395586in}{3.665338in}}%
\pgfpathlineto{\pgfqpoint{3.389377in}{3.670416in}}%
\pgfpathlineto{\pgfqpoint{3.383170in}{3.675149in}}%
\pgfpathlineto{\pgfqpoint{3.372149in}{3.669284in}}%
\pgfpathlineto{\pgfqpoint{3.361125in}{3.663405in}}%
\pgfpathlineto{\pgfqpoint{3.350097in}{3.657551in}}%
\pgfpathlineto{\pgfqpoint{3.339065in}{3.651762in}}%
\pgfpathlineto{\pgfqpoint{3.328031in}{3.646079in}}%
\pgfpathlineto{\pgfqpoint{3.334223in}{3.641645in}}%
\pgfpathlineto{\pgfqpoint{3.340418in}{3.636883in}}%
\pgfpathlineto{\pgfqpoint{3.346616in}{3.631779in}}%
\pgfpathlineto{\pgfqpoint{3.352816in}{3.626337in}}%
\pgfpathclose%
\pgfusepath{stroke,fill}%
\end{pgfscope}%
\begin{pgfscope}%
\pgfpathrectangle{\pgfqpoint{0.887500in}{0.275000in}}{\pgfqpoint{4.225000in}{4.225000in}}%
\pgfusepath{clip}%
\pgfsetbuttcap%
\pgfsetroundjoin%
\definecolor{currentfill}{rgb}{0.896320,0.893616,0.096335}%
\pgfsetfillcolor{currentfill}%
\pgfsetfillopacity{0.700000}%
\pgfsetlinewidth{0.501875pt}%
\definecolor{currentstroke}{rgb}{1.000000,1.000000,1.000000}%
\pgfsetstrokecolor{currentstroke}%
\pgfsetstrokeopacity{0.500000}%
\pgfsetdash{}{0pt}%
\pgfpathmoveto{\pgfqpoint{3.500500in}{3.640597in}}%
\pgfpathlineto{\pgfqpoint{3.511519in}{3.645739in}}%
\pgfpathlineto{\pgfqpoint{3.522534in}{3.650890in}}%
\pgfpathlineto{\pgfqpoint{3.533545in}{3.656051in}}%
\pgfpathlineto{\pgfqpoint{3.544552in}{3.661220in}}%
\pgfpathlineto{\pgfqpoint{3.555555in}{3.666393in}}%
\pgfpathlineto{\pgfqpoint{3.549305in}{3.673811in}}%
\pgfpathlineto{\pgfqpoint{3.543058in}{3.681095in}}%
\pgfpathlineto{\pgfqpoint{3.536813in}{3.688217in}}%
\pgfpathlineto{\pgfqpoint{3.530570in}{3.695147in}}%
\pgfpathlineto{\pgfqpoint{3.524329in}{3.701857in}}%
\pgfpathlineto{\pgfqpoint{3.513338in}{3.696626in}}%
\pgfpathlineto{\pgfqpoint{3.502343in}{3.691364in}}%
\pgfpathlineto{\pgfqpoint{3.491344in}{3.686069in}}%
\pgfpathlineto{\pgfqpoint{3.480339in}{3.680739in}}%
\pgfpathlineto{\pgfqpoint{3.469331in}{3.675372in}}%
\pgfpathlineto{\pgfqpoint{3.475560in}{3.668751in}}%
\pgfpathlineto{\pgfqpoint{3.481790in}{3.661915in}}%
\pgfpathlineto{\pgfqpoint{3.488023in}{3.654912in}}%
\pgfpathlineto{\pgfqpoint{3.494260in}{3.647790in}}%
\pgfpathclose%
\pgfusepath{stroke,fill}%
\end{pgfscope}%
\begin{pgfscope}%
\pgfpathrectangle{\pgfqpoint{0.887500in}{0.275000in}}{\pgfqpoint{4.225000in}{4.225000in}}%
\pgfusepath{clip}%
\pgfsetbuttcap%
\pgfsetroundjoin%
\definecolor{currentfill}{rgb}{0.119512,0.607464,0.540218}%
\pgfsetfillcolor{currentfill}%
\pgfsetfillopacity{0.700000}%
\pgfsetlinewidth{0.501875pt}%
\definecolor{currentstroke}{rgb}{1.000000,1.000000,1.000000}%
\pgfsetstrokecolor{currentstroke}%
\pgfsetstrokeopacity{0.500000}%
\pgfsetdash{}{0pt}%
\pgfpathmoveto{\pgfqpoint{2.182167in}{2.824637in}}%
\pgfpathlineto{\pgfqpoint{2.193489in}{2.827464in}}%
\pgfpathlineto{\pgfqpoint{2.204785in}{2.831210in}}%
\pgfpathlineto{\pgfqpoint{2.216050in}{2.836212in}}%
\pgfpathlineto{\pgfqpoint{2.227277in}{2.842803in}}%
\pgfpathlineto{\pgfqpoint{2.238467in}{2.851045in}}%
\pgfpathlineto{\pgfqpoint{2.232646in}{2.859307in}}%
\pgfpathlineto{\pgfqpoint{2.226823in}{2.867859in}}%
\pgfpathlineto{\pgfqpoint{2.220996in}{2.876752in}}%
\pgfpathlineto{\pgfqpoint{2.215164in}{2.886025in}}%
\pgfpathlineto{\pgfqpoint{2.209328in}{2.895652in}}%
\pgfpathlineto{\pgfqpoint{2.198150in}{2.887520in}}%
\pgfpathlineto{\pgfqpoint{2.186939in}{2.880768in}}%
\pgfpathlineto{\pgfqpoint{2.175695in}{2.875351in}}%
\pgfpathlineto{\pgfqpoint{2.164423in}{2.871002in}}%
\pgfpathlineto{\pgfqpoint{2.153128in}{2.867450in}}%
\pgfpathlineto{\pgfqpoint{2.158940in}{2.858411in}}%
\pgfpathlineto{\pgfqpoint{2.164749in}{2.849609in}}%
\pgfpathlineto{\pgfqpoint{2.170556in}{2.841063in}}%
\pgfpathlineto{\pgfqpoint{2.176362in}{2.832749in}}%
\pgfpathclose%
\pgfusepath{stroke,fill}%
\end{pgfscope}%
\begin{pgfscope}%
\pgfpathrectangle{\pgfqpoint{0.887500in}{0.275000in}}{\pgfqpoint{4.225000in}{4.225000in}}%
\pgfusepath{clip}%
\pgfsetbuttcap%
\pgfsetroundjoin%
\definecolor{currentfill}{rgb}{0.616293,0.852709,0.230052}%
\pgfsetfillcolor{currentfill}%
\pgfsetfillopacity{0.700000}%
\pgfsetlinewidth{0.501875pt}%
\definecolor{currentstroke}{rgb}{1.000000,1.000000,1.000000}%
\pgfsetstrokecolor{currentstroke}%
\pgfsetstrokeopacity{0.500000}%
\pgfsetdash{}{0pt}%
\pgfpathmoveto{\pgfqpoint{2.823246in}{3.456192in}}%
\pgfpathlineto{\pgfqpoint{2.834371in}{3.467153in}}%
\pgfpathlineto{\pgfqpoint{2.845500in}{3.477043in}}%
\pgfpathlineto{\pgfqpoint{2.856632in}{3.485929in}}%
\pgfpathlineto{\pgfqpoint{2.867767in}{3.493881in}}%
\pgfpathlineto{\pgfqpoint{2.878903in}{3.500978in}}%
\pgfpathlineto{\pgfqpoint{2.872930in}{3.495525in}}%
\pgfpathlineto{\pgfqpoint{2.866968in}{3.489582in}}%
\pgfpathlineto{\pgfqpoint{2.861014in}{3.483365in}}%
\pgfpathlineto{\pgfqpoint{2.855070in}{3.477088in}}%
\pgfpathlineto{\pgfqpoint{2.849132in}{3.470964in}}%
\pgfpathlineto{\pgfqpoint{2.838056in}{3.458977in}}%
\pgfpathlineto{\pgfqpoint{2.826981in}{3.446709in}}%
\pgfpathlineto{\pgfqpoint{2.815906in}{3.434236in}}%
\pgfpathlineto{\pgfqpoint{2.804833in}{3.421561in}}%
\pgfpathlineto{\pgfqpoint{2.793761in}{3.408686in}}%
\pgfpathlineto{\pgfqpoint{2.799639in}{3.418435in}}%
\pgfpathlineto{\pgfqpoint{2.805524in}{3.428383in}}%
\pgfpathlineto{\pgfqpoint{2.811419in}{3.438208in}}%
\pgfpathlineto{\pgfqpoint{2.817326in}{3.447586in}}%
\pgfpathclose%
\pgfusepath{stroke,fill}%
\end{pgfscope}%
\begin{pgfscope}%
\pgfpathrectangle{\pgfqpoint{0.887500in}{0.275000in}}{\pgfqpoint{4.225000in}{4.225000in}}%
\pgfusepath{clip}%
\pgfsetbuttcap%
\pgfsetroundjoin%
\definecolor{currentfill}{rgb}{0.137339,0.662252,0.515571}%
\pgfsetfillcolor{currentfill}%
\pgfsetfillopacity{0.700000}%
\pgfsetlinewidth{0.501875pt}%
\definecolor{currentstroke}{rgb}{1.000000,1.000000,1.000000}%
\pgfsetstrokecolor{currentstroke}%
\pgfsetstrokeopacity{0.500000}%
\pgfsetdash{}{0pt}%
\pgfpathmoveto{\pgfqpoint{2.520644in}{2.939774in}}%
\pgfpathlineto{\pgfqpoint{2.531849in}{2.945129in}}%
\pgfpathlineto{\pgfqpoint{2.543049in}{2.950517in}}%
\pgfpathlineto{\pgfqpoint{2.554242in}{2.956236in}}%
\pgfpathlineto{\pgfqpoint{2.565422in}{2.962581in}}%
\pgfpathlineto{\pgfqpoint{2.576588in}{2.969849in}}%
\pgfpathlineto{\pgfqpoint{2.570692in}{2.975191in}}%
\pgfpathlineto{\pgfqpoint{2.564793in}{2.981173in}}%
\pgfpathlineto{\pgfqpoint{2.558888in}{2.987947in}}%
\pgfpathlineto{\pgfqpoint{2.552977in}{2.995539in}}%
\pgfpathlineto{\pgfqpoint{2.547062in}{3.003820in}}%
\pgfpathlineto{\pgfqpoint{2.535901in}{2.997366in}}%
\pgfpathlineto{\pgfqpoint{2.524724in}{2.991946in}}%
\pgfpathlineto{\pgfqpoint{2.513532in}{2.987242in}}%
\pgfpathlineto{\pgfqpoint{2.502331in}{2.982938in}}%
\pgfpathlineto{\pgfqpoint{2.491123in}{2.978718in}}%
\pgfpathlineto{\pgfqpoint{2.497048in}{2.968924in}}%
\pgfpathlineto{\pgfqpoint{2.502963in}{2.960052in}}%
\pgfpathlineto{\pgfqpoint{2.508866in}{2.952293in}}%
\pgfpathlineto{\pgfqpoint{2.514759in}{2.945609in}}%
\pgfpathclose%
\pgfusepath{stroke,fill}%
\end{pgfscope}%
\begin{pgfscope}%
\pgfpathrectangle{\pgfqpoint{0.887500in}{0.275000in}}{\pgfqpoint{4.225000in}{4.225000in}}%
\pgfusepath{clip}%
\pgfsetbuttcap%
\pgfsetroundjoin%
\definecolor{currentfill}{rgb}{0.535621,0.835785,0.281908}%
\pgfsetfillcolor{currentfill}%
\pgfsetfillopacity{0.700000}%
\pgfsetlinewidth{0.501875pt}%
\definecolor{currentstroke}{rgb}{1.000000,1.000000,1.000000}%
\pgfsetstrokecolor{currentstroke}%
\pgfsetstrokeopacity{0.500000}%
\pgfsetdash{}{0pt}%
\pgfpathmoveto{\pgfqpoint{2.767739in}{3.383070in}}%
\pgfpathlineto{\pgfqpoint{2.778821in}{3.400300in}}%
\pgfpathlineto{\pgfqpoint{2.789913in}{3.416211in}}%
\pgfpathlineto{\pgfqpoint{2.801016in}{3.430788in}}%
\pgfpathlineto{\pgfqpoint{2.812128in}{3.444093in}}%
\pgfpathlineto{\pgfqpoint{2.823246in}{3.456192in}}%
\pgfpathlineto{\pgfqpoint{2.817326in}{3.447586in}}%
\pgfpathlineto{\pgfqpoint{2.811419in}{3.438208in}}%
\pgfpathlineto{\pgfqpoint{2.805524in}{3.428383in}}%
\pgfpathlineto{\pgfqpoint{2.799639in}{3.418435in}}%
\pgfpathlineto{\pgfqpoint{2.793761in}{3.408686in}}%
\pgfpathlineto{\pgfqpoint{2.782691in}{3.395611in}}%
\pgfpathlineto{\pgfqpoint{2.771622in}{3.382336in}}%
\pgfpathlineto{\pgfqpoint{2.760555in}{3.368862in}}%
\pgfpathlineto{\pgfqpoint{2.749490in}{3.355185in}}%
\pgfpathlineto{\pgfqpoint{2.738427in}{3.341228in}}%
\pgfpathlineto{\pgfqpoint{2.744260in}{3.351196in}}%
\pgfpathlineto{\pgfqpoint{2.750104in}{3.360802in}}%
\pgfpathlineto{\pgfqpoint{2.755964in}{3.369609in}}%
\pgfpathlineto{\pgfqpoint{2.761841in}{3.377178in}}%
\pgfpathclose%
\pgfusepath{stroke,fill}%
\end{pgfscope}%
\begin{pgfscope}%
\pgfpathrectangle{\pgfqpoint{0.887500in}{0.275000in}}{\pgfqpoint{4.225000in}{4.225000in}}%
\pgfusepath{clip}%
\pgfsetbuttcap%
\pgfsetroundjoin%
\definecolor{currentfill}{rgb}{0.119699,0.618490,0.536347}%
\pgfsetfillcolor{currentfill}%
\pgfsetfillopacity{0.700000}%
\pgfsetlinewidth{0.501875pt}%
\definecolor{currentstroke}{rgb}{1.000000,1.000000,1.000000}%
\pgfsetstrokecolor{currentstroke}%
\pgfsetstrokeopacity{0.500000}%
\pgfsetdash{}{0pt}%
\pgfpathmoveto{\pgfqpoint{1.954344in}{2.855137in}}%
\pgfpathlineto{\pgfqpoint{1.965712in}{2.858487in}}%
\pgfpathlineto{\pgfqpoint{1.977073in}{2.861877in}}%
\pgfpathlineto{\pgfqpoint{1.988427in}{2.865332in}}%
\pgfpathlineto{\pgfqpoint{1.999773in}{2.868874in}}%
\pgfpathlineto{\pgfqpoint{2.011111in}{2.872527in}}%
\pgfpathlineto{\pgfqpoint{2.005377in}{2.880240in}}%
\pgfpathlineto{\pgfqpoint{1.999647in}{2.887941in}}%
\pgfpathlineto{\pgfqpoint{1.993920in}{2.895630in}}%
\pgfpathlineto{\pgfqpoint{1.988198in}{2.903308in}}%
\pgfpathlineto{\pgfqpoint{1.982480in}{2.910974in}}%
\pgfpathlineto{\pgfqpoint{1.971155in}{2.907308in}}%
\pgfpathlineto{\pgfqpoint{1.959822in}{2.903746in}}%
\pgfpathlineto{\pgfqpoint{1.948481in}{2.900265in}}%
\pgfpathlineto{\pgfqpoint{1.937134in}{2.896845in}}%
\pgfpathlineto{\pgfqpoint{1.925780in}{2.893462in}}%
\pgfpathlineto{\pgfqpoint{1.931485in}{2.885821in}}%
\pgfpathlineto{\pgfqpoint{1.937193in}{2.878169in}}%
\pgfpathlineto{\pgfqpoint{1.942906in}{2.870504in}}%
\pgfpathlineto{\pgfqpoint{1.948623in}{2.862827in}}%
\pgfpathclose%
\pgfusepath{stroke,fill}%
\end{pgfscope}%
\begin{pgfscope}%
\pgfpathrectangle{\pgfqpoint{0.887500in}{0.275000in}}{\pgfqpoint{4.225000in}{4.225000in}}%
\pgfusepath{clip}%
\pgfsetbuttcap%
\pgfsetroundjoin%
\definecolor{currentfill}{rgb}{0.741388,0.873449,0.149561}%
\pgfsetfillcolor{currentfill}%
\pgfsetfillopacity{0.700000}%
\pgfsetlinewidth{0.501875pt}%
\definecolor{currentstroke}{rgb}{1.000000,1.000000,1.000000}%
\pgfsetstrokecolor{currentstroke}%
\pgfsetstrokeopacity{0.500000}%
\pgfsetdash{}{0pt}%
\pgfpathmoveto{\pgfqpoint{3.020442in}{3.541429in}}%
\pgfpathlineto{\pgfqpoint{3.031561in}{3.545959in}}%
\pgfpathlineto{\pgfqpoint{3.042675in}{3.550609in}}%
\pgfpathlineto{\pgfqpoint{3.053785in}{3.555286in}}%
\pgfpathlineto{\pgfqpoint{3.064891in}{3.559965in}}%
\pgfpathlineto{\pgfqpoint{3.075992in}{3.564660in}}%
\pgfpathlineto{\pgfqpoint{3.069896in}{3.565082in}}%
\pgfpathlineto{\pgfqpoint{3.063806in}{3.565481in}}%
\pgfpathlineto{\pgfqpoint{3.057722in}{3.565912in}}%
\pgfpathlineto{\pgfqpoint{3.051645in}{3.566403in}}%
\pgfpathlineto{\pgfqpoint{3.045573in}{3.566955in}}%
\pgfpathlineto{\pgfqpoint{3.034494in}{3.561930in}}%
\pgfpathlineto{\pgfqpoint{3.023411in}{3.557200in}}%
\pgfpathlineto{\pgfqpoint{3.012323in}{3.552815in}}%
\pgfpathlineto{\pgfqpoint{3.001230in}{3.548782in}}%
\pgfpathlineto{\pgfqpoint{2.990131in}{3.545015in}}%
\pgfpathlineto{\pgfqpoint{2.996181in}{3.544436in}}%
\pgfpathlineto{\pgfqpoint{3.002237in}{3.543804in}}%
\pgfpathlineto{\pgfqpoint{3.008299in}{3.543103in}}%
\pgfpathlineto{\pgfqpoint{3.014367in}{3.542318in}}%
\pgfpathclose%
\pgfusepath{stroke,fill}%
\end{pgfscope}%
\begin{pgfscope}%
\pgfpathrectangle{\pgfqpoint{0.887500in}{0.275000in}}{\pgfqpoint{4.225000in}{4.225000in}}%
\pgfusepath{clip}%
\pgfsetbuttcap%
\pgfsetroundjoin%
\definecolor{currentfill}{rgb}{0.896320,0.893616,0.096335}%
\pgfsetfillcolor{currentfill}%
\pgfsetfillopacity{0.700000}%
\pgfsetlinewidth{0.501875pt}%
\definecolor{currentstroke}{rgb}{1.000000,1.000000,1.000000}%
\pgfsetstrokecolor{currentstroke}%
\pgfsetstrokeopacity{0.500000}%
\pgfsetdash{}{0pt}%
\pgfpathmoveto{\pgfqpoint{3.728170in}{3.633551in}}%
\pgfpathlineto{\pgfqpoint{3.739147in}{3.638557in}}%
\pgfpathlineto{\pgfqpoint{3.750118in}{3.643493in}}%
\pgfpathlineto{\pgfqpoint{3.761085in}{3.648364in}}%
\pgfpathlineto{\pgfqpoint{3.772046in}{3.653179in}}%
\pgfpathlineto{\pgfqpoint{3.783002in}{3.657945in}}%
\pgfpathlineto{\pgfqpoint{3.776708in}{3.667288in}}%
\pgfpathlineto{\pgfqpoint{3.770414in}{3.676407in}}%
\pgfpathlineto{\pgfqpoint{3.764120in}{3.685307in}}%
\pgfpathlineto{\pgfqpoint{3.757826in}{3.693995in}}%
\pgfpathlineto{\pgfqpoint{3.751533in}{3.702478in}}%
\pgfpathlineto{\pgfqpoint{3.740586in}{3.697639in}}%
\pgfpathlineto{\pgfqpoint{3.729633in}{3.692758in}}%
\pgfpathlineto{\pgfqpoint{3.718676in}{3.687834in}}%
\pgfpathlineto{\pgfqpoint{3.707714in}{3.682863in}}%
\pgfpathlineto{\pgfqpoint{3.696747in}{3.677843in}}%
\pgfpathlineto{\pgfqpoint{3.703030in}{3.669375in}}%
\pgfpathlineto{\pgfqpoint{3.709314in}{3.660720in}}%
\pgfpathlineto{\pgfqpoint{3.715599in}{3.651871in}}%
\pgfpathlineto{\pgfqpoint{3.721884in}{3.642817in}}%
\pgfpathclose%
\pgfusepath{stroke,fill}%
\end{pgfscope}%
\begin{pgfscope}%
\pgfpathrectangle{\pgfqpoint{0.887500in}{0.275000in}}{\pgfqpoint{4.225000in}{4.225000in}}%
\pgfusepath{clip}%
\pgfsetbuttcap%
\pgfsetroundjoin%
\definecolor{currentfill}{rgb}{0.132268,0.655014,0.519661}%
\pgfsetfillcolor{currentfill}%
\pgfsetfillopacity{0.700000}%
\pgfsetlinewidth{0.501875pt}%
\definecolor{currentstroke}{rgb}{1.000000,1.000000,1.000000}%
\pgfsetstrokecolor{currentstroke}%
\pgfsetstrokeopacity{0.500000}%
\pgfsetdash{}{0pt}%
\pgfpathmoveto{\pgfqpoint{2.379167in}{2.913796in}}%
\pgfpathlineto{\pgfqpoint{2.390346in}{2.922260in}}%
\pgfpathlineto{\pgfqpoint{2.401530in}{2.930291in}}%
\pgfpathlineto{\pgfqpoint{2.412717in}{2.937899in}}%
\pgfpathlineto{\pgfqpoint{2.423909in}{2.945080in}}%
\pgfpathlineto{\pgfqpoint{2.435104in}{2.951826in}}%
\pgfpathlineto{\pgfqpoint{2.429188in}{2.961950in}}%
\pgfpathlineto{\pgfqpoint{2.423272in}{2.972260in}}%
\pgfpathlineto{\pgfqpoint{2.417359in}{2.982639in}}%
\pgfpathlineto{\pgfqpoint{2.411449in}{2.992971in}}%
\pgfpathlineto{\pgfqpoint{2.405545in}{3.003140in}}%
\pgfpathlineto{\pgfqpoint{2.394389in}{2.994657in}}%
\pgfpathlineto{\pgfqpoint{2.383238in}{2.985787in}}%
\pgfpathlineto{\pgfqpoint{2.372089in}{2.976682in}}%
\pgfpathlineto{\pgfqpoint{2.360940in}{2.967417in}}%
\pgfpathlineto{\pgfqpoint{2.349793in}{2.957948in}}%
\pgfpathlineto{\pgfqpoint{2.355656in}{2.949368in}}%
\pgfpathlineto{\pgfqpoint{2.361526in}{2.940654in}}%
\pgfpathlineto{\pgfqpoint{2.367401in}{2.931814in}}%
\pgfpathlineto{\pgfqpoint{2.373281in}{2.922859in}}%
\pgfpathclose%
\pgfusepath{stroke,fill}%
\end{pgfscope}%
\begin{pgfscope}%
\pgfpathrectangle{\pgfqpoint{0.887500in}{0.275000in}}{\pgfqpoint{4.225000in}{4.225000in}}%
\pgfusepath{clip}%
\pgfsetbuttcap%
\pgfsetroundjoin%
\definecolor{currentfill}{rgb}{0.122312,0.633153,0.530398}%
\pgfsetfillcolor{currentfill}%
\pgfsetfillopacity{0.700000}%
\pgfsetlinewidth{0.501875pt}%
\definecolor{currentstroke}{rgb}{1.000000,1.000000,1.000000}%
\pgfsetstrokecolor{currentstroke}%
\pgfsetstrokeopacity{0.500000}%
\pgfsetdash{}{0pt}%
\pgfpathmoveto{\pgfqpoint{1.726577in}{2.879919in}}%
\pgfpathlineto{\pgfqpoint{1.738000in}{2.883296in}}%
\pgfpathlineto{\pgfqpoint{1.749416in}{2.886683in}}%
\pgfpathlineto{\pgfqpoint{1.760827in}{2.890083in}}%
\pgfpathlineto{\pgfqpoint{1.772232in}{2.893497in}}%
\pgfpathlineto{\pgfqpoint{1.783630in}{2.896927in}}%
\pgfpathlineto{\pgfqpoint{1.777979in}{2.904409in}}%
\pgfpathlineto{\pgfqpoint{1.772331in}{2.911874in}}%
\pgfpathlineto{\pgfqpoint{1.766687in}{2.919322in}}%
\pgfpathlineto{\pgfqpoint{1.761048in}{2.926753in}}%
\pgfpathlineto{\pgfqpoint{1.749660in}{2.923315in}}%
\pgfpathlineto{\pgfqpoint{1.738266in}{2.919896in}}%
\pgfpathlineto{\pgfqpoint{1.726866in}{2.916493in}}%
\pgfpathlineto{\pgfqpoint{1.715460in}{2.913104in}}%
\pgfpathlineto{\pgfqpoint{1.704049in}{2.909723in}}%
\pgfpathlineto{\pgfqpoint{1.709674in}{2.902293in}}%
\pgfpathlineto{\pgfqpoint{1.715305in}{2.894849in}}%
\pgfpathlineto{\pgfqpoint{1.720939in}{2.887391in}}%
\pgfpathclose%
\pgfusepath{stroke,fill}%
\end{pgfscope}%
\begin{pgfscope}%
\pgfpathrectangle{\pgfqpoint{0.887500in}{0.275000in}}{\pgfqpoint{4.225000in}{4.225000in}}%
\pgfusepath{clip}%
\pgfsetbuttcap%
\pgfsetroundjoin%
\definecolor{currentfill}{rgb}{0.886271,0.892374,0.095374}%
\pgfsetfillcolor{currentfill}%
\pgfsetfillopacity{0.700000}%
\pgfsetlinewidth{0.501875pt}%
\definecolor{currentstroke}{rgb}{1.000000,1.000000,1.000000}%
\pgfsetstrokecolor{currentstroke}%
\pgfsetstrokeopacity{0.500000}%
\pgfsetdash{}{0pt}%
\pgfpathmoveto{\pgfqpoint{3.586855in}{3.628276in}}%
\pgfpathlineto{\pgfqpoint{3.597860in}{3.632977in}}%
\pgfpathlineto{\pgfqpoint{3.608861in}{3.637735in}}%
\pgfpathlineto{\pgfqpoint{3.619859in}{3.642564in}}%
\pgfpathlineto{\pgfqpoint{3.630854in}{3.647475in}}%
\pgfpathlineto{\pgfqpoint{3.641845in}{3.652464in}}%
\pgfpathlineto{\pgfqpoint{3.635574in}{3.660707in}}%
\pgfpathlineto{\pgfqpoint{3.629304in}{3.668806in}}%
\pgfpathlineto{\pgfqpoint{3.623037in}{3.676765in}}%
\pgfpathlineto{\pgfqpoint{3.616772in}{3.684573in}}%
\pgfpathlineto{\pgfqpoint{3.610508in}{3.692218in}}%
\pgfpathlineto{\pgfqpoint{3.599526in}{3.687070in}}%
\pgfpathlineto{\pgfqpoint{3.588540in}{3.681910in}}%
\pgfpathlineto{\pgfqpoint{3.577549in}{3.676741in}}%
\pgfpathlineto{\pgfqpoint{3.566554in}{3.671568in}}%
\pgfpathlineto{\pgfqpoint{3.555555in}{3.666393in}}%
\pgfpathlineto{\pgfqpoint{3.561808in}{3.658871in}}%
\pgfpathlineto{\pgfqpoint{3.568064in}{3.651275in}}%
\pgfpathlineto{\pgfqpoint{3.574324in}{3.643632in}}%
\pgfpathlineto{\pgfqpoint{3.580588in}{3.635969in}}%
\pgfpathclose%
\pgfusepath{stroke,fill}%
\end{pgfscope}%
\begin{pgfscope}%
\pgfpathrectangle{\pgfqpoint{0.887500in}{0.275000in}}{\pgfqpoint{4.225000in}{4.225000in}}%
\pgfusepath{clip}%
\pgfsetbuttcap%
\pgfsetroundjoin%
\definecolor{currentfill}{rgb}{0.119483,0.614817,0.537692}%
\pgfsetfillcolor{currentfill}%
\pgfsetfillopacity{0.700000}%
\pgfsetlinewidth{0.501875pt}%
\definecolor{currentstroke}{rgb}{1.000000,1.000000,1.000000}%
\pgfsetstrokecolor{currentstroke}%
\pgfsetstrokeopacity{0.500000}%
\pgfsetdash{}{0pt}%
\pgfpathmoveto{\pgfqpoint{2.039843in}{2.833780in}}%
\pgfpathlineto{\pgfqpoint{2.051186in}{2.837541in}}%
\pgfpathlineto{\pgfqpoint{2.062522in}{2.841363in}}%
\pgfpathlineto{\pgfqpoint{2.073854in}{2.845173in}}%
\pgfpathlineto{\pgfqpoint{2.085182in}{2.848901in}}%
\pgfpathlineto{\pgfqpoint{2.096509in}{2.852473in}}%
\pgfpathlineto{\pgfqpoint{2.090736in}{2.860466in}}%
\pgfpathlineto{\pgfqpoint{2.084967in}{2.868472in}}%
\pgfpathlineto{\pgfqpoint{2.079201in}{2.876486in}}%
\pgfpathlineto{\pgfqpoint{2.073439in}{2.884506in}}%
\pgfpathlineto{\pgfqpoint{2.067681in}{2.892528in}}%
\pgfpathlineto{\pgfqpoint{2.056382in}{2.888329in}}%
\pgfpathlineto{\pgfqpoint{2.045076in}{2.884219in}}%
\pgfpathlineto{\pgfqpoint{2.033762in}{2.880208in}}%
\pgfpathlineto{\pgfqpoint{2.022441in}{2.876307in}}%
\pgfpathlineto{\pgfqpoint{2.011111in}{2.872527in}}%
\pgfpathlineto{\pgfqpoint{2.016850in}{2.864802in}}%
\pgfpathlineto{\pgfqpoint{2.022592in}{2.857065in}}%
\pgfpathlineto{\pgfqpoint{2.028338in}{2.849316in}}%
\pgfpathlineto{\pgfqpoint{2.034089in}{2.841554in}}%
\pgfpathclose%
\pgfusepath{stroke,fill}%
\end{pgfscope}%
\begin{pgfscope}%
\pgfpathrectangle{\pgfqpoint{0.887500in}{0.275000in}}{\pgfqpoint{4.225000in}{4.225000in}}%
\pgfusepath{clip}%
\pgfsetbuttcap%
\pgfsetroundjoin%
\definecolor{currentfill}{rgb}{0.876168,0.891125,0.095250}%
\pgfsetfillcolor{currentfill}%
\pgfsetfillopacity{0.700000}%
\pgfsetlinewidth{0.501875pt}%
\definecolor{currentstroke}{rgb}{1.000000,1.000000,1.000000}%
\pgfsetstrokecolor{currentstroke}%
\pgfsetstrokeopacity{0.500000}%
\pgfsetdash{}{0pt}%
\pgfpathmoveto{\pgfqpoint{3.814460in}{3.607626in}}%
\pgfpathlineto{\pgfqpoint{3.825416in}{3.612209in}}%
\pgfpathlineto{\pgfqpoint{3.836367in}{3.616752in}}%
\pgfpathlineto{\pgfqpoint{3.847314in}{3.621266in}}%
\pgfpathlineto{\pgfqpoint{3.858255in}{3.625760in}}%
\pgfpathlineto{\pgfqpoint{3.869192in}{3.630246in}}%
\pgfpathlineto{\pgfqpoint{3.862899in}{3.640970in}}%
\pgfpathlineto{\pgfqpoint{3.856603in}{3.651436in}}%
\pgfpathlineto{\pgfqpoint{3.850306in}{3.661643in}}%
\pgfpathlineto{\pgfqpoint{3.844008in}{3.671595in}}%
\pgfpathlineto{\pgfqpoint{3.837708in}{3.681295in}}%
\pgfpathlineto{\pgfqpoint{3.826776in}{3.676664in}}%
\pgfpathlineto{\pgfqpoint{3.815840in}{3.672021in}}%
\pgfpathlineto{\pgfqpoint{3.804899in}{3.667358in}}%
\pgfpathlineto{\pgfqpoint{3.793953in}{3.662668in}}%
\pgfpathlineto{\pgfqpoint{3.783002in}{3.657945in}}%
\pgfpathlineto{\pgfqpoint{3.789295in}{3.648369in}}%
\pgfpathlineto{\pgfqpoint{3.795588in}{3.638556in}}%
\pgfpathlineto{\pgfqpoint{3.801880in}{3.628498in}}%
\pgfpathlineto{\pgfqpoint{3.808170in}{3.618188in}}%
\pgfpathclose%
\pgfusepath{stroke,fill}%
\end{pgfscope}%
\begin{pgfscope}%
\pgfpathrectangle{\pgfqpoint{0.887500in}{0.275000in}}{\pgfqpoint{4.225000in}{4.225000in}}%
\pgfusepath{clip}%
\pgfsetbuttcap%
\pgfsetroundjoin%
\definecolor{currentfill}{rgb}{0.804182,0.882046,0.114965}%
\pgfsetfillcolor{currentfill}%
\pgfsetfillopacity{0.700000}%
\pgfsetlinewidth{0.501875pt}%
\definecolor{currentstroke}{rgb}{1.000000,1.000000,1.000000}%
\pgfsetstrokecolor{currentstroke}%
\pgfsetstrokeopacity{0.500000}%
\pgfsetdash{}{0pt}%
\pgfpathmoveto{\pgfqpoint{3.162098in}{3.579296in}}%
\pgfpathlineto{\pgfqpoint{3.173190in}{3.583078in}}%
\pgfpathlineto{\pgfqpoint{3.184277in}{3.586923in}}%
\pgfpathlineto{\pgfqpoint{3.195360in}{3.590833in}}%
\pgfpathlineto{\pgfqpoint{3.206438in}{3.594810in}}%
\pgfpathlineto{\pgfqpoint{3.217512in}{3.598856in}}%
\pgfpathlineto{\pgfqpoint{3.211353in}{3.602252in}}%
\pgfpathlineto{\pgfqpoint{3.205199in}{3.605462in}}%
\pgfpathlineto{\pgfqpoint{3.199050in}{3.608471in}}%
\pgfpathlineto{\pgfqpoint{3.192906in}{3.611265in}}%
\pgfpathlineto{\pgfqpoint{3.186767in}{3.613828in}}%
\pgfpathlineto{\pgfqpoint{3.175708in}{3.608893in}}%
\pgfpathlineto{\pgfqpoint{3.164646in}{3.603895in}}%
\pgfpathlineto{\pgfqpoint{3.153579in}{3.598862in}}%
\pgfpathlineto{\pgfqpoint{3.142508in}{3.593824in}}%
\pgfpathlineto{\pgfqpoint{3.131432in}{3.588810in}}%
\pgfpathlineto{\pgfqpoint{3.137554in}{3.587282in}}%
\pgfpathlineto{\pgfqpoint{3.143682in}{3.585589in}}%
\pgfpathlineto{\pgfqpoint{3.149816in}{3.583708in}}%
\pgfpathlineto{\pgfqpoint{3.155954in}{3.581618in}}%
\pgfpathclose%
\pgfusepath{stroke,fill}%
\end{pgfscope}%
\begin{pgfscope}%
\pgfpathrectangle{\pgfqpoint{0.887500in}{0.275000in}}{\pgfqpoint{4.225000in}{4.225000in}}%
\pgfusepath{clip}%
\pgfsetbuttcap%
\pgfsetroundjoin%
\definecolor{currentfill}{rgb}{0.123444,0.636809,0.528763}%
\pgfsetfillcolor{currentfill}%
\pgfsetfillopacity{0.700000}%
\pgfsetlinewidth{0.501875pt}%
\definecolor{currentstroke}{rgb}{1.000000,1.000000,1.000000}%
\pgfsetstrokecolor{currentstroke}%
\pgfsetstrokeopacity{0.500000}%
\pgfsetdash{}{0pt}%
\pgfpathmoveto{\pgfqpoint{2.323361in}{2.864679in}}%
\pgfpathlineto{\pgfqpoint{2.334509in}{2.875436in}}%
\pgfpathlineto{\pgfqpoint{2.345664in}{2.885720in}}%
\pgfpathlineto{\pgfqpoint{2.356826in}{2.895535in}}%
\pgfpathlineto{\pgfqpoint{2.367994in}{2.904891in}}%
\pgfpathlineto{\pgfqpoint{2.379167in}{2.913796in}}%
\pgfpathlineto{\pgfqpoint{2.373281in}{2.922859in}}%
\pgfpathlineto{\pgfqpoint{2.367401in}{2.931814in}}%
\pgfpathlineto{\pgfqpoint{2.361526in}{2.940654in}}%
\pgfpathlineto{\pgfqpoint{2.355656in}{2.949368in}}%
\pgfpathlineto{\pgfqpoint{2.349793in}{2.957948in}}%
\pgfpathlineto{\pgfqpoint{2.338648in}{2.948222in}}%
\pgfpathlineto{\pgfqpoint{2.327508in}{2.938186in}}%
\pgfpathlineto{\pgfqpoint{2.316372in}{2.927786in}}%
\pgfpathlineto{\pgfqpoint{2.305243in}{2.916970in}}%
\pgfpathlineto{\pgfqpoint{2.294122in}{2.905692in}}%
\pgfpathlineto{\pgfqpoint{2.299962in}{2.897541in}}%
\pgfpathlineto{\pgfqpoint{2.305805in}{2.889397in}}%
\pgfpathlineto{\pgfqpoint{2.311652in}{2.881228in}}%
\pgfpathlineto{\pgfqpoint{2.317504in}{2.872999in}}%
\pgfpathclose%
\pgfusepath{stroke,fill}%
\end{pgfscope}%
\begin{pgfscope}%
\pgfpathrectangle{\pgfqpoint{0.887500in}{0.275000in}}{\pgfqpoint{4.225000in}{4.225000in}}%
\pgfusepath{clip}%
\pgfsetbuttcap%
\pgfsetroundjoin%
\definecolor{currentfill}{rgb}{0.845561,0.887322,0.099702}%
\pgfsetfillcolor{currentfill}%
\pgfsetfillopacity{0.700000}%
\pgfsetlinewidth{0.501875pt}%
\definecolor{currentstroke}{rgb}{1.000000,1.000000,1.000000}%
\pgfsetstrokecolor{currentstroke}%
\pgfsetstrokeopacity{0.500000}%
\pgfsetdash{}{0pt}%
\pgfpathmoveto{\pgfqpoint{3.303740in}{3.597616in}}%
\pgfpathlineto{\pgfqpoint{3.314802in}{3.601709in}}%
\pgfpathlineto{\pgfqpoint{3.325860in}{3.606035in}}%
\pgfpathlineto{\pgfqpoint{3.336915in}{3.610628in}}%
\pgfpathlineto{\pgfqpoint{3.347968in}{3.615497in}}%
\pgfpathlineto{\pgfqpoint{3.359019in}{3.620602in}}%
\pgfpathlineto{\pgfqpoint{3.352816in}{3.626337in}}%
\pgfpathlineto{\pgfqpoint{3.346616in}{3.631779in}}%
\pgfpathlineto{\pgfqpoint{3.340418in}{3.636883in}}%
\pgfpathlineto{\pgfqpoint{3.334223in}{3.641645in}}%
\pgfpathlineto{\pgfqpoint{3.328031in}{3.646079in}}%
\pgfpathlineto{\pgfqpoint{3.316993in}{3.640543in}}%
\pgfpathlineto{\pgfqpoint{3.305953in}{3.635193in}}%
\pgfpathlineto{\pgfqpoint{3.294910in}{3.630064in}}%
\pgfpathlineto{\pgfqpoint{3.283864in}{3.625149in}}%
\pgfpathlineto{\pgfqpoint{3.272815in}{3.620426in}}%
\pgfpathlineto{\pgfqpoint{3.278992in}{3.616300in}}%
\pgfpathlineto{\pgfqpoint{3.285174in}{3.611948in}}%
\pgfpathlineto{\pgfqpoint{3.291359in}{3.607375in}}%
\pgfpathlineto{\pgfqpoint{3.297548in}{3.602590in}}%
\pgfpathclose%
\pgfusepath{stroke,fill}%
\end{pgfscope}%
\begin{pgfscope}%
\pgfpathrectangle{\pgfqpoint{0.887500in}{0.275000in}}{\pgfqpoint{4.225000in}{4.225000in}}%
\pgfusepath{clip}%
\pgfsetbuttcap%
\pgfsetroundjoin%
\definecolor{currentfill}{rgb}{0.876168,0.891125,0.095250}%
\pgfsetfillcolor{currentfill}%
\pgfsetfillopacity{0.700000}%
\pgfsetlinewidth{0.501875pt}%
\definecolor{currentstroke}{rgb}{1.000000,1.000000,1.000000}%
\pgfsetstrokecolor{currentstroke}%
\pgfsetstrokeopacity{0.500000}%
\pgfsetdash{}{0pt}%
\pgfpathmoveto{\pgfqpoint{3.445341in}{3.614780in}}%
\pgfpathlineto{\pgfqpoint{3.456381in}{3.619980in}}%
\pgfpathlineto{\pgfqpoint{3.467417in}{3.625155in}}%
\pgfpathlineto{\pgfqpoint{3.478449in}{3.630312in}}%
\pgfpathlineto{\pgfqpoint{3.489476in}{3.635457in}}%
\pgfpathlineto{\pgfqpoint{3.500500in}{3.640597in}}%
\pgfpathlineto{\pgfqpoint{3.494260in}{3.647790in}}%
\pgfpathlineto{\pgfqpoint{3.488023in}{3.654912in}}%
\pgfpathlineto{\pgfqpoint{3.481790in}{3.661915in}}%
\pgfpathlineto{\pgfqpoint{3.475560in}{3.668751in}}%
\pgfpathlineto{\pgfqpoint{3.469331in}{3.675372in}}%
\pgfpathlineto{\pgfqpoint{3.458318in}{3.669970in}}%
\pgfpathlineto{\pgfqpoint{3.447301in}{3.664532in}}%
\pgfpathlineto{\pgfqpoint{3.436280in}{3.659059in}}%
\pgfpathlineto{\pgfqpoint{3.425254in}{3.653550in}}%
\pgfpathlineto{\pgfqpoint{3.414224in}{3.648008in}}%
\pgfpathlineto{\pgfqpoint{3.420441in}{3.641652in}}%
\pgfpathlineto{\pgfqpoint{3.426661in}{3.635096in}}%
\pgfpathlineto{\pgfqpoint{3.432884in}{3.628396in}}%
\pgfpathlineto{\pgfqpoint{3.439110in}{3.621605in}}%
\pgfpathclose%
\pgfusepath{stroke,fill}%
\end{pgfscope}%
\begin{pgfscope}%
\pgfpathrectangle{\pgfqpoint{0.887500in}{0.275000in}}{\pgfqpoint{4.225000in}{4.225000in}}%
\pgfusepath{clip}%
\pgfsetbuttcap%
\pgfsetroundjoin%
\definecolor{currentfill}{rgb}{0.120638,0.625828,0.533488}%
\pgfsetfillcolor{currentfill}%
\pgfsetfillopacity{0.700000}%
\pgfsetlinewidth{0.501875pt}%
\definecolor{currentstroke}{rgb}{1.000000,1.000000,1.000000}%
\pgfsetstrokecolor{currentstroke}%
\pgfsetstrokeopacity{0.500000}%
\pgfsetdash{}{0pt}%
\pgfpathmoveto{\pgfqpoint{1.811953in}{2.859275in}}%
\pgfpathlineto{\pgfqpoint{1.823360in}{2.862699in}}%
\pgfpathlineto{\pgfqpoint{1.834761in}{2.866135in}}%
\pgfpathlineto{\pgfqpoint{1.846156in}{2.869580in}}%
\pgfpathlineto{\pgfqpoint{1.857546in}{2.873027in}}%
\pgfpathlineto{\pgfqpoint{1.868931in}{2.876470in}}%
\pgfpathlineto{\pgfqpoint{1.863244in}{2.884061in}}%
\pgfpathlineto{\pgfqpoint{1.857561in}{2.891637in}}%
\pgfpathlineto{\pgfqpoint{1.851883in}{2.899199in}}%
\pgfpathlineto{\pgfqpoint{1.846208in}{2.906747in}}%
\pgfpathlineto{\pgfqpoint{1.840538in}{2.914279in}}%
\pgfpathlineto{\pgfqpoint{1.829168in}{2.910797in}}%
\pgfpathlineto{\pgfqpoint{1.817792in}{2.907314in}}%
\pgfpathlineto{\pgfqpoint{1.806410in}{2.903837in}}%
\pgfpathlineto{\pgfqpoint{1.795023in}{2.900373in}}%
\pgfpathlineto{\pgfqpoint{1.783630in}{2.896927in}}%
\pgfpathlineto{\pgfqpoint{1.789287in}{2.889428in}}%
\pgfpathlineto{\pgfqpoint{1.794947in}{2.881914in}}%
\pgfpathlineto{\pgfqpoint{1.800612in}{2.874383in}}%
\pgfpathlineto{\pgfqpoint{1.806280in}{2.866837in}}%
\pgfpathclose%
\pgfusepath{stroke,fill}%
\end{pgfscope}%
\begin{pgfscope}%
\pgfpathrectangle{\pgfqpoint{0.887500in}{0.275000in}}{\pgfqpoint{4.225000in}{4.225000in}}%
\pgfusepath{clip}%
\pgfsetbuttcap%
\pgfsetroundjoin%
\definecolor{currentfill}{rgb}{0.304148,0.764704,0.419943}%
\pgfsetfillcolor{currentfill}%
\pgfsetfillopacity{0.700000}%
\pgfsetlinewidth{0.501875pt}%
\definecolor{currentstroke}{rgb}{1.000000,1.000000,1.000000}%
\pgfsetstrokecolor{currentstroke}%
\pgfsetstrokeopacity{0.500000}%
\pgfsetdash{}{0pt}%
\pgfpathmoveto{\pgfqpoint{2.687467in}{3.108579in}}%
\pgfpathlineto{\pgfqpoint{2.698509in}{3.127958in}}%
\pgfpathlineto{\pgfqpoint{2.709542in}{3.148824in}}%
\pgfpathlineto{\pgfqpoint{2.720565in}{3.171452in}}%
\pgfpathlineto{\pgfqpoint{2.731577in}{3.196117in}}%
\pgfpathlineto{\pgfqpoint{2.742577in}{3.223096in}}%
\pgfpathlineto{\pgfqpoint{2.736504in}{3.241313in}}%
\pgfpathlineto{\pgfqpoint{2.730456in}{3.256498in}}%
\pgfpathlineto{\pgfqpoint{2.724441in}{3.267923in}}%
\pgfpathlineto{\pgfqpoint{2.718463in}{3.275384in}}%
\pgfpathlineto{\pgfqpoint{2.712518in}{3.279322in}}%
\pgfpathlineto{\pgfqpoint{2.701513in}{3.255525in}}%
\pgfpathlineto{\pgfqpoint{2.690521in}{3.231118in}}%
\pgfpathlineto{\pgfqpoint{2.679537in}{3.206497in}}%
\pgfpathlineto{\pgfqpoint{2.668557in}{3.182059in}}%
\pgfpathlineto{\pgfqpoint{2.657579in}{3.158198in}}%
\pgfpathlineto{\pgfqpoint{2.663512in}{3.152158in}}%
\pgfpathlineto{\pgfqpoint{2.669468in}{3.144231in}}%
\pgfpathlineto{\pgfqpoint{2.675447in}{3.134197in}}%
\pgfpathlineto{\pgfqpoint{2.681449in}{3.122179in}}%
\pgfpathclose%
\pgfusepath{stroke,fill}%
\end{pgfscope}%
\begin{pgfscope}%
\pgfpathrectangle{\pgfqpoint{0.887500in}{0.275000in}}{\pgfqpoint{4.225000in}{4.225000in}}%
\pgfusepath{clip}%
\pgfsetbuttcap%
\pgfsetroundjoin%
\definecolor{currentfill}{rgb}{0.132268,0.655014,0.519661}%
\pgfsetfillcolor{currentfill}%
\pgfsetfillopacity{0.700000}%
\pgfsetlinewidth{0.501875pt}%
\definecolor{currentstroke}{rgb}{1.000000,1.000000,1.000000}%
\pgfsetstrokecolor{currentstroke}%
\pgfsetstrokeopacity{0.500000}%
\pgfsetdash{}{0pt}%
\pgfpathmoveto{\pgfqpoint{2.464639in}{2.907365in}}%
\pgfpathlineto{\pgfqpoint{2.475836in}{2.914639in}}%
\pgfpathlineto{\pgfqpoint{2.487034in}{2.921560in}}%
\pgfpathlineto{\pgfqpoint{2.498235in}{2.928081in}}%
\pgfpathlineto{\pgfqpoint{2.509438in}{2.934157in}}%
\pgfpathlineto{\pgfqpoint{2.520644in}{2.939774in}}%
\pgfpathlineto{\pgfqpoint{2.514759in}{2.945609in}}%
\pgfpathlineto{\pgfqpoint{2.508866in}{2.952293in}}%
\pgfpathlineto{\pgfqpoint{2.502963in}{2.960052in}}%
\pgfpathlineto{\pgfqpoint{2.497048in}{2.968924in}}%
\pgfpathlineto{\pgfqpoint{2.491123in}{2.978718in}}%
\pgfpathlineto{\pgfqpoint{2.479915in}{2.974264in}}%
\pgfpathlineto{\pgfqpoint{2.468708in}{2.969355in}}%
\pgfpathlineto{\pgfqpoint{2.457503in}{2.963973in}}%
\pgfpathlineto{\pgfqpoint{2.446302in}{2.958127in}}%
\pgfpathlineto{\pgfqpoint{2.435104in}{2.951826in}}%
\pgfpathlineto{\pgfqpoint{2.441018in}{2.942005in}}%
\pgfpathlineto{\pgfqpoint{2.446929in}{2.932604in}}%
\pgfpathlineto{\pgfqpoint{2.452836in}{2.923731in}}%
\pgfpathlineto{\pgfqpoint{2.458738in}{2.915362in}}%
\pgfpathclose%
\pgfusepath{stroke,fill}%
\end{pgfscope}%
\begin{pgfscope}%
\pgfpathrectangle{\pgfqpoint{0.887500in}{0.275000in}}{\pgfqpoint{4.225000in}{4.225000in}}%
\pgfusepath{clip}%
\pgfsetbuttcap%
\pgfsetroundjoin%
\definecolor{currentfill}{rgb}{0.119483,0.614817,0.537692}%
\pgfsetfillcolor{currentfill}%
\pgfsetfillopacity{0.700000}%
\pgfsetlinewidth{0.501875pt}%
\definecolor{currentstroke}{rgb}{1.000000,1.000000,1.000000}%
\pgfsetstrokecolor{currentstroke}%
\pgfsetstrokeopacity{0.500000}%
\pgfsetdash{}{0pt}%
\pgfpathmoveto{\pgfqpoint{2.267577in}{2.812272in}}%
\pgfpathlineto{\pgfqpoint{2.278760in}{2.821527in}}%
\pgfpathlineto{\pgfqpoint{2.289924in}{2.831709in}}%
\pgfpathlineto{\pgfqpoint{2.301074in}{2.842505in}}%
\pgfpathlineto{\pgfqpoint{2.312218in}{2.853600in}}%
\pgfpathlineto{\pgfqpoint{2.323361in}{2.864679in}}%
\pgfpathlineto{\pgfqpoint{2.317504in}{2.872999in}}%
\pgfpathlineto{\pgfqpoint{2.311652in}{2.881228in}}%
\pgfpathlineto{\pgfqpoint{2.305805in}{2.889397in}}%
\pgfpathlineto{\pgfqpoint{2.299962in}{2.897541in}}%
\pgfpathlineto{\pgfqpoint{2.294122in}{2.905692in}}%
\pgfpathlineto{\pgfqpoint{2.283006in}{2.894088in}}%
\pgfpathlineto{\pgfqpoint{2.271890in}{2.882486in}}%
\pgfpathlineto{\pgfqpoint{2.260766in}{2.871222in}}%
\pgfpathlineto{\pgfqpoint{2.249627in}{2.860629in}}%
\pgfpathlineto{\pgfqpoint{2.238467in}{2.851045in}}%
\pgfpathlineto{\pgfqpoint{2.244287in}{2.843021in}}%
\pgfpathlineto{\pgfqpoint{2.250107in}{2.835184in}}%
\pgfpathlineto{\pgfqpoint{2.255928in}{2.827481in}}%
\pgfpathlineto{\pgfqpoint{2.261751in}{2.819861in}}%
\pgfpathclose%
\pgfusepath{stroke,fill}%
\end{pgfscope}%
\begin{pgfscope}%
\pgfpathrectangle{\pgfqpoint{0.887500in}{0.275000in}}{\pgfqpoint{4.225000in}{4.225000in}}%
\pgfusepath{clip}%
\pgfsetbuttcap%
\pgfsetroundjoin%
\definecolor{currentfill}{rgb}{0.835270,0.886029,0.102646}%
\pgfsetfillcolor{currentfill}%
\pgfsetfillopacity{0.700000}%
\pgfsetlinewidth{0.501875pt}%
\definecolor{currentstroke}{rgb}{1.000000,1.000000,1.000000}%
\pgfsetstrokecolor{currentstroke}%
\pgfsetstrokeopacity{0.500000}%
\pgfsetdash{}{0pt}%
\pgfpathmoveto{\pgfqpoint{3.900643in}{3.573407in}}%
\pgfpathlineto{\pgfqpoint{3.911579in}{3.577715in}}%
\pgfpathlineto{\pgfqpoint{3.922509in}{3.582031in}}%
\pgfpathlineto{\pgfqpoint{3.933436in}{3.586366in}}%
\pgfpathlineto{\pgfqpoint{3.944358in}{3.590730in}}%
\pgfpathlineto{\pgfqpoint{3.938068in}{3.602600in}}%
\pgfpathlineto{\pgfqpoint{3.931777in}{3.614319in}}%
\pgfpathlineto{\pgfqpoint{3.925486in}{3.625863in}}%
\pgfpathlineto{\pgfqpoint{3.919193in}{3.637204in}}%
\pgfpathlineto{\pgfqpoint{3.912899in}{3.648318in}}%
\pgfpathlineto{\pgfqpoint{3.901978in}{3.643760in}}%
\pgfpathlineto{\pgfqpoint{3.891053in}{3.639235in}}%
\pgfpathlineto{\pgfqpoint{3.880125in}{3.634734in}}%
\pgfpathlineto{\pgfqpoint{3.869192in}{3.630246in}}%
\pgfpathlineto{\pgfqpoint{3.875484in}{3.619282in}}%
\pgfpathlineto{\pgfqpoint{3.881775in}{3.608097in}}%
\pgfpathlineto{\pgfqpoint{3.888065in}{3.596710in}}%
\pgfpathlineto{\pgfqpoint{3.894355in}{3.585140in}}%
\pgfpathclose%
\pgfusepath{stroke,fill}%
\end{pgfscope}%
\begin{pgfscope}%
\pgfpathrectangle{\pgfqpoint{0.887500in}{0.275000in}}{\pgfqpoint{4.225000in}{4.225000in}}%
\pgfusepath{clip}%
\pgfsetbuttcap%
\pgfsetroundjoin%
\definecolor{currentfill}{rgb}{0.119738,0.603785,0.541400}%
\pgfsetfillcolor{currentfill}%
\pgfsetfillopacity{0.700000}%
\pgfsetlinewidth{0.501875pt}%
\definecolor{currentstroke}{rgb}{1.000000,1.000000,1.000000}%
\pgfsetstrokecolor{currentstroke}%
\pgfsetstrokeopacity{0.500000}%
\pgfsetdash{}{0pt}%
\pgfpathmoveto{\pgfqpoint{2.125419in}{2.812809in}}%
\pgfpathlineto{\pgfqpoint{2.136767in}{2.815765in}}%
\pgfpathlineto{\pgfqpoint{2.148119in}{2.818276in}}%
\pgfpathlineto{\pgfqpoint{2.159475in}{2.820381in}}%
\pgfpathlineto{\pgfqpoint{2.170827in}{2.822388in}}%
\pgfpathlineto{\pgfqpoint{2.182167in}{2.824637in}}%
\pgfpathlineto{\pgfqpoint{2.176362in}{2.832749in}}%
\pgfpathlineto{\pgfqpoint{2.170556in}{2.841063in}}%
\pgfpathlineto{\pgfqpoint{2.164749in}{2.849609in}}%
\pgfpathlineto{\pgfqpoint{2.158940in}{2.858411in}}%
\pgfpathlineto{\pgfqpoint{2.153128in}{2.867450in}}%
\pgfpathlineto{\pgfqpoint{2.141816in}{2.864424in}}%
\pgfpathlineto{\pgfqpoint{2.130493in}{2.861654in}}%
\pgfpathlineto{\pgfqpoint{2.119164in}{2.858868in}}%
\pgfpathlineto{\pgfqpoint{2.107835in}{2.855819in}}%
\pgfpathlineto{\pgfqpoint{2.096509in}{2.852473in}}%
\pgfpathlineto{\pgfqpoint{2.102284in}{2.844498in}}%
\pgfpathlineto{\pgfqpoint{2.108063in}{2.836544in}}%
\pgfpathlineto{\pgfqpoint{2.113845in}{2.828614in}}%
\pgfpathlineto{\pgfqpoint{2.119630in}{2.820704in}}%
\pgfpathclose%
\pgfusepath{stroke,fill}%
\end{pgfscope}%
\begin{pgfscope}%
\pgfpathrectangle{\pgfqpoint{0.887500in}{0.275000in}}{\pgfqpoint{4.225000in}{4.225000in}}%
\pgfusepath{clip}%
\pgfsetbuttcap%
\pgfsetroundjoin%
\definecolor{currentfill}{rgb}{0.876168,0.891125,0.095250}%
\pgfsetfillcolor{currentfill}%
\pgfsetfillopacity{0.700000}%
\pgfsetlinewidth{0.501875pt}%
\definecolor{currentstroke}{rgb}{1.000000,1.000000,1.000000}%
\pgfsetstrokecolor{currentstroke}%
\pgfsetstrokeopacity{0.500000}%
\pgfsetdash{}{0pt}%
\pgfpathmoveto{\pgfqpoint{3.673224in}{3.608653in}}%
\pgfpathlineto{\pgfqpoint{3.684219in}{3.613441in}}%
\pgfpathlineto{\pgfqpoint{3.695211in}{3.618378in}}%
\pgfpathlineto{\pgfqpoint{3.706201in}{3.623410in}}%
\pgfpathlineto{\pgfqpoint{3.717188in}{3.628485in}}%
\pgfpathlineto{\pgfqpoint{3.728170in}{3.633551in}}%
\pgfpathlineto{\pgfqpoint{3.721884in}{3.642817in}}%
\pgfpathlineto{\pgfqpoint{3.715599in}{3.651871in}}%
\pgfpathlineto{\pgfqpoint{3.709314in}{3.660720in}}%
\pgfpathlineto{\pgfqpoint{3.703030in}{3.669375in}}%
\pgfpathlineto{\pgfqpoint{3.696747in}{3.677843in}}%
\pgfpathlineto{\pgfqpoint{3.685775in}{3.672777in}}%
\pgfpathlineto{\pgfqpoint{3.674799in}{3.667686in}}%
\pgfpathlineto{\pgfqpoint{3.663818in}{3.662590in}}%
\pgfpathlineto{\pgfqpoint{3.652833in}{3.657509in}}%
\pgfpathlineto{\pgfqpoint{3.641845in}{3.652464in}}%
\pgfpathlineto{\pgfqpoint{3.648118in}{3.644064in}}%
\pgfpathlineto{\pgfqpoint{3.654393in}{3.635496in}}%
\pgfpathlineto{\pgfqpoint{3.660669in}{3.626747in}}%
\pgfpathlineto{\pgfqpoint{3.666946in}{3.617803in}}%
\pgfpathclose%
\pgfusepath{stroke,fill}%
\end{pgfscope}%
\begin{pgfscope}%
\pgfpathrectangle{\pgfqpoint{0.887500in}{0.275000in}}{\pgfqpoint{4.225000in}{4.225000in}}%
\pgfusepath{clip}%
\pgfsetbuttcap%
\pgfsetroundjoin%
\definecolor{currentfill}{rgb}{0.180653,0.701402,0.488189}%
\pgfsetfillcolor{currentfill}%
\pgfsetfillopacity{0.700000}%
\pgfsetlinewidth{0.501875pt}%
\definecolor{currentstroke}{rgb}{1.000000,1.000000,1.000000}%
\pgfsetstrokecolor{currentstroke}%
\pgfsetstrokeopacity{0.500000}%
\pgfsetdash{}{0pt}%
\pgfpathmoveto{\pgfqpoint{2.661884in}{2.992115in}}%
\pgfpathlineto{\pgfqpoint{2.673031in}{3.001592in}}%
\pgfpathlineto{\pgfqpoint{2.684183in}{3.010368in}}%
\pgfpathlineto{\pgfqpoint{2.695343in}{3.018063in}}%
\pgfpathlineto{\pgfqpoint{2.706509in}{3.024653in}}%
\pgfpathlineto{\pgfqpoint{2.717672in}{3.031269in}}%
\pgfpathlineto{\pgfqpoint{2.711629in}{3.046603in}}%
\pgfpathlineto{\pgfqpoint{2.705583in}{3.062411in}}%
\pgfpathlineto{\pgfqpoint{2.699539in}{3.078284in}}%
\pgfpathlineto{\pgfqpoint{2.693499in}{3.093811in}}%
\pgfpathlineto{\pgfqpoint{2.687467in}{3.108579in}}%
\pgfpathlineto{\pgfqpoint{2.676417in}{3.090413in}}%
\pgfpathlineto{\pgfqpoint{2.665361in}{3.073186in}}%
\pgfpathlineto{\pgfqpoint{2.654300in}{3.056681in}}%
\pgfpathlineto{\pgfqpoint{2.643234in}{3.040945in}}%
\pgfpathlineto{\pgfqpoint{2.632159in}{3.026108in}}%
\pgfpathlineto{\pgfqpoint{2.638096in}{3.019269in}}%
\pgfpathlineto{\pgfqpoint{2.644037in}{3.012428in}}%
\pgfpathlineto{\pgfqpoint{2.649982in}{3.005608in}}%
\pgfpathlineto{\pgfqpoint{2.655931in}{2.998830in}}%
\pgfpathclose%
\pgfusepath{stroke,fill}%
\end{pgfscope}%
\begin{pgfscope}%
\pgfpathrectangle{\pgfqpoint{0.887500in}{0.275000in}}{\pgfqpoint{4.225000in}{4.225000in}}%
\pgfusepath{clip}%
\pgfsetbuttcap%
\pgfsetroundjoin%
\definecolor{currentfill}{rgb}{0.119699,0.618490,0.536347}%
\pgfsetfillcolor{currentfill}%
\pgfsetfillopacity{0.700000}%
\pgfsetlinewidth{0.501875pt}%
\definecolor{currentstroke}{rgb}{1.000000,1.000000,1.000000}%
\pgfsetstrokecolor{currentstroke}%
\pgfsetstrokeopacity{0.500000}%
\pgfsetdash{}{0pt}%
\pgfpathmoveto{\pgfqpoint{1.897427in}{2.838311in}}%
\pgfpathlineto{\pgfqpoint{1.908820in}{2.841715in}}%
\pgfpathlineto{\pgfqpoint{1.920208in}{2.845102in}}%
\pgfpathlineto{\pgfqpoint{1.931592in}{2.848466in}}%
\pgfpathlineto{\pgfqpoint{1.942971in}{2.851805in}}%
\pgfpathlineto{\pgfqpoint{1.954344in}{2.855137in}}%
\pgfpathlineto{\pgfqpoint{1.948623in}{2.862827in}}%
\pgfpathlineto{\pgfqpoint{1.942906in}{2.870504in}}%
\pgfpathlineto{\pgfqpoint{1.937193in}{2.878169in}}%
\pgfpathlineto{\pgfqpoint{1.931485in}{2.885821in}}%
\pgfpathlineto{\pgfqpoint{1.925780in}{2.893462in}}%
\pgfpathlineto{\pgfqpoint{1.914420in}{2.890094in}}%
\pgfpathlineto{\pgfqpoint{1.903055in}{2.886720in}}%
\pgfpathlineto{\pgfqpoint{1.891685in}{2.883323in}}%
\pgfpathlineto{\pgfqpoint{1.880310in}{2.879905in}}%
\pgfpathlineto{\pgfqpoint{1.868931in}{2.876470in}}%
\pgfpathlineto{\pgfqpoint{1.874622in}{2.868866in}}%
\pgfpathlineto{\pgfqpoint{1.880317in}{2.861247in}}%
\pgfpathlineto{\pgfqpoint{1.886016in}{2.853615in}}%
\pgfpathlineto{\pgfqpoint{1.891719in}{2.845970in}}%
\pgfpathclose%
\pgfusepath{stroke,fill}%
\end{pgfscope}%
\begin{pgfscope}%
\pgfpathrectangle{\pgfqpoint{0.887500in}{0.275000in}}{\pgfqpoint{4.225000in}{4.225000in}}%
\pgfusepath{clip}%
\pgfsetbuttcap%
\pgfsetroundjoin%
\definecolor{currentfill}{rgb}{0.730889,0.871916,0.156029}%
\pgfsetfillcolor{currentfill}%
\pgfsetfillopacity{0.700000}%
\pgfsetlinewidth{0.501875pt}%
\definecolor{currentstroke}{rgb}{1.000000,1.000000,1.000000}%
\pgfsetstrokecolor{currentstroke}%
\pgfsetstrokeopacity{0.500000}%
\pgfsetdash{}{0pt}%
\pgfpathmoveto{\pgfqpoint{2.964766in}{3.524215in}}%
\pgfpathlineto{\pgfqpoint{2.975914in}{3.526591in}}%
\pgfpathlineto{\pgfqpoint{2.987055in}{3.529602in}}%
\pgfpathlineto{\pgfqpoint{2.998189in}{3.533147in}}%
\pgfpathlineto{\pgfqpoint{3.009318in}{3.537124in}}%
\pgfpathlineto{\pgfqpoint{3.020442in}{3.541429in}}%
\pgfpathlineto{\pgfqpoint{3.014367in}{3.542318in}}%
\pgfpathlineto{\pgfqpoint{3.008299in}{3.543103in}}%
\pgfpathlineto{\pgfqpoint{3.002237in}{3.543804in}}%
\pgfpathlineto{\pgfqpoint{2.996181in}{3.544436in}}%
\pgfpathlineto{\pgfqpoint{2.990131in}{3.545015in}}%
\pgfpathlineto{\pgfqpoint{2.979028in}{3.541419in}}%
\pgfpathlineto{\pgfqpoint{2.967919in}{3.537896in}}%
\pgfpathlineto{\pgfqpoint{2.956805in}{3.534352in}}%
\pgfpathlineto{\pgfqpoint{2.945686in}{3.530689in}}%
\pgfpathlineto{\pgfqpoint{2.934563in}{3.526813in}}%
\pgfpathlineto{\pgfqpoint{2.940589in}{3.527249in}}%
\pgfpathlineto{\pgfqpoint{2.946622in}{3.527250in}}%
\pgfpathlineto{\pgfqpoint{2.952663in}{3.526735in}}%
\pgfpathlineto{\pgfqpoint{2.958711in}{3.525700in}}%
\pgfpathclose%
\pgfusepath{stroke,fill}%
\end{pgfscope}%
\begin{pgfscope}%
\pgfpathrectangle{\pgfqpoint{0.887500in}{0.275000in}}{\pgfqpoint{4.225000in}{4.225000in}}%
\pgfusepath{clip}%
\pgfsetbuttcap%
\pgfsetroundjoin%
\definecolor{currentfill}{rgb}{0.146616,0.673050,0.508936}%
\pgfsetfillcolor{currentfill}%
\pgfsetfillopacity{0.700000}%
\pgfsetlinewidth{0.501875pt}%
\definecolor{currentstroke}{rgb}{1.000000,1.000000,1.000000}%
\pgfsetstrokecolor{currentstroke}%
\pgfsetstrokeopacity{0.500000}%
\pgfsetdash{}{0pt}%
\pgfpathmoveto{\pgfqpoint{2.606089in}{2.947309in}}%
\pgfpathlineto{\pgfqpoint{2.617269in}{2.954763in}}%
\pgfpathlineto{\pgfqpoint{2.628434in}{2.963261in}}%
\pgfpathlineto{\pgfqpoint{2.639589in}{2.972571in}}%
\pgfpathlineto{\pgfqpoint{2.650738in}{2.982315in}}%
\pgfpathlineto{\pgfqpoint{2.661884in}{2.992115in}}%
\pgfpathlineto{\pgfqpoint{2.655931in}{2.998830in}}%
\pgfpathlineto{\pgfqpoint{2.649982in}{3.005608in}}%
\pgfpathlineto{\pgfqpoint{2.644037in}{3.012428in}}%
\pgfpathlineto{\pgfqpoint{2.638096in}{3.019269in}}%
\pgfpathlineto{\pgfqpoint{2.632159in}{3.026108in}}%
\pgfpathlineto{\pgfqpoint{2.621074in}{3.012300in}}%
\pgfpathlineto{\pgfqpoint{2.609977in}{2.999649in}}%
\pgfpathlineto{\pgfqpoint{2.598865in}{2.988285in}}%
\pgfpathlineto{\pgfqpoint{2.587736in}{2.978337in}}%
\pgfpathlineto{\pgfqpoint{2.576588in}{2.969849in}}%
\pgfpathlineto{\pgfqpoint{2.582482in}{2.964993in}}%
\pgfpathlineto{\pgfqpoint{2.588378in}{2.960466in}}%
\pgfpathlineto{\pgfqpoint{2.594276in}{2.956114in}}%
\pgfpathlineto{\pgfqpoint{2.600179in}{2.951780in}}%
\pgfpathclose%
\pgfusepath{stroke,fill}%
\end{pgfscope}%
\begin{pgfscope}%
\pgfpathrectangle{\pgfqpoint{0.887500in}{0.275000in}}{\pgfqpoint{4.225000in}{4.225000in}}%
\pgfusepath{clip}%
\pgfsetbuttcap%
\pgfsetroundjoin%
\definecolor{currentfill}{rgb}{0.239374,0.735588,0.455688}%
\pgfsetfillcolor{currentfill}%
\pgfsetfillopacity{0.700000}%
\pgfsetlinewidth{0.501875pt}%
\definecolor{currentstroke}{rgb}{1.000000,1.000000,1.000000}%
\pgfsetstrokecolor{currentstroke}%
\pgfsetstrokeopacity{0.500000}%
\pgfsetdash{}{0pt}%
\pgfpathmoveto{\pgfqpoint{2.717672in}{3.031269in}}%
\pgfpathlineto{\pgfqpoint{2.728820in}{3.039276in}}%
\pgfpathlineto{\pgfqpoint{2.739943in}{3.050039in}}%
\pgfpathlineto{\pgfqpoint{2.751034in}{3.064927in}}%
\pgfpathlineto{\pgfqpoint{2.762085in}{3.085311in}}%
\pgfpathlineto{\pgfqpoint{2.773093in}{3.112566in}}%
\pgfpathlineto{\pgfqpoint{2.766989in}{3.134793in}}%
\pgfpathlineto{\pgfqpoint{2.760880in}{3.157700in}}%
\pgfpathlineto{\pgfqpoint{2.754772in}{3.180547in}}%
\pgfpathlineto{\pgfqpoint{2.748669in}{3.202593in}}%
\pgfpathlineto{\pgfqpoint{2.742577in}{3.223096in}}%
\pgfpathlineto{\pgfqpoint{2.731577in}{3.196117in}}%
\pgfpathlineto{\pgfqpoint{2.720565in}{3.171452in}}%
\pgfpathlineto{\pgfqpoint{2.709542in}{3.148824in}}%
\pgfpathlineto{\pgfqpoint{2.698509in}{3.127958in}}%
\pgfpathlineto{\pgfqpoint{2.687467in}{3.108579in}}%
\pgfpathlineto{\pgfqpoint{2.693499in}{3.093811in}}%
\pgfpathlineto{\pgfqpoint{2.699539in}{3.078284in}}%
\pgfpathlineto{\pgfqpoint{2.705583in}{3.062411in}}%
\pgfpathlineto{\pgfqpoint{2.711629in}{3.046603in}}%
\pgfpathclose%
\pgfusepath{stroke,fill}%
\end{pgfscope}%
\begin{pgfscope}%
\pgfpathrectangle{\pgfqpoint{0.887500in}{0.275000in}}{\pgfqpoint{4.225000in}{4.225000in}}%
\pgfusepath{clip}%
\pgfsetbuttcap%
\pgfsetroundjoin%
\definecolor{currentfill}{rgb}{0.876168,0.891125,0.095250}%
\pgfsetfillcolor{currentfill}%
\pgfsetfillopacity{0.700000}%
\pgfsetlinewidth{0.501875pt}%
\definecolor{currentstroke}{rgb}{1.000000,1.000000,1.000000}%
\pgfsetstrokecolor{currentstroke}%
\pgfsetstrokeopacity{0.500000}%
\pgfsetdash{}{0pt}%
\pgfpathmoveto{\pgfqpoint{3.531771in}{3.605186in}}%
\pgfpathlineto{\pgfqpoint{3.542796in}{3.609789in}}%
\pgfpathlineto{\pgfqpoint{3.553817in}{3.614390in}}%
\pgfpathlineto{\pgfqpoint{3.564834in}{3.618996in}}%
\pgfpathlineto{\pgfqpoint{3.575847in}{3.623620in}}%
\pgfpathlineto{\pgfqpoint{3.586855in}{3.628276in}}%
\pgfpathlineto{\pgfqpoint{3.580588in}{3.635969in}}%
\pgfpathlineto{\pgfqpoint{3.574324in}{3.643632in}}%
\pgfpathlineto{\pgfqpoint{3.568064in}{3.651275in}}%
\pgfpathlineto{\pgfqpoint{3.561808in}{3.658871in}}%
\pgfpathlineto{\pgfqpoint{3.555555in}{3.666393in}}%
\pgfpathlineto{\pgfqpoint{3.544552in}{3.661220in}}%
\pgfpathlineto{\pgfqpoint{3.533545in}{3.656051in}}%
\pgfpathlineto{\pgfqpoint{3.522534in}{3.650890in}}%
\pgfpathlineto{\pgfqpoint{3.511519in}{3.645739in}}%
\pgfpathlineto{\pgfqpoint{3.500500in}{3.640597in}}%
\pgfpathlineto{\pgfqpoint{3.506744in}{3.633383in}}%
\pgfpathlineto{\pgfqpoint{3.512992in}{3.626194in}}%
\pgfpathlineto{\pgfqpoint{3.519246in}{3.619080in}}%
\pgfpathlineto{\pgfqpoint{3.525506in}{3.612084in}}%
\pgfpathclose%
\pgfusepath{stroke,fill}%
\end{pgfscope}%
\begin{pgfscope}%
\pgfpathrectangle{\pgfqpoint{0.887500in}{0.275000in}}{\pgfqpoint{4.225000in}{4.225000in}}%
\pgfusepath{clip}%
\pgfsetbuttcap%
\pgfsetroundjoin%
\definecolor{currentfill}{rgb}{0.120565,0.596422,0.543611}%
\pgfsetfillcolor{currentfill}%
\pgfsetfillopacity{0.700000}%
\pgfsetlinewidth{0.501875pt}%
\definecolor{currentstroke}{rgb}{1.000000,1.000000,1.000000}%
\pgfsetstrokecolor{currentstroke}%
\pgfsetstrokeopacity{0.500000}%
\pgfsetdash{}{0pt}%
\pgfpathmoveto{\pgfqpoint{2.211206in}{2.785960in}}%
\pgfpathlineto{\pgfqpoint{2.222537in}{2.788937in}}%
\pgfpathlineto{\pgfqpoint{2.233844in}{2.792773in}}%
\pgfpathlineto{\pgfqpoint{2.245122in}{2.797778in}}%
\pgfpathlineto{\pgfqpoint{2.256367in}{2.804260in}}%
\pgfpathlineto{\pgfqpoint{2.267577in}{2.812272in}}%
\pgfpathlineto{\pgfqpoint{2.261751in}{2.819861in}}%
\pgfpathlineto{\pgfqpoint{2.255928in}{2.827481in}}%
\pgfpathlineto{\pgfqpoint{2.250107in}{2.835184in}}%
\pgfpathlineto{\pgfqpoint{2.244287in}{2.843021in}}%
\pgfpathlineto{\pgfqpoint{2.238467in}{2.851045in}}%
\pgfpathlineto{\pgfqpoint{2.227277in}{2.842803in}}%
\pgfpathlineto{\pgfqpoint{2.216050in}{2.836212in}}%
\pgfpathlineto{\pgfqpoint{2.204785in}{2.831210in}}%
\pgfpathlineto{\pgfqpoint{2.193489in}{2.827464in}}%
\pgfpathlineto{\pgfqpoint{2.182167in}{2.824637in}}%
\pgfpathlineto{\pgfqpoint{2.187972in}{2.816693in}}%
\pgfpathlineto{\pgfqpoint{2.193777in}{2.808885in}}%
\pgfpathlineto{\pgfqpoint{2.199584in}{2.801182in}}%
\pgfpathlineto{\pgfqpoint{2.205394in}{2.793551in}}%
\pgfpathclose%
\pgfusepath{stroke,fill}%
\end{pgfscope}%
\begin{pgfscope}%
\pgfpathrectangle{\pgfqpoint{0.887500in}{0.275000in}}{\pgfqpoint{4.225000in}{4.225000in}}%
\pgfusepath{clip}%
\pgfsetbuttcap%
\pgfsetroundjoin%
\definecolor{currentfill}{rgb}{0.121380,0.629492,0.531973}%
\pgfsetfillcolor{currentfill}%
\pgfsetfillopacity{0.700000}%
\pgfsetlinewidth{0.501875pt}%
\definecolor{currentstroke}{rgb}{1.000000,1.000000,1.000000}%
\pgfsetstrokecolor{currentstroke}%
\pgfsetstrokeopacity{0.500000}%
\pgfsetdash{}{0pt}%
\pgfpathmoveto{\pgfqpoint{1.669379in}{2.863156in}}%
\pgfpathlineto{\pgfqpoint{1.680830in}{2.866497in}}%
\pgfpathlineto{\pgfqpoint{1.692276in}{2.869842in}}%
\pgfpathlineto{\pgfqpoint{1.703715in}{2.873194in}}%
\pgfpathlineto{\pgfqpoint{1.715149in}{2.876552in}}%
\pgfpathlineto{\pgfqpoint{1.726577in}{2.879919in}}%
\pgfpathlineto{\pgfqpoint{1.720939in}{2.887391in}}%
\pgfpathlineto{\pgfqpoint{1.715305in}{2.894849in}}%
\pgfpathlineto{\pgfqpoint{1.709674in}{2.902293in}}%
\pgfpathlineto{\pgfqpoint{1.704049in}{2.909723in}}%
\pgfpathlineto{\pgfqpoint{1.692631in}{2.906348in}}%
\pgfpathlineto{\pgfqpoint{1.681209in}{2.902977in}}%
\pgfpathlineto{\pgfqpoint{1.669780in}{2.899607in}}%
\pgfpathlineto{\pgfqpoint{1.658347in}{2.896239in}}%
\pgfpathlineto{\pgfqpoint{1.646907in}{2.892872in}}%
\pgfpathlineto{\pgfqpoint{1.652519in}{2.885464in}}%
\pgfpathlineto{\pgfqpoint{1.658135in}{2.878042in}}%
\pgfpathlineto{\pgfqpoint{1.663755in}{2.870606in}}%
\pgfpathclose%
\pgfusepath{stroke,fill}%
\end{pgfscope}%
\begin{pgfscope}%
\pgfpathrectangle{\pgfqpoint{0.887500in}{0.275000in}}{\pgfqpoint{4.225000in}{4.225000in}}%
\pgfusepath{clip}%
\pgfsetbuttcap%
\pgfsetroundjoin%
\definecolor{currentfill}{rgb}{0.855810,0.888601,0.097452}%
\pgfsetfillcolor{currentfill}%
\pgfsetfillopacity{0.700000}%
\pgfsetlinewidth{0.501875pt}%
\definecolor{currentstroke}{rgb}{1.000000,1.000000,1.000000}%
\pgfsetstrokecolor{currentstroke}%
\pgfsetstrokeopacity{0.500000}%
\pgfsetdash{}{0pt}%
\pgfpathmoveto{\pgfqpoint{3.759594in}{3.583733in}}%
\pgfpathlineto{\pgfqpoint{3.770579in}{3.588677in}}%
\pgfpathlineto{\pgfqpoint{3.781557in}{3.593528in}}%
\pgfpathlineto{\pgfqpoint{3.792530in}{3.598296in}}%
\pgfpathlineto{\pgfqpoint{3.803498in}{3.602991in}}%
\pgfpathlineto{\pgfqpoint{3.814460in}{3.607626in}}%
\pgfpathlineto{\pgfqpoint{3.808170in}{3.618188in}}%
\pgfpathlineto{\pgfqpoint{3.801880in}{3.628498in}}%
\pgfpathlineto{\pgfqpoint{3.795588in}{3.638556in}}%
\pgfpathlineto{\pgfqpoint{3.789295in}{3.648369in}}%
\pgfpathlineto{\pgfqpoint{3.783002in}{3.657945in}}%
\pgfpathlineto{\pgfqpoint{3.772046in}{3.653179in}}%
\pgfpathlineto{\pgfqpoint{3.761085in}{3.648364in}}%
\pgfpathlineto{\pgfqpoint{3.750118in}{3.643493in}}%
\pgfpathlineto{\pgfqpoint{3.739147in}{3.638557in}}%
\pgfpathlineto{\pgfqpoint{3.728170in}{3.633551in}}%
\pgfpathlineto{\pgfqpoint{3.734455in}{3.624063in}}%
\pgfpathlineto{\pgfqpoint{3.740741in}{3.614345in}}%
\pgfpathlineto{\pgfqpoint{3.747026in}{3.604389in}}%
\pgfpathlineto{\pgfqpoint{3.753310in}{3.594185in}}%
\pgfpathclose%
\pgfusepath{stroke,fill}%
\end{pgfscope}%
\begin{pgfscope}%
\pgfpathrectangle{\pgfqpoint{0.887500in}{0.275000in}}{\pgfqpoint{4.225000in}{4.225000in}}%
\pgfusepath{clip}%
\pgfsetbuttcap%
\pgfsetroundjoin%
\definecolor{currentfill}{rgb}{0.855810,0.888601,0.097452}%
\pgfsetfillcolor{currentfill}%
\pgfsetfillopacity{0.700000}%
\pgfsetlinewidth{0.501875pt}%
\definecolor{currentstroke}{rgb}{1.000000,1.000000,1.000000}%
\pgfsetstrokecolor{currentstroke}%
\pgfsetstrokeopacity{0.500000}%
\pgfsetdash{}{0pt}%
\pgfpathmoveto{\pgfqpoint{3.390083in}{3.589461in}}%
\pgfpathlineto{\pgfqpoint{3.401140in}{3.594223in}}%
\pgfpathlineto{\pgfqpoint{3.412195in}{3.599205in}}%
\pgfpathlineto{\pgfqpoint{3.423248in}{3.604337in}}%
\pgfpathlineto{\pgfqpoint{3.434296in}{3.609552in}}%
\pgfpathlineto{\pgfqpoint{3.445341in}{3.614780in}}%
\pgfpathlineto{\pgfqpoint{3.439110in}{3.621605in}}%
\pgfpathlineto{\pgfqpoint{3.432884in}{3.628396in}}%
\pgfpathlineto{\pgfqpoint{3.426661in}{3.635096in}}%
\pgfpathlineto{\pgfqpoint{3.420441in}{3.641652in}}%
\pgfpathlineto{\pgfqpoint{3.414224in}{3.648008in}}%
\pgfpathlineto{\pgfqpoint{3.403190in}{3.642434in}}%
\pgfpathlineto{\pgfqpoint{3.392152in}{3.636859in}}%
\pgfpathlineto{\pgfqpoint{3.381111in}{3.631330in}}%
\pgfpathlineto{\pgfqpoint{3.370066in}{3.625895in}}%
\pgfpathlineto{\pgfqpoint{3.359019in}{3.620602in}}%
\pgfpathlineto{\pgfqpoint{3.365224in}{3.614629in}}%
\pgfpathlineto{\pgfqpoint{3.371433in}{3.608473in}}%
\pgfpathlineto{\pgfqpoint{3.377646in}{3.602190in}}%
\pgfpathlineto{\pgfqpoint{3.383862in}{3.595834in}}%
\pgfpathclose%
\pgfusepath{stroke,fill}%
\end{pgfscope}%
\begin{pgfscope}%
\pgfpathrectangle{\pgfqpoint{0.887500in}{0.275000in}}{\pgfqpoint{4.225000in}{4.225000in}}%
\pgfusepath{clip}%
\pgfsetbuttcap%
\pgfsetroundjoin%
\definecolor{currentfill}{rgb}{0.793760,0.880678,0.120005}%
\pgfsetfillcolor{currentfill}%
\pgfsetfillopacity{0.700000}%
\pgfsetlinewidth{0.501875pt}%
\definecolor{currentstroke}{rgb}{1.000000,1.000000,1.000000}%
\pgfsetstrokecolor{currentstroke}%
\pgfsetstrokeopacity{0.500000}%
\pgfsetdash{}{0pt}%
\pgfpathmoveto{\pgfqpoint{3.106563in}{3.560320in}}%
\pgfpathlineto{\pgfqpoint{3.117680in}{3.564354in}}%
\pgfpathlineto{\pgfqpoint{3.128792in}{3.568185in}}%
\pgfpathlineto{\pgfqpoint{3.139899in}{3.571897in}}%
\pgfpathlineto{\pgfqpoint{3.151001in}{3.575576in}}%
\pgfpathlineto{\pgfqpoint{3.162098in}{3.579296in}}%
\pgfpathlineto{\pgfqpoint{3.155954in}{3.581618in}}%
\pgfpathlineto{\pgfqpoint{3.149816in}{3.583708in}}%
\pgfpathlineto{\pgfqpoint{3.143682in}{3.585589in}}%
\pgfpathlineto{\pgfqpoint{3.137554in}{3.587282in}}%
\pgfpathlineto{\pgfqpoint{3.131432in}{3.588810in}}%
\pgfpathlineto{\pgfqpoint{3.120353in}{3.583851in}}%
\pgfpathlineto{\pgfqpoint{3.109269in}{3.578966in}}%
\pgfpathlineto{\pgfqpoint{3.098181in}{3.574148in}}%
\pgfpathlineto{\pgfqpoint{3.087089in}{3.569383in}}%
\pgfpathlineto{\pgfqpoint{3.075992in}{3.564660in}}%
\pgfpathlineto{\pgfqpoint{3.082094in}{3.564163in}}%
\pgfpathlineto{\pgfqpoint{3.088203in}{3.563535in}}%
\pgfpathlineto{\pgfqpoint{3.094317in}{3.562721in}}%
\pgfpathlineto{\pgfqpoint{3.100437in}{3.561668in}}%
\pgfpathclose%
\pgfusepath{stroke,fill}%
\end{pgfscope}%
\begin{pgfscope}%
\pgfpathrectangle{\pgfqpoint{0.887500in}{0.275000in}}{\pgfqpoint{4.225000in}{4.225000in}}%
\pgfusepath{clip}%
\pgfsetbuttcap%
\pgfsetroundjoin%
\definecolor{currentfill}{rgb}{0.126326,0.644107,0.525311}%
\pgfsetfillcolor{currentfill}%
\pgfsetfillopacity{0.700000}%
\pgfsetlinewidth{0.501875pt}%
\definecolor{currentstroke}{rgb}{1.000000,1.000000,1.000000}%
\pgfsetstrokecolor{currentstroke}%
\pgfsetstrokeopacity{0.500000}%
\pgfsetdash{}{0pt}%
\pgfpathmoveto{\pgfqpoint{2.408671in}{2.867098in}}%
\pgfpathlineto{\pgfqpoint{2.419863in}{2.875588in}}%
\pgfpathlineto{\pgfqpoint{2.431056in}{2.883860in}}%
\pgfpathlineto{\pgfqpoint{2.442250in}{2.891930in}}%
\pgfpathlineto{\pgfqpoint{2.453444in}{2.899780in}}%
\pgfpathlineto{\pgfqpoint{2.464639in}{2.907365in}}%
\pgfpathlineto{\pgfqpoint{2.458738in}{2.915362in}}%
\pgfpathlineto{\pgfqpoint{2.452836in}{2.923731in}}%
\pgfpathlineto{\pgfqpoint{2.446929in}{2.932604in}}%
\pgfpathlineto{\pgfqpoint{2.441018in}{2.942005in}}%
\pgfpathlineto{\pgfqpoint{2.435104in}{2.951826in}}%
\pgfpathlineto{\pgfqpoint{2.423909in}{2.945080in}}%
\pgfpathlineto{\pgfqpoint{2.412717in}{2.937899in}}%
\pgfpathlineto{\pgfqpoint{2.401530in}{2.930291in}}%
\pgfpathlineto{\pgfqpoint{2.390346in}{2.922260in}}%
\pgfpathlineto{\pgfqpoint{2.379167in}{2.913796in}}%
\pgfpathlineto{\pgfqpoint{2.385058in}{2.904635in}}%
\pgfpathlineto{\pgfqpoint{2.390954in}{2.895386in}}%
\pgfpathlineto{\pgfqpoint{2.396855in}{2.886055in}}%
\pgfpathlineto{\pgfqpoint{2.402761in}{2.876633in}}%
\pgfpathclose%
\pgfusepath{stroke,fill}%
\end{pgfscope}%
\begin{pgfscope}%
\pgfpathrectangle{\pgfqpoint{0.887500in}{0.275000in}}{\pgfqpoint{4.225000in}{4.225000in}}%
\pgfusepath{clip}%
\pgfsetbuttcap%
\pgfsetroundjoin%
\definecolor{currentfill}{rgb}{0.824940,0.884720,0.106217}%
\pgfsetfillcolor{currentfill}%
\pgfsetfillopacity{0.700000}%
\pgfsetlinewidth{0.501875pt}%
\definecolor{currentstroke}{rgb}{1.000000,1.000000,1.000000}%
\pgfsetstrokecolor{currentstroke}%
\pgfsetstrokeopacity{0.500000}%
\pgfsetdash{}{0pt}%
\pgfpathmoveto{\pgfqpoint{3.248373in}{3.579556in}}%
\pgfpathlineto{\pgfqpoint{3.259456in}{3.582951in}}%
\pgfpathlineto{\pgfqpoint{3.270533in}{3.586425in}}%
\pgfpathlineto{\pgfqpoint{3.281606in}{3.590006in}}%
\pgfpathlineto{\pgfqpoint{3.292675in}{3.593725in}}%
\pgfpathlineto{\pgfqpoint{3.303740in}{3.597616in}}%
\pgfpathlineto{\pgfqpoint{3.297548in}{3.602590in}}%
\pgfpathlineto{\pgfqpoint{3.291359in}{3.607375in}}%
\pgfpathlineto{\pgfqpoint{3.285174in}{3.611948in}}%
\pgfpathlineto{\pgfqpoint{3.278992in}{3.616300in}}%
\pgfpathlineto{\pgfqpoint{3.272815in}{3.620426in}}%
\pgfpathlineto{\pgfqpoint{3.261762in}{3.615870in}}%
\pgfpathlineto{\pgfqpoint{3.250706in}{3.611458in}}%
\pgfpathlineto{\pgfqpoint{3.239645in}{3.607167in}}%
\pgfpathlineto{\pgfqpoint{3.228581in}{3.602973in}}%
\pgfpathlineto{\pgfqpoint{3.217512in}{3.598856in}}%
\pgfpathlineto{\pgfqpoint{3.223675in}{3.595287in}}%
\pgfpathlineto{\pgfqpoint{3.229842in}{3.591562in}}%
\pgfpathlineto{\pgfqpoint{3.236015in}{3.587694in}}%
\pgfpathlineto{\pgfqpoint{3.242192in}{3.583693in}}%
\pgfpathclose%
\pgfusepath{stroke,fill}%
\end{pgfscope}%
\begin{pgfscope}%
\pgfpathrectangle{\pgfqpoint{0.887500in}{0.275000in}}{\pgfqpoint{4.225000in}{4.225000in}}%
\pgfusepath{clip}%
\pgfsetbuttcap%
\pgfsetroundjoin%
\definecolor{currentfill}{rgb}{0.119423,0.611141,0.538982}%
\pgfsetfillcolor{currentfill}%
\pgfsetfillopacity{0.700000}%
\pgfsetlinewidth{0.501875pt}%
\definecolor{currentstroke}{rgb}{1.000000,1.000000,1.000000}%
\pgfsetstrokecolor{currentstroke}%
\pgfsetstrokeopacity{0.500000}%
\pgfsetdash{}{0pt}%
\pgfpathmoveto{\pgfqpoint{1.983010in}{2.816508in}}%
\pgfpathlineto{\pgfqpoint{1.994391in}{2.819827in}}%
\pgfpathlineto{\pgfqpoint{2.005765in}{2.823188in}}%
\pgfpathlineto{\pgfqpoint{2.017133in}{2.826618in}}%
\pgfpathlineto{\pgfqpoint{2.028492in}{2.830140in}}%
\pgfpathlineto{\pgfqpoint{2.039843in}{2.833780in}}%
\pgfpathlineto{\pgfqpoint{2.034089in}{2.841554in}}%
\pgfpathlineto{\pgfqpoint{2.028338in}{2.849316in}}%
\pgfpathlineto{\pgfqpoint{2.022592in}{2.857065in}}%
\pgfpathlineto{\pgfqpoint{2.016850in}{2.864802in}}%
\pgfpathlineto{\pgfqpoint{2.011111in}{2.872527in}}%
\pgfpathlineto{\pgfqpoint{1.999773in}{2.868874in}}%
\pgfpathlineto{\pgfqpoint{1.988427in}{2.865332in}}%
\pgfpathlineto{\pgfqpoint{1.977073in}{2.861877in}}%
\pgfpathlineto{\pgfqpoint{1.965712in}{2.858487in}}%
\pgfpathlineto{\pgfqpoint{1.954344in}{2.855137in}}%
\pgfpathlineto{\pgfqpoint{1.960069in}{2.847436in}}%
\pgfpathlineto{\pgfqpoint{1.965798in}{2.839722in}}%
\pgfpathlineto{\pgfqpoint{1.971531in}{2.831996in}}%
\pgfpathlineto{\pgfqpoint{1.977268in}{2.824258in}}%
\pgfpathclose%
\pgfusepath{stroke,fill}%
\end{pgfscope}%
\begin{pgfscope}%
\pgfpathrectangle{\pgfqpoint{0.887500in}{0.275000in}}{\pgfqpoint{4.225000in}{4.225000in}}%
\pgfusepath{clip}%
\pgfsetbuttcap%
\pgfsetroundjoin%
\definecolor{currentfill}{rgb}{0.824940,0.884720,0.106217}%
\pgfsetfillcolor{currentfill}%
\pgfsetfillopacity{0.700000}%
\pgfsetlinewidth{0.501875pt}%
\definecolor{currentstroke}{rgb}{1.000000,1.000000,1.000000}%
\pgfsetstrokecolor{currentstroke}%
\pgfsetstrokeopacity{0.500000}%
\pgfsetdash{}{0pt}%
\pgfpathmoveto{\pgfqpoint{3.845895in}{3.551633in}}%
\pgfpathlineto{\pgfqpoint{3.856855in}{3.556053in}}%
\pgfpathlineto{\pgfqpoint{3.867810in}{3.560431in}}%
\pgfpathlineto{\pgfqpoint{3.878759in}{3.564775in}}%
\pgfpathlineto{\pgfqpoint{3.889704in}{3.569097in}}%
\pgfpathlineto{\pgfqpoint{3.900643in}{3.573407in}}%
\pgfpathlineto{\pgfqpoint{3.894355in}{3.585140in}}%
\pgfpathlineto{\pgfqpoint{3.888065in}{3.596710in}}%
\pgfpathlineto{\pgfqpoint{3.881775in}{3.608097in}}%
\pgfpathlineto{\pgfqpoint{3.875484in}{3.619282in}}%
\pgfpathlineto{\pgfqpoint{3.869192in}{3.630246in}}%
\pgfpathlineto{\pgfqpoint{3.858255in}{3.625760in}}%
\pgfpathlineto{\pgfqpoint{3.847314in}{3.621266in}}%
\pgfpathlineto{\pgfqpoint{3.836367in}{3.616752in}}%
\pgfpathlineto{\pgfqpoint{3.825416in}{3.612209in}}%
\pgfpathlineto{\pgfqpoint{3.814460in}{3.607626in}}%
\pgfpathlineto{\pgfqpoint{3.820748in}{3.596828in}}%
\pgfpathlineto{\pgfqpoint{3.827035in}{3.585811in}}%
\pgfpathlineto{\pgfqpoint{3.833322in}{3.574595in}}%
\pgfpathlineto{\pgfqpoint{3.839609in}{3.563196in}}%
\pgfpathclose%
\pgfusepath{stroke,fill}%
\end{pgfscope}%
\begin{pgfscope}%
\pgfpathrectangle{\pgfqpoint{0.887500in}{0.275000in}}{\pgfqpoint{4.225000in}{4.225000in}}%
\pgfusepath{clip}%
\pgfsetbuttcap%
\pgfsetroundjoin%
\definecolor{currentfill}{rgb}{0.120638,0.625828,0.533488}%
\pgfsetfillcolor{currentfill}%
\pgfsetfillopacity{0.700000}%
\pgfsetlinewidth{0.501875pt}%
\definecolor{currentstroke}{rgb}{1.000000,1.000000,1.000000}%
\pgfsetstrokecolor{currentstroke}%
\pgfsetstrokeopacity{0.500000}%
\pgfsetdash{}{0pt}%
\pgfpathmoveto{\pgfqpoint{1.754831in}{2.842349in}}%
\pgfpathlineto{\pgfqpoint{1.766267in}{2.845710in}}%
\pgfpathlineto{\pgfqpoint{1.777698in}{2.849083in}}%
\pgfpathlineto{\pgfqpoint{1.789122in}{2.852467in}}%
\pgfpathlineto{\pgfqpoint{1.800540in}{2.855865in}}%
\pgfpathlineto{\pgfqpoint{1.811953in}{2.859275in}}%
\pgfpathlineto{\pgfqpoint{1.806280in}{2.866837in}}%
\pgfpathlineto{\pgfqpoint{1.800612in}{2.874383in}}%
\pgfpathlineto{\pgfqpoint{1.794947in}{2.881914in}}%
\pgfpathlineto{\pgfqpoint{1.789287in}{2.889428in}}%
\pgfpathlineto{\pgfqpoint{1.783630in}{2.896927in}}%
\pgfpathlineto{\pgfqpoint{1.772232in}{2.893497in}}%
\pgfpathlineto{\pgfqpoint{1.760827in}{2.890083in}}%
\pgfpathlineto{\pgfqpoint{1.749416in}{2.886683in}}%
\pgfpathlineto{\pgfqpoint{1.738000in}{2.883296in}}%
\pgfpathlineto{\pgfqpoint{1.726577in}{2.879919in}}%
\pgfpathlineto{\pgfqpoint{1.732220in}{2.872433in}}%
\pgfpathlineto{\pgfqpoint{1.737866in}{2.864933in}}%
\pgfpathlineto{\pgfqpoint{1.743517in}{2.857419in}}%
\pgfpathlineto{\pgfqpoint{1.749172in}{2.849891in}}%
\pgfpathclose%
\pgfusepath{stroke,fill}%
\end{pgfscope}%
\begin{pgfscope}%
\pgfpathrectangle{\pgfqpoint{0.887500in}{0.275000in}}{\pgfqpoint{4.225000in}{4.225000in}}%
\pgfusepath{clip}%
\pgfsetbuttcap%
\pgfsetroundjoin%
\definecolor{currentfill}{rgb}{0.134692,0.658636,0.517649}%
\pgfsetfillcolor{currentfill}%
\pgfsetfillopacity{0.700000}%
\pgfsetlinewidth{0.501875pt}%
\definecolor{currentstroke}{rgb}{1.000000,1.000000,1.000000}%
\pgfsetstrokecolor{currentstroke}%
\pgfsetstrokeopacity{0.500000}%
\pgfsetdash{}{0pt}%
\pgfpathmoveto{\pgfqpoint{2.550078in}{2.915292in}}%
\pgfpathlineto{\pgfqpoint{2.561282in}{2.922124in}}%
\pgfpathlineto{\pgfqpoint{2.572489in}{2.928403in}}%
\pgfpathlineto{\pgfqpoint{2.583696in}{2.934468in}}%
\pgfpathlineto{\pgfqpoint{2.594897in}{2.940657in}}%
\pgfpathlineto{\pgfqpoint{2.606089in}{2.947309in}}%
\pgfpathlineto{\pgfqpoint{2.600179in}{2.951780in}}%
\pgfpathlineto{\pgfqpoint{2.594276in}{2.956114in}}%
\pgfpathlineto{\pgfqpoint{2.588378in}{2.960466in}}%
\pgfpathlineto{\pgfqpoint{2.582482in}{2.964993in}}%
\pgfpathlineto{\pgfqpoint{2.576588in}{2.969849in}}%
\pgfpathlineto{\pgfqpoint{2.565422in}{2.962581in}}%
\pgfpathlineto{\pgfqpoint{2.554242in}{2.956236in}}%
\pgfpathlineto{\pgfqpoint{2.543049in}{2.950517in}}%
\pgfpathlineto{\pgfqpoint{2.531849in}{2.945129in}}%
\pgfpathlineto{\pgfqpoint{2.520644in}{2.939774in}}%
\pgfpathlineto{\pgfqpoint{2.526525in}{2.934559in}}%
\pgfpathlineto{\pgfqpoint{2.532407in}{2.929734in}}%
\pgfpathlineto{\pgfqpoint{2.538290in}{2.925069in}}%
\pgfpathlineto{\pgfqpoint{2.544180in}{2.920332in}}%
\pgfpathclose%
\pgfusepath{stroke,fill}%
\end{pgfscope}%
\begin{pgfscope}%
\pgfpathrectangle{\pgfqpoint{0.887500in}{0.275000in}}{\pgfqpoint{4.225000in}{4.225000in}}%
\pgfusepath{clip}%
\pgfsetbuttcap%
\pgfsetroundjoin%
\definecolor{currentfill}{rgb}{0.180653,0.701402,0.488189}%
\pgfsetfillcolor{currentfill}%
\pgfsetfillopacity{0.700000}%
\pgfsetlinewidth{0.501875pt}%
\definecolor{currentstroke}{rgb}{1.000000,1.000000,1.000000}%
\pgfsetstrokecolor{currentstroke}%
\pgfsetstrokeopacity{0.500000}%
\pgfsetdash{}{0pt}%
\pgfpathmoveto{\pgfqpoint{2.747755in}{2.975051in}}%
\pgfpathlineto{\pgfqpoint{2.758963in}{2.976353in}}%
\pgfpathlineto{\pgfqpoint{2.770140in}{2.981063in}}%
\pgfpathlineto{\pgfqpoint{2.781275in}{2.991184in}}%
\pgfpathlineto{\pgfqpoint{2.792360in}{3.008723in}}%
\pgfpathlineto{\pgfqpoint{2.803388in}{3.035691in}}%
\pgfpathlineto{\pgfqpoint{2.797365in}{3.044940in}}%
\pgfpathlineto{\pgfqpoint{2.791325in}{3.057331in}}%
\pgfpathlineto{\pgfqpoint{2.785265in}{3.073099in}}%
\pgfpathlineto{\pgfqpoint{2.779186in}{3.091756in}}%
\pgfpathlineto{\pgfqpoint{2.773093in}{3.112566in}}%
\pgfpathlineto{\pgfqpoint{2.762085in}{3.085311in}}%
\pgfpathlineto{\pgfqpoint{2.751034in}{3.064927in}}%
\pgfpathlineto{\pgfqpoint{2.739943in}{3.050039in}}%
\pgfpathlineto{\pgfqpoint{2.728820in}{3.039276in}}%
\pgfpathlineto{\pgfqpoint{2.717672in}{3.031269in}}%
\pgfpathlineto{\pgfqpoint{2.723709in}{3.016820in}}%
\pgfpathlineto{\pgfqpoint{2.729738in}{3.003666in}}%
\pgfpathlineto{\pgfqpoint{2.735756in}{2.992216in}}%
\pgfpathlineto{\pgfqpoint{2.741761in}{2.982733in}}%
\pgfpathclose%
\pgfusepath{stroke,fill}%
\end{pgfscope}%
\begin{pgfscope}%
\pgfpathrectangle{\pgfqpoint{0.887500in}{0.275000in}}{\pgfqpoint{4.225000in}{4.225000in}}%
\pgfusepath{clip}%
\pgfsetbuttcap%
\pgfsetroundjoin%
\definecolor{currentfill}{rgb}{0.772852,0.877868,0.131109}%
\pgfsetfillcolor{currentfill}%
\pgfsetfillopacity{0.700000}%
\pgfsetlinewidth{0.501875pt}%
\definecolor{currentstroke}{rgb}{1.000000,1.000000,1.000000}%
\pgfsetstrokecolor{currentstroke}%
\pgfsetstrokeopacity{0.500000}%
\pgfsetdash{}{0pt}%
\pgfpathmoveto{\pgfqpoint{3.932095in}{3.512969in}}%
\pgfpathlineto{\pgfqpoint{3.943034in}{3.517189in}}%
\pgfpathlineto{\pgfqpoint{3.953968in}{3.521434in}}%
\pgfpathlineto{\pgfqpoint{3.964898in}{3.525710in}}%
\pgfpathlineto{\pgfqpoint{3.975825in}{3.530026in}}%
\pgfpathlineto{\pgfqpoint{3.969528in}{3.542262in}}%
\pgfpathlineto{\pgfqpoint{3.963233in}{3.554476in}}%
\pgfpathlineto{\pgfqpoint{3.956941in}{3.566642in}}%
\pgfpathlineto{\pgfqpoint{3.950649in}{3.578735in}}%
\pgfpathlineto{\pgfqpoint{3.944358in}{3.590730in}}%
\pgfpathlineto{\pgfqpoint{3.933436in}{3.586366in}}%
\pgfpathlineto{\pgfqpoint{3.922509in}{3.582031in}}%
\pgfpathlineto{\pgfqpoint{3.911579in}{3.577715in}}%
\pgfpathlineto{\pgfqpoint{3.900643in}{3.573407in}}%
\pgfpathlineto{\pgfqpoint{3.906932in}{3.561530in}}%
\pgfpathlineto{\pgfqpoint{3.913222in}{3.549529in}}%
\pgfpathlineto{\pgfqpoint{3.919512in}{3.537422in}}%
\pgfpathlineto{\pgfqpoint{3.925803in}{3.525229in}}%
\pgfpathclose%
\pgfusepath{stroke,fill}%
\end{pgfscope}%
\begin{pgfscope}%
\pgfpathrectangle{\pgfqpoint{0.887500in}{0.275000in}}{\pgfqpoint{4.225000in}{4.225000in}}%
\pgfusepath{clip}%
\pgfsetbuttcap%
\pgfsetroundjoin%
\definecolor{currentfill}{rgb}{0.866013,0.889868,0.095953}%
\pgfsetfillcolor{currentfill}%
\pgfsetfillopacity{0.700000}%
\pgfsetlinewidth{0.501875pt}%
\definecolor{currentstroke}{rgb}{1.000000,1.000000,1.000000}%
\pgfsetstrokecolor{currentstroke}%
\pgfsetstrokeopacity{0.500000}%
\pgfsetdash{}{0pt}%
\pgfpathmoveto{\pgfqpoint{3.618232in}{3.588061in}}%
\pgfpathlineto{\pgfqpoint{3.629234in}{3.591781in}}%
\pgfpathlineto{\pgfqpoint{3.640233in}{3.595652in}}%
\pgfpathlineto{\pgfqpoint{3.651231in}{3.599729in}}%
\pgfpathlineto{\pgfqpoint{3.662228in}{3.604065in}}%
\pgfpathlineto{\pgfqpoint{3.673224in}{3.608653in}}%
\pgfpathlineto{\pgfqpoint{3.666946in}{3.617803in}}%
\pgfpathlineto{\pgfqpoint{3.660669in}{3.626747in}}%
\pgfpathlineto{\pgfqpoint{3.654393in}{3.635496in}}%
\pgfpathlineto{\pgfqpoint{3.648118in}{3.644064in}}%
\pgfpathlineto{\pgfqpoint{3.641845in}{3.652464in}}%
\pgfpathlineto{\pgfqpoint{3.630854in}{3.647475in}}%
\pgfpathlineto{\pgfqpoint{3.619859in}{3.642564in}}%
\pgfpathlineto{\pgfqpoint{3.608861in}{3.637735in}}%
\pgfpathlineto{\pgfqpoint{3.597860in}{3.632977in}}%
\pgfpathlineto{\pgfqpoint{3.586855in}{3.628276in}}%
\pgfpathlineto{\pgfqpoint{3.593126in}{3.620516in}}%
\pgfpathlineto{\pgfqpoint{3.599400in}{3.612652in}}%
\pgfpathlineto{\pgfqpoint{3.605676in}{3.604647in}}%
\pgfpathlineto{\pgfqpoint{3.611953in}{3.596462in}}%
\pgfpathclose%
\pgfusepath{stroke,fill}%
\end{pgfscope}%
\begin{pgfscope}%
\pgfpathrectangle{\pgfqpoint{0.887500in}{0.275000in}}{\pgfqpoint{4.225000in}{4.225000in}}%
\pgfusepath{clip}%
\pgfsetbuttcap%
\pgfsetroundjoin%
\definecolor{currentfill}{rgb}{0.120638,0.625828,0.533488}%
\pgfsetfillcolor{currentfill}%
\pgfsetfillopacity{0.700000}%
\pgfsetlinewidth{0.501875pt}%
\definecolor{currentstroke}{rgb}{1.000000,1.000000,1.000000}%
\pgfsetstrokecolor{currentstroke}%
\pgfsetstrokeopacity{0.500000}%
\pgfsetdash{}{0pt}%
\pgfpathmoveto{\pgfqpoint{2.352745in}{2.820676in}}%
\pgfpathlineto{\pgfqpoint{2.363924in}{2.830553in}}%
\pgfpathlineto{\pgfqpoint{2.375107in}{2.840117in}}%
\pgfpathlineto{\pgfqpoint{2.386292in}{2.849383in}}%
\pgfpathlineto{\pgfqpoint{2.397481in}{2.858370in}}%
\pgfpathlineto{\pgfqpoint{2.408671in}{2.867098in}}%
\pgfpathlineto{\pgfqpoint{2.402761in}{2.876633in}}%
\pgfpathlineto{\pgfqpoint{2.396855in}{2.886055in}}%
\pgfpathlineto{\pgfqpoint{2.390954in}{2.895386in}}%
\pgfpathlineto{\pgfqpoint{2.385058in}{2.904635in}}%
\pgfpathlineto{\pgfqpoint{2.379167in}{2.913796in}}%
\pgfpathlineto{\pgfqpoint{2.367994in}{2.904891in}}%
\pgfpathlineto{\pgfqpoint{2.356826in}{2.895535in}}%
\pgfpathlineto{\pgfqpoint{2.345664in}{2.885720in}}%
\pgfpathlineto{\pgfqpoint{2.334509in}{2.875436in}}%
\pgfpathlineto{\pgfqpoint{2.323361in}{2.864679in}}%
\pgfpathlineto{\pgfqpoint{2.329224in}{2.856235in}}%
\pgfpathlineto{\pgfqpoint{2.335094in}{2.847634in}}%
\pgfpathlineto{\pgfqpoint{2.340970in}{2.838844in}}%
\pgfpathlineto{\pgfqpoint{2.346854in}{2.829856in}}%
\pgfpathclose%
\pgfusepath{stroke,fill}%
\end{pgfscope}%
\begin{pgfscope}%
\pgfpathrectangle{\pgfqpoint{0.887500in}{0.275000in}}{\pgfqpoint{4.225000in}{4.225000in}}%
\pgfusepath{clip}%
\pgfsetbuttcap%
\pgfsetroundjoin%
\definecolor{currentfill}{rgb}{0.119738,0.603785,0.541400}%
\pgfsetfillcolor{currentfill}%
\pgfsetfillopacity{0.700000}%
\pgfsetlinewidth{0.501875pt}%
\definecolor{currentstroke}{rgb}{1.000000,1.000000,1.000000}%
\pgfsetstrokecolor{currentstroke}%
\pgfsetstrokeopacity{0.500000}%
\pgfsetdash{}{0pt}%
\pgfpathmoveto{\pgfqpoint{2.068676in}{2.794718in}}%
\pgfpathlineto{\pgfqpoint{2.080032in}{2.798455in}}%
\pgfpathlineto{\pgfqpoint{2.091382in}{2.802230in}}%
\pgfpathlineto{\pgfqpoint{2.102729in}{2.805945in}}%
\pgfpathlineto{\pgfqpoint{2.114073in}{2.809503in}}%
\pgfpathlineto{\pgfqpoint{2.125419in}{2.812809in}}%
\pgfpathlineto{\pgfqpoint{2.119630in}{2.820704in}}%
\pgfpathlineto{\pgfqpoint{2.113845in}{2.828614in}}%
\pgfpathlineto{\pgfqpoint{2.108063in}{2.836544in}}%
\pgfpathlineto{\pgfqpoint{2.102284in}{2.844498in}}%
\pgfpathlineto{\pgfqpoint{2.096509in}{2.852473in}}%
\pgfpathlineto{\pgfqpoint{2.085182in}{2.848901in}}%
\pgfpathlineto{\pgfqpoint{2.073854in}{2.845173in}}%
\pgfpathlineto{\pgfqpoint{2.062522in}{2.841363in}}%
\pgfpathlineto{\pgfqpoint{2.051186in}{2.837541in}}%
\pgfpathlineto{\pgfqpoint{2.039843in}{2.833780in}}%
\pgfpathlineto{\pgfqpoint{2.045602in}{2.825993in}}%
\pgfpathlineto{\pgfqpoint{2.051364in}{2.818194in}}%
\pgfpathlineto{\pgfqpoint{2.057131in}{2.810381in}}%
\pgfpathlineto{\pgfqpoint{2.062901in}{2.802556in}}%
\pgfpathclose%
\pgfusepath{stroke,fill}%
\end{pgfscope}%
\begin{pgfscope}%
\pgfpathrectangle{\pgfqpoint{0.887500in}{0.275000in}}{\pgfqpoint{4.225000in}{4.225000in}}%
\pgfusepath{clip}%
\pgfsetbuttcap%
\pgfsetroundjoin%
\definecolor{currentfill}{rgb}{0.468053,0.818921,0.323998}%
\pgfsetfillcolor{currentfill}%
\pgfsetfillopacity{0.700000}%
\pgfsetlinewidth{0.501875pt}%
\definecolor{currentstroke}{rgb}{1.000000,1.000000,1.000000}%
\pgfsetstrokecolor{currentstroke}%
\pgfsetstrokeopacity{0.500000}%
\pgfsetdash{}{0pt}%
\pgfpathmoveto{\pgfqpoint{2.742577in}{3.223096in}}%
\pgfpathlineto{\pgfqpoint{2.753568in}{3.252326in}}%
\pgfpathlineto{\pgfqpoint{2.764558in}{3.282997in}}%
\pgfpathlineto{\pgfqpoint{2.775554in}{3.314196in}}%
\pgfpathlineto{\pgfqpoint{2.786563in}{3.345007in}}%
\pgfpathlineto{\pgfqpoint{2.797590in}{3.374509in}}%
\pgfpathlineto{\pgfqpoint{2.791573in}{3.381627in}}%
\pgfpathlineto{\pgfqpoint{2.785577in}{3.386294in}}%
\pgfpathlineto{\pgfqpoint{2.779605in}{3.388055in}}%
\pgfpathlineto{\pgfqpoint{2.773659in}{3.386843in}}%
\pgfpathlineto{\pgfqpoint{2.767739in}{3.383070in}}%
\pgfpathlineto{\pgfqpoint{2.756669in}{3.364582in}}%
\pgfpathlineto{\pgfqpoint{2.745612in}{3.344899in}}%
\pgfpathlineto{\pgfqpoint{2.734567in}{3.324086in}}%
\pgfpathlineto{\pgfqpoint{2.723536in}{3.302206in}}%
\pgfpathlineto{\pgfqpoint{2.712518in}{3.279322in}}%
\pgfpathlineto{\pgfqpoint{2.718463in}{3.275384in}}%
\pgfpathlineto{\pgfqpoint{2.724441in}{3.267923in}}%
\pgfpathlineto{\pgfqpoint{2.730456in}{3.256498in}}%
\pgfpathlineto{\pgfqpoint{2.736504in}{3.241313in}}%
\pgfpathclose%
\pgfusepath{stroke,fill}%
\end{pgfscope}%
\begin{pgfscope}%
\pgfpathrectangle{\pgfqpoint{0.887500in}{0.275000in}}{\pgfqpoint{4.225000in}{4.225000in}}%
\pgfusepath{clip}%
\pgfsetbuttcap%
\pgfsetroundjoin%
\definecolor{currentfill}{rgb}{0.119699,0.618490,0.536347}%
\pgfsetfillcolor{currentfill}%
\pgfsetfillopacity{0.700000}%
\pgfsetlinewidth{0.501875pt}%
\definecolor{currentstroke}{rgb}{1.000000,1.000000,1.000000}%
\pgfsetstrokecolor{currentstroke}%
\pgfsetstrokeopacity{0.500000}%
\pgfsetdash{}{0pt}%
\pgfpathmoveto{\pgfqpoint{1.840380in}{2.821248in}}%
\pgfpathlineto{\pgfqpoint{1.851800in}{2.824648in}}%
\pgfpathlineto{\pgfqpoint{1.863215in}{2.828059in}}%
\pgfpathlineto{\pgfqpoint{1.874624in}{2.831477in}}%
\pgfpathlineto{\pgfqpoint{1.886028in}{2.834897in}}%
\pgfpathlineto{\pgfqpoint{1.897427in}{2.838311in}}%
\pgfpathlineto{\pgfqpoint{1.891719in}{2.845970in}}%
\pgfpathlineto{\pgfqpoint{1.886016in}{2.853615in}}%
\pgfpathlineto{\pgfqpoint{1.880317in}{2.861247in}}%
\pgfpathlineto{\pgfqpoint{1.874622in}{2.868866in}}%
\pgfpathlineto{\pgfqpoint{1.868931in}{2.876470in}}%
\pgfpathlineto{\pgfqpoint{1.857546in}{2.873027in}}%
\pgfpathlineto{\pgfqpoint{1.846156in}{2.869580in}}%
\pgfpathlineto{\pgfqpoint{1.834761in}{2.866135in}}%
\pgfpathlineto{\pgfqpoint{1.823360in}{2.862699in}}%
\pgfpathlineto{\pgfqpoint{1.811953in}{2.859275in}}%
\pgfpathlineto{\pgfqpoint{1.817630in}{2.851699in}}%
\pgfpathlineto{\pgfqpoint{1.823311in}{2.844107in}}%
\pgfpathlineto{\pgfqpoint{1.828997in}{2.836502in}}%
\pgfpathlineto{\pgfqpoint{1.834686in}{2.828882in}}%
\pgfpathclose%
\pgfusepath{stroke,fill}%
\end{pgfscope}%
\begin{pgfscope}%
\pgfpathrectangle{\pgfqpoint{0.887500in}{0.275000in}}{\pgfqpoint{4.225000in}{4.225000in}}%
\pgfusepath{clip}%
\pgfsetbuttcap%
\pgfsetroundjoin%
\definecolor{currentfill}{rgb}{0.730889,0.871916,0.156029}%
\pgfsetfillcolor{currentfill}%
\pgfsetfillopacity{0.700000}%
\pgfsetlinewidth{0.501875pt}%
\definecolor{currentstroke}{rgb}{1.000000,1.000000,1.000000}%
\pgfsetstrokecolor{currentstroke}%
\pgfsetstrokeopacity{0.500000}%
\pgfsetdash{}{0pt}%
\pgfpathmoveto{\pgfqpoint{2.908935in}{3.514328in}}%
\pgfpathlineto{\pgfqpoint{2.920110in}{3.517127in}}%
\pgfpathlineto{\pgfqpoint{2.931282in}{3.519185in}}%
\pgfpathlineto{\pgfqpoint{2.942450in}{3.520833in}}%
\pgfpathlineto{\pgfqpoint{2.953611in}{3.522399in}}%
\pgfpathlineto{\pgfqpoint{2.964766in}{3.524215in}}%
\pgfpathlineto{\pgfqpoint{2.958711in}{3.525700in}}%
\pgfpathlineto{\pgfqpoint{2.952663in}{3.526735in}}%
\pgfpathlineto{\pgfqpoint{2.946622in}{3.527250in}}%
\pgfpathlineto{\pgfqpoint{2.940589in}{3.527249in}}%
\pgfpathlineto{\pgfqpoint{2.934563in}{3.526813in}}%
\pgfpathlineto{\pgfqpoint{2.923436in}{3.522628in}}%
\pgfpathlineto{\pgfqpoint{2.912306in}{3.518044in}}%
\pgfpathlineto{\pgfqpoint{2.901173in}{3.512968in}}%
\pgfpathlineto{\pgfqpoint{2.890038in}{3.507310in}}%
\pgfpathlineto{\pgfqpoint{2.878903in}{3.500978in}}%
\pgfpathlineto{\pgfqpoint{2.884886in}{3.505727in}}%
\pgfpathlineto{\pgfqpoint{2.890881in}{3.509555in}}%
\pgfpathlineto{\pgfqpoint{2.896888in}{3.512257in}}%
\pgfpathlineto{\pgfqpoint{2.902906in}{3.513807in}}%
\pgfpathclose%
\pgfusepath{stroke,fill}%
\end{pgfscope}%
\begin{pgfscope}%
\pgfpathrectangle{\pgfqpoint{0.887500in}{0.275000in}}{\pgfqpoint{4.225000in}{4.225000in}}%
\pgfusepath{clip}%
\pgfsetbuttcap%
\pgfsetroundjoin%
\definecolor{currentfill}{rgb}{0.855810,0.888601,0.097452}%
\pgfsetfillcolor{currentfill}%
\pgfsetfillopacity{0.700000}%
\pgfsetlinewidth{0.501875pt}%
\definecolor{currentstroke}{rgb}{1.000000,1.000000,1.000000}%
\pgfsetstrokecolor{currentstroke}%
\pgfsetstrokeopacity{0.500000}%
\pgfsetdash{}{0pt}%
\pgfpathmoveto{\pgfqpoint{3.476574in}{3.581954in}}%
\pgfpathlineto{\pgfqpoint{3.487622in}{3.586645in}}%
\pgfpathlineto{\pgfqpoint{3.498666in}{3.591308in}}%
\pgfpathlineto{\pgfqpoint{3.509706in}{3.595950in}}%
\pgfpathlineto{\pgfqpoint{3.520740in}{3.600574in}}%
\pgfpathlineto{\pgfqpoint{3.531771in}{3.605186in}}%
\pgfpathlineto{\pgfqpoint{3.525506in}{3.612084in}}%
\pgfpathlineto{\pgfqpoint{3.519246in}{3.619080in}}%
\pgfpathlineto{\pgfqpoint{3.512992in}{3.626194in}}%
\pgfpathlineto{\pgfqpoint{3.506744in}{3.633383in}}%
\pgfpathlineto{\pgfqpoint{3.500500in}{3.640597in}}%
\pgfpathlineto{\pgfqpoint{3.489476in}{3.635457in}}%
\pgfpathlineto{\pgfqpoint{3.478449in}{3.630312in}}%
\pgfpathlineto{\pgfqpoint{3.467417in}{3.625155in}}%
\pgfpathlineto{\pgfqpoint{3.456381in}{3.619980in}}%
\pgfpathlineto{\pgfqpoint{3.445341in}{3.614780in}}%
\pgfpathlineto{\pgfqpoint{3.451576in}{3.607974in}}%
\pgfpathlineto{\pgfqpoint{3.457816in}{3.601243in}}%
\pgfpathlineto{\pgfqpoint{3.464063in}{3.594643in}}%
\pgfpathlineto{\pgfqpoint{3.470315in}{3.588222in}}%
\pgfpathclose%
\pgfusepath{stroke,fill}%
\end{pgfscope}%
\begin{pgfscope}%
\pgfpathrectangle{\pgfqpoint{0.887500in}{0.275000in}}{\pgfqpoint{4.225000in}{4.225000in}}%
\pgfusepath{clip}%
\pgfsetbuttcap%
\pgfsetroundjoin%
\definecolor{currentfill}{rgb}{0.157851,0.683765,0.501686}%
\pgfsetfillcolor{currentfill}%
\pgfsetfillopacity{0.700000}%
\pgfsetlinewidth{0.501875pt}%
\definecolor{currentstroke}{rgb}{1.000000,1.000000,1.000000}%
\pgfsetstrokecolor{currentstroke}%
\pgfsetstrokeopacity{0.500000}%
\pgfsetdash{}{0pt}%
\pgfpathmoveto{\pgfqpoint{2.691695in}{2.960098in}}%
\pgfpathlineto{\pgfqpoint{2.702885in}{2.966875in}}%
\pgfpathlineto{\pgfqpoint{2.714086in}{2.971900in}}%
\pgfpathlineto{\pgfqpoint{2.725302in}{2.974660in}}%
\pgfpathlineto{\pgfqpoint{2.736529in}{2.975154in}}%
\pgfpathlineto{\pgfqpoint{2.747755in}{2.975051in}}%
\pgfpathlineto{\pgfqpoint{2.741761in}{2.982733in}}%
\pgfpathlineto{\pgfqpoint{2.735756in}{2.992216in}}%
\pgfpathlineto{\pgfqpoint{2.729738in}{3.003666in}}%
\pgfpathlineto{\pgfqpoint{2.723709in}{3.016820in}}%
\pgfpathlineto{\pgfqpoint{2.717672in}{3.031269in}}%
\pgfpathlineto{\pgfqpoint{2.706509in}{3.024653in}}%
\pgfpathlineto{\pgfqpoint{2.695343in}{3.018063in}}%
\pgfpathlineto{\pgfqpoint{2.684183in}{3.010368in}}%
\pgfpathlineto{\pgfqpoint{2.673031in}{3.001592in}}%
\pgfpathlineto{\pgfqpoint{2.661884in}{2.992115in}}%
\pgfpathlineto{\pgfqpoint{2.667840in}{2.985485in}}%
\pgfpathlineto{\pgfqpoint{2.673799in}{2.978960in}}%
\pgfpathlineto{\pgfqpoint{2.679761in}{2.972562in}}%
\pgfpathlineto{\pgfqpoint{2.685726in}{2.966294in}}%
\pgfpathclose%
\pgfusepath{stroke,fill}%
\end{pgfscope}%
\begin{pgfscope}%
\pgfpathrectangle{\pgfqpoint{0.887500in}{0.275000in}}{\pgfqpoint{4.225000in}{4.225000in}}%
\pgfusepath{clip}%
\pgfsetbuttcap%
\pgfsetroundjoin%
\definecolor{currentfill}{rgb}{0.119738,0.603785,0.541400}%
\pgfsetfillcolor{currentfill}%
\pgfsetfillopacity{0.700000}%
\pgfsetlinewidth{0.501875pt}%
\definecolor{currentstroke}{rgb}{1.000000,1.000000,1.000000}%
\pgfsetstrokecolor{currentstroke}%
\pgfsetstrokeopacity{0.500000}%
\pgfsetdash{}{0pt}%
\pgfpathmoveto{\pgfqpoint{2.296788in}{2.773170in}}%
\pgfpathlineto{\pgfqpoint{2.308003in}{2.781594in}}%
\pgfpathlineto{\pgfqpoint{2.319202in}{2.790806in}}%
\pgfpathlineto{\pgfqpoint{2.330388in}{2.800556in}}%
\pgfpathlineto{\pgfqpoint{2.341567in}{2.810596in}}%
\pgfpathlineto{\pgfqpoint{2.352745in}{2.820676in}}%
\pgfpathlineto{\pgfqpoint{2.346854in}{2.829856in}}%
\pgfpathlineto{\pgfqpoint{2.340970in}{2.838844in}}%
\pgfpathlineto{\pgfqpoint{2.335094in}{2.847634in}}%
\pgfpathlineto{\pgfqpoint{2.329224in}{2.856235in}}%
\pgfpathlineto{\pgfqpoint{2.323361in}{2.864679in}}%
\pgfpathlineto{\pgfqpoint{2.312218in}{2.853600in}}%
\pgfpathlineto{\pgfqpoint{2.301074in}{2.842505in}}%
\pgfpathlineto{\pgfqpoint{2.289924in}{2.831709in}}%
\pgfpathlineto{\pgfqpoint{2.278760in}{2.821527in}}%
\pgfpathlineto{\pgfqpoint{2.267577in}{2.812272in}}%
\pgfpathlineto{\pgfqpoint{2.273407in}{2.804663in}}%
\pgfpathlineto{\pgfqpoint{2.279243in}{2.796981in}}%
\pgfpathlineto{\pgfqpoint{2.285085in}{2.789177in}}%
\pgfpathlineto{\pgfqpoint{2.290933in}{2.781236in}}%
\pgfpathclose%
\pgfusepath{stroke,fill}%
\end{pgfscope}%
\begin{pgfscope}%
\pgfpathrectangle{\pgfqpoint{0.887500in}{0.275000in}}{\pgfqpoint{4.225000in}{4.225000in}}%
\pgfusepath{clip}%
\pgfsetbuttcap%
\pgfsetroundjoin%
\definecolor{currentfill}{rgb}{0.835270,0.886029,0.102646}%
\pgfsetfillcolor{currentfill}%
\pgfsetfillopacity{0.700000}%
\pgfsetlinewidth{0.501875pt}%
\definecolor{currentstroke}{rgb}{1.000000,1.000000,1.000000}%
\pgfsetstrokecolor{currentstroke}%
\pgfsetstrokeopacity{0.500000}%
\pgfsetdash{}{0pt}%
\pgfpathmoveto{\pgfqpoint{3.704613in}{3.559373in}}%
\pgfpathlineto{\pgfqpoint{3.715613in}{3.563945in}}%
\pgfpathlineto{\pgfqpoint{3.726612in}{3.568744in}}%
\pgfpathlineto{\pgfqpoint{3.737609in}{3.573692in}}%
\pgfpathlineto{\pgfqpoint{3.748604in}{3.578714in}}%
\pgfpathlineto{\pgfqpoint{3.759594in}{3.583733in}}%
\pgfpathlineto{\pgfqpoint{3.753310in}{3.594185in}}%
\pgfpathlineto{\pgfqpoint{3.747026in}{3.604389in}}%
\pgfpathlineto{\pgfqpoint{3.740741in}{3.614345in}}%
\pgfpathlineto{\pgfqpoint{3.734455in}{3.624063in}}%
\pgfpathlineto{\pgfqpoint{3.728170in}{3.633551in}}%
\pgfpathlineto{\pgfqpoint{3.717188in}{3.628485in}}%
\pgfpathlineto{\pgfqpoint{3.706201in}{3.623410in}}%
\pgfpathlineto{\pgfqpoint{3.695211in}{3.618378in}}%
\pgfpathlineto{\pgfqpoint{3.684219in}{3.613441in}}%
\pgfpathlineto{\pgfqpoint{3.673224in}{3.608653in}}%
\pgfpathlineto{\pgfqpoint{3.679502in}{3.599283in}}%
\pgfpathlineto{\pgfqpoint{3.685780in}{3.589682in}}%
\pgfpathlineto{\pgfqpoint{3.692058in}{3.579836in}}%
\pgfpathlineto{\pgfqpoint{3.698336in}{3.569733in}}%
\pgfpathclose%
\pgfusepath{stroke,fill}%
\end{pgfscope}%
\begin{pgfscope}%
\pgfpathrectangle{\pgfqpoint{0.887500in}{0.275000in}}{\pgfqpoint{4.225000in}{4.225000in}}%
\pgfusepath{clip}%
\pgfsetbuttcap%
\pgfsetroundjoin%
\definecolor{currentfill}{rgb}{0.835270,0.886029,0.102646}%
\pgfsetfillcolor{currentfill}%
\pgfsetfillopacity{0.700000}%
\pgfsetlinewidth{0.501875pt}%
\definecolor{currentstroke}{rgb}{1.000000,1.000000,1.000000}%
\pgfsetstrokecolor{currentstroke}%
\pgfsetstrokeopacity{0.500000}%
\pgfsetdash{}{0pt}%
\pgfpathmoveto{\pgfqpoint{3.334764in}{3.570759in}}%
\pgfpathlineto{\pgfqpoint{3.345833in}{3.573821in}}%
\pgfpathlineto{\pgfqpoint{3.356899in}{3.577162in}}%
\pgfpathlineto{\pgfqpoint{3.367962in}{3.580869in}}%
\pgfpathlineto{\pgfqpoint{3.379023in}{3.584987in}}%
\pgfpathlineto{\pgfqpoint{3.390083in}{3.589461in}}%
\pgfpathlineto{\pgfqpoint{3.383862in}{3.595834in}}%
\pgfpathlineto{\pgfqpoint{3.377646in}{3.602190in}}%
\pgfpathlineto{\pgfqpoint{3.371433in}{3.608473in}}%
\pgfpathlineto{\pgfqpoint{3.365224in}{3.614629in}}%
\pgfpathlineto{\pgfqpoint{3.359019in}{3.620602in}}%
\pgfpathlineto{\pgfqpoint{3.347968in}{3.615497in}}%
\pgfpathlineto{\pgfqpoint{3.336915in}{3.610628in}}%
\pgfpathlineto{\pgfqpoint{3.325860in}{3.606035in}}%
\pgfpathlineto{\pgfqpoint{3.314802in}{3.601709in}}%
\pgfpathlineto{\pgfqpoint{3.303740in}{3.597616in}}%
\pgfpathlineto{\pgfqpoint{3.309937in}{3.592477in}}%
\pgfpathlineto{\pgfqpoint{3.316138in}{3.587197in}}%
\pgfpathlineto{\pgfqpoint{3.322342in}{3.581802in}}%
\pgfpathlineto{\pgfqpoint{3.328551in}{3.576314in}}%
\pgfpathclose%
\pgfusepath{stroke,fill}%
\end{pgfscope}%
\begin{pgfscope}%
\pgfpathrectangle{\pgfqpoint{0.887500in}{0.275000in}}{\pgfqpoint{4.225000in}{4.225000in}}%
\pgfusepath{clip}%
\pgfsetbuttcap%
\pgfsetroundjoin%
\definecolor{currentfill}{rgb}{0.128087,0.647749,0.523491}%
\pgfsetfillcolor{currentfill}%
\pgfsetfillopacity{0.700000}%
\pgfsetlinewidth{0.501875pt}%
\definecolor{currentstroke}{rgb}{1.000000,1.000000,1.000000}%
\pgfsetstrokecolor{currentstroke}%
\pgfsetstrokeopacity{0.500000}%
\pgfsetdash{}{0pt}%
\pgfpathmoveto{\pgfqpoint{2.494188in}{2.868190in}}%
\pgfpathlineto{\pgfqpoint{2.505353in}{2.878854in}}%
\pgfpathlineto{\pgfqpoint{2.516521in}{2.889133in}}%
\pgfpathlineto{\pgfqpoint{2.527697in}{2.898785in}}%
\pgfpathlineto{\pgfqpoint{2.538882in}{2.907570in}}%
\pgfpathlineto{\pgfqpoint{2.550078in}{2.915292in}}%
\pgfpathlineto{\pgfqpoint{2.544180in}{2.920332in}}%
\pgfpathlineto{\pgfqpoint{2.538290in}{2.925069in}}%
\pgfpathlineto{\pgfqpoint{2.532407in}{2.929734in}}%
\pgfpathlineto{\pgfqpoint{2.526525in}{2.934559in}}%
\pgfpathlineto{\pgfqpoint{2.520644in}{2.939774in}}%
\pgfpathlineto{\pgfqpoint{2.509438in}{2.934157in}}%
\pgfpathlineto{\pgfqpoint{2.498235in}{2.928081in}}%
\pgfpathlineto{\pgfqpoint{2.487034in}{2.921560in}}%
\pgfpathlineto{\pgfqpoint{2.475836in}{2.914639in}}%
\pgfpathlineto{\pgfqpoint{2.464639in}{2.907365in}}%
\pgfpathlineto{\pgfqpoint{2.470540in}{2.899603in}}%
\pgfpathlineto{\pgfqpoint{2.476443in}{2.891940in}}%
\pgfpathlineto{\pgfqpoint{2.482351in}{2.884242in}}%
\pgfpathlineto{\pgfqpoint{2.488266in}{2.876370in}}%
\pgfpathclose%
\pgfusepath{stroke,fill}%
\end{pgfscope}%
\begin{pgfscope}%
\pgfpathrectangle{\pgfqpoint{0.887500in}{0.275000in}}{\pgfqpoint{4.225000in}{4.225000in}}%
\pgfusepath{clip}%
\pgfsetbuttcap%
\pgfsetroundjoin%
\definecolor{currentfill}{rgb}{0.783315,0.879285,0.125405}%
\pgfsetfillcolor{currentfill}%
\pgfsetfillopacity{0.700000}%
\pgfsetlinewidth{0.501875pt}%
\definecolor{currentstroke}{rgb}{1.000000,1.000000,1.000000}%
\pgfsetstrokecolor{currentstroke}%
\pgfsetstrokeopacity{0.500000}%
\pgfsetdash{}{0pt}%
\pgfpathmoveto{\pgfqpoint{3.050906in}{3.534804in}}%
\pgfpathlineto{\pgfqpoint{3.062046in}{3.540575in}}%
\pgfpathlineto{\pgfqpoint{3.073182in}{3.546156in}}%
\pgfpathlineto{\pgfqpoint{3.084314in}{3.551306in}}%
\pgfpathlineto{\pgfqpoint{3.095441in}{3.555999in}}%
\pgfpathlineto{\pgfqpoint{3.106563in}{3.560320in}}%
\pgfpathlineto{\pgfqpoint{3.100437in}{3.561668in}}%
\pgfpathlineto{\pgfqpoint{3.094317in}{3.562721in}}%
\pgfpathlineto{\pgfqpoint{3.088203in}{3.563535in}}%
\pgfpathlineto{\pgfqpoint{3.082094in}{3.564163in}}%
\pgfpathlineto{\pgfqpoint{3.075992in}{3.564660in}}%
\pgfpathlineto{\pgfqpoint{3.064891in}{3.559965in}}%
\pgfpathlineto{\pgfqpoint{3.053785in}{3.555286in}}%
\pgfpathlineto{\pgfqpoint{3.042675in}{3.550609in}}%
\pgfpathlineto{\pgfqpoint{3.031561in}{3.545959in}}%
\pgfpathlineto{\pgfqpoint{3.020442in}{3.541429in}}%
\pgfpathlineto{\pgfqpoint{3.026523in}{3.540419in}}%
\pgfpathlineto{\pgfqpoint{3.032609in}{3.539270in}}%
\pgfpathlineto{\pgfqpoint{3.038702in}{3.537963in}}%
\pgfpathlineto{\pgfqpoint{3.044801in}{3.536480in}}%
\pgfpathclose%
\pgfusepath{stroke,fill}%
\end{pgfscope}%
\begin{pgfscope}%
\pgfpathrectangle{\pgfqpoint{0.887500in}{0.275000in}}{\pgfqpoint{4.225000in}{4.225000in}}%
\pgfusepath{clip}%
\pgfsetbuttcap%
\pgfsetroundjoin%
\definecolor{currentfill}{rgb}{0.120565,0.596422,0.543611}%
\pgfsetfillcolor{currentfill}%
\pgfsetfillopacity{0.700000}%
\pgfsetlinewidth{0.501875pt}%
\definecolor{currentstroke}{rgb}{1.000000,1.000000,1.000000}%
\pgfsetstrokecolor{currentstroke}%
\pgfsetstrokeopacity{0.500000}%
\pgfsetdash{}{0pt}%
\pgfpathmoveto{\pgfqpoint{2.154416in}{2.773391in}}%
\pgfpathlineto{\pgfqpoint{2.165774in}{2.776424in}}%
\pgfpathlineto{\pgfqpoint{2.177136in}{2.779059in}}%
\pgfpathlineto{\pgfqpoint{2.188499in}{2.781333in}}%
\pgfpathlineto{\pgfqpoint{2.199858in}{2.783529in}}%
\pgfpathlineto{\pgfqpoint{2.211206in}{2.785960in}}%
\pgfpathlineto{\pgfqpoint{2.205394in}{2.793551in}}%
\pgfpathlineto{\pgfqpoint{2.199584in}{2.801182in}}%
\pgfpathlineto{\pgfqpoint{2.193777in}{2.808885in}}%
\pgfpathlineto{\pgfqpoint{2.187972in}{2.816693in}}%
\pgfpathlineto{\pgfqpoint{2.182167in}{2.824637in}}%
\pgfpathlineto{\pgfqpoint{2.170827in}{2.822388in}}%
\pgfpathlineto{\pgfqpoint{2.159475in}{2.820381in}}%
\pgfpathlineto{\pgfqpoint{2.148119in}{2.818276in}}%
\pgfpathlineto{\pgfqpoint{2.136767in}{2.815765in}}%
\pgfpathlineto{\pgfqpoint{2.125419in}{2.812809in}}%
\pgfpathlineto{\pgfqpoint{2.131211in}{2.804924in}}%
\pgfpathlineto{\pgfqpoint{2.137006in}{2.797045in}}%
\pgfpathlineto{\pgfqpoint{2.142806in}{2.789166in}}%
\pgfpathlineto{\pgfqpoint{2.148609in}{2.781283in}}%
\pgfpathclose%
\pgfusepath{stroke,fill}%
\end{pgfscope}%
\begin{pgfscope}%
\pgfpathrectangle{\pgfqpoint{0.887500in}{0.275000in}}{\pgfqpoint{4.225000in}{4.225000in}}%
\pgfusepath{clip}%
\pgfsetbuttcap%
\pgfsetroundjoin%
\definecolor{currentfill}{rgb}{0.121380,0.629492,0.531973}%
\pgfsetfillcolor{currentfill}%
\pgfsetfillopacity{0.700000}%
\pgfsetlinewidth{0.501875pt}%
\definecolor{currentstroke}{rgb}{1.000000,1.000000,1.000000}%
\pgfsetstrokecolor{currentstroke}%
\pgfsetstrokeopacity{0.500000}%
\pgfsetdash{}{0pt}%
\pgfpathmoveto{\pgfqpoint{1.612041in}{2.846494in}}%
\pgfpathlineto{\pgfqpoint{1.623520in}{2.849824in}}%
\pgfpathlineto{\pgfqpoint{1.634993in}{2.853155in}}%
\pgfpathlineto{\pgfqpoint{1.646461in}{2.856486in}}%
\pgfpathlineto{\pgfqpoint{1.657923in}{2.859820in}}%
\pgfpathlineto{\pgfqpoint{1.669379in}{2.863156in}}%
\pgfpathlineto{\pgfqpoint{1.663755in}{2.870606in}}%
\pgfpathlineto{\pgfqpoint{1.658135in}{2.878042in}}%
\pgfpathlineto{\pgfqpoint{1.652519in}{2.885464in}}%
\pgfpathlineto{\pgfqpoint{1.646907in}{2.892872in}}%
\pgfpathlineto{\pgfqpoint{1.635463in}{2.889505in}}%
\pgfpathlineto{\pgfqpoint{1.624012in}{2.886138in}}%
\pgfpathlineto{\pgfqpoint{1.612557in}{2.882769in}}%
\pgfpathlineto{\pgfqpoint{1.601096in}{2.879399in}}%
\pgfpathlineto{\pgfqpoint{1.589629in}{2.876029in}}%
\pgfpathlineto{\pgfqpoint{1.595226in}{2.868671in}}%
\pgfpathlineto{\pgfqpoint{1.600826in}{2.861296in}}%
\pgfpathlineto{\pgfqpoint{1.606431in}{2.853904in}}%
\pgfpathclose%
\pgfusepath{stroke,fill}%
\end{pgfscope}%
\begin{pgfscope}%
\pgfpathrectangle{\pgfqpoint{0.887500in}{0.275000in}}{\pgfqpoint{4.225000in}{4.225000in}}%
\pgfusepath{clip}%
\pgfsetbuttcap%
\pgfsetroundjoin%
\definecolor{currentfill}{rgb}{0.824940,0.884720,0.106217}%
\pgfsetfillcolor{currentfill}%
\pgfsetfillopacity{0.700000}%
\pgfsetlinewidth{0.501875pt}%
\definecolor{currentstroke}{rgb}{1.000000,1.000000,1.000000}%
\pgfsetstrokecolor{currentstroke}%
\pgfsetstrokeopacity{0.500000}%
\pgfsetdash{}{0pt}%
\pgfpathmoveto{\pgfqpoint{3.192888in}{3.563497in}}%
\pgfpathlineto{\pgfqpoint{3.203995in}{3.566606in}}%
\pgfpathlineto{\pgfqpoint{3.215097in}{3.569761in}}%
\pgfpathlineto{\pgfqpoint{3.226194in}{3.572967in}}%
\pgfpathlineto{\pgfqpoint{3.237286in}{3.576230in}}%
\pgfpathlineto{\pgfqpoint{3.248373in}{3.579556in}}%
\pgfpathlineto{\pgfqpoint{3.242192in}{3.583693in}}%
\pgfpathlineto{\pgfqpoint{3.236015in}{3.587694in}}%
\pgfpathlineto{\pgfqpoint{3.229842in}{3.591562in}}%
\pgfpathlineto{\pgfqpoint{3.223675in}{3.595287in}}%
\pgfpathlineto{\pgfqpoint{3.217512in}{3.598856in}}%
\pgfpathlineto{\pgfqpoint{3.206438in}{3.594810in}}%
\pgfpathlineto{\pgfqpoint{3.195360in}{3.590833in}}%
\pgfpathlineto{\pgfqpoint{3.184277in}{3.586923in}}%
\pgfpathlineto{\pgfqpoint{3.173190in}{3.583078in}}%
\pgfpathlineto{\pgfqpoint{3.162098in}{3.579296in}}%
\pgfpathlineto{\pgfqpoint{3.168246in}{3.576721in}}%
\pgfpathlineto{\pgfqpoint{3.174400in}{3.573871in}}%
\pgfpathlineto{\pgfqpoint{3.180558in}{3.570722in}}%
\pgfpathlineto{\pgfqpoint{3.186721in}{3.567262in}}%
\pgfpathclose%
\pgfusepath{stroke,fill}%
\end{pgfscope}%
\begin{pgfscope}%
\pgfpathrectangle{\pgfqpoint{0.887500in}{0.275000in}}{\pgfqpoint{4.225000in}{4.225000in}}%
\pgfusepath{clip}%
\pgfsetbuttcap%
\pgfsetroundjoin%
\definecolor{currentfill}{rgb}{0.804182,0.882046,0.114965}%
\pgfsetfillcolor{currentfill}%
\pgfsetfillopacity{0.700000}%
\pgfsetlinewidth{0.501875pt}%
\definecolor{currentstroke}{rgb}{1.000000,1.000000,1.000000}%
\pgfsetstrokecolor{currentstroke}%
\pgfsetstrokeopacity{0.500000}%
\pgfsetdash{}{0pt}%
\pgfpathmoveto{\pgfqpoint{3.791009in}{3.528522in}}%
\pgfpathlineto{\pgfqpoint{3.801999in}{3.533313in}}%
\pgfpathlineto{\pgfqpoint{3.812982in}{3.538009in}}%
\pgfpathlineto{\pgfqpoint{3.823959in}{3.542621in}}%
\pgfpathlineto{\pgfqpoint{3.834930in}{3.547159in}}%
\pgfpathlineto{\pgfqpoint{3.845895in}{3.551633in}}%
\pgfpathlineto{\pgfqpoint{3.839609in}{3.563196in}}%
\pgfpathlineto{\pgfqpoint{3.833322in}{3.574595in}}%
\pgfpathlineto{\pgfqpoint{3.827035in}{3.585811in}}%
\pgfpathlineto{\pgfqpoint{3.820748in}{3.596828in}}%
\pgfpathlineto{\pgfqpoint{3.814460in}{3.607626in}}%
\pgfpathlineto{\pgfqpoint{3.803498in}{3.602991in}}%
\pgfpathlineto{\pgfqpoint{3.792530in}{3.598296in}}%
\pgfpathlineto{\pgfqpoint{3.781557in}{3.593528in}}%
\pgfpathlineto{\pgfqpoint{3.770579in}{3.588677in}}%
\pgfpathlineto{\pgfqpoint{3.759594in}{3.583733in}}%
\pgfpathlineto{\pgfqpoint{3.765877in}{3.573053in}}%
\pgfpathlineto{\pgfqpoint{3.772160in}{3.562168in}}%
\pgfpathlineto{\pgfqpoint{3.778442in}{3.551103in}}%
\pgfpathlineto{\pgfqpoint{3.784726in}{3.539880in}}%
\pgfpathclose%
\pgfusepath{stroke,fill}%
\end{pgfscope}%
\begin{pgfscope}%
\pgfpathrectangle{\pgfqpoint{0.887500in}{0.275000in}}{\pgfqpoint{4.225000in}{4.225000in}}%
\pgfusepath{clip}%
\pgfsetbuttcap%
\pgfsetroundjoin%
\definecolor{currentfill}{rgb}{0.119423,0.611141,0.538982}%
\pgfsetfillcolor{currentfill}%
\pgfsetfillopacity{0.700000}%
\pgfsetlinewidth{0.501875pt}%
\definecolor{currentstroke}{rgb}{1.000000,1.000000,1.000000}%
\pgfsetstrokecolor{currentstroke}%
\pgfsetstrokeopacity{0.500000}%
\pgfsetdash{}{0pt}%
\pgfpathmoveto{\pgfqpoint{1.926024in}{2.799829in}}%
\pgfpathlineto{\pgfqpoint{1.937431in}{2.803206in}}%
\pgfpathlineto{\pgfqpoint{1.948833in}{2.806565in}}%
\pgfpathlineto{\pgfqpoint{1.960230in}{2.809899in}}%
\pgfpathlineto{\pgfqpoint{1.971622in}{2.813207in}}%
\pgfpathlineto{\pgfqpoint{1.983010in}{2.816508in}}%
\pgfpathlineto{\pgfqpoint{1.977268in}{2.824258in}}%
\pgfpathlineto{\pgfqpoint{1.971531in}{2.831996in}}%
\pgfpathlineto{\pgfqpoint{1.965798in}{2.839722in}}%
\pgfpathlineto{\pgfqpoint{1.960069in}{2.847436in}}%
\pgfpathlineto{\pgfqpoint{1.954344in}{2.855137in}}%
\pgfpathlineto{\pgfqpoint{1.942971in}{2.851805in}}%
\pgfpathlineto{\pgfqpoint{1.931592in}{2.848466in}}%
\pgfpathlineto{\pgfqpoint{1.920208in}{2.845102in}}%
\pgfpathlineto{\pgfqpoint{1.908820in}{2.841715in}}%
\pgfpathlineto{\pgfqpoint{1.897427in}{2.838311in}}%
\pgfpathlineto{\pgfqpoint{1.903138in}{2.830640in}}%
\pgfpathlineto{\pgfqpoint{1.908853in}{2.822955in}}%
\pgfpathlineto{\pgfqpoint{1.914573in}{2.815259in}}%
\pgfpathlineto{\pgfqpoint{1.920296in}{2.807550in}}%
\pgfpathclose%
\pgfusepath{stroke,fill}%
\end{pgfscope}%
\begin{pgfscope}%
\pgfpathrectangle{\pgfqpoint{0.887500in}{0.275000in}}{\pgfqpoint{4.225000in}{4.225000in}}%
\pgfusepath{clip}%
\pgfsetbuttcap%
\pgfsetroundjoin%
\definecolor{currentfill}{rgb}{0.762373,0.876424,0.137064}%
\pgfsetfillcolor{currentfill}%
\pgfsetfillopacity{0.700000}%
\pgfsetlinewidth{0.501875pt}%
\definecolor{currentstroke}{rgb}{1.000000,1.000000,1.000000}%
\pgfsetstrokecolor{currentstroke}%
\pgfsetstrokeopacity{0.500000}%
\pgfsetdash{}{0pt}%
\pgfpathmoveto{\pgfqpoint{3.877337in}{3.491986in}}%
\pgfpathlineto{\pgfqpoint{3.888298in}{3.496191in}}%
\pgfpathlineto{\pgfqpoint{3.899254in}{3.500385in}}%
\pgfpathlineto{\pgfqpoint{3.910206in}{3.504574in}}%
\pgfpathlineto{\pgfqpoint{3.921153in}{3.508766in}}%
\pgfpathlineto{\pgfqpoint{3.932095in}{3.512969in}}%
\pgfpathlineto{\pgfqpoint{3.925803in}{3.525229in}}%
\pgfpathlineto{\pgfqpoint{3.919512in}{3.537422in}}%
\pgfpathlineto{\pgfqpoint{3.913222in}{3.549529in}}%
\pgfpathlineto{\pgfqpoint{3.906932in}{3.561530in}}%
\pgfpathlineto{\pgfqpoint{3.900643in}{3.573407in}}%
\pgfpathlineto{\pgfqpoint{3.889704in}{3.569097in}}%
\pgfpathlineto{\pgfqpoint{3.878759in}{3.564775in}}%
\pgfpathlineto{\pgfqpoint{3.867810in}{3.560431in}}%
\pgfpathlineto{\pgfqpoint{3.856855in}{3.556053in}}%
\pgfpathlineto{\pgfqpoint{3.845895in}{3.551633in}}%
\pgfpathlineto{\pgfqpoint{3.852182in}{3.539924in}}%
\pgfpathlineto{\pgfqpoint{3.858469in}{3.528087in}}%
\pgfpathlineto{\pgfqpoint{3.864757in}{3.516139in}}%
\pgfpathlineto{\pgfqpoint{3.871046in}{3.504100in}}%
\pgfpathclose%
\pgfusepath{stroke,fill}%
\end{pgfscope}%
\begin{pgfscope}%
\pgfpathrectangle{\pgfqpoint{0.887500in}{0.275000in}}{\pgfqpoint{4.225000in}{4.225000in}}%
\pgfusepath{clip}%
\pgfsetbuttcap%
\pgfsetroundjoin%
\definecolor{currentfill}{rgb}{0.120081,0.622161,0.534946}%
\pgfsetfillcolor{currentfill}%
\pgfsetfillopacity{0.700000}%
\pgfsetlinewidth{0.501875pt}%
\definecolor{currentstroke}{rgb}{1.000000,1.000000,1.000000}%
\pgfsetstrokecolor{currentstroke}%
\pgfsetstrokeopacity{0.500000}%
\pgfsetdash{}{0pt}%
\pgfpathmoveto{\pgfqpoint{1.697563in}{2.825701in}}%
\pgfpathlineto{\pgfqpoint{1.709028in}{2.829014in}}%
\pgfpathlineto{\pgfqpoint{1.720488in}{2.832334in}}%
\pgfpathlineto{\pgfqpoint{1.731941in}{2.835662in}}%
\pgfpathlineto{\pgfqpoint{1.743389in}{2.839000in}}%
\pgfpathlineto{\pgfqpoint{1.754831in}{2.842349in}}%
\pgfpathlineto{\pgfqpoint{1.749172in}{2.849891in}}%
\pgfpathlineto{\pgfqpoint{1.743517in}{2.857419in}}%
\pgfpathlineto{\pgfqpoint{1.737866in}{2.864933in}}%
\pgfpathlineto{\pgfqpoint{1.732220in}{2.872433in}}%
\pgfpathlineto{\pgfqpoint{1.726577in}{2.879919in}}%
\pgfpathlineto{\pgfqpoint{1.715149in}{2.876552in}}%
\pgfpathlineto{\pgfqpoint{1.703715in}{2.873194in}}%
\pgfpathlineto{\pgfqpoint{1.692276in}{2.869842in}}%
\pgfpathlineto{\pgfqpoint{1.680830in}{2.866497in}}%
\pgfpathlineto{\pgfqpoint{1.669379in}{2.863156in}}%
\pgfpathlineto{\pgfqpoint{1.675008in}{2.855693in}}%
\pgfpathlineto{\pgfqpoint{1.680640in}{2.848216in}}%
\pgfpathlineto{\pgfqpoint{1.686277in}{2.840725in}}%
\pgfpathlineto{\pgfqpoint{1.691918in}{2.833220in}}%
\pgfpathclose%
\pgfusepath{stroke,fill}%
\end{pgfscope}%
\begin{pgfscope}%
\pgfpathrectangle{\pgfqpoint{0.887500in}{0.275000in}}{\pgfqpoint{4.225000in}{4.225000in}}%
\pgfusepath{clip}%
\pgfsetbuttcap%
\pgfsetroundjoin%
\definecolor{currentfill}{rgb}{0.855810,0.888601,0.097452}%
\pgfsetfillcolor{currentfill}%
\pgfsetfillopacity{0.700000}%
\pgfsetlinewidth{0.501875pt}%
\definecolor{currentstroke}{rgb}{1.000000,1.000000,1.000000}%
\pgfsetstrokecolor{currentstroke}%
\pgfsetstrokeopacity{0.500000}%
\pgfsetdash{}{0pt}%
\pgfpathmoveto{\pgfqpoint{3.563152in}{3.569840in}}%
\pgfpathlineto{\pgfqpoint{3.574179in}{3.573593in}}%
\pgfpathlineto{\pgfqpoint{3.585200in}{3.577258in}}%
\pgfpathlineto{\pgfqpoint{3.596216in}{3.580855in}}%
\pgfpathlineto{\pgfqpoint{3.607226in}{3.584437in}}%
\pgfpathlineto{\pgfqpoint{3.618232in}{3.588061in}}%
\pgfpathlineto{\pgfqpoint{3.611953in}{3.596462in}}%
\pgfpathlineto{\pgfqpoint{3.605676in}{3.604647in}}%
\pgfpathlineto{\pgfqpoint{3.599400in}{3.612652in}}%
\pgfpathlineto{\pgfqpoint{3.593126in}{3.620516in}}%
\pgfpathlineto{\pgfqpoint{3.586855in}{3.628276in}}%
\pgfpathlineto{\pgfqpoint{3.575847in}{3.623620in}}%
\pgfpathlineto{\pgfqpoint{3.564834in}{3.618996in}}%
\pgfpathlineto{\pgfqpoint{3.553817in}{3.614390in}}%
\pgfpathlineto{\pgfqpoint{3.542796in}{3.609789in}}%
\pgfpathlineto{\pgfqpoint{3.531771in}{3.605186in}}%
\pgfpathlineto{\pgfqpoint{3.538040in}{3.598320in}}%
\pgfpathlineto{\pgfqpoint{3.544314in}{3.591419in}}%
\pgfpathlineto{\pgfqpoint{3.550591in}{3.584416in}}%
\pgfpathlineto{\pgfqpoint{3.556871in}{3.577246in}}%
\pgfpathclose%
\pgfusepath{stroke,fill}%
\end{pgfscope}%
\begin{pgfscope}%
\pgfpathrectangle{\pgfqpoint{0.887500in}{0.275000in}}{\pgfqpoint{4.225000in}{4.225000in}}%
\pgfusepath{clip}%
\pgfsetbuttcap%
\pgfsetroundjoin%
\definecolor{currentfill}{rgb}{0.121380,0.629492,0.531973}%
\pgfsetfillcolor{currentfill}%
\pgfsetfillopacity{0.700000}%
\pgfsetlinewidth{0.501875pt}%
\definecolor{currentstroke}{rgb}{1.000000,1.000000,1.000000}%
\pgfsetstrokecolor{currentstroke}%
\pgfsetstrokeopacity{0.500000}%
\pgfsetdash{}{0pt}%
\pgfpathmoveto{\pgfqpoint{2.438315in}{2.816863in}}%
\pgfpathlineto{\pgfqpoint{2.449504in}{2.826392in}}%
\pgfpathlineto{\pgfqpoint{2.460685in}{2.836307in}}%
\pgfpathlineto{\pgfqpoint{2.471858in}{2.846677in}}%
\pgfpathlineto{\pgfqpoint{2.483024in}{2.857384in}}%
\pgfpathlineto{\pgfqpoint{2.494188in}{2.868190in}}%
\pgfpathlineto{\pgfqpoint{2.488266in}{2.876370in}}%
\pgfpathlineto{\pgfqpoint{2.482351in}{2.884242in}}%
\pgfpathlineto{\pgfqpoint{2.476443in}{2.891940in}}%
\pgfpathlineto{\pgfqpoint{2.470540in}{2.899603in}}%
\pgfpathlineto{\pgfqpoint{2.464639in}{2.907365in}}%
\pgfpathlineto{\pgfqpoint{2.453444in}{2.899780in}}%
\pgfpathlineto{\pgfqpoint{2.442250in}{2.891930in}}%
\pgfpathlineto{\pgfqpoint{2.431056in}{2.883860in}}%
\pgfpathlineto{\pgfqpoint{2.419863in}{2.875588in}}%
\pgfpathlineto{\pgfqpoint{2.408671in}{2.867098in}}%
\pgfpathlineto{\pgfqpoint{2.414588in}{2.857424in}}%
\pgfpathlineto{\pgfqpoint{2.420510in}{2.847588in}}%
\pgfpathlineto{\pgfqpoint{2.426438in}{2.837565in}}%
\pgfpathlineto{\pgfqpoint{2.432373in}{2.827332in}}%
\pgfpathclose%
\pgfusepath{stroke,fill}%
\end{pgfscope}%
\begin{pgfscope}%
\pgfpathrectangle{\pgfqpoint{0.887500in}{0.275000in}}{\pgfqpoint{4.225000in}{4.225000in}}%
\pgfusepath{clip}%
\pgfsetbuttcap%
\pgfsetroundjoin%
\definecolor{currentfill}{rgb}{0.709898,0.868751,0.169257}%
\pgfsetfillcolor{currentfill}%
\pgfsetfillopacity{0.700000}%
\pgfsetlinewidth{0.501875pt}%
\definecolor{currentstroke}{rgb}{1.000000,1.000000,1.000000}%
\pgfsetstrokecolor{currentstroke}%
\pgfsetstrokeopacity{0.500000}%
\pgfsetdash{}{0pt}%
\pgfpathmoveto{\pgfqpoint{3.963587in}{3.451061in}}%
\pgfpathlineto{\pgfqpoint{3.974531in}{3.455294in}}%
\pgfpathlineto{\pgfqpoint{3.985471in}{3.459600in}}%
\pgfpathlineto{\pgfqpoint{3.996409in}{3.463981in}}%
\pgfpathlineto{\pgfqpoint{4.007344in}{3.468441in}}%
\pgfpathlineto{\pgfqpoint{4.001036in}{3.480834in}}%
\pgfpathlineto{\pgfqpoint{3.994730in}{3.493183in}}%
\pgfpathlineto{\pgfqpoint{3.988426in}{3.505493in}}%
\pgfpathlineto{\pgfqpoint{3.982124in}{3.517772in}}%
\pgfpathlineto{\pgfqpoint{3.975825in}{3.530026in}}%
\pgfpathlineto{\pgfqpoint{3.964898in}{3.525710in}}%
\pgfpathlineto{\pgfqpoint{3.953968in}{3.521434in}}%
\pgfpathlineto{\pgfqpoint{3.943034in}{3.517189in}}%
\pgfpathlineto{\pgfqpoint{3.932095in}{3.512969in}}%
\pgfpathlineto{\pgfqpoint{3.938390in}{3.500655in}}%
\pgfpathlineto{\pgfqpoint{3.944686in}{3.488297in}}%
\pgfpathlineto{\pgfqpoint{3.950984in}{3.475905in}}%
\pgfpathlineto{\pgfqpoint{3.957284in}{3.463490in}}%
\pgfpathclose%
\pgfusepath{stroke,fill}%
\end{pgfscope}%
\begin{pgfscope}%
\pgfpathrectangle{\pgfqpoint{0.887500in}{0.275000in}}{\pgfqpoint{4.225000in}{4.225000in}}%
\pgfusepath{clip}%
\pgfsetbuttcap%
\pgfsetroundjoin%
\definecolor{currentfill}{rgb}{0.121831,0.589055,0.545623}%
\pgfsetfillcolor{currentfill}%
\pgfsetfillopacity{0.700000}%
\pgfsetlinewidth{0.501875pt}%
\definecolor{currentstroke}{rgb}{1.000000,1.000000,1.000000}%
\pgfsetstrokecolor{currentstroke}%
\pgfsetstrokeopacity{0.500000}%
\pgfsetdash{}{0pt}%
\pgfpathmoveto{\pgfqpoint{2.240334in}{2.747596in}}%
\pgfpathlineto{\pgfqpoint{2.251672in}{2.750817in}}%
\pgfpathlineto{\pgfqpoint{2.262990in}{2.754754in}}%
\pgfpathlineto{\pgfqpoint{2.274284in}{2.759659in}}%
\pgfpathlineto{\pgfqpoint{2.285550in}{2.765782in}}%
\pgfpathlineto{\pgfqpoint{2.296788in}{2.773170in}}%
\pgfpathlineto{\pgfqpoint{2.290933in}{2.781236in}}%
\pgfpathlineto{\pgfqpoint{2.285085in}{2.789177in}}%
\pgfpathlineto{\pgfqpoint{2.279243in}{2.796981in}}%
\pgfpathlineto{\pgfqpoint{2.273407in}{2.804663in}}%
\pgfpathlineto{\pgfqpoint{2.267577in}{2.812272in}}%
\pgfpathlineto{\pgfqpoint{2.256367in}{2.804260in}}%
\pgfpathlineto{\pgfqpoint{2.245122in}{2.797778in}}%
\pgfpathlineto{\pgfqpoint{2.233844in}{2.792773in}}%
\pgfpathlineto{\pgfqpoint{2.222537in}{2.788937in}}%
\pgfpathlineto{\pgfqpoint{2.211206in}{2.785960in}}%
\pgfpathlineto{\pgfqpoint{2.217022in}{2.778377in}}%
\pgfpathlineto{\pgfqpoint{2.222842in}{2.770770in}}%
\pgfpathlineto{\pgfqpoint{2.228668in}{2.763107in}}%
\pgfpathlineto{\pgfqpoint{2.234498in}{2.755380in}}%
\pgfpathclose%
\pgfusepath{stroke,fill}%
\end{pgfscope}%
\begin{pgfscope}%
\pgfpathrectangle{\pgfqpoint{0.887500in}{0.275000in}}{\pgfqpoint{4.225000in}{4.225000in}}%
\pgfusepath{clip}%
\pgfsetbuttcap%
\pgfsetroundjoin%
\definecolor{currentfill}{rgb}{0.146616,0.673050,0.508936}%
\pgfsetfillcolor{currentfill}%
\pgfsetfillopacity{0.700000}%
\pgfsetlinewidth{0.501875pt}%
\definecolor{currentstroke}{rgb}{1.000000,1.000000,1.000000}%
\pgfsetstrokecolor{currentstroke}%
\pgfsetstrokeopacity{0.500000}%
\pgfsetdash{}{0pt}%
\pgfpathmoveto{\pgfqpoint{2.635786in}{2.917751in}}%
\pgfpathlineto{\pgfqpoint{2.646977in}{2.925749in}}%
\pgfpathlineto{\pgfqpoint{2.658160in}{2.934395in}}%
\pgfpathlineto{\pgfqpoint{2.669337in}{2.943344in}}%
\pgfpathlineto{\pgfqpoint{2.680514in}{2.952083in}}%
\pgfpathlineto{\pgfqpoint{2.691695in}{2.960098in}}%
\pgfpathlineto{\pgfqpoint{2.685726in}{2.966294in}}%
\pgfpathlineto{\pgfqpoint{2.679761in}{2.972562in}}%
\pgfpathlineto{\pgfqpoint{2.673799in}{2.978960in}}%
\pgfpathlineto{\pgfqpoint{2.667840in}{2.985485in}}%
\pgfpathlineto{\pgfqpoint{2.661884in}{2.992115in}}%
\pgfpathlineto{\pgfqpoint{2.650738in}{2.982315in}}%
\pgfpathlineto{\pgfqpoint{2.639589in}{2.972571in}}%
\pgfpathlineto{\pgfqpoint{2.628434in}{2.963261in}}%
\pgfpathlineto{\pgfqpoint{2.617269in}{2.954763in}}%
\pgfpathlineto{\pgfqpoint{2.606089in}{2.947309in}}%
\pgfpathlineto{\pgfqpoint{2.612007in}{2.942544in}}%
\pgfpathlineto{\pgfqpoint{2.617934in}{2.937328in}}%
\pgfpathlineto{\pgfqpoint{2.623873in}{2.931507in}}%
\pgfpathlineto{\pgfqpoint{2.629823in}{2.924970in}}%
\pgfpathclose%
\pgfusepath{stroke,fill}%
\end{pgfscope}%
\begin{pgfscope}%
\pgfpathrectangle{\pgfqpoint{0.887500in}{0.275000in}}{\pgfqpoint{4.225000in}{4.225000in}}%
\pgfusepath{clip}%
\pgfsetbuttcap%
\pgfsetroundjoin%
\definecolor{currentfill}{rgb}{0.709898,0.868751,0.169257}%
\pgfsetfillcolor{currentfill}%
\pgfsetfillopacity{0.700000}%
\pgfsetlinewidth{0.501875pt}%
\definecolor{currentstroke}{rgb}{1.000000,1.000000,1.000000}%
\pgfsetstrokecolor{currentstroke}%
\pgfsetstrokeopacity{0.500000}%
\pgfsetdash{}{0pt}%
\pgfpathmoveto{\pgfqpoint{2.853088in}{3.477697in}}%
\pgfpathlineto{\pgfqpoint{2.864244in}{3.489132in}}%
\pgfpathlineto{\pgfqpoint{2.875409in}{3.498187in}}%
\pgfpathlineto{\pgfqpoint{2.886582in}{3.505187in}}%
\pgfpathlineto{\pgfqpoint{2.897758in}{3.510459in}}%
\pgfpathlineto{\pgfqpoint{2.908935in}{3.514328in}}%
\pgfpathlineto{\pgfqpoint{2.902906in}{3.513807in}}%
\pgfpathlineto{\pgfqpoint{2.896888in}{3.512257in}}%
\pgfpathlineto{\pgfqpoint{2.890881in}{3.509555in}}%
\pgfpathlineto{\pgfqpoint{2.884886in}{3.505727in}}%
\pgfpathlineto{\pgfqpoint{2.878903in}{3.500978in}}%
\pgfpathlineto{\pgfqpoint{2.867767in}{3.493881in}}%
\pgfpathlineto{\pgfqpoint{2.856632in}{3.485929in}}%
\pgfpathlineto{\pgfqpoint{2.845500in}{3.477043in}}%
\pgfpathlineto{\pgfqpoint{2.834371in}{3.467153in}}%
\pgfpathlineto{\pgfqpoint{2.823246in}{3.456192in}}%
\pgfpathlineto{\pgfqpoint{2.829182in}{3.463700in}}%
\pgfpathlineto{\pgfqpoint{2.835133in}{3.469782in}}%
\pgfpathlineto{\pgfqpoint{2.841102in}{3.474128in}}%
\pgfpathlineto{\pgfqpoint{2.847087in}{3.476702in}}%
\pgfpathclose%
\pgfusepath{stroke,fill}%
\end{pgfscope}%
\begin{pgfscope}%
\pgfpathrectangle{\pgfqpoint{0.887500in}{0.275000in}}{\pgfqpoint{4.225000in}{4.225000in}}%
\pgfusepath{clip}%
\pgfsetbuttcap%
\pgfsetroundjoin%
\definecolor{currentfill}{rgb}{0.404001,0.800275,0.362552}%
\pgfsetfillcolor{currentfill}%
\pgfsetfillopacity{0.700000}%
\pgfsetlinewidth{0.501875pt}%
\definecolor{currentstroke}{rgb}{1.000000,1.000000,1.000000}%
\pgfsetstrokecolor{currentstroke}%
\pgfsetstrokeopacity{0.500000}%
\pgfsetdash{}{0pt}%
\pgfpathmoveto{\pgfqpoint{2.773093in}{3.112566in}}%
\pgfpathlineto{\pgfqpoint{2.784059in}{3.147077in}}%
\pgfpathlineto{\pgfqpoint{2.795001in}{3.187042in}}%
\pgfpathlineto{\pgfqpoint{2.805938in}{3.230362in}}%
\pgfpathlineto{\pgfqpoint{2.816885in}{3.274924in}}%
\pgfpathlineto{\pgfqpoint{2.827854in}{3.318601in}}%
\pgfpathlineto{\pgfqpoint{2.821789in}{3.330924in}}%
\pgfpathlineto{\pgfqpoint{2.815727in}{3.343146in}}%
\pgfpathlineto{\pgfqpoint{2.809671in}{3.354798in}}%
\pgfpathlineto{\pgfqpoint{2.803624in}{3.365409in}}%
\pgfpathlineto{\pgfqpoint{2.797590in}{3.374509in}}%
\pgfpathlineto{\pgfqpoint{2.786563in}{3.345007in}}%
\pgfpathlineto{\pgfqpoint{2.775554in}{3.314196in}}%
\pgfpathlineto{\pgfqpoint{2.764558in}{3.282997in}}%
\pgfpathlineto{\pgfqpoint{2.753568in}{3.252326in}}%
\pgfpathlineto{\pgfqpoint{2.742577in}{3.223096in}}%
\pgfpathlineto{\pgfqpoint{2.748669in}{3.202593in}}%
\pgfpathlineto{\pgfqpoint{2.754772in}{3.180547in}}%
\pgfpathlineto{\pgfqpoint{2.760880in}{3.157700in}}%
\pgfpathlineto{\pgfqpoint{2.766989in}{3.134793in}}%
\pgfpathclose%
\pgfusepath{stroke,fill}%
\end{pgfscope}%
\begin{pgfscope}%
\pgfpathrectangle{\pgfqpoint{0.887500in}{0.275000in}}{\pgfqpoint{4.225000in}{4.225000in}}%
\pgfusepath{clip}%
\pgfsetbuttcap%
\pgfsetroundjoin%
\definecolor{currentfill}{rgb}{0.626579,0.854645,0.223353}%
\pgfsetfillcolor{currentfill}%
\pgfsetfillopacity{0.700000}%
\pgfsetlinewidth{0.501875pt}%
\definecolor{currentstroke}{rgb}{1.000000,1.000000,1.000000}%
\pgfsetstrokecolor{currentstroke}%
\pgfsetstrokeopacity{0.500000}%
\pgfsetdash{}{0pt}%
\pgfpathmoveto{\pgfqpoint{2.797590in}{3.374509in}}%
\pgfpathlineto{\pgfqpoint{2.808641in}{3.401776in}}%
\pgfpathlineto{\pgfqpoint{2.819717in}{3.425888in}}%
\pgfpathlineto{\pgfqpoint{2.830820in}{3.446396in}}%
\pgfpathlineto{\pgfqpoint{2.841945in}{3.463559in}}%
\pgfpathlineto{\pgfqpoint{2.853088in}{3.477697in}}%
\pgfpathlineto{\pgfqpoint{2.847087in}{3.476702in}}%
\pgfpathlineto{\pgfqpoint{2.841102in}{3.474128in}}%
\pgfpathlineto{\pgfqpoint{2.835133in}{3.469782in}}%
\pgfpathlineto{\pgfqpoint{2.829182in}{3.463700in}}%
\pgfpathlineto{\pgfqpoint{2.823246in}{3.456192in}}%
\pgfpathlineto{\pgfqpoint{2.812128in}{3.444093in}}%
\pgfpathlineto{\pgfqpoint{2.801016in}{3.430788in}}%
\pgfpathlineto{\pgfqpoint{2.789913in}{3.416211in}}%
\pgfpathlineto{\pgfqpoint{2.778821in}{3.400300in}}%
\pgfpathlineto{\pgfqpoint{2.767739in}{3.383070in}}%
\pgfpathlineto{\pgfqpoint{2.773659in}{3.386843in}}%
\pgfpathlineto{\pgfqpoint{2.779605in}{3.388055in}}%
\pgfpathlineto{\pgfqpoint{2.785577in}{3.386294in}}%
\pgfpathlineto{\pgfqpoint{2.791573in}{3.381627in}}%
\pgfpathclose%
\pgfusepath{stroke,fill}%
\end{pgfscope}%
\begin{pgfscope}%
\pgfpathrectangle{\pgfqpoint{0.887500in}{0.275000in}}{\pgfqpoint{4.225000in}{4.225000in}}%
\pgfusepath{clip}%
\pgfsetbuttcap%
\pgfsetroundjoin%
\definecolor{currentfill}{rgb}{0.119738,0.603785,0.541400}%
\pgfsetfillcolor{currentfill}%
\pgfsetfillopacity{0.700000}%
\pgfsetlinewidth{0.501875pt}%
\definecolor{currentstroke}{rgb}{1.000000,1.000000,1.000000}%
\pgfsetstrokecolor{currentstroke}%
\pgfsetstrokeopacity{0.500000}%
\pgfsetdash{}{0pt}%
\pgfpathmoveto{\pgfqpoint{2.011776in}{2.777577in}}%
\pgfpathlineto{\pgfqpoint{2.023170in}{2.780866in}}%
\pgfpathlineto{\pgfqpoint{2.034558in}{2.784199in}}%
\pgfpathlineto{\pgfqpoint{2.045939in}{2.787601in}}%
\pgfpathlineto{\pgfqpoint{2.057312in}{2.791099in}}%
\pgfpathlineto{\pgfqpoint{2.068676in}{2.794718in}}%
\pgfpathlineto{\pgfqpoint{2.062901in}{2.802556in}}%
\pgfpathlineto{\pgfqpoint{2.057131in}{2.810381in}}%
\pgfpathlineto{\pgfqpoint{2.051364in}{2.818194in}}%
\pgfpathlineto{\pgfqpoint{2.045602in}{2.825993in}}%
\pgfpathlineto{\pgfqpoint{2.039843in}{2.833780in}}%
\pgfpathlineto{\pgfqpoint{2.028492in}{2.830140in}}%
\pgfpathlineto{\pgfqpoint{2.017133in}{2.826618in}}%
\pgfpathlineto{\pgfqpoint{2.005765in}{2.823188in}}%
\pgfpathlineto{\pgfqpoint{1.994391in}{2.819827in}}%
\pgfpathlineto{\pgfqpoint{1.983010in}{2.816508in}}%
\pgfpathlineto{\pgfqpoint{1.988755in}{2.808746in}}%
\pgfpathlineto{\pgfqpoint{1.994504in}{2.800972in}}%
\pgfpathlineto{\pgfqpoint{2.000257in}{2.793186in}}%
\pgfpathlineto{\pgfqpoint{2.006014in}{2.785388in}}%
\pgfpathclose%
\pgfusepath{stroke,fill}%
\end{pgfscope}%
\begin{pgfscope}%
\pgfpathrectangle{\pgfqpoint{0.887500in}{0.275000in}}{\pgfqpoint{4.225000in}{4.225000in}}%
\pgfusepath{clip}%
\pgfsetbuttcap%
\pgfsetroundjoin%
\definecolor{currentfill}{rgb}{0.835270,0.886029,0.102646}%
\pgfsetfillcolor{currentfill}%
\pgfsetfillopacity{0.700000}%
\pgfsetlinewidth{0.501875pt}%
\definecolor{currentstroke}{rgb}{1.000000,1.000000,1.000000}%
\pgfsetstrokecolor{currentstroke}%
\pgfsetstrokeopacity{0.500000}%
\pgfsetdash{}{0pt}%
\pgfpathmoveto{\pgfqpoint{3.421270in}{3.559167in}}%
\pgfpathlineto{\pgfqpoint{3.432337in}{3.563429in}}%
\pgfpathlineto{\pgfqpoint{3.443401in}{3.567906in}}%
\pgfpathlineto{\pgfqpoint{3.454463in}{3.572531in}}%
\pgfpathlineto{\pgfqpoint{3.465520in}{3.577236in}}%
\pgfpathlineto{\pgfqpoint{3.476574in}{3.581954in}}%
\pgfpathlineto{\pgfqpoint{3.470315in}{3.588222in}}%
\pgfpathlineto{\pgfqpoint{3.464063in}{3.594643in}}%
\pgfpathlineto{\pgfqpoint{3.457816in}{3.601243in}}%
\pgfpathlineto{\pgfqpoint{3.451576in}{3.607974in}}%
\pgfpathlineto{\pgfqpoint{3.445341in}{3.614780in}}%
\pgfpathlineto{\pgfqpoint{3.434296in}{3.609552in}}%
\pgfpathlineto{\pgfqpoint{3.423248in}{3.604337in}}%
\pgfpathlineto{\pgfqpoint{3.412195in}{3.599205in}}%
\pgfpathlineto{\pgfqpoint{3.401140in}{3.594223in}}%
\pgfpathlineto{\pgfqpoint{3.390083in}{3.589461in}}%
\pgfpathlineto{\pgfqpoint{3.396309in}{3.583126in}}%
\pgfpathlineto{\pgfqpoint{3.402540in}{3.576884in}}%
\pgfpathlineto{\pgfqpoint{3.408777in}{3.570790in}}%
\pgfpathlineto{\pgfqpoint{3.415020in}{3.564894in}}%
\pgfpathclose%
\pgfusepath{stroke,fill}%
\end{pgfscope}%
\begin{pgfscope}%
\pgfpathrectangle{\pgfqpoint{0.887500in}{0.275000in}}{\pgfqpoint{4.225000in}{4.225000in}}%
\pgfusepath{clip}%
\pgfsetbuttcap%
\pgfsetroundjoin%
\definecolor{currentfill}{rgb}{0.166383,0.690856,0.496502}%
\pgfsetfillcolor{currentfill}%
\pgfsetfillopacity{0.700000}%
\pgfsetlinewidth{0.501875pt}%
\definecolor{currentstroke}{rgb}{1.000000,1.000000,1.000000}%
\pgfsetstrokecolor{currentstroke}%
\pgfsetstrokeopacity{0.500000}%
\pgfsetdash{}{0pt}%
\pgfpathmoveto{\pgfqpoint{2.777659in}{2.955053in}}%
\pgfpathlineto{\pgfqpoint{2.788866in}{2.958813in}}%
\pgfpathlineto{\pgfqpoint{2.800048in}{2.965707in}}%
\pgfpathlineto{\pgfqpoint{2.811195in}{2.977599in}}%
\pgfpathlineto{\pgfqpoint{2.822302in}{2.996355in}}%
\pgfpathlineto{\pgfqpoint{2.833364in}{3.023849in}}%
\pgfpathlineto{\pgfqpoint{2.827377in}{3.022841in}}%
\pgfpathlineto{\pgfqpoint{2.821389in}{3.023158in}}%
\pgfpathlineto{\pgfqpoint{2.815396in}{3.025164in}}%
\pgfpathlineto{\pgfqpoint{2.809397in}{3.029220in}}%
\pgfpathlineto{\pgfqpoint{2.803388in}{3.035691in}}%
\pgfpathlineto{\pgfqpoint{2.792360in}{3.008723in}}%
\pgfpathlineto{\pgfqpoint{2.781275in}{2.991184in}}%
\pgfpathlineto{\pgfqpoint{2.770140in}{2.981063in}}%
\pgfpathlineto{\pgfqpoint{2.758963in}{2.976353in}}%
\pgfpathlineto{\pgfqpoint{2.747755in}{2.975051in}}%
\pgfpathlineto{\pgfqpoint{2.753742in}{2.968923in}}%
\pgfpathlineto{\pgfqpoint{2.759723in}{2.964104in}}%
\pgfpathlineto{\pgfqpoint{2.765702in}{2.960350in}}%
\pgfpathlineto{\pgfqpoint{2.771680in}{2.957415in}}%
\pgfpathclose%
\pgfusepath{stroke,fill}%
\end{pgfscope}%
\begin{pgfscope}%
\pgfpathrectangle{\pgfqpoint{0.887500in}{0.275000in}}{\pgfqpoint{4.225000in}{4.225000in}}%
\pgfusepath{clip}%
\pgfsetbuttcap%
\pgfsetroundjoin%
\definecolor{currentfill}{rgb}{0.119699,0.618490,0.536347}%
\pgfsetfillcolor{currentfill}%
\pgfsetfillopacity{0.700000}%
\pgfsetlinewidth{0.501875pt}%
\definecolor{currentstroke}{rgb}{1.000000,1.000000,1.000000}%
\pgfsetstrokecolor{currentstroke}%
\pgfsetstrokeopacity{0.500000}%
\pgfsetdash{}{0pt}%
\pgfpathmoveto{\pgfqpoint{1.783189in}{2.804432in}}%
\pgfpathlineto{\pgfqpoint{1.794639in}{2.807771in}}%
\pgfpathlineto{\pgfqpoint{1.806083in}{2.811123in}}%
\pgfpathlineto{\pgfqpoint{1.817521in}{2.814486in}}%
\pgfpathlineto{\pgfqpoint{1.828953in}{2.817861in}}%
\pgfpathlineto{\pgfqpoint{1.840380in}{2.821248in}}%
\pgfpathlineto{\pgfqpoint{1.834686in}{2.828882in}}%
\pgfpathlineto{\pgfqpoint{1.828997in}{2.836502in}}%
\pgfpathlineto{\pgfqpoint{1.823311in}{2.844107in}}%
\pgfpathlineto{\pgfqpoint{1.817630in}{2.851699in}}%
\pgfpathlineto{\pgfqpoint{1.811953in}{2.859275in}}%
\pgfpathlineto{\pgfqpoint{1.800540in}{2.855865in}}%
\pgfpathlineto{\pgfqpoint{1.789122in}{2.852467in}}%
\pgfpathlineto{\pgfqpoint{1.777698in}{2.849083in}}%
\pgfpathlineto{\pgfqpoint{1.766267in}{2.845710in}}%
\pgfpathlineto{\pgfqpoint{1.754831in}{2.842349in}}%
\pgfpathlineto{\pgfqpoint{1.760494in}{2.834794in}}%
\pgfpathlineto{\pgfqpoint{1.766162in}{2.827224in}}%
\pgfpathlineto{\pgfqpoint{1.771833in}{2.819641in}}%
\pgfpathlineto{\pgfqpoint{1.777509in}{2.812043in}}%
\pgfpathclose%
\pgfusepath{stroke,fill}%
\end{pgfscope}%
\begin{pgfscope}%
\pgfpathrectangle{\pgfqpoint{0.887500in}{0.275000in}}{\pgfqpoint{4.225000in}{4.225000in}}%
\pgfusepath{clip}%
\pgfsetbuttcap%
\pgfsetroundjoin%
\definecolor{currentfill}{rgb}{0.824940,0.884720,0.106217}%
\pgfsetfillcolor{currentfill}%
\pgfsetfillopacity{0.700000}%
\pgfsetlinewidth{0.501875pt}%
\definecolor{currentstroke}{rgb}{1.000000,1.000000,1.000000}%
\pgfsetstrokecolor{currentstroke}%
\pgfsetstrokeopacity{0.500000}%
\pgfsetdash{}{0pt}%
\pgfpathmoveto{\pgfqpoint{3.649619in}{3.541528in}}%
\pgfpathlineto{\pgfqpoint{3.660619in}{3.544509in}}%
\pgfpathlineto{\pgfqpoint{3.671617in}{3.547710in}}%
\pgfpathlineto{\pgfqpoint{3.682615in}{3.551215in}}%
\pgfpathlineto{\pgfqpoint{3.693613in}{3.555105in}}%
\pgfpathlineto{\pgfqpoint{3.704613in}{3.559373in}}%
\pgfpathlineto{\pgfqpoint{3.698336in}{3.569733in}}%
\pgfpathlineto{\pgfqpoint{3.692058in}{3.579836in}}%
\pgfpathlineto{\pgfqpoint{3.685780in}{3.589682in}}%
\pgfpathlineto{\pgfqpoint{3.679502in}{3.599283in}}%
\pgfpathlineto{\pgfqpoint{3.673224in}{3.608653in}}%
\pgfpathlineto{\pgfqpoint{3.662228in}{3.604065in}}%
\pgfpathlineto{\pgfqpoint{3.651231in}{3.599729in}}%
\pgfpathlineto{\pgfqpoint{3.640233in}{3.595652in}}%
\pgfpathlineto{\pgfqpoint{3.629234in}{3.591781in}}%
\pgfpathlineto{\pgfqpoint{3.618232in}{3.588061in}}%
\pgfpathlineto{\pgfqpoint{3.624511in}{3.579406in}}%
\pgfpathlineto{\pgfqpoint{3.630790in}{3.570461in}}%
\pgfpathlineto{\pgfqpoint{3.637068in}{3.561187in}}%
\pgfpathlineto{\pgfqpoint{3.643345in}{3.551547in}}%
\pgfpathclose%
\pgfusepath{stroke,fill}%
\end{pgfscope}%
\begin{pgfscope}%
\pgfpathrectangle{\pgfqpoint{0.887500in}{0.275000in}}{\pgfqpoint{4.225000in}{4.225000in}}%
\pgfusepath{clip}%
\pgfsetbuttcap%
\pgfsetroundjoin%
\definecolor{currentfill}{rgb}{0.119423,0.611141,0.538982}%
\pgfsetfillcolor{currentfill}%
\pgfsetfillopacity{0.700000}%
\pgfsetlinewidth{0.501875pt}%
\definecolor{currentstroke}{rgb}{1.000000,1.000000,1.000000}%
\pgfsetstrokecolor{currentstroke}%
\pgfsetstrokeopacity{0.500000}%
\pgfsetdash{}{0pt}%
\pgfpathmoveto{\pgfqpoint{2.382293in}{2.772169in}}%
\pgfpathlineto{\pgfqpoint{2.393504in}{2.780985in}}%
\pgfpathlineto{\pgfqpoint{2.404713in}{2.789783in}}%
\pgfpathlineto{\pgfqpoint{2.415919in}{2.798641in}}%
\pgfpathlineto{\pgfqpoint{2.427120in}{2.807641in}}%
\pgfpathlineto{\pgfqpoint{2.438315in}{2.816863in}}%
\pgfpathlineto{\pgfqpoint{2.432373in}{2.827332in}}%
\pgfpathlineto{\pgfqpoint{2.426438in}{2.837565in}}%
\pgfpathlineto{\pgfqpoint{2.420510in}{2.847588in}}%
\pgfpathlineto{\pgfqpoint{2.414588in}{2.857424in}}%
\pgfpathlineto{\pgfqpoint{2.408671in}{2.867098in}}%
\pgfpathlineto{\pgfqpoint{2.397481in}{2.858370in}}%
\pgfpathlineto{\pgfqpoint{2.386292in}{2.849383in}}%
\pgfpathlineto{\pgfqpoint{2.375107in}{2.840117in}}%
\pgfpathlineto{\pgfqpoint{2.363924in}{2.830553in}}%
\pgfpathlineto{\pgfqpoint{2.352745in}{2.820676in}}%
\pgfpathlineto{\pgfqpoint{2.358642in}{2.811312in}}%
\pgfpathlineto{\pgfqpoint{2.364546in}{2.801771in}}%
\pgfpathlineto{\pgfqpoint{2.370455in}{2.792062in}}%
\pgfpathlineto{\pgfqpoint{2.376371in}{2.782192in}}%
\pgfpathclose%
\pgfusepath{stroke,fill}%
\end{pgfscope}%
\begin{pgfscope}%
\pgfpathrectangle{\pgfqpoint{0.887500in}{0.275000in}}{\pgfqpoint{4.225000in}{4.225000in}}%
\pgfusepath{clip}%
\pgfsetbuttcap%
\pgfsetroundjoin%
\definecolor{currentfill}{rgb}{0.762373,0.876424,0.137064}%
\pgfsetfillcolor{currentfill}%
\pgfsetfillopacity{0.700000}%
\pgfsetlinewidth{0.501875pt}%
\definecolor{currentstroke}{rgb}{1.000000,1.000000,1.000000}%
\pgfsetstrokecolor{currentstroke}%
\pgfsetstrokeopacity{0.500000}%
\pgfsetdash{}{0pt}%
\pgfpathmoveto{\pgfqpoint{2.995137in}{3.512429in}}%
\pgfpathlineto{\pgfqpoint{3.006302in}{3.515155in}}%
\pgfpathlineto{\pgfqpoint{3.017460in}{3.519018in}}%
\pgfpathlineto{\pgfqpoint{3.028613in}{3.523759in}}%
\pgfpathlineto{\pgfqpoint{3.039762in}{3.529110in}}%
\pgfpathlineto{\pgfqpoint{3.050906in}{3.534804in}}%
\pgfpathlineto{\pgfqpoint{3.044801in}{3.536480in}}%
\pgfpathlineto{\pgfqpoint{3.038702in}{3.537963in}}%
\pgfpathlineto{\pgfqpoint{3.032609in}{3.539270in}}%
\pgfpathlineto{\pgfqpoint{3.026523in}{3.540419in}}%
\pgfpathlineto{\pgfqpoint{3.020442in}{3.541429in}}%
\pgfpathlineto{\pgfqpoint{3.009318in}{3.537124in}}%
\pgfpathlineto{\pgfqpoint{2.998189in}{3.533147in}}%
\pgfpathlineto{\pgfqpoint{2.987055in}{3.529602in}}%
\pgfpathlineto{\pgfqpoint{2.975914in}{3.526591in}}%
\pgfpathlineto{\pgfqpoint{2.964766in}{3.524215in}}%
\pgfpathlineto{\pgfqpoint{2.970828in}{3.522346in}}%
\pgfpathlineto{\pgfqpoint{2.976896in}{3.520164in}}%
\pgfpathlineto{\pgfqpoint{2.982971in}{3.517738in}}%
\pgfpathlineto{\pgfqpoint{2.989051in}{3.515136in}}%
\pgfpathclose%
\pgfusepath{stroke,fill}%
\end{pgfscope}%
\begin{pgfscope}%
\pgfpathrectangle{\pgfqpoint{0.887500in}{0.275000in}}{\pgfqpoint{4.225000in}{4.225000in}}%
\pgfusepath{clip}%
\pgfsetbuttcap%
\pgfsetroundjoin%
\definecolor{currentfill}{rgb}{0.835270,0.886029,0.102646}%
\pgfsetfillcolor{currentfill}%
\pgfsetfillopacity{0.700000}%
\pgfsetlinewidth{0.501875pt}%
\definecolor{currentstroke}{rgb}{1.000000,1.000000,1.000000}%
\pgfsetstrokecolor{currentstroke}%
\pgfsetstrokeopacity{0.500000}%
\pgfsetdash{}{0pt}%
\pgfpathmoveto{\pgfqpoint{3.279347in}{3.556621in}}%
\pgfpathlineto{\pgfqpoint{3.290442in}{3.559570in}}%
\pgfpathlineto{\pgfqpoint{3.301531in}{3.562382in}}%
\pgfpathlineto{\pgfqpoint{3.312614in}{3.565127in}}%
\pgfpathlineto{\pgfqpoint{3.323692in}{3.567890in}}%
\pgfpathlineto{\pgfqpoint{3.334764in}{3.570759in}}%
\pgfpathlineto{\pgfqpoint{3.328551in}{3.576314in}}%
\pgfpathlineto{\pgfqpoint{3.322342in}{3.581802in}}%
\pgfpathlineto{\pgfqpoint{3.316138in}{3.587197in}}%
\pgfpathlineto{\pgfqpoint{3.309937in}{3.592477in}}%
\pgfpathlineto{\pgfqpoint{3.303740in}{3.597616in}}%
\pgfpathlineto{\pgfqpoint{3.292675in}{3.593725in}}%
\pgfpathlineto{\pgfqpoint{3.281606in}{3.590006in}}%
\pgfpathlineto{\pgfqpoint{3.270533in}{3.586425in}}%
\pgfpathlineto{\pgfqpoint{3.259456in}{3.582951in}}%
\pgfpathlineto{\pgfqpoint{3.248373in}{3.579556in}}%
\pgfpathlineto{\pgfqpoint{3.254560in}{3.575278in}}%
\pgfpathlineto{\pgfqpoint{3.260750in}{3.570851in}}%
\pgfpathlineto{\pgfqpoint{3.266945in}{3.566270in}}%
\pgfpathlineto{\pgfqpoint{3.273144in}{3.561529in}}%
\pgfpathclose%
\pgfusepath{stroke,fill}%
\end{pgfscope}%
\begin{pgfscope}%
\pgfpathrectangle{\pgfqpoint{0.887500in}{0.275000in}}{\pgfqpoint{4.225000in}{4.225000in}}%
\pgfusepath{clip}%
\pgfsetbuttcap%
\pgfsetroundjoin%
\definecolor{currentfill}{rgb}{0.134692,0.658636,0.517649}%
\pgfsetfillcolor{currentfill}%
\pgfsetfillopacity{0.700000}%
\pgfsetlinewidth{0.501875pt}%
\definecolor{currentstroke}{rgb}{1.000000,1.000000,1.000000}%
\pgfsetstrokecolor{currentstroke}%
\pgfsetstrokeopacity{0.500000}%
\pgfsetdash{}{0pt}%
\pgfpathmoveto{\pgfqpoint{2.579782in}{2.878088in}}%
\pgfpathlineto{\pgfqpoint{2.590976in}{2.887031in}}%
\pgfpathlineto{\pgfqpoint{2.602178in}{2.895153in}}%
\pgfpathlineto{\pgfqpoint{2.613383in}{2.902770in}}%
\pgfpathlineto{\pgfqpoint{2.624586in}{2.910197in}}%
\pgfpathlineto{\pgfqpoint{2.635786in}{2.917751in}}%
\pgfpathlineto{\pgfqpoint{2.629823in}{2.924970in}}%
\pgfpathlineto{\pgfqpoint{2.623873in}{2.931507in}}%
\pgfpathlineto{\pgfqpoint{2.617934in}{2.937328in}}%
\pgfpathlineto{\pgfqpoint{2.612007in}{2.942544in}}%
\pgfpathlineto{\pgfqpoint{2.606089in}{2.947309in}}%
\pgfpathlineto{\pgfqpoint{2.594897in}{2.940657in}}%
\pgfpathlineto{\pgfqpoint{2.583696in}{2.934468in}}%
\pgfpathlineto{\pgfqpoint{2.572489in}{2.928403in}}%
\pgfpathlineto{\pgfqpoint{2.561282in}{2.922124in}}%
\pgfpathlineto{\pgfqpoint{2.550078in}{2.915292in}}%
\pgfpathlineto{\pgfqpoint{2.555987in}{2.909720in}}%
\pgfpathlineto{\pgfqpoint{2.561911in}{2.903383in}}%
\pgfpathlineto{\pgfqpoint{2.567851in}{2.896050in}}%
\pgfpathlineto{\pgfqpoint{2.573808in}{2.887578in}}%
\pgfpathclose%
\pgfusepath{stroke,fill}%
\end{pgfscope}%
\begin{pgfscope}%
\pgfpathrectangle{\pgfqpoint{0.887500in}{0.275000in}}{\pgfqpoint{4.225000in}{4.225000in}}%
\pgfusepath{clip}%
\pgfsetbuttcap%
\pgfsetroundjoin%
\definecolor{currentfill}{rgb}{0.793760,0.880678,0.120005}%
\pgfsetfillcolor{currentfill}%
\pgfsetfillopacity{0.700000}%
\pgfsetlinewidth{0.501875pt}%
\definecolor{currentstroke}{rgb}{1.000000,1.000000,1.000000}%
\pgfsetstrokecolor{currentstroke}%
\pgfsetstrokeopacity{0.500000}%
\pgfsetdash{}{0pt}%
\pgfpathmoveto{\pgfqpoint{3.736004in}{3.504866in}}%
\pgfpathlineto{\pgfqpoint{3.747009in}{3.509312in}}%
\pgfpathlineto{\pgfqpoint{3.758013in}{3.513975in}}%
\pgfpathlineto{\pgfqpoint{3.769015in}{3.518781in}}%
\pgfpathlineto{\pgfqpoint{3.780015in}{3.523655in}}%
\pgfpathlineto{\pgfqpoint{3.791009in}{3.528522in}}%
\pgfpathlineto{\pgfqpoint{3.784726in}{3.539880in}}%
\pgfpathlineto{\pgfqpoint{3.778442in}{3.551103in}}%
\pgfpathlineto{\pgfqpoint{3.772160in}{3.562168in}}%
\pgfpathlineto{\pgfqpoint{3.765877in}{3.573053in}}%
\pgfpathlineto{\pgfqpoint{3.759594in}{3.583733in}}%
\pgfpathlineto{\pgfqpoint{3.748604in}{3.578714in}}%
\pgfpathlineto{\pgfqpoint{3.737609in}{3.573692in}}%
\pgfpathlineto{\pgfqpoint{3.726612in}{3.568744in}}%
\pgfpathlineto{\pgfqpoint{3.715613in}{3.563945in}}%
\pgfpathlineto{\pgfqpoint{3.704613in}{3.559373in}}%
\pgfpathlineto{\pgfqpoint{3.710890in}{3.548786in}}%
\pgfpathlineto{\pgfqpoint{3.717167in}{3.538007in}}%
\pgfpathlineto{\pgfqpoint{3.723444in}{3.527070in}}%
\pgfpathlineto{\pgfqpoint{3.729723in}{3.516012in}}%
\pgfpathclose%
\pgfusepath{stroke,fill}%
\end{pgfscope}%
\begin{pgfscope}%
\pgfpathrectangle{\pgfqpoint{0.887500in}{0.275000in}}{\pgfqpoint{4.225000in}{4.225000in}}%
\pgfusepath{clip}%
\pgfsetbuttcap%
\pgfsetroundjoin%
\definecolor{currentfill}{rgb}{0.120565,0.596422,0.543611}%
\pgfsetfillcolor{currentfill}%
\pgfsetfillopacity{0.700000}%
\pgfsetlinewidth{0.501875pt}%
\definecolor{currentstroke}{rgb}{1.000000,1.000000,1.000000}%
\pgfsetstrokecolor{currentstroke}%
\pgfsetstrokeopacity{0.500000}%
\pgfsetdash{}{0pt}%
\pgfpathmoveto{\pgfqpoint{2.097610in}{2.755334in}}%
\pgfpathlineto{\pgfqpoint{2.108979in}{2.759037in}}%
\pgfpathlineto{\pgfqpoint{2.120343in}{2.762783in}}%
\pgfpathlineto{\pgfqpoint{2.131702in}{2.766482in}}%
\pgfpathlineto{\pgfqpoint{2.143059in}{2.770048in}}%
\pgfpathlineto{\pgfqpoint{2.154416in}{2.773391in}}%
\pgfpathlineto{\pgfqpoint{2.148609in}{2.781283in}}%
\pgfpathlineto{\pgfqpoint{2.142806in}{2.789166in}}%
\pgfpathlineto{\pgfqpoint{2.137006in}{2.797045in}}%
\pgfpathlineto{\pgfqpoint{2.131211in}{2.804924in}}%
\pgfpathlineto{\pgfqpoint{2.125419in}{2.812809in}}%
\pgfpathlineto{\pgfqpoint{2.114073in}{2.809503in}}%
\pgfpathlineto{\pgfqpoint{2.102729in}{2.805945in}}%
\pgfpathlineto{\pgfqpoint{2.091382in}{2.802230in}}%
\pgfpathlineto{\pgfqpoint{2.080032in}{2.798455in}}%
\pgfpathlineto{\pgfqpoint{2.068676in}{2.794718in}}%
\pgfpathlineto{\pgfqpoint{2.074455in}{2.786866in}}%
\pgfpathlineto{\pgfqpoint{2.080237in}{2.779002in}}%
\pgfpathlineto{\pgfqpoint{2.086024in}{2.771126in}}%
\pgfpathlineto{\pgfqpoint{2.091815in}{2.763236in}}%
\pgfpathclose%
\pgfusepath{stroke,fill}%
\end{pgfscope}%
\begin{pgfscope}%
\pgfpathrectangle{\pgfqpoint{0.887500in}{0.275000in}}{\pgfqpoint{4.225000in}{4.225000in}}%
\pgfusepath{clip}%
\pgfsetbuttcap%
\pgfsetroundjoin%
\definecolor{currentfill}{rgb}{0.121380,0.629492,0.531973}%
\pgfsetfillcolor{currentfill}%
\pgfsetfillopacity{0.700000}%
\pgfsetlinewidth{0.501875pt}%
\definecolor{currentstroke}{rgb}{1.000000,1.000000,1.000000}%
\pgfsetstrokecolor{currentstroke}%
\pgfsetstrokeopacity{0.500000}%
\pgfsetdash{}{0pt}%
\pgfpathmoveto{\pgfqpoint{1.554563in}{2.829844in}}%
\pgfpathlineto{\pgfqpoint{1.566070in}{2.833175in}}%
\pgfpathlineto{\pgfqpoint{1.577571in}{2.836505in}}%
\pgfpathlineto{\pgfqpoint{1.589066in}{2.839835in}}%
\pgfpathlineto{\pgfqpoint{1.600556in}{2.843164in}}%
\pgfpathlineto{\pgfqpoint{1.612041in}{2.846494in}}%
\pgfpathlineto{\pgfqpoint{1.606431in}{2.853904in}}%
\pgfpathlineto{\pgfqpoint{1.600826in}{2.861296in}}%
\pgfpathlineto{\pgfqpoint{1.595226in}{2.868671in}}%
\pgfpathlineto{\pgfqpoint{1.589629in}{2.876029in}}%
\pgfpathlineto{\pgfqpoint{1.578157in}{2.872660in}}%
\pgfpathlineto{\pgfqpoint{1.566679in}{2.869294in}}%
\pgfpathlineto{\pgfqpoint{1.555196in}{2.865932in}}%
\pgfpathlineto{\pgfqpoint{1.543707in}{2.862576in}}%
\pgfpathlineto{\pgfqpoint{1.532212in}{2.859226in}}%
\pgfpathlineto{\pgfqpoint{1.537793in}{2.851911in}}%
\pgfpathlineto{\pgfqpoint{1.543379in}{2.844576in}}%
\pgfpathlineto{\pgfqpoint{1.548969in}{2.837220in}}%
\pgfpathclose%
\pgfusepath{stroke,fill}%
\end{pgfscope}%
\begin{pgfscope}%
\pgfpathrectangle{\pgfqpoint{0.887500in}{0.275000in}}{\pgfqpoint{4.225000in}{4.225000in}}%
\pgfusepath{clip}%
\pgfsetbuttcap%
\pgfsetroundjoin%
\definecolor{currentfill}{rgb}{0.814576,0.883393,0.110347}%
\pgfsetfillcolor{currentfill}%
\pgfsetfillopacity{0.700000}%
\pgfsetlinewidth{0.501875pt}%
\definecolor{currentstroke}{rgb}{1.000000,1.000000,1.000000}%
\pgfsetstrokecolor{currentstroke}%
\pgfsetstrokeopacity{0.500000}%
\pgfsetdash{}{0pt}%
\pgfpathmoveto{\pgfqpoint{3.137270in}{3.547379in}}%
\pgfpathlineto{\pgfqpoint{3.148405in}{3.550943in}}%
\pgfpathlineto{\pgfqpoint{3.159534in}{3.554241in}}%
\pgfpathlineto{\pgfqpoint{3.170657in}{3.557370in}}%
\pgfpathlineto{\pgfqpoint{3.181775in}{3.560426in}}%
\pgfpathlineto{\pgfqpoint{3.192888in}{3.563497in}}%
\pgfpathlineto{\pgfqpoint{3.186721in}{3.567262in}}%
\pgfpathlineto{\pgfqpoint{3.180558in}{3.570722in}}%
\pgfpathlineto{\pgfqpoint{3.174400in}{3.573871in}}%
\pgfpathlineto{\pgfqpoint{3.168246in}{3.576721in}}%
\pgfpathlineto{\pgfqpoint{3.162098in}{3.579296in}}%
\pgfpathlineto{\pgfqpoint{3.151001in}{3.575576in}}%
\pgfpathlineto{\pgfqpoint{3.139899in}{3.571897in}}%
\pgfpathlineto{\pgfqpoint{3.128792in}{3.568185in}}%
\pgfpathlineto{\pgfqpoint{3.117680in}{3.564354in}}%
\pgfpathlineto{\pgfqpoint{3.106563in}{3.560320in}}%
\pgfpathlineto{\pgfqpoint{3.112694in}{3.558623in}}%
\pgfpathlineto{\pgfqpoint{3.118831in}{3.556522in}}%
\pgfpathlineto{\pgfqpoint{3.124973in}{3.553962in}}%
\pgfpathlineto{\pgfqpoint{3.131119in}{3.550907in}}%
\pgfpathclose%
\pgfusepath{stroke,fill}%
\end{pgfscope}%
\begin{pgfscope}%
\pgfpathrectangle{\pgfqpoint{0.887500in}{0.275000in}}{\pgfqpoint{4.225000in}{4.225000in}}%
\pgfusepath{clip}%
\pgfsetbuttcap%
\pgfsetroundjoin%
\definecolor{currentfill}{rgb}{0.327796,0.773980,0.406640}%
\pgfsetfillcolor{currentfill}%
\pgfsetfillopacity{0.700000}%
\pgfsetlinewidth{0.501875pt}%
\definecolor{currentstroke}{rgb}{1.000000,1.000000,1.000000}%
\pgfsetstrokecolor{currentstroke}%
\pgfsetstrokeopacity{0.500000}%
\pgfsetdash{}{0pt}%
\pgfpathmoveto{\pgfqpoint{2.803388in}{3.035691in}}%
\pgfpathlineto{\pgfqpoint{2.814363in}{3.072747in}}%
\pgfpathlineto{\pgfqpoint{2.825307in}{3.117552in}}%
\pgfpathlineto{\pgfqpoint{2.836241in}{3.167362in}}%
\pgfpathlineto{\pgfqpoint{2.847183in}{3.219413in}}%
\pgfpathlineto{\pgfqpoint{2.858151in}{3.270921in}}%
\pgfpathlineto{\pgfqpoint{2.852099in}{3.277567in}}%
\pgfpathlineto{\pgfqpoint{2.846044in}{3.285736in}}%
\pgfpathlineto{\pgfqpoint{2.839984in}{3.295537in}}%
\pgfpathlineto{\pgfqpoint{2.833920in}{3.306649in}}%
\pgfpathlineto{\pgfqpoint{2.827854in}{3.318601in}}%
\pgfpathlineto{\pgfqpoint{2.816885in}{3.274924in}}%
\pgfpathlineto{\pgfqpoint{2.805938in}{3.230362in}}%
\pgfpathlineto{\pgfqpoint{2.795001in}{3.187042in}}%
\pgfpathlineto{\pgfqpoint{2.784059in}{3.147077in}}%
\pgfpathlineto{\pgfqpoint{2.773093in}{3.112566in}}%
\pgfpathlineto{\pgfqpoint{2.779186in}{3.091756in}}%
\pgfpathlineto{\pgfqpoint{2.785265in}{3.073099in}}%
\pgfpathlineto{\pgfqpoint{2.791325in}{3.057331in}}%
\pgfpathlineto{\pgfqpoint{2.797365in}{3.044940in}}%
\pgfpathclose%
\pgfusepath{stroke,fill}%
\end{pgfscope}%
\begin{pgfscope}%
\pgfpathrectangle{\pgfqpoint{0.887500in}{0.275000in}}{\pgfqpoint{4.225000in}{4.225000in}}%
\pgfusepath{clip}%
\pgfsetbuttcap%
\pgfsetroundjoin%
\definecolor{currentfill}{rgb}{0.751884,0.874951,0.143228}%
\pgfsetfillcolor{currentfill}%
\pgfsetfillopacity{0.700000}%
\pgfsetlinewidth{0.501875pt}%
\definecolor{currentstroke}{rgb}{1.000000,1.000000,1.000000}%
\pgfsetstrokecolor{currentstroke}%
\pgfsetstrokeopacity{0.500000}%
\pgfsetdash{}{0pt}%
\pgfpathmoveto{\pgfqpoint{3.822453in}{3.470538in}}%
\pgfpathlineto{\pgfqpoint{3.833441in}{3.474908in}}%
\pgfpathlineto{\pgfqpoint{3.844423in}{3.479231in}}%
\pgfpathlineto{\pgfqpoint{3.855399in}{3.483513in}}%
\pgfpathlineto{\pgfqpoint{3.866371in}{3.487763in}}%
\pgfpathlineto{\pgfqpoint{3.877337in}{3.491986in}}%
\pgfpathlineto{\pgfqpoint{3.871046in}{3.504100in}}%
\pgfpathlineto{\pgfqpoint{3.864757in}{3.516139in}}%
\pgfpathlineto{\pgfqpoint{3.858469in}{3.528087in}}%
\pgfpathlineto{\pgfqpoint{3.852182in}{3.539924in}}%
\pgfpathlineto{\pgfqpoint{3.845895in}{3.551633in}}%
\pgfpathlineto{\pgfqpoint{3.834930in}{3.547159in}}%
\pgfpathlineto{\pgfqpoint{3.823959in}{3.542621in}}%
\pgfpathlineto{\pgfqpoint{3.812982in}{3.538009in}}%
\pgfpathlineto{\pgfqpoint{3.801999in}{3.533313in}}%
\pgfpathlineto{\pgfqpoint{3.791009in}{3.528522in}}%
\pgfpathlineto{\pgfqpoint{3.797294in}{3.517054in}}%
\pgfpathlineto{\pgfqpoint{3.803581in}{3.505498in}}%
\pgfpathlineto{\pgfqpoint{3.809869in}{3.493878in}}%
\pgfpathlineto{\pgfqpoint{3.816160in}{3.482217in}}%
\pgfpathclose%
\pgfusepath{stroke,fill}%
\end{pgfscope}%
\begin{pgfscope}%
\pgfpathrectangle{\pgfqpoint{0.887500in}{0.275000in}}{\pgfqpoint{4.225000in}{4.225000in}}%
\pgfusepath{clip}%
\pgfsetbuttcap%
\pgfsetroundjoin%
\definecolor{currentfill}{rgb}{0.119423,0.611141,0.538982}%
\pgfsetfillcolor{currentfill}%
\pgfsetfillopacity{0.700000}%
\pgfsetlinewidth{0.501875pt}%
\definecolor{currentstroke}{rgb}{1.000000,1.000000,1.000000}%
\pgfsetstrokecolor{currentstroke}%
\pgfsetstrokeopacity{0.500000}%
\pgfsetdash{}{0pt}%
\pgfpathmoveto{\pgfqpoint{1.868909in}{2.782882in}}%
\pgfpathlineto{\pgfqpoint{1.880343in}{2.786260in}}%
\pgfpathlineto{\pgfqpoint{1.891772in}{2.789648in}}%
\pgfpathlineto{\pgfqpoint{1.903195in}{2.793043in}}%
\pgfpathlineto{\pgfqpoint{1.914612in}{2.796439in}}%
\pgfpathlineto{\pgfqpoint{1.926024in}{2.799829in}}%
\pgfpathlineto{\pgfqpoint{1.920296in}{2.807550in}}%
\pgfpathlineto{\pgfqpoint{1.914573in}{2.815259in}}%
\pgfpathlineto{\pgfqpoint{1.908853in}{2.822955in}}%
\pgfpathlineto{\pgfqpoint{1.903138in}{2.830640in}}%
\pgfpathlineto{\pgfqpoint{1.897427in}{2.838311in}}%
\pgfpathlineto{\pgfqpoint{1.886028in}{2.834897in}}%
\pgfpathlineto{\pgfqpoint{1.874624in}{2.831477in}}%
\pgfpathlineto{\pgfqpoint{1.863215in}{2.828059in}}%
\pgfpathlineto{\pgfqpoint{1.851800in}{2.824648in}}%
\pgfpathlineto{\pgfqpoint{1.840380in}{2.821248in}}%
\pgfpathlineto{\pgfqpoint{1.846077in}{2.813601in}}%
\pgfpathlineto{\pgfqpoint{1.851779in}{2.805940in}}%
\pgfpathlineto{\pgfqpoint{1.857485in}{2.798267in}}%
\pgfpathlineto{\pgfqpoint{1.863195in}{2.790581in}}%
\pgfpathclose%
\pgfusepath{stroke,fill}%
\end{pgfscope}%
\begin{pgfscope}%
\pgfpathrectangle{\pgfqpoint{0.887500in}{0.275000in}}{\pgfqpoint{4.225000in}{4.225000in}}%
\pgfusepath{clip}%
\pgfsetbuttcap%
\pgfsetroundjoin%
\definecolor{currentfill}{rgb}{0.121148,0.592739,0.544641}%
\pgfsetfillcolor{currentfill}%
\pgfsetfillopacity{0.700000}%
\pgfsetlinewidth{0.501875pt}%
\definecolor{currentstroke}{rgb}{1.000000,1.000000,1.000000}%
\pgfsetstrokecolor{currentstroke}%
\pgfsetstrokeopacity{0.500000}%
\pgfsetdash{}{0pt}%
\pgfpathmoveto{\pgfqpoint{2.326142in}{2.731461in}}%
\pgfpathlineto{\pgfqpoint{2.337395in}{2.738629in}}%
\pgfpathlineto{\pgfqpoint{2.348634in}{2.746442in}}%
\pgfpathlineto{\pgfqpoint{2.359861in}{2.754742in}}%
\pgfpathlineto{\pgfqpoint{2.371080in}{2.763371in}}%
\pgfpathlineto{\pgfqpoint{2.382293in}{2.772169in}}%
\pgfpathlineto{\pgfqpoint{2.376371in}{2.782192in}}%
\pgfpathlineto{\pgfqpoint{2.370455in}{2.792062in}}%
\pgfpathlineto{\pgfqpoint{2.364546in}{2.801771in}}%
\pgfpathlineto{\pgfqpoint{2.358642in}{2.811312in}}%
\pgfpathlineto{\pgfqpoint{2.352745in}{2.820676in}}%
\pgfpathlineto{\pgfqpoint{2.341567in}{2.810596in}}%
\pgfpathlineto{\pgfqpoint{2.330388in}{2.800556in}}%
\pgfpathlineto{\pgfqpoint{2.319202in}{2.790806in}}%
\pgfpathlineto{\pgfqpoint{2.308003in}{2.781594in}}%
\pgfpathlineto{\pgfqpoint{2.296788in}{2.773170in}}%
\pgfpathlineto{\pgfqpoint{2.302648in}{2.764994in}}%
\pgfpathlineto{\pgfqpoint{2.308514in}{2.756721in}}%
\pgfpathlineto{\pgfqpoint{2.314385in}{2.748366in}}%
\pgfpathlineto{\pgfqpoint{2.320261in}{2.739941in}}%
\pgfpathclose%
\pgfusepath{stroke,fill}%
\end{pgfscope}%
\begin{pgfscope}%
\pgfpathrectangle{\pgfqpoint{0.887500in}{0.275000in}}{\pgfqpoint{4.225000in}{4.225000in}}%
\pgfusepath{clip}%
\pgfsetbuttcap%
\pgfsetroundjoin%
\definecolor{currentfill}{rgb}{0.699415,0.867117,0.175971}%
\pgfsetfillcolor{currentfill}%
\pgfsetfillopacity{0.700000}%
\pgfsetlinewidth{0.501875pt}%
\definecolor{currentstroke}{rgb}{1.000000,1.000000,1.000000}%
\pgfsetstrokecolor{currentstroke}%
\pgfsetstrokeopacity{0.500000}%
\pgfsetdash{}{0pt}%
\pgfpathmoveto{\pgfqpoint{3.908822in}{3.430891in}}%
\pgfpathlineto{\pgfqpoint{3.919782in}{3.434802in}}%
\pgfpathlineto{\pgfqpoint{3.930738in}{3.438772in}}%
\pgfpathlineto{\pgfqpoint{3.941691in}{3.442803in}}%
\pgfpathlineto{\pgfqpoint{3.952641in}{3.446899in}}%
\pgfpathlineto{\pgfqpoint{3.963587in}{3.451061in}}%
\pgfpathlineto{\pgfqpoint{3.957284in}{3.463490in}}%
\pgfpathlineto{\pgfqpoint{3.950984in}{3.475905in}}%
\pgfpathlineto{\pgfqpoint{3.944686in}{3.488297in}}%
\pgfpathlineto{\pgfqpoint{3.938390in}{3.500655in}}%
\pgfpathlineto{\pgfqpoint{3.932095in}{3.512969in}}%
\pgfpathlineto{\pgfqpoint{3.921153in}{3.508766in}}%
\pgfpathlineto{\pgfqpoint{3.910206in}{3.504574in}}%
\pgfpathlineto{\pgfqpoint{3.899254in}{3.500385in}}%
\pgfpathlineto{\pgfqpoint{3.888298in}{3.496191in}}%
\pgfpathlineto{\pgfqpoint{3.877337in}{3.491986in}}%
\pgfpathlineto{\pgfqpoint{3.883629in}{3.479816in}}%
\pgfpathlineto{\pgfqpoint{3.889923in}{3.467604in}}%
\pgfpathlineto{\pgfqpoint{3.896220in}{3.455369in}}%
\pgfpathlineto{\pgfqpoint{3.902520in}{3.443126in}}%
\pgfpathclose%
\pgfusepath{stroke,fill}%
\end{pgfscope}%
\begin{pgfscope}%
\pgfpathrectangle{\pgfqpoint{0.887500in}{0.275000in}}{\pgfqpoint{4.225000in}{4.225000in}}%
\pgfusepath{clip}%
\pgfsetbuttcap%
\pgfsetroundjoin%
\definecolor{currentfill}{rgb}{0.835270,0.886029,0.102646}%
\pgfsetfillcolor{currentfill}%
\pgfsetfillopacity{0.700000}%
\pgfsetlinewidth{0.501875pt}%
\definecolor{currentstroke}{rgb}{1.000000,1.000000,1.000000}%
\pgfsetstrokecolor{currentstroke}%
\pgfsetstrokeopacity{0.500000}%
\pgfsetdash{}{0pt}%
\pgfpathmoveto{\pgfqpoint{3.507930in}{3.550093in}}%
\pgfpathlineto{\pgfqpoint{3.518985in}{3.554141in}}%
\pgfpathlineto{\pgfqpoint{3.530035in}{3.558149in}}%
\pgfpathlineto{\pgfqpoint{3.541080in}{3.562109in}}%
\pgfpathlineto{\pgfqpoint{3.552119in}{3.566009in}}%
\pgfpathlineto{\pgfqpoint{3.563152in}{3.569840in}}%
\pgfpathlineto{\pgfqpoint{3.556871in}{3.577246in}}%
\pgfpathlineto{\pgfqpoint{3.550591in}{3.584416in}}%
\pgfpathlineto{\pgfqpoint{3.544314in}{3.591419in}}%
\pgfpathlineto{\pgfqpoint{3.538040in}{3.598320in}}%
\pgfpathlineto{\pgfqpoint{3.531771in}{3.605186in}}%
\pgfpathlineto{\pgfqpoint{3.520740in}{3.600574in}}%
\pgfpathlineto{\pgfqpoint{3.509706in}{3.595950in}}%
\pgfpathlineto{\pgfqpoint{3.498666in}{3.591308in}}%
\pgfpathlineto{\pgfqpoint{3.487622in}{3.586645in}}%
\pgfpathlineto{\pgfqpoint{3.476574in}{3.581954in}}%
\pgfpathlineto{\pgfqpoint{3.482838in}{3.575760in}}%
\pgfpathlineto{\pgfqpoint{3.489106in}{3.569557in}}%
\pgfpathlineto{\pgfqpoint{3.495378in}{3.563267in}}%
\pgfpathlineto{\pgfqpoint{3.501653in}{3.556806in}}%
\pgfpathclose%
\pgfusepath{stroke,fill}%
\end{pgfscope}%
\begin{pgfscope}%
\pgfpathrectangle{\pgfqpoint{0.887500in}{0.275000in}}{\pgfqpoint{4.225000in}{4.225000in}}%
\pgfusepath{clip}%
\pgfsetbuttcap%
\pgfsetroundjoin%
\definecolor{currentfill}{rgb}{0.636902,0.856542,0.216620}%
\pgfsetfillcolor{currentfill}%
\pgfsetfillopacity{0.700000}%
\pgfsetlinewidth{0.501875pt}%
\definecolor{currentstroke}{rgb}{1.000000,1.000000,1.000000}%
\pgfsetstrokecolor{currentstroke}%
\pgfsetstrokeopacity{0.500000}%
\pgfsetdash{}{0pt}%
\pgfpathmoveto{\pgfqpoint{3.995149in}{3.389064in}}%
\pgfpathlineto{\pgfqpoint{4.006091in}{3.393058in}}%
\pgfpathlineto{\pgfqpoint{4.017031in}{3.397141in}}%
\pgfpathlineto{\pgfqpoint{4.027968in}{3.401310in}}%
\pgfpathlineto{\pgfqpoint{4.038903in}{3.405566in}}%
\pgfpathlineto{\pgfqpoint{4.032589in}{3.418284in}}%
\pgfpathlineto{\pgfqpoint{4.026276in}{3.430925in}}%
\pgfpathlineto{\pgfqpoint{4.019964in}{3.443493in}}%
\pgfpathlineto{\pgfqpoint{4.013653in}{3.455996in}}%
\pgfpathlineto{\pgfqpoint{4.007344in}{3.468441in}}%
\pgfpathlineto{\pgfqpoint{3.996409in}{3.463981in}}%
\pgfpathlineto{\pgfqpoint{3.985471in}{3.459600in}}%
\pgfpathlineto{\pgfqpoint{3.974531in}{3.455294in}}%
\pgfpathlineto{\pgfqpoint{3.963587in}{3.451061in}}%
\pgfpathlineto{\pgfqpoint{3.969893in}{3.438629in}}%
\pgfpathlineto{\pgfqpoint{3.976202in}{3.426203in}}%
\pgfpathlineto{\pgfqpoint{3.982514in}{3.413794in}}%
\pgfpathlineto{\pgfqpoint{3.988829in}{3.401411in}}%
\pgfpathclose%
\pgfusepath{stroke,fill}%
\end{pgfscope}%
\begin{pgfscope}%
\pgfpathrectangle{\pgfqpoint{0.887500in}{0.275000in}}{\pgfqpoint{4.225000in}{4.225000in}}%
\pgfusepath{clip}%
\pgfsetbuttcap%
\pgfsetroundjoin%
\definecolor{currentfill}{rgb}{0.120081,0.622161,0.534946}%
\pgfsetfillcolor{currentfill}%
\pgfsetfillopacity{0.700000}%
\pgfsetlinewidth{0.501875pt}%
\definecolor{currentstroke}{rgb}{1.000000,1.000000,1.000000}%
\pgfsetstrokecolor{currentstroke}%
\pgfsetstrokeopacity{0.500000}%
\pgfsetdash{}{0pt}%
\pgfpathmoveto{\pgfqpoint{1.640152in}{2.809191in}}%
\pgfpathlineto{\pgfqpoint{1.651645in}{2.812490in}}%
\pgfpathlineto{\pgfqpoint{1.663133in}{2.815789in}}%
\pgfpathlineto{\pgfqpoint{1.674615in}{2.819090in}}%
\pgfpathlineto{\pgfqpoint{1.686092in}{2.822393in}}%
\pgfpathlineto{\pgfqpoint{1.697563in}{2.825701in}}%
\pgfpathlineto{\pgfqpoint{1.691918in}{2.833220in}}%
\pgfpathlineto{\pgfqpoint{1.686277in}{2.840725in}}%
\pgfpathlineto{\pgfqpoint{1.680640in}{2.848216in}}%
\pgfpathlineto{\pgfqpoint{1.675008in}{2.855693in}}%
\pgfpathlineto{\pgfqpoint{1.669379in}{2.863156in}}%
\pgfpathlineto{\pgfqpoint{1.657923in}{2.859820in}}%
\pgfpathlineto{\pgfqpoint{1.646461in}{2.856486in}}%
\pgfpathlineto{\pgfqpoint{1.634993in}{2.853155in}}%
\pgfpathlineto{\pgfqpoint{1.623520in}{2.849824in}}%
\pgfpathlineto{\pgfqpoint{1.612041in}{2.846494in}}%
\pgfpathlineto{\pgfqpoint{1.617654in}{2.839067in}}%
\pgfpathlineto{\pgfqpoint{1.623272in}{2.831623in}}%
\pgfpathlineto{\pgfqpoint{1.628894in}{2.824163in}}%
\pgfpathlineto{\pgfqpoint{1.634521in}{2.816685in}}%
\pgfpathclose%
\pgfusepath{stroke,fill}%
\end{pgfscope}%
\begin{pgfscope}%
\pgfpathrectangle{\pgfqpoint{0.887500in}{0.275000in}}{\pgfqpoint{4.225000in}{4.225000in}}%
\pgfusepath{clip}%
\pgfsetbuttcap%
\pgfsetroundjoin%
\definecolor{currentfill}{rgb}{0.153894,0.680203,0.504172}%
\pgfsetfillcolor{currentfill}%
\pgfsetfillopacity{0.700000}%
\pgfsetlinewidth{0.501875pt}%
\definecolor{currentstroke}{rgb}{1.000000,1.000000,1.000000}%
\pgfsetstrokecolor{currentstroke}%
\pgfsetstrokeopacity{0.500000}%
\pgfsetdash{}{0pt}%
\pgfpathmoveto{\pgfqpoint{2.721618in}{2.927868in}}%
\pgfpathlineto{\pgfqpoint{2.732805in}{2.936915in}}%
\pgfpathlineto{\pgfqpoint{2.744004in}{2.944278in}}%
\pgfpathlineto{\pgfqpoint{2.755215in}{2.949492in}}%
\pgfpathlineto{\pgfqpoint{2.766438in}{2.952567in}}%
\pgfpathlineto{\pgfqpoint{2.777659in}{2.955053in}}%
\pgfpathlineto{\pgfqpoint{2.771680in}{2.957415in}}%
\pgfpathlineto{\pgfqpoint{2.765702in}{2.960350in}}%
\pgfpathlineto{\pgfqpoint{2.759723in}{2.964104in}}%
\pgfpathlineto{\pgfqpoint{2.753742in}{2.968923in}}%
\pgfpathlineto{\pgfqpoint{2.747755in}{2.975051in}}%
\pgfpathlineto{\pgfqpoint{2.736529in}{2.975154in}}%
\pgfpathlineto{\pgfqpoint{2.725302in}{2.974660in}}%
\pgfpathlineto{\pgfqpoint{2.714086in}{2.971900in}}%
\pgfpathlineto{\pgfqpoint{2.702885in}{2.966875in}}%
\pgfpathlineto{\pgfqpoint{2.691695in}{2.960098in}}%
\pgfpathlineto{\pgfqpoint{2.697669in}{2.953908in}}%
\pgfpathlineto{\pgfqpoint{2.703647in}{2.947657in}}%
\pgfpathlineto{\pgfqpoint{2.709631in}{2.941278in}}%
\pgfpathlineto{\pgfqpoint{2.715621in}{2.934704in}}%
\pgfpathclose%
\pgfusepath{stroke,fill}%
\end{pgfscope}%
\begin{pgfscope}%
\pgfpathrectangle{\pgfqpoint{0.887500in}{0.275000in}}{\pgfqpoint{4.225000in}{4.225000in}}%
\pgfusepath{clip}%
\pgfsetbuttcap%
\pgfsetroundjoin%
\definecolor{currentfill}{rgb}{0.121831,0.589055,0.545623}%
\pgfsetfillcolor{currentfill}%
\pgfsetfillopacity{0.700000}%
\pgfsetlinewidth{0.501875pt}%
\definecolor{currentstroke}{rgb}{1.000000,1.000000,1.000000}%
\pgfsetstrokecolor{currentstroke}%
\pgfsetstrokeopacity{0.500000}%
\pgfsetdash{}{0pt}%
\pgfpathmoveto{\pgfqpoint{2.183514in}{2.733642in}}%
\pgfpathlineto{\pgfqpoint{2.194881in}{2.736819in}}%
\pgfpathlineto{\pgfqpoint{2.206249in}{2.739690in}}%
\pgfpathlineto{\pgfqpoint{2.217618in}{2.742286in}}%
\pgfpathlineto{\pgfqpoint{2.228981in}{2.744837in}}%
\pgfpathlineto{\pgfqpoint{2.240334in}{2.747596in}}%
\pgfpathlineto{\pgfqpoint{2.234498in}{2.755380in}}%
\pgfpathlineto{\pgfqpoint{2.228668in}{2.763107in}}%
\pgfpathlineto{\pgfqpoint{2.222842in}{2.770770in}}%
\pgfpathlineto{\pgfqpoint{2.217022in}{2.778377in}}%
\pgfpathlineto{\pgfqpoint{2.211206in}{2.785960in}}%
\pgfpathlineto{\pgfqpoint{2.199858in}{2.783529in}}%
\pgfpathlineto{\pgfqpoint{2.188499in}{2.781333in}}%
\pgfpathlineto{\pgfqpoint{2.177136in}{2.779059in}}%
\pgfpathlineto{\pgfqpoint{2.165774in}{2.776424in}}%
\pgfpathlineto{\pgfqpoint{2.154416in}{2.773391in}}%
\pgfpathlineto{\pgfqpoint{2.160227in}{2.765485in}}%
\pgfpathlineto{\pgfqpoint{2.166043in}{2.757561in}}%
\pgfpathlineto{\pgfqpoint{2.171862in}{2.749612in}}%
\pgfpathlineto{\pgfqpoint{2.177686in}{2.741639in}}%
\pgfpathclose%
\pgfusepath{stroke,fill}%
\end{pgfscope}%
\begin{pgfscope}%
\pgfpathrectangle{\pgfqpoint{0.887500in}{0.275000in}}{\pgfqpoint{4.225000in}{4.225000in}}%
\pgfusepath{clip}%
\pgfsetbuttcap%
\pgfsetroundjoin%
\definecolor{currentfill}{rgb}{0.124780,0.640461,0.527068}%
\pgfsetfillcolor{currentfill}%
\pgfsetfillopacity{0.700000}%
\pgfsetlinewidth{0.501875pt}%
\definecolor{currentstroke}{rgb}{1.000000,1.000000,1.000000}%
\pgfsetstrokecolor{currentstroke}%
\pgfsetstrokeopacity{0.500000}%
\pgfsetdash{}{0pt}%
\pgfpathmoveto{\pgfqpoint{2.523977in}{2.818337in}}%
\pgfpathlineto{\pgfqpoint{2.535123in}{2.831480in}}%
\pgfpathlineto{\pgfqpoint{2.546273in}{2.844376in}}%
\pgfpathlineto{\pgfqpoint{2.557430in}{2.856671in}}%
\pgfpathlineto{\pgfqpoint{2.568599in}{2.868009in}}%
\pgfpathlineto{\pgfqpoint{2.579782in}{2.878088in}}%
\pgfpathlineto{\pgfqpoint{2.573808in}{2.887578in}}%
\pgfpathlineto{\pgfqpoint{2.567851in}{2.896050in}}%
\pgfpathlineto{\pgfqpoint{2.561911in}{2.903383in}}%
\pgfpathlineto{\pgfqpoint{2.555987in}{2.909720in}}%
\pgfpathlineto{\pgfqpoint{2.550078in}{2.915292in}}%
\pgfpathlineto{\pgfqpoint{2.538882in}{2.907570in}}%
\pgfpathlineto{\pgfqpoint{2.527697in}{2.898785in}}%
\pgfpathlineto{\pgfqpoint{2.516521in}{2.889133in}}%
\pgfpathlineto{\pgfqpoint{2.505353in}{2.878854in}}%
\pgfpathlineto{\pgfqpoint{2.494188in}{2.868190in}}%
\pgfpathlineto{\pgfqpoint{2.500120in}{2.859566in}}%
\pgfpathlineto{\pgfqpoint{2.506065in}{2.850360in}}%
\pgfpathlineto{\pgfqpoint{2.512022in}{2.840437in}}%
\pgfpathlineto{\pgfqpoint{2.517993in}{2.829725in}}%
\pgfpathclose%
\pgfusepath{stroke,fill}%
\end{pgfscope}%
\begin{pgfscope}%
\pgfpathrectangle{\pgfqpoint{0.887500in}{0.275000in}}{\pgfqpoint{4.225000in}{4.225000in}}%
\pgfusepath{clip}%
\pgfsetbuttcap%
\pgfsetroundjoin%
\definecolor{currentfill}{rgb}{0.824940,0.884720,0.106217}%
\pgfsetfillcolor{currentfill}%
\pgfsetfillopacity{0.700000}%
\pgfsetlinewidth{0.501875pt}%
\definecolor{currentstroke}{rgb}{1.000000,1.000000,1.000000}%
\pgfsetstrokecolor{currentstroke}%
\pgfsetstrokeopacity{0.500000}%
\pgfsetdash{}{0pt}%
\pgfpathmoveto{\pgfqpoint{3.365901in}{3.542775in}}%
\pgfpathlineto{\pgfqpoint{3.376981in}{3.545430in}}%
\pgfpathlineto{\pgfqpoint{3.388056in}{3.548322in}}%
\pgfpathlineto{\pgfqpoint{3.399129in}{3.551554in}}%
\pgfpathlineto{\pgfqpoint{3.410200in}{3.555186in}}%
\pgfpathlineto{\pgfqpoint{3.421270in}{3.559167in}}%
\pgfpathlineto{\pgfqpoint{3.415020in}{3.564894in}}%
\pgfpathlineto{\pgfqpoint{3.408777in}{3.570790in}}%
\pgfpathlineto{\pgfqpoint{3.402540in}{3.576884in}}%
\pgfpathlineto{\pgfqpoint{3.396309in}{3.583126in}}%
\pgfpathlineto{\pgfqpoint{3.390083in}{3.589461in}}%
\pgfpathlineto{\pgfqpoint{3.379023in}{3.584987in}}%
\pgfpathlineto{\pgfqpoint{3.367962in}{3.580869in}}%
\pgfpathlineto{\pgfqpoint{3.356899in}{3.577162in}}%
\pgfpathlineto{\pgfqpoint{3.345833in}{3.573821in}}%
\pgfpathlineto{\pgfqpoint{3.334764in}{3.570759in}}%
\pgfpathlineto{\pgfqpoint{3.340982in}{3.565162in}}%
\pgfpathlineto{\pgfqpoint{3.347205in}{3.559546in}}%
\pgfpathlineto{\pgfqpoint{3.353432in}{3.553937in}}%
\pgfpathlineto{\pgfqpoint{3.359664in}{3.548355in}}%
\pgfpathclose%
\pgfusepath{stroke,fill}%
\end{pgfscope}%
\begin{pgfscope}%
\pgfpathrectangle{\pgfqpoint{0.887500in}{0.275000in}}{\pgfqpoint{4.225000in}{4.225000in}}%
\pgfusepath{clip}%
\pgfsetbuttcap%
\pgfsetroundjoin%
\definecolor{currentfill}{rgb}{0.119738,0.603785,0.541400}%
\pgfsetfillcolor{currentfill}%
\pgfsetfillopacity{0.700000}%
\pgfsetlinewidth{0.501875pt}%
\definecolor{currentstroke}{rgb}{1.000000,1.000000,1.000000}%
\pgfsetstrokecolor{currentstroke}%
\pgfsetstrokeopacity{0.500000}%
\pgfsetdash{}{0pt}%
\pgfpathmoveto{\pgfqpoint{1.954723in}{2.761046in}}%
\pgfpathlineto{\pgfqpoint{1.966143in}{2.764395in}}%
\pgfpathlineto{\pgfqpoint{1.977559in}{2.767725in}}%
\pgfpathlineto{\pgfqpoint{1.988969in}{2.771030in}}%
\pgfpathlineto{\pgfqpoint{2.000375in}{2.774307in}}%
\pgfpathlineto{\pgfqpoint{2.011776in}{2.777577in}}%
\pgfpathlineto{\pgfqpoint{2.006014in}{2.785388in}}%
\pgfpathlineto{\pgfqpoint{2.000257in}{2.793186in}}%
\pgfpathlineto{\pgfqpoint{1.994504in}{2.800972in}}%
\pgfpathlineto{\pgfqpoint{1.988755in}{2.808746in}}%
\pgfpathlineto{\pgfqpoint{1.983010in}{2.816508in}}%
\pgfpathlineto{\pgfqpoint{1.971622in}{2.813207in}}%
\pgfpathlineto{\pgfqpoint{1.960230in}{2.809899in}}%
\pgfpathlineto{\pgfqpoint{1.948833in}{2.806565in}}%
\pgfpathlineto{\pgfqpoint{1.937431in}{2.803206in}}%
\pgfpathlineto{\pgfqpoint{1.926024in}{2.799829in}}%
\pgfpathlineto{\pgfqpoint{1.931756in}{2.792096in}}%
\pgfpathlineto{\pgfqpoint{1.937491in}{2.784351in}}%
\pgfpathlineto{\pgfqpoint{1.943231in}{2.776595in}}%
\pgfpathlineto{\pgfqpoint{1.948975in}{2.768826in}}%
\pgfpathclose%
\pgfusepath{stroke,fill}%
\end{pgfscope}%
\begin{pgfscope}%
\pgfpathrectangle{\pgfqpoint{0.887500in}{0.275000in}}{\pgfqpoint{4.225000in}{4.225000in}}%
\pgfusepath{clip}%
\pgfsetbuttcap%
\pgfsetroundjoin%
\definecolor{currentfill}{rgb}{0.119483,0.614817,0.537692}%
\pgfsetfillcolor{currentfill}%
\pgfsetfillopacity{0.700000}%
\pgfsetlinewidth{0.501875pt}%
\definecolor{currentstroke}{rgb}{1.000000,1.000000,1.000000}%
\pgfsetstrokecolor{currentstroke}%
\pgfsetstrokeopacity{0.500000}%
\pgfsetdash{}{0pt}%
\pgfpathmoveto{\pgfqpoint{2.468132in}{2.760330in}}%
\pgfpathlineto{\pgfqpoint{2.479328in}{2.770210in}}%
\pgfpathlineto{\pgfqpoint{2.490508in}{2.780979in}}%
\pgfpathlineto{\pgfqpoint{2.501674in}{2.792734in}}%
\pgfpathlineto{\pgfqpoint{2.512829in}{2.805304in}}%
\pgfpathlineto{\pgfqpoint{2.523977in}{2.818337in}}%
\pgfpathlineto{\pgfqpoint{2.517993in}{2.829725in}}%
\pgfpathlineto{\pgfqpoint{2.512022in}{2.840437in}}%
\pgfpathlineto{\pgfqpoint{2.506065in}{2.850360in}}%
\pgfpathlineto{\pgfqpoint{2.500120in}{2.859566in}}%
\pgfpathlineto{\pgfqpoint{2.494188in}{2.868190in}}%
\pgfpathlineto{\pgfqpoint{2.483024in}{2.857384in}}%
\pgfpathlineto{\pgfqpoint{2.471858in}{2.846677in}}%
\pgfpathlineto{\pgfqpoint{2.460685in}{2.836307in}}%
\pgfpathlineto{\pgfqpoint{2.449504in}{2.826392in}}%
\pgfpathlineto{\pgfqpoint{2.438315in}{2.816863in}}%
\pgfpathlineto{\pgfqpoint{2.444264in}{2.806137in}}%
\pgfpathlineto{\pgfqpoint{2.450220in}{2.795127in}}%
\pgfpathlineto{\pgfqpoint{2.456184in}{2.783810in}}%
\pgfpathlineto{\pgfqpoint{2.462155in}{2.772187in}}%
\pgfpathclose%
\pgfusepath{stroke,fill}%
\end{pgfscope}%
\begin{pgfscope}%
\pgfpathrectangle{\pgfqpoint{0.887500in}{0.275000in}}{\pgfqpoint{4.225000in}{4.225000in}}%
\pgfusepath{clip}%
\pgfsetbuttcap%
\pgfsetroundjoin%
\definecolor{currentfill}{rgb}{0.824940,0.884720,0.106217}%
\pgfsetfillcolor{currentfill}%
\pgfsetfillopacity{0.700000}%
\pgfsetlinewidth{0.501875pt}%
\definecolor{currentstroke}{rgb}{1.000000,1.000000,1.000000}%
\pgfsetstrokecolor{currentstroke}%
\pgfsetstrokeopacity{0.500000}%
\pgfsetdash{}{0pt}%
\pgfpathmoveto{\pgfqpoint{3.594544in}{3.526985in}}%
\pgfpathlineto{\pgfqpoint{3.605573in}{3.530089in}}%
\pgfpathlineto{\pgfqpoint{3.616594in}{3.533045in}}%
\pgfpathlineto{\pgfqpoint{3.627608in}{3.535882in}}%
\pgfpathlineto{\pgfqpoint{3.638616in}{3.538681in}}%
\pgfpathlineto{\pgfqpoint{3.649619in}{3.541528in}}%
\pgfpathlineto{\pgfqpoint{3.643345in}{3.551547in}}%
\pgfpathlineto{\pgfqpoint{3.637068in}{3.561187in}}%
\pgfpathlineto{\pgfqpoint{3.630790in}{3.570461in}}%
\pgfpathlineto{\pgfqpoint{3.624511in}{3.579406in}}%
\pgfpathlineto{\pgfqpoint{3.618232in}{3.588061in}}%
\pgfpathlineto{\pgfqpoint{3.607226in}{3.584437in}}%
\pgfpathlineto{\pgfqpoint{3.596216in}{3.580855in}}%
\pgfpathlineto{\pgfqpoint{3.585200in}{3.577258in}}%
\pgfpathlineto{\pgfqpoint{3.574179in}{3.573593in}}%
\pgfpathlineto{\pgfqpoint{3.563152in}{3.569840in}}%
\pgfpathlineto{\pgfqpoint{3.569433in}{3.562133in}}%
\pgfpathlineto{\pgfqpoint{3.575714in}{3.554057in}}%
\pgfpathlineto{\pgfqpoint{3.581994in}{3.545546in}}%
\pgfpathlineto{\pgfqpoint{3.588271in}{3.536532in}}%
\pgfpathclose%
\pgfusepath{stroke,fill}%
\end{pgfscope}%
\begin{pgfscope}%
\pgfpathrectangle{\pgfqpoint{0.887500in}{0.275000in}}{\pgfqpoint{4.225000in}{4.225000in}}%
\pgfusepath{clip}%
\pgfsetbuttcap%
\pgfsetroundjoin%
\definecolor{currentfill}{rgb}{0.119483,0.614817,0.537692}%
\pgfsetfillcolor{currentfill}%
\pgfsetfillopacity{0.700000}%
\pgfsetlinewidth{0.501875pt}%
\definecolor{currentstroke}{rgb}{1.000000,1.000000,1.000000}%
\pgfsetstrokecolor{currentstroke}%
\pgfsetstrokeopacity{0.500000}%
\pgfsetdash{}{0pt}%
\pgfpathmoveto{\pgfqpoint{1.725851in}{2.787892in}}%
\pgfpathlineto{\pgfqpoint{1.737330in}{2.791184in}}%
\pgfpathlineto{\pgfqpoint{1.748803in}{2.794482in}}%
\pgfpathlineto{\pgfqpoint{1.760271in}{2.797789in}}%
\pgfpathlineto{\pgfqpoint{1.771733in}{2.801105in}}%
\pgfpathlineto{\pgfqpoint{1.783189in}{2.804432in}}%
\pgfpathlineto{\pgfqpoint{1.777509in}{2.812043in}}%
\pgfpathlineto{\pgfqpoint{1.771833in}{2.819641in}}%
\pgfpathlineto{\pgfqpoint{1.766162in}{2.827224in}}%
\pgfpathlineto{\pgfqpoint{1.760494in}{2.834794in}}%
\pgfpathlineto{\pgfqpoint{1.754831in}{2.842349in}}%
\pgfpathlineto{\pgfqpoint{1.743389in}{2.839000in}}%
\pgfpathlineto{\pgfqpoint{1.731941in}{2.835662in}}%
\pgfpathlineto{\pgfqpoint{1.720488in}{2.832334in}}%
\pgfpathlineto{\pgfqpoint{1.709028in}{2.829014in}}%
\pgfpathlineto{\pgfqpoint{1.697563in}{2.825701in}}%
\pgfpathlineto{\pgfqpoint{1.703212in}{2.818168in}}%
\pgfpathlineto{\pgfqpoint{1.708866in}{2.810620in}}%
\pgfpathlineto{\pgfqpoint{1.714523in}{2.803059in}}%
\pgfpathlineto{\pgfqpoint{1.720185in}{2.795483in}}%
\pgfpathclose%
\pgfusepath{stroke,fill}%
\end{pgfscope}%
\begin{pgfscope}%
\pgfpathrectangle{\pgfqpoint{0.887500in}{0.275000in}}{\pgfqpoint{4.225000in}{4.225000in}}%
\pgfusepath{clip}%
\pgfsetbuttcap%
\pgfsetroundjoin%
\definecolor{currentfill}{rgb}{0.123463,0.581687,0.547445}%
\pgfsetfillcolor{currentfill}%
\pgfsetfillopacity{0.700000}%
\pgfsetlinewidth{0.501875pt}%
\definecolor{currentstroke}{rgb}{1.000000,1.000000,1.000000}%
\pgfsetstrokecolor{currentstroke}%
\pgfsetstrokeopacity{0.500000}%
\pgfsetdash{}{0pt}%
\pgfpathmoveto{\pgfqpoint{2.269580in}{2.708097in}}%
\pgfpathlineto{\pgfqpoint{2.280929in}{2.711365in}}%
\pgfpathlineto{\pgfqpoint{2.292263in}{2.715164in}}%
\pgfpathlineto{\pgfqpoint{2.303578in}{2.719679in}}%
\pgfpathlineto{\pgfqpoint{2.314871in}{2.725096in}}%
\pgfpathlineto{\pgfqpoint{2.326142in}{2.731461in}}%
\pgfpathlineto{\pgfqpoint{2.320261in}{2.739941in}}%
\pgfpathlineto{\pgfqpoint{2.314385in}{2.748366in}}%
\pgfpathlineto{\pgfqpoint{2.308514in}{2.756721in}}%
\pgfpathlineto{\pgfqpoint{2.302648in}{2.764994in}}%
\pgfpathlineto{\pgfqpoint{2.296788in}{2.773170in}}%
\pgfpathlineto{\pgfqpoint{2.285550in}{2.765782in}}%
\pgfpathlineto{\pgfqpoint{2.274284in}{2.759659in}}%
\pgfpathlineto{\pgfqpoint{2.262990in}{2.754754in}}%
\pgfpathlineto{\pgfqpoint{2.251672in}{2.750817in}}%
\pgfpathlineto{\pgfqpoint{2.240334in}{2.747596in}}%
\pgfpathlineto{\pgfqpoint{2.246174in}{2.739763in}}%
\pgfpathlineto{\pgfqpoint{2.252019in}{2.731888in}}%
\pgfpathlineto{\pgfqpoint{2.257869in}{2.723980in}}%
\pgfpathlineto{\pgfqpoint{2.263722in}{2.716047in}}%
\pgfpathclose%
\pgfusepath{stroke,fill}%
\end{pgfscope}%
\begin{pgfscope}%
\pgfpathrectangle{\pgfqpoint{0.887500in}{0.275000in}}{\pgfqpoint{4.225000in}{4.225000in}}%
\pgfusepath{clip}%
\pgfsetbuttcap%
\pgfsetroundjoin%
\definecolor{currentfill}{rgb}{0.824940,0.884720,0.106217}%
\pgfsetfillcolor{currentfill}%
\pgfsetfillopacity{0.700000}%
\pgfsetlinewidth{0.501875pt}%
\definecolor{currentstroke}{rgb}{1.000000,1.000000,1.000000}%
\pgfsetstrokecolor{currentstroke}%
\pgfsetstrokeopacity{0.500000}%
\pgfsetdash{}{0pt}%
\pgfpathmoveto{\pgfqpoint{3.223785in}{3.540470in}}%
\pgfpathlineto{\pgfqpoint{3.234908in}{3.543795in}}%
\pgfpathlineto{\pgfqpoint{3.246026in}{3.547100in}}%
\pgfpathlineto{\pgfqpoint{3.257139in}{3.550358in}}%
\pgfpathlineto{\pgfqpoint{3.268246in}{3.553541in}}%
\pgfpathlineto{\pgfqpoint{3.279347in}{3.556621in}}%
\pgfpathlineto{\pgfqpoint{3.273144in}{3.561529in}}%
\pgfpathlineto{\pgfqpoint{3.266945in}{3.566270in}}%
\pgfpathlineto{\pgfqpoint{3.260750in}{3.570851in}}%
\pgfpathlineto{\pgfqpoint{3.254560in}{3.575278in}}%
\pgfpathlineto{\pgfqpoint{3.248373in}{3.579556in}}%
\pgfpathlineto{\pgfqpoint{3.237286in}{3.576230in}}%
\pgfpathlineto{\pgfqpoint{3.226194in}{3.572967in}}%
\pgfpathlineto{\pgfqpoint{3.215097in}{3.569761in}}%
\pgfpathlineto{\pgfqpoint{3.203995in}{3.566606in}}%
\pgfpathlineto{\pgfqpoint{3.192888in}{3.563497in}}%
\pgfpathlineto{\pgfqpoint{3.199059in}{3.559436in}}%
\pgfpathlineto{\pgfqpoint{3.205234in}{3.555093in}}%
\pgfpathlineto{\pgfqpoint{3.211414in}{3.550477in}}%
\pgfpathlineto{\pgfqpoint{3.217597in}{3.545599in}}%
\pgfpathclose%
\pgfusepath{stroke,fill}%
\end{pgfscope}%
\begin{pgfscope}%
\pgfpathrectangle{\pgfqpoint{0.887500in}{0.275000in}}{\pgfqpoint{4.225000in}{4.225000in}}%
\pgfusepath{clip}%
\pgfsetbuttcap%
\pgfsetroundjoin%
\definecolor{currentfill}{rgb}{0.762373,0.876424,0.137064}%
\pgfsetfillcolor{currentfill}%
\pgfsetfillopacity{0.700000}%
\pgfsetlinewidth{0.501875pt}%
\definecolor{currentstroke}{rgb}{1.000000,1.000000,1.000000}%
\pgfsetstrokecolor{currentstroke}%
\pgfsetstrokeopacity{0.500000}%
\pgfsetdash{}{0pt}%
\pgfpathmoveto{\pgfqpoint{2.939207in}{3.505941in}}%
\pgfpathlineto{\pgfqpoint{2.950404in}{3.507873in}}%
\pgfpathlineto{\pgfqpoint{2.961597in}{3.509005in}}%
\pgfpathlineto{\pgfqpoint{2.972784in}{3.509821in}}%
\pgfpathlineto{\pgfqpoint{2.983964in}{3.510802in}}%
\pgfpathlineto{\pgfqpoint{2.995137in}{3.512429in}}%
\pgfpathlineto{\pgfqpoint{2.989051in}{3.515136in}}%
\pgfpathlineto{\pgfqpoint{2.982971in}{3.517738in}}%
\pgfpathlineto{\pgfqpoint{2.976896in}{3.520164in}}%
\pgfpathlineto{\pgfqpoint{2.970828in}{3.522346in}}%
\pgfpathlineto{\pgfqpoint{2.964766in}{3.524215in}}%
\pgfpathlineto{\pgfqpoint{2.953611in}{3.522399in}}%
\pgfpathlineto{\pgfqpoint{2.942450in}{3.520833in}}%
\pgfpathlineto{\pgfqpoint{2.931282in}{3.519185in}}%
\pgfpathlineto{\pgfqpoint{2.920110in}{3.517127in}}%
\pgfpathlineto{\pgfqpoint{2.908935in}{3.514328in}}%
\pgfpathlineto{\pgfqpoint{2.914973in}{3.513947in}}%
\pgfpathlineto{\pgfqpoint{2.921020in}{3.512790in}}%
\pgfpathlineto{\pgfqpoint{2.927075in}{3.510985in}}%
\pgfpathlineto{\pgfqpoint{2.933137in}{3.508659in}}%
\pgfpathclose%
\pgfusepath{stroke,fill}%
\end{pgfscope}%
\begin{pgfscope}%
\pgfpathrectangle{\pgfqpoint{0.887500in}{0.275000in}}{\pgfqpoint{4.225000in}{4.225000in}}%
\pgfusepath{clip}%
\pgfsetbuttcap%
\pgfsetroundjoin%
\definecolor{currentfill}{rgb}{0.783315,0.879285,0.125405}%
\pgfsetfillcolor{currentfill}%
\pgfsetfillopacity{0.700000}%
\pgfsetlinewidth{0.501875pt}%
\definecolor{currentstroke}{rgb}{1.000000,1.000000,1.000000}%
\pgfsetstrokecolor{currentstroke}%
\pgfsetstrokeopacity{0.500000}%
\pgfsetdash{}{0pt}%
\pgfpathmoveto{\pgfqpoint{3.680985in}{3.487494in}}%
\pgfpathlineto{\pgfqpoint{3.691990in}{3.490393in}}%
\pgfpathlineto{\pgfqpoint{3.702993in}{3.493510in}}%
\pgfpathlineto{\pgfqpoint{3.713996in}{3.496925in}}%
\pgfpathlineto{\pgfqpoint{3.724999in}{3.500713in}}%
\pgfpathlineto{\pgfqpoint{3.736004in}{3.504866in}}%
\pgfpathlineto{\pgfqpoint{3.729723in}{3.516012in}}%
\pgfpathlineto{\pgfqpoint{3.723444in}{3.527070in}}%
\pgfpathlineto{\pgfqpoint{3.717167in}{3.538007in}}%
\pgfpathlineto{\pgfqpoint{3.710890in}{3.548786in}}%
\pgfpathlineto{\pgfqpoint{3.704613in}{3.559373in}}%
\pgfpathlineto{\pgfqpoint{3.693613in}{3.555105in}}%
\pgfpathlineto{\pgfqpoint{3.682615in}{3.551215in}}%
\pgfpathlineto{\pgfqpoint{3.671617in}{3.547710in}}%
\pgfpathlineto{\pgfqpoint{3.660619in}{3.544509in}}%
\pgfpathlineto{\pgfqpoint{3.649619in}{3.541528in}}%
\pgfpathlineto{\pgfqpoint{3.655893in}{3.531174in}}%
\pgfpathlineto{\pgfqpoint{3.662165in}{3.520540in}}%
\pgfpathlineto{\pgfqpoint{3.668438in}{3.509679in}}%
\pgfpathlineto{\pgfqpoint{3.674711in}{3.498646in}}%
\pgfpathclose%
\pgfusepath{stroke,fill}%
\end{pgfscope}%
\begin{pgfscope}%
\pgfpathrectangle{\pgfqpoint{0.887500in}{0.275000in}}{\pgfqpoint{4.225000in}{4.225000in}}%
\pgfusepath{clip}%
\pgfsetbuttcap%
\pgfsetroundjoin%
\definecolor{currentfill}{rgb}{0.120565,0.596422,0.543611}%
\pgfsetfillcolor{currentfill}%
\pgfsetfillopacity{0.700000}%
\pgfsetlinewidth{0.501875pt}%
\definecolor{currentstroke}{rgb}{1.000000,1.000000,1.000000}%
\pgfsetstrokecolor{currentstroke}%
\pgfsetstrokeopacity{0.500000}%
\pgfsetdash{}{0pt}%
\pgfpathmoveto{\pgfqpoint{2.411982in}{2.720080in}}%
\pgfpathlineto{\pgfqpoint{2.423231in}{2.727305in}}%
\pgfpathlineto{\pgfqpoint{2.434472in}{2.734825in}}%
\pgfpathlineto{\pgfqpoint{2.445703in}{2.742757in}}%
\pgfpathlineto{\pgfqpoint{2.456924in}{2.751218in}}%
\pgfpathlineto{\pgfqpoint{2.468132in}{2.760330in}}%
\pgfpathlineto{\pgfqpoint{2.462155in}{2.772187in}}%
\pgfpathlineto{\pgfqpoint{2.456184in}{2.783810in}}%
\pgfpathlineto{\pgfqpoint{2.450220in}{2.795127in}}%
\pgfpathlineto{\pgfqpoint{2.444264in}{2.806137in}}%
\pgfpathlineto{\pgfqpoint{2.438315in}{2.816863in}}%
\pgfpathlineto{\pgfqpoint{2.427120in}{2.807641in}}%
\pgfpathlineto{\pgfqpoint{2.415919in}{2.798641in}}%
\pgfpathlineto{\pgfqpoint{2.404713in}{2.789783in}}%
\pgfpathlineto{\pgfqpoint{2.393504in}{2.780985in}}%
\pgfpathlineto{\pgfqpoint{2.382293in}{2.772169in}}%
\pgfpathlineto{\pgfqpoint{2.388221in}{2.762000in}}%
\pgfpathlineto{\pgfqpoint{2.394153in}{2.751693in}}%
\pgfpathlineto{\pgfqpoint{2.400092in}{2.741256in}}%
\pgfpathlineto{\pgfqpoint{2.406035in}{2.730705in}}%
\pgfpathclose%
\pgfusepath{stroke,fill}%
\end{pgfscope}%
\begin{pgfscope}%
\pgfpathrectangle{\pgfqpoint{0.887500in}{0.275000in}}{\pgfqpoint{4.225000in}{4.225000in}}%
\pgfusepath{clip}%
\pgfsetbuttcap%
\pgfsetroundjoin%
\definecolor{currentfill}{rgb}{0.140210,0.665859,0.513427}%
\pgfsetfillcolor{currentfill}%
\pgfsetfillopacity{0.700000}%
\pgfsetlinewidth{0.501875pt}%
\definecolor{currentstroke}{rgb}{1.000000,1.000000,1.000000}%
\pgfsetstrokecolor{currentstroke}%
\pgfsetstrokeopacity{0.500000}%
\pgfsetdash{}{0pt}%
\pgfpathmoveto{\pgfqpoint{2.665739in}{2.873484in}}%
\pgfpathlineto{\pgfqpoint{2.676919in}{2.884127in}}%
\pgfpathlineto{\pgfqpoint{2.688094in}{2.895266in}}%
\pgfpathlineto{\pgfqpoint{2.699266in}{2.906578in}}%
\pgfpathlineto{\pgfqpoint{2.710439in}{2.917600in}}%
\pgfpathlineto{\pgfqpoint{2.721618in}{2.927868in}}%
\pgfpathlineto{\pgfqpoint{2.715621in}{2.934704in}}%
\pgfpathlineto{\pgfqpoint{2.709631in}{2.941278in}}%
\pgfpathlineto{\pgfqpoint{2.703647in}{2.947657in}}%
\pgfpathlineto{\pgfqpoint{2.697669in}{2.953908in}}%
\pgfpathlineto{\pgfqpoint{2.691695in}{2.960098in}}%
\pgfpathlineto{\pgfqpoint{2.680514in}{2.952083in}}%
\pgfpathlineto{\pgfqpoint{2.669337in}{2.943344in}}%
\pgfpathlineto{\pgfqpoint{2.658160in}{2.934395in}}%
\pgfpathlineto{\pgfqpoint{2.646977in}{2.925749in}}%
\pgfpathlineto{\pgfqpoint{2.635786in}{2.917751in}}%
\pgfpathlineto{\pgfqpoint{2.641759in}{2.909910in}}%
\pgfpathlineto{\pgfqpoint{2.647741in}{2.901503in}}%
\pgfpathlineto{\pgfqpoint{2.653732in}{2.892591in}}%
\pgfpathlineto{\pgfqpoint{2.659732in}{2.883231in}}%
\pgfpathclose%
\pgfusepath{stroke,fill}%
\end{pgfscope}%
\begin{pgfscope}%
\pgfpathrectangle{\pgfqpoint{0.887500in}{0.275000in}}{\pgfqpoint{4.225000in}{4.225000in}}%
\pgfusepath{clip}%
\pgfsetbuttcap%
\pgfsetroundjoin%
\definecolor{currentfill}{rgb}{0.804182,0.882046,0.114965}%
\pgfsetfillcolor{currentfill}%
\pgfsetfillopacity{0.700000}%
\pgfsetlinewidth{0.501875pt}%
\definecolor{currentstroke}{rgb}{1.000000,1.000000,1.000000}%
\pgfsetstrokecolor{currentstroke}%
\pgfsetstrokeopacity{0.500000}%
\pgfsetdash{}{0pt}%
\pgfpathmoveto{\pgfqpoint{3.081515in}{3.522890in}}%
\pgfpathlineto{\pgfqpoint{3.092676in}{3.528651in}}%
\pgfpathlineto{\pgfqpoint{3.103832in}{3.534127in}}%
\pgfpathlineto{\pgfqpoint{3.114983in}{3.539067in}}%
\pgfpathlineto{\pgfqpoint{3.126129in}{3.543453in}}%
\pgfpathlineto{\pgfqpoint{3.137270in}{3.547379in}}%
\pgfpathlineto{\pgfqpoint{3.131119in}{3.550907in}}%
\pgfpathlineto{\pgfqpoint{3.124973in}{3.553962in}}%
\pgfpathlineto{\pgfqpoint{3.118831in}{3.556522in}}%
\pgfpathlineto{\pgfqpoint{3.112694in}{3.558623in}}%
\pgfpathlineto{\pgfqpoint{3.106563in}{3.560320in}}%
\pgfpathlineto{\pgfqpoint{3.095441in}{3.555999in}}%
\pgfpathlineto{\pgfqpoint{3.084314in}{3.551306in}}%
\pgfpathlineto{\pgfqpoint{3.073182in}{3.546156in}}%
\pgfpathlineto{\pgfqpoint{3.062046in}{3.540575in}}%
\pgfpathlineto{\pgfqpoint{3.050906in}{3.534804in}}%
\pgfpathlineto{\pgfqpoint{3.057016in}{3.532916in}}%
\pgfpathlineto{\pgfqpoint{3.063132in}{3.530797in}}%
\pgfpathlineto{\pgfqpoint{3.069254in}{3.528429in}}%
\pgfpathlineto{\pgfqpoint{3.075382in}{3.525797in}}%
\pgfpathclose%
\pgfusepath{stroke,fill}%
\end{pgfscope}%
\begin{pgfscope}%
\pgfpathrectangle{\pgfqpoint{0.887500in}{0.275000in}}{\pgfqpoint{4.225000in}{4.225000in}}%
\pgfusepath{clip}%
\pgfsetbuttcap%
\pgfsetroundjoin%
\definecolor{currentfill}{rgb}{0.120565,0.596422,0.543611}%
\pgfsetfillcolor{currentfill}%
\pgfsetfillopacity{0.700000}%
\pgfsetlinewidth{0.501875pt}%
\definecolor{currentstroke}{rgb}{1.000000,1.000000,1.000000}%
\pgfsetstrokecolor{currentstroke}%
\pgfsetstrokeopacity{0.500000}%
\pgfsetdash{}{0pt}%
\pgfpathmoveto{\pgfqpoint{2.040643in}{2.738335in}}%
\pgfpathlineto{\pgfqpoint{2.052051in}{2.741599in}}%
\pgfpathlineto{\pgfqpoint{2.063452in}{2.744906in}}%
\pgfpathlineto{\pgfqpoint{2.074846in}{2.748280in}}%
\pgfpathlineto{\pgfqpoint{2.086232in}{2.751748in}}%
\pgfpathlineto{\pgfqpoint{2.097610in}{2.755334in}}%
\pgfpathlineto{\pgfqpoint{2.091815in}{2.763236in}}%
\pgfpathlineto{\pgfqpoint{2.086024in}{2.771126in}}%
\pgfpathlineto{\pgfqpoint{2.080237in}{2.779002in}}%
\pgfpathlineto{\pgfqpoint{2.074455in}{2.786866in}}%
\pgfpathlineto{\pgfqpoint{2.068676in}{2.794718in}}%
\pgfpathlineto{\pgfqpoint{2.057312in}{2.791099in}}%
\pgfpathlineto{\pgfqpoint{2.045939in}{2.787601in}}%
\pgfpathlineto{\pgfqpoint{2.034558in}{2.784199in}}%
\pgfpathlineto{\pgfqpoint{2.023170in}{2.780866in}}%
\pgfpathlineto{\pgfqpoint{2.011776in}{2.777577in}}%
\pgfpathlineto{\pgfqpoint{2.017541in}{2.769754in}}%
\pgfpathlineto{\pgfqpoint{2.023311in}{2.761919in}}%
\pgfpathlineto{\pgfqpoint{2.029084in}{2.754071in}}%
\pgfpathlineto{\pgfqpoint{2.034862in}{2.746209in}}%
\pgfpathclose%
\pgfusepath{stroke,fill}%
\end{pgfscope}%
\begin{pgfscope}%
\pgfpathrectangle{\pgfqpoint{0.887500in}{0.275000in}}{\pgfqpoint{4.225000in}{4.225000in}}%
\pgfusepath{clip}%
\pgfsetbuttcap%
\pgfsetroundjoin%
\definecolor{currentfill}{rgb}{0.730889,0.871916,0.156029}%
\pgfsetfillcolor{currentfill}%
\pgfsetfillopacity{0.700000}%
\pgfsetlinewidth{0.501875pt}%
\definecolor{currentstroke}{rgb}{1.000000,1.000000,1.000000}%
\pgfsetstrokecolor{currentstroke}%
\pgfsetstrokeopacity{0.500000}%
\pgfsetdash{}{0pt}%
\pgfpathmoveto{\pgfqpoint{3.767454in}{3.449056in}}%
\pgfpathlineto{\pgfqpoint{3.778460in}{3.453147in}}%
\pgfpathlineto{\pgfqpoint{3.789463in}{3.457389in}}%
\pgfpathlineto{\pgfqpoint{3.800464in}{3.461733in}}%
\pgfpathlineto{\pgfqpoint{3.811461in}{3.466132in}}%
\pgfpathlineto{\pgfqpoint{3.822453in}{3.470538in}}%
\pgfpathlineto{\pgfqpoint{3.816160in}{3.482217in}}%
\pgfpathlineto{\pgfqpoint{3.809869in}{3.493878in}}%
\pgfpathlineto{\pgfqpoint{3.803581in}{3.505498in}}%
\pgfpathlineto{\pgfqpoint{3.797294in}{3.517054in}}%
\pgfpathlineto{\pgfqpoint{3.791009in}{3.528522in}}%
\pgfpathlineto{\pgfqpoint{3.780015in}{3.523655in}}%
\pgfpathlineto{\pgfqpoint{3.769015in}{3.518781in}}%
\pgfpathlineto{\pgfqpoint{3.758013in}{3.513975in}}%
\pgfpathlineto{\pgfqpoint{3.747009in}{3.509312in}}%
\pgfpathlineto{\pgfqpoint{3.736004in}{3.504866in}}%
\pgfpathlineto{\pgfqpoint{3.742287in}{3.493669in}}%
\pgfpathlineto{\pgfqpoint{3.748573in}{3.482454in}}%
\pgfpathlineto{\pgfqpoint{3.754862in}{3.471258in}}%
\pgfpathlineto{\pgfqpoint{3.761156in}{3.460114in}}%
\pgfpathclose%
\pgfusepath{stroke,fill}%
\end{pgfscope}%
\begin{pgfscope}%
\pgfpathrectangle{\pgfqpoint{0.887500in}{0.275000in}}{\pgfqpoint{4.225000in}{4.225000in}}%
\pgfusepath{clip}%
\pgfsetbuttcap%
\pgfsetroundjoin%
\definecolor{currentfill}{rgb}{0.121380,0.629492,0.531973}%
\pgfsetfillcolor{currentfill}%
\pgfsetfillopacity{0.700000}%
\pgfsetlinewidth{0.501875pt}%
\definecolor{currentstroke}{rgb}{1.000000,1.000000,1.000000}%
\pgfsetstrokecolor{currentstroke}%
\pgfsetstrokeopacity{0.500000}%
\pgfsetdash{}{0pt}%
\pgfpathmoveto{\pgfqpoint{1.496947in}{2.813175in}}%
\pgfpathlineto{\pgfqpoint{1.508482in}{2.816510in}}%
\pgfpathlineto{\pgfqpoint{1.520010in}{2.819845in}}%
\pgfpathlineto{\pgfqpoint{1.531533in}{2.823179in}}%
\pgfpathlineto{\pgfqpoint{1.543051in}{2.826512in}}%
\pgfpathlineto{\pgfqpoint{1.554563in}{2.829844in}}%
\pgfpathlineto{\pgfqpoint{1.548969in}{2.837220in}}%
\pgfpathlineto{\pgfqpoint{1.543379in}{2.844576in}}%
\pgfpathlineto{\pgfqpoint{1.537793in}{2.851911in}}%
\pgfpathlineto{\pgfqpoint{1.532212in}{2.859226in}}%
\pgfpathlineto{\pgfqpoint{1.520712in}{2.855881in}}%
\pgfpathlineto{\pgfqpoint{1.509205in}{2.852541in}}%
\pgfpathlineto{\pgfqpoint{1.497693in}{2.849203in}}%
\pgfpathlineto{\pgfqpoint{1.486176in}{2.845865in}}%
\pgfpathlineto{\pgfqpoint{1.474653in}{2.842526in}}%
\pgfpathlineto{\pgfqpoint{1.480220in}{2.835217in}}%
\pgfpathlineto{\pgfqpoint{1.485791in}{2.827888in}}%
\pgfpathlineto{\pgfqpoint{1.491367in}{2.820541in}}%
\pgfpathclose%
\pgfusepath{stroke,fill}%
\end{pgfscope}%
\begin{pgfscope}%
\pgfpathrectangle{\pgfqpoint{0.887500in}{0.275000in}}{\pgfqpoint{4.225000in}{4.225000in}}%
\pgfusepath{clip}%
\pgfsetbuttcap%
\pgfsetroundjoin%
\definecolor{currentfill}{rgb}{0.688944,0.865448,0.182725}%
\pgfsetfillcolor{currentfill}%
\pgfsetfillopacity{0.700000}%
\pgfsetlinewidth{0.501875pt}%
\definecolor{currentstroke}{rgb}{1.000000,1.000000,1.000000}%
\pgfsetstrokecolor{currentstroke}%
\pgfsetstrokeopacity{0.500000}%
\pgfsetdash{}{0pt}%
\pgfpathmoveto{\pgfqpoint{3.853966in}{3.412120in}}%
\pgfpathlineto{\pgfqpoint{3.864945in}{3.415779in}}%
\pgfpathlineto{\pgfqpoint{3.875920in}{3.419483in}}%
\pgfpathlineto{\pgfqpoint{3.886891in}{3.423235in}}%
\pgfpathlineto{\pgfqpoint{3.897859in}{3.427036in}}%
\pgfpathlineto{\pgfqpoint{3.908822in}{3.430891in}}%
\pgfpathlineto{\pgfqpoint{3.902520in}{3.443126in}}%
\pgfpathlineto{\pgfqpoint{3.896220in}{3.455369in}}%
\pgfpathlineto{\pgfqpoint{3.889923in}{3.467604in}}%
\pgfpathlineto{\pgfqpoint{3.883629in}{3.479816in}}%
\pgfpathlineto{\pgfqpoint{3.877337in}{3.491986in}}%
\pgfpathlineto{\pgfqpoint{3.866371in}{3.487763in}}%
\pgfpathlineto{\pgfqpoint{3.855399in}{3.483513in}}%
\pgfpathlineto{\pgfqpoint{3.844423in}{3.479231in}}%
\pgfpathlineto{\pgfqpoint{3.833441in}{3.474908in}}%
\pgfpathlineto{\pgfqpoint{3.822453in}{3.470538in}}%
\pgfpathlineto{\pgfqpoint{3.828750in}{3.458852in}}%
\pgfpathlineto{\pgfqpoint{3.835049in}{3.447162in}}%
\pgfpathlineto{\pgfqpoint{3.841351in}{3.435474in}}%
\pgfpathlineto{\pgfqpoint{3.847657in}{3.423792in}}%
\pgfpathclose%
\pgfusepath{stroke,fill}%
\end{pgfscope}%
\begin{pgfscope}%
\pgfpathrectangle{\pgfqpoint{0.887500in}{0.275000in}}{\pgfqpoint{4.225000in}{4.225000in}}%
\pgfusepath{clip}%
\pgfsetbuttcap%
\pgfsetroundjoin%
\definecolor{currentfill}{rgb}{0.626579,0.854645,0.223353}%
\pgfsetfillcolor{currentfill}%
\pgfsetfillopacity{0.700000}%
\pgfsetlinewidth{0.501875pt}%
\definecolor{currentstroke}{rgb}{1.000000,1.000000,1.000000}%
\pgfsetstrokecolor{currentstroke}%
\pgfsetstrokeopacity{0.500000}%
\pgfsetdash{}{0pt}%
\pgfpathmoveto{\pgfqpoint{3.940394in}{3.370425in}}%
\pgfpathlineto{\pgfqpoint{3.951351in}{3.373974in}}%
\pgfpathlineto{\pgfqpoint{3.962305in}{3.377612in}}%
\pgfpathlineto{\pgfqpoint{3.973255in}{3.381341in}}%
\pgfpathlineto{\pgfqpoint{3.984203in}{3.385158in}}%
\pgfpathlineto{\pgfqpoint{3.995149in}{3.389064in}}%
\pgfpathlineto{\pgfqpoint{3.988829in}{3.401411in}}%
\pgfpathlineto{\pgfqpoint{3.982514in}{3.413794in}}%
\pgfpathlineto{\pgfqpoint{3.976202in}{3.426203in}}%
\pgfpathlineto{\pgfqpoint{3.969893in}{3.438629in}}%
\pgfpathlineto{\pgfqpoint{3.963587in}{3.451061in}}%
\pgfpathlineto{\pgfqpoint{3.952641in}{3.446899in}}%
\pgfpathlineto{\pgfqpoint{3.941691in}{3.442803in}}%
\pgfpathlineto{\pgfqpoint{3.930738in}{3.438772in}}%
\pgfpathlineto{\pgfqpoint{3.919782in}{3.434802in}}%
\pgfpathlineto{\pgfqpoint{3.908822in}{3.430891in}}%
\pgfpathlineto{\pgfqpoint{3.915128in}{3.418682in}}%
\pgfpathlineto{\pgfqpoint{3.921438in}{3.406514in}}%
\pgfpathlineto{\pgfqpoint{3.927752in}{3.394405in}}%
\pgfpathlineto{\pgfqpoint{3.934070in}{3.382370in}}%
\pgfpathclose%
\pgfusepath{stroke,fill}%
\end{pgfscope}%
\begin{pgfscope}%
\pgfpathrectangle{\pgfqpoint{0.887500in}{0.275000in}}{\pgfqpoint{4.225000in}{4.225000in}}%
\pgfusepath{clip}%
\pgfsetbuttcap%
\pgfsetroundjoin%
\definecolor{currentfill}{rgb}{0.824940,0.884720,0.106217}%
\pgfsetfillcolor{currentfill}%
\pgfsetfillopacity{0.700000}%
\pgfsetlinewidth{0.501875pt}%
\definecolor{currentstroke}{rgb}{1.000000,1.000000,1.000000}%
\pgfsetstrokecolor{currentstroke}%
\pgfsetstrokeopacity{0.500000}%
\pgfsetdash{}{0pt}%
\pgfpathmoveto{\pgfqpoint{3.452585in}{3.530009in}}%
\pgfpathlineto{\pgfqpoint{3.463662in}{3.533902in}}%
\pgfpathlineto{\pgfqpoint{3.474736in}{3.537890in}}%
\pgfpathlineto{\pgfqpoint{3.485805in}{3.541939in}}%
\pgfpathlineto{\pgfqpoint{3.496870in}{3.546018in}}%
\pgfpathlineto{\pgfqpoint{3.507930in}{3.550093in}}%
\pgfpathlineto{\pgfqpoint{3.501653in}{3.556806in}}%
\pgfpathlineto{\pgfqpoint{3.495378in}{3.563267in}}%
\pgfpathlineto{\pgfqpoint{3.489106in}{3.569557in}}%
\pgfpathlineto{\pgfqpoint{3.482838in}{3.575760in}}%
\pgfpathlineto{\pgfqpoint{3.476574in}{3.581954in}}%
\pgfpathlineto{\pgfqpoint{3.465520in}{3.577236in}}%
\pgfpathlineto{\pgfqpoint{3.454463in}{3.572531in}}%
\pgfpathlineto{\pgfqpoint{3.443401in}{3.567906in}}%
\pgfpathlineto{\pgfqpoint{3.432337in}{3.563429in}}%
\pgfpathlineto{\pgfqpoint{3.421270in}{3.559167in}}%
\pgfpathlineto{\pgfqpoint{3.427525in}{3.553521in}}%
\pgfpathlineto{\pgfqpoint{3.433785in}{3.547869in}}%
\pgfpathlineto{\pgfqpoint{3.440049in}{3.542125in}}%
\pgfpathlineto{\pgfqpoint{3.446316in}{3.536201in}}%
\pgfpathclose%
\pgfusepath{stroke,fill}%
\end{pgfscope}%
\begin{pgfscope}%
\pgfpathrectangle{\pgfqpoint{0.887500in}{0.275000in}}{\pgfqpoint{4.225000in}{4.225000in}}%
\pgfusepath{clip}%
\pgfsetbuttcap%
\pgfsetroundjoin%
\definecolor{currentfill}{rgb}{0.119512,0.607464,0.540218}%
\pgfsetfillcolor{currentfill}%
\pgfsetfillopacity{0.700000}%
\pgfsetlinewidth{0.501875pt}%
\definecolor{currentstroke}{rgb}{1.000000,1.000000,1.000000}%
\pgfsetstrokecolor{currentstroke}%
\pgfsetstrokeopacity{0.500000}%
\pgfsetdash{}{0pt}%
\pgfpathmoveto{\pgfqpoint{1.811650in}{2.766171in}}%
\pgfpathlineto{\pgfqpoint{1.823113in}{2.769490in}}%
\pgfpathlineto{\pgfqpoint{1.834571in}{2.772821in}}%
\pgfpathlineto{\pgfqpoint{1.846023in}{2.776163in}}%
\pgfpathlineto{\pgfqpoint{1.857469in}{2.779517in}}%
\pgfpathlineto{\pgfqpoint{1.868909in}{2.782882in}}%
\pgfpathlineto{\pgfqpoint{1.863195in}{2.790581in}}%
\pgfpathlineto{\pgfqpoint{1.857485in}{2.798267in}}%
\pgfpathlineto{\pgfqpoint{1.851779in}{2.805940in}}%
\pgfpathlineto{\pgfqpoint{1.846077in}{2.813601in}}%
\pgfpathlineto{\pgfqpoint{1.840380in}{2.821248in}}%
\pgfpathlineto{\pgfqpoint{1.828953in}{2.817861in}}%
\pgfpathlineto{\pgfqpoint{1.817521in}{2.814486in}}%
\pgfpathlineto{\pgfqpoint{1.806083in}{2.811123in}}%
\pgfpathlineto{\pgfqpoint{1.794639in}{2.807771in}}%
\pgfpathlineto{\pgfqpoint{1.783189in}{2.804432in}}%
\pgfpathlineto{\pgfqpoint{1.788873in}{2.796807in}}%
\pgfpathlineto{\pgfqpoint{1.794561in}{2.789169in}}%
\pgfpathlineto{\pgfqpoint{1.800253in}{2.781516in}}%
\pgfpathlineto{\pgfqpoint{1.805949in}{2.773850in}}%
\pgfpathclose%
\pgfusepath{stroke,fill}%
\end{pgfscope}%
\begin{pgfscope}%
\pgfpathrectangle{\pgfqpoint{0.887500in}{0.275000in}}{\pgfqpoint{4.225000in}{4.225000in}}%
\pgfusepath{clip}%
\pgfsetbuttcap%
\pgfsetroundjoin%
\definecolor{currentfill}{rgb}{0.575563,0.844566,0.256415}%
\pgfsetfillcolor{currentfill}%
\pgfsetfillopacity{0.700000}%
\pgfsetlinewidth{0.501875pt}%
\definecolor{currentstroke}{rgb}{1.000000,1.000000,1.000000}%
\pgfsetstrokecolor{currentstroke}%
\pgfsetstrokeopacity{0.500000}%
\pgfsetdash{}{0pt}%
\pgfpathmoveto{\pgfqpoint{4.026802in}{3.327902in}}%
\pgfpathlineto{\pgfqpoint{4.037732in}{3.331209in}}%
\pgfpathlineto{\pgfqpoint{4.048658in}{3.334541in}}%
\pgfpathlineto{\pgfqpoint{4.059580in}{3.337898in}}%
\pgfpathlineto{\pgfqpoint{4.070497in}{3.341277in}}%
\pgfpathlineto{\pgfqpoint{4.064173in}{3.354150in}}%
\pgfpathlineto{\pgfqpoint{4.057852in}{3.367039in}}%
\pgfpathlineto{\pgfqpoint{4.051534in}{3.379920in}}%
\pgfpathlineto{\pgfqpoint{4.045218in}{3.392770in}}%
\pgfpathlineto{\pgfqpoint{4.038903in}{3.405566in}}%
\pgfpathlineto{\pgfqpoint{4.027968in}{3.401310in}}%
\pgfpathlineto{\pgfqpoint{4.017031in}{3.397141in}}%
\pgfpathlineto{\pgfqpoint{4.006091in}{3.393058in}}%
\pgfpathlineto{\pgfqpoint{3.995149in}{3.389064in}}%
\pgfpathlineto{\pgfqpoint{4.001472in}{3.376760in}}%
\pgfpathlineto{\pgfqpoint{4.007799in}{3.364496in}}%
\pgfpathlineto{\pgfqpoint{4.014129in}{3.352267in}}%
\pgfpathlineto{\pgfqpoint{4.020464in}{3.340071in}}%
\pgfpathclose%
\pgfusepath{stroke,fill}%
\end{pgfscope}%
\begin{pgfscope}%
\pgfpathrectangle{\pgfqpoint{0.887500in}{0.275000in}}{\pgfqpoint{4.225000in}{4.225000in}}%
\pgfusepath{clip}%
\pgfsetbuttcap%
\pgfsetroundjoin%
\definecolor{currentfill}{rgb}{0.128087,0.647749,0.523491}%
\pgfsetfillcolor{currentfill}%
\pgfsetfillopacity{0.700000}%
\pgfsetlinewidth{0.501875pt}%
\definecolor{currentstroke}{rgb}{1.000000,1.000000,1.000000}%
\pgfsetstrokecolor{currentstroke}%
\pgfsetstrokeopacity{0.500000}%
\pgfsetdash{}{0pt}%
\pgfpathmoveto{\pgfqpoint{2.609807in}{2.821192in}}%
\pgfpathlineto{\pgfqpoint{2.620989in}{2.832281in}}%
\pgfpathlineto{\pgfqpoint{2.632176in}{2.842829in}}%
\pgfpathlineto{\pgfqpoint{2.643365in}{2.853061in}}%
\pgfpathlineto{\pgfqpoint{2.654553in}{2.863204in}}%
\pgfpathlineto{\pgfqpoint{2.665739in}{2.873484in}}%
\pgfpathlineto{\pgfqpoint{2.659732in}{2.883231in}}%
\pgfpathlineto{\pgfqpoint{2.653732in}{2.892591in}}%
\pgfpathlineto{\pgfqpoint{2.647741in}{2.901503in}}%
\pgfpathlineto{\pgfqpoint{2.641759in}{2.909910in}}%
\pgfpathlineto{\pgfqpoint{2.635786in}{2.917751in}}%
\pgfpathlineto{\pgfqpoint{2.624586in}{2.910197in}}%
\pgfpathlineto{\pgfqpoint{2.613383in}{2.902770in}}%
\pgfpathlineto{\pgfqpoint{2.602178in}{2.895153in}}%
\pgfpathlineto{\pgfqpoint{2.590976in}{2.887031in}}%
\pgfpathlineto{\pgfqpoint{2.579782in}{2.878088in}}%
\pgfpathlineto{\pgfqpoint{2.585769in}{2.867746in}}%
\pgfpathlineto{\pgfqpoint{2.591767in}{2.856718in}}%
\pgfpathlineto{\pgfqpoint{2.597775in}{2.845171in}}%
\pgfpathlineto{\pgfqpoint{2.603789in}{2.833274in}}%
\pgfpathclose%
\pgfusepath{stroke,fill}%
\end{pgfscope}%
\begin{pgfscope}%
\pgfpathrectangle{\pgfqpoint{0.887500in}{0.275000in}}{\pgfqpoint{4.225000in}{4.225000in}}%
\pgfusepath{clip}%
\pgfsetbuttcap%
\pgfsetroundjoin%
\definecolor{currentfill}{rgb}{0.120081,0.622161,0.534946}%
\pgfsetfillcolor{currentfill}%
\pgfsetfillopacity{0.700000}%
\pgfsetlinewidth{0.501875pt}%
\definecolor{currentstroke}{rgb}{1.000000,1.000000,1.000000}%
\pgfsetstrokecolor{currentstroke}%
\pgfsetstrokeopacity{0.500000}%
\pgfsetdash{}{0pt}%
\pgfpathmoveto{\pgfqpoint{1.582601in}{2.792671in}}%
\pgfpathlineto{\pgfqpoint{1.594122in}{2.795982in}}%
\pgfpathlineto{\pgfqpoint{1.605638in}{2.799289in}}%
\pgfpathlineto{\pgfqpoint{1.617148in}{2.802592in}}%
\pgfpathlineto{\pgfqpoint{1.628653in}{2.805892in}}%
\pgfpathlineto{\pgfqpoint{1.640152in}{2.809191in}}%
\pgfpathlineto{\pgfqpoint{1.634521in}{2.816685in}}%
\pgfpathlineto{\pgfqpoint{1.628894in}{2.824163in}}%
\pgfpathlineto{\pgfqpoint{1.623272in}{2.831623in}}%
\pgfpathlineto{\pgfqpoint{1.617654in}{2.839067in}}%
\pgfpathlineto{\pgfqpoint{1.612041in}{2.846494in}}%
\pgfpathlineto{\pgfqpoint{1.600556in}{2.843164in}}%
\pgfpathlineto{\pgfqpoint{1.589066in}{2.839835in}}%
\pgfpathlineto{\pgfqpoint{1.577571in}{2.836505in}}%
\pgfpathlineto{\pgfqpoint{1.566070in}{2.833175in}}%
\pgfpathlineto{\pgfqpoint{1.554563in}{2.829844in}}%
\pgfpathlineto{\pgfqpoint{1.560162in}{2.822448in}}%
\pgfpathlineto{\pgfqpoint{1.565765in}{2.815032in}}%
\pgfpathlineto{\pgfqpoint{1.571373in}{2.807597in}}%
\pgfpathlineto{\pgfqpoint{1.576985in}{2.800143in}}%
\pgfpathclose%
\pgfusepath{stroke,fill}%
\end{pgfscope}%
\begin{pgfscope}%
\pgfpathrectangle{\pgfqpoint{0.887500in}{0.275000in}}{\pgfqpoint{4.225000in}{4.225000in}}%
\pgfusepath{clip}%
\pgfsetbuttcap%
\pgfsetroundjoin%
\definecolor{currentfill}{rgb}{0.121831,0.589055,0.545623}%
\pgfsetfillcolor{currentfill}%
\pgfsetfillopacity{0.700000}%
\pgfsetlinewidth{0.501875pt}%
\definecolor{currentstroke}{rgb}{1.000000,1.000000,1.000000}%
\pgfsetstrokecolor{currentstroke}%
\pgfsetstrokeopacity{0.500000}%
\pgfsetdash{}{0pt}%
\pgfpathmoveto{\pgfqpoint{2.126644in}{2.715633in}}%
\pgfpathlineto{\pgfqpoint{2.138028in}{2.719281in}}%
\pgfpathlineto{\pgfqpoint{2.149405in}{2.722977in}}%
\pgfpathlineto{\pgfqpoint{2.160778in}{2.726650in}}%
\pgfpathlineto{\pgfqpoint{2.172147in}{2.730228in}}%
\pgfpathlineto{\pgfqpoint{2.183514in}{2.733642in}}%
\pgfpathlineto{\pgfqpoint{2.177686in}{2.741639in}}%
\pgfpathlineto{\pgfqpoint{2.171862in}{2.749612in}}%
\pgfpathlineto{\pgfqpoint{2.166043in}{2.757561in}}%
\pgfpathlineto{\pgfqpoint{2.160227in}{2.765485in}}%
\pgfpathlineto{\pgfqpoint{2.154416in}{2.773391in}}%
\pgfpathlineto{\pgfqpoint{2.143059in}{2.770048in}}%
\pgfpathlineto{\pgfqpoint{2.131702in}{2.766482in}}%
\pgfpathlineto{\pgfqpoint{2.120343in}{2.762783in}}%
\pgfpathlineto{\pgfqpoint{2.108979in}{2.759037in}}%
\pgfpathlineto{\pgfqpoint{2.097610in}{2.755334in}}%
\pgfpathlineto{\pgfqpoint{2.103409in}{2.747419in}}%
\pgfpathlineto{\pgfqpoint{2.109211in}{2.739491in}}%
\pgfpathlineto{\pgfqpoint{2.115018in}{2.731551in}}%
\pgfpathlineto{\pgfqpoint{2.120829in}{2.723598in}}%
\pgfpathclose%
\pgfusepath{stroke,fill}%
\end{pgfscope}%
\begin{pgfscope}%
\pgfpathrectangle{\pgfqpoint{0.887500in}{0.275000in}}{\pgfqpoint{4.225000in}{4.225000in}}%
\pgfusepath{clip}%
\pgfsetbuttcap%
\pgfsetroundjoin%
\definecolor{currentfill}{rgb}{0.124395,0.578002,0.548287}%
\pgfsetfillcolor{currentfill}%
\pgfsetfillopacity{0.700000}%
\pgfsetlinewidth{0.501875pt}%
\definecolor{currentstroke}{rgb}{1.000000,1.000000,1.000000}%
\pgfsetstrokecolor{currentstroke}%
\pgfsetstrokeopacity{0.500000}%
\pgfsetdash{}{0pt}%
\pgfpathmoveto{\pgfqpoint{2.355607in}{2.688706in}}%
\pgfpathlineto{\pgfqpoint{2.366903in}{2.694193in}}%
\pgfpathlineto{\pgfqpoint{2.378188in}{2.700121in}}%
\pgfpathlineto{\pgfqpoint{2.389462in}{2.706442in}}%
\pgfpathlineto{\pgfqpoint{2.400726in}{2.713111in}}%
\pgfpathlineto{\pgfqpoint{2.411982in}{2.720080in}}%
\pgfpathlineto{\pgfqpoint{2.406035in}{2.730705in}}%
\pgfpathlineto{\pgfqpoint{2.400092in}{2.741256in}}%
\pgfpathlineto{\pgfqpoint{2.394153in}{2.751693in}}%
\pgfpathlineto{\pgfqpoint{2.388221in}{2.762000in}}%
\pgfpathlineto{\pgfqpoint{2.382293in}{2.772169in}}%
\pgfpathlineto{\pgfqpoint{2.371080in}{2.763371in}}%
\pgfpathlineto{\pgfqpoint{2.359861in}{2.754742in}}%
\pgfpathlineto{\pgfqpoint{2.348634in}{2.746442in}}%
\pgfpathlineto{\pgfqpoint{2.337395in}{2.738629in}}%
\pgfpathlineto{\pgfqpoint{2.326142in}{2.731461in}}%
\pgfpathlineto{\pgfqpoint{2.332027in}{2.722940in}}%
\pgfpathlineto{\pgfqpoint{2.337916in}{2.714392in}}%
\pgfpathlineto{\pgfqpoint{2.343810in}{2.705830in}}%
\pgfpathlineto{\pgfqpoint{2.349706in}{2.697266in}}%
\pgfpathclose%
\pgfusepath{stroke,fill}%
\end{pgfscope}%
\begin{pgfscope}%
\pgfpathrectangle{\pgfqpoint{0.887500in}{0.275000in}}{\pgfqpoint{4.225000in}{4.225000in}}%
\pgfusepath{clip}%
\pgfsetbuttcap%
\pgfsetroundjoin%
\definecolor{currentfill}{rgb}{0.647257,0.858400,0.209861}%
\pgfsetfillcolor{currentfill}%
\pgfsetfillopacity{0.700000}%
\pgfsetlinewidth{0.501875pt}%
\definecolor{currentstroke}{rgb}{1.000000,1.000000,1.000000}%
\pgfsetstrokecolor{currentstroke}%
\pgfsetstrokeopacity{0.500000}%
\pgfsetdash{}{0pt}%
\pgfpathmoveto{\pgfqpoint{2.827854in}{3.318601in}}%
\pgfpathlineto{\pgfqpoint{2.838856in}{3.359250in}}%
\pgfpathlineto{\pgfqpoint{2.849901in}{3.394733in}}%
\pgfpathlineto{\pgfqpoint{2.860988in}{3.423970in}}%
\pgfpathlineto{\pgfqpoint{2.872110in}{3.447488in}}%
\pgfpathlineto{\pgfqpoint{2.883260in}{3.465950in}}%
\pgfpathlineto{\pgfqpoint{2.877207in}{3.469862in}}%
\pgfpathlineto{\pgfqpoint{2.871162in}{3.473193in}}%
\pgfpathlineto{\pgfqpoint{2.865127in}{3.475742in}}%
\pgfpathlineto{\pgfqpoint{2.859101in}{3.477311in}}%
\pgfpathlineto{\pgfqpoint{2.853088in}{3.477697in}}%
\pgfpathlineto{\pgfqpoint{2.841945in}{3.463559in}}%
\pgfpathlineto{\pgfqpoint{2.830820in}{3.446396in}}%
\pgfpathlineto{\pgfqpoint{2.819717in}{3.425888in}}%
\pgfpathlineto{\pgfqpoint{2.808641in}{3.401776in}}%
\pgfpathlineto{\pgfqpoint{2.797590in}{3.374509in}}%
\pgfpathlineto{\pgfqpoint{2.803624in}{3.365409in}}%
\pgfpathlineto{\pgfqpoint{2.809671in}{3.354798in}}%
\pgfpathlineto{\pgfqpoint{2.815727in}{3.343146in}}%
\pgfpathlineto{\pgfqpoint{2.821789in}{3.330924in}}%
\pgfpathclose%
\pgfusepath{stroke,fill}%
\end{pgfscope}%
\begin{pgfscope}%
\pgfpathrectangle{\pgfqpoint{0.887500in}{0.275000in}}{\pgfqpoint{4.225000in}{4.225000in}}%
\pgfusepath{clip}%
\pgfsetbuttcap%
\pgfsetroundjoin%
\definecolor{currentfill}{rgb}{0.824940,0.884720,0.106217}%
\pgfsetfillcolor{currentfill}%
\pgfsetfillopacity{0.700000}%
\pgfsetlinewidth{0.501875pt}%
\definecolor{currentstroke}{rgb}{1.000000,1.000000,1.000000}%
\pgfsetstrokecolor{currentstroke}%
\pgfsetstrokeopacity{0.500000}%
\pgfsetdash{}{0pt}%
\pgfpathmoveto{\pgfqpoint{3.310421in}{3.529366in}}%
\pgfpathlineto{\pgfqpoint{3.321531in}{3.532401in}}%
\pgfpathlineto{\pgfqpoint{3.332633in}{3.535168in}}%
\pgfpathlineto{\pgfqpoint{3.343728in}{3.537749in}}%
\pgfpathlineto{\pgfqpoint{3.354818in}{3.540249in}}%
\pgfpathlineto{\pgfqpoint{3.365901in}{3.542775in}}%
\pgfpathlineto{\pgfqpoint{3.359664in}{3.548355in}}%
\pgfpathlineto{\pgfqpoint{3.353432in}{3.553937in}}%
\pgfpathlineto{\pgfqpoint{3.347205in}{3.559546in}}%
\pgfpathlineto{\pgfqpoint{3.340982in}{3.565162in}}%
\pgfpathlineto{\pgfqpoint{3.334764in}{3.570759in}}%
\pgfpathlineto{\pgfqpoint{3.323692in}{3.567890in}}%
\pgfpathlineto{\pgfqpoint{3.312614in}{3.565127in}}%
\pgfpathlineto{\pgfqpoint{3.301531in}{3.562382in}}%
\pgfpathlineto{\pgfqpoint{3.290442in}{3.559570in}}%
\pgfpathlineto{\pgfqpoint{3.279347in}{3.556621in}}%
\pgfpathlineto{\pgfqpoint{3.285554in}{3.551541in}}%
\pgfpathlineto{\pgfqpoint{3.291766in}{3.546283in}}%
\pgfpathlineto{\pgfqpoint{3.297980in}{3.540841in}}%
\pgfpathlineto{\pgfqpoint{3.304199in}{3.535207in}}%
\pgfpathclose%
\pgfusepath{stroke,fill}%
\end{pgfscope}%
\begin{pgfscope}%
\pgfpathrectangle{\pgfqpoint{0.887500in}{0.275000in}}{\pgfqpoint{4.225000in}{4.225000in}}%
\pgfusepath{clip}%
\pgfsetbuttcap%
\pgfsetroundjoin%
\definecolor{currentfill}{rgb}{0.311925,0.767822,0.415586}%
\pgfsetfillcolor{currentfill}%
\pgfsetfillopacity{0.700000}%
\pgfsetlinewidth{0.501875pt}%
\definecolor{currentstroke}{rgb}{1.000000,1.000000,1.000000}%
\pgfsetstrokecolor{currentstroke}%
\pgfsetstrokeopacity{0.500000}%
\pgfsetdash{}{0pt}%
\pgfpathmoveto{\pgfqpoint{2.833364in}{3.023849in}}%
\pgfpathlineto{\pgfqpoint{2.844384in}{3.060687in}}%
\pgfpathlineto{\pgfqpoint{2.855382in}{3.104684in}}%
\pgfpathlineto{\pgfqpoint{2.866375in}{3.153273in}}%
\pgfpathlineto{\pgfqpoint{2.877379in}{3.203872in}}%
\pgfpathlineto{\pgfqpoint{2.888407in}{3.253877in}}%
\pgfpathlineto{\pgfqpoint{2.882352in}{3.255759in}}%
\pgfpathlineto{\pgfqpoint{2.876300in}{3.258215in}}%
\pgfpathlineto{\pgfqpoint{2.870250in}{3.261434in}}%
\pgfpathlineto{\pgfqpoint{2.864201in}{3.265606in}}%
\pgfpathlineto{\pgfqpoint{2.858151in}{3.270921in}}%
\pgfpathlineto{\pgfqpoint{2.847183in}{3.219413in}}%
\pgfpathlineto{\pgfqpoint{2.836241in}{3.167362in}}%
\pgfpathlineto{\pgfqpoint{2.825307in}{3.117552in}}%
\pgfpathlineto{\pgfqpoint{2.814363in}{3.072747in}}%
\pgfpathlineto{\pgfqpoint{2.803388in}{3.035691in}}%
\pgfpathlineto{\pgfqpoint{2.809397in}{3.029220in}}%
\pgfpathlineto{\pgfqpoint{2.815396in}{3.025164in}}%
\pgfpathlineto{\pgfqpoint{2.821389in}{3.023158in}}%
\pgfpathlineto{\pgfqpoint{2.827377in}{3.022841in}}%
\pgfpathclose%
\pgfusepath{stroke,fill}%
\end{pgfscope}%
\begin{pgfscope}%
\pgfpathrectangle{\pgfqpoint{0.887500in}{0.275000in}}{\pgfqpoint{4.225000in}{4.225000in}}%
\pgfusepath{clip}%
\pgfsetbuttcap%
\pgfsetroundjoin%
\definecolor{currentfill}{rgb}{0.120638,0.625828,0.533488}%
\pgfsetfillcolor{currentfill}%
\pgfsetfillopacity{0.700000}%
\pgfsetlinewidth{0.501875pt}%
\definecolor{currentstroke}{rgb}{1.000000,1.000000,1.000000}%
\pgfsetstrokecolor{currentstroke}%
\pgfsetstrokeopacity{0.500000}%
\pgfsetdash{}{0pt}%
\pgfpathmoveto{\pgfqpoint{2.553999in}{2.756468in}}%
\pgfpathlineto{\pgfqpoint{2.565154in}{2.769896in}}%
\pgfpathlineto{\pgfqpoint{2.576310in}{2.783394in}}%
\pgfpathlineto{\pgfqpoint{2.587468in}{2.796646in}}%
\pgfpathlineto{\pgfqpoint{2.598633in}{2.809336in}}%
\pgfpathlineto{\pgfqpoint{2.609807in}{2.821192in}}%
\pgfpathlineto{\pgfqpoint{2.603789in}{2.833274in}}%
\pgfpathlineto{\pgfqpoint{2.597775in}{2.845171in}}%
\pgfpathlineto{\pgfqpoint{2.591767in}{2.856718in}}%
\pgfpathlineto{\pgfqpoint{2.585769in}{2.867746in}}%
\pgfpathlineto{\pgfqpoint{2.579782in}{2.878088in}}%
\pgfpathlineto{\pgfqpoint{2.568599in}{2.868009in}}%
\pgfpathlineto{\pgfqpoint{2.557430in}{2.856671in}}%
\pgfpathlineto{\pgfqpoint{2.546273in}{2.844376in}}%
\pgfpathlineto{\pgfqpoint{2.535123in}{2.831480in}}%
\pgfpathlineto{\pgfqpoint{2.523977in}{2.818337in}}%
\pgfpathlineto{\pgfqpoint{2.529970in}{2.806423in}}%
\pgfpathlineto{\pgfqpoint{2.535971in}{2.794131in}}%
\pgfpathlineto{\pgfqpoint{2.541977in}{2.781609in}}%
\pgfpathlineto{\pgfqpoint{2.547987in}{2.769005in}}%
\pgfpathclose%
\pgfusepath{stroke,fill}%
\end{pgfscope}%
\begin{pgfscope}%
\pgfpathrectangle{\pgfqpoint{0.887500in}{0.275000in}}{\pgfqpoint{4.225000in}{4.225000in}}%
\pgfusepath{clip}%
\pgfsetbuttcap%
\pgfsetroundjoin%
\definecolor{currentfill}{rgb}{0.120092,0.600104,0.542530}%
\pgfsetfillcolor{currentfill}%
\pgfsetfillopacity{0.700000}%
\pgfsetlinewidth{0.501875pt}%
\definecolor{currentstroke}{rgb}{1.000000,1.000000,1.000000}%
\pgfsetstrokecolor{currentstroke}%
\pgfsetstrokeopacity{0.500000}%
\pgfsetdash{}{0pt}%
\pgfpathmoveto{\pgfqpoint{1.897540in}{2.744210in}}%
\pgfpathlineto{\pgfqpoint{1.908988in}{2.747570in}}%
\pgfpathlineto{\pgfqpoint{1.920430in}{2.750938in}}%
\pgfpathlineto{\pgfqpoint{1.931866in}{2.754311in}}%
\pgfpathlineto{\pgfqpoint{1.943297in}{2.757682in}}%
\pgfpathlineto{\pgfqpoint{1.954723in}{2.761046in}}%
\pgfpathlineto{\pgfqpoint{1.948975in}{2.768826in}}%
\pgfpathlineto{\pgfqpoint{1.943231in}{2.776595in}}%
\pgfpathlineto{\pgfqpoint{1.937491in}{2.784351in}}%
\pgfpathlineto{\pgfqpoint{1.931756in}{2.792096in}}%
\pgfpathlineto{\pgfqpoint{1.926024in}{2.799829in}}%
\pgfpathlineto{\pgfqpoint{1.914612in}{2.796439in}}%
\pgfpathlineto{\pgfqpoint{1.903195in}{2.793043in}}%
\pgfpathlineto{\pgfqpoint{1.891772in}{2.789648in}}%
\pgfpathlineto{\pgfqpoint{1.880343in}{2.786260in}}%
\pgfpathlineto{\pgfqpoint{1.868909in}{2.782882in}}%
\pgfpathlineto{\pgfqpoint{1.874627in}{2.775172in}}%
\pgfpathlineto{\pgfqpoint{1.880349in}{2.767450in}}%
\pgfpathlineto{\pgfqpoint{1.886076in}{2.759716in}}%
\pgfpathlineto{\pgfqpoint{1.891806in}{2.751969in}}%
\pgfpathclose%
\pgfusepath{stroke,fill}%
\end{pgfscope}%
\begin{pgfscope}%
\pgfpathrectangle{\pgfqpoint{0.887500in}{0.275000in}}{\pgfqpoint{4.225000in}{4.225000in}}%
\pgfusepath{clip}%
\pgfsetbuttcap%
\pgfsetroundjoin%
\definecolor{currentfill}{rgb}{0.120565,0.596422,0.543611}%
\pgfsetfillcolor{currentfill}%
\pgfsetfillopacity{0.700000}%
\pgfsetlinewidth{0.501875pt}%
\definecolor{currentstroke}{rgb}{1.000000,1.000000,1.000000}%
\pgfsetstrokecolor{currentstroke}%
\pgfsetstrokeopacity{0.500000}%
\pgfsetdash{}{0pt}%
\pgfpathmoveto{\pgfqpoint{2.498067in}{2.700531in}}%
\pgfpathlineto{\pgfqpoint{2.509284in}{2.709611in}}%
\pgfpathlineto{\pgfqpoint{2.520485in}{2.719767in}}%
\pgfpathlineto{\pgfqpoint{2.531669in}{2.731090in}}%
\pgfpathlineto{\pgfqpoint{2.542838in}{2.743428in}}%
\pgfpathlineto{\pgfqpoint{2.553999in}{2.756468in}}%
\pgfpathlineto{\pgfqpoint{2.547987in}{2.769005in}}%
\pgfpathlineto{\pgfqpoint{2.541977in}{2.781609in}}%
\pgfpathlineto{\pgfqpoint{2.535971in}{2.794131in}}%
\pgfpathlineto{\pgfqpoint{2.529970in}{2.806423in}}%
\pgfpathlineto{\pgfqpoint{2.523977in}{2.818337in}}%
\pgfpathlineto{\pgfqpoint{2.512829in}{2.805304in}}%
\pgfpathlineto{\pgfqpoint{2.501674in}{2.792734in}}%
\pgfpathlineto{\pgfqpoint{2.490508in}{2.780979in}}%
\pgfpathlineto{\pgfqpoint{2.479328in}{2.770210in}}%
\pgfpathlineto{\pgfqpoint{2.468132in}{2.760330in}}%
\pgfpathlineto{\pgfqpoint{2.474115in}{2.748324in}}%
\pgfpathlineto{\pgfqpoint{2.480100in}{2.736256in}}%
\pgfpathlineto{\pgfqpoint{2.486089in}{2.724211in}}%
\pgfpathlineto{\pgfqpoint{2.492078in}{2.712274in}}%
\pgfpathclose%
\pgfusepath{stroke,fill}%
\end{pgfscope}%
\begin{pgfscope}%
\pgfpathrectangle{\pgfqpoint{0.887500in}{0.275000in}}{\pgfqpoint{4.225000in}{4.225000in}}%
\pgfusepath{clip}%
\pgfsetbuttcap%
\pgfsetroundjoin%
\definecolor{currentfill}{rgb}{0.814576,0.883393,0.110347}%
\pgfsetfillcolor{currentfill}%
\pgfsetfillopacity{0.700000}%
\pgfsetlinewidth{0.501875pt}%
\definecolor{currentstroke}{rgb}{1.000000,1.000000,1.000000}%
\pgfsetstrokecolor{currentstroke}%
\pgfsetstrokeopacity{0.500000}%
\pgfsetdash{}{0pt}%
\pgfpathmoveto{\pgfqpoint{3.539305in}{3.509958in}}%
\pgfpathlineto{\pgfqpoint{3.550364in}{3.513493in}}%
\pgfpathlineto{\pgfqpoint{3.561419in}{3.516985in}}%
\pgfpathlineto{\pgfqpoint{3.572467in}{3.520412in}}%
\pgfpathlineto{\pgfqpoint{3.583509in}{3.523752in}}%
\pgfpathlineto{\pgfqpoint{3.594544in}{3.526985in}}%
\pgfpathlineto{\pgfqpoint{3.588271in}{3.536532in}}%
\pgfpathlineto{\pgfqpoint{3.581994in}{3.545546in}}%
\pgfpathlineto{\pgfqpoint{3.575714in}{3.554057in}}%
\pgfpathlineto{\pgfqpoint{3.569433in}{3.562133in}}%
\pgfpathlineto{\pgfqpoint{3.563152in}{3.569840in}}%
\pgfpathlineto{\pgfqpoint{3.552119in}{3.566009in}}%
\pgfpathlineto{\pgfqpoint{3.541080in}{3.562109in}}%
\pgfpathlineto{\pgfqpoint{3.530035in}{3.558149in}}%
\pgfpathlineto{\pgfqpoint{3.518985in}{3.554141in}}%
\pgfpathlineto{\pgfqpoint{3.507930in}{3.550093in}}%
\pgfpathlineto{\pgfqpoint{3.514208in}{3.543049in}}%
\pgfpathlineto{\pgfqpoint{3.520485in}{3.535590in}}%
\pgfpathlineto{\pgfqpoint{3.526761in}{3.527636in}}%
\pgfpathlineto{\pgfqpoint{3.533035in}{3.519105in}}%
\pgfpathclose%
\pgfusepath{stroke,fill}%
\end{pgfscope}%
\begin{pgfscope}%
\pgfpathrectangle{\pgfqpoint{0.887500in}{0.275000in}}{\pgfqpoint{4.225000in}{4.225000in}}%
\pgfusepath{clip}%
\pgfsetbuttcap%
\pgfsetroundjoin%
\definecolor{currentfill}{rgb}{0.119483,0.614817,0.537692}%
\pgfsetfillcolor{currentfill}%
\pgfsetfillopacity{0.700000}%
\pgfsetlinewidth{0.501875pt}%
\definecolor{currentstroke}{rgb}{1.000000,1.000000,1.000000}%
\pgfsetstrokecolor{currentstroke}%
\pgfsetstrokeopacity{0.500000}%
\pgfsetdash{}{0pt}%
\pgfpathmoveto{\pgfqpoint{1.668369in}{2.771479in}}%
\pgfpathlineto{\pgfqpoint{1.679877in}{2.774760in}}%
\pgfpathlineto{\pgfqpoint{1.691379in}{2.778041in}}%
\pgfpathlineto{\pgfqpoint{1.702875in}{2.781322in}}%
\pgfpathlineto{\pgfqpoint{1.714366in}{2.784606in}}%
\pgfpathlineto{\pgfqpoint{1.725851in}{2.787892in}}%
\pgfpathlineto{\pgfqpoint{1.720185in}{2.795483in}}%
\pgfpathlineto{\pgfqpoint{1.714523in}{2.803059in}}%
\pgfpathlineto{\pgfqpoint{1.708866in}{2.810620in}}%
\pgfpathlineto{\pgfqpoint{1.703212in}{2.818168in}}%
\pgfpathlineto{\pgfqpoint{1.697563in}{2.825701in}}%
\pgfpathlineto{\pgfqpoint{1.686092in}{2.822393in}}%
\pgfpathlineto{\pgfqpoint{1.674615in}{2.819090in}}%
\pgfpathlineto{\pgfqpoint{1.663133in}{2.815789in}}%
\pgfpathlineto{\pgfqpoint{1.651645in}{2.812490in}}%
\pgfpathlineto{\pgfqpoint{1.640152in}{2.809191in}}%
\pgfpathlineto{\pgfqpoint{1.645787in}{2.801681in}}%
\pgfpathlineto{\pgfqpoint{1.651426in}{2.794155in}}%
\pgfpathlineto{\pgfqpoint{1.657069in}{2.786612in}}%
\pgfpathlineto{\pgfqpoint{1.662717in}{2.779054in}}%
\pgfpathclose%
\pgfusepath{stroke,fill}%
\end{pgfscope}%
\begin{pgfscope}%
\pgfpathrectangle{\pgfqpoint{0.887500in}{0.275000in}}{\pgfqpoint{4.225000in}{4.225000in}}%
\pgfusepath{clip}%
\pgfsetbuttcap%
\pgfsetroundjoin%
\definecolor{currentfill}{rgb}{0.124395,0.578002,0.548287}%
\pgfsetfillcolor{currentfill}%
\pgfsetfillopacity{0.700000}%
\pgfsetlinewidth{0.501875pt}%
\definecolor{currentstroke}{rgb}{1.000000,1.000000,1.000000}%
\pgfsetstrokecolor{currentstroke}%
\pgfsetstrokeopacity{0.500000}%
\pgfsetdash{}{0pt}%
\pgfpathmoveto{\pgfqpoint{2.212718in}{2.693338in}}%
\pgfpathlineto{\pgfqpoint{2.224096in}{2.696577in}}%
\pgfpathlineto{\pgfqpoint{2.235473in}{2.699589in}}%
\pgfpathlineto{\pgfqpoint{2.246848in}{2.702397in}}%
\pgfpathlineto{\pgfqpoint{2.258219in}{2.705170in}}%
\pgfpathlineto{\pgfqpoint{2.269580in}{2.708097in}}%
\pgfpathlineto{\pgfqpoint{2.263722in}{2.716047in}}%
\pgfpathlineto{\pgfqpoint{2.257869in}{2.723980in}}%
\pgfpathlineto{\pgfqpoint{2.252019in}{2.731888in}}%
\pgfpathlineto{\pgfqpoint{2.246174in}{2.739763in}}%
\pgfpathlineto{\pgfqpoint{2.240334in}{2.747596in}}%
\pgfpathlineto{\pgfqpoint{2.228981in}{2.744837in}}%
\pgfpathlineto{\pgfqpoint{2.217618in}{2.742286in}}%
\pgfpathlineto{\pgfqpoint{2.206249in}{2.739690in}}%
\pgfpathlineto{\pgfqpoint{2.194881in}{2.736819in}}%
\pgfpathlineto{\pgfqpoint{2.183514in}{2.733642in}}%
\pgfpathlineto{\pgfqpoint{2.189347in}{2.725622in}}%
\pgfpathlineto{\pgfqpoint{2.195183in}{2.717580in}}%
\pgfpathlineto{\pgfqpoint{2.201024in}{2.709518in}}%
\pgfpathlineto{\pgfqpoint{2.206869in}{2.701437in}}%
\pgfpathclose%
\pgfusepath{stroke,fill}%
\end{pgfscope}%
\begin{pgfscope}%
\pgfpathrectangle{\pgfqpoint{0.887500in}{0.275000in}}{\pgfqpoint{4.225000in}{4.225000in}}%
\pgfusepath{clip}%
\pgfsetbuttcap%
\pgfsetroundjoin%
\definecolor{currentfill}{rgb}{0.783315,0.879285,0.125405}%
\pgfsetfillcolor{currentfill}%
\pgfsetfillopacity{0.700000}%
\pgfsetlinewidth{0.501875pt}%
\definecolor{currentstroke}{rgb}{1.000000,1.000000,1.000000}%
\pgfsetstrokecolor{currentstroke}%
\pgfsetstrokeopacity{0.500000}%
\pgfsetdash{}{0pt}%
\pgfpathmoveto{\pgfqpoint{3.025649in}{3.499522in}}%
\pgfpathlineto{\pgfqpoint{3.036832in}{3.502550in}}%
\pgfpathlineto{\pgfqpoint{3.048010in}{3.506674in}}%
\pgfpathlineto{\pgfqpoint{3.059182in}{3.511624in}}%
\pgfpathlineto{\pgfqpoint{3.070351in}{3.517122in}}%
\pgfpathlineto{\pgfqpoint{3.081515in}{3.522890in}}%
\pgfpathlineto{\pgfqpoint{3.075382in}{3.525797in}}%
\pgfpathlineto{\pgfqpoint{3.069254in}{3.528429in}}%
\pgfpathlineto{\pgfqpoint{3.063132in}{3.530797in}}%
\pgfpathlineto{\pgfqpoint{3.057016in}{3.532916in}}%
\pgfpathlineto{\pgfqpoint{3.050906in}{3.534804in}}%
\pgfpathlineto{\pgfqpoint{3.039762in}{3.529110in}}%
\pgfpathlineto{\pgfqpoint{3.028613in}{3.523759in}}%
\pgfpathlineto{\pgfqpoint{3.017460in}{3.519018in}}%
\pgfpathlineto{\pgfqpoint{3.006302in}{3.515155in}}%
\pgfpathlineto{\pgfqpoint{2.995137in}{3.512429in}}%
\pgfpathlineto{\pgfqpoint{3.001228in}{3.509685in}}%
\pgfpathlineto{\pgfqpoint{3.007325in}{3.506975in}}%
\pgfpathlineto{\pgfqpoint{3.013428in}{3.504367in}}%
\pgfpathlineto{\pgfqpoint{3.019536in}{3.501901in}}%
\pgfpathclose%
\pgfusepath{stroke,fill}%
\end{pgfscope}%
\begin{pgfscope}%
\pgfpathrectangle{\pgfqpoint{0.887500in}{0.275000in}}{\pgfqpoint{4.225000in}{4.225000in}}%
\pgfusepath{clip}%
\pgfsetbuttcap%
\pgfsetroundjoin%
\definecolor{currentfill}{rgb}{0.751884,0.874951,0.143228}%
\pgfsetfillcolor{currentfill}%
\pgfsetfillopacity{0.700000}%
\pgfsetlinewidth{0.501875pt}%
\definecolor{currentstroke}{rgb}{1.000000,1.000000,1.000000}%
\pgfsetstrokecolor{currentstroke}%
\pgfsetstrokeopacity{0.500000}%
\pgfsetdash{}{0pt}%
\pgfpathmoveto{\pgfqpoint{2.883260in}{3.465950in}}%
\pgfpathlineto{\pgfqpoint{2.894430in}{3.480030in}}%
\pgfpathlineto{\pgfqpoint{2.905615in}{3.490400in}}%
\pgfpathlineto{\pgfqpoint{2.916809in}{3.497740in}}%
\pgfpathlineto{\pgfqpoint{2.928007in}{3.502726in}}%
\pgfpathlineto{\pgfqpoint{2.939207in}{3.505941in}}%
\pgfpathlineto{\pgfqpoint{2.933137in}{3.508659in}}%
\pgfpathlineto{\pgfqpoint{2.927075in}{3.510985in}}%
\pgfpathlineto{\pgfqpoint{2.921020in}{3.512790in}}%
\pgfpathlineto{\pgfqpoint{2.914973in}{3.513947in}}%
\pgfpathlineto{\pgfqpoint{2.908935in}{3.514328in}}%
\pgfpathlineto{\pgfqpoint{2.897758in}{3.510459in}}%
\pgfpathlineto{\pgfqpoint{2.886582in}{3.505187in}}%
\pgfpathlineto{\pgfqpoint{2.875409in}{3.498187in}}%
\pgfpathlineto{\pgfqpoint{2.864244in}{3.489132in}}%
\pgfpathlineto{\pgfqpoint{2.853088in}{3.477697in}}%
\pgfpathlineto{\pgfqpoint{2.859101in}{3.477311in}}%
\pgfpathlineto{\pgfqpoint{2.865127in}{3.475742in}}%
\pgfpathlineto{\pgfqpoint{2.871162in}{3.473193in}}%
\pgfpathlineto{\pgfqpoint{2.877207in}{3.469862in}}%
\pgfpathclose%
\pgfusepath{stroke,fill}%
\end{pgfscope}%
\begin{pgfscope}%
\pgfpathrectangle{\pgfqpoint{0.887500in}{0.275000in}}{\pgfqpoint{4.225000in}{4.225000in}}%
\pgfusepath{clip}%
\pgfsetbuttcap%
\pgfsetroundjoin%
\definecolor{currentfill}{rgb}{0.180653,0.701402,0.488189}%
\pgfsetfillcolor{currentfill}%
\pgfsetfillopacity{0.700000}%
\pgfsetlinewidth{0.501875pt}%
\definecolor{currentstroke}{rgb}{1.000000,1.000000,1.000000}%
\pgfsetstrokecolor{currentstroke}%
\pgfsetstrokeopacity{0.500000}%
\pgfsetdash{}{0pt}%
\pgfpathmoveto{\pgfqpoint{2.807638in}{2.943467in}}%
\pgfpathlineto{\pgfqpoint{2.818823in}{2.954354in}}%
\pgfpathlineto{\pgfqpoint{2.829991in}{2.967754in}}%
\pgfpathlineto{\pgfqpoint{2.841140in}{2.984987in}}%
\pgfpathlineto{\pgfqpoint{2.852264in}{3.007383in}}%
\pgfpathlineto{\pgfqpoint{2.863363in}{3.036269in}}%
\pgfpathlineto{\pgfqpoint{2.857348in}{3.033857in}}%
\pgfpathlineto{\pgfqpoint{2.851343in}{3.031180in}}%
\pgfpathlineto{\pgfqpoint{2.845344in}{3.028383in}}%
\pgfpathlineto{\pgfqpoint{2.839352in}{3.025818in}}%
\pgfpathlineto{\pgfqpoint{2.833364in}{3.023849in}}%
\pgfpathlineto{\pgfqpoint{2.822302in}{2.996355in}}%
\pgfpathlineto{\pgfqpoint{2.811195in}{2.977599in}}%
\pgfpathlineto{\pgfqpoint{2.800048in}{2.965707in}}%
\pgfpathlineto{\pgfqpoint{2.788866in}{2.958813in}}%
\pgfpathlineto{\pgfqpoint{2.777659in}{2.955053in}}%
\pgfpathlineto{\pgfqpoint{2.783642in}{2.953020in}}%
\pgfpathlineto{\pgfqpoint{2.789629in}{2.951068in}}%
\pgfpathlineto{\pgfqpoint{2.795624in}{2.948950in}}%
\pgfpathlineto{\pgfqpoint{2.801627in}{2.946435in}}%
\pgfpathclose%
\pgfusepath{stroke,fill}%
\end{pgfscope}%
\begin{pgfscope}%
\pgfpathrectangle{\pgfqpoint{0.887500in}{0.275000in}}{\pgfqpoint{4.225000in}{4.225000in}}%
\pgfusepath{clip}%
\pgfsetbuttcap%
\pgfsetroundjoin%
\definecolor{currentfill}{rgb}{0.824940,0.884720,0.106217}%
\pgfsetfillcolor{currentfill}%
\pgfsetfillopacity{0.700000}%
\pgfsetlinewidth{0.501875pt}%
\definecolor{currentstroke}{rgb}{1.000000,1.000000,1.000000}%
\pgfsetstrokecolor{currentstroke}%
\pgfsetstrokeopacity{0.500000}%
\pgfsetdash{}{0pt}%
\pgfpathmoveto{\pgfqpoint{3.168089in}{3.523735in}}%
\pgfpathlineto{\pgfqpoint{3.179239in}{3.527190in}}%
\pgfpathlineto{\pgfqpoint{3.190383in}{3.530551in}}%
\pgfpathlineto{\pgfqpoint{3.201522in}{3.533859in}}%
\pgfpathlineto{\pgfqpoint{3.212656in}{3.537154in}}%
\pgfpathlineto{\pgfqpoint{3.223785in}{3.540470in}}%
\pgfpathlineto{\pgfqpoint{3.217597in}{3.545599in}}%
\pgfpathlineto{\pgfqpoint{3.211414in}{3.550477in}}%
\pgfpathlineto{\pgfqpoint{3.205234in}{3.555093in}}%
\pgfpathlineto{\pgfqpoint{3.199059in}{3.559436in}}%
\pgfpathlineto{\pgfqpoint{3.192888in}{3.563497in}}%
\pgfpathlineto{\pgfqpoint{3.181775in}{3.560426in}}%
\pgfpathlineto{\pgfqpoint{3.170657in}{3.557370in}}%
\pgfpathlineto{\pgfqpoint{3.159534in}{3.554241in}}%
\pgfpathlineto{\pgfqpoint{3.148405in}{3.550943in}}%
\pgfpathlineto{\pgfqpoint{3.137270in}{3.547379in}}%
\pgfpathlineto{\pgfqpoint{3.143425in}{3.543408in}}%
\pgfpathlineto{\pgfqpoint{3.149585in}{3.539027in}}%
\pgfpathlineto{\pgfqpoint{3.155749in}{3.534267in}}%
\pgfpathlineto{\pgfqpoint{3.161917in}{3.529159in}}%
\pgfpathclose%
\pgfusepath{stroke,fill}%
\end{pgfscope}%
\begin{pgfscope}%
\pgfpathrectangle{\pgfqpoint{0.887500in}{0.275000in}}{\pgfqpoint{4.225000in}{4.225000in}}%
\pgfusepath{clip}%
\pgfsetbuttcap%
\pgfsetroundjoin%
\definecolor{currentfill}{rgb}{0.772852,0.877868,0.131109}%
\pgfsetfillcolor{currentfill}%
\pgfsetfillopacity{0.700000}%
\pgfsetlinewidth{0.501875pt}%
\definecolor{currentstroke}{rgb}{1.000000,1.000000,1.000000}%
\pgfsetstrokecolor{currentstroke}%
\pgfsetstrokeopacity{0.500000}%
\pgfsetdash{}{0pt}%
\pgfpathmoveto{\pgfqpoint{3.625887in}{3.473511in}}%
\pgfpathlineto{\pgfqpoint{3.636920in}{3.476468in}}%
\pgfpathlineto{\pgfqpoint{3.647946in}{3.479298in}}%
\pgfpathlineto{\pgfqpoint{3.658964in}{3.482026in}}%
\pgfpathlineto{\pgfqpoint{3.669977in}{3.484732in}}%
\pgfpathlineto{\pgfqpoint{3.680985in}{3.487494in}}%
\pgfpathlineto{\pgfqpoint{3.674711in}{3.498646in}}%
\pgfpathlineto{\pgfqpoint{3.668438in}{3.509679in}}%
\pgfpathlineto{\pgfqpoint{3.662165in}{3.520540in}}%
\pgfpathlineto{\pgfqpoint{3.655893in}{3.531174in}}%
\pgfpathlineto{\pgfqpoint{3.649619in}{3.541528in}}%
\pgfpathlineto{\pgfqpoint{3.638616in}{3.538681in}}%
\pgfpathlineto{\pgfqpoint{3.627608in}{3.535882in}}%
\pgfpathlineto{\pgfqpoint{3.616594in}{3.533045in}}%
\pgfpathlineto{\pgfqpoint{3.605573in}{3.530089in}}%
\pgfpathlineto{\pgfqpoint{3.594544in}{3.526985in}}%
\pgfpathlineto{\pgfqpoint{3.600815in}{3.516960in}}%
\pgfpathlineto{\pgfqpoint{3.607084in}{3.506530in}}%
\pgfpathlineto{\pgfqpoint{3.613352in}{3.495764in}}%
\pgfpathlineto{\pgfqpoint{3.619619in}{3.484734in}}%
\pgfpathclose%
\pgfusepath{stroke,fill}%
\end{pgfscope}%
\begin{pgfscope}%
\pgfpathrectangle{\pgfqpoint{0.887500in}{0.275000in}}{\pgfqpoint{4.225000in}{4.225000in}}%
\pgfusepath{clip}%
\pgfsetbuttcap%
\pgfsetroundjoin%
\definecolor{currentfill}{rgb}{0.720391,0.870350,0.162603}%
\pgfsetfillcolor{currentfill}%
\pgfsetfillopacity{0.700000}%
\pgfsetlinewidth{0.501875pt}%
\definecolor{currentstroke}{rgb}{1.000000,1.000000,1.000000}%
\pgfsetstrokecolor{currentstroke}%
\pgfsetstrokeopacity{0.500000}%
\pgfsetdash{}{0pt}%
\pgfpathmoveto{\pgfqpoint{3.712408in}{3.431844in}}%
\pgfpathlineto{\pgfqpoint{3.723421in}{3.434902in}}%
\pgfpathlineto{\pgfqpoint{3.734431in}{3.438107in}}%
\pgfpathlineto{\pgfqpoint{3.745440in}{3.441510in}}%
\pgfpathlineto{\pgfqpoint{3.756447in}{3.445161in}}%
\pgfpathlineto{\pgfqpoint{3.767454in}{3.449056in}}%
\pgfpathlineto{\pgfqpoint{3.761156in}{3.460114in}}%
\pgfpathlineto{\pgfqpoint{3.754862in}{3.471258in}}%
\pgfpathlineto{\pgfqpoint{3.748573in}{3.482454in}}%
\pgfpathlineto{\pgfqpoint{3.742287in}{3.493669in}}%
\pgfpathlineto{\pgfqpoint{3.736004in}{3.504866in}}%
\pgfpathlineto{\pgfqpoint{3.724999in}{3.500713in}}%
\pgfpathlineto{\pgfqpoint{3.713996in}{3.496925in}}%
\pgfpathlineto{\pgfqpoint{3.702993in}{3.493510in}}%
\pgfpathlineto{\pgfqpoint{3.691990in}{3.490393in}}%
\pgfpathlineto{\pgfqpoint{3.680985in}{3.487494in}}%
\pgfpathlineto{\pgfqpoint{3.687262in}{3.476277in}}%
\pgfpathlineto{\pgfqpoint{3.693542in}{3.465051in}}%
\pgfpathlineto{\pgfqpoint{3.699826in}{3.453868in}}%
\pgfpathlineto{\pgfqpoint{3.706114in}{3.442782in}}%
\pgfpathclose%
\pgfusepath{stroke,fill}%
\end{pgfscope}%
\begin{pgfscope}%
\pgfpathrectangle{\pgfqpoint{0.887500in}{0.275000in}}{\pgfqpoint{4.225000in}{4.225000in}}%
\pgfusepath{clip}%
\pgfsetbuttcap%
\pgfsetroundjoin%
\definecolor{currentfill}{rgb}{0.125394,0.574318,0.549086}%
\pgfsetfillcolor{currentfill}%
\pgfsetfillopacity{0.700000}%
\pgfsetlinewidth{0.501875pt}%
\definecolor{currentstroke}{rgb}{1.000000,1.000000,1.000000}%
\pgfsetstrokecolor{currentstroke}%
\pgfsetstrokeopacity{0.500000}%
\pgfsetdash{}{0pt}%
\pgfpathmoveto{\pgfqpoint{2.441759in}{2.667420in}}%
\pgfpathlineto{\pgfqpoint{2.453047in}{2.672769in}}%
\pgfpathlineto{\pgfqpoint{2.464323in}{2.678645in}}%
\pgfpathlineto{\pgfqpoint{2.475586in}{2.685157in}}%
\pgfpathlineto{\pgfqpoint{2.486834in}{2.692416in}}%
\pgfpathlineto{\pgfqpoint{2.498067in}{2.700531in}}%
\pgfpathlineto{\pgfqpoint{2.492078in}{2.712274in}}%
\pgfpathlineto{\pgfqpoint{2.486089in}{2.724211in}}%
\pgfpathlineto{\pgfqpoint{2.480100in}{2.736256in}}%
\pgfpathlineto{\pgfqpoint{2.474115in}{2.748324in}}%
\pgfpathlineto{\pgfqpoint{2.468132in}{2.760330in}}%
\pgfpathlineto{\pgfqpoint{2.456924in}{2.751218in}}%
\pgfpathlineto{\pgfqpoint{2.445703in}{2.742757in}}%
\pgfpathlineto{\pgfqpoint{2.434472in}{2.734825in}}%
\pgfpathlineto{\pgfqpoint{2.423231in}{2.727305in}}%
\pgfpathlineto{\pgfqpoint{2.411982in}{2.720080in}}%
\pgfpathlineto{\pgfqpoint{2.417933in}{2.709426in}}%
\pgfpathlineto{\pgfqpoint{2.423887in}{2.698788in}}%
\pgfpathlineto{\pgfqpoint{2.429843in}{2.688211in}}%
\pgfpathlineto{\pgfqpoint{2.435800in}{2.677740in}}%
\pgfpathclose%
\pgfusepath{stroke,fill}%
\end{pgfscope}%
\begin{pgfscope}%
\pgfpathrectangle{\pgfqpoint{0.887500in}{0.275000in}}{\pgfqpoint{4.225000in}{4.225000in}}%
\pgfusepath{clip}%
\pgfsetbuttcap%
\pgfsetroundjoin%
\definecolor{currentfill}{rgb}{0.121148,0.592739,0.544641}%
\pgfsetfillcolor{currentfill}%
\pgfsetfillopacity{0.700000}%
\pgfsetlinewidth{0.501875pt}%
\definecolor{currentstroke}{rgb}{1.000000,1.000000,1.000000}%
\pgfsetstrokecolor{currentstroke}%
\pgfsetstrokeopacity{0.500000}%
\pgfsetdash{}{0pt}%
\pgfpathmoveto{\pgfqpoint{1.983524in}{2.721931in}}%
\pgfpathlineto{\pgfqpoint{1.994958in}{2.725255in}}%
\pgfpathlineto{\pgfqpoint{2.006386in}{2.728559in}}%
\pgfpathlineto{\pgfqpoint{2.017810in}{2.731837in}}%
\pgfpathlineto{\pgfqpoint{2.029230in}{2.735090in}}%
\pgfpathlineto{\pgfqpoint{2.040643in}{2.738335in}}%
\pgfpathlineto{\pgfqpoint{2.034862in}{2.746209in}}%
\pgfpathlineto{\pgfqpoint{2.029084in}{2.754071in}}%
\pgfpathlineto{\pgfqpoint{2.023311in}{2.761919in}}%
\pgfpathlineto{\pgfqpoint{2.017541in}{2.769754in}}%
\pgfpathlineto{\pgfqpoint{2.011776in}{2.777577in}}%
\pgfpathlineto{\pgfqpoint{2.000375in}{2.774307in}}%
\pgfpathlineto{\pgfqpoint{1.988969in}{2.771030in}}%
\pgfpathlineto{\pgfqpoint{1.977559in}{2.767725in}}%
\pgfpathlineto{\pgfqpoint{1.966143in}{2.764395in}}%
\pgfpathlineto{\pgfqpoint{1.954723in}{2.761046in}}%
\pgfpathlineto{\pgfqpoint{1.960475in}{2.753252in}}%
\pgfpathlineto{\pgfqpoint{1.966231in}{2.745444in}}%
\pgfpathlineto{\pgfqpoint{1.971991in}{2.737621in}}%
\pgfpathlineto{\pgfqpoint{1.977755in}{2.729784in}}%
\pgfpathclose%
\pgfusepath{stroke,fill}%
\end{pgfscope}%
\begin{pgfscope}%
\pgfpathrectangle{\pgfqpoint{0.887500in}{0.275000in}}{\pgfqpoint{4.225000in}{4.225000in}}%
\pgfusepath{clip}%
\pgfsetbuttcap%
\pgfsetroundjoin%
\definecolor{currentfill}{rgb}{0.121380,0.629492,0.531973}%
\pgfsetfillcolor{currentfill}%
\pgfsetfillopacity{0.700000}%
\pgfsetlinewidth{0.501875pt}%
\definecolor{currentstroke}{rgb}{1.000000,1.000000,1.000000}%
\pgfsetstrokecolor{currentstroke}%
\pgfsetstrokeopacity{0.500000}%
\pgfsetdash{}{0pt}%
\pgfpathmoveto{\pgfqpoint{1.439193in}{2.796477in}}%
\pgfpathlineto{\pgfqpoint{1.450755in}{2.799821in}}%
\pgfpathlineto{\pgfqpoint{1.462311in}{2.803162in}}%
\pgfpathlineto{\pgfqpoint{1.473862in}{2.806500in}}%
\pgfpathlineto{\pgfqpoint{1.485408in}{2.809838in}}%
\pgfpathlineto{\pgfqpoint{1.496947in}{2.813175in}}%
\pgfpathlineto{\pgfqpoint{1.491367in}{2.820541in}}%
\pgfpathlineto{\pgfqpoint{1.485791in}{2.827888in}}%
\pgfpathlineto{\pgfqpoint{1.480220in}{2.835217in}}%
\pgfpathlineto{\pgfqpoint{1.474653in}{2.842526in}}%
\pgfpathlineto{\pgfqpoint{1.463125in}{2.839182in}}%
\pgfpathlineto{\pgfqpoint{1.451591in}{2.835834in}}%
\pgfpathlineto{\pgfqpoint{1.440052in}{2.832478in}}%
\pgfpathlineto{\pgfqpoint{1.428507in}{2.829117in}}%
\pgfpathlineto{\pgfqpoint{1.416958in}{2.825749in}}%
\pgfpathlineto{\pgfqpoint{1.422510in}{2.818460in}}%
\pgfpathlineto{\pgfqpoint{1.428067in}{2.811152in}}%
\pgfpathlineto{\pgfqpoint{1.433628in}{2.803824in}}%
\pgfpathclose%
\pgfusepath{stroke,fill}%
\end{pgfscope}%
\begin{pgfscope}%
\pgfpathrectangle{\pgfqpoint{0.887500in}{0.275000in}}{\pgfqpoint{4.225000in}{4.225000in}}%
\pgfusepath{clip}%
\pgfsetbuttcap%
\pgfsetroundjoin%
\definecolor{currentfill}{rgb}{0.153894,0.680203,0.504172}%
\pgfsetfillcolor{currentfill}%
\pgfsetfillopacity{0.700000}%
\pgfsetlinewidth{0.501875pt}%
\definecolor{currentstroke}{rgb}{1.000000,1.000000,1.000000}%
\pgfsetstrokecolor{currentstroke}%
\pgfsetstrokeopacity{0.500000}%
\pgfsetdash{}{0pt}%
\pgfpathmoveto{\pgfqpoint{2.751712in}{2.887541in}}%
\pgfpathlineto{\pgfqpoint{2.762885in}{2.900658in}}%
\pgfpathlineto{\pgfqpoint{2.774064in}{2.912911in}}%
\pgfpathlineto{\pgfqpoint{2.785251in}{2.923951in}}%
\pgfpathlineto{\pgfqpoint{2.796445in}{2.933772in}}%
\pgfpathlineto{\pgfqpoint{2.807638in}{2.943467in}}%
\pgfpathlineto{\pgfqpoint{2.801627in}{2.946435in}}%
\pgfpathlineto{\pgfqpoint{2.795624in}{2.948950in}}%
\pgfpathlineto{\pgfqpoint{2.789629in}{2.951068in}}%
\pgfpathlineto{\pgfqpoint{2.783642in}{2.953020in}}%
\pgfpathlineto{\pgfqpoint{2.777659in}{2.955053in}}%
\pgfpathlineto{\pgfqpoint{2.766438in}{2.952567in}}%
\pgfpathlineto{\pgfqpoint{2.755215in}{2.949492in}}%
\pgfpathlineto{\pgfqpoint{2.744004in}{2.944278in}}%
\pgfpathlineto{\pgfqpoint{2.732805in}{2.936915in}}%
\pgfpathlineto{\pgfqpoint{2.721618in}{2.927868in}}%
\pgfpathlineto{\pgfqpoint{2.727622in}{2.920702in}}%
\pgfpathlineto{\pgfqpoint{2.733632in}{2.913141in}}%
\pgfpathlineto{\pgfqpoint{2.739651in}{2.905117in}}%
\pgfpathlineto{\pgfqpoint{2.745678in}{2.896569in}}%
\pgfpathclose%
\pgfusepath{stroke,fill}%
\end{pgfscope}%
\begin{pgfscope}%
\pgfpathrectangle{\pgfqpoint{0.887500in}{0.275000in}}{\pgfqpoint{4.225000in}{4.225000in}}%
\pgfusepath{clip}%
\pgfsetbuttcap%
\pgfsetroundjoin%
\definecolor{currentfill}{rgb}{0.678489,0.863742,0.189503}%
\pgfsetfillcolor{currentfill}%
\pgfsetfillopacity{0.700000}%
\pgfsetlinewidth{0.501875pt}%
\definecolor{currentstroke}{rgb}{1.000000,1.000000,1.000000}%
\pgfsetstrokecolor{currentstroke}%
\pgfsetstrokeopacity{0.500000}%
\pgfsetdash{}{0pt}%
\pgfpathmoveto{\pgfqpoint{3.799009in}{3.394573in}}%
\pgfpathlineto{\pgfqpoint{3.810008in}{3.397968in}}%
\pgfpathlineto{\pgfqpoint{3.821004in}{3.401424in}}%
\pgfpathlineto{\pgfqpoint{3.831995in}{3.404938in}}%
\pgfpathlineto{\pgfqpoint{3.842983in}{3.408505in}}%
\pgfpathlineto{\pgfqpoint{3.853966in}{3.412120in}}%
\pgfpathlineto{\pgfqpoint{3.847657in}{3.423792in}}%
\pgfpathlineto{\pgfqpoint{3.841351in}{3.435474in}}%
\pgfpathlineto{\pgfqpoint{3.835049in}{3.447162in}}%
\pgfpathlineto{\pgfqpoint{3.828750in}{3.458852in}}%
\pgfpathlineto{\pgfqpoint{3.822453in}{3.470538in}}%
\pgfpathlineto{\pgfqpoint{3.811461in}{3.466132in}}%
\pgfpathlineto{\pgfqpoint{3.800464in}{3.461733in}}%
\pgfpathlineto{\pgfqpoint{3.789463in}{3.457389in}}%
\pgfpathlineto{\pgfqpoint{3.778460in}{3.453147in}}%
\pgfpathlineto{\pgfqpoint{3.767454in}{3.449056in}}%
\pgfpathlineto{\pgfqpoint{3.773757in}{3.438080in}}%
\pgfpathlineto{\pgfqpoint{3.780064in}{3.427167in}}%
\pgfpathlineto{\pgfqpoint{3.786376in}{3.416293in}}%
\pgfpathlineto{\pgfqpoint{3.792691in}{3.405436in}}%
\pgfpathclose%
\pgfusepath{stroke,fill}%
\end{pgfscope}%
\begin{pgfscope}%
\pgfpathrectangle{\pgfqpoint{0.887500in}{0.275000in}}{\pgfqpoint{4.225000in}{4.225000in}}%
\pgfusepath{clip}%
\pgfsetbuttcap%
\pgfsetroundjoin%
\definecolor{currentfill}{rgb}{0.565498,0.842430,0.262877}%
\pgfsetfillcolor{currentfill}%
\pgfsetfillopacity{0.700000}%
\pgfsetlinewidth{0.501875pt}%
\definecolor{currentstroke}{rgb}{1.000000,1.000000,1.000000}%
\pgfsetstrokecolor{currentstroke}%
\pgfsetstrokeopacity{0.500000}%
\pgfsetdash{}{0pt}%
\pgfpathmoveto{\pgfqpoint{3.972081in}{3.311788in}}%
\pgfpathlineto{\pgfqpoint{3.983034in}{3.314951in}}%
\pgfpathlineto{\pgfqpoint{3.993983in}{3.318145in}}%
\pgfpathlineto{\pgfqpoint{4.004927in}{3.321369in}}%
\pgfpathlineto{\pgfqpoint{4.015867in}{3.324622in}}%
\pgfpathlineto{\pgfqpoint{4.026802in}{3.327902in}}%
\pgfpathlineto{\pgfqpoint{4.020464in}{3.340071in}}%
\pgfpathlineto{\pgfqpoint{4.014129in}{3.352267in}}%
\pgfpathlineto{\pgfqpoint{4.007799in}{3.364496in}}%
\pgfpathlineto{\pgfqpoint{4.001472in}{3.376760in}}%
\pgfpathlineto{\pgfqpoint{3.995149in}{3.389064in}}%
\pgfpathlineto{\pgfqpoint{3.984203in}{3.385158in}}%
\pgfpathlineto{\pgfqpoint{3.973255in}{3.381341in}}%
\pgfpathlineto{\pgfqpoint{3.962305in}{3.377612in}}%
\pgfpathlineto{\pgfqpoint{3.951351in}{3.373974in}}%
\pgfpathlineto{\pgfqpoint{3.940394in}{3.370425in}}%
\pgfpathlineto{\pgfqpoint{3.946722in}{3.358578in}}%
\pgfpathlineto{\pgfqpoint{3.953056in}{3.346810in}}%
\pgfpathlineto{\pgfqpoint{3.959394in}{3.335102in}}%
\pgfpathlineto{\pgfqpoint{3.965736in}{3.323435in}}%
\pgfpathclose%
\pgfusepath{stroke,fill}%
\end{pgfscope}%
\begin{pgfscope}%
\pgfpathrectangle{\pgfqpoint{0.887500in}{0.275000in}}{\pgfqpoint{4.225000in}{4.225000in}}%
\pgfusepath{clip}%
\pgfsetbuttcap%
\pgfsetroundjoin%
\definecolor{currentfill}{rgb}{0.626579,0.854645,0.223353}%
\pgfsetfillcolor{currentfill}%
\pgfsetfillopacity{0.700000}%
\pgfsetlinewidth{0.501875pt}%
\definecolor{currentstroke}{rgb}{1.000000,1.000000,1.000000}%
\pgfsetstrokecolor{currentstroke}%
\pgfsetstrokeopacity{0.500000}%
\pgfsetdash{}{0pt}%
\pgfpathmoveto{\pgfqpoint{3.885562in}{3.354054in}}%
\pgfpathlineto{\pgfqpoint{3.896535in}{3.357144in}}%
\pgfpathlineto{\pgfqpoint{3.907505in}{3.360326in}}%
\pgfpathlineto{\pgfqpoint{3.918471in}{3.363601in}}%
\pgfpathlineto{\pgfqpoint{3.929434in}{3.366968in}}%
\pgfpathlineto{\pgfqpoint{3.940394in}{3.370425in}}%
\pgfpathlineto{\pgfqpoint{3.934070in}{3.382370in}}%
\pgfpathlineto{\pgfqpoint{3.927752in}{3.394405in}}%
\pgfpathlineto{\pgfqpoint{3.921438in}{3.406514in}}%
\pgfpathlineto{\pgfqpoint{3.915128in}{3.418682in}}%
\pgfpathlineto{\pgfqpoint{3.908822in}{3.430891in}}%
\pgfpathlineto{\pgfqpoint{3.897859in}{3.427036in}}%
\pgfpathlineto{\pgfqpoint{3.886891in}{3.423235in}}%
\pgfpathlineto{\pgfqpoint{3.875920in}{3.419483in}}%
\pgfpathlineto{\pgfqpoint{3.864945in}{3.415779in}}%
\pgfpathlineto{\pgfqpoint{3.853966in}{3.412120in}}%
\pgfpathlineto{\pgfqpoint{3.860278in}{3.400462in}}%
\pgfpathlineto{\pgfqpoint{3.866594in}{3.388822in}}%
\pgfpathlineto{\pgfqpoint{3.872913in}{3.377204in}}%
\pgfpathlineto{\pgfqpoint{3.879236in}{3.365614in}}%
\pgfpathclose%
\pgfusepath{stroke,fill}%
\end{pgfscope}%
\begin{pgfscope}%
\pgfpathrectangle{\pgfqpoint{0.887500in}{0.275000in}}{\pgfqpoint{4.225000in}{4.225000in}}%
\pgfusepath{clip}%
\pgfsetbuttcap%
\pgfsetroundjoin%
\definecolor{currentfill}{rgb}{0.824940,0.884720,0.106217}%
\pgfsetfillcolor{currentfill}%
\pgfsetfillopacity{0.700000}%
\pgfsetlinewidth{0.501875pt}%
\definecolor{currentstroke}{rgb}{1.000000,1.000000,1.000000}%
\pgfsetstrokecolor{currentstroke}%
\pgfsetstrokeopacity{0.500000}%
\pgfsetdash{}{0pt}%
\pgfpathmoveto{\pgfqpoint{3.397142in}{3.512772in}}%
\pgfpathlineto{\pgfqpoint{3.408239in}{3.515949in}}%
\pgfpathlineto{\pgfqpoint{3.419331in}{3.519219in}}%
\pgfpathlineto{\pgfqpoint{3.430419in}{3.522639in}}%
\pgfpathlineto{\pgfqpoint{3.441503in}{3.526244in}}%
\pgfpathlineto{\pgfqpoint{3.452585in}{3.530009in}}%
\pgfpathlineto{\pgfqpoint{3.446316in}{3.536201in}}%
\pgfpathlineto{\pgfqpoint{3.440049in}{3.542125in}}%
\pgfpathlineto{\pgfqpoint{3.433785in}{3.547869in}}%
\pgfpathlineto{\pgfqpoint{3.427525in}{3.553521in}}%
\pgfpathlineto{\pgfqpoint{3.421270in}{3.559167in}}%
\pgfpathlineto{\pgfqpoint{3.410200in}{3.555186in}}%
\pgfpathlineto{\pgfqpoint{3.399129in}{3.551554in}}%
\pgfpathlineto{\pgfqpoint{3.388056in}{3.548322in}}%
\pgfpathlineto{\pgfqpoint{3.376981in}{3.545430in}}%
\pgfpathlineto{\pgfqpoint{3.365901in}{3.542775in}}%
\pgfpathlineto{\pgfqpoint{3.372143in}{3.537135in}}%
\pgfpathlineto{\pgfqpoint{3.378388in}{3.531375in}}%
\pgfpathlineto{\pgfqpoint{3.384637in}{3.525436in}}%
\pgfpathlineto{\pgfqpoint{3.390889in}{3.519255in}}%
\pgfpathclose%
\pgfusepath{stroke,fill}%
\end{pgfscope}%
\begin{pgfscope}%
\pgfpathrectangle{\pgfqpoint{0.887500in}{0.275000in}}{\pgfqpoint{4.225000in}{4.225000in}}%
\pgfusepath{clip}%
\pgfsetbuttcap%
\pgfsetroundjoin%
\definecolor{currentfill}{rgb}{0.506271,0.828786,0.300362}%
\pgfsetfillcolor{currentfill}%
\pgfsetfillopacity{0.700000}%
\pgfsetlinewidth{0.501875pt}%
\definecolor{currentstroke}{rgb}{1.000000,1.000000,1.000000}%
\pgfsetstrokecolor{currentstroke}%
\pgfsetstrokeopacity{0.500000}%
\pgfsetdash{}{0pt}%
\pgfpathmoveto{\pgfqpoint{4.058543in}{3.267344in}}%
\pgfpathlineto{\pgfqpoint{4.069463in}{3.270066in}}%
\pgfpathlineto{\pgfqpoint{4.080377in}{3.272749in}}%
\pgfpathlineto{\pgfqpoint{4.091284in}{3.275393in}}%
\pgfpathlineto{\pgfqpoint{4.102185in}{3.277997in}}%
\pgfpathlineto{\pgfqpoint{4.095836in}{3.290429in}}%
\pgfpathlineto{\pgfqpoint{4.089493in}{3.302997in}}%
\pgfpathlineto{\pgfqpoint{4.083156in}{3.315677in}}%
\pgfpathlineto{\pgfqpoint{4.076824in}{3.328445in}}%
\pgfpathlineto{\pgfqpoint{4.070497in}{3.341277in}}%
\pgfpathlineto{\pgfqpoint{4.059580in}{3.337898in}}%
\pgfpathlineto{\pgfqpoint{4.048658in}{3.334541in}}%
\pgfpathlineto{\pgfqpoint{4.037732in}{3.331209in}}%
\pgfpathlineto{\pgfqpoint{4.026802in}{3.327902in}}%
\pgfpathlineto{\pgfqpoint{4.033143in}{3.315758in}}%
\pgfpathlineto{\pgfqpoint{4.039488in}{3.303634in}}%
\pgfpathlineto{\pgfqpoint{4.045837in}{3.291526in}}%
\pgfpathlineto{\pgfqpoint{4.052188in}{3.279431in}}%
\pgfpathclose%
\pgfusepath{stroke,fill}%
\end{pgfscope}%
\begin{pgfscope}%
\pgfpathrectangle{\pgfqpoint{0.887500in}{0.275000in}}{\pgfqpoint{4.225000in}{4.225000in}}%
\pgfusepath{clip}%
\pgfsetbuttcap%
\pgfsetroundjoin%
\definecolor{currentfill}{rgb}{0.119512,0.607464,0.540218}%
\pgfsetfillcolor{currentfill}%
\pgfsetfillopacity{0.700000}%
\pgfsetlinewidth{0.501875pt}%
\definecolor{currentstroke}{rgb}{1.000000,1.000000,1.000000}%
\pgfsetstrokecolor{currentstroke}%
\pgfsetstrokeopacity{0.500000}%
\pgfsetdash{}{0pt}%
\pgfpathmoveto{\pgfqpoint{1.754243in}{2.749721in}}%
\pgfpathlineto{\pgfqpoint{1.765736in}{2.752997in}}%
\pgfpathlineto{\pgfqpoint{1.777223in}{2.756277in}}%
\pgfpathlineto{\pgfqpoint{1.788704in}{2.759566in}}%
\pgfpathlineto{\pgfqpoint{1.800180in}{2.762863in}}%
\pgfpathlineto{\pgfqpoint{1.811650in}{2.766171in}}%
\pgfpathlineto{\pgfqpoint{1.805949in}{2.773850in}}%
\pgfpathlineto{\pgfqpoint{1.800253in}{2.781516in}}%
\pgfpathlineto{\pgfqpoint{1.794561in}{2.789169in}}%
\pgfpathlineto{\pgfqpoint{1.788873in}{2.796807in}}%
\pgfpathlineto{\pgfqpoint{1.783189in}{2.804432in}}%
\pgfpathlineto{\pgfqpoint{1.771733in}{2.801105in}}%
\pgfpathlineto{\pgfqpoint{1.760271in}{2.797789in}}%
\pgfpathlineto{\pgfqpoint{1.748803in}{2.794482in}}%
\pgfpathlineto{\pgfqpoint{1.737330in}{2.791184in}}%
\pgfpathlineto{\pgfqpoint{1.725851in}{2.787892in}}%
\pgfpathlineto{\pgfqpoint{1.731521in}{2.780287in}}%
\pgfpathlineto{\pgfqpoint{1.737195in}{2.772668in}}%
\pgfpathlineto{\pgfqpoint{1.742874in}{2.765034in}}%
\pgfpathlineto{\pgfqpoint{1.748556in}{2.757385in}}%
\pgfpathclose%
\pgfusepath{stroke,fill}%
\end{pgfscope}%
\begin{pgfscope}%
\pgfpathrectangle{\pgfqpoint{0.887500in}{0.275000in}}{\pgfqpoint{4.225000in}{4.225000in}}%
\pgfusepath{clip}%
\pgfsetbuttcap%
\pgfsetroundjoin%
\definecolor{currentfill}{rgb}{0.126453,0.570633,0.549841}%
\pgfsetfillcolor{currentfill}%
\pgfsetfillopacity{0.700000}%
\pgfsetlinewidth{0.501875pt}%
\definecolor{currentstroke}{rgb}{1.000000,1.000000,1.000000}%
\pgfsetstrokecolor{currentstroke}%
\pgfsetstrokeopacity{0.500000}%
\pgfsetdash{}{0pt}%
\pgfpathmoveto{\pgfqpoint{2.298927in}{2.668351in}}%
\pgfpathlineto{\pgfqpoint{2.310287in}{2.671661in}}%
\pgfpathlineto{\pgfqpoint{2.321637in}{2.675255in}}%
\pgfpathlineto{\pgfqpoint{2.332974in}{2.679237in}}%
\pgfpathlineto{\pgfqpoint{2.344298in}{2.683707in}}%
\pgfpathlineto{\pgfqpoint{2.355607in}{2.688706in}}%
\pgfpathlineto{\pgfqpoint{2.349706in}{2.697266in}}%
\pgfpathlineto{\pgfqpoint{2.343810in}{2.705830in}}%
\pgfpathlineto{\pgfqpoint{2.337916in}{2.714392in}}%
\pgfpathlineto{\pgfqpoint{2.332027in}{2.722940in}}%
\pgfpathlineto{\pgfqpoint{2.326142in}{2.731461in}}%
\pgfpathlineto{\pgfqpoint{2.314871in}{2.725096in}}%
\pgfpathlineto{\pgfqpoint{2.303578in}{2.719679in}}%
\pgfpathlineto{\pgfqpoint{2.292263in}{2.715164in}}%
\pgfpathlineto{\pgfqpoint{2.280929in}{2.711365in}}%
\pgfpathlineto{\pgfqpoint{2.269580in}{2.708097in}}%
\pgfpathlineto{\pgfqpoint{2.275442in}{2.700138in}}%
\pgfpathlineto{\pgfqpoint{2.281308in}{2.692178in}}%
\pgfpathlineto{\pgfqpoint{2.287177in}{2.684225in}}%
\pgfpathlineto{\pgfqpoint{2.293050in}{2.676284in}}%
\pgfpathclose%
\pgfusepath{stroke,fill}%
\end{pgfscope}%
\begin{pgfscope}%
\pgfpathrectangle{\pgfqpoint{0.887500in}{0.275000in}}{\pgfqpoint{4.225000in}{4.225000in}}%
\pgfusepath{clip}%
\pgfsetbuttcap%
\pgfsetroundjoin%
\definecolor{currentfill}{rgb}{0.134692,0.658636,0.517649}%
\pgfsetfillcolor{currentfill}%
\pgfsetfillopacity{0.700000}%
\pgfsetlinewidth{0.501875pt}%
\definecolor{currentstroke}{rgb}{1.000000,1.000000,1.000000}%
\pgfsetstrokecolor{currentstroke}%
\pgfsetstrokeopacity{0.500000}%
\pgfsetdash{}{0pt}%
\pgfpathmoveto{\pgfqpoint{2.695857in}{2.820987in}}%
\pgfpathlineto{\pgfqpoint{2.707036in}{2.833353in}}%
\pgfpathlineto{\pgfqpoint{2.718208in}{2.846467in}}%
\pgfpathlineto{\pgfqpoint{2.729376in}{2.860100in}}%
\pgfpathlineto{\pgfqpoint{2.740543in}{2.873906in}}%
\pgfpathlineto{\pgfqpoint{2.751712in}{2.887541in}}%
\pgfpathlineto{\pgfqpoint{2.745678in}{2.896569in}}%
\pgfpathlineto{\pgfqpoint{2.739651in}{2.905117in}}%
\pgfpathlineto{\pgfqpoint{2.733632in}{2.913141in}}%
\pgfpathlineto{\pgfqpoint{2.727622in}{2.920702in}}%
\pgfpathlineto{\pgfqpoint{2.721618in}{2.927868in}}%
\pgfpathlineto{\pgfqpoint{2.710439in}{2.917600in}}%
\pgfpathlineto{\pgfqpoint{2.699266in}{2.906578in}}%
\pgfpathlineto{\pgfqpoint{2.688094in}{2.895266in}}%
\pgfpathlineto{\pgfqpoint{2.676919in}{2.884127in}}%
\pgfpathlineto{\pgfqpoint{2.665739in}{2.873484in}}%
\pgfpathlineto{\pgfqpoint{2.671752in}{2.863408in}}%
\pgfpathlineto{\pgfqpoint{2.677771in}{2.853061in}}%
\pgfpathlineto{\pgfqpoint{2.683795in}{2.842503in}}%
\pgfpathlineto{\pgfqpoint{2.689824in}{2.831793in}}%
\pgfpathclose%
\pgfusepath{stroke,fill}%
\end{pgfscope}%
\begin{pgfscope}%
\pgfpathrectangle{\pgfqpoint{0.887500in}{0.275000in}}{\pgfqpoint{4.225000in}{4.225000in}}%
\pgfusepath{clip}%
\pgfsetbuttcap%
\pgfsetroundjoin%
\definecolor{currentfill}{rgb}{0.120081,0.622161,0.534946}%
\pgfsetfillcolor{currentfill}%
\pgfsetfillopacity{0.700000}%
\pgfsetlinewidth{0.501875pt}%
\definecolor{currentstroke}{rgb}{1.000000,1.000000,1.000000}%
\pgfsetstrokecolor{currentstroke}%
\pgfsetstrokeopacity{0.500000}%
\pgfsetdash{}{0pt}%
\pgfpathmoveto{\pgfqpoint{1.524914in}{2.776069in}}%
\pgfpathlineto{\pgfqpoint{1.536463in}{2.779392in}}%
\pgfpathlineto{\pgfqpoint{1.548006in}{2.782715in}}%
\pgfpathlineto{\pgfqpoint{1.559543in}{2.786036in}}%
\pgfpathlineto{\pgfqpoint{1.571075in}{2.789355in}}%
\pgfpathlineto{\pgfqpoint{1.582601in}{2.792671in}}%
\pgfpathlineto{\pgfqpoint{1.576985in}{2.800143in}}%
\pgfpathlineto{\pgfqpoint{1.571373in}{2.807597in}}%
\pgfpathlineto{\pgfqpoint{1.565765in}{2.815032in}}%
\pgfpathlineto{\pgfqpoint{1.560162in}{2.822448in}}%
\pgfpathlineto{\pgfqpoint{1.554563in}{2.829844in}}%
\pgfpathlineto{\pgfqpoint{1.543051in}{2.826512in}}%
\pgfpathlineto{\pgfqpoint{1.531533in}{2.823179in}}%
\pgfpathlineto{\pgfqpoint{1.520010in}{2.819845in}}%
\pgfpathlineto{\pgfqpoint{1.508482in}{2.816510in}}%
\pgfpathlineto{\pgfqpoint{1.496947in}{2.813175in}}%
\pgfpathlineto{\pgfqpoint{1.502532in}{2.805790in}}%
\pgfpathlineto{\pgfqpoint{1.508121in}{2.798387in}}%
\pgfpathlineto{\pgfqpoint{1.513714in}{2.790965in}}%
\pgfpathlineto{\pgfqpoint{1.519312in}{2.783526in}}%
\pgfpathclose%
\pgfusepath{stroke,fill}%
\end{pgfscope}%
\begin{pgfscope}%
\pgfpathrectangle{\pgfqpoint{0.887500in}{0.275000in}}{\pgfqpoint{4.225000in}{4.225000in}}%
\pgfusepath{clip}%
\pgfsetbuttcap%
\pgfsetroundjoin%
\definecolor{currentfill}{rgb}{0.122606,0.585371,0.546557}%
\pgfsetfillcolor{currentfill}%
\pgfsetfillopacity{0.700000}%
\pgfsetlinewidth{0.501875pt}%
\definecolor{currentstroke}{rgb}{1.000000,1.000000,1.000000}%
\pgfsetstrokecolor{currentstroke}%
\pgfsetstrokeopacity{0.500000}%
\pgfsetdash{}{0pt}%
\pgfpathmoveto{\pgfqpoint{2.069613in}{2.698761in}}%
\pgfpathlineto{\pgfqpoint{2.081033in}{2.702013in}}%
\pgfpathlineto{\pgfqpoint{2.092447in}{2.705304in}}%
\pgfpathlineto{\pgfqpoint{2.103854in}{2.708656in}}%
\pgfpathlineto{\pgfqpoint{2.115253in}{2.712091in}}%
\pgfpathlineto{\pgfqpoint{2.126644in}{2.715633in}}%
\pgfpathlineto{\pgfqpoint{2.120829in}{2.723598in}}%
\pgfpathlineto{\pgfqpoint{2.115018in}{2.731551in}}%
\pgfpathlineto{\pgfqpoint{2.109211in}{2.739491in}}%
\pgfpathlineto{\pgfqpoint{2.103409in}{2.747419in}}%
\pgfpathlineto{\pgfqpoint{2.097610in}{2.755334in}}%
\pgfpathlineto{\pgfqpoint{2.086232in}{2.751748in}}%
\pgfpathlineto{\pgfqpoint{2.074846in}{2.748280in}}%
\pgfpathlineto{\pgfqpoint{2.063452in}{2.744906in}}%
\pgfpathlineto{\pgfqpoint{2.052051in}{2.741599in}}%
\pgfpathlineto{\pgfqpoint{2.040643in}{2.738335in}}%
\pgfpathlineto{\pgfqpoint{2.046429in}{2.730447in}}%
\pgfpathlineto{\pgfqpoint{2.052219in}{2.722546in}}%
\pgfpathlineto{\pgfqpoint{2.058013in}{2.714631in}}%
\pgfpathlineto{\pgfqpoint{2.063810in}{2.706703in}}%
\pgfpathclose%
\pgfusepath{stroke,fill}%
\end{pgfscope}%
\begin{pgfscope}%
\pgfpathrectangle{\pgfqpoint{0.887500in}{0.275000in}}{\pgfqpoint{4.225000in}{4.225000in}}%
\pgfusepath{clip}%
\pgfsetbuttcap%
\pgfsetroundjoin%
\definecolor{currentfill}{rgb}{0.121380,0.629492,0.531973}%
\pgfsetfillcolor{currentfill}%
\pgfsetfillopacity{0.700000}%
\pgfsetlinewidth{0.501875pt}%
\definecolor{currentstroke}{rgb}{1.000000,1.000000,1.000000}%
\pgfsetstrokecolor{currentstroke}%
\pgfsetstrokeopacity{0.500000}%
\pgfsetdash{}{0pt}%
\pgfpathmoveto{\pgfqpoint{2.639912in}{2.763695in}}%
\pgfpathlineto{\pgfqpoint{2.651101in}{2.775292in}}%
\pgfpathlineto{\pgfqpoint{2.662292in}{2.786597in}}%
\pgfpathlineto{\pgfqpoint{2.673484in}{2.797833in}}%
\pgfpathlineto{\pgfqpoint{2.684673in}{2.809221in}}%
\pgfpathlineto{\pgfqpoint{2.695857in}{2.820987in}}%
\pgfpathlineto{\pgfqpoint{2.689824in}{2.831793in}}%
\pgfpathlineto{\pgfqpoint{2.683795in}{2.842503in}}%
\pgfpathlineto{\pgfqpoint{2.677771in}{2.853061in}}%
\pgfpathlineto{\pgfqpoint{2.671752in}{2.863408in}}%
\pgfpathlineto{\pgfqpoint{2.665739in}{2.873484in}}%
\pgfpathlineto{\pgfqpoint{2.654553in}{2.863204in}}%
\pgfpathlineto{\pgfqpoint{2.643365in}{2.853061in}}%
\pgfpathlineto{\pgfqpoint{2.632176in}{2.842829in}}%
\pgfpathlineto{\pgfqpoint{2.620989in}{2.832281in}}%
\pgfpathlineto{\pgfqpoint{2.609807in}{2.821192in}}%
\pgfpathlineto{\pgfqpoint{2.615829in}{2.809092in}}%
\pgfpathlineto{\pgfqpoint{2.621852in}{2.797143in}}%
\pgfpathlineto{\pgfqpoint{2.627874in}{2.785509in}}%
\pgfpathlineto{\pgfqpoint{2.633894in}{2.774350in}}%
\pgfpathclose%
\pgfusepath{stroke,fill}%
\end{pgfscope}%
\begin{pgfscope}%
\pgfpathrectangle{\pgfqpoint{0.887500in}{0.275000in}}{\pgfqpoint{4.225000in}{4.225000in}}%
\pgfusepath{clip}%
\pgfsetbuttcap%
\pgfsetroundjoin%
\definecolor{currentfill}{rgb}{0.119738,0.603785,0.541400}%
\pgfsetfillcolor{currentfill}%
\pgfsetfillopacity{0.700000}%
\pgfsetlinewidth{0.501875pt}%
\definecolor{currentstroke}{rgb}{1.000000,1.000000,1.000000}%
\pgfsetstrokecolor{currentstroke}%
\pgfsetstrokeopacity{0.500000}%
\pgfsetdash{}{0pt}%
\pgfpathmoveto{\pgfqpoint{2.584024in}{2.699804in}}%
\pgfpathlineto{\pgfqpoint{2.595200in}{2.712707in}}%
\pgfpathlineto{\pgfqpoint{2.606374in}{2.725826in}}%
\pgfpathlineto{\pgfqpoint{2.617548in}{2.738879in}}%
\pgfpathlineto{\pgfqpoint{2.628727in}{2.751583in}}%
\pgfpathlineto{\pgfqpoint{2.639912in}{2.763695in}}%
\pgfpathlineto{\pgfqpoint{2.633894in}{2.774350in}}%
\pgfpathlineto{\pgfqpoint{2.627874in}{2.785509in}}%
\pgfpathlineto{\pgfqpoint{2.621852in}{2.797143in}}%
\pgfpathlineto{\pgfqpoint{2.615829in}{2.809092in}}%
\pgfpathlineto{\pgfqpoint{2.609807in}{2.821192in}}%
\pgfpathlineto{\pgfqpoint{2.598633in}{2.809336in}}%
\pgfpathlineto{\pgfqpoint{2.587468in}{2.796646in}}%
\pgfpathlineto{\pgfqpoint{2.576310in}{2.783394in}}%
\pgfpathlineto{\pgfqpoint{2.565154in}{2.769896in}}%
\pgfpathlineto{\pgfqpoint{2.553999in}{2.756468in}}%
\pgfpathlineto{\pgfqpoint{2.560010in}{2.744147in}}%
\pgfpathlineto{\pgfqpoint{2.566020in}{2.732189in}}%
\pgfpathlineto{\pgfqpoint{2.572026in}{2.720742in}}%
\pgfpathlineto{\pgfqpoint{2.578027in}{2.709945in}}%
\pgfpathclose%
\pgfusepath{stroke,fill}%
\end{pgfscope}%
\begin{pgfscope}%
\pgfpathrectangle{\pgfqpoint{0.887500in}{0.275000in}}{\pgfqpoint{4.225000in}{4.225000in}}%
\pgfusepath{clip}%
\pgfsetbuttcap%
\pgfsetroundjoin%
\definecolor{currentfill}{rgb}{0.824940,0.884720,0.106217}%
\pgfsetfillcolor{currentfill}%
\pgfsetfillopacity{0.700000}%
\pgfsetlinewidth{0.501875pt}%
\definecolor{currentstroke}{rgb}{1.000000,1.000000,1.000000}%
\pgfsetstrokecolor{currentstroke}%
\pgfsetstrokeopacity{0.500000}%
\pgfsetdash{}{0pt}%
\pgfpathmoveto{\pgfqpoint{3.254779in}{3.511434in}}%
\pgfpathlineto{\pgfqpoint{3.265919in}{3.515226in}}%
\pgfpathlineto{\pgfqpoint{3.277053in}{3.518964in}}%
\pgfpathlineto{\pgfqpoint{3.288182in}{3.522598in}}%
\pgfpathlineto{\pgfqpoint{3.299305in}{3.526082in}}%
\pgfpathlineto{\pgfqpoint{3.310421in}{3.529366in}}%
\pgfpathlineto{\pgfqpoint{3.304199in}{3.535207in}}%
\pgfpathlineto{\pgfqpoint{3.297980in}{3.540841in}}%
\pgfpathlineto{\pgfqpoint{3.291766in}{3.546283in}}%
\pgfpathlineto{\pgfqpoint{3.285554in}{3.551541in}}%
\pgfpathlineto{\pgfqpoint{3.279347in}{3.556621in}}%
\pgfpathlineto{\pgfqpoint{3.268246in}{3.553541in}}%
\pgfpathlineto{\pgfqpoint{3.257139in}{3.550358in}}%
\pgfpathlineto{\pgfqpoint{3.246026in}{3.547100in}}%
\pgfpathlineto{\pgfqpoint{3.234908in}{3.543795in}}%
\pgfpathlineto{\pgfqpoint{3.223785in}{3.540470in}}%
\pgfpathlineto{\pgfqpoint{3.229976in}{3.535102in}}%
\pgfpathlineto{\pgfqpoint{3.236171in}{3.529505in}}%
\pgfpathlineto{\pgfqpoint{3.242370in}{3.523691in}}%
\pgfpathlineto{\pgfqpoint{3.248573in}{3.517668in}}%
\pgfpathclose%
\pgfusepath{stroke,fill}%
\end{pgfscope}%
\begin{pgfscope}%
\pgfpathrectangle{\pgfqpoint{0.887500in}{0.275000in}}{\pgfqpoint{4.225000in}{4.225000in}}%
\pgfusepath{clip}%
\pgfsetbuttcap%
\pgfsetroundjoin%
\definecolor{currentfill}{rgb}{0.120092,0.600104,0.542530}%
\pgfsetfillcolor{currentfill}%
\pgfsetfillopacity{0.700000}%
\pgfsetlinewidth{0.501875pt}%
\definecolor{currentstroke}{rgb}{1.000000,1.000000,1.000000}%
\pgfsetstrokecolor{currentstroke}%
\pgfsetstrokeopacity{0.500000}%
\pgfsetdash{}{0pt}%
\pgfpathmoveto{\pgfqpoint{1.840214in}{2.727565in}}%
\pgfpathlineto{\pgfqpoint{1.851691in}{2.730873in}}%
\pgfpathlineto{\pgfqpoint{1.863162in}{2.734191in}}%
\pgfpathlineto{\pgfqpoint{1.874627in}{2.737521in}}%
\pgfpathlineto{\pgfqpoint{1.886087in}{2.740860in}}%
\pgfpathlineto{\pgfqpoint{1.897540in}{2.744210in}}%
\pgfpathlineto{\pgfqpoint{1.891806in}{2.751969in}}%
\pgfpathlineto{\pgfqpoint{1.886076in}{2.759716in}}%
\pgfpathlineto{\pgfqpoint{1.880349in}{2.767450in}}%
\pgfpathlineto{\pgfqpoint{1.874627in}{2.775172in}}%
\pgfpathlineto{\pgfqpoint{1.868909in}{2.782882in}}%
\pgfpathlineto{\pgfqpoint{1.857469in}{2.779517in}}%
\pgfpathlineto{\pgfqpoint{1.846023in}{2.776163in}}%
\pgfpathlineto{\pgfqpoint{1.834571in}{2.772821in}}%
\pgfpathlineto{\pgfqpoint{1.823113in}{2.769490in}}%
\pgfpathlineto{\pgfqpoint{1.811650in}{2.766171in}}%
\pgfpathlineto{\pgfqpoint{1.817354in}{2.758477in}}%
\pgfpathlineto{\pgfqpoint{1.823063in}{2.750771in}}%
\pgfpathlineto{\pgfqpoint{1.828776in}{2.743050in}}%
\pgfpathlineto{\pgfqpoint{1.834493in}{2.735315in}}%
\pgfpathclose%
\pgfusepath{stroke,fill}%
\end{pgfscope}%
\begin{pgfscope}%
\pgfpathrectangle{\pgfqpoint{0.887500in}{0.275000in}}{\pgfqpoint{4.225000in}{4.225000in}}%
\pgfusepath{clip}%
\pgfsetbuttcap%
\pgfsetroundjoin%
\definecolor{currentfill}{rgb}{0.125394,0.574318,0.549086}%
\pgfsetfillcolor{currentfill}%
\pgfsetfillopacity{0.700000}%
\pgfsetlinewidth{0.501875pt}%
\definecolor{currentstroke}{rgb}{1.000000,1.000000,1.000000}%
\pgfsetstrokecolor{currentstroke}%
\pgfsetstrokeopacity{0.500000}%
\pgfsetdash{}{0pt}%
\pgfpathmoveto{\pgfqpoint{2.527979in}{2.647602in}}%
\pgfpathlineto{\pgfqpoint{2.539219in}{2.655853in}}%
\pgfpathlineto{\pgfqpoint{2.550442in}{2.665218in}}%
\pgfpathlineto{\pgfqpoint{2.561649in}{2.675779in}}%
\pgfpathlineto{\pgfqpoint{2.572841in}{2.687401in}}%
\pgfpathlineto{\pgfqpoint{2.584024in}{2.699804in}}%
\pgfpathlineto{\pgfqpoint{2.578027in}{2.709945in}}%
\pgfpathlineto{\pgfqpoint{2.572026in}{2.720742in}}%
\pgfpathlineto{\pgfqpoint{2.566020in}{2.732189in}}%
\pgfpathlineto{\pgfqpoint{2.560010in}{2.744147in}}%
\pgfpathlineto{\pgfqpoint{2.553999in}{2.756468in}}%
\pgfpathlineto{\pgfqpoint{2.542838in}{2.743428in}}%
\pgfpathlineto{\pgfqpoint{2.531669in}{2.731090in}}%
\pgfpathlineto{\pgfqpoint{2.520485in}{2.719767in}}%
\pgfpathlineto{\pgfqpoint{2.509284in}{2.709611in}}%
\pgfpathlineto{\pgfqpoint{2.498067in}{2.700531in}}%
\pgfpathlineto{\pgfqpoint{2.504055in}{2.689067in}}%
\pgfpathlineto{\pgfqpoint{2.510042in}{2.677968in}}%
\pgfpathlineto{\pgfqpoint{2.516024in}{2.667319in}}%
\pgfpathlineto{\pgfqpoint{2.522003in}{2.657199in}}%
\pgfpathclose%
\pgfusepath{stroke,fill}%
\end{pgfscope}%
\begin{pgfscope}%
\pgfpathrectangle{\pgfqpoint{0.887500in}{0.275000in}}{\pgfqpoint{4.225000in}{4.225000in}}%
\pgfusepath{clip}%
\pgfsetbuttcap%
\pgfsetroundjoin%
\definecolor{currentfill}{rgb}{0.814576,0.883393,0.110347}%
\pgfsetfillcolor{currentfill}%
\pgfsetfillopacity{0.700000}%
\pgfsetlinewidth{0.501875pt}%
\definecolor{currentstroke}{rgb}{1.000000,1.000000,1.000000}%
\pgfsetstrokecolor{currentstroke}%
\pgfsetstrokeopacity{0.500000}%
\pgfsetdash{}{0pt}%
\pgfpathmoveto{\pgfqpoint{3.483927in}{3.492012in}}%
\pgfpathlineto{\pgfqpoint{3.495013in}{3.495629in}}%
\pgfpathlineto{\pgfqpoint{3.506093in}{3.499233in}}%
\pgfpathlineto{\pgfqpoint{3.517169in}{3.502823in}}%
\pgfpathlineto{\pgfqpoint{3.528239in}{3.506398in}}%
\pgfpathlineto{\pgfqpoint{3.539305in}{3.509958in}}%
\pgfpathlineto{\pgfqpoint{3.533035in}{3.519105in}}%
\pgfpathlineto{\pgfqpoint{3.526761in}{3.527636in}}%
\pgfpathlineto{\pgfqpoint{3.520485in}{3.535590in}}%
\pgfpathlineto{\pgfqpoint{3.514208in}{3.543049in}}%
\pgfpathlineto{\pgfqpoint{3.507930in}{3.550093in}}%
\pgfpathlineto{\pgfqpoint{3.496870in}{3.546018in}}%
\pgfpathlineto{\pgfqpoint{3.485805in}{3.541939in}}%
\pgfpathlineto{\pgfqpoint{3.474736in}{3.537890in}}%
\pgfpathlineto{\pgfqpoint{3.463662in}{3.533902in}}%
\pgfpathlineto{\pgfqpoint{3.452585in}{3.530009in}}%
\pgfpathlineto{\pgfqpoint{3.458855in}{3.523462in}}%
\pgfpathlineto{\pgfqpoint{3.465125in}{3.516473in}}%
\pgfpathlineto{\pgfqpoint{3.471395in}{3.508952in}}%
\pgfpathlineto{\pgfqpoint{3.477662in}{3.500813in}}%
\pgfpathclose%
\pgfusepath{stroke,fill}%
\end{pgfscope}%
\begin{pgfscope}%
\pgfpathrectangle{\pgfqpoint{0.887500in}{0.275000in}}{\pgfqpoint{4.225000in}{4.225000in}}%
\pgfusepath{clip}%
\pgfsetbuttcap%
\pgfsetroundjoin%
\definecolor{currentfill}{rgb}{0.127568,0.566949,0.550556}%
\pgfsetfillcolor{currentfill}%
\pgfsetfillopacity{0.700000}%
\pgfsetlinewidth{0.501875pt}%
\definecolor{currentstroke}{rgb}{1.000000,1.000000,1.000000}%
\pgfsetstrokecolor{currentstroke}%
\pgfsetstrokeopacity{0.500000}%
\pgfsetdash{}{0pt}%
\pgfpathmoveto{\pgfqpoint{2.385161in}{2.646087in}}%
\pgfpathlineto{\pgfqpoint{2.396499in}{2.649835in}}%
\pgfpathlineto{\pgfqpoint{2.407829in}{2.653782in}}%
\pgfpathlineto{\pgfqpoint{2.419149in}{2.657988in}}%
\pgfpathlineto{\pgfqpoint{2.430459in}{2.662514in}}%
\pgfpathlineto{\pgfqpoint{2.441759in}{2.667420in}}%
\pgfpathlineto{\pgfqpoint{2.435800in}{2.677740in}}%
\pgfpathlineto{\pgfqpoint{2.429843in}{2.688211in}}%
\pgfpathlineto{\pgfqpoint{2.423887in}{2.698788in}}%
\pgfpathlineto{\pgfqpoint{2.417933in}{2.709426in}}%
\pgfpathlineto{\pgfqpoint{2.411982in}{2.720080in}}%
\pgfpathlineto{\pgfqpoint{2.400726in}{2.713111in}}%
\pgfpathlineto{\pgfqpoint{2.389462in}{2.706442in}}%
\pgfpathlineto{\pgfqpoint{2.378188in}{2.700121in}}%
\pgfpathlineto{\pgfqpoint{2.366903in}{2.694193in}}%
\pgfpathlineto{\pgfqpoint{2.355607in}{2.688706in}}%
\pgfpathlineto{\pgfqpoint{2.361511in}{2.680152in}}%
\pgfpathlineto{\pgfqpoint{2.367418in}{2.671610in}}%
\pgfpathlineto{\pgfqpoint{2.373329in}{2.663082in}}%
\pgfpathlineto{\pgfqpoint{2.379243in}{2.654574in}}%
\pgfpathclose%
\pgfusepath{stroke,fill}%
\end{pgfscope}%
\begin{pgfscope}%
\pgfpathrectangle{\pgfqpoint{0.887500in}{0.275000in}}{\pgfqpoint{4.225000in}{4.225000in}}%
\pgfusepath{clip}%
\pgfsetbuttcap%
\pgfsetroundjoin%
\definecolor{currentfill}{rgb}{0.119483,0.614817,0.537692}%
\pgfsetfillcolor{currentfill}%
\pgfsetfillopacity{0.700000}%
\pgfsetlinewidth{0.501875pt}%
\definecolor{currentstroke}{rgb}{1.000000,1.000000,1.000000}%
\pgfsetstrokecolor{currentstroke}%
\pgfsetstrokeopacity{0.500000}%
\pgfsetdash{}{0pt}%
\pgfpathmoveto{\pgfqpoint{1.610748in}{2.755041in}}%
\pgfpathlineto{\pgfqpoint{1.622284in}{2.758336in}}%
\pgfpathlineto{\pgfqpoint{1.633813in}{2.761626in}}%
\pgfpathlineto{\pgfqpoint{1.645337in}{2.764913in}}%
\pgfpathlineto{\pgfqpoint{1.656856in}{2.768197in}}%
\pgfpathlineto{\pgfqpoint{1.668369in}{2.771479in}}%
\pgfpathlineto{\pgfqpoint{1.662717in}{2.779054in}}%
\pgfpathlineto{\pgfqpoint{1.657069in}{2.786612in}}%
\pgfpathlineto{\pgfqpoint{1.651426in}{2.794155in}}%
\pgfpathlineto{\pgfqpoint{1.645787in}{2.801681in}}%
\pgfpathlineto{\pgfqpoint{1.640152in}{2.809191in}}%
\pgfpathlineto{\pgfqpoint{1.628653in}{2.805892in}}%
\pgfpathlineto{\pgfqpoint{1.617148in}{2.802592in}}%
\pgfpathlineto{\pgfqpoint{1.605638in}{2.799289in}}%
\pgfpathlineto{\pgfqpoint{1.594122in}{2.795982in}}%
\pgfpathlineto{\pgfqpoint{1.582601in}{2.792671in}}%
\pgfpathlineto{\pgfqpoint{1.588222in}{2.785180in}}%
\pgfpathlineto{\pgfqpoint{1.593847in}{2.777671in}}%
\pgfpathlineto{\pgfqpoint{1.599477in}{2.770145in}}%
\pgfpathlineto{\pgfqpoint{1.605110in}{2.762602in}}%
\pgfpathclose%
\pgfusepath{stroke,fill}%
\end{pgfscope}%
\begin{pgfscope}%
\pgfpathrectangle{\pgfqpoint{0.887500in}{0.275000in}}{\pgfqpoint{4.225000in}{4.225000in}}%
\pgfusepath{clip}%
\pgfsetbuttcap%
\pgfsetroundjoin%
\definecolor{currentfill}{rgb}{0.124395,0.578002,0.548287}%
\pgfsetfillcolor{currentfill}%
\pgfsetfillopacity{0.700000}%
\pgfsetlinewidth{0.501875pt}%
\definecolor{currentstroke}{rgb}{1.000000,1.000000,1.000000}%
\pgfsetstrokecolor{currentstroke}%
\pgfsetstrokeopacity{0.500000}%
\pgfsetdash{}{0pt}%
\pgfpathmoveto{\pgfqpoint{2.155780in}{2.675613in}}%
\pgfpathlineto{\pgfqpoint{2.167178in}{2.679183in}}%
\pgfpathlineto{\pgfqpoint{2.178570in}{2.682795in}}%
\pgfpathlineto{\pgfqpoint{2.189956in}{2.686393in}}%
\pgfpathlineto{\pgfqpoint{2.201339in}{2.689926in}}%
\pgfpathlineto{\pgfqpoint{2.212718in}{2.693338in}}%
\pgfpathlineto{\pgfqpoint{2.206869in}{2.701437in}}%
\pgfpathlineto{\pgfqpoint{2.201024in}{2.709518in}}%
\pgfpathlineto{\pgfqpoint{2.195183in}{2.717580in}}%
\pgfpathlineto{\pgfqpoint{2.189347in}{2.725622in}}%
\pgfpathlineto{\pgfqpoint{2.183514in}{2.733642in}}%
\pgfpathlineto{\pgfqpoint{2.172147in}{2.730228in}}%
\pgfpathlineto{\pgfqpoint{2.160778in}{2.726650in}}%
\pgfpathlineto{\pgfqpoint{2.149405in}{2.722977in}}%
\pgfpathlineto{\pgfqpoint{2.138028in}{2.719281in}}%
\pgfpathlineto{\pgfqpoint{2.126644in}{2.715633in}}%
\pgfpathlineto{\pgfqpoint{2.132464in}{2.707654in}}%
\pgfpathlineto{\pgfqpoint{2.138287in}{2.699663in}}%
\pgfpathlineto{\pgfqpoint{2.144114in}{2.691659in}}%
\pgfpathlineto{\pgfqpoint{2.149945in}{2.683643in}}%
\pgfpathclose%
\pgfusepath{stroke,fill}%
\end{pgfscope}%
\begin{pgfscope}%
\pgfpathrectangle{\pgfqpoint{0.887500in}{0.275000in}}{\pgfqpoint{4.225000in}{4.225000in}}%
\pgfusepath{clip}%
\pgfsetbuttcap%
\pgfsetroundjoin%
\definecolor{currentfill}{rgb}{0.783315,0.879285,0.125405}%
\pgfsetfillcolor{currentfill}%
\pgfsetfillopacity{0.700000}%
\pgfsetlinewidth{0.501875pt}%
\definecolor{currentstroke}{rgb}{1.000000,1.000000,1.000000}%
\pgfsetstrokecolor{currentstroke}%
\pgfsetstrokeopacity{0.500000}%
\pgfsetdash{}{0pt}%
\pgfpathmoveto{\pgfqpoint{2.969639in}{3.490667in}}%
\pgfpathlineto{\pgfqpoint{2.980852in}{3.493231in}}%
\pgfpathlineto{\pgfqpoint{2.992061in}{3.494902in}}%
\pgfpathlineto{\pgfqpoint{3.003264in}{3.496178in}}%
\pgfpathlineto{\pgfqpoint{3.014460in}{3.497553in}}%
\pgfpathlineto{\pgfqpoint{3.025649in}{3.499522in}}%
\pgfpathlineto{\pgfqpoint{3.019536in}{3.501901in}}%
\pgfpathlineto{\pgfqpoint{3.013428in}{3.504367in}}%
\pgfpathlineto{\pgfqpoint{3.007325in}{3.506975in}}%
\pgfpathlineto{\pgfqpoint{3.001228in}{3.509685in}}%
\pgfpathlineto{\pgfqpoint{2.995137in}{3.512429in}}%
\pgfpathlineto{\pgfqpoint{2.983964in}{3.510802in}}%
\pgfpathlineto{\pgfqpoint{2.972784in}{3.509821in}}%
\pgfpathlineto{\pgfqpoint{2.961597in}{3.509005in}}%
\pgfpathlineto{\pgfqpoint{2.950404in}{3.507873in}}%
\pgfpathlineto{\pgfqpoint{2.939207in}{3.505941in}}%
\pgfpathlineto{\pgfqpoint{2.945282in}{3.502958in}}%
\pgfpathlineto{\pgfqpoint{2.951363in}{3.499840in}}%
\pgfpathlineto{\pgfqpoint{2.957450in}{3.496714in}}%
\pgfpathlineto{\pgfqpoint{2.963542in}{3.493667in}}%
\pgfpathclose%
\pgfusepath{stroke,fill}%
\end{pgfscope}%
\begin{pgfscope}%
\pgfpathrectangle{\pgfqpoint{0.887500in}{0.275000in}}{\pgfqpoint{4.225000in}{4.225000in}}%
\pgfusepath{clip}%
\pgfsetbuttcap%
\pgfsetroundjoin%
\definecolor{currentfill}{rgb}{0.814576,0.883393,0.110347}%
\pgfsetfillcolor{currentfill}%
\pgfsetfillopacity{0.700000}%
\pgfsetlinewidth{0.501875pt}%
\definecolor{currentstroke}{rgb}{1.000000,1.000000,1.000000}%
\pgfsetstrokecolor{currentstroke}%
\pgfsetstrokeopacity{0.500000}%
\pgfsetdash{}{0pt}%
\pgfpathmoveto{\pgfqpoint{3.112257in}{3.503936in}}%
\pgfpathlineto{\pgfqpoint{3.123434in}{3.508200in}}%
\pgfpathlineto{\pgfqpoint{3.134606in}{3.512389in}}%
\pgfpathlineto{\pgfqpoint{3.145772in}{3.516375in}}%
\pgfpathlineto{\pgfqpoint{3.156933in}{3.520144in}}%
\pgfpathlineto{\pgfqpoint{3.168089in}{3.523735in}}%
\pgfpathlineto{\pgfqpoint{3.161917in}{3.529159in}}%
\pgfpathlineto{\pgfqpoint{3.155749in}{3.534267in}}%
\pgfpathlineto{\pgfqpoint{3.149585in}{3.539027in}}%
\pgfpathlineto{\pgfqpoint{3.143425in}{3.543408in}}%
\pgfpathlineto{\pgfqpoint{3.137270in}{3.547379in}}%
\pgfpathlineto{\pgfqpoint{3.126129in}{3.543453in}}%
\pgfpathlineto{\pgfqpoint{3.114983in}{3.539067in}}%
\pgfpathlineto{\pgfqpoint{3.103832in}{3.534127in}}%
\pgfpathlineto{\pgfqpoint{3.092676in}{3.528651in}}%
\pgfpathlineto{\pgfqpoint{3.081515in}{3.522890in}}%
\pgfpathlineto{\pgfqpoint{3.087653in}{3.519700in}}%
\pgfpathlineto{\pgfqpoint{3.093797in}{3.516218in}}%
\pgfpathlineto{\pgfqpoint{3.099945in}{3.512436in}}%
\pgfpathlineto{\pgfqpoint{3.106099in}{3.508345in}}%
\pgfpathclose%
\pgfusepath{stroke,fill}%
\end{pgfscope}%
\begin{pgfscope}%
\pgfpathrectangle{\pgfqpoint{0.887500in}{0.275000in}}{\pgfqpoint{4.225000in}{4.225000in}}%
\pgfusepath{clip}%
\pgfsetbuttcap%
\pgfsetroundjoin%
\definecolor{currentfill}{rgb}{0.772852,0.877868,0.131109}%
\pgfsetfillcolor{currentfill}%
\pgfsetfillopacity{0.700000}%
\pgfsetlinewidth{0.501875pt}%
\definecolor{currentstroke}{rgb}{1.000000,1.000000,1.000000}%
\pgfsetstrokecolor{currentstroke}%
\pgfsetstrokeopacity{0.500000}%
\pgfsetdash{}{0pt}%
\pgfpathmoveto{\pgfqpoint{3.570626in}{3.457452in}}%
\pgfpathlineto{\pgfqpoint{3.581690in}{3.460769in}}%
\pgfpathlineto{\pgfqpoint{3.592749in}{3.464053in}}%
\pgfpathlineto{\pgfqpoint{3.603801in}{3.467284in}}%
\pgfpathlineto{\pgfqpoint{3.614848in}{3.470443in}}%
\pgfpathlineto{\pgfqpoint{3.625887in}{3.473511in}}%
\pgfpathlineto{\pgfqpoint{3.619619in}{3.484734in}}%
\pgfpathlineto{\pgfqpoint{3.613352in}{3.495764in}}%
\pgfpathlineto{\pgfqpoint{3.607084in}{3.506530in}}%
\pgfpathlineto{\pgfqpoint{3.600815in}{3.516960in}}%
\pgfpathlineto{\pgfqpoint{3.594544in}{3.526985in}}%
\pgfpathlineto{\pgfqpoint{3.583509in}{3.523752in}}%
\pgfpathlineto{\pgfqpoint{3.572467in}{3.520412in}}%
\pgfpathlineto{\pgfqpoint{3.561419in}{3.516985in}}%
\pgfpathlineto{\pgfqpoint{3.550364in}{3.513493in}}%
\pgfpathlineto{\pgfqpoint{3.539305in}{3.509958in}}%
\pgfpathlineto{\pgfqpoint{3.545572in}{3.500255in}}%
\pgfpathlineto{\pgfqpoint{3.551837in}{3.490075in}}%
\pgfpathlineto{\pgfqpoint{3.558100in}{3.479496in}}%
\pgfpathlineto{\pgfqpoint{3.564363in}{3.468596in}}%
\pgfpathclose%
\pgfusepath{stroke,fill}%
\end{pgfscope}%
\begin{pgfscope}%
\pgfpathrectangle{\pgfqpoint{0.887500in}{0.275000in}}{\pgfqpoint{4.225000in}{4.225000in}}%
\pgfusepath{clip}%
\pgfsetbuttcap%
\pgfsetroundjoin%
\definecolor{currentfill}{rgb}{0.720391,0.870350,0.162603}%
\pgfsetfillcolor{currentfill}%
\pgfsetfillopacity{0.700000}%
\pgfsetlinewidth{0.501875pt}%
\definecolor{currentstroke}{rgb}{1.000000,1.000000,1.000000}%
\pgfsetstrokecolor{currentstroke}%
\pgfsetstrokeopacity{0.500000}%
\pgfsetdash{}{0pt}%
\pgfpathmoveto{\pgfqpoint{3.657268in}{3.416962in}}%
\pgfpathlineto{\pgfqpoint{3.668308in}{3.420034in}}%
\pgfpathlineto{\pgfqpoint{3.679342in}{3.423026in}}%
\pgfpathlineto{\pgfqpoint{3.690369in}{3.425958in}}%
\pgfpathlineto{\pgfqpoint{3.701391in}{3.428880in}}%
\pgfpathlineto{\pgfqpoint{3.712408in}{3.431844in}}%
\pgfpathlineto{\pgfqpoint{3.706114in}{3.442782in}}%
\pgfpathlineto{\pgfqpoint{3.699826in}{3.453868in}}%
\pgfpathlineto{\pgfqpoint{3.693542in}{3.465051in}}%
\pgfpathlineto{\pgfqpoint{3.687262in}{3.476277in}}%
\pgfpathlineto{\pgfqpoint{3.680985in}{3.487494in}}%
\pgfpathlineto{\pgfqpoint{3.669977in}{3.484732in}}%
\pgfpathlineto{\pgfqpoint{3.658964in}{3.482026in}}%
\pgfpathlineto{\pgfqpoint{3.647946in}{3.479298in}}%
\pgfpathlineto{\pgfqpoint{3.636920in}{3.476468in}}%
\pgfpathlineto{\pgfqpoint{3.625887in}{3.473511in}}%
\pgfpathlineto{\pgfqpoint{3.632157in}{3.462164in}}%
\pgfpathlineto{\pgfqpoint{3.638428in}{3.450766in}}%
\pgfpathlineto{\pgfqpoint{3.644704in}{3.439387in}}%
\pgfpathlineto{\pgfqpoint{3.650983in}{3.428097in}}%
\pgfpathclose%
\pgfusepath{stroke,fill}%
\end{pgfscope}%
\begin{pgfscope}%
\pgfpathrectangle{\pgfqpoint{0.887500in}{0.275000in}}{\pgfqpoint{4.225000in}{4.225000in}}%
\pgfusepath{clip}%
\pgfsetbuttcap%
\pgfsetroundjoin%
\definecolor{currentfill}{rgb}{0.626579,0.854645,0.223353}%
\pgfsetfillcolor{currentfill}%
\pgfsetfillopacity{0.700000}%
\pgfsetlinewidth{0.501875pt}%
\definecolor{currentstroke}{rgb}{1.000000,1.000000,1.000000}%
\pgfsetstrokecolor{currentstroke}%
\pgfsetstrokeopacity{0.500000}%
\pgfsetdash{}{0pt}%
\pgfpathmoveto{\pgfqpoint{2.858151in}{3.270921in}}%
\pgfpathlineto{\pgfqpoint{2.869155in}{3.319072in}}%
\pgfpathlineto{\pgfqpoint{2.880206in}{3.361060in}}%
\pgfpathlineto{\pgfqpoint{2.891305in}{3.395457in}}%
\pgfpathlineto{\pgfqpoint{2.902441in}{3.422933in}}%
\pgfpathlineto{\pgfqpoint{2.913607in}{3.444346in}}%
\pgfpathlineto{\pgfqpoint{2.907528in}{3.448473in}}%
\pgfpathlineto{\pgfqpoint{2.901454in}{3.452748in}}%
\pgfpathlineto{\pgfqpoint{2.895384in}{3.457192in}}%
\pgfpathlineto{\pgfqpoint{2.889319in}{3.461660in}}%
\pgfpathlineto{\pgfqpoint{2.883260in}{3.465950in}}%
\pgfpathlineto{\pgfqpoint{2.872110in}{3.447488in}}%
\pgfpathlineto{\pgfqpoint{2.860988in}{3.423970in}}%
\pgfpathlineto{\pgfqpoint{2.849901in}{3.394733in}}%
\pgfpathlineto{\pgfqpoint{2.838856in}{3.359250in}}%
\pgfpathlineto{\pgfqpoint{2.827854in}{3.318601in}}%
\pgfpathlineto{\pgfqpoint{2.833920in}{3.306649in}}%
\pgfpathlineto{\pgfqpoint{2.839984in}{3.295537in}}%
\pgfpathlineto{\pgfqpoint{2.846044in}{3.285736in}}%
\pgfpathlineto{\pgfqpoint{2.852099in}{3.277567in}}%
\pgfpathclose%
\pgfusepath{stroke,fill}%
\end{pgfscope}%
\begin{pgfscope}%
\pgfpathrectangle{\pgfqpoint{0.887500in}{0.275000in}}{\pgfqpoint{4.225000in}{4.225000in}}%
\pgfusepath{clip}%
\pgfsetbuttcap%
\pgfsetroundjoin%
\definecolor{currentfill}{rgb}{0.668054,0.861999,0.196293}%
\pgfsetfillcolor{currentfill}%
\pgfsetfillopacity{0.700000}%
\pgfsetlinewidth{0.501875pt}%
\definecolor{currentstroke}{rgb}{1.000000,1.000000,1.000000}%
\pgfsetstrokecolor{currentstroke}%
\pgfsetstrokeopacity{0.500000}%
\pgfsetdash{}{0pt}%
\pgfpathmoveto{\pgfqpoint{3.743951in}{3.378538in}}%
\pgfpathlineto{\pgfqpoint{3.754971in}{3.381651in}}%
\pgfpathlineto{\pgfqpoint{3.765987in}{3.384794in}}%
\pgfpathlineto{\pgfqpoint{3.776998in}{3.387987in}}%
\pgfpathlineto{\pgfqpoint{3.788006in}{3.391245in}}%
\pgfpathlineto{\pgfqpoint{3.799009in}{3.394573in}}%
\pgfpathlineto{\pgfqpoint{3.792691in}{3.405436in}}%
\pgfpathlineto{\pgfqpoint{3.786376in}{3.416293in}}%
\pgfpathlineto{\pgfqpoint{3.780064in}{3.427167in}}%
\pgfpathlineto{\pgfqpoint{3.773757in}{3.438080in}}%
\pgfpathlineto{\pgfqpoint{3.767454in}{3.449056in}}%
\pgfpathlineto{\pgfqpoint{3.756447in}{3.445161in}}%
\pgfpathlineto{\pgfqpoint{3.745440in}{3.441510in}}%
\pgfpathlineto{\pgfqpoint{3.734431in}{3.438107in}}%
\pgfpathlineto{\pgfqpoint{3.723421in}{3.434902in}}%
\pgfpathlineto{\pgfqpoint{3.712408in}{3.431844in}}%
\pgfpathlineto{\pgfqpoint{3.718708in}{3.421048in}}%
\pgfpathlineto{\pgfqpoint{3.725012in}{3.410357in}}%
\pgfpathlineto{\pgfqpoint{3.731321in}{3.399734in}}%
\pgfpathlineto{\pgfqpoint{3.737634in}{3.389140in}}%
\pgfpathclose%
\pgfusepath{stroke,fill}%
\end{pgfscope}%
\begin{pgfscope}%
\pgfpathrectangle{\pgfqpoint{0.887500in}{0.275000in}}{\pgfqpoint{4.225000in}{4.225000in}}%
\pgfusepath{clip}%
\pgfsetbuttcap%
\pgfsetroundjoin%
\definecolor{currentfill}{rgb}{0.121148,0.592739,0.544641}%
\pgfsetfillcolor{currentfill}%
\pgfsetfillopacity{0.700000}%
\pgfsetlinewidth{0.501875pt}%
\definecolor{currentstroke}{rgb}{1.000000,1.000000,1.000000}%
\pgfsetstrokecolor{currentstroke}%
\pgfsetstrokeopacity{0.500000}%
\pgfsetdash{}{0pt}%
\pgfpathmoveto{\pgfqpoint{1.926275in}{2.705192in}}%
\pgfpathlineto{\pgfqpoint{1.937736in}{2.708539in}}%
\pgfpathlineto{\pgfqpoint{1.949191in}{2.711891in}}%
\pgfpathlineto{\pgfqpoint{1.960641in}{2.715244in}}%
\pgfpathlineto{\pgfqpoint{1.972085in}{2.718593in}}%
\pgfpathlineto{\pgfqpoint{1.983524in}{2.721931in}}%
\pgfpathlineto{\pgfqpoint{1.977755in}{2.729784in}}%
\pgfpathlineto{\pgfqpoint{1.971991in}{2.737621in}}%
\pgfpathlineto{\pgfqpoint{1.966231in}{2.745444in}}%
\pgfpathlineto{\pgfqpoint{1.960475in}{2.753252in}}%
\pgfpathlineto{\pgfqpoint{1.954723in}{2.761046in}}%
\pgfpathlineto{\pgfqpoint{1.943297in}{2.757682in}}%
\pgfpathlineto{\pgfqpoint{1.931866in}{2.754311in}}%
\pgfpathlineto{\pgfqpoint{1.920430in}{2.750938in}}%
\pgfpathlineto{\pgfqpoint{1.908988in}{2.747570in}}%
\pgfpathlineto{\pgfqpoint{1.897540in}{2.744210in}}%
\pgfpathlineto{\pgfqpoint{1.903279in}{2.736437in}}%
\pgfpathlineto{\pgfqpoint{1.909022in}{2.728650in}}%
\pgfpathlineto{\pgfqpoint{1.914768in}{2.720847in}}%
\pgfpathlineto{\pgfqpoint{1.920519in}{2.713028in}}%
\pgfpathclose%
\pgfusepath{stroke,fill}%
\end{pgfscope}%
\begin{pgfscope}%
\pgfpathrectangle{\pgfqpoint{0.887500in}{0.275000in}}{\pgfqpoint{4.225000in}{4.225000in}}%
\pgfusepath{clip}%
\pgfsetbuttcap%
\pgfsetroundjoin%
\definecolor{currentfill}{rgb}{0.120638,0.625828,0.533488}%
\pgfsetfillcolor{currentfill}%
\pgfsetfillopacity{0.700000}%
\pgfsetlinewidth{0.501875pt}%
\definecolor{currentstroke}{rgb}{1.000000,1.000000,1.000000}%
\pgfsetstrokecolor{currentstroke}%
\pgfsetstrokeopacity{0.500000}%
\pgfsetdash{}{0pt}%
\pgfpathmoveto{\pgfqpoint{1.381302in}{2.779698in}}%
\pgfpathlineto{\pgfqpoint{1.392891in}{2.783064in}}%
\pgfpathlineto{\pgfqpoint{1.404475in}{2.786424in}}%
\pgfpathlineto{\pgfqpoint{1.416053in}{2.789780in}}%
\pgfpathlineto{\pgfqpoint{1.427626in}{2.793130in}}%
\pgfpathlineto{\pgfqpoint{1.439193in}{2.796477in}}%
\pgfpathlineto{\pgfqpoint{1.433628in}{2.803824in}}%
\pgfpathlineto{\pgfqpoint{1.428067in}{2.811152in}}%
\pgfpathlineto{\pgfqpoint{1.422510in}{2.818460in}}%
\pgfpathlineto{\pgfqpoint{1.416958in}{2.825749in}}%
\pgfpathlineto{\pgfqpoint{1.405403in}{2.822374in}}%
\pgfpathlineto{\pgfqpoint{1.393843in}{2.818992in}}%
\pgfpathlineto{\pgfqpoint{1.382277in}{2.815604in}}%
\pgfpathlineto{\pgfqpoint{1.370706in}{2.812208in}}%
\pgfpathlineto{\pgfqpoint{1.359130in}{2.808805in}}%
\pgfpathlineto{\pgfqpoint{1.364666in}{2.801563in}}%
\pgfpathlineto{\pgfqpoint{1.370207in}{2.794298in}}%
\pgfpathlineto{\pgfqpoint{1.375752in}{2.787009in}}%
\pgfpathclose%
\pgfusepath{stroke,fill}%
\end{pgfscope}%
\begin{pgfscope}%
\pgfpathrectangle{\pgfqpoint{0.887500in}{0.275000in}}{\pgfqpoint{4.225000in}{4.225000in}}%
\pgfusepath{clip}%
\pgfsetbuttcap%
\pgfsetroundjoin%
\definecolor{currentfill}{rgb}{0.506271,0.828786,0.300362}%
\pgfsetfillcolor{currentfill}%
\pgfsetfillopacity{0.700000}%
\pgfsetlinewidth{0.501875pt}%
\definecolor{currentstroke}{rgb}{1.000000,1.000000,1.000000}%
\pgfsetstrokecolor{currentstroke}%
\pgfsetstrokeopacity{0.500000}%
\pgfsetdash{}{0pt}%
\pgfpathmoveto{\pgfqpoint{4.003846in}{3.253165in}}%
\pgfpathlineto{\pgfqpoint{4.014798in}{3.256075in}}%
\pgfpathlineto{\pgfqpoint{4.025744in}{3.258949in}}%
\pgfpathlineto{\pgfqpoint{4.036683in}{3.261785in}}%
\pgfpathlineto{\pgfqpoint{4.047616in}{3.264584in}}%
\pgfpathlineto{\pgfqpoint{4.058543in}{3.267344in}}%
\pgfpathlineto{\pgfqpoint{4.052188in}{3.279431in}}%
\pgfpathlineto{\pgfqpoint{4.045837in}{3.291526in}}%
\pgfpathlineto{\pgfqpoint{4.039488in}{3.303634in}}%
\pgfpathlineto{\pgfqpoint{4.033143in}{3.315758in}}%
\pgfpathlineto{\pgfqpoint{4.026802in}{3.327902in}}%
\pgfpathlineto{\pgfqpoint{4.015867in}{3.324622in}}%
\pgfpathlineto{\pgfqpoint{4.004927in}{3.321369in}}%
\pgfpathlineto{\pgfqpoint{3.993983in}{3.318145in}}%
\pgfpathlineto{\pgfqpoint{3.983034in}{3.314951in}}%
\pgfpathlineto{\pgfqpoint{3.972081in}{3.311788in}}%
\pgfpathlineto{\pgfqpoint{3.978430in}{3.300141in}}%
\pgfpathlineto{\pgfqpoint{3.984781in}{3.288475in}}%
\pgfpathlineto{\pgfqpoint{3.991135in}{3.276771in}}%
\pgfpathlineto{\pgfqpoint{3.997490in}{3.265007in}}%
\pgfpathclose%
\pgfusepath{stroke,fill}%
\end{pgfscope}%
\begin{pgfscope}%
\pgfpathrectangle{\pgfqpoint{0.887500in}{0.275000in}}{\pgfqpoint{4.225000in}{4.225000in}}%
\pgfusepath{clip}%
\pgfsetbuttcap%
\pgfsetroundjoin%
\definecolor{currentfill}{rgb}{0.449368,0.813768,0.335384}%
\pgfsetfillcolor{currentfill}%
\pgfsetfillopacity{0.700000}%
\pgfsetlinewidth{0.501875pt}%
\definecolor{currentstroke}{rgb}{1.000000,1.000000,1.000000}%
\pgfsetstrokecolor{currentstroke}%
\pgfsetstrokeopacity{0.500000}%
\pgfsetdash{}{0pt}%
\pgfpathmoveto{\pgfqpoint{4.090362in}{3.206890in}}%
\pgfpathlineto{\pgfqpoint{4.101288in}{3.209665in}}%
\pgfpathlineto{\pgfqpoint{4.112207in}{3.212411in}}%
\pgfpathlineto{\pgfqpoint{4.123120in}{3.215127in}}%
\pgfpathlineto{\pgfqpoint{4.134027in}{3.217811in}}%
\pgfpathlineto{\pgfqpoint{4.127648in}{3.229661in}}%
\pgfpathlineto{\pgfqpoint{4.121273in}{3.241579in}}%
\pgfpathlineto{\pgfqpoint{4.114904in}{3.253589in}}%
\pgfpathlineto{\pgfqpoint{4.108541in}{3.265720in}}%
\pgfpathlineto{\pgfqpoint{4.102185in}{3.277997in}}%
\pgfpathlineto{\pgfqpoint{4.091284in}{3.275393in}}%
\pgfpathlineto{\pgfqpoint{4.080377in}{3.272749in}}%
\pgfpathlineto{\pgfqpoint{4.069463in}{3.270066in}}%
\pgfpathlineto{\pgfqpoint{4.058543in}{3.267344in}}%
\pgfpathlineto{\pgfqpoint{4.064901in}{3.255262in}}%
\pgfpathlineto{\pgfqpoint{4.071262in}{3.243181in}}%
\pgfpathlineto{\pgfqpoint{4.077626in}{3.231095in}}%
\pgfpathlineto{\pgfqpoint{4.083993in}{3.218999in}}%
\pgfpathclose%
\pgfusepath{stroke,fill}%
\end{pgfscope}%
\begin{pgfscope}%
\pgfpathrectangle{\pgfqpoint{0.887500in}{0.275000in}}{\pgfqpoint{4.225000in}{4.225000in}}%
\pgfusepath{clip}%
\pgfsetbuttcap%
\pgfsetroundjoin%
\definecolor{currentfill}{rgb}{0.565498,0.842430,0.262877}%
\pgfsetfillcolor{currentfill}%
\pgfsetfillopacity{0.700000}%
\pgfsetlinewidth{0.501875pt}%
\definecolor{currentstroke}{rgb}{1.000000,1.000000,1.000000}%
\pgfsetstrokecolor{currentstroke}%
\pgfsetstrokeopacity{0.500000}%
\pgfsetdash{}{0pt}%
\pgfpathmoveto{\pgfqpoint{3.917247in}{3.296469in}}%
\pgfpathlineto{\pgfqpoint{3.928223in}{3.299463in}}%
\pgfpathlineto{\pgfqpoint{3.939194in}{3.302493in}}%
\pgfpathlineto{\pgfqpoint{3.950161in}{3.305557in}}%
\pgfpathlineto{\pgfqpoint{3.961123in}{3.308656in}}%
\pgfpathlineto{\pgfqpoint{3.972081in}{3.311788in}}%
\pgfpathlineto{\pgfqpoint{3.965736in}{3.323435in}}%
\pgfpathlineto{\pgfqpoint{3.959394in}{3.335102in}}%
\pgfpathlineto{\pgfqpoint{3.953056in}{3.346810in}}%
\pgfpathlineto{\pgfqpoint{3.946722in}{3.358578in}}%
\pgfpathlineto{\pgfqpoint{3.940394in}{3.370425in}}%
\pgfpathlineto{\pgfqpoint{3.929434in}{3.366968in}}%
\pgfpathlineto{\pgfqpoint{3.918471in}{3.363601in}}%
\pgfpathlineto{\pgfqpoint{3.907505in}{3.360326in}}%
\pgfpathlineto{\pgfqpoint{3.896535in}{3.357144in}}%
\pgfpathlineto{\pgfqpoint{3.885562in}{3.354054in}}%
\pgfpathlineto{\pgfqpoint{3.891893in}{3.342525in}}%
\pgfpathlineto{\pgfqpoint{3.898227in}{3.331014in}}%
\pgfpathlineto{\pgfqpoint{3.904564in}{3.319510in}}%
\pgfpathlineto{\pgfqpoint{3.910904in}{3.308000in}}%
\pgfpathclose%
\pgfusepath{stroke,fill}%
\end{pgfscope}%
\begin{pgfscope}%
\pgfpathrectangle{\pgfqpoint{0.887500in}{0.275000in}}{\pgfqpoint{4.225000in}{4.225000in}}%
\pgfusepath{clip}%
\pgfsetbuttcap%
\pgfsetroundjoin%
\definecolor{currentfill}{rgb}{0.616293,0.852709,0.230052}%
\pgfsetfillcolor{currentfill}%
\pgfsetfillopacity{0.700000}%
\pgfsetlinewidth{0.501875pt}%
\definecolor{currentstroke}{rgb}{1.000000,1.000000,1.000000}%
\pgfsetstrokecolor{currentstroke}%
\pgfsetstrokeopacity{0.500000}%
\pgfsetdash{}{0pt}%
\pgfpathmoveto{\pgfqpoint{3.830634in}{3.339404in}}%
\pgfpathlineto{\pgfqpoint{3.841630in}{3.342308in}}%
\pgfpathlineto{\pgfqpoint{3.852620in}{3.345200in}}%
\pgfpathlineto{\pgfqpoint{3.863605in}{3.348106in}}%
\pgfpathlineto{\pgfqpoint{3.874586in}{3.351049in}}%
\pgfpathlineto{\pgfqpoint{3.885562in}{3.354054in}}%
\pgfpathlineto{\pgfqpoint{3.879236in}{3.365614in}}%
\pgfpathlineto{\pgfqpoint{3.872913in}{3.377204in}}%
\pgfpathlineto{\pgfqpoint{3.866594in}{3.388822in}}%
\pgfpathlineto{\pgfqpoint{3.860278in}{3.400462in}}%
\pgfpathlineto{\pgfqpoint{3.853966in}{3.412120in}}%
\pgfpathlineto{\pgfqpoint{3.842983in}{3.408505in}}%
\pgfpathlineto{\pgfqpoint{3.831995in}{3.404938in}}%
\pgfpathlineto{\pgfqpoint{3.821004in}{3.401424in}}%
\pgfpathlineto{\pgfqpoint{3.810008in}{3.397968in}}%
\pgfpathlineto{\pgfqpoint{3.799009in}{3.394573in}}%
\pgfpathlineto{\pgfqpoint{3.805330in}{3.383683in}}%
\pgfpathlineto{\pgfqpoint{3.811653in}{3.372744in}}%
\pgfpathlineto{\pgfqpoint{3.817979in}{3.361732in}}%
\pgfpathlineto{\pgfqpoint{3.824306in}{3.350627in}}%
\pgfpathclose%
\pgfusepath{stroke,fill}%
\end{pgfscope}%
\begin{pgfscope}%
\pgfpathrectangle{\pgfqpoint{0.887500in}{0.275000in}}{\pgfqpoint{4.225000in}{4.225000in}}%
\pgfusepath{clip}%
\pgfsetbuttcap%
\pgfsetroundjoin%
\definecolor{currentfill}{rgb}{0.129933,0.559582,0.551864}%
\pgfsetfillcolor{currentfill}%
\pgfsetfillopacity{0.700000}%
\pgfsetlinewidth{0.501875pt}%
\definecolor{currentstroke}{rgb}{1.000000,1.000000,1.000000}%
\pgfsetstrokecolor{currentstroke}%
\pgfsetstrokeopacity{0.500000}%
\pgfsetdash{}{0pt}%
\pgfpathmoveto{\pgfqpoint{2.471542in}{2.619593in}}%
\pgfpathlineto{\pgfqpoint{2.482858in}{2.623759in}}%
\pgfpathlineto{\pgfqpoint{2.494161in}{2.628544in}}%
\pgfpathlineto{\pgfqpoint{2.505449in}{2.634047in}}%
\pgfpathlineto{\pgfqpoint{2.516722in}{2.640367in}}%
\pgfpathlineto{\pgfqpoint{2.527979in}{2.647602in}}%
\pgfpathlineto{\pgfqpoint{2.522003in}{2.657199in}}%
\pgfpathlineto{\pgfqpoint{2.516024in}{2.667319in}}%
\pgfpathlineto{\pgfqpoint{2.510042in}{2.677968in}}%
\pgfpathlineto{\pgfqpoint{2.504055in}{2.689067in}}%
\pgfpathlineto{\pgfqpoint{2.498067in}{2.700531in}}%
\pgfpathlineto{\pgfqpoint{2.486834in}{2.692416in}}%
\pgfpathlineto{\pgfqpoint{2.475586in}{2.685157in}}%
\pgfpathlineto{\pgfqpoint{2.464323in}{2.678645in}}%
\pgfpathlineto{\pgfqpoint{2.453047in}{2.672769in}}%
\pgfpathlineto{\pgfqpoint{2.441759in}{2.667420in}}%
\pgfpathlineto{\pgfqpoint{2.447717in}{2.657296in}}%
\pgfpathlineto{\pgfqpoint{2.453675in}{2.647413in}}%
\pgfpathlineto{\pgfqpoint{2.459632in}{2.637815in}}%
\pgfpathlineto{\pgfqpoint{2.465587in}{2.628545in}}%
\pgfpathclose%
\pgfusepath{stroke,fill}%
\end{pgfscope}%
\begin{pgfscope}%
\pgfpathrectangle{\pgfqpoint{0.887500in}{0.275000in}}{\pgfqpoint{4.225000in}{4.225000in}}%
\pgfusepath{clip}%
\pgfsetbuttcap%
\pgfsetroundjoin%
\definecolor{currentfill}{rgb}{0.814576,0.883393,0.110347}%
\pgfsetfillcolor{currentfill}%
\pgfsetfillopacity{0.700000}%
\pgfsetlinewidth{0.501875pt}%
\definecolor{currentstroke}{rgb}{1.000000,1.000000,1.000000}%
\pgfsetstrokecolor{currentstroke}%
\pgfsetstrokeopacity{0.500000}%
\pgfsetdash{}{0pt}%
\pgfpathmoveto{\pgfqpoint{3.341576in}{3.496250in}}%
\pgfpathlineto{\pgfqpoint{3.352701in}{3.499824in}}%
\pgfpathlineto{\pgfqpoint{3.363821in}{3.503214in}}%
\pgfpathlineto{\pgfqpoint{3.374934in}{3.506463in}}%
\pgfpathlineto{\pgfqpoint{3.386041in}{3.509629in}}%
\pgfpathlineto{\pgfqpoint{3.397142in}{3.512772in}}%
\pgfpathlineto{\pgfqpoint{3.390889in}{3.519255in}}%
\pgfpathlineto{\pgfqpoint{3.384637in}{3.525436in}}%
\pgfpathlineto{\pgfqpoint{3.378388in}{3.531375in}}%
\pgfpathlineto{\pgfqpoint{3.372143in}{3.537135in}}%
\pgfpathlineto{\pgfqpoint{3.365901in}{3.542775in}}%
\pgfpathlineto{\pgfqpoint{3.354818in}{3.540249in}}%
\pgfpathlineto{\pgfqpoint{3.343728in}{3.537749in}}%
\pgfpathlineto{\pgfqpoint{3.332633in}{3.535168in}}%
\pgfpathlineto{\pgfqpoint{3.321531in}{3.532401in}}%
\pgfpathlineto{\pgfqpoint{3.310421in}{3.529366in}}%
\pgfpathlineto{\pgfqpoint{3.316647in}{3.523294in}}%
\pgfpathlineto{\pgfqpoint{3.322875in}{3.516970in}}%
\pgfpathlineto{\pgfqpoint{3.329106in}{3.510370in}}%
\pgfpathlineto{\pgfqpoint{3.335340in}{3.503471in}}%
\pgfpathclose%
\pgfusepath{stroke,fill}%
\end{pgfscope}%
\begin{pgfscope}%
\pgfpathrectangle{\pgfqpoint{0.887500in}{0.275000in}}{\pgfqpoint{4.225000in}{4.225000in}}%
\pgfusepath{clip}%
\pgfsetbuttcap%
\pgfsetroundjoin%
\definecolor{currentfill}{rgb}{0.126453,0.570633,0.549841}%
\pgfsetfillcolor{currentfill}%
\pgfsetfillopacity{0.700000}%
\pgfsetlinewidth{0.501875pt}%
\definecolor{currentstroke}{rgb}{1.000000,1.000000,1.000000}%
\pgfsetstrokecolor{currentstroke}%
\pgfsetstrokeopacity{0.500000}%
\pgfsetdash{}{0pt}%
\pgfpathmoveto{\pgfqpoint{2.242025in}{2.652613in}}%
\pgfpathlineto{\pgfqpoint{2.253413in}{2.655923in}}%
\pgfpathlineto{\pgfqpoint{2.264799in}{2.659108in}}%
\pgfpathlineto{\pgfqpoint{2.276180in}{2.662177in}}%
\pgfpathlineto{\pgfqpoint{2.287557in}{2.665224in}}%
\pgfpathlineto{\pgfqpoint{2.298927in}{2.668351in}}%
\pgfpathlineto{\pgfqpoint{2.293050in}{2.676284in}}%
\pgfpathlineto{\pgfqpoint{2.287177in}{2.684225in}}%
\pgfpathlineto{\pgfqpoint{2.281308in}{2.692178in}}%
\pgfpathlineto{\pgfqpoint{2.275442in}{2.700138in}}%
\pgfpathlineto{\pgfqpoint{2.269580in}{2.708097in}}%
\pgfpathlineto{\pgfqpoint{2.258219in}{2.705170in}}%
\pgfpathlineto{\pgfqpoint{2.246848in}{2.702397in}}%
\pgfpathlineto{\pgfqpoint{2.235473in}{2.699589in}}%
\pgfpathlineto{\pgfqpoint{2.224096in}{2.696577in}}%
\pgfpathlineto{\pgfqpoint{2.212718in}{2.693338in}}%
\pgfpathlineto{\pgfqpoint{2.218571in}{2.685222in}}%
\pgfpathlineto{\pgfqpoint{2.224429in}{2.677091in}}%
\pgfpathlineto{\pgfqpoint{2.230290in}{2.668945in}}%
\pgfpathlineto{\pgfqpoint{2.236156in}{2.660785in}}%
\pgfpathclose%
\pgfusepath{stroke,fill}%
\end{pgfscope}%
\begin{pgfscope}%
\pgfpathrectangle{\pgfqpoint{0.887500in}{0.275000in}}{\pgfqpoint{4.225000in}{4.225000in}}%
\pgfusepath{clip}%
\pgfsetbuttcap%
\pgfsetroundjoin%
\definecolor{currentfill}{rgb}{0.119512,0.607464,0.540218}%
\pgfsetfillcolor{currentfill}%
\pgfsetfillopacity{0.700000}%
\pgfsetlinewidth{0.501875pt}%
\definecolor{currentstroke}{rgb}{1.000000,1.000000,1.000000}%
\pgfsetstrokecolor{currentstroke}%
\pgfsetstrokeopacity{0.500000}%
\pgfsetdash{}{0pt}%
\pgfpathmoveto{\pgfqpoint{1.696693in}{2.733374in}}%
\pgfpathlineto{\pgfqpoint{1.708214in}{2.736644in}}%
\pgfpathlineto{\pgfqpoint{1.719730in}{2.739913in}}%
\pgfpathlineto{\pgfqpoint{1.731240in}{2.743181in}}%
\pgfpathlineto{\pgfqpoint{1.742744in}{2.746450in}}%
\pgfpathlineto{\pgfqpoint{1.754243in}{2.749721in}}%
\pgfpathlineto{\pgfqpoint{1.748556in}{2.757385in}}%
\pgfpathlineto{\pgfqpoint{1.742874in}{2.765034in}}%
\pgfpathlineto{\pgfqpoint{1.737195in}{2.772668in}}%
\pgfpathlineto{\pgfqpoint{1.731521in}{2.780287in}}%
\pgfpathlineto{\pgfqpoint{1.725851in}{2.787892in}}%
\pgfpathlineto{\pgfqpoint{1.714366in}{2.784606in}}%
\pgfpathlineto{\pgfqpoint{1.702875in}{2.781322in}}%
\pgfpathlineto{\pgfqpoint{1.691379in}{2.778041in}}%
\pgfpathlineto{\pgfqpoint{1.679877in}{2.774760in}}%
\pgfpathlineto{\pgfqpoint{1.668369in}{2.771479in}}%
\pgfpathlineto{\pgfqpoint{1.674026in}{2.763889in}}%
\pgfpathlineto{\pgfqpoint{1.679686in}{2.756283in}}%
\pgfpathlineto{\pgfqpoint{1.685351in}{2.748662in}}%
\pgfpathlineto{\pgfqpoint{1.691020in}{2.741026in}}%
\pgfpathclose%
\pgfusepath{stroke,fill}%
\end{pgfscope}%
\begin{pgfscope}%
\pgfpathrectangle{\pgfqpoint{0.887500in}{0.275000in}}{\pgfqpoint{4.225000in}{4.225000in}}%
\pgfusepath{clip}%
\pgfsetbuttcap%
\pgfsetroundjoin%
\definecolor{currentfill}{rgb}{0.120081,0.622161,0.534946}%
\pgfsetfillcolor{currentfill}%
\pgfsetfillopacity{0.700000}%
\pgfsetlinewidth{0.501875pt}%
\definecolor{currentstroke}{rgb}{1.000000,1.000000,1.000000}%
\pgfsetstrokecolor{currentstroke}%
\pgfsetstrokeopacity{0.500000}%
\pgfsetdash{}{0pt}%
\pgfpathmoveto{\pgfqpoint{1.467087in}{2.759462in}}%
\pgfpathlineto{\pgfqpoint{1.478664in}{2.762783in}}%
\pgfpathlineto{\pgfqpoint{1.490235in}{2.766103in}}%
\pgfpathlineto{\pgfqpoint{1.501800in}{2.769424in}}%
\pgfpathlineto{\pgfqpoint{1.513360in}{2.772746in}}%
\pgfpathlineto{\pgfqpoint{1.524914in}{2.776069in}}%
\pgfpathlineto{\pgfqpoint{1.519312in}{2.783526in}}%
\pgfpathlineto{\pgfqpoint{1.513714in}{2.790965in}}%
\pgfpathlineto{\pgfqpoint{1.508121in}{2.798387in}}%
\pgfpathlineto{\pgfqpoint{1.502532in}{2.805790in}}%
\pgfpathlineto{\pgfqpoint{1.496947in}{2.813175in}}%
\pgfpathlineto{\pgfqpoint{1.485408in}{2.809838in}}%
\pgfpathlineto{\pgfqpoint{1.473862in}{2.806500in}}%
\pgfpathlineto{\pgfqpoint{1.462311in}{2.803162in}}%
\pgfpathlineto{\pgfqpoint{1.450755in}{2.799821in}}%
\pgfpathlineto{\pgfqpoint{1.439193in}{2.796477in}}%
\pgfpathlineto{\pgfqpoint{1.444763in}{2.789111in}}%
\pgfpathlineto{\pgfqpoint{1.450337in}{2.781727in}}%
\pgfpathlineto{\pgfqpoint{1.455916in}{2.774323in}}%
\pgfpathlineto{\pgfqpoint{1.461499in}{2.766902in}}%
\pgfpathclose%
\pgfusepath{stroke,fill}%
\end{pgfscope}%
\begin{pgfscope}%
\pgfpathrectangle{\pgfqpoint{0.887500in}{0.275000in}}{\pgfqpoint{4.225000in}{4.225000in}}%
\pgfusepath{clip}%
\pgfsetbuttcap%
\pgfsetroundjoin%
\definecolor{currentfill}{rgb}{0.122606,0.585371,0.546557}%
\pgfsetfillcolor{currentfill}%
\pgfsetfillopacity{0.700000}%
\pgfsetlinewidth{0.501875pt}%
\definecolor{currentstroke}{rgb}{1.000000,1.000000,1.000000}%
\pgfsetstrokecolor{currentstroke}%
\pgfsetstrokeopacity{0.500000}%
\pgfsetdash{}{0pt}%
\pgfpathmoveto{\pgfqpoint{2.012429in}{2.682418in}}%
\pgfpathlineto{\pgfqpoint{2.023876in}{2.685726in}}%
\pgfpathlineto{\pgfqpoint{2.035318in}{2.689016in}}%
\pgfpathlineto{\pgfqpoint{2.046754in}{2.692283in}}%
\pgfpathlineto{\pgfqpoint{2.058186in}{2.695525in}}%
\pgfpathlineto{\pgfqpoint{2.069613in}{2.698761in}}%
\pgfpathlineto{\pgfqpoint{2.063810in}{2.706703in}}%
\pgfpathlineto{\pgfqpoint{2.058013in}{2.714631in}}%
\pgfpathlineto{\pgfqpoint{2.052219in}{2.722546in}}%
\pgfpathlineto{\pgfqpoint{2.046429in}{2.730447in}}%
\pgfpathlineto{\pgfqpoint{2.040643in}{2.738335in}}%
\pgfpathlineto{\pgfqpoint{2.029230in}{2.735090in}}%
\pgfpathlineto{\pgfqpoint{2.017810in}{2.731837in}}%
\pgfpathlineto{\pgfqpoint{2.006386in}{2.728559in}}%
\pgfpathlineto{\pgfqpoint{1.994958in}{2.725255in}}%
\pgfpathlineto{\pgfqpoint{1.983524in}{2.721931in}}%
\pgfpathlineto{\pgfqpoint{1.989296in}{2.714063in}}%
\pgfpathlineto{\pgfqpoint{1.995073in}{2.706177in}}%
\pgfpathlineto{\pgfqpoint{2.000854in}{2.698274in}}%
\pgfpathlineto{\pgfqpoint{2.006640in}{2.690354in}}%
\pgfpathclose%
\pgfusepath{stroke,fill}%
\end{pgfscope}%
\begin{pgfscope}%
\pgfpathrectangle{\pgfqpoint{0.887500in}{0.275000in}}{\pgfqpoint{4.225000in}{4.225000in}}%
\pgfusepath{clip}%
\pgfsetbuttcap%
\pgfsetroundjoin%
\definecolor{currentfill}{rgb}{0.124780,0.640461,0.527068}%
\pgfsetfillcolor{currentfill}%
\pgfsetfillopacity{0.700000}%
\pgfsetlinewidth{0.501875pt}%
\definecolor{currentstroke}{rgb}{1.000000,1.000000,1.000000}%
\pgfsetstrokecolor{currentstroke}%
\pgfsetstrokeopacity{0.500000}%
\pgfsetdash{}{0pt}%
\pgfpathmoveto{\pgfqpoint{2.726065in}{2.767419in}}%
\pgfpathlineto{\pgfqpoint{2.737260in}{2.779483in}}%
\pgfpathlineto{\pgfqpoint{2.748445in}{2.792861in}}%
\pgfpathlineto{\pgfqpoint{2.759622in}{2.807434in}}%
\pgfpathlineto{\pgfqpoint{2.770792in}{2.822967in}}%
\pgfpathlineto{\pgfqpoint{2.781959in}{2.839226in}}%
\pgfpathlineto{\pgfqpoint{2.775902in}{2.848926in}}%
\pgfpathlineto{\pgfqpoint{2.769848in}{2.858724in}}%
\pgfpathlineto{\pgfqpoint{2.763798in}{2.868503in}}%
\pgfpathlineto{\pgfqpoint{2.757752in}{2.878147in}}%
\pgfpathlineto{\pgfqpoint{2.751712in}{2.887541in}}%
\pgfpathlineto{\pgfqpoint{2.740543in}{2.873906in}}%
\pgfpathlineto{\pgfqpoint{2.729376in}{2.860100in}}%
\pgfpathlineto{\pgfqpoint{2.718208in}{2.846467in}}%
\pgfpathlineto{\pgfqpoint{2.707036in}{2.833353in}}%
\pgfpathlineto{\pgfqpoint{2.695857in}{2.820987in}}%
\pgfpathlineto{\pgfqpoint{2.701894in}{2.810139in}}%
\pgfpathlineto{\pgfqpoint{2.707933in}{2.799303in}}%
\pgfpathlineto{\pgfqpoint{2.713975in}{2.788535in}}%
\pgfpathlineto{\pgfqpoint{2.720019in}{2.777889in}}%
\pgfpathclose%
\pgfusepath{stroke,fill}%
\end{pgfscope}%
\begin{pgfscope}%
\pgfpathrectangle{\pgfqpoint{0.887500in}{0.275000in}}{\pgfqpoint{4.225000in}{4.225000in}}%
\pgfusepath{clip}%
\pgfsetbuttcap%
\pgfsetroundjoin%
\definecolor{currentfill}{rgb}{0.824940,0.884720,0.106217}%
\pgfsetfillcolor{currentfill}%
\pgfsetfillopacity{0.700000}%
\pgfsetlinewidth{0.501875pt}%
\definecolor{currentstroke}{rgb}{1.000000,1.000000,1.000000}%
\pgfsetstrokecolor{currentstroke}%
\pgfsetstrokeopacity{0.500000}%
\pgfsetdash{}{0pt}%
\pgfpathmoveto{\pgfqpoint{3.199008in}{3.492958in}}%
\pgfpathlineto{\pgfqpoint{3.210172in}{3.496521in}}%
\pgfpathlineto{\pgfqpoint{3.221331in}{3.500164in}}%
\pgfpathlineto{\pgfqpoint{3.232485in}{3.503873in}}%
\pgfpathlineto{\pgfqpoint{3.243635in}{3.507636in}}%
\pgfpathlineto{\pgfqpoint{3.254779in}{3.511434in}}%
\pgfpathlineto{\pgfqpoint{3.248573in}{3.517668in}}%
\pgfpathlineto{\pgfqpoint{3.242370in}{3.523691in}}%
\pgfpathlineto{\pgfqpoint{3.236171in}{3.529505in}}%
\pgfpathlineto{\pgfqpoint{3.229976in}{3.535102in}}%
\pgfpathlineto{\pgfqpoint{3.223785in}{3.540470in}}%
\pgfpathlineto{\pgfqpoint{3.212656in}{3.537154in}}%
\pgfpathlineto{\pgfqpoint{3.201522in}{3.533859in}}%
\pgfpathlineto{\pgfqpoint{3.190383in}{3.530551in}}%
\pgfpathlineto{\pgfqpoint{3.179239in}{3.527190in}}%
\pgfpathlineto{\pgfqpoint{3.168089in}{3.523735in}}%
\pgfpathlineto{\pgfqpoint{3.174265in}{3.518027in}}%
\pgfpathlineto{\pgfqpoint{3.180445in}{3.512067in}}%
\pgfpathlineto{\pgfqpoint{3.186628in}{3.505886in}}%
\pgfpathlineto{\pgfqpoint{3.192816in}{3.499515in}}%
\pgfpathclose%
\pgfusepath{stroke,fill}%
\end{pgfscope}%
\begin{pgfscope}%
\pgfpathrectangle{\pgfqpoint{0.887500in}{0.275000in}}{\pgfqpoint{4.225000in}{4.225000in}}%
\pgfusepath{clip}%
\pgfsetbuttcap%
\pgfsetroundjoin%
\definecolor{currentfill}{rgb}{0.119483,0.614817,0.537692}%
\pgfsetfillcolor{currentfill}%
\pgfsetfillopacity{0.700000}%
\pgfsetlinewidth{0.501875pt}%
\definecolor{currentstroke}{rgb}{1.000000,1.000000,1.000000}%
\pgfsetstrokecolor{currentstroke}%
\pgfsetstrokeopacity{0.500000}%
\pgfsetdash{}{0pt}%
\pgfpathmoveto{\pgfqpoint{2.669993in}{2.715907in}}%
\pgfpathlineto{\pgfqpoint{2.681211in}{2.726180in}}%
\pgfpathlineto{\pgfqpoint{2.692430in}{2.736125in}}%
\pgfpathlineto{\pgfqpoint{2.703647in}{2.746083in}}%
\pgfpathlineto{\pgfqpoint{2.714859in}{2.756400in}}%
\pgfpathlineto{\pgfqpoint{2.726065in}{2.767419in}}%
\pgfpathlineto{\pgfqpoint{2.720019in}{2.777889in}}%
\pgfpathlineto{\pgfqpoint{2.713975in}{2.788535in}}%
\pgfpathlineto{\pgfqpoint{2.707933in}{2.799303in}}%
\pgfpathlineto{\pgfqpoint{2.701894in}{2.810139in}}%
\pgfpathlineto{\pgfqpoint{2.695857in}{2.820987in}}%
\pgfpathlineto{\pgfqpoint{2.684673in}{2.809221in}}%
\pgfpathlineto{\pgfqpoint{2.673484in}{2.797833in}}%
\pgfpathlineto{\pgfqpoint{2.662292in}{2.786597in}}%
\pgfpathlineto{\pgfqpoint{2.651101in}{2.775292in}}%
\pgfpathlineto{\pgfqpoint{2.639912in}{2.763695in}}%
\pgfpathlineto{\pgfqpoint{2.645928in}{2.753486in}}%
\pgfpathlineto{\pgfqpoint{2.651943in}{2.743663in}}%
\pgfpathlineto{\pgfqpoint{2.657959in}{2.734166in}}%
\pgfpathlineto{\pgfqpoint{2.663975in}{2.724934in}}%
\pgfpathclose%
\pgfusepath{stroke,fill}%
\end{pgfscope}%
\begin{pgfscope}%
\pgfpathrectangle{\pgfqpoint{0.887500in}{0.275000in}}{\pgfqpoint{4.225000in}{4.225000in}}%
\pgfusepath{clip}%
\pgfsetbuttcap%
\pgfsetroundjoin%
\definecolor{currentfill}{rgb}{0.128729,0.563265,0.551229}%
\pgfsetfillcolor{currentfill}%
\pgfsetfillopacity{0.700000}%
\pgfsetlinewidth{0.501875pt}%
\definecolor{currentstroke}{rgb}{1.000000,1.000000,1.000000}%
\pgfsetstrokecolor{currentstroke}%
\pgfsetstrokeopacity{0.500000}%
\pgfsetdash{}{0pt}%
\pgfpathmoveto{\pgfqpoint{2.328369in}{2.628608in}}%
\pgfpathlineto{\pgfqpoint{2.339739in}{2.632023in}}%
\pgfpathlineto{\pgfqpoint{2.351104in}{2.635466in}}%
\pgfpathlineto{\pgfqpoint{2.362463in}{2.638948in}}%
\pgfpathlineto{\pgfqpoint{2.373815in}{2.642478in}}%
\pgfpathlineto{\pgfqpoint{2.385161in}{2.646087in}}%
\pgfpathlineto{\pgfqpoint{2.379243in}{2.654574in}}%
\pgfpathlineto{\pgfqpoint{2.373329in}{2.663082in}}%
\pgfpathlineto{\pgfqpoint{2.367418in}{2.671610in}}%
\pgfpathlineto{\pgfqpoint{2.361511in}{2.680152in}}%
\pgfpathlineto{\pgfqpoint{2.355607in}{2.688706in}}%
\pgfpathlineto{\pgfqpoint{2.344298in}{2.683707in}}%
\pgfpathlineto{\pgfqpoint{2.332974in}{2.679237in}}%
\pgfpathlineto{\pgfqpoint{2.321637in}{2.675255in}}%
\pgfpathlineto{\pgfqpoint{2.310287in}{2.671661in}}%
\pgfpathlineto{\pgfqpoint{2.298927in}{2.668351in}}%
\pgfpathlineto{\pgfqpoint{2.304807in}{2.660421in}}%
\pgfpathlineto{\pgfqpoint{2.310691in}{2.652487in}}%
\pgfpathlineto{\pgfqpoint{2.316580in}{2.644544in}}%
\pgfpathlineto{\pgfqpoint{2.322472in}{2.636586in}}%
\pgfpathclose%
\pgfusepath{stroke,fill}%
\end{pgfscope}%
\begin{pgfscope}%
\pgfpathrectangle{\pgfqpoint{0.887500in}{0.275000in}}{\pgfqpoint{4.225000in}{4.225000in}}%
\pgfusepath{clip}%
\pgfsetbuttcap%
\pgfsetroundjoin%
\definecolor{currentfill}{rgb}{0.804182,0.882046,0.114965}%
\pgfsetfillcolor{currentfill}%
\pgfsetfillopacity{0.700000}%
\pgfsetlinewidth{0.501875pt}%
\definecolor{currentstroke}{rgb}{1.000000,1.000000,1.000000}%
\pgfsetstrokecolor{currentstroke}%
\pgfsetstrokeopacity{0.500000}%
\pgfsetdash{}{0pt}%
\pgfpathmoveto{\pgfqpoint{3.428419in}{3.473729in}}%
\pgfpathlineto{\pgfqpoint{3.439531in}{3.477424in}}%
\pgfpathlineto{\pgfqpoint{3.450637in}{3.481093in}}%
\pgfpathlineto{\pgfqpoint{3.461739in}{3.484743in}}%
\pgfpathlineto{\pgfqpoint{3.472835in}{3.488383in}}%
\pgfpathlineto{\pgfqpoint{3.483927in}{3.492012in}}%
\pgfpathlineto{\pgfqpoint{3.477662in}{3.500813in}}%
\pgfpathlineto{\pgfqpoint{3.471395in}{3.508952in}}%
\pgfpathlineto{\pgfqpoint{3.465125in}{3.516473in}}%
\pgfpathlineto{\pgfqpoint{3.458855in}{3.523462in}}%
\pgfpathlineto{\pgfqpoint{3.452585in}{3.530009in}}%
\pgfpathlineto{\pgfqpoint{3.441503in}{3.526244in}}%
\pgfpathlineto{\pgfqpoint{3.430419in}{3.522639in}}%
\pgfpathlineto{\pgfqpoint{3.419331in}{3.519219in}}%
\pgfpathlineto{\pgfqpoint{3.408239in}{3.515949in}}%
\pgfpathlineto{\pgfqpoint{3.397142in}{3.512772in}}%
\pgfpathlineto{\pgfqpoint{3.403397in}{3.505926in}}%
\pgfpathlineto{\pgfqpoint{3.409653in}{3.498657in}}%
\pgfpathlineto{\pgfqpoint{3.415909in}{3.490903in}}%
\pgfpathlineto{\pgfqpoint{3.422164in}{3.482603in}}%
\pgfpathclose%
\pgfusepath{stroke,fill}%
\end{pgfscope}%
\begin{pgfscope}%
\pgfpathrectangle{\pgfqpoint{0.887500in}{0.275000in}}{\pgfqpoint{4.225000in}{4.225000in}}%
\pgfusepath{clip}%
\pgfsetbuttcap%
\pgfsetroundjoin%
\definecolor{currentfill}{rgb}{0.120092,0.600104,0.542530}%
\pgfsetfillcolor{currentfill}%
\pgfsetfillopacity{0.700000}%
\pgfsetlinewidth{0.501875pt}%
\definecolor{currentstroke}{rgb}{1.000000,1.000000,1.000000}%
\pgfsetstrokecolor{currentstroke}%
\pgfsetstrokeopacity{0.500000}%
\pgfsetdash{}{0pt}%
\pgfpathmoveto{\pgfqpoint{1.782740in}{2.711178in}}%
\pgfpathlineto{\pgfqpoint{1.794246in}{2.714440in}}%
\pgfpathlineto{\pgfqpoint{1.805747in}{2.717709in}}%
\pgfpathlineto{\pgfqpoint{1.817242in}{2.720984in}}%
\pgfpathlineto{\pgfqpoint{1.828731in}{2.724269in}}%
\pgfpathlineto{\pgfqpoint{1.840214in}{2.727565in}}%
\pgfpathlineto{\pgfqpoint{1.834493in}{2.735315in}}%
\pgfpathlineto{\pgfqpoint{1.828776in}{2.743050in}}%
\pgfpathlineto{\pgfqpoint{1.823063in}{2.750771in}}%
\pgfpathlineto{\pgfqpoint{1.817354in}{2.758477in}}%
\pgfpathlineto{\pgfqpoint{1.811650in}{2.766171in}}%
\pgfpathlineto{\pgfqpoint{1.800180in}{2.762863in}}%
\pgfpathlineto{\pgfqpoint{1.788704in}{2.759566in}}%
\pgfpathlineto{\pgfqpoint{1.777223in}{2.756277in}}%
\pgfpathlineto{\pgfqpoint{1.765736in}{2.752997in}}%
\pgfpathlineto{\pgfqpoint{1.754243in}{2.749721in}}%
\pgfpathlineto{\pgfqpoint{1.759934in}{2.742043in}}%
\pgfpathlineto{\pgfqpoint{1.765629in}{2.734350in}}%
\pgfpathlineto{\pgfqpoint{1.771328in}{2.726641in}}%
\pgfpathlineto{\pgfqpoint{1.777032in}{2.718918in}}%
\pgfpathclose%
\pgfusepath{stroke,fill}%
\end{pgfscope}%
\begin{pgfscope}%
\pgfpathrectangle{\pgfqpoint{0.887500in}{0.275000in}}{\pgfqpoint{4.225000in}{4.225000in}}%
\pgfusepath{clip}%
\pgfsetbuttcap%
\pgfsetroundjoin%
\definecolor{currentfill}{rgb}{0.150148,0.676631,0.506589}%
\pgfsetfillcolor{currentfill}%
\pgfsetfillopacity{0.700000}%
\pgfsetlinewidth{0.501875pt}%
\definecolor{currentstroke}{rgb}{1.000000,1.000000,1.000000}%
\pgfsetstrokecolor{currentstroke}%
\pgfsetstrokeopacity{0.500000}%
\pgfsetdash{}{0pt}%
\pgfpathmoveto{\pgfqpoint{2.781959in}{2.839226in}}%
\pgfpathlineto{\pgfqpoint{2.793124in}{2.855973in}}%
\pgfpathlineto{\pgfqpoint{2.804291in}{2.872972in}}%
\pgfpathlineto{\pgfqpoint{2.815459in}{2.889987in}}%
\pgfpathlineto{\pgfqpoint{2.826630in}{2.906969in}}%
\pgfpathlineto{\pgfqpoint{2.837800in}{2.924487in}}%
\pgfpathlineto{\pgfqpoint{2.831755in}{2.928579in}}%
\pgfpathlineto{\pgfqpoint{2.825716in}{2.932600in}}%
\pgfpathlineto{\pgfqpoint{2.819684in}{2.936473in}}%
\pgfpathlineto{\pgfqpoint{2.813657in}{2.940121in}}%
\pgfpathlineto{\pgfqpoint{2.807638in}{2.943467in}}%
\pgfpathlineto{\pgfqpoint{2.796445in}{2.933772in}}%
\pgfpathlineto{\pgfqpoint{2.785251in}{2.923951in}}%
\pgfpathlineto{\pgfqpoint{2.774064in}{2.912911in}}%
\pgfpathlineto{\pgfqpoint{2.762885in}{2.900658in}}%
\pgfpathlineto{\pgfqpoint{2.751712in}{2.887541in}}%
\pgfpathlineto{\pgfqpoint{2.757752in}{2.878147in}}%
\pgfpathlineto{\pgfqpoint{2.763798in}{2.868503in}}%
\pgfpathlineto{\pgfqpoint{2.769848in}{2.858724in}}%
\pgfpathlineto{\pgfqpoint{2.775902in}{2.848926in}}%
\pgfpathclose%
\pgfusepath{stroke,fill}%
\end{pgfscope}%
\begin{pgfscope}%
\pgfpathrectangle{\pgfqpoint{0.887500in}{0.275000in}}{\pgfqpoint{4.225000in}{4.225000in}}%
\pgfusepath{clip}%
\pgfsetbuttcap%
\pgfsetroundjoin%
\definecolor{currentfill}{rgb}{0.344074,0.780029,0.397381}%
\pgfsetfillcolor{currentfill}%
\pgfsetfillopacity{0.700000}%
\pgfsetlinewidth{0.501875pt}%
\definecolor{currentstroke}{rgb}{1.000000,1.000000,1.000000}%
\pgfsetstrokecolor{currentstroke}%
\pgfsetstrokeopacity{0.500000}%
\pgfsetdash{}{0pt}%
\pgfpathmoveto{\pgfqpoint{2.863363in}{3.036269in}}%
\pgfpathlineto{\pgfqpoint{2.874439in}{3.072028in}}%
\pgfpathlineto{\pgfqpoint{2.885506in}{3.112942in}}%
\pgfpathlineto{\pgfqpoint{2.896576in}{3.157010in}}%
\pgfpathlineto{\pgfqpoint{2.907659in}{3.202219in}}%
\pgfpathlineto{\pgfqpoint{2.918764in}{3.246537in}}%
\pgfpathlineto{\pgfqpoint{2.912680in}{3.248281in}}%
\pgfpathlineto{\pgfqpoint{2.906602in}{3.249767in}}%
\pgfpathlineto{\pgfqpoint{2.900531in}{3.251071in}}%
\pgfpathlineto{\pgfqpoint{2.894467in}{3.252378in}}%
\pgfpathlineto{\pgfqpoint{2.888407in}{3.253877in}}%
\pgfpathlineto{\pgfqpoint{2.877379in}{3.203872in}}%
\pgfpathlineto{\pgfqpoint{2.866375in}{3.153273in}}%
\pgfpathlineto{\pgfqpoint{2.855382in}{3.104684in}}%
\pgfpathlineto{\pgfqpoint{2.844384in}{3.060687in}}%
\pgfpathlineto{\pgfqpoint{2.833364in}{3.023849in}}%
\pgfpathlineto{\pgfqpoint{2.839352in}{3.025818in}}%
\pgfpathlineto{\pgfqpoint{2.845344in}{3.028383in}}%
\pgfpathlineto{\pgfqpoint{2.851343in}{3.031180in}}%
\pgfpathlineto{\pgfqpoint{2.857348in}{3.033857in}}%
\pgfpathclose%
\pgfusepath{stroke,fill}%
\end{pgfscope}%
\begin{pgfscope}%
\pgfpathrectangle{\pgfqpoint{0.887500in}{0.275000in}}{\pgfqpoint{4.225000in}{4.225000in}}%
\pgfusepath{clip}%
\pgfsetbuttcap%
\pgfsetroundjoin%
\definecolor{currentfill}{rgb}{0.122606,0.585371,0.546557}%
\pgfsetfillcolor{currentfill}%
\pgfsetfillopacity{0.700000}%
\pgfsetlinewidth{0.501875pt}%
\definecolor{currentstroke}{rgb}{1.000000,1.000000,1.000000}%
\pgfsetstrokecolor{currentstroke}%
\pgfsetstrokeopacity{0.500000}%
\pgfsetdash{}{0pt}%
\pgfpathmoveto{\pgfqpoint{2.613980in}{2.655846in}}%
\pgfpathlineto{\pgfqpoint{2.625179in}{2.668316in}}%
\pgfpathlineto{\pgfqpoint{2.636377in}{2.680873in}}%
\pgfpathlineto{\pgfqpoint{2.647576in}{2.693196in}}%
\pgfpathlineto{\pgfqpoint{2.658781in}{2.704962in}}%
\pgfpathlineto{\pgfqpoint{2.669993in}{2.715907in}}%
\pgfpathlineto{\pgfqpoint{2.663975in}{2.724934in}}%
\pgfpathlineto{\pgfqpoint{2.657959in}{2.734166in}}%
\pgfpathlineto{\pgfqpoint{2.651943in}{2.743663in}}%
\pgfpathlineto{\pgfqpoint{2.645928in}{2.753486in}}%
\pgfpathlineto{\pgfqpoint{2.639912in}{2.763695in}}%
\pgfpathlineto{\pgfqpoint{2.628727in}{2.751583in}}%
\pgfpathlineto{\pgfqpoint{2.617548in}{2.738879in}}%
\pgfpathlineto{\pgfqpoint{2.606374in}{2.725826in}}%
\pgfpathlineto{\pgfqpoint{2.595200in}{2.712707in}}%
\pgfpathlineto{\pgfqpoint{2.584024in}{2.699804in}}%
\pgfpathlineto{\pgfqpoint{2.590017in}{2.690232in}}%
\pgfpathlineto{\pgfqpoint{2.596008in}{2.681140in}}%
\pgfpathlineto{\pgfqpoint{2.601998in}{2.672437in}}%
\pgfpathlineto{\pgfqpoint{2.607988in}{2.664035in}}%
\pgfpathclose%
\pgfusepath{stroke,fill}%
\end{pgfscope}%
\begin{pgfscope}%
\pgfpathrectangle{\pgfqpoint{0.887500in}{0.275000in}}{\pgfqpoint{4.225000in}{4.225000in}}%
\pgfusepath{clip}%
\pgfsetbuttcap%
\pgfsetroundjoin%
\definecolor{currentfill}{rgb}{0.772852,0.877868,0.131109}%
\pgfsetfillcolor{currentfill}%
\pgfsetfillopacity{0.700000}%
\pgfsetlinewidth{0.501875pt}%
\definecolor{currentstroke}{rgb}{1.000000,1.000000,1.000000}%
\pgfsetstrokecolor{currentstroke}%
\pgfsetstrokeopacity{0.500000}%
\pgfsetdash{}{0pt}%
\pgfpathmoveto{\pgfqpoint{2.913607in}{3.444346in}}%
\pgfpathlineto{\pgfqpoint{2.924795in}{3.460559in}}%
\pgfpathlineto{\pgfqpoint{2.935998in}{3.472443in}}%
\pgfpathlineto{\pgfqpoint{2.947209in}{3.480872in}}%
\pgfpathlineto{\pgfqpoint{2.958424in}{3.486716in}}%
\pgfpathlineto{\pgfqpoint{2.969639in}{3.490667in}}%
\pgfpathlineto{\pgfqpoint{2.963542in}{3.493667in}}%
\pgfpathlineto{\pgfqpoint{2.957450in}{3.496714in}}%
\pgfpathlineto{\pgfqpoint{2.951363in}{3.499840in}}%
\pgfpathlineto{\pgfqpoint{2.945282in}{3.502958in}}%
\pgfpathlineto{\pgfqpoint{2.939207in}{3.505941in}}%
\pgfpathlineto{\pgfqpoint{2.928007in}{3.502726in}}%
\pgfpathlineto{\pgfqpoint{2.916809in}{3.497740in}}%
\pgfpathlineto{\pgfqpoint{2.905615in}{3.490400in}}%
\pgfpathlineto{\pgfqpoint{2.894430in}{3.480030in}}%
\pgfpathlineto{\pgfqpoint{2.883260in}{3.465950in}}%
\pgfpathlineto{\pgfqpoint{2.889319in}{3.461660in}}%
\pgfpathlineto{\pgfqpoint{2.895384in}{3.457192in}}%
\pgfpathlineto{\pgfqpoint{2.901454in}{3.452748in}}%
\pgfpathlineto{\pgfqpoint{2.907528in}{3.448473in}}%
\pgfpathclose%
\pgfusepath{stroke,fill}%
\end{pgfscope}%
\begin{pgfscope}%
\pgfpathrectangle{\pgfqpoint{0.887500in}{0.275000in}}{\pgfqpoint{4.225000in}{4.225000in}}%
\pgfusepath{clip}%
\pgfsetbuttcap%
\pgfsetroundjoin%
\definecolor{currentfill}{rgb}{0.804182,0.882046,0.114965}%
\pgfsetfillcolor{currentfill}%
\pgfsetfillopacity{0.700000}%
\pgfsetlinewidth{0.501875pt}%
\definecolor{currentstroke}{rgb}{1.000000,1.000000,1.000000}%
\pgfsetstrokecolor{currentstroke}%
\pgfsetstrokeopacity{0.500000}%
\pgfsetdash{}{0pt}%
\pgfpathmoveto{\pgfqpoint{3.056303in}{3.486440in}}%
\pgfpathlineto{\pgfqpoint{3.067504in}{3.488962in}}%
\pgfpathlineto{\pgfqpoint{3.078699in}{3.492111in}}%
\pgfpathlineto{\pgfqpoint{3.089890in}{3.495751in}}%
\pgfpathlineto{\pgfqpoint{3.101076in}{3.499740in}}%
\pgfpathlineto{\pgfqpoint{3.112257in}{3.503936in}}%
\pgfpathlineto{\pgfqpoint{3.106099in}{3.508345in}}%
\pgfpathlineto{\pgfqpoint{3.099945in}{3.512436in}}%
\pgfpathlineto{\pgfqpoint{3.093797in}{3.516218in}}%
\pgfpathlineto{\pgfqpoint{3.087653in}{3.519700in}}%
\pgfpathlineto{\pgfqpoint{3.081515in}{3.522890in}}%
\pgfpathlineto{\pgfqpoint{3.070351in}{3.517122in}}%
\pgfpathlineto{\pgfqpoint{3.059182in}{3.511624in}}%
\pgfpathlineto{\pgfqpoint{3.048010in}{3.506674in}}%
\pgfpathlineto{\pgfqpoint{3.036832in}{3.502550in}}%
\pgfpathlineto{\pgfqpoint{3.025649in}{3.499522in}}%
\pgfpathlineto{\pgfqpoint{3.031769in}{3.497159in}}%
\pgfpathlineto{\pgfqpoint{3.037894in}{3.494740in}}%
\pgfpathlineto{\pgfqpoint{3.044024in}{3.492196in}}%
\pgfpathlineto{\pgfqpoint{3.050161in}{3.489453in}}%
\pgfpathclose%
\pgfusepath{stroke,fill}%
\end{pgfscope}%
\begin{pgfscope}%
\pgfpathrectangle{\pgfqpoint{0.887500in}{0.275000in}}{\pgfqpoint{4.225000in}{4.225000in}}%
\pgfusepath{clip}%
\pgfsetbuttcap%
\pgfsetroundjoin%
\definecolor{currentfill}{rgb}{0.119483,0.614817,0.537692}%
\pgfsetfillcolor{currentfill}%
\pgfsetfillopacity{0.700000}%
\pgfsetlinewidth{0.501875pt}%
\definecolor{currentstroke}{rgb}{1.000000,1.000000,1.000000}%
\pgfsetstrokecolor{currentstroke}%
\pgfsetstrokeopacity{0.500000}%
\pgfsetdash{}{0pt}%
\pgfpathmoveto{\pgfqpoint{1.552990in}{2.738522in}}%
\pgfpathlineto{\pgfqpoint{1.564553in}{2.741829in}}%
\pgfpathlineto{\pgfqpoint{1.576110in}{2.745135in}}%
\pgfpathlineto{\pgfqpoint{1.587662in}{2.748440in}}%
\pgfpathlineto{\pgfqpoint{1.599208in}{2.751742in}}%
\pgfpathlineto{\pgfqpoint{1.610748in}{2.755041in}}%
\pgfpathlineto{\pgfqpoint{1.605110in}{2.762602in}}%
\pgfpathlineto{\pgfqpoint{1.599477in}{2.770145in}}%
\pgfpathlineto{\pgfqpoint{1.593847in}{2.777671in}}%
\pgfpathlineto{\pgfqpoint{1.588222in}{2.785180in}}%
\pgfpathlineto{\pgfqpoint{1.582601in}{2.792671in}}%
\pgfpathlineto{\pgfqpoint{1.571075in}{2.789355in}}%
\pgfpathlineto{\pgfqpoint{1.559543in}{2.786036in}}%
\pgfpathlineto{\pgfqpoint{1.548006in}{2.782715in}}%
\pgfpathlineto{\pgfqpoint{1.536463in}{2.779392in}}%
\pgfpathlineto{\pgfqpoint{1.524914in}{2.776069in}}%
\pgfpathlineto{\pgfqpoint{1.530521in}{2.768594in}}%
\pgfpathlineto{\pgfqpoint{1.536131in}{2.761101in}}%
\pgfpathlineto{\pgfqpoint{1.541747in}{2.753592in}}%
\pgfpathlineto{\pgfqpoint{1.547366in}{2.746065in}}%
\pgfpathclose%
\pgfusepath{stroke,fill}%
\end{pgfscope}%
\begin{pgfscope}%
\pgfpathrectangle{\pgfqpoint{0.887500in}{0.275000in}}{\pgfqpoint{4.225000in}{4.225000in}}%
\pgfusepath{clip}%
\pgfsetbuttcap%
\pgfsetroundjoin%
\definecolor{currentfill}{rgb}{0.124395,0.578002,0.548287}%
\pgfsetfillcolor{currentfill}%
\pgfsetfillopacity{0.700000}%
\pgfsetlinewidth{0.501875pt}%
\definecolor{currentstroke}{rgb}{1.000000,1.000000,1.000000}%
\pgfsetstrokecolor{currentstroke}%
\pgfsetstrokeopacity{0.500000}%
\pgfsetdash{}{0pt}%
\pgfpathmoveto{\pgfqpoint{2.098684in}{2.658857in}}%
\pgfpathlineto{\pgfqpoint{2.110116in}{2.662111in}}%
\pgfpathlineto{\pgfqpoint{2.121543in}{2.665396in}}%
\pgfpathlineto{\pgfqpoint{2.132962in}{2.668729in}}%
\pgfpathlineto{\pgfqpoint{2.144375in}{2.672129in}}%
\pgfpathlineto{\pgfqpoint{2.155780in}{2.675613in}}%
\pgfpathlineto{\pgfqpoint{2.149945in}{2.683643in}}%
\pgfpathlineto{\pgfqpoint{2.144114in}{2.691659in}}%
\pgfpathlineto{\pgfqpoint{2.138287in}{2.699663in}}%
\pgfpathlineto{\pgfqpoint{2.132464in}{2.707654in}}%
\pgfpathlineto{\pgfqpoint{2.126644in}{2.715633in}}%
\pgfpathlineto{\pgfqpoint{2.115253in}{2.712091in}}%
\pgfpathlineto{\pgfqpoint{2.103854in}{2.708656in}}%
\pgfpathlineto{\pgfqpoint{2.092447in}{2.705304in}}%
\pgfpathlineto{\pgfqpoint{2.081033in}{2.702013in}}%
\pgfpathlineto{\pgfqpoint{2.069613in}{2.698761in}}%
\pgfpathlineto{\pgfqpoint{2.075419in}{2.690806in}}%
\pgfpathlineto{\pgfqpoint{2.081229in}{2.682838in}}%
\pgfpathlineto{\pgfqpoint{2.087043in}{2.674857in}}%
\pgfpathlineto{\pgfqpoint{2.092861in}{2.666863in}}%
\pgfpathclose%
\pgfusepath{stroke,fill}%
\end{pgfscope}%
\begin{pgfscope}%
\pgfpathrectangle{\pgfqpoint{0.887500in}{0.275000in}}{\pgfqpoint{4.225000in}{4.225000in}}%
\pgfusepath{clip}%
\pgfsetbuttcap%
\pgfsetroundjoin%
\definecolor{currentfill}{rgb}{0.129933,0.559582,0.551864}%
\pgfsetfillcolor{currentfill}%
\pgfsetfillopacity{0.700000}%
\pgfsetlinewidth{0.501875pt}%
\definecolor{currentstroke}{rgb}{1.000000,1.000000,1.000000}%
\pgfsetstrokecolor{currentstroke}%
\pgfsetstrokeopacity{0.500000}%
\pgfsetdash{}{0pt}%
\pgfpathmoveto{\pgfqpoint{2.557837in}{2.605082in}}%
\pgfpathlineto{\pgfqpoint{2.569096in}{2.613062in}}%
\pgfpathlineto{\pgfqpoint{2.580338in}{2.622154in}}%
\pgfpathlineto{\pgfqpoint{2.591564in}{2.632446in}}%
\pgfpathlineto{\pgfqpoint{2.602777in}{2.643782in}}%
\pgfpathlineto{\pgfqpoint{2.613980in}{2.655846in}}%
\pgfpathlineto{\pgfqpoint{2.607988in}{2.664035in}}%
\pgfpathlineto{\pgfqpoint{2.601998in}{2.672437in}}%
\pgfpathlineto{\pgfqpoint{2.596008in}{2.681140in}}%
\pgfpathlineto{\pgfqpoint{2.590017in}{2.690232in}}%
\pgfpathlineto{\pgfqpoint{2.584024in}{2.699804in}}%
\pgfpathlineto{\pgfqpoint{2.572841in}{2.687401in}}%
\pgfpathlineto{\pgfqpoint{2.561649in}{2.675779in}}%
\pgfpathlineto{\pgfqpoint{2.550442in}{2.665218in}}%
\pgfpathlineto{\pgfqpoint{2.539219in}{2.655853in}}%
\pgfpathlineto{\pgfqpoint{2.527979in}{2.647602in}}%
\pgfpathlineto{\pgfqpoint{2.533951in}{2.638461in}}%
\pgfpathlineto{\pgfqpoint{2.539922in}{2.629707in}}%
\pgfpathlineto{\pgfqpoint{2.545893in}{2.621271in}}%
\pgfpathlineto{\pgfqpoint{2.551864in}{2.613086in}}%
\pgfpathclose%
\pgfusepath{stroke,fill}%
\end{pgfscope}%
\begin{pgfscope}%
\pgfpathrectangle{\pgfqpoint{0.887500in}{0.275000in}}{\pgfqpoint{4.225000in}{4.225000in}}%
\pgfusepath{clip}%
\pgfsetbuttcap%
\pgfsetroundjoin%
\definecolor{currentfill}{rgb}{0.772852,0.877868,0.131109}%
\pgfsetfillcolor{currentfill}%
\pgfsetfillopacity{0.700000}%
\pgfsetlinewidth{0.501875pt}%
\definecolor{currentstroke}{rgb}{1.000000,1.000000,1.000000}%
\pgfsetstrokecolor{currentstroke}%
\pgfsetstrokeopacity{0.500000}%
\pgfsetdash{}{0pt}%
\pgfpathmoveto{\pgfqpoint{3.515224in}{3.440588in}}%
\pgfpathlineto{\pgfqpoint{3.526315in}{3.444014in}}%
\pgfpathlineto{\pgfqpoint{3.537401in}{3.447407in}}%
\pgfpathlineto{\pgfqpoint{3.548482in}{3.450773in}}%
\pgfpathlineto{\pgfqpoint{3.559556in}{3.454119in}}%
\pgfpathlineto{\pgfqpoint{3.570626in}{3.457452in}}%
\pgfpathlineto{\pgfqpoint{3.564363in}{3.468596in}}%
\pgfpathlineto{\pgfqpoint{3.558100in}{3.479496in}}%
\pgfpathlineto{\pgfqpoint{3.551837in}{3.490075in}}%
\pgfpathlineto{\pgfqpoint{3.545572in}{3.500255in}}%
\pgfpathlineto{\pgfqpoint{3.539305in}{3.509958in}}%
\pgfpathlineto{\pgfqpoint{3.528239in}{3.506398in}}%
\pgfpathlineto{\pgfqpoint{3.517169in}{3.502823in}}%
\pgfpathlineto{\pgfqpoint{3.506093in}{3.499233in}}%
\pgfpathlineto{\pgfqpoint{3.495013in}{3.495629in}}%
\pgfpathlineto{\pgfqpoint{3.483927in}{3.492012in}}%
\pgfpathlineto{\pgfqpoint{3.490189in}{3.482610in}}%
\pgfpathlineto{\pgfqpoint{3.496449in}{3.472687in}}%
\pgfpathlineto{\pgfqpoint{3.502707in}{3.462323in}}%
\pgfpathlineto{\pgfqpoint{3.508965in}{3.451596in}}%
\pgfpathclose%
\pgfusepath{stroke,fill}%
\end{pgfscope}%
\begin{pgfscope}%
\pgfpathrectangle{\pgfqpoint{0.887500in}{0.275000in}}{\pgfqpoint{4.225000in}{4.225000in}}%
\pgfusepath{clip}%
\pgfsetbuttcap%
\pgfsetroundjoin%
\definecolor{currentfill}{rgb}{0.132444,0.552216,0.553018}%
\pgfsetfillcolor{currentfill}%
\pgfsetfillopacity{0.700000}%
\pgfsetlinewidth{0.501875pt}%
\definecolor{currentstroke}{rgb}{1.000000,1.000000,1.000000}%
\pgfsetstrokecolor{currentstroke}%
\pgfsetstrokeopacity{0.500000}%
\pgfsetdash{}{0pt}%
\pgfpathmoveto{\pgfqpoint{2.414797in}{2.604126in}}%
\pgfpathlineto{\pgfqpoint{2.426162in}{2.606890in}}%
\pgfpathlineto{\pgfqpoint{2.437521in}{2.609704in}}%
\pgfpathlineto{\pgfqpoint{2.448872in}{2.612681in}}%
\pgfpathlineto{\pgfqpoint{2.460212in}{2.615939in}}%
\pgfpathlineto{\pgfqpoint{2.471542in}{2.619593in}}%
\pgfpathlineto{\pgfqpoint{2.465587in}{2.628545in}}%
\pgfpathlineto{\pgfqpoint{2.459632in}{2.637815in}}%
\pgfpathlineto{\pgfqpoint{2.453675in}{2.647413in}}%
\pgfpathlineto{\pgfqpoint{2.447717in}{2.657296in}}%
\pgfpathlineto{\pgfqpoint{2.441759in}{2.667420in}}%
\pgfpathlineto{\pgfqpoint{2.430459in}{2.662514in}}%
\pgfpathlineto{\pgfqpoint{2.419149in}{2.657988in}}%
\pgfpathlineto{\pgfqpoint{2.407829in}{2.653782in}}%
\pgfpathlineto{\pgfqpoint{2.396499in}{2.649835in}}%
\pgfpathlineto{\pgfqpoint{2.385161in}{2.646087in}}%
\pgfpathlineto{\pgfqpoint{2.391082in}{2.637628in}}%
\pgfpathlineto{\pgfqpoint{2.397006in}{2.629199in}}%
\pgfpathlineto{\pgfqpoint{2.402933in}{2.620804in}}%
\pgfpathlineto{\pgfqpoint{2.408863in}{2.612447in}}%
\pgfpathclose%
\pgfusepath{stroke,fill}%
\end{pgfscope}%
\begin{pgfscope}%
\pgfpathrectangle{\pgfqpoint{0.887500in}{0.275000in}}{\pgfqpoint{4.225000in}{4.225000in}}%
\pgfusepath{clip}%
\pgfsetbuttcap%
\pgfsetroundjoin%
\definecolor{currentfill}{rgb}{0.720391,0.870350,0.162603}%
\pgfsetfillcolor{currentfill}%
\pgfsetfillopacity{0.700000}%
\pgfsetlinewidth{0.501875pt}%
\definecolor{currentstroke}{rgb}{1.000000,1.000000,1.000000}%
\pgfsetstrokecolor{currentstroke}%
\pgfsetstrokeopacity{0.500000}%
\pgfsetdash{}{0pt}%
\pgfpathmoveto{\pgfqpoint{3.601977in}{3.400809in}}%
\pgfpathlineto{\pgfqpoint{3.613047in}{3.404108in}}%
\pgfpathlineto{\pgfqpoint{3.624111in}{3.407384in}}%
\pgfpathlineto{\pgfqpoint{3.635169in}{3.410626in}}%
\pgfpathlineto{\pgfqpoint{3.646222in}{3.413822in}}%
\pgfpathlineto{\pgfqpoint{3.657268in}{3.416962in}}%
\pgfpathlineto{\pgfqpoint{3.650983in}{3.428097in}}%
\pgfpathlineto{\pgfqpoint{3.644704in}{3.439387in}}%
\pgfpathlineto{\pgfqpoint{3.638428in}{3.450766in}}%
\pgfpathlineto{\pgfqpoint{3.632157in}{3.462164in}}%
\pgfpathlineto{\pgfqpoint{3.625887in}{3.473511in}}%
\pgfpathlineto{\pgfqpoint{3.614848in}{3.470443in}}%
\pgfpathlineto{\pgfqpoint{3.603801in}{3.467284in}}%
\pgfpathlineto{\pgfqpoint{3.592749in}{3.464053in}}%
\pgfpathlineto{\pgfqpoint{3.581690in}{3.460769in}}%
\pgfpathlineto{\pgfqpoint{3.570626in}{3.457452in}}%
\pgfpathlineto{\pgfqpoint{3.576890in}{3.446143in}}%
\pgfpathlineto{\pgfqpoint{3.583156in}{3.434747in}}%
\pgfpathlineto{\pgfqpoint{3.589425in}{3.423342in}}%
\pgfpathlineto{\pgfqpoint{3.595699in}{3.412005in}}%
\pgfpathclose%
\pgfusepath{stroke,fill}%
\end{pgfscope}%
\begin{pgfscope}%
\pgfpathrectangle{\pgfqpoint{0.887500in}{0.275000in}}{\pgfqpoint{4.225000in}{4.225000in}}%
\pgfusepath{clip}%
\pgfsetbuttcap%
\pgfsetroundjoin%
\definecolor{currentfill}{rgb}{0.458674,0.816363,0.329727}%
\pgfsetfillcolor{currentfill}%
\pgfsetfillopacity{0.700000}%
\pgfsetlinewidth{0.501875pt}%
\definecolor{currentstroke}{rgb}{1.000000,1.000000,1.000000}%
\pgfsetstrokecolor{currentstroke}%
\pgfsetstrokeopacity{0.500000}%
\pgfsetdash{}{0pt}%
\pgfpathmoveto{\pgfqpoint{4.035642in}{3.192577in}}%
\pgfpathlineto{\pgfqpoint{4.046599in}{3.195496in}}%
\pgfpathlineto{\pgfqpoint{4.057549in}{3.198387in}}%
\pgfpathlineto{\pgfqpoint{4.068493in}{3.201250in}}%
\pgfpathlineto{\pgfqpoint{4.079431in}{3.204084in}}%
\pgfpathlineto{\pgfqpoint{4.090362in}{3.206890in}}%
\pgfpathlineto{\pgfqpoint{4.083993in}{3.218999in}}%
\pgfpathlineto{\pgfqpoint{4.077626in}{3.231095in}}%
\pgfpathlineto{\pgfqpoint{4.071262in}{3.243181in}}%
\pgfpathlineto{\pgfqpoint{4.064901in}{3.255262in}}%
\pgfpathlineto{\pgfqpoint{4.058543in}{3.267344in}}%
\pgfpathlineto{\pgfqpoint{4.047616in}{3.264584in}}%
\pgfpathlineto{\pgfqpoint{4.036683in}{3.261785in}}%
\pgfpathlineto{\pgfqpoint{4.025744in}{3.258949in}}%
\pgfpathlineto{\pgfqpoint{4.014798in}{3.256075in}}%
\pgfpathlineto{\pgfqpoint{4.003846in}{3.253165in}}%
\pgfpathlineto{\pgfqpoint{4.010204in}{3.241227in}}%
\pgfpathlineto{\pgfqpoint{4.016562in}{3.229191in}}%
\pgfpathlineto{\pgfqpoint{4.022921in}{3.217065in}}%
\pgfpathlineto{\pgfqpoint{4.029281in}{3.204858in}}%
\pgfpathclose%
\pgfusepath{stroke,fill}%
\end{pgfscope}%
\begin{pgfscope}%
\pgfpathrectangle{\pgfqpoint{0.887500in}{0.275000in}}{\pgfqpoint{4.225000in}{4.225000in}}%
\pgfusepath{clip}%
\pgfsetbuttcap%
\pgfsetroundjoin%
\definecolor{currentfill}{rgb}{0.395174,0.797475,0.367757}%
\pgfsetfillcolor{currentfill}%
\pgfsetfillopacity{0.700000}%
\pgfsetlinewidth{0.501875pt}%
\definecolor{currentstroke}{rgb}{1.000000,1.000000,1.000000}%
\pgfsetstrokecolor{currentstroke}%
\pgfsetstrokeopacity{0.500000}%
\pgfsetdash{}{0pt}%
\pgfpathmoveto{\pgfqpoint{4.122245in}{3.145964in}}%
\pgfpathlineto{\pgfqpoint{4.133185in}{3.149119in}}%
\pgfpathlineto{\pgfqpoint{4.144119in}{3.152285in}}%
\pgfpathlineto{\pgfqpoint{4.155049in}{3.155458in}}%
\pgfpathlineto{\pgfqpoint{4.165973in}{3.158640in}}%
\pgfpathlineto{\pgfqpoint{4.159580in}{3.170551in}}%
\pgfpathlineto{\pgfqpoint{4.153188in}{3.182397in}}%
\pgfpathlineto{\pgfqpoint{4.146798in}{3.194205in}}%
\pgfpathlineto{\pgfqpoint{4.140411in}{3.206001in}}%
\pgfpathlineto{\pgfqpoint{4.134027in}{3.217811in}}%
\pgfpathlineto{\pgfqpoint{4.123120in}{3.215127in}}%
\pgfpathlineto{\pgfqpoint{4.112207in}{3.212411in}}%
\pgfpathlineto{\pgfqpoint{4.101288in}{3.209665in}}%
\pgfpathlineto{\pgfqpoint{4.090362in}{3.206890in}}%
\pgfpathlineto{\pgfqpoint{4.096734in}{3.194761in}}%
\pgfpathlineto{\pgfqpoint{4.103109in}{3.182609in}}%
\pgfpathlineto{\pgfqpoint{4.109486in}{3.170429in}}%
\pgfpathlineto{\pgfqpoint{4.115864in}{3.158215in}}%
\pgfpathclose%
\pgfusepath{stroke,fill}%
\end{pgfscope}%
\begin{pgfscope}%
\pgfpathrectangle{\pgfqpoint{0.887500in}{0.275000in}}{\pgfqpoint{4.225000in}{4.225000in}}%
\pgfusepath{clip}%
\pgfsetbuttcap%
\pgfsetroundjoin%
\definecolor{currentfill}{rgb}{0.668054,0.861999,0.196293}%
\pgfsetfillcolor{currentfill}%
\pgfsetfillopacity{0.700000}%
\pgfsetlinewidth{0.501875pt}%
\definecolor{currentstroke}{rgb}{1.000000,1.000000,1.000000}%
\pgfsetstrokecolor{currentstroke}%
\pgfsetstrokeopacity{0.500000}%
\pgfsetdash{}{0pt}%
\pgfpathmoveto{\pgfqpoint{3.688766in}{3.362842in}}%
\pgfpathlineto{\pgfqpoint{3.699815in}{3.366049in}}%
\pgfpathlineto{\pgfqpoint{3.710858in}{3.369212in}}%
\pgfpathlineto{\pgfqpoint{3.721894in}{3.372336in}}%
\pgfpathlineto{\pgfqpoint{3.732925in}{3.375439in}}%
\pgfpathlineto{\pgfqpoint{3.743951in}{3.378538in}}%
\pgfpathlineto{\pgfqpoint{3.737634in}{3.389140in}}%
\pgfpathlineto{\pgfqpoint{3.731321in}{3.399734in}}%
\pgfpathlineto{\pgfqpoint{3.725012in}{3.410357in}}%
\pgfpathlineto{\pgfqpoint{3.718708in}{3.421048in}}%
\pgfpathlineto{\pgfqpoint{3.712408in}{3.431844in}}%
\pgfpathlineto{\pgfqpoint{3.701391in}{3.428880in}}%
\pgfpathlineto{\pgfqpoint{3.690369in}{3.425958in}}%
\pgfpathlineto{\pgfqpoint{3.679342in}{3.423026in}}%
\pgfpathlineto{\pgfqpoint{3.668308in}{3.420034in}}%
\pgfpathlineto{\pgfqpoint{3.657268in}{3.416962in}}%
\pgfpathlineto{\pgfqpoint{3.663559in}{3.405983in}}%
\pgfpathlineto{\pgfqpoint{3.669855in}{3.395120in}}%
\pgfpathlineto{\pgfqpoint{3.676155in}{3.384335in}}%
\pgfpathlineto{\pgfqpoint{3.682459in}{3.373589in}}%
\pgfpathclose%
\pgfusepath{stroke,fill}%
\end{pgfscope}%
\begin{pgfscope}%
\pgfpathrectangle{\pgfqpoint{0.887500in}{0.275000in}}{\pgfqpoint{4.225000in}{4.225000in}}%
\pgfusepath{clip}%
\pgfsetbuttcap%
\pgfsetroundjoin%
\definecolor{currentfill}{rgb}{0.121148,0.592739,0.544641}%
\pgfsetfillcolor{currentfill}%
\pgfsetfillopacity{0.700000}%
\pgfsetlinewidth{0.501875pt}%
\definecolor{currentstroke}{rgb}{1.000000,1.000000,1.000000}%
\pgfsetstrokecolor{currentstroke}%
\pgfsetstrokeopacity{0.500000}%
\pgfsetdash{}{0pt}%
\pgfpathmoveto{\pgfqpoint{1.868883in}{2.688577in}}%
\pgfpathlineto{\pgfqpoint{1.880373in}{2.691880in}}%
\pgfpathlineto{\pgfqpoint{1.891857in}{2.695194in}}%
\pgfpathlineto{\pgfqpoint{1.903335in}{2.698519in}}%
\pgfpathlineto{\pgfqpoint{1.914808in}{2.701852in}}%
\pgfpathlineto{\pgfqpoint{1.926275in}{2.705192in}}%
\pgfpathlineto{\pgfqpoint{1.920519in}{2.713028in}}%
\pgfpathlineto{\pgfqpoint{1.914768in}{2.720847in}}%
\pgfpathlineto{\pgfqpoint{1.909022in}{2.728650in}}%
\pgfpathlineto{\pgfqpoint{1.903279in}{2.736437in}}%
\pgfpathlineto{\pgfqpoint{1.897540in}{2.744210in}}%
\pgfpathlineto{\pgfqpoint{1.886087in}{2.740860in}}%
\pgfpathlineto{\pgfqpoint{1.874627in}{2.737521in}}%
\pgfpathlineto{\pgfqpoint{1.863162in}{2.734191in}}%
\pgfpathlineto{\pgfqpoint{1.851691in}{2.730873in}}%
\pgfpathlineto{\pgfqpoint{1.840214in}{2.727565in}}%
\pgfpathlineto{\pgfqpoint{1.845939in}{2.719801in}}%
\pgfpathlineto{\pgfqpoint{1.851669in}{2.712020in}}%
\pgfpathlineto{\pgfqpoint{1.857403in}{2.704223in}}%
\pgfpathlineto{\pgfqpoint{1.863141in}{2.696409in}}%
\pgfpathclose%
\pgfusepath{stroke,fill}%
\end{pgfscope}%
\begin{pgfscope}%
\pgfpathrectangle{\pgfqpoint{0.887500in}{0.275000in}}{\pgfqpoint{4.225000in}{4.225000in}}%
\pgfusepath{clip}%
\pgfsetbuttcap%
\pgfsetroundjoin%
\definecolor{currentfill}{rgb}{0.208030,0.718701,0.472873}%
\pgfsetfillcolor{currentfill}%
\pgfsetfillopacity{0.700000}%
\pgfsetlinewidth{0.501875pt}%
\definecolor{currentstroke}{rgb}{1.000000,1.000000,1.000000}%
\pgfsetstrokecolor{currentstroke}%
\pgfsetstrokeopacity{0.500000}%
\pgfsetdash{}{0pt}%
\pgfpathmoveto{\pgfqpoint{2.837800in}{2.924487in}}%
\pgfpathlineto{\pgfqpoint{2.848966in}{2.943236in}}%
\pgfpathlineto{\pgfqpoint{2.860126in}{2.963911in}}%
\pgfpathlineto{\pgfqpoint{2.871279in}{2.987213in}}%
\pgfpathlineto{\pgfqpoint{2.882422in}{3.013842in}}%
\pgfpathlineto{\pgfqpoint{2.893557in}{3.044503in}}%
\pgfpathlineto{\pgfqpoint{2.887502in}{3.043350in}}%
\pgfpathlineto{\pgfqpoint{2.881455in}{3.041955in}}%
\pgfpathlineto{\pgfqpoint{2.875416in}{3.040314in}}%
\pgfpathlineto{\pgfqpoint{2.869386in}{3.038420in}}%
\pgfpathlineto{\pgfqpoint{2.863363in}{3.036269in}}%
\pgfpathlineto{\pgfqpoint{2.852264in}{3.007383in}}%
\pgfpathlineto{\pgfqpoint{2.841140in}{2.984987in}}%
\pgfpathlineto{\pgfqpoint{2.829991in}{2.967754in}}%
\pgfpathlineto{\pgfqpoint{2.818823in}{2.954354in}}%
\pgfpathlineto{\pgfqpoint{2.807638in}{2.943467in}}%
\pgfpathlineto{\pgfqpoint{2.813657in}{2.940121in}}%
\pgfpathlineto{\pgfqpoint{2.819684in}{2.936473in}}%
\pgfpathlineto{\pgfqpoint{2.825716in}{2.932600in}}%
\pgfpathlineto{\pgfqpoint{2.831755in}{2.928579in}}%
\pgfpathclose%
\pgfusepath{stroke,fill}%
\end{pgfscope}%
\begin{pgfscope}%
\pgfpathrectangle{\pgfqpoint{0.887500in}{0.275000in}}{\pgfqpoint{4.225000in}{4.225000in}}%
\pgfusepath{clip}%
\pgfsetbuttcap%
\pgfsetroundjoin%
\definecolor{currentfill}{rgb}{0.120638,0.625828,0.533488}%
\pgfsetfillcolor{currentfill}%
\pgfsetfillopacity{0.700000}%
\pgfsetlinewidth{0.501875pt}%
\definecolor{currentstroke}{rgb}{1.000000,1.000000,1.000000}%
\pgfsetstrokecolor{currentstroke}%
\pgfsetstrokeopacity{0.500000}%
\pgfsetdash{}{0pt}%
\pgfpathmoveto{\pgfqpoint{1.323277in}{2.762757in}}%
\pgfpathlineto{\pgfqpoint{1.334893in}{2.766161in}}%
\pgfpathlineto{\pgfqpoint{1.346503in}{2.769557in}}%
\pgfpathlineto{\pgfqpoint{1.358108in}{2.772945in}}%
\pgfpathlineto{\pgfqpoint{1.369708in}{2.776325in}}%
\pgfpathlineto{\pgfqpoint{1.381302in}{2.779698in}}%
\pgfpathlineto{\pgfqpoint{1.375752in}{2.787009in}}%
\pgfpathlineto{\pgfqpoint{1.370207in}{2.794298in}}%
\pgfpathlineto{\pgfqpoint{1.364666in}{2.801563in}}%
\pgfpathlineto{\pgfqpoint{1.359130in}{2.808805in}}%
\pgfpathlineto{\pgfqpoint{1.347549in}{2.805395in}}%
\pgfpathlineto{\pgfqpoint{1.335962in}{2.801977in}}%
\pgfpathlineto{\pgfqpoint{1.324370in}{2.798551in}}%
\pgfpathlineto{\pgfqpoint{1.312773in}{2.795118in}}%
\pgfpathlineto{\pgfqpoint{1.301171in}{2.791677in}}%
\pgfpathlineto{\pgfqpoint{1.306690in}{2.784489in}}%
\pgfpathlineto{\pgfqpoint{1.312214in}{2.777273in}}%
\pgfpathlineto{\pgfqpoint{1.317743in}{2.770028in}}%
\pgfpathclose%
\pgfusepath{stroke,fill}%
\end{pgfscope}%
\begin{pgfscope}%
\pgfpathrectangle{\pgfqpoint{0.887500in}{0.275000in}}{\pgfqpoint{4.225000in}{4.225000in}}%
\pgfusepath{clip}%
\pgfsetbuttcap%
\pgfsetroundjoin%
\definecolor{currentfill}{rgb}{0.506271,0.828786,0.300362}%
\pgfsetfillcolor{currentfill}%
\pgfsetfillopacity{0.700000}%
\pgfsetlinewidth{0.501875pt}%
\definecolor{currentstroke}{rgb}{1.000000,1.000000,1.000000}%
\pgfsetstrokecolor{currentstroke}%
\pgfsetstrokeopacity{0.500000}%
\pgfsetdash{}{0pt}%
\pgfpathmoveto{\pgfqpoint{3.948994in}{3.238081in}}%
\pgfpathlineto{\pgfqpoint{3.959976in}{3.241167in}}%
\pgfpathlineto{\pgfqpoint{3.970953in}{3.244219in}}%
\pgfpathlineto{\pgfqpoint{3.981924in}{3.247237in}}%
\pgfpathlineto{\pgfqpoint{3.992888in}{3.250219in}}%
\pgfpathlineto{\pgfqpoint{4.003846in}{3.253165in}}%
\pgfpathlineto{\pgfqpoint{3.997490in}{3.265007in}}%
\pgfpathlineto{\pgfqpoint{3.991135in}{3.276771in}}%
\pgfpathlineto{\pgfqpoint{3.984781in}{3.288475in}}%
\pgfpathlineto{\pgfqpoint{3.978430in}{3.300141in}}%
\pgfpathlineto{\pgfqpoint{3.972081in}{3.311788in}}%
\pgfpathlineto{\pgfqpoint{3.961123in}{3.308656in}}%
\pgfpathlineto{\pgfqpoint{3.950161in}{3.305557in}}%
\pgfpathlineto{\pgfqpoint{3.939194in}{3.302493in}}%
\pgfpathlineto{\pgfqpoint{3.928223in}{3.299463in}}%
\pgfpathlineto{\pgfqpoint{3.917247in}{3.296469in}}%
\pgfpathlineto{\pgfqpoint{3.923593in}{3.284907in}}%
\pgfpathlineto{\pgfqpoint{3.929940in}{3.273300in}}%
\pgfpathlineto{\pgfqpoint{3.936290in}{3.261635in}}%
\pgfpathlineto{\pgfqpoint{3.942641in}{3.249899in}}%
\pgfpathclose%
\pgfusepath{stroke,fill}%
\end{pgfscope}%
\begin{pgfscope}%
\pgfpathrectangle{\pgfqpoint{0.887500in}{0.275000in}}{\pgfqpoint{4.225000in}{4.225000in}}%
\pgfusepath{clip}%
\pgfsetbuttcap%
\pgfsetroundjoin%
\definecolor{currentfill}{rgb}{0.616293,0.852709,0.230052}%
\pgfsetfillcolor{currentfill}%
\pgfsetfillopacity{0.700000}%
\pgfsetlinewidth{0.501875pt}%
\definecolor{currentstroke}{rgb}{1.000000,1.000000,1.000000}%
\pgfsetstrokecolor{currentstroke}%
\pgfsetstrokeopacity{0.500000}%
\pgfsetdash{}{0pt}%
\pgfpathmoveto{\pgfqpoint{3.775558in}{3.324084in}}%
\pgfpathlineto{\pgfqpoint{3.786586in}{3.327266in}}%
\pgfpathlineto{\pgfqpoint{3.797608in}{3.330396in}}%
\pgfpathlineto{\pgfqpoint{3.808623in}{3.333465in}}%
\pgfpathlineto{\pgfqpoint{3.819632in}{3.336465in}}%
\pgfpathlineto{\pgfqpoint{3.830634in}{3.339404in}}%
\pgfpathlineto{\pgfqpoint{3.824306in}{3.350627in}}%
\pgfpathlineto{\pgfqpoint{3.817979in}{3.361732in}}%
\pgfpathlineto{\pgfqpoint{3.811653in}{3.372744in}}%
\pgfpathlineto{\pgfqpoint{3.805330in}{3.383683in}}%
\pgfpathlineto{\pgfqpoint{3.799009in}{3.394573in}}%
\pgfpathlineto{\pgfqpoint{3.788006in}{3.391245in}}%
\pgfpathlineto{\pgfqpoint{3.776998in}{3.387987in}}%
\pgfpathlineto{\pgfqpoint{3.765987in}{3.384794in}}%
\pgfpathlineto{\pgfqpoint{3.754971in}{3.381651in}}%
\pgfpathlineto{\pgfqpoint{3.743951in}{3.378538in}}%
\pgfpathlineto{\pgfqpoint{3.750270in}{3.367890in}}%
\pgfpathlineto{\pgfqpoint{3.756591in}{3.357158in}}%
\pgfpathlineto{\pgfqpoint{3.762913in}{3.346306in}}%
\pgfpathlineto{\pgfqpoint{3.769236in}{3.335293in}}%
\pgfpathclose%
\pgfusepath{stroke,fill}%
\end{pgfscope}%
\begin{pgfscope}%
\pgfpathrectangle{\pgfqpoint{0.887500in}{0.275000in}}{\pgfqpoint{4.225000in}{4.225000in}}%
\pgfusepath{clip}%
\pgfsetbuttcap%
\pgfsetroundjoin%
\definecolor{currentfill}{rgb}{0.565498,0.842430,0.262877}%
\pgfsetfillcolor{currentfill}%
\pgfsetfillopacity{0.700000}%
\pgfsetlinewidth{0.501875pt}%
\definecolor{currentstroke}{rgb}{1.000000,1.000000,1.000000}%
\pgfsetstrokecolor{currentstroke}%
\pgfsetstrokeopacity{0.500000}%
\pgfsetdash{}{0pt}%
\pgfpathmoveto{\pgfqpoint{3.862289in}{3.281584in}}%
\pgfpathlineto{\pgfqpoint{3.873292in}{3.284613in}}%
\pgfpathlineto{\pgfqpoint{3.884289in}{3.287597in}}%
\pgfpathlineto{\pgfqpoint{3.895281in}{3.290555in}}%
\pgfpathlineto{\pgfqpoint{3.906267in}{3.293506in}}%
\pgfpathlineto{\pgfqpoint{3.917247in}{3.296469in}}%
\pgfpathlineto{\pgfqpoint{3.910904in}{3.308000in}}%
\pgfpathlineto{\pgfqpoint{3.904564in}{3.319510in}}%
\pgfpathlineto{\pgfqpoint{3.898227in}{3.331014in}}%
\pgfpathlineto{\pgfqpoint{3.891893in}{3.342525in}}%
\pgfpathlineto{\pgfqpoint{3.885562in}{3.354054in}}%
\pgfpathlineto{\pgfqpoint{3.874586in}{3.351049in}}%
\pgfpathlineto{\pgfqpoint{3.863605in}{3.348106in}}%
\pgfpathlineto{\pgfqpoint{3.852620in}{3.345200in}}%
\pgfpathlineto{\pgfqpoint{3.841630in}{3.342308in}}%
\pgfpathlineto{\pgfqpoint{3.830634in}{3.339404in}}%
\pgfpathlineto{\pgfqpoint{3.836963in}{3.328054in}}%
\pgfpathlineto{\pgfqpoint{3.843292in}{3.316584in}}%
\pgfpathlineto{\pgfqpoint{3.849623in}{3.305007in}}%
\pgfpathlineto{\pgfqpoint{3.855955in}{3.293336in}}%
\pgfpathclose%
\pgfusepath{stroke,fill}%
\end{pgfscope}%
\begin{pgfscope}%
\pgfpathrectangle{\pgfqpoint{0.887500in}{0.275000in}}{\pgfqpoint{4.225000in}{4.225000in}}%
\pgfusepath{clip}%
\pgfsetbuttcap%
\pgfsetroundjoin%
\definecolor{currentfill}{rgb}{0.814576,0.883393,0.110347}%
\pgfsetfillcolor{currentfill}%
\pgfsetfillopacity{0.700000}%
\pgfsetlinewidth{0.501875pt}%
\definecolor{currentstroke}{rgb}{1.000000,1.000000,1.000000}%
\pgfsetstrokecolor{currentstroke}%
\pgfsetstrokeopacity{0.500000}%
\pgfsetdash{}{0pt}%
\pgfpathmoveto{\pgfqpoint{3.285860in}{3.476554in}}%
\pgfpathlineto{\pgfqpoint{3.297014in}{3.480622in}}%
\pgfpathlineto{\pgfqpoint{3.308162in}{3.484660in}}%
\pgfpathlineto{\pgfqpoint{3.319306in}{3.488634in}}%
\pgfpathlineto{\pgfqpoint{3.330444in}{3.492509in}}%
\pgfpathlineto{\pgfqpoint{3.341576in}{3.496250in}}%
\pgfpathlineto{\pgfqpoint{3.335340in}{3.503471in}}%
\pgfpathlineto{\pgfqpoint{3.329106in}{3.510370in}}%
\pgfpathlineto{\pgfqpoint{3.322875in}{3.516970in}}%
\pgfpathlineto{\pgfqpoint{3.316647in}{3.523294in}}%
\pgfpathlineto{\pgfqpoint{3.310421in}{3.529366in}}%
\pgfpathlineto{\pgfqpoint{3.299305in}{3.526082in}}%
\pgfpathlineto{\pgfqpoint{3.288182in}{3.522598in}}%
\pgfpathlineto{\pgfqpoint{3.277053in}{3.518964in}}%
\pgfpathlineto{\pgfqpoint{3.265919in}{3.515226in}}%
\pgfpathlineto{\pgfqpoint{3.254779in}{3.511434in}}%
\pgfpathlineto{\pgfqpoint{3.260989in}{3.504972in}}%
\pgfpathlineto{\pgfqpoint{3.267203in}{3.498268in}}%
\pgfpathlineto{\pgfqpoint{3.273419in}{3.491307in}}%
\pgfpathlineto{\pgfqpoint{3.279638in}{3.484074in}}%
\pgfpathclose%
\pgfusepath{stroke,fill}%
\end{pgfscope}%
\begin{pgfscope}%
\pgfpathrectangle{\pgfqpoint{0.887500in}{0.275000in}}{\pgfqpoint{4.225000in}{4.225000in}}%
\pgfusepath{clip}%
\pgfsetbuttcap%
\pgfsetroundjoin%
\definecolor{currentfill}{rgb}{0.126453,0.570633,0.549841}%
\pgfsetfillcolor{currentfill}%
\pgfsetfillopacity{0.700000}%
\pgfsetlinewidth{0.501875pt}%
\definecolor{currentstroke}{rgb}{1.000000,1.000000,1.000000}%
\pgfsetstrokecolor{currentstroke}%
\pgfsetstrokeopacity{0.500000}%
\pgfsetdash{}{0pt}%
\pgfpathmoveto{\pgfqpoint{2.185017in}{2.635271in}}%
\pgfpathlineto{\pgfqpoint{2.196430in}{2.638741in}}%
\pgfpathlineto{\pgfqpoint{2.207836in}{2.642241in}}%
\pgfpathlineto{\pgfqpoint{2.219237in}{2.645741in}}%
\pgfpathlineto{\pgfqpoint{2.230633in}{2.649209in}}%
\pgfpathlineto{\pgfqpoint{2.242025in}{2.652613in}}%
\pgfpathlineto{\pgfqpoint{2.236156in}{2.660785in}}%
\pgfpathlineto{\pgfqpoint{2.230290in}{2.668945in}}%
\pgfpathlineto{\pgfqpoint{2.224429in}{2.677091in}}%
\pgfpathlineto{\pgfqpoint{2.218571in}{2.685222in}}%
\pgfpathlineto{\pgfqpoint{2.212718in}{2.693338in}}%
\pgfpathlineto{\pgfqpoint{2.201339in}{2.689926in}}%
\pgfpathlineto{\pgfqpoint{2.189956in}{2.686393in}}%
\pgfpathlineto{\pgfqpoint{2.178570in}{2.682795in}}%
\pgfpathlineto{\pgfqpoint{2.167178in}{2.679183in}}%
\pgfpathlineto{\pgfqpoint{2.155780in}{2.675613in}}%
\pgfpathlineto{\pgfqpoint{2.161620in}{2.667571in}}%
\pgfpathlineto{\pgfqpoint{2.167463in}{2.659515in}}%
\pgfpathlineto{\pgfqpoint{2.173310in}{2.651447in}}%
\pgfpathlineto{\pgfqpoint{2.179162in}{2.643366in}}%
\pgfpathclose%
\pgfusepath{stroke,fill}%
\end{pgfscope}%
\begin{pgfscope}%
\pgfpathrectangle{\pgfqpoint{0.887500in}{0.275000in}}{\pgfqpoint{4.225000in}{4.225000in}}%
\pgfusepath{clip}%
\pgfsetbuttcap%
\pgfsetroundjoin%
\definecolor{currentfill}{rgb}{0.119512,0.607464,0.540218}%
\pgfsetfillcolor{currentfill}%
\pgfsetfillopacity{0.700000}%
\pgfsetlinewidth{0.501875pt}%
\definecolor{currentstroke}{rgb}{1.000000,1.000000,1.000000}%
\pgfsetstrokecolor{currentstroke}%
\pgfsetstrokeopacity{0.500000}%
\pgfsetdash{}{0pt}%
\pgfpathmoveto{\pgfqpoint{1.639003in}{2.716997in}}%
\pgfpathlineto{\pgfqpoint{1.650552in}{2.720277in}}%
\pgfpathlineto{\pgfqpoint{1.662095in}{2.723555in}}%
\pgfpathlineto{\pgfqpoint{1.673633in}{2.726830in}}%
\pgfpathlineto{\pgfqpoint{1.685166in}{2.730103in}}%
\pgfpathlineto{\pgfqpoint{1.696693in}{2.733374in}}%
\pgfpathlineto{\pgfqpoint{1.691020in}{2.741026in}}%
\pgfpathlineto{\pgfqpoint{1.685351in}{2.748662in}}%
\pgfpathlineto{\pgfqpoint{1.679686in}{2.756283in}}%
\pgfpathlineto{\pgfqpoint{1.674026in}{2.763889in}}%
\pgfpathlineto{\pgfqpoint{1.668369in}{2.771479in}}%
\pgfpathlineto{\pgfqpoint{1.656856in}{2.768197in}}%
\pgfpathlineto{\pgfqpoint{1.645337in}{2.764913in}}%
\pgfpathlineto{\pgfqpoint{1.633813in}{2.761626in}}%
\pgfpathlineto{\pgfqpoint{1.622284in}{2.758336in}}%
\pgfpathlineto{\pgfqpoint{1.610748in}{2.755041in}}%
\pgfpathlineto{\pgfqpoint{1.616391in}{2.747464in}}%
\pgfpathlineto{\pgfqpoint{1.622037in}{2.739871in}}%
\pgfpathlineto{\pgfqpoint{1.627688in}{2.732261in}}%
\pgfpathlineto{\pgfqpoint{1.633343in}{2.724637in}}%
\pgfpathclose%
\pgfusepath{stroke,fill}%
\end{pgfscope}%
\begin{pgfscope}%
\pgfpathrectangle{\pgfqpoint{0.887500in}{0.275000in}}{\pgfqpoint{4.225000in}{4.225000in}}%
\pgfusepath{clip}%
\pgfsetbuttcap%
\pgfsetroundjoin%
\definecolor{currentfill}{rgb}{0.120081,0.622161,0.534946}%
\pgfsetfillcolor{currentfill}%
\pgfsetfillopacity{0.700000}%
\pgfsetlinewidth{0.501875pt}%
\definecolor{currentstroke}{rgb}{1.000000,1.000000,1.000000}%
\pgfsetstrokecolor{currentstroke}%
\pgfsetstrokeopacity{0.500000}%
\pgfsetdash{}{0pt}%
\pgfpathmoveto{\pgfqpoint{1.409120in}{2.742813in}}%
\pgfpathlineto{\pgfqpoint{1.420724in}{2.746152in}}%
\pgfpathlineto{\pgfqpoint{1.432323in}{2.749485in}}%
\pgfpathlineto{\pgfqpoint{1.443917in}{2.752814in}}%
\pgfpathlineto{\pgfqpoint{1.455504in}{2.756139in}}%
\pgfpathlineto{\pgfqpoint{1.467087in}{2.759462in}}%
\pgfpathlineto{\pgfqpoint{1.461499in}{2.766902in}}%
\pgfpathlineto{\pgfqpoint{1.455916in}{2.774323in}}%
\pgfpathlineto{\pgfqpoint{1.450337in}{2.781727in}}%
\pgfpathlineto{\pgfqpoint{1.444763in}{2.789111in}}%
\pgfpathlineto{\pgfqpoint{1.439193in}{2.796477in}}%
\pgfpathlineto{\pgfqpoint{1.427626in}{2.793130in}}%
\pgfpathlineto{\pgfqpoint{1.416053in}{2.789780in}}%
\pgfpathlineto{\pgfqpoint{1.404475in}{2.786424in}}%
\pgfpathlineto{\pgfqpoint{1.392891in}{2.783064in}}%
\pgfpathlineto{\pgfqpoint{1.381302in}{2.779698in}}%
\pgfpathlineto{\pgfqpoint{1.386856in}{2.772364in}}%
\pgfpathlineto{\pgfqpoint{1.392415in}{2.765008in}}%
\pgfpathlineto{\pgfqpoint{1.397979in}{2.757631in}}%
\pgfpathlineto{\pgfqpoint{1.403547in}{2.750232in}}%
\pgfpathclose%
\pgfusepath{stroke,fill}%
\end{pgfscope}%
\begin{pgfscope}%
\pgfpathrectangle{\pgfqpoint{0.887500in}{0.275000in}}{\pgfqpoint{4.225000in}{4.225000in}}%
\pgfusepath{clip}%
\pgfsetbuttcap%
\pgfsetroundjoin%
\definecolor{currentfill}{rgb}{0.135066,0.544853,0.554029}%
\pgfsetfillcolor{currentfill}%
\pgfsetfillopacity{0.700000}%
\pgfsetlinewidth{0.501875pt}%
\definecolor{currentstroke}{rgb}{1.000000,1.000000,1.000000}%
\pgfsetstrokecolor{currentstroke}%
\pgfsetstrokeopacity{0.500000}%
\pgfsetdash{}{0pt}%
\pgfpathmoveto{\pgfqpoint{2.501322in}{2.578103in}}%
\pgfpathlineto{\pgfqpoint{2.512652in}{2.582133in}}%
\pgfpathlineto{\pgfqpoint{2.523970in}{2.586739in}}%
\pgfpathlineto{\pgfqpoint{2.535274in}{2.592027in}}%
\pgfpathlineto{\pgfqpoint{2.546563in}{2.598106in}}%
\pgfpathlineto{\pgfqpoint{2.557837in}{2.605082in}}%
\pgfpathlineto{\pgfqpoint{2.551864in}{2.613086in}}%
\pgfpathlineto{\pgfqpoint{2.545893in}{2.621271in}}%
\pgfpathlineto{\pgfqpoint{2.539922in}{2.629707in}}%
\pgfpathlineto{\pgfqpoint{2.533951in}{2.638461in}}%
\pgfpathlineto{\pgfqpoint{2.527979in}{2.647602in}}%
\pgfpathlineto{\pgfqpoint{2.516722in}{2.640367in}}%
\pgfpathlineto{\pgfqpoint{2.505449in}{2.634047in}}%
\pgfpathlineto{\pgfqpoint{2.494161in}{2.628544in}}%
\pgfpathlineto{\pgfqpoint{2.482858in}{2.623759in}}%
\pgfpathlineto{\pgfqpoint{2.471542in}{2.619593in}}%
\pgfpathlineto{\pgfqpoint{2.477495in}{2.610918in}}%
\pgfpathlineto{\pgfqpoint{2.483450in}{2.602475in}}%
\pgfpathlineto{\pgfqpoint{2.489405in}{2.594221in}}%
\pgfpathlineto{\pgfqpoint{2.495362in}{2.586111in}}%
\pgfpathclose%
\pgfusepath{stroke,fill}%
\end{pgfscope}%
\begin{pgfscope}%
\pgfpathrectangle{\pgfqpoint{0.887500in}{0.275000in}}{\pgfqpoint{4.225000in}{4.225000in}}%
\pgfusepath{clip}%
\pgfsetbuttcap%
\pgfsetroundjoin%
\definecolor{currentfill}{rgb}{0.122606,0.585371,0.546557}%
\pgfsetfillcolor{currentfill}%
\pgfsetfillopacity{0.700000}%
\pgfsetlinewidth{0.501875pt}%
\definecolor{currentstroke}{rgb}{1.000000,1.000000,1.000000}%
\pgfsetstrokecolor{currentstroke}%
\pgfsetstrokeopacity{0.500000}%
\pgfsetdash{}{0pt}%
\pgfpathmoveto{\pgfqpoint{1.955115in}{2.665740in}}%
\pgfpathlineto{\pgfqpoint{1.966588in}{2.669079in}}%
\pgfpathlineto{\pgfqpoint{1.978057in}{2.672420in}}%
\pgfpathlineto{\pgfqpoint{1.989520in}{2.675760in}}%
\pgfpathlineto{\pgfqpoint{2.000977in}{2.679094in}}%
\pgfpathlineto{\pgfqpoint{2.012429in}{2.682418in}}%
\pgfpathlineto{\pgfqpoint{2.006640in}{2.690354in}}%
\pgfpathlineto{\pgfqpoint{2.000854in}{2.698274in}}%
\pgfpathlineto{\pgfqpoint{1.995073in}{2.706177in}}%
\pgfpathlineto{\pgfqpoint{1.989296in}{2.714063in}}%
\pgfpathlineto{\pgfqpoint{1.983524in}{2.721931in}}%
\pgfpathlineto{\pgfqpoint{1.972085in}{2.718593in}}%
\pgfpathlineto{\pgfqpoint{1.960641in}{2.715244in}}%
\pgfpathlineto{\pgfqpoint{1.949191in}{2.711891in}}%
\pgfpathlineto{\pgfqpoint{1.937736in}{2.708539in}}%
\pgfpathlineto{\pgfqpoint{1.926275in}{2.705192in}}%
\pgfpathlineto{\pgfqpoint{1.932034in}{2.697339in}}%
\pgfpathlineto{\pgfqpoint{1.937798in}{2.689467in}}%
\pgfpathlineto{\pgfqpoint{1.943566in}{2.681577in}}%
\pgfpathlineto{\pgfqpoint{1.949338in}{2.673667in}}%
\pgfpathclose%
\pgfusepath{stroke,fill}%
\end{pgfscope}%
\begin{pgfscope}%
\pgfpathrectangle{\pgfqpoint{0.887500in}{0.275000in}}{\pgfqpoint{4.225000in}{4.225000in}}%
\pgfusepath{clip}%
\pgfsetbuttcap%
\pgfsetroundjoin%
\definecolor{currentfill}{rgb}{0.814576,0.883393,0.110347}%
\pgfsetfillcolor{currentfill}%
\pgfsetfillopacity{0.700000}%
\pgfsetlinewidth{0.501875pt}%
\definecolor{currentstroke}{rgb}{1.000000,1.000000,1.000000}%
\pgfsetstrokecolor{currentstroke}%
\pgfsetstrokeopacity{0.500000}%
\pgfsetdash{}{0pt}%
\pgfpathmoveto{\pgfqpoint{3.143115in}{3.476838in}}%
\pgfpathlineto{\pgfqpoint{3.154303in}{3.479789in}}%
\pgfpathlineto{\pgfqpoint{3.165487in}{3.482891in}}%
\pgfpathlineto{\pgfqpoint{3.176666in}{3.486130in}}%
\pgfpathlineto{\pgfqpoint{3.187839in}{3.489490in}}%
\pgfpathlineto{\pgfqpoint{3.199008in}{3.492958in}}%
\pgfpathlineto{\pgfqpoint{3.192816in}{3.499515in}}%
\pgfpathlineto{\pgfqpoint{3.186628in}{3.505886in}}%
\pgfpathlineto{\pgfqpoint{3.180445in}{3.512067in}}%
\pgfpathlineto{\pgfqpoint{3.174265in}{3.518027in}}%
\pgfpathlineto{\pgfqpoint{3.168089in}{3.523735in}}%
\pgfpathlineto{\pgfqpoint{3.156933in}{3.520144in}}%
\pgfpathlineto{\pgfqpoint{3.145772in}{3.516375in}}%
\pgfpathlineto{\pgfqpoint{3.134606in}{3.512389in}}%
\pgfpathlineto{\pgfqpoint{3.123434in}{3.508200in}}%
\pgfpathlineto{\pgfqpoint{3.112257in}{3.503936in}}%
\pgfpathlineto{\pgfqpoint{3.118420in}{3.499201in}}%
\pgfpathlineto{\pgfqpoint{3.124587in}{3.494131in}}%
\pgfpathlineto{\pgfqpoint{3.130759in}{3.488718in}}%
\pgfpathlineto{\pgfqpoint{3.136935in}{3.482953in}}%
\pgfpathclose%
\pgfusepath{stroke,fill}%
\end{pgfscope}%
\begin{pgfscope}%
\pgfpathrectangle{\pgfqpoint{0.887500in}{0.275000in}}{\pgfqpoint{4.225000in}{4.225000in}}%
\pgfusepath{clip}%
\pgfsetbuttcap%
\pgfsetroundjoin%
\definecolor{currentfill}{rgb}{0.129933,0.559582,0.551864}%
\pgfsetfillcolor{currentfill}%
\pgfsetfillopacity{0.700000}%
\pgfsetlinewidth{0.501875pt}%
\definecolor{currentstroke}{rgb}{1.000000,1.000000,1.000000}%
\pgfsetstrokecolor{currentstroke}%
\pgfsetstrokeopacity{0.500000}%
\pgfsetdash{}{0pt}%
\pgfpathmoveto{\pgfqpoint{2.271431in}{2.611608in}}%
\pgfpathlineto{\pgfqpoint{2.282829in}{2.615022in}}%
\pgfpathlineto{\pgfqpoint{2.294222in}{2.618427in}}%
\pgfpathlineto{\pgfqpoint{2.305610in}{2.621820in}}%
\pgfpathlineto{\pgfqpoint{2.316992in}{2.625210in}}%
\pgfpathlineto{\pgfqpoint{2.328369in}{2.628608in}}%
\pgfpathlineto{\pgfqpoint{2.322472in}{2.636586in}}%
\pgfpathlineto{\pgfqpoint{2.316580in}{2.644544in}}%
\pgfpathlineto{\pgfqpoint{2.310691in}{2.652487in}}%
\pgfpathlineto{\pgfqpoint{2.304807in}{2.660421in}}%
\pgfpathlineto{\pgfqpoint{2.298927in}{2.668351in}}%
\pgfpathlineto{\pgfqpoint{2.287557in}{2.665224in}}%
\pgfpathlineto{\pgfqpoint{2.276180in}{2.662177in}}%
\pgfpathlineto{\pgfqpoint{2.264799in}{2.659108in}}%
\pgfpathlineto{\pgfqpoint{2.253413in}{2.655923in}}%
\pgfpathlineto{\pgfqpoint{2.242025in}{2.652613in}}%
\pgfpathlineto{\pgfqpoint{2.247898in}{2.644430in}}%
\pgfpathlineto{\pgfqpoint{2.253776in}{2.636237in}}%
\pgfpathlineto{\pgfqpoint{2.259657in}{2.628035in}}%
\pgfpathlineto{\pgfqpoint{2.265542in}{2.619825in}}%
\pgfpathclose%
\pgfusepath{stroke,fill}%
\end{pgfscope}%
\begin{pgfscope}%
\pgfpathrectangle{\pgfqpoint{0.887500in}{0.275000in}}{\pgfqpoint{4.225000in}{4.225000in}}%
\pgfusepath{clip}%
\pgfsetbuttcap%
\pgfsetroundjoin%
\definecolor{currentfill}{rgb}{0.804182,0.882046,0.114965}%
\pgfsetfillcolor{currentfill}%
\pgfsetfillopacity{0.700000}%
\pgfsetlinewidth{0.501875pt}%
\definecolor{currentstroke}{rgb}{1.000000,1.000000,1.000000}%
\pgfsetstrokecolor{currentstroke}%
\pgfsetstrokeopacity{0.500000}%
\pgfsetdash{}{0pt}%
\pgfpathmoveto{\pgfqpoint{3.372777in}{3.454534in}}%
\pgfpathlineto{\pgfqpoint{3.383916in}{3.458497in}}%
\pgfpathlineto{\pgfqpoint{3.395050in}{3.462391in}}%
\pgfpathlineto{\pgfqpoint{3.406178in}{3.466222in}}%
\pgfpathlineto{\pgfqpoint{3.417301in}{3.469998in}}%
\pgfpathlineto{\pgfqpoint{3.428419in}{3.473729in}}%
\pgfpathlineto{\pgfqpoint{3.422164in}{3.482603in}}%
\pgfpathlineto{\pgfqpoint{3.415909in}{3.490903in}}%
\pgfpathlineto{\pgfqpoint{3.409653in}{3.498657in}}%
\pgfpathlineto{\pgfqpoint{3.403397in}{3.505926in}}%
\pgfpathlineto{\pgfqpoint{3.397142in}{3.512772in}}%
\pgfpathlineto{\pgfqpoint{3.386041in}{3.509629in}}%
\pgfpathlineto{\pgfqpoint{3.374934in}{3.506463in}}%
\pgfpathlineto{\pgfqpoint{3.363821in}{3.503214in}}%
\pgfpathlineto{\pgfqpoint{3.352701in}{3.499824in}}%
\pgfpathlineto{\pgfqpoint{3.341576in}{3.496250in}}%
\pgfpathlineto{\pgfqpoint{3.347813in}{3.488685in}}%
\pgfpathlineto{\pgfqpoint{3.354053in}{3.480752in}}%
\pgfpathlineto{\pgfqpoint{3.360293in}{3.472428in}}%
\pgfpathlineto{\pgfqpoint{3.366535in}{3.463691in}}%
\pgfpathclose%
\pgfusepath{stroke,fill}%
\end{pgfscope}%
\begin{pgfscope}%
\pgfpathrectangle{\pgfqpoint{0.887500in}{0.275000in}}{\pgfqpoint{4.225000in}{4.225000in}}%
\pgfusepath{clip}%
\pgfsetbuttcap%
\pgfsetroundjoin%
\definecolor{currentfill}{rgb}{0.120092,0.600104,0.542530}%
\pgfsetfillcolor{currentfill}%
\pgfsetfillopacity{0.700000}%
\pgfsetlinewidth{0.501875pt}%
\definecolor{currentstroke}{rgb}{1.000000,1.000000,1.000000}%
\pgfsetstrokecolor{currentstroke}%
\pgfsetstrokeopacity{0.500000}%
\pgfsetdash{}{0pt}%
\pgfpathmoveto{\pgfqpoint{1.725122in}{2.694892in}}%
\pgfpathlineto{\pgfqpoint{1.736657in}{2.698150in}}%
\pgfpathlineto{\pgfqpoint{1.748186in}{2.701407in}}%
\pgfpathlineto{\pgfqpoint{1.759710in}{2.704663in}}%
\pgfpathlineto{\pgfqpoint{1.771228in}{2.707919in}}%
\pgfpathlineto{\pgfqpoint{1.782740in}{2.711178in}}%
\pgfpathlineto{\pgfqpoint{1.777032in}{2.718918in}}%
\pgfpathlineto{\pgfqpoint{1.771328in}{2.726641in}}%
\pgfpathlineto{\pgfqpoint{1.765629in}{2.734350in}}%
\pgfpathlineto{\pgfqpoint{1.759934in}{2.742043in}}%
\pgfpathlineto{\pgfqpoint{1.754243in}{2.749721in}}%
\pgfpathlineto{\pgfqpoint{1.742744in}{2.746450in}}%
\pgfpathlineto{\pgfqpoint{1.731240in}{2.743181in}}%
\pgfpathlineto{\pgfqpoint{1.719730in}{2.739913in}}%
\pgfpathlineto{\pgfqpoint{1.708214in}{2.736644in}}%
\pgfpathlineto{\pgfqpoint{1.696693in}{2.733374in}}%
\pgfpathlineto{\pgfqpoint{1.702370in}{2.725708in}}%
\pgfpathlineto{\pgfqpoint{1.708052in}{2.718027in}}%
\pgfpathlineto{\pgfqpoint{1.713737in}{2.710331in}}%
\pgfpathlineto{\pgfqpoint{1.719427in}{2.702619in}}%
\pgfpathclose%
\pgfusepath{stroke,fill}%
\end{pgfscope}%
\begin{pgfscope}%
\pgfpathrectangle{\pgfqpoint{0.887500in}{0.275000in}}{\pgfqpoint{4.225000in}{4.225000in}}%
\pgfusepath{clip}%
\pgfsetbuttcap%
\pgfsetroundjoin%
\definecolor{currentfill}{rgb}{0.626579,0.854645,0.223353}%
\pgfsetfillcolor{currentfill}%
\pgfsetfillopacity{0.700000}%
\pgfsetlinewidth{0.501875pt}%
\definecolor{currentstroke}{rgb}{1.000000,1.000000,1.000000}%
\pgfsetstrokecolor{currentstroke}%
\pgfsetstrokeopacity{0.500000}%
\pgfsetdash{}{0pt}%
\pgfpathmoveto{\pgfqpoint{2.888407in}{3.253877in}}%
\pgfpathlineto{\pgfqpoint{2.899468in}{3.300660in}}%
\pgfpathlineto{\pgfqpoint{2.910569in}{3.341598in}}%
\pgfpathlineto{\pgfqpoint{2.921710in}{3.375364in}}%
\pgfpathlineto{\pgfqpoint{2.932882in}{3.402595in}}%
\pgfpathlineto{\pgfqpoint{2.944078in}{3.424104in}}%
\pgfpathlineto{\pgfqpoint{2.937973in}{3.428273in}}%
\pgfpathlineto{\pgfqpoint{2.931874in}{3.432330in}}%
\pgfpathlineto{\pgfqpoint{2.925780in}{3.436326in}}%
\pgfpathlineto{\pgfqpoint{2.919691in}{3.440314in}}%
\pgfpathlineto{\pgfqpoint{2.913607in}{3.444346in}}%
\pgfpathlineto{\pgfqpoint{2.902441in}{3.422933in}}%
\pgfpathlineto{\pgfqpoint{2.891305in}{3.395457in}}%
\pgfpathlineto{\pgfqpoint{2.880206in}{3.361060in}}%
\pgfpathlineto{\pgfqpoint{2.869155in}{3.319072in}}%
\pgfpathlineto{\pgfqpoint{2.858151in}{3.270921in}}%
\pgfpathlineto{\pgfqpoint{2.864201in}{3.265606in}}%
\pgfpathlineto{\pgfqpoint{2.870250in}{3.261434in}}%
\pgfpathlineto{\pgfqpoint{2.876300in}{3.258215in}}%
\pgfpathlineto{\pgfqpoint{2.882352in}{3.255759in}}%
\pgfpathclose%
\pgfusepath{stroke,fill}%
\end{pgfscope}%
\begin{pgfscope}%
\pgfpathrectangle{\pgfqpoint{0.887500in}{0.275000in}}{\pgfqpoint{4.225000in}{4.225000in}}%
\pgfusepath{clip}%
\pgfsetbuttcap%
\pgfsetroundjoin%
\definecolor{currentfill}{rgb}{0.120638,0.625828,0.533488}%
\pgfsetfillcolor{currentfill}%
\pgfsetfillopacity{0.700000}%
\pgfsetlinewidth{0.501875pt}%
\definecolor{currentstroke}{rgb}{1.000000,1.000000,1.000000}%
\pgfsetstrokecolor{currentstroke}%
\pgfsetstrokeopacity{0.500000}%
\pgfsetdash{}{0pt}%
\pgfpathmoveto{\pgfqpoint{2.756307in}{2.719596in}}%
\pgfpathlineto{\pgfqpoint{2.767519in}{2.731608in}}%
\pgfpathlineto{\pgfqpoint{2.778719in}{2.745343in}}%
\pgfpathlineto{\pgfqpoint{2.789907in}{2.760779in}}%
\pgfpathlineto{\pgfqpoint{2.801087in}{2.777782in}}%
\pgfpathlineto{\pgfqpoint{2.812260in}{2.796220in}}%
\pgfpathlineto{\pgfqpoint{2.806199in}{2.803716in}}%
\pgfpathlineto{\pgfqpoint{2.800138in}{2.811868in}}%
\pgfpathlineto{\pgfqpoint{2.794078in}{2.820580in}}%
\pgfpathlineto{\pgfqpoint{2.788018in}{2.829738in}}%
\pgfpathlineto{\pgfqpoint{2.781959in}{2.839226in}}%
\pgfpathlineto{\pgfqpoint{2.770792in}{2.822967in}}%
\pgfpathlineto{\pgfqpoint{2.759622in}{2.807434in}}%
\pgfpathlineto{\pgfqpoint{2.748445in}{2.792861in}}%
\pgfpathlineto{\pgfqpoint{2.737260in}{2.779483in}}%
\pgfpathlineto{\pgfqpoint{2.726065in}{2.767419in}}%
\pgfpathlineto{\pgfqpoint{2.732112in}{2.757178in}}%
\pgfpathlineto{\pgfqpoint{2.738160in}{2.747223in}}%
\pgfpathlineto{\pgfqpoint{2.744208in}{2.737607in}}%
\pgfpathlineto{\pgfqpoint{2.750258in}{2.728385in}}%
\pgfpathclose%
\pgfusepath{stroke,fill}%
\end{pgfscope}%
\begin{pgfscope}%
\pgfpathrectangle{\pgfqpoint{0.887500in}{0.275000in}}{\pgfqpoint{4.225000in}{4.225000in}}%
\pgfusepath{clip}%
\pgfsetbuttcap%
\pgfsetroundjoin%
\definecolor{currentfill}{rgb}{0.804182,0.882046,0.114965}%
\pgfsetfillcolor{currentfill}%
\pgfsetfillopacity{0.700000}%
\pgfsetlinewidth{0.501875pt}%
\definecolor{currentstroke}{rgb}{1.000000,1.000000,1.000000}%
\pgfsetstrokecolor{currentstroke}%
\pgfsetstrokeopacity{0.500000}%
\pgfsetdash{}{0pt}%
\pgfpathmoveto{\pgfqpoint{3.000210in}{3.474397in}}%
\pgfpathlineto{\pgfqpoint{3.011439in}{3.478032in}}%
\pgfpathlineto{\pgfqpoint{3.022663in}{3.480659in}}%
\pgfpathlineto{\pgfqpoint{3.033882in}{3.482673in}}%
\pgfpathlineto{\pgfqpoint{3.045096in}{3.484468in}}%
\pgfpathlineto{\pgfqpoint{3.056303in}{3.486440in}}%
\pgfpathlineto{\pgfqpoint{3.050161in}{3.489453in}}%
\pgfpathlineto{\pgfqpoint{3.044024in}{3.492196in}}%
\pgfpathlineto{\pgfqpoint{3.037894in}{3.494740in}}%
\pgfpathlineto{\pgfqpoint{3.031769in}{3.497159in}}%
\pgfpathlineto{\pgfqpoint{3.025649in}{3.499522in}}%
\pgfpathlineto{\pgfqpoint{3.014460in}{3.497553in}}%
\pgfpathlineto{\pgfqpoint{3.003264in}{3.496178in}}%
\pgfpathlineto{\pgfqpoint{2.992061in}{3.494902in}}%
\pgfpathlineto{\pgfqpoint{2.980852in}{3.493231in}}%
\pgfpathlineto{\pgfqpoint{2.969639in}{3.490667in}}%
\pgfpathlineto{\pgfqpoint{2.975742in}{3.487657in}}%
\pgfpathlineto{\pgfqpoint{2.981850in}{3.484581in}}%
\pgfpathlineto{\pgfqpoint{2.987965in}{3.481383in}}%
\pgfpathlineto{\pgfqpoint{2.994084in}{3.478007in}}%
\pgfpathclose%
\pgfusepath{stroke,fill}%
\end{pgfscope}%
\begin{pgfscope}%
\pgfpathrectangle{\pgfqpoint{0.887500in}{0.275000in}}{\pgfqpoint{4.225000in}{4.225000in}}%
\pgfusepath{clip}%
\pgfsetbuttcap%
\pgfsetroundjoin%
\definecolor{currentfill}{rgb}{0.120092,0.600104,0.542530}%
\pgfsetfillcolor{currentfill}%
\pgfsetfillopacity{0.700000}%
\pgfsetlinewidth{0.501875pt}%
\definecolor{currentstroke}{rgb}{1.000000,1.000000,1.000000}%
\pgfsetstrokecolor{currentstroke}%
\pgfsetstrokeopacity{0.500000}%
\pgfsetdash{}{0pt}%
\pgfpathmoveto{\pgfqpoint{2.700130in}{2.671733in}}%
\pgfpathlineto{\pgfqpoint{2.711371in}{2.681082in}}%
\pgfpathlineto{\pgfqpoint{2.722613in}{2.690122in}}%
\pgfpathlineto{\pgfqpoint{2.733852in}{2.699274in}}%
\pgfpathlineto{\pgfqpoint{2.745084in}{2.708958in}}%
\pgfpathlineto{\pgfqpoint{2.756307in}{2.719596in}}%
\pgfpathlineto{\pgfqpoint{2.750258in}{2.728385in}}%
\pgfpathlineto{\pgfqpoint{2.744208in}{2.737607in}}%
\pgfpathlineto{\pgfqpoint{2.738160in}{2.747223in}}%
\pgfpathlineto{\pgfqpoint{2.732112in}{2.757178in}}%
\pgfpathlineto{\pgfqpoint{2.726065in}{2.767419in}}%
\pgfpathlineto{\pgfqpoint{2.714859in}{2.756400in}}%
\pgfpathlineto{\pgfqpoint{2.703647in}{2.746083in}}%
\pgfpathlineto{\pgfqpoint{2.692430in}{2.736125in}}%
\pgfpathlineto{\pgfqpoint{2.681211in}{2.726180in}}%
\pgfpathlineto{\pgfqpoint{2.669993in}{2.715907in}}%
\pgfpathlineto{\pgfqpoint{2.676014in}{2.707024in}}%
\pgfpathlineto{\pgfqpoint{2.682037in}{2.698226in}}%
\pgfpathlineto{\pgfqpoint{2.688064in}{2.689451in}}%
\pgfpathlineto{\pgfqpoint{2.694095in}{2.680639in}}%
\pgfpathclose%
\pgfusepath{stroke,fill}%
\end{pgfscope}%
\begin{pgfscope}%
\pgfpathrectangle{\pgfqpoint{0.887500in}{0.275000in}}{\pgfqpoint{4.225000in}{4.225000in}}%
\pgfusepath{clip}%
\pgfsetbuttcap%
\pgfsetroundjoin%
\definecolor{currentfill}{rgb}{0.119483,0.614817,0.537692}%
\pgfsetfillcolor{currentfill}%
\pgfsetfillopacity{0.700000}%
\pgfsetlinewidth{0.501875pt}%
\definecolor{currentstroke}{rgb}{1.000000,1.000000,1.000000}%
\pgfsetstrokecolor{currentstroke}%
\pgfsetstrokeopacity{0.500000}%
\pgfsetdash{}{0pt}%
\pgfpathmoveto{\pgfqpoint{1.495090in}{2.721995in}}%
\pgfpathlineto{\pgfqpoint{1.506682in}{2.725301in}}%
\pgfpathlineto{\pgfqpoint{1.518267in}{2.728605in}}%
\pgfpathlineto{\pgfqpoint{1.529847in}{2.731910in}}%
\pgfpathlineto{\pgfqpoint{1.541421in}{2.735215in}}%
\pgfpathlineto{\pgfqpoint{1.552990in}{2.738522in}}%
\pgfpathlineto{\pgfqpoint{1.547366in}{2.746065in}}%
\pgfpathlineto{\pgfqpoint{1.541747in}{2.753592in}}%
\pgfpathlineto{\pgfqpoint{1.536131in}{2.761101in}}%
\pgfpathlineto{\pgfqpoint{1.530521in}{2.768594in}}%
\pgfpathlineto{\pgfqpoint{1.524914in}{2.776069in}}%
\pgfpathlineto{\pgfqpoint{1.513360in}{2.772746in}}%
\pgfpathlineto{\pgfqpoint{1.501800in}{2.769424in}}%
\pgfpathlineto{\pgfqpoint{1.490235in}{2.766103in}}%
\pgfpathlineto{\pgfqpoint{1.478664in}{2.762783in}}%
\pgfpathlineto{\pgfqpoint{1.467087in}{2.759462in}}%
\pgfpathlineto{\pgfqpoint{1.472679in}{2.752004in}}%
\pgfpathlineto{\pgfqpoint{1.478275in}{2.744528in}}%
\pgfpathlineto{\pgfqpoint{1.483876in}{2.737034in}}%
\pgfpathlineto{\pgfqpoint{1.489481in}{2.729523in}}%
\pgfpathclose%
\pgfusepath{stroke,fill}%
\end{pgfscope}%
\begin{pgfscope}%
\pgfpathrectangle{\pgfqpoint{0.887500in}{0.275000in}}{\pgfqpoint{4.225000in}{4.225000in}}%
\pgfusepath{clip}%
\pgfsetbuttcap%
\pgfsetroundjoin%
\definecolor{currentfill}{rgb}{0.124395,0.578002,0.548287}%
\pgfsetfillcolor{currentfill}%
\pgfsetfillopacity{0.700000}%
\pgfsetlinewidth{0.501875pt}%
\definecolor{currentstroke}{rgb}{1.000000,1.000000,1.000000}%
\pgfsetstrokecolor{currentstroke}%
\pgfsetstrokeopacity{0.500000}%
\pgfsetdash{}{0pt}%
\pgfpathmoveto{\pgfqpoint{2.041438in}{2.642506in}}%
\pgfpathlineto{\pgfqpoint{2.052898in}{2.645810in}}%
\pgfpathlineto{\pgfqpoint{2.064352in}{2.649098in}}%
\pgfpathlineto{\pgfqpoint{2.075801in}{2.652366in}}%
\pgfpathlineto{\pgfqpoint{2.087245in}{2.655614in}}%
\pgfpathlineto{\pgfqpoint{2.098684in}{2.658857in}}%
\pgfpathlineto{\pgfqpoint{2.092861in}{2.666863in}}%
\pgfpathlineto{\pgfqpoint{2.087043in}{2.674857in}}%
\pgfpathlineto{\pgfqpoint{2.081229in}{2.682838in}}%
\pgfpathlineto{\pgfqpoint{2.075419in}{2.690806in}}%
\pgfpathlineto{\pgfqpoint{2.069613in}{2.698761in}}%
\pgfpathlineto{\pgfqpoint{2.058186in}{2.695525in}}%
\pgfpathlineto{\pgfqpoint{2.046754in}{2.692283in}}%
\pgfpathlineto{\pgfqpoint{2.035318in}{2.689016in}}%
\pgfpathlineto{\pgfqpoint{2.023876in}{2.685726in}}%
\pgfpathlineto{\pgfqpoint{2.012429in}{2.682418in}}%
\pgfpathlineto{\pgfqpoint{2.018222in}{2.674465in}}%
\pgfpathlineto{\pgfqpoint{2.024020in}{2.666497in}}%
\pgfpathlineto{\pgfqpoint{2.029822in}{2.658514in}}%
\pgfpathlineto{\pgfqpoint{2.035628in}{2.650517in}}%
\pgfpathclose%
\pgfusepath{stroke,fill}%
\end{pgfscope}%
\begin{pgfscope}%
\pgfpathrectangle{\pgfqpoint{0.887500in}{0.275000in}}{\pgfqpoint{4.225000in}{4.225000in}}%
\pgfusepath{clip}%
\pgfsetbuttcap%
\pgfsetroundjoin%
\definecolor{currentfill}{rgb}{0.772852,0.877868,0.131109}%
\pgfsetfillcolor{currentfill}%
\pgfsetfillopacity{0.700000}%
\pgfsetlinewidth{0.501875pt}%
\definecolor{currentstroke}{rgb}{1.000000,1.000000,1.000000}%
\pgfsetstrokecolor{currentstroke}%
\pgfsetstrokeopacity{0.500000}%
\pgfsetdash{}{0pt}%
\pgfpathmoveto{\pgfqpoint{3.459681in}{3.422752in}}%
\pgfpathlineto{\pgfqpoint{3.470801in}{3.426420in}}%
\pgfpathlineto{\pgfqpoint{3.481915in}{3.430040in}}%
\pgfpathlineto{\pgfqpoint{3.493024in}{3.433607in}}%
\pgfpathlineto{\pgfqpoint{3.504127in}{3.437121in}}%
\pgfpathlineto{\pgfqpoint{3.515224in}{3.440588in}}%
\pgfpathlineto{\pgfqpoint{3.508965in}{3.451596in}}%
\pgfpathlineto{\pgfqpoint{3.502707in}{3.462323in}}%
\pgfpathlineto{\pgfqpoint{3.496449in}{3.472687in}}%
\pgfpathlineto{\pgfqpoint{3.490189in}{3.482610in}}%
\pgfpathlineto{\pgfqpoint{3.483927in}{3.492012in}}%
\pgfpathlineto{\pgfqpoint{3.472835in}{3.488383in}}%
\pgfpathlineto{\pgfqpoint{3.461739in}{3.484743in}}%
\pgfpathlineto{\pgfqpoint{3.450637in}{3.481093in}}%
\pgfpathlineto{\pgfqpoint{3.439531in}{3.477424in}}%
\pgfpathlineto{\pgfqpoint{3.428419in}{3.473729in}}%
\pgfpathlineto{\pgfqpoint{3.434672in}{3.464331in}}%
\pgfpathlineto{\pgfqpoint{3.440924in}{3.454471in}}%
\pgfpathlineto{\pgfqpoint{3.447176in}{3.444212in}}%
\pgfpathlineto{\pgfqpoint{3.453428in}{3.433619in}}%
\pgfpathclose%
\pgfusepath{stroke,fill}%
\end{pgfscope}%
\begin{pgfscope}%
\pgfpathrectangle{\pgfqpoint{0.887500in}{0.275000in}}{\pgfqpoint{4.225000in}{4.225000in}}%
\pgfusepath{clip}%
\pgfsetbuttcap%
\pgfsetroundjoin%
\definecolor{currentfill}{rgb}{0.395174,0.797475,0.367757}%
\pgfsetfillcolor{currentfill}%
\pgfsetfillopacity{0.700000}%
\pgfsetlinewidth{0.501875pt}%
\definecolor{currentstroke}{rgb}{1.000000,1.000000,1.000000}%
\pgfsetstrokecolor{currentstroke}%
\pgfsetstrokeopacity{0.500000}%
\pgfsetdash{}{0pt}%
\pgfpathmoveto{\pgfqpoint{4.067473in}{3.130356in}}%
\pgfpathlineto{\pgfqpoint{4.078438in}{3.133452in}}%
\pgfpathlineto{\pgfqpoint{4.089397in}{3.136562in}}%
\pgfpathlineto{\pgfqpoint{4.100351in}{3.139684in}}%
\pgfpathlineto{\pgfqpoint{4.111301in}{3.142819in}}%
\pgfpathlineto{\pgfqpoint{4.122245in}{3.145964in}}%
\pgfpathlineto{\pgfqpoint{4.115864in}{3.158215in}}%
\pgfpathlineto{\pgfqpoint{4.109486in}{3.170429in}}%
\pgfpathlineto{\pgfqpoint{4.103109in}{3.182609in}}%
\pgfpathlineto{\pgfqpoint{4.096734in}{3.194761in}}%
\pgfpathlineto{\pgfqpoint{4.090362in}{3.206890in}}%
\pgfpathlineto{\pgfqpoint{4.079431in}{3.204084in}}%
\pgfpathlineto{\pgfqpoint{4.068493in}{3.201250in}}%
\pgfpathlineto{\pgfqpoint{4.057549in}{3.198387in}}%
\pgfpathlineto{\pgfqpoint{4.046599in}{3.195496in}}%
\pgfpathlineto{\pgfqpoint{4.035642in}{3.192577in}}%
\pgfpathlineto{\pgfqpoint{4.042005in}{3.180231in}}%
\pgfpathlineto{\pgfqpoint{4.048369in}{3.167828in}}%
\pgfpathlineto{\pgfqpoint{4.054735in}{3.155376in}}%
\pgfpathlineto{\pgfqpoint{4.061103in}{3.142883in}}%
\pgfpathclose%
\pgfusepath{stroke,fill}%
\end{pgfscope}%
\begin{pgfscope}%
\pgfpathrectangle{\pgfqpoint{0.887500in}{0.275000in}}{\pgfqpoint{4.225000in}{4.225000in}}%
\pgfusepath{clip}%
\pgfsetbuttcap%
\pgfsetroundjoin%
\definecolor{currentfill}{rgb}{0.132444,0.552216,0.553018}%
\pgfsetfillcolor{currentfill}%
\pgfsetfillopacity{0.700000}%
\pgfsetlinewidth{0.501875pt}%
\definecolor{currentstroke}{rgb}{1.000000,1.000000,1.000000}%
\pgfsetstrokecolor{currentstroke}%
\pgfsetstrokeopacity{0.500000}%
\pgfsetdash{}{0pt}%
\pgfpathmoveto{\pgfqpoint{2.357919in}{2.588214in}}%
\pgfpathlineto{\pgfqpoint{2.369300in}{2.591701in}}%
\pgfpathlineto{\pgfqpoint{2.380678in}{2.595072in}}%
\pgfpathlineto{\pgfqpoint{2.392054in}{2.598283in}}%
\pgfpathlineto{\pgfqpoint{2.403427in}{2.601293in}}%
\pgfpathlineto{\pgfqpoint{2.414797in}{2.604126in}}%
\pgfpathlineto{\pgfqpoint{2.408863in}{2.612447in}}%
\pgfpathlineto{\pgfqpoint{2.402933in}{2.620804in}}%
\pgfpathlineto{\pgfqpoint{2.397006in}{2.629199in}}%
\pgfpathlineto{\pgfqpoint{2.391082in}{2.637628in}}%
\pgfpathlineto{\pgfqpoint{2.385161in}{2.646087in}}%
\pgfpathlineto{\pgfqpoint{2.373815in}{2.642478in}}%
\pgfpathlineto{\pgfqpoint{2.362463in}{2.638948in}}%
\pgfpathlineto{\pgfqpoint{2.351104in}{2.635466in}}%
\pgfpathlineto{\pgfqpoint{2.339739in}{2.632023in}}%
\pgfpathlineto{\pgfqpoint{2.328369in}{2.628608in}}%
\pgfpathlineto{\pgfqpoint{2.334270in}{2.620603in}}%
\pgfpathlineto{\pgfqpoint{2.340175in}{2.612566in}}%
\pgfpathlineto{\pgfqpoint{2.346085in}{2.604492in}}%
\pgfpathlineto{\pgfqpoint{2.352000in}{2.596374in}}%
\pgfpathclose%
\pgfusepath{stroke,fill}%
\end{pgfscope}%
\begin{pgfscope}%
\pgfpathrectangle{\pgfqpoint{0.887500in}{0.275000in}}{\pgfqpoint{4.225000in}{4.225000in}}%
\pgfusepath{clip}%
\pgfsetbuttcap%
\pgfsetroundjoin%
\definecolor{currentfill}{rgb}{0.344074,0.780029,0.397381}%
\pgfsetfillcolor{currentfill}%
\pgfsetfillopacity{0.700000}%
\pgfsetlinewidth{0.501875pt}%
\definecolor{currentstroke}{rgb}{1.000000,1.000000,1.000000}%
\pgfsetstrokecolor{currentstroke}%
\pgfsetstrokeopacity{0.500000}%
\pgfsetdash{}{0pt}%
\pgfpathmoveto{\pgfqpoint{4.154173in}{3.083903in}}%
\pgfpathlineto{\pgfqpoint{4.165122in}{3.087279in}}%
\pgfpathlineto{\pgfqpoint{4.176067in}{3.090671in}}%
\pgfpathlineto{\pgfqpoint{4.187007in}{3.094077in}}%
\pgfpathlineto{\pgfqpoint{4.197942in}{3.097497in}}%
\pgfpathlineto{\pgfqpoint{4.191549in}{3.109958in}}%
\pgfpathlineto{\pgfqpoint{4.185155in}{3.122300in}}%
\pgfpathlineto{\pgfqpoint{4.178761in}{3.134526in}}%
\pgfpathlineto{\pgfqpoint{4.172367in}{3.146638in}}%
\pgfpathlineto{\pgfqpoint{4.165973in}{3.158640in}}%
\pgfpathlineto{\pgfqpoint{4.155049in}{3.155458in}}%
\pgfpathlineto{\pgfqpoint{4.144119in}{3.152285in}}%
\pgfpathlineto{\pgfqpoint{4.133185in}{3.149119in}}%
\pgfpathlineto{\pgfqpoint{4.122245in}{3.145964in}}%
\pgfpathlineto{\pgfqpoint{4.128628in}{3.133670in}}%
\pgfpathlineto{\pgfqpoint{4.135013in}{3.121327in}}%
\pgfpathlineto{\pgfqpoint{4.141398in}{3.108924in}}%
\pgfpathlineto{\pgfqpoint{4.147785in}{3.096453in}}%
\pgfpathclose%
\pgfusepath{stroke,fill}%
\end{pgfscope}%
\begin{pgfscope}%
\pgfpathrectangle{\pgfqpoint{0.887500in}{0.275000in}}{\pgfqpoint{4.225000in}{4.225000in}}%
\pgfusepath{clip}%
\pgfsetbuttcap%
\pgfsetroundjoin%
\definecolor{currentfill}{rgb}{0.720391,0.870350,0.162603}%
\pgfsetfillcolor{currentfill}%
\pgfsetfillopacity{0.700000}%
\pgfsetlinewidth{0.501875pt}%
\definecolor{currentstroke}{rgb}{1.000000,1.000000,1.000000}%
\pgfsetstrokecolor{currentstroke}%
\pgfsetstrokeopacity{0.500000}%
\pgfsetdash{}{0pt}%
\pgfpathmoveto{\pgfqpoint{3.546546in}{3.384082in}}%
\pgfpathlineto{\pgfqpoint{3.557643in}{3.387475in}}%
\pgfpathlineto{\pgfqpoint{3.568735in}{3.390838in}}%
\pgfpathlineto{\pgfqpoint{3.579821in}{3.394177in}}%
\pgfpathlineto{\pgfqpoint{3.590902in}{3.397499in}}%
\pgfpathlineto{\pgfqpoint{3.601977in}{3.400809in}}%
\pgfpathlineto{\pgfqpoint{3.595699in}{3.412005in}}%
\pgfpathlineto{\pgfqpoint{3.589425in}{3.423342in}}%
\pgfpathlineto{\pgfqpoint{3.583156in}{3.434747in}}%
\pgfpathlineto{\pgfqpoint{3.576890in}{3.446143in}}%
\pgfpathlineto{\pgfqpoint{3.570626in}{3.457452in}}%
\pgfpathlineto{\pgfqpoint{3.559556in}{3.454119in}}%
\pgfpathlineto{\pgfqpoint{3.548482in}{3.450773in}}%
\pgfpathlineto{\pgfqpoint{3.537401in}{3.447407in}}%
\pgfpathlineto{\pgfqpoint{3.526315in}{3.444014in}}%
\pgfpathlineto{\pgfqpoint{3.515224in}{3.440588in}}%
\pgfpathlineto{\pgfqpoint{3.521483in}{3.429376in}}%
\pgfpathlineto{\pgfqpoint{3.527744in}{3.418040in}}%
\pgfpathlineto{\pgfqpoint{3.534007in}{3.406661in}}%
\pgfpathlineto{\pgfqpoint{3.540275in}{3.395317in}}%
\pgfpathclose%
\pgfusepath{stroke,fill}%
\end{pgfscope}%
\begin{pgfscope}%
\pgfpathrectangle{\pgfqpoint{0.887500in}{0.275000in}}{\pgfqpoint{4.225000in}{4.225000in}}%
\pgfusepath{clip}%
\pgfsetbuttcap%
\pgfsetroundjoin%
\definecolor{currentfill}{rgb}{0.125394,0.574318,0.549086}%
\pgfsetfillcolor{currentfill}%
\pgfsetfillopacity{0.700000}%
\pgfsetlinewidth{0.501875pt}%
\definecolor{currentstroke}{rgb}{1.000000,1.000000,1.000000}%
\pgfsetstrokecolor{currentstroke}%
\pgfsetstrokeopacity{0.500000}%
\pgfsetdash{}{0pt}%
\pgfpathmoveto{\pgfqpoint{2.643998in}{2.614962in}}%
\pgfpathlineto{\pgfqpoint{2.655219in}{2.626972in}}%
\pgfpathlineto{\pgfqpoint{2.666440in}{2.638978in}}%
\pgfpathlineto{\pgfqpoint{2.677664in}{2.650649in}}%
\pgfpathlineto{\pgfqpoint{2.688894in}{2.661657in}}%
\pgfpathlineto{\pgfqpoint{2.700130in}{2.671733in}}%
\pgfpathlineto{\pgfqpoint{2.694095in}{2.680639in}}%
\pgfpathlineto{\pgfqpoint{2.688064in}{2.689451in}}%
\pgfpathlineto{\pgfqpoint{2.682037in}{2.698226in}}%
\pgfpathlineto{\pgfqpoint{2.676014in}{2.707024in}}%
\pgfpathlineto{\pgfqpoint{2.669993in}{2.715907in}}%
\pgfpathlineto{\pgfqpoint{2.658781in}{2.704962in}}%
\pgfpathlineto{\pgfqpoint{2.647576in}{2.693196in}}%
\pgfpathlineto{\pgfqpoint{2.636377in}{2.680873in}}%
\pgfpathlineto{\pgfqpoint{2.625179in}{2.668316in}}%
\pgfpathlineto{\pgfqpoint{2.613980in}{2.655846in}}%
\pgfpathlineto{\pgfqpoint{2.619975in}{2.647778in}}%
\pgfpathlineto{\pgfqpoint{2.625973in}{2.639744in}}%
\pgfpathlineto{\pgfqpoint{2.631975in}{2.631653in}}%
\pgfpathlineto{\pgfqpoint{2.637983in}{2.623417in}}%
\pgfpathclose%
\pgfusepath{stroke,fill}%
\end{pgfscope}%
\begin{pgfscope}%
\pgfpathrectangle{\pgfqpoint{0.887500in}{0.275000in}}{\pgfqpoint{4.225000in}{4.225000in}}%
\pgfusepath{clip}%
\pgfsetbuttcap%
\pgfsetroundjoin%
\definecolor{currentfill}{rgb}{0.668054,0.861999,0.196293}%
\pgfsetfillcolor{currentfill}%
\pgfsetfillopacity{0.700000}%
\pgfsetlinewidth{0.501875pt}%
\definecolor{currentstroke}{rgb}{1.000000,1.000000,1.000000}%
\pgfsetstrokecolor{currentstroke}%
\pgfsetstrokeopacity{0.500000}%
\pgfsetdash{}{0pt}%
\pgfpathmoveto{\pgfqpoint{3.633437in}{3.346312in}}%
\pgfpathlineto{\pgfqpoint{3.644514in}{3.349669in}}%
\pgfpathlineto{\pgfqpoint{3.655586in}{3.353005in}}%
\pgfpathlineto{\pgfqpoint{3.666652in}{3.356316in}}%
\pgfpathlineto{\pgfqpoint{3.677712in}{3.359596in}}%
\pgfpathlineto{\pgfqpoint{3.688766in}{3.362842in}}%
\pgfpathlineto{\pgfqpoint{3.682459in}{3.373589in}}%
\pgfpathlineto{\pgfqpoint{3.676155in}{3.384335in}}%
\pgfpathlineto{\pgfqpoint{3.669855in}{3.395120in}}%
\pgfpathlineto{\pgfqpoint{3.663559in}{3.405983in}}%
\pgfpathlineto{\pgfqpoint{3.657268in}{3.416962in}}%
\pgfpathlineto{\pgfqpoint{3.646222in}{3.413822in}}%
\pgfpathlineto{\pgfqpoint{3.635169in}{3.410626in}}%
\pgfpathlineto{\pgfqpoint{3.624111in}{3.407384in}}%
\pgfpathlineto{\pgfqpoint{3.613047in}{3.404108in}}%
\pgfpathlineto{\pgfqpoint{3.601977in}{3.400809in}}%
\pgfpathlineto{\pgfqpoint{3.608260in}{3.389760in}}%
\pgfpathlineto{\pgfqpoint{3.614548in}{3.378821in}}%
\pgfpathlineto{\pgfqpoint{3.620841in}{3.367957in}}%
\pgfpathlineto{\pgfqpoint{3.627137in}{3.357133in}}%
\pgfpathclose%
\pgfusepath{stroke,fill}%
\end{pgfscope}%
\begin{pgfscope}%
\pgfpathrectangle{\pgfqpoint{0.887500in}{0.275000in}}{\pgfqpoint{4.225000in}{4.225000in}}%
\pgfusepath{clip}%
\pgfsetbuttcap%
\pgfsetroundjoin%
\definecolor{currentfill}{rgb}{0.121148,0.592739,0.544641}%
\pgfsetfillcolor{currentfill}%
\pgfsetfillopacity{0.700000}%
\pgfsetlinewidth{0.501875pt}%
\definecolor{currentstroke}{rgb}{1.000000,1.000000,1.000000}%
\pgfsetstrokecolor{currentstroke}%
\pgfsetstrokeopacity{0.500000}%
\pgfsetdash{}{0pt}%
\pgfpathmoveto{\pgfqpoint{1.811343in}{2.672224in}}%
\pgfpathlineto{\pgfqpoint{1.822863in}{2.675478in}}%
\pgfpathlineto{\pgfqpoint{1.834377in}{2.678738in}}%
\pgfpathlineto{\pgfqpoint{1.845885in}{2.682007in}}%
\pgfpathlineto{\pgfqpoint{1.857387in}{2.685286in}}%
\pgfpathlineto{\pgfqpoint{1.868883in}{2.688577in}}%
\pgfpathlineto{\pgfqpoint{1.863141in}{2.696409in}}%
\pgfpathlineto{\pgfqpoint{1.857403in}{2.704223in}}%
\pgfpathlineto{\pgfqpoint{1.851669in}{2.712020in}}%
\pgfpathlineto{\pgfqpoint{1.845939in}{2.719801in}}%
\pgfpathlineto{\pgfqpoint{1.840214in}{2.727565in}}%
\pgfpathlineto{\pgfqpoint{1.828731in}{2.724269in}}%
\pgfpathlineto{\pgfqpoint{1.817242in}{2.720984in}}%
\pgfpathlineto{\pgfqpoint{1.805747in}{2.717709in}}%
\pgfpathlineto{\pgfqpoint{1.794246in}{2.714440in}}%
\pgfpathlineto{\pgfqpoint{1.782740in}{2.711178in}}%
\pgfpathlineto{\pgfqpoint{1.788452in}{2.703422in}}%
\pgfpathlineto{\pgfqpoint{1.794168in}{2.695649in}}%
\pgfpathlineto{\pgfqpoint{1.799889in}{2.687859in}}%
\pgfpathlineto{\pgfqpoint{1.805614in}{2.680051in}}%
\pgfpathclose%
\pgfusepath{stroke,fill}%
\end{pgfscope}%
\begin{pgfscope}%
\pgfpathrectangle{\pgfqpoint{0.887500in}{0.275000in}}{\pgfqpoint{4.225000in}{4.225000in}}%
\pgfusepath{clip}%
\pgfsetbuttcap%
\pgfsetroundjoin%
\definecolor{currentfill}{rgb}{0.804182,0.882046,0.114965}%
\pgfsetfillcolor{currentfill}%
\pgfsetfillopacity{0.700000}%
\pgfsetlinewidth{0.501875pt}%
\definecolor{currentstroke}{rgb}{1.000000,1.000000,1.000000}%
\pgfsetstrokecolor{currentstroke}%
\pgfsetstrokeopacity{0.500000}%
\pgfsetdash{}{0pt}%
\pgfpathmoveto{\pgfqpoint{3.230021in}{3.456930in}}%
\pgfpathlineto{\pgfqpoint{3.241198in}{3.460654in}}%
\pgfpathlineto{\pgfqpoint{3.252370in}{3.464510in}}%
\pgfpathlineto{\pgfqpoint{3.263538in}{3.468466in}}%
\pgfpathlineto{\pgfqpoint{3.274701in}{3.472491in}}%
\pgfpathlineto{\pgfqpoint{3.285860in}{3.476554in}}%
\pgfpathlineto{\pgfqpoint{3.279638in}{3.484074in}}%
\pgfpathlineto{\pgfqpoint{3.273419in}{3.491307in}}%
\pgfpathlineto{\pgfqpoint{3.267203in}{3.498268in}}%
\pgfpathlineto{\pgfqpoint{3.260989in}{3.504972in}}%
\pgfpathlineto{\pgfqpoint{3.254779in}{3.511434in}}%
\pgfpathlineto{\pgfqpoint{3.243635in}{3.507636in}}%
\pgfpathlineto{\pgfqpoint{3.232485in}{3.503873in}}%
\pgfpathlineto{\pgfqpoint{3.221331in}{3.500164in}}%
\pgfpathlineto{\pgfqpoint{3.210172in}{3.496521in}}%
\pgfpathlineto{\pgfqpoint{3.199008in}{3.492958in}}%
\pgfpathlineto{\pgfqpoint{3.205204in}{3.486203in}}%
\pgfpathlineto{\pgfqpoint{3.211403in}{3.479235in}}%
\pgfpathlineto{\pgfqpoint{3.217606in}{3.472043in}}%
\pgfpathlineto{\pgfqpoint{3.223812in}{3.464612in}}%
\pgfpathclose%
\pgfusepath{stroke,fill}%
\end{pgfscope}%
\begin{pgfscope}%
\pgfpathrectangle{\pgfqpoint{0.887500in}{0.275000in}}{\pgfqpoint{4.225000in}{4.225000in}}%
\pgfusepath{clip}%
\pgfsetbuttcap%
\pgfsetroundjoin%
\definecolor{currentfill}{rgb}{0.458674,0.816363,0.329727}%
\pgfsetfillcolor{currentfill}%
\pgfsetfillopacity{0.700000}%
\pgfsetlinewidth{0.501875pt}%
\definecolor{currentstroke}{rgb}{1.000000,1.000000,1.000000}%
\pgfsetstrokecolor{currentstroke}%
\pgfsetstrokeopacity{0.500000}%
\pgfsetdash{}{0pt}%
\pgfpathmoveto{\pgfqpoint{3.980771in}{3.177592in}}%
\pgfpathlineto{\pgfqpoint{3.991757in}{3.180640in}}%
\pgfpathlineto{\pgfqpoint{4.002737in}{3.183663in}}%
\pgfpathlineto{\pgfqpoint{4.013712in}{3.186660in}}%
\pgfpathlineto{\pgfqpoint{4.024680in}{3.189632in}}%
\pgfpathlineto{\pgfqpoint{4.035642in}{3.192577in}}%
\pgfpathlineto{\pgfqpoint{4.029281in}{3.204858in}}%
\pgfpathlineto{\pgfqpoint{4.022921in}{3.217065in}}%
\pgfpathlineto{\pgfqpoint{4.016562in}{3.229191in}}%
\pgfpathlineto{\pgfqpoint{4.010204in}{3.241227in}}%
\pgfpathlineto{\pgfqpoint{4.003846in}{3.253165in}}%
\pgfpathlineto{\pgfqpoint{3.992888in}{3.250219in}}%
\pgfpathlineto{\pgfqpoint{3.981924in}{3.247237in}}%
\pgfpathlineto{\pgfqpoint{3.970953in}{3.244219in}}%
\pgfpathlineto{\pgfqpoint{3.959976in}{3.241167in}}%
\pgfpathlineto{\pgfqpoint{3.948994in}{3.238081in}}%
\pgfpathlineto{\pgfqpoint{3.955347in}{3.226167in}}%
\pgfpathlineto{\pgfqpoint{3.961702in}{3.214158in}}%
\pgfpathlineto{\pgfqpoint{3.968057in}{3.202056in}}%
\pgfpathlineto{\pgfqpoint{3.974413in}{3.189866in}}%
\pgfpathclose%
\pgfusepath{stroke,fill}%
\end{pgfscope}%
\begin{pgfscope}%
\pgfpathrectangle{\pgfqpoint{0.887500in}{0.275000in}}{\pgfqpoint{4.225000in}{4.225000in}}%
\pgfusepath{clip}%
\pgfsetbuttcap%
\pgfsetroundjoin%
\definecolor{currentfill}{rgb}{0.506271,0.828786,0.300362}%
\pgfsetfillcolor{currentfill}%
\pgfsetfillopacity{0.700000}%
\pgfsetlinewidth{0.501875pt}%
\definecolor{currentstroke}{rgb}{1.000000,1.000000,1.000000}%
\pgfsetstrokecolor{currentstroke}%
\pgfsetstrokeopacity{0.500000}%
\pgfsetdash{}{0pt}%
\pgfpathmoveto{\pgfqpoint{3.893987in}{3.222025in}}%
\pgfpathlineto{\pgfqpoint{3.905001in}{3.225334in}}%
\pgfpathlineto{\pgfqpoint{3.916008in}{3.228590in}}%
\pgfpathlineto{\pgfqpoint{3.927010in}{3.231797in}}%
\pgfpathlineto{\pgfqpoint{3.938005in}{3.234959in}}%
\pgfpathlineto{\pgfqpoint{3.948994in}{3.238081in}}%
\pgfpathlineto{\pgfqpoint{3.942641in}{3.249899in}}%
\pgfpathlineto{\pgfqpoint{3.936290in}{3.261635in}}%
\pgfpathlineto{\pgfqpoint{3.929940in}{3.273300in}}%
\pgfpathlineto{\pgfqpoint{3.923593in}{3.284907in}}%
\pgfpathlineto{\pgfqpoint{3.917247in}{3.296469in}}%
\pgfpathlineto{\pgfqpoint{3.906267in}{3.293506in}}%
\pgfpathlineto{\pgfqpoint{3.895281in}{3.290555in}}%
\pgfpathlineto{\pgfqpoint{3.884289in}{3.287597in}}%
\pgfpathlineto{\pgfqpoint{3.873292in}{3.284613in}}%
\pgfpathlineto{\pgfqpoint{3.862289in}{3.281584in}}%
\pgfpathlineto{\pgfqpoint{3.868624in}{3.269761in}}%
\pgfpathlineto{\pgfqpoint{3.874961in}{3.257882in}}%
\pgfpathlineto{\pgfqpoint{3.881300in}{3.245958in}}%
\pgfpathlineto{\pgfqpoint{3.887642in}{3.234001in}}%
\pgfpathclose%
\pgfusepath{stroke,fill}%
\end{pgfscope}%
\begin{pgfscope}%
\pgfpathrectangle{\pgfqpoint{0.887500in}{0.275000in}}{\pgfqpoint{4.225000in}{4.225000in}}%
\pgfusepath{clip}%
\pgfsetbuttcap%
\pgfsetroundjoin%
\definecolor{currentfill}{rgb}{0.616293,0.852709,0.230052}%
\pgfsetfillcolor{currentfill}%
\pgfsetfillopacity{0.700000}%
\pgfsetlinewidth{0.501875pt}%
\definecolor{currentstroke}{rgb}{1.000000,1.000000,1.000000}%
\pgfsetstrokecolor{currentstroke}%
\pgfsetstrokeopacity{0.500000}%
\pgfsetdash{}{0pt}%
\pgfpathmoveto{\pgfqpoint{3.720332in}{3.307728in}}%
\pgfpathlineto{\pgfqpoint{3.731388in}{3.311032in}}%
\pgfpathlineto{\pgfqpoint{3.742439in}{3.314325in}}%
\pgfpathlineto{\pgfqpoint{3.753485in}{3.317604in}}%
\pgfpathlineto{\pgfqpoint{3.764524in}{3.320861in}}%
\pgfpathlineto{\pgfqpoint{3.775558in}{3.324084in}}%
\pgfpathlineto{\pgfqpoint{3.769236in}{3.335293in}}%
\pgfpathlineto{\pgfqpoint{3.762913in}{3.346306in}}%
\pgfpathlineto{\pgfqpoint{3.756591in}{3.357158in}}%
\pgfpathlineto{\pgfqpoint{3.750270in}{3.367890in}}%
\pgfpathlineto{\pgfqpoint{3.743951in}{3.378538in}}%
\pgfpathlineto{\pgfqpoint{3.732925in}{3.375439in}}%
\pgfpathlineto{\pgfqpoint{3.721894in}{3.372336in}}%
\pgfpathlineto{\pgfqpoint{3.710858in}{3.369212in}}%
\pgfpathlineto{\pgfqpoint{3.699815in}{3.366049in}}%
\pgfpathlineto{\pgfqpoint{3.688766in}{3.362842in}}%
\pgfpathlineto{\pgfqpoint{3.695077in}{3.352056in}}%
\pgfpathlineto{\pgfqpoint{3.701389in}{3.341190in}}%
\pgfpathlineto{\pgfqpoint{3.707703in}{3.330207in}}%
\pgfpathlineto{\pgfqpoint{3.714017in}{3.319066in}}%
\pgfpathclose%
\pgfusepath{stroke,fill}%
\end{pgfscope}%
\begin{pgfscope}%
\pgfpathrectangle{\pgfqpoint{0.887500in}{0.275000in}}{\pgfqpoint{4.225000in}{4.225000in}}%
\pgfusepath{clip}%
\pgfsetbuttcap%
\pgfsetroundjoin%
\definecolor{currentfill}{rgb}{0.565498,0.842430,0.262877}%
\pgfsetfillcolor{currentfill}%
\pgfsetfillopacity{0.700000}%
\pgfsetlinewidth{0.501875pt}%
\definecolor{currentstroke}{rgb}{1.000000,1.000000,1.000000}%
\pgfsetstrokecolor{currentstroke}%
\pgfsetstrokeopacity{0.500000}%
\pgfsetdash{}{0pt}%
\pgfpathmoveto{\pgfqpoint{3.807170in}{3.265333in}}%
\pgfpathlineto{\pgfqpoint{3.818207in}{3.268714in}}%
\pgfpathlineto{\pgfqpoint{3.829237in}{3.272046in}}%
\pgfpathlineto{\pgfqpoint{3.840261in}{3.275311in}}%
\pgfpathlineto{\pgfqpoint{3.851279in}{3.278489in}}%
\pgfpathlineto{\pgfqpoint{3.862289in}{3.281584in}}%
\pgfpathlineto{\pgfqpoint{3.855955in}{3.293336in}}%
\pgfpathlineto{\pgfqpoint{3.849623in}{3.305007in}}%
\pgfpathlineto{\pgfqpoint{3.843292in}{3.316584in}}%
\pgfpathlineto{\pgfqpoint{3.836963in}{3.328054in}}%
\pgfpathlineto{\pgfqpoint{3.830634in}{3.339404in}}%
\pgfpathlineto{\pgfqpoint{3.819632in}{3.336465in}}%
\pgfpathlineto{\pgfqpoint{3.808623in}{3.333465in}}%
\pgfpathlineto{\pgfqpoint{3.797608in}{3.330396in}}%
\pgfpathlineto{\pgfqpoint{3.786586in}{3.327266in}}%
\pgfpathlineto{\pgfqpoint{3.775558in}{3.324084in}}%
\pgfpathlineto{\pgfqpoint{3.781880in}{3.312660in}}%
\pgfpathlineto{\pgfqpoint{3.788202in}{3.301044in}}%
\pgfpathlineto{\pgfqpoint{3.794524in}{3.289265in}}%
\pgfpathlineto{\pgfqpoint{3.800846in}{3.277352in}}%
\pgfpathclose%
\pgfusepath{stroke,fill}%
\end{pgfscope}%
\begin{pgfscope}%
\pgfpathrectangle{\pgfqpoint{0.887500in}{0.275000in}}{\pgfqpoint{4.225000in}{4.225000in}}%
\pgfusepath{clip}%
\pgfsetbuttcap%
\pgfsetroundjoin%
\definecolor{currentfill}{rgb}{0.146616,0.673050,0.508936}%
\pgfsetfillcolor{currentfill}%
\pgfsetfillopacity{0.700000}%
\pgfsetlinewidth{0.501875pt}%
\definecolor{currentstroke}{rgb}{1.000000,1.000000,1.000000}%
\pgfsetstrokecolor{currentstroke}%
\pgfsetstrokeopacity{0.500000}%
\pgfsetdash{}{0pt}%
\pgfpathmoveto{\pgfqpoint{2.812260in}{2.796220in}}%
\pgfpathlineto{\pgfqpoint{2.823429in}{2.815957in}}%
\pgfpathlineto{\pgfqpoint{2.834595in}{2.836862in}}%
\pgfpathlineto{\pgfqpoint{2.845761in}{2.858799in}}%
\pgfpathlineto{\pgfqpoint{2.856926in}{2.881692in}}%
\pgfpathlineto{\pgfqpoint{2.868092in}{2.905654in}}%
\pgfpathlineto{\pgfqpoint{2.862025in}{2.908953in}}%
\pgfpathlineto{\pgfqpoint{2.855962in}{2.912557in}}%
\pgfpathlineto{\pgfqpoint{2.849904in}{2.916399in}}%
\pgfpathlineto{\pgfqpoint{2.843849in}{2.920402in}}%
\pgfpathlineto{\pgfqpoint{2.837800in}{2.924487in}}%
\pgfpathlineto{\pgfqpoint{2.826630in}{2.906969in}}%
\pgfpathlineto{\pgfqpoint{2.815459in}{2.889987in}}%
\pgfpathlineto{\pgfqpoint{2.804291in}{2.872972in}}%
\pgfpathlineto{\pgfqpoint{2.793124in}{2.855973in}}%
\pgfpathlineto{\pgfqpoint{2.781959in}{2.839226in}}%
\pgfpathlineto{\pgfqpoint{2.788018in}{2.829738in}}%
\pgfpathlineto{\pgfqpoint{2.794078in}{2.820580in}}%
\pgfpathlineto{\pgfqpoint{2.800138in}{2.811868in}}%
\pgfpathlineto{\pgfqpoint{2.806199in}{2.803716in}}%
\pgfpathclose%
\pgfusepath{stroke,fill}%
\end{pgfscope}%
\begin{pgfscope}%
\pgfpathrectangle{\pgfqpoint{0.887500in}{0.275000in}}{\pgfqpoint{4.225000in}{4.225000in}}%
\pgfusepath{clip}%
\pgfsetbuttcap%
\pgfsetroundjoin%
\definecolor{currentfill}{rgb}{0.132444,0.552216,0.553018}%
\pgfsetfillcolor{currentfill}%
\pgfsetfillopacity{0.700000}%
\pgfsetlinewidth{0.501875pt}%
\definecolor{currentstroke}{rgb}{1.000000,1.000000,1.000000}%
\pgfsetstrokecolor{currentstroke}%
\pgfsetstrokeopacity{0.500000}%
\pgfsetdash{}{0pt}%
\pgfpathmoveto{\pgfqpoint{2.587757in}{2.565406in}}%
\pgfpathlineto{\pgfqpoint{2.599032in}{2.573272in}}%
\pgfpathlineto{\pgfqpoint{2.610293in}{2.582188in}}%
\pgfpathlineto{\pgfqpoint{2.621538in}{2.592243in}}%
\pgfpathlineto{\pgfqpoint{2.632772in}{2.603276in}}%
\pgfpathlineto{\pgfqpoint{2.643998in}{2.614962in}}%
\pgfpathlineto{\pgfqpoint{2.637983in}{2.623417in}}%
\pgfpathlineto{\pgfqpoint{2.631975in}{2.631653in}}%
\pgfpathlineto{\pgfqpoint{2.625973in}{2.639744in}}%
\pgfpathlineto{\pgfqpoint{2.619975in}{2.647778in}}%
\pgfpathlineto{\pgfqpoint{2.613980in}{2.655846in}}%
\pgfpathlineto{\pgfqpoint{2.602777in}{2.643782in}}%
\pgfpathlineto{\pgfqpoint{2.591564in}{2.632446in}}%
\pgfpathlineto{\pgfqpoint{2.580338in}{2.622154in}}%
\pgfpathlineto{\pgfqpoint{2.569096in}{2.613062in}}%
\pgfpathlineto{\pgfqpoint{2.557837in}{2.605082in}}%
\pgfpathlineto{\pgfqpoint{2.563813in}{2.597191in}}%
\pgfpathlineto{\pgfqpoint{2.569792in}{2.589344in}}%
\pgfpathlineto{\pgfqpoint{2.575775in}{2.581473in}}%
\pgfpathlineto{\pgfqpoint{2.581763in}{2.573510in}}%
\pgfpathclose%
\pgfusepath{stroke,fill}%
\end{pgfscope}%
\begin{pgfscope}%
\pgfpathrectangle{\pgfqpoint{0.887500in}{0.275000in}}{\pgfqpoint{4.225000in}{4.225000in}}%
\pgfusepath{clip}%
\pgfsetbuttcap%
\pgfsetroundjoin%
\definecolor{currentfill}{rgb}{0.127568,0.566949,0.550556}%
\pgfsetfillcolor{currentfill}%
\pgfsetfillopacity{0.700000}%
\pgfsetlinewidth{0.501875pt}%
\definecolor{currentstroke}{rgb}{1.000000,1.000000,1.000000}%
\pgfsetstrokecolor{currentstroke}%
\pgfsetstrokeopacity{0.500000}%
\pgfsetdash{}{0pt}%
\pgfpathmoveto{\pgfqpoint{2.127856in}{2.618643in}}%
\pgfpathlineto{\pgfqpoint{2.139301in}{2.621906in}}%
\pgfpathlineto{\pgfqpoint{2.150740in}{2.625188in}}%
\pgfpathlineto{\pgfqpoint{2.162172in}{2.628502in}}%
\pgfpathlineto{\pgfqpoint{2.173598in}{2.631859in}}%
\pgfpathlineto{\pgfqpoint{2.185017in}{2.635271in}}%
\pgfpathlineto{\pgfqpoint{2.179162in}{2.643366in}}%
\pgfpathlineto{\pgfqpoint{2.173310in}{2.651447in}}%
\pgfpathlineto{\pgfqpoint{2.167463in}{2.659515in}}%
\pgfpathlineto{\pgfqpoint{2.161620in}{2.667571in}}%
\pgfpathlineto{\pgfqpoint{2.155780in}{2.675613in}}%
\pgfpathlineto{\pgfqpoint{2.144375in}{2.672129in}}%
\pgfpathlineto{\pgfqpoint{2.132962in}{2.668729in}}%
\pgfpathlineto{\pgfqpoint{2.121543in}{2.665396in}}%
\pgfpathlineto{\pgfqpoint{2.110116in}{2.662111in}}%
\pgfpathlineto{\pgfqpoint{2.098684in}{2.658857in}}%
\pgfpathlineto{\pgfqpoint{2.104510in}{2.650838in}}%
\pgfpathlineto{\pgfqpoint{2.110341in}{2.642807in}}%
\pgfpathlineto{\pgfqpoint{2.116175in}{2.634764in}}%
\pgfpathlineto{\pgfqpoint{2.122014in}{2.626710in}}%
\pgfpathclose%
\pgfusepath{stroke,fill}%
\end{pgfscope}%
\begin{pgfscope}%
\pgfpathrectangle{\pgfqpoint{0.887500in}{0.275000in}}{\pgfqpoint{4.225000in}{4.225000in}}%
\pgfusepath{clip}%
\pgfsetbuttcap%
\pgfsetroundjoin%
\definecolor{currentfill}{rgb}{0.119512,0.607464,0.540218}%
\pgfsetfillcolor{currentfill}%
\pgfsetfillopacity{0.700000}%
\pgfsetlinewidth{0.501875pt}%
\definecolor{currentstroke}{rgb}{1.000000,1.000000,1.000000}%
\pgfsetstrokecolor{currentstroke}%
\pgfsetstrokeopacity{0.500000}%
\pgfsetdash{}{0pt}%
\pgfpathmoveto{\pgfqpoint{1.581173in}{2.700559in}}%
\pgfpathlineto{\pgfqpoint{1.592750in}{2.703850in}}%
\pgfpathlineto{\pgfqpoint{1.604322in}{2.707139in}}%
\pgfpathlineto{\pgfqpoint{1.615888in}{2.710427in}}%
\pgfpathlineto{\pgfqpoint{1.627448in}{2.713713in}}%
\pgfpathlineto{\pgfqpoint{1.639003in}{2.716997in}}%
\pgfpathlineto{\pgfqpoint{1.633343in}{2.724637in}}%
\pgfpathlineto{\pgfqpoint{1.627688in}{2.732261in}}%
\pgfpathlineto{\pgfqpoint{1.622037in}{2.739871in}}%
\pgfpathlineto{\pgfqpoint{1.616391in}{2.747464in}}%
\pgfpathlineto{\pgfqpoint{1.610748in}{2.755041in}}%
\pgfpathlineto{\pgfqpoint{1.599208in}{2.751742in}}%
\pgfpathlineto{\pgfqpoint{1.587662in}{2.748440in}}%
\pgfpathlineto{\pgfqpoint{1.576110in}{2.745135in}}%
\pgfpathlineto{\pgfqpoint{1.564553in}{2.741829in}}%
\pgfpathlineto{\pgfqpoint{1.552990in}{2.738522in}}%
\pgfpathlineto{\pgfqpoint{1.558618in}{2.730962in}}%
\pgfpathlineto{\pgfqpoint{1.564250in}{2.723385in}}%
\pgfpathlineto{\pgfqpoint{1.569887in}{2.715793in}}%
\pgfpathlineto{\pgfqpoint{1.575528in}{2.708184in}}%
\pgfpathclose%
\pgfusepath{stroke,fill}%
\end{pgfscope}%
\begin{pgfscope}%
\pgfpathrectangle{\pgfqpoint{0.887500in}{0.275000in}}{\pgfqpoint{4.225000in}{4.225000in}}%
\pgfusepath{clip}%
\pgfsetbuttcap%
\pgfsetroundjoin%
\definecolor{currentfill}{rgb}{0.136408,0.541173,0.554483}%
\pgfsetfillcolor{currentfill}%
\pgfsetfillopacity{0.700000}%
\pgfsetlinewidth{0.501875pt}%
\definecolor{currentstroke}{rgb}{1.000000,1.000000,1.000000}%
\pgfsetstrokecolor{currentstroke}%
\pgfsetstrokeopacity{0.500000}%
\pgfsetdash{}{0pt}%
\pgfpathmoveto{\pgfqpoint{2.444517in}{2.562818in}}%
\pgfpathlineto{\pgfqpoint{2.455894in}{2.565587in}}%
\pgfpathlineto{\pgfqpoint{2.467264in}{2.568392in}}%
\pgfpathlineto{\pgfqpoint{2.478626in}{2.571341in}}%
\pgfpathlineto{\pgfqpoint{2.489979in}{2.574542in}}%
\pgfpathlineto{\pgfqpoint{2.501322in}{2.578103in}}%
\pgfpathlineto{\pgfqpoint{2.495362in}{2.586111in}}%
\pgfpathlineto{\pgfqpoint{2.489405in}{2.594221in}}%
\pgfpathlineto{\pgfqpoint{2.483450in}{2.602475in}}%
\pgfpathlineto{\pgfqpoint{2.477495in}{2.610918in}}%
\pgfpathlineto{\pgfqpoint{2.471542in}{2.619593in}}%
\pgfpathlineto{\pgfqpoint{2.460212in}{2.615939in}}%
\pgfpathlineto{\pgfqpoint{2.448872in}{2.612681in}}%
\pgfpathlineto{\pgfqpoint{2.437521in}{2.609704in}}%
\pgfpathlineto{\pgfqpoint{2.426162in}{2.606890in}}%
\pgfpathlineto{\pgfqpoint{2.414797in}{2.604126in}}%
\pgfpathlineto{\pgfqpoint{2.420734in}{2.595833in}}%
\pgfpathlineto{\pgfqpoint{2.426674in}{2.587562in}}%
\pgfpathlineto{\pgfqpoint{2.432618in}{2.579308in}}%
\pgfpathlineto{\pgfqpoint{2.438566in}{2.571061in}}%
\pgfpathclose%
\pgfusepath{stroke,fill}%
\end{pgfscope}%
\begin{pgfscope}%
\pgfpathrectangle{\pgfqpoint{0.887500in}{0.275000in}}{\pgfqpoint{4.225000in}{4.225000in}}%
\pgfusepath{clip}%
\pgfsetbuttcap%
\pgfsetroundjoin%
\definecolor{currentfill}{rgb}{0.119699,0.618490,0.536347}%
\pgfsetfillcolor{currentfill}%
\pgfsetfillopacity{0.700000}%
\pgfsetlinewidth{0.501875pt}%
\definecolor{currentstroke}{rgb}{1.000000,1.000000,1.000000}%
\pgfsetstrokecolor{currentstroke}%
\pgfsetstrokeopacity{0.500000}%
\pgfsetdash{}{0pt}%
\pgfpathmoveto{\pgfqpoint{1.351018in}{2.726011in}}%
\pgfpathlineto{\pgfqpoint{1.362649in}{2.729389in}}%
\pgfpathlineto{\pgfqpoint{1.374275in}{2.732757in}}%
\pgfpathlineto{\pgfqpoint{1.385895in}{2.736117in}}%
\pgfpathlineto{\pgfqpoint{1.397510in}{2.739469in}}%
\pgfpathlineto{\pgfqpoint{1.409120in}{2.742813in}}%
\pgfpathlineto{\pgfqpoint{1.403547in}{2.750232in}}%
\pgfpathlineto{\pgfqpoint{1.397979in}{2.757631in}}%
\pgfpathlineto{\pgfqpoint{1.392415in}{2.765008in}}%
\pgfpathlineto{\pgfqpoint{1.386856in}{2.772364in}}%
\pgfpathlineto{\pgfqpoint{1.381302in}{2.779698in}}%
\pgfpathlineto{\pgfqpoint{1.369708in}{2.776325in}}%
\pgfpathlineto{\pgfqpoint{1.358108in}{2.772945in}}%
\pgfpathlineto{\pgfqpoint{1.346503in}{2.769557in}}%
\pgfpathlineto{\pgfqpoint{1.334893in}{2.766161in}}%
\pgfpathlineto{\pgfqpoint{1.323277in}{2.762757in}}%
\pgfpathlineto{\pgfqpoint{1.328816in}{2.755458in}}%
\pgfpathlineto{\pgfqpoint{1.334359in}{2.748134in}}%
\pgfpathlineto{\pgfqpoint{1.339908in}{2.740784in}}%
\pgfpathlineto{\pgfqpoint{1.345460in}{2.733409in}}%
\pgfpathclose%
\pgfusepath{stroke,fill}%
\end{pgfscope}%
\begin{pgfscope}%
\pgfpathrectangle{\pgfqpoint{0.887500in}{0.275000in}}{\pgfqpoint{4.225000in}{4.225000in}}%
\pgfusepath{clip}%
\pgfsetbuttcap%
\pgfsetroundjoin%
\definecolor{currentfill}{rgb}{0.122606,0.585371,0.546557}%
\pgfsetfillcolor{currentfill}%
\pgfsetfillopacity{0.700000}%
\pgfsetlinewidth{0.501875pt}%
\definecolor{currentstroke}{rgb}{1.000000,1.000000,1.000000}%
\pgfsetstrokecolor{currentstroke}%
\pgfsetstrokeopacity{0.500000}%
\pgfsetdash{}{0pt}%
\pgfpathmoveto{\pgfqpoint{1.897658in}{2.649139in}}%
\pgfpathlineto{\pgfqpoint{1.909161in}{2.652441in}}%
\pgfpathlineto{\pgfqpoint{1.920658in}{2.655753in}}%
\pgfpathlineto{\pgfqpoint{1.932150in}{2.659075in}}%
\pgfpathlineto{\pgfqpoint{1.943635in}{2.662405in}}%
\pgfpathlineto{\pgfqpoint{1.955115in}{2.665740in}}%
\pgfpathlineto{\pgfqpoint{1.949338in}{2.673667in}}%
\pgfpathlineto{\pgfqpoint{1.943566in}{2.681577in}}%
\pgfpathlineto{\pgfqpoint{1.937798in}{2.689467in}}%
\pgfpathlineto{\pgfqpoint{1.932034in}{2.697339in}}%
\pgfpathlineto{\pgfqpoint{1.926275in}{2.705192in}}%
\pgfpathlineto{\pgfqpoint{1.914808in}{2.701852in}}%
\pgfpathlineto{\pgfqpoint{1.903335in}{2.698519in}}%
\pgfpathlineto{\pgfqpoint{1.891857in}{2.695194in}}%
\pgfpathlineto{\pgfqpoint{1.880373in}{2.691880in}}%
\pgfpathlineto{\pgfqpoint{1.868883in}{2.688577in}}%
\pgfpathlineto{\pgfqpoint{1.874629in}{2.680727in}}%
\pgfpathlineto{\pgfqpoint{1.880380in}{2.672859in}}%
\pgfpathlineto{\pgfqpoint{1.886135in}{2.664971in}}%
\pgfpathlineto{\pgfqpoint{1.891895in}{2.657064in}}%
\pgfpathclose%
\pgfusepath{stroke,fill}%
\end{pgfscope}%
\begin{pgfscope}%
\pgfpathrectangle{\pgfqpoint{0.887500in}{0.275000in}}{\pgfqpoint{4.225000in}{4.225000in}}%
\pgfusepath{clip}%
\pgfsetbuttcap%
\pgfsetroundjoin%
\definecolor{currentfill}{rgb}{0.814576,0.883393,0.110347}%
\pgfsetfillcolor{currentfill}%
\pgfsetfillopacity{0.700000}%
\pgfsetlinewidth{0.501875pt}%
\definecolor{currentstroke}{rgb}{1.000000,1.000000,1.000000}%
\pgfsetstrokecolor{currentstroke}%
\pgfsetstrokeopacity{0.500000}%
\pgfsetdash{}{0pt}%
\pgfpathmoveto{\pgfqpoint{3.087091in}{3.464903in}}%
\pgfpathlineto{\pgfqpoint{3.098307in}{3.466863in}}%
\pgfpathlineto{\pgfqpoint{3.109517in}{3.469051in}}%
\pgfpathlineto{\pgfqpoint{3.120722in}{3.471453in}}%
\pgfpathlineto{\pgfqpoint{3.131921in}{3.474054in}}%
\pgfpathlineto{\pgfqpoint{3.143115in}{3.476838in}}%
\pgfpathlineto{\pgfqpoint{3.136935in}{3.482953in}}%
\pgfpathlineto{\pgfqpoint{3.130759in}{3.488718in}}%
\pgfpathlineto{\pgfqpoint{3.124587in}{3.494131in}}%
\pgfpathlineto{\pgfqpoint{3.118420in}{3.499201in}}%
\pgfpathlineto{\pgfqpoint{3.112257in}{3.503936in}}%
\pgfpathlineto{\pgfqpoint{3.101076in}{3.499740in}}%
\pgfpathlineto{\pgfqpoint{3.089890in}{3.495751in}}%
\pgfpathlineto{\pgfqpoint{3.078699in}{3.492111in}}%
\pgfpathlineto{\pgfqpoint{3.067504in}{3.488962in}}%
\pgfpathlineto{\pgfqpoint{3.056303in}{3.486440in}}%
\pgfpathlineto{\pgfqpoint{3.062450in}{3.483087in}}%
\pgfpathlineto{\pgfqpoint{3.068603in}{3.479321in}}%
\pgfpathlineto{\pgfqpoint{3.074761in}{3.475071in}}%
\pgfpathlineto{\pgfqpoint{3.080923in}{3.470268in}}%
\pgfpathclose%
\pgfusepath{stroke,fill}%
\end{pgfscope}%
\begin{pgfscope}%
\pgfpathrectangle{\pgfqpoint{0.887500in}{0.275000in}}{\pgfqpoint{4.225000in}{4.225000in}}%
\pgfusepath{clip}%
\pgfsetbuttcap%
\pgfsetroundjoin%
\definecolor{currentfill}{rgb}{0.783315,0.879285,0.125405}%
\pgfsetfillcolor{currentfill}%
\pgfsetfillopacity{0.700000}%
\pgfsetlinewidth{0.501875pt}%
\definecolor{currentstroke}{rgb}{1.000000,1.000000,1.000000}%
\pgfsetstrokecolor{currentstroke}%
\pgfsetstrokeopacity{0.500000}%
\pgfsetdash{}{0pt}%
\pgfpathmoveto{\pgfqpoint{2.944078in}{3.424104in}}%
\pgfpathlineto{\pgfqpoint{2.955291in}{3.440709in}}%
\pgfpathlineto{\pgfqpoint{2.966516in}{3.453235in}}%
\pgfpathlineto{\pgfqpoint{2.977746in}{3.462510in}}%
\pgfpathlineto{\pgfqpoint{2.988978in}{3.469358in}}%
\pgfpathlineto{\pgfqpoint{3.000210in}{3.474397in}}%
\pgfpathlineto{\pgfqpoint{2.994084in}{3.478007in}}%
\pgfpathlineto{\pgfqpoint{2.987965in}{3.481383in}}%
\pgfpathlineto{\pgfqpoint{2.981850in}{3.484581in}}%
\pgfpathlineto{\pgfqpoint{2.975742in}{3.487657in}}%
\pgfpathlineto{\pgfqpoint{2.969639in}{3.490667in}}%
\pgfpathlineto{\pgfqpoint{2.958424in}{3.486716in}}%
\pgfpathlineto{\pgfqpoint{2.947209in}{3.480872in}}%
\pgfpathlineto{\pgfqpoint{2.935998in}{3.472443in}}%
\pgfpathlineto{\pgfqpoint{2.924795in}{3.460559in}}%
\pgfpathlineto{\pgfqpoint{2.913607in}{3.444346in}}%
\pgfpathlineto{\pgfqpoint{2.919691in}{3.440314in}}%
\pgfpathlineto{\pgfqpoint{2.925780in}{3.436326in}}%
\pgfpathlineto{\pgfqpoint{2.931874in}{3.432330in}}%
\pgfpathlineto{\pgfqpoint{2.937973in}{3.428273in}}%
\pgfpathclose%
\pgfusepath{stroke,fill}%
\end{pgfscope}%
\begin{pgfscope}%
\pgfpathrectangle{\pgfqpoint{0.887500in}{0.275000in}}{\pgfqpoint{4.225000in}{4.225000in}}%
\pgfusepath{clip}%
\pgfsetbuttcap%
\pgfsetroundjoin%
\definecolor{currentfill}{rgb}{0.793760,0.880678,0.120005}%
\pgfsetfillcolor{currentfill}%
\pgfsetfillopacity{0.700000}%
\pgfsetlinewidth{0.501875pt}%
\definecolor{currentstroke}{rgb}{1.000000,1.000000,1.000000}%
\pgfsetstrokecolor{currentstroke}%
\pgfsetstrokeopacity{0.500000}%
\pgfsetdash{}{0pt}%
\pgfpathmoveto{\pgfqpoint{3.317003in}{3.434134in}}%
\pgfpathlineto{\pgfqpoint{3.328167in}{3.438242in}}%
\pgfpathlineto{\pgfqpoint{3.339327in}{3.442351in}}%
\pgfpathlineto{\pgfqpoint{3.350482in}{3.446446in}}%
\pgfpathlineto{\pgfqpoint{3.361632in}{3.450512in}}%
\pgfpathlineto{\pgfqpoint{3.372777in}{3.454534in}}%
\pgfpathlineto{\pgfqpoint{3.366535in}{3.463691in}}%
\pgfpathlineto{\pgfqpoint{3.360293in}{3.472428in}}%
\pgfpathlineto{\pgfqpoint{3.354053in}{3.480752in}}%
\pgfpathlineto{\pgfqpoint{3.347813in}{3.488685in}}%
\pgfpathlineto{\pgfqpoint{3.341576in}{3.496250in}}%
\pgfpathlineto{\pgfqpoint{3.330444in}{3.492509in}}%
\pgfpathlineto{\pgfqpoint{3.319306in}{3.488634in}}%
\pgfpathlineto{\pgfqpoint{3.308162in}{3.484660in}}%
\pgfpathlineto{\pgfqpoint{3.297014in}{3.480622in}}%
\pgfpathlineto{\pgfqpoint{3.285860in}{3.476554in}}%
\pgfpathlineto{\pgfqpoint{3.292084in}{3.468732in}}%
\pgfpathlineto{\pgfqpoint{3.298311in}{3.460592in}}%
\pgfpathlineto{\pgfqpoint{3.304540in}{3.452121in}}%
\pgfpathlineto{\pgfqpoint{3.310770in}{3.443302in}}%
\pgfpathclose%
\pgfusepath{stroke,fill}%
\end{pgfscope}%
\begin{pgfscope}%
\pgfpathrectangle{\pgfqpoint{0.887500in}{0.275000in}}{\pgfqpoint{4.225000in}{4.225000in}}%
\pgfusepath{clip}%
\pgfsetbuttcap%
\pgfsetroundjoin%
\definecolor{currentfill}{rgb}{0.129933,0.559582,0.551864}%
\pgfsetfillcolor{currentfill}%
\pgfsetfillopacity{0.700000}%
\pgfsetlinewidth{0.501875pt}%
\definecolor{currentstroke}{rgb}{1.000000,1.000000,1.000000}%
\pgfsetstrokecolor{currentstroke}%
\pgfsetstrokeopacity{0.500000}%
\pgfsetdash{}{0pt}%
\pgfpathmoveto{\pgfqpoint{2.214356in}{2.594610in}}%
\pgfpathlineto{\pgfqpoint{2.225783in}{2.597978in}}%
\pgfpathlineto{\pgfqpoint{2.237203in}{2.601368in}}%
\pgfpathlineto{\pgfqpoint{2.248618in}{2.604774in}}%
\pgfpathlineto{\pgfqpoint{2.260028in}{2.608189in}}%
\pgfpathlineto{\pgfqpoint{2.271431in}{2.611608in}}%
\pgfpathlineto{\pgfqpoint{2.265542in}{2.619825in}}%
\pgfpathlineto{\pgfqpoint{2.259657in}{2.628035in}}%
\pgfpathlineto{\pgfqpoint{2.253776in}{2.636237in}}%
\pgfpathlineto{\pgfqpoint{2.247898in}{2.644430in}}%
\pgfpathlineto{\pgfqpoint{2.242025in}{2.652613in}}%
\pgfpathlineto{\pgfqpoint{2.230633in}{2.649209in}}%
\pgfpathlineto{\pgfqpoint{2.219237in}{2.645741in}}%
\pgfpathlineto{\pgfqpoint{2.207836in}{2.642241in}}%
\pgfpathlineto{\pgfqpoint{2.196430in}{2.638741in}}%
\pgfpathlineto{\pgfqpoint{2.185017in}{2.635271in}}%
\pgfpathlineto{\pgfqpoint{2.190877in}{2.627164in}}%
\pgfpathlineto{\pgfqpoint{2.196741in}{2.619045in}}%
\pgfpathlineto{\pgfqpoint{2.202608in}{2.610912in}}%
\pgfpathlineto{\pgfqpoint{2.208480in}{2.602768in}}%
\pgfpathclose%
\pgfusepath{stroke,fill}%
\end{pgfscope}%
\begin{pgfscope}%
\pgfpathrectangle{\pgfqpoint{0.887500in}{0.275000in}}{\pgfqpoint{4.225000in}{4.225000in}}%
\pgfusepath{clip}%
\pgfsetbuttcap%
\pgfsetroundjoin%
\definecolor{currentfill}{rgb}{0.120092,0.600104,0.542530}%
\pgfsetfillcolor{currentfill}%
\pgfsetfillopacity{0.700000}%
\pgfsetlinewidth{0.501875pt}%
\definecolor{currentstroke}{rgb}{1.000000,1.000000,1.000000}%
\pgfsetstrokecolor{currentstroke}%
\pgfsetstrokeopacity{0.500000}%
\pgfsetdash{}{0pt}%
\pgfpathmoveto{\pgfqpoint{1.667363in}{2.678574in}}%
\pgfpathlineto{\pgfqpoint{1.678926in}{2.681842in}}%
\pgfpathlineto{\pgfqpoint{1.690483in}{2.685108in}}%
\pgfpathlineto{\pgfqpoint{1.702035in}{2.688371in}}%
\pgfpathlineto{\pgfqpoint{1.713581in}{2.691632in}}%
\pgfpathlineto{\pgfqpoint{1.725122in}{2.694892in}}%
\pgfpathlineto{\pgfqpoint{1.719427in}{2.702619in}}%
\pgfpathlineto{\pgfqpoint{1.713737in}{2.710331in}}%
\pgfpathlineto{\pgfqpoint{1.708052in}{2.718027in}}%
\pgfpathlineto{\pgfqpoint{1.702370in}{2.725708in}}%
\pgfpathlineto{\pgfqpoint{1.696693in}{2.733374in}}%
\pgfpathlineto{\pgfqpoint{1.685166in}{2.730103in}}%
\pgfpathlineto{\pgfqpoint{1.673633in}{2.726830in}}%
\pgfpathlineto{\pgfqpoint{1.662095in}{2.723555in}}%
\pgfpathlineto{\pgfqpoint{1.650552in}{2.720277in}}%
\pgfpathlineto{\pgfqpoint{1.639003in}{2.716997in}}%
\pgfpathlineto{\pgfqpoint{1.644666in}{2.709342in}}%
\pgfpathlineto{\pgfqpoint{1.650334in}{2.701672in}}%
\pgfpathlineto{\pgfqpoint{1.656006in}{2.693988in}}%
\pgfpathlineto{\pgfqpoint{1.661682in}{2.686289in}}%
\pgfpathclose%
\pgfusepath{stroke,fill}%
\end{pgfscope}%
\begin{pgfscope}%
\pgfpathrectangle{\pgfqpoint{0.887500in}{0.275000in}}{\pgfqpoint{4.225000in}{4.225000in}}%
\pgfusepath{clip}%
\pgfsetbuttcap%
\pgfsetroundjoin%
\definecolor{currentfill}{rgb}{0.137770,0.537492,0.554906}%
\pgfsetfillcolor{currentfill}%
\pgfsetfillopacity{0.700000}%
\pgfsetlinewidth{0.501875pt}%
\definecolor{currentstroke}{rgb}{1.000000,1.000000,1.000000}%
\pgfsetstrokecolor{currentstroke}%
\pgfsetstrokeopacity{0.500000}%
\pgfsetdash{}{0pt}%
\pgfpathmoveto{\pgfqpoint{2.531178in}{2.538066in}}%
\pgfpathlineto{\pgfqpoint{2.542519in}{2.542294in}}%
\pgfpathlineto{\pgfqpoint{2.553847in}{2.547034in}}%
\pgfpathlineto{\pgfqpoint{2.565164in}{2.552395in}}%
\pgfpathlineto{\pgfqpoint{2.576467in}{2.558483in}}%
\pgfpathlineto{\pgfqpoint{2.587757in}{2.565406in}}%
\pgfpathlineto{\pgfqpoint{2.581763in}{2.573510in}}%
\pgfpathlineto{\pgfqpoint{2.575775in}{2.581473in}}%
\pgfpathlineto{\pgfqpoint{2.569792in}{2.589344in}}%
\pgfpathlineto{\pgfqpoint{2.563813in}{2.597191in}}%
\pgfpathlineto{\pgfqpoint{2.557837in}{2.605082in}}%
\pgfpathlineto{\pgfqpoint{2.546563in}{2.598106in}}%
\pgfpathlineto{\pgfqpoint{2.535274in}{2.592027in}}%
\pgfpathlineto{\pgfqpoint{2.523970in}{2.586739in}}%
\pgfpathlineto{\pgfqpoint{2.512652in}{2.582133in}}%
\pgfpathlineto{\pgfqpoint{2.501322in}{2.578103in}}%
\pgfpathlineto{\pgfqpoint{2.507284in}{2.570153in}}%
\pgfpathlineto{\pgfqpoint{2.513251in}{2.562216in}}%
\pgfpathlineto{\pgfqpoint{2.519221in}{2.554250in}}%
\pgfpathlineto{\pgfqpoint{2.525197in}{2.546210in}}%
\pgfpathclose%
\pgfusepath{stroke,fill}%
\end{pgfscope}%
\begin{pgfscope}%
\pgfpathrectangle{\pgfqpoint{0.887500in}{0.275000in}}{\pgfqpoint{4.225000in}{4.225000in}}%
\pgfusepath{clip}%
\pgfsetbuttcap%
\pgfsetroundjoin%
\definecolor{currentfill}{rgb}{0.119483,0.614817,0.537692}%
\pgfsetfillcolor{currentfill}%
\pgfsetfillopacity{0.700000}%
\pgfsetlinewidth{0.501875pt}%
\definecolor{currentstroke}{rgb}{1.000000,1.000000,1.000000}%
\pgfsetstrokecolor{currentstroke}%
\pgfsetstrokeopacity{0.500000}%
\pgfsetdash{}{0pt}%
\pgfpathmoveto{\pgfqpoint{1.437051in}{2.705420in}}%
\pgfpathlineto{\pgfqpoint{1.448670in}{2.708744in}}%
\pgfpathlineto{\pgfqpoint{1.460283in}{2.712063in}}%
\pgfpathlineto{\pgfqpoint{1.471891in}{2.715377in}}%
\pgfpathlineto{\pgfqpoint{1.483493in}{2.718688in}}%
\pgfpathlineto{\pgfqpoint{1.495090in}{2.721995in}}%
\pgfpathlineto{\pgfqpoint{1.489481in}{2.729523in}}%
\pgfpathlineto{\pgfqpoint{1.483876in}{2.737034in}}%
\pgfpathlineto{\pgfqpoint{1.478275in}{2.744528in}}%
\pgfpathlineto{\pgfqpoint{1.472679in}{2.752004in}}%
\pgfpathlineto{\pgfqpoint{1.467087in}{2.759462in}}%
\pgfpathlineto{\pgfqpoint{1.455504in}{2.756139in}}%
\pgfpathlineto{\pgfqpoint{1.443917in}{2.752814in}}%
\pgfpathlineto{\pgfqpoint{1.432323in}{2.749485in}}%
\pgfpathlineto{\pgfqpoint{1.420724in}{2.746152in}}%
\pgfpathlineto{\pgfqpoint{1.409120in}{2.742813in}}%
\pgfpathlineto{\pgfqpoint{1.414697in}{2.735374in}}%
\pgfpathlineto{\pgfqpoint{1.420279in}{2.727914in}}%
\pgfpathlineto{\pgfqpoint{1.425865in}{2.720435in}}%
\pgfpathlineto{\pgfqpoint{1.431456in}{2.712937in}}%
\pgfpathclose%
\pgfusepath{stroke,fill}%
\end{pgfscope}%
\begin{pgfscope}%
\pgfpathrectangle{\pgfqpoint{0.887500in}{0.275000in}}{\pgfqpoint{4.225000in}{4.225000in}}%
\pgfusepath{clip}%
\pgfsetbuttcap%
\pgfsetroundjoin%
\definecolor{currentfill}{rgb}{0.762373,0.876424,0.137064}%
\pgfsetfillcolor{currentfill}%
\pgfsetfillopacity{0.700000}%
\pgfsetlinewidth{0.501875pt}%
\definecolor{currentstroke}{rgb}{1.000000,1.000000,1.000000}%
\pgfsetstrokecolor{currentstroke}%
\pgfsetstrokeopacity{0.500000}%
\pgfsetdash{}{0pt}%
\pgfpathmoveto{\pgfqpoint{3.404001in}{3.403811in}}%
\pgfpathlineto{\pgfqpoint{3.415148in}{3.407667in}}%
\pgfpathlineto{\pgfqpoint{3.426290in}{3.411493in}}%
\pgfpathlineto{\pgfqpoint{3.437426in}{3.415285in}}%
\pgfpathlineto{\pgfqpoint{3.448556in}{3.419039in}}%
\pgfpathlineto{\pgfqpoint{3.459681in}{3.422752in}}%
\pgfpathlineto{\pgfqpoint{3.453428in}{3.433619in}}%
\pgfpathlineto{\pgfqpoint{3.447176in}{3.444212in}}%
\pgfpathlineto{\pgfqpoint{3.440924in}{3.454471in}}%
\pgfpathlineto{\pgfqpoint{3.434672in}{3.464331in}}%
\pgfpathlineto{\pgfqpoint{3.428419in}{3.473729in}}%
\pgfpathlineto{\pgfqpoint{3.417301in}{3.469998in}}%
\pgfpathlineto{\pgfqpoint{3.406178in}{3.466222in}}%
\pgfpathlineto{\pgfqpoint{3.395050in}{3.462391in}}%
\pgfpathlineto{\pgfqpoint{3.383916in}{3.458497in}}%
\pgfpathlineto{\pgfqpoint{3.372777in}{3.454534in}}%
\pgfpathlineto{\pgfqpoint{3.379019in}{3.444992in}}%
\pgfpathlineto{\pgfqpoint{3.385263in}{3.435107in}}%
\pgfpathlineto{\pgfqpoint{3.391508in}{3.424921in}}%
\pgfpathlineto{\pgfqpoint{3.397754in}{3.414475in}}%
\pgfpathclose%
\pgfusepath{stroke,fill}%
\end{pgfscope}%
\begin{pgfscope}%
\pgfpathrectangle{\pgfqpoint{0.887500in}{0.275000in}}{\pgfqpoint{4.225000in}{4.225000in}}%
\pgfusepath{clip}%
\pgfsetbuttcap%
\pgfsetroundjoin%
\definecolor{currentfill}{rgb}{0.124395,0.578002,0.548287}%
\pgfsetfillcolor{currentfill}%
\pgfsetfillopacity{0.700000}%
\pgfsetlinewidth{0.501875pt}%
\definecolor{currentstroke}{rgb}{1.000000,1.000000,1.000000}%
\pgfsetstrokecolor{currentstroke}%
\pgfsetstrokeopacity{0.500000}%
\pgfsetdash{}{0pt}%
\pgfpathmoveto{\pgfqpoint{1.984060in}{2.625861in}}%
\pgfpathlineto{\pgfqpoint{1.995546in}{2.629196in}}%
\pgfpathlineto{\pgfqpoint{2.007028in}{2.632530in}}%
\pgfpathlineto{\pgfqpoint{2.018503in}{2.635863in}}%
\pgfpathlineto{\pgfqpoint{2.029973in}{2.639189in}}%
\pgfpathlineto{\pgfqpoint{2.041438in}{2.642506in}}%
\pgfpathlineto{\pgfqpoint{2.035628in}{2.650517in}}%
\pgfpathlineto{\pgfqpoint{2.029822in}{2.658514in}}%
\pgfpathlineto{\pgfqpoint{2.024020in}{2.666497in}}%
\pgfpathlineto{\pgfqpoint{2.018222in}{2.674465in}}%
\pgfpathlineto{\pgfqpoint{2.012429in}{2.682418in}}%
\pgfpathlineto{\pgfqpoint{2.000977in}{2.679094in}}%
\pgfpathlineto{\pgfqpoint{1.989520in}{2.675760in}}%
\pgfpathlineto{\pgfqpoint{1.978057in}{2.672420in}}%
\pgfpathlineto{\pgfqpoint{1.966588in}{2.669079in}}%
\pgfpathlineto{\pgfqpoint{1.955115in}{2.665740in}}%
\pgfpathlineto{\pgfqpoint{1.960895in}{2.657795in}}%
\pgfpathlineto{\pgfqpoint{1.966680in}{2.649834in}}%
\pgfpathlineto{\pgfqpoint{1.972469in}{2.641857in}}%
\pgfpathlineto{\pgfqpoint{1.978262in}{2.633866in}}%
\pgfpathclose%
\pgfusepath{stroke,fill}%
\end{pgfscope}%
\begin{pgfscope}%
\pgfpathrectangle{\pgfqpoint{0.887500in}{0.275000in}}{\pgfqpoint{4.225000in}{4.225000in}}%
\pgfusepath{clip}%
\pgfsetbuttcap%
\pgfsetroundjoin%
\definecolor{currentfill}{rgb}{0.344074,0.780029,0.397381}%
\pgfsetfillcolor{currentfill}%
\pgfsetfillopacity{0.700000}%
\pgfsetlinewidth{0.501875pt}%
\definecolor{currentstroke}{rgb}{1.000000,1.000000,1.000000}%
\pgfsetstrokecolor{currentstroke}%
\pgfsetstrokeopacity{0.500000}%
\pgfsetdash{}{0pt}%
\pgfpathmoveto{\pgfqpoint{4.099355in}{3.067254in}}%
\pgfpathlineto{\pgfqpoint{4.110328in}{3.070551in}}%
\pgfpathlineto{\pgfqpoint{4.121296in}{3.073865in}}%
\pgfpathlineto{\pgfqpoint{4.132260in}{3.077195in}}%
\pgfpathlineto{\pgfqpoint{4.143219in}{3.080541in}}%
\pgfpathlineto{\pgfqpoint{4.154173in}{3.083903in}}%
\pgfpathlineto{\pgfqpoint{4.147785in}{3.096453in}}%
\pgfpathlineto{\pgfqpoint{4.141398in}{3.108924in}}%
\pgfpathlineto{\pgfqpoint{4.135013in}{3.121327in}}%
\pgfpathlineto{\pgfqpoint{4.128628in}{3.133670in}}%
\pgfpathlineto{\pgfqpoint{4.122245in}{3.145964in}}%
\pgfpathlineto{\pgfqpoint{4.111301in}{3.142819in}}%
\pgfpathlineto{\pgfqpoint{4.100351in}{3.139684in}}%
\pgfpathlineto{\pgfqpoint{4.089397in}{3.136562in}}%
\pgfpathlineto{\pgfqpoint{4.078438in}{3.133452in}}%
\pgfpathlineto{\pgfqpoint{4.067473in}{3.130356in}}%
\pgfpathlineto{\pgfqpoint{4.073845in}{3.117805in}}%
\pgfpathlineto{\pgfqpoint{4.080220in}{3.105230in}}%
\pgfpathlineto{\pgfqpoint{4.086597in}{3.092621in}}%
\pgfpathlineto{\pgfqpoint{4.092975in}{3.079966in}}%
\pgfpathclose%
\pgfusepath{stroke,fill}%
\end{pgfscope}%
\begin{pgfscope}%
\pgfpathrectangle{\pgfqpoint{0.887500in}{0.275000in}}{\pgfqpoint{4.225000in}{4.225000in}}%
\pgfusepath{clip}%
\pgfsetbuttcap%
\pgfsetroundjoin%
\definecolor{currentfill}{rgb}{0.296479,0.761561,0.424223}%
\pgfsetfillcolor{currentfill}%
\pgfsetfillopacity{0.700000}%
\pgfsetlinewidth{0.501875pt}%
\definecolor{currentstroke}{rgb}{1.000000,1.000000,1.000000}%
\pgfsetstrokecolor{currentstroke}%
\pgfsetstrokeopacity{0.500000}%
\pgfsetdash{}{0pt}%
\pgfpathmoveto{\pgfqpoint{4.186112in}{3.019608in}}%
\pgfpathlineto{\pgfqpoint{4.197066in}{3.023036in}}%
\pgfpathlineto{\pgfqpoint{4.208015in}{3.026462in}}%
\pgfpathlineto{\pgfqpoint{4.218959in}{3.029887in}}%
\pgfpathlineto{\pgfqpoint{4.229898in}{3.033310in}}%
\pgfpathlineto{\pgfqpoint{4.223509in}{3.046407in}}%
\pgfpathlineto{\pgfqpoint{4.217118in}{3.059371in}}%
\pgfpathlineto{\pgfqpoint{4.210727in}{3.072206in}}%
\pgfpathlineto{\pgfqpoint{4.204335in}{3.084913in}}%
\pgfpathlineto{\pgfqpoint{4.197942in}{3.097497in}}%
\pgfpathlineto{\pgfqpoint{4.187007in}{3.094077in}}%
\pgfpathlineto{\pgfqpoint{4.176067in}{3.090671in}}%
\pgfpathlineto{\pgfqpoint{4.165122in}{3.087279in}}%
\pgfpathlineto{\pgfqpoint{4.154173in}{3.083903in}}%
\pgfpathlineto{\pgfqpoint{4.160561in}{3.071263in}}%
\pgfpathlineto{\pgfqpoint{4.166950in}{3.058524in}}%
\pgfpathlineto{\pgfqpoint{4.173338in}{3.045675in}}%
\pgfpathlineto{\pgfqpoint{4.179725in}{3.032706in}}%
\pgfpathclose%
\pgfusepath{stroke,fill}%
\end{pgfscope}%
\begin{pgfscope}%
\pgfpathrectangle{\pgfqpoint{0.887500in}{0.275000in}}{\pgfqpoint{4.225000in}{4.225000in}}%
\pgfusepath{clip}%
\pgfsetbuttcap%
\pgfsetroundjoin%
\definecolor{currentfill}{rgb}{0.377779,0.791781,0.377939}%
\pgfsetfillcolor{currentfill}%
\pgfsetfillopacity{0.700000}%
\pgfsetlinewidth{0.501875pt}%
\definecolor{currentstroke}{rgb}{1.000000,1.000000,1.000000}%
\pgfsetstrokecolor{currentstroke}%
\pgfsetstrokeopacity{0.500000}%
\pgfsetdash{}{0pt}%
\pgfpathmoveto{\pgfqpoint{2.893557in}{3.044503in}}%
\pgfpathlineto{\pgfqpoint{2.904686in}{3.079303in}}%
\pgfpathlineto{\pgfqpoint{2.915818in}{3.117032in}}%
\pgfpathlineto{\pgfqpoint{2.926958in}{3.156301in}}%
\pgfpathlineto{\pgfqpoint{2.938112in}{3.195712in}}%
\pgfpathlineto{\pgfqpoint{2.949284in}{3.233858in}}%
\pgfpathlineto{\pgfqpoint{2.943167in}{3.236925in}}%
\pgfpathlineto{\pgfqpoint{2.937057in}{3.239726in}}%
\pgfpathlineto{\pgfqpoint{2.930953in}{3.242260in}}%
\pgfpathlineto{\pgfqpoint{2.924855in}{3.244530in}}%
\pgfpathlineto{\pgfqpoint{2.918764in}{3.246537in}}%
\pgfpathlineto{\pgfqpoint{2.907659in}{3.202219in}}%
\pgfpathlineto{\pgfqpoint{2.896576in}{3.157010in}}%
\pgfpathlineto{\pgfqpoint{2.885506in}{3.112942in}}%
\pgfpathlineto{\pgfqpoint{2.874439in}{3.072028in}}%
\pgfpathlineto{\pgfqpoint{2.863363in}{3.036269in}}%
\pgfpathlineto{\pgfqpoint{2.869386in}{3.038420in}}%
\pgfpathlineto{\pgfqpoint{2.875416in}{3.040314in}}%
\pgfpathlineto{\pgfqpoint{2.881455in}{3.041955in}}%
\pgfpathlineto{\pgfqpoint{2.887502in}{3.043350in}}%
\pgfpathclose%
\pgfusepath{stroke,fill}%
\end{pgfscope}%
\begin{pgfscope}%
\pgfpathrectangle{\pgfqpoint{0.887500in}{0.275000in}}{\pgfqpoint{4.225000in}{4.225000in}}%
\pgfusepath{clip}%
\pgfsetbuttcap%
\pgfsetroundjoin%
\definecolor{currentfill}{rgb}{0.720391,0.870350,0.162603}%
\pgfsetfillcolor{currentfill}%
\pgfsetfillopacity{0.700000}%
\pgfsetlinewidth{0.501875pt}%
\definecolor{currentstroke}{rgb}{1.000000,1.000000,1.000000}%
\pgfsetstrokecolor{currentstroke}%
\pgfsetstrokeopacity{0.500000}%
\pgfsetdash{}{0pt}%
\pgfpathmoveto{\pgfqpoint{3.490975in}{3.366530in}}%
\pgfpathlineto{\pgfqpoint{3.502101in}{3.370111in}}%
\pgfpathlineto{\pgfqpoint{3.513221in}{3.373666in}}%
\pgfpathlineto{\pgfqpoint{3.524335in}{3.377183in}}%
\pgfpathlineto{\pgfqpoint{3.535443in}{3.380653in}}%
\pgfpathlineto{\pgfqpoint{3.546546in}{3.384082in}}%
\pgfpathlineto{\pgfqpoint{3.540275in}{3.395317in}}%
\pgfpathlineto{\pgfqpoint{3.534007in}{3.406661in}}%
\pgfpathlineto{\pgfqpoint{3.527744in}{3.418040in}}%
\pgfpathlineto{\pgfqpoint{3.521483in}{3.429376in}}%
\pgfpathlineto{\pgfqpoint{3.515224in}{3.440588in}}%
\pgfpathlineto{\pgfqpoint{3.504127in}{3.437121in}}%
\pgfpathlineto{\pgfqpoint{3.493024in}{3.433607in}}%
\pgfpathlineto{\pgfqpoint{3.481915in}{3.430040in}}%
\pgfpathlineto{\pgfqpoint{3.470801in}{3.426420in}}%
\pgfpathlineto{\pgfqpoint{3.459681in}{3.422752in}}%
\pgfpathlineto{\pgfqpoint{3.465936in}{3.411676in}}%
\pgfpathlineto{\pgfqpoint{3.472192in}{3.400453in}}%
\pgfpathlineto{\pgfqpoint{3.478450in}{3.389146in}}%
\pgfpathlineto{\pgfqpoint{3.484711in}{3.377819in}}%
\pgfpathclose%
\pgfusepath{stroke,fill}%
\end{pgfscope}%
\begin{pgfscope}%
\pgfpathrectangle{\pgfqpoint{0.887500in}{0.275000in}}{\pgfqpoint{4.225000in}{4.225000in}}%
\pgfusepath{clip}%
\pgfsetbuttcap%
\pgfsetroundjoin%
\definecolor{currentfill}{rgb}{0.132444,0.552216,0.553018}%
\pgfsetfillcolor{currentfill}%
\pgfsetfillopacity{0.700000}%
\pgfsetlinewidth{0.501875pt}%
\definecolor{currentstroke}{rgb}{1.000000,1.000000,1.000000}%
\pgfsetstrokecolor{currentstroke}%
\pgfsetstrokeopacity{0.500000}%
\pgfsetdash{}{0pt}%
\pgfpathmoveto{\pgfqpoint{2.300935in}{2.570454in}}%
\pgfpathlineto{\pgfqpoint{2.312344in}{2.573934in}}%
\pgfpathlineto{\pgfqpoint{2.323746in}{2.577471in}}%
\pgfpathlineto{\pgfqpoint{2.335142in}{2.581056in}}%
\pgfpathlineto{\pgfqpoint{2.346533in}{2.584651in}}%
\pgfpathlineto{\pgfqpoint{2.357919in}{2.588214in}}%
\pgfpathlineto{\pgfqpoint{2.352000in}{2.596374in}}%
\pgfpathlineto{\pgfqpoint{2.346085in}{2.604492in}}%
\pgfpathlineto{\pgfqpoint{2.340175in}{2.612566in}}%
\pgfpathlineto{\pgfqpoint{2.334270in}{2.620603in}}%
\pgfpathlineto{\pgfqpoint{2.328369in}{2.628608in}}%
\pgfpathlineto{\pgfqpoint{2.316992in}{2.625210in}}%
\pgfpathlineto{\pgfqpoint{2.305610in}{2.621820in}}%
\pgfpathlineto{\pgfqpoint{2.294222in}{2.618427in}}%
\pgfpathlineto{\pgfqpoint{2.282829in}{2.615022in}}%
\pgfpathlineto{\pgfqpoint{2.271431in}{2.611608in}}%
\pgfpathlineto{\pgfqpoint{2.277324in}{2.603385in}}%
\pgfpathlineto{\pgfqpoint{2.283221in}{2.595157in}}%
\pgfpathlineto{\pgfqpoint{2.289122in}{2.586926in}}%
\pgfpathlineto{\pgfqpoint{2.295026in}{2.578692in}}%
\pgfpathclose%
\pgfusepath{stroke,fill}%
\end{pgfscope}%
\begin{pgfscope}%
\pgfpathrectangle{\pgfqpoint{0.887500in}{0.275000in}}{\pgfqpoint{4.225000in}{4.225000in}}%
\pgfusepath{clip}%
\pgfsetbuttcap%
\pgfsetroundjoin%
\definecolor{currentfill}{rgb}{0.395174,0.797475,0.367757}%
\pgfsetfillcolor{currentfill}%
\pgfsetfillopacity{0.700000}%
\pgfsetlinewidth{0.501875pt}%
\definecolor{currentstroke}{rgb}{1.000000,1.000000,1.000000}%
\pgfsetstrokecolor{currentstroke}%
\pgfsetstrokeopacity{0.500000}%
\pgfsetdash{}{0pt}%
\pgfpathmoveto{\pgfqpoint{4.012576in}{3.115107in}}%
\pgfpathlineto{\pgfqpoint{4.023566in}{3.118123in}}%
\pgfpathlineto{\pgfqpoint{4.034550in}{3.121157in}}%
\pgfpathlineto{\pgfqpoint{4.045529in}{3.124208in}}%
\pgfpathlineto{\pgfqpoint{4.056504in}{3.127275in}}%
\pgfpathlineto{\pgfqpoint{4.067473in}{3.130356in}}%
\pgfpathlineto{\pgfqpoint{4.061103in}{3.142883in}}%
\pgfpathlineto{\pgfqpoint{4.054735in}{3.155376in}}%
\pgfpathlineto{\pgfqpoint{4.048369in}{3.167828in}}%
\pgfpathlineto{\pgfqpoint{4.042005in}{3.180231in}}%
\pgfpathlineto{\pgfqpoint{4.035642in}{3.192577in}}%
\pgfpathlineto{\pgfqpoint{4.024680in}{3.189632in}}%
\pgfpathlineto{\pgfqpoint{4.013712in}{3.186660in}}%
\pgfpathlineto{\pgfqpoint{4.002737in}{3.183663in}}%
\pgfpathlineto{\pgfqpoint{3.991757in}{3.180640in}}%
\pgfpathlineto{\pgfqpoint{3.980771in}{3.177592in}}%
\pgfpathlineto{\pgfqpoint{3.987129in}{3.165238in}}%
\pgfpathlineto{\pgfqpoint{3.993489in}{3.152809in}}%
\pgfpathlineto{\pgfqpoint{3.999850in}{3.140308in}}%
\pgfpathlineto{\pgfqpoint{4.006213in}{3.127739in}}%
\pgfpathclose%
\pgfusepath{stroke,fill}%
\end{pgfscope}%
\begin{pgfscope}%
\pgfpathrectangle{\pgfqpoint{0.887500in}{0.275000in}}{\pgfqpoint{4.225000in}{4.225000in}}%
\pgfusepath{clip}%
\pgfsetbuttcap%
\pgfsetroundjoin%
\definecolor{currentfill}{rgb}{0.804182,0.882046,0.114965}%
\pgfsetfillcolor{currentfill}%
\pgfsetfillopacity{0.700000}%
\pgfsetlinewidth{0.501875pt}%
\definecolor{currentstroke}{rgb}{1.000000,1.000000,1.000000}%
\pgfsetstrokecolor{currentstroke}%
\pgfsetstrokeopacity{0.500000}%
\pgfsetdash{}{0pt}%
\pgfpathmoveto{\pgfqpoint{3.174069in}{3.441261in}}%
\pgfpathlineto{\pgfqpoint{3.185269in}{3.443942in}}%
\pgfpathlineto{\pgfqpoint{3.196464in}{3.446849in}}%
\pgfpathlineto{\pgfqpoint{3.207654in}{3.449996in}}%
\pgfpathlineto{\pgfqpoint{3.218840in}{3.453367in}}%
\pgfpathlineto{\pgfqpoint{3.230021in}{3.456930in}}%
\pgfpathlineto{\pgfqpoint{3.223812in}{3.464612in}}%
\pgfpathlineto{\pgfqpoint{3.217606in}{3.472043in}}%
\pgfpathlineto{\pgfqpoint{3.211403in}{3.479235in}}%
\pgfpathlineto{\pgfqpoint{3.205204in}{3.486203in}}%
\pgfpathlineto{\pgfqpoint{3.199008in}{3.492958in}}%
\pgfpathlineto{\pgfqpoint{3.187839in}{3.489490in}}%
\pgfpathlineto{\pgfqpoint{3.176666in}{3.486130in}}%
\pgfpathlineto{\pgfqpoint{3.165487in}{3.482891in}}%
\pgfpathlineto{\pgfqpoint{3.154303in}{3.479789in}}%
\pgfpathlineto{\pgfqpoint{3.143115in}{3.476838in}}%
\pgfpathlineto{\pgfqpoint{3.149298in}{3.470379in}}%
\pgfpathlineto{\pgfqpoint{3.155486in}{3.463585in}}%
\pgfpathlineto{\pgfqpoint{3.161677in}{3.456462in}}%
\pgfpathlineto{\pgfqpoint{3.167871in}{3.449018in}}%
\pgfpathclose%
\pgfusepath{stroke,fill}%
\end{pgfscope}%
\begin{pgfscope}%
\pgfpathrectangle{\pgfqpoint{0.887500in}{0.275000in}}{\pgfqpoint{4.225000in}{4.225000in}}%
\pgfusepath{clip}%
\pgfsetbuttcap%
\pgfsetroundjoin%
\definecolor{currentfill}{rgb}{0.668054,0.861999,0.196293}%
\pgfsetfillcolor{currentfill}%
\pgfsetfillopacity{0.700000}%
\pgfsetlinewidth{0.501875pt}%
\definecolor{currentstroke}{rgb}{1.000000,1.000000,1.000000}%
\pgfsetstrokecolor{currentstroke}%
\pgfsetstrokeopacity{0.500000}%
\pgfsetdash{}{0pt}%
\pgfpathmoveto{\pgfqpoint{3.577968in}{3.329145in}}%
\pgfpathlineto{\pgfqpoint{3.589073in}{3.332658in}}%
\pgfpathlineto{\pgfqpoint{3.600173in}{3.336121in}}%
\pgfpathlineto{\pgfqpoint{3.611266in}{3.339545in}}%
\pgfpathlineto{\pgfqpoint{3.622354in}{3.342939in}}%
\pgfpathlineto{\pgfqpoint{3.633437in}{3.346312in}}%
\pgfpathlineto{\pgfqpoint{3.627137in}{3.357133in}}%
\pgfpathlineto{\pgfqpoint{3.620841in}{3.367957in}}%
\pgfpathlineto{\pgfqpoint{3.614548in}{3.378821in}}%
\pgfpathlineto{\pgfqpoint{3.608260in}{3.389760in}}%
\pgfpathlineto{\pgfqpoint{3.601977in}{3.400809in}}%
\pgfpathlineto{\pgfqpoint{3.590902in}{3.397499in}}%
\pgfpathlineto{\pgfqpoint{3.579821in}{3.394177in}}%
\pgfpathlineto{\pgfqpoint{3.568735in}{3.390838in}}%
\pgfpathlineto{\pgfqpoint{3.557643in}{3.387475in}}%
\pgfpathlineto{\pgfqpoint{3.546546in}{3.384082in}}%
\pgfpathlineto{\pgfqpoint{3.552822in}{3.372967in}}%
\pgfpathlineto{\pgfqpoint{3.559103in}{3.361944in}}%
\pgfpathlineto{\pgfqpoint{3.565388in}{3.350985in}}%
\pgfpathlineto{\pgfqpoint{3.571676in}{3.340062in}}%
\pgfpathclose%
\pgfusepath{stroke,fill}%
\end{pgfscope}%
\begin{pgfscope}%
\pgfpathrectangle{\pgfqpoint{0.887500in}{0.275000in}}{\pgfqpoint{4.225000in}{4.225000in}}%
\pgfusepath{clip}%
\pgfsetbuttcap%
\pgfsetroundjoin%
\definecolor{currentfill}{rgb}{0.121148,0.592739,0.544641}%
\pgfsetfillcolor{currentfill}%
\pgfsetfillopacity{0.700000}%
\pgfsetlinewidth{0.501875pt}%
\definecolor{currentstroke}{rgb}{1.000000,1.000000,1.000000}%
\pgfsetstrokecolor{currentstroke}%
\pgfsetstrokeopacity{0.500000}%
\pgfsetdash{}{0pt}%
\pgfpathmoveto{\pgfqpoint{1.753657in}{2.655994in}}%
\pgfpathlineto{\pgfqpoint{1.765206in}{2.659240in}}%
\pgfpathlineto{\pgfqpoint{1.776748in}{2.662485in}}%
\pgfpathlineto{\pgfqpoint{1.788286in}{2.665730in}}%
\pgfpathlineto{\pgfqpoint{1.799817in}{2.668976in}}%
\pgfpathlineto{\pgfqpoint{1.811343in}{2.672224in}}%
\pgfpathlineto{\pgfqpoint{1.805614in}{2.680051in}}%
\pgfpathlineto{\pgfqpoint{1.799889in}{2.687859in}}%
\pgfpathlineto{\pgfqpoint{1.794168in}{2.695649in}}%
\pgfpathlineto{\pgfqpoint{1.788452in}{2.703422in}}%
\pgfpathlineto{\pgfqpoint{1.782740in}{2.711178in}}%
\pgfpathlineto{\pgfqpoint{1.771228in}{2.707919in}}%
\pgfpathlineto{\pgfqpoint{1.759710in}{2.704663in}}%
\pgfpathlineto{\pgfqpoint{1.748186in}{2.701407in}}%
\pgfpathlineto{\pgfqpoint{1.736657in}{2.698150in}}%
\pgfpathlineto{\pgfqpoint{1.725122in}{2.694892in}}%
\pgfpathlineto{\pgfqpoint{1.730820in}{2.687148in}}%
\pgfpathlineto{\pgfqpoint{1.736523in}{2.679386in}}%
\pgfpathlineto{\pgfqpoint{1.742230in}{2.671607in}}%
\pgfpathlineto{\pgfqpoint{1.747941in}{2.663810in}}%
\pgfpathclose%
\pgfusepath{stroke,fill}%
\end{pgfscope}%
\begin{pgfscope}%
\pgfpathrectangle{\pgfqpoint{0.887500in}{0.275000in}}{\pgfqpoint{4.225000in}{4.225000in}}%
\pgfusepath{clip}%
\pgfsetbuttcap%
\pgfsetroundjoin%
\definecolor{currentfill}{rgb}{0.458674,0.816363,0.329727}%
\pgfsetfillcolor{currentfill}%
\pgfsetfillopacity{0.700000}%
\pgfsetlinewidth{0.501875pt}%
\definecolor{currentstroke}{rgb}{1.000000,1.000000,1.000000}%
\pgfsetstrokecolor{currentstroke}%
\pgfsetstrokeopacity{0.500000}%
\pgfsetdash{}{0pt}%
\pgfpathmoveto{\pgfqpoint{3.925747in}{3.161818in}}%
\pgfpathlineto{\pgfqpoint{3.936765in}{3.165067in}}%
\pgfpathlineto{\pgfqpoint{3.947776in}{3.168262in}}%
\pgfpathlineto{\pgfqpoint{3.958780in}{3.171410in}}%
\pgfpathlineto{\pgfqpoint{3.969778in}{3.174517in}}%
\pgfpathlineto{\pgfqpoint{3.980771in}{3.177592in}}%
\pgfpathlineto{\pgfqpoint{3.974413in}{3.189866in}}%
\pgfpathlineto{\pgfqpoint{3.968057in}{3.202056in}}%
\pgfpathlineto{\pgfqpoint{3.961702in}{3.214158in}}%
\pgfpathlineto{\pgfqpoint{3.955347in}{3.226167in}}%
\pgfpathlineto{\pgfqpoint{3.948994in}{3.238081in}}%
\pgfpathlineto{\pgfqpoint{3.938005in}{3.234959in}}%
\pgfpathlineto{\pgfqpoint{3.927010in}{3.231797in}}%
\pgfpathlineto{\pgfqpoint{3.916008in}{3.228590in}}%
\pgfpathlineto{\pgfqpoint{3.905001in}{3.225334in}}%
\pgfpathlineto{\pgfqpoint{3.893987in}{3.222025in}}%
\pgfpathlineto{\pgfqpoint{3.900334in}{3.210039in}}%
\pgfpathlineto{\pgfqpoint{3.906684in}{3.198038in}}%
\pgfpathlineto{\pgfqpoint{3.913036in}{3.186010in}}%
\pgfpathlineto{\pgfqpoint{3.919391in}{3.173942in}}%
\pgfpathclose%
\pgfusepath{stroke,fill}%
\end{pgfscope}%
\begin{pgfscope}%
\pgfpathrectangle{\pgfqpoint{0.887500in}{0.275000in}}{\pgfqpoint{4.225000in}{4.225000in}}%
\pgfusepath{clip}%
\pgfsetbuttcap%
\pgfsetroundjoin%
\definecolor{currentfill}{rgb}{0.121831,0.589055,0.545623}%
\pgfsetfillcolor{currentfill}%
\pgfsetfillopacity{0.700000}%
\pgfsetlinewidth{0.501875pt}%
\definecolor{currentstroke}{rgb}{1.000000,1.000000,1.000000}%
\pgfsetstrokecolor{currentstroke}%
\pgfsetstrokeopacity{0.500000}%
\pgfsetdash{}{0pt}%
\pgfpathmoveto{\pgfqpoint{2.730390in}{2.623937in}}%
\pgfpathlineto{\pgfqpoint{2.741634in}{2.634666in}}%
\pgfpathlineto{\pgfqpoint{2.752876in}{2.645549in}}%
\pgfpathlineto{\pgfqpoint{2.764113in}{2.656875in}}%
\pgfpathlineto{\pgfqpoint{2.775345in}{2.668934in}}%
\pgfpathlineto{\pgfqpoint{2.786569in}{2.682015in}}%
\pgfpathlineto{\pgfqpoint{2.780514in}{2.688703in}}%
\pgfpathlineto{\pgfqpoint{2.774460in}{2.695798in}}%
\pgfpathlineto{\pgfqpoint{2.768408in}{2.703308in}}%
\pgfpathlineto{\pgfqpoint{2.762357in}{2.711239in}}%
\pgfpathlineto{\pgfqpoint{2.756307in}{2.719596in}}%
\pgfpathlineto{\pgfqpoint{2.745084in}{2.708958in}}%
\pgfpathlineto{\pgfqpoint{2.733852in}{2.699274in}}%
\pgfpathlineto{\pgfqpoint{2.722613in}{2.690122in}}%
\pgfpathlineto{\pgfqpoint{2.711371in}{2.681082in}}%
\pgfpathlineto{\pgfqpoint{2.700130in}{2.671733in}}%
\pgfpathlineto{\pgfqpoint{2.706171in}{2.662679in}}%
\pgfpathlineto{\pgfqpoint{2.712216in}{2.653424in}}%
\pgfpathlineto{\pgfqpoint{2.718268in}{2.643917in}}%
\pgfpathlineto{\pgfqpoint{2.724326in}{2.634106in}}%
\pgfpathclose%
\pgfusepath{stroke,fill}%
\end{pgfscope}%
\begin{pgfscope}%
\pgfpathrectangle{\pgfqpoint{0.887500in}{0.275000in}}{\pgfqpoint{4.225000in}{4.225000in}}%
\pgfusepath{clip}%
\pgfsetbuttcap%
\pgfsetroundjoin%
\definecolor{currentfill}{rgb}{0.119699,0.618490,0.536347}%
\pgfsetfillcolor{currentfill}%
\pgfsetfillopacity{0.700000}%
\pgfsetlinewidth{0.501875pt}%
\definecolor{currentstroke}{rgb}{1.000000,1.000000,1.000000}%
\pgfsetstrokecolor{currentstroke}%
\pgfsetstrokeopacity{0.500000}%
\pgfsetdash{}{0pt}%
\pgfpathmoveto{\pgfqpoint{2.786569in}{2.682015in}}%
\pgfpathlineto{\pgfqpoint{2.797784in}{2.696409in}}%
\pgfpathlineto{\pgfqpoint{2.808990in}{2.712364in}}%
\pgfpathlineto{\pgfqpoint{2.820189in}{2.729918in}}%
\pgfpathlineto{\pgfqpoint{2.831381in}{2.749044in}}%
\pgfpathlineto{\pgfqpoint{2.842569in}{2.769717in}}%
\pgfpathlineto{\pgfqpoint{2.836506in}{2.773456in}}%
\pgfpathlineto{\pgfqpoint{2.830444in}{2.778005in}}%
\pgfpathlineto{\pgfqpoint{2.824382in}{2.783335in}}%
\pgfpathlineto{\pgfqpoint{2.818321in}{2.789416in}}%
\pgfpathlineto{\pgfqpoint{2.812260in}{2.796220in}}%
\pgfpathlineto{\pgfqpoint{2.801087in}{2.777782in}}%
\pgfpathlineto{\pgfqpoint{2.789907in}{2.760779in}}%
\pgfpathlineto{\pgfqpoint{2.778719in}{2.745343in}}%
\pgfpathlineto{\pgfqpoint{2.767519in}{2.731608in}}%
\pgfpathlineto{\pgfqpoint{2.756307in}{2.719596in}}%
\pgfpathlineto{\pgfqpoint{2.762357in}{2.711239in}}%
\pgfpathlineto{\pgfqpoint{2.768408in}{2.703308in}}%
\pgfpathlineto{\pgfqpoint{2.774460in}{2.695798in}}%
\pgfpathlineto{\pgfqpoint{2.780514in}{2.688703in}}%
\pgfpathclose%
\pgfusepath{stroke,fill}%
\end{pgfscope}%
\begin{pgfscope}%
\pgfpathrectangle{\pgfqpoint{0.887500in}{0.275000in}}{\pgfqpoint{4.225000in}{4.225000in}}%
\pgfusepath{clip}%
\pgfsetbuttcap%
\pgfsetroundjoin%
\definecolor{currentfill}{rgb}{0.506271,0.828786,0.300362}%
\pgfsetfillcolor{currentfill}%
\pgfsetfillopacity{0.700000}%
\pgfsetlinewidth{0.501875pt}%
\definecolor{currentstroke}{rgb}{1.000000,1.000000,1.000000}%
\pgfsetstrokecolor{currentstroke}%
\pgfsetstrokeopacity{0.500000}%
\pgfsetdash{}{0pt}%
\pgfpathmoveto{\pgfqpoint{3.838822in}{3.204664in}}%
\pgfpathlineto{\pgfqpoint{3.849867in}{3.208211in}}%
\pgfpathlineto{\pgfqpoint{3.860906in}{3.211737in}}%
\pgfpathlineto{\pgfqpoint{3.871939in}{3.215226in}}%
\pgfpathlineto{\pgfqpoint{3.882966in}{3.218656in}}%
\pgfpathlineto{\pgfqpoint{3.893987in}{3.222025in}}%
\pgfpathlineto{\pgfqpoint{3.887642in}{3.234001in}}%
\pgfpathlineto{\pgfqpoint{3.881300in}{3.245958in}}%
\pgfpathlineto{\pgfqpoint{3.874961in}{3.257882in}}%
\pgfpathlineto{\pgfqpoint{3.868624in}{3.269761in}}%
\pgfpathlineto{\pgfqpoint{3.862289in}{3.281584in}}%
\pgfpathlineto{\pgfqpoint{3.851279in}{3.278489in}}%
\pgfpathlineto{\pgfqpoint{3.840261in}{3.275311in}}%
\pgfpathlineto{\pgfqpoint{3.829237in}{3.272046in}}%
\pgfpathlineto{\pgfqpoint{3.818207in}{3.268714in}}%
\pgfpathlineto{\pgfqpoint{3.807170in}{3.265333in}}%
\pgfpathlineto{\pgfqpoint{3.813495in}{3.253237in}}%
\pgfpathlineto{\pgfqpoint{3.819822in}{3.241094in}}%
\pgfpathlineto{\pgfqpoint{3.826152in}{3.228931in}}%
\pgfpathlineto{\pgfqpoint{3.832485in}{3.216778in}}%
\pgfpathclose%
\pgfusepath{stroke,fill}%
\end{pgfscope}%
\begin{pgfscope}%
\pgfpathrectangle{\pgfqpoint{0.887500in}{0.275000in}}{\pgfqpoint{4.225000in}{4.225000in}}%
\pgfusepath{clip}%
\pgfsetbuttcap%
\pgfsetroundjoin%
\definecolor{currentfill}{rgb}{0.616293,0.852709,0.230052}%
\pgfsetfillcolor{currentfill}%
\pgfsetfillopacity{0.700000}%
\pgfsetlinewidth{0.501875pt}%
\definecolor{currentstroke}{rgb}{1.000000,1.000000,1.000000}%
\pgfsetstrokecolor{currentstroke}%
\pgfsetstrokeopacity{0.500000}%
\pgfsetdash{}{0pt}%
\pgfpathmoveto{\pgfqpoint{3.664968in}{3.291008in}}%
\pgfpathlineto{\pgfqpoint{3.676052in}{3.294384in}}%
\pgfpathlineto{\pgfqpoint{3.687130in}{3.297743in}}%
\pgfpathlineto{\pgfqpoint{3.698203in}{3.301086in}}%
\pgfpathlineto{\pgfqpoint{3.709270in}{3.304414in}}%
\pgfpathlineto{\pgfqpoint{3.720332in}{3.307728in}}%
\pgfpathlineto{\pgfqpoint{3.714017in}{3.319066in}}%
\pgfpathlineto{\pgfqpoint{3.707703in}{3.330207in}}%
\pgfpathlineto{\pgfqpoint{3.701389in}{3.341190in}}%
\pgfpathlineto{\pgfqpoint{3.695077in}{3.352056in}}%
\pgfpathlineto{\pgfqpoint{3.688766in}{3.362842in}}%
\pgfpathlineto{\pgfqpoint{3.677712in}{3.359596in}}%
\pgfpathlineto{\pgfqpoint{3.666652in}{3.356316in}}%
\pgfpathlineto{\pgfqpoint{3.655586in}{3.353005in}}%
\pgfpathlineto{\pgfqpoint{3.644514in}{3.349669in}}%
\pgfpathlineto{\pgfqpoint{3.633437in}{3.346312in}}%
\pgfpathlineto{\pgfqpoint{3.639740in}{3.335459in}}%
\pgfpathlineto{\pgfqpoint{3.646045in}{3.324538in}}%
\pgfpathlineto{\pgfqpoint{3.652351in}{3.313513in}}%
\pgfpathlineto{\pgfqpoint{3.658659in}{3.302348in}}%
\pgfpathclose%
\pgfusepath{stroke,fill}%
\end{pgfscope}%
\begin{pgfscope}%
\pgfpathrectangle{\pgfqpoint{0.887500in}{0.275000in}}{\pgfqpoint{4.225000in}{4.225000in}}%
\pgfusepath{clip}%
\pgfsetbuttcap%
\pgfsetroundjoin%
\definecolor{currentfill}{rgb}{0.565498,0.842430,0.262877}%
\pgfsetfillcolor{currentfill}%
\pgfsetfillopacity{0.700000}%
\pgfsetlinewidth{0.501875pt}%
\definecolor{currentstroke}{rgb}{1.000000,1.000000,1.000000}%
\pgfsetstrokecolor{currentstroke}%
\pgfsetstrokeopacity{0.500000}%
\pgfsetdash{}{0pt}%
\pgfpathmoveto{\pgfqpoint{3.751907in}{3.248366in}}%
\pgfpathlineto{\pgfqpoint{3.762969in}{3.251713in}}%
\pgfpathlineto{\pgfqpoint{3.774027in}{3.255094in}}%
\pgfpathlineto{\pgfqpoint{3.785080in}{3.258503in}}%
\pgfpathlineto{\pgfqpoint{3.796127in}{3.261923in}}%
\pgfpathlineto{\pgfqpoint{3.807170in}{3.265333in}}%
\pgfpathlineto{\pgfqpoint{3.800846in}{3.277352in}}%
\pgfpathlineto{\pgfqpoint{3.794524in}{3.289265in}}%
\pgfpathlineto{\pgfqpoint{3.788202in}{3.301044in}}%
\pgfpathlineto{\pgfqpoint{3.781880in}{3.312660in}}%
\pgfpathlineto{\pgfqpoint{3.775558in}{3.324084in}}%
\pgfpathlineto{\pgfqpoint{3.764524in}{3.320861in}}%
\pgfpathlineto{\pgfqpoint{3.753485in}{3.317604in}}%
\pgfpathlineto{\pgfqpoint{3.742439in}{3.314325in}}%
\pgfpathlineto{\pgfqpoint{3.731388in}{3.311032in}}%
\pgfpathlineto{\pgfqpoint{3.720332in}{3.307728in}}%
\pgfpathlineto{\pgfqpoint{3.726646in}{3.296176in}}%
\pgfpathlineto{\pgfqpoint{3.732960in}{3.284433in}}%
\pgfpathlineto{\pgfqpoint{3.739275in}{3.272530in}}%
\pgfpathlineto{\pgfqpoint{3.745590in}{3.260498in}}%
\pgfpathclose%
\pgfusepath{stroke,fill}%
\end{pgfscope}%
\begin{pgfscope}%
\pgfpathrectangle{\pgfqpoint{0.887500in}{0.275000in}}{\pgfqpoint{4.225000in}{4.225000in}}%
\pgfusepath{clip}%
\pgfsetbuttcap%
\pgfsetroundjoin%
\definecolor{currentfill}{rgb}{0.226397,0.728888,0.462789}%
\pgfsetfillcolor{currentfill}%
\pgfsetfillopacity{0.700000}%
\pgfsetlinewidth{0.501875pt}%
\definecolor{currentstroke}{rgb}{1.000000,1.000000,1.000000}%
\pgfsetstrokecolor{currentstroke}%
\pgfsetstrokeopacity{0.500000}%
\pgfsetdash{}{0pt}%
\pgfpathmoveto{\pgfqpoint{2.868092in}{2.905654in}}%
\pgfpathlineto{\pgfqpoint{2.879259in}{2.930837in}}%
\pgfpathlineto{\pgfqpoint{2.890427in}{2.957392in}}%
\pgfpathlineto{\pgfqpoint{2.901597in}{2.985474in}}%
\pgfpathlineto{\pgfqpoint{2.912769in}{3.015235in}}%
\pgfpathlineto{\pgfqpoint{2.923944in}{3.046830in}}%
\pgfpathlineto{\pgfqpoint{2.917852in}{3.046801in}}%
\pgfpathlineto{\pgfqpoint{2.911767in}{3.046563in}}%
\pgfpathlineto{\pgfqpoint{2.905690in}{3.046105in}}%
\pgfpathlineto{\pgfqpoint{2.899620in}{3.045419in}}%
\pgfpathlineto{\pgfqpoint{2.893557in}{3.044503in}}%
\pgfpathlineto{\pgfqpoint{2.882422in}{3.013842in}}%
\pgfpathlineto{\pgfqpoint{2.871279in}{2.987213in}}%
\pgfpathlineto{\pgfqpoint{2.860126in}{2.963911in}}%
\pgfpathlineto{\pgfqpoint{2.848966in}{2.943236in}}%
\pgfpathlineto{\pgfqpoint{2.837800in}{2.924487in}}%
\pgfpathlineto{\pgfqpoint{2.843849in}{2.920402in}}%
\pgfpathlineto{\pgfqpoint{2.849904in}{2.916399in}}%
\pgfpathlineto{\pgfqpoint{2.855962in}{2.912557in}}%
\pgfpathlineto{\pgfqpoint{2.862025in}{2.908953in}}%
\pgfpathclose%
\pgfusepath{stroke,fill}%
\end{pgfscope}%
\begin{pgfscope}%
\pgfpathrectangle{\pgfqpoint{0.887500in}{0.275000in}}{\pgfqpoint{4.225000in}{4.225000in}}%
\pgfusepath{clip}%
\pgfsetbuttcap%
\pgfsetroundjoin%
\definecolor{currentfill}{rgb}{0.127568,0.566949,0.550556}%
\pgfsetfillcolor{currentfill}%
\pgfsetfillopacity{0.700000}%
\pgfsetlinewidth{0.501875pt}%
\definecolor{currentstroke}{rgb}{1.000000,1.000000,1.000000}%
\pgfsetstrokecolor{currentstroke}%
\pgfsetstrokeopacity{0.500000}%
\pgfsetdash{}{0pt}%
\pgfpathmoveto{\pgfqpoint{2.070549in}{2.602266in}}%
\pgfpathlineto{\pgfqpoint{2.082021in}{2.605566in}}%
\pgfpathlineto{\pgfqpoint{2.093488in}{2.608853in}}%
\pgfpathlineto{\pgfqpoint{2.104950in}{2.612127in}}%
\pgfpathlineto{\pgfqpoint{2.116406in}{2.615387in}}%
\pgfpathlineto{\pgfqpoint{2.127856in}{2.618643in}}%
\pgfpathlineto{\pgfqpoint{2.122014in}{2.626710in}}%
\pgfpathlineto{\pgfqpoint{2.116175in}{2.634764in}}%
\pgfpathlineto{\pgfqpoint{2.110341in}{2.642807in}}%
\pgfpathlineto{\pgfqpoint{2.104510in}{2.650838in}}%
\pgfpathlineto{\pgfqpoint{2.098684in}{2.658857in}}%
\pgfpathlineto{\pgfqpoint{2.087245in}{2.655614in}}%
\pgfpathlineto{\pgfqpoint{2.075801in}{2.652366in}}%
\pgfpathlineto{\pgfqpoint{2.064352in}{2.649098in}}%
\pgfpathlineto{\pgfqpoint{2.052898in}{2.645810in}}%
\pgfpathlineto{\pgfqpoint{2.041438in}{2.642506in}}%
\pgfpathlineto{\pgfqpoint{2.047252in}{2.634482in}}%
\pgfpathlineto{\pgfqpoint{2.053070in}{2.626446in}}%
\pgfpathlineto{\pgfqpoint{2.058893in}{2.618398in}}%
\pgfpathlineto{\pgfqpoint{2.064719in}{2.610338in}}%
\pgfpathclose%
\pgfusepath{stroke,fill}%
\end{pgfscope}%
\begin{pgfscope}%
\pgfpathrectangle{\pgfqpoint{0.887500in}{0.275000in}}{\pgfqpoint{4.225000in}{4.225000in}}%
\pgfusepath{clip}%
\pgfsetbuttcap%
\pgfsetroundjoin%
\definecolor{currentfill}{rgb}{0.119512,0.607464,0.540218}%
\pgfsetfillcolor{currentfill}%
\pgfsetfillopacity{0.700000}%
\pgfsetlinewidth{0.501875pt}%
\definecolor{currentstroke}{rgb}{1.000000,1.000000,1.000000}%
\pgfsetstrokecolor{currentstroke}%
\pgfsetstrokeopacity{0.500000}%
\pgfsetdash{}{0pt}%
\pgfpathmoveto{\pgfqpoint{1.523203in}{2.684100in}}%
\pgfpathlineto{\pgfqpoint{1.534808in}{2.687395in}}%
\pgfpathlineto{\pgfqpoint{1.546408in}{2.690687in}}%
\pgfpathlineto{\pgfqpoint{1.558002in}{2.693978in}}%
\pgfpathlineto{\pgfqpoint{1.569590in}{2.697269in}}%
\pgfpathlineto{\pgfqpoint{1.581173in}{2.700559in}}%
\pgfpathlineto{\pgfqpoint{1.575528in}{2.708184in}}%
\pgfpathlineto{\pgfqpoint{1.569887in}{2.715793in}}%
\pgfpathlineto{\pgfqpoint{1.564250in}{2.723385in}}%
\pgfpathlineto{\pgfqpoint{1.558618in}{2.730962in}}%
\pgfpathlineto{\pgfqpoint{1.552990in}{2.738522in}}%
\pgfpathlineto{\pgfqpoint{1.541421in}{2.735215in}}%
\pgfpathlineto{\pgfqpoint{1.529847in}{2.731910in}}%
\pgfpathlineto{\pgfqpoint{1.518267in}{2.728605in}}%
\pgfpathlineto{\pgfqpoint{1.506682in}{2.725301in}}%
\pgfpathlineto{\pgfqpoint{1.495090in}{2.721995in}}%
\pgfpathlineto{\pgfqpoint{1.500704in}{2.714450in}}%
\pgfpathlineto{\pgfqpoint{1.506322in}{2.706887in}}%
\pgfpathlineto{\pgfqpoint{1.511945in}{2.699308in}}%
\pgfpathlineto{\pgfqpoint{1.517571in}{2.691712in}}%
\pgfpathclose%
\pgfusepath{stroke,fill}%
\end{pgfscope}%
\begin{pgfscope}%
\pgfpathrectangle{\pgfqpoint{0.887500in}{0.275000in}}{\pgfqpoint{4.225000in}{4.225000in}}%
\pgfusepath{clip}%
\pgfsetbuttcap%
\pgfsetroundjoin%
\definecolor{currentfill}{rgb}{0.128729,0.563265,0.551229}%
\pgfsetfillcolor{currentfill}%
\pgfsetfillopacity{0.700000}%
\pgfsetlinewidth{0.501875pt}%
\definecolor{currentstroke}{rgb}{1.000000,1.000000,1.000000}%
\pgfsetstrokecolor{currentstroke}%
\pgfsetstrokeopacity{0.500000}%
\pgfsetdash{}{0pt}%
\pgfpathmoveto{\pgfqpoint{2.674167in}{2.568413in}}%
\pgfpathlineto{\pgfqpoint{2.685415in}{2.579406in}}%
\pgfpathlineto{\pgfqpoint{2.696659in}{2.590640in}}%
\pgfpathlineto{\pgfqpoint{2.707902in}{2.601925in}}%
\pgfpathlineto{\pgfqpoint{2.719145in}{2.613074in}}%
\pgfpathlineto{\pgfqpoint{2.730390in}{2.623937in}}%
\pgfpathlineto{\pgfqpoint{2.724326in}{2.634106in}}%
\pgfpathlineto{\pgfqpoint{2.718268in}{2.643917in}}%
\pgfpathlineto{\pgfqpoint{2.712216in}{2.653424in}}%
\pgfpathlineto{\pgfqpoint{2.706171in}{2.662679in}}%
\pgfpathlineto{\pgfqpoint{2.700130in}{2.671733in}}%
\pgfpathlineto{\pgfqpoint{2.688894in}{2.661657in}}%
\pgfpathlineto{\pgfqpoint{2.677664in}{2.650649in}}%
\pgfpathlineto{\pgfqpoint{2.666440in}{2.638978in}}%
\pgfpathlineto{\pgfqpoint{2.655219in}{2.626972in}}%
\pgfpathlineto{\pgfqpoint{2.643998in}{2.614962in}}%
\pgfpathlineto{\pgfqpoint{2.650018in}{2.606257in}}%
\pgfpathlineto{\pgfqpoint{2.656045in}{2.597275in}}%
\pgfpathlineto{\pgfqpoint{2.662079in}{2.587990in}}%
\pgfpathlineto{\pgfqpoint{2.668120in}{2.578378in}}%
\pgfpathclose%
\pgfusepath{stroke,fill}%
\end{pgfscope}%
\begin{pgfscope}%
\pgfpathrectangle{\pgfqpoint{0.887500in}{0.275000in}}{\pgfqpoint{4.225000in}{4.225000in}}%
\pgfusepath{clip}%
\pgfsetbuttcap%
\pgfsetroundjoin%
\definecolor{currentfill}{rgb}{0.135066,0.544853,0.554029}%
\pgfsetfillcolor{currentfill}%
\pgfsetfillopacity{0.700000}%
\pgfsetlinewidth{0.501875pt}%
\definecolor{currentstroke}{rgb}{1.000000,1.000000,1.000000}%
\pgfsetstrokecolor{currentstroke}%
\pgfsetstrokeopacity{0.500000}%
\pgfsetdash{}{0pt}%
\pgfpathmoveto{\pgfqpoint{2.387581in}{2.546897in}}%
\pgfpathlineto{\pgfqpoint{2.398974in}{2.550379in}}%
\pgfpathlineto{\pgfqpoint{2.410364in}{2.553747in}}%
\pgfpathlineto{\pgfqpoint{2.421751in}{2.556959in}}%
\pgfpathlineto{\pgfqpoint{2.433136in}{2.559976in}}%
\pgfpathlineto{\pgfqpoint{2.444517in}{2.562818in}}%
\pgfpathlineto{\pgfqpoint{2.438566in}{2.571061in}}%
\pgfpathlineto{\pgfqpoint{2.432618in}{2.579308in}}%
\pgfpathlineto{\pgfqpoint{2.426674in}{2.587562in}}%
\pgfpathlineto{\pgfqpoint{2.420734in}{2.595833in}}%
\pgfpathlineto{\pgfqpoint{2.414797in}{2.604126in}}%
\pgfpathlineto{\pgfqpoint{2.403427in}{2.601293in}}%
\pgfpathlineto{\pgfqpoint{2.392054in}{2.598283in}}%
\pgfpathlineto{\pgfqpoint{2.380678in}{2.595072in}}%
\pgfpathlineto{\pgfqpoint{2.369300in}{2.591701in}}%
\pgfpathlineto{\pgfqpoint{2.357919in}{2.588214in}}%
\pgfpathlineto{\pgfqpoint{2.363842in}{2.580014in}}%
\pgfpathlineto{\pgfqpoint{2.369770in}{2.571778in}}%
\pgfpathlineto{\pgfqpoint{2.375703in}{2.563511in}}%
\pgfpathlineto{\pgfqpoint{2.381640in}{2.555216in}}%
\pgfpathclose%
\pgfusepath{stroke,fill}%
\end{pgfscope}%
\begin{pgfscope}%
\pgfpathrectangle{\pgfqpoint{0.887500in}{0.275000in}}{\pgfqpoint{4.225000in}{4.225000in}}%
\pgfusepath{clip}%
\pgfsetbuttcap%
\pgfsetroundjoin%
\definecolor{currentfill}{rgb}{0.783315,0.879285,0.125405}%
\pgfsetfillcolor{currentfill}%
\pgfsetfillopacity{0.700000}%
\pgfsetlinewidth{0.501875pt}%
\definecolor{currentstroke}{rgb}{1.000000,1.000000,1.000000}%
\pgfsetstrokecolor{currentstroke}%
\pgfsetstrokeopacity{0.500000}%
\pgfsetdash{}{0pt}%
\pgfpathmoveto{\pgfqpoint{3.261109in}{3.414291in}}%
\pgfpathlineto{\pgfqpoint{3.272297in}{3.418081in}}%
\pgfpathlineto{\pgfqpoint{3.283480in}{3.421988in}}%
\pgfpathlineto{\pgfqpoint{3.294659in}{3.425985in}}%
\pgfpathlineto{\pgfqpoint{3.305833in}{3.430043in}}%
\pgfpathlineto{\pgfqpoint{3.317003in}{3.434134in}}%
\pgfpathlineto{\pgfqpoint{3.310770in}{3.443302in}}%
\pgfpathlineto{\pgfqpoint{3.304540in}{3.452121in}}%
\pgfpathlineto{\pgfqpoint{3.298311in}{3.460592in}}%
\pgfpathlineto{\pgfqpoint{3.292084in}{3.468732in}}%
\pgfpathlineto{\pgfqpoint{3.285860in}{3.476554in}}%
\pgfpathlineto{\pgfqpoint{3.274701in}{3.472491in}}%
\pgfpathlineto{\pgfqpoint{3.263538in}{3.468466in}}%
\pgfpathlineto{\pgfqpoint{3.252370in}{3.464510in}}%
\pgfpathlineto{\pgfqpoint{3.241198in}{3.460654in}}%
\pgfpathlineto{\pgfqpoint{3.230021in}{3.456930in}}%
\pgfpathlineto{\pgfqpoint{3.236233in}{3.448982in}}%
\pgfpathlineto{\pgfqpoint{3.242448in}{3.440757in}}%
\pgfpathlineto{\pgfqpoint{3.248666in}{3.432239in}}%
\pgfpathlineto{\pgfqpoint{3.254886in}{3.423417in}}%
\pgfpathclose%
\pgfusepath{stroke,fill}%
\end{pgfscope}%
\begin{pgfscope}%
\pgfpathrectangle{\pgfqpoint{0.887500in}{0.275000in}}{\pgfqpoint{4.225000in}{4.225000in}}%
\pgfusepath{clip}%
\pgfsetbuttcap%
\pgfsetroundjoin%
\definecolor{currentfill}{rgb}{0.122606,0.585371,0.546557}%
\pgfsetfillcolor{currentfill}%
\pgfsetfillopacity{0.700000}%
\pgfsetlinewidth{0.501875pt}%
\definecolor{currentstroke}{rgb}{1.000000,1.000000,1.000000}%
\pgfsetstrokecolor{currentstroke}%
\pgfsetstrokeopacity{0.500000}%
\pgfsetdash{}{0pt}%
\pgfpathmoveto{\pgfqpoint{1.840053in}{2.632806in}}%
\pgfpathlineto{\pgfqpoint{1.851586in}{2.636054in}}%
\pgfpathlineto{\pgfqpoint{1.863113in}{2.639310in}}%
\pgfpathlineto{\pgfqpoint{1.874634in}{2.642574in}}%
\pgfpathlineto{\pgfqpoint{1.886149in}{2.645850in}}%
\pgfpathlineto{\pgfqpoint{1.897658in}{2.649139in}}%
\pgfpathlineto{\pgfqpoint{1.891895in}{2.657064in}}%
\pgfpathlineto{\pgfqpoint{1.886135in}{2.664971in}}%
\pgfpathlineto{\pgfqpoint{1.880380in}{2.672859in}}%
\pgfpathlineto{\pgfqpoint{1.874629in}{2.680727in}}%
\pgfpathlineto{\pgfqpoint{1.868883in}{2.688577in}}%
\pgfpathlineto{\pgfqpoint{1.857387in}{2.685286in}}%
\pgfpathlineto{\pgfqpoint{1.845885in}{2.682007in}}%
\pgfpathlineto{\pgfqpoint{1.834377in}{2.678738in}}%
\pgfpathlineto{\pgfqpoint{1.822863in}{2.675478in}}%
\pgfpathlineto{\pgfqpoint{1.811343in}{2.672224in}}%
\pgfpathlineto{\pgfqpoint{1.817076in}{2.664379in}}%
\pgfpathlineto{\pgfqpoint{1.822814in}{2.656515in}}%
\pgfpathlineto{\pgfqpoint{1.828556in}{2.648631in}}%
\pgfpathlineto{\pgfqpoint{1.834302in}{2.640728in}}%
\pgfpathclose%
\pgfusepath{stroke,fill}%
\end{pgfscope}%
\begin{pgfscope}%
\pgfpathrectangle{\pgfqpoint{0.887500in}{0.275000in}}{\pgfqpoint{4.225000in}{4.225000in}}%
\pgfusepath{clip}%
\pgfsetbuttcap%
\pgfsetroundjoin%
\definecolor{currentfill}{rgb}{0.636902,0.856542,0.216620}%
\pgfsetfillcolor{currentfill}%
\pgfsetfillopacity{0.700000}%
\pgfsetlinewidth{0.501875pt}%
\definecolor{currentstroke}{rgb}{1.000000,1.000000,1.000000}%
\pgfsetstrokecolor{currentstroke}%
\pgfsetstrokeopacity{0.500000}%
\pgfsetdash{}{0pt}%
\pgfpathmoveto{\pgfqpoint{2.918764in}{3.246537in}}%
\pgfpathlineto{\pgfqpoint{2.929895in}{3.287920in}}%
\pgfpathlineto{\pgfqpoint{2.941056in}{3.324329in}}%
\pgfpathlineto{\pgfqpoint{2.952246in}{3.354740in}}%
\pgfpathlineto{\pgfqpoint{2.963458in}{3.379673in}}%
\pgfpathlineto{\pgfqpoint{2.974687in}{3.399780in}}%
\pgfpathlineto{\pgfqpoint{2.968554in}{3.405255in}}%
\pgfpathlineto{\pgfqpoint{2.962426in}{3.410396in}}%
\pgfpathlineto{\pgfqpoint{2.956305in}{3.415217in}}%
\pgfpathlineto{\pgfqpoint{2.950188in}{3.419769in}}%
\pgfpathlineto{\pgfqpoint{2.944078in}{3.424104in}}%
\pgfpathlineto{\pgfqpoint{2.932882in}{3.402595in}}%
\pgfpathlineto{\pgfqpoint{2.921710in}{3.375364in}}%
\pgfpathlineto{\pgfqpoint{2.910569in}{3.341598in}}%
\pgfpathlineto{\pgfqpoint{2.899468in}{3.300660in}}%
\pgfpathlineto{\pgfqpoint{2.888407in}{3.253877in}}%
\pgfpathlineto{\pgfqpoint{2.894467in}{3.252378in}}%
\pgfpathlineto{\pgfqpoint{2.900531in}{3.251071in}}%
\pgfpathlineto{\pgfqpoint{2.906602in}{3.249767in}}%
\pgfpathlineto{\pgfqpoint{2.912680in}{3.248281in}}%
\pgfpathclose%
\pgfusepath{stroke,fill}%
\end{pgfscope}%
\begin{pgfscope}%
\pgfpathrectangle{\pgfqpoint{0.887500in}{0.275000in}}{\pgfqpoint{4.225000in}{4.225000in}}%
\pgfusepath{clip}%
\pgfsetbuttcap%
\pgfsetroundjoin%
\definecolor{currentfill}{rgb}{0.814576,0.883393,0.110347}%
\pgfsetfillcolor{currentfill}%
\pgfsetfillopacity{0.700000}%
\pgfsetlinewidth{0.501875pt}%
\definecolor{currentstroke}{rgb}{1.000000,1.000000,1.000000}%
\pgfsetstrokecolor{currentstroke}%
\pgfsetstrokeopacity{0.500000}%
\pgfsetdash{}{0pt}%
\pgfpathmoveto{\pgfqpoint{3.030922in}{3.450907in}}%
\pgfpathlineto{\pgfqpoint{3.042166in}{3.455260in}}%
\pgfpathlineto{\pgfqpoint{3.053406in}{3.458535in}}%
\pgfpathlineto{\pgfqpoint{3.064640in}{3.461033in}}%
\pgfpathlineto{\pgfqpoint{3.075869in}{3.463055in}}%
\pgfpathlineto{\pgfqpoint{3.087091in}{3.464903in}}%
\pgfpathlineto{\pgfqpoint{3.080923in}{3.470268in}}%
\pgfpathlineto{\pgfqpoint{3.074761in}{3.475071in}}%
\pgfpathlineto{\pgfqpoint{3.068603in}{3.479321in}}%
\pgfpathlineto{\pgfqpoint{3.062450in}{3.483087in}}%
\pgfpathlineto{\pgfqpoint{3.056303in}{3.486440in}}%
\pgfpathlineto{\pgfqpoint{3.045096in}{3.484468in}}%
\pgfpathlineto{\pgfqpoint{3.033882in}{3.482673in}}%
\pgfpathlineto{\pgfqpoint{3.022663in}{3.480659in}}%
\pgfpathlineto{\pgfqpoint{3.011439in}{3.478032in}}%
\pgfpathlineto{\pgfqpoint{3.000210in}{3.474397in}}%
\pgfpathlineto{\pgfqpoint{3.006341in}{3.470495in}}%
\pgfpathlineto{\pgfqpoint{3.012478in}{3.466246in}}%
\pgfpathlineto{\pgfqpoint{3.018621in}{3.461594in}}%
\pgfpathlineto{\pgfqpoint{3.024768in}{3.456484in}}%
\pgfpathclose%
\pgfusepath{stroke,fill}%
\end{pgfscope}%
\begin{pgfscope}%
\pgfpathrectangle{\pgfqpoint{0.887500in}{0.275000in}}{\pgfqpoint{4.225000in}{4.225000in}}%
\pgfusepath{clip}%
\pgfsetbuttcap%
\pgfsetroundjoin%
\definecolor{currentfill}{rgb}{0.136408,0.541173,0.554483}%
\pgfsetfillcolor{currentfill}%
\pgfsetfillopacity{0.700000}%
\pgfsetlinewidth{0.501875pt}%
\definecolor{currentstroke}{rgb}{1.000000,1.000000,1.000000}%
\pgfsetstrokecolor{currentstroke}%
\pgfsetstrokeopacity{0.500000}%
\pgfsetdash{}{0pt}%
\pgfpathmoveto{\pgfqpoint{2.617805in}{2.523073in}}%
\pgfpathlineto{\pgfqpoint{2.629100in}{2.530494in}}%
\pgfpathlineto{\pgfqpoint{2.640383in}{2.538751in}}%
\pgfpathlineto{\pgfqpoint{2.651654in}{2.547898in}}%
\pgfpathlineto{\pgfqpoint{2.662914in}{2.557848in}}%
\pgfpathlineto{\pgfqpoint{2.674167in}{2.568413in}}%
\pgfpathlineto{\pgfqpoint{2.668120in}{2.578378in}}%
\pgfpathlineto{\pgfqpoint{2.662079in}{2.587990in}}%
\pgfpathlineto{\pgfqpoint{2.656045in}{2.597275in}}%
\pgfpathlineto{\pgfqpoint{2.650018in}{2.606257in}}%
\pgfpathlineto{\pgfqpoint{2.643998in}{2.614962in}}%
\pgfpathlineto{\pgfqpoint{2.632772in}{2.603276in}}%
\pgfpathlineto{\pgfqpoint{2.621538in}{2.592243in}}%
\pgfpathlineto{\pgfqpoint{2.610293in}{2.582188in}}%
\pgfpathlineto{\pgfqpoint{2.599032in}{2.573272in}}%
\pgfpathlineto{\pgfqpoint{2.587757in}{2.565406in}}%
\pgfpathlineto{\pgfqpoint{2.593756in}{2.557165in}}%
\pgfpathlineto{\pgfqpoint{2.599761in}{2.548799in}}%
\pgfpathlineto{\pgfqpoint{2.605771in}{2.540320in}}%
\pgfpathlineto{\pgfqpoint{2.611786in}{2.531741in}}%
\pgfpathclose%
\pgfusepath{stroke,fill}%
\end{pgfscope}%
\begin{pgfscope}%
\pgfpathrectangle{\pgfqpoint{0.887500in}{0.275000in}}{\pgfqpoint{4.225000in}{4.225000in}}%
\pgfusepath{clip}%
\pgfsetbuttcap%
\pgfsetroundjoin%
\definecolor{currentfill}{rgb}{0.129933,0.559582,0.551864}%
\pgfsetfillcolor{currentfill}%
\pgfsetfillopacity{0.700000}%
\pgfsetlinewidth{0.501875pt}%
\definecolor{currentstroke}{rgb}{1.000000,1.000000,1.000000}%
\pgfsetstrokecolor{currentstroke}%
\pgfsetstrokeopacity{0.500000}%
\pgfsetdash{}{0pt}%
\pgfpathmoveto{\pgfqpoint{2.157131in}{2.578111in}}%
\pgfpathlineto{\pgfqpoint{2.168588in}{2.581382in}}%
\pgfpathlineto{\pgfqpoint{2.180039in}{2.584663in}}%
\pgfpathlineto{\pgfqpoint{2.191484in}{2.587957in}}%
\pgfpathlineto{\pgfqpoint{2.202923in}{2.591271in}}%
\pgfpathlineto{\pgfqpoint{2.214356in}{2.594610in}}%
\pgfpathlineto{\pgfqpoint{2.208480in}{2.602768in}}%
\pgfpathlineto{\pgfqpoint{2.202608in}{2.610912in}}%
\pgfpathlineto{\pgfqpoint{2.196741in}{2.619045in}}%
\pgfpathlineto{\pgfqpoint{2.190877in}{2.627164in}}%
\pgfpathlineto{\pgfqpoint{2.185017in}{2.635271in}}%
\pgfpathlineto{\pgfqpoint{2.173598in}{2.631859in}}%
\pgfpathlineto{\pgfqpoint{2.162172in}{2.628502in}}%
\pgfpathlineto{\pgfqpoint{2.150740in}{2.625188in}}%
\pgfpathlineto{\pgfqpoint{2.139301in}{2.621906in}}%
\pgfpathlineto{\pgfqpoint{2.127856in}{2.618643in}}%
\pgfpathlineto{\pgfqpoint{2.133703in}{2.610563in}}%
\pgfpathlineto{\pgfqpoint{2.139554in}{2.602470in}}%
\pgfpathlineto{\pgfqpoint{2.145409in}{2.594364in}}%
\pgfpathlineto{\pgfqpoint{2.151268in}{2.586245in}}%
\pgfpathclose%
\pgfusepath{stroke,fill}%
\end{pgfscope}%
\begin{pgfscope}%
\pgfpathrectangle{\pgfqpoint{0.887500in}{0.275000in}}{\pgfqpoint{4.225000in}{4.225000in}}%
\pgfusepath{clip}%
\pgfsetbuttcap%
\pgfsetroundjoin%
\definecolor{currentfill}{rgb}{0.120092,0.600104,0.542530}%
\pgfsetfillcolor{currentfill}%
\pgfsetfillopacity{0.700000}%
\pgfsetlinewidth{0.501875pt}%
\definecolor{currentstroke}{rgb}{1.000000,1.000000,1.000000}%
\pgfsetstrokecolor{currentstroke}%
\pgfsetstrokeopacity{0.500000}%
\pgfsetdash{}{0pt}%
\pgfpathmoveto{\pgfqpoint{1.609464in}{2.662200in}}%
\pgfpathlineto{\pgfqpoint{1.621055in}{2.665478in}}%
\pgfpathlineto{\pgfqpoint{1.632640in}{2.668755in}}%
\pgfpathlineto{\pgfqpoint{1.644220in}{2.672030in}}%
\pgfpathlineto{\pgfqpoint{1.655794in}{2.675303in}}%
\pgfpathlineto{\pgfqpoint{1.667363in}{2.678574in}}%
\pgfpathlineto{\pgfqpoint{1.661682in}{2.686289in}}%
\pgfpathlineto{\pgfqpoint{1.656006in}{2.693988in}}%
\pgfpathlineto{\pgfqpoint{1.650334in}{2.701672in}}%
\pgfpathlineto{\pgfqpoint{1.644666in}{2.709342in}}%
\pgfpathlineto{\pgfqpoint{1.639003in}{2.716997in}}%
\pgfpathlineto{\pgfqpoint{1.627448in}{2.713713in}}%
\pgfpathlineto{\pgfqpoint{1.615888in}{2.710427in}}%
\pgfpathlineto{\pgfqpoint{1.604322in}{2.707139in}}%
\pgfpathlineto{\pgfqpoint{1.592750in}{2.703850in}}%
\pgfpathlineto{\pgfqpoint{1.581173in}{2.700559in}}%
\pgfpathlineto{\pgfqpoint{1.586823in}{2.692919in}}%
\pgfpathlineto{\pgfqpoint{1.592477in}{2.685264in}}%
\pgfpathlineto{\pgfqpoint{1.598135in}{2.677593in}}%
\pgfpathlineto{\pgfqpoint{1.603797in}{2.669905in}}%
\pgfpathclose%
\pgfusepath{stroke,fill}%
\end{pgfscope}%
\begin{pgfscope}%
\pgfpathrectangle{\pgfqpoint{0.887500in}{0.275000in}}{\pgfqpoint{4.225000in}{4.225000in}}%
\pgfusepath{clip}%
\pgfsetbuttcap%
\pgfsetroundjoin%
\definecolor{currentfill}{rgb}{0.139147,0.533812,0.555298}%
\pgfsetfillcolor{currentfill}%
\pgfsetfillopacity{0.700000}%
\pgfsetlinewidth{0.501875pt}%
\definecolor{currentstroke}{rgb}{1.000000,1.000000,1.000000}%
\pgfsetstrokecolor{currentstroke}%
\pgfsetstrokeopacity{0.500000}%
\pgfsetdash{}{0pt}%
\pgfpathmoveto{\pgfqpoint{2.474334in}{2.521401in}}%
\pgfpathlineto{\pgfqpoint{2.485719in}{2.524430in}}%
\pgfpathlineto{\pgfqpoint{2.497096in}{2.527523in}}%
\pgfpathlineto{\pgfqpoint{2.508466in}{2.530768in}}%
\pgfpathlineto{\pgfqpoint{2.519827in}{2.534254in}}%
\pgfpathlineto{\pgfqpoint{2.531178in}{2.538066in}}%
\pgfpathlineto{\pgfqpoint{2.525197in}{2.546210in}}%
\pgfpathlineto{\pgfqpoint{2.519221in}{2.554250in}}%
\pgfpathlineto{\pgfqpoint{2.513251in}{2.562216in}}%
\pgfpathlineto{\pgfqpoint{2.507284in}{2.570153in}}%
\pgfpathlineto{\pgfqpoint{2.501322in}{2.578103in}}%
\pgfpathlineto{\pgfqpoint{2.489979in}{2.574542in}}%
\pgfpathlineto{\pgfqpoint{2.478626in}{2.571341in}}%
\pgfpathlineto{\pgfqpoint{2.467264in}{2.568392in}}%
\pgfpathlineto{\pgfqpoint{2.455894in}{2.565587in}}%
\pgfpathlineto{\pgfqpoint{2.444517in}{2.562818in}}%
\pgfpathlineto{\pgfqpoint{2.450472in}{2.554569in}}%
\pgfpathlineto{\pgfqpoint{2.456432in}{2.546310in}}%
\pgfpathlineto{\pgfqpoint{2.462395in}{2.538033in}}%
\pgfpathlineto{\pgfqpoint{2.468363in}{2.529732in}}%
\pgfpathclose%
\pgfusepath{stroke,fill}%
\end{pgfscope}%
\begin{pgfscope}%
\pgfpathrectangle{\pgfqpoint{0.887500in}{0.275000in}}{\pgfqpoint{4.225000in}{4.225000in}}%
\pgfusepath{clip}%
\pgfsetbuttcap%
\pgfsetroundjoin%
\definecolor{currentfill}{rgb}{0.246070,0.738910,0.452024}%
\pgfsetfillcolor{currentfill}%
\pgfsetfillopacity{0.700000}%
\pgfsetlinewidth{0.501875pt}%
\definecolor{currentstroke}{rgb}{1.000000,1.000000,1.000000}%
\pgfsetstrokecolor{currentstroke}%
\pgfsetstrokeopacity{0.500000}%
\pgfsetdash{}{0pt}%
\pgfpathmoveto{\pgfqpoint{4.218028in}{2.952053in}}%
\pgfpathlineto{\pgfqpoint{4.228986in}{2.955518in}}%
\pgfpathlineto{\pgfqpoint{4.239939in}{2.958979in}}%
\pgfpathlineto{\pgfqpoint{4.250886in}{2.962438in}}%
\pgfpathlineto{\pgfqpoint{4.261828in}{2.965894in}}%
\pgfpathlineto{\pgfqpoint{4.255444in}{2.979605in}}%
\pgfpathlineto{\pgfqpoint{4.249059in}{2.993215in}}%
\pgfpathlineto{\pgfqpoint{4.242673in}{3.006711in}}%
\pgfpathlineto{\pgfqpoint{4.236286in}{3.020079in}}%
\pgfpathlineto{\pgfqpoint{4.229898in}{3.033310in}}%
\pgfpathlineto{\pgfqpoint{4.218959in}{3.029887in}}%
\pgfpathlineto{\pgfqpoint{4.208015in}{3.026462in}}%
\pgfpathlineto{\pgfqpoint{4.197066in}{3.023036in}}%
\pgfpathlineto{\pgfqpoint{4.186112in}{3.019608in}}%
\pgfpathlineto{\pgfqpoint{4.192498in}{3.006370in}}%
\pgfpathlineto{\pgfqpoint{4.198882in}{2.992984in}}%
\pgfpathlineto{\pgfqpoint{4.205265in}{2.979458in}}%
\pgfpathlineto{\pgfqpoint{4.211647in}{2.965809in}}%
\pgfpathclose%
\pgfusepath{stroke,fill}%
\end{pgfscope}%
\begin{pgfscope}%
\pgfpathrectangle{\pgfqpoint{0.887500in}{0.275000in}}{\pgfqpoint{4.225000in}{4.225000in}}%
\pgfusepath{clip}%
\pgfsetbuttcap%
\pgfsetroundjoin%
\definecolor{currentfill}{rgb}{0.751884,0.874951,0.143228}%
\pgfsetfillcolor{currentfill}%
\pgfsetfillopacity{0.700000}%
\pgfsetlinewidth{0.501875pt}%
\definecolor{currentstroke}{rgb}{1.000000,1.000000,1.000000}%
\pgfsetstrokecolor{currentstroke}%
\pgfsetstrokeopacity{0.500000}%
\pgfsetdash{}{0pt}%
\pgfpathmoveto{\pgfqpoint{3.348192in}{3.384330in}}%
\pgfpathlineto{\pgfqpoint{3.359364in}{3.388228in}}%
\pgfpathlineto{\pgfqpoint{3.370531in}{3.392133in}}%
\pgfpathlineto{\pgfqpoint{3.381693in}{3.396037in}}%
\pgfpathlineto{\pgfqpoint{3.392850in}{3.399932in}}%
\pgfpathlineto{\pgfqpoint{3.404001in}{3.403811in}}%
\pgfpathlineto{\pgfqpoint{3.397754in}{3.414475in}}%
\pgfpathlineto{\pgfqpoint{3.391508in}{3.424921in}}%
\pgfpathlineto{\pgfqpoint{3.385263in}{3.435107in}}%
\pgfpathlineto{\pgfqpoint{3.379019in}{3.444992in}}%
\pgfpathlineto{\pgfqpoint{3.372777in}{3.454534in}}%
\pgfpathlineto{\pgfqpoint{3.361632in}{3.450512in}}%
\pgfpathlineto{\pgfqpoint{3.350482in}{3.446446in}}%
\pgfpathlineto{\pgfqpoint{3.339327in}{3.442351in}}%
\pgfpathlineto{\pgfqpoint{3.328167in}{3.438242in}}%
\pgfpathlineto{\pgfqpoint{3.317003in}{3.434134in}}%
\pgfpathlineto{\pgfqpoint{3.323236in}{3.424652in}}%
\pgfpathlineto{\pgfqpoint{3.329472in}{3.414891in}}%
\pgfpathlineto{\pgfqpoint{3.335710in}{3.404892in}}%
\pgfpathlineto{\pgfqpoint{3.341950in}{3.394692in}}%
\pgfpathclose%
\pgfusepath{stroke,fill}%
\end{pgfscope}%
\begin{pgfscope}%
\pgfpathrectangle{\pgfqpoint{0.887500in}{0.275000in}}{\pgfqpoint{4.225000in}{4.225000in}}%
\pgfusepath{clip}%
\pgfsetbuttcap%
\pgfsetroundjoin%
\definecolor{currentfill}{rgb}{0.296479,0.761561,0.424223}%
\pgfsetfillcolor{currentfill}%
\pgfsetfillopacity{0.700000}%
\pgfsetlinewidth{0.501875pt}%
\definecolor{currentstroke}{rgb}{1.000000,1.000000,1.000000}%
\pgfsetstrokecolor{currentstroke}%
\pgfsetstrokeopacity{0.500000}%
\pgfsetdash{}{0pt}%
\pgfpathmoveto{\pgfqpoint{4.131262in}{3.002419in}}%
\pgfpathlineto{\pgfqpoint{4.142243in}{3.005865in}}%
\pgfpathlineto{\pgfqpoint{4.153218in}{3.009306in}}%
\pgfpathlineto{\pgfqpoint{4.164188in}{3.012743in}}%
\pgfpathlineto{\pgfqpoint{4.175153in}{3.016177in}}%
\pgfpathlineto{\pgfqpoint{4.186112in}{3.019608in}}%
\pgfpathlineto{\pgfqpoint{4.179725in}{3.032706in}}%
\pgfpathlineto{\pgfqpoint{4.173338in}{3.045675in}}%
\pgfpathlineto{\pgfqpoint{4.166950in}{3.058524in}}%
\pgfpathlineto{\pgfqpoint{4.160561in}{3.071263in}}%
\pgfpathlineto{\pgfqpoint{4.154173in}{3.083903in}}%
\pgfpathlineto{\pgfqpoint{4.143219in}{3.080541in}}%
\pgfpathlineto{\pgfqpoint{4.132260in}{3.077195in}}%
\pgfpathlineto{\pgfqpoint{4.121296in}{3.073865in}}%
\pgfpathlineto{\pgfqpoint{4.110328in}{3.070551in}}%
\pgfpathlineto{\pgfqpoint{4.099355in}{3.067254in}}%
\pgfpathlineto{\pgfqpoint{4.105735in}{3.054473in}}%
\pgfpathlineto{\pgfqpoint{4.112117in}{3.041610in}}%
\pgfpathlineto{\pgfqpoint{4.118499in}{3.028655in}}%
\pgfpathlineto{\pgfqpoint{4.124881in}{3.015595in}}%
\pgfpathclose%
\pgfusepath{stroke,fill}%
\end{pgfscope}%
\begin{pgfscope}%
\pgfpathrectangle{\pgfqpoint{0.887500in}{0.275000in}}{\pgfqpoint{4.225000in}{4.225000in}}%
\pgfusepath{clip}%
\pgfsetbuttcap%
\pgfsetroundjoin%
\definecolor{currentfill}{rgb}{0.119483,0.614817,0.537692}%
\pgfsetfillcolor{currentfill}%
\pgfsetfillopacity{0.700000}%
\pgfsetlinewidth{0.501875pt}%
\definecolor{currentstroke}{rgb}{1.000000,1.000000,1.000000}%
\pgfsetstrokecolor{currentstroke}%
\pgfsetstrokeopacity{0.500000}%
\pgfsetdash{}{0pt}%
\pgfpathmoveto{\pgfqpoint{1.378876in}{2.688681in}}%
\pgfpathlineto{\pgfqpoint{1.390521in}{2.692048in}}%
\pgfpathlineto{\pgfqpoint{1.402162in}{2.695404in}}%
\pgfpathlineto{\pgfqpoint{1.413797in}{2.698750in}}%
\pgfpathlineto{\pgfqpoint{1.425426in}{2.702089in}}%
\pgfpathlineto{\pgfqpoint{1.437051in}{2.705420in}}%
\pgfpathlineto{\pgfqpoint{1.431456in}{2.712937in}}%
\pgfpathlineto{\pgfqpoint{1.425865in}{2.720435in}}%
\pgfpathlineto{\pgfqpoint{1.420279in}{2.727914in}}%
\pgfpathlineto{\pgfqpoint{1.414697in}{2.735374in}}%
\pgfpathlineto{\pgfqpoint{1.409120in}{2.742813in}}%
\pgfpathlineto{\pgfqpoint{1.397510in}{2.739469in}}%
\pgfpathlineto{\pgfqpoint{1.385895in}{2.736117in}}%
\pgfpathlineto{\pgfqpoint{1.374275in}{2.732757in}}%
\pgfpathlineto{\pgfqpoint{1.362649in}{2.729389in}}%
\pgfpathlineto{\pgfqpoint{1.351018in}{2.726011in}}%
\pgfpathlineto{\pgfqpoint{1.356580in}{2.718589in}}%
\pgfpathlineto{\pgfqpoint{1.362147in}{2.711144in}}%
\pgfpathlineto{\pgfqpoint{1.367719in}{2.703678in}}%
\pgfpathlineto{\pgfqpoint{1.373295in}{2.696190in}}%
\pgfpathclose%
\pgfusepath{stroke,fill}%
\end{pgfscope}%
\begin{pgfscope}%
\pgfpathrectangle{\pgfqpoint{0.887500in}{0.275000in}}{\pgfqpoint{4.225000in}{4.225000in}}%
\pgfusepath{clip}%
\pgfsetbuttcap%
\pgfsetroundjoin%
\definecolor{currentfill}{rgb}{0.146616,0.673050,0.508936}%
\pgfsetfillcolor{currentfill}%
\pgfsetfillopacity{0.700000}%
\pgfsetlinewidth{0.501875pt}%
\definecolor{currentstroke}{rgb}{1.000000,1.000000,1.000000}%
\pgfsetstrokecolor{currentstroke}%
\pgfsetstrokeopacity{0.500000}%
\pgfsetdash{}{0pt}%
\pgfpathmoveto{\pgfqpoint{2.842569in}{2.769717in}}%
\pgfpathlineto{\pgfqpoint{2.853753in}{2.791910in}}%
\pgfpathlineto{\pgfqpoint{2.864935in}{2.815598in}}%
\pgfpathlineto{\pgfqpoint{2.876116in}{2.840755in}}%
\pgfpathlineto{\pgfqpoint{2.887299in}{2.867317in}}%
\pgfpathlineto{\pgfqpoint{2.898484in}{2.895100in}}%
\pgfpathlineto{\pgfqpoint{2.892398in}{2.896294in}}%
\pgfpathlineto{\pgfqpoint{2.886316in}{2.897984in}}%
\pgfpathlineto{\pgfqpoint{2.880237in}{2.900132in}}%
\pgfpathlineto{\pgfqpoint{2.874163in}{2.902701in}}%
\pgfpathlineto{\pgfqpoint{2.868092in}{2.905654in}}%
\pgfpathlineto{\pgfqpoint{2.856926in}{2.881692in}}%
\pgfpathlineto{\pgfqpoint{2.845761in}{2.858799in}}%
\pgfpathlineto{\pgfqpoint{2.834595in}{2.836862in}}%
\pgfpathlineto{\pgfqpoint{2.823429in}{2.815957in}}%
\pgfpathlineto{\pgfqpoint{2.812260in}{2.796220in}}%
\pgfpathlineto{\pgfqpoint{2.818321in}{2.789416in}}%
\pgfpathlineto{\pgfqpoint{2.824382in}{2.783335in}}%
\pgfpathlineto{\pgfqpoint{2.830444in}{2.778005in}}%
\pgfpathlineto{\pgfqpoint{2.836506in}{2.773456in}}%
\pgfpathclose%
\pgfusepath{stroke,fill}%
\end{pgfscope}%
\begin{pgfscope}%
\pgfpathrectangle{\pgfqpoint{0.887500in}{0.275000in}}{\pgfqpoint{4.225000in}{4.225000in}}%
\pgfusepath{clip}%
\pgfsetbuttcap%
\pgfsetroundjoin%
\definecolor{currentfill}{rgb}{0.124395,0.578002,0.548287}%
\pgfsetfillcolor{currentfill}%
\pgfsetfillopacity{0.700000}%
\pgfsetlinewidth{0.501875pt}%
\definecolor{currentstroke}{rgb}{1.000000,1.000000,1.000000}%
\pgfsetstrokecolor{currentstroke}%
\pgfsetstrokeopacity{0.500000}%
\pgfsetdash{}{0pt}%
\pgfpathmoveto{\pgfqpoint{1.926539in}{2.609269in}}%
\pgfpathlineto{\pgfqpoint{1.938055in}{2.612570in}}%
\pgfpathlineto{\pgfqpoint{1.949565in}{2.615882in}}%
\pgfpathlineto{\pgfqpoint{1.961069in}{2.619202in}}%
\pgfpathlineto{\pgfqpoint{1.972567in}{2.622529in}}%
\pgfpathlineto{\pgfqpoint{1.984060in}{2.625861in}}%
\pgfpathlineto{\pgfqpoint{1.978262in}{2.633866in}}%
\pgfpathlineto{\pgfqpoint{1.972469in}{2.641857in}}%
\pgfpathlineto{\pgfqpoint{1.966680in}{2.649834in}}%
\pgfpathlineto{\pgfqpoint{1.960895in}{2.657795in}}%
\pgfpathlineto{\pgfqpoint{1.955115in}{2.665740in}}%
\pgfpathlineto{\pgfqpoint{1.943635in}{2.662405in}}%
\pgfpathlineto{\pgfqpoint{1.932150in}{2.659075in}}%
\pgfpathlineto{\pgfqpoint{1.920658in}{2.655753in}}%
\pgfpathlineto{\pgfqpoint{1.909161in}{2.652441in}}%
\pgfpathlineto{\pgfqpoint{1.897658in}{2.649139in}}%
\pgfpathlineto{\pgfqpoint{1.903426in}{2.641197in}}%
\pgfpathlineto{\pgfqpoint{1.909198in}{2.633238in}}%
\pgfpathlineto{\pgfqpoint{1.914974in}{2.625263in}}%
\pgfpathlineto{\pgfqpoint{1.920755in}{2.617273in}}%
\pgfpathclose%
\pgfusepath{stroke,fill}%
\end{pgfscope}%
\begin{pgfscope}%
\pgfpathrectangle{\pgfqpoint{0.887500in}{0.275000in}}{\pgfqpoint{4.225000in}{4.225000in}}%
\pgfusepath{clip}%
\pgfsetbuttcap%
\pgfsetroundjoin%
\definecolor{currentfill}{rgb}{0.344074,0.780029,0.397381}%
\pgfsetfillcolor{currentfill}%
\pgfsetfillopacity{0.700000}%
\pgfsetlinewidth{0.501875pt}%
\definecolor{currentstroke}{rgb}{1.000000,1.000000,1.000000}%
\pgfsetstrokecolor{currentstroke}%
\pgfsetstrokeopacity{0.500000}%
\pgfsetdash{}{0pt}%
\pgfpathmoveto{\pgfqpoint{4.044416in}{3.051031in}}%
\pgfpathlineto{\pgfqpoint{4.055413in}{3.054240in}}%
\pgfpathlineto{\pgfqpoint{4.066406in}{3.057467in}}%
\pgfpathlineto{\pgfqpoint{4.077394in}{3.060712in}}%
\pgfpathlineto{\pgfqpoint{4.088377in}{3.063974in}}%
\pgfpathlineto{\pgfqpoint{4.099355in}{3.067254in}}%
\pgfpathlineto{\pgfqpoint{4.092975in}{3.079966in}}%
\pgfpathlineto{\pgfqpoint{4.086597in}{3.092621in}}%
\pgfpathlineto{\pgfqpoint{4.080220in}{3.105230in}}%
\pgfpathlineto{\pgfqpoint{4.073845in}{3.117805in}}%
\pgfpathlineto{\pgfqpoint{4.067473in}{3.130356in}}%
\pgfpathlineto{\pgfqpoint{4.056504in}{3.127275in}}%
\pgfpathlineto{\pgfqpoint{4.045529in}{3.124208in}}%
\pgfpathlineto{\pgfqpoint{4.034550in}{3.121157in}}%
\pgfpathlineto{\pgfqpoint{4.023566in}{3.118123in}}%
\pgfpathlineto{\pgfqpoint{4.012576in}{3.115107in}}%
\pgfpathlineto{\pgfqpoint{4.018942in}{3.102415in}}%
\pgfpathlineto{\pgfqpoint{4.025308in}{3.089666in}}%
\pgfpathlineto{\pgfqpoint{4.031676in}{3.076855in}}%
\pgfpathlineto{\pgfqpoint{4.038046in}{3.063978in}}%
\pgfpathclose%
\pgfusepath{stroke,fill}%
\end{pgfscope}%
\begin{pgfscope}%
\pgfpathrectangle{\pgfqpoint{0.887500in}{0.275000in}}{\pgfqpoint{4.225000in}{4.225000in}}%
\pgfusepath{clip}%
\pgfsetbuttcap%
\pgfsetroundjoin%
\definecolor{currentfill}{rgb}{0.709898,0.868751,0.169257}%
\pgfsetfillcolor{currentfill}%
\pgfsetfillopacity{0.700000}%
\pgfsetlinewidth{0.501875pt}%
\definecolor{currentstroke}{rgb}{1.000000,1.000000,1.000000}%
\pgfsetstrokecolor{currentstroke}%
\pgfsetstrokeopacity{0.500000}%
\pgfsetdash{}{0pt}%
\pgfpathmoveto{\pgfqpoint{3.435272in}{3.348673in}}%
\pgfpathlineto{\pgfqpoint{3.446423in}{3.352197in}}%
\pgfpathlineto{\pgfqpoint{3.457569in}{3.355756in}}%
\pgfpathlineto{\pgfqpoint{3.468709in}{3.359340in}}%
\pgfpathlineto{\pgfqpoint{3.479845in}{3.362935in}}%
\pgfpathlineto{\pgfqpoint{3.490975in}{3.366530in}}%
\pgfpathlineto{\pgfqpoint{3.484711in}{3.377819in}}%
\pgfpathlineto{\pgfqpoint{3.478450in}{3.389146in}}%
\pgfpathlineto{\pgfqpoint{3.472192in}{3.400453in}}%
\pgfpathlineto{\pgfqpoint{3.465936in}{3.411676in}}%
\pgfpathlineto{\pgfqpoint{3.459681in}{3.422752in}}%
\pgfpathlineto{\pgfqpoint{3.448556in}{3.419039in}}%
\pgfpathlineto{\pgfqpoint{3.437426in}{3.415285in}}%
\pgfpathlineto{\pgfqpoint{3.426290in}{3.411493in}}%
\pgfpathlineto{\pgfqpoint{3.415148in}{3.407667in}}%
\pgfpathlineto{\pgfqpoint{3.404001in}{3.403811in}}%
\pgfpathlineto{\pgfqpoint{3.410251in}{3.392970in}}%
\pgfpathlineto{\pgfqpoint{3.416502in}{3.381994in}}%
\pgfpathlineto{\pgfqpoint{3.422756in}{3.370926in}}%
\pgfpathlineto{\pgfqpoint{3.429013in}{3.359805in}}%
\pgfpathclose%
\pgfusepath{stroke,fill}%
\end{pgfscope}%
\begin{pgfscope}%
\pgfpathrectangle{\pgfqpoint{0.887500in}{0.275000in}}{\pgfqpoint{4.225000in}{4.225000in}}%
\pgfusepath{clip}%
\pgfsetbuttcap%
\pgfsetroundjoin%
\definecolor{currentfill}{rgb}{0.132444,0.552216,0.553018}%
\pgfsetfillcolor{currentfill}%
\pgfsetfillopacity{0.700000}%
\pgfsetlinewidth{0.501875pt}%
\definecolor{currentstroke}{rgb}{1.000000,1.000000,1.000000}%
\pgfsetstrokecolor{currentstroke}%
\pgfsetstrokeopacity{0.500000}%
\pgfsetdash{}{0pt}%
\pgfpathmoveto{\pgfqpoint{2.243795in}{2.553641in}}%
\pgfpathlineto{\pgfqpoint{2.255235in}{2.556951in}}%
\pgfpathlineto{\pgfqpoint{2.266670in}{2.560279in}}%
\pgfpathlineto{\pgfqpoint{2.278098in}{2.563634in}}%
\pgfpathlineto{\pgfqpoint{2.289519in}{2.567024in}}%
\pgfpathlineto{\pgfqpoint{2.300935in}{2.570454in}}%
\pgfpathlineto{\pgfqpoint{2.295026in}{2.578692in}}%
\pgfpathlineto{\pgfqpoint{2.289122in}{2.586926in}}%
\pgfpathlineto{\pgfqpoint{2.283221in}{2.595157in}}%
\pgfpathlineto{\pgfqpoint{2.277324in}{2.603385in}}%
\pgfpathlineto{\pgfqpoint{2.271431in}{2.611608in}}%
\pgfpathlineto{\pgfqpoint{2.260028in}{2.608189in}}%
\pgfpathlineto{\pgfqpoint{2.248618in}{2.604774in}}%
\pgfpathlineto{\pgfqpoint{2.237203in}{2.601368in}}%
\pgfpathlineto{\pgfqpoint{2.225783in}{2.597978in}}%
\pgfpathlineto{\pgfqpoint{2.214356in}{2.594610in}}%
\pgfpathlineto{\pgfqpoint{2.220236in}{2.586441in}}%
\pgfpathlineto{\pgfqpoint{2.226119in}{2.578259in}}%
\pgfpathlineto{\pgfqpoint{2.232007in}{2.570065in}}%
\pgfpathlineto{\pgfqpoint{2.237899in}{2.561859in}}%
\pgfpathclose%
\pgfusepath{stroke,fill}%
\end{pgfscope}%
\begin{pgfscope}%
\pgfpathrectangle{\pgfqpoint{0.887500in}{0.275000in}}{\pgfqpoint{4.225000in}{4.225000in}}%
\pgfusepath{clip}%
\pgfsetbuttcap%
\pgfsetroundjoin%
\definecolor{currentfill}{rgb}{0.657642,0.860219,0.203082}%
\pgfsetfillcolor{currentfill}%
\pgfsetfillopacity{0.700000}%
\pgfsetlinewidth{0.501875pt}%
\definecolor{currentstroke}{rgb}{1.000000,1.000000,1.000000}%
\pgfsetstrokecolor{currentstroke}%
\pgfsetstrokeopacity{0.500000}%
\pgfsetdash{}{0pt}%
\pgfpathmoveto{\pgfqpoint{3.522354in}{3.310706in}}%
\pgfpathlineto{\pgfqpoint{3.533488in}{3.314471in}}%
\pgfpathlineto{\pgfqpoint{3.544616in}{3.318228in}}%
\pgfpathlineto{\pgfqpoint{3.555739in}{3.321937in}}%
\pgfpathlineto{\pgfqpoint{3.566857in}{3.325575in}}%
\pgfpathlineto{\pgfqpoint{3.577968in}{3.329145in}}%
\pgfpathlineto{\pgfqpoint{3.571676in}{3.340062in}}%
\pgfpathlineto{\pgfqpoint{3.565388in}{3.350985in}}%
\pgfpathlineto{\pgfqpoint{3.559103in}{3.361944in}}%
\pgfpathlineto{\pgfqpoint{3.552822in}{3.372967in}}%
\pgfpathlineto{\pgfqpoint{3.546546in}{3.384082in}}%
\pgfpathlineto{\pgfqpoint{3.535443in}{3.380653in}}%
\pgfpathlineto{\pgfqpoint{3.524335in}{3.377183in}}%
\pgfpathlineto{\pgfqpoint{3.513221in}{3.373666in}}%
\pgfpathlineto{\pgfqpoint{3.502101in}{3.370111in}}%
\pgfpathlineto{\pgfqpoint{3.490975in}{3.366530in}}%
\pgfpathlineto{\pgfqpoint{3.497244in}{3.355295in}}%
\pgfpathlineto{\pgfqpoint{3.503516in}{3.344105in}}%
\pgfpathlineto{\pgfqpoint{3.509792in}{3.332951in}}%
\pgfpathlineto{\pgfqpoint{3.516071in}{3.321821in}}%
\pgfpathclose%
\pgfusepath{stroke,fill}%
\end{pgfscope}%
\begin{pgfscope}%
\pgfpathrectangle{\pgfqpoint{0.887500in}{0.275000in}}{\pgfqpoint{4.225000in}{4.225000in}}%
\pgfusepath{clip}%
\pgfsetbuttcap%
\pgfsetroundjoin%
\definecolor{currentfill}{rgb}{0.395174,0.797475,0.367757}%
\pgfsetfillcolor{currentfill}%
\pgfsetfillopacity{0.700000}%
\pgfsetlinewidth{0.501875pt}%
\definecolor{currentstroke}{rgb}{1.000000,1.000000,1.000000}%
\pgfsetstrokecolor{currentstroke}%
\pgfsetstrokeopacity{0.500000}%
\pgfsetdash{}{0pt}%
\pgfpathmoveto{\pgfqpoint{3.957546in}{3.099920in}}%
\pgfpathlineto{\pgfqpoint{3.968565in}{3.103022in}}%
\pgfpathlineto{\pgfqpoint{3.979576in}{3.106076in}}%
\pgfpathlineto{\pgfqpoint{3.990582in}{3.109098in}}%
\pgfpathlineto{\pgfqpoint{4.001582in}{3.112103in}}%
\pgfpathlineto{\pgfqpoint{4.012576in}{3.115107in}}%
\pgfpathlineto{\pgfqpoint{4.006213in}{3.127739in}}%
\pgfpathlineto{\pgfqpoint{3.999850in}{3.140308in}}%
\pgfpathlineto{\pgfqpoint{3.993489in}{3.152809in}}%
\pgfpathlineto{\pgfqpoint{3.987129in}{3.165238in}}%
\pgfpathlineto{\pgfqpoint{3.980771in}{3.177592in}}%
\pgfpathlineto{\pgfqpoint{3.969778in}{3.174517in}}%
\pgfpathlineto{\pgfqpoint{3.958780in}{3.171410in}}%
\pgfpathlineto{\pgfqpoint{3.947776in}{3.168262in}}%
\pgfpathlineto{\pgfqpoint{3.936765in}{3.165067in}}%
\pgfpathlineto{\pgfqpoint{3.925747in}{3.161818in}}%
\pgfpathlineto{\pgfqpoint{3.932105in}{3.149627in}}%
\pgfpathlineto{\pgfqpoint{3.938465in}{3.137355in}}%
\pgfpathlineto{\pgfqpoint{3.944825in}{3.124989in}}%
\pgfpathlineto{\pgfqpoint{3.951185in}{3.112515in}}%
\pgfpathclose%
\pgfusepath{stroke,fill}%
\end{pgfscope}%
\begin{pgfscope}%
\pgfpathrectangle{\pgfqpoint{0.887500in}{0.275000in}}{\pgfqpoint{4.225000in}{4.225000in}}%
\pgfusepath{clip}%
\pgfsetbuttcap%
\pgfsetroundjoin%
\definecolor{currentfill}{rgb}{0.449368,0.813768,0.335384}%
\pgfsetfillcolor{currentfill}%
\pgfsetfillopacity{0.700000}%
\pgfsetlinewidth{0.501875pt}%
\definecolor{currentstroke}{rgb}{1.000000,1.000000,1.000000}%
\pgfsetstrokecolor{currentstroke}%
\pgfsetstrokeopacity{0.500000}%
\pgfsetdash{}{0pt}%
\pgfpathmoveto{\pgfqpoint{3.870563in}{3.144683in}}%
\pgfpathlineto{\pgfqpoint{3.881612in}{3.148203in}}%
\pgfpathlineto{\pgfqpoint{3.892655in}{3.151691in}}%
\pgfpathlineto{\pgfqpoint{3.903693in}{3.155132in}}%
\pgfpathlineto{\pgfqpoint{3.914723in}{3.158509in}}%
\pgfpathlineto{\pgfqpoint{3.925747in}{3.161818in}}%
\pgfpathlineto{\pgfqpoint{3.919391in}{3.173942in}}%
\pgfpathlineto{\pgfqpoint{3.913036in}{3.186010in}}%
\pgfpathlineto{\pgfqpoint{3.906684in}{3.198038in}}%
\pgfpathlineto{\pgfqpoint{3.900334in}{3.210039in}}%
\pgfpathlineto{\pgfqpoint{3.893987in}{3.222025in}}%
\pgfpathlineto{\pgfqpoint{3.882966in}{3.218656in}}%
\pgfpathlineto{\pgfqpoint{3.871939in}{3.215226in}}%
\pgfpathlineto{\pgfqpoint{3.860906in}{3.211737in}}%
\pgfpathlineto{\pgfqpoint{3.849867in}{3.208211in}}%
\pgfpathlineto{\pgfqpoint{3.838822in}{3.204664in}}%
\pgfpathlineto{\pgfqpoint{3.845163in}{3.192613in}}%
\pgfpathlineto{\pgfqpoint{3.851508in}{3.180616in}}%
\pgfpathlineto{\pgfqpoint{3.857857in}{3.168647in}}%
\pgfpathlineto{\pgfqpoint{3.864208in}{3.156679in}}%
\pgfpathclose%
\pgfusepath{stroke,fill}%
\end{pgfscope}%
\begin{pgfscope}%
\pgfpathrectangle{\pgfqpoint{0.887500in}{0.275000in}}{\pgfqpoint{4.225000in}{4.225000in}}%
\pgfusepath{clip}%
\pgfsetbuttcap%
\pgfsetroundjoin%
\definecolor{currentfill}{rgb}{0.121148,0.592739,0.544641}%
\pgfsetfillcolor{currentfill}%
\pgfsetfillopacity{0.700000}%
\pgfsetlinewidth{0.501875pt}%
\definecolor{currentstroke}{rgb}{1.000000,1.000000,1.000000}%
\pgfsetstrokecolor{currentstroke}%
\pgfsetstrokeopacity{0.500000}%
\pgfsetdash{}{0pt}%
\pgfpathmoveto{\pgfqpoint{1.695830in}{2.639729in}}%
\pgfpathlineto{\pgfqpoint{1.707407in}{2.642988in}}%
\pgfpathlineto{\pgfqpoint{1.718978in}{2.646244in}}%
\pgfpathlineto{\pgfqpoint{1.730543in}{2.649497in}}%
\pgfpathlineto{\pgfqpoint{1.742103in}{2.652746in}}%
\pgfpathlineto{\pgfqpoint{1.753657in}{2.655994in}}%
\pgfpathlineto{\pgfqpoint{1.747941in}{2.663810in}}%
\pgfpathlineto{\pgfqpoint{1.742230in}{2.671607in}}%
\pgfpathlineto{\pgfqpoint{1.736523in}{2.679386in}}%
\pgfpathlineto{\pgfqpoint{1.730820in}{2.687148in}}%
\pgfpathlineto{\pgfqpoint{1.725122in}{2.694892in}}%
\pgfpathlineto{\pgfqpoint{1.713581in}{2.691632in}}%
\pgfpathlineto{\pgfqpoint{1.702035in}{2.688371in}}%
\pgfpathlineto{\pgfqpoint{1.690483in}{2.685108in}}%
\pgfpathlineto{\pgfqpoint{1.678926in}{2.681842in}}%
\pgfpathlineto{\pgfqpoint{1.667363in}{2.678574in}}%
\pgfpathlineto{\pgfqpoint{1.673048in}{2.670842in}}%
\pgfpathlineto{\pgfqpoint{1.678737in}{2.663092in}}%
\pgfpathlineto{\pgfqpoint{1.684430in}{2.655324in}}%
\pgfpathlineto{\pgfqpoint{1.690128in}{2.647536in}}%
\pgfpathclose%
\pgfusepath{stroke,fill}%
\end{pgfscope}%
\begin{pgfscope}%
\pgfpathrectangle{\pgfqpoint{0.887500in}{0.275000in}}{\pgfqpoint{4.225000in}{4.225000in}}%
\pgfusepath{clip}%
\pgfsetbuttcap%
\pgfsetroundjoin%
\definecolor{currentfill}{rgb}{0.804182,0.882046,0.114965}%
\pgfsetfillcolor{currentfill}%
\pgfsetfillopacity{0.700000}%
\pgfsetlinewidth{0.501875pt}%
\definecolor{currentstroke}{rgb}{1.000000,1.000000,1.000000}%
\pgfsetstrokecolor{currentstroke}%
\pgfsetstrokeopacity{0.500000}%
\pgfsetdash{}{0pt}%
\pgfpathmoveto{\pgfqpoint{3.117989in}{3.430702in}}%
\pgfpathlineto{\pgfqpoint{3.129217in}{3.432490in}}%
\pgfpathlineto{\pgfqpoint{3.140438in}{3.434422in}}%
\pgfpathlineto{\pgfqpoint{3.151654in}{3.436517in}}%
\pgfpathlineto{\pgfqpoint{3.162864in}{3.438792in}}%
\pgfpathlineto{\pgfqpoint{3.174069in}{3.441261in}}%
\pgfpathlineto{\pgfqpoint{3.167871in}{3.449018in}}%
\pgfpathlineto{\pgfqpoint{3.161677in}{3.456462in}}%
\pgfpathlineto{\pgfqpoint{3.155486in}{3.463585in}}%
\pgfpathlineto{\pgfqpoint{3.149298in}{3.470379in}}%
\pgfpathlineto{\pgfqpoint{3.143115in}{3.476838in}}%
\pgfpathlineto{\pgfqpoint{3.131921in}{3.474054in}}%
\pgfpathlineto{\pgfqpoint{3.120722in}{3.471453in}}%
\pgfpathlineto{\pgfqpoint{3.109517in}{3.469051in}}%
\pgfpathlineto{\pgfqpoint{3.098307in}{3.466863in}}%
\pgfpathlineto{\pgfqpoint{3.087091in}{3.464903in}}%
\pgfpathlineto{\pgfqpoint{3.093262in}{3.459005in}}%
\pgfpathlineto{\pgfqpoint{3.099438in}{3.452605in}}%
\pgfpathlineto{\pgfqpoint{3.105618in}{3.445734in}}%
\pgfpathlineto{\pgfqpoint{3.111802in}{3.438423in}}%
\pgfpathclose%
\pgfusepath{stroke,fill}%
\end{pgfscope}%
\begin{pgfscope}%
\pgfpathrectangle{\pgfqpoint{0.887500in}{0.275000in}}{\pgfqpoint{4.225000in}{4.225000in}}%
\pgfusepath{clip}%
\pgfsetbuttcap%
\pgfsetroundjoin%
\definecolor{currentfill}{rgb}{0.506271,0.828786,0.300362}%
\pgfsetfillcolor{currentfill}%
\pgfsetfillopacity{0.700000}%
\pgfsetlinewidth{0.501875pt}%
\definecolor{currentstroke}{rgb}{1.000000,1.000000,1.000000}%
\pgfsetstrokecolor{currentstroke}%
\pgfsetstrokeopacity{0.500000}%
\pgfsetdash{}{0pt}%
\pgfpathmoveto{\pgfqpoint{3.783528in}{3.187283in}}%
\pgfpathlineto{\pgfqpoint{3.794595in}{3.190661in}}%
\pgfpathlineto{\pgfqpoint{3.805658in}{3.194096in}}%
\pgfpathlineto{\pgfqpoint{3.816717in}{3.197588in}}%
\pgfpathlineto{\pgfqpoint{3.827772in}{3.201116in}}%
\pgfpathlineto{\pgfqpoint{3.838822in}{3.204664in}}%
\pgfpathlineto{\pgfqpoint{3.832485in}{3.216778in}}%
\pgfpathlineto{\pgfqpoint{3.826152in}{3.228931in}}%
\pgfpathlineto{\pgfqpoint{3.819822in}{3.241094in}}%
\pgfpathlineto{\pgfqpoint{3.813495in}{3.253237in}}%
\pgfpathlineto{\pgfqpoint{3.807170in}{3.265333in}}%
\pgfpathlineto{\pgfqpoint{3.796127in}{3.261923in}}%
\pgfpathlineto{\pgfqpoint{3.785080in}{3.258503in}}%
\pgfpathlineto{\pgfqpoint{3.774027in}{3.255094in}}%
\pgfpathlineto{\pgfqpoint{3.762969in}{3.251713in}}%
\pgfpathlineto{\pgfqpoint{3.751907in}{3.248366in}}%
\pgfpathlineto{\pgfqpoint{3.758226in}{3.236166in}}%
\pgfpathlineto{\pgfqpoint{3.764546in}{3.223927in}}%
\pgfpathlineto{\pgfqpoint{3.770870in}{3.211680in}}%
\pgfpathlineto{\pgfqpoint{3.777197in}{3.199455in}}%
\pgfpathclose%
\pgfusepath{stroke,fill}%
\end{pgfscope}%
\begin{pgfscope}%
\pgfpathrectangle{\pgfqpoint{0.887500in}{0.275000in}}{\pgfqpoint{4.225000in}{4.225000in}}%
\pgfusepath{clip}%
\pgfsetbuttcap%
\pgfsetroundjoin%
\definecolor{currentfill}{rgb}{0.616293,0.852709,0.230052}%
\pgfsetfillcolor{currentfill}%
\pgfsetfillopacity{0.700000}%
\pgfsetlinewidth{0.501875pt}%
\definecolor{currentstroke}{rgb}{1.000000,1.000000,1.000000}%
\pgfsetstrokecolor{currentstroke}%
\pgfsetstrokeopacity{0.500000}%
\pgfsetdash{}{0pt}%
\pgfpathmoveto{\pgfqpoint{3.609465in}{3.273675in}}%
\pgfpathlineto{\pgfqpoint{3.620577in}{3.277230in}}%
\pgfpathlineto{\pgfqpoint{3.631683in}{3.280731in}}%
\pgfpathlineto{\pgfqpoint{3.642784in}{3.284189in}}%
\pgfpathlineto{\pgfqpoint{3.653879in}{3.287611in}}%
\pgfpathlineto{\pgfqpoint{3.664968in}{3.291008in}}%
\pgfpathlineto{\pgfqpoint{3.658659in}{3.302348in}}%
\pgfpathlineto{\pgfqpoint{3.652351in}{3.313513in}}%
\pgfpathlineto{\pgfqpoint{3.646045in}{3.324538in}}%
\pgfpathlineto{\pgfqpoint{3.639740in}{3.335459in}}%
\pgfpathlineto{\pgfqpoint{3.633437in}{3.346312in}}%
\pgfpathlineto{\pgfqpoint{3.622354in}{3.342939in}}%
\pgfpathlineto{\pgfqpoint{3.611266in}{3.339545in}}%
\pgfpathlineto{\pgfqpoint{3.600173in}{3.336121in}}%
\pgfpathlineto{\pgfqpoint{3.589073in}{3.332658in}}%
\pgfpathlineto{\pgfqpoint{3.577968in}{3.329145in}}%
\pgfpathlineto{\pgfqpoint{3.584263in}{3.318207in}}%
\pgfpathlineto{\pgfqpoint{3.590560in}{3.307220in}}%
\pgfpathlineto{\pgfqpoint{3.596860in}{3.296154in}}%
\pgfpathlineto{\pgfqpoint{3.603162in}{3.284982in}}%
\pgfpathclose%
\pgfusepath{stroke,fill}%
\end{pgfscope}%
\begin{pgfscope}%
\pgfpathrectangle{\pgfqpoint{0.887500in}{0.275000in}}{\pgfqpoint{4.225000in}{4.225000in}}%
\pgfusepath{clip}%
\pgfsetbuttcap%
\pgfsetroundjoin%
\definecolor{currentfill}{rgb}{0.565498,0.842430,0.262877}%
\pgfsetfillcolor{currentfill}%
\pgfsetfillopacity{0.700000}%
\pgfsetlinewidth{0.501875pt}%
\definecolor{currentstroke}{rgb}{1.000000,1.000000,1.000000}%
\pgfsetstrokecolor{currentstroke}%
\pgfsetstrokeopacity{0.500000}%
\pgfsetdash{}{0pt}%
\pgfpathmoveto{\pgfqpoint{3.696521in}{3.231855in}}%
\pgfpathlineto{\pgfqpoint{3.707609in}{3.235154in}}%
\pgfpathlineto{\pgfqpoint{3.718691in}{3.238447in}}%
\pgfpathlineto{\pgfqpoint{3.729768in}{3.241741in}}%
\pgfpathlineto{\pgfqpoint{3.740840in}{3.245045in}}%
\pgfpathlineto{\pgfqpoint{3.751907in}{3.248366in}}%
\pgfpathlineto{\pgfqpoint{3.745590in}{3.260498in}}%
\pgfpathlineto{\pgfqpoint{3.739275in}{3.272530in}}%
\pgfpathlineto{\pgfqpoint{3.732960in}{3.284433in}}%
\pgfpathlineto{\pgfqpoint{3.726646in}{3.296176in}}%
\pgfpathlineto{\pgfqpoint{3.720332in}{3.307728in}}%
\pgfpathlineto{\pgfqpoint{3.709270in}{3.304414in}}%
\pgfpathlineto{\pgfqpoint{3.698203in}{3.301086in}}%
\pgfpathlineto{\pgfqpoint{3.687130in}{3.297743in}}%
\pgfpathlineto{\pgfqpoint{3.676052in}{3.294384in}}%
\pgfpathlineto{\pgfqpoint{3.664968in}{3.291008in}}%
\pgfpathlineto{\pgfqpoint{3.671277in}{3.279474in}}%
\pgfpathlineto{\pgfqpoint{3.677586in}{3.267767in}}%
\pgfpathlineto{\pgfqpoint{3.683897in}{3.255911in}}%
\pgfpathlineto{\pgfqpoint{3.690208in}{3.243932in}}%
\pgfpathclose%
\pgfusepath{stroke,fill}%
\end{pgfscope}%
\begin{pgfscope}%
\pgfpathrectangle{\pgfqpoint{0.887500in}{0.275000in}}{\pgfqpoint{4.225000in}{4.225000in}}%
\pgfusepath{clip}%
\pgfsetbuttcap%
\pgfsetroundjoin%
\definecolor{currentfill}{rgb}{0.141935,0.526453,0.555991}%
\pgfsetfillcolor{currentfill}%
\pgfsetfillopacity{0.700000}%
\pgfsetlinewidth{0.501875pt}%
\definecolor{currentstroke}{rgb}{1.000000,1.000000,1.000000}%
\pgfsetstrokecolor{currentstroke}%
\pgfsetstrokeopacity{0.500000}%
\pgfsetdash{}{0pt}%
\pgfpathmoveto{\pgfqpoint{2.561157in}{2.496130in}}%
\pgfpathlineto{\pgfqpoint{2.572509in}{2.500389in}}%
\pgfpathlineto{\pgfqpoint{2.583851in}{2.505145in}}%
\pgfpathlineto{\pgfqpoint{2.595180in}{2.510466in}}%
\pgfpathlineto{\pgfqpoint{2.606499in}{2.516420in}}%
\pgfpathlineto{\pgfqpoint{2.617805in}{2.523073in}}%
\pgfpathlineto{\pgfqpoint{2.611786in}{2.531741in}}%
\pgfpathlineto{\pgfqpoint{2.605771in}{2.540320in}}%
\pgfpathlineto{\pgfqpoint{2.599761in}{2.548799in}}%
\pgfpathlineto{\pgfqpoint{2.593756in}{2.557165in}}%
\pgfpathlineto{\pgfqpoint{2.587757in}{2.565406in}}%
\pgfpathlineto{\pgfqpoint{2.576467in}{2.558483in}}%
\pgfpathlineto{\pgfqpoint{2.565164in}{2.552395in}}%
\pgfpathlineto{\pgfqpoint{2.553847in}{2.547034in}}%
\pgfpathlineto{\pgfqpoint{2.542519in}{2.542294in}}%
\pgfpathlineto{\pgfqpoint{2.531178in}{2.538066in}}%
\pgfpathlineto{\pgfqpoint{2.537164in}{2.529826in}}%
\pgfpathlineto{\pgfqpoint{2.543155in}{2.521500in}}%
\pgfpathlineto{\pgfqpoint{2.549151in}{2.513101in}}%
\pgfpathlineto{\pgfqpoint{2.555152in}{2.504640in}}%
\pgfpathclose%
\pgfusepath{stroke,fill}%
\end{pgfscope}%
\begin{pgfscope}%
\pgfpathrectangle{\pgfqpoint{0.887500in}{0.275000in}}{\pgfqpoint{4.225000in}{4.225000in}}%
\pgfusepath{clip}%
\pgfsetbuttcap%
\pgfsetroundjoin%
\definecolor{currentfill}{rgb}{0.127568,0.566949,0.550556}%
\pgfsetfillcolor{currentfill}%
\pgfsetfillopacity{0.700000}%
\pgfsetlinewidth{0.501875pt}%
\definecolor{currentstroke}{rgb}{1.000000,1.000000,1.000000}%
\pgfsetstrokecolor{currentstroke}%
\pgfsetstrokeopacity{0.500000}%
\pgfsetdash{}{0pt}%
\pgfpathmoveto{\pgfqpoint{2.013107in}{2.585657in}}%
\pgfpathlineto{\pgfqpoint{2.024607in}{2.588987in}}%
\pgfpathlineto{\pgfqpoint{2.036101in}{2.592315in}}%
\pgfpathlineto{\pgfqpoint{2.047589in}{2.595639in}}%
\pgfpathlineto{\pgfqpoint{2.059072in}{2.598957in}}%
\pgfpathlineto{\pgfqpoint{2.070549in}{2.602266in}}%
\pgfpathlineto{\pgfqpoint{2.064719in}{2.610338in}}%
\pgfpathlineto{\pgfqpoint{2.058893in}{2.618398in}}%
\pgfpathlineto{\pgfqpoint{2.053070in}{2.626446in}}%
\pgfpathlineto{\pgfqpoint{2.047252in}{2.634482in}}%
\pgfpathlineto{\pgfqpoint{2.041438in}{2.642506in}}%
\pgfpathlineto{\pgfqpoint{2.029973in}{2.639189in}}%
\pgfpathlineto{\pgfqpoint{2.018503in}{2.635863in}}%
\pgfpathlineto{\pgfqpoint{2.007028in}{2.632530in}}%
\pgfpathlineto{\pgfqpoint{1.995546in}{2.629196in}}%
\pgfpathlineto{\pgfqpoint{1.984060in}{2.625861in}}%
\pgfpathlineto{\pgfqpoint{1.989861in}{2.617843in}}%
\pgfpathlineto{\pgfqpoint{1.995667in}{2.609813in}}%
\pgfpathlineto{\pgfqpoint{2.001476in}{2.601772in}}%
\pgfpathlineto{\pgfqpoint{2.007290in}{2.593720in}}%
\pgfpathclose%
\pgfusepath{stroke,fill}%
\end{pgfscope}%
\begin{pgfscope}%
\pgfpathrectangle{\pgfqpoint{0.887500in}{0.275000in}}{\pgfqpoint{4.225000in}{4.225000in}}%
\pgfusepath{clip}%
\pgfsetbuttcap%
\pgfsetroundjoin%
\definecolor{currentfill}{rgb}{0.119512,0.607464,0.540218}%
\pgfsetfillcolor{currentfill}%
\pgfsetfillopacity{0.700000}%
\pgfsetlinewidth{0.501875pt}%
\definecolor{currentstroke}{rgb}{1.000000,1.000000,1.000000}%
\pgfsetstrokecolor{currentstroke}%
\pgfsetstrokeopacity{0.500000}%
\pgfsetdash{}{0pt}%
\pgfpathmoveto{\pgfqpoint{1.465092in}{2.667567in}}%
\pgfpathlineto{\pgfqpoint{1.476725in}{2.670884in}}%
\pgfpathlineto{\pgfqpoint{1.488353in}{2.674196in}}%
\pgfpathlineto{\pgfqpoint{1.499975in}{2.677502in}}%
\pgfpathlineto{\pgfqpoint{1.511592in}{2.680803in}}%
\pgfpathlineto{\pgfqpoint{1.523203in}{2.684100in}}%
\pgfpathlineto{\pgfqpoint{1.517571in}{2.691712in}}%
\pgfpathlineto{\pgfqpoint{1.511945in}{2.699308in}}%
\pgfpathlineto{\pgfqpoint{1.506322in}{2.706887in}}%
\pgfpathlineto{\pgfqpoint{1.500704in}{2.714450in}}%
\pgfpathlineto{\pgfqpoint{1.495090in}{2.721995in}}%
\pgfpathlineto{\pgfqpoint{1.483493in}{2.718688in}}%
\pgfpathlineto{\pgfqpoint{1.471891in}{2.715377in}}%
\pgfpathlineto{\pgfqpoint{1.460283in}{2.712063in}}%
\pgfpathlineto{\pgfqpoint{1.448670in}{2.708744in}}%
\pgfpathlineto{\pgfqpoint{1.437051in}{2.705420in}}%
\pgfpathlineto{\pgfqpoint{1.442650in}{2.697884in}}%
\pgfpathlineto{\pgfqpoint{1.448254in}{2.690331in}}%
\pgfpathlineto{\pgfqpoint{1.453862in}{2.682760in}}%
\pgfpathlineto{\pgfqpoint{1.459475in}{2.675172in}}%
\pgfpathclose%
\pgfusepath{stroke,fill}%
\end{pgfscope}%
\begin{pgfscope}%
\pgfpathrectangle{\pgfqpoint{0.887500in}{0.275000in}}{\pgfqpoint{4.225000in}{4.225000in}}%
\pgfusepath{clip}%
\pgfsetbuttcap%
\pgfsetroundjoin%
\definecolor{currentfill}{rgb}{0.793760,0.880678,0.120005}%
\pgfsetfillcolor{currentfill}%
\pgfsetfillopacity{0.700000}%
\pgfsetlinewidth{0.501875pt}%
\definecolor{currentstroke}{rgb}{1.000000,1.000000,1.000000}%
\pgfsetstrokecolor{currentstroke}%
\pgfsetstrokeopacity{0.500000}%
\pgfsetdash{}{0pt}%
\pgfpathmoveto{\pgfqpoint{2.974687in}{3.399780in}}%
\pgfpathlineto{\pgfqpoint{2.985927in}{3.415720in}}%
\pgfpathlineto{\pgfqpoint{2.997174in}{3.428157in}}%
\pgfpathlineto{\pgfqpoint{3.008424in}{3.437755in}}%
\pgfpathlineto{\pgfqpoint{3.019674in}{3.445175in}}%
\pgfpathlineto{\pgfqpoint{3.030922in}{3.450907in}}%
\pgfpathlineto{\pgfqpoint{3.024768in}{3.456484in}}%
\pgfpathlineto{\pgfqpoint{3.018621in}{3.461594in}}%
\pgfpathlineto{\pgfqpoint{3.012478in}{3.466246in}}%
\pgfpathlineto{\pgfqpoint{3.006341in}{3.470495in}}%
\pgfpathlineto{\pgfqpoint{3.000210in}{3.474397in}}%
\pgfpathlineto{\pgfqpoint{2.988978in}{3.469358in}}%
\pgfpathlineto{\pgfqpoint{2.977746in}{3.462510in}}%
\pgfpathlineto{\pgfqpoint{2.966516in}{3.453235in}}%
\pgfpathlineto{\pgfqpoint{2.955291in}{3.440709in}}%
\pgfpathlineto{\pgfqpoint{2.944078in}{3.424104in}}%
\pgfpathlineto{\pgfqpoint{2.950188in}{3.419769in}}%
\pgfpathlineto{\pgfqpoint{2.956305in}{3.415217in}}%
\pgfpathlineto{\pgfqpoint{2.962426in}{3.410396in}}%
\pgfpathlineto{\pgfqpoint{2.968554in}{3.405255in}}%
\pgfpathclose%
\pgfusepath{stroke,fill}%
\end{pgfscope}%
\begin{pgfscope}%
\pgfpathrectangle{\pgfqpoint{0.887500in}{0.275000in}}{\pgfqpoint{4.225000in}{4.225000in}}%
\pgfusepath{clip}%
\pgfsetbuttcap%
\pgfsetroundjoin%
\definecolor{currentfill}{rgb}{0.136408,0.541173,0.554483}%
\pgfsetfillcolor{currentfill}%
\pgfsetfillopacity{0.700000}%
\pgfsetlinewidth{0.501875pt}%
\definecolor{currentstroke}{rgb}{1.000000,1.000000,1.000000}%
\pgfsetstrokecolor{currentstroke}%
\pgfsetstrokeopacity{0.500000}%
\pgfsetdash{}{0pt}%
\pgfpathmoveto{\pgfqpoint{2.330536in}{2.529167in}}%
\pgfpathlineto{\pgfqpoint{2.341957in}{2.532644in}}%
\pgfpathlineto{\pgfqpoint{2.353372in}{2.536175in}}%
\pgfpathlineto{\pgfqpoint{2.364780in}{2.539754in}}%
\pgfpathlineto{\pgfqpoint{2.376183in}{2.543341in}}%
\pgfpathlineto{\pgfqpoint{2.387581in}{2.546897in}}%
\pgfpathlineto{\pgfqpoint{2.381640in}{2.555216in}}%
\pgfpathlineto{\pgfqpoint{2.375703in}{2.563511in}}%
\pgfpathlineto{\pgfqpoint{2.369770in}{2.571778in}}%
\pgfpathlineto{\pgfqpoint{2.363842in}{2.580014in}}%
\pgfpathlineto{\pgfqpoint{2.357919in}{2.588214in}}%
\pgfpathlineto{\pgfqpoint{2.346533in}{2.584651in}}%
\pgfpathlineto{\pgfqpoint{2.335142in}{2.581056in}}%
\pgfpathlineto{\pgfqpoint{2.323746in}{2.577471in}}%
\pgfpathlineto{\pgfqpoint{2.312344in}{2.573934in}}%
\pgfpathlineto{\pgfqpoint{2.300935in}{2.570454in}}%
\pgfpathlineto{\pgfqpoint{2.306847in}{2.562212in}}%
\pgfpathlineto{\pgfqpoint{2.312763in}{2.553964in}}%
\pgfpathlineto{\pgfqpoint{2.318684in}{2.545708in}}%
\pgfpathlineto{\pgfqpoint{2.324608in}{2.537443in}}%
\pgfpathclose%
\pgfusepath{stroke,fill}%
\end{pgfscope}%
\begin{pgfscope}%
\pgfpathrectangle{\pgfqpoint{0.887500in}{0.275000in}}{\pgfqpoint{4.225000in}{4.225000in}}%
\pgfusepath{clip}%
\pgfsetbuttcap%
\pgfsetroundjoin%
\definecolor{currentfill}{rgb}{0.772852,0.877868,0.131109}%
\pgfsetfillcolor{currentfill}%
\pgfsetfillopacity{0.700000}%
\pgfsetlinewidth{0.501875pt}%
\definecolor{currentstroke}{rgb}{1.000000,1.000000,1.000000}%
\pgfsetstrokecolor{currentstroke}%
\pgfsetstrokeopacity{0.500000}%
\pgfsetdash{}{0pt}%
\pgfpathmoveto{\pgfqpoint{3.205105in}{3.398042in}}%
\pgfpathlineto{\pgfqpoint{3.216315in}{3.400867in}}%
\pgfpathlineto{\pgfqpoint{3.227520in}{3.403912in}}%
\pgfpathlineto{\pgfqpoint{3.238721in}{3.407178in}}%
\pgfpathlineto{\pgfqpoint{3.249917in}{3.410647in}}%
\pgfpathlineto{\pgfqpoint{3.261109in}{3.414291in}}%
\pgfpathlineto{\pgfqpoint{3.254886in}{3.423417in}}%
\pgfpathlineto{\pgfqpoint{3.248666in}{3.432239in}}%
\pgfpathlineto{\pgfqpoint{3.242448in}{3.440757in}}%
\pgfpathlineto{\pgfqpoint{3.236233in}{3.448982in}}%
\pgfpathlineto{\pgfqpoint{3.230021in}{3.456930in}}%
\pgfpathlineto{\pgfqpoint{3.218840in}{3.453367in}}%
\pgfpathlineto{\pgfqpoint{3.207654in}{3.449996in}}%
\pgfpathlineto{\pgfqpoint{3.196464in}{3.446849in}}%
\pgfpathlineto{\pgfqpoint{3.185269in}{3.443942in}}%
\pgfpathlineto{\pgfqpoint{3.174069in}{3.441261in}}%
\pgfpathlineto{\pgfqpoint{3.180270in}{3.433198in}}%
\pgfpathlineto{\pgfqpoint{3.186474in}{3.424836in}}%
\pgfpathlineto{\pgfqpoint{3.192682in}{3.416183in}}%
\pgfpathlineto{\pgfqpoint{3.198892in}{3.407247in}}%
\pgfpathclose%
\pgfusepath{stroke,fill}%
\end{pgfscope}%
\begin{pgfscope}%
\pgfpathrectangle{\pgfqpoint{0.887500in}{0.275000in}}{\pgfqpoint{4.225000in}{4.225000in}}%
\pgfusepath{clip}%
\pgfsetbuttcap%
\pgfsetroundjoin%
\definecolor{currentfill}{rgb}{0.124395,0.578002,0.548287}%
\pgfsetfillcolor{currentfill}%
\pgfsetfillopacity{0.700000}%
\pgfsetlinewidth{0.501875pt}%
\definecolor{currentstroke}{rgb}{1.000000,1.000000,1.000000}%
\pgfsetstrokecolor{currentstroke}%
\pgfsetstrokeopacity{0.500000}%
\pgfsetdash{}{0pt}%
\pgfpathmoveto{\pgfqpoint{2.760815in}{2.565922in}}%
\pgfpathlineto{\pgfqpoint{2.772039in}{2.581127in}}%
\pgfpathlineto{\pgfqpoint{2.783255in}{2.597760in}}%
\pgfpathlineto{\pgfqpoint{2.794464in}{2.615647in}}%
\pgfpathlineto{\pgfqpoint{2.805670in}{2.634613in}}%
\pgfpathlineto{\pgfqpoint{2.816873in}{2.654485in}}%
\pgfpathlineto{\pgfqpoint{2.810808in}{2.659228in}}%
\pgfpathlineto{\pgfqpoint{2.804744in}{2.664343in}}%
\pgfpathlineto{\pgfqpoint{2.798684in}{2.669841in}}%
\pgfpathlineto{\pgfqpoint{2.792625in}{2.675730in}}%
\pgfpathlineto{\pgfqpoint{2.786569in}{2.682015in}}%
\pgfpathlineto{\pgfqpoint{2.775345in}{2.668934in}}%
\pgfpathlineto{\pgfqpoint{2.764113in}{2.656875in}}%
\pgfpathlineto{\pgfqpoint{2.752876in}{2.645549in}}%
\pgfpathlineto{\pgfqpoint{2.741634in}{2.634666in}}%
\pgfpathlineto{\pgfqpoint{2.730390in}{2.623937in}}%
\pgfpathlineto{\pgfqpoint{2.736461in}{2.613359in}}%
\pgfpathlineto{\pgfqpoint{2.742538in}{2.602320in}}%
\pgfpathlineto{\pgfqpoint{2.748623in}{2.590767in}}%
\pgfpathlineto{\pgfqpoint{2.754715in}{2.578648in}}%
\pgfpathclose%
\pgfusepath{stroke,fill}%
\end{pgfscope}%
\begin{pgfscope}%
\pgfpathrectangle{\pgfqpoint{0.887500in}{0.275000in}}{\pgfqpoint{4.225000in}{4.225000in}}%
\pgfusepath{clip}%
\pgfsetbuttcap%
\pgfsetroundjoin%
\definecolor{currentfill}{rgb}{0.133743,0.548535,0.553541}%
\pgfsetfillcolor{currentfill}%
\pgfsetfillopacity{0.700000}%
\pgfsetlinewidth{0.501875pt}%
\definecolor{currentstroke}{rgb}{1.000000,1.000000,1.000000}%
\pgfsetstrokecolor{currentstroke}%
\pgfsetstrokeopacity{0.500000}%
\pgfsetdash{}{0pt}%
\pgfpathmoveto{\pgfqpoint{2.704512in}{2.512396in}}%
\pgfpathlineto{\pgfqpoint{2.715796in}{2.520641in}}%
\pgfpathlineto{\pgfqpoint{2.727069in}{2.529891in}}%
\pgfpathlineto{\pgfqpoint{2.738331in}{2.540373in}}%
\pgfpathlineto{\pgfqpoint{2.749579in}{2.552318in}}%
\pgfpathlineto{\pgfqpoint{2.760815in}{2.565922in}}%
\pgfpathlineto{\pgfqpoint{2.754715in}{2.578648in}}%
\pgfpathlineto{\pgfqpoint{2.748623in}{2.590767in}}%
\pgfpathlineto{\pgfqpoint{2.742538in}{2.602320in}}%
\pgfpathlineto{\pgfqpoint{2.736461in}{2.613359in}}%
\pgfpathlineto{\pgfqpoint{2.730390in}{2.623937in}}%
\pgfpathlineto{\pgfqpoint{2.719145in}{2.613074in}}%
\pgfpathlineto{\pgfqpoint{2.707902in}{2.601925in}}%
\pgfpathlineto{\pgfqpoint{2.696659in}{2.590640in}}%
\pgfpathlineto{\pgfqpoint{2.685415in}{2.579406in}}%
\pgfpathlineto{\pgfqpoint{2.674167in}{2.568413in}}%
\pgfpathlineto{\pgfqpoint{2.680222in}{2.558069in}}%
\pgfpathlineto{\pgfqpoint{2.686284in}{2.547319in}}%
\pgfpathlineto{\pgfqpoint{2.692353in}{2.536140in}}%
\pgfpathlineto{\pgfqpoint{2.698429in}{2.524505in}}%
\pgfpathclose%
\pgfusepath{stroke,fill}%
\end{pgfscope}%
\begin{pgfscope}%
\pgfpathrectangle{\pgfqpoint{0.887500in}{0.275000in}}{\pgfqpoint{4.225000in}{4.225000in}}%
\pgfusepath{clip}%
\pgfsetbuttcap%
\pgfsetroundjoin%
\definecolor{currentfill}{rgb}{0.122606,0.585371,0.546557}%
\pgfsetfillcolor{currentfill}%
\pgfsetfillopacity{0.700000}%
\pgfsetlinewidth{0.501875pt}%
\definecolor{currentstroke}{rgb}{1.000000,1.000000,1.000000}%
\pgfsetstrokecolor{currentstroke}%
\pgfsetstrokeopacity{0.500000}%
\pgfsetdash{}{0pt}%
\pgfpathmoveto{\pgfqpoint{1.782301in}{2.616610in}}%
\pgfpathlineto{\pgfqpoint{1.793863in}{2.619849in}}%
\pgfpathlineto{\pgfqpoint{1.805419in}{2.623087in}}%
\pgfpathlineto{\pgfqpoint{1.816970in}{2.626324in}}%
\pgfpathlineto{\pgfqpoint{1.828514in}{2.629564in}}%
\pgfpathlineto{\pgfqpoint{1.840053in}{2.632806in}}%
\pgfpathlineto{\pgfqpoint{1.834302in}{2.640728in}}%
\pgfpathlineto{\pgfqpoint{1.828556in}{2.648631in}}%
\pgfpathlineto{\pgfqpoint{1.822814in}{2.656515in}}%
\pgfpathlineto{\pgfqpoint{1.817076in}{2.664379in}}%
\pgfpathlineto{\pgfqpoint{1.811343in}{2.672224in}}%
\pgfpathlineto{\pgfqpoint{1.799817in}{2.668976in}}%
\pgfpathlineto{\pgfqpoint{1.788286in}{2.665730in}}%
\pgfpathlineto{\pgfqpoint{1.776748in}{2.662485in}}%
\pgfpathlineto{\pgfqpoint{1.765206in}{2.659240in}}%
\pgfpathlineto{\pgfqpoint{1.753657in}{2.655994in}}%
\pgfpathlineto{\pgfqpoint{1.759377in}{2.648158in}}%
\pgfpathlineto{\pgfqpoint{1.765102in}{2.640302in}}%
\pgfpathlineto{\pgfqpoint{1.770830in}{2.632425in}}%
\pgfpathlineto{\pgfqpoint{1.776564in}{2.624527in}}%
\pgfpathclose%
\pgfusepath{stroke,fill}%
\end{pgfscope}%
\begin{pgfscope}%
\pgfpathrectangle{\pgfqpoint{0.887500in}{0.275000in}}{\pgfqpoint{4.225000in}{4.225000in}}%
\pgfusepath{clip}%
\pgfsetbuttcap%
\pgfsetroundjoin%
\definecolor{currentfill}{rgb}{0.202219,0.715272,0.476084}%
\pgfsetfillcolor{currentfill}%
\pgfsetfillopacity{0.700000}%
\pgfsetlinewidth{0.501875pt}%
\definecolor{currentstroke}{rgb}{1.000000,1.000000,1.000000}%
\pgfsetstrokecolor{currentstroke}%
\pgfsetstrokeopacity{0.500000}%
\pgfsetdash{}{0pt}%
\pgfpathmoveto{\pgfqpoint{4.249943in}{2.882211in}}%
\pgfpathlineto{\pgfqpoint{4.260903in}{2.885716in}}%
\pgfpathlineto{\pgfqpoint{4.271860in}{2.889227in}}%
\pgfpathlineto{\pgfqpoint{4.282811in}{2.892746in}}%
\pgfpathlineto{\pgfqpoint{4.293757in}{2.896275in}}%
\pgfpathlineto{\pgfqpoint{4.287369in}{2.910297in}}%
\pgfpathlineto{\pgfqpoint{4.280983in}{2.924283in}}%
\pgfpathlineto{\pgfqpoint{4.274597in}{2.938220in}}%
\pgfpathlineto{\pgfqpoint{4.268213in}{2.952094in}}%
\pgfpathlineto{\pgfqpoint{4.261828in}{2.965894in}}%
\pgfpathlineto{\pgfqpoint{4.250886in}{2.962438in}}%
\pgfpathlineto{\pgfqpoint{4.239939in}{2.958979in}}%
\pgfpathlineto{\pgfqpoint{4.228986in}{2.955518in}}%
\pgfpathlineto{\pgfqpoint{4.218028in}{2.952053in}}%
\pgfpathlineto{\pgfqpoint{4.224410in}{2.938204in}}%
\pgfpathlineto{\pgfqpoint{4.230791in}{2.924280in}}%
\pgfpathlineto{\pgfqpoint{4.237174in}{2.910296in}}%
\pgfpathlineto{\pgfqpoint{4.243557in}{2.896268in}}%
\pgfpathclose%
\pgfusepath{stroke,fill}%
\end{pgfscope}%
\begin{pgfscope}%
\pgfpathrectangle{\pgfqpoint{0.887500in}{0.275000in}}{\pgfqpoint{4.225000in}{4.225000in}}%
\pgfusepath{clip}%
\pgfsetbuttcap%
\pgfsetroundjoin%
\definecolor{currentfill}{rgb}{0.129933,0.559582,0.551864}%
\pgfsetfillcolor{currentfill}%
\pgfsetfillopacity{0.700000}%
\pgfsetlinewidth{0.501875pt}%
\definecolor{currentstroke}{rgb}{1.000000,1.000000,1.000000}%
\pgfsetstrokecolor{currentstroke}%
\pgfsetstrokeopacity{0.500000}%
\pgfsetdash{}{0pt}%
\pgfpathmoveto{\pgfqpoint{2.099762in}{2.561719in}}%
\pgfpathlineto{\pgfqpoint{2.111247in}{2.565011in}}%
\pgfpathlineto{\pgfqpoint{2.122726in}{2.568295in}}%
\pgfpathlineto{\pgfqpoint{2.134200in}{2.571572in}}%
\pgfpathlineto{\pgfqpoint{2.145668in}{2.574843in}}%
\pgfpathlineto{\pgfqpoint{2.157131in}{2.578111in}}%
\pgfpathlineto{\pgfqpoint{2.151268in}{2.586245in}}%
\pgfpathlineto{\pgfqpoint{2.145409in}{2.594364in}}%
\pgfpathlineto{\pgfqpoint{2.139554in}{2.602470in}}%
\pgfpathlineto{\pgfqpoint{2.133703in}{2.610563in}}%
\pgfpathlineto{\pgfqpoint{2.127856in}{2.618643in}}%
\pgfpathlineto{\pgfqpoint{2.116406in}{2.615387in}}%
\pgfpathlineto{\pgfqpoint{2.104950in}{2.612127in}}%
\pgfpathlineto{\pgfqpoint{2.093488in}{2.608853in}}%
\pgfpathlineto{\pgfqpoint{2.082021in}{2.605566in}}%
\pgfpathlineto{\pgfqpoint{2.070549in}{2.602266in}}%
\pgfpathlineto{\pgfqpoint{2.076384in}{2.594183in}}%
\pgfpathlineto{\pgfqpoint{2.082222in}{2.586086in}}%
\pgfpathlineto{\pgfqpoint{2.088065in}{2.577977in}}%
\pgfpathlineto{\pgfqpoint{2.093911in}{2.569855in}}%
\pgfpathclose%
\pgfusepath{stroke,fill}%
\end{pgfscope}%
\begin{pgfscope}%
\pgfpathrectangle{\pgfqpoint{0.887500in}{0.275000in}}{\pgfqpoint{4.225000in}{4.225000in}}%
\pgfusepath{clip}%
\pgfsetbuttcap%
\pgfsetroundjoin%
\definecolor{currentfill}{rgb}{0.246070,0.738910,0.452024}%
\pgfsetfillcolor{currentfill}%
\pgfsetfillopacity{0.700000}%
\pgfsetlinewidth{0.501875pt}%
\definecolor{currentstroke}{rgb}{1.000000,1.000000,1.000000}%
\pgfsetstrokecolor{currentstroke}%
\pgfsetstrokeopacity{0.500000}%
\pgfsetdash{}{0pt}%
\pgfpathmoveto{\pgfqpoint{4.163159in}{2.934626in}}%
\pgfpathlineto{\pgfqpoint{4.174144in}{2.938128in}}%
\pgfpathlineto{\pgfqpoint{4.185123in}{2.941621in}}%
\pgfpathlineto{\pgfqpoint{4.196097in}{2.945105in}}%
\pgfpathlineto{\pgfqpoint{4.207065in}{2.948582in}}%
\pgfpathlineto{\pgfqpoint{4.218028in}{2.952053in}}%
\pgfpathlineto{\pgfqpoint{4.211647in}{2.965809in}}%
\pgfpathlineto{\pgfqpoint{4.205265in}{2.979458in}}%
\pgfpathlineto{\pgfqpoint{4.198882in}{2.992984in}}%
\pgfpathlineto{\pgfqpoint{4.192498in}{3.006370in}}%
\pgfpathlineto{\pgfqpoint{4.186112in}{3.019608in}}%
\pgfpathlineto{\pgfqpoint{4.175153in}{3.016177in}}%
\pgfpathlineto{\pgfqpoint{4.164188in}{3.012743in}}%
\pgfpathlineto{\pgfqpoint{4.153218in}{3.009306in}}%
\pgfpathlineto{\pgfqpoint{4.142243in}{3.005865in}}%
\pgfpathlineto{\pgfqpoint{4.131262in}{3.002419in}}%
\pgfpathlineto{\pgfqpoint{4.137643in}{2.989114in}}%
\pgfpathlineto{\pgfqpoint{4.144023in}{2.975672in}}%
\pgfpathlineto{\pgfqpoint{4.150402in}{2.962099in}}%
\pgfpathlineto{\pgfqpoint{4.156781in}{2.948413in}}%
\pgfpathclose%
\pgfusepath{stroke,fill}%
\end{pgfscope}%
\begin{pgfscope}%
\pgfpathrectangle{\pgfqpoint{0.887500in}{0.275000in}}{\pgfqpoint{4.225000in}{4.225000in}}%
\pgfusepath{clip}%
\pgfsetbuttcap%
\pgfsetroundjoin%
\definecolor{currentfill}{rgb}{0.120092,0.600104,0.542530}%
\pgfsetfillcolor{currentfill}%
\pgfsetfillopacity{0.700000}%
\pgfsetlinewidth{0.501875pt}%
\definecolor{currentstroke}{rgb}{1.000000,1.000000,1.000000}%
\pgfsetstrokecolor{currentstroke}%
\pgfsetstrokeopacity{0.500000}%
\pgfsetdash{}{0pt}%
\pgfpathmoveto{\pgfqpoint{1.551423in}{2.645794in}}%
\pgfpathlineto{\pgfqpoint{1.563042in}{2.649079in}}%
\pgfpathlineto{\pgfqpoint{1.574656in}{2.652362in}}%
\pgfpathlineto{\pgfqpoint{1.586264in}{2.655642in}}%
\pgfpathlineto{\pgfqpoint{1.597867in}{2.658922in}}%
\pgfpathlineto{\pgfqpoint{1.609464in}{2.662200in}}%
\pgfpathlineto{\pgfqpoint{1.603797in}{2.669905in}}%
\pgfpathlineto{\pgfqpoint{1.598135in}{2.677593in}}%
\pgfpathlineto{\pgfqpoint{1.592477in}{2.685264in}}%
\pgfpathlineto{\pgfqpoint{1.586823in}{2.692919in}}%
\pgfpathlineto{\pgfqpoint{1.581173in}{2.700559in}}%
\pgfpathlineto{\pgfqpoint{1.569590in}{2.697269in}}%
\pgfpathlineto{\pgfqpoint{1.558002in}{2.693978in}}%
\pgfpathlineto{\pgfqpoint{1.546408in}{2.690687in}}%
\pgfpathlineto{\pgfqpoint{1.534808in}{2.687395in}}%
\pgfpathlineto{\pgfqpoint{1.523203in}{2.684100in}}%
\pgfpathlineto{\pgfqpoint{1.528838in}{2.676472in}}%
\pgfpathlineto{\pgfqpoint{1.534478in}{2.668828in}}%
\pgfpathlineto{\pgfqpoint{1.540122in}{2.661167in}}%
\pgfpathlineto{\pgfqpoint{1.545770in}{2.653490in}}%
\pgfpathclose%
\pgfusepath{stroke,fill}%
\end{pgfscope}%
\begin{pgfscope}%
\pgfpathrectangle{\pgfqpoint{0.887500in}{0.275000in}}{\pgfqpoint{4.225000in}{4.225000in}}%
\pgfusepath{clip}%
\pgfsetbuttcap%
\pgfsetroundjoin%
\definecolor{currentfill}{rgb}{0.741388,0.873449,0.149561}%
\pgfsetfillcolor{currentfill}%
\pgfsetfillopacity{0.700000}%
\pgfsetlinewidth{0.501875pt}%
\definecolor{currentstroke}{rgb}{1.000000,1.000000,1.000000}%
\pgfsetstrokecolor{currentstroke}%
\pgfsetstrokeopacity{0.500000}%
\pgfsetdash{}{0pt}%
\pgfpathmoveto{\pgfqpoint{3.292260in}{3.365358in}}%
\pgfpathlineto{\pgfqpoint{3.303456in}{3.369018in}}%
\pgfpathlineto{\pgfqpoint{3.314647in}{3.372768in}}%
\pgfpathlineto{\pgfqpoint{3.325833in}{3.376584in}}%
\pgfpathlineto{\pgfqpoint{3.337015in}{3.380445in}}%
\pgfpathlineto{\pgfqpoint{3.348192in}{3.384330in}}%
\pgfpathlineto{\pgfqpoint{3.341950in}{3.394692in}}%
\pgfpathlineto{\pgfqpoint{3.335710in}{3.404892in}}%
\pgfpathlineto{\pgfqpoint{3.329472in}{3.414891in}}%
\pgfpathlineto{\pgfqpoint{3.323236in}{3.424652in}}%
\pgfpathlineto{\pgfqpoint{3.317003in}{3.434134in}}%
\pgfpathlineto{\pgfqpoint{3.305833in}{3.430043in}}%
\pgfpathlineto{\pgfqpoint{3.294659in}{3.425985in}}%
\pgfpathlineto{\pgfqpoint{3.283480in}{3.421988in}}%
\pgfpathlineto{\pgfqpoint{3.272297in}{3.418081in}}%
\pgfpathlineto{\pgfqpoint{3.261109in}{3.414291in}}%
\pgfpathlineto{\pgfqpoint{3.267334in}{3.404893in}}%
\pgfpathlineto{\pgfqpoint{3.273562in}{3.395262in}}%
\pgfpathlineto{\pgfqpoint{3.279792in}{3.385437in}}%
\pgfpathlineto{\pgfqpoint{3.286024in}{3.375456in}}%
\pgfpathclose%
\pgfusepath{stroke,fill}%
\end{pgfscope}%
\begin{pgfscope}%
\pgfpathrectangle{\pgfqpoint{0.887500in}{0.275000in}}{\pgfqpoint{4.225000in}{4.225000in}}%
\pgfusepath{clip}%
\pgfsetbuttcap%
\pgfsetroundjoin%
\definecolor{currentfill}{rgb}{0.139147,0.533812,0.555298}%
\pgfsetfillcolor{currentfill}%
\pgfsetfillopacity{0.700000}%
\pgfsetlinewidth{0.501875pt}%
\definecolor{currentstroke}{rgb}{1.000000,1.000000,1.000000}%
\pgfsetstrokecolor{currentstroke}%
\pgfsetstrokeopacity{0.500000}%
\pgfsetdash{}{0pt}%
\pgfpathmoveto{\pgfqpoint{2.417346in}{2.505072in}}%
\pgfpathlineto{\pgfqpoint{2.428752in}{2.508529in}}%
\pgfpathlineto{\pgfqpoint{2.440154in}{2.511912in}}%
\pgfpathlineto{\pgfqpoint{2.451551in}{2.515195in}}%
\pgfpathlineto{\pgfqpoint{2.462945in}{2.518351in}}%
\pgfpathlineto{\pgfqpoint{2.474334in}{2.521401in}}%
\pgfpathlineto{\pgfqpoint{2.468363in}{2.529732in}}%
\pgfpathlineto{\pgfqpoint{2.462395in}{2.538033in}}%
\pgfpathlineto{\pgfqpoint{2.456432in}{2.546310in}}%
\pgfpathlineto{\pgfqpoint{2.450472in}{2.554569in}}%
\pgfpathlineto{\pgfqpoint{2.444517in}{2.562818in}}%
\pgfpathlineto{\pgfqpoint{2.433136in}{2.559976in}}%
\pgfpathlineto{\pgfqpoint{2.421751in}{2.556959in}}%
\pgfpathlineto{\pgfqpoint{2.410364in}{2.553747in}}%
\pgfpathlineto{\pgfqpoint{2.398974in}{2.550379in}}%
\pgfpathlineto{\pgfqpoint{2.387581in}{2.546897in}}%
\pgfpathlineto{\pgfqpoint{2.393526in}{2.538557in}}%
\pgfpathlineto{\pgfqpoint{2.399475in}{2.530201in}}%
\pgfpathlineto{\pgfqpoint{2.405428in}{2.521833in}}%
\pgfpathlineto{\pgfqpoint{2.411385in}{2.513456in}}%
\pgfpathclose%
\pgfusepath{stroke,fill}%
\end{pgfscope}%
\begin{pgfscope}%
\pgfpathrectangle{\pgfqpoint{0.887500in}{0.275000in}}{\pgfqpoint{4.225000in}{4.225000in}}%
\pgfusepath{clip}%
\pgfsetbuttcap%
\pgfsetroundjoin%
\definecolor{currentfill}{rgb}{0.288921,0.758394,0.428426}%
\pgfsetfillcolor{currentfill}%
\pgfsetfillopacity{0.700000}%
\pgfsetlinewidth{0.501875pt}%
\definecolor{currentstroke}{rgb}{1.000000,1.000000,1.000000}%
\pgfsetstrokecolor{currentstroke}%
\pgfsetstrokeopacity{0.500000}%
\pgfsetdash{}{0pt}%
\pgfpathmoveto{\pgfqpoint{4.076280in}{2.985104in}}%
\pgfpathlineto{\pgfqpoint{4.087287in}{2.988580in}}%
\pgfpathlineto{\pgfqpoint{4.098289in}{2.992049in}}%
\pgfpathlineto{\pgfqpoint{4.109285in}{2.995511in}}%
\pgfpathlineto{\pgfqpoint{4.120277in}{2.998968in}}%
\pgfpathlineto{\pgfqpoint{4.131262in}{3.002419in}}%
\pgfpathlineto{\pgfqpoint{4.124881in}{3.015595in}}%
\pgfpathlineto{\pgfqpoint{4.118499in}{3.028655in}}%
\pgfpathlineto{\pgfqpoint{4.112117in}{3.041610in}}%
\pgfpathlineto{\pgfqpoint{4.105735in}{3.054473in}}%
\pgfpathlineto{\pgfqpoint{4.099355in}{3.067254in}}%
\pgfpathlineto{\pgfqpoint{4.088377in}{3.063974in}}%
\pgfpathlineto{\pgfqpoint{4.077394in}{3.060712in}}%
\pgfpathlineto{\pgfqpoint{4.066406in}{3.057467in}}%
\pgfpathlineto{\pgfqpoint{4.055413in}{3.054240in}}%
\pgfpathlineto{\pgfqpoint{4.044416in}{3.051031in}}%
\pgfpathlineto{\pgfqpoint{4.050787in}{3.038011in}}%
\pgfpathlineto{\pgfqpoint{4.057160in}{3.024912in}}%
\pgfpathlineto{\pgfqpoint{4.063533in}{3.011730in}}%
\pgfpathlineto{\pgfqpoint{4.069906in}{2.998463in}}%
\pgfpathclose%
\pgfusepath{stroke,fill}%
\end{pgfscope}%
\begin{pgfscope}%
\pgfpathrectangle{\pgfqpoint{0.887500in}{0.275000in}}{\pgfqpoint{4.225000in}{4.225000in}}%
\pgfusepath{clip}%
\pgfsetbuttcap%
\pgfsetroundjoin%
\definecolor{currentfill}{rgb}{0.141935,0.526453,0.555991}%
\pgfsetfillcolor{currentfill}%
\pgfsetfillopacity{0.700000}%
\pgfsetlinewidth{0.501875pt}%
\definecolor{currentstroke}{rgb}{1.000000,1.000000,1.000000}%
\pgfsetstrokecolor{currentstroke}%
\pgfsetstrokeopacity{0.500000}%
\pgfsetdash{}{0pt}%
\pgfpathmoveto{\pgfqpoint{2.647970in}{2.478823in}}%
\pgfpathlineto{\pgfqpoint{2.659291in}{2.484993in}}%
\pgfpathlineto{\pgfqpoint{2.670606in}{2.491398in}}%
\pgfpathlineto{\pgfqpoint{2.681915in}{2.498002in}}%
\pgfpathlineto{\pgfqpoint{2.693218in}{2.504926in}}%
\pgfpathlineto{\pgfqpoint{2.704512in}{2.512396in}}%
\pgfpathlineto{\pgfqpoint{2.698429in}{2.524505in}}%
\pgfpathlineto{\pgfqpoint{2.692353in}{2.536140in}}%
\pgfpathlineto{\pgfqpoint{2.686284in}{2.547319in}}%
\pgfpathlineto{\pgfqpoint{2.680222in}{2.558069in}}%
\pgfpathlineto{\pgfqpoint{2.674167in}{2.568413in}}%
\pgfpathlineto{\pgfqpoint{2.662914in}{2.557848in}}%
\pgfpathlineto{\pgfqpoint{2.651654in}{2.547898in}}%
\pgfpathlineto{\pgfqpoint{2.640383in}{2.538751in}}%
\pgfpathlineto{\pgfqpoint{2.629100in}{2.530494in}}%
\pgfpathlineto{\pgfqpoint{2.617805in}{2.523073in}}%
\pgfpathlineto{\pgfqpoint{2.623830in}{2.514328in}}%
\pgfpathlineto{\pgfqpoint{2.629858in}{2.505519in}}%
\pgfpathlineto{\pgfqpoint{2.635891in}{2.496657in}}%
\pgfpathlineto{\pgfqpoint{2.641929in}{2.487754in}}%
\pgfpathclose%
\pgfusepath{stroke,fill}%
\end{pgfscope}%
\begin{pgfscope}%
\pgfpathrectangle{\pgfqpoint{0.887500in}{0.275000in}}{\pgfqpoint{4.225000in}{4.225000in}}%
\pgfusepath{clip}%
\pgfsetbuttcap%
\pgfsetroundjoin%
\definecolor{currentfill}{rgb}{0.120081,0.622161,0.534946}%
\pgfsetfillcolor{currentfill}%
\pgfsetfillopacity{0.700000}%
\pgfsetlinewidth{0.501875pt}%
\definecolor{currentstroke}{rgb}{1.000000,1.000000,1.000000}%
\pgfsetstrokecolor{currentstroke}%
\pgfsetstrokeopacity{0.500000}%
\pgfsetdash{}{0pt}%
\pgfpathmoveto{\pgfqpoint{2.816873in}{2.654485in}}%
\pgfpathlineto{\pgfqpoint{2.828077in}{2.675088in}}%
\pgfpathlineto{\pgfqpoint{2.839281in}{2.696284in}}%
\pgfpathlineto{\pgfqpoint{2.850486in}{2.718128in}}%
\pgfpathlineto{\pgfqpoint{2.861693in}{2.740731in}}%
\pgfpathlineto{\pgfqpoint{2.872902in}{2.764206in}}%
\pgfpathlineto{\pgfqpoint{2.866832in}{2.763462in}}%
\pgfpathlineto{\pgfqpoint{2.860764in}{2.763661in}}%
\pgfpathlineto{\pgfqpoint{2.854697in}{2.764790in}}%
\pgfpathlineto{\pgfqpoint{2.848632in}{2.766818in}}%
\pgfpathlineto{\pgfqpoint{2.842569in}{2.769717in}}%
\pgfpathlineto{\pgfqpoint{2.831381in}{2.749044in}}%
\pgfpathlineto{\pgfqpoint{2.820189in}{2.729918in}}%
\pgfpathlineto{\pgfqpoint{2.808990in}{2.712364in}}%
\pgfpathlineto{\pgfqpoint{2.797784in}{2.696409in}}%
\pgfpathlineto{\pgfqpoint{2.786569in}{2.682015in}}%
\pgfpathlineto{\pgfqpoint{2.792625in}{2.675730in}}%
\pgfpathlineto{\pgfqpoint{2.798684in}{2.669841in}}%
\pgfpathlineto{\pgfqpoint{2.804744in}{2.664343in}}%
\pgfpathlineto{\pgfqpoint{2.810808in}{2.659228in}}%
\pgfpathclose%
\pgfusepath{stroke,fill}%
\end{pgfscope}%
\begin{pgfscope}%
\pgfpathrectangle{\pgfqpoint{0.887500in}{0.275000in}}{\pgfqpoint{4.225000in}{4.225000in}}%
\pgfusepath{clip}%
\pgfsetbuttcap%
\pgfsetroundjoin%
\definecolor{currentfill}{rgb}{0.125394,0.574318,0.549086}%
\pgfsetfillcolor{currentfill}%
\pgfsetfillopacity{0.700000}%
\pgfsetlinewidth{0.501875pt}%
\definecolor{currentstroke}{rgb}{1.000000,1.000000,1.000000}%
\pgfsetstrokecolor{currentstroke}%
\pgfsetstrokeopacity{0.500000}%
\pgfsetdash{}{0pt}%
\pgfpathmoveto{\pgfqpoint{1.868870in}{2.592941in}}%
\pgfpathlineto{\pgfqpoint{1.880416in}{2.596188in}}%
\pgfpathlineto{\pgfqpoint{1.891956in}{2.599442in}}%
\pgfpathlineto{\pgfqpoint{1.903490in}{2.602705in}}%
\pgfpathlineto{\pgfqpoint{1.915018in}{2.605980in}}%
\pgfpathlineto{\pgfqpoint{1.926539in}{2.609269in}}%
\pgfpathlineto{\pgfqpoint{1.920755in}{2.617273in}}%
\pgfpathlineto{\pgfqpoint{1.914974in}{2.625263in}}%
\pgfpathlineto{\pgfqpoint{1.909198in}{2.633238in}}%
\pgfpathlineto{\pgfqpoint{1.903426in}{2.641197in}}%
\pgfpathlineto{\pgfqpoint{1.897658in}{2.649139in}}%
\pgfpathlineto{\pgfqpoint{1.886149in}{2.645850in}}%
\pgfpathlineto{\pgfqpoint{1.874634in}{2.642574in}}%
\pgfpathlineto{\pgfqpoint{1.863113in}{2.639310in}}%
\pgfpathlineto{\pgfqpoint{1.851586in}{2.636054in}}%
\pgfpathlineto{\pgfqpoint{1.840053in}{2.632806in}}%
\pgfpathlineto{\pgfqpoint{1.845808in}{2.624866in}}%
\pgfpathlineto{\pgfqpoint{1.851567in}{2.616909in}}%
\pgfpathlineto{\pgfqpoint{1.857331in}{2.608935in}}%
\pgfpathlineto{\pgfqpoint{1.863098in}{2.600946in}}%
\pgfpathclose%
\pgfusepath{stroke,fill}%
\end{pgfscope}%
\begin{pgfscope}%
\pgfpathrectangle{\pgfqpoint{0.887500in}{0.275000in}}{\pgfqpoint{4.225000in}{4.225000in}}%
\pgfusepath{clip}%
\pgfsetbuttcap%
\pgfsetroundjoin%
\definecolor{currentfill}{rgb}{0.709898,0.868751,0.169257}%
\pgfsetfillcolor{currentfill}%
\pgfsetfillopacity{0.700000}%
\pgfsetlinewidth{0.501875pt}%
\definecolor{currentstroke}{rgb}{1.000000,1.000000,1.000000}%
\pgfsetstrokecolor{currentstroke}%
\pgfsetstrokeopacity{0.500000}%
\pgfsetdash{}{0pt}%
\pgfpathmoveto{\pgfqpoint{3.379445in}{3.331420in}}%
\pgfpathlineto{\pgfqpoint{3.390621in}{3.334840in}}%
\pgfpathlineto{\pgfqpoint{3.401791in}{3.338270in}}%
\pgfpathlineto{\pgfqpoint{3.412957in}{3.341715in}}%
\pgfpathlineto{\pgfqpoint{3.424117in}{3.345181in}}%
\pgfpathlineto{\pgfqpoint{3.435272in}{3.348673in}}%
\pgfpathlineto{\pgfqpoint{3.429013in}{3.359805in}}%
\pgfpathlineto{\pgfqpoint{3.422756in}{3.370926in}}%
\pgfpathlineto{\pgfqpoint{3.416502in}{3.381994in}}%
\pgfpathlineto{\pgfqpoint{3.410251in}{3.392970in}}%
\pgfpathlineto{\pgfqpoint{3.404001in}{3.403811in}}%
\pgfpathlineto{\pgfqpoint{3.392850in}{3.399932in}}%
\pgfpathlineto{\pgfqpoint{3.381693in}{3.396037in}}%
\pgfpathlineto{\pgfqpoint{3.370531in}{3.392133in}}%
\pgfpathlineto{\pgfqpoint{3.359364in}{3.388228in}}%
\pgfpathlineto{\pgfqpoint{3.348192in}{3.384330in}}%
\pgfpathlineto{\pgfqpoint{3.354436in}{3.373843in}}%
\pgfpathlineto{\pgfqpoint{3.360684in}{3.363271in}}%
\pgfpathlineto{\pgfqpoint{3.366934in}{3.352651in}}%
\pgfpathlineto{\pgfqpoint{3.373188in}{3.342023in}}%
\pgfpathclose%
\pgfusepath{stroke,fill}%
\end{pgfscope}%
\begin{pgfscope}%
\pgfpathrectangle{\pgfqpoint{0.887500in}{0.275000in}}{\pgfqpoint{4.225000in}{4.225000in}}%
\pgfusepath{clip}%
\pgfsetbuttcap%
\pgfsetroundjoin%
\definecolor{currentfill}{rgb}{0.344074,0.780029,0.397381}%
\pgfsetfillcolor{currentfill}%
\pgfsetfillopacity{0.700000}%
\pgfsetlinewidth{0.501875pt}%
\definecolor{currentstroke}{rgb}{1.000000,1.000000,1.000000}%
\pgfsetstrokecolor{currentstroke}%
\pgfsetstrokeopacity{0.500000}%
\pgfsetdash{}{0pt}%
\pgfpathmoveto{\pgfqpoint{3.989349in}{3.034972in}}%
\pgfpathlineto{\pgfqpoint{4.000374in}{3.038226in}}%
\pgfpathlineto{\pgfqpoint{4.011393in}{3.041447in}}%
\pgfpathlineto{\pgfqpoint{4.022406in}{3.044647in}}%
\pgfpathlineto{\pgfqpoint{4.033414in}{3.047838in}}%
\pgfpathlineto{\pgfqpoint{4.044416in}{3.051031in}}%
\pgfpathlineto{\pgfqpoint{4.038046in}{3.063978in}}%
\pgfpathlineto{\pgfqpoint{4.031676in}{3.076855in}}%
\pgfpathlineto{\pgfqpoint{4.025308in}{3.089666in}}%
\pgfpathlineto{\pgfqpoint{4.018942in}{3.102415in}}%
\pgfpathlineto{\pgfqpoint{4.012576in}{3.115107in}}%
\pgfpathlineto{\pgfqpoint{4.001582in}{3.112103in}}%
\pgfpathlineto{\pgfqpoint{3.990582in}{3.109098in}}%
\pgfpathlineto{\pgfqpoint{3.979576in}{3.106076in}}%
\pgfpathlineto{\pgfqpoint{3.968565in}{3.103022in}}%
\pgfpathlineto{\pgfqpoint{3.957546in}{3.099920in}}%
\pgfpathlineto{\pgfqpoint{3.963907in}{3.087190in}}%
\pgfpathlineto{\pgfqpoint{3.970268in}{3.074321in}}%
\pgfpathlineto{\pgfqpoint{3.976628in}{3.061321in}}%
\pgfpathlineto{\pgfqpoint{3.982988in}{3.048201in}}%
\pgfpathclose%
\pgfusepath{stroke,fill}%
\end{pgfscope}%
\begin{pgfscope}%
\pgfpathrectangle{\pgfqpoint{0.887500in}{0.275000in}}{\pgfqpoint{4.225000in}{4.225000in}}%
\pgfusepath{clip}%
\pgfsetbuttcap%
\pgfsetroundjoin%
\definecolor{currentfill}{rgb}{0.657642,0.860219,0.203082}%
\pgfsetfillcolor{currentfill}%
\pgfsetfillopacity{0.700000}%
\pgfsetlinewidth{0.501875pt}%
\definecolor{currentstroke}{rgb}{1.000000,1.000000,1.000000}%
\pgfsetstrokecolor{currentstroke}%
\pgfsetstrokeopacity{0.500000}%
\pgfsetdash{}{0pt}%
\pgfpathmoveto{\pgfqpoint{3.466620in}{3.293087in}}%
\pgfpathlineto{\pgfqpoint{3.477774in}{3.296329in}}%
\pgfpathlineto{\pgfqpoint{3.488925in}{3.299744in}}%
\pgfpathlineto{\pgfqpoint{3.500072in}{3.303304in}}%
\pgfpathlineto{\pgfqpoint{3.511215in}{3.306971in}}%
\pgfpathlineto{\pgfqpoint{3.522354in}{3.310706in}}%
\pgfpathlineto{\pgfqpoint{3.516071in}{3.321821in}}%
\pgfpathlineto{\pgfqpoint{3.509792in}{3.332951in}}%
\pgfpathlineto{\pgfqpoint{3.503516in}{3.344105in}}%
\pgfpathlineto{\pgfqpoint{3.497244in}{3.355295in}}%
\pgfpathlineto{\pgfqpoint{3.490975in}{3.366530in}}%
\pgfpathlineto{\pgfqpoint{3.479845in}{3.362935in}}%
\pgfpathlineto{\pgfqpoint{3.468709in}{3.359340in}}%
\pgfpathlineto{\pgfqpoint{3.457569in}{3.355756in}}%
\pgfpathlineto{\pgfqpoint{3.446423in}{3.352197in}}%
\pgfpathlineto{\pgfqpoint{3.435272in}{3.348673in}}%
\pgfpathlineto{\pgfqpoint{3.441535in}{3.337544in}}%
\pgfpathlineto{\pgfqpoint{3.447802in}{3.326420in}}%
\pgfpathlineto{\pgfqpoint{3.454071in}{3.315302in}}%
\pgfpathlineto{\pgfqpoint{3.460344in}{3.304191in}}%
\pgfpathclose%
\pgfusepath{stroke,fill}%
\end{pgfscope}%
\begin{pgfscope}%
\pgfpathrectangle{\pgfqpoint{0.887500in}{0.275000in}}{\pgfqpoint{4.225000in}{4.225000in}}%
\pgfusepath{clip}%
\pgfsetbuttcap%
\pgfsetroundjoin%
\definecolor{currentfill}{rgb}{0.132444,0.552216,0.553018}%
\pgfsetfillcolor{currentfill}%
\pgfsetfillopacity{0.700000}%
\pgfsetlinewidth{0.501875pt}%
\definecolor{currentstroke}{rgb}{1.000000,1.000000,1.000000}%
\pgfsetstrokecolor{currentstroke}%
\pgfsetstrokeopacity{0.500000}%
\pgfsetdash{}{0pt}%
\pgfpathmoveto{\pgfqpoint{2.186508in}{2.537218in}}%
\pgfpathlineto{\pgfqpoint{2.197977in}{2.540493in}}%
\pgfpathlineto{\pgfqpoint{2.209440in}{2.543771in}}%
\pgfpathlineto{\pgfqpoint{2.220898in}{2.547053in}}%
\pgfpathlineto{\pgfqpoint{2.232349in}{2.550343in}}%
\pgfpathlineto{\pgfqpoint{2.243795in}{2.553641in}}%
\pgfpathlineto{\pgfqpoint{2.237899in}{2.561859in}}%
\pgfpathlineto{\pgfqpoint{2.232007in}{2.570065in}}%
\pgfpathlineto{\pgfqpoint{2.226119in}{2.578259in}}%
\pgfpathlineto{\pgfqpoint{2.220236in}{2.586441in}}%
\pgfpathlineto{\pgfqpoint{2.214356in}{2.594610in}}%
\pgfpathlineto{\pgfqpoint{2.202923in}{2.591271in}}%
\pgfpathlineto{\pgfqpoint{2.191484in}{2.587957in}}%
\pgfpathlineto{\pgfqpoint{2.180039in}{2.584663in}}%
\pgfpathlineto{\pgfqpoint{2.168588in}{2.581382in}}%
\pgfpathlineto{\pgfqpoint{2.157131in}{2.578111in}}%
\pgfpathlineto{\pgfqpoint{2.162998in}{2.569963in}}%
\pgfpathlineto{\pgfqpoint{2.168869in}{2.561800in}}%
\pgfpathlineto{\pgfqpoint{2.174744in}{2.553621in}}%
\pgfpathlineto{\pgfqpoint{2.180624in}{2.545427in}}%
\pgfpathclose%
\pgfusepath{stroke,fill}%
\end{pgfscope}%
\begin{pgfscope}%
\pgfpathrectangle{\pgfqpoint{0.887500in}{0.275000in}}{\pgfqpoint{4.225000in}{4.225000in}}%
\pgfusepath{clip}%
\pgfsetbuttcap%
\pgfsetroundjoin%
\definecolor{currentfill}{rgb}{0.395174,0.797475,0.367757}%
\pgfsetfillcolor{currentfill}%
\pgfsetfillopacity{0.700000}%
\pgfsetlinewidth{0.501875pt}%
\definecolor{currentstroke}{rgb}{1.000000,1.000000,1.000000}%
\pgfsetstrokecolor{currentstroke}%
\pgfsetstrokeopacity{0.500000}%
\pgfsetdash{}{0pt}%
\pgfpathmoveto{\pgfqpoint{3.902352in}{3.083355in}}%
\pgfpathlineto{\pgfqpoint{3.913404in}{3.086797in}}%
\pgfpathlineto{\pgfqpoint{3.924450in}{3.090187in}}%
\pgfpathlineto{\pgfqpoint{3.935489in}{3.093511in}}%
\pgfpathlineto{\pgfqpoint{3.946521in}{3.096755in}}%
\pgfpathlineto{\pgfqpoint{3.957546in}{3.099920in}}%
\pgfpathlineto{\pgfqpoint{3.951185in}{3.112515in}}%
\pgfpathlineto{\pgfqpoint{3.944825in}{3.124989in}}%
\pgfpathlineto{\pgfqpoint{3.938465in}{3.137355in}}%
\pgfpathlineto{\pgfqpoint{3.932105in}{3.149627in}}%
\pgfpathlineto{\pgfqpoint{3.925747in}{3.161818in}}%
\pgfpathlineto{\pgfqpoint{3.914723in}{3.158509in}}%
\pgfpathlineto{\pgfqpoint{3.903693in}{3.155132in}}%
\pgfpathlineto{\pgfqpoint{3.892655in}{3.151691in}}%
\pgfpathlineto{\pgfqpoint{3.881612in}{3.148203in}}%
\pgfpathlineto{\pgfqpoint{3.870563in}{3.144683in}}%
\pgfpathlineto{\pgfqpoint{3.876919in}{3.132634in}}%
\pgfpathlineto{\pgfqpoint{3.883277in}{3.120504in}}%
\pgfpathlineto{\pgfqpoint{3.889635in}{3.108265in}}%
\pgfpathlineto{\pgfqpoint{3.895994in}{3.095891in}}%
\pgfpathclose%
\pgfusepath{stroke,fill}%
\end{pgfscope}%
\begin{pgfscope}%
\pgfpathrectangle{\pgfqpoint{0.887500in}{0.275000in}}{\pgfqpoint{4.225000in}{4.225000in}}%
\pgfusepath{clip}%
\pgfsetbuttcap%
\pgfsetroundjoin%
\definecolor{currentfill}{rgb}{0.449368,0.813768,0.335384}%
\pgfsetfillcolor{currentfill}%
\pgfsetfillopacity{0.700000}%
\pgfsetlinewidth{0.501875pt}%
\definecolor{currentstroke}{rgb}{1.000000,1.000000,1.000000}%
\pgfsetstrokecolor{currentstroke}%
\pgfsetstrokeopacity{0.500000}%
\pgfsetdash{}{0pt}%
\pgfpathmoveto{\pgfqpoint{3.815242in}{3.127210in}}%
\pgfpathlineto{\pgfqpoint{3.826315in}{3.130642in}}%
\pgfpathlineto{\pgfqpoint{3.837384in}{3.134113in}}%
\pgfpathlineto{\pgfqpoint{3.848449in}{3.137622in}}%
\pgfpathlineto{\pgfqpoint{3.859508in}{3.141151in}}%
\pgfpathlineto{\pgfqpoint{3.870563in}{3.144683in}}%
\pgfpathlineto{\pgfqpoint{3.864208in}{3.156679in}}%
\pgfpathlineto{\pgfqpoint{3.857857in}{3.168647in}}%
\pgfpathlineto{\pgfqpoint{3.851508in}{3.180616in}}%
\pgfpathlineto{\pgfqpoint{3.845163in}{3.192613in}}%
\pgfpathlineto{\pgfqpoint{3.838822in}{3.204664in}}%
\pgfpathlineto{\pgfqpoint{3.827772in}{3.201116in}}%
\pgfpathlineto{\pgfqpoint{3.816717in}{3.197588in}}%
\pgfpathlineto{\pgfqpoint{3.805658in}{3.194096in}}%
\pgfpathlineto{\pgfqpoint{3.794595in}{3.190661in}}%
\pgfpathlineto{\pgfqpoint{3.783528in}{3.187283in}}%
\pgfpathlineto{\pgfqpoint{3.789863in}{3.175191in}}%
\pgfpathlineto{\pgfqpoint{3.796203in}{3.163167in}}%
\pgfpathlineto{\pgfqpoint{3.802546in}{3.151182in}}%
\pgfpathlineto{\pgfqpoint{3.808892in}{3.139207in}}%
\pgfpathclose%
\pgfusepath{stroke,fill}%
\end{pgfscope}%
\begin{pgfscope}%
\pgfpathrectangle{\pgfqpoint{0.887500in}{0.275000in}}{\pgfqpoint{4.225000in}{4.225000in}}%
\pgfusepath{clip}%
\pgfsetbuttcap%
\pgfsetroundjoin%
\definecolor{currentfill}{rgb}{0.143343,0.522773,0.556295}%
\pgfsetfillcolor{currentfill}%
\pgfsetfillopacity{0.700000}%
\pgfsetlinewidth{0.501875pt}%
\definecolor{currentstroke}{rgb}{1.000000,1.000000,1.000000}%
\pgfsetstrokecolor{currentstroke}%
\pgfsetstrokeopacity{0.500000}%
\pgfsetdash{}{0pt}%
\pgfpathmoveto{\pgfqpoint{2.504257in}{2.479377in}}%
\pgfpathlineto{\pgfqpoint{2.515652in}{2.482414in}}%
\pgfpathlineto{\pgfqpoint{2.527041in}{2.485520in}}%
\pgfpathlineto{\pgfqpoint{2.538422in}{2.488783in}}%
\pgfpathlineto{\pgfqpoint{2.549794in}{2.492290in}}%
\pgfpathlineto{\pgfqpoint{2.561157in}{2.496130in}}%
\pgfpathlineto{\pgfqpoint{2.555152in}{2.504640in}}%
\pgfpathlineto{\pgfqpoint{2.549151in}{2.513101in}}%
\pgfpathlineto{\pgfqpoint{2.543155in}{2.521500in}}%
\pgfpathlineto{\pgfqpoint{2.537164in}{2.529826in}}%
\pgfpathlineto{\pgfqpoint{2.531178in}{2.538066in}}%
\pgfpathlineto{\pgfqpoint{2.519827in}{2.534254in}}%
\pgfpathlineto{\pgfqpoint{2.508466in}{2.530768in}}%
\pgfpathlineto{\pgfqpoint{2.497096in}{2.527523in}}%
\pgfpathlineto{\pgfqpoint{2.485719in}{2.524430in}}%
\pgfpathlineto{\pgfqpoint{2.474334in}{2.521401in}}%
\pgfpathlineto{\pgfqpoint{2.480311in}{2.513044in}}%
\pgfpathlineto{\pgfqpoint{2.486291in}{2.504660in}}%
\pgfpathlineto{\pgfqpoint{2.492275in}{2.496253in}}%
\pgfpathlineto{\pgfqpoint{2.498264in}{2.487825in}}%
\pgfpathclose%
\pgfusepath{stroke,fill}%
\end{pgfscope}%
\begin{pgfscope}%
\pgfpathrectangle{\pgfqpoint{0.887500in}{0.275000in}}{\pgfqpoint{4.225000in}{4.225000in}}%
\pgfusepath{clip}%
\pgfsetbuttcap%
\pgfsetroundjoin%
\definecolor{currentfill}{rgb}{0.412913,0.803041,0.357269}%
\pgfsetfillcolor{currentfill}%
\pgfsetfillopacity{0.700000}%
\pgfsetlinewidth{0.501875pt}%
\definecolor{currentstroke}{rgb}{1.000000,1.000000,1.000000}%
\pgfsetstrokecolor{currentstroke}%
\pgfsetstrokeopacity{0.500000}%
\pgfsetdash{}{0pt}%
\pgfpathmoveto{\pgfqpoint{2.923944in}{3.046830in}}%
\pgfpathlineto{\pgfqpoint{2.935125in}{3.080130in}}%
\pgfpathlineto{\pgfqpoint{2.946315in}{3.114384in}}%
\pgfpathlineto{\pgfqpoint{2.957516in}{3.148753in}}%
\pgfpathlineto{\pgfqpoint{2.968731in}{3.182398in}}%
\pgfpathlineto{\pgfqpoint{2.979960in}{3.214469in}}%
\pgfpathlineto{\pgfqpoint{2.973813in}{3.218889in}}%
\pgfpathlineto{\pgfqpoint{2.967672in}{3.223039in}}%
\pgfpathlineto{\pgfqpoint{2.961537in}{3.226916in}}%
\pgfpathlineto{\pgfqpoint{2.955408in}{3.230522in}}%
\pgfpathlineto{\pgfqpoint{2.949284in}{3.233858in}}%
\pgfpathlineto{\pgfqpoint{2.938112in}{3.195712in}}%
\pgfpathlineto{\pgfqpoint{2.926958in}{3.156301in}}%
\pgfpathlineto{\pgfqpoint{2.915818in}{3.117032in}}%
\pgfpathlineto{\pgfqpoint{2.904686in}{3.079303in}}%
\pgfpathlineto{\pgfqpoint{2.893557in}{3.044503in}}%
\pgfpathlineto{\pgfqpoint{2.899620in}{3.045419in}}%
\pgfpathlineto{\pgfqpoint{2.905690in}{3.046105in}}%
\pgfpathlineto{\pgfqpoint{2.911767in}{3.046563in}}%
\pgfpathlineto{\pgfqpoint{2.917852in}{3.046801in}}%
\pgfpathclose%
\pgfusepath{stroke,fill}%
\end{pgfscope}%
\begin{pgfscope}%
\pgfpathrectangle{\pgfqpoint{0.887500in}{0.275000in}}{\pgfqpoint{4.225000in}{4.225000in}}%
\pgfusepath{clip}%
\pgfsetbuttcap%
\pgfsetroundjoin%
\definecolor{currentfill}{rgb}{0.606045,0.850733,0.236712}%
\pgfsetfillcolor{currentfill}%
\pgfsetfillopacity{0.700000}%
\pgfsetlinewidth{0.501875pt}%
\definecolor{currentstroke}{rgb}{1.000000,1.000000,1.000000}%
\pgfsetstrokecolor{currentstroke}%
\pgfsetstrokeopacity{0.500000}%
\pgfsetdash{}{0pt}%
\pgfpathmoveto{\pgfqpoint{3.553814in}{3.255001in}}%
\pgfpathlineto{\pgfqpoint{3.564955in}{3.258802in}}%
\pgfpathlineto{\pgfqpoint{3.576091in}{3.262607in}}%
\pgfpathlineto{\pgfqpoint{3.587222in}{3.266370in}}%
\pgfpathlineto{\pgfqpoint{3.598346in}{3.270058in}}%
\pgfpathlineto{\pgfqpoint{3.609465in}{3.273675in}}%
\pgfpathlineto{\pgfqpoint{3.603162in}{3.284982in}}%
\pgfpathlineto{\pgfqpoint{3.596860in}{3.296154in}}%
\pgfpathlineto{\pgfqpoint{3.590560in}{3.307220in}}%
\pgfpathlineto{\pgfqpoint{3.584263in}{3.318207in}}%
\pgfpathlineto{\pgfqpoint{3.577968in}{3.329145in}}%
\pgfpathlineto{\pgfqpoint{3.566857in}{3.325575in}}%
\pgfpathlineto{\pgfqpoint{3.555739in}{3.321937in}}%
\pgfpathlineto{\pgfqpoint{3.544616in}{3.318228in}}%
\pgfpathlineto{\pgfqpoint{3.533488in}{3.314471in}}%
\pgfpathlineto{\pgfqpoint{3.522354in}{3.310706in}}%
\pgfpathlineto{\pgfqpoint{3.528640in}{3.299596in}}%
\pgfpathlineto{\pgfqpoint{3.534929in}{3.288480in}}%
\pgfpathlineto{\pgfqpoint{3.541221in}{3.277349in}}%
\pgfpathlineto{\pgfqpoint{3.547516in}{3.266193in}}%
\pgfpathclose%
\pgfusepath{stroke,fill}%
\end{pgfscope}%
\begin{pgfscope}%
\pgfpathrectangle{\pgfqpoint{0.887500in}{0.275000in}}{\pgfqpoint{4.225000in}{4.225000in}}%
\pgfusepath{clip}%
\pgfsetbuttcap%
\pgfsetroundjoin%
\definecolor{currentfill}{rgb}{0.506271,0.828786,0.300362}%
\pgfsetfillcolor{currentfill}%
\pgfsetfillopacity{0.700000}%
\pgfsetlinewidth{0.501875pt}%
\definecolor{currentstroke}{rgb}{1.000000,1.000000,1.000000}%
\pgfsetstrokecolor{currentstroke}%
\pgfsetstrokeopacity{0.500000}%
\pgfsetdash{}{0pt}%
\pgfpathmoveto{\pgfqpoint{3.728120in}{3.170871in}}%
\pgfpathlineto{\pgfqpoint{3.739212in}{3.174129in}}%
\pgfpathlineto{\pgfqpoint{3.750298in}{3.177388in}}%
\pgfpathlineto{\pgfqpoint{3.761379in}{3.180659in}}%
\pgfpathlineto{\pgfqpoint{3.772456in}{3.183953in}}%
\pgfpathlineto{\pgfqpoint{3.783528in}{3.187283in}}%
\pgfpathlineto{\pgfqpoint{3.777197in}{3.199455in}}%
\pgfpathlineto{\pgfqpoint{3.770870in}{3.211680in}}%
\pgfpathlineto{\pgfqpoint{3.764546in}{3.223927in}}%
\pgfpathlineto{\pgfqpoint{3.758226in}{3.236166in}}%
\pgfpathlineto{\pgfqpoint{3.751907in}{3.248366in}}%
\pgfpathlineto{\pgfqpoint{3.740840in}{3.245045in}}%
\pgfpathlineto{\pgfqpoint{3.729768in}{3.241741in}}%
\pgfpathlineto{\pgfqpoint{3.718691in}{3.238447in}}%
\pgfpathlineto{\pgfqpoint{3.707609in}{3.235154in}}%
\pgfpathlineto{\pgfqpoint{3.696521in}{3.231855in}}%
\pgfpathlineto{\pgfqpoint{3.702835in}{3.219705in}}%
\pgfpathlineto{\pgfqpoint{3.709152in}{3.207506in}}%
\pgfpathlineto{\pgfqpoint{3.715472in}{3.195284in}}%
\pgfpathlineto{\pgfqpoint{3.721794in}{3.183064in}}%
\pgfpathclose%
\pgfusepath{stroke,fill}%
\end{pgfscope}%
\begin{pgfscope}%
\pgfpathrectangle{\pgfqpoint{0.887500in}{0.275000in}}{\pgfqpoint{4.225000in}{4.225000in}}%
\pgfusepath{clip}%
\pgfsetbuttcap%
\pgfsetroundjoin%
\definecolor{currentfill}{rgb}{0.121148,0.592739,0.544641}%
\pgfsetfillcolor{currentfill}%
\pgfsetfillopacity{0.700000}%
\pgfsetlinewidth{0.501875pt}%
\definecolor{currentstroke}{rgb}{1.000000,1.000000,1.000000}%
\pgfsetstrokecolor{currentstroke}%
\pgfsetstrokeopacity{0.500000}%
\pgfsetdash{}{0pt}%
\pgfpathmoveto{\pgfqpoint{1.637863in}{2.623383in}}%
\pgfpathlineto{\pgfqpoint{1.649468in}{2.626656in}}%
\pgfpathlineto{\pgfqpoint{1.661067in}{2.629928in}}%
\pgfpathlineto{\pgfqpoint{1.672660in}{2.633198in}}%
\pgfpathlineto{\pgfqpoint{1.684248in}{2.636465in}}%
\pgfpathlineto{\pgfqpoint{1.695830in}{2.639729in}}%
\pgfpathlineto{\pgfqpoint{1.690128in}{2.647536in}}%
\pgfpathlineto{\pgfqpoint{1.684430in}{2.655324in}}%
\pgfpathlineto{\pgfqpoint{1.678737in}{2.663092in}}%
\pgfpathlineto{\pgfqpoint{1.673048in}{2.670842in}}%
\pgfpathlineto{\pgfqpoint{1.667363in}{2.678574in}}%
\pgfpathlineto{\pgfqpoint{1.655794in}{2.675303in}}%
\pgfpathlineto{\pgfqpoint{1.644220in}{2.672030in}}%
\pgfpathlineto{\pgfqpoint{1.632640in}{2.668755in}}%
\pgfpathlineto{\pgfqpoint{1.621055in}{2.665478in}}%
\pgfpathlineto{\pgfqpoint{1.609464in}{2.662200in}}%
\pgfpathlineto{\pgfqpoint{1.615135in}{2.654478in}}%
\pgfpathlineto{\pgfqpoint{1.620810in}{2.646735in}}%
\pgfpathlineto{\pgfqpoint{1.626490in}{2.638973in}}%
\pgfpathlineto{\pgfqpoint{1.632174in}{2.631189in}}%
\pgfpathclose%
\pgfusepath{stroke,fill}%
\end{pgfscope}%
\begin{pgfscope}%
\pgfpathrectangle{\pgfqpoint{0.887500in}{0.275000in}}{\pgfqpoint{4.225000in}{4.225000in}}%
\pgfusepath{clip}%
\pgfsetbuttcap%
\pgfsetroundjoin%
\definecolor{currentfill}{rgb}{0.555484,0.840254,0.269281}%
\pgfsetfillcolor{currentfill}%
\pgfsetfillopacity{0.700000}%
\pgfsetlinewidth{0.501875pt}%
\definecolor{currentstroke}{rgb}{1.000000,1.000000,1.000000}%
\pgfsetstrokecolor{currentstroke}%
\pgfsetstrokeopacity{0.500000}%
\pgfsetdash{}{0pt}%
\pgfpathmoveto{\pgfqpoint{3.640999in}{3.215137in}}%
\pgfpathlineto{\pgfqpoint{3.652114in}{3.218508in}}%
\pgfpathlineto{\pgfqpoint{3.663224in}{3.221867in}}%
\pgfpathlineto{\pgfqpoint{3.674329in}{3.225212in}}%
\pgfpathlineto{\pgfqpoint{3.685427in}{3.228542in}}%
\pgfpathlineto{\pgfqpoint{3.696521in}{3.231855in}}%
\pgfpathlineto{\pgfqpoint{3.690208in}{3.243932in}}%
\pgfpathlineto{\pgfqpoint{3.683897in}{3.255911in}}%
\pgfpathlineto{\pgfqpoint{3.677586in}{3.267767in}}%
\pgfpathlineto{\pgfqpoint{3.671277in}{3.279474in}}%
\pgfpathlineto{\pgfqpoint{3.664968in}{3.291008in}}%
\pgfpathlineto{\pgfqpoint{3.653879in}{3.287611in}}%
\pgfpathlineto{\pgfqpoint{3.642784in}{3.284189in}}%
\pgfpathlineto{\pgfqpoint{3.631683in}{3.280731in}}%
\pgfpathlineto{\pgfqpoint{3.620577in}{3.277230in}}%
\pgfpathlineto{\pgfqpoint{3.609465in}{3.273675in}}%
\pgfpathlineto{\pgfqpoint{3.615769in}{3.262217in}}%
\pgfpathlineto{\pgfqpoint{3.622074in}{3.250620in}}%
\pgfpathlineto{\pgfqpoint{3.628381in}{3.238898in}}%
\pgfpathlineto{\pgfqpoint{3.634689in}{3.227065in}}%
\pgfpathclose%
\pgfusepath{stroke,fill}%
\end{pgfscope}%
\begin{pgfscope}%
\pgfpathrectangle{\pgfqpoint{0.887500in}{0.275000in}}{\pgfqpoint{4.225000in}{4.225000in}}%
\pgfusepath{clip}%
\pgfsetbuttcap%
\pgfsetroundjoin%
\definecolor{currentfill}{rgb}{0.814576,0.883393,0.110347}%
\pgfsetfillcolor{currentfill}%
\pgfsetfillopacity{0.700000}%
\pgfsetlinewidth{0.501875pt}%
\definecolor{currentstroke}{rgb}{1.000000,1.000000,1.000000}%
\pgfsetstrokecolor{currentstroke}%
\pgfsetstrokeopacity{0.500000}%
\pgfsetdash{}{0pt}%
\pgfpathmoveto{\pgfqpoint{3.061758in}{3.416722in}}%
\pgfpathlineto{\pgfqpoint{3.073016in}{3.421099in}}%
\pgfpathlineto{\pgfqpoint{3.084269in}{3.424411in}}%
\pgfpathlineto{\pgfqpoint{3.095516in}{3.426933in}}%
\pgfpathlineto{\pgfqpoint{3.106756in}{3.428938in}}%
\pgfpathlineto{\pgfqpoint{3.117989in}{3.430702in}}%
\pgfpathlineto{\pgfqpoint{3.111802in}{3.438423in}}%
\pgfpathlineto{\pgfqpoint{3.105618in}{3.445734in}}%
\pgfpathlineto{\pgfqpoint{3.099438in}{3.452605in}}%
\pgfpathlineto{\pgfqpoint{3.093262in}{3.459005in}}%
\pgfpathlineto{\pgfqpoint{3.087091in}{3.464903in}}%
\pgfpathlineto{\pgfqpoint{3.075869in}{3.463055in}}%
\pgfpathlineto{\pgfqpoint{3.064640in}{3.461033in}}%
\pgfpathlineto{\pgfqpoint{3.053406in}{3.458535in}}%
\pgfpathlineto{\pgfqpoint{3.042166in}{3.455260in}}%
\pgfpathlineto{\pgfqpoint{3.030922in}{3.450907in}}%
\pgfpathlineto{\pgfqpoint{3.037080in}{3.444882in}}%
\pgfpathlineto{\pgfqpoint{3.043243in}{3.438431in}}%
\pgfpathlineto{\pgfqpoint{3.049410in}{3.431573in}}%
\pgfpathlineto{\pgfqpoint{3.055582in}{3.424330in}}%
\pgfpathclose%
\pgfusepath{stroke,fill}%
\end{pgfscope}%
\begin{pgfscope}%
\pgfpathrectangle{\pgfqpoint{0.887500in}{0.275000in}}{\pgfqpoint{4.225000in}{4.225000in}}%
\pgfusepath{clip}%
\pgfsetbuttcap%
\pgfsetroundjoin%
\definecolor{currentfill}{rgb}{0.246070,0.738910,0.452024}%
\pgfsetfillcolor{currentfill}%
\pgfsetfillopacity{0.700000}%
\pgfsetlinewidth{0.501875pt}%
\definecolor{currentstroke}{rgb}{1.000000,1.000000,1.000000}%
\pgfsetstrokecolor{currentstroke}%
\pgfsetstrokeopacity{0.500000}%
\pgfsetdash{}{0pt}%
\pgfpathmoveto{\pgfqpoint{2.898484in}{2.895100in}}%
\pgfpathlineto{\pgfqpoint{2.909674in}{2.923890in}}%
\pgfpathlineto{\pgfqpoint{2.920870in}{2.953474in}}%
\pgfpathlineto{\pgfqpoint{2.932074in}{2.983638in}}%
\pgfpathlineto{\pgfqpoint{2.943286in}{3.014168in}}%
\pgfpathlineto{\pgfqpoint{2.954508in}{3.044847in}}%
\pgfpathlineto{\pgfqpoint{2.948382in}{3.045427in}}%
\pgfpathlineto{\pgfqpoint{2.942263in}{3.045945in}}%
\pgfpathlineto{\pgfqpoint{2.936150in}{3.046372in}}%
\pgfpathlineto{\pgfqpoint{2.930044in}{3.046677in}}%
\pgfpathlineto{\pgfqpoint{2.923944in}{3.046830in}}%
\pgfpathlineto{\pgfqpoint{2.912769in}{3.015235in}}%
\pgfpathlineto{\pgfqpoint{2.901597in}{2.985474in}}%
\pgfpathlineto{\pgfqpoint{2.890427in}{2.957392in}}%
\pgfpathlineto{\pgfqpoint{2.879259in}{2.930837in}}%
\pgfpathlineto{\pgfqpoint{2.868092in}{2.905654in}}%
\pgfpathlineto{\pgfqpoint{2.874163in}{2.902701in}}%
\pgfpathlineto{\pgfqpoint{2.880237in}{2.900132in}}%
\pgfpathlineto{\pgfqpoint{2.886316in}{2.897984in}}%
\pgfpathlineto{\pgfqpoint{2.892398in}{2.896294in}}%
\pgfpathclose%
\pgfusepath{stroke,fill}%
\end{pgfscope}%
\begin{pgfscope}%
\pgfpathrectangle{\pgfqpoint{0.887500in}{0.275000in}}{\pgfqpoint{4.225000in}{4.225000in}}%
\pgfusepath{clip}%
\pgfsetbuttcap%
\pgfsetroundjoin%
\definecolor{currentfill}{rgb}{0.127568,0.566949,0.550556}%
\pgfsetfillcolor{currentfill}%
\pgfsetfillopacity{0.700000}%
\pgfsetlinewidth{0.501875pt}%
\definecolor{currentstroke}{rgb}{1.000000,1.000000,1.000000}%
\pgfsetstrokecolor{currentstroke}%
\pgfsetstrokeopacity{0.500000}%
\pgfsetdash{}{0pt}%
\pgfpathmoveto{\pgfqpoint{1.955524in}{2.569057in}}%
\pgfpathlineto{\pgfqpoint{1.967052in}{2.572362in}}%
\pgfpathlineto{\pgfqpoint{1.978575in}{2.575677in}}%
\pgfpathlineto{\pgfqpoint{1.990091in}{2.578999in}}%
\pgfpathlineto{\pgfqpoint{2.001602in}{2.582327in}}%
\pgfpathlineto{\pgfqpoint{2.013107in}{2.585657in}}%
\pgfpathlineto{\pgfqpoint{2.007290in}{2.593720in}}%
\pgfpathlineto{\pgfqpoint{2.001476in}{2.601772in}}%
\pgfpathlineto{\pgfqpoint{1.995667in}{2.609813in}}%
\pgfpathlineto{\pgfqpoint{1.989861in}{2.617843in}}%
\pgfpathlineto{\pgfqpoint{1.984060in}{2.625861in}}%
\pgfpathlineto{\pgfqpoint{1.972567in}{2.622529in}}%
\pgfpathlineto{\pgfqpoint{1.961069in}{2.619202in}}%
\pgfpathlineto{\pgfqpoint{1.949565in}{2.615882in}}%
\pgfpathlineto{\pgfqpoint{1.938055in}{2.612570in}}%
\pgfpathlineto{\pgfqpoint{1.926539in}{2.609269in}}%
\pgfpathlineto{\pgfqpoint{1.932328in}{2.601251in}}%
\pgfpathlineto{\pgfqpoint{1.938121in}{2.593220in}}%
\pgfpathlineto{\pgfqpoint{1.943918in}{2.585178in}}%
\pgfpathlineto{\pgfqpoint{1.949719in}{2.577124in}}%
\pgfpathclose%
\pgfusepath{stroke,fill}%
\end{pgfscope}%
\begin{pgfscope}%
\pgfpathrectangle{\pgfqpoint{0.887500in}{0.275000in}}{\pgfqpoint{4.225000in}{4.225000in}}%
\pgfusepath{clip}%
\pgfsetbuttcap%
\pgfsetroundjoin%
\definecolor{currentfill}{rgb}{0.119512,0.607464,0.540218}%
\pgfsetfillcolor{currentfill}%
\pgfsetfillopacity{0.700000}%
\pgfsetlinewidth{0.501875pt}%
\definecolor{currentstroke}{rgb}{1.000000,1.000000,1.000000}%
\pgfsetstrokecolor{currentstroke}%
\pgfsetstrokeopacity{0.500000}%
\pgfsetdash{}{0pt}%
\pgfpathmoveto{\pgfqpoint{1.406846in}{2.650852in}}%
\pgfpathlineto{\pgfqpoint{1.418506in}{2.654214in}}%
\pgfpathlineto{\pgfqpoint{1.430160in}{2.657566in}}%
\pgfpathlineto{\pgfqpoint{1.441810in}{2.660908in}}%
\pgfpathlineto{\pgfqpoint{1.453454in}{2.664241in}}%
\pgfpathlineto{\pgfqpoint{1.465092in}{2.667567in}}%
\pgfpathlineto{\pgfqpoint{1.459475in}{2.675172in}}%
\pgfpathlineto{\pgfqpoint{1.453862in}{2.682760in}}%
\pgfpathlineto{\pgfqpoint{1.448254in}{2.690331in}}%
\pgfpathlineto{\pgfqpoint{1.442650in}{2.697884in}}%
\pgfpathlineto{\pgfqpoint{1.437051in}{2.705420in}}%
\pgfpathlineto{\pgfqpoint{1.425426in}{2.702089in}}%
\pgfpathlineto{\pgfqpoint{1.413797in}{2.698750in}}%
\pgfpathlineto{\pgfqpoint{1.402162in}{2.695404in}}%
\pgfpathlineto{\pgfqpoint{1.390521in}{2.692048in}}%
\pgfpathlineto{\pgfqpoint{1.378876in}{2.688681in}}%
\pgfpathlineto{\pgfqpoint{1.384461in}{2.681153in}}%
\pgfpathlineto{\pgfqpoint{1.390050in}{2.673605in}}%
\pgfpathlineto{\pgfqpoint{1.395645in}{2.666038in}}%
\pgfpathlineto{\pgfqpoint{1.401243in}{2.658454in}}%
\pgfpathclose%
\pgfusepath{stroke,fill}%
\end{pgfscope}%
\begin{pgfscope}%
\pgfpathrectangle{\pgfqpoint{0.887500in}{0.275000in}}{\pgfqpoint{4.225000in}{4.225000in}}%
\pgfusepath{clip}%
\pgfsetbuttcap%
\pgfsetroundjoin%
\definecolor{currentfill}{rgb}{0.647257,0.858400,0.209861}%
\pgfsetfillcolor{currentfill}%
\pgfsetfillopacity{0.700000}%
\pgfsetlinewidth{0.501875pt}%
\definecolor{currentstroke}{rgb}{1.000000,1.000000,1.000000}%
\pgfsetstrokecolor{currentstroke}%
\pgfsetstrokeopacity{0.500000}%
\pgfsetdash{}{0pt}%
\pgfpathmoveto{\pgfqpoint{2.949284in}{3.233858in}}%
\pgfpathlineto{\pgfqpoint{2.960477in}{3.269322in}}%
\pgfpathlineto{\pgfqpoint{2.971692in}{3.300695in}}%
\pgfpathlineto{\pgfqpoint{2.982925in}{3.327278in}}%
\pgfpathlineto{\pgfqpoint{2.994173in}{3.349456in}}%
\pgfpathlineto{\pgfqpoint{3.005430in}{3.367702in}}%
\pgfpathlineto{\pgfqpoint{2.999272in}{3.374713in}}%
\pgfpathlineto{\pgfqpoint{2.993118in}{3.381435in}}%
\pgfpathlineto{\pgfqpoint{2.986969in}{3.387860in}}%
\pgfpathlineto{\pgfqpoint{2.980825in}{3.393978in}}%
\pgfpathlineto{\pgfqpoint{2.974687in}{3.399780in}}%
\pgfpathlineto{\pgfqpoint{2.963458in}{3.379673in}}%
\pgfpathlineto{\pgfqpoint{2.952246in}{3.354740in}}%
\pgfpathlineto{\pgfqpoint{2.941056in}{3.324329in}}%
\pgfpathlineto{\pgfqpoint{2.929895in}{3.287920in}}%
\pgfpathlineto{\pgfqpoint{2.918764in}{3.246537in}}%
\pgfpathlineto{\pgfqpoint{2.924855in}{3.244530in}}%
\pgfpathlineto{\pgfqpoint{2.930953in}{3.242260in}}%
\pgfpathlineto{\pgfqpoint{2.937057in}{3.239726in}}%
\pgfpathlineto{\pgfqpoint{2.943167in}{3.236925in}}%
\pgfpathclose%
\pgfusepath{stroke,fill}%
\end{pgfscope}%
\begin{pgfscope}%
\pgfpathrectangle{\pgfqpoint{0.887500in}{0.275000in}}{\pgfqpoint{4.225000in}{4.225000in}}%
\pgfusepath{clip}%
\pgfsetbuttcap%
\pgfsetroundjoin%
\definecolor{currentfill}{rgb}{0.162016,0.687316,0.499129}%
\pgfsetfillcolor{currentfill}%
\pgfsetfillopacity{0.700000}%
\pgfsetlinewidth{0.501875pt}%
\definecolor{currentstroke}{rgb}{1.000000,1.000000,1.000000}%
\pgfsetstrokecolor{currentstroke}%
\pgfsetstrokeopacity{0.500000}%
\pgfsetdash{}{0pt}%
\pgfpathmoveto{\pgfqpoint{4.281907in}{2.811940in}}%
\pgfpathlineto{\pgfqpoint{4.292869in}{2.815435in}}%
\pgfpathlineto{\pgfqpoint{4.303827in}{2.818941in}}%
\pgfpathlineto{\pgfqpoint{4.314780in}{2.822458in}}%
\pgfpathlineto{\pgfqpoint{4.325728in}{2.825989in}}%
\pgfpathlineto{\pgfqpoint{4.319329in}{2.840058in}}%
\pgfpathlineto{\pgfqpoint{4.312932in}{2.854120in}}%
\pgfpathlineto{\pgfqpoint{4.306538in}{2.868176in}}%
\pgfpathlineto{\pgfqpoint{4.300147in}{2.882231in}}%
\pgfpathlineto{\pgfqpoint{4.293757in}{2.896275in}}%
\pgfpathlineto{\pgfqpoint{4.282811in}{2.892746in}}%
\pgfpathlineto{\pgfqpoint{4.271860in}{2.889227in}}%
\pgfpathlineto{\pgfqpoint{4.260903in}{2.885716in}}%
\pgfpathlineto{\pgfqpoint{4.249943in}{2.882211in}}%
\pgfpathlineto{\pgfqpoint{4.256330in}{2.868143in}}%
\pgfpathlineto{\pgfqpoint{4.262720in}{2.854078in}}%
\pgfpathlineto{\pgfqpoint{4.269113in}{2.840025in}}%
\pgfpathlineto{\pgfqpoint{4.275508in}{2.825981in}}%
\pgfpathclose%
\pgfusepath{stroke,fill}%
\end{pgfscope}%
\begin{pgfscope}%
\pgfpathrectangle{\pgfqpoint{0.887500in}{0.275000in}}{\pgfqpoint{4.225000in}{4.225000in}}%
\pgfusepath{clip}%
\pgfsetbuttcap%
\pgfsetroundjoin%
\definecolor{currentfill}{rgb}{0.143343,0.522773,0.556295}%
\pgfsetfillcolor{currentfill}%
\pgfsetfillopacity{0.700000}%
\pgfsetlinewidth{0.501875pt}%
\definecolor{currentstroke}{rgb}{1.000000,1.000000,1.000000}%
\pgfsetstrokecolor{currentstroke}%
\pgfsetstrokeopacity{0.500000}%
\pgfsetdash{}{0pt}%
\pgfpathmoveto{\pgfqpoint{2.735000in}{2.448090in}}%
\pgfpathlineto{\pgfqpoint{2.746329in}{2.451729in}}%
\pgfpathlineto{\pgfqpoint{2.757637in}{2.457562in}}%
\pgfpathlineto{\pgfqpoint{2.768917in}{2.466483in}}%
\pgfpathlineto{\pgfqpoint{2.780168in}{2.479386in}}%
\pgfpathlineto{\pgfqpoint{2.791386in}{2.497020in}}%
\pgfpathlineto{\pgfqpoint{2.785266in}{2.511037in}}%
\pgfpathlineto{\pgfqpoint{2.779147in}{2.525076in}}%
\pgfpathlineto{\pgfqpoint{2.773032in}{2.538996in}}%
\pgfpathlineto{\pgfqpoint{2.766921in}{2.552658in}}%
\pgfpathlineto{\pgfqpoint{2.760815in}{2.565922in}}%
\pgfpathlineto{\pgfqpoint{2.749579in}{2.552318in}}%
\pgfpathlineto{\pgfqpoint{2.738331in}{2.540373in}}%
\pgfpathlineto{\pgfqpoint{2.727069in}{2.529891in}}%
\pgfpathlineto{\pgfqpoint{2.715796in}{2.520641in}}%
\pgfpathlineto{\pgfqpoint{2.704512in}{2.512396in}}%
\pgfpathlineto{\pgfqpoint{2.710601in}{2.499879in}}%
\pgfpathlineto{\pgfqpoint{2.716696in}{2.487072in}}%
\pgfpathlineto{\pgfqpoint{2.722795in}{2.474093in}}%
\pgfpathlineto{\pgfqpoint{2.728896in}{2.461059in}}%
\pgfpathclose%
\pgfusepath{stroke,fill}%
\end{pgfscope}%
\begin{pgfscope}%
\pgfpathrectangle{\pgfqpoint{0.887500in}{0.275000in}}{\pgfqpoint{4.225000in}{4.225000in}}%
\pgfusepath{clip}%
\pgfsetbuttcap%
\pgfsetroundjoin%
\definecolor{currentfill}{rgb}{0.136408,0.541173,0.554483}%
\pgfsetfillcolor{currentfill}%
\pgfsetfillopacity{0.700000}%
\pgfsetlinewidth{0.501875pt}%
\definecolor{currentstroke}{rgb}{1.000000,1.000000,1.000000}%
\pgfsetstrokecolor{currentstroke}%
\pgfsetstrokeopacity{0.500000}%
\pgfsetdash{}{0pt}%
\pgfpathmoveto{\pgfqpoint{2.273336in}{2.512356in}}%
\pgfpathlineto{\pgfqpoint{2.284788in}{2.515666in}}%
\pgfpathlineto{\pgfqpoint{2.296234in}{2.518995in}}%
\pgfpathlineto{\pgfqpoint{2.307675in}{2.522350in}}%
\pgfpathlineto{\pgfqpoint{2.319108in}{2.525738in}}%
\pgfpathlineto{\pgfqpoint{2.330536in}{2.529167in}}%
\pgfpathlineto{\pgfqpoint{2.324608in}{2.537443in}}%
\pgfpathlineto{\pgfqpoint{2.318684in}{2.545708in}}%
\pgfpathlineto{\pgfqpoint{2.312763in}{2.553964in}}%
\pgfpathlineto{\pgfqpoint{2.306847in}{2.562212in}}%
\pgfpathlineto{\pgfqpoint{2.300935in}{2.570454in}}%
\pgfpathlineto{\pgfqpoint{2.289519in}{2.567024in}}%
\pgfpathlineto{\pgfqpoint{2.278098in}{2.563634in}}%
\pgfpathlineto{\pgfqpoint{2.266670in}{2.560279in}}%
\pgfpathlineto{\pgfqpoint{2.255235in}{2.556951in}}%
\pgfpathlineto{\pgfqpoint{2.243795in}{2.553641in}}%
\pgfpathlineto{\pgfqpoint{2.249695in}{2.545410in}}%
\pgfpathlineto{\pgfqpoint{2.255599in}{2.537167in}}%
\pgfpathlineto{\pgfqpoint{2.261508in}{2.528910in}}%
\pgfpathlineto{\pgfqpoint{2.267420in}{2.520640in}}%
\pgfpathclose%
\pgfusepath{stroke,fill}%
\end{pgfscope}%
\begin{pgfscope}%
\pgfpathrectangle{\pgfqpoint{0.887500in}{0.275000in}}{\pgfqpoint{4.225000in}{4.225000in}}%
\pgfusepath{clip}%
\pgfsetbuttcap%
\pgfsetroundjoin%
\definecolor{currentfill}{rgb}{0.202219,0.715272,0.476084}%
\pgfsetfillcolor{currentfill}%
\pgfsetfillopacity{0.700000}%
\pgfsetlinewidth{0.501875pt}%
\definecolor{currentstroke}{rgb}{1.000000,1.000000,1.000000}%
\pgfsetstrokecolor{currentstroke}%
\pgfsetstrokeopacity{0.500000}%
\pgfsetdash{}{0pt}%
\pgfpathmoveto{\pgfqpoint{4.195062in}{2.864749in}}%
\pgfpathlineto{\pgfqpoint{4.206048in}{2.868238in}}%
\pgfpathlineto{\pgfqpoint{4.217030in}{2.871727in}}%
\pgfpathlineto{\pgfqpoint{4.228006in}{2.875218in}}%
\pgfpathlineto{\pgfqpoint{4.238977in}{2.878713in}}%
\pgfpathlineto{\pgfqpoint{4.249943in}{2.882211in}}%
\pgfpathlineto{\pgfqpoint{4.243557in}{2.896268in}}%
\pgfpathlineto{\pgfqpoint{4.237174in}{2.910296in}}%
\pgfpathlineto{\pgfqpoint{4.230791in}{2.924280in}}%
\pgfpathlineto{\pgfqpoint{4.224410in}{2.938204in}}%
\pgfpathlineto{\pgfqpoint{4.218028in}{2.952053in}}%
\pgfpathlineto{\pgfqpoint{4.207065in}{2.948582in}}%
\pgfpathlineto{\pgfqpoint{4.196097in}{2.945105in}}%
\pgfpathlineto{\pgfqpoint{4.185123in}{2.941621in}}%
\pgfpathlineto{\pgfqpoint{4.174144in}{2.938128in}}%
\pgfpathlineto{\pgfqpoint{4.163159in}{2.934626in}}%
\pgfpathlineto{\pgfqpoint{4.169537in}{2.920757in}}%
\pgfpathlineto{\pgfqpoint{4.175917in}{2.906819in}}%
\pgfpathlineto{\pgfqpoint{4.182297in}{2.892828in}}%
\pgfpathlineto{\pgfqpoint{4.188679in}{2.878800in}}%
\pgfpathclose%
\pgfusepath{stroke,fill}%
\end{pgfscope}%
\begin{pgfscope}%
\pgfpathrectangle{\pgfqpoint{0.887500in}{0.275000in}}{\pgfqpoint{4.225000in}{4.225000in}}%
\pgfusepath{clip}%
\pgfsetbuttcap%
\pgfsetroundjoin%
\definecolor{currentfill}{rgb}{0.783315,0.879285,0.125405}%
\pgfsetfillcolor{currentfill}%
\pgfsetfillopacity{0.700000}%
\pgfsetlinewidth{0.501875pt}%
\definecolor{currentstroke}{rgb}{1.000000,1.000000,1.000000}%
\pgfsetstrokecolor{currentstroke}%
\pgfsetstrokeopacity{0.500000}%
\pgfsetdash{}{0pt}%
\pgfpathmoveto{\pgfqpoint{3.148977in}{3.387050in}}%
\pgfpathlineto{\pgfqpoint{3.160214in}{3.388847in}}%
\pgfpathlineto{\pgfqpoint{3.171445in}{3.390839in}}%
\pgfpathlineto{\pgfqpoint{3.182670in}{3.393032in}}%
\pgfpathlineto{\pgfqpoint{3.193890in}{3.395432in}}%
\pgfpathlineto{\pgfqpoint{3.205105in}{3.398042in}}%
\pgfpathlineto{\pgfqpoint{3.198892in}{3.407247in}}%
\pgfpathlineto{\pgfqpoint{3.192682in}{3.416183in}}%
\pgfpathlineto{\pgfqpoint{3.186474in}{3.424836in}}%
\pgfpathlineto{\pgfqpoint{3.180270in}{3.433198in}}%
\pgfpathlineto{\pgfqpoint{3.174069in}{3.441261in}}%
\pgfpathlineto{\pgfqpoint{3.162864in}{3.438792in}}%
\pgfpathlineto{\pgfqpoint{3.151654in}{3.436517in}}%
\pgfpathlineto{\pgfqpoint{3.140438in}{3.434422in}}%
\pgfpathlineto{\pgfqpoint{3.129217in}{3.432490in}}%
\pgfpathlineto{\pgfqpoint{3.117989in}{3.430702in}}%
\pgfpathlineto{\pgfqpoint{3.124180in}{3.422604in}}%
\pgfpathlineto{\pgfqpoint{3.130375in}{3.414159in}}%
\pgfpathlineto{\pgfqpoint{3.136572in}{3.405398in}}%
\pgfpathlineto{\pgfqpoint{3.142773in}{3.396352in}}%
\pgfpathclose%
\pgfusepath{stroke,fill}%
\end{pgfscope}%
\begin{pgfscope}%
\pgfpathrectangle{\pgfqpoint{0.887500in}{0.275000in}}{\pgfqpoint{4.225000in}{4.225000in}}%
\pgfusepath{clip}%
\pgfsetbuttcap%
\pgfsetroundjoin%
\definecolor{currentfill}{rgb}{0.144759,0.519093,0.556572}%
\pgfsetfillcolor{currentfill}%
\pgfsetfillopacity{0.700000}%
\pgfsetlinewidth{0.501875pt}%
\definecolor{currentstroke}{rgb}{1.000000,1.000000,1.000000}%
\pgfsetstrokecolor{currentstroke}%
\pgfsetstrokeopacity{0.500000}%
\pgfsetdash{}{0pt}%
\pgfpathmoveto{\pgfqpoint{2.591244in}{2.453241in}}%
\pgfpathlineto{\pgfqpoint{2.602608in}{2.457490in}}%
\pgfpathlineto{\pgfqpoint{2.613962in}{2.462222in}}%
\pgfpathlineto{\pgfqpoint{2.625306in}{2.467387in}}%
\pgfpathlineto{\pgfqpoint{2.636642in}{2.472938in}}%
\pgfpathlineto{\pgfqpoint{2.647970in}{2.478823in}}%
\pgfpathlineto{\pgfqpoint{2.641929in}{2.487754in}}%
\pgfpathlineto{\pgfqpoint{2.635891in}{2.496657in}}%
\pgfpathlineto{\pgfqpoint{2.629858in}{2.505519in}}%
\pgfpathlineto{\pgfqpoint{2.623830in}{2.514328in}}%
\pgfpathlineto{\pgfqpoint{2.617805in}{2.523073in}}%
\pgfpathlineto{\pgfqpoint{2.606499in}{2.516420in}}%
\pgfpathlineto{\pgfqpoint{2.595180in}{2.510466in}}%
\pgfpathlineto{\pgfqpoint{2.583851in}{2.505145in}}%
\pgfpathlineto{\pgfqpoint{2.572509in}{2.500389in}}%
\pgfpathlineto{\pgfqpoint{2.561157in}{2.496130in}}%
\pgfpathlineto{\pgfqpoint{2.567166in}{2.487581in}}%
\pgfpathlineto{\pgfqpoint{2.573180in}{2.479006in}}%
\pgfpathlineto{\pgfqpoint{2.579197in}{2.470416in}}%
\pgfpathlineto{\pgfqpoint{2.585219in}{2.461824in}}%
\pgfpathclose%
\pgfusepath{stroke,fill}%
\end{pgfscope}%
\begin{pgfscope}%
\pgfpathrectangle{\pgfqpoint{0.887500in}{0.275000in}}{\pgfqpoint{4.225000in}{4.225000in}}%
\pgfusepath{clip}%
\pgfsetbuttcap%
\pgfsetroundjoin%
\definecolor{currentfill}{rgb}{0.122606,0.585371,0.546557}%
\pgfsetfillcolor{currentfill}%
\pgfsetfillopacity{0.700000}%
\pgfsetlinewidth{0.501875pt}%
\definecolor{currentstroke}{rgb}{1.000000,1.000000,1.000000}%
\pgfsetstrokecolor{currentstroke}%
\pgfsetstrokeopacity{0.500000}%
\pgfsetdash{}{0pt}%
\pgfpathmoveto{\pgfqpoint{1.724408in}{2.600362in}}%
\pgfpathlineto{\pgfqpoint{1.735998in}{2.603622in}}%
\pgfpathlineto{\pgfqpoint{1.747582in}{2.606876in}}%
\pgfpathlineto{\pgfqpoint{1.759161in}{2.610124in}}%
\pgfpathlineto{\pgfqpoint{1.770734in}{2.613369in}}%
\pgfpathlineto{\pgfqpoint{1.782301in}{2.616610in}}%
\pgfpathlineto{\pgfqpoint{1.776564in}{2.624527in}}%
\pgfpathlineto{\pgfqpoint{1.770830in}{2.632425in}}%
\pgfpathlineto{\pgfqpoint{1.765102in}{2.640302in}}%
\pgfpathlineto{\pgfqpoint{1.759377in}{2.648158in}}%
\pgfpathlineto{\pgfqpoint{1.753657in}{2.655994in}}%
\pgfpathlineto{\pgfqpoint{1.742103in}{2.652746in}}%
\pgfpathlineto{\pgfqpoint{1.730543in}{2.649497in}}%
\pgfpathlineto{\pgfqpoint{1.718978in}{2.646244in}}%
\pgfpathlineto{\pgfqpoint{1.707407in}{2.642988in}}%
\pgfpathlineto{\pgfqpoint{1.695830in}{2.639729in}}%
\pgfpathlineto{\pgfqpoint{1.701537in}{2.631900in}}%
\pgfpathlineto{\pgfqpoint{1.707248in}{2.624049in}}%
\pgfpathlineto{\pgfqpoint{1.712963in}{2.616176in}}%
\pgfpathlineto{\pgfqpoint{1.718684in}{2.608280in}}%
\pgfpathclose%
\pgfusepath{stroke,fill}%
\end{pgfscope}%
\begin{pgfscope}%
\pgfpathrectangle{\pgfqpoint{0.887500in}{0.275000in}}{\pgfqpoint{4.225000in}{4.225000in}}%
\pgfusepath{clip}%
\pgfsetbuttcap%
\pgfsetroundjoin%
\definecolor{currentfill}{rgb}{0.246070,0.738910,0.452024}%
\pgfsetfillcolor{currentfill}%
\pgfsetfillopacity{0.700000}%
\pgfsetlinewidth{0.501875pt}%
\definecolor{currentstroke}{rgb}{1.000000,1.000000,1.000000}%
\pgfsetstrokecolor{currentstroke}%
\pgfsetstrokeopacity{0.500000}%
\pgfsetdash{}{0pt}%
\pgfpathmoveto{\pgfqpoint{4.108153in}{2.916939in}}%
\pgfpathlineto{\pgfqpoint{4.119166in}{2.920504in}}%
\pgfpathlineto{\pgfqpoint{4.130172in}{2.924054in}}%
\pgfpathlineto{\pgfqpoint{4.141173in}{2.927590in}}%
\pgfpathlineto{\pgfqpoint{4.152169in}{2.931114in}}%
\pgfpathlineto{\pgfqpoint{4.163159in}{2.934626in}}%
\pgfpathlineto{\pgfqpoint{4.156781in}{2.948413in}}%
\pgfpathlineto{\pgfqpoint{4.150402in}{2.962099in}}%
\pgfpathlineto{\pgfqpoint{4.144023in}{2.975672in}}%
\pgfpathlineto{\pgfqpoint{4.137643in}{2.989114in}}%
\pgfpathlineto{\pgfqpoint{4.131262in}{3.002419in}}%
\pgfpathlineto{\pgfqpoint{4.120277in}{2.998968in}}%
\pgfpathlineto{\pgfqpoint{4.109285in}{2.995511in}}%
\pgfpathlineto{\pgfqpoint{4.098289in}{2.992049in}}%
\pgfpathlineto{\pgfqpoint{4.087287in}{2.988580in}}%
\pgfpathlineto{\pgfqpoint{4.076280in}{2.985104in}}%
\pgfpathlineto{\pgfqpoint{4.082654in}{2.971651in}}%
\pgfpathlineto{\pgfqpoint{4.089028in}{2.958100in}}%
\pgfpathlineto{\pgfqpoint{4.095403in}{2.944457in}}%
\pgfpathlineto{\pgfqpoint{4.101778in}{2.930733in}}%
\pgfpathclose%
\pgfusepath{stroke,fill}%
\end{pgfscope}%
\begin{pgfscope}%
\pgfpathrectangle{\pgfqpoint{0.887500in}{0.275000in}}{\pgfqpoint{4.225000in}{4.225000in}}%
\pgfusepath{clip}%
\pgfsetbuttcap%
\pgfsetroundjoin%
\definecolor{currentfill}{rgb}{0.129933,0.559582,0.551864}%
\pgfsetfillcolor{currentfill}%
\pgfsetfillopacity{0.700000}%
\pgfsetlinewidth{0.501875pt}%
\definecolor{currentstroke}{rgb}{1.000000,1.000000,1.000000}%
\pgfsetstrokecolor{currentstroke}%
\pgfsetstrokeopacity{0.500000}%
\pgfsetdash{}{0pt}%
\pgfpathmoveto{\pgfqpoint{2.042257in}{2.545150in}}%
\pgfpathlineto{\pgfqpoint{2.053769in}{2.548477in}}%
\pgfpathlineto{\pgfqpoint{2.065275in}{2.551798in}}%
\pgfpathlineto{\pgfqpoint{2.076776in}{2.555112in}}%
\pgfpathlineto{\pgfqpoint{2.088272in}{2.558419in}}%
\pgfpathlineto{\pgfqpoint{2.099762in}{2.561719in}}%
\pgfpathlineto{\pgfqpoint{2.093911in}{2.569855in}}%
\pgfpathlineto{\pgfqpoint{2.088065in}{2.577977in}}%
\pgfpathlineto{\pgfqpoint{2.082222in}{2.586086in}}%
\pgfpathlineto{\pgfqpoint{2.076384in}{2.594183in}}%
\pgfpathlineto{\pgfqpoint{2.070549in}{2.602266in}}%
\pgfpathlineto{\pgfqpoint{2.059072in}{2.598957in}}%
\pgfpathlineto{\pgfqpoint{2.047589in}{2.595639in}}%
\pgfpathlineto{\pgfqpoint{2.036101in}{2.592315in}}%
\pgfpathlineto{\pgfqpoint{2.024607in}{2.588987in}}%
\pgfpathlineto{\pgfqpoint{2.013107in}{2.585657in}}%
\pgfpathlineto{\pgfqpoint{2.018929in}{2.577582in}}%
\pgfpathlineto{\pgfqpoint{2.024755in}{2.569494in}}%
\pgfpathlineto{\pgfqpoint{2.030585in}{2.561394in}}%
\pgfpathlineto{\pgfqpoint{2.036419in}{2.553279in}}%
\pgfpathclose%
\pgfusepath{stroke,fill}%
\end{pgfscope}%
\begin{pgfscope}%
\pgfpathrectangle{\pgfqpoint{0.887500in}{0.275000in}}{\pgfqpoint{4.225000in}{4.225000in}}%
\pgfusepath{clip}%
\pgfsetbuttcap%
\pgfsetroundjoin%
\definecolor{currentfill}{rgb}{0.741388,0.873449,0.149561}%
\pgfsetfillcolor{currentfill}%
\pgfsetfillopacity{0.700000}%
\pgfsetlinewidth{0.501875pt}%
\definecolor{currentstroke}{rgb}{1.000000,1.000000,1.000000}%
\pgfsetstrokecolor{currentstroke}%
\pgfsetstrokeopacity{0.500000}%
\pgfsetdash{}{0pt}%
\pgfpathmoveto{\pgfqpoint{3.236213in}{3.349142in}}%
\pgfpathlineto{\pgfqpoint{3.247432in}{3.352047in}}%
\pgfpathlineto{\pgfqpoint{3.258646in}{3.355134in}}%
\pgfpathlineto{\pgfqpoint{3.269855in}{3.358394in}}%
\pgfpathlineto{\pgfqpoint{3.281060in}{3.361809in}}%
\pgfpathlineto{\pgfqpoint{3.292260in}{3.365358in}}%
\pgfpathlineto{\pgfqpoint{3.286024in}{3.375456in}}%
\pgfpathlineto{\pgfqpoint{3.279792in}{3.385437in}}%
\pgfpathlineto{\pgfqpoint{3.273562in}{3.395262in}}%
\pgfpathlineto{\pgfqpoint{3.267334in}{3.404893in}}%
\pgfpathlineto{\pgfqpoint{3.261109in}{3.414291in}}%
\pgfpathlineto{\pgfqpoint{3.249917in}{3.410647in}}%
\pgfpathlineto{\pgfqpoint{3.238721in}{3.407178in}}%
\pgfpathlineto{\pgfqpoint{3.227520in}{3.403912in}}%
\pgfpathlineto{\pgfqpoint{3.216315in}{3.400867in}}%
\pgfpathlineto{\pgfqpoint{3.205105in}{3.398042in}}%
\pgfpathlineto{\pgfqpoint{3.211320in}{3.388599in}}%
\pgfpathlineto{\pgfqpoint{3.217539in}{3.378954in}}%
\pgfpathlineto{\pgfqpoint{3.223761in}{3.369140in}}%
\pgfpathlineto{\pgfqpoint{3.229985in}{3.359191in}}%
\pgfpathclose%
\pgfusepath{stroke,fill}%
\end{pgfscope}%
\begin{pgfscope}%
\pgfpathrectangle{\pgfqpoint{0.887500in}{0.275000in}}{\pgfqpoint{4.225000in}{4.225000in}}%
\pgfusepath{clip}%
\pgfsetbuttcap%
\pgfsetroundjoin%
\definecolor{currentfill}{rgb}{0.120092,0.600104,0.542530}%
\pgfsetfillcolor{currentfill}%
\pgfsetfillopacity{0.700000}%
\pgfsetlinewidth{0.501875pt}%
\definecolor{currentstroke}{rgb}{1.000000,1.000000,1.000000}%
\pgfsetstrokecolor{currentstroke}%
\pgfsetstrokeopacity{0.500000}%
\pgfsetdash{}{0pt}%
\pgfpathmoveto{\pgfqpoint{1.493242in}{2.629294in}}%
\pgfpathlineto{\pgfqpoint{1.504889in}{2.632607in}}%
\pgfpathlineto{\pgfqpoint{1.516531in}{2.635912in}}%
\pgfpathlineto{\pgfqpoint{1.528167in}{2.639211in}}%
\pgfpathlineto{\pgfqpoint{1.539798in}{2.642505in}}%
\pgfpathlineto{\pgfqpoint{1.551423in}{2.645794in}}%
\pgfpathlineto{\pgfqpoint{1.545770in}{2.653490in}}%
\pgfpathlineto{\pgfqpoint{1.540122in}{2.661167in}}%
\pgfpathlineto{\pgfqpoint{1.534478in}{2.668828in}}%
\pgfpathlineto{\pgfqpoint{1.528838in}{2.676472in}}%
\pgfpathlineto{\pgfqpoint{1.523203in}{2.684100in}}%
\pgfpathlineto{\pgfqpoint{1.511592in}{2.680803in}}%
\pgfpathlineto{\pgfqpoint{1.499975in}{2.677502in}}%
\pgfpathlineto{\pgfqpoint{1.488353in}{2.674196in}}%
\pgfpathlineto{\pgfqpoint{1.476725in}{2.670884in}}%
\pgfpathlineto{\pgfqpoint{1.465092in}{2.667567in}}%
\pgfpathlineto{\pgfqpoint{1.470713in}{2.659945in}}%
\pgfpathlineto{\pgfqpoint{1.476339in}{2.652308in}}%
\pgfpathlineto{\pgfqpoint{1.481969in}{2.644654in}}%
\pgfpathlineto{\pgfqpoint{1.487603in}{2.636984in}}%
\pgfpathclose%
\pgfusepath{stroke,fill}%
\end{pgfscope}%
\begin{pgfscope}%
\pgfpathrectangle{\pgfqpoint{0.887500in}{0.275000in}}{\pgfqpoint{4.225000in}{4.225000in}}%
\pgfusepath{clip}%
\pgfsetbuttcap%
\pgfsetroundjoin%
\definecolor{currentfill}{rgb}{0.127568,0.566949,0.550556}%
\pgfsetfillcolor{currentfill}%
\pgfsetfillopacity{0.700000}%
\pgfsetlinewidth{0.501875pt}%
\definecolor{currentstroke}{rgb}{1.000000,1.000000,1.000000}%
\pgfsetstrokecolor{currentstroke}%
\pgfsetstrokeopacity{0.500000}%
\pgfsetdash{}{0pt}%
\pgfpathmoveto{\pgfqpoint{2.791386in}{2.497020in}}%
\pgfpathlineto{\pgfqpoint{2.802580in}{2.518994in}}%
\pgfpathlineto{\pgfqpoint{2.813757in}{2.544397in}}%
\pgfpathlineto{\pgfqpoint{2.824927in}{2.572314in}}%
\pgfpathlineto{\pgfqpoint{2.836097in}{2.601827in}}%
\pgfpathlineto{\pgfqpoint{2.847272in}{2.632014in}}%
\pgfpathlineto{\pgfqpoint{2.841180in}{2.636827in}}%
\pgfpathlineto{\pgfqpoint{2.835095in}{2.641335in}}%
\pgfpathlineto{\pgfqpoint{2.829016in}{2.645683in}}%
\pgfpathlineto{\pgfqpoint{2.822943in}{2.650018in}}%
\pgfpathlineto{\pgfqpoint{2.816873in}{2.654485in}}%
\pgfpathlineto{\pgfqpoint{2.805670in}{2.634613in}}%
\pgfpathlineto{\pgfqpoint{2.794464in}{2.615647in}}%
\pgfpathlineto{\pgfqpoint{2.783255in}{2.597760in}}%
\pgfpathlineto{\pgfqpoint{2.772039in}{2.581127in}}%
\pgfpathlineto{\pgfqpoint{2.760815in}{2.565922in}}%
\pgfpathlineto{\pgfqpoint{2.766921in}{2.552658in}}%
\pgfpathlineto{\pgfqpoint{2.773032in}{2.538996in}}%
\pgfpathlineto{\pgfqpoint{2.779147in}{2.525076in}}%
\pgfpathlineto{\pgfqpoint{2.785266in}{2.511037in}}%
\pgfpathclose%
\pgfusepath{stroke,fill}%
\end{pgfscope}%
\begin{pgfscope}%
\pgfpathrectangle{\pgfqpoint{0.887500in}{0.275000in}}{\pgfqpoint{4.225000in}{4.225000in}}%
\pgfusepath{clip}%
\pgfsetbuttcap%
\pgfsetroundjoin%
\definecolor{currentfill}{rgb}{0.139147,0.533812,0.555298}%
\pgfsetfillcolor{currentfill}%
\pgfsetfillopacity{0.700000}%
\pgfsetlinewidth{0.501875pt}%
\definecolor{currentstroke}{rgb}{1.000000,1.000000,1.000000}%
\pgfsetstrokecolor{currentstroke}%
\pgfsetstrokeopacity{0.500000}%
\pgfsetdash{}{0pt}%
\pgfpathmoveto{\pgfqpoint{2.360238in}{2.487574in}}%
\pgfpathlineto{\pgfqpoint{2.371671in}{2.491029in}}%
\pgfpathlineto{\pgfqpoint{2.383099in}{2.494521in}}%
\pgfpathlineto{\pgfqpoint{2.394520in}{2.498041in}}%
\pgfpathlineto{\pgfqpoint{2.405936in}{2.501568in}}%
\pgfpathlineto{\pgfqpoint{2.417346in}{2.505072in}}%
\pgfpathlineto{\pgfqpoint{2.411385in}{2.513456in}}%
\pgfpathlineto{\pgfqpoint{2.405428in}{2.521833in}}%
\pgfpathlineto{\pgfqpoint{2.399475in}{2.530201in}}%
\pgfpathlineto{\pgfqpoint{2.393526in}{2.538557in}}%
\pgfpathlineto{\pgfqpoint{2.387581in}{2.546897in}}%
\pgfpathlineto{\pgfqpoint{2.376183in}{2.543341in}}%
\pgfpathlineto{\pgfqpoint{2.364780in}{2.539754in}}%
\pgfpathlineto{\pgfqpoint{2.353372in}{2.536175in}}%
\pgfpathlineto{\pgfqpoint{2.341957in}{2.532644in}}%
\pgfpathlineto{\pgfqpoint{2.330536in}{2.529167in}}%
\pgfpathlineto{\pgfqpoint{2.336468in}{2.520879in}}%
\pgfpathlineto{\pgfqpoint{2.342404in}{2.512577in}}%
\pgfpathlineto{\pgfqpoint{2.348345in}{2.504260in}}%
\pgfpathlineto{\pgfqpoint{2.354289in}{2.495926in}}%
\pgfpathclose%
\pgfusepath{stroke,fill}%
\end{pgfscope}%
\begin{pgfscope}%
\pgfpathrectangle{\pgfqpoint{0.887500in}{0.275000in}}{\pgfqpoint{4.225000in}{4.225000in}}%
\pgfusepath{clip}%
\pgfsetbuttcap%
\pgfsetroundjoin%
\definecolor{currentfill}{rgb}{0.288921,0.758394,0.428426}%
\pgfsetfillcolor{currentfill}%
\pgfsetfillopacity{0.700000}%
\pgfsetlinewidth{0.501875pt}%
\definecolor{currentstroke}{rgb}{1.000000,1.000000,1.000000}%
\pgfsetstrokecolor{currentstroke}%
\pgfsetstrokeopacity{0.500000}%
\pgfsetdash{}{0pt}%
\pgfpathmoveto{\pgfqpoint{4.021162in}{2.967565in}}%
\pgfpathlineto{\pgfqpoint{4.032197in}{2.971101in}}%
\pgfpathlineto{\pgfqpoint{4.043226in}{2.974621in}}%
\pgfpathlineto{\pgfqpoint{4.054249in}{2.978127in}}%
\pgfpathlineto{\pgfqpoint{4.065267in}{2.981620in}}%
\pgfpathlineto{\pgfqpoint{4.076280in}{2.985104in}}%
\pgfpathlineto{\pgfqpoint{4.069906in}{2.998463in}}%
\pgfpathlineto{\pgfqpoint{4.063533in}{3.011730in}}%
\pgfpathlineto{\pgfqpoint{4.057160in}{3.024912in}}%
\pgfpathlineto{\pgfqpoint{4.050787in}{3.038011in}}%
\pgfpathlineto{\pgfqpoint{4.044416in}{3.051031in}}%
\pgfpathlineto{\pgfqpoint{4.033414in}{3.047838in}}%
\pgfpathlineto{\pgfqpoint{4.022406in}{3.044647in}}%
\pgfpathlineto{\pgfqpoint{4.011393in}{3.041447in}}%
\pgfpathlineto{\pgfqpoint{4.000374in}{3.038226in}}%
\pgfpathlineto{\pgfqpoint{3.989349in}{3.034972in}}%
\pgfpathlineto{\pgfqpoint{3.995710in}{3.021644in}}%
\pgfpathlineto{\pgfqpoint{4.002072in}{3.008229in}}%
\pgfpathlineto{\pgfqpoint{4.008434in}{2.994737in}}%
\pgfpathlineto{\pgfqpoint{4.014797in}{2.981179in}}%
\pgfpathclose%
\pgfusepath{stroke,fill}%
\end{pgfscope}%
\begin{pgfscope}%
\pgfpathrectangle{\pgfqpoint{0.887500in}{0.275000in}}{\pgfqpoint{4.225000in}{4.225000in}}%
\pgfusepath{clip}%
\pgfsetbuttcap%
\pgfsetroundjoin%
\definecolor{currentfill}{rgb}{0.699415,0.867117,0.175971}%
\pgfsetfillcolor{currentfill}%
\pgfsetfillopacity{0.700000}%
\pgfsetlinewidth{0.501875pt}%
\definecolor{currentstroke}{rgb}{1.000000,1.000000,1.000000}%
\pgfsetstrokecolor{currentstroke}%
\pgfsetstrokeopacity{0.500000}%
\pgfsetdash{}{0pt}%
\pgfpathmoveto{\pgfqpoint{3.323488in}{3.314456in}}%
\pgfpathlineto{\pgfqpoint{3.334690in}{3.317805in}}%
\pgfpathlineto{\pgfqpoint{3.345886in}{3.321186in}}%
\pgfpathlineto{\pgfqpoint{3.357078in}{3.324589in}}%
\pgfpathlineto{\pgfqpoint{3.368264in}{3.328004in}}%
\pgfpathlineto{\pgfqpoint{3.379445in}{3.331420in}}%
\pgfpathlineto{\pgfqpoint{3.373188in}{3.342023in}}%
\pgfpathlineto{\pgfqpoint{3.366934in}{3.352651in}}%
\pgfpathlineto{\pgfqpoint{3.360684in}{3.363271in}}%
\pgfpathlineto{\pgfqpoint{3.354436in}{3.373843in}}%
\pgfpathlineto{\pgfqpoint{3.348192in}{3.384330in}}%
\pgfpathlineto{\pgfqpoint{3.337015in}{3.380445in}}%
\pgfpathlineto{\pgfqpoint{3.325833in}{3.376584in}}%
\pgfpathlineto{\pgfqpoint{3.314647in}{3.372768in}}%
\pgfpathlineto{\pgfqpoint{3.303456in}{3.369018in}}%
\pgfpathlineto{\pgfqpoint{3.292260in}{3.365358in}}%
\pgfpathlineto{\pgfqpoint{3.298498in}{3.355181in}}%
\pgfpathlineto{\pgfqpoint{3.304740in}{3.344965in}}%
\pgfpathlineto{\pgfqpoint{3.310986in}{3.334747in}}%
\pgfpathlineto{\pgfqpoint{3.317235in}{3.324566in}}%
\pgfpathclose%
\pgfusepath{stroke,fill}%
\end{pgfscope}%
\begin{pgfscope}%
\pgfpathrectangle{\pgfqpoint{0.887500in}{0.275000in}}{\pgfqpoint{4.225000in}{4.225000in}}%
\pgfusepath{clip}%
\pgfsetbuttcap%
\pgfsetroundjoin%
\definecolor{currentfill}{rgb}{0.153894,0.680203,0.504172}%
\pgfsetfillcolor{currentfill}%
\pgfsetfillopacity{0.700000}%
\pgfsetlinewidth{0.501875pt}%
\definecolor{currentstroke}{rgb}{1.000000,1.000000,1.000000}%
\pgfsetstrokecolor{currentstroke}%
\pgfsetstrokeopacity{0.500000}%
\pgfsetdash{}{0pt}%
\pgfpathmoveto{\pgfqpoint{2.872902in}{2.764206in}}%
\pgfpathlineto{\pgfqpoint{2.884112in}{2.788664in}}%
\pgfpathlineto{\pgfqpoint{2.895324in}{2.814220in}}%
\pgfpathlineto{\pgfqpoint{2.906538in}{2.840987in}}%
\pgfpathlineto{\pgfqpoint{2.917756in}{2.868978in}}%
\pgfpathlineto{\pgfqpoint{2.928980in}{2.897870in}}%
\pgfpathlineto{\pgfqpoint{2.922872in}{2.896035in}}%
\pgfpathlineto{\pgfqpoint{2.916768in}{2.894869in}}%
\pgfpathlineto{\pgfqpoint{2.910669in}{2.894349in}}%
\pgfpathlineto{\pgfqpoint{2.904575in}{2.894439in}}%
\pgfpathlineto{\pgfqpoint{2.898484in}{2.895100in}}%
\pgfpathlineto{\pgfqpoint{2.887299in}{2.867317in}}%
\pgfpathlineto{\pgfqpoint{2.876116in}{2.840755in}}%
\pgfpathlineto{\pgfqpoint{2.864935in}{2.815598in}}%
\pgfpathlineto{\pgfqpoint{2.853753in}{2.791910in}}%
\pgfpathlineto{\pgfqpoint{2.842569in}{2.769717in}}%
\pgfpathlineto{\pgfqpoint{2.848632in}{2.766818in}}%
\pgfpathlineto{\pgfqpoint{2.854697in}{2.764790in}}%
\pgfpathlineto{\pgfqpoint{2.860764in}{2.763661in}}%
\pgfpathlineto{\pgfqpoint{2.866832in}{2.763462in}}%
\pgfpathclose%
\pgfusepath{stroke,fill}%
\end{pgfscope}%
\begin{pgfscope}%
\pgfpathrectangle{\pgfqpoint{0.887500in}{0.275000in}}{\pgfqpoint{4.225000in}{4.225000in}}%
\pgfusepath{clip}%
\pgfsetbuttcap%
\pgfsetroundjoin%
\definecolor{currentfill}{rgb}{0.783315,0.879285,0.125405}%
\pgfsetfillcolor{currentfill}%
\pgfsetfillopacity{0.700000}%
\pgfsetlinewidth{0.501875pt}%
\definecolor{currentstroke}{rgb}{1.000000,1.000000,1.000000}%
\pgfsetstrokecolor{currentstroke}%
\pgfsetstrokeopacity{0.500000}%
\pgfsetdash{}{0pt}%
\pgfpathmoveto{\pgfqpoint{3.005430in}{3.367702in}}%
\pgfpathlineto{\pgfqpoint{3.016695in}{3.382497in}}%
\pgfpathlineto{\pgfqpoint{3.027962in}{3.394323in}}%
\pgfpathlineto{\pgfqpoint{3.039230in}{3.403665in}}%
\pgfpathlineto{\pgfqpoint{3.050496in}{3.411003in}}%
\pgfpathlineto{\pgfqpoint{3.061758in}{3.416722in}}%
\pgfpathlineto{\pgfqpoint{3.055582in}{3.424330in}}%
\pgfpathlineto{\pgfqpoint{3.049410in}{3.431573in}}%
\pgfpathlineto{\pgfqpoint{3.043243in}{3.438431in}}%
\pgfpathlineto{\pgfqpoint{3.037080in}{3.444882in}}%
\pgfpathlineto{\pgfqpoint{3.030922in}{3.450907in}}%
\pgfpathlineto{\pgfqpoint{3.019674in}{3.445175in}}%
\pgfpathlineto{\pgfqpoint{3.008424in}{3.437755in}}%
\pgfpathlineto{\pgfqpoint{2.997174in}{3.428157in}}%
\pgfpathlineto{\pgfqpoint{2.985927in}{3.415720in}}%
\pgfpathlineto{\pgfqpoint{2.974687in}{3.399780in}}%
\pgfpathlineto{\pgfqpoint{2.980825in}{3.393978in}}%
\pgfpathlineto{\pgfqpoint{2.986969in}{3.387860in}}%
\pgfpathlineto{\pgfqpoint{2.993118in}{3.381435in}}%
\pgfpathlineto{\pgfqpoint{2.999272in}{3.374713in}}%
\pgfpathclose%
\pgfusepath{stroke,fill}%
\end{pgfscope}%
\begin{pgfscope}%
\pgfpathrectangle{\pgfqpoint{0.887500in}{0.275000in}}{\pgfqpoint{4.225000in}{4.225000in}}%
\pgfusepath{clip}%
\pgfsetbuttcap%
\pgfsetroundjoin%
\definecolor{currentfill}{rgb}{0.125394,0.574318,0.549086}%
\pgfsetfillcolor{currentfill}%
\pgfsetfillopacity{0.700000}%
\pgfsetlinewidth{0.501875pt}%
\definecolor{currentstroke}{rgb}{1.000000,1.000000,1.000000}%
\pgfsetstrokecolor{currentstroke}%
\pgfsetstrokeopacity{0.500000}%
\pgfsetdash{}{0pt}%
\pgfpathmoveto{\pgfqpoint{1.811054in}{2.576748in}}%
\pgfpathlineto{\pgfqpoint{1.822628in}{2.579987in}}%
\pgfpathlineto{\pgfqpoint{1.834197in}{2.583225in}}%
\pgfpathlineto{\pgfqpoint{1.845761in}{2.586461in}}%
\pgfpathlineto{\pgfqpoint{1.857318in}{2.589699in}}%
\pgfpathlineto{\pgfqpoint{1.868870in}{2.592941in}}%
\pgfpathlineto{\pgfqpoint{1.863098in}{2.600946in}}%
\pgfpathlineto{\pgfqpoint{1.857331in}{2.608935in}}%
\pgfpathlineto{\pgfqpoint{1.851567in}{2.616909in}}%
\pgfpathlineto{\pgfqpoint{1.845808in}{2.624866in}}%
\pgfpathlineto{\pgfqpoint{1.840053in}{2.632806in}}%
\pgfpathlineto{\pgfqpoint{1.828514in}{2.629564in}}%
\pgfpathlineto{\pgfqpoint{1.816970in}{2.626324in}}%
\pgfpathlineto{\pgfqpoint{1.805419in}{2.623087in}}%
\pgfpathlineto{\pgfqpoint{1.793863in}{2.619849in}}%
\pgfpathlineto{\pgfqpoint{1.782301in}{2.616610in}}%
\pgfpathlineto{\pgfqpoint{1.788043in}{2.608673in}}%
\pgfpathlineto{\pgfqpoint{1.793789in}{2.600718in}}%
\pgfpathlineto{\pgfqpoint{1.799540in}{2.592745in}}%
\pgfpathlineto{\pgfqpoint{1.805295in}{2.584755in}}%
\pgfpathclose%
\pgfusepath{stroke,fill}%
\end{pgfscope}%
\begin{pgfscope}%
\pgfpathrectangle{\pgfqpoint{0.887500in}{0.275000in}}{\pgfqpoint{4.225000in}{4.225000in}}%
\pgfusepath{clip}%
\pgfsetbuttcap%
\pgfsetroundjoin%
\definecolor{currentfill}{rgb}{0.344074,0.780029,0.397381}%
\pgfsetfillcolor{currentfill}%
\pgfsetfillopacity{0.700000}%
\pgfsetlinewidth{0.501875pt}%
\definecolor{currentstroke}{rgb}{1.000000,1.000000,1.000000}%
\pgfsetstrokecolor{currentstroke}%
\pgfsetstrokeopacity{0.500000}%
\pgfsetdash{}{0pt}%
\pgfpathmoveto{\pgfqpoint{3.934127in}{3.017924in}}%
\pgfpathlineto{\pgfqpoint{3.945184in}{3.021428in}}%
\pgfpathlineto{\pgfqpoint{3.956235in}{3.024895in}}%
\pgfpathlineto{\pgfqpoint{3.967280in}{3.028313in}}%
\pgfpathlineto{\pgfqpoint{3.978318in}{3.031671in}}%
\pgfpathlineto{\pgfqpoint{3.989349in}{3.034972in}}%
\pgfpathlineto{\pgfqpoint{3.982988in}{3.048201in}}%
\pgfpathlineto{\pgfqpoint{3.976628in}{3.061321in}}%
\pgfpathlineto{\pgfqpoint{3.970268in}{3.074321in}}%
\pgfpathlineto{\pgfqpoint{3.963907in}{3.087190in}}%
\pgfpathlineto{\pgfqpoint{3.957546in}{3.099920in}}%
\pgfpathlineto{\pgfqpoint{3.946521in}{3.096755in}}%
\pgfpathlineto{\pgfqpoint{3.935489in}{3.093511in}}%
\pgfpathlineto{\pgfqpoint{3.924450in}{3.090187in}}%
\pgfpathlineto{\pgfqpoint{3.913404in}{3.086797in}}%
\pgfpathlineto{\pgfqpoint{3.902352in}{3.083355in}}%
\pgfpathlineto{\pgfqpoint{3.908709in}{3.070629in}}%
\pgfpathlineto{\pgfqpoint{3.915065in}{3.057702in}}%
\pgfpathlineto{\pgfqpoint{3.921419in}{3.044595in}}%
\pgfpathlineto{\pgfqpoint{3.927773in}{3.031328in}}%
\pgfpathclose%
\pgfusepath{stroke,fill}%
\end{pgfscope}%
\begin{pgfscope}%
\pgfpathrectangle{\pgfqpoint{0.887500in}{0.275000in}}{\pgfqpoint{4.225000in}{4.225000in}}%
\pgfusepath{clip}%
\pgfsetbuttcap%
\pgfsetroundjoin%
\definecolor{currentfill}{rgb}{0.657642,0.860219,0.203082}%
\pgfsetfillcolor{currentfill}%
\pgfsetfillopacity{0.700000}%
\pgfsetlinewidth{0.501875pt}%
\definecolor{currentstroke}{rgb}{1.000000,1.000000,1.000000}%
\pgfsetstrokecolor{currentstroke}%
\pgfsetstrokeopacity{0.500000}%
\pgfsetdash{}{0pt}%
\pgfpathmoveto{\pgfqpoint{3.410783in}{3.278548in}}%
\pgfpathlineto{\pgfqpoint{3.421961in}{3.281340in}}%
\pgfpathlineto{\pgfqpoint{3.433133in}{3.284157in}}%
\pgfpathlineto{\pgfqpoint{3.444300in}{3.287033in}}%
\pgfpathlineto{\pgfqpoint{3.455462in}{3.289999in}}%
\pgfpathlineto{\pgfqpoint{3.466620in}{3.293087in}}%
\pgfpathlineto{\pgfqpoint{3.460344in}{3.304191in}}%
\pgfpathlineto{\pgfqpoint{3.454071in}{3.315302in}}%
\pgfpathlineto{\pgfqpoint{3.447802in}{3.326420in}}%
\pgfpathlineto{\pgfqpoint{3.441535in}{3.337544in}}%
\pgfpathlineto{\pgfqpoint{3.435272in}{3.348673in}}%
\pgfpathlineto{\pgfqpoint{3.424117in}{3.345181in}}%
\pgfpathlineto{\pgfqpoint{3.412957in}{3.341715in}}%
\pgfpathlineto{\pgfqpoint{3.401791in}{3.338270in}}%
\pgfpathlineto{\pgfqpoint{3.390621in}{3.334840in}}%
\pgfpathlineto{\pgfqpoint{3.379445in}{3.331420in}}%
\pgfpathlineto{\pgfqpoint{3.385706in}{3.320847in}}%
\pgfpathlineto{\pgfqpoint{3.391970in}{3.310290in}}%
\pgfpathlineto{\pgfqpoint{3.398238in}{3.299731in}}%
\pgfpathlineto{\pgfqpoint{3.404509in}{3.289156in}}%
\pgfpathclose%
\pgfusepath{stroke,fill}%
\end{pgfscope}%
\begin{pgfscope}%
\pgfpathrectangle{\pgfqpoint{0.887500in}{0.275000in}}{\pgfqpoint{4.225000in}{4.225000in}}%
\pgfusepath{clip}%
\pgfsetbuttcap%
\pgfsetroundjoin%
\definecolor{currentfill}{rgb}{0.147607,0.511733,0.557049}%
\pgfsetfillcolor{currentfill}%
\pgfsetfillopacity{0.700000}%
\pgfsetlinewidth{0.501875pt}%
\definecolor{currentstroke}{rgb}{1.000000,1.000000,1.000000}%
\pgfsetstrokecolor{currentstroke}%
\pgfsetstrokeopacity{0.500000}%
\pgfsetdash{}{0pt}%
\pgfpathmoveto{\pgfqpoint{2.678235in}{2.433946in}}%
\pgfpathlineto{\pgfqpoint{2.689588in}{2.438053in}}%
\pgfpathlineto{\pgfqpoint{2.700942in}{2.441405in}}%
\pgfpathlineto{\pgfqpoint{2.712299in}{2.443818in}}%
\pgfpathlineto{\pgfqpoint{2.723655in}{2.445750in}}%
\pgfpathlineto{\pgfqpoint{2.735000in}{2.448090in}}%
\pgfpathlineto{\pgfqpoint{2.728896in}{2.461059in}}%
\pgfpathlineto{\pgfqpoint{2.722795in}{2.474093in}}%
\pgfpathlineto{\pgfqpoint{2.716696in}{2.487072in}}%
\pgfpathlineto{\pgfqpoint{2.710601in}{2.499879in}}%
\pgfpathlineto{\pgfqpoint{2.704512in}{2.512396in}}%
\pgfpathlineto{\pgfqpoint{2.693218in}{2.504926in}}%
\pgfpathlineto{\pgfqpoint{2.681915in}{2.498002in}}%
\pgfpathlineto{\pgfqpoint{2.670606in}{2.491398in}}%
\pgfpathlineto{\pgfqpoint{2.659291in}{2.484993in}}%
\pgfpathlineto{\pgfqpoint{2.647970in}{2.478823in}}%
\pgfpathlineto{\pgfqpoint{2.654015in}{2.469870in}}%
\pgfpathlineto{\pgfqpoint{2.660064in}{2.460901in}}%
\pgfpathlineto{\pgfqpoint{2.666117in}{2.451920in}}%
\pgfpathlineto{\pgfqpoint{2.672174in}{2.442934in}}%
\pgfpathclose%
\pgfusepath{stroke,fill}%
\end{pgfscope}%
\begin{pgfscope}%
\pgfpathrectangle{\pgfqpoint{0.887500in}{0.275000in}}{\pgfqpoint{4.225000in}{4.225000in}}%
\pgfusepath{clip}%
\pgfsetbuttcap%
\pgfsetroundjoin%
\definecolor{currentfill}{rgb}{0.132444,0.552216,0.553018}%
\pgfsetfillcolor{currentfill}%
\pgfsetfillopacity{0.700000}%
\pgfsetlinewidth{0.501875pt}%
\definecolor{currentstroke}{rgb}{1.000000,1.000000,1.000000}%
\pgfsetstrokecolor{currentstroke}%
\pgfsetstrokeopacity{0.500000}%
\pgfsetdash{}{0pt}%
\pgfpathmoveto{\pgfqpoint{2.129078in}{2.520818in}}%
\pgfpathlineto{\pgfqpoint{2.140575in}{2.524106in}}%
\pgfpathlineto{\pgfqpoint{2.152067in}{2.527389in}}%
\pgfpathlineto{\pgfqpoint{2.163553in}{2.530668in}}%
\pgfpathlineto{\pgfqpoint{2.175033in}{2.533943in}}%
\pgfpathlineto{\pgfqpoint{2.186508in}{2.537218in}}%
\pgfpathlineto{\pgfqpoint{2.180624in}{2.545427in}}%
\pgfpathlineto{\pgfqpoint{2.174744in}{2.553621in}}%
\pgfpathlineto{\pgfqpoint{2.168869in}{2.561800in}}%
\pgfpathlineto{\pgfqpoint{2.162998in}{2.569963in}}%
\pgfpathlineto{\pgfqpoint{2.157131in}{2.578111in}}%
\pgfpathlineto{\pgfqpoint{2.145668in}{2.574843in}}%
\pgfpathlineto{\pgfqpoint{2.134200in}{2.571572in}}%
\pgfpathlineto{\pgfqpoint{2.122726in}{2.568295in}}%
\pgfpathlineto{\pgfqpoint{2.111247in}{2.565011in}}%
\pgfpathlineto{\pgfqpoint{2.099762in}{2.561719in}}%
\pgfpathlineto{\pgfqpoint{2.105617in}{2.553568in}}%
\pgfpathlineto{\pgfqpoint{2.111476in}{2.545403in}}%
\pgfpathlineto{\pgfqpoint{2.117339in}{2.537224in}}%
\pgfpathlineto{\pgfqpoint{2.123207in}{2.529029in}}%
\pgfpathclose%
\pgfusepath{stroke,fill}%
\end{pgfscope}%
\begin{pgfscope}%
\pgfpathrectangle{\pgfqpoint{0.887500in}{0.275000in}}{\pgfqpoint{4.225000in}{4.225000in}}%
\pgfusepath{clip}%
\pgfsetbuttcap%
\pgfsetroundjoin%
\definecolor{currentfill}{rgb}{0.395174,0.797475,0.367757}%
\pgfsetfillcolor{currentfill}%
\pgfsetfillopacity{0.700000}%
\pgfsetlinewidth{0.501875pt}%
\definecolor{currentstroke}{rgb}{1.000000,1.000000,1.000000}%
\pgfsetstrokecolor{currentstroke}%
\pgfsetstrokeopacity{0.500000}%
\pgfsetdash{}{0pt}%
\pgfpathmoveto{\pgfqpoint{3.847007in}{3.065873in}}%
\pgfpathlineto{\pgfqpoint{3.858087in}{3.069364in}}%
\pgfpathlineto{\pgfqpoint{3.869161in}{3.072865in}}%
\pgfpathlineto{\pgfqpoint{3.880230in}{3.076374in}}%
\pgfpathlineto{\pgfqpoint{3.891294in}{3.079875in}}%
\pgfpathlineto{\pgfqpoint{3.902352in}{3.083355in}}%
\pgfpathlineto{\pgfqpoint{3.895994in}{3.095891in}}%
\pgfpathlineto{\pgfqpoint{3.889635in}{3.108265in}}%
\pgfpathlineto{\pgfqpoint{3.883277in}{3.120504in}}%
\pgfpathlineto{\pgfqpoint{3.876919in}{3.132634in}}%
\pgfpathlineto{\pgfqpoint{3.870563in}{3.144683in}}%
\pgfpathlineto{\pgfqpoint{3.859508in}{3.141151in}}%
\pgfpathlineto{\pgfqpoint{3.848449in}{3.137622in}}%
\pgfpathlineto{\pgfqpoint{3.837384in}{3.134113in}}%
\pgfpathlineto{\pgfqpoint{3.826315in}{3.130642in}}%
\pgfpathlineto{\pgfqpoint{3.815242in}{3.127210in}}%
\pgfpathlineto{\pgfqpoint{3.821593in}{3.115164in}}%
\pgfpathlineto{\pgfqpoint{3.827946in}{3.103036in}}%
\pgfpathlineto{\pgfqpoint{3.834300in}{3.090799in}}%
\pgfpathlineto{\pgfqpoint{3.840654in}{3.078421in}}%
\pgfpathclose%
\pgfusepath{stroke,fill}%
\end{pgfscope}%
\begin{pgfscope}%
\pgfpathrectangle{\pgfqpoint{0.887500in}{0.275000in}}{\pgfqpoint{4.225000in}{4.225000in}}%
\pgfusepath{clip}%
\pgfsetbuttcap%
\pgfsetroundjoin%
\definecolor{currentfill}{rgb}{0.606045,0.850733,0.236712}%
\pgfsetfillcolor{currentfill}%
\pgfsetfillopacity{0.700000}%
\pgfsetlinewidth{0.501875pt}%
\definecolor{currentstroke}{rgb}{1.000000,1.000000,1.000000}%
\pgfsetstrokecolor{currentstroke}%
\pgfsetstrokeopacity{0.500000}%
\pgfsetdash{}{0pt}%
\pgfpathmoveto{\pgfqpoint{3.498051in}{3.237716in}}%
\pgfpathlineto{\pgfqpoint{3.509210in}{3.240795in}}%
\pgfpathlineto{\pgfqpoint{3.520366in}{3.244102in}}%
\pgfpathlineto{\pgfqpoint{3.531519in}{3.247604in}}%
\pgfpathlineto{\pgfqpoint{3.542669in}{3.251253in}}%
\pgfpathlineto{\pgfqpoint{3.553814in}{3.255001in}}%
\pgfpathlineto{\pgfqpoint{3.547516in}{3.266193in}}%
\pgfpathlineto{\pgfqpoint{3.541221in}{3.277349in}}%
\pgfpathlineto{\pgfqpoint{3.534929in}{3.288480in}}%
\pgfpathlineto{\pgfqpoint{3.528640in}{3.299596in}}%
\pgfpathlineto{\pgfqpoint{3.522354in}{3.310706in}}%
\pgfpathlineto{\pgfqpoint{3.511215in}{3.306971in}}%
\pgfpathlineto{\pgfqpoint{3.500072in}{3.303304in}}%
\pgfpathlineto{\pgfqpoint{3.488925in}{3.299744in}}%
\pgfpathlineto{\pgfqpoint{3.477774in}{3.296329in}}%
\pgfpathlineto{\pgfqpoint{3.466620in}{3.293087in}}%
\pgfpathlineto{\pgfqpoint{3.472900in}{3.281992in}}%
\pgfpathlineto{\pgfqpoint{3.479183in}{3.270907in}}%
\pgfpathlineto{\pgfqpoint{3.485469in}{3.259831in}}%
\pgfpathlineto{\pgfqpoint{3.491758in}{3.248767in}}%
\pgfpathclose%
\pgfusepath{stroke,fill}%
\end{pgfscope}%
\begin{pgfscope}%
\pgfpathrectangle{\pgfqpoint{0.887500in}{0.275000in}}{\pgfqpoint{4.225000in}{4.225000in}}%
\pgfusepath{clip}%
\pgfsetbuttcap%
\pgfsetroundjoin%
\definecolor{currentfill}{rgb}{0.137339,0.662252,0.515571}%
\pgfsetfillcolor{currentfill}%
\pgfsetfillopacity{0.700000}%
\pgfsetlinewidth{0.501875pt}%
\definecolor{currentstroke}{rgb}{1.000000,1.000000,1.000000}%
\pgfsetstrokecolor{currentstroke}%
\pgfsetstrokeopacity{0.500000}%
\pgfsetdash{}{0pt}%
\pgfpathmoveto{\pgfqpoint{4.313933in}{2.741646in}}%
\pgfpathlineto{\pgfqpoint{4.324895in}{2.745083in}}%
\pgfpathlineto{\pgfqpoint{4.335852in}{2.748525in}}%
\pgfpathlineto{\pgfqpoint{4.346805in}{2.751974in}}%
\pgfpathlineto{\pgfqpoint{4.357752in}{2.755430in}}%
\pgfpathlineto{\pgfqpoint{4.351344in}{2.769580in}}%
\pgfpathlineto{\pgfqpoint{4.344937in}{2.783708in}}%
\pgfpathlineto{\pgfqpoint{4.338532in}{2.797817in}}%
\pgfpathlineto{\pgfqpoint{4.332129in}{2.811910in}}%
\pgfpathlineto{\pgfqpoint{4.325728in}{2.825989in}}%
\pgfpathlineto{\pgfqpoint{4.314780in}{2.822458in}}%
\pgfpathlineto{\pgfqpoint{4.303827in}{2.818941in}}%
\pgfpathlineto{\pgfqpoint{4.292869in}{2.815435in}}%
\pgfpathlineto{\pgfqpoint{4.281907in}{2.811940in}}%
\pgfpathlineto{\pgfqpoint{4.288307in}{2.797900in}}%
\pgfpathlineto{\pgfqpoint{4.294710in}{2.783855in}}%
\pgfpathlineto{\pgfqpoint{4.301116in}{2.769800in}}%
\pgfpathlineto{\pgfqpoint{4.307523in}{2.755732in}}%
\pgfpathclose%
\pgfusepath{stroke,fill}%
\end{pgfscope}%
\begin{pgfscope}%
\pgfpathrectangle{\pgfqpoint{0.887500in}{0.275000in}}{\pgfqpoint{4.225000in}{4.225000in}}%
\pgfusepath{clip}%
\pgfsetbuttcap%
\pgfsetroundjoin%
\definecolor{currentfill}{rgb}{0.449368,0.813768,0.335384}%
\pgfsetfillcolor{currentfill}%
\pgfsetfillopacity{0.700000}%
\pgfsetlinewidth{0.501875pt}%
\definecolor{currentstroke}{rgb}{1.000000,1.000000,1.000000}%
\pgfsetstrokecolor{currentstroke}%
\pgfsetstrokeopacity{0.500000}%
\pgfsetdash{}{0pt}%
\pgfpathmoveto{\pgfqpoint{3.759801in}{3.110313in}}%
\pgfpathlineto{\pgfqpoint{3.770899in}{3.113690in}}%
\pgfpathlineto{\pgfqpoint{3.781993in}{3.117059in}}%
\pgfpathlineto{\pgfqpoint{3.793081in}{3.120429in}}%
\pgfpathlineto{\pgfqpoint{3.804164in}{3.123809in}}%
\pgfpathlineto{\pgfqpoint{3.815242in}{3.127210in}}%
\pgfpathlineto{\pgfqpoint{3.808892in}{3.139207in}}%
\pgfpathlineto{\pgfqpoint{3.802546in}{3.151182in}}%
\pgfpathlineto{\pgfqpoint{3.796203in}{3.163167in}}%
\pgfpathlineto{\pgfqpoint{3.789863in}{3.175191in}}%
\pgfpathlineto{\pgfqpoint{3.783528in}{3.187283in}}%
\pgfpathlineto{\pgfqpoint{3.772456in}{3.183953in}}%
\pgfpathlineto{\pgfqpoint{3.761379in}{3.180659in}}%
\pgfpathlineto{\pgfqpoint{3.750298in}{3.177388in}}%
\pgfpathlineto{\pgfqpoint{3.739212in}{3.174129in}}%
\pgfpathlineto{\pgfqpoint{3.728120in}{3.170871in}}%
\pgfpathlineto{\pgfqpoint{3.734449in}{3.158726in}}%
\pgfpathlineto{\pgfqpoint{3.740783in}{3.146622in}}%
\pgfpathlineto{\pgfqpoint{3.747119in}{3.134534in}}%
\pgfpathlineto{\pgfqpoint{3.753459in}{3.122439in}}%
\pgfpathclose%
\pgfusepath{stroke,fill}%
\end{pgfscope}%
\begin{pgfscope}%
\pgfpathrectangle{\pgfqpoint{0.887500in}{0.275000in}}{\pgfqpoint{4.225000in}{4.225000in}}%
\pgfusepath{clip}%
\pgfsetbuttcap%
\pgfsetroundjoin%
\definecolor{currentfill}{rgb}{0.143343,0.522773,0.556295}%
\pgfsetfillcolor{currentfill}%
\pgfsetfillopacity{0.700000}%
\pgfsetlinewidth{0.501875pt}%
\definecolor{currentstroke}{rgb}{1.000000,1.000000,1.000000}%
\pgfsetstrokecolor{currentstroke}%
\pgfsetstrokeopacity{0.500000}%
\pgfsetdash{}{0pt}%
\pgfpathmoveto{\pgfqpoint{2.447211in}{2.463042in}}%
\pgfpathlineto{\pgfqpoint{2.458629in}{2.466496in}}%
\pgfpathlineto{\pgfqpoint{2.470042in}{2.469878in}}%
\pgfpathlineto{\pgfqpoint{2.481451in}{2.473161in}}%
\pgfpathlineto{\pgfqpoint{2.492856in}{2.476320in}}%
\pgfpathlineto{\pgfqpoint{2.504257in}{2.479377in}}%
\pgfpathlineto{\pgfqpoint{2.498264in}{2.487825in}}%
\pgfpathlineto{\pgfqpoint{2.492275in}{2.496253in}}%
\pgfpathlineto{\pgfqpoint{2.486291in}{2.504660in}}%
\pgfpathlineto{\pgfqpoint{2.480311in}{2.513044in}}%
\pgfpathlineto{\pgfqpoint{2.474334in}{2.521401in}}%
\pgfpathlineto{\pgfqpoint{2.462945in}{2.518351in}}%
\pgfpathlineto{\pgfqpoint{2.451551in}{2.515195in}}%
\pgfpathlineto{\pgfqpoint{2.440154in}{2.511912in}}%
\pgfpathlineto{\pgfqpoint{2.428752in}{2.508529in}}%
\pgfpathlineto{\pgfqpoint{2.417346in}{2.505072in}}%
\pgfpathlineto{\pgfqpoint{2.423311in}{2.496682in}}%
\pgfpathlineto{\pgfqpoint{2.429280in}{2.488285in}}%
\pgfpathlineto{\pgfqpoint{2.435253in}{2.479880in}}%
\pgfpathlineto{\pgfqpoint{2.441230in}{2.471466in}}%
\pgfpathclose%
\pgfusepath{stroke,fill}%
\end{pgfscope}%
\begin{pgfscope}%
\pgfpathrectangle{\pgfqpoint{0.887500in}{0.275000in}}{\pgfqpoint{4.225000in}{4.225000in}}%
\pgfusepath{clip}%
\pgfsetbuttcap%
\pgfsetroundjoin%
\definecolor{currentfill}{rgb}{0.496615,0.826376,0.306377}%
\pgfsetfillcolor{currentfill}%
\pgfsetfillopacity{0.700000}%
\pgfsetlinewidth{0.501875pt}%
\definecolor{currentstroke}{rgb}{1.000000,1.000000,1.000000}%
\pgfsetstrokecolor{currentstroke}%
\pgfsetstrokeopacity{0.500000}%
\pgfsetdash{}{0pt}%
\pgfpathmoveto{\pgfqpoint{3.672580in}{3.154570in}}%
\pgfpathlineto{\pgfqpoint{3.683698in}{3.157799in}}%
\pgfpathlineto{\pgfqpoint{3.694812in}{3.161055in}}%
\pgfpathlineto{\pgfqpoint{3.705920in}{3.164326in}}%
\pgfpathlineto{\pgfqpoint{3.717022in}{3.167602in}}%
\pgfpathlineto{\pgfqpoint{3.728120in}{3.170871in}}%
\pgfpathlineto{\pgfqpoint{3.721794in}{3.183064in}}%
\pgfpathlineto{\pgfqpoint{3.715472in}{3.195284in}}%
\pgfpathlineto{\pgfqpoint{3.709152in}{3.207506in}}%
\pgfpathlineto{\pgfqpoint{3.702835in}{3.219705in}}%
\pgfpathlineto{\pgfqpoint{3.696521in}{3.231855in}}%
\pgfpathlineto{\pgfqpoint{3.685427in}{3.228542in}}%
\pgfpathlineto{\pgfqpoint{3.674329in}{3.225212in}}%
\pgfpathlineto{\pgfqpoint{3.663224in}{3.221867in}}%
\pgfpathlineto{\pgfqpoint{3.652114in}{3.218508in}}%
\pgfpathlineto{\pgfqpoint{3.640999in}{3.215137in}}%
\pgfpathlineto{\pgfqpoint{3.647311in}{3.203127in}}%
\pgfpathlineto{\pgfqpoint{3.653625in}{3.191051in}}%
\pgfpathlineto{\pgfqpoint{3.659941in}{3.178923in}}%
\pgfpathlineto{\pgfqpoint{3.666259in}{3.166758in}}%
\pgfpathclose%
\pgfusepath{stroke,fill}%
\end{pgfscope}%
\begin{pgfscope}%
\pgfpathrectangle{\pgfqpoint{0.887500in}{0.275000in}}{\pgfqpoint{4.225000in}{4.225000in}}%
\pgfusepath{clip}%
\pgfsetbuttcap%
\pgfsetroundjoin%
\definecolor{currentfill}{rgb}{0.555484,0.840254,0.269281}%
\pgfsetfillcolor{currentfill}%
\pgfsetfillopacity{0.700000}%
\pgfsetlinewidth{0.501875pt}%
\definecolor{currentstroke}{rgb}{1.000000,1.000000,1.000000}%
\pgfsetstrokecolor{currentstroke}%
\pgfsetstrokeopacity{0.500000}%
\pgfsetdash{}{0pt}%
\pgfpathmoveto{\pgfqpoint{3.585342in}{3.198183in}}%
\pgfpathlineto{\pgfqpoint{3.596484in}{3.201561in}}%
\pgfpathlineto{\pgfqpoint{3.607620in}{3.204959in}}%
\pgfpathlineto{\pgfqpoint{3.618752in}{3.208360in}}%
\pgfpathlineto{\pgfqpoint{3.629878in}{3.211754in}}%
\pgfpathlineto{\pgfqpoint{3.640999in}{3.215137in}}%
\pgfpathlineto{\pgfqpoint{3.634689in}{3.227065in}}%
\pgfpathlineto{\pgfqpoint{3.628381in}{3.238898in}}%
\pgfpathlineto{\pgfqpoint{3.622074in}{3.250620in}}%
\pgfpathlineto{\pgfqpoint{3.615769in}{3.262217in}}%
\pgfpathlineto{\pgfqpoint{3.609465in}{3.273675in}}%
\pgfpathlineto{\pgfqpoint{3.598346in}{3.270058in}}%
\pgfpathlineto{\pgfqpoint{3.587222in}{3.266370in}}%
\pgfpathlineto{\pgfqpoint{3.576091in}{3.262607in}}%
\pgfpathlineto{\pgfqpoint{3.564955in}{3.258802in}}%
\pgfpathlineto{\pgfqpoint{3.553814in}{3.255001in}}%
\pgfpathlineto{\pgfqpoint{3.560115in}{3.243764in}}%
\pgfpathlineto{\pgfqpoint{3.566418in}{3.232473in}}%
\pgfpathlineto{\pgfqpoint{3.572724in}{3.221118in}}%
\pgfpathlineto{\pgfqpoint{3.579032in}{3.209691in}}%
\pgfpathclose%
\pgfusepath{stroke,fill}%
\end{pgfscope}%
\begin{pgfscope}%
\pgfpathrectangle{\pgfqpoint{0.887500in}{0.275000in}}{\pgfqpoint{4.225000in}{4.225000in}}%
\pgfusepath{clip}%
\pgfsetbuttcap%
\pgfsetroundjoin%
\definecolor{currentfill}{rgb}{0.121148,0.592739,0.544641}%
\pgfsetfillcolor{currentfill}%
\pgfsetfillopacity{0.700000}%
\pgfsetlinewidth{0.501875pt}%
\definecolor{currentstroke}{rgb}{1.000000,1.000000,1.000000}%
\pgfsetstrokecolor{currentstroke}%
\pgfsetstrokeopacity{0.500000}%
\pgfsetdash{}{0pt}%
\pgfpathmoveto{\pgfqpoint{1.579754in}{2.606999in}}%
\pgfpathlineto{\pgfqpoint{1.591387in}{2.610280in}}%
\pgfpathlineto{\pgfqpoint{1.603014in}{2.613557in}}%
\pgfpathlineto{\pgfqpoint{1.614636in}{2.616833in}}%
\pgfpathlineto{\pgfqpoint{1.626253in}{2.620108in}}%
\pgfpathlineto{\pgfqpoint{1.637863in}{2.623383in}}%
\pgfpathlineto{\pgfqpoint{1.632174in}{2.631189in}}%
\pgfpathlineto{\pgfqpoint{1.626490in}{2.638973in}}%
\pgfpathlineto{\pgfqpoint{1.620810in}{2.646735in}}%
\pgfpathlineto{\pgfqpoint{1.615135in}{2.654478in}}%
\pgfpathlineto{\pgfqpoint{1.609464in}{2.662200in}}%
\pgfpathlineto{\pgfqpoint{1.597867in}{2.658922in}}%
\pgfpathlineto{\pgfqpoint{1.586264in}{2.655642in}}%
\pgfpathlineto{\pgfqpoint{1.574656in}{2.652362in}}%
\pgfpathlineto{\pgfqpoint{1.563042in}{2.649079in}}%
\pgfpathlineto{\pgfqpoint{1.551423in}{2.645794in}}%
\pgfpathlineto{\pgfqpoint{1.557080in}{2.638078in}}%
\pgfpathlineto{\pgfqpoint{1.562742in}{2.630343in}}%
\pgfpathlineto{\pgfqpoint{1.568408in}{2.622585in}}%
\pgfpathlineto{\pgfqpoint{1.574078in}{2.614804in}}%
\pgfpathclose%
\pgfusepath{stroke,fill}%
\end{pgfscope}%
\begin{pgfscope}%
\pgfpathrectangle{\pgfqpoint{0.887500in}{0.275000in}}{\pgfqpoint{4.225000in}{4.225000in}}%
\pgfusepath{clip}%
\pgfsetbuttcap%
\pgfsetroundjoin%
\definecolor{currentfill}{rgb}{0.162016,0.687316,0.499129}%
\pgfsetfillcolor{currentfill}%
\pgfsetfillopacity{0.700000}%
\pgfsetlinewidth{0.501875pt}%
\definecolor{currentstroke}{rgb}{1.000000,1.000000,1.000000}%
\pgfsetstrokecolor{currentstroke}%
\pgfsetstrokeopacity{0.500000}%
\pgfsetdash{}{0pt}%
\pgfpathmoveto{\pgfqpoint{4.227020in}{2.794587in}}%
\pgfpathlineto{\pgfqpoint{4.238007in}{2.798046in}}%
\pgfpathlineto{\pgfqpoint{4.248990in}{2.801509in}}%
\pgfpathlineto{\pgfqpoint{4.259967in}{2.804979in}}%
\pgfpathlineto{\pgfqpoint{4.270939in}{2.808455in}}%
\pgfpathlineto{\pgfqpoint{4.281907in}{2.811940in}}%
\pgfpathlineto{\pgfqpoint{4.275508in}{2.825981in}}%
\pgfpathlineto{\pgfqpoint{4.269113in}{2.840025in}}%
\pgfpathlineto{\pgfqpoint{4.262720in}{2.854078in}}%
\pgfpathlineto{\pgfqpoint{4.256330in}{2.868143in}}%
\pgfpathlineto{\pgfqpoint{4.249943in}{2.882211in}}%
\pgfpathlineto{\pgfqpoint{4.238977in}{2.878713in}}%
\pgfpathlineto{\pgfqpoint{4.228006in}{2.875218in}}%
\pgfpathlineto{\pgfqpoint{4.217030in}{2.871727in}}%
\pgfpathlineto{\pgfqpoint{4.206048in}{2.868238in}}%
\pgfpathlineto{\pgfqpoint{4.195062in}{2.864749in}}%
\pgfpathlineto{\pgfqpoint{4.201448in}{2.850693in}}%
\pgfpathlineto{\pgfqpoint{4.207837in}{2.836645in}}%
\pgfpathlineto{\pgfqpoint{4.214228in}{2.822614in}}%
\pgfpathlineto{\pgfqpoint{4.220623in}{2.808597in}}%
\pgfpathclose%
\pgfusepath{stroke,fill}%
\end{pgfscope}%
\begin{pgfscope}%
\pgfpathrectangle{\pgfqpoint{0.887500in}{0.275000in}}{\pgfqpoint{4.225000in}{4.225000in}}%
\pgfusepath{clip}%
\pgfsetbuttcap%
\pgfsetroundjoin%
\definecolor{currentfill}{rgb}{0.153364,0.497000,0.557724}%
\pgfsetfillcolor{currentfill}%
\pgfsetfillopacity{0.700000}%
\pgfsetlinewidth{0.501875pt}%
\definecolor{currentstroke}{rgb}{1.000000,1.000000,1.000000}%
\pgfsetstrokecolor{currentstroke}%
\pgfsetstrokeopacity{0.500000}%
\pgfsetdash{}{0pt}%
\pgfpathmoveto{\pgfqpoint{2.765520in}{2.388329in}}%
\pgfpathlineto{\pgfqpoint{2.776881in}{2.388366in}}%
\pgfpathlineto{\pgfqpoint{2.788214in}{2.391420in}}%
\pgfpathlineto{\pgfqpoint{2.799513in}{2.398858in}}%
\pgfpathlineto{\pgfqpoint{2.810773in}{2.412053in}}%
\pgfpathlineto{\pgfqpoint{2.821992in}{2.432148in}}%
\pgfpathlineto{\pgfqpoint{2.815872in}{2.443962in}}%
\pgfpathlineto{\pgfqpoint{2.809752in}{2.456496in}}%
\pgfpathlineto{\pgfqpoint{2.803630in}{2.469610in}}%
\pgfpathlineto{\pgfqpoint{2.797508in}{2.483165in}}%
\pgfpathlineto{\pgfqpoint{2.791386in}{2.497020in}}%
\pgfpathlineto{\pgfqpoint{2.780168in}{2.479386in}}%
\pgfpathlineto{\pgfqpoint{2.768917in}{2.466483in}}%
\pgfpathlineto{\pgfqpoint{2.757637in}{2.457562in}}%
\pgfpathlineto{\pgfqpoint{2.746329in}{2.451729in}}%
\pgfpathlineto{\pgfqpoint{2.735000in}{2.448090in}}%
\pgfpathlineto{\pgfqpoint{2.741105in}{2.435302in}}%
\pgfpathlineto{\pgfqpoint{2.747210in}{2.422814in}}%
\pgfpathlineto{\pgfqpoint{2.753315in}{2.410744in}}%
\pgfpathlineto{\pgfqpoint{2.759418in}{2.399210in}}%
\pgfpathclose%
\pgfusepath{stroke,fill}%
\end{pgfscope}%
\begin{pgfscope}%
\pgfpathrectangle{\pgfqpoint{0.887500in}{0.275000in}}{\pgfqpoint{4.225000in}{4.225000in}}%
\pgfusepath{clip}%
\pgfsetbuttcap%
\pgfsetroundjoin%
\definecolor{currentfill}{rgb}{0.127568,0.566949,0.550556}%
\pgfsetfillcolor{currentfill}%
\pgfsetfillopacity{0.700000}%
\pgfsetlinewidth{0.501875pt}%
\definecolor{currentstroke}{rgb}{1.000000,1.000000,1.000000}%
\pgfsetstrokecolor{currentstroke}%
\pgfsetstrokeopacity{0.500000}%
\pgfsetdash{}{0pt}%
\pgfpathmoveto{\pgfqpoint{1.897791in}{2.552712in}}%
\pgfpathlineto{\pgfqpoint{1.909350in}{2.555962in}}%
\pgfpathlineto{\pgfqpoint{1.920902in}{2.559220in}}%
\pgfpathlineto{\pgfqpoint{1.932449in}{2.562487in}}%
\pgfpathlineto{\pgfqpoint{1.943989in}{2.565766in}}%
\pgfpathlineto{\pgfqpoint{1.955524in}{2.569057in}}%
\pgfpathlineto{\pgfqpoint{1.949719in}{2.577124in}}%
\pgfpathlineto{\pgfqpoint{1.943918in}{2.585178in}}%
\pgfpathlineto{\pgfqpoint{1.938121in}{2.593220in}}%
\pgfpathlineto{\pgfqpoint{1.932328in}{2.601251in}}%
\pgfpathlineto{\pgfqpoint{1.926539in}{2.609269in}}%
\pgfpathlineto{\pgfqpoint{1.915018in}{2.605980in}}%
\pgfpathlineto{\pgfqpoint{1.903490in}{2.602705in}}%
\pgfpathlineto{\pgfqpoint{1.891956in}{2.599442in}}%
\pgfpathlineto{\pgfqpoint{1.880416in}{2.596188in}}%
\pgfpathlineto{\pgfqpoint{1.868870in}{2.592941in}}%
\pgfpathlineto{\pgfqpoint{1.874646in}{2.584922in}}%
\pgfpathlineto{\pgfqpoint{1.880426in}{2.576889in}}%
\pgfpathlineto{\pgfqpoint{1.886210in}{2.568842in}}%
\pgfpathlineto{\pgfqpoint{1.891999in}{2.560784in}}%
\pgfpathclose%
\pgfusepath{stroke,fill}%
\end{pgfscope}%
\begin{pgfscope}%
\pgfpathrectangle{\pgfqpoint{0.887500in}{0.275000in}}{\pgfqpoint{4.225000in}{4.225000in}}%
\pgfusepath{clip}%
\pgfsetbuttcap%
\pgfsetroundjoin%
\definecolor{currentfill}{rgb}{0.136408,0.541173,0.554483}%
\pgfsetfillcolor{currentfill}%
\pgfsetfillopacity{0.700000}%
\pgfsetlinewidth{0.501875pt}%
\definecolor{currentstroke}{rgb}{1.000000,1.000000,1.000000}%
\pgfsetstrokecolor{currentstroke}%
\pgfsetstrokeopacity{0.500000}%
\pgfsetdash{}{0pt}%
\pgfpathmoveto{\pgfqpoint{2.215988in}{2.495940in}}%
\pgfpathlineto{\pgfqpoint{2.227469in}{2.499214in}}%
\pgfpathlineto{\pgfqpoint{2.238945in}{2.502490in}}%
\pgfpathlineto{\pgfqpoint{2.250414in}{2.505771in}}%
\pgfpathlineto{\pgfqpoint{2.261878in}{2.509059in}}%
\pgfpathlineto{\pgfqpoint{2.273336in}{2.512356in}}%
\pgfpathlineto{\pgfqpoint{2.267420in}{2.520640in}}%
\pgfpathlineto{\pgfqpoint{2.261508in}{2.528910in}}%
\pgfpathlineto{\pgfqpoint{2.255599in}{2.537167in}}%
\pgfpathlineto{\pgfqpoint{2.249695in}{2.545410in}}%
\pgfpathlineto{\pgfqpoint{2.243795in}{2.553641in}}%
\pgfpathlineto{\pgfqpoint{2.232349in}{2.550343in}}%
\pgfpathlineto{\pgfqpoint{2.220898in}{2.547053in}}%
\pgfpathlineto{\pgfqpoint{2.209440in}{2.543771in}}%
\pgfpathlineto{\pgfqpoint{2.197977in}{2.540493in}}%
\pgfpathlineto{\pgfqpoint{2.186508in}{2.537218in}}%
\pgfpathlineto{\pgfqpoint{2.192395in}{2.528992in}}%
\pgfpathlineto{\pgfqpoint{2.198287in}{2.520752in}}%
\pgfpathlineto{\pgfqpoint{2.204183in}{2.512496in}}%
\pgfpathlineto{\pgfqpoint{2.210084in}{2.504226in}}%
\pgfpathclose%
\pgfusepath{stroke,fill}%
\end{pgfscope}%
\begin{pgfscope}%
\pgfpathrectangle{\pgfqpoint{0.887500in}{0.275000in}}{\pgfqpoint{4.225000in}{4.225000in}}%
\pgfusepath{clip}%
\pgfsetbuttcap%
\pgfsetroundjoin%
\definecolor{currentfill}{rgb}{0.196571,0.711827,0.479221}%
\pgfsetfillcolor{currentfill}%
\pgfsetfillopacity{0.700000}%
\pgfsetlinewidth{0.501875pt}%
\definecolor{currentstroke}{rgb}{1.000000,1.000000,1.000000}%
\pgfsetstrokecolor{currentstroke}%
\pgfsetstrokeopacity{0.500000}%
\pgfsetdash{}{0pt}%
\pgfpathmoveto{\pgfqpoint{4.140052in}{2.847275in}}%
\pgfpathlineto{\pgfqpoint{4.151065in}{2.850778in}}%
\pgfpathlineto{\pgfqpoint{4.162072in}{2.854276in}}%
\pgfpathlineto{\pgfqpoint{4.173074in}{2.857770in}}%
\pgfpathlineto{\pgfqpoint{4.184071in}{2.861260in}}%
\pgfpathlineto{\pgfqpoint{4.195062in}{2.864749in}}%
\pgfpathlineto{\pgfqpoint{4.188679in}{2.878800in}}%
\pgfpathlineto{\pgfqpoint{4.182297in}{2.892828in}}%
\pgfpathlineto{\pgfqpoint{4.175917in}{2.906819in}}%
\pgfpathlineto{\pgfqpoint{4.169537in}{2.920757in}}%
\pgfpathlineto{\pgfqpoint{4.163159in}{2.934626in}}%
\pgfpathlineto{\pgfqpoint{4.152169in}{2.931114in}}%
\pgfpathlineto{\pgfqpoint{4.141173in}{2.927590in}}%
\pgfpathlineto{\pgfqpoint{4.130172in}{2.924054in}}%
\pgfpathlineto{\pgfqpoint{4.119166in}{2.920504in}}%
\pgfpathlineto{\pgfqpoint{4.108153in}{2.916939in}}%
\pgfpathlineto{\pgfqpoint{4.114530in}{2.903085in}}%
\pgfpathlineto{\pgfqpoint{4.120908in}{2.889181in}}%
\pgfpathlineto{\pgfqpoint{4.127288in}{2.875237in}}%
\pgfpathlineto{\pgfqpoint{4.133669in}{2.861265in}}%
\pgfpathclose%
\pgfusepath{stroke,fill}%
\end{pgfscope}%
\begin{pgfscope}%
\pgfpathrectangle{\pgfqpoint{0.887500in}{0.275000in}}{\pgfqpoint{4.225000in}{4.225000in}}%
\pgfusepath{clip}%
\pgfsetbuttcap%
\pgfsetroundjoin%
\definecolor{currentfill}{rgb}{0.146180,0.515413,0.556823}%
\pgfsetfillcolor{currentfill}%
\pgfsetfillopacity{0.700000}%
\pgfsetlinewidth{0.501875pt}%
\definecolor{currentstroke}{rgb}{1.000000,1.000000,1.000000}%
\pgfsetstrokecolor{currentstroke}%
\pgfsetstrokeopacity{0.500000}%
\pgfsetdash{}{0pt}%
\pgfpathmoveto{\pgfqpoint{2.534282in}{2.436919in}}%
\pgfpathlineto{\pgfqpoint{2.545690in}{2.439856in}}%
\pgfpathlineto{\pgfqpoint{2.557091in}{2.442857in}}%
\pgfpathlineto{\pgfqpoint{2.568484in}{2.446022in}}%
\pgfpathlineto{\pgfqpoint{2.579869in}{2.449450in}}%
\pgfpathlineto{\pgfqpoint{2.591244in}{2.453241in}}%
\pgfpathlineto{\pgfqpoint{2.585219in}{2.461824in}}%
\pgfpathlineto{\pgfqpoint{2.579197in}{2.470416in}}%
\pgfpathlineto{\pgfqpoint{2.573180in}{2.479006in}}%
\pgfpathlineto{\pgfqpoint{2.567166in}{2.487581in}}%
\pgfpathlineto{\pgfqpoint{2.561157in}{2.496130in}}%
\pgfpathlineto{\pgfqpoint{2.549794in}{2.492290in}}%
\pgfpathlineto{\pgfqpoint{2.538422in}{2.488783in}}%
\pgfpathlineto{\pgfqpoint{2.527041in}{2.485520in}}%
\pgfpathlineto{\pgfqpoint{2.515652in}{2.482414in}}%
\pgfpathlineto{\pgfqpoint{2.504257in}{2.479377in}}%
\pgfpathlineto{\pgfqpoint{2.510254in}{2.470912in}}%
\pgfpathlineto{\pgfqpoint{2.516255in}{2.462431in}}%
\pgfpathlineto{\pgfqpoint{2.522260in}{2.453937in}}%
\pgfpathlineto{\pgfqpoint{2.528269in}{2.445433in}}%
\pgfpathclose%
\pgfusepath{stroke,fill}%
\end{pgfscope}%
\begin{pgfscope}%
\pgfpathrectangle{\pgfqpoint{0.887500in}{0.275000in}}{\pgfqpoint{4.225000in}{4.225000in}}%
\pgfusepath{clip}%
\pgfsetbuttcap%
\pgfsetroundjoin%
\definecolor{currentfill}{rgb}{0.122606,0.585371,0.546557}%
\pgfsetfillcolor{currentfill}%
\pgfsetfillopacity{0.700000}%
\pgfsetlinewidth{0.501875pt}%
\definecolor{currentstroke}{rgb}{1.000000,1.000000,1.000000}%
\pgfsetstrokecolor{currentstroke}%
\pgfsetstrokeopacity{0.500000}%
\pgfsetdash{}{0pt}%
\pgfpathmoveto{\pgfqpoint{1.666375in}{2.583989in}}%
\pgfpathlineto{\pgfqpoint{1.677993in}{2.587270in}}%
\pgfpathlineto{\pgfqpoint{1.689605in}{2.590549in}}%
\pgfpathlineto{\pgfqpoint{1.701212in}{2.593824in}}%
\pgfpathlineto{\pgfqpoint{1.712813in}{2.597096in}}%
\pgfpathlineto{\pgfqpoint{1.724408in}{2.600362in}}%
\pgfpathlineto{\pgfqpoint{1.718684in}{2.608280in}}%
\pgfpathlineto{\pgfqpoint{1.712963in}{2.616176in}}%
\pgfpathlineto{\pgfqpoint{1.707248in}{2.624049in}}%
\pgfpathlineto{\pgfqpoint{1.701537in}{2.631900in}}%
\pgfpathlineto{\pgfqpoint{1.695830in}{2.639729in}}%
\pgfpathlineto{\pgfqpoint{1.684248in}{2.636465in}}%
\pgfpathlineto{\pgfqpoint{1.672660in}{2.633198in}}%
\pgfpathlineto{\pgfqpoint{1.661067in}{2.629928in}}%
\pgfpathlineto{\pgfqpoint{1.649468in}{2.626656in}}%
\pgfpathlineto{\pgfqpoint{1.637863in}{2.623383in}}%
\pgfpathlineto{\pgfqpoint{1.643556in}{2.615554in}}%
\pgfpathlineto{\pgfqpoint{1.649254in}{2.607700in}}%
\pgfpathlineto{\pgfqpoint{1.654957in}{2.599821in}}%
\pgfpathlineto{\pgfqpoint{1.660664in}{2.591917in}}%
\pgfpathclose%
\pgfusepath{stroke,fill}%
\end{pgfscope}%
\begin{pgfscope}%
\pgfpathrectangle{\pgfqpoint{0.887500in}{0.275000in}}{\pgfqpoint{4.225000in}{4.225000in}}%
\pgfusepath{clip}%
\pgfsetbuttcap%
\pgfsetroundjoin%
\definecolor{currentfill}{rgb}{0.783315,0.879285,0.125405}%
\pgfsetfillcolor{currentfill}%
\pgfsetfillopacity{0.700000}%
\pgfsetlinewidth{0.501875pt}%
\definecolor{currentstroke}{rgb}{1.000000,1.000000,1.000000}%
\pgfsetstrokecolor{currentstroke}%
\pgfsetstrokeopacity{0.500000}%
\pgfsetdash{}{0pt}%
\pgfpathmoveto{\pgfqpoint{3.092699in}{3.373932in}}%
\pgfpathlineto{\pgfqpoint{3.103968in}{3.378009in}}%
\pgfpathlineto{\pgfqpoint{3.115230in}{3.381087in}}%
\pgfpathlineto{\pgfqpoint{3.126486in}{3.383438in}}%
\pgfpathlineto{\pgfqpoint{3.137735in}{3.385335in}}%
\pgfpathlineto{\pgfqpoint{3.148977in}{3.387050in}}%
\pgfpathlineto{\pgfqpoint{3.142773in}{3.396352in}}%
\pgfpathlineto{\pgfqpoint{3.136572in}{3.405398in}}%
\pgfpathlineto{\pgfqpoint{3.130375in}{3.414159in}}%
\pgfpathlineto{\pgfqpoint{3.124180in}{3.422604in}}%
\pgfpathlineto{\pgfqpoint{3.117989in}{3.430702in}}%
\pgfpathlineto{\pgfqpoint{3.106756in}{3.428938in}}%
\pgfpathlineto{\pgfqpoint{3.095516in}{3.426933in}}%
\pgfpathlineto{\pgfqpoint{3.084269in}{3.424411in}}%
\pgfpathlineto{\pgfqpoint{3.073016in}{3.421099in}}%
\pgfpathlineto{\pgfqpoint{3.061758in}{3.416722in}}%
\pgfpathlineto{\pgfqpoint{3.067939in}{3.408769in}}%
\pgfpathlineto{\pgfqpoint{3.074123in}{3.400493in}}%
\pgfpathlineto{\pgfqpoint{3.080311in}{3.391915in}}%
\pgfpathlineto{\pgfqpoint{3.086503in}{3.383055in}}%
\pgfpathclose%
\pgfusepath{stroke,fill}%
\end{pgfscope}%
\begin{pgfscope}%
\pgfpathrectangle{\pgfqpoint{0.887500in}{0.275000in}}{\pgfqpoint{4.225000in}{4.225000in}}%
\pgfusepath{clip}%
\pgfsetbuttcap%
\pgfsetroundjoin%
\definecolor{currentfill}{rgb}{0.239374,0.735588,0.455688}%
\pgfsetfillcolor{currentfill}%
\pgfsetfillopacity{0.700000}%
\pgfsetlinewidth{0.501875pt}%
\definecolor{currentstroke}{rgb}{1.000000,1.000000,1.000000}%
\pgfsetstrokecolor{currentstroke}%
\pgfsetstrokeopacity{0.500000}%
\pgfsetdash{}{0pt}%
\pgfpathmoveto{\pgfqpoint{4.053012in}{2.898960in}}%
\pgfpathlineto{\pgfqpoint{4.064051in}{2.902566in}}%
\pgfpathlineto{\pgfqpoint{4.075084in}{2.906171in}}%
\pgfpathlineto{\pgfqpoint{4.086113in}{2.909770in}}%
\pgfpathlineto{\pgfqpoint{4.097136in}{2.913360in}}%
\pgfpathlineto{\pgfqpoint{4.108153in}{2.916939in}}%
\pgfpathlineto{\pgfqpoint{4.101778in}{2.930733in}}%
\pgfpathlineto{\pgfqpoint{4.095403in}{2.944457in}}%
\pgfpathlineto{\pgfqpoint{4.089028in}{2.958100in}}%
\pgfpathlineto{\pgfqpoint{4.082654in}{2.971651in}}%
\pgfpathlineto{\pgfqpoint{4.076280in}{2.985104in}}%
\pgfpathlineto{\pgfqpoint{4.065267in}{2.981620in}}%
\pgfpathlineto{\pgfqpoint{4.054249in}{2.978127in}}%
\pgfpathlineto{\pgfqpoint{4.043226in}{2.974621in}}%
\pgfpathlineto{\pgfqpoint{4.032197in}{2.971101in}}%
\pgfpathlineto{\pgfqpoint{4.021162in}{2.967565in}}%
\pgfpathlineto{\pgfqpoint{4.027528in}{2.953907in}}%
\pgfpathlineto{\pgfqpoint{4.033896in}{2.940214in}}%
\pgfpathlineto{\pgfqpoint{4.040266in}{2.926491in}}%
\pgfpathlineto{\pgfqpoint{4.046638in}{2.912739in}}%
\pgfpathclose%
\pgfusepath{stroke,fill}%
\end{pgfscope}%
\begin{pgfscope}%
\pgfpathrectangle{\pgfqpoint{0.887500in}{0.275000in}}{\pgfqpoint{4.225000in}{4.225000in}}%
\pgfusepath{clip}%
\pgfsetbuttcap%
\pgfsetroundjoin%
\definecolor{currentfill}{rgb}{0.129933,0.559582,0.551864}%
\pgfsetfillcolor{currentfill}%
\pgfsetfillopacity{0.700000}%
\pgfsetlinewidth{0.501875pt}%
\definecolor{currentstroke}{rgb}{1.000000,1.000000,1.000000}%
\pgfsetstrokecolor{currentstroke}%
\pgfsetstrokeopacity{0.500000}%
\pgfsetdash{}{0pt}%
\pgfpathmoveto{\pgfqpoint{1.984611in}{2.528524in}}%
\pgfpathlineto{\pgfqpoint{1.996152in}{2.531837in}}%
\pgfpathlineto{\pgfqpoint{2.007687in}{2.535159in}}%
\pgfpathlineto{\pgfqpoint{2.019216in}{2.538487in}}%
\pgfpathlineto{\pgfqpoint{2.030739in}{2.541819in}}%
\pgfpathlineto{\pgfqpoint{2.042257in}{2.545150in}}%
\pgfpathlineto{\pgfqpoint{2.036419in}{2.553279in}}%
\pgfpathlineto{\pgfqpoint{2.030585in}{2.561394in}}%
\pgfpathlineto{\pgfqpoint{2.024755in}{2.569494in}}%
\pgfpathlineto{\pgfqpoint{2.018929in}{2.577582in}}%
\pgfpathlineto{\pgfqpoint{2.013107in}{2.585657in}}%
\pgfpathlineto{\pgfqpoint{2.001602in}{2.582327in}}%
\pgfpathlineto{\pgfqpoint{1.990091in}{2.578999in}}%
\pgfpathlineto{\pgfqpoint{1.978575in}{2.575677in}}%
\pgfpathlineto{\pgfqpoint{1.967052in}{2.572362in}}%
\pgfpathlineto{\pgfqpoint{1.955524in}{2.569057in}}%
\pgfpathlineto{\pgfqpoint{1.961333in}{2.560979in}}%
\pgfpathlineto{\pgfqpoint{1.967146in}{2.552887in}}%
\pgfpathlineto{\pgfqpoint{1.972964in}{2.544781in}}%
\pgfpathlineto{\pgfqpoint{1.978785in}{2.536660in}}%
\pgfpathclose%
\pgfusepath{stroke,fill}%
\end{pgfscope}%
\begin{pgfscope}%
\pgfpathrectangle{\pgfqpoint{0.887500in}{0.275000in}}{\pgfqpoint{4.225000in}{4.225000in}}%
\pgfusepath{clip}%
\pgfsetbuttcap%
\pgfsetroundjoin%
\definecolor{currentfill}{rgb}{0.741388,0.873449,0.149561}%
\pgfsetfillcolor{currentfill}%
\pgfsetfillopacity{0.700000}%
\pgfsetlinewidth{0.501875pt}%
\definecolor{currentstroke}{rgb}{1.000000,1.000000,1.000000}%
\pgfsetstrokecolor{currentstroke}%
\pgfsetstrokeopacity{0.500000}%
\pgfsetdash{}{0pt}%
\pgfpathmoveto{\pgfqpoint{3.180045in}{3.337568in}}%
\pgfpathlineto{\pgfqpoint{3.191290in}{3.339468in}}%
\pgfpathlineto{\pgfqpoint{3.202528in}{3.341581in}}%
\pgfpathlineto{\pgfqpoint{3.213762in}{3.343902in}}%
\pgfpathlineto{\pgfqpoint{3.224990in}{3.346425in}}%
\pgfpathlineto{\pgfqpoint{3.236213in}{3.349142in}}%
\pgfpathlineto{\pgfqpoint{3.229985in}{3.359191in}}%
\pgfpathlineto{\pgfqpoint{3.223761in}{3.369140in}}%
\pgfpathlineto{\pgfqpoint{3.217539in}{3.378954in}}%
\pgfpathlineto{\pgfqpoint{3.211320in}{3.388599in}}%
\pgfpathlineto{\pgfqpoint{3.205105in}{3.398042in}}%
\pgfpathlineto{\pgfqpoint{3.193890in}{3.395432in}}%
\pgfpathlineto{\pgfqpoint{3.182670in}{3.393032in}}%
\pgfpathlineto{\pgfqpoint{3.171445in}{3.390839in}}%
\pgfpathlineto{\pgfqpoint{3.160214in}{3.388847in}}%
\pgfpathlineto{\pgfqpoint{3.148977in}{3.387050in}}%
\pgfpathlineto{\pgfqpoint{3.155185in}{3.377518in}}%
\pgfpathlineto{\pgfqpoint{3.161395in}{3.367780in}}%
\pgfpathlineto{\pgfqpoint{3.167609in}{3.357859in}}%
\pgfpathlineto{\pgfqpoint{3.173826in}{3.347780in}}%
\pgfpathclose%
\pgfusepath{stroke,fill}%
\end{pgfscope}%
\begin{pgfscope}%
\pgfpathrectangle{\pgfqpoint{0.887500in}{0.275000in}}{\pgfqpoint{4.225000in}{4.225000in}}%
\pgfusepath{clip}%
\pgfsetbuttcap%
\pgfsetroundjoin%
\definecolor{currentfill}{rgb}{0.120092,0.600104,0.542530}%
\pgfsetfillcolor{currentfill}%
\pgfsetfillopacity{0.700000}%
\pgfsetlinewidth{0.501875pt}%
\definecolor{currentstroke}{rgb}{1.000000,1.000000,1.000000}%
\pgfsetstrokecolor{currentstroke}%
\pgfsetstrokeopacity{0.500000}%
\pgfsetdash{}{0pt}%
\pgfpathmoveto{\pgfqpoint{1.434925in}{2.612596in}}%
\pgfpathlineto{\pgfqpoint{1.446599in}{2.615957in}}%
\pgfpathlineto{\pgfqpoint{1.458268in}{2.619306in}}%
\pgfpathlineto{\pgfqpoint{1.469931in}{2.622645in}}%
\pgfpathlineto{\pgfqpoint{1.481589in}{2.625974in}}%
\pgfpathlineto{\pgfqpoint{1.493242in}{2.629294in}}%
\pgfpathlineto{\pgfqpoint{1.487603in}{2.636984in}}%
\pgfpathlineto{\pgfqpoint{1.481969in}{2.644654in}}%
\pgfpathlineto{\pgfqpoint{1.476339in}{2.652308in}}%
\pgfpathlineto{\pgfqpoint{1.470713in}{2.659945in}}%
\pgfpathlineto{\pgfqpoint{1.465092in}{2.667567in}}%
\pgfpathlineto{\pgfqpoint{1.453454in}{2.664241in}}%
\pgfpathlineto{\pgfqpoint{1.441810in}{2.660908in}}%
\pgfpathlineto{\pgfqpoint{1.430160in}{2.657566in}}%
\pgfpathlineto{\pgfqpoint{1.418506in}{2.654214in}}%
\pgfpathlineto{\pgfqpoint{1.406846in}{2.650852in}}%
\pgfpathlineto{\pgfqpoint{1.412453in}{2.643233in}}%
\pgfpathlineto{\pgfqpoint{1.418065in}{2.635599in}}%
\pgfpathlineto{\pgfqpoint{1.423681in}{2.627949in}}%
\pgfpathlineto{\pgfqpoint{1.429301in}{2.620282in}}%
\pgfpathclose%
\pgfusepath{stroke,fill}%
\end{pgfscope}%
\begin{pgfscope}%
\pgfpathrectangle{\pgfqpoint{0.887500in}{0.275000in}}{\pgfqpoint{4.225000in}{4.225000in}}%
\pgfusepath{clip}%
\pgfsetbuttcap%
\pgfsetroundjoin%
\definecolor{currentfill}{rgb}{0.139147,0.533812,0.555298}%
\pgfsetfillcolor{currentfill}%
\pgfsetfillopacity{0.700000}%
\pgfsetlinewidth{0.501875pt}%
\definecolor{currentstroke}{rgb}{1.000000,1.000000,1.000000}%
\pgfsetstrokecolor{currentstroke}%
\pgfsetstrokeopacity{0.500000}%
\pgfsetdash{}{0pt}%
\pgfpathmoveto{\pgfqpoint{2.302979in}{2.470724in}}%
\pgfpathlineto{\pgfqpoint{2.314443in}{2.474047in}}%
\pgfpathlineto{\pgfqpoint{2.325901in}{2.477390in}}%
\pgfpathlineto{\pgfqpoint{2.337353in}{2.480757in}}%
\pgfpathlineto{\pgfqpoint{2.348798in}{2.484151in}}%
\pgfpathlineto{\pgfqpoint{2.360238in}{2.487574in}}%
\pgfpathlineto{\pgfqpoint{2.354289in}{2.495926in}}%
\pgfpathlineto{\pgfqpoint{2.348345in}{2.504260in}}%
\pgfpathlineto{\pgfqpoint{2.342404in}{2.512577in}}%
\pgfpathlineto{\pgfqpoint{2.336468in}{2.520879in}}%
\pgfpathlineto{\pgfqpoint{2.330536in}{2.529167in}}%
\pgfpathlineto{\pgfqpoint{2.319108in}{2.525738in}}%
\pgfpathlineto{\pgfqpoint{2.307675in}{2.522350in}}%
\pgfpathlineto{\pgfqpoint{2.296234in}{2.518995in}}%
\pgfpathlineto{\pgfqpoint{2.284788in}{2.515666in}}%
\pgfpathlineto{\pgfqpoint{2.273336in}{2.512356in}}%
\pgfpathlineto{\pgfqpoint{2.279257in}{2.504059in}}%
\pgfpathlineto{\pgfqpoint{2.285181in}{2.495747in}}%
\pgfpathlineto{\pgfqpoint{2.291110in}{2.487421in}}%
\pgfpathlineto{\pgfqpoint{2.297042in}{2.479080in}}%
\pgfpathclose%
\pgfusepath{stroke,fill}%
\end{pgfscope}%
\begin{pgfscope}%
\pgfpathrectangle{\pgfqpoint{0.887500in}{0.275000in}}{\pgfqpoint{4.225000in}{4.225000in}}%
\pgfusepath{clip}%
\pgfsetbuttcap%
\pgfsetroundjoin%
\definecolor{currentfill}{rgb}{0.288921,0.758394,0.428426}%
\pgfsetfillcolor{currentfill}%
\pgfsetfillopacity{0.700000}%
\pgfsetlinewidth{0.501875pt}%
\definecolor{currentstroke}{rgb}{1.000000,1.000000,1.000000}%
\pgfsetstrokecolor{currentstroke}%
\pgfsetstrokeopacity{0.500000}%
\pgfsetdash{}{0pt}%
\pgfpathmoveto{\pgfqpoint{3.965906in}{2.949634in}}%
\pgfpathlineto{\pgfqpoint{3.976968in}{2.953240in}}%
\pgfpathlineto{\pgfqpoint{3.988025in}{2.956844in}}%
\pgfpathlineto{\pgfqpoint{3.999076in}{2.960437in}}%
\pgfpathlineto{\pgfqpoint{4.010122in}{2.964011in}}%
\pgfpathlineto{\pgfqpoint{4.021162in}{2.967565in}}%
\pgfpathlineto{\pgfqpoint{4.014797in}{2.981179in}}%
\pgfpathlineto{\pgfqpoint{4.008434in}{2.994737in}}%
\pgfpathlineto{\pgfqpoint{4.002072in}{3.008229in}}%
\pgfpathlineto{\pgfqpoint{3.995710in}{3.021644in}}%
\pgfpathlineto{\pgfqpoint{3.989349in}{3.034972in}}%
\pgfpathlineto{\pgfqpoint{3.978318in}{3.031671in}}%
\pgfpathlineto{\pgfqpoint{3.967280in}{3.028313in}}%
\pgfpathlineto{\pgfqpoint{3.956235in}{3.024895in}}%
\pgfpathlineto{\pgfqpoint{3.945184in}{3.021428in}}%
\pgfpathlineto{\pgfqpoint{3.934127in}{3.017924in}}%
\pgfpathlineto{\pgfqpoint{3.940481in}{3.004406in}}%
\pgfpathlineto{\pgfqpoint{3.946835in}{2.990796in}}%
\pgfpathlineto{\pgfqpoint{3.953190in}{2.977116in}}%
\pgfpathlineto{\pgfqpoint{3.959547in}{2.963387in}}%
\pgfpathclose%
\pgfusepath{stroke,fill}%
\end{pgfscope}%
\begin{pgfscope}%
\pgfpathrectangle{\pgfqpoint{0.887500in}{0.275000in}}{\pgfqpoint{4.225000in}{4.225000in}}%
\pgfusepath{clip}%
\pgfsetbuttcap%
\pgfsetroundjoin%
\definecolor{currentfill}{rgb}{0.122312,0.633153,0.530398}%
\pgfsetfillcolor{currentfill}%
\pgfsetfillopacity{0.700000}%
\pgfsetlinewidth{0.501875pt}%
\definecolor{currentstroke}{rgb}{1.000000,1.000000,1.000000}%
\pgfsetstrokecolor{currentstroke}%
\pgfsetstrokeopacity{0.500000}%
\pgfsetdash{}{0pt}%
\pgfpathmoveto{\pgfqpoint{4.346004in}{2.670817in}}%
\pgfpathlineto{\pgfqpoint{4.356966in}{2.674181in}}%
\pgfpathlineto{\pgfqpoint{4.367923in}{2.677546in}}%
\pgfpathlineto{\pgfqpoint{4.378874in}{2.680911in}}%
\pgfpathlineto{\pgfqpoint{4.389820in}{2.684279in}}%
\pgfpathlineto{\pgfqpoint{4.383403in}{2.698560in}}%
\pgfpathlineto{\pgfqpoint{4.376988in}{2.712820in}}%
\pgfpathlineto{\pgfqpoint{4.370575in}{2.727052in}}%
\pgfpathlineto{\pgfqpoint{4.364163in}{2.741255in}}%
\pgfpathlineto{\pgfqpoint{4.357752in}{2.755430in}}%
\pgfpathlineto{\pgfqpoint{4.346805in}{2.751974in}}%
\pgfpathlineto{\pgfqpoint{4.335852in}{2.748525in}}%
\pgfpathlineto{\pgfqpoint{4.324895in}{2.745083in}}%
\pgfpathlineto{\pgfqpoint{4.313933in}{2.741646in}}%
\pgfpathlineto{\pgfqpoint{4.320344in}{2.727537in}}%
\pgfpathlineto{\pgfqpoint{4.326757in}{2.713401in}}%
\pgfpathlineto{\pgfqpoint{4.333171in}{2.699235in}}%
\pgfpathlineto{\pgfqpoint{4.339587in}{2.685039in}}%
\pgfpathclose%
\pgfusepath{stroke,fill}%
\end{pgfscope}%
\begin{pgfscope}%
\pgfpathrectangle{\pgfqpoint{0.887500in}{0.275000in}}{\pgfqpoint{4.225000in}{4.225000in}}%
\pgfusepath{clip}%
\pgfsetbuttcap%
\pgfsetroundjoin%
\definecolor{currentfill}{rgb}{0.149039,0.508051,0.557250}%
\pgfsetfillcolor{currentfill}%
\pgfsetfillopacity{0.700000}%
\pgfsetlinewidth{0.501875pt}%
\definecolor{currentstroke}{rgb}{1.000000,1.000000,1.000000}%
\pgfsetstrokecolor{currentstroke}%
\pgfsetstrokeopacity{0.500000}%
\pgfsetdash{}{0pt}%
\pgfpathmoveto{\pgfqpoint{2.621426in}{2.410414in}}%
\pgfpathlineto{\pgfqpoint{2.632802in}{2.414731in}}%
\pgfpathlineto{\pgfqpoint{2.644168in}{2.419477in}}%
\pgfpathlineto{\pgfqpoint{2.655527in}{2.424420in}}%
\pgfpathlineto{\pgfqpoint{2.666882in}{2.429323in}}%
\pgfpathlineto{\pgfqpoint{2.678235in}{2.433946in}}%
\pgfpathlineto{\pgfqpoint{2.672174in}{2.442934in}}%
\pgfpathlineto{\pgfqpoint{2.666117in}{2.451920in}}%
\pgfpathlineto{\pgfqpoint{2.660064in}{2.460901in}}%
\pgfpathlineto{\pgfqpoint{2.654015in}{2.469870in}}%
\pgfpathlineto{\pgfqpoint{2.647970in}{2.478823in}}%
\pgfpathlineto{\pgfqpoint{2.636642in}{2.472938in}}%
\pgfpathlineto{\pgfqpoint{2.625306in}{2.467387in}}%
\pgfpathlineto{\pgfqpoint{2.613962in}{2.462222in}}%
\pgfpathlineto{\pgfqpoint{2.602608in}{2.457490in}}%
\pgfpathlineto{\pgfqpoint{2.591244in}{2.453241in}}%
\pgfpathlineto{\pgfqpoint{2.597273in}{2.444668in}}%
\pgfpathlineto{\pgfqpoint{2.603305in}{2.436102in}}%
\pgfpathlineto{\pgfqpoint{2.609342in}{2.427540in}}%
\pgfpathlineto{\pgfqpoint{2.615382in}{2.418978in}}%
\pgfpathclose%
\pgfusepath{stroke,fill}%
\end{pgfscope}%
\begin{pgfscope}%
\pgfpathrectangle{\pgfqpoint{0.887500in}{0.275000in}}{\pgfqpoint{4.225000in}{4.225000in}}%
\pgfusepath{clip}%
\pgfsetbuttcap%
\pgfsetroundjoin%
\definecolor{currentfill}{rgb}{0.699415,0.867117,0.175971}%
\pgfsetfillcolor{currentfill}%
\pgfsetfillopacity{0.700000}%
\pgfsetlinewidth{0.501875pt}%
\definecolor{currentstroke}{rgb}{1.000000,1.000000,1.000000}%
\pgfsetstrokecolor{currentstroke}%
\pgfsetstrokeopacity{0.500000}%
\pgfsetdash{}{0pt}%
\pgfpathmoveto{\pgfqpoint{3.267403in}{3.298597in}}%
\pgfpathlineto{\pgfqpoint{3.278630in}{3.301612in}}%
\pgfpathlineto{\pgfqpoint{3.289852in}{3.304718in}}%
\pgfpathlineto{\pgfqpoint{3.301069in}{3.307902in}}%
\pgfpathlineto{\pgfqpoint{3.312281in}{3.311152in}}%
\pgfpathlineto{\pgfqpoint{3.323488in}{3.314456in}}%
\pgfpathlineto{\pgfqpoint{3.317235in}{3.324566in}}%
\pgfpathlineto{\pgfqpoint{3.310986in}{3.334747in}}%
\pgfpathlineto{\pgfqpoint{3.304740in}{3.344965in}}%
\pgfpathlineto{\pgfqpoint{3.298498in}{3.355181in}}%
\pgfpathlineto{\pgfqpoint{3.292260in}{3.365358in}}%
\pgfpathlineto{\pgfqpoint{3.281060in}{3.361809in}}%
\pgfpathlineto{\pgfqpoint{3.269855in}{3.358394in}}%
\pgfpathlineto{\pgfqpoint{3.258646in}{3.355134in}}%
\pgfpathlineto{\pgfqpoint{3.247432in}{3.352047in}}%
\pgfpathlineto{\pgfqpoint{3.236213in}{3.349142in}}%
\pgfpathlineto{\pgfqpoint{3.242444in}{3.339028in}}%
\pgfpathlineto{\pgfqpoint{3.248679in}{3.328882in}}%
\pgfpathlineto{\pgfqpoint{3.254916in}{3.318739in}}%
\pgfpathlineto{\pgfqpoint{3.261158in}{3.308634in}}%
\pgfpathclose%
\pgfusepath{stroke,fill}%
\end{pgfscope}%
\begin{pgfscope}%
\pgfpathrectangle{\pgfqpoint{0.887500in}{0.275000in}}{\pgfqpoint{4.225000in}{4.225000in}}%
\pgfusepath{clip}%
\pgfsetbuttcap%
\pgfsetroundjoin%
\definecolor{currentfill}{rgb}{0.121380,0.629492,0.531973}%
\pgfsetfillcolor{currentfill}%
\pgfsetfillopacity{0.700000}%
\pgfsetlinewidth{0.501875pt}%
\definecolor{currentstroke}{rgb}{1.000000,1.000000,1.000000}%
\pgfsetstrokecolor{currentstroke}%
\pgfsetstrokeopacity{0.500000}%
\pgfsetdash{}{0pt}%
\pgfpathmoveto{\pgfqpoint{2.847272in}{2.632014in}}%
\pgfpathlineto{\pgfqpoint{2.858458in}{2.661950in}}%
\pgfpathlineto{\pgfqpoint{2.869659in}{2.690866in}}%
\pgfpathlineto{\pgfqpoint{2.880874in}{2.718795in}}%
\pgfpathlineto{\pgfqpoint{2.892099in}{2.746015in}}%
\pgfpathlineto{\pgfqpoint{2.903335in}{2.772806in}}%
\pgfpathlineto{\pgfqpoint{2.897232in}{2.771500in}}%
\pgfpathlineto{\pgfqpoint{2.891139in}{2.769667in}}%
\pgfpathlineto{\pgfqpoint{2.885054in}{2.767626in}}%
\pgfpathlineto{\pgfqpoint{2.878975in}{2.765699in}}%
\pgfpathlineto{\pgfqpoint{2.872902in}{2.764206in}}%
\pgfpathlineto{\pgfqpoint{2.861693in}{2.740731in}}%
\pgfpathlineto{\pgfqpoint{2.850486in}{2.718128in}}%
\pgfpathlineto{\pgfqpoint{2.839281in}{2.696284in}}%
\pgfpathlineto{\pgfqpoint{2.828077in}{2.675088in}}%
\pgfpathlineto{\pgfqpoint{2.816873in}{2.654485in}}%
\pgfpathlineto{\pgfqpoint{2.822943in}{2.650018in}}%
\pgfpathlineto{\pgfqpoint{2.829016in}{2.645683in}}%
\pgfpathlineto{\pgfqpoint{2.835095in}{2.641335in}}%
\pgfpathlineto{\pgfqpoint{2.841180in}{2.636827in}}%
\pgfpathclose%
\pgfusepath{stroke,fill}%
\end{pgfscope}%
\begin{pgfscope}%
\pgfpathrectangle{\pgfqpoint{0.887500in}{0.275000in}}{\pgfqpoint{4.225000in}{4.225000in}}%
\pgfusepath{clip}%
\pgfsetbuttcap%
\pgfsetroundjoin%
\definecolor{currentfill}{rgb}{0.125394,0.574318,0.549086}%
\pgfsetfillcolor{currentfill}%
\pgfsetfillopacity{0.700000}%
\pgfsetlinewidth{0.501875pt}%
\definecolor{currentstroke}{rgb}{1.000000,1.000000,1.000000}%
\pgfsetstrokecolor{currentstroke}%
\pgfsetstrokeopacity{0.500000}%
\pgfsetdash{}{0pt}%
\pgfpathmoveto{\pgfqpoint{1.753096in}{2.560475in}}%
\pgfpathlineto{\pgfqpoint{1.764699in}{2.563745in}}%
\pgfpathlineto{\pgfqpoint{1.776296in}{2.567006in}}%
\pgfpathlineto{\pgfqpoint{1.787887in}{2.570259in}}%
\pgfpathlineto{\pgfqpoint{1.799473in}{2.573506in}}%
\pgfpathlineto{\pgfqpoint{1.811054in}{2.576748in}}%
\pgfpathlineto{\pgfqpoint{1.805295in}{2.584755in}}%
\pgfpathlineto{\pgfqpoint{1.799540in}{2.592745in}}%
\pgfpathlineto{\pgfqpoint{1.793789in}{2.600718in}}%
\pgfpathlineto{\pgfqpoint{1.788043in}{2.608673in}}%
\pgfpathlineto{\pgfqpoint{1.782301in}{2.616610in}}%
\pgfpathlineto{\pgfqpoint{1.770734in}{2.613369in}}%
\pgfpathlineto{\pgfqpoint{1.759161in}{2.610124in}}%
\pgfpathlineto{\pgfqpoint{1.747582in}{2.606876in}}%
\pgfpathlineto{\pgfqpoint{1.735998in}{2.603622in}}%
\pgfpathlineto{\pgfqpoint{1.724408in}{2.600362in}}%
\pgfpathlineto{\pgfqpoint{1.730137in}{2.592423in}}%
\pgfpathlineto{\pgfqpoint{1.735870in}{2.584464in}}%
\pgfpathlineto{\pgfqpoint{1.741608in}{2.576486in}}%
\pgfpathlineto{\pgfqpoint{1.747350in}{2.568490in}}%
\pgfpathclose%
\pgfusepath{stroke,fill}%
\end{pgfscope}%
\begin{pgfscope}%
\pgfpathrectangle{\pgfqpoint{0.887500in}{0.275000in}}{\pgfqpoint{4.225000in}{4.225000in}}%
\pgfusepath{clip}%
\pgfsetbuttcap%
\pgfsetroundjoin%
\definecolor{currentfill}{rgb}{0.344074,0.780029,0.397381}%
\pgfsetfillcolor{currentfill}%
\pgfsetfillopacity{0.700000}%
\pgfsetlinewidth{0.501875pt}%
\definecolor{currentstroke}{rgb}{1.000000,1.000000,1.000000}%
\pgfsetstrokecolor{currentstroke}%
\pgfsetstrokeopacity{0.500000}%
\pgfsetdash{}{0pt}%
\pgfpathmoveto{\pgfqpoint{3.878761in}{3.000252in}}%
\pgfpathlineto{\pgfqpoint{3.889844in}{3.003775in}}%
\pgfpathlineto{\pgfqpoint{3.900922in}{3.007309in}}%
\pgfpathlineto{\pgfqpoint{3.911996in}{3.010853in}}%
\pgfpathlineto{\pgfqpoint{3.923064in}{3.014396in}}%
\pgfpathlineto{\pgfqpoint{3.934127in}{3.017924in}}%
\pgfpathlineto{\pgfqpoint{3.927773in}{3.031328in}}%
\pgfpathlineto{\pgfqpoint{3.921419in}{3.044595in}}%
\pgfpathlineto{\pgfqpoint{3.915065in}{3.057702in}}%
\pgfpathlineto{\pgfqpoint{3.908709in}{3.070629in}}%
\pgfpathlineto{\pgfqpoint{3.902352in}{3.083355in}}%
\pgfpathlineto{\pgfqpoint{3.891294in}{3.079875in}}%
\pgfpathlineto{\pgfqpoint{3.880230in}{3.076374in}}%
\pgfpathlineto{\pgfqpoint{3.869161in}{3.072865in}}%
\pgfpathlineto{\pgfqpoint{3.858087in}{3.069364in}}%
\pgfpathlineto{\pgfqpoint{3.847007in}{3.065873in}}%
\pgfpathlineto{\pgfqpoint{3.853360in}{3.053125in}}%
\pgfpathlineto{\pgfqpoint{3.859711in}{3.040166in}}%
\pgfpathlineto{\pgfqpoint{3.866062in}{3.027017in}}%
\pgfpathlineto{\pgfqpoint{3.872411in}{3.013704in}}%
\pgfpathclose%
\pgfusepath{stroke,fill}%
\end{pgfscope}%
\begin{pgfscope}%
\pgfpathrectangle{\pgfqpoint{0.887500in}{0.275000in}}{\pgfqpoint{4.225000in}{4.225000in}}%
\pgfusepath{clip}%
\pgfsetbuttcap%
\pgfsetroundjoin%
\definecolor{currentfill}{rgb}{0.647257,0.858400,0.209861}%
\pgfsetfillcolor{currentfill}%
\pgfsetfillopacity{0.700000}%
\pgfsetlinewidth{0.501875pt}%
\definecolor{currentstroke}{rgb}{1.000000,1.000000,1.000000}%
\pgfsetstrokecolor{currentstroke}%
\pgfsetstrokeopacity{0.500000}%
\pgfsetdash{}{0pt}%
\pgfpathmoveto{\pgfqpoint{2.979960in}{3.214469in}}%
\pgfpathlineto{\pgfqpoint{2.991205in}{3.244118in}}%
\pgfpathlineto{\pgfqpoint{3.002462in}{3.270499in}}%
\pgfpathlineto{\pgfqpoint{3.013732in}{3.293203in}}%
\pgfpathlineto{\pgfqpoint{3.025009in}{3.312486in}}%
\pgfpathlineto{\pgfqpoint{3.036291in}{3.328661in}}%
\pgfpathlineto{\pgfqpoint{3.030110in}{3.336970in}}%
\pgfpathlineto{\pgfqpoint{3.023934in}{3.345037in}}%
\pgfpathlineto{\pgfqpoint{3.017762in}{3.352855in}}%
\pgfpathlineto{\pgfqpoint{3.011594in}{3.360413in}}%
\pgfpathlineto{\pgfqpoint{3.005430in}{3.367702in}}%
\pgfpathlineto{\pgfqpoint{2.994173in}{3.349456in}}%
\pgfpathlineto{\pgfqpoint{2.982925in}{3.327278in}}%
\pgfpathlineto{\pgfqpoint{2.971692in}{3.300695in}}%
\pgfpathlineto{\pgfqpoint{2.960477in}{3.269322in}}%
\pgfpathlineto{\pgfqpoint{2.949284in}{3.233858in}}%
\pgfpathlineto{\pgfqpoint{2.955408in}{3.230522in}}%
\pgfpathlineto{\pgfqpoint{2.961537in}{3.226916in}}%
\pgfpathlineto{\pgfqpoint{2.967672in}{3.223039in}}%
\pgfpathlineto{\pgfqpoint{2.973813in}{3.218889in}}%
\pgfpathclose%
\pgfusepath{stroke,fill}%
\end{pgfscope}%
\begin{pgfscope}%
\pgfpathrectangle{\pgfqpoint{0.887500in}{0.275000in}}{\pgfqpoint{4.225000in}{4.225000in}}%
\pgfusepath{clip}%
\pgfsetbuttcap%
\pgfsetroundjoin%
\definecolor{currentfill}{rgb}{0.137339,0.662252,0.515571}%
\pgfsetfillcolor{currentfill}%
\pgfsetfillopacity{0.700000}%
\pgfsetlinewidth{0.501875pt}%
\definecolor{currentstroke}{rgb}{1.000000,1.000000,1.000000}%
\pgfsetstrokecolor{currentstroke}%
\pgfsetstrokeopacity{0.500000}%
\pgfsetdash{}{0pt}%
\pgfpathmoveto{\pgfqpoint{4.259043in}{2.724504in}}%
\pgfpathlineto{\pgfqpoint{4.270031in}{2.727930in}}%
\pgfpathlineto{\pgfqpoint{4.281014in}{2.731357in}}%
\pgfpathlineto{\pgfqpoint{4.291992in}{2.734784in}}%
\pgfpathlineto{\pgfqpoint{4.302965in}{2.738213in}}%
\pgfpathlineto{\pgfqpoint{4.313933in}{2.741646in}}%
\pgfpathlineto{\pgfqpoint{4.307523in}{2.755732in}}%
\pgfpathlineto{\pgfqpoint{4.301116in}{2.769800in}}%
\pgfpathlineto{\pgfqpoint{4.294710in}{2.783855in}}%
\pgfpathlineto{\pgfqpoint{4.288307in}{2.797900in}}%
\pgfpathlineto{\pgfqpoint{4.281907in}{2.811940in}}%
\pgfpathlineto{\pgfqpoint{4.270939in}{2.808455in}}%
\pgfpathlineto{\pgfqpoint{4.259967in}{2.804979in}}%
\pgfpathlineto{\pgfqpoint{4.248990in}{2.801509in}}%
\pgfpathlineto{\pgfqpoint{4.238007in}{2.798046in}}%
\pgfpathlineto{\pgfqpoint{4.227020in}{2.794587in}}%
\pgfpathlineto{\pgfqpoint{4.233420in}{2.780582in}}%
\pgfpathlineto{\pgfqpoint{4.239822in}{2.766576in}}%
\pgfpathlineto{\pgfqpoint{4.246227in}{2.752564in}}%
\pgfpathlineto{\pgfqpoint{4.252634in}{2.738542in}}%
\pgfpathclose%
\pgfusepath{stroke,fill}%
\end{pgfscope}%
\begin{pgfscope}%
\pgfpathrectangle{\pgfqpoint{0.887500in}{0.275000in}}{\pgfqpoint{4.225000in}{4.225000in}}%
\pgfusepath{clip}%
\pgfsetbuttcap%
\pgfsetroundjoin%
\definecolor{currentfill}{rgb}{0.132444,0.552216,0.553018}%
\pgfsetfillcolor{currentfill}%
\pgfsetfillopacity{0.700000}%
\pgfsetlinewidth{0.501875pt}%
\definecolor{currentstroke}{rgb}{1.000000,1.000000,1.000000}%
\pgfsetstrokecolor{currentstroke}%
\pgfsetstrokeopacity{0.500000}%
\pgfsetdash{}{0pt}%
\pgfpathmoveto{\pgfqpoint{2.821992in}{2.432148in}}%
\pgfpathlineto{\pgfqpoint{2.833179in}{2.458525in}}%
\pgfpathlineto{\pgfqpoint{2.844348in}{2.489759in}}%
\pgfpathlineto{\pgfqpoint{2.855510in}{2.524418in}}%
\pgfpathlineto{\pgfqpoint{2.866675in}{2.561066in}}%
\pgfpathlineto{\pgfqpoint{2.877851in}{2.598257in}}%
\pgfpathlineto{\pgfqpoint{2.871718in}{2.606788in}}%
\pgfpathlineto{\pgfqpoint{2.865593in}{2.614283in}}%
\pgfpathlineto{\pgfqpoint{2.859478in}{2.620888in}}%
\pgfpathlineto{\pgfqpoint{2.853371in}{2.626750in}}%
\pgfpathlineto{\pgfqpoint{2.847272in}{2.632014in}}%
\pgfpathlineto{\pgfqpoint{2.836097in}{2.601827in}}%
\pgfpathlineto{\pgfqpoint{2.824927in}{2.572314in}}%
\pgfpathlineto{\pgfqpoint{2.813757in}{2.544397in}}%
\pgfpathlineto{\pgfqpoint{2.802580in}{2.518994in}}%
\pgfpathlineto{\pgfqpoint{2.791386in}{2.497020in}}%
\pgfpathlineto{\pgfqpoint{2.797508in}{2.483165in}}%
\pgfpathlineto{\pgfqpoint{2.803630in}{2.469610in}}%
\pgfpathlineto{\pgfqpoint{2.809752in}{2.456496in}}%
\pgfpathlineto{\pgfqpoint{2.815872in}{2.443962in}}%
\pgfpathclose%
\pgfusepath{stroke,fill}%
\end{pgfscope}%
\begin{pgfscope}%
\pgfpathrectangle{\pgfqpoint{0.887500in}{0.275000in}}{\pgfqpoint{4.225000in}{4.225000in}}%
\pgfusepath{clip}%
\pgfsetbuttcap%
\pgfsetroundjoin%
\definecolor{currentfill}{rgb}{0.657642,0.860219,0.203082}%
\pgfsetfillcolor{currentfill}%
\pgfsetfillopacity{0.700000}%
\pgfsetlinewidth{0.501875pt}%
\definecolor{currentstroke}{rgb}{1.000000,1.000000,1.000000}%
\pgfsetstrokecolor{currentstroke}%
\pgfsetstrokeopacity{0.500000}%
\pgfsetdash{}{0pt}%
\pgfpathmoveto{\pgfqpoint{3.354808in}{3.264175in}}%
\pgfpathlineto{\pgfqpoint{3.366015in}{3.267138in}}%
\pgfpathlineto{\pgfqpoint{3.377216in}{3.270051in}}%
\pgfpathlineto{\pgfqpoint{3.388411in}{3.272921in}}%
\pgfpathlineto{\pgfqpoint{3.399600in}{3.275750in}}%
\pgfpathlineto{\pgfqpoint{3.410783in}{3.278548in}}%
\pgfpathlineto{\pgfqpoint{3.404509in}{3.289156in}}%
\pgfpathlineto{\pgfqpoint{3.398238in}{3.299731in}}%
\pgfpathlineto{\pgfqpoint{3.391970in}{3.310290in}}%
\pgfpathlineto{\pgfqpoint{3.385706in}{3.320847in}}%
\pgfpathlineto{\pgfqpoint{3.379445in}{3.331420in}}%
\pgfpathlineto{\pgfqpoint{3.368264in}{3.328004in}}%
\pgfpathlineto{\pgfqpoint{3.357078in}{3.324589in}}%
\pgfpathlineto{\pgfqpoint{3.345886in}{3.321186in}}%
\pgfpathlineto{\pgfqpoint{3.334690in}{3.317805in}}%
\pgfpathlineto{\pgfqpoint{3.323488in}{3.314456in}}%
\pgfpathlineto{\pgfqpoint{3.329745in}{3.304411in}}%
\pgfpathlineto{\pgfqpoint{3.336006in}{3.294395in}}%
\pgfpathlineto{\pgfqpoint{3.342270in}{3.284373in}}%
\pgfpathlineto{\pgfqpoint{3.348538in}{3.274312in}}%
\pgfpathclose%
\pgfusepath{stroke,fill}%
\end{pgfscope}%
\begin{pgfscope}%
\pgfpathrectangle{\pgfqpoint{0.887500in}{0.275000in}}{\pgfqpoint{4.225000in}{4.225000in}}%
\pgfusepath{clip}%
\pgfsetbuttcap%
\pgfsetroundjoin%
\definecolor{currentfill}{rgb}{0.395174,0.797475,0.367757}%
\pgfsetfillcolor{currentfill}%
\pgfsetfillopacity{0.700000}%
\pgfsetlinewidth{0.501875pt}%
\definecolor{currentstroke}{rgb}{1.000000,1.000000,1.000000}%
\pgfsetstrokecolor{currentstroke}%
\pgfsetstrokeopacity{0.500000}%
\pgfsetdash{}{0pt}%
\pgfpathmoveto{\pgfqpoint{3.791532in}{3.048371in}}%
\pgfpathlineto{\pgfqpoint{3.802638in}{3.051898in}}%
\pgfpathlineto{\pgfqpoint{3.813739in}{3.055406in}}%
\pgfpathlineto{\pgfqpoint{3.824834in}{3.058900in}}%
\pgfpathlineto{\pgfqpoint{3.835923in}{3.062387in}}%
\pgfpathlineto{\pgfqpoint{3.847007in}{3.065873in}}%
\pgfpathlineto{\pgfqpoint{3.840654in}{3.078421in}}%
\pgfpathlineto{\pgfqpoint{3.834300in}{3.090799in}}%
\pgfpathlineto{\pgfqpoint{3.827946in}{3.103036in}}%
\pgfpathlineto{\pgfqpoint{3.821593in}{3.115164in}}%
\pgfpathlineto{\pgfqpoint{3.815242in}{3.127210in}}%
\pgfpathlineto{\pgfqpoint{3.804164in}{3.123809in}}%
\pgfpathlineto{\pgfqpoint{3.793081in}{3.120429in}}%
\pgfpathlineto{\pgfqpoint{3.781993in}{3.117059in}}%
\pgfpathlineto{\pgfqpoint{3.770899in}{3.113690in}}%
\pgfpathlineto{\pgfqpoint{3.759801in}{3.110313in}}%
\pgfpathlineto{\pgfqpoint{3.766145in}{3.098131in}}%
\pgfpathlineto{\pgfqpoint{3.772490in}{3.085870in}}%
\pgfpathlineto{\pgfqpoint{3.778837in}{3.073506in}}%
\pgfpathlineto{\pgfqpoint{3.785184in}{3.061014in}}%
\pgfpathclose%
\pgfusepath{stroke,fill}%
\end{pgfscope}%
\begin{pgfscope}%
\pgfpathrectangle{\pgfqpoint{0.887500in}{0.275000in}}{\pgfqpoint{4.225000in}{4.225000in}}%
\pgfusepath{clip}%
\pgfsetbuttcap%
\pgfsetroundjoin%
\definecolor{currentfill}{rgb}{0.133743,0.548535,0.553541}%
\pgfsetfillcolor{currentfill}%
\pgfsetfillopacity{0.700000}%
\pgfsetlinewidth{0.501875pt}%
\definecolor{currentstroke}{rgb}{1.000000,1.000000,1.000000}%
\pgfsetstrokecolor{currentstroke}%
\pgfsetstrokeopacity{0.500000}%
\pgfsetdash{}{0pt}%
\pgfpathmoveto{\pgfqpoint{2.071510in}{2.504261in}}%
\pgfpathlineto{\pgfqpoint{2.083035in}{2.507591in}}%
\pgfpathlineto{\pgfqpoint{2.094554in}{2.510911in}}%
\pgfpathlineto{\pgfqpoint{2.106067in}{2.514222in}}%
\pgfpathlineto{\pgfqpoint{2.117575in}{2.517524in}}%
\pgfpathlineto{\pgfqpoint{2.129078in}{2.520818in}}%
\pgfpathlineto{\pgfqpoint{2.123207in}{2.529029in}}%
\pgfpathlineto{\pgfqpoint{2.117339in}{2.537224in}}%
\pgfpathlineto{\pgfqpoint{2.111476in}{2.545403in}}%
\pgfpathlineto{\pgfqpoint{2.105617in}{2.553568in}}%
\pgfpathlineto{\pgfqpoint{2.099762in}{2.561719in}}%
\pgfpathlineto{\pgfqpoint{2.088272in}{2.558419in}}%
\pgfpathlineto{\pgfqpoint{2.076776in}{2.555112in}}%
\pgfpathlineto{\pgfqpoint{2.065275in}{2.551798in}}%
\pgfpathlineto{\pgfqpoint{2.053769in}{2.548477in}}%
\pgfpathlineto{\pgfqpoint{2.042257in}{2.545150in}}%
\pgfpathlineto{\pgfqpoint{2.048099in}{2.537005in}}%
\pgfpathlineto{\pgfqpoint{2.053945in}{2.528845in}}%
\pgfpathlineto{\pgfqpoint{2.059796in}{2.520668in}}%
\pgfpathlineto{\pgfqpoint{2.065651in}{2.512473in}}%
\pgfpathclose%
\pgfusepath{stroke,fill}%
\end{pgfscope}%
\begin{pgfscope}%
\pgfpathrectangle{\pgfqpoint{0.887500in}{0.275000in}}{\pgfqpoint{4.225000in}{4.225000in}}%
\pgfusepath{clip}%
\pgfsetbuttcap%
\pgfsetroundjoin%
\definecolor{currentfill}{rgb}{0.440137,0.811138,0.340967}%
\pgfsetfillcolor{currentfill}%
\pgfsetfillopacity{0.700000}%
\pgfsetlinewidth{0.501875pt}%
\definecolor{currentstroke}{rgb}{1.000000,1.000000,1.000000}%
\pgfsetstrokecolor{currentstroke}%
\pgfsetstrokeopacity{0.500000}%
\pgfsetdash{}{0pt}%
\pgfpathmoveto{\pgfqpoint{3.704226in}{3.093312in}}%
\pgfpathlineto{\pgfqpoint{3.715351in}{3.096696in}}%
\pgfpathlineto{\pgfqpoint{3.726471in}{3.100099in}}%
\pgfpathlineto{\pgfqpoint{3.737586in}{3.103510in}}%
\pgfpathlineto{\pgfqpoint{3.748696in}{3.106918in}}%
\pgfpathlineto{\pgfqpoint{3.759801in}{3.110313in}}%
\pgfpathlineto{\pgfqpoint{3.753459in}{3.122439in}}%
\pgfpathlineto{\pgfqpoint{3.747119in}{3.134534in}}%
\pgfpathlineto{\pgfqpoint{3.740783in}{3.146622in}}%
\pgfpathlineto{\pgfqpoint{3.734449in}{3.158726in}}%
\pgfpathlineto{\pgfqpoint{3.728120in}{3.170871in}}%
\pgfpathlineto{\pgfqpoint{3.717022in}{3.167602in}}%
\pgfpathlineto{\pgfqpoint{3.705920in}{3.164326in}}%
\pgfpathlineto{\pgfqpoint{3.694812in}{3.161055in}}%
\pgfpathlineto{\pgfqpoint{3.683698in}{3.157799in}}%
\pgfpathlineto{\pgfqpoint{3.672580in}{3.154570in}}%
\pgfpathlineto{\pgfqpoint{3.678904in}{3.142371in}}%
\pgfpathlineto{\pgfqpoint{3.685231in}{3.130159in}}%
\pgfpathlineto{\pgfqpoint{3.691560in}{3.117921in}}%
\pgfpathlineto{\pgfqpoint{3.697892in}{3.105643in}}%
\pgfpathclose%
\pgfusepath{stroke,fill}%
\end{pgfscope}%
\begin{pgfscope}%
\pgfpathrectangle{\pgfqpoint{0.887500in}{0.275000in}}{\pgfqpoint{4.225000in}{4.225000in}}%
\pgfusepath{clip}%
\pgfsetbuttcap%
\pgfsetroundjoin%
\definecolor{currentfill}{rgb}{0.606045,0.850733,0.236712}%
\pgfsetfillcolor{currentfill}%
\pgfsetfillopacity{0.700000}%
\pgfsetlinewidth{0.501875pt}%
\definecolor{currentstroke}{rgb}{1.000000,1.000000,1.000000}%
\pgfsetstrokecolor{currentstroke}%
\pgfsetstrokeopacity{0.500000}%
\pgfsetdash{}{0pt}%
\pgfpathmoveto{\pgfqpoint{3.442196in}{3.224470in}}%
\pgfpathlineto{\pgfqpoint{3.453377in}{3.226978in}}%
\pgfpathlineto{\pgfqpoint{3.464553in}{3.229513in}}%
\pgfpathlineto{\pgfqpoint{3.475723in}{3.232118in}}%
\pgfpathlineto{\pgfqpoint{3.486889in}{3.234838in}}%
\pgfpathlineto{\pgfqpoint{3.498051in}{3.237716in}}%
\pgfpathlineto{\pgfqpoint{3.491758in}{3.248767in}}%
\pgfpathlineto{\pgfqpoint{3.485469in}{3.259831in}}%
\pgfpathlineto{\pgfqpoint{3.479183in}{3.270907in}}%
\pgfpathlineto{\pgfqpoint{3.472900in}{3.281992in}}%
\pgfpathlineto{\pgfqpoint{3.466620in}{3.293087in}}%
\pgfpathlineto{\pgfqpoint{3.455462in}{3.289999in}}%
\pgfpathlineto{\pgfqpoint{3.444300in}{3.287033in}}%
\pgfpathlineto{\pgfqpoint{3.433133in}{3.284157in}}%
\pgfpathlineto{\pgfqpoint{3.421961in}{3.281340in}}%
\pgfpathlineto{\pgfqpoint{3.410783in}{3.278548in}}%
\pgfpathlineto{\pgfqpoint{3.417061in}{3.267892in}}%
\pgfpathlineto{\pgfqpoint{3.423341in}{3.257172in}}%
\pgfpathlineto{\pgfqpoint{3.429623in}{3.246372in}}%
\pgfpathlineto{\pgfqpoint{3.435908in}{3.235477in}}%
\pgfpathclose%
\pgfusepath{stroke,fill}%
\end{pgfscope}%
\begin{pgfscope}%
\pgfpathrectangle{\pgfqpoint{0.887500in}{0.275000in}}{\pgfqpoint{4.225000in}{4.225000in}}%
\pgfusepath{clip}%
\pgfsetbuttcap%
\pgfsetroundjoin%
\definecolor{currentfill}{rgb}{0.440137,0.811138,0.340967}%
\pgfsetfillcolor{currentfill}%
\pgfsetfillopacity{0.700000}%
\pgfsetlinewidth{0.501875pt}%
\definecolor{currentstroke}{rgb}{1.000000,1.000000,1.000000}%
\pgfsetstrokecolor{currentstroke}%
\pgfsetstrokeopacity{0.500000}%
\pgfsetdash{}{0pt}%
\pgfpathmoveto{\pgfqpoint{2.954508in}{3.044847in}}%
\pgfpathlineto{\pgfqpoint{2.965740in}{3.075408in}}%
\pgfpathlineto{\pgfqpoint{2.976983in}{3.105468in}}%
\pgfpathlineto{\pgfqpoint{2.988237in}{3.134632in}}%
\pgfpathlineto{\pgfqpoint{2.999502in}{3.162498in}}%
\pgfpathlineto{\pgfqpoint{3.010777in}{3.188668in}}%
\pgfpathlineto{\pgfqpoint{3.004603in}{3.194286in}}%
\pgfpathlineto{\pgfqpoint{2.998435in}{3.199686in}}%
\pgfpathlineto{\pgfqpoint{2.992271in}{3.204857in}}%
\pgfpathlineto{\pgfqpoint{2.986113in}{3.209788in}}%
\pgfpathlineto{\pgfqpoint{2.979960in}{3.214469in}}%
\pgfpathlineto{\pgfqpoint{2.968731in}{3.182398in}}%
\pgfpathlineto{\pgfqpoint{2.957516in}{3.148753in}}%
\pgfpathlineto{\pgfqpoint{2.946315in}{3.114384in}}%
\pgfpathlineto{\pgfqpoint{2.935125in}{3.080130in}}%
\pgfpathlineto{\pgfqpoint{2.923944in}{3.046830in}}%
\pgfpathlineto{\pgfqpoint{2.930044in}{3.046677in}}%
\pgfpathlineto{\pgfqpoint{2.936150in}{3.046372in}}%
\pgfpathlineto{\pgfqpoint{2.942263in}{3.045945in}}%
\pgfpathlineto{\pgfqpoint{2.948382in}{3.045427in}}%
\pgfpathclose%
\pgfusepath{stroke,fill}%
\end{pgfscope}%
\begin{pgfscope}%
\pgfpathrectangle{\pgfqpoint{0.887500in}{0.275000in}}{\pgfqpoint{4.225000in}{4.225000in}}%
\pgfusepath{clip}%
\pgfsetbuttcap%
\pgfsetroundjoin%
\definecolor{currentfill}{rgb}{0.555484,0.840254,0.269281}%
\pgfsetfillcolor{currentfill}%
\pgfsetfillopacity{0.700000}%
\pgfsetlinewidth{0.501875pt}%
\definecolor{currentstroke}{rgb}{1.000000,1.000000,1.000000}%
\pgfsetstrokecolor{currentstroke}%
\pgfsetstrokeopacity{0.500000}%
\pgfsetdash{}{0pt}%
\pgfpathmoveto{\pgfqpoint{3.529562in}{3.182112in}}%
\pgfpathlineto{\pgfqpoint{3.540727in}{3.185169in}}%
\pgfpathlineto{\pgfqpoint{3.551888in}{3.188317in}}%
\pgfpathlineto{\pgfqpoint{3.563044in}{3.191546in}}%
\pgfpathlineto{\pgfqpoint{3.574195in}{3.194839in}}%
\pgfpathlineto{\pgfqpoint{3.585342in}{3.198183in}}%
\pgfpathlineto{\pgfqpoint{3.579032in}{3.209691in}}%
\pgfpathlineto{\pgfqpoint{3.572724in}{3.221118in}}%
\pgfpathlineto{\pgfqpoint{3.566418in}{3.232473in}}%
\pgfpathlineto{\pgfqpoint{3.560115in}{3.243764in}}%
\pgfpathlineto{\pgfqpoint{3.553814in}{3.255001in}}%
\pgfpathlineto{\pgfqpoint{3.542669in}{3.251253in}}%
\pgfpathlineto{\pgfqpoint{3.531519in}{3.247604in}}%
\pgfpathlineto{\pgfqpoint{3.520366in}{3.244102in}}%
\pgfpathlineto{\pgfqpoint{3.509210in}{3.240795in}}%
\pgfpathlineto{\pgfqpoint{3.498051in}{3.237716in}}%
\pgfpathlineto{\pgfqpoint{3.504348in}{3.226670in}}%
\pgfpathlineto{\pgfqpoint{3.510647in}{3.215610in}}%
\pgfpathlineto{\pgfqpoint{3.516950in}{3.204512in}}%
\pgfpathlineto{\pgfqpoint{3.523255in}{3.193354in}}%
\pgfpathclose%
\pgfusepath{stroke,fill}%
\end{pgfscope}%
\begin{pgfscope}%
\pgfpathrectangle{\pgfqpoint{0.887500in}{0.275000in}}{\pgfqpoint{4.225000in}{4.225000in}}%
\pgfusepath{clip}%
\pgfsetbuttcap%
\pgfsetroundjoin%
\definecolor{currentfill}{rgb}{0.143343,0.522773,0.556295}%
\pgfsetfillcolor{currentfill}%
\pgfsetfillopacity{0.700000}%
\pgfsetlinewidth{0.501875pt}%
\definecolor{currentstroke}{rgb}{1.000000,1.000000,1.000000}%
\pgfsetstrokecolor{currentstroke}%
\pgfsetstrokeopacity{0.500000}%
\pgfsetdash{}{0pt}%
\pgfpathmoveto{\pgfqpoint{2.390043in}{2.445560in}}%
\pgfpathlineto{\pgfqpoint{2.401488in}{2.449013in}}%
\pgfpathlineto{\pgfqpoint{2.412928in}{2.452501in}}%
\pgfpathlineto{\pgfqpoint{2.424361in}{2.456018in}}%
\pgfpathlineto{\pgfqpoint{2.435789in}{2.459541in}}%
\pgfpathlineto{\pgfqpoint{2.447211in}{2.463042in}}%
\pgfpathlineto{\pgfqpoint{2.441230in}{2.471466in}}%
\pgfpathlineto{\pgfqpoint{2.435253in}{2.479880in}}%
\pgfpathlineto{\pgfqpoint{2.429280in}{2.488285in}}%
\pgfpathlineto{\pgfqpoint{2.423311in}{2.496682in}}%
\pgfpathlineto{\pgfqpoint{2.417346in}{2.505072in}}%
\pgfpathlineto{\pgfqpoint{2.405936in}{2.501568in}}%
\pgfpathlineto{\pgfqpoint{2.394520in}{2.498041in}}%
\pgfpathlineto{\pgfqpoint{2.383099in}{2.494521in}}%
\pgfpathlineto{\pgfqpoint{2.371671in}{2.491029in}}%
\pgfpathlineto{\pgfqpoint{2.360238in}{2.487574in}}%
\pgfpathlineto{\pgfqpoint{2.366191in}{2.479204in}}%
\pgfpathlineto{\pgfqpoint{2.372148in}{2.470817in}}%
\pgfpathlineto{\pgfqpoint{2.378109in}{2.462414in}}%
\pgfpathlineto{\pgfqpoint{2.384074in}{2.453995in}}%
\pgfpathclose%
\pgfusepath{stroke,fill}%
\end{pgfscope}%
\begin{pgfscope}%
\pgfpathrectangle{\pgfqpoint{0.887500in}{0.275000in}}{\pgfqpoint{4.225000in}{4.225000in}}%
\pgfusepath{clip}%
\pgfsetbuttcap%
\pgfsetroundjoin%
\definecolor{currentfill}{rgb}{0.496615,0.826376,0.306377}%
\pgfsetfillcolor{currentfill}%
\pgfsetfillopacity{0.700000}%
\pgfsetlinewidth{0.501875pt}%
\definecolor{currentstroke}{rgb}{1.000000,1.000000,1.000000}%
\pgfsetstrokecolor{currentstroke}%
\pgfsetstrokeopacity{0.500000}%
\pgfsetdash{}{0pt}%
\pgfpathmoveto{\pgfqpoint{3.616919in}{3.139111in}}%
\pgfpathlineto{\pgfqpoint{3.628061in}{3.142118in}}%
\pgfpathlineto{\pgfqpoint{3.639198in}{3.145154in}}%
\pgfpathlineto{\pgfqpoint{3.650330in}{3.148237in}}%
\pgfpathlineto{\pgfqpoint{3.661457in}{3.151379in}}%
\pgfpathlineto{\pgfqpoint{3.672580in}{3.154570in}}%
\pgfpathlineto{\pgfqpoint{3.666259in}{3.166758in}}%
\pgfpathlineto{\pgfqpoint{3.659941in}{3.178923in}}%
\pgfpathlineto{\pgfqpoint{3.653625in}{3.191051in}}%
\pgfpathlineto{\pgfqpoint{3.647311in}{3.203127in}}%
\pgfpathlineto{\pgfqpoint{3.640999in}{3.215137in}}%
\pgfpathlineto{\pgfqpoint{3.629878in}{3.211754in}}%
\pgfpathlineto{\pgfqpoint{3.618752in}{3.208360in}}%
\pgfpathlineto{\pgfqpoint{3.607620in}{3.204959in}}%
\pgfpathlineto{\pgfqpoint{3.596484in}{3.201561in}}%
\pgfpathlineto{\pgfqpoint{3.585342in}{3.198183in}}%
\pgfpathlineto{\pgfqpoint{3.591654in}{3.186585in}}%
\pgfpathlineto{\pgfqpoint{3.597968in}{3.174888in}}%
\pgfpathlineto{\pgfqpoint{3.604283in}{3.163082in}}%
\pgfpathlineto{\pgfqpoint{3.610600in}{3.151159in}}%
\pgfpathclose%
\pgfusepath{stroke,fill}%
\end{pgfscope}%
\begin{pgfscope}%
\pgfpathrectangle{\pgfqpoint{0.887500in}{0.275000in}}{\pgfqpoint{4.225000in}{4.225000in}}%
\pgfusepath{clip}%
\pgfsetbuttcap%
\pgfsetroundjoin%
\definecolor{currentfill}{rgb}{0.153364,0.497000,0.557724}%
\pgfsetfillcolor{currentfill}%
\pgfsetfillopacity{0.700000}%
\pgfsetlinewidth{0.501875pt}%
\definecolor{currentstroke}{rgb}{1.000000,1.000000,1.000000}%
\pgfsetstrokecolor{currentstroke}%
\pgfsetstrokeopacity{0.500000}%
\pgfsetdash{}{0pt}%
\pgfpathmoveto{\pgfqpoint{2.708593in}{2.389169in}}%
\pgfpathlineto{\pgfqpoint{2.719971in}{2.391646in}}%
\pgfpathlineto{\pgfqpoint{2.731355in}{2.392628in}}%
\pgfpathlineto{\pgfqpoint{2.742747in}{2.391825in}}%
\pgfpathlineto{\pgfqpoint{2.754138in}{2.389939in}}%
\pgfpathlineto{\pgfqpoint{2.765520in}{2.388329in}}%
\pgfpathlineto{\pgfqpoint{2.759418in}{2.399210in}}%
\pgfpathlineto{\pgfqpoint{2.753315in}{2.410744in}}%
\pgfpathlineto{\pgfqpoint{2.747210in}{2.422814in}}%
\pgfpathlineto{\pgfqpoint{2.741105in}{2.435302in}}%
\pgfpathlineto{\pgfqpoint{2.735000in}{2.448090in}}%
\pgfpathlineto{\pgfqpoint{2.723655in}{2.445750in}}%
\pgfpathlineto{\pgfqpoint{2.712299in}{2.443818in}}%
\pgfpathlineto{\pgfqpoint{2.700942in}{2.441405in}}%
\pgfpathlineto{\pgfqpoint{2.689588in}{2.438053in}}%
\pgfpathlineto{\pgfqpoint{2.678235in}{2.433946in}}%
\pgfpathlineto{\pgfqpoint{2.684299in}{2.424962in}}%
\pgfpathlineto{\pgfqpoint{2.690367in}{2.415988in}}%
\pgfpathlineto{\pgfqpoint{2.696439in}{2.407027in}}%
\pgfpathlineto{\pgfqpoint{2.702514in}{2.398086in}}%
\pgfpathclose%
\pgfusepath{stroke,fill}%
\end{pgfscope}%
\begin{pgfscope}%
\pgfpathrectangle{\pgfqpoint{0.887500in}{0.275000in}}{\pgfqpoint{4.225000in}{4.225000in}}%
\pgfusepath{clip}%
\pgfsetbuttcap%
\pgfsetroundjoin%
\definecolor{currentfill}{rgb}{0.121148,0.592739,0.544641}%
\pgfsetfillcolor{currentfill}%
\pgfsetfillopacity{0.700000}%
\pgfsetlinewidth{0.501875pt}%
\definecolor{currentstroke}{rgb}{1.000000,1.000000,1.000000}%
\pgfsetstrokecolor{currentstroke}%
\pgfsetstrokeopacity{0.500000}%
\pgfsetdash{}{0pt}%
\pgfpathmoveto{\pgfqpoint{1.521504in}{2.590520in}}%
\pgfpathlineto{\pgfqpoint{1.533165in}{2.593829in}}%
\pgfpathlineto{\pgfqpoint{1.544820in}{2.597130in}}%
\pgfpathlineto{\pgfqpoint{1.556470in}{2.600425in}}%
\pgfpathlineto{\pgfqpoint{1.568115in}{2.603714in}}%
\pgfpathlineto{\pgfqpoint{1.579754in}{2.606999in}}%
\pgfpathlineto{\pgfqpoint{1.574078in}{2.614804in}}%
\pgfpathlineto{\pgfqpoint{1.568408in}{2.622585in}}%
\pgfpathlineto{\pgfqpoint{1.562742in}{2.630343in}}%
\pgfpathlineto{\pgfqpoint{1.557080in}{2.638078in}}%
\pgfpathlineto{\pgfqpoint{1.551423in}{2.645794in}}%
\pgfpathlineto{\pgfqpoint{1.539798in}{2.642505in}}%
\pgfpathlineto{\pgfqpoint{1.528167in}{2.639211in}}%
\pgfpathlineto{\pgfqpoint{1.516531in}{2.635912in}}%
\pgfpathlineto{\pgfqpoint{1.504889in}{2.632607in}}%
\pgfpathlineto{\pgfqpoint{1.493242in}{2.629294in}}%
\pgfpathlineto{\pgfqpoint{1.498885in}{2.621585in}}%
\pgfpathlineto{\pgfqpoint{1.504533in}{2.613855in}}%
\pgfpathlineto{\pgfqpoint{1.510185in}{2.606101in}}%
\pgfpathlineto{\pgfqpoint{1.515842in}{2.598323in}}%
\pgfpathclose%
\pgfusepath{stroke,fill}%
\end{pgfscope}%
\begin{pgfscope}%
\pgfpathrectangle{\pgfqpoint{0.887500in}{0.275000in}}{\pgfqpoint{4.225000in}{4.225000in}}%
\pgfusepath{clip}%
\pgfsetbuttcap%
\pgfsetroundjoin%
\definecolor{currentfill}{rgb}{0.162016,0.687316,0.499129}%
\pgfsetfillcolor{currentfill}%
\pgfsetfillopacity{0.700000}%
\pgfsetlinewidth{0.501875pt}%
\definecolor{currentstroke}{rgb}{1.000000,1.000000,1.000000}%
\pgfsetstrokecolor{currentstroke}%
\pgfsetstrokeopacity{0.500000}%
\pgfsetdash{}{0pt}%
\pgfpathmoveto{\pgfqpoint{4.172006in}{2.777328in}}%
\pgfpathlineto{\pgfqpoint{4.183019in}{2.780780in}}%
\pgfpathlineto{\pgfqpoint{4.194027in}{2.784230in}}%
\pgfpathlineto{\pgfqpoint{4.205030in}{2.787681in}}%
\pgfpathlineto{\pgfqpoint{4.216027in}{2.791133in}}%
\pgfpathlineto{\pgfqpoint{4.227020in}{2.794587in}}%
\pgfpathlineto{\pgfqpoint{4.220623in}{2.808597in}}%
\pgfpathlineto{\pgfqpoint{4.214228in}{2.822614in}}%
\pgfpathlineto{\pgfqpoint{4.207837in}{2.836645in}}%
\pgfpathlineto{\pgfqpoint{4.201448in}{2.850693in}}%
\pgfpathlineto{\pgfqpoint{4.195062in}{2.864749in}}%
\pgfpathlineto{\pgfqpoint{4.184071in}{2.861260in}}%
\pgfpathlineto{\pgfqpoint{4.173074in}{2.857770in}}%
\pgfpathlineto{\pgfqpoint{4.162072in}{2.854276in}}%
\pgfpathlineto{\pgfqpoint{4.151065in}{2.850778in}}%
\pgfpathlineto{\pgfqpoint{4.140052in}{2.847275in}}%
\pgfpathlineto{\pgfqpoint{4.146437in}{2.833276in}}%
\pgfpathlineto{\pgfqpoint{4.152825in}{2.819280in}}%
\pgfpathlineto{\pgfqpoint{4.159216in}{2.805292in}}%
\pgfpathlineto{\pgfqpoint{4.165610in}{2.791309in}}%
\pgfpathclose%
\pgfusepath{stroke,fill}%
\end{pgfscope}%
\begin{pgfscope}%
\pgfpathrectangle{\pgfqpoint{0.887500in}{0.275000in}}{\pgfqpoint{4.225000in}{4.225000in}}%
\pgfusepath{clip}%
\pgfsetbuttcap%
\pgfsetroundjoin%
\definecolor{currentfill}{rgb}{0.159194,0.482237,0.558073}%
\pgfsetfillcolor{currentfill}%
\pgfsetfillopacity{0.700000}%
\pgfsetlinewidth{0.501875pt}%
\definecolor{currentstroke}{rgb}{1.000000,1.000000,1.000000}%
\pgfsetstrokecolor{currentstroke}%
\pgfsetstrokeopacity{0.500000}%
\pgfsetdash{}{0pt}%
\pgfpathmoveto{\pgfqpoint{2.796017in}{2.342786in}}%
\pgfpathlineto{\pgfqpoint{2.807389in}{2.342432in}}%
\pgfpathlineto{\pgfqpoint{2.818735in}{2.344977in}}%
\pgfpathlineto{\pgfqpoint{2.830051in}{2.351754in}}%
\pgfpathlineto{\pgfqpoint{2.841332in}{2.364101in}}%
\pgfpathlineto{\pgfqpoint{2.852579in}{2.383137in}}%
\pgfpathlineto{\pgfqpoint{2.846460in}{2.391914in}}%
\pgfpathlineto{\pgfqpoint{2.840342in}{2.401105in}}%
\pgfpathlineto{\pgfqpoint{2.834226in}{2.410809in}}%
\pgfpathlineto{\pgfqpoint{2.828109in}{2.421123in}}%
\pgfpathlineto{\pgfqpoint{2.821992in}{2.432148in}}%
\pgfpathlineto{\pgfqpoint{2.810773in}{2.412053in}}%
\pgfpathlineto{\pgfqpoint{2.799513in}{2.398858in}}%
\pgfpathlineto{\pgfqpoint{2.788214in}{2.391420in}}%
\pgfpathlineto{\pgfqpoint{2.776881in}{2.388366in}}%
\pgfpathlineto{\pgfqpoint{2.765520in}{2.388329in}}%
\pgfpathlineto{\pgfqpoint{2.771619in}{2.378156in}}%
\pgfpathlineto{\pgfqpoint{2.777718in}{2.368613in}}%
\pgfpathlineto{\pgfqpoint{2.783817in}{2.359602in}}%
\pgfpathlineto{\pgfqpoint{2.789916in}{2.351025in}}%
\pgfpathclose%
\pgfusepath{stroke,fill}%
\end{pgfscope}%
\begin{pgfscope}%
\pgfpathrectangle{\pgfqpoint{0.887500in}{0.275000in}}{\pgfqpoint{4.225000in}{4.225000in}}%
\pgfusepath{clip}%
\pgfsetbuttcap%
\pgfsetroundjoin%
\definecolor{currentfill}{rgb}{0.127568,0.566949,0.550556}%
\pgfsetfillcolor{currentfill}%
\pgfsetfillopacity{0.700000}%
\pgfsetlinewidth{0.501875pt}%
\definecolor{currentstroke}{rgb}{1.000000,1.000000,1.000000}%
\pgfsetstrokecolor{currentstroke}%
\pgfsetstrokeopacity{0.500000}%
\pgfsetdash{}{0pt}%
\pgfpathmoveto{\pgfqpoint{1.839911in}{2.536494in}}%
\pgfpathlineto{\pgfqpoint{1.851499in}{2.539740in}}%
\pgfpathlineto{\pgfqpoint{1.863081in}{2.542983in}}%
\pgfpathlineto{\pgfqpoint{1.874657in}{2.546224in}}%
\pgfpathlineto{\pgfqpoint{1.886227in}{2.549467in}}%
\pgfpathlineto{\pgfqpoint{1.897791in}{2.552712in}}%
\pgfpathlineto{\pgfqpoint{1.891999in}{2.560784in}}%
\pgfpathlineto{\pgfqpoint{1.886210in}{2.568842in}}%
\pgfpathlineto{\pgfqpoint{1.880426in}{2.576889in}}%
\pgfpathlineto{\pgfqpoint{1.874646in}{2.584922in}}%
\pgfpathlineto{\pgfqpoint{1.868870in}{2.592941in}}%
\pgfpathlineto{\pgfqpoint{1.857318in}{2.589699in}}%
\pgfpathlineto{\pgfqpoint{1.845761in}{2.586461in}}%
\pgfpathlineto{\pgfqpoint{1.834197in}{2.583225in}}%
\pgfpathlineto{\pgfqpoint{1.822628in}{2.579987in}}%
\pgfpathlineto{\pgfqpoint{1.811054in}{2.576748in}}%
\pgfpathlineto{\pgfqpoint{1.816817in}{2.568726in}}%
\pgfpathlineto{\pgfqpoint{1.822584in}{2.560689in}}%
\pgfpathlineto{\pgfqpoint{1.828356in}{2.552638in}}%
\pgfpathlineto{\pgfqpoint{1.834131in}{2.544573in}}%
\pgfpathclose%
\pgfusepath{stroke,fill}%
\end{pgfscope}%
\begin{pgfscope}%
\pgfpathrectangle{\pgfqpoint{0.887500in}{0.275000in}}{\pgfqpoint{4.225000in}{4.225000in}}%
\pgfusepath{clip}%
\pgfsetbuttcap%
\pgfsetroundjoin%
\definecolor{currentfill}{rgb}{0.762373,0.876424,0.137064}%
\pgfsetfillcolor{currentfill}%
\pgfsetfillopacity{0.700000}%
\pgfsetlinewidth{0.501875pt}%
\definecolor{currentstroke}{rgb}{1.000000,1.000000,1.000000}%
\pgfsetstrokecolor{currentstroke}%
\pgfsetstrokeopacity{0.500000}%
\pgfsetdash{}{0pt}%
\pgfpathmoveto{\pgfqpoint{3.036291in}{3.328661in}}%
\pgfpathlineto{\pgfqpoint{3.047576in}{3.342041in}}%
\pgfpathlineto{\pgfqpoint{3.058861in}{3.352943in}}%
\pgfpathlineto{\pgfqpoint{3.070144in}{3.361684in}}%
\pgfpathlineto{\pgfqpoint{3.081424in}{3.368581in}}%
\pgfpathlineto{\pgfqpoint{3.092699in}{3.373932in}}%
\pgfpathlineto{\pgfqpoint{3.086503in}{3.383055in}}%
\pgfpathlineto{\pgfqpoint{3.080311in}{3.391915in}}%
\pgfpathlineto{\pgfqpoint{3.074123in}{3.400493in}}%
\pgfpathlineto{\pgfqpoint{3.067939in}{3.408769in}}%
\pgfpathlineto{\pgfqpoint{3.061758in}{3.416722in}}%
\pgfpathlineto{\pgfqpoint{3.050496in}{3.411003in}}%
\pgfpathlineto{\pgfqpoint{3.039230in}{3.403665in}}%
\pgfpathlineto{\pgfqpoint{3.027962in}{3.394323in}}%
\pgfpathlineto{\pgfqpoint{3.016695in}{3.382497in}}%
\pgfpathlineto{\pgfqpoint{3.005430in}{3.367702in}}%
\pgfpathlineto{\pgfqpoint{3.011594in}{3.360413in}}%
\pgfpathlineto{\pgfqpoint{3.017762in}{3.352855in}}%
\pgfpathlineto{\pgfqpoint{3.023934in}{3.345037in}}%
\pgfpathlineto{\pgfqpoint{3.030110in}{3.336970in}}%
\pgfpathclose%
\pgfusepath{stroke,fill}%
\end{pgfscope}%
\begin{pgfscope}%
\pgfpathrectangle{\pgfqpoint{0.887500in}{0.275000in}}{\pgfqpoint{4.225000in}{4.225000in}}%
\pgfusepath{clip}%
\pgfsetbuttcap%
\pgfsetroundjoin%
\definecolor{currentfill}{rgb}{0.196571,0.711827,0.479221}%
\pgfsetfillcolor{currentfill}%
\pgfsetfillopacity{0.700000}%
\pgfsetlinewidth{0.501875pt}%
\definecolor{currentstroke}{rgb}{1.000000,1.000000,1.000000}%
\pgfsetstrokecolor{currentstroke}%
\pgfsetstrokeopacity{0.500000}%
\pgfsetdash{}{0pt}%
\pgfpathmoveto{\pgfqpoint{4.084910in}{2.829729in}}%
\pgfpathlineto{\pgfqpoint{4.095948in}{2.833233in}}%
\pgfpathlineto{\pgfqpoint{4.106982in}{2.836742in}}%
\pgfpathlineto{\pgfqpoint{4.118011in}{2.840254in}}%
\pgfpathlineto{\pgfqpoint{4.129034in}{2.843766in}}%
\pgfpathlineto{\pgfqpoint{4.140052in}{2.847275in}}%
\pgfpathlineto{\pgfqpoint{4.133669in}{2.861265in}}%
\pgfpathlineto{\pgfqpoint{4.127288in}{2.875237in}}%
\pgfpathlineto{\pgfqpoint{4.120908in}{2.889181in}}%
\pgfpathlineto{\pgfqpoint{4.114530in}{2.903085in}}%
\pgfpathlineto{\pgfqpoint{4.108153in}{2.916939in}}%
\pgfpathlineto{\pgfqpoint{4.097136in}{2.913360in}}%
\pgfpathlineto{\pgfqpoint{4.086113in}{2.909770in}}%
\pgfpathlineto{\pgfqpoint{4.075084in}{2.906171in}}%
\pgfpathlineto{\pgfqpoint{4.064051in}{2.902566in}}%
\pgfpathlineto{\pgfqpoint{4.053012in}{2.898960in}}%
\pgfpathlineto{\pgfqpoint{4.059387in}{2.885157in}}%
\pgfpathlineto{\pgfqpoint{4.065765in}{2.871330in}}%
\pgfpathlineto{\pgfqpoint{4.072144in}{2.857482in}}%
\pgfpathlineto{\pgfqpoint{4.078526in}{2.843614in}}%
\pgfpathclose%
\pgfusepath{stroke,fill}%
\end{pgfscope}%
\begin{pgfscope}%
\pgfpathrectangle{\pgfqpoint{0.887500in}{0.275000in}}{\pgfqpoint{4.225000in}{4.225000in}}%
\pgfusepath{clip}%
\pgfsetbuttcap%
\pgfsetroundjoin%
\definecolor{currentfill}{rgb}{0.136408,0.541173,0.554483}%
\pgfsetfillcolor{currentfill}%
\pgfsetfillopacity{0.700000}%
\pgfsetlinewidth{0.501875pt}%
\definecolor{currentstroke}{rgb}{1.000000,1.000000,1.000000}%
\pgfsetstrokecolor{currentstroke}%
\pgfsetstrokeopacity{0.500000}%
\pgfsetdash{}{0pt}%
\pgfpathmoveto{\pgfqpoint{2.158497in}{2.479533in}}%
\pgfpathlineto{\pgfqpoint{2.170007in}{2.482825in}}%
\pgfpathlineto{\pgfqpoint{2.181510in}{2.486110in}}%
\pgfpathlineto{\pgfqpoint{2.193009in}{2.489390in}}%
\pgfpathlineto{\pgfqpoint{2.204501in}{2.492666in}}%
\pgfpathlineto{\pgfqpoint{2.215988in}{2.495940in}}%
\pgfpathlineto{\pgfqpoint{2.210084in}{2.504226in}}%
\pgfpathlineto{\pgfqpoint{2.204183in}{2.512496in}}%
\pgfpathlineto{\pgfqpoint{2.198287in}{2.520752in}}%
\pgfpathlineto{\pgfqpoint{2.192395in}{2.528992in}}%
\pgfpathlineto{\pgfqpoint{2.186508in}{2.537218in}}%
\pgfpathlineto{\pgfqpoint{2.175033in}{2.533943in}}%
\pgfpathlineto{\pgfqpoint{2.163553in}{2.530668in}}%
\pgfpathlineto{\pgfqpoint{2.152067in}{2.527389in}}%
\pgfpathlineto{\pgfqpoint{2.140575in}{2.524106in}}%
\pgfpathlineto{\pgfqpoint{2.129078in}{2.520818in}}%
\pgfpathlineto{\pgfqpoint{2.134954in}{2.512592in}}%
\pgfpathlineto{\pgfqpoint{2.140833in}{2.504351in}}%
\pgfpathlineto{\pgfqpoint{2.146717in}{2.496094in}}%
\pgfpathlineto{\pgfqpoint{2.152605in}{2.487821in}}%
\pgfpathclose%
\pgfusepath{stroke,fill}%
\end{pgfscope}%
\begin{pgfscope}%
\pgfpathrectangle{\pgfqpoint{0.887500in}{0.275000in}}{\pgfqpoint{4.225000in}{4.225000in}}%
\pgfusepath{clip}%
\pgfsetbuttcap%
\pgfsetroundjoin%
\definecolor{currentfill}{rgb}{0.274149,0.751988,0.436601}%
\pgfsetfillcolor{currentfill}%
\pgfsetfillopacity{0.700000}%
\pgfsetlinewidth{0.501875pt}%
\definecolor{currentstroke}{rgb}{1.000000,1.000000,1.000000}%
\pgfsetstrokecolor{currentstroke}%
\pgfsetstrokeopacity{0.500000}%
\pgfsetdash{}{0pt}%
\pgfpathmoveto{\pgfqpoint{2.928980in}{2.897870in}}%
\pgfpathlineto{\pgfqpoint{2.940211in}{2.927271in}}%
\pgfpathlineto{\pgfqpoint{2.951452in}{2.956788in}}%
\pgfpathlineto{\pgfqpoint{2.962702in}{2.986027in}}%
\pgfpathlineto{\pgfqpoint{2.973962in}{3.014591in}}%
\pgfpathlineto{\pgfqpoint{2.985234in}{3.042082in}}%
\pgfpathlineto{\pgfqpoint{2.979076in}{3.042521in}}%
\pgfpathlineto{\pgfqpoint{2.972925in}{3.043043in}}%
\pgfpathlineto{\pgfqpoint{2.966780in}{3.043625in}}%
\pgfpathlineto{\pgfqpoint{2.960641in}{3.044236in}}%
\pgfpathlineto{\pgfqpoint{2.954508in}{3.044847in}}%
\pgfpathlineto{\pgfqpoint{2.943286in}{3.014168in}}%
\pgfpathlineto{\pgfqpoint{2.932074in}{2.983638in}}%
\pgfpathlineto{\pgfqpoint{2.920870in}{2.953474in}}%
\pgfpathlineto{\pgfqpoint{2.909674in}{2.923890in}}%
\pgfpathlineto{\pgfqpoint{2.898484in}{2.895100in}}%
\pgfpathlineto{\pgfqpoint{2.904575in}{2.894439in}}%
\pgfpathlineto{\pgfqpoint{2.910669in}{2.894349in}}%
\pgfpathlineto{\pgfqpoint{2.916768in}{2.894869in}}%
\pgfpathlineto{\pgfqpoint{2.922872in}{2.896035in}}%
\pgfpathclose%
\pgfusepath{stroke,fill}%
\end{pgfscope}%
\begin{pgfscope}%
\pgfpathrectangle{\pgfqpoint{0.887500in}{0.275000in}}{\pgfqpoint{4.225000in}{4.225000in}}%
\pgfusepath{clip}%
\pgfsetbuttcap%
\pgfsetroundjoin%
\definecolor{currentfill}{rgb}{0.146180,0.515413,0.556823}%
\pgfsetfillcolor{currentfill}%
\pgfsetfillopacity{0.700000}%
\pgfsetlinewidth{0.501875pt}%
\definecolor{currentstroke}{rgb}{1.000000,1.000000,1.000000}%
\pgfsetstrokecolor{currentstroke}%
\pgfsetstrokeopacity{0.500000}%
\pgfsetdash{}{0pt}%
\pgfpathmoveto{\pgfqpoint{2.477176in}{2.420748in}}%
\pgfpathlineto{\pgfqpoint{2.488605in}{2.424213in}}%
\pgfpathlineto{\pgfqpoint{2.500030in}{2.427589in}}%
\pgfpathlineto{\pgfqpoint{2.511451in}{2.430844in}}%
\pgfpathlineto{\pgfqpoint{2.522868in}{2.433946in}}%
\pgfpathlineto{\pgfqpoint{2.534282in}{2.436919in}}%
\pgfpathlineto{\pgfqpoint{2.528269in}{2.445433in}}%
\pgfpathlineto{\pgfqpoint{2.522260in}{2.453937in}}%
\pgfpathlineto{\pgfqpoint{2.516255in}{2.462431in}}%
\pgfpathlineto{\pgfqpoint{2.510254in}{2.470912in}}%
\pgfpathlineto{\pgfqpoint{2.504257in}{2.479377in}}%
\pgfpathlineto{\pgfqpoint{2.492856in}{2.476320in}}%
\pgfpathlineto{\pgfqpoint{2.481451in}{2.473161in}}%
\pgfpathlineto{\pgfqpoint{2.470042in}{2.469878in}}%
\pgfpathlineto{\pgfqpoint{2.458629in}{2.466496in}}%
\pgfpathlineto{\pgfqpoint{2.447211in}{2.463042in}}%
\pgfpathlineto{\pgfqpoint{2.453196in}{2.454607in}}%
\pgfpathlineto{\pgfqpoint{2.459185in}{2.446162in}}%
\pgfpathlineto{\pgfqpoint{2.465178in}{2.437704in}}%
\pgfpathlineto{\pgfqpoint{2.471175in}{2.429233in}}%
\pgfpathclose%
\pgfusepath{stroke,fill}%
\end{pgfscope}%
\begin{pgfscope}%
\pgfpathrectangle{\pgfqpoint{0.887500in}{0.275000in}}{\pgfqpoint{4.225000in}{4.225000in}}%
\pgfusepath{clip}%
\pgfsetbuttcap%
\pgfsetroundjoin%
\definecolor{currentfill}{rgb}{0.122606,0.585371,0.546557}%
\pgfsetfillcolor{currentfill}%
\pgfsetfillopacity{0.700000}%
\pgfsetlinewidth{0.501875pt}%
\definecolor{currentstroke}{rgb}{1.000000,1.000000,1.000000}%
\pgfsetstrokecolor{currentstroke}%
\pgfsetstrokeopacity{0.500000}%
\pgfsetdash{}{0pt}%
\pgfpathmoveto{\pgfqpoint{1.608200in}{2.567575in}}%
\pgfpathlineto{\pgfqpoint{1.619847in}{2.570861in}}%
\pgfpathlineto{\pgfqpoint{1.631487in}{2.574144in}}%
\pgfpathlineto{\pgfqpoint{1.643123in}{2.577426in}}%
\pgfpathlineto{\pgfqpoint{1.654752in}{2.580708in}}%
\pgfpathlineto{\pgfqpoint{1.666375in}{2.583989in}}%
\pgfpathlineto{\pgfqpoint{1.660664in}{2.591917in}}%
\pgfpathlineto{\pgfqpoint{1.654957in}{2.599821in}}%
\pgfpathlineto{\pgfqpoint{1.649254in}{2.607700in}}%
\pgfpathlineto{\pgfqpoint{1.643556in}{2.615554in}}%
\pgfpathlineto{\pgfqpoint{1.637863in}{2.623383in}}%
\pgfpathlineto{\pgfqpoint{1.626253in}{2.620108in}}%
\pgfpathlineto{\pgfqpoint{1.614636in}{2.616833in}}%
\pgfpathlineto{\pgfqpoint{1.603014in}{2.613557in}}%
\pgfpathlineto{\pgfqpoint{1.591387in}{2.610280in}}%
\pgfpathlineto{\pgfqpoint{1.579754in}{2.606999in}}%
\pgfpathlineto{\pgfqpoint{1.585434in}{2.599168in}}%
\pgfpathlineto{\pgfqpoint{1.591118in}{2.591311in}}%
\pgfpathlineto{\pgfqpoint{1.596807in}{2.583426in}}%
\pgfpathlineto{\pgfqpoint{1.602501in}{2.575514in}}%
\pgfpathclose%
\pgfusepath{stroke,fill}%
\end{pgfscope}%
\begin{pgfscope}%
\pgfpathrectangle{\pgfqpoint{0.887500in}{0.275000in}}{\pgfqpoint{4.225000in}{4.225000in}}%
\pgfusepath{clip}%
\pgfsetbuttcap%
\pgfsetroundjoin%
\definecolor{currentfill}{rgb}{0.119512,0.607464,0.540218}%
\pgfsetfillcolor{currentfill}%
\pgfsetfillopacity{0.700000}%
\pgfsetlinewidth{0.501875pt}%
\definecolor{currentstroke}{rgb}{1.000000,1.000000,1.000000}%
\pgfsetstrokecolor{currentstroke}%
\pgfsetstrokeopacity{0.500000}%
\pgfsetdash{}{0pt}%
\pgfpathmoveto{\pgfqpoint{4.378120in}{2.599418in}}%
\pgfpathlineto{\pgfqpoint{4.389082in}{2.602746in}}%
\pgfpathlineto{\pgfqpoint{4.400039in}{2.606075in}}%
\pgfpathlineto{\pgfqpoint{4.410990in}{2.609405in}}%
\pgfpathlineto{\pgfqpoint{4.421936in}{2.612739in}}%
\pgfpathlineto{\pgfqpoint{4.415508in}{2.627048in}}%
\pgfpathlineto{\pgfqpoint{4.409083in}{2.641361in}}%
\pgfpathlineto{\pgfqpoint{4.402659in}{2.655674in}}%
\pgfpathlineto{\pgfqpoint{4.396239in}{2.669982in}}%
\pgfpathlineto{\pgfqpoint{4.389820in}{2.684279in}}%
\pgfpathlineto{\pgfqpoint{4.378874in}{2.680911in}}%
\pgfpathlineto{\pgfqpoint{4.367923in}{2.677546in}}%
\pgfpathlineto{\pgfqpoint{4.356966in}{2.674181in}}%
\pgfpathlineto{\pgfqpoint{4.346004in}{2.670817in}}%
\pgfpathlineto{\pgfqpoint{4.352424in}{2.656571in}}%
\pgfpathlineto{\pgfqpoint{4.358845in}{2.642306in}}%
\pgfpathlineto{\pgfqpoint{4.365268in}{2.628023in}}%
\pgfpathlineto{\pgfqpoint{4.371693in}{2.613726in}}%
\pgfpathclose%
\pgfusepath{stroke,fill}%
\end{pgfscope}%
\begin{pgfscope}%
\pgfpathrectangle{\pgfqpoint{0.887500in}{0.275000in}}{\pgfqpoint{4.225000in}{4.225000in}}%
\pgfusepath{clip}%
\pgfsetbuttcap%
\pgfsetroundjoin%
\definecolor{currentfill}{rgb}{0.239374,0.735588,0.455688}%
\pgfsetfillcolor{currentfill}%
\pgfsetfillopacity{0.700000}%
\pgfsetlinewidth{0.501875pt}%
\definecolor{currentstroke}{rgb}{1.000000,1.000000,1.000000}%
\pgfsetstrokecolor{currentstroke}%
\pgfsetstrokeopacity{0.500000}%
\pgfsetdash{}{0pt}%
\pgfpathmoveto{\pgfqpoint{3.997741in}{2.881000in}}%
\pgfpathlineto{\pgfqpoint{4.008805in}{2.884573in}}%
\pgfpathlineto{\pgfqpoint{4.019864in}{2.888158in}}%
\pgfpathlineto{\pgfqpoint{4.030918in}{2.891753in}}%
\pgfpathlineto{\pgfqpoint{4.041968in}{2.895355in}}%
\pgfpathlineto{\pgfqpoint{4.053012in}{2.898960in}}%
\pgfpathlineto{\pgfqpoint{4.046638in}{2.912739in}}%
\pgfpathlineto{\pgfqpoint{4.040266in}{2.926491in}}%
\pgfpathlineto{\pgfqpoint{4.033896in}{2.940214in}}%
\pgfpathlineto{\pgfqpoint{4.027528in}{2.953907in}}%
\pgfpathlineto{\pgfqpoint{4.021162in}{2.967565in}}%
\pgfpathlineto{\pgfqpoint{4.010122in}{2.964011in}}%
\pgfpathlineto{\pgfqpoint{3.999076in}{2.960437in}}%
\pgfpathlineto{\pgfqpoint{3.988025in}{2.956844in}}%
\pgfpathlineto{\pgfqpoint{3.976968in}{2.953240in}}%
\pgfpathlineto{\pgfqpoint{3.965906in}{2.949634in}}%
\pgfpathlineto{\pgfqpoint{3.972267in}{2.935877in}}%
\pgfpathlineto{\pgfqpoint{3.978631in}{2.922136in}}%
\pgfpathlineto{\pgfqpoint{3.984998in}{2.908414in}}%
\pgfpathlineto{\pgfqpoint{3.991368in}{2.894704in}}%
\pgfpathclose%
\pgfusepath{stroke,fill}%
\end{pgfscope}%
\begin{pgfscope}%
\pgfpathrectangle{\pgfqpoint{0.887500in}{0.275000in}}{\pgfqpoint{4.225000in}{4.225000in}}%
\pgfusepath{clip}%
\pgfsetbuttcap%
\pgfsetroundjoin%
\definecolor{currentfill}{rgb}{0.129933,0.559582,0.551864}%
\pgfsetfillcolor{currentfill}%
\pgfsetfillopacity{0.700000}%
\pgfsetlinewidth{0.501875pt}%
\definecolor{currentstroke}{rgb}{1.000000,1.000000,1.000000}%
\pgfsetstrokecolor{currentstroke}%
\pgfsetstrokeopacity{0.500000}%
\pgfsetdash{}{0pt}%
\pgfpathmoveto{\pgfqpoint{1.926816in}{2.512136in}}%
\pgfpathlineto{\pgfqpoint{1.938387in}{2.515395in}}%
\pgfpathlineto{\pgfqpoint{1.949952in}{2.518661in}}%
\pgfpathlineto{\pgfqpoint{1.961511in}{2.521936in}}%
\pgfpathlineto{\pgfqpoint{1.973064in}{2.525223in}}%
\pgfpathlineto{\pgfqpoint{1.984611in}{2.528524in}}%
\pgfpathlineto{\pgfqpoint{1.978785in}{2.536660in}}%
\pgfpathlineto{\pgfqpoint{1.972964in}{2.544781in}}%
\pgfpathlineto{\pgfqpoint{1.967146in}{2.552887in}}%
\pgfpathlineto{\pgfqpoint{1.961333in}{2.560979in}}%
\pgfpathlineto{\pgfqpoint{1.955524in}{2.569057in}}%
\pgfpathlineto{\pgfqpoint{1.943989in}{2.565766in}}%
\pgfpathlineto{\pgfqpoint{1.932449in}{2.562487in}}%
\pgfpathlineto{\pgfqpoint{1.920902in}{2.559220in}}%
\pgfpathlineto{\pgfqpoint{1.909350in}{2.555962in}}%
\pgfpathlineto{\pgfqpoint{1.897791in}{2.552712in}}%
\pgfpathlineto{\pgfqpoint{1.903588in}{2.544627in}}%
\pgfpathlineto{\pgfqpoint{1.909389in}{2.536527in}}%
\pgfpathlineto{\pgfqpoint{1.915194in}{2.528412in}}%
\pgfpathlineto{\pgfqpoint{1.921003in}{2.520282in}}%
\pgfpathclose%
\pgfusepath{stroke,fill}%
\end{pgfscope}%
\begin{pgfscope}%
\pgfpathrectangle{\pgfqpoint{0.887500in}{0.275000in}}{\pgfqpoint{4.225000in}{4.225000in}}%
\pgfusepath{clip}%
\pgfsetbuttcap%
\pgfsetroundjoin%
\definecolor{currentfill}{rgb}{0.122312,0.633153,0.530398}%
\pgfsetfillcolor{currentfill}%
\pgfsetfillopacity{0.700000}%
\pgfsetlinewidth{0.501875pt}%
\definecolor{currentstroke}{rgb}{1.000000,1.000000,1.000000}%
\pgfsetstrokecolor{currentstroke}%
\pgfsetstrokeopacity{0.500000}%
\pgfsetdash{}{0pt}%
\pgfpathmoveto{\pgfqpoint{4.291115in}{2.653950in}}%
\pgfpathlineto{\pgfqpoint{4.302104in}{2.657333in}}%
\pgfpathlineto{\pgfqpoint{4.313087in}{2.660710in}}%
\pgfpathlineto{\pgfqpoint{4.324065in}{2.664082in}}%
\pgfpathlineto{\pgfqpoint{4.335037in}{2.667451in}}%
\pgfpathlineto{\pgfqpoint{4.346004in}{2.670817in}}%
\pgfpathlineto{\pgfqpoint{4.339587in}{2.685039in}}%
\pgfpathlineto{\pgfqpoint{4.333171in}{2.699235in}}%
\pgfpathlineto{\pgfqpoint{4.326757in}{2.713401in}}%
\pgfpathlineto{\pgfqpoint{4.320344in}{2.727537in}}%
\pgfpathlineto{\pgfqpoint{4.313933in}{2.741646in}}%
\pgfpathlineto{\pgfqpoint{4.302965in}{2.738213in}}%
\pgfpathlineto{\pgfqpoint{4.291992in}{2.734784in}}%
\pgfpathlineto{\pgfqpoint{4.281014in}{2.731357in}}%
\pgfpathlineto{\pgfqpoint{4.270031in}{2.727930in}}%
\pgfpathlineto{\pgfqpoint{4.259043in}{2.724504in}}%
\pgfpathlineto{\pgfqpoint{4.265454in}{2.710447in}}%
\pgfpathlineto{\pgfqpoint{4.271867in}{2.696366in}}%
\pgfpathlineto{\pgfqpoint{4.278281in}{2.682255in}}%
\pgfpathlineto{\pgfqpoint{4.284697in}{2.668116in}}%
\pgfpathclose%
\pgfusepath{stroke,fill}%
\end{pgfscope}%
\begin{pgfscope}%
\pgfpathrectangle{\pgfqpoint{0.887500in}{0.275000in}}{\pgfqpoint{4.225000in}{4.225000in}}%
\pgfusepath{clip}%
\pgfsetbuttcap%
\pgfsetroundjoin%
\definecolor{currentfill}{rgb}{0.139147,0.533812,0.555298}%
\pgfsetfillcolor{currentfill}%
\pgfsetfillopacity{0.700000}%
\pgfsetlinewidth{0.501875pt}%
\definecolor{currentstroke}{rgb}{1.000000,1.000000,1.000000}%
\pgfsetstrokecolor{currentstroke}%
\pgfsetstrokeopacity{0.500000}%
\pgfsetdash{}{0pt}%
\pgfpathmoveto{\pgfqpoint{2.245571in}{2.454295in}}%
\pgfpathlineto{\pgfqpoint{2.257064in}{2.457568in}}%
\pgfpathlineto{\pgfqpoint{2.268552in}{2.460844in}}%
\pgfpathlineto{\pgfqpoint{2.280033in}{2.464126in}}%
\pgfpathlineto{\pgfqpoint{2.291509in}{2.467418in}}%
\pgfpathlineto{\pgfqpoint{2.302979in}{2.470724in}}%
\pgfpathlineto{\pgfqpoint{2.297042in}{2.479080in}}%
\pgfpathlineto{\pgfqpoint{2.291110in}{2.487421in}}%
\pgfpathlineto{\pgfqpoint{2.285181in}{2.495747in}}%
\pgfpathlineto{\pgfqpoint{2.279257in}{2.504059in}}%
\pgfpathlineto{\pgfqpoint{2.273336in}{2.512356in}}%
\pgfpathlineto{\pgfqpoint{2.261878in}{2.509059in}}%
\pgfpathlineto{\pgfqpoint{2.250414in}{2.505771in}}%
\pgfpathlineto{\pgfqpoint{2.238945in}{2.502490in}}%
\pgfpathlineto{\pgfqpoint{2.227469in}{2.499214in}}%
\pgfpathlineto{\pgfqpoint{2.215988in}{2.495940in}}%
\pgfpathlineto{\pgfqpoint{2.221896in}{2.487640in}}%
\pgfpathlineto{\pgfqpoint{2.227809in}{2.479325in}}%
\pgfpathlineto{\pgfqpoint{2.233726in}{2.470996in}}%
\pgfpathlineto{\pgfqpoint{2.239646in}{2.462652in}}%
\pgfpathclose%
\pgfusepath{stroke,fill}%
\end{pgfscope}%
\begin{pgfscope}%
\pgfpathrectangle{\pgfqpoint{0.887500in}{0.275000in}}{\pgfqpoint{4.225000in}{4.225000in}}%
\pgfusepath{clip}%
\pgfsetbuttcap%
\pgfsetroundjoin%
\definecolor{currentfill}{rgb}{0.288921,0.758394,0.428426}%
\pgfsetfillcolor{currentfill}%
\pgfsetfillopacity{0.700000}%
\pgfsetlinewidth{0.501875pt}%
\definecolor{currentstroke}{rgb}{1.000000,1.000000,1.000000}%
\pgfsetstrokecolor{currentstroke}%
\pgfsetstrokeopacity{0.500000}%
\pgfsetdash{}{0pt}%
\pgfpathmoveto{\pgfqpoint{3.910522in}{2.931809in}}%
\pgfpathlineto{\pgfqpoint{3.921608in}{2.935326in}}%
\pgfpathlineto{\pgfqpoint{3.932690in}{2.938870in}}%
\pgfpathlineto{\pgfqpoint{3.943767in}{2.942441in}}%
\pgfpathlineto{\pgfqpoint{3.954839in}{2.946031in}}%
\pgfpathlineto{\pgfqpoint{3.965906in}{2.949634in}}%
\pgfpathlineto{\pgfqpoint{3.959547in}{2.963387in}}%
\pgfpathlineto{\pgfqpoint{3.953190in}{2.977116in}}%
\pgfpathlineto{\pgfqpoint{3.946835in}{2.990796in}}%
\pgfpathlineto{\pgfqpoint{3.940481in}{3.004406in}}%
\pgfpathlineto{\pgfqpoint{3.934127in}{3.017924in}}%
\pgfpathlineto{\pgfqpoint{3.923064in}{3.014396in}}%
\pgfpathlineto{\pgfqpoint{3.911996in}{3.010853in}}%
\pgfpathlineto{\pgfqpoint{3.900922in}{3.007309in}}%
\pgfpathlineto{\pgfqpoint{3.889844in}{3.003775in}}%
\pgfpathlineto{\pgfqpoint{3.878761in}{3.000252in}}%
\pgfpathlineto{\pgfqpoint{3.885110in}{2.986688in}}%
\pgfpathlineto{\pgfqpoint{3.891461in}{2.973036in}}%
\pgfpathlineto{\pgfqpoint{3.897813in}{2.959322in}}%
\pgfpathlineto{\pgfqpoint{3.904166in}{2.945571in}}%
\pgfpathclose%
\pgfusepath{stroke,fill}%
\end{pgfscope}%
\begin{pgfscope}%
\pgfpathrectangle{\pgfqpoint{0.887500in}{0.275000in}}{\pgfqpoint{4.225000in}{4.225000in}}%
\pgfusepath{clip}%
\pgfsetbuttcap%
\pgfsetroundjoin%
\definecolor{currentfill}{rgb}{0.150476,0.504369,0.557430}%
\pgfsetfillcolor{currentfill}%
\pgfsetfillopacity{0.700000}%
\pgfsetlinewidth{0.501875pt}%
\definecolor{currentstroke}{rgb}{1.000000,1.000000,1.000000}%
\pgfsetstrokecolor{currentstroke}%
\pgfsetstrokeopacity{0.500000}%
\pgfsetdash{}{0pt}%
\pgfpathmoveto{\pgfqpoint{2.564406in}{2.394195in}}%
\pgfpathlineto{\pgfqpoint{2.575826in}{2.397082in}}%
\pgfpathlineto{\pgfqpoint{2.587239in}{2.400039in}}%
\pgfpathlineto{\pgfqpoint{2.598644in}{2.403173in}}%
\pgfpathlineto{\pgfqpoint{2.610040in}{2.406595in}}%
\pgfpathlineto{\pgfqpoint{2.621426in}{2.410414in}}%
\pgfpathlineto{\pgfqpoint{2.615382in}{2.418978in}}%
\pgfpathlineto{\pgfqpoint{2.609342in}{2.427540in}}%
\pgfpathlineto{\pgfqpoint{2.603305in}{2.436102in}}%
\pgfpathlineto{\pgfqpoint{2.597273in}{2.444668in}}%
\pgfpathlineto{\pgfqpoint{2.591244in}{2.453241in}}%
\pgfpathlineto{\pgfqpoint{2.579869in}{2.449450in}}%
\pgfpathlineto{\pgfqpoint{2.568484in}{2.446022in}}%
\pgfpathlineto{\pgfqpoint{2.557091in}{2.442857in}}%
\pgfpathlineto{\pgfqpoint{2.545690in}{2.439856in}}%
\pgfpathlineto{\pgfqpoint{2.534282in}{2.436919in}}%
\pgfpathlineto{\pgfqpoint{2.540298in}{2.428395in}}%
\pgfpathlineto{\pgfqpoint{2.546319in}{2.419862in}}%
\pgfpathlineto{\pgfqpoint{2.552344in}{2.411319in}}%
\pgfpathlineto{\pgfqpoint{2.558373in}{2.402763in}}%
\pgfpathclose%
\pgfusepath{stroke,fill}%
\end{pgfscope}%
\begin{pgfscope}%
\pgfpathrectangle{\pgfqpoint{0.887500in}{0.275000in}}{\pgfqpoint{4.225000in}{4.225000in}}%
\pgfusepath{clip}%
\pgfsetbuttcap%
\pgfsetroundjoin%
\definecolor{currentfill}{rgb}{0.751884,0.874951,0.143228}%
\pgfsetfillcolor{currentfill}%
\pgfsetfillopacity{0.700000}%
\pgfsetlinewidth{0.501875pt}%
\definecolor{currentstroke}{rgb}{1.000000,1.000000,1.000000}%
\pgfsetstrokecolor{currentstroke}%
\pgfsetstrokeopacity{0.500000}%
\pgfsetdash{}{0pt}%
\pgfpathmoveto{\pgfqpoint{3.123727in}{3.324746in}}%
\pgfpathlineto{\pgfqpoint{3.135005in}{3.328598in}}%
\pgfpathlineto{\pgfqpoint{3.146275in}{3.331550in}}%
\pgfpathlineto{\pgfqpoint{3.157539in}{3.333859in}}%
\pgfpathlineto{\pgfqpoint{3.168796in}{3.335780in}}%
\pgfpathlineto{\pgfqpoint{3.180045in}{3.337568in}}%
\pgfpathlineto{\pgfqpoint{3.173826in}{3.347780in}}%
\pgfpathlineto{\pgfqpoint{3.167609in}{3.357859in}}%
\pgfpathlineto{\pgfqpoint{3.161395in}{3.367780in}}%
\pgfpathlineto{\pgfqpoint{3.155185in}{3.377518in}}%
\pgfpathlineto{\pgfqpoint{3.148977in}{3.387050in}}%
\pgfpathlineto{\pgfqpoint{3.137735in}{3.385335in}}%
\pgfpathlineto{\pgfqpoint{3.126486in}{3.383438in}}%
\pgfpathlineto{\pgfqpoint{3.115230in}{3.381087in}}%
\pgfpathlineto{\pgfqpoint{3.103968in}{3.378009in}}%
\pgfpathlineto{\pgfqpoint{3.092699in}{3.373932in}}%
\pgfpathlineto{\pgfqpoint{3.098898in}{3.364557in}}%
\pgfpathlineto{\pgfqpoint{3.105100in}{3.354940in}}%
\pgfpathlineto{\pgfqpoint{3.111306in}{3.345092in}}%
\pgfpathlineto{\pgfqpoint{3.117515in}{3.335024in}}%
\pgfpathclose%
\pgfusepath{stroke,fill}%
\end{pgfscope}%
\begin{pgfscope}%
\pgfpathrectangle{\pgfqpoint{0.887500in}{0.275000in}}{\pgfqpoint{4.225000in}{4.225000in}}%
\pgfusepath{clip}%
\pgfsetbuttcap%
\pgfsetroundjoin%
\definecolor{currentfill}{rgb}{0.699415,0.867117,0.175971}%
\pgfsetfillcolor{currentfill}%
\pgfsetfillopacity{0.700000}%
\pgfsetlinewidth{0.501875pt}%
\definecolor{currentstroke}{rgb}{1.000000,1.000000,1.000000}%
\pgfsetstrokecolor{currentstroke}%
\pgfsetstrokeopacity{0.500000}%
\pgfsetdash{}{0pt}%
\pgfpathmoveto{\pgfqpoint{3.211192in}{3.285340in}}%
\pgfpathlineto{\pgfqpoint{3.222445in}{3.287706in}}%
\pgfpathlineto{\pgfqpoint{3.233692in}{3.290226in}}%
\pgfpathlineto{\pgfqpoint{3.244934in}{3.292890in}}%
\pgfpathlineto{\pgfqpoint{3.256171in}{3.295685in}}%
\pgfpathlineto{\pgfqpoint{3.267403in}{3.298597in}}%
\pgfpathlineto{\pgfqpoint{3.261158in}{3.308634in}}%
\pgfpathlineto{\pgfqpoint{3.254916in}{3.318739in}}%
\pgfpathlineto{\pgfqpoint{3.248679in}{3.328882in}}%
\pgfpathlineto{\pgfqpoint{3.242444in}{3.339028in}}%
\pgfpathlineto{\pgfqpoint{3.236213in}{3.349142in}}%
\pgfpathlineto{\pgfqpoint{3.224990in}{3.346425in}}%
\pgfpathlineto{\pgfqpoint{3.213762in}{3.343902in}}%
\pgfpathlineto{\pgfqpoint{3.202528in}{3.341581in}}%
\pgfpathlineto{\pgfqpoint{3.191290in}{3.339468in}}%
\pgfpathlineto{\pgfqpoint{3.180045in}{3.337568in}}%
\pgfpathlineto{\pgfqpoint{3.186268in}{3.327245in}}%
\pgfpathlineto{\pgfqpoint{3.192494in}{3.316837in}}%
\pgfpathlineto{\pgfqpoint{3.198724in}{3.306368in}}%
\pgfpathlineto{\pgfqpoint{3.204956in}{3.295861in}}%
\pgfpathclose%
\pgfusepath{stroke,fill}%
\end{pgfscope}%
\begin{pgfscope}%
\pgfpathrectangle{\pgfqpoint{0.887500in}{0.275000in}}{\pgfqpoint{4.225000in}{4.225000in}}%
\pgfusepath{clip}%
\pgfsetbuttcap%
\pgfsetroundjoin%
\definecolor{currentfill}{rgb}{0.335885,0.777018,0.402049}%
\pgfsetfillcolor{currentfill}%
\pgfsetfillopacity{0.700000}%
\pgfsetlinewidth{0.501875pt}%
\definecolor{currentstroke}{rgb}{1.000000,1.000000,1.000000}%
\pgfsetstrokecolor{currentstroke}%
\pgfsetstrokeopacity{0.500000}%
\pgfsetdash{}{0pt}%
\pgfpathmoveto{\pgfqpoint{3.823265in}{2.982641in}}%
\pgfpathlineto{\pgfqpoint{3.834375in}{2.986181in}}%
\pgfpathlineto{\pgfqpoint{3.845479in}{2.989707in}}%
\pgfpathlineto{\pgfqpoint{3.856578in}{2.993224in}}%
\pgfpathlineto{\pgfqpoint{3.867672in}{2.996737in}}%
\pgfpathlineto{\pgfqpoint{3.878761in}{3.000252in}}%
\pgfpathlineto{\pgfqpoint{3.872411in}{3.013704in}}%
\pgfpathlineto{\pgfqpoint{3.866062in}{3.027017in}}%
\pgfpathlineto{\pgfqpoint{3.859711in}{3.040166in}}%
\pgfpathlineto{\pgfqpoint{3.853360in}{3.053125in}}%
\pgfpathlineto{\pgfqpoint{3.847007in}{3.065873in}}%
\pgfpathlineto{\pgfqpoint{3.835923in}{3.062387in}}%
\pgfpathlineto{\pgfqpoint{3.824834in}{3.058900in}}%
\pgfpathlineto{\pgfqpoint{3.813739in}{3.055406in}}%
\pgfpathlineto{\pgfqpoint{3.802638in}{3.051898in}}%
\pgfpathlineto{\pgfqpoint{3.791532in}{3.048371in}}%
\pgfpathlineto{\pgfqpoint{3.797879in}{3.035553in}}%
\pgfpathlineto{\pgfqpoint{3.804226in}{3.022551in}}%
\pgfpathlineto{\pgfqpoint{3.810572in}{3.009384in}}%
\pgfpathlineto{\pgfqpoint{3.816919in}{2.996073in}}%
\pgfpathclose%
\pgfusepath{stroke,fill}%
\end{pgfscope}%
\begin{pgfscope}%
\pgfpathrectangle{\pgfqpoint{0.887500in}{0.275000in}}{\pgfqpoint{4.225000in}{4.225000in}}%
\pgfusepath{clip}%
\pgfsetbuttcap%
\pgfsetroundjoin%
\definecolor{currentfill}{rgb}{0.125394,0.574318,0.549086}%
\pgfsetfillcolor{currentfill}%
\pgfsetfillopacity{0.700000}%
\pgfsetlinewidth{0.501875pt}%
\definecolor{currentstroke}{rgb}{1.000000,1.000000,1.000000}%
\pgfsetstrokecolor{currentstroke}%
\pgfsetstrokeopacity{0.500000}%
\pgfsetdash{}{0pt}%
\pgfpathmoveto{\pgfqpoint{1.695000in}{2.544029in}}%
\pgfpathlineto{\pgfqpoint{1.706631in}{2.547326in}}%
\pgfpathlineto{\pgfqpoint{1.718255in}{2.550621in}}%
\pgfpathlineto{\pgfqpoint{1.729874in}{2.553912in}}%
\pgfpathlineto{\pgfqpoint{1.741488in}{2.557197in}}%
\pgfpathlineto{\pgfqpoint{1.753096in}{2.560475in}}%
\pgfpathlineto{\pgfqpoint{1.747350in}{2.568490in}}%
\pgfpathlineto{\pgfqpoint{1.741608in}{2.576486in}}%
\pgfpathlineto{\pgfqpoint{1.735870in}{2.584464in}}%
\pgfpathlineto{\pgfqpoint{1.730137in}{2.592423in}}%
\pgfpathlineto{\pgfqpoint{1.724408in}{2.600362in}}%
\pgfpathlineto{\pgfqpoint{1.712813in}{2.597096in}}%
\pgfpathlineto{\pgfqpoint{1.701212in}{2.593824in}}%
\pgfpathlineto{\pgfqpoint{1.689605in}{2.590549in}}%
\pgfpathlineto{\pgfqpoint{1.677993in}{2.587270in}}%
\pgfpathlineto{\pgfqpoint{1.666375in}{2.583989in}}%
\pgfpathlineto{\pgfqpoint{1.672091in}{2.576039in}}%
\pgfpathlineto{\pgfqpoint{1.677812in}{2.568066in}}%
\pgfpathlineto{\pgfqpoint{1.683537in}{2.560073in}}%
\pgfpathlineto{\pgfqpoint{1.689266in}{2.552060in}}%
\pgfpathclose%
\pgfusepath{stroke,fill}%
\end{pgfscope}%
\begin{pgfscope}%
\pgfpathrectangle{\pgfqpoint{0.887500in}{0.275000in}}{\pgfqpoint{4.225000in}{4.225000in}}%
\pgfusepath{clip}%
\pgfsetbuttcap%
\pgfsetroundjoin%
\definecolor{currentfill}{rgb}{0.134692,0.658636,0.517649}%
\pgfsetfillcolor{currentfill}%
\pgfsetfillopacity{0.700000}%
\pgfsetlinewidth{0.501875pt}%
\definecolor{currentstroke}{rgb}{1.000000,1.000000,1.000000}%
\pgfsetstrokecolor{currentstroke}%
\pgfsetstrokeopacity{0.500000}%
\pgfsetdash{}{0pt}%
\pgfpathmoveto{\pgfqpoint{4.204022in}{2.707334in}}%
\pgfpathlineto{\pgfqpoint{4.215037in}{2.710777in}}%
\pgfpathlineto{\pgfqpoint{4.226046in}{2.714214in}}%
\pgfpathlineto{\pgfqpoint{4.237051in}{2.717647in}}%
\pgfpathlineto{\pgfqpoint{4.248049in}{2.721077in}}%
\pgfpathlineto{\pgfqpoint{4.259043in}{2.724504in}}%
\pgfpathlineto{\pgfqpoint{4.252634in}{2.738542in}}%
\pgfpathlineto{\pgfqpoint{4.246227in}{2.752564in}}%
\pgfpathlineto{\pgfqpoint{4.239822in}{2.766576in}}%
\pgfpathlineto{\pgfqpoint{4.233420in}{2.780582in}}%
\pgfpathlineto{\pgfqpoint{4.227020in}{2.794587in}}%
\pgfpathlineto{\pgfqpoint{4.216027in}{2.791133in}}%
\pgfpathlineto{\pgfqpoint{4.205030in}{2.787681in}}%
\pgfpathlineto{\pgfqpoint{4.194027in}{2.784230in}}%
\pgfpathlineto{\pgfqpoint{4.183019in}{2.780780in}}%
\pgfpathlineto{\pgfqpoint{4.172006in}{2.777328in}}%
\pgfpathlineto{\pgfqpoint{4.178404in}{2.763346in}}%
\pgfpathlineto{\pgfqpoint{4.184805in}{2.749358in}}%
\pgfpathlineto{\pgfqpoint{4.191208in}{2.735363in}}%
\pgfpathlineto{\pgfqpoint{4.197614in}{2.721356in}}%
\pgfpathclose%
\pgfusepath{stroke,fill}%
\end{pgfscope}%
\begin{pgfscope}%
\pgfpathrectangle{\pgfqpoint{0.887500in}{0.275000in}}{\pgfqpoint{4.225000in}{4.225000in}}%
\pgfusepath{clip}%
\pgfsetbuttcap%
\pgfsetroundjoin%
\definecolor{currentfill}{rgb}{0.386433,0.794644,0.372886}%
\pgfsetfillcolor{currentfill}%
\pgfsetfillopacity{0.700000}%
\pgfsetlinewidth{0.501875pt}%
\definecolor{currentstroke}{rgb}{1.000000,1.000000,1.000000}%
\pgfsetstrokecolor{currentstroke}%
\pgfsetstrokeopacity{0.500000}%
\pgfsetdash{}{0pt}%
\pgfpathmoveto{\pgfqpoint{3.735918in}{3.030429in}}%
\pgfpathlineto{\pgfqpoint{3.747052in}{3.034044in}}%
\pgfpathlineto{\pgfqpoint{3.758180in}{3.037651in}}%
\pgfpathlineto{\pgfqpoint{3.769303in}{3.041244in}}%
\pgfpathlineto{\pgfqpoint{3.780420in}{3.044820in}}%
\pgfpathlineto{\pgfqpoint{3.791532in}{3.048371in}}%
\pgfpathlineto{\pgfqpoint{3.785184in}{3.061014in}}%
\pgfpathlineto{\pgfqpoint{3.778837in}{3.073506in}}%
\pgfpathlineto{\pgfqpoint{3.772490in}{3.085870in}}%
\pgfpathlineto{\pgfqpoint{3.766145in}{3.098131in}}%
\pgfpathlineto{\pgfqpoint{3.759801in}{3.110313in}}%
\pgfpathlineto{\pgfqpoint{3.748696in}{3.106918in}}%
\pgfpathlineto{\pgfqpoint{3.737586in}{3.103510in}}%
\pgfpathlineto{\pgfqpoint{3.726471in}{3.100099in}}%
\pgfpathlineto{\pgfqpoint{3.715351in}{3.096696in}}%
\pgfpathlineto{\pgfqpoint{3.704226in}{3.093312in}}%
\pgfpathlineto{\pgfqpoint{3.710561in}{3.080917in}}%
\pgfpathlineto{\pgfqpoint{3.716899in}{3.068444in}}%
\pgfpathlineto{\pgfqpoint{3.723237in}{3.055880in}}%
\pgfpathlineto{\pgfqpoint{3.729577in}{3.043212in}}%
\pgfpathclose%
\pgfusepath{stroke,fill}%
\end{pgfscope}%
\begin{pgfscope}%
\pgfpathrectangle{\pgfqpoint{0.887500in}{0.275000in}}{\pgfqpoint{4.225000in}{4.225000in}}%
\pgfusepath{clip}%
\pgfsetbuttcap%
\pgfsetroundjoin%
\definecolor{currentfill}{rgb}{0.647257,0.858400,0.209861}%
\pgfsetfillcolor{currentfill}%
\pgfsetfillopacity{0.700000}%
\pgfsetlinewidth{0.501875pt}%
\definecolor{currentstroke}{rgb}{1.000000,1.000000,1.000000}%
\pgfsetstrokecolor{currentstroke}%
\pgfsetstrokeopacity{0.500000}%
\pgfsetdash{}{0pt}%
\pgfpathmoveto{\pgfqpoint{3.298685in}{3.248507in}}%
\pgfpathlineto{\pgfqpoint{3.309921in}{3.251751in}}%
\pgfpathlineto{\pgfqpoint{3.321152in}{3.254949in}}%
\pgfpathlineto{\pgfqpoint{3.332376in}{3.258085in}}%
\pgfpathlineto{\pgfqpoint{3.343595in}{3.261159in}}%
\pgfpathlineto{\pgfqpoint{3.354808in}{3.264175in}}%
\pgfpathlineto{\pgfqpoint{3.348538in}{3.274312in}}%
\pgfpathlineto{\pgfqpoint{3.342270in}{3.284373in}}%
\pgfpathlineto{\pgfqpoint{3.336006in}{3.294395in}}%
\pgfpathlineto{\pgfqpoint{3.329745in}{3.304411in}}%
\pgfpathlineto{\pgfqpoint{3.323488in}{3.314456in}}%
\pgfpathlineto{\pgfqpoint{3.312281in}{3.311152in}}%
\pgfpathlineto{\pgfqpoint{3.301069in}{3.307902in}}%
\pgfpathlineto{\pgfqpoint{3.289852in}{3.304718in}}%
\pgfpathlineto{\pgfqpoint{3.278630in}{3.301612in}}%
\pgfpathlineto{\pgfqpoint{3.267403in}{3.298597in}}%
\pgfpathlineto{\pgfqpoint{3.273653in}{3.288616in}}%
\pgfpathlineto{\pgfqpoint{3.279906in}{3.278654in}}%
\pgfpathlineto{\pgfqpoint{3.286162in}{3.268674in}}%
\pgfpathlineto{\pgfqpoint{3.292422in}{3.258638in}}%
\pgfpathclose%
\pgfusepath{stroke,fill}%
\end{pgfscope}%
\begin{pgfscope}%
\pgfpathrectangle{\pgfqpoint{0.887500in}{0.275000in}}{\pgfqpoint{4.225000in}{4.225000in}}%
\pgfusepath{clip}%
\pgfsetbuttcap%
\pgfsetroundjoin%
\definecolor{currentfill}{rgb}{0.133743,0.548535,0.553541}%
\pgfsetfillcolor{currentfill}%
\pgfsetfillopacity{0.700000}%
\pgfsetlinewidth{0.501875pt}%
\definecolor{currentstroke}{rgb}{1.000000,1.000000,1.000000}%
\pgfsetstrokecolor{currentstroke}%
\pgfsetstrokeopacity{0.500000}%
\pgfsetdash{}{0pt}%
\pgfpathmoveto{\pgfqpoint{2.013803in}{2.487591in}}%
\pgfpathlineto{\pgfqpoint{2.025356in}{2.490916in}}%
\pgfpathlineto{\pgfqpoint{2.036903in}{2.494249in}}%
\pgfpathlineto{\pgfqpoint{2.048444in}{2.497586in}}%
\pgfpathlineto{\pgfqpoint{2.059980in}{2.500925in}}%
\pgfpathlineto{\pgfqpoint{2.071510in}{2.504261in}}%
\pgfpathlineto{\pgfqpoint{2.065651in}{2.512473in}}%
\pgfpathlineto{\pgfqpoint{2.059796in}{2.520668in}}%
\pgfpathlineto{\pgfqpoint{2.053945in}{2.528845in}}%
\pgfpathlineto{\pgfqpoint{2.048099in}{2.537005in}}%
\pgfpathlineto{\pgfqpoint{2.042257in}{2.545150in}}%
\pgfpathlineto{\pgfqpoint{2.030739in}{2.541819in}}%
\pgfpathlineto{\pgfqpoint{2.019216in}{2.538487in}}%
\pgfpathlineto{\pgfqpoint{2.007687in}{2.535159in}}%
\pgfpathlineto{\pgfqpoint{1.996152in}{2.531837in}}%
\pgfpathlineto{\pgfqpoint{1.984611in}{2.528524in}}%
\pgfpathlineto{\pgfqpoint{1.990441in}{2.520372in}}%
\pgfpathlineto{\pgfqpoint{1.996275in}{2.512203in}}%
\pgfpathlineto{\pgfqpoint{2.002113in}{2.504017in}}%
\pgfpathlineto{\pgfqpoint{2.007956in}{2.495813in}}%
\pgfpathclose%
\pgfusepath{stroke,fill}%
\end{pgfscope}%
\begin{pgfscope}%
\pgfpathrectangle{\pgfqpoint{0.887500in}{0.275000in}}{\pgfqpoint{4.225000in}{4.225000in}}%
\pgfusepath{clip}%
\pgfsetbuttcap%
\pgfsetroundjoin%
\definecolor{currentfill}{rgb}{0.440137,0.811138,0.340967}%
\pgfsetfillcolor{currentfill}%
\pgfsetfillopacity{0.700000}%
\pgfsetlinewidth{0.501875pt}%
\definecolor{currentstroke}{rgb}{1.000000,1.000000,1.000000}%
\pgfsetstrokecolor{currentstroke}%
\pgfsetstrokeopacity{0.500000}%
\pgfsetdash{}{0pt}%
\pgfpathmoveto{\pgfqpoint{3.648528in}{3.076975in}}%
\pgfpathlineto{\pgfqpoint{3.659677in}{3.080167in}}%
\pgfpathlineto{\pgfqpoint{3.670821in}{3.083387in}}%
\pgfpathlineto{\pgfqpoint{3.681961in}{3.086648in}}%
\pgfpathlineto{\pgfqpoint{3.693096in}{3.089959in}}%
\pgfpathlineto{\pgfqpoint{3.704226in}{3.093312in}}%
\pgfpathlineto{\pgfqpoint{3.697892in}{3.105643in}}%
\pgfpathlineto{\pgfqpoint{3.691560in}{3.117921in}}%
\pgfpathlineto{\pgfqpoint{3.685231in}{3.130159in}}%
\pgfpathlineto{\pgfqpoint{3.678904in}{3.142371in}}%
\pgfpathlineto{\pgfqpoint{3.672580in}{3.154570in}}%
\pgfpathlineto{\pgfqpoint{3.661457in}{3.151379in}}%
\pgfpathlineto{\pgfqpoint{3.650330in}{3.148237in}}%
\pgfpathlineto{\pgfqpoint{3.639198in}{3.145154in}}%
\pgfpathlineto{\pgfqpoint{3.628061in}{3.142118in}}%
\pgfpathlineto{\pgfqpoint{3.616919in}{3.139111in}}%
\pgfpathlineto{\pgfqpoint{3.623238in}{3.126929in}}%
\pgfpathlineto{\pgfqpoint{3.629559in}{3.114615in}}%
\pgfpathlineto{\pgfqpoint{3.635880in}{3.102180in}}%
\pgfpathlineto{\pgfqpoint{3.642203in}{3.089630in}}%
\pgfpathclose%
\pgfusepath{stroke,fill}%
\end{pgfscope}%
\begin{pgfscope}%
\pgfpathrectangle{\pgfqpoint{0.887500in}{0.275000in}}{\pgfqpoint{4.225000in}{4.225000in}}%
\pgfusepath{clip}%
\pgfsetbuttcap%
\pgfsetroundjoin%
\definecolor{currentfill}{rgb}{0.143343,0.522773,0.556295}%
\pgfsetfillcolor{currentfill}%
\pgfsetfillopacity{0.700000}%
\pgfsetlinewidth{0.501875pt}%
\definecolor{currentstroke}{rgb}{1.000000,1.000000,1.000000}%
\pgfsetstrokecolor{currentstroke}%
\pgfsetstrokeopacity{0.500000}%
\pgfsetdash{}{0pt}%
\pgfpathmoveto{\pgfqpoint{2.332725in}{2.428717in}}%
\pgfpathlineto{\pgfqpoint{2.344200in}{2.432039in}}%
\pgfpathlineto{\pgfqpoint{2.355670in}{2.435381in}}%
\pgfpathlineto{\pgfqpoint{2.367134in}{2.438747in}}%
\pgfpathlineto{\pgfqpoint{2.378591in}{2.442139in}}%
\pgfpathlineto{\pgfqpoint{2.390043in}{2.445560in}}%
\pgfpathlineto{\pgfqpoint{2.384074in}{2.453995in}}%
\pgfpathlineto{\pgfqpoint{2.378109in}{2.462414in}}%
\pgfpathlineto{\pgfqpoint{2.372148in}{2.470817in}}%
\pgfpathlineto{\pgfqpoint{2.366191in}{2.479204in}}%
\pgfpathlineto{\pgfqpoint{2.360238in}{2.487574in}}%
\pgfpathlineto{\pgfqpoint{2.348798in}{2.484151in}}%
\pgfpathlineto{\pgfqpoint{2.337353in}{2.480757in}}%
\pgfpathlineto{\pgfqpoint{2.325901in}{2.477390in}}%
\pgfpathlineto{\pgfqpoint{2.314443in}{2.474047in}}%
\pgfpathlineto{\pgfqpoint{2.302979in}{2.470724in}}%
\pgfpathlineto{\pgfqpoint{2.308920in}{2.462353in}}%
\pgfpathlineto{\pgfqpoint{2.314865in}{2.453967in}}%
\pgfpathlineto{\pgfqpoint{2.320814in}{2.445565in}}%
\pgfpathlineto{\pgfqpoint{2.326767in}{2.437149in}}%
\pgfpathclose%
\pgfusepath{stroke,fill}%
\end{pgfscope}%
\begin{pgfscope}%
\pgfpathrectangle{\pgfqpoint{0.887500in}{0.275000in}}{\pgfqpoint{4.225000in}{4.225000in}}%
\pgfusepath{clip}%
\pgfsetbuttcap%
\pgfsetroundjoin%
\definecolor{currentfill}{rgb}{0.606045,0.850733,0.236712}%
\pgfsetfillcolor{currentfill}%
\pgfsetfillopacity{0.700000}%
\pgfsetlinewidth{0.501875pt}%
\definecolor{currentstroke}{rgb}{1.000000,1.000000,1.000000}%
\pgfsetstrokecolor{currentstroke}%
\pgfsetstrokeopacity{0.500000}%
\pgfsetdash{}{0pt}%
\pgfpathmoveto{\pgfqpoint{3.386196in}{3.211154in}}%
\pgfpathlineto{\pgfqpoint{3.397409in}{3.213985in}}%
\pgfpathlineto{\pgfqpoint{3.408615in}{3.216718in}}%
\pgfpathlineto{\pgfqpoint{3.419815in}{3.219367in}}%
\pgfpathlineto{\pgfqpoint{3.431008in}{3.221945in}}%
\pgfpathlineto{\pgfqpoint{3.442196in}{3.224470in}}%
\pgfpathlineto{\pgfqpoint{3.435908in}{3.235477in}}%
\pgfpathlineto{\pgfqpoint{3.429623in}{3.246372in}}%
\pgfpathlineto{\pgfqpoint{3.423341in}{3.257172in}}%
\pgfpathlineto{\pgfqpoint{3.417061in}{3.267892in}}%
\pgfpathlineto{\pgfqpoint{3.410783in}{3.278548in}}%
\pgfpathlineto{\pgfqpoint{3.399600in}{3.275750in}}%
\pgfpathlineto{\pgfqpoint{3.388411in}{3.272921in}}%
\pgfpathlineto{\pgfqpoint{3.377216in}{3.270051in}}%
\pgfpathlineto{\pgfqpoint{3.366015in}{3.267138in}}%
\pgfpathlineto{\pgfqpoint{3.354808in}{3.264175in}}%
\pgfpathlineto{\pgfqpoint{3.361082in}{3.253929in}}%
\pgfpathlineto{\pgfqpoint{3.367357in}{3.243539in}}%
\pgfpathlineto{\pgfqpoint{3.373635in}{3.232969in}}%
\pgfpathlineto{\pgfqpoint{3.379915in}{3.222186in}}%
\pgfpathclose%
\pgfusepath{stroke,fill}%
\end{pgfscope}%
\begin{pgfscope}%
\pgfpathrectangle{\pgfqpoint{0.887500in}{0.275000in}}{\pgfqpoint{4.225000in}{4.225000in}}%
\pgfusepath{clip}%
\pgfsetbuttcap%
\pgfsetroundjoin%
\definecolor{currentfill}{rgb}{0.496615,0.826376,0.306377}%
\pgfsetfillcolor{currentfill}%
\pgfsetfillopacity{0.700000}%
\pgfsetlinewidth{0.501875pt}%
\definecolor{currentstroke}{rgb}{1.000000,1.000000,1.000000}%
\pgfsetstrokecolor{currentstroke}%
\pgfsetstrokeopacity{0.500000}%
\pgfsetdash{}{0pt}%
\pgfpathmoveto{\pgfqpoint{3.561122in}{3.123852in}}%
\pgfpathlineto{\pgfqpoint{3.572294in}{3.126994in}}%
\pgfpathlineto{\pgfqpoint{3.583459in}{3.130074in}}%
\pgfpathlineto{\pgfqpoint{3.594618in}{3.133108in}}%
\pgfpathlineto{\pgfqpoint{3.605771in}{3.136114in}}%
\pgfpathlineto{\pgfqpoint{3.616919in}{3.139111in}}%
\pgfpathlineto{\pgfqpoint{3.610600in}{3.151159in}}%
\pgfpathlineto{\pgfqpoint{3.604283in}{3.163082in}}%
\pgfpathlineto{\pgfqpoint{3.597968in}{3.174888in}}%
\pgfpathlineto{\pgfqpoint{3.591654in}{3.186585in}}%
\pgfpathlineto{\pgfqpoint{3.585342in}{3.198183in}}%
\pgfpathlineto{\pgfqpoint{3.574195in}{3.194839in}}%
\pgfpathlineto{\pgfqpoint{3.563044in}{3.191546in}}%
\pgfpathlineto{\pgfqpoint{3.551888in}{3.188317in}}%
\pgfpathlineto{\pgfqpoint{3.540727in}{3.185169in}}%
\pgfpathlineto{\pgfqpoint{3.529562in}{3.182112in}}%
\pgfpathlineto{\pgfqpoint{3.535872in}{3.170764in}}%
\pgfpathlineto{\pgfqpoint{3.542183in}{3.159287in}}%
\pgfpathlineto{\pgfqpoint{3.548495in}{3.147658in}}%
\pgfpathlineto{\pgfqpoint{3.554808in}{3.135854in}}%
\pgfpathclose%
\pgfusepath{stroke,fill}%
\end{pgfscope}%
\begin{pgfscope}%
\pgfpathrectangle{\pgfqpoint{0.887500in}{0.275000in}}{\pgfqpoint{4.225000in}{4.225000in}}%
\pgfusepath{clip}%
\pgfsetbuttcap%
\pgfsetroundjoin%
\definecolor{currentfill}{rgb}{0.555484,0.840254,0.269281}%
\pgfsetfillcolor{currentfill}%
\pgfsetfillopacity{0.700000}%
\pgfsetlinewidth{0.501875pt}%
\definecolor{currentstroke}{rgb}{1.000000,1.000000,1.000000}%
\pgfsetstrokecolor{currentstroke}%
\pgfsetstrokeopacity{0.500000}%
\pgfsetdash{}{0pt}%
\pgfpathmoveto{\pgfqpoint{3.473663in}{3.167550in}}%
\pgfpathlineto{\pgfqpoint{3.484854in}{3.170436in}}%
\pgfpathlineto{\pgfqpoint{3.496039in}{3.173315in}}%
\pgfpathlineto{\pgfqpoint{3.507218in}{3.176206in}}%
\pgfpathlineto{\pgfqpoint{3.518393in}{3.179131in}}%
\pgfpathlineto{\pgfqpoint{3.529562in}{3.182112in}}%
\pgfpathlineto{\pgfqpoint{3.523255in}{3.193354in}}%
\pgfpathlineto{\pgfqpoint{3.516950in}{3.204512in}}%
\pgfpathlineto{\pgfqpoint{3.510647in}{3.215610in}}%
\pgfpathlineto{\pgfqpoint{3.504348in}{3.226670in}}%
\pgfpathlineto{\pgfqpoint{3.498051in}{3.237716in}}%
\pgfpathlineto{\pgfqpoint{3.486889in}{3.234838in}}%
\pgfpathlineto{\pgfqpoint{3.475723in}{3.232118in}}%
\pgfpathlineto{\pgfqpoint{3.464553in}{3.229513in}}%
\pgfpathlineto{\pgfqpoint{3.453377in}{3.226978in}}%
\pgfpathlineto{\pgfqpoint{3.442196in}{3.224470in}}%
\pgfpathlineto{\pgfqpoint{3.448485in}{3.213341in}}%
\pgfpathlineto{\pgfqpoint{3.454777in}{3.202086in}}%
\pgfpathlineto{\pgfqpoint{3.461070in}{3.190704in}}%
\pgfpathlineto{\pgfqpoint{3.467366in}{3.179192in}}%
\pgfpathclose%
\pgfusepath{stroke,fill}%
\end{pgfscope}%
\begin{pgfscope}%
\pgfpathrectangle{\pgfqpoint{0.887500in}{0.275000in}}{\pgfqpoint{4.225000in}{4.225000in}}%
\pgfusepath{clip}%
\pgfsetbuttcap%
\pgfsetroundjoin%
\definecolor{currentfill}{rgb}{0.137770,0.537492,0.554906}%
\pgfsetfillcolor{currentfill}%
\pgfsetfillopacity{0.700000}%
\pgfsetlinewidth{0.501875pt}%
\definecolor{currentstroke}{rgb}{1.000000,1.000000,1.000000}%
\pgfsetstrokecolor{currentstroke}%
\pgfsetstrokeopacity{0.500000}%
\pgfsetdash{}{0pt}%
\pgfpathmoveto{\pgfqpoint{2.852579in}{2.383137in}}%
\pgfpathlineto{\pgfqpoint{2.863800in}{2.408278in}}%
\pgfpathlineto{\pgfqpoint{2.875005in}{2.438167in}}%
\pgfpathlineto{\pgfqpoint{2.886206in}{2.471438in}}%
\pgfpathlineto{\pgfqpoint{2.897410in}{2.506721in}}%
\pgfpathlineto{\pgfqpoint{2.908624in}{2.542637in}}%
\pgfpathlineto{\pgfqpoint{2.902459in}{2.554906in}}%
\pgfpathlineto{\pgfqpoint{2.896297in}{2.566784in}}%
\pgfpathlineto{\pgfqpoint{2.890142in}{2.578088in}}%
\pgfpathlineto{\pgfqpoint{2.883992in}{2.588639in}}%
\pgfpathlineto{\pgfqpoint{2.877851in}{2.598257in}}%
\pgfpathlineto{\pgfqpoint{2.866675in}{2.561066in}}%
\pgfpathlineto{\pgfqpoint{2.855510in}{2.524418in}}%
\pgfpathlineto{\pgfqpoint{2.844348in}{2.489759in}}%
\pgfpathlineto{\pgfqpoint{2.833179in}{2.458525in}}%
\pgfpathlineto{\pgfqpoint{2.821992in}{2.432148in}}%
\pgfpathlineto{\pgfqpoint{2.828109in}{2.421123in}}%
\pgfpathlineto{\pgfqpoint{2.834226in}{2.410809in}}%
\pgfpathlineto{\pgfqpoint{2.840342in}{2.401105in}}%
\pgfpathlineto{\pgfqpoint{2.846460in}{2.391914in}}%
\pgfpathclose%
\pgfusepath{stroke,fill}%
\end{pgfscope}%
\begin{pgfscope}%
\pgfpathrectangle{\pgfqpoint{0.887500in}{0.275000in}}{\pgfqpoint{4.225000in}{4.225000in}}%
\pgfusepath{clip}%
\pgfsetbuttcap%
\pgfsetroundjoin%
\definecolor{currentfill}{rgb}{0.121148,0.592739,0.544641}%
\pgfsetfillcolor{currentfill}%
\pgfsetfillopacity{0.700000}%
\pgfsetlinewidth{0.501875pt}%
\definecolor{currentstroke}{rgb}{1.000000,1.000000,1.000000}%
\pgfsetstrokecolor{currentstroke}%
\pgfsetstrokeopacity{0.500000}%
\pgfsetdash{}{0pt}%
\pgfpathmoveto{\pgfqpoint{1.463118in}{2.573828in}}%
\pgfpathlineto{\pgfqpoint{1.474806in}{2.577189in}}%
\pgfpathlineto{\pgfqpoint{1.486488in}{2.580538in}}%
\pgfpathlineto{\pgfqpoint{1.498165in}{2.583875in}}%
\pgfpathlineto{\pgfqpoint{1.509837in}{2.587202in}}%
\pgfpathlineto{\pgfqpoint{1.521504in}{2.590520in}}%
\pgfpathlineto{\pgfqpoint{1.515842in}{2.598323in}}%
\pgfpathlineto{\pgfqpoint{1.510185in}{2.606101in}}%
\pgfpathlineto{\pgfqpoint{1.504533in}{2.613855in}}%
\pgfpathlineto{\pgfqpoint{1.498885in}{2.621585in}}%
\pgfpathlineto{\pgfqpoint{1.493242in}{2.629294in}}%
\pgfpathlineto{\pgfqpoint{1.481589in}{2.625974in}}%
\pgfpathlineto{\pgfqpoint{1.469931in}{2.622645in}}%
\pgfpathlineto{\pgfqpoint{1.458268in}{2.619306in}}%
\pgfpathlineto{\pgfqpoint{1.446599in}{2.615957in}}%
\pgfpathlineto{\pgfqpoint{1.434925in}{2.612596in}}%
\pgfpathlineto{\pgfqpoint{1.440555in}{2.604890in}}%
\pgfpathlineto{\pgfqpoint{1.446188in}{2.597161in}}%
\pgfpathlineto{\pgfqpoint{1.451827in}{2.589410in}}%
\pgfpathlineto{\pgfqpoint{1.457470in}{2.581632in}}%
\pgfpathclose%
\pgfusepath{stroke,fill}%
\end{pgfscope}%
\begin{pgfscope}%
\pgfpathrectangle{\pgfqpoint{0.887500in}{0.275000in}}{\pgfqpoint{4.225000in}{4.225000in}}%
\pgfusepath{clip}%
\pgfsetbuttcap%
\pgfsetroundjoin%
\definecolor{currentfill}{rgb}{0.157851,0.683765,0.501686}%
\pgfsetfillcolor{currentfill}%
\pgfsetfillopacity{0.700000}%
\pgfsetlinewidth{0.501875pt}%
\definecolor{currentstroke}{rgb}{1.000000,1.000000,1.000000}%
\pgfsetstrokecolor{currentstroke}%
\pgfsetstrokeopacity{0.500000}%
\pgfsetdash{}{0pt}%
\pgfpathmoveto{\pgfqpoint{4.116861in}{2.760085in}}%
\pgfpathlineto{\pgfqpoint{4.127900in}{2.763526in}}%
\pgfpathlineto{\pgfqpoint{4.138934in}{2.766973in}}%
\pgfpathlineto{\pgfqpoint{4.149963in}{2.770423in}}%
\pgfpathlineto{\pgfqpoint{4.160987in}{2.773876in}}%
\pgfpathlineto{\pgfqpoint{4.172006in}{2.777328in}}%
\pgfpathlineto{\pgfqpoint{4.165610in}{2.791309in}}%
\pgfpathlineto{\pgfqpoint{4.159216in}{2.805292in}}%
\pgfpathlineto{\pgfqpoint{4.152825in}{2.819280in}}%
\pgfpathlineto{\pgfqpoint{4.146437in}{2.833276in}}%
\pgfpathlineto{\pgfqpoint{4.140052in}{2.847275in}}%
\pgfpathlineto{\pgfqpoint{4.129034in}{2.843766in}}%
\pgfpathlineto{\pgfqpoint{4.118011in}{2.840254in}}%
\pgfpathlineto{\pgfqpoint{4.106982in}{2.836742in}}%
\pgfpathlineto{\pgfqpoint{4.095948in}{2.833233in}}%
\pgfpathlineto{\pgfqpoint{4.084910in}{2.829729in}}%
\pgfpathlineto{\pgfqpoint{4.091295in}{2.815828in}}%
\pgfpathlineto{\pgfqpoint{4.097684in}{2.801912in}}%
\pgfpathlineto{\pgfqpoint{4.104074in}{2.787983in}}%
\pgfpathlineto{\pgfqpoint{4.110466in}{2.774041in}}%
\pgfpathclose%
\pgfusepath{stroke,fill}%
\end{pgfscope}%
\begin{pgfscope}%
\pgfpathrectangle{\pgfqpoint{0.887500in}{0.275000in}}{\pgfqpoint{4.225000in}{4.225000in}}%
\pgfusepath{clip}%
\pgfsetbuttcap%
\pgfsetroundjoin%
\definecolor{currentfill}{rgb}{0.153364,0.497000,0.557724}%
\pgfsetfillcolor{currentfill}%
\pgfsetfillopacity{0.700000}%
\pgfsetlinewidth{0.501875pt}%
\definecolor{currentstroke}{rgb}{1.000000,1.000000,1.000000}%
\pgfsetstrokecolor{currentstroke}%
\pgfsetstrokeopacity{0.500000}%
\pgfsetdash{}{0pt}%
\pgfpathmoveto{\pgfqpoint{2.651707in}{2.367434in}}%
\pgfpathlineto{\pgfqpoint{2.663094in}{2.371786in}}%
\pgfpathlineto{\pgfqpoint{2.674473in}{2.376505in}}%
\pgfpathlineto{\pgfqpoint{2.685846in}{2.381224in}}%
\pgfpathlineto{\pgfqpoint{2.697219in}{2.385570in}}%
\pgfpathlineto{\pgfqpoint{2.708593in}{2.389169in}}%
\pgfpathlineto{\pgfqpoint{2.702514in}{2.398086in}}%
\pgfpathlineto{\pgfqpoint{2.696439in}{2.407027in}}%
\pgfpathlineto{\pgfqpoint{2.690367in}{2.415988in}}%
\pgfpathlineto{\pgfqpoint{2.684299in}{2.424962in}}%
\pgfpathlineto{\pgfqpoint{2.678235in}{2.433946in}}%
\pgfpathlineto{\pgfqpoint{2.666882in}{2.429323in}}%
\pgfpathlineto{\pgfqpoint{2.655527in}{2.424420in}}%
\pgfpathlineto{\pgfqpoint{2.644168in}{2.419477in}}%
\pgfpathlineto{\pgfqpoint{2.632802in}{2.414731in}}%
\pgfpathlineto{\pgfqpoint{2.621426in}{2.410414in}}%
\pgfpathlineto{\pgfqpoint{2.627474in}{2.401844in}}%
\pgfpathlineto{\pgfqpoint{2.633526in}{2.393264in}}%
\pgfpathlineto{\pgfqpoint{2.639582in}{2.384671in}}%
\pgfpathlineto{\pgfqpoint{2.645643in}{2.376062in}}%
\pgfpathclose%
\pgfusepath{stroke,fill}%
\end{pgfscope}%
\begin{pgfscope}%
\pgfpathrectangle{\pgfqpoint{0.887500in}{0.275000in}}{\pgfqpoint{4.225000in}{4.225000in}}%
\pgfusepath{clip}%
\pgfsetbuttcap%
\pgfsetroundjoin%
\definecolor{currentfill}{rgb}{0.180653,0.701402,0.488189}%
\pgfsetfillcolor{currentfill}%
\pgfsetfillopacity{0.700000}%
\pgfsetlinewidth{0.501875pt}%
\definecolor{currentstroke}{rgb}{1.000000,1.000000,1.000000}%
\pgfsetstrokecolor{currentstroke}%
\pgfsetstrokeopacity{0.500000}%
\pgfsetdash{}{0pt}%
\pgfpathmoveto{\pgfqpoint{2.903335in}{2.772806in}}%
\pgfpathlineto{\pgfqpoint{2.914579in}{2.799448in}}%
\pgfpathlineto{\pgfqpoint{2.925830in}{2.826224in}}%
\pgfpathlineto{\pgfqpoint{2.937087in}{2.853418in}}%
\pgfpathlineto{\pgfqpoint{2.948352in}{2.881168in}}%
\pgfpathlineto{\pgfqpoint{2.959624in}{2.909142in}}%
\pgfpathlineto{\pgfqpoint{2.953479in}{2.907529in}}%
\pgfpathlineto{\pgfqpoint{2.947343in}{2.905318in}}%
\pgfpathlineto{\pgfqpoint{2.941215in}{2.902786in}}%
\pgfpathlineto{\pgfqpoint{2.935094in}{2.900211in}}%
\pgfpathlineto{\pgfqpoint{2.928980in}{2.897870in}}%
\pgfpathlineto{\pgfqpoint{2.917756in}{2.868978in}}%
\pgfpathlineto{\pgfqpoint{2.906538in}{2.840987in}}%
\pgfpathlineto{\pgfqpoint{2.895324in}{2.814220in}}%
\pgfpathlineto{\pgfqpoint{2.884112in}{2.788664in}}%
\pgfpathlineto{\pgfqpoint{2.872902in}{2.764206in}}%
\pgfpathlineto{\pgfqpoint{2.878975in}{2.765699in}}%
\pgfpathlineto{\pgfqpoint{2.885054in}{2.767626in}}%
\pgfpathlineto{\pgfqpoint{2.891139in}{2.769667in}}%
\pgfpathlineto{\pgfqpoint{2.897232in}{2.771500in}}%
\pgfpathclose%
\pgfusepath{stroke,fill}%
\end{pgfscope}%
\begin{pgfscope}%
\pgfpathrectangle{\pgfqpoint{0.887500in}{0.275000in}}{\pgfqpoint{4.225000in}{4.225000in}}%
\pgfusepath{clip}%
\pgfsetbuttcap%
\pgfsetroundjoin%
\definecolor{currentfill}{rgb}{0.124395,0.578002,0.548287}%
\pgfsetfillcolor{currentfill}%
\pgfsetfillopacity{0.700000}%
\pgfsetlinewidth{0.501875pt}%
\definecolor{currentstroke}{rgb}{1.000000,1.000000,1.000000}%
\pgfsetstrokecolor{currentstroke}%
\pgfsetstrokeopacity{0.500000}%
\pgfsetdash{}{0pt}%
\pgfpathmoveto{\pgfqpoint{4.410292in}{2.527829in}}%
\pgfpathlineto{\pgfqpoint{4.421257in}{2.531215in}}%
\pgfpathlineto{\pgfqpoint{4.432217in}{2.534613in}}%
\pgfpathlineto{\pgfqpoint{4.443173in}{2.538023in}}%
\pgfpathlineto{\pgfqpoint{4.454123in}{2.541446in}}%
\pgfpathlineto{\pgfqpoint{4.447679in}{2.555660in}}%
\pgfpathlineto{\pgfqpoint{4.441239in}{2.569897in}}%
\pgfpathlineto{\pgfqpoint{4.434801in}{2.584159in}}%
\pgfpathlineto{\pgfqpoint{4.428367in}{2.598441in}}%
\pgfpathlineto{\pgfqpoint{4.421936in}{2.612739in}}%
\pgfpathlineto{\pgfqpoint{4.410990in}{2.609405in}}%
\pgfpathlineto{\pgfqpoint{4.400039in}{2.606075in}}%
\pgfpathlineto{\pgfqpoint{4.389082in}{2.602746in}}%
\pgfpathlineto{\pgfqpoint{4.378120in}{2.599418in}}%
\pgfpathlineto{\pgfqpoint{4.384549in}{2.585103in}}%
\pgfpathlineto{\pgfqpoint{4.390981in}{2.570783in}}%
\pgfpathlineto{\pgfqpoint{4.397415in}{2.556461in}}%
\pgfpathlineto{\pgfqpoint{4.403852in}{2.542141in}}%
\pgfpathclose%
\pgfusepath{stroke,fill}%
\end{pgfscope}%
\begin{pgfscope}%
\pgfpathrectangle{\pgfqpoint{0.887500in}{0.275000in}}{\pgfqpoint{4.225000in}{4.225000in}}%
\pgfusepath{clip}%
\pgfsetbuttcap%
\pgfsetroundjoin%
\definecolor{currentfill}{rgb}{0.127568,0.566949,0.550556}%
\pgfsetfillcolor{currentfill}%
\pgfsetfillopacity{0.700000}%
\pgfsetlinewidth{0.501875pt}%
\definecolor{currentstroke}{rgb}{1.000000,1.000000,1.000000}%
\pgfsetstrokecolor{currentstroke}%
\pgfsetstrokeopacity{0.500000}%
\pgfsetdash{}{0pt}%
\pgfpathmoveto{\pgfqpoint{1.781891in}{2.520167in}}%
\pgfpathlineto{\pgfqpoint{1.793506in}{2.523451in}}%
\pgfpathlineto{\pgfqpoint{1.805115in}{2.526724in}}%
\pgfpathlineto{\pgfqpoint{1.816720in}{2.529988in}}%
\pgfpathlineto{\pgfqpoint{1.828318in}{2.533244in}}%
\pgfpathlineto{\pgfqpoint{1.839911in}{2.536494in}}%
\pgfpathlineto{\pgfqpoint{1.834131in}{2.544573in}}%
\pgfpathlineto{\pgfqpoint{1.828356in}{2.552638in}}%
\pgfpathlineto{\pgfqpoint{1.822584in}{2.560689in}}%
\pgfpathlineto{\pgfqpoint{1.816817in}{2.568726in}}%
\pgfpathlineto{\pgfqpoint{1.811054in}{2.576748in}}%
\pgfpathlineto{\pgfqpoint{1.799473in}{2.573506in}}%
\pgfpathlineto{\pgfqpoint{1.787887in}{2.570259in}}%
\pgfpathlineto{\pgfqpoint{1.776296in}{2.567006in}}%
\pgfpathlineto{\pgfqpoint{1.764699in}{2.563745in}}%
\pgfpathlineto{\pgfqpoint{1.753096in}{2.560475in}}%
\pgfpathlineto{\pgfqpoint{1.758847in}{2.552444in}}%
\pgfpathlineto{\pgfqpoint{1.764601in}{2.544397in}}%
\pgfpathlineto{\pgfqpoint{1.770360in}{2.536335in}}%
\pgfpathlineto{\pgfqpoint{1.776123in}{2.528259in}}%
\pgfpathclose%
\pgfusepath{stroke,fill}%
\end{pgfscope}%
\begin{pgfscope}%
\pgfpathrectangle{\pgfqpoint{0.887500in}{0.275000in}}{\pgfqpoint{4.225000in}{4.225000in}}%
\pgfusepath{clip}%
\pgfsetbuttcap%
\pgfsetroundjoin%
\definecolor{currentfill}{rgb}{0.159194,0.482237,0.558073}%
\pgfsetfillcolor{currentfill}%
\pgfsetfillopacity{0.700000}%
\pgfsetlinewidth{0.501875pt}%
\definecolor{currentstroke}{rgb}{1.000000,1.000000,1.000000}%
\pgfsetstrokecolor{currentstroke}%
\pgfsetstrokeopacity{0.500000}%
\pgfsetdash{}{0pt}%
\pgfpathmoveto{\pgfqpoint{2.739041in}{2.344892in}}%
\pgfpathlineto{\pgfqpoint{2.750431in}{2.347139in}}%
\pgfpathlineto{\pgfqpoint{2.761826in}{2.347884in}}%
\pgfpathlineto{\pgfqpoint{2.773227in}{2.346844in}}%
\pgfpathlineto{\pgfqpoint{2.784627in}{2.344702in}}%
\pgfpathlineto{\pgfqpoint{2.796017in}{2.342786in}}%
\pgfpathlineto{\pgfqpoint{2.789916in}{2.351025in}}%
\pgfpathlineto{\pgfqpoint{2.783817in}{2.359602in}}%
\pgfpathlineto{\pgfqpoint{2.777718in}{2.368613in}}%
\pgfpathlineto{\pgfqpoint{2.771619in}{2.378156in}}%
\pgfpathlineto{\pgfqpoint{2.765520in}{2.388329in}}%
\pgfpathlineto{\pgfqpoint{2.754138in}{2.389939in}}%
\pgfpathlineto{\pgfqpoint{2.742747in}{2.391825in}}%
\pgfpathlineto{\pgfqpoint{2.731355in}{2.392628in}}%
\pgfpathlineto{\pgfqpoint{2.719971in}{2.391646in}}%
\pgfpathlineto{\pgfqpoint{2.708593in}{2.389169in}}%
\pgfpathlineto{\pgfqpoint{2.714675in}{2.380278in}}%
\pgfpathlineto{\pgfqpoint{2.720761in}{2.371410in}}%
\pgfpathlineto{\pgfqpoint{2.726851in}{2.362559in}}%
\pgfpathlineto{\pgfqpoint{2.732944in}{2.353721in}}%
\pgfpathclose%
\pgfusepath{stroke,fill}%
\end{pgfscope}%
\begin{pgfscope}%
\pgfpathrectangle{\pgfqpoint{0.887500in}{0.275000in}}{\pgfqpoint{4.225000in}{4.225000in}}%
\pgfusepath{clip}%
\pgfsetbuttcap%
\pgfsetroundjoin%
\definecolor{currentfill}{rgb}{0.196571,0.711827,0.479221}%
\pgfsetfillcolor{currentfill}%
\pgfsetfillopacity{0.700000}%
\pgfsetlinewidth{0.501875pt}%
\definecolor{currentstroke}{rgb}{1.000000,1.000000,1.000000}%
\pgfsetstrokecolor{currentstroke}%
\pgfsetstrokeopacity{0.500000}%
\pgfsetdash{}{0pt}%
\pgfpathmoveto{\pgfqpoint{4.029641in}{2.812357in}}%
\pgfpathlineto{\pgfqpoint{4.040705in}{2.815810in}}%
\pgfpathlineto{\pgfqpoint{4.051763in}{2.819274in}}%
\pgfpathlineto{\pgfqpoint{4.062817in}{2.822747in}}%
\pgfpathlineto{\pgfqpoint{4.073866in}{2.826233in}}%
\pgfpathlineto{\pgfqpoint{4.084910in}{2.829729in}}%
\pgfpathlineto{\pgfqpoint{4.078526in}{2.843614in}}%
\pgfpathlineto{\pgfqpoint{4.072144in}{2.857482in}}%
\pgfpathlineto{\pgfqpoint{4.065765in}{2.871330in}}%
\pgfpathlineto{\pgfqpoint{4.059387in}{2.885157in}}%
\pgfpathlineto{\pgfqpoint{4.053012in}{2.898960in}}%
\pgfpathlineto{\pgfqpoint{4.041968in}{2.895355in}}%
\pgfpathlineto{\pgfqpoint{4.030918in}{2.891753in}}%
\pgfpathlineto{\pgfqpoint{4.019864in}{2.888158in}}%
\pgfpathlineto{\pgfqpoint{4.008805in}{2.884573in}}%
\pgfpathlineto{\pgfqpoint{3.997741in}{2.881000in}}%
\pgfpathlineto{\pgfqpoint{4.004117in}{2.867297in}}%
\pgfpathlineto{\pgfqpoint{4.010494in}{2.853587in}}%
\pgfpathlineto{\pgfqpoint{4.016875in}{2.839864in}}%
\pgfpathlineto{\pgfqpoint{4.023257in}{2.826123in}}%
\pgfpathclose%
\pgfusepath{stroke,fill}%
\end{pgfscope}%
\begin{pgfscope}%
\pgfpathrectangle{\pgfqpoint{0.887500in}{0.275000in}}{\pgfqpoint{4.225000in}{4.225000in}}%
\pgfusepath{clip}%
\pgfsetbuttcap%
\pgfsetroundjoin%
\definecolor{currentfill}{rgb}{0.136408,0.541173,0.554483}%
\pgfsetfillcolor{currentfill}%
\pgfsetfillopacity{0.700000}%
\pgfsetlinewidth{0.501875pt}%
\definecolor{currentstroke}{rgb}{1.000000,1.000000,1.000000}%
\pgfsetstrokecolor{currentstroke}%
\pgfsetstrokeopacity{0.500000}%
\pgfsetdash{}{0pt}%
\pgfpathmoveto{\pgfqpoint{2.100869in}{2.462937in}}%
\pgfpathlineto{\pgfqpoint{2.112405in}{2.466277in}}%
\pgfpathlineto{\pgfqpoint{2.123937in}{2.469606in}}%
\pgfpathlineto{\pgfqpoint{2.135462in}{2.472925in}}%
\pgfpathlineto{\pgfqpoint{2.146983in}{2.476233in}}%
\pgfpathlineto{\pgfqpoint{2.158497in}{2.479533in}}%
\pgfpathlineto{\pgfqpoint{2.152605in}{2.487821in}}%
\pgfpathlineto{\pgfqpoint{2.146717in}{2.496094in}}%
\pgfpathlineto{\pgfqpoint{2.140833in}{2.504351in}}%
\pgfpathlineto{\pgfqpoint{2.134954in}{2.512592in}}%
\pgfpathlineto{\pgfqpoint{2.129078in}{2.520818in}}%
\pgfpathlineto{\pgfqpoint{2.117575in}{2.517524in}}%
\pgfpathlineto{\pgfqpoint{2.106067in}{2.514222in}}%
\pgfpathlineto{\pgfqpoint{2.094554in}{2.510911in}}%
\pgfpathlineto{\pgfqpoint{2.083035in}{2.507591in}}%
\pgfpathlineto{\pgfqpoint{2.071510in}{2.504261in}}%
\pgfpathlineto{\pgfqpoint{2.077373in}{2.496032in}}%
\pgfpathlineto{\pgfqpoint{2.083241in}{2.487785in}}%
\pgfpathlineto{\pgfqpoint{2.089113in}{2.479520in}}%
\pgfpathlineto{\pgfqpoint{2.094989in}{2.471237in}}%
\pgfpathclose%
\pgfusepath{stroke,fill}%
\end{pgfscope}%
\begin{pgfscope}%
\pgfpathrectangle{\pgfqpoint{0.887500in}{0.275000in}}{\pgfqpoint{4.225000in}{4.225000in}}%
\pgfusepath{clip}%
\pgfsetbuttcap%
\pgfsetroundjoin%
\definecolor{currentfill}{rgb}{0.626579,0.854645,0.223353}%
\pgfsetfillcolor{currentfill}%
\pgfsetfillopacity{0.700000}%
\pgfsetlinewidth{0.501875pt}%
\definecolor{currentstroke}{rgb}{1.000000,1.000000,1.000000}%
\pgfsetstrokecolor{currentstroke}%
\pgfsetstrokeopacity{0.500000}%
\pgfsetdash{}{0pt}%
\pgfpathmoveto{\pgfqpoint{3.010777in}{3.188668in}}%
\pgfpathlineto{\pgfqpoint{3.022062in}{3.212738in}}%
\pgfpathlineto{\pgfqpoint{3.033354in}{3.234310in}}%
\pgfpathlineto{\pgfqpoint{3.044652in}{3.253203in}}%
\pgfpathlineto{\pgfqpoint{3.055953in}{3.269568in}}%
\pgfpathlineto{\pgfqpoint{3.067255in}{3.283583in}}%
\pgfpathlineto{\pgfqpoint{3.061055in}{3.293061in}}%
\pgfpathlineto{\pgfqpoint{3.054858in}{3.302310in}}%
\pgfpathlineto{\pgfqpoint{3.048665in}{3.311328in}}%
\pgfpathlineto{\pgfqpoint{3.042476in}{3.320113in}}%
\pgfpathlineto{\pgfqpoint{3.036291in}{3.328661in}}%
\pgfpathlineto{\pgfqpoint{3.025009in}{3.312486in}}%
\pgfpathlineto{\pgfqpoint{3.013732in}{3.293203in}}%
\pgfpathlineto{\pgfqpoint{3.002462in}{3.270499in}}%
\pgfpathlineto{\pgfqpoint{2.991205in}{3.244118in}}%
\pgfpathlineto{\pgfqpoint{2.979960in}{3.214469in}}%
\pgfpathlineto{\pgfqpoint{2.986113in}{3.209788in}}%
\pgfpathlineto{\pgfqpoint{2.992271in}{3.204857in}}%
\pgfpathlineto{\pgfqpoint{2.998435in}{3.199686in}}%
\pgfpathlineto{\pgfqpoint{3.004603in}{3.194286in}}%
\pgfpathclose%
\pgfusepath{stroke,fill}%
\end{pgfscope}%
\begin{pgfscope}%
\pgfpathrectangle{\pgfqpoint{0.887500in}{0.275000in}}{\pgfqpoint{4.225000in}{4.225000in}}%
\pgfusepath{clip}%
\pgfsetbuttcap%
\pgfsetroundjoin%
\definecolor{currentfill}{rgb}{0.146180,0.515413,0.556823}%
\pgfsetfillcolor{currentfill}%
\pgfsetfillopacity{0.700000}%
\pgfsetlinewidth{0.501875pt}%
\definecolor{currentstroke}{rgb}{1.000000,1.000000,1.000000}%
\pgfsetstrokecolor{currentstroke}%
\pgfsetstrokeopacity{0.500000}%
\pgfsetdash{}{0pt}%
\pgfpathmoveto{\pgfqpoint{2.419950in}{2.403165in}}%
\pgfpathlineto{\pgfqpoint{2.431407in}{2.406628in}}%
\pgfpathlineto{\pgfqpoint{2.442858in}{2.410135in}}%
\pgfpathlineto{\pgfqpoint{2.454303in}{2.413677in}}%
\pgfpathlineto{\pgfqpoint{2.465742in}{2.417225in}}%
\pgfpathlineto{\pgfqpoint{2.477176in}{2.420748in}}%
\pgfpathlineto{\pgfqpoint{2.471175in}{2.429233in}}%
\pgfpathlineto{\pgfqpoint{2.465178in}{2.437704in}}%
\pgfpathlineto{\pgfqpoint{2.459185in}{2.446162in}}%
\pgfpathlineto{\pgfqpoint{2.453196in}{2.454607in}}%
\pgfpathlineto{\pgfqpoint{2.447211in}{2.463042in}}%
\pgfpathlineto{\pgfqpoint{2.435789in}{2.459541in}}%
\pgfpathlineto{\pgfqpoint{2.424361in}{2.456018in}}%
\pgfpathlineto{\pgfqpoint{2.412928in}{2.452501in}}%
\pgfpathlineto{\pgfqpoint{2.401488in}{2.449013in}}%
\pgfpathlineto{\pgfqpoint{2.390043in}{2.445560in}}%
\pgfpathlineto{\pgfqpoint{2.396016in}{2.437109in}}%
\pgfpathlineto{\pgfqpoint{2.401994in}{2.428644in}}%
\pgfpathlineto{\pgfqpoint{2.407975in}{2.420165in}}%
\pgfpathlineto{\pgfqpoint{2.413961in}{2.411672in}}%
\pgfpathclose%
\pgfusepath{stroke,fill}%
\end{pgfscope}%
\begin{pgfscope}%
\pgfpathrectangle{\pgfqpoint{0.887500in}{0.275000in}}{\pgfqpoint{4.225000in}{4.225000in}}%
\pgfusepath{clip}%
\pgfsetbuttcap%
\pgfsetroundjoin%
\definecolor{currentfill}{rgb}{0.119738,0.603785,0.541400}%
\pgfsetfillcolor{currentfill}%
\pgfsetfillopacity{0.700000}%
\pgfsetlinewidth{0.501875pt}%
\definecolor{currentstroke}{rgb}{1.000000,1.000000,1.000000}%
\pgfsetstrokecolor{currentstroke}%
\pgfsetstrokeopacity{0.500000}%
\pgfsetdash{}{0pt}%
\pgfpathmoveto{\pgfqpoint{4.323230in}{2.582746in}}%
\pgfpathlineto{\pgfqpoint{4.334219in}{2.586089in}}%
\pgfpathlineto{\pgfqpoint{4.345202in}{2.589426in}}%
\pgfpathlineto{\pgfqpoint{4.356180in}{2.592760in}}%
\pgfpathlineto{\pgfqpoint{4.367153in}{2.596090in}}%
\pgfpathlineto{\pgfqpoint{4.378120in}{2.599418in}}%
\pgfpathlineto{\pgfqpoint{4.371693in}{2.613726in}}%
\pgfpathlineto{\pgfqpoint{4.365268in}{2.628023in}}%
\pgfpathlineto{\pgfqpoint{4.358845in}{2.642306in}}%
\pgfpathlineto{\pgfqpoint{4.352424in}{2.656571in}}%
\pgfpathlineto{\pgfqpoint{4.346004in}{2.670817in}}%
\pgfpathlineto{\pgfqpoint{4.335037in}{2.667451in}}%
\pgfpathlineto{\pgfqpoint{4.324065in}{2.664082in}}%
\pgfpathlineto{\pgfqpoint{4.313087in}{2.660710in}}%
\pgfpathlineto{\pgfqpoint{4.302104in}{2.657333in}}%
\pgfpathlineto{\pgfqpoint{4.291115in}{2.653950in}}%
\pgfpathlineto{\pgfqpoint{4.297535in}{2.639757in}}%
\pgfpathlineto{\pgfqpoint{4.303956in}{2.625539in}}%
\pgfpathlineto{\pgfqpoint{4.310379in}{2.611297in}}%
\pgfpathlineto{\pgfqpoint{4.316803in}{2.597032in}}%
\pgfpathclose%
\pgfusepath{stroke,fill}%
\end{pgfscope}%
\begin{pgfscope}%
\pgfpathrectangle{\pgfqpoint{0.887500in}{0.275000in}}{\pgfqpoint{4.225000in}{4.225000in}}%
\pgfusepath{clip}%
\pgfsetbuttcap%
\pgfsetroundjoin%
\definecolor{currentfill}{rgb}{0.232815,0.732247,0.459277}%
\pgfsetfillcolor{currentfill}%
\pgfsetfillopacity{0.700000}%
\pgfsetlinewidth{0.501875pt}%
\definecolor{currentstroke}{rgb}{1.000000,1.000000,1.000000}%
\pgfsetstrokecolor{currentstroke}%
\pgfsetstrokeopacity{0.500000}%
\pgfsetdash{}{0pt}%
\pgfpathmoveto{\pgfqpoint{3.942349in}{2.863405in}}%
\pgfpathlineto{\pgfqpoint{3.953437in}{2.866883in}}%
\pgfpathlineto{\pgfqpoint{3.964520in}{2.870381in}}%
\pgfpathlineto{\pgfqpoint{3.975599in}{2.873901in}}%
\pgfpathlineto{\pgfqpoint{3.986672in}{2.877442in}}%
\pgfpathlineto{\pgfqpoint{3.997741in}{2.881000in}}%
\pgfpathlineto{\pgfqpoint{3.991368in}{2.894704in}}%
\pgfpathlineto{\pgfqpoint{3.984998in}{2.908414in}}%
\pgfpathlineto{\pgfqpoint{3.978631in}{2.922136in}}%
\pgfpathlineto{\pgfqpoint{3.972267in}{2.935877in}}%
\pgfpathlineto{\pgfqpoint{3.965906in}{2.949634in}}%
\pgfpathlineto{\pgfqpoint{3.954839in}{2.946031in}}%
\pgfpathlineto{\pgfqpoint{3.943767in}{2.942441in}}%
\pgfpathlineto{\pgfqpoint{3.932690in}{2.938870in}}%
\pgfpathlineto{\pgfqpoint{3.921608in}{2.935326in}}%
\pgfpathlineto{\pgfqpoint{3.910522in}{2.931809in}}%
\pgfpathlineto{\pgfqpoint{3.916881in}{2.918062in}}%
\pgfpathlineto{\pgfqpoint{3.923243in}{2.904352in}}%
\pgfpathlineto{\pgfqpoint{3.929609in}{2.890679in}}%
\pgfpathlineto{\pgfqpoint{3.935977in}{2.877033in}}%
\pgfpathclose%
\pgfusepath{stroke,fill}%
\end{pgfscope}%
\begin{pgfscope}%
\pgfpathrectangle{\pgfqpoint{0.887500in}{0.275000in}}{\pgfqpoint{4.225000in}{4.225000in}}%
\pgfusepath{clip}%
\pgfsetbuttcap%
\pgfsetroundjoin%
\definecolor{currentfill}{rgb}{0.122606,0.585371,0.546557}%
\pgfsetfillcolor{currentfill}%
\pgfsetfillopacity{0.700000}%
\pgfsetlinewidth{0.501875pt}%
\definecolor{currentstroke}{rgb}{1.000000,1.000000,1.000000}%
\pgfsetstrokecolor{currentstroke}%
\pgfsetstrokeopacity{0.500000}%
\pgfsetdash{}{0pt}%
\pgfpathmoveto{\pgfqpoint{1.549884in}{2.551074in}}%
\pgfpathlineto{\pgfqpoint{1.561558in}{2.554387in}}%
\pgfpathlineto{\pgfqpoint{1.573227in}{2.557693in}}%
\pgfpathlineto{\pgfqpoint{1.584890in}{2.560993in}}%
\pgfpathlineto{\pgfqpoint{1.596548in}{2.564286in}}%
\pgfpathlineto{\pgfqpoint{1.608200in}{2.567575in}}%
\pgfpathlineto{\pgfqpoint{1.602501in}{2.575514in}}%
\pgfpathlineto{\pgfqpoint{1.596807in}{2.583426in}}%
\pgfpathlineto{\pgfqpoint{1.591118in}{2.591311in}}%
\pgfpathlineto{\pgfqpoint{1.585434in}{2.599168in}}%
\pgfpathlineto{\pgfqpoint{1.579754in}{2.606999in}}%
\pgfpathlineto{\pgfqpoint{1.568115in}{2.603714in}}%
\pgfpathlineto{\pgfqpoint{1.556470in}{2.600425in}}%
\pgfpathlineto{\pgfqpoint{1.544820in}{2.597130in}}%
\pgfpathlineto{\pgfqpoint{1.533165in}{2.593829in}}%
\pgfpathlineto{\pgfqpoint{1.521504in}{2.590520in}}%
\pgfpathlineto{\pgfqpoint{1.527170in}{2.582689in}}%
\pgfpathlineto{\pgfqpoint{1.532841in}{2.574830in}}%
\pgfpathlineto{\pgfqpoint{1.538517in}{2.566940in}}%
\pgfpathlineto{\pgfqpoint{1.544198in}{2.559021in}}%
\pgfpathclose%
\pgfusepath{stroke,fill}%
\end{pgfscope}%
\begin{pgfscope}%
\pgfpathrectangle{\pgfqpoint{0.887500in}{0.275000in}}{\pgfqpoint{4.225000in}{4.225000in}}%
\pgfusepath{clip}%
\pgfsetbuttcap%
\pgfsetroundjoin%
\definecolor{currentfill}{rgb}{0.730889,0.871916,0.156029}%
\pgfsetfillcolor{currentfill}%
\pgfsetfillopacity{0.700000}%
\pgfsetlinewidth{0.501875pt}%
\definecolor{currentstroke}{rgb}{1.000000,1.000000,1.000000}%
\pgfsetstrokecolor{currentstroke}%
\pgfsetstrokeopacity{0.500000}%
\pgfsetdash{}{0pt}%
\pgfpathmoveto{\pgfqpoint{3.067255in}{3.283583in}}%
\pgfpathlineto{\pgfqpoint{3.078557in}{3.295427in}}%
\pgfpathlineto{\pgfqpoint{3.089856in}{3.305280in}}%
\pgfpathlineto{\pgfqpoint{3.101152in}{3.313323in}}%
\pgfpathlineto{\pgfqpoint{3.112442in}{3.319739in}}%
\pgfpathlineto{\pgfqpoint{3.123727in}{3.324746in}}%
\pgfpathlineto{\pgfqpoint{3.117515in}{3.335024in}}%
\pgfpathlineto{\pgfqpoint{3.111306in}{3.345092in}}%
\pgfpathlineto{\pgfqpoint{3.105100in}{3.354940in}}%
\pgfpathlineto{\pgfqpoint{3.098898in}{3.364557in}}%
\pgfpathlineto{\pgfqpoint{3.092699in}{3.373932in}}%
\pgfpathlineto{\pgfqpoint{3.081424in}{3.368581in}}%
\pgfpathlineto{\pgfqpoint{3.070144in}{3.361684in}}%
\pgfpathlineto{\pgfqpoint{3.058861in}{3.352943in}}%
\pgfpathlineto{\pgfqpoint{3.047576in}{3.342041in}}%
\pgfpathlineto{\pgfqpoint{3.036291in}{3.328661in}}%
\pgfpathlineto{\pgfqpoint{3.042476in}{3.320113in}}%
\pgfpathlineto{\pgfqpoint{3.048665in}{3.311328in}}%
\pgfpathlineto{\pgfqpoint{3.054858in}{3.302310in}}%
\pgfpathlineto{\pgfqpoint{3.061055in}{3.293061in}}%
\pgfpathclose%
\pgfusepath{stroke,fill}%
\end{pgfscope}%
\begin{pgfscope}%
\pgfpathrectangle{\pgfqpoint{0.887500in}{0.275000in}}{\pgfqpoint{4.225000in}{4.225000in}}%
\pgfusepath{clip}%
\pgfsetbuttcap%
\pgfsetroundjoin%
\definecolor{currentfill}{rgb}{0.163625,0.471133,0.558148}%
\pgfsetfillcolor{currentfill}%
\pgfsetfillopacity{0.700000}%
\pgfsetlinewidth{0.501875pt}%
\definecolor{currentstroke}{rgb}{1.000000,1.000000,1.000000}%
\pgfsetstrokecolor{currentstroke}%
\pgfsetstrokeopacity{0.500000}%
\pgfsetdash{}{0pt}%
\pgfpathmoveto{\pgfqpoint{2.826574in}{2.303252in}}%
\pgfpathlineto{\pgfqpoint{2.837949in}{2.303926in}}%
\pgfpathlineto{\pgfqpoint{2.849304in}{2.307045in}}%
\pgfpathlineto{\pgfqpoint{2.860636in}{2.313717in}}%
\pgfpathlineto{\pgfqpoint{2.871942in}{2.325054in}}%
\pgfpathlineto{\pgfqpoint{2.883222in}{2.341987in}}%
\pgfpathlineto{\pgfqpoint{2.877085in}{2.350182in}}%
\pgfpathlineto{\pgfqpoint{2.870953in}{2.358295in}}%
\pgfpathlineto{\pgfqpoint{2.864825in}{2.366426in}}%
\pgfpathlineto{\pgfqpoint{2.858701in}{2.374674in}}%
\pgfpathlineto{\pgfqpoint{2.852579in}{2.383137in}}%
\pgfpathlineto{\pgfqpoint{2.841332in}{2.364101in}}%
\pgfpathlineto{\pgfqpoint{2.830051in}{2.351754in}}%
\pgfpathlineto{\pgfqpoint{2.818735in}{2.344977in}}%
\pgfpathlineto{\pgfqpoint{2.807389in}{2.342432in}}%
\pgfpathlineto{\pgfqpoint{2.796017in}{2.342786in}}%
\pgfpathlineto{\pgfqpoint{2.802121in}{2.334787in}}%
\pgfpathlineto{\pgfqpoint{2.808228in}{2.326932in}}%
\pgfpathlineto{\pgfqpoint{2.814339in}{2.319123in}}%
\pgfpathlineto{\pgfqpoint{2.820454in}{2.311262in}}%
\pgfpathclose%
\pgfusepath{stroke,fill}%
\end{pgfscope}%
\begin{pgfscope}%
\pgfpathrectangle{\pgfqpoint{0.887500in}{0.275000in}}{\pgfqpoint{4.225000in}{4.225000in}}%
\pgfusepath{clip}%
\pgfsetbuttcap%
\pgfsetroundjoin%
\definecolor{currentfill}{rgb}{0.129933,0.559582,0.551864}%
\pgfsetfillcolor{currentfill}%
\pgfsetfillopacity{0.700000}%
\pgfsetlinewidth{0.501875pt}%
\definecolor{currentstroke}{rgb}{1.000000,1.000000,1.000000}%
\pgfsetstrokecolor{currentstroke}%
\pgfsetstrokeopacity{0.500000}%
\pgfsetdash{}{0pt}%
\pgfpathmoveto{\pgfqpoint{1.868874in}{2.495861in}}%
\pgfpathlineto{\pgfqpoint{1.880474in}{2.499120in}}%
\pgfpathlineto{\pgfqpoint{1.892068in}{2.502375in}}%
\pgfpathlineto{\pgfqpoint{1.903656in}{2.505628in}}%
\pgfpathlineto{\pgfqpoint{1.915239in}{2.508881in}}%
\pgfpathlineto{\pgfqpoint{1.926816in}{2.512136in}}%
\pgfpathlineto{\pgfqpoint{1.921003in}{2.520282in}}%
\pgfpathlineto{\pgfqpoint{1.915194in}{2.528412in}}%
\pgfpathlineto{\pgfqpoint{1.909389in}{2.536527in}}%
\pgfpathlineto{\pgfqpoint{1.903588in}{2.544627in}}%
\pgfpathlineto{\pgfqpoint{1.897791in}{2.552712in}}%
\pgfpathlineto{\pgfqpoint{1.886227in}{2.549467in}}%
\pgfpathlineto{\pgfqpoint{1.874657in}{2.546224in}}%
\pgfpathlineto{\pgfqpoint{1.863081in}{2.542983in}}%
\pgfpathlineto{\pgfqpoint{1.851499in}{2.539740in}}%
\pgfpathlineto{\pgfqpoint{1.839911in}{2.536494in}}%
\pgfpathlineto{\pgfqpoint{1.845695in}{2.528400in}}%
\pgfpathlineto{\pgfqpoint{1.851484in}{2.520290in}}%
\pgfpathlineto{\pgfqpoint{1.857276in}{2.512164in}}%
\pgfpathlineto{\pgfqpoint{1.863073in}{2.504021in}}%
\pgfpathclose%
\pgfusepath{stroke,fill}%
\end{pgfscope}%
\begin{pgfscope}%
\pgfpathrectangle{\pgfqpoint{0.887500in}{0.275000in}}{\pgfqpoint{4.225000in}{4.225000in}}%
\pgfusepath{clip}%
\pgfsetbuttcap%
\pgfsetroundjoin%
\definecolor{currentfill}{rgb}{0.122312,0.633153,0.530398}%
\pgfsetfillcolor{currentfill}%
\pgfsetfillopacity{0.700000}%
\pgfsetlinewidth{0.501875pt}%
\definecolor{currentstroke}{rgb}{1.000000,1.000000,1.000000}%
\pgfsetstrokecolor{currentstroke}%
\pgfsetstrokeopacity{0.500000}%
\pgfsetdash{}{0pt}%
\pgfpathmoveto{\pgfqpoint{4.236088in}{2.636900in}}%
\pgfpathlineto{\pgfqpoint{4.247105in}{2.640332in}}%
\pgfpathlineto{\pgfqpoint{4.258116in}{2.643751in}}%
\pgfpathlineto{\pgfqpoint{4.269121in}{2.647160in}}%
\pgfpathlineto{\pgfqpoint{4.280121in}{2.650559in}}%
\pgfpathlineto{\pgfqpoint{4.291115in}{2.653950in}}%
\pgfpathlineto{\pgfqpoint{4.284697in}{2.668116in}}%
\pgfpathlineto{\pgfqpoint{4.278281in}{2.682255in}}%
\pgfpathlineto{\pgfqpoint{4.271867in}{2.696366in}}%
\pgfpathlineto{\pgfqpoint{4.265454in}{2.710447in}}%
\pgfpathlineto{\pgfqpoint{4.259043in}{2.724504in}}%
\pgfpathlineto{\pgfqpoint{4.248049in}{2.721077in}}%
\pgfpathlineto{\pgfqpoint{4.237051in}{2.717647in}}%
\pgfpathlineto{\pgfqpoint{4.226046in}{2.714214in}}%
\pgfpathlineto{\pgfqpoint{4.215037in}{2.710777in}}%
\pgfpathlineto{\pgfqpoint{4.204022in}{2.707334in}}%
\pgfpathlineto{\pgfqpoint{4.210431in}{2.693293in}}%
\pgfpathlineto{\pgfqpoint{4.216843in}{2.679232in}}%
\pgfpathlineto{\pgfqpoint{4.223256in}{2.665146in}}%
\pgfpathlineto{\pgfqpoint{4.229671in}{2.651035in}}%
\pgfpathclose%
\pgfusepath{stroke,fill}%
\end{pgfscope}%
\begin{pgfscope}%
\pgfpathrectangle{\pgfqpoint{0.887500in}{0.275000in}}{\pgfqpoint{4.225000in}{4.225000in}}%
\pgfusepath{clip}%
\pgfsetbuttcap%
\pgfsetroundjoin%
\definecolor{currentfill}{rgb}{0.281477,0.755203,0.432552}%
\pgfsetfillcolor{currentfill}%
\pgfsetfillopacity{0.700000}%
\pgfsetlinewidth{0.501875pt}%
\definecolor{currentstroke}{rgb}{1.000000,1.000000,1.000000}%
\pgfsetstrokecolor{currentstroke}%
\pgfsetstrokeopacity{0.500000}%
\pgfsetdash{}{0pt}%
\pgfpathmoveto{\pgfqpoint{3.855017in}{2.914422in}}%
\pgfpathlineto{\pgfqpoint{3.866129in}{2.917894in}}%
\pgfpathlineto{\pgfqpoint{3.877235in}{2.921362in}}%
\pgfpathlineto{\pgfqpoint{3.888336in}{2.924833in}}%
\pgfpathlineto{\pgfqpoint{3.899431in}{2.928314in}}%
\pgfpathlineto{\pgfqpoint{3.910522in}{2.931809in}}%
\pgfpathlineto{\pgfqpoint{3.904166in}{2.945571in}}%
\pgfpathlineto{\pgfqpoint{3.897813in}{2.959322in}}%
\pgfpathlineto{\pgfqpoint{3.891461in}{2.973036in}}%
\pgfpathlineto{\pgfqpoint{3.885110in}{2.986688in}}%
\pgfpathlineto{\pgfqpoint{3.878761in}{3.000252in}}%
\pgfpathlineto{\pgfqpoint{3.867672in}{2.996737in}}%
\pgfpathlineto{\pgfqpoint{3.856578in}{2.993224in}}%
\pgfpathlineto{\pgfqpoint{3.845479in}{2.989707in}}%
\pgfpathlineto{\pgfqpoint{3.834375in}{2.986181in}}%
\pgfpathlineto{\pgfqpoint{3.823265in}{2.982641in}}%
\pgfpathlineto{\pgfqpoint{3.829613in}{2.969110in}}%
\pgfpathlineto{\pgfqpoint{3.835961in}{2.955500in}}%
\pgfpathlineto{\pgfqpoint{3.842311in}{2.941834in}}%
\pgfpathlineto{\pgfqpoint{3.848663in}{2.928135in}}%
\pgfpathclose%
\pgfusepath{stroke,fill}%
\end{pgfscope}%
\begin{pgfscope}%
\pgfpathrectangle{\pgfqpoint{0.887500in}{0.275000in}}{\pgfqpoint{4.225000in}{4.225000in}}%
\pgfusepath{clip}%
\pgfsetbuttcap%
\pgfsetroundjoin%
\definecolor{currentfill}{rgb}{0.139147,0.533812,0.555298}%
\pgfsetfillcolor{currentfill}%
\pgfsetfillopacity{0.700000}%
\pgfsetlinewidth{0.501875pt}%
\definecolor{currentstroke}{rgb}{1.000000,1.000000,1.000000}%
\pgfsetstrokecolor{currentstroke}%
\pgfsetstrokeopacity{0.500000}%
\pgfsetdash{}{0pt}%
\pgfpathmoveto{\pgfqpoint{2.188021in}{2.437853in}}%
\pgfpathlineto{\pgfqpoint{2.199542in}{2.441158in}}%
\pgfpathlineto{\pgfqpoint{2.211057in}{2.444454in}}%
\pgfpathlineto{\pgfqpoint{2.222568in}{2.447741in}}%
\pgfpathlineto{\pgfqpoint{2.234072in}{2.451020in}}%
\pgfpathlineto{\pgfqpoint{2.245571in}{2.454295in}}%
\pgfpathlineto{\pgfqpoint{2.239646in}{2.462652in}}%
\pgfpathlineto{\pgfqpoint{2.233726in}{2.470996in}}%
\pgfpathlineto{\pgfqpoint{2.227809in}{2.479325in}}%
\pgfpathlineto{\pgfqpoint{2.221896in}{2.487640in}}%
\pgfpathlineto{\pgfqpoint{2.215988in}{2.495940in}}%
\pgfpathlineto{\pgfqpoint{2.204501in}{2.492666in}}%
\pgfpathlineto{\pgfqpoint{2.193009in}{2.489390in}}%
\pgfpathlineto{\pgfqpoint{2.181510in}{2.486110in}}%
\pgfpathlineto{\pgfqpoint{2.170007in}{2.482825in}}%
\pgfpathlineto{\pgfqpoint{2.158497in}{2.479533in}}%
\pgfpathlineto{\pgfqpoint{2.164394in}{2.471229in}}%
\pgfpathlineto{\pgfqpoint{2.170294in}{2.462909in}}%
\pgfpathlineto{\pgfqpoint{2.176199in}{2.454573in}}%
\pgfpathlineto{\pgfqpoint{2.182108in}{2.446221in}}%
\pgfpathclose%
\pgfusepath{stroke,fill}%
\end{pgfscope}%
\begin{pgfscope}%
\pgfpathrectangle{\pgfqpoint{0.887500in}{0.275000in}}{\pgfqpoint{4.225000in}{4.225000in}}%
\pgfusepath{clip}%
\pgfsetbuttcap%
\pgfsetroundjoin%
\definecolor{currentfill}{rgb}{0.150476,0.504369,0.557430}%
\pgfsetfillcolor{currentfill}%
\pgfsetfillopacity{0.700000}%
\pgfsetlinewidth{0.501875pt}%
\definecolor{currentstroke}{rgb}{1.000000,1.000000,1.000000}%
\pgfsetstrokecolor{currentstroke}%
\pgfsetstrokeopacity{0.500000}%
\pgfsetdash{}{0pt}%
\pgfpathmoveto{\pgfqpoint{2.507242in}{2.378094in}}%
\pgfpathlineto{\pgfqpoint{2.518683in}{2.381572in}}%
\pgfpathlineto{\pgfqpoint{2.530119in}{2.384949in}}%
\pgfpathlineto{\pgfqpoint{2.541552in}{2.388192in}}%
\pgfpathlineto{\pgfqpoint{2.552981in}{2.391266in}}%
\pgfpathlineto{\pgfqpoint{2.564406in}{2.394195in}}%
\pgfpathlineto{\pgfqpoint{2.558373in}{2.402763in}}%
\pgfpathlineto{\pgfqpoint{2.552344in}{2.411319in}}%
\pgfpathlineto{\pgfqpoint{2.546319in}{2.419862in}}%
\pgfpathlineto{\pgfqpoint{2.540298in}{2.428395in}}%
\pgfpathlineto{\pgfqpoint{2.534282in}{2.436919in}}%
\pgfpathlineto{\pgfqpoint{2.522868in}{2.433946in}}%
\pgfpathlineto{\pgfqpoint{2.511451in}{2.430844in}}%
\pgfpathlineto{\pgfqpoint{2.500030in}{2.427589in}}%
\pgfpathlineto{\pgfqpoint{2.488605in}{2.424213in}}%
\pgfpathlineto{\pgfqpoint{2.477176in}{2.420748in}}%
\pgfpathlineto{\pgfqpoint{2.483181in}{2.412249in}}%
\pgfpathlineto{\pgfqpoint{2.489190in}{2.403734in}}%
\pgfpathlineto{\pgfqpoint{2.495203in}{2.395204in}}%
\pgfpathlineto{\pgfqpoint{2.501221in}{2.386658in}}%
\pgfpathclose%
\pgfusepath{stroke,fill}%
\end{pgfscope}%
\begin{pgfscope}%
\pgfpathrectangle{\pgfqpoint{0.887500in}{0.275000in}}{\pgfqpoint{4.225000in}{4.225000in}}%
\pgfusepath{clip}%
\pgfsetbuttcap%
\pgfsetroundjoin%
\definecolor{currentfill}{rgb}{0.124780,0.640461,0.527068}%
\pgfsetfillcolor{currentfill}%
\pgfsetfillopacity{0.700000}%
\pgfsetlinewidth{0.501875pt}%
\definecolor{currentstroke}{rgb}{1.000000,1.000000,1.000000}%
\pgfsetstrokecolor{currentstroke}%
\pgfsetstrokeopacity{0.500000}%
\pgfsetdash{}{0pt}%
\pgfpathmoveto{\pgfqpoint{2.877851in}{2.598257in}}%
\pgfpathlineto{\pgfqpoint{2.889044in}{2.634537in}}%
\pgfpathlineto{\pgfqpoint{2.900259in}{2.668694in}}%
\pgfpathlineto{\pgfqpoint{2.911494in}{2.700744in}}%
\pgfpathlineto{\pgfqpoint{2.922746in}{2.731086in}}%
\pgfpathlineto{\pgfqpoint{2.934011in}{2.760117in}}%
\pgfpathlineto{\pgfqpoint{2.927853in}{2.766294in}}%
\pgfpathlineto{\pgfqpoint{2.921706in}{2.770328in}}%
\pgfpathlineto{\pgfqpoint{2.915571in}{2.772542in}}%
\pgfpathlineto{\pgfqpoint{2.909447in}{2.773261in}}%
\pgfpathlineto{\pgfqpoint{2.903335in}{2.772806in}}%
\pgfpathlineto{\pgfqpoint{2.892099in}{2.746015in}}%
\pgfpathlineto{\pgfqpoint{2.880874in}{2.718795in}}%
\pgfpathlineto{\pgfqpoint{2.869659in}{2.690866in}}%
\pgfpathlineto{\pgfqpoint{2.858458in}{2.661950in}}%
\pgfpathlineto{\pgfqpoint{2.847272in}{2.632014in}}%
\pgfpathlineto{\pgfqpoint{2.853371in}{2.626750in}}%
\pgfpathlineto{\pgfqpoint{2.859478in}{2.620888in}}%
\pgfpathlineto{\pgfqpoint{2.865593in}{2.614283in}}%
\pgfpathlineto{\pgfqpoint{2.871718in}{2.606788in}}%
\pgfpathclose%
\pgfusepath{stroke,fill}%
\end{pgfscope}%
\begin{pgfscope}%
\pgfpathrectangle{\pgfqpoint{0.887500in}{0.275000in}}{\pgfqpoint{4.225000in}{4.225000in}}%
\pgfusepath{clip}%
\pgfsetbuttcap%
\pgfsetroundjoin%
\definecolor{currentfill}{rgb}{0.335885,0.777018,0.402049}%
\pgfsetfillcolor{currentfill}%
\pgfsetfillopacity{0.700000}%
\pgfsetlinewidth{0.501875pt}%
\definecolor{currentstroke}{rgb}{1.000000,1.000000,1.000000}%
\pgfsetstrokecolor{currentstroke}%
\pgfsetstrokeopacity{0.500000}%
\pgfsetdash{}{0pt}%
\pgfpathmoveto{\pgfqpoint{3.767632in}{2.964667in}}%
\pgfpathlineto{\pgfqpoint{3.778770in}{2.968294in}}%
\pgfpathlineto{\pgfqpoint{3.789902in}{2.971906in}}%
\pgfpathlineto{\pgfqpoint{3.801029in}{2.975503in}}%
\pgfpathlineto{\pgfqpoint{3.812150in}{2.979082in}}%
\pgfpathlineto{\pgfqpoint{3.823265in}{2.982641in}}%
\pgfpathlineto{\pgfqpoint{3.816919in}{2.996073in}}%
\pgfpathlineto{\pgfqpoint{3.810572in}{3.009384in}}%
\pgfpathlineto{\pgfqpoint{3.804226in}{3.022551in}}%
\pgfpathlineto{\pgfqpoint{3.797879in}{3.035553in}}%
\pgfpathlineto{\pgfqpoint{3.791532in}{3.048371in}}%
\pgfpathlineto{\pgfqpoint{3.780420in}{3.044820in}}%
\pgfpathlineto{\pgfqpoint{3.769303in}{3.041244in}}%
\pgfpathlineto{\pgfqpoint{3.758180in}{3.037651in}}%
\pgfpathlineto{\pgfqpoint{3.747052in}{3.034044in}}%
\pgfpathlineto{\pgfqpoint{3.735918in}{3.030429in}}%
\pgfpathlineto{\pgfqpoint{3.742260in}{3.017516in}}%
\pgfpathlineto{\pgfqpoint{3.748602in}{3.004471in}}%
\pgfpathlineto{\pgfqpoint{3.754944in}{2.991305in}}%
\pgfpathlineto{\pgfqpoint{3.761288in}{2.978032in}}%
\pgfpathclose%
\pgfusepath{stroke,fill}%
\end{pgfscope}%
\begin{pgfscope}%
\pgfpathrectangle{\pgfqpoint{0.887500in}{0.275000in}}{\pgfqpoint{4.225000in}{4.225000in}}%
\pgfusepath{clip}%
\pgfsetbuttcap%
\pgfsetroundjoin%
\definecolor{currentfill}{rgb}{0.134692,0.658636,0.517649}%
\pgfsetfillcolor{currentfill}%
\pgfsetfillopacity{0.700000}%
\pgfsetlinewidth{0.501875pt}%
\definecolor{currentstroke}{rgb}{1.000000,1.000000,1.000000}%
\pgfsetstrokecolor{currentstroke}%
\pgfsetstrokeopacity{0.500000}%
\pgfsetdash{}{0pt}%
\pgfpathmoveto{\pgfqpoint{4.148866in}{2.690072in}}%
\pgfpathlineto{\pgfqpoint{4.159907in}{2.693524in}}%
\pgfpathlineto{\pgfqpoint{4.170944in}{2.696978in}}%
\pgfpathlineto{\pgfqpoint{4.181975in}{2.700433in}}%
\pgfpathlineto{\pgfqpoint{4.193001in}{2.703885in}}%
\pgfpathlineto{\pgfqpoint{4.204022in}{2.707334in}}%
\pgfpathlineto{\pgfqpoint{4.197614in}{2.721356in}}%
\pgfpathlineto{\pgfqpoint{4.191208in}{2.735363in}}%
\pgfpathlineto{\pgfqpoint{4.184805in}{2.749358in}}%
\pgfpathlineto{\pgfqpoint{4.178404in}{2.763346in}}%
\pgfpathlineto{\pgfqpoint{4.172006in}{2.777328in}}%
\pgfpathlineto{\pgfqpoint{4.160987in}{2.773876in}}%
\pgfpathlineto{\pgfqpoint{4.149963in}{2.770423in}}%
\pgfpathlineto{\pgfqpoint{4.138934in}{2.766973in}}%
\pgfpathlineto{\pgfqpoint{4.127900in}{2.763526in}}%
\pgfpathlineto{\pgfqpoint{4.116861in}{2.760085in}}%
\pgfpathlineto{\pgfqpoint{4.123258in}{2.746113in}}%
\pgfpathlineto{\pgfqpoint{4.129656in}{2.732127in}}%
\pgfpathlineto{\pgfqpoint{4.136057in}{2.718125in}}%
\pgfpathlineto{\pgfqpoint{4.142461in}{2.704107in}}%
\pgfpathclose%
\pgfusepath{stroke,fill}%
\end{pgfscope}%
\begin{pgfscope}%
\pgfpathrectangle{\pgfqpoint{0.887500in}{0.275000in}}{\pgfqpoint{4.225000in}{4.225000in}}%
\pgfusepath{clip}%
\pgfsetbuttcap%
\pgfsetroundjoin%
\definecolor{currentfill}{rgb}{0.125394,0.574318,0.549086}%
\pgfsetfillcolor{currentfill}%
\pgfsetfillopacity{0.700000}%
\pgfsetlinewidth{0.501875pt}%
\definecolor{currentstroke}{rgb}{1.000000,1.000000,1.000000}%
\pgfsetstrokecolor{currentstroke}%
\pgfsetstrokeopacity{0.500000}%
\pgfsetdash{}{0pt}%
\pgfpathmoveto{\pgfqpoint{1.636761in}{2.527538in}}%
\pgfpathlineto{\pgfqpoint{1.648420in}{2.530838in}}%
\pgfpathlineto{\pgfqpoint{1.660074in}{2.534136in}}%
\pgfpathlineto{\pgfqpoint{1.671722in}{2.537433in}}%
\pgfpathlineto{\pgfqpoint{1.683364in}{2.540731in}}%
\pgfpathlineto{\pgfqpoint{1.695000in}{2.544029in}}%
\pgfpathlineto{\pgfqpoint{1.689266in}{2.552060in}}%
\pgfpathlineto{\pgfqpoint{1.683537in}{2.560073in}}%
\pgfpathlineto{\pgfqpoint{1.677812in}{2.568066in}}%
\pgfpathlineto{\pgfqpoint{1.672091in}{2.576039in}}%
\pgfpathlineto{\pgfqpoint{1.666375in}{2.583989in}}%
\pgfpathlineto{\pgfqpoint{1.654752in}{2.580708in}}%
\pgfpathlineto{\pgfqpoint{1.643123in}{2.577426in}}%
\pgfpathlineto{\pgfqpoint{1.631487in}{2.574144in}}%
\pgfpathlineto{\pgfqpoint{1.619847in}{2.570861in}}%
\pgfpathlineto{\pgfqpoint{1.608200in}{2.567575in}}%
\pgfpathlineto{\pgfqpoint{1.613903in}{2.559612in}}%
\pgfpathlineto{\pgfqpoint{1.619611in}{2.551625in}}%
\pgfpathlineto{\pgfqpoint{1.625323in}{2.543617in}}%
\pgfpathlineto{\pgfqpoint{1.631040in}{2.535587in}}%
\pgfpathclose%
\pgfusepath{stroke,fill}%
\end{pgfscope}%
\begin{pgfscope}%
\pgfpathrectangle{\pgfqpoint{0.887500in}{0.275000in}}{\pgfqpoint{4.225000in}{4.225000in}}%
\pgfusepath{clip}%
\pgfsetbuttcap%
\pgfsetroundjoin%
\definecolor{currentfill}{rgb}{0.699415,0.867117,0.175971}%
\pgfsetfillcolor{currentfill}%
\pgfsetfillopacity{0.700000}%
\pgfsetlinewidth{0.501875pt}%
\definecolor{currentstroke}{rgb}{1.000000,1.000000,1.000000}%
\pgfsetstrokecolor{currentstroke}%
\pgfsetstrokeopacity{0.500000}%
\pgfsetdash{}{0pt}%
\pgfpathmoveto{\pgfqpoint{3.154833in}{3.270577in}}%
\pgfpathlineto{\pgfqpoint{3.166119in}{3.274612in}}%
\pgfpathlineto{\pgfqpoint{3.177398in}{3.277896in}}%
\pgfpathlineto{\pgfqpoint{3.188669in}{3.280640in}}%
\pgfpathlineto{\pgfqpoint{3.199934in}{3.283051in}}%
\pgfpathlineto{\pgfqpoint{3.211192in}{3.285340in}}%
\pgfpathlineto{\pgfqpoint{3.204956in}{3.295861in}}%
\pgfpathlineto{\pgfqpoint{3.198724in}{3.306368in}}%
\pgfpathlineto{\pgfqpoint{3.192494in}{3.316837in}}%
\pgfpathlineto{\pgfqpoint{3.186268in}{3.327245in}}%
\pgfpathlineto{\pgfqpoint{3.180045in}{3.337568in}}%
\pgfpathlineto{\pgfqpoint{3.168796in}{3.335780in}}%
\pgfpathlineto{\pgfqpoint{3.157539in}{3.333859in}}%
\pgfpathlineto{\pgfqpoint{3.146275in}{3.331550in}}%
\pgfpathlineto{\pgfqpoint{3.135005in}{3.328598in}}%
\pgfpathlineto{\pgfqpoint{3.123727in}{3.324746in}}%
\pgfpathlineto{\pgfqpoint{3.129942in}{3.314268in}}%
\pgfpathlineto{\pgfqpoint{3.136160in}{3.303602in}}%
\pgfpathlineto{\pgfqpoint{3.142381in}{3.292758in}}%
\pgfpathlineto{\pgfqpoint{3.148606in}{3.281746in}}%
\pgfpathclose%
\pgfusepath{stroke,fill}%
\end{pgfscope}%
\begin{pgfscope}%
\pgfpathrectangle{\pgfqpoint{0.887500in}{0.275000in}}{\pgfqpoint{4.225000in}{4.225000in}}%
\pgfusepath{clip}%
\pgfsetbuttcap%
\pgfsetroundjoin%
\definecolor{currentfill}{rgb}{0.449368,0.813768,0.335384}%
\pgfsetfillcolor{currentfill}%
\pgfsetfillopacity{0.700000}%
\pgfsetlinewidth{0.501875pt}%
\definecolor{currentstroke}{rgb}{1.000000,1.000000,1.000000}%
\pgfsetstrokecolor{currentstroke}%
\pgfsetstrokeopacity{0.500000}%
\pgfsetdash{}{0pt}%
\pgfpathmoveto{\pgfqpoint{2.985234in}{3.042082in}}%
\pgfpathlineto{\pgfqpoint{2.996515in}{3.068200in}}%
\pgfpathlineto{\pgfqpoint{3.007806in}{3.092868in}}%
\pgfpathlineto{\pgfqpoint{3.019105in}{3.116037in}}%
\pgfpathlineto{\pgfqpoint{3.030410in}{3.137658in}}%
\pgfpathlineto{\pgfqpoint{3.041721in}{3.157681in}}%
\pgfpathlineto{\pgfqpoint{3.035523in}{3.164230in}}%
\pgfpathlineto{\pgfqpoint{3.029329in}{3.170612in}}%
\pgfpathlineto{\pgfqpoint{3.023140in}{3.176820in}}%
\pgfpathlineto{\pgfqpoint{3.016956in}{3.182842in}}%
\pgfpathlineto{\pgfqpoint{3.010777in}{3.188668in}}%
\pgfpathlineto{\pgfqpoint{2.999502in}{3.162498in}}%
\pgfpathlineto{\pgfqpoint{2.988237in}{3.134632in}}%
\pgfpathlineto{\pgfqpoint{2.976983in}{3.105468in}}%
\pgfpathlineto{\pgfqpoint{2.965740in}{3.075408in}}%
\pgfpathlineto{\pgfqpoint{2.954508in}{3.044847in}}%
\pgfpathlineto{\pgfqpoint{2.960641in}{3.044236in}}%
\pgfpathlineto{\pgfqpoint{2.966780in}{3.043625in}}%
\pgfpathlineto{\pgfqpoint{2.972925in}{3.043043in}}%
\pgfpathlineto{\pgfqpoint{2.979076in}{3.042521in}}%
\pgfpathclose%
\pgfusepath{stroke,fill}%
\end{pgfscope}%
\begin{pgfscope}%
\pgfpathrectangle{\pgfqpoint{0.887500in}{0.275000in}}{\pgfqpoint{4.225000in}{4.225000in}}%
\pgfusepath{clip}%
\pgfsetbuttcap%
\pgfsetroundjoin%
\definecolor{currentfill}{rgb}{0.386433,0.794644,0.372886}%
\pgfsetfillcolor{currentfill}%
\pgfsetfillopacity{0.700000}%
\pgfsetlinewidth{0.501875pt}%
\definecolor{currentstroke}{rgb}{1.000000,1.000000,1.000000}%
\pgfsetstrokecolor{currentstroke}%
\pgfsetstrokeopacity{0.500000}%
\pgfsetdash{}{0pt}%
\pgfpathmoveto{\pgfqpoint{3.680173in}{3.012421in}}%
\pgfpathlineto{\pgfqpoint{3.691332in}{3.015996in}}%
\pgfpathlineto{\pgfqpoint{3.702486in}{3.019589in}}%
\pgfpathlineto{\pgfqpoint{3.713635in}{3.023196in}}%
\pgfpathlineto{\pgfqpoint{3.724779in}{3.026811in}}%
\pgfpathlineto{\pgfqpoint{3.735918in}{3.030429in}}%
\pgfpathlineto{\pgfqpoint{3.729577in}{3.043212in}}%
\pgfpathlineto{\pgfqpoint{3.723237in}{3.055880in}}%
\pgfpathlineto{\pgfqpoint{3.716899in}{3.068444in}}%
\pgfpathlineto{\pgfqpoint{3.710561in}{3.080917in}}%
\pgfpathlineto{\pgfqpoint{3.704226in}{3.093312in}}%
\pgfpathlineto{\pgfqpoint{3.693096in}{3.089959in}}%
\pgfpathlineto{\pgfqpoint{3.681961in}{3.086648in}}%
\pgfpathlineto{\pgfqpoint{3.670821in}{3.083387in}}%
\pgfpathlineto{\pgfqpoint{3.659677in}{3.080167in}}%
\pgfpathlineto{\pgfqpoint{3.648528in}{3.076975in}}%
\pgfpathlineto{\pgfqpoint{3.654854in}{3.064224in}}%
\pgfpathlineto{\pgfqpoint{3.661181in}{3.051384in}}%
\pgfpathlineto{\pgfqpoint{3.667510in}{3.038464in}}%
\pgfpathlineto{\pgfqpoint{3.673840in}{3.025474in}}%
\pgfpathclose%
\pgfusepath{stroke,fill}%
\end{pgfscope}%
\begin{pgfscope}%
\pgfpathrectangle{\pgfqpoint{0.887500in}{0.275000in}}{\pgfqpoint{4.225000in}{4.225000in}}%
\pgfusepath{clip}%
\pgfsetbuttcap%
\pgfsetroundjoin%
\definecolor{currentfill}{rgb}{0.647257,0.858400,0.209861}%
\pgfsetfillcolor{currentfill}%
\pgfsetfillopacity{0.700000}%
\pgfsetlinewidth{0.501875pt}%
\definecolor{currentstroke}{rgb}{1.000000,1.000000,1.000000}%
\pgfsetstrokecolor{currentstroke}%
\pgfsetstrokeopacity{0.500000}%
\pgfsetdash{}{0pt}%
\pgfpathmoveto{\pgfqpoint{3.242420in}{3.232138in}}%
\pgfpathlineto{\pgfqpoint{3.253684in}{3.235380in}}%
\pgfpathlineto{\pgfqpoint{3.264942in}{3.238653in}}%
\pgfpathlineto{\pgfqpoint{3.276195in}{3.241942in}}%
\pgfpathlineto{\pgfqpoint{3.287443in}{3.245232in}}%
\pgfpathlineto{\pgfqpoint{3.298685in}{3.248507in}}%
\pgfpathlineto{\pgfqpoint{3.292422in}{3.258638in}}%
\pgfpathlineto{\pgfqpoint{3.286162in}{3.268674in}}%
\pgfpathlineto{\pgfqpoint{3.279906in}{3.278654in}}%
\pgfpathlineto{\pgfqpoint{3.273653in}{3.288616in}}%
\pgfpathlineto{\pgfqpoint{3.267403in}{3.298597in}}%
\pgfpathlineto{\pgfqpoint{3.256171in}{3.295685in}}%
\pgfpathlineto{\pgfqpoint{3.244934in}{3.292890in}}%
\pgfpathlineto{\pgfqpoint{3.233692in}{3.290226in}}%
\pgfpathlineto{\pgfqpoint{3.222445in}{3.287706in}}%
\pgfpathlineto{\pgfqpoint{3.211192in}{3.285340in}}%
\pgfpathlineto{\pgfqpoint{3.217431in}{3.274801in}}%
\pgfpathlineto{\pgfqpoint{3.223674in}{3.264228in}}%
\pgfpathlineto{\pgfqpoint{3.229919in}{3.253605in}}%
\pgfpathlineto{\pgfqpoint{3.236168in}{3.242914in}}%
\pgfpathclose%
\pgfusepath{stroke,fill}%
\end{pgfscope}%
\begin{pgfscope}%
\pgfpathrectangle{\pgfqpoint{0.887500in}{0.275000in}}{\pgfqpoint{4.225000in}{4.225000in}}%
\pgfusepath{clip}%
\pgfsetbuttcap%
\pgfsetroundjoin%
\definecolor{currentfill}{rgb}{0.133743,0.548535,0.553541}%
\pgfsetfillcolor{currentfill}%
\pgfsetfillopacity{0.700000}%
\pgfsetlinewidth{0.501875pt}%
\definecolor{currentstroke}{rgb}{1.000000,1.000000,1.000000}%
\pgfsetstrokecolor{currentstroke}%
\pgfsetstrokeopacity{0.500000}%
\pgfsetdash{}{0pt}%
\pgfpathmoveto{\pgfqpoint{4.442537in}{2.456554in}}%
\pgfpathlineto{\pgfqpoint{4.453508in}{2.460067in}}%
\pgfpathlineto{\pgfqpoint{4.464474in}{2.463603in}}%
\pgfpathlineto{\pgfqpoint{4.475437in}{2.467163in}}%
\pgfpathlineto{\pgfqpoint{4.486395in}{2.470748in}}%
\pgfpathlineto{\pgfqpoint{4.479933in}{2.484838in}}%
\pgfpathlineto{\pgfqpoint{4.473476in}{2.498954in}}%
\pgfpathlineto{\pgfqpoint{4.467021in}{2.513093in}}%
\pgfpathlineto{\pgfqpoint{4.460571in}{2.527258in}}%
\pgfpathlineto{\pgfqpoint{4.454123in}{2.541446in}}%
\pgfpathlineto{\pgfqpoint{4.443173in}{2.538023in}}%
\pgfpathlineto{\pgfqpoint{4.432217in}{2.534613in}}%
\pgfpathlineto{\pgfqpoint{4.421257in}{2.531215in}}%
\pgfpathlineto{\pgfqpoint{4.410292in}{2.527829in}}%
\pgfpathlineto{\pgfqpoint{4.416734in}{2.513529in}}%
\pgfpathlineto{\pgfqpoint{4.423179in}{2.499247in}}%
\pgfpathlineto{\pgfqpoint{4.429628in}{2.484987in}}%
\pgfpathlineto{\pgfqpoint{4.436081in}{2.470754in}}%
\pgfpathclose%
\pgfusepath{stroke,fill}%
\end{pgfscope}%
\begin{pgfscope}%
\pgfpathrectangle{\pgfqpoint{0.887500in}{0.275000in}}{\pgfqpoint{4.225000in}{4.225000in}}%
\pgfusepath{clip}%
\pgfsetbuttcap%
\pgfsetroundjoin%
\definecolor{currentfill}{rgb}{0.133743,0.548535,0.553541}%
\pgfsetfillcolor{currentfill}%
\pgfsetfillopacity{0.700000}%
\pgfsetlinewidth{0.501875pt}%
\definecolor{currentstroke}{rgb}{1.000000,1.000000,1.000000}%
\pgfsetstrokecolor{currentstroke}%
\pgfsetstrokeopacity{0.500000}%
\pgfsetdash{}{0pt}%
\pgfpathmoveto{\pgfqpoint{1.955947in}{2.471139in}}%
\pgfpathlineto{\pgfqpoint{1.967530in}{2.474412in}}%
\pgfpathlineto{\pgfqpoint{1.979107in}{2.477692in}}%
\pgfpathlineto{\pgfqpoint{1.990678in}{2.480979in}}%
\pgfpathlineto{\pgfqpoint{2.002244in}{2.484278in}}%
\pgfpathlineto{\pgfqpoint{2.013803in}{2.487591in}}%
\pgfpathlineto{\pgfqpoint{2.007956in}{2.495813in}}%
\pgfpathlineto{\pgfqpoint{2.002113in}{2.504017in}}%
\pgfpathlineto{\pgfqpoint{1.996275in}{2.512203in}}%
\pgfpathlineto{\pgfqpoint{1.990441in}{2.520372in}}%
\pgfpathlineto{\pgfqpoint{1.984611in}{2.528524in}}%
\pgfpathlineto{\pgfqpoint{1.973064in}{2.525223in}}%
\pgfpathlineto{\pgfqpoint{1.961511in}{2.521936in}}%
\pgfpathlineto{\pgfqpoint{1.949952in}{2.518661in}}%
\pgfpathlineto{\pgfqpoint{1.938387in}{2.515395in}}%
\pgfpathlineto{\pgfqpoint{1.926816in}{2.512136in}}%
\pgfpathlineto{\pgfqpoint{1.932634in}{2.503972in}}%
\pgfpathlineto{\pgfqpoint{1.938455in}{2.495792in}}%
\pgfpathlineto{\pgfqpoint{1.944282in}{2.487593in}}%
\pgfpathlineto{\pgfqpoint{1.950112in}{2.479375in}}%
\pgfpathclose%
\pgfusepath{stroke,fill}%
\end{pgfscope}%
\begin{pgfscope}%
\pgfpathrectangle{\pgfqpoint{0.887500in}{0.275000in}}{\pgfqpoint{4.225000in}{4.225000in}}%
\pgfusepath{clip}%
\pgfsetbuttcap%
\pgfsetroundjoin%
\definecolor{currentfill}{rgb}{0.440137,0.811138,0.340967}%
\pgfsetfillcolor{currentfill}%
\pgfsetfillopacity{0.700000}%
\pgfsetlinewidth{0.501875pt}%
\definecolor{currentstroke}{rgb}{1.000000,1.000000,1.000000}%
\pgfsetstrokecolor{currentstroke}%
\pgfsetstrokeopacity{0.500000}%
\pgfsetdash{}{0pt}%
\pgfpathmoveto{\pgfqpoint{3.592698in}{3.060891in}}%
\pgfpathlineto{\pgfqpoint{3.603876in}{3.064173in}}%
\pgfpathlineto{\pgfqpoint{3.615048in}{3.067409in}}%
\pgfpathlineto{\pgfqpoint{3.626213in}{3.070611in}}%
\pgfpathlineto{\pgfqpoint{3.637373in}{3.073795in}}%
\pgfpathlineto{\pgfqpoint{3.648528in}{3.076975in}}%
\pgfpathlineto{\pgfqpoint{3.642203in}{3.089630in}}%
\pgfpathlineto{\pgfqpoint{3.635880in}{3.102180in}}%
\pgfpathlineto{\pgfqpoint{3.629559in}{3.114615in}}%
\pgfpathlineto{\pgfqpoint{3.623238in}{3.126929in}}%
\pgfpathlineto{\pgfqpoint{3.616919in}{3.139111in}}%
\pgfpathlineto{\pgfqpoint{3.605771in}{3.136114in}}%
\pgfpathlineto{\pgfqpoint{3.594618in}{3.133108in}}%
\pgfpathlineto{\pgfqpoint{3.583459in}{3.130074in}}%
\pgfpathlineto{\pgfqpoint{3.572294in}{3.126994in}}%
\pgfpathlineto{\pgfqpoint{3.561122in}{3.123852in}}%
\pgfpathlineto{\pgfqpoint{3.567436in}{3.111632in}}%
\pgfpathlineto{\pgfqpoint{3.573751in}{3.099203in}}%
\pgfpathlineto{\pgfqpoint{3.580066in}{3.086588in}}%
\pgfpathlineto{\pgfqpoint{3.586382in}{3.073809in}}%
\pgfpathclose%
\pgfusepath{stroke,fill}%
\end{pgfscope}%
\begin{pgfscope}%
\pgfpathrectangle{\pgfqpoint{0.887500in}{0.275000in}}{\pgfqpoint{4.225000in}{4.225000in}}%
\pgfusepath{clip}%
\pgfsetbuttcap%
\pgfsetroundjoin%
\definecolor{currentfill}{rgb}{0.143343,0.522773,0.556295}%
\pgfsetfillcolor{currentfill}%
\pgfsetfillopacity{0.700000}%
\pgfsetlinewidth{0.501875pt}%
\definecolor{currentstroke}{rgb}{1.000000,1.000000,1.000000}%
\pgfsetstrokecolor{currentstroke}%
\pgfsetstrokeopacity{0.500000}%
\pgfsetdash{}{0pt}%
\pgfpathmoveto{\pgfqpoint{2.275257in}{2.412283in}}%
\pgfpathlineto{\pgfqpoint{2.286762in}{2.415561in}}%
\pgfpathlineto{\pgfqpoint{2.298261in}{2.418838in}}%
\pgfpathlineto{\pgfqpoint{2.309755in}{2.422121in}}%
\pgfpathlineto{\pgfqpoint{2.321243in}{2.425412in}}%
\pgfpathlineto{\pgfqpoint{2.332725in}{2.428717in}}%
\pgfpathlineto{\pgfqpoint{2.326767in}{2.437149in}}%
\pgfpathlineto{\pgfqpoint{2.320814in}{2.445565in}}%
\pgfpathlineto{\pgfqpoint{2.314865in}{2.453967in}}%
\pgfpathlineto{\pgfqpoint{2.308920in}{2.462353in}}%
\pgfpathlineto{\pgfqpoint{2.302979in}{2.470724in}}%
\pgfpathlineto{\pgfqpoint{2.291509in}{2.467418in}}%
\pgfpathlineto{\pgfqpoint{2.280033in}{2.464126in}}%
\pgfpathlineto{\pgfqpoint{2.268552in}{2.460844in}}%
\pgfpathlineto{\pgfqpoint{2.257064in}{2.457568in}}%
\pgfpathlineto{\pgfqpoint{2.245571in}{2.454295in}}%
\pgfpathlineto{\pgfqpoint{2.251500in}{2.445923in}}%
\pgfpathlineto{\pgfqpoint{2.257433in}{2.437536in}}%
\pgfpathlineto{\pgfqpoint{2.263370in}{2.429134in}}%
\pgfpathlineto{\pgfqpoint{2.269311in}{2.420716in}}%
\pgfpathclose%
\pgfusepath{stroke,fill}%
\end{pgfscope}%
\begin{pgfscope}%
\pgfpathrectangle{\pgfqpoint{0.887500in}{0.275000in}}{\pgfqpoint{4.225000in}{4.225000in}}%
\pgfusepath{clip}%
\pgfsetbuttcap%
\pgfsetroundjoin%
\definecolor{currentfill}{rgb}{0.154815,0.493313,0.557840}%
\pgfsetfillcolor{currentfill}%
\pgfsetfillopacity{0.700000}%
\pgfsetlinewidth{0.501875pt}%
\definecolor{currentstroke}{rgb}{1.000000,1.000000,1.000000}%
\pgfsetstrokecolor{currentstroke}%
\pgfsetstrokeopacity{0.500000}%
\pgfsetdash{}{0pt}%
\pgfpathmoveto{\pgfqpoint{2.594631in}{2.351131in}}%
\pgfpathlineto{\pgfqpoint{2.606062in}{2.354026in}}%
\pgfpathlineto{\pgfqpoint{2.617486in}{2.356993in}}%
\pgfpathlineto{\pgfqpoint{2.628902in}{2.360142in}}%
\pgfpathlineto{\pgfqpoint{2.640309in}{2.363586in}}%
\pgfpathlineto{\pgfqpoint{2.651707in}{2.367434in}}%
\pgfpathlineto{\pgfqpoint{2.645643in}{2.376062in}}%
\pgfpathlineto{\pgfqpoint{2.639582in}{2.384671in}}%
\pgfpathlineto{\pgfqpoint{2.633526in}{2.393264in}}%
\pgfpathlineto{\pgfqpoint{2.627474in}{2.401844in}}%
\pgfpathlineto{\pgfqpoint{2.621426in}{2.410414in}}%
\pgfpathlineto{\pgfqpoint{2.610040in}{2.406595in}}%
\pgfpathlineto{\pgfqpoint{2.598644in}{2.403173in}}%
\pgfpathlineto{\pgfqpoint{2.587239in}{2.400039in}}%
\pgfpathlineto{\pgfqpoint{2.575826in}{2.397082in}}%
\pgfpathlineto{\pgfqpoint{2.564406in}{2.394195in}}%
\pgfpathlineto{\pgfqpoint{2.570443in}{2.385613in}}%
\pgfpathlineto{\pgfqpoint{2.576484in}{2.377016in}}%
\pgfpathlineto{\pgfqpoint{2.582529in}{2.368404in}}%
\pgfpathlineto{\pgfqpoint{2.588578in}{2.359776in}}%
\pgfpathclose%
\pgfusepath{stroke,fill}%
\end{pgfscope}%
\begin{pgfscope}%
\pgfpathrectangle{\pgfqpoint{0.887500in}{0.275000in}}{\pgfqpoint{4.225000in}{4.225000in}}%
\pgfusepath{clip}%
\pgfsetbuttcap%
\pgfsetroundjoin%
\definecolor{currentfill}{rgb}{0.496615,0.826376,0.306377}%
\pgfsetfillcolor{currentfill}%
\pgfsetfillopacity{0.700000}%
\pgfsetlinewidth{0.501875pt}%
\definecolor{currentstroke}{rgb}{1.000000,1.000000,1.000000}%
\pgfsetstrokecolor{currentstroke}%
\pgfsetstrokeopacity{0.500000}%
\pgfsetdash{}{0pt}%
\pgfpathmoveto{\pgfqpoint{3.505173in}{3.107318in}}%
\pgfpathlineto{\pgfqpoint{3.516375in}{3.110719in}}%
\pgfpathlineto{\pgfqpoint{3.527571in}{3.114077in}}%
\pgfpathlineto{\pgfqpoint{3.538761in}{3.117389in}}%
\pgfpathlineto{\pgfqpoint{3.549944in}{3.120648in}}%
\pgfpathlineto{\pgfqpoint{3.561122in}{3.123852in}}%
\pgfpathlineto{\pgfqpoint{3.554808in}{3.135854in}}%
\pgfpathlineto{\pgfqpoint{3.548495in}{3.147658in}}%
\pgfpathlineto{\pgfqpoint{3.542183in}{3.159287in}}%
\pgfpathlineto{\pgfqpoint{3.535872in}{3.170764in}}%
\pgfpathlineto{\pgfqpoint{3.529562in}{3.182112in}}%
\pgfpathlineto{\pgfqpoint{3.518393in}{3.179131in}}%
\pgfpathlineto{\pgfqpoint{3.507218in}{3.176206in}}%
\pgfpathlineto{\pgfqpoint{3.496039in}{3.173315in}}%
\pgfpathlineto{\pgfqpoint{3.484854in}{3.170436in}}%
\pgfpathlineto{\pgfqpoint{3.473663in}{3.167550in}}%
\pgfpathlineto{\pgfqpoint{3.479962in}{3.155776in}}%
\pgfpathlineto{\pgfqpoint{3.486262in}{3.143867in}}%
\pgfpathlineto{\pgfqpoint{3.492565in}{3.131822in}}%
\pgfpathlineto{\pgfqpoint{3.498868in}{3.119640in}}%
\pgfpathclose%
\pgfusepath{stroke,fill}%
\end{pgfscope}%
\begin{pgfscope}%
\pgfpathrectangle{\pgfqpoint{0.887500in}{0.275000in}}{\pgfqpoint{4.225000in}{4.225000in}}%
\pgfusepath{clip}%
\pgfsetbuttcap%
\pgfsetroundjoin%
\definecolor{currentfill}{rgb}{0.606045,0.850733,0.236712}%
\pgfsetfillcolor{currentfill}%
\pgfsetfillopacity{0.700000}%
\pgfsetlinewidth{0.501875pt}%
\definecolor{currentstroke}{rgb}{1.000000,1.000000,1.000000}%
\pgfsetstrokecolor{currentstroke}%
\pgfsetstrokeopacity{0.500000}%
\pgfsetdash{}{0pt}%
\pgfpathmoveto{\pgfqpoint{3.330033in}{3.195129in}}%
\pgfpathlineto{\pgfqpoint{3.341279in}{3.198601in}}%
\pgfpathlineto{\pgfqpoint{3.352518in}{3.201945in}}%
\pgfpathlineto{\pgfqpoint{3.363750in}{3.205148in}}%
\pgfpathlineto{\pgfqpoint{3.374976in}{3.208213in}}%
\pgfpathlineto{\pgfqpoint{3.386196in}{3.211154in}}%
\pgfpathlineto{\pgfqpoint{3.379915in}{3.222186in}}%
\pgfpathlineto{\pgfqpoint{3.373635in}{3.232969in}}%
\pgfpathlineto{\pgfqpoint{3.367357in}{3.243539in}}%
\pgfpathlineto{\pgfqpoint{3.361082in}{3.253929in}}%
\pgfpathlineto{\pgfqpoint{3.354808in}{3.264175in}}%
\pgfpathlineto{\pgfqpoint{3.343595in}{3.261159in}}%
\pgfpathlineto{\pgfqpoint{3.332376in}{3.258085in}}%
\pgfpathlineto{\pgfqpoint{3.321152in}{3.254949in}}%
\pgfpathlineto{\pgfqpoint{3.309921in}{3.251751in}}%
\pgfpathlineto{\pgfqpoint{3.298685in}{3.248507in}}%
\pgfpathlineto{\pgfqpoint{3.304950in}{3.238246in}}%
\pgfpathlineto{\pgfqpoint{3.311218in}{3.227815in}}%
\pgfpathlineto{\pgfqpoint{3.317488in}{3.217177in}}%
\pgfpathlineto{\pgfqpoint{3.323760in}{3.206294in}}%
\pgfpathclose%
\pgfusepath{stroke,fill}%
\end{pgfscope}%
\begin{pgfscope}%
\pgfpathrectangle{\pgfqpoint{0.887500in}{0.275000in}}{\pgfqpoint{4.225000in}{4.225000in}}%
\pgfusepath{clip}%
\pgfsetbuttcap%
\pgfsetroundjoin%
\definecolor{currentfill}{rgb}{0.157851,0.683765,0.501686}%
\pgfsetfillcolor{currentfill}%
\pgfsetfillopacity{0.700000}%
\pgfsetlinewidth{0.501875pt}%
\definecolor{currentstroke}{rgb}{1.000000,1.000000,1.000000}%
\pgfsetstrokecolor{currentstroke}%
\pgfsetstrokeopacity{0.500000}%
\pgfsetdash{}{0pt}%
\pgfpathmoveto{\pgfqpoint{4.061588in}{2.742995in}}%
\pgfpathlineto{\pgfqpoint{4.072653in}{2.746397in}}%
\pgfpathlineto{\pgfqpoint{4.083712in}{2.749807in}}%
\pgfpathlineto{\pgfqpoint{4.094767in}{2.753224in}}%
\pgfpathlineto{\pgfqpoint{4.105816in}{2.756650in}}%
\pgfpathlineto{\pgfqpoint{4.116861in}{2.760085in}}%
\pgfpathlineto{\pgfqpoint{4.110466in}{2.774041in}}%
\pgfpathlineto{\pgfqpoint{4.104074in}{2.787983in}}%
\pgfpathlineto{\pgfqpoint{4.097684in}{2.801912in}}%
\pgfpathlineto{\pgfqpoint{4.091295in}{2.815828in}}%
\pgfpathlineto{\pgfqpoint{4.084910in}{2.829729in}}%
\pgfpathlineto{\pgfqpoint{4.073866in}{2.826233in}}%
\pgfpathlineto{\pgfqpoint{4.062817in}{2.822747in}}%
\pgfpathlineto{\pgfqpoint{4.051763in}{2.819274in}}%
\pgfpathlineto{\pgfqpoint{4.040705in}{2.815810in}}%
\pgfpathlineto{\pgfqpoint{4.029641in}{2.812357in}}%
\pgfpathlineto{\pgfqpoint{4.036027in}{2.798559in}}%
\pgfpathlineto{\pgfqpoint{4.042415in}{2.784725in}}%
\pgfpathlineto{\pgfqpoint{4.048804in}{2.770850in}}%
\pgfpathlineto{\pgfqpoint{4.055195in}{2.756939in}}%
\pgfpathclose%
\pgfusepath{stroke,fill}%
\end{pgfscope}%
\begin{pgfscope}%
\pgfpathrectangle{\pgfqpoint{0.887500in}{0.275000in}}{\pgfqpoint{4.225000in}{4.225000in}}%
\pgfusepath{clip}%
\pgfsetbuttcap%
\pgfsetroundjoin%
\definecolor{currentfill}{rgb}{0.555484,0.840254,0.269281}%
\pgfsetfillcolor{currentfill}%
\pgfsetfillopacity{0.700000}%
\pgfsetlinewidth{0.501875pt}%
\definecolor{currentstroke}{rgb}{1.000000,1.000000,1.000000}%
\pgfsetstrokecolor{currentstroke}%
\pgfsetstrokeopacity{0.500000}%
\pgfsetdash{}{0pt}%
\pgfpathmoveto{\pgfqpoint{3.417618in}{3.152418in}}%
\pgfpathlineto{\pgfqpoint{3.428839in}{3.155572in}}%
\pgfpathlineto{\pgfqpoint{3.440054in}{3.158654in}}%
\pgfpathlineto{\pgfqpoint{3.451263in}{3.161673in}}%
\pgfpathlineto{\pgfqpoint{3.462466in}{3.164635in}}%
\pgfpathlineto{\pgfqpoint{3.473663in}{3.167550in}}%
\pgfpathlineto{\pgfqpoint{3.467366in}{3.179192in}}%
\pgfpathlineto{\pgfqpoint{3.461070in}{3.190704in}}%
\pgfpathlineto{\pgfqpoint{3.454777in}{3.202086in}}%
\pgfpathlineto{\pgfqpoint{3.448485in}{3.213341in}}%
\pgfpathlineto{\pgfqpoint{3.442196in}{3.224470in}}%
\pgfpathlineto{\pgfqpoint{3.431008in}{3.221945in}}%
\pgfpathlineto{\pgfqpoint{3.419815in}{3.219367in}}%
\pgfpathlineto{\pgfqpoint{3.408615in}{3.216718in}}%
\pgfpathlineto{\pgfqpoint{3.397409in}{3.213985in}}%
\pgfpathlineto{\pgfqpoint{3.386196in}{3.211154in}}%
\pgfpathlineto{\pgfqpoint{3.392478in}{3.199856in}}%
\pgfpathlineto{\pgfqpoint{3.398761in}{3.188310in}}%
\pgfpathlineto{\pgfqpoint{3.405045in}{3.176539in}}%
\pgfpathlineto{\pgfqpoint{3.411331in}{3.164568in}}%
\pgfpathclose%
\pgfusepath{stroke,fill}%
\end{pgfscope}%
\begin{pgfscope}%
\pgfpathrectangle{\pgfqpoint{0.887500in}{0.275000in}}{\pgfqpoint{4.225000in}{4.225000in}}%
\pgfusepath{clip}%
\pgfsetbuttcap%
\pgfsetroundjoin%
\definecolor{currentfill}{rgb}{0.124395,0.578002,0.548287}%
\pgfsetfillcolor{currentfill}%
\pgfsetfillopacity{0.700000}%
\pgfsetlinewidth{0.501875pt}%
\definecolor{currentstroke}{rgb}{1.000000,1.000000,1.000000}%
\pgfsetstrokecolor{currentstroke}%
\pgfsetstrokeopacity{0.500000}%
\pgfsetdash{}{0pt}%
\pgfpathmoveto{\pgfqpoint{4.355391in}{2.511047in}}%
\pgfpathlineto{\pgfqpoint{4.366381in}{2.514386in}}%
\pgfpathlineto{\pgfqpoint{4.377366in}{2.517733in}}%
\pgfpathlineto{\pgfqpoint{4.388346in}{2.521089in}}%
\pgfpathlineto{\pgfqpoint{4.399321in}{2.524454in}}%
\pgfpathlineto{\pgfqpoint{4.410292in}{2.527829in}}%
\pgfpathlineto{\pgfqpoint{4.403852in}{2.542141in}}%
\pgfpathlineto{\pgfqpoint{4.397415in}{2.556461in}}%
\pgfpathlineto{\pgfqpoint{4.390981in}{2.570783in}}%
\pgfpathlineto{\pgfqpoint{4.384549in}{2.585103in}}%
\pgfpathlineto{\pgfqpoint{4.378120in}{2.599418in}}%
\pgfpathlineto{\pgfqpoint{4.367153in}{2.596090in}}%
\pgfpathlineto{\pgfqpoint{4.356180in}{2.592760in}}%
\pgfpathlineto{\pgfqpoint{4.345202in}{2.589426in}}%
\pgfpathlineto{\pgfqpoint{4.334219in}{2.586089in}}%
\pgfpathlineto{\pgfqpoint{4.323230in}{2.582746in}}%
\pgfpathlineto{\pgfqpoint{4.329658in}{2.568439in}}%
\pgfpathlineto{\pgfqpoint{4.336088in}{2.554114in}}%
\pgfpathlineto{\pgfqpoint{4.342521in}{2.539771in}}%
\pgfpathlineto{\pgfqpoint{4.348955in}{2.525413in}}%
\pgfpathclose%
\pgfusepath{stroke,fill}%
\end{pgfscope}%
\begin{pgfscope}%
\pgfpathrectangle{\pgfqpoint{0.887500in}{0.275000in}}{\pgfqpoint{4.225000in}{4.225000in}}%
\pgfusepath{clip}%
\pgfsetbuttcap%
\pgfsetroundjoin%
\definecolor{currentfill}{rgb}{0.144759,0.519093,0.556572}%
\pgfsetfillcolor{currentfill}%
\pgfsetfillopacity{0.700000}%
\pgfsetlinewidth{0.501875pt}%
\definecolor{currentstroke}{rgb}{1.000000,1.000000,1.000000}%
\pgfsetstrokecolor{currentstroke}%
\pgfsetstrokeopacity{0.500000}%
\pgfsetdash{}{0pt}%
\pgfpathmoveto{\pgfqpoint{2.883222in}{2.341987in}}%
\pgfpathlineto{\pgfqpoint{2.894483in}{2.364056in}}%
\pgfpathlineto{\pgfqpoint{2.905734in}{2.390167in}}%
\pgfpathlineto{\pgfqpoint{2.916982in}{2.419218in}}%
\pgfpathlineto{\pgfqpoint{2.928234in}{2.450106in}}%
\pgfpathlineto{\pgfqpoint{2.939493in}{2.481724in}}%
\pgfpathlineto{\pgfqpoint{2.933316in}{2.493248in}}%
\pgfpathlineto{\pgfqpoint{2.927140in}{2.505281in}}%
\pgfpathlineto{\pgfqpoint{2.920965in}{2.517644in}}%
\pgfpathlineto{\pgfqpoint{2.914793in}{2.530156in}}%
\pgfpathlineto{\pgfqpoint{2.908624in}{2.542637in}}%
\pgfpathlineto{\pgfqpoint{2.897410in}{2.506721in}}%
\pgfpathlineto{\pgfqpoint{2.886206in}{2.471438in}}%
\pgfpathlineto{\pgfqpoint{2.875005in}{2.438167in}}%
\pgfpathlineto{\pgfqpoint{2.863800in}{2.408278in}}%
\pgfpathlineto{\pgfqpoint{2.852579in}{2.383137in}}%
\pgfpathlineto{\pgfqpoint{2.858701in}{2.374674in}}%
\pgfpathlineto{\pgfqpoint{2.864825in}{2.366426in}}%
\pgfpathlineto{\pgfqpoint{2.870953in}{2.358295in}}%
\pgfpathlineto{\pgfqpoint{2.877085in}{2.350182in}}%
\pgfpathclose%
\pgfusepath{stroke,fill}%
\end{pgfscope}%
\begin{pgfscope}%
\pgfpathrectangle{\pgfqpoint{0.887500in}{0.275000in}}{\pgfqpoint{4.225000in}{4.225000in}}%
\pgfusepath{clip}%
\pgfsetbuttcap%
\pgfsetroundjoin%
\definecolor{currentfill}{rgb}{0.127568,0.566949,0.550556}%
\pgfsetfillcolor{currentfill}%
\pgfsetfillopacity{0.700000}%
\pgfsetlinewidth{0.501875pt}%
\definecolor{currentstroke}{rgb}{1.000000,1.000000,1.000000}%
\pgfsetstrokecolor{currentstroke}%
\pgfsetstrokeopacity{0.500000}%
\pgfsetdash{}{0pt}%
\pgfpathmoveto{\pgfqpoint{1.723732in}{2.503631in}}%
\pgfpathlineto{\pgfqpoint{1.735375in}{2.506947in}}%
\pgfpathlineto{\pgfqpoint{1.747012in}{2.510261in}}%
\pgfpathlineto{\pgfqpoint{1.758644in}{2.513570in}}%
\pgfpathlineto{\pgfqpoint{1.770270in}{2.516873in}}%
\pgfpathlineto{\pgfqpoint{1.781891in}{2.520167in}}%
\pgfpathlineto{\pgfqpoint{1.776123in}{2.528259in}}%
\pgfpathlineto{\pgfqpoint{1.770360in}{2.536335in}}%
\pgfpathlineto{\pgfqpoint{1.764601in}{2.544397in}}%
\pgfpathlineto{\pgfqpoint{1.758847in}{2.552444in}}%
\pgfpathlineto{\pgfqpoint{1.753096in}{2.560475in}}%
\pgfpathlineto{\pgfqpoint{1.741488in}{2.557197in}}%
\pgfpathlineto{\pgfqpoint{1.729874in}{2.553912in}}%
\pgfpathlineto{\pgfqpoint{1.718255in}{2.550621in}}%
\pgfpathlineto{\pgfqpoint{1.706631in}{2.547326in}}%
\pgfpathlineto{\pgfqpoint{1.695000in}{2.544029in}}%
\pgfpathlineto{\pgfqpoint{1.700738in}{2.535980in}}%
\pgfpathlineto{\pgfqpoint{1.706480in}{2.527915in}}%
\pgfpathlineto{\pgfqpoint{1.712227in}{2.519835in}}%
\pgfpathlineto{\pgfqpoint{1.717977in}{2.511741in}}%
\pgfpathclose%
\pgfusepath{stroke,fill}%
\end{pgfscope}%
\begin{pgfscope}%
\pgfpathrectangle{\pgfqpoint{0.887500in}{0.275000in}}{\pgfqpoint{4.225000in}{4.225000in}}%
\pgfusepath{clip}%
\pgfsetbuttcap%
\pgfsetroundjoin%
\definecolor{currentfill}{rgb}{0.191090,0.708366,0.482284}%
\pgfsetfillcolor{currentfill}%
\pgfsetfillopacity{0.700000}%
\pgfsetlinewidth{0.501875pt}%
\definecolor{currentstroke}{rgb}{1.000000,1.000000,1.000000}%
\pgfsetstrokecolor{currentstroke}%
\pgfsetstrokeopacity{0.500000}%
\pgfsetdash{}{0pt}%
\pgfpathmoveto{\pgfqpoint{3.974245in}{2.795187in}}%
\pgfpathlineto{\pgfqpoint{3.985335in}{2.798612in}}%
\pgfpathlineto{\pgfqpoint{3.996419in}{2.802040in}}%
\pgfpathlineto{\pgfqpoint{4.007498in}{2.805473in}}%
\pgfpathlineto{\pgfqpoint{4.018572in}{2.808911in}}%
\pgfpathlineto{\pgfqpoint{4.029641in}{2.812357in}}%
\pgfpathlineto{\pgfqpoint{4.023257in}{2.826123in}}%
\pgfpathlineto{\pgfqpoint{4.016875in}{2.839864in}}%
\pgfpathlineto{\pgfqpoint{4.010494in}{2.853587in}}%
\pgfpathlineto{\pgfqpoint{4.004117in}{2.867297in}}%
\pgfpathlineto{\pgfqpoint{3.997741in}{2.881000in}}%
\pgfpathlineto{\pgfqpoint{3.986672in}{2.877442in}}%
\pgfpathlineto{\pgfqpoint{3.975599in}{2.873901in}}%
\pgfpathlineto{\pgfqpoint{3.964520in}{2.870381in}}%
\pgfpathlineto{\pgfqpoint{3.953437in}{2.866883in}}%
\pgfpathlineto{\pgfqpoint{3.942349in}{2.863405in}}%
\pgfpathlineto{\pgfqpoint{3.948724in}{2.849785in}}%
\pgfpathlineto{\pgfqpoint{3.955101in}{2.836163in}}%
\pgfpathlineto{\pgfqpoint{3.961480in}{2.822530in}}%
\pgfpathlineto{\pgfqpoint{3.967862in}{2.808874in}}%
\pgfpathclose%
\pgfusepath{stroke,fill}%
\end{pgfscope}%
\begin{pgfscope}%
\pgfpathrectangle{\pgfqpoint{0.887500in}{0.275000in}}{\pgfqpoint{4.225000in}{4.225000in}}%
\pgfusepath{clip}%
\pgfsetbuttcap%
\pgfsetroundjoin%
\definecolor{currentfill}{rgb}{0.136408,0.541173,0.554483}%
\pgfsetfillcolor{currentfill}%
\pgfsetfillopacity{0.700000}%
\pgfsetlinewidth{0.501875pt}%
\definecolor{currentstroke}{rgb}{1.000000,1.000000,1.000000}%
\pgfsetstrokecolor{currentstroke}%
\pgfsetstrokeopacity{0.500000}%
\pgfsetdash{}{0pt}%
\pgfpathmoveto{\pgfqpoint{2.043101in}{2.446202in}}%
\pgfpathlineto{\pgfqpoint{2.054666in}{2.449541in}}%
\pgfpathlineto{\pgfqpoint{2.066225in}{2.452888in}}%
\pgfpathlineto{\pgfqpoint{2.077779in}{2.456239in}}%
\pgfpathlineto{\pgfqpoint{2.089327in}{2.459590in}}%
\pgfpathlineto{\pgfqpoint{2.100869in}{2.462937in}}%
\pgfpathlineto{\pgfqpoint{2.094989in}{2.471237in}}%
\pgfpathlineto{\pgfqpoint{2.089113in}{2.479520in}}%
\pgfpathlineto{\pgfqpoint{2.083241in}{2.487785in}}%
\pgfpathlineto{\pgfqpoint{2.077373in}{2.496032in}}%
\pgfpathlineto{\pgfqpoint{2.071510in}{2.504261in}}%
\pgfpathlineto{\pgfqpoint{2.059980in}{2.500925in}}%
\pgfpathlineto{\pgfqpoint{2.048444in}{2.497586in}}%
\pgfpathlineto{\pgfqpoint{2.036903in}{2.494249in}}%
\pgfpathlineto{\pgfqpoint{2.025356in}{2.490916in}}%
\pgfpathlineto{\pgfqpoint{2.013803in}{2.487591in}}%
\pgfpathlineto{\pgfqpoint{2.019654in}{2.479350in}}%
\pgfpathlineto{\pgfqpoint{2.025509in}{2.471091in}}%
\pgfpathlineto{\pgfqpoint{2.031369in}{2.462814in}}%
\pgfpathlineto{\pgfqpoint{2.037233in}{2.454517in}}%
\pgfpathclose%
\pgfusepath{stroke,fill}%
\end{pgfscope}%
\begin{pgfscope}%
\pgfpathrectangle{\pgfqpoint{0.887500in}{0.275000in}}{\pgfqpoint{4.225000in}{4.225000in}}%
\pgfusepath{clip}%
\pgfsetbuttcap%
\pgfsetroundjoin%
\definecolor{currentfill}{rgb}{0.147607,0.511733,0.557049}%
\pgfsetfillcolor{currentfill}%
\pgfsetfillopacity{0.700000}%
\pgfsetlinewidth{0.501875pt}%
\definecolor{currentstroke}{rgb}{1.000000,1.000000,1.000000}%
\pgfsetstrokecolor{currentstroke}%
\pgfsetstrokeopacity{0.500000}%
\pgfsetdash{}{0pt}%
\pgfpathmoveto{\pgfqpoint{2.362573in}{2.386326in}}%
\pgfpathlineto{\pgfqpoint{2.374060in}{2.389646in}}%
\pgfpathlineto{\pgfqpoint{2.385542in}{2.392984in}}%
\pgfpathlineto{\pgfqpoint{2.397018in}{2.396347in}}%
\pgfpathlineto{\pgfqpoint{2.408487in}{2.399739in}}%
\pgfpathlineto{\pgfqpoint{2.419950in}{2.403165in}}%
\pgfpathlineto{\pgfqpoint{2.413961in}{2.411672in}}%
\pgfpathlineto{\pgfqpoint{2.407975in}{2.420165in}}%
\pgfpathlineto{\pgfqpoint{2.401994in}{2.428644in}}%
\pgfpathlineto{\pgfqpoint{2.396016in}{2.437109in}}%
\pgfpathlineto{\pgfqpoint{2.390043in}{2.445560in}}%
\pgfpathlineto{\pgfqpoint{2.378591in}{2.442139in}}%
\pgfpathlineto{\pgfqpoint{2.367134in}{2.438747in}}%
\pgfpathlineto{\pgfqpoint{2.355670in}{2.435381in}}%
\pgfpathlineto{\pgfqpoint{2.344200in}{2.432039in}}%
\pgfpathlineto{\pgfqpoint{2.332725in}{2.428717in}}%
\pgfpathlineto{\pgfqpoint{2.338686in}{2.420270in}}%
\pgfpathlineto{\pgfqpoint{2.344652in}{2.411807in}}%
\pgfpathlineto{\pgfqpoint{2.350621in}{2.403329in}}%
\pgfpathlineto{\pgfqpoint{2.356595in}{2.394835in}}%
\pgfpathclose%
\pgfusepath{stroke,fill}%
\end{pgfscope}%
\begin{pgfscope}%
\pgfpathrectangle{\pgfqpoint{0.887500in}{0.275000in}}{\pgfqpoint{4.225000in}{4.225000in}}%
\pgfusepath{clip}%
\pgfsetbuttcap%
\pgfsetroundjoin%
\definecolor{currentfill}{rgb}{0.119738,0.603785,0.541400}%
\pgfsetfillcolor{currentfill}%
\pgfsetfillopacity{0.700000}%
\pgfsetlinewidth{0.501875pt}%
\definecolor{currentstroke}{rgb}{1.000000,1.000000,1.000000}%
\pgfsetstrokecolor{currentstroke}%
\pgfsetstrokeopacity{0.500000}%
\pgfsetdash{}{0pt}%
\pgfpathmoveto{\pgfqpoint{4.268201in}{2.565903in}}%
\pgfpathlineto{\pgfqpoint{4.279218in}{2.569293in}}%
\pgfpathlineto{\pgfqpoint{4.290230in}{2.572670in}}%
\pgfpathlineto{\pgfqpoint{4.301235in}{2.576038in}}%
\pgfpathlineto{\pgfqpoint{4.312235in}{2.579396in}}%
\pgfpathlineto{\pgfqpoint{4.323230in}{2.582746in}}%
\pgfpathlineto{\pgfqpoint{4.316803in}{2.597032in}}%
\pgfpathlineto{\pgfqpoint{4.310379in}{2.611297in}}%
\pgfpathlineto{\pgfqpoint{4.303956in}{2.625539in}}%
\pgfpathlineto{\pgfqpoint{4.297535in}{2.639757in}}%
\pgfpathlineto{\pgfqpoint{4.291115in}{2.653950in}}%
\pgfpathlineto{\pgfqpoint{4.280121in}{2.650559in}}%
\pgfpathlineto{\pgfqpoint{4.269121in}{2.647160in}}%
\pgfpathlineto{\pgfqpoint{4.258116in}{2.643751in}}%
\pgfpathlineto{\pgfqpoint{4.247105in}{2.640332in}}%
\pgfpathlineto{\pgfqpoint{4.236088in}{2.636900in}}%
\pgfpathlineto{\pgfqpoint{4.242507in}{2.622743in}}%
\pgfpathlineto{\pgfqpoint{4.248928in}{2.608563in}}%
\pgfpathlineto{\pgfqpoint{4.255350in}{2.594363in}}%
\pgfpathlineto{\pgfqpoint{4.261775in}{2.580143in}}%
\pgfpathclose%
\pgfusepath{stroke,fill}%
\end{pgfscope}%
\begin{pgfscope}%
\pgfpathrectangle{\pgfqpoint{0.887500in}{0.275000in}}{\pgfqpoint{4.225000in}{4.225000in}}%
\pgfusepath{clip}%
\pgfsetbuttcap%
\pgfsetroundjoin%
\definecolor{currentfill}{rgb}{0.157729,0.485932,0.558013}%
\pgfsetfillcolor{currentfill}%
\pgfsetfillopacity{0.700000}%
\pgfsetlinewidth{0.501875pt}%
\definecolor{currentstroke}{rgb}{1.000000,1.000000,1.000000}%
\pgfsetstrokecolor{currentstroke}%
\pgfsetstrokeopacity{0.500000}%
\pgfsetdash{}{0pt}%
\pgfpathmoveto{\pgfqpoint{2.682090in}{2.323953in}}%
\pgfpathlineto{\pgfqpoint{2.693490in}{2.328219in}}%
\pgfpathlineto{\pgfqpoint{2.704882in}{2.332807in}}%
\pgfpathlineto{\pgfqpoint{2.716269in}{2.337360in}}%
\pgfpathlineto{\pgfqpoint{2.727655in}{2.341510in}}%
\pgfpathlineto{\pgfqpoint{2.739041in}{2.344892in}}%
\pgfpathlineto{\pgfqpoint{2.732944in}{2.353721in}}%
\pgfpathlineto{\pgfqpoint{2.726851in}{2.362559in}}%
\pgfpathlineto{\pgfqpoint{2.720761in}{2.371410in}}%
\pgfpathlineto{\pgfqpoint{2.714675in}{2.380278in}}%
\pgfpathlineto{\pgfqpoint{2.708593in}{2.389169in}}%
\pgfpathlineto{\pgfqpoint{2.697219in}{2.385570in}}%
\pgfpathlineto{\pgfqpoint{2.685846in}{2.381224in}}%
\pgfpathlineto{\pgfqpoint{2.674473in}{2.376505in}}%
\pgfpathlineto{\pgfqpoint{2.663094in}{2.371786in}}%
\pgfpathlineto{\pgfqpoint{2.651707in}{2.367434in}}%
\pgfpathlineto{\pgfqpoint{2.657775in}{2.358783in}}%
\pgfpathlineto{\pgfqpoint{2.663848in}{2.350110in}}%
\pgfpathlineto{\pgfqpoint{2.669924in}{2.341414in}}%
\pgfpathlineto{\pgfqpoint{2.676005in}{2.332695in}}%
\pgfpathclose%
\pgfusepath{stroke,fill}%
\end{pgfscope}%
\begin{pgfscope}%
\pgfpathrectangle{\pgfqpoint{0.887500in}{0.275000in}}{\pgfqpoint{4.225000in}{4.225000in}}%
\pgfusepath{clip}%
\pgfsetbuttcap%
\pgfsetroundjoin%
\definecolor{currentfill}{rgb}{0.232815,0.732247,0.459277}%
\pgfsetfillcolor{currentfill}%
\pgfsetfillopacity{0.700000}%
\pgfsetlinewidth{0.501875pt}%
\definecolor{currentstroke}{rgb}{1.000000,1.000000,1.000000}%
\pgfsetstrokecolor{currentstroke}%
\pgfsetstrokeopacity{0.500000}%
\pgfsetdash{}{0pt}%
\pgfpathmoveto{\pgfqpoint{3.886833in}{2.846165in}}%
\pgfpathlineto{\pgfqpoint{3.897947in}{2.849610in}}%
\pgfpathlineto{\pgfqpoint{3.909055in}{2.853052in}}%
\pgfpathlineto{\pgfqpoint{3.920158in}{2.856495in}}%
\pgfpathlineto{\pgfqpoint{3.931256in}{2.859944in}}%
\pgfpathlineto{\pgfqpoint{3.942349in}{2.863405in}}%
\pgfpathlineto{\pgfqpoint{3.935977in}{2.877033in}}%
\pgfpathlineto{\pgfqpoint{3.929609in}{2.890679in}}%
\pgfpathlineto{\pgfqpoint{3.923243in}{2.904352in}}%
\pgfpathlineto{\pgfqpoint{3.916881in}{2.918062in}}%
\pgfpathlineto{\pgfqpoint{3.910522in}{2.931809in}}%
\pgfpathlineto{\pgfqpoint{3.899431in}{2.928314in}}%
\pgfpathlineto{\pgfqpoint{3.888336in}{2.924833in}}%
\pgfpathlineto{\pgfqpoint{3.877235in}{2.921362in}}%
\pgfpathlineto{\pgfqpoint{3.866129in}{2.917894in}}%
\pgfpathlineto{\pgfqpoint{3.855017in}{2.914422in}}%
\pgfpathlineto{\pgfqpoint{3.861374in}{2.900720in}}%
\pgfpathlineto{\pgfqpoint{3.867734in}{2.887046in}}%
\pgfpathlineto{\pgfqpoint{3.874098in}{2.873401in}}%
\pgfpathlineto{\pgfqpoint{3.880464in}{2.859777in}}%
\pgfpathclose%
\pgfusepath{stroke,fill}%
\end{pgfscope}%
\begin{pgfscope}%
\pgfpathrectangle{\pgfqpoint{0.887500in}{0.275000in}}{\pgfqpoint{4.225000in}{4.225000in}}%
\pgfusepath{clip}%
\pgfsetbuttcap%
\pgfsetroundjoin%
\definecolor{currentfill}{rgb}{0.123463,0.581687,0.547445}%
\pgfsetfillcolor{currentfill}%
\pgfsetfillopacity{0.700000}%
\pgfsetlinewidth{0.501875pt}%
\definecolor{currentstroke}{rgb}{1.000000,1.000000,1.000000}%
\pgfsetstrokecolor{currentstroke}%
\pgfsetstrokeopacity{0.500000}%
\pgfsetdash{}{0pt}%
\pgfpathmoveto{\pgfqpoint{1.491430in}{2.534355in}}%
\pgfpathlineto{\pgfqpoint{1.503132in}{2.537723in}}%
\pgfpathlineto{\pgfqpoint{1.514828in}{2.541077in}}%
\pgfpathlineto{\pgfqpoint{1.526518in}{2.544420in}}%
\pgfpathlineto{\pgfqpoint{1.538204in}{2.547752in}}%
\pgfpathlineto{\pgfqpoint{1.549884in}{2.551074in}}%
\pgfpathlineto{\pgfqpoint{1.544198in}{2.559021in}}%
\pgfpathlineto{\pgfqpoint{1.538517in}{2.566940in}}%
\pgfpathlineto{\pgfqpoint{1.532841in}{2.574830in}}%
\pgfpathlineto{\pgfqpoint{1.527170in}{2.582689in}}%
\pgfpathlineto{\pgfqpoint{1.521504in}{2.590520in}}%
\pgfpathlineto{\pgfqpoint{1.509837in}{2.587202in}}%
\pgfpathlineto{\pgfqpoint{1.498165in}{2.583875in}}%
\pgfpathlineto{\pgfqpoint{1.486488in}{2.580538in}}%
\pgfpathlineto{\pgfqpoint{1.474806in}{2.577189in}}%
\pgfpathlineto{\pgfqpoint{1.463118in}{2.573828in}}%
\pgfpathlineto{\pgfqpoint{1.468771in}{2.565996in}}%
\pgfpathlineto{\pgfqpoint{1.474428in}{2.558133in}}%
\pgfpathlineto{\pgfqpoint{1.480091in}{2.550238in}}%
\pgfpathlineto{\pgfqpoint{1.485758in}{2.542311in}}%
\pgfpathclose%
\pgfusepath{stroke,fill}%
\end{pgfscope}%
\begin{pgfscope}%
\pgfpathrectangle{\pgfqpoint{0.887500in}{0.275000in}}{\pgfqpoint{4.225000in}{4.225000in}}%
\pgfusepath{clip}%
\pgfsetbuttcap%
\pgfsetroundjoin%
\definecolor{currentfill}{rgb}{0.163625,0.471133,0.558148}%
\pgfsetfillcolor{currentfill}%
\pgfsetfillopacity{0.700000}%
\pgfsetlinewidth{0.501875pt}%
\definecolor{currentstroke}{rgb}{1.000000,1.000000,1.000000}%
\pgfsetstrokecolor{currentstroke}%
\pgfsetstrokeopacity{0.500000}%
\pgfsetdash{}{0pt}%
\pgfpathmoveto{\pgfqpoint{2.769586in}{2.300702in}}%
\pgfpathlineto{\pgfqpoint{2.780983in}{2.303357in}}%
\pgfpathlineto{\pgfqpoint{2.792382in}{2.304801in}}%
\pgfpathlineto{\pgfqpoint{2.803784in}{2.304796in}}%
\pgfpathlineto{\pgfqpoint{2.815184in}{2.303913in}}%
\pgfpathlineto{\pgfqpoint{2.826574in}{2.303252in}}%
\pgfpathlineto{\pgfqpoint{2.820454in}{2.311262in}}%
\pgfpathlineto{\pgfqpoint{2.814339in}{2.319123in}}%
\pgfpathlineto{\pgfqpoint{2.808228in}{2.326932in}}%
\pgfpathlineto{\pgfqpoint{2.802121in}{2.334787in}}%
\pgfpathlineto{\pgfqpoint{2.796017in}{2.342786in}}%
\pgfpathlineto{\pgfqpoint{2.784627in}{2.344702in}}%
\pgfpathlineto{\pgfqpoint{2.773227in}{2.346844in}}%
\pgfpathlineto{\pgfqpoint{2.761826in}{2.347884in}}%
\pgfpathlineto{\pgfqpoint{2.750431in}{2.347139in}}%
\pgfpathlineto{\pgfqpoint{2.739041in}{2.344892in}}%
\pgfpathlineto{\pgfqpoint{2.745142in}{2.336066in}}%
\pgfpathlineto{\pgfqpoint{2.751247in}{2.327239in}}%
\pgfpathlineto{\pgfqpoint{2.757356in}{2.318406in}}%
\pgfpathlineto{\pgfqpoint{2.763469in}{2.309562in}}%
\pgfpathclose%
\pgfusepath{stroke,fill}%
\end{pgfscope}%
\begin{pgfscope}%
\pgfpathrectangle{\pgfqpoint{0.887500in}{0.275000in}}{\pgfqpoint{4.225000in}{4.225000in}}%
\pgfusepath{clip}%
\pgfsetbuttcap%
\pgfsetroundjoin%
\definecolor{currentfill}{rgb}{0.311925,0.767822,0.415586}%
\pgfsetfillcolor{currentfill}%
\pgfsetfillopacity{0.700000}%
\pgfsetlinewidth{0.501875pt}%
\definecolor{currentstroke}{rgb}{1.000000,1.000000,1.000000}%
\pgfsetstrokecolor{currentstroke}%
\pgfsetstrokeopacity{0.500000}%
\pgfsetdash{}{0pt}%
\pgfpathmoveto{\pgfqpoint{2.959624in}{2.909142in}}%
\pgfpathlineto{\pgfqpoint{2.970904in}{2.936913in}}%
\pgfpathlineto{\pgfqpoint{2.982195in}{2.964049in}}%
\pgfpathlineto{\pgfqpoint{2.993495in}{2.990117in}}%
\pgfpathlineto{\pgfqpoint{3.004804in}{3.014683in}}%
\pgfpathlineto{\pgfqpoint{3.016122in}{3.037312in}}%
\pgfpathlineto{\pgfqpoint{3.009931in}{3.039057in}}%
\pgfpathlineto{\pgfqpoint{3.003746in}{3.040272in}}%
\pgfpathlineto{\pgfqpoint{2.997569in}{3.041091in}}%
\pgfpathlineto{\pgfqpoint{2.991398in}{3.041649in}}%
\pgfpathlineto{\pgfqpoint{2.985234in}{3.042082in}}%
\pgfpathlineto{\pgfqpoint{2.973962in}{3.014591in}}%
\pgfpathlineto{\pgfqpoint{2.962702in}{2.986027in}}%
\pgfpathlineto{\pgfqpoint{2.951452in}{2.956788in}}%
\pgfpathlineto{\pgfqpoint{2.940211in}{2.927271in}}%
\pgfpathlineto{\pgfqpoint{2.928980in}{2.897870in}}%
\pgfpathlineto{\pgfqpoint{2.935094in}{2.900211in}}%
\pgfpathlineto{\pgfqpoint{2.941215in}{2.902786in}}%
\pgfpathlineto{\pgfqpoint{2.947343in}{2.905318in}}%
\pgfpathlineto{\pgfqpoint{2.953479in}{2.907529in}}%
\pgfpathclose%
\pgfusepath{stroke,fill}%
\end{pgfscope}%
\begin{pgfscope}%
\pgfpathrectangle{\pgfqpoint{0.887500in}{0.275000in}}{\pgfqpoint{4.225000in}{4.225000in}}%
\pgfusepath{clip}%
\pgfsetbuttcap%
\pgfsetroundjoin%
\definecolor{currentfill}{rgb}{0.122312,0.633153,0.530398}%
\pgfsetfillcolor{currentfill}%
\pgfsetfillopacity{0.700000}%
\pgfsetlinewidth{0.501875pt}%
\definecolor{currentstroke}{rgb}{1.000000,1.000000,1.000000}%
\pgfsetstrokecolor{currentstroke}%
\pgfsetstrokeopacity{0.500000}%
\pgfsetdash{}{0pt}%
\pgfpathmoveto{\pgfqpoint{4.180923in}{2.619634in}}%
\pgfpathlineto{\pgfqpoint{4.191966in}{2.623090in}}%
\pgfpathlineto{\pgfqpoint{4.203005in}{2.626548in}}%
\pgfpathlineto{\pgfqpoint{4.214038in}{2.630005in}}%
\pgfpathlineto{\pgfqpoint{4.225066in}{2.633457in}}%
\pgfpathlineto{\pgfqpoint{4.236088in}{2.636900in}}%
\pgfpathlineto{\pgfqpoint{4.229671in}{2.651035in}}%
\pgfpathlineto{\pgfqpoint{4.223256in}{2.665146in}}%
\pgfpathlineto{\pgfqpoint{4.216843in}{2.679232in}}%
\pgfpathlineto{\pgfqpoint{4.210431in}{2.693293in}}%
\pgfpathlineto{\pgfqpoint{4.204022in}{2.707334in}}%
\pgfpathlineto{\pgfqpoint{4.193001in}{2.703885in}}%
\pgfpathlineto{\pgfqpoint{4.181975in}{2.700433in}}%
\pgfpathlineto{\pgfqpoint{4.170944in}{2.696978in}}%
\pgfpathlineto{\pgfqpoint{4.159907in}{2.693524in}}%
\pgfpathlineto{\pgfqpoint{4.148866in}{2.690072in}}%
\pgfpathlineto{\pgfqpoint{4.155273in}{2.676019in}}%
\pgfpathlineto{\pgfqpoint{4.161682in}{2.661949in}}%
\pgfpathlineto{\pgfqpoint{4.168094in}{2.647860in}}%
\pgfpathlineto{\pgfqpoint{4.174507in}{2.633755in}}%
\pgfpathclose%
\pgfusepath{stroke,fill}%
\end{pgfscope}%
\begin{pgfscope}%
\pgfpathrectangle{\pgfqpoint{0.887500in}{0.275000in}}{\pgfqpoint{4.225000in}{4.225000in}}%
\pgfusepath{clip}%
\pgfsetbuttcap%
\pgfsetroundjoin%
\definecolor{currentfill}{rgb}{0.129933,0.559582,0.551864}%
\pgfsetfillcolor{currentfill}%
\pgfsetfillopacity{0.700000}%
\pgfsetlinewidth{0.501875pt}%
\definecolor{currentstroke}{rgb}{1.000000,1.000000,1.000000}%
\pgfsetstrokecolor{currentstroke}%
\pgfsetstrokeopacity{0.500000}%
\pgfsetdash{}{0pt}%
\pgfpathmoveto{\pgfqpoint{1.810791in}{2.479455in}}%
\pgfpathlineto{\pgfqpoint{1.822419in}{2.482756in}}%
\pgfpathlineto{\pgfqpoint{1.834041in}{2.486046in}}%
\pgfpathlineto{\pgfqpoint{1.845657in}{2.489325in}}%
\pgfpathlineto{\pgfqpoint{1.857268in}{2.492597in}}%
\pgfpathlineto{\pgfqpoint{1.868874in}{2.495861in}}%
\pgfpathlineto{\pgfqpoint{1.863073in}{2.504021in}}%
\pgfpathlineto{\pgfqpoint{1.857276in}{2.512164in}}%
\pgfpathlineto{\pgfqpoint{1.851484in}{2.520290in}}%
\pgfpathlineto{\pgfqpoint{1.845695in}{2.528400in}}%
\pgfpathlineto{\pgfqpoint{1.839911in}{2.536494in}}%
\pgfpathlineto{\pgfqpoint{1.828318in}{2.533244in}}%
\pgfpathlineto{\pgfqpoint{1.816720in}{2.529988in}}%
\pgfpathlineto{\pgfqpoint{1.805115in}{2.526724in}}%
\pgfpathlineto{\pgfqpoint{1.793506in}{2.523451in}}%
\pgfpathlineto{\pgfqpoint{1.781891in}{2.520167in}}%
\pgfpathlineto{\pgfqpoint{1.787662in}{2.512059in}}%
\pgfpathlineto{\pgfqpoint{1.793438in}{2.503935in}}%
\pgfpathlineto{\pgfqpoint{1.799218in}{2.495793in}}%
\pgfpathlineto{\pgfqpoint{1.805002in}{2.487633in}}%
\pgfpathclose%
\pgfusepath{stroke,fill}%
\end{pgfscope}%
\begin{pgfscope}%
\pgfpathrectangle{\pgfqpoint{0.887500in}{0.275000in}}{\pgfqpoint{4.225000in}{4.225000in}}%
\pgfusepath{clip}%
\pgfsetbuttcap%
\pgfsetroundjoin%
\definecolor{currentfill}{rgb}{0.281477,0.755203,0.432552}%
\pgfsetfillcolor{currentfill}%
\pgfsetfillopacity{0.700000}%
\pgfsetlinewidth{0.501875pt}%
\definecolor{currentstroke}{rgb}{1.000000,1.000000,1.000000}%
\pgfsetstrokecolor{currentstroke}%
\pgfsetstrokeopacity{0.500000}%
\pgfsetdash{}{0pt}%
\pgfpathmoveto{\pgfqpoint{3.799378in}{2.896950in}}%
\pgfpathlineto{\pgfqpoint{3.810517in}{2.900455in}}%
\pgfpathlineto{\pgfqpoint{3.821650in}{2.903956in}}%
\pgfpathlineto{\pgfqpoint{3.832778in}{2.907453in}}%
\pgfpathlineto{\pgfqpoint{3.843900in}{2.910942in}}%
\pgfpathlineto{\pgfqpoint{3.855017in}{2.914422in}}%
\pgfpathlineto{\pgfqpoint{3.848663in}{2.928135in}}%
\pgfpathlineto{\pgfqpoint{3.842311in}{2.941834in}}%
\pgfpathlineto{\pgfqpoint{3.835961in}{2.955500in}}%
\pgfpathlineto{\pgfqpoint{3.829613in}{2.969110in}}%
\pgfpathlineto{\pgfqpoint{3.823265in}{2.982641in}}%
\pgfpathlineto{\pgfqpoint{3.812150in}{2.979082in}}%
\pgfpathlineto{\pgfqpoint{3.801029in}{2.975503in}}%
\pgfpathlineto{\pgfqpoint{3.789902in}{2.971906in}}%
\pgfpathlineto{\pgfqpoint{3.778770in}{2.968294in}}%
\pgfpathlineto{\pgfqpoint{3.767632in}{2.964667in}}%
\pgfpathlineto{\pgfqpoint{3.773978in}{2.951224in}}%
\pgfpathlineto{\pgfqpoint{3.780325in}{2.937717in}}%
\pgfpathlineto{\pgfqpoint{3.786674in}{2.924159in}}%
\pgfpathlineto{\pgfqpoint{3.793025in}{2.910566in}}%
\pgfpathclose%
\pgfusepath{stroke,fill}%
\end{pgfscope}%
\begin{pgfscope}%
\pgfpathrectangle{\pgfqpoint{0.887500in}{0.275000in}}{\pgfqpoint{4.225000in}{4.225000in}}%
\pgfusepath{clip}%
\pgfsetbuttcap%
\pgfsetroundjoin%
\definecolor{currentfill}{rgb}{0.688944,0.865448,0.182725}%
\pgfsetfillcolor{currentfill}%
\pgfsetfillopacity{0.700000}%
\pgfsetlinewidth{0.501875pt}%
\definecolor{currentstroke}{rgb}{1.000000,1.000000,1.000000}%
\pgfsetstrokecolor{currentstroke}%
\pgfsetstrokeopacity{0.500000}%
\pgfsetdash{}{0pt}%
\pgfpathmoveto{\pgfqpoint{3.098308in}{3.232857in}}%
\pgfpathlineto{\pgfqpoint{3.109623in}{3.243129in}}%
\pgfpathlineto{\pgfqpoint{3.120934in}{3.251962in}}%
\pgfpathlineto{\pgfqpoint{3.132239in}{3.259423in}}%
\pgfpathlineto{\pgfqpoint{3.143539in}{3.265583in}}%
\pgfpathlineto{\pgfqpoint{3.154833in}{3.270577in}}%
\pgfpathlineto{\pgfqpoint{3.148606in}{3.281746in}}%
\pgfpathlineto{\pgfqpoint{3.142381in}{3.292758in}}%
\pgfpathlineto{\pgfqpoint{3.136160in}{3.303602in}}%
\pgfpathlineto{\pgfqpoint{3.129942in}{3.314268in}}%
\pgfpathlineto{\pgfqpoint{3.123727in}{3.324746in}}%
\pgfpathlineto{\pgfqpoint{3.112442in}{3.319739in}}%
\pgfpathlineto{\pgfqpoint{3.101152in}{3.313323in}}%
\pgfpathlineto{\pgfqpoint{3.089856in}{3.305280in}}%
\pgfpathlineto{\pgfqpoint{3.078557in}{3.295427in}}%
\pgfpathlineto{\pgfqpoint{3.067255in}{3.283583in}}%
\pgfpathlineto{\pgfqpoint{3.073459in}{3.273879in}}%
\pgfpathlineto{\pgfqpoint{3.079666in}{3.263952in}}%
\pgfpathlineto{\pgfqpoint{3.085877in}{3.253804in}}%
\pgfpathlineto{\pgfqpoint{3.092091in}{3.243438in}}%
\pgfpathclose%
\pgfusepath{stroke,fill}%
\end{pgfscope}%
\begin{pgfscope}%
\pgfpathrectangle{\pgfqpoint{0.887500in}{0.275000in}}{\pgfqpoint{4.225000in}{4.225000in}}%
\pgfusepath{clip}%
\pgfsetbuttcap%
\pgfsetroundjoin%
\definecolor{currentfill}{rgb}{0.139147,0.533812,0.555298}%
\pgfsetfillcolor{currentfill}%
\pgfsetfillopacity{0.700000}%
\pgfsetlinewidth{0.501875pt}%
\definecolor{currentstroke}{rgb}{1.000000,1.000000,1.000000}%
\pgfsetstrokecolor{currentstroke}%
\pgfsetstrokeopacity{0.500000}%
\pgfsetdash{}{0pt}%
\pgfpathmoveto{\pgfqpoint{2.130333in}{2.421170in}}%
\pgfpathlineto{\pgfqpoint{2.141881in}{2.424528in}}%
\pgfpathlineto{\pgfqpoint{2.153424in}{2.427876in}}%
\pgfpathlineto{\pgfqpoint{2.164962in}{2.431212in}}%
\pgfpathlineto{\pgfqpoint{2.176494in}{2.434538in}}%
\pgfpathlineto{\pgfqpoint{2.188021in}{2.437853in}}%
\pgfpathlineto{\pgfqpoint{2.182108in}{2.446221in}}%
\pgfpathlineto{\pgfqpoint{2.176199in}{2.454573in}}%
\pgfpathlineto{\pgfqpoint{2.170294in}{2.462909in}}%
\pgfpathlineto{\pgfqpoint{2.164394in}{2.471229in}}%
\pgfpathlineto{\pgfqpoint{2.158497in}{2.479533in}}%
\pgfpathlineto{\pgfqpoint{2.146983in}{2.476233in}}%
\pgfpathlineto{\pgfqpoint{2.135462in}{2.472925in}}%
\pgfpathlineto{\pgfqpoint{2.123937in}{2.469606in}}%
\pgfpathlineto{\pgfqpoint{2.112405in}{2.466277in}}%
\pgfpathlineto{\pgfqpoint{2.100869in}{2.462937in}}%
\pgfpathlineto{\pgfqpoint{2.106753in}{2.454619in}}%
\pgfpathlineto{\pgfqpoint{2.112642in}{2.446284in}}%
\pgfpathlineto{\pgfqpoint{2.118535in}{2.437930in}}%
\pgfpathlineto{\pgfqpoint{2.124432in}{2.429559in}}%
\pgfpathclose%
\pgfusepath{stroke,fill}%
\end{pgfscope}%
\begin{pgfscope}%
\pgfpathrectangle{\pgfqpoint{0.887500in}{0.275000in}}{\pgfqpoint{4.225000in}{4.225000in}}%
\pgfusepath{clip}%
\pgfsetbuttcap%
\pgfsetroundjoin%
\definecolor{currentfill}{rgb}{0.150476,0.504369,0.557430}%
\pgfsetfillcolor{currentfill}%
\pgfsetfillopacity{0.700000}%
\pgfsetlinewidth{0.501875pt}%
\definecolor{currentstroke}{rgb}{1.000000,1.000000,1.000000}%
\pgfsetstrokecolor{currentstroke}%
\pgfsetstrokeopacity{0.500000}%
\pgfsetdash{}{0pt}%
\pgfpathmoveto{\pgfqpoint{2.449959in}{2.360418in}}%
\pgfpathlineto{\pgfqpoint{2.461428in}{2.363895in}}%
\pgfpathlineto{\pgfqpoint{2.472890in}{2.367420in}}%
\pgfpathlineto{\pgfqpoint{2.484346in}{2.370983in}}%
\pgfpathlineto{\pgfqpoint{2.495797in}{2.374553in}}%
\pgfpathlineto{\pgfqpoint{2.507242in}{2.378094in}}%
\pgfpathlineto{\pgfqpoint{2.501221in}{2.386658in}}%
\pgfpathlineto{\pgfqpoint{2.495203in}{2.395204in}}%
\pgfpathlineto{\pgfqpoint{2.489190in}{2.403734in}}%
\pgfpathlineto{\pgfqpoint{2.483181in}{2.412249in}}%
\pgfpathlineto{\pgfqpoint{2.477176in}{2.420748in}}%
\pgfpathlineto{\pgfqpoint{2.465742in}{2.417225in}}%
\pgfpathlineto{\pgfqpoint{2.454303in}{2.413677in}}%
\pgfpathlineto{\pgfqpoint{2.442858in}{2.410135in}}%
\pgfpathlineto{\pgfqpoint{2.431407in}{2.406628in}}%
\pgfpathlineto{\pgfqpoint{2.419950in}{2.403165in}}%
\pgfpathlineto{\pgfqpoint{2.425944in}{2.394645in}}%
\pgfpathlineto{\pgfqpoint{2.431942in}{2.386110in}}%
\pgfpathlineto{\pgfqpoint{2.437943in}{2.377561in}}%
\pgfpathlineto{\pgfqpoint{2.443949in}{2.368997in}}%
\pgfpathclose%
\pgfusepath{stroke,fill}%
\end{pgfscope}%
\begin{pgfscope}%
\pgfpathrectangle{\pgfqpoint{0.887500in}{0.275000in}}{\pgfqpoint{4.225000in}{4.225000in}}%
\pgfusepath{clip}%
\pgfsetbuttcap%
\pgfsetroundjoin%
\definecolor{currentfill}{rgb}{0.143343,0.522773,0.556295}%
\pgfsetfillcolor{currentfill}%
\pgfsetfillopacity{0.700000}%
\pgfsetlinewidth{0.501875pt}%
\definecolor{currentstroke}{rgb}{1.000000,1.000000,1.000000}%
\pgfsetstrokecolor{currentstroke}%
\pgfsetstrokeopacity{0.500000}%
\pgfsetdash{}{0pt}%
\pgfpathmoveto{\pgfqpoint{4.474878in}{2.386199in}}%
\pgfpathlineto{\pgfqpoint{4.485854in}{2.389789in}}%
\pgfpathlineto{\pgfqpoint{4.496825in}{2.393393in}}%
\pgfpathlineto{\pgfqpoint{4.507791in}{2.397015in}}%
\pgfpathlineto{\pgfqpoint{4.518753in}{2.400657in}}%
\pgfpathlineto{\pgfqpoint{4.512275in}{2.414633in}}%
\pgfpathlineto{\pgfqpoint{4.505799in}{2.428625in}}%
\pgfpathlineto{\pgfqpoint{4.499328in}{2.442641in}}%
\pgfpathlineto{\pgfqpoint{4.492859in}{2.456682in}}%
\pgfpathlineto{\pgfqpoint{4.486395in}{2.470748in}}%
\pgfpathlineto{\pgfqpoint{4.475437in}{2.467163in}}%
\pgfpathlineto{\pgfqpoint{4.464474in}{2.463603in}}%
\pgfpathlineto{\pgfqpoint{4.453508in}{2.460067in}}%
\pgfpathlineto{\pgfqpoint{4.442537in}{2.456554in}}%
\pgfpathlineto{\pgfqpoint{4.448996in}{2.442392in}}%
\pgfpathlineto{\pgfqpoint{4.455460in}{2.428272in}}%
\pgfpathlineto{\pgfqpoint{4.461929in}{2.414200in}}%
\pgfpathlineto{\pgfqpoint{4.468401in}{2.400179in}}%
\pgfpathclose%
\pgfusepath{stroke,fill}%
\end{pgfscope}%
\begin{pgfscope}%
\pgfpathrectangle{\pgfqpoint{0.887500in}{0.275000in}}{\pgfqpoint{4.225000in}{4.225000in}}%
\pgfusepath{clip}%
\pgfsetbuttcap%
\pgfsetroundjoin%
\definecolor{currentfill}{rgb}{0.606045,0.850733,0.236712}%
\pgfsetfillcolor{currentfill}%
\pgfsetfillopacity{0.700000}%
\pgfsetlinewidth{0.501875pt}%
\definecolor{currentstroke}{rgb}{1.000000,1.000000,1.000000}%
\pgfsetstrokecolor{currentstroke}%
\pgfsetstrokeopacity{0.500000}%
\pgfsetdash{}{0pt}%
\pgfpathmoveto{\pgfqpoint{3.041721in}{3.157681in}}%
\pgfpathlineto{\pgfqpoint{3.053036in}{3.176058in}}%
\pgfpathlineto{\pgfqpoint{3.064354in}{3.192743in}}%
\pgfpathlineto{\pgfqpoint{3.075672in}{3.207730in}}%
\pgfpathlineto{\pgfqpoint{3.086991in}{3.221079in}}%
\pgfpathlineto{\pgfqpoint{3.098308in}{3.232857in}}%
\pgfpathlineto{\pgfqpoint{3.092091in}{3.243438in}}%
\pgfpathlineto{\pgfqpoint{3.085877in}{3.253804in}}%
\pgfpathlineto{\pgfqpoint{3.079666in}{3.263952in}}%
\pgfpathlineto{\pgfqpoint{3.073459in}{3.273879in}}%
\pgfpathlineto{\pgfqpoint{3.067255in}{3.283583in}}%
\pgfpathlineto{\pgfqpoint{3.055953in}{3.269568in}}%
\pgfpathlineto{\pgfqpoint{3.044652in}{3.253203in}}%
\pgfpathlineto{\pgfqpoint{3.033354in}{3.234310in}}%
\pgfpathlineto{\pgfqpoint{3.022062in}{3.212738in}}%
\pgfpathlineto{\pgfqpoint{3.010777in}{3.188668in}}%
\pgfpathlineto{\pgfqpoint{3.016956in}{3.182842in}}%
\pgfpathlineto{\pgfqpoint{3.023140in}{3.176820in}}%
\pgfpathlineto{\pgfqpoint{3.029329in}{3.170612in}}%
\pgfpathlineto{\pgfqpoint{3.035523in}{3.164230in}}%
\pgfpathclose%
\pgfusepath{stroke,fill}%
\end{pgfscope}%
\begin{pgfscope}%
\pgfpathrectangle{\pgfqpoint{0.887500in}{0.275000in}}{\pgfqpoint{4.225000in}{4.225000in}}%
\pgfusepath{clip}%
\pgfsetbuttcap%
\pgfsetroundjoin%
\definecolor{currentfill}{rgb}{0.335885,0.777018,0.402049}%
\pgfsetfillcolor{currentfill}%
\pgfsetfillopacity{0.700000}%
\pgfsetlinewidth{0.501875pt}%
\definecolor{currentstroke}{rgb}{1.000000,1.000000,1.000000}%
\pgfsetstrokecolor{currentstroke}%
\pgfsetstrokeopacity{0.500000}%
\pgfsetdash{}{0pt}%
\pgfpathmoveto{\pgfqpoint{3.711864in}{2.946433in}}%
\pgfpathlineto{\pgfqpoint{3.723028in}{2.950075in}}%
\pgfpathlineto{\pgfqpoint{3.734187in}{2.953728in}}%
\pgfpathlineto{\pgfqpoint{3.745341in}{2.957382in}}%
\pgfpathlineto{\pgfqpoint{3.756489in}{2.961029in}}%
\pgfpathlineto{\pgfqpoint{3.767632in}{2.964667in}}%
\pgfpathlineto{\pgfqpoint{3.761288in}{2.978032in}}%
\pgfpathlineto{\pgfqpoint{3.754944in}{2.991305in}}%
\pgfpathlineto{\pgfqpoint{3.748602in}{3.004471in}}%
\pgfpathlineto{\pgfqpoint{3.742260in}{3.017516in}}%
\pgfpathlineto{\pgfqpoint{3.735918in}{3.030429in}}%
\pgfpathlineto{\pgfqpoint{3.724779in}{3.026811in}}%
\pgfpathlineto{\pgfqpoint{3.713635in}{3.023196in}}%
\pgfpathlineto{\pgfqpoint{3.702486in}{3.019589in}}%
\pgfpathlineto{\pgfqpoint{3.691332in}{3.015996in}}%
\pgfpathlineto{\pgfqpoint{3.680173in}{3.012421in}}%
\pgfpathlineto{\pgfqpoint{3.686507in}{2.999314in}}%
\pgfpathlineto{\pgfqpoint{3.692843in}{2.986159in}}%
\pgfpathlineto{\pgfqpoint{3.699182in}{2.972959in}}%
\pgfpathlineto{\pgfqpoint{3.705522in}{2.959717in}}%
\pgfpathclose%
\pgfusepath{stroke,fill}%
\end{pgfscope}%
\begin{pgfscope}%
\pgfpathrectangle{\pgfqpoint{0.887500in}{0.275000in}}{\pgfqpoint{4.225000in}{4.225000in}}%
\pgfusepath{clip}%
\pgfsetbuttcap%
\pgfsetroundjoin%
\definecolor{currentfill}{rgb}{0.134692,0.658636,0.517649}%
\pgfsetfillcolor{currentfill}%
\pgfsetfillopacity{0.700000}%
\pgfsetlinewidth{0.501875pt}%
\definecolor{currentstroke}{rgb}{1.000000,1.000000,1.000000}%
\pgfsetstrokecolor{currentstroke}%
\pgfsetstrokeopacity{0.500000}%
\pgfsetdash{}{0pt}%
\pgfpathmoveto{\pgfqpoint{4.093580in}{2.672898in}}%
\pgfpathlineto{\pgfqpoint{4.104648in}{2.676317in}}%
\pgfpathlineto{\pgfqpoint{4.115710in}{2.679745in}}%
\pgfpathlineto{\pgfqpoint{4.126767in}{2.683180in}}%
\pgfpathlineto{\pgfqpoint{4.137819in}{2.686623in}}%
\pgfpathlineto{\pgfqpoint{4.148866in}{2.690072in}}%
\pgfpathlineto{\pgfqpoint{4.142461in}{2.704107in}}%
\pgfpathlineto{\pgfqpoint{4.136057in}{2.718125in}}%
\pgfpathlineto{\pgfqpoint{4.129656in}{2.732127in}}%
\pgfpathlineto{\pgfqpoint{4.123258in}{2.746113in}}%
\pgfpathlineto{\pgfqpoint{4.116861in}{2.760085in}}%
\pgfpathlineto{\pgfqpoint{4.105816in}{2.756650in}}%
\pgfpathlineto{\pgfqpoint{4.094767in}{2.753224in}}%
\pgfpathlineto{\pgfqpoint{4.083712in}{2.749807in}}%
\pgfpathlineto{\pgfqpoint{4.072653in}{2.746397in}}%
\pgfpathlineto{\pgfqpoint{4.061588in}{2.742995in}}%
\pgfpathlineto{\pgfqpoint{4.067983in}{2.729020in}}%
\pgfpathlineto{\pgfqpoint{4.074379in}{2.715020in}}%
\pgfpathlineto{\pgfqpoint{4.080777in}{2.700997in}}%
\pgfpathlineto{\pgfqpoint{4.087178in}{2.686955in}}%
\pgfpathclose%
\pgfusepath{stroke,fill}%
\end{pgfscope}%
\begin{pgfscope}%
\pgfpathrectangle{\pgfqpoint{0.887500in}{0.275000in}}{\pgfqpoint{4.225000in}{4.225000in}}%
\pgfusepath{clip}%
\pgfsetbuttcap%
\pgfsetroundjoin%
\definecolor{currentfill}{rgb}{0.133743,0.548535,0.553541}%
\pgfsetfillcolor{currentfill}%
\pgfsetfillopacity{0.700000}%
\pgfsetlinewidth{0.501875pt}%
\definecolor{currentstroke}{rgb}{1.000000,1.000000,1.000000}%
\pgfsetstrokecolor{currentstroke}%
\pgfsetstrokeopacity{0.500000}%
\pgfsetdash{}{0pt}%
\pgfpathmoveto{\pgfqpoint{4.387614in}{2.439329in}}%
\pgfpathlineto{\pgfqpoint{4.398608in}{2.442731in}}%
\pgfpathlineto{\pgfqpoint{4.409597in}{2.446153in}}%
\pgfpathlineto{\pgfqpoint{4.420581in}{2.449598in}}%
\pgfpathlineto{\pgfqpoint{4.431561in}{2.453065in}}%
\pgfpathlineto{\pgfqpoint{4.442537in}{2.456554in}}%
\pgfpathlineto{\pgfqpoint{4.436081in}{2.470754in}}%
\pgfpathlineto{\pgfqpoint{4.429628in}{2.484987in}}%
\pgfpathlineto{\pgfqpoint{4.423179in}{2.499247in}}%
\pgfpathlineto{\pgfqpoint{4.416734in}{2.513529in}}%
\pgfpathlineto{\pgfqpoint{4.410292in}{2.527829in}}%
\pgfpathlineto{\pgfqpoint{4.399321in}{2.524454in}}%
\pgfpathlineto{\pgfqpoint{4.388346in}{2.521089in}}%
\pgfpathlineto{\pgfqpoint{4.377366in}{2.517733in}}%
\pgfpathlineto{\pgfqpoint{4.366381in}{2.514386in}}%
\pgfpathlineto{\pgfqpoint{4.355391in}{2.511047in}}%
\pgfpathlineto{\pgfqpoint{4.361830in}{2.496681in}}%
\pgfpathlineto{\pgfqpoint{4.368271in}{2.482319in}}%
\pgfpathlineto{\pgfqpoint{4.374716in}{2.467968in}}%
\pgfpathlineto{\pgfqpoint{4.381163in}{2.453636in}}%
\pgfpathclose%
\pgfusepath{stroke,fill}%
\end{pgfscope}%
\begin{pgfscope}%
\pgfpathrectangle{\pgfqpoint{0.887500in}{0.275000in}}{\pgfqpoint{4.225000in}{4.225000in}}%
\pgfusepath{clip}%
\pgfsetbuttcap%
\pgfsetroundjoin%
\definecolor{currentfill}{rgb}{0.125394,0.574318,0.549086}%
\pgfsetfillcolor{currentfill}%
\pgfsetfillopacity{0.700000}%
\pgfsetlinewidth{0.501875pt}%
\definecolor{currentstroke}{rgb}{1.000000,1.000000,1.000000}%
\pgfsetstrokecolor{currentstroke}%
\pgfsetstrokeopacity{0.500000}%
\pgfsetdash{}{0pt}%
\pgfpathmoveto{\pgfqpoint{1.578380in}{2.510973in}}%
\pgfpathlineto{\pgfqpoint{1.590067in}{2.514298in}}%
\pgfpathlineto{\pgfqpoint{1.601749in}{2.517617in}}%
\pgfpathlineto{\pgfqpoint{1.613426in}{2.520929in}}%
\pgfpathlineto{\pgfqpoint{1.625096in}{2.524236in}}%
\pgfpathlineto{\pgfqpoint{1.636761in}{2.527538in}}%
\pgfpathlineto{\pgfqpoint{1.631040in}{2.535587in}}%
\pgfpathlineto{\pgfqpoint{1.625323in}{2.543617in}}%
\pgfpathlineto{\pgfqpoint{1.619611in}{2.551625in}}%
\pgfpathlineto{\pgfqpoint{1.613903in}{2.559612in}}%
\pgfpathlineto{\pgfqpoint{1.608200in}{2.567575in}}%
\pgfpathlineto{\pgfqpoint{1.596548in}{2.564286in}}%
\pgfpathlineto{\pgfqpoint{1.584890in}{2.560993in}}%
\pgfpathlineto{\pgfqpoint{1.573227in}{2.557693in}}%
\pgfpathlineto{\pgfqpoint{1.561558in}{2.554387in}}%
\pgfpathlineto{\pgfqpoint{1.549884in}{2.551074in}}%
\pgfpathlineto{\pgfqpoint{1.555574in}{2.543100in}}%
\pgfpathlineto{\pgfqpoint{1.561269in}{2.535102in}}%
\pgfpathlineto{\pgfqpoint{1.566968in}{2.527080in}}%
\pgfpathlineto{\pgfqpoint{1.572672in}{2.519036in}}%
\pgfpathclose%
\pgfusepath{stroke,fill}%
\end{pgfscope}%
\begin{pgfscope}%
\pgfpathrectangle{\pgfqpoint{0.887500in}{0.275000in}}{\pgfqpoint{4.225000in}{4.225000in}}%
\pgfusepath{clip}%
\pgfsetbuttcap%
\pgfsetroundjoin%
\definecolor{currentfill}{rgb}{0.386433,0.794644,0.372886}%
\pgfsetfillcolor{currentfill}%
\pgfsetfillopacity{0.700000}%
\pgfsetlinewidth{0.501875pt}%
\definecolor{currentstroke}{rgb}{1.000000,1.000000,1.000000}%
\pgfsetstrokecolor{currentstroke}%
\pgfsetstrokeopacity{0.500000}%
\pgfsetdash{}{0pt}%
\pgfpathmoveto{\pgfqpoint{3.624305in}{2.995009in}}%
\pgfpathlineto{\pgfqpoint{3.635488in}{2.998414in}}%
\pgfpathlineto{\pgfqpoint{3.646666in}{3.001861in}}%
\pgfpathlineto{\pgfqpoint{3.657840in}{3.005348in}}%
\pgfpathlineto{\pgfqpoint{3.669009in}{3.008870in}}%
\pgfpathlineto{\pgfqpoint{3.680173in}{3.012421in}}%
\pgfpathlineto{\pgfqpoint{3.673840in}{3.025474in}}%
\pgfpathlineto{\pgfqpoint{3.667510in}{3.038464in}}%
\pgfpathlineto{\pgfqpoint{3.661181in}{3.051384in}}%
\pgfpathlineto{\pgfqpoint{3.654854in}{3.064224in}}%
\pgfpathlineto{\pgfqpoint{3.648528in}{3.076975in}}%
\pgfpathlineto{\pgfqpoint{3.637373in}{3.073795in}}%
\pgfpathlineto{\pgfqpoint{3.626213in}{3.070611in}}%
\pgfpathlineto{\pgfqpoint{3.615048in}{3.067409in}}%
\pgfpathlineto{\pgfqpoint{3.603876in}{3.064173in}}%
\pgfpathlineto{\pgfqpoint{3.592698in}{3.060891in}}%
\pgfpathlineto{\pgfqpoint{3.599016in}{3.047855in}}%
\pgfpathlineto{\pgfqpoint{3.605336in}{3.034726in}}%
\pgfpathlineto{\pgfqpoint{3.611657in}{3.021527in}}%
\pgfpathlineto{\pgfqpoint{3.617980in}{3.008280in}}%
\pgfpathclose%
\pgfusepath{stroke,fill}%
\end{pgfscope}%
\begin{pgfscope}%
\pgfpathrectangle{\pgfqpoint{0.887500in}{0.275000in}}{\pgfqpoint{4.225000in}{4.225000in}}%
\pgfusepath{clip}%
\pgfsetbuttcap%
\pgfsetroundjoin%
\definecolor{currentfill}{rgb}{0.168126,0.459988,0.558082}%
\pgfsetfillcolor{currentfill}%
\pgfsetfillopacity{0.700000}%
\pgfsetlinewidth{0.501875pt}%
\definecolor{currentstroke}{rgb}{1.000000,1.000000,1.000000}%
\pgfsetstrokecolor{currentstroke}%
\pgfsetstrokeopacity{0.500000}%
\pgfsetdash{}{0pt}%
\pgfpathmoveto{\pgfqpoint{2.857255in}{2.259868in}}%
\pgfpathlineto{\pgfqpoint{2.868638in}{2.260825in}}%
\pgfpathlineto{\pgfqpoint{2.880003in}{2.264121in}}%
\pgfpathlineto{\pgfqpoint{2.891348in}{2.270812in}}%
\pgfpathlineto{\pgfqpoint{2.902672in}{2.281954in}}%
\pgfpathlineto{\pgfqpoint{2.913975in}{2.298432in}}%
\pgfpathlineto{\pgfqpoint{2.907816in}{2.307447in}}%
\pgfpathlineto{\pgfqpoint{2.901660in}{2.316330in}}%
\pgfpathlineto{\pgfqpoint{2.895509in}{2.325060in}}%
\pgfpathlineto{\pgfqpoint{2.889363in}{2.333620in}}%
\pgfpathlineto{\pgfqpoint{2.883222in}{2.341987in}}%
\pgfpathlineto{\pgfqpoint{2.871942in}{2.325054in}}%
\pgfpathlineto{\pgfqpoint{2.860636in}{2.313717in}}%
\pgfpathlineto{\pgfqpoint{2.849304in}{2.307045in}}%
\pgfpathlineto{\pgfqpoint{2.837949in}{2.303926in}}%
\pgfpathlineto{\pgfqpoint{2.826574in}{2.303252in}}%
\pgfpathlineto{\pgfqpoint{2.832700in}{2.295005in}}%
\pgfpathlineto{\pgfqpoint{2.838831in}{2.286511in}}%
\pgfpathlineto{\pgfqpoint{2.844968in}{2.277801in}}%
\pgfpathlineto{\pgfqpoint{2.851109in}{2.268909in}}%
\pgfpathclose%
\pgfusepath{stroke,fill}%
\end{pgfscope}%
\begin{pgfscope}%
\pgfpathrectangle{\pgfqpoint{0.887500in}{0.275000in}}{\pgfqpoint{4.225000in}{4.225000in}}%
\pgfusepath{clip}%
\pgfsetbuttcap%
\pgfsetroundjoin%
\definecolor{currentfill}{rgb}{0.647257,0.858400,0.209861}%
\pgfsetfillcolor{currentfill}%
\pgfsetfillopacity{0.700000}%
\pgfsetlinewidth{0.501875pt}%
\definecolor{currentstroke}{rgb}{1.000000,1.000000,1.000000}%
\pgfsetstrokecolor{currentstroke}%
\pgfsetstrokeopacity{0.500000}%
\pgfsetdash{}{0pt}%
\pgfpathmoveto{\pgfqpoint{3.186010in}{3.212971in}}%
\pgfpathlineto{\pgfqpoint{3.197305in}{3.217656in}}%
\pgfpathlineto{\pgfqpoint{3.208594in}{3.221776in}}%
\pgfpathlineto{\pgfqpoint{3.219875in}{3.225471in}}%
\pgfpathlineto{\pgfqpoint{3.231151in}{3.228879in}}%
\pgfpathlineto{\pgfqpoint{3.242420in}{3.232138in}}%
\pgfpathlineto{\pgfqpoint{3.236168in}{3.242914in}}%
\pgfpathlineto{\pgfqpoint{3.229919in}{3.253605in}}%
\pgfpathlineto{\pgfqpoint{3.223674in}{3.264228in}}%
\pgfpathlineto{\pgfqpoint{3.217431in}{3.274801in}}%
\pgfpathlineto{\pgfqpoint{3.211192in}{3.285340in}}%
\pgfpathlineto{\pgfqpoint{3.199934in}{3.283051in}}%
\pgfpathlineto{\pgfqpoint{3.188669in}{3.280640in}}%
\pgfpathlineto{\pgfqpoint{3.177398in}{3.277896in}}%
\pgfpathlineto{\pgfqpoint{3.166119in}{3.274612in}}%
\pgfpathlineto{\pgfqpoint{3.154833in}{3.270577in}}%
\pgfpathlineto{\pgfqpoint{3.161062in}{3.259267in}}%
\pgfpathlineto{\pgfqpoint{3.167295in}{3.247833in}}%
\pgfpathlineto{\pgfqpoint{3.173530in}{3.236294in}}%
\pgfpathlineto{\pgfqpoint{3.179769in}{3.224667in}}%
\pgfpathclose%
\pgfusepath{stroke,fill}%
\end{pgfscope}%
\begin{pgfscope}%
\pgfpathrectangle{\pgfqpoint{0.887500in}{0.275000in}}{\pgfqpoint{4.225000in}{4.225000in}}%
\pgfusepath{clip}%
\pgfsetbuttcap%
\pgfsetroundjoin%
\definecolor{currentfill}{rgb}{0.133743,0.548535,0.553541}%
\pgfsetfillcolor{currentfill}%
\pgfsetfillopacity{0.700000}%
\pgfsetlinewidth{0.501875pt}%
\definecolor{currentstroke}{rgb}{1.000000,1.000000,1.000000}%
\pgfsetstrokecolor{currentstroke}%
\pgfsetstrokeopacity{0.500000}%
\pgfsetdash{}{0pt}%
\pgfpathmoveto{\pgfqpoint{1.897944in}{2.454774in}}%
\pgfpathlineto{\pgfqpoint{1.909556in}{2.458054in}}%
\pgfpathlineto{\pgfqpoint{1.921162in}{2.461329in}}%
\pgfpathlineto{\pgfqpoint{1.932763in}{2.464600in}}%
\pgfpathlineto{\pgfqpoint{1.944357in}{2.467869in}}%
\pgfpathlineto{\pgfqpoint{1.955947in}{2.471139in}}%
\pgfpathlineto{\pgfqpoint{1.950112in}{2.479375in}}%
\pgfpathlineto{\pgfqpoint{1.944282in}{2.487593in}}%
\pgfpathlineto{\pgfqpoint{1.938455in}{2.495792in}}%
\pgfpathlineto{\pgfqpoint{1.932634in}{2.503972in}}%
\pgfpathlineto{\pgfqpoint{1.926816in}{2.512136in}}%
\pgfpathlineto{\pgfqpoint{1.915239in}{2.508881in}}%
\pgfpathlineto{\pgfqpoint{1.903656in}{2.505628in}}%
\pgfpathlineto{\pgfqpoint{1.892068in}{2.502375in}}%
\pgfpathlineto{\pgfqpoint{1.880474in}{2.499120in}}%
\pgfpathlineto{\pgfqpoint{1.868874in}{2.495861in}}%
\pgfpathlineto{\pgfqpoint{1.874679in}{2.487683in}}%
\pgfpathlineto{\pgfqpoint{1.880489in}{2.479485in}}%
\pgfpathlineto{\pgfqpoint{1.886303in}{2.471268in}}%
\pgfpathlineto{\pgfqpoint{1.892121in}{2.463031in}}%
\pgfpathclose%
\pgfusepath{stroke,fill}%
\end{pgfscope}%
\begin{pgfscope}%
\pgfpathrectangle{\pgfqpoint{0.887500in}{0.275000in}}{\pgfqpoint{4.225000in}{4.225000in}}%
\pgfusepath{clip}%
\pgfsetbuttcap%
\pgfsetroundjoin%
\definecolor{currentfill}{rgb}{0.440137,0.811138,0.340967}%
\pgfsetfillcolor{currentfill}%
\pgfsetfillopacity{0.700000}%
\pgfsetlinewidth{0.501875pt}%
\definecolor{currentstroke}{rgb}{1.000000,1.000000,1.000000}%
\pgfsetstrokecolor{currentstroke}%
\pgfsetstrokeopacity{0.500000}%
\pgfsetdash{}{0pt}%
\pgfpathmoveto{\pgfqpoint{3.536720in}{3.043809in}}%
\pgfpathlineto{\pgfqpoint{3.547928in}{3.047308in}}%
\pgfpathlineto{\pgfqpoint{3.559129in}{3.050768in}}%
\pgfpathlineto{\pgfqpoint{3.570325in}{3.054186in}}%
\pgfpathlineto{\pgfqpoint{3.581515in}{3.057561in}}%
\pgfpathlineto{\pgfqpoint{3.592698in}{3.060891in}}%
\pgfpathlineto{\pgfqpoint{3.586382in}{3.073809in}}%
\pgfpathlineto{\pgfqpoint{3.580066in}{3.086588in}}%
\pgfpathlineto{\pgfqpoint{3.573751in}{3.099203in}}%
\pgfpathlineto{\pgfqpoint{3.567436in}{3.111632in}}%
\pgfpathlineto{\pgfqpoint{3.561122in}{3.123852in}}%
\pgfpathlineto{\pgfqpoint{3.549944in}{3.120648in}}%
\pgfpathlineto{\pgfqpoint{3.538761in}{3.117389in}}%
\pgfpathlineto{\pgfqpoint{3.527571in}{3.114077in}}%
\pgfpathlineto{\pgfqpoint{3.516375in}{3.110719in}}%
\pgfpathlineto{\pgfqpoint{3.505173in}{3.107318in}}%
\pgfpathlineto{\pgfqpoint{3.511480in}{3.094856in}}%
\pgfpathlineto{\pgfqpoint{3.517788in}{3.082262in}}%
\pgfpathlineto{\pgfqpoint{3.524097in}{3.069548in}}%
\pgfpathlineto{\pgfqpoint{3.530408in}{3.056726in}}%
\pgfpathclose%
\pgfusepath{stroke,fill}%
\end{pgfscope}%
\begin{pgfscope}%
\pgfpathrectangle{\pgfqpoint{0.887500in}{0.275000in}}{\pgfqpoint{4.225000in}{4.225000in}}%
\pgfusepath{clip}%
\pgfsetbuttcap%
\pgfsetroundjoin%
\definecolor{currentfill}{rgb}{0.143343,0.522773,0.556295}%
\pgfsetfillcolor{currentfill}%
\pgfsetfillopacity{0.700000}%
\pgfsetlinewidth{0.501875pt}%
\definecolor{currentstroke}{rgb}{1.000000,1.000000,1.000000}%
\pgfsetstrokecolor{currentstroke}%
\pgfsetstrokeopacity{0.500000}%
\pgfsetdash{}{0pt}%
\pgfpathmoveto{\pgfqpoint{2.217648in}{2.395766in}}%
\pgfpathlineto{\pgfqpoint{2.229181in}{2.399093in}}%
\pgfpathlineto{\pgfqpoint{2.240708in}{2.402408in}}%
\pgfpathlineto{\pgfqpoint{2.252230in}{2.405711in}}%
\pgfpathlineto{\pgfqpoint{2.263746in}{2.409001in}}%
\pgfpathlineto{\pgfqpoint{2.275257in}{2.412283in}}%
\pgfpathlineto{\pgfqpoint{2.269311in}{2.420716in}}%
\pgfpathlineto{\pgfqpoint{2.263370in}{2.429134in}}%
\pgfpathlineto{\pgfqpoint{2.257433in}{2.437536in}}%
\pgfpathlineto{\pgfqpoint{2.251500in}{2.445923in}}%
\pgfpathlineto{\pgfqpoint{2.245571in}{2.454295in}}%
\pgfpathlineto{\pgfqpoint{2.234072in}{2.451020in}}%
\pgfpathlineto{\pgfqpoint{2.222568in}{2.447741in}}%
\pgfpathlineto{\pgfqpoint{2.211057in}{2.444454in}}%
\pgfpathlineto{\pgfqpoint{2.199542in}{2.441158in}}%
\pgfpathlineto{\pgfqpoint{2.188021in}{2.437853in}}%
\pgfpathlineto{\pgfqpoint{2.193938in}{2.429469in}}%
\pgfpathlineto{\pgfqpoint{2.199859in}{2.421068in}}%
\pgfpathlineto{\pgfqpoint{2.205784in}{2.412651in}}%
\pgfpathlineto{\pgfqpoint{2.211714in}{2.404217in}}%
\pgfpathclose%
\pgfusepath{stroke,fill}%
\end{pgfscope}%
\begin{pgfscope}%
\pgfpathrectangle{\pgfqpoint{0.887500in}{0.275000in}}{\pgfqpoint{4.225000in}{4.225000in}}%
\pgfusepath{clip}%
\pgfsetbuttcap%
\pgfsetroundjoin%
\definecolor{currentfill}{rgb}{0.606045,0.850733,0.236712}%
\pgfsetfillcolor{currentfill}%
\pgfsetfillopacity{0.700000}%
\pgfsetlinewidth{0.501875pt}%
\definecolor{currentstroke}{rgb}{1.000000,1.000000,1.000000}%
\pgfsetstrokecolor{currentstroke}%
\pgfsetstrokeopacity{0.500000}%
\pgfsetdash{}{0pt}%
\pgfpathmoveto{\pgfqpoint{3.273719in}{3.176392in}}%
\pgfpathlineto{\pgfqpoint{3.284993in}{3.180272in}}%
\pgfpathlineto{\pgfqpoint{3.296261in}{3.184100in}}%
\pgfpathlineto{\pgfqpoint{3.307524in}{3.187863in}}%
\pgfpathlineto{\pgfqpoint{3.318782in}{3.191544in}}%
\pgfpathlineto{\pgfqpoint{3.330033in}{3.195129in}}%
\pgfpathlineto{\pgfqpoint{3.323760in}{3.206294in}}%
\pgfpathlineto{\pgfqpoint{3.317488in}{3.217177in}}%
\pgfpathlineto{\pgfqpoint{3.311218in}{3.227815in}}%
\pgfpathlineto{\pgfqpoint{3.304950in}{3.238246in}}%
\pgfpathlineto{\pgfqpoint{3.298685in}{3.248507in}}%
\pgfpathlineto{\pgfqpoint{3.287443in}{3.245232in}}%
\pgfpathlineto{\pgfqpoint{3.276195in}{3.241942in}}%
\pgfpathlineto{\pgfqpoint{3.264942in}{3.238653in}}%
\pgfpathlineto{\pgfqpoint{3.253684in}{3.235380in}}%
\pgfpathlineto{\pgfqpoint{3.242420in}{3.232138in}}%
\pgfpathlineto{\pgfqpoint{3.248675in}{3.221260in}}%
\pgfpathlineto{\pgfqpoint{3.254932in}{3.210264in}}%
\pgfpathlineto{\pgfqpoint{3.261192in}{3.199131in}}%
\pgfpathlineto{\pgfqpoint{3.267454in}{3.187847in}}%
\pgfpathclose%
\pgfusepath{stroke,fill}%
\end{pgfscope}%
\begin{pgfscope}%
\pgfpathrectangle{\pgfqpoint{0.887500in}{0.275000in}}{\pgfqpoint{4.225000in}{4.225000in}}%
\pgfusepath{clip}%
\pgfsetbuttcap%
\pgfsetroundjoin%
\definecolor{currentfill}{rgb}{0.496615,0.826376,0.306377}%
\pgfsetfillcolor{currentfill}%
\pgfsetfillopacity{0.700000}%
\pgfsetlinewidth{0.501875pt}%
\definecolor{currentstroke}{rgb}{1.000000,1.000000,1.000000}%
\pgfsetstrokecolor{currentstroke}%
\pgfsetstrokeopacity{0.500000}%
\pgfsetdash{}{0pt}%
\pgfpathmoveto{\pgfqpoint{3.449080in}{3.089816in}}%
\pgfpathlineto{\pgfqpoint{3.460310in}{3.093375in}}%
\pgfpathlineto{\pgfqpoint{3.471534in}{3.096906in}}%
\pgfpathlineto{\pgfqpoint{3.482753in}{3.100409in}}%
\pgfpathlineto{\pgfqpoint{3.493966in}{3.103880in}}%
\pgfpathlineto{\pgfqpoint{3.505173in}{3.107318in}}%
\pgfpathlineto{\pgfqpoint{3.498868in}{3.119640in}}%
\pgfpathlineto{\pgfqpoint{3.492565in}{3.131822in}}%
\pgfpathlineto{\pgfqpoint{3.486262in}{3.143867in}}%
\pgfpathlineto{\pgfqpoint{3.479962in}{3.155776in}}%
\pgfpathlineto{\pgfqpoint{3.473663in}{3.167550in}}%
\pgfpathlineto{\pgfqpoint{3.462466in}{3.164635in}}%
\pgfpathlineto{\pgfqpoint{3.451263in}{3.161673in}}%
\pgfpathlineto{\pgfqpoint{3.440054in}{3.158654in}}%
\pgfpathlineto{\pgfqpoint{3.428839in}{3.155572in}}%
\pgfpathlineto{\pgfqpoint{3.417618in}{3.152418in}}%
\pgfpathlineto{\pgfqpoint{3.423906in}{3.140113in}}%
\pgfpathlineto{\pgfqpoint{3.430197in}{3.127678in}}%
\pgfpathlineto{\pgfqpoint{3.436489in}{3.115134in}}%
\pgfpathlineto{\pgfqpoint{3.442783in}{3.102506in}}%
\pgfpathclose%
\pgfusepath{stroke,fill}%
\end{pgfscope}%
\begin{pgfscope}%
\pgfpathrectangle{\pgfqpoint{0.887500in}{0.275000in}}{\pgfqpoint{4.225000in}{4.225000in}}%
\pgfusepath{clip}%
\pgfsetbuttcap%
\pgfsetroundjoin%
\definecolor{currentfill}{rgb}{0.154815,0.493313,0.557840}%
\pgfsetfillcolor{currentfill}%
\pgfsetfillopacity{0.700000}%
\pgfsetlinewidth{0.501875pt}%
\definecolor{currentstroke}{rgb}{1.000000,1.000000,1.000000}%
\pgfsetstrokecolor{currentstroke}%
\pgfsetstrokeopacity{0.500000}%
\pgfsetdash{}{0pt}%
\pgfpathmoveto{\pgfqpoint{2.537411in}{2.335001in}}%
\pgfpathlineto{\pgfqpoint{2.548863in}{2.338485in}}%
\pgfpathlineto{\pgfqpoint{2.560311in}{2.341869in}}%
\pgfpathlineto{\pgfqpoint{2.571755in}{2.345117in}}%
\pgfpathlineto{\pgfqpoint{2.583195in}{2.348196in}}%
\pgfpathlineto{\pgfqpoint{2.594631in}{2.351131in}}%
\pgfpathlineto{\pgfqpoint{2.588578in}{2.359776in}}%
\pgfpathlineto{\pgfqpoint{2.582529in}{2.368404in}}%
\pgfpathlineto{\pgfqpoint{2.576484in}{2.377016in}}%
\pgfpathlineto{\pgfqpoint{2.570443in}{2.385613in}}%
\pgfpathlineto{\pgfqpoint{2.564406in}{2.394195in}}%
\pgfpathlineto{\pgfqpoint{2.552981in}{2.391266in}}%
\pgfpathlineto{\pgfqpoint{2.541552in}{2.388192in}}%
\pgfpathlineto{\pgfqpoint{2.530119in}{2.384949in}}%
\pgfpathlineto{\pgfqpoint{2.518683in}{2.381572in}}%
\pgfpathlineto{\pgfqpoint{2.507242in}{2.378094in}}%
\pgfpathlineto{\pgfqpoint{2.513268in}{2.369513in}}%
\pgfpathlineto{\pgfqpoint{2.519297in}{2.360914in}}%
\pgfpathlineto{\pgfqpoint{2.525331in}{2.352296in}}%
\pgfpathlineto{\pgfqpoint{2.531369in}{2.343658in}}%
\pgfpathclose%
\pgfusepath{stroke,fill}%
\end{pgfscope}%
\begin{pgfscope}%
\pgfpathrectangle{\pgfqpoint{0.887500in}{0.275000in}}{\pgfqpoint{4.225000in}{4.225000in}}%
\pgfusepath{clip}%
\pgfsetbuttcap%
\pgfsetroundjoin%
\definecolor{currentfill}{rgb}{0.157851,0.683765,0.501686}%
\pgfsetfillcolor{currentfill}%
\pgfsetfillopacity{0.700000}%
\pgfsetlinewidth{0.501875pt}%
\definecolor{currentstroke}{rgb}{1.000000,1.000000,1.000000}%
\pgfsetstrokecolor{currentstroke}%
\pgfsetstrokeopacity{0.500000}%
\pgfsetdash{}{0pt}%
\pgfpathmoveto{\pgfqpoint{4.006185in}{2.726031in}}%
\pgfpathlineto{\pgfqpoint{4.017277in}{2.729422in}}%
\pgfpathlineto{\pgfqpoint{4.028362in}{2.732812in}}%
\pgfpathlineto{\pgfqpoint{4.039443in}{2.736203in}}%
\pgfpathlineto{\pgfqpoint{4.050518in}{2.739597in}}%
\pgfpathlineto{\pgfqpoint{4.061588in}{2.742995in}}%
\pgfpathlineto{\pgfqpoint{4.055195in}{2.756939in}}%
\pgfpathlineto{\pgfqpoint{4.048804in}{2.770850in}}%
\pgfpathlineto{\pgfqpoint{4.042415in}{2.784725in}}%
\pgfpathlineto{\pgfqpoint{4.036027in}{2.798559in}}%
\pgfpathlineto{\pgfqpoint{4.029641in}{2.812357in}}%
\pgfpathlineto{\pgfqpoint{4.018572in}{2.808911in}}%
\pgfpathlineto{\pgfqpoint{4.007498in}{2.805473in}}%
\pgfpathlineto{\pgfqpoint{3.996419in}{2.802040in}}%
\pgfpathlineto{\pgfqpoint{3.985335in}{2.798612in}}%
\pgfpathlineto{\pgfqpoint{3.974245in}{2.795187in}}%
\pgfpathlineto{\pgfqpoint{3.980631in}{2.781458in}}%
\pgfpathlineto{\pgfqpoint{3.987017in}{2.767679in}}%
\pgfpathlineto{\pgfqpoint{3.993405in}{2.753844in}}%
\pgfpathlineto{\pgfqpoint{3.999795in}{2.739960in}}%
\pgfpathclose%
\pgfusepath{stroke,fill}%
\end{pgfscope}%
\begin{pgfscope}%
\pgfpathrectangle{\pgfqpoint{0.887500in}{0.275000in}}{\pgfqpoint{4.225000in}{4.225000in}}%
\pgfusepath{clip}%
\pgfsetbuttcap%
\pgfsetroundjoin%
\definecolor{currentfill}{rgb}{0.555484,0.840254,0.269281}%
\pgfsetfillcolor{currentfill}%
\pgfsetfillopacity{0.700000}%
\pgfsetlinewidth{0.501875pt}%
\definecolor{currentstroke}{rgb}{1.000000,1.000000,1.000000}%
\pgfsetstrokecolor{currentstroke}%
\pgfsetstrokeopacity{0.500000}%
\pgfsetdash{}{0pt}%
\pgfpathmoveto{\pgfqpoint{3.361419in}{3.135349in}}%
\pgfpathlineto{\pgfqpoint{3.372671in}{3.138942in}}%
\pgfpathlineto{\pgfqpoint{3.383917in}{3.142451in}}%
\pgfpathlineto{\pgfqpoint{3.395157in}{3.145865in}}%
\pgfpathlineto{\pgfqpoint{3.406390in}{3.149185in}}%
\pgfpathlineto{\pgfqpoint{3.417618in}{3.152418in}}%
\pgfpathlineto{\pgfqpoint{3.411331in}{3.164568in}}%
\pgfpathlineto{\pgfqpoint{3.405045in}{3.176539in}}%
\pgfpathlineto{\pgfqpoint{3.398761in}{3.188310in}}%
\pgfpathlineto{\pgfqpoint{3.392478in}{3.199856in}}%
\pgfpathlineto{\pgfqpoint{3.386196in}{3.211154in}}%
\pgfpathlineto{\pgfqpoint{3.374976in}{3.208213in}}%
\pgfpathlineto{\pgfqpoint{3.363750in}{3.205148in}}%
\pgfpathlineto{\pgfqpoint{3.352518in}{3.201945in}}%
\pgfpathlineto{\pgfqpoint{3.341279in}{3.198601in}}%
\pgfpathlineto{\pgfqpoint{3.330033in}{3.195129in}}%
\pgfpathlineto{\pgfqpoint{3.336308in}{3.183664in}}%
\pgfpathlineto{\pgfqpoint{3.342583in}{3.171923in}}%
\pgfpathlineto{\pgfqpoint{3.348860in}{3.159937in}}%
\pgfpathlineto{\pgfqpoint{3.355139in}{3.147736in}}%
\pgfpathclose%
\pgfusepath{stroke,fill}%
\end{pgfscope}%
\begin{pgfscope}%
\pgfpathrectangle{\pgfqpoint{0.887500in}{0.275000in}}{\pgfqpoint{4.225000in}{4.225000in}}%
\pgfusepath{clip}%
\pgfsetbuttcap%
\pgfsetroundjoin%
\definecolor{currentfill}{rgb}{0.124395,0.578002,0.548287}%
\pgfsetfillcolor{currentfill}%
\pgfsetfillopacity{0.700000}%
\pgfsetlinewidth{0.501875pt}%
\definecolor{currentstroke}{rgb}{1.000000,1.000000,1.000000}%
\pgfsetstrokecolor{currentstroke}%
\pgfsetstrokeopacity{0.500000}%
\pgfsetdash{}{0pt}%
\pgfpathmoveto{\pgfqpoint{4.300363in}{2.494448in}}%
\pgfpathlineto{\pgfqpoint{4.311379in}{2.497758in}}%
\pgfpathlineto{\pgfqpoint{4.322390in}{2.501072in}}%
\pgfpathlineto{\pgfqpoint{4.333395in}{2.504391in}}%
\pgfpathlineto{\pgfqpoint{4.344396in}{2.507716in}}%
\pgfpathlineto{\pgfqpoint{4.355391in}{2.511047in}}%
\pgfpathlineto{\pgfqpoint{4.348955in}{2.525413in}}%
\pgfpathlineto{\pgfqpoint{4.342521in}{2.539771in}}%
\pgfpathlineto{\pgfqpoint{4.336088in}{2.554114in}}%
\pgfpathlineto{\pgfqpoint{4.329658in}{2.568439in}}%
\pgfpathlineto{\pgfqpoint{4.323230in}{2.582746in}}%
\pgfpathlineto{\pgfqpoint{4.312235in}{2.579396in}}%
\pgfpathlineto{\pgfqpoint{4.301235in}{2.576038in}}%
\pgfpathlineto{\pgfqpoint{4.290230in}{2.572670in}}%
\pgfpathlineto{\pgfqpoint{4.279218in}{2.569293in}}%
\pgfpathlineto{\pgfqpoint{4.268201in}{2.565903in}}%
\pgfpathlineto{\pgfqpoint{4.274630in}{2.551645in}}%
\pgfpathlineto{\pgfqpoint{4.281060in}{2.537370in}}%
\pgfpathlineto{\pgfqpoint{4.287492in}{2.523078in}}%
\pgfpathlineto{\pgfqpoint{4.293927in}{2.508770in}}%
\pgfpathclose%
\pgfusepath{stroke,fill}%
\end{pgfscope}%
\begin{pgfscope}%
\pgfpathrectangle{\pgfqpoint{0.887500in}{0.275000in}}{\pgfqpoint{4.225000in}{4.225000in}}%
\pgfusepath{clip}%
\pgfsetbuttcap%
\pgfsetroundjoin%
\definecolor{currentfill}{rgb}{0.121380,0.629492,0.531973}%
\pgfsetfillcolor{currentfill}%
\pgfsetfillopacity{0.700000}%
\pgfsetlinewidth{0.501875pt}%
\definecolor{currentstroke}{rgb}{1.000000,1.000000,1.000000}%
\pgfsetstrokecolor{currentstroke}%
\pgfsetstrokeopacity{0.500000}%
\pgfsetdash{}{0pt}%
\pgfpathmoveto{\pgfqpoint{2.908624in}{2.542637in}}%
\pgfpathlineto{\pgfqpoint{2.919854in}{2.577803in}}%
\pgfpathlineto{\pgfqpoint{2.931103in}{2.611067in}}%
\pgfpathlineto{\pgfqpoint{2.942369in}{2.642471in}}%
\pgfpathlineto{\pgfqpoint{2.953649in}{2.672426in}}%
\pgfpathlineto{\pgfqpoint{2.964940in}{2.701346in}}%
\pgfpathlineto{\pgfqpoint{2.958742in}{2.715728in}}%
\pgfpathlineto{\pgfqpoint{2.952549in}{2.729135in}}%
\pgfpathlineto{\pgfqpoint{2.946361in}{2.741228in}}%
\pgfpathlineto{\pgfqpoint{2.940181in}{2.751668in}}%
\pgfpathlineto{\pgfqpoint{2.934011in}{2.760117in}}%
\pgfpathlineto{\pgfqpoint{2.922746in}{2.731086in}}%
\pgfpathlineto{\pgfqpoint{2.911494in}{2.700744in}}%
\pgfpathlineto{\pgfqpoint{2.900259in}{2.668694in}}%
\pgfpathlineto{\pgfqpoint{2.889044in}{2.634537in}}%
\pgfpathlineto{\pgfqpoint{2.877851in}{2.598257in}}%
\pgfpathlineto{\pgfqpoint{2.883992in}{2.588639in}}%
\pgfpathlineto{\pgfqpoint{2.890142in}{2.578088in}}%
\pgfpathlineto{\pgfqpoint{2.896297in}{2.566784in}}%
\pgfpathlineto{\pgfqpoint{2.902459in}{2.554906in}}%
\pgfpathclose%
\pgfusepath{stroke,fill}%
\end{pgfscope}%
\begin{pgfscope}%
\pgfpathrectangle{\pgfqpoint{0.887500in}{0.275000in}}{\pgfqpoint{4.225000in}{4.225000in}}%
\pgfusepath{clip}%
\pgfsetbuttcap%
\pgfsetroundjoin%
\definecolor{currentfill}{rgb}{0.191090,0.708366,0.482284}%
\pgfsetfillcolor{currentfill}%
\pgfsetfillopacity{0.700000}%
\pgfsetlinewidth{0.501875pt}%
\definecolor{currentstroke}{rgb}{1.000000,1.000000,1.000000}%
\pgfsetstrokecolor{currentstroke}%
\pgfsetstrokeopacity{0.500000}%
\pgfsetdash{}{0pt}%
\pgfpathmoveto{\pgfqpoint{3.918717in}{2.778001in}}%
\pgfpathlineto{\pgfqpoint{3.929833in}{2.781456in}}%
\pgfpathlineto{\pgfqpoint{3.940945in}{2.784898in}}%
\pgfpathlineto{\pgfqpoint{3.952050in}{2.788333in}}%
\pgfpathlineto{\pgfqpoint{3.963151in}{2.791761in}}%
\pgfpathlineto{\pgfqpoint{3.974245in}{2.795187in}}%
\pgfpathlineto{\pgfqpoint{3.967862in}{2.808874in}}%
\pgfpathlineto{\pgfqpoint{3.961480in}{2.822530in}}%
\pgfpathlineto{\pgfqpoint{3.955101in}{2.836163in}}%
\pgfpathlineto{\pgfqpoint{3.948724in}{2.849785in}}%
\pgfpathlineto{\pgfqpoint{3.942349in}{2.863405in}}%
\pgfpathlineto{\pgfqpoint{3.931256in}{2.859944in}}%
\pgfpathlineto{\pgfqpoint{3.920158in}{2.856495in}}%
\pgfpathlineto{\pgfqpoint{3.909055in}{2.853052in}}%
\pgfpathlineto{\pgfqpoint{3.897947in}{2.849610in}}%
\pgfpathlineto{\pgfqpoint{3.886833in}{2.846165in}}%
\pgfpathlineto{\pgfqpoint{3.893205in}{2.832557in}}%
\pgfpathlineto{\pgfqpoint{3.899580in}{2.818946in}}%
\pgfpathlineto{\pgfqpoint{3.905957in}{2.805321in}}%
\pgfpathlineto{\pgfqpoint{3.912336in}{2.791676in}}%
\pgfpathclose%
\pgfusepath{stroke,fill}%
\end{pgfscope}%
\begin{pgfscope}%
\pgfpathrectangle{\pgfqpoint{0.887500in}{0.275000in}}{\pgfqpoint{4.225000in}{4.225000in}}%
\pgfusepath{clip}%
\pgfsetbuttcap%
\pgfsetroundjoin%
\definecolor{currentfill}{rgb}{0.127568,0.566949,0.550556}%
\pgfsetfillcolor{currentfill}%
\pgfsetfillopacity{0.700000}%
\pgfsetlinewidth{0.501875pt}%
\definecolor{currentstroke}{rgb}{1.000000,1.000000,1.000000}%
\pgfsetstrokecolor{currentstroke}%
\pgfsetstrokeopacity{0.500000}%
\pgfsetdash{}{0pt}%
\pgfpathmoveto{\pgfqpoint{1.665430in}{2.487051in}}%
\pgfpathlineto{\pgfqpoint{1.677102in}{2.490367in}}%
\pgfpathlineto{\pgfqpoint{1.688769in}{2.493683in}}%
\pgfpathlineto{\pgfqpoint{1.700429in}{2.496998in}}%
\pgfpathlineto{\pgfqpoint{1.712083in}{2.500314in}}%
\pgfpathlineto{\pgfqpoint{1.723732in}{2.503631in}}%
\pgfpathlineto{\pgfqpoint{1.717977in}{2.511741in}}%
\pgfpathlineto{\pgfqpoint{1.712227in}{2.519835in}}%
\pgfpathlineto{\pgfqpoint{1.706480in}{2.527915in}}%
\pgfpathlineto{\pgfqpoint{1.700738in}{2.535980in}}%
\pgfpathlineto{\pgfqpoint{1.695000in}{2.544029in}}%
\pgfpathlineto{\pgfqpoint{1.683364in}{2.540731in}}%
\pgfpathlineto{\pgfqpoint{1.671722in}{2.537433in}}%
\pgfpathlineto{\pgfqpoint{1.660074in}{2.534136in}}%
\pgfpathlineto{\pgfqpoint{1.648420in}{2.530838in}}%
\pgfpathlineto{\pgfqpoint{1.636761in}{2.527538in}}%
\pgfpathlineto{\pgfqpoint{1.642487in}{2.519472in}}%
\pgfpathlineto{\pgfqpoint{1.648216in}{2.511389in}}%
\pgfpathlineto{\pgfqpoint{1.653950in}{2.503291in}}%
\pgfpathlineto{\pgfqpoint{1.659688in}{2.495178in}}%
\pgfpathclose%
\pgfusepath{stroke,fill}%
\end{pgfscope}%
\begin{pgfscope}%
\pgfpathrectangle{\pgfqpoint{0.887500in}{0.275000in}}{\pgfqpoint{4.225000in}{4.225000in}}%
\pgfusepath{clip}%
\pgfsetbuttcap%
\pgfsetroundjoin%
\definecolor{currentfill}{rgb}{0.136408,0.541173,0.554483}%
\pgfsetfillcolor{currentfill}%
\pgfsetfillopacity{0.700000}%
\pgfsetlinewidth{0.501875pt}%
\definecolor{currentstroke}{rgb}{1.000000,1.000000,1.000000}%
\pgfsetstrokecolor{currentstroke}%
\pgfsetstrokeopacity{0.500000}%
\pgfsetdash{}{0pt}%
\pgfpathmoveto{\pgfqpoint{1.985184in}{2.429667in}}%
\pgfpathlineto{\pgfqpoint{1.996779in}{2.432959in}}%
\pgfpathlineto{\pgfqpoint{2.008369in}{2.436255in}}%
\pgfpathlineto{\pgfqpoint{2.019952in}{2.439560in}}%
\pgfpathlineto{\pgfqpoint{2.031530in}{2.442874in}}%
\pgfpathlineto{\pgfqpoint{2.043101in}{2.446202in}}%
\pgfpathlineto{\pgfqpoint{2.037233in}{2.454517in}}%
\pgfpathlineto{\pgfqpoint{2.031369in}{2.462814in}}%
\pgfpathlineto{\pgfqpoint{2.025509in}{2.471091in}}%
\pgfpathlineto{\pgfqpoint{2.019654in}{2.479350in}}%
\pgfpathlineto{\pgfqpoint{2.013803in}{2.487591in}}%
\pgfpathlineto{\pgfqpoint{2.002244in}{2.484278in}}%
\pgfpathlineto{\pgfqpoint{1.990678in}{2.480979in}}%
\pgfpathlineto{\pgfqpoint{1.979107in}{2.477692in}}%
\pgfpathlineto{\pgfqpoint{1.967530in}{2.474412in}}%
\pgfpathlineto{\pgfqpoint{1.955947in}{2.471139in}}%
\pgfpathlineto{\pgfqpoint{1.961785in}{2.462883in}}%
\pgfpathlineto{\pgfqpoint{1.967629in}{2.454609in}}%
\pgfpathlineto{\pgfqpoint{1.973476in}{2.446314in}}%
\pgfpathlineto{\pgfqpoint{1.979328in}{2.438000in}}%
\pgfpathclose%
\pgfusepath{stroke,fill}%
\end{pgfscope}%
\begin{pgfscope}%
\pgfpathrectangle{\pgfqpoint{0.887500in}{0.275000in}}{\pgfqpoint{4.225000in}{4.225000in}}%
\pgfusepath{clip}%
\pgfsetbuttcap%
\pgfsetroundjoin%
\definecolor{currentfill}{rgb}{0.202219,0.715272,0.476084}%
\pgfsetfillcolor{currentfill}%
\pgfsetfillopacity{0.700000}%
\pgfsetlinewidth{0.501875pt}%
\definecolor{currentstroke}{rgb}{1.000000,1.000000,1.000000}%
\pgfsetstrokecolor{currentstroke}%
\pgfsetstrokeopacity{0.500000}%
\pgfsetdash{}{0pt}%
\pgfpathmoveto{\pgfqpoint{2.934011in}{2.760117in}}%
\pgfpathlineto{\pgfqpoint{2.945288in}{2.788241in}}%
\pgfpathlineto{\pgfqpoint{2.956575in}{2.815861in}}%
\pgfpathlineto{\pgfqpoint{2.967871in}{2.843383in}}%
\pgfpathlineto{\pgfqpoint{2.979174in}{2.871035in}}%
\pgfpathlineto{\pgfqpoint{2.990486in}{2.898469in}}%
\pgfpathlineto{\pgfqpoint{2.984295in}{2.904033in}}%
\pgfpathlineto{\pgfqpoint{2.978114in}{2.907604in}}%
\pgfpathlineto{\pgfqpoint{2.971941in}{2.909459in}}%
\pgfpathlineto{\pgfqpoint{2.965778in}{2.909879in}}%
\pgfpathlineto{\pgfqpoint{2.959624in}{2.909142in}}%
\pgfpathlineto{\pgfqpoint{2.948352in}{2.881168in}}%
\pgfpathlineto{\pgfqpoint{2.937087in}{2.853418in}}%
\pgfpathlineto{\pgfqpoint{2.925830in}{2.826224in}}%
\pgfpathlineto{\pgfqpoint{2.914579in}{2.799448in}}%
\pgfpathlineto{\pgfqpoint{2.903335in}{2.772806in}}%
\pgfpathlineto{\pgfqpoint{2.909447in}{2.773261in}}%
\pgfpathlineto{\pgfqpoint{2.915571in}{2.772542in}}%
\pgfpathlineto{\pgfqpoint{2.921706in}{2.770328in}}%
\pgfpathlineto{\pgfqpoint{2.927853in}{2.766294in}}%
\pgfpathclose%
\pgfusepath{stroke,fill}%
\end{pgfscope}%
\begin{pgfscope}%
\pgfpathrectangle{\pgfqpoint{0.887500in}{0.275000in}}{\pgfqpoint{4.225000in}{4.225000in}}%
\pgfusepath{clip}%
\pgfsetbuttcap%
\pgfsetroundjoin%
\definecolor{currentfill}{rgb}{0.159194,0.482237,0.558073}%
\pgfsetfillcolor{currentfill}%
\pgfsetfillopacity{0.700000}%
\pgfsetlinewidth{0.501875pt}%
\definecolor{currentstroke}{rgb}{1.000000,1.000000,1.000000}%
\pgfsetstrokecolor{currentstroke}%
\pgfsetstrokeopacity{0.500000}%
\pgfsetdash{}{0pt}%
\pgfpathmoveto{\pgfqpoint{2.624959in}{2.307614in}}%
\pgfpathlineto{\pgfqpoint{2.636400in}{2.310549in}}%
\pgfpathlineto{\pgfqpoint{2.647835in}{2.313548in}}%
\pgfpathlineto{\pgfqpoint{2.659262in}{2.316714in}}%
\pgfpathlineto{\pgfqpoint{2.670680in}{2.320149in}}%
\pgfpathlineto{\pgfqpoint{2.682090in}{2.323953in}}%
\pgfpathlineto{\pgfqpoint{2.676005in}{2.332695in}}%
\pgfpathlineto{\pgfqpoint{2.669924in}{2.341414in}}%
\pgfpathlineto{\pgfqpoint{2.663848in}{2.350110in}}%
\pgfpathlineto{\pgfqpoint{2.657775in}{2.358783in}}%
\pgfpathlineto{\pgfqpoint{2.651707in}{2.367434in}}%
\pgfpathlineto{\pgfqpoint{2.640309in}{2.363586in}}%
\pgfpathlineto{\pgfqpoint{2.628902in}{2.360142in}}%
\pgfpathlineto{\pgfqpoint{2.617486in}{2.356993in}}%
\pgfpathlineto{\pgfqpoint{2.606062in}{2.354026in}}%
\pgfpathlineto{\pgfqpoint{2.594631in}{2.351131in}}%
\pgfpathlineto{\pgfqpoint{2.600689in}{2.342467in}}%
\pgfpathlineto{\pgfqpoint{2.606750in}{2.333784in}}%
\pgfpathlineto{\pgfqpoint{2.612816in}{2.325082in}}%
\pgfpathlineto{\pgfqpoint{2.618885in}{2.316359in}}%
\pgfpathclose%
\pgfusepath{stroke,fill}%
\end{pgfscope}%
\begin{pgfscope}%
\pgfpathrectangle{\pgfqpoint{0.887500in}{0.275000in}}{\pgfqpoint{4.225000in}{4.225000in}}%
\pgfusepath{clip}%
\pgfsetbuttcap%
\pgfsetroundjoin%
\definecolor{currentfill}{rgb}{0.119738,0.603785,0.541400}%
\pgfsetfillcolor{currentfill}%
\pgfsetfillopacity{0.700000}%
\pgfsetlinewidth{0.501875pt}%
\definecolor{currentstroke}{rgb}{1.000000,1.000000,1.000000}%
\pgfsetstrokecolor{currentstroke}%
\pgfsetstrokeopacity{0.500000}%
\pgfsetdash{}{0pt}%
\pgfpathmoveto{\pgfqpoint{4.213033in}{2.548853in}}%
\pgfpathlineto{\pgfqpoint{4.224077in}{2.552265in}}%
\pgfpathlineto{\pgfqpoint{4.235116in}{2.555679in}}%
\pgfpathlineto{\pgfqpoint{4.246150in}{2.559093in}}%
\pgfpathlineto{\pgfqpoint{4.257178in}{2.562502in}}%
\pgfpathlineto{\pgfqpoint{4.268201in}{2.565903in}}%
\pgfpathlineto{\pgfqpoint{4.261775in}{2.580143in}}%
\pgfpathlineto{\pgfqpoint{4.255350in}{2.594363in}}%
\pgfpathlineto{\pgfqpoint{4.248928in}{2.608563in}}%
\pgfpathlineto{\pgfqpoint{4.242507in}{2.622743in}}%
\pgfpathlineto{\pgfqpoint{4.236088in}{2.636900in}}%
\pgfpathlineto{\pgfqpoint{4.225066in}{2.633457in}}%
\pgfpathlineto{\pgfqpoint{4.214038in}{2.630005in}}%
\pgfpathlineto{\pgfqpoint{4.203005in}{2.626548in}}%
\pgfpathlineto{\pgfqpoint{4.191966in}{2.623090in}}%
\pgfpathlineto{\pgfqpoint{4.180923in}{2.619634in}}%
\pgfpathlineto{\pgfqpoint{4.187340in}{2.605499in}}%
\pgfpathlineto{\pgfqpoint{4.193760in}{2.591352in}}%
\pgfpathlineto{\pgfqpoint{4.200182in}{2.577194in}}%
\pgfpathlineto{\pgfqpoint{4.206606in}{2.563027in}}%
\pgfpathclose%
\pgfusepath{stroke,fill}%
\end{pgfscope}%
\begin{pgfscope}%
\pgfpathrectangle{\pgfqpoint{0.887500in}{0.275000in}}{\pgfqpoint{4.225000in}{4.225000in}}%
\pgfusepath{clip}%
\pgfsetbuttcap%
\pgfsetroundjoin%
\definecolor{currentfill}{rgb}{0.147607,0.511733,0.557049}%
\pgfsetfillcolor{currentfill}%
\pgfsetfillopacity{0.700000}%
\pgfsetlinewidth{0.501875pt}%
\definecolor{currentstroke}{rgb}{1.000000,1.000000,1.000000}%
\pgfsetstrokecolor{currentstroke}%
\pgfsetstrokeopacity{0.500000}%
\pgfsetdash{}{0pt}%
\pgfpathmoveto{\pgfqpoint{2.305046in}{2.369867in}}%
\pgfpathlineto{\pgfqpoint{2.316563in}{2.373156in}}%
\pgfpathlineto{\pgfqpoint{2.328074in}{2.376442in}}%
\pgfpathlineto{\pgfqpoint{2.339580in}{2.379729in}}%
\pgfpathlineto{\pgfqpoint{2.351079in}{2.383022in}}%
\pgfpathlineto{\pgfqpoint{2.362573in}{2.386326in}}%
\pgfpathlineto{\pgfqpoint{2.356595in}{2.394835in}}%
\pgfpathlineto{\pgfqpoint{2.350621in}{2.403329in}}%
\pgfpathlineto{\pgfqpoint{2.344652in}{2.411807in}}%
\pgfpathlineto{\pgfqpoint{2.338686in}{2.420270in}}%
\pgfpathlineto{\pgfqpoint{2.332725in}{2.428717in}}%
\pgfpathlineto{\pgfqpoint{2.321243in}{2.425412in}}%
\pgfpathlineto{\pgfqpoint{2.309755in}{2.422121in}}%
\pgfpathlineto{\pgfqpoint{2.298261in}{2.418838in}}%
\pgfpathlineto{\pgfqpoint{2.286762in}{2.415561in}}%
\pgfpathlineto{\pgfqpoint{2.275257in}{2.412283in}}%
\pgfpathlineto{\pgfqpoint{2.281206in}{2.403833in}}%
\pgfpathlineto{\pgfqpoint{2.287160in}{2.395367in}}%
\pgfpathlineto{\pgfqpoint{2.293118in}{2.386885in}}%
\pgfpathlineto{\pgfqpoint{2.299080in}{2.378385in}}%
\pgfpathclose%
\pgfusepath{stroke,fill}%
\end{pgfscope}%
\begin{pgfscope}%
\pgfpathrectangle{\pgfqpoint{0.887500in}{0.275000in}}{\pgfqpoint{4.225000in}{4.225000in}}%
\pgfusepath{clip}%
\pgfsetbuttcap%
\pgfsetroundjoin%
\definecolor{currentfill}{rgb}{0.149039,0.508051,0.557250}%
\pgfsetfillcolor{currentfill}%
\pgfsetfillopacity{0.700000}%
\pgfsetlinewidth{0.501875pt}%
\definecolor{currentstroke}{rgb}{1.000000,1.000000,1.000000}%
\pgfsetstrokecolor{currentstroke}%
\pgfsetstrokeopacity{0.500000}%
\pgfsetdash{}{0pt}%
\pgfpathmoveto{\pgfqpoint{2.913975in}{2.298432in}}%
\pgfpathlineto{\pgfqpoint{2.925265in}{2.319794in}}%
\pgfpathlineto{\pgfqpoint{2.936547in}{2.344979in}}%
\pgfpathlineto{\pgfqpoint{2.947828in}{2.372919in}}%
\pgfpathlineto{\pgfqpoint{2.959114in}{2.402547in}}%
\pgfpathlineto{\pgfqpoint{2.970408in}{2.432786in}}%
\pgfpathlineto{\pgfqpoint{2.964220in}{2.441742in}}%
\pgfpathlineto{\pgfqpoint{2.958035in}{2.450996in}}%
\pgfpathlineto{\pgfqpoint{2.951852in}{2.460666in}}%
\pgfpathlineto{\pgfqpoint{2.945672in}{2.470870in}}%
\pgfpathlineto{\pgfqpoint{2.939493in}{2.481724in}}%
\pgfpathlineto{\pgfqpoint{2.928234in}{2.450106in}}%
\pgfpathlineto{\pgfqpoint{2.916982in}{2.419218in}}%
\pgfpathlineto{\pgfqpoint{2.905734in}{2.390167in}}%
\pgfpathlineto{\pgfqpoint{2.894483in}{2.364056in}}%
\pgfpathlineto{\pgfqpoint{2.883222in}{2.341987in}}%
\pgfpathlineto{\pgfqpoint{2.889363in}{2.333620in}}%
\pgfpathlineto{\pgfqpoint{2.895509in}{2.325060in}}%
\pgfpathlineto{\pgfqpoint{2.901660in}{2.316330in}}%
\pgfpathlineto{\pgfqpoint{2.907816in}{2.307447in}}%
\pgfpathclose%
\pgfusepath{stroke,fill}%
\end{pgfscope}%
\begin{pgfscope}%
\pgfpathrectangle{\pgfqpoint{0.887500in}{0.275000in}}{\pgfqpoint{4.225000in}{4.225000in}}%
\pgfusepath{clip}%
\pgfsetbuttcap%
\pgfsetroundjoin%
\definecolor{currentfill}{rgb}{0.232815,0.732247,0.459277}%
\pgfsetfillcolor{currentfill}%
\pgfsetfillopacity{0.700000}%
\pgfsetlinewidth{0.501875pt}%
\definecolor{currentstroke}{rgb}{1.000000,1.000000,1.000000}%
\pgfsetstrokecolor{currentstroke}%
\pgfsetstrokeopacity{0.500000}%
\pgfsetdash{}{0pt}%
\pgfpathmoveto{\pgfqpoint{3.831183in}{2.828869in}}%
\pgfpathlineto{\pgfqpoint{3.842324in}{2.832327in}}%
\pgfpathlineto{\pgfqpoint{3.853459in}{2.835789in}}%
\pgfpathlineto{\pgfqpoint{3.864589in}{2.839252in}}%
\pgfpathlineto{\pgfqpoint{3.875714in}{2.842712in}}%
\pgfpathlineto{\pgfqpoint{3.886833in}{2.846165in}}%
\pgfpathlineto{\pgfqpoint{3.880464in}{2.859777in}}%
\pgfpathlineto{\pgfqpoint{3.874098in}{2.873401in}}%
\pgfpathlineto{\pgfqpoint{3.867734in}{2.887046in}}%
\pgfpathlineto{\pgfqpoint{3.861374in}{2.900720in}}%
\pgfpathlineto{\pgfqpoint{3.855017in}{2.914422in}}%
\pgfpathlineto{\pgfqpoint{3.843900in}{2.910942in}}%
\pgfpathlineto{\pgfqpoint{3.832778in}{2.907453in}}%
\pgfpathlineto{\pgfqpoint{3.821650in}{2.903956in}}%
\pgfpathlineto{\pgfqpoint{3.810517in}{2.900455in}}%
\pgfpathlineto{\pgfqpoint{3.799378in}{2.896950in}}%
\pgfpathlineto{\pgfqpoint{3.805734in}{2.883327in}}%
\pgfpathlineto{\pgfqpoint{3.812092in}{2.869708in}}%
\pgfpathlineto{\pgfqpoint{3.818453in}{2.856095in}}%
\pgfpathlineto{\pgfqpoint{3.824817in}{2.842484in}}%
\pgfpathclose%
\pgfusepath{stroke,fill}%
\end{pgfscope}%
\begin{pgfscope}%
\pgfpathrectangle{\pgfqpoint{0.887500in}{0.275000in}}{\pgfqpoint{4.225000in}{4.225000in}}%
\pgfusepath{clip}%
\pgfsetbuttcap%
\pgfsetroundjoin%
\definecolor{currentfill}{rgb}{0.154815,0.493313,0.557840}%
\pgfsetfillcolor{currentfill}%
\pgfsetfillopacity{0.700000}%
\pgfsetlinewidth{0.501875pt}%
\definecolor{currentstroke}{rgb}{1.000000,1.000000,1.000000}%
\pgfsetstrokecolor{currentstroke}%
\pgfsetstrokeopacity{0.500000}%
\pgfsetdash{}{0pt}%
\pgfpathmoveto{\pgfqpoint{4.507309in}{2.316478in}}%
\pgfpathlineto{\pgfqpoint{4.518285in}{2.320044in}}%
\pgfpathlineto{\pgfqpoint{4.529256in}{2.323595in}}%
\pgfpathlineto{\pgfqpoint{4.540222in}{2.327140in}}%
\pgfpathlineto{\pgfqpoint{4.551182in}{2.330683in}}%
\pgfpathlineto{\pgfqpoint{4.544693in}{2.344723in}}%
\pgfpathlineto{\pgfqpoint{4.538205in}{2.358730in}}%
\pgfpathlineto{\pgfqpoint{4.531718in}{2.372715in}}%
\pgfpathlineto{\pgfqpoint{4.525234in}{2.386687in}}%
\pgfpathlineto{\pgfqpoint{4.518753in}{2.400657in}}%
\pgfpathlineto{\pgfqpoint{4.507791in}{2.397015in}}%
\pgfpathlineto{\pgfqpoint{4.496825in}{2.393393in}}%
\pgfpathlineto{\pgfqpoint{4.485854in}{2.389789in}}%
\pgfpathlineto{\pgfqpoint{4.474878in}{2.386199in}}%
\pgfpathlineto{\pgfqpoint{4.481359in}{2.372248in}}%
\pgfpathlineto{\pgfqpoint{4.487842in}{2.358314in}}%
\pgfpathlineto{\pgfqpoint{4.494329in}{2.344382in}}%
\pgfpathlineto{\pgfqpoint{4.500818in}{2.330441in}}%
\pgfpathclose%
\pgfusepath{stroke,fill}%
\end{pgfscope}%
\begin{pgfscope}%
\pgfpathrectangle{\pgfqpoint{0.887500in}{0.275000in}}{\pgfqpoint{4.225000in}{4.225000in}}%
\pgfusepath{clip}%
\pgfsetbuttcap%
\pgfsetroundjoin%
\definecolor{currentfill}{rgb}{0.281477,0.755203,0.432552}%
\pgfsetfillcolor{currentfill}%
\pgfsetfillopacity{0.700000}%
\pgfsetlinewidth{0.501875pt}%
\definecolor{currentstroke}{rgb}{1.000000,1.000000,1.000000}%
\pgfsetstrokecolor{currentstroke}%
\pgfsetstrokeopacity{0.500000}%
\pgfsetdash{}{0pt}%
\pgfpathmoveto{\pgfqpoint{3.743607in}{2.879470in}}%
\pgfpathlineto{\pgfqpoint{3.754772in}{2.882954in}}%
\pgfpathlineto{\pgfqpoint{3.765931in}{2.886445in}}%
\pgfpathlineto{\pgfqpoint{3.777085in}{2.889944in}}%
\pgfpathlineto{\pgfqpoint{3.788235in}{2.893446in}}%
\pgfpathlineto{\pgfqpoint{3.799378in}{2.896950in}}%
\pgfpathlineto{\pgfqpoint{3.793025in}{2.910566in}}%
\pgfpathlineto{\pgfqpoint{3.786674in}{2.924159in}}%
\pgfpathlineto{\pgfqpoint{3.780325in}{2.937717in}}%
\pgfpathlineto{\pgfqpoint{3.773978in}{2.951224in}}%
\pgfpathlineto{\pgfqpoint{3.767632in}{2.964667in}}%
\pgfpathlineto{\pgfqpoint{3.756489in}{2.961029in}}%
\pgfpathlineto{\pgfqpoint{3.745341in}{2.957382in}}%
\pgfpathlineto{\pgfqpoint{3.734187in}{2.953728in}}%
\pgfpathlineto{\pgfqpoint{3.723028in}{2.950075in}}%
\pgfpathlineto{\pgfqpoint{3.711864in}{2.946433in}}%
\pgfpathlineto{\pgfqpoint{3.718209in}{2.933110in}}%
\pgfpathlineto{\pgfqpoint{3.724555in}{2.919750in}}%
\pgfpathlineto{\pgfqpoint{3.730904in}{2.906356in}}%
\pgfpathlineto{\pgfqpoint{3.737254in}{2.892928in}}%
\pgfpathclose%
\pgfusepath{stroke,fill}%
\end{pgfscope}%
\begin{pgfscope}%
\pgfpathrectangle{\pgfqpoint{0.887500in}{0.275000in}}{\pgfqpoint{4.225000in}{4.225000in}}%
\pgfusepath{clip}%
\pgfsetbuttcap%
\pgfsetroundjoin%
\definecolor{currentfill}{rgb}{0.121380,0.629492,0.531973}%
\pgfsetfillcolor{currentfill}%
\pgfsetfillopacity{0.700000}%
\pgfsetlinewidth{0.501875pt}%
\definecolor{currentstroke}{rgb}{1.000000,1.000000,1.000000}%
\pgfsetstrokecolor{currentstroke}%
\pgfsetstrokeopacity{0.500000}%
\pgfsetdash{}{0pt}%
\pgfpathmoveto{\pgfqpoint{4.125628in}{2.602489in}}%
\pgfpathlineto{\pgfqpoint{4.136697in}{2.605895in}}%
\pgfpathlineto{\pgfqpoint{4.147761in}{2.609313in}}%
\pgfpathlineto{\pgfqpoint{4.158820in}{2.612743in}}%
\pgfpathlineto{\pgfqpoint{4.169874in}{2.616184in}}%
\pgfpathlineto{\pgfqpoint{4.180923in}{2.619634in}}%
\pgfpathlineto{\pgfqpoint{4.174507in}{2.633755in}}%
\pgfpathlineto{\pgfqpoint{4.168094in}{2.647860in}}%
\pgfpathlineto{\pgfqpoint{4.161682in}{2.661949in}}%
\pgfpathlineto{\pgfqpoint{4.155273in}{2.676019in}}%
\pgfpathlineto{\pgfqpoint{4.148866in}{2.690072in}}%
\pgfpathlineto{\pgfqpoint{4.137819in}{2.686623in}}%
\pgfpathlineto{\pgfqpoint{4.126767in}{2.683180in}}%
\pgfpathlineto{\pgfqpoint{4.115710in}{2.679745in}}%
\pgfpathlineto{\pgfqpoint{4.104648in}{2.676317in}}%
\pgfpathlineto{\pgfqpoint{4.093580in}{2.672898in}}%
\pgfpathlineto{\pgfqpoint{4.099985in}{2.658829in}}%
\pgfpathlineto{\pgfqpoint{4.106392in}{2.644751in}}%
\pgfpathlineto{\pgfqpoint{4.112802in}{2.630669in}}%
\pgfpathlineto{\pgfqpoint{4.119214in}{2.616582in}}%
\pgfpathclose%
\pgfusepath{stroke,fill}%
\end{pgfscope}%
\begin{pgfscope}%
\pgfpathrectangle{\pgfqpoint{0.887500in}{0.275000in}}{\pgfqpoint{4.225000in}{4.225000in}}%
\pgfusepath{clip}%
\pgfsetbuttcap%
\pgfsetroundjoin%
\definecolor{currentfill}{rgb}{0.143343,0.522773,0.556295}%
\pgfsetfillcolor{currentfill}%
\pgfsetfillopacity{0.700000}%
\pgfsetlinewidth{0.501875pt}%
\definecolor{currentstroke}{rgb}{1.000000,1.000000,1.000000}%
\pgfsetstrokecolor{currentstroke}%
\pgfsetstrokeopacity{0.500000}%
\pgfsetdash{}{0pt}%
\pgfpathmoveto{\pgfqpoint{4.419926in}{2.368373in}}%
\pgfpathlineto{\pgfqpoint{4.430927in}{2.371931in}}%
\pgfpathlineto{\pgfqpoint{4.441922in}{2.375490in}}%
\pgfpathlineto{\pgfqpoint{4.452912in}{2.379052in}}%
\pgfpathlineto{\pgfqpoint{4.463898in}{2.382621in}}%
\pgfpathlineto{\pgfqpoint{4.474878in}{2.386199in}}%
\pgfpathlineto{\pgfqpoint{4.468401in}{2.400179in}}%
\pgfpathlineto{\pgfqpoint{4.461929in}{2.414200in}}%
\pgfpathlineto{\pgfqpoint{4.455460in}{2.428272in}}%
\pgfpathlineto{\pgfqpoint{4.448996in}{2.442392in}}%
\pgfpathlineto{\pgfqpoint{4.442537in}{2.456554in}}%
\pgfpathlineto{\pgfqpoint{4.431561in}{2.453065in}}%
\pgfpathlineto{\pgfqpoint{4.420581in}{2.449598in}}%
\pgfpathlineto{\pgfqpoint{4.409597in}{2.446153in}}%
\pgfpathlineto{\pgfqpoint{4.398608in}{2.442731in}}%
\pgfpathlineto{\pgfqpoint{4.387614in}{2.439329in}}%
\pgfpathlineto{\pgfqpoint{4.394068in}{2.425053in}}%
\pgfpathlineto{\pgfqpoint{4.400527in}{2.410814in}}%
\pgfpathlineto{\pgfqpoint{4.406989in}{2.396620in}}%
\pgfpathlineto{\pgfqpoint{4.413455in}{2.382475in}}%
\pgfpathclose%
\pgfusepath{stroke,fill}%
\end{pgfscope}%
\begin{pgfscope}%
\pgfpathrectangle{\pgfqpoint{0.887500in}{0.275000in}}{\pgfqpoint{4.225000in}{4.225000in}}%
\pgfusepath{clip}%
\pgfsetbuttcap%
\pgfsetroundjoin%
\definecolor{currentfill}{rgb}{0.129933,0.559582,0.551864}%
\pgfsetfillcolor{currentfill}%
\pgfsetfillopacity{0.700000}%
\pgfsetlinewidth{0.501875pt}%
\definecolor{currentstroke}{rgb}{1.000000,1.000000,1.000000}%
\pgfsetstrokecolor{currentstroke}%
\pgfsetstrokeopacity{0.500000}%
\pgfsetdash{}{0pt}%
\pgfpathmoveto{\pgfqpoint{1.752571in}{2.462826in}}%
\pgfpathlineto{\pgfqpoint{1.764226in}{2.466161in}}%
\pgfpathlineto{\pgfqpoint{1.775876in}{2.469494in}}%
\pgfpathlineto{\pgfqpoint{1.787520in}{2.472822in}}%
\pgfpathlineto{\pgfqpoint{1.799158in}{2.476143in}}%
\pgfpathlineto{\pgfqpoint{1.810791in}{2.479455in}}%
\pgfpathlineto{\pgfqpoint{1.805002in}{2.487633in}}%
\pgfpathlineto{\pgfqpoint{1.799218in}{2.495793in}}%
\pgfpathlineto{\pgfqpoint{1.793438in}{2.503935in}}%
\pgfpathlineto{\pgfqpoint{1.787662in}{2.512059in}}%
\pgfpathlineto{\pgfqpoint{1.781891in}{2.520167in}}%
\pgfpathlineto{\pgfqpoint{1.770270in}{2.516873in}}%
\pgfpathlineto{\pgfqpoint{1.758644in}{2.513570in}}%
\pgfpathlineto{\pgfqpoint{1.747012in}{2.510261in}}%
\pgfpathlineto{\pgfqpoint{1.735375in}{2.506947in}}%
\pgfpathlineto{\pgfqpoint{1.723732in}{2.503631in}}%
\pgfpathlineto{\pgfqpoint{1.729491in}{2.495505in}}%
\pgfpathlineto{\pgfqpoint{1.735255in}{2.487362in}}%
\pgfpathlineto{\pgfqpoint{1.741022in}{2.479202in}}%
\pgfpathlineto{\pgfqpoint{1.746794in}{2.471024in}}%
\pgfpathclose%
\pgfusepath{stroke,fill}%
\end{pgfscope}%
\begin{pgfscope}%
\pgfpathrectangle{\pgfqpoint{0.887500in}{0.275000in}}{\pgfqpoint{4.225000in}{4.225000in}}%
\pgfusepath{clip}%
\pgfsetbuttcap%
\pgfsetroundjoin%
\definecolor{currentfill}{rgb}{0.140536,0.530132,0.555659}%
\pgfsetfillcolor{currentfill}%
\pgfsetfillopacity{0.700000}%
\pgfsetlinewidth{0.501875pt}%
\definecolor{currentstroke}{rgb}{1.000000,1.000000,1.000000}%
\pgfsetstrokecolor{currentstroke}%
\pgfsetstrokeopacity{0.500000}%
\pgfsetdash{}{0pt}%
\pgfpathmoveto{\pgfqpoint{2.072505in}{2.404340in}}%
\pgfpathlineto{\pgfqpoint{2.084082in}{2.407699in}}%
\pgfpathlineto{\pgfqpoint{2.095654in}{2.411065in}}%
\pgfpathlineto{\pgfqpoint{2.107219in}{2.414434in}}%
\pgfpathlineto{\pgfqpoint{2.118779in}{2.417804in}}%
\pgfpathlineto{\pgfqpoint{2.130333in}{2.421170in}}%
\pgfpathlineto{\pgfqpoint{2.124432in}{2.429559in}}%
\pgfpathlineto{\pgfqpoint{2.118535in}{2.437930in}}%
\pgfpathlineto{\pgfqpoint{2.112642in}{2.446284in}}%
\pgfpathlineto{\pgfqpoint{2.106753in}{2.454619in}}%
\pgfpathlineto{\pgfqpoint{2.100869in}{2.462937in}}%
\pgfpathlineto{\pgfqpoint{2.089327in}{2.459590in}}%
\pgfpathlineto{\pgfqpoint{2.077779in}{2.456239in}}%
\pgfpathlineto{\pgfqpoint{2.066225in}{2.452888in}}%
\pgfpathlineto{\pgfqpoint{2.054666in}{2.449541in}}%
\pgfpathlineto{\pgfqpoint{2.043101in}{2.446202in}}%
\pgfpathlineto{\pgfqpoint{2.048973in}{2.437868in}}%
\pgfpathlineto{\pgfqpoint{2.054850in}{2.429515in}}%
\pgfpathlineto{\pgfqpoint{2.060731in}{2.421143in}}%
\pgfpathlineto{\pgfqpoint{2.066616in}{2.412751in}}%
\pgfpathclose%
\pgfusepath{stroke,fill}%
\end{pgfscope}%
\begin{pgfscope}%
\pgfpathrectangle{\pgfqpoint{0.887500in}{0.275000in}}{\pgfqpoint{4.225000in}{4.225000in}}%
\pgfusepath{clip}%
\pgfsetbuttcap%
\pgfsetroundjoin%
\definecolor{currentfill}{rgb}{0.468053,0.818921,0.323998}%
\pgfsetfillcolor{currentfill}%
\pgfsetfillopacity{0.700000}%
\pgfsetlinewidth{0.501875pt}%
\definecolor{currentstroke}{rgb}{1.000000,1.000000,1.000000}%
\pgfsetstrokecolor{currentstroke}%
\pgfsetstrokeopacity{0.500000}%
\pgfsetdash{}{0pt}%
\pgfpathmoveto{\pgfqpoint{3.016122in}{3.037312in}}%
\pgfpathlineto{\pgfqpoint{3.027447in}{3.057760in}}%
\pgfpathlineto{\pgfqpoint{3.038777in}{3.076200in}}%
\pgfpathlineto{\pgfqpoint{3.050110in}{3.092861in}}%
\pgfpathlineto{\pgfqpoint{3.061445in}{3.107973in}}%
\pgfpathlineto{\pgfqpoint{3.072781in}{3.121767in}}%
\pgfpathlineto{\pgfqpoint{3.066560in}{3.129456in}}%
\pgfpathlineto{\pgfqpoint{3.060344in}{3.136867in}}%
\pgfpathlineto{\pgfqpoint{3.054132in}{3.144024in}}%
\pgfpathlineto{\pgfqpoint{3.047924in}{3.150954in}}%
\pgfpathlineto{\pgfqpoint{3.041721in}{3.157681in}}%
\pgfpathlineto{\pgfqpoint{3.030410in}{3.137658in}}%
\pgfpathlineto{\pgfqpoint{3.019105in}{3.116037in}}%
\pgfpathlineto{\pgfqpoint{3.007806in}{3.092868in}}%
\pgfpathlineto{\pgfqpoint{2.996515in}{3.068200in}}%
\pgfpathlineto{\pgfqpoint{2.985234in}{3.042082in}}%
\pgfpathlineto{\pgfqpoint{2.991398in}{3.041649in}}%
\pgfpathlineto{\pgfqpoint{2.997569in}{3.041091in}}%
\pgfpathlineto{\pgfqpoint{3.003746in}{3.040272in}}%
\pgfpathlineto{\pgfqpoint{3.009931in}{3.039057in}}%
\pgfpathclose%
\pgfusepath{stroke,fill}%
\end{pgfscope}%
\begin{pgfscope}%
\pgfpathrectangle{\pgfqpoint{0.887500in}{0.275000in}}{\pgfqpoint{4.225000in}{4.225000in}}%
\pgfusepath{clip}%
\pgfsetbuttcap%
\pgfsetroundjoin%
\definecolor{currentfill}{rgb}{0.150476,0.504369,0.557430}%
\pgfsetfillcolor{currentfill}%
\pgfsetfillopacity{0.700000}%
\pgfsetlinewidth{0.501875pt}%
\definecolor{currentstroke}{rgb}{1.000000,1.000000,1.000000}%
\pgfsetstrokecolor{currentstroke}%
\pgfsetstrokeopacity{0.500000}%
\pgfsetdash{}{0pt}%
\pgfpathmoveto{\pgfqpoint{2.392524in}{2.343535in}}%
\pgfpathlineto{\pgfqpoint{2.404023in}{2.346862in}}%
\pgfpathlineto{\pgfqpoint{2.415516in}{2.350209in}}%
\pgfpathlineto{\pgfqpoint{2.427004in}{2.353580in}}%
\pgfpathlineto{\pgfqpoint{2.438485in}{2.356981in}}%
\pgfpathlineto{\pgfqpoint{2.449959in}{2.360418in}}%
\pgfpathlineto{\pgfqpoint{2.443949in}{2.368997in}}%
\pgfpathlineto{\pgfqpoint{2.437943in}{2.377561in}}%
\pgfpathlineto{\pgfqpoint{2.431942in}{2.386110in}}%
\pgfpathlineto{\pgfqpoint{2.425944in}{2.394645in}}%
\pgfpathlineto{\pgfqpoint{2.419950in}{2.403165in}}%
\pgfpathlineto{\pgfqpoint{2.408487in}{2.399739in}}%
\pgfpathlineto{\pgfqpoint{2.397018in}{2.396347in}}%
\pgfpathlineto{\pgfqpoint{2.385542in}{2.392984in}}%
\pgfpathlineto{\pgfqpoint{2.374060in}{2.389646in}}%
\pgfpathlineto{\pgfqpoint{2.362573in}{2.386326in}}%
\pgfpathlineto{\pgfqpoint{2.368555in}{2.377802in}}%
\pgfpathlineto{\pgfqpoint{2.374541in}{2.369260in}}%
\pgfpathlineto{\pgfqpoint{2.380531in}{2.360702in}}%
\pgfpathlineto{\pgfqpoint{2.386525in}{2.352127in}}%
\pgfpathclose%
\pgfusepath{stroke,fill}%
\end{pgfscope}%
\begin{pgfscope}%
\pgfpathrectangle{\pgfqpoint{0.887500in}{0.275000in}}{\pgfqpoint{4.225000in}{4.225000in}}%
\pgfusepath{clip}%
\pgfsetbuttcap%
\pgfsetroundjoin%
\definecolor{currentfill}{rgb}{0.162142,0.474838,0.558140}%
\pgfsetfillcolor{currentfill}%
\pgfsetfillopacity{0.700000}%
\pgfsetlinewidth{0.501875pt}%
\definecolor{currentstroke}{rgb}{1.000000,1.000000,1.000000}%
\pgfsetstrokecolor{currentstroke}%
\pgfsetstrokeopacity{0.500000}%
\pgfsetdash{}{0pt}%
\pgfpathmoveto{\pgfqpoint{2.712575in}{2.279885in}}%
\pgfpathlineto{\pgfqpoint{2.723988in}{2.284045in}}%
\pgfpathlineto{\pgfqpoint{2.735393in}{2.288507in}}%
\pgfpathlineto{\pgfqpoint{2.746793in}{2.292973in}}%
\pgfpathlineto{\pgfqpoint{2.758190in}{2.297140in}}%
\pgfpathlineto{\pgfqpoint{2.769586in}{2.300702in}}%
\pgfpathlineto{\pgfqpoint{2.763469in}{2.309562in}}%
\pgfpathlineto{\pgfqpoint{2.757356in}{2.318406in}}%
\pgfpathlineto{\pgfqpoint{2.751247in}{2.327239in}}%
\pgfpathlineto{\pgfqpoint{2.745142in}{2.336066in}}%
\pgfpathlineto{\pgfqpoint{2.739041in}{2.344892in}}%
\pgfpathlineto{\pgfqpoint{2.727655in}{2.341510in}}%
\pgfpathlineto{\pgfqpoint{2.716269in}{2.337360in}}%
\pgfpathlineto{\pgfqpoint{2.704882in}{2.332807in}}%
\pgfpathlineto{\pgfqpoint{2.693490in}{2.328219in}}%
\pgfpathlineto{\pgfqpoint{2.682090in}{2.323953in}}%
\pgfpathlineto{\pgfqpoint{2.688179in}{2.315187in}}%
\pgfpathlineto{\pgfqpoint{2.694272in}{2.306398in}}%
\pgfpathlineto{\pgfqpoint{2.700369in}{2.297585in}}%
\pgfpathlineto{\pgfqpoint{2.706470in}{2.288747in}}%
\pgfpathclose%
\pgfusepath{stroke,fill}%
\end{pgfscope}%
\begin{pgfscope}%
\pgfpathrectangle{\pgfqpoint{0.887500in}{0.275000in}}{\pgfqpoint{4.225000in}{4.225000in}}%
\pgfusepath{clip}%
\pgfsetbuttcap%
\pgfsetroundjoin%
\definecolor{currentfill}{rgb}{0.327796,0.773980,0.406640}%
\pgfsetfillcolor{currentfill}%
\pgfsetfillopacity{0.700000}%
\pgfsetlinewidth{0.501875pt}%
\definecolor{currentstroke}{rgb}{1.000000,1.000000,1.000000}%
\pgfsetstrokecolor{currentstroke}%
\pgfsetstrokeopacity{0.500000}%
\pgfsetdash{}{0pt}%
\pgfpathmoveto{\pgfqpoint{3.655974in}{2.928729in}}%
\pgfpathlineto{\pgfqpoint{3.667161in}{2.932172in}}%
\pgfpathlineto{\pgfqpoint{3.678344in}{2.935671in}}%
\pgfpathlineto{\pgfqpoint{3.689522in}{2.939221in}}%
\pgfpathlineto{\pgfqpoint{3.700695in}{2.942811in}}%
\pgfpathlineto{\pgfqpoint{3.711864in}{2.946433in}}%
\pgfpathlineto{\pgfqpoint{3.705522in}{2.959717in}}%
\pgfpathlineto{\pgfqpoint{3.699182in}{2.972959in}}%
\pgfpathlineto{\pgfqpoint{3.692843in}{2.986159in}}%
\pgfpathlineto{\pgfqpoint{3.686507in}{2.999314in}}%
\pgfpathlineto{\pgfqpoint{3.680173in}{3.012421in}}%
\pgfpathlineto{\pgfqpoint{3.669009in}{3.008870in}}%
\pgfpathlineto{\pgfqpoint{3.657840in}{3.005348in}}%
\pgfpathlineto{\pgfqpoint{3.646666in}{3.001861in}}%
\pgfpathlineto{\pgfqpoint{3.635488in}{2.998414in}}%
\pgfpathlineto{\pgfqpoint{3.624305in}{2.995009in}}%
\pgfpathlineto{\pgfqpoint{3.630633in}{2.981737in}}%
\pgfpathlineto{\pgfqpoint{3.636964in}{2.968477in}}%
\pgfpathlineto{\pgfqpoint{3.643298in}{2.955226in}}%
\pgfpathlineto{\pgfqpoint{3.649634in}{2.941979in}}%
\pgfpathclose%
\pgfusepath{stroke,fill}%
\end{pgfscope}%
\begin{pgfscope}%
\pgfpathrectangle{\pgfqpoint{0.887500in}{0.275000in}}{\pgfqpoint{4.225000in}{4.225000in}}%
\pgfusepath{clip}%
\pgfsetbuttcap%
\pgfsetroundjoin%
\definecolor{currentfill}{rgb}{0.636902,0.856542,0.216620}%
\pgfsetfillcolor{currentfill}%
\pgfsetfillopacity{0.700000}%
\pgfsetlinewidth{0.501875pt}%
\definecolor{currentstroke}{rgb}{1.000000,1.000000,1.000000}%
\pgfsetstrokecolor{currentstroke}%
\pgfsetstrokeopacity{0.500000}%
\pgfsetdash{}{0pt}%
\pgfpathmoveto{\pgfqpoint{3.129441in}{3.177455in}}%
\pgfpathlineto{\pgfqpoint{3.140765in}{3.186266in}}%
\pgfpathlineto{\pgfqpoint{3.152085in}{3.194243in}}%
\pgfpathlineto{\pgfqpoint{3.163399in}{3.201358in}}%
\pgfpathlineto{\pgfqpoint{3.174708in}{3.207583in}}%
\pgfpathlineto{\pgfqpoint{3.186010in}{3.212971in}}%
\pgfpathlineto{\pgfqpoint{3.179769in}{3.224667in}}%
\pgfpathlineto{\pgfqpoint{3.173530in}{3.236294in}}%
\pgfpathlineto{\pgfqpoint{3.167295in}{3.247833in}}%
\pgfpathlineto{\pgfqpoint{3.161062in}{3.259267in}}%
\pgfpathlineto{\pgfqpoint{3.154833in}{3.270577in}}%
\pgfpathlineto{\pgfqpoint{3.143539in}{3.265583in}}%
\pgfpathlineto{\pgfqpoint{3.132239in}{3.259423in}}%
\pgfpathlineto{\pgfqpoint{3.120934in}{3.251962in}}%
\pgfpathlineto{\pgfqpoint{3.109623in}{3.243129in}}%
\pgfpathlineto{\pgfqpoint{3.098308in}{3.232857in}}%
\pgfpathlineto{\pgfqpoint{3.104529in}{3.222077in}}%
\pgfpathlineto{\pgfqpoint{3.110752in}{3.211124in}}%
\pgfpathlineto{\pgfqpoint{3.116979in}{3.200019in}}%
\pgfpathlineto{\pgfqpoint{3.123209in}{3.188789in}}%
\pgfpathclose%
\pgfusepath{stroke,fill}%
\end{pgfscope}%
\begin{pgfscope}%
\pgfpathrectangle{\pgfqpoint{0.887500in}{0.275000in}}{\pgfqpoint{4.225000in}{4.225000in}}%
\pgfusepath{clip}%
\pgfsetbuttcap%
\pgfsetroundjoin%
\definecolor{currentfill}{rgb}{0.134692,0.658636,0.517649}%
\pgfsetfillcolor{currentfill}%
\pgfsetfillopacity{0.700000}%
\pgfsetlinewidth{0.501875pt}%
\definecolor{currentstroke}{rgb}{1.000000,1.000000,1.000000}%
\pgfsetstrokecolor{currentstroke}%
\pgfsetstrokeopacity{0.500000}%
\pgfsetdash{}{0pt}%
\pgfpathmoveto{\pgfqpoint{4.038167in}{2.655945in}}%
\pgfpathlineto{\pgfqpoint{4.049260in}{2.659316in}}%
\pgfpathlineto{\pgfqpoint{4.060348in}{2.662696in}}%
\pgfpathlineto{\pgfqpoint{4.071430in}{2.666087in}}%
\pgfpathlineto{\pgfqpoint{4.082508in}{2.669488in}}%
\pgfpathlineto{\pgfqpoint{4.093580in}{2.672898in}}%
\pgfpathlineto{\pgfqpoint{4.087178in}{2.686955in}}%
\pgfpathlineto{\pgfqpoint{4.080777in}{2.700997in}}%
\pgfpathlineto{\pgfqpoint{4.074379in}{2.715020in}}%
\pgfpathlineto{\pgfqpoint{4.067983in}{2.729020in}}%
\pgfpathlineto{\pgfqpoint{4.061588in}{2.742995in}}%
\pgfpathlineto{\pgfqpoint{4.050518in}{2.739597in}}%
\pgfpathlineto{\pgfqpoint{4.039443in}{2.736203in}}%
\pgfpathlineto{\pgfqpoint{4.028362in}{2.732812in}}%
\pgfpathlineto{\pgfqpoint{4.017277in}{2.729422in}}%
\pgfpathlineto{\pgfqpoint{4.006185in}{2.726031in}}%
\pgfpathlineto{\pgfqpoint{4.012578in}{2.712065in}}%
\pgfpathlineto{\pgfqpoint{4.018972in}{2.698067in}}%
\pgfpathlineto{\pgfqpoint{4.025368in}{2.684044in}}%
\pgfpathlineto{\pgfqpoint{4.031767in}{2.670002in}}%
\pgfpathclose%
\pgfusepath{stroke,fill}%
\end{pgfscope}%
\begin{pgfscope}%
\pgfpathrectangle{\pgfqpoint{0.887500in}{0.275000in}}{\pgfqpoint{4.225000in}{4.225000in}}%
\pgfusepath{clip}%
\pgfsetbuttcap%
\pgfsetroundjoin%
\definecolor{currentfill}{rgb}{0.386433,0.794644,0.372886}%
\pgfsetfillcolor{currentfill}%
\pgfsetfillopacity{0.700000}%
\pgfsetlinewidth{0.501875pt}%
\definecolor{currentstroke}{rgb}{1.000000,1.000000,1.000000}%
\pgfsetstrokecolor{currentstroke}%
\pgfsetstrokeopacity{0.500000}%
\pgfsetdash{}{0pt}%
\pgfpathmoveto{\pgfqpoint{3.568312in}{2.978219in}}%
\pgfpathlineto{\pgfqpoint{3.579522in}{2.981590in}}%
\pgfpathlineto{\pgfqpoint{3.590726in}{2.984941in}}%
\pgfpathlineto{\pgfqpoint{3.601924in}{2.988286in}}%
\pgfpathlineto{\pgfqpoint{3.613117in}{2.991637in}}%
\pgfpathlineto{\pgfqpoint{3.624305in}{2.995009in}}%
\pgfpathlineto{\pgfqpoint{3.617980in}{3.008280in}}%
\pgfpathlineto{\pgfqpoint{3.611657in}{3.021527in}}%
\pgfpathlineto{\pgfqpoint{3.605336in}{3.034726in}}%
\pgfpathlineto{\pgfqpoint{3.599016in}{3.047855in}}%
\pgfpathlineto{\pgfqpoint{3.592698in}{3.060891in}}%
\pgfpathlineto{\pgfqpoint{3.581515in}{3.057561in}}%
\pgfpathlineto{\pgfqpoint{3.570325in}{3.054186in}}%
\pgfpathlineto{\pgfqpoint{3.559129in}{3.050768in}}%
\pgfpathlineto{\pgfqpoint{3.547928in}{3.047308in}}%
\pgfpathlineto{\pgfqpoint{3.536720in}{3.043809in}}%
\pgfpathlineto{\pgfqpoint{3.543035in}{3.030809in}}%
\pgfpathlineto{\pgfqpoint{3.549351in}{3.017737in}}%
\pgfpathlineto{\pgfqpoint{3.555669in}{3.004607in}}%
\pgfpathlineto{\pgfqpoint{3.561989in}{2.991430in}}%
\pgfpathclose%
\pgfusepath{stroke,fill}%
\end{pgfscope}%
\begin{pgfscope}%
\pgfpathrectangle{\pgfqpoint{0.887500in}{0.275000in}}{\pgfqpoint{4.225000in}{4.225000in}}%
\pgfusepath{clip}%
\pgfsetbuttcap%
\pgfsetroundjoin%
\definecolor{currentfill}{rgb}{0.133743,0.548535,0.553541}%
\pgfsetfillcolor{currentfill}%
\pgfsetfillopacity{0.700000}%
\pgfsetlinewidth{0.501875pt}%
\definecolor{currentstroke}{rgb}{1.000000,1.000000,1.000000}%
\pgfsetstrokecolor{currentstroke}%
\pgfsetstrokeopacity{0.500000}%
\pgfsetdash{}{0pt}%
\pgfpathmoveto{\pgfqpoint{4.332576in}{2.422623in}}%
\pgfpathlineto{\pgfqpoint{4.343593in}{2.425926in}}%
\pgfpathlineto{\pgfqpoint{4.354605in}{2.429247in}}%
\pgfpathlineto{\pgfqpoint{4.365613in}{2.432588in}}%
\pgfpathlineto{\pgfqpoint{4.376616in}{2.435948in}}%
\pgfpathlineto{\pgfqpoint{4.387614in}{2.439329in}}%
\pgfpathlineto{\pgfqpoint{4.381163in}{2.453636in}}%
\pgfpathlineto{\pgfqpoint{4.374716in}{2.467968in}}%
\pgfpathlineto{\pgfqpoint{4.368271in}{2.482319in}}%
\pgfpathlineto{\pgfqpoint{4.361830in}{2.496681in}}%
\pgfpathlineto{\pgfqpoint{4.355391in}{2.511047in}}%
\pgfpathlineto{\pgfqpoint{4.344396in}{2.507716in}}%
\pgfpathlineto{\pgfqpoint{4.333395in}{2.504391in}}%
\pgfpathlineto{\pgfqpoint{4.322390in}{2.501072in}}%
\pgfpathlineto{\pgfqpoint{4.311379in}{2.497758in}}%
\pgfpathlineto{\pgfqpoint{4.300363in}{2.494448in}}%
\pgfpathlineto{\pgfqpoint{4.306801in}{2.480110in}}%
\pgfpathlineto{\pgfqpoint{4.313242in}{2.465759in}}%
\pgfpathlineto{\pgfqpoint{4.319684in}{2.451393in}}%
\pgfpathlineto{\pgfqpoint{4.326129in}{2.437015in}}%
\pgfpathclose%
\pgfusepath{stroke,fill}%
\end{pgfscope}%
\begin{pgfscope}%
\pgfpathrectangle{\pgfqpoint{0.887500in}{0.275000in}}{\pgfqpoint{4.225000in}{4.225000in}}%
\pgfusepath{clip}%
\pgfsetbuttcap%
\pgfsetroundjoin%
\definecolor{currentfill}{rgb}{0.125394,0.574318,0.549086}%
\pgfsetfillcolor{currentfill}%
\pgfsetfillopacity{0.700000}%
\pgfsetlinewidth{0.501875pt}%
\definecolor{currentstroke}{rgb}{1.000000,1.000000,1.000000}%
\pgfsetstrokecolor{currentstroke}%
\pgfsetstrokeopacity{0.500000}%
\pgfsetdash{}{0pt}%
\pgfpathmoveto{\pgfqpoint{1.519862in}{2.494193in}}%
\pgfpathlineto{\pgfqpoint{1.531576in}{2.497572in}}%
\pgfpathlineto{\pgfqpoint{1.543285in}{2.500939in}}%
\pgfpathlineto{\pgfqpoint{1.554989in}{2.504294in}}%
\pgfpathlineto{\pgfqpoint{1.566687in}{2.507638in}}%
\pgfpathlineto{\pgfqpoint{1.578380in}{2.510973in}}%
\pgfpathlineto{\pgfqpoint{1.572672in}{2.519036in}}%
\pgfpathlineto{\pgfqpoint{1.566968in}{2.527080in}}%
\pgfpathlineto{\pgfqpoint{1.561269in}{2.535102in}}%
\pgfpathlineto{\pgfqpoint{1.555574in}{2.543100in}}%
\pgfpathlineto{\pgfqpoint{1.549884in}{2.551074in}}%
\pgfpathlineto{\pgfqpoint{1.538204in}{2.547752in}}%
\pgfpathlineto{\pgfqpoint{1.526518in}{2.544420in}}%
\pgfpathlineto{\pgfqpoint{1.514828in}{2.541077in}}%
\pgfpathlineto{\pgfqpoint{1.503132in}{2.537723in}}%
\pgfpathlineto{\pgfqpoint{1.491430in}{2.534355in}}%
\pgfpathlineto{\pgfqpoint{1.497107in}{2.526371in}}%
\pgfpathlineto{\pgfqpoint{1.502789in}{2.518361in}}%
\pgfpathlineto{\pgfqpoint{1.508475in}{2.510327in}}%
\pgfpathlineto{\pgfqpoint{1.514166in}{2.502270in}}%
\pgfpathclose%
\pgfusepath{stroke,fill}%
\end{pgfscope}%
\begin{pgfscope}%
\pgfpathrectangle{\pgfqpoint{0.887500in}{0.275000in}}{\pgfqpoint{4.225000in}{4.225000in}}%
\pgfusepath{clip}%
\pgfsetbuttcap%
\pgfsetroundjoin%
\definecolor{currentfill}{rgb}{0.168126,0.459988,0.558082}%
\pgfsetfillcolor{currentfill}%
\pgfsetfillopacity{0.700000}%
\pgfsetlinewidth{0.501875pt}%
\definecolor{currentstroke}{rgb}{1.000000,1.000000,1.000000}%
\pgfsetstrokecolor{currentstroke}%
\pgfsetstrokeopacity{0.500000}%
\pgfsetdash{}{0pt}%
\pgfpathmoveto{\pgfqpoint{2.800231in}{2.256064in}}%
\pgfpathlineto{\pgfqpoint{2.811637in}{2.258839in}}%
\pgfpathlineto{\pgfqpoint{2.823045in}{2.260474in}}%
\pgfpathlineto{\pgfqpoint{2.834453in}{2.260745in}}%
\pgfpathlineto{\pgfqpoint{2.845859in}{2.260194in}}%
\pgfpathlineto{\pgfqpoint{2.857255in}{2.259868in}}%
\pgfpathlineto{\pgfqpoint{2.851109in}{2.268909in}}%
\pgfpathlineto{\pgfqpoint{2.844968in}{2.277801in}}%
\pgfpathlineto{\pgfqpoint{2.838831in}{2.286511in}}%
\pgfpathlineto{\pgfqpoint{2.832700in}{2.295005in}}%
\pgfpathlineto{\pgfqpoint{2.826574in}{2.303252in}}%
\pgfpathlineto{\pgfqpoint{2.815184in}{2.303913in}}%
\pgfpathlineto{\pgfqpoint{2.803784in}{2.304796in}}%
\pgfpathlineto{\pgfqpoint{2.792382in}{2.304801in}}%
\pgfpathlineto{\pgfqpoint{2.780983in}{2.303357in}}%
\pgfpathlineto{\pgfqpoint{2.769586in}{2.300702in}}%
\pgfpathlineto{\pgfqpoint{2.775707in}{2.291823in}}%
\pgfpathlineto{\pgfqpoint{2.781832in}{2.282920in}}%
\pgfpathlineto{\pgfqpoint{2.787961in}{2.273994in}}%
\pgfpathlineto{\pgfqpoint{2.794094in}{2.265043in}}%
\pgfpathclose%
\pgfusepath{stroke,fill}%
\end{pgfscope}%
\begin{pgfscope}%
\pgfpathrectangle{\pgfqpoint{0.887500in}{0.275000in}}{\pgfqpoint{4.225000in}{4.225000in}}%
\pgfusepath{clip}%
\pgfsetbuttcap%
\pgfsetroundjoin%
\definecolor{currentfill}{rgb}{0.440137,0.811138,0.340967}%
\pgfsetfillcolor{currentfill}%
\pgfsetfillopacity{0.700000}%
\pgfsetlinewidth{0.501875pt}%
\definecolor{currentstroke}{rgb}{1.000000,1.000000,1.000000}%
\pgfsetstrokecolor{currentstroke}%
\pgfsetstrokeopacity{0.500000}%
\pgfsetdash{}{0pt}%
\pgfpathmoveto{\pgfqpoint{3.480599in}{3.025812in}}%
\pgfpathlineto{\pgfqpoint{3.491834in}{3.029468in}}%
\pgfpathlineto{\pgfqpoint{3.503064in}{3.033099in}}%
\pgfpathlineto{\pgfqpoint{3.514289in}{3.036702in}}%
\pgfpathlineto{\pgfqpoint{3.525507in}{3.040273in}}%
\pgfpathlineto{\pgfqpoint{3.536720in}{3.043809in}}%
\pgfpathlineto{\pgfqpoint{3.530408in}{3.056726in}}%
\pgfpathlineto{\pgfqpoint{3.524097in}{3.069548in}}%
\pgfpathlineto{\pgfqpoint{3.517788in}{3.082262in}}%
\pgfpathlineto{\pgfqpoint{3.511480in}{3.094856in}}%
\pgfpathlineto{\pgfqpoint{3.505173in}{3.107318in}}%
\pgfpathlineto{\pgfqpoint{3.493966in}{3.103880in}}%
\pgfpathlineto{\pgfqpoint{3.482753in}{3.100409in}}%
\pgfpathlineto{\pgfqpoint{3.471534in}{3.096906in}}%
\pgfpathlineto{\pgfqpoint{3.460310in}{3.093375in}}%
\pgfpathlineto{\pgfqpoint{3.449080in}{3.089816in}}%
\pgfpathlineto{\pgfqpoint{3.455379in}{3.077086in}}%
\pgfpathlineto{\pgfqpoint{3.461680in}{3.064322in}}%
\pgfpathlineto{\pgfqpoint{3.467984in}{3.051524in}}%
\pgfpathlineto{\pgfqpoint{3.474290in}{3.038687in}}%
\pgfpathclose%
\pgfusepath{stroke,fill}%
\end{pgfscope}%
\begin{pgfscope}%
\pgfpathrectangle{\pgfqpoint{0.887500in}{0.275000in}}{\pgfqpoint{4.225000in}{4.225000in}}%
\pgfusepath{clip}%
\pgfsetbuttcap%
\pgfsetroundjoin%
\definecolor{currentfill}{rgb}{0.595839,0.848717,0.243329}%
\pgfsetfillcolor{currentfill}%
\pgfsetfillopacity{0.700000}%
\pgfsetlinewidth{0.501875pt}%
\definecolor{currentstroke}{rgb}{1.000000,1.000000,1.000000}%
\pgfsetstrokecolor{currentstroke}%
\pgfsetstrokeopacity{0.500000}%
\pgfsetdash{}{0pt}%
\pgfpathmoveto{\pgfqpoint{3.217260in}{3.154074in}}%
\pgfpathlineto{\pgfqpoint{3.228564in}{3.159223in}}%
\pgfpathlineto{\pgfqpoint{3.239862in}{3.163940in}}%
\pgfpathlineto{\pgfqpoint{3.251154in}{3.168314in}}%
\pgfpathlineto{\pgfqpoint{3.262439in}{3.172435in}}%
\pgfpathlineto{\pgfqpoint{3.273719in}{3.176392in}}%
\pgfpathlineto{\pgfqpoint{3.267454in}{3.187847in}}%
\pgfpathlineto{\pgfqpoint{3.261192in}{3.199131in}}%
\pgfpathlineto{\pgfqpoint{3.254932in}{3.210264in}}%
\pgfpathlineto{\pgfqpoint{3.248675in}{3.221260in}}%
\pgfpathlineto{\pgfqpoint{3.242420in}{3.232138in}}%
\pgfpathlineto{\pgfqpoint{3.231151in}{3.228879in}}%
\pgfpathlineto{\pgfqpoint{3.219875in}{3.225471in}}%
\pgfpathlineto{\pgfqpoint{3.208594in}{3.221776in}}%
\pgfpathlineto{\pgfqpoint{3.197305in}{3.217656in}}%
\pgfpathlineto{\pgfqpoint{3.186010in}{3.212971in}}%
\pgfpathlineto{\pgfqpoint{3.192254in}{3.201223in}}%
\pgfpathlineto{\pgfqpoint{3.198501in}{3.189441in}}%
\pgfpathlineto{\pgfqpoint{3.204751in}{3.177644in}}%
\pgfpathlineto{\pgfqpoint{3.211004in}{3.165849in}}%
\pgfpathclose%
\pgfusepath{stroke,fill}%
\end{pgfscope}%
\begin{pgfscope}%
\pgfpathrectangle{\pgfqpoint{0.887500in}{0.275000in}}{\pgfqpoint{4.225000in}{4.225000in}}%
\pgfusepath{clip}%
\pgfsetbuttcap%
\pgfsetroundjoin%
\definecolor{currentfill}{rgb}{0.133743,0.548535,0.553541}%
\pgfsetfillcolor{currentfill}%
\pgfsetfillopacity{0.700000}%
\pgfsetlinewidth{0.501875pt}%
\definecolor{currentstroke}{rgb}{1.000000,1.000000,1.000000}%
\pgfsetstrokecolor{currentstroke}%
\pgfsetstrokeopacity{0.500000}%
\pgfsetdash{}{0pt}%
\pgfpathmoveto{\pgfqpoint{1.839800in}{2.438255in}}%
\pgfpathlineto{\pgfqpoint{1.851440in}{2.441579in}}%
\pgfpathlineto{\pgfqpoint{1.863074in}{2.444891in}}%
\pgfpathlineto{\pgfqpoint{1.874703in}{2.448194in}}%
\pgfpathlineto{\pgfqpoint{1.886326in}{2.451488in}}%
\pgfpathlineto{\pgfqpoint{1.897944in}{2.454774in}}%
\pgfpathlineto{\pgfqpoint{1.892121in}{2.463031in}}%
\pgfpathlineto{\pgfqpoint{1.886303in}{2.471268in}}%
\pgfpathlineto{\pgfqpoint{1.880489in}{2.479485in}}%
\pgfpathlineto{\pgfqpoint{1.874679in}{2.487683in}}%
\pgfpathlineto{\pgfqpoint{1.868874in}{2.495861in}}%
\pgfpathlineto{\pgfqpoint{1.857268in}{2.492597in}}%
\pgfpathlineto{\pgfqpoint{1.845657in}{2.489325in}}%
\pgfpathlineto{\pgfqpoint{1.834041in}{2.486046in}}%
\pgfpathlineto{\pgfqpoint{1.822419in}{2.482756in}}%
\pgfpathlineto{\pgfqpoint{1.810791in}{2.479455in}}%
\pgfpathlineto{\pgfqpoint{1.816584in}{2.471257in}}%
\pgfpathlineto{\pgfqpoint{1.822382in}{2.463038in}}%
\pgfpathlineto{\pgfqpoint{1.828183in}{2.454799in}}%
\pgfpathlineto{\pgfqpoint{1.833990in}{2.446538in}}%
\pgfpathclose%
\pgfusepath{stroke,fill}%
\end{pgfscope}%
\begin{pgfscope}%
\pgfpathrectangle{\pgfqpoint{0.887500in}{0.275000in}}{\pgfqpoint{4.225000in}{4.225000in}}%
\pgfusepath{clip}%
\pgfsetbuttcap%
\pgfsetroundjoin%
\definecolor{currentfill}{rgb}{0.119512,0.607464,0.540218}%
\pgfsetfillcolor{currentfill}%
\pgfsetfillopacity{0.700000}%
\pgfsetlinewidth{0.501875pt}%
\definecolor{currentstroke}{rgb}{1.000000,1.000000,1.000000}%
\pgfsetstrokecolor{currentstroke}%
\pgfsetstrokeopacity{0.500000}%
\pgfsetdash{}{0pt}%
\pgfpathmoveto{\pgfqpoint{2.939493in}{2.481724in}}%
\pgfpathlineto{\pgfqpoint{2.950764in}{2.512957in}}%
\pgfpathlineto{\pgfqpoint{2.962049in}{2.542885in}}%
\pgfpathlineto{\pgfqpoint{2.973346in}{2.571566in}}%
\pgfpathlineto{\pgfqpoint{2.984653in}{2.599359in}}%
\pgfpathlineto{\pgfqpoint{2.995970in}{2.626624in}}%
\pgfpathlineto{\pgfqpoint{2.989761in}{2.640818in}}%
\pgfpathlineto{\pgfqpoint{2.983553in}{2.655724in}}%
\pgfpathlineto{\pgfqpoint{2.977347in}{2.671005in}}%
\pgfpathlineto{\pgfqpoint{2.971142in}{2.686325in}}%
\pgfpathlineto{\pgfqpoint{2.964940in}{2.701346in}}%
\pgfpathlineto{\pgfqpoint{2.953649in}{2.672426in}}%
\pgfpathlineto{\pgfqpoint{2.942369in}{2.642471in}}%
\pgfpathlineto{\pgfqpoint{2.931103in}{2.611067in}}%
\pgfpathlineto{\pgfqpoint{2.919854in}{2.577803in}}%
\pgfpathlineto{\pgfqpoint{2.908624in}{2.542637in}}%
\pgfpathlineto{\pgfqpoint{2.914793in}{2.530156in}}%
\pgfpathlineto{\pgfqpoint{2.920965in}{2.517644in}}%
\pgfpathlineto{\pgfqpoint{2.927140in}{2.505281in}}%
\pgfpathlineto{\pgfqpoint{2.933316in}{2.493248in}}%
\pgfpathclose%
\pgfusepath{stroke,fill}%
\end{pgfscope}%
\begin{pgfscope}%
\pgfpathrectangle{\pgfqpoint{0.887500in}{0.275000in}}{\pgfqpoint{4.225000in}{4.225000in}}%
\pgfusepath{clip}%
\pgfsetbuttcap%
\pgfsetroundjoin%
\definecolor{currentfill}{rgb}{0.496615,0.826376,0.306377}%
\pgfsetfillcolor{currentfill}%
\pgfsetfillopacity{0.700000}%
\pgfsetlinewidth{0.501875pt}%
\definecolor{currentstroke}{rgb}{1.000000,1.000000,1.000000}%
\pgfsetstrokecolor{currentstroke}%
\pgfsetstrokeopacity{0.500000}%
\pgfsetdash{}{0pt}%
\pgfpathmoveto{\pgfqpoint{3.392847in}{3.071691in}}%
\pgfpathlineto{\pgfqpoint{3.404105in}{3.075347in}}%
\pgfpathlineto{\pgfqpoint{3.415357in}{3.078994in}}%
\pgfpathlineto{\pgfqpoint{3.426603in}{3.082623in}}%
\pgfpathlineto{\pgfqpoint{3.437844in}{3.086231in}}%
\pgfpathlineto{\pgfqpoint{3.449080in}{3.089816in}}%
\pgfpathlineto{\pgfqpoint{3.442783in}{3.102506in}}%
\pgfpathlineto{\pgfqpoint{3.436489in}{3.115134in}}%
\pgfpathlineto{\pgfqpoint{3.430197in}{3.127678in}}%
\pgfpathlineto{\pgfqpoint{3.423906in}{3.140113in}}%
\pgfpathlineto{\pgfqpoint{3.417618in}{3.152418in}}%
\pgfpathlineto{\pgfqpoint{3.406390in}{3.149185in}}%
\pgfpathlineto{\pgfqpoint{3.395157in}{3.145865in}}%
\pgfpathlineto{\pgfqpoint{3.383917in}{3.142451in}}%
\pgfpathlineto{\pgfqpoint{3.372671in}{3.138942in}}%
\pgfpathlineto{\pgfqpoint{3.361419in}{3.135349in}}%
\pgfpathlineto{\pgfqpoint{3.367700in}{3.122807in}}%
\pgfpathlineto{\pgfqpoint{3.373984in}{3.110141in}}%
\pgfpathlineto{\pgfqpoint{3.380269in}{3.097379in}}%
\pgfpathlineto{\pgfqpoint{3.386557in}{3.084552in}}%
\pgfpathclose%
\pgfusepath{stroke,fill}%
\end{pgfscope}%
\begin{pgfscope}%
\pgfpathrectangle{\pgfqpoint{0.887500in}{0.275000in}}{\pgfqpoint{4.225000in}{4.225000in}}%
\pgfusepath{clip}%
\pgfsetbuttcap%
\pgfsetroundjoin%
\definecolor{currentfill}{rgb}{0.575563,0.844566,0.256415}%
\pgfsetfillcolor{currentfill}%
\pgfsetfillopacity{0.700000}%
\pgfsetlinewidth{0.501875pt}%
\definecolor{currentstroke}{rgb}{1.000000,1.000000,1.000000}%
\pgfsetstrokecolor{currentstroke}%
\pgfsetstrokeopacity{0.500000}%
\pgfsetdash{}{0pt}%
\pgfpathmoveto{\pgfqpoint{3.072781in}{3.121767in}}%
\pgfpathlineto{\pgfqpoint{3.084116in}{3.134475in}}%
\pgfpathlineto{\pgfqpoint{3.095450in}{3.146328in}}%
\pgfpathlineto{\pgfqpoint{3.106783in}{3.157456in}}%
\pgfpathlineto{\pgfqpoint{3.118114in}{3.167842in}}%
\pgfpathlineto{\pgfqpoint{3.129441in}{3.177455in}}%
\pgfpathlineto{\pgfqpoint{3.123209in}{3.188789in}}%
\pgfpathlineto{\pgfqpoint{3.116979in}{3.200019in}}%
\pgfpathlineto{\pgfqpoint{3.110752in}{3.211124in}}%
\pgfpathlineto{\pgfqpoint{3.104529in}{3.222077in}}%
\pgfpathlineto{\pgfqpoint{3.098308in}{3.232857in}}%
\pgfpathlineto{\pgfqpoint{3.086991in}{3.221079in}}%
\pgfpathlineto{\pgfqpoint{3.075672in}{3.207730in}}%
\pgfpathlineto{\pgfqpoint{3.064354in}{3.192743in}}%
\pgfpathlineto{\pgfqpoint{3.053036in}{3.176058in}}%
\pgfpathlineto{\pgfqpoint{3.041721in}{3.157681in}}%
\pgfpathlineto{\pgfqpoint{3.047924in}{3.150954in}}%
\pgfpathlineto{\pgfqpoint{3.054132in}{3.144024in}}%
\pgfpathlineto{\pgfqpoint{3.060344in}{3.136867in}}%
\pgfpathlineto{\pgfqpoint{3.066560in}{3.129456in}}%
\pgfpathclose%
\pgfusepath{stroke,fill}%
\end{pgfscope}%
\begin{pgfscope}%
\pgfpathrectangle{\pgfqpoint{0.887500in}{0.275000in}}{\pgfqpoint{4.225000in}{4.225000in}}%
\pgfusepath{clip}%
\pgfsetbuttcap%
\pgfsetroundjoin%
\definecolor{currentfill}{rgb}{0.143343,0.522773,0.556295}%
\pgfsetfillcolor{currentfill}%
\pgfsetfillopacity{0.700000}%
\pgfsetlinewidth{0.501875pt}%
\definecolor{currentstroke}{rgb}{1.000000,1.000000,1.000000}%
\pgfsetstrokecolor{currentstroke}%
\pgfsetstrokeopacity{0.500000}%
\pgfsetdash{}{0pt}%
\pgfpathmoveto{\pgfqpoint{2.159902in}{2.378954in}}%
\pgfpathlineto{\pgfqpoint{2.171462in}{2.382338in}}%
\pgfpathlineto{\pgfqpoint{2.183017in}{2.385713in}}%
\pgfpathlineto{\pgfqpoint{2.194566in}{2.389075in}}%
\pgfpathlineto{\pgfqpoint{2.206110in}{2.392427in}}%
\pgfpathlineto{\pgfqpoint{2.217648in}{2.395766in}}%
\pgfpathlineto{\pgfqpoint{2.211714in}{2.404217in}}%
\pgfpathlineto{\pgfqpoint{2.205784in}{2.412651in}}%
\pgfpathlineto{\pgfqpoint{2.199859in}{2.421068in}}%
\pgfpathlineto{\pgfqpoint{2.193938in}{2.429469in}}%
\pgfpathlineto{\pgfqpoint{2.188021in}{2.437853in}}%
\pgfpathlineto{\pgfqpoint{2.176494in}{2.434538in}}%
\pgfpathlineto{\pgfqpoint{2.164962in}{2.431212in}}%
\pgfpathlineto{\pgfqpoint{2.153424in}{2.427876in}}%
\pgfpathlineto{\pgfqpoint{2.141881in}{2.424528in}}%
\pgfpathlineto{\pgfqpoint{2.130333in}{2.421170in}}%
\pgfpathlineto{\pgfqpoint{2.136238in}{2.412763in}}%
\pgfpathlineto{\pgfqpoint{2.142148in}{2.404338in}}%
\pgfpathlineto{\pgfqpoint{2.148062in}{2.395895in}}%
\pgfpathlineto{\pgfqpoint{2.153980in}{2.387433in}}%
\pgfpathclose%
\pgfusepath{stroke,fill}%
\end{pgfscope}%
\begin{pgfscope}%
\pgfpathrectangle{\pgfqpoint{0.887500in}{0.275000in}}{\pgfqpoint{4.225000in}{4.225000in}}%
\pgfusepath{clip}%
\pgfsetbuttcap%
\pgfsetroundjoin%
\definecolor{currentfill}{rgb}{0.157851,0.683765,0.501686}%
\pgfsetfillcolor{currentfill}%
\pgfsetfillopacity{0.700000}%
\pgfsetlinewidth{0.501875pt}%
\definecolor{currentstroke}{rgb}{1.000000,1.000000,1.000000}%
\pgfsetstrokecolor{currentstroke}%
\pgfsetstrokeopacity{0.500000}%
\pgfsetdash{}{0pt}%
\pgfpathmoveto{\pgfqpoint{3.950647in}{2.708979in}}%
\pgfpathlineto{\pgfqpoint{3.961766in}{2.712411in}}%
\pgfpathlineto{\pgfqpoint{3.972879in}{2.715830in}}%
\pgfpathlineto{\pgfqpoint{3.983987in}{2.719238in}}%
\pgfpathlineto{\pgfqpoint{3.995089in}{2.722637in}}%
\pgfpathlineto{\pgfqpoint{4.006185in}{2.726031in}}%
\pgfpathlineto{\pgfqpoint{3.999795in}{2.739960in}}%
\pgfpathlineto{\pgfqpoint{3.993405in}{2.753844in}}%
\pgfpathlineto{\pgfqpoint{3.987017in}{2.767679in}}%
\pgfpathlineto{\pgfqpoint{3.980631in}{2.781458in}}%
\pgfpathlineto{\pgfqpoint{3.974245in}{2.795187in}}%
\pgfpathlineto{\pgfqpoint{3.963151in}{2.791761in}}%
\pgfpathlineto{\pgfqpoint{3.952050in}{2.788333in}}%
\pgfpathlineto{\pgfqpoint{3.940945in}{2.784898in}}%
\pgfpathlineto{\pgfqpoint{3.929833in}{2.781456in}}%
\pgfpathlineto{\pgfqpoint{3.918717in}{2.778001in}}%
\pgfpathlineto{\pgfqpoint{3.925100in}{2.764289in}}%
\pgfpathlineto{\pgfqpoint{3.931484in}{2.750531in}}%
\pgfpathlineto{\pgfqpoint{3.937870in}{2.736724in}}%
\pgfpathlineto{\pgfqpoint{3.944258in}{2.722871in}}%
\pgfpathclose%
\pgfusepath{stroke,fill}%
\end{pgfscope}%
\begin{pgfscope}%
\pgfpathrectangle{\pgfqpoint{0.887500in}{0.275000in}}{\pgfqpoint{4.225000in}{4.225000in}}%
\pgfusepath{clip}%
\pgfsetbuttcap%
\pgfsetroundjoin%
\definecolor{currentfill}{rgb}{0.154815,0.493313,0.557840}%
\pgfsetfillcolor{currentfill}%
\pgfsetfillopacity{0.700000}%
\pgfsetlinewidth{0.501875pt}%
\definecolor{currentstroke}{rgb}{1.000000,1.000000,1.000000}%
\pgfsetstrokecolor{currentstroke}%
\pgfsetstrokeopacity{0.500000}%
\pgfsetdash{}{0pt}%
\pgfpathmoveto{\pgfqpoint{2.480071in}{2.317268in}}%
\pgfpathlineto{\pgfqpoint{2.491551in}{2.320760in}}%
\pgfpathlineto{\pgfqpoint{2.503025in}{2.324297in}}%
\pgfpathlineto{\pgfqpoint{2.514492in}{2.327871in}}%
\pgfpathlineto{\pgfqpoint{2.525954in}{2.331451in}}%
\pgfpathlineto{\pgfqpoint{2.537411in}{2.335001in}}%
\pgfpathlineto{\pgfqpoint{2.531369in}{2.343658in}}%
\pgfpathlineto{\pgfqpoint{2.525331in}{2.352296in}}%
\pgfpathlineto{\pgfqpoint{2.519297in}{2.360914in}}%
\pgfpathlineto{\pgfqpoint{2.513268in}{2.369513in}}%
\pgfpathlineto{\pgfqpoint{2.507242in}{2.378094in}}%
\pgfpathlineto{\pgfqpoint{2.495797in}{2.374553in}}%
\pgfpathlineto{\pgfqpoint{2.484346in}{2.370983in}}%
\pgfpathlineto{\pgfqpoint{2.472890in}{2.367420in}}%
\pgfpathlineto{\pgfqpoint{2.461428in}{2.363895in}}%
\pgfpathlineto{\pgfqpoint{2.449959in}{2.360418in}}%
\pgfpathlineto{\pgfqpoint{2.455973in}{2.351822in}}%
\pgfpathlineto{\pgfqpoint{2.461992in}{2.343210in}}%
\pgfpathlineto{\pgfqpoint{2.468014in}{2.334580in}}%
\pgfpathlineto{\pgfqpoint{2.474040in}{2.325933in}}%
\pgfpathclose%
\pgfusepath{stroke,fill}%
\end{pgfscope}%
\begin{pgfscope}%
\pgfpathrectangle{\pgfqpoint{0.887500in}{0.275000in}}{\pgfqpoint{4.225000in}{4.225000in}}%
\pgfusepath{clip}%
\pgfsetbuttcap%
\pgfsetroundjoin%
\definecolor{currentfill}{rgb}{0.555484,0.840254,0.269281}%
\pgfsetfillcolor{currentfill}%
\pgfsetfillopacity{0.700000}%
\pgfsetlinewidth{0.501875pt}%
\definecolor{currentstroke}{rgb}{1.000000,1.000000,1.000000}%
\pgfsetstrokecolor{currentstroke}%
\pgfsetstrokeopacity{0.500000}%
\pgfsetdash{}{0pt}%
\pgfpathmoveto{\pgfqpoint{3.305074in}{3.116621in}}%
\pgfpathlineto{\pgfqpoint{3.316354in}{3.120424in}}%
\pgfpathlineto{\pgfqpoint{3.327628in}{3.124211in}}%
\pgfpathlineto{\pgfqpoint{3.338897in}{3.127970in}}%
\pgfpathlineto{\pgfqpoint{3.350161in}{3.131688in}}%
\pgfpathlineto{\pgfqpoint{3.361419in}{3.135349in}}%
\pgfpathlineto{\pgfqpoint{3.355139in}{3.147736in}}%
\pgfpathlineto{\pgfqpoint{3.348860in}{3.159937in}}%
\pgfpathlineto{\pgfqpoint{3.342583in}{3.171923in}}%
\pgfpathlineto{\pgfqpoint{3.336308in}{3.183664in}}%
\pgfpathlineto{\pgfqpoint{3.330033in}{3.195129in}}%
\pgfpathlineto{\pgfqpoint{3.318782in}{3.191544in}}%
\pgfpathlineto{\pgfqpoint{3.307524in}{3.187863in}}%
\pgfpathlineto{\pgfqpoint{3.296261in}{3.184100in}}%
\pgfpathlineto{\pgfqpoint{3.284993in}{3.180272in}}%
\pgfpathlineto{\pgfqpoint{3.273719in}{3.176392in}}%
\pgfpathlineto{\pgfqpoint{3.279986in}{3.164759in}}%
\pgfpathlineto{\pgfqpoint{3.286255in}{3.152956in}}%
\pgfpathlineto{\pgfqpoint{3.292526in}{3.140992in}}%
\pgfpathlineto{\pgfqpoint{3.298799in}{3.128877in}}%
\pgfpathclose%
\pgfusepath{stroke,fill}%
\end{pgfscope}%
\begin{pgfscope}%
\pgfpathrectangle{\pgfqpoint{0.887500in}{0.275000in}}{\pgfqpoint{4.225000in}{4.225000in}}%
\pgfusepath{clip}%
\pgfsetbuttcap%
\pgfsetroundjoin%
\definecolor{currentfill}{rgb}{0.124395,0.578002,0.548287}%
\pgfsetfillcolor{currentfill}%
\pgfsetfillopacity{0.700000}%
\pgfsetlinewidth{0.501875pt}%
\definecolor{currentstroke}{rgb}{1.000000,1.000000,1.000000}%
\pgfsetstrokecolor{currentstroke}%
\pgfsetstrokeopacity{0.500000}%
\pgfsetdash{}{0pt}%
\pgfpathmoveto{\pgfqpoint{4.245202in}{2.477897in}}%
\pgfpathlineto{\pgfqpoint{4.256245in}{2.481216in}}%
\pgfpathlineto{\pgfqpoint{4.267283in}{2.484527in}}%
\pgfpathlineto{\pgfqpoint{4.278315in}{2.487835in}}%
\pgfpathlineto{\pgfqpoint{4.289342in}{2.491141in}}%
\pgfpathlineto{\pgfqpoint{4.300363in}{2.494448in}}%
\pgfpathlineto{\pgfqpoint{4.293927in}{2.508770in}}%
\pgfpathlineto{\pgfqpoint{4.287492in}{2.523078in}}%
\pgfpathlineto{\pgfqpoint{4.281060in}{2.537370in}}%
\pgfpathlineto{\pgfqpoint{4.274630in}{2.551645in}}%
\pgfpathlineto{\pgfqpoint{4.268201in}{2.565903in}}%
\pgfpathlineto{\pgfqpoint{4.257178in}{2.562502in}}%
\pgfpathlineto{\pgfqpoint{4.246150in}{2.559093in}}%
\pgfpathlineto{\pgfqpoint{4.235116in}{2.555679in}}%
\pgfpathlineto{\pgfqpoint{4.224077in}{2.552265in}}%
\pgfpathlineto{\pgfqpoint{4.213033in}{2.548853in}}%
\pgfpathlineto{\pgfqpoint{4.219462in}{2.534673in}}%
\pgfpathlineto{\pgfqpoint{4.225893in}{2.520488in}}%
\pgfpathlineto{\pgfqpoint{4.232327in}{2.506301in}}%
\pgfpathlineto{\pgfqpoint{4.238764in}{2.492107in}}%
\pgfpathclose%
\pgfusepath{stroke,fill}%
\end{pgfscope}%
\begin{pgfscope}%
\pgfpathrectangle{\pgfqpoint{0.887500in}{0.275000in}}{\pgfqpoint{4.225000in}{4.225000in}}%
\pgfusepath{clip}%
\pgfsetbuttcap%
\pgfsetroundjoin%
\definecolor{currentfill}{rgb}{0.172719,0.448791,0.557885}%
\pgfsetfillcolor{currentfill}%
\pgfsetfillopacity{0.700000}%
\pgfsetlinewidth{0.501875pt}%
\definecolor{currentstroke}{rgb}{1.000000,1.000000,1.000000}%
\pgfsetstrokecolor{currentstroke}%
\pgfsetstrokeopacity{0.500000}%
\pgfsetdash{}{0pt}%
\pgfpathmoveto{\pgfqpoint{2.888048in}{2.213564in}}%
\pgfpathlineto{\pgfqpoint{2.899440in}{2.214295in}}%
\pgfpathlineto{\pgfqpoint{2.910816in}{2.217437in}}%
\pgfpathlineto{\pgfqpoint{2.922174in}{2.224082in}}%
\pgfpathlineto{\pgfqpoint{2.933514in}{2.235326in}}%
\pgfpathlineto{\pgfqpoint{2.944836in}{2.252082in}}%
\pgfpathlineto{\pgfqpoint{2.938656in}{2.261456in}}%
\pgfpathlineto{\pgfqpoint{2.932480in}{2.270797in}}%
\pgfpathlineto{\pgfqpoint{2.926308in}{2.280087in}}%
\pgfpathlineto{\pgfqpoint{2.920139in}{2.289305in}}%
\pgfpathlineto{\pgfqpoint{2.913975in}{2.298432in}}%
\pgfpathlineto{\pgfqpoint{2.902672in}{2.281954in}}%
\pgfpathlineto{\pgfqpoint{2.891348in}{2.270812in}}%
\pgfpathlineto{\pgfqpoint{2.880003in}{2.264121in}}%
\pgfpathlineto{\pgfqpoint{2.868638in}{2.260825in}}%
\pgfpathlineto{\pgfqpoint{2.857255in}{2.259868in}}%
\pgfpathlineto{\pgfqpoint{2.863406in}{2.250710in}}%
\pgfpathlineto{\pgfqpoint{2.869561in}{2.241467in}}%
\pgfpathlineto{\pgfqpoint{2.875719in}{2.232174in}}%
\pgfpathlineto{\pgfqpoint{2.881882in}{2.222861in}}%
\pgfpathclose%
\pgfusepath{stroke,fill}%
\end{pgfscope}%
\begin{pgfscope}%
\pgfpathrectangle{\pgfqpoint{0.887500in}{0.275000in}}{\pgfqpoint{4.225000in}{4.225000in}}%
\pgfusepath{clip}%
\pgfsetbuttcap%
\pgfsetroundjoin%
\definecolor{currentfill}{rgb}{0.191090,0.708366,0.482284}%
\pgfsetfillcolor{currentfill}%
\pgfsetfillopacity{0.700000}%
\pgfsetlinewidth{0.501875pt}%
\definecolor{currentstroke}{rgb}{1.000000,1.000000,1.000000}%
\pgfsetstrokecolor{currentstroke}%
\pgfsetstrokeopacity{0.500000}%
\pgfsetdash{}{0pt}%
\pgfpathmoveto{\pgfqpoint{3.863050in}{2.760601in}}%
\pgfpathlineto{\pgfqpoint{3.874194in}{2.764082in}}%
\pgfpathlineto{\pgfqpoint{3.885333in}{2.767568in}}%
\pgfpathlineto{\pgfqpoint{3.896466in}{2.771054in}}%
\pgfpathlineto{\pgfqpoint{3.907594in}{2.774533in}}%
\pgfpathlineto{\pgfqpoint{3.918717in}{2.778001in}}%
\pgfpathlineto{\pgfqpoint{3.912336in}{2.791676in}}%
\pgfpathlineto{\pgfqpoint{3.905957in}{2.805321in}}%
\pgfpathlineto{\pgfqpoint{3.899580in}{2.818946in}}%
\pgfpathlineto{\pgfqpoint{3.893205in}{2.832557in}}%
\pgfpathlineto{\pgfqpoint{3.886833in}{2.846165in}}%
\pgfpathlineto{\pgfqpoint{3.875714in}{2.842712in}}%
\pgfpathlineto{\pgfqpoint{3.864589in}{2.839252in}}%
\pgfpathlineto{\pgfqpoint{3.853459in}{2.835789in}}%
\pgfpathlineto{\pgfqpoint{3.842324in}{2.832327in}}%
\pgfpathlineto{\pgfqpoint{3.831183in}{2.828869in}}%
\pgfpathlineto{\pgfqpoint{3.837552in}{2.815248in}}%
\pgfpathlineto{\pgfqpoint{3.843923in}{2.801614in}}%
\pgfpathlineto{\pgfqpoint{3.850297in}{2.787965in}}%
\pgfpathlineto{\pgfqpoint{3.856672in}{2.774295in}}%
\pgfpathclose%
\pgfusepath{stroke,fill}%
\end{pgfscope}%
\begin{pgfscope}%
\pgfpathrectangle{\pgfqpoint{0.887500in}{0.275000in}}{\pgfqpoint{4.225000in}{4.225000in}}%
\pgfusepath{clip}%
\pgfsetbuttcap%
\pgfsetroundjoin%
\definecolor{currentfill}{rgb}{0.127568,0.566949,0.550556}%
\pgfsetfillcolor{currentfill}%
\pgfsetfillopacity{0.700000}%
\pgfsetlinewidth{0.501875pt}%
\definecolor{currentstroke}{rgb}{1.000000,1.000000,1.000000}%
\pgfsetstrokecolor{currentstroke}%
\pgfsetstrokeopacity{0.500000}%
\pgfsetdash{}{0pt}%
\pgfpathmoveto{\pgfqpoint{1.606986in}{2.470407in}}%
\pgfpathlineto{\pgfqpoint{1.618686in}{2.473747in}}%
\pgfpathlineto{\pgfqpoint{1.630380in}{2.477081in}}%
\pgfpathlineto{\pgfqpoint{1.642069in}{2.480409in}}%
\pgfpathlineto{\pgfqpoint{1.653753in}{2.483731in}}%
\pgfpathlineto{\pgfqpoint{1.665430in}{2.487051in}}%
\pgfpathlineto{\pgfqpoint{1.659688in}{2.495178in}}%
\pgfpathlineto{\pgfqpoint{1.653950in}{2.503291in}}%
\pgfpathlineto{\pgfqpoint{1.648216in}{2.511389in}}%
\pgfpathlineto{\pgfqpoint{1.642487in}{2.519472in}}%
\pgfpathlineto{\pgfqpoint{1.636761in}{2.527538in}}%
\pgfpathlineto{\pgfqpoint{1.625096in}{2.524236in}}%
\pgfpathlineto{\pgfqpoint{1.613426in}{2.520929in}}%
\pgfpathlineto{\pgfqpoint{1.601749in}{2.517617in}}%
\pgfpathlineto{\pgfqpoint{1.590067in}{2.514298in}}%
\pgfpathlineto{\pgfqpoint{1.578380in}{2.510973in}}%
\pgfpathlineto{\pgfqpoint{1.584093in}{2.502890in}}%
\pgfpathlineto{\pgfqpoint{1.589810in}{2.494792in}}%
\pgfpathlineto{\pgfqpoint{1.595531in}{2.486678in}}%
\pgfpathlineto{\pgfqpoint{1.601256in}{2.478550in}}%
\pgfpathclose%
\pgfusepath{stroke,fill}%
\end{pgfscope}%
\begin{pgfscope}%
\pgfpathrectangle{\pgfqpoint{0.887500in}{0.275000in}}{\pgfqpoint{4.225000in}{4.225000in}}%
\pgfusepath{clip}%
\pgfsetbuttcap%
\pgfsetroundjoin%
\definecolor{currentfill}{rgb}{0.165117,0.467423,0.558141}%
\pgfsetfillcolor{currentfill}%
\pgfsetfillopacity{0.700000}%
\pgfsetlinewidth{0.501875pt}%
\definecolor{currentstroke}{rgb}{1.000000,1.000000,1.000000}%
\pgfsetstrokecolor{currentstroke}%
\pgfsetstrokeopacity{0.500000}%
\pgfsetdash{}{0pt}%
\pgfpathmoveto{\pgfqpoint{4.539774in}{2.245885in}}%
\pgfpathlineto{\pgfqpoint{4.550749in}{2.249370in}}%
\pgfpathlineto{\pgfqpoint{4.561718in}{2.252826in}}%
\pgfpathlineto{\pgfqpoint{4.572681in}{2.256261in}}%
\pgfpathlineto{\pgfqpoint{4.583638in}{2.259682in}}%
\pgfpathlineto{\pgfqpoint{4.577147in}{2.274016in}}%
\pgfpathlineto{\pgfqpoint{4.570655in}{2.288279in}}%
\pgfpathlineto{\pgfqpoint{4.564164in}{2.302471in}}%
\pgfpathlineto{\pgfqpoint{4.557673in}{2.316603in}}%
\pgfpathlineto{\pgfqpoint{4.551182in}{2.330683in}}%
\pgfpathlineto{\pgfqpoint{4.540222in}{2.327140in}}%
\pgfpathlineto{\pgfqpoint{4.529256in}{2.323595in}}%
\pgfpathlineto{\pgfqpoint{4.518285in}{2.320044in}}%
\pgfpathlineto{\pgfqpoint{4.507309in}{2.316478in}}%
\pgfpathlineto{\pgfqpoint{4.513801in}{2.302479in}}%
\pgfpathlineto{\pgfqpoint{4.520294in}{2.288432in}}%
\pgfpathlineto{\pgfqpoint{4.526787in}{2.274324in}}%
\pgfpathlineto{\pgfqpoint{4.533281in}{2.260143in}}%
\pgfpathclose%
\pgfusepath{stroke,fill}%
\end{pgfscope}%
\begin{pgfscope}%
\pgfpathrectangle{\pgfqpoint{0.887500in}{0.275000in}}{\pgfqpoint{4.225000in}{4.225000in}}%
\pgfusepath{clip}%
\pgfsetbuttcap%
\pgfsetroundjoin%
\definecolor{currentfill}{rgb}{0.136408,0.541173,0.554483}%
\pgfsetfillcolor{currentfill}%
\pgfsetfillopacity{0.700000}%
\pgfsetlinewidth{0.501875pt}%
\definecolor{currentstroke}{rgb}{1.000000,1.000000,1.000000}%
\pgfsetstrokecolor{currentstroke}%
\pgfsetstrokeopacity{0.500000}%
\pgfsetdash{}{0pt}%
\pgfpathmoveto{\pgfqpoint{1.927121in}{2.413183in}}%
\pgfpathlineto{\pgfqpoint{1.938745in}{2.416490in}}%
\pgfpathlineto{\pgfqpoint{1.950364in}{2.419791in}}%
\pgfpathlineto{\pgfqpoint{1.961976in}{2.423086in}}%
\pgfpathlineto{\pgfqpoint{1.973583in}{2.426377in}}%
\pgfpathlineto{\pgfqpoint{1.985184in}{2.429667in}}%
\pgfpathlineto{\pgfqpoint{1.979328in}{2.438000in}}%
\pgfpathlineto{\pgfqpoint{1.973476in}{2.446314in}}%
\pgfpathlineto{\pgfqpoint{1.967629in}{2.454609in}}%
\pgfpathlineto{\pgfqpoint{1.961785in}{2.462883in}}%
\pgfpathlineto{\pgfqpoint{1.955947in}{2.471139in}}%
\pgfpathlineto{\pgfqpoint{1.944357in}{2.467869in}}%
\pgfpathlineto{\pgfqpoint{1.932763in}{2.464600in}}%
\pgfpathlineto{\pgfqpoint{1.921162in}{2.461329in}}%
\pgfpathlineto{\pgfqpoint{1.909556in}{2.458054in}}%
\pgfpathlineto{\pgfqpoint{1.897944in}{2.454774in}}%
\pgfpathlineto{\pgfqpoint{1.903770in}{2.446497in}}%
\pgfpathlineto{\pgfqpoint{1.909602in}{2.438199in}}%
\pgfpathlineto{\pgfqpoint{1.915437in}{2.429880in}}%
\pgfpathlineto{\pgfqpoint{1.921277in}{2.421542in}}%
\pgfpathclose%
\pgfusepath{stroke,fill}%
\end{pgfscope}%
\begin{pgfscope}%
\pgfpathrectangle{\pgfqpoint{0.887500in}{0.275000in}}{\pgfqpoint{4.225000in}{4.225000in}}%
\pgfusepath{clip}%
\pgfsetbuttcap%
\pgfsetroundjoin%
\definecolor{currentfill}{rgb}{0.119738,0.603785,0.541400}%
\pgfsetfillcolor{currentfill}%
\pgfsetfillopacity{0.700000}%
\pgfsetlinewidth{0.501875pt}%
\definecolor{currentstroke}{rgb}{1.000000,1.000000,1.000000}%
\pgfsetstrokecolor{currentstroke}%
\pgfsetstrokeopacity{0.500000}%
\pgfsetdash{}{0pt}%
\pgfpathmoveto{\pgfqpoint{4.157735in}{2.531934in}}%
\pgfpathlineto{\pgfqpoint{4.168805in}{2.535297in}}%
\pgfpathlineto{\pgfqpoint{4.179870in}{2.538670in}}%
\pgfpathlineto{\pgfqpoint{4.190929in}{2.542053in}}%
\pgfpathlineto{\pgfqpoint{4.201984in}{2.545448in}}%
\pgfpathlineto{\pgfqpoint{4.213033in}{2.548853in}}%
\pgfpathlineto{\pgfqpoint{4.206606in}{2.563027in}}%
\pgfpathlineto{\pgfqpoint{4.200182in}{2.577194in}}%
\pgfpathlineto{\pgfqpoint{4.193760in}{2.591352in}}%
\pgfpathlineto{\pgfqpoint{4.187340in}{2.605499in}}%
\pgfpathlineto{\pgfqpoint{4.180923in}{2.619634in}}%
\pgfpathlineto{\pgfqpoint{4.169874in}{2.616184in}}%
\pgfpathlineto{\pgfqpoint{4.158820in}{2.612743in}}%
\pgfpathlineto{\pgfqpoint{4.147761in}{2.609313in}}%
\pgfpathlineto{\pgfqpoint{4.136697in}{2.605895in}}%
\pgfpathlineto{\pgfqpoint{4.125628in}{2.602489in}}%
\pgfpathlineto{\pgfqpoint{4.132045in}{2.588391in}}%
\pgfpathlineto{\pgfqpoint{4.138464in}{2.574288in}}%
\pgfpathlineto{\pgfqpoint{4.144885in}{2.560177in}}%
\pgfpathlineto{\pgfqpoint{4.151309in}{2.546059in}}%
\pgfpathclose%
\pgfusepath{stroke,fill}%
\end{pgfscope}%
\begin{pgfscope}%
\pgfpathrectangle{\pgfqpoint{0.887500in}{0.275000in}}{\pgfqpoint{4.225000in}{4.225000in}}%
\pgfusepath{clip}%
\pgfsetbuttcap%
\pgfsetroundjoin%
\definecolor{currentfill}{rgb}{0.157729,0.485932,0.558013}%
\pgfsetfillcolor{currentfill}%
\pgfsetfillopacity{0.700000}%
\pgfsetlinewidth{0.501875pt}%
\definecolor{currentstroke}{rgb}{1.000000,1.000000,1.000000}%
\pgfsetstrokecolor{currentstroke}%
\pgfsetstrokeopacity{0.500000}%
\pgfsetdash{}{0pt}%
\pgfpathmoveto{\pgfqpoint{2.567684in}{2.291393in}}%
\pgfpathlineto{\pgfqpoint{2.579147in}{2.294879in}}%
\pgfpathlineto{\pgfqpoint{2.590606in}{2.298270in}}%
\pgfpathlineto{\pgfqpoint{2.602061in}{2.301534in}}%
\pgfpathlineto{\pgfqpoint{2.613512in}{2.304641in}}%
\pgfpathlineto{\pgfqpoint{2.624959in}{2.307614in}}%
\pgfpathlineto{\pgfqpoint{2.618885in}{2.316359in}}%
\pgfpathlineto{\pgfqpoint{2.612816in}{2.325082in}}%
\pgfpathlineto{\pgfqpoint{2.606750in}{2.333784in}}%
\pgfpathlineto{\pgfqpoint{2.600689in}{2.342467in}}%
\pgfpathlineto{\pgfqpoint{2.594631in}{2.351131in}}%
\pgfpathlineto{\pgfqpoint{2.583195in}{2.348196in}}%
\pgfpathlineto{\pgfqpoint{2.571755in}{2.345117in}}%
\pgfpathlineto{\pgfqpoint{2.560311in}{2.341869in}}%
\pgfpathlineto{\pgfqpoint{2.548863in}{2.338485in}}%
\pgfpathlineto{\pgfqpoint{2.537411in}{2.335001in}}%
\pgfpathlineto{\pgfqpoint{2.543457in}{2.326322in}}%
\pgfpathlineto{\pgfqpoint{2.549508in}{2.317623in}}%
\pgfpathlineto{\pgfqpoint{2.555562in}{2.308902in}}%
\pgfpathlineto{\pgfqpoint{2.561621in}{2.300158in}}%
\pgfpathclose%
\pgfusepath{stroke,fill}%
\end{pgfscope}%
\begin{pgfscope}%
\pgfpathrectangle{\pgfqpoint{0.887500in}{0.275000in}}{\pgfqpoint{4.225000in}{4.225000in}}%
\pgfusepath{clip}%
\pgfsetbuttcap%
\pgfsetroundjoin%
\definecolor{currentfill}{rgb}{0.154815,0.493313,0.557840}%
\pgfsetfillcolor{currentfill}%
\pgfsetfillopacity{0.700000}%
\pgfsetlinewidth{0.501875pt}%
\definecolor{currentstroke}{rgb}{1.000000,1.000000,1.000000}%
\pgfsetstrokecolor{currentstroke}%
\pgfsetstrokeopacity{0.500000}%
\pgfsetdash{}{0pt}%
\pgfpathmoveto{\pgfqpoint{4.452330in}{2.298196in}}%
\pgfpathlineto{\pgfqpoint{4.463339in}{2.301935in}}%
\pgfpathlineto{\pgfqpoint{4.474341in}{2.305626in}}%
\pgfpathlineto{\pgfqpoint{4.485337in}{2.309276in}}%
\pgfpathlineto{\pgfqpoint{4.496326in}{2.312891in}}%
\pgfpathlineto{\pgfqpoint{4.507309in}{2.316478in}}%
\pgfpathlineto{\pgfqpoint{4.500818in}{2.330441in}}%
\pgfpathlineto{\pgfqpoint{4.494329in}{2.344382in}}%
\pgfpathlineto{\pgfqpoint{4.487842in}{2.358314in}}%
\pgfpathlineto{\pgfqpoint{4.481359in}{2.372248in}}%
\pgfpathlineto{\pgfqpoint{4.474878in}{2.386199in}}%
\pgfpathlineto{\pgfqpoint{4.463898in}{2.382621in}}%
\pgfpathlineto{\pgfqpoint{4.452912in}{2.379052in}}%
\pgfpathlineto{\pgfqpoint{4.441922in}{2.375490in}}%
\pgfpathlineto{\pgfqpoint{4.430927in}{2.371931in}}%
\pgfpathlineto{\pgfqpoint{4.419926in}{2.368373in}}%
\pgfpathlineto{\pgfqpoint{4.426400in}{2.354306in}}%
\pgfpathlineto{\pgfqpoint{4.432878in}{2.340263in}}%
\pgfpathlineto{\pgfqpoint{4.439360in}{2.326237in}}%
\pgfpathlineto{\pgfqpoint{4.445844in}{2.312217in}}%
\pgfpathclose%
\pgfusepath{stroke,fill}%
\end{pgfscope}%
\begin{pgfscope}%
\pgfpathrectangle{\pgfqpoint{0.887500in}{0.275000in}}{\pgfqpoint{4.225000in}{4.225000in}}%
\pgfusepath{clip}%
\pgfsetbuttcap%
\pgfsetroundjoin%
\definecolor{currentfill}{rgb}{0.147607,0.511733,0.557049}%
\pgfsetfillcolor{currentfill}%
\pgfsetfillopacity{0.700000}%
\pgfsetlinewidth{0.501875pt}%
\definecolor{currentstroke}{rgb}{1.000000,1.000000,1.000000}%
\pgfsetstrokecolor{currentstroke}%
\pgfsetstrokeopacity{0.500000}%
\pgfsetdash{}{0pt}%
\pgfpathmoveto{\pgfqpoint{2.247379in}{2.353246in}}%
\pgfpathlineto{\pgfqpoint{2.258924in}{2.356600in}}%
\pgfpathlineto{\pgfqpoint{2.270462in}{2.359939in}}%
\pgfpathlineto{\pgfqpoint{2.281996in}{2.363263in}}%
\pgfpathlineto{\pgfqpoint{2.293524in}{2.366571in}}%
\pgfpathlineto{\pgfqpoint{2.305046in}{2.369867in}}%
\pgfpathlineto{\pgfqpoint{2.299080in}{2.378385in}}%
\pgfpathlineto{\pgfqpoint{2.293118in}{2.386885in}}%
\pgfpathlineto{\pgfqpoint{2.287160in}{2.395367in}}%
\pgfpathlineto{\pgfqpoint{2.281206in}{2.403833in}}%
\pgfpathlineto{\pgfqpoint{2.275257in}{2.412283in}}%
\pgfpathlineto{\pgfqpoint{2.263746in}{2.409001in}}%
\pgfpathlineto{\pgfqpoint{2.252230in}{2.405711in}}%
\pgfpathlineto{\pgfqpoint{2.240708in}{2.402408in}}%
\pgfpathlineto{\pgfqpoint{2.229181in}{2.399093in}}%
\pgfpathlineto{\pgfqpoint{2.217648in}{2.395766in}}%
\pgfpathlineto{\pgfqpoint{2.223586in}{2.387297in}}%
\pgfpathlineto{\pgfqpoint{2.229528in}{2.378811in}}%
\pgfpathlineto{\pgfqpoint{2.235474in}{2.370308in}}%
\pgfpathlineto{\pgfqpoint{2.241425in}{2.361786in}}%
\pgfpathclose%
\pgfusepath{stroke,fill}%
\end{pgfscope}%
\begin{pgfscope}%
\pgfpathrectangle{\pgfqpoint{0.887500in}{0.275000in}}{\pgfqpoint{4.225000in}{4.225000in}}%
\pgfusepath{clip}%
\pgfsetbuttcap%
\pgfsetroundjoin%
\definecolor{currentfill}{rgb}{0.232815,0.732247,0.459277}%
\pgfsetfillcolor{currentfill}%
\pgfsetfillopacity{0.700000}%
\pgfsetlinewidth{0.501875pt}%
\definecolor{currentstroke}{rgb}{1.000000,1.000000,1.000000}%
\pgfsetstrokecolor{currentstroke}%
\pgfsetstrokeopacity{0.500000}%
\pgfsetdash{}{0pt}%
\pgfpathmoveto{\pgfqpoint{3.775403in}{2.811772in}}%
\pgfpathlineto{\pgfqpoint{3.786570in}{2.815162in}}%
\pgfpathlineto{\pgfqpoint{3.797731in}{2.818566in}}%
\pgfpathlineto{\pgfqpoint{3.808887in}{2.821985in}}%
\pgfpathlineto{\pgfqpoint{3.820038in}{2.825421in}}%
\pgfpathlineto{\pgfqpoint{3.831183in}{2.828869in}}%
\pgfpathlineto{\pgfqpoint{3.824817in}{2.842484in}}%
\pgfpathlineto{\pgfqpoint{3.818453in}{2.856095in}}%
\pgfpathlineto{\pgfqpoint{3.812092in}{2.869708in}}%
\pgfpathlineto{\pgfqpoint{3.805734in}{2.883327in}}%
\pgfpathlineto{\pgfqpoint{3.799378in}{2.896950in}}%
\pgfpathlineto{\pgfqpoint{3.788235in}{2.893446in}}%
\pgfpathlineto{\pgfqpoint{3.777085in}{2.889944in}}%
\pgfpathlineto{\pgfqpoint{3.765931in}{2.886445in}}%
\pgfpathlineto{\pgfqpoint{3.754772in}{2.882954in}}%
\pgfpathlineto{\pgfqpoint{3.743607in}{2.879470in}}%
\pgfpathlineto{\pgfqpoint{3.749962in}{2.865983in}}%
\pgfpathlineto{\pgfqpoint{3.756319in}{2.852470in}}%
\pgfpathlineto{\pgfqpoint{3.762678in}{2.838930in}}%
\pgfpathlineto{\pgfqpoint{3.769040in}{2.825364in}}%
\pgfpathclose%
\pgfusepath{stroke,fill}%
\end{pgfscope}%
\begin{pgfscope}%
\pgfpathrectangle{\pgfqpoint{0.887500in}{0.275000in}}{\pgfqpoint{4.225000in}{4.225000in}}%
\pgfusepath{clip}%
\pgfsetbuttcap%
\pgfsetroundjoin%
\definecolor{currentfill}{rgb}{0.274149,0.751988,0.436601}%
\pgfsetfillcolor{currentfill}%
\pgfsetfillopacity{0.700000}%
\pgfsetlinewidth{0.501875pt}%
\definecolor{currentstroke}{rgb}{1.000000,1.000000,1.000000}%
\pgfsetstrokecolor{currentstroke}%
\pgfsetstrokeopacity{0.500000}%
\pgfsetdash{}{0pt}%
\pgfpathmoveto{\pgfqpoint{3.687706in}{2.862247in}}%
\pgfpathlineto{\pgfqpoint{3.698897in}{2.865660in}}%
\pgfpathlineto{\pgfqpoint{3.710082in}{2.869090in}}%
\pgfpathlineto{\pgfqpoint{3.721262in}{2.872537in}}%
\pgfpathlineto{\pgfqpoint{3.732437in}{2.875997in}}%
\pgfpathlineto{\pgfqpoint{3.743607in}{2.879470in}}%
\pgfpathlineto{\pgfqpoint{3.737254in}{2.892928in}}%
\pgfpathlineto{\pgfqpoint{3.730904in}{2.906356in}}%
\pgfpathlineto{\pgfqpoint{3.724555in}{2.919750in}}%
\pgfpathlineto{\pgfqpoint{3.718209in}{2.933110in}}%
\pgfpathlineto{\pgfqpoint{3.711864in}{2.946433in}}%
\pgfpathlineto{\pgfqpoint{3.700695in}{2.942811in}}%
\pgfpathlineto{\pgfqpoint{3.689522in}{2.939221in}}%
\pgfpathlineto{\pgfqpoint{3.678344in}{2.935671in}}%
\pgfpathlineto{\pgfqpoint{3.667161in}{2.932172in}}%
\pgfpathlineto{\pgfqpoint{3.655974in}{2.928729in}}%
\pgfpathlineto{\pgfqpoint{3.662315in}{2.915471in}}%
\pgfpathlineto{\pgfqpoint{3.668660in}{2.902199in}}%
\pgfpathlineto{\pgfqpoint{3.675006in}{2.888909in}}%
\pgfpathlineto{\pgfqpoint{3.681355in}{2.875593in}}%
\pgfpathclose%
\pgfusepath{stroke,fill}%
\end{pgfscope}%
\begin{pgfscope}%
\pgfpathrectangle{\pgfqpoint{0.887500in}{0.275000in}}{\pgfqpoint{4.225000in}{4.225000in}}%
\pgfusepath{clip}%
\pgfsetbuttcap%
\pgfsetroundjoin%
\definecolor{currentfill}{rgb}{0.121380,0.629492,0.531973}%
\pgfsetfillcolor{currentfill}%
\pgfsetfillopacity{0.700000}%
\pgfsetlinewidth{0.501875pt}%
\definecolor{currentstroke}{rgb}{1.000000,1.000000,1.000000}%
\pgfsetstrokecolor{currentstroke}%
\pgfsetstrokeopacity{0.500000}%
\pgfsetdash{}{0pt}%
\pgfpathmoveto{\pgfqpoint{4.070207in}{2.585640in}}%
\pgfpathlineto{\pgfqpoint{4.081302in}{2.588987in}}%
\pgfpathlineto{\pgfqpoint{4.092391in}{2.592344in}}%
\pgfpathlineto{\pgfqpoint{4.103475in}{2.595713in}}%
\pgfpathlineto{\pgfqpoint{4.114554in}{2.599095in}}%
\pgfpathlineto{\pgfqpoint{4.125628in}{2.602489in}}%
\pgfpathlineto{\pgfqpoint{4.119214in}{2.616582in}}%
\pgfpathlineto{\pgfqpoint{4.112802in}{2.630669in}}%
\pgfpathlineto{\pgfqpoint{4.106392in}{2.644751in}}%
\pgfpathlineto{\pgfqpoint{4.099985in}{2.658829in}}%
\pgfpathlineto{\pgfqpoint{4.093580in}{2.672898in}}%
\pgfpathlineto{\pgfqpoint{4.082508in}{2.669488in}}%
\pgfpathlineto{\pgfqpoint{4.071430in}{2.666087in}}%
\pgfpathlineto{\pgfqpoint{4.060348in}{2.662696in}}%
\pgfpathlineto{\pgfqpoint{4.049260in}{2.659316in}}%
\pgfpathlineto{\pgfqpoint{4.038167in}{2.655945in}}%
\pgfpathlineto{\pgfqpoint{4.044570in}{2.641882in}}%
\pgfpathlineto{\pgfqpoint{4.050975in}{2.627817in}}%
\pgfpathlineto{\pgfqpoint{4.057383in}{2.613756in}}%
\pgfpathlineto{\pgfqpoint{4.063794in}{2.599699in}}%
\pgfpathclose%
\pgfusepath{stroke,fill}%
\end{pgfscope}%
\begin{pgfscope}%
\pgfpathrectangle{\pgfqpoint{0.887500in}{0.275000in}}{\pgfqpoint{4.225000in}{4.225000in}}%
\pgfusepath{clip}%
\pgfsetbuttcap%
\pgfsetroundjoin%
\definecolor{currentfill}{rgb}{0.144759,0.519093,0.556572}%
\pgfsetfillcolor{currentfill}%
\pgfsetfillopacity{0.700000}%
\pgfsetlinewidth{0.501875pt}%
\definecolor{currentstroke}{rgb}{1.000000,1.000000,1.000000}%
\pgfsetstrokecolor{currentstroke}%
\pgfsetstrokeopacity{0.500000}%
\pgfsetdash{}{0pt}%
\pgfpathmoveto{\pgfqpoint{4.364841in}{2.350502in}}%
\pgfpathlineto{\pgfqpoint{4.375869in}{2.354096in}}%
\pgfpathlineto{\pgfqpoint{4.386892in}{2.357678in}}%
\pgfpathlineto{\pgfqpoint{4.397909in}{2.361249in}}%
\pgfpathlineto{\pgfqpoint{4.408920in}{2.364814in}}%
\pgfpathlineto{\pgfqpoint{4.419926in}{2.368373in}}%
\pgfpathlineto{\pgfqpoint{4.413455in}{2.382475in}}%
\pgfpathlineto{\pgfqpoint{4.406989in}{2.396620in}}%
\pgfpathlineto{\pgfqpoint{4.400527in}{2.410814in}}%
\pgfpathlineto{\pgfqpoint{4.394068in}{2.425053in}}%
\pgfpathlineto{\pgfqpoint{4.387614in}{2.439329in}}%
\pgfpathlineto{\pgfqpoint{4.376616in}{2.435948in}}%
\pgfpathlineto{\pgfqpoint{4.365613in}{2.432588in}}%
\pgfpathlineto{\pgfqpoint{4.354605in}{2.429247in}}%
\pgfpathlineto{\pgfqpoint{4.343593in}{2.425926in}}%
\pgfpathlineto{\pgfqpoint{4.332576in}{2.422623in}}%
\pgfpathlineto{\pgfqpoint{4.339024in}{2.408220in}}%
\pgfpathlineto{\pgfqpoint{4.345475in}{2.393805in}}%
\pgfpathlineto{\pgfqpoint{4.351928in}{2.379378in}}%
\pgfpathlineto{\pgfqpoint{4.358384in}{2.364942in}}%
\pgfpathclose%
\pgfusepath{stroke,fill}%
\end{pgfscope}%
\begin{pgfscope}%
\pgfpathrectangle{\pgfqpoint{0.887500in}{0.275000in}}{\pgfqpoint{4.225000in}{4.225000in}}%
\pgfusepath{clip}%
\pgfsetbuttcap%
\pgfsetroundjoin%
\definecolor{currentfill}{rgb}{0.129933,0.559582,0.551864}%
\pgfsetfillcolor{currentfill}%
\pgfsetfillopacity{0.700000}%
\pgfsetlinewidth{0.501875pt}%
\definecolor{currentstroke}{rgb}{1.000000,1.000000,1.000000}%
\pgfsetstrokecolor{currentstroke}%
\pgfsetstrokeopacity{0.500000}%
\pgfsetdash{}{0pt}%
\pgfpathmoveto{\pgfqpoint{1.694206in}{2.446155in}}%
\pgfpathlineto{\pgfqpoint{1.705891in}{2.449489in}}%
\pgfpathlineto{\pgfqpoint{1.717570in}{2.452822in}}%
\pgfpathlineto{\pgfqpoint{1.729242in}{2.456155in}}%
\pgfpathlineto{\pgfqpoint{1.740910in}{2.459490in}}%
\pgfpathlineto{\pgfqpoint{1.752571in}{2.462826in}}%
\pgfpathlineto{\pgfqpoint{1.746794in}{2.471024in}}%
\pgfpathlineto{\pgfqpoint{1.741022in}{2.479202in}}%
\pgfpathlineto{\pgfqpoint{1.735255in}{2.487362in}}%
\pgfpathlineto{\pgfqpoint{1.729491in}{2.495505in}}%
\pgfpathlineto{\pgfqpoint{1.723732in}{2.503631in}}%
\pgfpathlineto{\pgfqpoint{1.712083in}{2.500314in}}%
\pgfpathlineto{\pgfqpoint{1.700429in}{2.496998in}}%
\pgfpathlineto{\pgfqpoint{1.688769in}{2.493683in}}%
\pgfpathlineto{\pgfqpoint{1.677102in}{2.490367in}}%
\pgfpathlineto{\pgfqpoint{1.665430in}{2.487051in}}%
\pgfpathlineto{\pgfqpoint{1.671177in}{2.478907in}}%
\pgfpathlineto{\pgfqpoint{1.676928in}{2.470747in}}%
\pgfpathlineto{\pgfqpoint{1.682683in}{2.462569in}}%
\pgfpathlineto{\pgfqpoint{1.688443in}{2.454372in}}%
\pgfpathclose%
\pgfusepath{stroke,fill}%
\end{pgfscope}%
\begin{pgfscope}%
\pgfpathrectangle{\pgfqpoint{0.887500in}{0.275000in}}{\pgfqpoint{4.225000in}{4.225000in}}%
\pgfusepath{clip}%
\pgfsetbuttcap%
\pgfsetroundjoin%
\definecolor{currentfill}{rgb}{0.162142,0.474838,0.558140}%
\pgfsetfillcolor{currentfill}%
\pgfsetfillopacity{0.700000}%
\pgfsetlinewidth{0.501875pt}%
\definecolor{currentstroke}{rgb}{1.000000,1.000000,1.000000}%
\pgfsetstrokecolor{currentstroke}%
\pgfsetstrokeopacity{0.500000}%
\pgfsetdash{}{0pt}%
\pgfpathmoveto{\pgfqpoint{2.655390in}{2.263536in}}%
\pgfpathlineto{\pgfqpoint{2.666841in}{2.266512in}}%
\pgfpathlineto{\pgfqpoint{2.678286in}{2.269544in}}%
\pgfpathlineto{\pgfqpoint{2.689724in}{2.272722in}}%
\pgfpathlineto{\pgfqpoint{2.701154in}{2.276138in}}%
\pgfpathlineto{\pgfqpoint{2.712575in}{2.279885in}}%
\pgfpathlineto{\pgfqpoint{2.706470in}{2.288747in}}%
\pgfpathlineto{\pgfqpoint{2.700369in}{2.297585in}}%
\pgfpathlineto{\pgfqpoint{2.694272in}{2.306398in}}%
\pgfpathlineto{\pgfqpoint{2.688179in}{2.315187in}}%
\pgfpathlineto{\pgfqpoint{2.682090in}{2.323953in}}%
\pgfpathlineto{\pgfqpoint{2.670680in}{2.320149in}}%
\pgfpathlineto{\pgfqpoint{2.659262in}{2.316714in}}%
\pgfpathlineto{\pgfqpoint{2.647835in}{2.313548in}}%
\pgfpathlineto{\pgfqpoint{2.636400in}{2.310549in}}%
\pgfpathlineto{\pgfqpoint{2.624959in}{2.307614in}}%
\pgfpathlineto{\pgfqpoint{2.631037in}{2.298847in}}%
\pgfpathlineto{\pgfqpoint{2.637119in}{2.290056in}}%
\pgfpathlineto{\pgfqpoint{2.643205in}{2.281242in}}%
\pgfpathlineto{\pgfqpoint{2.649295in}{2.272402in}}%
\pgfpathclose%
\pgfusepath{stroke,fill}%
\end{pgfscope}%
\begin{pgfscope}%
\pgfpathrectangle{\pgfqpoint{0.887500in}{0.275000in}}{\pgfqpoint{4.225000in}{4.225000in}}%
\pgfusepath{clip}%
\pgfsetbuttcap%
\pgfsetroundjoin%
\definecolor{currentfill}{rgb}{0.153364,0.497000,0.557724}%
\pgfsetfillcolor{currentfill}%
\pgfsetfillopacity{0.700000}%
\pgfsetlinewidth{0.501875pt}%
\definecolor{currentstroke}{rgb}{1.000000,1.000000,1.000000}%
\pgfsetstrokecolor{currentstroke}%
\pgfsetstrokeopacity{0.500000}%
\pgfsetdash{}{0pt}%
\pgfpathmoveto{\pgfqpoint{2.944836in}{2.252082in}}%
\pgfpathlineto{\pgfqpoint{2.956149in}{2.273858in}}%
\pgfpathlineto{\pgfqpoint{2.967457in}{2.299518in}}%
\pgfpathlineto{\pgfqpoint{2.978767in}{2.327922in}}%
\pgfpathlineto{\pgfqpoint{2.990083in}{2.357925in}}%
\pgfpathlineto{\pgfqpoint{3.001409in}{2.388379in}}%
\pgfpathlineto{\pgfqpoint{2.995200in}{2.397603in}}%
\pgfpathlineto{\pgfqpoint{2.988996in}{2.406538in}}%
\pgfpathlineto{\pgfqpoint{2.982796in}{2.415302in}}%
\pgfpathlineto{\pgfqpoint{2.976600in}{2.424012in}}%
\pgfpathlineto{\pgfqpoint{2.970408in}{2.432786in}}%
\pgfpathlineto{\pgfqpoint{2.959114in}{2.402547in}}%
\pgfpathlineto{\pgfqpoint{2.947828in}{2.372919in}}%
\pgfpathlineto{\pgfqpoint{2.936547in}{2.344979in}}%
\pgfpathlineto{\pgfqpoint{2.925265in}{2.319794in}}%
\pgfpathlineto{\pgfqpoint{2.913975in}{2.298432in}}%
\pgfpathlineto{\pgfqpoint{2.920139in}{2.289305in}}%
\pgfpathlineto{\pgfqpoint{2.926308in}{2.280087in}}%
\pgfpathlineto{\pgfqpoint{2.932480in}{2.270797in}}%
\pgfpathlineto{\pgfqpoint{2.938656in}{2.261456in}}%
\pgfpathclose%
\pgfusepath{stroke,fill}%
\end{pgfscope}%
\begin{pgfscope}%
\pgfpathrectangle{\pgfqpoint{0.887500in}{0.275000in}}{\pgfqpoint{4.225000in}{4.225000in}}%
\pgfusepath{clip}%
\pgfsetbuttcap%
\pgfsetroundjoin%
\definecolor{currentfill}{rgb}{0.140536,0.530132,0.555659}%
\pgfsetfillcolor{currentfill}%
\pgfsetfillopacity{0.700000}%
\pgfsetlinewidth{0.501875pt}%
\definecolor{currentstroke}{rgb}{1.000000,1.000000,1.000000}%
\pgfsetstrokecolor{currentstroke}%
\pgfsetstrokeopacity{0.500000}%
\pgfsetdash{}{0pt}%
\pgfpathmoveto{\pgfqpoint{2.014529in}{2.387699in}}%
\pgfpathlineto{\pgfqpoint{2.026136in}{2.391013in}}%
\pgfpathlineto{\pgfqpoint{2.037737in}{2.394332in}}%
\pgfpathlineto{\pgfqpoint{2.049333in}{2.397658in}}%
\pgfpathlineto{\pgfqpoint{2.060922in}{2.400993in}}%
\pgfpathlineto{\pgfqpoint{2.072505in}{2.404340in}}%
\pgfpathlineto{\pgfqpoint{2.066616in}{2.412751in}}%
\pgfpathlineto{\pgfqpoint{2.060731in}{2.421143in}}%
\pgfpathlineto{\pgfqpoint{2.054850in}{2.429515in}}%
\pgfpathlineto{\pgfqpoint{2.048973in}{2.437868in}}%
\pgfpathlineto{\pgfqpoint{2.043101in}{2.446202in}}%
\pgfpathlineto{\pgfqpoint{2.031530in}{2.442874in}}%
\pgfpathlineto{\pgfqpoint{2.019952in}{2.439560in}}%
\pgfpathlineto{\pgfqpoint{2.008369in}{2.436255in}}%
\pgfpathlineto{\pgfqpoint{1.996779in}{2.432959in}}%
\pgfpathlineto{\pgfqpoint{1.985184in}{2.429667in}}%
\pgfpathlineto{\pgfqpoint{1.991045in}{2.421313in}}%
\pgfpathlineto{\pgfqpoint{1.996909in}{2.412940in}}%
\pgfpathlineto{\pgfqpoint{2.002778in}{2.404546in}}%
\pgfpathlineto{\pgfqpoint{2.008651in}{2.396133in}}%
\pgfpathclose%
\pgfusepath{stroke,fill}%
\end{pgfscope}%
\begin{pgfscope}%
\pgfpathrectangle{\pgfqpoint{0.887500in}{0.275000in}}{\pgfqpoint{4.225000in}{4.225000in}}%
\pgfusepath{clip}%
\pgfsetbuttcap%
\pgfsetroundjoin%
\definecolor{currentfill}{rgb}{0.344074,0.780029,0.397381}%
\pgfsetfillcolor{currentfill}%
\pgfsetfillopacity{0.700000}%
\pgfsetlinewidth{0.501875pt}%
\definecolor{currentstroke}{rgb}{1.000000,1.000000,1.000000}%
\pgfsetstrokecolor{currentstroke}%
\pgfsetstrokeopacity{0.500000}%
\pgfsetdash{}{0pt}%
\pgfpathmoveto{\pgfqpoint{2.990486in}{2.898469in}}%
\pgfpathlineto{\pgfqpoint{3.001807in}{2.925215in}}%
\pgfpathlineto{\pgfqpoint{3.013137in}{2.950801in}}%
\pgfpathlineto{\pgfqpoint{3.024476in}{2.974756in}}%
\pgfpathlineto{\pgfqpoint{3.035822in}{2.996605in}}%
\pgfpathlineto{\pgfqpoint{3.047173in}{3.015875in}}%
\pgfpathlineto{\pgfqpoint{3.040951in}{3.022309in}}%
\pgfpathlineto{\pgfqpoint{3.034735in}{3.027534in}}%
\pgfpathlineto{\pgfqpoint{3.028524in}{3.031686in}}%
\pgfpathlineto{\pgfqpoint{3.022320in}{3.034900in}}%
\pgfpathlineto{\pgfqpoint{3.016122in}{3.037312in}}%
\pgfpathlineto{\pgfqpoint{3.004804in}{3.014683in}}%
\pgfpathlineto{\pgfqpoint{2.993495in}{2.990117in}}%
\pgfpathlineto{\pgfqpoint{2.982195in}{2.964049in}}%
\pgfpathlineto{\pgfqpoint{2.970904in}{2.936913in}}%
\pgfpathlineto{\pgfqpoint{2.959624in}{2.909142in}}%
\pgfpathlineto{\pgfqpoint{2.965778in}{2.909879in}}%
\pgfpathlineto{\pgfqpoint{2.971941in}{2.909459in}}%
\pgfpathlineto{\pgfqpoint{2.978114in}{2.907604in}}%
\pgfpathlineto{\pgfqpoint{2.984295in}{2.904033in}}%
\pgfpathclose%
\pgfusepath{stroke,fill}%
\end{pgfscope}%
\begin{pgfscope}%
\pgfpathrectangle{\pgfqpoint{0.887500in}{0.275000in}}{\pgfqpoint{4.225000in}{4.225000in}}%
\pgfusepath{clip}%
\pgfsetbuttcap%
\pgfsetroundjoin%
\definecolor{currentfill}{rgb}{0.327796,0.773980,0.406640}%
\pgfsetfillcolor{currentfill}%
\pgfsetfillopacity{0.700000}%
\pgfsetlinewidth{0.501875pt}%
\definecolor{currentstroke}{rgb}{1.000000,1.000000,1.000000}%
\pgfsetstrokecolor{currentstroke}%
\pgfsetstrokeopacity{0.500000}%
\pgfsetdash{}{0pt}%
\pgfpathmoveto{\pgfqpoint{3.599962in}{2.911910in}}%
\pgfpathlineto{\pgfqpoint{3.611175in}{2.915271in}}%
\pgfpathlineto{\pgfqpoint{3.622383in}{2.918619in}}%
\pgfpathlineto{\pgfqpoint{3.633585in}{2.921967in}}%
\pgfpathlineto{\pgfqpoint{3.644782in}{2.925332in}}%
\pgfpathlineto{\pgfqpoint{3.655974in}{2.928729in}}%
\pgfpathlineto{\pgfqpoint{3.649634in}{2.941979in}}%
\pgfpathlineto{\pgfqpoint{3.643298in}{2.955226in}}%
\pgfpathlineto{\pgfqpoint{3.636964in}{2.968477in}}%
\pgfpathlineto{\pgfqpoint{3.630633in}{2.981737in}}%
\pgfpathlineto{\pgfqpoint{3.624305in}{2.995009in}}%
\pgfpathlineto{\pgfqpoint{3.613117in}{2.991637in}}%
\pgfpathlineto{\pgfqpoint{3.601924in}{2.988286in}}%
\pgfpathlineto{\pgfqpoint{3.590726in}{2.984941in}}%
\pgfpathlineto{\pgfqpoint{3.579522in}{2.981590in}}%
\pgfpathlineto{\pgfqpoint{3.568312in}{2.978219in}}%
\pgfpathlineto{\pgfqpoint{3.574637in}{2.964987in}}%
\pgfpathlineto{\pgfqpoint{3.580965in}{2.951740in}}%
\pgfpathlineto{\pgfqpoint{3.587295in}{2.938479in}}%
\pgfpathlineto{\pgfqpoint{3.593627in}{2.925202in}}%
\pgfpathclose%
\pgfusepath{stroke,fill}%
\end{pgfscope}%
\begin{pgfscope}%
\pgfpathrectangle{\pgfqpoint{0.887500in}{0.275000in}}{\pgfqpoint{4.225000in}{4.225000in}}%
\pgfusepath{clip}%
\pgfsetbuttcap%
\pgfsetroundjoin%
\definecolor{currentfill}{rgb}{0.150476,0.504369,0.557430}%
\pgfsetfillcolor{currentfill}%
\pgfsetfillopacity{0.700000}%
\pgfsetlinewidth{0.501875pt}%
\definecolor{currentstroke}{rgb}{1.000000,1.000000,1.000000}%
\pgfsetstrokecolor{currentstroke}%
\pgfsetstrokeopacity{0.500000}%
\pgfsetdash{}{0pt}%
\pgfpathmoveto{\pgfqpoint{2.334940in}{2.327003in}}%
\pgfpathlineto{\pgfqpoint{2.346468in}{2.330311in}}%
\pgfpathlineto{\pgfqpoint{2.357991in}{2.333613in}}%
\pgfpathlineto{\pgfqpoint{2.369508in}{2.336914in}}%
\pgfpathlineto{\pgfqpoint{2.381019in}{2.340220in}}%
\pgfpathlineto{\pgfqpoint{2.392524in}{2.343535in}}%
\pgfpathlineto{\pgfqpoint{2.386525in}{2.352127in}}%
\pgfpathlineto{\pgfqpoint{2.380531in}{2.360702in}}%
\pgfpathlineto{\pgfqpoint{2.374541in}{2.369260in}}%
\pgfpathlineto{\pgfqpoint{2.368555in}{2.377802in}}%
\pgfpathlineto{\pgfqpoint{2.362573in}{2.386326in}}%
\pgfpathlineto{\pgfqpoint{2.351079in}{2.383022in}}%
\pgfpathlineto{\pgfqpoint{2.339580in}{2.379729in}}%
\pgfpathlineto{\pgfqpoint{2.328074in}{2.376442in}}%
\pgfpathlineto{\pgfqpoint{2.316563in}{2.373156in}}%
\pgfpathlineto{\pgfqpoint{2.305046in}{2.369867in}}%
\pgfpathlineto{\pgfqpoint{2.311017in}{2.361332in}}%
\pgfpathlineto{\pgfqpoint{2.316991in}{2.352778in}}%
\pgfpathlineto{\pgfqpoint{2.322970in}{2.344206in}}%
\pgfpathlineto{\pgfqpoint{2.328953in}{2.335614in}}%
\pgfpathclose%
\pgfusepath{stroke,fill}%
\end{pgfscope}%
\begin{pgfscope}%
\pgfpathrectangle{\pgfqpoint{0.887500in}{0.275000in}}{\pgfqpoint{4.225000in}{4.225000in}}%
\pgfusepath{clip}%
\pgfsetbuttcap%
\pgfsetroundjoin%
\definecolor{currentfill}{rgb}{0.575563,0.844566,0.256415}%
\pgfsetfillcolor{currentfill}%
\pgfsetfillopacity{0.700000}%
\pgfsetlinewidth{0.501875pt}%
\definecolor{currentstroke}{rgb}{1.000000,1.000000,1.000000}%
\pgfsetstrokecolor{currentstroke}%
\pgfsetstrokeopacity{0.500000}%
\pgfsetdash{}{0pt}%
\pgfpathmoveto{\pgfqpoint{3.160650in}{3.120088in}}%
\pgfpathlineto{\pgfqpoint{3.171982in}{3.127869in}}%
\pgfpathlineto{\pgfqpoint{3.183310in}{3.135248in}}%
\pgfpathlineto{\pgfqpoint{3.194632in}{3.142125in}}%
\pgfpathlineto{\pgfqpoint{3.205949in}{3.148403in}}%
\pgfpathlineto{\pgfqpoint{3.217260in}{3.154074in}}%
\pgfpathlineto{\pgfqpoint{3.211004in}{3.165849in}}%
\pgfpathlineto{\pgfqpoint{3.204751in}{3.177644in}}%
\pgfpathlineto{\pgfqpoint{3.198501in}{3.189441in}}%
\pgfpathlineto{\pgfqpoint{3.192254in}{3.201223in}}%
\pgfpathlineto{\pgfqpoint{3.186010in}{3.212971in}}%
\pgfpathlineto{\pgfqpoint{3.174708in}{3.207583in}}%
\pgfpathlineto{\pgfqpoint{3.163399in}{3.201358in}}%
\pgfpathlineto{\pgfqpoint{3.152085in}{3.194243in}}%
\pgfpathlineto{\pgfqpoint{3.140765in}{3.186266in}}%
\pgfpathlineto{\pgfqpoint{3.129441in}{3.177455in}}%
\pgfpathlineto{\pgfqpoint{3.135677in}{3.166043in}}%
\pgfpathlineto{\pgfqpoint{3.141915in}{3.154576in}}%
\pgfpathlineto{\pgfqpoint{3.148157in}{3.143079in}}%
\pgfpathlineto{\pgfqpoint{3.154402in}{3.131575in}}%
\pgfpathclose%
\pgfusepath{stroke,fill}%
\end{pgfscope}%
\begin{pgfscope}%
\pgfpathrectangle{\pgfqpoint{0.887500in}{0.275000in}}{\pgfqpoint{4.225000in}{4.225000in}}%
\pgfusepath{clip}%
\pgfsetbuttcap%
\pgfsetroundjoin%
\definecolor{currentfill}{rgb}{0.134692,0.658636,0.517649}%
\pgfsetfillcolor{currentfill}%
\pgfsetfillopacity{0.700000}%
\pgfsetlinewidth{0.501875pt}%
\definecolor{currentstroke}{rgb}{1.000000,1.000000,1.000000}%
\pgfsetstrokecolor{currentstroke}%
\pgfsetstrokeopacity{0.500000}%
\pgfsetdash{}{0pt}%
\pgfpathmoveto{\pgfqpoint{3.982622in}{2.639095in}}%
\pgfpathlineto{\pgfqpoint{3.993742in}{2.642477in}}%
\pgfpathlineto{\pgfqpoint{4.004857in}{2.645849in}}%
\pgfpathlineto{\pgfqpoint{4.015966in}{2.649216in}}%
\pgfpathlineto{\pgfqpoint{4.027069in}{2.652580in}}%
\pgfpathlineto{\pgfqpoint{4.038167in}{2.655945in}}%
\pgfpathlineto{\pgfqpoint{4.031767in}{2.670002in}}%
\pgfpathlineto{\pgfqpoint{4.025368in}{2.684044in}}%
\pgfpathlineto{\pgfqpoint{4.018972in}{2.698067in}}%
\pgfpathlineto{\pgfqpoint{4.012578in}{2.712065in}}%
\pgfpathlineto{\pgfqpoint{4.006185in}{2.726031in}}%
\pgfpathlineto{\pgfqpoint{3.995089in}{2.722637in}}%
\pgfpathlineto{\pgfqpoint{3.983987in}{2.719238in}}%
\pgfpathlineto{\pgfqpoint{3.972879in}{2.715830in}}%
\pgfpathlineto{\pgfqpoint{3.961766in}{2.712411in}}%
\pgfpathlineto{\pgfqpoint{3.950647in}{2.708979in}}%
\pgfpathlineto{\pgfqpoint{3.957038in}{2.695052in}}%
\pgfpathlineto{\pgfqpoint{3.963431in}{2.681095in}}%
\pgfpathlineto{\pgfqpoint{3.969826in}{2.667113in}}%
\pgfpathlineto{\pgfqpoint{3.976223in}{2.653112in}}%
\pgfpathclose%
\pgfusepath{stroke,fill}%
\end{pgfscope}%
\begin{pgfscope}%
\pgfpathrectangle{\pgfqpoint{0.887500in}{0.275000in}}{\pgfqpoint{4.225000in}{4.225000in}}%
\pgfusepath{clip}%
\pgfsetbuttcap%
\pgfsetroundjoin%
\definecolor{currentfill}{rgb}{0.377779,0.791781,0.377939}%
\pgfsetfillcolor{currentfill}%
\pgfsetfillopacity{0.700000}%
\pgfsetlinewidth{0.501875pt}%
\definecolor{currentstroke}{rgb}{1.000000,1.000000,1.000000}%
\pgfsetstrokecolor{currentstroke}%
\pgfsetstrokeopacity{0.500000}%
\pgfsetdash{}{0pt}%
\pgfpathmoveto{\pgfqpoint{3.512176in}{2.960777in}}%
\pgfpathlineto{\pgfqpoint{3.523415in}{2.964351in}}%
\pgfpathlineto{\pgfqpoint{3.534648in}{2.967882in}}%
\pgfpathlineto{\pgfqpoint{3.545875in}{2.971371in}}%
\pgfpathlineto{\pgfqpoint{3.557097in}{2.974817in}}%
\pgfpathlineto{\pgfqpoint{3.568312in}{2.978219in}}%
\pgfpathlineto{\pgfqpoint{3.561989in}{2.991430in}}%
\pgfpathlineto{\pgfqpoint{3.555669in}{3.004607in}}%
\pgfpathlineto{\pgfqpoint{3.549351in}{3.017737in}}%
\pgfpathlineto{\pgfqpoint{3.543035in}{3.030809in}}%
\pgfpathlineto{\pgfqpoint{3.536720in}{3.043809in}}%
\pgfpathlineto{\pgfqpoint{3.525507in}{3.040273in}}%
\pgfpathlineto{\pgfqpoint{3.514289in}{3.036702in}}%
\pgfpathlineto{\pgfqpoint{3.503064in}{3.033099in}}%
\pgfpathlineto{\pgfqpoint{3.491834in}{3.029468in}}%
\pgfpathlineto{\pgfqpoint{3.480599in}{3.025812in}}%
\pgfpathlineto{\pgfqpoint{3.486910in}{3.012895in}}%
\pgfpathlineto{\pgfqpoint{3.493223in}{2.999935in}}%
\pgfpathlineto{\pgfqpoint{3.499538in}{2.986930in}}%
\pgfpathlineto{\pgfqpoint{3.505856in}{2.973878in}}%
\pgfpathclose%
\pgfusepath{stroke,fill}%
\end{pgfscope}%
\begin{pgfscope}%
\pgfpathrectangle{\pgfqpoint{0.887500in}{0.275000in}}{\pgfqpoint{4.225000in}{4.225000in}}%
\pgfusepath{clip}%
\pgfsetbuttcap%
\pgfsetroundjoin%
\definecolor{currentfill}{rgb}{0.196571,0.711827,0.479221}%
\pgfsetfillcolor{currentfill}%
\pgfsetfillopacity{0.700000}%
\pgfsetlinewidth{0.501875pt}%
\definecolor{currentstroke}{rgb}{1.000000,1.000000,1.000000}%
\pgfsetstrokecolor{currentstroke}%
\pgfsetstrokeopacity{0.500000}%
\pgfsetdash{}{0pt}%
\pgfpathmoveto{\pgfqpoint{2.964940in}{2.701346in}}%
\pgfpathlineto{\pgfqpoint{2.976243in}{2.729645in}}%
\pgfpathlineto{\pgfqpoint{2.987554in}{2.757742in}}%
\pgfpathlineto{\pgfqpoint{2.998874in}{2.786055in}}%
\pgfpathlineto{\pgfqpoint{3.010201in}{2.814800in}}%
\pgfpathlineto{\pgfqpoint{3.021538in}{2.843518in}}%
\pgfpathlineto{\pgfqpoint{3.015318in}{2.857307in}}%
\pgfpathlineto{\pgfqpoint{3.009102in}{2.869951in}}%
\pgfpathlineto{\pgfqpoint{3.002891in}{2.881196in}}%
\pgfpathlineto{\pgfqpoint{2.996685in}{2.890787in}}%
\pgfpathlineto{\pgfqpoint{2.990486in}{2.898469in}}%
\pgfpathlineto{\pgfqpoint{2.979174in}{2.871035in}}%
\pgfpathlineto{\pgfqpoint{2.967871in}{2.843383in}}%
\pgfpathlineto{\pgfqpoint{2.956575in}{2.815861in}}%
\pgfpathlineto{\pgfqpoint{2.945288in}{2.788241in}}%
\pgfpathlineto{\pgfqpoint{2.934011in}{2.760117in}}%
\pgfpathlineto{\pgfqpoint{2.940181in}{2.751668in}}%
\pgfpathlineto{\pgfqpoint{2.946361in}{2.741228in}}%
\pgfpathlineto{\pgfqpoint{2.952549in}{2.729135in}}%
\pgfpathlineto{\pgfqpoint{2.958742in}{2.715728in}}%
\pgfpathclose%
\pgfusepath{stroke,fill}%
\end{pgfscope}%
\begin{pgfscope}%
\pgfpathrectangle{\pgfqpoint{0.887500in}{0.275000in}}{\pgfqpoint{4.225000in}{4.225000in}}%
\pgfusepath{clip}%
\pgfsetbuttcap%
\pgfsetroundjoin%
\definecolor{currentfill}{rgb}{0.133743,0.548535,0.553541}%
\pgfsetfillcolor{currentfill}%
\pgfsetfillopacity{0.700000}%
\pgfsetlinewidth{0.501875pt}%
\definecolor{currentstroke}{rgb}{1.000000,1.000000,1.000000}%
\pgfsetstrokecolor{currentstroke}%
\pgfsetstrokeopacity{0.500000}%
\pgfsetdash{}{0pt}%
\pgfpathmoveto{\pgfqpoint{4.277413in}{2.406236in}}%
\pgfpathlineto{\pgfqpoint{4.288456in}{2.409518in}}%
\pgfpathlineto{\pgfqpoint{4.299494in}{2.412791in}}%
\pgfpathlineto{\pgfqpoint{4.310526in}{2.416062in}}%
\pgfpathlineto{\pgfqpoint{4.321554in}{2.419338in}}%
\pgfpathlineto{\pgfqpoint{4.332576in}{2.422623in}}%
\pgfpathlineto{\pgfqpoint{4.326129in}{2.437015in}}%
\pgfpathlineto{\pgfqpoint{4.319684in}{2.451393in}}%
\pgfpathlineto{\pgfqpoint{4.313242in}{2.465759in}}%
\pgfpathlineto{\pgfqpoint{4.306801in}{2.480110in}}%
\pgfpathlineto{\pgfqpoint{4.300363in}{2.494448in}}%
\pgfpathlineto{\pgfqpoint{4.289342in}{2.491141in}}%
\pgfpathlineto{\pgfqpoint{4.278315in}{2.487835in}}%
\pgfpathlineto{\pgfqpoint{4.267283in}{2.484527in}}%
\pgfpathlineto{\pgfqpoint{4.256245in}{2.481216in}}%
\pgfpathlineto{\pgfqpoint{4.245202in}{2.477897in}}%
\pgfpathlineto{\pgfqpoint{4.251642in}{2.463661in}}%
\pgfpathlineto{\pgfqpoint{4.258083in}{2.449387in}}%
\pgfpathlineto{\pgfqpoint{4.264526in}{2.435065in}}%
\pgfpathlineto{\pgfqpoint{4.270969in}{2.420685in}}%
\pgfpathclose%
\pgfusepath{stroke,fill}%
\end{pgfscope}%
\begin{pgfscope}%
\pgfpathrectangle{\pgfqpoint{0.887500in}{0.275000in}}{\pgfqpoint{4.225000in}{4.225000in}}%
\pgfusepath{clip}%
\pgfsetbuttcap%
\pgfsetroundjoin%
\definecolor{currentfill}{rgb}{0.430983,0.808473,0.346476}%
\pgfsetfillcolor{currentfill}%
\pgfsetfillopacity{0.700000}%
\pgfsetlinewidth{0.501875pt}%
\definecolor{currentstroke}{rgb}{1.000000,1.000000,1.000000}%
\pgfsetstrokecolor{currentstroke}%
\pgfsetstrokeopacity{0.500000}%
\pgfsetdash{}{0pt}%
\pgfpathmoveto{\pgfqpoint{3.424339in}{3.007282in}}%
\pgfpathlineto{\pgfqpoint{3.435602in}{3.011012in}}%
\pgfpathlineto{\pgfqpoint{3.446859in}{3.014732in}}%
\pgfpathlineto{\pgfqpoint{3.458111in}{3.018440in}}%
\pgfpathlineto{\pgfqpoint{3.469358in}{3.022135in}}%
\pgfpathlineto{\pgfqpoint{3.480599in}{3.025812in}}%
\pgfpathlineto{\pgfqpoint{3.474290in}{3.038687in}}%
\pgfpathlineto{\pgfqpoint{3.467984in}{3.051524in}}%
\pgfpathlineto{\pgfqpoint{3.461680in}{3.064322in}}%
\pgfpathlineto{\pgfqpoint{3.455379in}{3.077086in}}%
\pgfpathlineto{\pgfqpoint{3.449080in}{3.089816in}}%
\pgfpathlineto{\pgfqpoint{3.437844in}{3.086231in}}%
\pgfpathlineto{\pgfqpoint{3.426603in}{3.082623in}}%
\pgfpathlineto{\pgfqpoint{3.415357in}{3.078994in}}%
\pgfpathlineto{\pgfqpoint{3.404105in}{3.075347in}}%
\pgfpathlineto{\pgfqpoint{3.392847in}{3.071691in}}%
\pgfpathlineto{\pgfqpoint{3.399140in}{3.058822in}}%
\pgfpathlineto{\pgfqpoint{3.405436in}{3.045953in}}%
\pgfpathlineto{\pgfqpoint{3.411735in}{3.033077in}}%
\pgfpathlineto{\pgfqpoint{3.418036in}{3.020188in}}%
\pgfpathclose%
\pgfusepath{stroke,fill}%
\end{pgfscope}%
\begin{pgfscope}%
\pgfpathrectangle{\pgfqpoint{0.887500in}{0.275000in}}{\pgfqpoint{4.225000in}{4.225000in}}%
\pgfusepath{clip}%
\pgfsetbuttcap%
\pgfsetroundjoin%
\definecolor{currentfill}{rgb}{0.545524,0.838039,0.275626}%
\pgfsetfillcolor{currentfill}%
\pgfsetfillopacity{0.700000}%
\pgfsetlinewidth{0.501875pt}%
\definecolor{currentstroke}{rgb}{1.000000,1.000000,1.000000}%
\pgfsetstrokecolor{currentstroke}%
\pgfsetstrokeopacity{0.500000}%
\pgfsetdash{}{0pt}%
\pgfpathmoveto{\pgfqpoint{3.248586in}{3.095072in}}%
\pgfpathlineto{\pgfqpoint{3.259897in}{3.100032in}}%
\pgfpathlineto{\pgfqpoint{3.271201in}{3.104573in}}%
\pgfpathlineto{\pgfqpoint{3.282498in}{3.108788in}}%
\pgfpathlineto{\pgfqpoint{3.293789in}{3.112773in}}%
\pgfpathlineto{\pgfqpoint{3.305074in}{3.116621in}}%
\pgfpathlineto{\pgfqpoint{3.298799in}{3.128877in}}%
\pgfpathlineto{\pgfqpoint{3.292526in}{3.140992in}}%
\pgfpathlineto{\pgfqpoint{3.286255in}{3.152956in}}%
\pgfpathlineto{\pgfqpoint{3.279986in}{3.164759in}}%
\pgfpathlineto{\pgfqpoint{3.273719in}{3.176392in}}%
\pgfpathlineto{\pgfqpoint{3.262439in}{3.172435in}}%
\pgfpathlineto{\pgfqpoint{3.251154in}{3.168314in}}%
\pgfpathlineto{\pgfqpoint{3.239862in}{3.163940in}}%
\pgfpathlineto{\pgfqpoint{3.228564in}{3.159223in}}%
\pgfpathlineto{\pgfqpoint{3.217260in}{3.154074in}}%
\pgfpathlineto{\pgfqpoint{3.223519in}{3.142324in}}%
\pgfpathlineto{\pgfqpoint{3.229782in}{3.130575in}}%
\pgfpathlineto{\pgfqpoint{3.236047in}{3.118802in}}%
\pgfpathlineto{\pgfqpoint{3.242315in}{3.106976in}}%
\pgfpathclose%
\pgfusepath{stroke,fill}%
\end{pgfscope}%
\begin{pgfscope}%
\pgfpathrectangle{\pgfqpoint{0.887500in}{0.275000in}}{\pgfqpoint{4.225000in}{4.225000in}}%
\pgfusepath{clip}%
\pgfsetbuttcap%
\pgfsetroundjoin%
\definecolor{currentfill}{rgb}{0.122606,0.585371,0.546557}%
\pgfsetfillcolor{currentfill}%
\pgfsetfillopacity{0.700000}%
\pgfsetlinewidth{0.501875pt}%
\definecolor{currentstroke}{rgb}{1.000000,1.000000,1.000000}%
\pgfsetstrokecolor{currentstroke}%
\pgfsetstrokeopacity{0.500000}%
\pgfsetdash{}{0pt}%
\pgfpathmoveto{\pgfqpoint{2.970408in}{2.432786in}}%
\pgfpathlineto{\pgfqpoint{2.981713in}{2.462560in}}%
\pgfpathlineto{\pgfqpoint{2.993030in}{2.490970in}}%
\pgfpathlineto{\pgfqpoint{3.004357in}{2.518032in}}%
\pgfpathlineto{\pgfqpoint{3.015693in}{2.544045in}}%
\pgfpathlineto{\pgfqpoint{3.027038in}{2.569310in}}%
\pgfpathlineto{\pgfqpoint{3.020819in}{2.579382in}}%
\pgfpathlineto{\pgfqpoint{3.014604in}{2.589987in}}%
\pgfpathlineto{\pgfqpoint{3.008391in}{2.601289in}}%
\pgfpathlineto{\pgfqpoint{3.002180in}{2.613448in}}%
\pgfpathlineto{\pgfqpoint{2.995970in}{2.626624in}}%
\pgfpathlineto{\pgfqpoint{2.984653in}{2.599359in}}%
\pgfpathlineto{\pgfqpoint{2.973346in}{2.571566in}}%
\pgfpathlineto{\pgfqpoint{2.962049in}{2.542885in}}%
\pgfpathlineto{\pgfqpoint{2.950764in}{2.512957in}}%
\pgfpathlineto{\pgfqpoint{2.939493in}{2.481724in}}%
\pgfpathlineto{\pgfqpoint{2.945672in}{2.470870in}}%
\pgfpathlineto{\pgfqpoint{2.951852in}{2.460666in}}%
\pgfpathlineto{\pgfqpoint{2.958035in}{2.450996in}}%
\pgfpathlineto{\pgfqpoint{2.964220in}{2.441742in}}%
\pgfpathclose%
\pgfusepath{stroke,fill}%
\end{pgfscope}%
\begin{pgfscope}%
\pgfpathrectangle{\pgfqpoint{0.887500in}{0.275000in}}{\pgfqpoint{4.225000in}{4.225000in}}%
\pgfusepath{clip}%
\pgfsetbuttcap%
\pgfsetroundjoin%
\definecolor{currentfill}{rgb}{0.487026,0.823929,0.312321}%
\pgfsetfillcolor{currentfill}%
\pgfsetfillopacity{0.700000}%
\pgfsetlinewidth{0.501875pt}%
\definecolor{currentstroke}{rgb}{1.000000,1.000000,1.000000}%
\pgfsetstrokecolor{currentstroke}%
\pgfsetstrokeopacity{0.500000}%
\pgfsetdash{}{0pt}%
\pgfpathmoveto{\pgfqpoint{3.336483in}{3.053592in}}%
\pgfpathlineto{\pgfqpoint{3.347766in}{3.057158in}}%
\pgfpathlineto{\pgfqpoint{3.359044in}{3.060759in}}%
\pgfpathlineto{\pgfqpoint{3.370317in}{3.064388in}}%
\pgfpathlineto{\pgfqpoint{3.381585in}{3.068035in}}%
\pgfpathlineto{\pgfqpoint{3.392847in}{3.071691in}}%
\pgfpathlineto{\pgfqpoint{3.386557in}{3.084552in}}%
\pgfpathlineto{\pgfqpoint{3.380269in}{3.097379in}}%
\pgfpathlineto{\pgfqpoint{3.373984in}{3.110141in}}%
\pgfpathlineto{\pgfqpoint{3.367700in}{3.122807in}}%
\pgfpathlineto{\pgfqpoint{3.361419in}{3.135349in}}%
\pgfpathlineto{\pgfqpoint{3.350161in}{3.131688in}}%
\pgfpathlineto{\pgfqpoint{3.338897in}{3.127970in}}%
\pgfpathlineto{\pgfqpoint{3.327628in}{3.124211in}}%
\pgfpathlineto{\pgfqpoint{3.316354in}{3.120424in}}%
\pgfpathlineto{\pgfqpoint{3.305074in}{3.116621in}}%
\pgfpathlineto{\pgfqpoint{3.311352in}{3.104236in}}%
\pgfpathlineto{\pgfqpoint{3.317631in}{3.091730in}}%
\pgfpathlineto{\pgfqpoint{3.323913in}{3.079114in}}%
\pgfpathlineto{\pgfqpoint{3.330197in}{3.066398in}}%
\pgfpathclose%
\pgfusepath{stroke,fill}%
\end{pgfscope}%
\begin{pgfscope}%
\pgfpathrectangle{\pgfqpoint{0.887500in}{0.275000in}}{\pgfqpoint{4.225000in}{4.225000in}}%
\pgfusepath{clip}%
\pgfsetbuttcap%
\pgfsetroundjoin%
\definecolor{currentfill}{rgb}{0.133743,0.548535,0.553541}%
\pgfsetfillcolor{currentfill}%
\pgfsetfillopacity{0.700000}%
\pgfsetlinewidth{0.501875pt}%
\definecolor{currentstroke}{rgb}{1.000000,1.000000,1.000000}%
\pgfsetstrokecolor{currentstroke}%
\pgfsetstrokeopacity{0.500000}%
\pgfsetdash{}{0pt}%
\pgfpathmoveto{\pgfqpoint{1.781518in}{2.421518in}}%
\pgfpathlineto{\pgfqpoint{1.793186in}{2.424874in}}%
\pgfpathlineto{\pgfqpoint{1.804848in}{2.428228in}}%
\pgfpathlineto{\pgfqpoint{1.816504in}{2.431578in}}%
\pgfpathlineto{\pgfqpoint{1.828155in}{2.434921in}}%
\pgfpathlineto{\pgfqpoint{1.839800in}{2.438255in}}%
\pgfpathlineto{\pgfqpoint{1.833990in}{2.446538in}}%
\pgfpathlineto{\pgfqpoint{1.828183in}{2.454799in}}%
\pgfpathlineto{\pgfqpoint{1.822382in}{2.463038in}}%
\pgfpathlineto{\pgfqpoint{1.816584in}{2.471257in}}%
\pgfpathlineto{\pgfqpoint{1.810791in}{2.479455in}}%
\pgfpathlineto{\pgfqpoint{1.799158in}{2.476143in}}%
\pgfpathlineto{\pgfqpoint{1.787520in}{2.472822in}}%
\pgfpathlineto{\pgfqpoint{1.775876in}{2.469494in}}%
\pgfpathlineto{\pgfqpoint{1.764226in}{2.466161in}}%
\pgfpathlineto{\pgfqpoint{1.752571in}{2.462826in}}%
\pgfpathlineto{\pgfqpoint{1.758351in}{2.454608in}}%
\pgfpathlineto{\pgfqpoint{1.764137in}{2.446368in}}%
\pgfpathlineto{\pgfqpoint{1.769926in}{2.438108in}}%
\pgfpathlineto{\pgfqpoint{1.775720in}{2.429824in}}%
\pgfpathclose%
\pgfusepath{stroke,fill}%
\end{pgfscope}%
\begin{pgfscope}%
\pgfpathrectangle{\pgfqpoint{0.887500in}{0.275000in}}{\pgfqpoint{4.225000in}{4.225000in}}%
\pgfusepath{clip}%
\pgfsetbuttcap%
\pgfsetroundjoin%
\definecolor{currentfill}{rgb}{0.175841,0.441290,0.557685}%
\pgfsetfillcolor{currentfill}%
\pgfsetfillopacity{0.700000}%
\pgfsetlinewidth{0.501875pt}%
\definecolor{currentstroke}{rgb}{1.000000,1.000000,1.000000}%
\pgfsetstrokecolor{currentstroke}%
\pgfsetstrokeopacity{0.500000}%
\pgfsetdash{}{0pt}%
\pgfpathmoveto{\pgfqpoint{4.572245in}{2.173746in}}%
\pgfpathlineto{\pgfqpoint{4.583218in}{2.177140in}}%
\pgfpathlineto{\pgfqpoint{4.594186in}{2.180518in}}%
\pgfpathlineto{\pgfqpoint{4.605147in}{2.183886in}}%
\pgfpathlineto{\pgfqpoint{4.616103in}{2.187246in}}%
\pgfpathlineto{\pgfqpoint{4.609607in}{2.201806in}}%
\pgfpathlineto{\pgfqpoint{4.603114in}{2.216338in}}%
\pgfpathlineto{\pgfqpoint{4.596621in}{2.230834in}}%
\pgfpathlineto{\pgfqpoint{4.590129in}{2.245285in}}%
\pgfpathlineto{\pgfqpoint{4.583638in}{2.259682in}}%
\pgfpathlineto{\pgfqpoint{4.572681in}{2.256261in}}%
\pgfpathlineto{\pgfqpoint{4.561718in}{2.252826in}}%
\pgfpathlineto{\pgfqpoint{4.550749in}{2.249370in}}%
\pgfpathlineto{\pgfqpoint{4.539774in}{2.245885in}}%
\pgfpathlineto{\pgfqpoint{4.546267in}{2.231558in}}%
\pgfpathlineto{\pgfqpoint{4.552760in}{2.217173in}}%
\pgfpathlineto{\pgfqpoint{4.559254in}{2.202736in}}%
\pgfpathlineto{\pgfqpoint{4.565749in}{2.188258in}}%
\pgfpathclose%
\pgfusepath{stroke,fill}%
\end{pgfscope}%
\begin{pgfscope}%
\pgfpathrectangle{\pgfqpoint{0.887500in}{0.275000in}}{\pgfqpoint{4.225000in}{4.225000in}}%
\pgfusepath{clip}%
\pgfsetbuttcap%
\pgfsetroundjoin%
\definecolor{currentfill}{rgb}{0.166617,0.463708,0.558119}%
\pgfsetfillcolor{currentfill}%
\pgfsetfillopacity{0.700000}%
\pgfsetlinewidth{0.501875pt}%
\definecolor{currentstroke}{rgb}{1.000000,1.000000,1.000000}%
\pgfsetstrokecolor{currentstroke}%
\pgfsetstrokeopacity{0.500000}%
\pgfsetdash{}{0pt}%
\pgfpathmoveto{\pgfqpoint{2.743163in}{2.235199in}}%
\pgfpathlineto{\pgfqpoint{2.754587in}{2.239346in}}%
\pgfpathlineto{\pgfqpoint{2.766004in}{2.243791in}}%
\pgfpathlineto{\pgfqpoint{2.777416in}{2.248251in}}%
\pgfpathlineto{\pgfqpoint{2.788824in}{2.252439in}}%
\pgfpathlineto{\pgfqpoint{2.800231in}{2.256064in}}%
\pgfpathlineto{\pgfqpoint{2.794094in}{2.265043in}}%
\pgfpathlineto{\pgfqpoint{2.787961in}{2.273994in}}%
\pgfpathlineto{\pgfqpoint{2.781832in}{2.282920in}}%
\pgfpathlineto{\pgfqpoint{2.775707in}{2.291823in}}%
\pgfpathlineto{\pgfqpoint{2.769586in}{2.300702in}}%
\pgfpathlineto{\pgfqpoint{2.758190in}{2.297140in}}%
\pgfpathlineto{\pgfqpoint{2.746793in}{2.292973in}}%
\pgfpathlineto{\pgfqpoint{2.735393in}{2.288507in}}%
\pgfpathlineto{\pgfqpoint{2.723988in}{2.284045in}}%
\pgfpathlineto{\pgfqpoint{2.712575in}{2.279885in}}%
\pgfpathlineto{\pgfqpoint{2.718685in}{2.270999in}}%
\pgfpathlineto{\pgfqpoint{2.724798in}{2.262087in}}%
\pgfpathlineto{\pgfqpoint{2.730916in}{2.253150in}}%
\pgfpathlineto{\pgfqpoint{2.737038in}{2.244188in}}%
\pgfpathclose%
\pgfusepath{stroke,fill}%
\end{pgfscope}%
\begin{pgfscope}%
\pgfpathrectangle{\pgfqpoint{0.887500in}{0.275000in}}{\pgfqpoint{4.225000in}{4.225000in}}%
\pgfusepath{clip}%
\pgfsetbuttcap%
\pgfsetroundjoin%
\definecolor{currentfill}{rgb}{0.157851,0.683765,0.501686}%
\pgfsetfillcolor{currentfill}%
\pgfsetfillopacity{0.700000}%
\pgfsetlinewidth{0.501875pt}%
\definecolor{currentstroke}{rgb}{1.000000,1.000000,1.000000}%
\pgfsetstrokecolor{currentstroke}%
\pgfsetstrokeopacity{0.500000}%
\pgfsetdash{}{0pt}%
\pgfpathmoveto{\pgfqpoint{3.894968in}{2.691654in}}%
\pgfpathlineto{\pgfqpoint{3.906115in}{2.695126in}}%
\pgfpathlineto{\pgfqpoint{3.917256in}{2.698600in}}%
\pgfpathlineto{\pgfqpoint{3.928392in}{2.702070in}}%
\pgfpathlineto{\pgfqpoint{3.939522in}{2.705531in}}%
\pgfpathlineto{\pgfqpoint{3.950647in}{2.708979in}}%
\pgfpathlineto{\pgfqpoint{3.944258in}{2.722871in}}%
\pgfpathlineto{\pgfqpoint{3.937870in}{2.736724in}}%
\pgfpathlineto{\pgfqpoint{3.931484in}{2.750531in}}%
\pgfpathlineto{\pgfqpoint{3.925100in}{2.764289in}}%
\pgfpathlineto{\pgfqpoint{3.918717in}{2.778001in}}%
\pgfpathlineto{\pgfqpoint{3.907594in}{2.774533in}}%
\pgfpathlineto{\pgfqpoint{3.896466in}{2.771054in}}%
\pgfpathlineto{\pgfqpoint{3.885333in}{2.767568in}}%
\pgfpathlineto{\pgfqpoint{3.874194in}{2.764082in}}%
\pgfpathlineto{\pgfqpoint{3.863050in}{2.760601in}}%
\pgfpathlineto{\pgfqpoint{3.869430in}{2.746878in}}%
\pgfpathlineto{\pgfqpoint{3.875812in}{2.733122in}}%
\pgfpathlineto{\pgfqpoint{3.882195in}{2.719331in}}%
\pgfpathlineto{\pgfqpoint{3.888581in}{2.705507in}}%
\pgfpathclose%
\pgfusepath{stroke,fill}%
\end{pgfscope}%
\begin{pgfscope}%
\pgfpathrectangle{\pgfqpoint{0.887500in}{0.275000in}}{\pgfqpoint{4.225000in}{4.225000in}}%
\pgfusepath{clip}%
\pgfsetbuttcap%
\pgfsetroundjoin%
\definecolor{currentfill}{rgb}{0.143343,0.522773,0.556295}%
\pgfsetfillcolor{currentfill}%
\pgfsetfillopacity{0.700000}%
\pgfsetlinewidth{0.501875pt}%
\definecolor{currentstroke}{rgb}{1.000000,1.000000,1.000000}%
\pgfsetstrokecolor{currentstroke}%
\pgfsetstrokeopacity{0.500000}%
\pgfsetdash{}{0pt}%
\pgfpathmoveto{\pgfqpoint{2.102016in}{2.361996in}}%
\pgfpathlineto{\pgfqpoint{2.113605in}{2.365381in}}%
\pgfpathlineto{\pgfqpoint{2.125188in}{2.368772in}}%
\pgfpathlineto{\pgfqpoint{2.136765in}{2.372167in}}%
\pgfpathlineto{\pgfqpoint{2.148336in}{2.375562in}}%
\pgfpathlineto{\pgfqpoint{2.159902in}{2.378954in}}%
\pgfpathlineto{\pgfqpoint{2.153980in}{2.387433in}}%
\pgfpathlineto{\pgfqpoint{2.148062in}{2.395895in}}%
\pgfpathlineto{\pgfqpoint{2.142148in}{2.404338in}}%
\pgfpathlineto{\pgfqpoint{2.136238in}{2.412763in}}%
\pgfpathlineto{\pgfqpoint{2.130333in}{2.421170in}}%
\pgfpathlineto{\pgfqpoint{2.118779in}{2.417804in}}%
\pgfpathlineto{\pgfqpoint{2.107219in}{2.414434in}}%
\pgfpathlineto{\pgfqpoint{2.095654in}{2.411065in}}%
\pgfpathlineto{\pgfqpoint{2.084082in}{2.407699in}}%
\pgfpathlineto{\pgfqpoint{2.072505in}{2.404340in}}%
\pgfpathlineto{\pgfqpoint{2.078399in}{2.395910in}}%
\pgfpathlineto{\pgfqpoint{2.084297in}{2.387461in}}%
\pgfpathlineto{\pgfqpoint{2.090199in}{2.378992in}}%
\pgfpathlineto{\pgfqpoint{2.096105in}{2.370504in}}%
\pgfpathclose%
\pgfusepath{stroke,fill}%
\end{pgfscope}%
\begin{pgfscope}%
\pgfpathrectangle{\pgfqpoint{0.887500in}{0.275000in}}{\pgfqpoint{4.225000in}{4.225000in}}%
\pgfusepath{clip}%
\pgfsetbuttcap%
\pgfsetroundjoin%
\definecolor{currentfill}{rgb}{0.154815,0.493313,0.557840}%
\pgfsetfillcolor{currentfill}%
\pgfsetfillopacity{0.700000}%
\pgfsetlinewidth{0.501875pt}%
\definecolor{currentstroke}{rgb}{1.000000,1.000000,1.000000}%
\pgfsetstrokecolor{currentstroke}%
\pgfsetstrokeopacity{0.500000}%
\pgfsetdash{}{0pt}%
\pgfpathmoveto{\pgfqpoint{2.422579in}{2.300291in}}%
\pgfpathlineto{\pgfqpoint{2.434089in}{2.303640in}}%
\pgfpathlineto{\pgfqpoint{2.445594in}{2.307007in}}%
\pgfpathlineto{\pgfqpoint{2.457092in}{2.310397in}}%
\pgfpathlineto{\pgfqpoint{2.468585in}{2.313815in}}%
\pgfpathlineto{\pgfqpoint{2.480071in}{2.317268in}}%
\pgfpathlineto{\pgfqpoint{2.474040in}{2.325933in}}%
\pgfpathlineto{\pgfqpoint{2.468014in}{2.334580in}}%
\pgfpathlineto{\pgfqpoint{2.461992in}{2.343210in}}%
\pgfpathlineto{\pgfqpoint{2.455973in}{2.351822in}}%
\pgfpathlineto{\pgfqpoint{2.449959in}{2.360418in}}%
\pgfpathlineto{\pgfqpoint{2.438485in}{2.356981in}}%
\pgfpathlineto{\pgfqpoint{2.427004in}{2.353580in}}%
\pgfpathlineto{\pgfqpoint{2.415516in}{2.350209in}}%
\pgfpathlineto{\pgfqpoint{2.404023in}{2.346862in}}%
\pgfpathlineto{\pgfqpoint{2.392524in}{2.343535in}}%
\pgfpathlineto{\pgfqpoint{2.398527in}{2.334924in}}%
\pgfpathlineto{\pgfqpoint{2.404533in}{2.326295in}}%
\pgfpathlineto{\pgfqpoint{2.410544in}{2.317646in}}%
\pgfpathlineto{\pgfqpoint{2.416559in}{2.308979in}}%
\pgfpathclose%
\pgfusepath{stroke,fill}%
\end{pgfscope}%
\begin{pgfscope}%
\pgfpathrectangle{\pgfqpoint{0.887500in}{0.275000in}}{\pgfqpoint{4.225000in}{4.225000in}}%
\pgfusepath{clip}%
\pgfsetbuttcap%
\pgfsetroundjoin%
\definecolor{currentfill}{rgb}{0.124395,0.578002,0.548287}%
\pgfsetfillcolor{currentfill}%
\pgfsetfillopacity{0.700000}%
\pgfsetlinewidth{0.501875pt}%
\definecolor{currentstroke}{rgb}{1.000000,1.000000,1.000000}%
\pgfsetstrokecolor{currentstroke}%
\pgfsetstrokeopacity{0.500000}%
\pgfsetdash{}{0pt}%
\pgfpathmoveto{\pgfqpoint{4.189900in}{2.461157in}}%
\pgfpathlineto{\pgfqpoint{4.200972in}{2.464523in}}%
\pgfpathlineto{\pgfqpoint{4.212038in}{2.467882in}}%
\pgfpathlineto{\pgfqpoint{4.223098in}{2.471232in}}%
\pgfpathlineto{\pgfqpoint{4.234153in}{2.474570in}}%
\pgfpathlineto{\pgfqpoint{4.245202in}{2.477897in}}%
\pgfpathlineto{\pgfqpoint{4.238764in}{2.492107in}}%
\pgfpathlineto{\pgfqpoint{4.232327in}{2.506301in}}%
\pgfpathlineto{\pgfqpoint{4.225893in}{2.520488in}}%
\pgfpathlineto{\pgfqpoint{4.219462in}{2.534673in}}%
\pgfpathlineto{\pgfqpoint{4.213033in}{2.548853in}}%
\pgfpathlineto{\pgfqpoint{4.201984in}{2.545448in}}%
\pgfpathlineto{\pgfqpoint{4.190929in}{2.542053in}}%
\pgfpathlineto{\pgfqpoint{4.179870in}{2.538670in}}%
\pgfpathlineto{\pgfqpoint{4.168805in}{2.535297in}}%
\pgfpathlineto{\pgfqpoint{4.157735in}{2.531934in}}%
\pgfpathlineto{\pgfqpoint{4.164164in}{2.517801in}}%
\pgfpathlineto{\pgfqpoint{4.170594in}{2.503659in}}%
\pgfpathlineto{\pgfqpoint{4.177028in}{2.489508in}}%
\pgfpathlineto{\pgfqpoint{4.183463in}{2.475344in}}%
\pgfpathclose%
\pgfusepath{stroke,fill}%
\end{pgfscope}%
\begin{pgfscope}%
\pgfpathrectangle{\pgfqpoint{0.887500in}{0.275000in}}{\pgfqpoint{4.225000in}{4.225000in}}%
\pgfusepath{clip}%
\pgfsetbuttcap%
\pgfsetroundjoin%
\definecolor{currentfill}{rgb}{0.165117,0.467423,0.558141}%
\pgfsetfillcolor{currentfill}%
\pgfsetfillopacity{0.700000}%
\pgfsetlinewidth{0.501875pt}%
\definecolor{currentstroke}{rgb}{1.000000,1.000000,1.000000}%
\pgfsetstrokecolor{currentstroke}%
\pgfsetstrokeopacity{0.500000}%
\pgfsetdash{}{0pt}%
\pgfpathmoveto{\pgfqpoint{4.484791in}{2.227744in}}%
\pgfpathlineto{\pgfqpoint{4.495803in}{2.231494in}}%
\pgfpathlineto{\pgfqpoint{4.506806in}{2.235175in}}%
\pgfpathlineto{\pgfqpoint{4.517803in}{2.238796in}}%
\pgfpathlineto{\pgfqpoint{4.528792in}{2.242363in}}%
\pgfpathlineto{\pgfqpoint{4.539774in}{2.245885in}}%
\pgfpathlineto{\pgfqpoint{4.533281in}{2.260143in}}%
\pgfpathlineto{\pgfqpoint{4.526787in}{2.274324in}}%
\pgfpathlineto{\pgfqpoint{4.520294in}{2.288432in}}%
\pgfpathlineto{\pgfqpoint{4.513801in}{2.302479in}}%
\pgfpathlineto{\pgfqpoint{4.507309in}{2.316478in}}%
\pgfpathlineto{\pgfqpoint{4.496326in}{2.312891in}}%
\pgfpathlineto{\pgfqpoint{4.485337in}{2.309276in}}%
\pgfpathlineto{\pgfqpoint{4.474341in}{2.305626in}}%
\pgfpathlineto{\pgfqpoint{4.463339in}{2.301935in}}%
\pgfpathlineto{\pgfqpoint{4.452330in}{2.298196in}}%
\pgfpathlineto{\pgfqpoint{4.458819in}{2.284163in}}%
\pgfpathlineto{\pgfqpoint{4.465310in}{2.270110in}}%
\pgfpathlineto{\pgfqpoint{4.471803in}{2.256028in}}%
\pgfpathlineto{\pgfqpoint{4.478296in}{2.241907in}}%
\pgfpathclose%
\pgfusepath{stroke,fill}%
\end{pgfscope}%
\begin{pgfscope}%
\pgfpathrectangle{\pgfqpoint{0.887500in}{0.275000in}}{\pgfqpoint{4.225000in}{4.225000in}}%
\pgfusepath{clip}%
\pgfsetbuttcap%
\pgfsetroundjoin%
\definecolor{currentfill}{rgb}{0.191090,0.708366,0.482284}%
\pgfsetfillcolor{currentfill}%
\pgfsetfillopacity{0.700000}%
\pgfsetlinewidth{0.501875pt}%
\definecolor{currentstroke}{rgb}{1.000000,1.000000,1.000000}%
\pgfsetstrokecolor{currentstroke}%
\pgfsetstrokeopacity{0.500000}%
\pgfsetdash{}{0pt}%
\pgfpathmoveto{\pgfqpoint{3.807254in}{2.743426in}}%
\pgfpathlineto{\pgfqpoint{3.818423in}{2.746824in}}%
\pgfpathlineto{\pgfqpoint{3.829588in}{2.750239in}}%
\pgfpathlineto{\pgfqpoint{3.840747in}{2.753674in}}%
\pgfpathlineto{\pgfqpoint{3.851901in}{2.757130in}}%
\pgfpathlineto{\pgfqpoint{3.863050in}{2.760601in}}%
\pgfpathlineto{\pgfqpoint{3.856672in}{2.774295in}}%
\pgfpathlineto{\pgfqpoint{3.850297in}{2.787965in}}%
\pgfpathlineto{\pgfqpoint{3.843923in}{2.801614in}}%
\pgfpathlineto{\pgfqpoint{3.837552in}{2.815248in}}%
\pgfpathlineto{\pgfqpoint{3.831183in}{2.828869in}}%
\pgfpathlineto{\pgfqpoint{3.820038in}{2.825421in}}%
\pgfpathlineto{\pgfqpoint{3.808887in}{2.821985in}}%
\pgfpathlineto{\pgfqpoint{3.797731in}{2.818566in}}%
\pgfpathlineto{\pgfqpoint{3.786570in}{2.815162in}}%
\pgfpathlineto{\pgfqpoint{3.775403in}{2.811772in}}%
\pgfpathlineto{\pgfqpoint{3.781769in}{2.798154in}}%
\pgfpathlineto{\pgfqpoint{3.788137in}{2.784510in}}%
\pgfpathlineto{\pgfqpoint{3.794507in}{2.770841in}}%
\pgfpathlineto{\pgfqpoint{3.800880in}{2.757146in}}%
\pgfpathclose%
\pgfusepath{stroke,fill}%
\end{pgfscope}%
\begin{pgfscope}%
\pgfpathrectangle{\pgfqpoint{0.887500in}{0.275000in}}{\pgfqpoint{4.225000in}{4.225000in}}%
\pgfusepath{clip}%
\pgfsetbuttcap%
\pgfsetroundjoin%
\definecolor{currentfill}{rgb}{0.535621,0.835785,0.281908}%
\pgfsetfillcolor{currentfill}%
\pgfsetfillopacity{0.700000}%
\pgfsetlinewidth{0.501875pt}%
\definecolor{currentstroke}{rgb}{1.000000,1.000000,1.000000}%
\pgfsetstrokecolor{currentstroke}%
\pgfsetstrokeopacity{0.500000}%
\pgfsetdash{}{0pt}%
\pgfpathmoveto{\pgfqpoint{3.103942in}{3.078270in}}%
\pgfpathlineto{\pgfqpoint{3.115290in}{3.086934in}}%
\pgfpathlineto{\pgfqpoint{3.126634in}{3.095341in}}%
\pgfpathlineto{\pgfqpoint{3.137976in}{3.103724in}}%
\pgfpathlineto{\pgfqpoint{3.149315in}{3.112006in}}%
\pgfpathlineto{\pgfqpoint{3.160650in}{3.120088in}}%
\pgfpathlineto{\pgfqpoint{3.154402in}{3.131575in}}%
\pgfpathlineto{\pgfqpoint{3.148157in}{3.143079in}}%
\pgfpathlineto{\pgfqpoint{3.141915in}{3.154576in}}%
\pgfpathlineto{\pgfqpoint{3.135677in}{3.166043in}}%
\pgfpathlineto{\pgfqpoint{3.129441in}{3.177455in}}%
\pgfpathlineto{\pgfqpoint{3.118114in}{3.167842in}}%
\pgfpathlineto{\pgfqpoint{3.106783in}{3.157456in}}%
\pgfpathlineto{\pgfqpoint{3.095450in}{3.146328in}}%
\pgfpathlineto{\pgfqpoint{3.084116in}{3.134475in}}%
\pgfpathlineto{\pgfqpoint{3.072781in}{3.121767in}}%
\pgfpathlineto{\pgfqpoint{3.079005in}{3.113775in}}%
\pgfpathlineto{\pgfqpoint{3.085234in}{3.105454in}}%
\pgfpathlineto{\pgfqpoint{3.091467in}{3.096780in}}%
\pgfpathlineto{\pgfqpoint{3.097703in}{3.087727in}}%
\pgfpathclose%
\pgfusepath{stroke,fill}%
\end{pgfscope}%
\begin{pgfscope}%
\pgfpathrectangle{\pgfqpoint{0.887500in}{0.275000in}}{\pgfqpoint{4.225000in}{4.225000in}}%
\pgfusepath{clip}%
\pgfsetbuttcap%
\pgfsetroundjoin%
\definecolor{currentfill}{rgb}{0.172719,0.448791,0.557885}%
\pgfsetfillcolor{currentfill}%
\pgfsetfillopacity{0.700000}%
\pgfsetlinewidth{0.501875pt}%
\definecolor{currentstroke}{rgb}{1.000000,1.000000,1.000000}%
\pgfsetstrokecolor{currentstroke}%
\pgfsetstrokeopacity{0.500000}%
\pgfsetdash{}{0pt}%
\pgfpathmoveto{\pgfqpoint{2.830977in}{2.210719in}}%
\pgfpathlineto{\pgfqpoint{2.842394in}{2.213402in}}%
\pgfpathlineto{\pgfqpoint{2.853812in}{2.214891in}}%
\pgfpathlineto{\pgfqpoint{2.865229in}{2.214953in}}%
\pgfpathlineto{\pgfqpoint{2.876643in}{2.214148in}}%
\pgfpathlineto{\pgfqpoint{2.888048in}{2.213564in}}%
\pgfpathlineto{\pgfqpoint{2.881882in}{2.222861in}}%
\pgfpathlineto{\pgfqpoint{2.875719in}{2.232174in}}%
\pgfpathlineto{\pgfqpoint{2.869561in}{2.241467in}}%
\pgfpathlineto{\pgfqpoint{2.863406in}{2.250710in}}%
\pgfpathlineto{\pgfqpoint{2.857255in}{2.259868in}}%
\pgfpathlineto{\pgfqpoint{2.845859in}{2.260194in}}%
\pgfpathlineto{\pgfqpoint{2.834453in}{2.260745in}}%
\pgfpathlineto{\pgfqpoint{2.823045in}{2.260474in}}%
\pgfpathlineto{\pgfqpoint{2.811637in}{2.258839in}}%
\pgfpathlineto{\pgfqpoint{2.800231in}{2.256064in}}%
\pgfpathlineto{\pgfqpoint{2.806372in}{2.247058in}}%
\pgfpathlineto{\pgfqpoint{2.812517in}{2.238021in}}%
\pgfpathlineto{\pgfqpoint{2.818666in}{2.228954in}}%
\pgfpathlineto{\pgfqpoint{2.824820in}{2.219853in}}%
\pgfpathclose%
\pgfusepath{stroke,fill}%
\end{pgfscope}%
\begin{pgfscope}%
\pgfpathrectangle{\pgfqpoint{0.887500in}{0.275000in}}{\pgfqpoint{4.225000in}{4.225000in}}%
\pgfusepath{clip}%
\pgfsetbuttcap%
\pgfsetroundjoin%
\definecolor{currentfill}{rgb}{0.127568,0.566949,0.550556}%
\pgfsetfillcolor{currentfill}%
\pgfsetfillopacity{0.700000}%
\pgfsetlinewidth{0.501875pt}%
\definecolor{currentstroke}{rgb}{1.000000,1.000000,1.000000}%
\pgfsetstrokecolor{currentstroke}%
\pgfsetstrokeopacity{0.500000}%
\pgfsetdash{}{0pt}%
\pgfpathmoveto{\pgfqpoint{1.548403in}{2.453555in}}%
\pgfpathlineto{\pgfqpoint{1.560130in}{2.456949in}}%
\pgfpathlineto{\pgfqpoint{1.571852in}{2.460330in}}%
\pgfpathlineto{\pgfqpoint{1.583569in}{2.463699in}}%
\pgfpathlineto{\pgfqpoint{1.595280in}{2.467058in}}%
\pgfpathlineto{\pgfqpoint{1.606986in}{2.470407in}}%
\pgfpathlineto{\pgfqpoint{1.601256in}{2.478550in}}%
\pgfpathlineto{\pgfqpoint{1.595531in}{2.486678in}}%
\pgfpathlineto{\pgfqpoint{1.589810in}{2.494792in}}%
\pgfpathlineto{\pgfqpoint{1.584093in}{2.502890in}}%
\pgfpathlineto{\pgfqpoint{1.578380in}{2.510973in}}%
\pgfpathlineto{\pgfqpoint{1.566687in}{2.507638in}}%
\pgfpathlineto{\pgfqpoint{1.554989in}{2.504294in}}%
\pgfpathlineto{\pgfqpoint{1.543285in}{2.500939in}}%
\pgfpathlineto{\pgfqpoint{1.531576in}{2.497572in}}%
\pgfpathlineto{\pgfqpoint{1.519862in}{2.494193in}}%
\pgfpathlineto{\pgfqpoint{1.525561in}{2.486096in}}%
\pgfpathlineto{\pgfqpoint{1.531265in}{2.477983in}}%
\pgfpathlineto{\pgfqpoint{1.536974in}{2.469855in}}%
\pgfpathlineto{\pgfqpoint{1.542686in}{2.461713in}}%
\pgfpathclose%
\pgfusepath{stroke,fill}%
\end{pgfscope}%
\begin{pgfscope}%
\pgfpathrectangle{\pgfqpoint{0.887500in}{0.275000in}}{\pgfqpoint{4.225000in}{4.225000in}}%
\pgfusepath{clip}%
\pgfsetbuttcap%
\pgfsetroundjoin%
\definecolor{currentfill}{rgb}{0.468053,0.818921,0.323998}%
\pgfsetfillcolor{currentfill}%
\pgfsetfillopacity{0.700000}%
\pgfsetlinewidth{0.501875pt}%
\definecolor{currentstroke}{rgb}{1.000000,1.000000,1.000000}%
\pgfsetstrokecolor{currentstroke}%
\pgfsetstrokeopacity{0.500000}%
\pgfsetdash{}{0pt}%
\pgfpathmoveto{\pgfqpoint{3.047173in}{3.015875in}}%
\pgfpathlineto{\pgfqpoint{3.058528in}{3.032353in}}%
\pgfpathlineto{\pgfqpoint{3.069884in}{3.046395in}}%
\pgfpathlineto{\pgfqpoint{3.081239in}{3.058438in}}%
\pgfpathlineto{\pgfqpoint{3.092592in}{3.068918in}}%
\pgfpathlineto{\pgfqpoint{3.103942in}{3.078270in}}%
\pgfpathlineto{\pgfqpoint{3.097703in}{3.087727in}}%
\pgfpathlineto{\pgfqpoint{3.091467in}{3.096780in}}%
\pgfpathlineto{\pgfqpoint{3.085234in}{3.105454in}}%
\pgfpathlineto{\pgfqpoint{3.079005in}{3.113775in}}%
\pgfpathlineto{\pgfqpoint{3.072781in}{3.121767in}}%
\pgfpathlineto{\pgfqpoint{3.061445in}{3.107973in}}%
\pgfpathlineto{\pgfqpoint{3.050110in}{3.092861in}}%
\pgfpathlineto{\pgfqpoint{3.038777in}{3.076200in}}%
\pgfpathlineto{\pgfqpoint{3.027447in}{3.057760in}}%
\pgfpathlineto{\pgfqpoint{3.016122in}{3.037312in}}%
\pgfpathlineto{\pgfqpoint{3.022320in}{3.034900in}}%
\pgfpathlineto{\pgfqpoint{3.028524in}{3.031686in}}%
\pgfpathlineto{\pgfqpoint{3.034735in}{3.027534in}}%
\pgfpathlineto{\pgfqpoint{3.040951in}{3.022309in}}%
\pgfpathclose%
\pgfusepath{stroke,fill}%
\end{pgfscope}%
\begin{pgfscope}%
\pgfpathrectangle{\pgfqpoint{0.887500in}{0.275000in}}{\pgfqpoint{4.225000in}{4.225000in}}%
\pgfusepath{clip}%
\pgfsetbuttcap%
\pgfsetroundjoin%
\definecolor{currentfill}{rgb}{0.136408,0.541173,0.554483}%
\pgfsetfillcolor{currentfill}%
\pgfsetfillopacity{0.700000}%
\pgfsetlinewidth{0.501875pt}%
\definecolor{currentstroke}{rgb}{1.000000,1.000000,1.000000}%
\pgfsetstrokecolor{currentstroke}%
\pgfsetstrokeopacity{0.500000}%
\pgfsetdash{}{0pt}%
\pgfpathmoveto{\pgfqpoint{1.868919in}{2.396519in}}%
\pgfpathlineto{\pgfqpoint{1.880570in}{2.399871in}}%
\pgfpathlineto{\pgfqpoint{1.892216in}{2.403213in}}%
\pgfpathlineto{\pgfqpoint{1.903857in}{2.406545in}}%
\pgfpathlineto{\pgfqpoint{1.915492in}{2.409868in}}%
\pgfpathlineto{\pgfqpoint{1.927121in}{2.413183in}}%
\pgfpathlineto{\pgfqpoint{1.921277in}{2.421542in}}%
\pgfpathlineto{\pgfqpoint{1.915437in}{2.429880in}}%
\pgfpathlineto{\pgfqpoint{1.909602in}{2.438199in}}%
\pgfpathlineto{\pgfqpoint{1.903770in}{2.446497in}}%
\pgfpathlineto{\pgfqpoint{1.897944in}{2.454774in}}%
\pgfpathlineto{\pgfqpoint{1.886326in}{2.451488in}}%
\pgfpathlineto{\pgfqpoint{1.874703in}{2.448194in}}%
\pgfpathlineto{\pgfqpoint{1.863074in}{2.444891in}}%
\pgfpathlineto{\pgfqpoint{1.851440in}{2.441579in}}%
\pgfpathlineto{\pgfqpoint{1.839800in}{2.438255in}}%
\pgfpathlineto{\pgfqpoint{1.845615in}{2.429950in}}%
\pgfpathlineto{\pgfqpoint{1.851435in}{2.421624in}}%
\pgfpathlineto{\pgfqpoint{1.857258in}{2.413277in}}%
\pgfpathlineto{\pgfqpoint{1.863086in}{2.404908in}}%
\pgfpathclose%
\pgfusepath{stroke,fill}%
\end{pgfscope}%
\begin{pgfscope}%
\pgfpathrectangle{\pgfqpoint{0.887500in}{0.275000in}}{\pgfqpoint{4.225000in}{4.225000in}}%
\pgfusepath{clip}%
\pgfsetbuttcap%
\pgfsetroundjoin%
\definecolor{currentfill}{rgb}{0.154815,0.493313,0.557840}%
\pgfsetfillcolor{currentfill}%
\pgfsetfillopacity{0.700000}%
\pgfsetlinewidth{0.501875pt}%
\definecolor{currentstroke}{rgb}{1.000000,1.000000,1.000000}%
\pgfsetstrokecolor{currentstroke}%
\pgfsetstrokeopacity{0.500000}%
\pgfsetdash{}{0pt}%
\pgfpathmoveto{\pgfqpoint{4.397175in}{2.278535in}}%
\pgfpathlineto{\pgfqpoint{4.408221in}{2.282618in}}%
\pgfpathlineto{\pgfqpoint{4.419260in}{2.286619in}}%
\pgfpathlineto{\pgfqpoint{4.430291in}{2.290545in}}%
\pgfpathlineto{\pgfqpoint{4.441314in}{2.294401in}}%
\pgfpathlineto{\pgfqpoint{4.452330in}{2.298196in}}%
\pgfpathlineto{\pgfqpoint{4.445844in}{2.312217in}}%
\pgfpathlineto{\pgfqpoint{4.439360in}{2.326237in}}%
\pgfpathlineto{\pgfqpoint{4.432878in}{2.340263in}}%
\pgfpathlineto{\pgfqpoint{4.426400in}{2.354306in}}%
\pgfpathlineto{\pgfqpoint{4.419926in}{2.368373in}}%
\pgfpathlineto{\pgfqpoint{4.408920in}{2.364814in}}%
\pgfpathlineto{\pgfqpoint{4.397909in}{2.361249in}}%
\pgfpathlineto{\pgfqpoint{4.386892in}{2.357678in}}%
\pgfpathlineto{\pgfqpoint{4.375869in}{2.354096in}}%
\pgfpathlineto{\pgfqpoint{4.364841in}{2.350502in}}%
\pgfpathlineto{\pgfqpoint{4.371301in}{2.336067in}}%
\pgfpathlineto{\pgfqpoint{4.377765in}{2.321645in}}%
\pgfpathlineto{\pgfqpoint{4.384231in}{2.307243in}}%
\pgfpathlineto{\pgfqpoint{4.390701in}{2.292871in}}%
\pgfpathclose%
\pgfusepath{stroke,fill}%
\end{pgfscope}%
\begin{pgfscope}%
\pgfpathrectangle{\pgfqpoint{0.887500in}{0.275000in}}{\pgfqpoint{4.225000in}{4.225000in}}%
\pgfusepath{clip}%
\pgfsetbuttcap%
\pgfsetroundjoin%
\definecolor{currentfill}{rgb}{0.119738,0.603785,0.541400}%
\pgfsetfillcolor{currentfill}%
\pgfsetfillopacity{0.700000}%
\pgfsetlinewidth{0.501875pt}%
\definecolor{currentstroke}{rgb}{1.000000,1.000000,1.000000}%
\pgfsetstrokecolor{currentstroke}%
\pgfsetstrokeopacity{0.500000}%
\pgfsetdash{}{0pt}%
\pgfpathmoveto{\pgfqpoint{4.102308in}{2.515230in}}%
\pgfpathlineto{\pgfqpoint{4.113404in}{2.518560in}}%
\pgfpathlineto{\pgfqpoint{4.124495in}{2.521894in}}%
\pgfpathlineto{\pgfqpoint{4.135580in}{2.525234in}}%
\pgfpathlineto{\pgfqpoint{4.146660in}{2.528580in}}%
\pgfpathlineto{\pgfqpoint{4.157735in}{2.531934in}}%
\pgfpathlineto{\pgfqpoint{4.151309in}{2.546059in}}%
\pgfpathlineto{\pgfqpoint{4.144885in}{2.560177in}}%
\pgfpathlineto{\pgfqpoint{4.138464in}{2.574288in}}%
\pgfpathlineto{\pgfqpoint{4.132045in}{2.588391in}}%
\pgfpathlineto{\pgfqpoint{4.125628in}{2.602489in}}%
\pgfpathlineto{\pgfqpoint{4.114554in}{2.599095in}}%
\pgfpathlineto{\pgfqpoint{4.103475in}{2.595713in}}%
\pgfpathlineto{\pgfqpoint{4.092391in}{2.592344in}}%
\pgfpathlineto{\pgfqpoint{4.081302in}{2.588987in}}%
\pgfpathlineto{\pgfqpoint{4.070207in}{2.585640in}}%
\pgfpathlineto{\pgfqpoint{4.076623in}{2.571579in}}%
\pgfpathlineto{\pgfqpoint{4.083041in}{2.557510in}}%
\pgfpathlineto{\pgfqpoint{4.089461in}{2.543431in}}%
\pgfpathlineto{\pgfqpoint{4.095884in}{2.529339in}}%
\pgfpathclose%
\pgfusepath{stroke,fill}%
\end{pgfscope}%
\begin{pgfscope}%
\pgfpathrectangle{\pgfqpoint{0.887500in}{0.275000in}}{\pgfqpoint{4.225000in}{4.225000in}}%
\pgfusepath{clip}%
\pgfsetbuttcap%
\pgfsetroundjoin%
\definecolor{currentfill}{rgb}{0.159194,0.482237,0.558073}%
\pgfsetfillcolor{currentfill}%
\pgfsetfillopacity{0.700000}%
\pgfsetlinewidth{0.501875pt}%
\definecolor{currentstroke}{rgb}{1.000000,1.000000,1.000000}%
\pgfsetstrokecolor{currentstroke}%
\pgfsetstrokeopacity{0.500000}%
\pgfsetdash{}{0pt}%
\pgfpathmoveto{\pgfqpoint{2.510286in}{2.273641in}}%
\pgfpathlineto{\pgfqpoint{2.521777in}{2.277147in}}%
\pgfpathlineto{\pgfqpoint{2.533262in}{2.280691in}}%
\pgfpathlineto{\pgfqpoint{2.544742in}{2.284267in}}%
\pgfpathlineto{\pgfqpoint{2.556215in}{2.287845in}}%
\pgfpathlineto{\pgfqpoint{2.567684in}{2.291393in}}%
\pgfpathlineto{\pgfqpoint{2.561621in}{2.300158in}}%
\pgfpathlineto{\pgfqpoint{2.555562in}{2.308902in}}%
\pgfpathlineto{\pgfqpoint{2.549508in}{2.317623in}}%
\pgfpathlineto{\pgfqpoint{2.543457in}{2.326322in}}%
\pgfpathlineto{\pgfqpoint{2.537411in}{2.335001in}}%
\pgfpathlineto{\pgfqpoint{2.525954in}{2.331451in}}%
\pgfpathlineto{\pgfqpoint{2.514492in}{2.327871in}}%
\pgfpathlineto{\pgfqpoint{2.503025in}{2.324297in}}%
\pgfpathlineto{\pgfqpoint{2.491551in}{2.320760in}}%
\pgfpathlineto{\pgfqpoint{2.480071in}{2.317268in}}%
\pgfpathlineto{\pgfqpoint{2.486106in}{2.308584in}}%
\pgfpathlineto{\pgfqpoint{2.492144in}{2.299880in}}%
\pgfpathlineto{\pgfqpoint{2.498187in}{2.291155in}}%
\pgfpathlineto{\pgfqpoint{2.504234in}{2.282409in}}%
\pgfpathclose%
\pgfusepath{stroke,fill}%
\end{pgfscope}%
\begin{pgfscope}%
\pgfpathrectangle{\pgfqpoint{0.887500in}{0.275000in}}{\pgfqpoint{4.225000in}{4.225000in}}%
\pgfusepath{clip}%
\pgfsetbuttcap%
\pgfsetroundjoin%
\definecolor{currentfill}{rgb}{0.147607,0.511733,0.557049}%
\pgfsetfillcolor{currentfill}%
\pgfsetfillopacity{0.700000}%
\pgfsetlinewidth{0.501875pt}%
\definecolor{currentstroke}{rgb}{1.000000,1.000000,1.000000}%
\pgfsetstrokecolor{currentstroke}%
\pgfsetstrokeopacity{0.500000}%
\pgfsetdash{}{0pt}%
\pgfpathmoveto{\pgfqpoint{2.189576in}{2.336281in}}%
\pgfpathlineto{\pgfqpoint{2.201148in}{2.339697in}}%
\pgfpathlineto{\pgfqpoint{2.212714in}{2.343102in}}%
\pgfpathlineto{\pgfqpoint{2.224274in}{2.346496in}}%
\pgfpathlineto{\pgfqpoint{2.235830in}{2.349878in}}%
\pgfpathlineto{\pgfqpoint{2.247379in}{2.353246in}}%
\pgfpathlineto{\pgfqpoint{2.241425in}{2.361786in}}%
\pgfpathlineto{\pgfqpoint{2.235474in}{2.370308in}}%
\pgfpathlineto{\pgfqpoint{2.229528in}{2.378811in}}%
\pgfpathlineto{\pgfqpoint{2.223586in}{2.387297in}}%
\pgfpathlineto{\pgfqpoint{2.217648in}{2.395766in}}%
\pgfpathlineto{\pgfqpoint{2.206110in}{2.392427in}}%
\pgfpathlineto{\pgfqpoint{2.194566in}{2.389075in}}%
\pgfpathlineto{\pgfqpoint{2.183017in}{2.385713in}}%
\pgfpathlineto{\pgfqpoint{2.171462in}{2.382338in}}%
\pgfpathlineto{\pgfqpoint{2.159902in}{2.378954in}}%
\pgfpathlineto{\pgfqpoint{2.165828in}{2.370456in}}%
\pgfpathlineto{\pgfqpoint{2.171759in}{2.361940in}}%
\pgfpathlineto{\pgfqpoint{2.177694in}{2.353405in}}%
\pgfpathlineto{\pgfqpoint{2.183633in}{2.344852in}}%
\pgfpathclose%
\pgfusepath{stroke,fill}%
\end{pgfscope}%
\begin{pgfscope}%
\pgfpathrectangle{\pgfqpoint{0.887500in}{0.275000in}}{\pgfqpoint{4.225000in}{4.225000in}}%
\pgfusepath{clip}%
\pgfsetbuttcap%
\pgfsetroundjoin%
\definecolor{currentfill}{rgb}{0.226397,0.728888,0.462789}%
\pgfsetfillcolor{currentfill}%
\pgfsetfillopacity{0.700000}%
\pgfsetlinewidth{0.501875pt}%
\definecolor{currentstroke}{rgb}{1.000000,1.000000,1.000000}%
\pgfsetstrokecolor{currentstroke}%
\pgfsetstrokeopacity{0.500000}%
\pgfsetdash{}{0pt}%
\pgfpathmoveto{\pgfqpoint{3.719493in}{2.794923in}}%
\pgfpathlineto{\pgfqpoint{3.730686in}{2.798288in}}%
\pgfpathlineto{\pgfqpoint{3.741873in}{2.801652in}}%
\pgfpathlineto{\pgfqpoint{3.753055in}{2.805019in}}%
\pgfpathlineto{\pgfqpoint{3.764232in}{2.808391in}}%
\pgfpathlineto{\pgfqpoint{3.775403in}{2.811772in}}%
\pgfpathlineto{\pgfqpoint{3.769040in}{2.825364in}}%
\pgfpathlineto{\pgfqpoint{3.762678in}{2.838930in}}%
\pgfpathlineto{\pgfqpoint{3.756319in}{2.852470in}}%
\pgfpathlineto{\pgfqpoint{3.749962in}{2.865983in}}%
\pgfpathlineto{\pgfqpoint{3.743607in}{2.879470in}}%
\pgfpathlineto{\pgfqpoint{3.732437in}{2.875997in}}%
\pgfpathlineto{\pgfqpoint{3.721262in}{2.872537in}}%
\pgfpathlineto{\pgfqpoint{3.710082in}{2.869090in}}%
\pgfpathlineto{\pgfqpoint{3.698897in}{2.865660in}}%
\pgfpathlineto{\pgfqpoint{3.687706in}{2.862247in}}%
\pgfpathlineto{\pgfqpoint{3.694060in}{2.848864in}}%
\pgfpathlineto{\pgfqpoint{3.700415in}{2.835441in}}%
\pgfpathlineto{\pgfqpoint{3.706772in}{2.821976in}}%
\pgfpathlineto{\pgfqpoint{3.713132in}{2.808469in}}%
\pgfpathclose%
\pgfusepath{stroke,fill}%
\end{pgfscope}%
\begin{pgfscope}%
\pgfpathrectangle{\pgfqpoint{0.887500in}{0.275000in}}{\pgfqpoint{4.225000in}{4.225000in}}%
\pgfusepath{clip}%
\pgfsetbuttcap%
\pgfsetroundjoin%
\definecolor{currentfill}{rgb}{0.177423,0.437527,0.557565}%
\pgfsetfillcolor{currentfill}%
\pgfsetfillopacity{0.700000}%
\pgfsetlinewidth{0.501875pt}%
\definecolor{currentstroke}{rgb}{1.000000,1.000000,1.000000}%
\pgfsetstrokecolor{currentstroke}%
\pgfsetstrokeopacity{0.500000}%
\pgfsetdash{}{0pt}%
\pgfpathmoveto{\pgfqpoint{2.918935in}{2.167826in}}%
\pgfpathlineto{\pgfqpoint{2.930335in}{2.168625in}}%
\pgfpathlineto{\pgfqpoint{2.941721in}{2.171727in}}%
\pgfpathlineto{\pgfqpoint{2.953093in}{2.178178in}}%
\pgfpathlineto{\pgfqpoint{2.964450in}{2.189028in}}%
\pgfpathlineto{\pgfqpoint{2.975795in}{2.205153in}}%
\pgfpathlineto{\pgfqpoint{2.969595in}{2.214544in}}%
\pgfpathlineto{\pgfqpoint{2.963400in}{2.223930in}}%
\pgfpathlineto{\pgfqpoint{2.957208in}{2.233314in}}%
\pgfpathlineto{\pgfqpoint{2.951020in}{2.242698in}}%
\pgfpathlineto{\pgfqpoint{2.944836in}{2.252082in}}%
\pgfpathlineto{\pgfqpoint{2.933514in}{2.235326in}}%
\pgfpathlineto{\pgfqpoint{2.922174in}{2.224082in}}%
\pgfpathlineto{\pgfqpoint{2.910816in}{2.217437in}}%
\pgfpathlineto{\pgfqpoint{2.899440in}{2.214295in}}%
\pgfpathlineto{\pgfqpoint{2.888048in}{2.213564in}}%
\pgfpathlineto{\pgfqpoint{2.894219in}{2.204313in}}%
\pgfpathlineto{\pgfqpoint{2.900392in}{2.195126in}}%
\pgfpathlineto{\pgfqpoint{2.906569in}{2.185992in}}%
\pgfpathlineto{\pgfqpoint{2.912751in}{2.176897in}}%
\pgfpathclose%
\pgfusepath{stroke,fill}%
\end{pgfscope}%
\begin{pgfscope}%
\pgfpathrectangle{\pgfqpoint{0.887500in}{0.275000in}}{\pgfqpoint{4.225000in}{4.225000in}}%
\pgfusepath{clip}%
\pgfsetbuttcap%
\pgfsetroundjoin%
\definecolor{currentfill}{rgb}{0.274149,0.751988,0.436601}%
\pgfsetfillcolor{currentfill}%
\pgfsetfillopacity{0.700000}%
\pgfsetlinewidth{0.501875pt}%
\definecolor{currentstroke}{rgb}{1.000000,1.000000,1.000000}%
\pgfsetstrokecolor{currentstroke}%
\pgfsetstrokeopacity{0.500000}%
\pgfsetdash{}{0pt}%
\pgfpathmoveto{\pgfqpoint{3.631673in}{2.845164in}}%
\pgfpathlineto{\pgfqpoint{3.642891in}{2.848611in}}%
\pgfpathlineto{\pgfqpoint{3.654103in}{2.852034in}}%
\pgfpathlineto{\pgfqpoint{3.665310in}{2.855442in}}%
\pgfpathlineto{\pgfqpoint{3.676511in}{2.858844in}}%
\pgfpathlineto{\pgfqpoint{3.687706in}{2.862247in}}%
\pgfpathlineto{\pgfqpoint{3.681355in}{2.875593in}}%
\pgfpathlineto{\pgfqpoint{3.675006in}{2.888909in}}%
\pgfpathlineto{\pgfqpoint{3.668660in}{2.902199in}}%
\pgfpathlineto{\pgfqpoint{3.662315in}{2.915471in}}%
\pgfpathlineto{\pgfqpoint{3.655974in}{2.928729in}}%
\pgfpathlineto{\pgfqpoint{3.644782in}{2.925332in}}%
\pgfpathlineto{\pgfqpoint{3.633585in}{2.921967in}}%
\pgfpathlineto{\pgfqpoint{3.622383in}{2.918619in}}%
\pgfpathlineto{\pgfqpoint{3.611175in}{2.915271in}}%
\pgfpathlineto{\pgfqpoint{3.599962in}{2.911910in}}%
\pgfpathlineto{\pgfqpoint{3.606299in}{2.898600in}}%
\pgfpathlineto{\pgfqpoint{3.612639in}{2.885271in}}%
\pgfpathlineto{\pgfqpoint{3.618982in}{2.871923in}}%
\pgfpathlineto{\pgfqpoint{3.625326in}{2.858554in}}%
\pgfpathclose%
\pgfusepath{stroke,fill}%
\end{pgfscope}%
\begin{pgfscope}%
\pgfpathrectangle{\pgfqpoint{0.887500in}{0.275000in}}{\pgfqpoint{4.225000in}{4.225000in}}%
\pgfusepath{clip}%
\pgfsetbuttcap%
\pgfsetroundjoin%
\definecolor{currentfill}{rgb}{0.121380,0.629492,0.531973}%
\pgfsetfillcolor{currentfill}%
\pgfsetfillopacity{0.700000}%
\pgfsetlinewidth{0.501875pt}%
\definecolor{currentstroke}{rgb}{1.000000,1.000000,1.000000}%
\pgfsetstrokecolor{currentstroke}%
\pgfsetstrokeopacity{0.500000}%
\pgfsetdash{}{0pt}%
\pgfpathmoveto{\pgfqpoint{4.014655in}{2.568937in}}%
\pgfpathlineto{\pgfqpoint{4.025777in}{2.572286in}}%
\pgfpathlineto{\pgfqpoint{4.036892in}{2.575628in}}%
\pgfpathlineto{\pgfqpoint{4.048003in}{2.578964in}}%
\pgfpathlineto{\pgfqpoint{4.059108in}{2.582301in}}%
\pgfpathlineto{\pgfqpoint{4.070207in}{2.585640in}}%
\pgfpathlineto{\pgfqpoint{4.063794in}{2.599699in}}%
\pgfpathlineto{\pgfqpoint{4.057383in}{2.613756in}}%
\pgfpathlineto{\pgfqpoint{4.050975in}{2.627817in}}%
\pgfpathlineto{\pgfqpoint{4.044570in}{2.641882in}}%
\pgfpathlineto{\pgfqpoint{4.038167in}{2.655945in}}%
\pgfpathlineto{\pgfqpoint{4.027069in}{2.652580in}}%
\pgfpathlineto{\pgfqpoint{4.015966in}{2.649216in}}%
\pgfpathlineto{\pgfqpoint{4.004857in}{2.645849in}}%
\pgfpathlineto{\pgfqpoint{3.993742in}{2.642477in}}%
\pgfpathlineto{\pgfqpoint{3.982622in}{2.639095in}}%
\pgfpathlineto{\pgfqpoint{3.989024in}{2.625070in}}%
\pgfpathlineto{\pgfqpoint{3.995428in}{2.611039in}}%
\pgfpathlineto{\pgfqpoint{4.001834in}{2.597008in}}%
\pgfpathlineto{\pgfqpoint{4.008244in}{2.582976in}}%
\pgfpathclose%
\pgfusepath{stroke,fill}%
\end{pgfscope}%
\begin{pgfscope}%
\pgfpathrectangle{\pgfqpoint{0.887500in}{0.275000in}}{\pgfqpoint{4.225000in}{4.225000in}}%
\pgfusepath{clip}%
\pgfsetbuttcap%
\pgfsetroundjoin%
\definecolor{currentfill}{rgb}{0.144759,0.519093,0.556572}%
\pgfsetfillcolor{currentfill}%
\pgfsetfillopacity{0.700000}%
\pgfsetlinewidth{0.501875pt}%
\definecolor{currentstroke}{rgb}{1.000000,1.000000,1.000000}%
\pgfsetstrokecolor{currentstroke}%
\pgfsetstrokeopacity{0.500000}%
\pgfsetdash{}{0pt}%
\pgfpathmoveto{\pgfqpoint{4.309625in}{2.332637in}}%
\pgfpathlineto{\pgfqpoint{4.320677in}{2.336153in}}%
\pgfpathlineto{\pgfqpoint{4.331725in}{2.339710in}}%
\pgfpathlineto{\pgfqpoint{4.342769in}{2.343296in}}%
\pgfpathlineto{\pgfqpoint{4.353807in}{2.346898in}}%
\pgfpathlineto{\pgfqpoint{4.364841in}{2.350502in}}%
\pgfpathlineto{\pgfqpoint{4.358384in}{2.364942in}}%
\pgfpathlineto{\pgfqpoint{4.351928in}{2.379378in}}%
\pgfpathlineto{\pgfqpoint{4.345475in}{2.393805in}}%
\pgfpathlineto{\pgfqpoint{4.339024in}{2.408220in}}%
\pgfpathlineto{\pgfqpoint{4.332576in}{2.422623in}}%
\pgfpathlineto{\pgfqpoint{4.321554in}{2.419338in}}%
\pgfpathlineto{\pgfqpoint{4.310526in}{2.416062in}}%
\pgfpathlineto{\pgfqpoint{4.299494in}{2.412791in}}%
\pgfpathlineto{\pgfqpoint{4.288456in}{2.409518in}}%
\pgfpathlineto{\pgfqpoint{4.277413in}{2.406236in}}%
\pgfpathlineto{\pgfqpoint{4.283856in}{2.391707in}}%
\pgfpathlineto{\pgfqpoint{4.290299in}{2.377088in}}%
\pgfpathlineto{\pgfqpoint{4.296742in}{2.362369in}}%
\pgfpathlineto{\pgfqpoint{4.303184in}{2.347544in}}%
\pgfpathclose%
\pgfusepath{stroke,fill}%
\end{pgfscope}%
\begin{pgfscope}%
\pgfpathrectangle{\pgfqpoint{0.887500in}{0.275000in}}{\pgfqpoint{4.225000in}{4.225000in}}%
\pgfusepath{clip}%
\pgfsetbuttcap%
\pgfsetroundjoin%
\definecolor{currentfill}{rgb}{0.129933,0.559582,0.551864}%
\pgfsetfillcolor{currentfill}%
\pgfsetfillopacity{0.700000}%
\pgfsetlinewidth{0.501875pt}%
\definecolor{currentstroke}{rgb}{1.000000,1.000000,1.000000}%
\pgfsetstrokecolor{currentstroke}%
\pgfsetstrokeopacity{0.500000}%
\pgfsetdash{}{0pt}%
\pgfpathmoveto{\pgfqpoint{1.635698in}{2.429429in}}%
\pgfpathlineto{\pgfqpoint{1.647411in}{2.432786in}}%
\pgfpathlineto{\pgfqpoint{1.659119in}{2.436136in}}%
\pgfpathlineto{\pgfqpoint{1.670820in}{2.439480in}}%
\pgfpathlineto{\pgfqpoint{1.682516in}{2.442819in}}%
\pgfpathlineto{\pgfqpoint{1.694206in}{2.446155in}}%
\pgfpathlineto{\pgfqpoint{1.688443in}{2.454372in}}%
\pgfpathlineto{\pgfqpoint{1.682683in}{2.462569in}}%
\pgfpathlineto{\pgfqpoint{1.676928in}{2.470747in}}%
\pgfpathlineto{\pgfqpoint{1.671177in}{2.478907in}}%
\pgfpathlineto{\pgfqpoint{1.665430in}{2.487051in}}%
\pgfpathlineto{\pgfqpoint{1.653753in}{2.483731in}}%
\pgfpathlineto{\pgfqpoint{1.642069in}{2.480409in}}%
\pgfpathlineto{\pgfqpoint{1.630380in}{2.477081in}}%
\pgfpathlineto{\pgfqpoint{1.618686in}{2.473747in}}%
\pgfpathlineto{\pgfqpoint{1.606986in}{2.470407in}}%
\pgfpathlineto{\pgfqpoint{1.612719in}{2.462247in}}%
\pgfpathlineto{\pgfqpoint{1.618458in}{2.454071in}}%
\pgfpathlineto{\pgfqpoint{1.624200in}{2.445877in}}%
\pgfpathlineto{\pgfqpoint{1.629947in}{2.437663in}}%
\pgfpathclose%
\pgfusepath{stroke,fill}%
\end{pgfscope}%
\begin{pgfscope}%
\pgfpathrectangle{\pgfqpoint{0.887500in}{0.275000in}}{\pgfqpoint{4.225000in}{4.225000in}}%
\pgfusepath{clip}%
\pgfsetbuttcap%
\pgfsetroundjoin%
\definecolor{currentfill}{rgb}{0.162016,0.687316,0.499129}%
\pgfsetfillcolor{currentfill}%
\pgfsetfillopacity{0.700000}%
\pgfsetlinewidth{0.501875pt}%
\definecolor{currentstroke}{rgb}{1.000000,1.000000,1.000000}%
\pgfsetstrokecolor{currentstroke}%
\pgfsetstrokeopacity{0.500000}%
\pgfsetdash{}{0pt}%
\pgfpathmoveto{\pgfqpoint{2.995970in}{2.626624in}}%
\pgfpathlineto{\pgfqpoint{3.007295in}{2.653725in}}%
\pgfpathlineto{\pgfqpoint{3.018628in}{2.681025in}}%
\pgfpathlineto{\pgfqpoint{3.029968in}{2.708891in}}%
\pgfpathlineto{\pgfqpoint{3.041317in}{2.737489in}}%
\pgfpathlineto{\pgfqpoint{3.052674in}{2.766339in}}%
\pgfpathlineto{\pgfqpoint{3.046443in}{2.782024in}}%
\pgfpathlineto{\pgfqpoint{3.040214in}{2.797839in}}%
\pgfpathlineto{\pgfqpoint{3.033986in}{2.813529in}}%
\pgfpathlineto{\pgfqpoint{3.027761in}{2.828841in}}%
\pgfpathlineto{\pgfqpoint{3.021538in}{2.843518in}}%
\pgfpathlineto{\pgfqpoint{3.010201in}{2.814800in}}%
\pgfpathlineto{\pgfqpoint{2.998874in}{2.786055in}}%
\pgfpathlineto{\pgfqpoint{2.987554in}{2.757742in}}%
\pgfpathlineto{\pgfqpoint{2.976243in}{2.729645in}}%
\pgfpathlineto{\pgfqpoint{2.964940in}{2.701346in}}%
\pgfpathlineto{\pgfqpoint{2.971142in}{2.686325in}}%
\pgfpathlineto{\pgfqpoint{2.977347in}{2.671005in}}%
\pgfpathlineto{\pgfqpoint{2.983553in}{2.655724in}}%
\pgfpathlineto{\pgfqpoint{2.989761in}{2.640818in}}%
\pgfpathclose%
\pgfusepath{stroke,fill}%
\end{pgfscope}%
\begin{pgfscope}%
\pgfpathrectangle{\pgfqpoint{0.887500in}{0.275000in}}{\pgfqpoint{4.225000in}{4.225000in}}%
\pgfusepath{clip}%
\pgfsetbuttcap%
\pgfsetroundjoin%
\definecolor{currentfill}{rgb}{0.327796,0.773980,0.406640}%
\pgfsetfillcolor{currentfill}%
\pgfsetfillopacity{0.700000}%
\pgfsetlinewidth{0.501875pt}%
\definecolor{currentstroke}{rgb}{1.000000,1.000000,1.000000}%
\pgfsetstrokecolor{currentstroke}%
\pgfsetstrokeopacity{0.500000}%
\pgfsetdash{}{0pt}%
\pgfpathmoveto{\pgfqpoint{3.543808in}{2.894545in}}%
\pgfpathlineto{\pgfqpoint{3.555051in}{2.898102in}}%
\pgfpathlineto{\pgfqpoint{3.566287in}{2.901616in}}%
\pgfpathlineto{\pgfqpoint{3.577518in}{2.905089in}}%
\pgfpathlineto{\pgfqpoint{3.588743in}{2.908520in}}%
\pgfpathlineto{\pgfqpoint{3.599962in}{2.911910in}}%
\pgfpathlineto{\pgfqpoint{3.593627in}{2.925202in}}%
\pgfpathlineto{\pgfqpoint{3.587295in}{2.938479in}}%
\pgfpathlineto{\pgfqpoint{3.580965in}{2.951740in}}%
\pgfpathlineto{\pgfqpoint{3.574637in}{2.964987in}}%
\pgfpathlineto{\pgfqpoint{3.568312in}{2.978219in}}%
\pgfpathlineto{\pgfqpoint{3.557097in}{2.974817in}}%
\pgfpathlineto{\pgfqpoint{3.545875in}{2.971371in}}%
\pgfpathlineto{\pgfqpoint{3.534648in}{2.967882in}}%
\pgfpathlineto{\pgfqpoint{3.523415in}{2.964351in}}%
\pgfpathlineto{\pgfqpoint{3.512176in}{2.960777in}}%
\pgfpathlineto{\pgfqpoint{3.518498in}{2.947625in}}%
\pgfpathlineto{\pgfqpoint{3.524822in}{2.934422in}}%
\pgfpathlineto{\pgfqpoint{3.531149in}{2.921172in}}%
\pgfpathlineto{\pgfqpoint{3.537477in}{2.907878in}}%
\pgfpathclose%
\pgfusepath{stroke,fill}%
\end{pgfscope}%
\begin{pgfscope}%
\pgfpathrectangle{\pgfqpoint{0.887500in}{0.275000in}}{\pgfqpoint{4.225000in}{4.225000in}}%
\pgfusepath{clip}%
\pgfsetbuttcap%
\pgfsetroundjoin%
\definecolor{currentfill}{rgb}{0.162142,0.474838,0.558140}%
\pgfsetfillcolor{currentfill}%
\pgfsetfillopacity{0.700000}%
\pgfsetlinewidth{0.501875pt}%
\definecolor{currentstroke}{rgb}{1.000000,1.000000,1.000000}%
\pgfsetstrokecolor{currentstroke}%
\pgfsetstrokeopacity{0.500000}%
\pgfsetdash{}{0pt}%
\pgfpathmoveto{\pgfqpoint{2.598060in}{2.247214in}}%
\pgfpathlineto{\pgfqpoint{2.609535in}{2.250703in}}%
\pgfpathlineto{\pgfqpoint{2.621005in}{2.254102in}}%
\pgfpathlineto{\pgfqpoint{2.632471in}{2.257385in}}%
\pgfpathlineto{\pgfqpoint{2.643932in}{2.260523in}}%
\pgfpathlineto{\pgfqpoint{2.655390in}{2.263536in}}%
\pgfpathlineto{\pgfqpoint{2.649295in}{2.272402in}}%
\pgfpathlineto{\pgfqpoint{2.643205in}{2.281242in}}%
\pgfpathlineto{\pgfqpoint{2.637119in}{2.290056in}}%
\pgfpathlineto{\pgfqpoint{2.631037in}{2.298847in}}%
\pgfpathlineto{\pgfqpoint{2.624959in}{2.307614in}}%
\pgfpathlineto{\pgfqpoint{2.613512in}{2.304641in}}%
\pgfpathlineto{\pgfqpoint{2.602061in}{2.301534in}}%
\pgfpathlineto{\pgfqpoint{2.590606in}{2.298270in}}%
\pgfpathlineto{\pgfqpoint{2.579147in}{2.294879in}}%
\pgfpathlineto{\pgfqpoint{2.567684in}{2.291393in}}%
\pgfpathlineto{\pgfqpoint{2.573751in}{2.282605in}}%
\pgfpathlineto{\pgfqpoint{2.579822in}{2.273793in}}%
\pgfpathlineto{\pgfqpoint{2.585897in}{2.264958in}}%
\pgfpathlineto{\pgfqpoint{2.591976in}{2.256098in}}%
\pgfpathclose%
\pgfusepath{stroke,fill}%
\end{pgfscope}%
\begin{pgfscope}%
\pgfpathrectangle{\pgfqpoint{0.887500in}{0.275000in}}{\pgfqpoint{4.225000in}{4.225000in}}%
\pgfusepath{clip}%
\pgfsetbuttcap%
\pgfsetroundjoin%
\definecolor{currentfill}{rgb}{0.140536,0.530132,0.555659}%
\pgfsetfillcolor{currentfill}%
\pgfsetfillopacity{0.700000}%
\pgfsetlinewidth{0.501875pt}%
\definecolor{currentstroke}{rgb}{1.000000,1.000000,1.000000}%
\pgfsetstrokecolor{currentstroke}%
\pgfsetstrokeopacity{0.500000}%
\pgfsetdash{}{0pt}%
\pgfpathmoveto{\pgfqpoint{1.956407in}{2.371084in}}%
\pgfpathlineto{\pgfqpoint{1.968043in}{2.374419in}}%
\pgfpathlineto{\pgfqpoint{1.979673in}{2.377747in}}%
\pgfpathlineto{\pgfqpoint{1.991297in}{2.381068in}}%
\pgfpathlineto{\pgfqpoint{2.002916in}{2.384385in}}%
\pgfpathlineto{\pgfqpoint{2.014529in}{2.387699in}}%
\pgfpathlineto{\pgfqpoint{2.008651in}{2.396133in}}%
\pgfpathlineto{\pgfqpoint{2.002778in}{2.404546in}}%
\pgfpathlineto{\pgfqpoint{1.996909in}{2.412940in}}%
\pgfpathlineto{\pgfqpoint{1.991045in}{2.421313in}}%
\pgfpathlineto{\pgfqpoint{1.985184in}{2.429667in}}%
\pgfpathlineto{\pgfqpoint{1.973583in}{2.426377in}}%
\pgfpathlineto{\pgfqpoint{1.961976in}{2.423086in}}%
\pgfpathlineto{\pgfqpoint{1.950364in}{2.419791in}}%
\pgfpathlineto{\pgfqpoint{1.938745in}{2.416490in}}%
\pgfpathlineto{\pgfqpoint{1.927121in}{2.413183in}}%
\pgfpathlineto{\pgfqpoint{1.932970in}{2.404803in}}%
\pgfpathlineto{\pgfqpoint{1.938823in}{2.396404in}}%
\pgfpathlineto{\pgfqpoint{1.944680in}{2.387984in}}%
\pgfpathlineto{\pgfqpoint{1.950542in}{2.379544in}}%
\pgfpathclose%
\pgfusepath{stroke,fill}%
\end{pgfscope}%
\begin{pgfscope}%
\pgfpathrectangle{\pgfqpoint{0.887500in}{0.275000in}}{\pgfqpoint{4.225000in}{4.225000in}}%
\pgfusepath{clip}%
\pgfsetbuttcap%
\pgfsetroundjoin%
\definecolor{currentfill}{rgb}{0.150476,0.504369,0.557430}%
\pgfsetfillcolor{currentfill}%
\pgfsetfillopacity{0.700000}%
\pgfsetlinewidth{0.501875pt}%
\definecolor{currentstroke}{rgb}{1.000000,1.000000,1.000000}%
\pgfsetstrokecolor{currentstroke}%
\pgfsetstrokeopacity{0.500000}%
\pgfsetdash{}{0pt}%
\pgfpathmoveto{\pgfqpoint{2.277216in}{2.310255in}}%
\pgfpathlineto{\pgfqpoint{2.288771in}{2.313639in}}%
\pgfpathlineto{\pgfqpoint{2.300322in}{2.317006in}}%
\pgfpathlineto{\pgfqpoint{2.311866in}{2.320355in}}%
\pgfpathlineto{\pgfqpoint{2.323406in}{2.323686in}}%
\pgfpathlineto{\pgfqpoint{2.334940in}{2.327003in}}%
\pgfpathlineto{\pgfqpoint{2.328953in}{2.335614in}}%
\pgfpathlineto{\pgfqpoint{2.322970in}{2.344206in}}%
\pgfpathlineto{\pgfqpoint{2.316991in}{2.352778in}}%
\pgfpathlineto{\pgfqpoint{2.311017in}{2.361332in}}%
\pgfpathlineto{\pgfqpoint{2.305046in}{2.369867in}}%
\pgfpathlineto{\pgfqpoint{2.293524in}{2.366571in}}%
\pgfpathlineto{\pgfqpoint{2.281996in}{2.363263in}}%
\pgfpathlineto{\pgfqpoint{2.270462in}{2.359939in}}%
\pgfpathlineto{\pgfqpoint{2.258924in}{2.356600in}}%
\pgfpathlineto{\pgfqpoint{2.247379in}{2.353246in}}%
\pgfpathlineto{\pgfqpoint{2.253338in}{2.344687in}}%
\pgfpathlineto{\pgfqpoint{2.259301in}{2.336109in}}%
\pgfpathlineto{\pgfqpoint{2.265269in}{2.327512in}}%
\pgfpathlineto{\pgfqpoint{2.271240in}{2.318894in}}%
\pgfpathclose%
\pgfusepath{stroke,fill}%
\end{pgfscope}%
\begin{pgfscope}%
\pgfpathrectangle{\pgfqpoint{0.887500in}{0.275000in}}{\pgfqpoint{4.225000in}{4.225000in}}%
\pgfusepath{clip}%
\pgfsetbuttcap%
\pgfsetroundjoin%
\definecolor{currentfill}{rgb}{0.525776,0.833491,0.288127}%
\pgfsetfillcolor{currentfill}%
\pgfsetfillopacity{0.700000}%
\pgfsetlinewidth{0.501875pt}%
\definecolor{currentstroke}{rgb}{1.000000,1.000000,1.000000}%
\pgfsetstrokecolor{currentstroke}%
\pgfsetstrokeopacity{0.500000}%
\pgfsetdash{}{0pt}%
\pgfpathmoveto{\pgfqpoint{3.191936in}{3.062301in}}%
\pgfpathlineto{\pgfqpoint{3.203277in}{3.069739in}}%
\pgfpathlineto{\pgfqpoint{3.214613in}{3.076855in}}%
\pgfpathlineto{\pgfqpoint{3.225944in}{3.083517in}}%
\pgfpathlineto{\pgfqpoint{3.237268in}{3.089596in}}%
\pgfpathlineto{\pgfqpoint{3.248586in}{3.095072in}}%
\pgfpathlineto{\pgfqpoint{3.242315in}{3.106976in}}%
\pgfpathlineto{\pgfqpoint{3.236047in}{3.118802in}}%
\pgfpathlineto{\pgfqpoint{3.229782in}{3.130575in}}%
\pgfpathlineto{\pgfqpoint{3.223519in}{3.142324in}}%
\pgfpathlineto{\pgfqpoint{3.217260in}{3.154074in}}%
\pgfpathlineto{\pgfqpoint{3.205949in}{3.148403in}}%
\pgfpathlineto{\pgfqpoint{3.194632in}{3.142125in}}%
\pgfpathlineto{\pgfqpoint{3.183310in}{3.135248in}}%
\pgfpathlineto{\pgfqpoint{3.171982in}{3.127869in}}%
\pgfpathlineto{\pgfqpoint{3.160650in}{3.120088in}}%
\pgfpathlineto{\pgfqpoint{3.166901in}{3.108624in}}%
\pgfpathlineto{\pgfqpoint{3.173156in}{3.097151in}}%
\pgfpathlineto{\pgfqpoint{3.179413in}{3.085631in}}%
\pgfpathlineto{\pgfqpoint{3.185673in}{3.074027in}}%
\pgfpathclose%
\pgfusepath{stroke,fill}%
\end{pgfscope}%
\begin{pgfscope}%
\pgfpathrectangle{\pgfqpoint{0.887500in}{0.275000in}}{\pgfqpoint{4.225000in}{4.225000in}}%
\pgfusepath{clip}%
\pgfsetbuttcap%
\pgfsetroundjoin%
\definecolor{currentfill}{rgb}{0.188923,0.410910,0.556326}%
\pgfsetfillcolor{currentfill}%
\pgfsetfillopacity{0.700000}%
\pgfsetlinewidth{0.501875pt}%
\definecolor{currentstroke}{rgb}{1.000000,1.000000,1.000000}%
\pgfsetstrokecolor{currentstroke}%
\pgfsetstrokeopacity{0.500000}%
\pgfsetdash{}{0pt}%
\pgfpathmoveto{\pgfqpoint{4.604757in}{2.101003in}}%
\pgfpathlineto{\pgfqpoint{4.615729in}{2.104335in}}%
\pgfpathlineto{\pgfqpoint{4.626695in}{2.107670in}}%
\pgfpathlineto{\pgfqpoint{4.637656in}{2.111009in}}%
\pgfpathlineto{\pgfqpoint{4.648612in}{2.114353in}}%
\pgfpathlineto{\pgfqpoint{4.642105in}{2.128916in}}%
\pgfpathlineto{\pgfqpoint{4.635600in}{2.143494in}}%
\pgfpathlineto{\pgfqpoint{4.629099in}{2.158082in}}%
\pgfpathlineto{\pgfqpoint{4.622599in}{2.172668in}}%
\pgfpathlineto{\pgfqpoint{4.616103in}{2.187246in}}%
\pgfpathlineto{\pgfqpoint{4.605147in}{2.183886in}}%
\pgfpathlineto{\pgfqpoint{4.594186in}{2.180518in}}%
\pgfpathlineto{\pgfqpoint{4.583218in}{2.177140in}}%
\pgfpathlineto{\pgfqpoint{4.572245in}{2.173746in}}%
\pgfpathlineto{\pgfqpoint{4.578743in}{2.159211in}}%
\pgfpathlineto{\pgfqpoint{4.585243in}{2.144660in}}%
\pgfpathlineto{\pgfqpoint{4.591745in}{2.130103in}}%
\pgfpathlineto{\pgfqpoint{4.598250in}{2.115548in}}%
\pgfpathclose%
\pgfusepath{stroke,fill}%
\end{pgfscope}%
\begin{pgfscope}%
\pgfpathrectangle{\pgfqpoint{0.887500in}{0.275000in}}{\pgfqpoint{4.225000in}{4.225000in}}%
\pgfusepath{clip}%
\pgfsetbuttcap%
\pgfsetroundjoin%
\definecolor{currentfill}{rgb}{0.377779,0.791781,0.377939}%
\pgfsetfillcolor{currentfill}%
\pgfsetfillopacity{0.700000}%
\pgfsetlinewidth{0.501875pt}%
\definecolor{currentstroke}{rgb}{1.000000,1.000000,1.000000}%
\pgfsetstrokecolor{currentstroke}%
\pgfsetstrokeopacity{0.500000}%
\pgfsetdash{}{0pt}%
\pgfpathmoveto{\pgfqpoint{3.455895in}{2.942289in}}%
\pgfpathlineto{\pgfqpoint{3.467163in}{2.946071in}}%
\pgfpathlineto{\pgfqpoint{3.478425in}{2.949809in}}%
\pgfpathlineto{\pgfqpoint{3.489681in}{2.953505in}}%
\pgfpathlineto{\pgfqpoint{3.500931in}{2.957162in}}%
\pgfpathlineto{\pgfqpoint{3.512176in}{2.960777in}}%
\pgfpathlineto{\pgfqpoint{3.505856in}{2.973878in}}%
\pgfpathlineto{\pgfqpoint{3.499538in}{2.986930in}}%
\pgfpathlineto{\pgfqpoint{3.493223in}{2.999935in}}%
\pgfpathlineto{\pgfqpoint{3.486910in}{3.012895in}}%
\pgfpathlineto{\pgfqpoint{3.480599in}{3.025812in}}%
\pgfpathlineto{\pgfqpoint{3.469358in}{3.022135in}}%
\pgfpathlineto{\pgfqpoint{3.458111in}{3.018440in}}%
\pgfpathlineto{\pgfqpoint{3.446859in}{3.014732in}}%
\pgfpathlineto{\pgfqpoint{3.435602in}{3.011012in}}%
\pgfpathlineto{\pgfqpoint{3.424339in}{3.007282in}}%
\pgfpathlineto{\pgfqpoint{3.430646in}{2.994353in}}%
\pgfpathlineto{\pgfqpoint{3.436955in}{2.981394in}}%
\pgfpathlineto{\pgfqpoint{3.443266in}{2.968401in}}%
\pgfpathlineto{\pgfqpoint{3.449579in}{2.955368in}}%
\pgfpathclose%
\pgfusepath{stroke,fill}%
\end{pgfscope}%
\begin{pgfscope}%
\pgfpathrectangle{\pgfqpoint{0.887500in}{0.275000in}}{\pgfqpoint{4.225000in}{4.225000in}}%
\pgfusepath{clip}%
\pgfsetbuttcap%
\pgfsetroundjoin%
\definecolor{currentfill}{rgb}{0.132268,0.655014,0.519661}%
\pgfsetfillcolor{currentfill}%
\pgfsetfillopacity{0.700000}%
\pgfsetlinewidth{0.501875pt}%
\definecolor{currentstroke}{rgb}{1.000000,1.000000,1.000000}%
\pgfsetstrokecolor{currentstroke}%
\pgfsetstrokeopacity{0.500000}%
\pgfsetdash{}{0pt}%
\pgfpathmoveto{\pgfqpoint{3.926937in}{2.622033in}}%
\pgfpathlineto{\pgfqpoint{3.938085in}{2.625457in}}%
\pgfpathlineto{\pgfqpoint{3.949228in}{2.628878in}}%
\pgfpathlineto{\pgfqpoint{3.960365in}{2.632294in}}%
\pgfpathlineto{\pgfqpoint{3.971496in}{2.635701in}}%
\pgfpathlineto{\pgfqpoint{3.982622in}{2.639095in}}%
\pgfpathlineto{\pgfqpoint{3.976223in}{2.653112in}}%
\pgfpathlineto{\pgfqpoint{3.969826in}{2.667113in}}%
\pgfpathlineto{\pgfqpoint{3.963431in}{2.681095in}}%
\pgfpathlineto{\pgfqpoint{3.957038in}{2.695052in}}%
\pgfpathlineto{\pgfqpoint{3.950647in}{2.708979in}}%
\pgfpathlineto{\pgfqpoint{3.939522in}{2.705531in}}%
\pgfpathlineto{\pgfqpoint{3.928392in}{2.702070in}}%
\pgfpathlineto{\pgfqpoint{3.917256in}{2.698600in}}%
\pgfpathlineto{\pgfqpoint{3.906115in}{2.695126in}}%
\pgfpathlineto{\pgfqpoint{3.894968in}{2.691654in}}%
\pgfpathlineto{\pgfqpoint{3.901358in}{2.677773in}}%
\pgfpathlineto{\pgfqpoint{3.907750in}{2.663868in}}%
\pgfpathlineto{\pgfqpoint{3.914143in}{2.649941in}}%
\pgfpathlineto{\pgfqpoint{3.920539in}{2.635995in}}%
\pgfpathclose%
\pgfusepath{stroke,fill}%
\end{pgfscope}%
\begin{pgfscope}%
\pgfpathrectangle{\pgfqpoint{0.887500in}{0.275000in}}{\pgfqpoint{4.225000in}{4.225000in}}%
\pgfusepath{clip}%
\pgfsetbuttcap%
\pgfsetroundjoin%
\definecolor{currentfill}{rgb}{0.133743,0.548535,0.553541}%
\pgfsetfillcolor{currentfill}%
\pgfsetfillopacity{0.700000}%
\pgfsetlinewidth{0.501875pt}%
\definecolor{currentstroke}{rgb}{1.000000,1.000000,1.000000}%
\pgfsetstrokecolor{currentstroke}%
\pgfsetstrokeopacity{0.500000}%
\pgfsetdash{}{0pt}%
\pgfpathmoveto{\pgfqpoint{4.222105in}{2.389547in}}%
\pgfpathlineto{\pgfqpoint{4.233178in}{2.392920in}}%
\pgfpathlineto{\pgfqpoint{4.244246in}{2.396280in}}%
\pgfpathlineto{\pgfqpoint{4.255307in}{2.399621in}}%
\pgfpathlineto{\pgfqpoint{4.266363in}{2.402939in}}%
\pgfpathlineto{\pgfqpoint{4.277413in}{2.406236in}}%
\pgfpathlineto{\pgfqpoint{4.270969in}{2.420685in}}%
\pgfpathlineto{\pgfqpoint{4.264526in}{2.435065in}}%
\pgfpathlineto{\pgfqpoint{4.258083in}{2.449387in}}%
\pgfpathlineto{\pgfqpoint{4.251642in}{2.463661in}}%
\pgfpathlineto{\pgfqpoint{4.245202in}{2.477897in}}%
\pgfpathlineto{\pgfqpoint{4.234153in}{2.474570in}}%
\pgfpathlineto{\pgfqpoint{4.223098in}{2.471232in}}%
\pgfpathlineto{\pgfqpoint{4.212038in}{2.467882in}}%
\pgfpathlineto{\pgfqpoint{4.200972in}{2.464523in}}%
\pgfpathlineto{\pgfqpoint{4.189900in}{2.461157in}}%
\pgfpathlineto{\pgfqpoint{4.196339in}{2.446938in}}%
\pgfpathlineto{\pgfqpoint{4.202779in}{2.432677in}}%
\pgfpathlineto{\pgfqpoint{4.209220in}{2.418365in}}%
\pgfpathlineto{\pgfqpoint{4.215662in}{2.403991in}}%
\pgfpathclose%
\pgfusepath{stroke,fill}%
\end{pgfscope}%
\begin{pgfscope}%
\pgfpathrectangle{\pgfqpoint{0.887500in}{0.275000in}}{\pgfqpoint{4.225000in}{4.225000in}}%
\pgfusepath{clip}%
\pgfsetbuttcap%
\pgfsetroundjoin%
\definecolor{currentfill}{rgb}{0.430983,0.808473,0.346476}%
\pgfsetfillcolor{currentfill}%
\pgfsetfillopacity{0.700000}%
\pgfsetlinewidth{0.501875pt}%
\definecolor{currentstroke}{rgb}{1.000000,1.000000,1.000000}%
\pgfsetstrokecolor{currentstroke}%
\pgfsetstrokeopacity{0.500000}%
\pgfsetdash{}{0pt}%
\pgfpathmoveto{\pgfqpoint{3.367947in}{2.988537in}}%
\pgfpathlineto{\pgfqpoint{3.379236in}{2.992295in}}%
\pgfpathlineto{\pgfqpoint{3.390520in}{2.996050in}}%
\pgfpathlineto{\pgfqpoint{3.401798in}{2.999800in}}%
\pgfpathlineto{\pgfqpoint{3.413072in}{3.003544in}}%
\pgfpathlineto{\pgfqpoint{3.424339in}{3.007282in}}%
\pgfpathlineto{\pgfqpoint{3.418036in}{3.020188in}}%
\pgfpathlineto{\pgfqpoint{3.411735in}{3.033077in}}%
\pgfpathlineto{\pgfqpoint{3.405436in}{3.045953in}}%
\pgfpathlineto{\pgfqpoint{3.399140in}{3.058822in}}%
\pgfpathlineto{\pgfqpoint{3.392847in}{3.071691in}}%
\pgfpathlineto{\pgfqpoint{3.381585in}{3.068035in}}%
\pgfpathlineto{\pgfqpoint{3.370317in}{3.064388in}}%
\pgfpathlineto{\pgfqpoint{3.359044in}{3.060759in}}%
\pgfpathlineto{\pgfqpoint{3.347766in}{3.057158in}}%
\pgfpathlineto{\pgfqpoint{3.336483in}{3.053592in}}%
\pgfpathlineto{\pgfqpoint{3.342771in}{3.040705in}}%
\pgfpathlineto{\pgfqpoint{3.349062in}{3.027748in}}%
\pgfpathlineto{\pgfqpoint{3.355354in}{3.014729in}}%
\pgfpathlineto{\pgfqpoint{3.361650in}{3.001656in}}%
\pgfpathclose%
\pgfusepath{stroke,fill}%
\end{pgfscope}%
\begin{pgfscope}%
\pgfpathrectangle{\pgfqpoint{0.887500in}{0.275000in}}{\pgfqpoint{4.225000in}{4.225000in}}%
\pgfusepath{clip}%
\pgfsetbuttcap%
\pgfsetroundjoin%
\definecolor{currentfill}{rgb}{0.177423,0.437527,0.557565}%
\pgfsetfillcolor{currentfill}%
\pgfsetfillopacity{0.700000}%
\pgfsetlinewidth{0.501875pt}%
\definecolor{currentstroke}{rgb}{1.000000,1.000000,1.000000}%
\pgfsetstrokecolor{currentstroke}%
\pgfsetstrokeopacity{0.500000}%
\pgfsetdash{}{0pt}%
\pgfpathmoveto{\pgfqpoint{4.517285in}{2.156386in}}%
\pgfpathlineto{\pgfqpoint{4.528290in}{2.159925in}}%
\pgfpathlineto{\pgfqpoint{4.539289in}{2.163426in}}%
\pgfpathlineto{\pgfqpoint{4.550281in}{2.166894in}}%
\pgfpathlineto{\pgfqpoint{4.561266in}{2.170332in}}%
\pgfpathlineto{\pgfqpoint{4.572245in}{2.173746in}}%
\pgfpathlineto{\pgfqpoint{4.565749in}{2.188258in}}%
\pgfpathlineto{\pgfqpoint{4.559254in}{2.202736in}}%
\pgfpathlineto{\pgfqpoint{4.552760in}{2.217173in}}%
\pgfpathlineto{\pgfqpoint{4.546267in}{2.231558in}}%
\pgfpathlineto{\pgfqpoint{4.539774in}{2.245885in}}%
\pgfpathlineto{\pgfqpoint{4.528792in}{2.242363in}}%
\pgfpathlineto{\pgfqpoint{4.517803in}{2.238796in}}%
\pgfpathlineto{\pgfqpoint{4.506806in}{2.235175in}}%
\pgfpathlineto{\pgfqpoint{4.495803in}{2.231494in}}%
\pgfpathlineto{\pgfqpoint{4.484791in}{2.227744in}}%
\pgfpathlineto{\pgfqpoint{4.491287in}{2.213542in}}%
\pgfpathlineto{\pgfqpoint{4.497785in}{2.199302in}}%
\pgfpathlineto{\pgfqpoint{4.504283in}{2.185028in}}%
\pgfpathlineto{\pgfqpoint{4.510783in}{2.170721in}}%
\pgfpathclose%
\pgfusepath{stroke,fill}%
\end{pgfscope}%
\begin{pgfscope}%
\pgfpathrectangle{\pgfqpoint{0.887500in}{0.275000in}}{\pgfqpoint{4.225000in}{4.225000in}}%
\pgfusepath{clip}%
\pgfsetbuttcap%
\pgfsetroundjoin%
\definecolor{currentfill}{rgb}{0.487026,0.823929,0.312321}%
\pgfsetfillcolor{currentfill}%
\pgfsetfillopacity{0.700000}%
\pgfsetlinewidth{0.501875pt}%
\definecolor{currentstroke}{rgb}{1.000000,1.000000,1.000000}%
\pgfsetstrokecolor{currentstroke}%
\pgfsetstrokeopacity{0.500000}%
\pgfsetdash{}{0pt}%
\pgfpathmoveto{\pgfqpoint{3.279974in}{3.033423in}}%
\pgfpathlineto{\pgfqpoint{3.291290in}{3.038147in}}%
\pgfpathlineto{\pgfqpoint{3.302599in}{3.042411in}}%
\pgfpathlineto{\pgfqpoint{3.313900in}{3.046330in}}%
\pgfpathlineto{\pgfqpoint{3.325194in}{3.050019in}}%
\pgfpathlineto{\pgfqpoint{3.336483in}{3.053592in}}%
\pgfpathlineto{\pgfqpoint{3.330197in}{3.066398in}}%
\pgfpathlineto{\pgfqpoint{3.323913in}{3.079114in}}%
\pgfpathlineto{\pgfqpoint{3.317631in}{3.091730in}}%
\pgfpathlineto{\pgfqpoint{3.311352in}{3.104236in}}%
\pgfpathlineto{\pgfqpoint{3.305074in}{3.116621in}}%
\pgfpathlineto{\pgfqpoint{3.293789in}{3.112773in}}%
\pgfpathlineto{\pgfqpoint{3.282498in}{3.108788in}}%
\pgfpathlineto{\pgfqpoint{3.271201in}{3.104573in}}%
\pgfpathlineto{\pgfqpoint{3.259897in}{3.100032in}}%
\pgfpathlineto{\pgfqpoint{3.248586in}{3.095072in}}%
\pgfpathlineto{\pgfqpoint{3.254859in}{3.083061in}}%
\pgfpathlineto{\pgfqpoint{3.261135in}{3.070918in}}%
\pgfpathlineto{\pgfqpoint{3.267413in}{3.058615in}}%
\pgfpathlineto{\pgfqpoint{3.273693in}{3.046126in}}%
\pgfpathclose%
\pgfusepath{stroke,fill}%
\end{pgfscope}%
\begin{pgfscope}%
\pgfpathrectangle{\pgfqpoint{0.887500in}{0.275000in}}{\pgfqpoint{4.225000in}{4.225000in}}%
\pgfusepath{clip}%
\pgfsetbuttcap%
\pgfsetroundjoin%
\definecolor{currentfill}{rgb}{0.159194,0.482237,0.558073}%
\pgfsetfillcolor{currentfill}%
\pgfsetfillopacity{0.700000}%
\pgfsetlinewidth{0.501875pt}%
\definecolor{currentstroke}{rgb}{1.000000,1.000000,1.000000}%
\pgfsetstrokecolor{currentstroke}%
\pgfsetstrokeopacity{0.500000}%
\pgfsetdash{}{0pt}%
\pgfpathmoveto{\pgfqpoint{2.975795in}{2.205153in}}%
\pgfpathlineto{\pgfqpoint{2.987133in}{2.226085in}}%
\pgfpathlineto{\pgfqpoint{2.998470in}{2.250748in}}%
\pgfpathlineto{\pgfqpoint{3.009810in}{2.278056in}}%
\pgfpathlineto{\pgfqpoint{3.021158in}{2.306925in}}%
\pgfpathlineto{\pgfqpoint{3.032514in}{2.336263in}}%
\pgfpathlineto{\pgfqpoint{3.026285in}{2.347343in}}%
\pgfpathlineto{\pgfqpoint{3.020060in}{2.358166in}}%
\pgfpathlineto{\pgfqpoint{3.013839in}{2.368659in}}%
\pgfpathlineto{\pgfqpoint{3.007622in}{2.378750in}}%
\pgfpathlineto{\pgfqpoint{3.001409in}{2.388379in}}%
\pgfpathlineto{\pgfqpoint{2.990083in}{2.357925in}}%
\pgfpathlineto{\pgfqpoint{2.978767in}{2.327922in}}%
\pgfpathlineto{\pgfqpoint{2.967457in}{2.299518in}}%
\pgfpathlineto{\pgfqpoint{2.956149in}{2.273858in}}%
\pgfpathlineto{\pgfqpoint{2.944836in}{2.252082in}}%
\pgfpathlineto{\pgfqpoint{2.951020in}{2.242698in}}%
\pgfpathlineto{\pgfqpoint{2.957208in}{2.233314in}}%
\pgfpathlineto{\pgfqpoint{2.963400in}{2.223930in}}%
\pgfpathlineto{\pgfqpoint{2.969595in}{2.214544in}}%
\pgfpathclose%
\pgfusepath{stroke,fill}%
\end{pgfscope}%
\begin{pgfscope}%
\pgfpathrectangle{\pgfqpoint{0.887500in}{0.275000in}}{\pgfqpoint{4.225000in}{4.225000in}}%
\pgfusepath{clip}%
\pgfsetbuttcap%
\pgfsetroundjoin%
\definecolor{currentfill}{rgb}{0.166617,0.463708,0.558119}%
\pgfsetfillcolor{currentfill}%
\pgfsetfillopacity{0.700000}%
\pgfsetlinewidth{0.501875pt}%
\definecolor{currentstroke}{rgb}{1.000000,1.000000,1.000000}%
\pgfsetstrokecolor{currentstroke}%
\pgfsetstrokeopacity{0.500000}%
\pgfsetdash{}{0pt}%
\pgfpathmoveto{\pgfqpoint{2.685924in}{2.218800in}}%
\pgfpathlineto{\pgfqpoint{2.697386in}{2.221795in}}%
\pgfpathlineto{\pgfqpoint{2.708841in}{2.224843in}}%
\pgfpathlineto{\pgfqpoint{2.720290in}{2.228033in}}%
\pgfpathlineto{\pgfqpoint{2.731731in}{2.231455in}}%
\pgfpathlineto{\pgfqpoint{2.743163in}{2.235199in}}%
\pgfpathlineto{\pgfqpoint{2.737038in}{2.244188in}}%
\pgfpathlineto{\pgfqpoint{2.730916in}{2.253150in}}%
\pgfpathlineto{\pgfqpoint{2.724798in}{2.262087in}}%
\pgfpathlineto{\pgfqpoint{2.718685in}{2.270999in}}%
\pgfpathlineto{\pgfqpoint{2.712575in}{2.279885in}}%
\pgfpathlineto{\pgfqpoint{2.701154in}{2.276138in}}%
\pgfpathlineto{\pgfqpoint{2.689724in}{2.272722in}}%
\pgfpathlineto{\pgfqpoint{2.678286in}{2.269544in}}%
\pgfpathlineto{\pgfqpoint{2.666841in}{2.266512in}}%
\pgfpathlineto{\pgfqpoint{2.655390in}{2.263536in}}%
\pgfpathlineto{\pgfqpoint{2.661488in}{2.254644in}}%
\pgfpathlineto{\pgfqpoint{2.667591in}{2.245724in}}%
\pgfpathlineto{\pgfqpoint{2.673698in}{2.236777in}}%
\pgfpathlineto{\pgfqpoint{2.679808in}{2.227803in}}%
\pgfpathclose%
\pgfusepath{stroke,fill}%
\end{pgfscope}%
\begin{pgfscope}%
\pgfpathrectangle{\pgfqpoint{0.887500in}{0.275000in}}{\pgfqpoint{4.225000in}{4.225000in}}%
\pgfusepath{clip}%
\pgfsetbuttcap%
\pgfsetroundjoin%
\definecolor{currentfill}{rgb}{0.133743,0.548535,0.553541}%
\pgfsetfillcolor{currentfill}%
\pgfsetfillopacity{0.700000}%
\pgfsetlinewidth{0.501875pt}%
\definecolor{currentstroke}{rgb}{1.000000,1.000000,1.000000}%
\pgfsetstrokecolor{currentstroke}%
\pgfsetstrokeopacity{0.500000}%
\pgfsetdash{}{0pt}%
\pgfpathmoveto{\pgfqpoint{1.723093in}{2.404743in}}%
\pgfpathlineto{\pgfqpoint{1.734789in}{2.408098in}}%
\pgfpathlineto{\pgfqpoint{1.746480in}{2.411451in}}%
\pgfpathlineto{\pgfqpoint{1.758166in}{2.414805in}}%
\pgfpathlineto{\pgfqpoint{1.769845in}{2.418161in}}%
\pgfpathlineto{\pgfqpoint{1.781518in}{2.421518in}}%
\pgfpathlineto{\pgfqpoint{1.775720in}{2.429824in}}%
\pgfpathlineto{\pgfqpoint{1.769926in}{2.438108in}}%
\pgfpathlineto{\pgfqpoint{1.764137in}{2.446368in}}%
\pgfpathlineto{\pgfqpoint{1.758351in}{2.454608in}}%
\pgfpathlineto{\pgfqpoint{1.752571in}{2.462826in}}%
\pgfpathlineto{\pgfqpoint{1.740910in}{2.459490in}}%
\pgfpathlineto{\pgfqpoint{1.729242in}{2.456155in}}%
\pgfpathlineto{\pgfqpoint{1.717570in}{2.452822in}}%
\pgfpathlineto{\pgfqpoint{1.705891in}{2.449489in}}%
\pgfpathlineto{\pgfqpoint{1.694206in}{2.446155in}}%
\pgfpathlineto{\pgfqpoint{1.699975in}{2.437918in}}%
\pgfpathlineto{\pgfqpoint{1.705747in}{2.429659in}}%
\pgfpathlineto{\pgfqpoint{1.711525in}{2.421377in}}%
\pgfpathlineto{\pgfqpoint{1.717306in}{2.413072in}}%
\pgfpathclose%
\pgfusepath{stroke,fill}%
\end{pgfscope}%
\begin{pgfscope}%
\pgfpathrectangle{\pgfqpoint{0.887500in}{0.275000in}}{\pgfqpoint{4.225000in}{4.225000in}}%
\pgfusepath{clip}%
\pgfsetbuttcap%
\pgfsetroundjoin%
\definecolor{currentfill}{rgb}{0.153894,0.680203,0.504172}%
\pgfsetfillcolor{currentfill}%
\pgfsetfillopacity{0.700000}%
\pgfsetlinewidth{0.501875pt}%
\definecolor{currentstroke}{rgb}{1.000000,1.000000,1.000000}%
\pgfsetstrokecolor{currentstroke}%
\pgfsetstrokeopacity{0.500000}%
\pgfsetdash{}{0pt}%
\pgfpathmoveto{\pgfqpoint{3.839158in}{2.674456in}}%
\pgfpathlineto{\pgfqpoint{3.850331in}{2.677868in}}%
\pgfpathlineto{\pgfqpoint{3.861498in}{2.681293in}}%
\pgfpathlineto{\pgfqpoint{3.872660in}{2.684733in}}%
\pgfpathlineto{\pgfqpoint{3.883817in}{2.688188in}}%
\pgfpathlineto{\pgfqpoint{3.894968in}{2.691654in}}%
\pgfpathlineto{\pgfqpoint{3.888581in}{2.705507in}}%
\pgfpathlineto{\pgfqpoint{3.882195in}{2.719331in}}%
\pgfpathlineto{\pgfqpoint{3.875812in}{2.733122in}}%
\pgfpathlineto{\pgfqpoint{3.869430in}{2.746878in}}%
\pgfpathlineto{\pgfqpoint{3.863050in}{2.760601in}}%
\pgfpathlineto{\pgfqpoint{3.851901in}{2.757130in}}%
\pgfpathlineto{\pgfqpoint{3.840747in}{2.753674in}}%
\pgfpathlineto{\pgfqpoint{3.829588in}{2.750239in}}%
\pgfpathlineto{\pgfqpoint{3.818423in}{2.746824in}}%
\pgfpathlineto{\pgfqpoint{3.807254in}{2.743426in}}%
\pgfpathlineto{\pgfqpoint{3.813631in}{2.729681in}}%
\pgfpathlineto{\pgfqpoint{3.820009in}{2.715910in}}%
\pgfpathlineto{\pgfqpoint{3.826390in}{2.702115in}}%
\pgfpathlineto{\pgfqpoint{3.832773in}{2.688297in}}%
\pgfpathclose%
\pgfusepath{stroke,fill}%
\end{pgfscope}%
\begin{pgfscope}%
\pgfpathrectangle{\pgfqpoint{0.887500in}{0.275000in}}{\pgfqpoint{4.225000in}{4.225000in}}%
\pgfusepath{clip}%
\pgfsetbuttcap%
\pgfsetroundjoin%
\definecolor{currentfill}{rgb}{0.143343,0.522773,0.556295}%
\pgfsetfillcolor{currentfill}%
\pgfsetfillopacity{0.700000}%
\pgfsetlinewidth{0.501875pt}%
\definecolor{currentstroke}{rgb}{1.000000,1.000000,1.000000}%
\pgfsetstrokecolor{currentstroke}%
\pgfsetstrokeopacity{0.500000}%
\pgfsetdash{}{0pt}%
\pgfpathmoveto{\pgfqpoint{2.043982in}{2.345219in}}%
\pgfpathlineto{\pgfqpoint{2.055600in}{2.348562in}}%
\pgfpathlineto{\pgfqpoint{2.067213in}{2.351908in}}%
\pgfpathlineto{\pgfqpoint{2.078820in}{2.355260in}}%
\pgfpathlineto{\pgfqpoint{2.090421in}{2.358622in}}%
\pgfpathlineto{\pgfqpoint{2.102016in}{2.361996in}}%
\pgfpathlineto{\pgfqpoint{2.096105in}{2.370504in}}%
\pgfpathlineto{\pgfqpoint{2.090199in}{2.378992in}}%
\pgfpathlineto{\pgfqpoint{2.084297in}{2.387461in}}%
\pgfpathlineto{\pgfqpoint{2.078399in}{2.395910in}}%
\pgfpathlineto{\pgfqpoint{2.072505in}{2.404340in}}%
\pgfpathlineto{\pgfqpoint{2.060922in}{2.400993in}}%
\pgfpathlineto{\pgfqpoint{2.049333in}{2.397658in}}%
\pgfpathlineto{\pgfqpoint{2.037737in}{2.394332in}}%
\pgfpathlineto{\pgfqpoint{2.026136in}{2.391013in}}%
\pgfpathlineto{\pgfqpoint{2.014529in}{2.387699in}}%
\pgfpathlineto{\pgfqpoint{2.020411in}{2.379244in}}%
\pgfpathlineto{\pgfqpoint{2.026297in}{2.370769in}}%
\pgfpathlineto{\pgfqpoint{2.032188in}{2.362273in}}%
\pgfpathlineto{\pgfqpoint{2.038082in}{2.353756in}}%
\pgfpathclose%
\pgfusepath{stroke,fill}%
\end{pgfscope}%
\begin{pgfscope}%
\pgfpathrectangle{\pgfqpoint{0.887500in}{0.275000in}}{\pgfqpoint{4.225000in}{4.225000in}}%
\pgfusepath{clip}%
\pgfsetbuttcap%
\pgfsetroundjoin%
\definecolor{currentfill}{rgb}{0.154815,0.493313,0.557840}%
\pgfsetfillcolor{currentfill}%
\pgfsetfillopacity{0.700000}%
\pgfsetlinewidth{0.501875pt}%
\definecolor{currentstroke}{rgb}{1.000000,1.000000,1.000000}%
\pgfsetstrokecolor{currentstroke}%
\pgfsetstrokeopacity{0.500000}%
\pgfsetdash{}{0pt}%
\pgfpathmoveto{\pgfqpoint{2.364938in}{2.283637in}}%
\pgfpathlineto{\pgfqpoint{2.376478in}{2.286972in}}%
\pgfpathlineto{\pgfqpoint{2.388012in}{2.290300in}}%
\pgfpathlineto{\pgfqpoint{2.399540in}{2.293626in}}%
\pgfpathlineto{\pgfqpoint{2.411062in}{2.296955in}}%
\pgfpathlineto{\pgfqpoint{2.422579in}{2.300291in}}%
\pgfpathlineto{\pgfqpoint{2.416559in}{2.308979in}}%
\pgfpathlineto{\pgfqpoint{2.410544in}{2.317646in}}%
\pgfpathlineto{\pgfqpoint{2.404533in}{2.326295in}}%
\pgfpathlineto{\pgfqpoint{2.398527in}{2.334924in}}%
\pgfpathlineto{\pgfqpoint{2.392524in}{2.343535in}}%
\pgfpathlineto{\pgfqpoint{2.381019in}{2.340220in}}%
\pgfpathlineto{\pgfqpoint{2.369508in}{2.336914in}}%
\pgfpathlineto{\pgfqpoint{2.357991in}{2.333613in}}%
\pgfpathlineto{\pgfqpoint{2.346468in}{2.330311in}}%
\pgfpathlineto{\pgfqpoint{2.334940in}{2.327003in}}%
\pgfpathlineto{\pgfqpoint{2.340931in}{2.318372in}}%
\pgfpathlineto{\pgfqpoint{2.346926in}{2.309720in}}%
\pgfpathlineto{\pgfqpoint{2.352926in}{2.301047in}}%
\pgfpathlineto{\pgfqpoint{2.358930in}{2.292353in}}%
\pgfpathclose%
\pgfusepath{stroke,fill}%
\end{pgfscope}%
\begin{pgfscope}%
\pgfpathrectangle{\pgfqpoint{0.887500in}{0.275000in}}{\pgfqpoint{4.225000in}{4.225000in}}%
\pgfusepath{clip}%
\pgfsetbuttcap%
\pgfsetroundjoin%
\definecolor{currentfill}{rgb}{0.166617,0.463708,0.558119}%
\pgfsetfillcolor{currentfill}%
\pgfsetfillopacity{0.700000}%
\pgfsetlinewidth{0.501875pt}%
\definecolor{currentstroke}{rgb}{1.000000,1.000000,1.000000}%
\pgfsetstrokecolor{currentstroke}%
\pgfsetstrokeopacity{0.500000}%
\pgfsetdash{}{0pt}%
\pgfpathmoveto{\pgfqpoint{4.429611in}{2.207682in}}%
\pgfpathlineto{\pgfqpoint{4.440664in}{2.211896in}}%
\pgfpathlineto{\pgfqpoint{4.451708in}{2.216001in}}%
\pgfpathlineto{\pgfqpoint{4.462744in}{2.220006in}}%
\pgfpathlineto{\pgfqpoint{4.473772in}{2.223918in}}%
\pgfpathlineto{\pgfqpoint{4.484791in}{2.227744in}}%
\pgfpathlineto{\pgfqpoint{4.478296in}{2.241907in}}%
\pgfpathlineto{\pgfqpoint{4.471803in}{2.256028in}}%
\pgfpathlineto{\pgfqpoint{4.465310in}{2.270110in}}%
\pgfpathlineto{\pgfqpoint{4.458819in}{2.284163in}}%
\pgfpathlineto{\pgfqpoint{4.452330in}{2.298196in}}%
\pgfpathlineto{\pgfqpoint{4.441314in}{2.294401in}}%
\pgfpathlineto{\pgfqpoint{4.430291in}{2.290545in}}%
\pgfpathlineto{\pgfqpoint{4.419260in}{2.286619in}}%
\pgfpathlineto{\pgfqpoint{4.408221in}{2.282618in}}%
\pgfpathlineto{\pgfqpoint{4.397175in}{2.278535in}}%
\pgfpathlineto{\pgfqpoint{4.403652in}{2.264245in}}%
\pgfpathlineto{\pgfqpoint{4.410135in}{2.250007in}}%
\pgfpathlineto{\pgfqpoint{4.416622in}{2.235831in}}%
\pgfpathlineto{\pgfqpoint{4.423114in}{2.221723in}}%
\pgfpathclose%
\pgfusepath{stroke,fill}%
\end{pgfscope}%
\begin{pgfscope}%
\pgfpathrectangle{\pgfqpoint{0.887500in}{0.275000in}}{\pgfqpoint{4.225000in}{4.225000in}}%
\pgfusepath{clip}%
\pgfsetbuttcap%
\pgfsetroundjoin%
\definecolor{currentfill}{rgb}{0.125394,0.574318,0.549086}%
\pgfsetfillcolor{currentfill}%
\pgfsetfillopacity{0.700000}%
\pgfsetlinewidth{0.501875pt}%
\definecolor{currentstroke}{rgb}{1.000000,1.000000,1.000000}%
\pgfsetstrokecolor{currentstroke}%
\pgfsetstrokeopacity{0.500000}%
\pgfsetdash{}{0pt}%
\pgfpathmoveto{\pgfqpoint{4.134460in}{2.444324in}}%
\pgfpathlineto{\pgfqpoint{4.145559in}{2.447685in}}%
\pgfpathlineto{\pgfqpoint{4.156652in}{2.451050in}}%
\pgfpathlineto{\pgfqpoint{4.167740in}{2.454418in}}%
\pgfpathlineto{\pgfqpoint{4.178823in}{2.457788in}}%
\pgfpathlineto{\pgfqpoint{4.189900in}{2.461157in}}%
\pgfpathlineto{\pgfqpoint{4.183463in}{2.475344in}}%
\pgfpathlineto{\pgfqpoint{4.177028in}{2.489508in}}%
\pgfpathlineto{\pgfqpoint{4.170594in}{2.503659in}}%
\pgfpathlineto{\pgfqpoint{4.164164in}{2.517801in}}%
\pgfpathlineto{\pgfqpoint{4.157735in}{2.531934in}}%
\pgfpathlineto{\pgfqpoint{4.146660in}{2.528580in}}%
\pgfpathlineto{\pgfqpoint{4.135580in}{2.525234in}}%
\pgfpathlineto{\pgfqpoint{4.124495in}{2.521894in}}%
\pgfpathlineto{\pgfqpoint{4.113404in}{2.518560in}}%
\pgfpathlineto{\pgfqpoint{4.102308in}{2.515230in}}%
\pgfpathlineto{\pgfqpoint{4.108735in}{2.501101in}}%
\pgfpathlineto{\pgfqpoint{4.115164in}{2.486949in}}%
\pgfpathlineto{\pgfqpoint{4.121594in}{2.472770in}}%
\pgfpathlineto{\pgfqpoint{4.128026in}{2.458562in}}%
\pgfpathclose%
\pgfusepath{stroke,fill}%
\end{pgfscope}%
\begin{pgfscope}%
\pgfpathrectangle{\pgfqpoint{0.887500in}{0.275000in}}{\pgfqpoint{4.225000in}{4.225000in}}%
\pgfusepath{clip}%
\pgfsetbuttcap%
\pgfsetroundjoin%
\definecolor{currentfill}{rgb}{0.126453,0.570633,0.549841}%
\pgfsetfillcolor{currentfill}%
\pgfsetfillopacity{0.700000}%
\pgfsetlinewidth{0.501875pt}%
\definecolor{currentstroke}{rgb}{1.000000,1.000000,1.000000}%
\pgfsetstrokecolor{currentstroke}%
\pgfsetstrokeopacity{0.500000}%
\pgfsetdash{}{0pt}%
\pgfpathmoveto{\pgfqpoint{3.001409in}{2.388379in}}%
\pgfpathlineto{\pgfqpoint{3.012745in}{2.418131in}}%
\pgfpathlineto{\pgfqpoint{3.024092in}{2.446211in}}%
\pgfpathlineto{\pgfqpoint{3.035449in}{2.472591in}}%
\pgfpathlineto{\pgfqpoint{3.046813in}{2.497533in}}%
\pgfpathlineto{\pgfqpoint{3.058185in}{2.521301in}}%
\pgfpathlineto{\pgfqpoint{3.051948in}{2.531128in}}%
\pgfpathlineto{\pgfqpoint{3.045715in}{2.540681in}}%
\pgfpathlineto{\pgfqpoint{3.039485in}{2.550121in}}%
\pgfpathlineto{\pgfqpoint{3.033260in}{2.559610in}}%
\pgfpathlineto{\pgfqpoint{3.027038in}{2.569310in}}%
\pgfpathlineto{\pgfqpoint{3.015693in}{2.544045in}}%
\pgfpathlineto{\pgfqpoint{3.004357in}{2.518032in}}%
\pgfpathlineto{\pgfqpoint{2.993030in}{2.490970in}}%
\pgfpathlineto{\pgfqpoint{2.981713in}{2.462560in}}%
\pgfpathlineto{\pgfqpoint{2.970408in}{2.432786in}}%
\pgfpathlineto{\pgfqpoint{2.976600in}{2.424012in}}%
\pgfpathlineto{\pgfqpoint{2.982796in}{2.415302in}}%
\pgfpathlineto{\pgfqpoint{2.988996in}{2.406538in}}%
\pgfpathlineto{\pgfqpoint{2.995200in}{2.397603in}}%
\pgfpathclose%
\pgfusepath{stroke,fill}%
\end{pgfscope}%
\begin{pgfscope}%
\pgfpathrectangle{\pgfqpoint{0.887500in}{0.275000in}}{\pgfqpoint{4.225000in}{4.225000in}}%
\pgfusepath{clip}%
\pgfsetbuttcap%
\pgfsetroundjoin%
\definecolor{currentfill}{rgb}{0.327796,0.773980,0.406640}%
\pgfsetfillcolor{currentfill}%
\pgfsetfillopacity{0.700000}%
\pgfsetlinewidth{0.501875pt}%
\definecolor{currentstroke}{rgb}{1.000000,1.000000,1.000000}%
\pgfsetstrokecolor{currentstroke}%
\pgfsetstrokeopacity{0.500000}%
\pgfsetdash{}{0pt}%
\pgfpathmoveto{\pgfqpoint{3.021538in}{2.843518in}}%
\pgfpathlineto{\pgfqpoint{3.032884in}{2.871616in}}%
\pgfpathlineto{\pgfqpoint{3.044239in}{2.898494in}}%
\pgfpathlineto{\pgfqpoint{3.055601in}{2.923552in}}%
\pgfpathlineto{\pgfqpoint{3.066970in}{2.946189in}}%
\pgfpathlineto{\pgfqpoint{3.078343in}{2.965803in}}%
\pgfpathlineto{\pgfqpoint{3.072102in}{2.977962in}}%
\pgfpathlineto{\pgfqpoint{3.065865in}{2.989126in}}%
\pgfpathlineto{\pgfqpoint{3.059630in}{2.999216in}}%
\pgfpathlineto{\pgfqpoint{3.053399in}{3.008158in}}%
\pgfpathlineto{\pgfqpoint{3.047173in}{3.015875in}}%
\pgfpathlineto{\pgfqpoint{3.035822in}{2.996605in}}%
\pgfpathlineto{\pgfqpoint{3.024476in}{2.974756in}}%
\pgfpathlineto{\pgfqpoint{3.013137in}{2.950801in}}%
\pgfpathlineto{\pgfqpoint{3.001807in}{2.925215in}}%
\pgfpathlineto{\pgfqpoint{2.990486in}{2.898469in}}%
\pgfpathlineto{\pgfqpoint{2.996685in}{2.890787in}}%
\pgfpathlineto{\pgfqpoint{3.002891in}{2.881196in}}%
\pgfpathlineto{\pgfqpoint{3.009102in}{2.869951in}}%
\pgfpathlineto{\pgfqpoint{3.015318in}{2.857307in}}%
\pgfpathclose%
\pgfusepath{stroke,fill}%
\end{pgfscope}%
\begin{pgfscope}%
\pgfpathrectangle{\pgfqpoint{0.887500in}{0.275000in}}{\pgfqpoint{4.225000in}{4.225000in}}%
\pgfusepath{clip}%
\pgfsetbuttcap%
\pgfsetroundjoin%
\definecolor{currentfill}{rgb}{0.185783,0.704891,0.485273}%
\pgfsetfillcolor{currentfill}%
\pgfsetfillopacity{0.700000}%
\pgfsetlinewidth{0.501875pt}%
\definecolor{currentstroke}{rgb}{1.000000,1.000000,1.000000}%
\pgfsetstrokecolor{currentstroke}%
\pgfsetstrokeopacity{0.500000}%
\pgfsetdash{}{0pt}%
\pgfpathmoveto{\pgfqpoint{3.751330in}{2.726628in}}%
\pgfpathlineto{\pgfqpoint{3.762525in}{2.729971in}}%
\pgfpathlineto{\pgfqpoint{3.773715in}{2.733319in}}%
\pgfpathlineto{\pgfqpoint{3.784900in}{2.736676in}}%
\pgfpathlineto{\pgfqpoint{3.796080in}{2.740044in}}%
\pgfpathlineto{\pgfqpoint{3.807254in}{2.743426in}}%
\pgfpathlineto{\pgfqpoint{3.800880in}{2.757146in}}%
\pgfpathlineto{\pgfqpoint{3.794507in}{2.770841in}}%
\pgfpathlineto{\pgfqpoint{3.788137in}{2.784510in}}%
\pgfpathlineto{\pgfqpoint{3.781769in}{2.798154in}}%
\pgfpathlineto{\pgfqpoint{3.775403in}{2.811772in}}%
\pgfpathlineto{\pgfqpoint{3.764232in}{2.808391in}}%
\pgfpathlineto{\pgfqpoint{3.753055in}{2.805019in}}%
\pgfpathlineto{\pgfqpoint{3.741873in}{2.801652in}}%
\pgfpathlineto{\pgfqpoint{3.730686in}{2.798288in}}%
\pgfpathlineto{\pgfqpoint{3.719493in}{2.794923in}}%
\pgfpathlineto{\pgfqpoint{3.725856in}{2.781338in}}%
\pgfpathlineto{\pgfqpoint{3.732222in}{2.767714in}}%
\pgfpathlineto{\pgfqpoint{3.738589in}{2.754054in}}%
\pgfpathlineto{\pgfqpoint{3.744959in}{2.740358in}}%
\pgfpathclose%
\pgfusepath{stroke,fill}%
\end{pgfscope}%
\begin{pgfscope}%
\pgfpathrectangle{\pgfqpoint{0.887500in}{0.275000in}}{\pgfqpoint{4.225000in}{4.225000in}}%
\pgfusepath{clip}%
\pgfsetbuttcap%
\pgfsetroundjoin%
\definecolor{currentfill}{rgb}{0.156270,0.489624,0.557936}%
\pgfsetfillcolor{currentfill}%
\pgfsetfillopacity{0.700000}%
\pgfsetlinewidth{0.501875pt}%
\definecolor{currentstroke}{rgb}{1.000000,1.000000,1.000000}%
\pgfsetstrokecolor{currentstroke}%
\pgfsetstrokeopacity{0.500000}%
\pgfsetdash{}{0pt}%
\pgfpathmoveto{\pgfqpoint{4.341865in}{2.258044in}}%
\pgfpathlineto{\pgfqpoint{4.352932in}{2.261986in}}%
\pgfpathlineto{\pgfqpoint{4.363998in}{2.266057in}}%
\pgfpathlineto{\pgfqpoint{4.375061in}{2.270206in}}%
\pgfpathlineto{\pgfqpoint{4.386121in}{2.274382in}}%
\pgfpathlineto{\pgfqpoint{4.397175in}{2.278535in}}%
\pgfpathlineto{\pgfqpoint{4.390701in}{2.292871in}}%
\pgfpathlineto{\pgfqpoint{4.384231in}{2.307243in}}%
\pgfpathlineto{\pgfqpoint{4.377765in}{2.321645in}}%
\pgfpathlineto{\pgfqpoint{4.371301in}{2.336067in}}%
\pgfpathlineto{\pgfqpoint{4.364841in}{2.350502in}}%
\pgfpathlineto{\pgfqpoint{4.353807in}{2.346898in}}%
\pgfpathlineto{\pgfqpoint{4.342769in}{2.343296in}}%
\pgfpathlineto{\pgfqpoint{4.331725in}{2.339710in}}%
\pgfpathlineto{\pgfqpoint{4.320677in}{2.336153in}}%
\pgfpathlineto{\pgfqpoint{4.309625in}{2.332637in}}%
\pgfpathlineto{\pgfqpoint{4.316067in}{2.317682in}}%
\pgfpathlineto{\pgfqpoint{4.322511in}{2.302711in}}%
\pgfpathlineto{\pgfqpoint{4.328958in}{2.287759in}}%
\pgfpathlineto{\pgfqpoint{4.335409in}{2.272859in}}%
\pgfpathclose%
\pgfusepath{stroke,fill}%
\end{pgfscope}%
\begin{pgfscope}%
\pgfpathrectangle{\pgfqpoint{0.887500in}{0.275000in}}{\pgfqpoint{4.225000in}{4.225000in}}%
\pgfusepath{clip}%
\pgfsetbuttcap%
\pgfsetroundjoin%
\definecolor{currentfill}{rgb}{0.171176,0.452530,0.557965}%
\pgfsetfillcolor{currentfill}%
\pgfsetfillopacity{0.700000}%
\pgfsetlinewidth{0.501875pt}%
\definecolor{currentstroke}{rgb}{1.000000,1.000000,1.000000}%
\pgfsetstrokecolor{currentstroke}%
\pgfsetstrokeopacity{0.500000}%
\pgfsetdash{}{0pt}%
\pgfpathmoveto{\pgfqpoint{2.773853in}{2.189855in}}%
\pgfpathlineto{\pgfqpoint{2.785288in}{2.194028in}}%
\pgfpathlineto{\pgfqpoint{2.796716in}{2.198496in}}%
\pgfpathlineto{\pgfqpoint{2.808139in}{2.202967in}}%
\pgfpathlineto{\pgfqpoint{2.819559in}{2.207141in}}%
\pgfpathlineto{\pgfqpoint{2.830977in}{2.210719in}}%
\pgfpathlineto{\pgfqpoint{2.824820in}{2.219853in}}%
\pgfpathlineto{\pgfqpoint{2.818666in}{2.228954in}}%
\pgfpathlineto{\pgfqpoint{2.812517in}{2.238021in}}%
\pgfpathlineto{\pgfqpoint{2.806372in}{2.247058in}}%
\pgfpathlineto{\pgfqpoint{2.800231in}{2.256064in}}%
\pgfpathlineto{\pgfqpoint{2.788824in}{2.252439in}}%
\pgfpathlineto{\pgfqpoint{2.777416in}{2.248251in}}%
\pgfpathlineto{\pgfqpoint{2.766004in}{2.243791in}}%
\pgfpathlineto{\pgfqpoint{2.754587in}{2.239346in}}%
\pgfpathlineto{\pgfqpoint{2.743163in}{2.235199in}}%
\pgfpathlineto{\pgfqpoint{2.749293in}{2.226184in}}%
\pgfpathlineto{\pgfqpoint{2.755427in}{2.217143in}}%
\pgfpathlineto{\pgfqpoint{2.761565in}{2.208075in}}%
\pgfpathlineto{\pgfqpoint{2.767707in}{2.198979in}}%
\pgfpathclose%
\pgfusepath{stroke,fill}%
\end{pgfscope}%
\begin{pgfscope}%
\pgfpathrectangle{\pgfqpoint{0.887500in}{0.275000in}}{\pgfqpoint{4.225000in}{4.225000in}}%
\pgfusepath{clip}%
\pgfsetbuttcap%
\pgfsetroundjoin%
\definecolor{currentfill}{rgb}{0.487026,0.823929,0.312321}%
\pgfsetfillcolor{currentfill}%
\pgfsetfillopacity{0.700000}%
\pgfsetlinewidth{0.501875pt}%
\definecolor{currentstroke}{rgb}{1.000000,1.000000,1.000000}%
\pgfsetstrokecolor{currentstroke}%
\pgfsetstrokeopacity{0.500000}%
\pgfsetdash{}{0pt}%
\pgfpathmoveto{\pgfqpoint{3.135184in}{3.024383in}}%
\pgfpathlineto{\pgfqpoint{3.146541in}{3.031934in}}%
\pgfpathlineto{\pgfqpoint{3.157894in}{3.039373in}}%
\pgfpathlineto{\pgfqpoint{3.169244in}{3.046986in}}%
\pgfpathlineto{\pgfqpoint{3.180592in}{3.054672in}}%
\pgfpathlineto{\pgfqpoint{3.191936in}{3.062301in}}%
\pgfpathlineto{\pgfqpoint{3.185673in}{3.074027in}}%
\pgfpathlineto{\pgfqpoint{3.179413in}{3.085631in}}%
\pgfpathlineto{\pgfqpoint{3.173156in}{3.097151in}}%
\pgfpathlineto{\pgfqpoint{3.166901in}{3.108624in}}%
\pgfpathlineto{\pgfqpoint{3.160650in}{3.120088in}}%
\pgfpathlineto{\pgfqpoint{3.149315in}{3.112006in}}%
\pgfpathlineto{\pgfqpoint{3.137976in}{3.103724in}}%
\pgfpathlineto{\pgfqpoint{3.126634in}{3.095341in}}%
\pgfpathlineto{\pgfqpoint{3.115290in}{3.086934in}}%
\pgfpathlineto{\pgfqpoint{3.103942in}{3.078270in}}%
\pgfpathlineto{\pgfqpoint{3.110185in}{3.068389in}}%
\pgfpathlineto{\pgfqpoint{3.116431in}{3.058071in}}%
\pgfpathlineto{\pgfqpoint{3.122680in}{3.047304in}}%
\pgfpathlineto{\pgfqpoint{3.128931in}{3.036079in}}%
\pgfpathclose%
\pgfusepath{stroke,fill}%
\end{pgfscope}%
\begin{pgfscope}%
\pgfpathrectangle{\pgfqpoint{0.887500in}{0.275000in}}{\pgfqpoint{4.225000in}{4.225000in}}%
\pgfusepath{clip}%
\pgfsetbuttcap%
\pgfsetroundjoin%
\definecolor{currentfill}{rgb}{0.136408,0.541173,0.554483}%
\pgfsetfillcolor{currentfill}%
\pgfsetfillopacity{0.700000}%
\pgfsetlinewidth{0.501875pt}%
\definecolor{currentstroke}{rgb}{1.000000,1.000000,1.000000}%
\pgfsetstrokecolor{currentstroke}%
\pgfsetstrokeopacity{0.500000}%
\pgfsetdash{}{0pt}%
\pgfpathmoveto{\pgfqpoint{1.810577in}{2.379648in}}%
\pgfpathlineto{\pgfqpoint{1.822257in}{2.383030in}}%
\pgfpathlineto{\pgfqpoint{1.833931in}{2.386411in}}%
\pgfpathlineto{\pgfqpoint{1.845599in}{2.389787in}}%
\pgfpathlineto{\pgfqpoint{1.857262in}{2.393157in}}%
\pgfpathlineto{\pgfqpoint{1.868919in}{2.396519in}}%
\pgfpathlineto{\pgfqpoint{1.863086in}{2.404908in}}%
\pgfpathlineto{\pgfqpoint{1.857258in}{2.413277in}}%
\pgfpathlineto{\pgfqpoint{1.851435in}{2.421624in}}%
\pgfpathlineto{\pgfqpoint{1.845615in}{2.429950in}}%
\pgfpathlineto{\pgfqpoint{1.839800in}{2.438255in}}%
\pgfpathlineto{\pgfqpoint{1.828155in}{2.434921in}}%
\pgfpathlineto{\pgfqpoint{1.816504in}{2.431578in}}%
\pgfpathlineto{\pgfqpoint{1.804848in}{2.428228in}}%
\pgfpathlineto{\pgfqpoint{1.793186in}{2.424874in}}%
\pgfpathlineto{\pgfqpoint{1.781518in}{2.421518in}}%
\pgfpathlineto{\pgfqpoint{1.787321in}{2.413188in}}%
\pgfpathlineto{\pgfqpoint{1.793129in}{2.404837in}}%
\pgfpathlineto{\pgfqpoint{1.798940in}{2.396463in}}%
\pgfpathlineto{\pgfqpoint{1.804757in}{2.388066in}}%
\pgfpathclose%
\pgfusepath{stroke,fill}%
\end{pgfscope}%
\begin{pgfscope}%
\pgfpathrectangle{\pgfqpoint{0.887500in}{0.275000in}}{\pgfqpoint{4.225000in}{4.225000in}}%
\pgfusepath{clip}%
\pgfsetbuttcap%
\pgfsetroundjoin%
\definecolor{currentfill}{rgb}{0.119738,0.603785,0.541400}%
\pgfsetfillcolor{currentfill}%
\pgfsetfillopacity{0.700000}%
\pgfsetlinewidth{0.501875pt}%
\definecolor{currentstroke}{rgb}{1.000000,1.000000,1.000000}%
\pgfsetstrokecolor{currentstroke}%
\pgfsetstrokeopacity{0.500000}%
\pgfsetdash{}{0pt}%
\pgfpathmoveto{\pgfqpoint{4.046746in}{2.498529in}}%
\pgfpathlineto{\pgfqpoint{4.057870in}{2.501885in}}%
\pgfpathlineto{\pgfqpoint{4.068988in}{2.505231in}}%
\pgfpathlineto{\pgfqpoint{4.080100in}{2.508568in}}%
\pgfpathlineto{\pgfqpoint{4.091207in}{2.511900in}}%
\pgfpathlineto{\pgfqpoint{4.102308in}{2.515230in}}%
\pgfpathlineto{\pgfqpoint{4.095884in}{2.529339in}}%
\pgfpathlineto{\pgfqpoint{4.089461in}{2.543431in}}%
\pgfpathlineto{\pgfqpoint{4.083041in}{2.557510in}}%
\pgfpathlineto{\pgfqpoint{4.076623in}{2.571579in}}%
\pgfpathlineto{\pgfqpoint{4.070207in}{2.585640in}}%
\pgfpathlineto{\pgfqpoint{4.059108in}{2.582301in}}%
\pgfpathlineto{\pgfqpoint{4.048003in}{2.578964in}}%
\pgfpathlineto{\pgfqpoint{4.036892in}{2.575628in}}%
\pgfpathlineto{\pgfqpoint{4.025777in}{2.572286in}}%
\pgfpathlineto{\pgfqpoint{4.014655in}{2.568937in}}%
\pgfpathlineto{\pgfqpoint{4.021069in}{2.554889in}}%
\pgfpathlineto{\pgfqpoint{4.027485in}{2.540828in}}%
\pgfpathlineto{\pgfqpoint{4.033903in}{2.526750in}}%
\pgfpathlineto{\pgfqpoint{4.040324in}{2.512652in}}%
\pgfpathclose%
\pgfusepath{stroke,fill}%
\end{pgfscope}%
\begin{pgfscope}%
\pgfpathrectangle{\pgfqpoint{0.887500in}{0.275000in}}{\pgfqpoint{4.225000in}{4.225000in}}%
\pgfusepath{clip}%
\pgfsetbuttcap%
\pgfsetroundjoin%
\definecolor{currentfill}{rgb}{0.159194,0.482237,0.558073}%
\pgfsetfillcolor{currentfill}%
\pgfsetfillopacity{0.700000}%
\pgfsetlinewidth{0.501875pt}%
\definecolor{currentstroke}{rgb}{1.000000,1.000000,1.000000}%
\pgfsetstrokecolor{currentstroke}%
\pgfsetstrokeopacity{0.500000}%
\pgfsetdash{}{0pt}%
\pgfpathmoveto{\pgfqpoint{2.452738in}{2.256533in}}%
\pgfpathlineto{\pgfqpoint{2.464259in}{2.259914in}}%
\pgfpathlineto{\pgfqpoint{2.475775in}{2.263310in}}%
\pgfpathlineto{\pgfqpoint{2.487285in}{2.266727in}}%
\pgfpathlineto{\pgfqpoint{2.498788in}{2.270169in}}%
\pgfpathlineto{\pgfqpoint{2.510286in}{2.273641in}}%
\pgfpathlineto{\pgfqpoint{2.504234in}{2.282409in}}%
\pgfpathlineto{\pgfqpoint{2.498187in}{2.291155in}}%
\pgfpathlineto{\pgfqpoint{2.492144in}{2.299880in}}%
\pgfpathlineto{\pgfqpoint{2.486106in}{2.308584in}}%
\pgfpathlineto{\pgfqpoint{2.480071in}{2.317268in}}%
\pgfpathlineto{\pgfqpoint{2.468585in}{2.313815in}}%
\pgfpathlineto{\pgfqpoint{2.457092in}{2.310397in}}%
\pgfpathlineto{\pgfqpoint{2.445594in}{2.307007in}}%
\pgfpathlineto{\pgfqpoint{2.434089in}{2.303640in}}%
\pgfpathlineto{\pgfqpoint{2.422579in}{2.300291in}}%
\pgfpathlineto{\pgfqpoint{2.428602in}{2.291584in}}%
\pgfpathlineto{\pgfqpoint{2.434630in}{2.282855in}}%
\pgfpathlineto{\pgfqpoint{2.440661in}{2.274104in}}%
\pgfpathlineto{\pgfqpoint{2.446697in}{2.265330in}}%
\pgfpathclose%
\pgfusepath{stroke,fill}%
\end{pgfscope}%
\begin{pgfscope}%
\pgfpathrectangle{\pgfqpoint{0.887500in}{0.275000in}}{\pgfqpoint{4.225000in}{4.225000in}}%
\pgfusepath{clip}%
\pgfsetbuttcap%
\pgfsetroundjoin%
\definecolor{currentfill}{rgb}{0.226397,0.728888,0.462789}%
\pgfsetfillcolor{currentfill}%
\pgfsetfillopacity{0.700000}%
\pgfsetlinewidth{0.501875pt}%
\definecolor{currentstroke}{rgb}{1.000000,1.000000,1.000000}%
\pgfsetstrokecolor{currentstroke}%
\pgfsetstrokeopacity{0.500000}%
\pgfsetdash{}{0pt}%
\pgfpathmoveto{\pgfqpoint{3.663443in}{2.777835in}}%
\pgfpathlineto{\pgfqpoint{3.674665in}{2.781314in}}%
\pgfpathlineto{\pgfqpoint{3.685881in}{2.784754in}}%
\pgfpathlineto{\pgfqpoint{3.697091in}{2.788164in}}%
\pgfpathlineto{\pgfqpoint{3.708295in}{2.791551in}}%
\pgfpathlineto{\pgfqpoint{3.719493in}{2.794923in}}%
\pgfpathlineto{\pgfqpoint{3.713132in}{2.808469in}}%
\pgfpathlineto{\pgfqpoint{3.706772in}{2.821976in}}%
\pgfpathlineto{\pgfqpoint{3.700415in}{2.835441in}}%
\pgfpathlineto{\pgfqpoint{3.694060in}{2.848864in}}%
\pgfpathlineto{\pgfqpoint{3.687706in}{2.862247in}}%
\pgfpathlineto{\pgfqpoint{3.676511in}{2.858844in}}%
\pgfpathlineto{\pgfqpoint{3.665310in}{2.855442in}}%
\pgfpathlineto{\pgfqpoint{3.654103in}{2.852034in}}%
\pgfpathlineto{\pgfqpoint{3.642891in}{2.848611in}}%
\pgfpathlineto{\pgfqpoint{3.631673in}{2.845164in}}%
\pgfpathlineto{\pgfqpoint{3.638023in}{2.831752in}}%
\pgfpathlineto{\pgfqpoint{3.644374in}{2.818315in}}%
\pgfpathlineto{\pgfqpoint{3.650728in}{2.804852in}}%
\pgfpathlineto{\pgfqpoint{3.657085in}{2.791359in}}%
\pgfpathclose%
\pgfusepath{stroke,fill}%
\end{pgfscope}%
\begin{pgfscope}%
\pgfpathrectangle{\pgfqpoint{0.887500in}{0.275000in}}{\pgfqpoint{4.225000in}{4.225000in}}%
\pgfusepath{clip}%
\pgfsetbuttcap%
\pgfsetroundjoin%
\definecolor{currentfill}{rgb}{0.147607,0.511733,0.557049}%
\pgfsetfillcolor{currentfill}%
\pgfsetfillopacity{0.700000}%
\pgfsetlinewidth{0.501875pt}%
\definecolor{currentstroke}{rgb}{1.000000,1.000000,1.000000}%
\pgfsetstrokecolor{currentstroke}%
\pgfsetstrokeopacity{0.500000}%
\pgfsetdash{}{0pt}%
\pgfpathmoveto{\pgfqpoint{2.131633in}{2.319164in}}%
\pgfpathlineto{\pgfqpoint{2.143233in}{2.322581in}}%
\pgfpathlineto{\pgfqpoint{2.154827in}{2.326004in}}%
\pgfpathlineto{\pgfqpoint{2.166416in}{2.329431in}}%
\pgfpathlineto{\pgfqpoint{2.177999in}{2.332858in}}%
\pgfpathlineto{\pgfqpoint{2.189576in}{2.336281in}}%
\pgfpathlineto{\pgfqpoint{2.183633in}{2.344852in}}%
\pgfpathlineto{\pgfqpoint{2.177694in}{2.353405in}}%
\pgfpathlineto{\pgfqpoint{2.171759in}{2.361940in}}%
\pgfpathlineto{\pgfqpoint{2.165828in}{2.370456in}}%
\pgfpathlineto{\pgfqpoint{2.159902in}{2.378954in}}%
\pgfpathlineto{\pgfqpoint{2.148336in}{2.375562in}}%
\pgfpathlineto{\pgfqpoint{2.136765in}{2.372167in}}%
\pgfpathlineto{\pgfqpoint{2.125188in}{2.368772in}}%
\pgfpathlineto{\pgfqpoint{2.113605in}{2.365381in}}%
\pgfpathlineto{\pgfqpoint{2.102016in}{2.361996in}}%
\pgfpathlineto{\pgfqpoint{2.107931in}{2.353469in}}%
\pgfpathlineto{\pgfqpoint{2.113850in}{2.344922in}}%
\pgfpathlineto{\pgfqpoint{2.119773in}{2.336356in}}%
\pgfpathlineto{\pgfqpoint{2.125701in}{2.327770in}}%
\pgfpathclose%
\pgfusepath{stroke,fill}%
\end{pgfscope}%
\begin{pgfscope}%
\pgfpathrectangle{\pgfqpoint{0.887500in}{0.275000in}}{\pgfqpoint{4.225000in}{4.225000in}}%
\pgfusepath{clip}%
\pgfsetbuttcap%
\pgfsetroundjoin%
\definecolor{currentfill}{rgb}{0.134692,0.658636,0.517649}%
\pgfsetfillcolor{currentfill}%
\pgfsetfillopacity{0.700000}%
\pgfsetlinewidth{0.501875pt}%
\definecolor{currentstroke}{rgb}{1.000000,1.000000,1.000000}%
\pgfsetstrokecolor{currentstroke}%
\pgfsetstrokeopacity{0.500000}%
\pgfsetdash{}{0pt}%
\pgfpathmoveto{\pgfqpoint{3.027038in}{2.569310in}}%
\pgfpathlineto{\pgfqpoint{3.038389in}{2.594127in}}%
\pgfpathlineto{\pgfqpoint{3.049747in}{2.618800in}}%
\pgfpathlineto{\pgfqpoint{3.061112in}{2.643632in}}%
\pgfpathlineto{\pgfqpoint{3.072484in}{2.668792in}}%
\pgfpathlineto{\pgfqpoint{3.083862in}{2.694002in}}%
\pgfpathlineto{\pgfqpoint{3.077620in}{2.707646in}}%
\pgfpathlineto{\pgfqpoint{3.071380in}{2.721688in}}%
\pgfpathlineto{\pgfqpoint{3.065143in}{2.736141in}}%
\pgfpathlineto{\pgfqpoint{3.058908in}{2.751020in}}%
\pgfpathlineto{\pgfqpoint{3.052674in}{2.766339in}}%
\pgfpathlineto{\pgfqpoint{3.041317in}{2.737489in}}%
\pgfpathlineto{\pgfqpoint{3.029968in}{2.708891in}}%
\pgfpathlineto{\pgfqpoint{3.018628in}{2.681025in}}%
\pgfpathlineto{\pgfqpoint{3.007295in}{2.653725in}}%
\pgfpathlineto{\pgfqpoint{2.995970in}{2.626624in}}%
\pgfpathlineto{\pgfqpoint{3.002180in}{2.613448in}}%
\pgfpathlineto{\pgfqpoint{3.008391in}{2.601289in}}%
\pgfpathlineto{\pgfqpoint{3.014604in}{2.589987in}}%
\pgfpathlineto{\pgfqpoint{3.020819in}{2.579382in}}%
\pgfpathclose%
\pgfusepath{stroke,fill}%
\end{pgfscope}%
\begin{pgfscope}%
\pgfpathrectangle{\pgfqpoint{0.887500in}{0.275000in}}{\pgfqpoint{4.225000in}{4.225000in}}%
\pgfusepath{clip}%
\pgfsetbuttcap%
\pgfsetroundjoin%
\definecolor{currentfill}{rgb}{0.203063,0.379716,0.553925}%
\pgfsetfillcolor{currentfill}%
\pgfsetfillopacity{0.700000}%
\pgfsetlinewidth{0.501875pt}%
\definecolor{currentstroke}{rgb}{1.000000,1.000000,1.000000}%
\pgfsetstrokecolor{currentstroke}%
\pgfsetstrokeopacity{0.500000}%
\pgfsetdash{}{0pt}%
\pgfpathmoveto{\pgfqpoint{4.637337in}{2.028357in}}%
\pgfpathlineto{\pgfqpoint{4.648309in}{2.031676in}}%
\pgfpathlineto{\pgfqpoint{4.659276in}{2.035005in}}%
\pgfpathlineto{\pgfqpoint{4.670238in}{2.038343in}}%
\pgfpathlineto{\pgfqpoint{4.681194in}{2.041691in}}%
\pgfpathlineto{\pgfqpoint{4.674673in}{2.056215in}}%
\pgfpathlineto{\pgfqpoint{4.668153in}{2.070740in}}%
\pgfpathlineto{\pgfqpoint{4.661637in}{2.085269in}}%
\pgfpathlineto{\pgfqpoint{4.655123in}{2.099805in}}%
\pgfpathlineto{\pgfqpoint{4.648612in}{2.114353in}}%
\pgfpathlineto{\pgfqpoint{4.637656in}{2.111009in}}%
\pgfpathlineto{\pgfqpoint{4.626695in}{2.107670in}}%
\pgfpathlineto{\pgfqpoint{4.615729in}{2.104335in}}%
\pgfpathlineto{\pgfqpoint{4.604757in}{2.101003in}}%
\pgfpathlineto{\pgfqpoint{4.611268in}{2.086467in}}%
\pgfpathlineto{\pgfqpoint{4.617781in}{2.071937in}}%
\pgfpathlineto{\pgfqpoint{4.624297in}{2.057411in}}%
\pgfpathlineto{\pgfqpoint{4.630816in}{2.042885in}}%
\pgfpathclose%
\pgfusepath{stroke,fill}%
\end{pgfscope}%
\begin{pgfscope}%
\pgfpathrectangle{\pgfqpoint{0.887500in}{0.275000in}}{\pgfqpoint{4.225000in}{4.225000in}}%
\pgfusepath{clip}%
\pgfsetbuttcap%
\pgfsetroundjoin%
\definecolor{currentfill}{rgb}{0.177423,0.437527,0.557565}%
\pgfsetfillcolor{currentfill}%
\pgfsetfillopacity{0.700000}%
\pgfsetlinewidth{0.501875pt}%
\definecolor{currentstroke}{rgb}{1.000000,1.000000,1.000000}%
\pgfsetstrokecolor{currentstroke}%
\pgfsetstrokeopacity{0.500000}%
\pgfsetdash{}{0pt}%
\pgfpathmoveto{\pgfqpoint{2.861825in}{2.164516in}}%
\pgfpathlineto{\pgfqpoint{2.873252in}{2.167215in}}%
\pgfpathlineto{\pgfqpoint{2.884678in}{2.168765in}}%
\pgfpathlineto{\pgfqpoint{2.896103in}{2.168941in}}%
\pgfpathlineto{\pgfqpoint{2.907524in}{2.168281in}}%
\pgfpathlineto{\pgfqpoint{2.918935in}{2.167826in}}%
\pgfpathlineto{\pgfqpoint{2.912751in}{2.176897in}}%
\pgfpathlineto{\pgfqpoint{2.906569in}{2.185992in}}%
\pgfpathlineto{\pgfqpoint{2.900392in}{2.195126in}}%
\pgfpathlineto{\pgfqpoint{2.894219in}{2.204313in}}%
\pgfpathlineto{\pgfqpoint{2.888048in}{2.213564in}}%
\pgfpathlineto{\pgfqpoint{2.876643in}{2.214148in}}%
\pgfpathlineto{\pgfqpoint{2.865229in}{2.214953in}}%
\pgfpathlineto{\pgfqpoint{2.853812in}{2.214891in}}%
\pgfpathlineto{\pgfqpoint{2.842394in}{2.213402in}}%
\pgfpathlineto{\pgfqpoint{2.830977in}{2.210719in}}%
\pgfpathlineto{\pgfqpoint{2.837139in}{2.201550in}}%
\pgfpathlineto{\pgfqpoint{2.843304in}{2.192344in}}%
\pgfpathlineto{\pgfqpoint{2.849474in}{2.183103in}}%
\pgfpathlineto{\pgfqpoint{2.855647in}{2.173827in}}%
\pgfpathclose%
\pgfusepath{stroke,fill}%
\end{pgfscope}%
\begin{pgfscope}%
\pgfpathrectangle{\pgfqpoint{0.887500in}{0.275000in}}{\pgfqpoint{4.225000in}{4.225000in}}%
\pgfusepath{clip}%
\pgfsetbuttcap%
\pgfsetroundjoin%
\definecolor{currentfill}{rgb}{0.274149,0.751988,0.436601}%
\pgfsetfillcolor{currentfill}%
\pgfsetfillopacity{0.700000}%
\pgfsetlinewidth{0.501875pt}%
\definecolor{currentstroke}{rgb}{1.000000,1.000000,1.000000}%
\pgfsetstrokecolor{currentstroke}%
\pgfsetstrokeopacity{0.500000}%
\pgfsetdash{}{0pt}%
\pgfpathmoveto{\pgfqpoint{3.575496in}{2.827416in}}%
\pgfpathlineto{\pgfqpoint{3.586743in}{2.831031in}}%
\pgfpathlineto{\pgfqpoint{3.597984in}{2.834616in}}%
\pgfpathlineto{\pgfqpoint{3.609220in}{2.838169in}}%
\pgfpathlineto{\pgfqpoint{3.620450in}{2.841686in}}%
\pgfpathlineto{\pgfqpoint{3.631673in}{2.845164in}}%
\pgfpathlineto{\pgfqpoint{3.625326in}{2.858554in}}%
\pgfpathlineto{\pgfqpoint{3.618982in}{2.871923in}}%
\pgfpathlineto{\pgfqpoint{3.612639in}{2.885271in}}%
\pgfpathlineto{\pgfqpoint{3.606299in}{2.898600in}}%
\pgfpathlineto{\pgfqpoint{3.599962in}{2.911910in}}%
\pgfpathlineto{\pgfqpoint{3.588743in}{2.908520in}}%
\pgfpathlineto{\pgfqpoint{3.577518in}{2.905089in}}%
\pgfpathlineto{\pgfqpoint{3.566287in}{2.901616in}}%
\pgfpathlineto{\pgfqpoint{3.555051in}{2.898102in}}%
\pgfpathlineto{\pgfqpoint{3.543808in}{2.894545in}}%
\pgfpathlineto{\pgfqpoint{3.550141in}{2.881175in}}%
\pgfpathlineto{\pgfqpoint{3.556476in}{2.867774in}}%
\pgfpathlineto{\pgfqpoint{3.562814in}{2.854344in}}%
\pgfpathlineto{\pgfqpoint{3.569154in}{2.840890in}}%
\pgfpathclose%
\pgfusepath{stroke,fill}%
\end{pgfscope}%
\begin{pgfscope}%
\pgfpathrectangle{\pgfqpoint{0.887500in}{0.275000in}}{\pgfqpoint{4.225000in}{4.225000in}}%
\pgfusepath{clip}%
\pgfsetbuttcap%
\pgfsetroundjoin%
\definecolor{currentfill}{rgb}{0.121380,0.629492,0.531973}%
\pgfsetfillcolor{currentfill}%
\pgfsetfillopacity{0.700000}%
\pgfsetlinewidth{0.501875pt}%
\definecolor{currentstroke}{rgb}{1.000000,1.000000,1.000000}%
\pgfsetstrokecolor{currentstroke}%
\pgfsetstrokeopacity{0.500000}%
\pgfsetdash{}{0pt}%
\pgfpathmoveto{\pgfqpoint{3.958963in}{2.552045in}}%
\pgfpathlineto{\pgfqpoint{3.970112in}{2.555436in}}%
\pgfpathlineto{\pgfqpoint{3.981256in}{2.558823in}}%
\pgfpathlineto{\pgfqpoint{3.992395in}{2.562204in}}%
\pgfpathlineto{\pgfqpoint{4.003528in}{2.565576in}}%
\pgfpathlineto{\pgfqpoint{4.014655in}{2.568937in}}%
\pgfpathlineto{\pgfqpoint{4.008244in}{2.582976in}}%
\pgfpathlineto{\pgfqpoint{4.001834in}{2.597008in}}%
\pgfpathlineto{\pgfqpoint{3.995428in}{2.611039in}}%
\pgfpathlineto{\pgfqpoint{3.989024in}{2.625070in}}%
\pgfpathlineto{\pgfqpoint{3.982622in}{2.639095in}}%
\pgfpathlineto{\pgfqpoint{3.971496in}{2.635701in}}%
\pgfpathlineto{\pgfqpoint{3.960365in}{2.632294in}}%
\pgfpathlineto{\pgfqpoint{3.949228in}{2.628878in}}%
\pgfpathlineto{\pgfqpoint{3.938085in}{2.625457in}}%
\pgfpathlineto{\pgfqpoint{3.926937in}{2.622033in}}%
\pgfpathlineto{\pgfqpoint{3.933338in}{2.608057in}}%
\pgfpathlineto{\pgfqpoint{3.939740in}{2.594070in}}%
\pgfpathlineto{\pgfqpoint{3.946145in}{2.580074in}}%
\pgfpathlineto{\pgfqpoint{3.952553in}{2.566067in}}%
\pgfpathclose%
\pgfusepath{stroke,fill}%
\end{pgfscope}%
\begin{pgfscope}%
\pgfpathrectangle{\pgfqpoint{0.887500in}{0.275000in}}{\pgfqpoint{4.225000in}{4.225000in}}%
\pgfusepath{clip}%
\pgfsetbuttcap%
\pgfsetroundjoin%
\definecolor{currentfill}{rgb}{0.144759,0.519093,0.556572}%
\pgfsetfillcolor{currentfill}%
\pgfsetfillopacity{0.700000}%
\pgfsetlinewidth{0.501875pt}%
\definecolor{currentstroke}{rgb}{1.000000,1.000000,1.000000}%
\pgfsetstrokecolor{currentstroke}%
\pgfsetstrokeopacity{0.500000}%
\pgfsetdash{}{0pt}%
\pgfpathmoveto{\pgfqpoint{4.254313in}{2.315978in}}%
\pgfpathlineto{\pgfqpoint{4.265383in}{2.319201in}}%
\pgfpathlineto{\pgfqpoint{4.276448in}{2.322466in}}%
\pgfpathlineto{\pgfqpoint{4.287511in}{2.325787in}}%
\pgfpathlineto{\pgfqpoint{4.298569in}{2.329178in}}%
\pgfpathlineto{\pgfqpoint{4.309625in}{2.332637in}}%
\pgfpathlineto{\pgfqpoint{4.303184in}{2.347544in}}%
\pgfpathlineto{\pgfqpoint{4.296742in}{2.362369in}}%
\pgfpathlineto{\pgfqpoint{4.290299in}{2.377088in}}%
\pgfpathlineto{\pgfqpoint{4.283856in}{2.391707in}}%
\pgfpathlineto{\pgfqpoint{4.277413in}{2.406236in}}%
\pgfpathlineto{\pgfqpoint{4.266363in}{2.402939in}}%
\pgfpathlineto{\pgfqpoint{4.255307in}{2.399621in}}%
\pgfpathlineto{\pgfqpoint{4.244246in}{2.396280in}}%
\pgfpathlineto{\pgfqpoint{4.233178in}{2.392920in}}%
\pgfpathlineto{\pgfqpoint{4.222105in}{2.389547in}}%
\pgfpathlineto{\pgfqpoint{4.228547in}{2.375023in}}%
\pgfpathlineto{\pgfqpoint{4.234989in}{2.360409in}}%
\pgfpathlineto{\pgfqpoint{4.241431in}{2.345695in}}%
\pgfpathlineto{\pgfqpoint{4.247872in}{2.330877in}}%
\pgfpathclose%
\pgfusepath{stroke,fill}%
\end{pgfscope}%
\begin{pgfscope}%
\pgfpathrectangle{\pgfqpoint{0.887500in}{0.275000in}}{\pgfqpoint{4.225000in}{4.225000in}}%
\pgfusepath{clip}%
\pgfsetbuttcap%
\pgfsetroundjoin%
\definecolor{currentfill}{rgb}{0.129933,0.559582,0.551864}%
\pgfsetfillcolor{currentfill}%
\pgfsetfillopacity{0.700000}%
\pgfsetlinewidth{0.501875pt}%
\definecolor{currentstroke}{rgb}{1.000000,1.000000,1.000000}%
\pgfsetstrokecolor{currentstroke}%
\pgfsetstrokeopacity{0.500000}%
\pgfsetdash{}{0pt}%
\pgfpathmoveto{\pgfqpoint{1.577051in}{2.412503in}}%
\pgfpathlineto{\pgfqpoint{1.588792in}{2.415911in}}%
\pgfpathlineto{\pgfqpoint{1.600526in}{2.419307in}}%
\pgfpathlineto{\pgfqpoint{1.612256in}{2.422691in}}%
\pgfpathlineto{\pgfqpoint{1.623980in}{2.426065in}}%
\pgfpathlineto{\pgfqpoint{1.635698in}{2.429429in}}%
\pgfpathlineto{\pgfqpoint{1.629947in}{2.437663in}}%
\pgfpathlineto{\pgfqpoint{1.624200in}{2.445877in}}%
\pgfpathlineto{\pgfqpoint{1.618458in}{2.454071in}}%
\pgfpathlineto{\pgfqpoint{1.612719in}{2.462247in}}%
\pgfpathlineto{\pgfqpoint{1.606986in}{2.470407in}}%
\pgfpathlineto{\pgfqpoint{1.595280in}{2.467058in}}%
\pgfpathlineto{\pgfqpoint{1.583569in}{2.463699in}}%
\pgfpathlineto{\pgfqpoint{1.571852in}{2.460330in}}%
\pgfpathlineto{\pgfqpoint{1.560130in}{2.456949in}}%
\pgfpathlineto{\pgfqpoint{1.548403in}{2.453555in}}%
\pgfpathlineto{\pgfqpoint{1.554124in}{2.445382in}}%
\pgfpathlineto{\pgfqpoint{1.559849in}{2.437192in}}%
\pgfpathlineto{\pgfqpoint{1.565579in}{2.428982in}}%
\pgfpathlineto{\pgfqpoint{1.571313in}{2.420753in}}%
\pgfpathclose%
\pgfusepath{stroke,fill}%
\end{pgfscope}%
\begin{pgfscope}%
\pgfpathrectangle{\pgfqpoint{0.887500in}{0.275000in}}{\pgfqpoint{4.225000in}{4.225000in}}%
\pgfusepath{clip}%
\pgfsetbuttcap%
\pgfsetroundjoin%
\definecolor{currentfill}{rgb}{0.188923,0.410910,0.556326}%
\pgfsetfillcolor{currentfill}%
\pgfsetfillopacity{0.700000}%
\pgfsetlinewidth{0.501875pt}%
\definecolor{currentstroke}{rgb}{1.000000,1.000000,1.000000}%
\pgfsetstrokecolor{currentstroke}%
\pgfsetstrokeopacity{0.500000}%
\pgfsetdash{}{0pt}%
\pgfpathmoveto{\pgfqpoint{4.549819in}{2.084357in}}%
\pgfpathlineto{\pgfqpoint{4.560818in}{2.087688in}}%
\pgfpathlineto{\pgfqpoint{4.571811in}{2.091016in}}%
\pgfpathlineto{\pgfqpoint{4.582798in}{2.094345in}}%
\pgfpathlineto{\pgfqpoint{4.593781in}{2.097673in}}%
\pgfpathlineto{\pgfqpoint{4.604757in}{2.101003in}}%
\pgfpathlineto{\pgfqpoint{4.598250in}{2.115548in}}%
\pgfpathlineto{\pgfqpoint{4.591745in}{2.130103in}}%
\pgfpathlineto{\pgfqpoint{4.585243in}{2.144660in}}%
\pgfpathlineto{\pgfqpoint{4.578743in}{2.159211in}}%
\pgfpathlineto{\pgfqpoint{4.572245in}{2.173746in}}%
\pgfpathlineto{\pgfqpoint{4.561266in}{2.170332in}}%
\pgfpathlineto{\pgfqpoint{4.550281in}{2.166894in}}%
\pgfpathlineto{\pgfqpoint{4.539289in}{2.163426in}}%
\pgfpathlineto{\pgfqpoint{4.528290in}{2.159925in}}%
\pgfpathlineto{\pgfqpoint{4.517285in}{2.156386in}}%
\pgfpathlineto{\pgfqpoint{4.523788in}{2.142023in}}%
\pgfpathlineto{\pgfqpoint{4.530293in}{2.127636in}}%
\pgfpathlineto{\pgfqpoint{4.536800in}{2.113228in}}%
\pgfpathlineto{\pgfqpoint{4.543308in}{2.098801in}}%
\pgfpathclose%
\pgfusepath{stroke,fill}%
\end{pgfscope}%
\begin{pgfscope}%
\pgfpathrectangle{\pgfqpoint{0.887500in}{0.275000in}}{\pgfqpoint{4.225000in}{4.225000in}}%
\pgfusepath{clip}%
\pgfsetbuttcap%
\pgfsetroundjoin%
\definecolor{currentfill}{rgb}{0.319809,0.770914,0.411152}%
\pgfsetfillcolor{currentfill}%
\pgfsetfillopacity{0.700000}%
\pgfsetlinewidth{0.501875pt}%
\definecolor{currentstroke}{rgb}{1.000000,1.000000,1.000000}%
\pgfsetstrokecolor{currentstroke}%
\pgfsetstrokeopacity{0.500000}%
\pgfsetdash{}{0pt}%
\pgfpathmoveto{\pgfqpoint{3.487508in}{2.876130in}}%
\pgfpathlineto{\pgfqpoint{3.498779in}{2.879899in}}%
\pgfpathlineto{\pgfqpoint{3.510045in}{2.883624in}}%
\pgfpathlineto{\pgfqpoint{3.521305in}{2.887306in}}%
\pgfpathlineto{\pgfqpoint{3.532559in}{2.890946in}}%
\pgfpathlineto{\pgfqpoint{3.543808in}{2.894545in}}%
\pgfpathlineto{\pgfqpoint{3.537477in}{2.907878in}}%
\pgfpathlineto{\pgfqpoint{3.531149in}{2.921172in}}%
\pgfpathlineto{\pgfqpoint{3.524822in}{2.934422in}}%
\pgfpathlineto{\pgfqpoint{3.518498in}{2.947625in}}%
\pgfpathlineto{\pgfqpoint{3.512176in}{2.960777in}}%
\pgfpathlineto{\pgfqpoint{3.500931in}{2.957162in}}%
\pgfpathlineto{\pgfqpoint{3.489681in}{2.953505in}}%
\pgfpathlineto{\pgfqpoint{3.478425in}{2.949809in}}%
\pgfpathlineto{\pgfqpoint{3.467163in}{2.946071in}}%
\pgfpathlineto{\pgfqpoint{3.455895in}{2.942289in}}%
\pgfpathlineto{\pgfqpoint{3.462213in}{2.929158in}}%
\pgfpathlineto{\pgfqpoint{3.468534in}{2.915973in}}%
\pgfpathlineto{\pgfqpoint{3.474856in}{2.902738in}}%
\pgfpathlineto{\pgfqpoint{3.481181in}{2.889455in}}%
\pgfpathclose%
\pgfusepath{stroke,fill}%
\end{pgfscope}%
\begin{pgfscope}%
\pgfpathrectangle{\pgfqpoint{0.887500in}{0.275000in}}{\pgfqpoint{4.225000in}{4.225000in}}%
\pgfusepath{clip}%
\pgfsetbuttcap%
\pgfsetroundjoin%
\definecolor{currentfill}{rgb}{0.440137,0.811138,0.340967}%
\pgfsetfillcolor{currentfill}%
\pgfsetfillopacity{0.700000}%
\pgfsetlinewidth{0.501875pt}%
\definecolor{currentstroke}{rgb}{1.000000,1.000000,1.000000}%
\pgfsetstrokecolor{currentstroke}%
\pgfsetstrokeopacity{0.500000}%
\pgfsetdash{}{0pt}%
\pgfpathmoveto{\pgfqpoint{3.078343in}{2.965803in}}%
\pgfpathlineto{\pgfqpoint{3.089716in}{2.982121in}}%
\pgfpathlineto{\pgfqpoint{3.101089in}{2.995591in}}%
\pgfpathlineto{\pgfqpoint{3.112458in}{3.006760in}}%
\pgfpathlineto{\pgfqpoint{3.123823in}{3.016174in}}%
\pgfpathlineto{\pgfqpoint{3.135184in}{3.024383in}}%
\pgfpathlineto{\pgfqpoint{3.128931in}{3.036079in}}%
\pgfpathlineto{\pgfqpoint{3.122680in}{3.047304in}}%
\pgfpathlineto{\pgfqpoint{3.116431in}{3.058071in}}%
\pgfpathlineto{\pgfqpoint{3.110185in}{3.068389in}}%
\pgfpathlineto{\pgfqpoint{3.103942in}{3.078270in}}%
\pgfpathlineto{\pgfqpoint{3.092592in}{3.068918in}}%
\pgfpathlineto{\pgfqpoint{3.081239in}{3.058438in}}%
\pgfpathlineto{\pgfqpoint{3.069884in}{3.046395in}}%
\pgfpathlineto{\pgfqpoint{3.058528in}{3.032353in}}%
\pgfpathlineto{\pgfqpoint{3.047173in}{3.015875in}}%
\pgfpathlineto{\pgfqpoint{3.053399in}{3.008158in}}%
\pgfpathlineto{\pgfqpoint{3.059630in}{2.999216in}}%
\pgfpathlineto{\pgfqpoint{3.065865in}{2.989126in}}%
\pgfpathlineto{\pgfqpoint{3.072102in}{2.977962in}}%
\pgfpathclose%
\pgfusepath{stroke,fill}%
\end{pgfscope}%
\begin{pgfscope}%
\pgfpathrectangle{\pgfqpoint{0.887500in}{0.275000in}}{\pgfqpoint{4.225000in}{4.225000in}}%
\pgfusepath{clip}%
\pgfsetbuttcap%
\pgfsetroundjoin%
\definecolor{currentfill}{rgb}{0.162142,0.474838,0.558140}%
\pgfsetfillcolor{currentfill}%
\pgfsetfillopacity{0.700000}%
\pgfsetlinewidth{0.501875pt}%
\definecolor{currentstroke}{rgb}{1.000000,1.000000,1.000000}%
\pgfsetstrokecolor{currentstroke}%
\pgfsetstrokeopacity{0.500000}%
\pgfsetdash{}{0pt}%
\pgfpathmoveto{\pgfqpoint{2.540605in}{2.229431in}}%
\pgfpathlineto{\pgfqpoint{2.552108in}{2.232954in}}%
\pgfpathlineto{\pgfqpoint{2.563604in}{2.236508in}}%
\pgfpathlineto{\pgfqpoint{2.575095in}{2.240088in}}%
\pgfpathlineto{\pgfqpoint{2.586580in}{2.243667in}}%
\pgfpathlineto{\pgfqpoint{2.598060in}{2.247214in}}%
\pgfpathlineto{\pgfqpoint{2.591976in}{2.256098in}}%
\pgfpathlineto{\pgfqpoint{2.585897in}{2.264958in}}%
\pgfpathlineto{\pgfqpoint{2.579822in}{2.273793in}}%
\pgfpathlineto{\pgfqpoint{2.573751in}{2.282605in}}%
\pgfpathlineto{\pgfqpoint{2.567684in}{2.291393in}}%
\pgfpathlineto{\pgfqpoint{2.556215in}{2.287845in}}%
\pgfpathlineto{\pgfqpoint{2.544742in}{2.284267in}}%
\pgfpathlineto{\pgfqpoint{2.533262in}{2.280691in}}%
\pgfpathlineto{\pgfqpoint{2.521777in}{2.277147in}}%
\pgfpathlineto{\pgfqpoint{2.510286in}{2.273641in}}%
\pgfpathlineto{\pgfqpoint{2.516341in}{2.264849in}}%
\pgfpathlineto{\pgfqpoint{2.522401in}{2.256033in}}%
\pgfpathlineto{\pgfqpoint{2.528465in}{2.247192in}}%
\pgfpathlineto{\pgfqpoint{2.534533in}{2.238325in}}%
\pgfpathclose%
\pgfusepath{stroke,fill}%
\end{pgfscope}%
\begin{pgfscope}%
\pgfpathrectangle{\pgfqpoint{0.887500in}{0.275000in}}{\pgfqpoint{4.225000in}{4.225000in}}%
\pgfusepath{clip}%
\pgfsetbuttcap%
\pgfsetroundjoin%
\definecolor{currentfill}{rgb}{0.140536,0.530132,0.555659}%
\pgfsetfillcolor{currentfill}%
\pgfsetfillopacity{0.700000}%
\pgfsetlinewidth{0.501875pt}%
\definecolor{currentstroke}{rgb}{1.000000,1.000000,1.000000}%
\pgfsetstrokecolor{currentstroke}%
\pgfsetstrokeopacity{0.500000}%
\pgfsetdash{}{0pt}%
\pgfpathmoveto{\pgfqpoint{1.898146in}{2.354266in}}%
\pgfpathlineto{\pgfqpoint{1.909809in}{2.357650in}}%
\pgfpathlineto{\pgfqpoint{1.921467in}{2.361024in}}%
\pgfpathlineto{\pgfqpoint{1.933120in}{2.364387in}}%
\pgfpathlineto{\pgfqpoint{1.944766in}{2.367740in}}%
\pgfpathlineto{\pgfqpoint{1.956407in}{2.371084in}}%
\pgfpathlineto{\pgfqpoint{1.950542in}{2.379544in}}%
\pgfpathlineto{\pgfqpoint{1.944680in}{2.387984in}}%
\pgfpathlineto{\pgfqpoint{1.938823in}{2.396404in}}%
\pgfpathlineto{\pgfqpoint{1.932970in}{2.404803in}}%
\pgfpathlineto{\pgfqpoint{1.927121in}{2.413183in}}%
\pgfpathlineto{\pgfqpoint{1.915492in}{2.409868in}}%
\pgfpathlineto{\pgfqpoint{1.903857in}{2.406545in}}%
\pgfpathlineto{\pgfqpoint{1.892216in}{2.403213in}}%
\pgfpathlineto{\pgfqpoint{1.880570in}{2.399871in}}%
\pgfpathlineto{\pgfqpoint{1.868919in}{2.396519in}}%
\pgfpathlineto{\pgfqpoint{1.874756in}{2.388109in}}%
\pgfpathlineto{\pgfqpoint{1.880597in}{2.379678in}}%
\pgfpathlineto{\pgfqpoint{1.886442in}{2.371228in}}%
\pgfpathlineto{\pgfqpoint{1.892292in}{2.362757in}}%
\pgfpathclose%
\pgfusepath{stroke,fill}%
\end{pgfscope}%
\begin{pgfscope}%
\pgfpathrectangle{\pgfqpoint{0.887500in}{0.275000in}}{\pgfqpoint{4.225000in}{4.225000in}}%
\pgfusepath{clip}%
\pgfsetbuttcap%
\pgfsetroundjoin%
\definecolor{currentfill}{rgb}{0.468053,0.818921,0.323998}%
\pgfsetfillcolor{currentfill}%
\pgfsetfillopacity{0.700000}%
\pgfsetlinewidth{0.501875pt}%
\definecolor{currentstroke}{rgb}{1.000000,1.000000,1.000000}%
\pgfsetstrokecolor{currentstroke}%
\pgfsetstrokeopacity{0.500000}%
\pgfsetdash{}{0pt}%
\pgfpathmoveto{\pgfqpoint{3.223285in}{3.000530in}}%
\pgfpathlineto{\pgfqpoint{3.234635in}{3.008236in}}%
\pgfpathlineto{\pgfqpoint{3.245980in}{3.015477in}}%
\pgfpathlineto{\pgfqpoint{3.257319in}{3.022143in}}%
\pgfpathlineto{\pgfqpoint{3.268650in}{3.028125in}}%
\pgfpathlineto{\pgfqpoint{3.279974in}{3.033423in}}%
\pgfpathlineto{\pgfqpoint{3.273693in}{3.046126in}}%
\pgfpathlineto{\pgfqpoint{3.267413in}{3.058615in}}%
\pgfpathlineto{\pgfqpoint{3.261135in}{3.070918in}}%
\pgfpathlineto{\pgfqpoint{3.254859in}{3.083061in}}%
\pgfpathlineto{\pgfqpoint{3.248586in}{3.095072in}}%
\pgfpathlineto{\pgfqpoint{3.237268in}{3.089596in}}%
\pgfpathlineto{\pgfqpoint{3.225944in}{3.083517in}}%
\pgfpathlineto{\pgfqpoint{3.214613in}{3.076855in}}%
\pgfpathlineto{\pgfqpoint{3.203277in}{3.069739in}}%
\pgfpathlineto{\pgfqpoint{3.191936in}{3.062301in}}%
\pgfpathlineto{\pgfqpoint{3.198202in}{3.050415in}}%
\pgfpathlineto{\pgfqpoint{3.204470in}{3.038333in}}%
\pgfpathlineto{\pgfqpoint{3.210740in}{3.026016in}}%
\pgfpathlineto{\pgfqpoint{3.217011in}{3.013428in}}%
\pgfpathclose%
\pgfusepath{stroke,fill}%
\end{pgfscope}%
\begin{pgfscope}%
\pgfpathrectangle{\pgfqpoint{0.887500in}{0.275000in}}{\pgfqpoint{4.225000in}{4.225000in}}%
\pgfusepath{clip}%
\pgfsetbuttcap%
\pgfsetroundjoin%
\definecolor{currentfill}{rgb}{0.150476,0.504369,0.557430}%
\pgfsetfillcolor{currentfill}%
\pgfsetfillopacity{0.700000}%
\pgfsetlinewidth{0.501875pt}%
\definecolor{currentstroke}{rgb}{1.000000,1.000000,1.000000}%
\pgfsetstrokecolor{currentstroke}%
\pgfsetstrokeopacity{0.500000}%
\pgfsetdash{}{0pt}%
\pgfpathmoveto{\pgfqpoint{2.219356in}{2.293123in}}%
\pgfpathlineto{\pgfqpoint{2.230939in}{2.296574in}}%
\pgfpathlineto{\pgfqpoint{2.242516in}{2.300014in}}%
\pgfpathlineto{\pgfqpoint{2.254088in}{2.303442in}}%
\pgfpathlineto{\pgfqpoint{2.265655in}{2.306856in}}%
\pgfpathlineto{\pgfqpoint{2.277216in}{2.310255in}}%
\pgfpathlineto{\pgfqpoint{2.271240in}{2.318894in}}%
\pgfpathlineto{\pgfqpoint{2.265269in}{2.327512in}}%
\pgfpathlineto{\pgfqpoint{2.259301in}{2.336109in}}%
\pgfpathlineto{\pgfqpoint{2.253338in}{2.344687in}}%
\pgfpathlineto{\pgfqpoint{2.247379in}{2.353246in}}%
\pgfpathlineto{\pgfqpoint{2.235830in}{2.349878in}}%
\pgfpathlineto{\pgfqpoint{2.224274in}{2.346496in}}%
\pgfpathlineto{\pgfqpoint{2.212714in}{2.343102in}}%
\pgfpathlineto{\pgfqpoint{2.201148in}{2.339697in}}%
\pgfpathlineto{\pgfqpoint{2.189576in}{2.336281in}}%
\pgfpathlineto{\pgfqpoint{2.195524in}{2.327690in}}%
\pgfpathlineto{\pgfqpoint{2.201475in}{2.319079in}}%
\pgfpathlineto{\pgfqpoint{2.207431in}{2.310448in}}%
\pgfpathlineto{\pgfqpoint{2.213391in}{2.301797in}}%
\pgfpathclose%
\pgfusepath{stroke,fill}%
\end{pgfscope}%
\begin{pgfscope}%
\pgfpathrectangle{\pgfqpoint{0.887500in}{0.275000in}}{\pgfqpoint{4.225000in}{4.225000in}}%
\pgfusepath{clip}%
\pgfsetbuttcap%
\pgfsetroundjoin%
\definecolor{currentfill}{rgb}{0.369214,0.788888,0.382914}%
\pgfsetfillcolor{currentfill}%
\pgfsetfillopacity{0.700000}%
\pgfsetlinewidth{0.501875pt}%
\definecolor{currentstroke}{rgb}{1.000000,1.000000,1.000000}%
\pgfsetstrokecolor{currentstroke}%
\pgfsetstrokeopacity{0.500000}%
\pgfsetdash{}{0pt}%
\pgfpathmoveto{\pgfqpoint{3.399472in}{2.922569in}}%
\pgfpathlineto{\pgfqpoint{3.410768in}{2.926634in}}%
\pgfpathlineto{\pgfqpoint{3.422058in}{2.930634in}}%
\pgfpathlineto{\pgfqpoint{3.433343in}{2.934574in}}%
\pgfpathlineto{\pgfqpoint{3.444622in}{2.938458in}}%
\pgfpathlineto{\pgfqpoint{3.455895in}{2.942289in}}%
\pgfpathlineto{\pgfqpoint{3.449579in}{2.955368in}}%
\pgfpathlineto{\pgfqpoint{3.443266in}{2.968401in}}%
\pgfpathlineto{\pgfqpoint{3.436955in}{2.981394in}}%
\pgfpathlineto{\pgfqpoint{3.430646in}{2.994353in}}%
\pgfpathlineto{\pgfqpoint{3.424339in}{3.007282in}}%
\pgfpathlineto{\pgfqpoint{3.413072in}{3.003544in}}%
\pgfpathlineto{\pgfqpoint{3.401798in}{2.999800in}}%
\pgfpathlineto{\pgfqpoint{3.390520in}{2.996050in}}%
\pgfpathlineto{\pgfqpoint{3.379236in}{2.992295in}}%
\pgfpathlineto{\pgfqpoint{3.367947in}{2.988537in}}%
\pgfpathlineto{\pgfqpoint{3.374247in}{2.975382in}}%
\pgfpathlineto{\pgfqpoint{3.380549in}{2.962200in}}%
\pgfpathlineto{\pgfqpoint{3.386854in}{2.948997in}}%
\pgfpathlineto{\pgfqpoint{3.393162in}{2.935784in}}%
\pgfpathclose%
\pgfusepath{stroke,fill}%
\end{pgfscope}%
\begin{pgfscope}%
\pgfpathrectangle{\pgfqpoint{0.887500in}{0.275000in}}{\pgfqpoint{4.225000in}{4.225000in}}%
\pgfusepath{clip}%
\pgfsetbuttcap%
\pgfsetroundjoin%
\definecolor{currentfill}{rgb}{0.182256,0.426184,0.557120}%
\pgfsetfillcolor{currentfill}%
\pgfsetfillopacity{0.700000}%
\pgfsetlinewidth{0.501875pt}%
\definecolor{currentstroke}{rgb}{1.000000,1.000000,1.000000}%
\pgfsetstrokecolor{currentstroke}%
\pgfsetstrokeopacity{0.500000}%
\pgfsetdash{}{0pt}%
\pgfpathmoveto{\pgfqpoint{2.949919in}{2.122322in}}%
\pgfpathlineto{\pgfqpoint{2.961327in}{2.123421in}}%
\pgfpathlineto{\pgfqpoint{2.972723in}{2.126606in}}%
\pgfpathlineto{\pgfqpoint{2.984108in}{2.132828in}}%
\pgfpathlineto{\pgfqpoint{2.995482in}{2.143036in}}%
\pgfpathlineto{\pgfqpoint{3.006849in}{2.158026in}}%
\pgfpathlineto{\pgfqpoint{3.000630in}{2.167483in}}%
\pgfpathlineto{\pgfqpoint{2.994416in}{2.176921in}}%
\pgfpathlineto{\pgfqpoint{2.988205in}{2.186344in}}%
\pgfpathlineto{\pgfqpoint{2.981998in}{2.195753in}}%
\pgfpathlineto{\pgfqpoint{2.975795in}{2.205153in}}%
\pgfpathlineto{\pgfqpoint{2.964450in}{2.189028in}}%
\pgfpathlineto{\pgfqpoint{2.953093in}{2.178178in}}%
\pgfpathlineto{\pgfqpoint{2.941721in}{2.171727in}}%
\pgfpathlineto{\pgfqpoint{2.930335in}{2.168625in}}%
\pgfpathlineto{\pgfqpoint{2.918935in}{2.167826in}}%
\pgfpathlineto{\pgfqpoint{2.925124in}{2.158764in}}%
\pgfpathlineto{\pgfqpoint{2.931317in}{2.149698in}}%
\pgfpathlineto{\pgfqpoint{2.937514in}{2.140611in}}%
\pgfpathlineto{\pgfqpoint{2.943714in}{2.131491in}}%
\pgfpathclose%
\pgfusepath{stroke,fill}%
\end{pgfscope}%
\begin{pgfscope}%
\pgfpathrectangle{\pgfqpoint{0.887500in}{0.275000in}}{\pgfqpoint{4.225000in}{4.225000in}}%
\pgfusepath{clip}%
\pgfsetbuttcap%
\pgfsetroundjoin%
\definecolor{currentfill}{rgb}{0.132268,0.655014,0.519661}%
\pgfsetfillcolor{currentfill}%
\pgfsetfillopacity{0.700000}%
\pgfsetlinewidth{0.501875pt}%
\definecolor{currentstroke}{rgb}{1.000000,1.000000,1.000000}%
\pgfsetstrokecolor{currentstroke}%
\pgfsetstrokeopacity{0.500000}%
\pgfsetdash{}{0pt}%
\pgfpathmoveto{\pgfqpoint{3.871117in}{2.604966in}}%
\pgfpathlineto{\pgfqpoint{3.882292in}{2.608371in}}%
\pgfpathlineto{\pgfqpoint{3.893461in}{2.611778in}}%
\pgfpathlineto{\pgfqpoint{3.904625in}{2.615191in}}%
\pgfpathlineto{\pgfqpoint{3.915784in}{2.618610in}}%
\pgfpathlineto{\pgfqpoint{3.926937in}{2.622033in}}%
\pgfpathlineto{\pgfqpoint{3.920539in}{2.635995in}}%
\pgfpathlineto{\pgfqpoint{3.914143in}{2.649941in}}%
\pgfpathlineto{\pgfqpoint{3.907750in}{2.663868in}}%
\pgfpathlineto{\pgfqpoint{3.901358in}{2.677773in}}%
\pgfpathlineto{\pgfqpoint{3.894968in}{2.691654in}}%
\pgfpathlineto{\pgfqpoint{3.883817in}{2.688188in}}%
\pgfpathlineto{\pgfqpoint{3.872660in}{2.684733in}}%
\pgfpathlineto{\pgfqpoint{3.861498in}{2.681293in}}%
\pgfpathlineto{\pgfqpoint{3.850331in}{2.677868in}}%
\pgfpathlineto{\pgfqpoint{3.839158in}{2.674456in}}%
\pgfpathlineto{\pgfqpoint{3.845546in}{2.660594in}}%
\pgfpathlineto{\pgfqpoint{3.851935in}{2.646713in}}%
\pgfpathlineto{\pgfqpoint{3.858327in}{2.632814in}}%
\pgfpathlineto{\pgfqpoint{3.864721in}{2.618898in}}%
\pgfpathclose%
\pgfusepath{stroke,fill}%
\end{pgfscope}%
\begin{pgfscope}%
\pgfpathrectangle{\pgfqpoint{0.887500in}{0.275000in}}{\pgfqpoint{4.225000in}{4.225000in}}%
\pgfusepath{clip}%
\pgfsetbuttcap%
\pgfsetroundjoin%
\definecolor{currentfill}{rgb}{0.421908,0.805774,0.351910}%
\pgfsetfillcolor{currentfill}%
\pgfsetfillopacity{0.700000}%
\pgfsetlinewidth{0.501875pt}%
\definecolor{currentstroke}{rgb}{1.000000,1.000000,1.000000}%
\pgfsetstrokecolor{currentstroke}%
\pgfsetstrokeopacity{0.500000}%
\pgfsetdash{}{0pt}%
\pgfpathmoveto{\pgfqpoint{3.311405in}{2.966944in}}%
\pgfpathlineto{\pgfqpoint{3.322728in}{2.971991in}}%
\pgfpathlineto{\pgfqpoint{3.334043in}{2.976567in}}%
\pgfpathlineto{\pgfqpoint{3.345351in}{2.980778in}}%
\pgfpathlineto{\pgfqpoint{3.356652in}{2.984733in}}%
\pgfpathlineto{\pgfqpoint{3.367947in}{2.988537in}}%
\pgfpathlineto{\pgfqpoint{3.361650in}{3.001656in}}%
\pgfpathlineto{\pgfqpoint{3.355354in}{3.014729in}}%
\pgfpathlineto{\pgfqpoint{3.349062in}{3.027748in}}%
\pgfpathlineto{\pgfqpoint{3.342771in}{3.040705in}}%
\pgfpathlineto{\pgfqpoint{3.336483in}{3.053592in}}%
\pgfpathlineto{\pgfqpoint{3.325194in}{3.050019in}}%
\pgfpathlineto{\pgfqpoint{3.313900in}{3.046330in}}%
\pgfpathlineto{\pgfqpoint{3.302599in}{3.042411in}}%
\pgfpathlineto{\pgfqpoint{3.291290in}{3.038147in}}%
\pgfpathlineto{\pgfqpoint{3.279974in}{3.033423in}}%
\pgfpathlineto{\pgfqpoint{3.286257in}{3.020485in}}%
\pgfpathlineto{\pgfqpoint{3.292541in}{3.007328in}}%
\pgfpathlineto{\pgfqpoint{3.298827in}{2.993993in}}%
\pgfpathlineto{\pgfqpoint{3.305115in}{2.980519in}}%
\pgfpathclose%
\pgfusepath{stroke,fill}%
\end{pgfscope}%
\begin{pgfscope}%
\pgfpathrectangle{\pgfqpoint{0.887500in}{0.275000in}}{\pgfqpoint{4.225000in}{4.225000in}}%
\pgfusepath{clip}%
\pgfsetbuttcap%
\pgfsetroundjoin%
\definecolor{currentfill}{rgb}{0.133743,0.548535,0.553541}%
\pgfsetfillcolor{currentfill}%
\pgfsetfillopacity{0.700000}%
\pgfsetlinewidth{0.501875pt}%
\definecolor{currentstroke}{rgb}{1.000000,1.000000,1.000000}%
\pgfsetstrokecolor{currentstroke}%
\pgfsetstrokeopacity{0.500000}%
\pgfsetdash{}{0pt}%
\pgfpathmoveto{\pgfqpoint{4.166654in}{2.372631in}}%
\pgfpathlineto{\pgfqpoint{4.177755in}{2.376009in}}%
\pgfpathlineto{\pgfqpoint{4.188850in}{2.379391in}}%
\pgfpathlineto{\pgfqpoint{4.199941in}{2.382778in}}%
\pgfpathlineto{\pgfqpoint{4.211025in}{2.386165in}}%
\pgfpathlineto{\pgfqpoint{4.222105in}{2.389547in}}%
\pgfpathlineto{\pgfqpoint{4.215662in}{2.403991in}}%
\pgfpathlineto{\pgfqpoint{4.209220in}{2.418365in}}%
\pgfpathlineto{\pgfqpoint{4.202779in}{2.432677in}}%
\pgfpathlineto{\pgfqpoint{4.196339in}{2.446938in}}%
\pgfpathlineto{\pgfqpoint{4.189900in}{2.461157in}}%
\pgfpathlineto{\pgfqpoint{4.178823in}{2.457788in}}%
\pgfpathlineto{\pgfqpoint{4.167740in}{2.454418in}}%
\pgfpathlineto{\pgfqpoint{4.156652in}{2.451050in}}%
\pgfpathlineto{\pgfqpoint{4.145559in}{2.447685in}}%
\pgfpathlineto{\pgfqpoint{4.134460in}{2.444324in}}%
\pgfpathlineto{\pgfqpoint{4.140896in}{2.430054in}}%
\pgfpathlineto{\pgfqpoint{4.147333in}{2.415751in}}%
\pgfpathlineto{\pgfqpoint{4.153772in}{2.401414in}}%
\pgfpathlineto{\pgfqpoint{4.160212in}{2.387041in}}%
\pgfpathclose%
\pgfusepath{stroke,fill}%
\end{pgfscope}%
\begin{pgfscope}%
\pgfpathrectangle{\pgfqpoint{0.887500in}{0.275000in}}{\pgfqpoint{4.225000in}{4.225000in}}%
\pgfusepath{clip}%
\pgfsetbuttcap%
\pgfsetroundjoin%
\definecolor{currentfill}{rgb}{0.177423,0.437527,0.557565}%
\pgfsetfillcolor{currentfill}%
\pgfsetfillopacity{0.700000}%
\pgfsetlinewidth{0.501875pt}%
\definecolor{currentstroke}{rgb}{1.000000,1.000000,1.000000}%
\pgfsetstrokecolor{currentstroke}%
\pgfsetstrokeopacity{0.500000}%
\pgfsetdash{}{0pt}%
\pgfpathmoveto{\pgfqpoint{4.462153in}{2.137959in}}%
\pgfpathlineto{\pgfqpoint{4.473194in}{2.141756in}}%
\pgfpathlineto{\pgfqpoint{4.484227in}{2.145493in}}%
\pgfpathlineto{\pgfqpoint{4.495254in}{2.149174in}}%
\pgfpathlineto{\pgfqpoint{4.506273in}{2.152804in}}%
\pgfpathlineto{\pgfqpoint{4.517285in}{2.156386in}}%
\pgfpathlineto{\pgfqpoint{4.510783in}{2.170721in}}%
\pgfpathlineto{\pgfqpoint{4.504283in}{2.185028in}}%
\pgfpathlineto{\pgfqpoint{4.497785in}{2.199302in}}%
\pgfpathlineto{\pgfqpoint{4.491287in}{2.213542in}}%
\pgfpathlineto{\pgfqpoint{4.484791in}{2.227744in}}%
\pgfpathlineto{\pgfqpoint{4.473772in}{2.223918in}}%
\pgfpathlineto{\pgfqpoint{4.462744in}{2.220006in}}%
\pgfpathlineto{\pgfqpoint{4.451708in}{2.216001in}}%
\pgfpathlineto{\pgfqpoint{4.440664in}{2.211896in}}%
\pgfpathlineto{\pgfqpoint{4.429611in}{2.207682in}}%
\pgfpathlineto{\pgfqpoint{4.436112in}{2.193694in}}%
\pgfpathlineto{\pgfqpoint{4.442617in}{2.179742in}}%
\pgfpathlineto{\pgfqpoint{4.449126in}{2.165813in}}%
\pgfpathlineto{\pgfqpoint{4.455638in}{2.151890in}}%
\pgfpathclose%
\pgfusepath{stroke,fill}%
\end{pgfscope}%
\begin{pgfscope}%
\pgfpathrectangle{\pgfqpoint{0.887500in}{0.275000in}}{\pgfqpoint{4.225000in}{4.225000in}}%
\pgfusepath{clip}%
\pgfsetbuttcap%
\pgfsetroundjoin%
\definecolor{currentfill}{rgb}{0.133743,0.548535,0.553541}%
\pgfsetfillcolor{currentfill}%
\pgfsetfillopacity{0.700000}%
\pgfsetlinewidth{0.501875pt}%
\definecolor{currentstroke}{rgb}{1.000000,1.000000,1.000000}%
\pgfsetstrokecolor{currentstroke}%
\pgfsetstrokeopacity{0.500000}%
\pgfsetdash{}{0pt}%
\pgfpathmoveto{\pgfqpoint{1.664522in}{2.387916in}}%
\pgfpathlineto{\pgfqpoint{1.676248in}{2.391293in}}%
\pgfpathlineto{\pgfqpoint{1.687967in}{2.394662in}}%
\pgfpathlineto{\pgfqpoint{1.699682in}{2.398027in}}%
\pgfpathlineto{\pgfqpoint{1.711390in}{2.401386in}}%
\pgfpathlineto{\pgfqpoint{1.723093in}{2.404743in}}%
\pgfpathlineto{\pgfqpoint{1.717306in}{2.413072in}}%
\pgfpathlineto{\pgfqpoint{1.711525in}{2.421377in}}%
\pgfpathlineto{\pgfqpoint{1.705747in}{2.429659in}}%
\pgfpathlineto{\pgfqpoint{1.699975in}{2.437918in}}%
\pgfpathlineto{\pgfqpoint{1.694206in}{2.446155in}}%
\pgfpathlineto{\pgfqpoint{1.682516in}{2.442819in}}%
\pgfpathlineto{\pgfqpoint{1.670820in}{2.439480in}}%
\pgfpathlineto{\pgfqpoint{1.659119in}{2.436136in}}%
\pgfpathlineto{\pgfqpoint{1.647411in}{2.432786in}}%
\pgfpathlineto{\pgfqpoint{1.635698in}{2.429429in}}%
\pgfpathlineto{\pgfqpoint{1.641454in}{2.421174in}}%
\pgfpathlineto{\pgfqpoint{1.647214in}{2.412896in}}%
\pgfpathlineto{\pgfqpoint{1.652979in}{2.404594in}}%
\pgfpathlineto{\pgfqpoint{1.658748in}{2.396268in}}%
\pgfpathclose%
\pgfusepath{stroke,fill}%
\end{pgfscope}%
\begin{pgfscope}%
\pgfpathrectangle{\pgfqpoint{0.887500in}{0.275000in}}{\pgfqpoint{4.225000in}{4.225000in}}%
\pgfusepath{clip}%
\pgfsetbuttcap%
\pgfsetroundjoin%
\definecolor{currentfill}{rgb}{0.153894,0.680203,0.504172}%
\pgfsetfillcolor{currentfill}%
\pgfsetfillopacity{0.700000}%
\pgfsetlinewidth{0.501875pt}%
\definecolor{currentstroke}{rgb}{1.000000,1.000000,1.000000}%
\pgfsetstrokecolor{currentstroke}%
\pgfsetstrokeopacity{0.500000}%
\pgfsetdash{}{0pt}%
\pgfpathmoveto{\pgfqpoint{3.783217in}{2.657502in}}%
\pgfpathlineto{\pgfqpoint{3.794416in}{2.660887in}}%
\pgfpathlineto{\pgfqpoint{3.805610in}{2.664272in}}%
\pgfpathlineto{\pgfqpoint{3.816798in}{2.667660in}}%
\pgfpathlineto{\pgfqpoint{3.827981in}{2.671054in}}%
\pgfpathlineto{\pgfqpoint{3.839158in}{2.674456in}}%
\pgfpathlineto{\pgfqpoint{3.832773in}{2.688297in}}%
\pgfpathlineto{\pgfqpoint{3.826390in}{2.702115in}}%
\pgfpathlineto{\pgfqpoint{3.820009in}{2.715910in}}%
\pgfpathlineto{\pgfqpoint{3.813631in}{2.729681in}}%
\pgfpathlineto{\pgfqpoint{3.807254in}{2.743426in}}%
\pgfpathlineto{\pgfqpoint{3.796080in}{2.740044in}}%
\pgfpathlineto{\pgfqpoint{3.784900in}{2.736676in}}%
\pgfpathlineto{\pgfqpoint{3.773715in}{2.733319in}}%
\pgfpathlineto{\pgfqpoint{3.762525in}{2.729971in}}%
\pgfpathlineto{\pgfqpoint{3.751330in}{2.726628in}}%
\pgfpathlineto{\pgfqpoint{3.757703in}{2.712863in}}%
\pgfpathlineto{\pgfqpoint{3.764079in}{2.699066in}}%
\pgfpathlineto{\pgfqpoint{3.770456in}{2.685238in}}%
\pgfpathlineto{\pgfqpoint{3.776836in}{2.671382in}}%
\pgfpathclose%
\pgfusepath{stroke,fill}%
\end{pgfscope}%
\begin{pgfscope}%
\pgfpathrectangle{\pgfqpoint{0.887500in}{0.275000in}}{\pgfqpoint{4.225000in}{4.225000in}}%
\pgfusepath{clip}%
\pgfsetbuttcap%
\pgfsetroundjoin%
\definecolor{currentfill}{rgb}{0.166617,0.463708,0.558119}%
\pgfsetfillcolor{currentfill}%
\pgfsetfillopacity{0.700000}%
\pgfsetlinewidth{0.501875pt}%
\definecolor{currentstroke}{rgb}{1.000000,1.000000,1.000000}%
\pgfsetstrokecolor{currentstroke}%
\pgfsetstrokeopacity{0.500000}%
\pgfsetdash{}{0pt}%
\pgfpathmoveto{\pgfqpoint{2.628540in}{2.202403in}}%
\pgfpathlineto{\pgfqpoint{2.640026in}{2.205904in}}%
\pgfpathlineto{\pgfqpoint{2.651507in}{2.209317in}}%
\pgfpathlineto{\pgfqpoint{2.662984in}{2.212614in}}%
\pgfpathlineto{\pgfqpoint{2.674456in}{2.215768in}}%
\pgfpathlineto{\pgfqpoint{2.685924in}{2.218800in}}%
\pgfpathlineto{\pgfqpoint{2.679808in}{2.227803in}}%
\pgfpathlineto{\pgfqpoint{2.673698in}{2.236777in}}%
\pgfpathlineto{\pgfqpoint{2.667591in}{2.245724in}}%
\pgfpathlineto{\pgfqpoint{2.661488in}{2.254644in}}%
\pgfpathlineto{\pgfqpoint{2.655390in}{2.263536in}}%
\pgfpathlineto{\pgfqpoint{2.643932in}{2.260523in}}%
\pgfpathlineto{\pgfqpoint{2.632471in}{2.257385in}}%
\pgfpathlineto{\pgfqpoint{2.621005in}{2.254102in}}%
\pgfpathlineto{\pgfqpoint{2.609535in}{2.250703in}}%
\pgfpathlineto{\pgfqpoint{2.598060in}{2.247214in}}%
\pgfpathlineto{\pgfqpoint{2.604148in}{2.238305in}}%
\pgfpathlineto{\pgfqpoint{2.610239in}{2.229370in}}%
\pgfpathlineto{\pgfqpoint{2.616335in}{2.220409in}}%
\pgfpathlineto{\pgfqpoint{2.622436in}{2.211420in}}%
\pgfpathclose%
\pgfusepath{stroke,fill}%
\end{pgfscope}%
\begin{pgfscope}%
\pgfpathrectangle{\pgfqpoint{0.887500in}{0.275000in}}{\pgfqpoint{4.225000in}{4.225000in}}%
\pgfusepath{clip}%
\pgfsetbuttcap%
\pgfsetroundjoin%
\definecolor{currentfill}{rgb}{0.166617,0.463708,0.558119}%
\pgfsetfillcolor{currentfill}%
\pgfsetfillopacity{0.700000}%
\pgfsetlinewidth{0.501875pt}%
\definecolor{currentstroke}{rgb}{1.000000,1.000000,1.000000}%
\pgfsetstrokecolor{currentstroke}%
\pgfsetstrokeopacity{0.500000}%
\pgfsetdash{}{0pt}%
\pgfpathmoveto{\pgfqpoint{4.374262in}{2.186382in}}%
\pgfpathlineto{\pgfqpoint{4.385336in}{2.190473in}}%
\pgfpathlineto{\pgfqpoint{4.396410in}{2.194710in}}%
\pgfpathlineto{\pgfqpoint{4.407481in}{2.199030in}}%
\pgfpathlineto{\pgfqpoint{4.418549in}{2.203375in}}%
\pgfpathlineto{\pgfqpoint{4.429611in}{2.207682in}}%
\pgfpathlineto{\pgfqpoint{4.423114in}{2.221723in}}%
\pgfpathlineto{\pgfqpoint{4.416622in}{2.235831in}}%
\pgfpathlineto{\pgfqpoint{4.410135in}{2.250007in}}%
\pgfpathlineto{\pgfqpoint{4.403652in}{2.264245in}}%
\pgfpathlineto{\pgfqpoint{4.397175in}{2.278535in}}%
\pgfpathlineto{\pgfqpoint{4.386121in}{2.274382in}}%
\pgfpathlineto{\pgfqpoint{4.375061in}{2.270206in}}%
\pgfpathlineto{\pgfqpoint{4.363998in}{2.266057in}}%
\pgfpathlineto{\pgfqpoint{4.352932in}{2.261986in}}%
\pgfpathlineto{\pgfqpoint{4.341865in}{2.258044in}}%
\pgfpathlineto{\pgfqpoint{4.348327in}{2.243347in}}%
\pgfpathlineto{\pgfqpoint{4.354797in}{2.228803in}}%
\pgfpathlineto{\pgfqpoint{4.361275in}{2.214444in}}%
\pgfpathlineto{\pgfqpoint{4.367764in}{2.200303in}}%
\pgfpathclose%
\pgfusepath{stroke,fill}%
\end{pgfscope}%
\begin{pgfscope}%
\pgfpathrectangle{\pgfqpoint{0.887500in}{0.275000in}}{\pgfqpoint{4.225000in}{4.225000in}}%
\pgfusepath{clip}%
\pgfsetbuttcap%
\pgfsetroundjoin%
\definecolor{currentfill}{rgb}{0.143343,0.522773,0.556295}%
\pgfsetfillcolor{currentfill}%
\pgfsetfillopacity{0.700000}%
\pgfsetlinewidth{0.501875pt}%
\definecolor{currentstroke}{rgb}{1.000000,1.000000,1.000000}%
\pgfsetstrokecolor{currentstroke}%
\pgfsetstrokeopacity{0.500000}%
\pgfsetdash{}{0pt}%
\pgfpathmoveto{\pgfqpoint{1.985802in}{2.328465in}}%
\pgfpathlineto{\pgfqpoint{1.997449in}{2.331829in}}%
\pgfpathlineto{\pgfqpoint{2.009090in}{2.335185in}}%
\pgfpathlineto{\pgfqpoint{2.020727in}{2.338533in}}%
\pgfpathlineto{\pgfqpoint{2.032357in}{2.341877in}}%
\pgfpathlineto{\pgfqpoint{2.043982in}{2.345219in}}%
\pgfpathlineto{\pgfqpoint{2.038082in}{2.353756in}}%
\pgfpathlineto{\pgfqpoint{2.032188in}{2.362273in}}%
\pgfpathlineto{\pgfqpoint{2.026297in}{2.370769in}}%
\pgfpathlineto{\pgfqpoint{2.020411in}{2.379244in}}%
\pgfpathlineto{\pgfqpoint{2.014529in}{2.387699in}}%
\pgfpathlineto{\pgfqpoint{2.002916in}{2.384385in}}%
\pgfpathlineto{\pgfqpoint{1.991297in}{2.381068in}}%
\pgfpathlineto{\pgfqpoint{1.979673in}{2.377747in}}%
\pgfpathlineto{\pgfqpoint{1.968043in}{2.374419in}}%
\pgfpathlineto{\pgfqpoint{1.956407in}{2.371084in}}%
\pgfpathlineto{\pgfqpoint{1.962278in}{2.362603in}}%
\pgfpathlineto{\pgfqpoint{1.968152in}{2.354101in}}%
\pgfpathlineto{\pgfqpoint{1.974031in}{2.345577in}}%
\pgfpathlineto{\pgfqpoint{1.979914in}{2.337032in}}%
\pgfpathclose%
\pgfusepath{stroke,fill}%
\end{pgfscope}%
\begin{pgfscope}%
\pgfpathrectangle{\pgfqpoint{0.887500in}{0.275000in}}{\pgfqpoint{4.225000in}{4.225000in}}%
\pgfusepath{clip}%
\pgfsetbuttcap%
\pgfsetroundjoin%
\definecolor{currentfill}{rgb}{0.154815,0.493313,0.557840}%
\pgfsetfillcolor{currentfill}%
\pgfsetfillopacity{0.700000}%
\pgfsetlinewidth{0.501875pt}%
\definecolor{currentstroke}{rgb}{1.000000,1.000000,1.000000}%
\pgfsetstrokecolor{currentstroke}%
\pgfsetstrokeopacity{0.500000}%
\pgfsetdash{}{0pt}%
\pgfpathmoveto{\pgfqpoint{2.307158in}{2.266730in}}%
\pgfpathlineto{\pgfqpoint{2.318725in}{2.270148in}}%
\pgfpathlineto{\pgfqpoint{2.330286in}{2.273549in}}%
\pgfpathlineto{\pgfqpoint{2.341842in}{2.276930in}}%
\pgfpathlineto{\pgfqpoint{2.353393in}{2.280291in}}%
\pgfpathlineto{\pgfqpoint{2.364938in}{2.283637in}}%
\pgfpathlineto{\pgfqpoint{2.358930in}{2.292353in}}%
\pgfpathlineto{\pgfqpoint{2.352926in}{2.301047in}}%
\pgfpathlineto{\pgfqpoint{2.346926in}{2.309720in}}%
\pgfpathlineto{\pgfqpoint{2.340931in}{2.318372in}}%
\pgfpathlineto{\pgfqpoint{2.334940in}{2.327003in}}%
\pgfpathlineto{\pgfqpoint{2.323406in}{2.323686in}}%
\pgfpathlineto{\pgfqpoint{2.311866in}{2.320355in}}%
\pgfpathlineto{\pgfqpoint{2.300322in}{2.317006in}}%
\pgfpathlineto{\pgfqpoint{2.288771in}{2.313639in}}%
\pgfpathlineto{\pgfqpoint{2.277216in}{2.310255in}}%
\pgfpathlineto{\pgfqpoint{2.283196in}{2.301595in}}%
\pgfpathlineto{\pgfqpoint{2.289180in}{2.292913in}}%
\pgfpathlineto{\pgfqpoint{2.295168in}{2.284209in}}%
\pgfpathlineto{\pgfqpoint{2.301161in}{2.275481in}}%
\pgfpathclose%
\pgfusepath{stroke,fill}%
\end{pgfscope}%
\begin{pgfscope}%
\pgfpathrectangle{\pgfqpoint{0.887500in}{0.275000in}}{\pgfqpoint{4.225000in}{4.225000in}}%
\pgfusepath{clip}%
\pgfsetbuttcap%
\pgfsetroundjoin%
\definecolor{currentfill}{rgb}{0.281477,0.755203,0.432552}%
\pgfsetfillcolor{currentfill}%
\pgfsetfillopacity{0.700000}%
\pgfsetlinewidth{0.501875pt}%
\definecolor{currentstroke}{rgb}{1.000000,1.000000,1.000000}%
\pgfsetstrokecolor{currentstroke}%
\pgfsetstrokeopacity{0.500000}%
\pgfsetdash{}{0pt}%
\pgfpathmoveto{\pgfqpoint{3.052674in}{2.766339in}}%
\pgfpathlineto{\pgfqpoint{3.064041in}{2.794825in}}%
\pgfpathlineto{\pgfqpoint{3.075415in}{2.822330in}}%
\pgfpathlineto{\pgfqpoint{3.086797in}{2.848233in}}%
\pgfpathlineto{\pgfqpoint{3.098184in}{2.871912in}}%
\pgfpathlineto{\pgfqpoint{3.109573in}{2.892743in}}%
\pgfpathlineto{\pgfqpoint{3.103324in}{2.908735in}}%
\pgfpathlineto{\pgfqpoint{3.097076in}{2.924113in}}%
\pgfpathlineto{\pgfqpoint{3.090830in}{2.938802in}}%
\pgfpathlineto{\pgfqpoint{3.084585in}{2.952724in}}%
\pgfpathlineto{\pgfqpoint{3.078343in}{2.965803in}}%
\pgfpathlineto{\pgfqpoint{3.066970in}{2.946189in}}%
\pgfpathlineto{\pgfqpoint{3.055601in}{2.923552in}}%
\pgfpathlineto{\pgfqpoint{3.044239in}{2.898494in}}%
\pgfpathlineto{\pgfqpoint{3.032884in}{2.871616in}}%
\pgfpathlineto{\pgfqpoint{3.021538in}{2.843518in}}%
\pgfpathlineto{\pgfqpoint{3.027761in}{2.828841in}}%
\pgfpathlineto{\pgfqpoint{3.033986in}{2.813529in}}%
\pgfpathlineto{\pgfqpoint{3.040214in}{2.797839in}}%
\pgfpathlineto{\pgfqpoint{3.046443in}{2.782024in}}%
\pgfpathclose%
\pgfusepath{stroke,fill}%
\end{pgfscope}%
\begin{pgfscope}%
\pgfpathrectangle{\pgfqpoint{0.887500in}{0.275000in}}{\pgfqpoint{4.225000in}{4.225000in}}%
\pgfusepath{clip}%
\pgfsetbuttcap%
\pgfsetroundjoin%
\definecolor{currentfill}{rgb}{0.165117,0.467423,0.558141}%
\pgfsetfillcolor{currentfill}%
\pgfsetfillopacity{0.700000}%
\pgfsetlinewidth{0.501875pt}%
\definecolor{currentstroke}{rgb}{1.000000,1.000000,1.000000}%
\pgfsetstrokecolor{currentstroke}%
\pgfsetstrokeopacity{0.500000}%
\pgfsetdash{}{0pt}%
\pgfpathmoveto{\pgfqpoint{3.006849in}{2.158026in}}%
\pgfpathlineto{\pgfqpoint{3.018212in}{2.177392in}}%
\pgfpathlineto{\pgfqpoint{3.029577in}{2.200181in}}%
\pgfpathlineto{\pgfqpoint{3.040946in}{2.225437in}}%
\pgfpathlineto{\pgfqpoint{3.052322in}{2.252201in}}%
\pgfpathlineto{\pgfqpoint{3.063706in}{2.279511in}}%
\pgfpathlineto{\pgfqpoint{3.057461in}{2.290801in}}%
\pgfpathlineto{\pgfqpoint{3.051219in}{2.302193in}}%
\pgfpathlineto{\pgfqpoint{3.044981in}{2.313615in}}%
\pgfpathlineto{\pgfqpoint{3.038746in}{2.324996in}}%
\pgfpathlineto{\pgfqpoint{3.032514in}{2.336263in}}%
\pgfpathlineto{\pgfqpoint{3.021158in}{2.306925in}}%
\pgfpathlineto{\pgfqpoint{3.009810in}{2.278056in}}%
\pgfpathlineto{\pgfqpoint{2.998470in}{2.250748in}}%
\pgfpathlineto{\pgfqpoint{2.987133in}{2.226085in}}%
\pgfpathlineto{\pgfqpoint{2.975795in}{2.205153in}}%
\pgfpathlineto{\pgfqpoint{2.981998in}{2.195753in}}%
\pgfpathlineto{\pgfqpoint{2.988205in}{2.186344in}}%
\pgfpathlineto{\pgfqpoint{2.994416in}{2.176921in}}%
\pgfpathlineto{\pgfqpoint{3.000630in}{2.167483in}}%
\pgfpathclose%
\pgfusepath{stroke,fill}%
\end{pgfscope}%
\begin{pgfscope}%
\pgfpathrectangle{\pgfqpoint{0.887500in}{0.275000in}}{\pgfqpoint{4.225000in}{4.225000in}}%
\pgfusepath{clip}%
\pgfsetbuttcap%
\pgfsetroundjoin%
\definecolor{currentfill}{rgb}{0.125394,0.574318,0.549086}%
\pgfsetfillcolor{currentfill}%
\pgfsetfillopacity{0.700000}%
\pgfsetlinewidth{0.501875pt}%
\definecolor{currentstroke}{rgb}{1.000000,1.000000,1.000000}%
\pgfsetstrokecolor{currentstroke}%
\pgfsetstrokeopacity{0.500000}%
\pgfsetdash{}{0pt}%
\pgfpathmoveto{\pgfqpoint{4.078883in}{2.427436in}}%
\pgfpathlineto{\pgfqpoint{4.090010in}{2.430838in}}%
\pgfpathlineto{\pgfqpoint{4.101131in}{2.434223in}}%
\pgfpathlineto{\pgfqpoint{4.112247in}{2.437597in}}%
\pgfpathlineto{\pgfqpoint{4.123356in}{2.440963in}}%
\pgfpathlineto{\pgfqpoint{4.134460in}{2.444324in}}%
\pgfpathlineto{\pgfqpoint{4.128026in}{2.458562in}}%
\pgfpathlineto{\pgfqpoint{4.121594in}{2.472770in}}%
\pgfpathlineto{\pgfqpoint{4.115164in}{2.486949in}}%
\pgfpathlineto{\pgfqpoint{4.108735in}{2.501101in}}%
\pgfpathlineto{\pgfqpoint{4.102308in}{2.515230in}}%
\pgfpathlineto{\pgfqpoint{4.091207in}{2.511900in}}%
\pgfpathlineto{\pgfqpoint{4.080100in}{2.508568in}}%
\pgfpathlineto{\pgfqpoint{4.068988in}{2.505231in}}%
\pgfpathlineto{\pgfqpoint{4.057870in}{2.501885in}}%
\pgfpathlineto{\pgfqpoint{4.046746in}{2.498529in}}%
\pgfpathlineto{\pgfqpoint{4.053170in}{2.484378in}}%
\pgfpathlineto{\pgfqpoint{4.059596in}{2.470195in}}%
\pgfpathlineto{\pgfqpoint{4.066023in}{2.455977in}}%
\pgfpathlineto{\pgfqpoint{4.072453in}{2.441723in}}%
\pgfpathclose%
\pgfusepath{stroke,fill}%
\end{pgfscope}%
\begin{pgfscope}%
\pgfpathrectangle{\pgfqpoint{0.887500in}{0.275000in}}{\pgfqpoint{4.225000in}{4.225000in}}%
\pgfusepath{clip}%
\pgfsetbuttcap%
\pgfsetroundjoin%
\definecolor{currentfill}{rgb}{0.185783,0.704891,0.485273}%
\pgfsetfillcolor{currentfill}%
\pgfsetfillopacity{0.700000}%
\pgfsetlinewidth{0.501875pt}%
\definecolor{currentstroke}{rgb}{1.000000,1.000000,1.000000}%
\pgfsetstrokecolor{currentstroke}%
\pgfsetstrokeopacity{0.500000}%
\pgfsetdash{}{0pt}%
\pgfpathmoveto{\pgfqpoint{3.695266in}{2.709620in}}%
\pgfpathlineto{\pgfqpoint{3.706491in}{2.713099in}}%
\pgfpathlineto{\pgfqpoint{3.717709in}{2.716528in}}%
\pgfpathlineto{\pgfqpoint{3.728922in}{2.719918in}}%
\pgfpathlineto{\pgfqpoint{3.740129in}{2.723281in}}%
\pgfpathlineto{\pgfqpoint{3.751330in}{2.726628in}}%
\pgfpathlineto{\pgfqpoint{3.744959in}{2.740358in}}%
\pgfpathlineto{\pgfqpoint{3.738589in}{2.754054in}}%
\pgfpathlineto{\pgfqpoint{3.732222in}{2.767714in}}%
\pgfpathlineto{\pgfqpoint{3.725856in}{2.781338in}}%
\pgfpathlineto{\pgfqpoint{3.719493in}{2.794923in}}%
\pgfpathlineto{\pgfqpoint{3.708295in}{2.791551in}}%
\pgfpathlineto{\pgfqpoint{3.697091in}{2.788164in}}%
\pgfpathlineto{\pgfqpoint{3.685881in}{2.784754in}}%
\pgfpathlineto{\pgfqpoint{3.674665in}{2.781314in}}%
\pgfpathlineto{\pgfqpoint{3.663443in}{2.777835in}}%
\pgfpathlineto{\pgfqpoint{3.669804in}{2.764275in}}%
\pgfpathlineto{\pgfqpoint{3.676166in}{2.750676in}}%
\pgfpathlineto{\pgfqpoint{3.682531in}{2.737036in}}%
\pgfpathlineto{\pgfqpoint{3.688897in}{2.723352in}}%
\pgfpathclose%
\pgfusepath{stroke,fill}%
\end{pgfscope}%
\begin{pgfscope}%
\pgfpathrectangle{\pgfqpoint{0.887500in}{0.275000in}}{\pgfqpoint{4.225000in}{4.225000in}}%
\pgfusepath{clip}%
\pgfsetbuttcap%
\pgfsetroundjoin%
\definecolor{currentfill}{rgb}{0.218130,0.347432,0.550038}%
\pgfsetfillcolor{currentfill}%
\pgfsetfillopacity{0.700000}%
\pgfsetlinewidth{0.501875pt}%
\definecolor{currentstroke}{rgb}{1.000000,1.000000,1.000000}%
\pgfsetstrokecolor{currentstroke}%
\pgfsetstrokeopacity{0.500000}%
\pgfsetdash{}{0pt}%
\pgfpathmoveto{\pgfqpoint{4.669976in}{1.955586in}}%
\pgfpathlineto{\pgfqpoint{4.680949in}{1.958917in}}%
\pgfpathlineto{\pgfqpoint{4.691917in}{1.962256in}}%
\pgfpathlineto{\pgfqpoint{4.702880in}{1.965604in}}%
\pgfpathlineto{\pgfqpoint{4.713838in}{1.968962in}}%
\pgfpathlineto{\pgfqpoint{4.707305in}{1.983534in}}%
\pgfpathlineto{\pgfqpoint{4.700774in}{1.998089in}}%
\pgfpathlineto{\pgfqpoint{4.694246in}{2.012632in}}%
\pgfpathlineto{\pgfqpoint{4.687719in}{2.027164in}}%
\pgfpathlineto{\pgfqpoint{4.681194in}{2.041691in}}%
\pgfpathlineto{\pgfqpoint{4.670238in}{2.038343in}}%
\pgfpathlineto{\pgfqpoint{4.659276in}{2.035005in}}%
\pgfpathlineto{\pgfqpoint{4.648309in}{2.031676in}}%
\pgfpathlineto{\pgfqpoint{4.637337in}{2.028357in}}%
\pgfpathlineto{\pgfqpoint{4.643860in}{2.013824in}}%
\pgfpathlineto{\pgfqpoint{4.650386in}{1.999283in}}%
\pgfpathlineto{\pgfqpoint{4.656914in}{1.984732in}}%
\pgfpathlineto{\pgfqpoint{4.663444in}{1.970167in}}%
\pgfpathclose%
\pgfusepath{stroke,fill}%
\end{pgfscope}%
\begin{pgfscope}%
\pgfpathrectangle{\pgfqpoint{0.887500in}{0.275000in}}{\pgfqpoint{4.225000in}{4.225000in}}%
\pgfusepath{clip}%
\pgfsetbuttcap%
\pgfsetroundjoin%
\definecolor{currentfill}{rgb}{0.171176,0.452530,0.557965}%
\pgfsetfillcolor{currentfill}%
\pgfsetfillopacity{0.700000}%
\pgfsetlinewidth{0.501875pt}%
\definecolor{currentstroke}{rgb}{1.000000,1.000000,1.000000}%
\pgfsetstrokecolor{currentstroke}%
\pgfsetstrokeopacity{0.500000}%
\pgfsetdash{}{0pt}%
\pgfpathmoveto{\pgfqpoint{2.716561in}{2.173355in}}%
\pgfpathlineto{\pgfqpoint{2.728033in}{2.176368in}}%
\pgfpathlineto{\pgfqpoint{2.739499in}{2.179434in}}%
\pgfpathlineto{\pgfqpoint{2.750958in}{2.182644in}}%
\pgfpathlineto{\pgfqpoint{2.762410in}{2.186088in}}%
\pgfpathlineto{\pgfqpoint{2.773853in}{2.189855in}}%
\pgfpathlineto{\pgfqpoint{2.767707in}{2.198979in}}%
\pgfpathlineto{\pgfqpoint{2.761565in}{2.208075in}}%
\pgfpathlineto{\pgfqpoint{2.755427in}{2.217143in}}%
\pgfpathlineto{\pgfqpoint{2.749293in}{2.226184in}}%
\pgfpathlineto{\pgfqpoint{2.743163in}{2.235199in}}%
\pgfpathlineto{\pgfqpoint{2.731731in}{2.231455in}}%
\pgfpathlineto{\pgfqpoint{2.720290in}{2.228033in}}%
\pgfpathlineto{\pgfqpoint{2.708841in}{2.224843in}}%
\pgfpathlineto{\pgfqpoint{2.697386in}{2.221795in}}%
\pgfpathlineto{\pgfqpoint{2.685924in}{2.218800in}}%
\pgfpathlineto{\pgfqpoint{2.692043in}{2.209769in}}%
\pgfpathlineto{\pgfqpoint{2.698166in}{2.200709in}}%
\pgfpathlineto{\pgfqpoint{2.704294in}{2.191620in}}%
\pgfpathlineto{\pgfqpoint{2.710425in}{2.182503in}}%
\pgfpathclose%
\pgfusepath{stroke,fill}%
\end{pgfscope}%
\begin{pgfscope}%
\pgfpathrectangle{\pgfqpoint{0.887500in}{0.275000in}}{\pgfqpoint{4.225000in}{4.225000in}}%
\pgfusepath{clip}%
\pgfsetbuttcap%
\pgfsetroundjoin%
\definecolor{currentfill}{rgb}{0.156270,0.489624,0.557936}%
\pgfsetfillcolor{currentfill}%
\pgfsetfillopacity{0.700000}%
\pgfsetlinewidth{0.501875pt}%
\definecolor{currentstroke}{rgb}{1.000000,1.000000,1.000000}%
\pgfsetstrokecolor{currentstroke}%
\pgfsetstrokeopacity{0.500000}%
\pgfsetdash{}{0pt}%
\pgfpathmoveto{\pgfqpoint{4.286549in}{2.241364in}}%
\pgfpathlineto{\pgfqpoint{4.297611in}{2.244335in}}%
\pgfpathlineto{\pgfqpoint{4.308673in}{2.247446in}}%
\pgfpathlineto{\pgfqpoint{4.319735in}{2.250745in}}%
\pgfpathlineto{\pgfqpoint{4.330799in}{2.254280in}}%
\pgfpathlineto{\pgfqpoint{4.341865in}{2.258044in}}%
\pgfpathlineto{\pgfqpoint{4.335409in}{2.272859in}}%
\pgfpathlineto{\pgfqpoint{4.328958in}{2.287759in}}%
\pgfpathlineto{\pgfqpoint{4.322511in}{2.302711in}}%
\pgfpathlineto{\pgfqpoint{4.316067in}{2.317682in}}%
\pgfpathlineto{\pgfqpoint{4.309625in}{2.332637in}}%
\pgfpathlineto{\pgfqpoint{4.298569in}{2.329178in}}%
\pgfpathlineto{\pgfqpoint{4.287511in}{2.325787in}}%
\pgfpathlineto{\pgfqpoint{4.276448in}{2.322466in}}%
\pgfpathlineto{\pgfqpoint{4.265383in}{2.319201in}}%
\pgfpathlineto{\pgfqpoint{4.254313in}{2.315978in}}%
\pgfpathlineto{\pgfqpoint{4.260754in}{2.301029in}}%
\pgfpathlineto{\pgfqpoint{4.267198in}{2.286061in}}%
\pgfpathlineto{\pgfqpoint{4.273644in}{2.271106in}}%
\pgfpathlineto{\pgfqpoint{4.280094in}{2.256197in}}%
\pgfpathclose%
\pgfusepath{stroke,fill}%
\end{pgfscope}%
\begin{pgfscope}%
\pgfpathrectangle{\pgfqpoint{0.887500in}{0.275000in}}{\pgfqpoint{4.225000in}{4.225000in}}%
\pgfusepath{clip}%
\pgfsetbuttcap%
\pgfsetroundjoin%
\definecolor{currentfill}{rgb}{0.430983,0.808473,0.346476}%
\pgfsetfillcolor{currentfill}%
\pgfsetfillopacity{0.700000}%
\pgfsetlinewidth{0.501875pt}%
\definecolor{currentstroke}{rgb}{1.000000,1.000000,1.000000}%
\pgfsetstrokecolor{currentstroke}%
\pgfsetstrokeopacity{0.500000}%
\pgfsetdash{}{0pt}%
\pgfpathmoveto{\pgfqpoint{3.166478in}{2.958462in}}%
\pgfpathlineto{\pgfqpoint{3.177845in}{2.967253in}}%
\pgfpathlineto{\pgfqpoint{3.189210in}{2.975735in}}%
\pgfpathlineto{\pgfqpoint{3.200571in}{2.984169in}}%
\pgfpathlineto{\pgfqpoint{3.211930in}{2.992471in}}%
\pgfpathlineto{\pgfqpoint{3.223285in}{3.000530in}}%
\pgfpathlineto{\pgfqpoint{3.217011in}{3.013428in}}%
\pgfpathlineto{\pgfqpoint{3.210740in}{3.026016in}}%
\pgfpathlineto{\pgfqpoint{3.204470in}{3.038333in}}%
\pgfpathlineto{\pgfqpoint{3.198202in}{3.050415in}}%
\pgfpathlineto{\pgfqpoint{3.191936in}{3.062301in}}%
\pgfpathlineto{\pgfqpoint{3.180592in}{3.054672in}}%
\pgfpathlineto{\pgfqpoint{3.169244in}{3.046986in}}%
\pgfpathlineto{\pgfqpoint{3.157894in}{3.039373in}}%
\pgfpathlineto{\pgfqpoint{3.146541in}{3.031934in}}%
\pgfpathlineto{\pgfqpoint{3.135184in}{3.024383in}}%
\pgfpathlineto{\pgfqpoint{3.141440in}{3.012205in}}%
\pgfpathlineto{\pgfqpoint{3.147697in}{2.999535in}}%
\pgfpathlineto{\pgfqpoint{3.153956in}{2.986362in}}%
\pgfpathlineto{\pgfqpoint{3.160216in}{2.972675in}}%
\pgfpathclose%
\pgfusepath{stroke,fill}%
\end{pgfscope}%
\begin{pgfscope}%
\pgfpathrectangle{\pgfqpoint{0.887500in}{0.275000in}}{\pgfqpoint{4.225000in}{4.225000in}}%
\pgfusepath{clip}%
\pgfsetbuttcap%
\pgfsetroundjoin%
\definecolor{currentfill}{rgb}{0.226397,0.728888,0.462789}%
\pgfsetfillcolor{currentfill}%
\pgfsetfillopacity{0.700000}%
\pgfsetlinewidth{0.501875pt}%
\definecolor{currentstroke}{rgb}{1.000000,1.000000,1.000000}%
\pgfsetstrokecolor{currentstroke}%
\pgfsetstrokeopacity{0.500000}%
\pgfsetdash{}{0pt}%
\pgfpathmoveto{\pgfqpoint{3.607244in}{2.759788in}}%
\pgfpathlineto{\pgfqpoint{3.618495in}{2.763465in}}%
\pgfpathlineto{\pgfqpoint{3.629741in}{2.767117in}}%
\pgfpathlineto{\pgfqpoint{3.640981in}{2.770735in}}%
\pgfpathlineto{\pgfqpoint{3.652215in}{2.774310in}}%
\pgfpathlineto{\pgfqpoint{3.663443in}{2.777835in}}%
\pgfpathlineto{\pgfqpoint{3.657085in}{2.791359in}}%
\pgfpathlineto{\pgfqpoint{3.650728in}{2.804852in}}%
\pgfpathlineto{\pgfqpoint{3.644374in}{2.818315in}}%
\pgfpathlineto{\pgfqpoint{3.638023in}{2.831752in}}%
\pgfpathlineto{\pgfqpoint{3.631673in}{2.845164in}}%
\pgfpathlineto{\pgfqpoint{3.620450in}{2.841686in}}%
\pgfpathlineto{\pgfqpoint{3.609220in}{2.838169in}}%
\pgfpathlineto{\pgfqpoint{3.597984in}{2.834616in}}%
\pgfpathlineto{\pgfqpoint{3.586743in}{2.831031in}}%
\pgfpathlineto{\pgfqpoint{3.575496in}{2.827416in}}%
\pgfpathlineto{\pgfqpoint{3.581841in}{2.813924in}}%
\pgfpathlineto{\pgfqpoint{3.588188in}{2.800419in}}%
\pgfpathlineto{\pgfqpoint{3.594538in}{2.786898in}}%
\pgfpathlineto{\pgfqpoint{3.600890in}{2.773356in}}%
\pgfpathclose%
\pgfusepath{stroke,fill}%
\end{pgfscope}%
\begin{pgfscope}%
\pgfpathrectangle{\pgfqpoint{0.887500in}{0.275000in}}{\pgfqpoint{4.225000in}{4.225000in}}%
\pgfusepath{clip}%
\pgfsetbuttcap%
\pgfsetroundjoin%
\definecolor{currentfill}{rgb}{0.119738,0.603785,0.541400}%
\pgfsetfillcolor{currentfill}%
\pgfsetfillopacity{0.700000}%
\pgfsetlinewidth{0.501875pt}%
\definecolor{currentstroke}{rgb}{1.000000,1.000000,1.000000}%
\pgfsetstrokecolor{currentstroke}%
\pgfsetstrokeopacity{0.500000}%
\pgfsetdash{}{0pt}%
\pgfpathmoveto{\pgfqpoint{3.991041in}{2.481581in}}%
\pgfpathlineto{\pgfqpoint{4.002193in}{2.484984in}}%
\pgfpathlineto{\pgfqpoint{4.013340in}{2.488383in}}%
\pgfpathlineto{\pgfqpoint{4.024481in}{2.491776in}}%
\pgfpathlineto{\pgfqpoint{4.035616in}{2.495159in}}%
\pgfpathlineto{\pgfqpoint{4.046746in}{2.498529in}}%
\pgfpathlineto{\pgfqpoint{4.040324in}{2.512652in}}%
\pgfpathlineto{\pgfqpoint{4.033903in}{2.526750in}}%
\pgfpathlineto{\pgfqpoint{4.027485in}{2.540828in}}%
\pgfpathlineto{\pgfqpoint{4.021069in}{2.554889in}}%
\pgfpathlineto{\pgfqpoint{4.014655in}{2.568937in}}%
\pgfpathlineto{\pgfqpoint{4.003528in}{2.565576in}}%
\pgfpathlineto{\pgfqpoint{3.992395in}{2.562204in}}%
\pgfpathlineto{\pgfqpoint{3.981256in}{2.558823in}}%
\pgfpathlineto{\pgfqpoint{3.970112in}{2.555436in}}%
\pgfpathlineto{\pgfqpoint{3.958963in}{2.552045in}}%
\pgfpathlineto{\pgfqpoint{3.965374in}{2.538004in}}%
\pgfpathlineto{\pgfqpoint{3.971788in}{2.523941in}}%
\pgfpathlineto{\pgfqpoint{3.978204in}{2.509853in}}%
\pgfpathlineto{\pgfqpoint{3.984622in}{2.495734in}}%
\pgfpathclose%
\pgfusepath{stroke,fill}%
\end{pgfscope}%
\begin{pgfscope}%
\pgfpathrectangle{\pgfqpoint{0.887500in}{0.275000in}}{\pgfqpoint{4.225000in}{4.225000in}}%
\pgfusepath{clip}%
\pgfsetbuttcap%
\pgfsetroundjoin%
\definecolor{currentfill}{rgb}{0.136408,0.541173,0.554483}%
\pgfsetfillcolor{currentfill}%
\pgfsetfillopacity{0.700000}%
\pgfsetlinewidth{0.501875pt}%
\definecolor{currentstroke}{rgb}{1.000000,1.000000,1.000000}%
\pgfsetstrokecolor{currentstroke}%
\pgfsetstrokeopacity{0.500000}%
\pgfsetdash{}{0pt}%
\pgfpathmoveto{\pgfqpoint{1.752091in}{2.362741in}}%
\pgfpathlineto{\pgfqpoint{1.763800in}{2.366123in}}%
\pgfpathlineto{\pgfqpoint{1.775503in}{2.369503in}}%
\pgfpathlineto{\pgfqpoint{1.787200in}{2.372883in}}%
\pgfpathlineto{\pgfqpoint{1.798892in}{2.376265in}}%
\pgfpathlineto{\pgfqpoint{1.810577in}{2.379648in}}%
\pgfpathlineto{\pgfqpoint{1.804757in}{2.388066in}}%
\pgfpathlineto{\pgfqpoint{1.798940in}{2.396463in}}%
\pgfpathlineto{\pgfqpoint{1.793129in}{2.404837in}}%
\pgfpathlineto{\pgfqpoint{1.787321in}{2.413188in}}%
\pgfpathlineto{\pgfqpoint{1.781518in}{2.421518in}}%
\pgfpathlineto{\pgfqpoint{1.769845in}{2.418161in}}%
\pgfpathlineto{\pgfqpoint{1.758166in}{2.414805in}}%
\pgfpathlineto{\pgfqpoint{1.746480in}{2.411451in}}%
\pgfpathlineto{\pgfqpoint{1.734789in}{2.408098in}}%
\pgfpathlineto{\pgfqpoint{1.723093in}{2.404743in}}%
\pgfpathlineto{\pgfqpoint{1.728883in}{2.396390in}}%
\pgfpathlineto{\pgfqpoint{1.734678in}{2.388013in}}%
\pgfpathlineto{\pgfqpoint{1.740478in}{2.379613in}}%
\pgfpathlineto{\pgfqpoint{1.746282in}{2.371189in}}%
\pgfpathclose%
\pgfusepath{stroke,fill}%
\end{pgfscope}%
\begin{pgfscope}%
\pgfpathrectangle{\pgfqpoint{0.887500in}{0.275000in}}{\pgfqpoint{4.225000in}{4.225000in}}%
\pgfusepath{clip}%
\pgfsetbuttcap%
\pgfsetroundjoin%
\definecolor{currentfill}{rgb}{0.159194,0.482237,0.558073}%
\pgfsetfillcolor{currentfill}%
\pgfsetfillopacity{0.700000}%
\pgfsetlinewidth{0.501875pt}%
\definecolor{currentstroke}{rgb}{1.000000,1.000000,1.000000}%
\pgfsetstrokecolor{currentstroke}%
\pgfsetstrokeopacity{0.500000}%
\pgfsetdash{}{0pt}%
\pgfpathmoveto{\pgfqpoint{2.395042in}{2.239709in}}%
\pgfpathlineto{\pgfqpoint{2.406593in}{2.243080in}}%
\pgfpathlineto{\pgfqpoint{2.418138in}{2.246443in}}%
\pgfpathlineto{\pgfqpoint{2.429677in}{2.249803in}}%
\pgfpathlineto{\pgfqpoint{2.441210in}{2.253165in}}%
\pgfpathlineto{\pgfqpoint{2.452738in}{2.256533in}}%
\pgfpathlineto{\pgfqpoint{2.446697in}{2.265330in}}%
\pgfpathlineto{\pgfqpoint{2.440661in}{2.274104in}}%
\pgfpathlineto{\pgfqpoint{2.434630in}{2.282855in}}%
\pgfpathlineto{\pgfqpoint{2.428602in}{2.291584in}}%
\pgfpathlineto{\pgfqpoint{2.422579in}{2.300291in}}%
\pgfpathlineto{\pgfqpoint{2.411062in}{2.296955in}}%
\pgfpathlineto{\pgfqpoint{2.399540in}{2.293626in}}%
\pgfpathlineto{\pgfqpoint{2.388012in}{2.290300in}}%
\pgfpathlineto{\pgfqpoint{2.376478in}{2.286972in}}%
\pgfpathlineto{\pgfqpoint{2.364938in}{2.283637in}}%
\pgfpathlineto{\pgfqpoint{2.370950in}{2.274899in}}%
\pgfpathlineto{\pgfqpoint{2.376967in}{2.266137in}}%
\pgfpathlineto{\pgfqpoint{2.382988in}{2.257353in}}%
\pgfpathlineto{\pgfqpoint{2.389013in}{2.248543in}}%
\pgfpathclose%
\pgfusepath{stroke,fill}%
\end{pgfscope}%
\begin{pgfscope}%
\pgfpathrectangle{\pgfqpoint{0.887500in}{0.275000in}}{\pgfqpoint{4.225000in}{4.225000in}}%
\pgfusepath{clip}%
\pgfsetbuttcap%
\pgfsetroundjoin%
\definecolor{currentfill}{rgb}{0.203063,0.379716,0.553925}%
\pgfsetfillcolor{currentfill}%
\pgfsetfillopacity{0.700000}%
\pgfsetlinewidth{0.501875pt}%
\definecolor{currentstroke}{rgb}{1.000000,1.000000,1.000000}%
\pgfsetstrokecolor{currentstroke}%
\pgfsetstrokeopacity{0.500000}%
\pgfsetdash{}{0pt}%
\pgfpathmoveto{\pgfqpoint{4.582401in}{2.011890in}}%
\pgfpathlineto{\pgfqpoint{4.593398in}{2.015167in}}%
\pgfpathlineto{\pgfqpoint{4.604391in}{2.018452in}}%
\pgfpathlineto{\pgfqpoint{4.615378in}{2.021745in}}%
\pgfpathlineto{\pgfqpoint{4.626360in}{2.025047in}}%
\pgfpathlineto{\pgfqpoint{4.637337in}{2.028357in}}%
\pgfpathlineto{\pgfqpoint{4.630816in}{2.042885in}}%
\pgfpathlineto{\pgfqpoint{4.624297in}{2.057411in}}%
\pgfpathlineto{\pgfqpoint{4.617781in}{2.071937in}}%
\pgfpathlineto{\pgfqpoint{4.611268in}{2.086467in}}%
\pgfpathlineto{\pgfqpoint{4.604757in}{2.101003in}}%
\pgfpathlineto{\pgfqpoint{4.593781in}{2.097673in}}%
\pgfpathlineto{\pgfqpoint{4.582798in}{2.094345in}}%
\pgfpathlineto{\pgfqpoint{4.571811in}{2.091016in}}%
\pgfpathlineto{\pgfqpoint{4.560818in}{2.087688in}}%
\pgfpathlineto{\pgfqpoint{4.549819in}{2.084357in}}%
\pgfpathlineto{\pgfqpoint{4.556331in}{2.069897in}}%
\pgfpathlineto{\pgfqpoint{4.562846in}{2.055420in}}%
\pgfpathlineto{\pgfqpoint{4.569362in}{2.040927in}}%
\pgfpathlineto{\pgfqpoint{4.575881in}{2.026417in}}%
\pgfpathclose%
\pgfusepath{stroke,fill}%
\end{pgfscope}%
\begin{pgfscope}%
\pgfpathrectangle{\pgfqpoint{0.887500in}{0.275000in}}{\pgfqpoint{4.225000in}{4.225000in}}%
\pgfusepath{clip}%
\pgfsetbuttcap%
\pgfsetroundjoin%
\definecolor{currentfill}{rgb}{0.147607,0.511733,0.557049}%
\pgfsetfillcolor{currentfill}%
\pgfsetfillopacity{0.700000}%
\pgfsetlinewidth{0.501875pt}%
\definecolor{currentstroke}{rgb}{1.000000,1.000000,1.000000}%
\pgfsetstrokecolor{currentstroke}%
\pgfsetstrokeopacity{0.500000}%
\pgfsetdash{}{0pt}%
\pgfpathmoveto{\pgfqpoint{2.073541in}{2.302218in}}%
\pgfpathlineto{\pgfqpoint{2.085171in}{2.305595in}}%
\pgfpathlineto{\pgfqpoint{2.096796in}{2.308976in}}%
\pgfpathlineto{\pgfqpoint{2.108414in}{2.312362in}}%
\pgfpathlineto{\pgfqpoint{2.120026in}{2.315757in}}%
\pgfpathlineto{\pgfqpoint{2.131633in}{2.319164in}}%
\pgfpathlineto{\pgfqpoint{2.125701in}{2.327770in}}%
\pgfpathlineto{\pgfqpoint{2.119773in}{2.336356in}}%
\pgfpathlineto{\pgfqpoint{2.113850in}{2.344922in}}%
\pgfpathlineto{\pgfqpoint{2.107931in}{2.353469in}}%
\pgfpathlineto{\pgfqpoint{2.102016in}{2.361996in}}%
\pgfpathlineto{\pgfqpoint{2.090421in}{2.358622in}}%
\pgfpathlineto{\pgfqpoint{2.078820in}{2.355260in}}%
\pgfpathlineto{\pgfqpoint{2.067213in}{2.351908in}}%
\pgfpathlineto{\pgfqpoint{2.055600in}{2.348562in}}%
\pgfpathlineto{\pgfqpoint{2.043982in}{2.345219in}}%
\pgfpathlineto{\pgfqpoint{2.049885in}{2.336661in}}%
\pgfpathlineto{\pgfqpoint{2.055793in}{2.328082in}}%
\pgfpathlineto{\pgfqpoint{2.061705in}{2.319482in}}%
\pgfpathlineto{\pgfqpoint{2.067621in}{2.310861in}}%
\pgfpathclose%
\pgfusepath{stroke,fill}%
\end{pgfscope}%
\begin{pgfscope}%
\pgfpathrectangle{\pgfqpoint{0.887500in}{0.275000in}}{\pgfqpoint{4.225000in}{4.225000in}}%
\pgfusepath{clip}%
\pgfsetbuttcap%
\pgfsetroundjoin%
\definecolor{currentfill}{rgb}{0.129933,0.559582,0.551864}%
\pgfsetfillcolor{currentfill}%
\pgfsetfillopacity{0.700000}%
\pgfsetlinewidth{0.501875pt}%
\definecolor{currentstroke}{rgb}{1.000000,1.000000,1.000000}%
\pgfsetstrokecolor{currentstroke}%
\pgfsetstrokeopacity{0.500000}%
\pgfsetdash{}{0pt}%
\pgfpathmoveto{\pgfqpoint{3.032514in}{2.336263in}}%
\pgfpathlineto{\pgfqpoint{3.043879in}{2.364974in}}%
\pgfpathlineto{\pgfqpoint{3.055254in}{2.392142in}}%
\pgfpathlineto{\pgfqpoint{3.066637in}{2.417753in}}%
\pgfpathlineto{\pgfqpoint{3.078026in}{2.442072in}}%
\pgfpathlineto{\pgfqpoint{3.089420in}{2.465366in}}%
\pgfpathlineto{\pgfqpoint{3.083167in}{2.477349in}}%
\pgfpathlineto{\pgfqpoint{3.076917in}{2.489001in}}%
\pgfpathlineto{\pgfqpoint{3.070670in}{2.500254in}}%
\pgfpathlineto{\pgfqpoint{3.064425in}{2.511038in}}%
\pgfpathlineto{\pgfqpoint{3.058185in}{2.521301in}}%
\pgfpathlineto{\pgfqpoint{3.046813in}{2.497533in}}%
\pgfpathlineto{\pgfqpoint{3.035449in}{2.472591in}}%
\pgfpathlineto{\pgfqpoint{3.024092in}{2.446211in}}%
\pgfpathlineto{\pgfqpoint{3.012745in}{2.418131in}}%
\pgfpathlineto{\pgfqpoint{3.001409in}{2.388379in}}%
\pgfpathlineto{\pgfqpoint{3.007622in}{2.378750in}}%
\pgfpathlineto{\pgfqpoint{3.013839in}{2.368659in}}%
\pgfpathlineto{\pgfqpoint{3.020060in}{2.358166in}}%
\pgfpathlineto{\pgfqpoint{3.026285in}{2.347343in}}%
\pgfpathclose%
\pgfusepath{stroke,fill}%
\end{pgfscope}%
\begin{pgfscope}%
\pgfpathrectangle{\pgfqpoint{0.887500in}{0.275000in}}{\pgfqpoint{4.225000in}{4.225000in}}%
\pgfusepath{clip}%
\pgfsetbuttcap%
\pgfsetroundjoin%
\definecolor{currentfill}{rgb}{0.124780,0.640461,0.527068}%
\pgfsetfillcolor{currentfill}%
\pgfsetfillopacity{0.700000}%
\pgfsetlinewidth{0.501875pt}%
\definecolor{currentstroke}{rgb}{1.000000,1.000000,1.000000}%
\pgfsetstrokecolor{currentstroke}%
\pgfsetstrokeopacity{0.500000}%
\pgfsetdash{}{0pt}%
\pgfpathmoveto{\pgfqpoint{3.058185in}{2.521301in}}%
\pgfpathlineto{\pgfqpoint{3.069562in}{2.544159in}}%
\pgfpathlineto{\pgfqpoint{3.080944in}{2.566373in}}%
\pgfpathlineto{\pgfqpoint{3.092332in}{2.588209in}}%
\pgfpathlineto{\pgfqpoint{3.103724in}{2.609855in}}%
\pgfpathlineto{\pgfqpoint{3.115120in}{2.631240in}}%
\pgfpathlineto{\pgfqpoint{3.108862in}{2.643112in}}%
\pgfpathlineto{\pgfqpoint{3.102607in}{2.655310in}}%
\pgfpathlineto{\pgfqpoint{3.096356in}{2.667848in}}%
\pgfpathlineto{\pgfqpoint{3.090108in}{2.680741in}}%
\pgfpathlineto{\pgfqpoint{3.083862in}{2.694002in}}%
\pgfpathlineto{\pgfqpoint{3.072484in}{2.668792in}}%
\pgfpathlineto{\pgfqpoint{3.061112in}{2.643632in}}%
\pgfpathlineto{\pgfqpoint{3.049747in}{2.618800in}}%
\pgfpathlineto{\pgfqpoint{3.038389in}{2.594127in}}%
\pgfpathlineto{\pgfqpoint{3.027038in}{2.569310in}}%
\pgfpathlineto{\pgfqpoint{3.033260in}{2.559610in}}%
\pgfpathlineto{\pgfqpoint{3.039485in}{2.550121in}}%
\pgfpathlineto{\pgfqpoint{3.045715in}{2.540681in}}%
\pgfpathlineto{\pgfqpoint{3.051948in}{2.531128in}}%
\pgfpathclose%
\pgfusepath{stroke,fill}%
\end{pgfscope}%
\begin{pgfscope}%
\pgfpathrectangle{\pgfqpoint{0.887500in}{0.275000in}}{\pgfqpoint{4.225000in}{4.225000in}}%
\pgfusepath{clip}%
\pgfsetbuttcap%
\pgfsetroundjoin%
\definecolor{currentfill}{rgb}{0.274149,0.751988,0.436601}%
\pgfsetfillcolor{currentfill}%
\pgfsetfillopacity{0.700000}%
\pgfsetlinewidth{0.501875pt}%
\definecolor{currentstroke}{rgb}{1.000000,1.000000,1.000000}%
\pgfsetstrokecolor{currentstroke}%
\pgfsetstrokeopacity{0.500000}%
\pgfsetdash{}{0pt}%
\pgfpathmoveto{\pgfqpoint{3.519176in}{2.808991in}}%
\pgfpathlineto{\pgfqpoint{3.530452in}{2.812722in}}%
\pgfpathlineto{\pgfqpoint{3.541721in}{2.816426in}}%
\pgfpathlineto{\pgfqpoint{3.552985in}{2.820110in}}%
\pgfpathlineto{\pgfqpoint{3.564243in}{2.823774in}}%
\pgfpathlineto{\pgfqpoint{3.575496in}{2.827416in}}%
\pgfpathlineto{\pgfqpoint{3.569154in}{2.840890in}}%
\pgfpathlineto{\pgfqpoint{3.562814in}{2.854344in}}%
\pgfpathlineto{\pgfqpoint{3.556476in}{2.867774in}}%
\pgfpathlineto{\pgfqpoint{3.550141in}{2.881175in}}%
\pgfpathlineto{\pgfqpoint{3.543808in}{2.894545in}}%
\pgfpathlineto{\pgfqpoint{3.532559in}{2.890946in}}%
\pgfpathlineto{\pgfqpoint{3.521305in}{2.887306in}}%
\pgfpathlineto{\pgfqpoint{3.510045in}{2.883624in}}%
\pgfpathlineto{\pgfqpoint{3.498779in}{2.879899in}}%
\pgfpathlineto{\pgfqpoint{3.487508in}{2.876130in}}%
\pgfpathlineto{\pgfqpoint{3.493837in}{2.862765in}}%
\pgfpathlineto{\pgfqpoint{3.500168in}{2.849365in}}%
\pgfpathlineto{\pgfqpoint{3.506502in}{2.835934in}}%
\pgfpathlineto{\pgfqpoint{3.512838in}{2.822474in}}%
\pgfpathclose%
\pgfusepath{stroke,fill}%
\end{pgfscope}%
\begin{pgfscope}%
\pgfpathrectangle{\pgfqpoint{0.887500in}{0.275000in}}{\pgfqpoint{4.225000in}{4.225000in}}%
\pgfusepath{clip}%
\pgfsetbuttcap%
\pgfsetroundjoin%
\definecolor{currentfill}{rgb}{0.121380,0.629492,0.531973}%
\pgfsetfillcolor{currentfill}%
\pgfsetfillopacity{0.700000}%
\pgfsetlinewidth{0.501875pt}%
\definecolor{currentstroke}{rgb}{1.000000,1.000000,1.000000}%
\pgfsetstrokecolor{currentstroke}%
\pgfsetstrokeopacity{0.500000}%
\pgfsetdash{}{0pt}%
\pgfpathmoveto{\pgfqpoint{3.903133in}{2.535096in}}%
\pgfpathlineto{\pgfqpoint{3.914310in}{2.538484in}}%
\pgfpathlineto{\pgfqpoint{3.925481in}{2.541871in}}%
\pgfpathlineto{\pgfqpoint{3.936647in}{2.545260in}}%
\pgfpathlineto{\pgfqpoint{3.947807in}{2.548652in}}%
\pgfpathlineto{\pgfqpoint{3.958963in}{2.552045in}}%
\pgfpathlineto{\pgfqpoint{3.952553in}{2.566067in}}%
\pgfpathlineto{\pgfqpoint{3.946145in}{2.580074in}}%
\pgfpathlineto{\pgfqpoint{3.939740in}{2.594070in}}%
\pgfpathlineto{\pgfqpoint{3.933338in}{2.608057in}}%
\pgfpathlineto{\pgfqpoint{3.926937in}{2.622033in}}%
\pgfpathlineto{\pgfqpoint{3.915784in}{2.618610in}}%
\pgfpathlineto{\pgfqpoint{3.904625in}{2.615191in}}%
\pgfpathlineto{\pgfqpoint{3.893461in}{2.611778in}}%
\pgfpathlineto{\pgfqpoint{3.882292in}{2.608371in}}%
\pgfpathlineto{\pgfqpoint{3.871117in}{2.604966in}}%
\pgfpathlineto{\pgfqpoint{3.877516in}{2.591020in}}%
\pgfpathlineto{\pgfqpoint{3.883916in}{2.577061in}}%
\pgfpathlineto{\pgfqpoint{3.890320in}{2.563089in}}%
\pgfpathlineto{\pgfqpoint{3.896725in}{2.549103in}}%
\pgfpathclose%
\pgfusepath{stroke,fill}%
\end{pgfscope}%
\begin{pgfscope}%
\pgfpathrectangle{\pgfqpoint{0.887500in}{0.275000in}}{\pgfqpoint{4.225000in}{4.225000in}}%
\pgfusepath{clip}%
\pgfsetbuttcap%
\pgfsetroundjoin%
\definecolor{currentfill}{rgb}{0.412913,0.803041,0.357269}%
\pgfsetfillcolor{currentfill}%
\pgfsetfillopacity{0.700000}%
\pgfsetlinewidth{0.501875pt}%
\definecolor{currentstroke}{rgb}{1.000000,1.000000,1.000000}%
\pgfsetstrokecolor{currentstroke}%
\pgfsetstrokeopacity{0.500000}%
\pgfsetdash{}{0pt}%
\pgfpathmoveto{\pgfqpoint{3.254672in}{2.931925in}}%
\pgfpathlineto{\pgfqpoint{3.266033in}{2.940275in}}%
\pgfpathlineto{\pgfqpoint{3.277387in}{2.947982in}}%
\pgfpathlineto{\pgfqpoint{3.288734in}{2.955007in}}%
\pgfpathlineto{\pgfqpoint{3.300074in}{2.961317in}}%
\pgfpathlineto{\pgfqpoint{3.311405in}{2.966944in}}%
\pgfpathlineto{\pgfqpoint{3.305115in}{2.980519in}}%
\pgfpathlineto{\pgfqpoint{3.298827in}{2.993993in}}%
\pgfpathlineto{\pgfqpoint{3.292541in}{3.007328in}}%
\pgfpathlineto{\pgfqpoint{3.286257in}{3.020485in}}%
\pgfpathlineto{\pgfqpoint{3.279974in}{3.033423in}}%
\pgfpathlineto{\pgfqpoint{3.268650in}{3.028125in}}%
\pgfpathlineto{\pgfqpoint{3.257319in}{3.022143in}}%
\pgfpathlineto{\pgfqpoint{3.245980in}{3.015477in}}%
\pgfpathlineto{\pgfqpoint{3.234635in}{3.008236in}}%
\pgfpathlineto{\pgfqpoint{3.223285in}{3.000530in}}%
\pgfpathlineto{\pgfqpoint{3.229559in}{2.987293in}}%
\pgfpathlineto{\pgfqpoint{3.235835in}{2.973748in}}%
\pgfpathlineto{\pgfqpoint{3.242112in}{2.959961in}}%
\pgfpathlineto{\pgfqpoint{3.248391in}{2.945998in}}%
\pgfpathclose%
\pgfusepath{stroke,fill}%
\end{pgfscope}%
\begin{pgfscope}%
\pgfpathrectangle{\pgfqpoint{0.887500in}{0.275000in}}{\pgfqpoint{4.225000in}{4.225000in}}%
\pgfusepath{clip}%
\pgfsetbuttcap%
\pgfsetroundjoin%
\definecolor{currentfill}{rgb}{0.319809,0.770914,0.411152}%
\pgfsetfillcolor{currentfill}%
\pgfsetfillopacity{0.700000}%
\pgfsetlinewidth{0.501875pt}%
\definecolor{currentstroke}{rgb}{1.000000,1.000000,1.000000}%
\pgfsetstrokecolor{currentstroke}%
\pgfsetstrokeopacity{0.500000}%
\pgfsetdash{}{0pt}%
\pgfpathmoveto{\pgfqpoint{3.431063in}{2.856515in}}%
\pgfpathlineto{\pgfqpoint{3.442364in}{2.860550in}}%
\pgfpathlineto{\pgfqpoint{3.453658in}{2.864526in}}%
\pgfpathlineto{\pgfqpoint{3.464947in}{2.868446in}}%
\pgfpathlineto{\pgfqpoint{3.476230in}{2.872313in}}%
\pgfpathlineto{\pgfqpoint{3.487508in}{2.876130in}}%
\pgfpathlineto{\pgfqpoint{3.481181in}{2.889455in}}%
\pgfpathlineto{\pgfqpoint{3.474856in}{2.902738in}}%
\pgfpathlineto{\pgfqpoint{3.468534in}{2.915973in}}%
\pgfpathlineto{\pgfqpoint{3.462213in}{2.929158in}}%
\pgfpathlineto{\pgfqpoint{3.455895in}{2.942289in}}%
\pgfpathlineto{\pgfqpoint{3.444622in}{2.938458in}}%
\pgfpathlineto{\pgfqpoint{3.433343in}{2.934574in}}%
\pgfpathlineto{\pgfqpoint{3.422058in}{2.930634in}}%
\pgfpathlineto{\pgfqpoint{3.410768in}{2.926634in}}%
\pgfpathlineto{\pgfqpoint{3.399472in}{2.922569in}}%
\pgfpathlineto{\pgfqpoint{3.405785in}{2.909360in}}%
\pgfpathlineto{\pgfqpoint{3.412100in}{2.896159in}}%
\pgfpathlineto{\pgfqpoint{3.418419in}{2.882958in}}%
\pgfpathlineto{\pgfqpoint{3.424740in}{2.869747in}}%
\pgfpathclose%
\pgfusepath{stroke,fill}%
\end{pgfscope}%
\begin{pgfscope}%
\pgfpathrectangle{\pgfqpoint{0.887500in}{0.275000in}}{\pgfqpoint{4.225000in}{4.225000in}}%
\pgfusepath{clip}%
\pgfsetbuttcap%
\pgfsetroundjoin%
\definecolor{currentfill}{rgb}{0.175841,0.441290,0.557685}%
\pgfsetfillcolor{currentfill}%
\pgfsetfillopacity{0.700000}%
\pgfsetlinewidth{0.501875pt}%
\definecolor{currentstroke}{rgb}{1.000000,1.000000,1.000000}%
\pgfsetstrokecolor{currentstroke}%
\pgfsetstrokeopacity{0.500000}%
\pgfsetdash{}{0pt}%
\pgfpathmoveto{\pgfqpoint{2.804645in}{2.143801in}}%
\pgfpathlineto{\pgfqpoint{2.816091in}{2.147955in}}%
\pgfpathlineto{\pgfqpoint{2.827531in}{2.152386in}}%
\pgfpathlineto{\pgfqpoint{2.838965in}{2.156815in}}%
\pgfpathlineto{\pgfqpoint{2.850396in}{2.160954in}}%
\pgfpathlineto{\pgfqpoint{2.861825in}{2.164516in}}%
\pgfpathlineto{\pgfqpoint{2.855647in}{2.173827in}}%
\pgfpathlineto{\pgfqpoint{2.849474in}{2.183103in}}%
\pgfpathlineto{\pgfqpoint{2.843304in}{2.192344in}}%
\pgfpathlineto{\pgfqpoint{2.837139in}{2.201550in}}%
\pgfpathlineto{\pgfqpoint{2.830977in}{2.210719in}}%
\pgfpathlineto{\pgfqpoint{2.819559in}{2.207141in}}%
\pgfpathlineto{\pgfqpoint{2.808139in}{2.202967in}}%
\pgfpathlineto{\pgfqpoint{2.796716in}{2.198496in}}%
\pgfpathlineto{\pgfqpoint{2.785288in}{2.194028in}}%
\pgfpathlineto{\pgfqpoint{2.773853in}{2.189855in}}%
\pgfpathlineto{\pgfqpoint{2.780003in}{2.180704in}}%
\pgfpathlineto{\pgfqpoint{2.786158in}{2.171524in}}%
\pgfpathlineto{\pgfqpoint{2.792316in}{2.162314in}}%
\pgfpathlineto{\pgfqpoint{2.798478in}{2.153074in}}%
\pgfpathclose%
\pgfusepath{stroke,fill}%
\end{pgfscope}%
\begin{pgfscope}%
\pgfpathrectangle{\pgfqpoint{0.887500in}{0.275000in}}{\pgfqpoint{4.225000in}{4.225000in}}%
\pgfusepath{clip}%
\pgfsetbuttcap%
\pgfsetroundjoin%
\definecolor{currentfill}{rgb}{0.144759,0.519093,0.556572}%
\pgfsetfillcolor{currentfill}%
\pgfsetfillopacity{0.700000}%
\pgfsetlinewidth{0.501875pt}%
\definecolor{currentstroke}{rgb}{1.000000,1.000000,1.000000}%
\pgfsetstrokecolor{currentstroke}%
\pgfsetstrokeopacity{0.500000}%
\pgfsetdash{}{0pt}%
\pgfpathmoveto{\pgfqpoint{4.198884in}{2.299994in}}%
\pgfpathlineto{\pgfqpoint{4.209981in}{2.303216in}}%
\pgfpathlineto{\pgfqpoint{4.221073in}{2.306416in}}%
\pgfpathlineto{\pgfqpoint{4.232158in}{2.309600in}}%
\pgfpathlineto{\pgfqpoint{4.243238in}{2.312783in}}%
\pgfpathlineto{\pgfqpoint{4.254313in}{2.315978in}}%
\pgfpathlineto{\pgfqpoint{4.247872in}{2.330877in}}%
\pgfpathlineto{\pgfqpoint{4.241431in}{2.345695in}}%
\pgfpathlineto{\pgfqpoint{4.234989in}{2.360409in}}%
\pgfpathlineto{\pgfqpoint{4.228547in}{2.375023in}}%
\pgfpathlineto{\pgfqpoint{4.222105in}{2.389547in}}%
\pgfpathlineto{\pgfqpoint{4.211025in}{2.386165in}}%
\pgfpathlineto{\pgfqpoint{4.199941in}{2.382778in}}%
\pgfpathlineto{\pgfqpoint{4.188850in}{2.379391in}}%
\pgfpathlineto{\pgfqpoint{4.177755in}{2.376009in}}%
\pgfpathlineto{\pgfqpoint{4.166654in}{2.372631in}}%
\pgfpathlineto{\pgfqpoint{4.173097in}{2.358182in}}%
\pgfpathlineto{\pgfqpoint{4.179542in}{2.343694in}}%
\pgfpathlineto{\pgfqpoint{4.185988in}{2.329165in}}%
\pgfpathlineto{\pgfqpoint{4.192435in}{2.314596in}}%
\pgfpathclose%
\pgfusepath{stroke,fill}%
\end{pgfscope}%
\begin{pgfscope}%
\pgfpathrectangle{\pgfqpoint{0.887500in}{0.275000in}}{\pgfqpoint{4.225000in}{4.225000in}}%
\pgfusepath{clip}%
\pgfsetbuttcap%
\pgfsetroundjoin%
\definecolor{currentfill}{rgb}{0.386433,0.794644,0.372886}%
\pgfsetfillcolor{currentfill}%
\pgfsetfillopacity{0.700000}%
\pgfsetlinewidth{0.501875pt}%
\definecolor{currentstroke}{rgb}{1.000000,1.000000,1.000000}%
\pgfsetstrokecolor{currentstroke}%
\pgfsetstrokeopacity{0.500000}%
\pgfsetdash{}{0pt}%
\pgfpathmoveto{\pgfqpoint{3.109573in}{2.892743in}}%
\pgfpathlineto{\pgfqpoint{3.120961in}{2.910423in}}%
\pgfpathlineto{\pgfqpoint{3.132346in}{2.925346in}}%
\pgfpathlineto{\pgfqpoint{3.143728in}{2.938000in}}%
\pgfpathlineto{\pgfqpoint{3.155105in}{2.948875in}}%
\pgfpathlineto{\pgfqpoint{3.166478in}{2.958462in}}%
\pgfpathlineto{\pgfqpoint{3.160216in}{2.972675in}}%
\pgfpathlineto{\pgfqpoint{3.153956in}{2.986362in}}%
\pgfpathlineto{\pgfqpoint{3.147697in}{2.999535in}}%
\pgfpathlineto{\pgfqpoint{3.141440in}{3.012205in}}%
\pgfpathlineto{\pgfqpoint{3.135184in}{3.024383in}}%
\pgfpathlineto{\pgfqpoint{3.123823in}{3.016174in}}%
\pgfpathlineto{\pgfqpoint{3.112458in}{3.006760in}}%
\pgfpathlineto{\pgfqpoint{3.101089in}{2.995591in}}%
\pgfpathlineto{\pgfqpoint{3.089716in}{2.982121in}}%
\pgfpathlineto{\pgfqpoint{3.078343in}{2.965803in}}%
\pgfpathlineto{\pgfqpoint{3.084585in}{2.952724in}}%
\pgfpathlineto{\pgfqpoint{3.090830in}{2.938802in}}%
\pgfpathlineto{\pgfqpoint{3.097076in}{2.924113in}}%
\pgfpathlineto{\pgfqpoint{3.103324in}{2.908735in}}%
\pgfpathclose%
\pgfusepath{stroke,fill}%
\end{pgfscope}%
\begin{pgfscope}%
\pgfpathrectangle{\pgfqpoint{0.887500in}{0.275000in}}{\pgfqpoint{4.225000in}{4.225000in}}%
\pgfusepath{clip}%
\pgfsetbuttcap%
\pgfsetroundjoin%
\definecolor{currentfill}{rgb}{0.369214,0.788888,0.382914}%
\pgfsetfillcolor{currentfill}%
\pgfsetfillopacity{0.700000}%
\pgfsetlinewidth{0.501875pt}%
\definecolor{currentstroke}{rgb}{1.000000,1.000000,1.000000}%
\pgfsetstrokecolor{currentstroke}%
\pgfsetstrokeopacity{0.500000}%
\pgfsetdash{}{0pt}%
\pgfpathmoveto{\pgfqpoint{3.342891in}{2.898946in}}%
\pgfpathlineto{\pgfqpoint{3.354222in}{2.904404in}}%
\pgfpathlineto{\pgfqpoint{3.365545in}{2.909406in}}%
\pgfpathlineto{\pgfqpoint{3.376861in}{2.914042in}}%
\pgfpathlineto{\pgfqpoint{3.388170in}{2.918400in}}%
\pgfpathlineto{\pgfqpoint{3.399472in}{2.922569in}}%
\pgfpathlineto{\pgfqpoint{3.393162in}{2.935784in}}%
\pgfpathlineto{\pgfqpoint{3.386854in}{2.948997in}}%
\pgfpathlineto{\pgfqpoint{3.380549in}{2.962200in}}%
\pgfpathlineto{\pgfqpoint{3.374247in}{2.975382in}}%
\pgfpathlineto{\pgfqpoint{3.367947in}{2.988537in}}%
\pgfpathlineto{\pgfqpoint{3.356652in}{2.984733in}}%
\pgfpathlineto{\pgfqpoint{3.345351in}{2.980778in}}%
\pgfpathlineto{\pgfqpoint{3.334043in}{2.976567in}}%
\pgfpathlineto{\pgfqpoint{3.322728in}{2.971991in}}%
\pgfpathlineto{\pgfqpoint{3.311405in}{2.966944in}}%
\pgfpathlineto{\pgfqpoint{3.317697in}{2.953308in}}%
\pgfpathlineto{\pgfqpoint{3.323991in}{2.939652in}}%
\pgfpathlineto{\pgfqpoint{3.330288in}{2.926013in}}%
\pgfpathlineto{\pgfqpoint{3.336588in}{2.912431in}}%
\pgfpathclose%
\pgfusepath{stroke,fill}%
\end{pgfscope}%
\begin{pgfscope}%
\pgfpathrectangle{\pgfqpoint{0.887500in}{0.275000in}}{\pgfqpoint{4.225000in}{4.225000in}}%
\pgfusepath{clip}%
\pgfsetbuttcap%
\pgfsetroundjoin%
\definecolor{currentfill}{rgb}{0.188923,0.410910,0.556326}%
\pgfsetfillcolor{currentfill}%
\pgfsetfillopacity{0.700000}%
\pgfsetlinewidth{0.501875pt}%
\definecolor{currentstroke}{rgb}{1.000000,1.000000,1.000000}%
\pgfsetstrokecolor{currentstroke}%
\pgfsetstrokeopacity{0.500000}%
\pgfsetdash{}{0pt}%
\pgfpathmoveto{\pgfqpoint{4.494742in}{2.067651in}}%
\pgfpathlineto{\pgfqpoint{4.505769in}{2.071002in}}%
\pgfpathlineto{\pgfqpoint{4.516790in}{2.074348in}}%
\pgfpathlineto{\pgfqpoint{4.527805in}{2.077688in}}%
\pgfpathlineto{\pgfqpoint{4.538815in}{2.081024in}}%
\pgfpathlineto{\pgfqpoint{4.549819in}{2.084357in}}%
\pgfpathlineto{\pgfqpoint{4.543308in}{2.098801in}}%
\pgfpathlineto{\pgfqpoint{4.536800in}{2.113228in}}%
\pgfpathlineto{\pgfqpoint{4.530293in}{2.127636in}}%
\pgfpathlineto{\pgfqpoint{4.523788in}{2.142023in}}%
\pgfpathlineto{\pgfqpoint{4.517285in}{2.156386in}}%
\pgfpathlineto{\pgfqpoint{4.506273in}{2.152804in}}%
\pgfpathlineto{\pgfqpoint{4.495254in}{2.149174in}}%
\pgfpathlineto{\pgfqpoint{4.484227in}{2.145493in}}%
\pgfpathlineto{\pgfqpoint{4.473194in}{2.141756in}}%
\pgfpathlineto{\pgfqpoint{4.462153in}{2.137959in}}%
\pgfpathlineto{\pgfqpoint{4.468669in}{2.124004in}}%
\pgfpathlineto{\pgfqpoint{4.475187in}{2.110011in}}%
\pgfpathlineto{\pgfqpoint{4.481705in}{2.095964in}}%
\pgfpathlineto{\pgfqpoint{4.488224in}{2.081848in}}%
\pgfpathclose%
\pgfusepath{stroke,fill}%
\end{pgfscope}%
\begin{pgfscope}%
\pgfpathrectangle{\pgfqpoint{0.887500in}{0.275000in}}{\pgfqpoint{4.225000in}{4.225000in}}%
\pgfusepath{clip}%
\pgfsetbuttcap%
\pgfsetroundjoin%
\definecolor{currentfill}{rgb}{0.220124,0.725509,0.466226}%
\pgfsetfillcolor{currentfill}%
\pgfsetfillopacity{0.700000}%
\pgfsetlinewidth{0.501875pt}%
\definecolor{currentstroke}{rgb}{1.000000,1.000000,1.000000}%
\pgfsetstrokecolor{currentstroke}%
\pgfsetstrokeopacity{0.500000}%
\pgfsetdash{}{0pt}%
\pgfpathmoveto{\pgfqpoint{3.083862in}{2.694002in}}%
\pgfpathlineto{\pgfqpoint{3.095247in}{2.718894in}}%
\pgfpathlineto{\pgfqpoint{3.106638in}{2.743100in}}%
\pgfpathlineto{\pgfqpoint{3.118034in}{2.766247in}}%
\pgfpathlineto{\pgfqpoint{3.129433in}{2.787964in}}%
\pgfpathlineto{\pgfqpoint{3.140833in}{2.807879in}}%
\pgfpathlineto{\pgfqpoint{3.134578in}{2.824969in}}%
\pgfpathlineto{\pgfqpoint{3.128325in}{2.842167in}}%
\pgfpathlineto{\pgfqpoint{3.122073in}{2.859307in}}%
\pgfpathlineto{\pgfqpoint{3.115822in}{2.876221in}}%
\pgfpathlineto{\pgfqpoint{3.109573in}{2.892743in}}%
\pgfpathlineto{\pgfqpoint{3.098184in}{2.871912in}}%
\pgfpathlineto{\pgfqpoint{3.086797in}{2.848233in}}%
\pgfpathlineto{\pgfqpoint{3.075415in}{2.822330in}}%
\pgfpathlineto{\pgfqpoint{3.064041in}{2.794825in}}%
\pgfpathlineto{\pgfqpoint{3.052674in}{2.766339in}}%
\pgfpathlineto{\pgfqpoint{3.058908in}{2.751020in}}%
\pgfpathlineto{\pgfqpoint{3.065143in}{2.736141in}}%
\pgfpathlineto{\pgfqpoint{3.071380in}{2.721688in}}%
\pgfpathlineto{\pgfqpoint{3.077620in}{2.707646in}}%
\pgfpathclose%
\pgfusepath{stroke,fill}%
\end{pgfscope}%
\begin{pgfscope}%
\pgfpathrectangle{\pgfqpoint{0.887500in}{0.275000in}}{\pgfqpoint{4.225000in}{4.225000in}}%
\pgfusepath{clip}%
\pgfsetbuttcap%
\pgfsetroundjoin%
\definecolor{currentfill}{rgb}{0.163625,0.471133,0.558148}%
\pgfsetfillcolor{currentfill}%
\pgfsetfillopacity{0.700000}%
\pgfsetlinewidth{0.501875pt}%
\definecolor{currentstroke}{rgb}{1.000000,1.000000,1.000000}%
\pgfsetstrokecolor{currentstroke}%
\pgfsetstrokeopacity{0.500000}%
\pgfsetdash{}{0pt}%
\pgfpathmoveto{\pgfqpoint{2.483002in}{2.212168in}}%
\pgfpathlineto{\pgfqpoint{2.494535in}{2.215585in}}%
\pgfpathlineto{\pgfqpoint{2.506061in}{2.219016in}}%
\pgfpathlineto{\pgfqpoint{2.517582in}{2.222465in}}%
\pgfpathlineto{\pgfqpoint{2.529096in}{2.225936in}}%
\pgfpathlineto{\pgfqpoint{2.540605in}{2.229431in}}%
\pgfpathlineto{\pgfqpoint{2.534533in}{2.238325in}}%
\pgfpathlineto{\pgfqpoint{2.528465in}{2.247192in}}%
\pgfpathlineto{\pgfqpoint{2.522401in}{2.256033in}}%
\pgfpathlineto{\pgfqpoint{2.516341in}{2.264849in}}%
\pgfpathlineto{\pgfqpoint{2.510286in}{2.273641in}}%
\pgfpathlineto{\pgfqpoint{2.498788in}{2.270169in}}%
\pgfpathlineto{\pgfqpoint{2.487285in}{2.266727in}}%
\pgfpathlineto{\pgfqpoint{2.475775in}{2.263310in}}%
\pgfpathlineto{\pgfqpoint{2.464259in}{2.259914in}}%
\pgfpathlineto{\pgfqpoint{2.452738in}{2.256533in}}%
\pgfpathlineto{\pgfqpoint{2.458782in}{2.247712in}}%
\pgfpathlineto{\pgfqpoint{2.464831in}{2.238866in}}%
\pgfpathlineto{\pgfqpoint{2.470884in}{2.229993in}}%
\pgfpathlineto{\pgfqpoint{2.476941in}{2.221094in}}%
\pgfpathclose%
\pgfusepath{stroke,fill}%
\end{pgfscope}%
\begin{pgfscope}%
\pgfpathrectangle{\pgfqpoint{0.887500in}{0.275000in}}{\pgfqpoint{4.225000in}{4.225000in}}%
\pgfusepath{clip}%
\pgfsetbuttcap%
\pgfsetroundjoin%
\definecolor{currentfill}{rgb}{0.140536,0.530132,0.555659}%
\pgfsetfillcolor{currentfill}%
\pgfsetfillopacity{0.700000}%
\pgfsetlinewidth{0.501875pt}%
\definecolor{currentstroke}{rgb}{1.000000,1.000000,1.000000}%
\pgfsetstrokecolor{currentstroke}%
\pgfsetstrokeopacity{0.500000}%
\pgfsetdash{}{0pt}%
\pgfpathmoveto{\pgfqpoint{1.839746in}{2.337228in}}%
\pgfpathlineto{\pgfqpoint{1.851437in}{2.340645in}}%
\pgfpathlineto{\pgfqpoint{1.863123in}{2.344059in}}%
\pgfpathlineto{\pgfqpoint{1.874803in}{2.347468in}}%
\pgfpathlineto{\pgfqpoint{1.886477in}{2.350871in}}%
\pgfpathlineto{\pgfqpoint{1.898146in}{2.354266in}}%
\pgfpathlineto{\pgfqpoint{1.892292in}{2.362757in}}%
\pgfpathlineto{\pgfqpoint{1.886442in}{2.371228in}}%
\pgfpathlineto{\pgfqpoint{1.880597in}{2.379678in}}%
\pgfpathlineto{\pgfqpoint{1.874756in}{2.388109in}}%
\pgfpathlineto{\pgfqpoint{1.868919in}{2.396519in}}%
\pgfpathlineto{\pgfqpoint{1.857262in}{2.393157in}}%
\pgfpathlineto{\pgfqpoint{1.845599in}{2.389787in}}%
\pgfpathlineto{\pgfqpoint{1.833931in}{2.386411in}}%
\pgfpathlineto{\pgfqpoint{1.822257in}{2.383030in}}%
\pgfpathlineto{\pgfqpoint{1.810577in}{2.379648in}}%
\pgfpathlineto{\pgfqpoint{1.816402in}{2.371207in}}%
\pgfpathlineto{\pgfqpoint{1.822232in}{2.362745in}}%
\pgfpathlineto{\pgfqpoint{1.828065in}{2.354261in}}%
\pgfpathlineto{\pgfqpoint{1.833903in}{2.345755in}}%
\pgfpathclose%
\pgfusepath{stroke,fill}%
\end{pgfscope}%
\begin{pgfscope}%
\pgfpathrectangle{\pgfqpoint{0.887500in}{0.275000in}}{\pgfqpoint{4.225000in}{4.225000in}}%
\pgfusepath{clip}%
\pgfsetbuttcap%
\pgfsetroundjoin%
\definecolor{currentfill}{rgb}{0.150476,0.504369,0.557430}%
\pgfsetfillcolor{currentfill}%
\pgfsetfillopacity{0.700000}%
\pgfsetlinewidth{0.501875pt}%
\definecolor{currentstroke}{rgb}{1.000000,1.000000,1.000000}%
\pgfsetstrokecolor{currentstroke}%
\pgfsetstrokeopacity{0.500000}%
\pgfsetdash{}{0pt}%
\pgfpathmoveto{\pgfqpoint{2.161356in}{2.275820in}}%
\pgfpathlineto{\pgfqpoint{2.172968in}{2.279275in}}%
\pgfpathlineto{\pgfqpoint{2.184573in}{2.282736in}}%
\pgfpathlineto{\pgfqpoint{2.196173in}{2.286201in}}%
\pgfpathlineto{\pgfqpoint{2.207767in}{2.289664in}}%
\pgfpathlineto{\pgfqpoint{2.219356in}{2.293123in}}%
\pgfpathlineto{\pgfqpoint{2.213391in}{2.301797in}}%
\pgfpathlineto{\pgfqpoint{2.207431in}{2.310448in}}%
\pgfpathlineto{\pgfqpoint{2.201475in}{2.319079in}}%
\pgfpathlineto{\pgfqpoint{2.195524in}{2.327690in}}%
\pgfpathlineto{\pgfqpoint{2.189576in}{2.336281in}}%
\pgfpathlineto{\pgfqpoint{2.177999in}{2.332858in}}%
\pgfpathlineto{\pgfqpoint{2.166416in}{2.329431in}}%
\pgfpathlineto{\pgfqpoint{2.154827in}{2.326004in}}%
\pgfpathlineto{\pgfqpoint{2.143233in}{2.322581in}}%
\pgfpathlineto{\pgfqpoint{2.131633in}{2.319164in}}%
\pgfpathlineto{\pgfqpoint{2.137569in}{2.310538in}}%
\pgfpathlineto{\pgfqpoint{2.143509in}{2.301891in}}%
\pgfpathlineto{\pgfqpoint{2.149454in}{2.293223in}}%
\pgfpathlineto{\pgfqpoint{2.155403in}{2.284533in}}%
\pgfpathclose%
\pgfusepath{stroke,fill}%
\end{pgfscope}%
\begin{pgfscope}%
\pgfpathrectangle{\pgfqpoint{0.887500in}{0.275000in}}{\pgfqpoint{4.225000in}{4.225000in}}%
\pgfusepath{clip}%
\pgfsetbuttcap%
\pgfsetroundjoin%
\definecolor{currentfill}{rgb}{0.132268,0.655014,0.519661}%
\pgfsetfillcolor{currentfill}%
\pgfsetfillopacity{0.700000}%
\pgfsetlinewidth{0.501875pt}%
\definecolor{currentstroke}{rgb}{1.000000,1.000000,1.000000}%
\pgfsetstrokecolor{currentstroke}%
\pgfsetstrokeopacity{0.500000}%
\pgfsetdash{}{0pt}%
\pgfpathmoveto{\pgfqpoint{3.815160in}{2.587851in}}%
\pgfpathlineto{\pgfqpoint{3.826363in}{2.591298in}}%
\pgfpathlineto{\pgfqpoint{3.837560in}{2.594729in}}%
\pgfpathlineto{\pgfqpoint{3.848751in}{2.598148in}}%
\pgfpathlineto{\pgfqpoint{3.859937in}{2.601559in}}%
\pgfpathlineto{\pgfqpoint{3.871117in}{2.604966in}}%
\pgfpathlineto{\pgfqpoint{3.864721in}{2.618898in}}%
\pgfpathlineto{\pgfqpoint{3.858327in}{2.632814in}}%
\pgfpathlineto{\pgfqpoint{3.851935in}{2.646713in}}%
\pgfpathlineto{\pgfqpoint{3.845546in}{2.660594in}}%
\pgfpathlineto{\pgfqpoint{3.839158in}{2.674456in}}%
\pgfpathlineto{\pgfqpoint{3.827981in}{2.671054in}}%
\pgfpathlineto{\pgfqpoint{3.816798in}{2.667660in}}%
\pgfpathlineto{\pgfqpoint{3.805610in}{2.664272in}}%
\pgfpathlineto{\pgfqpoint{3.794416in}{2.660887in}}%
\pgfpathlineto{\pgfqpoint{3.783217in}{2.657502in}}%
\pgfpathlineto{\pgfqpoint{3.789601in}{2.643600in}}%
\pgfpathlineto{\pgfqpoint{3.795987in}{2.629681in}}%
\pgfpathlineto{\pgfqpoint{3.802376in}{2.615748in}}%
\pgfpathlineto{\pgfqpoint{3.808767in}{2.601803in}}%
\pgfpathclose%
\pgfusepath{stroke,fill}%
\end{pgfscope}%
\begin{pgfscope}%
\pgfpathrectangle{\pgfqpoint{0.887500in}{0.275000in}}{\pgfqpoint{4.225000in}{4.225000in}}%
\pgfusepath{clip}%
\pgfsetbuttcap%
\pgfsetroundjoin%
\definecolor{currentfill}{rgb}{0.182256,0.426184,0.557120}%
\pgfsetfillcolor{currentfill}%
\pgfsetfillopacity{0.700000}%
\pgfsetlinewidth{0.501875pt}%
\definecolor{currentstroke}{rgb}{1.000000,1.000000,1.000000}%
\pgfsetstrokecolor{currentstroke}%
\pgfsetstrokeopacity{0.500000}%
\pgfsetdash{}{0pt}%
\pgfpathmoveto{\pgfqpoint{2.892773in}{2.117479in}}%
\pgfpathlineto{\pgfqpoint{2.904209in}{2.120289in}}%
\pgfpathlineto{\pgfqpoint{2.915643in}{2.122060in}}%
\pgfpathlineto{\pgfqpoint{2.927075in}{2.122589in}}%
\pgfpathlineto{\pgfqpoint{2.938501in}{2.122362in}}%
\pgfpathlineto{\pgfqpoint{2.949919in}{2.122322in}}%
\pgfpathlineto{\pgfqpoint{2.943714in}{2.131491in}}%
\pgfpathlineto{\pgfqpoint{2.937514in}{2.140611in}}%
\pgfpathlineto{\pgfqpoint{2.931317in}{2.149698in}}%
\pgfpathlineto{\pgfqpoint{2.925124in}{2.158764in}}%
\pgfpathlineto{\pgfqpoint{2.918935in}{2.167826in}}%
\pgfpathlineto{\pgfqpoint{2.907524in}{2.168281in}}%
\pgfpathlineto{\pgfqpoint{2.896103in}{2.168941in}}%
\pgfpathlineto{\pgfqpoint{2.884678in}{2.168765in}}%
\pgfpathlineto{\pgfqpoint{2.873252in}{2.167215in}}%
\pgfpathlineto{\pgfqpoint{2.861825in}{2.164516in}}%
\pgfpathlineto{\pgfqpoint{2.868007in}{2.155173in}}%
\pgfpathlineto{\pgfqpoint{2.874192in}{2.145797in}}%
\pgfpathlineto{\pgfqpoint{2.880382in}{2.136388in}}%
\pgfpathlineto{\pgfqpoint{2.886575in}{2.126949in}}%
\pgfpathclose%
\pgfusepath{stroke,fill}%
\end{pgfscope}%
\begin{pgfscope}%
\pgfpathrectangle{\pgfqpoint{0.887500in}{0.275000in}}{\pgfqpoint{4.225000in}{4.225000in}}%
\pgfusepath{clip}%
\pgfsetbuttcap%
\pgfsetroundjoin%
\definecolor{currentfill}{rgb}{0.177423,0.437527,0.557565}%
\pgfsetfillcolor{currentfill}%
\pgfsetfillopacity{0.700000}%
\pgfsetlinewidth{0.501875pt}%
\definecolor{currentstroke}{rgb}{1.000000,1.000000,1.000000}%
\pgfsetstrokecolor{currentstroke}%
\pgfsetstrokeopacity{0.500000}%
\pgfsetdash{}{0pt}%
\pgfpathmoveto{\pgfqpoint{4.406861in}{2.118717in}}%
\pgfpathlineto{\pgfqpoint{4.417928in}{2.122504in}}%
\pgfpathlineto{\pgfqpoint{4.428991in}{2.126351in}}%
\pgfpathlineto{\pgfqpoint{4.440051in}{2.130228in}}%
\pgfpathlineto{\pgfqpoint{4.451105in}{2.134107in}}%
\pgfpathlineto{\pgfqpoint{4.462153in}{2.137959in}}%
\pgfpathlineto{\pgfqpoint{4.455638in}{2.151890in}}%
\pgfpathlineto{\pgfqpoint{4.449126in}{2.165813in}}%
\pgfpathlineto{\pgfqpoint{4.442617in}{2.179742in}}%
\pgfpathlineto{\pgfqpoint{4.436112in}{2.193694in}}%
\pgfpathlineto{\pgfqpoint{4.429611in}{2.207682in}}%
\pgfpathlineto{\pgfqpoint{4.418549in}{2.203375in}}%
\pgfpathlineto{\pgfqpoint{4.407481in}{2.199030in}}%
\pgfpathlineto{\pgfqpoint{4.396410in}{2.194710in}}%
\pgfpathlineto{\pgfqpoint{4.385336in}{2.190473in}}%
\pgfpathlineto{\pgfqpoint{4.374262in}{2.186382in}}%
\pgfpathlineto{\pgfqpoint{4.380769in}{2.172642in}}%
\pgfpathlineto{\pgfqpoint{4.387284in}{2.159045in}}%
\pgfpathlineto{\pgfqpoint{4.393805in}{2.145551in}}%
\pgfpathlineto{\pgfqpoint{4.400331in}{2.132122in}}%
\pgfpathclose%
\pgfusepath{stroke,fill}%
\end{pgfscope}%
\begin{pgfscope}%
\pgfpathrectangle{\pgfqpoint{0.887500in}{0.275000in}}{\pgfqpoint{4.225000in}{4.225000in}}%
\pgfusepath{clip}%
\pgfsetbuttcap%
\pgfsetroundjoin%
\definecolor{currentfill}{rgb}{0.133743,0.548535,0.553541}%
\pgfsetfillcolor{currentfill}%
\pgfsetfillopacity{0.700000}%
\pgfsetlinewidth{0.501875pt}%
\definecolor{currentstroke}{rgb}{1.000000,1.000000,1.000000}%
\pgfsetstrokecolor{currentstroke}%
\pgfsetstrokeopacity{0.500000}%
\pgfsetdash{}{0pt}%
\pgfpathmoveto{\pgfqpoint{4.111066in}{2.355632in}}%
\pgfpathlineto{\pgfqpoint{4.122195in}{2.359063in}}%
\pgfpathlineto{\pgfqpoint{4.133319in}{2.362473in}}%
\pgfpathlineto{\pgfqpoint{4.144436in}{2.365868in}}%
\pgfpathlineto{\pgfqpoint{4.155548in}{2.369252in}}%
\pgfpathlineto{\pgfqpoint{4.166654in}{2.372631in}}%
\pgfpathlineto{\pgfqpoint{4.160212in}{2.387041in}}%
\pgfpathlineto{\pgfqpoint{4.153772in}{2.401414in}}%
\pgfpathlineto{\pgfqpoint{4.147333in}{2.415751in}}%
\pgfpathlineto{\pgfqpoint{4.140896in}{2.430054in}}%
\pgfpathlineto{\pgfqpoint{4.134460in}{2.444324in}}%
\pgfpathlineto{\pgfqpoint{4.123356in}{2.440963in}}%
\pgfpathlineto{\pgfqpoint{4.112247in}{2.437597in}}%
\pgfpathlineto{\pgfqpoint{4.101131in}{2.434223in}}%
\pgfpathlineto{\pgfqpoint{4.090010in}{2.430838in}}%
\pgfpathlineto{\pgfqpoint{4.078883in}{2.427436in}}%
\pgfpathlineto{\pgfqpoint{4.085316in}{2.413120in}}%
\pgfpathlineto{\pgfqpoint{4.091750in}{2.398778in}}%
\pgfpathlineto{\pgfqpoint{4.098187in}{2.384414in}}%
\pgfpathlineto{\pgfqpoint{4.104625in}{2.370030in}}%
\pgfpathclose%
\pgfusepath{stroke,fill}%
\end{pgfscope}%
\begin{pgfscope}%
\pgfpathrectangle{\pgfqpoint{0.887500in}{0.275000in}}{\pgfqpoint{4.225000in}{4.225000in}}%
\pgfusepath{clip}%
\pgfsetbuttcap%
\pgfsetroundjoin%
\definecolor{currentfill}{rgb}{0.231674,0.318106,0.544834}%
\pgfsetfillcolor{currentfill}%
\pgfsetfillopacity{0.700000}%
\pgfsetlinewidth{0.501875pt}%
\definecolor{currentstroke}{rgb}{1.000000,1.000000,1.000000}%
\pgfsetstrokecolor{currentstroke}%
\pgfsetstrokeopacity{0.500000}%
\pgfsetdash{}{0pt}%
\pgfpathmoveto{\pgfqpoint{4.702659in}{1.882339in}}%
\pgfpathlineto{\pgfqpoint{4.713633in}{1.885677in}}%
\pgfpathlineto{\pgfqpoint{4.724602in}{1.889019in}}%
\pgfpathlineto{\pgfqpoint{4.735566in}{1.892370in}}%
\pgfpathlineto{\pgfqpoint{4.746525in}{1.895731in}}%
\pgfpathlineto{\pgfqpoint{4.739985in}{1.910438in}}%
\pgfpathlineto{\pgfqpoint{4.733446in}{1.925111in}}%
\pgfpathlineto{\pgfqpoint{4.726909in}{1.939754in}}%
\pgfpathlineto{\pgfqpoint{4.720373in}{1.954370in}}%
\pgfpathlineto{\pgfqpoint{4.713838in}{1.968962in}}%
\pgfpathlineto{\pgfqpoint{4.702880in}{1.965604in}}%
\pgfpathlineto{\pgfqpoint{4.691917in}{1.962256in}}%
\pgfpathlineto{\pgfqpoint{4.680949in}{1.958917in}}%
\pgfpathlineto{\pgfqpoint{4.669976in}{1.955586in}}%
\pgfpathlineto{\pgfqpoint{4.676509in}{1.940986in}}%
\pgfpathlineto{\pgfqpoint{4.683044in}{1.926364in}}%
\pgfpathlineto{\pgfqpoint{4.689581in}{1.911717in}}%
\pgfpathlineto{\pgfqpoint{4.696119in}{1.897044in}}%
\pgfpathclose%
\pgfusepath{stroke,fill}%
\end{pgfscope}%
\begin{pgfscope}%
\pgfpathrectangle{\pgfqpoint{0.887500in}{0.275000in}}{\pgfqpoint{4.225000in}{4.225000in}}%
\pgfusepath{clip}%
\pgfsetbuttcap%
\pgfsetroundjoin%
\definecolor{currentfill}{rgb}{0.153894,0.680203,0.504172}%
\pgfsetfillcolor{currentfill}%
\pgfsetfillopacity{0.700000}%
\pgfsetlinewidth{0.501875pt}%
\definecolor{currentstroke}{rgb}{1.000000,1.000000,1.000000}%
\pgfsetstrokecolor{currentstroke}%
\pgfsetstrokeopacity{0.500000}%
\pgfsetdash{}{0pt}%
\pgfpathmoveto{\pgfqpoint{3.727134in}{2.640213in}}%
\pgfpathlineto{\pgfqpoint{3.738363in}{2.643755in}}%
\pgfpathlineto{\pgfqpoint{3.749586in}{2.647245in}}%
\pgfpathlineto{\pgfqpoint{3.760802in}{2.650691in}}%
\pgfpathlineto{\pgfqpoint{3.772013in}{2.654107in}}%
\pgfpathlineto{\pgfqpoint{3.783217in}{2.657502in}}%
\pgfpathlineto{\pgfqpoint{3.776836in}{2.671382in}}%
\pgfpathlineto{\pgfqpoint{3.770456in}{2.685238in}}%
\pgfpathlineto{\pgfqpoint{3.764079in}{2.699066in}}%
\pgfpathlineto{\pgfqpoint{3.757703in}{2.712863in}}%
\pgfpathlineto{\pgfqpoint{3.751330in}{2.726628in}}%
\pgfpathlineto{\pgfqpoint{3.740129in}{2.723281in}}%
\pgfpathlineto{\pgfqpoint{3.728922in}{2.719918in}}%
\pgfpathlineto{\pgfqpoint{3.717709in}{2.716528in}}%
\pgfpathlineto{\pgfqpoint{3.706491in}{2.713099in}}%
\pgfpathlineto{\pgfqpoint{3.695266in}{2.709620in}}%
\pgfpathlineto{\pgfqpoint{3.701636in}{2.695838in}}%
\pgfpathlineto{\pgfqpoint{3.708008in}{2.682002in}}%
\pgfpathlineto{\pgfqpoint{3.714381in}{2.668114in}}%
\pgfpathlineto{\pgfqpoint{3.720757in}{2.654181in}}%
\pgfpathclose%
\pgfusepath{stroke,fill}%
\end{pgfscope}%
\begin{pgfscope}%
\pgfpathrectangle{\pgfqpoint{0.887500in}{0.275000in}}{\pgfqpoint{4.225000in}{4.225000in}}%
\pgfusepath{clip}%
\pgfsetbuttcap%
\pgfsetroundjoin%
\definecolor{currentfill}{rgb}{0.133743,0.548535,0.553541}%
\pgfsetfillcolor{currentfill}%
\pgfsetfillopacity{0.700000}%
\pgfsetlinewidth{0.501875pt}%
\definecolor{currentstroke}{rgb}{1.000000,1.000000,1.000000}%
\pgfsetstrokecolor{currentstroke}%
\pgfsetstrokeopacity{0.500000}%
\pgfsetdash{}{0pt}%
\pgfpathmoveto{\pgfqpoint{1.605813in}{2.370894in}}%
\pgfpathlineto{\pgfqpoint{1.617566in}{2.374321in}}%
\pgfpathlineto{\pgfqpoint{1.629313in}{2.377736in}}%
\pgfpathlineto{\pgfqpoint{1.641055in}{2.381139in}}%
\pgfpathlineto{\pgfqpoint{1.652791in}{2.384532in}}%
\pgfpathlineto{\pgfqpoint{1.664522in}{2.387916in}}%
\pgfpathlineto{\pgfqpoint{1.658748in}{2.396268in}}%
\pgfpathlineto{\pgfqpoint{1.652979in}{2.404594in}}%
\pgfpathlineto{\pgfqpoint{1.647214in}{2.412896in}}%
\pgfpathlineto{\pgfqpoint{1.641454in}{2.421174in}}%
\pgfpathlineto{\pgfqpoint{1.635698in}{2.429429in}}%
\pgfpathlineto{\pgfqpoint{1.623980in}{2.426065in}}%
\pgfpathlineto{\pgfqpoint{1.612256in}{2.422691in}}%
\pgfpathlineto{\pgfqpoint{1.600526in}{2.419307in}}%
\pgfpathlineto{\pgfqpoint{1.588792in}{2.415911in}}%
\pgfpathlineto{\pgfqpoint{1.577051in}{2.412503in}}%
\pgfpathlineto{\pgfqpoint{1.582794in}{2.404231in}}%
\pgfpathlineto{\pgfqpoint{1.588542in}{2.395935in}}%
\pgfpathlineto{\pgfqpoint{1.594294in}{2.387614in}}%
\pgfpathlineto{\pgfqpoint{1.600051in}{2.379267in}}%
\pgfpathclose%
\pgfusepath{stroke,fill}%
\end{pgfscope}%
\begin{pgfscope}%
\pgfpathrectangle{\pgfqpoint{0.887500in}{0.275000in}}{\pgfqpoint{4.225000in}{4.225000in}}%
\pgfusepath{clip}%
\pgfsetbuttcap%
\pgfsetroundjoin%
\definecolor{currentfill}{rgb}{0.168126,0.459988,0.558082}%
\pgfsetfillcolor{currentfill}%
\pgfsetfillopacity{0.700000}%
\pgfsetlinewidth{0.501875pt}%
\definecolor{currentstroke}{rgb}{1.000000,1.000000,1.000000}%
\pgfsetstrokecolor{currentstroke}%
\pgfsetstrokeopacity{0.500000}%
\pgfsetdash{}{0pt}%
\pgfpathmoveto{\pgfqpoint{4.318929in}{2.169415in}}%
\pgfpathlineto{\pgfqpoint{4.329993in}{2.172395in}}%
\pgfpathlineto{\pgfqpoint{4.341057in}{2.175528in}}%
\pgfpathlineto{\pgfqpoint{4.352121in}{2.178876in}}%
\pgfpathlineto{\pgfqpoint{4.363190in}{2.182496in}}%
\pgfpathlineto{\pgfqpoint{4.374262in}{2.186382in}}%
\pgfpathlineto{\pgfqpoint{4.367764in}{2.200303in}}%
\pgfpathlineto{\pgfqpoint{4.361275in}{2.214444in}}%
\pgfpathlineto{\pgfqpoint{4.354797in}{2.228803in}}%
\pgfpathlineto{\pgfqpoint{4.348327in}{2.243347in}}%
\pgfpathlineto{\pgfqpoint{4.341865in}{2.258044in}}%
\pgfpathlineto{\pgfqpoint{4.330799in}{2.254280in}}%
\pgfpathlineto{\pgfqpoint{4.319735in}{2.250745in}}%
\pgfpathlineto{\pgfqpoint{4.308673in}{2.247446in}}%
\pgfpathlineto{\pgfqpoint{4.297611in}{2.244335in}}%
\pgfpathlineto{\pgfqpoint{4.286549in}{2.241364in}}%
\pgfpathlineto{\pgfqpoint{4.293009in}{2.226638in}}%
\pgfpathlineto{\pgfqpoint{4.299476in}{2.212052in}}%
\pgfpathlineto{\pgfqpoint{4.305951in}{2.197638in}}%
\pgfpathlineto{\pgfqpoint{4.312436in}{2.183424in}}%
\pgfpathclose%
\pgfusepath{stroke,fill}%
\end{pgfscope}%
\begin{pgfscope}%
\pgfpathrectangle{\pgfqpoint{0.887500in}{0.275000in}}{\pgfqpoint{4.225000in}{4.225000in}}%
\pgfusepath{clip}%
\pgfsetbuttcap%
\pgfsetroundjoin%
\definecolor{currentfill}{rgb}{0.166617,0.463708,0.558119}%
\pgfsetfillcolor{currentfill}%
\pgfsetfillopacity{0.700000}%
\pgfsetlinewidth{0.501875pt}%
\definecolor{currentstroke}{rgb}{1.000000,1.000000,1.000000}%
\pgfsetstrokecolor{currentstroke}%
\pgfsetstrokeopacity{0.500000}%
\pgfsetdash{}{0pt}%
\pgfpathmoveto{\pgfqpoint{2.571030in}{2.184527in}}%
\pgfpathlineto{\pgfqpoint{2.582543in}{2.188075in}}%
\pgfpathlineto{\pgfqpoint{2.594051in}{2.191651in}}%
\pgfpathlineto{\pgfqpoint{2.605553in}{2.195249in}}%
\pgfpathlineto{\pgfqpoint{2.617049in}{2.198842in}}%
\pgfpathlineto{\pgfqpoint{2.628540in}{2.202403in}}%
\pgfpathlineto{\pgfqpoint{2.622436in}{2.211420in}}%
\pgfpathlineto{\pgfqpoint{2.616335in}{2.220409in}}%
\pgfpathlineto{\pgfqpoint{2.610239in}{2.229370in}}%
\pgfpathlineto{\pgfqpoint{2.604148in}{2.238305in}}%
\pgfpathlineto{\pgfqpoint{2.598060in}{2.247214in}}%
\pgfpathlineto{\pgfqpoint{2.586580in}{2.243667in}}%
\pgfpathlineto{\pgfqpoint{2.575095in}{2.240088in}}%
\pgfpathlineto{\pgfqpoint{2.563604in}{2.236508in}}%
\pgfpathlineto{\pgfqpoint{2.552108in}{2.232954in}}%
\pgfpathlineto{\pgfqpoint{2.540605in}{2.229431in}}%
\pgfpathlineto{\pgfqpoint{2.546682in}{2.220508in}}%
\pgfpathlineto{\pgfqpoint{2.552762in}{2.211557in}}%
\pgfpathlineto{\pgfqpoint{2.558847in}{2.202577in}}%
\pgfpathlineto{\pgfqpoint{2.564936in}{2.193567in}}%
\pgfpathclose%
\pgfusepath{stroke,fill}%
\end{pgfscope}%
\begin{pgfscope}%
\pgfpathrectangle{\pgfqpoint{0.887500in}{0.275000in}}{\pgfqpoint{4.225000in}{4.225000in}}%
\pgfusepath{clip}%
\pgfsetbuttcap%
\pgfsetroundjoin%
\definecolor{currentfill}{rgb}{0.187231,0.414746,0.556547}%
\pgfsetfillcolor{currentfill}%
\pgfsetfillopacity{0.700000}%
\pgfsetlinewidth{0.501875pt}%
\definecolor{currentstroke}{rgb}{1.000000,1.000000,1.000000}%
\pgfsetstrokecolor{currentstroke}%
\pgfsetstrokeopacity{0.500000}%
\pgfsetdash{}{0pt}%
\pgfpathmoveto{\pgfqpoint{2.981004in}{2.075445in}}%
\pgfpathlineto{\pgfqpoint{2.992420in}{2.076725in}}%
\pgfpathlineto{\pgfqpoint{3.003826in}{2.079975in}}%
\pgfpathlineto{\pgfqpoint{3.015223in}{2.086090in}}%
\pgfpathlineto{\pgfqpoint{3.026613in}{2.095966in}}%
\pgfpathlineto{\pgfqpoint{3.038000in}{2.110352in}}%
\pgfpathlineto{\pgfqpoint{3.031762in}{2.119954in}}%
\pgfpathlineto{\pgfqpoint{3.025528in}{2.129517in}}%
\pgfpathlineto{\pgfqpoint{3.019298in}{2.139047in}}%
\pgfpathlineto{\pgfqpoint{3.013071in}{2.148548in}}%
\pgfpathlineto{\pgfqpoint{3.006849in}{2.158026in}}%
\pgfpathlineto{\pgfqpoint{2.995482in}{2.143036in}}%
\pgfpathlineto{\pgfqpoint{2.984108in}{2.132828in}}%
\pgfpathlineto{\pgfqpoint{2.972723in}{2.126606in}}%
\pgfpathlineto{\pgfqpoint{2.961327in}{2.123421in}}%
\pgfpathlineto{\pgfqpoint{2.949919in}{2.122322in}}%
\pgfpathlineto{\pgfqpoint{2.956128in}{2.113091in}}%
\pgfpathlineto{\pgfqpoint{2.962341in}{2.103783in}}%
\pgfpathlineto{\pgfqpoint{2.968558in}{2.094401in}}%
\pgfpathlineto{\pgfqpoint{2.974779in}{2.084953in}}%
\pgfpathclose%
\pgfusepath{stroke,fill}%
\end{pgfscope}%
\begin{pgfscope}%
\pgfpathrectangle{\pgfqpoint{0.887500in}{0.275000in}}{\pgfqpoint{4.225000in}{4.225000in}}%
\pgfusepath{clip}%
\pgfsetbuttcap%
\pgfsetroundjoin%
\definecolor{currentfill}{rgb}{0.143343,0.522773,0.556295}%
\pgfsetfillcolor{currentfill}%
\pgfsetfillopacity{0.700000}%
\pgfsetlinewidth{0.501875pt}%
\definecolor{currentstroke}{rgb}{1.000000,1.000000,1.000000}%
\pgfsetstrokecolor{currentstroke}%
\pgfsetstrokeopacity{0.500000}%
\pgfsetdash{}{0pt}%
\pgfpathmoveto{\pgfqpoint{1.927482in}{2.311489in}}%
\pgfpathlineto{\pgfqpoint{1.939157in}{2.314908in}}%
\pgfpathlineto{\pgfqpoint{1.950826in}{2.318315in}}%
\pgfpathlineto{\pgfqpoint{1.962490in}{2.321709in}}%
\pgfpathlineto{\pgfqpoint{1.974149in}{2.325092in}}%
\pgfpathlineto{\pgfqpoint{1.985802in}{2.328465in}}%
\pgfpathlineto{\pgfqpoint{1.979914in}{2.337032in}}%
\pgfpathlineto{\pgfqpoint{1.974031in}{2.345577in}}%
\pgfpathlineto{\pgfqpoint{1.968152in}{2.354101in}}%
\pgfpathlineto{\pgfqpoint{1.962278in}{2.362603in}}%
\pgfpathlineto{\pgfqpoint{1.956407in}{2.371084in}}%
\pgfpathlineto{\pgfqpoint{1.944766in}{2.367740in}}%
\pgfpathlineto{\pgfqpoint{1.933120in}{2.364387in}}%
\pgfpathlineto{\pgfqpoint{1.921467in}{2.361024in}}%
\pgfpathlineto{\pgfqpoint{1.909809in}{2.357650in}}%
\pgfpathlineto{\pgfqpoint{1.898146in}{2.354266in}}%
\pgfpathlineto{\pgfqpoint{1.904005in}{2.345754in}}%
\pgfpathlineto{\pgfqpoint{1.909867in}{2.337221in}}%
\pgfpathlineto{\pgfqpoint{1.915735in}{2.328667in}}%
\pgfpathlineto{\pgfqpoint{1.921606in}{2.320089in}}%
\pgfpathclose%
\pgfusepath{stroke,fill}%
\end{pgfscope}%
\begin{pgfscope}%
\pgfpathrectangle{\pgfqpoint{0.887500in}{0.275000in}}{\pgfqpoint{4.225000in}{4.225000in}}%
\pgfusepath{clip}%
\pgfsetbuttcap%
\pgfsetroundjoin%
\definecolor{currentfill}{rgb}{0.154815,0.493313,0.557840}%
\pgfsetfillcolor{currentfill}%
\pgfsetfillopacity{0.700000}%
\pgfsetlinewidth{0.501875pt}%
\definecolor{currentstroke}{rgb}{1.000000,1.000000,1.000000}%
\pgfsetstrokecolor{currentstroke}%
\pgfsetstrokeopacity{0.500000}%
\pgfsetdash{}{0pt}%
\pgfpathmoveto{\pgfqpoint{2.249242in}{2.249405in}}%
\pgfpathlineto{\pgfqpoint{2.260837in}{2.252897in}}%
\pgfpathlineto{\pgfqpoint{2.272425in}{2.256377in}}%
\pgfpathlineto{\pgfqpoint{2.284008in}{2.259843in}}%
\pgfpathlineto{\pgfqpoint{2.295586in}{2.263294in}}%
\pgfpathlineto{\pgfqpoint{2.307158in}{2.266730in}}%
\pgfpathlineto{\pgfqpoint{2.301161in}{2.275481in}}%
\pgfpathlineto{\pgfqpoint{2.295168in}{2.284209in}}%
\pgfpathlineto{\pgfqpoint{2.289180in}{2.292913in}}%
\pgfpathlineto{\pgfqpoint{2.283196in}{2.301595in}}%
\pgfpathlineto{\pgfqpoint{2.277216in}{2.310255in}}%
\pgfpathlineto{\pgfqpoint{2.265655in}{2.306856in}}%
\pgfpathlineto{\pgfqpoint{2.254088in}{2.303442in}}%
\pgfpathlineto{\pgfqpoint{2.242516in}{2.300014in}}%
\pgfpathlineto{\pgfqpoint{2.230939in}{2.296574in}}%
\pgfpathlineto{\pgfqpoint{2.219356in}{2.293123in}}%
\pgfpathlineto{\pgfqpoint{2.225325in}{2.284427in}}%
\pgfpathlineto{\pgfqpoint{2.231298in}{2.275708in}}%
\pgfpathlineto{\pgfqpoint{2.237275in}{2.266965in}}%
\pgfpathlineto{\pgfqpoint{2.243257in}{2.258198in}}%
\pgfpathclose%
\pgfusepath{stroke,fill}%
\end{pgfscope}%
\begin{pgfscope}%
\pgfpathrectangle{\pgfqpoint{0.887500in}{0.275000in}}{\pgfqpoint{4.225000in}{4.225000in}}%
\pgfusepath{clip}%
\pgfsetbuttcap%
\pgfsetroundjoin%
\definecolor{currentfill}{rgb}{0.369214,0.788888,0.382914}%
\pgfsetfillcolor{currentfill}%
\pgfsetfillopacity{0.700000}%
\pgfsetlinewidth{0.501875pt}%
\definecolor{currentstroke}{rgb}{1.000000,1.000000,1.000000}%
\pgfsetstrokecolor{currentstroke}%
\pgfsetstrokeopacity{0.500000}%
\pgfsetdash{}{0pt}%
\pgfpathmoveto{\pgfqpoint{3.197799in}{2.881646in}}%
\pgfpathlineto{\pgfqpoint{3.209182in}{2.892838in}}%
\pgfpathlineto{\pgfqpoint{3.220560in}{2.903395in}}%
\pgfpathlineto{\pgfqpoint{3.231935in}{2.913447in}}%
\pgfpathlineto{\pgfqpoint{3.243306in}{2.922969in}}%
\pgfpathlineto{\pgfqpoint{3.254672in}{2.931925in}}%
\pgfpathlineto{\pgfqpoint{3.248391in}{2.945998in}}%
\pgfpathlineto{\pgfqpoint{3.242112in}{2.959961in}}%
\pgfpathlineto{\pgfqpoint{3.235835in}{2.973748in}}%
\pgfpathlineto{\pgfqpoint{3.229559in}{2.987293in}}%
\pgfpathlineto{\pgfqpoint{3.223285in}{3.000530in}}%
\pgfpathlineto{\pgfqpoint{3.211930in}{2.992471in}}%
\pgfpathlineto{\pgfqpoint{3.200571in}{2.984169in}}%
\pgfpathlineto{\pgfqpoint{3.189210in}{2.975735in}}%
\pgfpathlineto{\pgfqpoint{3.177845in}{2.967253in}}%
\pgfpathlineto{\pgfqpoint{3.166478in}{2.958462in}}%
\pgfpathlineto{\pgfqpoint{3.172740in}{2.943723in}}%
\pgfpathlineto{\pgfqpoint{3.179003in}{2.928542in}}%
\pgfpathlineto{\pgfqpoint{3.185267in}{2.913048in}}%
\pgfpathlineto{\pgfqpoint{3.191532in}{2.897373in}}%
\pgfpathclose%
\pgfusepath{stroke,fill}%
\end{pgfscope}%
\begin{pgfscope}%
\pgfpathrectangle{\pgfqpoint{0.887500in}{0.275000in}}{\pgfqpoint{4.225000in}{4.225000in}}%
\pgfusepath{clip}%
\pgfsetbuttcap%
\pgfsetroundjoin%
\definecolor{currentfill}{rgb}{0.125394,0.574318,0.549086}%
\pgfsetfillcolor{currentfill}%
\pgfsetfillopacity{0.700000}%
\pgfsetlinewidth{0.501875pt}%
\definecolor{currentstroke}{rgb}{1.000000,1.000000,1.000000}%
\pgfsetstrokecolor{currentstroke}%
\pgfsetstrokeopacity{0.500000}%
\pgfsetdash{}{0pt}%
\pgfpathmoveto{\pgfqpoint{4.023162in}{2.410209in}}%
\pgfpathlineto{\pgfqpoint{4.034317in}{2.413670in}}%
\pgfpathlineto{\pgfqpoint{4.045467in}{2.417128in}}%
\pgfpathlineto{\pgfqpoint{4.056612in}{2.420578in}}%
\pgfpathlineto{\pgfqpoint{4.067751in}{2.424016in}}%
\pgfpathlineto{\pgfqpoint{4.078883in}{2.427436in}}%
\pgfpathlineto{\pgfqpoint{4.072453in}{2.441723in}}%
\pgfpathlineto{\pgfqpoint{4.066023in}{2.455977in}}%
\pgfpathlineto{\pgfqpoint{4.059596in}{2.470195in}}%
\pgfpathlineto{\pgfqpoint{4.053170in}{2.484378in}}%
\pgfpathlineto{\pgfqpoint{4.046746in}{2.498529in}}%
\pgfpathlineto{\pgfqpoint{4.035616in}{2.495159in}}%
\pgfpathlineto{\pgfqpoint{4.024481in}{2.491776in}}%
\pgfpathlineto{\pgfqpoint{4.013340in}{2.488383in}}%
\pgfpathlineto{\pgfqpoint{4.002193in}{2.484984in}}%
\pgfpathlineto{\pgfqpoint{3.991041in}{2.481581in}}%
\pgfpathlineto{\pgfqpoint{3.997462in}{2.467391in}}%
\pgfpathlineto{\pgfqpoint{4.003885in}{2.453160in}}%
\pgfpathlineto{\pgfqpoint{4.010309in}{2.438884in}}%
\pgfpathlineto{\pgfqpoint{4.016735in}{2.424565in}}%
\pgfpathclose%
\pgfusepath{stroke,fill}%
\end{pgfscope}%
\begin{pgfscope}%
\pgfpathrectangle{\pgfqpoint{0.887500in}{0.275000in}}{\pgfqpoint{4.225000in}{4.225000in}}%
\pgfusepath{clip}%
\pgfsetbuttcap%
\pgfsetroundjoin%
\definecolor{currentfill}{rgb}{0.218130,0.347432,0.550038}%
\pgfsetfillcolor{currentfill}%
\pgfsetfillopacity{0.700000}%
\pgfsetlinewidth{0.501875pt}%
\definecolor{currentstroke}{rgb}{1.000000,1.000000,1.000000}%
\pgfsetstrokecolor{currentstroke}%
\pgfsetstrokeopacity{0.500000}%
\pgfsetdash{}{0pt}%
\pgfpathmoveto{\pgfqpoint{4.615030in}{1.938997in}}%
\pgfpathlineto{\pgfqpoint{4.626030in}{1.942310in}}%
\pgfpathlineto{\pgfqpoint{4.637024in}{1.945625in}}%
\pgfpathlineto{\pgfqpoint{4.648013in}{1.948941in}}%
\pgfpathlineto{\pgfqpoint{4.658997in}{1.952261in}}%
\pgfpathlineto{\pgfqpoint{4.669976in}{1.955586in}}%
\pgfpathlineto{\pgfqpoint{4.663444in}{1.970167in}}%
\pgfpathlineto{\pgfqpoint{4.656914in}{1.984732in}}%
\pgfpathlineto{\pgfqpoint{4.650386in}{1.999283in}}%
\pgfpathlineto{\pgfqpoint{4.643860in}{2.013824in}}%
\pgfpathlineto{\pgfqpoint{4.637337in}{2.028357in}}%
\pgfpathlineto{\pgfqpoint{4.626360in}{2.025047in}}%
\pgfpathlineto{\pgfqpoint{4.615378in}{2.021745in}}%
\pgfpathlineto{\pgfqpoint{4.604391in}{2.018452in}}%
\pgfpathlineto{\pgfqpoint{4.593398in}{2.015167in}}%
\pgfpathlineto{\pgfqpoint{4.582401in}{2.011890in}}%
\pgfpathlineto{\pgfqpoint{4.588923in}{1.997346in}}%
\pgfpathlineto{\pgfqpoint{4.595447in}{1.982785in}}%
\pgfpathlineto{\pgfqpoint{4.601973in}{1.968206in}}%
\pgfpathlineto{\pgfqpoint{4.608500in}{1.953610in}}%
\pgfpathclose%
\pgfusepath{stroke,fill}%
\end{pgfscope}%
\begin{pgfscope}%
\pgfpathrectangle{\pgfqpoint{0.887500in}{0.275000in}}{\pgfqpoint{4.225000in}{4.225000in}}%
\pgfusepath{clip}%
\pgfsetbuttcap%
\pgfsetroundjoin%
\definecolor{currentfill}{rgb}{0.185783,0.704891,0.485273}%
\pgfsetfillcolor{currentfill}%
\pgfsetfillopacity{0.700000}%
\pgfsetlinewidth{0.501875pt}%
\definecolor{currentstroke}{rgb}{1.000000,1.000000,1.000000}%
\pgfsetstrokecolor{currentstroke}%
\pgfsetstrokeopacity{0.500000}%
\pgfsetdash{}{0pt}%
\pgfpathmoveto{\pgfqpoint{3.639047in}{2.691372in}}%
\pgfpathlineto{\pgfqpoint{3.650302in}{2.695104in}}%
\pgfpathlineto{\pgfqpoint{3.661552in}{2.698808in}}%
\pgfpathlineto{\pgfqpoint{3.672796in}{2.702472in}}%
\pgfpathlineto{\pgfqpoint{3.684034in}{2.706080in}}%
\pgfpathlineto{\pgfqpoint{3.695266in}{2.709620in}}%
\pgfpathlineto{\pgfqpoint{3.688897in}{2.723352in}}%
\pgfpathlineto{\pgfqpoint{3.682531in}{2.737036in}}%
\pgfpathlineto{\pgfqpoint{3.676166in}{2.750676in}}%
\pgfpathlineto{\pgfqpoint{3.669804in}{2.764275in}}%
\pgfpathlineto{\pgfqpoint{3.663443in}{2.777835in}}%
\pgfpathlineto{\pgfqpoint{3.652215in}{2.774310in}}%
\pgfpathlineto{\pgfqpoint{3.640981in}{2.770735in}}%
\pgfpathlineto{\pgfqpoint{3.629741in}{2.767117in}}%
\pgfpathlineto{\pgfqpoint{3.618495in}{2.763465in}}%
\pgfpathlineto{\pgfqpoint{3.607244in}{2.759788in}}%
\pgfpathlineto{\pgfqpoint{3.613600in}{2.746188in}}%
\pgfpathlineto{\pgfqpoint{3.619959in}{2.732551in}}%
\pgfpathlineto{\pgfqpoint{3.626320in}{2.718873in}}%
\pgfpathlineto{\pgfqpoint{3.632682in}{2.705149in}}%
\pgfpathclose%
\pgfusepath{stroke,fill}%
\end{pgfscope}%
\begin{pgfscope}%
\pgfpathrectangle{\pgfqpoint{0.887500in}{0.275000in}}{\pgfqpoint{4.225000in}{4.225000in}}%
\pgfusepath{clip}%
\pgfsetbuttcap%
\pgfsetroundjoin%
\definecolor{currentfill}{rgb}{0.171176,0.452530,0.557965}%
\pgfsetfillcolor{currentfill}%
\pgfsetfillopacity{0.700000}%
\pgfsetlinewidth{0.501875pt}%
\definecolor{currentstroke}{rgb}{1.000000,1.000000,1.000000}%
\pgfsetstrokecolor{currentstroke}%
\pgfsetstrokeopacity{0.500000}%
\pgfsetdash{}{0pt}%
\pgfpathmoveto{\pgfqpoint{3.038000in}{2.110352in}}%
\pgfpathlineto{\pgfqpoint{3.049386in}{2.128859in}}%
\pgfpathlineto{\pgfqpoint{3.060775in}{2.150574in}}%
\pgfpathlineto{\pgfqpoint{3.072170in}{2.174586in}}%
\pgfpathlineto{\pgfqpoint{3.083573in}{2.199977in}}%
\pgfpathlineto{\pgfqpoint{3.094983in}{2.225828in}}%
\pgfpathlineto{\pgfqpoint{3.088720in}{2.236301in}}%
\pgfpathlineto{\pgfqpoint{3.082461in}{2.246840in}}%
\pgfpathlineto{\pgfqpoint{3.076206in}{2.257515in}}%
\pgfpathlineto{\pgfqpoint{3.069954in}{2.268396in}}%
\pgfpathlineto{\pgfqpoint{3.063706in}{2.279511in}}%
\pgfpathlineto{\pgfqpoint{3.052322in}{2.252201in}}%
\pgfpathlineto{\pgfqpoint{3.040946in}{2.225437in}}%
\pgfpathlineto{\pgfqpoint{3.029577in}{2.200181in}}%
\pgfpathlineto{\pgfqpoint{3.018212in}{2.177392in}}%
\pgfpathlineto{\pgfqpoint{3.006849in}{2.158026in}}%
\pgfpathlineto{\pgfqpoint{3.013071in}{2.148548in}}%
\pgfpathlineto{\pgfqpoint{3.019298in}{2.139047in}}%
\pgfpathlineto{\pgfqpoint{3.025528in}{2.129517in}}%
\pgfpathlineto{\pgfqpoint{3.031762in}{2.119954in}}%
\pgfpathclose%
\pgfusepath{stroke,fill}%
\end{pgfscope}%
\begin{pgfscope}%
\pgfpathrectangle{\pgfqpoint{0.887500in}{0.275000in}}{\pgfqpoint{4.225000in}{4.225000in}}%
\pgfusepath{clip}%
\pgfsetbuttcap%
\pgfsetroundjoin%
\definecolor{currentfill}{rgb}{0.319809,0.770914,0.411152}%
\pgfsetfillcolor{currentfill}%
\pgfsetfillopacity{0.700000}%
\pgfsetlinewidth{0.501875pt}%
\definecolor{currentstroke}{rgb}{1.000000,1.000000,1.000000}%
\pgfsetstrokecolor{currentstroke}%
\pgfsetstrokeopacity{0.500000}%
\pgfsetdash{}{0pt}%
\pgfpathmoveto{\pgfqpoint{3.140833in}{2.807879in}}%
\pgfpathlineto{\pgfqpoint{3.152232in}{2.825794in}}%
\pgfpathlineto{\pgfqpoint{3.163629in}{2.841893in}}%
\pgfpathlineto{\pgfqpoint{3.175022in}{2.856410in}}%
\pgfpathlineto{\pgfqpoint{3.186413in}{2.869582in}}%
\pgfpathlineto{\pgfqpoint{3.197799in}{2.881646in}}%
\pgfpathlineto{\pgfqpoint{3.191532in}{2.897373in}}%
\pgfpathlineto{\pgfqpoint{3.185267in}{2.913048in}}%
\pgfpathlineto{\pgfqpoint{3.179003in}{2.928542in}}%
\pgfpathlineto{\pgfqpoint{3.172740in}{2.943723in}}%
\pgfpathlineto{\pgfqpoint{3.166478in}{2.958462in}}%
\pgfpathlineto{\pgfqpoint{3.155105in}{2.948875in}}%
\pgfpathlineto{\pgfqpoint{3.143728in}{2.938000in}}%
\pgfpathlineto{\pgfqpoint{3.132346in}{2.925346in}}%
\pgfpathlineto{\pgfqpoint{3.120961in}{2.910423in}}%
\pgfpathlineto{\pgfqpoint{3.109573in}{2.892743in}}%
\pgfpathlineto{\pgfqpoint{3.115822in}{2.876221in}}%
\pgfpathlineto{\pgfqpoint{3.122073in}{2.859307in}}%
\pgfpathlineto{\pgfqpoint{3.128325in}{2.842167in}}%
\pgfpathlineto{\pgfqpoint{3.134578in}{2.824969in}}%
\pgfpathclose%
\pgfusepath{stroke,fill}%
\end{pgfscope}%
\begin{pgfscope}%
\pgfpathrectangle{\pgfqpoint{0.887500in}{0.275000in}}{\pgfqpoint{4.225000in}{4.225000in}}%
\pgfusepath{clip}%
\pgfsetbuttcap%
\pgfsetroundjoin%
\definecolor{currentfill}{rgb}{0.171176,0.452530,0.557965}%
\pgfsetfillcolor{currentfill}%
\pgfsetfillopacity{0.700000}%
\pgfsetlinewidth{0.501875pt}%
\definecolor{currentstroke}{rgb}{1.000000,1.000000,1.000000}%
\pgfsetstrokecolor{currentstroke}%
\pgfsetstrokeopacity{0.500000}%
\pgfsetdash{}{0pt}%
\pgfpathmoveto{\pgfqpoint{2.659124in}{2.156861in}}%
\pgfpathlineto{\pgfqpoint{2.670621in}{2.160383in}}%
\pgfpathlineto{\pgfqpoint{2.682113in}{2.163816in}}%
\pgfpathlineto{\pgfqpoint{2.693600in}{2.167133in}}%
\pgfpathlineto{\pgfqpoint{2.705083in}{2.170305in}}%
\pgfpathlineto{\pgfqpoint{2.716561in}{2.173355in}}%
\pgfpathlineto{\pgfqpoint{2.710425in}{2.182503in}}%
\pgfpathlineto{\pgfqpoint{2.704294in}{2.191620in}}%
\pgfpathlineto{\pgfqpoint{2.698166in}{2.200709in}}%
\pgfpathlineto{\pgfqpoint{2.692043in}{2.209769in}}%
\pgfpathlineto{\pgfqpoint{2.685924in}{2.218800in}}%
\pgfpathlineto{\pgfqpoint{2.674456in}{2.215768in}}%
\pgfpathlineto{\pgfqpoint{2.662984in}{2.212614in}}%
\pgfpathlineto{\pgfqpoint{2.651507in}{2.209317in}}%
\pgfpathlineto{\pgfqpoint{2.640026in}{2.205904in}}%
\pgfpathlineto{\pgfqpoint{2.628540in}{2.202403in}}%
\pgfpathlineto{\pgfqpoint{2.634648in}{2.193356in}}%
\pgfpathlineto{\pgfqpoint{2.640761in}{2.184280in}}%
\pgfpathlineto{\pgfqpoint{2.646878in}{2.175172in}}%
\pgfpathlineto{\pgfqpoint{2.652999in}{2.166033in}}%
\pgfpathclose%
\pgfusepath{stroke,fill}%
\end{pgfscope}%
\begin{pgfscope}%
\pgfpathrectangle{\pgfqpoint{0.887500in}{0.275000in}}{\pgfqpoint{4.225000in}{4.225000in}}%
\pgfusepath{clip}%
\pgfsetbuttcap%
\pgfsetroundjoin%
\definecolor{currentfill}{rgb}{0.156270,0.489624,0.557936}%
\pgfsetfillcolor{currentfill}%
\pgfsetfillopacity{0.700000}%
\pgfsetlinewidth{0.501875pt}%
\definecolor{currentstroke}{rgb}{1.000000,1.000000,1.000000}%
\pgfsetstrokecolor{currentstroke}%
\pgfsetstrokeopacity{0.500000}%
\pgfsetdash{}{0pt}%
\pgfpathmoveto{\pgfqpoint{4.231162in}{2.226894in}}%
\pgfpathlineto{\pgfqpoint{4.242253in}{2.229879in}}%
\pgfpathlineto{\pgfqpoint{4.253337in}{2.232789in}}%
\pgfpathlineto{\pgfqpoint{4.264413in}{2.235640in}}%
\pgfpathlineto{\pgfqpoint{4.275483in}{2.238482in}}%
\pgfpathlineto{\pgfqpoint{4.286549in}{2.241364in}}%
\pgfpathlineto{\pgfqpoint{4.280094in}{2.256197in}}%
\pgfpathlineto{\pgfqpoint{4.273644in}{2.271106in}}%
\pgfpathlineto{\pgfqpoint{4.267198in}{2.286061in}}%
\pgfpathlineto{\pgfqpoint{4.260754in}{2.301029in}}%
\pgfpathlineto{\pgfqpoint{4.254313in}{2.315978in}}%
\pgfpathlineto{\pgfqpoint{4.243238in}{2.312783in}}%
\pgfpathlineto{\pgfqpoint{4.232158in}{2.309600in}}%
\pgfpathlineto{\pgfqpoint{4.221073in}{2.306416in}}%
\pgfpathlineto{\pgfqpoint{4.209981in}{2.303216in}}%
\pgfpathlineto{\pgfqpoint{4.198884in}{2.299994in}}%
\pgfpathlineto{\pgfqpoint{4.205334in}{2.285371in}}%
\pgfpathlineto{\pgfqpoint{4.211787in}{2.270738in}}%
\pgfpathlineto{\pgfqpoint{4.218242in}{2.256106in}}%
\pgfpathlineto{\pgfqpoint{4.224700in}{2.241488in}}%
\pgfpathclose%
\pgfusepath{stroke,fill}%
\end{pgfscope}%
\begin{pgfscope}%
\pgfpathrectangle{\pgfqpoint{0.887500in}{0.275000in}}{\pgfqpoint{4.225000in}{4.225000in}}%
\pgfusepath{clip}%
\pgfsetbuttcap%
\pgfsetroundjoin%
\definecolor{currentfill}{rgb}{0.226397,0.728888,0.462789}%
\pgfsetfillcolor{currentfill}%
\pgfsetfillopacity{0.700000}%
\pgfsetlinewidth{0.501875pt}%
\definecolor{currentstroke}{rgb}{1.000000,1.000000,1.000000}%
\pgfsetstrokecolor{currentstroke}%
\pgfsetstrokeopacity{0.500000}%
\pgfsetdash{}{0pt}%
\pgfpathmoveto{\pgfqpoint{3.550905in}{2.741271in}}%
\pgfpathlineto{\pgfqpoint{3.562184in}{2.745000in}}%
\pgfpathlineto{\pgfqpoint{3.573457in}{2.748699in}}%
\pgfpathlineto{\pgfqpoint{3.584724in}{2.752396in}}%
\pgfpathlineto{\pgfqpoint{3.595987in}{2.756095in}}%
\pgfpathlineto{\pgfqpoint{3.607244in}{2.759788in}}%
\pgfpathlineto{\pgfqpoint{3.600890in}{2.773356in}}%
\pgfpathlineto{\pgfqpoint{3.594538in}{2.786898in}}%
\pgfpathlineto{\pgfqpoint{3.588188in}{2.800419in}}%
\pgfpathlineto{\pgfqpoint{3.581841in}{2.813924in}}%
\pgfpathlineto{\pgfqpoint{3.575496in}{2.827416in}}%
\pgfpathlineto{\pgfqpoint{3.564243in}{2.823774in}}%
\pgfpathlineto{\pgfqpoint{3.552985in}{2.820110in}}%
\pgfpathlineto{\pgfqpoint{3.541721in}{2.816426in}}%
\pgfpathlineto{\pgfqpoint{3.530452in}{2.812722in}}%
\pgfpathlineto{\pgfqpoint{3.519176in}{2.808991in}}%
\pgfpathlineto{\pgfqpoint{3.525517in}{2.795488in}}%
\pgfpathlineto{\pgfqpoint{3.531861in}{2.781968in}}%
\pgfpathlineto{\pgfqpoint{3.538206in}{2.768428in}}%
\pgfpathlineto{\pgfqpoint{3.544554in}{2.754864in}}%
\pgfpathclose%
\pgfusepath{stroke,fill}%
\end{pgfscope}%
\begin{pgfscope}%
\pgfpathrectangle{\pgfqpoint{0.887500in}{0.275000in}}{\pgfqpoint{4.225000in}{4.225000in}}%
\pgfusepath{clip}%
\pgfsetbuttcap%
\pgfsetroundjoin%
\definecolor{currentfill}{rgb}{0.175707,0.697900,0.491033}%
\pgfsetfillcolor{currentfill}%
\pgfsetfillopacity{0.700000}%
\pgfsetlinewidth{0.501875pt}%
\definecolor{currentstroke}{rgb}{1.000000,1.000000,1.000000}%
\pgfsetstrokecolor{currentstroke}%
\pgfsetstrokeopacity{0.500000}%
\pgfsetdash{}{0pt}%
\pgfpathmoveto{\pgfqpoint{3.115120in}{2.631240in}}%
\pgfpathlineto{\pgfqpoint{3.126521in}{2.652239in}}%
\pgfpathlineto{\pgfqpoint{3.137925in}{2.672729in}}%
\pgfpathlineto{\pgfqpoint{3.149332in}{2.692584in}}%
\pgfpathlineto{\pgfqpoint{3.160740in}{2.711679in}}%
\pgfpathlineto{\pgfqpoint{3.172150in}{2.729889in}}%
\pgfpathlineto{\pgfqpoint{3.165878in}{2.743939in}}%
\pgfpathlineto{\pgfqpoint{3.159612in}{2.758928in}}%
\pgfpathlineto{\pgfqpoint{3.153349in}{2.774693in}}%
\pgfpathlineto{\pgfqpoint{3.147090in}{2.791065in}}%
\pgfpathlineto{\pgfqpoint{3.140833in}{2.807879in}}%
\pgfpathlineto{\pgfqpoint{3.129433in}{2.787964in}}%
\pgfpathlineto{\pgfqpoint{3.118034in}{2.766247in}}%
\pgfpathlineto{\pgfqpoint{3.106638in}{2.743100in}}%
\pgfpathlineto{\pgfqpoint{3.095247in}{2.718894in}}%
\pgfpathlineto{\pgfqpoint{3.083862in}{2.694002in}}%
\pgfpathlineto{\pgfqpoint{3.090108in}{2.680741in}}%
\pgfpathlineto{\pgfqpoint{3.096356in}{2.667848in}}%
\pgfpathlineto{\pgfqpoint{3.102607in}{2.655310in}}%
\pgfpathlineto{\pgfqpoint{3.108862in}{2.643112in}}%
\pgfpathclose%
\pgfusepath{stroke,fill}%
\end{pgfscope}%
\begin{pgfscope}%
\pgfpathrectangle{\pgfqpoint{0.887500in}{0.275000in}}{\pgfqpoint{4.225000in}{4.225000in}}%
\pgfusepath{clip}%
\pgfsetbuttcap%
\pgfsetroundjoin%
\definecolor{currentfill}{rgb}{0.119738,0.603785,0.541400}%
\pgfsetfillcolor{currentfill}%
\pgfsetfillopacity{0.700000}%
\pgfsetlinewidth{0.501875pt}%
\definecolor{currentstroke}{rgb}{1.000000,1.000000,1.000000}%
\pgfsetstrokecolor{currentstroke}%
\pgfsetstrokeopacity{0.500000}%
\pgfsetdash{}{0pt}%
\pgfpathmoveto{\pgfqpoint{3.935199in}{2.464608in}}%
\pgfpathlineto{\pgfqpoint{3.946379in}{2.467994in}}%
\pgfpathlineto{\pgfqpoint{3.957552in}{2.471384in}}%
\pgfpathlineto{\pgfqpoint{3.968721in}{2.474778in}}%
\pgfpathlineto{\pgfqpoint{3.979884in}{2.478178in}}%
\pgfpathlineto{\pgfqpoint{3.991041in}{2.481581in}}%
\pgfpathlineto{\pgfqpoint{3.984622in}{2.495734in}}%
\pgfpathlineto{\pgfqpoint{3.978204in}{2.509853in}}%
\pgfpathlineto{\pgfqpoint{3.971788in}{2.523941in}}%
\pgfpathlineto{\pgfqpoint{3.965374in}{2.538004in}}%
\pgfpathlineto{\pgfqpoint{3.958963in}{2.552045in}}%
\pgfpathlineto{\pgfqpoint{3.947807in}{2.548652in}}%
\pgfpathlineto{\pgfqpoint{3.936647in}{2.545260in}}%
\pgfpathlineto{\pgfqpoint{3.925481in}{2.541871in}}%
\pgfpathlineto{\pgfqpoint{3.914310in}{2.538484in}}%
\pgfpathlineto{\pgfqpoint{3.903133in}{2.535096in}}%
\pgfpathlineto{\pgfqpoint{3.909542in}{2.521065in}}%
\pgfpathlineto{\pgfqpoint{3.915954in}{2.507005in}}%
\pgfpathlineto{\pgfqpoint{3.922367in}{2.492912in}}%
\pgfpathlineto{\pgfqpoint{3.928783in}{2.478781in}}%
\pgfpathclose%
\pgfusepath{stroke,fill}%
\end{pgfscope}%
\begin{pgfscope}%
\pgfpathrectangle{\pgfqpoint{0.887500in}{0.275000in}}{\pgfqpoint{4.225000in}{4.225000in}}%
\pgfusepath{clip}%
\pgfsetbuttcap%
\pgfsetroundjoin%
\definecolor{currentfill}{rgb}{0.136408,0.541173,0.554483}%
\pgfsetfillcolor{currentfill}%
\pgfsetfillopacity{0.700000}%
\pgfsetlinewidth{0.501875pt}%
\definecolor{currentstroke}{rgb}{1.000000,1.000000,1.000000}%
\pgfsetstrokecolor{currentstroke}%
\pgfsetstrokeopacity{0.500000}%
\pgfsetdash{}{0pt}%
\pgfpathmoveto{\pgfqpoint{1.693460in}{2.345781in}}%
\pgfpathlineto{\pgfqpoint{1.705198in}{2.349184in}}%
\pgfpathlineto{\pgfqpoint{1.716930in}{2.352581in}}%
\pgfpathlineto{\pgfqpoint{1.728656in}{2.355972in}}%
\pgfpathlineto{\pgfqpoint{1.740376in}{2.359358in}}%
\pgfpathlineto{\pgfqpoint{1.752091in}{2.362741in}}%
\pgfpathlineto{\pgfqpoint{1.746282in}{2.371189in}}%
\pgfpathlineto{\pgfqpoint{1.740478in}{2.379613in}}%
\pgfpathlineto{\pgfqpoint{1.734678in}{2.388013in}}%
\pgfpathlineto{\pgfqpoint{1.728883in}{2.396390in}}%
\pgfpathlineto{\pgfqpoint{1.723093in}{2.404743in}}%
\pgfpathlineto{\pgfqpoint{1.711390in}{2.401386in}}%
\pgfpathlineto{\pgfqpoint{1.699682in}{2.398027in}}%
\pgfpathlineto{\pgfqpoint{1.687967in}{2.394662in}}%
\pgfpathlineto{\pgfqpoint{1.676248in}{2.391293in}}%
\pgfpathlineto{\pgfqpoint{1.664522in}{2.387916in}}%
\pgfpathlineto{\pgfqpoint{1.670301in}{2.379540in}}%
\pgfpathlineto{\pgfqpoint{1.676084in}{2.371138in}}%
\pgfpathlineto{\pgfqpoint{1.681872in}{2.362711in}}%
\pgfpathlineto{\pgfqpoint{1.687664in}{2.354258in}}%
\pgfpathclose%
\pgfusepath{stroke,fill}%
\end{pgfscope}%
\begin{pgfscope}%
\pgfpathrectangle{\pgfqpoint{0.887500in}{0.275000in}}{\pgfqpoint{4.225000in}{4.225000in}}%
\pgfusepath{clip}%
\pgfsetbuttcap%
\pgfsetroundjoin%
\definecolor{currentfill}{rgb}{0.159194,0.482237,0.558073}%
\pgfsetfillcolor{currentfill}%
\pgfsetfillopacity{0.700000}%
\pgfsetlinewidth{0.501875pt}%
\definecolor{currentstroke}{rgb}{1.000000,1.000000,1.000000}%
\pgfsetstrokecolor{currentstroke}%
\pgfsetstrokeopacity{0.500000}%
\pgfsetdash{}{0pt}%
\pgfpathmoveto{\pgfqpoint{2.337207in}{2.222601in}}%
\pgfpathlineto{\pgfqpoint{2.348785in}{2.226062in}}%
\pgfpathlineto{\pgfqpoint{2.360357in}{2.229504in}}%
\pgfpathlineto{\pgfqpoint{2.371924in}{2.232926in}}%
\pgfpathlineto{\pgfqpoint{2.383486in}{2.236326in}}%
\pgfpathlineto{\pgfqpoint{2.395042in}{2.239709in}}%
\pgfpathlineto{\pgfqpoint{2.389013in}{2.248543in}}%
\pgfpathlineto{\pgfqpoint{2.382988in}{2.257353in}}%
\pgfpathlineto{\pgfqpoint{2.376967in}{2.266137in}}%
\pgfpathlineto{\pgfqpoint{2.370950in}{2.274899in}}%
\pgfpathlineto{\pgfqpoint{2.364938in}{2.283637in}}%
\pgfpathlineto{\pgfqpoint{2.353393in}{2.280291in}}%
\pgfpathlineto{\pgfqpoint{2.341842in}{2.276930in}}%
\pgfpathlineto{\pgfqpoint{2.330286in}{2.273549in}}%
\pgfpathlineto{\pgfqpoint{2.318725in}{2.270148in}}%
\pgfpathlineto{\pgfqpoint{2.307158in}{2.266730in}}%
\pgfpathlineto{\pgfqpoint{2.313159in}{2.257954in}}%
\pgfpathlineto{\pgfqpoint{2.319165in}{2.249154in}}%
\pgfpathlineto{\pgfqpoint{2.325175in}{2.240329in}}%
\pgfpathlineto{\pgfqpoint{2.331189in}{2.231478in}}%
\pgfpathclose%
\pgfusepath{stroke,fill}%
\end{pgfscope}%
\begin{pgfscope}%
\pgfpathrectangle{\pgfqpoint{0.887500in}{0.275000in}}{\pgfqpoint{4.225000in}{4.225000in}}%
\pgfusepath{clip}%
\pgfsetbuttcap%
\pgfsetroundjoin%
\definecolor{currentfill}{rgb}{0.147607,0.511733,0.557049}%
\pgfsetfillcolor{currentfill}%
\pgfsetfillopacity{0.700000}%
\pgfsetlinewidth{0.501875pt}%
\definecolor{currentstroke}{rgb}{1.000000,1.000000,1.000000}%
\pgfsetstrokecolor{currentstroke}%
\pgfsetstrokeopacity{0.500000}%
\pgfsetdash{}{0pt}%
\pgfpathmoveto{\pgfqpoint{2.015304in}{2.285286in}}%
\pgfpathlineto{\pgfqpoint{2.026963in}{2.288687in}}%
\pgfpathlineto{\pgfqpoint{2.038616in}{2.292078in}}%
\pgfpathlineto{\pgfqpoint{2.050264in}{2.295463in}}%
\pgfpathlineto{\pgfqpoint{2.061905in}{2.298842in}}%
\pgfpathlineto{\pgfqpoint{2.073541in}{2.302218in}}%
\pgfpathlineto{\pgfqpoint{2.067621in}{2.310861in}}%
\pgfpathlineto{\pgfqpoint{2.061705in}{2.319482in}}%
\pgfpathlineto{\pgfqpoint{2.055793in}{2.328082in}}%
\pgfpathlineto{\pgfqpoint{2.049885in}{2.336661in}}%
\pgfpathlineto{\pgfqpoint{2.043982in}{2.345219in}}%
\pgfpathlineto{\pgfqpoint{2.032357in}{2.341877in}}%
\pgfpathlineto{\pgfqpoint{2.020727in}{2.338533in}}%
\pgfpathlineto{\pgfqpoint{2.009090in}{2.335185in}}%
\pgfpathlineto{\pgfqpoint{1.997449in}{2.331829in}}%
\pgfpathlineto{\pgfqpoint{1.985802in}{2.328465in}}%
\pgfpathlineto{\pgfqpoint{1.991693in}{2.319875in}}%
\pgfpathlineto{\pgfqpoint{1.997590in}{2.311263in}}%
\pgfpathlineto{\pgfqpoint{2.003490in}{2.302628in}}%
\pgfpathlineto{\pgfqpoint{2.009395in}{2.293969in}}%
\pgfpathclose%
\pgfusepath{stroke,fill}%
\end{pgfscope}%
\begin{pgfscope}%
\pgfpathrectangle{\pgfqpoint{0.887500in}{0.275000in}}{\pgfqpoint{4.225000in}{4.225000in}}%
\pgfusepath{clip}%
\pgfsetbuttcap%
\pgfsetroundjoin%
\definecolor{currentfill}{rgb}{0.344074,0.780029,0.397381}%
\pgfsetfillcolor{currentfill}%
\pgfsetfillopacity{0.700000}%
\pgfsetlinewidth{0.501875pt}%
\definecolor{currentstroke}{rgb}{1.000000,1.000000,1.000000}%
\pgfsetstrokecolor{currentstroke}%
\pgfsetstrokeopacity{0.500000}%
\pgfsetdash{}{0pt}%
\pgfpathmoveto{\pgfqpoint{3.286116in}{2.862211in}}%
\pgfpathlineto{\pgfqpoint{3.297486in}{2.870966in}}%
\pgfpathlineto{\pgfqpoint{3.308849in}{2.878992in}}%
\pgfpathlineto{\pgfqpoint{3.320205in}{2.886311in}}%
\pgfpathlineto{\pgfqpoint{3.331552in}{2.892944in}}%
\pgfpathlineto{\pgfqpoint{3.342891in}{2.898946in}}%
\pgfpathlineto{\pgfqpoint{3.336588in}{2.912431in}}%
\pgfpathlineto{\pgfqpoint{3.330288in}{2.926013in}}%
\pgfpathlineto{\pgfqpoint{3.323991in}{2.939652in}}%
\pgfpathlineto{\pgfqpoint{3.317697in}{2.953308in}}%
\pgfpathlineto{\pgfqpoint{3.311405in}{2.966944in}}%
\pgfpathlineto{\pgfqpoint{3.300074in}{2.961317in}}%
\pgfpathlineto{\pgfqpoint{3.288734in}{2.955007in}}%
\pgfpathlineto{\pgfqpoint{3.277387in}{2.947982in}}%
\pgfpathlineto{\pgfqpoint{3.266033in}{2.940275in}}%
\pgfpathlineto{\pgfqpoint{3.254672in}{2.931925in}}%
\pgfpathlineto{\pgfqpoint{3.260955in}{2.917807in}}%
\pgfpathlineto{\pgfqpoint{3.267241in}{2.903711in}}%
\pgfpathlineto{\pgfqpoint{3.273529in}{2.889703in}}%
\pgfpathlineto{\pgfqpoint{3.279821in}{2.875848in}}%
\pgfpathclose%
\pgfusepath{stroke,fill}%
\end{pgfscope}%
\begin{pgfscope}%
\pgfpathrectangle{\pgfqpoint{0.887500in}{0.275000in}}{\pgfqpoint{4.225000in}{4.225000in}}%
\pgfusepath{clip}%
\pgfsetbuttcap%
\pgfsetroundjoin%
\definecolor{currentfill}{rgb}{0.203063,0.379716,0.553925}%
\pgfsetfillcolor{currentfill}%
\pgfsetfillopacity{0.700000}%
\pgfsetlinewidth{0.501875pt}%
\definecolor{currentstroke}{rgb}{1.000000,1.000000,1.000000}%
\pgfsetstrokecolor{currentstroke}%
\pgfsetstrokeopacity{0.500000}%
\pgfsetdash{}{0pt}%
\pgfpathmoveto{\pgfqpoint{4.527335in}{1.995612in}}%
\pgfpathlineto{\pgfqpoint{4.538358in}{1.998854in}}%
\pgfpathlineto{\pgfqpoint{4.549377in}{2.002103in}}%
\pgfpathlineto{\pgfqpoint{4.560390in}{2.005358in}}%
\pgfpathlineto{\pgfqpoint{4.571398in}{2.008621in}}%
\pgfpathlineto{\pgfqpoint{4.582401in}{2.011890in}}%
\pgfpathlineto{\pgfqpoint{4.575881in}{2.026417in}}%
\pgfpathlineto{\pgfqpoint{4.569362in}{2.040927in}}%
\pgfpathlineto{\pgfqpoint{4.562846in}{2.055420in}}%
\pgfpathlineto{\pgfqpoint{4.556331in}{2.069897in}}%
\pgfpathlineto{\pgfqpoint{4.549819in}{2.084357in}}%
\pgfpathlineto{\pgfqpoint{4.538815in}{2.081024in}}%
\pgfpathlineto{\pgfqpoint{4.527805in}{2.077688in}}%
\pgfpathlineto{\pgfqpoint{4.516790in}{2.074348in}}%
\pgfpathlineto{\pgfqpoint{4.505769in}{2.071002in}}%
\pgfpathlineto{\pgfqpoint{4.494742in}{2.067651in}}%
\pgfpathlineto{\pgfqpoint{4.501260in}{2.053375in}}%
\pgfpathlineto{\pgfqpoint{4.507779in}{2.039028in}}%
\pgfpathlineto{\pgfqpoint{4.514297in}{2.024613in}}%
\pgfpathlineto{\pgfqpoint{4.520815in}{2.010140in}}%
\pgfpathclose%
\pgfusepath{stroke,fill}%
\end{pgfscope}%
\begin{pgfscope}%
\pgfpathrectangle{\pgfqpoint{0.887500in}{0.275000in}}{\pgfqpoint{4.225000in}{4.225000in}}%
\pgfusepath{clip}%
\pgfsetbuttcap%
\pgfsetroundjoin%
\definecolor{currentfill}{rgb}{0.137770,0.537492,0.554906}%
\pgfsetfillcolor{currentfill}%
\pgfsetfillopacity{0.700000}%
\pgfsetlinewidth{0.501875pt}%
\definecolor{currentstroke}{rgb}{1.000000,1.000000,1.000000}%
\pgfsetstrokecolor{currentstroke}%
\pgfsetstrokeopacity{0.500000}%
\pgfsetdash{}{0pt}%
\pgfpathmoveto{\pgfqpoint{3.063706in}{2.279511in}}%
\pgfpathlineto{\pgfqpoint{3.075098in}{2.306400in}}%
\pgfpathlineto{\pgfqpoint{3.086497in}{2.332065in}}%
\pgfpathlineto{\pgfqpoint{3.097902in}{2.356524in}}%
\pgfpathlineto{\pgfqpoint{3.109312in}{2.380047in}}%
\pgfpathlineto{\pgfqpoint{3.120728in}{2.402907in}}%
\pgfpathlineto{\pgfqpoint{3.114460in}{2.415508in}}%
\pgfpathlineto{\pgfqpoint{3.108196in}{2.428123in}}%
\pgfpathlineto{\pgfqpoint{3.101935in}{2.440684in}}%
\pgfpathlineto{\pgfqpoint{3.095676in}{2.453121in}}%
\pgfpathlineto{\pgfqpoint{3.089420in}{2.465366in}}%
\pgfpathlineto{\pgfqpoint{3.078026in}{2.442072in}}%
\pgfpathlineto{\pgfqpoint{3.066637in}{2.417753in}}%
\pgfpathlineto{\pgfqpoint{3.055254in}{2.392142in}}%
\pgfpathlineto{\pgfqpoint{3.043879in}{2.364974in}}%
\pgfpathlineto{\pgfqpoint{3.032514in}{2.336263in}}%
\pgfpathlineto{\pgfqpoint{3.038746in}{2.324996in}}%
\pgfpathlineto{\pgfqpoint{3.044981in}{2.313615in}}%
\pgfpathlineto{\pgfqpoint{3.051219in}{2.302193in}}%
\pgfpathlineto{\pgfqpoint{3.057461in}{2.290801in}}%
\pgfpathclose%
\pgfusepath{stroke,fill}%
\end{pgfscope}%
\begin{pgfscope}%
\pgfpathrectangle{\pgfqpoint{0.887500in}{0.275000in}}{\pgfqpoint{4.225000in}{4.225000in}}%
\pgfusepath{clip}%
\pgfsetbuttcap%
\pgfsetroundjoin%
\definecolor{currentfill}{rgb}{0.266941,0.748751,0.440573}%
\pgfsetfillcolor{currentfill}%
\pgfsetfillopacity{0.700000}%
\pgfsetlinewidth{0.501875pt}%
\definecolor{currentstroke}{rgb}{1.000000,1.000000,1.000000}%
\pgfsetstrokecolor{currentstroke}%
\pgfsetstrokeopacity{0.500000}%
\pgfsetdash{}{0pt}%
\pgfpathmoveto{\pgfqpoint{3.462715in}{2.789692in}}%
\pgfpathlineto{\pgfqpoint{3.474019in}{2.793663in}}%
\pgfpathlineto{\pgfqpoint{3.485317in}{2.797570in}}%
\pgfpathlineto{\pgfqpoint{3.496609in}{2.801423in}}%
\pgfpathlineto{\pgfqpoint{3.507896in}{2.805228in}}%
\pgfpathlineto{\pgfqpoint{3.519176in}{2.808991in}}%
\pgfpathlineto{\pgfqpoint{3.512838in}{2.822474in}}%
\pgfpathlineto{\pgfqpoint{3.506502in}{2.835934in}}%
\pgfpathlineto{\pgfqpoint{3.500168in}{2.849365in}}%
\pgfpathlineto{\pgfqpoint{3.493837in}{2.862765in}}%
\pgfpathlineto{\pgfqpoint{3.487508in}{2.876130in}}%
\pgfpathlineto{\pgfqpoint{3.476230in}{2.872313in}}%
\pgfpathlineto{\pgfqpoint{3.464947in}{2.868446in}}%
\pgfpathlineto{\pgfqpoint{3.453658in}{2.864526in}}%
\pgfpathlineto{\pgfqpoint{3.442364in}{2.860550in}}%
\pgfpathlineto{\pgfqpoint{3.431063in}{2.856515in}}%
\pgfpathlineto{\pgfqpoint{3.437389in}{2.843253in}}%
\pgfpathlineto{\pgfqpoint{3.443717in}{2.829949in}}%
\pgfpathlineto{\pgfqpoint{3.450048in}{2.816595in}}%
\pgfpathlineto{\pgfqpoint{3.456380in}{2.803179in}}%
\pgfpathclose%
\pgfusepath{stroke,fill}%
\end{pgfscope}%
\begin{pgfscope}%
\pgfpathrectangle{\pgfqpoint{0.887500in}{0.275000in}}{\pgfqpoint{4.225000in}{4.225000in}}%
\pgfusepath{clip}%
\pgfsetbuttcap%
\pgfsetroundjoin%
\definecolor{currentfill}{rgb}{0.120081,0.622161,0.534946}%
\pgfsetfillcolor{currentfill}%
\pgfsetfillopacity{0.700000}%
\pgfsetlinewidth{0.501875pt}%
\definecolor{currentstroke}{rgb}{1.000000,1.000000,1.000000}%
\pgfsetstrokecolor{currentstroke}%
\pgfsetstrokeopacity{0.500000}%
\pgfsetdash{}{0pt}%
\pgfpathmoveto{\pgfqpoint{3.089420in}{2.465366in}}%
\pgfpathlineto{\pgfqpoint{3.100820in}{2.487901in}}%
\pgfpathlineto{\pgfqpoint{3.112224in}{2.509947in}}%
\pgfpathlineto{\pgfqpoint{3.123633in}{2.531773in}}%
\pgfpathlineto{\pgfqpoint{3.135047in}{2.553555in}}%
\pgfpathlineto{\pgfqpoint{3.146465in}{2.575163in}}%
\pgfpathlineto{\pgfqpoint{3.140189in}{2.586231in}}%
\pgfpathlineto{\pgfqpoint{3.133917in}{2.597272in}}%
\pgfpathlineto{\pgfqpoint{3.127648in}{2.608389in}}%
\pgfpathlineto{\pgfqpoint{3.121382in}{2.619680in}}%
\pgfpathlineto{\pgfqpoint{3.115120in}{2.631240in}}%
\pgfpathlineto{\pgfqpoint{3.103724in}{2.609855in}}%
\pgfpathlineto{\pgfqpoint{3.092332in}{2.588209in}}%
\pgfpathlineto{\pgfqpoint{3.080944in}{2.566373in}}%
\pgfpathlineto{\pgfqpoint{3.069562in}{2.544159in}}%
\pgfpathlineto{\pgfqpoint{3.058185in}{2.521301in}}%
\pgfpathlineto{\pgfqpoint{3.064425in}{2.511038in}}%
\pgfpathlineto{\pgfqpoint{3.070670in}{2.500254in}}%
\pgfpathlineto{\pgfqpoint{3.076917in}{2.489001in}}%
\pgfpathlineto{\pgfqpoint{3.083167in}{2.477349in}}%
\pgfpathclose%
\pgfusepath{stroke,fill}%
\end{pgfscope}%
\begin{pgfscope}%
\pgfpathrectangle{\pgfqpoint{0.887500in}{0.275000in}}{\pgfqpoint{4.225000in}{4.225000in}}%
\pgfusepath{clip}%
\pgfsetbuttcap%
\pgfsetroundjoin%
\definecolor{currentfill}{rgb}{0.311925,0.767822,0.415586}%
\pgfsetfillcolor{currentfill}%
\pgfsetfillopacity{0.700000}%
\pgfsetlinewidth{0.501875pt}%
\definecolor{currentstroke}{rgb}{1.000000,1.000000,1.000000}%
\pgfsetstrokecolor{currentstroke}%
\pgfsetstrokeopacity{0.500000}%
\pgfsetdash{}{0pt}%
\pgfpathmoveto{\pgfqpoint{3.374460in}{2.833241in}}%
\pgfpathlineto{\pgfqpoint{3.385796in}{2.838591in}}%
\pgfpathlineto{\pgfqpoint{3.397124in}{2.843508in}}%
\pgfpathlineto{\pgfqpoint{3.408444in}{2.848078in}}%
\pgfpathlineto{\pgfqpoint{3.419757in}{2.852385in}}%
\pgfpathlineto{\pgfqpoint{3.431063in}{2.856515in}}%
\pgfpathlineto{\pgfqpoint{3.424740in}{2.869747in}}%
\pgfpathlineto{\pgfqpoint{3.418419in}{2.882958in}}%
\pgfpathlineto{\pgfqpoint{3.412100in}{2.896159in}}%
\pgfpathlineto{\pgfqpoint{3.405785in}{2.909360in}}%
\pgfpathlineto{\pgfqpoint{3.399472in}{2.922569in}}%
\pgfpathlineto{\pgfqpoint{3.388170in}{2.918400in}}%
\pgfpathlineto{\pgfqpoint{3.376861in}{2.914042in}}%
\pgfpathlineto{\pgfqpoint{3.365545in}{2.909406in}}%
\pgfpathlineto{\pgfqpoint{3.354222in}{2.904404in}}%
\pgfpathlineto{\pgfqpoint{3.342891in}{2.898946in}}%
\pgfpathlineto{\pgfqpoint{3.349198in}{2.885597in}}%
\pgfpathlineto{\pgfqpoint{3.355508in}{2.872394in}}%
\pgfpathlineto{\pgfqpoint{3.361822in}{2.859298in}}%
\pgfpathlineto{\pgfqpoint{3.368140in}{2.846262in}}%
\pgfpathclose%
\pgfusepath{stroke,fill}%
\end{pgfscope}%
\begin{pgfscope}%
\pgfpathrectangle{\pgfqpoint{0.887500in}{0.275000in}}{\pgfqpoint{4.225000in}{4.225000in}}%
\pgfusepath{clip}%
\pgfsetbuttcap%
\pgfsetroundjoin%
\definecolor{currentfill}{rgb}{0.175841,0.441290,0.557685}%
\pgfsetfillcolor{currentfill}%
\pgfsetfillopacity{0.700000}%
\pgfsetlinewidth{0.501875pt}%
\definecolor{currentstroke}{rgb}{1.000000,1.000000,1.000000}%
\pgfsetstrokecolor{currentstroke}%
\pgfsetstrokeopacity{0.500000}%
\pgfsetdash{}{0pt}%
\pgfpathmoveto{\pgfqpoint{2.747301in}{2.127154in}}%
\pgfpathlineto{\pgfqpoint{2.758783in}{2.130213in}}%
\pgfpathlineto{\pgfqpoint{2.770259in}{2.133322in}}%
\pgfpathlineto{\pgfqpoint{2.781728in}{2.136566in}}%
\pgfpathlineto{\pgfqpoint{2.793190in}{2.140030in}}%
\pgfpathlineto{\pgfqpoint{2.804645in}{2.143801in}}%
\pgfpathlineto{\pgfqpoint{2.798478in}{2.153074in}}%
\pgfpathlineto{\pgfqpoint{2.792316in}{2.162314in}}%
\pgfpathlineto{\pgfqpoint{2.786158in}{2.171524in}}%
\pgfpathlineto{\pgfqpoint{2.780003in}{2.180704in}}%
\pgfpathlineto{\pgfqpoint{2.773853in}{2.189855in}}%
\pgfpathlineto{\pgfqpoint{2.762410in}{2.186088in}}%
\pgfpathlineto{\pgfqpoint{2.750958in}{2.182644in}}%
\pgfpathlineto{\pgfqpoint{2.739499in}{2.179434in}}%
\pgfpathlineto{\pgfqpoint{2.728033in}{2.176368in}}%
\pgfpathlineto{\pgfqpoint{2.716561in}{2.173355in}}%
\pgfpathlineto{\pgfqpoint{2.722701in}{2.164178in}}%
\pgfpathlineto{\pgfqpoint{2.728844in}{2.154971in}}%
\pgfpathlineto{\pgfqpoint{2.734992in}{2.145732in}}%
\pgfpathlineto{\pgfqpoint{2.741144in}{2.136460in}}%
\pgfpathclose%
\pgfusepath{stroke,fill}%
\end{pgfscope}%
\begin{pgfscope}%
\pgfpathrectangle{\pgfqpoint{0.887500in}{0.275000in}}{\pgfqpoint{4.225000in}{4.225000in}}%
\pgfusepath{clip}%
\pgfsetbuttcap%
\pgfsetroundjoin%
\definecolor{currentfill}{rgb}{0.121380,0.629492,0.531973}%
\pgfsetfillcolor{currentfill}%
\pgfsetfillopacity{0.700000}%
\pgfsetlinewidth{0.501875pt}%
\definecolor{currentstroke}{rgb}{1.000000,1.000000,1.000000}%
\pgfsetstrokecolor{currentstroke}%
\pgfsetstrokeopacity{0.500000}%
\pgfsetdash{}{0pt}%
\pgfpathmoveto{\pgfqpoint{3.847164in}{2.518041in}}%
\pgfpathlineto{\pgfqpoint{3.858369in}{2.521477in}}%
\pgfpathlineto{\pgfqpoint{3.869569in}{2.524897in}}%
\pgfpathlineto{\pgfqpoint{3.880762in}{2.528305in}}%
\pgfpathlineto{\pgfqpoint{3.891950in}{2.531704in}}%
\pgfpathlineto{\pgfqpoint{3.903133in}{2.535096in}}%
\pgfpathlineto{\pgfqpoint{3.896725in}{2.549103in}}%
\pgfpathlineto{\pgfqpoint{3.890320in}{2.563089in}}%
\pgfpathlineto{\pgfqpoint{3.883916in}{2.577061in}}%
\pgfpathlineto{\pgfqpoint{3.877516in}{2.591020in}}%
\pgfpathlineto{\pgfqpoint{3.871117in}{2.604966in}}%
\pgfpathlineto{\pgfqpoint{3.859937in}{2.601559in}}%
\pgfpathlineto{\pgfqpoint{3.848751in}{2.598148in}}%
\pgfpathlineto{\pgfqpoint{3.837560in}{2.594729in}}%
\pgfpathlineto{\pgfqpoint{3.826363in}{2.591298in}}%
\pgfpathlineto{\pgfqpoint{3.815160in}{2.587851in}}%
\pgfpathlineto{\pgfqpoint{3.821556in}{2.573894in}}%
\pgfpathlineto{\pgfqpoint{3.827954in}{2.559936in}}%
\pgfpathlineto{\pgfqpoint{3.834355in}{2.545979in}}%
\pgfpathlineto{\pgfqpoint{3.840758in}{2.532017in}}%
\pgfpathclose%
\pgfusepath{stroke,fill}%
\end{pgfscope}%
\begin{pgfscope}%
\pgfpathrectangle{\pgfqpoint{0.887500in}{0.275000in}}{\pgfqpoint{4.225000in}{4.225000in}}%
\pgfusepath{clip}%
\pgfsetbuttcap%
\pgfsetroundjoin%
\definecolor{currentfill}{rgb}{0.144759,0.519093,0.556572}%
\pgfsetfillcolor{currentfill}%
\pgfsetfillopacity{0.700000}%
\pgfsetlinewidth{0.501875pt}%
\definecolor{currentstroke}{rgb}{1.000000,1.000000,1.000000}%
\pgfsetstrokecolor{currentstroke}%
\pgfsetstrokeopacity{0.500000}%
\pgfsetdash{}{0pt}%
\pgfpathmoveto{\pgfqpoint{4.143303in}{2.283528in}}%
\pgfpathlineto{\pgfqpoint{4.154431in}{2.286869in}}%
\pgfpathlineto{\pgfqpoint{4.165554in}{2.290186in}}%
\pgfpathlineto{\pgfqpoint{4.176670in}{2.293479in}}%
\pgfpathlineto{\pgfqpoint{4.187780in}{2.296748in}}%
\pgfpathlineto{\pgfqpoint{4.198884in}{2.299994in}}%
\pgfpathlineto{\pgfqpoint{4.192435in}{2.314596in}}%
\pgfpathlineto{\pgfqpoint{4.185988in}{2.329165in}}%
\pgfpathlineto{\pgfqpoint{4.179542in}{2.343694in}}%
\pgfpathlineto{\pgfqpoint{4.173097in}{2.358182in}}%
\pgfpathlineto{\pgfqpoint{4.166654in}{2.372631in}}%
\pgfpathlineto{\pgfqpoint{4.155548in}{2.369252in}}%
\pgfpathlineto{\pgfqpoint{4.144436in}{2.365868in}}%
\pgfpathlineto{\pgfqpoint{4.133319in}{2.362473in}}%
\pgfpathlineto{\pgfqpoint{4.122195in}{2.359063in}}%
\pgfpathlineto{\pgfqpoint{4.111066in}{2.355632in}}%
\pgfpathlineto{\pgfqpoint{4.117508in}{2.341222in}}%
\pgfpathlineto{\pgfqpoint{4.123953in}{2.326804in}}%
\pgfpathlineto{\pgfqpoint{4.130401in}{2.312382in}}%
\pgfpathlineto{\pgfqpoint{4.136850in}{2.297957in}}%
\pgfpathclose%
\pgfusepath{stroke,fill}%
\end{pgfscope}%
\begin{pgfscope}%
\pgfpathrectangle{\pgfqpoint{0.887500in}{0.275000in}}{\pgfqpoint{4.225000in}{4.225000in}}%
\pgfusepath{clip}%
\pgfsetbuttcap%
\pgfsetroundjoin%
\definecolor{currentfill}{rgb}{0.188923,0.410910,0.556326}%
\pgfsetfillcolor{currentfill}%
\pgfsetfillopacity{0.700000}%
\pgfsetlinewidth{0.501875pt}%
\definecolor{currentstroke}{rgb}{1.000000,1.000000,1.000000}%
\pgfsetstrokecolor{currentstroke}%
\pgfsetstrokeopacity{0.500000}%
\pgfsetdash{}{0pt}%
\pgfpathmoveto{\pgfqpoint{4.439520in}{2.050703in}}%
\pgfpathlineto{\pgfqpoint{4.450577in}{2.054130in}}%
\pgfpathlineto{\pgfqpoint{4.461627in}{2.057535in}}%
\pgfpathlineto{\pgfqpoint{4.472672in}{2.060921in}}%
\pgfpathlineto{\pgfqpoint{4.483710in}{2.064291in}}%
\pgfpathlineto{\pgfqpoint{4.494742in}{2.067651in}}%
\pgfpathlineto{\pgfqpoint{4.488224in}{2.081848in}}%
\pgfpathlineto{\pgfqpoint{4.481705in}{2.095964in}}%
\pgfpathlineto{\pgfqpoint{4.475187in}{2.110011in}}%
\pgfpathlineto{\pgfqpoint{4.468669in}{2.124004in}}%
\pgfpathlineto{\pgfqpoint{4.462153in}{2.137959in}}%
\pgfpathlineto{\pgfqpoint{4.451105in}{2.134107in}}%
\pgfpathlineto{\pgfqpoint{4.440051in}{2.130228in}}%
\pgfpathlineto{\pgfqpoint{4.428991in}{2.126351in}}%
\pgfpathlineto{\pgfqpoint{4.417928in}{2.122504in}}%
\pgfpathlineto{\pgfqpoint{4.406861in}{2.118717in}}%
\pgfpathlineto{\pgfqpoint{4.413394in}{2.105299in}}%
\pgfpathlineto{\pgfqpoint{4.419927in}{2.091826in}}%
\pgfpathlineto{\pgfqpoint{4.426460in}{2.078261in}}%
\pgfpathlineto{\pgfqpoint{4.432992in}{2.064564in}}%
\pgfpathclose%
\pgfusepath{stroke,fill}%
\end{pgfscope}%
\begin{pgfscope}%
\pgfpathrectangle{\pgfqpoint{0.887500in}{0.275000in}}{\pgfqpoint{4.225000in}{4.225000in}}%
\pgfusepath{clip}%
\pgfsetbuttcap%
\pgfsetroundjoin%
\definecolor{currentfill}{rgb}{0.246811,0.283237,0.535941}%
\pgfsetfillcolor{currentfill}%
\pgfsetfillopacity{0.700000}%
\pgfsetlinewidth{0.501875pt}%
\definecolor{currentstroke}{rgb}{1.000000,1.000000,1.000000}%
\pgfsetstrokecolor{currentstroke}%
\pgfsetstrokeopacity{0.500000}%
\pgfsetdash{}{0pt}%
\pgfpathmoveto{\pgfqpoint{4.735370in}{1.808268in}}%
\pgfpathlineto{\pgfqpoint{4.746344in}{1.811575in}}%
\pgfpathlineto{\pgfqpoint{4.757312in}{1.814893in}}%
\pgfpathlineto{\pgfqpoint{4.768276in}{1.818223in}}%
\pgfpathlineto{\pgfqpoint{4.779234in}{1.821566in}}%
\pgfpathlineto{\pgfqpoint{4.772691in}{1.836494in}}%
\pgfpathlineto{\pgfqpoint{4.766149in}{1.851372in}}%
\pgfpathlineto{\pgfqpoint{4.759607in}{1.866201in}}%
\pgfpathlineto{\pgfqpoint{4.753065in}{1.880986in}}%
\pgfpathlineto{\pgfqpoint{4.746525in}{1.895731in}}%
\pgfpathlineto{\pgfqpoint{4.735566in}{1.892370in}}%
\pgfpathlineto{\pgfqpoint{4.724602in}{1.889019in}}%
\pgfpathlineto{\pgfqpoint{4.713633in}{1.885677in}}%
\pgfpathlineto{\pgfqpoint{4.702659in}{1.882339in}}%
\pgfpathlineto{\pgfqpoint{4.709199in}{1.867602in}}%
\pgfpathlineto{\pgfqpoint{4.715741in}{1.852829in}}%
\pgfpathlineto{\pgfqpoint{4.722283in}{1.838018in}}%
\pgfpathlineto{\pgfqpoint{4.728827in}{1.823165in}}%
\pgfpathclose%
\pgfusepath{stroke,fill}%
\end{pgfscope}%
\begin{pgfscope}%
\pgfpathrectangle{\pgfqpoint{0.887500in}{0.275000in}}{\pgfqpoint{4.225000in}{4.225000in}}%
\pgfusepath{clip}%
\pgfsetbuttcap%
\pgfsetroundjoin%
\definecolor{currentfill}{rgb}{0.163625,0.471133,0.558148}%
\pgfsetfillcolor{currentfill}%
\pgfsetfillopacity{0.700000}%
\pgfsetlinewidth{0.501875pt}%
\definecolor{currentstroke}{rgb}{1.000000,1.000000,1.000000}%
\pgfsetstrokecolor{currentstroke}%
\pgfsetstrokeopacity{0.500000}%
\pgfsetdash{}{0pt}%
\pgfpathmoveto{\pgfqpoint{2.425252in}{2.195143in}}%
\pgfpathlineto{\pgfqpoint{2.436814in}{2.198556in}}%
\pgfpathlineto{\pgfqpoint{2.448369in}{2.201960in}}%
\pgfpathlineto{\pgfqpoint{2.459919in}{2.205360in}}%
\pgfpathlineto{\pgfqpoint{2.471464in}{2.208761in}}%
\pgfpathlineto{\pgfqpoint{2.483002in}{2.212168in}}%
\pgfpathlineto{\pgfqpoint{2.476941in}{2.221094in}}%
\pgfpathlineto{\pgfqpoint{2.470884in}{2.229993in}}%
\pgfpathlineto{\pgfqpoint{2.464831in}{2.238866in}}%
\pgfpathlineto{\pgfqpoint{2.458782in}{2.247712in}}%
\pgfpathlineto{\pgfqpoint{2.452738in}{2.256533in}}%
\pgfpathlineto{\pgfqpoint{2.441210in}{2.253165in}}%
\pgfpathlineto{\pgfqpoint{2.429677in}{2.249803in}}%
\pgfpathlineto{\pgfqpoint{2.418138in}{2.246443in}}%
\pgfpathlineto{\pgfqpoint{2.406593in}{2.243080in}}%
\pgfpathlineto{\pgfqpoint{2.395042in}{2.239709in}}%
\pgfpathlineto{\pgfqpoint{2.401076in}{2.230850in}}%
\pgfpathlineto{\pgfqpoint{2.407113in}{2.221964in}}%
\pgfpathlineto{\pgfqpoint{2.413155in}{2.213051in}}%
\pgfpathlineto{\pgfqpoint{2.419202in}{2.204111in}}%
\pgfpathclose%
\pgfusepath{stroke,fill}%
\end{pgfscope}%
\begin{pgfscope}%
\pgfpathrectangle{\pgfqpoint{0.887500in}{0.275000in}}{\pgfqpoint{4.225000in}{4.225000in}}%
\pgfusepath{clip}%
\pgfsetbuttcap%
\pgfsetroundjoin%
\definecolor{currentfill}{rgb}{0.140536,0.530132,0.555659}%
\pgfsetfillcolor{currentfill}%
\pgfsetfillopacity{0.700000}%
\pgfsetlinewidth{0.501875pt}%
\definecolor{currentstroke}{rgb}{1.000000,1.000000,1.000000}%
\pgfsetstrokecolor{currentstroke}%
\pgfsetstrokeopacity{0.500000}%
\pgfsetdash{}{0pt}%
\pgfpathmoveto{\pgfqpoint{1.781201in}{2.320151in}}%
\pgfpathlineto{\pgfqpoint{1.792922in}{2.323567in}}%
\pgfpathlineto{\pgfqpoint{1.804637in}{2.326981in}}%
\pgfpathlineto{\pgfqpoint{1.816346in}{2.330396in}}%
\pgfpathlineto{\pgfqpoint{1.828049in}{2.333811in}}%
\pgfpathlineto{\pgfqpoint{1.839746in}{2.337228in}}%
\pgfpathlineto{\pgfqpoint{1.833903in}{2.345755in}}%
\pgfpathlineto{\pgfqpoint{1.828065in}{2.354261in}}%
\pgfpathlineto{\pgfqpoint{1.822232in}{2.362745in}}%
\pgfpathlineto{\pgfqpoint{1.816402in}{2.371207in}}%
\pgfpathlineto{\pgfqpoint{1.810577in}{2.379648in}}%
\pgfpathlineto{\pgfqpoint{1.798892in}{2.376265in}}%
\pgfpathlineto{\pgfqpoint{1.787200in}{2.372883in}}%
\pgfpathlineto{\pgfqpoint{1.775503in}{2.369503in}}%
\pgfpathlineto{\pgfqpoint{1.763800in}{2.366123in}}%
\pgfpathlineto{\pgfqpoint{1.752091in}{2.362741in}}%
\pgfpathlineto{\pgfqpoint{1.757904in}{2.354270in}}%
\pgfpathlineto{\pgfqpoint{1.763722in}{2.345776in}}%
\pgfpathlineto{\pgfqpoint{1.769544in}{2.337257in}}%
\pgfpathlineto{\pgfqpoint{1.775370in}{2.328716in}}%
\pgfpathclose%
\pgfusepath{stroke,fill}%
\end{pgfscope}%
\begin{pgfscope}%
\pgfpathrectangle{\pgfqpoint{0.887500in}{0.275000in}}{\pgfqpoint{4.225000in}{4.225000in}}%
\pgfusepath{clip}%
\pgfsetbuttcap%
\pgfsetroundjoin%
\definecolor{currentfill}{rgb}{0.151918,0.500685,0.557587}%
\pgfsetfillcolor{currentfill}%
\pgfsetfillopacity{0.700000}%
\pgfsetlinewidth{0.501875pt}%
\definecolor{currentstroke}{rgb}{1.000000,1.000000,1.000000}%
\pgfsetstrokecolor{currentstroke}%
\pgfsetstrokeopacity{0.500000}%
\pgfsetdash{}{0pt}%
\pgfpathmoveto{\pgfqpoint{2.103209in}{2.258671in}}%
\pgfpathlineto{\pgfqpoint{2.114850in}{2.262091in}}%
\pgfpathlineto{\pgfqpoint{2.126486in}{2.265512in}}%
\pgfpathlineto{\pgfqpoint{2.138115in}{2.268939in}}%
\pgfpathlineto{\pgfqpoint{2.149739in}{2.272374in}}%
\pgfpathlineto{\pgfqpoint{2.161356in}{2.275820in}}%
\pgfpathlineto{\pgfqpoint{2.155403in}{2.284533in}}%
\pgfpathlineto{\pgfqpoint{2.149454in}{2.293223in}}%
\pgfpathlineto{\pgfqpoint{2.143509in}{2.301891in}}%
\pgfpathlineto{\pgfqpoint{2.137569in}{2.310538in}}%
\pgfpathlineto{\pgfqpoint{2.131633in}{2.319164in}}%
\pgfpathlineto{\pgfqpoint{2.120026in}{2.315757in}}%
\pgfpathlineto{\pgfqpoint{2.108414in}{2.312362in}}%
\pgfpathlineto{\pgfqpoint{2.096796in}{2.308976in}}%
\pgfpathlineto{\pgfqpoint{2.085171in}{2.305595in}}%
\pgfpathlineto{\pgfqpoint{2.073541in}{2.302218in}}%
\pgfpathlineto{\pgfqpoint{2.079466in}{2.293554in}}%
\pgfpathlineto{\pgfqpoint{2.085395in}{2.284868in}}%
\pgfpathlineto{\pgfqpoint{2.091329in}{2.276159in}}%
\pgfpathlineto{\pgfqpoint{2.097267in}{2.267427in}}%
\pgfpathclose%
\pgfusepath{stroke,fill}%
\end{pgfscope}%
\begin{pgfscope}%
\pgfpathrectangle{\pgfqpoint{0.887500in}{0.275000in}}{\pgfqpoint{4.225000in}{4.225000in}}%
\pgfusepath{clip}%
\pgfsetbuttcap%
\pgfsetroundjoin%
\definecolor{currentfill}{rgb}{0.132268,0.655014,0.519661}%
\pgfsetfillcolor{currentfill}%
\pgfsetfillopacity{0.700000}%
\pgfsetlinewidth{0.501875pt}%
\definecolor{currentstroke}{rgb}{1.000000,1.000000,1.000000}%
\pgfsetstrokecolor{currentstroke}%
\pgfsetstrokeopacity{0.500000}%
\pgfsetdash{}{0pt}%
\pgfpathmoveto{\pgfqpoint{3.759055in}{2.570122in}}%
\pgfpathlineto{\pgfqpoint{3.770289in}{2.573761in}}%
\pgfpathlineto{\pgfqpoint{3.781516in}{2.577345in}}%
\pgfpathlineto{\pgfqpoint{3.792737in}{2.580882in}}%
\pgfpathlineto{\pgfqpoint{3.803951in}{2.584382in}}%
\pgfpathlineto{\pgfqpoint{3.815160in}{2.587851in}}%
\pgfpathlineto{\pgfqpoint{3.808767in}{2.601803in}}%
\pgfpathlineto{\pgfqpoint{3.802376in}{2.615748in}}%
\pgfpathlineto{\pgfqpoint{3.795987in}{2.629681in}}%
\pgfpathlineto{\pgfqpoint{3.789601in}{2.643600in}}%
\pgfpathlineto{\pgfqpoint{3.783217in}{2.657502in}}%
\pgfpathlineto{\pgfqpoint{3.772013in}{2.654107in}}%
\pgfpathlineto{\pgfqpoint{3.760802in}{2.650691in}}%
\pgfpathlineto{\pgfqpoint{3.749586in}{2.647245in}}%
\pgfpathlineto{\pgfqpoint{3.738363in}{2.643755in}}%
\pgfpathlineto{\pgfqpoint{3.727134in}{2.640213in}}%
\pgfpathlineto{\pgfqpoint{3.733514in}{2.626217in}}%
\pgfpathlineto{\pgfqpoint{3.739895in}{2.612201in}}%
\pgfpathlineto{\pgfqpoint{3.746280in}{2.598175in}}%
\pgfpathlineto{\pgfqpoint{3.752666in}{2.584145in}}%
\pgfpathclose%
\pgfusepath{stroke,fill}%
\end{pgfscope}%
\begin{pgfscope}%
\pgfpathrectangle{\pgfqpoint{0.887500in}{0.275000in}}{\pgfqpoint{4.225000in}{4.225000in}}%
\pgfusepath{clip}%
\pgfsetbuttcap%
\pgfsetroundjoin%
\definecolor{currentfill}{rgb}{0.179019,0.433756,0.557430}%
\pgfsetfillcolor{currentfill}%
\pgfsetfillopacity{0.700000}%
\pgfsetlinewidth{0.501875pt}%
\definecolor{currentstroke}{rgb}{1.000000,1.000000,1.000000}%
\pgfsetstrokecolor{currentstroke}%
\pgfsetstrokeopacity{0.500000}%
\pgfsetdash{}{0pt}%
\pgfpathmoveto{\pgfqpoint{4.351498in}{2.101250in}}%
\pgfpathlineto{\pgfqpoint{4.362577in}{2.104589in}}%
\pgfpathlineto{\pgfqpoint{4.373651in}{2.107973in}}%
\pgfpathlineto{\pgfqpoint{4.384722in}{2.111438in}}%
\pgfpathlineto{\pgfqpoint{4.395792in}{2.115019in}}%
\pgfpathlineto{\pgfqpoint{4.406861in}{2.118717in}}%
\pgfpathlineto{\pgfqpoint{4.400331in}{2.132122in}}%
\pgfpathlineto{\pgfqpoint{4.393805in}{2.145551in}}%
\pgfpathlineto{\pgfqpoint{4.387284in}{2.159045in}}%
\pgfpathlineto{\pgfqpoint{4.380769in}{2.172642in}}%
\pgfpathlineto{\pgfqpoint{4.374262in}{2.186382in}}%
\pgfpathlineto{\pgfqpoint{4.363190in}{2.182496in}}%
\pgfpathlineto{\pgfqpoint{4.352121in}{2.178876in}}%
\pgfpathlineto{\pgfqpoint{4.341057in}{2.175528in}}%
\pgfpathlineto{\pgfqpoint{4.329993in}{2.172395in}}%
\pgfpathlineto{\pgfqpoint{4.318929in}{2.169415in}}%
\pgfpathlineto{\pgfqpoint{4.325431in}{2.155577in}}%
\pgfpathlineto{\pgfqpoint{4.331939in}{2.141876in}}%
\pgfpathlineto{\pgfqpoint{4.338454in}{2.128277in}}%
\pgfpathlineto{\pgfqpoint{4.344974in}{2.114747in}}%
\pgfpathclose%
\pgfusepath{stroke,fill}%
\end{pgfscope}%
\begin{pgfscope}%
\pgfpathrectangle{\pgfqpoint{0.887500in}{0.275000in}}{\pgfqpoint{4.225000in}{4.225000in}}%
\pgfusepath{clip}%
\pgfsetbuttcap%
\pgfsetroundjoin%
\definecolor{currentfill}{rgb}{0.311925,0.767822,0.415586}%
\pgfsetfillcolor{currentfill}%
\pgfsetfillopacity{0.700000}%
\pgfsetlinewidth{0.501875pt}%
\definecolor{currentstroke}{rgb}{1.000000,1.000000,1.000000}%
\pgfsetstrokecolor{currentstroke}%
\pgfsetstrokeopacity{0.500000}%
\pgfsetdash{}{0pt}%
\pgfpathmoveto{\pgfqpoint{3.229179in}{2.806790in}}%
\pgfpathlineto{\pgfqpoint{3.240576in}{2.819498in}}%
\pgfpathlineto{\pgfqpoint{3.251969in}{2.831371in}}%
\pgfpathlineto{\pgfqpoint{3.263357in}{2.842434in}}%
\pgfpathlineto{\pgfqpoint{3.274740in}{2.852707in}}%
\pgfpathlineto{\pgfqpoint{3.286116in}{2.862211in}}%
\pgfpathlineto{\pgfqpoint{3.279821in}{2.875848in}}%
\pgfpathlineto{\pgfqpoint{3.273529in}{2.889703in}}%
\pgfpathlineto{\pgfqpoint{3.267241in}{2.903711in}}%
\pgfpathlineto{\pgfqpoint{3.260955in}{2.917807in}}%
\pgfpathlineto{\pgfqpoint{3.254672in}{2.931925in}}%
\pgfpathlineto{\pgfqpoint{3.243306in}{2.922969in}}%
\pgfpathlineto{\pgfqpoint{3.231935in}{2.913447in}}%
\pgfpathlineto{\pgfqpoint{3.220560in}{2.903395in}}%
\pgfpathlineto{\pgfqpoint{3.209182in}{2.892838in}}%
\pgfpathlineto{\pgfqpoint{3.197799in}{2.881646in}}%
\pgfpathlineto{\pgfqpoint{3.204068in}{2.865998in}}%
\pgfpathlineto{\pgfqpoint{3.210340in}{2.850558in}}%
\pgfpathlineto{\pgfqpoint{3.216615in}{2.835457in}}%
\pgfpathlineto{\pgfqpoint{3.222895in}{2.820825in}}%
\pgfpathclose%
\pgfusepath{stroke,fill}%
\end{pgfscope}%
\begin{pgfscope}%
\pgfpathrectangle{\pgfqpoint{0.887500in}{0.275000in}}{\pgfqpoint{4.225000in}{4.225000in}}%
\pgfusepath{clip}%
\pgfsetbuttcap%
\pgfsetroundjoin%
\definecolor{currentfill}{rgb}{0.231674,0.318106,0.544834}%
\pgfsetfillcolor{currentfill}%
\pgfsetfillopacity{0.700000}%
\pgfsetlinewidth{0.501875pt}%
\definecolor{currentstroke}{rgb}{1.000000,1.000000,1.000000}%
\pgfsetstrokecolor{currentstroke}%
\pgfsetstrokeopacity{0.500000}%
\pgfsetdash{}{0pt}%
\pgfpathmoveto{\pgfqpoint{4.647706in}{1.865658in}}%
\pgfpathlineto{\pgfqpoint{4.658708in}{1.869001in}}%
\pgfpathlineto{\pgfqpoint{4.669704in}{1.872339in}}%
\pgfpathlineto{\pgfqpoint{4.680694in}{1.875673in}}%
\pgfpathlineto{\pgfqpoint{4.691679in}{1.879006in}}%
\pgfpathlineto{\pgfqpoint{4.702659in}{1.882339in}}%
\pgfpathlineto{\pgfqpoint{4.696119in}{1.897044in}}%
\pgfpathlineto{\pgfqpoint{4.689581in}{1.911717in}}%
\pgfpathlineto{\pgfqpoint{4.683044in}{1.926364in}}%
\pgfpathlineto{\pgfqpoint{4.676509in}{1.940986in}}%
\pgfpathlineto{\pgfqpoint{4.669976in}{1.955586in}}%
\pgfpathlineto{\pgfqpoint{4.658997in}{1.952261in}}%
\pgfpathlineto{\pgfqpoint{4.648013in}{1.948941in}}%
\pgfpathlineto{\pgfqpoint{4.637024in}{1.945625in}}%
\pgfpathlineto{\pgfqpoint{4.626030in}{1.942310in}}%
\pgfpathlineto{\pgfqpoint{4.615030in}{1.938997in}}%
\pgfpathlineto{\pgfqpoint{4.621562in}{1.924365in}}%
\pgfpathlineto{\pgfqpoint{4.628095in}{1.909716in}}%
\pgfpathlineto{\pgfqpoint{4.634630in}{1.895048in}}%
\pgfpathlineto{\pgfqpoint{4.641167in}{1.880362in}}%
\pgfpathclose%
\pgfusepath{stroke,fill}%
\end{pgfscope}%
\begin{pgfscope}%
\pgfpathrectangle{\pgfqpoint{0.887500in}{0.275000in}}{\pgfqpoint{4.225000in}{4.225000in}}%
\pgfusepath{clip}%
\pgfsetbuttcap%
\pgfsetroundjoin%
\definecolor{currentfill}{rgb}{0.252899,0.742211,0.448284}%
\pgfsetfillcolor{currentfill}%
\pgfsetfillopacity{0.700000}%
\pgfsetlinewidth{0.501875pt}%
\definecolor{currentstroke}{rgb}{1.000000,1.000000,1.000000}%
\pgfsetstrokecolor{currentstroke}%
\pgfsetstrokeopacity{0.500000}%
\pgfsetdash{}{0pt}%
\pgfpathmoveto{\pgfqpoint{3.172150in}{2.729889in}}%
\pgfpathlineto{\pgfqpoint{3.183559in}{2.747133in}}%
\pgfpathlineto{\pgfqpoint{3.194967in}{2.763420in}}%
\pgfpathlineto{\pgfqpoint{3.206374in}{2.778776in}}%
\pgfpathlineto{\pgfqpoint{3.217778in}{2.793225in}}%
\pgfpathlineto{\pgfqpoint{3.229179in}{2.806790in}}%
\pgfpathlineto{\pgfqpoint{3.222895in}{2.820825in}}%
\pgfpathlineto{\pgfqpoint{3.216615in}{2.835457in}}%
\pgfpathlineto{\pgfqpoint{3.210340in}{2.850558in}}%
\pgfpathlineto{\pgfqpoint{3.204068in}{2.865998in}}%
\pgfpathlineto{\pgfqpoint{3.197799in}{2.881646in}}%
\pgfpathlineto{\pgfqpoint{3.186413in}{2.869582in}}%
\pgfpathlineto{\pgfqpoint{3.175022in}{2.856410in}}%
\pgfpathlineto{\pgfqpoint{3.163629in}{2.841893in}}%
\pgfpathlineto{\pgfqpoint{3.152232in}{2.825794in}}%
\pgfpathlineto{\pgfqpoint{3.140833in}{2.807879in}}%
\pgfpathlineto{\pgfqpoint{3.147090in}{2.791065in}}%
\pgfpathlineto{\pgfqpoint{3.153349in}{2.774693in}}%
\pgfpathlineto{\pgfqpoint{3.159612in}{2.758928in}}%
\pgfpathlineto{\pgfqpoint{3.165878in}{2.743939in}}%
\pgfpathclose%
\pgfusepath{stroke,fill}%
\end{pgfscope}%
\begin{pgfscope}%
\pgfpathrectangle{\pgfqpoint{0.887500in}{0.275000in}}{\pgfqpoint{4.225000in}{4.225000in}}%
\pgfusepath{clip}%
\pgfsetbuttcap%
\pgfsetroundjoin%
\definecolor{currentfill}{rgb}{0.133743,0.548535,0.553541}%
\pgfsetfillcolor{currentfill}%
\pgfsetfillopacity{0.700000}%
\pgfsetlinewidth{0.501875pt}%
\definecolor{currentstroke}{rgb}{1.000000,1.000000,1.000000}%
\pgfsetstrokecolor{currentstroke}%
\pgfsetstrokeopacity{0.500000}%
\pgfsetdash{}{0pt}%
\pgfpathmoveto{\pgfqpoint{4.055330in}{2.338199in}}%
\pgfpathlineto{\pgfqpoint{4.066488in}{2.341705in}}%
\pgfpathlineto{\pgfqpoint{4.077641in}{2.345207in}}%
\pgfpathlineto{\pgfqpoint{4.088789in}{2.348700in}}%
\pgfpathlineto{\pgfqpoint{4.099930in}{2.352177in}}%
\pgfpathlineto{\pgfqpoint{4.111066in}{2.355632in}}%
\pgfpathlineto{\pgfqpoint{4.104625in}{2.370030in}}%
\pgfpathlineto{\pgfqpoint{4.098187in}{2.384414in}}%
\pgfpathlineto{\pgfqpoint{4.091750in}{2.398778in}}%
\pgfpathlineto{\pgfqpoint{4.085316in}{2.413120in}}%
\pgfpathlineto{\pgfqpoint{4.078883in}{2.427436in}}%
\pgfpathlineto{\pgfqpoint{4.067751in}{2.424016in}}%
\pgfpathlineto{\pgfqpoint{4.056612in}{2.420578in}}%
\pgfpathlineto{\pgfqpoint{4.045467in}{2.417128in}}%
\pgfpathlineto{\pgfqpoint{4.034317in}{2.413670in}}%
\pgfpathlineto{\pgfqpoint{4.023162in}{2.410209in}}%
\pgfpathlineto{\pgfqpoint{4.029591in}{2.395827in}}%
\pgfpathlineto{\pgfqpoint{4.036022in}{2.381427in}}%
\pgfpathlineto{\pgfqpoint{4.042455in}{2.367016in}}%
\pgfpathlineto{\pgfqpoint{4.048891in}{2.352604in}}%
\pgfpathclose%
\pgfusepath{stroke,fill}%
\end{pgfscope}%
\begin{pgfscope}%
\pgfpathrectangle{\pgfqpoint{0.887500in}{0.275000in}}{\pgfqpoint{4.225000in}{4.225000in}}%
\pgfusepath{clip}%
\pgfsetbuttcap%
\pgfsetroundjoin%
\definecolor{currentfill}{rgb}{0.180629,0.429975,0.557282}%
\pgfsetfillcolor{currentfill}%
\pgfsetfillopacity{0.700000}%
\pgfsetlinewidth{0.501875pt}%
\definecolor{currentstroke}{rgb}{1.000000,1.000000,1.000000}%
\pgfsetstrokecolor{currentstroke}%
\pgfsetstrokeopacity{0.500000}%
\pgfsetdash{}{0pt}%
\pgfpathmoveto{\pgfqpoint{2.835538in}{2.096915in}}%
\pgfpathlineto{\pgfqpoint{2.846995in}{2.101033in}}%
\pgfpathlineto{\pgfqpoint{2.858446in}{2.105405in}}%
\pgfpathlineto{\pgfqpoint{2.869892in}{2.109777in}}%
\pgfpathlineto{\pgfqpoint{2.881334in}{2.113888in}}%
\pgfpathlineto{\pgfqpoint{2.892773in}{2.117479in}}%
\pgfpathlineto{\pgfqpoint{2.886575in}{2.126949in}}%
\pgfpathlineto{\pgfqpoint{2.880382in}{2.136388in}}%
\pgfpathlineto{\pgfqpoint{2.874192in}{2.145797in}}%
\pgfpathlineto{\pgfqpoint{2.868007in}{2.155173in}}%
\pgfpathlineto{\pgfqpoint{2.861825in}{2.164516in}}%
\pgfpathlineto{\pgfqpoint{2.850396in}{2.160954in}}%
\pgfpathlineto{\pgfqpoint{2.838965in}{2.156815in}}%
\pgfpathlineto{\pgfqpoint{2.827531in}{2.152386in}}%
\pgfpathlineto{\pgfqpoint{2.816091in}{2.147955in}}%
\pgfpathlineto{\pgfqpoint{2.804645in}{2.143801in}}%
\pgfpathlineto{\pgfqpoint{2.810815in}{2.134495in}}%
\pgfpathlineto{\pgfqpoint{2.816990in}{2.125155in}}%
\pgfpathlineto{\pgfqpoint{2.823168in}{2.115779in}}%
\pgfpathlineto{\pgfqpoint{2.829351in}{2.106366in}}%
\pgfpathclose%
\pgfusepath{stroke,fill}%
\end{pgfscope}%
\begin{pgfscope}%
\pgfpathrectangle{\pgfqpoint{0.887500in}{0.275000in}}{\pgfqpoint{4.225000in}{4.225000in}}%
\pgfusepath{clip}%
\pgfsetbuttcap%
\pgfsetroundjoin%
\definecolor{currentfill}{rgb}{0.153894,0.680203,0.504172}%
\pgfsetfillcolor{currentfill}%
\pgfsetfillopacity{0.700000}%
\pgfsetlinewidth{0.501875pt}%
\definecolor{currentstroke}{rgb}{1.000000,1.000000,1.000000}%
\pgfsetstrokecolor{currentstroke}%
\pgfsetstrokeopacity{0.500000}%
\pgfsetdash{}{0pt}%
\pgfpathmoveto{\pgfqpoint{3.670895in}{2.621625in}}%
\pgfpathlineto{\pgfqpoint{3.682155in}{2.625424in}}%
\pgfpathlineto{\pgfqpoint{3.693409in}{2.629197in}}%
\pgfpathlineto{\pgfqpoint{3.704657in}{2.632930in}}%
\pgfpathlineto{\pgfqpoint{3.715899in}{2.636606in}}%
\pgfpathlineto{\pgfqpoint{3.727134in}{2.640213in}}%
\pgfpathlineto{\pgfqpoint{3.720757in}{2.654181in}}%
\pgfpathlineto{\pgfqpoint{3.714381in}{2.668114in}}%
\pgfpathlineto{\pgfqpoint{3.708008in}{2.682002in}}%
\pgfpathlineto{\pgfqpoint{3.701636in}{2.695838in}}%
\pgfpathlineto{\pgfqpoint{3.695266in}{2.709620in}}%
\pgfpathlineto{\pgfqpoint{3.684034in}{2.706080in}}%
\pgfpathlineto{\pgfqpoint{3.672796in}{2.702472in}}%
\pgfpathlineto{\pgfqpoint{3.661552in}{2.698808in}}%
\pgfpathlineto{\pgfqpoint{3.650302in}{2.695104in}}%
\pgfpathlineto{\pgfqpoint{3.639047in}{2.691372in}}%
\pgfpathlineto{\pgfqpoint{3.645413in}{2.677539in}}%
\pgfpathlineto{\pgfqpoint{3.651781in}{2.663644in}}%
\pgfpathlineto{\pgfqpoint{3.658150in}{2.649687in}}%
\pgfpathlineto{\pgfqpoint{3.664522in}{2.635678in}}%
\pgfpathclose%
\pgfusepath{stroke,fill}%
\end{pgfscope}%
\begin{pgfscope}%
\pgfpathrectangle{\pgfqpoint{0.887500in}{0.275000in}}{\pgfqpoint{4.225000in}{4.225000in}}%
\pgfusepath{clip}%
\pgfsetbuttcap%
\pgfsetroundjoin%
\definecolor{currentfill}{rgb}{0.166617,0.463708,0.558119}%
\pgfsetfillcolor{currentfill}%
\pgfsetfillopacity{0.700000}%
\pgfsetlinewidth{0.501875pt}%
\definecolor{currentstroke}{rgb}{1.000000,1.000000,1.000000}%
\pgfsetstrokecolor{currentstroke}%
\pgfsetstrokeopacity{0.500000}%
\pgfsetdash{}{0pt}%
\pgfpathmoveto{\pgfqpoint{2.513373in}{2.167096in}}%
\pgfpathlineto{\pgfqpoint{2.524916in}{2.170552in}}%
\pgfpathlineto{\pgfqpoint{2.536453in}{2.174020in}}%
\pgfpathlineto{\pgfqpoint{2.547985in}{2.177503in}}%
\pgfpathlineto{\pgfqpoint{2.559510in}{2.181004in}}%
\pgfpathlineto{\pgfqpoint{2.571030in}{2.184527in}}%
\pgfpathlineto{\pgfqpoint{2.564936in}{2.193567in}}%
\pgfpathlineto{\pgfqpoint{2.558847in}{2.202577in}}%
\pgfpathlineto{\pgfqpoint{2.552762in}{2.211557in}}%
\pgfpathlineto{\pgfqpoint{2.546682in}{2.220508in}}%
\pgfpathlineto{\pgfqpoint{2.540605in}{2.229431in}}%
\pgfpathlineto{\pgfqpoint{2.529096in}{2.225936in}}%
\pgfpathlineto{\pgfqpoint{2.517582in}{2.222465in}}%
\pgfpathlineto{\pgfqpoint{2.506061in}{2.219016in}}%
\pgfpathlineto{\pgfqpoint{2.494535in}{2.215585in}}%
\pgfpathlineto{\pgfqpoint{2.483002in}{2.212168in}}%
\pgfpathlineto{\pgfqpoint{2.489068in}{2.203213in}}%
\pgfpathlineto{\pgfqpoint{2.495138in}{2.194229in}}%
\pgfpathlineto{\pgfqpoint{2.501212in}{2.185215in}}%
\pgfpathlineto{\pgfqpoint{2.507290in}{2.176171in}}%
\pgfpathclose%
\pgfusepath{stroke,fill}%
\end{pgfscope}%
\begin{pgfscope}%
\pgfpathrectangle{\pgfqpoint{0.887500in}{0.275000in}}{\pgfqpoint{4.225000in}{4.225000in}}%
\pgfusepath{clip}%
\pgfsetbuttcap%
\pgfsetroundjoin%
\definecolor{currentfill}{rgb}{0.166617,0.463708,0.558119}%
\pgfsetfillcolor{currentfill}%
\pgfsetfillopacity{0.700000}%
\pgfsetlinewidth{0.501875pt}%
\definecolor{currentstroke}{rgb}{1.000000,1.000000,1.000000}%
\pgfsetstrokecolor{currentstroke}%
\pgfsetstrokeopacity{0.500000}%
\pgfsetdash{}{0pt}%
\pgfpathmoveto{\pgfqpoint{4.263528in}{2.154666in}}%
\pgfpathlineto{\pgfqpoint{4.274624in}{2.157773in}}%
\pgfpathlineto{\pgfqpoint{4.285712in}{2.160765in}}%
\pgfpathlineto{\pgfqpoint{4.296790in}{2.163662in}}%
\pgfpathlineto{\pgfqpoint{4.307862in}{2.166524in}}%
\pgfpathlineto{\pgfqpoint{4.318929in}{2.169415in}}%
\pgfpathlineto{\pgfqpoint{4.312436in}{2.183424in}}%
\pgfpathlineto{\pgfqpoint{4.305951in}{2.197638in}}%
\pgfpathlineto{\pgfqpoint{4.299476in}{2.212052in}}%
\pgfpathlineto{\pgfqpoint{4.293009in}{2.226638in}}%
\pgfpathlineto{\pgfqpoint{4.286549in}{2.241364in}}%
\pgfpathlineto{\pgfqpoint{4.275483in}{2.238482in}}%
\pgfpathlineto{\pgfqpoint{4.264413in}{2.235640in}}%
\pgfpathlineto{\pgfqpoint{4.253337in}{2.232789in}}%
\pgfpathlineto{\pgfqpoint{4.242253in}{2.229879in}}%
\pgfpathlineto{\pgfqpoint{4.231162in}{2.226894in}}%
\pgfpathlineto{\pgfqpoint{4.237626in}{2.212334in}}%
\pgfpathlineto{\pgfqpoint{4.244095in}{2.197822in}}%
\pgfpathlineto{\pgfqpoint{4.250568in}{2.183367in}}%
\pgfpathlineto{\pgfqpoint{4.257046in}{2.168981in}}%
\pgfpathclose%
\pgfusepath{stroke,fill}%
\end{pgfscope}%
\begin{pgfscope}%
\pgfpathrectangle{\pgfqpoint{0.887500in}{0.275000in}}{\pgfqpoint{4.225000in}{4.225000in}}%
\pgfusepath{clip}%
\pgfsetbuttcap%
\pgfsetroundjoin%
\definecolor{currentfill}{rgb}{0.154815,0.493313,0.557840}%
\pgfsetfillcolor{currentfill}%
\pgfsetfillopacity{0.700000}%
\pgfsetlinewidth{0.501875pt}%
\definecolor{currentstroke}{rgb}{1.000000,1.000000,1.000000}%
\pgfsetstrokecolor{currentstroke}%
\pgfsetstrokeopacity{0.500000}%
\pgfsetdash{}{0pt}%
\pgfpathmoveto{\pgfqpoint{2.191187in}{2.231890in}}%
\pgfpathlineto{\pgfqpoint{2.202810in}{2.235389in}}%
\pgfpathlineto{\pgfqpoint{2.214427in}{2.238894in}}%
\pgfpathlineto{\pgfqpoint{2.226038in}{2.242400in}}%
\pgfpathlineto{\pgfqpoint{2.237643in}{2.245905in}}%
\pgfpathlineto{\pgfqpoint{2.249242in}{2.249405in}}%
\pgfpathlineto{\pgfqpoint{2.243257in}{2.258198in}}%
\pgfpathlineto{\pgfqpoint{2.237275in}{2.266965in}}%
\pgfpathlineto{\pgfqpoint{2.231298in}{2.275708in}}%
\pgfpathlineto{\pgfqpoint{2.225325in}{2.284427in}}%
\pgfpathlineto{\pgfqpoint{2.219356in}{2.293123in}}%
\pgfpathlineto{\pgfqpoint{2.207767in}{2.289664in}}%
\pgfpathlineto{\pgfqpoint{2.196173in}{2.286201in}}%
\pgfpathlineto{\pgfqpoint{2.184573in}{2.282736in}}%
\pgfpathlineto{\pgfqpoint{2.172968in}{2.279275in}}%
\pgfpathlineto{\pgfqpoint{2.161356in}{2.275820in}}%
\pgfpathlineto{\pgfqpoint{2.167314in}{2.267083in}}%
\pgfpathlineto{\pgfqpoint{2.173276in}{2.258323in}}%
\pgfpathlineto{\pgfqpoint{2.179242in}{2.249537in}}%
\pgfpathlineto{\pgfqpoint{2.185212in}{2.240727in}}%
\pgfpathclose%
\pgfusepath{stroke,fill}%
\end{pgfscope}%
\begin{pgfscope}%
\pgfpathrectangle{\pgfqpoint{0.887500in}{0.275000in}}{\pgfqpoint{4.225000in}{4.225000in}}%
\pgfusepath{clip}%
\pgfsetbuttcap%
\pgfsetroundjoin%
\definecolor{currentfill}{rgb}{0.187231,0.414746,0.556547}%
\pgfsetfillcolor{currentfill}%
\pgfsetfillopacity{0.700000}%
\pgfsetlinewidth{0.501875pt}%
\definecolor{currentstroke}{rgb}{1.000000,1.000000,1.000000}%
\pgfsetstrokecolor{currentstroke}%
\pgfsetstrokeopacity{0.500000}%
\pgfsetdash{}{0pt}%
\pgfpathmoveto{\pgfqpoint{2.923820in}{2.069661in}}%
\pgfpathlineto{\pgfqpoint{2.935265in}{2.072555in}}%
\pgfpathlineto{\pgfqpoint{2.946708in}{2.074468in}}%
\pgfpathlineto{\pgfqpoint{2.958146in}{2.075210in}}%
\pgfpathlineto{\pgfqpoint{2.969579in}{2.075239in}}%
\pgfpathlineto{\pgfqpoint{2.981004in}{2.075445in}}%
\pgfpathlineto{\pgfqpoint{2.974779in}{2.084953in}}%
\pgfpathlineto{\pgfqpoint{2.968558in}{2.094401in}}%
\pgfpathlineto{\pgfqpoint{2.962341in}{2.103783in}}%
\pgfpathlineto{\pgfqpoint{2.956128in}{2.113091in}}%
\pgfpathlineto{\pgfqpoint{2.949919in}{2.122322in}}%
\pgfpathlineto{\pgfqpoint{2.938501in}{2.122362in}}%
\pgfpathlineto{\pgfqpoint{2.927075in}{2.122589in}}%
\pgfpathlineto{\pgfqpoint{2.915643in}{2.122060in}}%
\pgfpathlineto{\pgfqpoint{2.904209in}{2.120289in}}%
\pgfpathlineto{\pgfqpoint{2.892773in}{2.117479in}}%
\pgfpathlineto{\pgfqpoint{2.898974in}{2.107979in}}%
\pgfpathlineto{\pgfqpoint{2.905180in}{2.098450in}}%
\pgfpathlineto{\pgfqpoint{2.911389in}{2.088889in}}%
\pgfpathlineto{\pgfqpoint{2.917603in}{2.079294in}}%
\pgfpathclose%
\pgfusepath{stroke,fill}%
\end{pgfscope}%
\begin{pgfscope}%
\pgfpathrectangle{\pgfqpoint{0.887500in}{0.275000in}}{\pgfqpoint{4.225000in}{4.225000in}}%
\pgfusepath{clip}%
\pgfsetbuttcap%
\pgfsetroundjoin%
\definecolor{currentfill}{rgb}{0.143343,0.522773,0.556295}%
\pgfsetfillcolor{currentfill}%
\pgfsetfillopacity{0.700000}%
\pgfsetlinewidth{0.501875pt}%
\definecolor{currentstroke}{rgb}{1.000000,1.000000,1.000000}%
\pgfsetstrokecolor{currentstroke}%
\pgfsetstrokeopacity{0.500000}%
\pgfsetdash{}{0pt}%
\pgfpathmoveto{\pgfqpoint{1.869025in}{2.294255in}}%
\pgfpathlineto{\pgfqpoint{1.880727in}{2.297712in}}%
\pgfpathlineto{\pgfqpoint{1.892425in}{2.301167in}}%
\pgfpathlineto{\pgfqpoint{1.904116in}{2.304616in}}%
\pgfpathlineto{\pgfqpoint{1.915802in}{2.308058in}}%
\pgfpathlineto{\pgfqpoint{1.927482in}{2.311489in}}%
\pgfpathlineto{\pgfqpoint{1.921606in}{2.320089in}}%
\pgfpathlineto{\pgfqpoint{1.915735in}{2.328667in}}%
\pgfpathlineto{\pgfqpoint{1.909867in}{2.337221in}}%
\pgfpathlineto{\pgfqpoint{1.904005in}{2.345754in}}%
\pgfpathlineto{\pgfqpoint{1.898146in}{2.354266in}}%
\pgfpathlineto{\pgfqpoint{1.886477in}{2.350871in}}%
\pgfpathlineto{\pgfqpoint{1.874803in}{2.347468in}}%
\pgfpathlineto{\pgfqpoint{1.863123in}{2.344059in}}%
\pgfpathlineto{\pgfqpoint{1.851437in}{2.340645in}}%
\pgfpathlineto{\pgfqpoint{1.839746in}{2.337228in}}%
\pgfpathlineto{\pgfqpoint{1.845593in}{2.328679in}}%
\pgfpathlineto{\pgfqpoint{1.851444in}{2.320108in}}%
\pgfpathlineto{\pgfqpoint{1.857300in}{2.311514in}}%
\pgfpathlineto{\pgfqpoint{1.863160in}{2.302896in}}%
\pgfpathclose%
\pgfusepath{stroke,fill}%
\end{pgfscope}%
\begin{pgfscope}%
\pgfpathrectangle{\pgfqpoint{0.887500in}{0.275000in}}{\pgfqpoint{4.225000in}{4.225000in}}%
\pgfusepath{clip}%
\pgfsetbuttcap%
\pgfsetroundjoin%
\definecolor{currentfill}{rgb}{0.125394,0.574318,0.549086}%
\pgfsetfillcolor{currentfill}%
\pgfsetfillopacity{0.700000}%
\pgfsetlinewidth{0.501875pt}%
\definecolor{currentstroke}{rgb}{1.000000,1.000000,1.000000}%
\pgfsetstrokecolor{currentstroke}%
\pgfsetstrokeopacity{0.500000}%
\pgfsetdash{}{0pt}%
\pgfpathmoveto{\pgfqpoint{3.967304in}{2.392990in}}%
\pgfpathlineto{\pgfqpoint{3.978487in}{2.396423in}}%
\pgfpathlineto{\pgfqpoint{3.989663in}{2.399857in}}%
\pgfpathlineto{\pgfqpoint{4.000835in}{2.403298in}}%
\pgfpathlineto{\pgfqpoint{4.012001in}{2.406751in}}%
\pgfpathlineto{\pgfqpoint{4.023162in}{2.410209in}}%
\pgfpathlineto{\pgfqpoint{4.016735in}{2.424565in}}%
\pgfpathlineto{\pgfqpoint{4.010309in}{2.438884in}}%
\pgfpathlineto{\pgfqpoint{4.003885in}{2.453160in}}%
\pgfpathlineto{\pgfqpoint{3.997462in}{2.467391in}}%
\pgfpathlineto{\pgfqpoint{3.991041in}{2.481581in}}%
\pgfpathlineto{\pgfqpoint{3.979884in}{2.478178in}}%
\pgfpathlineto{\pgfqpoint{3.968721in}{2.474778in}}%
\pgfpathlineto{\pgfqpoint{3.957552in}{2.471384in}}%
\pgfpathlineto{\pgfqpoint{3.946379in}{2.467994in}}%
\pgfpathlineto{\pgfqpoint{3.935199in}{2.464608in}}%
\pgfpathlineto{\pgfqpoint{3.941618in}{2.450388in}}%
\pgfpathlineto{\pgfqpoint{3.948037in}{2.436116in}}%
\pgfpathlineto{\pgfqpoint{3.954458in}{2.421789in}}%
\pgfpathlineto{\pgfqpoint{3.960881in}{2.407410in}}%
\pgfpathclose%
\pgfusepath{stroke,fill}%
\end{pgfscope}%
\begin{pgfscope}%
\pgfpathrectangle{\pgfqpoint{0.887500in}{0.275000in}}{\pgfqpoint{4.225000in}{4.225000in}}%
\pgfusepath{clip}%
\pgfsetbuttcap%
\pgfsetroundjoin%
\definecolor{currentfill}{rgb}{0.185783,0.704891,0.485273}%
\pgfsetfillcolor{currentfill}%
\pgfsetfillopacity{0.700000}%
\pgfsetlinewidth{0.501875pt}%
\definecolor{currentstroke}{rgb}{1.000000,1.000000,1.000000}%
\pgfsetstrokecolor{currentstroke}%
\pgfsetstrokeopacity{0.500000}%
\pgfsetdash{}{0pt}%
\pgfpathmoveto{\pgfqpoint{3.582688in}{2.672684in}}%
\pgfpathlineto{\pgfqpoint{3.593971in}{2.676449in}}%
\pgfpathlineto{\pgfqpoint{3.605248in}{2.680165in}}%
\pgfpathlineto{\pgfqpoint{3.616519in}{2.683889in}}%
\pgfpathlineto{\pgfqpoint{3.627786in}{2.687629in}}%
\pgfpathlineto{\pgfqpoint{3.639047in}{2.691372in}}%
\pgfpathlineto{\pgfqpoint{3.632682in}{2.705149in}}%
\pgfpathlineto{\pgfqpoint{3.626320in}{2.718873in}}%
\pgfpathlineto{\pgfqpoint{3.619959in}{2.732551in}}%
\pgfpathlineto{\pgfqpoint{3.613600in}{2.746188in}}%
\pgfpathlineto{\pgfqpoint{3.607244in}{2.759788in}}%
\pgfpathlineto{\pgfqpoint{3.595987in}{2.756095in}}%
\pgfpathlineto{\pgfqpoint{3.584724in}{2.752396in}}%
\pgfpathlineto{\pgfqpoint{3.573457in}{2.748699in}}%
\pgfpathlineto{\pgfqpoint{3.562184in}{2.745000in}}%
\pgfpathlineto{\pgfqpoint{3.550905in}{2.741271in}}%
\pgfpathlineto{\pgfqpoint{3.557257in}{2.727643in}}%
\pgfpathlineto{\pgfqpoint{3.563612in}{2.713975in}}%
\pgfpathlineto{\pgfqpoint{3.569969in}{2.700263in}}%
\pgfpathlineto{\pgfqpoint{3.576328in}{2.686501in}}%
\pgfpathclose%
\pgfusepath{stroke,fill}%
\end{pgfscope}%
\begin{pgfscope}%
\pgfpathrectangle{\pgfqpoint{0.887500in}{0.275000in}}{\pgfqpoint{4.225000in}{4.225000in}}%
\pgfusepath{clip}%
\pgfsetbuttcap%
\pgfsetroundjoin%
\definecolor{currentfill}{rgb}{0.216210,0.351535,0.550627}%
\pgfsetfillcolor{currentfill}%
\pgfsetfillopacity{0.700000}%
\pgfsetlinewidth{0.501875pt}%
\definecolor{currentstroke}{rgb}{1.000000,1.000000,1.000000}%
\pgfsetstrokecolor{currentstroke}%
\pgfsetstrokeopacity{0.500000}%
\pgfsetdash{}{0pt}%
\pgfpathmoveto{\pgfqpoint{4.559947in}{1.922392in}}%
\pgfpathlineto{\pgfqpoint{4.570975in}{1.925722in}}%
\pgfpathlineto{\pgfqpoint{4.581997in}{1.929047in}}%
\pgfpathlineto{\pgfqpoint{4.593014in}{1.932366in}}%
\pgfpathlineto{\pgfqpoint{4.604025in}{1.935682in}}%
\pgfpathlineto{\pgfqpoint{4.615030in}{1.938997in}}%
\pgfpathlineto{\pgfqpoint{4.608500in}{1.953610in}}%
\pgfpathlineto{\pgfqpoint{4.601973in}{1.968206in}}%
\pgfpathlineto{\pgfqpoint{4.595447in}{1.982785in}}%
\pgfpathlineto{\pgfqpoint{4.588923in}{1.997346in}}%
\pgfpathlineto{\pgfqpoint{4.582401in}{2.011890in}}%
\pgfpathlineto{\pgfqpoint{4.571398in}{2.008621in}}%
\pgfpathlineto{\pgfqpoint{4.560390in}{2.005358in}}%
\pgfpathlineto{\pgfqpoint{4.549377in}{2.002103in}}%
\pgfpathlineto{\pgfqpoint{4.538358in}{1.998854in}}%
\pgfpathlineto{\pgfqpoint{4.527335in}{1.995612in}}%
\pgfpathlineto{\pgfqpoint{4.533855in}{1.981037in}}%
\pgfpathlineto{\pgfqpoint{4.540376in}{1.966422in}}%
\pgfpathlineto{\pgfqpoint{4.546898in}{1.951771in}}%
\pgfpathlineto{\pgfqpoint{4.553422in}{1.937093in}}%
\pgfpathclose%
\pgfusepath{stroke,fill}%
\end{pgfscope}%
\begin{pgfscope}%
\pgfpathrectangle{\pgfqpoint{0.887500in}{0.275000in}}{\pgfqpoint{4.225000in}{4.225000in}}%
\pgfusepath{clip}%
\pgfsetbuttcap%
\pgfsetroundjoin%
\definecolor{currentfill}{rgb}{0.296479,0.761561,0.424223}%
\pgfsetfillcolor{currentfill}%
\pgfsetfillopacity{0.700000}%
\pgfsetlinewidth{0.501875pt}%
\definecolor{currentstroke}{rgb}{1.000000,1.000000,1.000000}%
\pgfsetstrokecolor{currentstroke}%
\pgfsetstrokeopacity{0.500000}%
\pgfsetdash{}{0pt}%
\pgfpathmoveto{\pgfqpoint{3.317655in}{2.797644in}}%
\pgfpathlineto{\pgfqpoint{3.329032in}{2.806040in}}%
\pgfpathlineto{\pgfqpoint{3.340401in}{2.813793in}}%
\pgfpathlineto{\pgfqpoint{3.351762in}{2.820904in}}%
\pgfpathlineto{\pgfqpoint{3.363115in}{2.827373in}}%
\pgfpathlineto{\pgfqpoint{3.374460in}{2.833241in}}%
\pgfpathlineto{\pgfqpoint{3.368140in}{2.846262in}}%
\pgfpathlineto{\pgfqpoint{3.361822in}{2.859298in}}%
\pgfpathlineto{\pgfqpoint{3.355508in}{2.872394in}}%
\pgfpathlineto{\pgfqpoint{3.349198in}{2.885597in}}%
\pgfpathlineto{\pgfqpoint{3.342891in}{2.898946in}}%
\pgfpathlineto{\pgfqpoint{3.331552in}{2.892944in}}%
\pgfpathlineto{\pgfqpoint{3.320205in}{2.886311in}}%
\pgfpathlineto{\pgfqpoint{3.308849in}{2.878992in}}%
\pgfpathlineto{\pgfqpoint{3.297486in}{2.870966in}}%
\pgfpathlineto{\pgfqpoint{3.286116in}{2.862211in}}%
\pgfpathlineto{\pgfqpoint{3.292416in}{2.848858in}}%
\pgfpathlineto{\pgfqpoint{3.298720in}{2.835805in}}%
\pgfpathlineto{\pgfqpoint{3.305028in}{2.822976in}}%
\pgfpathlineto{\pgfqpoint{3.311340in}{2.810285in}}%
\pgfpathclose%
\pgfusepath{stroke,fill}%
\end{pgfscope}%
\begin{pgfscope}%
\pgfpathrectangle{\pgfqpoint{0.887500in}{0.275000in}}{\pgfqpoint{4.225000in}{4.225000in}}%
\pgfusepath{clip}%
\pgfsetbuttcap%
\pgfsetroundjoin%
\definecolor{currentfill}{rgb}{0.150148,0.676631,0.506589}%
\pgfsetfillcolor{currentfill}%
\pgfsetfillopacity{0.700000}%
\pgfsetlinewidth{0.501875pt}%
\definecolor{currentstroke}{rgb}{1.000000,1.000000,1.000000}%
\pgfsetstrokecolor{currentstroke}%
\pgfsetstrokeopacity{0.500000}%
\pgfsetdash{}{0pt}%
\pgfpathmoveto{\pgfqpoint{3.146465in}{2.575163in}}%
\pgfpathlineto{\pgfqpoint{3.157887in}{2.596406in}}%
\pgfpathlineto{\pgfqpoint{3.169313in}{2.617090in}}%
\pgfpathlineto{\pgfqpoint{3.180740in}{2.637022in}}%
\pgfpathlineto{\pgfqpoint{3.192168in}{2.656009in}}%
\pgfpathlineto{\pgfqpoint{3.203595in}{2.673857in}}%
\pgfpathlineto{\pgfqpoint{3.197295in}{2.683860in}}%
\pgfpathlineto{\pgfqpoint{3.191000in}{2.694192in}}%
\pgfpathlineto{\pgfqpoint{3.184710in}{2.705129in}}%
\pgfpathlineto{\pgfqpoint{3.178427in}{2.716946in}}%
\pgfpathlineto{\pgfqpoint{3.172150in}{2.729889in}}%
\pgfpathlineto{\pgfqpoint{3.160740in}{2.711679in}}%
\pgfpathlineto{\pgfqpoint{3.149332in}{2.692584in}}%
\pgfpathlineto{\pgfqpoint{3.137925in}{2.672729in}}%
\pgfpathlineto{\pgfqpoint{3.126521in}{2.652239in}}%
\pgfpathlineto{\pgfqpoint{3.115120in}{2.631240in}}%
\pgfpathlineto{\pgfqpoint{3.121382in}{2.619680in}}%
\pgfpathlineto{\pgfqpoint{3.127648in}{2.608389in}}%
\pgfpathlineto{\pgfqpoint{3.133917in}{2.597272in}}%
\pgfpathlineto{\pgfqpoint{3.140189in}{2.586231in}}%
\pgfpathclose%
\pgfusepath{stroke,fill}%
\end{pgfscope}%
\begin{pgfscope}%
\pgfpathrectangle{\pgfqpoint{0.887500in}{0.275000in}}{\pgfqpoint{4.225000in}{4.225000in}}%
\pgfusepath{clip}%
\pgfsetbuttcap%
\pgfsetroundjoin%
\definecolor{currentfill}{rgb}{0.220124,0.725509,0.466226}%
\pgfsetfillcolor{currentfill}%
\pgfsetfillopacity{0.700000}%
\pgfsetlinewidth{0.501875pt}%
\definecolor{currentstroke}{rgb}{1.000000,1.000000,1.000000}%
\pgfsetstrokecolor{currentstroke}%
\pgfsetstrokeopacity{0.500000}%
\pgfsetdash{}{0pt}%
\pgfpathmoveto{\pgfqpoint{3.494413in}{2.721012in}}%
\pgfpathlineto{\pgfqpoint{3.505726in}{2.725388in}}%
\pgfpathlineto{\pgfqpoint{3.517031in}{2.729569in}}%
\pgfpathlineto{\pgfqpoint{3.528329in}{2.733588in}}%
\pgfpathlineto{\pgfqpoint{3.539620in}{2.737478in}}%
\pgfpathlineto{\pgfqpoint{3.550905in}{2.741271in}}%
\pgfpathlineto{\pgfqpoint{3.544554in}{2.754864in}}%
\pgfpathlineto{\pgfqpoint{3.538206in}{2.768428in}}%
\pgfpathlineto{\pgfqpoint{3.531861in}{2.781968in}}%
\pgfpathlineto{\pgfqpoint{3.525517in}{2.795488in}}%
\pgfpathlineto{\pgfqpoint{3.519176in}{2.808991in}}%
\pgfpathlineto{\pgfqpoint{3.507896in}{2.805228in}}%
\pgfpathlineto{\pgfqpoint{3.496609in}{2.801423in}}%
\pgfpathlineto{\pgfqpoint{3.485317in}{2.797570in}}%
\pgfpathlineto{\pgfqpoint{3.474019in}{2.793663in}}%
\pgfpathlineto{\pgfqpoint{3.462715in}{2.789692in}}%
\pgfpathlineto{\pgfqpoint{3.469051in}{2.776123in}}%
\pgfpathlineto{\pgfqpoint{3.475389in}{2.762464in}}%
\pgfpathlineto{\pgfqpoint{3.481729in}{2.748720in}}%
\pgfpathlineto{\pgfqpoint{3.488070in}{2.734899in}}%
\pgfpathclose%
\pgfusepath{stroke,fill}%
\end{pgfscope}%
\begin{pgfscope}%
\pgfpathrectangle{\pgfqpoint{0.887500in}{0.275000in}}{\pgfqpoint{4.225000in}{4.225000in}}%
\pgfusepath{clip}%
\pgfsetbuttcap%
\pgfsetroundjoin%
\definecolor{currentfill}{rgb}{0.194100,0.399323,0.555565}%
\pgfsetfillcolor{currentfill}%
\pgfsetfillopacity{0.700000}%
\pgfsetlinewidth{0.501875pt}%
\definecolor{currentstroke}{rgb}{1.000000,1.000000,1.000000}%
\pgfsetstrokecolor{currentstroke}%
\pgfsetstrokeopacity{0.500000}%
\pgfsetdash{}{0pt}%
\pgfpathmoveto{\pgfqpoint{3.012188in}{2.027299in}}%
\pgfpathlineto{\pgfqpoint{3.023612in}{2.028706in}}%
\pgfpathlineto{\pgfqpoint{3.035027in}{2.031996in}}%
\pgfpathlineto{\pgfqpoint{3.046437in}{2.038024in}}%
\pgfpathlineto{\pgfqpoint{3.057842in}{2.047649in}}%
\pgfpathlineto{\pgfqpoint{3.069246in}{2.061584in}}%
\pgfpathlineto{\pgfqpoint{3.062989in}{2.071456in}}%
\pgfpathlineto{\pgfqpoint{3.056736in}{2.081264in}}%
\pgfpathlineto{\pgfqpoint{3.050487in}{2.091013in}}%
\pgfpathlineto{\pgfqpoint{3.044241in}{2.100707in}}%
\pgfpathlineto{\pgfqpoint{3.038000in}{2.110352in}}%
\pgfpathlineto{\pgfqpoint{3.026613in}{2.095966in}}%
\pgfpathlineto{\pgfqpoint{3.015223in}{2.086090in}}%
\pgfpathlineto{\pgfqpoint{3.003826in}{2.079975in}}%
\pgfpathlineto{\pgfqpoint{2.992420in}{2.076725in}}%
\pgfpathlineto{\pgfqpoint{2.981004in}{2.075445in}}%
\pgfpathlineto{\pgfqpoint{2.987233in}{2.065887in}}%
\pgfpathlineto{\pgfqpoint{2.993466in}{2.056285in}}%
\pgfpathlineto{\pgfqpoint{2.999703in}{2.046648in}}%
\pgfpathlineto{\pgfqpoint{3.005943in}{2.036984in}}%
\pgfpathclose%
\pgfusepath{stroke,fill}%
\end{pgfscope}%
\begin{pgfscope}%
\pgfpathrectangle{\pgfqpoint{0.887500in}{0.275000in}}{\pgfqpoint{4.225000in}{4.225000in}}%
\pgfusepath{clip}%
\pgfsetbuttcap%
\pgfsetroundjoin%
\definecolor{currentfill}{rgb}{0.171176,0.452530,0.557965}%
\pgfsetfillcolor{currentfill}%
\pgfsetfillopacity{0.700000}%
\pgfsetlinewidth{0.501875pt}%
\definecolor{currentstroke}{rgb}{1.000000,1.000000,1.000000}%
\pgfsetstrokecolor{currentstroke}%
\pgfsetstrokeopacity{0.500000}%
\pgfsetdash{}{0pt}%
\pgfpathmoveto{\pgfqpoint{2.601560in}{2.138850in}}%
\pgfpathlineto{\pgfqpoint{2.613084in}{2.142429in}}%
\pgfpathlineto{\pgfqpoint{2.624603in}{2.146034in}}%
\pgfpathlineto{\pgfqpoint{2.636115in}{2.149658in}}%
\pgfpathlineto{\pgfqpoint{2.647622in}{2.153276in}}%
\pgfpathlineto{\pgfqpoint{2.659124in}{2.156861in}}%
\pgfpathlineto{\pgfqpoint{2.652999in}{2.166033in}}%
\pgfpathlineto{\pgfqpoint{2.646878in}{2.175172in}}%
\pgfpathlineto{\pgfqpoint{2.640761in}{2.184280in}}%
\pgfpathlineto{\pgfqpoint{2.634648in}{2.193356in}}%
\pgfpathlineto{\pgfqpoint{2.628540in}{2.202403in}}%
\pgfpathlineto{\pgfqpoint{2.617049in}{2.198842in}}%
\pgfpathlineto{\pgfqpoint{2.605553in}{2.195249in}}%
\pgfpathlineto{\pgfqpoint{2.594051in}{2.191651in}}%
\pgfpathlineto{\pgfqpoint{2.582543in}{2.188075in}}%
\pgfpathlineto{\pgfqpoint{2.571030in}{2.184527in}}%
\pgfpathlineto{\pgfqpoint{2.577127in}{2.175456in}}%
\pgfpathlineto{\pgfqpoint{2.583229in}{2.166353in}}%
\pgfpathlineto{\pgfqpoint{2.589335in}{2.157218in}}%
\pgfpathlineto{\pgfqpoint{2.595446in}{2.148051in}}%
\pgfpathclose%
\pgfusepath{stroke,fill}%
\end{pgfscope}%
\begin{pgfscope}%
\pgfpathrectangle{\pgfqpoint{0.887500in}{0.275000in}}{\pgfqpoint{4.225000in}{4.225000in}}%
\pgfusepath{clip}%
\pgfsetbuttcap%
\pgfsetroundjoin%
\definecolor{currentfill}{rgb}{0.154815,0.493313,0.557840}%
\pgfsetfillcolor{currentfill}%
\pgfsetfillopacity{0.700000}%
\pgfsetlinewidth{0.501875pt}%
\definecolor{currentstroke}{rgb}{1.000000,1.000000,1.000000}%
\pgfsetstrokecolor{currentstroke}%
\pgfsetstrokeopacity{0.500000}%
\pgfsetdash{}{0pt}%
\pgfpathmoveto{\pgfqpoint{4.175594in}{2.211134in}}%
\pgfpathlineto{\pgfqpoint{4.186721in}{2.214366in}}%
\pgfpathlineto{\pgfqpoint{4.197842in}{2.217567in}}%
\pgfpathlineto{\pgfqpoint{4.208956in}{2.220728in}}%
\pgfpathlineto{\pgfqpoint{4.220062in}{2.223840in}}%
\pgfpathlineto{\pgfqpoint{4.231162in}{2.226894in}}%
\pgfpathlineto{\pgfqpoint{4.224700in}{2.241488in}}%
\pgfpathlineto{\pgfqpoint{4.218242in}{2.256106in}}%
\pgfpathlineto{\pgfqpoint{4.211787in}{2.270738in}}%
\pgfpathlineto{\pgfqpoint{4.205334in}{2.285371in}}%
\pgfpathlineto{\pgfqpoint{4.198884in}{2.299994in}}%
\pgfpathlineto{\pgfqpoint{4.187780in}{2.296748in}}%
\pgfpathlineto{\pgfqpoint{4.176670in}{2.293479in}}%
\pgfpathlineto{\pgfqpoint{4.165554in}{2.290186in}}%
\pgfpathlineto{\pgfqpoint{4.154431in}{2.286869in}}%
\pgfpathlineto{\pgfqpoint{4.143303in}{2.283528in}}%
\pgfpathlineto{\pgfqpoint{4.149757in}{2.269089in}}%
\pgfpathlineto{\pgfqpoint{4.156214in}{2.254634in}}%
\pgfpathlineto{\pgfqpoint{4.162672in}{2.240160in}}%
\pgfpathlineto{\pgfqpoint{4.169132in}{2.225662in}}%
\pgfpathclose%
\pgfusepath{stroke,fill}%
\end{pgfscope}%
\begin{pgfscope}%
\pgfpathrectangle{\pgfqpoint{0.887500in}{0.275000in}}{\pgfqpoint{4.225000in}{4.225000in}}%
\pgfusepath{clip}%
\pgfsetbuttcap%
\pgfsetroundjoin%
\definecolor{currentfill}{rgb}{0.259857,0.745492,0.444467}%
\pgfsetfillcolor{currentfill}%
\pgfsetfillopacity{0.700000}%
\pgfsetlinewidth{0.501875pt}%
\definecolor{currentstroke}{rgb}{1.000000,1.000000,1.000000}%
\pgfsetstrokecolor{currentstroke}%
\pgfsetstrokeopacity{0.500000}%
\pgfsetdash{}{0pt}%
\pgfpathmoveto{\pgfqpoint{3.406090in}{2.766751in}}%
\pgfpathlineto{\pgfqpoint{3.417431in}{2.772017in}}%
\pgfpathlineto{\pgfqpoint{3.428763in}{2.776863in}}%
\pgfpathlineto{\pgfqpoint{3.440088in}{2.781371in}}%
\pgfpathlineto{\pgfqpoint{3.451405in}{2.785620in}}%
\pgfpathlineto{\pgfqpoint{3.462715in}{2.789692in}}%
\pgfpathlineto{\pgfqpoint{3.456380in}{2.803179in}}%
\pgfpathlineto{\pgfqpoint{3.450048in}{2.816595in}}%
\pgfpathlineto{\pgfqpoint{3.443717in}{2.829949in}}%
\pgfpathlineto{\pgfqpoint{3.437389in}{2.843253in}}%
\pgfpathlineto{\pgfqpoint{3.431063in}{2.856515in}}%
\pgfpathlineto{\pgfqpoint{3.419757in}{2.852385in}}%
\pgfpathlineto{\pgfqpoint{3.408444in}{2.848078in}}%
\pgfpathlineto{\pgfqpoint{3.397124in}{2.843508in}}%
\pgfpathlineto{\pgfqpoint{3.385796in}{2.838591in}}%
\pgfpathlineto{\pgfqpoint{3.374460in}{2.833241in}}%
\pgfpathlineto{\pgfqpoint{3.380782in}{2.820189in}}%
\pgfpathlineto{\pgfqpoint{3.387107in}{2.807060in}}%
\pgfpathlineto{\pgfqpoint{3.393434in}{2.793807in}}%
\pgfpathlineto{\pgfqpoint{3.399761in}{2.780386in}}%
\pgfpathclose%
\pgfusepath{stroke,fill}%
\end{pgfscope}%
\begin{pgfscope}%
\pgfpathrectangle{\pgfqpoint{0.887500in}{0.275000in}}{\pgfqpoint{4.225000in}{4.225000in}}%
\pgfusepath{clip}%
\pgfsetbuttcap%
\pgfsetroundjoin%
\definecolor{currentfill}{rgb}{0.120092,0.600104,0.542530}%
\pgfsetfillcolor{currentfill}%
\pgfsetfillopacity{0.700000}%
\pgfsetlinewidth{0.501875pt}%
\definecolor{currentstroke}{rgb}{1.000000,1.000000,1.000000}%
\pgfsetstrokecolor{currentstroke}%
\pgfsetstrokeopacity{0.500000}%
\pgfsetdash{}{0pt}%
\pgfpathmoveto{\pgfqpoint{3.879221in}{2.447646in}}%
\pgfpathlineto{\pgfqpoint{3.890428in}{2.451050in}}%
\pgfpathlineto{\pgfqpoint{3.901629in}{2.454445in}}%
\pgfpathlineto{\pgfqpoint{3.912825in}{2.457835in}}%
\pgfpathlineto{\pgfqpoint{3.924015in}{2.461222in}}%
\pgfpathlineto{\pgfqpoint{3.935199in}{2.464608in}}%
\pgfpathlineto{\pgfqpoint{3.928783in}{2.478781in}}%
\pgfpathlineto{\pgfqpoint{3.922367in}{2.492912in}}%
\pgfpathlineto{\pgfqpoint{3.915954in}{2.507005in}}%
\pgfpathlineto{\pgfqpoint{3.909542in}{2.521065in}}%
\pgfpathlineto{\pgfqpoint{3.903133in}{2.535096in}}%
\pgfpathlineto{\pgfqpoint{3.891950in}{2.531704in}}%
\pgfpathlineto{\pgfqpoint{3.880762in}{2.528305in}}%
\pgfpathlineto{\pgfqpoint{3.869569in}{2.524897in}}%
\pgfpathlineto{\pgfqpoint{3.858369in}{2.521477in}}%
\pgfpathlineto{\pgfqpoint{3.847164in}{2.518041in}}%
\pgfpathlineto{\pgfqpoint{3.853572in}{2.504043in}}%
\pgfpathlineto{\pgfqpoint{3.859982in}{2.490013in}}%
\pgfpathlineto{\pgfqpoint{3.866393in}{2.475943in}}%
\pgfpathlineto{\pgfqpoint{3.872806in}{2.461823in}}%
\pgfpathclose%
\pgfusepath{stroke,fill}%
\end{pgfscope}%
\begin{pgfscope}%
\pgfpathrectangle{\pgfqpoint{0.887500in}{0.275000in}}{\pgfqpoint{4.225000in}{4.225000in}}%
\pgfusepath{clip}%
\pgfsetbuttcap%
\pgfsetroundjoin%
\definecolor{currentfill}{rgb}{0.136408,0.541173,0.554483}%
\pgfsetfillcolor{currentfill}%
\pgfsetfillopacity{0.700000}%
\pgfsetlinewidth{0.501875pt}%
\definecolor{currentstroke}{rgb}{1.000000,1.000000,1.000000}%
\pgfsetstrokecolor{currentstroke}%
\pgfsetstrokeopacity{0.500000}%
\pgfsetdash{}{0pt}%
\pgfpathmoveto{\pgfqpoint{1.634690in}{2.328627in}}%
\pgfpathlineto{\pgfqpoint{1.646455in}{2.332080in}}%
\pgfpathlineto{\pgfqpoint{1.658215in}{2.335521in}}%
\pgfpathlineto{\pgfqpoint{1.669969in}{2.338951in}}%
\pgfpathlineto{\pgfqpoint{1.681717in}{2.342370in}}%
\pgfpathlineto{\pgfqpoint{1.693460in}{2.345781in}}%
\pgfpathlineto{\pgfqpoint{1.687664in}{2.354258in}}%
\pgfpathlineto{\pgfqpoint{1.681872in}{2.362711in}}%
\pgfpathlineto{\pgfqpoint{1.676084in}{2.371138in}}%
\pgfpathlineto{\pgfqpoint{1.670301in}{2.379540in}}%
\pgfpathlineto{\pgfqpoint{1.664522in}{2.387916in}}%
\pgfpathlineto{\pgfqpoint{1.652791in}{2.384532in}}%
\pgfpathlineto{\pgfqpoint{1.641055in}{2.381139in}}%
\pgfpathlineto{\pgfqpoint{1.629313in}{2.377736in}}%
\pgfpathlineto{\pgfqpoint{1.617566in}{2.374321in}}%
\pgfpathlineto{\pgfqpoint{1.605813in}{2.370894in}}%
\pgfpathlineto{\pgfqpoint{1.611579in}{2.362494in}}%
\pgfpathlineto{\pgfqpoint{1.617350in}{2.354068in}}%
\pgfpathlineto{\pgfqpoint{1.623125in}{2.345614in}}%
\pgfpathlineto{\pgfqpoint{1.628905in}{2.337134in}}%
\pgfpathclose%
\pgfusepath{stroke,fill}%
\end{pgfscope}%
\begin{pgfscope}%
\pgfpathrectangle{\pgfqpoint{0.887500in}{0.275000in}}{\pgfqpoint{4.225000in}{4.225000in}}%
\pgfusepath{clip}%
\pgfsetbuttcap%
\pgfsetroundjoin%
\definecolor{currentfill}{rgb}{0.159194,0.482237,0.558073}%
\pgfsetfillcolor{currentfill}%
\pgfsetfillopacity{0.700000}%
\pgfsetlinewidth{0.501875pt}%
\definecolor{currentstroke}{rgb}{1.000000,1.000000,1.000000}%
\pgfsetstrokecolor{currentstroke}%
\pgfsetstrokeopacity{0.500000}%
\pgfsetdash{}{0pt}%
\pgfpathmoveto{\pgfqpoint{2.279237in}{2.205052in}}%
\pgfpathlineto{\pgfqpoint{2.290842in}{2.208589in}}%
\pgfpathlineto{\pgfqpoint{2.302441in}{2.212114in}}%
\pgfpathlineto{\pgfqpoint{2.314035in}{2.215625in}}%
\pgfpathlineto{\pgfqpoint{2.325624in}{2.219121in}}%
\pgfpathlineto{\pgfqpoint{2.337207in}{2.222601in}}%
\pgfpathlineto{\pgfqpoint{2.331189in}{2.231478in}}%
\pgfpathlineto{\pgfqpoint{2.325175in}{2.240329in}}%
\pgfpathlineto{\pgfqpoint{2.319165in}{2.249154in}}%
\pgfpathlineto{\pgfqpoint{2.313159in}{2.257954in}}%
\pgfpathlineto{\pgfqpoint{2.307158in}{2.266730in}}%
\pgfpathlineto{\pgfqpoint{2.295586in}{2.263294in}}%
\pgfpathlineto{\pgfqpoint{2.284008in}{2.259843in}}%
\pgfpathlineto{\pgfqpoint{2.272425in}{2.256377in}}%
\pgfpathlineto{\pgfqpoint{2.260837in}{2.252897in}}%
\pgfpathlineto{\pgfqpoint{2.249242in}{2.249405in}}%
\pgfpathlineto{\pgfqpoint{2.255233in}{2.240587in}}%
\pgfpathlineto{\pgfqpoint{2.261227in}{2.231744in}}%
\pgfpathlineto{\pgfqpoint{2.267226in}{2.222873in}}%
\pgfpathlineto{\pgfqpoint{2.273229in}{2.213976in}}%
\pgfpathclose%
\pgfusepath{stroke,fill}%
\end{pgfscope}%
\begin{pgfscope}%
\pgfpathrectangle{\pgfqpoint{0.887500in}{0.275000in}}{\pgfqpoint{4.225000in}{4.225000in}}%
\pgfusepath{clip}%
\pgfsetbuttcap%
\pgfsetroundjoin%
\definecolor{currentfill}{rgb}{0.147607,0.511733,0.557049}%
\pgfsetfillcolor{currentfill}%
\pgfsetfillopacity{0.700000}%
\pgfsetlinewidth{0.501875pt}%
\definecolor{currentstroke}{rgb}{1.000000,1.000000,1.000000}%
\pgfsetstrokecolor{currentstroke}%
\pgfsetstrokeopacity{0.500000}%
\pgfsetdash{}{0pt}%
\pgfpathmoveto{\pgfqpoint{1.956928in}{2.268119in}}%
\pgfpathlineto{\pgfqpoint{1.968614in}{2.271578in}}%
\pgfpathlineto{\pgfqpoint{1.980295in}{2.275024in}}%
\pgfpathlineto{\pgfqpoint{1.991970in}{2.278456in}}%
\pgfpathlineto{\pgfqpoint{2.003640in}{2.281877in}}%
\pgfpathlineto{\pgfqpoint{2.015304in}{2.285286in}}%
\pgfpathlineto{\pgfqpoint{2.009395in}{2.293969in}}%
\pgfpathlineto{\pgfqpoint{2.003490in}{2.302628in}}%
\pgfpathlineto{\pgfqpoint{1.997590in}{2.311263in}}%
\pgfpathlineto{\pgfqpoint{1.991693in}{2.319875in}}%
\pgfpathlineto{\pgfqpoint{1.985802in}{2.328465in}}%
\pgfpathlineto{\pgfqpoint{1.974149in}{2.325092in}}%
\pgfpathlineto{\pgfqpoint{1.962490in}{2.321709in}}%
\pgfpathlineto{\pgfqpoint{1.950826in}{2.318315in}}%
\pgfpathlineto{\pgfqpoint{1.939157in}{2.314908in}}%
\pgfpathlineto{\pgfqpoint{1.927482in}{2.311489in}}%
\pgfpathlineto{\pgfqpoint{1.933362in}{2.302865in}}%
\pgfpathlineto{\pgfqpoint{1.939247in}{2.294217in}}%
\pgfpathlineto{\pgfqpoint{1.945136in}{2.285543in}}%
\pgfpathlineto{\pgfqpoint{1.951030in}{2.276844in}}%
\pgfpathclose%
\pgfusepath{stroke,fill}%
\end{pgfscope}%
\begin{pgfscope}%
\pgfpathrectangle{\pgfqpoint{0.887500in}{0.275000in}}{\pgfqpoint{4.225000in}{4.225000in}}%
\pgfusepath{clip}%
\pgfsetbuttcap%
\pgfsetroundjoin%
\definecolor{currentfill}{rgb}{0.203063,0.379716,0.553925}%
\pgfsetfillcolor{currentfill}%
\pgfsetfillopacity{0.700000}%
\pgfsetlinewidth{0.501875pt}%
\definecolor{currentstroke}{rgb}{1.000000,1.000000,1.000000}%
\pgfsetstrokecolor{currentstroke}%
\pgfsetstrokeopacity{0.500000}%
\pgfsetdash{}{0pt}%
\pgfpathmoveto{\pgfqpoint{4.472129in}{1.979297in}}%
\pgfpathlineto{\pgfqpoint{4.483183in}{1.982600in}}%
\pgfpathlineto{\pgfqpoint{4.494230in}{1.985875in}}%
\pgfpathlineto{\pgfqpoint{4.505271in}{1.989130in}}%
\pgfpathlineto{\pgfqpoint{4.516306in}{1.992373in}}%
\pgfpathlineto{\pgfqpoint{4.527335in}{1.995612in}}%
\pgfpathlineto{\pgfqpoint{4.520815in}{2.010140in}}%
\pgfpathlineto{\pgfqpoint{4.514297in}{2.024613in}}%
\pgfpathlineto{\pgfqpoint{4.507779in}{2.039028in}}%
\pgfpathlineto{\pgfqpoint{4.501260in}{2.053375in}}%
\pgfpathlineto{\pgfqpoint{4.494742in}{2.067651in}}%
\pgfpathlineto{\pgfqpoint{4.483710in}{2.064291in}}%
\pgfpathlineto{\pgfqpoint{4.472672in}{2.060921in}}%
\pgfpathlineto{\pgfqpoint{4.461627in}{2.057535in}}%
\pgfpathlineto{\pgfqpoint{4.450577in}{2.054130in}}%
\pgfpathlineto{\pgfqpoint{4.439520in}{2.050703in}}%
\pgfpathlineto{\pgfqpoint{4.446046in}{2.036683in}}%
\pgfpathlineto{\pgfqpoint{4.452569in}{2.022517in}}%
\pgfpathlineto{\pgfqpoint{4.459091in}{2.008222in}}%
\pgfpathlineto{\pgfqpoint{4.465610in}{1.993810in}}%
\pgfpathclose%
\pgfusepath{stroke,fill}%
\end{pgfscope}%
\begin{pgfscope}%
\pgfpathrectangle{\pgfqpoint{0.887500in}{0.275000in}}{\pgfqpoint{4.225000in}{4.225000in}}%
\pgfusepath{clip}%
\pgfsetbuttcap%
\pgfsetroundjoin%
\definecolor{currentfill}{rgb}{0.144759,0.519093,0.556572}%
\pgfsetfillcolor{currentfill}%
\pgfsetfillopacity{0.700000}%
\pgfsetlinewidth{0.501875pt}%
\definecolor{currentstroke}{rgb}{1.000000,1.000000,1.000000}%
\pgfsetstrokecolor{currentstroke}%
\pgfsetstrokeopacity{0.500000}%
\pgfsetdash{}{0pt}%
\pgfpathmoveto{\pgfqpoint{3.094983in}{2.225828in}}%
\pgfpathlineto{\pgfqpoint{3.106400in}{2.251218in}}%
\pgfpathlineto{\pgfqpoint{3.117822in}{2.275375in}}%
\pgfpathlineto{\pgfqpoint{3.129248in}{2.298293in}}%
\pgfpathlineto{\pgfqpoint{3.140677in}{2.320204in}}%
\pgfpathlineto{\pgfqpoint{3.152110in}{2.341338in}}%
\pgfpathlineto{\pgfqpoint{3.145827in}{2.353547in}}%
\pgfpathlineto{\pgfqpoint{3.139548in}{2.365751in}}%
\pgfpathlineto{\pgfqpoint{3.133271in}{2.378011in}}%
\pgfpathlineto{\pgfqpoint{3.126998in}{2.390388in}}%
\pgfpathlineto{\pgfqpoint{3.120728in}{2.402907in}}%
\pgfpathlineto{\pgfqpoint{3.109312in}{2.380047in}}%
\pgfpathlineto{\pgfqpoint{3.097902in}{2.356524in}}%
\pgfpathlineto{\pgfqpoint{3.086497in}{2.332065in}}%
\pgfpathlineto{\pgfqpoint{3.075098in}{2.306400in}}%
\pgfpathlineto{\pgfqpoint{3.063706in}{2.279511in}}%
\pgfpathlineto{\pgfqpoint{3.069954in}{2.268396in}}%
\pgfpathlineto{\pgfqpoint{3.076206in}{2.257515in}}%
\pgfpathlineto{\pgfqpoint{3.082461in}{2.246840in}}%
\pgfpathlineto{\pgfqpoint{3.088720in}{2.236301in}}%
\pgfpathclose%
\pgfusepath{stroke,fill}%
\end{pgfscope}%
\begin{pgfscope}%
\pgfpathrectangle{\pgfqpoint{0.887500in}{0.275000in}}{\pgfqpoint{4.225000in}{4.225000in}}%
\pgfusepath{clip}%
\pgfsetbuttcap%
\pgfsetroundjoin%
\definecolor{currentfill}{rgb}{0.177423,0.437527,0.557565}%
\pgfsetfillcolor{currentfill}%
\pgfsetfillopacity{0.700000}%
\pgfsetlinewidth{0.501875pt}%
\definecolor{currentstroke}{rgb}{1.000000,1.000000,1.000000}%
\pgfsetstrokecolor{currentstroke}%
\pgfsetstrokeopacity{0.500000}%
\pgfsetdash{}{0pt}%
\pgfpathmoveto{\pgfqpoint{3.069246in}{2.061584in}}%
\pgfpathlineto{\pgfqpoint{3.080653in}{2.079436in}}%
\pgfpathlineto{\pgfqpoint{3.092066in}{2.100306in}}%
\pgfpathlineto{\pgfqpoint{3.103486in}{2.123292in}}%
\pgfpathlineto{\pgfqpoint{3.114913in}{2.147491in}}%
\pgfpathlineto{\pgfqpoint{3.126347in}{2.171995in}}%
\pgfpathlineto{\pgfqpoint{3.120068in}{2.183191in}}%
\pgfpathlineto{\pgfqpoint{3.113791in}{2.194103in}}%
\pgfpathlineto{\pgfqpoint{3.107519in}{2.204799in}}%
\pgfpathlineto{\pgfqpoint{3.101249in}{2.215351in}}%
\pgfpathlineto{\pgfqpoint{3.094983in}{2.225828in}}%
\pgfpathlineto{\pgfqpoint{3.083573in}{2.199977in}}%
\pgfpathlineto{\pgfqpoint{3.072170in}{2.174586in}}%
\pgfpathlineto{\pgfqpoint{3.060775in}{2.150574in}}%
\pgfpathlineto{\pgfqpoint{3.049386in}{2.128859in}}%
\pgfpathlineto{\pgfqpoint{3.038000in}{2.110352in}}%
\pgfpathlineto{\pgfqpoint{3.044241in}{2.100707in}}%
\pgfpathlineto{\pgfqpoint{3.050487in}{2.091013in}}%
\pgfpathlineto{\pgfqpoint{3.056736in}{2.081264in}}%
\pgfpathlineto{\pgfqpoint{3.062989in}{2.071456in}}%
\pgfpathclose%
\pgfusepath{stroke,fill}%
\end{pgfscope}%
\begin{pgfscope}%
\pgfpathrectangle{\pgfqpoint{0.887500in}{0.275000in}}{\pgfqpoint{4.225000in}{4.225000in}}%
\pgfusepath{clip}%
\pgfsetbuttcap%
\pgfsetroundjoin%
\definecolor{currentfill}{rgb}{0.119738,0.603785,0.541400}%
\pgfsetfillcolor{currentfill}%
\pgfsetfillopacity{0.700000}%
\pgfsetlinewidth{0.501875pt}%
\definecolor{currentstroke}{rgb}{1.000000,1.000000,1.000000}%
\pgfsetstrokecolor{currentstroke}%
\pgfsetstrokeopacity{0.500000}%
\pgfsetdash{}{0pt}%
\pgfpathmoveto{\pgfqpoint{3.120728in}{2.402907in}}%
\pgfpathlineto{\pgfqpoint{3.132148in}{2.425376in}}%
\pgfpathlineto{\pgfqpoint{3.143573in}{2.447729in}}%
\pgfpathlineto{\pgfqpoint{3.155003in}{2.470241in}}%
\pgfpathlineto{\pgfqpoint{3.166440in}{2.493063in}}%
\pgfpathlineto{\pgfqpoint{3.177883in}{2.515936in}}%
\pgfpathlineto{\pgfqpoint{3.171596in}{2.528634in}}%
\pgfpathlineto{\pgfqpoint{3.165310in}{2.540806in}}%
\pgfpathlineto{\pgfqpoint{3.159026in}{2.552551in}}%
\pgfpathlineto{\pgfqpoint{3.152744in}{2.563970in}}%
\pgfpathlineto{\pgfqpoint{3.146465in}{2.575163in}}%
\pgfpathlineto{\pgfqpoint{3.135047in}{2.553555in}}%
\pgfpathlineto{\pgfqpoint{3.123633in}{2.531773in}}%
\pgfpathlineto{\pgfqpoint{3.112224in}{2.509947in}}%
\pgfpathlineto{\pgfqpoint{3.100820in}{2.487901in}}%
\pgfpathlineto{\pgfqpoint{3.089420in}{2.465366in}}%
\pgfpathlineto{\pgfqpoint{3.095676in}{2.453121in}}%
\pgfpathlineto{\pgfqpoint{3.101935in}{2.440684in}}%
\pgfpathlineto{\pgfqpoint{3.108196in}{2.428123in}}%
\pgfpathlineto{\pgfqpoint{3.114460in}{2.415508in}}%
\pgfpathclose%
\pgfusepath{stroke,fill}%
\end{pgfscope}%
\begin{pgfscope}%
\pgfpathrectangle{\pgfqpoint{0.887500in}{0.275000in}}{\pgfqpoint{4.225000in}{4.225000in}}%
\pgfusepath{clip}%
\pgfsetbuttcap%
\pgfsetroundjoin%
\definecolor{currentfill}{rgb}{0.121380,0.629492,0.531973}%
\pgfsetfillcolor{currentfill}%
\pgfsetfillopacity{0.700000}%
\pgfsetlinewidth{0.501875pt}%
\definecolor{currentstroke}{rgb}{1.000000,1.000000,1.000000}%
\pgfsetstrokecolor{currentstroke}%
\pgfsetstrokeopacity{0.500000}%
\pgfsetdash{}{0pt}%
\pgfpathmoveto{\pgfqpoint{3.791045in}{2.500286in}}%
\pgfpathlineto{\pgfqpoint{3.802282in}{2.503948in}}%
\pgfpathlineto{\pgfqpoint{3.813512in}{2.507544in}}%
\pgfpathlineto{\pgfqpoint{3.824736in}{2.511085in}}%
\pgfpathlineto{\pgfqpoint{3.835953in}{2.514581in}}%
\pgfpathlineto{\pgfqpoint{3.847164in}{2.518041in}}%
\pgfpathlineto{\pgfqpoint{3.840758in}{2.532017in}}%
\pgfpathlineto{\pgfqpoint{3.834355in}{2.545979in}}%
\pgfpathlineto{\pgfqpoint{3.827954in}{2.559936in}}%
\pgfpathlineto{\pgfqpoint{3.821556in}{2.573894in}}%
\pgfpathlineto{\pgfqpoint{3.815160in}{2.587851in}}%
\pgfpathlineto{\pgfqpoint{3.803951in}{2.584382in}}%
\pgfpathlineto{\pgfqpoint{3.792737in}{2.580882in}}%
\pgfpathlineto{\pgfqpoint{3.781516in}{2.577345in}}%
\pgfpathlineto{\pgfqpoint{3.770289in}{2.573761in}}%
\pgfpathlineto{\pgfqpoint{3.759055in}{2.570122in}}%
\pgfpathlineto{\pgfqpoint{3.765447in}{2.556113in}}%
\pgfpathlineto{\pgfqpoint{3.771842in}{2.542127in}}%
\pgfpathlineto{\pgfqpoint{3.778241in}{2.528169in}}%
\pgfpathlineto{\pgfqpoint{3.784642in}{2.514228in}}%
\pgfpathclose%
\pgfusepath{stroke,fill}%
\end{pgfscope}%
\begin{pgfscope}%
\pgfpathrectangle{\pgfqpoint{0.887500in}{0.275000in}}{\pgfqpoint{4.225000in}{4.225000in}}%
\pgfusepath{clip}%
\pgfsetbuttcap%
\pgfsetroundjoin%
\definecolor{currentfill}{rgb}{0.246811,0.283237,0.535941}%
\pgfsetfillcolor{currentfill}%
\pgfsetfillopacity{0.700000}%
\pgfsetlinewidth{0.501875pt}%
\definecolor{currentstroke}{rgb}{1.000000,1.000000,1.000000}%
\pgfsetstrokecolor{currentstroke}%
\pgfsetstrokeopacity{0.500000}%
\pgfsetdash{}{0pt}%
\pgfpathmoveto{\pgfqpoint{4.680426in}{1.791855in}}%
\pgfpathlineto{\pgfqpoint{4.691426in}{1.795125in}}%
\pgfpathlineto{\pgfqpoint{4.702420in}{1.798401in}}%
\pgfpathlineto{\pgfqpoint{4.713408in}{1.801682in}}%
\pgfpathlineto{\pgfqpoint{4.724392in}{1.804971in}}%
\pgfpathlineto{\pgfqpoint{4.735370in}{1.808268in}}%
\pgfpathlineto{\pgfqpoint{4.728827in}{1.823165in}}%
\pgfpathlineto{\pgfqpoint{4.722283in}{1.838018in}}%
\pgfpathlineto{\pgfqpoint{4.715741in}{1.852829in}}%
\pgfpathlineto{\pgfqpoint{4.709199in}{1.867602in}}%
\pgfpathlineto{\pgfqpoint{4.702659in}{1.882339in}}%
\pgfpathlineto{\pgfqpoint{4.691679in}{1.879006in}}%
\pgfpathlineto{\pgfqpoint{4.680694in}{1.875673in}}%
\pgfpathlineto{\pgfqpoint{4.669704in}{1.872339in}}%
\pgfpathlineto{\pgfqpoint{4.658708in}{1.869001in}}%
\pgfpathlineto{\pgfqpoint{4.647706in}{1.865658in}}%
\pgfpathlineto{\pgfqpoint{4.654246in}{1.850935in}}%
\pgfpathlineto{\pgfqpoint{4.660789in}{1.836194in}}%
\pgfpathlineto{\pgfqpoint{4.667333in}{1.821433in}}%
\pgfpathlineto{\pgfqpoint{4.673879in}{1.806654in}}%
\pgfpathclose%
\pgfusepath{stroke,fill}%
\end{pgfscope}%
\begin{pgfscope}%
\pgfpathrectangle{\pgfqpoint{0.887500in}{0.275000in}}{\pgfqpoint{4.225000in}{4.225000in}}%
\pgfusepath{clip}%
\pgfsetbuttcap%
\pgfsetroundjoin%
\definecolor{currentfill}{rgb}{0.175841,0.441290,0.557685}%
\pgfsetfillcolor{currentfill}%
\pgfsetfillopacity{0.700000}%
\pgfsetlinewidth{0.501875pt}%
\definecolor{currentstroke}{rgb}{1.000000,1.000000,1.000000}%
\pgfsetstrokecolor{currentstroke}%
\pgfsetstrokeopacity{0.500000}%
\pgfsetdash{}{0pt}%
\pgfpathmoveto{\pgfqpoint{2.689813in}{2.110479in}}%
\pgfpathlineto{\pgfqpoint{2.701320in}{2.114030in}}%
\pgfpathlineto{\pgfqpoint{2.712822in}{2.117493in}}%
\pgfpathlineto{\pgfqpoint{2.724320in}{2.120845in}}%
\pgfpathlineto{\pgfqpoint{2.735813in}{2.124058in}}%
\pgfpathlineto{\pgfqpoint{2.747301in}{2.127154in}}%
\pgfpathlineto{\pgfqpoint{2.741144in}{2.136460in}}%
\pgfpathlineto{\pgfqpoint{2.734992in}{2.145732in}}%
\pgfpathlineto{\pgfqpoint{2.728844in}{2.154971in}}%
\pgfpathlineto{\pgfqpoint{2.722701in}{2.164178in}}%
\pgfpathlineto{\pgfqpoint{2.716561in}{2.173355in}}%
\pgfpathlineto{\pgfqpoint{2.705083in}{2.170305in}}%
\pgfpathlineto{\pgfqpoint{2.693600in}{2.167133in}}%
\pgfpathlineto{\pgfqpoint{2.682113in}{2.163816in}}%
\pgfpathlineto{\pgfqpoint{2.670621in}{2.160383in}}%
\pgfpathlineto{\pgfqpoint{2.659124in}{2.156861in}}%
\pgfpathlineto{\pgfqpoint{2.665253in}{2.147655in}}%
\pgfpathlineto{\pgfqpoint{2.671387in}{2.138415in}}%
\pgfpathlineto{\pgfqpoint{2.677525in}{2.129140in}}%
\pgfpathlineto{\pgfqpoint{2.683667in}{2.119828in}}%
\pgfpathclose%
\pgfusepath{stroke,fill}%
\end{pgfscope}%
\begin{pgfscope}%
\pgfpathrectangle{\pgfqpoint{0.887500in}{0.275000in}}{\pgfqpoint{4.225000in}{4.225000in}}%
\pgfusepath{clip}%
\pgfsetbuttcap%
\pgfsetroundjoin%
\definecolor{currentfill}{rgb}{0.144759,0.519093,0.556572}%
\pgfsetfillcolor{currentfill}%
\pgfsetfillopacity{0.700000}%
\pgfsetlinewidth{0.501875pt}%
\definecolor{currentstroke}{rgb}{1.000000,1.000000,1.000000}%
\pgfsetstrokecolor{currentstroke}%
\pgfsetstrokeopacity{0.500000}%
\pgfsetdash{}{0pt}%
\pgfpathmoveto{\pgfqpoint{4.087570in}{2.266546in}}%
\pgfpathlineto{\pgfqpoint{4.098728in}{2.269961in}}%
\pgfpathlineto{\pgfqpoint{4.109880in}{2.273372in}}%
\pgfpathlineto{\pgfqpoint{4.121027in}{2.276775in}}%
\pgfpathlineto{\pgfqpoint{4.132168in}{2.280162in}}%
\pgfpathlineto{\pgfqpoint{4.143303in}{2.283528in}}%
\pgfpathlineto{\pgfqpoint{4.136850in}{2.297957in}}%
\pgfpathlineto{\pgfqpoint{4.130401in}{2.312382in}}%
\pgfpathlineto{\pgfqpoint{4.123953in}{2.326804in}}%
\pgfpathlineto{\pgfqpoint{4.117508in}{2.341222in}}%
\pgfpathlineto{\pgfqpoint{4.111066in}{2.355632in}}%
\pgfpathlineto{\pgfqpoint{4.099930in}{2.352177in}}%
\pgfpathlineto{\pgfqpoint{4.088789in}{2.348700in}}%
\pgfpathlineto{\pgfqpoint{4.077641in}{2.345207in}}%
\pgfpathlineto{\pgfqpoint{4.066488in}{2.341705in}}%
\pgfpathlineto{\pgfqpoint{4.055330in}{2.338199in}}%
\pgfpathlineto{\pgfqpoint{4.061771in}{2.323810in}}%
\pgfpathlineto{\pgfqpoint{4.068216in}{2.309445in}}%
\pgfpathlineto{\pgfqpoint{4.074664in}{2.295113in}}%
\pgfpathlineto{\pgfqpoint{4.081115in}{2.280819in}}%
\pgfpathclose%
\pgfusepath{stroke,fill}%
\end{pgfscope}%
\begin{pgfscope}%
\pgfpathrectangle{\pgfqpoint{0.887500in}{0.275000in}}{\pgfqpoint{4.225000in}{4.225000in}}%
\pgfusepath{clip}%
\pgfsetbuttcap%
\pgfsetroundjoin%
\definecolor{currentfill}{rgb}{0.190631,0.407061,0.556089}%
\pgfsetfillcolor{currentfill}%
\pgfsetfillopacity{0.700000}%
\pgfsetlinewidth{0.501875pt}%
\definecolor{currentstroke}{rgb}{1.000000,1.000000,1.000000}%
\pgfsetstrokecolor{currentstroke}%
\pgfsetstrokeopacity{0.500000}%
\pgfsetdash{}{0pt}%
\pgfpathmoveto{\pgfqpoint{4.384139in}{2.033067in}}%
\pgfpathlineto{\pgfqpoint{4.395229in}{2.036684in}}%
\pgfpathlineto{\pgfqpoint{4.406312in}{2.040247in}}%
\pgfpathlineto{\pgfqpoint{4.417388in}{2.043766in}}%
\pgfpathlineto{\pgfqpoint{4.428458in}{2.047249in}}%
\pgfpathlineto{\pgfqpoint{4.439520in}{2.050703in}}%
\pgfpathlineto{\pgfqpoint{4.432992in}{2.064564in}}%
\pgfpathlineto{\pgfqpoint{4.426460in}{2.078261in}}%
\pgfpathlineto{\pgfqpoint{4.419927in}{2.091826in}}%
\pgfpathlineto{\pgfqpoint{4.413394in}{2.105299in}}%
\pgfpathlineto{\pgfqpoint{4.406861in}{2.118717in}}%
\pgfpathlineto{\pgfqpoint{4.395792in}{2.115019in}}%
\pgfpathlineto{\pgfqpoint{4.384722in}{2.111438in}}%
\pgfpathlineto{\pgfqpoint{4.373651in}{2.107973in}}%
\pgfpathlineto{\pgfqpoint{4.362577in}{2.104589in}}%
\pgfpathlineto{\pgfqpoint{4.351498in}{2.101250in}}%
\pgfpathlineto{\pgfqpoint{4.358025in}{2.087751in}}%
\pgfpathlineto{\pgfqpoint{4.364554in}{2.074218in}}%
\pgfpathlineto{\pgfqpoint{4.371083in}{2.060614in}}%
\pgfpathlineto{\pgfqpoint{4.377612in}{2.046907in}}%
\pgfpathclose%
\pgfusepath{stroke,fill}%
\end{pgfscope}%
\begin{pgfscope}%
\pgfpathrectangle{\pgfqpoint{0.887500in}{0.275000in}}{\pgfqpoint{4.225000in}{4.225000in}}%
\pgfusepath{clip}%
\pgfsetbuttcap%
\pgfsetroundjoin%
\definecolor{currentfill}{rgb}{0.259857,0.745492,0.444467}%
\pgfsetfillcolor{currentfill}%
\pgfsetfillopacity{0.700000}%
\pgfsetlinewidth{0.501875pt}%
\definecolor{currentstroke}{rgb}{1.000000,1.000000,1.000000}%
\pgfsetstrokecolor{currentstroke}%
\pgfsetstrokeopacity{0.500000}%
\pgfsetdash{}{0pt}%
\pgfpathmoveto{\pgfqpoint{3.260683in}{2.746005in}}%
\pgfpathlineto{\pgfqpoint{3.272088in}{2.757662in}}%
\pgfpathlineto{\pgfqpoint{3.283488in}{2.768618in}}%
\pgfpathlineto{\pgfqpoint{3.294883in}{2.778932in}}%
\pgfpathlineto{\pgfqpoint{3.306272in}{2.788608in}}%
\pgfpathlineto{\pgfqpoint{3.317655in}{2.797644in}}%
\pgfpathlineto{\pgfqpoint{3.311340in}{2.810285in}}%
\pgfpathlineto{\pgfqpoint{3.305028in}{2.822976in}}%
\pgfpathlineto{\pgfqpoint{3.298720in}{2.835805in}}%
\pgfpathlineto{\pgfqpoint{3.292416in}{2.848858in}}%
\pgfpathlineto{\pgfqpoint{3.286116in}{2.862211in}}%
\pgfpathlineto{\pgfqpoint{3.274740in}{2.852707in}}%
\pgfpathlineto{\pgfqpoint{3.263357in}{2.842434in}}%
\pgfpathlineto{\pgfqpoint{3.251969in}{2.831371in}}%
\pgfpathlineto{\pgfqpoint{3.240576in}{2.819498in}}%
\pgfpathlineto{\pgfqpoint{3.229179in}{2.806790in}}%
\pgfpathlineto{\pgfqpoint{3.235468in}{2.793483in}}%
\pgfpathlineto{\pgfqpoint{3.241764in}{2.780927in}}%
\pgfpathlineto{\pgfqpoint{3.248066in}{2.768955in}}%
\pgfpathlineto{\pgfqpoint{3.254372in}{2.757378in}}%
\pgfpathclose%
\pgfusepath{stroke,fill}%
\end{pgfscope}%
\begin{pgfscope}%
\pgfpathrectangle{\pgfqpoint{0.887500in}{0.275000in}}{\pgfqpoint{4.225000in}{4.225000in}}%
\pgfusepath{clip}%
\pgfsetbuttcap%
\pgfsetroundjoin%
\definecolor{currentfill}{rgb}{0.208030,0.718701,0.472873}%
\pgfsetfillcolor{currentfill}%
\pgfsetfillopacity{0.700000}%
\pgfsetlinewidth{0.501875pt}%
\definecolor{currentstroke}{rgb}{1.000000,1.000000,1.000000}%
\pgfsetstrokecolor{currentstroke}%
\pgfsetstrokeopacity{0.500000}%
\pgfsetdash{}{0pt}%
\pgfpathmoveto{\pgfqpoint{3.203595in}{2.673857in}}%
\pgfpathlineto{\pgfqpoint{3.215020in}{2.690454in}}%
\pgfpathlineto{\pgfqpoint{3.226442in}{2.705872in}}%
\pgfpathlineto{\pgfqpoint{3.237860in}{2.720206in}}%
\pgfpathlineto{\pgfqpoint{3.249274in}{2.733552in}}%
\pgfpathlineto{\pgfqpoint{3.260683in}{2.746005in}}%
\pgfpathlineto{\pgfqpoint{3.254372in}{2.757378in}}%
\pgfpathlineto{\pgfqpoint{3.248066in}{2.768955in}}%
\pgfpathlineto{\pgfqpoint{3.241764in}{2.780927in}}%
\pgfpathlineto{\pgfqpoint{3.235468in}{2.793483in}}%
\pgfpathlineto{\pgfqpoint{3.229179in}{2.806790in}}%
\pgfpathlineto{\pgfqpoint{3.217778in}{2.793225in}}%
\pgfpathlineto{\pgfqpoint{3.206374in}{2.778776in}}%
\pgfpathlineto{\pgfqpoint{3.194967in}{2.763420in}}%
\pgfpathlineto{\pgfqpoint{3.183559in}{2.747133in}}%
\pgfpathlineto{\pgfqpoint{3.172150in}{2.729889in}}%
\pgfpathlineto{\pgfqpoint{3.178427in}{2.716946in}}%
\pgfpathlineto{\pgfqpoint{3.184710in}{2.705129in}}%
\pgfpathlineto{\pgfqpoint{3.191000in}{2.694192in}}%
\pgfpathlineto{\pgfqpoint{3.197295in}{2.683860in}}%
\pgfpathclose%
\pgfusepath{stroke,fill}%
\end{pgfscope}%
\begin{pgfscope}%
\pgfpathrectangle{\pgfqpoint{0.887500in}{0.275000in}}{\pgfqpoint{4.225000in}{4.225000in}}%
\pgfusepath{clip}%
\pgfsetbuttcap%
\pgfsetroundjoin%
\definecolor{currentfill}{rgb}{0.163625,0.471133,0.558148}%
\pgfsetfillcolor{currentfill}%
\pgfsetfillopacity{0.700000}%
\pgfsetlinewidth{0.501875pt}%
\definecolor{currentstroke}{rgb}{1.000000,1.000000,1.000000}%
\pgfsetstrokecolor{currentstroke}%
\pgfsetstrokeopacity{0.500000}%
\pgfsetdash{}{0pt}%
\pgfpathmoveto{\pgfqpoint{2.367364in}{2.177805in}}%
\pgfpathlineto{\pgfqpoint{2.378952in}{2.181314in}}%
\pgfpathlineto{\pgfqpoint{2.390535in}{2.184804in}}%
\pgfpathlineto{\pgfqpoint{2.402113in}{2.188271in}}%
\pgfpathlineto{\pgfqpoint{2.413685in}{2.191717in}}%
\pgfpathlineto{\pgfqpoint{2.425252in}{2.195143in}}%
\pgfpathlineto{\pgfqpoint{2.419202in}{2.204111in}}%
\pgfpathlineto{\pgfqpoint{2.413155in}{2.213051in}}%
\pgfpathlineto{\pgfqpoint{2.407113in}{2.221964in}}%
\pgfpathlineto{\pgfqpoint{2.401076in}{2.230850in}}%
\pgfpathlineto{\pgfqpoint{2.395042in}{2.239709in}}%
\pgfpathlineto{\pgfqpoint{2.383486in}{2.236326in}}%
\pgfpathlineto{\pgfqpoint{2.371924in}{2.232926in}}%
\pgfpathlineto{\pgfqpoint{2.360357in}{2.229504in}}%
\pgfpathlineto{\pgfqpoint{2.348785in}{2.226062in}}%
\pgfpathlineto{\pgfqpoint{2.337207in}{2.222601in}}%
\pgfpathlineto{\pgfqpoint{2.343230in}{2.213697in}}%
\pgfpathlineto{\pgfqpoint{2.349257in}{2.204766in}}%
\pgfpathlineto{\pgfqpoint{2.355288in}{2.195807in}}%
\pgfpathlineto{\pgfqpoint{2.361324in}{2.186820in}}%
\pgfpathclose%
\pgfusepath{stroke,fill}%
\end{pgfscope}%
\begin{pgfscope}%
\pgfpathrectangle{\pgfqpoint{0.887500in}{0.275000in}}{\pgfqpoint{4.225000in}{4.225000in}}%
\pgfusepath{clip}%
\pgfsetbuttcap%
\pgfsetroundjoin%
\definecolor{currentfill}{rgb}{0.140536,0.530132,0.555659}%
\pgfsetfillcolor{currentfill}%
\pgfsetfillopacity{0.700000}%
\pgfsetlinewidth{0.501875pt}%
\definecolor{currentstroke}{rgb}{1.000000,1.000000,1.000000}%
\pgfsetstrokecolor{currentstroke}%
\pgfsetstrokeopacity{0.500000}%
\pgfsetdash{}{0pt}%
\pgfpathmoveto{\pgfqpoint{1.722512in}{2.303017in}}%
\pgfpathlineto{\pgfqpoint{1.734261in}{2.306455in}}%
\pgfpathlineto{\pgfqpoint{1.746005in}{2.309887in}}%
\pgfpathlineto{\pgfqpoint{1.757743in}{2.313312in}}%
\pgfpathlineto{\pgfqpoint{1.769475in}{2.316734in}}%
\pgfpathlineto{\pgfqpoint{1.781201in}{2.320151in}}%
\pgfpathlineto{\pgfqpoint{1.775370in}{2.328716in}}%
\pgfpathlineto{\pgfqpoint{1.769544in}{2.337257in}}%
\pgfpathlineto{\pgfqpoint{1.763722in}{2.345776in}}%
\pgfpathlineto{\pgfqpoint{1.757904in}{2.354270in}}%
\pgfpathlineto{\pgfqpoint{1.752091in}{2.362741in}}%
\pgfpathlineto{\pgfqpoint{1.740376in}{2.359358in}}%
\pgfpathlineto{\pgfqpoint{1.728656in}{2.355972in}}%
\pgfpathlineto{\pgfqpoint{1.716930in}{2.352581in}}%
\pgfpathlineto{\pgfqpoint{1.705198in}{2.349184in}}%
\pgfpathlineto{\pgfqpoint{1.693460in}{2.345781in}}%
\pgfpathlineto{\pgfqpoint{1.699262in}{2.337279in}}%
\pgfpathlineto{\pgfqpoint{1.705067in}{2.328751in}}%
\pgfpathlineto{\pgfqpoint{1.710878in}{2.320198in}}%
\pgfpathlineto{\pgfqpoint{1.716693in}{2.311620in}}%
\pgfpathclose%
\pgfusepath{stroke,fill}%
\end{pgfscope}%
\begin{pgfscope}%
\pgfpathrectangle{\pgfqpoint{0.887500in}{0.275000in}}{\pgfqpoint{4.225000in}{4.225000in}}%
\pgfusepath{clip}%
\pgfsetbuttcap%
\pgfsetroundjoin%
\definecolor{currentfill}{rgb}{0.151918,0.500685,0.557587}%
\pgfsetfillcolor{currentfill}%
\pgfsetfillopacity{0.700000}%
\pgfsetlinewidth{0.501875pt}%
\definecolor{currentstroke}{rgb}{1.000000,1.000000,1.000000}%
\pgfsetstrokecolor{currentstroke}%
\pgfsetstrokeopacity{0.500000}%
\pgfsetdash{}{0pt}%
\pgfpathmoveto{\pgfqpoint{2.044917in}{2.241508in}}%
\pgfpathlineto{\pgfqpoint{2.056586in}{2.244957in}}%
\pgfpathlineto{\pgfqpoint{2.068250in}{2.248396in}}%
\pgfpathlineto{\pgfqpoint{2.079909in}{2.251827in}}%
\pgfpathlineto{\pgfqpoint{2.091562in}{2.255251in}}%
\pgfpathlineto{\pgfqpoint{2.103209in}{2.258671in}}%
\pgfpathlineto{\pgfqpoint{2.097267in}{2.267427in}}%
\pgfpathlineto{\pgfqpoint{2.091329in}{2.276159in}}%
\pgfpathlineto{\pgfqpoint{2.085395in}{2.284868in}}%
\pgfpathlineto{\pgfqpoint{2.079466in}{2.293554in}}%
\pgfpathlineto{\pgfqpoint{2.073541in}{2.302218in}}%
\pgfpathlineto{\pgfqpoint{2.061905in}{2.298842in}}%
\pgfpathlineto{\pgfqpoint{2.050264in}{2.295463in}}%
\pgfpathlineto{\pgfqpoint{2.038616in}{2.292078in}}%
\pgfpathlineto{\pgfqpoint{2.026963in}{2.288687in}}%
\pgfpathlineto{\pgfqpoint{2.015304in}{2.285286in}}%
\pgfpathlineto{\pgfqpoint{2.021218in}{2.276580in}}%
\pgfpathlineto{\pgfqpoint{2.027136in}{2.267849in}}%
\pgfpathlineto{\pgfqpoint{2.033059in}{2.259093in}}%
\pgfpathlineto{\pgfqpoint{2.038985in}{2.250313in}}%
\pgfpathclose%
\pgfusepath{stroke,fill}%
\end{pgfscope}%
\begin{pgfscope}%
\pgfpathrectangle{\pgfqpoint{0.887500in}{0.275000in}}{\pgfqpoint{4.225000in}{4.225000in}}%
\pgfusepath{clip}%
\pgfsetbuttcap%
\pgfsetroundjoin%
\definecolor{currentfill}{rgb}{0.132268,0.655014,0.519661}%
\pgfsetfillcolor{currentfill}%
\pgfsetfillopacity{0.700000}%
\pgfsetlinewidth{0.501875pt}%
\definecolor{currentstroke}{rgb}{1.000000,1.000000,1.000000}%
\pgfsetstrokecolor{currentstroke}%
\pgfsetstrokeopacity{0.500000}%
\pgfsetdash{}{0pt}%
\pgfpathmoveto{\pgfqpoint{3.702794in}{2.551054in}}%
\pgfpathlineto{\pgfqpoint{3.714058in}{2.554961in}}%
\pgfpathlineto{\pgfqpoint{3.725317in}{2.558832in}}%
\pgfpathlineto{\pgfqpoint{3.736569in}{2.562656in}}%
\pgfpathlineto{\pgfqpoint{3.747816in}{2.566422in}}%
\pgfpathlineto{\pgfqpoint{3.759055in}{2.570122in}}%
\pgfpathlineto{\pgfqpoint{3.752666in}{2.584145in}}%
\pgfpathlineto{\pgfqpoint{3.746280in}{2.598175in}}%
\pgfpathlineto{\pgfqpoint{3.739895in}{2.612201in}}%
\pgfpathlineto{\pgfqpoint{3.733514in}{2.626217in}}%
\pgfpathlineto{\pgfqpoint{3.727134in}{2.640213in}}%
\pgfpathlineto{\pgfqpoint{3.715899in}{2.636606in}}%
\pgfpathlineto{\pgfqpoint{3.704657in}{2.632930in}}%
\pgfpathlineto{\pgfqpoint{3.693409in}{2.629197in}}%
\pgfpathlineto{\pgfqpoint{3.682155in}{2.625424in}}%
\pgfpathlineto{\pgfqpoint{3.670895in}{2.621625in}}%
\pgfpathlineto{\pgfqpoint{3.677270in}{2.607539in}}%
\pgfpathlineto{\pgfqpoint{3.683647in}{2.593429in}}%
\pgfpathlineto{\pgfqpoint{3.690027in}{2.579306in}}%
\pgfpathlineto{\pgfqpoint{3.696409in}{2.565177in}}%
\pgfpathclose%
\pgfusepath{stroke,fill}%
\end{pgfscope}%
\begin{pgfscope}%
\pgfpathrectangle{\pgfqpoint{0.887500in}{0.275000in}}{\pgfqpoint{4.225000in}{4.225000in}}%
\pgfusepath{clip}%
\pgfsetbuttcap%
\pgfsetroundjoin%
\definecolor{currentfill}{rgb}{0.179019,0.433756,0.557430}%
\pgfsetfillcolor{currentfill}%
\pgfsetfillopacity{0.700000}%
\pgfsetlinewidth{0.501875pt}%
\definecolor{currentstroke}{rgb}{1.000000,1.000000,1.000000}%
\pgfsetstrokecolor{currentstroke}%
\pgfsetstrokeopacity{0.500000}%
\pgfsetdash{}{0pt}%
\pgfpathmoveto{\pgfqpoint{4.296005in}{2.083920in}}%
\pgfpathlineto{\pgfqpoint{4.307120in}{2.087577in}}%
\pgfpathlineto{\pgfqpoint{4.318227in}{2.091117in}}%
\pgfpathlineto{\pgfqpoint{4.329325in}{2.094551in}}%
\pgfpathlineto{\pgfqpoint{4.340415in}{2.097916in}}%
\pgfpathlineto{\pgfqpoint{4.351498in}{2.101250in}}%
\pgfpathlineto{\pgfqpoint{4.344974in}{2.114747in}}%
\pgfpathlineto{\pgfqpoint{4.338454in}{2.128277in}}%
\pgfpathlineto{\pgfqpoint{4.331939in}{2.141876in}}%
\pgfpathlineto{\pgfqpoint{4.325431in}{2.155577in}}%
\pgfpathlineto{\pgfqpoint{4.318929in}{2.169415in}}%
\pgfpathlineto{\pgfqpoint{4.307862in}{2.166524in}}%
\pgfpathlineto{\pgfqpoint{4.296790in}{2.163662in}}%
\pgfpathlineto{\pgfqpoint{4.285712in}{2.160765in}}%
\pgfpathlineto{\pgfqpoint{4.274624in}{2.157773in}}%
\pgfpathlineto{\pgfqpoint{4.263528in}{2.154666in}}%
\pgfpathlineto{\pgfqpoint{4.270015in}{2.140417in}}%
\pgfpathlineto{\pgfqpoint{4.276506in}{2.126225in}}%
\pgfpathlineto{\pgfqpoint{4.283002in}{2.112083in}}%
\pgfpathlineto{\pgfqpoint{4.289501in}{2.097984in}}%
\pgfpathclose%
\pgfusepath{stroke,fill}%
\end{pgfscope}%
\begin{pgfscope}%
\pgfpathrectangle{\pgfqpoint{0.887500in}{0.275000in}}{\pgfqpoint{4.225000in}{4.225000in}}%
\pgfusepath{clip}%
\pgfsetbuttcap%
\pgfsetroundjoin%
\definecolor{currentfill}{rgb}{0.182256,0.426184,0.557120}%
\pgfsetfillcolor{currentfill}%
\pgfsetfillopacity{0.700000}%
\pgfsetlinewidth{0.501875pt}%
\definecolor{currentstroke}{rgb}{1.000000,1.000000,1.000000}%
\pgfsetstrokecolor{currentstroke}%
\pgfsetstrokeopacity{0.500000}%
\pgfsetdash{}{0pt}%
\pgfpathmoveto{\pgfqpoint{2.778143in}{2.080070in}}%
\pgfpathlineto{\pgfqpoint{2.789635in}{2.083196in}}%
\pgfpathlineto{\pgfqpoint{2.801121in}{2.086365in}}%
\pgfpathlineto{\pgfqpoint{2.812600in}{2.089655in}}%
\pgfpathlineto{\pgfqpoint{2.824073in}{2.093146in}}%
\pgfpathlineto{\pgfqpoint{2.835538in}{2.096915in}}%
\pgfpathlineto{\pgfqpoint{2.829351in}{2.106366in}}%
\pgfpathlineto{\pgfqpoint{2.823168in}{2.115779in}}%
\pgfpathlineto{\pgfqpoint{2.816990in}{2.125155in}}%
\pgfpathlineto{\pgfqpoint{2.810815in}{2.134495in}}%
\pgfpathlineto{\pgfqpoint{2.804645in}{2.143801in}}%
\pgfpathlineto{\pgfqpoint{2.793190in}{2.140030in}}%
\pgfpathlineto{\pgfqpoint{2.781728in}{2.136566in}}%
\pgfpathlineto{\pgfqpoint{2.770259in}{2.133322in}}%
\pgfpathlineto{\pgfqpoint{2.758783in}{2.130213in}}%
\pgfpathlineto{\pgfqpoint{2.747301in}{2.127154in}}%
\pgfpathlineto{\pgfqpoint{2.753461in}{2.117813in}}%
\pgfpathlineto{\pgfqpoint{2.759625in}{2.108435in}}%
\pgfpathlineto{\pgfqpoint{2.765794in}{2.099020in}}%
\pgfpathlineto{\pgfqpoint{2.771967in}{2.089565in}}%
\pgfpathclose%
\pgfusepath{stroke,fill}%
\end{pgfscope}%
\begin{pgfscope}%
\pgfpathrectangle{\pgfqpoint{0.887500in}{0.275000in}}{\pgfqpoint{4.225000in}{4.225000in}}%
\pgfusepath{clip}%
\pgfsetbuttcap%
\pgfsetroundjoin%
\definecolor{currentfill}{rgb}{0.133743,0.548535,0.553541}%
\pgfsetfillcolor{currentfill}%
\pgfsetfillopacity{0.700000}%
\pgfsetlinewidth{0.501875pt}%
\definecolor{currentstroke}{rgb}{1.000000,1.000000,1.000000}%
\pgfsetstrokecolor{currentstroke}%
\pgfsetstrokeopacity{0.500000}%
\pgfsetdash{}{0pt}%
\pgfpathmoveto{\pgfqpoint{3.999458in}{2.320766in}}%
\pgfpathlineto{\pgfqpoint{4.010643in}{2.324251in}}%
\pgfpathlineto{\pgfqpoint{4.021823in}{2.327725in}}%
\pgfpathlineto{\pgfqpoint{4.032997in}{2.331204in}}%
\pgfpathlineto{\pgfqpoint{4.044166in}{2.334697in}}%
\pgfpathlineto{\pgfqpoint{4.055330in}{2.338199in}}%
\pgfpathlineto{\pgfqpoint{4.048891in}{2.352604in}}%
\pgfpathlineto{\pgfqpoint{4.042455in}{2.367016in}}%
\pgfpathlineto{\pgfqpoint{4.036022in}{2.381427in}}%
\pgfpathlineto{\pgfqpoint{4.029591in}{2.395827in}}%
\pgfpathlineto{\pgfqpoint{4.023162in}{2.410209in}}%
\pgfpathlineto{\pgfqpoint{4.012001in}{2.406751in}}%
\pgfpathlineto{\pgfqpoint{4.000835in}{2.403298in}}%
\pgfpathlineto{\pgfqpoint{3.989663in}{2.399857in}}%
\pgfpathlineto{\pgfqpoint{3.978487in}{2.396423in}}%
\pgfpathlineto{\pgfqpoint{3.967304in}{2.392990in}}%
\pgfpathlineto{\pgfqpoint{3.973730in}{2.378544in}}%
\pgfpathlineto{\pgfqpoint{3.980158in}{2.364085in}}%
\pgfpathlineto{\pgfqpoint{3.986588in}{2.349627in}}%
\pgfpathlineto{\pgfqpoint{3.993021in}{2.335183in}}%
\pgfpathclose%
\pgfusepath{stroke,fill}%
\end{pgfscope}%
\begin{pgfscope}%
\pgfpathrectangle{\pgfqpoint{0.887500in}{0.275000in}}{\pgfqpoint{4.225000in}{4.225000in}}%
\pgfusepath{clip}%
\pgfsetbuttcap%
\pgfsetroundjoin%
\definecolor{currentfill}{rgb}{0.231674,0.318106,0.544834}%
\pgfsetfillcolor{currentfill}%
\pgfsetfillopacity{0.700000}%
\pgfsetlinewidth{0.501875pt}%
\definecolor{currentstroke}{rgb}{1.000000,1.000000,1.000000}%
\pgfsetstrokecolor{currentstroke}%
\pgfsetstrokeopacity{0.500000}%
\pgfsetdash{}{0pt}%
\pgfpathmoveto{\pgfqpoint{4.592608in}{1.848775in}}%
\pgfpathlineto{\pgfqpoint{4.603640in}{1.852181in}}%
\pgfpathlineto{\pgfqpoint{4.614665in}{1.855570in}}%
\pgfpathlineto{\pgfqpoint{4.625685in}{1.858945in}}%
\pgfpathlineto{\pgfqpoint{4.636698in}{1.862306in}}%
\pgfpathlineto{\pgfqpoint{4.647706in}{1.865658in}}%
\pgfpathlineto{\pgfqpoint{4.641167in}{1.880362in}}%
\pgfpathlineto{\pgfqpoint{4.634630in}{1.895048in}}%
\pgfpathlineto{\pgfqpoint{4.628095in}{1.909716in}}%
\pgfpathlineto{\pgfqpoint{4.621562in}{1.924365in}}%
\pgfpathlineto{\pgfqpoint{4.615030in}{1.938997in}}%
\pgfpathlineto{\pgfqpoint{4.604025in}{1.935682in}}%
\pgfpathlineto{\pgfqpoint{4.593014in}{1.932366in}}%
\pgfpathlineto{\pgfqpoint{4.581997in}{1.929047in}}%
\pgfpathlineto{\pgfqpoint{4.570975in}{1.925722in}}%
\pgfpathlineto{\pgfqpoint{4.559947in}{1.922392in}}%
\pgfpathlineto{\pgfqpoint{4.566474in}{1.907675in}}%
\pgfpathlineto{\pgfqpoint{4.573004in}{1.892949in}}%
\pgfpathlineto{\pgfqpoint{4.579536in}{1.878219in}}%
\pgfpathlineto{\pgfqpoint{4.586070in}{1.863492in}}%
\pgfpathclose%
\pgfusepath{stroke,fill}%
\end{pgfscope}%
\begin{pgfscope}%
\pgfpathrectangle{\pgfqpoint{0.887500in}{0.275000in}}{\pgfqpoint{4.225000in}{4.225000in}}%
\pgfusepath{clip}%
\pgfsetbuttcap%
\pgfsetroundjoin%
\definecolor{currentfill}{rgb}{0.150148,0.676631,0.506589}%
\pgfsetfillcolor{currentfill}%
\pgfsetfillopacity{0.700000}%
\pgfsetlinewidth{0.501875pt}%
\definecolor{currentstroke}{rgb}{1.000000,1.000000,1.000000}%
\pgfsetstrokecolor{currentstroke}%
\pgfsetstrokeopacity{0.500000}%
\pgfsetdash{}{0pt}%
\pgfpathmoveto{\pgfqpoint{3.614517in}{2.602657in}}%
\pgfpathlineto{\pgfqpoint{3.625803in}{2.606472in}}%
\pgfpathlineto{\pgfqpoint{3.637084in}{2.610238in}}%
\pgfpathlineto{\pgfqpoint{3.648359in}{2.614016in}}%
\pgfpathlineto{\pgfqpoint{3.659630in}{2.617817in}}%
\pgfpathlineto{\pgfqpoint{3.670895in}{2.621625in}}%
\pgfpathlineto{\pgfqpoint{3.664522in}{2.635678in}}%
\pgfpathlineto{\pgfqpoint{3.658150in}{2.649687in}}%
\pgfpathlineto{\pgfqpoint{3.651781in}{2.663644in}}%
\pgfpathlineto{\pgfqpoint{3.645413in}{2.677539in}}%
\pgfpathlineto{\pgfqpoint{3.639047in}{2.691372in}}%
\pgfpathlineto{\pgfqpoint{3.627786in}{2.687629in}}%
\pgfpathlineto{\pgfqpoint{3.616519in}{2.683889in}}%
\pgfpathlineto{\pgfqpoint{3.605248in}{2.680165in}}%
\pgfpathlineto{\pgfqpoint{3.593971in}{2.676449in}}%
\pgfpathlineto{\pgfqpoint{3.582688in}{2.672684in}}%
\pgfpathlineto{\pgfqpoint{3.589051in}{2.658808in}}%
\pgfpathlineto{\pgfqpoint{3.595415in}{2.644867in}}%
\pgfpathlineto{\pgfqpoint{3.601780in}{2.630859in}}%
\pgfpathlineto{\pgfqpoint{3.608148in}{2.616788in}}%
\pgfpathclose%
\pgfusepath{stroke,fill}%
\end{pgfscope}%
\begin{pgfscope}%
\pgfpathrectangle{\pgfqpoint{0.887500in}{0.275000in}}{\pgfqpoint{4.225000in}{4.225000in}}%
\pgfusepath{clip}%
\pgfsetbuttcap%
\pgfsetroundjoin%
\definecolor{currentfill}{rgb}{0.168126,0.459988,0.558082}%
\pgfsetfillcolor{currentfill}%
\pgfsetfillopacity{0.700000}%
\pgfsetlinewidth{0.501875pt}%
\definecolor{currentstroke}{rgb}{1.000000,1.000000,1.000000}%
\pgfsetstrokecolor{currentstroke}%
\pgfsetstrokeopacity{0.500000}%
\pgfsetdash{}{0pt}%
\pgfpathmoveto{\pgfqpoint{2.455570in}{2.149857in}}%
\pgfpathlineto{\pgfqpoint{2.467142in}{2.153316in}}%
\pgfpathlineto{\pgfqpoint{2.478708in}{2.156764in}}%
\pgfpathlineto{\pgfqpoint{2.490269in}{2.160207in}}%
\pgfpathlineto{\pgfqpoint{2.501824in}{2.163650in}}%
\pgfpathlineto{\pgfqpoint{2.513373in}{2.167096in}}%
\pgfpathlineto{\pgfqpoint{2.507290in}{2.176171in}}%
\pgfpathlineto{\pgfqpoint{2.501212in}{2.185215in}}%
\pgfpathlineto{\pgfqpoint{2.495138in}{2.194229in}}%
\pgfpathlineto{\pgfqpoint{2.489068in}{2.203213in}}%
\pgfpathlineto{\pgfqpoint{2.483002in}{2.212168in}}%
\pgfpathlineto{\pgfqpoint{2.471464in}{2.208761in}}%
\pgfpathlineto{\pgfqpoint{2.459919in}{2.205360in}}%
\pgfpathlineto{\pgfqpoint{2.448369in}{2.201960in}}%
\pgfpathlineto{\pgfqpoint{2.436814in}{2.198556in}}%
\pgfpathlineto{\pgfqpoint{2.425252in}{2.195143in}}%
\pgfpathlineto{\pgfqpoint{2.431307in}{2.186146in}}%
\pgfpathlineto{\pgfqpoint{2.437366in}{2.177120in}}%
\pgfpathlineto{\pgfqpoint{2.443430in}{2.168064in}}%
\pgfpathlineto{\pgfqpoint{2.449497in}{2.158976in}}%
\pgfpathclose%
\pgfusepath{stroke,fill}%
\end{pgfscope}%
\begin{pgfscope}%
\pgfpathrectangle{\pgfqpoint{0.887500in}{0.275000in}}{\pgfqpoint{4.225000in}{4.225000in}}%
\pgfusepath{clip}%
\pgfsetbuttcap%
\pgfsetroundjoin%
\definecolor{currentfill}{rgb}{0.154815,0.493313,0.557840}%
\pgfsetfillcolor{currentfill}%
\pgfsetfillopacity{0.700000}%
\pgfsetlinewidth{0.501875pt}%
\definecolor{currentstroke}{rgb}{1.000000,1.000000,1.000000}%
\pgfsetstrokecolor{currentstroke}%
\pgfsetstrokeopacity{0.500000}%
\pgfsetdash{}{0pt}%
\pgfpathmoveto{\pgfqpoint{2.132985in}{2.214512in}}%
\pgfpathlineto{\pgfqpoint{2.144637in}{2.217979in}}%
\pgfpathlineto{\pgfqpoint{2.156284in}{2.221447in}}%
\pgfpathlineto{\pgfqpoint{2.167924in}{2.224920in}}%
\pgfpathlineto{\pgfqpoint{2.179559in}{2.228400in}}%
\pgfpathlineto{\pgfqpoint{2.191187in}{2.231890in}}%
\pgfpathlineto{\pgfqpoint{2.185212in}{2.240727in}}%
\pgfpathlineto{\pgfqpoint{2.179242in}{2.249537in}}%
\pgfpathlineto{\pgfqpoint{2.173276in}{2.258323in}}%
\pgfpathlineto{\pgfqpoint{2.167314in}{2.267083in}}%
\pgfpathlineto{\pgfqpoint{2.161356in}{2.275820in}}%
\pgfpathlineto{\pgfqpoint{2.149739in}{2.272374in}}%
\pgfpathlineto{\pgfqpoint{2.138115in}{2.268939in}}%
\pgfpathlineto{\pgfqpoint{2.126486in}{2.265512in}}%
\pgfpathlineto{\pgfqpoint{2.114850in}{2.262091in}}%
\pgfpathlineto{\pgfqpoint{2.103209in}{2.258671in}}%
\pgfpathlineto{\pgfqpoint{2.109155in}{2.249890in}}%
\pgfpathlineto{\pgfqpoint{2.115106in}{2.241085in}}%
\pgfpathlineto{\pgfqpoint{2.121061in}{2.232253in}}%
\pgfpathlineto{\pgfqpoint{2.127021in}{2.223396in}}%
\pgfpathclose%
\pgfusepath{stroke,fill}%
\end{pgfscope}%
\begin{pgfscope}%
\pgfpathrectangle{\pgfqpoint{0.887500in}{0.275000in}}{\pgfqpoint{4.225000in}{4.225000in}}%
\pgfusepath{clip}%
\pgfsetbuttcap%
\pgfsetroundjoin%
\definecolor{currentfill}{rgb}{0.143343,0.522773,0.556295}%
\pgfsetfillcolor{currentfill}%
\pgfsetfillopacity{0.700000}%
\pgfsetlinewidth{0.501875pt}%
\definecolor{currentstroke}{rgb}{1.000000,1.000000,1.000000}%
\pgfsetstrokecolor{currentstroke}%
\pgfsetstrokeopacity{0.500000}%
\pgfsetdash{}{0pt}%
\pgfpathmoveto{\pgfqpoint{1.810423in}{2.276968in}}%
\pgfpathlineto{\pgfqpoint{1.822155in}{2.280426in}}%
\pgfpathlineto{\pgfqpoint{1.833881in}{2.283882in}}%
\pgfpathlineto{\pgfqpoint{1.845602in}{2.287338in}}%
\pgfpathlineto{\pgfqpoint{1.857316in}{2.290796in}}%
\pgfpathlineto{\pgfqpoint{1.869025in}{2.294255in}}%
\pgfpathlineto{\pgfqpoint{1.863160in}{2.302896in}}%
\pgfpathlineto{\pgfqpoint{1.857300in}{2.311514in}}%
\pgfpathlineto{\pgfqpoint{1.851444in}{2.320108in}}%
\pgfpathlineto{\pgfqpoint{1.845593in}{2.328679in}}%
\pgfpathlineto{\pgfqpoint{1.839746in}{2.337228in}}%
\pgfpathlineto{\pgfqpoint{1.828049in}{2.333811in}}%
\pgfpathlineto{\pgfqpoint{1.816346in}{2.330396in}}%
\pgfpathlineto{\pgfqpoint{1.804637in}{2.326981in}}%
\pgfpathlineto{\pgfqpoint{1.792922in}{2.323567in}}%
\pgfpathlineto{\pgfqpoint{1.781201in}{2.320151in}}%
\pgfpathlineto{\pgfqpoint{1.787037in}{2.311563in}}%
\pgfpathlineto{\pgfqpoint{1.792877in}{2.302951in}}%
\pgfpathlineto{\pgfqpoint{1.798721in}{2.294314in}}%
\pgfpathlineto{\pgfqpoint{1.804570in}{2.285654in}}%
\pgfpathclose%
\pgfusepath{stroke,fill}%
\end{pgfscope}%
\begin{pgfscope}%
\pgfpathrectangle{\pgfqpoint{0.887500in}{0.275000in}}{\pgfqpoint{4.225000in}{4.225000in}}%
\pgfusepath{clip}%
\pgfsetbuttcap%
\pgfsetroundjoin%
\definecolor{currentfill}{rgb}{0.180653,0.701402,0.488189}%
\pgfsetfillcolor{currentfill}%
\pgfsetfillopacity{0.700000}%
\pgfsetlinewidth{0.501875pt}%
\definecolor{currentstroke}{rgb}{1.000000,1.000000,1.000000}%
\pgfsetstrokecolor{currentstroke}%
\pgfsetstrokeopacity{0.500000}%
\pgfsetdash{}{0pt}%
\pgfpathmoveto{\pgfqpoint{3.526160in}{2.650935in}}%
\pgfpathlineto{\pgfqpoint{3.537483in}{2.655883in}}%
\pgfpathlineto{\pgfqpoint{3.548797in}{2.660469in}}%
\pgfpathlineto{\pgfqpoint{3.560102in}{2.664757in}}%
\pgfpathlineto{\pgfqpoint{3.571399in}{2.668808in}}%
\pgfpathlineto{\pgfqpoint{3.582688in}{2.672684in}}%
\pgfpathlineto{\pgfqpoint{3.576328in}{2.686501in}}%
\pgfpathlineto{\pgfqpoint{3.569969in}{2.700263in}}%
\pgfpathlineto{\pgfqpoint{3.563612in}{2.713975in}}%
\pgfpathlineto{\pgfqpoint{3.557257in}{2.727643in}}%
\pgfpathlineto{\pgfqpoint{3.550905in}{2.741271in}}%
\pgfpathlineto{\pgfqpoint{3.539620in}{2.737478in}}%
\pgfpathlineto{\pgfqpoint{3.528329in}{2.733588in}}%
\pgfpathlineto{\pgfqpoint{3.517031in}{2.729569in}}%
\pgfpathlineto{\pgfqpoint{3.505726in}{2.725388in}}%
\pgfpathlineto{\pgfqpoint{3.494413in}{2.721012in}}%
\pgfpathlineto{\pgfqpoint{3.500759in}{2.707069in}}%
\pgfpathlineto{\pgfqpoint{3.507106in}{2.693080in}}%
\pgfpathlineto{\pgfqpoint{3.513455in}{2.679055in}}%
\pgfpathlineto{\pgfqpoint{3.519806in}{2.665003in}}%
\pgfpathclose%
\pgfusepath{stroke,fill}%
\end{pgfscope}%
\begin{pgfscope}%
\pgfpathrectangle{\pgfqpoint{0.887500in}{0.275000in}}{\pgfqpoint{4.225000in}{4.225000in}}%
\pgfusepath{clip}%
\pgfsetbuttcap%
\pgfsetroundjoin%
\definecolor{currentfill}{rgb}{0.166617,0.463708,0.558119}%
\pgfsetfillcolor{currentfill}%
\pgfsetfillopacity{0.700000}%
\pgfsetlinewidth{0.501875pt}%
\definecolor{currentstroke}{rgb}{1.000000,1.000000,1.000000}%
\pgfsetstrokecolor{currentstroke}%
\pgfsetstrokeopacity{0.500000}%
\pgfsetdash{}{0pt}%
\pgfpathmoveto{\pgfqpoint{4.207924in}{2.137894in}}%
\pgfpathlineto{\pgfqpoint{4.219059in}{2.141364in}}%
\pgfpathlineto{\pgfqpoint{4.230188in}{2.144792in}}%
\pgfpathlineto{\pgfqpoint{4.241309in}{2.148161in}}%
\pgfpathlineto{\pgfqpoint{4.252423in}{2.151457in}}%
\pgfpathlineto{\pgfqpoint{4.263528in}{2.154666in}}%
\pgfpathlineto{\pgfqpoint{4.257046in}{2.168981in}}%
\pgfpathlineto{\pgfqpoint{4.250568in}{2.183367in}}%
\pgfpathlineto{\pgfqpoint{4.244095in}{2.197822in}}%
\pgfpathlineto{\pgfqpoint{4.237626in}{2.212334in}}%
\pgfpathlineto{\pgfqpoint{4.231162in}{2.226894in}}%
\pgfpathlineto{\pgfqpoint{4.220062in}{2.223840in}}%
\pgfpathlineto{\pgfqpoint{4.208956in}{2.220728in}}%
\pgfpathlineto{\pgfqpoint{4.197842in}{2.217567in}}%
\pgfpathlineto{\pgfqpoint{4.186721in}{2.214366in}}%
\pgfpathlineto{\pgfqpoint{4.175594in}{2.211134in}}%
\pgfpathlineto{\pgfqpoint{4.182058in}{2.196572in}}%
\pgfpathlineto{\pgfqpoint{4.188523in}{2.181970in}}%
\pgfpathlineto{\pgfqpoint{4.194989in}{2.167325in}}%
\pgfpathlineto{\pgfqpoint{4.201456in}{2.152630in}}%
\pgfpathclose%
\pgfusepath{stroke,fill}%
\end{pgfscope}%
\begin{pgfscope}%
\pgfpathrectangle{\pgfqpoint{0.887500in}{0.275000in}}{\pgfqpoint{4.225000in}{4.225000in}}%
\pgfusepath{clip}%
\pgfsetbuttcap%
\pgfsetroundjoin%
\definecolor{currentfill}{rgb}{0.246070,0.738910,0.452024}%
\pgfsetfillcolor{currentfill}%
\pgfsetfillopacity{0.700000}%
\pgfsetlinewidth{0.501875pt}%
\definecolor{currentstroke}{rgb}{1.000000,1.000000,1.000000}%
\pgfsetstrokecolor{currentstroke}%
\pgfsetstrokeopacity{0.500000}%
\pgfsetdash{}{0pt}%
\pgfpathmoveto{\pgfqpoint{3.349258in}{2.732192in}}%
\pgfpathlineto{\pgfqpoint{3.360640in}{2.740205in}}%
\pgfpathlineto{\pgfqpoint{3.372015in}{2.747702in}}%
\pgfpathlineto{\pgfqpoint{3.383382in}{2.754641in}}%
\pgfpathlineto{\pgfqpoint{3.394740in}{2.760985in}}%
\pgfpathlineto{\pgfqpoint{3.406090in}{2.766751in}}%
\pgfpathlineto{\pgfqpoint{3.399761in}{2.780386in}}%
\pgfpathlineto{\pgfqpoint{3.393434in}{2.793807in}}%
\pgfpathlineto{\pgfqpoint{3.387107in}{2.807060in}}%
\pgfpathlineto{\pgfqpoint{3.380782in}{2.820189in}}%
\pgfpathlineto{\pgfqpoint{3.374460in}{2.833241in}}%
\pgfpathlineto{\pgfqpoint{3.363115in}{2.827373in}}%
\pgfpathlineto{\pgfqpoint{3.351762in}{2.820904in}}%
\pgfpathlineto{\pgfqpoint{3.340401in}{2.813793in}}%
\pgfpathlineto{\pgfqpoint{3.329032in}{2.806040in}}%
\pgfpathlineto{\pgfqpoint{3.317655in}{2.797644in}}%
\pgfpathlineto{\pgfqpoint{3.323973in}{2.784968in}}%
\pgfpathlineto{\pgfqpoint{3.330293in}{2.772171in}}%
\pgfpathlineto{\pgfqpoint{3.336614in}{2.759167in}}%
\pgfpathlineto{\pgfqpoint{3.342936in}{2.745869in}}%
\pgfpathclose%
\pgfusepath{stroke,fill}%
\end{pgfscope}%
\begin{pgfscope}%
\pgfpathrectangle{\pgfqpoint{0.887500in}{0.275000in}}{\pgfqpoint{4.225000in}{4.225000in}}%
\pgfusepath{clip}%
\pgfsetbuttcap%
\pgfsetroundjoin%
\definecolor{currentfill}{rgb}{0.125394,0.574318,0.549086}%
\pgfsetfillcolor{currentfill}%
\pgfsetfillopacity{0.700000}%
\pgfsetlinewidth{0.501875pt}%
\definecolor{currentstroke}{rgb}{1.000000,1.000000,1.000000}%
\pgfsetstrokecolor{currentstroke}%
\pgfsetstrokeopacity{0.500000}%
\pgfsetdash{}{0pt}%
\pgfpathmoveto{\pgfqpoint{3.911307in}{2.375642in}}%
\pgfpathlineto{\pgfqpoint{3.922519in}{2.379157in}}%
\pgfpathlineto{\pgfqpoint{3.933724in}{2.382643in}}%
\pgfpathlineto{\pgfqpoint{3.944923in}{2.386106in}}%
\pgfpathlineto{\pgfqpoint{3.956117in}{2.389553in}}%
\pgfpathlineto{\pgfqpoint{3.967304in}{2.392990in}}%
\pgfpathlineto{\pgfqpoint{3.960881in}{2.407410in}}%
\pgfpathlineto{\pgfqpoint{3.954458in}{2.421789in}}%
\pgfpathlineto{\pgfqpoint{3.948037in}{2.436116in}}%
\pgfpathlineto{\pgfqpoint{3.941618in}{2.450388in}}%
\pgfpathlineto{\pgfqpoint{3.935199in}{2.464608in}}%
\pgfpathlineto{\pgfqpoint{3.924015in}{2.461222in}}%
\pgfpathlineto{\pgfqpoint{3.912825in}{2.457835in}}%
\pgfpathlineto{\pgfqpoint{3.901629in}{2.454445in}}%
\pgfpathlineto{\pgfqpoint{3.890428in}{2.451050in}}%
\pgfpathlineto{\pgfqpoint{3.879221in}{2.447646in}}%
\pgfpathlineto{\pgfqpoint{3.885636in}{2.433401in}}%
\pgfpathlineto{\pgfqpoint{3.892053in}{2.419081in}}%
\pgfpathlineto{\pgfqpoint{3.898470in}{2.404677in}}%
\pgfpathlineto{\pgfqpoint{3.904888in}{2.390191in}}%
\pgfpathclose%
\pgfusepath{stroke,fill}%
\end{pgfscope}%
\begin{pgfscope}%
\pgfpathrectangle{\pgfqpoint{0.887500in}{0.275000in}}{\pgfqpoint{4.225000in}{4.225000in}}%
\pgfusepath{clip}%
\pgfsetbuttcap%
\pgfsetroundjoin%
\definecolor{currentfill}{rgb}{0.187231,0.414746,0.556547}%
\pgfsetfillcolor{currentfill}%
\pgfsetfillopacity{0.700000}%
\pgfsetlinewidth{0.501875pt}%
\definecolor{currentstroke}{rgb}{1.000000,1.000000,1.000000}%
\pgfsetstrokecolor{currentstroke}%
\pgfsetstrokeopacity{0.500000}%
\pgfsetdash{}{0pt}%
\pgfpathmoveto{\pgfqpoint{2.866532in}{2.049056in}}%
\pgfpathlineto{\pgfqpoint{2.878000in}{2.053181in}}%
\pgfpathlineto{\pgfqpoint{2.889462in}{2.057547in}}%
\pgfpathlineto{\pgfqpoint{2.900918in}{2.061912in}}%
\pgfpathlineto{\pgfqpoint{2.912371in}{2.066032in}}%
\pgfpathlineto{\pgfqpoint{2.923820in}{2.069661in}}%
\pgfpathlineto{\pgfqpoint{2.917603in}{2.079294in}}%
\pgfpathlineto{\pgfqpoint{2.911389in}{2.088889in}}%
\pgfpathlineto{\pgfqpoint{2.905180in}{2.098450in}}%
\pgfpathlineto{\pgfqpoint{2.898974in}{2.107979in}}%
\pgfpathlineto{\pgfqpoint{2.892773in}{2.117479in}}%
\pgfpathlineto{\pgfqpoint{2.881334in}{2.113888in}}%
\pgfpathlineto{\pgfqpoint{2.869892in}{2.109777in}}%
\pgfpathlineto{\pgfqpoint{2.858446in}{2.105405in}}%
\pgfpathlineto{\pgfqpoint{2.846995in}{2.101033in}}%
\pgfpathlineto{\pgfqpoint{2.835538in}{2.096915in}}%
\pgfpathlineto{\pgfqpoint{2.841728in}{2.087425in}}%
\pgfpathlineto{\pgfqpoint{2.847923in}{2.077895in}}%
\pgfpathlineto{\pgfqpoint{2.854122in}{2.068323in}}%
\pgfpathlineto{\pgfqpoint{2.860325in}{2.058710in}}%
\pgfpathclose%
\pgfusepath{stroke,fill}%
\end{pgfscope}%
\begin{pgfscope}%
\pgfpathrectangle{\pgfqpoint{0.887500in}{0.275000in}}{\pgfqpoint{4.225000in}{4.225000in}}%
\pgfusepath{clip}%
\pgfsetbuttcap%
\pgfsetroundjoin%
\definecolor{currentfill}{rgb}{0.214000,0.722114,0.469588}%
\pgfsetfillcolor{currentfill}%
\pgfsetfillopacity{0.700000}%
\pgfsetlinewidth{0.501875pt}%
\definecolor{currentstroke}{rgb}{1.000000,1.000000,1.000000}%
\pgfsetstrokecolor{currentstroke}%
\pgfsetstrokeopacity{0.500000}%
\pgfsetdash{}{0pt}%
\pgfpathmoveto{\pgfqpoint{3.437735in}{2.694665in}}%
\pgfpathlineto{\pgfqpoint{3.449087in}{2.700691in}}%
\pgfpathlineto{\pgfqpoint{3.460431in}{2.706290in}}%
\pgfpathlineto{\pgfqpoint{3.471766in}{2.711511in}}%
\pgfpathlineto{\pgfqpoint{3.483094in}{2.716403in}}%
\pgfpathlineto{\pgfqpoint{3.494413in}{2.721012in}}%
\pgfpathlineto{\pgfqpoint{3.488070in}{2.734899in}}%
\pgfpathlineto{\pgfqpoint{3.481729in}{2.748720in}}%
\pgfpathlineto{\pgfqpoint{3.475389in}{2.762464in}}%
\pgfpathlineto{\pgfqpoint{3.469051in}{2.776123in}}%
\pgfpathlineto{\pgfqpoint{3.462715in}{2.789692in}}%
\pgfpathlineto{\pgfqpoint{3.451405in}{2.785620in}}%
\pgfpathlineto{\pgfqpoint{3.440088in}{2.781371in}}%
\pgfpathlineto{\pgfqpoint{3.428763in}{2.776863in}}%
\pgfpathlineto{\pgfqpoint{3.417431in}{2.772017in}}%
\pgfpathlineto{\pgfqpoint{3.406090in}{2.766751in}}%
\pgfpathlineto{\pgfqpoint{3.412419in}{2.752855in}}%
\pgfpathlineto{\pgfqpoint{3.418748in}{2.738661in}}%
\pgfpathlineto{\pgfqpoint{3.425076in}{2.724194in}}%
\pgfpathlineto{\pgfqpoint{3.431405in}{2.709511in}}%
\pgfpathclose%
\pgfusepath{stroke,fill}%
\end{pgfscope}%
\begin{pgfscope}%
\pgfpathrectangle{\pgfqpoint{0.887500in}{0.275000in}}{\pgfqpoint{4.225000in}{4.225000in}}%
\pgfusepath{clip}%
\pgfsetbuttcap%
\pgfsetroundjoin%
\definecolor{currentfill}{rgb}{0.216210,0.351535,0.550627}%
\pgfsetfillcolor{currentfill}%
\pgfsetfillopacity{0.700000}%
\pgfsetlinewidth{0.501875pt}%
\definecolor{currentstroke}{rgb}{1.000000,1.000000,1.000000}%
\pgfsetstrokecolor{currentstroke}%
\pgfsetstrokeopacity{0.500000}%
\pgfsetdash{}{0pt}%
\pgfpathmoveto{\pgfqpoint{4.504724in}{1.905716in}}%
\pgfpathlineto{\pgfqpoint{4.515779in}{1.909043in}}%
\pgfpathlineto{\pgfqpoint{4.526829in}{1.912378in}}%
\pgfpathlineto{\pgfqpoint{4.537874in}{1.915716in}}%
\pgfpathlineto{\pgfqpoint{4.548913in}{1.919055in}}%
\pgfpathlineto{\pgfqpoint{4.559947in}{1.922392in}}%
\pgfpathlineto{\pgfqpoint{4.553422in}{1.937093in}}%
\pgfpathlineto{\pgfqpoint{4.546898in}{1.951771in}}%
\pgfpathlineto{\pgfqpoint{4.540376in}{1.966422in}}%
\pgfpathlineto{\pgfqpoint{4.533855in}{1.981037in}}%
\pgfpathlineto{\pgfqpoint{4.527335in}{1.995612in}}%
\pgfpathlineto{\pgfqpoint{4.516306in}{1.992373in}}%
\pgfpathlineto{\pgfqpoint{4.505271in}{1.989130in}}%
\pgfpathlineto{\pgfqpoint{4.494230in}{1.985875in}}%
\pgfpathlineto{\pgfqpoint{4.483183in}{1.982600in}}%
\pgfpathlineto{\pgfqpoint{4.472129in}{1.979297in}}%
\pgfpathlineto{\pgfqpoint{4.478647in}{1.964697in}}%
\pgfpathlineto{\pgfqpoint{4.485165in}{1.950024in}}%
\pgfpathlineto{\pgfqpoint{4.491684in}{1.935294in}}%
\pgfpathlineto{\pgfqpoint{4.498203in}{1.920520in}}%
\pgfpathclose%
\pgfusepath{stroke,fill}%
\end{pgfscope}%
\begin{pgfscope}%
\pgfpathrectangle{\pgfqpoint{0.887500in}{0.275000in}}{\pgfqpoint{4.225000in}{4.225000in}}%
\pgfusepath{clip}%
\pgfsetbuttcap%
\pgfsetroundjoin%
\definecolor{currentfill}{rgb}{0.171176,0.452530,0.557965}%
\pgfsetfillcolor{currentfill}%
\pgfsetfillopacity{0.700000}%
\pgfsetlinewidth{0.501875pt}%
\definecolor{currentstroke}{rgb}{1.000000,1.000000,1.000000}%
\pgfsetstrokecolor{currentstroke}%
\pgfsetstrokeopacity{0.500000}%
\pgfsetdash{}{0pt}%
\pgfpathmoveto{\pgfqpoint{2.543850in}{2.121231in}}%
\pgfpathlineto{\pgfqpoint{2.555404in}{2.124728in}}%
\pgfpathlineto{\pgfqpoint{2.566951in}{2.128236in}}%
\pgfpathlineto{\pgfqpoint{2.578494in}{2.131756in}}%
\pgfpathlineto{\pgfqpoint{2.590030in}{2.135293in}}%
\pgfpathlineto{\pgfqpoint{2.601560in}{2.138850in}}%
\pgfpathlineto{\pgfqpoint{2.595446in}{2.148051in}}%
\pgfpathlineto{\pgfqpoint{2.589335in}{2.157218in}}%
\pgfpathlineto{\pgfqpoint{2.583229in}{2.166353in}}%
\pgfpathlineto{\pgfqpoint{2.577127in}{2.175456in}}%
\pgfpathlineto{\pgfqpoint{2.571030in}{2.184527in}}%
\pgfpathlineto{\pgfqpoint{2.559510in}{2.181004in}}%
\pgfpathlineto{\pgfqpoint{2.547985in}{2.177503in}}%
\pgfpathlineto{\pgfqpoint{2.536453in}{2.174020in}}%
\pgfpathlineto{\pgfqpoint{2.524916in}{2.170552in}}%
\pgfpathlineto{\pgfqpoint{2.513373in}{2.167096in}}%
\pgfpathlineto{\pgfqpoint{2.519460in}{2.157990in}}%
\pgfpathlineto{\pgfqpoint{2.525551in}{2.148850in}}%
\pgfpathlineto{\pgfqpoint{2.531646in}{2.139678in}}%
\pgfpathlineto{\pgfqpoint{2.537746in}{2.130472in}}%
\pgfpathclose%
\pgfusepath{stroke,fill}%
\end{pgfscope}%
\begin{pgfscope}%
\pgfpathrectangle{\pgfqpoint{0.887500in}{0.275000in}}{\pgfqpoint{4.225000in}{4.225000in}}%
\pgfusepath{clip}%
\pgfsetbuttcap%
\pgfsetroundjoin%
\definecolor{currentfill}{rgb}{0.194100,0.399323,0.555565}%
\pgfsetfillcolor{currentfill}%
\pgfsetfillopacity{0.700000}%
\pgfsetlinewidth{0.501875pt}%
\definecolor{currentstroke}{rgb}{1.000000,1.000000,1.000000}%
\pgfsetstrokecolor{currentstroke}%
\pgfsetstrokeopacity{0.500000}%
\pgfsetdash{}{0pt}%
\pgfpathmoveto{\pgfqpoint{2.954965in}{2.020828in}}%
\pgfpathlineto{\pgfqpoint{2.966420in}{2.023787in}}%
\pgfpathlineto{\pgfqpoint{2.977870in}{2.025807in}}%
\pgfpathlineto{\pgfqpoint{2.989316in}{2.026704in}}%
\pgfpathlineto{\pgfqpoint{3.000756in}{2.026918in}}%
\pgfpathlineto{\pgfqpoint{3.012188in}{2.027299in}}%
\pgfpathlineto{\pgfqpoint{3.005943in}{2.036984in}}%
\pgfpathlineto{\pgfqpoint{2.999703in}{2.046648in}}%
\pgfpathlineto{\pgfqpoint{2.993466in}{2.056285in}}%
\pgfpathlineto{\pgfqpoint{2.987233in}{2.065887in}}%
\pgfpathlineto{\pgfqpoint{2.981004in}{2.075445in}}%
\pgfpathlineto{\pgfqpoint{2.969579in}{2.075239in}}%
\pgfpathlineto{\pgfqpoint{2.958146in}{2.075210in}}%
\pgfpathlineto{\pgfqpoint{2.946708in}{2.074468in}}%
\pgfpathlineto{\pgfqpoint{2.935265in}{2.072555in}}%
\pgfpathlineto{\pgfqpoint{2.923820in}{2.069661in}}%
\pgfpathlineto{\pgfqpoint{2.930041in}{2.059988in}}%
\pgfpathlineto{\pgfqpoint{2.936266in}{2.050271in}}%
\pgfpathlineto{\pgfqpoint{2.942495in}{2.040508in}}%
\pgfpathlineto{\pgfqpoint{2.948728in}{2.030694in}}%
\pgfpathclose%
\pgfusepath{stroke,fill}%
\end{pgfscope}%
\begin{pgfscope}%
\pgfpathrectangle{\pgfqpoint{0.887500in}{0.275000in}}{\pgfqpoint{4.225000in}{4.225000in}}%
\pgfusepath{clip}%
\pgfsetbuttcap%
\pgfsetroundjoin%
\definecolor{currentfill}{rgb}{0.134692,0.658636,0.517649}%
\pgfsetfillcolor{currentfill}%
\pgfsetfillopacity{0.700000}%
\pgfsetlinewidth{0.501875pt}%
\definecolor{currentstroke}{rgb}{1.000000,1.000000,1.000000}%
\pgfsetstrokecolor{currentstroke}%
\pgfsetstrokeopacity{0.500000}%
\pgfsetdash{}{0pt}%
\pgfpathmoveto{\pgfqpoint{3.177883in}{2.515936in}}%
\pgfpathlineto{\pgfqpoint{3.189330in}{2.538519in}}%
\pgfpathlineto{\pgfqpoint{3.200781in}{2.560471in}}%
\pgfpathlineto{\pgfqpoint{3.212233in}{2.581448in}}%
\pgfpathlineto{\pgfqpoint{3.223683in}{2.601105in}}%
\pgfpathlineto{\pgfqpoint{3.235130in}{2.619098in}}%
\pgfpathlineto{\pgfqpoint{3.228823in}{2.631602in}}%
\pgfpathlineto{\pgfqpoint{3.222514in}{2.643054in}}%
\pgfpathlineto{\pgfqpoint{3.216206in}{2.653729in}}%
\pgfpathlineto{\pgfqpoint{3.209899in}{2.663905in}}%
\pgfpathlineto{\pgfqpoint{3.203595in}{2.673857in}}%
\pgfpathlineto{\pgfqpoint{3.192168in}{2.656009in}}%
\pgfpathlineto{\pgfqpoint{3.180740in}{2.637022in}}%
\pgfpathlineto{\pgfqpoint{3.169313in}{2.617090in}}%
\pgfpathlineto{\pgfqpoint{3.157887in}{2.596406in}}%
\pgfpathlineto{\pgfqpoint{3.146465in}{2.575163in}}%
\pgfpathlineto{\pgfqpoint{3.152744in}{2.563970in}}%
\pgfpathlineto{\pgfqpoint{3.159026in}{2.552551in}}%
\pgfpathlineto{\pgfqpoint{3.165310in}{2.540806in}}%
\pgfpathlineto{\pgfqpoint{3.171596in}{2.528634in}}%
\pgfpathclose%
\pgfusepath{stroke,fill}%
\end{pgfscope}%
\begin{pgfscope}%
\pgfpathrectangle{\pgfqpoint{0.887500in}{0.275000in}}{\pgfqpoint{4.225000in}{4.225000in}}%
\pgfusepath{clip}%
\pgfsetbuttcap%
\pgfsetroundjoin%
\definecolor{currentfill}{rgb}{0.120092,0.600104,0.542530}%
\pgfsetfillcolor{currentfill}%
\pgfsetfillopacity{0.700000}%
\pgfsetlinewidth{0.501875pt}%
\definecolor{currentstroke}{rgb}{1.000000,1.000000,1.000000}%
\pgfsetstrokecolor{currentstroke}%
\pgfsetstrokeopacity{0.500000}%
\pgfsetdash{}{0pt}%
\pgfpathmoveto{\pgfqpoint{3.823090in}{2.429955in}}%
\pgfpathlineto{\pgfqpoint{3.834330in}{2.433637in}}%
\pgfpathlineto{\pgfqpoint{3.845563in}{2.437230in}}%
\pgfpathlineto{\pgfqpoint{3.856789in}{2.440752in}}%
\pgfpathlineto{\pgfqpoint{3.868008in}{2.444218in}}%
\pgfpathlineto{\pgfqpoint{3.879221in}{2.447646in}}%
\pgfpathlineto{\pgfqpoint{3.872806in}{2.461823in}}%
\pgfpathlineto{\pgfqpoint{3.866393in}{2.475943in}}%
\pgfpathlineto{\pgfqpoint{3.859982in}{2.490013in}}%
\pgfpathlineto{\pgfqpoint{3.853572in}{2.504043in}}%
\pgfpathlineto{\pgfqpoint{3.847164in}{2.518041in}}%
\pgfpathlineto{\pgfqpoint{3.835953in}{2.514581in}}%
\pgfpathlineto{\pgfqpoint{3.824736in}{2.511085in}}%
\pgfpathlineto{\pgfqpoint{3.813512in}{2.507544in}}%
\pgfpathlineto{\pgfqpoint{3.802282in}{2.503948in}}%
\pgfpathlineto{\pgfqpoint{3.791045in}{2.500286in}}%
\pgfpathlineto{\pgfqpoint{3.797451in}{2.486327in}}%
\pgfpathlineto{\pgfqpoint{3.803859in}{2.472332in}}%
\pgfpathlineto{\pgfqpoint{3.810268in}{2.458283in}}%
\pgfpathlineto{\pgfqpoint{3.816679in}{2.444164in}}%
\pgfpathclose%
\pgfusepath{stroke,fill}%
\end{pgfscope}%
\begin{pgfscope}%
\pgfpathrectangle{\pgfqpoint{0.887500in}{0.275000in}}{\pgfqpoint{4.225000in}{4.225000in}}%
\pgfusepath{clip}%
\pgfsetbuttcap%
\pgfsetroundjoin%
\definecolor{currentfill}{rgb}{0.154815,0.493313,0.557840}%
\pgfsetfillcolor{currentfill}%
\pgfsetfillopacity{0.700000}%
\pgfsetlinewidth{0.501875pt}%
\definecolor{currentstroke}{rgb}{1.000000,1.000000,1.000000}%
\pgfsetstrokecolor{currentstroke}%
\pgfsetstrokeopacity{0.500000}%
\pgfsetdash{}{0pt}%
\pgfpathmoveto{\pgfqpoint{4.119869in}{2.194746in}}%
\pgfpathlineto{\pgfqpoint{4.131026in}{2.198040in}}%
\pgfpathlineto{\pgfqpoint{4.142177in}{2.201329in}}%
\pgfpathlineto{\pgfqpoint{4.153322in}{2.204611in}}%
\pgfpathlineto{\pgfqpoint{4.164461in}{2.207880in}}%
\pgfpathlineto{\pgfqpoint{4.175594in}{2.211134in}}%
\pgfpathlineto{\pgfqpoint{4.169132in}{2.225662in}}%
\pgfpathlineto{\pgfqpoint{4.162672in}{2.240160in}}%
\pgfpathlineto{\pgfqpoint{4.156214in}{2.254634in}}%
\pgfpathlineto{\pgfqpoint{4.149757in}{2.269089in}}%
\pgfpathlineto{\pgfqpoint{4.143303in}{2.283528in}}%
\pgfpathlineto{\pgfqpoint{4.132168in}{2.280162in}}%
\pgfpathlineto{\pgfqpoint{4.121027in}{2.276775in}}%
\pgfpathlineto{\pgfqpoint{4.109880in}{2.273372in}}%
\pgfpathlineto{\pgfqpoint{4.098728in}{2.269961in}}%
\pgfpathlineto{\pgfqpoint{4.087570in}{2.266546in}}%
\pgfpathlineto{\pgfqpoint{4.094027in}{2.252274in}}%
\pgfpathlineto{\pgfqpoint{4.100486in}{2.237980in}}%
\pgfpathlineto{\pgfqpoint{4.106946in}{2.223642in}}%
\pgfpathlineto{\pgfqpoint{4.113408in}{2.209238in}}%
\pgfpathclose%
\pgfusepath{stroke,fill}%
\end{pgfscope}%
\begin{pgfscope}%
\pgfpathrectangle{\pgfqpoint{0.887500in}{0.275000in}}{\pgfqpoint{4.225000in}{4.225000in}}%
\pgfusepath{clip}%
\pgfsetbuttcap%
\pgfsetroundjoin%
\definecolor{currentfill}{rgb}{0.159194,0.482237,0.558073}%
\pgfsetfillcolor{currentfill}%
\pgfsetfillopacity{0.700000}%
\pgfsetlinewidth{0.501875pt}%
\definecolor{currentstroke}{rgb}{1.000000,1.000000,1.000000}%
\pgfsetstrokecolor{currentstroke}%
\pgfsetstrokeopacity{0.500000}%
\pgfsetdash{}{0pt}%
\pgfpathmoveto{\pgfqpoint{2.221127in}{2.187301in}}%
\pgfpathlineto{\pgfqpoint{2.232761in}{2.190848in}}%
\pgfpathlineto{\pgfqpoint{2.244388in}{2.194400in}}%
\pgfpathlineto{\pgfqpoint{2.256010in}{2.197954in}}%
\pgfpathlineto{\pgfqpoint{2.267626in}{2.201505in}}%
\pgfpathlineto{\pgfqpoint{2.279237in}{2.205052in}}%
\pgfpathlineto{\pgfqpoint{2.273229in}{2.213976in}}%
\pgfpathlineto{\pgfqpoint{2.267226in}{2.222873in}}%
\pgfpathlineto{\pgfqpoint{2.261227in}{2.231744in}}%
\pgfpathlineto{\pgfqpoint{2.255233in}{2.240587in}}%
\pgfpathlineto{\pgfqpoint{2.249242in}{2.249405in}}%
\pgfpathlineto{\pgfqpoint{2.237643in}{2.245905in}}%
\pgfpathlineto{\pgfqpoint{2.226038in}{2.242400in}}%
\pgfpathlineto{\pgfqpoint{2.214427in}{2.238894in}}%
\pgfpathlineto{\pgfqpoint{2.202810in}{2.235389in}}%
\pgfpathlineto{\pgfqpoint{2.191187in}{2.231890in}}%
\pgfpathlineto{\pgfqpoint{2.197167in}{2.223027in}}%
\pgfpathlineto{\pgfqpoint{2.203150in}{2.214136in}}%
\pgfpathlineto{\pgfqpoint{2.209138in}{2.205219in}}%
\pgfpathlineto{\pgfqpoint{2.215130in}{2.196274in}}%
\pgfpathclose%
\pgfusepath{stroke,fill}%
\end{pgfscope}%
\begin{pgfscope}%
\pgfpathrectangle{\pgfqpoint{0.887500in}{0.275000in}}{\pgfqpoint{4.225000in}{4.225000in}}%
\pgfusepath{clip}%
\pgfsetbuttcap%
\pgfsetroundjoin%
\definecolor{currentfill}{rgb}{0.147607,0.511733,0.557049}%
\pgfsetfillcolor{currentfill}%
\pgfsetfillopacity{0.700000}%
\pgfsetlinewidth{0.501875pt}%
\definecolor{currentstroke}{rgb}{1.000000,1.000000,1.000000}%
\pgfsetstrokecolor{currentstroke}%
\pgfsetstrokeopacity{0.500000}%
\pgfsetdash{}{0pt}%
\pgfpathmoveto{\pgfqpoint{1.898414in}{2.250672in}}%
\pgfpathlineto{\pgfqpoint{1.910128in}{2.254174in}}%
\pgfpathlineto{\pgfqpoint{1.921837in}{2.257671in}}%
\pgfpathlineto{\pgfqpoint{1.933539in}{2.261163in}}%
\pgfpathlineto{\pgfqpoint{1.945237in}{2.264647in}}%
\pgfpathlineto{\pgfqpoint{1.956928in}{2.268119in}}%
\pgfpathlineto{\pgfqpoint{1.951030in}{2.276844in}}%
\pgfpathlineto{\pgfqpoint{1.945136in}{2.285543in}}%
\pgfpathlineto{\pgfqpoint{1.939247in}{2.294217in}}%
\pgfpathlineto{\pgfqpoint{1.933362in}{2.302865in}}%
\pgfpathlineto{\pgfqpoint{1.927482in}{2.311489in}}%
\pgfpathlineto{\pgfqpoint{1.915802in}{2.308058in}}%
\pgfpathlineto{\pgfqpoint{1.904116in}{2.304616in}}%
\pgfpathlineto{\pgfqpoint{1.892425in}{2.301167in}}%
\pgfpathlineto{\pgfqpoint{1.880727in}{2.297712in}}%
\pgfpathlineto{\pgfqpoint{1.869025in}{2.294255in}}%
\pgfpathlineto{\pgfqpoint{1.874894in}{2.285589in}}%
\pgfpathlineto{\pgfqpoint{1.880767in}{2.276898in}}%
\pgfpathlineto{\pgfqpoint{1.886645in}{2.268182in}}%
\pgfpathlineto{\pgfqpoint{1.892527in}{2.259440in}}%
\pgfpathclose%
\pgfusepath{stroke,fill}%
\end{pgfscope}%
\begin{pgfscope}%
\pgfpathrectangle{\pgfqpoint{0.887500in}{0.275000in}}{\pgfqpoint{4.225000in}{4.225000in}}%
\pgfusepath{clip}%
\pgfsetbuttcap%
\pgfsetroundjoin%
\definecolor{currentfill}{rgb}{0.124395,0.578002,0.548287}%
\pgfsetfillcolor{currentfill}%
\pgfsetfillopacity{0.700000}%
\pgfsetlinewidth{0.501875pt}%
\definecolor{currentstroke}{rgb}{1.000000,1.000000,1.000000}%
\pgfsetstrokecolor{currentstroke}%
\pgfsetstrokeopacity{0.500000}%
\pgfsetdash{}{0pt}%
\pgfpathmoveto{\pgfqpoint{3.152110in}{2.341338in}}%
\pgfpathlineto{\pgfqpoint{3.163546in}{2.361928in}}%
\pgfpathlineto{\pgfqpoint{3.174985in}{2.382206in}}%
\pgfpathlineto{\pgfqpoint{3.186428in}{2.402405in}}%
\pgfpathlineto{\pgfqpoint{3.197876in}{2.422673in}}%
\pgfpathlineto{\pgfqpoint{3.209327in}{2.442875in}}%
\pgfpathlineto{\pgfqpoint{3.203038in}{2.458607in}}%
\pgfpathlineto{\pgfqpoint{3.196749in}{2.473875in}}%
\pgfpathlineto{\pgfqpoint{3.190460in}{2.488576in}}%
\pgfpathlineto{\pgfqpoint{3.184171in}{2.502611in}}%
\pgfpathlineto{\pgfqpoint{3.177883in}{2.515936in}}%
\pgfpathlineto{\pgfqpoint{3.166440in}{2.493063in}}%
\pgfpathlineto{\pgfqpoint{3.155003in}{2.470241in}}%
\pgfpathlineto{\pgfqpoint{3.143573in}{2.447729in}}%
\pgfpathlineto{\pgfqpoint{3.132148in}{2.425376in}}%
\pgfpathlineto{\pgfqpoint{3.120728in}{2.402907in}}%
\pgfpathlineto{\pgfqpoint{3.126998in}{2.390388in}}%
\pgfpathlineto{\pgfqpoint{3.133271in}{2.378011in}}%
\pgfpathlineto{\pgfqpoint{3.139548in}{2.365751in}}%
\pgfpathlineto{\pgfqpoint{3.145827in}{2.353547in}}%
\pgfpathclose%
\pgfusepath{stroke,fill}%
\end{pgfscope}%
\begin{pgfscope}%
\pgfpathrectangle{\pgfqpoint{0.887500in}{0.275000in}}{\pgfqpoint{4.225000in}{4.225000in}}%
\pgfusepath{clip}%
\pgfsetbuttcap%
\pgfsetroundjoin%
\definecolor{currentfill}{rgb}{0.203063,0.379716,0.553925}%
\pgfsetfillcolor{currentfill}%
\pgfsetfillopacity{0.700000}%
\pgfsetlinewidth{0.501875pt}%
\definecolor{currentstroke}{rgb}{1.000000,1.000000,1.000000}%
\pgfsetstrokecolor{currentstroke}%
\pgfsetstrokeopacity{0.500000}%
\pgfsetdash{}{0pt}%
\pgfpathmoveto{\pgfqpoint{4.416755in}{1.962125in}}%
\pgfpathlineto{\pgfqpoint{4.427844in}{1.965658in}}%
\pgfpathlineto{\pgfqpoint{4.438926in}{1.969140in}}%
\pgfpathlineto{\pgfqpoint{4.450001in}{1.972573in}}%
\pgfpathlineto{\pgfqpoint{4.461069in}{1.975957in}}%
\pgfpathlineto{\pgfqpoint{4.472129in}{1.979297in}}%
\pgfpathlineto{\pgfqpoint{4.465610in}{1.993810in}}%
\pgfpathlineto{\pgfqpoint{4.459091in}{2.008222in}}%
\pgfpathlineto{\pgfqpoint{4.452569in}{2.022517in}}%
\pgfpathlineto{\pgfqpoint{4.446046in}{2.036683in}}%
\pgfpathlineto{\pgfqpoint{4.439520in}{2.050703in}}%
\pgfpathlineto{\pgfqpoint{4.428458in}{2.047249in}}%
\pgfpathlineto{\pgfqpoint{4.417388in}{2.043766in}}%
\pgfpathlineto{\pgfqpoint{4.406312in}{2.040247in}}%
\pgfpathlineto{\pgfqpoint{4.395229in}{2.036684in}}%
\pgfpathlineto{\pgfqpoint{4.384139in}{2.033067in}}%
\pgfpathlineto{\pgfqpoint{4.390664in}{2.019096in}}%
\pgfpathlineto{\pgfqpoint{4.397188in}{2.005006in}}%
\pgfpathlineto{\pgfqpoint{4.403711in}{1.990807in}}%
\pgfpathlineto{\pgfqpoint{4.410233in}{1.976510in}}%
\pgfpathclose%
\pgfusepath{stroke,fill}%
\end{pgfscope}%
\begin{pgfscope}%
\pgfpathrectangle{\pgfqpoint{0.887500in}{0.275000in}}{\pgfqpoint{4.225000in}{4.225000in}}%
\pgfusepath{clip}%
\pgfsetbuttcap%
\pgfsetroundjoin%
\definecolor{currentfill}{rgb}{0.151918,0.500685,0.557587}%
\pgfsetfillcolor{currentfill}%
\pgfsetfillopacity{0.700000}%
\pgfsetlinewidth{0.501875pt}%
\definecolor{currentstroke}{rgb}{1.000000,1.000000,1.000000}%
\pgfsetstrokecolor{currentstroke}%
\pgfsetstrokeopacity{0.500000}%
\pgfsetdash{}{0pt}%
\pgfpathmoveto{\pgfqpoint{3.126347in}{2.171995in}}%
\pgfpathlineto{\pgfqpoint{3.137786in}{2.195894in}}%
\pgfpathlineto{\pgfqpoint{3.149228in}{2.218419in}}%
\pgfpathlineto{\pgfqpoint{3.160672in}{2.239513in}}%
\pgfpathlineto{\pgfqpoint{3.172117in}{2.259339in}}%
\pgfpathlineto{\pgfqpoint{3.183562in}{2.278061in}}%
\pgfpathlineto{\pgfqpoint{3.177268in}{2.291219in}}%
\pgfpathlineto{\pgfqpoint{3.170975in}{2.304065in}}%
\pgfpathlineto{\pgfqpoint{3.164684in}{2.316658in}}%
\pgfpathlineto{\pgfqpoint{3.158396in}{2.329063in}}%
\pgfpathlineto{\pgfqpoint{3.152110in}{2.341338in}}%
\pgfpathlineto{\pgfqpoint{3.140677in}{2.320204in}}%
\pgfpathlineto{\pgfqpoint{3.129248in}{2.298293in}}%
\pgfpathlineto{\pgfqpoint{3.117822in}{2.275375in}}%
\pgfpathlineto{\pgfqpoint{3.106400in}{2.251218in}}%
\pgfpathlineto{\pgfqpoint{3.094983in}{2.225828in}}%
\pgfpathlineto{\pgfqpoint{3.101249in}{2.215351in}}%
\pgfpathlineto{\pgfqpoint{3.107519in}{2.204799in}}%
\pgfpathlineto{\pgfqpoint{3.113791in}{2.194103in}}%
\pgfpathlineto{\pgfqpoint{3.120068in}{2.183191in}}%
\pgfpathclose%
\pgfusepath{stroke,fill}%
\end{pgfscope}%
\begin{pgfscope}%
\pgfpathrectangle{\pgfqpoint{0.887500in}{0.275000in}}{\pgfqpoint{4.225000in}{4.225000in}}%
\pgfusepath{clip}%
\pgfsetbuttcap%
\pgfsetroundjoin%
\definecolor{currentfill}{rgb}{0.199430,0.387607,0.554642}%
\pgfsetfillcolor{currentfill}%
\pgfsetfillopacity{0.700000}%
\pgfsetlinewidth{0.501875pt}%
\definecolor{currentstroke}{rgb}{1.000000,1.000000,1.000000}%
\pgfsetstrokecolor{currentstroke}%
\pgfsetstrokeopacity{0.500000}%
\pgfsetdash{}{0pt}%
\pgfpathmoveto{\pgfqpoint{3.043468in}{1.978791in}}%
\pgfpathlineto{\pgfqpoint{3.054900in}{1.980522in}}%
\pgfpathlineto{\pgfqpoint{3.066325in}{1.983887in}}%
\pgfpathlineto{\pgfqpoint{3.077746in}{1.989630in}}%
\pgfpathlineto{\pgfqpoint{3.089165in}{1.998497in}}%
\pgfpathlineto{\pgfqpoint{3.100586in}{2.011106in}}%
\pgfpathlineto{\pgfqpoint{3.094311in}{2.021356in}}%
\pgfpathlineto{\pgfqpoint{3.088039in}{2.031529in}}%
\pgfpathlineto{\pgfqpoint{3.081771in}{2.041624in}}%
\pgfpathlineto{\pgfqpoint{3.075507in}{2.051642in}}%
\pgfpathlineto{\pgfqpoint{3.069246in}{2.061584in}}%
\pgfpathlineto{\pgfqpoint{3.057842in}{2.047649in}}%
\pgfpathlineto{\pgfqpoint{3.046437in}{2.038024in}}%
\pgfpathlineto{\pgfqpoint{3.035027in}{2.031996in}}%
\pgfpathlineto{\pgfqpoint{3.023612in}{2.028706in}}%
\pgfpathlineto{\pgfqpoint{3.012188in}{2.027299in}}%
\pgfpathlineto{\pgfqpoint{3.018436in}{2.017603in}}%
\pgfpathlineto{\pgfqpoint{3.024688in}{2.007903in}}%
\pgfpathlineto{\pgfqpoint{3.030944in}{1.998202in}}%
\pgfpathlineto{\pgfqpoint{3.037204in}{1.988499in}}%
\pgfpathclose%
\pgfusepath{stroke,fill}%
\end{pgfscope}%
\begin{pgfscope}%
\pgfpathrectangle{\pgfqpoint{0.887500in}{0.275000in}}{\pgfqpoint{4.225000in}{4.225000in}}%
\pgfusepath{clip}%
\pgfsetbuttcap%
\pgfsetroundjoin%
\definecolor{currentfill}{rgb}{0.120638,0.625828,0.533488}%
\pgfsetfillcolor{currentfill}%
\pgfsetfillopacity{0.700000}%
\pgfsetlinewidth{0.501875pt}%
\definecolor{currentstroke}{rgb}{1.000000,1.000000,1.000000}%
\pgfsetstrokecolor{currentstroke}%
\pgfsetstrokeopacity{0.500000}%
\pgfsetdash{}{0pt}%
\pgfpathmoveto{\pgfqpoint{3.734760in}{2.480734in}}%
\pgfpathlineto{\pgfqpoint{3.746030in}{2.484820in}}%
\pgfpathlineto{\pgfqpoint{3.757294in}{2.488818in}}%
\pgfpathlineto{\pgfqpoint{3.768551in}{2.492726in}}%
\pgfpathlineto{\pgfqpoint{3.779802in}{2.496549in}}%
\pgfpathlineto{\pgfqpoint{3.791045in}{2.500286in}}%
\pgfpathlineto{\pgfqpoint{3.784642in}{2.514228in}}%
\pgfpathlineto{\pgfqpoint{3.778241in}{2.528169in}}%
\pgfpathlineto{\pgfqpoint{3.771842in}{2.542127in}}%
\pgfpathlineto{\pgfqpoint{3.765447in}{2.556113in}}%
\pgfpathlineto{\pgfqpoint{3.759055in}{2.570122in}}%
\pgfpathlineto{\pgfqpoint{3.747816in}{2.566422in}}%
\pgfpathlineto{\pgfqpoint{3.736569in}{2.562656in}}%
\pgfpathlineto{\pgfqpoint{3.725317in}{2.558832in}}%
\pgfpathlineto{\pgfqpoint{3.714058in}{2.554961in}}%
\pgfpathlineto{\pgfqpoint{3.702794in}{2.551054in}}%
\pgfpathlineto{\pgfqpoint{3.709181in}{2.536945in}}%
\pgfpathlineto{\pgfqpoint{3.715571in}{2.522861in}}%
\pgfpathlineto{\pgfqpoint{3.721965in}{2.508808in}}%
\pgfpathlineto{\pgfqpoint{3.728361in}{2.494772in}}%
\pgfpathclose%
\pgfusepath{stroke,fill}%
\end{pgfscope}%
\begin{pgfscope}%
\pgfpathrectangle{\pgfqpoint{0.887500in}{0.275000in}}{\pgfqpoint{4.225000in}{4.225000in}}%
\pgfusepath{clip}%
\pgfsetbuttcap%
\pgfsetroundjoin%
\definecolor{currentfill}{rgb}{0.220124,0.725509,0.466226}%
\pgfsetfillcolor{currentfill}%
\pgfsetfillopacity{0.700000}%
\pgfsetlinewidth{0.501875pt}%
\definecolor{currentstroke}{rgb}{1.000000,1.000000,1.000000}%
\pgfsetstrokecolor{currentstroke}%
\pgfsetstrokeopacity{0.500000}%
\pgfsetdash{}{0pt}%
\pgfpathmoveto{\pgfqpoint{3.292266in}{2.685586in}}%
\pgfpathlineto{\pgfqpoint{3.303674in}{2.695769in}}%
\pgfpathlineto{\pgfqpoint{3.315078in}{2.705456in}}%
\pgfpathlineto{\pgfqpoint{3.326476in}{2.714776in}}%
\pgfpathlineto{\pgfqpoint{3.337870in}{2.723702in}}%
\pgfpathlineto{\pgfqpoint{3.349258in}{2.732192in}}%
\pgfpathlineto{\pgfqpoint{3.342936in}{2.745869in}}%
\pgfpathlineto{\pgfqpoint{3.336614in}{2.759167in}}%
\pgfpathlineto{\pgfqpoint{3.330293in}{2.772171in}}%
\pgfpathlineto{\pgfqpoint{3.323973in}{2.784968in}}%
\pgfpathlineto{\pgfqpoint{3.317655in}{2.797644in}}%
\pgfpathlineto{\pgfqpoint{3.306272in}{2.788608in}}%
\pgfpathlineto{\pgfqpoint{3.294883in}{2.778932in}}%
\pgfpathlineto{\pgfqpoint{3.283488in}{2.768618in}}%
\pgfpathlineto{\pgfqpoint{3.272088in}{2.757662in}}%
\pgfpathlineto{\pgfqpoint{3.260683in}{2.746005in}}%
\pgfpathlineto{\pgfqpoint{3.266998in}{2.734648in}}%
\pgfpathlineto{\pgfqpoint{3.273314in}{2.723116in}}%
\pgfpathlineto{\pgfqpoint{3.279632in}{2.711222in}}%
\pgfpathlineto{\pgfqpoint{3.285950in}{2.698775in}}%
\pgfpathclose%
\pgfusepath{stroke,fill}%
\end{pgfscope}%
\begin{pgfscope}%
\pgfpathrectangle{\pgfqpoint{0.887500in}{0.275000in}}{\pgfqpoint{4.225000in}{4.225000in}}%
\pgfusepath{clip}%
\pgfsetbuttcap%
\pgfsetroundjoin%
\definecolor{currentfill}{rgb}{0.175841,0.441290,0.557685}%
\pgfsetfillcolor{currentfill}%
\pgfsetfillopacity{0.700000}%
\pgfsetlinewidth{0.501875pt}%
\definecolor{currentstroke}{rgb}{1.000000,1.000000,1.000000}%
\pgfsetstrokecolor{currentstroke}%
\pgfsetstrokeopacity{0.500000}%
\pgfsetdash{}{0pt}%
\pgfpathmoveto{\pgfqpoint{2.632196in}{2.092319in}}%
\pgfpathlineto{\pgfqpoint{2.643730in}{2.095934in}}%
\pgfpathlineto{\pgfqpoint{2.655259in}{2.099571in}}%
\pgfpathlineto{\pgfqpoint{2.666783in}{2.103224in}}%
\pgfpathlineto{\pgfqpoint{2.678300in}{2.106869in}}%
\pgfpathlineto{\pgfqpoint{2.689813in}{2.110479in}}%
\pgfpathlineto{\pgfqpoint{2.683667in}{2.119828in}}%
\pgfpathlineto{\pgfqpoint{2.677525in}{2.129140in}}%
\pgfpathlineto{\pgfqpoint{2.671387in}{2.138415in}}%
\pgfpathlineto{\pgfqpoint{2.665253in}{2.147655in}}%
\pgfpathlineto{\pgfqpoint{2.659124in}{2.156861in}}%
\pgfpathlineto{\pgfqpoint{2.647622in}{2.153276in}}%
\pgfpathlineto{\pgfqpoint{2.636115in}{2.149658in}}%
\pgfpathlineto{\pgfqpoint{2.624603in}{2.146034in}}%
\pgfpathlineto{\pgfqpoint{2.613084in}{2.142429in}}%
\pgfpathlineto{\pgfqpoint{2.601560in}{2.138850in}}%
\pgfpathlineto{\pgfqpoint{2.607679in}{2.129615in}}%
\pgfpathlineto{\pgfqpoint{2.613802in}{2.120346in}}%
\pgfpathlineto{\pgfqpoint{2.619929in}{2.111041in}}%
\pgfpathlineto{\pgfqpoint{2.626060in}{2.101699in}}%
\pgfpathclose%
\pgfusepath{stroke,fill}%
\end{pgfscope}%
\begin{pgfscope}%
\pgfpathrectangle{\pgfqpoint{0.887500in}{0.275000in}}{\pgfqpoint{4.225000in}{4.225000in}}%
\pgfusepath{clip}%
\pgfsetbuttcap%
\pgfsetroundjoin%
\definecolor{currentfill}{rgb}{0.246811,0.283237,0.535941}%
\pgfsetfillcolor{currentfill}%
\pgfsetfillopacity{0.700000}%
\pgfsetlinewidth{0.501875pt}%
\definecolor{currentstroke}{rgb}{1.000000,1.000000,1.000000}%
\pgfsetstrokecolor{currentstroke}%
\pgfsetstrokeopacity{0.500000}%
\pgfsetdash{}{0pt}%
\pgfpathmoveto{\pgfqpoint{4.625348in}{1.775542in}}%
\pgfpathlineto{\pgfqpoint{4.636375in}{1.778803in}}%
\pgfpathlineto{\pgfqpoint{4.647396in}{1.782064in}}%
\pgfpathlineto{\pgfqpoint{4.658411in}{1.785326in}}%
\pgfpathlineto{\pgfqpoint{4.669422in}{1.788589in}}%
\pgfpathlineto{\pgfqpoint{4.680426in}{1.791855in}}%
\pgfpathlineto{\pgfqpoint{4.673879in}{1.806654in}}%
\pgfpathlineto{\pgfqpoint{4.667333in}{1.821433in}}%
\pgfpathlineto{\pgfqpoint{4.660789in}{1.836194in}}%
\pgfpathlineto{\pgfqpoint{4.654246in}{1.850935in}}%
\pgfpathlineto{\pgfqpoint{4.647706in}{1.865658in}}%
\pgfpathlineto{\pgfqpoint{4.636698in}{1.862306in}}%
\pgfpathlineto{\pgfqpoint{4.625685in}{1.858945in}}%
\pgfpathlineto{\pgfqpoint{4.614665in}{1.855570in}}%
\pgfpathlineto{\pgfqpoint{4.603640in}{1.852181in}}%
\pgfpathlineto{\pgfqpoint{4.592608in}{1.848775in}}%
\pgfpathlineto{\pgfqpoint{4.599148in}{1.834072in}}%
\pgfpathlineto{\pgfqpoint{4.605693in}{1.819392in}}%
\pgfpathlineto{\pgfqpoint{4.612240in}{1.804739in}}%
\pgfpathlineto{\pgfqpoint{4.618792in}{1.790120in}}%
\pgfpathclose%
\pgfusepath{stroke,fill}%
\end{pgfscope}%
\begin{pgfscope}%
\pgfpathrectangle{\pgfqpoint{0.887500in}{0.275000in}}{\pgfqpoint{4.225000in}{4.225000in}}%
\pgfusepath{clip}%
\pgfsetbuttcap%
\pgfsetroundjoin%
\definecolor{currentfill}{rgb}{0.183898,0.422383,0.556944}%
\pgfsetfillcolor{currentfill}%
\pgfsetfillopacity{0.700000}%
\pgfsetlinewidth{0.501875pt}%
\definecolor{currentstroke}{rgb}{1.000000,1.000000,1.000000}%
\pgfsetstrokecolor{currentstroke}%
\pgfsetstrokeopacity{0.500000}%
\pgfsetdash{}{0pt}%
\pgfpathmoveto{\pgfqpoint{3.100586in}{2.011106in}}%
\pgfpathlineto{\pgfqpoint{3.112011in}{2.027123in}}%
\pgfpathlineto{\pgfqpoint{3.123443in}{2.045777in}}%
\pgfpathlineto{\pgfqpoint{3.134881in}{2.066293in}}%
\pgfpathlineto{\pgfqpoint{3.146326in}{2.087898in}}%
\pgfpathlineto{\pgfqpoint{3.157776in}{2.109812in}}%
\pgfpathlineto{\pgfqpoint{3.151487in}{2.123138in}}%
\pgfpathlineto{\pgfqpoint{3.145199in}{2.136029in}}%
\pgfpathlineto{\pgfqpoint{3.138912in}{2.148468in}}%
\pgfpathlineto{\pgfqpoint{3.132628in}{2.160444in}}%
\pgfpathlineto{\pgfqpoint{3.126347in}{2.171995in}}%
\pgfpathlineto{\pgfqpoint{3.114913in}{2.147491in}}%
\pgfpathlineto{\pgfqpoint{3.103486in}{2.123292in}}%
\pgfpathlineto{\pgfqpoint{3.092066in}{2.100306in}}%
\pgfpathlineto{\pgfqpoint{3.080653in}{2.079436in}}%
\pgfpathlineto{\pgfqpoint{3.069246in}{2.061584in}}%
\pgfpathlineto{\pgfqpoint{3.075507in}{2.051642in}}%
\pgfpathlineto{\pgfqpoint{3.081771in}{2.041624in}}%
\pgfpathlineto{\pgfqpoint{3.088039in}{2.031529in}}%
\pgfpathlineto{\pgfqpoint{3.094311in}{2.021356in}}%
\pgfpathclose%
\pgfusepath{stroke,fill}%
\end{pgfscope}%
\begin{pgfscope}%
\pgfpathrectangle{\pgfqpoint{0.887500in}{0.275000in}}{\pgfqpoint{4.225000in}{4.225000in}}%
\pgfusepath{clip}%
\pgfsetbuttcap%
\pgfsetroundjoin%
\definecolor{currentfill}{rgb}{0.144759,0.519093,0.556572}%
\pgfsetfillcolor{currentfill}%
\pgfsetfillopacity{0.700000}%
\pgfsetlinewidth{0.501875pt}%
\definecolor{currentstroke}{rgb}{1.000000,1.000000,1.000000}%
\pgfsetstrokecolor{currentstroke}%
\pgfsetstrokeopacity{0.500000}%
\pgfsetdash{}{0pt}%
\pgfpathmoveto{\pgfqpoint{4.031698in}{2.249513in}}%
\pgfpathlineto{\pgfqpoint{4.042884in}{2.252941in}}%
\pgfpathlineto{\pgfqpoint{4.054063in}{2.256339in}}%
\pgfpathlineto{\pgfqpoint{4.065237in}{2.259731in}}%
\pgfpathlineto{\pgfqpoint{4.076406in}{2.263134in}}%
\pgfpathlineto{\pgfqpoint{4.087570in}{2.266546in}}%
\pgfpathlineto{\pgfqpoint{4.081115in}{2.280819in}}%
\pgfpathlineto{\pgfqpoint{4.074664in}{2.295113in}}%
\pgfpathlineto{\pgfqpoint{4.068216in}{2.309445in}}%
\pgfpathlineto{\pgfqpoint{4.061771in}{2.323810in}}%
\pgfpathlineto{\pgfqpoint{4.055330in}{2.338199in}}%
\pgfpathlineto{\pgfqpoint{4.044166in}{2.334697in}}%
\pgfpathlineto{\pgfqpoint{4.032997in}{2.331204in}}%
\pgfpathlineto{\pgfqpoint{4.021823in}{2.327725in}}%
\pgfpathlineto{\pgfqpoint{4.010643in}{2.324251in}}%
\pgfpathlineto{\pgfqpoint{3.999458in}{2.320766in}}%
\pgfpathlineto{\pgfqpoint{4.005898in}{2.306391in}}%
\pgfpathlineto{\pgfqpoint{4.012341in}{2.292070in}}%
\pgfpathlineto{\pgfqpoint{4.018789in}{2.277818in}}%
\pgfpathlineto{\pgfqpoint{4.025242in}{2.263641in}}%
\pgfpathclose%
\pgfusepath{stroke,fill}%
\end{pgfscope}%
\begin{pgfscope}%
\pgfpathrectangle{\pgfqpoint{0.887500in}{0.275000in}}{\pgfqpoint{4.225000in}{4.225000in}}%
\pgfusepath{clip}%
\pgfsetbuttcap%
\pgfsetroundjoin%
\definecolor{currentfill}{rgb}{0.190631,0.407061,0.556089}%
\pgfsetfillcolor{currentfill}%
\pgfsetfillopacity{0.700000}%
\pgfsetlinewidth{0.501875pt}%
\definecolor{currentstroke}{rgb}{1.000000,1.000000,1.000000}%
\pgfsetstrokecolor{currentstroke}%
\pgfsetstrokeopacity{0.500000}%
\pgfsetdash{}{0pt}%
\pgfpathmoveto{\pgfqpoint{4.328571in}{2.013871in}}%
\pgfpathlineto{\pgfqpoint{4.339701in}{2.017884in}}%
\pgfpathlineto{\pgfqpoint{4.350822in}{2.021804in}}%
\pgfpathlineto{\pgfqpoint{4.361936in}{2.025636in}}%
\pgfpathlineto{\pgfqpoint{4.373041in}{2.029387in}}%
\pgfpathlineto{\pgfqpoint{4.384139in}{2.033067in}}%
\pgfpathlineto{\pgfqpoint{4.377612in}{2.046907in}}%
\pgfpathlineto{\pgfqpoint{4.371083in}{2.060614in}}%
\pgfpathlineto{\pgfqpoint{4.364554in}{2.074218in}}%
\pgfpathlineto{\pgfqpoint{4.358025in}{2.087751in}}%
\pgfpathlineto{\pgfqpoint{4.351498in}{2.101250in}}%
\pgfpathlineto{\pgfqpoint{4.340415in}{2.097916in}}%
\pgfpathlineto{\pgfqpoint{4.329325in}{2.094551in}}%
\pgfpathlineto{\pgfqpoint{4.318227in}{2.091117in}}%
\pgfpathlineto{\pgfqpoint{4.307120in}{2.087577in}}%
\pgfpathlineto{\pgfqpoint{4.296005in}{2.083920in}}%
\pgfpathlineto{\pgfqpoint{4.302512in}{2.069884in}}%
\pgfpathlineto{\pgfqpoint{4.309022in}{2.055868in}}%
\pgfpathlineto{\pgfqpoint{4.315536in}{2.041866in}}%
\pgfpathlineto{\pgfqpoint{4.322052in}{2.027869in}}%
\pgfpathclose%
\pgfusepath{stroke,fill}%
\end{pgfscope}%
\begin{pgfscope}%
\pgfpathrectangle{\pgfqpoint{0.887500in}{0.275000in}}{\pgfqpoint{4.225000in}{4.225000in}}%
\pgfusepath{clip}%
\pgfsetbuttcap%
\pgfsetroundjoin%
\definecolor{currentfill}{rgb}{0.163625,0.471133,0.558148}%
\pgfsetfillcolor{currentfill}%
\pgfsetfillopacity{0.700000}%
\pgfsetlinewidth{0.501875pt}%
\definecolor{currentstroke}{rgb}{1.000000,1.000000,1.000000}%
\pgfsetstrokecolor{currentstroke}%
\pgfsetstrokeopacity{0.500000}%
\pgfsetdash{}{0pt}%
\pgfpathmoveto{\pgfqpoint{2.309340in}{2.160009in}}%
\pgfpathlineto{\pgfqpoint{2.320955in}{2.163596in}}%
\pgfpathlineto{\pgfqpoint{2.332566in}{2.167170in}}%
\pgfpathlineto{\pgfqpoint{2.344170in}{2.170731in}}%
\pgfpathlineto{\pgfqpoint{2.355770in}{2.174276in}}%
\pgfpathlineto{\pgfqpoint{2.367364in}{2.177805in}}%
\pgfpathlineto{\pgfqpoint{2.361324in}{2.186820in}}%
\pgfpathlineto{\pgfqpoint{2.355288in}{2.195807in}}%
\pgfpathlineto{\pgfqpoint{2.349257in}{2.204766in}}%
\pgfpathlineto{\pgfqpoint{2.343230in}{2.213697in}}%
\pgfpathlineto{\pgfqpoint{2.337207in}{2.222601in}}%
\pgfpathlineto{\pgfqpoint{2.325624in}{2.219121in}}%
\pgfpathlineto{\pgfqpoint{2.314035in}{2.215625in}}%
\pgfpathlineto{\pgfqpoint{2.302441in}{2.212114in}}%
\pgfpathlineto{\pgfqpoint{2.290842in}{2.208589in}}%
\pgfpathlineto{\pgfqpoint{2.279237in}{2.205052in}}%
\pgfpathlineto{\pgfqpoint{2.285249in}{2.196100in}}%
\pgfpathlineto{\pgfqpoint{2.291265in}{2.187120in}}%
\pgfpathlineto{\pgfqpoint{2.297286in}{2.178112in}}%
\pgfpathlineto{\pgfqpoint{2.303310in}{2.169075in}}%
\pgfpathclose%
\pgfusepath{stroke,fill}%
\end{pgfscope}%
\begin{pgfscope}%
\pgfpathrectangle{\pgfqpoint{0.887500in}{0.275000in}}{\pgfqpoint{4.225000in}{4.225000in}}%
\pgfusepath{clip}%
\pgfsetbuttcap%
\pgfsetroundjoin%
\definecolor{currentfill}{rgb}{0.130067,0.651384,0.521608}%
\pgfsetfillcolor{currentfill}%
\pgfsetfillopacity{0.700000}%
\pgfsetlinewidth{0.501875pt}%
\definecolor{currentstroke}{rgb}{1.000000,1.000000,1.000000}%
\pgfsetstrokecolor{currentstroke}%
\pgfsetstrokeopacity{0.500000}%
\pgfsetdash{}{0pt}%
\pgfpathmoveto{\pgfqpoint{3.646388in}{2.531233in}}%
\pgfpathlineto{\pgfqpoint{3.657681in}{2.535255in}}%
\pgfpathlineto{\pgfqpoint{3.668967in}{2.539220in}}%
\pgfpathlineto{\pgfqpoint{3.680248in}{2.543173in}}%
\pgfpathlineto{\pgfqpoint{3.691523in}{2.547121in}}%
\pgfpathlineto{\pgfqpoint{3.702794in}{2.551054in}}%
\pgfpathlineto{\pgfqpoint{3.696409in}{2.565177in}}%
\pgfpathlineto{\pgfqpoint{3.690027in}{2.579306in}}%
\pgfpathlineto{\pgfqpoint{3.683647in}{2.593429in}}%
\pgfpathlineto{\pgfqpoint{3.677270in}{2.607539in}}%
\pgfpathlineto{\pgfqpoint{3.670895in}{2.621625in}}%
\pgfpathlineto{\pgfqpoint{3.659630in}{2.617817in}}%
\pgfpathlineto{\pgfqpoint{3.648359in}{2.614016in}}%
\pgfpathlineto{\pgfqpoint{3.637084in}{2.610238in}}%
\pgfpathlineto{\pgfqpoint{3.625803in}{2.606472in}}%
\pgfpathlineto{\pgfqpoint{3.614517in}{2.602657in}}%
\pgfpathlineto{\pgfqpoint{3.620888in}{2.588470in}}%
\pgfpathlineto{\pgfqpoint{3.627260in}{2.574230in}}%
\pgfpathlineto{\pgfqpoint{3.633634in}{2.559942in}}%
\pgfpathlineto{\pgfqpoint{3.640010in}{2.545608in}}%
\pgfpathclose%
\pgfusepath{stroke,fill}%
\end{pgfscope}%
\begin{pgfscope}%
\pgfpathrectangle{\pgfqpoint{0.887500in}{0.275000in}}{\pgfqpoint{4.225000in}{4.225000in}}%
\pgfusepath{clip}%
\pgfsetbuttcap%
\pgfsetroundjoin%
\definecolor{currentfill}{rgb}{0.140536,0.530132,0.555659}%
\pgfsetfillcolor{currentfill}%
\pgfsetfillopacity{0.700000}%
\pgfsetlinewidth{0.501875pt}%
\definecolor{currentstroke}{rgb}{1.000000,1.000000,1.000000}%
\pgfsetstrokecolor{currentstroke}%
\pgfsetstrokeopacity{0.500000}%
\pgfsetdash{}{0pt}%
\pgfpathmoveto{\pgfqpoint{1.663683in}{2.285684in}}%
\pgfpathlineto{\pgfqpoint{1.675460in}{2.289173in}}%
\pgfpathlineto{\pgfqpoint{1.687231in}{2.292650in}}%
\pgfpathlineto{\pgfqpoint{1.698997in}{2.296116in}}%
\pgfpathlineto{\pgfqpoint{1.710757in}{2.299571in}}%
\pgfpathlineto{\pgfqpoint{1.722512in}{2.303017in}}%
\pgfpathlineto{\pgfqpoint{1.716693in}{2.311620in}}%
\pgfpathlineto{\pgfqpoint{1.710878in}{2.320198in}}%
\pgfpathlineto{\pgfqpoint{1.705067in}{2.328751in}}%
\pgfpathlineto{\pgfqpoint{1.699262in}{2.337279in}}%
\pgfpathlineto{\pgfqpoint{1.693460in}{2.345781in}}%
\pgfpathlineto{\pgfqpoint{1.681717in}{2.342370in}}%
\pgfpathlineto{\pgfqpoint{1.669969in}{2.338951in}}%
\pgfpathlineto{\pgfqpoint{1.658215in}{2.335521in}}%
\pgfpathlineto{\pgfqpoint{1.646455in}{2.332080in}}%
\pgfpathlineto{\pgfqpoint{1.634690in}{2.328627in}}%
\pgfpathlineto{\pgfqpoint{1.640479in}{2.320093in}}%
\pgfpathlineto{\pgfqpoint{1.646273in}{2.311531in}}%
\pgfpathlineto{\pgfqpoint{1.652072in}{2.302943in}}%
\pgfpathlineto{\pgfqpoint{1.657875in}{2.294327in}}%
\pgfpathclose%
\pgfusepath{stroke,fill}%
\end{pgfscope}%
\begin{pgfscope}%
\pgfpathrectangle{\pgfqpoint{0.887500in}{0.275000in}}{\pgfqpoint{4.225000in}{4.225000in}}%
\pgfusepath{clip}%
\pgfsetbuttcap%
\pgfsetroundjoin%
\definecolor{currentfill}{rgb}{0.151918,0.500685,0.557587}%
\pgfsetfillcolor{currentfill}%
\pgfsetfillopacity{0.700000}%
\pgfsetlinewidth{0.501875pt}%
\definecolor{currentstroke}{rgb}{1.000000,1.000000,1.000000}%
\pgfsetstrokecolor{currentstroke}%
\pgfsetstrokeopacity{0.500000}%
\pgfsetdash{}{0pt}%
\pgfpathmoveto{\pgfqpoint{1.986485in}{2.224099in}}%
\pgfpathlineto{\pgfqpoint{1.998183in}{2.227605in}}%
\pgfpathlineto{\pgfqpoint{2.009874in}{2.231098in}}%
\pgfpathlineto{\pgfqpoint{2.021561in}{2.234579in}}%
\pgfpathlineto{\pgfqpoint{2.033241in}{2.238049in}}%
\pgfpathlineto{\pgfqpoint{2.044917in}{2.241508in}}%
\pgfpathlineto{\pgfqpoint{2.038985in}{2.250313in}}%
\pgfpathlineto{\pgfqpoint{2.033059in}{2.259093in}}%
\pgfpathlineto{\pgfqpoint{2.027136in}{2.267849in}}%
\pgfpathlineto{\pgfqpoint{2.021218in}{2.276580in}}%
\pgfpathlineto{\pgfqpoint{2.015304in}{2.285286in}}%
\pgfpathlineto{\pgfqpoint{2.003640in}{2.281877in}}%
\pgfpathlineto{\pgfqpoint{1.991970in}{2.278456in}}%
\pgfpathlineto{\pgfqpoint{1.980295in}{2.275024in}}%
\pgfpathlineto{\pgfqpoint{1.968614in}{2.271578in}}%
\pgfpathlineto{\pgfqpoint{1.956928in}{2.268119in}}%
\pgfpathlineto{\pgfqpoint{1.962831in}{2.259368in}}%
\pgfpathlineto{\pgfqpoint{1.968738in}{2.250590in}}%
\pgfpathlineto{\pgfqpoint{1.974649in}{2.241786in}}%
\pgfpathlineto{\pgfqpoint{1.980565in}{2.232955in}}%
\pgfpathclose%
\pgfusepath{stroke,fill}%
\end{pgfscope}%
\begin{pgfscope}%
\pgfpathrectangle{\pgfqpoint{0.887500in}{0.275000in}}{\pgfqpoint{4.225000in}{4.225000in}}%
\pgfusepath{clip}%
\pgfsetbuttcap%
\pgfsetroundjoin%
\definecolor{currentfill}{rgb}{0.180653,0.701402,0.488189}%
\pgfsetfillcolor{currentfill}%
\pgfsetfillopacity{0.700000}%
\pgfsetlinewidth{0.501875pt}%
\definecolor{currentstroke}{rgb}{1.000000,1.000000,1.000000}%
\pgfsetstrokecolor{currentstroke}%
\pgfsetstrokeopacity{0.500000}%
\pgfsetdash{}{0pt}%
\pgfpathmoveto{\pgfqpoint{3.235130in}{2.619098in}}%
\pgfpathlineto{\pgfqpoint{3.246571in}{2.635249in}}%
\pgfpathlineto{\pgfqpoint{3.258005in}{2.649741in}}%
\pgfpathlineto{\pgfqpoint{3.269432in}{2.662806in}}%
\pgfpathlineto{\pgfqpoint{3.280852in}{2.674677in}}%
\pgfpathlineto{\pgfqpoint{3.292266in}{2.685586in}}%
\pgfpathlineto{\pgfqpoint{3.285950in}{2.698775in}}%
\pgfpathlineto{\pgfqpoint{3.279632in}{2.711222in}}%
\pgfpathlineto{\pgfqpoint{3.273314in}{2.723116in}}%
\pgfpathlineto{\pgfqpoint{3.266998in}{2.734648in}}%
\pgfpathlineto{\pgfqpoint{3.260683in}{2.746005in}}%
\pgfpathlineto{\pgfqpoint{3.249274in}{2.733552in}}%
\pgfpathlineto{\pgfqpoint{3.237860in}{2.720206in}}%
\pgfpathlineto{\pgfqpoint{3.226442in}{2.705872in}}%
\pgfpathlineto{\pgfqpoint{3.215020in}{2.690454in}}%
\pgfpathlineto{\pgfqpoint{3.203595in}{2.673857in}}%
\pgfpathlineto{\pgfqpoint{3.209899in}{2.663905in}}%
\pgfpathlineto{\pgfqpoint{3.216206in}{2.653729in}}%
\pgfpathlineto{\pgfqpoint{3.222514in}{2.643054in}}%
\pgfpathlineto{\pgfqpoint{3.228823in}{2.631602in}}%
\pgfpathclose%
\pgfusepath{stroke,fill}%
\end{pgfscope}%
\begin{pgfscope}%
\pgfpathrectangle{\pgfqpoint{0.887500in}{0.275000in}}{\pgfqpoint{4.225000in}{4.225000in}}%
\pgfusepath{clip}%
\pgfsetbuttcap%
\pgfsetroundjoin%
\definecolor{currentfill}{rgb}{0.179019,0.433756,0.557430}%
\pgfsetfillcolor{currentfill}%
\pgfsetfillopacity{0.700000}%
\pgfsetlinewidth{0.501875pt}%
\definecolor{currentstroke}{rgb}{1.000000,1.000000,1.000000}%
\pgfsetstrokecolor{currentstroke}%
\pgfsetstrokeopacity{0.500000}%
\pgfsetdash{}{0pt}%
\pgfpathmoveto{\pgfqpoint{4.240304in}{2.064261in}}%
\pgfpathlineto{\pgfqpoint{4.251459in}{2.068339in}}%
\pgfpathlineto{\pgfqpoint{4.262607in}{2.072355in}}%
\pgfpathlineto{\pgfqpoint{4.273748in}{2.076298in}}%
\pgfpathlineto{\pgfqpoint{4.284880in}{2.080157in}}%
\pgfpathlineto{\pgfqpoint{4.296005in}{2.083920in}}%
\pgfpathlineto{\pgfqpoint{4.289501in}{2.097984in}}%
\pgfpathlineto{\pgfqpoint{4.283002in}{2.112083in}}%
\pgfpathlineto{\pgfqpoint{4.276506in}{2.126225in}}%
\pgfpathlineto{\pgfqpoint{4.270015in}{2.140417in}}%
\pgfpathlineto{\pgfqpoint{4.263528in}{2.154666in}}%
\pgfpathlineto{\pgfqpoint{4.252423in}{2.151457in}}%
\pgfpathlineto{\pgfqpoint{4.241309in}{2.148161in}}%
\pgfpathlineto{\pgfqpoint{4.230188in}{2.144792in}}%
\pgfpathlineto{\pgfqpoint{4.219059in}{2.141364in}}%
\pgfpathlineto{\pgfqpoint{4.207924in}{2.137894in}}%
\pgfpathlineto{\pgfqpoint{4.214394in}{2.123134in}}%
\pgfpathlineto{\pgfqpoint{4.220867in}{2.108370in}}%
\pgfpathlineto{\pgfqpoint{4.227342in}{2.093624in}}%
\pgfpathlineto{\pgfqpoint{4.233821in}{2.078914in}}%
\pgfpathclose%
\pgfusepath{stroke,fill}%
\end{pgfscope}%
\begin{pgfscope}%
\pgfpathrectangle{\pgfqpoint{0.887500in}{0.275000in}}{\pgfqpoint{4.225000in}{4.225000in}}%
\pgfusepath{clip}%
\pgfsetbuttcap%
\pgfsetroundjoin%
\definecolor{currentfill}{rgb}{0.135066,0.544853,0.554029}%
\pgfsetfillcolor{currentfill}%
\pgfsetfillopacity{0.700000}%
\pgfsetlinewidth{0.501875pt}%
\definecolor{currentstroke}{rgb}{1.000000,1.000000,1.000000}%
\pgfsetstrokecolor{currentstroke}%
\pgfsetstrokeopacity{0.500000}%
\pgfsetdash{}{0pt}%
\pgfpathmoveto{\pgfqpoint{3.943434in}{2.302660in}}%
\pgfpathlineto{\pgfqpoint{3.954653in}{2.306422in}}%
\pgfpathlineto{\pgfqpoint{3.965864in}{2.310099in}}%
\pgfpathlineto{\pgfqpoint{3.977069in}{2.313705in}}%
\pgfpathlineto{\pgfqpoint{3.988267in}{2.317256in}}%
\pgfpathlineto{\pgfqpoint{3.999458in}{2.320766in}}%
\pgfpathlineto{\pgfqpoint{3.993021in}{2.335183in}}%
\pgfpathlineto{\pgfqpoint{3.986588in}{2.349627in}}%
\pgfpathlineto{\pgfqpoint{3.980158in}{2.364085in}}%
\pgfpathlineto{\pgfqpoint{3.973730in}{2.378544in}}%
\pgfpathlineto{\pgfqpoint{3.967304in}{2.392990in}}%
\pgfpathlineto{\pgfqpoint{3.956117in}{2.389553in}}%
\pgfpathlineto{\pgfqpoint{3.944923in}{2.386106in}}%
\pgfpathlineto{\pgfqpoint{3.933724in}{2.382643in}}%
\pgfpathlineto{\pgfqpoint{3.922519in}{2.379157in}}%
\pgfpathlineto{\pgfqpoint{3.911307in}{2.375642in}}%
\pgfpathlineto{\pgfqpoint{3.917728in}{2.361051in}}%
\pgfpathlineto{\pgfqpoint{3.924151in}{2.346437in}}%
\pgfpathlineto{\pgfqpoint{3.930575in}{2.331820in}}%
\pgfpathlineto{\pgfqpoint{3.937003in}{2.317221in}}%
\pgfpathclose%
\pgfusepath{stroke,fill}%
\end{pgfscope}%
\begin{pgfscope}%
\pgfpathrectangle{\pgfqpoint{0.887500in}{0.275000in}}{\pgfqpoint{4.225000in}{4.225000in}}%
\pgfusepath{clip}%
\pgfsetbuttcap%
\pgfsetroundjoin%
\definecolor{currentfill}{rgb}{0.150148,0.676631,0.506589}%
\pgfsetfillcolor{currentfill}%
\pgfsetfillopacity{0.700000}%
\pgfsetlinewidth{0.501875pt}%
\definecolor{currentstroke}{rgb}{1.000000,1.000000,1.000000}%
\pgfsetstrokecolor{currentstroke}%
\pgfsetstrokeopacity{0.500000}%
\pgfsetdash{}{0pt}%
\pgfpathmoveto{\pgfqpoint{3.557966in}{2.580507in}}%
\pgfpathlineto{\pgfqpoint{3.569294in}{2.585570in}}%
\pgfpathlineto{\pgfqpoint{3.580613in}{2.590250in}}%
\pgfpathlineto{\pgfqpoint{3.591922in}{2.594613in}}%
\pgfpathlineto{\pgfqpoint{3.603223in}{2.598726in}}%
\pgfpathlineto{\pgfqpoint{3.614517in}{2.602657in}}%
\pgfpathlineto{\pgfqpoint{3.608148in}{2.616788in}}%
\pgfpathlineto{\pgfqpoint{3.601780in}{2.630859in}}%
\pgfpathlineto{\pgfqpoint{3.595415in}{2.644867in}}%
\pgfpathlineto{\pgfqpoint{3.589051in}{2.658808in}}%
\pgfpathlineto{\pgfqpoint{3.582688in}{2.672684in}}%
\pgfpathlineto{\pgfqpoint{3.571399in}{2.668808in}}%
\pgfpathlineto{\pgfqpoint{3.560102in}{2.664757in}}%
\pgfpathlineto{\pgfqpoint{3.548797in}{2.660469in}}%
\pgfpathlineto{\pgfqpoint{3.537483in}{2.655883in}}%
\pgfpathlineto{\pgfqpoint{3.526160in}{2.650935in}}%
\pgfpathlineto{\pgfqpoint{3.532516in}{2.636860in}}%
\pgfpathlineto{\pgfqpoint{3.538875in}{2.622788in}}%
\pgfpathlineto{\pgfqpoint{3.545236in}{2.608718in}}%
\pgfpathlineto{\pgfqpoint{3.551600in}{2.594631in}}%
\pgfpathclose%
\pgfusepath{stroke,fill}%
\end{pgfscope}%
\begin{pgfscope}%
\pgfpathrectangle{\pgfqpoint{0.887500in}{0.275000in}}{\pgfqpoint{4.225000in}{4.225000in}}%
\pgfusepath{clip}%
\pgfsetbuttcap%
\pgfsetroundjoin%
\definecolor{currentfill}{rgb}{0.182256,0.426184,0.557120}%
\pgfsetfillcolor{currentfill}%
\pgfsetfillopacity{0.700000}%
\pgfsetlinewidth{0.501875pt}%
\definecolor{currentstroke}{rgb}{1.000000,1.000000,1.000000}%
\pgfsetstrokecolor{currentstroke}%
\pgfsetstrokeopacity{0.500000}%
\pgfsetdash{}{0pt}%
\pgfpathmoveto{\pgfqpoint{2.720605in}{2.063149in}}%
\pgfpathlineto{\pgfqpoint{2.732123in}{2.066735in}}%
\pgfpathlineto{\pgfqpoint{2.743635in}{2.070239in}}%
\pgfpathlineto{\pgfqpoint{2.755143in}{2.073638in}}%
\pgfpathlineto{\pgfqpoint{2.766646in}{2.076909in}}%
\pgfpathlineto{\pgfqpoint{2.778143in}{2.080070in}}%
\pgfpathlineto{\pgfqpoint{2.771967in}{2.089565in}}%
\pgfpathlineto{\pgfqpoint{2.765794in}{2.099020in}}%
\pgfpathlineto{\pgfqpoint{2.759625in}{2.108435in}}%
\pgfpathlineto{\pgfqpoint{2.753461in}{2.117813in}}%
\pgfpathlineto{\pgfqpoint{2.747301in}{2.127154in}}%
\pgfpathlineto{\pgfqpoint{2.735813in}{2.124058in}}%
\pgfpathlineto{\pgfqpoint{2.724320in}{2.120845in}}%
\pgfpathlineto{\pgfqpoint{2.712822in}{2.117493in}}%
\pgfpathlineto{\pgfqpoint{2.701320in}{2.114030in}}%
\pgfpathlineto{\pgfqpoint{2.689813in}{2.110479in}}%
\pgfpathlineto{\pgfqpoint{2.695963in}{2.101093in}}%
\pgfpathlineto{\pgfqpoint{2.702117in}{2.091667in}}%
\pgfpathlineto{\pgfqpoint{2.708276in}{2.082202in}}%
\pgfpathlineto{\pgfqpoint{2.714438in}{2.072696in}}%
\pgfpathclose%
\pgfusepath{stroke,fill}%
\end{pgfscope}%
\begin{pgfscope}%
\pgfpathrectangle{\pgfqpoint{0.887500in}{0.275000in}}{\pgfqpoint{4.225000in}{4.225000in}}%
\pgfusepath{clip}%
\pgfsetbuttcap%
\pgfsetroundjoin%
\definecolor{currentfill}{rgb}{0.231674,0.318106,0.544834}%
\pgfsetfillcolor{currentfill}%
\pgfsetfillopacity{0.700000}%
\pgfsetlinewidth{0.501875pt}%
\definecolor{currentstroke}{rgb}{1.000000,1.000000,1.000000}%
\pgfsetstrokecolor{currentstroke}%
\pgfsetstrokeopacity{0.500000}%
\pgfsetdash{}{0pt}%
\pgfpathmoveto{\pgfqpoint{4.537368in}{1.831766in}}%
\pgfpathlineto{\pgfqpoint{4.548425in}{1.835124in}}%
\pgfpathlineto{\pgfqpoint{4.559478in}{1.838516in}}%
\pgfpathlineto{\pgfqpoint{4.570527in}{1.841930in}}%
\pgfpathlineto{\pgfqpoint{4.581570in}{1.845354in}}%
\pgfpathlineto{\pgfqpoint{4.592608in}{1.848775in}}%
\pgfpathlineto{\pgfqpoint{4.586070in}{1.863492in}}%
\pgfpathlineto{\pgfqpoint{4.579536in}{1.878219in}}%
\pgfpathlineto{\pgfqpoint{4.573004in}{1.892949in}}%
\pgfpathlineto{\pgfqpoint{4.566474in}{1.907675in}}%
\pgfpathlineto{\pgfqpoint{4.559947in}{1.922392in}}%
\pgfpathlineto{\pgfqpoint{4.548913in}{1.919055in}}%
\pgfpathlineto{\pgfqpoint{4.537874in}{1.915716in}}%
\pgfpathlineto{\pgfqpoint{4.526829in}{1.912378in}}%
\pgfpathlineto{\pgfqpoint{4.515779in}{1.909043in}}%
\pgfpathlineto{\pgfqpoint{4.504724in}{1.905716in}}%
\pgfpathlineto{\pgfqpoint{4.511247in}{1.890898in}}%
\pgfpathlineto{\pgfqpoint{4.517772in}{1.876080in}}%
\pgfpathlineto{\pgfqpoint{4.524300in}{1.861275in}}%
\pgfpathlineto{\pgfqpoint{4.530832in}{1.846499in}}%
\pgfpathclose%
\pgfusepath{stroke,fill}%
\end{pgfscope}%
\begin{pgfscope}%
\pgfpathrectangle{\pgfqpoint{0.887500in}{0.275000in}}{\pgfqpoint{4.225000in}{4.225000in}}%
\pgfusepath{clip}%
\pgfsetbuttcap%
\pgfsetroundjoin%
\definecolor{currentfill}{rgb}{0.202219,0.715272,0.476084}%
\pgfsetfillcolor{currentfill}%
\pgfsetfillopacity{0.700000}%
\pgfsetlinewidth{0.501875pt}%
\definecolor{currentstroke}{rgb}{1.000000,1.000000,1.000000}%
\pgfsetstrokecolor{currentstroke}%
\pgfsetstrokeopacity{0.500000}%
\pgfsetdash{}{0pt}%
\pgfpathmoveto{\pgfqpoint{3.380850in}{2.656810in}}%
\pgfpathlineto{\pgfqpoint{3.392242in}{2.665463in}}%
\pgfpathlineto{\pgfqpoint{3.403627in}{2.673574in}}%
\pgfpathlineto{\pgfqpoint{3.415005in}{2.681141in}}%
\pgfpathlineto{\pgfqpoint{3.426374in}{2.688164in}}%
\pgfpathlineto{\pgfqpoint{3.437735in}{2.694665in}}%
\pgfpathlineto{\pgfqpoint{3.431405in}{2.709511in}}%
\pgfpathlineto{\pgfqpoint{3.425076in}{2.724194in}}%
\pgfpathlineto{\pgfqpoint{3.418748in}{2.738661in}}%
\pgfpathlineto{\pgfqpoint{3.412419in}{2.752855in}}%
\pgfpathlineto{\pgfqpoint{3.406090in}{2.766751in}}%
\pgfpathlineto{\pgfqpoint{3.394740in}{2.760985in}}%
\pgfpathlineto{\pgfqpoint{3.383382in}{2.754641in}}%
\pgfpathlineto{\pgfqpoint{3.372015in}{2.747702in}}%
\pgfpathlineto{\pgfqpoint{3.360640in}{2.740205in}}%
\pgfpathlineto{\pgfqpoint{3.349258in}{2.732192in}}%
\pgfpathlineto{\pgfqpoint{3.355579in}{2.718048in}}%
\pgfpathlineto{\pgfqpoint{3.361899in}{2.703369in}}%
\pgfpathlineto{\pgfqpoint{3.368216in}{2.688201in}}%
\pgfpathlineto{\pgfqpoint{3.374533in}{2.672648in}}%
\pgfpathclose%
\pgfusepath{stroke,fill}%
\end{pgfscope}%
\begin{pgfscope}%
\pgfpathrectangle{\pgfqpoint{0.887500in}{0.275000in}}{\pgfqpoint{4.225000in}{4.225000in}}%
\pgfusepath{clip}%
\pgfsetbuttcap%
\pgfsetroundjoin%
\definecolor{currentfill}{rgb}{0.175707,0.697900,0.491033}%
\pgfsetfillcolor{currentfill}%
\pgfsetfillopacity{0.700000}%
\pgfsetlinewidth{0.501875pt}%
\definecolor{currentstroke}{rgb}{1.000000,1.000000,1.000000}%
\pgfsetstrokecolor{currentstroke}%
\pgfsetstrokeopacity{0.500000}%
\pgfsetdash{}{0pt}%
\pgfpathmoveto{\pgfqpoint{3.469411in}{2.619920in}}%
\pgfpathlineto{\pgfqpoint{3.480778in}{2.626996in}}%
\pgfpathlineto{\pgfqpoint{3.492137in}{2.633625in}}%
\pgfpathlineto{\pgfqpoint{3.503487in}{2.639818in}}%
\pgfpathlineto{\pgfqpoint{3.514828in}{2.645584in}}%
\pgfpathlineto{\pgfqpoint{3.526160in}{2.650935in}}%
\pgfpathlineto{\pgfqpoint{3.519806in}{2.665003in}}%
\pgfpathlineto{\pgfqpoint{3.513455in}{2.679055in}}%
\pgfpathlineto{\pgfqpoint{3.507106in}{2.693080in}}%
\pgfpathlineto{\pgfqpoint{3.500759in}{2.707069in}}%
\pgfpathlineto{\pgfqpoint{3.494413in}{2.721012in}}%
\pgfpathlineto{\pgfqpoint{3.483094in}{2.716403in}}%
\pgfpathlineto{\pgfqpoint{3.471766in}{2.711511in}}%
\pgfpathlineto{\pgfqpoint{3.460431in}{2.706290in}}%
\pgfpathlineto{\pgfqpoint{3.449087in}{2.700691in}}%
\pgfpathlineto{\pgfqpoint{3.437735in}{2.694665in}}%
\pgfpathlineto{\pgfqpoint{3.444065in}{2.679711in}}%
\pgfpathlineto{\pgfqpoint{3.450398in}{2.664705in}}%
\pgfpathlineto{\pgfqpoint{3.456733in}{2.649702in}}%
\pgfpathlineto{\pgfqpoint{3.463070in}{2.634755in}}%
\pgfpathclose%
\pgfusepath{stroke,fill}%
\end{pgfscope}%
\begin{pgfscope}%
\pgfpathrectangle{\pgfqpoint{0.887500in}{0.275000in}}{\pgfqpoint{4.225000in}{4.225000in}}%
\pgfusepath{clip}%
\pgfsetbuttcap%
\pgfsetroundjoin%
\definecolor{currentfill}{rgb}{0.168126,0.459988,0.558082}%
\pgfsetfillcolor{currentfill}%
\pgfsetfillopacity{0.700000}%
\pgfsetlinewidth{0.501875pt}%
\definecolor{currentstroke}{rgb}{1.000000,1.000000,1.000000}%
\pgfsetstrokecolor{currentstroke}%
\pgfsetstrokeopacity{0.500000}%
\pgfsetdash{}{0pt}%
\pgfpathmoveto{\pgfqpoint{2.397628in}{2.132274in}}%
\pgfpathlineto{\pgfqpoint{2.409227in}{2.135834in}}%
\pgfpathlineto{\pgfqpoint{2.420821in}{2.139374in}}%
\pgfpathlineto{\pgfqpoint{2.432409in}{2.142890in}}%
\pgfpathlineto{\pgfqpoint{2.443992in}{2.146384in}}%
\pgfpathlineto{\pgfqpoint{2.455570in}{2.149857in}}%
\pgfpathlineto{\pgfqpoint{2.449497in}{2.158976in}}%
\pgfpathlineto{\pgfqpoint{2.443430in}{2.168064in}}%
\pgfpathlineto{\pgfqpoint{2.437366in}{2.177120in}}%
\pgfpathlineto{\pgfqpoint{2.431307in}{2.186146in}}%
\pgfpathlineto{\pgfqpoint{2.425252in}{2.195143in}}%
\pgfpathlineto{\pgfqpoint{2.413685in}{2.191717in}}%
\pgfpathlineto{\pgfqpoint{2.402113in}{2.188271in}}%
\pgfpathlineto{\pgfqpoint{2.390535in}{2.184804in}}%
\pgfpathlineto{\pgfqpoint{2.378952in}{2.181314in}}%
\pgfpathlineto{\pgfqpoint{2.367364in}{2.177805in}}%
\pgfpathlineto{\pgfqpoint{2.373408in}{2.168760in}}%
\pgfpathlineto{\pgfqpoint{2.379456in}{2.159685in}}%
\pgfpathlineto{\pgfqpoint{2.385509in}{2.150580in}}%
\pgfpathlineto{\pgfqpoint{2.391566in}{2.141443in}}%
\pgfpathclose%
\pgfusepath{stroke,fill}%
\end{pgfscope}%
\begin{pgfscope}%
\pgfpathrectangle{\pgfqpoint{0.887500in}{0.275000in}}{\pgfqpoint{4.225000in}{4.225000in}}%
\pgfusepath{clip}%
\pgfsetbuttcap%
\pgfsetroundjoin%
\definecolor{currentfill}{rgb}{0.154815,0.493313,0.557840}%
\pgfsetfillcolor{currentfill}%
\pgfsetfillopacity{0.700000}%
\pgfsetlinewidth{0.501875pt}%
\definecolor{currentstroke}{rgb}{1.000000,1.000000,1.000000}%
\pgfsetstrokecolor{currentstroke}%
\pgfsetstrokeopacity{0.500000}%
\pgfsetdash{}{0pt}%
\pgfpathmoveto{\pgfqpoint{2.074639in}{2.197095in}}%
\pgfpathlineto{\pgfqpoint{2.086319in}{2.200596in}}%
\pgfpathlineto{\pgfqpoint{2.097994in}{2.204087in}}%
\pgfpathlineto{\pgfqpoint{2.109663in}{2.207569in}}%
\pgfpathlineto{\pgfqpoint{2.121327in}{2.211043in}}%
\pgfpathlineto{\pgfqpoint{2.132985in}{2.214512in}}%
\pgfpathlineto{\pgfqpoint{2.127021in}{2.223396in}}%
\pgfpathlineto{\pgfqpoint{2.121061in}{2.232253in}}%
\pgfpathlineto{\pgfqpoint{2.115106in}{2.241085in}}%
\pgfpathlineto{\pgfqpoint{2.109155in}{2.249890in}}%
\pgfpathlineto{\pgfqpoint{2.103209in}{2.258671in}}%
\pgfpathlineto{\pgfqpoint{2.091562in}{2.255251in}}%
\pgfpathlineto{\pgfqpoint{2.079909in}{2.251827in}}%
\pgfpathlineto{\pgfqpoint{2.068250in}{2.248396in}}%
\pgfpathlineto{\pgfqpoint{2.056586in}{2.244957in}}%
\pgfpathlineto{\pgfqpoint{2.044917in}{2.241508in}}%
\pgfpathlineto{\pgfqpoint{2.050852in}{2.232677in}}%
\pgfpathlineto{\pgfqpoint{2.056792in}{2.223820in}}%
\pgfpathlineto{\pgfqpoint{2.062737in}{2.214938in}}%
\pgfpathlineto{\pgfqpoint{2.068685in}{2.206030in}}%
\pgfpathclose%
\pgfusepath{stroke,fill}%
\end{pgfscope}%
\begin{pgfscope}%
\pgfpathrectangle{\pgfqpoint{0.887500in}{0.275000in}}{\pgfqpoint{4.225000in}{4.225000in}}%
\pgfusepath{clip}%
\pgfsetbuttcap%
\pgfsetroundjoin%
\definecolor{currentfill}{rgb}{0.144759,0.519093,0.556572}%
\pgfsetfillcolor{currentfill}%
\pgfsetfillopacity{0.700000}%
\pgfsetlinewidth{0.501875pt}%
\definecolor{currentstroke}{rgb}{1.000000,1.000000,1.000000}%
\pgfsetstrokecolor{currentstroke}%
\pgfsetstrokeopacity{0.500000}%
\pgfsetdash{}{0pt}%
\pgfpathmoveto{\pgfqpoint{1.751677in}{2.259621in}}%
\pgfpathlineto{\pgfqpoint{1.763437in}{2.263102in}}%
\pgfpathlineto{\pgfqpoint{1.775192in}{2.266576in}}%
\pgfpathlineto{\pgfqpoint{1.786942in}{2.270045in}}%
\pgfpathlineto{\pgfqpoint{1.798685in}{2.273508in}}%
\pgfpathlineto{\pgfqpoint{1.810423in}{2.276968in}}%
\pgfpathlineto{\pgfqpoint{1.804570in}{2.285654in}}%
\pgfpathlineto{\pgfqpoint{1.798721in}{2.294314in}}%
\pgfpathlineto{\pgfqpoint{1.792877in}{2.302951in}}%
\pgfpathlineto{\pgfqpoint{1.787037in}{2.311563in}}%
\pgfpathlineto{\pgfqpoint{1.781201in}{2.320151in}}%
\pgfpathlineto{\pgfqpoint{1.769475in}{2.316734in}}%
\pgfpathlineto{\pgfqpoint{1.757743in}{2.313312in}}%
\pgfpathlineto{\pgfqpoint{1.746005in}{2.309887in}}%
\pgfpathlineto{\pgfqpoint{1.734261in}{2.306455in}}%
\pgfpathlineto{\pgfqpoint{1.722512in}{2.303017in}}%
\pgfpathlineto{\pgfqpoint{1.728336in}{2.294389in}}%
\pgfpathlineto{\pgfqpoint{1.734164in}{2.285735in}}%
\pgfpathlineto{\pgfqpoint{1.739997in}{2.277056in}}%
\pgfpathlineto{\pgfqpoint{1.745835in}{2.268351in}}%
\pgfpathclose%
\pgfusepath{stroke,fill}%
\end{pgfscope}%
\begin{pgfscope}%
\pgfpathrectangle{\pgfqpoint{0.887500in}{0.275000in}}{\pgfqpoint{4.225000in}{4.225000in}}%
\pgfusepath{clip}%
\pgfsetbuttcap%
\pgfsetroundjoin%
\definecolor{currentfill}{rgb}{0.166617,0.463708,0.558119}%
\pgfsetfillcolor{currentfill}%
\pgfsetfillopacity{0.700000}%
\pgfsetlinewidth{0.501875pt}%
\definecolor{currentstroke}{rgb}{1.000000,1.000000,1.000000}%
\pgfsetstrokecolor{currentstroke}%
\pgfsetstrokeopacity{0.500000}%
\pgfsetdash{}{0pt}%
\pgfpathmoveto{\pgfqpoint{4.152157in}{2.120227in}}%
\pgfpathlineto{\pgfqpoint{4.163322in}{2.123785in}}%
\pgfpathlineto{\pgfqpoint{4.174481in}{2.127335in}}%
\pgfpathlineto{\pgfqpoint{4.185635in}{2.130873in}}%
\pgfpathlineto{\pgfqpoint{4.196783in}{2.134394in}}%
\pgfpathlineto{\pgfqpoint{4.207924in}{2.137894in}}%
\pgfpathlineto{\pgfqpoint{4.201456in}{2.152630in}}%
\pgfpathlineto{\pgfqpoint{4.194989in}{2.167325in}}%
\pgfpathlineto{\pgfqpoint{4.188523in}{2.181970in}}%
\pgfpathlineto{\pgfqpoint{4.182058in}{2.196572in}}%
\pgfpathlineto{\pgfqpoint{4.175594in}{2.211134in}}%
\pgfpathlineto{\pgfqpoint{4.164461in}{2.207880in}}%
\pgfpathlineto{\pgfqpoint{4.153322in}{2.204611in}}%
\pgfpathlineto{\pgfqpoint{4.142177in}{2.201329in}}%
\pgfpathlineto{\pgfqpoint{4.131026in}{2.198040in}}%
\pgfpathlineto{\pgfqpoint{4.119869in}{2.194746in}}%
\pgfpathlineto{\pgfqpoint{4.126330in}{2.180144in}}%
\pgfpathlineto{\pgfqpoint{4.132790in}{2.165409in}}%
\pgfpathlineto{\pgfqpoint{4.139248in}{2.150521in}}%
\pgfpathlineto{\pgfqpoint{4.145704in}{2.135457in}}%
\pgfpathclose%
\pgfusepath{stroke,fill}%
\end{pgfscope}%
\begin{pgfscope}%
\pgfpathrectangle{\pgfqpoint{0.887500in}{0.275000in}}{\pgfqpoint{4.225000in}{4.225000in}}%
\pgfusepath{clip}%
\pgfsetbuttcap%
\pgfsetroundjoin%
\definecolor{currentfill}{rgb}{0.187231,0.414746,0.556547}%
\pgfsetfillcolor{currentfill}%
\pgfsetfillopacity{0.700000}%
\pgfsetlinewidth{0.501875pt}%
\definecolor{currentstroke}{rgb}{1.000000,1.000000,1.000000}%
\pgfsetstrokecolor{currentstroke}%
\pgfsetstrokeopacity{0.500000}%
\pgfsetdash{}{0pt}%
\pgfpathmoveto{\pgfqpoint{2.809088in}{2.031959in}}%
\pgfpathlineto{\pgfqpoint{2.820590in}{2.035150in}}%
\pgfpathlineto{\pgfqpoint{2.832085in}{2.038381in}}%
\pgfpathlineto{\pgfqpoint{2.843574in}{2.041726in}}%
\pgfpathlineto{\pgfqpoint{2.855056in}{2.045259in}}%
\pgfpathlineto{\pgfqpoint{2.866532in}{2.049056in}}%
\pgfpathlineto{\pgfqpoint{2.860325in}{2.058710in}}%
\pgfpathlineto{\pgfqpoint{2.854122in}{2.068323in}}%
\pgfpathlineto{\pgfqpoint{2.847923in}{2.077895in}}%
\pgfpathlineto{\pgfqpoint{2.841728in}{2.087425in}}%
\pgfpathlineto{\pgfqpoint{2.835538in}{2.096915in}}%
\pgfpathlineto{\pgfqpoint{2.824073in}{2.093146in}}%
\pgfpathlineto{\pgfqpoint{2.812600in}{2.089655in}}%
\pgfpathlineto{\pgfqpoint{2.801121in}{2.086365in}}%
\pgfpathlineto{\pgfqpoint{2.789635in}{2.083196in}}%
\pgfpathlineto{\pgfqpoint{2.778143in}{2.080070in}}%
\pgfpathlineto{\pgfqpoint{2.784324in}{2.070534in}}%
\pgfpathlineto{\pgfqpoint{2.790509in}{2.060955in}}%
\pgfpathlineto{\pgfqpoint{2.796698in}{2.051333in}}%
\pgfpathlineto{\pgfqpoint{2.802891in}{2.041667in}}%
\pgfpathclose%
\pgfusepath{stroke,fill}%
\end{pgfscope}%
\begin{pgfscope}%
\pgfpathrectangle{\pgfqpoint{0.887500in}{0.275000in}}{\pgfqpoint{4.225000in}{4.225000in}}%
\pgfusepath{clip}%
\pgfsetbuttcap%
\pgfsetroundjoin%
\definecolor{currentfill}{rgb}{0.125394,0.574318,0.549086}%
\pgfsetfillcolor{currentfill}%
\pgfsetfillopacity{0.700000}%
\pgfsetlinewidth{0.501875pt}%
\definecolor{currentstroke}{rgb}{1.000000,1.000000,1.000000}%
\pgfsetstrokecolor{currentstroke}%
\pgfsetstrokeopacity{0.500000}%
\pgfsetdash{}{0pt}%
\pgfpathmoveto{\pgfqpoint{3.855149in}{2.357070in}}%
\pgfpathlineto{\pgfqpoint{3.866395in}{2.360973in}}%
\pgfpathlineto{\pgfqpoint{3.877634in}{2.364764in}}%
\pgfpathlineto{\pgfqpoint{3.888865in}{2.368462in}}%
\pgfpathlineto{\pgfqpoint{3.900090in}{2.372083in}}%
\pgfpathlineto{\pgfqpoint{3.911307in}{2.375642in}}%
\pgfpathlineto{\pgfqpoint{3.904888in}{2.390191in}}%
\pgfpathlineto{\pgfqpoint{3.898470in}{2.404677in}}%
\pgfpathlineto{\pgfqpoint{3.892053in}{2.419081in}}%
\pgfpathlineto{\pgfqpoint{3.885636in}{2.433401in}}%
\pgfpathlineto{\pgfqpoint{3.879221in}{2.447646in}}%
\pgfpathlineto{\pgfqpoint{3.868008in}{2.444218in}}%
\pgfpathlineto{\pgfqpoint{3.856789in}{2.440752in}}%
\pgfpathlineto{\pgfqpoint{3.845563in}{2.437230in}}%
\pgfpathlineto{\pgfqpoint{3.834330in}{2.433637in}}%
\pgfpathlineto{\pgfqpoint{3.823090in}{2.429955in}}%
\pgfpathlineto{\pgfqpoint{3.829502in}{2.415641in}}%
\pgfpathlineto{\pgfqpoint{3.835914in}{2.401202in}}%
\pgfpathlineto{\pgfqpoint{3.842326in}{2.386622in}}%
\pgfpathlineto{\pgfqpoint{3.848737in}{2.371903in}}%
\pgfpathclose%
\pgfusepath{stroke,fill}%
\end{pgfscope}%
\begin{pgfscope}%
\pgfpathrectangle{\pgfqpoint{0.887500in}{0.275000in}}{\pgfqpoint{4.225000in}{4.225000in}}%
\pgfusepath{clip}%
\pgfsetbuttcap%
\pgfsetroundjoin%
\definecolor{currentfill}{rgb}{0.216210,0.351535,0.550627}%
\pgfsetfillcolor{currentfill}%
\pgfsetfillopacity{0.700000}%
\pgfsetlinewidth{0.501875pt}%
\definecolor{currentstroke}{rgb}{1.000000,1.000000,1.000000}%
\pgfsetstrokecolor{currentstroke}%
\pgfsetstrokeopacity{0.500000}%
\pgfsetdash{}{0pt}%
\pgfpathmoveto{\pgfqpoint{4.449370in}{1.889256in}}%
\pgfpathlineto{\pgfqpoint{4.460451in}{1.892530in}}%
\pgfpathlineto{\pgfqpoint{4.471527in}{1.895809in}}%
\pgfpathlineto{\pgfqpoint{4.482598in}{1.899098in}}%
\pgfpathlineto{\pgfqpoint{4.493663in}{1.902400in}}%
\pgfpathlineto{\pgfqpoint{4.504724in}{1.905716in}}%
\pgfpathlineto{\pgfqpoint{4.498203in}{1.920520in}}%
\pgfpathlineto{\pgfqpoint{4.491684in}{1.935294in}}%
\pgfpathlineto{\pgfqpoint{4.485165in}{1.950024in}}%
\pgfpathlineto{\pgfqpoint{4.478647in}{1.964697in}}%
\pgfpathlineto{\pgfqpoint{4.472129in}{1.979297in}}%
\pgfpathlineto{\pgfqpoint{4.461069in}{1.975957in}}%
\pgfpathlineto{\pgfqpoint{4.450001in}{1.972573in}}%
\pgfpathlineto{\pgfqpoint{4.438926in}{1.969140in}}%
\pgfpathlineto{\pgfqpoint{4.427844in}{1.965658in}}%
\pgfpathlineto{\pgfqpoint{4.416755in}{1.962125in}}%
\pgfpathlineto{\pgfqpoint{4.423277in}{1.947663in}}%
\pgfpathlineto{\pgfqpoint{4.429799in}{1.933134in}}%
\pgfpathlineto{\pgfqpoint{4.436322in}{1.918550in}}%
\pgfpathlineto{\pgfqpoint{4.442845in}{1.903920in}}%
\pgfpathclose%
\pgfusepath{stroke,fill}%
\end{pgfscope}%
\begin{pgfscope}%
\pgfpathrectangle{\pgfqpoint{0.887500in}{0.275000in}}{\pgfqpoint{4.225000in}{4.225000in}}%
\pgfusepath{clip}%
\pgfsetbuttcap%
\pgfsetroundjoin%
\definecolor{currentfill}{rgb}{0.122312,0.633153,0.530398}%
\pgfsetfillcolor{currentfill}%
\pgfsetfillopacity{0.700000}%
\pgfsetlinewidth{0.501875pt}%
\definecolor{currentstroke}{rgb}{1.000000,1.000000,1.000000}%
\pgfsetstrokecolor{currentstroke}%
\pgfsetstrokeopacity{0.500000}%
\pgfsetdash{}{0pt}%
\pgfpathmoveto{\pgfqpoint{3.209327in}{2.442875in}}%
\pgfpathlineto{\pgfqpoint{3.220781in}{2.462818in}}%
\pgfpathlineto{\pgfqpoint{3.232237in}{2.482308in}}%
\pgfpathlineto{\pgfqpoint{3.243693in}{2.501151in}}%
\pgfpathlineto{\pgfqpoint{3.255149in}{2.519154in}}%
\pgfpathlineto{\pgfqpoint{3.266602in}{2.536121in}}%
\pgfpathlineto{\pgfqpoint{3.260316in}{2.555031in}}%
\pgfpathlineto{\pgfqpoint{3.254027in}{2.573041in}}%
\pgfpathlineto{\pgfqpoint{3.247733in}{2.589876in}}%
\pgfpathlineto{\pgfqpoint{3.241434in}{2.605266in}}%
\pgfpathlineto{\pgfqpoint{3.235130in}{2.619098in}}%
\pgfpathlineto{\pgfqpoint{3.223683in}{2.601105in}}%
\pgfpathlineto{\pgfqpoint{3.212233in}{2.581448in}}%
\pgfpathlineto{\pgfqpoint{3.200781in}{2.560471in}}%
\pgfpathlineto{\pgfqpoint{3.189330in}{2.538519in}}%
\pgfpathlineto{\pgfqpoint{3.177883in}{2.515936in}}%
\pgfpathlineto{\pgfqpoint{3.184171in}{2.502611in}}%
\pgfpathlineto{\pgfqpoint{3.190460in}{2.488576in}}%
\pgfpathlineto{\pgfqpoint{3.196749in}{2.473875in}}%
\pgfpathlineto{\pgfqpoint{3.203038in}{2.458607in}}%
\pgfpathclose%
\pgfusepath{stroke,fill}%
\end{pgfscope}%
\begin{pgfscope}%
\pgfpathrectangle{\pgfqpoint{0.887500in}{0.275000in}}{\pgfqpoint{4.225000in}{4.225000in}}%
\pgfusepath{clip}%
\pgfsetbuttcap%
\pgfsetroundjoin%
\definecolor{currentfill}{rgb}{0.132444,0.552216,0.553018}%
\pgfsetfillcolor{currentfill}%
\pgfsetfillopacity{0.700000}%
\pgfsetlinewidth{0.501875pt}%
\definecolor{currentstroke}{rgb}{1.000000,1.000000,1.000000}%
\pgfsetstrokecolor{currentstroke}%
\pgfsetstrokeopacity{0.500000}%
\pgfsetdash{}{0pt}%
\pgfpathmoveto{\pgfqpoint{3.183562in}{2.278061in}}%
\pgfpathlineto{\pgfqpoint{3.195007in}{2.295844in}}%
\pgfpathlineto{\pgfqpoint{3.206452in}{2.312850in}}%
\pgfpathlineto{\pgfqpoint{3.217896in}{2.329247in}}%
\pgfpathlineto{\pgfqpoint{3.229341in}{2.345188in}}%
\pgfpathlineto{\pgfqpoint{3.240785in}{2.360789in}}%
\pgfpathlineto{\pgfqpoint{3.234491in}{2.377326in}}%
\pgfpathlineto{\pgfqpoint{3.228198in}{2.393904in}}%
\pgfpathlineto{\pgfqpoint{3.221906in}{2.410422in}}%
\pgfpathlineto{\pgfqpoint{3.215616in}{2.426780in}}%
\pgfpathlineto{\pgfqpoint{3.209327in}{2.442875in}}%
\pgfpathlineto{\pgfqpoint{3.197876in}{2.422673in}}%
\pgfpathlineto{\pgfqpoint{3.186428in}{2.402405in}}%
\pgfpathlineto{\pgfqpoint{3.174985in}{2.382206in}}%
\pgfpathlineto{\pgfqpoint{3.163546in}{2.361928in}}%
\pgfpathlineto{\pgfqpoint{3.152110in}{2.341338in}}%
\pgfpathlineto{\pgfqpoint{3.158396in}{2.329063in}}%
\pgfpathlineto{\pgfqpoint{3.164684in}{2.316658in}}%
\pgfpathlineto{\pgfqpoint{3.170975in}{2.304065in}}%
\pgfpathlineto{\pgfqpoint{3.177268in}{2.291219in}}%
\pgfpathclose%
\pgfusepath{stroke,fill}%
\end{pgfscope}%
\begin{pgfscope}%
\pgfpathrectangle{\pgfqpoint{0.887500in}{0.275000in}}{\pgfqpoint{4.225000in}{4.225000in}}%
\pgfusepath{clip}%
\pgfsetbuttcap%
\pgfsetroundjoin%
\definecolor{currentfill}{rgb}{0.172719,0.448791,0.557885}%
\pgfsetfillcolor{currentfill}%
\pgfsetfillopacity{0.700000}%
\pgfsetlinewidth{0.501875pt}%
\definecolor{currentstroke}{rgb}{1.000000,1.000000,1.000000}%
\pgfsetstrokecolor{currentstroke}%
\pgfsetstrokeopacity{0.500000}%
\pgfsetdash{}{0pt}%
\pgfpathmoveto{\pgfqpoint{2.485994in}{2.103760in}}%
\pgfpathlineto{\pgfqpoint{2.497577in}{2.107268in}}%
\pgfpathlineto{\pgfqpoint{2.509153in}{2.110764in}}%
\pgfpathlineto{\pgfqpoint{2.520725in}{2.114253in}}%
\pgfpathlineto{\pgfqpoint{2.532290in}{2.117741in}}%
\pgfpathlineto{\pgfqpoint{2.543850in}{2.121231in}}%
\pgfpathlineto{\pgfqpoint{2.537746in}{2.130472in}}%
\pgfpathlineto{\pgfqpoint{2.531646in}{2.139678in}}%
\pgfpathlineto{\pgfqpoint{2.525551in}{2.148850in}}%
\pgfpathlineto{\pgfqpoint{2.519460in}{2.157990in}}%
\pgfpathlineto{\pgfqpoint{2.513373in}{2.167096in}}%
\pgfpathlineto{\pgfqpoint{2.501824in}{2.163650in}}%
\pgfpathlineto{\pgfqpoint{2.490269in}{2.160207in}}%
\pgfpathlineto{\pgfqpoint{2.478708in}{2.156764in}}%
\pgfpathlineto{\pgfqpoint{2.467142in}{2.153316in}}%
\pgfpathlineto{\pgfqpoint{2.455570in}{2.149857in}}%
\pgfpathlineto{\pgfqpoint{2.461646in}{2.140706in}}%
\pgfpathlineto{\pgfqpoint{2.467726in}{2.131521in}}%
\pgfpathlineto{\pgfqpoint{2.473811in}{2.122302in}}%
\pgfpathlineto{\pgfqpoint{2.479901in}{2.113049in}}%
\pgfpathclose%
\pgfusepath{stroke,fill}%
\end{pgfscope}%
\begin{pgfscope}%
\pgfpathrectangle{\pgfqpoint{0.887500in}{0.275000in}}{\pgfqpoint{4.225000in}{4.225000in}}%
\pgfusepath{clip}%
\pgfsetbuttcap%
\pgfsetroundjoin%
\definecolor{currentfill}{rgb}{0.120092,0.600104,0.542530}%
\pgfsetfillcolor{currentfill}%
\pgfsetfillopacity{0.700000}%
\pgfsetlinewidth{0.501875pt}%
\definecolor{currentstroke}{rgb}{1.000000,1.000000,1.000000}%
\pgfsetstrokecolor{currentstroke}%
\pgfsetstrokeopacity{0.500000}%
\pgfsetdash{}{0pt}%
\pgfpathmoveto{\pgfqpoint{3.766779in}{2.409763in}}%
\pgfpathlineto{\pgfqpoint{3.778056in}{2.414069in}}%
\pgfpathlineto{\pgfqpoint{3.789326in}{2.418234in}}%
\pgfpathlineto{\pgfqpoint{3.800588in}{2.422265in}}%
\pgfpathlineto{\pgfqpoint{3.811843in}{2.426170in}}%
\pgfpathlineto{\pgfqpoint{3.823090in}{2.429955in}}%
\pgfpathlineto{\pgfqpoint{3.816679in}{2.444164in}}%
\pgfpathlineto{\pgfqpoint{3.810268in}{2.458283in}}%
\pgfpathlineto{\pgfqpoint{3.803859in}{2.472332in}}%
\pgfpathlineto{\pgfqpoint{3.797451in}{2.486327in}}%
\pgfpathlineto{\pgfqpoint{3.791045in}{2.500286in}}%
\pgfpathlineto{\pgfqpoint{3.779802in}{2.496549in}}%
\pgfpathlineto{\pgfqpoint{3.768551in}{2.492726in}}%
\pgfpathlineto{\pgfqpoint{3.757294in}{2.488818in}}%
\pgfpathlineto{\pgfqpoint{3.746030in}{2.484820in}}%
\pgfpathlineto{\pgfqpoint{3.734760in}{2.480734in}}%
\pgfpathlineto{\pgfqpoint{3.741161in}{2.466671in}}%
\pgfpathlineto{\pgfqpoint{3.747564in}{2.452563in}}%
\pgfpathlineto{\pgfqpoint{3.753968in}{2.438390in}}%
\pgfpathlineto{\pgfqpoint{3.760373in}{2.424130in}}%
\pgfpathclose%
\pgfusepath{stroke,fill}%
\end{pgfscope}%
\begin{pgfscope}%
\pgfpathrectangle{\pgfqpoint{0.887500in}{0.275000in}}{\pgfqpoint{4.225000in}{4.225000in}}%
\pgfusepath{clip}%
\pgfsetbuttcap%
\pgfsetroundjoin%
\definecolor{currentfill}{rgb}{0.175707,0.697900,0.491033}%
\pgfsetfillcolor{currentfill}%
\pgfsetfillopacity{0.700000}%
\pgfsetlinewidth{0.501875pt}%
\definecolor{currentstroke}{rgb}{1.000000,1.000000,1.000000}%
\pgfsetstrokecolor{currentstroke}%
\pgfsetstrokeopacity{0.500000}%
\pgfsetdash{}{0pt}%
\pgfpathmoveto{\pgfqpoint{3.323794in}{2.605422in}}%
\pgfpathlineto{\pgfqpoint{3.335216in}{2.616821in}}%
\pgfpathlineto{\pgfqpoint{3.346633in}{2.627624in}}%
\pgfpathlineto{\pgfqpoint{3.358045in}{2.637889in}}%
\pgfpathlineto{\pgfqpoint{3.369450in}{2.647618in}}%
\pgfpathlineto{\pgfqpoint{3.380850in}{2.656810in}}%
\pgfpathlineto{\pgfqpoint{3.374533in}{2.672648in}}%
\pgfpathlineto{\pgfqpoint{3.368216in}{2.688201in}}%
\pgfpathlineto{\pgfqpoint{3.361899in}{2.703369in}}%
\pgfpathlineto{\pgfqpoint{3.355579in}{2.718048in}}%
\pgfpathlineto{\pgfqpoint{3.349258in}{2.732192in}}%
\pgfpathlineto{\pgfqpoint{3.337870in}{2.723702in}}%
\pgfpathlineto{\pgfqpoint{3.326476in}{2.714776in}}%
\pgfpathlineto{\pgfqpoint{3.315078in}{2.705456in}}%
\pgfpathlineto{\pgfqpoint{3.303674in}{2.695769in}}%
\pgfpathlineto{\pgfqpoint{3.292266in}{2.685586in}}%
\pgfpathlineto{\pgfqpoint{3.298580in}{2.671467in}}%
\pgfpathlineto{\pgfqpoint{3.304889in}{2.656259in}}%
\pgfpathlineto{\pgfqpoint{3.311194in}{2.640050in}}%
\pgfpathlineto{\pgfqpoint{3.317495in}{2.623038in}}%
\pgfpathclose%
\pgfusepath{stroke,fill}%
\end{pgfscope}%
\begin{pgfscope}%
\pgfpathrectangle{\pgfqpoint{0.887500in}{0.275000in}}{\pgfqpoint{4.225000in}{4.225000in}}%
\pgfusepath{clip}%
\pgfsetbuttcap%
\pgfsetroundjoin%
\definecolor{currentfill}{rgb}{0.159194,0.482237,0.558073}%
\pgfsetfillcolor{currentfill}%
\pgfsetfillopacity{0.700000}%
\pgfsetlinewidth{0.501875pt}%
\definecolor{currentstroke}{rgb}{1.000000,1.000000,1.000000}%
\pgfsetstrokecolor{currentstroke}%
\pgfsetstrokeopacity{0.500000}%
\pgfsetdash{}{0pt}%
\pgfpathmoveto{\pgfqpoint{2.162871in}{2.169675in}}%
\pgfpathlineto{\pgfqpoint{2.174534in}{2.173193in}}%
\pgfpathlineto{\pgfqpoint{2.186191in}{2.176711in}}%
\pgfpathlineto{\pgfqpoint{2.197842in}{2.180233in}}%
\pgfpathlineto{\pgfqpoint{2.209488in}{2.183763in}}%
\pgfpathlineto{\pgfqpoint{2.221127in}{2.187301in}}%
\pgfpathlineto{\pgfqpoint{2.215130in}{2.196274in}}%
\pgfpathlineto{\pgfqpoint{2.209138in}{2.205219in}}%
\pgfpathlineto{\pgfqpoint{2.203150in}{2.214136in}}%
\pgfpathlineto{\pgfqpoint{2.197167in}{2.223027in}}%
\pgfpathlineto{\pgfqpoint{2.191187in}{2.231890in}}%
\pgfpathlineto{\pgfqpoint{2.179559in}{2.228400in}}%
\pgfpathlineto{\pgfqpoint{2.167924in}{2.224920in}}%
\pgfpathlineto{\pgfqpoint{2.156284in}{2.221447in}}%
\pgfpathlineto{\pgfqpoint{2.144637in}{2.217979in}}%
\pgfpathlineto{\pgfqpoint{2.132985in}{2.214512in}}%
\pgfpathlineto{\pgfqpoint{2.138953in}{2.205600in}}%
\pgfpathlineto{\pgfqpoint{2.144926in}{2.196661in}}%
\pgfpathlineto{\pgfqpoint{2.150903in}{2.187694in}}%
\pgfpathlineto{\pgfqpoint{2.156885in}{2.178699in}}%
\pgfpathclose%
\pgfusepath{stroke,fill}%
\end{pgfscope}%
\begin{pgfscope}%
\pgfpathrectangle{\pgfqpoint{0.887500in}{0.275000in}}{\pgfqpoint{4.225000in}{4.225000in}}%
\pgfusepath{clip}%
\pgfsetbuttcap%
\pgfsetroundjoin%
\definecolor{currentfill}{rgb}{0.192357,0.403199,0.555836}%
\pgfsetfillcolor{currentfill}%
\pgfsetfillopacity{0.700000}%
\pgfsetlinewidth{0.501875pt}%
\definecolor{currentstroke}{rgb}{1.000000,1.000000,1.000000}%
\pgfsetstrokecolor{currentstroke}%
\pgfsetstrokeopacity{0.500000}%
\pgfsetdash{}{0pt}%
\pgfpathmoveto{\pgfqpoint{2.897625in}{2.000134in}}%
\pgfpathlineto{\pgfqpoint{2.909104in}{2.004281in}}%
\pgfpathlineto{\pgfqpoint{2.920576in}{2.008655in}}%
\pgfpathlineto{\pgfqpoint{2.932044in}{2.013027in}}%
\pgfpathlineto{\pgfqpoint{2.943506in}{2.017163in}}%
\pgfpathlineto{\pgfqpoint{2.954965in}{2.020828in}}%
\pgfpathlineto{\pgfqpoint{2.948728in}{2.030694in}}%
\pgfpathlineto{\pgfqpoint{2.942495in}{2.040508in}}%
\pgfpathlineto{\pgfqpoint{2.936266in}{2.050271in}}%
\pgfpathlineto{\pgfqpoint{2.930041in}{2.059988in}}%
\pgfpathlineto{\pgfqpoint{2.923820in}{2.069661in}}%
\pgfpathlineto{\pgfqpoint{2.912371in}{2.066032in}}%
\pgfpathlineto{\pgfqpoint{2.900918in}{2.061912in}}%
\pgfpathlineto{\pgfqpoint{2.889462in}{2.057547in}}%
\pgfpathlineto{\pgfqpoint{2.878000in}{2.053181in}}%
\pgfpathlineto{\pgfqpoint{2.866532in}{2.049056in}}%
\pgfpathlineto{\pgfqpoint{2.872742in}{2.039358in}}%
\pgfpathlineto{\pgfqpoint{2.878957in}{2.029618in}}%
\pgfpathlineto{\pgfqpoint{2.885176in}{2.019834in}}%
\pgfpathlineto{\pgfqpoint{2.891399in}{2.010006in}}%
\pgfpathclose%
\pgfusepath{stroke,fill}%
\end{pgfscope}%
\begin{pgfscope}%
\pgfpathrectangle{\pgfqpoint{0.887500in}{0.275000in}}{\pgfqpoint{4.225000in}{4.225000in}}%
\pgfusepath{clip}%
\pgfsetbuttcap%
\pgfsetroundjoin%
\definecolor{currentfill}{rgb}{0.154815,0.493313,0.557840}%
\pgfsetfillcolor{currentfill}%
\pgfsetfillopacity{0.700000}%
\pgfsetlinewidth{0.501875pt}%
\definecolor{currentstroke}{rgb}{1.000000,1.000000,1.000000}%
\pgfsetstrokecolor{currentstroke}%
\pgfsetstrokeopacity{0.500000}%
\pgfsetdash{}{0pt}%
\pgfpathmoveto{\pgfqpoint{4.064001in}{2.178228in}}%
\pgfpathlineto{\pgfqpoint{4.075187in}{2.181577in}}%
\pgfpathlineto{\pgfqpoint{4.086366in}{2.184879in}}%
\pgfpathlineto{\pgfqpoint{4.097539in}{2.188163in}}%
\pgfpathlineto{\pgfqpoint{4.108707in}{2.191453in}}%
\pgfpathlineto{\pgfqpoint{4.119869in}{2.194746in}}%
\pgfpathlineto{\pgfqpoint{4.113408in}{2.209238in}}%
\pgfpathlineto{\pgfqpoint{4.106946in}{2.223642in}}%
\pgfpathlineto{\pgfqpoint{4.100486in}{2.237980in}}%
\pgfpathlineto{\pgfqpoint{4.094027in}{2.252274in}}%
\pgfpathlineto{\pgfqpoint{4.087570in}{2.266546in}}%
\pgfpathlineto{\pgfqpoint{4.076406in}{2.263134in}}%
\pgfpathlineto{\pgfqpoint{4.065237in}{2.259731in}}%
\pgfpathlineto{\pgfqpoint{4.054063in}{2.256339in}}%
\pgfpathlineto{\pgfqpoint{4.042884in}{2.252941in}}%
\pgfpathlineto{\pgfqpoint{4.031698in}{2.249513in}}%
\pgfpathlineto{\pgfqpoint{4.038157in}{2.235393in}}%
\pgfpathlineto{\pgfqpoint{4.044617in}{2.221244in}}%
\pgfpathlineto{\pgfqpoint{4.051079in}{2.207026in}}%
\pgfpathlineto{\pgfqpoint{4.057541in}{2.192700in}}%
\pgfpathclose%
\pgfusepath{stroke,fill}%
\end{pgfscope}%
\begin{pgfscope}%
\pgfpathrectangle{\pgfqpoint{0.887500in}{0.275000in}}{\pgfqpoint{4.225000in}{4.225000in}}%
\pgfusepath{clip}%
\pgfsetbuttcap%
\pgfsetroundjoin%
\definecolor{currentfill}{rgb}{0.147607,0.511733,0.557049}%
\pgfsetfillcolor{currentfill}%
\pgfsetfillopacity{0.700000}%
\pgfsetlinewidth{0.501875pt}%
\definecolor{currentstroke}{rgb}{1.000000,1.000000,1.000000}%
\pgfsetstrokecolor{currentstroke}%
\pgfsetstrokeopacity{0.500000}%
\pgfsetdash{}{0pt}%
\pgfpathmoveto{\pgfqpoint{1.839757in}{2.233162in}}%
\pgfpathlineto{\pgfqpoint{1.851500in}{2.236665in}}%
\pgfpathlineto{\pgfqpoint{1.863237in}{2.240166in}}%
\pgfpathlineto{\pgfqpoint{1.874969in}{2.243667in}}%
\pgfpathlineto{\pgfqpoint{1.886694in}{2.247169in}}%
\pgfpathlineto{\pgfqpoint{1.898414in}{2.250672in}}%
\pgfpathlineto{\pgfqpoint{1.892527in}{2.259440in}}%
\pgfpathlineto{\pgfqpoint{1.886645in}{2.268182in}}%
\pgfpathlineto{\pgfqpoint{1.880767in}{2.276898in}}%
\pgfpathlineto{\pgfqpoint{1.874894in}{2.285589in}}%
\pgfpathlineto{\pgfqpoint{1.869025in}{2.294255in}}%
\pgfpathlineto{\pgfqpoint{1.857316in}{2.290796in}}%
\pgfpathlineto{\pgfqpoint{1.845602in}{2.287338in}}%
\pgfpathlineto{\pgfqpoint{1.833881in}{2.283882in}}%
\pgfpathlineto{\pgfqpoint{1.822155in}{2.280426in}}%
\pgfpathlineto{\pgfqpoint{1.810423in}{2.276968in}}%
\pgfpathlineto{\pgfqpoint{1.816281in}{2.268258in}}%
\pgfpathlineto{\pgfqpoint{1.822143in}{2.259522in}}%
\pgfpathlineto{\pgfqpoint{1.828010in}{2.250761in}}%
\pgfpathlineto{\pgfqpoint{1.833881in}{2.241974in}}%
\pgfpathclose%
\pgfusepath{stroke,fill}%
\end{pgfscope}%
\begin{pgfscope}%
\pgfpathrectangle{\pgfqpoint{0.887500in}{0.275000in}}{\pgfqpoint{4.225000in}{4.225000in}}%
\pgfusepath{clip}%
\pgfsetbuttcap%
\pgfsetroundjoin%
\definecolor{currentfill}{rgb}{0.203063,0.379716,0.553925}%
\pgfsetfillcolor{currentfill}%
\pgfsetfillopacity{0.700000}%
\pgfsetlinewidth{0.501875pt}%
\definecolor{currentstroke}{rgb}{1.000000,1.000000,1.000000}%
\pgfsetstrokecolor{currentstroke}%
\pgfsetstrokeopacity{0.500000}%
\pgfsetdash{}{0pt}%
\pgfpathmoveto{\pgfqpoint{4.361202in}{1.943693in}}%
\pgfpathlineto{\pgfqpoint{4.372327in}{1.947485in}}%
\pgfpathlineto{\pgfqpoint{4.383445in}{1.951222in}}%
\pgfpathlineto{\pgfqpoint{4.394555in}{1.954908in}}%
\pgfpathlineto{\pgfqpoint{4.405659in}{1.958542in}}%
\pgfpathlineto{\pgfqpoint{4.416755in}{1.962125in}}%
\pgfpathlineto{\pgfqpoint{4.410233in}{1.976510in}}%
\pgfpathlineto{\pgfqpoint{4.403711in}{1.990807in}}%
\pgfpathlineto{\pgfqpoint{4.397188in}{2.005006in}}%
\pgfpathlineto{\pgfqpoint{4.390664in}{2.019096in}}%
\pgfpathlineto{\pgfqpoint{4.384139in}{2.033067in}}%
\pgfpathlineto{\pgfqpoint{4.373041in}{2.029387in}}%
\pgfpathlineto{\pgfqpoint{4.361936in}{2.025636in}}%
\pgfpathlineto{\pgfqpoint{4.350822in}{2.021804in}}%
\pgfpathlineto{\pgfqpoint{4.339701in}{2.017884in}}%
\pgfpathlineto{\pgfqpoint{4.328571in}{2.013871in}}%
\pgfpathlineto{\pgfqpoint{4.335093in}{1.999867in}}%
\pgfpathlineto{\pgfqpoint{4.341617in}{1.985851in}}%
\pgfpathlineto{\pgfqpoint{4.348143in}{1.971820in}}%
\pgfpathlineto{\pgfqpoint{4.354672in}{1.957769in}}%
\pgfpathclose%
\pgfusepath{stroke,fill}%
\end{pgfscope}%
\begin{pgfscope}%
\pgfpathrectangle{\pgfqpoint{0.887500in}{0.275000in}}{\pgfqpoint{4.225000in}{4.225000in}}%
\pgfusepath{clip}%
\pgfsetbuttcap%
\pgfsetroundjoin%
\definecolor{currentfill}{rgb}{0.160665,0.478540,0.558115}%
\pgfsetfillcolor{currentfill}%
\pgfsetfillopacity{0.700000}%
\pgfsetlinewidth{0.501875pt}%
\definecolor{currentstroke}{rgb}{1.000000,1.000000,1.000000}%
\pgfsetstrokecolor{currentstroke}%
\pgfsetstrokeopacity{0.500000}%
\pgfsetdash{}{0pt}%
\pgfpathmoveto{\pgfqpoint{3.157776in}{2.109812in}}%
\pgfpathlineto{\pgfqpoint{3.169231in}{2.131256in}}%
\pgfpathlineto{\pgfqpoint{3.180686in}{2.151570in}}%
\pgfpathlineto{\pgfqpoint{3.192142in}{2.170687in}}%
\pgfpathlineto{\pgfqpoint{3.203597in}{2.188728in}}%
\pgfpathlineto{\pgfqpoint{3.215051in}{2.205810in}}%
\pgfpathlineto{\pgfqpoint{3.208752in}{2.221204in}}%
\pgfpathlineto{\pgfqpoint{3.202454in}{2.236122in}}%
\pgfpathlineto{\pgfqpoint{3.196156in}{2.250563in}}%
\pgfpathlineto{\pgfqpoint{3.189858in}{2.264530in}}%
\pgfpathlineto{\pgfqpoint{3.183562in}{2.278061in}}%
\pgfpathlineto{\pgfqpoint{3.172117in}{2.259339in}}%
\pgfpathlineto{\pgfqpoint{3.160672in}{2.239513in}}%
\pgfpathlineto{\pgfqpoint{3.149228in}{2.218419in}}%
\pgfpathlineto{\pgfqpoint{3.137786in}{2.195894in}}%
\pgfpathlineto{\pgfqpoint{3.126347in}{2.171995in}}%
\pgfpathlineto{\pgfqpoint{3.132628in}{2.160444in}}%
\pgfpathlineto{\pgfqpoint{3.138912in}{2.148468in}}%
\pgfpathlineto{\pgfqpoint{3.145199in}{2.136029in}}%
\pgfpathlineto{\pgfqpoint{3.151487in}{2.123138in}}%
\pgfpathclose%
\pgfusepath{stroke,fill}%
\end{pgfscope}%
\begin{pgfscope}%
\pgfpathrectangle{\pgfqpoint{0.887500in}{0.275000in}}{\pgfqpoint{4.225000in}{4.225000in}}%
\pgfusepath{clip}%
\pgfsetbuttcap%
\pgfsetroundjoin%
\definecolor{currentfill}{rgb}{0.120638,0.625828,0.533488}%
\pgfsetfillcolor{currentfill}%
\pgfsetfillopacity{0.700000}%
\pgfsetlinewidth{0.501875pt}%
\definecolor{currentstroke}{rgb}{1.000000,1.000000,1.000000}%
\pgfsetstrokecolor{currentstroke}%
\pgfsetstrokeopacity{0.500000}%
\pgfsetdash{}{0pt}%
\pgfpathmoveto{\pgfqpoint{3.678307in}{2.458853in}}%
\pgfpathlineto{\pgfqpoint{3.689611in}{2.463451in}}%
\pgfpathlineto{\pgfqpoint{3.700908in}{2.467920in}}%
\pgfpathlineto{\pgfqpoint{3.712199in}{2.472285in}}%
\pgfpathlineto{\pgfqpoint{3.723482in}{2.476556in}}%
\pgfpathlineto{\pgfqpoint{3.734760in}{2.480734in}}%
\pgfpathlineto{\pgfqpoint{3.728361in}{2.494772in}}%
\pgfpathlineto{\pgfqpoint{3.721965in}{2.508808in}}%
\pgfpathlineto{\pgfqpoint{3.715571in}{2.522861in}}%
\pgfpathlineto{\pgfqpoint{3.709181in}{2.536945in}}%
\pgfpathlineto{\pgfqpoint{3.702794in}{2.551054in}}%
\pgfpathlineto{\pgfqpoint{3.691523in}{2.547121in}}%
\pgfpathlineto{\pgfqpoint{3.680248in}{2.543173in}}%
\pgfpathlineto{\pgfqpoint{3.668967in}{2.539220in}}%
\pgfpathlineto{\pgfqpoint{3.657681in}{2.535255in}}%
\pgfpathlineto{\pgfqpoint{3.646388in}{2.531233in}}%
\pgfpathlineto{\pgfqpoint{3.652768in}{2.516820in}}%
\pgfpathlineto{\pgfqpoint{3.659150in}{2.502373in}}%
\pgfpathlineto{\pgfqpoint{3.665533in}{2.487895in}}%
\pgfpathlineto{\pgfqpoint{3.671919in}{2.473388in}}%
\pgfpathclose%
\pgfusepath{stroke,fill}%
\end{pgfscope}%
\begin{pgfscope}%
\pgfpathrectangle{\pgfqpoint{0.887500in}{0.275000in}}{\pgfqpoint{4.225000in}{4.225000in}}%
\pgfusepath{clip}%
\pgfsetbuttcap%
\pgfsetroundjoin%
\definecolor{currentfill}{rgb}{0.199430,0.387607,0.554642}%
\pgfsetfillcolor{currentfill}%
\pgfsetfillopacity{0.700000}%
\pgfsetlinewidth{0.501875pt}%
\definecolor{currentstroke}{rgb}{1.000000,1.000000,1.000000}%
\pgfsetstrokecolor{currentstroke}%
\pgfsetstrokeopacity{0.500000}%
\pgfsetdash{}{0pt}%
\pgfpathmoveto{\pgfqpoint{2.986207in}{1.970617in}}%
\pgfpathlineto{\pgfqpoint{2.997671in}{1.973699in}}%
\pgfpathlineto{\pgfqpoint{3.009129in}{1.975964in}}%
\pgfpathlineto{\pgfqpoint{3.020582in}{1.977255in}}%
\pgfpathlineto{\pgfqpoint{3.032029in}{1.977951in}}%
\pgfpathlineto{\pgfqpoint{3.043468in}{1.978791in}}%
\pgfpathlineto{\pgfqpoint{3.037204in}{1.988499in}}%
\pgfpathlineto{\pgfqpoint{3.030944in}{1.998202in}}%
\pgfpathlineto{\pgfqpoint{3.024688in}{2.007903in}}%
\pgfpathlineto{\pgfqpoint{3.018436in}{2.017603in}}%
\pgfpathlineto{\pgfqpoint{3.012188in}{2.027299in}}%
\pgfpathlineto{\pgfqpoint{3.000756in}{2.026918in}}%
\pgfpathlineto{\pgfqpoint{2.989316in}{2.026704in}}%
\pgfpathlineto{\pgfqpoint{2.977870in}{2.025807in}}%
\pgfpathlineto{\pgfqpoint{2.966420in}{2.023787in}}%
\pgfpathlineto{\pgfqpoint{2.954965in}{2.020828in}}%
\pgfpathlineto{\pgfqpoint{2.961206in}{2.010905in}}%
\pgfpathlineto{\pgfqpoint{2.967450in}{2.000924in}}%
\pgfpathlineto{\pgfqpoint{2.973699in}{1.990882in}}%
\pgfpathlineto{\pgfqpoint{2.979951in}{1.980779in}}%
\pgfpathclose%
\pgfusepath{stroke,fill}%
\end{pgfscope}%
\begin{pgfscope}%
\pgfpathrectangle{\pgfqpoint{0.887500in}{0.275000in}}{\pgfqpoint{4.225000in}{4.225000in}}%
\pgfusepath{clip}%
\pgfsetbuttcap%
\pgfsetroundjoin%
\definecolor{currentfill}{rgb}{0.146616,0.673050,0.508936}%
\pgfsetfillcolor{currentfill}%
\pgfsetfillopacity{0.700000}%
\pgfsetlinewidth{0.501875pt}%
\definecolor{currentstroke}{rgb}{1.000000,1.000000,1.000000}%
\pgfsetstrokecolor{currentstroke}%
\pgfsetstrokeopacity{0.500000}%
\pgfsetdash{}{0pt}%
\pgfpathmoveto{\pgfqpoint{3.266602in}{2.536121in}}%
\pgfpathlineto{\pgfqpoint{3.278050in}{2.551941in}}%
\pgfpathlineto{\pgfqpoint{3.289494in}{2.566685in}}%
\pgfpathlineto{\pgfqpoint{3.300933in}{2.580450in}}%
\pgfpathlineto{\pgfqpoint{3.312366in}{2.593330in}}%
\pgfpathlineto{\pgfqpoint{3.323794in}{2.605422in}}%
\pgfpathlineto{\pgfqpoint{3.317495in}{2.623038in}}%
\pgfpathlineto{\pgfqpoint{3.311194in}{2.640050in}}%
\pgfpathlineto{\pgfqpoint{3.304889in}{2.656259in}}%
\pgfpathlineto{\pgfqpoint{3.298580in}{2.671467in}}%
\pgfpathlineto{\pgfqpoint{3.292266in}{2.685586in}}%
\pgfpathlineto{\pgfqpoint{3.280852in}{2.674677in}}%
\pgfpathlineto{\pgfqpoint{3.269432in}{2.662806in}}%
\pgfpathlineto{\pgfqpoint{3.258005in}{2.649741in}}%
\pgfpathlineto{\pgfqpoint{3.246571in}{2.635249in}}%
\pgfpathlineto{\pgfqpoint{3.235130in}{2.619098in}}%
\pgfpathlineto{\pgfqpoint{3.241434in}{2.605266in}}%
\pgfpathlineto{\pgfqpoint{3.247733in}{2.589876in}}%
\pgfpathlineto{\pgfqpoint{3.254027in}{2.573041in}}%
\pgfpathlineto{\pgfqpoint{3.260316in}{2.555031in}}%
\pgfpathclose%
\pgfusepath{stroke,fill}%
\end{pgfscope}%
\begin{pgfscope}%
\pgfpathrectangle{\pgfqpoint{0.887500in}{0.275000in}}{\pgfqpoint{4.225000in}{4.225000in}}%
\pgfusepath{clip}%
\pgfsetbuttcap%
\pgfsetroundjoin%
\definecolor{currentfill}{rgb}{0.162016,0.687316,0.499129}%
\pgfsetfillcolor{currentfill}%
\pgfsetfillopacity{0.700000}%
\pgfsetlinewidth{0.501875pt}%
\definecolor{currentstroke}{rgb}{1.000000,1.000000,1.000000}%
\pgfsetstrokecolor{currentstroke}%
\pgfsetstrokeopacity{0.500000}%
\pgfsetdash{}{0pt}%
\pgfpathmoveto{\pgfqpoint{3.412455in}{2.576951in}}%
\pgfpathlineto{\pgfqpoint{3.423862in}{2.586745in}}%
\pgfpathlineto{\pgfqpoint{3.435262in}{2.595861in}}%
\pgfpathlineto{\pgfqpoint{3.446653in}{2.604382in}}%
\pgfpathlineto{\pgfqpoint{3.458036in}{2.612386in}}%
\pgfpathlineto{\pgfqpoint{3.469411in}{2.619920in}}%
\pgfpathlineto{\pgfqpoint{3.463070in}{2.634755in}}%
\pgfpathlineto{\pgfqpoint{3.456733in}{2.649702in}}%
\pgfpathlineto{\pgfqpoint{3.450398in}{2.664705in}}%
\pgfpathlineto{\pgfqpoint{3.444065in}{2.679711in}}%
\pgfpathlineto{\pgfqpoint{3.437735in}{2.694665in}}%
\pgfpathlineto{\pgfqpoint{3.426374in}{2.688164in}}%
\pgfpathlineto{\pgfqpoint{3.415005in}{2.681141in}}%
\pgfpathlineto{\pgfqpoint{3.403627in}{2.673574in}}%
\pgfpathlineto{\pgfqpoint{3.392242in}{2.665463in}}%
\pgfpathlineto{\pgfqpoint{3.380850in}{2.656810in}}%
\pgfpathlineto{\pgfqpoint{3.387167in}{2.640792in}}%
\pgfpathlineto{\pgfqpoint{3.393485in}{2.624695in}}%
\pgfpathlineto{\pgfqpoint{3.399805in}{2.608621in}}%
\pgfpathlineto{\pgfqpoint{3.406128in}{2.592672in}}%
\pgfpathclose%
\pgfusepath{stroke,fill}%
\end{pgfscope}%
\begin{pgfscope}%
\pgfpathrectangle{\pgfqpoint{0.887500in}{0.275000in}}{\pgfqpoint{4.225000in}{4.225000in}}%
\pgfusepath{clip}%
\pgfsetbuttcap%
\pgfsetroundjoin%
\definecolor{currentfill}{rgb}{0.177423,0.437527,0.557565}%
\pgfsetfillcolor{currentfill}%
\pgfsetfillopacity{0.700000}%
\pgfsetlinewidth{0.501875pt}%
\definecolor{currentstroke}{rgb}{1.000000,1.000000,1.000000}%
\pgfsetstrokecolor{currentstroke}%
\pgfsetstrokeopacity{0.500000}%
\pgfsetdash{}{0pt}%
\pgfpathmoveto{\pgfqpoint{2.574433in}{2.074482in}}%
\pgfpathlineto{\pgfqpoint{2.585997in}{2.078027in}}%
\pgfpathlineto{\pgfqpoint{2.597556in}{2.081581in}}%
\pgfpathlineto{\pgfqpoint{2.609108in}{2.085146in}}%
\pgfpathlineto{\pgfqpoint{2.620655in}{2.088724in}}%
\pgfpathlineto{\pgfqpoint{2.632196in}{2.092319in}}%
\pgfpathlineto{\pgfqpoint{2.626060in}{2.101699in}}%
\pgfpathlineto{\pgfqpoint{2.619929in}{2.111041in}}%
\pgfpathlineto{\pgfqpoint{2.613802in}{2.120346in}}%
\pgfpathlineto{\pgfqpoint{2.607679in}{2.129615in}}%
\pgfpathlineto{\pgfqpoint{2.601560in}{2.138850in}}%
\pgfpathlineto{\pgfqpoint{2.590030in}{2.135293in}}%
\pgfpathlineto{\pgfqpoint{2.578494in}{2.131756in}}%
\pgfpathlineto{\pgfqpoint{2.566951in}{2.128236in}}%
\pgfpathlineto{\pgfqpoint{2.555404in}{2.124728in}}%
\pgfpathlineto{\pgfqpoint{2.543850in}{2.121231in}}%
\pgfpathlineto{\pgfqpoint{2.549958in}{2.111956in}}%
\pgfpathlineto{\pgfqpoint{2.556070in}{2.102644in}}%
\pgfpathlineto{\pgfqpoint{2.562187in}{2.093295in}}%
\pgfpathlineto{\pgfqpoint{2.568308in}{2.083908in}}%
\pgfpathclose%
\pgfusepath{stroke,fill}%
\end{pgfscope}%
\begin{pgfscope}%
\pgfpathrectangle{\pgfqpoint{0.887500in}{0.275000in}}{\pgfqpoint{4.225000in}{4.225000in}}%
\pgfusepath{clip}%
\pgfsetbuttcap%
\pgfsetroundjoin%
\definecolor{currentfill}{rgb}{0.190631,0.407061,0.556089}%
\pgfsetfillcolor{currentfill}%
\pgfsetfillopacity{0.700000}%
\pgfsetlinewidth{0.501875pt}%
\definecolor{currentstroke}{rgb}{1.000000,1.000000,1.000000}%
\pgfsetstrokecolor{currentstroke}%
\pgfsetstrokeopacity{0.500000}%
\pgfsetdash{}{0pt}%
\pgfpathmoveto{\pgfqpoint{4.272806in}{1.992541in}}%
\pgfpathlineto{\pgfqpoint{4.283974in}{1.996966in}}%
\pgfpathlineto{\pgfqpoint{4.295135in}{2.001315in}}%
\pgfpathlineto{\pgfqpoint{4.306288in}{2.005584in}}%
\pgfpathlineto{\pgfqpoint{4.317433in}{2.009770in}}%
\pgfpathlineto{\pgfqpoint{4.328571in}{2.013871in}}%
\pgfpathlineto{\pgfqpoint{4.322052in}{2.027869in}}%
\pgfpathlineto{\pgfqpoint{4.315536in}{2.041866in}}%
\pgfpathlineto{\pgfqpoint{4.309022in}{2.055868in}}%
\pgfpathlineto{\pgfqpoint{4.302512in}{2.069884in}}%
\pgfpathlineto{\pgfqpoint{4.296005in}{2.083920in}}%
\pgfpathlineto{\pgfqpoint{4.284880in}{2.080157in}}%
\pgfpathlineto{\pgfqpoint{4.273748in}{2.076298in}}%
\pgfpathlineto{\pgfqpoint{4.262607in}{2.072355in}}%
\pgfpathlineto{\pgfqpoint{4.251459in}{2.068339in}}%
\pgfpathlineto{\pgfqpoint{4.240304in}{2.064261in}}%
\pgfpathlineto{\pgfqpoint{4.246792in}{2.049685in}}%
\pgfpathlineto{\pgfqpoint{4.253286in}{2.035206in}}%
\pgfpathlineto{\pgfqpoint{4.259785in}{2.020843in}}%
\pgfpathlineto{\pgfqpoint{4.266292in}{2.006617in}}%
\pgfpathclose%
\pgfusepath{stroke,fill}%
\end{pgfscope}%
\begin{pgfscope}%
\pgfpathrectangle{\pgfqpoint{0.887500in}{0.275000in}}{\pgfqpoint{4.225000in}{4.225000in}}%
\pgfusepath{clip}%
\pgfsetbuttcap%
\pgfsetroundjoin%
\definecolor{currentfill}{rgb}{0.144759,0.519093,0.556572}%
\pgfsetfillcolor{currentfill}%
\pgfsetfillopacity{0.700000}%
\pgfsetlinewidth{0.501875pt}%
\definecolor{currentstroke}{rgb}{1.000000,1.000000,1.000000}%
\pgfsetstrokecolor{currentstroke}%
\pgfsetstrokeopacity{0.500000}%
\pgfsetdash{}{0pt}%
\pgfpathmoveto{\pgfqpoint{3.975654in}{2.231032in}}%
\pgfpathlineto{\pgfqpoint{3.986879in}{2.234989in}}%
\pgfpathlineto{\pgfqpoint{3.998096in}{2.238792in}}%
\pgfpathlineto{\pgfqpoint{4.009305in}{2.242463in}}%
\pgfpathlineto{\pgfqpoint{4.020505in}{2.246029in}}%
\pgfpathlineto{\pgfqpoint{4.031698in}{2.249513in}}%
\pgfpathlineto{\pgfqpoint{4.025242in}{2.263641in}}%
\pgfpathlineto{\pgfqpoint{4.018789in}{2.277818in}}%
\pgfpathlineto{\pgfqpoint{4.012341in}{2.292070in}}%
\pgfpathlineto{\pgfqpoint{4.005898in}{2.306391in}}%
\pgfpathlineto{\pgfqpoint{3.999458in}{2.320766in}}%
\pgfpathlineto{\pgfqpoint{3.988267in}{2.317256in}}%
\pgfpathlineto{\pgfqpoint{3.977069in}{2.313705in}}%
\pgfpathlineto{\pgfqpoint{3.965864in}{2.310099in}}%
\pgfpathlineto{\pgfqpoint{3.954653in}{2.306422in}}%
\pgfpathlineto{\pgfqpoint{3.943434in}{2.302660in}}%
\pgfpathlineto{\pgfqpoint{3.949869in}{2.288155in}}%
\pgfpathlineto{\pgfqpoint{3.956308in}{2.273728in}}%
\pgfpathlineto{\pgfqpoint{3.962752in}{2.259399in}}%
\pgfpathlineto{\pgfqpoint{3.969201in}{2.245178in}}%
\pgfpathclose%
\pgfusepath{stroke,fill}%
\end{pgfscope}%
\begin{pgfscope}%
\pgfpathrectangle{\pgfqpoint{0.887500in}{0.275000in}}{\pgfqpoint{4.225000in}{4.225000in}}%
\pgfusepath{clip}%
\pgfsetbuttcap%
\pgfsetroundjoin%
\definecolor{currentfill}{rgb}{0.246811,0.283237,0.535941}%
\pgfsetfillcolor{currentfill}%
\pgfsetfillopacity{0.700000}%
\pgfsetlinewidth{0.501875pt}%
\definecolor{currentstroke}{rgb}{1.000000,1.000000,1.000000}%
\pgfsetstrokecolor{currentstroke}%
\pgfsetstrokeopacity{0.500000}%
\pgfsetdash{}{0pt}%
\pgfpathmoveto{\pgfqpoint{4.570131in}{1.759246in}}%
\pgfpathlineto{\pgfqpoint{4.581185in}{1.762499in}}%
\pgfpathlineto{\pgfqpoint{4.592234in}{1.765757in}}%
\pgfpathlineto{\pgfqpoint{4.603278in}{1.769017in}}%
\pgfpathlineto{\pgfqpoint{4.614315in}{1.772279in}}%
\pgfpathlineto{\pgfqpoint{4.625348in}{1.775542in}}%
\pgfpathlineto{\pgfqpoint{4.618792in}{1.790120in}}%
\pgfpathlineto{\pgfqpoint{4.612240in}{1.804739in}}%
\pgfpathlineto{\pgfqpoint{4.605693in}{1.819392in}}%
\pgfpathlineto{\pgfqpoint{4.599148in}{1.834072in}}%
\pgfpathlineto{\pgfqpoint{4.592608in}{1.848775in}}%
\pgfpathlineto{\pgfqpoint{4.581570in}{1.845354in}}%
\pgfpathlineto{\pgfqpoint{4.570527in}{1.841930in}}%
\pgfpathlineto{\pgfqpoint{4.559478in}{1.838516in}}%
\pgfpathlineto{\pgfqpoint{4.548425in}{1.835124in}}%
\pgfpathlineto{\pgfqpoint{4.537368in}{1.831766in}}%
\pgfpathlineto{\pgfqpoint{4.543909in}{1.817090in}}%
\pgfpathlineto{\pgfqpoint{4.550455in}{1.802486in}}%
\pgfpathlineto{\pgfqpoint{4.557007in}{1.787968in}}%
\pgfpathlineto{\pgfqpoint{4.563565in}{1.773550in}}%
\pgfpathclose%
\pgfusepath{stroke,fill}%
\end{pgfscope}%
\begin{pgfscope}%
\pgfpathrectangle{\pgfqpoint{0.887500in}{0.275000in}}{\pgfqpoint{4.225000in}{4.225000in}}%
\pgfusepath{clip}%
\pgfsetbuttcap%
\pgfsetroundjoin%
\definecolor{currentfill}{rgb}{0.128087,0.647749,0.523491}%
\pgfsetfillcolor{currentfill}%
\pgfsetfillopacity{0.700000}%
\pgfsetlinewidth{0.501875pt}%
\definecolor{currentstroke}{rgb}{1.000000,1.000000,1.000000}%
\pgfsetstrokecolor{currentstroke}%
\pgfsetstrokeopacity{0.500000}%
\pgfsetdash{}{0pt}%
\pgfpathmoveto{\pgfqpoint{3.589814in}{2.508608in}}%
\pgfpathlineto{\pgfqpoint{3.601146in}{2.513629in}}%
\pgfpathlineto{\pgfqpoint{3.612468in}{2.518354in}}%
\pgfpathlineto{\pgfqpoint{3.623783in}{2.522831in}}%
\pgfpathlineto{\pgfqpoint{3.635089in}{2.527108in}}%
\pgfpathlineto{\pgfqpoint{3.646388in}{2.531233in}}%
\pgfpathlineto{\pgfqpoint{3.640010in}{2.545608in}}%
\pgfpathlineto{\pgfqpoint{3.633634in}{2.559942in}}%
\pgfpathlineto{\pgfqpoint{3.627260in}{2.574230in}}%
\pgfpathlineto{\pgfqpoint{3.620888in}{2.588470in}}%
\pgfpathlineto{\pgfqpoint{3.614517in}{2.602657in}}%
\pgfpathlineto{\pgfqpoint{3.603223in}{2.598726in}}%
\pgfpathlineto{\pgfqpoint{3.591922in}{2.594613in}}%
\pgfpathlineto{\pgfqpoint{3.580613in}{2.590250in}}%
\pgfpathlineto{\pgfqpoint{3.569294in}{2.585570in}}%
\pgfpathlineto{\pgfqpoint{3.557966in}{2.580507in}}%
\pgfpathlineto{\pgfqpoint{3.564333in}{2.566324in}}%
\pgfpathlineto{\pgfqpoint{3.570702in}{2.552064in}}%
\pgfpathlineto{\pgfqpoint{3.577072in}{2.537705in}}%
\pgfpathlineto{\pgfqpoint{3.583443in}{2.523226in}}%
\pgfpathclose%
\pgfusepath{stroke,fill}%
\end{pgfscope}%
\begin{pgfscope}%
\pgfpathrectangle{\pgfqpoint{0.887500in}{0.275000in}}{\pgfqpoint{4.225000in}{4.225000in}}%
\pgfusepath{clip}%
\pgfsetbuttcap%
\pgfsetroundjoin%
\definecolor{currentfill}{rgb}{0.192357,0.403199,0.555836}%
\pgfsetfillcolor{currentfill}%
\pgfsetfillopacity{0.700000}%
\pgfsetlinewidth{0.501875pt}%
\definecolor{currentstroke}{rgb}{1.000000,1.000000,1.000000}%
\pgfsetstrokecolor{currentstroke}%
\pgfsetstrokeopacity{0.500000}%
\pgfsetdash{}{0pt}%
\pgfpathmoveto{\pgfqpoint{3.132015in}{1.958713in}}%
\pgfpathlineto{\pgfqpoint{3.143452in}{1.971405in}}%
\pgfpathlineto{\pgfqpoint{3.154893in}{1.986107in}}%
\pgfpathlineto{\pgfqpoint{3.166340in}{2.002316in}}%
\pgfpathlineto{\pgfqpoint{3.177791in}{2.019527in}}%
\pgfpathlineto{\pgfqpoint{3.189246in}{2.037235in}}%
\pgfpathlineto{\pgfqpoint{3.182950in}{2.052488in}}%
\pgfpathlineto{\pgfqpoint{3.176655in}{2.067389in}}%
\pgfpathlineto{\pgfqpoint{3.170361in}{2.081921in}}%
\pgfpathlineto{\pgfqpoint{3.164068in}{2.096068in}}%
\pgfpathlineto{\pgfqpoint{3.157776in}{2.109812in}}%
\pgfpathlineto{\pgfqpoint{3.146326in}{2.087898in}}%
\pgfpathlineto{\pgfqpoint{3.134881in}{2.066293in}}%
\pgfpathlineto{\pgfqpoint{3.123443in}{2.045777in}}%
\pgfpathlineto{\pgfqpoint{3.112011in}{2.027123in}}%
\pgfpathlineto{\pgfqpoint{3.100586in}{2.011106in}}%
\pgfpathlineto{\pgfqpoint{3.106865in}{2.000779in}}%
\pgfpathlineto{\pgfqpoint{3.113147in}{1.990375in}}%
\pgfpathlineto{\pgfqpoint{3.119433in}{1.979896in}}%
\pgfpathlineto{\pgfqpoint{3.125723in}{1.969342in}}%
\pgfpathclose%
\pgfusepath{stroke,fill}%
\end{pgfscope}%
\begin{pgfscope}%
\pgfpathrectangle{\pgfqpoint{0.887500in}{0.275000in}}{\pgfqpoint{4.225000in}{4.225000in}}%
\pgfusepath{clip}%
\pgfsetbuttcap%
\pgfsetroundjoin%
\definecolor{currentfill}{rgb}{0.120565,0.596422,0.543611}%
\pgfsetfillcolor{currentfill}%
\pgfsetfillopacity{0.700000}%
\pgfsetlinewidth{0.501875pt}%
\definecolor{currentstroke}{rgb}{1.000000,1.000000,1.000000}%
\pgfsetstrokecolor{currentstroke}%
\pgfsetstrokeopacity{0.500000}%
\pgfsetdash{}{0pt}%
\pgfpathmoveto{\pgfqpoint{3.240785in}{2.360789in}}%
\pgfpathlineto{\pgfqpoint{3.252230in}{2.376161in}}%
\pgfpathlineto{\pgfqpoint{3.263675in}{2.391413in}}%
\pgfpathlineto{\pgfqpoint{3.275121in}{2.406655in}}%
\pgfpathlineto{\pgfqpoint{3.286569in}{2.421997in}}%
\pgfpathlineto{\pgfqpoint{3.298020in}{2.437550in}}%
\pgfpathlineto{\pgfqpoint{3.291733in}{2.456897in}}%
\pgfpathlineto{\pgfqpoint{3.285449in}{2.476698in}}%
\pgfpathlineto{\pgfqpoint{3.279167in}{2.496682in}}%
\pgfpathlineto{\pgfqpoint{3.272885in}{2.516580in}}%
\pgfpathlineto{\pgfqpoint{3.266602in}{2.536121in}}%
\pgfpathlineto{\pgfqpoint{3.255149in}{2.519154in}}%
\pgfpathlineto{\pgfqpoint{3.243693in}{2.501151in}}%
\pgfpathlineto{\pgfqpoint{3.232237in}{2.482308in}}%
\pgfpathlineto{\pgfqpoint{3.220781in}{2.462818in}}%
\pgfpathlineto{\pgfqpoint{3.209327in}{2.442875in}}%
\pgfpathlineto{\pgfqpoint{3.215616in}{2.426780in}}%
\pgfpathlineto{\pgfqpoint{3.221906in}{2.410422in}}%
\pgfpathlineto{\pgfqpoint{3.228198in}{2.393904in}}%
\pgfpathlineto{\pgfqpoint{3.234491in}{2.377326in}}%
\pgfpathclose%
\pgfusepath{stroke,fill}%
\end{pgfscope}%
\begin{pgfscope}%
\pgfpathrectangle{\pgfqpoint{0.887500in}{0.275000in}}{\pgfqpoint{4.225000in}{4.225000in}}%
\pgfusepath{clip}%
\pgfsetbuttcap%
\pgfsetroundjoin%
\definecolor{currentfill}{rgb}{0.143303,0.669459,0.511215}%
\pgfsetfillcolor{currentfill}%
\pgfsetfillopacity{0.700000}%
\pgfsetlinewidth{0.501875pt}%
\definecolor{currentstroke}{rgb}{1.000000,1.000000,1.000000}%
\pgfsetstrokecolor{currentstroke}%
\pgfsetstrokeopacity{0.500000}%
\pgfsetdash{}{0pt}%
\pgfpathmoveto{\pgfqpoint{3.501181in}{2.548577in}}%
\pgfpathlineto{\pgfqpoint{3.512556in}{2.555877in}}%
\pgfpathlineto{\pgfqpoint{3.523922in}{2.562710in}}%
\pgfpathlineto{\pgfqpoint{3.535280in}{2.569087in}}%
\pgfpathlineto{\pgfqpoint{3.546627in}{2.575016in}}%
\pgfpathlineto{\pgfqpoint{3.557966in}{2.580507in}}%
\pgfpathlineto{\pgfqpoint{3.551600in}{2.594631in}}%
\pgfpathlineto{\pgfqpoint{3.545236in}{2.608718in}}%
\pgfpathlineto{\pgfqpoint{3.538875in}{2.622788in}}%
\pgfpathlineto{\pgfqpoint{3.532516in}{2.636860in}}%
\pgfpathlineto{\pgfqpoint{3.526160in}{2.650935in}}%
\pgfpathlineto{\pgfqpoint{3.514828in}{2.645584in}}%
\pgfpathlineto{\pgfqpoint{3.503487in}{2.639818in}}%
\pgfpathlineto{\pgfqpoint{3.492137in}{2.633625in}}%
\pgfpathlineto{\pgfqpoint{3.480778in}{2.626996in}}%
\pgfpathlineto{\pgfqpoint{3.469411in}{2.619920in}}%
\pgfpathlineto{\pgfqpoint{3.475756in}{2.605252in}}%
\pgfpathlineto{\pgfqpoint{3.482105in}{2.590804in}}%
\pgfpathlineto{\pgfqpoint{3.488460in}{2.576586in}}%
\pgfpathlineto{\pgfqpoint{3.494818in}{2.562534in}}%
\pgfpathclose%
\pgfusepath{stroke,fill}%
\end{pgfscope}%
\begin{pgfscope}%
\pgfpathrectangle{\pgfqpoint{0.887500in}{0.275000in}}{\pgfqpoint{4.225000in}{4.225000in}}%
\pgfusepath{clip}%
\pgfsetbuttcap%
\pgfsetroundjoin%
\definecolor{currentfill}{rgb}{0.163625,0.471133,0.558148}%
\pgfsetfillcolor{currentfill}%
\pgfsetfillopacity{0.700000}%
\pgfsetlinewidth{0.501875pt}%
\definecolor{currentstroke}{rgb}{1.000000,1.000000,1.000000}%
\pgfsetstrokecolor{currentstroke}%
\pgfsetstrokeopacity{0.500000}%
\pgfsetdash{}{0pt}%
\pgfpathmoveto{\pgfqpoint{2.251176in}{2.142012in}}%
\pgfpathlineto{\pgfqpoint{2.262820in}{2.145609in}}%
\pgfpathlineto{\pgfqpoint{2.274459in}{2.149211in}}%
\pgfpathlineto{\pgfqpoint{2.286091in}{2.152813in}}%
\pgfpathlineto{\pgfqpoint{2.297718in}{2.156414in}}%
\pgfpathlineto{\pgfqpoint{2.309340in}{2.160009in}}%
\pgfpathlineto{\pgfqpoint{2.303310in}{2.169075in}}%
\pgfpathlineto{\pgfqpoint{2.297286in}{2.178112in}}%
\pgfpathlineto{\pgfqpoint{2.291265in}{2.187120in}}%
\pgfpathlineto{\pgfqpoint{2.285249in}{2.196100in}}%
\pgfpathlineto{\pgfqpoint{2.279237in}{2.205052in}}%
\pgfpathlineto{\pgfqpoint{2.267626in}{2.201505in}}%
\pgfpathlineto{\pgfqpoint{2.256010in}{2.197954in}}%
\pgfpathlineto{\pgfqpoint{2.244388in}{2.194400in}}%
\pgfpathlineto{\pgfqpoint{2.232761in}{2.190848in}}%
\pgfpathlineto{\pgfqpoint{2.221127in}{2.187301in}}%
\pgfpathlineto{\pgfqpoint{2.227128in}{2.178301in}}%
\pgfpathlineto{\pgfqpoint{2.233134in}{2.169272in}}%
\pgfpathlineto{\pgfqpoint{2.239143in}{2.160214in}}%
\pgfpathlineto{\pgfqpoint{2.245158in}{2.151127in}}%
\pgfpathclose%
\pgfusepath{stroke,fill}%
\end{pgfscope}%
\begin{pgfscope}%
\pgfpathrectangle{\pgfqpoint{0.887500in}{0.275000in}}{\pgfqpoint{4.225000in}{4.225000in}}%
\pgfusepath{clip}%
\pgfsetbuttcap%
\pgfsetroundjoin%
\definecolor{currentfill}{rgb}{0.151918,0.500685,0.557587}%
\pgfsetfillcolor{currentfill}%
\pgfsetfillopacity{0.700000}%
\pgfsetlinewidth{0.501875pt}%
\definecolor{currentstroke}{rgb}{1.000000,1.000000,1.000000}%
\pgfsetstrokecolor{currentstroke}%
\pgfsetstrokeopacity{0.500000}%
\pgfsetdash{}{0pt}%
\pgfpathmoveto{\pgfqpoint{1.927916in}{2.206432in}}%
\pgfpathlineto{\pgfqpoint{1.939641in}{2.209977in}}%
\pgfpathlineto{\pgfqpoint{1.951361in}{2.213518in}}%
\pgfpathlineto{\pgfqpoint{1.963074in}{2.217053in}}%
\pgfpathlineto{\pgfqpoint{1.974783in}{2.220581in}}%
\pgfpathlineto{\pgfqpoint{1.986485in}{2.224099in}}%
\pgfpathlineto{\pgfqpoint{1.980565in}{2.232955in}}%
\pgfpathlineto{\pgfqpoint{1.974649in}{2.241786in}}%
\pgfpathlineto{\pgfqpoint{1.968738in}{2.250590in}}%
\pgfpathlineto{\pgfqpoint{1.962831in}{2.259368in}}%
\pgfpathlineto{\pgfqpoint{1.956928in}{2.268119in}}%
\pgfpathlineto{\pgfqpoint{1.945237in}{2.264647in}}%
\pgfpathlineto{\pgfqpoint{1.933539in}{2.261163in}}%
\pgfpathlineto{\pgfqpoint{1.921837in}{2.257671in}}%
\pgfpathlineto{\pgfqpoint{1.910128in}{2.254174in}}%
\pgfpathlineto{\pgfqpoint{1.898414in}{2.250672in}}%
\pgfpathlineto{\pgfqpoint{1.904306in}{2.241877in}}%
\pgfpathlineto{\pgfqpoint{1.910201in}{2.233056in}}%
\pgfpathlineto{\pgfqpoint{1.916102in}{2.224208in}}%
\pgfpathlineto{\pgfqpoint{1.922007in}{2.215333in}}%
\pgfpathclose%
\pgfusepath{stroke,fill}%
\end{pgfscope}%
\begin{pgfscope}%
\pgfpathrectangle{\pgfqpoint{0.887500in}{0.275000in}}{\pgfqpoint{4.225000in}{4.225000in}}%
\pgfusepath{clip}%
\pgfsetbuttcap%
\pgfsetroundjoin%
\definecolor{currentfill}{rgb}{0.206756,0.371758,0.553117}%
\pgfsetfillcolor{currentfill}%
\pgfsetfillopacity{0.700000}%
\pgfsetlinewidth{0.501875pt}%
\definecolor{currentstroke}{rgb}{1.000000,1.000000,1.000000}%
\pgfsetstrokecolor{currentstroke}%
\pgfsetstrokeopacity{0.500000}%
\pgfsetdash{}{0pt}%
\pgfpathmoveto{\pgfqpoint{3.074844in}{1.930077in}}%
\pgfpathlineto{\pgfqpoint{3.086284in}{1.932377in}}%
\pgfpathlineto{\pgfqpoint{3.097720in}{1.935861in}}%
\pgfpathlineto{\pgfqpoint{3.109152in}{1.941066in}}%
\pgfpathlineto{\pgfqpoint{3.120583in}{1.948533in}}%
\pgfpathlineto{\pgfqpoint{3.132015in}{1.958713in}}%
\pgfpathlineto{\pgfqpoint{3.125723in}{1.969342in}}%
\pgfpathlineto{\pgfqpoint{3.119433in}{1.979896in}}%
\pgfpathlineto{\pgfqpoint{3.113147in}{1.990375in}}%
\pgfpathlineto{\pgfqpoint{3.106865in}{2.000779in}}%
\pgfpathlineto{\pgfqpoint{3.100586in}{2.011106in}}%
\pgfpathlineto{\pgfqpoint{3.089165in}{1.998497in}}%
\pgfpathlineto{\pgfqpoint{3.077746in}{1.989630in}}%
\pgfpathlineto{\pgfqpoint{3.066325in}{1.983887in}}%
\pgfpathlineto{\pgfqpoint{3.054900in}{1.980522in}}%
\pgfpathlineto{\pgfqpoint{3.043468in}{1.978791in}}%
\pgfpathlineto{\pgfqpoint{3.049735in}{1.969076in}}%
\pgfpathlineto{\pgfqpoint{3.056007in}{1.959349in}}%
\pgfpathlineto{\pgfqpoint{3.062282in}{1.949609in}}%
\pgfpathlineto{\pgfqpoint{3.068561in}{1.939853in}}%
\pgfpathclose%
\pgfusepath{stroke,fill}%
\end{pgfscope}%
\begin{pgfscope}%
\pgfpathrectangle{\pgfqpoint{0.887500in}{0.275000in}}{\pgfqpoint{4.225000in}{4.225000in}}%
\pgfusepath{clip}%
\pgfsetbuttcap%
\pgfsetroundjoin%
\definecolor{currentfill}{rgb}{0.179019,0.433756,0.557430}%
\pgfsetfillcolor{currentfill}%
\pgfsetfillopacity{0.700000}%
\pgfsetlinewidth{0.501875pt}%
\definecolor{currentstroke}{rgb}{1.000000,1.000000,1.000000}%
\pgfsetstrokecolor{currentstroke}%
\pgfsetstrokeopacity{0.500000}%
\pgfsetdash{}{0pt}%
\pgfpathmoveto{\pgfqpoint{4.184436in}{2.043320in}}%
\pgfpathlineto{\pgfqpoint{4.195621in}{2.047547in}}%
\pgfpathlineto{\pgfqpoint{4.206801in}{2.051765in}}%
\pgfpathlineto{\pgfqpoint{4.217975in}{2.055964in}}%
\pgfpathlineto{\pgfqpoint{4.229143in}{2.060133in}}%
\pgfpathlineto{\pgfqpoint{4.240304in}{2.064261in}}%
\pgfpathlineto{\pgfqpoint{4.233821in}{2.078914in}}%
\pgfpathlineto{\pgfqpoint{4.227342in}{2.093624in}}%
\pgfpathlineto{\pgfqpoint{4.220867in}{2.108370in}}%
\pgfpathlineto{\pgfqpoint{4.214394in}{2.123134in}}%
\pgfpathlineto{\pgfqpoint{4.207924in}{2.137894in}}%
\pgfpathlineto{\pgfqpoint{4.196783in}{2.134394in}}%
\pgfpathlineto{\pgfqpoint{4.185635in}{2.130873in}}%
\pgfpathlineto{\pgfqpoint{4.174481in}{2.127335in}}%
\pgfpathlineto{\pgfqpoint{4.163322in}{2.123785in}}%
\pgfpathlineto{\pgfqpoint{4.152157in}{2.120227in}}%
\pgfpathlineto{\pgfqpoint{4.158609in}{2.104881in}}%
\pgfpathlineto{\pgfqpoint{4.165062in}{2.089466in}}%
\pgfpathlineto{\pgfqpoint{4.171516in}{2.074035in}}%
\pgfpathlineto{\pgfqpoint{4.177974in}{2.058636in}}%
\pgfpathclose%
\pgfusepath{stroke,fill}%
\end{pgfscope}%
\begin{pgfscope}%
\pgfpathrectangle{\pgfqpoint{0.887500in}{0.275000in}}{\pgfqpoint{4.225000in}{4.225000in}}%
\pgfusepath{clip}%
\pgfsetbuttcap%
\pgfsetroundjoin%
\definecolor{currentfill}{rgb}{0.135066,0.544853,0.554029}%
\pgfsetfillcolor{currentfill}%
\pgfsetfillopacity{0.700000}%
\pgfsetlinewidth{0.501875pt}%
\definecolor{currentstroke}{rgb}{1.000000,1.000000,1.000000}%
\pgfsetstrokecolor{currentstroke}%
\pgfsetstrokeopacity{0.500000}%
\pgfsetdash{}{0pt}%
\pgfpathmoveto{\pgfqpoint{3.887226in}{2.282234in}}%
\pgfpathlineto{\pgfqpoint{3.898484in}{2.286566in}}%
\pgfpathlineto{\pgfqpoint{3.909733in}{2.290765in}}%
\pgfpathlineto{\pgfqpoint{3.920974in}{2.294841in}}%
\pgfpathlineto{\pgfqpoint{3.932208in}{2.298802in}}%
\pgfpathlineto{\pgfqpoint{3.943434in}{2.302660in}}%
\pgfpathlineto{\pgfqpoint{3.937003in}{2.317221in}}%
\pgfpathlineto{\pgfqpoint{3.930575in}{2.331820in}}%
\pgfpathlineto{\pgfqpoint{3.924151in}{2.346437in}}%
\pgfpathlineto{\pgfqpoint{3.917728in}{2.361051in}}%
\pgfpathlineto{\pgfqpoint{3.911307in}{2.375642in}}%
\pgfpathlineto{\pgfqpoint{3.900090in}{2.372083in}}%
\pgfpathlineto{\pgfqpoint{3.888865in}{2.368462in}}%
\pgfpathlineto{\pgfqpoint{3.877634in}{2.364764in}}%
\pgfpathlineto{\pgfqpoint{3.866395in}{2.360973in}}%
\pgfpathlineto{\pgfqpoint{3.855149in}{2.357070in}}%
\pgfpathlineto{\pgfqpoint{3.861561in}{2.342154in}}%
\pgfpathlineto{\pgfqpoint{3.867974in}{2.327183in}}%
\pgfpathlineto{\pgfqpoint{3.874389in}{2.312187in}}%
\pgfpathlineto{\pgfqpoint{3.880806in}{2.297194in}}%
\pgfpathclose%
\pgfusepath{stroke,fill}%
\end{pgfscope}%
\begin{pgfscope}%
\pgfpathrectangle{\pgfqpoint{0.887500in}{0.275000in}}{\pgfqpoint{4.225000in}{4.225000in}}%
\pgfusepath{clip}%
\pgfsetbuttcap%
\pgfsetroundjoin%
\definecolor{currentfill}{rgb}{0.182256,0.426184,0.557120}%
\pgfsetfillcolor{currentfill}%
\pgfsetfillopacity{0.700000}%
\pgfsetlinewidth{0.501875pt}%
\definecolor{currentstroke}{rgb}{1.000000,1.000000,1.000000}%
\pgfsetstrokecolor{currentstroke}%
\pgfsetstrokeopacity{0.500000}%
\pgfsetdash{}{0pt}%
\pgfpathmoveto{\pgfqpoint{2.662937in}{2.044814in}}%
\pgfpathlineto{\pgfqpoint{2.674482in}{2.048470in}}%
\pgfpathlineto{\pgfqpoint{2.686021in}{2.052143in}}%
\pgfpathlineto{\pgfqpoint{2.697555in}{2.055830in}}%
\pgfpathlineto{\pgfqpoint{2.709083in}{2.059506in}}%
\pgfpathlineto{\pgfqpoint{2.720605in}{2.063149in}}%
\pgfpathlineto{\pgfqpoint{2.714438in}{2.072696in}}%
\pgfpathlineto{\pgfqpoint{2.708276in}{2.082202in}}%
\pgfpathlineto{\pgfqpoint{2.702117in}{2.091667in}}%
\pgfpathlineto{\pgfqpoint{2.695963in}{2.101093in}}%
\pgfpathlineto{\pgfqpoint{2.689813in}{2.110479in}}%
\pgfpathlineto{\pgfqpoint{2.678300in}{2.106869in}}%
\pgfpathlineto{\pgfqpoint{2.666783in}{2.103224in}}%
\pgfpathlineto{\pgfqpoint{2.655259in}{2.099571in}}%
\pgfpathlineto{\pgfqpoint{2.643730in}{2.095934in}}%
\pgfpathlineto{\pgfqpoint{2.632196in}{2.092319in}}%
\pgfpathlineto{\pgfqpoint{2.638335in}{2.082901in}}%
\pgfpathlineto{\pgfqpoint{2.644479in}{2.073442in}}%
\pgfpathlineto{\pgfqpoint{2.650627in}{2.063942in}}%
\pgfpathlineto{\pgfqpoint{2.656780in}{2.054400in}}%
\pgfpathclose%
\pgfusepath{stroke,fill}%
\end{pgfscope}%
\begin{pgfscope}%
\pgfpathrectangle{\pgfqpoint{0.887500in}{0.275000in}}{\pgfqpoint{4.225000in}{4.225000in}}%
\pgfusepath{clip}%
\pgfsetbuttcap%
\pgfsetroundjoin%
\definecolor{currentfill}{rgb}{0.143343,0.522773,0.556295}%
\pgfsetfillcolor{currentfill}%
\pgfsetfillopacity{0.700000}%
\pgfsetlinewidth{0.501875pt}%
\definecolor{currentstroke}{rgb}{1.000000,1.000000,1.000000}%
\pgfsetstrokecolor{currentstroke}%
\pgfsetstrokeopacity{0.500000}%
\pgfsetdash{}{0pt}%
\pgfpathmoveto{\pgfqpoint{3.215051in}{2.205810in}}%
\pgfpathlineto{\pgfqpoint{3.226504in}{2.222055in}}%
\pgfpathlineto{\pgfqpoint{3.237955in}{2.237581in}}%
\pgfpathlineto{\pgfqpoint{3.249404in}{2.252510in}}%
\pgfpathlineto{\pgfqpoint{3.260853in}{2.266976in}}%
\pgfpathlineto{\pgfqpoint{3.272300in}{2.281158in}}%
\pgfpathlineto{\pgfqpoint{3.265991in}{2.296696in}}%
\pgfpathlineto{\pgfqpoint{3.259685in}{2.312364in}}%
\pgfpathlineto{\pgfqpoint{3.253382in}{2.328240in}}%
\pgfpathlineto{\pgfqpoint{3.247082in}{2.344394in}}%
\pgfpathlineto{\pgfqpoint{3.240785in}{2.360789in}}%
\pgfpathlineto{\pgfqpoint{3.229341in}{2.345188in}}%
\pgfpathlineto{\pgfqpoint{3.217896in}{2.329247in}}%
\pgfpathlineto{\pgfqpoint{3.206452in}{2.312850in}}%
\pgfpathlineto{\pgfqpoint{3.195007in}{2.295844in}}%
\pgfpathlineto{\pgfqpoint{3.183562in}{2.278061in}}%
\pgfpathlineto{\pgfqpoint{3.189858in}{2.264530in}}%
\pgfpathlineto{\pgfqpoint{3.196156in}{2.250563in}}%
\pgfpathlineto{\pgfqpoint{3.202454in}{2.236122in}}%
\pgfpathlineto{\pgfqpoint{3.208752in}{2.221204in}}%
\pgfpathclose%
\pgfusepath{stroke,fill}%
\end{pgfscope}%
\begin{pgfscope}%
\pgfpathrectangle{\pgfqpoint{0.887500in}{0.275000in}}{\pgfqpoint{4.225000in}{4.225000in}}%
\pgfusepath{clip}%
\pgfsetbuttcap%
\pgfsetroundjoin%
\definecolor{currentfill}{rgb}{0.140210,0.665859,0.513427}%
\pgfsetfillcolor{currentfill}%
\pgfsetfillopacity{0.700000}%
\pgfsetlinewidth{0.501875pt}%
\definecolor{currentstroke}{rgb}{1.000000,1.000000,1.000000}%
\pgfsetstrokecolor{currentstroke}%
\pgfsetstrokeopacity{0.500000}%
\pgfsetdash{}{0pt}%
\pgfpathmoveto{\pgfqpoint{3.355292in}{2.515219in}}%
\pgfpathlineto{\pgfqpoint{3.366740in}{2.529410in}}%
\pgfpathlineto{\pgfqpoint{3.378181in}{2.542703in}}%
\pgfpathlineto{\pgfqpoint{3.389614in}{2.555011in}}%
\pgfpathlineto{\pgfqpoint{3.401039in}{2.566400in}}%
\pgfpathlineto{\pgfqpoint{3.412455in}{2.576951in}}%
\pgfpathlineto{\pgfqpoint{3.406128in}{2.592672in}}%
\pgfpathlineto{\pgfqpoint{3.399805in}{2.608621in}}%
\pgfpathlineto{\pgfqpoint{3.393485in}{2.624695in}}%
\pgfpathlineto{\pgfqpoint{3.387167in}{2.640792in}}%
\pgfpathlineto{\pgfqpoint{3.380850in}{2.656810in}}%
\pgfpathlineto{\pgfqpoint{3.369450in}{2.647618in}}%
\pgfpathlineto{\pgfqpoint{3.358045in}{2.637889in}}%
\pgfpathlineto{\pgfqpoint{3.346633in}{2.627624in}}%
\pgfpathlineto{\pgfqpoint{3.335216in}{2.616821in}}%
\pgfpathlineto{\pgfqpoint{3.323794in}{2.605422in}}%
\pgfpathlineto{\pgfqpoint{3.330091in}{2.587399in}}%
\pgfpathlineto{\pgfqpoint{3.336388in}{2.569170in}}%
\pgfpathlineto{\pgfqpoint{3.342687in}{2.550932in}}%
\pgfpathlineto{\pgfqpoint{3.348987in}{2.532882in}}%
\pgfpathclose%
\pgfusepath{stroke,fill}%
\end{pgfscope}%
\begin{pgfscope}%
\pgfpathrectangle{\pgfqpoint{0.887500in}{0.275000in}}{\pgfqpoint{4.225000in}{4.225000in}}%
\pgfusepath{clip}%
\pgfsetbuttcap%
\pgfsetroundjoin%
\definecolor{currentfill}{rgb}{0.231674,0.318106,0.544834}%
\pgfsetfillcolor{currentfill}%
\pgfsetfillopacity{0.700000}%
\pgfsetlinewidth{0.501875pt}%
\definecolor{currentstroke}{rgb}{1.000000,1.000000,1.000000}%
\pgfsetstrokecolor{currentstroke}%
\pgfsetstrokeopacity{0.500000}%
\pgfsetdash{}{0pt}%
\pgfpathmoveto{\pgfqpoint{4.482028in}{1.815786in}}%
\pgfpathlineto{\pgfqpoint{4.493104in}{1.818881in}}%
\pgfpathlineto{\pgfqpoint{4.504175in}{1.822018in}}%
\pgfpathlineto{\pgfqpoint{4.515243in}{1.825206in}}%
\pgfpathlineto{\pgfqpoint{4.526307in}{1.828456in}}%
\pgfpathlineto{\pgfqpoint{4.537368in}{1.831766in}}%
\pgfpathlineto{\pgfqpoint{4.530832in}{1.846499in}}%
\pgfpathlineto{\pgfqpoint{4.524300in}{1.861275in}}%
\pgfpathlineto{\pgfqpoint{4.517772in}{1.876080in}}%
\pgfpathlineto{\pgfqpoint{4.511247in}{1.890898in}}%
\pgfpathlineto{\pgfqpoint{4.504724in}{1.905716in}}%
\pgfpathlineto{\pgfqpoint{4.493663in}{1.902400in}}%
\pgfpathlineto{\pgfqpoint{4.482598in}{1.899098in}}%
\pgfpathlineto{\pgfqpoint{4.471527in}{1.895809in}}%
\pgfpathlineto{\pgfqpoint{4.460451in}{1.892530in}}%
\pgfpathlineto{\pgfqpoint{4.449370in}{1.889256in}}%
\pgfpathlineto{\pgfqpoint{4.455897in}{1.874568in}}%
\pgfpathlineto{\pgfqpoint{4.462426in}{1.859866in}}%
\pgfpathlineto{\pgfqpoint{4.468957in}{1.845161in}}%
\pgfpathlineto{\pgfqpoint{4.475491in}{1.830465in}}%
\pgfpathclose%
\pgfusepath{stroke,fill}%
\end{pgfscope}%
\begin{pgfscope}%
\pgfpathrectangle{\pgfqpoint{0.887500in}{0.275000in}}{\pgfqpoint{4.225000in}{4.225000in}}%
\pgfusepath{clip}%
\pgfsetbuttcap%
\pgfsetroundjoin%
\definecolor{currentfill}{rgb}{0.123444,0.636809,0.528763}%
\pgfsetfillcolor{currentfill}%
\pgfsetfillopacity{0.700000}%
\pgfsetlinewidth{0.501875pt}%
\definecolor{currentstroke}{rgb}{1.000000,1.000000,1.000000}%
\pgfsetstrokecolor{currentstroke}%
\pgfsetstrokeopacity{0.500000}%
\pgfsetdash{}{0pt}%
\pgfpathmoveto{\pgfqpoint{3.298020in}{2.437550in}}%
\pgfpathlineto{\pgfqpoint{3.309473in}{2.453336in}}%
\pgfpathlineto{\pgfqpoint{3.320929in}{2.469189in}}%
\pgfpathlineto{\pgfqpoint{3.332384in}{2.484916in}}%
\pgfpathlineto{\pgfqpoint{3.343840in}{2.500323in}}%
\pgfpathlineto{\pgfqpoint{3.355292in}{2.515219in}}%
\pgfpathlineto{\pgfqpoint{3.348987in}{2.532882in}}%
\pgfpathlineto{\pgfqpoint{3.342687in}{2.550932in}}%
\pgfpathlineto{\pgfqpoint{3.336388in}{2.569170in}}%
\pgfpathlineto{\pgfqpoint{3.330091in}{2.587399in}}%
\pgfpathlineto{\pgfqpoint{3.323794in}{2.605422in}}%
\pgfpathlineto{\pgfqpoint{3.312366in}{2.593330in}}%
\pgfpathlineto{\pgfqpoint{3.300933in}{2.580450in}}%
\pgfpathlineto{\pgfqpoint{3.289494in}{2.566685in}}%
\pgfpathlineto{\pgfqpoint{3.278050in}{2.551941in}}%
\pgfpathlineto{\pgfqpoint{3.266602in}{2.536121in}}%
\pgfpathlineto{\pgfqpoint{3.272885in}{2.516580in}}%
\pgfpathlineto{\pgfqpoint{3.279167in}{2.496682in}}%
\pgfpathlineto{\pgfqpoint{3.285449in}{2.476698in}}%
\pgfpathlineto{\pgfqpoint{3.291733in}{2.456897in}}%
\pgfpathclose%
\pgfusepath{stroke,fill}%
\end{pgfscope}%
\begin{pgfscope}%
\pgfpathrectangle{\pgfqpoint{0.887500in}{0.275000in}}{\pgfqpoint{4.225000in}{4.225000in}}%
\pgfusepath{clip}%
\pgfsetbuttcap%
\pgfsetroundjoin%
\definecolor{currentfill}{rgb}{0.168126,0.459988,0.558082}%
\pgfsetfillcolor{currentfill}%
\pgfsetfillopacity{0.700000}%
\pgfsetlinewidth{0.501875pt}%
\definecolor{currentstroke}{rgb}{1.000000,1.000000,1.000000}%
\pgfsetstrokecolor{currentstroke}%
\pgfsetstrokeopacity{0.500000}%
\pgfsetdash{}{0pt}%
\pgfpathmoveto{\pgfqpoint{2.339551in}{2.114226in}}%
\pgfpathlineto{\pgfqpoint{2.351177in}{2.117862in}}%
\pgfpathlineto{\pgfqpoint{2.362798in}{2.121487in}}%
\pgfpathlineto{\pgfqpoint{2.374413in}{2.125099in}}%
\pgfpathlineto{\pgfqpoint{2.386023in}{2.128695in}}%
\pgfpathlineto{\pgfqpoint{2.397628in}{2.132274in}}%
\pgfpathlineto{\pgfqpoint{2.391566in}{2.141443in}}%
\pgfpathlineto{\pgfqpoint{2.385509in}{2.150580in}}%
\pgfpathlineto{\pgfqpoint{2.379456in}{2.159685in}}%
\pgfpathlineto{\pgfqpoint{2.373408in}{2.168760in}}%
\pgfpathlineto{\pgfqpoint{2.367364in}{2.177805in}}%
\pgfpathlineto{\pgfqpoint{2.355770in}{2.174276in}}%
\pgfpathlineto{\pgfqpoint{2.344170in}{2.170731in}}%
\pgfpathlineto{\pgfqpoint{2.332566in}{2.167170in}}%
\pgfpathlineto{\pgfqpoint{2.320955in}{2.163596in}}%
\pgfpathlineto{\pgfqpoint{2.309340in}{2.160009in}}%
\pgfpathlineto{\pgfqpoint{2.315373in}{2.150914in}}%
\pgfpathlineto{\pgfqpoint{2.321411in}{2.141789in}}%
\pgfpathlineto{\pgfqpoint{2.327453in}{2.132633in}}%
\pgfpathlineto{\pgfqpoint{2.333500in}{2.123445in}}%
\pgfpathclose%
\pgfusepath{stroke,fill}%
\end{pgfscope}%
\begin{pgfscope}%
\pgfpathrectangle{\pgfqpoint{0.887500in}{0.275000in}}{\pgfqpoint{4.225000in}{4.225000in}}%
\pgfusepath{clip}%
\pgfsetbuttcap%
\pgfsetroundjoin%
\definecolor{currentfill}{rgb}{0.154815,0.493313,0.557840}%
\pgfsetfillcolor{currentfill}%
\pgfsetfillopacity{0.700000}%
\pgfsetlinewidth{0.501875pt}%
\definecolor{currentstroke}{rgb}{1.000000,1.000000,1.000000}%
\pgfsetstrokecolor{currentstroke}%
\pgfsetstrokeopacity{0.500000}%
\pgfsetdash{}{0pt}%
\pgfpathmoveto{\pgfqpoint{2.016153in}{2.179425in}}%
\pgfpathlineto{\pgfqpoint{2.027861in}{2.182982in}}%
\pgfpathlineto{\pgfqpoint{2.039564in}{2.186527in}}%
\pgfpathlineto{\pgfqpoint{2.051261in}{2.190060in}}%
\pgfpathlineto{\pgfqpoint{2.062953in}{2.193583in}}%
\pgfpathlineto{\pgfqpoint{2.074639in}{2.197095in}}%
\pgfpathlineto{\pgfqpoint{2.068685in}{2.206030in}}%
\pgfpathlineto{\pgfqpoint{2.062737in}{2.214938in}}%
\pgfpathlineto{\pgfqpoint{2.056792in}{2.223820in}}%
\pgfpathlineto{\pgfqpoint{2.050852in}{2.232677in}}%
\pgfpathlineto{\pgfqpoint{2.044917in}{2.241508in}}%
\pgfpathlineto{\pgfqpoint{2.033241in}{2.238049in}}%
\pgfpathlineto{\pgfqpoint{2.021561in}{2.234579in}}%
\pgfpathlineto{\pgfqpoint{2.009874in}{2.231098in}}%
\pgfpathlineto{\pgfqpoint{1.998183in}{2.227605in}}%
\pgfpathlineto{\pgfqpoint{1.986485in}{2.224099in}}%
\pgfpathlineto{\pgfqpoint{1.992410in}{2.215216in}}%
\pgfpathlineto{\pgfqpoint{1.998339in}{2.206307in}}%
\pgfpathlineto{\pgfqpoint{2.004273in}{2.197373in}}%
\pgfpathlineto{\pgfqpoint{2.010211in}{2.188412in}}%
\pgfpathclose%
\pgfusepath{stroke,fill}%
\end{pgfscope}%
\begin{pgfscope}%
\pgfpathrectangle{\pgfqpoint{0.887500in}{0.275000in}}{\pgfqpoint{4.225000in}{4.225000in}}%
\pgfusepath{clip}%
\pgfsetbuttcap%
\pgfsetroundjoin%
\definecolor{currentfill}{rgb}{0.144759,0.519093,0.556572}%
\pgfsetfillcolor{currentfill}%
\pgfsetfillopacity{0.700000}%
\pgfsetlinewidth{0.501875pt}%
\definecolor{currentstroke}{rgb}{1.000000,1.000000,1.000000}%
\pgfsetstrokecolor{currentstroke}%
\pgfsetstrokeopacity{0.500000}%
\pgfsetdash{}{0pt}%
\pgfpathmoveto{\pgfqpoint{1.692791in}{2.242064in}}%
\pgfpathlineto{\pgfqpoint{1.704579in}{2.245599in}}%
\pgfpathlineto{\pgfqpoint{1.716362in}{2.249121in}}%
\pgfpathlineto{\pgfqpoint{1.728139in}{2.252631in}}%
\pgfpathlineto{\pgfqpoint{1.739911in}{2.256131in}}%
\pgfpathlineto{\pgfqpoint{1.751677in}{2.259621in}}%
\pgfpathlineto{\pgfqpoint{1.745835in}{2.268351in}}%
\pgfpathlineto{\pgfqpoint{1.739997in}{2.277056in}}%
\pgfpathlineto{\pgfqpoint{1.734164in}{2.285735in}}%
\pgfpathlineto{\pgfqpoint{1.728336in}{2.294389in}}%
\pgfpathlineto{\pgfqpoint{1.722512in}{2.303017in}}%
\pgfpathlineto{\pgfqpoint{1.710757in}{2.299571in}}%
\pgfpathlineto{\pgfqpoint{1.698997in}{2.296116in}}%
\pgfpathlineto{\pgfqpoint{1.687231in}{2.292650in}}%
\pgfpathlineto{\pgfqpoint{1.675460in}{2.289173in}}%
\pgfpathlineto{\pgfqpoint{1.663683in}{2.285684in}}%
\pgfpathlineto{\pgfqpoint{1.669495in}{2.277014in}}%
\pgfpathlineto{\pgfqpoint{1.675312in}{2.268317in}}%
\pgfpathlineto{\pgfqpoint{1.681134in}{2.259593in}}%
\pgfpathlineto{\pgfqpoint{1.686960in}{2.250842in}}%
\pgfpathclose%
\pgfusepath{stroke,fill}%
\end{pgfscope}%
\begin{pgfscope}%
\pgfpathrectangle{\pgfqpoint{0.887500in}{0.275000in}}{\pgfqpoint{4.225000in}{4.225000in}}%
\pgfusepath{clip}%
\pgfsetbuttcap%
\pgfsetroundjoin%
\definecolor{currentfill}{rgb}{0.126453,0.570633,0.549841}%
\pgfsetfillcolor{currentfill}%
\pgfsetfillopacity{0.700000}%
\pgfsetlinewidth{0.501875pt}%
\definecolor{currentstroke}{rgb}{1.000000,1.000000,1.000000}%
\pgfsetstrokecolor{currentstroke}%
\pgfsetstrokeopacity{0.500000}%
\pgfsetdash{}{0pt}%
\pgfpathmoveto{\pgfqpoint{3.798803in}{2.335702in}}%
\pgfpathlineto{\pgfqpoint{3.810086in}{2.340188in}}%
\pgfpathlineto{\pgfqpoint{3.821363in}{2.344585in}}%
\pgfpathlineto{\pgfqpoint{3.832632in}{2.348874in}}%
\pgfpathlineto{\pgfqpoint{3.843894in}{2.353040in}}%
\pgfpathlineto{\pgfqpoint{3.855149in}{2.357070in}}%
\pgfpathlineto{\pgfqpoint{3.848737in}{2.371903in}}%
\pgfpathlineto{\pgfqpoint{3.842326in}{2.386622in}}%
\pgfpathlineto{\pgfqpoint{3.835914in}{2.401202in}}%
\pgfpathlineto{\pgfqpoint{3.829502in}{2.415641in}}%
\pgfpathlineto{\pgfqpoint{3.823090in}{2.429955in}}%
\pgfpathlineto{\pgfqpoint{3.811843in}{2.426170in}}%
\pgfpathlineto{\pgfqpoint{3.800588in}{2.422265in}}%
\pgfpathlineto{\pgfqpoint{3.789326in}{2.418234in}}%
\pgfpathlineto{\pgfqpoint{3.778056in}{2.414069in}}%
\pgfpathlineto{\pgfqpoint{3.766779in}{2.409763in}}%
\pgfpathlineto{\pgfqpoint{3.773184in}{2.395268in}}%
\pgfpathlineto{\pgfqpoint{3.779590in}{2.380623in}}%
\pgfpathlineto{\pgfqpoint{3.785995in}{2.365809in}}%
\pgfpathlineto{\pgfqpoint{3.792399in}{2.350826in}}%
\pgfpathclose%
\pgfusepath{stroke,fill}%
\end{pgfscope}%
\begin{pgfscope}%
\pgfpathrectangle{\pgfqpoint{0.887500in}{0.275000in}}{\pgfqpoint{4.225000in}{4.225000in}}%
\pgfusepath{clip}%
\pgfsetbuttcap%
\pgfsetroundjoin%
\definecolor{currentfill}{rgb}{0.129933,0.559582,0.551864}%
\pgfsetfillcolor{currentfill}%
\pgfsetfillopacity{0.700000}%
\pgfsetlinewidth{0.501875pt}%
\definecolor{currentstroke}{rgb}{1.000000,1.000000,1.000000}%
\pgfsetstrokecolor{currentstroke}%
\pgfsetstrokeopacity{0.500000}%
\pgfsetdash{}{0pt}%
\pgfpathmoveto{\pgfqpoint{3.272300in}{2.281158in}}%
\pgfpathlineto{\pgfqpoint{3.283747in}{2.295245in}}%
\pgfpathlineto{\pgfqpoint{3.295195in}{2.309427in}}%
\pgfpathlineto{\pgfqpoint{3.306646in}{2.323892in}}%
\pgfpathlineto{\pgfqpoint{3.318101in}{2.338831in}}%
\pgfpathlineto{\pgfqpoint{3.329561in}{2.354433in}}%
\pgfpathlineto{\pgfqpoint{3.323237in}{2.369200in}}%
\pgfpathlineto{\pgfqpoint{3.316920in}{2.384766in}}%
\pgfpathlineto{\pgfqpoint{3.310612in}{2.401295in}}%
\pgfpathlineto{\pgfqpoint{3.304312in}{2.418927in}}%
\pgfpathlineto{\pgfqpoint{3.298020in}{2.437550in}}%
\pgfpathlineto{\pgfqpoint{3.286569in}{2.421997in}}%
\pgfpathlineto{\pgfqpoint{3.275121in}{2.406655in}}%
\pgfpathlineto{\pgfqpoint{3.263675in}{2.391413in}}%
\pgfpathlineto{\pgfqpoint{3.252230in}{2.376161in}}%
\pgfpathlineto{\pgfqpoint{3.240785in}{2.360789in}}%
\pgfpathlineto{\pgfqpoint{3.247082in}{2.344394in}}%
\pgfpathlineto{\pgfqpoint{3.253382in}{2.328240in}}%
\pgfpathlineto{\pgfqpoint{3.259685in}{2.312364in}}%
\pgfpathlineto{\pgfqpoint{3.265991in}{2.296696in}}%
\pgfpathclose%
\pgfusepath{stroke,fill}%
\end{pgfscope}%
\begin{pgfscope}%
\pgfpathrectangle{\pgfqpoint{0.887500in}{0.275000in}}{\pgfqpoint{4.225000in}{4.225000in}}%
\pgfusepath{clip}%
\pgfsetbuttcap%
\pgfsetroundjoin%
\definecolor{currentfill}{rgb}{0.166617,0.463708,0.558119}%
\pgfsetfillcolor{currentfill}%
\pgfsetfillopacity{0.700000}%
\pgfsetlinewidth{0.501875pt}%
\definecolor{currentstroke}{rgb}{1.000000,1.000000,1.000000}%
\pgfsetstrokecolor{currentstroke}%
\pgfsetstrokeopacity{0.500000}%
\pgfsetdash{}{0pt}%
\pgfpathmoveto{\pgfqpoint{4.096249in}{2.102367in}}%
\pgfpathlineto{\pgfqpoint{4.107442in}{2.105978in}}%
\pgfpathlineto{\pgfqpoint{4.118629in}{2.109550in}}%
\pgfpathlineto{\pgfqpoint{4.129811in}{2.113107in}}%
\pgfpathlineto{\pgfqpoint{4.140987in}{2.116667in}}%
\pgfpathlineto{\pgfqpoint{4.152157in}{2.120227in}}%
\pgfpathlineto{\pgfqpoint{4.145704in}{2.135457in}}%
\pgfpathlineto{\pgfqpoint{4.139248in}{2.150521in}}%
\pgfpathlineto{\pgfqpoint{4.132790in}{2.165409in}}%
\pgfpathlineto{\pgfqpoint{4.126330in}{2.180144in}}%
\pgfpathlineto{\pgfqpoint{4.119869in}{2.194746in}}%
\pgfpathlineto{\pgfqpoint{4.108707in}{2.191453in}}%
\pgfpathlineto{\pgfqpoint{4.097539in}{2.188163in}}%
\pgfpathlineto{\pgfqpoint{4.086366in}{2.184879in}}%
\pgfpathlineto{\pgfqpoint{4.075187in}{2.181577in}}%
\pgfpathlineto{\pgfqpoint{4.064001in}{2.178228in}}%
\pgfpathlineto{\pgfqpoint{4.070459in}{2.163570in}}%
\pgfpathlineto{\pgfqpoint{4.076914in}{2.148688in}}%
\pgfpathlineto{\pgfqpoint{4.083364in}{2.133543in}}%
\pgfpathlineto{\pgfqpoint{4.089809in}{2.118097in}}%
\pgfpathclose%
\pgfusepath{stroke,fill}%
\end{pgfscope}%
\begin{pgfscope}%
\pgfpathrectangle{\pgfqpoint{0.887500in}{0.275000in}}{\pgfqpoint{4.225000in}{4.225000in}}%
\pgfusepath{clip}%
\pgfsetbuttcap%
\pgfsetroundjoin%
\definecolor{currentfill}{rgb}{0.187231,0.414746,0.556547}%
\pgfsetfillcolor{currentfill}%
\pgfsetfillopacity{0.700000}%
\pgfsetlinewidth{0.501875pt}%
\definecolor{currentstroke}{rgb}{1.000000,1.000000,1.000000}%
\pgfsetstrokecolor{currentstroke}%
\pgfsetstrokeopacity{0.500000}%
\pgfsetdash{}{0pt}%
\pgfpathmoveto{\pgfqpoint{2.751501in}{2.014764in}}%
\pgfpathlineto{\pgfqpoint{2.763029in}{2.018396in}}%
\pgfpathlineto{\pgfqpoint{2.774551in}{2.021949in}}%
\pgfpathlineto{\pgfqpoint{2.786069in}{2.025402in}}%
\pgfpathlineto{\pgfqpoint{2.797581in}{2.028733in}}%
\pgfpathlineto{\pgfqpoint{2.809088in}{2.031959in}}%
\pgfpathlineto{\pgfqpoint{2.802891in}{2.041667in}}%
\pgfpathlineto{\pgfqpoint{2.796698in}{2.051333in}}%
\pgfpathlineto{\pgfqpoint{2.790509in}{2.060955in}}%
\pgfpathlineto{\pgfqpoint{2.784324in}{2.070534in}}%
\pgfpathlineto{\pgfqpoint{2.778143in}{2.080070in}}%
\pgfpathlineto{\pgfqpoint{2.766646in}{2.076909in}}%
\pgfpathlineto{\pgfqpoint{2.755143in}{2.073638in}}%
\pgfpathlineto{\pgfqpoint{2.743635in}{2.070239in}}%
\pgfpathlineto{\pgfqpoint{2.732123in}{2.066735in}}%
\pgfpathlineto{\pgfqpoint{2.720605in}{2.063149in}}%
\pgfpathlineto{\pgfqpoint{2.726776in}{2.053560in}}%
\pgfpathlineto{\pgfqpoint{2.732951in}{2.043927in}}%
\pgfpathlineto{\pgfqpoint{2.739131in}{2.034250in}}%
\pgfpathlineto{\pgfqpoint{2.745314in}{2.024529in}}%
\pgfpathclose%
\pgfusepath{stroke,fill}%
\end{pgfscope}%
\begin{pgfscope}%
\pgfpathrectangle{\pgfqpoint{0.887500in}{0.275000in}}{\pgfqpoint{4.225000in}{4.225000in}}%
\pgfusepath{clip}%
\pgfsetbuttcap%
\pgfsetroundjoin%
\definecolor{currentfill}{rgb}{0.172719,0.448791,0.557885}%
\pgfsetfillcolor{currentfill}%
\pgfsetfillopacity{0.700000}%
\pgfsetlinewidth{0.501875pt}%
\definecolor{currentstroke}{rgb}{1.000000,1.000000,1.000000}%
\pgfsetstrokecolor{currentstroke}%
\pgfsetstrokeopacity{0.500000}%
\pgfsetdash{}{0pt}%
\pgfpathmoveto{\pgfqpoint{3.189246in}{2.037235in}}%
\pgfpathlineto{\pgfqpoint{3.200704in}{2.054935in}}%
\pgfpathlineto{\pgfqpoint{3.212163in}{2.072203in}}%
\pgfpathlineto{\pgfqpoint{3.223622in}{2.089017in}}%
\pgfpathlineto{\pgfqpoint{3.235081in}{2.105484in}}%
\pgfpathlineto{\pgfqpoint{3.246542in}{2.121707in}}%
\pgfpathlineto{\pgfqpoint{3.240245in}{2.139477in}}%
\pgfpathlineto{\pgfqpoint{3.233947in}{2.156773in}}%
\pgfpathlineto{\pgfqpoint{3.227649in}{2.173594in}}%
\pgfpathlineto{\pgfqpoint{3.221350in}{2.189940in}}%
\pgfpathlineto{\pgfqpoint{3.215051in}{2.205810in}}%
\pgfpathlineto{\pgfqpoint{3.203597in}{2.188728in}}%
\pgfpathlineto{\pgfqpoint{3.192142in}{2.170687in}}%
\pgfpathlineto{\pgfqpoint{3.180686in}{2.151570in}}%
\pgfpathlineto{\pgfqpoint{3.169231in}{2.131256in}}%
\pgfpathlineto{\pgfqpoint{3.157776in}{2.109812in}}%
\pgfpathlineto{\pgfqpoint{3.164068in}{2.096068in}}%
\pgfpathlineto{\pgfqpoint{3.170361in}{2.081921in}}%
\pgfpathlineto{\pgfqpoint{3.176655in}{2.067389in}}%
\pgfpathlineto{\pgfqpoint{3.182950in}{2.052488in}}%
\pgfpathclose%
\pgfusepath{stroke,fill}%
\end{pgfscope}%
\begin{pgfscope}%
\pgfpathrectangle{\pgfqpoint{0.887500in}{0.275000in}}{\pgfqpoint{4.225000in}{4.225000in}}%
\pgfusepath{clip}%
\pgfsetbuttcap%
\pgfsetroundjoin%
\definecolor{currentfill}{rgb}{0.216210,0.351535,0.550627}%
\pgfsetfillcolor{currentfill}%
\pgfsetfillopacity{0.700000}%
\pgfsetlinewidth{0.501875pt}%
\definecolor{currentstroke}{rgb}{1.000000,1.000000,1.000000}%
\pgfsetstrokecolor{currentstroke}%
\pgfsetstrokeopacity{0.500000}%
\pgfsetdash{}{0pt}%
\pgfpathmoveto{\pgfqpoint{4.393877in}{1.872779in}}%
\pgfpathlineto{\pgfqpoint{4.404988in}{1.876106in}}%
\pgfpathlineto{\pgfqpoint{4.416092in}{1.879411in}}%
\pgfpathlineto{\pgfqpoint{4.427191in}{1.882701in}}%
\pgfpathlineto{\pgfqpoint{4.438283in}{1.885981in}}%
\pgfpathlineto{\pgfqpoint{4.449370in}{1.889256in}}%
\pgfpathlineto{\pgfqpoint{4.442845in}{1.903920in}}%
\pgfpathlineto{\pgfqpoint{4.436322in}{1.918550in}}%
\pgfpathlineto{\pgfqpoint{4.429799in}{1.933134in}}%
\pgfpathlineto{\pgfqpoint{4.423277in}{1.947663in}}%
\pgfpathlineto{\pgfqpoint{4.416755in}{1.962125in}}%
\pgfpathlineto{\pgfqpoint{4.405659in}{1.958542in}}%
\pgfpathlineto{\pgfqpoint{4.394555in}{1.954908in}}%
\pgfpathlineto{\pgfqpoint{4.383445in}{1.951222in}}%
\pgfpathlineto{\pgfqpoint{4.372327in}{1.947485in}}%
\pgfpathlineto{\pgfqpoint{4.361202in}{1.943693in}}%
\pgfpathlineto{\pgfqpoint{4.367734in}{1.929587in}}%
\pgfpathlineto{\pgfqpoint{4.374268in}{1.915447in}}%
\pgfpathlineto{\pgfqpoint{4.380803in}{1.901269in}}%
\pgfpathlineto{\pgfqpoint{4.387339in}{1.887048in}}%
\pgfpathclose%
\pgfusepath{stroke,fill}%
\end{pgfscope}%
\begin{pgfscope}%
\pgfpathrectangle{\pgfqpoint{0.887500in}{0.275000in}}{\pgfqpoint{4.225000in}{4.225000in}}%
\pgfusepath{clip}%
\pgfsetbuttcap%
\pgfsetroundjoin%
\definecolor{currentfill}{rgb}{0.120565,0.596422,0.543611}%
\pgfsetfillcolor{currentfill}%
\pgfsetfillopacity{0.700000}%
\pgfsetlinewidth{0.501875pt}%
\definecolor{currentstroke}{rgb}{1.000000,1.000000,1.000000}%
\pgfsetstrokecolor{currentstroke}%
\pgfsetstrokeopacity{0.500000}%
\pgfsetdash{}{0pt}%
\pgfpathmoveto{\pgfqpoint{3.710275in}{2.385813in}}%
\pgfpathlineto{\pgfqpoint{3.721591in}{2.390979in}}%
\pgfpathlineto{\pgfqpoint{3.732900in}{2.395929in}}%
\pgfpathlineto{\pgfqpoint{3.744200in}{2.400700in}}%
\pgfpathlineto{\pgfqpoint{3.755493in}{2.405309in}}%
\pgfpathlineto{\pgfqpoint{3.766779in}{2.409763in}}%
\pgfpathlineto{\pgfqpoint{3.760373in}{2.424130in}}%
\pgfpathlineto{\pgfqpoint{3.753968in}{2.438390in}}%
\pgfpathlineto{\pgfqpoint{3.747564in}{2.452563in}}%
\pgfpathlineto{\pgfqpoint{3.741161in}{2.466671in}}%
\pgfpathlineto{\pgfqpoint{3.734760in}{2.480734in}}%
\pgfpathlineto{\pgfqpoint{3.723482in}{2.476556in}}%
\pgfpathlineto{\pgfqpoint{3.712199in}{2.472285in}}%
\pgfpathlineto{\pgfqpoint{3.700908in}{2.467920in}}%
\pgfpathlineto{\pgfqpoint{3.689611in}{2.463451in}}%
\pgfpathlineto{\pgfqpoint{3.678307in}{2.458853in}}%
\pgfpathlineto{\pgfqpoint{3.684696in}{2.444292in}}%
\pgfpathlineto{\pgfqpoint{3.691088in}{2.429706in}}%
\pgfpathlineto{\pgfqpoint{3.697481in}{2.415097in}}%
\pgfpathlineto{\pgfqpoint{3.703877in}{2.400466in}}%
\pgfpathclose%
\pgfusepath{stroke,fill}%
\end{pgfscope}%
\begin{pgfscope}%
\pgfpathrectangle{\pgfqpoint{0.887500in}{0.275000in}}{\pgfqpoint{4.225000in}{4.225000in}}%
\pgfusepath{clip}%
\pgfsetbuttcap%
\pgfsetroundjoin%
\definecolor{currentfill}{rgb}{0.132268,0.655014,0.519661}%
\pgfsetfillcolor{currentfill}%
\pgfsetfillopacity{0.700000}%
\pgfsetlinewidth{0.501875pt}%
\definecolor{currentstroke}{rgb}{1.000000,1.000000,1.000000}%
\pgfsetstrokecolor{currentstroke}%
\pgfsetstrokeopacity{0.500000}%
\pgfsetdash{}{0pt}%
\pgfpathmoveto{\pgfqpoint{3.444175in}{2.504011in}}%
\pgfpathlineto{\pgfqpoint{3.455594in}{2.514259in}}%
\pgfpathlineto{\pgfqpoint{3.467004in}{2.523731in}}%
\pgfpathlineto{\pgfqpoint{3.478405in}{2.532541in}}%
\pgfpathlineto{\pgfqpoint{3.489797in}{2.540802in}}%
\pgfpathlineto{\pgfqpoint{3.501181in}{2.548577in}}%
\pgfpathlineto{\pgfqpoint{3.494818in}{2.562534in}}%
\pgfpathlineto{\pgfqpoint{3.488460in}{2.576586in}}%
\pgfpathlineto{\pgfqpoint{3.482105in}{2.590804in}}%
\pgfpathlineto{\pgfqpoint{3.475756in}{2.605252in}}%
\pgfpathlineto{\pgfqpoint{3.469411in}{2.619920in}}%
\pgfpathlineto{\pgfqpoint{3.458036in}{2.612386in}}%
\pgfpathlineto{\pgfqpoint{3.446653in}{2.604382in}}%
\pgfpathlineto{\pgfqpoint{3.435262in}{2.595861in}}%
\pgfpathlineto{\pgfqpoint{3.423862in}{2.586745in}}%
\pgfpathlineto{\pgfqpoint{3.412455in}{2.576951in}}%
\pgfpathlineto{\pgfqpoint{3.418787in}{2.561559in}}%
\pgfpathlineto{\pgfqpoint{3.425125in}{2.546597in}}%
\pgfpathlineto{\pgfqpoint{3.431469in}{2.532089in}}%
\pgfpathlineto{\pgfqpoint{3.437820in}{2.517931in}}%
\pgfpathclose%
\pgfusepath{stroke,fill}%
\end{pgfscope}%
\begin{pgfscope}%
\pgfpathrectangle{\pgfqpoint{0.887500in}{0.275000in}}{\pgfqpoint{4.225000in}{4.225000in}}%
\pgfusepath{clip}%
\pgfsetbuttcap%
\pgfsetroundjoin%
\definecolor{currentfill}{rgb}{0.172719,0.448791,0.557885}%
\pgfsetfillcolor{currentfill}%
\pgfsetfillopacity{0.700000}%
\pgfsetlinewidth{0.501875pt}%
\definecolor{currentstroke}{rgb}{1.000000,1.000000,1.000000}%
\pgfsetstrokecolor{currentstroke}%
\pgfsetstrokeopacity{0.500000}%
\pgfsetdash{}{0pt}%
\pgfpathmoveto{\pgfqpoint{2.428000in}{2.085928in}}%
\pgfpathlineto{\pgfqpoint{2.439610in}{2.089539in}}%
\pgfpathlineto{\pgfqpoint{2.451214in}{2.093128in}}%
\pgfpathlineto{\pgfqpoint{2.462813in}{2.096695in}}%
\pgfpathlineto{\pgfqpoint{2.474406in}{2.100238in}}%
\pgfpathlineto{\pgfqpoint{2.485994in}{2.103760in}}%
\pgfpathlineto{\pgfqpoint{2.479901in}{2.113049in}}%
\pgfpathlineto{\pgfqpoint{2.473811in}{2.122302in}}%
\pgfpathlineto{\pgfqpoint{2.467726in}{2.131521in}}%
\pgfpathlineto{\pgfqpoint{2.461646in}{2.140706in}}%
\pgfpathlineto{\pgfqpoint{2.455570in}{2.149857in}}%
\pgfpathlineto{\pgfqpoint{2.443992in}{2.146384in}}%
\pgfpathlineto{\pgfqpoint{2.432409in}{2.142890in}}%
\pgfpathlineto{\pgfqpoint{2.420821in}{2.139374in}}%
\pgfpathlineto{\pgfqpoint{2.409227in}{2.135834in}}%
\pgfpathlineto{\pgfqpoint{2.397628in}{2.132274in}}%
\pgfpathlineto{\pgfqpoint{2.403694in}{2.123073in}}%
\pgfpathlineto{\pgfqpoint{2.409764in}{2.113839in}}%
\pgfpathlineto{\pgfqpoint{2.415838in}{2.104570in}}%
\pgfpathlineto{\pgfqpoint{2.421917in}{2.095267in}}%
\pgfpathclose%
\pgfusepath{stroke,fill}%
\end{pgfscope}%
\begin{pgfscope}%
\pgfpathrectangle{\pgfqpoint{0.887500in}{0.275000in}}{\pgfqpoint{4.225000in}{4.225000in}}%
\pgfusepath{clip}%
\pgfsetbuttcap%
\pgfsetroundjoin%
\definecolor{currentfill}{rgb}{0.120092,0.600104,0.542530}%
\pgfsetfillcolor{currentfill}%
\pgfsetfillopacity{0.700000}%
\pgfsetlinewidth{0.501875pt}%
\definecolor{currentstroke}{rgb}{1.000000,1.000000,1.000000}%
\pgfsetstrokecolor{currentstroke}%
\pgfsetstrokeopacity{0.500000}%
\pgfsetdash{}{0pt}%
\pgfpathmoveto{\pgfqpoint{3.329561in}{2.354433in}}%
\pgfpathlineto{\pgfqpoint{3.341028in}{2.370747in}}%
\pgfpathlineto{\pgfqpoint{3.352499in}{2.387511in}}%
\pgfpathlineto{\pgfqpoint{3.363974in}{2.404422in}}%
\pgfpathlineto{\pgfqpoint{3.375450in}{2.421178in}}%
\pgfpathlineto{\pgfqpoint{3.386924in}{2.437475in}}%
\pgfpathlineto{\pgfqpoint{3.380582in}{2.451651in}}%
\pgfpathlineto{\pgfqpoint{3.374248in}{2.466388in}}%
\pgfpathlineto{\pgfqpoint{3.367921in}{2.481840in}}%
\pgfpathlineto{\pgfqpoint{3.361603in}{2.498140in}}%
\pgfpathlineto{\pgfqpoint{3.355292in}{2.515219in}}%
\pgfpathlineto{\pgfqpoint{3.343840in}{2.500323in}}%
\pgfpathlineto{\pgfqpoint{3.332384in}{2.484916in}}%
\pgfpathlineto{\pgfqpoint{3.320929in}{2.469189in}}%
\pgfpathlineto{\pgfqpoint{3.309473in}{2.453336in}}%
\pgfpathlineto{\pgfqpoint{3.298020in}{2.437550in}}%
\pgfpathlineto{\pgfqpoint{3.304312in}{2.418927in}}%
\pgfpathlineto{\pgfqpoint{3.310612in}{2.401295in}}%
\pgfpathlineto{\pgfqpoint{3.316920in}{2.384766in}}%
\pgfpathlineto{\pgfqpoint{3.323237in}{2.369200in}}%
\pgfpathclose%
\pgfusepath{stroke,fill}%
\end{pgfscope}%
\begin{pgfscope}%
\pgfpathrectangle{\pgfqpoint{0.887500in}{0.275000in}}{\pgfqpoint{4.225000in}{4.225000in}}%
\pgfusepath{clip}%
\pgfsetbuttcap%
\pgfsetroundjoin%
\definecolor{currentfill}{rgb}{0.192357,0.403199,0.555836}%
\pgfsetfillcolor{currentfill}%
\pgfsetfillopacity{0.700000}%
\pgfsetlinewidth{0.501875pt}%
\definecolor{currentstroke}{rgb}{1.000000,1.000000,1.000000}%
\pgfsetstrokecolor{currentstroke}%
\pgfsetstrokeopacity{0.500000}%
\pgfsetdash{}{0pt}%
\pgfpathmoveto{\pgfqpoint{2.840134in}{1.982757in}}%
\pgfpathlineto{\pgfqpoint{2.851645in}{1.986017in}}%
\pgfpathlineto{\pgfqpoint{2.863150in}{1.989313in}}%
\pgfpathlineto{\pgfqpoint{2.874648in}{1.992718in}}%
\pgfpathlineto{\pgfqpoint{2.886140in}{1.996301in}}%
\pgfpathlineto{\pgfqpoint{2.897625in}{2.000134in}}%
\pgfpathlineto{\pgfqpoint{2.891399in}{2.010006in}}%
\pgfpathlineto{\pgfqpoint{2.885176in}{2.019834in}}%
\pgfpathlineto{\pgfqpoint{2.878957in}{2.029618in}}%
\pgfpathlineto{\pgfqpoint{2.872742in}{2.039358in}}%
\pgfpathlineto{\pgfqpoint{2.866532in}{2.049056in}}%
\pgfpathlineto{\pgfqpoint{2.855056in}{2.045259in}}%
\pgfpathlineto{\pgfqpoint{2.843574in}{2.041726in}}%
\pgfpathlineto{\pgfqpoint{2.832085in}{2.038381in}}%
\pgfpathlineto{\pgfqpoint{2.820590in}{2.035150in}}%
\pgfpathlineto{\pgfqpoint{2.809088in}{2.031959in}}%
\pgfpathlineto{\pgfqpoint{2.815289in}{2.022206in}}%
\pgfpathlineto{\pgfqpoint{2.821495in}{2.012410in}}%
\pgfpathlineto{\pgfqpoint{2.827704in}{2.002570in}}%
\pgfpathlineto{\pgfqpoint{2.833917in}{1.992685in}}%
\pgfpathclose%
\pgfusepath{stroke,fill}%
\end{pgfscope}%
\begin{pgfscope}%
\pgfpathrectangle{\pgfqpoint{0.887500in}{0.275000in}}{\pgfqpoint{4.225000in}{4.225000in}}%
\pgfusepath{clip}%
\pgfsetbuttcap%
\pgfsetroundjoin%
\definecolor{currentfill}{rgb}{0.120081,0.622161,0.534946}%
\pgfsetfillcolor{currentfill}%
\pgfsetfillopacity{0.700000}%
\pgfsetlinewidth{0.501875pt}%
\definecolor{currentstroke}{rgb}{1.000000,1.000000,1.000000}%
\pgfsetstrokecolor{currentstroke}%
\pgfsetstrokeopacity{0.500000}%
\pgfsetdash{}{0pt}%
\pgfpathmoveto{\pgfqpoint{3.621670in}{2.432957in}}%
\pgfpathlineto{\pgfqpoint{3.633014in}{2.438616in}}%
\pgfpathlineto{\pgfqpoint{3.644349in}{2.444007in}}%
\pgfpathlineto{\pgfqpoint{3.655676in}{2.449158in}}%
\pgfpathlineto{\pgfqpoint{3.666995in}{2.454098in}}%
\pgfpathlineto{\pgfqpoint{3.678307in}{2.458853in}}%
\pgfpathlineto{\pgfqpoint{3.671919in}{2.473388in}}%
\pgfpathlineto{\pgfqpoint{3.665533in}{2.487895in}}%
\pgfpathlineto{\pgfqpoint{3.659150in}{2.502373in}}%
\pgfpathlineto{\pgfqpoint{3.652768in}{2.516820in}}%
\pgfpathlineto{\pgfqpoint{3.646388in}{2.531233in}}%
\pgfpathlineto{\pgfqpoint{3.635089in}{2.527108in}}%
\pgfpathlineto{\pgfqpoint{3.623783in}{2.522831in}}%
\pgfpathlineto{\pgfqpoint{3.612468in}{2.518354in}}%
\pgfpathlineto{\pgfqpoint{3.601146in}{2.513629in}}%
\pgfpathlineto{\pgfqpoint{3.589814in}{2.508608in}}%
\pgfpathlineto{\pgfqpoint{3.596186in}{2.493830in}}%
\pgfpathlineto{\pgfqpoint{3.602557in}{2.478872in}}%
\pgfpathlineto{\pgfqpoint{3.608928in}{2.463720in}}%
\pgfpathlineto{\pgfqpoint{3.615299in}{2.448400in}}%
\pgfpathclose%
\pgfusepath{stroke,fill}%
\end{pgfscope}%
\begin{pgfscope}%
\pgfpathrectangle{\pgfqpoint{0.887500in}{0.275000in}}{\pgfqpoint{4.225000in}{4.225000in}}%
\pgfusepath{clip}%
\pgfsetbuttcap%
\pgfsetroundjoin%
\definecolor{currentfill}{rgb}{0.159194,0.482237,0.558073}%
\pgfsetfillcolor{currentfill}%
\pgfsetfillopacity{0.700000}%
\pgfsetlinewidth{0.501875pt}%
\definecolor{currentstroke}{rgb}{1.000000,1.000000,1.000000}%
\pgfsetstrokecolor{currentstroke}%
\pgfsetstrokeopacity{0.500000}%
\pgfsetdash{}{0pt}%
\pgfpathmoveto{\pgfqpoint{2.104471in}{2.152011in}}%
\pgfpathlineto{\pgfqpoint{2.116162in}{2.155561in}}%
\pgfpathlineto{\pgfqpoint{2.127848in}{2.159102in}}%
\pgfpathlineto{\pgfqpoint{2.139528in}{2.162633in}}%
\pgfpathlineto{\pgfqpoint{2.151202in}{2.166157in}}%
\pgfpathlineto{\pgfqpoint{2.162871in}{2.169675in}}%
\pgfpathlineto{\pgfqpoint{2.156885in}{2.178699in}}%
\pgfpathlineto{\pgfqpoint{2.150903in}{2.187694in}}%
\pgfpathlineto{\pgfqpoint{2.144926in}{2.196661in}}%
\pgfpathlineto{\pgfqpoint{2.138953in}{2.205600in}}%
\pgfpathlineto{\pgfqpoint{2.132985in}{2.214512in}}%
\pgfpathlineto{\pgfqpoint{2.121327in}{2.211043in}}%
\pgfpathlineto{\pgfqpoint{2.109663in}{2.207569in}}%
\pgfpathlineto{\pgfqpoint{2.097994in}{2.204087in}}%
\pgfpathlineto{\pgfqpoint{2.086319in}{2.200596in}}%
\pgfpathlineto{\pgfqpoint{2.074639in}{2.197095in}}%
\pgfpathlineto{\pgfqpoint{2.080596in}{2.188133in}}%
\pgfpathlineto{\pgfqpoint{2.086558in}{2.179144in}}%
\pgfpathlineto{\pgfqpoint{2.092525in}{2.170128in}}%
\pgfpathlineto{\pgfqpoint{2.098495in}{2.161083in}}%
\pgfpathclose%
\pgfusepath{stroke,fill}%
\end{pgfscope}%
\begin{pgfscope}%
\pgfpathrectangle{\pgfqpoint{0.887500in}{0.275000in}}{\pgfqpoint{4.225000in}{4.225000in}}%
\pgfusepath{clip}%
\pgfsetbuttcap%
\pgfsetroundjoin%
\definecolor{currentfill}{rgb}{0.156270,0.489624,0.557936}%
\pgfsetfillcolor{currentfill}%
\pgfsetfillopacity{0.700000}%
\pgfsetlinewidth{0.501875pt}%
\definecolor{currentstroke}{rgb}{1.000000,1.000000,1.000000}%
\pgfsetstrokecolor{currentstroke}%
\pgfsetstrokeopacity{0.500000}%
\pgfsetdash{}{0pt}%
\pgfpathmoveto{\pgfqpoint{4.007944in}{2.159681in}}%
\pgfpathlineto{\pgfqpoint{4.019175in}{2.163730in}}%
\pgfpathlineto{\pgfqpoint{4.030395in}{2.167582in}}%
\pgfpathlineto{\pgfqpoint{4.041606in}{2.171261in}}%
\pgfpathlineto{\pgfqpoint{4.052808in}{2.174799in}}%
\pgfpathlineto{\pgfqpoint{4.064001in}{2.178228in}}%
\pgfpathlineto{\pgfqpoint{4.057541in}{2.192700in}}%
\pgfpathlineto{\pgfqpoint{4.051079in}{2.207026in}}%
\pgfpathlineto{\pgfqpoint{4.044617in}{2.221244in}}%
\pgfpathlineto{\pgfqpoint{4.038157in}{2.235393in}}%
\pgfpathlineto{\pgfqpoint{4.031698in}{2.249513in}}%
\pgfpathlineto{\pgfqpoint{4.020505in}{2.246029in}}%
\pgfpathlineto{\pgfqpoint{4.009305in}{2.242463in}}%
\pgfpathlineto{\pgfqpoint{3.998096in}{2.238792in}}%
\pgfpathlineto{\pgfqpoint{3.986879in}{2.234989in}}%
\pgfpathlineto{\pgfqpoint{3.975654in}{2.231032in}}%
\pgfpathlineto{\pgfqpoint{3.982110in}{2.216911in}}%
\pgfpathlineto{\pgfqpoint{3.988569in}{2.202765in}}%
\pgfpathlineto{\pgfqpoint{3.995028in}{2.188544in}}%
\pgfpathlineto{\pgfqpoint{4.001487in}{2.174200in}}%
\pgfpathclose%
\pgfusepath{stroke,fill}%
\end{pgfscope}%
\begin{pgfscope}%
\pgfpathrectangle{\pgfqpoint{0.887500in}{0.275000in}}{\pgfqpoint{4.225000in}{4.225000in}}%
\pgfusepath{clip}%
\pgfsetbuttcap%
\pgfsetroundjoin%
\definecolor{currentfill}{rgb}{0.203063,0.379716,0.553925}%
\pgfsetfillcolor{currentfill}%
\pgfsetfillopacity{0.700000}%
\pgfsetlinewidth{0.501875pt}%
\definecolor{currentstroke}{rgb}{1.000000,1.000000,1.000000}%
\pgfsetstrokecolor{currentstroke}%
\pgfsetstrokeopacity{0.500000}%
\pgfsetdash{}{0pt}%
\pgfpathmoveto{\pgfqpoint{4.305466in}{1.923743in}}%
\pgfpathlineto{\pgfqpoint{4.316628in}{1.927881in}}%
\pgfpathlineto{\pgfqpoint{4.327783in}{1.931940in}}%
\pgfpathlineto{\pgfqpoint{4.338930in}{1.935925in}}%
\pgfpathlineto{\pgfqpoint{4.350070in}{1.939841in}}%
\pgfpathlineto{\pgfqpoint{4.361202in}{1.943693in}}%
\pgfpathlineto{\pgfqpoint{4.354672in}{1.957769in}}%
\pgfpathlineto{\pgfqpoint{4.348143in}{1.971820in}}%
\pgfpathlineto{\pgfqpoint{4.341617in}{1.985851in}}%
\pgfpathlineto{\pgfqpoint{4.335093in}{1.999867in}}%
\pgfpathlineto{\pgfqpoint{4.328571in}{2.013871in}}%
\pgfpathlineto{\pgfqpoint{4.317433in}{2.009770in}}%
\pgfpathlineto{\pgfqpoint{4.306288in}{2.005584in}}%
\pgfpathlineto{\pgfqpoint{4.295135in}{2.001315in}}%
\pgfpathlineto{\pgfqpoint{4.283974in}{1.996966in}}%
\pgfpathlineto{\pgfqpoint{4.272806in}{1.992541in}}%
\pgfpathlineto{\pgfqpoint{4.279326in}{1.978601in}}%
\pgfpathlineto{\pgfqpoint{4.285853in}{1.964774in}}%
\pgfpathlineto{\pgfqpoint{4.292386in}{1.951038in}}%
\pgfpathlineto{\pgfqpoint{4.298924in}{1.937368in}}%
\pgfpathclose%
\pgfusepath{stroke,fill}%
\end{pgfscope}%
\begin{pgfscope}%
\pgfpathrectangle{\pgfqpoint{0.887500in}{0.275000in}}{\pgfqpoint{4.225000in}{4.225000in}}%
\pgfusepath{clip}%
\pgfsetbuttcap%
\pgfsetroundjoin%
\definecolor{currentfill}{rgb}{0.147607,0.511733,0.557049}%
\pgfsetfillcolor{currentfill}%
\pgfsetfillopacity{0.700000}%
\pgfsetlinewidth{0.501875pt}%
\definecolor{currentstroke}{rgb}{1.000000,1.000000,1.000000}%
\pgfsetstrokecolor{currentstroke}%
\pgfsetstrokeopacity{0.500000}%
\pgfsetdash{}{0pt}%
\pgfpathmoveto{\pgfqpoint{1.780955in}{2.215581in}}%
\pgfpathlineto{\pgfqpoint{1.792726in}{2.219110in}}%
\pgfpathlineto{\pgfqpoint{1.804492in}{2.222631in}}%
\pgfpathlineto{\pgfqpoint{1.816253in}{2.226146in}}%
\pgfpathlineto{\pgfqpoint{1.828008in}{2.229656in}}%
\pgfpathlineto{\pgfqpoint{1.839757in}{2.233162in}}%
\pgfpathlineto{\pgfqpoint{1.833881in}{2.241974in}}%
\pgfpathlineto{\pgfqpoint{1.828010in}{2.250761in}}%
\pgfpathlineto{\pgfqpoint{1.822143in}{2.259522in}}%
\pgfpathlineto{\pgfqpoint{1.816281in}{2.268258in}}%
\pgfpathlineto{\pgfqpoint{1.810423in}{2.276968in}}%
\pgfpathlineto{\pgfqpoint{1.798685in}{2.273508in}}%
\pgfpathlineto{\pgfqpoint{1.786942in}{2.270045in}}%
\pgfpathlineto{\pgfqpoint{1.775192in}{2.266576in}}%
\pgfpathlineto{\pgfqpoint{1.763437in}{2.263102in}}%
\pgfpathlineto{\pgfqpoint{1.751677in}{2.259621in}}%
\pgfpathlineto{\pgfqpoint{1.757523in}{2.250865in}}%
\pgfpathlineto{\pgfqpoint{1.763375in}{2.242083in}}%
\pgfpathlineto{\pgfqpoint{1.769230in}{2.233275in}}%
\pgfpathlineto{\pgfqpoint{1.775090in}{2.224441in}}%
\pgfpathclose%
\pgfusepath{stroke,fill}%
\end{pgfscope}%
\begin{pgfscope}%
\pgfpathrectangle{\pgfqpoint{0.887500in}{0.275000in}}{\pgfqpoint{4.225000in}{4.225000in}}%
\pgfusepath{clip}%
\pgfsetbuttcap%
\pgfsetroundjoin%
\definecolor{currentfill}{rgb}{0.126326,0.644107,0.525311}%
\pgfsetfillcolor{currentfill}%
\pgfsetfillopacity{0.700000}%
\pgfsetlinewidth{0.501875pt}%
\definecolor{currentstroke}{rgb}{1.000000,1.000000,1.000000}%
\pgfsetstrokecolor{currentstroke}%
\pgfsetstrokeopacity{0.500000}%
\pgfsetdash{}{0pt}%
\pgfpathmoveto{\pgfqpoint{3.533018in}{2.477768in}}%
\pgfpathlineto{\pgfqpoint{3.544396in}{2.484812in}}%
\pgfpathlineto{\pgfqpoint{3.555765in}{2.491385in}}%
\pgfpathlineto{\pgfqpoint{3.567124in}{2.497520in}}%
\pgfpathlineto{\pgfqpoint{3.578474in}{2.503250in}}%
\pgfpathlineto{\pgfqpoint{3.589814in}{2.508608in}}%
\pgfpathlineto{\pgfqpoint{3.583443in}{2.523226in}}%
\pgfpathlineto{\pgfqpoint{3.577072in}{2.537705in}}%
\pgfpathlineto{\pgfqpoint{3.570702in}{2.552064in}}%
\pgfpathlineto{\pgfqpoint{3.564333in}{2.566324in}}%
\pgfpathlineto{\pgfqpoint{3.557966in}{2.580507in}}%
\pgfpathlineto{\pgfqpoint{3.546627in}{2.575016in}}%
\pgfpathlineto{\pgfqpoint{3.535280in}{2.569087in}}%
\pgfpathlineto{\pgfqpoint{3.523922in}{2.562710in}}%
\pgfpathlineto{\pgfqpoint{3.512556in}{2.555877in}}%
\pgfpathlineto{\pgfqpoint{3.501181in}{2.548577in}}%
\pgfpathlineto{\pgfqpoint{3.507546in}{2.534646in}}%
\pgfpathlineto{\pgfqpoint{3.513913in}{2.520669in}}%
\pgfpathlineto{\pgfqpoint{3.520281in}{2.506578in}}%
\pgfpathlineto{\pgfqpoint{3.526650in}{2.492300in}}%
\pgfpathclose%
\pgfusepath{stroke,fill}%
\end{pgfscope}%
\begin{pgfscope}%
\pgfpathrectangle{\pgfqpoint{0.887500in}{0.275000in}}{\pgfqpoint{4.225000in}{4.225000in}}%
\pgfusepath{clip}%
\pgfsetbuttcap%
\pgfsetroundjoin%
\definecolor{currentfill}{rgb}{0.122312,0.633153,0.530398}%
\pgfsetfillcolor{currentfill}%
\pgfsetfillopacity{0.700000}%
\pgfsetlinewidth{0.501875pt}%
\definecolor{currentstroke}{rgb}{1.000000,1.000000,1.000000}%
\pgfsetstrokecolor{currentstroke}%
\pgfsetstrokeopacity{0.500000}%
\pgfsetdash{}{0pt}%
\pgfpathmoveto{\pgfqpoint{3.386924in}{2.437475in}}%
\pgfpathlineto{\pgfqpoint{3.398393in}{2.453008in}}%
\pgfpathlineto{\pgfqpoint{3.409854in}{2.467475in}}%
\pgfpathlineto{\pgfqpoint{3.421305in}{2.480733in}}%
\pgfpathlineto{\pgfqpoint{3.432745in}{2.492873in}}%
\pgfpathlineto{\pgfqpoint{3.444175in}{2.504011in}}%
\pgfpathlineto{\pgfqpoint{3.437820in}{2.517931in}}%
\pgfpathlineto{\pgfqpoint{3.431469in}{2.532089in}}%
\pgfpathlineto{\pgfqpoint{3.425125in}{2.546597in}}%
\pgfpathlineto{\pgfqpoint{3.418787in}{2.561559in}}%
\pgfpathlineto{\pgfqpoint{3.412455in}{2.576951in}}%
\pgfpathlineto{\pgfqpoint{3.401039in}{2.566400in}}%
\pgfpathlineto{\pgfqpoint{3.389614in}{2.555011in}}%
\pgfpathlineto{\pgfqpoint{3.378181in}{2.542703in}}%
\pgfpathlineto{\pgfqpoint{3.366740in}{2.529410in}}%
\pgfpathlineto{\pgfqpoint{3.355292in}{2.515219in}}%
\pgfpathlineto{\pgfqpoint{3.361603in}{2.498140in}}%
\pgfpathlineto{\pgfqpoint{3.367921in}{2.481840in}}%
\pgfpathlineto{\pgfqpoint{3.374248in}{2.466388in}}%
\pgfpathlineto{\pgfqpoint{3.380582in}{2.451651in}}%
\pgfpathclose%
\pgfusepath{stroke,fill}%
\end{pgfscope}%
\begin{pgfscope}%
\pgfpathrectangle{\pgfqpoint{0.887500in}{0.275000in}}{\pgfqpoint{4.225000in}{4.225000in}}%
\pgfusepath{clip}%
\pgfsetbuttcap%
\pgfsetroundjoin%
\definecolor{currentfill}{rgb}{0.192357,0.403199,0.555836}%
\pgfsetfillcolor{currentfill}%
\pgfsetfillopacity{0.700000}%
\pgfsetlinewidth{0.501875pt}%
\definecolor{currentstroke}{rgb}{1.000000,1.000000,1.000000}%
\pgfsetstrokecolor{currentstroke}%
\pgfsetstrokeopacity{0.500000}%
\pgfsetdash{}{0pt}%
\pgfpathmoveto{\pgfqpoint{4.216869in}{1.969716in}}%
\pgfpathlineto{\pgfqpoint{4.228067in}{1.974304in}}%
\pgfpathlineto{\pgfqpoint{4.239260in}{1.978903in}}%
\pgfpathlineto{\pgfqpoint{4.250448in}{1.983491in}}%
\pgfpathlineto{\pgfqpoint{4.261630in}{1.988044in}}%
\pgfpathlineto{\pgfqpoint{4.272806in}{1.992541in}}%
\pgfpathlineto{\pgfqpoint{4.266292in}{2.006617in}}%
\pgfpathlineto{\pgfqpoint{4.259785in}{2.020843in}}%
\pgfpathlineto{\pgfqpoint{4.253286in}{2.035206in}}%
\pgfpathlineto{\pgfqpoint{4.246792in}{2.049685in}}%
\pgfpathlineto{\pgfqpoint{4.240304in}{2.064261in}}%
\pgfpathlineto{\pgfqpoint{4.229143in}{2.060133in}}%
\pgfpathlineto{\pgfqpoint{4.217975in}{2.055964in}}%
\pgfpathlineto{\pgfqpoint{4.206801in}{2.051765in}}%
\pgfpathlineto{\pgfqpoint{4.195621in}{2.047547in}}%
\pgfpathlineto{\pgfqpoint{4.184436in}{2.043320in}}%
\pgfpathlineto{\pgfqpoint{4.190904in}{2.028137in}}%
\pgfpathlineto{\pgfqpoint{4.197380in}{2.013136in}}%
\pgfpathlineto{\pgfqpoint{4.203865in}{1.998368in}}%
\pgfpathlineto{\pgfqpoint{4.210361in}{1.983883in}}%
\pgfpathclose%
\pgfusepath{stroke,fill}%
\end{pgfscope}%
\begin{pgfscope}%
\pgfpathrectangle{\pgfqpoint{0.887500in}{0.275000in}}{\pgfqpoint{4.225000in}{4.225000in}}%
\pgfusepath{clip}%
\pgfsetbuttcap%
\pgfsetroundjoin%
\definecolor{currentfill}{rgb}{0.154815,0.493313,0.557840}%
\pgfsetfillcolor{currentfill}%
\pgfsetfillopacity{0.700000}%
\pgfsetlinewidth{0.501875pt}%
\definecolor{currentstroke}{rgb}{1.000000,1.000000,1.000000}%
\pgfsetstrokecolor{currentstroke}%
\pgfsetstrokeopacity{0.500000}%
\pgfsetdash{}{0pt}%
\pgfpathmoveto{\pgfqpoint{3.246542in}{2.121707in}}%
\pgfpathlineto{\pgfqpoint{3.258003in}{2.137793in}}%
\pgfpathlineto{\pgfqpoint{3.269466in}{2.153847in}}%
\pgfpathlineto{\pgfqpoint{3.280931in}{2.169974in}}%
\pgfpathlineto{\pgfqpoint{3.292399in}{2.186257in}}%
\pgfpathlineto{\pgfqpoint{3.303869in}{2.202699in}}%
\pgfpathlineto{\pgfqpoint{3.297553in}{2.218753in}}%
\pgfpathlineto{\pgfqpoint{3.291238in}{2.234548in}}%
\pgfpathlineto{\pgfqpoint{3.284924in}{2.250162in}}%
\pgfpathlineto{\pgfqpoint{3.278611in}{2.265673in}}%
\pgfpathlineto{\pgfqpoint{3.272300in}{2.281158in}}%
\pgfpathlineto{\pgfqpoint{3.260853in}{2.266976in}}%
\pgfpathlineto{\pgfqpoint{3.249404in}{2.252510in}}%
\pgfpathlineto{\pgfqpoint{3.237955in}{2.237581in}}%
\pgfpathlineto{\pgfqpoint{3.226504in}{2.222055in}}%
\pgfpathlineto{\pgfqpoint{3.215051in}{2.205810in}}%
\pgfpathlineto{\pgfqpoint{3.221350in}{2.189940in}}%
\pgfpathlineto{\pgfqpoint{3.227649in}{2.173594in}}%
\pgfpathlineto{\pgfqpoint{3.233947in}{2.156773in}}%
\pgfpathlineto{\pgfqpoint{3.240245in}{2.139477in}}%
\pgfpathclose%
\pgfusepath{stroke,fill}%
\end{pgfscope}%
\begin{pgfscope}%
\pgfpathrectangle{\pgfqpoint{0.887500in}{0.275000in}}{\pgfqpoint{4.225000in}{4.225000in}}%
\pgfusepath{clip}%
\pgfsetbuttcap%
\pgfsetroundjoin%
\definecolor{currentfill}{rgb}{0.204903,0.375746,0.553533}%
\pgfsetfillcolor{currentfill}%
\pgfsetfillopacity{0.700000}%
\pgfsetlinewidth{0.501875pt}%
\definecolor{currentstroke}{rgb}{1.000000,1.000000,1.000000}%
\pgfsetstrokecolor{currentstroke}%
\pgfsetstrokeopacity{0.500000}%
\pgfsetdash{}{0pt}%
\pgfpathmoveto{\pgfqpoint{3.163531in}{1.904500in}}%
\pgfpathlineto{\pgfqpoint{3.174970in}{1.912823in}}%
\pgfpathlineto{\pgfqpoint{3.186410in}{1.922341in}}%
\pgfpathlineto{\pgfqpoint{3.197851in}{1.932907in}}%
\pgfpathlineto{\pgfqpoint{3.209295in}{1.944374in}}%
\pgfpathlineto{\pgfqpoint{3.220741in}{1.956594in}}%
\pgfpathlineto{\pgfqpoint{3.214440in}{1.973119in}}%
\pgfpathlineto{\pgfqpoint{3.208140in}{1.989534in}}%
\pgfpathlineto{\pgfqpoint{3.201841in}{2.005737in}}%
\pgfpathlineto{\pgfqpoint{3.195543in}{2.021646in}}%
\pgfpathlineto{\pgfqpoint{3.189246in}{2.037235in}}%
\pgfpathlineto{\pgfqpoint{3.177791in}{2.019527in}}%
\pgfpathlineto{\pgfqpoint{3.166340in}{2.002316in}}%
\pgfpathlineto{\pgfqpoint{3.154893in}{1.986107in}}%
\pgfpathlineto{\pgfqpoint{3.143452in}{1.971405in}}%
\pgfpathlineto{\pgfqpoint{3.132015in}{1.958713in}}%
\pgfpathlineto{\pgfqpoint{3.138312in}{1.948010in}}%
\pgfpathlineto{\pgfqpoint{3.144611in}{1.937234in}}%
\pgfpathlineto{\pgfqpoint{3.150915in}{1.926385in}}%
\pgfpathlineto{\pgfqpoint{3.157221in}{1.915471in}}%
\pgfpathclose%
\pgfusepath{stroke,fill}%
\end{pgfscope}%
\begin{pgfscope}%
\pgfpathrectangle{\pgfqpoint{0.887500in}{0.275000in}}{\pgfqpoint{4.225000in}{4.225000in}}%
\pgfusepath{clip}%
\pgfsetbuttcap%
\pgfsetroundjoin%
\definecolor{currentfill}{rgb}{0.146180,0.515413,0.556823}%
\pgfsetfillcolor{currentfill}%
\pgfsetfillopacity{0.700000}%
\pgfsetlinewidth{0.501875pt}%
\definecolor{currentstroke}{rgb}{1.000000,1.000000,1.000000}%
\pgfsetstrokecolor{currentstroke}%
\pgfsetstrokeopacity{0.500000}%
\pgfsetdash{}{0pt}%
\pgfpathmoveto{\pgfqpoint{3.919394in}{2.208846in}}%
\pgfpathlineto{\pgfqpoint{3.930663in}{2.213607in}}%
\pgfpathlineto{\pgfqpoint{3.941924in}{2.218205in}}%
\pgfpathlineto{\pgfqpoint{3.953176in}{2.222642in}}%
\pgfpathlineto{\pgfqpoint{3.964419in}{2.226917in}}%
\pgfpathlineto{\pgfqpoint{3.975654in}{2.231032in}}%
\pgfpathlineto{\pgfqpoint{3.969201in}{2.245178in}}%
\pgfpathlineto{\pgfqpoint{3.962752in}{2.259399in}}%
\pgfpathlineto{\pgfqpoint{3.956308in}{2.273728in}}%
\pgfpathlineto{\pgfqpoint{3.949869in}{2.288155in}}%
\pgfpathlineto{\pgfqpoint{3.943434in}{2.302660in}}%
\pgfpathlineto{\pgfqpoint{3.932208in}{2.298802in}}%
\pgfpathlineto{\pgfqpoint{3.920974in}{2.294841in}}%
\pgfpathlineto{\pgfqpoint{3.909733in}{2.290765in}}%
\pgfpathlineto{\pgfqpoint{3.898484in}{2.286566in}}%
\pgfpathlineto{\pgfqpoint{3.887226in}{2.282234in}}%
\pgfpathlineto{\pgfqpoint{3.893650in}{2.267336in}}%
\pgfpathlineto{\pgfqpoint{3.900079in}{2.252528in}}%
\pgfpathlineto{\pgfqpoint{3.906512in}{2.237839in}}%
\pgfpathlineto{\pgfqpoint{3.912951in}{2.223290in}}%
\pgfpathclose%
\pgfusepath{stroke,fill}%
\end{pgfscope}%
\begin{pgfscope}%
\pgfpathrectangle{\pgfqpoint{0.887500in}{0.275000in}}{\pgfqpoint{4.225000in}{4.225000in}}%
\pgfusepath{clip}%
\pgfsetbuttcap%
\pgfsetroundjoin%
\definecolor{currentfill}{rgb}{0.177423,0.437527,0.557565}%
\pgfsetfillcolor{currentfill}%
\pgfsetfillopacity{0.700000}%
\pgfsetlinewidth{0.501875pt}%
\definecolor{currentstroke}{rgb}{1.000000,1.000000,1.000000}%
\pgfsetstrokecolor{currentstroke}%
\pgfsetstrokeopacity{0.500000}%
\pgfsetdash{}{0pt}%
\pgfpathmoveto{\pgfqpoint{2.516526in}{2.056757in}}%
\pgfpathlineto{\pgfqpoint{2.528119in}{2.060316in}}%
\pgfpathlineto{\pgfqpoint{2.539706in}{2.063864in}}%
\pgfpathlineto{\pgfqpoint{2.551288in}{2.067405in}}%
\pgfpathlineto{\pgfqpoint{2.562863in}{2.070943in}}%
\pgfpathlineto{\pgfqpoint{2.574433in}{2.074482in}}%
\pgfpathlineto{\pgfqpoint{2.568308in}{2.083908in}}%
\pgfpathlineto{\pgfqpoint{2.562187in}{2.093295in}}%
\pgfpathlineto{\pgfqpoint{2.556070in}{2.102644in}}%
\pgfpathlineto{\pgfqpoint{2.549958in}{2.111956in}}%
\pgfpathlineto{\pgfqpoint{2.543850in}{2.121231in}}%
\pgfpathlineto{\pgfqpoint{2.532290in}{2.117741in}}%
\pgfpathlineto{\pgfqpoint{2.520725in}{2.114253in}}%
\pgfpathlineto{\pgfqpoint{2.509153in}{2.110764in}}%
\pgfpathlineto{\pgfqpoint{2.497577in}{2.107268in}}%
\pgfpathlineto{\pgfqpoint{2.485994in}{2.103760in}}%
\pgfpathlineto{\pgfqpoint{2.492092in}{2.094436in}}%
\pgfpathlineto{\pgfqpoint{2.498194in}{2.085074in}}%
\pgfpathlineto{\pgfqpoint{2.504301in}{2.075674in}}%
\pgfpathlineto{\pgfqpoint{2.510411in}{2.066236in}}%
\pgfpathclose%
\pgfusepath{stroke,fill}%
\end{pgfscope}%
\begin{pgfscope}%
\pgfpathrectangle{\pgfqpoint{0.887500in}{0.275000in}}{\pgfqpoint{4.225000in}{4.225000in}}%
\pgfusepath{clip}%
\pgfsetbuttcap%
\pgfsetroundjoin%
\definecolor{currentfill}{rgb}{0.199430,0.387607,0.554642}%
\pgfsetfillcolor{currentfill}%
\pgfsetfillopacity{0.700000}%
\pgfsetlinewidth{0.501875pt}%
\definecolor{currentstroke}{rgb}{1.000000,1.000000,1.000000}%
\pgfsetstrokecolor{currentstroke}%
\pgfsetstrokeopacity{0.500000}%
\pgfsetdash{}{0pt}%
\pgfpathmoveto{\pgfqpoint{2.928817in}{1.950091in}}%
\pgfpathlineto{\pgfqpoint{2.940306in}{1.954201in}}%
\pgfpathlineto{\pgfqpoint{2.951789in}{1.958510in}}%
\pgfpathlineto{\pgfqpoint{2.963267in}{1.962818in}}%
\pgfpathlineto{\pgfqpoint{2.974739in}{1.966922in}}%
\pgfpathlineto{\pgfqpoint{2.986207in}{1.970617in}}%
\pgfpathlineto{\pgfqpoint{2.979951in}{1.980779in}}%
\pgfpathlineto{\pgfqpoint{2.973699in}{1.990882in}}%
\pgfpathlineto{\pgfqpoint{2.967450in}{2.000924in}}%
\pgfpathlineto{\pgfqpoint{2.961206in}{2.010905in}}%
\pgfpathlineto{\pgfqpoint{2.954965in}{2.020828in}}%
\pgfpathlineto{\pgfqpoint{2.943506in}{2.017163in}}%
\pgfpathlineto{\pgfqpoint{2.932044in}{2.013027in}}%
\pgfpathlineto{\pgfqpoint{2.920576in}{2.008655in}}%
\pgfpathlineto{\pgfqpoint{2.909104in}{2.004281in}}%
\pgfpathlineto{\pgfqpoint{2.897625in}{2.000134in}}%
\pgfpathlineto{\pgfqpoint{2.903856in}{1.990217in}}%
\pgfpathlineto{\pgfqpoint{2.910091in}{1.980255in}}%
\pgfpathlineto{\pgfqpoint{2.916329in}{1.970246in}}%
\pgfpathlineto{\pgfqpoint{2.922571in}{1.960192in}}%
\pgfpathclose%
\pgfusepath{stroke,fill}%
\end{pgfscope}%
\begin{pgfscope}%
\pgfpathrectangle{\pgfqpoint{0.887500in}{0.275000in}}{\pgfqpoint{4.225000in}{4.225000in}}%
\pgfusepath{clip}%
\pgfsetbuttcap%
\pgfsetroundjoin%
\definecolor{currentfill}{rgb}{0.246811,0.283237,0.535941}%
\pgfsetfillcolor{currentfill}%
\pgfsetfillopacity{0.700000}%
\pgfsetlinewidth{0.501875pt}%
\definecolor{currentstroke}{rgb}{1.000000,1.000000,1.000000}%
\pgfsetstrokecolor{currentstroke}%
\pgfsetstrokeopacity{0.500000}%
\pgfsetdash{}{0pt}%
\pgfpathmoveto{\pgfqpoint{4.514777in}{1.743038in}}%
\pgfpathlineto{\pgfqpoint{4.525859in}{1.746284in}}%
\pgfpathlineto{\pgfqpoint{4.536936in}{1.749522in}}%
\pgfpathlineto{\pgfqpoint{4.548006in}{1.752758in}}%
\pgfpathlineto{\pgfqpoint{4.559071in}{1.755998in}}%
\pgfpathlineto{\pgfqpoint{4.570131in}{1.759246in}}%
\pgfpathlineto{\pgfqpoint{4.563565in}{1.773550in}}%
\pgfpathlineto{\pgfqpoint{4.557007in}{1.787968in}}%
\pgfpathlineto{\pgfqpoint{4.550455in}{1.802486in}}%
\pgfpathlineto{\pgfqpoint{4.543909in}{1.817090in}}%
\pgfpathlineto{\pgfqpoint{4.537368in}{1.831766in}}%
\pgfpathlineto{\pgfqpoint{4.526307in}{1.828456in}}%
\pgfpathlineto{\pgfqpoint{4.515243in}{1.825206in}}%
\pgfpathlineto{\pgfqpoint{4.504175in}{1.822018in}}%
\pgfpathlineto{\pgfqpoint{4.493104in}{1.818881in}}%
\pgfpathlineto{\pgfqpoint{4.482028in}{1.815786in}}%
\pgfpathlineto{\pgfqpoint{4.488569in}{1.801137in}}%
\pgfpathlineto{\pgfqpoint{4.495114in}{1.786527in}}%
\pgfpathlineto{\pgfqpoint{4.501663in}{1.771967in}}%
\pgfpathlineto{\pgfqpoint{4.508218in}{1.757467in}}%
\pgfpathclose%
\pgfusepath{stroke,fill}%
\end{pgfscope}%
\begin{pgfscope}%
\pgfpathrectangle{\pgfqpoint{0.887500in}{0.275000in}}{\pgfqpoint{4.225000in}{4.225000in}}%
\pgfusepath{clip}%
\pgfsetbuttcap%
\pgfsetroundjoin%
\definecolor{currentfill}{rgb}{0.187231,0.414746,0.556547}%
\pgfsetfillcolor{currentfill}%
\pgfsetfillopacity{0.700000}%
\pgfsetlinewidth{0.501875pt}%
\definecolor{currentstroke}{rgb}{1.000000,1.000000,1.000000}%
\pgfsetstrokecolor{currentstroke}%
\pgfsetstrokeopacity{0.500000}%
\pgfsetdash{}{0pt}%
\pgfpathmoveto{\pgfqpoint{3.220741in}{1.956594in}}%
\pgfpathlineto{\pgfqpoint{3.232188in}{1.969421in}}%
\pgfpathlineto{\pgfqpoint{3.243638in}{1.982740in}}%
\pgfpathlineto{\pgfqpoint{3.255091in}{1.996601in}}%
\pgfpathlineto{\pgfqpoint{3.266548in}{2.011101in}}%
\pgfpathlineto{\pgfqpoint{3.278009in}{2.026342in}}%
\pgfpathlineto{\pgfqpoint{3.271717in}{2.046054in}}%
\pgfpathlineto{\pgfqpoint{3.265425in}{2.065573in}}%
\pgfpathlineto{\pgfqpoint{3.259132in}{2.084747in}}%
\pgfpathlineto{\pgfqpoint{3.252838in}{2.103464in}}%
\pgfpathlineto{\pgfqpoint{3.246542in}{2.121707in}}%
\pgfpathlineto{\pgfqpoint{3.235081in}{2.105484in}}%
\pgfpathlineto{\pgfqpoint{3.223622in}{2.089017in}}%
\pgfpathlineto{\pgfqpoint{3.212163in}{2.072203in}}%
\pgfpathlineto{\pgfqpoint{3.200704in}{2.054935in}}%
\pgfpathlineto{\pgfqpoint{3.189246in}{2.037235in}}%
\pgfpathlineto{\pgfqpoint{3.195543in}{2.021646in}}%
\pgfpathlineto{\pgfqpoint{3.201841in}{2.005737in}}%
\pgfpathlineto{\pgfqpoint{3.208140in}{1.989534in}}%
\pgfpathlineto{\pgfqpoint{3.214440in}{1.973119in}}%
\pgfpathclose%
\pgfusepath{stroke,fill}%
\end{pgfscope}%
\begin{pgfscope}%
\pgfpathrectangle{\pgfqpoint{0.887500in}{0.275000in}}{\pgfqpoint{4.225000in}{4.225000in}}%
\pgfusepath{clip}%
\pgfsetbuttcap%
\pgfsetroundjoin%
\definecolor{currentfill}{rgb}{0.163625,0.471133,0.558148}%
\pgfsetfillcolor{currentfill}%
\pgfsetfillopacity{0.700000}%
\pgfsetlinewidth{0.501875pt}%
\definecolor{currentstroke}{rgb}{1.000000,1.000000,1.000000}%
\pgfsetstrokecolor{currentstroke}%
\pgfsetstrokeopacity{0.500000}%
\pgfsetdash{}{0pt}%
\pgfpathmoveto{\pgfqpoint{2.192867in}{2.124126in}}%
\pgfpathlineto{\pgfqpoint{2.204540in}{2.127696in}}%
\pgfpathlineto{\pgfqpoint{2.216208in}{2.131267in}}%
\pgfpathlineto{\pgfqpoint{2.227870in}{2.134841in}}%
\pgfpathlineto{\pgfqpoint{2.239526in}{2.138422in}}%
\pgfpathlineto{\pgfqpoint{2.251176in}{2.142012in}}%
\pgfpathlineto{\pgfqpoint{2.245158in}{2.151127in}}%
\pgfpathlineto{\pgfqpoint{2.239143in}{2.160214in}}%
\pgfpathlineto{\pgfqpoint{2.233134in}{2.169272in}}%
\pgfpathlineto{\pgfqpoint{2.227128in}{2.178301in}}%
\pgfpathlineto{\pgfqpoint{2.221127in}{2.187301in}}%
\pgfpathlineto{\pgfqpoint{2.209488in}{2.183763in}}%
\pgfpathlineto{\pgfqpoint{2.197842in}{2.180233in}}%
\pgfpathlineto{\pgfqpoint{2.186191in}{2.176711in}}%
\pgfpathlineto{\pgfqpoint{2.174534in}{2.173193in}}%
\pgfpathlineto{\pgfqpoint{2.162871in}{2.169675in}}%
\pgfpathlineto{\pgfqpoint{2.168861in}{2.160624in}}%
\pgfpathlineto{\pgfqpoint{2.174856in}{2.151543in}}%
\pgfpathlineto{\pgfqpoint{2.180855in}{2.142433in}}%
\pgfpathlineto{\pgfqpoint{2.186859in}{2.133295in}}%
\pgfpathclose%
\pgfusepath{stroke,fill}%
\end{pgfscope}%
\begin{pgfscope}%
\pgfpathrectangle{\pgfqpoint{0.887500in}{0.275000in}}{\pgfqpoint{4.225000in}{4.225000in}}%
\pgfusepath{clip}%
\pgfsetbuttcap%
\pgfsetroundjoin%
\definecolor{currentfill}{rgb}{0.151918,0.500685,0.557587}%
\pgfsetfillcolor{currentfill}%
\pgfsetfillopacity{0.700000}%
\pgfsetlinewidth{0.501875pt}%
\definecolor{currentstroke}{rgb}{1.000000,1.000000,1.000000}%
\pgfsetstrokecolor{currentstroke}%
\pgfsetstrokeopacity{0.500000}%
\pgfsetdash{}{0pt}%
\pgfpathmoveto{\pgfqpoint{1.869203in}{2.188697in}}%
\pgfpathlineto{\pgfqpoint{1.880957in}{2.192246in}}%
\pgfpathlineto{\pgfqpoint{1.892705in}{2.195793in}}%
\pgfpathlineto{\pgfqpoint{1.904448in}{2.199339in}}%
\pgfpathlineto{\pgfqpoint{1.916185in}{2.202885in}}%
\pgfpathlineto{\pgfqpoint{1.927916in}{2.206432in}}%
\pgfpathlineto{\pgfqpoint{1.922007in}{2.215333in}}%
\pgfpathlineto{\pgfqpoint{1.916102in}{2.224208in}}%
\pgfpathlineto{\pgfqpoint{1.910201in}{2.233056in}}%
\pgfpathlineto{\pgfqpoint{1.904306in}{2.241877in}}%
\pgfpathlineto{\pgfqpoint{1.898414in}{2.250672in}}%
\pgfpathlineto{\pgfqpoint{1.886694in}{2.247169in}}%
\pgfpathlineto{\pgfqpoint{1.874969in}{2.243667in}}%
\pgfpathlineto{\pgfqpoint{1.863237in}{2.240166in}}%
\pgfpathlineto{\pgfqpoint{1.851500in}{2.236665in}}%
\pgfpathlineto{\pgfqpoint{1.839757in}{2.233162in}}%
\pgfpathlineto{\pgfqpoint{1.845637in}{2.224322in}}%
\pgfpathlineto{\pgfqpoint{1.851521in}{2.215457in}}%
\pgfpathlineto{\pgfqpoint{1.857411in}{2.206564in}}%
\pgfpathlineto{\pgfqpoint{1.863304in}{2.197644in}}%
\pgfpathclose%
\pgfusepath{stroke,fill}%
\end{pgfscope}%
\begin{pgfscope}%
\pgfpathrectangle{\pgfqpoint{0.887500in}{0.275000in}}{\pgfqpoint{4.225000in}{4.225000in}}%
\pgfusepath{clip}%
\pgfsetbuttcap%
\pgfsetroundjoin%
\definecolor{currentfill}{rgb}{0.206756,0.371758,0.553117}%
\pgfsetfillcolor{currentfill}%
\pgfsetfillopacity{0.700000}%
\pgfsetlinewidth{0.501875pt}%
\definecolor{currentstroke}{rgb}{1.000000,1.000000,1.000000}%
\pgfsetstrokecolor{currentstroke}%
\pgfsetstrokeopacity{0.500000}%
\pgfsetdash{}{0pt}%
\pgfpathmoveto{\pgfqpoint{3.017545in}{1.918928in}}%
\pgfpathlineto{\pgfqpoint{3.029017in}{1.922204in}}%
\pgfpathlineto{\pgfqpoint{3.040483in}{1.924891in}}%
\pgfpathlineto{\pgfqpoint{3.051943in}{1.926871in}}%
\pgfpathlineto{\pgfqpoint{3.063397in}{1.928422in}}%
\pgfpathlineto{\pgfqpoint{3.074844in}{1.930077in}}%
\pgfpathlineto{\pgfqpoint{3.068561in}{1.939853in}}%
\pgfpathlineto{\pgfqpoint{3.062282in}{1.949609in}}%
\pgfpathlineto{\pgfqpoint{3.056007in}{1.959349in}}%
\pgfpathlineto{\pgfqpoint{3.049735in}{1.969076in}}%
\pgfpathlineto{\pgfqpoint{3.043468in}{1.978791in}}%
\pgfpathlineto{\pgfqpoint{3.032029in}{1.977951in}}%
\pgfpathlineto{\pgfqpoint{3.020582in}{1.977255in}}%
\pgfpathlineto{\pgfqpoint{3.009129in}{1.975964in}}%
\pgfpathlineto{\pgfqpoint{2.997671in}{1.973699in}}%
\pgfpathlineto{\pgfqpoint{2.986207in}{1.970617in}}%
\pgfpathlineto{\pgfqpoint{2.992467in}{1.960395in}}%
\pgfpathlineto{\pgfqpoint{2.998731in}{1.950115in}}%
\pgfpathlineto{\pgfqpoint{3.004999in}{1.939776in}}%
\pgfpathlineto{\pgfqpoint{3.011270in}{1.929381in}}%
\pgfpathclose%
\pgfusepath{stroke,fill}%
\end{pgfscope}%
\begin{pgfscope}%
\pgfpathrectangle{\pgfqpoint{0.887500in}{0.275000in}}{\pgfqpoint{4.225000in}{4.225000in}}%
\pgfusepath{clip}%
\pgfsetbuttcap%
\pgfsetroundjoin%
\definecolor{currentfill}{rgb}{0.136408,0.541173,0.554483}%
\pgfsetfillcolor{currentfill}%
\pgfsetfillopacity{0.700000}%
\pgfsetlinewidth{0.501875pt}%
\definecolor{currentstroke}{rgb}{1.000000,1.000000,1.000000}%
\pgfsetstrokecolor{currentstroke}%
\pgfsetstrokeopacity{0.500000}%
\pgfsetdash{}{0pt}%
\pgfpathmoveto{\pgfqpoint{3.830835in}{2.259059in}}%
\pgfpathlineto{\pgfqpoint{3.842124in}{2.263741in}}%
\pgfpathlineto{\pgfqpoint{3.853409in}{2.268458in}}%
\pgfpathlineto{\pgfqpoint{3.864688in}{2.273151in}}%
\pgfpathlineto{\pgfqpoint{3.875961in}{2.277761in}}%
\pgfpathlineto{\pgfqpoint{3.887226in}{2.282234in}}%
\pgfpathlineto{\pgfqpoint{3.880806in}{2.297194in}}%
\pgfpathlineto{\pgfqpoint{3.874389in}{2.312187in}}%
\pgfpathlineto{\pgfqpoint{3.867974in}{2.327183in}}%
\pgfpathlineto{\pgfqpoint{3.861561in}{2.342154in}}%
\pgfpathlineto{\pgfqpoint{3.855149in}{2.357070in}}%
\pgfpathlineto{\pgfqpoint{3.843894in}{2.353040in}}%
\pgfpathlineto{\pgfqpoint{3.832632in}{2.348874in}}%
\pgfpathlineto{\pgfqpoint{3.821363in}{2.344585in}}%
\pgfpathlineto{\pgfqpoint{3.810086in}{2.340188in}}%
\pgfpathlineto{\pgfqpoint{3.798803in}{2.335702in}}%
\pgfpathlineto{\pgfqpoint{3.805207in}{2.320469in}}%
\pgfpathlineto{\pgfqpoint{3.811611in}{2.305157in}}%
\pgfpathlineto{\pgfqpoint{3.818017in}{2.289798in}}%
\pgfpathlineto{\pgfqpoint{3.824425in}{2.274422in}}%
\pgfpathclose%
\pgfusepath{stroke,fill}%
\end{pgfscope}%
\begin{pgfscope}%
\pgfpathrectangle{\pgfqpoint{0.887500in}{0.275000in}}{\pgfqpoint{4.225000in}{4.225000in}}%
\pgfusepath{clip}%
\pgfsetbuttcap%
\pgfsetroundjoin%
\definecolor{currentfill}{rgb}{0.140536,0.530132,0.555659}%
\pgfsetfillcolor{currentfill}%
\pgfsetfillopacity{0.700000}%
\pgfsetlinewidth{0.501875pt}%
\definecolor{currentstroke}{rgb}{1.000000,1.000000,1.000000}%
\pgfsetstrokecolor{currentstroke}%
\pgfsetstrokeopacity{0.500000}%
\pgfsetdash{}{0pt}%
\pgfpathmoveto{\pgfqpoint{3.303869in}{2.202699in}}%
\pgfpathlineto{\pgfqpoint{3.315342in}{2.219291in}}%
\pgfpathlineto{\pgfqpoint{3.326818in}{2.236020in}}%
\pgfpathlineto{\pgfqpoint{3.338297in}{2.252875in}}%
\pgfpathlineto{\pgfqpoint{3.349779in}{2.269846in}}%
\pgfpathlineto{\pgfqpoint{3.361264in}{2.286920in}}%
\pgfpathlineto{\pgfqpoint{3.354916in}{2.300121in}}%
\pgfpathlineto{\pgfqpoint{3.348570in}{2.313310in}}%
\pgfpathlineto{\pgfqpoint{3.342229in}{2.326651in}}%
\pgfpathlineto{\pgfqpoint{3.335892in}{2.340305in}}%
\pgfpathlineto{\pgfqpoint{3.329561in}{2.354433in}}%
\pgfpathlineto{\pgfqpoint{3.318101in}{2.338831in}}%
\pgfpathlineto{\pgfqpoint{3.306646in}{2.323892in}}%
\pgfpathlineto{\pgfqpoint{3.295195in}{2.309427in}}%
\pgfpathlineto{\pgfqpoint{3.283747in}{2.295245in}}%
\pgfpathlineto{\pgfqpoint{3.272300in}{2.281158in}}%
\pgfpathlineto{\pgfqpoint{3.278611in}{2.265673in}}%
\pgfpathlineto{\pgfqpoint{3.284924in}{2.250162in}}%
\pgfpathlineto{\pgfqpoint{3.291238in}{2.234548in}}%
\pgfpathlineto{\pgfqpoint{3.297553in}{2.218753in}}%
\pgfpathclose%
\pgfusepath{stroke,fill}%
\end{pgfscope}%
\begin{pgfscope}%
\pgfpathrectangle{\pgfqpoint{0.887500in}{0.275000in}}{\pgfqpoint{4.225000in}{4.225000in}}%
\pgfusepath{clip}%
\pgfsetbuttcap%
\pgfsetroundjoin%
\definecolor{currentfill}{rgb}{0.180629,0.429975,0.557282}%
\pgfsetfillcolor{currentfill}%
\pgfsetfillopacity{0.700000}%
\pgfsetlinewidth{0.501875pt}%
\definecolor{currentstroke}{rgb}{1.000000,1.000000,1.000000}%
\pgfsetstrokecolor{currentstroke}%
\pgfsetstrokeopacity{0.500000}%
\pgfsetdash{}{0pt}%
\pgfpathmoveto{\pgfqpoint{4.128438in}{2.022302in}}%
\pgfpathlineto{\pgfqpoint{4.139648in}{2.026497in}}%
\pgfpathlineto{\pgfqpoint{4.150852in}{2.030683in}}%
\pgfpathlineto{\pgfqpoint{4.162051in}{2.034879in}}%
\pgfpathlineto{\pgfqpoint{4.173246in}{2.039094in}}%
\pgfpathlineto{\pgfqpoint{4.184436in}{2.043320in}}%
\pgfpathlineto{\pgfqpoint{4.177974in}{2.058636in}}%
\pgfpathlineto{\pgfqpoint{4.171516in}{2.074035in}}%
\pgfpathlineto{\pgfqpoint{4.165062in}{2.089466in}}%
\pgfpathlineto{\pgfqpoint{4.158609in}{2.104881in}}%
\pgfpathlineto{\pgfqpoint{4.152157in}{2.120227in}}%
\pgfpathlineto{\pgfqpoint{4.140987in}{2.116667in}}%
\pgfpathlineto{\pgfqpoint{4.129811in}{2.113107in}}%
\pgfpathlineto{\pgfqpoint{4.118629in}{2.109550in}}%
\pgfpathlineto{\pgfqpoint{4.107442in}{2.105978in}}%
\pgfpathlineto{\pgfqpoint{4.096249in}{2.102367in}}%
\pgfpathlineto{\pgfqpoint{4.102685in}{2.086431in}}%
\pgfpathlineto{\pgfqpoint{4.109119in}{2.070373in}}%
\pgfpathlineto{\pgfqpoint{4.115555in}{2.054276in}}%
\pgfpathlineto{\pgfqpoint{4.121994in}{2.038225in}}%
\pgfpathclose%
\pgfusepath{stroke,fill}%
\end{pgfscope}%
\begin{pgfscope}%
\pgfpathrectangle{\pgfqpoint{0.887500in}{0.275000in}}{\pgfqpoint{4.225000in}{4.225000in}}%
\pgfusepath{clip}%
\pgfsetbuttcap%
\pgfsetroundjoin%
\definecolor{currentfill}{rgb}{0.182256,0.426184,0.557120}%
\pgfsetfillcolor{currentfill}%
\pgfsetfillopacity{0.700000}%
\pgfsetlinewidth{0.501875pt}%
\definecolor{currentstroke}{rgb}{1.000000,1.000000,1.000000}%
\pgfsetstrokecolor{currentstroke}%
\pgfsetstrokeopacity{0.500000}%
\pgfsetdash{}{0pt}%
\pgfpathmoveto{\pgfqpoint{2.605124in}{2.026732in}}%
\pgfpathlineto{\pgfqpoint{2.616698in}{2.030330in}}%
\pgfpathlineto{\pgfqpoint{2.628266in}{2.033935in}}%
\pgfpathlineto{\pgfqpoint{2.639829in}{2.037549in}}%
\pgfpathlineto{\pgfqpoint{2.651386in}{2.041175in}}%
\pgfpathlineto{\pgfqpoint{2.662937in}{2.044814in}}%
\pgfpathlineto{\pgfqpoint{2.656780in}{2.054400in}}%
\pgfpathlineto{\pgfqpoint{2.650627in}{2.063942in}}%
\pgfpathlineto{\pgfqpoint{2.644479in}{2.073442in}}%
\pgfpathlineto{\pgfqpoint{2.638335in}{2.082901in}}%
\pgfpathlineto{\pgfqpoint{2.632196in}{2.092319in}}%
\pgfpathlineto{\pgfqpoint{2.620655in}{2.088724in}}%
\pgfpathlineto{\pgfqpoint{2.609108in}{2.085146in}}%
\pgfpathlineto{\pgfqpoint{2.597556in}{2.081581in}}%
\pgfpathlineto{\pgfqpoint{2.585997in}{2.078027in}}%
\pgfpathlineto{\pgfqpoint{2.574433in}{2.074482in}}%
\pgfpathlineto{\pgfqpoint{2.580563in}{2.065016in}}%
\pgfpathlineto{\pgfqpoint{2.586697in}{2.055509in}}%
\pgfpathlineto{\pgfqpoint{2.592835in}{2.045960in}}%
\pgfpathlineto{\pgfqpoint{2.598977in}{2.036368in}}%
\pgfpathclose%
\pgfusepath{stroke,fill}%
\end{pgfscope}%
\begin{pgfscope}%
\pgfpathrectangle{\pgfqpoint{0.887500in}{0.275000in}}{\pgfqpoint{4.225000in}{4.225000in}}%
\pgfusepath{clip}%
\pgfsetbuttcap%
\pgfsetroundjoin%
\definecolor{currentfill}{rgb}{0.214298,0.355619,0.551184}%
\pgfsetfillcolor{currentfill}%
\pgfsetfillopacity{0.700000}%
\pgfsetlinewidth{0.501875pt}%
\definecolor{currentstroke}{rgb}{1.000000,1.000000,1.000000}%
\pgfsetstrokecolor{currentstroke}%
\pgfsetstrokeopacity{0.500000}%
\pgfsetdash{}{0pt}%
\pgfpathmoveto{\pgfqpoint{3.106316in}{1.880792in}}%
\pgfpathlineto{\pgfqpoint{3.117766in}{1.883815in}}%
\pgfpathlineto{\pgfqpoint{3.129212in}{1.887430in}}%
\pgfpathlineto{\pgfqpoint{3.140653in}{1.891908in}}%
\pgfpathlineto{\pgfqpoint{3.152092in}{1.897519in}}%
\pgfpathlineto{\pgfqpoint{3.163531in}{1.904500in}}%
\pgfpathlineto{\pgfqpoint{3.157221in}{1.915471in}}%
\pgfpathlineto{\pgfqpoint{3.150915in}{1.926385in}}%
\pgfpathlineto{\pgfqpoint{3.144611in}{1.937234in}}%
\pgfpathlineto{\pgfqpoint{3.138312in}{1.948010in}}%
\pgfpathlineto{\pgfqpoint{3.132015in}{1.958713in}}%
\pgfpathlineto{\pgfqpoint{3.120583in}{1.948533in}}%
\pgfpathlineto{\pgfqpoint{3.109152in}{1.941066in}}%
\pgfpathlineto{\pgfqpoint{3.097720in}{1.935861in}}%
\pgfpathlineto{\pgfqpoint{3.086284in}{1.932377in}}%
\pgfpathlineto{\pgfqpoint{3.074844in}{1.930077in}}%
\pgfpathlineto{\pgfqpoint{3.081131in}{1.920280in}}%
\pgfpathlineto{\pgfqpoint{3.087421in}{1.910459in}}%
\pgfpathlineto{\pgfqpoint{3.093716in}{1.900609in}}%
\pgfpathlineto{\pgfqpoint{3.100014in}{1.890724in}}%
\pgfpathclose%
\pgfusepath{stroke,fill}%
\end{pgfscope}%
\begin{pgfscope}%
\pgfpathrectangle{\pgfqpoint{0.887500in}{0.275000in}}{\pgfqpoint{4.225000in}{4.225000in}}%
\pgfusepath{clip}%
\pgfsetbuttcap%
\pgfsetroundjoin%
\definecolor{currentfill}{rgb}{0.231674,0.318106,0.544834}%
\pgfsetfillcolor{currentfill}%
\pgfsetfillopacity{0.700000}%
\pgfsetlinewidth{0.501875pt}%
\definecolor{currentstroke}{rgb}{1.000000,1.000000,1.000000}%
\pgfsetstrokecolor{currentstroke}%
\pgfsetstrokeopacity{0.500000}%
\pgfsetdash{}{0pt}%
\pgfpathmoveto{\pgfqpoint{4.426574in}{1.800556in}}%
\pgfpathlineto{\pgfqpoint{4.437676in}{1.803604in}}%
\pgfpathlineto{\pgfqpoint{4.448772in}{1.806641in}}%
\pgfpathlineto{\pgfqpoint{4.459863in}{1.809677in}}%
\pgfpathlineto{\pgfqpoint{4.470948in}{1.812722in}}%
\pgfpathlineto{\pgfqpoint{4.482028in}{1.815786in}}%
\pgfpathlineto{\pgfqpoint{4.475491in}{1.830465in}}%
\pgfpathlineto{\pgfqpoint{4.468957in}{1.845161in}}%
\pgfpathlineto{\pgfqpoint{4.462426in}{1.859866in}}%
\pgfpathlineto{\pgfqpoint{4.455897in}{1.874568in}}%
\pgfpathlineto{\pgfqpoint{4.449370in}{1.889256in}}%
\pgfpathlineto{\pgfqpoint{4.438283in}{1.885981in}}%
\pgfpathlineto{\pgfqpoint{4.427191in}{1.882701in}}%
\pgfpathlineto{\pgfqpoint{4.416092in}{1.879411in}}%
\pgfpathlineto{\pgfqpoint{4.404988in}{1.876106in}}%
\pgfpathlineto{\pgfqpoint{4.393877in}{1.872779in}}%
\pgfpathlineto{\pgfqpoint{4.400415in}{1.858457in}}%
\pgfpathlineto{\pgfqpoint{4.406954in}{1.844079in}}%
\pgfpathlineto{\pgfqpoint{4.413494in}{1.829639in}}%
\pgfpathlineto{\pgfqpoint{4.420033in}{1.815133in}}%
\pgfpathclose%
\pgfusepath{stroke,fill}%
\end{pgfscope}%
\begin{pgfscope}%
\pgfpathrectangle{\pgfqpoint{0.887500in}{0.275000in}}{\pgfqpoint{4.225000in}{4.225000in}}%
\pgfusepath{clip}%
\pgfsetbuttcap%
\pgfsetroundjoin%
\definecolor{currentfill}{rgb}{0.127568,0.566949,0.550556}%
\pgfsetfillcolor{currentfill}%
\pgfsetfillopacity{0.700000}%
\pgfsetlinewidth{0.501875pt}%
\definecolor{currentstroke}{rgb}{1.000000,1.000000,1.000000}%
\pgfsetstrokecolor{currentstroke}%
\pgfsetstrokeopacity{0.500000}%
\pgfsetdash{}{0pt}%
\pgfpathmoveto{\pgfqpoint{3.742294in}{2.312264in}}%
\pgfpathlineto{\pgfqpoint{3.753609in}{2.317155in}}%
\pgfpathlineto{\pgfqpoint{3.764916in}{2.321874in}}%
\pgfpathlineto{\pgfqpoint{3.776217in}{2.326528in}}%
\pgfpathlineto{\pgfqpoint{3.787513in}{2.331143in}}%
\pgfpathlineto{\pgfqpoint{3.798803in}{2.335702in}}%
\pgfpathlineto{\pgfqpoint{3.792399in}{2.350826in}}%
\pgfpathlineto{\pgfqpoint{3.785995in}{2.365809in}}%
\pgfpathlineto{\pgfqpoint{3.779590in}{2.380623in}}%
\pgfpathlineto{\pgfqpoint{3.773184in}{2.395268in}}%
\pgfpathlineto{\pgfqpoint{3.766779in}{2.409763in}}%
\pgfpathlineto{\pgfqpoint{3.755493in}{2.405309in}}%
\pgfpathlineto{\pgfqpoint{3.744200in}{2.400700in}}%
\pgfpathlineto{\pgfqpoint{3.732900in}{2.395929in}}%
\pgfpathlineto{\pgfqpoint{3.721591in}{2.390979in}}%
\pgfpathlineto{\pgfqpoint{3.710275in}{2.385813in}}%
\pgfpathlineto{\pgfqpoint{3.716674in}{2.371142in}}%
\pgfpathlineto{\pgfqpoint{3.723076in}{2.356452in}}%
\pgfpathlineto{\pgfqpoint{3.729480in}{2.341745in}}%
\pgfpathlineto{\pgfqpoint{3.735886in}{2.327019in}}%
\pgfpathclose%
\pgfusepath{stroke,fill}%
\end{pgfscope}%
\begin{pgfscope}%
\pgfpathrectangle{\pgfqpoint{0.887500in}{0.275000in}}{\pgfqpoint{4.225000in}{4.225000in}}%
\pgfusepath{clip}%
\pgfsetbuttcap%
\pgfsetroundjoin%
\definecolor{currentfill}{rgb}{0.169646,0.456262,0.558030}%
\pgfsetfillcolor{currentfill}%
\pgfsetfillopacity{0.700000}%
\pgfsetlinewidth{0.501875pt}%
\definecolor{currentstroke}{rgb}{1.000000,1.000000,1.000000}%
\pgfsetstrokecolor{currentstroke}%
\pgfsetstrokeopacity{0.500000}%
\pgfsetdash{}{0pt}%
\pgfpathmoveto{\pgfqpoint{3.278009in}{2.026342in}}%
\pgfpathlineto{\pgfqpoint{3.289477in}{2.042423in}}%
\pgfpathlineto{\pgfqpoint{3.300952in}{2.059444in}}%
\pgfpathlineto{\pgfqpoint{3.312436in}{2.077506in}}%
\pgfpathlineto{\pgfqpoint{3.323930in}{2.096620in}}%
\pgfpathlineto{\pgfqpoint{3.335433in}{2.116511in}}%
\pgfpathlineto{\pgfqpoint{3.329123in}{2.134529in}}%
\pgfpathlineto{\pgfqpoint{3.322811in}{2.152227in}}%
\pgfpathlineto{\pgfqpoint{3.316499in}{2.169505in}}%
\pgfpathlineto{\pgfqpoint{3.310184in}{2.186309in}}%
\pgfpathlineto{\pgfqpoint{3.303869in}{2.202699in}}%
\pgfpathlineto{\pgfqpoint{3.292399in}{2.186257in}}%
\pgfpathlineto{\pgfqpoint{3.280931in}{2.169974in}}%
\pgfpathlineto{\pgfqpoint{3.269466in}{2.153847in}}%
\pgfpathlineto{\pgfqpoint{3.258003in}{2.137793in}}%
\pgfpathlineto{\pgfqpoint{3.246542in}{2.121707in}}%
\pgfpathlineto{\pgfqpoint{3.252838in}{2.103464in}}%
\pgfpathlineto{\pgfqpoint{3.259132in}{2.084747in}}%
\pgfpathlineto{\pgfqpoint{3.265425in}{2.065573in}}%
\pgfpathlineto{\pgfqpoint{3.271717in}{2.046054in}}%
\pgfpathclose%
\pgfusepath{stroke,fill}%
\end{pgfscope}%
\begin{pgfscope}%
\pgfpathrectangle{\pgfqpoint{0.887500in}{0.275000in}}{\pgfqpoint{4.225000in}{4.225000in}}%
\pgfusepath{clip}%
\pgfsetbuttcap%
\pgfsetroundjoin%
\definecolor{currentfill}{rgb}{0.168126,0.459988,0.558082}%
\pgfsetfillcolor{currentfill}%
\pgfsetfillopacity{0.700000}%
\pgfsetlinewidth{0.501875pt}%
\definecolor{currentstroke}{rgb}{1.000000,1.000000,1.000000}%
\pgfsetstrokecolor{currentstroke}%
\pgfsetstrokeopacity{0.500000}%
\pgfsetdash{}{0pt}%
\pgfpathmoveto{\pgfqpoint{2.281334in}{2.095976in}}%
\pgfpathlineto{\pgfqpoint{2.292989in}{2.099625in}}%
\pgfpathlineto{\pgfqpoint{2.304638in}{2.103277in}}%
\pgfpathlineto{\pgfqpoint{2.316281in}{2.106930in}}%
\pgfpathlineto{\pgfqpoint{2.327919in}{2.110581in}}%
\pgfpathlineto{\pgfqpoint{2.339551in}{2.114226in}}%
\pgfpathlineto{\pgfqpoint{2.333500in}{2.123445in}}%
\pgfpathlineto{\pgfqpoint{2.327453in}{2.132633in}}%
\pgfpathlineto{\pgfqpoint{2.321411in}{2.141789in}}%
\pgfpathlineto{\pgfqpoint{2.315373in}{2.150914in}}%
\pgfpathlineto{\pgfqpoint{2.309340in}{2.160009in}}%
\pgfpathlineto{\pgfqpoint{2.297718in}{2.156414in}}%
\pgfpathlineto{\pgfqpoint{2.286091in}{2.152813in}}%
\pgfpathlineto{\pgfqpoint{2.274459in}{2.149211in}}%
\pgfpathlineto{\pgfqpoint{2.262820in}{2.145609in}}%
\pgfpathlineto{\pgfqpoint{2.251176in}{2.142012in}}%
\pgfpathlineto{\pgfqpoint{2.257199in}{2.132867in}}%
\pgfpathlineto{\pgfqpoint{2.263226in}{2.123691in}}%
\pgfpathlineto{\pgfqpoint{2.269258in}{2.114485in}}%
\pgfpathlineto{\pgfqpoint{2.275294in}{2.105247in}}%
\pgfpathclose%
\pgfusepath{stroke,fill}%
\end{pgfscope}%
\begin{pgfscope}%
\pgfpathrectangle{\pgfqpoint{0.887500in}{0.275000in}}{\pgfqpoint{4.225000in}{4.225000in}}%
\pgfusepath{clip}%
\pgfsetbuttcap%
\pgfsetroundjoin%
\definecolor{currentfill}{rgb}{0.121380,0.629492,0.531973}%
\pgfsetfillcolor{currentfill}%
\pgfsetfillopacity{0.700000}%
\pgfsetlinewidth{0.501875pt}%
\definecolor{currentstroke}{rgb}{1.000000,1.000000,1.000000}%
\pgfsetstrokecolor{currentstroke}%
\pgfsetstrokeopacity{0.500000}%
\pgfsetdash{}{0pt}%
\pgfpathmoveto{\pgfqpoint{3.475984in}{2.433985in}}%
\pgfpathlineto{\pgfqpoint{3.487411in}{2.444091in}}%
\pgfpathlineto{\pgfqpoint{3.498827in}{2.453447in}}%
\pgfpathlineto{\pgfqpoint{3.510234in}{2.462130in}}%
\pgfpathlineto{\pgfqpoint{3.521631in}{2.470218in}}%
\pgfpathlineto{\pgfqpoint{3.533018in}{2.477768in}}%
\pgfpathlineto{\pgfqpoint{3.526650in}{2.492300in}}%
\pgfpathlineto{\pgfqpoint{3.520281in}{2.506578in}}%
\pgfpathlineto{\pgfqpoint{3.513913in}{2.520669in}}%
\pgfpathlineto{\pgfqpoint{3.507546in}{2.534646in}}%
\pgfpathlineto{\pgfqpoint{3.501181in}{2.548577in}}%
\pgfpathlineto{\pgfqpoint{3.489797in}{2.540802in}}%
\pgfpathlineto{\pgfqpoint{3.478405in}{2.532541in}}%
\pgfpathlineto{\pgfqpoint{3.467004in}{2.523731in}}%
\pgfpathlineto{\pgfqpoint{3.455594in}{2.514259in}}%
\pgfpathlineto{\pgfqpoint{3.444175in}{2.504011in}}%
\pgfpathlineto{\pgfqpoint{3.450534in}{2.490213in}}%
\pgfpathlineto{\pgfqpoint{3.456895in}{2.476426in}}%
\pgfpathlineto{\pgfqpoint{3.463258in}{2.462534in}}%
\pgfpathlineto{\pgfqpoint{3.469622in}{2.448425in}}%
\pgfpathclose%
\pgfusepath{stroke,fill}%
\end{pgfscope}%
\begin{pgfscope}%
\pgfpathrectangle{\pgfqpoint{0.887500in}{0.275000in}}{\pgfqpoint{4.225000in}{4.225000in}}%
\pgfusepath{clip}%
\pgfsetbuttcap%
\pgfsetroundjoin%
\definecolor{currentfill}{rgb}{0.156270,0.489624,0.557936}%
\pgfsetfillcolor{currentfill}%
\pgfsetfillopacity{0.700000}%
\pgfsetlinewidth{0.501875pt}%
\definecolor{currentstroke}{rgb}{1.000000,1.000000,1.000000}%
\pgfsetstrokecolor{currentstroke}%
\pgfsetstrokeopacity{0.500000}%
\pgfsetdash{}{0pt}%
\pgfpathmoveto{\pgfqpoint{1.957529in}{2.161516in}}%
\pgfpathlineto{\pgfqpoint{1.969265in}{2.165109in}}%
\pgfpathlineto{\pgfqpoint{1.980996in}{2.168698in}}%
\pgfpathlineto{\pgfqpoint{1.992721in}{2.172281in}}%
\pgfpathlineto{\pgfqpoint{2.004440in}{2.175858in}}%
\pgfpathlineto{\pgfqpoint{2.016153in}{2.179425in}}%
\pgfpathlineto{\pgfqpoint{2.010211in}{2.188412in}}%
\pgfpathlineto{\pgfqpoint{2.004273in}{2.197373in}}%
\pgfpathlineto{\pgfqpoint{1.998339in}{2.206307in}}%
\pgfpathlineto{\pgfqpoint{1.992410in}{2.215216in}}%
\pgfpathlineto{\pgfqpoint{1.986485in}{2.224099in}}%
\pgfpathlineto{\pgfqpoint{1.974783in}{2.220581in}}%
\pgfpathlineto{\pgfqpoint{1.963074in}{2.217053in}}%
\pgfpathlineto{\pgfqpoint{1.951361in}{2.213518in}}%
\pgfpathlineto{\pgfqpoint{1.939641in}{2.209977in}}%
\pgfpathlineto{\pgfqpoint{1.927916in}{2.206432in}}%
\pgfpathlineto{\pgfqpoint{1.933830in}{2.197503in}}%
\pgfpathlineto{\pgfqpoint{1.939748in}{2.188547in}}%
\pgfpathlineto{\pgfqpoint{1.945670in}{2.179564in}}%
\pgfpathlineto{\pgfqpoint{1.951598in}{2.170554in}}%
\pgfpathclose%
\pgfusepath{stroke,fill}%
\end{pgfscope}%
\begin{pgfscope}%
\pgfpathrectangle{\pgfqpoint{0.887500in}{0.275000in}}{\pgfqpoint{4.225000in}{4.225000in}}%
\pgfusepath{clip}%
\pgfsetbuttcap%
\pgfsetroundjoin%
\definecolor{currentfill}{rgb}{0.121148,0.592739,0.544641}%
\pgfsetfillcolor{currentfill}%
\pgfsetfillopacity{0.700000}%
\pgfsetlinewidth{0.501875pt}%
\definecolor{currentstroke}{rgb}{1.000000,1.000000,1.000000}%
\pgfsetstrokecolor{currentstroke}%
\pgfsetstrokeopacity{0.500000}%
\pgfsetdash{}{0pt}%
\pgfpathmoveto{\pgfqpoint{3.653553in}{2.355426in}}%
\pgfpathlineto{\pgfqpoint{3.664917in}{2.362239in}}%
\pgfpathlineto{\pgfqpoint{3.676271in}{2.368646in}}%
\pgfpathlineto{\pgfqpoint{3.687615in}{2.374685in}}%
\pgfpathlineto{\pgfqpoint{3.698949in}{2.380395in}}%
\pgfpathlineto{\pgfqpoint{3.710275in}{2.385813in}}%
\pgfpathlineto{\pgfqpoint{3.703877in}{2.400466in}}%
\pgfpathlineto{\pgfqpoint{3.697481in}{2.415097in}}%
\pgfpathlineto{\pgfqpoint{3.691088in}{2.429706in}}%
\pgfpathlineto{\pgfqpoint{3.684696in}{2.444292in}}%
\pgfpathlineto{\pgfqpoint{3.678307in}{2.458853in}}%
\pgfpathlineto{\pgfqpoint{3.666995in}{2.454098in}}%
\pgfpathlineto{\pgfqpoint{3.655676in}{2.449158in}}%
\pgfpathlineto{\pgfqpoint{3.644349in}{2.444007in}}%
\pgfpathlineto{\pgfqpoint{3.633014in}{2.438616in}}%
\pgfpathlineto{\pgfqpoint{3.621670in}{2.432957in}}%
\pgfpathlineto{\pgfqpoint{3.628043in}{2.417435in}}%
\pgfpathlineto{\pgfqpoint{3.634416in}{2.401878in}}%
\pgfpathlineto{\pgfqpoint{3.640792in}{2.386328in}}%
\pgfpathlineto{\pgfqpoint{3.647171in}{2.370830in}}%
\pgfpathclose%
\pgfusepath{stroke,fill}%
\end{pgfscope}%
\begin{pgfscope}%
\pgfpathrectangle{\pgfqpoint{0.887500in}{0.275000in}}{\pgfqpoint{4.225000in}{4.225000in}}%
\pgfusepath{clip}%
\pgfsetbuttcap%
\pgfsetroundjoin%
\definecolor{currentfill}{rgb}{0.206756,0.371758,0.553117}%
\pgfsetfillcolor{currentfill}%
\pgfsetfillopacity{0.700000}%
\pgfsetlinewidth{0.501875pt}%
\definecolor{currentstroke}{rgb}{1.000000,1.000000,1.000000}%
\pgfsetstrokecolor{currentstroke}%
\pgfsetstrokeopacity{0.500000}%
\pgfsetdash{}{0pt}%
\pgfpathmoveto{\pgfqpoint{3.252281in}{1.875952in}}%
\pgfpathlineto{\pgfqpoint{3.263711in}{1.884000in}}%
\pgfpathlineto{\pgfqpoint{3.275146in}{1.893346in}}%
\pgfpathlineto{\pgfqpoint{3.286586in}{1.904097in}}%
\pgfpathlineto{\pgfqpoint{3.298035in}{1.916341in}}%
\pgfpathlineto{\pgfqpoint{3.309493in}{1.930165in}}%
\pgfpathlineto{\pgfqpoint{3.303189in}{1.948581in}}%
\pgfpathlineto{\pgfqpoint{3.296890in}{1.967557in}}%
\pgfpathlineto{\pgfqpoint{3.290595in}{1.986943in}}%
\pgfpathlineto{\pgfqpoint{3.284301in}{2.006588in}}%
\pgfpathlineto{\pgfqpoint{3.278009in}{2.026342in}}%
\pgfpathlineto{\pgfqpoint{3.266548in}{2.011101in}}%
\pgfpathlineto{\pgfqpoint{3.255091in}{1.996601in}}%
\pgfpathlineto{\pgfqpoint{3.243638in}{1.982740in}}%
\pgfpathlineto{\pgfqpoint{3.232188in}{1.969421in}}%
\pgfpathlineto{\pgfqpoint{3.220741in}{1.956594in}}%
\pgfpathlineto{\pgfqpoint{3.227043in}{1.940064in}}%
\pgfpathlineto{\pgfqpoint{3.233348in}{1.923631in}}%
\pgfpathlineto{\pgfqpoint{3.239655in}{1.907400in}}%
\pgfpathlineto{\pgfqpoint{3.245966in}{1.891472in}}%
\pgfpathclose%
\pgfusepath{stroke,fill}%
\end{pgfscope}%
\begin{pgfscope}%
\pgfpathrectangle{\pgfqpoint{0.887500in}{0.275000in}}{\pgfqpoint{4.225000in}{4.225000in}}%
\pgfusepath{clip}%
\pgfsetbuttcap%
\pgfsetroundjoin%
\definecolor{currentfill}{rgb}{0.126453,0.570633,0.549841}%
\pgfsetfillcolor{currentfill}%
\pgfsetfillopacity{0.700000}%
\pgfsetlinewidth{0.501875pt}%
\definecolor{currentstroke}{rgb}{1.000000,1.000000,1.000000}%
\pgfsetstrokecolor{currentstroke}%
\pgfsetstrokeopacity{0.500000}%
\pgfsetdash{}{0pt}%
\pgfpathmoveto{\pgfqpoint{3.361264in}{2.286920in}}%
\pgfpathlineto{\pgfqpoint{3.372752in}{2.304040in}}%
\pgfpathlineto{\pgfqpoint{3.384241in}{2.321048in}}%
\pgfpathlineto{\pgfqpoint{3.395729in}{2.337772in}}%
\pgfpathlineto{\pgfqpoint{3.407216in}{2.354041in}}%
\pgfpathlineto{\pgfqpoint{3.418698in}{2.369682in}}%
\pgfpathlineto{\pgfqpoint{3.412339in}{2.383337in}}%
\pgfpathlineto{\pgfqpoint{3.405980in}{2.396792in}}%
\pgfpathlineto{\pgfqpoint{3.399624in}{2.410198in}}%
\pgfpathlineto{\pgfqpoint{3.393272in}{2.423708in}}%
\pgfpathlineto{\pgfqpoint{3.386924in}{2.437475in}}%
\pgfpathlineto{\pgfqpoint{3.375450in}{2.421178in}}%
\pgfpathlineto{\pgfqpoint{3.363974in}{2.404422in}}%
\pgfpathlineto{\pgfqpoint{3.352499in}{2.387511in}}%
\pgfpathlineto{\pgfqpoint{3.341028in}{2.370747in}}%
\pgfpathlineto{\pgfqpoint{3.329561in}{2.354433in}}%
\pgfpathlineto{\pgfqpoint{3.335892in}{2.340305in}}%
\pgfpathlineto{\pgfqpoint{3.342229in}{2.326651in}}%
\pgfpathlineto{\pgfqpoint{3.348570in}{2.313310in}}%
\pgfpathlineto{\pgfqpoint{3.354916in}{2.300121in}}%
\pgfpathclose%
\pgfusepath{stroke,fill}%
\end{pgfscope}%
\begin{pgfscope}%
\pgfpathrectangle{\pgfqpoint{0.887500in}{0.275000in}}{\pgfqpoint{4.225000in}{4.225000in}}%
\pgfusepath{clip}%
\pgfsetbuttcap%
\pgfsetroundjoin%
\definecolor{currentfill}{rgb}{0.166617,0.463708,0.558119}%
\pgfsetfillcolor{currentfill}%
\pgfsetfillopacity{0.700000}%
\pgfsetlinewidth{0.501875pt}%
\definecolor{currentstroke}{rgb}{1.000000,1.000000,1.000000}%
\pgfsetstrokecolor{currentstroke}%
\pgfsetstrokeopacity{0.500000}%
\pgfsetdash{}{0pt}%
\pgfpathmoveto{\pgfqpoint{4.040161in}{2.082813in}}%
\pgfpathlineto{\pgfqpoint{4.051396in}{2.087007in}}%
\pgfpathlineto{\pgfqpoint{4.062622in}{2.091035in}}%
\pgfpathlineto{\pgfqpoint{4.073839in}{2.094920in}}%
\pgfpathlineto{\pgfqpoint{4.085048in}{2.098689in}}%
\pgfpathlineto{\pgfqpoint{4.096249in}{2.102367in}}%
\pgfpathlineto{\pgfqpoint{4.089809in}{2.118097in}}%
\pgfpathlineto{\pgfqpoint{4.083364in}{2.133543in}}%
\pgfpathlineto{\pgfqpoint{4.076914in}{2.148688in}}%
\pgfpathlineto{\pgfqpoint{4.070459in}{2.163570in}}%
\pgfpathlineto{\pgfqpoint{4.064001in}{2.178228in}}%
\pgfpathlineto{\pgfqpoint{4.052808in}{2.174799in}}%
\pgfpathlineto{\pgfqpoint{4.041606in}{2.171261in}}%
\pgfpathlineto{\pgfqpoint{4.030395in}{2.167582in}}%
\pgfpathlineto{\pgfqpoint{4.019175in}{2.163730in}}%
\pgfpathlineto{\pgfqpoint{4.007944in}{2.159681in}}%
\pgfpathlineto{\pgfqpoint{4.014398in}{2.144938in}}%
\pgfpathlineto{\pgfqpoint{4.020848in}{2.129923in}}%
\pgfpathlineto{\pgfqpoint{4.027293in}{2.114585in}}%
\pgfpathlineto{\pgfqpoint{4.033730in}{2.098876in}}%
\pgfpathclose%
\pgfusepath{stroke,fill}%
\end{pgfscope}%
\begin{pgfscope}%
\pgfpathrectangle{\pgfqpoint{0.887500in}{0.275000in}}{\pgfqpoint{4.225000in}{4.225000in}}%
\pgfusepath{clip}%
\pgfsetbuttcap%
\pgfsetroundjoin%
\definecolor{currentfill}{rgb}{0.187231,0.414746,0.556547}%
\pgfsetfillcolor{currentfill}%
\pgfsetfillopacity{0.700000}%
\pgfsetlinewidth{0.501875pt}%
\definecolor{currentstroke}{rgb}{1.000000,1.000000,1.000000}%
\pgfsetstrokecolor{currentstroke}%
\pgfsetstrokeopacity{0.500000}%
\pgfsetdash{}{0pt}%
\pgfpathmoveto{\pgfqpoint{2.693783in}{1.996205in}}%
\pgfpathlineto{\pgfqpoint{2.705338in}{1.999908in}}%
\pgfpathlineto{\pgfqpoint{2.716887in}{2.003628in}}%
\pgfpathlineto{\pgfqpoint{2.728431in}{2.007358in}}%
\pgfpathlineto{\pgfqpoint{2.739969in}{2.011077in}}%
\pgfpathlineto{\pgfqpoint{2.751501in}{2.014764in}}%
\pgfpathlineto{\pgfqpoint{2.745314in}{2.024529in}}%
\pgfpathlineto{\pgfqpoint{2.739131in}{2.034250in}}%
\pgfpathlineto{\pgfqpoint{2.732951in}{2.043927in}}%
\pgfpathlineto{\pgfqpoint{2.726776in}{2.053560in}}%
\pgfpathlineto{\pgfqpoint{2.720605in}{2.063149in}}%
\pgfpathlineto{\pgfqpoint{2.709083in}{2.059506in}}%
\pgfpathlineto{\pgfqpoint{2.697555in}{2.055830in}}%
\pgfpathlineto{\pgfqpoint{2.686021in}{2.052143in}}%
\pgfpathlineto{\pgfqpoint{2.674482in}{2.048470in}}%
\pgfpathlineto{\pgfqpoint{2.662937in}{2.044814in}}%
\pgfpathlineto{\pgfqpoint{2.669097in}{2.035184in}}%
\pgfpathlineto{\pgfqpoint{2.675262in}{2.025508in}}%
\pgfpathlineto{\pgfqpoint{2.681432in}{2.015786in}}%
\pgfpathlineto{\pgfqpoint{2.687605in}{2.006019in}}%
\pgfpathclose%
\pgfusepath{stroke,fill}%
\end{pgfscope}%
\begin{pgfscope}%
\pgfpathrectangle{\pgfqpoint{0.887500in}{0.275000in}}{\pgfqpoint{4.225000in}{4.225000in}}%
\pgfusepath{clip}%
\pgfsetbuttcap%
\pgfsetroundjoin%
\definecolor{currentfill}{rgb}{0.119483,0.614817,0.537692}%
\pgfsetfillcolor{currentfill}%
\pgfsetfillopacity{0.700000}%
\pgfsetlinewidth{0.501875pt}%
\definecolor{currentstroke}{rgb}{1.000000,1.000000,1.000000}%
\pgfsetstrokecolor{currentstroke}%
\pgfsetstrokeopacity{0.500000}%
\pgfsetdash{}{0pt}%
\pgfpathmoveto{\pgfqpoint{3.564821in}{2.399384in}}%
\pgfpathlineto{\pgfqpoint{3.576209in}{2.406934in}}%
\pgfpathlineto{\pgfqpoint{3.587588in}{2.414027in}}%
\pgfpathlineto{\pgfqpoint{3.598958in}{2.420703in}}%
\pgfpathlineto{\pgfqpoint{3.610319in}{2.426999in}}%
\pgfpathlineto{\pgfqpoint{3.621670in}{2.432957in}}%
\pgfpathlineto{\pgfqpoint{3.615299in}{2.448400in}}%
\pgfpathlineto{\pgfqpoint{3.608928in}{2.463720in}}%
\pgfpathlineto{\pgfqpoint{3.602557in}{2.478872in}}%
\pgfpathlineto{\pgfqpoint{3.596186in}{2.493830in}}%
\pgfpathlineto{\pgfqpoint{3.589814in}{2.508608in}}%
\pgfpathlineto{\pgfqpoint{3.578474in}{2.503250in}}%
\pgfpathlineto{\pgfqpoint{3.567124in}{2.497520in}}%
\pgfpathlineto{\pgfqpoint{3.555765in}{2.491385in}}%
\pgfpathlineto{\pgfqpoint{3.544396in}{2.484812in}}%
\pgfpathlineto{\pgfqpoint{3.533018in}{2.477768in}}%
\pgfpathlineto{\pgfqpoint{3.539385in}{2.462909in}}%
\pgfpathlineto{\pgfqpoint{3.545749in}{2.447655in}}%
\pgfpathlineto{\pgfqpoint{3.552109in}{2.431950in}}%
\pgfpathlineto{\pgfqpoint{3.558466in}{2.415834in}}%
\pgfpathclose%
\pgfusepath{stroke,fill}%
\end{pgfscope}%
\begin{pgfscope}%
\pgfpathrectangle{\pgfqpoint{0.887500in}{0.275000in}}{\pgfqpoint{4.225000in}{4.225000in}}%
\pgfusepath{clip}%
\pgfsetbuttcap%
\pgfsetroundjoin%
\definecolor{currentfill}{rgb}{0.216210,0.351535,0.550627}%
\pgfsetfillcolor{currentfill}%
\pgfsetfillopacity{0.700000}%
\pgfsetlinewidth{0.501875pt}%
\definecolor{currentstroke}{rgb}{1.000000,1.000000,1.000000}%
\pgfsetstrokecolor{currentstroke}%
\pgfsetstrokeopacity{0.500000}%
\pgfsetdash{}{0pt}%
\pgfpathmoveto{\pgfqpoint{4.338216in}{1.855484in}}%
\pgfpathlineto{\pgfqpoint{4.349363in}{1.859062in}}%
\pgfpathlineto{\pgfqpoint{4.360502in}{1.862571in}}%
\pgfpathlineto{\pgfqpoint{4.371634in}{1.866020in}}%
\pgfpathlineto{\pgfqpoint{4.382759in}{1.869420in}}%
\pgfpathlineto{\pgfqpoint{4.393877in}{1.872779in}}%
\pgfpathlineto{\pgfqpoint{4.387339in}{1.887048in}}%
\pgfpathlineto{\pgfqpoint{4.380803in}{1.901269in}}%
\pgfpathlineto{\pgfqpoint{4.374268in}{1.915447in}}%
\pgfpathlineto{\pgfqpoint{4.367734in}{1.929587in}}%
\pgfpathlineto{\pgfqpoint{4.361202in}{1.943693in}}%
\pgfpathlineto{\pgfqpoint{4.350070in}{1.939841in}}%
\pgfpathlineto{\pgfqpoint{4.338930in}{1.935925in}}%
\pgfpathlineto{\pgfqpoint{4.327783in}{1.931940in}}%
\pgfpathlineto{\pgfqpoint{4.316628in}{1.927881in}}%
\pgfpathlineto{\pgfqpoint{4.305466in}{1.923743in}}%
\pgfpathlineto{\pgfqpoint{4.312012in}{1.910139in}}%
\pgfpathlineto{\pgfqpoint{4.318560in}{1.896534in}}%
\pgfpathlineto{\pgfqpoint{4.325111in}{1.882905in}}%
\pgfpathlineto{\pgfqpoint{4.331663in}{1.869229in}}%
\pgfpathclose%
\pgfusepath{stroke,fill}%
\end{pgfscope}%
\begin{pgfscope}%
\pgfpathrectangle{\pgfqpoint{0.887500in}{0.275000in}}{\pgfqpoint{4.225000in}{4.225000in}}%
\pgfusepath{clip}%
\pgfsetbuttcap%
\pgfsetroundjoin%
\definecolor{currentfill}{rgb}{0.119512,0.607464,0.540218}%
\pgfsetfillcolor{currentfill}%
\pgfsetfillopacity{0.700000}%
\pgfsetlinewidth{0.501875pt}%
\definecolor{currentstroke}{rgb}{1.000000,1.000000,1.000000}%
\pgfsetstrokecolor{currentstroke}%
\pgfsetstrokeopacity{0.500000}%
\pgfsetdash{}{0pt}%
\pgfpathmoveto{\pgfqpoint{3.418698in}{2.369682in}}%
\pgfpathlineto{\pgfqpoint{3.430174in}{2.384522in}}%
\pgfpathlineto{\pgfqpoint{3.441642in}{2.398392in}}%
\pgfpathlineto{\pgfqpoint{3.453100in}{2.411213in}}%
\pgfpathlineto{\pgfqpoint{3.464547in}{2.423052in}}%
\pgfpathlineto{\pgfqpoint{3.475984in}{2.433985in}}%
\pgfpathlineto{\pgfqpoint{3.469622in}{2.448425in}}%
\pgfpathlineto{\pgfqpoint{3.463258in}{2.462534in}}%
\pgfpathlineto{\pgfqpoint{3.456895in}{2.476426in}}%
\pgfpathlineto{\pgfqpoint{3.450534in}{2.490213in}}%
\pgfpathlineto{\pgfqpoint{3.444175in}{2.504011in}}%
\pgfpathlineto{\pgfqpoint{3.432745in}{2.492873in}}%
\pgfpathlineto{\pgfqpoint{3.421305in}{2.480733in}}%
\pgfpathlineto{\pgfqpoint{3.409854in}{2.467475in}}%
\pgfpathlineto{\pgfqpoint{3.398393in}{2.453008in}}%
\pgfpathlineto{\pgfqpoint{3.386924in}{2.437475in}}%
\pgfpathlineto{\pgfqpoint{3.393272in}{2.423708in}}%
\pgfpathlineto{\pgfqpoint{3.399624in}{2.410198in}}%
\pgfpathlineto{\pgfqpoint{3.405980in}{2.396792in}}%
\pgfpathlineto{\pgfqpoint{3.412339in}{2.383337in}}%
\pgfpathclose%
\pgfusepath{stroke,fill}%
\end{pgfscope}%
\begin{pgfscope}%
\pgfpathrectangle{\pgfqpoint{0.887500in}{0.275000in}}{\pgfqpoint{4.225000in}{4.225000in}}%
\pgfusepath{clip}%
\pgfsetbuttcap%
\pgfsetroundjoin%
\definecolor{currentfill}{rgb}{0.188923,0.410910,0.556326}%
\pgfsetfillcolor{currentfill}%
\pgfsetfillopacity{0.700000}%
\pgfsetlinewidth{0.501875pt}%
\definecolor{currentstroke}{rgb}{1.000000,1.000000,1.000000}%
\pgfsetstrokecolor{currentstroke}%
\pgfsetstrokeopacity{0.500000}%
\pgfsetdash{}{0pt}%
\pgfpathmoveto{\pgfqpoint{3.309493in}{1.930165in}}%
\pgfpathlineto{\pgfqpoint{3.320963in}{1.945657in}}%
\pgfpathlineto{\pgfqpoint{3.332446in}{1.962906in}}%
\pgfpathlineto{\pgfqpoint{3.343945in}{1.982001in}}%
\pgfpathlineto{\pgfqpoint{3.355461in}{2.002892in}}%
\pgfpathlineto{\pgfqpoint{3.366992in}{2.025072in}}%
\pgfpathlineto{\pgfqpoint{3.360677in}{2.043210in}}%
\pgfpathlineto{\pgfqpoint{3.354364in}{2.061522in}}%
\pgfpathlineto{\pgfqpoint{3.348053in}{2.079908in}}%
\pgfpathlineto{\pgfqpoint{3.341743in}{2.098271in}}%
\pgfpathlineto{\pgfqpoint{3.335433in}{2.116511in}}%
\pgfpathlineto{\pgfqpoint{3.323930in}{2.096620in}}%
\pgfpathlineto{\pgfqpoint{3.312436in}{2.077506in}}%
\pgfpathlineto{\pgfqpoint{3.300952in}{2.059444in}}%
\pgfpathlineto{\pgfqpoint{3.289477in}{2.042423in}}%
\pgfpathlineto{\pgfqpoint{3.278009in}{2.026342in}}%
\pgfpathlineto{\pgfqpoint{3.284301in}{2.006588in}}%
\pgfpathlineto{\pgfqpoint{3.290595in}{1.986943in}}%
\pgfpathlineto{\pgfqpoint{3.296890in}{1.967557in}}%
\pgfpathlineto{\pgfqpoint{3.303189in}{1.948581in}}%
\pgfpathclose%
\pgfusepath{stroke,fill}%
\end{pgfscope}%
\begin{pgfscope}%
\pgfpathrectangle{\pgfqpoint{0.887500in}{0.275000in}}{\pgfqpoint{4.225000in}{4.225000in}}%
\pgfusepath{clip}%
\pgfsetbuttcap%
\pgfsetroundjoin%
\definecolor{currentfill}{rgb}{0.172719,0.448791,0.557885}%
\pgfsetfillcolor{currentfill}%
\pgfsetfillopacity{0.700000}%
\pgfsetlinewidth{0.501875pt}%
\definecolor{currentstroke}{rgb}{1.000000,1.000000,1.000000}%
\pgfsetstrokecolor{currentstroke}%
\pgfsetstrokeopacity{0.500000}%
\pgfsetdash{}{0pt}%
\pgfpathmoveto{\pgfqpoint{2.369871in}{2.067623in}}%
\pgfpathlineto{\pgfqpoint{2.381508in}{2.071312in}}%
\pgfpathlineto{\pgfqpoint{2.393139in}{2.074989in}}%
\pgfpathlineto{\pgfqpoint{2.404765in}{2.078651in}}%
\pgfpathlineto{\pgfqpoint{2.416385in}{2.082299in}}%
\pgfpathlineto{\pgfqpoint{2.428000in}{2.085928in}}%
\pgfpathlineto{\pgfqpoint{2.421917in}{2.095267in}}%
\pgfpathlineto{\pgfqpoint{2.415838in}{2.104570in}}%
\pgfpathlineto{\pgfqpoint{2.409764in}{2.113839in}}%
\pgfpathlineto{\pgfqpoint{2.403694in}{2.123073in}}%
\pgfpathlineto{\pgfqpoint{2.397628in}{2.132274in}}%
\pgfpathlineto{\pgfqpoint{2.386023in}{2.128695in}}%
\pgfpathlineto{\pgfqpoint{2.374413in}{2.125099in}}%
\pgfpathlineto{\pgfqpoint{2.362798in}{2.121487in}}%
\pgfpathlineto{\pgfqpoint{2.351177in}{2.117862in}}%
\pgfpathlineto{\pgfqpoint{2.339551in}{2.114226in}}%
\pgfpathlineto{\pgfqpoint{2.345606in}{2.104973in}}%
\pgfpathlineto{\pgfqpoint{2.351666in}{2.095688in}}%
\pgfpathlineto{\pgfqpoint{2.357730in}{2.086368in}}%
\pgfpathlineto{\pgfqpoint{2.363798in}{2.077013in}}%
\pgfpathclose%
\pgfusepath{stroke,fill}%
\end{pgfscope}%
\begin{pgfscope}%
\pgfpathrectangle{\pgfqpoint{0.887500in}{0.275000in}}{\pgfqpoint{4.225000in}{4.225000in}}%
\pgfusepath{clip}%
\pgfsetbuttcap%
\pgfsetroundjoin%
\definecolor{currentfill}{rgb}{0.204903,0.375746,0.553533}%
\pgfsetfillcolor{currentfill}%
\pgfsetfillopacity{0.700000}%
\pgfsetlinewidth{0.501875pt}%
\definecolor{currentstroke}{rgb}{1.000000,1.000000,1.000000}%
\pgfsetstrokecolor{currentstroke}%
\pgfsetstrokeopacity{0.500000}%
\pgfsetdash{}{0pt}%
\pgfpathmoveto{\pgfqpoint{4.249557in}{1.902379in}}%
\pgfpathlineto{\pgfqpoint{4.260748in}{1.906635in}}%
\pgfpathlineto{\pgfqpoint{4.271936in}{1.910935in}}%
\pgfpathlineto{\pgfqpoint{4.283119in}{1.915242in}}%
\pgfpathlineto{\pgfqpoint{4.294296in}{1.919523in}}%
\pgfpathlineto{\pgfqpoint{4.305466in}{1.923743in}}%
\pgfpathlineto{\pgfqpoint{4.298924in}{1.937368in}}%
\pgfpathlineto{\pgfqpoint{4.292386in}{1.951038in}}%
\pgfpathlineto{\pgfqpoint{4.285853in}{1.964774in}}%
\pgfpathlineto{\pgfqpoint{4.279326in}{1.978601in}}%
\pgfpathlineto{\pgfqpoint{4.272806in}{1.992541in}}%
\pgfpathlineto{\pgfqpoint{4.261630in}{1.988044in}}%
\pgfpathlineto{\pgfqpoint{4.250448in}{1.983491in}}%
\pgfpathlineto{\pgfqpoint{4.239260in}{1.978903in}}%
\pgfpathlineto{\pgfqpoint{4.228067in}{1.974304in}}%
\pgfpathlineto{\pgfqpoint{4.216869in}{1.969716in}}%
\pgfpathlineto{\pgfqpoint{4.223389in}{1.955842in}}%
\pgfpathlineto{\pgfqpoint{4.229918in}{1.942216in}}%
\pgfpathlineto{\pgfqpoint{4.236457in}{1.928793in}}%
\pgfpathlineto{\pgfqpoint{4.243004in}{1.915529in}}%
\pgfpathclose%
\pgfusepath{stroke,fill}%
\end{pgfscope}%
\begin{pgfscope}%
\pgfpathrectangle{\pgfqpoint{0.887500in}{0.275000in}}{\pgfqpoint{4.225000in}{4.225000in}}%
\pgfusepath{clip}%
\pgfsetbuttcap%
\pgfsetroundjoin%
\definecolor{currentfill}{rgb}{0.156270,0.489624,0.557936}%
\pgfsetfillcolor{currentfill}%
\pgfsetfillopacity{0.700000}%
\pgfsetlinewidth{0.501875pt}%
\definecolor{currentstroke}{rgb}{1.000000,1.000000,1.000000}%
\pgfsetstrokecolor{currentstroke}%
\pgfsetstrokeopacity{0.500000}%
\pgfsetdash{}{0pt}%
\pgfpathmoveto{\pgfqpoint{3.951645in}{2.136427in}}%
\pgfpathlineto{\pgfqpoint{3.962923in}{2.141473in}}%
\pgfpathlineto{\pgfqpoint{3.974193in}{2.146323in}}%
\pgfpathlineto{\pgfqpoint{3.985453in}{2.150976in}}%
\pgfpathlineto{\pgfqpoint{3.996703in}{2.155429in}}%
\pgfpathlineto{\pgfqpoint{4.007944in}{2.159681in}}%
\pgfpathlineto{\pgfqpoint{4.001487in}{2.174200in}}%
\pgfpathlineto{\pgfqpoint{3.995028in}{2.188544in}}%
\pgfpathlineto{\pgfqpoint{3.988569in}{2.202765in}}%
\pgfpathlineto{\pgfqpoint{3.982110in}{2.216911in}}%
\pgfpathlineto{\pgfqpoint{3.975654in}{2.231032in}}%
\pgfpathlineto{\pgfqpoint{3.964419in}{2.226917in}}%
\pgfpathlineto{\pgfqpoint{3.953176in}{2.222642in}}%
\pgfpathlineto{\pgfqpoint{3.941924in}{2.218205in}}%
\pgfpathlineto{\pgfqpoint{3.930663in}{2.213607in}}%
\pgfpathlineto{\pgfqpoint{3.919394in}{2.208846in}}%
\pgfpathlineto{\pgfqpoint{3.925841in}{2.194457in}}%
\pgfpathlineto{\pgfqpoint{3.932291in}{2.180072in}}%
\pgfpathlineto{\pgfqpoint{3.938742in}{2.165639in}}%
\pgfpathlineto{\pgfqpoint{3.945194in}{2.151108in}}%
\pgfpathclose%
\pgfusepath{stroke,fill}%
\end{pgfscope}%
\begin{pgfscope}%
\pgfpathrectangle{\pgfqpoint{0.887500in}{0.275000in}}{\pgfqpoint{4.225000in}{4.225000in}}%
\pgfusepath{clip}%
\pgfsetbuttcap%
\pgfsetroundjoin%
\definecolor{currentfill}{rgb}{0.192357,0.403199,0.555836}%
\pgfsetfillcolor{currentfill}%
\pgfsetfillopacity{0.700000}%
\pgfsetlinewidth{0.501875pt}%
\definecolor{currentstroke}{rgb}{1.000000,1.000000,1.000000}%
\pgfsetstrokecolor{currentstroke}%
\pgfsetstrokeopacity{0.500000}%
\pgfsetdash{}{0pt}%
\pgfpathmoveto{\pgfqpoint{2.782500in}{1.965260in}}%
\pgfpathlineto{\pgfqpoint{2.794037in}{1.968945in}}%
\pgfpathlineto{\pgfqpoint{2.805569in}{1.972555in}}%
\pgfpathlineto{\pgfqpoint{2.817096in}{1.976068in}}%
\pgfpathlineto{\pgfqpoint{2.828618in}{1.979463in}}%
\pgfpathlineto{\pgfqpoint{2.840134in}{1.982757in}}%
\pgfpathlineto{\pgfqpoint{2.833917in}{1.992685in}}%
\pgfpathlineto{\pgfqpoint{2.827704in}{2.002570in}}%
\pgfpathlineto{\pgfqpoint{2.821495in}{2.012410in}}%
\pgfpathlineto{\pgfqpoint{2.815289in}{2.022206in}}%
\pgfpathlineto{\pgfqpoint{2.809088in}{2.031959in}}%
\pgfpathlineto{\pgfqpoint{2.797581in}{2.028733in}}%
\pgfpathlineto{\pgfqpoint{2.786069in}{2.025402in}}%
\pgfpathlineto{\pgfqpoint{2.774551in}{2.021949in}}%
\pgfpathlineto{\pgfqpoint{2.763029in}{2.018396in}}%
\pgfpathlineto{\pgfqpoint{2.751501in}{2.014764in}}%
\pgfpathlineto{\pgfqpoint{2.757693in}{2.004954in}}%
\pgfpathlineto{\pgfqpoint{2.763889in}{1.995099in}}%
\pgfpathlineto{\pgfqpoint{2.770088in}{1.985199in}}%
\pgfpathlineto{\pgfqpoint{2.776292in}{1.975253in}}%
\pgfpathclose%
\pgfusepath{stroke,fill}%
\end{pgfscope}%
\begin{pgfscope}%
\pgfpathrectangle{\pgfqpoint{0.887500in}{0.275000in}}{\pgfqpoint{4.225000in}{4.225000in}}%
\pgfusepath{clip}%
\pgfsetbuttcap%
\pgfsetroundjoin%
\definecolor{currentfill}{rgb}{0.159194,0.482237,0.558073}%
\pgfsetfillcolor{currentfill}%
\pgfsetfillopacity{0.700000}%
\pgfsetlinewidth{0.501875pt}%
\definecolor{currentstroke}{rgb}{1.000000,1.000000,1.000000}%
\pgfsetstrokecolor{currentstroke}%
\pgfsetstrokeopacity{0.500000}%
\pgfsetdash{}{0pt}%
\pgfpathmoveto{\pgfqpoint{2.045932in}{2.134088in}}%
\pgfpathlineto{\pgfqpoint{2.057650in}{2.137697in}}%
\pgfpathlineto{\pgfqpoint{2.069364in}{2.141293in}}%
\pgfpathlineto{\pgfqpoint{2.081071in}{2.144877in}}%
\pgfpathlineto{\pgfqpoint{2.092774in}{2.148450in}}%
\pgfpathlineto{\pgfqpoint{2.104471in}{2.152011in}}%
\pgfpathlineto{\pgfqpoint{2.098495in}{2.161083in}}%
\pgfpathlineto{\pgfqpoint{2.092525in}{2.170128in}}%
\pgfpathlineto{\pgfqpoint{2.086558in}{2.179144in}}%
\pgfpathlineto{\pgfqpoint{2.080596in}{2.188133in}}%
\pgfpathlineto{\pgfqpoint{2.074639in}{2.197095in}}%
\pgfpathlineto{\pgfqpoint{2.062953in}{2.193583in}}%
\pgfpathlineto{\pgfqpoint{2.051261in}{2.190060in}}%
\pgfpathlineto{\pgfqpoint{2.039564in}{2.186527in}}%
\pgfpathlineto{\pgfqpoint{2.027861in}{2.182982in}}%
\pgfpathlineto{\pgfqpoint{2.016153in}{2.179425in}}%
\pgfpathlineto{\pgfqpoint{2.022100in}{2.170412in}}%
\pgfpathlineto{\pgfqpoint{2.028051in}{2.161373in}}%
\pgfpathlineto{\pgfqpoint{2.034007in}{2.152306in}}%
\pgfpathlineto{\pgfqpoint{2.039967in}{2.143211in}}%
\pgfpathclose%
\pgfusepath{stroke,fill}%
\end{pgfscope}%
\begin{pgfscope}%
\pgfpathrectangle{\pgfqpoint{0.887500in}{0.275000in}}{\pgfqpoint{4.225000in}{4.225000in}}%
\pgfusepath{clip}%
\pgfsetbuttcap%
\pgfsetroundjoin%
\definecolor{currentfill}{rgb}{0.216210,0.351535,0.550627}%
\pgfsetfillcolor{currentfill}%
\pgfsetfillopacity{0.700000}%
\pgfsetlinewidth{0.501875pt}%
\definecolor{currentstroke}{rgb}{1.000000,1.000000,1.000000}%
\pgfsetstrokecolor{currentstroke}%
\pgfsetstrokeopacity{0.500000}%
\pgfsetdash{}{0pt}%
\pgfpathmoveto{\pgfqpoint{3.195131in}{1.849100in}}%
\pgfpathlineto{\pgfqpoint{3.206566in}{1.853285in}}%
\pgfpathlineto{\pgfqpoint{3.217997in}{1.857883in}}%
\pgfpathlineto{\pgfqpoint{3.229425in}{1.863073in}}%
\pgfpathlineto{\pgfqpoint{3.240853in}{1.869037in}}%
\pgfpathlineto{\pgfqpoint{3.252281in}{1.875952in}}%
\pgfpathlineto{\pgfqpoint{3.245966in}{1.891472in}}%
\pgfpathlineto{\pgfqpoint{3.239655in}{1.907400in}}%
\pgfpathlineto{\pgfqpoint{3.233348in}{1.923631in}}%
\pgfpathlineto{\pgfqpoint{3.227043in}{1.940064in}}%
\pgfpathlineto{\pgfqpoint{3.220741in}{1.956594in}}%
\pgfpathlineto{\pgfqpoint{3.209295in}{1.944374in}}%
\pgfpathlineto{\pgfqpoint{3.197851in}{1.932907in}}%
\pgfpathlineto{\pgfqpoint{3.186410in}{1.922341in}}%
\pgfpathlineto{\pgfqpoint{3.174970in}{1.912823in}}%
\pgfpathlineto{\pgfqpoint{3.163531in}{1.904500in}}%
\pgfpathlineto{\pgfqpoint{3.169844in}{1.893481in}}%
\pgfpathlineto{\pgfqpoint{3.176161in}{1.882423in}}%
\pgfpathlineto{\pgfqpoint{3.182481in}{1.871334in}}%
\pgfpathlineto{\pgfqpoint{3.188804in}{1.860224in}}%
\pgfpathclose%
\pgfusepath{stroke,fill}%
\end{pgfscope}%
\begin{pgfscope}%
\pgfpathrectangle{\pgfqpoint{0.887500in}{0.275000in}}{\pgfqpoint{4.225000in}{4.225000in}}%
\pgfusepath{clip}%
\pgfsetbuttcap%
\pgfsetroundjoin%
\definecolor{currentfill}{rgb}{0.147607,0.511733,0.557049}%
\pgfsetfillcolor{currentfill}%
\pgfsetfillopacity{0.700000}%
\pgfsetlinewidth{0.501875pt}%
\definecolor{currentstroke}{rgb}{1.000000,1.000000,1.000000}%
\pgfsetstrokecolor{currentstroke}%
\pgfsetstrokeopacity{0.500000}%
\pgfsetdash{}{0pt}%
\pgfpathmoveto{\pgfqpoint{1.722014in}{2.197781in}}%
\pgfpathlineto{\pgfqpoint{1.733813in}{2.201366in}}%
\pgfpathlineto{\pgfqpoint{1.745606in}{2.204937in}}%
\pgfpathlineto{\pgfqpoint{1.757395in}{2.208496in}}%
\pgfpathlineto{\pgfqpoint{1.769178in}{2.212043in}}%
\pgfpathlineto{\pgfqpoint{1.780955in}{2.215581in}}%
\pgfpathlineto{\pgfqpoint{1.775090in}{2.224441in}}%
\pgfpathlineto{\pgfqpoint{1.769230in}{2.233275in}}%
\pgfpathlineto{\pgfqpoint{1.763375in}{2.242083in}}%
\pgfpathlineto{\pgfqpoint{1.757523in}{2.250865in}}%
\pgfpathlineto{\pgfqpoint{1.751677in}{2.259621in}}%
\pgfpathlineto{\pgfqpoint{1.739911in}{2.256131in}}%
\pgfpathlineto{\pgfqpoint{1.728139in}{2.252631in}}%
\pgfpathlineto{\pgfqpoint{1.716362in}{2.249121in}}%
\pgfpathlineto{\pgfqpoint{1.704579in}{2.245599in}}%
\pgfpathlineto{\pgfqpoint{1.692791in}{2.242064in}}%
\pgfpathlineto{\pgfqpoint{1.698627in}{2.233260in}}%
\pgfpathlineto{\pgfqpoint{1.704466in}{2.224430in}}%
\pgfpathlineto{\pgfqpoint{1.710311in}{2.215573in}}%
\pgfpathlineto{\pgfqpoint{1.716160in}{2.206690in}}%
\pgfpathclose%
\pgfusepath{stroke,fill}%
\end{pgfscope}%
\begin{pgfscope}%
\pgfpathrectangle{\pgfqpoint{0.887500in}{0.275000in}}{\pgfqpoint{4.225000in}{4.225000in}}%
\pgfusepath{clip}%
\pgfsetbuttcap%
\pgfsetroundjoin%
\definecolor{currentfill}{rgb}{0.194100,0.399323,0.555565}%
\pgfsetfillcolor{currentfill}%
\pgfsetfillopacity{0.700000}%
\pgfsetlinewidth{0.501875pt}%
\definecolor{currentstroke}{rgb}{1.000000,1.000000,1.000000}%
\pgfsetstrokecolor{currentstroke}%
\pgfsetstrokeopacity{0.500000}%
\pgfsetdash{}{0pt}%
\pgfpathmoveto{\pgfqpoint{4.160828in}{1.947496in}}%
\pgfpathlineto{\pgfqpoint{4.172044in}{1.951865in}}%
\pgfpathlineto{\pgfqpoint{4.183255in}{1.956240in}}%
\pgfpathlineto{\pgfqpoint{4.194463in}{1.960665in}}%
\pgfpathlineto{\pgfqpoint{4.205668in}{1.965162in}}%
\pgfpathlineto{\pgfqpoint{4.216869in}{1.969716in}}%
\pgfpathlineto{\pgfqpoint{4.210361in}{1.983883in}}%
\pgfpathlineto{\pgfqpoint{4.203865in}{1.998368in}}%
\pgfpathlineto{\pgfqpoint{4.197380in}{2.013136in}}%
\pgfpathlineto{\pgfqpoint{4.190904in}{2.028137in}}%
\pgfpathlineto{\pgfqpoint{4.184436in}{2.043320in}}%
\pgfpathlineto{\pgfqpoint{4.173246in}{2.039094in}}%
\pgfpathlineto{\pgfqpoint{4.162051in}{2.034879in}}%
\pgfpathlineto{\pgfqpoint{4.150852in}{2.030683in}}%
\pgfpathlineto{\pgfqpoint{4.139648in}{2.026497in}}%
\pgfpathlineto{\pgfqpoint{4.128438in}{2.022302in}}%
\pgfpathlineto{\pgfqpoint{4.134891in}{2.006590in}}%
\pgfpathlineto{\pgfqpoint{4.141353in}{1.991173in}}%
\pgfpathlineto{\pgfqpoint{4.147829in}{1.976134in}}%
\pgfpathlineto{\pgfqpoint{4.154320in}{1.961555in}}%
\pgfpathclose%
\pgfusepath{stroke,fill}%
\end{pgfscope}%
\begin{pgfscope}%
\pgfpathrectangle{\pgfqpoint{0.887500in}{0.275000in}}{\pgfqpoint{4.225000in}{4.225000in}}%
\pgfusepath{clip}%
\pgfsetbuttcap%
\pgfsetroundjoin%
\definecolor{currentfill}{rgb}{0.147607,0.511733,0.557049}%
\pgfsetfillcolor{currentfill}%
\pgfsetfillopacity{0.700000}%
\pgfsetlinewidth{0.501875pt}%
\definecolor{currentstroke}{rgb}{1.000000,1.000000,1.000000}%
\pgfsetstrokecolor{currentstroke}%
\pgfsetstrokeopacity{0.500000}%
\pgfsetdash{}{0pt}%
\pgfpathmoveto{\pgfqpoint{3.862944in}{2.183454in}}%
\pgfpathlineto{\pgfqpoint{3.874243in}{2.188538in}}%
\pgfpathlineto{\pgfqpoint{3.885540in}{2.193695in}}%
\pgfpathlineto{\pgfqpoint{3.896831in}{2.198848in}}%
\pgfpathlineto{\pgfqpoint{3.908117in}{2.203921in}}%
\pgfpathlineto{\pgfqpoint{3.919394in}{2.208846in}}%
\pgfpathlineto{\pgfqpoint{3.912951in}{2.223290in}}%
\pgfpathlineto{\pgfqpoint{3.906512in}{2.237839in}}%
\pgfpathlineto{\pgfqpoint{3.900079in}{2.252528in}}%
\pgfpathlineto{\pgfqpoint{3.893650in}{2.267336in}}%
\pgfpathlineto{\pgfqpoint{3.887226in}{2.282234in}}%
\pgfpathlineto{\pgfqpoint{3.875961in}{2.277761in}}%
\pgfpathlineto{\pgfqpoint{3.864688in}{2.273151in}}%
\pgfpathlineto{\pgfqpoint{3.853409in}{2.268458in}}%
\pgfpathlineto{\pgfqpoint{3.842124in}{2.263741in}}%
\pgfpathlineto{\pgfqpoint{3.830835in}{2.259059in}}%
\pgfpathlineto{\pgfqpoint{3.837248in}{2.243742in}}%
\pgfpathlineto{\pgfqpoint{3.843665in}{2.228501in}}%
\pgfpathlineto{\pgfqpoint{3.850087in}{2.213366in}}%
\pgfpathlineto{\pgfqpoint{3.856513in}{2.198360in}}%
\pgfpathclose%
\pgfusepath{stroke,fill}%
\end{pgfscope}%
\begin{pgfscope}%
\pgfpathrectangle{\pgfqpoint{0.887500in}{0.275000in}}{\pgfqpoint{4.225000in}{4.225000in}}%
\pgfusepath{clip}%
\pgfsetbuttcap%
\pgfsetroundjoin%
\definecolor{currentfill}{rgb}{0.151918,0.500685,0.557587}%
\pgfsetfillcolor{currentfill}%
\pgfsetfillopacity{0.700000}%
\pgfsetlinewidth{0.501875pt}%
\definecolor{currentstroke}{rgb}{1.000000,1.000000,1.000000}%
\pgfsetstrokecolor{currentstroke}%
\pgfsetstrokeopacity{0.500000}%
\pgfsetdash{}{0pt}%
\pgfpathmoveto{\pgfqpoint{3.335433in}{2.116511in}}%
\pgfpathlineto{\pgfqpoint{3.346943in}{2.136842in}}%
\pgfpathlineto{\pgfqpoint{3.358458in}{2.157277in}}%
\pgfpathlineto{\pgfqpoint{3.369975in}{2.177477in}}%
\pgfpathlineto{\pgfqpoint{3.381492in}{2.197105in}}%
\pgfpathlineto{\pgfqpoint{3.393005in}{2.215822in}}%
\pgfpathlineto{\pgfqpoint{3.386659in}{2.230969in}}%
\pgfpathlineto{\pgfqpoint{3.380312in}{2.245651in}}%
\pgfpathlineto{\pgfqpoint{3.373964in}{2.259837in}}%
\pgfpathlineto{\pgfqpoint{3.367614in}{2.273546in}}%
\pgfpathlineto{\pgfqpoint{3.361264in}{2.286920in}}%
\pgfpathlineto{\pgfqpoint{3.349779in}{2.269846in}}%
\pgfpathlineto{\pgfqpoint{3.338297in}{2.252875in}}%
\pgfpathlineto{\pgfqpoint{3.326818in}{2.236020in}}%
\pgfpathlineto{\pgfqpoint{3.315342in}{2.219291in}}%
\pgfpathlineto{\pgfqpoint{3.303869in}{2.202699in}}%
\pgfpathlineto{\pgfqpoint{3.310184in}{2.186309in}}%
\pgfpathlineto{\pgfqpoint{3.316499in}{2.169505in}}%
\pgfpathlineto{\pgfqpoint{3.322811in}{2.152227in}}%
\pgfpathlineto{\pgfqpoint{3.329123in}{2.134529in}}%
\pgfpathclose%
\pgfusepath{stroke,fill}%
\end{pgfscope}%
\begin{pgfscope}%
\pgfpathrectangle{\pgfqpoint{0.887500in}{0.275000in}}{\pgfqpoint{4.225000in}{4.225000in}}%
\pgfusepath{clip}%
\pgfsetbuttcap%
\pgfsetroundjoin%
\definecolor{currentfill}{rgb}{0.199430,0.387607,0.554642}%
\pgfsetfillcolor{currentfill}%
\pgfsetfillopacity{0.700000}%
\pgfsetlinewidth{0.501875pt}%
\definecolor{currentstroke}{rgb}{1.000000,1.000000,1.000000}%
\pgfsetstrokecolor{currentstroke}%
\pgfsetstrokeopacity{0.500000}%
\pgfsetdash{}{0pt}%
\pgfpathmoveto{\pgfqpoint{2.871279in}{1.932445in}}%
\pgfpathlineto{\pgfqpoint{2.882799in}{1.935792in}}%
\pgfpathlineto{\pgfqpoint{2.894313in}{1.939167in}}%
\pgfpathlineto{\pgfqpoint{2.905821in}{1.942634in}}%
\pgfpathlineto{\pgfqpoint{2.917322in}{1.946254in}}%
\pgfpathlineto{\pgfqpoint{2.928817in}{1.950091in}}%
\pgfpathlineto{\pgfqpoint{2.922571in}{1.960192in}}%
\pgfpathlineto{\pgfqpoint{2.916329in}{1.970246in}}%
\pgfpathlineto{\pgfqpoint{2.910091in}{1.980255in}}%
\pgfpathlineto{\pgfqpoint{2.903856in}{1.990217in}}%
\pgfpathlineto{\pgfqpoint{2.897625in}{2.000134in}}%
\pgfpathlineto{\pgfqpoint{2.886140in}{1.996301in}}%
\pgfpathlineto{\pgfqpoint{2.874648in}{1.992718in}}%
\pgfpathlineto{\pgfqpoint{2.863150in}{1.989313in}}%
\pgfpathlineto{\pgfqpoint{2.851645in}{1.986017in}}%
\pgfpathlineto{\pgfqpoint{2.840134in}{1.982757in}}%
\pgfpathlineto{\pgfqpoint{2.846355in}{1.972784in}}%
\pgfpathlineto{\pgfqpoint{2.852580in}{1.962767in}}%
\pgfpathlineto{\pgfqpoint{2.858809in}{1.952705in}}%
\pgfpathlineto{\pgfqpoint{2.865042in}{1.942597in}}%
\pgfpathclose%
\pgfusepath{stroke,fill}%
\end{pgfscope}%
\begin{pgfscope}%
\pgfpathrectangle{\pgfqpoint{0.887500in}{0.275000in}}{\pgfqpoint{4.225000in}{4.225000in}}%
\pgfusepath{clip}%
\pgfsetbuttcap%
\pgfsetroundjoin%
\definecolor{currentfill}{rgb}{0.177423,0.437527,0.557565}%
\pgfsetfillcolor{currentfill}%
\pgfsetfillopacity{0.700000}%
\pgfsetlinewidth{0.501875pt}%
\definecolor{currentstroke}{rgb}{1.000000,1.000000,1.000000}%
\pgfsetstrokecolor{currentstroke}%
\pgfsetstrokeopacity{0.500000}%
\pgfsetdash{}{0pt}%
\pgfpathmoveto{\pgfqpoint{2.458481in}{2.038675in}}%
\pgfpathlineto{\pgfqpoint{2.470101in}{2.042335in}}%
\pgfpathlineto{\pgfqpoint{2.481715in}{2.045973in}}%
\pgfpathlineto{\pgfqpoint{2.493325in}{2.049590in}}%
\pgfpathlineto{\pgfqpoint{2.504928in}{2.053183in}}%
\pgfpathlineto{\pgfqpoint{2.516526in}{2.056757in}}%
\pgfpathlineto{\pgfqpoint{2.510411in}{2.066236in}}%
\pgfpathlineto{\pgfqpoint{2.504301in}{2.075674in}}%
\pgfpathlineto{\pgfqpoint{2.498194in}{2.085074in}}%
\pgfpathlineto{\pgfqpoint{2.492092in}{2.094436in}}%
\pgfpathlineto{\pgfqpoint{2.485994in}{2.103760in}}%
\pgfpathlineto{\pgfqpoint{2.474406in}{2.100238in}}%
\pgfpathlineto{\pgfqpoint{2.462813in}{2.096695in}}%
\pgfpathlineto{\pgfqpoint{2.451214in}{2.093128in}}%
\pgfpathlineto{\pgfqpoint{2.439610in}{2.089539in}}%
\pgfpathlineto{\pgfqpoint{2.428000in}{2.085928in}}%
\pgfpathlineto{\pgfqpoint{2.434088in}{2.076554in}}%
\pgfpathlineto{\pgfqpoint{2.440179in}{2.067142in}}%
\pgfpathlineto{\pgfqpoint{2.446276in}{2.057692in}}%
\pgfpathlineto{\pgfqpoint{2.452376in}{2.048204in}}%
\pgfpathclose%
\pgfusepath{stroke,fill}%
\end{pgfscope}%
\begin{pgfscope}%
\pgfpathrectangle{\pgfqpoint{0.887500in}{0.275000in}}{\pgfqpoint{4.225000in}{4.225000in}}%
\pgfusepath{clip}%
\pgfsetbuttcap%
\pgfsetroundjoin%
\definecolor{currentfill}{rgb}{0.223925,0.334994,0.548053}%
\pgfsetfillcolor{currentfill}%
\pgfsetfillopacity{0.700000}%
\pgfsetlinewidth{0.501875pt}%
\definecolor{currentstroke}{rgb}{1.000000,1.000000,1.000000}%
\pgfsetstrokecolor{currentstroke}%
\pgfsetstrokeopacity{0.500000}%
\pgfsetdash{}{0pt}%
\pgfpathmoveto{\pgfqpoint{3.283939in}{1.807623in}}%
\pgfpathlineto{\pgfqpoint{3.295361in}{1.813002in}}%
\pgfpathlineto{\pgfqpoint{3.306788in}{1.819945in}}%
\pgfpathlineto{\pgfqpoint{3.318223in}{1.828565in}}%
\pgfpathlineto{\pgfqpoint{3.329667in}{1.838921in}}%
\pgfpathlineto{\pgfqpoint{3.341122in}{1.851075in}}%
\pgfpathlineto{\pgfqpoint{3.334779in}{1.864970in}}%
\pgfpathlineto{\pgfqpoint{3.328444in}{1.879771in}}%
\pgfpathlineto{\pgfqpoint{3.322119in}{1.895612in}}%
\pgfpathlineto{\pgfqpoint{3.315802in}{1.912458in}}%
\pgfpathlineto{\pgfqpoint{3.309493in}{1.930165in}}%
\pgfpathlineto{\pgfqpoint{3.298035in}{1.916341in}}%
\pgfpathlineto{\pgfqpoint{3.286586in}{1.904097in}}%
\pgfpathlineto{\pgfqpoint{3.275146in}{1.893346in}}%
\pgfpathlineto{\pgfqpoint{3.263711in}{1.884000in}}%
\pgfpathlineto{\pgfqpoint{3.252281in}{1.875952in}}%
\pgfpathlineto{\pgfqpoint{3.258600in}{1.860943in}}%
\pgfpathlineto{\pgfqpoint{3.264925in}{1.846547in}}%
\pgfpathlineto{\pgfqpoint{3.271256in}{1.832865in}}%
\pgfpathlineto{\pgfqpoint{3.277594in}{1.819920in}}%
\pgfpathclose%
\pgfusepath{stroke,fill}%
\end{pgfscope}%
\begin{pgfscope}%
\pgfpathrectangle{\pgfqpoint{0.887500in}{0.275000in}}{\pgfqpoint{4.225000in}{4.225000in}}%
\pgfusepath{clip}%
\pgfsetbuttcap%
\pgfsetroundjoin%
\definecolor{currentfill}{rgb}{0.137770,0.537492,0.554906}%
\pgfsetfillcolor{currentfill}%
\pgfsetfillopacity{0.700000}%
\pgfsetlinewidth{0.501875pt}%
\definecolor{currentstroke}{rgb}{1.000000,1.000000,1.000000}%
\pgfsetstrokecolor{currentstroke}%
\pgfsetstrokeopacity{0.500000}%
\pgfsetdash{}{0pt}%
\pgfpathmoveto{\pgfqpoint{3.774355in}{2.237746in}}%
\pgfpathlineto{\pgfqpoint{3.785658in}{2.241803in}}%
\pgfpathlineto{\pgfqpoint{3.796954in}{2.245832in}}%
\pgfpathlineto{\pgfqpoint{3.808249in}{2.250045in}}%
\pgfpathlineto{\pgfqpoint{3.819543in}{2.254474in}}%
\pgfpathlineto{\pgfqpoint{3.830835in}{2.259059in}}%
\pgfpathlineto{\pgfqpoint{3.824425in}{2.274422in}}%
\pgfpathlineto{\pgfqpoint{3.818017in}{2.289798in}}%
\pgfpathlineto{\pgfqpoint{3.811611in}{2.305157in}}%
\pgfpathlineto{\pgfqpoint{3.805207in}{2.320469in}}%
\pgfpathlineto{\pgfqpoint{3.798803in}{2.335702in}}%
\pgfpathlineto{\pgfqpoint{3.787513in}{2.331143in}}%
\pgfpathlineto{\pgfqpoint{3.776217in}{2.326528in}}%
\pgfpathlineto{\pgfqpoint{3.764916in}{2.321874in}}%
\pgfpathlineto{\pgfqpoint{3.753609in}{2.317155in}}%
\pgfpathlineto{\pgfqpoint{3.742294in}{2.312264in}}%
\pgfpathlineto{\pgfqpoint{3.748704in}{2.297472in}}%
\pgfpathlineto{\pgfqpoint{3.755115in}{2.282633in}}%
\pgfpathlineto{\pgfqpoint{3.761527in}{2.267738in}}%
\pgfpathlineto{\pgfqpoint{3.767941in}{2.252779in}}%
\pgfpathclose%
\pgfusepath{stroke,fill}%
\end{pgfscope}%
\begin{pgfscope}%
\pgfpathrectangle{\pgfqpoint{0.887500in}{0.275000in}}{\pgfqpoint{4.225000in}{4.225000in}}%
\pgfusepath{clip}%
\pgfsetbuttcap%
\pgfsetroundjoin%
\definecolor{currentfill}{rgb}{0.163625,0.471133,0.558148}%
\pgfsetfillcolor{currentfill}%
\pgfsetfillopacity{0.700000}%
\pgfsetlinewidth{0.501875pt}%
\definecolor{currentstroke}{rgb}{1.000000,1.000000,1.000000}%
\pgfsetstrokecolor{currentstroke}%
\pgfsetstrokeopacity{0.500000}%
\pgfsetdash{}{0pt}%
\pgfpathmoveto{\pgfqpoint{2.134413in}{2.106206in}}%
\pgfpathlineto{\pgfqpoint{2.146115in}{2.109807in}}%
\pgfpathlineto{\pgfqpoint{2.157811in}{2.113398in}}%
\pgfpathlineto{\pgfqpoint{2.169502in}{2.116980in}}%
\pgfpathlineto{\pgfqpoint{2.181187in}{2.120555in}}%
\pgfpathlineto{\pgfqpoint{2.192867in}{2.124126in}}%
\pgfpathlineto{\pgfqpoint{2.186859in}{2.133295in}}%
\pgfpathlineto{\pgfqpoint{2.180855in}{2.142433in}}%
\pgfpathlineto{\pgfqpoint{2.174856in}{2.151543in}}%
\pgfpathlineto{\pgfqpoint{2.168861in}{2.160624in}}%
\pgfpathlineto{\pgfqpoint{2.162871in}{2.169675in}}%
\pgfpathlineto{\pgfqpoint{2.151202in}{2.166157in}}%
\pgfpathlineto{\pgfqpoint{2.139528in}{2.162633in}}%
\pgfpathlineto{\pgfqpoint{2.127848in}{2.159102in}}%
\pgfpathlineto{\pgfqpoint{2.116162in}{2.155561in}}%
\pgfpathlineto{\pgfqpoint{2.104471in}{2.152011in}}%
\pgfpathlineto{\pgfqpoint{2.110450in}{2.142909in}}%
\pgfpathlineto{\pgfqpoint{2.116434in}{2.133778in}}%
\pgfpathlineto{\pgfqpoint{2.122423in}{2.124618in}}%
\pgfpathlineto{\pgfqpoint{2.128416in}{2.115427in}}%
\pgfpathclose%
\pgfusepath{stroke,fill}%
\end{pgfscope}%
\begin{pgfscope}%
\pgfpathrectangle{\pgfqpoint{0.887500in}{0.275000in}}{\pgfqpoint{4.225000in}{4.225000in}}%
\pgfusepath{clip}%
\pgfsetbuttcap%
\pgfsetroundjoin%
\definecolor{currentfill}{rgb}{0.244972,0.287675,0.537260}%
\pgfsetfillcolor{currentfill}%
\pgfsetfillopacity{0.700000}%
\pgfsetlinewidth{0.501875pt}%
\definecolor{currentstroke}{rgb}{1.000000,1.000000,1.000000}%
\pgfsetstrokecolor{currentstroke}%
\pgfsetstrokeopacity{0.500000}%
\pgfsetdash{}{0pt}%
\pgfpathmoveto{\pgfqpoint{4.459269in}{1.726455in}}%
\pgfpathlineto{\pgfqpoint{4.470384in}{1.729837in}}%
\pgfpathlineto{\pgfqpoint{4.481493in}{1.733183in}}%
\pgfpathlineto{\pgfqpoint{4.492594in}{1.736495in}}%
\pgfpathlineto{\pgfqpoint{4.503689in}{1.739777in}}%
\pgfpathlineto{\pgfqpoint{4.514777in}{1.743038in}}%
\pgfpathlineto{\pgfqpoint{4.508218in}{1.757467in}}%
\pgfpathlineto{\pgfqpoint{4.501663in}{1.771967in}}%
\pgfpathlineto{\pgfqpoint{4.495114in}{1.786527in}}%
\pgfpathlineto{\pgfqpoint{4.488569in}{1.801137in}}%
\pgfpathlineto{\pgfqpoint{4.482028in}{1.815786in}}%
\pgfpathlineto{\pgfqpoint{4.470948in}{1.812722in}}%
\pgfpathlineto{\pgfqpoint{4.459863in}{1.809677in}}%
\pgfpathlineto{\pgfqpoint{4.448772in}{1.806641in}}%
\pgfpathlineto{\pgfqpoint{4.437676in}{1.803604in}}%
\pgfpathlineto{\pgfqpoint{4.426574in}{1.800556in}}%
\pgfpathlineto{\pgfqpoint{4.433114in}{1.785905in}}%
\pgfpathlineto{\pgfqpoint{4.439653in}{1.771173in}}%
\pgfpathlineto{\pgfqpoint{4.446193in}{1.756357in}}%
\pgfpathlineto{\pgfqpoint{4.452731in}{1.741453in}}%
\pgfpathclose%
\pgfusepath{stroke,fill}%
\end{pgfscope}%
\begin{pgfscope}%
\pgfpathrectangle{\pgfqpoint{0.887500in}{0.275000in}}{\pgfqpoint{4.225000in}{4.225000in}}%
\pgfusepath{clip}%
\pgfsetbuttcap%
\pgfsetroundjoin%
\definecolor{currentfill}{rgb}{0.206756,0.371758,0.553117}%
\pgfsetfillcolor{currentfill}%
\pgfsetfillopacity{0.700000}%
\pgfsetlinewidth{0.501875pt}%
\definecolor{currentstroke}{rgb}{1.000000,1.000000,1.000000}%
\pgfsetstrokecolor{currentstroke}%
\pgfsetstrokeopacity{0.500000}%
\pgfsetdash{}{0pt}%
\pgfpathmoveto{\pgfqpoint{3.341122in}{1.851075in}}%
\pgfpathlineto{\pgfqpoint{3.352591in}{1.865087in}}%
\pgfpathlineto{\pgfqpoint{3.364076in}{1.881019in}}%
\pgfpathlineto{\pgfqpoint{3.375578in}{1.898932in}}%
\pgfpathlineto{\pgfqpoint{3.387099in}{1.918753in}}%
\pgfpathlineto{\pgfqpoint{3.398636in}{1.939963in}}%
\pgfpathlineto{\pgfqpoint{3.392297in}{1.956129in}}%
\pgfpathlineto{\pgfqpoint{3.385963in}{1.972678in}}%
\pgfpathlineto{\pgfqpoint{3.379634in}{1.989708in}}%
\pgfpathlineto{\pgfqpoint{3.373311in}{2.007205in}}%
\pgfpathlineto{\pgfqpoint{3.366992in}{2.025072in}}%
\pgfpathlineto{\pgfqpoint{3.355461in}{2.002892in}}%
\pgfpathlineto{\pgfqpoint{3.343945in}{1.982001in}}%
\pgfpathlineto{\pgfqpoint{3.332446in}{1.962906in}}%
\pgfpathlineto{\pgfqpoint{3.320963in}{1.945657in}}%
\pgfpathlineto{\pgfqpoint{3.309493in}{1.930165in}}%
\pgfpathlineto{\pgfqpoint{3.315802in}{1.912458in}}%
\pgfpathlineto{\pgfqpoint{3.322119in}{1.895612in}}%
\pgfpathlineto{\pgfqpoint{3.328444in}{1.879771in}}%
\pgfpathlineto{\pgfqpoint{3.334779in}{1.864970in}}%
\pgfpathclose%
\pgfusepath{stroke,fill}%
\end{pgfscope}%
\begin{pgfscope}%
\pgfpathrectangle{\pgfqpoint{0.887500in}{0.275000in}}{\pgfqpoint{4.225000in}{4.225000in}}%
\pgfusepath{clip}%
\pgfsetbuttcap%
\pgfsetroundjoin%
\definecolor{currentfill}{rgb}{0.120092,0.600104,0.542530}%
\pgfsetfillcolor{currentfill}%
\pgfsetfillopacity{0.700000}%
\pgfsetlinewidth{0.501875pt}%
\definecolor{currentstroke}{rgb}{1.000000,1.000000,1.000000}%
\pgfsetstrokecolor{currentstroke}%
\pgfsetstrokeopacity{0.500000}%
\pgfsetdash{}{0pt}%
\pgfpathmoveto{\pgfqpoint{3.507740in}{2.353668in}}%
\pgfpathlineto{\pgfqpoint{3.519174in}{2.363921in}}%
\pgfpathlineto{\pgfqpoint{3.530599in}{2.373616in}}%
\pgfpathlineto{\pgfqpoint{3.542016in}{2.382754in}}%
\pgfpathlineto{\pgfqpoint{3.553423in}{2.391337in}}%
\pgfpathlineto{\pgfqpoint{3.564821in}{2.399384in}}%
\pgfpathlineto{\pgfqpoint{3.558466in}{2.415834in}}%
\pgfpathlineto{\pgfqpoint{3.552109in}{2.431950in}}%
\pgfpathlineto{\pgfqpoint{3.545749in}{2.447655in}}%
\pgfpathlineto{\pgfqpoint{3.539385in}{2.462909in}}%
\pgfpathlineto{\pgfqpoint{3.533018in}{2.477768in}}%
\pgfpathlineto{\pgfqpoint{3.521631in}{2.470218in}}%
\pgfpathlineto{\pgfqpoint{3.510234in}{2.462130in}}%
\pgfpathlineto{\pgfqpoint{3.498827in}{2.453447in}}%
\pgfpathlineto{\pgfqpoint{3.487411in}{2.444091in}}%
\pgfpathlineto{\pgfqpoint{3.475984in}{2.433985in}}%
\pgfpathlineto{\pgfqpoint{3.482344in}{2.419101in}}%
\pgfpathlineto{\pgfqpoint{3.488700in}{2.403659in}}%
\pgfpathlineto{\pgfqpoint{3.495052in}{2.387568in}}%
\pgfpathlineto{\pgfqpoint{3.501398in}{2.370872in}}%
\pgfpathclose%
\pgfusepath{stroke,fill}%
\end{pgfscope}%
\begin{pgfscope}%
\pgfpathrectangle{\pgfqpoint{0.887500in}{0.275000in}}{\pgfqpoint{4.225000in}{4.225000in}}%
\pgfusepath{clip}%
\pgfsetbuttcap%
\pgfsetroundjoin%
\definecolor{currentfill}{rgb}{0.180629,0.429975,0.557282}%
\pgfsetfillcolor{currentfill}%
\pgfsetfillopacity{0.700000}%
\pgfsetlinewidth{0.501875pt}%
\definecolor{currentstroke}{rgb}{1.000000,1.000000,1.000000}%
\pgfsetstrokecolor{currentstroke}%
\pgfsetstrokeopacity{0.500000}%
\pgfsetdash{}{0pt}%
\pgfpathmoveto{\pgfqpoint{4.072295in}{2.000602in}}%
\pgfpathlineto{\pgfqpoint{4.083538in}{2.005097in}}%
\pgfpathlineto{\pgfqpoint{4.094773in}{2.009497in}}%
\pgfpathlineto{\pgfqpoint{4.106002in}{2.013819in}}%
\pgfpathlineto{\pgfqpoint{4.117223in}{2.018082in}}%
\pgfpathlineto{\pgfqpoint{4.128438in}{2.022302in}}%
\pgfpathlineto{\pgfqpoint{4.121994in}{2.038225in}}%
\pgfpathlineto{\pgfqpoint{4.115555in}{2.054276in}}%
\pgfpathlineto{\pgfqpoint{4.109119in}{2.070373in}}%
\pgfpathlineto{\pgfqpoint{4.102685in}{2.086431in}}%
\pgfpathlineto{\pgfqpoint{4.096249in}{2.102367in}}%
\pgfpathlineto{\pgfqpoint{4.085048in}{2.098689in}}%
\pgfpathlineto{\pgfqpoint{4.073839in}{2.094920in}}%
\pgfpathlineto{\pgfqpoint{4.062622in}{2.091035in}}%
\pgfpathlineto{\pgfqpoint{4.051396in}{2.087007in}}%
\pgfpathlineto{\pgfqpoint{4.040161in}{2.082813in}}%
\pgfpathlineto{\pgfqpoint{4.046587in}{2.066492in}}%
\pgfpathlineto{\pgfqpoint{4.053011in}{2.050010in}}%
\pgfpathlineto{\pgfqpoint{4.059435in}{2.033469in}}%
\pgfpathlineto{\pgfqpoint{4.065862in}{2.016966in}}%
\pgfpathclose%
\pgfusepath{stroke,fill}%
\end{pgfscope}%
\begin{pgfscope}%
\pgfpathrectangle{\pgfqpoint{0.887500in}{0.275000in}}{\pgfqpoint{4.225000in}{4.225000in}}%
\pgfusepath{clip}%
\pgfsetbuttcap%
\pgfsetroundjoin%
\definecolor{currentfill}{rgb}{0.151918,0.500685,0.557587}%
\pgfsetfillcolor{currentfill}%
\pgfsetfillopacity{0.700000}%
\pgfsetlinewidth{0.501875pt}%
\definecolor{currentstroke}{rgb}{1.000000,1.000000,1.000000}%
\pgfsetstrokecolor{currentstroke}%
\pgfsetstrokeopacity{0.500000}%
\pgfsetdash{}{0pt}%
\pgfpathmoveto{\pgfqpoint{1.810346in}{2.170877in}}%
\pgfpathlineto{\pgfqpoint{1.822128in}{2.174455in}}%
\pgfpathlineto{\pgfqpoint{1.833905in}{2.178024in}}%
\pgfpathlineto{\pgfqpoint{1.845677in}{2.181587in}}%
\pgfpathlineto{\pgfqpoint{1.857443in}{2.185144in}}%
\pgfpathlineto{\pgfqpoint{1.869203in}{2.188697in}}%
\pgfpathlineto{\pgfqpoint{1.863304in}{2.197644in}}%
\pgfpathlineto{\pgfqpoint{1.857411in}{2.206564in}}%
\pgfpathlineto{\pgfqpoint{1.851521in}{2.215457in}}%
\pgfpathlineto{\pgfqpoint{1.845637in}{2.224322in}}%
\pgfpathlineto{\pgfqpoint{1.839757in}{2.233162in}}%
\pgfpathlineto{\pgfqpoint{1.828008in}{2.229656in}}%
\pgfpathlineto{\pgfqpoint{1.816253in}{2.226146in}}%
\pgfpathlineto{\pgfqpoint{1.804492in}{2.222631in}}%
\pgfpathlineto{\pgfqpoint{1.792726in}{2.219110in}}%
\pgfpathlineto{\pgfqpoint{1.780955in}{2.215581in}}%
\pgfpathlineto{\pgfqpoint{1.786824in}{2.206695in}}%
\pgfpathlineto{\pgfqpoint{1.792698in}{2.197782in}}%
\pgfpathlineto{\pgfqpoint{1.798576in}{2.188841in}}%
\pgfpathlineto{\pgfqpoint{1.804458in}{2.179873in}}%
\pgfpathclose%
\pgfusepath{stroke,fill}%
\end{pgfscope}%
\begin{pgfscope}%
\pgfpathrectangle{\pgfqpoint{0.887500in}{0.275000in}}{\pgfqpoint{4.225000in}{4.225000in}}%
\pgfusepath{clip}%
\pgfsetbuttcap%
\pgfsetroundjoin%
\definecolor{currentfill}{rgb}{0.123463,0.581687,0.547445}%
\pgfsetfillcolor{currentfill}%
\pgfsetfillopacity{0.700000}%
\pgfsetlinewidth{0.501875pt}%
\definecolor{currentstroke}{rgb}{1.000000,1.000000,1.000000}%
\pgfsetstrokecolor{currentstroke}%
\pgfsetstrokeopacity{0.500000}%
\pgfsetdash{}{0pt}%
\pgfpathmoveto{\pgfqpoint{3.596590in}{2.314825in}}%
\pgfpathlineto{\pgfqpoint{3.608001in}{2.323832in}}%
\pgfpathlineto{\pgfqpoint{3.619403in}{2.332391in}}%
\pgfpathlineto{\pgfqpoint{3.630796in}{2.340507in}}%
\pgfpathlineto{\pgfqpoint{3.642180in}{2.348184in}}%
\pgfpathlineto{\pgfqpoint{3.653553in}{2.355426in}}%
\pgfpathlineto{\pgfqpoint{3.647171in}{2.370830in}}%
\pgfpathlineto{\pgfqpoint{3.640792in}{2.386328in}}%
\pgfpathlineto{\pgfqpoint{3.634416in}{2.401878in}}%
\pgfpathlineto{\pgfqpoint{3.628043in}{2.417435in}}%
\pgfpathlineto{\pgfqpoint{3.621670in}{2.432957in}}%
\pgfpathlineto{\pgfqpoint{3.610319in}{2.426999in}}%
\pgfpathlineto{\pgfqpoint{3.598958in}{2.420703in}}%
\pgfpathlineto{\pgfqpoint{3.587588in}{2.414027in}}%
\pgfpathlineto{\pgfqpoint{3.576209in}{2.406934in}}%
\pgfpathlineto{\pgfqpoint{3.564821in}{2.399384in}}%
\pgfpathlineto{\pgfqpoint{3.571174in}{2.382677in}}%
\pgfpathlineto{\pgfqpoint{3.577527in}{2.365791in}}%
\pgfpathlineto{\pgfqpoint{3.583880in}{2.348802in}}%
\pgfpathlineto{\pgfqpoint{3.590234in}{2.331788in}}%
\pgfpathclose%
\pgfusepath{stroke,fill}%
\end{pgfscope}%
\begin{pgfscope}%
\pgfpathrectangle{\pgfqpoint{0.887500in}{0.275000in}}{\pgfqpoint{4.225000in}{4.225000in}}%
\pgfusepath{clip}%
\pgfsetbuttcap%
\pgfsetroundjoin%
\definecolor{currentfill}{rgb}{0.128729,0.563265,0.551229}%
\pgfsetfillcolor{currentfill}%
\pgfsetfillopacity{0.700000}%
\pgfsetlinewidth{0.501875pt}%
\definecolor{currentstroke}{rgb}{1.000000,1.000000,1.000000}%
\pgfsetstrokecolor{currentstroke}%
\pgfsetstrokeopacity{0.500000}%
\pgfsetdash{}{0pt}%
\pgfpathmoveto{\pgfqpoint{3.685540in}{2.281117in}}%
\pgfpathlineto{\pgfqpoint{3.696920in}{2.288629in}}%
\pgfpathlineto{\pgfqpoint{3.708284in}{2.295383in}}%
\pgfpathlineto{\pgfqpoint{3.719633in}{2.301496in}}%
\pgfpathlineto{\pgfqpoint{3.730969in}{2.307083in}}%
\pgfpathlineto{\pgfqpoint{3.742294in}{2.312264in}}%
\pgfpathlineto{\pgfqpoint{3.735886in}{2.327019in}}%
\pgfpathlineto{\pgfqpoint{3.729480in}{2.341745in}}%
\pgfpathlineto{\pgfqpoint{3.723076in}{2.356452in}}%
\pgfpathlineto{\pgfqpoint{3.716674in}{2.371142in}}%
\pgfpathlineto{\pgfqpoint{3.710275in}{2.385813in}}%
\pgfpathlineto{\pgfqpoint{3.698949in}{2.380395in}}%
\pgfpathlineto{\pgfqpoint{3.687615in}{2.374685in}}%
\pgfpathlineto{\pgfqpoint{3.676271in}{2.368646in}}%
\pgfpathlineto{\pgfqpoint{3.664917in}{2.362239in}}%
\pgfpathlineto{\pgfqpoint{3.653553in}{2.355426in}}%
\pgfpathlineto{\pgfqpoint{3.659940in}{2.340162in}}%
\pgfpathlineto{\pgfqpoint{3.666332in}{2.325080in}}%
\pgfpathlineto{\pgfqpoint{3.672729in}{2.310222in}}%
\pgfpathlineto{\pgfqpoint{3.679132in}{2.295589in}}%
\pgfpathclose%
\pgfusepath{stroke,fill}%
\end{pgfscope}%
\begin{pgfscope}%
\pgfpathrectangle{\pgfqpoint{0.887500in}{0.275000in}}{\pgfqpoint{4.225000in}{4.225000in}}%
\pgfusepath{clip}%
\pgfsetbuttcap%
\pgfsetroundjoin%
\definecolor{currentfill}{rgb}{0.206756,0.371758,0.553117}%
\pgfsetfillcolor{currentfill}%
\pgfsetfillopacity{0.700000}%
\pgfsetlinewidth{0.501875pt}%
\definecolor{currentstroke}{rgb}{1.000000,1.000000,1.000000}%
\pgfsetstrokecolor{currentstroke}%
\pgfsetstrokeopacity{0.500000}%
\pgfsetdash{}{0pt}%
\pgfpathmoveto{\pgfqpoint{2.960106in}{1.898890in}}%
\pgfpathlineto{\pgfqpoint{2.971605in}{1.902893in}}%
\pgfpathlineto{\pgfqpoint{2.983098in}{1.907045in}}%
\pgfpathlineto{\pgfqpoint{2.994585in}{1.911200in}}%
\pgfpathlineto{\pgfqpoint{3.006068in}{1.915211in}}%
\pgfpathlineto{\pgfqpoint{3.017545in}{1.918928in}}%
\pgfpathlineto{\pgfqpoint{3.011270in}{1.929381in}}%
\pgfpathlineto{\pgfqpoint{3.004999in}{1.939776in}}%
\pgfpathlineto{\pgfqpoint{2.998731in}{1.950115in}}%
\pgfpathlineto{\pgfqpoint{2.992467in}{1.960395in}}%
\pgfpathlineto{\pgfqpoint{2.986207in}{1.970617in}}%
\pgfpathlineto{\pgfqpoint{2.974739in}{1.966922in}}%
\pgfpathlineto{\pgfqpoint{2.963267in}{1.962818in}}%
\pgfpathlineto{\pgfqpoint{2.951789in}{1.958510in}}%
\pgfpathlineto{\pgfqpoint{2.940306in}{1.954201in}}%
\pgfpathlineto{\pgfqpoint{2.928817in}{1.950091in}}%
\pgfpathlineto{\pgfqpoint{2.935068in}{1.939945in}}%
\pgfpathlineto{\pgfqpoint{2.941321in}{1.929751in}}%
\pgfpathlineto{\pgfqpoint{2.947579in}{1.919511in}}%
\pgfpathlineto{\pgfqpoint{2.953841in}{1.909224in}}%
\pgfpathclose%
\pgfusepath{stroke,fill}%
\end{pgfscope}%
\begin{pgfscope}%
\pgfpathrectangle{\pgfqpoint{0.887500in}{0.275000in}}{\pgfqpoint{4.225000in}{4.225000in}}%
\pgfusepath{clip}%
\pgfsetbuttcap%
\pgfsetroundjoin%
\definecolor{currentfill}{rgb}{0.182256,0.426184,0.557120}%
\pgfsetfillcolor{currentfill}%
\pgfsetfillopacity{0.700000}%
\pgfsetlinewidth{0.501875pt}%
\definecolor{currentstroke}{rgb}{1.000000,1.000000,1.000000}%
\pgfsetstrokecolor{currentstroke}%
\pgfsetstrokeopacity{0.500000}%
\pgfsetdash{}{0pt}%
\pgfpathmoveto{\pgfqpoint{2.547167in}{2.008734in}}%
\pgfpathlineto{\pgfqpoint{2.558769in}{2.012347in}}%
\pgfpathlineto{\pgfqpoint{2.570366in}{2.015950in}}%
\pgfpathlineto{\pgfqpoint{2.581958in}{2.019545in}}%
\pgfpathlineto{\pgfqpoint{2.593544in}{2.023138in}}%
\pgfpathlineto{\pgfqpoint{2.605124in}{2.026732in}}%
\pgfpathlineto{\pgfqpoint{2.598977in}{2.036368in}}%
\pgfpathlineto{\pgfqpoint{2.592835in}{2.045960in}}%
\pgfpathlineto{\pgfqpoint{2.586697in}{2.055509in}}%
\pgfpathlineto{\pgfqpoint{2.580563in}{2.065016in}}%
\pgfpathlineto{\pgfqpoint{2.574433in}{2.074482in}}%
\pgfpathlineto{\pgfqpoint{2.562863in}{2.070943in}}%
\pgfpathlineto{\pgfqpoint{2.551288in}{2.067405in}}%
\pgfpathlineto{\pgfqpoint{2.539706in}{2.063864in}}%
\pgfpathlineto{\pgfqpoint{2.528119in}{2.060316in}}%
\pgfpathlineto{\pgfqpoint{2.516526in}{2.056757in}}%
\pgfpathlineto{\pgfqpoint{2.522646in}{2.047238in}}%
\pgfpathlineto{\pgfqpoint{2.528770in}{2.037677in}}%
\pgfpathlineto{\pgfqpoint{2.534898in}{2.028073in}}%
\pgfpathlineto{\pgfqpoint{2.541030in}{2.018426in}}%
\pgfpathclose%
\pgfusepath{stroke,fill}%
\end{pgfscope}%
\begin{pgfscope}%
\pgfpathrectangle{\pgfqpoint{0.887500in}{0.275000in}}{\pgfqpoint{4.225000in}{4.225000in}}%
\pgfusepath{clip}%
\pgfsetbuttcap%
\pgfsetroundjoin%
\definecolor{currentfill}{rgb}{0.135066,0.544853,0.554029}%
\pgfsetfillcolor{currentfill}%
\pgfsetfillopacity{0.700000}%
\pgfsetlinewidth{0.501875pt}%
\definecolor{currentstroke}{rgb}{1.000000,1.000000,1.000000}%
\pgfsetstrokecolor{currentstroke}%
\pgfsetstrokeopacity{0.500000}%
\pgfsetdash{}{0pt}%
\pgfpathmoveto{\pgfqpoint{3.393005in}{2.215822in}}%
\pgfpathlineto{\pgfqpoint{3.404510in}{2.233411in}}%
\pgfpathlineto{\pgfqpoint{3.416009in}{2.249926in}}%
\pgfpathlineto{\pgfqpoint{3.427500in}{2.265458in}}%
\pgfpathlineto{\pgfqpoint{3.438984in}{2.280096in}}%
\pgfpathlineto{\pgfqpoint{3.450460in}{2.293930in}}%
\pgfpathlineto{\pgfqpoint{3.444115in}{2.310247in}}%
\pgfpathlineto{\pgfqpoint{3.437766in}{2.326014in}}%
\pgfpathlineto{\pgfqpoint{3.431413in}{2.341161in}}%
\pgfpathlineto{\pgfqpoint{3.425057in}{2.355674in}}%
\pgfpathlineto{\pgfqpoint{3.418698in}{2.369682in}}%
\pgfpathlineto{\pgfqpoint{3.407216in}{2.354041in}}%
\pgfpathlineto{\pgfqpoint{3.395729in}{2.337772in}}%
\pgfpathlineto{\pgfqpoint{3.384241in}{2.321048in}}%
\pgfpathlineto{\pgfqpoint{3.372752in}{2.304040in}}%
\pgfpathlineto{\pgfqpoint{3.361264in}{2.286920in}}%
\pgfpathlineto{\pgfqpoint{3.367614in}{2.273546in}}%
\pgfpathlineto{\pgfqpoint{3.373964in}{2.259837in}}%
\pgfpathlineto{\pgfqpoint{3.380312in}{2.245651in}}%
\pgfpathlineto{\pgfqpoint{3.386659in}{2.230969in}}%
\pgfpathclose%
\pgfusepath{stroke,fill}%
\end{pgfscope}%
\begin{pgfscope}%
\pgfpathrectangle{\pgfqpoint{0.887500in}{0.275000in}}{\pgfqpoint{4.225000in}{4.225000in}}%
\pgfusepath{clip}%
\pgfsetbuttcap%
\pgfsetroundjoin%
\definecolor{currentfill}{rgb}{0.214298,0.355619,0.551184}%
\pgfsetfillcolor{currentfill}%
\pgfsetfillopacity{0.700000}%
\pgfsetlinewidth{0.501875pt}%
\definecolor{currentstroke}{rgb}{1.000000,1.000000,1.000000}%
\pgfsetstrokecolor{currentstroke}%
\pgfsetstrokeopacity{0.500000}%
\pgfsetdash{}{0pt}%
\pgfpathmoveto{\pgfqpoint{3.048974in}{1.865844in}}%
\pgfpathlineto{\pgfqpoint{3.060455in}{1.869358in}}%
\pgfpathlineto{\pgfqpoint{3.071929in}{1.872578in}}%
\pgfpathlineto{\pgfqpoint{3.083398in}{1.875444in}}%
\pgfpathlineto{\pgfqpoint{3.094860in}{1.878092in}}%
\pgfpathlineto{\pgfqpoint{3.106316in}{1.880792in}}%
\pgfpathlineto{\pgfqpoint{3.100014in}{1.890724in}}%
\pgfpathlineto{\pgfqpoint{3.093716in}{1.900609in}}%
\pgfpathlineto{\pgfqpoint{3.087421in}{1.910459in}}%
\pgfpathlineto{\pgfqpoint{3.081131in}{1.920280in}}%
\pgfpathlineto{\pgfqpoint{3.074844in}{1.930077in}}%
\pgfpathlineto{\pgfqpoint{3.063397in}{1.928422in}}%
\pgfpathlineto{\pgfqpoint{3.051943in}{1.926871in}}%
\pgfpathlineto{\pgfqpoint{3.040483in}{1.924891in}}%
\pgfpathlineto{\pgfqpoint{3.029017in}{1.922204in}}%
\pgfpathlineto{\pgfqpoint{3.017545in}{1.918928in}}%
\pgfpathlineto{\pgfqpoint{3.023823in}{1.908420in}}%
\pgfpathlineto{\pgfqpoint{3.030106in}{1.897855in}}%
\pgfpathlineto{\pgfqpoint{3.036392in}{1.887237in}}%
\pgfpathlineto{\pgfqpoint{3.042681in}{1.876565in}}%
\pgfpathclose%
\pgfusepath{stroke,fill}%
\end{pgfscope}%
\begin{pgfscope}%
\pgfpathrectangle{\pgfqpoint{0.887500in}{0.275000in}}{\pgfqpoint{4.225000in}{4.225000in}}%
\pgfusepath{clip}%
\pgfsetbuttcap%
\pgfsetroundjoin%
\definecolor{currentfill}{rgb}{0.168126,0.459988,0.558082}%
\pgfsetfillcolor{currentfill}%
\pgfsetfillopacity{0.700000}%
\pgfsetlinewidth{0.501875pt}%
\definecolor{currentstroke}{rgb}{1.000000,1.000000,1.000000}%
\pgfsetstrokecolor{currentstroke}%
\pgfsetstrokeopacity{0.500000}%
\pgfsetdash{}{0pt}%
\pgfpathmoveto{\pgfqpoint{2.222972in}{2.077821in}}%
\pgfpathlineto{\pgfqpoint{2.234656in}{2.081446in}}%
\pgfpathlineto{\pgfqpoint{2.246334in}{2.085071in}}%
\pgfpathlineto{\pgfqpoint{2.258007in}{2.088700in}}%
\pgfpathlineto{\pgfqpoint{2.269673in}{2.092334in}}%
\pgfpathlineto{\pgfqpoint{2.281334in}{2.095976in}}%
\pgfpathlineto{\pgfqpoint{2.275294in}{2.105247in}}%
\pgfpathlineto{\pgfqpoint{2.269258in}{2.114485in}}%
\pgfpathlineto{\pgfqpoint{2.263226in}{2.123691in}}%
\pgfpathlineto{\pgfqpoint{2.257199in}{2.132867in}}%
\pgfpathlineto{\pgfqpoint{2.251176in}{2.142012in}}%
\pgfpathlineto{\pgfqpoint{2.239526in}{2.138422in}}%
\pgfpathlineto{\pgfqpoint{2.227870in}{2.134841in}}%
\pgfpathlineto{\pgfqpoint{2.216208in}{2.131267in}}%
\pgfpathlineto{\pgfqpoint{2.204540in}{2.127696in}}%
\pgfpathlineto{\pgfqpoint{2.192867in}{2.124126in}}%
\pgfpathlineto{\pgfqpoint{2.198879in}{2.114928in}}%
\pgfpathlineto{\pgfqpoint{2.204896in}{2.105699in}}%
\pgfpathlineto{\pgfqpoint{2.210917in}{2.096439in}}%
\pgfpathlineto{\pgfqpoint{2.216942in}{2.087147in}}%
\pgfpathclose%
\pgfusepath{stroke,fill}%
\end{pgfscope}%
\begin{pgfscope}%
\pgfpathrectangle{\pgfqpoint{0.887500in}{0.275000in}}{\pgfqpoint{4.225000in}{4.225000in}}%
\pgfusepath{clip}%
\pgfsetbuttcap%
\pgfsetroundjoin%
\definecolor{currentfill}{rgb}{0.124395,0.578002,0.548287}%
\pgfsetfillcolor{currentfill}%
\pgfsetfillopacity{0.700000}%
\pgfsetlinewidth{0.501875pt}%
\definecolor{currentstroke}{rgb}{1.000000,1.000000,1.000000}%
\pgfsetstrokecolor{currentstroke}%
\pgfsetstrokeopacity{0.500000}%
\pgfsetdash{}{0pt}%
\pgfpathmoveto{\pgfqpoint{3.450460in}{2.293930in}}%
\pgfpathlineto{\pgfqpoint{3.461930in}{2.307052in}}%
\pgfpathlineto{\pgfqpoint{3.473392in}{2.319551in}}%
\pgfpathlineto{\pgfqpoint{3.484849in}{2.331484in}}%
\pgfpathlineto{\pgfqpoint{3.496298in}{2.342856in}}%
\pgfpathlineto{\pgfqpoint{3.507740in}{2.353668in}}%
\pgfpathlineto{\pgfqpoint{3.501398in}{2.370872in}}%
\pgfpathlineto{\pgfqpoint{3.495052in}{2.387568in}}%
\pgfpathlineto{\pgfqpoint{3.488700in}{2.403659in}}%
\pgfpathlineto{\pgfqpoint{3.482344in}{2.419101in}}%
\pgfpathlineto{\pgfqpoint{3.475984in}{2.433985in}}%
\pgfpathlineto{\pgfqpoint{3.464547in}{2.423052in}}%
\pgfpathlineto{\pgfqpoint{3.453100in}{2.411213in}}%
\pgfpathlineto{\pgfqpoint{3.441642in}{2.398392in}}%
\pgfpathlineto{\pgfqpoint{3.430174in}{2.384522in}}%
\pgfpathlineto{\pgfqpoint{3.418698in}{2.369682in}}%
\pgfpathlineto{\pgfqpoint{3.425057in}{2.355674in}}%
\pgfpathlineto{\pgfqpoint{3.431413in}{2.341161in}}%
\pgfpathlineto{\pgfqpoint{3.437766in}{2.326014in}}%
\pgfpathlineto{\pgfqpoint{3.444115in}{2.310247in}}%
\pgfpathclose%
\pgfusepath{stroke,fill}%
\end{pgfscope}%
\begin{pgfscope}%
\pgfpathrectangle{\pgfqpoint{0.887500in}{0.275000in}}{\pgfqpoint{4.225000in}{4.225000in}}%
\pgfusepath{clip}%
\pgfsetbuttcap%
\pgfsetroundjoin%
\definecolor{currentfill}{rgb}{0.231674,0.318106,0.544834}%
\pgfsetfillcolor{currentfill}%
\pgfsetfillopacity{0.700000}%
\pgfsetlinewidth{0.501875pt}%
\definecolor{currentstroke}{rgb}{1.000000,1.000000,1.000000}%
\pgfsetstrokecolor{currentstroke}%
\pgfsetstrokeopacity{0.500000}%
\pgfsetdash{}{0pt}%
\pgfpathmoveto{\pgfqpoint{4.370961in}{1.784915in}}%
\pgfpathlineto{\pgfqpoint{4.382098in}{1.788120in}}%
\pgfpathlineto{\pgfqpoint{4.393227in}{1.791280in}}%
\pgfpathlineto{\pgfqpoint{4.404349in}{1.794401in}}%
\pgfpathlineto{\pgfqpoint{4.415465in}{1.797491in}}%
\pgfpathlineto{\pgfqpoint{4.426574in}{1.800556in}}%
\pgfpathlineto{\pgfqpoint{4.420033in}{1.815133in}}%
\pgfpathlineto{\pgfqpoint{4.413494in}{1.829639in}}%
\pgfpathlineto{\pgfqpoint{4.406954in}{1.844079in}}%
\pgfpathlineto{\pgfqpoint{4.400415in}{1.858457in}}%
\pgfpathlineto{\pgfqpoint{4.393877in}{1.872779in}}%
\pgfpathlineto{\pgfqpoint{4.382759in}{1.869420in}}%
\pgfpathlineto{\pgfqpoint{4.371634in}{1.866020in}}%
\pgfpathlineto{\pgfqpoint{4.360502in}{1.862571in}}%
\pgfpathlineto{\pgfqpoint{4.349363in}{1.859062in}}%
\pgfpathlineto{\pgfqpoint{4.338216in}{1.855484in}}%
\pgfpathlineto{\pgfqpoint{4.344769in}{1.841646in}}%
\pgfpathlineto{\pgfqpoint{4.351320in}{1.827693in}}%
\pgfpathlineto{\pgfqpoint{4.357870in}{1.813602in}}%
\pgfpathlineto{\pgfqpoint{4.364417in}{1.799350in}}%
\pgfpathclose%
\pgfusepath{stroke,fill}%
\end{pgfscope}%
\begin{pgfscope}%
\pgfpathrectangle{\pgfqpoint{0.887500in}{0.275000in}}{\pgfqpoint{4.225000in}{4.225000in}}%
\pgfusepath{clip}%
\pgfsetbuttcap%
\pgfsetroundjoin%
\definecolor{currentfill}{rgb}{0.165117,0.467423,0.558141}%
\pgfsetfillcolor{currentfill}%
\pgfsetfillopacity{0.700000}%
\pgfsetlinewidth{0.501875pt}%
\definecolor{currentstroke}{rgb}{1.000000,1.000000,1.000000}%
\pgfsetstrokecolor{currentstroke}%
\pgfsetstrokeopacity{0.500000}%
\pgfsetdash{}{0pt}%
\pgfpathmoveto{\pgfqpoint{3.366992in}{2.025072in}}%
\pgfpathlineto{\pgfqpoint{3.378534in}{2.047939in}}%
\pgfpathlineto{\pgfqpoint{3.390082in}{2.070891in}}%
\pgfpathlineto{\pgfqpoint{3.401632in}{2.093321in}}%
\pgfpathlineto{\pgfqpoint{3.413176in}{2.114623in}}%
\pgfpathlineto{\pgfqpoint{3.424709in}{2.134189in}}%
\pgfpathlineto{\pgfqpoint{3.418370in}{2.151199in}}%
\pgfpathlineto{\pgfqpoint{3.412031in}{2.167898in}}%
\pgfpathlineto{\pgfqpoint{3.405690in}{2.184255in}}%
\pgfpathlineto{\pgfqpoint{3.399348in}{2.200240in}}%
\pgfpathlineto{\pgfqpoint{3.393005in}{2.215822in}}%
\pgfpathlineto{\pgfqpoint{3.381492in}{2.197105in}}%
\pgfpathlineto{\pgfqpoint{3.369975in}{2.177477in}}%
\pgfpathlineto{\pgfqpoint{3.358458in}{2.157277in}}%
\pgfpathlineto{\pgfqpoint{3.346943in}{2.136842in}}%
\pgfpathlineto{\pgfqpoint{3.335433in}{2.116511in}}%
\pgfpathlineto{\pgfqpoint{3.341743in}{2.098271in}}%
\pgfpathlineto{\pgfqpoint{3.348053in}{2.079908in}}%
\pgfpathlineto{\pgfqpoint{3.354364in}{2.061522in}}%
\pgfpathlineto{\pgfqpoint{3.360677in}{2.043210in}}%
\pgfpathclose%
\pgfusepath{stroke,fill}%
\end{pgfscope}%
\begin{pgfscope}%
\pgfpathrectangle{\pgfqpoint{0.887500in}{0.275000in}}{\pgfqpoint{4.225000in}{4.225000in}}%
\pgfusepath{clip}%
\pgfsetbuttcap%
\pgfsetroundjoin%
\definecolor{currentfill}{rgb}{0.168126,0.459988,0.558082}%
\pgfsetfillcolor{currentfill}%
\pgfsetfillopacity{0.700000}%
\pgfsetlinewidth{0.501875pt}%
\definecolor{currentstroke}{rgb}{1.000000,1.000000,1.000000}%
\pgfsetstrokecolor{currentstroke}%
\pgfsetstrokeopacity{0.500000}%
\pgfsetdash{}{0pt}%
\pgfpathmoveto{\pgfqpoint{3.983842in}{2.059067in}}%
\pgfpathlineto{\pgfqpoint{3.995125in}{2.064211in}}%
\pgfpathlineto{\pgfqpoint{4.006398in}{2.069150in}}%
\pgfpathlineto{\pgfqpoint{4.017662in}{2.073892in}}%
\pgfpathlineto{\pgfqpoint{4.028916in}{2.078444in}}%
\pgfpathlineto{\pgfqpoint{4.040161in}{2.082813in}}%
\pgfpathlineto{\pgfqpoint{4.033730in}{2.098876in}}%
\pgfpathlineto{\pgfqpoint{4.027293in}{2.114585in}}%
\pgfpathlineto{\pgfqpoint{4.020848in}{2.129923in}}%
\pgfpathlineto{\pgfqpoint{4.014398in}{2.144938in}}%
\pgfpathlineto{\pgfqpoint{4.007944in}{2.159681in}}%
\pgfpathlineto{\pgfqpoint{3.996703in}{2.155429in}}%
\pgfpathlineto{\pgfqpoint{3.985453in}{2.150976in}}%
\pgfpathlineto{\pgfqpoint{3.974193in}{2.146323in}}%
\pgfpathlineto{\pgfqpoint{3.962923in}{2.141473in}}%
\pgfpathlineto{\pgfqpoint{3.951645in}{2.136427in}}%
\pgfpathlineto{\pgfqpoint{3.958093in}{2.121546in}}%
\pgfpathlineto{\pgfqpoint{3.964538in}{2.106414in}}%
\pgfpathlineto{\pgfqpoint{3.970979in}{2.090980in}}%
\pgfpathlineto{\pgfqpoint{3.977413in}{2.075195in}}%
\pgfpathclose%
\pgfusepath{stroke,fill}%
\end{pgfscope}%
\begin{pgfscope}%
\pgfpathrectangle{\pgfqpoint{0.887500in}{0.275000in}}{\pgfqpoint{4.225000in}{4.225000in}}%
\pgfusepath{clip}%
\pgfsetbuttcap%
\pgfsetroundjoin%
\definecolor{currentfill}{rgb}{0.156270,0.489624,0.557936}%
\pgfsetfillcolor{currentfill}%
\pgfsetfillopacity{0.700000}%
\pgfsetlinewidth{0.501875pt}%
\definecolor{currentstroke}{rgb}{1.000000,1.000000,1.000000}%
\pgfsetstrokecolor{currentstroke}%
\pgfsetstrokeopacity{0.500000}%
\pgfsetdash{}{0pt}%
\pgfpathmoveto{\pgfqpoint{1.898761in}{2.143536in}}%
\pgfpathlineto{\pgfqpoint{1.910527in}{2.147135in}}%
\pgfpathlineto{\pgfqpoint{1.922286in}{2.150732in}}%
\pgfpathlineto{\pgfqpoint{1.934039in}{2.154327in}}%
\pgfpathlineto{\pgfqpoint{1.945787in}{2.157921in}}%
\pgfpathlineto{\pgfqpoint{1.957529in}{2.161516in}}%
\pgfpathlineto{\pgfqpoint{1.951598in}{2.170554in}}%
\pgfpathlineto{\pgfqpoint{1.945670in}{2.179564in}}%
\pgfpathlineto{\pgfqpoint{1.939748in}{2.188547in}}%
\pgfpathlineto{\pgfqpoint{1.933830in}{2.197503in}}%
\pgfpathlineto{\pgfqpoint{1.927916in}{2.206432in}}%
\pgfpathlineto{\pgfqpoint{1.916185in}{2.202885in}}%
\pgfpathlineto{\pgfqpoint{1.904448in}{2.199339in}}%
\pgfpathlineto{\pgfqpoint{1.892705in}{2.195793in}}%
\pgfpathlineto{\pgfqpoint{1.880957in}{2.192246in}}%
\pgfpathlineto{\pgfqpoint{1.869203in}{2.188697in}}%
\pgfpathlineto{\pgfqpoint{1.875105in}{2.179721in}}%
\pgfpathlineto{\pgfqpoint{1.881013in}{2.170718in}}%
\pgfpathlineto{\pgfqpoint{1.886924in}{2.161686in}}%
\pgfpathlineto{\pgfqpoint{1.892841in}{2.152626in}}%
\pgfpathclose%
\pgfusepath{stroke,fill}%
\end{pgfscope}%
\begin{pgfscope}%
\pgfpathrectangle{\pgfqpoint{0.887500in}{0.275000in}}{\pgfqpoint{4.225000in}{4.225000in}}%
\pgfusepath{clip}%
\pgfsetbuttcap%
\pgfsetroundjoin%
\definecolor{currentfill}{rgb}{0.187231,0.414746,0.556547}%
\pgfsetfillcolor{currentfill}%
\pgfsetfillopacity{0.700000}%
\pgfsetlinewidth{0.501875pt}%
\definecolor{currentstroke}{rgb}{1.000000,1.000000,1.000000}%
\pgfsetstrokecolor{currentstroke}%
\pgfsetstrokeopacity{0.500000}%
\pgfsetdash{}{0pt}%
\pgfpathmoveto{\pgfqpoint{2.635920in}{1.977861in}}%
\pgfpathlineto{\pgfqpoint{2.647504in}{1.981514in}}%
\pgfpathlineto{\pgfqpoint{2.659083in}{1.985173in}}%
\pgfpathlineto{\pgfqpoint{2.670655in}{1.988840in}}%
\pgfpathlineto{\pgfqpoint{2.682222in}{1.992517in}}%
\pgfpathlineto{\pgfqpoint{2.693783in}{1.996205in}}%
\pgfpathlineto{\pgfqpoint{2.687605in}{2.006019in}}%
\pgfpathlineto{\pgfqpoint{2.681432in}{2.015786in}}%
\pgfpathlineto{\pgfqpoint{2.675262in}{2.025508in}}%
\pgfpathlineto{\pgfqpoint{2.669097in}{2.035184in}}%
\pgfpathlineto{\pgfqpoint{2.662937in}{2.044814in}}%
\pgfpathlineto{\pgfqpoint{2.651386in}{2.041175in}}%
\pgfpathlineto{\pgfqpoint{2.639829in}{2.037549in}}%
\pgfpathlineto{\pgfqpoint{2.628266in}{2.033935in}}%
\pgfpathlineto{\pgfqpoint{2.616698in}{2.030330in}}%
\pgfpathlineto{\pgfqpoint{2.605124in}{2.026732in}}%
\pgfpathlineto{\pgfqpoint{2.611274in}{2.017050in}}%
\pgfpathlineto{\pgfqpoint{2.617430in}{2.007322in}}%
\pgfpathlineto{\pgfqpoint{2.623589in}{1.997548in}}%
\pgfpathlineto{\pgfqpoint{2.629752in}{1.987728in}}%
\pgfpathclose%
\pgfusepath{stroke,fill}%
\end{pgfscope}%
\begin{pgfscope}%
\pgfpathrectangle{\pgfqpoint{0.887500in}{0.275000in}}{\pgfqpoint{4.225000in}{4.225000in}}%
\pgfusepath{clip}%
\pgfsetbuttcap%
\pgfsetroundjoin%
\definecolor{currentfill}{rgb}{0.221989,0.339161,0.548752}%
\pgfsetfillcolor{currentfill}%
\pgfsetfillopacity{0.700000}%
\pgfsetlinewidth{0.501875pt}%
\definecolor{currentstroke}{rgb}{1.000000,1.000000,1.000000}%
\pgfsetstrokecolor{currentstroke}%
\pgfsetstrokeopacity{0.500000}%
\pgfsetdash{}{0pt}%
\pgfpathmoveto{\pgfqpoint{3.137882in}{1.830100in}}%
\pgfpathlineto{\pgfqpoint{3.149342in}{1.833793in}}%
\pgfpathlineto{\pgfqpoint{3.160798in}{1.837521in}}%
\pgfpathlineto{\pgfqpoint{3.172247in}{1.841300in}}%
\pgfpathlineto{\pgfqpoint{3.183692in}{1.845150in}}%
\pgfpathlineto{\pgfqpoint{3.195131in}{1.849100in}}%
\pgfpathlineto{\pgfqpoint{3.188804in}{1.860224in}}%
\pgfpathlineto{\pgfqpoint{3.182481in}{1.871334in}}%
\pgfpathlineto{\pgfqpoint{3.176161in}{1.882423in}}%
\pgfpathlineto{\pgfqpoint{3.169844in}{1.893481in}}%
\pgfpathlineto{\pgfqpoint{3.163531in}{1.904500in}}%
\pgfpathlineto{\pgfqpoint{3.152092in}{1.897519in}}%
\pgfpathlineto{\pgfqpoint{3.140653in}{1.891908in}}%
\pgfpathlineto{\pgfqpoint{3.129212in}{1.887430in}}%
\pgfpathlineto{\pgfqpoint{3.117766in}{1.883815in}}%
\pgfpathlineto{\pgfqpoint{3.106316in}{1.880792in}}%
\pgfpathlineto{\pgfqpoint{3.112622in}{1.870805in}}%
\pgfpathlineto{\pgfqpoint{3.118931in}{1.860752in}}%
\pgfpathlineto{\pgfqpoint{3.125245in}{1.850624in}}%
\pgfpathlineto{\pgfqpoint{3.131561in}{1.840410in}}%
\pgfpathclose%
\pgfusepath{stroke,fill}%
\end{pgfscope}%
\begin{pgfscope}%
\pgfpathrectangle{\pgfqpoint{0.887500in}{0.275000in}}{\pgfqpoint{4.225000in}{4.225000in}}%
\pgfusepath{clip}%
\pgfsetbuttcap%
\pgfsetroundjoin%
\definecolor{currentfill}{rgb}{0.157729,0.485932,0.558013}%
\pgfsetfillcolor{currentfill}%
\pgfsetfillopacity{0.700000}%
\pgfsetlinewidth{0.501875pt}%
\definecolor{currentstroke}{rgb}{1.000000,1.000000,1.000000}%
\pgfsetstrokecolor{currentstroke}%
\pgfsetstrokeopacity{0.500000}%
\pgfsetdash{}{0pt}%
\pgfpathmoveto{\pgfqpoint{3.895131in}{2.108852in}}%
\pgfpathlineto{\pgfqpoint{3.906446in}{2.114548in}}%
\pgfpathlineto{\pgfqpoint{3.917757in}{2.120201in}}%
\pgfpathlineto{\pgfqpoint{3.929061in}{2.125764in}}%
\pgfpathlineto{\pgfqpoint{3.940357in}{2.131186in}}%
\pgfpathlineto{\pgfqpoint{3.951645in}{2.136427in}}%
\pgfpathlineto{\pgfqpoint{3.945194in}{2.151108in}}%
\pgfpathlineto{\pgfqpoint{3.938742in}{2.165639in}}%
\pgfpathlineto{\pgfqpoint{3.932291in}{2.180072in}}%
\pgfpathlineto{\pgfqpoint{3.925841in}{2.194457in}}%
\pgfpathlineto{\pgfqpoint{3.919394in}{2.208846in}}%
\pgfpathlineto{\pgfqpoint{3.908117in}{2.203921in}}%
\pgfpathlineto{\pgfqpoint{3.896831in}{2.198848in}}%
\pgfpathlineto{\pgfqpoint{3.885540in}{2.193695in}}%
\pgfpathlineto{\pgfqpoint{3.874243in}{2.188538in}}%
\pgfpathlineto{\pgfqpoint{3.862944in}{2.183454in}}%
\pgfpathlineto{\pgfqpoint{3.869378in}{2.168603in}}%
\pgfpathlineto{\pgfqpoint{3.875815in}{2.153762in}}%
\pgfpathlineto{\pgfqpoint{3.882253in}{2.138887in}}%
\pgfpathlineto{\pgfqpoint{3.888692in}{2.123932in}}%
\pgfpathclose%
\pgfusepath{stroke,fill}%
\end{pgfscope}%
\begin{pgfscope}%
\pgfpathrectangle{\pgfqpoint{0.887500in}{0.275000in}}{\pgfqpoint{4.225000in}{4.225000in}}%
\pgfusepath{clip}%
\pgfsetbuttcap%
\pgfsetroundjoin%
\definecolor{currentfill}{rgb}{0.218130,0.347432,0.550038}%
\pgfsetfillcolor{currentfill}%
\pgfsetfillopacity{0.700000}%
\pgfsetlinewidth{0.501875pt}%
\definecolor{currentstroke}{rgb}{1.000000,1.000000,1.000000}%
\pgfsetstrokecolor{currentstroke}%
\pgfsetstrokeopacity{0.500000}%
\pgfsetdash{}{0pt}%
\pgfpathmoveto{\pgfqpoint{4.282373in}{1.836791in}}%
\pgfpathlineto{\pgfqpoint{4.293554in}{1.840569in}}%
\pgfpathlineto{\pgfqpoint{4.304729in}{1.844350in}}%
\pgfpathlineto{\pgfqpoint{4.315898in}{1.848111in}}%
\pgfpathlineto{\pgfqpoint{4.327061in}{1.851830in}}%
\pgfpathlineto{\pgfqpoint{4.338216in}{1.855484in}}%
\pgfpathlineto{\pgfqpoint{4.331663in}{1.869229in}}%
\pgfpathlineto{\pgfqpoint{4.325111in}{1.882905in}}%
\pgfpathlineto{\pgfqpoint{4.318560in}{1.896534in}}%
\pgfpathlineto{\pgfqpoint{4.312012in}{1.910139in}}%
\pgfpathlineto{\pgfqpoint{4.305466in}{1.923743in}}%
\pgfpathlineto{\pgfqpoint{4.294296in}{1.919523in}}%
\pgfpathlineto{\pgfqpoint{4.283119in}{1.915242in}}%
\pgfpathlineto{\pgfqpoint{4.271936in}{1.910935in}}%
\pgfpathlineto{\pgfqpoint{4.260748in}{1.906635in}}%
\pgfpathlineto{\pgfqpoint{4.249557in}{1.902379in}}%
\pgfpathlineto{\pgfqpoint{4.256115in}{1.889300in}}%
\pgfpathlineto{\pgfqpoint{4.262677in}{1.876246in}}%
\pgfpathlineto{\pgfqpoint{4.269242in}{1.863172in}}%
\pgfpathlineto{\pgfqpoint{4.275807in}{1.850036in}}%
\pgfpathclose%
\pgfusepath{stroke,fill}%
\end{pgfscope}%
\begin{pgfscope}%
\pgfpathrectangle{\pgfqpoint{0.887500in}{0.275000in}}{\pgfqpoint{4.225000in}{4.225000in}}%
\pgfusepath{clip}%
\pgfsetbuttcap%
\pgfsetroundjoin%
\definecolor{currentfill}{rgb}{0.206756,0.371758,0.553117}%
\pgfsetfillcolor{currentfill}%
\pgfsetfillopacity{0.700000}%
\pgfsetlinewidth{0.501875pt}%
\definecolor{currentstroke}{rgb}{1.000000,1.000000,1.000000}%
\pgfsetstrokecolor{currentstroke}%
\pgfsetstrokeopacity{0.500000}%
\pgfsetdash{}{0pt}%
\pgfpathmoveto{\pgfqpoint{4.193561in}{1.882441in}}%
\pgfpathlineto{\pgfqpoint{4.204768in}{1.886342in}}%
\pgfpathlineto{\pgfqpoint{4.215969in}{1.890205in}}%
\pgfpathlineto{\pgfqpoint{4.227166in}{1.894136in}}%
\pgfpathlineto{\pgfqpoint{4.238362in}{1.898201in}}%
\pgfpathlineto{\pgfqpoint{4.249557in}{1.902379in}}%
\pgfpathlineto{\pgfqpoint{4.243004in}{1.915529in}}%
\pgfpathlineto{\pgfqpoint{4.236457in}{1.928793in}}%
\pgfpathlineto{\pgfqpoint{4.229918in}{1.942216in}}%
\pgfpathlineto{\pgfqpoint{4.223389in}{1.955842in}}%
\pgfpathlineto{\pgfqpoint{4.216869in}{1.969716in}}%
\pgfpathlineto{\pgfqpoint{4.205668in}{1.965162in}}%
\pgfpathlineto{\pgfqpoint{4.194463in}{1.960665in}}%
\pgfpathlineto{\pgfqpoint{4.183255in}{1.956240in}}%
\pgfpathlineto{\pgfqpoint{4.172044in}{1.951865in}}%
\pgfpathlineto{\pgfqpoint{4.160828in}{1.947496in}}%
\pgfpathlineto{\pgfqpoint{4.167351in}{1.933902in}}%
\pgfpathlineto{\pgfqpoint{4.173889in}{1.920686in}}%
\pgfpathlineto{\pgfqpoint{4.180438in}{1.907762in}}%
\pgfpathlineto{\pgfqpoint{4.186996in}{1.895043in}}%
\pgfpathclose%
\pgfusepath{stroke,fill}%
\end{pgfscope}%
\begin{pgfscope}%
\pgfpathrectangle{\pgfqpoint{0.887500in}{0.275000in}}{\pgfqpoint{4.225000in}{4.225000in}}%
\pgfusepath{clip}%
\pgfsetbuttcap%
\pgfsetroundjoin%
\definecolor{currentfill}{rgb}{0.127568,0.566949,0.550556}%
\pgfsetfillcolor{currentfill}%
\pgfsetfillopacity{0.700000}%
\pgfsetlinewidth{0.501875pt}%
\definecolor{currentstroke}{rgb}{1.000000,1.000000,1.000000}%
\pgfsetstrokecolor{currentstroke}%
\pgfsetstrokeopacity{0.500000}%
\pgfsetdash{}{0pt}%
\pgfpathmoveto{\pgfqpoint{3.539422in}{2.263489in}}%
\pgfpathlineto{\pgfqpoint{3.550868in}{2.274450in}}%
\pgfpathlineto{\pgfqpoint{3.562309in}{2.285130in}}%
\pgfpathlineto{\pgfqpoint{3.573744in}{2.295459in}}%
\pgfpathlineto{\pgfqpoint{3.585171in}{2.305369in}}%
\pgfpathlineto{\pgfqpoint{3.596590in}{2.314825in}}%
\pgfpathlineto{\pgfqpoint{3.590234in}{2.331788in}}%
\pgfpathlineto{\pgfqpoint{3.583880in}{2.348802in}}%
\pgfpathlineto{\pgfqpoint{3.577527in}{2.365791in}}%
\pgfpathlineto{\pgfqpoint{3.571174in}{2.382677in}}%
\pgfpathlineto{\pgfqpoint{3.564821in}{2.399384in}}%
\pgfpathlineto{\pgfqpoint{3.553423in}{2.391337in}}%
\pgfpathlineto{\pgfqpoint{3.542016in}{2.382754in}}%
\pgfpathlineto{\pgfqpoint{3.530599in}{2.373616in}}%
\pgfpathlineto{\pgfqpoint{3.519174in}{2.363921in}}%
\pgfpathlineto{\pgfqpoint{3.507740in}{2.353668in}}%
\pgfpathlineto{\pgfqpoint{3.514079in}{2.336056in}}%
\pgfpathlineto{\pgfqpoint{3.520416in}{2.318134in}}%
\pgfpathlineto{\pgfqpoint{3.526751in}{2.300000in}}%
\pgfpathlineto{\pgfqpoint{3.533086in}{2.281752in}}%
\pgfpathclose%
\pgfusepath{stroke,fill}%
\end{pgfscope}%
\begin{pgfscope}%
\pgfpathrectangle{\pgfqpoint{0.887500in}{0.275000in}}{\pgfqpoint{4.225000in}{4.225000in}}%
\pgfusepath{clip}%
\pgfsetbuttcap%
\pgfsetroundjoin%
\definecolor{currentfill}{rgb}{0.172719,0.448791,0.557885}%
\pgfsetfillcolor{currentfill}%
\pgfsetfillopacity{0.700000}%
\pgfsetlinewidth{0.501875pt}%
\definecolor{currentstroke}{rgb}{1.000000,1.000000,1.000000}%
\pgfsetstrokecolor{currentstroke}%
\pgfsetstrokeopacity{0.500000}%
\pgfsetdash{}{0pt}%
\pgfpathmoveto{\pgfqpoint{2.311602in}{2.049104in}}%
\pgfpathlineto{\pgfqpoint{2.323267in}{2.052808in}}%
\pgfpathlineto{\pgfqpoint{2.334927in}{2.056515in}}%
\pgfpathlineto{\pgfqpoint{2.346580in}{2.060222in}}%
\pgfpathlineto{\pgfqpoint{2.358228in}{2.063925in}}%
\pgfpathlineto{\pgfqpoint{2.369871in}{2.067623in}}%
\pgfpathlineto{\pgfqpoint{2.363798in}{2.077013in}}%
\pgfpathlineto{\pgfqpoint{2.357730in}{2.086368in}}%
\pgfpathlineto{\pgfqpoint{2.351666in}{2.095688in}}%
\pgfpathlineto{\pgfqpoint{2.345606in}{2.104973in}}%
\pgfpathlineto{\pgfqpoint{2.339551in}{2.114226in}}%
\pgfpathlineto{\pgfqpoint{2.327919in}{2.110581in}}%
\pgfpathlineto{\pgfqpoint{2.316281in}{2.106930in}}%
\pgfpathlineto{\pgfqpoint{2.304638in}{2.103277in}}%
\pgfpathlineto{\pgfqpoint{2.292989in}{2.099625in}}%
\pgfpathlineto{\pgfqpoint{2.281334in}{2.095976in}}%
\pgfpathlineto{\pgfqpoint{2.287379in}{2.086672in}}%
\pgfpathlineto{\pgfqpoint{2.293428in}{2.077333in}}%
\pgfpathlineto{\pgfqpoint{2.299482in}{2.067960in}}%
\pgfpathlineto{\pgfqpoint{2.305540in}{2.058550in}}%
\pgfpathclose%
\pgfusepath{stroke,fill}%
\end{pgfscope}%
\begin{pgfscope}%
\pgfpathrectangle{\pgfqpoint{0.887500in}{0.275000in}}{\pgfqpoint{4.225000in}{4.225000in}}%
\pgfusepath{clip}%
\pgfsetbuttcap%
\pgfsetroundjoin%
\definecolor{currentfill}{rgb}{0.229739,0.322361,0.545706}%
\pgfsetfillcolor{currentfill}%
\pgfsetfillopacity{0.700000}%
\pgfsetlinewidth{0.501875pt}%
\definecolor{currentstroke}{rgb}{1.000000,1.000000,1.000000}%
\pgfsetstrokecolor{currentstroke}%
\pgfsetstrokeopacity{0.500000}%
\pgfsetdash{}{0pt}%
\pgfpathmoveto{\pgfqpoint{3.226818in}{1.793552in}}%
\pgfpathlineto{\pgfqpoint{3.238251in}{1.795721in}}%
\pgfpathlineto{\pgfqpoint{3.249678in}{1.797893in}}%
\pgfpathlineto{\pgfqpoint{3.261101in}{1.800388in}}%
\pgfpathlineto{\pgfqpoint{3.272520in}{1.803525in}}%
\pgfpathlineto{\pgfqpoint{3.283939in}{1.807623in}}%
\pgfpathlineto{\pgfqpoint{3.277594in}{1.819920in}}%
\pgfpathlineto{\pgfqpoint{3.271256in}{1.832865in}}%
\pgfpathlineto{\pgfqpoint{3.264925in}{1.846547in}}%
\pgfpathlineto{\pgfqpoint{3.258600in}{1.860943in}}%
\pgfpathlineto{\pgfqpoint{3.252281in}{1.875952in}}%
\pgfpathlineto{\pgfqpoint{3.240853in}{1.869037in}}%
\pgfpathlineto{\pgfqpoint{3.229425in}{1.863073in}}%
\pgfpathlineto{\pgfqpoint{3.217997in}{1.857883in}}%
\pgfpathlineto{\pgfqpoint{3.206566in}{1.853285in}}%
\pgfpathlineto{\pgfqpoint{3.195131in}{1.849100in}}%
\pgfpathlineto{\pgfqpoint{3.201461in}{1.837972in}}%
\pgfpathlineto{\pgfqpoint{3.207795in}{1.826849in}}%
\pgfpathlineto{\pgfqpoint{3.214132in}{1.815739in}}%
\pgfpathlineto{\pgfqpoint{3.220473in}{1.804643in}}%
\pgfpathclose%
\pgfusepath{stroke,fill}%
\end{pgfscope}%
\begin{pgfscope}%
\pgfpathrectangle{\pgfqpoint{0.887500in}{0.275000in}}{\pgfqpoint{4.225000in}{4.225000in}}%
\pgfusepath{clip}%
\pgfsetbuttcap%
\pgfsetroundjoin%
\definecolor{currentfill}{rgb}{0.149039,0.508051,0.557250}%
\pgfsetfillcolor{currentfill}%
\pgfsetfillopacity{0.700000}%
\pgfsetlinewidth{0.501875pt}%
\definecolor{currentstroke}{rgb}{1.000000,1.000000,1.000000}%
\pgfsetstrokecolor{currentstroke}%
\pgfsetstrokeopacity{0.500000}%
\pgfsetdash{}{0pt}%
\pgfpathmoveto{\pgfqpoint{3.806436in}{2.161149in}}%
\pgfpathlineto{\pgfqpoint{3.817741in}{2.165273in}}%
\pgfpathlineto{\pgfqpoint{3.829041in}{2.169409in}}%
\pgfpathlineto{\pgfqpoint{3.840341in}{2.173813in}}%
\pgfpathlineto{\pgfqpoint{3.851643in}{2.178520in}}%
\pgfpathlineto{\pgfqpoint{3.862944in}{2.183454in}}%
\pgfpathlineto{\pgfqpoint{3.856513in}{2.198360in}}%
\pgfpathlineto{\pgfqpoint{3.850087in}{2.213366in}}%
\pgfpathlineto{\pgfqpoint{3.843665in}{2.228501in}}%
\pgfpathlineto{\pgfqpoint{3.837248in}{2.243742in}}%
\pgfpathlineto{\pgfqpoint{3.830835in}{2.259059in}}%
\pgfpathlineto{\pgfqpoint{3.819543in}{2.254474in}}%
\pgfpathlineto{\pgfqpoint{3.808249in}{2.250045in}}%
\pgfpathlineto{\pgfqpoint{3.796954in}{2.245832in}}%
\pgfpathlineto{\pgfqpoint{3.785658in}{2.241803in}}%
\pgfpathlineto{\pgfqpoint{3.774355in}{2.237746in}}%
\pgfpathlineto{\pgfqpoint{3.780771in}{2.222630in}}%
\pgfpathlineto{\pgfqpoint{3.787187in}{2.207423in}}%
\pgfpathlineto{\pgfqpoint{3.793603in}{2.192114in}}%
\pgfpathlineto{\pgfqpoint{3.800020in}{2.176694in}}%
\pgfpathclose%
\pgfusepath{stroke,fill}%
\end{pgfscope}%
\begin{pgfscope}%
\pgfpathrectangle{\pgfqpoint{0.887500in}{0.275000in}}{\pgfqpoint{4.225000in}{4.225000in}}%
\pgfusepath{clip}%
\pgfsetbuttcap%
\pgfsetroundjoin%
\definecolor{currentfill}{rgb}{0.195860,0.395433,0.555276}%
\pgfsetfillcolor{currentfill}%
\pgfsetfillopacity{0.700000}%
\pgfsetlinewidth{0.501875pt}%
\definecolor{currentstroke}{rgb}{1.000000,1.000000,1.000000}%
\pgfsetstrokecolor{currentstroke}%
\pgfsetstrokeopacity{0.500000}%
\pgfsetdash{}{0pt}%
\pgfpathmoveto{\pgfqpoint{4.104634in}{1.924291in}}%
\pgfpathlineto{\pgfqpoint{4.115891in}{1.929249in}}%
\pgfpathlineto{\pgfqpoint{4.127138in}{1.934010in}}%
\pgfpathlineto{\pgfqpoint{4.138376in}{1.938610in}}%
\pgfpathlineto{\pgfqpoint{4.149605in}{1.943092in}}%
\pgfpathlineto{\pgfqpoint{4.160828in}{1.947496in}}%
\pgfpathlineto{\pgfqpoint{4.154320in}{1.961555in}}%
\pgfpathlineto{\pgfqpoint{4.147829in}{1.976134in}}%
\pgfpathlineto{\pgfqpoint{4.141353in}{1.991173in}}%
\pgfpathlineto{\pgfqpoint{4.134891in}{2.006590in}}%
\pgfpathlineto{\pgfqpoint{4.128438in}{2.022302in}}%
\pgfpathlineto{\pgfqpoint{4.117223in}{2.018082in}}%
\pgfpathlineto{\pgfqpoint{4.106002in}{2.013819in}}%
\pgfpathlineto{\pgfqpoint{4.094773in}{2.009497in}}%
\pgfpathlineto{\pgfqpoint{4.083538in}{2.005097in}}%
\pgfpathlineto{\pgfqpoint{4.072295in}{2.000602in}}%
\pgfpathlineto{\pgfqpoint{4.078736in}{1.984475in}}%
\pgfpathlineto{\pgfqpoint{4.085187in}{1.968685in}}%
\pgfpathlineto{\pgfqpoint{4.091652in}{1.953330in}}%
\pgfpathlineto{\pgfqpoint{4.098134in}{1.938509in}}%
\pgfpathclose%
\pgfusepath{stroke,fill}%
\end{pgfscope}%
\begin{pgfscope}%
\pgfpathrectangle{\pgfqpoint{0.887500in}{0.275000in}}{\pgfqpoint{4.225000in}{4.225000in}}%
\pgfusepath{clip}%
\pgfsetbuttcap%
\pgfsetroundjoin%
\definecolor{currentfill}{rgb}{0.133743,0.548535,0.553541}%
\pgfsetfillcolor{currentfill}%
\pgfsetfillopacity{0.700000}%
\pgfsetlinewidth{0.501875pt}%
\definecolor{currentstroke}{rgb}{1.000000,1.000000,1.000000}%
\pgfsetstrokecolor{currentstroke}%
\pgfsetstrokeopacity{0.500000}%
\pgfsetdash{}{0pt}%
\pgfpathmoveto{\pgfqpoint{3.628442in}{2.233145in}}%
\pgfpathlineto{\pgfqpoint{3.639883in}{2.243869in}}%
\pgfpathlineto{\pgfqpoint{3.651315in}{2.254108in}}%
\pgfpathlineto{\pgfqpoint{3.662736in}{2.263781in}}%
\pgfpathlineto{\pgfqpoint{3.674145in}{2.272811in}}%
\pgfpathlineto{\pgfqpoint{3.685540in}{2.281117in}}%
\pgfpathlineto{\pgfqpoint{3.679132in}{2.295589in}}%
\pgfpathlineto{\pgfqpoint{3.672729in}{2.310222in}}%
\pgfpathlineto{\pgfqpoint{3.666332in}{2.325080in}}%
\pgfpathlineto{\pgfqpoint{3.659940in}{2.340162in}}%
\pgfpathlineto{\pgfqpoint{3.653553in}{2.355426in}}%
\pgfpathlineto{\pgfqpoint{3.642180in}{2.348184in}}%
\pgfpathlineto{\pgfqpoint{3.630796in}{2.340507in}}%
\pgfpathlineto{\pgfqpoint{3.619403in}{2.332391in}}%
\pgfpathlineto{\pgfqpoint{3.608001in}{2.323832in}}%
\pgfpathlineto{\pgfqpoint{3.596590in}{2.314825in}}%
\pgfpathlineto{\pgfqpoint{3.602950in}{2.297992in}}%
\pgfpathlineto{\pgfqpoint{3.609315in}{2.281363in}}%
\pgfpathlineto{\pgfqpoint{3.615685in}{2.265015in}}%
\pgfpathlineto{\pgfqpoint{3.622061in}{2.248965in}}%
\pgfpathclose%
\pgfusepath{stroke,fill}%
\end{pgfscope}%
\begin{pgfscope}%
\pgfpathrectangle{\pgfqpoint{0.887500in}{0.275000in}}{\pgfqpoint{4.225000in}{4.225000in}}%
\pgfusepath{clip}%
\pgfsetbuttcap%
\pgfsetroundjoin%
\definecolor{currentfill}{rgb}{0.194100,0.399323,0.555565}%
\pgfsetfillcolor{currentfill}%
\pgfsetfillopacity{0.700000}%
\pgfsetlinewidth{0.501875pt}%
\definecolor{currentstroke}{rgb}{1.000000,1.000000,1.000000}%
\pgfsetstrokecolor{currentstroke}%
\pgfsetstrokeopacity{0.500000}%
\pgfsetdash{}{0pt}%
\pgfpathmoveto{\pgfqpoint{2.724732in}{1.946446in}}%
\pgfpathlineto{\pgfqpoint{2.736297in}{1.950201in}}%
\pgfpathlineto{\pgfqpoint{2.747856in}{1.953971in}}%
\pgfpathlineto{\pgfqpoint{2.759410in}{1.957752in}}%
\pgfpathlineto{\pgfqpoint{2.770957in}{1.961522in}}%
\pgfpathlineto{\pgfqpoint{2.782500in}{1.965260in}}%
\pgfpathlineto{\pgfqpoint{2.776292in}{1.975253in}}%
\pgfpathlineto{\pgfqpoint{2.770088in}{1.985199in}}%
\pgfpathlineto{\pgfqpoint{2.763889in}{1.995099in}}%
\pgfpathlineto{\pgfqpoint{2.757693in}{2.004954in}}%
\pgfpathlineto{\pgfqpoint{2.751501in}{2.014764in}}%
\pgfpathlineto{\pgfqpoint{2.739969in}{2.011077in}}%
\pgfpathlineto{\pgfqpoint{2.728431in}{2.007358in}}%
\pgfpathlineto{\pgfqpoint{2.716887in}{2.003628in}}%
\pgfpathlineto{\pgfqpoint{2.705338in}{1.999908in}}%
\pgfpathlineto{\pgfqpoint{2.693783in}{1.996205in}}%
\pgfpathlineto{\pgfqpoint{2.699964in}{1.986346in}}%
\pgfpathlineto{\pgfqpoint{2.706150in}{1.976440in}}%
\pgfpathlineto{\pgfqpoint{2.712340in}{1.966488in}}%
\pgfpathlineto{\pgfqpoint{2.718534in}{1.956490in}}%
\pgfpathclose%
\pgfusepath{stroke,fill}%
\end{pgfscope}%
\begin{pgfscope}%
\pgfpathrectangle{\pgfqpoint{0.887500in}{0.275000in}}{\pgfqpoint{4.225000in}{4.225000in}}%
\pgfusepath{clip}%
\pgfsetbuttcap%
\pgfsetroundjoin%
\definecolor{currentfill}{rgb}{0.160665,0.478540,0.558115}%
\pgfsetfillcolor{currentfill}%
\pgfsetfillopacity{0.700000}%
\pgfsetlinewidth{0.501875pt}%
\definecolor{currentstroke}{rgb}{1.000000,1.000000,1.000000}%
\pgfsetstrokecolor{currentstroke}%
\pgfsetstrokeopacity{0.500000}%
\pgfsetdash{}{0pt}%
\pgfpathmoveto{\pgfqpoint{1.987254in}{2.115903in}}%
\pgfpathlineto{\pgfqpoint{1.999001in}{2.119552in}}%
\pgfpathlineto{\pgfqpoint{2.010742in}{2.123197in}}%
\pgfpathlineto{\pgfqpoint{2.022477in}{2.126836in}}%
\pgfpathlineto{\pgfqpoint{2.034207in}{2.130467in}}%
\pgfpathlineto{\pgfqpoint{2.045932in}{2.134088in}}%
\pgfpathlineto{\pgfqpoint{2.039967in}{2.143211in}}%
\pgfpathlineto{\pgfqpoint{2.034007in}{2.152306in}}%
\pgfpathlineto{\pgfqpoint{2.028051in}{2.161373in}}%
\pgfpathlineto{\pgfqpoint{2.022100in}{2.170412in}}%
\pgfpathlineto{\pgfqpoint{2.016153in}{2.179425in}}%
\pgfpathlineto{\pgfqpoint{2.004440in}{2.175858in}}%
\pgfpathlineto{\pgfqpoint{1.992721in}{2.172281in}}%
\pgfpathlineto{\pgfqpoint{1.980996in}{2.168698in}}%
\pgfpathlineto{\pgfqpoint{1.969265in}{2.165109in}}%
\pgfpathlineto{\pgfqpoint{1.957529in}{2.161516in}}%
\pgfpathlineto{\pgfqpoint{1.963465in}{2.152450in}}%
\pgfpathlineto{\pgfqpoint{1.969406in}{2.143357in}}%
\pgfpathlineto{\pgfqpoint{1.975351in}{2.134235in}}%
\pgfpathlineto{\pgfqpoint{1.981300in}{2.125084in}}%
\pgfpathclose%
\pgfusepath{stroke,fill}%
\end{pgfscope}%
\begin{pgfscope}%
\pgfpathrectangle{\pgfqpoint{0.887500in}{0.275000in}}{\pgfqpoint{4.225000in}{4.225000in}}%
\pgfusepath{clip}%
\pgfsetbuttcap%
\pgfsetroundjoin%
\definecolor{currentfill}{rgb}{0.221989,0.339161,0.548752}%
\pgfsetfillcolor{currentfill}%
\pgfsetfillopacity{0.700000}%
\pgfsetlinewidth{0.501875pt}%
\definecolor{currentstroke}{rgb}{1.000000,1.000000,1.000000}%
\pgfsetstrokecolor{currentstroke}%
\pgfsetstrokeopacity{0.500000}%
\pgfsetdash{}{0pt}%
\pgfpathmoveto{\pgfqpoint{3.372948in}{1.790186in}}%
\pgfpathlineto{\pgfqpoint{3.384411in}{1.801726in}}%
\pgfpathlineto{\pgfqpoint{3.395885in}{1.814628in}}%
\pgfpathlineto{\pgfqpoint{3.407371in}{1.828904in}}%
\pgfpathlineto{\pgfqpoint{3.418870in}{1.844498in}}%
\pgfpathlineto{\pgfqpoint{3.430380in}{1.861131in}}%
\pgfpathlineto{\pgfqpoint{3.424028in}{1.876988in}}%
\pgfpathlineto{\pgfqpoint{3.417676in}{1.892692in}}%
\pgfpathlineto{\pgfqpoint{3.411327in}{1.908352in}}%
\pgfpathlineto{\pgfqpoint{3.404980in}{1.924073in}}%
\pgfpathlineto{\pgfqpoint{3.398636in}{1.939963in}}%
\pgfpathlineto{\pgfqpoint{3.387099in}{1.918753in}}%
\pgfpathlineto{\pgfqpoint{3.375578in}{1.898932in}}%
\pgfpathlineto{\pgfqpoint{3.364076in}{1.881019in}}%
\pgfpathlineto{\pgfqpoint{3.352591in}{1.865087in}}%
\pgfpathlineto{\pgfqpoint{3.341122in}{1.851075in}}%
\pgfpathlineto{\pgfqpoint{3.347474in}{1.837944in}}%
\pgfpathlineto{\pgfqpoint{3.353833in}{1.825432in}}%
\pgfpathlineto{\pgfqpoint{3.360199in}{1.813398in}}%
\pgfpathlineto{\pgfqpoint{3.366572in}{1.801697in}}%
\pgfpathclose%
\pgfusepath{stroke,fill}%
\end{pgfscope}%
\begin{pgfscope}%
\pgfpathrectangle{\pgfqpoint{0.887500in}{0.275000in}}{\pgfqpoint{4.225000in}{4.225000in}}%
\pgfusepath{clip}%
\pgfsetbuttcap%
\pgfsetroundjoin%
\definecolor{currentfill}{rgb}{0.180629,0.429975,0.557282}%
\pgfsetfillcolor{currentfill}%
\pgfsetfillopacity{0.700000}%
\pgfsetlinewidth{0.501875pt}%
\definecolor{currentstroke}{rgb}{1.000000,1.000000,1.000000}%
\pgfsetstrokecolor{currentstroke}%
\pgfsetstrokeopacity{0.500000}%
\pgfsetdash{}{0pt}%
\pgfpathmoveto{\pgfqpoint{3.398636in}{1.939963in}}%
\pgfpathlineto{\pgfqpoint{3.410185in}{1.961950in}}%
\pgfpathlineto{\pgfqpoint{3.421740in}{1.984101in}}%
\pgfpathlineto{\pgfqpoint{3.433297in}{2.005801in}}%
\pgfpathlineto{\pgfqpoint{3.444847in}{2.026433in}}%
\pgfpathlineto{\pgfqpoint{3.456385in}{2.045381in}}%
\pgfpathlineto{\pgfqpoint{3.450052in}{2.063624in}}%
\pgfpathlineto{\pgfqpoint{3.443717in}{2.081602in}}%
\pgfpathlineto{\pgfqpoint{3.437382in}{2.099360in}}%
\pgfpathlineto{\pgfqpoint{3.431046in}{2.116899in}}%
\pgfpathlineto{\pgfqpoint{3.424709in}{2.134189in}}%
\pgfpathlineto{\pgfqpoint{3.413176in}{2.114623in}}%
\pgfpathlineto{\pgfqpoint{3.401632in}{2.093321in}}%
\pgfpathlineto{\pgfqpoint{3.390082in}{2.070891in}}%
\pgfpathlineto{\pgfqpoint{3.378534in}{2.047939in}}%
\pgfpathlineto{\pgfqpoint{3.366992in}{2.025072in}}%
\pgfpathlineto{\pgfqpoint{3.373311in}{2.007205in}}%
\pgfpathlineto{\pgfqpoint{3.379634in}{1.989708in}}%
\pgfpathlineto{\pgfqpoint{3.385963in}{1.972678in}}%
\pgfpathlineto{\pgfqpoint{3.392297in}{1.956129in}}%
\pgfpathclose%
\pgfusepath{stroke,fill}%
\end{pgfscope}%
\begin{pgfscope}%
\pgfpathrectangle{\pgfqpoint{0.887500in}{0.275000in}}{\pgfqpoint{4.225000in}{4.225000in}}%
\pgfusepath{clip}%
\pgfsetbuttcap%
\pgfsetroundjoin%
\definecolor{currentfill}{rgb}{0.136408,0.541173,0.554483}%
\pgfsetfillcolor{currentfill}%
\pgfsetfillopacity{0.700000}%
\pgfsetlinewidth{0.501875pt}%
\definecolor{currentstroke}{rgb}{1.000000,1.000000,1.000000}%
\pgfsetstrokecolor{currentstroke}%
\pgfsetstrokeopacity{0.500000}%
\pgfsetdash{}{0pt}%
\pgfpathmoveto{\pgfqpoint{3.482152in}{2.206600in}}%
\pgfpathlineto{\pgfqpoint{3.493611in}{2.218206in}}%
\pgfpathlineto{\pgfqpoint{3.505066in}{2.229615in}}%
\pgfpathlineto{\pgfqpoint{3.516521in}{2.241002in}}%
\pgfpathlineto{\pgfqpoint{3.527973in}{2.252316in}}%
\pgfpathlineto{\pgfqpoint{3.539422in}{2.263489in}}%
\pgfpathlineto{\pgfqpoint{3.533086in}{2.281752in}}%
\pgfpathlineto{\pgfqpoint{3.526751in}{2.300000in}}%
\pgfpathlineto{\pgfqpoint{3.520416in}{2.318134in}}%
\pgfpathlineto{\pgfqpoint{3.514079in}{2.336056in}}%
\pgfpathlineto{\pgfqpoint{3.507740in}{2.353668in}}%
\pgfpathlineto{\pgfqpoint{3.496298in}{2.342856in}}%
\pgfpathlineto{\pgfqpoint{3.484849in}{2.331484in}}%
\pgfpathlineto{\pgfqpoint{3.473392in}{2.319551in}}%
\pgfpathlineto{\pgfqpoint{3.461930in}{2.307052in}}%
\pgfpathlineto{\pgfqpoint{3.450460in}{2.293930in}}%
\pgfpathlineto{\pgfqpoint{3.456802in}{2.277135in}}%
\pgfpathlineto{\pgfqpoint{3.463142in}{2.259934in}}%
\pgfpathlineto{\pgfqpoint{3.469480in}{2.242397in}}%
\pgfpathlineto{\pgfqpoint{3.475816in}{2.224595in}}%
\pgfpathclose%
\pgfusepath{stroke,fill}%
\end{pgfscope}%
\begin{pgfscope}%
\pgfpathrectangle{\pgfqpoint{0.887500in}{0.275000in}}{\pgfqpoint{4.225000in}{4.225000in}}%
\pgfusepath{clip}%
\pgfsetbuttcap%
\pgfsetroundjoin%
\definecolor{currentfill}{rgb}{0.139147,0.533812,0.555298}%
\pgfsetfillcolor{currentfill}%
\pgfsetfillopacity{0.700000}%
\pgfsetlinewidth{0.501875pt}%
\definecolor{currentstroke}{rgb}{1.000000,1.000000,1.000000}%
\pgfsetstrokecolor{currentstroke}%
\pgfsetstrokeopacity{0.500000}%
\pgfsetdash{}{0pt}%
\pgfpathmoveto{\pgfqpoint{3.717621in}{2.208870in}}%
\pgfpathlineto{\pgfqpoint{3.729009in}{2.216567in}}%
\pgfpathlineto{\pgfqpoint{3.740373in}{2.223072in}}%
\pgfpathlineto{\pgfqpoint{3.751717in}{2.228615in}}%
\pgfpathlineto{\pgfqpoint{3.763043in}{2.233428in}}%
\pgfpathlineto{\pgfqpoint{3.774355in}{2.237746in}}%
\pgfpathlineto{\pgfqpoint{3.767941in}{2.252779in}}%
\pgfpathlineto{\pgfqpoint{3.761527in}{2.267738in}}%
\pgfpathlineto{\pgfqpoint{3.755115in}{2.282633in}}%
\pgfpathlineto{\pgfqpoint{3.748704in}{2.297472in}}%
\pgfpathlineto{\pgfqpoint{3.742294in}{2.312264in}}%
\pgfpathlineto{\pgfqpoint{3.730969in}{2.307083in}}%
\pgfpathlineto{\pgfqpoint{3.719633in}{2.301496in}}%
\pgfpathlineto{\pgfqpoint{3.708284in}{2.295383in}}%
\pgfpathlineto{\pgfqpoint{3.696920in}{2.288629in}}%
\pgfpathlineto{\pgfqpoint{3.685540in}{2.281117in}}%
\pgfpathlineto{\pgfqpoint{3.691953in}{2.266742in}}%
\pgfpathlineto{\pgfqpoint{3.698368in}{2.252395in}}%
\pgfpathlineto{\pgfqpoint{3.704785in}{2.238012in}}%
\pgfpathlineto{\pgfqpoint{3.711203in}{2.223526in}}%
\pgfpathclose%
\pgfusepath{stroke,fill}%
\end{pgfscope}%
\begin{pgfscope}%
\pgfpathrectangle{\pgfqpoint{0.887500in}{0.275000in}}{\pgfqpoint{4.225000in}{4.225000in}}%
\pgfusepath{clip}%
\pgfsetbuttcap%
\pgfsetroundjoin%
\definecolor{currentfill}{rgb}{0.147607,0.511733,0.557049}%
\pgfsetfillcolor{currentfill}%
\pgfsetfillopacity{0.700000}%
\pgfsetlinewidth{0.501875pt}%
\definecolor{currentstroke}{rgb}{1.000000,1.000000,1.000000}%
\pgfsetstrokecolor{currentstroke}%
\pgfsetstrokeopacity{0.500000}%
\pgfsetdash{}{0pt}%
\pgfpathmoveto{\pgfqpoint{3.424709in}{2.134189in}}%
\pgfpathlineto{\pgfqpoint{3.436225in}{2.151678in}}%
\pgfpathlineto{\pgfqpoint{3.447725in}{2.167338in}}%
\pgfpathlineto{\pgfqpoint{3.459212in}{2.181494in}}%
\pgfpathlineto{\pgfqpoint{3.470686in}{2.194472in}}%
\pgfpathlineto{\pgfqpoint{3.482152in}{2.206600in}}%
\pgfpathlineto{\pgfqpoint{3.475816in}{2.224595in}}%
\pgfpathlineto{\pgfqpoint{3.469480in}{2.242397in}}%
\pgfpathlineto{\pgfqpoint{3.463142in}{2.259934in}}%
\pgfpathlineto{\pgfqpoint{3.456802in}{2.277135in}}%
\pgfpathlineto{\pgfqpoint{3.450460in}{2.293930in}}%
\pgfpathlineto{\pgfqpoint{3.438984in}{2.280096in}}%
\pgfpathlineto{\pgfqpoint{3.427500in}{2.265458in}}%
\pgfpathlineto{\pgfqpoint{3.416009in}{2.249926in}}%
\pgfpathlineto{\pgfqpoint{3.404510in}{2.233411in}}%
\pgfpathlineto{\pgfqpoint{3.393005in}{2.215822in}}%
\pgfpathlineto{\pgfqpoint{3.399348in}{2.200240in}}%
\pgfpathlineto{\pgfqpoint{3.405690in}{2.184255in}}%
\pgfpathlineto{\pgfqpoint{3.412031in}{2.167898in}}%
\pgfpathlineto{\pgfqpoint{3.418370in}{2.151199in}}%
\pgfpathclose%
\pgfusepath{stroke,fill}%
\end{pgfscope}%
\begin{pgfscope}%
\pgfpathrectangle{\pgfqpoint{0.887500in}{0.275000in}}{\pgfqpoint{4.225000in}{4.225000in}}%
\pgfusepath{clip}%
\pgfsetbuttcap%
\pgfsetroundjoin%
\definecolor{currentfill}{rgb}{0.235526,0.309527,0.542944}%
\pgfsetfillcolor{currentfill}%
\pgfsetfillopacity{0.700000}%
\pgfsetlinewidth{0.501875pt}%
\definecolor{currentstroke}{rgb}{1.000000,1.000000,1.000000}%
\pgfsetstrokecolor{currentstroke}%
\pgfsetstrokeopacity{0.500000}%
\pgfsetdash{}{0pt}%
\pgfpathmoveto{\pgfqpoint{3.315752in}{1.752441in}}%
\pgfpathlineto{\pgfqpoint{3.327180in}{1.757401in}}%
\pgfpathlineto{\pgfqpoint{3.338612in}{1.763613in}}%
\pgfpathlineto{\pgfqpoint{3.350050in}{1.771141in}}%
\pgfpathlineto{\pgfqpoint{3.361495in}{1.779994in}}%
\pgfpathlineto{\pgfqpoint{3.372948in}{1.790186in}}%
\pgfpathlineto{\pgfqpoint{3.366572in}{1.801697in}}%
\pgfpathlineto{\pgfqpoint{3.360199in}{1.813398in}}%
\pgfpathlineto{\pgfqpoint{3.353833in}{1.825432in}}%
\pgfpathlineto{\pgfqpoint{3.347474in}{1.837944in}}%
\pgfpathlineto{\pgfqpoint{3.341122in}{1.851075in}}%
\pgfpathlineto{\pgfqpoint{3.329667in}{1.838921in}}%
\pgfpathlineto{\pgfqpoint{3.318223in}{1.828565in}}%
\pgfpathlineto{\pgfqpoint{3.306788in}{1.819945in}}%
\pgfpathlineto{\pgfqpoint{3.295361in}{1.813002in}}%
\pgfpathlineto{\pgfqpoint{3.283939in}{1.807623in}}%
\pgfpathlineto{\pgfqpoint{3.290290in}{1.795876in}}%
\pgfpathlineto{\pgfqpoint{3.296648in}{1.784582in}}%
\pgfpathlineto{\pgfqpoint{3.303011in}{1.773643in}}%
\pgfpathlineto{\pgfqpoint{3.309379in}{1.762962in}}%
\pgfpathclose%
\pgfusepath{stroke,fill}%
\end{pgfscope}%
\begin{pgfscope}%
\pgfpathrectangle{\pgfqpoint{0.887500in}{0.275000in}}{\pgfqpoint{4.225000in}{4.225000in}}%
\pgfusepath{clip}%
\pgfsetbuttcap%
\pgfsetroundjoin%
\definecolor{currentfill}{rgb}{0.177423,0.437527,0.557565}%
\pgfsetfillcolor{currentfill}%
\pgfsetfillopacity{0.700000}%
\pgfsetlinewidth{0.501875pt}%
\definecolor{currentstroke}{rgb}{1.000000,1.000000,1.000000}%
\pgfsetstrokecolor{currentstroke}%
\pgfsetstrokeopacity{0.500000}%
\pgfsetdash{}{0pt}%
\pgfpathmoveto{\pgfqpoint{2.400300in}{2.020110in}}%
\pgfpathlineto{\pgfqpoint{2.411947in}{2.023854in}}%
\pgfpathlineto{\pgfqpoint{2.423589in}{2.027584in}}%
\pgfpathlineto{\pgfqpoint{2.435225in}{2.031299in}}%
\pgfpathlineto{\pgfqpoint{2.446856in}{2.034996in}}%
\pgfpathlineto{\pgfqpoint{2.458481in}{2.038675in}}%
\pgfpathlineto{\pgfqpoint{2.452376in}{2.048204in}}%
\pgfpathlineto{\pgfqpoint{2.446276in}{2.057692in}}%
\pgfpathlineto{\pgfqpoint{2.440179in}{2.067142in}}%
\pgfpathlineto{\pgfqpoint{2.434088in}{2.076554in}}%
\pgfpathlineto{\pgfqpoint{2.428000in}{2.085928in}}%
\pgfpathlineto{\pgfqpoint{2.416385in}{2.082299in}}%
\pgfpathlineto{\pgfqpoint{2.404765in}{2.078651in}}%
\pgfpathlineto{\pgfqpoint{2.393139in}{2.074989in}}%
\pgfpathlineto{\pgfqpoint{2.381508in}{2.071312in}}%
\pgfpathlineto{\pgfqpoint{2.369871in}{2.067623in}}%
\pgfpathlineto{\pgfqpoint{2.375948in}{2.058197in}}%
\pgfpathlineto{\pgfqpoint{2.382029in}{2.048733in}}%
\pgfpathlineto{\pgfqpoint{2.388115in}{2.039232in}}%
\pgfpathlineto{\pgfqpoint{2.394205in}{2.029691in}}%
\pgfpathclose%
\pgfusepath{stroke,fill}%
\end{pgfscope}%
\begin{pgfscope}%
\pgfpathrectangle{\pgfqpoint{0.887500in}{0.275000in}}{\pgfqpoint{4.225000in}{4.225000in}}%
\pgfusepath{clip}%
\pgfsetbuttcap%
\pgfsetroundjoin%
\definecolor{currentfill}{rgb}{0.199430,0.387607,0.554642}%
\pgfsetfillcolor{currentfill}%
\pgfsetfillopacity{0.700000}%
\pgfsetlinewidth{0.501875pt}%
\definecolor{currentstroke}{rgb}{1.000000,1.000000,1.000000}%
\pgfsetstrokecolor{currentstroke}%
\pgfsetstrokeopacity{0.500000}%
\pgfsetdash{}{0pt}%
\pgfpathmoveto{\pgfqpoint{2.813599in}{1.914593in}}%
\pgfpathlineto{\pgfqpoint{2.825146in}{1.918334in}}%
\pgfpathlineto{\pgfqpoint{2.836687in}{1.922006in}}%
\pgfpathlineto{\pgfqpoint{2.848223in}{1.925589in}}%
\pgfpathlineto{\pgfqpoint{2.859754in}{1.929064in}}%
\pgfpathlineto{\pgfqpoint{2.871279in}{1.932445in}}%
\pgfpathlineto{\pgfqpoint{2.865042in}{1.942597in}}%
\pgfpathlineto{\pgfqpoint{2.858809in}{1.952705in}}%
\pgfpathlineto{\pgfqpoint{2.852580in}{1.962767in}}%
\pgfpathlineto{\pgfqpoint{2.846355in}{1.972784in}}%
\pgfpathlineto{\pgfqpoint{2.840134in}{1.982757in}}%
\pgfpathlineto{\pgfqpoint{2.828618in}{1.979463in}}%
\pgfpathlineto{\pgfqpoint{2.817096in}{1.976068in}}%
\pgfpathlineto{\pgfqpoint{2.805569in}{1.972555in}}%
\pgfpathlineto{\pgfqpoint{2.794037in}{1.968945in}}%
\pgfpathlineto{\pgfqpoint{2.782500in}{1.965260in}}%
\pgfpathlineto{\pgfqpoint{2.788712in}{1.955221in}}%
\pgfpathlineto{\pgfqpoint{2.794927in}{1.945136in}}%
\pgfpathlineto{\pgfqpoint{2.801147in}{1.935003in}}%
\pgfpathlineto{\pgfqpoint{2.807371in}{1.924822in}}%
\pgfpathclose%
\pgfusepath{stroke,fill}%
\end{pgfscope}%
\begin{pgfscope}%
\pgfpathrectangle{\pgfqpoint{0.887500in}{0.275000in}}{\pgfqpoint{4.225000in}{4.225000in}}%
\pgfusepath{clip}%
\pgfsetbuttcap%
\pgfsetroundjoin%
\definecolor{currentfill}{rgb}{0.182256,0.426184,0.557120}%
\pgfsetfillcolor{currentfill}%
\pgfsetfillopacity{0.700000}%
\pgfsetlinewidth{0.501875pt}%
\definecolor{currentstroke}{rgb}{1.000000,1.000000,1.000000}%
\pgfsetstrokecolor{currentstroke}%
\pgfsetstrokeopacity{0.500000}%
\pgfsetdash{}{0pt}%
\pgfpathmoveto{\pgfqpoint{4.015954in}{1.976013in}}%
\pgfpathlineto{\pgfqpoint{4.027240in}{1.981279in}}%
\pgfpathlineto{\pgfqpoint{4.038517in}{1.986351in}}%
\pgfpathlineto{\pgfqpoint{4.049785in}{1.991249in}}%
\pgfpathlineto{\pgfqpoint{4.061044in}{1.995993in}}%
\pgfpathlineto{\pgfqpoint{4.072295in}{2.000602in}}%
\pgfpathlineto{\pgfqpoint{4.065862in}{2.016966in}}%
\pgfpathlineto{\pgfqpoint{4.059435in}{2.033469in}}%
\pgfpathlineto{\pgfqpoint{4.053011in}{2.050010in}}%
\pgfpathlineto{\pgfqpoint{4.046587in}{2.066492in}}%
\pgfpathlineto{\pgfqpoint{4.040161in}{2.082813in}}%
\pgfpathlineto{\pgfqpoint{4.028916in}{2.078444in}}%
\pgfpathlineto{\pgfqpoint{4.017662in}{2.073892in}}%
\pgfpathlineto{\pgfqpoint{4.006398in}{2.069150in}}%
\pgfpathlineto{\pgfqpoint{3.995125in}{2.064211in}}%
\pgfpathlineto{\pgfqpoint{3.983842in}{2.059067in}}%
\pgfpathlineto{\pgfqpoint{3.990266in}{2.042670in}}%
\pgfpathlineto{\pgfqpoint{3.996687in}{2.026086in}}%
\pgfpathlineto{\pgfqpoint{4.003108in}{2.009395in}}%
\pgfpathlineto{\pgfqpoint{4.009529in}{1.992677in}}%
\pgfpathclose%
\pgfusepath{stroke,fill}%
\end{pgfscope}%
\begin{pgfscope}%
\pgfpathrectangle{\pgfqpoint{0.887500in}{0.275000in}}{\pgfqpoint{4.225000in}{4.225000in}}%
\pgfusepath{clip}%
\pgfsetbuttcap%
\pgfsetroundjoin%
\definecolor{currentfill}{rgb}{0.163625,0.471133,0.558148}%
\pgfsetfillcolor{currentfill}%
\pgfsetfillopacity{0.700000}%
\pgfsetlinewidth{0.501875pt}%
\definecolor{currentstroke}{rgb}{1.000000,1.000000,1.000000}%
\pgfsetstrokecolor{currentstroke}%
\pgfsetstrokeopacity{0.500000}%
\pgfsetdash{}{0pt}%
\pgfpathmoveto{\pgfqpoint{2.075821in}{2.088021in}}%
\pgfpathlineto{\pgfqpoint{2.087550in}{2.091686in}}%
\pgfpathlineto{\pgfqpoint{2.099274in}{2.095336in}}%
\pgfpathlineto{\pgfqpoint{2.110993in}{2.098972in}}%
\pgfpathlineto{\pgfqpoint{2.122706in}{2.102595in}}%
\pgfpathlineto{\pgfqpoint{2.134413in}{2.106206in}}%
\pgfpathlineto{\pgfqpoint{2.128416in}{2.115427in}}%
\pgfpathlineto{\pgfqpoint{2.122423in}{2.124618in}}%
\pgfpathlineto{\pgfqpoint{2.116434in}{2.133778in}}%
\pgfpathlineto{\pgfqpoint{2.110450in}{2.142909in}}%
\pgfpathlineto{\pgfqpoint{2.104471in}{2.152011in}}%
\pgfpathlineto{\pgfqpoint{2.092774in}{2.148450in}}%
\pgfpathlineto{\pgfqpoint{2.081071in}{2.144877in}}%
\pgfpathlineto{\pgfqpoint{2.069364in}{2.141293in}}%
\pgfpathlineto{\pgfqpoint{2.057650in}{2.137697in}}%
\pgfpathlineto{\pgfqpoint{2.045932in}{2.134088in}}%
\pgfpathlineto{\pgfqpoint{2.051900in}{2.124935in}}%
\pgfpathlineto{\pgfqpoint{2.057874in}{2.115753in}}%
\pgfpathlineto{\pgfqpoint{2.063852in}{2.106540in}}%
\pgfpathlineto{\pgfqpoint{2.069834in}{2.097297in}}%
\pgfpathclose%
\pgfusepath{stroke,fill}%
\end{pgfscope}%
\begin{pgfscope}%
\pgfpathrectangle{\pgfqpoint{0.887500in}{0.275000in}}{\pgfqpoint{4.225000in}{4.225000in}}%
\pgfusepath{clip}%
\pgfsetbuttcap%
\pgfsetroundjoin%
\definecolor{currentfill}{rgb}{0.140536,0.530132,0.555659}%
\pgfsetfillcolor{currentfill}%
\pgfsetfillopacity{0.700000}%
\pgfsetlinewidth{0.501875pt}%
\definecolor{currentstroke}{rgb}{1.000000,1.000000,1.000000}%
\pgfsetstrokecolor{currentstroke}%
\pgfsetstrokeopacity{0.500000}%
\pgfsetdash{}{0pt}%
\pgfpathmoveto{\pgfqpoint{3.571158in}{2.174983in}}%
\pgfpathlineto{\pgfqpoint{3.582622in}{2.186965in}}%
\pgfpathlineto{\pgfqpoint{3.594083in}{2.198847in}}%
\pgfpathlineto{\pgfqpoint{3.605541in}{2.210556in}}%
\pgfpathlineto{\pgfqpoint{3.616994in}{2.222014in}}%
\pgfpathlineto{\pgfqpoint{3.628442in}{2.233145in}}%
\pgfpathlineto{\pgfqpoint{3.622061in}{2.248965in}}%
\pgfpathlineto{\pgfqpoint{3.615685in}{2.265015in}}%
\pgfpathlineto{\pgfqpoint{3.609315in}{2.281363in}}%
\pgfpathlineto{\pgfqpoint{3.602950in}{2.297992in}}%
\pgfpathlineto{\pgfqpoint{3.596590in}{2.314825in}}%
\pgfpathlineto{\pgfqpoint{3.585171in}{2.305369in}}%
\pgfpathlineto{\pgfqpoint{3.573744in}{2.295459in}}%
\pgfpathlineto{\pgfqpoint{3.562309in}{2.285130in}}%
\pgfpathlineto{\pgfqpoint{3.550868in}{2.274450in}}%
\pgfpathlineto{\pgfqpoint{3.539422in}{2.263489in}}%
\pgfpathlineto{\pgfqpoint{3.545761in}{2.245307in}}%
\pgfpathlineto{\pgfqpoint{3.552103in}{2.227305in}}%
\pgfpathlineto{\pgfqpoint{3.558450in}{2.209578in}}%
\pgfpathlineto{\pgfqpoint{3.564802in}{2.192158in}}%
\pgfpathclose%
\pgfusepath{stroke,fill}%
\end{pgfscope}%
\begin{pgfscope}%
\pgfpathrectangle{\pgfqpoint{0.887500in}{0.275000in}}{\pgfqpoint{4.225000in}{4.225000in}}%
\pgfusepath{clip}%
\pgfsetbuttcap%
\pgfsetroundjoin%
\definecolor{currentfill}{rgb}{0.151918,0.500685,0.557587}%
\pgfsetfillcolor{currentfill}%
\pgfsetfillopacity{0.700000}%
\pgfsetlinewidth{0.501875pt}%
\definecolor{currentstroke}{rgb}{1.000000,1.000000,1.000000}%
\pgfsetstrokecolor{currentstroke}%
\pgfsetstrokeopacity{0.500000}%
\pgfsetdash{}{0pt}%
\pgfpathmoveto{\pgfqpoint{1.751350in}{2.152830in}}%
\pgfpathlineto{\pgfqpoint{1.763160in}{2.156465in}}%
\pgfpathlineto{\pgfqpoint{1.774964in}{2.160086in}}%
\pgfpathlineto{\pgfqpoint{1.786764in}{2.163694in}}%
\pgfpathlineto{\pgfqpoint{1.798557in}{2.167291in}}%
\pgfpathlineto{\pgfqpoint{1.810346in}{2.170877in}}%
\pgfpathlineto{\pgfqpoint{1.804458in}{2.179873in}}%
\pgfpathlineto{\pgfqpoint{1.798576in}{2.188841in}}%
\pgfpathlineto{\pgfqpoint{1.792698in}{2.197782in}}%
\pgfpathlineto{\pgfqpoint{1.786824in}{2.206695in}}%
\pgfpathlineto{\pgfqpoint{1.780955in}{2.215581in}}%
\pgfpathlineto{\pgfqpoint{1.769178in}{2.212043in}}%
\pgfpathlineto{\pgfqpoint{1.757395in}{2.208496in}}%
\pgfpathlineto{\pgfqpoint{1.745606in}{2.204937in}}%
\pgfpathlineto{\pgfqpoint{1.733813in}{2.201366in}}%
\pgfpathlineto{\pgfqpoint{1.722014in}{2.197781in}}%
\pgfpathlineto{\pgfqpoint{1.727872in}{2.188845in}}%
\pgfpathlineto{\pgfqpoint{1.733734in}{2.179883in}}%
\pgfpathlineto{\pgfqpoint{1.739601in}{2.170893in}}%
\pgfpathlineto{\pgfqpoint{1.745473in}{2.161876in}}%
\pgfpathclose%
\pgfusepath{stroke,fill}%
\end{pgfscope}%
\begin{pgfscope}%
\pgfpathrectangle{\pgfqpoint{0.887500in}{0.275000in}}{\pgfqpoint{4.225000in}{4.225000in}}%
\pgfusepath{clip}%
\pgfsetbuttcap%
\pgfsetroundjoin%
\definecolor{currentfill}{rgb}{0.244972,0.287675,0.537260}%
\pgfsetfillcolor{currentfill}%
\pgfsetfillopacity{0.700000}%
\pgfsetlinewidth{0.501875pt}%
\definecolor{currentstroke}{rgb}{1.000000,1.000000,1.000000}%
\pgfsetstrokecolor{currentstroke}%
\pgfsetstrokeopacity{0.500000}%
\pgfsetdash{}{0pt}%
\pgfpathmoveto{\pgfqpoint{4.403597in}{1.709198in}}%
\pgfpathlineto{\pgfqpoint{4.414744in}{1.712678in}}%
\pgfpathlineto{\pgfqpoint{4.425884in}{1.716150in}}%
\pgfpathlineto{\pgfqpoint{4.437019in}{1.719606in}}%
\pgfpathlineto{\pgfqpoint{4.448147in}{1.723043in}}%
\pgfpathlineto{\pgfqpoint{4.459269in}{1.726455in}}%
\pgfpathlineto{\pgfqpoint{4.452731in}{1.741453in}}%
\pgfpathlineto{\pgfqpoint{4.446193in}{1.756357in}}%
\pgfpathlineto{\pgfqpoint{4.439653in}{1.771173in}}%
\pgfpathlineto{\pgfqpoint{4.433114in}{1.785905in}}%
\pgfpathlineto{\pgfqpoint{4.426574in}{1.800556in}}%
\pgfpathlineto{\pgfqpoint{4.415465in}{1.797491in}}%
\pgfpathlineto{\pgfqpoint{4.404349in}{1.794401in}}%
\pgfpathlineto{\pgfqpoint{4.393227in}{1.791280in}}%
\pgfpathlineto{\pgfqpoint{4.382098in}{1.788120in}}%
\pgfpathlineto{\pgfqpoint{4.370961in}{1.784915in}}%
\pgfpathlineto{\pgfqpoint{4.377500in}{1.770274in}}%
\pgfpathlineto{\pgfqpoint{4.384035in}{1.755405in}}%
\pgfpathlineto{\pgfqpoint{4.390563in}{1.740284in}}%
\pgfpathlineto{\pgfqpoint{4.397084in}{1.724889in}}%
\pgfpathclose%
\pgfusepath{stroke,fill}%
\end{pgfscope}%
\begin{pgfscope}%
\pgfpathrectangle{\pgfqpoint{0.887500in}{0.275000in}}{\pgfqpoint{4.225000in}{4.225000in}}%
\pgfusepath{clip}%
\pgfsetbuttcap%
\pgfsetroundjoin%
\definecolor{currentfill}{rgb}{0.206756,0.371758,0.553117}%
\pgfsetfillcolor{currentfill}%
\pgfsetfillopacity{0.700000}%
\pgfsetlinewidth{0.501875pt}%
\definecolor{currentstroke}{rgb}{1.000000,1.000000,1.000000}%
\pgfsetstrokecolor{currentstroke}%
\pgfsetstrokeopacity{0.500000}%
\pgfsetdash{}{0pt}%
\pgfpathmoveto{\pgfqpoint{2.902522in}{1.880983in}}%
\pgfpathlineto{\pgfqpoint{2.914051in}{1.884441in}}%
\pgfpathlineto{\pgfqpoint{2.925574in}{1.887913in}}%
\pgfpathlineto{\pgfqpoint{2.937091in}{1.891446in}}%
\pgfpathlineto{\pgfqpoint{2.948601in}{1.895089in}}%
\pgfpathlineto{\pgfqpoint{2.960106in}{1.898890in}}%
\pgfpathlineto{\pgfqpoint{2.953841in}{1.909224in}}%
\pgfpathlineto{\pgfqpoint{2.947579in}{1.919511in}}%
\pgfpathlineto{\pgfqpoint{2.941321in}{1.929751in}}%
\pgfpathlineto{\pgfqpoint{2.935068in}{1.939945in}}%
\pgfpathlineto{\pgfqpoint{2.928817in}{1.950091in}}%
\pgfpathlineto{\pgfqpoint{2.917322in}{1.946254in}}%
\pgfpathlineto{\pgfqpoint{2.905821in}{1.942634in}}%
\pgfpathlineto{\pgfqpoint{2.894313in}{1.939167in}}%
\pgfpathlineto{\pgfqpoint{2.882799in}{1.935792in}}%
\pgfpathlineto{\pgfqpoint{2.871279in}{1.932445in}}%
\pgfpathlineto{\pgfqpoint{2.877520in}{1.922246in}}%
\pgfpathlineto{\pgfqpoint{2.883765in}{1.912001in}}%
\pgfpathlineto{\pgfqpoint{2.890014in}{1.901709in}}%
\pgfpathlineto{\pgfqpoint{2.896266in}{1.891370in}}%
\pgfpathclose%
\pgfusepath{stroke,fill}%
\end{pgfscope}%
\begin{pgfscope}%
\pgfpathrectangle{\pgfqpoint{0.887500in}{0.275000in}}{\pgfqpoint{4.225000in}{4.225000in}}%
\pgfusepath{clip}%
\pgfsetbuttcap%
\pgfsetroundjoin%
\definecolor{currentfill}{rgb}{0.182256,0.426184,0.557120}%
\pgfsetfillcolor{currentfill}%
\pgfsetfillopacity{0.700000}%
\pgfsetlinewidth{0.501875pt}%
\definecolor{currentstroke}{rgb}{1.000000,1.000000,1.000000}%
\pgfsetstrokecolor{currentstroke}%
\pgfsetstrokeopacity{0.500000}%
\pgfsetdash{}{0pt}%
\pgfpathmoveto{\pgfqpoint{2.489071in}{1.990393in}}%
\pgfpathlineto{\pgfqpoint{2.500701in}{1.994103in}}%
\pgfpathlineto{\pgfqpoint{2.512326in}{1.997792in}}%
\pgfpathlineto{\pgfqpoint{2.523945in}{2.001460in}}%
\pgfpathlineto{\pgfqpoint{2.535558in}{2.005106in}}%
\pgfpathlineto{\pgfqpoint{2.547167in}{2.008734in}}%
\pgfpathlineto{\pgfqpoint{2.541030in}{2.018426in}}%
\pgfpathlineto{\pgfqpoint{2.534898in}{2.028073in}}%
\pgfpathlineto{\pgfqpoint{2.528770in}{2.037677in}}%
\pgfpathlineto{\pgfqpoint{2.522646in}{2.047238in}}%
\pgfpathlineto{\pgfqpoint{2.516526in}{2.056757in}}%
\pgfpathlineto{\pgfqpoint{2.504928in}{2.053183in}}%
\pgfpathlineto{\pgfqpoint{2.493325in}{2.049590in}}%
\pgfpathlineto{\pgfqpoint{2.481715in}{2.045973in}}%
\pgfpathlineto{\pgfqpoint{2.470101in}{2.042335in}}%
\pgfpathlineto{\pgfqpoint{2.458481in}{2.038675in}}%
\pgfpathlineto{\pgfqpoint{2.464590in}{2.029106in}}%
\pgfpathlineto{\pgfqpoint{2.470704in}{2.019494in}}%
\pgfpathlineto{\pgfqpoint{2.476822in}{2.009838in}}%
\pgfpathlineto{\pgfqpoint{2.482944in}{2.000139in}}%
\pgfpathclose%
\pgfusepath{stroke,fill}%
\end{pgfscope}%
\begin{pgfscope}%
\pgfpathrectangle{\pgfqpoint{0.887500in}{0.275000in}}{\pgfqpoint{4.225000in}{4.225000in}}%
\pgfusepath{clip}%
\pgfsetbuttcap%
\pgfsetroundjoin%
\definecolor{currentfill}{rgb}{0.171176,0.452530,0.557965}%
\pgfsetfillcolor{currentfill}%
\pgfsetfillopacity{0.700000}%
\pgfsetlinewidth{0.501875pt}%
\definecolor{currentstroke}{rgb}{1.000000,1.000000,1.000000}%
\pgfsetstrokecolor{currentstroke}%
\pgfsetstrokeopacity{0.500000}%
\pgfsetdash{}{0pt}%
\pgfpathmoveto{\pgfqpoint{3.927284in}{2.030060in}}%
\pgfpathlineto{\pgfqpoint{3.938615in}{2.036316in}}%
\pgfpathlineto{\pgfqpoint{3.949935in}{2.042340in}}%
\pgfpathlineto{\pgfqpoint{3.961247in}{2.048137in}}%
\pgfpathlineto{\pgfqpoint{3.972549in}{2.053711in}}%
\pgfpathlineto{\pgfqpoint{3.983842in}{2.059067in}}%
\pgfpathlineto{\pgfqpoint{3.977413in}{2.075195in}}%
\pgfpathlineto{\pgfqpoint{3.970979in}{2.090980in}}%
\pgfpathlineto{\pgfqpoint{3.964538in}{2.106414in}}%
\pgfpathlineto{\pgfqpoint{3.958093in}{2.121546in}}%
\pgfpathlineto{\pgfqpoint{3.951645in}{2.136427in}}%
\pgfpathlineto{\pgfqpoint{3.940357in}{2.131186in}}%
\pgfpathlineto{\pgfqpoint{3.929061in}{2.125764in}}%
\pgfpathlineto{\pgfqpoint{3.917757in}{2.120201in}}%
\pgfpathlineto{\pgfqpoint{3.906446in}{2.114548in}}%
\pgfpathlineto{\pgfqpoint{3.895131in}{2.108852in}}%
\pgfpathlineto{\pgfqpoint{3.901568in}{2.093602in}}%
\pgfpathlineto{\pgfqpoint{3.908003in}{2.078138in}}%
\pgfpathlineto{\pgfqpoint{3.914435in}{2.062415in}}%
\pgfpathlineto{\pgfqpoint{3.920862in}{2.046390in}}%
\pgfpathclose%
\pgfusepath{stroke,fill}%
\end{pgfscope}%
\begin{pgfscope}%
\pgfpathrectangle{\pgfqpoint{0.887500in}{0.275000in}}{\pgfqpoint{4.225000in}{4.225000in}}%
\pgfusepath{clip}%
\pgfsetbuttcap%
\pgfsetroundjoin%
\definecolor{currentfill}{rgb}{0.150476,0.504369,0.557430}%
\pgfsetfillcolor{currentfill}%
\pgfsetfillopacity{0.700000}%
\pgfsetlinewidth{0.501875pt}%
\definecolor{currentstroke}{rgb}{1.000000,1.000000,1.000000}%
\pgfsetstrokecolor{currentstroke}%
\pgfsetstrokeopacity{0.500000}%
\pgfsetdash{}{0pt}%
\pgfpathmoveto{\pgfqpoint{3.513840in}{2.115924in}}%
\pgfpathlineto{\pgfqpoint{3.525304in}{2.127567in}}%
\pgfpathlineto{\pgfqpoint{3.536766in}{2.139206in}}%
\pgfpathlineto{\pgfqpoint{3.548230in}{2.151028in}}%
\pgfpathlineto{\pgfqpoint{3.559694in}{2.162979in}}%
\pgfpathlineto{\pgfqpoint{3.571158in}{2.174983in}}%
\pgfpathlineto{\pgfqpoint{3.564802in}{2.192158in}}%
\pgfpathlineto{\pgfqpoint{3.558450in}{2.209578in}}%
\pgfpathlineto{\pgfqpoint{3.552103in}{2.227305in}}%
\pgfpathlineto{\pgfqpoint{3.545761in}{2.245307in}}%
\pgfpathlineto{\pgfqpoint{3.539422in}{2.263489in}}%
\pgfpathlineto{\pgfqpoint{3.527973in}{2.252316in}}%
\pgfpathlineto{\pgfqpoint{3.516521in}{2.241002in}}%
\pgfpathlineto{\pgfqpoint{3.505066in}{2.229615in}}%
\pgfpathlineto{\pgfqpoint{3.493611in}{2.218206in}}%
\pgfpathlineto{\pgfqpoint{3.482152in}{2.206600in}}%
\pgfpathlineto{\pgfqpoint{3.488487in}{2.188483in}}%
\pgfpathlineto{\pgfqpoint{3.494823in}{2.170313in}}%
\pgfpathlineto{\pgfqpoint{3.501161in}{2.152162in}}%
\pgfpathlineto{\pgfqpoint{3.507500in}{2.134049in}}%
\pgfpathclose%
\pgfusepath{stroke,fill}%
\end{pgfscope}%
\begin{pgfscope}%
\pgfpathrectangle{\pgfqpoint{0.887500in}{0.275000in}}{\pgfqpoint{4.225000in}{4.225000in}}%
\pgfusepath{clip}%
\pgfsetbuttcap%
\pgfsetroundjoin%
\definecolor{currentfill}{rgb}{0.197636,0.391528,0.554969}%
\pgfsetfillcolor{currentfill}%
\pgfsetfillopacity{0.700000}%
\pgfsetlinewidth{0.501875pt}%
\definecolor{currentstroke}{rgb}{1.000000,1.000000,1.000000}%
\pgfsetstrokecolor{currentstroke}%
\pgfsetstrokeopacity{0.500000}%
\pgfsetdash{}{0pt}%
\pgfpathmoveto{\pgfqpoint{3.430380in}{1.861131in}}%
\pgfpathlineto{\pgfqpoint{3.441899in}{1.878479in}}%
\pgfpathlineto{\pgfqpoint{3.453424in}{1.896215in}}%
\pgfpathlineto{\pgfqpoint{3.464953in}{1.914015in}}%
\pgfpathlineto{\pgfqpoint{3.476481in}{1.931551in}}%
\pgfpathlineto{\pgfqpoint{3.488005in}{1.948496in}}%
\pgfpathlineto{\pgfqpoint{3.481690in}{1.968789in}}%
\pgfpathlineto{\pgfqpoint{3.475369in}{1.988576in}}%
\pgfpathlineto{\pgfqpoint{3.469044in}{2.007905in}}%
\pgfpathlineto{\pgfqpoint{3.462716in}{2.026824in}}%
\pgfpathlineto{\pgfqpoint{3.456385in}{2.045381in}}%
\pgfpathlineto{\pgfqpoint{3.444847in}{2.026433in}}%
\pgfpathlineto{\pgfqpoint{3.433297in}{2.005801in}}%
\pgfpathlineto{\pgfqpoint{3.421740in}{1.984101in}}%
\pgfpathlineto{\pgfqpoint{3.410185in}{1.961950in}}%
\pgfpathlineto{\pgfqpoint{3.398636in}{1.939963in}}%
\pgfpathlineto{\pgfqpoint{3.404980in}{1.924073in}}%
\pgfpathlineto{\pgfqpoint{3.411327in}{1.908352in}}%
\pgfpathlineto{\pgfqpoint{3.417676in}{1.892692in}}%
\pgfpathlineto{\pgfqpoint{3.424028in}{1.876988in}}%
\pgfpathclose%
\pgfusepath{stroke,fill}%
\end{pgfscope}%
\begin{pgfscope}%
\pgfpathrectangle{\pgfqpoint{0.887500in}{0.275000in}}{\pgfqpoint{4.225000in}{4.225000in}}%
\pgfusepath{clip}%
\pgfsetbuttcap%
\pgfsetroundjoin%
\definecolor{currentfill}{rgb}{0.162142,0.474838,0.558140}%
\pgfsetfillcolor{currentfill}%
\pgfsetfillopacity{0.700000}%
\pgfsetlinewidth{0.501875pt}%
\definecolor{currentstroke}{rgb}{1.000000,1.000000,1.000000}%
\pgfsetstrokecolor{currentstroke}%
\pgfsetstrokeopacity{0.500000}%
\pgfsetdash{}{0pt}%
\pgfpathmoveto{\pgfqpoint{3.456385in}{2.045381in}}%
\pgfpathlineto{\pgfqpoint{3.467904in}{2.062303in}}%
\pgfpathlineto{\pgfqpoint{3.479407in}{2.077468in}}%
\pgfpathlineto{\pgfqpoint{3.490896in}{2.091226in}}%
\pgfpathlineto{\pgfqpoint{3.502372in}{2.103927in}}%
\pgfpathlineto{\pgfqpoint{3.513840in}{2.115924in}}%
\pgfpathlineto{\pgfqpoint{3.507500in}{2.134049in}}%
\pgfpathlineto{\pgfqpoint{3.501161in}{2.152162in}}%
\pgfpathlineto{\pgfqpoint{3.494823in}{2.170313in}}%
\pgfpathlineto{\pgfqpoint{3.488487in}{2.188483in}}%
\pgfpathlineto{\pgfqpoint{3.482152in}{2.206600in}}%
\pgfpathlineto{\pgfqpoint{3.470686in}{2.194472in}}%
\pgfpathlineto{\pgfqpoint{3.459212in}{2.181494in}}%
\pgfpathlineto{\pgfqpoint{3.447725in}{2.167338in}}%
\pgfpathlineto{\pgfqpoint{3.436225in}{2.151678in}}%
\pgfpathlineto{\pgfqpoint{3.424709in}{2.134189in}}%
\pgfpathlineto{\pgfqpoint{3.431046in}{2.116899in}}%
\pgfpathlineto{\pgfqpoint{3.437382in}{2.099360in}}%
\pgfpathlineto{\pgfqpoint{3.443717in}{2.081602in}}%
\pgfpathlineto{\pgfqpoint{3.450052in}{2.063624in}}%
\pgfpathclose%
\pgfusepath{stroke,fill}%
\end{pgfscope}%
\begin{pgfscope}%
\pgfpathrectangle{\pgfqpoint{0.887500in}{0.275000in}}{\pgfqpoint{4.225000in}{4.225000in}}%
\pgfusepath{clip}%
\pgfsetbuttcap%
\pgfsetroundjoin%
\definecolor{currentfill}{rgb}{0.168126,0.459988,0.558082}%
\pgfsetfillcolor{currentfill}%
\pgfsetfillopacity{0.700000}%
\pgfsetlinewidth{0.501875pt}%
\definecolor{currentstroke}{rgb}{1.000000,1.000000,1.000000}%
\pgfsetstrokecolor{currentstroke}%
\pgfsetstrokeopacity{0.500000}%
\pgfsetdash{}{0pt}%
\pgfpathmoveto{\pgfqpoint{2.164467in}{2.059621in}}%
\pgfpathlineto{\pgfqpoint{2.176179in}{2.063278in}}%
\pgfpathlineto{\pgfqpoint{2.187886in}{2.066925in}}%
\pgfpathlineto{\pgfqpoint{2.199587in}{2.070563in}}%
\pgfpathlineto{\pgfqpoint{2.211282in}{2.074194in}}%
\pgfpathlineto{\pgfqpoint{2.222972in}{2.077821in}}%
\pgfpathlineto{\pgfqpoint{2.216942in}{2.087147in}}%
\pgfpathlineto{\pgfqpoint{2.210917in}{2.096439in}}%
\pgfpathlineto{\pgfqpoint{2.204896in}{2.105699in}}%
\pgfpathlineto{\pgfqpoint{2.198879in}{2.114928in}}%
\pgfpathlineto{\pgfqpoint{2.192867in}{2.124126in}}%
\pgfpathlineto{\pgfqpoint{2.181187in}{2.120555in}}%
\pgfpathlineto{\pgfqpoint{2.169502in}{2.116980in}}%
\pgfpathlineto{\pgfqpoint{2.157811in}{2.113398in}}%
\pgfpathlineto{\pgfqpoint{2.146115in}{2.109807in}}%
\pgfpathlineto{\pgfqpoint{2.134413in}{2.106206in}}%
\pgfpathlineto{\pgfqpoint{2.140415in}{2.096954in}}%
\pgfpathlineto{\pgfqpoint{2.146421in}{2.087670in}}%
\pgfpathlineto{\pgfqpoint{2.152432in}{2.078354in}}%
\pgfpathlineto{\pgfqpoint{2.158447in}{2.069005in}}%
\pgfpathclose%
\pgfusepath{stroke,fill}%
\end{pgfscope}%
\begin{pgfscope}%
\pgfpathrectangle{\pgfqpoint{0.887500in}{0.275000in}}{\pgfqpoint{4.225000in}{4.225000in}}%
\pgfusepath{clip}%
\pgfsetbuttcap%
\pgfsetroundjoin%
\definecolor{currentfill}{rgb}{0.214298,0.355619,0.551184}%
\pgfsetfillcolor{currentfill}%
\pgfsetfillopacity{0.700000}%
\pgfsetlinewidth{0.501875pt}%
\definecolor{currentstroke}{rgb}{1.000000,1.000000,1.000000}%
\pgfsetstrokecolor{currentstroke}%
\pgfsetstrokeopacity{0.500000}%
\pgfsetdash{}{0pt}%
\pgfpathmoveto{\pgfqpoint{2.991489in}{1.846505in}}%
\pgfpathlineto{\pgfqpoint{3.002997in}{1.850356in}}%
\pgfpathlineto{\pgfqpoint{3.014500in}{1.854290in}}%
\pgfpathlineto{\pgfqpoint{3.025997in}{1.858234in}}%
\pgfpathlineto{\pgfqpoint{3.037488in}{1.862110in}}%
\pgfpathlineto{\pgfqpoint{3.048974in}{1.865844in}}%
\pgfpathlineto{\pgfqpoint{3.042681in}{1.876565in}}%
\pgfpathlineto{\pgfqpoint{3.036392in}{1.887237in}}%
\pgfpathlineto{\pgfqpoint{3.030106in}{1.897855in}}%
\pgfpathlineto{\pgfqpoint{3.023823in}{1.908420in}}%
\pgfpathlineto{\pgfqpoint{3.017545in}{1.918928in}}%
\pgfpathlineto{\pgfqpoint{3.006068in}{1.915211in}}%
\pgfpathlineto{\pgfqpoint{2.994585in}{1.911200in}}%
\pgfpathlineto{\pgfqpoint{2.983098in}{1.907045in}}%
\pgfpathlineto{\pgfqpoint{2.971605in}{1.902893in}}%
\pgfpathlineto{\pgfqpoint{2.960106in}{1.898890in}}%
\pgfpathlineto{\pgfqpoint{2.966375in}{1.888509in}}%
\pgfpathlineto{\pgfqpoint{2.972648in}{1.878081in}}%
\pgfpathlineto{\pgfqpoint{2.978925in}{1.867605in}}%
\pgfpathlineto{\pgfqpoint{2.985205in}{1.857080in}}%
\pgfpathclose%
\pgfusepath{stroke,fill}%
\end{pgfscope}%
\begin{pgfscope}%
\pgfpathrectangle{\pgfqpoint{0.887500in}{0.275000in}}{\pgfqpoint{4.225000in}{4.225000in}}%
\pgfusepath{clip}%
\pgfsetbuttcap%
\pgfsetroundjoin%
\definecolor{currentfill}{rgb}{0.160665,0.478540,0.558115}%
\pgfsetfillcolor{currentfill}%
\pgfsetfillopacity{0.700000}%
\pgfsetlinewidth{0.501875pt}%
\definecolor{currentstroke}{rgb}{1.000000,1.000000,1.000000}%
\pgfsetstrokecolor{currentstroke}%
\pgfsetstrokeopacity{0.500000}%
\pgfsetdash{}{0pt}%
\pgfpathmoveto{\pgfqpoint{3.838499in}{2.080917in}}%
\pgfpathlineto{\pgfqpoint{3.849835in}{2.086522in}}%
\pgfpathlineto{\pgfqpoint{3.861163in}{2.091990in}}%
\pgfpathlineto{\pgfqpoint{3.872488in}{2.097525in}}%
\pgfpathlineto{\pgfqpoint{3.883811in}{2.103161in}}%
\pgfpathlineto{\pgfqpoint{3.895131in}{2.108852in}}%
\pgfpathlineto{\pgfqpoint{3.888692in}{2.123932in}}%
\pgfpathlineto{\pgfqpoint{3.882253in}{2.138887in}}%
\pgfpathlineto{\pgfqpoint{3.875815in}{2.153762in}}%
\pgfpathlineto{\pgfqpoint{3.869378in}{2.168603in}}%
\pgfpathlineto{\pgfqpoint{3.862944in}{2.183454in}}%
\pgfpathlineto{\pgfqpoint{3.851643in}{2.178520in}}%
\pgfpathlineto{\pgfqpoint{3.840341in}{2.173813in}}%
\pgfpathlineto{\pgfqpoint{3.829041in}{2.169409in}}%
\pgfpathlineto{\pgfqpoint{3.817741in}{2.165273in}}%
\pgfpathlineto{\pgfqpoint{3.806436in}{2.161149in}}%
\pgfpathlineto{\pgfqpoint{3.812851in}{2.145459in}}%
\pgfpathlineto{\pgfqpoint{3.819266in}{2.129608in}}%
\pgfpathlineto{\pgfqpoint{3.825679in}{2.113579in}}%
\pgfpathlineto{\pgfqpoint{3.832090in}{2.097355in}}%
\pgfpathclose%
\pgfusepath{stroke,fill}%
\end{pgfscope}%
\begin{pgfscope}%
\pgfpathrectangle{\pgfqpoint{0.887500in}{0.275000in}}{\pgfqpoint{4.225000in}{4.225000in}}%
\pgfusepath{clip}%
\pgfsetbuttcap%
\pgfsetroundjoin%
\definecolor{currentfill}{rgb}{0.231674,0.318106,0.544834}%
\pgfsetfillcolor{currentfill}%
\pgfsetfillopacity{0.700000}%
\pgfsetlinewidth{0.501875pt}%
\definecolor{currentstroke}{rgb}{1.000000,1.000000,1.000000}%
\pgfsetstrokecolor{currentstroke}%
\pgfsetstrokeopacity{0.500000}%
\pgfsetdash{}{0pt}%
\pgfpathmoveto{\pgfqpoint{4.315144in}{1.767399in}}%
\pgfpathlineto{\pgfqpoint{4.326329in}{1.771226in}}%
\pgfpathlineto{\pgfqpoint{4.337502in}{1.774851in}}%
\pgfpathlineto{\pgfqpoint{4.348664in}{1.778314in}}%
\pgfpathlineto{\pgfqpoint{4.359817in}{1.781656in}}%
\pgfpathlineto{\pgfqpoint{4.370961in}{1.784915in}}%
\pgfpathlineto{\pgfqpoint{4.364417in}{1.799350in}}%
\pgfpathlineto{\pgfqpoint{4.357870in}{1.813602in}}%
\pgfpathlineto{\pgfqpoint{4.351320in}{1.827693in}}%
\pgfpathlineto{\pgfqpoint{4.344769in}{1.841646in}}%
\pgfpathlineto{\pgfqpoint{4.338216in}{1.855484in}}%
\pgfpathlineto{\pgfqpoint{4.327061in}{1.851830in}}%
\pgfpathlineto{\pgfqpoint{4.315898in}{1.848111in}}%
\pgfpathlineto{\pgfqpoint{4.304729in}{1.844350in}}%
\pgfpathlineto{\pgfqpoint{4.293554in}{1.840569in}}%
\pgfpathlineto{\pgfqpoint{4.282373in}{1.836791in}}%
\pgfpathlineto{\pgfqpoint{4.288937in}{1.823395in}}%
\pgfpathlineto{\pgfqpoint{4.295497in}{1.809801in}}%
\pgfpathlineto{\pgfqpoint{4.302053in}{1.795967in}}%
\pgfpathlineto{\pgfqpoint{4.308603in}{1.781848in}}%
\pgfpathclose%
\pgfusepath{stroke,fill}%
\end{pgfscope}%
\begin{pgfscope}%
\pgfpathrectangle{\pgfqpoint{0.887500in}{0.275000in}}{\pgfqpoint{4.225000in}{4.225000in}}%
\pgfusepath{clip}%
\pgfsetbuttcap%
\pgfsetroundjoin%
\definecolor{currentfill}{rgb}{0.144759,0.519093,0.556572}%
\pgfsetfillcolor{currentfill}%
\pgfsetfillopacity{0.700000}%
\pgfsetlinewidth{0.501875pt}%
\definecolor{currentstroke}{rgb}{1.000000,1.000000,1.000000}%
\pgfsetstrokecolor{currentstroke}%
\pgfsetstrokeopacity{0.500000}%
\pgfsetdash{}{0pt}%
\pgfpathmoveto{\pgfqpoint{3.660394in}{2.154960in}}%
\pgfpathlineto{\pgfqpoint{3.671865in}{2.167210in}}%
\pgfpathlineto{\pgfqpoint{3.683327in}{2.178906in}}%
\pgfpathlineto{\pgfqpoint{3.694776in}{2.189868in}}%
\pgfpathlineto{\pgfqpoint{3.706208in}{2.199916in}}%
\pgfpathlineto{\pgfqpoint{3.717621in}{2.208870in}}%
\pgfpathlineto{\pgfqpoint{3.711203in}{2.223526in}}%
\pgfpathlineto{\pgfqpoint{3.704785in}{2.238012in}}%
\pgfpathlineto{\pgfqpoint{3.698368in}{2.252395in}}%
\pgfpathlineto{\pgfqpoint{3.691953in}{2.266742in}}%
\pgfpathlineto{\pgfqpoint{3.685540in}{2.281117in}}%
\pgfpathlineto{\pgfqpoint{3.674145in}{2.272811in}}%
\pgfpathlineto{\pgfqpoint{3.662736in}{2.263781in}}%
\pgfpathlineto{\pgfqpoint{3.651315in}{2.254108in}}%
\pgfpathlineto{\pgfqpoint{3.639883in}{2.243869in}}%
\pgfpathlineto{\pgfqpoint{3.628442in}{2.233145in}}%
\pgfpathlineto{\pgfqpoint{3.634828in}{2.217483in}}%
\pgfpathlineto{\pgfqpoint{3.641217in}{2.201906in}}%
\pgfpathlineto{\pgfqpoint{3.647609in}{2.186342in}}%
\pgfpathlineto{\pgfqpoint{3.654001in}{2.170717in}}%
\pgfpathclose%
\pgfusepath{stroke,fill}%
\end{pgfscope}%
\begin{pgfscope}%
\pgfpathrectangle{\pgfqpoint{0.887500in}{0.275000in}}{\pgfqpoint{4.225000in}{4.225000in}}%
\pgfusepath{clip}%
\pgfsetbuttcap%
\pgfsetroundjoin%
\definecolor{currentfill}{rgb}{0.156270,0.489624,0.557936}%
\pgfsetfillcolor{currentfill}%
\pgfsetfillopacity{0.700000}%
\pgfsetlinewidth{0.501875pt}%
\definecolor{currentstroke}{rgb}{1.000000,1.000000,1.000000}%
\pgfsetstrokecolor{currentstroke}%
\pgfsetstrokeopacity{0.500000}%
\pgfsetdash{}{0pt}%
\pgfpathmoveto{\pgfqpoint{1.839850in}{2.125463in}}%
\pgfpathlineto{\pgfqpoint{1.851644in}{2.129092in}}%
\pgfpathlineto{\pgfqpoint{1.863432in}{2.132712in}}%
\pgfpathlineto{\pgfqpoint{1.875214in}{2.136326in}}%
\pgfpathlineto{\pgfqpoint{1.886991in}{2.139933in}}%
\pgfpathlineto{\pgfqpoint{1.898761in}{2.143536in}}%
\pgfpathlineto{\pgfqpoint{1.892841in}{2.152626in}}%
\pgfpathlineto{\pgfqpoint{1.886924in}{2.161686in}}%
\pgfpathlineto{\pgfqpoint{1.881013in}{2.170718in}}%
\pgfpathlineto{\pgfqpoint{1.875105in}{2.179721in}}%
\pgfpathlineto{\pgfqpoint{1.869203in}{2.188697in}}%
\pgfpathlineto{\pgfqpoint{1.857443in}{2.185144in}}%
\pgfpathlineto{\pgfqpoint{1.845677in}{2.181587in}}%
\pgfpathlineto{\pgfqpoint{1.833905in}{2.178024in}}%
\pgfpathlineto{\pgfqpoint{1.822128in}{2.174455in}}%
\pgfpathlineto{\pgfqpoint{1.810346in}{2.170877in}}%
\pgfpathlineto{\pgfqpoint{1.816237in}{2.161853in}}%
\pgfpathlineto{\pgfqpoint{1.822134in}{2.152800in}}%
\pgfpathlineto{\pgfqpoint{1.828035in}{2.143718in}}%
\pgfpathlineto{\pgfqpoint{1.833940in}{2.134605in}}%
\pgfpathclose%
\pgfusepath{stroke,fill}%
\end{pgfscope}%
\begin{pgfscope}%
\pgfpathrectangle{\pgfqpoint{0.887500in}{0.275000in}}{\pgfqpoint{4.225000in}{4.225000in}}%
\pgfusepath{clip}%
\pgfsetbuttcap%
\pgfsetroundjoin%
\definecolor{currentfill}{rgb}{0.187231,0.414746,0.556547}%
\pgfsetfillcolor{currentfill}%
\pgfsetfillopacity{0.700000}%
\pgfsetlinewidth{0.501875pt}%
\definecolor{currentstroke}{rgb}{1.000000,1.000000,1.000000}%
\pgfsetstrokecolor{currentstroke}%
\pgfsetstrokeopacity{0.500000}%
\pgfsetdash{}{0pt}%
\pgfpathmoveto{\pgfqpoint{2.577915in}{1.959577in}}%
\pgfpathlineto{\pgfqpoint{2.589527in}{1.963248in}}%
\pgfpathlineto{\pgfqpoint{2.601134in}{1.966908in}}%
\pgfpathlineto{\pgfqpoint{2.612735in}{1.970561in}}%
\pgfpathlineto{\pgfqpoint{2.624330in}{1.974211in}}%
\pgfpathlineto{\pgfqpoint{2.635920in}{1.977861in}}%
\pgfpathlineto{\pgfqpoint{2.629752in}{1.987728in}}%
\pgfpathlineto{\pgfqpoint{2.623589in}{1.997548in}}%
\pgfpathlineto{\pgfqpoint{2.617430in}{2.007322in}}%
\pgfpathlineto{\pgfqpoint{2.611274in}{2.017050in}}%
\pgfpathlineto{\pgfqpoint{2.605124in}{2.026732in}}%
\pgfpathlineto{\pgfqpoint{2.593544in}{2.023138in}}%
\pgfpathlineto{\pgfqpoint{2.581958in}{2.019545in}}%
\pgfpathlineto{\pgfqpoint{2.570366in}{2.015950in}}%
\pgfpathlineto{\pgfqpoint{2.558769in}{2.012347in}}%
\pgfpathlineto{\pgfqpoint{2.547167in}{2.008734in}}%
\pgfpathlineto{\pgfqpoint{2.553308in}{1.998996in}}%
\pgfpathlineto{\pgfqpoint{2.559453in}{1.989211in}}%
\pgfpathlineto{\pgfqpoint{2.565603in}{1.979380in}}%
\pgfpathlineto{\pgfqpoint{2.571756in}{1.969502in}}%
\pgfpathclose%
\pgfusepath{stroke,fill}%
\end{pgfscope}%
\begin{pgfscope}%
\pgfpathrectangle{\pgfqpoint{0.887500in}{0.275000in}}{\pgfqpoint{4.225000in}{4.225000in}}%
\pgfusepath{clip}%
\pgfsetbuttcap%
\pgfsetroundjoin%
\definecolor{currentfill}{rgb}{0.221989,0.339161,0.548752}%
\pgfsetfillcolor{currentfill}%
\pgfsetfillopacity{0.700000}%
\pgfsetlinewidth{0.501875pt}%
\definecolor{currentstroke}{rgb}{1.000000,1.000000,1.000000}%
\pgfsetstrokecolor{currentstroke}%
\pgfsetstrokeopacity{0.500000}%
\pgfsetdash{}{0pt}%
\pgfpathmoveto{\pgfqpoint{3.080494in}{1.811582in}}%
\pgfpathlineto{\pgfqpoint{3.091983in}{1.815326in}}%
\pgfpathlineto{\pgfqpoint{3.103466in}{1.819049in}}%
\pgfpathlineto{\pgfqpoint{3.114944in}{1.822746in}}%
\pgfpathlineto{\pgfqpoint{3.126415in}{1.826424in}}%
\pgfpathlineto{\pgfqpoint{3.137882in}{1.830100in}}%
\pgfpathlineto{\pgfqpoint{3.131561in}{1.840410in}}%
\pgfpathlineto{\pgfqpoint{3.125245in}{1.850624in}}%
\pgfpathlineto{\pgfqpoint{3.118931in}{1.860752in}}%
\pgfpathlineto{\pgfqpoint{3.112622in}{1.870805in}}%
\pgfpathlineto{\pgfqpoint{3.106316in}{1.880792in}}%
\pgfpathlineto{\pgfqpoint{3.094860in}{1.878092in}}%
\pgfpathlineto{\pgfqpoint{3.083398in}{1.875444in}}%
\pgfpathlineto{\pgfqpoint{3.071929in}{1.872578in}}%
\pgfpathlineto{\pgfqpoint{3.060455in}{1.869358in}}%
\pgfpathlineto{\pgfqpoint{3.048974in}{1.865844in}}%
\pgfpathlineto{\pgfqpoint{3.055271in}{1.855075in}}%
\pgfpathlineto{\pgfqpoint{3.061572in}{1.844262in}}%
\pgfpathlineto{\pgfqpoint{3.067875in}{1.833407in}}%
\pgfpathlineto{\pgfqpoint{3.074183in}{1.822513in}}%
\pgfpathclose%
\pgfusepath{stroke,fill}%
\end{pgfscope}%
\begin{pgfscope}%
\pgfpathrectangle{\pgfqpoint{0.887500in}{0.275000in}}{\pgfqpoint{4.225000in}{4.225000in}}%
\pgfusepath{clip}%
\pgfsetbuttcap%
\pgfsetroundjoin%
\definecolor{currentfill}{rgb}{0.218130,0.347432,0.550038}%
\pgfsetfillcolor{currentfill}%
\pgfsetfillopacity{0.700000}%
\pgfsetlinewidth{0.501875pt}%
\definecolor{currentstroke}{rgb}{1.000000,1.000000,1.000000}%
\pgfsetstrokecolor{currentstroke}%
\pgfsetstrokeopacity{0.500000}%
\pgfsetdash{}{0pt}%
\pgfpathmoveto{\pgfqpoint{4.226396in}{1.818157in}}%
\pgfpathlineto{\pgfqpoint{4.237606in}{1.822010in}}%
\pgfpathlineto{\pgfqpoint{4.248806in}{1.825692in}}%
\pgfpathlineto{\pgfqpoint{4.259999in}{1.829337in}}%
\pgfpathlineto{\pgfqpoint{4.271188in}{1.833040in}}%
\pgfpathlineto{\pgfqpoint{4.282373in}{1.836791in}}%
\pgfpathlineto{\pgfqpoint{4.275807in}{1.850036in}}%
\pgfpathlineto{\pgfqpoint{4.269242in}{1.863172in}}%
\pgfpathlineto{\pgfqpoint{4.262677in}{1.876246in}}%
\pgfpathlineto{\pgfqpoint{4.256115in}{1.889300in}}%
\pgfpathlineto{\pgfqpoint{4.249557in}{1.902379in}}%
\pgfpathlineto{\pgfqpoint{4.238362in}{1.898201in}}%
\pgfpathlineto{\pgfqpoint{4.227166in}{1.894136in}}%
\pgfpathlineto{\pgfqpoint{4.215969in}{1.890205in}}%
\pgfpathlineto{\pgfqpoint{4.204768in}{1.886342in}}%
\pgfpathlineto{\pgfqpoint{4.193561in}{1.882441in}}%
\pgfpathlineto{\pgfqpoint{4.200130in}{1.869870in}}%
\pgfpathlineto{\pgfqpoint{4.206701in}{1.857243in}}%
\pgfpathlineto{\pgfqpoint{4.213271in}{1.844473in}}%
\pgfpathlineto{\pgfqpoint{4.219837in}{1.831473in}}%
\pgfpathclose%
\pgfusepath{stroke,fill}%
\end{pgfscope}%
\begin{pgfscope}%
\pgfpathrectangle{\pgfqpoint{0.887500in}{0.275000in}}{\pgfqpoint{4.225000in}{4.225000in}}%
\pgfusepath{clip}%
\pgfsetbuttcap%
\pgfsetroundjoin%
\definecolor{currentfill}{rgb}{0.208623,0.367752,0.552675}%
\pgfsetfillcolor{currentfill}%
\pgfsetfillopacity{0.700000}%
\pgfsetlinewidth{0.501875pt}%
\definecolor{currentstroke}{rgb}{1.000000,1.000000,1.000000}%
\pgfsetstrokecolor{currentstroke}%
\pgfsetstrokeopacity{0.500000}%
\pgfsetdash{}{0pt}%
\pgfpathmoveto{\pgfqpoint{4.137330in}{1.858735in}}%
\pgfpathlineto{\pgfqpoint{4.148611in}{1.864371in}}%
\pgfpathlineto{\pgfqpoint{4.159872in}{1.869468in}}%
\pgfpathlineto{\pgfqpoint{4.171116in}{1.874109in}}%
\pgfpathlineto{\pgfqpoint{4.182345in}{1.878398in}}%
\pgfpathlineto{\pgfqpoint{4.193561in}{1.882441in}}%
\pgfpathlineto{\pgfqpoint{4.186996in}{1.895043in}}%
\pgfpathlineto{\pgfqpoint{4.180438in}{1.907762in}}%
\pgfpathlineto{\pgfqpoint{4.173889in}{1.920686in}}%
\pgfpathlineto{\pgfqpoint{4.167351in}{1.933902in}}%
\pgfpathlineto{\pgfqpoint{4.160828in}{1.947496in}}%
\pgfpathlineto{\pgfqpoint{4.149605in}{1.943092in}}%
\pgfpathlineto{\pgfqpoint{4.138376in}{1.938610in}}%
\pgfpathlineto{\pgfqpoint{4.127138in}{1.934010in}}%
\pgfpathlineto{\pgfqpoint{4.115891in}{1.929249in}}%
\pgfpathlineto{\pgfqpoint{4.104634in}{1.924291in}}%
\pgfpathlineto{\pgfqpoint{4.111152in}{1.910599in}}%
\pgfpathlineto{\pgfqpoint{4.117683in}{1.897315in}}%
\pgfpathlineto{\pgfqpoint{4.124225in}{1.884322in}}%
\pgfpathlineto{\pgfqpoint{4.130775in}{1.871501in}}%
\pgfpathclose%
\pgfusepath{stroke,fill}%
\end{pgfscope}%
\begin{pgfscope}%
\pgfpathrectangle{\pgfqpoint{0.887500in}{0.275000in}}{\pgfqpoint{4.225000in}{4.225000in}}%
\pgfusepath{clip}%
\pgfsetbuttcap%
\pgfsetroundjoin%
\definecolor{currentfill}{rgb}{0.150476,0.504369,0.557430}%
\pgfsetfillcolor{currentfill}%
\pgfsetfillopacity{0.700000}%
\pgfsetlinewidth{0.501875pt}%
\definecolor{currentstroke}{rgb}{1.000000,1.000000,1.000000}%
\pgfsetstrokecolor{currentstroke}%
\pgfsetstrokeopacity{0.500000}%
\pgfsetdash{}{0pt}%
\pgfpathmoveto{\pgfqpoint{3.749661in}{2.130831in}}%
\pgfpathlineto{\pgfqpoint{3.761064in}{2.139125in}}%
\pgfpathlineto{\pgfqpoint{3.772439in}{2.146025in}}%
\pgfpathlineto{\pgfqpoint{3.783790in}{2.151807in}}%
\pgfpathlineto{\pgfqpoint{3.795120in}{2.156754in}}%
\pgfpathlineto{\pgfqpoint{3.806436in}{2.161149in}}%
\pgfpathlineto{\pgfqpoint{3.800020in}{2.176694in}}%
\pgfpathlineto{\pgfqpoint{3.793603in}{2.192114in}}%
\pgfpathlineto{\pgfqpoint{3.787187in}{2.207423in}}%
\pgfpathlineto{\pgfqpoint{3.780771in}{2.222630in}}%
\pgfpathlineto{\pgfqpoint{3.774355in}{2.237746in}}%
\pgfpathlineto{\pgfqpoint{3.763043in}{2.233428in}}%
\pgfpathlineto{\pgfqpoint{3.751717in}{2.228615in}}%
\pgfpathlineto{\pgfqpoint{3.740373in}{2.223072in}}%
\pgfpathlineto{\pgfqpoint{3.729009in}{2.216567in}}%
\pgfpathlineto{\pgfqpoint{3.717621in}{2.208870in}}%
\pgfpathlineto{\pgfqpoint{3.724037in}{2.193978in}}%
\pgfpathlineto{\pgfqpoint{3.730450in}{2.178785in}}%
\pgfpathlineto{\pgfqpoint{3.736859in}{2.163224in}}%
\pgfpathlineto{\pgfqpoint{3.743263in}{2.147239in}}%
\pgfpathclose%
\pgfusepath{stroke,fill}%
\end{pgfscope}%
\begin{pgfscope}%
\pgfpathrectangle{\pgfqpoint{0.887500in}{0.275000in}}{\pgfqpoint{4.225000in}{4.225000in}}%
\pgfusepath{clip}%
\pgfsetbuttcap%
\pgfsetroundjoin%
\definecolor{currentfill}{rgb}{0.197636,0.391528,0.554969}%
\pgfsetfillcolor{currentfill}%
\pgfsetfillopacity{0.700000}%
\pgfsetlinewidth{0.501875pt}%
\definecolor{currentstroke}{rgb}{1.000000,1.000000,1.000000}%
\pgfsetstrokecolor{currentstroke}%
\pgfsetstrokeopacity{0.500000}%
\pgfsetdash{}{0pt}%
\pgfpathmoveto{\pgfqpoint{4.048196in}{1.896256in}}%
\pgfpathlineto{\pgfqpoint{4.059504in}{1.902312in}}%
\pgfpathlineto{\pgfqpoint{4.070802in}{1.908139in}}%
\pgfpathlineto{\pgfqpoint{4.082089in}{1.913741in}}%
\pgfpathlineto{\pgfqpoint{4.093367in}{1.919123in}}%
\pgfpathlineto{\pgfqpoint{4.104634in}{1.924291in}}%
\pgfpathlineto{\pgfqpoint{4.098134in}{1.938509in}}%
\pgfpathlineto{\pgfqpoint{4.091652in}{1.953330in}}%
\pgfpathlineto{\pgfqpoint{4.085187in}{1.968685in}}%
\pgfpathlineto{\pgfqpoint{4.078736in}{1.984475in}}%
\pgfpathlineto{\pgfqpoint{4.072295in}{2.000602in}}%
\pgfpathlineto{\pgfqpoint{4.061044in}{1.995993in}}%
\pgfpathlineto{\pgfqpoint{4.049785in}{1.991249in}}%
\pgfpathlineto{\pgfqpoint{4.038517in}{1.986351in}}%
\pgfpathlineto{\pgfqpoint{4.027240in}{1.981279in}}%
\pgfpathlineto{\pgfqpoint{4.015954in}{1.976013in}}%
\pgfpathlineto{\pgfqpoint{4.022384in}{1.959481in}}%
\pgfpathlineto{\pgfqpoint{4.028821in}{1.943162in}}%
\pgfpathlineto{\pgfqpoint{4.035267in}{1.927136in}}%
\pgfpathlineto{\pgfqpoint{4.041725in}{1.911482in}}%
\pgfpathclose%
\pgfusepath{stroke,fill}%
\end{pgfscope}%
\begin{pgfscope}%
\pgfpathrectangle{\pgfqpoint{0.887500in}{0.275000in}}{\pgfqpoint{4.225000in}{4.225000in}}%
\pgfusepath{clip}%
\pgfsetbuttcap%
\pgfsetroundjoin%
\definecolor{currentfill}{rgb}{0.235526,0.309527,0.542944}%
\pgfsetfillcolor{currentfill}%
\pgfsetfillopacity{0.700000}%
\pgfsetlinewidth{0.501875pt}%
\definecolor{currentstroke}{rgb}{1.000000,1.000000,1.000000}%
\pgfsetstrokecolor{currentstroke}%
\pgfsetstrokeopacity{0.500000}%
\pgfsetdash{}{0pt}%
\pgfpathmoveto{\pgfqpoint{3.404868in}{1.730701in}}%
\pgfpathlineto{\pgfqpoint{3.416313in}{1.738908in}}%
\pgfpathlineto{\pgfqpoint{3.427759in}{1.747592in}}%
\pgfpathlineto{\pgfqpoint{3.439207in}{1.756690in}}%
\pgfpathlineto{\pgfqpoint{3.450657in}{1.766180in}}%
\pgfpathlineto{\pgfqpoint{3.462110in}{1.776166in}}%
\pgfpathlineto{\pgfqpoint{3.455770in}{1.794091in}}%
\pgfpathlineto{\pgfqpoint{3.449427in}{1.811586in}}%
\pgfpathlineto{\pgfqpoint{3.443081in}{1.828537in}}%
\pgfpathlineto{\pgfqpoint{3.436731in}{1.845017in}}%
\pgfpathlineto{\pgfqpoint{3.430380in}{1.861131in}}%
\pgfpathlineto{\pgfqpoint{3.418870in}{1.844498in}}%
\pgfpathlineto{\pgfqpoint{3.407371in}{1.828904in}}%
\pgfpathlineto{\pgfqpoint{3.395885in}{1.814628in}}%
\pgfpathlineto{\pgfqpoint{3.384411in}{1.801726in}}%
\pgfpathlineto{\pgfqpoint{3.372948in}{1.790186in}}%
\pgfpathlineto{\pgfqpoint{3.379329in}{1.778722in}}%
\pgfpathlineto{\pgfqpoint{3.385713in}{1.767162in}}%
\pgfpathlineto{\pgfqpoint{3.392097in}{1.755364in}}%
\pgfpathlineto{\pgfqpoint{3.398482in}{1.743207in}}%
\pgfpathclose%
\pgfusepath{stroke,fill}%
\end{pgfscope}%
\begin{pgfscope}%
\pgfpathrectangle{\pgfqpoint{0.887500in}{0.275000in}}{\pgfqpoint{4.225000in}{4.225000in}}%
\pgfusepath{clip}%
\pgfsetbuttcap%
\pgfsetroundjoin%
\definecolor{currentfill}{rgb}{0.154815,0.493313,0.557840}%
\pgfsetfillcolor{currentfill}%
\pgfsetfillopacity{0.700000}%
\pgfsetlinewidth{0.501875pt}%
\definecolor{currentstroke}{rgb}{1.000000,1.000000,1.000000}%
\pgfsetstrokecolor{currentstroke}%
\pgfsetstrokeopacity{0.500000}%
\pgfsetdash{}{0pt}%
\pgfpathmoveto{\pgfqpoint{3.602983in}{2.090392in}}%
\pgfpathlineto{\pgfqpoint{3.614469in}{2.103534in}}%
\pgfpathlineto{\pgfqpoint{3.625953in}{2.116574in}}%
\pgfpathlineto{\pgfqpoint{3.637436in}{2.129510in}}%
\pgfpathlineto{\pgfqpoint{3.648916in}{2.142335in}}%
\pgfpathlineto{\pgfqpoint{3.660394in}{2.154960in}}%
\pgfpathlineto{\pgfqpoint{3.654001in}{2.170717in}}%
\pgfpathlineto{\pgfqpoint{3.647609in}{2.186342in}}%
\pgfpathlineto{\pgfqpoint{3.641217in}{2.201906in}}%
\pgfpathlineto{\pgfqpoint{3.634828in}{2.217483in}}%
\pgfpathlineto{\pgfqpoint{3.628442in}{2.233145in}}%
\pgfpathlineto{\pgfqpoint{3.616994in}{2.222014in}}%
\pgfpathlineto{\pgfqpoint{3.605541in}{2.210556in}}%
\pgfpathlineto{\pgfqpoint{3.594083in}{2.198847in}}%
\pgfpathlineto{\pgfqpoint{3.582622in}{2.186965in}}%
\pgfpathlineto{\pgfqpoint{3.571158in}{2.174983in}}%
\pgfpathlineto{\pgfqpoint{3.577519in}{2.157985in}}%
\pgfpathlineto{\pgfqpoint{3.583883in}{2.141095in}}%
\pgfpathlineto{\pgfqpoint{3.590248in}{2.124245in}}%
\pgfpathlineto{\pgfqpoint{3.596616in}{2.107367in}}%
\pgfpathclose%
\pgfusepath{stroke,fill}%
\end{pgfscope}%
\begin{pgfscope}%
\pgfpathrectangle{\pgfqpoint{0.887500in}{0.275000in}}{\pgfqpoint{4.225000in}{4.225000in}}%
\pgfusepath{clip}%
\pgfsetbuttcap%
\pgfsetroundjoin%
\definecolor{currentfill}{rgb}{0.218130,0.347432,0.550038}%
\pgfsetfillcolor{currentfill}%
\pgfsetfillopacity{0.700000}%
\pgfsetlinewidth{0.501875pt}%
\definecolor{currentstroke}{rgb}{1.000000,1.000000,1.000000}%
\pgfsetstrokecolor{currentstroke}%
\pgfsetstrokeopacity{0.500000}%
\pgfsetdash{}{0pt}%
\pgfpathmoveto{\pgfqpoint{3.462110in}{1.776166in}}%
\pgfpathlineto{\pgfqpoint{3.473567in}{1.786779in}}%
\pgfpathlineto{\pgfqpoint{3.485031in}{1.798150in}}%
\pgfpathlineto{\pgfqpoint{3.496504in}{1.810412in}}%
\pgfpathlineto{\pgfqpoint{3.507989in}{1.823695in}}%
\pgfpathlineto{\pgfqpoint{3.519487in}{1.838130in}}%
\pgfpathlineto{\pgfqpoint{3.513204in}{1.861345in}}%
\pgfpathlineto{\pgfqpoint{3.506915in}{1.884102in}}%
\pgfpathlineto{\pgfqpoint{3.500619in}{1.906200in}}%
\pgfpathlineto{\pgfqpoint{3.494316in}{1.927649in}}%
\pgfpathlineto{\pgfqpoint{3.488005in}{1.948496in}}%
\pgfpathlineto{\pgfqpoint{3.476481in}{1.931551in}}%
\pgfpathlineto{\pgfqpoint{3.464953in}{1.914015in}}%
\pgfpathlineto{\pgfqpoint{3.453424in}{1.896215in}}%
\pgfpathlineto{\pgfqpoint{3.441899in}{1.878479in}}%
\pgfpathlineto{\pgfqpoint{3.430380in}{1.861131in}}%
\pgfpathlineto{\pgfqpoint{3.436731in}{1.845017in}}%
\pgfpathlineto{\pgfqpoint{3.443081in}{1.828537in}}%
\pgfpathlineto{\pgfqpoint{3.449427in}{1.811586in}}%
\pgfpathlineto{\pgfqpoint{3.455770in}{1.794091in}}%
\pgfpathclose%
\pgfusepath{stroke,fill}%
\end{pgfscope}%
\begin{pgfscope}%
\pgfpathrectangle{\pgfqpoint{0.887500in}{0.275000in}}{\pgfqpoint{4.225000in}{4.225000in}}%
\pgfusepath{clip}%
\pgfsetbuttcap%
\pgfsetroundjoin%
\definecolor{currentfill}{rgb}{0.172719,0.448791,0.557885}%
\pgfsetfillcolor{currentfill}%
\pgfsetfillopacity{0.700000}%
\pgfsetlinewidth{0.501875pt}%
\definecolor{currentstroke}{rgb}{1.000000,1.000000,1.000000}%
\pgfsetstrokecolor{currentstroke}%
\pgfsetstrokeopacity{0.500000}%
\pgfsetdash{}{0pt}%
\pgfpathmoveto{\pgfqpoint{2.253189in}{2.030661in}}%
\pgfpathlineto{\pgfqpoint{2.264883in}{2.034345in}}%
\pgfpathlineto{\pgfqpoint{2.276571in}{2.038029in}}%
\pgfpathlineto{\pgfqpoint{2.288254in}{2.041715in}}%
\pgfpathlineto{\pgfqpoint{2.299931in}{2.045406in}}%
\pgfpathlineto{\pgfqpoint{2.311602in}{2.049104in}}%
\pgfpathlineto{\pgfqpoint{2.305540in}{2.058550in}}%
\pgfpathlineto{\pgfqpoint{2.299482in}{2.067960in}}%
\pgfpathlineto{\pgfqpoint{2.293428in}{2.077333in}}%
\pgfpathlineto{\pgfqpoint{2.287379in}{2.086672in}}%
\pgfpathlineto{\pgfqpoint{2.281334in}{2.095976in}}%
\pgfpathlineto{\pgfqpoint{2.269673in}{2.092334in}}%
\pgfpathlineto{\pgfqpoint{2.258007in}{2.088700in}}%
\pgfpathlineto{\pgfqpoint{2.246334in}{2.085071in}}%
\pgfpathlineto{\pgfqpoint{2.234656in}{2.081446in}}%
\pgfpathlineto{\pgfqpoint{2.222972in}{2.077821in}}%
\pgfpathlineto{\pgfqpoint{2.229006in}{2.068461in}}%
\pgfpathlineto{\pgfqpoint{2.235045in}{2.059066in}}%
\pgfpathlineto{\pgfqpoint{2.241089in}{2.049635in}}%
\pgfpathlineto{\pgfqpoint{2.247136in}{2.040167in}}%
\pgfpathclose%
\pgfusepath{stroke,fill}%
\end{pgfscope}%
\begin{pgfscope}%
\pgfpathrectangle{\pgfqpoint{0.887500in}{0.275000in}}{\pgfqpoint{4.225000in}{4.225000in}}%
\pgfusepath{clip}%
\pgfsetbuttcap%
\pgfsetroundjoin%
\definecolor{currentfill}{rgb}{0.179019,0.433756,0.557430}%
\pgfsetfillcolor{currentfill}%
\pgfsetfillopacity{0.700000}%
\pgfsetlinewidth{0.501875pt}%
\definecolor{currentstroke}{rgb}{1.000000,1.000000,1.000000}%
\pgfsetstrokecolor{currentstroke}%
\pgfsetstrokeopacity{0.500000}%
\pgfsetdash{}{0pt}%
\pgfpathmoveto{\pgfqpoint{3.488005in}{1.948496in}}%
\pgfpathlineto{\pgfqpoint{3.499523in}{1.964650in}}%
\pgfpathlineto{\pgfqpoint{3.511034in}{1.980086in}}%
\pgfpathlineto{\pgfqpoint{3.522539in}{1.994915in}}%
\pgfpathlineto{\pgfqpoint{3.534039in}{2.009249in}}%
\pgfpathlineto{\pgfqpoint{3.545534in}{2.023200in}}%
\pgfpathlineto{\pgfqpoint{3.539199in}{2.042207in}}%
\pgfpathlineto{\pgfqpoint{3.532861in}{2.060928in}}%
\pgfpathlineto{\pgfqpoint{3.526522in}{2.079418in}}%
\pgfpathlineto{\pgfqpoint{3.520181in}{2.097732in}}%
\pgfpathlineto{\pgfqpoint{3.513840in}{2.115924in}}%
\pgfpathlineto{\pgfqpoint{3.502372in}{2.103927in}}%
\pgfpathlineto{\pgfqpoint{3.490896in}{2.091226in}}%
\pgfpathlineto{\pgfqpoint{3.479407in}{2.077468in}}%
\pgfpathlineto{\pgfqpoint{3.467904in}{2.062303in}}%
\pgfpathlineto{\pgfqpoint{3.456385in}{2.045381in}}%
\pgfpathlineto{\pgfqpoint{3.462716in}{2.026824in}}%
\pgfpathlineto{\pgfqpoint{3.469044in}{2.007905in}}%
\pgfpathlineto{\pgfqpoint{3.475369in}{1.988576in}}%
\pgfpathlineto{\pgfqpoint{3.481690in}{1.968789in}}%
\pgfpathclose%
\pgfusepath{stroke,fill}%
\end{pgfscope}%
\begin{pgfscope}%
\pgfpathrectangle{\pgfqpoint{0.887500in}{0.275000in}}{\pgfqpoint{4.225000in}{4.225000in}}%
\pgfusepath{clip}%
\pgfsetbuttcap%
\pgfsetroundjoin%
\definecolor{currentfill}{rgb}{0.165117,0.467423,0.558141}%
\pgfsetfillcolor{currentfill}%
\pgfsetfillopacity{0.700000}%
\pgfsetlinewidth{0.501875pt}%
\definecolor{currentstroke}{rgb}{1.000000,1.000000,1.000000}%
\pgfsetstrokecolor{currentstroke}%
\pgfsetstrokeopacity{0.500000}%
\pgfsetdash{}{0pt}%
\pgfpathmoveto{\pgfqpoint{3.545534in}{2.023200in}}%
\pgfpathlineto{\pgfqpoint{3.557027in}{2.036878in}}%
\pgfpathlineto{\pgfqpoint{3.568518in}{2.050396in}}%
\pgfpathlineto{\pgfqpoint{3.580007in}{2.063820in}}%
\pgfpathlineto{\pgfqpoint{3.591496in}{2.077154in}}%
\pgfpathlineto{\pgfqpoint{3.602983in}{2.090392in}}%
\pgfpathlineto{\pgfqpoint{3.596616in}{2.107367in}}%
\pgfpathlineto{\pgfqpoint{3.590248in}{2.124245in}}%
\pgfpathlineto{\pgfqpoint{3.583883in}{2.141095in}}%
\pgfpathlineto{\pgfqpoint{3.577519in}{2.157985in}}%
\pgfpathlineto{\pgfqpoint{3.571158in}{2.174983in}}%
\pgfpathlineto{\pgfqpoint{3.559694in}{2.162979in}}%
\pgfpathlineto{\pgfqpoint{3.548230in}{2.151028in}}%
\pgfpathlineto{\pgfqpoint{3.536766in}{2.139206in}}%
\pgfpathlineto{\pgfqpoint{3.525304in}{2.127567in}}%
\pgfpathlineto{\pgfqpoint{3.513840in}{2.115924in}}%
\pgfpathlineto{\pgfqpoint{3.520181in}{2.097732in}}%
\pgfpathlineto{\pgfqpoint{3.526522in}{2.079418in}}%
\pgfpathlineto{\pgfqpoint{3.532861in}{2.060928in}}%
\pgfpathlineto{\pgfqpoint{3.539199in}{2.042207in}}%
\pgfpathclose%
\pgfusepath{stroke,fill}%
\end{pgfscope}%
\begin{pgfscope}%
\pgfpathrectangle{\pgfqpoint{0.887500in}{0.275000in}}{\pgfqpoint{4.225000in}{4.225000in}}%
\pgfusepath{clip}%
\pgfsetbuttcap%
\pgfsetroundjoin%
\definecolor{currentfill}{rgb}{0.229739,0.322361,0.545706}%
\pgfsetfillcolor{currentfill}%
\pgfsetfillopacity{0.700000}%
\pgfsetlinewidth{0.501875pt}%
\definecolor{currentstroke}{rgb}{1.000000,1.000000,1.000000}%
\pgfsetstrokecolor{currentstroke}%
\pgfsetstrokeopacity{0.500000}%
\pgfsetdash{}{0pt}%
\pgfpathmoveto{\pgfqpoint{3.169535in}{1.776811in}}%
\pgfpathlineto{\pgfqpoint{3.181006in}{1.780837in}}%
\pgfpathlineto{\pgfqpoint{3.192470in}{1.784624in}}%
\pgfpathlineto{\pgfqpoint{3.203927in}{1.788069in}}%
\pgfpathlineto{\pgfqpoint{3.215376in}{1.791067in}}%
\pgfpathlineto{\pgfqpoint{3.226818in}{1.793552in}}%
\pgfpathlineto{\pgfqpoint{3.220473in}{1.804643in}}%
\pgfpathlineto{\pgfqpoint{3.214132in}{1.815739in}}%
\pgfpathlineto{\pgfqpoint{3.207795in}{1.826849in}}%
\pgfpathlineto{\pgfqpoint{3.201461in}{1.837972in}}%
\pgfpathlineto{\pgfqpoint{3.195131in}{1.849100in}}%
\pgfpathlineto{\pgfqpoint{3.183692in}{1.845150in}}%
\pgfpathlineto{\pgfqpoint{3.172247in}{1.841300in}}%
\pgfpathlineto{\pgfqpoint{3.160798in}{1.837521in}}%
\pgfpathlineto{\pgfqpoint{3.149342in}{1.833793in}}%
\pgfpathlineto{\pgfqpoint{3.137882in}{1.830100in}}%
\pgfpathlineto{\pgfqpoint{3.144206in}{1.819686in}}%
\pgfpathlineto{\pgfqpoint{3.150533in}{1.809156in}}%
\pgfpathlineto{\pgfqpoint{3.156864in}{1.798502in}}%
\pgfpathlineto{\pgfqpoint{3.163198in}{1.787719in}}%
\pgfpathclose%
\pgfusepath{stroke,fill}%
\end{pgfscope}%
\begin{pgfscope}%
\pgfpathrectangle{\pgfqpoint{0.887500in}{0.275000in}}{\pgfqpoint{4.225000in}{4.225000in}}%
\pgfusepath{clip}%
\pgfsetbuttcap%
\pgfsetroundjoin%
\definecolor{currentfill}{rgb}{0.194100,0.399323,0.555565}%
\pgfsetfillcolor{currentfill}%
\pgfsetfillopacity{0.700000}%
\pgfsetlinewidth{0.501875pt}%
\definecolor{currentstroke}{rgb}{1.000000,1.000000,1.000000}%
\pgfsetstrokecolor{currentstroke}%
\pgfsetstrokeopacity{0.500000}%
\pgfsetdash{}{0pt}%
\pgfpathmoveto{\pgfqpoint{2.666822in}{1.927827in}}%
\pgfpathlineto{\pgfqpoint{2.678416in}{1.931537in}}%
\pgfpathlineto{\pgfqpoint{2.690003in}{1.935252in}}%
\pgfpathlineto{\pgfqpoint{2.701586in}{1.938973in}}%
\pgfpathlineto{\pgfqpoint{2.713162in}{1.942704in}}%
\pgfpathlineto{\pgfqpoint{2.724732in}{1.946446in}}%
\pgfpathlineto{\pgfqpoint{2.718534in}{1.956490in}}%
\pgfpathlineto{\pgfqpoint{2.712340in}{1.966488in}}%
\pgfpathlineto{\pgfqpoint{2.706150in}{1.976440in}}%
\pgfpathlineto{\pgfqpoint{2.699964in}{1.986346in}}%
\pgfpathlineto{\pgfqpoint{2.693783in}{1.996205in}}%
\pgfpathlineto{\pgfqpoint{2.682222in}{1.992517in}}%
\pgfpathlineto{\pgfqpoint{2.670655in}{1.988840in}}%
\pgfpathlineto{\pgfqpoint{2.659083in}{1.985173in}}%
\pgfpathlineto{\pgfqpoint{2.647504in}{1.981514in}}%
\pgfpathlineto{\pgfqpoint{2.635920in}{1.977861in}}%
\pgfpathlineto{\pgfqpoint{2.642092in}{1.967947in}}%
\pgfpathlineto{\pgfqpoint{2.648268in}{1.957987in}}%
\pgfpathlineto{\pgfqpoint{2.654449in}{1.947980in}}%
\pgfpathlineto{\pgfqpoint{2.660633in}{1.937927in}}%
\pgfpathclose%
\pgfusepath{stroke,fill}%
\end{pgfscope}%
\begin{pgfscope}%
\pgfpathrectangle{\pgfqpoint{0.887500in}{0.275000in}}{\pgfqpoint{4.225000in}{4.225000in}}%
\pgfusepath{clip}%
\pgfsetbuttcap%
\pgfsetroundjoin%
\definecolor{currentfill}{rgb}{0.160665,0.478540,0.558115}%
\pgfsetfillcolor{currentfill}%
\pgfsetfillopacity{0.700000}%
\pgfsetlinewidth{0.501875pt}%
\definecolor{currentstroke}{rgb}{1.000000,1.000000,1.000000}%
\pgfsetstrokecolor{currentstroke}%
\pgfsetstrokeopacity{0.500000}%
\pgfsetdash{}{0pt}%
\pgfpathmoveto{\pgfqpoint{1.928433in}{2.097643in}}%
\pgfpathlineto{\pgfqpoint{1.940209in}{2.101298in}}%
\pgfpathlineto{\pgfqpoint{1.951979in}{2.104950in}}%
\pgfpathlineto{\pgfqpoint{1.963743in}{2.108601in}}%
\pgfpathlineto{\pgfqpoint{1.975502in}{2.112252in}}%
\pgfpathlineto{\pgfqpoint{1.987254in}{2.115903in}}%
\pgfpathlineto{\pgfqpoint{1.981300in}{2.125084in}}%
\pgfpathlineto{\pgfqpoint{1.975351in}{2.134235in}}%
\pgfpathlineto{\pgfqpoint{1.969406in}{2.143357in}}%
\pgfpathlineto{\pgfqpoint{1.963465in}{2.152450in}}%
\pgfpathlineto{\pgfqpoint{1.957529in}{2.161516in}}%
\pgfpathlineto{\pgfqpoint{1.945787in}{2.157921in}}%
\pgfpathlineto{\pgfqpoint{1.934039in}{2.154327in}}%
\pgfpathlineto{\pgfqpoint{1.922286in}{2.150732in}}%
\pgfpathlineto{\pgfqpoint{1.910527in}{2.147135in}}%
\pgfpathlineto{\pgfqpoint{1.898761in}{2.143536in}}%
\pgfpathlineto{\pgfqpoint{1.904687in}{2.134417in}}%
\pgfpathlineto{\pgfqpoint{1.910617in}{2.125269in}}%
\pgfpathlineto{\pgfqpoint{1.916551in}{2.116091in}}%
\pgfpathlineto{\pgfqpoint{1.922490in}{2.106882in}}%
\pgfpathclose%
\pgfusepath{stroke,fill}%
\end{pgfscope}%
\begin{pgfscope}%
\pgfpathrectangle{\pgfqpoint{0.887500in}{0.275000in}}{\pgfqpoint{4.225000in}{4.225000in}}%
\pgfusepath{clip}%
\pgfsetbuttcap%
\pgfsetroundjoin%
\definecolor{currentfill}{rgb}{0.239346,0.300855,0.540844}%
\pgfsetfillcolor{currentfill}%
\pgfsetfillopacity{0.700000}%
\pgfsetlinewidth{0.501875pt}%
\definecolor{currentstroke}{rgb}{1.000000,1.000000,1.000000}%
\pgfsetstrokecolor{currentstroke}%
\pgfsetstrokeopacity{0.500000}%
\pgfsetdash{}{0pt}%
\pgfpathmoveto{\pgfqpoint{3.258592in}{1.737825in}}%
\pgfpathlineto{\pgfqpoint{3.270033in}{1.740260in}}%
\pgfpathlineto{\pgfqpoint{3.281468in}{1.742678in}}%
\pgfpathlineto{\pgfqpoint{3.292899in}{1.745340in}}%
\pgfpathlineto{\pgfqpoint{3.304326in}{1.748508in}}%
\pgfpathlineto{\pgfqpoint{3.315752in}{1.752441in}}%
\pgfpathlineto{\pgfqpoint{3.309379in}{1.762962in}}%
\pgfpathlineto{\pgfqpoint{3.303011in}{1.773643in}}%
\pgfpathlineto{\pgfqpoint{3.296648in}{1.784582in}}%
\pgfpathlineto{\pgfqpoint{3.290290in}{1.795876in}}%
\pgfpathlineto{\pgfqpoint{3.283939in}{1.807623in}}%
\pgfpathlineto{\pgfqpoint{3.272520in}{1.803525in}}%
\pgfpathlineto{\pgfqpoint{3.261101in}{1.800388in}}%
\pgfpathlineto{\pgfqpoint{3.249678in}{1.797893in}}%
\pgfpathlineto{\pgfqpoint{3.238251in}{1.795721in}}%
\pgfpathlineto{\pgfqpoint{3.226818in}{1.793552in}}%
\pgfpathlineto{\pgfqpoint{3.233166in}{1.782457in}}%
\pgfpathlineto{\pgfqpoint{3.239517in}{1.771346in}}%
\pgfpathlineto{\pgfqpoint{3.245872in}{1.760211in}}%
\pgfpathlineto{\pgfqpoint{3.252230in}{1.749040in}}%
\pgfpathclose%
\pgfusepath{stroke,fill}%
\end{pgfscope}%
\begin{pgfscope}%
\pgfpathrectangle{\pgfqpoint{0.887500in}{0.275000in}}{\pgfqpoint{4.225000in}{4.225000in}}%
\pgfusepath{clip}%
\pgfsetbuttcap%
\pgfsetroundjoin%
\definecolor{currentfill}{rgb}{0.243113,0.292092,0.538516}%
\pgfsetfillcolor{currentfill}%
\pgfsetfillopacity{0.700000}%
\pgfsetlinewidth{0.501875pt}%
\definecolor{currentstroke}{rgb}{1.000000,1.000000,1.000000}%
\pgfsetstrokecolor{currentstroke}%
\pgfsetstrokeopacity{0.500000}%
\pgfsetdash{}{0pt}%
\pgfpathmoveto{\pgfqpoint{3.493824in}{1.686240in}}%
\pgfpathlineto{\pgfqpoint{3.505213in}{1.690622in}}%
\pgfpathlineto{\pgfqpoint{3.516610in}{1.696115in}}%
\pgfpathlineto{\pgfqpoint{3.528022in}{1.703259in}}%
\pgfpathlineto{\pgfqpoint{3.539458in}{1.712596in}}%
\pgfpathlineto{\pgfqpoint{3.550926in}{1.724665in}}%
\pgfpathlineto{\pgfqpoint{3.544623in}{1.746111in}}%
\pgfpathlineto{\pgfqpoint{3.538332in}{1.768449in}}%
\pgfpathlineto{\pgfqpoint{3.532047in}{1.791412in}}%
\pgfpathlineto{\pgfqpoint{3.525767in}{1.814729in}}%
\pgfpathlineto{\pgfqpoint{3.519487in}{1.838130in}}%
\pgfpathlineto{\pgfqpoint{3.507989in}{1.823695in}}%
\pgfpathlineto{\pgfqpoint{3.496504in}{1.810412in}}%
\pgfpathlineto{\pgfqpoint{3.485031in}{1.798150in}}%
\pgfpathlineto{\pgfqpoint{3.473567in}{1.786779in}}%
\pgfpathlineto{\pgfqpoint{3.462110in}{1.776166in}}%
\pgfpathlineto{\pgfqpoint{3.468448in}{1.757986in}}%
\pgfpathlineto{\pgfqpoint{3.474787in}{1.739728in}}%
\pgfpathlineto{\pgfqpoint{3.481127in}{1.721567in}}%
\pgfpathlineto{\pgfqpoint{3.487473in}{1.703679in}}%
\pgfpathclose%
\pgfusepath{stroke,fill}%
\end{pgfscope}%
\begin{pgfscope}%
\pgfpathrectangle{\pgfqpoint{0.887500in}{0.275000in}}{\pgfqpoint{4.225000in}{4.225000in}}%
\pgfusepath{clip}%
\pgfsetbuttcap%
\pgfsetroundjoin%
\definecolor{currentfill}{rgb}{0.185556,0.418570,0.556753}%
\pgfsetfillcolor{currentfill}%
\pgfsetfillopacity{0.700000}%
\pgfsetlinewidth{0.501875pt}%
\definecolor{currentstroke}{rgb}{1.000000,1.000000,1.000000}%
\pgfsetstrokecolor{currentstroke}%
\pgfsetstrokeopacity{0.500000}%
\pgfsetdash{}{0pt}%
\pgfpathmoveto{\pgfqpoint{3.959362in}{1.945694in}}%
\pgfpathlineto{\pgfqpoint{3.970704in}{1.952444in}}%
\pgfpathlineto{\pgfqpoint{3.982034in}{1.958799in}}%
\pgfpathlineto{\pgfqpoint{3.993352in}{1.964811in}}%
\pgfpathlineto{\pgfqpoint{4.004658in}{1.970532in}}%
\pgfpathlineto{\pgfqpoint{4.015954in}{1.976013in}}%
\pgfpathlineto{\pgfqpoint{4.009529in}{1.992677in}}%
\pgfpathlineto{\pgfqpoint{4.003108in}{2.009395in}}%
\pgfpathlineto{\pgfqpoint{3.996687in}{2.026086in}}%
\pgfpathlineto{\pgfqpoint{3.990266in}{2.042670in}}%
\pgfpathlineto{\pgfqpoint{3.983842in}{2.059067in}}%
\pgfpathlineto{\pgfqpoint{3.972549in}{2.053711in}}%
\pgfpathlineto{\pgfqpoint{3.961247in}{2.048137in}}%
\pgfpathlineto{\pgfqpoint{3.949935in}{2.042340in}}%
\pgfpathlineto{\pgfqpoint{3.938615in}{2.036316in}}%
\pgfpathlineto{\pgfqpoint{3.927284in}{2.030060in}}%
\pgfpathlineto{\pgfqpoint{3.933703in}{2.013477in}}%
\pgfpathlineto{\pgfqpoint{3.940119in}{1.996696in}}%
\pgfpathlineto{\pgfqpoint{3.946533in}{1.979769in}}%
\pgfpathlineto{\pgfqpoint{3.952947in}{1.962750in}}%
\pgfpathclose%
\pgfusepath{stroke,fill}%
\end{pgfscope}%
\begin{pgfscope}%
\pgfpathrectangle{\pgfqpoint{0.887500in}{0.275000in}}{\pgfqpoint{4.225000in}{4.225000in}}%
\pgfusepath{clip}%
\pgfsetbuttcap%
\pgfsetroundjoin%
\definecolor{currentfill}{rgb}{0.225863,0.330805,0.547314}%
\pgfsetfillcolor{currentfill}%
\pgfsetfillopacity{0.700000}%
\pgfsetlinewidth{0.501875pt}%
\definecolor{currentstroke}{rgb}{1.000000,1.000000,1.000000}%
\pgfsetstrokecolor{currentstroke}%
\pgfsetstrokeopacity{0.500000}%
\pgfsetdash{}{0pt}%
\pgfpathmoveto{\pgfqpoint{3.550926in}{1.724665in}}%
\pgfpathlineto{\pgfqpoint{3.562432in}{1.739663in}}%
\pgfpathlineto{\pgfqpoint{3.573970in}{1.757027in}}%
\pgfpathlineto{\pgfqpoint{3.585534in}{1.776093in}}%
\pgfpathlineto{\pgfqpoint{3.597116in}{1.796196in}}%
\pgfpathlineto{\pgfqpoint{3.608706in}{1.816672in}}%
\pgfpathlineto{\pgfqpoint{3.602387in}{1.837318in}}%
\pgfpathlineto{\pgfqpoint{3.596074in}{1.858345in}}%
\pgfpathlineto{\pgfqpoint{3.589764in}{1.879602in}}%
\pgfpathlineto{\pgfqpoint{3.583455in}{1.900938in}}%
\pgfpathlineto{\pgfqpoint{3.577146in}{1.922204in}}%
\pgfpathlineto{\pgfqpoint{3.565602in}{1.904763in}}%
\pgfpathlineto{\pgfqpoint{3.554060in}{1.887340in}}%
\pgfpathlineto{\pgfqpoint{3.542525in}{1.870230in}}%
\pgfpathlineto{\pgfqpoint{3.531000in}{1.853728in}}%
\pgfpathlineto{\pgfqpoint{3.519487in}{1.838130in}}%
\pgfpathlineto{\pgfqpoint{3.525767in}{1.814729in}}%
\pgfpathlineto{\pgfqpoint{3.532047in}{1.791412in}}%
\pgfpathlineto{\pgfqpoint{3.538332in}{1.768449in}}%
\pgfpathlineto{\pgfqpoint{3.544623in}{1.746111in}}%
\pgfpathclose%
\pgfusepath{stroke,fill}%
\end{pgfscope}%
\begin{pgfscope}%
\pgfpathrectangle{\pgfqpoint{0.887500in}{0.275000in}}{\pgfqpoint{4.225000in}{4.225000in}}%
\pgfusepath{clip}%
\pgfsetbuttcap%
\pgfsetroundjoin%
\definecolor{currentfill}{rgb}{0.199430,0.387607,0.554642}%
\pgfsetfillcolor{currentfill}%
\pgfsetfillopacity{0.700000}%
\pgfsetlinewidth{0.501875pt}%
\definecolor{currentstroke}{rgb}{1.000000,1.000000,1.000000}%
\pgfsetstrokecolor{currentstroke}%
\pgfsetstrokeopacity{0.500000}%
\pgfsetdash{}{0pt}%
\pgfpathmoveto{\pgfqpoint{3.519487in}{1.838130in}}%
\pgfpathlineto{\pgfqpoint{3.531000in}{1.853728in}}%
\pgfpathlineto{\pgfqpoint{3.542525in}{1.870230in}}%
\pgfpathlineto{\pgfqpoint{3.554060in}{1.887340in}}%
\pgfpathlineto{\pgfqpoint{3.565602in}{1.904763in}}%
\pgfpathlineto{\pgfqpoint{3.577146in}{1.922204in}}%
\pgfpathlineto{\pgfqpoint{3.570833in}{1.943248in}}%
\pgfpathlineto{\pgfqpoint{3.564517in}{1.963918in}}%
\pgfpathlineto{\pgfqpoint{3.558194in}{1.984110in}}%
\pgfpathlineto{\pgfqpoint{3.551866in}{2.003852in}}%
\pgfpathlineto{\pgfqpoint{3.545534in}{2.023200in}}%
\pgfpathlineto{\pgfqpoint{3.534039in}{2.009249in}}%
\pgfpathlineto{\pgfqpoint{3.522539in}{1.994915in}}%
\pgfpathlineto{\pgfqpoint{3.511034in}{1.980086in}}%
\pgfpathlineto{\pgfqpoint{3.499523in}{1.964650in}}%
\pgfpathlineto{\pgfqpoint{3.488005in}{1.948496in}}%
\pgfpathlineto{\pgfqpoint{3.494316in}{1.927649in}}%
\pgfpathlineto{\pgfqpoint{3.500619in}{1.906200in}}%
\pgfpathlineto{\pgfqpoint{3.506915in}{1.884102in}}%
\pgfpathlineto{\pgfqpoint{3.513204in}{1.861345in}}%
\pgfpathclose%
\pgfusepath{stroke,fill}%
\end{pgfscope}%
\begin{pgfscope}%
\pgfpathrectangle{\pgfqpoint{0.887500in}{0.275000in}}{\pgfqpoint{4.225000in}{4.225000in}}%
\pgfusepath{clip}%
\pgfsetbuttcap%
\pgfsetroundjoin%
\definecolor{currentfill}{rgb}{0.177423,0.437527,0.557565}%
\pgfsetfillcolor{currentfill}%
\pgfsetfillopacity{0.700000}%
\pgfsetlinewidth{0.501875pt}%
\definecolor{currentstroke}{rgb}{1.000000,1.000000,1.000000}%
\pgfsetstrokecolor{currentstroke}%
\pgfsetstrokeopacity{0.500000}%
\pgfsetdash{}{0pt}%
\pgfpathmoveto{\pgfqpoint{2.341981in}{2.001294in}}%
\pgfpathlineto{\pgfqpoint{2.353656in}{2.005059in}}%
\pgfpathlineto{\pgfqpoint{2.365325in}{2.008827in}}%
\pgfpathlineto{\pgfqpoint{2.376989in}{2.012594in}}%
\pgfpathlineto{\pgfqpoint{2.388647in}{2.016356in}}%
\pgfpathlineto{\pgfqpoint{2.400300in}{2.020110in}}%
\pgfpathlineto{\pgfqpoint{2.394205in}{2.029691in}}%
\pgfpathlineto{\pgfqpoint{2.388115in}{2.039232in}}%
\pgfpathlineto{\pgfqpoint{2.382029in}{2.048733in}}%
\pgfpathlineto{\pgfqpoint{2.375948in}{2.058197in}}%
\pgfpathlineto{\pgfqpoint{2.369871in}{2.067623in}}%
\pgfpathlineto{\pgfqpoint{2.358228in}{2.063925in}}%
\pgfpathlineto{\pgfqpoint{2.346580in}{2.060222in}}%
\pgfpathlineto{\pgfqpoint{2.334927in}{2.056515in}}%
\pgfpathlineto{\pgfqpoint{2.323267in}{2.052808in}}%
\pgfpathlineto{\pgfqpoint{2.311602in}{2.049104in}}%
\pgfpathlineto{\pgfqpoint{2.317669in}{2.039621in}}%
\pgfpathlineto{\pgfqpoint{2.323740in}{2.030098in}}%
\pgfpathlineto{\pgfqpoint{2.329816in}{2.020537in}}%
\pgfpathlineto{\pgfqpoint{2.335896in}{2.010936in}}%
\pgfpathclose%
\pgfusepath{stroke,fill}%
\end{pgfscope}%
\begin{pgfscope}%
\pgfpathrectangle{\pgfqpoint{0.887500in}{0.275000in}}{\pgfqpoint{4.225000in}{4.225000in}}%
\pgfusepath{clip}%
\pgfsetbuttcap%
\pgfsetroundjoin%
\definecolor{currentfill}{rgb}{0.199430,0.387607,0.554642}%
\pgfsetfillcolor{currentfill}%
\pgfsetfillopacity{0.700000}%
\pgfsetlinewidth{0.501875pt}%
\definecolor{currentstroke}{rgb}{1.000000,1.000000,1.000000}%
\pgfsetstrokecolor{currentstroke}%
\pgfsetstrokeopacity{0.500000}%
\pgfsetdash{}{0pt}%
\pgfpathmoveto{\pgfqpoint{2.755784in}{1.895526in}}%
\pgfpathlineto{\pgfqpoint{2.767358in}{1.899333in}}%
\pgfpathlineto{\pgfqpoint{2.778927in}{1.903154in}}%
\pgfpathlineto{\pgfqpoint{2.790490in}{1.906983in}}%
\pgfpathlineto{\pgfqpoint{2.802047in}{1.910803in}}%
\pgfpathlineto{\pgfqpoint{2.813599in}{1.914593in}}%
\pgfpathlineto{\pgfqpoint{2.807371in}{1.924822in}}%
\pgfpathlineto{\pgfqpoint{2.801147in}{1.935003in}}%
\pgfpathlineto{\pgfqpoint{2.794927in}{1.945136in}}%
\pgfpathlineto{\pgfqpoint{2.788712in}{1.955221in}}%
\pgfpathlineto{\pgfqpoint{2.782500in}{1.965260in}}%
\pgfpathlineto{\pgfqpoint{2.770957in}{1.961522in}}%
\pgfpathlineto{\pgfqpoint{2.759410in}{1.957752in}}%
\pgfpathlineto{\pgfqpoint{2.747856in}{1.953971in}}%
\pgfpathlineto{\pgfqpoint{2.736297in}{1.950201in}}%
\pgfpathlineto{\pgfqpoint{2.724732in}{1.946446in}}%
\pgfpathlineto{\pgfqpoint{2.730935in}{1.936355in}}%
\pgfpathlineto{\pgfqpoint{2.737141in}{1.926218in}}%
\pgfpathlineto{\pgfqpoint{2.743351in}{1.916035in}}%
\pgfpathlineto{\pgfqpoint{2.749566in}{1.905804in}}%
\pgfpathclose%
\pgfusepath{stroke,fill}%
\end{pgfscope}%
\begin{pgfscope}%
\pgfpathrectangle{\pgfqpoint{0.887500in}{0.275000in}}{\pgfqpoint{4.225000in}{4.225000in}}%
\pgfusepath{clip}%
\pgfsetbuttcap%
\pgfsetroundjoin%
\definecolor{currentfill}{rgb}{0.174274,0.445044,0.557792}%
\pgfsetfillcolor{currentfill}%
\pgfsetfillopacity{0.700000}%
\pgfsetlinewidth{0.501875pt}%
\definecolor{currentstroke}{rgb}{1.000000,1.000000,1.000000}%
\pgfsetstrokecolor{currentstroke}%
\pgfsetstrokeopacity{0.500000}%
\pgfsetdash{}{0pt}%
\pgfpathmoveto{\pgfqpoint{3.870494in}{1.994974in}}%
\pgfpathlineto{\pgfqpoint{3.881872in}{2.002606in}}%
\pgfpathlineto{\pgfqpoint{3.893239in}{2.009862in}}%
\pgfpathlineto{\pgfqpoint{3.904596in}{2.016838in}}%
\pgfpathlineto{\pgfqpoint{3.915945in}{2.023569in}}%
\pgfpathlineto{\pgfqpoint{3.927284in}{2.030060in}}%
\pgfpathlineto{\pgfqpoint{3.920862in}{2.046390in}}%
\pgfpathlineto{\pgfqpoint{3.914435in}{2.062415in}}%
\pgfpathlineto{\pgfqpoint{3.908003in}{2.078138in}}%
\pgfpathlineto{\pgfqpoint{3.901568in}{2.093602in}}%
\pgfpathlineto{\pgfqpoint{3.895131in}{2.108852in}}%
\pgfpathlineto{\pgfqpoint{3.883811in}{2.103161in}}%
\pgfpathlineto{\pgfqpoint{3.872488in}{2.097525in}}%
\pgfpathlineto{\pgfqpoint{3.861163in}{2.091990in}}%
\pgfpathlineto{\pgfqpoint{3.849835in}{2.086522in}}%
\pgfpathlineto{\pgfqpoint{3.838499in}{2.080917in}}%
\pgfpathlineto{\pgfqpoint{3.844905in}{2.064250in}}%
\pgfpathlineto{\pgfqpoint{3.851308in}{2.047336in}}%
\pgfpathlineto{\pgfqpoint{3.857708in}{2.030157in}}%
\pgfpathlineto{\pgfqpoint{3.864103in}{2.012698in}}%
\pgfpathclose%
\pgfusepath{stroke,fill}%
\end{pgfscope}%
\begin{pgfscope}%
\pgfpathrectangle{\pgfqpoint{0.887500in}{0.275000in}}{\pgfqpoint{4.225000in}{4.225000in}}%
\pgfusepath{clip}%
\pgfsetbuttcap%
\pgfsetroundjoin%
\definecolor{currentfill}{rgb}{0.244972,0.287675,0.537260}%
\pgfsetfillcolor{currentfill}%
\pgfsetfillopacity{0.700000}%
\pgfsetlinewidth{0.501875pt}%
\definecolor{currentstroke}{rgb}{1.000000,1.000000,1.000000}%
\pgfsetstrokecolor{currentstroke}%
\pgfsetstrokeopacity{0.500000}%
\pgfsetdash{}{0pt}%
\pgfpathmoveto{\pgfqpoint{3.347667in}{1.698957in}}%
\pgfpathlineto{\pgfqpoint{3.359104in}{1.703887in}}%
\pgfpathlineto{\pgfqpoint{3.370543in}{1.709564in}}%
\pgfpathlineto{\pgfqpoint{3.381982in}{1.715967in}}%
\pgfpathlineto{\pgfqpoint{3.393424in}{1.723033in}}%
\pgfpathlineto{\pgfqpoint{3.404868in}{1.730701in}}%
\pgfpathlineto{\pgfqpoint{3.398482in}{1.743207in}}%
\pgfpathlineto{\pgfqpoint{3.392097in}{1.755364in}}%
\pgfpathlineto{\pgfqpoint{3.385713in}{1.767162in}}%
\pgfpathlineto{\pgfqpoint{3.379329in}{1.778722in}}%
\pgfpathlineto{\pgfqpoint{3.372948in}{1.790186in}}%
\pgfpathlineto{\pgfqpoint{3.361495in}{1.779994in}}%
\pgfpathlineto{\pgfqpoint{3.350050in}{1.771141in}}%
\pgfpathlineto{\pgfqpoint{3.338612in}{1.763613in}}%
\pgfpathlineto{\pgfqpoint{3.327180in}{1.757401in}}%
\pgfpathlineto{\pgfqpoint{3.315752in}{1.752441in}}%
\pgfpathlineto{\pgfqpoint{3.322129in}{1.741983in}}%
\pgfpathlineto{\pgfqpoint{3.328510in}{1.731491in}}%
\pgfpathlineto{\pgfqpoint{3.334894in}{1.720867in}}%
\pgfpathlineto{\pgfqpoint{3.341279in}{1.710027in}}%
\pgfpathclose%
\pgfusepath{stroke,fill}%
\end{pgfscope}%
\begin{pgfscope}%
\pgfpathrectangle{\pgfqpoint{0.887500in}{0.275000in}}{\pgfqpoint{4.225000in}{4.225000in}}%
\pgfusepath{clip}%
\pgfsetbuttcap%
\pgfsetroundjoin%
\definecolor{currentfill}{rgb}{0.248629,0.278775,0.534556}%
\pgfsetfillcolor{currentfill}%
\pgfsetfillopacity{0.700000}%
\pgfsetlinewidth{0.501875pt}%
\definecolor{currentstroke}{rgb}{1.000000,1.000000,1.000000}%
\pgfsetstrokecolor{currentstroke}%
\pgfsetstrokeopacity{0.500000}%
\pgfsetdash{}{0pt}%
\pgfpathmoveto{\pgfqpoint{3.582748in}{1.639930in}}%
\pgfpathlineto{\pgfqpoint{3.594237in}{1.653110in}}%
\pgfpathlineto{\pgfqpoint{3.605762in}{1.668790in}}%
\pgfpathlineto{\pgfqpoint{3.617314in}{1.686309in}}%
\pgfpathlineto{\pgfqpoint{3.628887in}{1.705008in}}%
\pgfpathlineto{\pgfqpoint{3.640470in}{1.724223in}}%
\pgfpathlineto{\pgfqpoint{3.634088in}{1.740879in}}%
\pgfpathlineto{\pgfqpoint{3.627722in}{1.758512in}}%
\pgfpathlineto{\pgfqpoint{3.621371in}{1.777119in}}%
\pgfpathlineto{\pgfqpoint{3.615034in}{1.796556in}}%
\pgfpathlineto{\pgfqpoint{3.608706in}{1.816672in}}%
\pgfpathlineto{\pgfqpoint{3.597116in}{1.796196in}}%
\pgfpathlineto{\pgfqpoint{3.585534in}{1.776093in}}%
\pgfpathlineto{\pgfqpoint{3.573970in}{1.757027in}}%
\pgfpathlineto{\pgfqpoint{3.562432in}{1.739663in}}%
\pgfpathlineto{\pgfqpoint{3.550926in}{1.724665in}}%
\pgfpathlineto{\pgfqpoint{3.557244in}{1.704381in}}%
\pgfpathlineto{\pgfqpoint{3.563582in}{1.685527in}}%
\pgfpathlineto{\pgfqpoint{3.569942in}{1.668370in}}%
\pgfpathlineto{\pgfqpoint{3.576331in}{1.653171in}}%
\pgfpathclose%
\pgfusepath{stroke,fill}%
\end{pgfscope}%
\begin{pgfscope}%
\pgfpathrectangle{\pgfqpoint{0.887500in}{0.275000in}}{\pgfqpoint{4.225000in}{4.225000in}}%
\pgfusepath{clip}%
\pgfsetbuttcap%
\pgfsetroundjoin%
\definecolor{currentfill}{rgb}{0.165117,0.467423,0.558141}%
\pgfsetfillcolor{currentfill}%
\pgfsetfillopacity{0.700000}%
\pgfsetlinewidth{0.501875pt}%
\definecolor{currentstroke}{rgb}{1.000000,1.000000,1.000000}%
\pgfsetstrokecolor{currentstroke}%
\pgfsetstrokeopacity{0.500000}%
\pgfsetdash{}{0pt}%
\pgfpathmoveto{\pgfqpoint{2.017091in}{2.069539in}}%
\pgfpathlineto{\pgfqpoint{2.028848in}{2.073249in}}%
\pgfpathlineto{\pgfqpoint{2.040600in}{2.076954in}}%
\pgfpathlineto{\pgfqpoint{2.052346in}{2.080653in}}%
\pgfpathlineto{\pgfqpoint{2.064086in}{2.084343in}}%
\pgfpathlineto{\pgfqpoint{2.075821in}{2.088021in}}%
\pgfpathlineto{\pgfqpoint{2.069834in}{2.097297in}}%
\pgfpathlineto{\pgfqpoint{2.063852in}{2.106540in}}%
\pgfpathlineto{\pgfqpoint{2.057874in}{2.115753in}}%
\pgfpathlineto{\pgfqpoint{2.051900in}{2.124935in}}%
\pgfpathlineto{\pgfqpoint{2.045932in}{2.134088in}}%
\pgfpathlineto{\pgfqpoint{2.034207in}{2.130467in}}%
\pgfpathlineto{\pgfqpoint{2.022477in}{2.126836in}}%
\pgfpathlineto{\pgfqpoint{2.010742in}{2.123197in}}%
\pgfpathlineto{\pgfqpoint{1.999001in}{2.119552in}}%
\pgfpathlineto{\pgfqpoint{1.987254in}{2.115903in}}%
\pgfpathlineto{\pgfqpoint{1.993213in}{2.106692in}}%
\pgfpathlineto{\pgfqpoint{1.999175in}{2.097451in}}%
\pgfpathlineto{\pgfqpoint{2.005143in}{2.088179in}}%
\pgfpathlineto{\pgfqpoint{2.011115in}{2.078875in}}%
\pgfpathclose%
\pgfusepath{stroke,fill}%
\end{pgfscope}%
\begin{pgfscope}%
\pgfpathrectangle{\pgfqpoint{0.887500in}{0.275000in}}{\pgfqpoint{4.225000in}{4.225000in}}%
\pgfusepath{clip}%
\pgfsetbuttcap%
\pgfsetroundjoin%
\definecolor{currentfill}{rgb}{0.182256,0.426184,0.557120}%
\pgfsetfillcolor{currentfill}%
\pgfsetfillopacity{0.700000}%
\pgfsetlinewidth{0.501875pt}%
\definecolor{currentstroke}{rgb}{1.000000,1.000000,1.000000}%
\pgfsetstrokecolor{currentstroke}%
\pgfsetstrokeopacity{0.500000}%
\pgfsetdash{}{0pt}%
\pgfpathmoveto{\pgfqpoint{3.577146in}{1.922204in}}%
\pgfpathlineto{\pgfqpoint{3.588688in}{1.939366in}}%
\pgfpathlineto{\pgfqpoint{3.600225in}{1.955955in}}%
\pgfpathlineto{\pgfqpoint{3.611754in}{1.971831in}}%
\pgfpathlineto{\pgfqpoint{3.623276in}{1.987087in}}%
\pgfpathlineto{\pgfqpoint{3.634791in}{2.001837in}}%
\pgfpathlineto{\pgfqpoint{3.628436in}{2.020193in}}%
\pgfpathlineto{\pgfqpoint{3.622077in}{2.038215in}}%
\pgfpathlineto{\pgfqpoint{3.615715in}{2.055885in}}%
\pgfpathlineto{\pgfqpoint{3.609350in}{2.073254in}}%
\pgfpathlineto{\pgfqpoint{3.602983in}{2.090392in}}%
\pgfpathlineto{\pgfqpoint{3.591496in}{2.077154in}}%
\pgfpathlineto{\pgfqpoint{3.580007in}{2.063820in}}%
\pgfpathlineto{\pgfqpoint{3.568518in}{2.050396in}}%
\pgfpathlineto{\pgfqpoint{3.557027in}{2.036878in}}%
\pgfpathlineto{\pgfqpoint{3.545534in}{2.023200in}}%
\pgfpathlineto{\pgfqpoint{3.551866in}{2.003852in}}%
\pgfpathlineto{\pgfqpoint{3.558194in}{1.984110in}}%
\pgfpathlineto{\pgfqpoint{3.564517in}{1.963918in}}%
\pgfpathlineto{\pgfqpoint{3.570833in}{1.943248in}}%
\pgfpathclose%
\pgfusepath{stroke,fill}%
\end{pgfscope}%
\begin{pgfscope}%
\pgfpathrectangle{\pgfqpoint{0.887500in}{0.275000in}}{\pgfqpoint{4.225000in}{4.225000in}}%
\pgfusepath{clip}%
\pgfsetbuttcap%
\pgfsetroundjoin%
\definecolor{currentfill}{rgb}{0.156270,0.489624,0.557936}%
\pgfsetfillcolor{currentfill}%
\pgfsetfillopacity{0.700000}%
\pgfsetlinewidth{0.501875pt}%
\definecolor{currentstroke}{rgb}{1.000000,1.000000,1.000000}%
\pgfsetstrokecolor{currentstroke}%
\pgfsetstrokeopacity{0.500000}%
\pgfsetdash{}{0pt}%
\pgfpathmoveto{\pgfqpoint{3.692316in}{2.071768in}}%
\pgfpathlineto{\pgfqpoint{3.703813in}{2.085171in}}%
\pgfpathlineto{\pgfqpoint{3.715301in}{2.098009in}}%
\pgfpathlineto{\pgfqpoint{3.726774in}{2.110053in}}%
\pgfpathlineto{\pgfqpoint{3.738229in}{2.121070in}}%
\pgfpathlineto{\pgfqpoint{3.749661in}{2.130831in}}%
\pgfpathlineto{\pgfqpoint{3.743263in}{2.147239in}}%
\pgfpathlineto{\pgfqpoint{3.736859in}{2.163224in}}%
\pgfpathlineto{\pgfqpoint{3.730450in}{2.178785in}}%
\pgfpathlineto{\pgfqpoint{3.724037in}{2.193978in}}%
\pgfpathlineto{\pgfqpoint{3.717621in}{2.208870in}}%
\pgfpathlineto{\pgfqpoint{3.706208in}{2.199916in}}%
\pgfpathlineto{\pgfqpoint{3.694776in}{2.189868in}}%
\pgfpathlineto{\pgfqpoint{3.683327in}{2.178906in}}%
\pgfpathlineto{\pgfqpoint{3.671865in}{2.167210in}}%
\pgfpathlineto{\pgfqpoint{3.660394in}{2.154960in}}%
\pgfpathlineto{\pgfqpoint{3.666785in}{2.138998in}}%
\pgfpathlineto{\pgfqpoint{3.673174in}{2.122759in}}%
\pgfpathlineto{\pgfqpoint{3.679560in}{2.106170in}}%
\pgfpathlineto{\pgfqpoint{3.685941in}{2.089171in}}%
\pgfpathclose%
\pgfusepath{stroke,fill}%
\end{pgfscope}%
\begin{pgfscope}%
\pgfpathrectangle{\pgfqpoint{0.887500in}{0.275000in}}{\pgfqpoint{4.225000in}{4.225000in}}%
\pgfusepath{clip}%
\pgfsetbuttcap%
\pgfsetroundjoin%
\definecolor{currentfill}{rgb}{0.262138,0.242286,0.520837}%
\pgfsetfillcolor{currentfill}%
\pgfsetfillopacity{0.700000}%
\pgfsetlinewidth{0.501875pt}%
\definecolor{currentstroke}{rgb}{1.000000,1.000000,1.000000}%
\pgfsetstrokecolor{currentstroke}%
\pgfsetstrokeopacity{0.500000}%
\pgfsetdash{}{0pt}%
\pgfpathmoveto{\pgfqpoint{3.525758in}{1.611714in}}%
\pgfpathlineto{\pgfqpoint{3.537124in}{1.614049in}}%
\pgfpathlineto{\pgfqpoint{3.548497in}{1.617451in}}%
\pgfpathlineto{\pgfqpoint{3.559886in}{1.622507in}}%
\pgfpathlineto{\pgfqpoint{3.571300in}{1.629805in}}%
\pgfpathlineto{\pgfqpoint{3.582748in}{1.639930in}}%
\pgfpathlineto{\pgfqpoint{3.576331in}{1.653171in}}%
\pgfpathlineto{\pgfqpoint{3.569942in}{1.668370in}}%
\pgfpathlineto{\pgfqpoint{3.563582in}{1.685527in}}%
\pgfpathlineto{\pgfqpoint{3.557244in}{1.704381in}}%
\pgfpathlineto{\pgfqpoint{3.550926in}{1.724665in}}%
\pgfpathlineto{\pgfqpoint{3.539458in}{1.712596in}}%
\pgfpathlineto{\pgfqpoint{3.528022in}{1.703259in}}%
\pgfpathlineto{\pgfqpoint{3.516610in}{1.696115in}}%
\pgfpathlineto{\pgfqpoint{3.505213in}{1.690622in}}%
\pgfpathlineto{\pgfqpoint{3.493824in}{1.686240in}}%
\pgfpathlineto{\pgfqpoint{3.500184in}{1.669424in}}%
\pgfpathlineto{\pgfqpoint{3.506556in}{1.653406in}}%
\pgfpathlineto{\pgfqpoint{3.512940in}{1.638362in}}%
\pgfpathlineto{\pgfqpoint{3.519341in}{1.624461in}}%
\pgfpathclose%
\pgfusepath{stroke,fill}%
\end{pgfscope}%
\begin{pgfscope}%
\pgfpathrectangle{\pgfqpoint{0.887500in}{0.275000in}}{\pgfqpoint{4.225000in}{4.225000in}}%
\pgfusepath{clip}%
\pgfsetbuttcap%
\pgfsetroundjoin%
\definecolor{currentfill}{rgb}{0.168126,0.459988,0.558082}%
\pgfsetfillcolor{currentfill}%
\pgfsetfillopacity{0.700000}%
\pgfsetlinewidth{0.501875pt}%
\definecolor{currentstroke}{rgb}{1.000000,1.000000,1.000000}%
\pgfsetstrokecolor{currentstroke}%
\pgfsetstrokeopacity{0.500000}%
\pgfsetdash{}{0pt}%
\pgfpathmoveto{\pgfqpoint{3.634791in}{2.001837in}}%
\pgfpathlineto{\pgfqpoint{3.646300in}{2.016193in}}%
\pgfpathlineto{\pgfqpoint{3.657807in}{2.030270in}}%
\pgfpathlineto{\pgfqpoint{3.669311in}{2.044180in}}%
\pgfpathlineto{\pgfqpoint{3.680814in}{2.058030in}}%
\pgfpathlineto{\pgfqpoint{3.692316in}{2.071768in}}%
\pgfpathlineto{\pgfqpoint{3.685941in}{2.089171in}}%
\pgfpathlineto{\pgfqpoint{3.679560in}{2.106170in}}%
\pgfpathlineto{\pgfqpoint{3.673174in}{2.122759in}}%
\pgfpathlineto{\pgfqpoint{3.666785in}{2.138998in}}%
\pgfpathlineto{\pgfqpoint{3.660394in}{2.154960in}}%
\pgfpathlineto{\pgfqpoint{3.648916in}{2.142335in}}%
\pgfpathlineto{\pgfqpoint{3.637436in}{2.129510in}}%
\pgfpathlineto{\pgfqpoint{3.625953in}{2.116574in}}%
\pgfpathlineto{\pgfqpoint{3.614469in}{2.103534in}}%
\pgfpathlineto{\pgfqpoint{3.602983in}{2.090392in}}%
\pgfpathlineto{\pgfqpoint{3.609350in}{2.073254in}}%
\pgfpathlineto{\pgfqpoint{3.615715in}{2.055885in}}%
\pgfpathlineto{\pgfqpoint{3.622077in}{2.038215in}}%
\pgfpathlineto{\pgfqpoint{3.628436in}{2.020193in}}%
\pgfpathclose%
\pgfusepath{stroke,fill}%
\end{pgfscope}%
\begin{pgfscope}%
\pgfpathrectangle{\pgfqpoint{0.887500in}{0.275000in}}{\pgfqpoint{4.225000in}{4.225000in}}%
\pgfusepath{clip}%
\pgfsetbuttcap%
\pgfsetroundjoin%
\definecolor{currentfill}{rgb}{0.244972,0.287675,0.537260}%
\pgfsetfillcolor{currentfill}%
\pgfsetfillopacity{0.700000}%
\pgfsetlinewidth{0.501875pt}%
\definecolor{currentstroke}{rgb}{1.000000,1.000000,1.000000}%
\pgfsetstrokecolor{currentstroke}%
\pgfsetstrokeopacity{0.500000}%
\pgfsetdash{}{0pt}%
\pgfpathmoveto{\pgfqpoint{4.347677in}{1.688670in}}%
\pgfpathlineto{\pgfqpoint{4.358904in}{1.693743in}}%
\pgfpathlineto{\pgfqpoint{4.370104in}{1.698154in}}%
\pgfpathlineto{\pgfqpoint{4.381282in}{1.702081in}}%
\pgfpathlineto{\pgfqpoint{4.392444in}{1.705703in}}%
\pgfpathlineto{\pgfqpoint{4.403597in}{1.709198in}}%
\pgfpathlineto{\pgfqpoint{4.397084in}{1.724889in}}%
\pgfpathlineto{\pgfqpoint{4.390563in}{1.740284in}}%
\pgfpathlineto{\pgfqpoint{4.384035in}{1.755405in}}%
\pgfpathlineto{\pgfqpoint{4.377500in}{1.770274in}}%
\pgfpathlineto{\pgfqpoint{4.370961in}{1.784915in}}%
\pgfpathlineto{\pgfqpoint{4.359817in}{1.781656in}}%
\pgfpathlineto{\pgfqpoint{4.348664in}{1.778314in}}%
\pgfpathlineto{\pgfqpoint{4.337502in}{1.774851in}}%
\pgfpathlineto{\pgfqpoint{4.326329in}{1.771226in}}%
\pgfpathlineto{\pgfqpoint{4.315144in}{1.767399in}}%
\pgfpathlineto{\pgfqpoint{4.321676in}{1.752577in}}%
\pgfpathlineto{\pgfqpoint{4.328197in}{1.737337in}}%
\pgfpathlineto{\pgfqpoint{4.334706in}{1.721635in}}%
\pgfpathlineto{\pgfqpoint{4.341199in}{1.705427in}}%
\pgfpathclose%
\pgfusepath{stroke,fill}%
\end{pgfscope}%
\begin{pgfscope}%
\pgfpathrectangle{\pgfqpoint{0.887500in}{0.275000in}}{\pgfqpoint{4.225000in}{4.225000in}}%
\pgfusepath{clip}%
\pgfsetbuttcap%
\pgfsetroundjoin%
\definecolor{currentfill}{rgb}{0.163625,0.471133,0.558148}%
\pgfsetfillcolor{currentfill}%
\pgfsetfillopacity{0.700000}%
\pgfsetlinewidth{0.501875pt}%
\definecolor{currentstroke}{rgb}{1.000000,1.000000,1.000000}%
\pgfsetstrokecolor{currentstroke}%
\pgfsetstrokeopacity{0.500000}%
\pgfsetdash{}{0pt}%
\pgfpathmoveto{\pgfqpoint{3.781574in}{2.043074in}}%
\pgfpathlineto{\pgfqpoint{3.793004in}{2.052689in}}%
\pgfpathlineto{\pgfqpoint{3.804409in}{2.061062in}}%
\pgfpathlineto{\pgfqpoint{3.815790in}{2.068410in}}%
\pgfpathlineto{\pgfqpoint{3.827152in}{2.074954in}}%
\pgfpathlineto{\pgfqpoint{3.838499in}{2.080917in}}%
\pgfpathlineto{\pgfqpoint{3.832090in}{2.097355in}}%
\pgfpathlineto{\pgfqpoint{3.825679in}{2.113579in}}%
\pgfpathlineto{\pgfqpoint{3.819266in}{2.129608in}}%
\pgfpathlineto{\pgfqpoint{3.812851in}{2.145459in}}%
\pgfpathlineto{\pgfqpoint{3.806436in}{2.161149in}}%
\pgfpathlineto{\pgfqpoint{3.795120in}{2.156754in}}%
\pgfpathlineto{\pgfqpoint{3.783790in}{2.151807in}}%
\pgfpathlineto{\pgfqpoint{3.772439in}{2.146025in}}%
\pgfpathlineto{\pgfqpoint{3.761064in}{2.139125in}}%
\pgfpathlineto{\pgfqpoint{3.749661in}{2.130831in}}%
\pgfpathlineto{\pgfqpoint{3.756054in}{2.114017in}}%
\pgfpathlineto{\pgfqpoint{3.762441in}{2.096815in}}%
\pgfpathlineto{\pgfqpoint{3.768824in}{2.079245in}}%
\pgfpathlineto{\pgfqpoint{3.775202in}{2.061325in}}%
\pgfpathclose%
\pgfusepath{stroke,fill}%
\end{pgfscope}%
\begin{pgfscope}%
\pgfpathrectangle{\pgfqpoint{0.887500in}{0.275000in}}{\pgfqpoint{4.225000in}{4.225000in}}%
\pgfusepath{clip}%
\pgfsetbuttcap%
\pgfsetroundjoin%
\definecolor{currentfill}{rgb}{0.206756,0.371758,0.553117}%
\pgfsetfillcolor{currentfill}%
\pgfsetfillopacity{0.700000}%
\pgfsetlinewidth{0.501875pt}%
\definecolor{currentstroke}{rgb}{1.000000,1.000000,1.000000}%
\pgfsetstrokecolor{currentstroke}%
\pgfsetstrokeopacity{0.500000}%
\pgfsetdash{}{0pt}%
\pgfpathmoveto{\pgfqpoint{2.844797in}{1.862710in}}%
\pgfpathlineto{\pgfqpoint{2.856353in}{1.866509in}}%
\pgfpathlineto{\pgfqpoint{2.867903in}{1.870250in}}%
\pgfpathlineto{\pgfqpoint{2.879448in}{1.873915in}}%
\pgfpathlineto{\pgfqpoint{2.890988in}{1.877489in}}%
\pgfpathlineto{\pgfqpoint{2.902522in}{1.880983in}}%
\pgfpathlineto{\pgfqpoint{2.896266in}{1.891370in}}%
\pgfpathlineto{\pgfqpoint{2.890014in}{1.901709in}}%
\pgfpathlineto{\pgfqpoint{2.883765in}{1.912001in}}%
\pgfpathlineto{\pgfqpoint{2.877520in}{1.922246in}}%
\pgfpathlineto{\pgfqpoint{2.871279in}{1.932445in}}%
\pgfpathlineto{\pgfqpoint{2.859754in}{1.929064in}}%
\pgfpathlineto{\pgfqpoint{2.848223in}{1.925589in}}%
\pgfpathlineto{\pgfqpoint{2.836687in}{1.922006in}}%
\pgfpathlineto{\pgfqpoint{2.825146in}{1.918334in}}%
\pgfpathlineto{\pgfqpoint{2.813599in}{1.914593in}}%
\pgfpathlineto{\pgfqpoint{2.819831in}{1.904316in}}%
\pgfpathlineto{\pgfqpoint{2.826066in}{1.893989in}}%
\pgfpathlineto{\pgfqpoint{2.832306in}{1.883613in}}%
\pgfpathlineto{\pgfqpoint{2.838550in}{1.873187in}}%
\pgfpathclose%
\pgfusepath{stroke,fill}%
\end{pgfscope}%
\begin{pgfscope}%
\pgfpathrectangle{\pgfqpoint{0.887500in}{0.275000in}}{\pgfqpoint{4.225000in}{4.225000in}}%
\pgfusepath{clip}%
\pgfsetbuttcap%
\pgfsetroundjoin%
\definecolor{currentfill}{rgb}{0.182256,0.426184,0.557120}%
\pgfsetfillcolor{currentfill}%
\pgfsetfillopacity{0.700000}%
\pgfsetlinewidth{0.501875pt}%
\definecolor{currentstroke}{rgb}{1.000000,1.000000,1.000000}%
\pgfsetstrokecolor{currentstroke}%
\pgfsetstrokeopacity{0.500000}%
\pgfsetdash{}{0pt}%
\pgfpathmoveto{\pgfqpoint{2.430839in}{1.971565in}}%
\pgfpathlineto{\pgfqpoint{2.442497in}{1.975365in}}%
\pgfpathlineto{\pgfqpoint{2.454148in}{1.979149in}}%
\pgfpathlineto{\pgfqpoint{2.465795in}{1.982915in}}%
\pgfpathlineto{\pgfqpoint{2.477435in}{1.986664in}}%
\pgfpathlineto{\pgfqpoint{2.489071in}{1.990393in}}%
\pgfpathlineto{\pgfqpoint{2.482944in}{2.000139in}}%
\pgfpathlineto{\pgfqpoint{2.476822in}{2.009838in}}%
\pgfpathlineto{\pgfqpoint{2.470704in}{2.019494in}}%
\pgfpathlineto{\pgfqpoint{2.464590in}{2.029106in}}%
\pgfpathlineto{\pgfqpoint{2.458481in}{2.038675in}}%
\pgfpathlineto{\pgfqpoint{2.446856in}{2.034996in}}%
\pgfpathlineto{\pgfqpoint{2.435225in}{2.031299in}}%
\pgfpathlineto{\pgfqpoint{2.423589in}{2.027584in}}%
\pgfpathlineto{\pgfqpoint{2.411947in}{2.023854in}}%
\pgfpathlineto{\pgfqpoint{2.400300in}{2.020110in}}%
\pgfpathlineto{\pgfqpoint{2.406399in}{2.010488in}}%
\pgfpathlineto{\pgfqpoint{2.412503in}{2.000824in}}%
\pgfpathlineto{\pgfqpoint{2.418610in}{1.991116in}}%
\pgfpathlineto{\pgfqpoint{2.424723in}{1.981363in}}%
\pgfpathclose%
\pgfusepath{stroke,fill}%
\end{pgfscope}%
\begin{pgfscope}%
\pgfpathrectangle{\pgfqpoint{0.887500in}{0.275000in}}{\pgfqpoint{4.225000in}{4.225000in}}%
\pgfusepath{clip}%
\pgfsetbuttcap%
\pgfsetroundjoin%
\definecolor{currentfill}{rgb}{0.203063,0.379716,0.553925}%
\pgfsetfillcolor{currentfill}%
\pgfsetfillopacity{0.700000}%
\pgfsetlinewidth{0.501875pt}%
\definecolor{currentstroke}{rgb}{1.000000,1.000000,1.000000}%
\pgfsetstrokecolor{currentstroke}%
\pgfsetstrokeopacity{0.500000}%
\pgfsetdash{}{0pt}%
\pgfpathmoveto{\pgfqpoint{3.608706in}{1.816672in}}%
\pgfpathlineto{\pgfqpoint{3.620297in}{1.836851in}}%
\pgfpathlineto{\pgfqpoint{3.631878in}{1.856071in}}%
\pgfpathlineto{\pgfqpoint{3.643442in}{1.873998in}}%
\pgfpathlineto{\pgfqpoint{3.654992in}{1.890804in}}%
\pgfpathlineto{\pgfqpoint{3.666530in}{1.906697in}}%
\pgfpathlineto{\pgfqpoint{3.660185in}{1.926019in}}%
\pgfpathlineto{\pgfqpoint{3.653839in}{1.945240in}}%
\pgfpathlineto{\pgfqpoint{3.647492in}{1.964314in}}%
\pgfpathlineto{\pgfqpoint{3.641142in}{1.983196in}}%
\pgfpathlineto{\pgfqpoint{3.634791in}{2.001837in}}%
\pgfpathlineto{\pgfqpoint{3.623276in}{1.987087in}}%
\pgfpathlineto{\pgfqpoint{3.611754in}{1.971831in}}%
\pgfpathlineto{\pgfqpoint{3.600225in}{1.955955in}}%
\pgfpathlineto{\pgfqpoint{3.588688in}{1.939366in}}%
\pgfpathlineto{\pgfqpoint{3.577146in}{1.922204in}}%
\pgfpathlineto{\pgfqpoint{3.583455in}{1.900938in}}%
\pgfpathlineto{\pgfqpoint{3.589764in}{1.879602in}}%
\pgfpathlineto{\pgfqpoint{3.596074in}{1.858345in}}%
\pgfpathlineto{\pgfqpoint{3.602387in}{1.837318in}}%
\pgfpathclose%
\pgfusepath{stroke,fill}%
\end{pgfscope}%
\begin{pgfscope}%
\pgfpathrectangle{\pgfqpoint{0.887500in}{0.275000in}}{\pgfqpoint{4.225000in}{4.225000in}}%
\pgfusepath{clip}%
\pgfsetbuttcap%
\pgfsetroundjoin%
\definecolor{currentfill}{rgb}{0.250425,0.274290,0.533103}%
\pgfsetfillcolor{currentfill}%
\pgfsetfillopacity{0.700000}%
\pgfsetlinewidth{0.501875pt}%
\definecolor{currentstroke}{rgb}{1.000000,1.000000,1.000000}%
\pgfsetstrokecolor{currentstroke}%
\pgfsetstrokeopacity{0.500000}%
\pgfsetdash{}{0pt}%
\pgfpathmoveto{\pgfqpoint{3.436808in}{1.664875in}}%
\pgfpathlineto{\pgfqpoint{3.448228in}{1.669800in}}%
\pgfpathlineto{\pgfqpoint{3.459641in}{1.674433in}}%
\pgfpathlineto{\pgfqpoint{3.471043in}{1.678641in}}%
\pgfpathlineto{\pgfqpoint{3.482436in}{1.682426in}}%
\pgfpathlineto{\pgfqpoint{3.493824in}{1.686240in}}%
\pgfpathlineto{\pgfqpoint{3.487473in}{1.703679in}}%
\pgfpathlineto{\pgfqpoint{3.481127in}{1.721567in}}%
\pgfpathlineto{\pgfqpoint{3.474787in}{1.739728in}}%
\pgfpathlineto{\pgfqpoint{3.468448in}{1.757986in}}%
\pgfpathlineto{\pgfqpoint{3.462110in}{1.776166in}}%
\pgfpathlineto{\pgfqpoint{3.450657in}{1.766180in}}%
\pgfpathlineto{\pgfqpoint{3.439207in}{1.756690in}}%
\pgfpathlineto{\pgfqpoint{3.427759in}{1.747592in}}%
\pgfpathlineto{\pgfqpoint{3.416313in}{1.738908in}}%
\pgfpathlineto{\pgfqpoint{3.404868in}{1.730701in}}%
\pgfpathlineto{\pgfqpoint{3.411253in}{1.717903in}}%
\pgfpathlineto{\pgfqpoint{3.417639in}{1.704866in}}%
\pgfpathlineto{\pgfqpoint{3.424027in}{1.691646in}}%
\pgfpathlineto{\pgfqpoint{3.430416in}{1.678297in}}%
\pgfpathclose%
\pgfusepath{stroke,fill}%
\end{pgfscope}%
\begin{pgfscope}%
\pgfpathrectangle{\pgfqpoint{0.887500in}{0.275000in}}{\pgfqpoint{4.225000in}{4.225000in}}%
\pgfusepath{clip}%
\pgfsetbuttcap%
\pgfsetroundjoin%
\definecolor{currentfill}{rgb}{0.210503,0.363727,0.552206}%
\pgfsetfillcolor{currentfill}%
\pgfsetfillopacity{0.700000}%
\pgfsetlinewidth{0.501875pt}%
\definecolor{currentstroke}{rgb}{1.000000,1.000000,1.000000}%
\pgfsetstrokecolor{currentstroke}%
\pgfsetstrokeopacity{0.500000}%
\pgfsetdash{}{0pt}%
\pgfpathmoveto{\pgfqpoint{4.080679in}{1.823708in}}%
\pgfpathlineto{\pgfqpoint{4.092036in}{1.831471in}}%
\pgfpathlineto{\pgfqpoint{4.103382in}{1.838901in}}%
\pgfpathlineto{\pgfqpoint{4.114713in}{1.845954in}}%
\pgfpathlineto{\pgfqpoint{4.126030in}{1.852581in}}%
\pgfpathlineto{\pgfqpoint{4.137330in}{1.858735in}}%
\pgfpathlineto{\pgfqpoint{4.130775in}{1.871501in}}%
\pgfpathlineto{\pgfqpoint{4.124225in}{1.884322in}}%
\pgfpathlineto{\pgfqpoint{4.117683in}{1.897315in}}%
\pgfpathlineto{\pgfqpoint{4.111152in}{1.910599in}}%
\pgfpathlineto{\pgfqpoint{4.104634in}{1.924291in}}%
\pgfpathlineto{\pgfqpoint{4.093367in}{1.919123in}}%
\pgfpathlineto{\pgfqpoint{4.082089in}{1.913741in}}%
\pgfpathlineto{\pgfqpoint{4.070802in}{1.908139in}}%
\pgfpathlineto{\pgfqpoint{4.059504in}{1.902312in}}%
\pgfpathlineto{\pgfqpoint{4.048196in}{1.896256in}}%
\pgfpathlineto{\pgfqpoint{4.054678in}{1.881395in}}%
\pgfpathlineto{\pgfqpoint{4.061170in}{1.866805in}}%
\pgfpathlineto{\pgfqpoint{4.067669in}{1.852391in}}%
\pgfpathlineto{\pgfqpoint{4.074173in}{1.838057in}}%
\pgfpathclose%
\pgfusepath{stroke,fill}%
\end{pgfscope}%
\begin{pgfscope}%
\pgfpathrectangle{\pgfqpoint{0.887500in}{0.275000in}}{\pgfqpoint{4.225000in}{4.225000in}}%
\pgfusepath{clip}%
\pgfsetbuttcap%
\pgfsetroundjoin%
\definecolor{currentfill}{rgb}{0.214298,0.355619,0.551184}%
\pgfsetfillcolor{currentfill}%
\pgfsetfillopacity{0.700000}%
\pgfsetlinewidth{0.501875pt}%
\definecolor{currentstroke}{rgb}{1.000000,1.000000,1.000000}%
\pgfsetstrokecolor{currentstroke}%
\pgfsetstrokeopacity{0.500000}%
\pgfsetdash{}{0pt}%
\pgfpathmoveto{\pgfqpoint{2.933861in}{1.828313in}}%
\pgfpathlineto{\pgfqpoint{2.945398in}{1.831906in}}%
\pgfpathlineto{\pgfqpoint{2.956930in}{1.835492in}}%
\pgfpathlineto{\pgfqpoint{2.968455in}{1.839101in}}%
\pgfpathlineto{\pgfqpoint{2.979975in}{1.842763in}}%
\pgfpathlineto{\pgfqpoint{2.991489in}{1.846505in}}%
\pgfpathlineto{\pgfqpoint{2.985205in}{1.857080in}}%
\pgfpathlineto{\pgfqpoint{2.978925in}{1.867605in}}%
\pgfpathlineto{\pgfqpoint{2.972648in}{1.878081in}}%
\pgfpathlineto{\pgfqpoint{2.966375in}{1.888509in}}%
\pgfpathlineto{\pgfqpoint{2.960106in}{1.898890in}}%
\pgfpathlineto{\pgfqpoint{2.948601in}{1.895089in}}%
\pgfpathlineto{\pgfqpoint{2.937091in}{1.891446in}}%
\pgfpathlineto{\pgfqpoint{2.925574in}{1.887913in}}%
\pgfpathlineto{\pgfqpoint{2.914051in}{1.884441in}}%
\pgfpathlineto{\pgfqpoint{2.902522in}{1.880983in}}%
\pgfpathlineto{\pgfqpoint{2.908782in}{1.870548in}}%
\pgfpathlineto{\pgfqpoint{2.915046in}{1.860064in}}%
\pgfpathlineto{\pgfqpoint{2.921314in}{1.849532in}}%
\pgfpathlineto{\pgfqpoint{2.927586in}{1.838949in}}%
\pgfpathclose%
\pgfusepath{stroke,fill}%
\end{pgfscope}%
\begin{pgfscope}%
\pgfpathrectangle{\pgfqpoint{0.887500in}{0.275000in}}{\pgfqpoint{4.225000in}{4.225000in}}%
\pgfusepath{clip}%
\pgfsetbuttcap%
\pgfsetroundjoin%
\definecolor{currentfill}{rgb}{0.169646,0.456262,0.558030}%
\pgfsetfillcolor{currentfill}%
\pgfsetfillopacity{0.700000}%
\pgfsetlinewidth{0.501875pt}%
\definecolor{currentstroke}{rgb}{1.000000,1.000000,1.000000}%
\pgfsetstrokecolor{currentstroke}%
\pgfsetstrokeopacity{0.500000}%
\pgfsetdash{}{0pt}%
\pgfpathmoveto{\pgfqpoint{2.105823in}{2.041146in}}%
\pgfpathlineto{\pgfqpoint{2.117562in}{2.044870in}}%
\pgfpathlineto{\pgfqpoint{2.129297in}{2.048578in}}%
\pgfpathlineto{\pgfqpoint{2.141025in}{2.052272in}}%
\pgfpathlineto{\pgfqpoint{2.152749in}{2.055952in}}%
\pgfpathlineto{\pgfqpoint{2.164467in}{2.059621in}}%
\pgfpathlineto{\pgfqpoint{2.158447in}{2.069005in}}%
\pgfpathlineto{\pgfqpoint{2.152432in}{2.078354in}}%
\pgfpathlineto{\pgfqpoint{2.146421in}{2.087670in}}%
\pgfpathlineto{\pgfqpoint{2.140415in}{2.096954in}}%
\pgfpathlineto{\pgfqpoint{2.134413in}{2.106206in}}%
\pgfpathlineto{\pgfqpoint{2.122706in}{2.102595in}}%
\pgfpathlineto{\pgfqpoint{2.110993in}{2.098972in}}%
\pgfpathlineto{\pgfqpoint{2.099274in}{2.095336in}}%
\pgfpathlineto{\pgfqpoint{2.087550in}{2.091686in}}%
\pgfpathlineto{\pgfqpoint{2.075821in}{2.088021in}}%
\pgfpathlineto{\pgfqpoint{2.081812in}{2.078714in}}%
\pgfpathlineto{\pgfqpoint{2.087808in}{2.069373in}}%
\pgfpathlineto{\pgfqpoint{2.093808in}{2.059999in}}%
\pgfpathlineto{\pgfqpoint{2.099813in}{2.050590in}}%
\pgfpathclose%
\pgfusepath{stroke,fill}%
\end{pgfscope}%
\begin{pgfscope}%
\pgfpathrectangle{\pgfqpoint{0.887500in}{0.275000in}}{\pgfqpoint{4.225000in}{4.225000in}}%
\pgfusepath{clip}%
\pgfsetbuttcap%
\pgfsetroundjoin%
\definecolor{currentfill}{rgb}{0.201239,0.383670,0.554294}%
\pgfsetfillcolor{currentfill}%
\pgfsetfillopacity{0.700000}%
\pgfsetlinewidth{0.501875pt}%
\definecolor{currentstroke}{rgb}{1.000000,1.000000,1.000000}%
\pgfsetstrokecolor{currentstroke}%
\pgfsetstrokeopacity{0.500000}%
\pgfsetdash{}{0pt}%
\pgfpathmoveto{\pgfqpoint{3.991491in}{1.861705in}}%
\pgfpathlineto{\pgfqpoint{4.002857in}{1.869349in}}%
\pgfpathlineto{\pgfqpoint{4.014209in}{1.876567in}}%
\pgfpathlineto{\pgfqpoint{4.025549in}{1.883419in}}%
\pgfpathlineto{\pgfqpoint{4.036878in}{1.889963in}}%
\pgfpathlineto{\pgfqpoint{4.048196in}{1.896256in}}%
\pgfpathlineto{\pgfqpoint{4.041725in}{1.911482in}}%
\pgfpathlineto{\pgfqpoint{4.035267in}{1.927136in}}%
\pgfpathlineto{\pgfqpoint{4.028821in}{1.943162in}}%
\pgfpathlineto{\pgfqpoint{4.022384in}{1.959481in}}%
\pgfpathlineto{\pgfqpoint{4.015954in}{1.976013in}}%
\pgfpathlineto{\pgfqpoint{4.004658in}{1.970532in}}%
\pgfpathlineto{\pgfqpoint{3.993352in}{1.964811in}}%
\pgfpathlineto{\pgfqpoint{3.982034in}{1.958799in}}%
\pgfpathlineto{\pgfqpoint{3.970704in}{1.952444in}}%
\pgfpathlineto{\pgfqpoint{3.959362in}{1.945694in}}%
\pgfpathlineto{\pgfqpoint{3.965778in}{1.928653in}}%
\pgfpathlineto{\pgfqpoint{3.972198in}{1.911682in}}%
\pgfpathlineto{\pgfqpoint{3.978623in}{1.894834in}}%
\pgfpathlineto{\pgfqpoint{3.985054in}{1.878162in}}%
\pgfpathclose%
\pgfusepath{stroke,fill}%
\end{pgfscope}%
\begin{pgfscope}%
\pgfpathrectangle{\pgfqpoint{0.887500in}{0.275000in}}{\pgfqpoint{4.225000in}{4.225000in}}%
\pgfusepath{clip}%
\pgfsetbuttcap%
\pgfsetroundjoin%
\definecolor{currentfill}{rgb}{0.231674,0.318106,0.544834}%
\pgfsetfillcolor{currentfill}%
\pgfsetfillopacity{0.700000}%
\pgfsetlinewidth{0.501875pt}%
\definecolor{currentstroke}{rgb}{1.000000,1.000000,1.000000}%
\pgfsetstrokecolor{currentstroke}%
\pgfsetstrokeopacity{0.500000}%
\pgfsetdash{}{0pt}%
\pgfpathmoveto{\pgfqpoint{4.259000in}{1.743804in}}%
\pgfpathlineto{\pgfqpoint{4.270262in}{1.749272in}}%
\pgfpathlineto{\pgfqpoint{4.281506in}{1.754314in}}%
\pgfpathlineto{\pgfqpoint{4.292734in}{1.758984in}}%
\pgfpathlineto{\pgfqpoint{4.303946in}{1.763332in}}%
\pgfpathlineto{\pgfqpoint{4.315144in}{1.767399in}}%
\pgfpathlineto{\pgfqpoint{4.308603in}{1.781848in}}%
\pgfpathlineto{\pgfqpoint{4.302053in}{1.795967in}}%
\pgfpathlineto{\pgfqpoint{4.295497in}{1.809801in}}%
\pgfpathlineto{\pgfqpoint{4.288937in}{1.823395in}}%
\pgfpathlineto{\pgfqpoint{4.282373in}{1.836791in}}%
\pgfpathlineto{\pgfqpoint{4.271188in}{1.833040in}}%
\pgfpathlineto{\pgfqpoint{4.259999in}{1.829337in}}%
\pgfpathlineto{\pgfqpoint{4.248806in}{1.825692in}}%
\pgfpathlineto{\pgfqpoint{4.237606in}{1.822010in}}%
\pgfpathlineto{\pgfqpoint{4.226396in}{1.818157in}}%
\pgfpathlineto{\pgfqpoint{4.232947in}{1.804438in}}%
\pgfpathlineto{\pgfqpoint{4.239486in}{1.790230in}}%
\pgfpathlineto{\pgfqpoint{4.246009in}{1.775446in}}%
\pgfpathlineto{\pgfqpoint{4.252515in}{1.759999in}}%
\pgfpathclose%
\pgfusepath{stroke,fill}%
\end{pgfscope}%
\begin{pgfscope}%
\pgfpathrectangle{\pgfqpoint{0.887500in}{0.275000in}}{\pgfqpoint{4.225000in}{4.225000in}}%
\pgfusepath{clip}%
\pgfsetbuttcap%
\pgfsetroundjoin%
\definecolor{currentfill}{rgb}{0.156270,0.489624,0.557936}%
\pgfsetfillcolor{currentfill}%
\pgfsetfillopacity{0.700000}%
\pgfsetlinewidth{0.501875pt}%
\definecolor{currentstroke}{rgb}{1.000000,1.000000,1.000000}%
\pgfsetstrokecolor{currentstroke}%
\pgfsetstrokeopacity{0.500000}%
\pgfsetdash{}{0pt}%
\pgfpathmoveto{\pgfqpoint{1.780800in}{2.107159in}}%
\pgfpathlineto{\pgfqpoint{1.792621in}{2.110845in}}%
\pgfpathlineto{\pgfqpoint{1.804437in}{2.114518in}}%
\pgfpathlineto{\pgfqpoint{1.816247in}{2.118177in}}%
\pgfpathlineto{\pgfqpoint{1.828051in}{2.121826in}}%
\pgfpathlineto{\pgfqpoint{1.839850in}{2.125463in}}%
\pgfpathlineto{\pgfqpoint{1.833940in}{2.134605in}}%
\pgfpathlineto{\pgfqpoint{1.828035in}{2.143718in}}%
\pgfpathlineto{\pgfqpoint{1.822134in}{2.152800in}}%
\pgfpathlineto{\pgfqpoint{1.816237in}{2.161853in}}%
\pgfpathlineto{\pgfqpoint{1.810346in}{2.170877in}}%
\pgfpathlineto{\pgfqpoint{1.798557in}{2.167291in}}%
\pgfpathlineto{\pgfqpoint{1.786764in}{2.163694in}}%
\pgfpathlineto{\pgfqpoint{1.774964in}{2.160086in}}%
\pgfpathlineto{\pgfqpoint{1.763160in}{2.156465in}}%
\pgfpathlineto{\pgfqpoint{1.751350in}{2.152830in}}%
\pgfpathlineto{\pgfqpoint{1.757231in}{2.143755in}}%
\pgfpathlineto{\pgfqpoint{1.763116in}{2.134652in}}%
\pgfpathlineto{\pgfqpoint{1.769006in}{2.125518in}}%
\pgfpathlineto{\pgfqpoint{1.774901in}{2.116354in}}%
\pgfpathclose%
\pgfusepath{stroke,fill}%
\end{pgfscope}%
\begin{pgfscope}%
\pgfpathrectangle{\pgfqpoint{0.887500in}{0.275000in}}{\pgfqpoint{4.225000in}{4.225000in}}%
\pgfusepath{clip}%
\pgfsetbuttcap%
\pgfsetroundjoin%
\definecolor{currentfill}{rgb}{0.220057,0.343307,0.549413}%
\pgfsetfillcolor{currentfill}%
\pgfsetfillopacity{0.700000}%
\pgfsetlinewidth{0.501875pt}%
\definecolor{currentstroke}{rgb}{1.000000,1.000000,1.000000}%
\pgfsetstrokecolor{currentstroke}%
\pgfsetstrokeopacity{0.500000}%
\pgfsetdash{}{0pt}%
\pgfpathmoveto{\pgfqpoint{4.170057in}{1.791610in}}%
\pgfpathlineto{\pgfqpoint{4.181375in}{1.798319in}}%
\pgfpathlineto{\pgfqpoint{4.192665in}{1.804214in}}%
\pgfpathlineto{\pgfqpoint{4.203929in}{1.809394in}}%
\pgfpathlineto{\pgfqpoint{4.215172in}{1.813997in}}%
\pgfpathlineto{\pgfqpoint{4.226396in}{1.818157in}}%
\pgfpathlineto{\pgfqpoint{4.219837in}{1.831473in}}%
\pgfpathlineto{\pgfqpoint{4.213271in}{1.844473in}}%
\pgfpathlineto{\pgfqpoint{4.206701in}{1.857243in}}%
\pgfpathlineto{\pgfqpoint{4.200130in}{1.869870in}}%
\pgfpathlineto{\pgfqpoint{4.193561in}{1.882441in}}%
\pgfpathlineto{\pgfqpoint{4.182345in}{1.878398in}}%
\pgfpathlineto{\pgfqpoint{4.171116in}{1.874109in}}%
\pgfpathlineto{\pgfqpoint{4.159872in}{1.869468in}}%
\pgfpathlineto{\pgfqpoint{4.148611in}{1.864371in}}%
\pgfpathlineto{\pgfqpoint{4.137330in}{1.858735in}}%
\pgfpathlineto{\pgfqpoint{4.143886in}{1.845907in}}%
\pgfpathlineto{\pgfqpoint{4.150440in}{1.832897in}}%
\pgfpathlineto{\pgfqpoint{4.156989in}{1.819590in}}%
\pgfpathlineto{\pgfqpoint{4.163530in}{1.805866in}}%
\pgfpathclose%
\pgfusepath{stroke,fill}%
\end{pgfscope}%
\begin{pgfscope}%
\pgfpathrectangle{\pgfqpoint{0.887500in}{0.275000in}}{\pgfqpoint{4.225000in}{4.225000in}}%
\pgfusepath{clip}%
\pgfsetbuttcap%
\pgfsetroundjoin%
\definecolor{currentfill}{rgb}{0.188923,0.410910,0.556326}%
\pgfsetfillcolor{currentfill}%
\pgfsetfillopacity{0.700000}%
\pgfsetlinewidth{0.501875pt}%
\definecolor{currentstroke}{rgb}{1.000000,1.000000,1.000000}%
\pgfsetstrokecolor{currentstroke}%
\pgfsetstrokeopacity{0.500000}%
\pgfsetdash{}{0pt}%
\pgfpathmoveto{\pgfqpoint{2.519770in}{1.940959in}}%
\pgfpathlineto{\pgfqpoint{2.531410in}{1.944723in}}%
\pgfpathlineto{\pgfqpoint{2.543044in}{1.948466in}}%
\pgfpathlineto{\pgfqpoint{2.554673in}{1.952190in}}%
\pgfpathlineto{\pgfqpoint{2.566297in}{1.955892in}}%
\pgfpathlineto{\pgfqpoint{2.577915in}{1.959577in}}%
\pgfpathlineto{\pgfqpoint{2.571756in}{1.969502in}}%
\pgfpathlineto{\pgfqpoint{2.565603in}{1.979380in}}%
\pgfpathlineto{\pgfqpoint{2.559453in}{1.989211in}}%
\pgfpathlineto{\pgfqpoint{2.553308in}{1.998996in}}%
\pgfpathlineto{\pgfqpoint{2.547167in}{2.008734in}}%
\pgfpathlineto{\pgfqpoint{2.535558in}{2.005106in}}%
\pgfpathlineto{\pgfqpoint{2.523945in}{2.001460in}}%
\pgfpathlineto{\pgfqpoint{2.512326in}{1.997792in}}%
\pgfpathlineto{\pgfqpoint{2.500701in}{1.994103in}}%
\pgfpathlineto{\pgfqpoint{2.489071in}{1.990393in}}%
\pgfpathlineto{\pgfqpoint{2.495202in}{1.980602in}}%
\pgfpathlineto{\pgfqpoint{2.501337in}{1.970762in}}%
\pgfpathlineto{\pgfqpoint{2.507477in}{1.960875in}}%
\pgfpathlineto{\pgfqpoint{2.513621in}{1.950941in}}%
\pgfpathclose%
\pgfusepath{stroke,fill}%
\end{pgfscope}%
\begin{pgfscope}%
\pgfpathrectangle{\pgfqpoint{0.887500in}{0.275000in}}{\pgfqpoint{4.225000in}{4.225000in}}%
\pgfusepath{clip}%
\pgfsetbuttcap%
\pgfsetroundjoin%
\definecolor{currentfill}{rgb}{0.188923,0.410910,0.556326}%
\pgfsetfillcolor{currentfill}%
\pgfsetfillopacity{0.700000}%
\pgfsetlinewidth{0.501875pt}%
\definecolor{currentstroke}{rgb}{1.000000,1.000000,1.000000}%
\pgfsetstrokecolor{currentstroke}%
\pgfsetstrokeopacity{0.500000}%
\pgfsetdash{}{0pt}%
\pgfpathmoveto{\pgfqpoint{3.902425in}{1.904270in}}%
\pgfpathlineto{\pgfqpoint{3.913843in}{1.913699in}}%
\pgfpathlineto{\pgfqpoint{3.925246in}{1.922541in}}%
\pgfpathlineto{\pgfqpoint{3.936633in}{1.930795in}}%
\pgfpathlineto{\pgfqpoint{3.948005in}{1.938495in}}%
\pgfpathlineto{\pgfqpoint{3.959362in}{1.945694in}}%
\pgfpathlineto{\pgfqpoint{3.952947in}{1.962750in}}%
\pgfpathlineto{\pgfqpoint{3.946533in}{1.979769in}}%
\pgfpathlineto{\pgfqpoint{3.940119in}{1.996696in}}%
\pgfpathlineto{\pgfqpoint{3.933703in}{2.013477in}}%
\pgfpathlineto{\pgfqpoint{3.927284in}{2.030060in}}%
\pgfpathlineto{\pgfqpoint{3.915945in}{2.023569in}}%
\pgfpathlineto{\pgfqpoint{3.904596in}{2.016838in}}%
\pgfpathlineto{\pgfqpoint{3.893239in}{2.009862in}}%
\pgfpathlineto{\pgfqpoint{3.881872in}{2.002606in}}%
\pgfpathlineto{\pgfqpoint{3.870494in}{1.994974in}}%
\pgfpathlineto{\pgfqpoint{3.876882in}{1.977037in}}%
\pgfpathlineto{\pgfqpoint{3.883268in}{1.958943in}}%
\pgfpathlineto{\pgfqpoint{3.889653in}{1.940747in}}%
\pgfpathlineto{\pgfqpoint{3.896038in}{1.922504in}}%
\pgfpathclose%
\pgfusepath{stroke,fill}%
\end{pgfscope}%
\begin{pgfscope}%
\pgfpathrectangle{\pgfqpoint{0.887500in}{0.275000in}}{\pgfqpoint{4.225000in}{4.225000in}}%
\pgfusepath{clip}%
\pgfsetbuttcap%
\pgfsetroundjoin%
\definecolor{currentfill}{rgb}{0.221989,0.339161,0.548752}%
\pgfsetfillcolor{currentfill}%
\pgfsetfillopacity{0.700000}%
\pgfsetlinewidth{0.501875pt}%
\definecolor{currentstroke}{rgb}{1.000000,1.000000,1.000000}%
\pgfsetstrokecolor{currentstroke}%
\pgfsetstrokeopacity{0.500000}%
\pgfsetdash{}{0pt}%
\pgfpathmoveto{\pgfqpoint{3.022965in}{1.792840in}}%
\pgfpathlineto{\pgfqpoint{3.034482in}{1.796565in}}%
\pgfpathlineto{\pgfqpoint{3.045993in}{1.800310in}}%
\pgfpathlineto{\pgfqpoint{3.057499in}{1.804066in}}%
\pgfpathlineto{\pgfqpoint{3.068999in}{1.807827in}}%
\pgfpathlineto{\pgfqpoint{3.080494in}{1.811582in}}%
\pgfpathlineto{\pgfqpoint{3.074183in}{1.822513in}}%
\pgfpathlineto{\pgfqpoint{3.067875in}{1.833407in}}%
\pgfpathlineto{\pgfqpoint{3.061572in}{1.844262in}}%
\pgfpathlineto{\pgfqpoint{3.055271in}{1.855075in}}%
\pgfpathlineto{\pgfqpoint{3.048974in}{1.865844in}}%
\pgfpathlineto{\pgfqpoint{3.037488in}{1.862110in}}%
\pgfpathlineto{\pgfqpoint{3.025997in}{1.858234in}}%
\pgfpathlineto{\pgfqpoint{3.014500in}{1.854290in}}%
\pgfpathlineto{\pgfqpoint{3.002997in}{1.850356in}}%
\pgfpathlineto{\pgfqpoint{2.991489in}{1.846505in}}%
\pgfpathlineto{\pgfqpoint{2.997777in}{1.835880in}}%
\pgfpathlineto{\pgfqpoint{3.004068in}{1.825201in}}%
\pgfpathlineto{\pgfqpoint{3.010364in}{1.814470in}}%
\pgfpathlineto{\pgfqpoint{3.016662in}{1.803683in}}%
\pgfpathclose%
\pgfusepath{stroke,fill}%
\end{pgfscope}%
\begin{pgfscope}%
\pgfpathrectangle{\pgfqpoint{0.887500in}{0.275000in}}{\pgfqpoint{4.225000in}{4.225000in}}%
\pgfusepath{clip}%
\pgfsetbuttcap%
\pgfsetroundjoin%
\definecolor{currentfill}{rgb}{0.185556,0.418570,0.556753}%
\pgfsetfillcolor{currentfill}%
\pgfsetfillopacity{0.700000}%
\pgfsetlinewidth{0.501875pt}%
\definecolor{currentstroke}{rgb}{1.000000,1.000000,1.000000}%
\pgfsetstrokecolor{currentstroke}%
\pgfsetstrokeopacity{0.500000}%
\pgfsetdash{}{0pt}%
\pgfpathmoveto{\pgfqpoint{3.666530in}{1.906697in}}%
\pgfpathlineto{\pgfqpoint{3.678058in}{1.921890in}}%
\pgfpathlineto{\pgfqpoint{3.689580in}{1.936596in}}%
\pgfpathlineto{\pgfqpoint{3.701098in}{1.951026in}}%
\pgfpathlineto{\pgfqpoint{3.712615in}{1.965385in}}%
\pgfpathlineto{\pgfqpoint{3.724132in}{1.979681in}}%
\pgfpathlineto{\pgfqpoint{3.717776in}{1.998681in}}%
\pgfpathlineto{\pgfqpoint{3.711417in}{2.017417in}}%
\pgfpathlineto{\pgfqpoint{3.705054in}{2.035863in}}%
\pgfpathlineto{\pgfqpoint{3.698688in}{2.053989in}}%
\pgfpathlineto{\pgfqpoint{3.692316in}{2.071768in}}%
\pgfpathlineto{\pgfqpoint{3.680814in}{2.058030in}}%
\pgfpathlineto{\pgfqpoint{3.669311in}{2.044180in}}%
\pgfpathlineto{\pgfqpoint{3.657807in}{2.030270in}}%
\pgfpathlineto{\pgfqpoint{3.646300in}{2.016193in}}%
\pgfpathlineto{\pgfqpoint{3.634791in}{2.001837in}}%
\pgfpathlineto{\pgfqpoint{3.641142in}{1.983196in}}%
\pgfpathlineto{\pgfqpoint{3.647492in}{1.964314in}}%
\pgfpathlineto{\pgfqpoint{3.653839in}{1.945240in}}%
\pgfpathlineto{\pgfqpoint{3.660185in}{1.926019in}}%
\pgfpathclose%
\pgfusepath{stroke,fill}%
\end{pgfscope}%
\begin{pgfscope}%
\pgfpathrectangle{\pgfqpoint{0.887500in}{0.275000in}}{\pgfqpoint{4.225000in}{4.225000in}}%
\pgfusepath{clip}%
\pgfsetbuttcap%
\pgfsetroundjoin%
\definecolor{currentfill}{rgb}{0.225863,0.330805,0.547314}%
\pgfsetfillcolor{currentfill}%
\pgfsetfillopacity{0.700000}%
\pgfsetlinewidth{0.501875pt}%
\definecolor{currentstroke}{rgb}{1.000000,1.000000,1.000000}%
\pgfsetstrokecolor{currentstroke}%
\pgfsetstrokeopacity{0.500000}%
\pgfsetdash{}{0pt}%
\pgfpathmoveto{\pgfqpoint{3.640470in}{1.724223in}}%
\pgfpathlineto{\pgfqpoint{3.652054in}{1.743292in}}%
\pgfpathlineto{\pgfqpoint{3.663630in}{1.761555in}}%
\pgfpathlineto{\pgfqpoint{3.675190in}{1.778677in}}%
\pgfpathlineto{\pgfqpoint{3.686736in}{1.794812in}}%
\pgfpathlineto{\pgfqpoint{3.698271in}{1.810158in}}%
\pgfpathlineto{\pgfqpoint{3.691919in}{1.829325in}}%
\pgfpathlineto{\pgfqpoint{3.685569in}{1.848589in}}%
\pgfpathlineto{\pgfqpoint{3.679222in}{1.867936in}}%
\pgfpathlineto{\pgfqpoint{3.672875in}{1.887321in}}%
\pgfpathlineto{\pgfqpoint{3.666530in}{1.906697in}}%
\pgfpathlineto{\pgfqpoint{3.654992in}{1.890804in}}%
\pgfpathlineto{\pgfqpoint{3.643442in}{1.873998in}}%
\pgfpathlineto{\pgfqpoint{3.631878in}{1.856071in}}%
\pgfpathlineto{\pgfqpoint{3.620297in}{1.836851in}}%
\pgfpathlineto{\pgfqpoint{3.608706in}{1.816672in}}%
\pgfpathlineto{\pgfqpoint{3.615034in}{1.796556in}}%
\pgfpathlineto{\pgfqpoint{3.621371in}{1.777119in}}%
\pgfpathlineto{\pgfqpoint{3.627722in}{1.758512in}}%
\pgfpathlineto{\pgfqpoint{3.634088in}{1.740879in}}%
\pgfpathclose%
\pgfusepath{stroke,fill}%
\end{pgfscope}%
\begin{pgfscope}%
\pgfpathrectangle{\pgfqpoint{0.887500in}{0.275000in}}{\pgfqpoint{4.225000in}{4.225000in}}%
\pgfusepath{clip}%
\pgfsetbuttcap%
\pgfsetroundjoin%
\definecolor{currentfill}{rgb}{0.171176,0.452530,0.557965}%
\pgfsetfillcolor{currentfill}%
\pgfsetfillopacity{0.700000}%
\pgfsetlinewidth{0.501875pt}%
\definecolor{currentstroke}{rgb}{1.000000,1.000000,1.000000}%
\pgfsetstrokecolor{currentstroke}%
\pgfsetstrokeopacity{0.500000}%
\pgfsetdash{}{0pt}%
\pgfpathmoveto{\pgfqpoint{3.724132in}{1.979681in}}%
\pgfpathlineto{\pgfqpoint{3.735644in}{1.993716in}}%
\pgfpathlineto{\pgfqpoint{3.747149in}{2.007282in}}%
\pgfpathlineto{\pgfqpoint{3.758642in}{2.020169in}}%
\pgfpathlineto{\pgfqpoint{3.770119in}{2.032170in}}%
\pgfpathlineto{\pgfqpoint{3.781574in}{2.043074in}}%
\pgfpathlineto{\pgfqpoint{3.775202in}{2.061325in}}%
\pgfpathlineto{\pgfqpoint{3.768824in}{2.079245in}}%
\pgfpathlineto{\pgfqpoint{3.762441in}{2.096815in}}%
\pgfpathlineto{\pgfqpoint{3.756054in}{2.114017in}}%
\pgfpathlineto{\pgfqpoint{3.749661in}{2.130831in}}%
\pgfpathlineto{\pgfqpoint{3.738229in}{2.121070in}}%
\pgfpathlineto{\pgfqpoint{3.726774in}{2.110053in}}%
\pgfpathlineto{\pgfqpoint{3.715301in}{2.098009in}}%
\pgfpathlineto{\pgfqpoint{3.703813in}{2.085171in}}%
\pgfpathlineto{\pgfqpoint{3.692316in}{2.071768in}}%
\pgfpathlineto{\pgfqpoint{3.698688in}{2.053989in}}%
\pgfpathlineto{\pgfqpoint{3.705054in}{2.035863in}}%
\pgfpathlineto{\pgfqpoint{3.711417in}{2.017417in}}%
\pgfpathlineto{\pgfqpoint{3.717776in}{1.998681in}}%
\pgfpathclose%
\pgfusepath{stroke,fill}%
\end{pgfscope}%
\begin{pgfscope}%
\pgfpathrectangle{\pgfqpoint{0.887500in}{0.275000in}}{\pgfqpoint{4.225000in}{4.225000in}}%
\pgfusepath{clip}%
\pgfsetbuttcap%
\pgfsetroundjoin%
\definecolor{currentfill}{rgb}{0.179019,0.433756,0.557430}%
\pgfsetfillcolor{currentfill}%
\pgfsetfillopacity{0.700000}%
\pgfsetlinewidth{0.501875pt}%
\definecolor{currentstroke}{rgb}{1.000000,1.000000,1.000000}%
\pgfsetstrokecolor{currentstroke}%
\pgfsetstrokeopacity{0.500000}%
\pgfsetdash{}{0pt}%
\pgfpathmoveto{\pgfqpoint{3.813374in}{1.947512in}}%
\pgfpathlineto{\pgfqpoint{3.824833in}{1.958589in}}%
\pgfpathlineto{\pgfqpoint{3.836274in}{1.968772in}}%
\pgfpathlineto{\pgfqpoint{3.847696in}{1.978161in}}%
\pgfpathlineto{\pgfqpoint{3.859102in}{1.986860in}}%
\pgfpathlineto{\pgfqpoint{3.870494in}{1.994974in}}%
\pgfpathlineto{\pgfqpoint{3.864103in}{2.012698in}}%
\pgfpathlineto{\pgfqpoint{3.857708in}{2.030157in}}%
\pgfpathlineto{\pgfqpoint{3.851308in}{2.047336in}}%
\pgfpathlineto{\pgfqpoint{3.844905in}{2.064250in}}%
\pgfpathlineto{\pgfqpoint{3.838499in}{2.080917in}}%
\pgfpathlineto{\pgfqpoint{3.827152in}{2.074954in}}%
\pgfpathlineto{\pgfqpoint{3.815790in}{2.068410in}}%
\pgfpathlineto{\pgfqpoint{3.804409in}{2.061062in}}%
\pgfpathlineto{\pgfqpoint{3.793004in}{2.052689in}}%
\pgfpathlineto{\pgfqpoint{3.781574in}{2.043074in}}%
\pgfpathlineto{\pgfqpoint{3.787943in}{2.024509in}}%
\pgfpathlineto{\pgfqpoint{3.794306in}{2.005650in}}%
\pgfpathlineto{\pgfqpoint{3.800666in}{1.986514in}}%
\pgfpathlineto{\pgfqpoint{3.807022in}{1.967122in}}%
\pgfpathclose%
\pgfusepath{stroke,fill}%
\end{pgfscope}%
\begin{pgfscope}%
\pgfpathrectangle{\pgfqpoint{0.887500in}{0.275000in}}{\pgfqpoint{4.225000in}{4.225000in}}%
\pgfusepath{clip}%
\pgfsetbuttcap%
\pgfsetroundjoin%
\definecolor{currentfill}{rgb}{0.172719,0.448791,0.557885}%
\pgfsetfillcolor{currentfill}%
\pgfsetfillopacity{0.700000}%
\pgfsetlinewidth{0.501875pt}%
\definecolor{currentstroke}{rgb}{1.000000,1.000000,1.000000}%
\pgfsetstrokecolor{currentstroke}%
\pgfsetstrokeopacity{0.500000}%
\pgfsetdash{}{0pt}%
\pgfpathmoveto{\pgfqpoint{2.194632in}{2.012155in}}%
\pgfpathlineto{\pgfqpoint{2.206355in}{2.015874in}}%
\pgfpathlineto{\pgfqpoint{2.218072in}{2.019583in}}%
\pgfpathlineto{\pgfqpoint{2.229783in}{2.023283in}}%
\pgfpathlineto{\pgfqpoint{2.241489in}{2.026975in}}%
\pgfpathlineto{\pgfqpoint{2.253189in}{2.030661in}}%
\pgfpathlineto{\pgfqpoint{2.247136in}{2.040167in}}%
\pgfpathlineto{\pgfqpoint{2.241089in}{2.049635in}}%
\pgfpathlineto{\pgfqpoint{2.235045in}{2.059066in}}%
\pgfpathlineto{\pgfqpoint{2.229006in}{2.068461in}}%
\pgfpathlineto{\pgfqpoint{2.222972in}{2.077821in}}%
\pgfpathlineto{\pgfqpoint{2.211282in}{2.074194in}}%
\pgfpathlineto{\pgfqpoint{2.199587in}{2.070563in}}%
\pgfpathlineto{\pgfqpoint{2.187886in}{2.066925in}}%
\pgfpathlineto{\pgfqpoint{2.176179in}{2.063278in}}%
\pgfpathlineto{\pgfqpoint{2.164467in}{2.059621in}}%
\pgfpathlineto{\pgfqpoint{2.170491in}{2.050201in}}%
\pgfpathlineto{\pgfqpoint{2.176519in}{2.040746in}}%
\pgfpathlineto{\pgfqpoint{2.182552in}{2.031254in}}%
\pgfpathlineto{\pgfqpoint{2.188590in}{2.021724in}}%
\pgfpathclose%
\pgfusepath{stroke,fill}%
\end{pgfscope}%
\begin{pgfscope}%
\pgfpathrectangle{\pgfqpoint{0.887500in}{0.275000in}}{\pgfqpoint{4.225000in}{4.225000in}}%
\pgfusepath{clip}%
\pgfsetbuttcap%
\pgfsetroundjoin%
\definecolor{currentfill}{rgb}{0.229739,0.322361,0.545706}%
\pgfsetfillcolor{currentfill}%
\pgfsetfillopacity{0.700000}%
\pgfsetlinewidth{0.501875pt}%
\definecolor{currentstroke}{rgb}{1.000000,1.000000,1.000000}%
\pgfsetstrokecolor{currentstroke}%
\pgfsetstrokeopacity{0.500000}%
\pgfsetdash{}{0pt}%
\pgfpathmoveto{\pgfqpoint{3.112101in}{1.756454in}}%
\pgfpathlineto{\pgfqpoint{3.123599in}{1.760342in}}%
\pgfpathlineto{\pgfqpoint{3.135091in}{1.764337in}}%
\pgfpathlineto{\pgfqpoint{3.146577in}{1.768458in}}%
\pgfpathlineto{\pgfqpoint{3.158059in}{1.772651in}}%
\pgfpathlineto{\pgfqpoint{3.169535in}{1.776811in}}%
\pgfpathlineto{\pgfqpoint{3.163198in}{1.787719in}}%
\pgfpathlineto{\pgfqpoint{3.156864in}{1.798502in}}%
\pgfpathlineto{\pgfqpoint{3.150533in}{1.809156in}}%
\pgfpathlineto{\pgfqpoint{3.144206in}{1.819686in}}%
\pgfpathlineto{\pgfqpoint{3.137882in}{1.830100in}}%
\pgfpathlineto{\pgfqpoint{3.126415in}{1.826424in}}%
\pgfpathlineto{\pgfqpoint{3.114944in}{1.822746in}}%
\pgfpathlineto{\pgfqpoint{3.103466in}{1.819049in}}%
\pgfpathlineto{\pgfqpoint{3.091983in}{1.815326in}}%
\pgfpathlineto{\pgfqpoint{3.080494in}{1.811582in}}%
\pgfpathlineto{\pgfqpoint{3.086808in}{1.800618in}}%
\pgfpathlineto{\pgfqpoint{3.093126in}{1.789622in}}%
\pgfpathlineto{\pgfqpoint{3.099448in}{1.778598in}}%
\pgfpathlineto{\pgfqpoint{3.105773in}{1.767544in}}%
\pgfpathclose%
\pgfusepath{stroke,fill}%
\end{pgfscope}%
\begin{pgfscope}%
\pgfpathrectangle{\pgfqpoint{0.887500in}{0.275000in}}{\pgfqpoint{4.225000in}{4.225000in}}%
\pgfusepath{clip}%
\pgfsetbuttcap%
\pgfsetroundjoin%
\definecolor{currentfill}{rgb}{0.194100,0.399323,0.555565}%
\pgfsetfillcolor{currentfill}%
\pgfsetfillopacity{0.700000}%
\pgfsetlinewidth{0.501875pt}%
\definecolor{currentstroke}{rgb}{1.000000,1.000000,1.000000}%
\pgfsetstrokecolor{currentstroke}%
\pgfsetstrokeopacity{0.500000}%
\pgfsetdash{}{0pt}%
\pgfpathmoveto{\pgfqpoint{2.608769in}{1.909252in}}%
\pgfpathlineto{\pgfqpoint{2.620391in}{1.912982in}}%
\pgfpathlineto{\pgfqpoint{2.632007in}{1.916701in}}%
\pgfpathlineto{\pgfqpoint{2.643618in}{1.920412in}}%
\pgfpathlineto{\pgfqpoint{2.655223in}{1.924120in}}%
\pgfpathlineto{\pgfqpoint{2.666822in}{1.927827in}}%
\pgfpathlineto{\pgfqpoint{2.660633in}{1.937927in}}%
\pgfpathlineto{\pgfqpoint{2.654449in}{1.947980in}}%
\pgfpathlineto{\pgfqpoint{2.648268in}{1.957987in}}%
\pgfpathlineto{\pgfqpoint{2.642092in}{1.967947in}}%
\pgfpathlineto{\pgfqpoint{2.635920in}{1.977861in}}%
\pgfpathlineto{\pgfqpoint{2.624330in}{1.974211in}}%
\pgfpathlineto{\pgfqpoint{2.612735in}{1.970561in}}%
\pgfpathlineto{\pgfqpoint{2.601134in}{1.966908in}}%
\pgfpathlineto{\pgfqpoint{2.589527in}{1.963248in}}%
\pgfpathlineto{\pgfqpoint{2.577915in}{1.959577in}}%
\pgfpathlineto{\pgfqpoint{2.584077in}{1.949605in}}%
\pgfpathlineto{\pgfqpoint{2.590244in}{1.939587in}}%
\pgfpathlineto{\pgfqpoint{2.596414in}{1.929522in}}%
\pgfpathlineto{\pgfqpoint{2.602589in}{1.919410in}}%
\pgfpathclose%
\pgfusepath{stroke,fill}%
\end{pgfscope}%
\begin{pgfscope}%
\pgfpathrectangle{\pgfqpoint{0.887500in}{0.275000in}}{\pgfqpoint{4.225000in}{4.225000in}}%
\pgfusepath{clip}%
\pgfsetbuttcap%
\pgfsetroundjoin%
\definecolor{currentfill}{rgb}{0.160665,0.478540,0.558115}%
\pgfsetfillcolor{currentfill}%
\pgfsetfillopacity{0.700000}%
\pgfsetlinewidth{0.501875pt}%
\definecolor{currentstroke}{rgb}{1.000000,1.000000,1.000000}%
\pgfsetstrokecolor{currentstroke}%
\pgfsetstrokeopacity{0.500000}%
\pgfsetdash{}{0pt}%
\pgfpathmoveto{\pgfqpoint{1.869470in}{2.079294in}}%
\pgfpathlineto{\pgfqpoint{1.881274in}{2.082977in}}%
\pgfpathlineto{\pgfqpoint{1.893072in}{2.086653in}}%
\pgfpathlineto{\pgfqpoint{1.904865in}{2.090322in}}%
\pgfpathlineto{\pgfqpoint{1.916652in}{2.093985in}}%
\pgfpathlineto{\pgfqpoint{1.928433in}{2.097643in}}%
\pgfpathlineto{\pgfqpoint{1.922490in}{2.106882in}}%
\pgfpathlineto{\pgfqpoint{1.916551in}{2.116091in}}%
\pgfpathlineto{\pgfqpoint{1.910617in}{2.125269in}}%
\pgfpathlineto{\pgfqpoint{1.904687in}{2.134417in}}%
\pgfpathlineto{\pgfqpoint{1.898761in}{2.143536in}}%
\pgfpathlineto{\pgfqpoint{1.886991in}{2.139933in}}%
\pgfpathlineto{\pgfqpoint{1.875214in}{2.136326in}}%
\pgfpathlineto{\pgfqpoint{1.863432in}{2.132712in}}%
\pgfpathlineto{\pgfqpoint{1.851644in}{2.129092in}}%
\pgfpathlineto{\pgfqpoint{1.839850in}{2.125463in}}%
\pgfpathlineto{\pgfqpoint{1.845765in}{2.116291in}}%
\pgfpathlineto{\pgfqpoint{1.851684in}{2.107088in}}%
\pgfpathlineto{\pgfqpoint{1.857608in}{2.097854in}}%
\pgfpathlineto{\pgfqpoint{1.863537in}{2.088590in}}%
\pgfpathclose%
\pgfusepath{stroke,fill}%
\end{pgfscope}%
\begin{pgfscope}%
\pgfpathrectangle{\pgfqpoint{0.887500in}{0.275000in}}{\pgfqpoint{4.225000in}{4.225000in}}%
\pgfusepath{clip}%
\pgfsetbuttcap%
\pgfsetroundjoin%
\definecolor{currentfill}{rgb}{0.262138,0.242286,0.520837}%
\pgfsetfillcolor{currentfill}%
\pgfsetfillopacity{0.700000}%
\pgfsetlinewidth{0.501875pt}%
\definecolor{currentstroke}{rgb}{1.000000,1.000000,1.000000}%
\pgfsetstrokecolor{currentstroke}%
\pgfsetstrokeopacity{0.500000}%
\pgfsetdash{}{0pt}%
\pgfpathmoveto{\pgfqpoint{3.615134in}{1.591435in}}%
\pgfpathlineto{\pgfqpoint{3.626592in}{1.601260in}}%
\pgfpathlineto{\pgfqpoint{3.638064in}{1.612195in}}%
\pgfpathlineto{\pgfqpoint{3.649548in}{1.624020in}}%
\pgfpathlineto{\pgfqpoint{3.661042in}{1.636516in}}%
\pgfpathlineto{\pgfqpoint{3.672543in}{1.649462in}}%
\pgfpathlineto{\pgfqpoint{3.666115in}{1.663879in}}%
\pgfpathlineto{\pgfqpoint{3.659691in}{1.678384in}}%
\pgfpathlineto{\pgfqpoint{3.653274in}{1.693156in}}%
\pgfpathlineto{\pgfqpoint{3.646866in}{1.708376in}}%
\pgfpathlineto{\pgfqpoint{3.640470in}{1.724223in}}%
\pgfpathlineto{\pgfqpoint{3.628887in}{1.705008in}}%
\pgfpathlineto{\pgfqpoint{3.617314in}{1.686309in}}%
\pgfpathlineto{\pgfqpoint{3.605762in}{1.668790in}}%
\pgfpathlineto{\pgfqpoint{3.594237in}{1.653110in}}%
\pgfpathlineto{\pgfqpoint{3.582748in}{1.639930in}}%
\pgfpathlineto{\pgfqpoint{3.589191in}{1.628323in}}%
\pgfpathlineto{\pgfqpoint{3.595655in}{1.618011in}}%
\pgfpathlineto{\pgfqpoint{3.602137in}{1.608653in}}%
\pgfpathlineto{\pgfqpoint{3.608631in}{1.599908in}}%
\pgfpathclose%
\pgfusepath{stroke,fill}%
\end{pgfscope}%
\begin{pgfscope}%
\pgfpathrectangle{\pgfqpoint{0.887500in}{0.275000in}}{\pgfqpoint{4.225000in}{4.225000in}}%
\pgfusepath{clip}%
\pgfsetbuttcap%
\pgfsetroundjoin%
\definecolor{currentfill}{rgb}{0.237441,0.305202,0.541921}%
\pgfsetfillcolor{currentfill}%
\pgfsetfillopacity{0.700000}%
\pgfsetlinewidth{0.501875pt}%
\definecolor{currentstroke}{rgb}{1.000000,1.000000,1.000000}%
\pgfsetstrokecolor{currentstroke}%
\pgfsetstrokeopacity{0.500000}%
\pgfsetdash{}{0pt}%
\pgfpathmoveto{\pgfqpoint{3.201269in}{1.720585in}}%
\pgfpathlineto{\pgfqpoint{3.212748in}{1.724613in}}%
\pgfpathlineto{\pgfqpoint{3.224220in}{1.728435in}}%
\pgfpathlineto{\pgfqpoint{3.235685in}{1.731964in}}%
\pgfpathlineto{\pgfqpoint{3.247142in}{1.735113in}}%
\pgfpathlineto{\pgfqpoint{3.258592in}{1.737825in}}%
\pgfpathlineto{\pgfqpoint{3.252230in}{1.749040in}}%
\pgfpathlineto{\pgfqpoint{3.245872in}{1.760211in}}%
\pgfpathlineto{\pgfqpoint{3.239517in}{1.771346in}}%
\pgfpathlineto{\pgfqpoint{3.233166in}{1.782457in}}%
\pgfpathlineto{\pgfqpoint{3.226818in}{1.793552in}}%
\pgfpathlineto{\pgfqpoint{3.215376in}{1.791067in}}%
\pgfpathlineto{\pgfqpoint{3.203927in}{1.788069in}}%
\pgfpathlineto{\pgfqpoint{3.192470in}{1.784624in}}%
\pgfpathlineto{\pgfqpoint{3.181006in}{1.780837in}}%
\pgfpathlineto{\pgfqpoint{3.169535in}{1.776811in}}%
\pgfpathlineto{\pgfqpoint{3.175876in}{1.765785in}}%
\pgfpathlineto{\pgfqpoint{3.182219in}{1.754644in}}%
\pgfpathlineto{\pgfqpoint{3.188566in}{1.743393in}}%
\pgfpathlineto{\pgfqpoint{3.194916in}{1.732039in}}%
\pgfpathclose%
\pgfusepath{stroke,fill}%
\end{pgfscope}%
\begin{pgfscope}%
\pgfpathrectangle{\pgfqpoint{0.887500in}{0.275000in}}{\pgfqpoint{4.225000in}{4.225000in}}%
\pgfusepath{clip}%
\pgfsetbuttcap%
\pgfsetroundjoin%
\definecolor{currentfill}{rgb}{0.204903,0.375746,0.553533}%
\pgfsetfillcolor{currentfill}%
\pgfsetfillopacity{0.700000}%
\pgfsetlinewidth{0.501875pt}%
\definecolor{currentstroke}{rgb}{1.000000,1.000000,1.000000}%
\pgfsetstrokecolor{currentstroke}%
\pgfsetstrokeopacity{0.500000}%
\pgfsetdash{}{0pt}%
\pgfpathmoveto{\pgfqpoint{3.698271in}{1.810158in}}%
\pgfpathlineto{\pgfqpoint{3.709797in}{1.824910in}}%
\pgfpathlineto{\pgfqpoint{3.721317in}{1.839264in}}%
\pgfpathlineto{\pgfqpoint{3.732835in}{1.853419in}}%
\pgfpathlineto{\pgfqpoint{3.744352in}{1.867564in}}%
\pgfpathlineto{\pgfqpoint{3.755871in}{1.881727in}}%
\pgfpathlineto{\pgfqpoint{3.749527in}{1.901617in}}%
\pgfpathlineto{\pgfqpoint{3.743181in}{1.921387in}}%
\pgfpathlineto{\pgfqpoint{3.736834in}{1.941006in}}%
\pgfpathlineto{\pgfqpoint{3.730484in}{1.960447in}}%
\pgfpathlineto{\pgfqpoint{3.724132in}{1.979681in}}%
\pgfpathlineto{\pgfqpoint{3.712615in}{1.965385in}}%
\pgfpathlineto{\pgfqpoint{3.701098in}{1.951026in}}%
\pgfpathlineto{\pgfqpoint{3.689580in}{1.936596in}}%
\pgfpathlineto{\pgfqpoint{3.678058in}{1.921890in}}%
\pgfpathlineto{\pgfqpoint{3.666530in}{1.906697in}}%
\pgfpathlineto{\pgfqpoint{3.672875in}{1.887321in}}%
\pgfpathlineto{\pgfqpoint{3.679222in}{1.867936in}}%
\pgfpathlineto{\pgfqpoint{3.685569in}{1.848589in}}%
\pgfpathlineto{\pgfqpoint{3.691919in}{1.829325in}}%
\pgfpathclose%
\pgfusepath{stroke,fill}%
\end{pgfscope}%
\begin{pgfscope}%
\pgfpathrectangle{\pgfqpoint{0.887500in}{0.275000in}}{\pgfqpoint{4.225000in}{4.225000in}}%
\pgfusepath{clip}%
\pgfsetbuttcap%
\pgfsetroundjoin%
\definecolor{currentfill}{rgb}{0.246811,0.283237,0.535941}%
\pgfsetfillcolor{currentfill}%
\pgfsetfillopacity{0.700000}%
\pgfsetlinewidth{0.501875pt}%
\definecolor{currentstroke}{rgb}{1.000000,1.000000,1.000000}%
\pgfsetstrokecolor{currentstroke}%
\pgfsetstrokeopacity{0.500000}%
\pgfsetdash{}{0pt}%
\pgfpathmoveto{\pgfqpoint{3.290446in}{1.680734in}}%
\pgfpathlineto{\pgfqpoint{3.301899in}{1.684016in}}%
\pgfpathlineto{\pgfqpoint{3.313347in}{1.687330in}}%
\pgfpathlineto{\pgfqpoint{3.324790in}{1.690827in}}%
\pgfpathlineto{\pgfqpoint{3.336229in}{1.694653in}}%
\pgfpathlineto{\pgfqpoint{3.347667in}{1.698957in}}%
\pgfpathlineto{\pgfqpoint{3.341279in}{1.710027in}}%
\pgfpathlineto{\pgfqpoint{3.334894in}{1.720867in}}%
\pgfpathlineto{\pgfqpoint{3.328510in}{1.731491in}}%
\pgfpathlineto{\pgfqpoint{3.322129in}{1.741983in}}%
\pgfpathlineto{\pgfqpoint{3.315752in}{1.752441in}}%
\pgfpathlineto{\pgfqpoint{3.304326in}{1.748508in}}%
\pgfpathlineto{\pgfqpoint{3.292899in}{1.745340in}}%
\pgfpathlineto{\pgfqpoint{3.281468in}{1.742678in}}%
\pgfpathlineto{\pgfqpoint{3.270033in}{1.740260in}}%
\pgfpathlineto{\pgfqpoint{3.258592in}{1.737825in}}%
\pgfpathlineto{\pgfqpoint{3.264956in}{1.726555in}}%
\pgfpathlineto{\pgfqpoint{3.271324in}{1.715220in}}%
\pgfpathlineto{\pgfqpoint{3.277695in}{1.703810in}}%
\pgfpathlineto{\pgfqpoint{3.284069in}{1.692316in}}%
\pgfpathclose%
\pgfusepath{stroke,fill}%
\end{pgfscope}%
\begin{pgfscope}%
\pgfpathrectangle{\pgfqpoint{0.887500in}{0.275000in}}{\pgfqpoint{4.225000in}{4.225000in}}%
\pgfusepath{clip}%
\pgfsetbuttcap%
\pgfsetroundjoin%
\definecolor{currentfill}{rgb}{0.188923,0.410910,0.556326}%
\pgfsetfillcolor{currentfill}%
\pgfsetfillopacity{0.700000}%
\pgfsetlinewidth{0.501875pt}%
\definecolor{currentstroke}{rgb}{1.000000,1.000000,1.000000}%
\pgfsetstrokecolor{currentstroke}%
\pgfsetstrokeopacity{0.500000}%
\pgfsetdash{}{0pt}%
\pgfpathmoveto{\pgfqpoint{3.755871in}{1.881727in}}%
\pgfpathlineto{\pgfqpoint{3.767387in}{1.895760in}}%
\pgfpathlineto{\pgfqpoint{3.778899in}{1.909510in}}%
\pgfpathlineto{\pgfqpoint{3.790403in}{1.922822in}}%
\pgfpathlineto{\pgfqpoint{3.801896in}{1.935541in}}%
\pgfpathlineto{\pgfqpoint{3.813374in}{1.947512in}}%
\pgfpathlineto{\pgfqpoint{3.807022in}{1.967122in}}%
\pgfpathlineto{\pgfqpoint{3.800666in}{1.986514in}}%
\pgfpathlineto{\pgfqpoint{3.794306in}{2.005650in}}%
\pgfpathlineto{\pgfqpoint{3.787943in}{2.024509in}}%
\pgfpathlineto{\pgfqpoint{3.781574in}{2.043074in}}%
\pgfpathlineto{\pgfqpoint{3.770119in}{2.032170in}}%
\pgfpathlineto{\pgfqpoint{3.758642in}{2.020169in}}%
\pgfpathlineto{\pgfqpoint{3.747149in}{2.007282in}}%
\pgfpathlineto{\pgfqpoint{3.735644in}{1.993716in}}%
\pgfpathlineto{\pgfqpoint{3.724132in}{1.979681in}}%
\pgfpathlineto{\pgfqpoint{3.730484in}{1.960447in}}%
\pgfpathlineto{\pgfqpoint{3.736834in}{1.941006in}}%
\pgfpathlineto{\pgfqpoint{3.743181in}{1.921387in}}%
\pgfpathlineto{\pgfqpoint{3.749527in}{1.901617in}}%
\pgfpathclose%
\pgfusepath{stroke,fill}%
\end{pgfscope}%
\begin{pgfscope}%
\pgfpathrectangle{\pgfqpoint{0.887500in}{0.275000in}}{\pgfqpoint{4.225000in}{4.225000in}}%
\pgfusepath{clip}%
\pgfsetbuttcap%
\pgfsetroundjoin%
\definecolor{currentfill}{rgb}{0.179019,0.433756,0.557430}%
\pgfsetfillcolor{currentfill}%
\pgfsetfillopacity{0.700000}%
\pgfsetlinewidth{0.501875pt}%
\definecolor{currentstroke}{rgb}{1.000000,1.000000,1.000000}%
\pgfsetstrokecolor{currentstroke}%
\pgfsetstrokeopacity{0.500000}%
\pgfsetdash{}{0pt}%
\pgfpathmoveto{\pgfqpoint{2.283517in}{1.982536in}}%
\pgfpathlineto{\pgfqpoint{2.295221in}{1.986284in}}%
\pgfpathlineto{\pgfqpoint{2.306920in}{1.990031in}}%
\pgfpathlineto{\pgfqpoint{2.318613in}{1.993780in}}%
\pgfpathlineto{\pgfqpoint{2.330300in}{1.997534in}}%
\pgfpathlineto{\pgfqpoint{2.341981in}{2.001294in}}%
\pgfpathlineto{\pgfqpoint{2.335896in}{2.010936in}}%
\pgfpathlineto{\pgfqpoint{2.329816in}{2.020537in}}%
\pgfpathlineto{\pgfqpoint{2.323740in}{2.030098in}}%
\pgfpathlineto{\pgfqpoint{2.317669in}{2.039621in}}%
\pgfpathlineto{\pgfqpoint{2.311602in}{2.049104in}}%
\pgfpathlineto{\pgfqpoint{2.299931in}{2.045406in}}%
\pgfpathlineto{\pgfqpoint{2.288254in}{2.041715in}}%
\pgfpathlineto{\pgfqpoint{2.276571in}{2.038029in}}%
\pgfpathlineto{\pgfqpoint{2.264883in}{2.034345in}}%
\pgfpathlineto{\pgfqpoint{2.253189in}{2.030661in}}%
\pgfpathlineto{\pgfqpoint{2.259245in}{2.021116in}}%
\pgfpathlineto{\pgfqpoint{2.265307in}{2.011532in}}%
\pgfpathlineto{\pgfqpoint{2.271372in}{2.001907in}}%
\pgfpathlineto{\pgfqpoint{2.277442in}{1.992242in}}%
\pgfpathclose%
\pgfusepath{stroke,fill}%
\end{pgfscope}%
\begin{pgfscope}%
\pgfpathrectangle{\pgfqpoint{0.887500in}{0.275000in}}{\pgfqpoint{4.225000in}{4.225000in}}%
\pgfusepath{clip}%
\pgfsetbuttcap%
\pgfsetroundjoin%
\definecolor{currentfill}{rgb}{0.244972,0.287675,0.537260}%
\pgfsetfillcolor{currentfill}%
\pgfsetfillopacity{0.700000}%
\pgfsetlinewidth{0.501875pt}%
\definecolor{currentstroke}{rgb}{1.000000,1.000000,1.000000}%
\pgfsetstrokecolor{currentstroke}%
\pgfsetstrokeopacity{0.500000}%
\pgfsetdash{}{0pt}%
\pgfpathmoveto{\pgfqpoint{3.672543in}{1.649462in}}%
\pgfpathlineto{\pgfqpoint{3.684047in}{1.662639in}}%
\pgfpathlineto{\pgfqpoint{3.695551in}{1.675828in}}%
\pgfpathlineto{\pgfqpoint{3.707054in}{1.688913in}}%
\pgfpathlineto{\pgfqpoint{3.718555in}{1.701935in}}%
\pgfpathlineto{\pgfqpoint{3.730056in}{1.714946in}}%
\pgfpathlineto{\pgfqpoint{3.723697in}{1.733989in}}%
\pgfpathlineto{\pgfqpoint{3.717339in}{1.753006in}}%
\pgfpathlineto{\pgfqpoint{3.710981in}{1.772024in}}%
\pgfpathlineto{\pgfqpoint{3.704625in}{1.791066in}}%
\pgfpathlineto{\pgfqpoint{3.698271in}{1.810158in}}%
\pgfpathlineto{\pgfqpoint{3.686736in}{1.794812in}}%
\pgfpathlineto{\pgfqpoint{3.675190in}{1.778677in}}%
\pgfpathlineto{\pgfqpoint{3.663630in}{1.761555in}}%
\pgfpathlineto{\pgfqpoint{3.652054in}{1.743292in}}%
\pgfpathlineto{\pgfqpoint{3.640470in}{1.724223in}}%
\pgfpathlineto{\pgfqpoint{3.646866in}{1.708376in}}%
\pgfpathlineto{\pgfqpoint{3.653274in}{1.693156in}}%
\pgfpathlineto{\pgfqpoint{3.659691in}{1.678384in}}%
\pgfpathlineto{\pgfqpoint{3.666115in}{1.663879in}}%
\pgfpathclose%
\pgfusepath{stroke,fill}%
\end{pgfscope}%
\begin{pgfscope}%
\pgfpathrectangle{\pgfqpoint{0.887500in}{0.275000in}}{\pgfqpoint{4.225000in}{4.225000in}}%
\pgfusepath{clip}%
\pgfsetbuttcap%
\pgfsetroundjoin%
\definecolor{currentfill}{rgb}{0.201239,0.383670,0.554294}%
\pgfsetfillcolor{currentfill}%
\pgfsetfillopacity{0.700000}%
\pgfsetlinewidth{0.501875pt}%
\definecolor{currentstroke}{rgb}{1.000000,1.000000,1.000000}%
\pgfsetstrokecolor{currentstroke}%
\pgfsetstrokeopacity{0.500000}%
\pgfsetdash{}{0pt}%
\pgfpathmoveto{\pgfqpoint{2.697827in}{1.876626in}}%
\pgfpathlineto{\pgfqpoint{2.709430in}{1.880394in}}%
\pgfpathlineto{\pgfqpoint{2.721027in}{1.884166in}}%
\pgfpathlineto{\pgfqpoint{2.732618in}{1.887944in}}%
\pgfpathlineto{\pgfqpoint{2.744204in}{1.891730in}}%
\pgfpathlineto{\pgfqpoint{2.755784in}{1.895526in}}%
\pgfpathlineto{\pgfqpoint{2.749566in}{1.905804in}}%
\pgfpathlineto{\pgfqpoint{2.743351in}{1.916035in}}%
\pgfpathlineto{\pgfqpoint{2.737141in}{1.926218in}}%
\pgfpathlineto{\pgfqpoint{2.730935in}{1.936355in}}%
\pgfpathlineto{\pgfqpoint{2.724732in}{1.946446in}}%
\pgfpathlineto{\pgfqpoint{2.713162in}{1.942704in}}%
\pgfpathlineto{\pgfqpoint{2.701586in}{1.938973in}}%
\pgfpathlineto{\pgfqpoint{2.690003in}{1.935252in}}%
\pgfpathlineto{\pgfqpoint{2.678416in}{1.931537in}}%
\pgfpathlineto{\pgfqpoint{2.666822in}{1.927827in}}%
\pgfpathlineto{\pgfqpoint{2.673015in}{1.917681in}}%
\pgfpathlineto{\pgfqpoint{2.679212in}{1.907488in}}%
\pgfpathlineto{\pgfqpoint{2.685413in}{1.897248in}}%
\pgfpathlineto{\pgfqpoint{2.691618in}{1.886961in}}%
\pgfpathclose%
\pgfusepath{stroke,fill}%
\end{pgfscope}%
\begin{pgfscope}%
\pgfpathrectangle{\pgfqpoint{0.887500in}{0.275000in}}{\pgfqpoint{4.225000in}{4.225000in}}%
\pgfusepath{clip}%
\pgfsetbuttcap%
\pgfsetroundjoin%
\definecolor{currentfill}{rgb}{0.197636,0.391528,0.554969}%
\pgfsetfillcolor{currentfill}%
\pgfsetfillopacity{0.700000}%
\pgfsetlinewidth{0.501875pt}%
\definecolor{currentstroke}{rgb}{1.000000,1.000000,1.000000}%
\pgfsetstrokecolor{currentstroke}%
\pgfsetstrokeopacity{0.500000}%
\pgfsetdash{}{0pt}%
\pgfpathmoveto{\pgfqpoint{3.845126in}{1.848479in}}%
\pgfpathlineto{\pgfqpoint{3.856612in}{1.860772in}}%
\pgfpathlineto{\pgfqpoint{3.868085in}{1.872504in}}%
\pgfpathlineto{\pgfqpoint{3.879545in}{1.883667in}}%
\pgfpathlineto{\pgfqpoint{3.890992in}{1.894257in}}%
\pgfpathlineto{\pgfqpoint{3.902425in}{1.904270in}}%
\pgfpathlineto{\pgfqpoint{3.896038in}{1.922504in}}%
\pgfpathlineto{\pgfqpoint{3.889653in}{1.940747in}}%
\pgfpathlineto{\pgfqpoint{3.883268in}{1.958943in}}%
\pgfpathlineto{\pgfqpoint{3.876882in}{1.977037in}}%
\pgfpathlineto{\pgfqpoint{3.870494in}{1.994974in}}%
\pgfpathlineto{\pgfqpoint{3.859102in}{1.986860in}}%
\pgfpathlineto{\pgfqpoint{3.847696in}{1.978161in}}%
\pgfpathlineto{\pgfqpoint{3.836274in}{1.968772in}}%
\pgfpathlineto{\pgfqpoint{3.824833in}{1.958589in}}%
\pgfpathlineto{\pgfqpoint{3.813374in}{1.947512in}}%
\pgfpathlineto{\pgfqpoint{3.819724in}{1.927748in}}%
\pgfpathlineto{\pgfqpoint{3.826073in}{1.907898in}}%
\pgfpathlineto{\pgfqpoint{3.832422in}{1.888025in}}%
\pgfpathlineto{\pgfqpoint{3.838773in}{1.868197in}}%
\pgfpathclose%
\pgfusepath{stroke,fill}%
\end{pgfscope}%
\begin{pgfscope}%
\pgfpathrectangle{\pgfqpoint{0.887500in}{0.275000in}}{\pgfqpoint{4.225000in}{4.225000in}}%
\pgfusepath{clip}%
\pgfsetbuttcap%
\pgfsetroundjoin%
\definecolor{currentfill}{rgb}{0.206756,0.371758,0.553117}%
\pgfsetfillcolor{currentfill}%
\pgfsetfillopacity{0.700000}%
\pgfsetlinewidth{0.501875pt}%
\definecolor{currentstroke}{rgb}{1.000000,1.000000,1.000000}%
\pgfsetstrokecolor{currentstroke}%
\pgfsetstrokeopacity{0.500000}%
\pgfsetdash{}{0pt}%
\pgfpathmoveto{\pgfqpoint{3.934424in}{1.815131in}}%
\pgfpathlineto{\pgfqpoint{3.945870in}{1.825682in}}%
\pgfpathlineto{\pgfqpoint{3.957301in}{1.835609in}}%
\pgfpathlineto{\pgfqpoint{3.968714in}{1.844894in}}%
\pgfpathlineto{\pgfqpoint{3.980110in}{1.853573in}}%
\pgfpathlineto{\pgfqpoint{3.991491in}{1.861705in}}%
\pgfpathlineto{\pgfqpoint{3.985054in}{1.878162in}}%
\pgfpathlineto{\pgfqpoint{3.978623in}{1.894834in}}%
\pgfpathlineto{\pgfqpoint{3.972198in}{1.911682in}}%
\pgfpathlineto{\pgfqpoint{3.965778in}{1.928653in}}%
\pgfpathlineto{\pgfqpoint{3.959362in}{1.945694in}}%
\pgfpathlineto{\pgfqpoint{3.948005in}{1.938495in}}%
\pgfpathlineto{\pgfqpoint{3.936633in}{1.930795in}}%
\pgfpathlineto{\pgfqpoint{3.925246in}{1.922541in}}%
\pgfpathlineto{\pgfqpoint{3.913843in}{1.913699in}}%
\pgfpathlineto{\pgfqpoint{3.902425in}{1.904270in}}%
\pgfpathlineto{\pgfqpoint{3.908815in}{1.886099in}}%
\pgfpathlineto{\pgfqpoint{3.915208in}{1.868047in}}%
\pgfpathlineto{\pgfqpoint{3.921607in}{1.850167in}}%
\pgfpathlineto{\pgfqpoint{3.928012in}{1.832516in}}%
\pgfpathclose%
\pgfusepath{stroke,fill}%
\end{pgfscope}%
\begin{pgfscope}%
\pgfpathrectangle{\pgfqpoint{0.887500in}{0.275000in}}{\pgfqpoint{4.225000in}{4.225000in}}%
\pgfusepath{clip}%
\pgfsetbuttcap%
\pgfsetroundjoin%
\definecolor{currentfill}{rgb}{0.216210,0.351535,0.550627}%
\pgfsetfillcolor{currentfill}%
\pgfsetfillopacity{0.700000}%
\pgfsetlinewidth{0.501875pt}%
\definecolor{currentstroke}{rgb}{1.000000,1.000000,1.000000}%
\pgfsetstrokecolor{currentstroke}%
\pgfsetstrokeopacity{0.500000}%
\pgfsetdash{}{0pt}%
\pgfpathmoveto{\pgfqpoint{4.023744in}{1.780870in}}%
\pgfpathlineto{\pgfqpoint{4.035150in}{1.789981in}}%
\pgfpathlineto{\pgfqpoint{4.046546in}{1.798808in}}%
\pgfpathlineto{\pgfqpoint{4.057933in}{1.807363in}}%
\pgfpathlineto{\pgfqpoint{4.069311in}{1.815661in}}%
\pgfpathlineto{\pgfqpoint{4.080679in}{1.823708in}}%
\pgfpathlineto{\pgfqpoint{4.074173in}{1.838057in}}%
\pgfpathlineto{\pgfqpoint{4.067669in}{1.852391in}}%
\pgfpathlineto{\pgfqpoint{4.061170in}{1.866805in}}%
\pgfpathlineto{\pgfqpoint{4.054678in}{1.881395in}}%
\pgfpathlineto{\pgfqpoint{4.048196in}{1.896256in}}%
\pgfpathlineto{\pgfqpoint{4.036878in}{1.889963in}}%
\pgfpathlineto{\pgfqpoint{4.025549in}{1.883419in}}%
\pgfpathlineto{\pgfqpoint{4.014209in}{1.876567in}}%
\pgfpathlineto{\pgfqpoint{4.002857in}{1.869349in}}%
\pgfpathlineto{\pgfqpoint{3.991491in}{1.861705in}}%
\pgfpathlineto{\pgfqpoint{3.997935in}{1.845423in}}%
\pgfpathlineto{\pgfqpoint{4.004383in}{1.829258in}}%
\pgfpathlineto{\pgfqpoint{4.010835in}{1.813151in}}%
\pgfpathlineto{\pgfqpoint{4.017289in}{1.797041in}}%
\pgfpathclose%
\pgfusepath{stroke,fill}%
\end{pgfscope}%
\begin{pgfscope}%
\pgfpathrectangle{\pgfqpoint{0.887500in}{0.275000in}}{\pgfqpoint{4.225000in}{4.225000in}}%
\pgfusepath{clip}%
\pgfsetbuttcap%
\pgfsetroundjoin%
\definecolor{currentfill}{rgb}{0.246811,0.283237,0.535941}%
\pgfsetfillcolor{currentfill}%
\pgfsetfillopacity{0.700000}%
\pgfsetlinewidth{0.501875pt}%
\definecolor{currentstroke}{rgb}{1.000000,1.000000,1.000000}%
\pgfsetstrokecolor{currentstroke}%
\pgfsetstrokeopacity{0.500000}%
\pgfsetdash{}{0pt}%
\pgfpathmoveto{\pgfqpoint{4.291003in}{1.648595in}}%
\pgfpathlineto{\pgfqpoint{4.302407in}{1.658569in}}%
\pgfpathlineto{\pgfqpoint{4.313782in}{1.667717in}}%
\pgfpathlineto{\pgfqpoint{4.325120in}{1.675816in}}%
\pgfpathlineto{\pgfqpoint{4.336417in}{1.682754in}}%
\pgfpathlineto{\pgfqpoint{4.347677in}{1.688670in}}%
\pgfpathlineto{\pgfqpoint{4.341199in}{1.705427in}}%
\pgfpathlineto{\pgfqpoint{4.334706in}{1.721635in}}%
\pgfpathlineto{\pgfqpoint{4.328197in}{1.737337in}}%
\pgfpathlineto{\pgfqpoint{4.321676in}{1.752577in}}%
\pgfpathlineto{\pgfqpoint{4.315144in}{1.767399in}}%
\pgfpathlineto{\pgfqpoint{4.303946in}{1.763332in}}%
\pgfpathlineto{\pgfqpoint{4.292734in}{1.758984in}}%
\pgfpathlineto{\pgfqpoint{4.281506in}{1.754314in}}%
\pgfpathlineto{\pgfqpoint{4.270262in}{1.749272in}}%
\pgfpathlineto{\pgfqpoint{4.259000in}{1.743804in}}%
\pgfpathlineto{\pgfqpoint{4.265462in}{1.726774in}}%
\pgfpathlineto{\pgfqpoint{4.271896in}{1.708824in}}%
\pgfpathlineto{\pgfqpoint{4.278299in}{1.689867in}}%
\pgfpathlineto{\pgfqpoint{4.284670in}{1.669819in}}%
\pgfpathclose%
\pgfusepath{stroke,fill}%
\end{pgfscope}%
\begin{pgfscope}%
\pgfpathrectangle{\pgfqpoint{0.887500in}{0.275000in}}{\pgfqpoint{4.225000in}{4.225000in}}%
\pgfusepath{clip}%
\pgfsetbuttcap%
\pgfsetroundjoin%
\definecolor{currentfill}{rgb}{0.165117,0.467423,0.558141}%
\pgfsetfillcolor{currentfill}%
\pgfsetfillopacity{0.700000}%
\pgfsetlinewidth{0.501875pt}%
\definecolor{currentstroke}{rgb}{1.000000,1.000000,1.000000}%
\pgfsetstrokecolor{currentstroke}%
\pgfsetstrokeopacity{0.500000}%
\pgfsetdash{}{0pt}%
\pgfpathmoveto{\pgfqpoint{1.958219in}{2.050979in}}%
\pgfpathlineto{\pgfqpoint{1.970005in}{2.054693in}}%
\pgfpathlineto{\pgfqpoint{1.981785in}{2.058405in}}%
\pgfpathlineto{\pgfqpoint{1.993560in}{2.062116in}}%
\pgfpathlineto{\pgfqpoint{2.005328in}{2.065827in}}%
\pgfpathlineto{\pgfqpoint{2.017091in}{2.069539in}}%
\pgfpathlineto{\pgfqpoint{2.011115in}{2.078875in}}%
\pgfpathlineto{\pgfqpoint{2.005143in}{2.088179in}}%
\pgfpathlineto{\pgfqpoint{1.999175in}{2.097451in}}%
\pgfpathlineto{\pgfqpoint{1.993213in}{2.106692in}}%
\pgfpathlineto{\pgfqpoint{1.987254in}{2.115903in}}%
\pgfpathlineto{\pgfqpoint{1.975502in}{2.112252in}}%
\pgfpathlineto{\pgfqpoint{1.963743in}{2.108601in}}%
\pgfpathlineto{\pgfqpoint{1.951979in}{2.104950in}}%
\pgfpathlineto{\pgfqpoint{1.940209in}{2.101298in}}%
\pgfpathlineto{\pgfqpoint{1.928433in}{2.097643in}}%
\pgfpathlineto{\pgfqpoint{1.934381in}{2.088373in}}%
\pgfpathlineto{\pgfqpoint{1.940334in}{2.079072in}}%
\pgfpathlineto{\pgfqpoint{1.946291in}{2.069740in}}%
\pgfpathlineto{\pgfqpoint{1.952253in}{2.060375in}}%
\pgfpathclose%
\pgfusepath{stroke,fill}%
\end{pgfscope}%
\begin{pgfscope}%
\pgfpathrectangle{\pgfqpoint{0.887500in}{0.275000in}}{\pgfqpoint{4.225000in}{4.225000in}}%
\pgfusepath{clip}%
\pgfsetbuttcap%
\pgfsetroundjoin%
\definecolor{currentfill}{rgb}{0.263663,0.237631,0.518762}%
\pgfsetfillcolor{currentfill}%
\pgfsetfillopacity{0.700000}%
\pgfsetlinewidth{0.501875pt}%
\definecolor{currentstroke}{rgb}{1.000000,1.000000,1.000000}%
\pgfsetstrokecolor{currentstroke}%
\pgfsetstrokeopacity{0.500000}%
\pgfsetdash{}{0pt}%
\pgfpathmoveto{\pgfqpoint{3.468815in}{1.598532in}}%
\pgfpathlineto{\pgfqpoint{3.480227in}{1.602172in}}%
\pgfpathlineto{\pgfqpoint{3.491628in}{1.605340in}}%
\pgfpathlineto{\pgfqpoint{3.503016in}{1.607899in}}%
\pgfpathlineto{\pgfqpoint{3.514391in}{1.609860in}}%
\pgfpathlineto{\pgfqpoint{3.525758in}{1.611714in}}%
\pgfpathlineto{\pgfqpoint{3.519341in}{1.624461in}}%
\pgfpathlineto{\pgfqpoint{3.512940in}{1.638362in}}%
\pgfpathlineto{\pgfqpoint{3.506556in}{1.653406in}}%
\pgfpathlineto{\pgfqpoint{3.500184in}{1.669424in}}%
\pgfpathlineto{\pgfqpoint{3.493824in}{1.686240in}}%
\pgfpathlineto{\pgfqpoint{3.482436in}{1.682426in}}%
\pgfpathlineto{\pgfqpoint{3.471043in}{1.678641in}}%
\pgfpathlineto{\pgfqpoint{3.459641in}{1.674433in}}%
\pgfpathlineto{\pgfqpoint{3.448228in}{1.669800in}}%
\pgfpathlineto{\pgfqpoint{3.436808in}{1.664875in}}%
\pgfpathlineto{\pgfqpoint{3.443202in}{1.651433in}}%
\pgfpathlineto{\pgfqpoint{3.449599in}{1.638027in}}%
\pgfpathlineto{\pgfqpoint{3.456000in}{1.624712in}}%
\pgfpathlineto{\pgfqpoint{3.462405in}{1.611541in}}%
\pgfpathclose%
\pgfusepath{stroke,fill}%
\end{pgfscope}%
\begin{pgfscope}%
\pgfpathrectangle{\pgfqpoint{0.887500in}{0.275000in}}{\pgfqpoint{4.225000in}{4.225000in}}%
\pgfusepath{clip}%
\pgfsetbuttcap%
\pgfsetroundjoin%
\definecolor{currentfill}{rgb}{0.253935,0.265254,0.529983}%
\pgfsetfillcolor{currentfill}%
\pgfsetfillopacity{0.700000}%
\pgfsetlinewidth{0.501875pt}%
\definecolor{currentstroke}{rgb}{1.000000,1.000000,1.000000}%
\pgfsetstrokecolor{currentstroke}%
\pgfsetstrokeopacity{0.500000}%
\pgfsetdash{}{0pt}%
\pgfpathmoveto{\pgfqpoint{3.379634in}{1.640539in}}%
\pgfpathlineto{\pgfqpoint{3.391075in}{1.644948in}}%
\pgfpathlineto{\pgfqpoint{3.402513in}{1.649695in}}%
\pgfpathlineto{\pgfqpoint{3.413949in}{1.654688in}}%
\pgfpathlineto{\pgfqpoint{3.425381in}{1.659793in}}%
\pgfpathlineto{\pgfqpoint{3.436808in}{1.664875in}}%
\pgfpathlineto{\pgfqpoint{3.430416in}{1.678297in}}%
\pgfpathlineto{\pgfqpoint{3.424027in}{1.691646in}}%
\pgfpathlineto{\pgfqpoint{3.417639in}{1.704866in}}%
\pgfpathlineto{\pgfqpoint{3.411253in}{1.717903in}}%
\pgfpathlineto{\pgfqpoint{3.404868in}{1.730701in}}%
\pgfpathlineto{\pgfqpoint{3.393424in}{1.723033in}}%
\pgfpathlineto{\pgfqpoint{3.381982in}{1.715967in}}%
\pgfpathlineto{\pgfqpoint{3.370543in}{1.709564in}}%
\pgfpathlineto{\pgfqpoint{3.359104in}{1.703887in}}%
\pgfpathlineto{\pgfqpoint{3.347667in}{1.698957in}}%
\pgfpathlineto{\pgfqpoint{3.354057in}{1.687668in}}%
\pgfpathlineto{\pgfqpoint{3.360448in}{1.676171in}}%
\pgfpathlineto{\pgfqpoint{3.366842in}{1.664476in}}%
\pgfpathlineto{\pgfqpoint{3.373237in}{1.652596in}}%
\pgfpathclose%
\pgfusepath{stroke,fill}%
\end{pgfscope}%
\begin{pgfscope}%
\pgfpathrectangle{\pgfqpoint{0.887500in}{0.275000in}}{\pgfqpoint{4.225000in}{4.225000in}}%
\pgfusepath{clip}%
\pgfsetbuttcap%
\pgfsetroundjoin%
\definecolor{currentfill}{rgb}{0.227802,0.326594,0.546532}%
\pgfsetfillcolor{currentfill}%
\pgfsetfillopacity{0.700000}%
\pgfsetlinewidth{0.501875pt}%
\definecolor{currentstroke}{rgb}{1.000000,1.000000,1.000000}%
\pgfsetstrokecolor{currentstroke}%
\pgfsetstrokeopacity{0.500000}%
\pgfsetdash{}{0pt}%
\pgfpathmoveto{\pgfqpoint{3.730056in}{1.714946in}}%
\pgfpathlineto{\pgfqpoint{3.741558in}{1.727999in}}%
\pgfpathlineto{\pgfqpoint{3.753061in}{1.741147in}}%
\pgfpathlineto{\pgfqpoint{3.764567in}{1.754442in}}%
\pgfpathlineto{\pgfqpoint{3.776076in}{1.767934in}}%
\pgfpathlineto{\pgfqpoint{3.787589in}{1.781595in}}%
\pgfpathlineto{\pgfqpoint{3.781243in}{1.801605in}}%
\pgfpathlineto{\pgfqpoint{3.774900in}{1.821655in}}%
\pgfpathlineto{\pgfqpoint{3.768557in}{1.841714in}}%
\pgfpathlineto{\pgfqpoint{3.762214in}{1.861748in}}%
\pgfpathlineto{\pgfqpoint{3.755871in}{1.881727in}}%
\pgfpathlineto{\pgfqpoint{3.744352in}{1.867564in}}%
\pgfpathlineto{\pgfqpoint{3.732835in}{1.853419in}}%
\pgfpathlineto{\pgfqpoint{3.721317in}{1.839264in}}%
\pgfpathlineto{\pgfqpoint{3.709797in}{1.824910in}}%
\pgfpathlineto{\pgfqpoint{3.698271in}{1.810158in}}%
\pgfpathlineto{\pgfqpoint{3.704625in}{1.791066in}}%
\pgfpathlineto{\pgfqpoint{3.710981in}{1.772024in}}%
\pgfpathlineto{\pgfqpoint{3.717339in}{1.753006in}}%
\pgfpathlineto{\pgfqpoint{3.723697in}{1.733989in}}%
\pgfpathclose%
\pgfusepath{stroke,fill}%
\end{pgfscope}%
\begin{pgfscope}%
\pgfpathrectangle{\pgfqpoint{0.887500in}{0.275000in}}{\pgfqpoint{4.225000in}{4.225000in}}%
\pgfusepath{clip}%
\pgfsetbuttcap%
\pgfsetroundjoin%
\definecolor{currentfill}{rgb}{0.270595,0.214069,0.507052}%
\pgfsetfillcolor{currentfill}%
\pgfsetfillopacity{0.700000}%
\pgfsetlinewidth{0.501875pt}%
\definecolor{currentstroke}{rgb}{1.000000,1.000000,1.000000}%
\pgfsetstrokecolor{currentstroke}%
\pgfsetstrokeopacity{0.500000}%
\pgfsetdash{}{0pt}%
\pgfpathmoveto{\pgfqpoint{3.558027in}{1.558591in}}%
\pgfpathlineto{\pgfqpoint{3.569437in}{1.563633in}}%
\pgfpathlineto{\pgfqpoint{3.580848in}{1.569220in}}%
\pgfpathlineto{\pgfqpoint{3.592266in}{1.575570in}}%
\pgfpathlineto{\pgfqpoint{3.603693in}{1.582902in}}%
\pgfpathlineto{\pgfqpoint{3.615134in}{1.591435in}}%
\pgfpathlineto{\pgfqpoint{3.608631in}{1.599908in}}%
\pgfpathlineto{\pgfqpoint{3.602137in}{1.608653in}}%
\pgfpathlineto{\pgfqpoint{3.595655in}{1.618011in}}%
\pgfpathlineto{\pgfqpoint{3.589191in}{1.628323in}}%
\pgfpathlineto{\pgfqpoint{3.582748in}{1.639930in}}%
\pgfpathlineto{\pgfqpoint{3.571300in}{1.629805in}}%
\pgfpathlineto{\pgfqpoint{3.559886in}{1.622507in}}%
\pgfpathlineto{\pgfqpoint{3.548497in}{1.617451in}}%
\pgfpathlineto{\pgfqpoint{3.537124in}{1.614049in}}%
\pgfpathlineto{\pgfqpoint{3.525758in}{1.611714in}}%
\pgfpathlineto{\pgfqpoint{3.532190in}{1.599937in}}%
\pgfpathlineto{\pgfqpoint{3.538635in}{1.588932in}}%
\pgfpathlineto{\pgfqpoint{3.545091in}{1.578504in}}%
\pgfpathlineto{\pgfqpoint{3.551556in}{1.568456in}}%
\pgfpathclose%
\pgfusepath{stroke,fill}%
\end{pgfscope}%
\begin{pgfscope}%
\pgfpathrectangle{\pgfqpoint{0.887500in}{0.275000in}}{\pgfqpoint{4.225000in}{4.225000in}}%
\pgfusepath{clip}%
\pgfsetbuttcap%
\pgfsetroundjoin%
\definecolor{currentfill}{rgb}{0.210503,0.363727,0.552206}%
\pgfsetfillcolor{currentfill}%
\pgfsetfillopacity{0.700000}%
\pgfsetlinewidth{0.501875pt}%
\definecolor{currentstroke}{rgb}{1.000000,1.000000,1.000000}%
\pgfsetstrokecolor{currentstroke}%
\pgfsetstrokeopacity{0.500000}%
\pgfsetdash{}{0pt}%
\pgfpathmoveto{\pgfqpoint{3.787589in}{1.781595in}}%
\pgfpathlineto{\pgfqpoint{3.799103in}{1.795314in}}%
\pgfpathlineto{\pgfqpoint{3.810616in}{1.808977in}}%
\pgfpathlineto{\pgfqpoint{3.822126in}{1.822468in}}%
\pgfpathlineto{\pgfqpoint{3.833630in}{1.835674in}}%
\pgfpathlineto{\pgfqpoint{3.845126in}{1.848479in}}%
\pgfpathlineto{\pgfqpoint{3.838773in}{1.868197in}}%
\pgfpathlineto{\pgfqpoint{3.832422in}{1.888025in}}%
\pgfpathlineto{\pgfqpoint{3.826073in}{1.907898in}}%
\pgfpathlineto{\pgfqpoint{3.819724in}{1.927748in}}%
\pgfpathlineto{\pgfqpoint{3.813374in}{1.947512in}}%
\pgfpathlineto{\pgfqpoint{3.801896in}{1.935541in}}%
\pgfpathlineto{\pgfqpoint{3.790403in}{1.922822in}}%
\pgfpathlineto{\pgfqpoint{3.778899in}{1.909510in}}%
\pgfpathlineto{\pgfqpoint{3.767387in}{1.895760in}}%
\pgfpathlineto{\pgfqpoint{3.755871in}{1.881727in}}%
\pgfpathlineto{\pgfqpoint{3.762214in}{1.861748in}}%
\pgfpathlineto{\pgfqpoint{3.768557in}{1.841714in}}%
\pgfpathlineto{\pgfqpoint{3.774900in}{1.821655in}}%
\pgfpathlineto{\pgfqpoint{3.781243in}{1.801605in}}%
\pgfpathclose%
\pgfusepath{stroke,fill}%
\end{pgfscope}%
\begin{pgfscope}%
\pgfpathrectangle{\pgfqpoint{0.887500in}{0.275000in}}{\pgfqpoint{4.225000in}{4.225000in}}%
\pgfusepath{clip}%
\pgfsetbuttcap%
\pgfsetroundjoin%
\definecolor{currentfill}{rgb}{0.206756,0.371758,0.553117}%
\pgfsetfillcolor{currentfill}%
\pgfsetfillopacity{0.700000}%
\pgfsetlinewidth{0.501875pt}%
\definecolor{currentstroke}{rgb}{1.000000,1.000000,1.000000}%
\pgfsetstrokecolor{currentstroke}%
\pgfsetstrokeopacity{0.500000}%
\pgfsetdash{}{0pt}%
\pgfpathmoveto{\pgfqpoint{2.786936in}{1.843397in}}%
\pgfpathlineto{\pgfqpoint{2.798520in}{1.847257in}}%
\pgfpathlineto{\pgfqpoint{2.810097in}{1.851127in}}%
\pgfpathlineto{\pgfqpoint{2.821669in}{1.855003in}}%
\pgfpathlineto{\pgfqpoint{2.833236in}{1.858869in}}%
\pgfpathlineto{\pgfqpoint{2.844797in}{1.862710in}}%
\pgfpathlineto{\pgfqpoint{2.838550in}{1.873187in}}%
\pgfpathlineto{\pgfqpoint{2.832306in}{1.883613in}}%
\pgfpathlineto{\pgfqpoint{2.826066in}{1.893989in}}%
\pgfpathlineto{\pgfqpoint{2.819831in}{1.904316in}}%
\pgfpathlineto{\pgfqpoint{2.813599in}{1.914593in}}%
\pgfpathlineto{\pgfqpoint{2.802047in}{1.910803in}}%
\pgfpathlineto{\pgfqpoint{2.790490in}{1.906983in}}%
\pgfpathlineto{\pgfqpoint{2.778927in}{1.903154in}}%
\pgfpathlineto{\pgfqpoint{2.767358in}{1.899333in}}%
\pgfpathlineto{\pgfqpoint{2.755784in}{1.895526in}}%
\pgfpathlineto{\pgfqpoint{2.762007in}{1.885199in}}%
\pgfpathlineto{\pgfqpoint{2.768233in}{1.874823in}}%
\pgfpathlineto{\pgfqpoint{2.774463in}{1.864398in}}%
\pgfpathlineto{\pgfqpoint{2.780698in}{1.853923in}}%
\pgfpathclose%
\pgfusepath{stroke,fill}%
\end{pgfscope}%
\begin{pgfscope}%
\pgfpathrectangle{\pgfqpoint{0.887500in}{0.275000in}}{\pgfqpoint{4.225000in}{4.225000in}}%
\pgfusepath{clip}%
\pgfsetbuttcap%
\pgfsetroundjoin%
\definecolor{currentfill}{rgb}{0.223925,0.334994,0.548053}%
\pgfsetfillcolor{currentfill}%
\pgfsetfillopacity{0.700000}%
\pgfsetlinewidth{0.501875pt}%
\definecolor{currentstroke}{rgb}{1.000000,1.000000,1.000000}%
\pgfsetstrokecolor{currentstroke}%
\pgfsetstrokeopacity{0.500000}%
\pgfsetdash{}{0pt}%
\pgfpathmoveto{\pgfqpoint{4.113152in}{1.748421in}}%
\pgfpathlineto{\pgfqpoint{4.124565in}{1.758038in}}%
\pgfpathlineto{\pgfqpoint{4.135966in}{1.767258in}}%
\pgfpathlineto{\pgfqpoint{4.147350in}{1.775987in}}%
\pgfpathlineto{\pgfqpoint{4.158715in}{1.784135in}}%
\pgfpathlineto{\pgfqpoint{4.170057in}{1.791610in}}%
\pgfpathlineto{\pgfqpoint{4.163530in}{1.805866in}}%
\pgfpathlineto{\pgfqpoint{4.156989in}{1.819590in}}%
\pgfpathlineto{\pgfqpoint{4.150440in}{1.832897in}}%
\pgfpathlineto{\pgfqpoint{4.143886in}{1.845907in}}%
\pgfpathlineto{\pgfqpoint{4.137330in}{1.858735in}}%
\pgfpathlineto{\pgfqpoint{4.126030in}{1.852581in}}%
\pgfpathlineto{\pgfqpoint{4.114713in}{1.845954in}}%
\pgfpathlineto{\pgfqpoint{4.103382in}{1.838901in}}%
\pgfpathlineto{\pgfqpoint{4.092036in}{1.831471in}}%
\pgfpathlineto{\pgfqpoint{4.080679in}{1.823708in}}%
\pgfpathlineto{\pgfqpoint{4.087184in}{1.809250in}}%
\pgfpathlineto{\pgfqpoint{4.093687in}{1.794587in}}%
\pgfpathlineto{\pgfqpoint{4.100185in}{1.779624in}}%
\pgfpathlineto{\pgfqpoint{4.106674in}{1.764267in}}%
\pgfpathclose%
\pgfusepath{stroke,fill}%
\end{pgfscope}%
\begin{pgfscope}%
\pgfpathrectangle{\pgfqpoint{0.887500in}{0.275000in}}{\pgfqpoint{4.225000in}{4.225000in}}%
\pgfusepath{clip}%
\pgfsetbuttcap%
\pgfsetroundjoin%
\definecolor{currentfill}{rgb}{0.183898,0.422383,0.556944}%
\pgfsetfillcolor{currentfill}%
\pgfsetfillopacity{0.700000}%
\pgfsetlinewidth{0.501875pt}%
\definecolor{currentstroke}{rgb}{1.000000,1.000000,1.000000}%
\pgfsetstrokecolor{currentstroke}%
\pgfsetstrokeopacity{0.500000}%
\pgfsetdash{}{0pt}%
\pgfpathmoveto{\pgfqpoint{2.372470in}{1.952444in}}%
\pgfpathlineto{\pgfqpoint{2.384156in}{1.956272in}}%
\pgfpathlineto{\pgfqpoint{2.395835in}{1.960102in}}%
\pgfpathlineto{\pgfqpoint{2.407509in}{1.963930in}}%
\pgfpathlineto{\pgfqpoint{2.419177in}{1.967752in}}%
\pgfpathlineto{\pgfqpoint{2.430839in}{1.971565in}}%
\pgfpathlineto{\pgfqpoint{2.424723in}{1.981363in}}%
\pgfpathlineto{\pgfqpoint{2.418610in}{1.991116in}}%
\pgfpathlineto{\pgfqpoint{2.412503in}{2.000824in}}%
\pgfpathlineto{\pgfqpoint{2.406399in}{2.010488in}}%
\pgfpathlineto{\pgfqpoint{2.400300in}{2.020110in}}%
\pgfpathlineto{\pgfqpoint{2.388647in}{2.016356in}}%
\pgfpathlineto{\pgfqpoint{2.376989in}{2.012594in}}%
\pgfpathlineto{\pgfqpoint{2.365325in}{2.008827in}}%
\pgfpathlineto{\pgfqpoint{2.353656in}{2.005059in}}%
\pgfpathlineto{\pgfqpoint{2.341981in}{2.001294in}}%
\pgfpathlineto{\pgfqpoint{2.348070in}{1.991610in}}%
\pgfpathlineto{\pgfqpoint{2.354163in}{1.981884in}}%
\pgfpathlineto{\pgfqpoint{2.360261in}{1.972115in}}%
\pgfpathlineto{\pgfqpoint{2.366364in}{1.962302in}}%
\pgfpathclose%
\pgfusepath{stroke,fill}%
\end{pgfscope}%
\begin{pgfscope}%
\pgfpathrectangle{\pgfqpoint{0.887500in}{0.275000in}}{\pgfqpoint{4.225000in}{4.225000in}}%
\pgfusepath{clip}%
\pgfsetbuttcap%
\pgfsetroundjoin%
\definecolor{currentfill}{rgb}{0.233603,0.313828,0.543914}%
\pgfsetfillcolor{currentfill}%
\pgfsetfillopacity{0.700000}%
\pgfsetlinewidth{0.501875pt}%
\definecolor{currentstroke}{rgb}{1.000000,1.000000,1.000000}%
\pgfsetstrokecolor{currentstroke}%
\pgfsetstrokeopacity{0.500000}%
\pgfsetdash{}{0pt}%
\pgfpathmoveto{\pgfqpoint{4.202381in}{1.708226in}}%
\pgfpathlineto{\pgfqpoint{4.213749in}{1.716606in}}%
\pgfpathlineto{\pgfqpoint{4.225095in}{1.724312in}}%
\pgfpathlineto{\pgfqpoint{4.236418in}{1.731378in}}%
\pgfpathlineto{\pgfqpoint{4.247719in}{1.737857in}}%
\pgfpathlineto{\pgfqpoint{4.259000in}{1.743804in}}%
\pgfpathlineto{\pgfqpoint{4.252515in}{1.759999in}}%
\pgfpathlineto{\pgfqpoint{4.246009in}{1.775446in}}%
\pgfpathlineto{\pgfqpoint{4.239486in}{1.790230in}}%
\pgfpathlineto{\pgfqpoint{4.232947in}{1.804438in}}%
\pgfpathlineto{\pgfqpoint{4.226396in}{1.818157in}}%
\pgfpathlineto{\pgfqpoint{4.215172in}{1.813997in}}%
\pgfpathlineto{\pgfqpoint{4.203929in}{1.809394in}}%
\pgfpathlineto{\pgfqpoint{4.192665in}{1.804214in}}%
\pgfpathlineto{\pgfqpoint{4.181375in}{1.798319in}}%
\pgfpathlineto{\pgfqpoint{4.170057in}{1.791610in}}%
\pgfpathlineto{\pgfqpoint{4.176569in}{1.776703in}}%
\pgfpathlineto{\pgfqpoint{4.183061in}{1.761027in}}%
\pgfpathlineto{\pgfqpoint{4.189530in}{1.744467in}}%
\pgfpathlineto{\pgfqpoint{4.195971in}{1.726906in}}%
\pgfpathclose%
\pgfusepath{stroke,fill}%
\end{pgfscope}%
\begin{pgfscope}%
\pgfpathrectangle{\pgfqpoint{0.887500in}{0.275000in}}{\pgfqpoint{4.225000in}{4.225000in}}%
\pgfusepath{clip}%
\pgfsetbuttcap%
\pgfsetroundjoin%
\definecolor{currentfill}{rgb}{0.218130,0.347432,0.550038}%
\pgfsetfillcolor{currentfill}%
\pgfsetfillopacity{0.700000}%
\pgfsetlinewidth{0.501875pt}%
\definecolor{currentstroke}{rgb}{1.000000,1.000000,1.000000}%
\pgfsetstrokecolor{currentstroke}%
\pgfsetstrokeopacity{0.500000}%
\pgfsetdash{}{0pt}%
\pgfpathmoveto{\pgfqpoint{3.876989in}{1.753802in}}%
\pgfpathlineto{\pgfqpoint{3.888500in}{1.767135in}}%
\pgfpathlineto{\pgfqpoint{3.900000in}{1.779957in}}%
\pgfpathlineto{\pgfqpoint{3.911488in}{1.792246in}}%
\pgfpathlineto{\pgfqpoint{3.922963in}{1.803978in}}%
\pgfpathlineto{\pgfqpoint{3.934424in}{1.815131in}}%
\pgfpathlineto{\pgfqpoint{3.928012in}{1.832516in}}%
\pgfpathlineto{\pgfqpoint{3.921607in}{1.850167in}}%
\pgfpathlineto{\pgfqpoint{3.915208in}{1.868047in}}%
\pgfpathlineto{\pgfqpoint{3.908815in}{1.886099in}}%
\pgfpathlineto{\pgfqpoint{3.902425in}{1.904270in}}%
\pgfpathlineto{\pgfqpoint{3.890992in}{1.894257in}}%
\pgfpathlineto{\pgfqpoint{3.879545in}{1.883667in}}%
\pgfpathlineto{\pgfqpoint{3.868085in}{1.872504in}}%
\pgfpathlineto{\pgfqpoint{3.856612in}{1.860772in}}%
\pgfpathlineto{\pgfqpoint{3.845126in}{1.848479in}}%
\pgfpathlineto{\pgfqpoint{3.851484in}{1.828936in}}%
\pgfpathlineto{\pgfqpoint{3.857848in}{1.809634in}}%
\pgfpathlineto{\pgfqpoint{3.864219in}{1.790637in}}%
\pgfpathlineto{\pgfqpoint{3.870599in}{1.772011in}}%
\pgfpathclose%
\pgfusepath{stroke,fill}%
\end{pgfscope}%
\begin{pgfscope}%
\pgfpathrectangle{\pgfqpoint{0.887500in}{0.275000in}}{\pgfqpoint{4.225000in}{4.225000in}}%
\pgfusepath{clip}%
\pgfsetbuttcap%
\pgfsetroundjoin%
\definecolor{currentfill}{rgb}{0.250425,0.274290,0.533103}%
\pgfsetfillcolor{currentfill}%
\pgfsetfillopacity{0.700000}%
\pgfsetlinewidth{0.501875pt}%
\definecolor{currentstroke}{rgb}{1.000000,1.000000,1.000000}%
\pgfsetstrokecolor{currentstroke}%
\pgfsetstrokeopacity{0.500000}%
\pgfsetdash{}{0pt}%
\pgfpathmoveto{\pgfqpoint{3.761844in}{1.618513in}}%
\pgfpathlineto{\pgfqpoint{3.773325in}{1.630302in}}%
\pgfpathlineto{\pgfqpoint{3.784819in}{1.642762in}}%
\pgfpathlineto{\pgfqpoint{3.796323in}{1.655805in}}%
\pgfpathlineto{\pgfqpoint{3.807836in}{1.669342in}}%
\pgfpathlineto{\pgfqpoint{3.819358in}{1.683263in}}%
\pgfpathlineto{\pgfqpoint{3.812996in}{1.702596in}}%
\pgfpathlineto{\pgfqpoint{3.806639in}{1.722126in}}%
\pgfpathlineto{\pgfqpoint{3.800286in}{1.741823in}}%
\pgfpathlineto{\pgfqpoint{3.793936in}{1.761657in}}%
\pgfpathlineto{\pgfqpoint{3.787589in}{1.781595in}}%
\pgfpathlineto{\pgfqpoint{3.776076in}{1.767934in}}%
\pgfpathlineto{\pgfqpoint{3.764567in}{1.754442in}}%
\pgfpathlineto{\pgfqpoint{3.753061in}{1.741147in}}%
\pgfpathlineto{\pgfqpoint{3.741558in}{1.727999in}}%
\pgfpathlineto{\pgfqpoint{3.730056in}{1.714946in}}%
\pgfpathlineto{\pgfqpoint{3.736415in}{1.695855in}}%
\pgfpathlineto{\pgfqpoint{3.742774in}{1.676689in}}%
\pgfpathlineto{\pgfqpoint{3.749132in}{1.657424in}}%
\pgfpathlineto{\pgfqpoint{3.755489in}{1.638035in}}%
\pgfpathclose%
\pgfusepath{stroke,fill}%
\end{pgfscope}%
\begin{pgfscope}%
\pgfpathrectangle{\pgfqpoint{0.887500in}{0.275000in}}{\pgfqpoint{4.225000in}{4.225000in}}%
\pgfusepath{clip}%
\pgfsetbuttcap%
\pgfsetroundjoin%
\definecolor{currentfill}{rgb}{0.233603,0.313828,0.543914}%
\pgfsetfillcolor{currentfill}%
\pgfsetfillopacity{0.700000}%
\pgfsetlinewidth{0.501875pt}%
\definecolor{currentstroke}{rgb}{1.000000,1.000000,1.000000}%
\pgfsetstrokecolor{currentstroke}%
\pgfsetstrokeopacity{0.500000}%
\pgfsetdash{}{0pt}%
\pgfpathmoveto{\pgfqpoint{3.819358in}{1.683263in}}%
\pgfpathlineto{\pgfqpoint{3.830884in}{1.697430in}}%
\pgfpathlineto{\pgfqpoint{3.842413in}{1.711705in}}%
\pgfpathlineto{\pgfqpoint{3.853942in}{1.725951in}}%
\pgfpathlineto{\pgfqpoint{3.865469in}{1.740029in}}%
\pgfpathlineto{\pgfqpoint{3.876989in}{1.753802in}}%
\pgfpathlineto{\pgfqpoint{3.870599in}{1.772011in}}%
\pgfpathlineto{\pgfqpoint{3.864219in}{1.790637in}}%
\pgfpathlineto{\pgfqpoint{3.857848in}{1.809634in}}%
\pgfpathlineto{\pgfqpoint{3.851484in}{1.828936in}}%
\pgfpathlineto{\pgfqpoint{3.845126in}{1.848479in}}%
\pgfpathlineto{\pgfqpoint{3.833630in}{1.835674in}}%
\pgfpathlineto{\pgfqpoint{3.822126in}{1.822468in}}%
\pgfpathlineto{\pgfqpoint{3.810616in}{1.808977in}}%
\pgfpathlineto{\pgfqpoint{3.799103in}{1.795314in}}%
\pgfpathlineto{\pgfqpoint{3.787589in}{1.781595in}}%
\pgfpathlineto{\pgfqpoint{3.793936in}{1.761657in}}%
\pgfpathlineto{\pgfqpoint{3.800286in}{1.741823in}}%
\pgfpathlineto{\pgfqpoint{3.806639in}{1.722126in}}%
\pgfpathlineto{\pgfqpoint{3.812996in}{1.702596in}}%
\pgfpathclose%
\pgfusepath{stroke,fill}%
\end{pgfscope}%
\begin{pgfscope}%
\pgfpathrectangle{\pgfqpoint{0.887500in}{0.275000in}}{\pgfqpoint{4.225000in}{4.225000in}}%
\pgfusepath{clip}%
\pgfsetbuttcap%
\pgfsetroundjoin%
\definecolor{currentfill}{rgb}{0.169646,0.456262,0.558030}%
\pgfsetfillcolor{currentfill}%
\pgfsetfillopacity{0.700000}%
\pgfsetlinewidth{0.501875pt}%
\definecolor{currentstroke}{rgb}{1.000000,1.000000,1.000000}%
\pgfsetstrokecolor{currentstroke}%
\pgfsetstrokeopacity{0.500000}%
\pgfsetdash{}{0pt}%
\pgfpathmoveto{\pgfqpoint{2.047042in}{2.022362in}}%
\pgfpathlineto{\pgfqpoint{2.058809in}{2.026132in}}%
\pgfpathlineto{\pgfqpoint{2.070571in}{2.029898in}}%
\pgfpathlineto{\pgfqpoint{2.082327in}{2.033658in}}%
\pgfpathlineto{\pgfqpoint{2.094077in}{2.037408in}}%
\pgfpathlineto{\pgfqpoint{2.105823in}{2.041146in}}%
\pgfpathlineto{\pgfqpoint{2.099813in}{2.050590in}}%
\pgfpathlineto{\pgfqpoint{2.093808in}{2.059999in}}%
\pgfpathlineto{\pgfqpoint{2.087808in}{2.069373in}}%
\pgfpathlineto{\pgfqpoint{2.081812in}{2.078714in}}%
\pgfpathlineto{\pgfqpoint{2.075821in}{2.088021in}}%
\pgfpathlineto{\pgfqpoint{2.064086in}{2.084343in}}%
\pgfpathlineto{\pgfqpoint{2.052346in}{2.080653in}}%
\pgfpathlineto{\pgfqpoint{2.040600in}{2.076954in}}%
\pgfpathlineto{\pgfqpoint{2.028848in}{2.073249in}}%
\pgfpathlineto{\pgfqpoint{2.017091in}{2.069539in}}%
\pgfpathlineto{\pgfqpoint{2.023072in}{2.060171in}}%
\pgfpathlineto{\pgfqpoint{2.029058in}{2.050769in}}%
\pgfpathlineto{\pgfqpoint{2.035048in}{2.041334in}}%
\pgfpathlineto{\pgfqpoint{2.041042in}{2.031865in}}%
\pgfpathclose%
\pgfusepath{stroke,fill}%
\end{pgfscope}%
\begin{pgfscope}%
\pgfpathrectangle{\pgfqpoint{0.887500in}{0.275000in}}{\pgfqpoint{4.225000in}{4.225000in}}%
\pgfusepath{clip}%
\pgfsetbuttcap%
\pgfsetroundjoin%
\definecolor{currentfill}{rgb}{0.214298,0.355619,0.551184}%
\pgfsetfillcolor{currentfill}%
\pgfsetfillopacity{0.700000}%
\pgfsetlinewidth{0.501875pt}%
\definecolor{currentstroke}{rgb}{1.000000,1.000000,1.000000}%
\pgfsetstrokecolor{currentstroke}%
\pgfsetstrokeopacity{0.500000}%
\pgfsetdash{}{0pt}%
\pgfpathmoveto{\pgfqpoint{2.876093in}{1.809553in}}%
\pgfpathlineto{\pgfqpoint{2.887657in}{1.813415in}}%
\pgfpathlineto{\pgfqpoint{2.899216in}{1.817232in}}%
\pgfpathlineto{\pgfqpoint{2.910770in}{1.820992in}}%
\pgfpathlineto{\pgfqpoint{2.922318in}{1.824684in}}%
\pgfpathlineto{\pgfqpoint{2.933861in}{1.828313in}}%
\pgfpathlineto{\pgfqpoint{2.927586in}{1.838949in}}%
\pgfpathlineto{\pgfqpoint{2.921314in}{1.849532in}}%
\pgfpathlineto{\pgfqpoint{2.915046in}{1.860064in}}%
\pgfpathlineto{\pgfqpoint{2.908782in}{1.870548in}}%
\pgfpathlineto{\pgfqpoint{2.902522in}{1.880983in}}%
\pgfpathlineto{\pgfqpoint{2.890988in}{1.877489in}}%
\pgfpathlineto{\pgfqpoint{2.879448in}{1.873915in}}%
\pgfpathlineto{\pgfqpoint{2.867903in}{1.870250in}}%
\pgfpathlineto{\pgfqpoint{2.856353in}{1.866509in}}%
\pgfpathlineto{\pgfqpoint{2.844797in}{1.862710in}}%
\pgfpathlineto{\pgfqpoint{2.851049in}{1.852183in}}%
\pgfpathlineto{\pgfqpoint{2.857304in}{1.841604in}}%
\pgfpathlineto{\pgfqpoint{2.863563in}{1.830973in}}%
\pgfpathlineto{\pgfqpoint{2.869826in}{1.820290in}}%
\pgfpathclose%
\pgfusepath{stroke,fill}%
\end{pgfscope}%
\begin{pgfscope}%
\pgfpathrectangle{\pgfqpoint{0.887500in}{0.275000in}}{\pgfqpoint{4.225000in}{4.225000in}}%
\pgfusepath{clip}%
\pgfsetbuttcap%
\pgfsetroundjoin%
\definecolor{currentfill}{rgb}{0.262138,0.242286,0.520837}%
\pgfsetfillcolor{currentfill}%
\pgfsetfillopacity{0.700000}%
\pgfsetlinewidth{0.501875pt}%
\definecolor{currentstroke}{rgb}{1.000000,1.000000,1.000000}%
\pgfsetstrokecolor{currentstroke}%
\pgfsetstrokeopacity{0.500000}%
\pgfsetdash{}{0pt}%
\pgfpathmoveto{\pgfqpoint{3.704638in}{1.572456in}}%
\pgfpathlineto{\pgfqpoint{3.716052in}{1.579801in}}%
\pgfpathlineto{\pgfqpoint{3.727478in}{1.588046in}}%
\pgfpathlineto{\pgfqpoint{3.738919in}{1.597297in}}%
\pgfpathlineto{\pgfqpoint{3.750375in}{1.607482in}}%
\pgfpathlineto{\pgfqpoint{3.761844in}{1.618513in}}%
\pgfpathlineto{\pgfqpoint{3.755489in}{1.638035in}}%
\pgfpathlineto{\pgfqpoint{3.749132in}{1.657424in}}%
\pgfpathlineto{\pgfqpoint{3.742774in}{1.676689in}}%
\pgfpathlineto{\pgfqpoint{3.736415in}{1.695855in}}%
\pgfpathlineto{\pgfqpoint{3.730056in}{1.714946in}}%
\pgfpathlineto{\pgfqpoint{3.718555in}{1.701935in}}%
\pgfpathlineto{\pgfqpoint{3.707054in}{1.688913in}}%
\pgfpathlineto{\pgfqpoint{3.695551in}{1.675828in}}%
\pgfpathlineto{\pgfqpoint{3.684047in}{1.662639in}}%
\pgfpathlineto{\pgfqpoint{3.672543in}{1.649462in}}%
\pgfpathlineto{\pgfqpoint{3.678972in}{1.634953in}}%
\pgfpathlineto{\pgfqpoint{3.685399in}{1.620173in}}%
\pgfpathlineto{\pgfqpoint{3.691821in}{1.604942in}}%
\pgfpathlineto{\pgfqpoint{3.698234in}{1.589081in}}%
\pgfpathclose%
\pgfusepath{stroke,fill}%
\end{pgfscope}%
\begin{pgfscope}%
\pgfpathrectangle{\pgfqpoint{0.887500in}{0.275000in}}{\pgfqpoint{4.225000in}{4.225000in}}%
\pgfusepath{clip}%
\pgfsetbuttcap%
\pgfsetroundjoin%
\definecolor{currentfill}{rgb}{0.188923,0.410910,0.556326}%
\pgfsetfillcolor{currentfill}%
\pgfsetfillopacity{0.700000}%
\pgfsetlinewidth{0.501875pt}%
\definecolor{currentstroke}{rgb}{1.000000,1.000000,1.000000}%
\pgfsetstrokecolor{currentstroke}%
\pgfsetstrokeopacity{0.500000}%
\pgfsetdash{}{0pt}%
\pgfpathmoveto{\pgfqpoint{2.461489in}{1.921860in}}%
\pgfpathlineto{\pgfqpoint{2.473156in}{1.925715in}}%
\pgfpathlineto{\pgfqpoint{2.484818in}{1.929553in}}%
\pgfpathlineto{\pgfqpoint{2.496474in}{1.933373in}}%
\pgfpathlineto{\pgfqpoint{2.508125in}{1.937175in}}%
\pgfpathlineto{\pgfqpoint{2.519770in}{1.940959in}}%
\pgfpathlineto{\pgfqpoint{2.513621in}{1.950941in}}%
\pgfpathlineto{\pgfqpoint{2.507477in}{1.960875in}}%
\pgfpathlineto{\pgfqpoint{2.501337in}{1.970762in}}%
\pgfpathlineto{\pgfqpoint{2.495202in}{1.980602in}}%
\pgfpathlineto{\pgfqpoint{2.489071in}{1.990393in}}%
\pgfpathlineto{\pgfqpoint{2.477435in}{1.986664in}}%
\pgfpathlineto{\pgfqpoint{2.465795in}{1.982915in}}%
\pgfpathlineto{\pgfqpoint{2.454148in}{1.979149in}}%
\pgfpathlineto{\pgfqpoint{2.442497in}{1.975365in}}%
\pgfpathlineto{\pgfqpoint{2.430839in}{1.971565in}}%
\pgfpathlineto{\pgfqpoint{2.436961in}{1.961719in}}%
\pgfpathlineto{\pgfqpoint{2.443086in}{1.951826in}}%
\pgfpathlineto{\pgfqpoint{2.449216in}{1.941885in}}%
\pgfpathlineto{\pgfqpoint{2.455351in}{1.931897in}}%
\pgfpathclose%
\pgfusepath{stroke,fill}%
\end{pgfscope}%
\begin{pgfscope}%
\pgfpathrectangle{\pgfqpoint{0.887500in}{0.275000in}}{\pgfqpoint{4.225000in}{4.225000in}}%
\pgfusepath{clip}%
\pgfsetbuttcap%
\pgfsetroundjoin%
\definecolor{currentfill}{rgb}{0.225863,0.330805,0.547314}%
\pgfsetfillcolor{currentfill}%
\pgfsetfillopacity{0.700000}%
\pgfsetlinewidth{0.501875pt}%
\definecolor{currentstroke}{rgb}{1.000000,1.000000,1.000000}%
\pgfsetstrokecolor{currentstroke}%
\pgfsetstrokeopacity{0.500000}%
\pgfsetdash{}{0pt}%
\pgfpathmoveto{\pgfqpoint{3.966563in}{1.730441in}}%
\pgfpathlineto{\pgfqpoint{3.978020in}{1.741279in}}%
\pgfpathlineto{\pgfqpoint{3.989466in}{1.751687in}}%
\pgfpathlineto{\pgfqpoint{4.000902in}{1.761736in}}%
\pgfpathlineto{\pgfqpoint{4.012328in}{1.771459in}}%
\pgfpathlineto{\pgfqpoint{4.023744in}{1.780870in}}%
\pgfpathlineto{\pgfqpoint{4.017289in}{1.797041in}}%
\pgfpathlineto{\pgfqpoint{4.010835in}{1.813151in}}%
\pgfpathlineto{\pgfqpoint{4.004383in}{1.829258in}}%
\pgfpathlineto{\pgfqpoint{3.997935in}{1.845423in}}%
\pgfpathlineto{\pgfqpoint{3.991491in}{1.861705in}}%
\pgfpathlineto{\pgfqpoint{3.980110in}{1.853573in}}%
\pgfpathlineto{\pgfqpoint{3.968714in}{1.844894in}}%
\pgfpathlineto{\pgfqpoint{3.957301in}{1.835609in}}%
\pgfpathlineto{\pgfqpoint{3.945870in}{1.825682in}}%
\pgfpathlineto{\pgfqpoint{3.934424in}{1.815131in}}%
\pgfpathlineto{\pgfqpoint{3.940844in}{1.797975in}}%
\pgfpathlineto{\pgfqpoint{3.947268in}{1.780988in}}%
\pgfpathlineto{\pgfqpoint{3.953697in}{1.764110in}}%
\pgfpathlineto{\pgfqpoint{3.960129in}{1.747281in}}%
\pgfpathclose%
\pgfusepath{stroke,fill}%
\end{pgfscope}%
\begin{pgfscope}%
\pgfpathrectangle{\pgfqpoint{0.887500in}{0.275000in}}{\pgfqpoint{4.225000in}{4.225000in}}%
\pgfusepath{clip}%
\pgfsetbuttcap%
\pgfsetroundjoin%
\definecolor{currentfill}{rgb}{0.221989,0.339161,0.548752}%
\pgfsetfillcolor{currentfill}%
\pgfsetfillopacity{0.700000}%
\pgfsetlinewidth{0.501875pt}%
\definecolor{currentstroke}{rgb}{1.000000,1.000000,1.000000}%
\pgfsetstrokecolor{currentstroke}%
\pgfsetstrokeopacity{0.500000}%
\pgfsetdash{}{0pt}%
\pgfpathmoveto{\pgfqpoint{2.965294in}{1.774285in}}%
\pgfpathlineto{\pgfqpoint{2.976839in}{1.778020in}}%
\pgfpathlineto{\pgfqpoint{2.988379in}{1.781733in}}%
\pgfpathlineto{\pgfqpoint{2.999913in}{1.785434in}}%
\pgfpathlineto{\pgfqpoint{3.011442in}{1.789132in}}%
\pgfpathlineto{\pgfqpoint{3.022965in}{1.792840in}}%
\pgfpathlineto{\pgfqpoint{3.016662in}{1.803683in}}%
\pgfpathlineto{\pgfqpoint{3.010364in}{1.814470in}}%
\pgfpathlineto{\pgfqpoint{3.004068in}{1.825201in}}%
\pgfpathlineto{\pgfqpoint{2.997777in}{1.835880in}}%
\pgfpathlineto{\pgfqpoint{2.991489in}{1.846505in}}%
\pgfpathlineto{\pgfqpoint{2.979975in}{1.842763in}}%
\pgfpathlineto{\pgfqpoint{2.968455in}{1.839101in}}%
\pgfpathlineto{\pgfqpoint{2.956930in}{1.835492in}}%
\pgfpathlineto{\pgfqpoint{2.945398in}{1.831906in}}%
\pgfpathlineto{\pgfqpoint{2.933861in}{1.828313in}}%
\pgfpathlineto{\pgfqpoint{2.940140in}{1.817623in}}%
\pgfpathlineto{\pgfqpoint{2.946423in}{1.806878in}}%
\pgfpathlineto{\pgfqpoint{2.952710in}{1.796074in}}%
\pgfpathlineto{\pgfqpoint{2.959000in}{1.785210in}}%
\pgfpathclose%
\pgfusepath{stroke,fill}%
\end{pgfscope}%
\begin{pgfscope}%
\pgfpathrectangle{\pgfqpoint{0.887500in}{0.275000in}}{\pgfqpoint{4.225000in}{4.225000in}}%
\pgfusepath{clip}%
\pgfsetbuttcap%
\pgfsetroundjoin%
\definecolor{currentfill}{rgb}{0.267968,0.223549,0.512008}%
\pgfsetfillcolor{currentfill}%
\pgfsetfillopacity{0.700000}%
\pgfsetlinewidth{0.501875pt}%
\definecolor{currentstroke}{rgb}{1.000000,1.000000,1.000000}%
\pgfsetstrokecolor{currentstroke}%
\pgfsetstrokeopacity{0.500000}%
\pgfsetdash{}{0pt}%
\pgfpathmoveto{\pgfqpoint{3.793637in}{1.521227in}}%
\pgfpathlineto{\pgfqpoint{3.805123in}{1.533112in}}%
\pgfpathlineto{\pgfqpoint{3.816629in}{1.546044in}}%
\pgfpathlineto{\pgfqpoint{3.828152in}{1.559895in}}%
\pgfpathlineto{\pgfqpoint{3.839692in}{1.574532in}}%
\pgfpathlineto{\pgfqpoint{3.851245in}{1.589778in}}%
\pgfpathlineto{\pgfqpoint{3.844857in}{1.608035in}}%
\pgfpathlineto{\pgfqpoint{3.838474in}{1.626516in}}%
\pgfpathlineto{\pgfqpoint{3.832096in}{1.645218in}}%
\pgfpathlineto{\pgfqpoint{3.825724in}{1.664135in}}%
\pgfpathlineto{\pgfqpoint{3.819358in}{1.683263in}}%
\pgfpathlineto{\pgfqpoint{3.807836in}{1.669342in}}%
\pgfpathlineto{\pgfqpoint{3.796323in}{1.655805in}}%
\pgfpathlineto{\pgfqpoint{3.784819in}{1.642762in}}%
\pgfpathlineto{\pgfqpoint{3.773325in}{1.630302in}}%
\pgfpathlineto{\pgfqpoint{3.761844in}{1.618513in}}%
\pgfpathlineto{\pgfqpoint{3.768198in}{1.598912in}}%
\pgfpathlineto{\pgfqpoint{3.774553in}{1.579308in}}%
\pgfpathlineto{\pgfqpoint{3.780910in}{1.559777in}}%
\pgfpathlineto{\pgfqpoint{3.787271in}{1.540391in}}%
\pgfpathclose%
\pgfusepath{stroke,fill}%
\end{pgfscope}%
\begin{pgfscope}%
\pgfpathrectangle{\pgfqpoint{0.887500in}{0.275000in}}{\pgfqpoint{4.225000in}{4.225000in}}%
\pgfusepath{clip}%
\pgfsetbuttcap%
\pgfsetroundjoin%
\definecolor{currentfill}{rgb}{0.253935,0.265254,0.529983}%
\pgfsetfillcolor{currentfill}%
\pgfsetfillopacity{0.700000}%
\pgfsetlinewidth{0.501875pt}%
\definecolor{currentstroke}{rgb}{1.000000,1.000000,1.000000}%
\pgfsetstrokecolor{currentstroke}%
\pgfsetstrokeopacity{0.500000}%
\pgfsetdash{}{0pt}%
\pgfpathmoveto{\pgfqpoint{4.233813in}{1.594011in}}%
\pgfpathlineto{\pgfqpoint{4.245252in}{1.604847in}}%
\pgfpathlineto{\pgfqpoint{4.256696in}{1.615920in}}%
\pgfpathlineto{\pgfqpoint{4.268140in}{1.627050in}}%
\pgfpathlineto{\pgfqpoint{4.279579in}{1.638015in}}%
\pgfpathlineto{\pgfqpoint{4.291003in}{1.648595in}}%
\pgfpathlineto{\pgfqpoint{4.284670in}{1.669819in}}%
\pgfpathlineto{\pgfqpoint{4.278299in}{1.689867in}}%
\pgfpathlineto{\pgfqpoint{4.271896in}{1.708824in}}%
\pgfpathlineto{\pgfqpoint{4.265462in}{1.726774in}}%
\pgfpathlineto{\pgfqpoint{4.259000in}{1.743804in}}%
\pgfpathlineto{\pgfqpoint{4.247719in}{1.737857in}}%
\pgfpathlineto{\pgfqpoint{4.236418in}{1.731378in}}%
\pgfpathlineto{\pgfqpoint{4.225095in}{1.724312in}}%
\pgfpathlineto{\pgfqpoint{4.213749in}{1.716606in}}%
\pgfpathlineto{\pgfqpoint{4.202381in}{1.708226in}}%
\pgfpathlineto{\pgfqpoint{4.208755in}{1.688312in}}%
\pgfpathlineto{\pgfqpoint{4.215089in}{1.667049in}}%
\pgfpathlineto{\pgfqpoint{4.221380in}{1.644320in}}%
\pgfpathlineto{\pgfqpoint{4.227623in}{1.620012in}}%
\pgfpathclose%
\pgfusepath{stroke,fill}%
\end{pgfscope}%
\begin{pgfscope}%
\pgfpathrectangle{\pgfqpoint{0.887500in}{0.275000in}}{\pgfqpoint{4.225000in}{4.225000in}}%
\pgfusepath{clip}%
\pgfsetbuttcap%
\pgfsetroundjoin%
\definecolor{currentfill}{rgb}{0.174274,0.445044,0.557792}%
\pgfsetfillcolor{currentfill}%
\pgfsetfillopacity{0.700000}%
\pgfsetlinewidth{0.501875pt}%
\definecolor{currentstroke}{rgb}{1.000000,1.000000,1.000000}%
\pgfsetstrokecolor{currentstroke}%
\pgfsetstrokeopacity{0.500000}%
\pgfsetdash{}{0pt}%
\pgfpathmoveto{\pgfqpoint{2.135938in}{1.993375in}}%
\pgfpathlineto{\pgfqpoint{2.147687in}{1.997159in}}%
\pgfpathlineto{\pgfqpoint{2.159432in}{2.000928in}}%
\pgfpathlineto{\pgfqpoint{2.171171in}{2.004683in}}%
\pgfpathlineto{\pgfqpoint{2.182904in}{2.008425in}}%
\pgfpathlineto{\pgfqpoint{2.194632in}{2.012155in}}%
\pgfpathlineto{\pgfqpoint{2.188590in}{2.021724in}}%
\pgfpathlineto{\pgfqpoint{2.182552in}{2.031254in}}%
\pgfpathlineto{\pgfqpoint{2.176519in}{2.040746in}}%
\pgfpathlineto{\pgfqpoint{2.170491in}{2.050201in}}%
\pgfpathlineto{\pgfqpoint{2.164467in}{2.059621in}}%
\pgfpathlineto{\pgfqpoint{2.152749in}{2.055952in}}%
\pgfpathlineto{\pgfqpoint{2.141025in}{2.052272in}}%
\pgfpathlineto{\pgfqpoint{2.129297in}{2.048578in}}%
\pgfpathlineto{\pgfqpoint{2.117562in}{2.044870in}}%
\pgfpathlineto{\pgfqpoint{2.105823in}{2.041146in}}%
\pgfpathlineto{\pgfqpoint{2.111837in}{2.031666in}}%
\pgfpathlineto{\pgfqpoint{2.117855in}{2.022150in}}%
\pgfpathlineto{\pgfqpoint{2.123878in}{2.012597in}}%
\pgfpathlineto{\pgfqpoint{2.129906in}{2.003005in}}%
\pgfpathclose%
\pgfusepath{stroke,fill}%
\end{pgfscope}%
\begin{pgfscope}%
\pgfpathrectangle{\pgfqpoint{0.887500in}{0.275000in}}{\pgfqpoint{4.225000in}{4.225000in}}%
\pgfusepath{clip}%
\pgfsetbuttcap%
\pgfsetroundjoin%
\definecolor{currentfill}{rgb}{0.233603,0.313828,0.543914}%
\pgfsetfillcolor{currentfill}%
\pgfsetfillopacity{0.700000}%
\pgfsetlinewidth{0.501875pt}%
\definecolor{currentstroke}{rgb}{1.000000,1.000000,1.000000}%
\pgfsetstrokecolor{currentstroke}%
\pgfsetstrokeopacity{0.500000}%
\pgfsetdash{}{0pt}%
\pgfpathmoveto{\pgfqpoint{4.055969in}{1.697020in}}%
\pgfpathlineto{\pgfqpoint{4.067416in}{1.707568in}}%
\pgfpathlineto{\pgfqpoint{4.078859in}{1.718026in}}%
\pgfpathlineto{\pgfqpoint{4.090297in}{1.728350in}}%
\pgfpathlineto{\pgfqpoint{4.101728in}{1.738498in}}%
\pgfpathlineto{\pgfqpoint{4.113152in}{1.748421in}}%
\pgfpathlineto{\pgfqpoint{4.106674in}{1.764267in}}%
\pgfpathlineto{\pgfqpoint{4.100185in}{1.779624in}}%
\pgfpathlineto{\pgfqpoint{4.093687in}{1.794587in}}%
\pgfpathlineto{\pgfqpoint{4.087184in}{1.809250in}}%
\pgfpathlineto{\pgfqpoint{4.080679in}{1.823708in}}%
\pgfpathlineto{\pgfqpoint{4.069311in}{1.815661in}}%
\pgfpathlineto{\pgfqpoint{4.057933in}{1.807363in}}%
\pgfpathlineto{\pgfqpoint{4.046546in}{1.798808in}}%
\pgfpathlineto{\pgfqpoint{4.035150in}{1.789981in}}%
\pgfpathlineto{\pgfqpoint{4.023744in}{1.780870in}}%
\pgfpathlineto{\pgfqpoint{4.030197in}{1.764578in}}%
\pgfpathlineto{\pgfqpoint{4.036648in}{1.748106in}}%
\pgfpathlineto{\pgfqpoint{4.043094in}{1.731395in}}%
\pgfpathlineto{\pgfqpoint{4.049535in}{1.714386in}}%
\pgfpathclose%
\pgfusepath{stroke,fill}%
\end{pgfscope}%
\begin{pgfscope}%
\pgfpathrectangle{\pgfqpoint{0.887500in}{0.275000in}}{\pgfqpoint{4.225000in}{4.225000in}}%
\pgfusepath{clip}%
\pgfsetbuttcap%
\pgfsetroundjoin%
\definecolor{currentfill}{rgb}{0.253935,0.265254,0.529983}%
\pgfsetfillcolor{currentfill}%
\pgfsetfillopacity{0.700000}%
\pgfsetlinewidth{0.501875pt}%
\definecolor{currentstroke}{rgb}{1.000000,1.000000,1.000000}%
\pgfsetstrokecolor{currentstroke}%
\pgfsetstrokeopacity{0.500000}%
\pgfsetdash{}{0pt}%
\pgfpathmoveto{\pgfqpoint{3.851245in}{1.589778in}}%
\pgfpathlineto{\pgfqpoint{3.862807in}{1.605407in}}%
\pgfpathlineto{\pgfqpoint{3.874374in}{1.621189in}}%
\pgfpathlineto{\pgfqpoint{3.885941in}{1.636896in}}%
\pgfpathlineto{\pgfqpoint{3.897503in}{1.652298in}}%
\pgfpathlineto{\pgfqpoint{3.909053in}{1.667166in}}%
\pgfpathlineto{\pgfqpoint{3.902629in}{1.684118in}}%
\pgfpathlineto{\pgfqpoint{3.896209in}{1.701194in}}%
\pgfpathlineto{\pgfqpoint{3.889795in}{1.718457in}}%
\pgfpathlineto{\pgfqpoint{3.883388in}{1.735972in}}%
\pgfpathlineto{\pgfqpoint{3.876989in}{1.753802in}}%
\pgfpathlineto{\pgfqpoint{3.865469in}{1.740029in}}%
\pgfpathlineto{\pgfqpoint{3.853942in}{1.725951in}}%
\pgfpathlineto{\pgfqpoint{3.842413in}{1.711705in}}%
\pgfpathlineto{\pgfqpoint{3.830884in}{1.697430in}}%
\pgfpathlineto{\pgfqpoint{3.819358in}{1.683263in}}%
\pgfpathlineto{\pgfqpoint{3.825724in}{1.664135in}}%
\pgfpathlineto{\pgfqpoint{3.832096in}{1.645218in}}%
\pgfpathlineto{\pgfqpoint{3.838474in}{1.626516in}}%
\pgfpathlineto{\pgfqpoint{3.844857in}{1.608035in}}%
\pgfpathclose%
\pgfusepath{stroke,fill}%
\end{pgfscope}%
\begin{pgfscope}%
\pgfpathrectangle{\pgfqpoint{0.887500in}{0.275000in}}{\pgfqpoint{4.225000in}{4.225000in}}%
\pgfusepath{clip}%
\pgfsetbuttcap%
\pgfsetroundjoin%
\definecolor{currentfill}{rgb}{0.229739,0.322361,0.545706}%
\pgfsetfillcolor{currentfill}%
\pgfsetfillopacity{0.700000}%
\pgfsetlinewidth{0.501875pt}%
\definecolor{currentstroke}{rgb}{1.000000,1.000000,1.000000}%
\pgfsetstrokecolor{currentstroke}%
\pgfsetstrokeopacity{0.500000}%
\pgfsetdash{}{0pt}%
\pgfpathmoveto{\pgfqpoint{3.054531in}{1.737738in}}%
\pgfpathlineto{\pgfqpoint{3.066056in}{1.741467in}}%
\pgfpathlineto{\pgfqpoint{3.077576in}{1.745179in}}%
\pgfpathlineto{\pgfqpoint{3.089090in}{1.748897in}}%
\pgfpathlineto{\pgfqpoint{3.100598in}{1.752647in}}%
\pgfpathlineto{\pgfqpoint{3.112101in}{1.756454in}}%
\pgfpathlineto{\pgfqpoint{3.105773in}{1.767544in}}%
\pgfpathlineto{\pgfqpoint{3.099448in}{1.778598in}}%
\pgfpathlineto{\pgfqpoint{3.093126in}{1.789622in}}%
\pgfpathlineto{\pgfqpoint{3.086808in}{1.800618in}}%
\pgfpathlineto{\pgfqpoint{3.080494in}{1.811582in}}%
\pgfpathlineto{\pgfqpoint{3.068999in}{1.807827in}}%
\pgfpathlineto{\pgfqpoint{3.057499in}{1.804066in}}%
\pgfpathlineto{\pgfqpoint{3.045993in}{1.800310in}}%
\pgfpathlineto{\pgfqpoint{3.034482in}{1.796565in}}%
\pgfpathlineto{\pgfqpoint{3.022965in}{1.792840in}}%
\pgfpathlineto{\pgfqpoint{3.029271in}{1.781939in}}%
\pgfpathlineto{\pgfqpoint{3.035580in}{1.770980in}}%
\pgfpathlineto{\pgfqpoint{3.041894in}{1.759961in}}%
\pgfpathlineto{\pgfqpoint{3.048210in}{1.748880in}}%
\pgfpathclose%
\pgfusepath{stroke,fill}%
\end{pgfscope}%
\begin{pgfscope}%
\pgfpathrectangle{\pgfqpoint{0.887500in}{0.275000in}}{\pgfqpoint{4.225000in}{4.225000in}}%
\pgfusepath{clip}%
\pgfsetbuttcap%
\pgfsetroundjoin%
\definecolor{currentfill}{rgb}{0.237441,0.305202,0.541921}%
\pgfsetfillcolor{currentfill}%
\pgfsetfillopacity{0.700000}%
\pgfsetlinewidth{0.501875pt}%
\definecolor{currentstroke}{rgb}{1.000000,1.000000,1.000000}%
\pgfsetstrokecolor{currentstroke}%
\pgfsetstrokeopacity{0.500000}%
\pgfsetdash{}{0pt}%
\pgfpathmoveto{\pgfqpoint{3.909053in}{1.667166in}}%
\pgfpathlineto{\pgfqpoint{3.920589in}{1.681281in}}%
\pgfpathlineto{\pgfqpoint{3.932106in}{1.694592in}}%
\pgfpathlineto{\pgfqpoint{3.943607in}{1.707173in}}%
\pgfpathlineto{\pgfqpoint{3.955092in}{1.719098in}}%
\pgfpathlineto{\pgfqpoint{3.966563in}{1.730441in}}%
\pgfpathlineto{\pgfqpoint{3.960129in}{1.747281in}}%
\pgfpathlineto{\pgfqpoint{3.953697in}{1.764110in}}%
\pgfpathlineto{\pgfqpoint{3.947268in}{1.780988in}}%
\pgfpathlineto{\pgfqpoint{3.940844in}{1.797975in}}%
\pgfpathlineto{\pgfqpoint{3.934424in}{1.815131in}}%
\pgfpathlineto{\pgfqpoint{3.922963in}{1.803978in}}%
\pgfpathlineto{\pgfqpoint{3.911488in}{1.792246in}}%
\pgfpathlineto{\pgfqpoint{3.900000in}{1.779957in}}%
\pgfpathlineto{\pgfqpoint{3.888500in}{1.767135in}}%
\pgfpathlineto{\pgfqpoint{3.876989in}{1.753802in}}%
\pgfpathlineto{\pgfqpoint{3.883388in}{1.735972in}}%
\pgfpathlineto{\pgfqpoint{3.889795in}{1.718457in}}%
\pgfpathlineto{\pgfqpoint{3.896209in}{1.701194in}}%
\pgfpathlineto{\pgfqpoint{3.902629in}{1.684118in}}%
\pgfpathclose%
\pgfusepath{stroke,fill}%
\end{pgfscope}%
\begin{pgfscope}%
\pgfpathrectangle{\pgfqpoint{0.887500in}{0.275000in}}{\pgfqpoint{4.225000in}{4.225000in}}%
\pgfusepath{clip}%
\pgfsetbuttcap%
\pgfsetroundjoin%
\definecolor{currentfill}{rgb}{0.194100,0.399323,0.555565}%
\pgfsetfillcolor{currentfill}%
\pgfsetfillopacity{0.700000}%
\pgfsetlinewidth{0.501875pt}%
\definecolor{currentstroke}{rgb}{1.000000,1.000000,1.000000}%
\pgfsetstrokecolor{currentstroke}%
\pgfsetstrokeopacity{0.500000}%
\pgfsetdash{}{0pt}%
\pgfpathmoveto{\pgfqpoint{2.550577in}{1.890341in}}%
\pgfpathlineto{\pgfqpoint{2.562226in}{1.894163in}}%
\pgfpathlineto{\pgfqpoint{2.573870in}{1.897964in}}%
\pgfpathlineto{\pgfqpoint{2.585508in}{1.901746in}}%
\pgfpathlineto{\pgfqpoint{2.597141in}{1.905508in}}%
\pgfpathlineto{\pgfqpoint{2.608769in}{1.909252in}}%
\pgfpathlineto{\pgfqpoint{2.602589in}{1.919410in}}%
\pgfpathlineto{\pgfqpoint{2.596414in}{1.929522in}}%
\pgfpathlineto{\pgfqpoint{2.590244in}{1.939587in}}%
\pgfpathlineto{\pgfqpoint{2.584077in}{1.949605in}}%
\pgfpathlineto{\pgfqpoint{2.577915in}{1.959577in}}%
\pgfpathlineto{\pgfqpoint{2.566297in}{1.955892in}}%
\pgfpathlineto{\pgfqpoint{2.554673in}{1.952190in}}%
\pgfpathlineto{\pgfqpoint{2.543044in}{1.948466in}}%
\pgfpathlineto{\pgfqpoint{2.531410in}{1.944723in}}%
\pgfpathlineto{\pgfqpoint{2.519770in}{1.940959in}}%
\pgfpathlineto{\pgfqpoint{2.525923in}{1.930929in}}%
\pgfpathlineto{\pgfqpoint{2.532080in}{1.920853in}}%
\pgfpathlineto{\pgfqpoint{2.538241in}{1.910729in}}%
\pgfpathlineto{\pgfqpoint{2.544407in}{1.900559in}}%
\pgfpathclose%
\pgfusepath{stroke,fill}%
\end{pgfscope}%
\begin{pgfscope}%
\pgfpathrectangle{\pgfqpoint{0.887500in}{0.275000in}}{\pgfqpoint{4.225000in}{4.225000in}}%
\pgfusepath{clip}%
\pgfsetbuttcap%
\pgfsetroundjoin%
\definecolor{currentfill}{rgb}{0.160665,0.478540,0.558115}%
\pgfsetfillcolor{currentfill}%
\pgfsetfillopacity{0.700000}%
\pgfsetlinewidth{0.501875pt}%
\definecolor{currentstroke}{rgb}{1.000000,1.000000,1.000000}%
\pgfsetstrokecolor{currentstroke}%
\pgfsetstrokeopacity{0.500000}%
\pgfsetdash{}{0pt}%
\pgfpathmoveto{\pgfqpoint{1.810366in}{2.060717in}}%
\pgfpathlineto{\pgfqpoint{1.822198in}{2.064457in}}%
\pgfpathlineto{\pgfqpoint{1.834024in}{2.068184in}}%
\pgfpathlineto{\pgfqpoint{1.845845in}{2.071898in}}%
\pgfpathlineto{\pgfqpoint{1.857660in}{2.075601in}}%
\pgfpathlineto{\pgfqpoint{1.869470in}{2.079294in}}%
\pgfpathlineto{\pgfqpoint{1.863537in}{2.088590in}}%
\pgfpathlineto{\pgfqpoint{1.857608in}{2.097854in}}%
\pgfpathlineto{\pgfqpoint{1.851684in}{2.107088in}}%
\pgfpathlineto{\pgfqpoint{1.845765in}{2.116291in}}%
\pgfpathlineto{\pgfqpoint{1.839850in}{2.125463in}}%
\pgfpathlineto{\pgfqpoint{1.828051in}{2.121826in}}%
\pgfpathlineto{\pgfqpoint{1.816247in}{2.118177in}}%
\pgfpathlineto{\pgfqpoint{1.804437in}{2.114518in}}%
\pgfpathlineto{\pgfqpoint{1.792621in}{2.110845in}}%
\pgfpathlineto{\pgfqpoint{1.780800in}{2.107159in}}%
\pgfpathlineto{\pgfqpoint{1.786704in}{2.097933in}}%
\pgfpathlineto{\pgfqpoint{1.792613in}{2.088676in}}%
\pgfpathlineto{\pgfqpoint{1.798526in}{2.079388in}}%
\pgfpathlineto{\pgfqpoint{1.804444in}{2.070068in}}%
\pgfpathclose%
\pgfusepath{stroke,fill}%
\end{pgfscope}%
\begin{pgfscope}%
\pgfpathrectangle{\pgfqpoint{0.887500in}{0.275000in}}{\pgfqpoint{4.225000in}{4.225000in}}%
\pgfusepath{clip}%
\pgfsetbuttcap%
\pgfsetroundjoin%
\definecolor{currentfill}{rgb}{0.269308,0.218818,0.509577}%
\pgfsetfillcolor{currentfill}%
\pgfsetfillopacity{0.700000}%
\pgfsetlinewidth{0.501875pt}%
\definecolor{currentstroke}{rgb}{1.000000,1.000000,1.000000}%
\pgfsetstrokecolor{currentstroke}%
\pgfsetstrokeopacity{0.500000}%
\pgfsetdash{}{0pt}%
\pgfpathmoveto{\pgfqpoint{3.647599in}{1.541291in}}%
\pgfpathlineto{\pgfqpoint{3.659016in}{1.547542in}}%
\pgfpathlineto{\pgfqpoint{3.670424in}{1.553556in}}%
\pgfpathlineto{\pgfqpoint{3.681828in}{1.559561in}}%
\pgfpathlineto{\pgfqpoint{3.693231in}{1.565785in}}%
\pgfpathlineto{\pgfqpoint{3.704638in}{1.572456in}}%
\pgfpathlineto{\pgfqpoint{3.698234in}{1.589081in}}%
\pgfpathlineto{\pgfqpoint{3.691821in}{1.604942in}}%
\pgfpathlineto{\pgfqpoint{3.685399in}{1.620173in}}%
\pgfpathlineto{\pgfqpoint{3.678972in}{1.634953in}}%
\pgfpathlineto{\pgfqpoint{3.672543in}{1.649462in}}%
\pgfpathlineto{\pgfqpoint{3.661042in}{1.636516in}}%
\pgfpathlineto{\pgfqpoint{3.649548in}{1.624020in}}%
\pgfpathlineto{\pgfqpoint{3.638064in}{1.612195in}}%
\pgfpathlineto{\pgfqpoint{3.626592in}{1.601260in}}%
\pgfpathlineto{\pgfqpoint{3.615134in}{1.591435in}}%
\pgfpathlineto{\pgfqpoint{3.621641in}{1.582893in}}%
\pgfpathlineto{\pgfqpoint{3.628146in}{1.573942in}}%
\pgfpathlineto{\pgfqpoint{3.634644in}{1.564240in}}%
\pgfpathlineto{\pgfqpoint{3.641130in}{1.553447in}}%
\pgfpathclose%
\pgfusepath{stroke,fill}%
\end{pgfscope}%
\begin{pgfscope}%
\pgfpathrectangle{\pgfqpoint{0.887500in}{0.275000in}}{\pgfqpoint{4.225000in}{4.225000in}}%
\pgfusepath{clip}%
\pgfsetbuttcap%
\pgfsetroundjoin%
\definecolor{currentfill}{rgb}{0.276194,0.190074,0.493001}%
\pgfsetfillcolor{currentfill}%
\pgfsetfillopacity{0.700000}%
\pgfsetlinewidth{0.501875pt}%
\definecolor{currentstroke}{rgb}{1.000000,1.000000,1.000000}%
\pgfsetstrokecolor{currentstroke}%
\pgfsetstrokeopacity{0.500000}%
\pgfsetdash{}{0pt}%
\pgfpathmoveto{\pgfqpoint{3.736546in}{1.481707in}}%
\pgfpathlineto{\pgfqpoint{3.747920in}{1.486750in}}%
\pgfpathlineto{\pgfqpoint{3.759313in}{1.493164in}}%
\pgfpathlineto{\pgfqpoint{3.770731in}{1.501123in}}%
\pgfpathlineto{\pgfqpoint{3.782173in}{1.510521in}}%
\pgfpathlineto{\pgfqpoint{3.793637in}{1.521227in}}%
\pgfpathlineto{\pgfqpoint{3.787271in}{1.540391in}}%
\pgfpathlineto{\pgfqpoint{3.780910in}{1.559777in}}%
\pgfpathlineto{\pgfqpoint{3.774553in}{1.579308in}}%
\pgfpathlineto{\pgfqpoint{3.768198in}{1.598912in}}%
\pgfpathlineto{\pgfqpoint{3.761844in}{1.618513in}}%
\pgfpathlineto{\pgfqpoint{3.750375in}{1.607482in}}%
\pgfpathlineto{\pgfqpoint{3.738919in}{1.597297in}}%
\pgfpathlineto{\pgfqpoint{3.727478in}{1.588046in}}%
\pgfpathlineto{\pgfqpoint{3.716052in}{1.579801in}}%
\pgfpathlineto{\pgfqpoint{3.704638in}{1.572456in}}%
\pgfpathlineto{\pgfqpoint{3.711031in}{1.555141in}}%
\pgfpathlineto{\pgfqpoint{3.717416in}{1.537274in}}%
\pgfpathlineto{\pgfqpoint{3.723796in}{1.518988in}}%
\pgfpathlineto{\pgfqpoint{3.730172in}{1.500421in}}%
\pgfpathclose%
\pgfusepath{stroke,fill}%
\end{pgfscope}%
\begin{pgfscope}%
\pgfpathrectangle{\pgfqpoint{0.887500in}{0.275000in}}{\pgfqpoint{4.225000in}{4.225000in}}%
\pgfusepath{clip}%
\pgfsetbuttcap%
\pgfsetroundjoin%
\definecolor{currentfill}{rgb}{0.239346,0.300855,0.540844}%
\pgfsetfillcolor{currentfill}%
\pgfsetfillopacity{0.700000}%
\pgfsetlinewidth{0.501875pt}%
\definecolor{currentstroke}{rgb}{1.000000,1.000000,1.000000}%
\pgfsetstrokecolor{currentstroke}%
\pgfsetstrokeopacity{0.500000}%
\pgfsetdash{}{0pt}%
\pgfpathmoveto{\pgfqpoint{3.143794in}{1.700186in}}%
\pgfpathlineto{\pgfqpoint{3.155300in}{1.704121in}}%
\pgfpathlineto{\pgfqpoint{3.166800in}{1.708141in}}%
\pgfpathlineto{\pgfqpoint{3.178295in}{1.712262in}}%
\pgfpathlineto{\pgfqpoint{3.189785in}{1.716439in}}%
\pgfpathlineto{\pgfqpoint{3.201269in}{1.720585in}}%
\pgfpathlineto{\pgfqpoint{3.194916in}{1.732039in}}%
\pgfpathlineto{\pgfqpoint{3.188566in}{1.743393in}}%
\pgfpathlineto{\pgfqpoint{3.182219in}{1.754644in}}%
\pgfpathlineto{\pgfqpoint{3.175876in}{1.765785in}}%
\pgfpathlineto{\pgfqpoint{3.169535in}{1.776811in}}%
\pgfpathlineto{\pgfqpoint{3.158059in}{1.772651in}}%
\pgfpathlineto{\pgfqpoint{3.146577in}{1.768458in}}%
\pgfpathlineto{\pgfqpoint{3.135091in}{1.764337in}}%
\pgfpathlineto{\pgfqpoint{3.123599in}{1.760342in}}%
\pgfpathlineto{\pgfqpoint{3.112101in}{1.756454in}}%
\pgfpathlineto{\pgfqpoint{3.118433in}{1.745319in}}%
\pgfpathlineto{\pgfqpoint{3.124768in}{1.734132in}}%
\pgfpathlineto{\pgfqpoint{3.131107in}{1.722886in}}%
\pgfpathlineto{\pgfqpoint{3.137449in}{1.711573in}}%
\pgfpathclose%
\pgfusepath{stroke,fill}%
\end{pgfscope}%
\begin{pgfscope}%
\pgfpathrectangle{\pgfqpoint{0.887500in}{0.275000in}}{\pgfqpoint{4.225000in}{4.225000in}}%
\pgfusepath{clip}%
\pgfsetbuttcap%
\pgfsetroundjoin%
\definecolor{currentfill}{rgb}{0.241237,0.296485,0.539709}%
\pgfsetfillcolor{currentfill}%
\pgfsetfillopacity{0.700000}%
\pgfsetlinewidth{0.501875pt}%
\definecolor{currentstroke}{rgb}{1.000000,1.000000,1.000000}%
\pgfsetstrokecolor{currentstroke}%
\pgfsetstrokeopacity{0.500000}%
\pgfsetdash{}{0pt}%
\pgfpathmoveto{\pgfqpoint{4.145277in}{1.658557in}}%
\pgfpathlineto{\pgfqpoint{4.156724in}{1.669270in}}%
\pgfpathlineto{\pgfqpoint{4.168160in}{1.679671in}}%
\pgfpathlineto{\pgfqpoint{4.179583in}{1.689682in}}%
\pgfpathlineto{\pgfqpoint{4.190991in}{1.699226in}}%
\pgfpathlineto{\pgfqpoint{4.202381in}{1.708226in}}%
\pgfpathlineto{\pgfqpoint{4.195971in}{1.726906in}}%
\pgfpathlineto{\pgfqpoint{4.189530in}{1.744467in}}%
\pgfpathlineto{\pgfqpoint{4.183061in}{1.761027in}}%
\pgfpathlineto{\pgfqpoint{4.176569in}{1.776703in}}%
\pgfpathlineto{\pgfqpoint{4.170057in}{1.791610in}}%
\pgfpathlineto{\pgfqpoint{4.158715in}{1.784135in}}%
\pgfpathlineto{\pgfqpoint{4.147350in}{1.775987in}}%
\pgfpathlineto{\pgfqpoint{4.135966in}{1.767258in}}%
\pgfpathlineto{\pgfqpoint{4.124565in}{1.758038in}}%
\pgfpathlineto{\pgfqpoint{4.113152in}{1.748421in}}%
\pgfpathlineto{\pgfqpoint{4.119617in}{1.731992in}}%
\pgfpathlineto{\pgfqpoint{4.126064in}{1.714885in}}%
\pgfpathlineto{\pgfqpoint{4.132492in}{1.697006in}}%
\pgfpathlineto{\pgfqpoint{4.138898in}{1.678261in}}%
\pgfpathclose%
\pgfusepath{stroke,fill}%
\end{pgfscope}%
\begin{pgfscope}%
\pgfpathrectangle{\pgfqpoint{0.887500in}{0.275000in}}{\pgfqpoint{4.225000in}{4.225000in}}%
\pgfusepath{clip}%
\pgfsetbuttcap%
\pgfsetroundjoin%
\definecolor{currentfill}{rgb}{0.246811,0.283237,0.535941}%
\pgfsetfillcolor{currentfill}%
\pgfsetfillopacity{0.700000}%
\pgfsetlinewidth{0.501875pt}%
\definecolor{currentstroke}{rgb}{1.000000,1.000000,1.000000}%
\pgfsetstrokecolor{currentstroke}%
\pgfsetstrokeopacity{0.500000}%
\pgfsetdash{}{0pt}%
\pgfpathmoveto{\pgfqpoint{3.233080in}{1.662005in}}%
\pgfpathlineto{\pgfqpoint{3.244566in}{1.666023in}}%
\pgfpathlineto{\pgfqpoint{3.256046in}{1.669947in}}%
\pgfpathlineto{\pgfqpoint{3.267519in}{1.673733in}}%
\pgfpathlineto{\pgfqpoint{3.278986in}{1.677337in}}%
\pgfpathlineto{\pgfqpoint{3.290446in}{1.680734in}}%
\pgfpathlineto{\pgfqpoint{3.284069in}{1.692316in}}%
\pgfpathlineto{\pgfqpoint{3.277695in}{1.703810in}}%
\pgfpathlineto{\pgfqpoint{3.271324in}{1.715220in}}%
\pgfpathlineto{\pgfqpoint{3.264956in}{1.726555in}}%
\pgfpathlineto{\pgfqpoint{3.258592in}{1.737825in}}%
\pgfpathlineto{\pgfqpoint{3.247142in}{1.735113in}}%
\pgfpathlineto{\pgfqpoint{3.235685in}{1.731964in}}%
\pgfpathlineto{\pgfqpoint{3.224220in}{1.728435in}}%
\pgfpathlineto{\pgfqpoint{3.212748in}{1.724613in}}%
\pgfpathlineto{\pgfqpoint{3.201269in}{1.720585in}}%
\pgfpathlineto{\pgfqpoint{3.207626in}{1.709037in}}%
\pgfpathlineto{\pgfqpoint{3.213985in}{1.697400in}}%
\pgfpathlineto{\pgfqpoint{3.220347in}{1.685679in}}%
\pgfpathlineto{\pgfqpoint{3.226712in}{1.673879in}}%
\pgfpathclose%
\pgfusepath{stroke,fill}%
\end{pgfscope}%
\begin{pgfscope}%
\pgfpathrectangle{\pgfqpoint{0.887500in}{0.275000in}}{\pgfqpoint{4.225000in}{4.225000in}}%
\pgfusepath{clip}%
\pgfsetbuttcap%
\pgfsetroundjoin%
\definecolor{currentfill}{rgb}{0.279574,0.170599,0.479997}%
\pgfsetfillcolor{currentfill}%
\pgfsetfillopacity{0.700000}%
\pgfsetlinewidth{0.501875pt}%
\definecolor{currentstroke}{rgb}{1.000000,1.000000,1.000000}%
\pgfsetstrokecolor{currentstroke}%
\pgfsetstrokeopacity{0.500000}%
\pgfsetdash{}{0pt}%
\pgfpathmoveto{\pgfqpoint{3.825599in}{1.431334in}}%
\pgfpathlineto{\pgfqpoint{3.837095in}{1.443469in}}%
\pgfpathlineto{\pgfqpoint{3.848612in}{1.456718in}}%
\pgfpathlineto{\pgfqpoint{3.860151in}{1.470981in}}%
\pgfpathlineto{\pgfqpoint{3.871709in}{1.486151in}}%
\pgfpathlineto{\pgfqpoint{3.883283in}{1.502034in}}%
\pgfpathlineto{\pgfqpoint{3.876862in}{1.519095in}}%
\pgfpathlineto{\pgfqpoint{3.870448in}{1.536405in}}%
\pgfpathlineto{\pgfqpoint{3.864041in}{1.553958in}}%
\pgfpathlineto{\pgfqpoint{3.857640in}{1.571751in}}%
\pgfpathlineto{\pgfqpoint{3.851245in}{1.589778in}}%
\pgfpathlineto{\pgfqpoint{3.839692in}{1.574532in}}%
\pgfpathlineto{\pgfqpoint{3.828152in}{1.559895in}}%
\pgfpathlineto{\pgfqpoint{3.816629in}{1.546044in}}%
\pgfpathlineto{\pgfqpoint{3.805123in}{1.533112in}}%
\pgfpathlineto{\pgfqpoint{3.793637in}{1.521227in}}%
\pgfpathlineto{\pgfqpoint{3.800010in}{1.502359in}}%
\pgfpathlineto{\pgfqpoint{3.806391in}{1.483862in}}%
\pgfpathlineto{\pgfqpoint{3.812781in}{1.465809in}}%
\pgfpathlineto{\pgfqpoint{3.819184in}{1.448275in}}%
\pgfpathclose%
\pgfusepath{stroke,fill}%
\end{pgfscope}%
\begin{pgfscope}%
\pgfpathrectangle{\pgfqpoint{0.887500in}{0.275000in}}{\pgfqpoint{4.225000in}{4.225000in}}%
\pgfusepath{clip}%
\pgfsetbuttcap%
\pgfsetroundjoin%
\definecolor{currentfill}{rgb}{0.179019,0.433756,0.557430}%
\pgfsetfillcolor{currentfill}%
\pgfsetfillopacity{0.700000}%
\pgfsetlinewidth{0.501875pt}%
\definecolor{currentstroke}{rgb}{1.000000,1.000000,1.000000}%
\pgfsetstrokecolor{currentstroke}%
\pgfsetstrokeopacity{0.500000}%
\pgfsetdash{}{0pt}%
\pgfpathmoveto{\pgfqpoint{2.224911in}{1.963708in}}%
\pgfpathlineto{\pgfqpoint{2.236644in}{1.967492in}}%
\pgfpathlineto{\pgfqpoint{2.248370in}{1.971265in}}%
\pgfpathlineto{\pgfqpoint{2.260092in}{1.975029in}}%
\pgfpathlineto{\pgfqpoint{2.271807in}{1.978785in}}%
\pgfpathlineto{\pgfqpoint{2.283517in}{1.982536in}}%
\pgfpathlineto{\pgfqpoint{2.277442in}{1.992242in}}%
\pgfpathlineto{\pgfqpoint{2.271372in}{2.001907in}}%
\pgfpathlineto{\pgfqpoint{2.265307in}{2.011532in}}%
\pgfpathlineto{\pgfqpoint{2.259245in}{2.021116in}}%
\pgfpathlineto{\pgfqpoint{2.253189in}{2.030661in}}%
\pgfpathlineto{\pgfqpoint{2.241489in}{2.026975in}}%
\pgfpathlineto{\pgfqpoint{2.229783in}{2.023283in}}%
\pgfpathlineto{\pgfqpoint{2.218072in}{2.019583in}}%
\pgfpathlineto{\pgfqpoint{2.206355in}{2.015874in}}%
\pgfpathlineto{\pgfqpoint{2.194632in}{2.012155in}}%
\pgfpathlineto{\pgfqpoint{2.200679in}{2.002547in}}%
\pgfpathlineto{\pgfqpoint{2.206730in}{1.992898in}}%
\pgfpathlineto{\pgfqpoint{2.212786in}{1.983209in}}%
\pgfpathlineto{\pgfqpoint{2.218846in}{1.973479in}}%
\pgfpathclose%
\pgfusepath{stroke,fill}%
\end{pgfscope}%
\begin{pgfscope}%
\pgfpathrectangle{\pgfqpoint{0.887500in}{0.275000in}}{\pgfqpoint{4.225000in}{4.225000in}}%
\pgfusepath{clip}%
\pgfsetbuttcap%
\pgfsetroundjoin%
\definecolor{currentfill}{rgb}{0.201239,0.383670,0.554294}%
\pgfsetfillcolor{currentfill}%
\pgfsetfillopacity{0.700000}%
\pgfsetlinewidth{0.501875pt}%
\definecolor{currentstroke}{rgb}{1.000000,1.000000,1.000000}%
\pgfsetstrokecolor{currentstroke}%
\pgfsetstrokeopacity{0.500000}%
\pgfsetdash{}{0pt}%
\pgfpathmoveto{\pgfqpoint{2.639727in}{1.857758in}}%
\pgfpathlineto{\pgfqpoint{2.651358in}{1.861546in}}%
\pgfpathlineto{\pgfqpoint{2.662984in}{1.865324in}}%
\pgfpathlineto{\pgfqpoint{2.674604in}{1.869094in}}%
\pgfpathlineto{\pgfqpoint{2.686218in}{1.872860in}}%
\pgfpathlineto{\pgfqpoint{2.697827in}{1.876626in}}%
\pgfpathlineto{\pgfqpoint{2.691618in}{1.886961in}}%
\pgfpathlineto{\pgfqpoint{2.685413in}{1.897248in}}%
\pgfpathlineto{\pgfqpoint{2.679212in}{1.907488in}}%
\pgfpathlineto{\pgfqpoint{2.673015in}{1.917681in}}%
\pgfpathlineto{\pgfqpoint{2.666822in}{1.927827in}}%
\pgfpathlineto{\pgfqpoint{2.655223in}{1.924120in}}%
\pgfpathlineto{\pgfqpoint{2.643618in}{1.920412in}}%
\pgfpathlineto{\pgfqpoint{2.632007in}{1.916701in}}%
\pgfpathlineto{\pgfqpoint{2.620391in}{1.912982in}}%
\pgfpathlineto{\pgfqpoint{2.608769in}{1.909252in}}%
\pgfpathlineto{\pgfqpoint{2.614952in}{1.899047in}}%
\pgfpathlineto{\pgfqpoint{2.621140in}{1.888795in}}%
\pgfpathlineto{\pgfqpoint{2.627331in}{1.878497in}}%
\pgfpathlineto{\pgfqpoint{2.633527in}{1.868151in}}%
\pgfpathclose%
\pgfusepath{stroke,fill}%
\end{pgfscope}%
\begin{pgfscope}%
\pgfpathrectangle{\pgfqpoint{0.887500in}{0.275000in}}{\pgfqpoint{4.225000in}{4.225000in}}%
\pgfusepath{clip}%
\pgfsetbuttcap%
\pgfsetroundjoin%
\definecolor{currentfill}{rgb}{0.255645,0.260703,0.528312}%
\pgfsetfillcolor{currentfill}%
\pgfsetfillopacity{0.700000}%
\pgfsetlinewidth{0.501875pt}%
\definecolor{currentstroke}{rgb}{1.000000,1.000000,1.000000}%
\pgfsetstrokecolor{currentstroke}%
\pgfsetstrokeopacity{0.500000}%
\pgfsetdash{}{0pt}%
\pgfpathmoveto{\pgfqpoint{3.322372in}{1.621433in}}%
\pgfpathlineto{\pgfqpoint{3.333835in}{1.625107in}}%
\pgfpathlineto{\pgfqpoint{3.345291in}{1.628781in}}%
\pgfpathlineto{\pgfqpoint{3.356743in}{1.632529in}}%
\pgfpathlineto{\pgfqpoint{3.368190in}{1.636424in}}%
\pgfpathlineto{\pgfqpoint{3.379634in}{1.640539in}}%
\pgfpathlineto{\pgfqpoint{3.373237in}{1.652596in}}%
\pgfpathlineto{\pgfqpoint{3.366842in}{1.664476in}}%
\pgfpathlineto{\pgfqpoint{3.360448in}{1.676171in}}%
\pgfpathlineto{\pgfqpoint{3.354057in}{1.687668in}}%
\pgfpathlineto{\pgfqpoint{3.347667in}{1.698957in}}%
\pgfpathlineto{\pgfqpoint{3.336229in}{1.694653in}}%
\pgfpathlineto{\pgfqpoint{3.324790in}{1.690827in}}%
\pgfpathlineto{\pgfqpoint{3.313347in}{1.687330in}}%
\pgfpathlineto{\pgfqpoint{3.301899in}{1.684016in}}%
\pgfpathlineto{\pgfqpoint{3.290446in}{1.680734in}}%
\pgfpathlineto{\pgfqpoint{3.296826in}{1.669062in}}%
\pgfpathlineto{\pgfqpoint{3.303208in}{1.657298in}}%
\pgfpathlineto{\pgfqpoint{3.309594in}{1.645440in}}%
\pgfpathlineto{\pgfqpoint{3.315982in}{1.633486in}}%
\pgfpathclose%
\pgfusepath{stroke,fill}%
\end{pgfscope}%
\begin{pgfscope}%
\pgfpathrectangle{\pgfqpoint{0.887500in}{0.275000in}}{\pgfqpoint{4.225000in}{4.225000in}}%
\pgfusepath{clip}%
\pgfsetbuttcap%
\pgfsetroundjoin%
\definecolor{currentfill}{rgb}{0.262138,0.242286,0.520837}%
\pgfsetfillcolor{currentfill}%
\pgfsetfillopacity{0.700000}%
\pgfsetlinewidth{0.501875pt}%
\definecolor{currentstroke}{rgb}{1.000000,1.000000,1.000000}%
\pgfsetstrokecolor{currentstroke}%
\pgfsetstrokeopacity{0.500000}%
\pgfsetdash{}{0pt}%
\pgfpathmoveto{\pgfqpoint{4.176675in}{1.542385in}}%
\pgfpathlineto{\pgfqpoint{4.188101in}{1.552534in}}%
\pgfpathlineto{\pgfqpoint{4.199526in}{1.562722in}}%
\pgfpathlineto{\pgfqpoint{4.210952in}{1.572998in}}%
\pgfpathlineto{\pgfqpoint{4.222381in}{1.583411in}}%
\pgfpathlineto{\pgfqpoint{4.233813in}{1.594011in}}%
\pgfpathlineto{\pgfqpoint{4.227623in}{1.620012in}}%
\pgfpathlineto{\pgfqpoint{4.221380in}{1.644320in}}%
\pgfpathlineto{\pgfqpoint{4.215089in}{1.667049in}}%
\pgfpathlineto{\pgfqpoint{4.208755in}{1.688312in}}%
\pgfpathlineto{\pgfqpoint{4.202381in}{1.708226in}}%
\pgfpathlineto{\pgfqpoint{4.190991in}{1.699226in}}%
\pgfpathlineto{\pgfqpoint{4.179583in}{1.689682in}}%
\pgfpathlineto{\pgfqpoint{4.168160in}{1.679671in}}%
\pgfpathlineto{\pgfqpoint{4.156724in}{1.669270in}}%
\pgfpathlineto{\pgfqpoint{4.145277in}{1.658557in}}%
\pgfpathlineto{\pgfqpoint{4.151628in}{1.637799in}}%
\pgfpathlineto{\pgfqpoint{4.157946in}{1.615896in}}%
\pgfpathlineto{\pgfqpoint{4.164229in}{1.592754in}}%
\pgfpathlineto{\pgfqpoint{4.170473in}{1.568281in}}%
\pgfpathclose%
\pgfusepath{stroke,fill}%
\end{pgfscope}%
\begin{pgfscope}%
\pgfpathrectangle{\pgfqpoint{0.887500in}{0.275000in}}{\pgfqpoint{4.225000in}{4.225000in}}%
\pgfusepath{clip}%
\pgfsetbuttcap%
\pgfsetroundjoin%
\definecolor{currentfill}{rgb}{0.271828,0.209303,0.504434}%
\pgfsetfillcolor{currentfill}%
\pgfsetfillopacity{0.700000}%
\pgfsetlinewidth{0.501875pt}%
\definecolor{currentstroke}{rgb}{1.000000,1.000000,1.000000}%
\pgfsetstrokecolor{currentstroke}%
\pgfsetstrokeopacity{0.500000}%
\pgfsetdash{}{0pt}%
\pgfpathmoveto{\pgfqpoint{3.500929in}{1.535058in}}%
\pgfpathlineto{\pgfqpoint{3.512359in}{1.539829in}}%
\pgfpathlineto{\pgfqpoint{3.523785in}{1.544585in}}%
\pgfpathlineto{\pgfqpoint{3.535204in}{1.549267in}}%
\pgfpathlineto{\pgfqpoint{3.546618in}{1.553875in}}%
\pgfpathlineto{\pgfqpoint{3.558027in}{1.558591in}}%
\pgfpathlineto{\pgfqpoint{3.551556in}{1.568456in}}%
\pgfpathlineto{\pgfqpoint{3.545091in}{1.578504in}}%
\pgfpathlineto{\pgfqpoint{3.538635in}{1.588932in}}%
\pgfpathlineto{\pgfqpoint{3.532190in}{1.599937in}}%
\pgfpathlineto{\pgfqpoint{3.525758in}{1.611714in}}%
\pgfpathlineto{\pgfqpoint{3.514391in}{1.609860in}}%
\pgfpathlineto{\pgfqpoint{3.503016in}{1.607899in}}%
\pgfpathlineto{\pgfqpoint{3.491628in}{1.605340in}}%
\pgfpathlineto{\pgfqpoint{3.480227in}{1.602172in}}%
\pgfpathlineto{\pgfqpoint{3.468815in}{1.598532in}}%
\pgfpathlineto{\pgfqpoint{3.475230in}{1.585664in}}%
\pgfpathlineto{\pgfqpoint{3.481649in}{1.572908in}}%
\pgfpathlineto{\pgfqpoint{3.488072in}{1.560240in}}%
\pgfpathlineto{\pgfqpoint{3.494498in}{1.547632in}}%
\pgfpathclose%
\pgfusepath{stroke,fill}%
\end{pgfscope}%
\begin{pgfscope}%
\pgfpathrectangle{\pgfqpoint{0.887500in}{0.275000in}}{\pgfqpoint{4.225000in}{4.225000in}}%
\pgfusepath{clip}%
\pgfsetbuttcap%
\pgfsetroundjoin%
\definecolor{currentfill}{rgb}{0.243113,0.292092,0.538516}%
\pgfsetfillcolor{currentfill}%
\pgfsetfillopacity{0.700000}%
\pgfsetlinewidth{0.501875pt}%
\definecolor{currentstroke}{rgb}{1.000000,1.000000,1.000000}%
\pgfsetstrokecolor{currentstroke}%
\pgfsetstrokeopacity{0.500000}%
\pgfsetdash{}{0pt}%
\pgfpathmoveto{\pgfqpoint{3.998702in}{1.644000in}}%
\pgfpathlineto{\pgfqpoint{4.010161in}{1.654703in}}%
\pgfpathlineto{\pgfqpoint{4.021615in}{1.665257in}}%
\pgfpathlineto{\pgfqpoint{4.033067in}{1.675821in}}%
\pgfpathlineto{\pgfqpoint{4.044519in}{1.686423in}}%
\pgfpathlineto{\pgfqpoint{4.055969in}{1.697020in}}%
\pgfpathlineto{\pgfqpoint{4.049535in}{1.714386in}}%
\pgfpathlineto{\pgfqpoint{4.043094in}{1.731395in}}%
\pgfpathlineto{\pgfqpoint{4.036648in}{1.748106in}}%
\pgfpathlineto{\pgfqpoint{4.030197in}{1.764578in}}%
\pgfpathlineto{\pgfqpoint{4.023744in}{1.780870in}}%
\pgfpathlineto{\pgfqpoint{4.012328in}{1.771459in}}%
\pgfpathlineto{\pgfqpoint{4.000902in}{1.761736in}}%
\pgfpathlineto{\pgfqpoint{3.989466in}{1.751687in}}%
\pgfpathlineto{\pgfqpoint{3.978020in}{1.741279in}}%
\pgfpathlineto{\pgfqpoint{3.966563in}{1.730441in}}%
\pgfpathlineto{\pgfqpoint{3.972996in}{1.713531in}}%
\pgfpathlineto{\pgfqpoint{3.979429in}{1.696492in}}%
\pgfpathlineto{\pgfqpoint{3.985858in}{1.679263in}}%
\pgfpathlineto{\pgfqpoint{3.992283in}{1.661785in}}%
\pgfpathclose%
\pgfusepath{stroke,fill}%
\end{pgfscope}%
\begin{pgfscope}%
\pgfpathrectangle{\pgfqpoint{0.887500in}{0.275000in}}{\pgfqpoint{4.225000in}{4.225000in}}%
\pgfusepath{clip}%
\pgfsetbuttcap%
\pgfsetroundjoin%
\definecolor{currentfill}{rgb}{0.263663,0.237631,0.518762}%
\pgfsetfillcolor{currentfill}%
\pgfsetfillopacity{0.700000}%
\pgfsetlinewidth{0.501875pt}%
\definecolor{currentstroke}{rgb}{1.000000,1.000000,1.000000}%
\pgfsetstrokecolor{currentstroke}%
\pgfsetstrokeopacity{0.500000}%
\pgfsetdash{}{0pt}%
\pgfpathmoveto{\pgfqpoint{3.411649in}{1.578006in}}%
\pgfpathlineto{\pgfqpoint{3.423092in}{1.581989in}}%
\pgfpathlineto{\pgfqpoint{3.434531in}{1.586149in}}%
\pgfpathlineto{\pgfqpoint{3.445966in}{1.590384in}}%
\pgfpathlineto{\pgfqpoint{3.457394in}{1.594557in}}%
\pgfpathlineto{\pgfqpoint{3.468815in}{1.598532in}}%
\pgfpathlineto{\pgfqpoint{3.462405in}{1.611541in}}%
\pgfpathlineto{\pgfqpoint{3.456000in}{1.624712in}}%
\pgfpathlineto{\pgfqpoint{3.449599in}{1.638027in}}%
\pgfpathlineto{\pgfqpoint{3.443202in}{1.651433in}}%
\pgfpathlineto{\pgfqpoint{3.436808in}{1.664875in}}%
\pgfpathlineto{\pgfqpoint{3.425381in}{1.659793in}}%
\pgfpathlineto{\pgfqpoint{3.413949in}{1.654688in}}%
\pgfpathlineto{\pgfqpoint{3.402513in}{1.649695in}}%
\pgfpathlineto{\pgfqpoint{3.391075in}{1.644948in}}%
\pgfpathlineto{\pgfqpoint{3.379634in}{1.640539in}}%
\pgfpathlineto{\pgfqpoint{3.386033in}{1.628318in}}%
\pgfpathlineto{\pgfqpoint{3.392434in}{1.615943in}}%
\pgfpathlineto{\pgfqpoint{3.398837in}{1.603425in}}%
\pgfpathlineto{\pgfqpoint{3.405242in}{1.590775in}}%
\pgfpathclose%
\pgfusepath{stroke,fill}%
\end{pgfscope}%
\begin{pgfscope}%
\pgfpathrectangle{\pgfqpoint{0.887500in}{0.275000in}}{\pgfqpoint{4.225000in}{4.225000in}}%
\pgfusepath{clip}%
\pgfsetbuttcap%
\pgfsetroundjoin%
\definecolor{currentfill}{rgb}{0.165117,0.467423,0.558141}%
\pgfsetfillcolor{currentfill}%
\pgfsetfillopacity{0.700000}%
\pgfsetlinewidth{0.501875pt}%
\definecolor{currentstroke}{rgb}{1.000000,1.000000,1.000000}%
\pgfsetstrokecolor{currentstroke}%
\pgfsetstrokeopacity{0.500000}%
\pgfsetdash{}{0pt}%
\pgfpathmoveto{\pgfqpoint{1.899203in}{2.032344in}}%
\pgfpathlineto{\pgfqpoint{1.911017in}{2.036083in}}%
\pgfpathlineto{\pgfqpoint{1.922826in}{2.039816in}}%
\pgfpathlineto{\pgfqpoint{1.934629in}{2.043542in}}%
\pgfpathlineto{\pgfqpoint{1.946427in}{2.047263in}}%
\pgfpathlineto{\pgfqpoint{1.958219in}{2.050979in}}%
\pgfpathlineto{\pgfqpoint{1.952253in}{2.060375in}}%
\pgfpathlineto{\pgfqpoint{1.946291in}{2.069740in}}%
\pgfpathlineto{\pgfqpoint{1.940334in}{2.079072in}}%
\pgfpathlineto{\pgfqpoint{1.934381in}{2.088373in}}%
\pgfpathlineto{\pgfqpoint{1.928433in}{2.097643in}}%
\pgfpathlineto{\pgfqpoint{1.916652in}{2.093985in}}%
\pgfpathlineto{\pgfqpoint{1.904865in}{2.090322in}}%
\pgfpathlineto{\pgfqpoint{1.893072in}{2.086653in}}%
\pgfpathlineto{\pgfqpoint{1.881274in}{2.082977in}}%
\pgfpathlineto{\pgfqpoint{1.869470in}{2.079294in}}%
\pgfpathlineto{\pgfqpoint{1.875407in}{2.069967in}}%
\pgfpathlineto{\pgfqpoint{1.881349in}{2.060609in}}%
\pgfpathlineto{\pgfqpoint{1.887296in}{2.051219in}}%
\pgfpathlineto{\pgfqpoint{1.893247in}{2.041797in}}%
\pgfpathclose%
\pgfusepath{stroke,fill}%
\end{pgfscope}%
\begin{pgfscope}%
\pgfpathrectangle{\pgfqpoint{0.887500in}{0.275000in}}{\pgfqpoint{4.225000in}{4.225000in}}%
\pgfusepath{clip}%
\pgfsetbuttcap%
\pgfsetroundjoin%
\definecolor{currentfill}{rgb}{0.269308,0.218818,0.509577}%
\pgfsetfillcolor{currentfill}%
\pgfsetfillopacity{0.700000}%
\pgfsetlinewidth{0.501875pt}%
\definecolor{currentstroke}{rgb}{1.000000,1.000000,1.000000}%
\pgfsetstrokecolor{currentstroke}%
\pgfsetstrokeopacity{0.500000}%
\pgfsetdash{}{0pt}%
\pgfpathmoveto{\pgfqpoint{3.883283in}{1.502034in}}%
\pgfpathlineto{\pgfqpoint{3.894869in}{1.518345in}}%
\pgfpathlineto{\pgfqpoint{3.906459in}{1.534794in}}%
\pgfpathlineto{\pgfqpoint{3.918048in}{1.551089in}}%
\pgfpathlineto{\pgfqpoint{3.929628in}{1.566938in}}%
\pgfpathlineto{\pgfqpoint{3.941193in}{1.582049in}}%
\pgfpathlineto{\pgfqpoint{3.934767in}{1.599331in}}%
\pgfpathlineto{\pgfqpoint{3.928339in}{1.616420in}}%
\pgfpathlineto{\pgfqpoint{3.921910in}{1.633380in}}%
\pgfpathlineto{\pgfqpoint{3.915481in}{1.650274in}}%
\pgfpathlineto{\pgfqpoint{3.909053in}{1.667166in}}%
\pgfpathlineto{\pgfqpoint{3.897503in}{1.652298in}}%
\pgfpathlineto{\pgfqpoint{3.885941in}{1.636896in}}%
\pgfpathlineto{\pgfqpoint{3.874374in}{1.621189in}}%
\pgfpathlineto{\pgfqpoint{3.862807in}{1.605407in}}%
\pgfpathlineto{\pgfqpoint{3.851245in}{1.589778in}}%
\pgfpathlineto{\pgfqpoint{3.857640in}{1.571751in}}%
\pgfpathlineto{\pgfqpoint{3.864041in}{1.553958in}}%
\pgfpathlineto{\pgfqpoint{3.870448in}{1.536405in}}%
\pgfpathlineto{\pgfqpoint{3.876862in}{1.519095in}}%
\pgfpathclose%
\pgfusepath{stroke,fill}%
\end{pgfscope}%
\begin{pgfscope}%
\pgfpathrectangle{\pgfqpoint{0.887500in}{0.275000in}}{\pgfqpoint{4.225000in}{4.225000in}}%
\pgfusepath{clip}%
\pgfsetbuttcap%
\pgfsetroundjoin%
\definecolor{currentfill}{rgb}{0.208623,0.367752,0.552675}%
\pgfsetfillcolor{currentfill}%
\pgfsetfillopacity{0.700000}%
\pgfsetlinewidth{0.501875pt}%
\definecolor{currentstroke}{rgb}{1.000000,1.000000,1.000000}%
\pgfsetstrokecolor{currentstroke}%
\pgfsetstrokeopacity{0.500000}%
\pgfsetdash{}{0pt}%
\pgfpathmoveto{\pgfqpoint{2.728933in}{1.824209in}}%
\pgfpathlineto{\pgfqpoint{2.740545in}{1.828036in}}%
\pgfpathlineto{\pgfqpoint{2.752152in}{1.831867in}}%
\pgfpathlineto{\pgfqpoint{2.763752in}{1.835703in}}%
\pgfpathlineto{\pgfqpoint{2.775347in}{1.839546in}}%
\pgfpathlineto{\pgfqpoint{2.786936in}{1.843397in}}%
\pgfpathlineto{\pgfqpoint{2.780698in}{1.853923in}}%
\pgfpathlineto{\pgfqpoint{2.774463in}{1.864398in}}%
\pgfpathlineto{\pgfqpoint{2.768233in}{1.874823in}}%
\pgfpathlineto{\pgfqpoint{2.762007in}{1.885199in}}%
\pgfpathlineto{\pgfqpoint{2.755784in}{1.895526in}}%
\pgfpathlineto{\pgfqpoint{2.744204in}{1.891730in}}%
\pgfpathlineto{\pgfqpoint{2.732618in}{1.887944in}}%
\pgfpathlineto{\pgfqpoint{2.721027in}{1.884166in}}%
\pgfpathlineto{\pgfqpoint{2.709430in}{1.880394in}}%
\pgfpathlineto{\pgfqpoint{2.697827in}{1.876626in}}%
\pgfpathlineto{\pgfqpoint{2.704040in}{1.866242in}}%
\pgfpathlineto{\pgfqpoint{2.710257in}{1.855809in}}%
\pgfpathlineto{\pgfqpoint{2.716479in}{1.845326in}}%
\pgfpathlineto{\pgfqpoint{2.722704in}{1.834793in}}%
\pgfpathclose%
\pgfusepath{stroke,fill}%
\end{pgfscope}%
\begin{pgfscope}%
\pgfpathrectangle{\pgfqpoint{0.887500in}{0.275000in}}{\pgfqpoint{4.225000in}{4.225000in}}%
\pgfusepath{clip}%
\pgfsetbuttcap%
\pgfsetroundjoin%
\definecolor{currentfill}{rgb}{0.252194,0.269783,0.531579}%
\pgfsetfillcolor{currentfill}%
\pgfsetfillopacity{0.700000}%
\pgfsetlinewidth{0.501875pt}%
\definecolor{currentstroke}{rgb}{1.000000,1.000000,1.000000}%
\pgfsetstrokecolor{currentstroke}%
\pgfsetstrokeopacity{0.500000}%
\pgfsetdash{}{0pt}%
\pgfpathmoveto{\pgfqpoint{4.087962in}{1.602778in}}%
\pgfpathlineto{\pgfqpoint{4.099430in}{1.614023in}}%
\pgfpathlineto{\pgfqpoint{4.110897in}{1.625282in}}%
\pgfpathlineto{\pgfqpoint{4.122362in}{1.636496in}}%
\pgfpathlineto{\pgfqpoint{4.133822in}{1.647608in}}%
\pgfpathlineto{\pgfqpoint{4.145277in}{1.658557in}}%
\pgfpathlineto{\pgfqpoint{4.138898in}{1.678261in}}%
\pgfpathlineto{\pgfqpoint{4.132492in}{1.697006in}}%
\pgfpathlineto{\pgfqpoint{4.126064in}{1.714885in}}%
\pgfpathlineto{\pgfqpoint{4.119617in}{1.731992in}}%
\pgfpathlineto{\pgfqpoint{4.113152in}{1.748421in}}%
\pgfpathlineto{\pgfqpoint{4.101728in}{1.738498in}}%
\pgfpathlineto{\pgfqpoint{4.090297in}{1.728350in}}%
\pgfpathlineto{\pgfqpoint{4.078859in}{1.718026in}}%
\pgfpathlineto{\pgfqpoint{4.067416in}{1.707568in}}%
\pgfpathlineto{\pgfqpoint{4.055969in}{1.697020in}}%
\pgfpathlineto{\pgfqpoint{4.062393in}{1.679238in}}%
\pgfpathlineto{\pgfqpoint{4.068806in}{1.660981in}}%
\pgfpathlineto{\pgfqpoint{4.075207in}{1.642191in}}%
\pgfpathlineto{\pgfqpoint{4.081593in}{1.622809in}}%
\pgfpathclose%
\pgfusepath{stroke,fill}%
\end{pgfscope}%
\begin{pgfscope}%
\pgfpathrectangle{\pgfqpoint{0.887500in}{0.275000in}}{\pgfqpoint{4.225000in}{4.225000in}}%
\pgfusepath{clip}%
\pgfsetbuttcap%
\pgfsetroundjoin%
\definecolor{currentfill}{rgb}{0.282884,0.135920,0.453427}%
\pgfsetfillcolor{currentfill}%
\pgfsetfillopacity{0.700000}%
\pgfsetlinewidth{0.501875pt}%
\definecolor{currentstroke}{rgb}{1.000000,1.000000,1.000000}%
\pgfsetstrokecolor{currentstroke}%
\pgfsetstrokeopacity{0.500000}%
\pgfsetdash{}{0pt}%
\pgfpathmoveto{\pgfqpoint{3.768482in}{1.390667in}}%
\pgfpathlineto{\pgfqpoint{3.779858in}{1.395966in}}%
\pgfpathlineto{\pgfqpoint{3.791256in}{1.402638in}}%
\pgfpathlineto{\pgfqpoint{3.802679in}{1.410816in}}%
\pgfpathlineto{\pgfqpoint{3.814127in}{1.420415in}}%
\pgfpathlineto{\pgfqpoint{3.825599in}{1.431334in}}%
\pgfpathlineto{\pgfqpoint{3.819184in}{1.448275in}}%
\pgfpathlineto{\pgfqpoint{3.812781in}{1.465809in}}%
\pgfpathlineto{\pgfqpoint{3.806391in}{1.483862in}}%
\pgfpathlineto{\pgfqpoint{3.800010in}{1.502359in}}%
\pgfpathlineto{\pgfqpoint{3.793637in}{1.521227in}}%
\pgfpathlineto{\pgfqpoint{3.782173in}{1.510521in}}%
\pgfpathlineto{\pgfqpoint{3.770731in}{1.501123in}}%
\pgfpathlineto{\pgfqpoint{3.759313in}{1.493164in}}%
\pgfpathlineto{\pgfqpoint{3.747920in}{1.486750in}}%
\pgfpathlineto{\pgfqpoint{3.736546in}{1.481707in}}%
\pgfpathlineto{\pgfqpoint{3.742922in}{1.462981in}}%
\pgfpathlineto{\pgfqpoint{3.749300in}{1.444380in}}%
\pgfpathlineto{\pgfqpoint{3.755685in}{1.426038in}}%
\pgfpathlineto{\pgfqpoint{3.762078in}{1.408088in}}%
\pgfpathclose%
\pgfusepath{stroke,fill}%
\end{pgfscope}%
\begin{pgfscope}%
\pgfpathrectangle{\pgfqpoint{0.887500in}{0.275000in}}{\pgfqpoint{4.225000in}{4.225000in}}%
\pgfusepath{clip}%
\pgfsetbuttcap%
\pgfsetroundjoin%
\definecolor{currentfill}{rgb}{0.183898,0.422383,0.556944}%
\pgfsetfillcolor{currentfill}%
\pgfsetfillopacity{0.700000}%
\pgfsetlinewidth{0.501875pt}%
\definecolor{currentstroke}{rgb}{1.000000,1.000000,1.000000}%
\pgfsetstrokecolor{currentstroke}%
\pgfsetstrokeopacity{0.500000}%
\pgfsetdash{}{0pt}%
\pgfpathmoveto{\pgfqpoint{2.313958in}{1.933369in}}%
\pgfpathlineto{\pgfqpoint{2.325672in}{1.937180in}}%
\pgfpathlineto{\pgfqpoint{2.337380in}{1.940991in}}%
\pgfpathlineto{\pgfqpoint{2.349083in}{1.944804in}}%
\pgfpathlineto{\pgfqpoint{2.360779in}{1.948621in}}%
\pgfpathlineto{\pgfqpoint{2.372470in}{1.952444in}}%
\pgfpathlineto{\pgfqpoint{2.366364in}{1.962302in}}%
\pgfpathlineto{\pgfqpoint{2.360261in}{1.972115in}}%
\pgfpathlineto{\pgfqpoint{2.354163in}{1.981884in}}%
\pgfpathlineto{\pgfqpoint{2.348070in}{1.991610in}}%
\pgfpathlineto{\pgfqpoint{2.341981in}{2.001294in}}%
\pgfpathlineto{\pgfqpoint{2.330300in}{1.997534in}}%
\pgfpathlineto{\pgfqpoint{2.318613in}{1.993780in}}%
\pgfpathlineto{\pgfqpoint{2.306920in}{1.990031in}}%
\pgfpathlineto{\pgfqpoint{2.295221in}{1.986284in}}%
\pgfpathlineto{\pgfqpoint{2.283517in}{1.982536in}}%
\pgfpathlineto{\pgfqpoint{2.289596in}{1.972788in}}%
\pgfpathlineto{\pgfqpoint{2.295680in}{1.962997in}}%
\pgfpathlineto{\pgfqpoint{2.301768in}{1.953164in}}%
\pgfpathlineto{\pgfqpoint{2.307861in}{1.943288in}}%
\pgfpathclose%
\pgfusepath{stroke,fill}%
\end{pgfscope}%
\begin{pgfscope}%
\pgfpathrectangle{\pgfqpoint{0.887500in}{0.275000in}}{\pgfqpoint{4.225000in}{4.225000in}}%
\pgfusepath{clip}%
\pgfsetbuttcap%
\pgfsetroundjoin%
\definecolor{currentfill}{rgb}{0.255645,0.260703,0.528312}%
\pgfsetfillcolor{currentfill}%
\pgfsetfillopacity{0.700000}%
\pgfsetlinewidth{0.501875pt}%
\definecolor{currentstroke}{rgb}{1.000000,1.000000,1.000000}%
\pgfsetstrokecolor{currentstroke}%
\pgfsetstrokeopacity{0.500000}%
\pgfsetdash{}{0pt}%
\pgfpathmoveto{\pgfqpoint{3.941193in}{1.582049in}}%
\pgfpathlineto{\pgfqpoint{3.952735in}{1.596149in}}%
\pgfpathlineto{\pgfqpoint{3.964254in}{1.609220in}}%
\pgfpathlineto{\pgfqpoint{3.975753in}{1.621435in}}%
\pgfpathlineto{\pgfqpoint{3.987234in}{1.632969in}}%
\pgfpathlineto{\pgfqpoint{3.998702in}{1.644000in}}%
\pgfpathlineto{\pgfqpoint{3.992283in}{1.661785in}}%
\pgfpathlineto{\pgfqpoint{3.985858in}{1.679263in}}%
\pgfpathlineto{\pgfqpoint{3.979429in}{1.696492in}}%
\pgfpathlineto{\pgfqpoint{3.972996in}{1.713531in}}%
\pgfpathlineto{\pgfqpoint{3.966563in}{1.730441in}}%
\pgfpathlineto{\pgfqpoint{3.955092in}{1.719098in}}%
\pgfpathlineto{\pgfqpoint{3.943607in}{1.707173in}}%
\pgfpathlineto{\pgfqpoint{3.932106in}{1.694592in}}%
\pgfpathlineto{\pgfqpoint{3.920589in}{1.681281in}}%
\pgfpathlineto{\pgfqpoint{3.909053in}{1.667166in}}%
\pgfpathlineto{\pgfqpoint{3.915481in}{1.650274in}}%
\pgfpathlineto{\pgfqpoint{3.921910in}{1.633380in}}%
\pgfpathlineto{\pgfqpoint{3.928339in}{1.616420in}}%
\pgfpathlineto{\pgfqpoint{3.934767in}{1.599331in}}%
\pgfpathclose%
\pgfusepath{stroke,fill}%
\end{pgfscope}%
\begin{pgfscope}%
\pgfpathrectangle{\pgfqpoint{0.887500in}{0.275000in}}{\pgfqpoint{4.225000in}{4.225000in}}%
\pgfusepath{clip}%
\pgfsetbuttcap%
\pgfsetroundjoin%
\definecolor{currentfill}{rgb}{0.270595,0.214069,0.507052}%
\pgfsetfillcolor{currentfill}%
\pgfsetfillopacity{0.700000}%
\pgfsetlinewidth{0.501875pt}%
\definecolor{currentstroke}{rgb}{1.000000,1.000000,1.000000}%
\pgfsetstrokecolor{currentstroke}%
\pgfsetstrokeopacity{0.500000}%
\pgfsetdash{}{0pt}%
\pgfpathmoveto{\pgfqpoint{4.119496in}{1.490838in}}%
\pgfpathlineto{\pgfqpoint{4.130941in}{1.501337in}}%
\pgfpathlineto{\pgfqpoint{4.142380in}{1.511722in}}%
\pgfpathlineto{\pgfqpoint{4.153816in}{1.522012in}}%
\pgfpathlineto{\pgfqpoint{4.165247in}{1.532225in}}%
\pgfpathlineto{\pgfqpoint{4.176675in}{1.542385in}}%
\pgfpathlineto{\pgfqpoint{4.170473in}{1.568281in}}%
\pgfpathlineto{\pgfqpoint{4.164229in}{1.592754in}}%
\pgfpathlineto{\pgfqpoint{4.157946in}{1.615896in}}%
\pgfpathlineto{\pgfqpoint{4.151628in}{1.637799in}}%
\pgfpathlineto{\pgfqpoint{4.145277in}{1.658557in}}%
\pgfpathlineto{\pgfqpoint{4.133822in}{1.647608in}}%
\pgfpathlineto{\pgfqpoint{4.122362in}{1.636496in}}%
\pgfpathlineto{\pgfqpoint{4.110897in}{1.625282in}}%
\pgfpathlineto{\pgfqpoint{4.099430in}{1.614023in}}%
\pgfpathlineto{\pgfqpoint{4.087962in}{1.602778in}}%
\pgfpathlineto{\pgfqpoint{4.094313in}{1.582037in}}%
\pgfpathlineto{\pgfqpoint{4.100645in}{1.560531in}}%
\pgfpathlineto{\pgfqpoint{4.106953in}{1.538201in}}%
\pgfpathlineto{\pgfqpoint{4.113238in}{1.514989in}}%
\pgfpathclose%
\pgfusepath{stroke,fill}%
\end{pgfscope}%
\begin{pgfscope}%
\pgfpathrectangle{\pgfqpoint{0.887500in}{0.275000in}}{\pgfqpoint{4.225000in}{4.225000in}}%
\pgfusepath{clip}%
\pgfsetbuttcap%
\pgfsetroundjoin%
\definecolor{currentfill}{rgb}{0.169646,0.456262,0.558030}%
\pgfsetfillcolor{currentfill}%
\pgfsetfillopacity{0.700000}%
\pgfsetlinewidth{0.501875pt}%
\definecolor{currentstroke}{rgb}{1.000000,1.000000,1.000000}%
\pgfsetstrokecolor{currentstroke}%
\pgfsetstrokeopacity{0.500000}%
\pgfsetdash{}{0pt}%
\pgfpathmoveto{\pgfqpoint{1.988118in}{2.003506in}}%
\pgfpathlineto{\pgfqpoint{1.999914in}{2.007278in}}%
\pgfpathlineto{\pgfqpoint{2.011705in}{2.011048in}}%
\pgfpathlineto{\pgfqpoint{2.023490in}{2.014818in}}%
\pgfpathlineto{\pgfqpoint{2.035268in}{2.018590in}}%
\pgfpathlineto{\pgfqpoint{2.047042in}{2.022362in}}%
\pgfpathlineto{\pgfqpoint{2.041042in}{2.031865in}}%
\pgfpathlineto{\pgfqpoint{2.035048in}{2.041334in}}%
\pgfpathlineto{\pgfqpoint{2.029058in}{2.050769in}}%
\pgfpathlineto{\pgfqpoint{2.023072in}{2.060171in}}%
\pgfpathlineto{\pgfqpoint{2.017091in}{2.069539in}}%
\pgfpathlineto{\pgfqpoint{2.005328in}{2.065827in}}%
\pgfpathlineto{\pgfqpoint{1.993560in}{2.062116in}}%
\pgfpathlineto{\pgfqpoint{1.981785in}{2.058405in}}%
\pgfpathlineto{\pgfqpoint{1.970005in}{2.054693in}}%
\pgfpathlineto{\pgfqpoint{1.958219in}{2.050979in}}%
\pgfpathlineto{\pgfqpoint{1.964190in}{2.041551in}}%
\pgfpathlineto{\pgfqpoint{1.970165in}{2.032090in}}%
\pgfpathlineto{\pgfqpoint{1.976145in}{2.022595in}}%
\pgfpathlineto{\pgfqpoint{1.982129in}{2.013068in}}%
\pgfpathclose%
\pgfusepath{stroke,fill}%
\end{pgfscope}%
\begin{pgfscope}%
\pgfpathrectangle{\pgfqpoint{0.887500in}{0.275000in}}{\pgfqpoint{4.225000in}{4.225000in}}%
\pgfusepath{clip}%
\pgfsetbuttcap%
\pgfsetroundjoin%
\definecolor{currentfill}{rgb}{0.274128,0.199721,0.498911}%
\pgfsetfillcolor{currentfill}%
\pgfsetfillopacity{0.700000}%
\pgfsetlinewidth{0.501875pt}%
\definecolor{currentstroke}{rgb}{1.000000,1.000000,1.000000}%
\pgfsetstrokecolor{currentstroke}%
\pgfsetstrokeopacity{0.500000}%
\pgfsetdash{}{0pt}%
\pgfpathmoveto{\pgfqpoint{3.590389in}{1.505172in}}%
\pgfpathlineto{\pgfqpoint{3.601842in}{1.512666in}}%
\pgfpathlineto{\pgfqpoint{3.613292in}{1.520159in}}%
\pgfpathlineto{\pgfqpoint{3.624736in}{1.527518in}}%
\pgfpathlineto{\pgfqpoint{3.636173in}{1.534606in}}%
\pgfpathlineto{\pgfqpoint{3.647599in}{1.541291in}}%
\pgfpathlineto{\pgfqpoint{3.641130in}{1.553447in}}%
\pgfpathlineto{\pgfqpoint{3.634644in}{1.564240in}}%
\pgfpathlineto{\pgfqpoint{3.628146in}{1.573942in}}%
\pgfpathlineto{\pgfqpoint{3.621641in}{1.582893in}}%
\pgfpathlineto{\pgfqpoint{3.615134in}{1.591435in}}%
\pgfpathlineto{\pgfqpoint{3.603693in}{1.582902in}}%
\pgfpathlineto{\pgfqpoint{3.592266in}{1.575570in}}%
\pgfpathlineto{\pgfqpoint{3.580848in}{1.569220in}}%
\pgfpathlineto{\pgfqpoint{3.569437in}{1.563633in}}%
\pgfpathlineto{\pgfqpoint{3.558027in}{1.558591in}}%
\pgfpathlineto{\pgfqpoint{3.564503in}{1.548713in}}%
\pgfpathlineto{\pgfqpoint{3.570979in}{1.538624in}}%
\pgfpathlineto{\pgfqpoint{3.577454in}{1.528130in}}%
\pgfpathlineto{\pgfqpoint{3.583925in}{1.517032in}}%
\pgfpathclose%
\pgfusepath{stroke,fill}%
\end{pgfscope}%
\begin{pgfscope}%
\pgfpathrectangle{\pgfqpoint{0.887500in}{0.275000in}}{\pgfqpoint{4.225000in}{4.225000in}}%
\pgfusepath{clip}%
\pgfsetbuttcap%
\pgfsetroundjoin%
\definecolor{currentfill}{rgb}{0.214298,0.355619,0.551184}%
\pgfsetfillcolor{currentfill}%
\pgfsetfillopacity{0.700000}%
\pgfsetlinewidth{0.501875pt}%
\definecolor{currentstroke}{rgb}{1.000000,1.000000,1.000000}%
\pgfsetstrokecolor{currentstroke}%
\pgfsetstrokeopacity{0.500000}%
\pgfsetdash{}{0pt}%
\pgfpathmoveto{\pgfqpoint{2.818187in}{1.789990in}}%
\pgfpathlineto{\pgfqpoint{2.829779in}{1.793903in}}%
\pgfpathlineto{\pgfqpoint{2.841366in}{1.797823in}}%
\pgfpathlineto{\pgfqpoint{2.852947in}{1.801745in}}%
\pgfpathlineto{\pgfqpoint{2.864522in}{1.805659in}}%
\pgfpathlineto{\pgfqpoint{2.876093in}{1.809553in}}%
\pgfpathlineto{\pgfqpoint{2.869826in}{1.820290in}}%
\pgfpathlineto{\pgfqpoint{2.863563in}{1.830973in}}%
\pgfpathlineto{\pgfqpoint{2.857304in}{1.841604in}}%
\pgfpathlineto{\pgfqpoint{2.851049in}{1.852183in}}%
\pgfpathlineto{\pgfqpoint{2.844797in}{1.862710in}}%
\pgfpathlineto{\pgfqpoint{2.833236in}{1.858869in}}%
\pgfpathlineto{\pgfqpoint{2.821669in}{1.855003in}}%
\pgfpathlineto{\pgfqpoint{2.810097in}{1.851127in}}%
\pgfpathlineto{\pgfqpoint{2.798520in}{1.847257in}}%
\pgfpathlineto{\pgfqpoint{2.786936in}{1.843397in}}%
\pgfpathlineto{\pgfqpoint{2.793178in}{1.832820in}}%
\pgfpathlineto{\pgfqpoint{2.799425in}{1.822191in}}%
\pgfpathlineto{\pgfqpoint{2.805675in}{1.811510in}}%
\pgfpathlineto{\pgfqpoint{2.811929in}{1.800776in}}%
\pgfpathclose%
\pgfusepath{stroke,fill}%
\end{pgfscope}%
\begin{pgfscope}%
\pgfpathrectangle{\pgfqpoint{0.887500in}{0.275000in}}{\pgfqpoint{4.225000in}{4.225000in}}%
\pgfusepath{clip}%
\pgfsetbuttcap%
\pgfsetroundjoin%
\definecolor{currentfill}{rgb}{0.262138,0.242286,0.520837}%
\pgfsetfillcolor{currentfill}%
\pgfsetfillopacity{0.700000}%
\pgfsetlinewidth{0.501875pt}%
\definecolor{currentstroke}{rgb}{1.000000,1.000000,1.000000}%
\pgfsetstrokecolor{currentstroke}%
\pgfsetstrokeopacity{0.500000}%
\pgfsetdash{}{0pt}%
\pgfpathmoveto{\pgfqpoint{4.030651in}{1.548383in}}%
\pgfpathlineto{\pgfqpoint{4.042110in}{1.559020in}}%
\pgfpathlineto{\pgfqpoint{4.053568in}{1.569699in}}%
\pgfpathlineto{\pgfqpoint{4.065030in}{1.580558in}}%
\pgfpathlineto{\pgfqpoint{4.076495in}{1.591603in}}%
\pgfpathlineto{\pgfqpoint{4.087962in}{1.602778in}}%
\pgfpathlineto{\pgfqpoint{4.081593in}{1.622809in}}%
\pgfpathlineto{\pgfqpoint{4.075207in}{1.642191in}}%
\pgfpathlineto{\pgfqpoint{4.068806in}{1.660981in}}%
\pgfpathlineto{\pgfqpoint{4.062393in}{1.679238in}}%
\pgfpathlineto{\pgfqpoint{4.055969in}{1.697020in}}%
\pgfpathlineto{\pgfqpoint{4.044519in}{1.686423in}}%
\pgfpathlineto{\pgfqpoint{4.033067in}{1.675821in}}%
\pgfpathlineto{\pgfqpoint{4.021615in}{1.665257in}}%
\pgfpathlineto{\pgfqpoint{4.010161in}{1.654703in}}%
\pgfpathlineto{\pgfqpoint{3.998702in}{1.644000in}}%
\pgfpathlineto{\pgfqpoint{4.005114in}{1.625847in}}%
\pgfpathlineto{\pgfqpoint{4.011516in}{1.607268in}}%
\pgfpathlineto{\pgfqpoint{4.017908in}{1.588203in}}%
\pgfpathlineto{\pgfqpoint{4.024287in}{1.568594in}}%
\pgfpathclose%
\pgfusepath{stroke,fill}%
\end{pgfscope}%
\begin{pgfscope}%
\pgfpathrectangle{\pgfqpoint{0.887500in}{0.275000in}}{\pgfqpoint{4.225000in}{4.225000in}}%
\pgfusepath{clip}%
\pgfsetbuttcap%
\pgfsetroundjoin%
\definecolor{currentfill}{rgb}{0.188923,0.410910,0.556326}%
\pgfsetfillcolor{currentfill}%
\pgfsetfillopacity{0.700000}%
\pgfsetlinewidth{0.501875pt}%
\definecolor{currentstroke}{rgb}{1.000000,1.000000,1.000000}%
\pgfsetstrokecolor{currentstroke}%
\pgfsetstrokeopacity{0.500000}%
\pgfsetdash{}{0pt}%
\pgfpathmoveto{\pgfqpoint{2.403071in}{1.902463in}}%
\pgfpathlineto{\pgfqpoint{2.414766in}{1.906347in}}%
\pgfpathlineto{\pgfqpoint{2.426455in}{1.910233in}}%
\pgfpathlineto{\pgfqpoint{2.438139in}{1.914116in}}%
\pgfpathlineto{\pgfqpoint{2.449817in}{1.917993in}}%
\pgfpathlineto{\pgfqpoint{2.461489in}{1.921860in}}%
\pgfpathlineto{\pgfqpoint{2.455351in}{1.931897in}}%
\pgfpathlineto{\pgfqpoint{2.449216in}{1.941885in}}%
\pgfpathlineto{\pgfqpoint{2.443086in}{1.951826in}}%
\pgfpathlineto{\pgfqpoint{2.436961in}{1.961719in}}%
\pgfpathlineto{\pgfqpoint{2.430839in}{1.971565in}}%
\pgfpathlineto{\pgfqpoint{2.419177in}{1.967752in}}%
\pgfpathlineto{\pgfqpoint{2.407509in}{1.963930in}}%
\pgfpathlineto{\pgfqpoint{2.395835in}{1.960102in}}%
\pgfpathlineto{\pgfqpoint{2.384156in}{1.956272in}}%
\pgfpathlineto{\pgfqpoint{2.372470in}{1.952444in}}%
\pgfpathlineto{\pgfqpoint{2.378582in}{1.942541in}}%
\pgfpathlineto{\pgfqpoint{2.384697in}{1.932591in}}%
\pgfpathlineto{\pgfqpoint{2.390818in}{1.922594in}}%
\pgfpathlineto{\pgfqpoint{2.396942in}{1.912552in}}%
\pgfpathclose%
\pgfusepath{stroke,fill}%
\end{pgfscope}%
\begin{pgfscope}%
\pgfpathrectangle{\pgfqpoint{0.887500in}{0.275000in}}{\pgfqpoint{4.225000in}{4.225000in}}%
\pgfusepath{clip}%
\pgfsetbuttcap%
\pgfsetroundjoin%
\definecolor{currentfill}{rgb}{0.283187,0.125848,0.444960}%
\pgfsetfillcolor{currentfill}%
\pgfsetfillopacity{0.700000}%
\pgfsetlinewidth{0.501875pt}%
\definecolor{currentstroke}{rgb}{1.000000,1.000000,1.000000}%
\pgfsetstrokecolor{currentstroke}%
\pgfsetstrokeopacity{0.500000}%
\pgfsetdash{}{0pt}%
\pgfpathmoveto{\pgfqpoint{3.857943in}{1.358123in}}%
\pgfpathlineto{\pgfqpoint{3.869425in}{1.369173in}}%
\pgfpathlineto{\pgfqpoint{3.880922in}{1.381004in}}%
\pgfpathlineto{\pgfqpoint{3.892435in}{1.393570in}}%
\pgfpathlineto{\pgfqpoint{3.903962in}{1.406819in}}%
\pgfpathlineto{\pgfqpoint{3.915501in}{1.420623in}}%
\pgfpathlineto{\pgfqpoint{3.909042in}{1.436370in}}%
\pgfpathlineto{\pgfqpoint{3.902591in}{1.452389in}}%
\pgfpathlineto{\pgfqpoint{3.896147in}{1.468676in}}%
\pgfpathlineto{\pgfqpoint{3.889712in}{1.485226in}}%
\pgfpathlineto{\pgfqpoint{3.883283in}{1.502034in}}%
\pgfpathlineto{\pgfqpoint{3.871709in}{1.486151in}}%
\pgfpathlineto{\pgfqpoint{3.860151in}{1.470981in}}%
\pgfpathlineto{\pgfqpoint{3.848612in}{1.456718in}}%
\pgfpathlineto{\pgfqpoint{3.837095in}{1.443469in}}%
\pgfpathlineto{\pgfqpoint{3.825599in}{1.431334in}}%
\pgfpathlineto{\pgfqpoint{3.832030in}{1.415060in}}%
\pgfpathlineto{\pgfqpoint{3.838478in}{1.399528in}}%
\pgfpathlineto{\pgfqpoint{3.844945in}{1.384812in}}%
\pgfpathlineto{\pgfqpoint{3.851432in}{1.370985in}}%
\pgfpathclose%
\pgfusepath{stroke,fill}%
\end{pgfscope}%
\begin{pgfscope}%
\pgfpathrectangle{\pgfqpoint{0.887500in}{0.275000in}}{\pgfqpoint{4.225000in}{4.225000in}}%
\pgfusepath{clip}%
\pgfsetbuttcap%
\pgfsetroundjoin%
\definecolor{currentfill}{rgb}{0.221989,0.339161,0.548752}%
\pgfsetfillcolor{currentfill}%
\pgfsetfillopacity{0.700000}%
\pgfsetlinewidth{0.501875pt}%
\definecolor{currentstroke}{rgb}{1.000000,1.000000,1.000000}%
\pgfsetstrokecolor{currentstroke}%
\pgfsetstrokeopacity{0.500000}%
\pgfsetdash{}{0pt}%
\pgfpathmoveto{\pgfqpoint{2.907483in}{1.755050in}}%
\pgfpathlineto{\pgfqpoint{2.919056in}{1.758966in}}%
\pgfpathlineto{\pgfqpoint{2.930624in}{1.762854in}}%
\pgfpathlineto{\pgfqpoint{2.942186in}{1.766707in}}%
\pgfpathlineto{\pgfqpoint{2.953743in}{1.770517in}}%
\pgfpathlineto{\pgfqpoint{2.965294in}{1.774285in}}%
\pgfpathlineto{\pgfqpoint{2.959000in}{1.785210in}}%
\pgfpathlineto{\pgfqpoint{2.952710in}{1.796074in}}%
\pgfpathlineto{\pgfqpoint{2.946423in}{1.806878in}}%
\pgfpathlineto{\pgfqpoint{2.940140in}{1.817623in}}%
\pgfpathlineto{\pgfqpoint{2.933861in}{1.828313in}}%
\pgfpathlineto{\pgfqpoint{2.922318in}{1.824684in}}%
\pgfpathlineto{\pgfqpoint{2.910770in}{1.820992in}}%
\pgfpathlineto{\pgfqpoint{2.899216in}{1.817232in}}%
\pgfpathlineto{\pgfqpoint{2.887657in}{1.813415in}}%
\pgfpathlineto{\pgfqpoint{2.876093in}{1.809553in}}%
\pgfpathlineto{\pgfqpoint{2.882363in}{1.798763in}}%
\pgfpathlineto{\pgfqpoint{2.888637in}{1.787918in}}%
\pgfpathlineto{\pgfqpoint{2.894916in}{1.777018in}}%
\pgfpathlineto{\pgfqpoint{2.901197in}{1.766062in}}%
\pgfpathclose%
\pgfusepath{stroke,fill}%
\end{pgfscope}%
\begin{pgfscope}%
\pgfpathrectangle{\pgfqpoint{0.887500in}{0.275000in}}{\pgfqpoint{4.225000in}{4.225000in}}%
\pgfusepath{clip}%
\pgfsetbuttcap%
\pgfsetroundjoin%
\definecolor{currentfill}{rgb}{0.278012,0.180367,0.486697}%
\pgfsetfillcolor{currentfill}%
\pgfsetfillopacity{0.700000}%
\pgfsetlinewidth{0.501875pt}%
\definecolor{currentstroke}{rgb}{1.000000,1.000000,1.000000}%
\pgfsetstrokecolor{currentstroke}%
\pgfsetstrokeopacity{0.500000}%
\pgfsetdash{}{0pt}%
\pgfpathmoveto{\pgfqpoint{3.679733in}{1.464266in}}%
\pgfpathlineto{\pgfqpoint{3.691110in}{1.467940in}}%
\pgfpathlineto{\pgfqpoint{3.702474in}{1.471154in}}%
\pgfpathlineto{\pgfqpoint{3.713830in}{1.474275in}}%
\pgfpathlineto{\pgfqpoint{3.725186in}{1.477671in}}%
\pgfpathlineto{\pgfqpoint{3.736546in}{1.481707in}}%
\pgfpathlineto{\pgfqpoint{3.730172in}{1.500421in}}%
\pgfpathlineto{\pgfqpoint{3.723796in}{1.518988in}}%
\pgfpathlineto{\pgfqpoint{3.717416in}{1.537274in}}%
\pgfpathlineto{\pgfqpoint{3.711031in}{1.555141in}}%
\pgfpathlineto{\pgfqpoint{3.704638in}{1.572456in}}%
\pgfpathlineto{\pgfqpoint{3.693231in}{1.565785in}}%
\pgfpathlineto{\pgfqpoint{3.681828in}{1.559561in}}%
\pgfpathlineto{\pgfqpoint{3.670424in}{1.553556in}}%
\pgfpathlineto{\pgfqpoint{3.659016in}{1.547542in}}%
\pgfpathlineto{\pgfqpoint{3.647599in}{1.541291in}}%
\pgfpathlineto{\pgfqpoint{3.654052in}{1.527837in}}%
\pgfpathlineto{\pgfqpoint{3.660489in}{1.513246in}}%
\pgfpathlineto{\pgfqpoint{3.666914in}{1.497680in}}%
\pgfpathlineto{\pgfqpoint{3.673328in}{1.481299in}}%
\pgfpathclose%
\pgfusepath{stroke,fill}%
\end{pgfscope}%
\begin{pgfscope}%
\pgfpathrectangle{\pgfqpoint{0.887500in}{0.275000in}}{\pgfqpoint{4.225000in}{4.225000in}}%
\pgfusepath{clip}%
\pgfsetbuttcap%
\pgfsetroundjoin%
\definecolor{currentfill}{rgb}{0.277134,0.185228,0.489898}%
\pgfsetfillcolor{currentfill}%
\pgfsetfillopacity{0.700000}%
\pgfsetlinewidth{0.501875pt}%
\definecolor{currentstroke}{rgb}{1.000000,1.000000,1.000000}%
\pgfsetstrokecolor{currentstroke}%
\pgfsetstrokeopacity{0.500000}%
\pgfsetdash{}{0pt}%
\pgfpathmoveto{\pgfqpoint{4.062194in}{1.436238in}}%
\pgfpathlineto{\pgfqpoint{4.073663in}{1.447362in}}%
\pgfpathlineto{\pgfqpoint{4.085129in}{1.458463in}}%
\pgfpathlineto{\pgfqpoint{4.096591in}{1.469419in}}%
\pgfpathlineto{\pgfqpoint{4.108046in}{1.480205in}}%
\pgfpathlineto{\pgfqpoint{4.119496in}{1.490838in}}%
\pgfpathlineto{\pgfqpoint{4.113238in}{1.514989in}}%
\pgfpathlineto{\pgfqpoint{4.106953in}{1.538201in}}%
\pgfpathlineto{\pgfqpoint{4.100645in}{1.560531in}}%
\pgfpathlineto{\pgfqpoint{4.094313in}{1.582037in}}%
\pgfpathlineto{\pgfqpoint{4.087962in}{1.602778in}}%
\pgfpathlineto{\pgfqpoint{4.076495in}{1.591603in}}%
\pgfpathlineto{\pgfqpoint{4.065030in}{1.580558in}}%
\pgfpathlineto{\pgfqpoint{4.053568in}{1.569699in}}%
\pgfpathlineto{\pgfqpoint{4.042110in}{1.559020in}}%
\pgfpathlineto{\pgfqpoint{4.030651in}{1.548383in}}%
\pgfpathlineto{\pgfqpoint{4.037000in}{1.527510in}}%
\pgfpathlineto{\pgfqpoint{4.043330in}{1.505917in}}%
\pgfpathlineto{\pgfqpoint{4.049640in}{1.483546in}}%
\pgfpathlineto{\pgfqpoint{4.055929in}{1.460339in}}%
\pgfpathclose%
\pgfusepath{stroke,fill}%
\end{pgfscope}%
\begin{pgfscope}%
\pgfpathrectangle{\pgfqpoint{0.887500in}{0.275000in}}{\pgfqpoint{4.225000in}{4.225000in}}%
\pgfusepath{clip}%
\pgfsetbuttcap%
\pgfsetroundjoin%
\definecolor{currentfill}{rgb}{0.279574,0.170599,0.479997}%
\pgfsetfillcolor{currentfill}%
\pgfsetfillopacity{0.700000}%
\pgfsetlinewidth{0.501875pt}%
\definecolor{currentstroke}{rgb}{1.000000,1.000000,1.000000}%
\pgfsetstrokecolor{currentstroke}%
\pgfsetstrokeopacity{0.500000}%
\pgfsetdash{}{0pt}%
\pgfpathmoveto{\pgfqpoint{3.915501in}{1.420623in}}%
\pgfpathlineto{\pgfqpoint{3.927048in}{1.434772in}}%
\pgfpathlineto{\pgfqpoint{3.938599in}{1.449054in}}%
\pgfpathlineto{\pgfqpoint{3.950149in}{1.463254in}}%
\pgfpathlineto{\pgfqpoint{3.961692in}{1.477158in}}%
\pgfpathlineto{\pgfqpoint{3.973224in}{1.490554in}}%
\pgfpathlineto{\pgfqpoint{3.966836in}{1.509740in}}%
\pgfpathlineto{\pgfqpoint{3.960437in}{1.528421in}}%
\pgfpathlineto{\pgfqpoint{3.954029in}{1.546657in}}%
\pgfpathlineto{\pgfqpoint{3.947614in}{1.564512in}}%
\pgfpathlineto{\pgfqpoint{3.941193in}{1.582049in}}%
\pgfpathlineto{\pgfqpoint{3.929628in}{1.566938in}}%
\pgfpathlineto{\pgfqpoint{3.918048in}{1.551089in}}%
\pgfpathlineto{\pgfqpoint{3.906459in}{1.534794in}}%
\pgfpathlineto{\pgfqpoint{3.894869in}{1.518345in}}%
\pgfpathlineto{\pgfqpoint{3.883283in}{1.502034in}}%
\pgfpathlineto{\pgfqpoint{3.889712in}{1.485226in}}%
\pgfpathlineto{\pgfqpoint{3.896147in}{1.468676in}}%
\pgfpathlineto{\pgfqpoint{3.902591in}{1.452389in}}%
\pgfpathlineto{\pgfqpoint{3.909042in}{1.436370in}}%
\pgfpathclose%
\pgfusepath{stroke,fill}%
\end{pgfscope}%
\begin{pgfscope}%
\pgfpathrectangle{\pgfqpoint{0.887500in}{0.275000in}}{\pgfqpoint{4.225000in}{4.225000in}}%
\pgfusepath{clip}%
\pgfsetbuttcap%
\pgfsetroundjoin%
\definecolor{currentfill}{rgb}{0.270595,0.214069,0.507052}%
\pgfsetfillcolor{currentfill}%
\pgfsetfillopacity{0.700000}%
\pgfsetlinewidth{0.501875pt}%
\definecolor{currentstroke}{rgb}{1.000000,1.000000,1.000000}%
\pgfsetstrokecolor{currentstroke}%
\pgfsetstrokeopacity{0.500000}%
\pgfsetdash{}{0pt}%
\pgfpathmoveto{\pgfqpoint{3.973224in}{1.490554in}}%
\pgfpathlineto{\pgfqpoint{3.984738in}{1.503239in}}%
\pgfpathlineto{\pgfqpoint{3.996235in}{1.515217in}}%
\pgfpathlineto{\pgfqpoint{4.007717in}{1.526632in}}%
\pgfpathlineto{\pgfqpoint{4.019188in}{1.537637in}}%
\pgfpathlineto{\pgfqpoint{4.030651in}{1.548383in}}%
\pgfpathlineto{\pgfqpoint{4.024287in}{1.568594in}}%
\pgfpathlineto{\pgfqpoint{4.017908in}{1.588203in}}%
\pgfpathlineto{\pgfqpoint{4.011516in}{1.607268in}}%
\pgfpathlineto{\pgfqpoint{4.005114in}{1.625847in}}%
\pgfpathlineto{\pgfqpoint{3.998702in}{1.644000in}}%
\pgfpathlineto{\pgfqpoint{3.987234in}{1.632969in}}%
\pgfpathlineto{\pgfqpoint{3.975753in}{1.621435in}}%
\pgfpathlineto{\pgfqpoint{3.964254in}{1.609220in}}%
\pgfpathlineto{\pgfqpoint{3.952735in}{1.596149in}}%
\pgfpathlineto{\pgfqpoint{3.941193in}{1.582049in}}%
\pgfpathlineto{\pgfqpoint{3.947614in}{1.564512in}}%
\pgfpathlineto{\pgfqpoint{3.954029in}{1.546657in}}%
\pgfpathlineto{\pgfqpoint{3.960437in}{1.528421in}}%
\pgfpathlineto{\pgfqpoint{3.966836in}{1.509740in}}%
\pgfpathclose%
\pgfusepath{stroke,fill}%
\end{pgfscope}%
\begin{pgfscope}%
\pgfpathrectangle{\pgfqpoint{0.887500in}{0.275000in}}{\pgfqpoint{4.225000in}{4.225000in}}%
\pgfusepath{clip}%
\pgfsetbuttcap%
\pgfsetroundjoin%
\definecolor{currentfill}{rgb}{0.174274,0.445044,0.557792}%
\pgfsetfillcolor{currentfill}%
\pgfsetfillopacity{0.700000}%
\pgfsetlinewidth{0.501875pt}%
\definecolor{currentstroke}{rgb}{1.000000,1.000000,1.000000}%
\pgfsetstrokecolor{currentstroke}%
\pgfsetstrokeopacity{0.500000}%
\pgfsetdash{}{0pt}%
\pgfpathmoveto{\pgfqpoint{2.077106in}{1.974303in}}%
\pgfpathlineto{\pgfqpoint{2.088883in}{1.978130in}}%
\pgfpathlineto{\pgfqpoint{2.100655in}{1.981953in}}%
\pgfpathlineto{\pgfqpoint{2.112422in}{1.985770in}}%
\pgfpathlineto{\pgfqpoint{2.124182in}{1.989578in}}%
\pgfpathlineto{\pgfqpoint{2.135938in}{1.993375in}}%
\pgfpathlineto{\pgfqpoint{2.129906in}{2.003005in}}%
\pgfpathlineto{\pgfqpoint{2.123878in}{2.012597in}}%
\pgfpathlineto{\pgfqpoint{2.117855in}{2.022150in}}%
\pgfpathlineto{\pgfqpoint{2.111837in}{2.031666in}}%
\pgfpathlineto{\pgfqpoint{2.105823in}{2.041146in}}%
\pgfpathlineto{\pgfqpoint{2.094077in}{2.037408in}}%
\pgfpathlineto{\pgfqpoint{2.082327in}{2.033658in}}%
\pgfpathlineto{\pgfqpoint{2.070571in}{2.029898in}}%
\pgfpathlineto{\pgfqpoint{2.058809in}{2.026132in}}%
\pgfpathlineto{\pgfqpoint{2.047042in}{2.022362in}}%
\pgfpathlineto{\pgfqpoint{2.053045in}{2.012823in}}%
\pgfpathlineto{\pgfqpoint{2.059054in}{2.003248in}}%
\pgfpathlineto{\pgfqpoint{2.065066in}{1.993637in}}%
\pgfpathlineto{\pgfqpoint{2.071084in}{1.983989in}}%
\pgfpathclose%
\pgfusepath{stroke,fill}%
\end{pgfscope}%
\begin{pgfscope}%
\pgfpathrectangle{\pgfqpoint{0.887500in}{0.275000in}}{\pgfqpoint{4.225000in}{4.225000in}}%
\pgfusepath{clip}%
\pgfsetbuttcap%
\pgfsetroundjoin%
\definecolor{currentfill}{rgb}{0.231674,0.318106,0.544834}%
\pgfsetfillcolor{currentfill}%
\pgfsetfillopacity{0.700000}%
\pgfsetlinewidth{0.501875pt}%
\definecolor{currentstroke}{rgb}{1.000000,1.000000,1.000000}%
\pgfsetstrokecolor{currentstroke}%
\pgfsetstrokeopacity{0.500000}%
\pgfsetdash{}{0pt}%
\pgfpathmoveto{\pgfqpoint{2.996819in}{1.718681in}}%
\pgfpathlineto{\pgfqpoint{3.008372in}{1.722541in}}%
\pgfpathlineto{\pgfqpoint{3.019920in}{1.726378in}}%
\pgfpathlineto{\pgfqpoint{3.031463in}{1.730192in}}%
\pgfpathlineto{\pgfqpoint{3.042999in}{1.733979in}}%
\pgfpathlineto{\pgfqpoint{3.054531in}{1.737738in}}%
\pgfpathlineto{\pgfqpoint{3.048210in}{1.748880in}}%
\pgfpathlineto{\pgfqpoint{3.041894in}{1.759961in}}%
\pgfpathlineto{\pgfqpoint{3.035580in}{1.770980in}}%
\pgfpathlineto{\pgfqpoint{3.029271in}{1.781939in}}%
\pgfpathlineto{\pgfqpoint{3.022965in}{1.792840in}}%
\pgfpathlineto{\pgfqpoint{3.011442in}{1.789132in}}%
\pgfpathlineto{\pgfqpoint{2.999913in}{1.785434in}}%
\pgfpathlineto{\pgfqpoint{2.988379in}{1.781733in}}%
\pgfpathlineto{\pgfqpoint{2.976839in}{1.778020in}}%
\pgfpathlineto{\pgfqpoint{2.965294in}{1.774285in}}%
\pgfpathlineto{\pgfqpoint{2.971592in}{1.763296in}}%
\pgfpathlineto{\pgfqpoint{2.977893in}{1.752242in}}%
\pgfpathlineto{\pgfqpoint{2.984198in}{1.741121in}}%
\pgfpathlineto{\pgfqpoint{2.990506in}{1.729933in}}%
\pgfpathclose%
\pgfusepath{stroke,fill}%
\end{pgfscope}%
\begin{pgfscope}%
\pgfpathrectangle{\pgfqpoint{0.887500in}{0.275000in}}{\pgfqpoint{4.225000in}{4.225000in}}%
\pgfusepath{clip}%
\pgfsetbuttcap%
\pgfsetroundjoin%
\definecolor{currentfill}{rgb}{0.195860,0.395433,0.555276}%
\pgfsetfillcolor{currentfill}%
\pgfsetfillopacity{0.700000}%
\pgfsetlinewidth{0.501875pt}%
\definecolor{currentstroke}{rgb}{1.000000,1.000000,1.000000}%
\pgfsetstrokecolor{currentstroke}%
\pgfsetstrokeopacity{0.500000}%
\pgfsetdash{}{0pt}%
\pgfpathmoveto{\pgfqpoint{2.492248in}{1.870972in}}%
\pgfpathlineto{\pgfqpoint{2.503925in}{1.874878in}}%
\pgfpathlineto{\pgfqpoint{2.515596in}{1.878769in}}%
\pgfpathlineto{\pgfqpoint{2.527261in}{1.882644in}}%
\pgfpathlineto{\pgfqpoint{2.538922in}{1.886501in}}%
\pgfpathlineto{\pgfqpoint{2.550577in}{1.890341in}}%
\pgfpathlineto{\pgfqpoint{2.544407in}{1.900559in}}%
\pgfpathlineto{\pgfqpoint{2.538241in}{1.910729in}}%
\pgfpathlineto{\pgfqpoint{2.532080in}{1.920853in}}%
\pgfpathlineto{\pgfqpoint{2.525923in}{1.930929in}}%
\pgfpathlineto{\pgfqpoint{2.519770in}{1.940959in}}%
\pgfpathlineto{\pgfqpoint{2.508125in}{1.937175in}}%
\pgfpathlineto{\pgfqpoint{2.496474in}{1.933373in}}%
\pgfpathlineto{\pgfqpoint{2.484818in}{1.929553in}}%
\pgfpathlineto{\pgfqpoint{2.473156in}{1.925715in}}%
\pgfpathlineto{\pgfqpoint{2.461489in}{1.921860in}}%
\pgfpathlineto{\pgfqpoint{2.467632in}{1.911777in}}%
\pgfpathlineto{\pgfqpoint{2.473780in}{1.901646in}}%
\pgfpathlineto{\pgfqpoint{2.479932in}{1.891468in}}%
\pgfpathlineto{\pgfqpoint{2.486088in}{1.881243in}}%
\pgfpathclose%
\pgfusepath{stroke,fill}%
\end{pgfscope}%
\begin{pgfscope}%
\pgfpathrectangle{\pgfqpoint{0.887500in}{0.275000in}}{\pgfqpoint{4.225000in}{4.225000in}}%
\pgfusepath{clip}%
\pgfsetbuttcap%
\pgfsetroundjoin%
\definecolor{currentfill}{rgb}{0.281924,0.089666,0.412415}%
\pgfsetfillcolor{currentfill}%
\pgfsetfillopacity{0.700000}%
\pgfsetlinewidth{0.501875pt}%
\definecolor{currentstroke}{rgb}{1.000000,1.000000,1.000000}%
\pgfsetstrokecolor{currentstroke}%
\pgfsetstrokeopacity{0.500000}%
\pgfsetdash{}{0pt}%
\pgfpathmoveto{\pgfqpoint{3.800770in}{1.316172in}}%
\pgfpathlineto{\pgfqpoint{3.812174in}{1.322702in}}%
\pgfpathlineto{\pgfqpoint{3.823592in}{1.330148in}}%
\pgfpathlineto{\pgfqpoint{3.835026in}{1.338562in}}%
\pgfpathlineto{\pgfqpoint{3.846477in}{1.347903in}}%
\pgfpathlineto{\pgfqpoint{3.857943in}{1.358123in}}%
\pgfpathlineto{\pgfqpoint{3.851432in}{1.370985in}}%
\pgfpathlineto{\pgfqpoint{3.844945in}{1.384812in}}%
\pgfpathlineto{\pgfqpoint{3.838478in}{1.399528in}}%
\pgfpathlineto{\pgfqpoint{3.832030in}{1.415060in}}%
\pgfpathlineto{\pgfqpoint{3.825599in}{1.431334in}}%
\pgfpathlineto{\pgfqpoint{3.814127in}{1.420415in}}%
\pgfpathlineto{\pgfqpoint{3.802679in}{1.410816in}}%
\pgfpathlineto{\pgfqpoint{3.791256in}{1.402638in}}%
\pgfpathlineto{\pgfqpoint{3.779858in}{1.395966in}}%
\pgfpathlineto{\pgfqpoint{3.768482in}{1.390667in}}%
\pgfpathlineto{\pgfqpoint{3.774900in}{1.373907in}}%
\pgfpathlineto{\pgfqpoint{3.781335in}{1.357944in}}%
\pgfpathlineto{\pgfqpoint{3.787790in}{1.342911in}}%
\pgfpathlineto{\pgfqpoint{3.794267in}{1.328942in}}%
\pgfpathclose%
\pgfusepath{stroke,fill}%
\end{pgfscope}%
\begin{pgfscope}%
\pgfpathrectangle{\pgfqpoint{0.887500in}{0.275000in}}{\pgfqpoint{4.225000in}{4.225000in}}%
\pgfusepath{clip}%
\pgfsetbuttcap%
\pgfsetroundjoin%
\definecolor{currentfill}{rgb}{0.239346,0.300855,0.540844}%
\pgfsetfillcolor{currentfill}%
\pgfsetfillopacity{0.700000}%
\pgfsetlinewidth{0.501875pt}%
\definecolor{currentstroke}{rgb}{1.000000,1.000000,1.000000}%
\pgfsetstrokecolor{currentstroke}%
\pgfsetstrokeopacity{0.500000}%
\pgfsetdash{}{0pt}%
\pgfpathmoveto{\pgfqpoint{3.086184in}{1.681061in}}%
\pgfpathlineto{\pgfqpoint{3.097717in}{1.684883in}}%
\pgfpathlineto{\pgfqpoint{3.109245in}{1.688685in}}%
\pgfpathlineto{\pgfqpoint{3.120767in}{1.692489in}}%
\pgfpathlineto{\pgfqpoint{3.132284in}{1.696316in}}%
\pgfpathlineto{\pgfqpoint{3.143794in}{1.700186in}}%
\pgfpathlineto{\pgfqpoint{3.137449in}{1.711573in}}%
\pgfpathlineto{\pgfqpoint{3.131107in}{1.722886in}}%
\pgfpathlineto{\pgfqpoint{3.124768in}{1.734132in}}%
\pgfpathlineto{\pgfqpoint{3.118433in}{1.745319in}}%
\pgfpathlineto{\pgfqpoint{3.112101in}{1.756454in}}%
\pgfpathlineto{\pgfqpoint{3.100598in}{1.752647in}}%
\pgfpathlineto{\pgfqpoint{3.089090in}{1.748897in}}%
\pgfpathlineto{\pgfqpoint{3.077576in}{1.745179in}}%
\pgfpathlineto{\pgfqpoint{3.066056in}{1.741467in}}%
\pgfpathlineto{\pgfqpoint{3.054531in}{1.737738in}}%
\pgfpathlineto{\pgfqpoint{3.060854in}{1.726532in}}%
\pgfpathlineto{\pgfqpoint{3.067182in}{1.715263in}}%
\pgfpathlineto{\pgfqpoint{3.073512in}{1.703928in}}%
\pgfpathlineto{\pgfqpoint{3.079846in}{1.692528in}}%
\pgfpathclose%
\pgfusepath{stroke,fill}%
\end{pgfscope}%
\begin{pgfscope}%
\pgfpathrectangle{\pgfqpoint{0.887500in}{0.275000in}}{\pgfqpoint{4.225000in}{4.225000in}}%
\pgfusepath{clip}%
\pgfsetbuttcap%
\pgfsetroundjoin%
\definecolor{currentfill}{rgb}{0.281887,0.150881,0.465405}%
\pgfsetfillcolor{currentfill}%
\pgfsetfillopacity{0.700000}%
\pgfsetlinewidth{0.501875pt}%
\definecolor{currentstroke}{rgb}{1.000000,1.000000,1.000000}%
\pgfsetstrokecolor{currentstroke}%
\pgfsetstrokeopacity{0.500000}%
\pgfsetdash{}{0pt}%
\pgfpathmoveto{\pgfqpoint{4.004936in}{1.384843in}}%
\pgfpathlineto{\pgfqpoint{4.016367in}{1.394125in}}%
\pgfpathlineto{\pgfqpoint{4.027810in}{1.404035in}}%
\pgfpathlineto{\pgfqpoint{4.039265in}{1.414444in}}%
\pgfpathlineto{\pgfqpoint{4.050727in}{1.425222in}}%
\pgfpathlineto{\pgfqpoint{4.062194in}{1.436238in}}%
\pgfpathlineto{\pgfqpoint{4.055929in}{1.460339in}}%
\pgfpathlineto{\pgfqpoint{4.049640in}{1.483546in}}%
\pgfpathlineto{\pgfqpoint{4.043330in}{1.505917in}}%
\pgfpathlineto{\pgfqpoint{4.037000in}{1.527510in}}%
\pgfpathlineto{\pgfqpoint{4.030651in}{1.548383in}}%
\pgfpathlineto{\pgfqpoint{4.019188in}{1.537637in}}%
\pgfpathlineto{\pgfqpoint{4.007717in}{1.526632in}}%
\pgfpathlineto{\pgfqpoint{3.996235in}{1.515217in}}%
\pgfpathlineto{\pgfqpoint{3.984738in}{1.503239in}}%
\pgfpathlineto{\pgfqpoint{3.973224in}{1.490554in}}%
\pgfpathlineto{\pgfqpoint{3.979599in}{1.470798in}}%
\pgfpathlineto{\pgfqpoint{3.985959in}{1.450411in}}%
\pgfpathlineto{\pgfqpoint{3.992304in}{1.429331in}}%
\pgfpathlineto{\pgfqpoint{3.998630in}{1.407496in}}%
\pgfpathclose%
\pgfusepath{stroke,fill}%
\end{pgfscope}%
\begin{pgfscope}%
\pgfpathrectangle{\pgfqpoint{0.887500in}{0.275000in}}{\pgfqpoint{4.225000in}{4.225000in}}%
\pgfusepath{clip}%
\pgfsetbuttcap%
\pgfsetroundjoin%
\definecolor{currentfill}{rgb}{0.246811,0.283237,0.535941}%
\pgfsetfillcolor{currentfill}%
\pgfsetfillopacity{0.700000}%
\pgfsetlinewidth{0.501875pt}%
\definecolor{currentstroke}{rgb}{1.000000,1.000000,1.000000}%
\pgfsetstrokecolor{currentstroke}%
\pgfsetstrokeopacity{0.500000}%
\pgfsetdash{}{0pt}%
\pgfpathmoveto{\pgfqpoint{3.175569in}{1.641902in}}%
\pgfpathlineto{\pgfqpoint{3.187082in}{1.645834in}}%
\pgfpathlineto{\pgfqpoint{3.198590in}{1.649818in}}%
\pgfpathlineto{\pgfqpoint{3.210092in}{1.653860in}}%
\pgfpathlineto{\pgfqpoint{3.221589in}{1.657936in}}%
\pgfpathlineto{\pgfqpoint{3.233080in}{1.662005in}}%
\pgfpathlineto{\pgfqpoint{3.226712in}{1.673879in}}%
\pgfpathlineto{\pgfqpoint{3.220347in}{1.685679in}}%
\pgfpathlineto{\pgfqpoint{3.213985in}{1.697400in}}%
\pgfpathlineto{\pgfqpoint{3.207626in}{1.709037in}}%
\pgfpathlineto{\pgfqpoint{3.201269in}{1.720585in}}%
\pgfpathlineto{\pgfqpoint{3.189785in}{1.716439in}}%
\pgfpathlineto{\pgfqpoint{3.178295in}{1.712262in}}%
\pgfpathlineto{\pgfqpoint{3.166800in}{1.708141in}}%
\pgfpathlineto{\pgfqpoint{3.155300in}{1.704121in}}%
\pgfpathlineto{\pgfqpoint{3.143794in}{1.700186in}}%
\pgfpathlineto{\pgfqpoint{3.150143in}{1.688718in}}%
\pgfpathlineto{\pgfqpoint{3.156495in}{1.677161in}}%
\pgfpathlineto{\pgfqpoint{3.162850in}{1.665507in}}%
\pgfpathlineto{\pgfqpoint{3.169208in}{1.653752in}}%
\pgfpathclose%
\pgfusepath{stroke,fill}%
\end{pgfscope}%
\begin{pgfscope}%
\pgfpathrectangle{\pgfqpoint{0.887500in}{0.275000in}}{\pgfqpoint{4.225000in}{4.225000in}}%
\pgfusepath{clip}%
\pgfsetbuttcap%
\pgfsetroundjoin%
\definecolor{currentfill}{rgb}{0.179019,0.433756,0.557430}%
\pgfsetfillcolor{currentfill}%
\pgfsetfillopacity{0.700000}%
\pgfsetlinewidth{0.501875pt}%
\definecolor{currentstroke}{rgb}{1.000000,1.000000,1.000000}%
\pgfsetstrokecolor{currentstroke}%
\pgfsetstrokeopacity{0.500000}%
\pgfsetdash{}{0pt}%
\pgfpathmoveto{\pgfqpoint{2.166167in}{1.944626in}}%
\pgfpathlineto{\pgfqpoint{2.177927in}{1.948467in}}%
\pgfpathlineto{\pgfqpoint{2.189681in}{1.952294in}}%
\pgfpathlineto{\pgfqpoint{2.201430in}{1.956110in}}%
\pgfpathlineto{\pgfqpoint{2.213173in}{1.959915in}}%
\pgfpathlineto{\pgfqpoint{2.224911in}{1.963708in}}%
\pgfpathlineto{\pgfqpoint{2.218846in}{1.973479in}}%
\pgfpathlineto{\pgfqpoint{2.212786in}{1.983209in}}%
\pgfpathlineto{\pgfqpoint{2.206730in}{1.992898in}}%
\pgfpathlineto{\pgfqpoint{2.200679in}{2.002547in}}%
\pgfpathlineto{\pgfqpoint{2.194632in}{2.012155in}}%
\pgfpathlineto{\pgfqpoint{2.182904in}{2.008425in}}%
\pgfpathlineto{\pgfqpoint{2.171171in}{2.004683in}}%
\pgfpathlineto{\pgfqpoint{2.159432in}{2.000928in}}%
\pgfpathlineto{\pgfqpoint{2.147687in}{1.997159in}}%
\pgfpathlineto{\pgfqpoint{2.135938in}{1.993375in}}%
\pgfpathlineto{\pgfqpoint{2.141974in}{1.983706in}}%
\pgfpathlineto{\pgfqpoint{2.148016in}{1.973996in}}%
\pgfpathlineto{\pgfqpoint{2.154061in}{1.964247in}}%
\pgfpathlineto{\pgfqpoint{2.160112in}{1.954457in}}%
\pgfpathclose%
\pgfusepath{stroke,fill}%
\end{pgfscope}%
\begin{pgfscope}%
\pgfpathrectangle{\pgfqpoint{0.887500in}{0.275000in}}{\pgfqpoint{4.225000in}{4.225000in}}%
\pgfusepath{clip}%
\pgfsetbuttcap%
\pgfsetroundjoin%
\definecolor{currentfill}{rgb}{0.201239,0.383670,0.554294}%
\pgfsetfillcolor{currentfill}%
\pgfsetfillopacity{0.700000}%
\pgfsetlinewidth{0.501875pt}%
\definecolor{currentstroke}{rgb}{1.000000,1.000000,1.000000}%
\pgfsetstrokecolor{currentstroke}%
\pgfsetstrokeopacity{0.500000}%
\pgfsetdash{}{0pt}%
\pgfpathmoveto{\pgfqpoint{2.581489in}{1.838554in}}%
\pgfpathlineto{\pgfqpoint{2.593147in}{1.842433in}}%
\pgfpathlineto{\pgfqpoint{2.604800in}{1.846294in}}%
\pgfpathlineto{\pgfqpoint{2.616448in}{1.850134in}}%
\pgfpathlineto{\pgfqpoint{2.628090in}{1.853955in}}%
\pgfpathlineto{\pgfqpoint{2.639727in}{1.857758in}}%
\pgfpathlineto{\pgfqpoint{2.633527in}{1.868151in}}%
\pgfpathlineto{\pgfqpoint{2.627331in}{1.878497in}}%
\pgfpathlineto{\pgfqpoint{2.621140in}{1.888795in}}%
\pgfpathlineto{\pgfqpoint{2.614952in}{1.899047in}}%
\pgfpathlineto{\pgfqpoint{2.608769in}{1.909252in}}%
\pgfpathlineto{\pgfqpoint{2.597141in}{1.905508in}}%
\pgfpathlineto{\pgfqpoint{2.585508in}{1.901746in}}%
\pgfpathlineto{\pgfqpoint{2.573870in}{1.897964in}}%
\pgfpathlineto{\pgfqpoint{2.562226in}{1.894163in}}%
\pgfpathlineto{\pgfqpoint{2.550577in}{1.890341in}}%
\pgfpathlineto{\pgfqpoint{2.556751in}{1.880077in}}%
\pgfpathlineto{\pgfqpoint{2.562929in}{1.869767in}}%
\pgfpathlineto{\pgfqpoint{2.569111in}{1.859410in}}%
\pgfpathlineto{\pgfqpoint{2.575298in}{1.849006in}}%
\pgfpathclose%
\pgfusepath{stroke,fill}%
\end{pgfscope}%
\begin{pgfscope}%
\pgfpathrectangle{\pgfqpoint{0.887500in}{0.275000in}}{\pgfqpoint{4.225000in}{4.225000in}}%
\pgfusepath{clip}%
\pgfsetbuttcap%
\pgfsetroundjoin%
\definecolor{currentfill}{rgb}{0.255645,0.260703,0.528312}%
\pgfsetfillcolor{currentfill}%
\pgfsetfillopacity{0.700000}%
\pgfsetlinewidth{0.501875pt}%
\definecolor{currentstroke}{rgb}{1.000000,1.000000,1.000000}%
\pgfsetstrokecolor{currentstroke}%
\pgfsetstrokeopacity{0.500000}%
\pgfsetdash{}{0pt}%
\pgfpathmoveto{\pgfqpoint{3.264967in}{1.601762in}}%
\pgfpathlineto{\pgfqpoint{3.276460in}{1.605836in}}%
\pgfpathlineto{\pgfqpoint{3.287947in}{1.609866in}}%
\pgfpathlineto{\pgfqpoint{3.299428in}{1.613826in}}%
\pgfpathlineto{\pgfqpoint{3.310903in}{1.617688in}}%
\pgfpathlineto{\pgfqpoint{3.322372in}{1.621433in}}%
\pgfpathlineto{\pgfqpoint{3.315982in}{1.633486in}}%
\pgfpathlineto{\pgfqpoint{3.309594in}{1.645440in}}%
\pgfpathlineto{\pgfqpoint{3.303208in}{1.657298in}}%
\pgfpathlineto{\pgfqpoint{3.296826in}{1.669062in}}%
\pgfpathlineto{\pgfqpoint{3.290446in}{1.680734in}}%
\pgfpathlineto{\pgfqpoint{3.278986in}{1.677337in}}%
\pgfpathlineto{\pgfqpoint{3.267519in}{1.673733in}}%
\pgfpathlineto{\pgfqpoint{3.256046in}{1.669947in}}%
\pgfpathlineto{\pgfqpoint{3.244566in}{1.666023in}}%
\pgfpathlineto{\pgfqpoint{3.233080in}{1.662005in}}%
\pgfpathlineto{\pgfqpoint{3.239452in}{1.650066in}}%
\pgfpathlineto{\pgfqpoint{3.245826in}{1.638065in}}%
\pgfpathlineto{\pgfqpoint{3.252203in}{1.626010in}}%
\pgfpathlineto{\pgfqpoint{3.258584in}{1.613907in}}%
\pgfpathclose%
\pgfusepath{stroke,fill}%
\end{pgfscope}%
\begin{pgfscope}%
\pgfpathrectangle{\pgfqpoint{0.887500in}{0.275000in}}{\pgfqpoint{4.225000in}{4.225000in}}%
\pgfusepath{clip}%
\pgfsetbuttcap%
\pgfsetroundjoin%
\definecolor{currentfill}{rgb}{0.271828,0.209303,0.504434}%
\pgfsetfillcolor{currentfill}%
\pgfsetfillopacity{0.700000}%
\pgfsetlinewidth{0.501875pt}%
\definecolor{currentstroke}{rgb}{1.000000,1.000000,1.000000}%
\pgfsetstrokecolor{currentstroke}%
\pgfsetstrokeopacity{0.500000}%
\pgfsetdash{}{0pt}%
\pgfpathmoveto{\pgfqpoint{3.443718in}{1.512950in}}%
\pgfpathlineto{\pgfqpoint{3.455166in}{1.516962in}}%
\pgfpathlineto{\pgfqpoint{3.466612in}{1.521223in}}%
\pgfpathlineto{\pgfqpoint{3.478054in}{1.525697in}}%
\pgfpathlineto{\pgfqpoint{3.489494in}{1.530328in}}%
\pgfpathlineto{\pgfqpoint{3.500929in}{1.535058in}}%
\pgfpathlineto{\pgfqpoint{3.494498in}{1.547632in}}%
\pgfpathlineto{\pgfqpoint{3.488072in}{1.560240in}}%
\pgfpathlineto{\pgfqpoint{3.481649in}{1.572908in}}%
\pgfpathlineto{\pgfqpoint{3.475230in}{1.585664in}}%
\pgfpathlineto{\pgfqpoint{3.468815in}{1.598532in}}%
\pgfpathlineto{\pgfqpoint{3.457394in}{1.594557in}}%
\pgfpathlineto{\pgfqpoint{3.445966in}{1.590384in}}%
\pgfpathlineto{\pgfqpoint{3.434531in}{1.586149in}}%
\pgfpathlineto{\pgfqpoint{3.423092in}{1.581989in}}%
\pgfpathlineto{\pgfqpoint{3.411649in}{1.578006in}}%
\pgfpathlineto{\pgfqpoint{3.418058in}{1.565134in}}%
\pgfpathlineto{\pgfqpoint{3.424469in}{1.552177in}}%
\pgfpathlineto{\pgfqpoint{3.430883in}{1.539149in}}%
\pgfpathlineto{\pgfqpoint{3.437299in}{1.526068in}}%
\pgfpathclose%
\pgfusepath{stroke,fill}%
\end{pgfscope}%
\begin{pgfscope}%
\pgfpathrectangle{\pgfqpoint{0.887500in}{0.275000in}}{\pgfqpoint{4.225000in}{4.225000in}}%
\pgfusepath{clip}%
\pgfsetbuttcap%
\pgfsetroundjoin%
\definecolor{currentfill}{rgb}{0.283072,0.130895,0.449241}%
\pgfsetfillcolor{currentfill}%
\pgfsetfillopacity{0.700000}%
\pgfsetlinewidth{0.501875pt}%
\definecolor{currentstroke}{rgb}{1.000000,1.000000,1.000000}%
\pgfsetstrokecolor{currentstroke}%
\pgfsetstrokeopacity{0.500000}%
\pgfsetdash{}{0pt}%
\pgfpathmoveto{\pgfqpoint{3.711714in}{1.374945in}}%
\pgfpathlineto{\pgfqpoint{3.723071in}{1.377591in}}%
\pgfpathlineto{\pgfqpoint{3.734421in}{1.380204in}}%
\pgfpathlineto{\pgfqpoint{3.745769in}{1.383067in}}%
\pgfpathlineto{\pgfqpoint{3.757121in}{1.386461in}}%
\pgfpathlineto{\pgfqpoint{3.768482in}{1.390667in}}%
\pgfpathlineto{\pgfqpoint{3.762078in}{1.408088in}}%
\pgfpathlineto{\pgfqpoint{3.755685in}{1.426038in}}%
\pgfpathlineto{\pgfqpoint{3.749300in}{1.444380in}}%
\pgfpathlineto{\pgfqpoint{3.742922in}{1.462981in}}%
\pgfpathlineto{\pgfqpoint{3.736546in}{1.481707in}}%
\pgfpathlineto{\pgfqpoint{3.725186in}{1.477671in}}%
\pgfpathlineto{\pgfqpoint{3.713830in}{1.474275in}}%
\pgfpathlineto{\pgfqpoint{3.702474in}{1.471154in}}%
\pgfpathlineto{\pgfqpoint{3.691110in}{1.467940in}}%
\pgfpathlineto{\pgfqpoint{3.679733in}{1.464266in}}%
\pgfpathlineto{\pgfqpoint{3.686132in}{1.446741in}}%
\pgfpathlineto{\pgfqpoint{3.692527in}{1.428886in}}%
\pgfpathlineto{\pgfqpoint{3.698920in}{1.410861in}}%
\pgfpathlineto{\pgfqpoint{3.705315in}{1.392827in}}%
\pgfpathclose%
\pgfusepath{stroke,fill}%
\end{pgfscope}%
\begin{pgfscope}%
\pgfpathrectangle{\pgfqpoint{0.887500in}{0.275000in}}{\pgfqpoint{4.225000in}{4.225000in}}%
\pgfusepath{clip}%
\pgfsetbuttcap%
\pgfsetroundjoin%
\definecolor{currentfill}{rgb}{0.263663,0.237631,0.518762}%
\pgfsetfillcolor{currentfill}%
\pgfsetfillopacity{0.700000}%
\pgfsetlinewidth{0.501875pt}%
\definecolor{currentstroke}{rgb}{1.000000,1.000000,1.000000}%
\pgfsetstrokecolor{currentstroke}%
\pgfsetstrokeopacity{0.500000}%
\pgfsetdash{}{0pt}%
\pgfpathmoveto{\pgfqpoint{3.354363in}{1.559622in}}%
\pgfpathlineto{\pgfqpoint{3.365831in}{1.563234in}}%
\pgfpathlineto{\pgfqpoint{3.377293in}{1.566836in}}%
\pgfpathlineto{\pgfqpoint{3.388750in}{1.570471in}}%
\pgfpathlineto{\pgfqpoint{3.400202in}{1.574180in}}%
\pgfpathlineto{\pgfqpoint{3.411649in}{1.578006in}}%
\pgfpathlineto{\pgfqpoint{3.405242in}{1.590775in}}%
\pgfpathlineto{\pgfqpoint{3.398837in}{1.603425in}}%
\pgfpathlineto{\pgfqpoint{3.392434in}{1.615943in}}%
\pgfpathlineto{\pgfqpoint{3.386033in}{1.628318in}}%
\pgfpathlineto{\pgfqpoint{3.379634in}{1.640539in}}%
\pgfpathlineto{\pgfqpoint{3.368190in}{1.636424in}}%
\pgfpathlineto{\pgfqpoint{3.356743in}{1.632529in}}%
\pgfpathlineto{\pgfqpoint{3.345291in}{1.628781in}}%
\pgfpathlineto{\pgfqpoint{3.333835in}{1.625107in}}%
\pgfpathlineto{\pgfqpoint{3.322372in}{1.621433in}}%
\pgfpathlineto{\pgfqpoint{3.328765in}{1.609280in}}%
\pgfpathlineto{\pgfqpoint{3.335161in}{1.597024in}}%
\pgfpathlineto{\pgfqpoint{3.341559in}{1.584664in}}%
\pgfpathlineto{\pgfqpoint{3.347960in}{1.572197in}}%
\pgfpathclose%
\pgfusepath{stroke,fill}%
\end{pgfscope}%
\begin{pgfscope}%
\pgfpathrectangle{\pgfqpoint{0.887500in}{0.275000in}}{\pgfqpoint{4.225000in}{4.225000in}}%
\pgfusepath{clip}%
\pgfsetbuttcap%
\pgfsetroundjoin%
\definecolor{currentfill}{rgb}{0.165117,0.467423,0.558141}%
\pgfsetfillcolor{currentfill}%
\pgfsetfillopacity{0.700000}%
\pgfsetlinewidth{0.501875pt}%
\definecolor{currentstroke}{rgb}{1.000000,1.000000,1.000000}%
\pgfsetstrokecolor{currentstroke}%
\pgfsetstrokeopacity{0.500000}%
\pgfsetdash{}{0pt}%
\pgfpathmoveto{\pgfqpoint{1.840048in}{2.013492in}}%
\pgfpathlineto{\pgfqpoint{1.851890in}{2.017286in}}%
\pgfpathlineto{\pgfqpoint{1.863726in}{2.021068in}}%
\pgfpathlineto{\pgfqpoint{1.875557in}{2.024837in}}%
\pgfpathlineto{\pgfqpoint{1.887383in}{2.028595in}}%
\pgfpathlineto{\pgfqpoint{1.899203in}{2.032344in}}%
\pgfpathlineto{\pgfqpoint{1.893247in}{2.041797in}}%
\pgfpathlineto{\pgfqpoint{1.887296in}{2.051219in}}%
\pgfpathlineto{\pgfqpoint{1.881349in}{2.060609in}}%
\pgfpathlineto{\pgfqpoint{1.875407in}{2.069967in}}%
\pgfpathlineto{\pgfqpoint{1.869470in}{2.079294in}}%
\pgfpathlineto{\pgfqpoint{1.857660in}{2.075601in}}%
\pgfpathlineto{\pgfqpoint{1.845845in}{2.071898in}}%
\pgfpathlineto{\pgfqpoint{1.834024in}{2.068184in}}%
\pgfpathlineto{\pgfqpoint{1.822198in}{2.064457in}}%
\pgfpathlineto{\pgfqpoint{1.810366in}{2.060717in}}%
\pgfpathlineto{\pgfqpoint{1.816294in}{2.051335in}}%
\pgfpathlineto{\pgfqpoint{1.822225in}{2.041921in}}%
\pgfpathlineto{\pgfqpoint{1.828161in}{2.032476in}}%
\pgfpathlineto{\pgfqpoint{1.834102in}{2.023000in}}%
\pgfpathclose%
\pgfusepath{stroke,fill}%
\end{pgfscope}%
\begin{pgfscope}%
\pgfpathrectangle{\pgfqpoint{0.887500in}{0.275000in}}{\pgfqpoint{4.225000in}{4.225000in}}%
\pgfusepath{clip}%
\pgfsetbuttcap%
\pgfsetroundjoin%
\definecolor{currentfill}{rgb}{0.283229,0.120777,0.440584}%
\pgfsetfillcolor{currentfill}%
\pgfsetfillopacity{0.700000}%
\pgfsetlinewidth{0.501875pt}%
\definecolor{currentstroke}{rgb}{1.000000,1.000000,1.000000}%
\pgfsetstrokecolor{currentstroke}%
\pgfsetstrokeopacity{0.500000}%
\pgfsetdash{}{0pt}%
\pgfpathmoveto{\pgfqpoint{3.947929in}{1.346141in}}%
\pgfpathlineto{\pgfqpoint{3.959321in}{1.353226in}}%
\pgfpathlineto{\pgfqpoint{3.970715in}{1.360525in}}%
\pgfpathlineto{\pgfqpoint{3.982114in}{1.368152in}}%
\pgfpathlineto{\pgfqpoint{3.993520in}{1.376220in}}%
\pgfpathlineto{\pgfqpoint{4.004936in}{1.384843in}}%
\pgfpathlineto{\pgfqpoint{3.998630in}{1.407496in}}%
\pgfpathlineto{\pgfqpoint{3.992304in}{1.429331in}}%
\pgfpathlineto{\pgfqpoint{3.985959in}{1.450411in}}%
\pgfpathlineto{\pgfqpoint{3.979599in}{1.470798in}}%
\pgfpathlineto{\pgfqpoint{3.973224in}{1.490554in}}%
\pgfpathlineto{\pgfqpoint{3.961692in}{1.477158in}}%
\pgfpathlineto{\pgfqpoint{3.950149in}{1.463254in}}%
\pgfpathlineto{\pgfqpoint{3.938599in}{1.449054in}}%
\pgfpathlineto{\pgfqpoint{3.927048in}{1.434772in}}%
\pgfpathlineto{\pgfqpoint{3.915501in}{1.420623in}}%
\pgfpathlineto{\pgfqpoint{3.921969in}{1.405153in}}%
\pgfpathlineto{\pgfqpoint{3.928445in}{1.389965in}}%
\pgfpathlineto{\pgfqpoint{3.934930in}{1.375064in}}%
\pgfpathlineto{\pgfqpoint{3.941425in}{1.360455in}}%
\pgfpathclose%
\pgfusepath{stroke,fill}%
\end{pgfscope}%
\begin{pgfscope}%
\pgfpathrectangle{\pgfqpoint{0.887500in}{0.275000in}}{\pgfqpoint{4.225000in}{4.225000in}}%
\pgfusepath{clip}%
\pgfsetbuttcap%
\pgfsetroundjoin%
\definecolor{currentfill}{rgb}{0.277134,0.185228,0.489898}%
\pgfsetfillcolor{currentfill}%
\pgfsetfillopacity{0.700000}%
\pgfsetlinewidth{0.501875pt}%
\definecolor{currentstroke}{rgb}{1.000000,1.000000,1.000000}%
\pgfsetstrokecolor{currentstroke}%
\pgfsetstrokeopacity{0.500000}%
\pgfsetdash{}{0pt}%
\pgfpathmoveto{\pgfqpoint{3.533123in}{1.471787in}}%
\pgfpathlineto{\pgfqpoint{3.544575in}{1.477711in}}%
\pgfpathlineto{\pgfqpoint{3.556027in}{1.484018in}}%
\pgfpathlineto{\pgfqpoint{3.567480in}{1.490724in}}%
\pgfpathlineto{\pgfqpoint{3.578935in}{1.497813in}}%
\pgfpathlineto{\pgfqpoint{3.590389in}{1.505172in}}%
\pgfpathlineto{\pgfqpoint{3.583925in}{1.517032in}}%
\pgfpathlineto{\pgfqpoint{3.577454in}{1.528130in}}%
\pgfpathlineto{\pgfqpoint{3.570979in}{1.538624in}}%
\pgfpathlineto{\pgfqpoint{3.564503in}{1.548713in}}%
\pgfpathlineto{\pgfqpoint{3.558027in}{1.558591in}}%
\pgfpathlineto{\pgfqpoint{3.546618in}{1.553875in}}%
\pgfpathlineto{\pgfqpoint{3.535204in}{1.549267in}}%
\pgfpathlineto{\pgfqpoint{3.523785in}{1.544585in}}%
\pgfpathlineto{\pgfqpoint{3.512359in}{1.539829in}}%
\pgfpathlineto{\pgfqpoint{3.500929in}{1.535058in}}%
\pgfpathlineto{\pgfqpoint{3.507362in}{1.522492in}}%
\pgfpathlineto{\pgfqpoint{3.513799in}{1.509908in}}%
\pgfpathlineto{\pgfqpoint{3.520238in}{1.497280in}}%
\pgfpathlineto{\pgfqpoint{3.526680in}{1.484581in}}%
\pgfpathclose%
\pgfusepath{stroke,fill}%
\end{pgfscope}%
\begin{pgfscope}%
\pgfpathrectangle{\pgfqpoint{0.887500in}{0.275000in}}{\pgfqpoint{4.225000in}{4.225000in}}%
\pgfusepath{clip}%
\pgfsetbuttcap%
\pgfsetroundjoin%
\definecolor{currentfill}{rgb}{0.208623,0.367752,0.552675}%
\pgfsetfillcolor{currentfill}%
\pgfsetfillopacity{0.700000}%
\pgfsetlinewidth{0.501875pt}%
\definecolor{currentstroke}{rgb}{1.000000,1.000000,1.000000}%
\pgfsetstrokecolor{currentstroke}%
\pgfsetstrokeopacity{0.500000}%
\pgfsetdash{}{0pt}%
\pgfpathmoveto{\pgfqpoint{2.670788in}{1.805051in}}%
\pgfpathlineto{\pgfqpoint{2.682429in}{1.808897in}}%
\pgfpathlineto{\pgfqpoint{2.694063in}{1.812732in}}%
\pgfpathlineto{\pgfqpoint{2.705692in}{1.816560in}}%
\pgfpathlineto{\pgfqpoint{2.717316in}{1.820385in}}%
\pgfpathlineto{\pgfqpoint{2.728933in}{1.824209in}}%
\pgfpathlineto{\pgfqpoint{2.722704in}{1.834793in}}%
\pgfpathlineto{\pgfqpoint{2.716479in}{1.845326in}}%
\pgfpathlineto{\pgfqpoint{2.710257in}{1.855809in}}%
\pgfpathlineto{\pgfqpoint{2.704040in}{1.866242in}}%
\pgfpathlineto{\pgfqpoint{2.697827in}{1.876626in}}%
\pgfpathlineto{\pgfqpoint{2.686218in}{1.872860in}}%
\pgfpathlineto{\pgfqpoint{2.674604in}{1.869094in}}%
\pgfpathlineto{\pgfqpoint{2.662984in}{1.865324in}}%
\pgfpathlineto{\pgfqpoint{2.651358in}{1.861546in}}%
\pgfpathlineto{\pgfqpoint{2.639727in}{1.857758in}}%
\pgfpathlineto{\pgfqpoint{2.645931in}{1.847315in}}%
\pgfpathlineto{\pgfqpoint{2.652139in}{1.836824in}}%
\pgfpathlineto{\pgfqpoint{2.658352in}{1.826284in}}%
\pgfpathlineto{\pgfqpoint{2.664568in}{1.815693in}}%
\pgfpathclose%
\pgfusepath{stroke,fill}%
\end{pgfscope}%
\begin{pgfscope}%
\pgfpathrectangle{\pgfqpoint{0.887500in}{0.275000in}}{\pgfqpoint{4.225000in}{4.225000in}}%
\pgfusepath{clip}%
\pgfsetbuttcap%
\pgfsetroundjoin%
\definecolor{currentfill}{rgb}{0.183898,0.422383,0.556944}%
\pgfsetfillcolor{currentfill}%
\pgfsetfillopacity{0.700000}%
\pgfsetlinewidth{0.501875pt}%
\definecolor{currentstroke}{rgb}{1.000000,1.000000,1.000000}%
\pgfsetstrokecolor{currentstroke}%
\pgfsetstrokeopacity{0.500000}%
\pgfsetdash{}{0pt}%
\pgfpathmoveto{\pgfqpoint{2.255303in}{1.914225in}}%
\pgfpathlineto{\pgfqpoint{2.267045in}{1.918071in}}%
\pgfpathlineto{\pgfqpoint{2.278782in}{1.921907in}}%
\pgfpathlineto{\pgfqpoint{2.290513in}{1.925735in}}%
\pgfpathlineto{\pgfqpoint{2.302238in}{1.929554in}}%
\pgfpathlineto{\pgfqpoint{2.313958in}{1.933369in}}%
\pgfpathlineto{\pgfqpoint{2.307861in}{1.943288in}}%
\pgfpathlineto{\pgfqpoint{2.301768in}{1.953164in}}%
\pgfpathlineto{\pgfqpoint{2.295680in}{1.962997in}}%
\pgfpathlineto{\pgfqpoint{2.289596in}{1.972788in}}%
\pgfpathlineto{\pgfqpoint{2.283517in}{1.982536in}}%
\pgfpathlineto{\pgfqpoint{2.271807in}{1.978785in}}%
\pgfpathlineto{\pgfqpoint{2.260092in}{1.975029in}}%
\pgfpathlineto{\pgfqpoint{2.248370in}{1.971265in}}%
\pgfpathlineto{\pgfqpoint{2.236644in}{1.967492in}}%
\pgfpathlineto{\pgfqpoint{2.224911in}{1.963708in}}%
\pgfpathlineto{\pgfqpoint{2.230981in}{1.953896in}}%
\pgfpathlineto{\pgfqpoint{2.237054in}{1.944041in}}%
\pgfpathlineto{\pgfqpoint{2.243133in}{1.934145in}}%
\pgfpathlineto{\pgfqpoint{2.249216in}{1.924207in}}%
\pgfpathclose%
\pgfusepath{stroke,fill}%
\end{pgfscope}%
\begin{pgfscope}%
\pgfpathrectangle{\pgfqpoint{0.887500in}{0.275000in}}{\pgfqpoint{4.225000in}{4.225000in}}%
\pgfusepath{clip}%
\pgfsetbuttcap%
\pgfsetroundjoin%
\definecolor{currentfill}{rgb}{0.216210,0.351535,0.550627}%
\pgfsetfillcolor{currentfill}%
\pgfsetfillopacity{0.700000}%
\pgfsetlinewidth{0.501875pt}%
\definecolor{currentstroke}{rgb}{1.000000,1.000000,1.000000}%
\pgfsetstrokecolor{currentstroke}%
\pgfsetstrokeopacity{0.500000}%
\pgfsetdash{}{0pt}%
\pgfpathmoveto{\pgfqpoint{2.760140in}{1.770507in}}%
\pgfpathlineto{\pgfqpoint{2.771761in}{1.774394in}}%
\pgfpathlineto{\pgfqpoint{2.783376in}{1.778285in}}%
\pgfpathlineto{\pgfqpoint{2.794985in}{1.782182in}}%
\pgfpathlineto{\pgfqpoint{2.806589in}{1.786083in}}%
\pgfpathlineto{\pgfqpoint{2.818187in}{1.789990in}}%
\pgfpathlineto{\pgfqpoint{2.811929in}{1.800776in}}%
\pgfpathlineto{\pgfqpoint{2.805675in}{1.811510in}}%
\pgfpathlineto{\pgfqpoint{2.799425in}{1.822191in}}%
\pgfpathlineto{\pgfqpoint{2.793178in}{1.832820in}}%
\pgfpathlineto{\pgfqpoint{2.786936in}{1.843397in}}%
\pgfpathlineto{\pgfqpoint{2.775347in}{1.839546in}}%
\pgfpathlineto{\pgfqpoint{2.763752in}{1.835703in}}%
\pgfpathlineto{\pgfqpoint{2.752152in}{1.831867in}}%
\pgfpathlineto{\pgfqpoint{2.740545in}{1.828036in}}%
\pgfpathlineto{\pgfqpoint{2.728933in}{1.824209in}}%
\pgfpathlineto{\pgfqpoint{2.735167in}{1.813573in}}%
\pgfpathlineto{\pgfqpoint{2.741404in}{1.802885in}}%
\pgfpathlineto{\pgfqpoint{2.747645in}{1.792145in}}%
\pgfpathlineto{\pgfqpoint{2.753891in}{1.781352in}}%
\pgfpathclose%
\pgfusepath{stroke,fill}%
\end{pgfscope}%
\begin{pgfscope}%
\pgfpathrectangle{\pgfqpoint{0.887500in}{0.275000in}}{\pgfqpoint{4.225000in}{4.225000in}}%
\pgfusepath{clip}%
\pgfsetbuttcap%
\pgfsetroundjoin%
\definecolor{currentfill}{rgb}{0.281924,0.089666,0.412415}%
\pgfsetfillcolor{currentfill}%
\pgfsetfillopacity{0.700000}%
\pgfsetlinewidth{0.501875pt}%
\definecolor{currentstroke}{rgb}{1.000000,1.000000,1.000000}%
\pgfsetstrokecolor{currentstroke}%
\pgfsetstrokeopacity{0.500000}%
\pgfsetdash{}{0pt}%
\pgfpathmoveto{\pgfqpoint{3.890914in}{1.310882in}}%
\pgfpathlineto{\pgfqpoint{3.902326in}{1.318018in}}%
\pgfpathlineto{\pgfqpoint{3.913734in}{1.325117in}}%
\pgfpathlineto{\pgfqpoint{3.925137in}{1.332167in}}%
\pgfpathlineto{\pgfqpoint{3.936535in}{1.339159in}}%
\pgfpathlineto{\pgfqpoint{3.947929in}{1.346141in}}%
\pgfpathlineto{\pgfqpoint{3.941425in}{1.360455in}}%
\pgfpathlineto{\pgfqpoint{3.934930in}{1.375064in}}%
\pgfpathlineto{\pgfqpoint{3.928445in}{1.389965in}}%
\pgfpathlineto{\pgfqpoint{3.921969in}{1.405153in}}%
\pgfpathlineto{\pgfqpoint{3.915501in}{1.420623in}}%
\pgfpathlineto{\pgfqpoint{3.903962in}{1.406819in}}%
\pgfpathlineto{\pgfqpoint{3.892435in}{1.393570in}}%
\pgfpathlineto{\pgfqpoint{3.880922in}{1.381004in}}%
\pgfpathlineto{\pgfqpoint{3.869425in}{1.369173in}}%
\pgfpathlineto{\pgfqpoint{3.857943in}{1.358123in}}%
\pgfpathlineto{\pgfqpoint{3.864478in}{1.346299in}}%
\pgfpathlineto{\pgfqpoint{3.871041in}{1.335588in}}%
\pgfpathlineto{\pgfqpoint{3.877633in}{1.326065in}}%
\pgfpathlineto{\pgfqpoint{3.884257in}{1.317804in}}%
\pgfpathclose%
\pgfusepath{stroke,fill}%
\end{pgfscope}%
\begin{pgfscope}%
\pgfpathrectangle{\pgfqpoint{0.887500in}{0.275000in}}{\pgfqpoint{4.225000in}{4.225000in}}%
\pgfusepath{clip}%
\pgfsetbuttcap%
\pgfsetroundjoin%
\definecolor{currentfill}{rgb}{0.169646,0.456262,0.558030}%
\pgfsetfillcolor{currentfill}%
\pgfsetfillopacity{0.700000}%
\pgfsetlinewidth{0.501875pt}%
\definecolor{currentstroke}{rgb}{1.000000,1.000000,1.000000}%
\pgfsetstrokecolor{currentstroke}%
\pgfsetstrokeopacity{0.500000}%
\pgfsetdash{}{0pt}%
\pgfpathmoveto{\pgfqpoint{1.929050in}{1.984588in}}%
\pgfpathlineto{\pgfqpoint{1.940875in}{1.988383in}}%
\pgfpathlineto{\pgfqpoint{1.952694in}{1.992171in}}%
\pgfpathlineto{\pgfqpoint{1.964508in}{1.995954in}}%
\pgfpathlineto{\pgfqpoint{1.976316in}{1.999731in}}%
\pgfpathlineto{\pgfqpoint{1.988118in}{2.003506in}}%
\pgfpathlineto{\pgfqpoint{1.982129in}{2.013068in}}%
\pgfpathlineto{\pgfqpoint{1.976145in}{2.022595in}}%
\pgfpathlineto{\pgfqpoint{1.970165in}{2.032090in}}%
\pgfpathlineto{\pgfqpoint{1.964190in}{2.041551in}}%
\pgfpathlineto{\pgfqpoint{1.958219in}{2.050979in}}%
\pgfpathlineto{\pgfqpoint{1.946427in}{2.047263in}}%
\pgfpathlineto{\pgfqpoint{1.934629in}{2.043542in}}%
\pgfpathlineto{\pgfqpoint{1.922826in}{2.039816in}}%
\pgfpathlineto{\pgfqpoint{1.911017in}{2.036083in}}%
\pgfpathlineto{\pgfqpoint{1.899203in}{2.032344in}}%
\pgfpathlineto{\pgfqpoint{1.905163in}{2.022858in}}%
\pgfpathlineto{\pgfqpoint{1.911128in}{2.013340in}}%
\pgfpathlineto{\pgfqpoint{1.917098in}{2.003789in}}%
\pgfpathlineto{\pgfqpoint{1.923072in}{1.994206in}}%
\pgfpathclose%
\pgfusepath{stroke,fill}%
\end{pgfscope}%
\begin{pgfscope}%
\pgfpathrectangle{\pgfqpoint{0.887500in}{0.275000in}}{\pgfqpoint{4.225000in}{4.225000in}}%
\pgfusepath{clip}%
\pgfsetbuttcap%
\pgfsetroundjoin%
\definecolor{currentfill}{rgb}{0.279574,0.170599,0.479997}%
\pgfsetfillcolor{currentfill}%
\pgfsetfillopacity{0.700000}%
\pgfsetlinewidth{0.501875pt}%
\definecolor{currentstroke}{rgb}{1.000000,1.000000,1.000000}%
\pgfsetstrokecolor{currentstroke}%
\pgfsetstrokeopacity{0.500000}%
\pgfsetdash{}{0pt}%
\pgfpathmoveto{\pgfqpoint{3.622618in}{1.436137in}}%
\pgfpathlineto{\pgfqpoint{3.634062in}{1.442577in}}%
\pgfpathlineto{\pgfqpoint{3.645499in}{1.448788in}}%
\pgfpathlineto{\pgfqpoint{3.656925in}{1.454591in}}%
\pgfpathlineto{\pgfqpoint{3.668338in}{1.459809in}}%
\pgfpathlineto{\pgfqpoint{3.679733in}{1.464266in}}%
\pgfpathlineto{\pgfqpoint{3.673328in}{1.481299in}}%
\pgfpathlineto{\pgfqpoint{3.666914in}{1.497680in}}%
\pgfpathlineto{\pgfqpoint{3.660489in}{1.513246in}}%
\pgfpathlineto{\pgfqpoint{3.654052in}{1.527837in}}%
\pgfpathlineto{\pgfqpoint{3.647599in}{1.541291in}}%
\pgfpathlineto{\pgfqpoint{3.636173in}{1.534606in}}%
\pgfpathlineto{\pgfqpoint{3.624736in}{1.527518in}}%
\pgfpathlineto{\pgfqpoint{3.613292in}{1.520159in}}%
\pgfpathlineto{\pgfqpoint{3.601842in}{1.512666in}}%
\pgfpathlineto{\pgfqpoint{3.590389in}{1.505172in}}%
\pgfpathlineto{\pgfqpoint{3.596846in}{1.492570in}}%
\pgfpathlineto{\pgfqpoint{3.603297in}{1.479294in}}%
\pgfpathlineto{\pgfqpoint{3.609742in}{1.465416in}}%
\pgfpathlineto{\pgfqpoint{3.616182in}{1.451007in}}%
\pgfpathclose%
\pgfusepath{stroke,fill}%
\end{pgfscope}%
\begin{pgfscope}%
\pgfpathrectangle{\pgfqpoint{0.887500in}{0.275000in}}{\pgfqpoint{4.225000in}{4.225000in}}%
\pgfusepath{clip}%
\pgfsetbuttcap%
\pgfsetroundjoin%
\definecolor{currentfill}{rgb}{0.190631,0.407061,0.556089}%
\pgfsetfillcolor{currentfill}%
\pgfsetfillopacity{0.700000}%
\pgfsetlinewidth{0.501875pt}%
\definecolor{currentstroke}{rgb}{1.000000,1.000000,1.000000}%
\pgfsetstrokecolor{currentstroke}%
\pgfsetstrokeopacity{0.500000}%
\pgfsetdash{}{0pt}%
\pgfpathmoveto{\pgfqpoint{2.344510in}{1.883102in}}%
\pgfpathlineto{\pgfqpoint{2.356234in}{1.886971in}}%
\pgfpathlineto{\pgfqpoint{2.367952in}{1.890839in}}%
\pgfpathlineto{\pgfqpoint{2.379664in}{1.894709in}}%
\pgfpathlineto{\pgfqpoint{2.391371in}{1.898583in}}%
\pgfpathlineto{\pgfqpoint{2.403071in}{1.902463in}}%
\pgfpathlineto{\pgfqpoint{2.396942in}{1.912552in}}%
\pgfpathlineto{\pgfqpoint{2.390818in}{1.922594in}}%
\pgfpathlineto{\pgfqpoint{2.384697in}{1.932591in}}%
\pgfpathlineto{\pgfqpoint{2.378582in}{1.942541in}}%
\pgfpathlineto{\pgfqpoint{2.372470in}{1.952444in}}%
\pgfpathlineto{\pgfqpoint{2.360779in}{1.948621in}}%
\pgfpathlineto{\pgfqpoint{2.349083in}{1.944804in}}%
\pgfpathlineto{\pgfqpoint{2.337380in}{1.940991in}}%
\pgfpathlineto{\pgfqpoint{2.325672in}{1.937180in}}%
\pgfpathlineto{\pgfqpoint{2.313958in}{1.933369in}}%
\pgfpathlineto{\pgfqpoint{2.320059in}{1.923405in}}%
\pgfpathlineto{\pgfqpoint{2.326165in}{1.913396in}}%
\pgfpathlineto{\pgfqpoint{2.332276in}{1.903343in}}%
\pgfpathlineto{\pgfqpoint{2.338391in}{1.893245in}}%
\pgfpathclose%
\pgfusepath{stroke,fill}%
\end{pgfscope}%
\begin{pgfscope}%
\pgfpathrectangle{\pgfqpoint{0.887500in}{0.275000in}}{\pgfqpoint{4.225000in}{4.225000in}}%
\pgfusepath{clip}%
\pgfsetbuttcap%
\pgfsetroundjoin%
\definecolor{currentfill}{rgb}{0.280894,0.078907,0.402329}%
\pgfsetfillcolor{currentfill}%
\pgfsetfillopacity{0.700000}%
\pgfsetlinewidth{0.501875pt}%
\definecolor{currentstroke}{rgb}{1.000000,1.000000,1.000000}%
\pgfsetstrokecolor{currentstroke}%
\pgfsetstrokeopacity{0.500000}%
\pgfsetdash{}{0pt}%
\pgfpathmoveto{\pgfqpoint{3.743869in}{1.293406in}}%
\pgfpathlineto{\pgfqpoint{3.755242in}{1.297014in}}%
\pgfpathlineto{\pgfqpoint{3.766617in}{1.300982in}}%
\pgfpathlineto{\pgfqpoint{3.777994in}{1.305424in}}%
\pgfpathlineto{\pgfqpoint{3.789378in}{1.310451in}}%
\pgfpathlineto{\pgfqpoint{3.800770in}{1.316172in}}%
\pgfpathlineto{\pgfqpoint{3.794267in}{1.328942in}}%
\pgfpathlineto{\pgfqpoint{3.787790in}{1.342911in}}%
\pgfpathlineto{\pgfqpoint{3.781335in}{1.357944in}}%
\pgfpathlineto{\pgfqpoint{3.774900in}{1.373907in}}%
\pgfpathlineto{\pgfqpoint{3.768482in}{1.390667in}}%
\pgfpathlineto{\pgfqpoint{3.757121in}{1.386461in}}%
\pgfpathlineto{\pgfqpoint{3.745769in}{1.383067in}}%
\pgfpathlineto{\pgfqpoint{3.734421in}{1.380204in}}%
\pgfpathlineto{\pgfqpoint{3.723071in}{1.377591in}}%
\pgfpathlineto{\pgfqpoint{3.711714in}{1.374945in}}%
\pgfpathlineto{\pgfqpoint{3.718119in}{1.357375in}}%
\pgfpathlineto{\pgfqpoint{3.724534in}{1.340276in}}%
\pgfpathlineto{\pgfqpoint{3.730962in}{1.323808in}}%
\pgfpathlineto{\pgfqpoint{3.737406in}{1.308132in}}%
\pgfpathclose%
\pgfusepath{stroke,fill}%
\end{pgfscope}%
\begin{pgfscope}%
\pgfpathrectangle{\pgfqpoint{0.887500in}{0.275000in}}{\pgfqpoint{4.225000in}{4.225000in}}%
\pgfusepath{clip}%
\pgfsetbuttcap%
\pgfsetroundjoin%
\definecolor{currentfill}{rgb}{0.223925,0.334994,0.548053}%
\pgfsetfillcolor{currentfill}%
\pgfsetfillopacity{0.700000}%
\pgfsetlinewidth{0.501875pt}%
\definecolor{currentstroke}{rgb}{1.000000,1.000000,1.000000}%
\pgfsetstrokecolor{currentstroke}%
\pgfsetstrokeopacity{0.500000}%
\pgfsetdash{}{0pt}%
\pgfpathmoveto{\pgfqpoint{2.849534in}{1.735278in}}%
\pgfpathlineto{\pgfqpoint{2.861135in}{1.739238in}}%
\pgfpathlineto{\pgfqpoint{2.872730in}{1.743199in}}%
\pgfpathlineto{\pgfqpoint{2.884320in}{1.747159in}}%
\pgfpathlineto{\pgfqpoint{2.895904in}{1.751112in}}%
\pgfpathlineto{\pgfqpoint{2.907483in}{1.755050in}}%
\pgfpathlineto{\pgfqpoint{2.901197in}{1.766062in}}%
\pgfpathlineto{\pgfqpoint{2.894916in}{1.777018in}}%
\pgfpathlineto{\pgfqpoint{2.888637in}{1.787918in}}%
\pgfpathlineto{\pgfqpoint{2.882363in}{1.798763in}}%
\pgfpathlineto{\pgfqpoint{2.876093in}{1.809553in}}%
\pgfpathlineto{\pgfqpoint{2.864522in}{1.805659in}}%
\pgfpathlineto{\pgfqpoint{2.852947in}{1.801745in}}%
\pgfpathlineto{\pgfqpoint{2.841366in}{1.797823in}}%
\pgfpathlineto{\pgfqpoint{2.829779in}{1.793903in}}%
\pgfpathlineto{\pgfqpoint{2.818187in}{1.789990in}}%
\pgfpathlineto{\pgfqpoint{2.824448in}{1.779152in}}%
\pgfpathlineto{\pgfqpoint{2.830714in}{1.768261in}}%
\pgfpathlineto{\pgfqpoint{2.836984in}{1.757319in}}%
\pgfpathlineto{\pgfqpoint{2.843257in}{1.746324in}}%
\pgfpathclose%
\pgfusepath{stroke,fill}%
\end{pgfscope}%
\begin{pgfscope}%
\pgfpathrectangle{\pgfqpoint{0.887500in}{0.275000in}}{\pgfqpoint{4.225000in}{4.225000in}}%
\pgfusepath{clip}%
\pgfsetbuttcap%
\pgfsetroundjoin%
\definecolor{currentfill}{rgb}{0.174274,0.445044,0.557792}%
\pgfsetfillcolor{currentfill}%
\pgfsetfillopacity{0.700000}%
\pgfsetlinewidth{0.501875pt}%
\definecolor{currentstroke}{rgb}{1.000000,1.000000,1.000000}%
\pgfsetstrokecolor{currentstroke}%
\pgfsetstrokeopacity{0.500000}%
\pgfsetdash{}{0pt}%
\pgfpathmoveto{\pgfqpoint{2.018131in}{1.955170in}}%
\pgfpathlineto{\pgfqpoint{2.029937in}{1.958997in}}%
\pgfpathlineto{\pgfqpoint{2.041738in}{1.962822in}}%
\pgfpathlineto{\pgfqpoint{2.053533in}{1.966648in}}%
\pgfpathlineto{\pgfqpoint{2.065322in}{1.970475in}}%
\pgfpathlineto{\pgfqpoint{2.077106in}{1.974303in}}%
\pgfpathlineto{\pgfqpoint{2.071084in}{1.983989in}}%
\pgfpathlineto{\pgfqpoint{2.065066in}{1.993637in}}%
\pgfpathlineto{\pgfqpoint{2.059054in}{2.003248in}}%
\pgfpathlineto{\pgfqpoint{2.053045in}{2.012823in}}%
\pgfpathlineto{\pgfqpoint{2.047042in}{2.022362in}}%
\pgfpathlineto{\pgfqpoint{2.035268in}{2.018590in}}%
\pgfpathlineto{\pgfqpoint{2.023490in}{2.014818in}}%
\pgfpathlineto{\pgfqpoint{2.011705in}{2.011048in}}%
\pgfpathlineto{\pgfqpoint{1.999914in}{2.007278in}}%
\pgfpathlineto{\pgfqpoint{1.988118in}{2.003506in}}%
\pgfpathlineto{\pgfqpoint{1.994111in}{1.993909in}}%
\pgfpathlineto{\pgfqpoint{2.000109in}{1.984278in}}%
\pgfpathlineto{\pgfqpoint{2.006112in}{1.974611in}}%
\pgfpathlineto{\pgfqpoint{2.012119in}{1.964909in}}%
\pgfpathclose%
\pgfusepath{stroke,fill}%
\end{pgfscope}%
\begin{pgfscope}%
\pgfpathrectangle{\pgfqpoint{0.887500in}{0.275000in}}{\pgfqpoint{4.225000in}{4.225000in}}%
\pgfusepath{clip}%
\pgfsetbuttcap%
\pgfsetroundjoin%
\definecolor{currentfill}{rgb}{0.231674,0.318106,0.544834}%
\pgfsetfillcolor{currentfill}%
\pgfsetfillopacity{0.700000}%
\pgfsetlinewidth{0.501875pt}%
\definecolor{currentstroke}{rgb}{1.000000,1.000000,1.000000}%
\pgfsetstrokecolor{currentstroke}%
\pgfsetstrokeopacity{0.500000}%
\pgfsetdash{}{0pt}%
\pgfpathmoveto{\pgfqpoint{2.938967in}{1.699120in}}%
\pgfpathlineto{\pgfqpoint{2.950548in}{1.703062in}}%
\pgfpathlineto{\pgfqpoint{2.962124in}{1.706991in}}%
\pgfpathlineto{\pgfqpoint{2.973695in}{1.710904in}}%
\pgfpathlineto{\pgfqpoint{2.985259in}{1.714802in}}%
\pgfpathlineto{\pgfqpoint{2.996819in}{1.718681in}}%
\pgfpathlineto{\pgfqpoint{2.990506in}{1.729933in}}%
\pgfpathlineto{\pgfqpoint{2.984198in}{1.741121in}}%
\pgfpathlineto{\pgfqpoint{2.977893in}{1.752242in}}%
\pgfpathlineto{\pgfqpoint{2.971592in}{1.763296in}}%
\pgfpathlineto{\pgfqpoint{2.965294in}{1.774285in}}%
\pgfpathlineto{\pgfqpoint{2.953743in}{1.770517in}}%
\pgfpathlineto{\pgfqpoint{2.942186in}{1.766707in}}%
\pgfpathlineto{\pgfqpoint{2.930624in}{1.762854in}}%
\pgfpathlineto{\pgfqpoint{2.919056in}{1.758966in}}%
\pgfpathlineto{\pgfqpoint{2.907483in}{1.755050in}}%
\pgfpathlineto{\pgfqpoint{2.913772in}{1.743980in}}%
\pgfpathlineto{\pgfqpoint{2.920065in}{1.732852in}}%
\pgfpathlineto{\pgfqpoint{2.926362in}{1.721666in}}%
\pgfpathlineto{\pgfqpoint{2.932663in}{1.710422in}}%
\pgfpathclose%
\pgfusepath{stroke,fill}%
\end{pgfscope}%
\begin{pgfscope}%
\pgfpathrectangle{\pgfqpoint{0.887500in}{0.275000in}}{\pgfqpoint{4.225000in}{4.225000in}}%
\pgfusepath{clip}%
\pgfsetbuttcap%
\pgfsetroundjoin%
\definecolor{currentfill}{rgb}{0.195860,0.395433,0.555276}%
\pgfsetfillcolor{currentfill}%
\pgfsetfillopacity{0.700000}%
\pgfsetlinewidth{0.501875pt}%
\definecolor{currentstroke}{rgb}{1.000000,1.000000,1.000000}%
\pgfsetstrokecolor{currentstroke}%
\pgfsetstrokeopacity{0.500000}%
\pgfsetdash{}{0pt}%
\pgfpathmoveto{\pgfqpoint{2.433782in}{1.851328in}}%
\pgfpathlineto{\pgfqpoint{2.445486in}{1.855261in}}%
\pgfpathlineto{\pgfqpoint{2.457185in}{1.859195in}}%
\pgfpathlineto{\pgfqpoint{2.468878in}{1.863127in}}%
\pgfpathlineto{\pgfqpoint{2.480566in}{1.867054in}}%
\pgfpathlineto{\pgfqpoint{2.492248in}{1.870972in}}%
\pgfpathlineto{\pgfqpoint{2.486088in}{1.881243in}}%
\pgfpathlineto{\pgfqpoint{2.479932in}{1.891468in}}%
\pgfpathlineto{\pgfqpoint{2.473780in}{1.901646in}}%
\pgfpathlineto{\pgfqpoint{2.467632in}{1.911777in}}%
\pgfpathlineto{\pgfqpoint{2.461489in}{1.921860in}}%
\pgfpathlineto{\pgfqpoint{2.449817in}{1.917993in}}%
\pgfpathlineto{\pgfqpoint{2.438139in}{1.914116in}}%
\pgfpathlineto{\pgfqpoint{2.426455in}{1.910233in}}%
\pgfpathlineto{\pgfqpoint{2.414766in}{1.906347in}}%
\pgfpathlineto{\pgfqpoint{2.403071in}{1.902463in}}%
\pgfpathlineto{\pgfqpoint{2.409205in}{1.892328in}}%
\pgfpathlineto{\pgfqpoint{2.415342in}{1.882146in}}%
\pgfpathlineto{\pgfqpoint{2.421484in}{1.871919in}}%
\pgfpathlineto{\pgfqpoint{2.427631in}{1.861646in}}%
\pgfpathclose%
\pgfusepath{stroke,fill}%
\end{pgfscope}%
\begin{pgfscope}%
\pgfpathrectangle{\pgfqpoint{0.887500in}{0.275000in}}{\pgfqpoint{4.225000in}{4.225000in}}%
\pgfusepath{clip}%
\pgfsetbuttcap%
\pgfsetroundjoin%
\definecolor{currentfill}{rgb}{0.239346,0.300855,0.540844}%
\pgfsetfillcolor{currentfill}%
\pgfsetfillopacity{0.700000}%
\pgfsetlinewidth{0.501875pt}%
\definecolor{currentstroke}{rgb}{1.000000,1.000000,1.000000}%
\pgfsetstrokecolor{currentstroke}%
\pgfsetstrokeopacity{0.500000}%
\pgfsetdash{}{0pt}%
\pgfpathmoveto{\pgfqpoint{3.028432in}{1.661596in}}%
\pgfpathlineto{\pgfqpoint{3.039994in}{1.665526in}}%
\pgfpathlineto{\pgfqpoint{3.051550in}{1.669441in}}%
\pgfpathlineto{\pgfqpoint{3.063100in}{1.673337in}}%
\pgfpathlineto{\pgfqpoint{3.074645in}{1.677212in}}%
\pgfpathlineto{\pgfqpoint{3.086184in}{1.681061in}}%
\pgfpathlineto{\pgfqpoint{3.079846in}{1.692528in}}%
\pgfpathlineto{\pgfqpoint{3.073512in}{1.703928in}}%
\pgfpathlineto{\pgfqpoint{3.067182in}{1.715263in}}%
\pgfpathlineto{\pgfqpoint{3.060854in}{1.726532in}}%
\pgfpathlineto{\pgfqpoint{3.054531in}{1.737738in}}%
\pgfpathlineto{\pgfqpoint{3.042999in}{1.733979in}}%
\pgfpathlineto{\pgfqpoint{3.031463in}{1.730192in}}%
\pgfpathlineto{\pgfqpoint{3.019920in}{1.726378in}}%
\pgfpathlineto{\pgfqpoint{3.008372in}{1.722541in}}%
\pgfpathlineto{\pgfqpoint{2.996819in}{1.718681in}}%
\pgfpathlineto{\pgfqpoint{3.003134in}{1.707370in}}%
\pgfpathlineto{\pgfqpoint{3.009454in}{1.696002in}}%
\pgfpathlineto{\pgfqpoint{3.015776in}{1.684581in}}%
\pgfpathlineto{\pgfqpoint{3.022102in}{1.673112in}}%
\pgfpathclose%
\pgfusepath{stroke,fill}%
\end{pgfscope}%
\begin{pgfscope}%
\pgfpathrectangle{\pgfqpoint{0.887500in}{0.275000in}}{\pgfqpoint{4.225000in}{4.225000in}}%
\pgfusepath{clip}%
\pgfsetbuttcap%
\pgfsetroundjoin%
\definecolor{currentfill}{rgb}{0.278791,0.062145,0.386592}%
\pgfsetfillcolor{currentfill}%
\pgfsetfillopacity{0.700000}%
\pgfsetlinewidth{0.501875pt}%
\definecolor{currentstroke}{rgb}{1.000000,1.000000,1.000000}%
\pgfsetstrokecolor{currentstroke}%
\pgfsetstrokeopacity{0.500000}%
\pgfsetdash{}{0pt}%
\pgfpathmoveto{\pgfqpoint{3.833795in}{1.275020in}}%
\pgfpathlineto{\pgfqpoint{3.845226in}{1.282181in}}%
\pgfpathlineto{\pgfqpoint{3.856653in}{1.289357in}}%
\pgfpathlineto{\pgfqpoint{3.868077in}{1.296540in}}%
\pgfpathlineto{\pgfqpoint{3.879498in}{1.303718in}}%
\pgfpathlineto{\pgfqpoint{3.890914in}{1.310882in}}%
\pgfpathlineto{\pgfqpoint{3.884257in}{1.317804in}}%
\pgfpathlineto{\pgfqpoint{3.877633in}{1.326065in}}%
\pgfpathlineto{\pgfqpoint{3.871041in}{1.335588in}}%
\pgfpathlineto{\pgfqpoint{3.864478in}{1.346299in}}%
\pgfpathlineto{\pgfqpoint{3.857943in}{1.358123in}}%
\pgfpathlineto{\pgfqpoint{3.846477in}{1.347903in}}%
\pgfpathlineto{\pgfqpoint{3.835026in}{1.338562in}}%
\pgfpathlineto{\pgfqpoint{3.823592in}{1.330148in}}%
\pgfpathlineto{\pgfqpoint{3.812174in}{1.322702in}}%
\pgfpathlineto{\pgfqpoint{3.800770in}{1.316172in}}%
\pgfpathlineto{\pgfqpoint{3.807303in}{1.304736in}}%
\pgfpathlineto{\pgfqpoint{3.813868in}{1.294767in}}%
\pgfpathlineto{\pgfqpoint{3.820470in}{1.286401in}}%
\pgfpathlineto{\pgfqpoint{3.827111in}{1.279774in}}%
\pgfpathclose%
\pgfusepath{stroke,fill}%
\end{pgfscope}%
\begin{pgfscope}%
\pgfpathrectangle{\pgfqpoint{0.887500in}{0.275000in}}{\pgfqpoint{4.225000in}{4.225000in}}%
\pgfusepath{clip}%
\pgfsetbuttcap%
\pgfsetroundjoin%
\definecolor{currentfill}{rgb}{0.248629,0.278775,0.534556}%
\pgfsetfillcolor{currentfill}%
\pgfsetfillopacity{0.700000}%
\pgfsetlinewidth{0.501875pt}%
\definecolor{currentstroke}{rgb}{1.000000,1.000000,1.000000}%
\pgfsetstrokecolor{currentstroke}%
\pgfsetstrokeopacity{0.500000}%
\pgfsetdash{}{0pt}%
\pgfpathmoveto{\pgfqpoint{3.117921in}{1.622693in}}%
\pgfpathlineto{\pgfqpoint{3.129462in}{1.626505in}}%
\pgfpathlineto{\pgfqpoint{3.140997in}{1.630324in}}%
\pgfpathlineto{\pgfqpoint{3.152527in}{1.634156in}}%
\pgfpathlineto{\pgfqpoint{3.164051in}{1.638013in}}%
\pgfpathlineto{\pgfqpoint{3.175569in}{1.641902in}}%
\pgfpathlineto{\pgfqpoint{3.169208in}{1.653752in}}%
\pgfpathlineto{\pgfqpoint{3.162850in}{1.665507in}}%
\pgfpathlineto{\pgfqpoint{3.156495in}{1.677161in}}%
\pgfpathlineto{\pgfqpoint{3.150143in}{1.688718in}}%
\pgfpathlineto{\pgfqpoint{3.143794in}{1.700186in}}%
\pgfpathlineto{\pgfqpoint{3.132284in}{1.696316in}}%
\pgfpathlineto{\pgfqpoint{3.120767in}{1.692489in}}%
\pgfpathlineto{\pgfqpoint{3.109245in}{1.688685in}}%
\pgfpathlineto{\pgfqpoint{3.097717in}{1.684883in}}%
\pgfpathlineto{\pgfqpoint{3.086184in}{1.681061in}}%
\pgfpathlineto{\pgfqpoint{3.092525in}{1.669527in}}%
\pgfpathlineto{\pgfqpoint{3.098869in}{1.657923in}}%
\pgfpathlineto{\pgfqpoint{3.105216in}{1.646251in}}%
\pgfpathlineto{\pgfqpoint{3.111567in}{1.634508in}}%
\pgfpathclose%
\pgfusepath{stroke,fill}%
\end{pgfscope}%
\begin{pgfscope}%
\pgfpathrectangle{\pgfqpoint{0.887500in}{0.275000in}}{\pgfqpoint{4.225000in}{4.225000in}}%
\pgfusepath{clip}%
\pgfsetbuttcap%
\pgfsetroundjoin%
\definecolor{currentfill}{rgb}{0.179019,0.433756,0.557430}%
\pgfsetfillcolor{currentfill}%
\pgfsetfillopacity{0.700000}%
\pgfsetlinewidth{0.501875pt}%
\definecolor{currentstroke}{rgb}{1.000000,1.000000,1.000000}%
\pgfsetstrokecolor{currentstroke}%
\pgfsetstrokeopacity{0.500000}%
\pgfsetdash{}{0pt}%
\pgfpathmoveto{\pgfqpoint{2.107284in}{1.925297in}}%
\pgfpathlineto{\pgfqpoint{2.119072in}{1.929173in}}%
\pgfpathlineto{\pgfqpoint{2.130854in}{1.933046in}}%
\pgfpathlineto{\pgfqpoint{2.142630in}{1.936914in}}%
\pgfpathlineto{\pgfqpoint{2.154401in}{1.940775in}}%
\pgfpathlineto{\pgfqpoint{2.166167in}{1.944626in}}%
\pgfpathlineto{\pgfqpoint{2.160112in}{1.954457in}}%
\pgfpathlineto{\pgfqpoint{2.154061in}{1.964247in}}%
\pgfpathlineto{\pgfqpoint{2.148016in}{1.973996in}}%
\pgfpathlineto{\pgfqpoint{2.141974in}{1.983706in}}%
\pgfpathlineto{\pgfqpoint{2.135938in}{1.993375in}}%
\pgfpathlineto{\pgfqpoint{2.124182in}{1.989578in}}%
\pgfpathlineto{\pgfqpoint{2.112422in}{1.985770in}}%
\pgfpathlineto{\pgfqpoint{2.100655in}{1.981953in}}%
\pgfpathlineto{\pgfqpoint{2.088883in}{1.978130in}}%
\pgfpathlineto{\pgfqpoint{2.077106in}{1.974303in}}%
\pgfpathlineto{\pgfqpoint{2.083132in}{1.964580in}}%
\pgfpathlineto{\pgfqpoint{2.089163in}{1.954818in}}%
\pgfpathlineto{\pgfqpoint{2.095199in}{1.945017in}}%
\pgfpathlineto{\pgfqpoint{2.101239in}{1.935177in}}%
\pgfpathclose%
\pgfusepath{stroke,fill}%
\end{pgfscope}%
\begin{pgfscope}%
\pgfpathrectangle{\pgfqpoint{0.887500in}{0.275000in}}{\pgfqpoint{4.225000in}{4.225000in}}%
\pgfusepath{clip}%
\pgfsetbuttcap%
\pgfsetroundjoin%
\definecolor{currentfill}{rgb}{0.201239,0.383670,0.554294}%
\pgfsetfillcolor{currentfill}%
\pgfsetfillopacity{0.700000}%
\pgfsetlinewidth{0.501875pt}%
\definecolor{currentstroke}{rgb}{1.000000,1.000000,1.000000}%
\pgfsetstrokecolor{currentstroke}%
\pgfsetstrokeopacity{0.500000}%
\pgfsetdash{}{0pt}%
\pgfpathmoveto{\pgfqpoint{2.523113in}{1.818919in}}%
\pgfpathlineto{\pgfqpoint{2.534799in}{1.822874in}}%
\pgfpathlineto{\pgfqpoint{2.546480in}{1.826817in}}%
\pgfpathlineto{\pgfqpoint{2.558155in}{1.830745in}}%
\pgfpathlineto{\pgfqpoint{2.569825in}{1.834658in}}%
\pgfpathlineto{\pgfqpoint{2.581489in}{1.838554in}}%
\pgfpathlineto{\pgfqpoint{2.575298in}{1.849006in}}%
\pgfpathlineto{\pgfqpoint{2.569111in}{1.859410in}}%
\pgfpathlineto{\pgfqpoint{2.562929in}{1.869767in}}%
\pgfpathlineto{\pgfqpoint{2.556751in}{1.880077in}}%
\pgfpathlineto{\pgfqpoint{2.550577in}{1.890341in}}%
\pgfpathlineto{\pgfqpoint{2.538922in}{1.886501in}}%
\pgfpathlineto{\pgfqpoint{2.527261in}{1.882644in}}%
\pgfpathlineto{\pgfqpoint{2.515596in}{1.878769in}}%
\pgfpathlineto{\pgfqpoint{2.503925in}{1.874878in}}%
\pgfpathlineto{\pgfqpoint{2.492248in}{1.870972in}}%
\pgfpathlineto{\pgfqpoint{2.498413in}{1.860654in}}%
\pgfpathlineto{\pgfqpoint{2.504581in}{1.850290in}}%
\pgfpathlineto{\pgfqpoint{2.510755in}{1.839880in}}%
\pgfpathlineto{\pgfqpoint{2.516932in}{1.829423in}}%
\pgfpathclose%
\pgfusepath{stroke,fill}%
\end{pgfscope}%
\begin{pgfscope}%
\pgfpathrectangle{\pgfqpoint{0.887500in}{0.275000in}}{\pgfqpoint{4.225000in}{4.225000in}}%
\pgfusepath{clip}%
\pgfsetbuttcap%
\pgfsetroundjoin%
\definecolor{currentfill}{rgb}{0.278012,0.180367,0.486697}%
\pgfsetfillcolor{currentfill}%
\pgfsetfillopacity{0.700000}%
\pgfsetlinewidth{0.501875pt}%
\definecolor{currentstroke}{rgb}{1.000000,1.000000,1.000000}%
\pgfsetstrokecolor{currentstroke}%
\pgfsetstrokeopacity{0.500000}%
\pgfsetdash{}{0pt}%
\pgfpathmoveto{\pgfqpoint{3.475856in}{1.447371in}}%
\pgfpathlineto{\pgfqpoint{3.487313in}{1.451619in}}%
\pgfpathlineto{\pgfqpoint{3.498767in}{1.456163in}}%
\pgfpathlineto{\pgfqpoint{3.510220in}{1.461028in}}%
\pgfpathlineto{\pgfqpoint{3.521672in}{1.466231in}}%
\pgfpathlineto{\pgfqpoint{3.533123in}{1.471787in}}%
\pgfpathlineto{\pgfqpoint{3.526680in}{1.484581in}}%
\pgfpathlineto{\pgfqpoint{3.520238in}{1.497280in}}%
\pgfpathlineto{\pgfqpoint{3.513799in}{1.509908in}}%
\pgfpathlineto{\pgfqpoint{3.507362in}{1.522492in}}%
\pgfpathlineto{\pgfqpoint{3.500929in}{1.535058in}}%
\pgfpathlineto{\pgfqpoint{3.489494in}{1.530328in}}%
\pgfpathlineto{\pgfqpoint{3.478054in}{1.525697in}}%
\pgfpathlineto{\pgfqpoint{3.466612in}{1.521223in}}%
\pgfpathlineto{\pgfqpoint{3.455166in}{1.516962in}}%
\pgfpathlineto{\pgfqpoint{3.443718in}{1.512950in}}%
\pgfpathlineto{\pgfqpoint{3.450139in}{1.499812in}}%
\pgfpathlineto{\pgfqpoint{3.456564in}{1.486669in}}%
\pgfpathlineto{\pgfqpoint{3.462991in}{1.473538in}}%
\pgfpathlineto{\pgfqpoint{3.469422in}{1.460435in}}%
\pgfpathclose%
\pgfusepath{stroke,fill}%
\end{pgfscope}%
\begin{pgfscope}%
\pgfpathrectangle{\pgfqpoint{0.887500in}{0.275000in}}{\pgfqpoint{4.225000in}{4.225000in}}%
\pgfusepath{clip}%
\pgfsetbuttcap%
\pgfsetroundjoin%
\definecolor{currentfill}{rgb}{0.257322,0.256130,0.526563}%
\pgfsetfillcolor{currentfill}%
\pgfsetfillopacity{0.700000}%
\pgfsetlinewidth{0.501875pt}%
\definecolor{currentstroke}{rgb}{1.000000,1.000000,1.000000}%
\pgfsetstrokecolor{currentstroke}%
\pgfsetstrokeopacity{0.500000}%
\pgfsetdash{}{0pt}%
\pgfpathmoveto{\pgfqpoint{3.207421in}{1.581621in}}%
\pgfpathlineto{\pgfqpoint{3.218941in}{1.585561in}}%
\pgfpathlineto{\pgfqpoint{3.230455in}{1.589550in}}%
\pgfpathlineto{\pgfqpoint{3.241965in}{1.593592in}}%
\pgfpathlineto{\pgfqpoint{3.253469in}{1.597672in}}%
\pgfpathlineto{\pgfqpoint{3.264967in}{1.601762in}}%
\pgfpathlineto{\pgfqpoint{3.258584in}{1.613907in}}%
\pgfpathlineto{\pgfqpoint{3.252203in}{1.626010in}}%
\pgfpathlineto{\pgfqpoint{3.245826in}{1.638065in}}%
\pgfpathlineto{\pgfqpoint{3.239452in}{1.650066in}}%
\pgfpathlineto{\pgfqpoint{3.233080in}{1.662005in}}%
\pgfpathlineto{\pgfqpoint{3.221589in}{1.657936in}}%
\pgfpathlineto{\pgfqpoint{3.210092in}{1.653860in}}%
\pgfpathlineto{\pgfqpoint{3.198590in}{1.649818in}}%
\pgfpathlineto{\pgfqpoint{3.187082in}{1.645834in}}%
\pgfpathlineto{\pgfqpoint{3.175569in}{1.641902in}}%
\pgfpathlineto{\pgfqpoint{3.181933in}{1.629969in}}%
\pgfpathlineto{\pgfqpoint{3.188301in}{1.617963in}}%
\pgfpathlineto{\pgfqpoint{3.194671in}{1.605896in}}%
\pgfpathlineto{\pgfqpoint{3.201044in}{1.593778in}}%
\pgfpathclose%
\pgfusepath{stroke,fill}%
\end{pgfscope}%
\begin{pgfscope}%
\pgfpathrectangle{\pgfqpoint{0.887500in}{0.275000in}}{\pgfqpoint{4.225000in}{4.225000in}}%
\pgfusepath{clip}%
\pgfsetbuttcap%
\pgfsetroundjoin%
\definecolor{currentfill}{rgb}{0.271828,0.209303,0.504434}%
\pgfsetfillcolor{currentfill}%
\pgfsetfillopacity{0.700000}%
\pgfsetlinewidth{0.501875pt}%
\definecolor{currentstroke}{rgb}{1.000000,1.000000,1.000000}%
\pgfsetstrokecolor{currentstroke}%
\pgfsetstrokeopacity{0.500000}%
\pgfsetdash{}{0pt}%
\pgfpathmoveto{\pgfqpoint{3.386412in}{1.495192in}}%
\pgfpathlineto{\pgfqpoint{3.397883in}{1.498595in}}%
\pgfpathlineto{\pgfqpoint{3.409349in}{1.502025in}}%
\pgfpathlineto{\pgfqpoint{3.420810in}{1.505529in}}%
\pgfpathlineto{\pgfqpoint{3.432266in}{1.509155in}}%
\pgfpathlineto{\pgfqpoint{3.443718in}{1.512950in}}%
\pgfpathlineto{\pgfqpoint{3.437299in}{1.526068in}}%
\pgfpathlineto{\pgfqpoint{3.430883in}{1.539149in}}%
\pgfpathlineto{\pgfqpoint{3.424469in}{1.552177in}}%
\pgfpathlineto{\pgfqpoint{3.418058in}{1.565134in}}%
\pgfpathlineto{\pgfqpoint{3.411649in}{1.578006in}}%
\pgfpathlineto{\pgfqpoint{3.400202in}{1.574180in}}%
\pgfpathlineto{\pgfqpoint{3.388750in}{1.570471in}}%
\pgfpathlineto{\pgfqpoint{3.377293in}{1.566836in}}%
\pgfpathlineto{\pgfqpoint{3.365831in}{1.563234in}}%
\pgfpathlineto{\pgfqpoint{3.354363in}{1.559622in}}%
\pgfpathlineto{\pgfqpoint{3.360768in}{1.546942in}}%
\pgfpathlineto{\pgfqpoint{3.367176in}{1.534157in}}%
\pgfpathlineto{\pgfqpoint{3.373586in}{1.521269in}}%
\pgfpathlineto{\pgfqpoint{3.379998in}{1.508281in}}%
\pgfpathclose%
\pgfusepath{stroke,fill}%
\end{pgfscope}%
\begin{pgfscope}%
\pgfpathrectangle{\pgfqpoint{0.887500in}{0.275000in}}{\pgfqpoint{4.225000in}{4.225000in}}%
\pgfusepath{clip}%
\pgfsetbuttcap%
\pgfsetroundjoin%
\definecolor{currentfill}{rgb}{0.283072,0.130895,0.449241}%
\pgfsetfillcolor{currentfill}%
\pgfsetfillopacity{0.700000}%
\pgfsetlinewidth{0.501875pt}%
\definecolor{currentstroke}{rgb}{1.000000,1.000000,1.000000}%
\pgfsetstrokecolor{currentstroke}%
\pgfsetstrokeopacity{0.500000}%
\pgfsetdash{}{0pt}%
\pgfpathmoveto{\pgfqpoint{3.654766in}{1.357341in}}%
\pgfpathlineto{\pgfqpoint{3.666175in}{1.361363in}}%
\pgfpathlineto{\pgfqpoint{3.677575in}{1.365178in}}%
\pgfpathlineto{\pgfqpoint{3.688966in}{1.368742in}}%
\pgfpathlineto{\pgfqpoint{3.700345in}{1.372012in}}%
\pgfpathlineto{\pgfqpoint{3.711714in}{1.374945in}}%
\pgfpathlineto{\pgfqpoint{3.705315in}{1.392827in}}%
\pgfpathlineto{\pgfqpoint{3.698920in}{1.410861in}}%
\pgfpathlineto{\pgfqpoint{3.692527in}{1.428886in}}%
\pgfpathlineto{\pgfqpoint{3.686132in}{1.446741in}}%
\pgfpathlineto{\pgfqpoint{3.679733in}{1.464266in}}%
\pgfpathlineto{\pgfqpoint{3.668338in}{1.459809in}}%
\pgfpathlineto{\pgfqpoint{3.656925in}{1.454591in}}%
\pgfpathlineto{\pgfqpoint{3.645499in}{1.448788in}}%
\pgfpathlineto{\pgfqpoint{3.634062in}{1.442577in}}%
\pgfpathlineto{\pgfqpoint{3.622618in}{1.436137in}}%
\pgfpathlineto{\pgfqpoint{3.629051in}{1.420877in}}%
\pgfpathlineto{\pgfqpoint{3.635481in}{1.405296in}}%
\pgfpathlineto{\pgfqpoint{3.641910in}{1.389467in}}%
\pgfpathlineto{\pgfqpoint{3.648338in}{1.373458in}}%
\pgfpathclose%
\pgfusepath{stroke,fill}%
\end{pgfscope}%
\begin{pgfscope}%
\pgfpathrectangle{\pgfqpoint{0.887500in}{0.275000in}}{\pgfqpoint{4.225000in}{4.225000in}}%
\pgfusepath{clip}%
\pgfsetbuttcap%
\pgfsetroundjoin%
\definecolor{currentfill}{rgb}{0.265145,0.232956,0.516599}%
\pgfsetfillcolor{currentfill}%
\pgfsetfillopacity{0.700000}%
\pgfsetlinewidth{0.501875pt}%
\definecolor{currentstroke}{rgb}{1.000000,1.000000,1.000000}%
\pgfsetstrokecolor{currentstroke}%
\pgfsetstrokeopacity{0.500000}%
\pgfsetdash{}{0pt}%
\pgfpathmoveto{\pgfqpoint{3.296930in}{1.540605in}}%
\pgfpathlineto{\pgfqpoint{3.308429in}{1.544528in}}%
\pgfpathlineto{\pgfqpoint{3.319922in}{1.548403in}}%
\pgfpathlineto{\pgfqpoint{3.331408in}{1.552216in}}%
\pgfpathlineto{\pgfqpoint{3.342889in}{1.555958in}}%
\pgfpathlineto{\pgfqpoint{3.354363in}{1.559622in}}%
\pgfpathlineto{\pgfqpoint{3.347960in}{1.572197in}}%
\pgfpathlineto{\pgfqpoint{3.341559in}{1.584664in}}%
\pgfpathlineto{\pgfqpoint{3.335161in}{1.597024in}}%
\pgfpathlineto{\pgfqpoint{3.328765in}{1.609280in}}%
\pgfpathlineto{\pgfqpoint{3.322372in}{1.621433in}}%
\pgfpathlineto{\pgfqpoint{3.310903in}{1.617688in}}%
\pgfpathlineto{\pgfqpoint{3.299428in}{1.613826in}}%
\pgfpathlineto{\pgfqpoint{3.287947in}{1.609866in}}%
\pgfpathlineto{\pgfqpoint{3.276460in}{1.605836in}}%
\pgfpathlineto{\pgfqpoint{3.264967in}{1.601762in}}%
\pgfpathlineto{\pgfqpoint{3.271353in}{1.589581in}}%
\pgfpathlineto{\pgfqpoint{3.277743in}{1.577371in}}%
\pgfpathlineto{\pgfqpoint{3.284136in}{1.565138in}}%
\pgfpathlineto{\pgfqpoint{3.290531in}{1.552886in}}%
\pgfpathclose%
\pgfusepath{stroke,fill}%
\end{pgfscope}%
\begin{pgfscope}%
\pgfpathrectangle{\pgfqpoint{0.887500in}{0.275000in}}{\pgfqpoint{4.225000in}{4.225000in}}%
\pgfusepath{clip}%
\pgfsetbuttcap%
\pgfsetroundjoin%
\definecolor{currentfill}{rgb}{0.280868,0.160771,0.472899}%
\pgfsetfillcolor{currentfill}%
\pgfsetfillopacity{0.700000}%
\pgfsetlinewidth{0.501875pt}%
\definecolor{currentstroke}{rgb}{1.000000,1.000000,1.000000}%
\pgfsetstrokecolor{currentstroke}%
\pgfsetstrokeopacity{0.500000}%
\pgfsetdash{}{0pt}%
\pgfpathmoveto{\pgfqpoint{3.565362in}{1.405887in}}%
\pgfpathlineto{\pgfqpoint{3.576815in}{1.411400in}}%
\pgfpathlineto{\pgfqpoint{3.588267in}{1.417191in}}%
\pgfpathlineto{\pgfqpoint{3.599718in}{1.423280in}}%
\pgfpathlineto{\pgfqpoint{3.611169in}{1.429645in}}%
\pgfpathlineto{\pgfqpoint{3.622618in}{1.436137in}}%
\pgfpathlineto{\pgfqpoint{3.616182in}{1.451007in}}%
\pgfpathlineto{\pgfqpoint{3.609742in}{1.465416in}}%
\pgfpathlineto{\pgfqpoint{3.603297in}{1.479294in}}%
\pgfpathlineto{\pgfqpoint{3.596846in}{1.492570in}}%
\pgfpathlineto{\pgfqpoint{3.590389in}{1.505172in}}%
\pgfpathlineto{\pgfqpoint{3.578935in}{1.497813in}}%
\pgfpathlineto{\pgfqpoint{3.567480in}{1.490724in}}%
\pgfpathlineto{\pgfqpoint{3.556027in}{1.484018in}}%
\pgfpathlineto{\pgfqpoint{3.544575in}{1.477711in}}%
\pgfpathlineto{\pgfqpoint{3.533123in}{1.471787in}}%
\pgfpathlineto{\pgfqpoint{3.539568in}{1.458882in}}%
\pgfpathlineto{\pgfqpoint{3.546015in}{1.445853in}}%
\pgfpathlineto{\pgfqpoint{3.552463in}{1.432686in}}%
\pgfpathlineto{\pgfqpoint{3.558912in}{1.419368in}}%
\pgfpathclose%
\pgfusepath{stroke,fill}%
\end{pgfscope}%
\begin{pgfscope}%
\pgfpathrectangle{\pgfqpoint{0.887500in}{0.275000in}}{\pgfqpoint{4.225000in}{4.225000in}}%
\pgfusepath{clip}%
\pgfsetbuttcap%
\pgfsetroundjoin%
\definecolor{currentfill}{rgb}{0.208623,0.367752,0.552675}%
\pgfsetfillcolor{currentfill}%
\pgfsetfillopacity{0.700000}%
\pgfsetlinewidth{0.501875pt}%
\definecolor{currentstroke}{rgb}{1.000000,1.000000,1.000000}%
\pgfsetstrokecolor{currentstroke}%
\pgfsetstrokeopacity{0.500000}%
\pgfsetdash{}{0pt}%
\pgfpathmoveto{\pgfqpoint{2.612505in}{1.785570in}}%
\pgfpathlineto{\pgfqpoint{2.624172in}{1.789503in}}%
\pgfpathlineto{\pgfqpoint{2.635835in}{1.793418in}}%
\pgfpathlineto{\pgfqpoint{2.647491in}{1.797315in}}%
\pgfpathlineto{\pgfqpoint{2.659143in}{1.801191in}}%
\pgfpathlineto{\pgfqpoint{2.670788in}{1.805051in}}%
\pgfpathlineto{\pgfqpoint{2.664568in}{1.815693in}}%
\pgfpathlineto{\pgfqpoint{2.658352in}{1.826284in}}%
\pgfpathlineto{\pgfqpoint{2.652139in}{1.836824in}}%
\pgfpathlineto{\pgfqpoint{2.645931in}{1.847315in}}%
\pgfpathlineto{\pgfqpoint{2.639727in}{1.857758in}}%
\pgfpathlineto{\pgfqpoint{2.628090in}{1.853955in}}%
\pgfpathlineto{\pgfqpoint{2.616448in}{1.850134in}}%
\pgfpathlineto{\pgfqpoint{2.604800in}{1.846294in}}%
\pgfpathlineto{\pgfqpoint{2.593147in}{1.842433in}}%
\pgfpathlineto{\pgfqpoint{2.581489in}{1.838554in}}%
\pgfpathlineto{\pgfqpoint{2.587684in}{1.828055in}}%
\pgfpathlineto{\pgfqpoint{2.593883in}{1.817508in}}%
\pgfpathlineto{\pgfqpoint{2.600086in}{1.806911in}}%
\pgfpathlineto{\pgfqpoint{2.606293in}{1.796266in}}%
\pgfpathclose%
\pgfusepath{stroke,fill}%
\end{pgfscope}%
\begin{pgfscope}%
\pgfpathrectangle{\pgfqpoint{0.887500in}{0.275000in}}{\pgfqpoint{4.225000in}{4.225000in}}%
\pgfusepath{clip}%
\pgfsetbuttcap%
\pgfsetroundjoin%
\definecolor{currentfill}{rgb}{0.185556,0.418570,0.556753}%
\pgfsetfillcolor{currentfill}%
\pgfsetfillopacity{0.700000}%
\pgfsetlinewidth{0.501875pt}%
\definecolor{currentstroke}{rgb}{1.000000,1.000000,1.000000}%
\pgfsetstrokecolor{currentstroke}%
\pgfsetstrokeopacity{0.500000}%
\pgfsetdash{}{0pt}%
\pgfpathmoveto{\pgfqpoint{2.196509in}{1.894858in}}%
\pgfpathlineto{\pgfqpoint{2.208279in}{1.898750in}}%
\pgfpathlineto{\pgfqpoint{2.220043in}{1.902633in}}%
\pgfpathlineto{\pgfqpoint{2.231802in}{1.906507in}}%
\pgfpathlineto{\pgfqpoint{2.243555in}{1.910371in}}%
\pgfpathlineto{\pgfqpoint{2.255303in}{1.914225in}}%
\pgfpathlineto{\pgfqpoint{2.249216in}{1.924207in}}%
\pgfpathlineto{\pgfqpoint{2.243133in}{1.934145in}}%
\pgfpathlineto{\pgfqpoint{2.237054in}{1.944041in}}%
\pgfpathlineto{\pgfqpoint{2.230981in}{1.953896in}}%
\pgfpathlineto{\pgfqpoint{2.224911in}{1.963708in}}%
\pgfpathlineto{\pgfqpoint{2.213173in}{1.959915in}}%
\pgfpathlineto{\pgfqpoint{2.201430in}{1.956110in}}%
\pgfpathlineto{\pgfqpoint{2.189681in}{1.952294in}}%
\pgfpathlineto{\pgfqpoint{2.177927in}{1.948467in}}%
\pgfpathlineto{\pgfqpoint{2.166167in}{1.944626in}}%
\pgfpathlineto{\pgfqpoint{2.172226in}{1.934755in}}%
\pgfpathlineto{\pgfqpoint{2.178290in}{1.924843in}}%
\pgfpathlineto{\pgfqpoint{2.184359in}{1.914889in}}%
\pgfpathlineto{\pgfqpoint{2.190432in}{1.904895in}}%
\pgfpathclose%
\pgfusepath{stroke,fill}%
\end{pgfscope}%
\begin{pgfscope}%
\pgfpathrectangle{\pgfqpoint{0.887500in}{0.275000in}}{\pgfqpoint{4.225000in}{4.225000in}}%
\pgfusepath{clip}%
\pgfsetbuttcap%
\pgfsetroundjoin%
\definecolor{currentfill}{rgb}{0.216210,0.351535,0.550627}%
\pgfsetfillcolor{currentfill}%
\pgfsetfillopacity{0.700000}%
\pgfsetlinewidth{0.501875pt}%
\definecolor{currentstroke}{rgb}{1.000000,1.000000,1.000000}%
\pgfsetstrokecolor{currentstroke}%
\pgfsetstrokeopacity{0.500000}%
\pgfsetdash{}{0pt}%
\pgfpathmoveto{\pgfqpoint{2.701951in}{1.751066in}}%
\pgfpathlineto{\pgfqpoint{2.713600in}{1.754965in}}%
\pgfpathlineto{\pgfqpoint{2.725243in}{1.758856in}}%
\pgfpathlineto{\pgfqpoint{2.736881in}{1.762741in}}%
\pgfpathlineto{\pgfqpoint{2.748513in}{1.766624in}}%
\pgfpathlineto{\pgfqpoint{2.760140in}{1.770507in}}%
\pgfpathlineto{\pgfqpoint{2.753891in}{1.781352in}}%
\pgfpathlineto{\pgfqpoint{2.747645in}{1.792145in}}%
\pgfpathlineto{\pgfqpoint{2.741404in}{1.802885in}}%
\pgfpathlineto{\pgfqpoint{2.735167in}{1.813573in}}%
\pgfpathlineto{\pgfqpoint{2.728933in}{1.824209in}}%
\pgfpathlineto{\pgfqpoint{2.717316in}{1.820385in}}%
\pgfpathlineto{\pgfqpoint{2.705692in}{1.816560in}}%
\pgfpathlineto{\pgfqpoint{2.694063in}{1.812732in}}%
\pgfpathlineto{\pgfqpoint{2.682429in}{1.808897in}}%
\pgfpathlineto{\pgfqpoint{2.670788in}{1.805051in}}%
\pgfpathlineto{\pgfqpoint{2.677013in}{1.794358in}}%
\pgfpathlineto{\pgfqpoint{2.683241in}{1.783613in}}%
\pgfpathlineto{\pgfqpoint{2.689474in}{1.772815in}}%
\pgfpathlineto{\pgfqpoint{2.695710in}{1.761966in}}%
\pgfpathclose%
\pgfusepath{stroke,fill}%
\end{pgfscope}%
\begin{pgfscope}%
\pgfpathrectangle{\pgfqpoint{0.887500in}{0.275000in}}{\pgfqpoint{4.225000in}{4.225000in}}%
\pgfusepath{clip}%
\pgfsetbuttcap%
\pgfsetroundjoin%
\definecolor{currentfill}{rgb}{0.171176,0.452530,0.557965}%
\pgfsetfillcolor{currentfill}%
\pgfsetfillopacity{0.700000}%
\pgfsetlinewidth{0.501875pt}%
\definecolor{currentstroke}{rgb}{1.000000,1.000000,1.000000}%
\pgfsetstrokecolor{currentstroke}%
\pgfsetstrokeopacity{0.500000}%
\pgfsetdash{}{0pt}%
\pgfpathmoveto{\pgfqpoint{1.869843in}{1.965473in}}%
\pgfpathlineto{\pgfqpoint{1.881696in}{1.969319in}}%
\pgfpathlineto{\pgfqpoint{1.893543in}{1.973152in}}%
\pgfpathlineto{\pgfqpoint{1.905384in}{1.976974in}}%
\pgfpathlineto{\pgfqpoint{1.917220in}{1.980785in}}%
\pgfpathlineto{\pgfqpoint{1.929050in}{1.984588in}}%
\pgfpathlineto{\pgfqpoint{1.923072in}{1.994206in}}%
\pgfpathlineto{\pgfqpoint{1.917098in}{2.003789in}}%
\pgfpathlineto{\pgfqpoint{1.911128in}{2.013340in}}%
\pgfpathlineto{\pgfqpoint{1.905163in}{2.022858in}}%
\pgfpathlineto{\pgfqpoint{1.899203in}{2.032344in}}%
\pgfpathlineto{\pgfqpoint{1.887383in}{2.028595in}}%
\pgfpathlineto{\pgfqpoint{1.875557in}{2.024837in}}%
\pgfpathlineto{\pgfqpoint{1.863726in}{2.021068in}}%
\pgfpathlineto{\pgfqpoint{1.851890in}{2.017286in}}%
\pgfpathlineto{\pgfqpoint{1.840048in}{2.013492in}}%
\pgfpathlineto{\pgfqpoint{1.845998in}{2.003952in}}%
\pgfpathlineto{\pgfqpoint{1.851952in}{1.994381in}}%
\pgfpathlineto{\pgfqpoint{1.857911in}{1.984778in}}%
\pgfpathlineto{\pgfqpoint{1.863875in}{1.975142in}}%
\pgfpathclose%
\pgfusepath{stroke,fill}%
\end{pgfscope}%
\begin{pgfscope}%
\pgfpathrectangle{\pgfqpoint{0.887500in}{0.275000in}}{\pgfqpoint{4.225000in}{4.225000in}}%
\pgfusepath{clip}%
\pgfsetbuttcap%
\pgfsetroundjoin%
\definecolor{currentfill}{rgb}{0.190631,0.407061,0.556089}%
\pgfsetfillcolor{currentfill}%
\pgfsetfillopacity{0.700000}%
\pgfsetlinewidth{0.501875pt}%
\definecolor{currentstroke}{rgb}{1.000000,1.000000,1.000000}%
\pgfsetstrokecolor{currentstroke}%
\pgfsetstrokeopacity{0.500000}%
\pgfsetdash{}{0pt}%
\pgfpathmoveto{\pgfqpoint{2.285807in}{1.863673in}}%
\pgfpathlineto{\pgfqpoint{2.297559in}{1.867575in}}%
\pgfpathlineto{\pgfqpoint{2.309305in}{1.871468in}}%
\pgfpathlineto{\pgfqpoint{2.321046in}{1.875353in}}%
\pgfpathlineto{\pgfqpoint{2.332781in}{1.879230in}}%
\pgfpathlineto{\pgfqpoint{2.344510in}{1.883102in}}%
\pgfpathlineto{\pgfqpoint{2.338391in}{1.893245in}}%
\pgfpathlineto{\pgfqpoint{2.332276in}{1.903343in}}%
\pgfpathlineto{\pgfqpoint{2.326165in}{1.913396in}}%
\pgfpathlineto{\pgfqpoint{2.320059in}{1.923405in}}%
\pgfpathlineto{\pgfqpoint{2.313958in}{1.933369in}}%
\pgfpathlineto{\pgfqpoint{2.302238in}{1.929554in}}%
\pgfpathlineto{\pgfqpoint{2.290513in}{1.925735in}}%
\pgfpathlineto{\pgfqpoint{2.278782in}{1.921907in}}%
\pgfpathlineto{\pgfqpoint{2.267045in}{1.918071in}}%
\pgfpathlineto{\pgfqpoint{2.255303in}{1.914225in}}%
\pgfpathlineto{\pgfqpoint{2.261395in}{1.904201in}}%
\pgfpathlineto{\pgfqpoint{2.267491in}{1.894134in}}%
\pgfpathlineto{\pgfqpoint{2.273592in}{1.884024in}}%
\pgfpathlineto{\pgfqpoint{2.279697in}{1.873870in}}%
\pgfpathclose%
\pgfusepath{stroke,fill}%
\end{pgfscope}%
\begin{pgfscope}%
\pgfpathrectangle{\pgfqpoint{0.887500in}{0.275000in}}{\pgfqpoint{4.225000in}{4.225000in}}%
\pgfusepath{clip}%
\pgfsetbuttcap%
\pgfsetroundjoin%
\definecolor{currentfill}{rgb}{0.277018,0.050344,0.375715}%
\pgfsetfillcolor{currentfill}%
\pgfsetfillopacity{0.700000}%
\pgfsetlinewidth{0.501875pt}%
\definecolor{currentstroke}{rgb}{1.000000,1.000000,1.000000}%
\pgfsetstrokecolor{currentstroke}%
\pgfsetstrokeopacity{0.500000}%
\pgfsetdash{}{0pt}%
\pgfpathmoveto{\pgfqpoint{3.776594in}{1.239628in}}%
\pgfpathlineto{\pgfqpoint{3.788040in}{1.246634in}}%
\pgfpathlineto{\pgfqpoint{3.799484in}{1.253681in}}%
\pgfpathlineto{\pgfqpoint{3.810924in}{1.260765in}}%
\pgfpathlineto{\pgfqpoint{3.822361in}{1.267880in}}%
\pgfpathlineto{\pgfqpoint{3.833795in}{1.275020in}}%
\pgfpathlineto{\pgfqpoint{3.827111in}{1.279774in}}%
\pgfpathlineto{\pgfqpoint{3.820470in}{1.286401in}}%
\pgfpathlineto{\pgfqpoint{3.813868in}{1.294767in}}%
\pgfpathlineto{\pgfqpoint{3.807303in}{1.304736in}}%
\pgfpathlineto{\pgfqpoint{3.800770in}{1.316172in}}%
\pgfpathlineto{\pgfqpoint{3.789378in}{1.310451in}}%
\pgfpathlineto{\pgfqpoint{3.777994in}{1.305424in}}%
\pgfpathlineto{\pgfqpoint{3.766617in}{1.300982in}}%
\pgfpathlineto{\pgfqpoint{3.755242in}{1.297014in}}%
\pgfpathlineto{\pgfqpoint{3.743869in}{1.293406in}}%
\pgfpathlineto{\pgfqpoint{3.750354in}{1.279791in}}%
\pgfpathlineto{\pgfqpoint{3.756866in}{1.267445in}}%
\pgfpathlineto{\pgfqpoint{3.763407in}{1.256529in}}%
\pgfpathlineto{\pgfqpoint{3.769982in}{1.247204in}}%
\pgfpathclose%
\pgfusepath{stroke,fill}%
\end{pgfscope}%
\begin{pgfscope}%
\pgfpathrectangle{\pgfqpoint{0.887500in}{0.275000in}}{\pgfqpoint{4.225000in}{4.225000in}}%
\pgfusepath{clip}%
\pgfsetbuttcap%
\pgfsetroundjoin%
\definecolor{currentfill}{rgb}{0.223925,0.334994,0.548053}%
\pgfsetfillcolor{currentfill}%
\pgfsetfillopacity{0.700000}%
\pgfsetlinewidth{0.501875pt}%
\definecolor{currentstroke}{rgb}{1.000000,1.000000,1.000000}%
\pgfsetstrokecolor{currentstroke}%
\pgfsetstrokeopacity{0.500000}%
\pgfsetdash{}{0pt}%
\pgfpathmoveto{\pgfqpoint{2.791444in}{1.715527in}}%
\pgfpathlineto{\pgfqpoint{2.803073in}{1.719469in}}%
\pgfpathlineto{\pgfqpoint{2.814697in}{1.723416in}}%
\pgfpathlineto{\pgfqpoint{2.826315in}{1.727367in}}%
\pgfpathlineto{\pgfqpoint{2.837927in}{1.731321in}}%
\pgfpathlineto{\pgfqpoint{2.849534in}{1.735278in}}%
\pgfpathlineto{\pgfqpoint{2.843257in}{1.746324in}}%
\pgfpathlineto{\pgfqpoint{2.836984in}{1.757319in}}%
\pgfpathlineto{\pgfqpoint{2.830714in}{1.768261in}}%
\pgfpathlineto{\pgfqpoint{2.824448in}{1.779152in}}%
\pgfpathlineto{\pgfqpoint{2.818187in}{1.789990in}}%
\pgfpathlineto{\pgfqpoint{2.806589in}{1.786083in}}%
\pgfpathlineto{\pgfqpoint{2.794985in}{1.782182in}}%
\pgfpathlineto{\pgfqpoint{2.783376in}{1.778285in}}%
\pgfpathlineto{\pgfqpoint{2.771761in}{1.774394in}}%
\pgfpathlineto{\pgfqpoint{2.760140in}{1.770507in}}%
\pgfpathlineto{\pgfqpoint{2.766393in}{1.759611in}}%
\pgfpathlineto{\pgfqpoint{2.772650in}{1.748664in}}%
\pgfpathlineto{\pgfqpoint{2.778911in}{1.737668in}}%
\pgfpathlineto{\pgfqpoint{2.785175in}{1.726622in}}%
\pgfpathclose%
\pgfusepath{stroke,fill}%
\end{pgfscope}%
\begin{pgfscope}%
\pgfpathrectangle{\pgfqpoint{0.887500in}{0.275000in}}{\pgfqpoint{4.225000in}{4.225000in}}%
\pgfusepath{clip}%
\pgfsetbuttcap%
\pgfsetroundjoin%
\definecolor{currentfill}{rgb}{0.281446,0.084320,0.407414}%
\pgfsetfillcolor{currentfill}%
\pgfsetfillopacity{0.700000}%
\pgfsetlinewidth{0.501875pt}%
\definecolor{currentstroke}{rgb}{1.000000,1.000000,1.000000}%
\pgfsetstrokecolor{currentstroke}%
\pgfsetstrokeopacity{0.500000}%
\pgfsetdash{}{0pt}%
\pgfpathmoveto{\pgfqpoint{3.686952in}{1.277584in}}%
\pgfpathlineto{\pgfqpoint{3.698346in}{1.280707in}}%
\pgfpathlineto{\pgfqpoint{3.709734in}{1.283776in}}%
\pgfpathlineto{\pgfqpoint{3.721116in}{1.286867in}}%
\pgfpathlineto{\pgfqpoint{3.732494in}{1.290052in}}%
\pgfpathlineto{\pgfqpoint{3.743869in}{1.293406in}}%
\pgfpathlineto{\pgfqpoint{3.737406in}{1.308132in}}%
\pgfpathlineto{\pgfqpoint{3.730962in}{1.323808in}}%
\pgfpathlineto{\pgfqpoint{3.724534in}{1.340276in}}%
\pgfpathlineto{\pgfqpoint{3.718119in}{1.357375in}}%
\pgfpathlineto{\pgfqpoint{3.711714in}{1.374945in}}%
\pgfpathlineto{\pgfqpoint{3.700345in}{1.372012in}}%
\pgfpathlineto{\pgfqpoint{3.688966in}{1.368742in}}%
\pgfpathlineto{\pgfqpoint{3.677575in}{1.365178in}}%
\pgfpathlineto{\pgfqpoint{3.666175in}{1.361363in}}%
\pgfpathlineto{\pgfqpoint{3.654766in}{1.357341in}}%
\pgfpathlineto{\pgfqpoint{3.661196in}{1.341186in}}%
\pgfpathlineto{\pgfqpoint{3.667628in}{1.325062in}}%
\pgfpathlineto{\pgfqpoint{3.674064in}{1.309041in}}%
\pgfpathlineto{\pgfqpoint{3.680505in}{1.293191in}}%
\pgfpathclose%
\pgfusepath{stroke,fill}%
\end{pgfscope}%
\begin{pgfscope}%
\pgfpathrectangle{\pgfqpoint{0.887500in}{0.275000in}}{\pgfqpoint{4.225000in}{4.225000in}}%
\pgfusepath{clip}%
\pgfsetbuttcap%
\pgfsetroundjoin%
\definecolor{currentfill}{rgb}{0.175841,0.441290,0.557685}%
\pgfsetfillcolor{currentfill}%
\pgfsetfillopacity{0.700000}%
\pgfsetlinewidth{0.501875pt}%
\definecolor{currentstroke}{rgb}{1.000000,1.000000,1.000000}%
\pgfsetstrokecolor{currentstroke}%
\pgfsetstrokeopacity{0.500000}%
\pgfsetdash{}{0pt}%
\pgfpathmoveto{\pgfqpoint{1.959012in}{1.935987in}}%
\pgfpathlineto{\pgfqpoint{1.970847in}{1.939833in}}%
\pgfpathlineto{\pgfqpoint{1.982677in}{1.943674in}}%
\pgfpathlineto{\pgfqpoint{1.994500in}{1.947510in}}%
\pgfpathlineto{\pgfqpoint{2.006318in}{1.951341in}}%
\pgfpathlineto{\pgfqpoint{2.018131in}{1.955170in}}%
\pgfpathlineto{\pgfqpoint{2.012119in}{1.964909in}}%
\pgfpathlineto{\pgfqpoint{2.006112in}{1.974611in}}%
\pgfpathlineto{\pgfqpoint{2.000109in}{1.984278in}}%
\pgfpathlineto{\pgfqpoint{1.994111in}{1.993909in}}%
\pgfpathlineto{\pgfqpoint{1.988118in}{2.003506in}}%
\pgfpathlineto{\pgfqpoint{1.976316in}{1.999731in}}%
\pgfpathlineto{\pgfqpoint{1.964508in}{1.995954in}}%
\pgfpathlineto{\pgfqpoint{1.952694in}{1.992171in}}%
\pgfpathlineto{\pgfqpoint{1.940875in}{1.988383in}}%
\pgfpathlineto{\pgfqpoint{1.929050in}{1.984588in}}%
\pgfpathlineto{\pgfqpoint{1.935034in}{1.974937in}}%
\pgfpathlineto{\pgfqpoint{1.941021in}{1.965252in}}%
\pgfpathlineto{\pgfqpoint{1.947014in}{1.955532in}}%
\pgfpathlineto{\pgfqpoint{1.953011in}{1.945777in}}%
\pgfpathclose%
\pgfusepath{stroke,fill}%
\end{pgfscope}%
\begin{pgfscope}%
\pgfpathrectangle{\pgfqpoint{0.887500in}{0.275000in}}{\pgfqpoint{4.225000in}{4.225000in}}%
\pgfusepath{clip}%
\pgfsetbuttcap%
\pgfsetroundjoin%
\definecolor{currentfill}{rgb}{0.231674,0.318106,0.544834}%
\pgfsetfillcolor{currentfill}%
\pgfsetfillopacity{0.700000}%
\pgfsetlinewidth{0.501875pt}%
\definecolor{currentstroke}{rgb}{1.000000,1.000000,1.000000}%
\pgfsetstrokecolor{currentstroke}%
\pgfsetstrokeopacity{0.500000}%
\pgfsetdash{}{0pt}%
\pgfpathmoveto{\pgfqpoint{2.880975in}{1.679267in}}%
\pgfpathlineto{\pgfqpoint{2.892584in}{1.683252in}}%
\pgfpathlineto{\pgfqpoint{2.904188in}{1.687231in}}%
\pgfpathlineto{\pgfqpoint{2.915787in}{1.691203in}}%
\pgfpathlineto{\pgfqpoint{2.927379in}{1.695167in}}%
\pgfpathlineto{\pgfqpoint{2.938967in}{1.699120in}}%
\pgfpathlineto{\pgfqpoint{2.932663in}{1.710422in}}%
\pgfpathlineto{\pgfqpoint{2.926362in}{1.721666in}}%
\pgfpathlineto{\pgfqpoint{2.920065in}{1.732852in}}%
\pgfpathlineto{\pgfqpoint{2.913772in}{1.743980in}}%
\pgfpathlineto{\pgfqpoint{2.907483in}{1.755050in}}%
\pgfpathlineto{\pgfqpoint{2.895904in}{1.751112in}}%
\pgfpathlineto{\pgfqpoint{2.884320in}{1.747159in}}%
\pgfpathlineto{\pgfqpoint{2.872730in}{1.743199in}}%
\pgfpathlineto{\pgfqpoint{2.861135in}{1.739238in}}%
\pgfpathlineto{\pgfqpoint{2.849534in}{1.735278in}}%
\pgfpathlineto{\pgfqpoint{2.855815in}{1.724180in}}%
\pgfpathlineto{\pgfqpoint{2.862099in}{1.713030in}}%
\pgfpathlineto{\pgfqpoint{2.868387in}{1.701829in}}%
\pgfpathlineto{\pgfqpoint{2.874679in}{1.690575in}}%
\pgfpathclose%
\pgfusepath{stroke,fill}%
\end{pgfscope}%
\begin{pgfscope}%
\pgfpathrectangle{\pgfqpoint{0.887500in}{0.275000in}}{\pgfqpoint{4.225000in}{4.225000in}}%
\pgfusepath{clip}%
\pgfsetbuttcap%
\pgfsetroundjoin%
\definecolor{currentfill}{rgb}{0.195860,0.395433,0.555276}%
\pgfsetfillcolor{currentfill}%
\pgfsetfillopacity{0.700000}%
\pgfsetlinewidth{0.501875pt}%
\definecolor{currentstroke}{rgb}{1.000000,1.000000,1.000000}%
\pgfsetstrokecolor{currentstroke}%
\pgfsetstrokeopacity{0.500000}%
\pgfsetdash{}{0pt}%
\pgfpathmoveto{\pgfqpoint{2.375172in}{1.831718in}}%
\pgfpathlineto{\pgfqpoint{2.386906in}{1.835637in}}%
\pgfpathlineto{\pgfqpoint{2.398633in}{1.839556in}}%
\pgfpathlineto{\pgfqpoint{2.410355in}{1.843475in}}%
\pgfpathlineto{\pgfqpoint{2.422071in}{1.847399in}}%
\pgfpathlineto{\pgfqpoint{2.433782in}{1.851328in}}%
\pgfpathlineto{\pgfqpoint{2.427631in}{1.861646in}}%
\pgfpathlineto{\pgfqpoint{2.421484in}{1.871919in}}%
\pgfpathlineto{\pgfqpoint{2.415342in}{1.882146in}}%
\pgfpathlineto{\pgfqpoint{2.409205in}{1.892328in}}%
\pgfpathlineto{\pgfqpoint{2.403071in}{1.902463in}}%
\pgfpathlineto{\pgfqpoint{2.391371in}{1.898583in}}%
\pgfpathlineto{\pgfqpoint{2.379664in}{1.894709in}}%
\pgfpathlineto{\pgfqpoint{2.367952in}{1.890839in}}%
\pgfpathlineto{\pgfqpoint{2.356234in}{1.886971in}}%
\pgfpathlineto{\pgfqpoint{2.344510in}{1.883102in}}%
\pgfpathlineto{\pgfqpoint{2.350634in}{1.872914in}}%
\pgfpathlineto{\pgfqpoint{2.356762in}{1.862682in}}%
\pgfpathlineto{\pgfqpoint{2.362894in}{1.852406in}}%
\pgfpathlineto{\pgfqpoint{2.369031in}{1.842084in}}%
\pgfpathclose%
\pgfusepath{stroke,fill}%
\end{pgfscope}%
\begin{pgfscope}%
\pgfpathrectangle{\pgfqpoint{0.887500in}{0.275000in}}{\pgfqpoint{4.225000in}{4.225000in}}%
\pgfusepath{clip}%
\pgfsetbuttcap%
\pgfsetroundjoin%
\definecolor{currentfill}{rgb}{0.241237,0.296485,0.539709}%
\pgfsetfillcolor{currentfill}%
\pgfsetfillopacity{0.700000}%
\pgfsetlinewidth{0.501875pt}%
\definecolor{currentstroke}{rgb}{1.000000,1.000000,1.000000}%
\pgfsetstrokecolor{currentstroke}%
\pgfsetstrokeopacity{0.500000}%
\pgfsetdash{}{0pt}%
\pgfpathmoveto{\pgfqpoint{2.970540in}{1.641813in}}%
\pgfpathlineto{\pgfqpoint{2.982130in}{1.645784in}}%
\pgfpathlineto{\pgfqpoint{2.993714in}{1.649748in}}%
\pgfpathlineto{\pgfqpoint{3.005292in}{1.653705in}}%
\pgfpathlineto{\pgfqpoint{3.016865in}{1.657655in}}%
\pgfpathlineto{\pgfqpoint{3.028432in}{1.661596in}}%
\pgfpathlineto{\pgfqpoint{3.022102in}{1.673112in}}%
\pgfpathlineto{\pgfqpoint{3.015776in}{1.684581in}}%
\pgfpathlineto{\pgfqpoint{3.009454in}{1.696002in}}%
\pgfpathlineto{\pgfqpoint{3.003134in}{1.707370in}}%
\pgfpathlineto{\pgfqpoint{2.996819in}{1.718681in}}%
\pgfpathlineto{\pgfqpoint{2.985259in}{1.714802in}}%
\pgfpathlineto{\pgfqpoint{2.973695in}{1.710904in}}%
\pgfpathlineto{\pgfqpoint{2.962124in}{1.706991in}}%
\pgfpathlineto{\pgfqpoint{2.950548in}{1.703062in}}%
\pgfpathlineto{\pgfqpoint{2.938967in}{1.699120in}}%
\pgfpathlineto{\pgfqpoint{2.945274in}{1.687763in}}%
\pgfpathlineto{\pgfqpoint{2.951585in}{1.676353in}}%
\pgfpathlineto{\pgfqpoint{2.957900in}{1.664890in}}%
\pgfpathlineto{\pgfqpoint{2.964218in}{1.653376in}}%
\pgfpathclose%
\pgfusepath{stroke,fill}%
\end{pgfscope}%
\begin{pgfscope}%
\pgfpathrectangle{\pgfqpoint{0.887500in}{0.275000in}}{\pgfqpoint{4.225000in}{4.225000in}}%
\pgfusepath{clip}%
\pgfsetbuttcap%
\pgfsetroundjoin%
\definecolor{currentfill}{rgb}{0.281887,0.150881,0.465405}%
\pgfsetfillcolor{currentfill}%
\pgfsetfillopacity{0.700000}%
\pgfsetlinewidth{0.501875pt}%
\definecolor{currentstroke}{rgb}{1.000000,1.000000,1.000000}%
\pgfsetstrokecolor{currentstroke}%
\pgfsetstrokeopacity{0.500000}%
\pgfsetdash{}{0pt}%
\pgfpathmoveto{\pgfqpoint{3.508070in}{1.381750in}}%
\pgfpathlineto{\pgfqpoint{3.519534in}{1.386196in}}%
\pgfpathlineto{\pgfqpoint{3.530995in}{1.390804in}}%
\pgfpathlineto{\pgfqpoint{3.542453in}{1.395609in}}%
\pgfpathlineto{\pgfqpoint{3.553908in}{1.400629in}}%
\pgfpathlineto{\pgfqpoint{3.565362in}{1.405887in}}%
\pgfpathlineto{\pgfqpoint{3.558912in}{1.419368in}}%
\pgfpathlineto{\pgfqpoint{3.552463in}{1.432686in}}%
\pgfpathlineto{\pgfqpoint{3.546015in}{1.445853in}}%
\pgfpathlineto{\pgfqpoint{3.539568in}{1.458882in}}%
\pgfpathlineto{\pgfqpoint{3.533123in}{1.471787in}}%
\pgfpathlineto{\pgfqpoint{3.521672in}{1.466231in}}%
\pgfpathlineto{\pgfqpoint{3.510220in}{1.461028in}}%
\pgfpathlineto{\pgfqpoint{3.498767in}{1.456163in}}%
\pgfpathlineto{\pgfqpoint{3.487313in}{1.451619in}}%
\pgfpathlineto{\pgfqpoint{3.475856in}{1.447371in}}%
\pgfpathlineto{\pgfqpoint{3.482294in}{1.434325in}}%
\pgfpathlineto{\pgfqpoint{3.488734in}{1.421269in}}%
\pgfpathlineto{\pgfqpoint{3.495178in}{1.408174in}}%
\pgfpathlineto{\pgfqpoint{3.501623in}{1.395010in}}%
\pgfpathclose%
\pgfusepath{stroke,fill}%
\end{pgfscope}%
\begin{pgfscope}%
\pgfpathrectangle{\pgfqpoint{0.887500in}{0.275000in}}{\pgfqpoint{4.225000in}{4.225000in}}%
\pgfusepath{clip}%
\pgfsetbuttcap%
\pgfsetroundjoin%
\definecolor{currentfill}{rgb}{0.283187,0.125848,0.444960}%
\pgfsetfillcolor{currentfill}%
\pgfsetfillopacity{0.700000}%
\pgfsetlinewidth{0.501875pt}%
\definecolor{currentstroke}{rgb}{1.000000,1.000000,1.000000}%
\pgfsetstrokecolor{currentstroke}%
\pgfsetstrokeopacity{0.500000}%
\pgfsetdash{}{0pt}%
\pgfpathmoveto{\pgfqpoint{3.597615in}{1.335548in}}%
\pgfpathlineto{\pgfqpoint{3.609057in}{1.340031in}}%
\pgfpathlineto{\pgfqpoint{3.620494in}{1.344471in}}%
\pgfpathlineto{\pgfqpoint{3.631925in}{1.348853in}}%
\pgfpathlineto{\pgfqpoint{3.643349in}{1.353157in}}%
\pgfpathlineto{\pgfqpoint{3.654766in}{1.357341in}}%
\pgfpathlineto{\pgfqpoint{3.648338in}{1.373458in}}%
\pgfpathlineto{\pgfqpoint{3.641910in}{1.389467in}}%
\pgfpathlineto{\pgfqpoint{3.635481in}{1.405296in}}%
\pgfpathlineto{\pgfqpoint{3.629051in}{1.420877in}}%
\pgfpathlineto{\pgfqpoint{3.622618in}{1.436137in}}%
\pgfpathlineto{\pgfqpoint{3.611169in}{1.429645in}}%
\pgfpathlineto{\pgfqpoint{3.599718in}{1.423280in}}%
\pgfpathlineto{\pgfqpoint{3.588267in}{1.417191in}}%
\pgfpathlineto{\pgfqpoint{3.576815in}{1.411400in}}%
\pgfpathlineto{\pgfqpoint{3.565362in}{1.405887in}}%
\pgfpathlineto{\pgfqpoint{3.571813in}{1.392227in}}%
\pgfpathlineto{\pgfqpoint{3.578264in}{1.378377in}}%
\pgfpathlineto{\pgfqpoint{3.584714in}{1.364322in}}%
\pgfpathlineto{\pgfqpoint{3.591165in}{1.350050in}}%
\pgfpathclose%
\pgfusepath{stroke,fill}%
\end{pgfscope}%
\begin{pgfscope}%
\pgfpathrectangle{\pgfqpoint{0.887500in}{0.275000in}}{\pgfqpoint{4.225000in}{4.225000in}}%
\pgfusepath{clip}%
\pgfsetbuttcap%
\pgfsetroundjoin%
\definecolor{currentfill}{rgb}{0.248629,0.278775,0.534556}%
\pgfsetfillcolor{currentfill}%
\pgfsetfillopacity{0.700000}%
\pgfsetlinewidth{0.501875pt}%
\definecolor{currentstroke}{rgb}{1.000000,1.000000,1.000000}%
\pgfsetstrokecolor{currentstroke}%
\pgfsetstrokeopacity{0.500000}%
\pgfsetdash{}{0pt}%
\pgfpathmoveto{\pgfqpoint{3.060131in}{1.603433in}}%
\pgfpathlineto{\pgfqpoint{3.071701in}{1.607336in}}%
\pgfpathlineto{\pgfqpoint{3.083264in}{1.611206in}}%
\pgfpathlineto{\pgfqpoint{3.094822in}{1.615050in}}%
\pgfpathlineto{\pgfqpoint{3.106374in}{1.618877in}}%
\pgfpathlineto{\pgfqpoint{3.117921in}{1.622693in}}%
\pgfpathlineto{\pgfqpoint{3.111567in}{1.634508in}}%
\pgfpathlineto{\pgfqpoint{3.105216in}{1.646251in}}%
\pgfpathlineto{\pgfqpoint{3.098869in}{1.657923in}}%
\pgfpathlineto{\pgfqpoint{3.092525in}{1.669527in}}%
\pgfpathlineto{\pgfqpoint{3.086184in}{1.681061in}}%
\pgfpathlineto{\pgfqpoint{3.074645in}{1.677212in}}%
\pgfpathlineto{\pgfqpoint{3.063100in}{1.673337in}}%
\pgfpathlineto{\pgfqpoint{3.051550in}{1.669441in}}%
\pgfpathlineto{\pgfqpoint{3.039994in}{1.665526in}}%
\pgfpathlineto{\pgfqpoint{3.028432in}{1.661596in}}%
\pgfpathlineto{\pgfqpoint{3.034765in}{1.650038in}}%
\pgfpathlineto{\pgfqpoint{3.041102in}{1.638442in}}%
\pgfpathlineto{\pgfqpoint{3.047441in}{1.626811in}}%
\pgfpathlineto{\pgfqpoint{3.053785in}{1.615146in}}%
\pgfpathclose%
\pgfusepath{stroke,fill}%
\end{pgfscope}%
\begin{pgfscope}%
\pgfpathrectangle{\pgfqpoint{0.887500in}{0.275000in}}{\pgfqpoint{4.225000in}{4.225000in}}%
\pgfusepath{clip}%
\pgfsetbuttcap%
\pgfsetroundjoin%
\definecolor{currentfill}{rgb}{0.180629,0.429975,0.557282}%
\pgfsetfillcolor{currentfill}%
\pgfsetfillopacity{0.700000}%
\pgfsetlinewidth{0.501875pt}%
\definecolor{currentstroke}{rgb}{1.000000,1.000000,1.000000}%
\pgfsetstrokecolor{currentstroke}%
\pgfsetstrokeopacity{0.500000}%
\pgfsetdash{}{0pt}%
\pgfpathmoveto{\pgfqpoint{2.048258in}{1.905918in}}%
\pgfpathlineto{\pgfqpoint{2.060075in}{1.909794in}}%
\pgfpathlineto{\pgfqpoint{2.071886in}{1.913669in}}%
\pgfpathlineto{\pgfqpoint{2.083691in}{1.917544in}}%
\pgfpathlineto{\pgfqpoint{2.095490in}{1.921420in}}%
\pgfpathlineto{\pgfqpoint{2.107284in}{1.925297in}}%
\pgfpathlineto{\pgfqpoint{2.101239in}{1.935177in}}%
\pgfpathlineto{\pgfqpoint{2.095199in}{1.945017in}}%
\pgfpathlineto{\pgfqpoint{2.089163in}{1.954818in}}%
\pgfpathlineto{\pgfqpoint{2.083132in}{1.964580in}}%
\pgfpathlineto{\pgfqpoint{2.077106in}{1.974303in}}%
\pgfpathlineto{\pgfqpoint{2.065322in}{1.970475in}}%
\pgfpathlineto{\pgfqpoint{2.053533in}{1.966648in}}%
\pgfpathlineto{\pgfqpoint{2.041738in}{1.962822in}}%
\pgfpathlineto{\pgfqpoint{2.029937in}{1.958997in}}%
\pgfpathlineto{\pgfqpoint{2.018131in}{1.955170in}}%
\pgfpathlineto{\pgfqpoint{2.024147in}{1.945394in}}%
\pgfpathlineto{\pgfqpoint{2.030168in}{1.935582in}}%
\pgfpathlineto{\pgfqpoint{2.036193in}{1.925732in}}%
\pgfpathlineto{\pgfqpoint{2.042223in}{1.915844in}}%
\pgfpathclose%
\pgfusepath{stroke,fill}%
\end{pgfscope}%
\begin{pgfscope}%
\pgfpathrectangle{\pgfqpoint{0.887500in}{0.275000in}}{\pgfqpoint{4.225000in}{4.225000in}}%
\pgfusepath{clip}%
\pgfsetbuttcap%
\pgfsetroundjoin%
\definecolor{currentfill}{rgb}{0.278012,0.180367,0.486697}%
\pgfsetfillcolor{currentfill}%
\pgfsetfillopacity{0.700000}%
\pgfsetlinewidth{0.501875pt}%
\definecolor{currentstroke}{rgb}{1.000000,1.000000,1.000000}%
\pgfsetstrokecolor{currentstroke}%
\pgfsetstrokeopacity{0.500000}%
\pgfsetdash{}{0pt}%
\pgfpathmoveto{\pgfqpoint{3.418515in}{1.428306in}}%
\pgfpathlineto{\pgfqpoint{3.429994in}{1.432055in}}%
\pgfpathlineto{\pgfqpoint{3.441467in}{1.435769in}}%
\pgfpathlineto{\pgfqpoint{3.452934in}{1.439514in}}%
\pgfpathlineto{\pgfqpoint{3.464397in}{1.443359in}}%
\pgfpathlineto{\pgfqpoint{3.475856in}{1.447371in}}%
\pgfpathlineto{\pgfqpoint{3.469422in}{1.460435in}}%
\pgfpathlineto{\pgfqpoint{3.462991in}{1.473538in}}%
\pgfpathlineto{\pgfqpoint{3.456564in}{1.486669in}}%
\pgfpathlineto{\pgfqpoint{3.450139in}{1.499812in}}%
\pgfpathlineto{\pgfqpoint{3.443718in}{1.512950in}}%
\pgfpathlineto{\pgfqpoint{3.432266in}{1.509155in}}%
\pgfpathlineto{\pgfqpoint{3.420810in}{1.505529in}}%
\pgfpathlineto{\pgfqpoint{3.409349in}{1.502025in}}%
\pgfpathlineto{\pgfqpoint{3.397883in}{1.498595in}}%
\pgfpathlineto{\pgfqpoint{3.386412in}{1.495192in}}%
\pgfpathlineto{\pgfqpoint{3.392828in}{1.482005in}}%
\pgfpathlineto{\pgfqpoint{3.399247in}{1.468722in}}%
\pgfpathlineto{\pgfqpoint{3.405667in}{1.455343in}}%
\pgfpathlineto{\pgfqpoint{3.412090in}{1.441871in}}%
\pgfpathclose%
\pgfusepath{stroke,fill}%
\end{pgfscope}%
\begin{pgfscope}%
\pgfpathrectangle{\pgfqpoint{0.887500in}{0.275000in}}{\pgfqpoint{4.225000in}{4.225000in}}%
\pgfusepath{clip}%
\pgfsetbuttcap%
\pgfsetroundjoin%
\definecolor{currentfill}{rgb}{0.203063,0.379716,0.553925}%
\pgfsetfillcolor{currentfill}%
\pgfsetfillopacity{0.700000}%
\pgfsetlinewidth{0.501875pt}%
\definecolor{currentstroke}{rgb}{1.000000,1.000000,1.000000}%
\pgfsetstrokecolor{currentstroke}%
\pgfsetstrokeopacity{0.500000}%
\pgfsetdash{}{0pt}%
\pgfpathmoveto{\pgfqpoint{2.464600in}{1.799050in}}%
\pgfpathlineto{\pgfqpoint{2.476314in}{1.803027in}}%
\pgfpathlineto{\pgfqpoint{2.488022in}{1.807005in}}%
\pgfpathlineto{\pgfqpoint{2.499725in}{1.810982in}}%
\pgfpathlineto{\pgfqpoint{2.511422in}{1.814954in}}%
\pgfpathlineto{\pgfqpoint{2.523113in}{1.818919in}}%
\pgfpathlineto{\pgfqpoint{2.516932in}{1.829423in}}%
\pgfpathlineto{\pgfqpoint{2.510755in}{1.839880in}}%
\pgfpathlineto{\pgfqpoint{2.504581in}{1.850290in}}%
\pgfpathlineto{\pgfqpoint{2.498413in}{1.860654in}}%
\pgfpathlineto{\pgfqpoint{2.492248in}{1.870972in}}%
\pgfpathlineto{\pgfqpoint{2.480566in}{1.867054in}}%
\pgfpathlineto{\pgfqpoint{2.468878in}{1.863127in}}%
\pgfpathlineto{\pgfqpoint{2.457185in}{1.859195in}}%
\pgfpathlineto{\pgfqpoint{2.445486in}{1.855261in}}%
\pgfpathlineto{\pgfqpoint{2.433782in}{1.851328in}}%
\pgfpathlineto{\pgfqpoint{2.439937in}{1.840963in}}%
\pgfpathlineto{\pgfqpoint{2.446096in}{1.830553in}}%
\pgfpathlineto{\pgfqpoint{2.452260in}{1.820098in}}%
\pgfpathlineto{\pgfqpoint{2.458427in}{1.809597in}}%
\pgfpathclose%
\pgfusepath{stroke,fill}%
\end{pgfscope}%
\begin{pgfscope}%
\pgfpathrectangle{\pgfqpoint{0.887500in}{0.275000in}}{\pgfqpoint{4.225000in}{4.225000in}}%
\pgfusepath{clip}%
\pgfsetbuttcap%
\pgfsetroundjoin%
\definecolor{currentfill}{rgb}{0.257322,0.256130,0.526563}%
\pgfsetfillcolor{currentfill}%
\pgfsetfillopacity{0.700000}%
\pgfsetlinewidth{0.501875pt}%
\definecolor{currentstroke}{rgb}{1.000000,1.000000,1.000000}%
\pgfsetstrokecolor{currentstroke}%
\pgfsetstrokeopacity{0.500000}%
\pgfsetdash{}{0pt}%
\pgfpathmoveto{\pgfqpoint{3.149739in}{1.562481in}}%
\pgfpathlineto{\pgfqpoint{3.161286in}{1.566251in}}%
\pgfpathlineto{\pgfqpoint{3.172828in}{1.570045in}}%
\pgfpathlineto{\pgfqpoint{3.184365in}{1.573867in}}%
\pgfpathlineto{\pgfqpoint{3.195896in}{1.577725in}}%
\pgfpathlineto{\pgfqpoint{3.207421in}{1.581621in}}%
\pgfpathlineto{\pgfqpoint{3.201044in}{1.593778in}}%
\pgfpathlineto{\pgfqpoint{3.194671in}{1.605896in}}%
\pgfpathlineto{\pgfqpoint{3.188301in}{1.617963in}}%
\pgfpathlineto{\pgfqpoint{3.181933in}{1.629969in}}%
\pgfpathlineto{\pgfqpoint{3.175569in}{1.641902in}}%
\pgfpathlineto{\pgfqpoint{3.164051in}{1.638013in}}%
\pgfpathlineto{\pgfqpoint{3.152527in}{1.634156in}}%
\pgfpathlineto{\pgfqpoint{3.140997in}{1.630324in}}%
\pgfpathlineto{\pgfqpoint{3.129462in}{1.626505in}}%
\pgfpathlineto{\pgfqpoint{3.117921in}{1.622693in}}%
\pgfpathlineto{\pgfqpoint{3.124278in}{1.610804in}}%
\pgfpathlineto{\pgfqpoint{3.130639in}{1.598840in}}%
\pgfpathlineto{\pgfqpoint{3.137002in}{1.586799in}}%
\pgfpathlineto{\pgfqpoint{3.143369in}{1.574680in}}%
\pgfpathclose%
\pgfusepath{stroke,fill}%
\end{pgfscope}%
\begin{pgfscope}%
\pgfpathrectangle{\pgfqpoint{0.887500in}{0.275000in}}{\pgfqpoint{4.225000in}{4.225000in}}%
\pgfusepath{clip}%
\pgfsetbuttcap%
\pgfsetroundjoin%
\definecolor{currentfill}{rgb}{0.265145,0.232956,0.516599}%
\pgfsetfillcolor{currentfill}%
\pgfsetfillopacity{0.700000}%
\pgfsetlinewidth{0.501875pt}%
\definecolor{currentstroke}{rgb}{1.000000,1.000000,1.000000}%
\pgfsetstrokecolor{currentstroke}%
\pgfsetstrokeopacity{0.500000}%
\pgfsetdash{}{0pt}%
\pgfpathmoveto{\pgfqpoint{3.239351in}{1.520602in}}%
\pgfpathlineto{\pgfqpoint{3.250878in}{1.524630in}}%
\pgfpathlineto{\pgfqpoint{3.262400in}{1.528646in}}%
\pgfpathlineto{\pgfqpoint{3.273916in}{1.532653in}}%
\pgfpathlineto{\pgfqpoint{3.285426in}{1.536642in}}%
\pgfpathlineto{\pgfqpoint{3.296930in}{1.540605in}}%
\pgfpathlineto{\pgfqpoint{3.290531in}{1.552886in}}%
\pgfpathlineto{\pgfqpoint{3.284136in}{1.565138in}}%
\pgfpathlineto{\pgfqpoint{3.277743in}{1.577371in}}%
\pgfpathlineto{\pgfqpoint{3.271353in}{1.589581in}}%
\pgfpathlineto{\pgfqpoint{3.264967in}{1.601762in}}%
\pgfpathlineto{\pgfqpoint{3.253469in}{1.597672in}}%
\pgfpathlineto{\pgfqpoint{3.241965in}{1.593592in}}%
\pgfpathlineto{\pgfqpoint{3.230455in}{1.589550in}}%
\pgfpathlineto{\pgfqpoint{3.218941in}{1.585561in}}%
\pgfpathlineto{\pgfqpoint{3.207421in}{1.581621in}}%
\pgfpathlineto{\pgfqpoint{3.213801in}{1.569435in}}%
\pgfpathlineto{\pgfqpoint{3.220183in}{1.557232in}}%
\pgfpathlineto{\pgfqpoint{3.226569in}{1.545023in}}%
\pgfpathlineto{\pgfqpoint{3.232959in}{1.532818in}}%
\pgfpathclose%
\pgfusepath{stroke,fill}%
\end{pgfscope}%
\begin{pgfscope}%
\pgfpathrectangle{\pgfqpoint{0.887500in}{0.275000in}}{\pgfqpoint{4.225000in}{4.225000in}}%
\pgfusepath{clip}%
\pgfsetbuttcap%
\pgfsetroundjoin%
\definecolor{currentfill}{rgb}{0.271828,0.209303,0.504434}%
\pgfsetfillcolor{currentfill}%
\pgfsetfillopacity{0.700000}%
\pgfsetlinewidth{0.501875pt}%
\definecolor{currentstroke}{rgb}{1.000000,1.000000,1.000000}%
\pgfsetstrokecolor{currentstroke}%
\pgfsetstrokeopacity{0.500000}%
\pgfsetdash{}{0pt}%
\pgfpathmoveto{\pgfqpoint{3.328961in}{1.477450in}}%
\pgfpathlineto{\pgfqpoint{3.340464in}{1.481135in}}%
\pgfpathlineto{\pgfqpoint{3.351960in}{1.484747in}}%
\pgfpathlineto{\pgfqpoint{3.363451in}{1.488290in}}%
\pgfpathlineto{\pgfqpoint{3.374934in}{1.491769in}}%
\pgfpathlineto{\pgfqpoint{3.386412in}{1.495192in}}%
\pgfpathlineto{\pgfqpoint{3.379998in}{1.508281in}}%
\pgfpathlineto{\pgfqpoint{3.373586in}{1.521269in}}%
\pgfpathlineto{\pgfqpoint{3.367176in}{1.534157in}}%
\pgfpathlineto{\pgfqpoint{3.360768in}{1.546942in}}%
\pgfpathlineto{\pgfqpoint{3.354363in}{1.559622in}}%
\pgfpathlineto{\pgfqpoint{3.342889in}{1.555958in}}%
\pgfpathlineto{\pgfqpoint{3.331408in}{1.552216in}}%
\pgfpathlineto{\pgfqpoint{3.319922in}{1.548403in}}%
\pgfpathlineto{\pgfqpoint{3.308429in}{1.544528in}}%
\pgfpathlineto{\pgfqpoint{3.296930in}{1.540605in}}%
\pgfpathlineto{\pgfqpoint{3.303332in}{1.528256in}}%
\pgfpathlineto{\pgfqpoint{3.309736in}{1.515804in}}%
\pgfpathlineto{\pgfqpoint{3.316143in}{1.503210in}}%
\pgfpathlineto{\pgfqpoint{3.322551in}{1.490438in}}%
\pgfpathclose%
\pgfusepath{stroke,fill}%
\end{pgfscope}%
\begin{pgfscope}%
\pgfpathrectangle{\pgfqpoint{0.887500in}{0.275000in}}{\pgfqpoint{4.225000in}{4.225000in}}%
\pgfusepath{clip}%
\pgfsetbuttcap%
\pgfsetroundjoin%
\definecolor{currentfill}{rgb}{0.210503,0.363727,0.552206}%
\pgfsetfillcolor{currentfill}%
\pgfsetfillopacity{0.700000}%
\pgfsetlinewidth{0.501875pt}%
\definecolor{currentstroke}{rgb}{1.000000,1.000000,1.000000}%
\pgfsetstrokecolor{currentstroke}%
\pgfsetstrokeopacity{0.500000}%
\pgfsetdash{}{0pt}%
\pgfpathmoveto{\pgfqpoint{2.554083in}{1.765692in}}%
\pgfpathlineto{\pgfqpoint{2.565779in}{1.769691in}}%
\pgfpathlineto{\pgfqpoint{2.577468in}{1.773680in}}%
\pgfpathlineto{\pgfqpoint{2.589153in}{1.777657in}}%
\pgfpathlineto{\pgfqpoint{2.600831in}{1.781621in}}%
\pgfpathlineto{\pgfqpoint{2.612505in}{1.785570in}}%
\pgfpathlineto{\pgfqpoint{2.606293in}{1.796266in}}%
\pgfpathlineto{\pgfqpoint{2.600086in}{1.806911in}}%
\pgfpathlineto{\pgfqpoint{2.593883in}{1.817508in}}%
\pgfpathlineto{\pgfqpoint{2.587684in}{1.828055in}}%
\pgfpathlineto{\pgfqpoint{2.581489in}{1.838554in}}%
\pgfpathlineto{\pgfqpoint{2.569825in}{1.834658in}}%
\pgfpathlineto{\pgfqpoint{2.558155in}{1.830745in}}%
\pgfpathlineto{\pgfqpoint{2.546480in}{1.826817in}}%
\pgfpathlineto{\pgfqpoint{2.534799in}{1.822874in}}%
\pgfpathlineto{\pgfqpoint{2.523113in}{1.818919in}}%
\pgfpathlineto{\pgfqpoint{2.529299in}{1.808369in}}%
\pgfpathlineto{\pgfqpoint{2.535489in}{1.797771in}}%
\pgfpathlineto{\pgfqpoint{2.541683in}{1.787126in}}%
\pgfpathlineto{\pgfqpoint{2.547881in}{1.776433in}}%
\pgfpathclose%
\pgfusepath{stroke,fill}%
\end{pgfscope}%
\begin{pgfscope}%
\pgfpathrectangle{\pgfqpoint{0.887500in}{0.275000in}}{\pgfqpoint{4.225000in}{4.225000in}}%
\pgfusepath{clip}%
\pgfsetbuttcap%
\pgfsetroundjoin%
\definecolor{currentfill}{rgb}{0.185556,0.418570,0.556753}%
\pgfsetfillcolor{currentfill}%
\pgfsetfillopacity{0.700000}%
\pgfsetlinewidth{0.501875pt}%
\definecolor{currentstroke}{rgb}{1.000000,1.000000,1.000000}%
\pgfsetstrokecolor{currentstroke}%
\pgfsetstrokeopacity{0.500000}%
\pgfsetdash{}{0pt}%
\pgfpathmoveto{\pgfqpoint{2.137576in}{1.875298in}}%
\pgfpathlineto{\pgfqpoint{2.149374in}{1.879218in}}%
\pgfpathlineto{\pgfqpoint{2.161166in}{1.883136in}}%
\pgfpathlineto{\pgfqpoint{2.172953in}{1.887049in}}%
\pgfpathlineto{\pgfqpoint{2.184734in}{1.890957in}}%
\pgfpathlineto{\pgfqpoint{2.196509in}{1.894858in}}%
\pgfpathlineto{\pgfqpoint{2.190432in}{1.904895in}}%
\pgfpathlineto{\pgfqpoint{2.184359in}{1.914889in}}%
\pgfpathlineto{\pgfqpoint{2.178290in}{1.924843in}}%
\pgfpathlineto{\pgfqpoint{2.172226in}{1.934755in}}%
\pgfpathlineto{\pgfqpoint{2.166167in}{1.944626in}}%
\pgfpathlineto{\pgfqpoint{2.154401in}{1.940775in}}%
\pgfpathlineto{\pgfqpoint{2.142630in}{1.936914in}}%
\pgfpathlineto{\pgfqpoint{2.130854in}{1.933046in}}%
\pgfpathlineto{\pgfqpoint{2.119072in}{1.929173in}}%
\pgfpathlineto{\pgfqpoint{2.107284in}{1.925297in}}%
\pgfpathlineto{\pgfqpoint{2.113333in}{1.915378in}}%
\pgfpathlineto{\pgfqpoint{2.119387in}{1.905419in}}%
\pgfpathlineto{\pgfqpoint{2.125445in}{1.895419in}}%
\pgfpathlineto{\pgfqpoint{2.131508in}{1.885379in}}%
\pgfpathclose%
\pgfusepath{stroke,fill}%
\end{pgfscope}%
\begin{pgfscope}%
\pgfpathrectangle{\pgfqpoint{0.887500in}{0.275000in}}{\pgfqpoint{4.225000in}{4.225000in}}%
\pgfusepath{clip}%
\pgfsetbuttcap%
\pgfsetroundjoin%
\definecolor{currentfill}{rgb}{0.216210,0.351535,0.550627}%
\pgfsetfillcolor{currentfill}%
\pgfsetfillopacity{0.700000}%
\pgfsetlinewidth{0.501875pt}%
\definecolor{currentstroke}{rgb}{1.000000,1.000000,1.000000}%
\pgfsetstrokecolor{currentstroke}%
\pgfsetstrokeopacity{0.500000}%
\pgfsetdash{}{0pt}%
\pgfpathmoveto{\pgfqpoint{2.643623in}{1.731339in}}%
\pgfpathlineto{\pgfqpoint{2.655299in}{1.735318in}}%
\pgfpathlineto{\pgfqpoint{2.666971in}{1.739281in}}%
\pgfpathlineto{\pgfqpoint{2.678636in}{1.743226in}}%
\pgfpathlineto{\pgfqpoint{2.690296in}{1.747154in}}%
\pgfpathlineto{\pgfqpoint{2.701951in}{1.751066in}}%
\pgfpathlineto{\pgfqpoint{2.695710in}{1.761966in}}%
\pgfpathlineto{\pgfqpoint{2.689474in}{1.772815in}}%
\pgfpathlineto{\pgfqpoint{2.683241in}{1.783613in}}%
\pgfpathlineto{\pgfqpoint{2.677013in}{1.794358in}}%
\pgfpathlineto{\pgfqpoint{2.670788in}{1.805051in}}%
\pgfpathlineto{\pgfqpoint{2.659143in}{1.801191in}}%
\pgfpathlineto{\pgfqpoint{2.647491in}{1.797315in}}%
\pgfpathlineto{\pgfqpoint{2.635835in}{1.793418in}}%
\pgfpathlineto{\pgfqpoint{2.624172in}{1.789503in}}%
\pgfpathlineto{\pgfqpoint{2.612505in}{1.785570in}}%
\pgfpathlineto{\pgfqpoint{2.618720in}{1.774825in}}%
\pgfpathlineto{\pgfqpoint{2.624940in}{1.764029in}}%
\pgfpathlineto{\pgfqpoint{2.631163in}{1.753182in}}%
\pgfpathlineto{\pgfqpoint{2.637391in}{1.742285in}}%
\pgfpathclose%
\pgfusepath{stroke,fill}%
\end{pgfscope}%
\begin{pgfscope}%
\pgfpathrectangle{\pgfqpoint{0.887500in}{0.275000in}}{\pgfqpoint{4.225000in}{4.225000in}}%
\pgfusepath{clip}%
\pgfsetbuttcap%
\pgfsetroundjoin%
\definecolor{currentfill}{rgb}{0.274952,0.037752,0.364543}%
\pgfsetfillcolor{currentfill}%
\pgfsetfillopacity{0.700000}%
\pgfsetlinewidth{0.501875pt}%
\definecolor{currentstroke}{rgb}{1.000000,1.000000,1.000000}%
\pgfsetstrokecolor{currentstroke}%
\pgfsetstrokeopacity{0.500000}%
\pgfsetdash{}{0pt}%
\pgfpathmoveto{\pgfqpoint{3.719325in}{1.205626in}}%
\pgfpathlineto{\pgfqpoint{3.730783in}{1.212238in}}%
\pgfpathlineto{\pgfqpoint{3.742239in}{1.218960in}}%
\pgfpathlineto{\pgfqpoint{3.753694in}{1.225777in}}%
\pgfpathlineto{\pgfqpoint{3.765145in}{1.232672in}}%
\pgfpathlineto{\pgfqpoint{3.776594in}{1.239628in}}%
\pgfpathlineto{\pgfqpoint{3.769982in}{1.247204in}}%
\pgfpathlineto{\pgfqpoint{3.763407in}{1.256529in}}%
\pgfpathlineto{\pgfqpoint{3.756866in}{1.267445in}}%
\pgfpathlineto{\pgfqpoint{3.750354in}{1.279791in}}%
\pgfpathlineto{\pgfqpoint{3.743869in}{1.293406in}}%
\pgfpathlineto{\pgfqpoint{3.732494in}{1.290052in}}%
\pgfpathlineto{\pgfqpoint{3.721116in}{1.286867in}}%
\pgfpathlineto{\pgfqpoint{3.709734in}{1.283776in}}%
\pgfpathlineto{\pgfqpoint{3.698346in}{1.280707in}}%
\pgfpathlineto{\pgfqpoint{3.686952in}{1.277584in}}%
\pgfpathlineto{\pgfqpoint{3.693406in}{1.262289in}}%
\pgfpathlineto{\pgfqpoint{3.699869in}{1.247376in}}%
\pgfpathlineto{\pgfqpoint{3.706342in}{1.232914in}}%
\pgfpathlineto{\pgfqpoint{3.712827in}{1.218975in}}%
\pgfpathclose%
\pgfusepath{stroke,fill}%
\end{pgfscope}%
\begin{pgfscope}%
\pgfpathrectangle{\pgfqpoint{0.887500in}{0.275000in}}{\pgfqpoint{4.225000in}{4.225000in}}%
\pgfusepath{clip}%
\pgfsetbuttcap%
\pgfsetroundjoin%
\definecolor{currentfill}{rgb}{0.190631,0.407061,0.556089}%
\pgfsetfillcolor{currentfill}%
\pgfsetfillopacity{0.700000}%
\pgfsetlinewidth{0.501875pt}%
\definecolor{currentstroke}{rgb}{1.000000,1.000000,1.000000}%
\pgfsetstrokecolor{currentstroke}%
\pgfsetstrokeopacity{0.500000}%
\pgfsetdash{}{0pt}%
\pgfpathmoveto{\pgfqpoint{2.226964in}{1.844046in}}%
\pgfpathlineto{\pgfqpoint{2.238744in}{1.847987in}}%
\pgfpathlineto{\pgfqpoint{2.250518in}{1.851920in}}%
\pgfpathlineto{\pgfqpoint{2.262287in}{1.855845in}}%
\pgfpathlineto{\pgfqpoint{2.274050in}{1.859763in}}%
\pgfpathlineto{\pgfqpoint{2.285807in}{1.863673in}}%
\pgfpathlineto{\pgfqpoint{2.279697in}{1.873870in}}%
\pgfpathlineto{\pgfqpoint{2.273592in}{1.884024in}}%
\pgfpathlineto{\pgfqpoint{2.267491in}{1.894134in}}%
\pgfpathlineto{\pgfqpoint{2.261395in}{1.904201in}}%
\pgfpathlineto{\pgfqpoint{2.255303in}{1.914225in}}%
\pgfpathlineto{\pgfqpoint{2.243555in}{1.910371in}}%
\pgfpathlineto{\pgfqpoint{2.231802in}{1.906507in}}%
\pgfpathlineto{\pgfqpoint{2.220043in}{1.902633in}}%
\pgfpathlineto{\pgfqpoint{2.208279in}{1.898750in}}%
\pgfpathlineto{\pgfqpoint{2.196509in}{1.894858in}}%
\pgfpathlineto{\pgfqpoint{2.202591in}{1.884780in}}%
\pgfpathlineto{\pgfqpoint{2.208678in}{1.874659in}}%
\pgfpathlineto{\pgfqpoint{2.214769in}{1.864497in}}%
\pgfpathlineto{\pgfqpoint{2.220864in}{1.854293in}}%
\pgfpathclose%
\pgfusepath{stroke,fill}%
\end{pgfscope}%
\begin{pgfscope}%
\pgfpathrectangle{\pgfqpoint{0.887500in}{0.275000in}}{\pgfqpoint{4.225000in}{4.225000in}}%
\pgfusepath{clip}%
\pgfsetbuttcap%
\pgfsetroundjoin%
\definecolor{currentfill}{rgb}{0.223925,0.334994,0.548053}%
\pgfsetfillcolor{currentfill}%
\pgfsetfillopacity{0.700000}%
\pgfsetlinewidth{0.501875pt}%
\definecolor{currentstroke}{rgb}{1.000000,1.000000,1.000000}%
\pgfsetstrokecolor{currentstroke}%
\pgfsetstrokeopacity{0.500000}%
\pgfsetdash{}{0pt}%
\pgfpathmoveto{\pgfqpoint{2.733212in}{1.695832in}}%
\pgfpathlineto{\pgfqpoint{2.744870in}{1.699779in}}%
\pgfpathlineto{\pgfqpoint{2.756522in}{1.703719in}}%
\pgfpathlineto{\pgfqpoint{2.768168in}{1.707654in}}%
\pgfpathlineto{\pgfqpoint{2.779809in}{1.711589in}}%
\pgfpathlineto{\pgfqpoint{2.791444in}{1.715527in}}%
\pgfpathlineto{\pgfqpoint{2.785175in}{1.726622in}}%
\pgfpathlineto{\pgfqpoint{2.778911in}{1.737668in}}%
\pgfpathlineto{\pgfqpoint{2.772650in}{1.748664in}}%
\pgfpathlineto{\pgfqpoint{2.766393in}{1.759611in}}%
\pgfpathlineto{\pgfqpoint{2.760140in}{1.770507in}}%
\pgfpathlineto{\pgfqpoint{2.748513in}{1.766624in}}%
\pgfpathlineto{\pgfqpoint{2.736881in}{1.762741in}}%
\pgfpathlineto{\pgfqpoint{2.725243in}{1.758856in}}%
\pgfpathlineto{\pgfqpoint{2.713600in}{1.754965in}}%
\pgfpathlineto{\pgfqpoint{2.701951in}{1.751066in}}%
\pgfpathlineto{\pgfqpoint{2.708195in}{1.740116in}}%
\pgfpathlineto{\pgfqpoint{2.714444in}{1.729116in}}%
\pgfpathlineto{\pgfqpoint{2.720696in}{1.718068in}}%
\pgfpathlineto{\pgfqpoint{2.726952in}{1.706973in}}%
\pgfpathclose%
\pgfusepath{stroke,fill}%
\end{pgfscope}%
\begin{pgfscope}%
\pgfpathrectangle{\pgfqpoint{0.887500in}{0.275000in}}{\pgfqpoint{4.225000in}{4.225000in}}%
\pgfusepath{clip}%
\pgfsetbuttcap%
\pgfsetroundjoin%
\definecolor{currentfill}{rgb}{0.281446,0.084320,0.407414}%
\pgfsetfillcolor{currentfill}%
\pgfsetfillopacity{0.700000}%
\pgfsetlinewidth{0.501875pt}%
\definecolor{currentstroke}{rgb}{1.000000,1.000000,1.000000}%
\pgfsetstrokecolor{currentstroke}%
\pgfsetstrokeopacity{0.500000}%
\pgfsetdash{}{0pt}%
\pgfpathmoveto{\pgfqpoint{3.629844in}{1.259113in}}%
\pgfpathlineto{\pgfqpoint{3.641284in}{1.263236in}}%
\pgfpathlineto{\pgfqpoint{3.652715in}{1.267177in}}%
\pgfpathlineto{\pgfqpoint{3.664137in}{1.270885in}}%
\pgfpathlineto{\pgfqpoint{3.675549in}{1.274335in}}%
\pgfpathlineto{\pgfqpoint{3.686952in}{1.277584in}}%
\pgfpathlineto{\pgfqpoint{3.680505in}{1.293191in}}%
\pgfpathlineto{\pgfqpoint{3.674064in}{1.309041in}}%
\pgfpathlineto{\pgfqpoint{3.667628in}{1.325062in}}%
\pgfpathlineto{\pgfqpoint{3.661196in}{1.341186in}}%
\pgfpathlineto{\pgfqpoint{3.654766in}{1.357341in}}%
\pgfpathlineto{\pgfqpoint{3.643349in}{1.353157in}}%
\pgfpathlineto{\pgfqpoint{3.631925in}{1.348853in}}%
\pgfpathlineto{\pgfqpoint{3.620494in}{1.344471in}}%
\pgfpathlineto{\pgfqpoint{3.609057in}{1.340031in}}%
\pgfpathlineto{\pgfqpoint{3.597615in}{1.335548in}}%
\pgfpathlineto{\pgfqpoint{3.604064in}{1.320802in}}%
\pgfpathlineto{\pgfqpoint{3.610511in}{1.305798in}}%
\pgfpathlineto{\pgfqpoint{3.616958in}{1.290524in}}%
\pgfpathlineto{\pgfqpoint{3.623402in}{1.274967in}}%
\pgfpathclose%
\pgfusepath{stroke,fill}%
\end{pgfscope}%
\begin{pgfscope}%
\pgfpathrectangle{\pgfqpoint{0.887500in}{0.275000in}}{\pgfqpoint{4.225000in}{4.225000in}}%
\pgfusepath{clip}%
\pgfsetbuttcap%
\pgfsetroundjoin%
\definecolor{currentfill}{rgb}{0.175841,0.441290,0.557685}%
\pgfsetfillcolor{currentfill}%
\pgfsetfillopacity{0.700000}%
\pgfsetlinewidth{0.501875pt}%
\definecolor{currentstroke}{rgb}{1.000000,1.000000,1.000000}%
\pgfsetstrokecolor{currentstroke}%
\pgfsetstrokeopacity{0.500000}%
\pgfsetdash{}{0pt}%
\pgfpathmoveto{\pgfqpoint{1.899753in}{1.916625in}}%
\pgfpathlineto{\pgfqpoint{1.911616in}{1.920518in}}%
\pgfpathlineto{\pgfqpoint{1.923473in}{1.924400in}}%
\pgfpathlineto{\pgfqpoint{1.935325in}{1.928271in}}%
\pgfpathlineto{\pgfqpoint{1.947172in}{1.932133in}}%
\pgfpathlineto{\pgfqpoint{1.959012in}{1.935987in}}%
\pgfpathlineto{\pgfqpoint{1.953011in}{1.945777in}}%
\pgfpathlineto{\pgfqpoint{1.947014in}{1.955532in}}%
\pgfpathlineto{\pgfqpoint{1.941021in}{1.965252in}}%
\pgfpathlineto{\pgfqpoint{1.935034in}{1.974937in}}%
\pgfpathlineto{\pgfqpoint{1.929050in}{1.984588in}}%
\pgfpathlineto{\pgfqpoint{1.917220in}{1.980785in}}%
\pgfpathlineto{\pgfqpoint{1.905384in}{1.976974in}}%
\pgfpathlineto{\pgfqpoint{1.893543in}{1.973152in}}%
\pgfpathlineto{\pgfqpoint{1.881696in}{1.969319in}}%
\pgfpathlineto{\pgfqpoint{1.869843in}{1.965473in}}%
\pgfpathlineto{\pgfqpoint{1.875816in}{1.955771in}}%
\pgfpathlineto{\pgfqpoint{1.881793in}{1.946036in}}%
\pgfpathlineto{\pgfqpoint{1.887775in}{1.936267in}}%
\pgfpathlineto{\pgfqpoint{1.893762in}{1.926463in}}%
\pgfpathclose%
\pgfusepath{stroke,fill}%
\end{pgfscope}%
\begin{pgfscope}%
\pgfpathrectangle{\pgfqpoint{0.887500in}{0.275000in}}{\pgfqpoint{4.225000in}{4.225000in}}%
\pgfusepath{clip}%
\pgfsetbuttcap%
\pgfsetroundjoin%
\definecolor{currentfill}{rgb}{0.233603,0.313828,0.543914}%
\pgfsetfillcolor{currentfill}%
\pgfsetfillopacity{0.700000}%
\pgfsetlinewidth{0.501875pt}%
\definecolor{currentstroke}{rgb}{1.000000,1.000000,1.000000}%
\pgfsetstrokecolor{currentstroke}%
\pgfsetstrokeopacity{0.500000}%
\pgfsetdash{}{0pt}%
\pgfpathmoveto{\pgfqpoint{2.822843in}{1.659331in}}%
\pgfpathlineto{\pgfqpoint{2.834481in}{1.663314in}}%
\pgfpathlineto{\pgfqpoint{2.846113in}{1.667300in}}%
\pgfpathlineto{\pgfqpoint{2.857739in}{1.671289in}}%
\pgfpathlineto{\pgfqpoint{2.869360in}{1.675279in}}%
\pgfpathlineto{\pgfqpoint{2.880975in}{1.679267in}}%
\pgfpathlineto{\pgfqpoint{2.874679in}{1.690575in}}%
\pgfpathlineto{\pgfqpoint{2.868387in}{1.701829in}}%
\pgfpathlineto{\pgfqpoint{2.862099in}{1.713030in}}%
\pgfpathlineto{\pgfqpoint{2.855815in}{1.724180in}}%
\pgfpathlineto{\pgfqpoint{2.849534in}{1.735278in}}%
\pgfpathlineto{\pgfqpoint{2.837927in}{1.731321in}}%
\pgfpathlineto{\pgfqpoint{2.826315in}{1.727367in}}%
\pgfpathlineto{\pgfqpoint{2.814697in}{1.723416in}}%
\pgfpathlineto{\pgfqpoint{2.803073in}{1.719469in}}%
\pgfpathlineto{\pgfqpoint{2.791444in}{1.715527in}}%
\pgfpathlineto{\pgfqpoint{2.797716in}{1.704383in}}%
\pgfpathlineto{\pgfqpoint{2.803992in}{1.693192in}}%
\pgfpathlineto{\pgfqpoint{2.810272in}{1.681954in}}%
\pgfpathlineto{\pgfqpoint{2.816556in}{1.670668in}}%
\pgfpathclose%
\pgfusepath{stroke,fill}%
\end{pgfscope}%
\begin{pgfscope}%
\pgfpathrectangle{\pgfqpoint{0.887500in}{0.275000in}}{\pgfqpoint{4.225000in}{4.225000in}}%
\pgfusepath{clip}%
\pgfsetbuttcap%
\pgfsetroundjoin%
\definecolor{currentfill}{rgb}{0.197636,0.391528,0.554969}%
\pgfsetfillcolor{currentfill}%
\pgfsetfillopacity{0.700000}%
\pgfsetlinewidth{0.501875pt}%
\definecolor{currentstroke}{rgb}{1.000000,1.000000,1.000000}%
\pgfsetstrokecolor{currentstroke}%
\pgfsetstrokeopacity{0.500000}%
\pgfsetdash{}{0pt}%
\pgfpathmoveto{\pgfqpoint{2.316421in}{1.812042in}}%
\pgfpathlineto{\pgfqpoint{2.328183in}{1.815992in}}%
\pgfpathlineto{\pgfqpoint{2.339938in}{1.819935in}}%
\pgfpathlineto{\pgfqpoint{2.351689in}{1.823869in}}%
\pgfpathlineto{\pgfqpoint{2.363433in}{1.827796in}}%
\pgfpathlineto{\pgfqpoint{2.375172in}{1.831718in}}%
\pgfpathlineto{\pgfqpoint{2.369031in}{1.842084in}}%
\pgfpathlineto{\pgfqpoint{2.362894in}{1.852406in}}%
\pgfpathlineto{\pgfqpoint{2.356762in}{1.862682in}}%
\pgfpathlineto{\pgfqpoint{2.350634in}{1.872914in}}%
\pgfpathlineto{\pgfqpoint{2.344510in}{1.883102in}}%
\pgfpathlineto{\pgfqpoint{2.332781in}{1.879230in}}%
\pgfpathlineto{\pgfqpoint{2.321046in}{1.875353in}}%
\pgfpathlineto{\pgfqpoint{2.309305in}{1.871468in}}%
\pgfpathlineto{\pgfqpoint{2.297559in}{1.867575in}}%
\pgfpathlineto{\pgfqpoint{2.285807in}{1.863673in}}%
\pgfpathlineto{\pgfqpoint{2.291921in}{1.853433in}}%
\pgfpathlineto{\pgfqpoint{2.298039in}{1.843150in}}%
\pgfpathlineto{\pgfqpoint{2.304162in}{1.832824in}}%
\pgfpathlineto{\pgfqpoint{2.310290in}{1.822454in}}%
\pgfpathclose%
\pgfusepath{stroke,fill}%
\end{pgfscope}%
\begin{pgfscope}%
\pgfpathrectangle{\pgfqpoint{0.887500in}{0.275000in}}{\pgfqpoint{4.225000in}{4.225000in}}%
\pgfusepath{clip}%
\pgfsetbuttcap%
\pgfsetroundjoin%
\definecolor{currentfill}{rgb}{0.241237,0.296485,0.539709}%
\pgfsetfillcolor{currentfill}%
\pgfsetfillopacity{0.700000}%
\pgfsetlinewidth{0.501875pt}%
\definecolor{currentstroke}{rgb}{1.000000,1.000000,1.000000}%
\pgfsetstrokecolor{currentstroke}%
\pgfsetstrokeopacity{0.500000}%
\pgfsetdash{}{0pt}%
\pgfpathmoveto{\pgfqpoint{2.912508in}{1.621840in}}%
\pgfpathlineto{\pgfqpoint{2.924126in}{1.625849in}}%
\pgfpathlineto{\pgfqpoint{2.935738in}{1.629852in}}%
\pgfpathlineto{\pgfqpoint{2.947344in}{1.633847in}}%
\pgfpathlineto{\pgfqpoint{2.958945in}{1.637834in}}%
\pgfpathlineto{\pgfqpoint{2.970540in}{1.641813in}}%
\pgfpathlineto{\pgfqpoint{2.964218in}{1.653376in}}%
\pgfpathlineto{\pgfqpoint{2.957900in}{1.664890in}}%
\pgfpathlineto{\pgfqpoint{2.951585in}{1.676353in}}%
\pgfpathlineto{\pgfqpoint{2.945274in}{1.687763in}}%
\pgfpathlineto{\pgfqpoint{2.938967in}{1.699120in}}%
\pgfpathlineto{\pgfqpoint{2.927379in}{1.695167in}}%
\pgfpathlineto{\pgfqpoint{2.915787in}{1.691203in}}%
\pgfpathlineto{\pgfqpoint{2.904188in}{1.687231in}}%
\pgfpathlineto{\pgfqpoint{2.892584in}{1.683252in}}%
\pgfpathlineto{\pgfqpoint{2.880975in}{1.679267in}}%
\pgfpathlineto{\pgfqpoint{2.887274in}{1.667902in}}%
\pgfpathlineto{\pgfqpoint{2.893577in}{1.656480in}}%
\pgfpathlineto{\pgfqpoint{2.899884in}{1.644996in}}%
\pgfpathlineto{\pgfqpoint{2.906194in}{1.633450in}}%
\pgfpathclose%
\pgfusepath{stroke,fill}%
\end{pgfscope}%
\begin{pgfscope}%
\pgfpathrectangle{\pgfqpoint{0.887500in}{0.275000in}}{\pgfqpoint{4.225000in}{4.225000in}}%
\pgfusepath{clip}%
\pgfsetbuttcap%
\pgfsetroundjoin%
\definecolor{currentfill}{rgb}{0.283197,0.115680,0.436115}%
\pgfsetfillcolor{currentfill}%
\pgfsetfillopacity{0.700000}%
\pgfsetlinewidth{0.501875pt}%
\definecolor{currentstroke}{rgb}{1.000000,1.000000,1.000000}%
\pgfsetstrokecolor{currentstroke}%
\pgfsetstrokeopacity{0.500000}%
\pgfsetdash{}{0pt}%
\pgfpathmoveto{\pgfqpoint{3.540320in}{1.312991in}}%
\pgfpathlineto{\pgfqpoint{3.551790in}{1.317489in}}%
\pgfpathlineto{\pgfqpoint{3.563254in}{1.321992in}}%
\pgfpathlineto{\pgfqpoint{3.574713in}{1.326514in}}%
\pgfpathlineto{\pgfqpoint{3.586166in}{1.331037in}}%
\pgfpathlineto{\pgfqpoint{3.597615in}{1.335548in}}%
\pgfpathlineto{\pgfqpoint{3.591165in}{1.350050in}}%
\pgfpathlineto{\pgfqpoint{3.584714in}{1.364322in}}%
\pgfpathlineto{\pgfqpoint{3.578264in}{1.378377in}}%
\pgfpathlineto{\pgfqpoint{3.571813in}{1.392227in}}%
\pgfpathlineto{\pgfqpoint{3.565362in}{1.405887in}}%
\pgfpathlineto{\pgfqpoint{3.553908in}{1.400629in}}%
\pgfpathlineto{\pgfqpoint{3.542453in}{1.395609in}}%
\pgfpathlineto{\pgfqpoint{3.530995in}{1.390804in}}%
\pgfpathlineto{\pgfqpoint{3.519534in}{1.386196in}}%
\pgfpathlineto{\pgfqpoint{3.508070in}{1.381750in}}%
\pgfpathlineto{\pgfqpoint{3.514519in}{1.368364in}}%
\pgfpathlineto{\pgfqpoint{3.520969in}{1.354824in}}%
\pgfpathlineto{\pgfqpoint{3.527419in}{1.341100in}}%
\pgfpathlineto{\pgfqpoint{3.533870in}{1.327166in}}%
\pgfpathclose%
\pgfusepath{stroke,fill}%
\end{pgfscope}%
\begin{pgfscope}%
\pgfpathrectangle{\pgfqpoint{0.887500in}{0.275000in}}{\pgfqpoint{4.225000in}{4.225000in}}%
\pgfusepath{clip}%
\pgfsetbuttcap%
\pgfsetroundjoin%
\definecolor{currentfill}{rgb}{0.282290,0.145912,0.461510}%
\pgfsetfillcolor{currentfill}%
\pgfsetfillopacity{0.700000}%
\pgfsetlinewidth{0.501875pt}%
\definecolor{currentstroke}{rgb}{1.000000,1.000000,1.000000}%
\pgfsetstrokecolor{currentstroke}%
\pgfsetstrokeopacity{0.500000}%
\pgfsetdash{}{0pt}%
\pgfpathmoveto{\pgfqpoint{3.450665in}{1.358975in}}%
\pgfpathlineto{\pgfqpoint{3.462161in}{1.363898in}}%
\pgfpathlineto{\pgfqpoint{3.473648in}{1.368548in}}%
\pgfpathlineto{\pgfqpoint{3.485128in}{1.373015in}}%
\pgfpathlineto{\pgfqpoint{3.496602in}{1.377385in}}%
\pgfpathlineto{\pgfqpoint{3.508070in}{1.381750in}}%
\pgfpathlineto{\pgfqpoint{3.501623in}{1.395010in}}%
\pgfpathlineto{\pgfqpoint{3.495178in}{1.408174in}}%
\pgfpathlineto{\pgfqpoint{3.488734in}{1.421269in}}%
\pgfpathlineto{\pgfqpoint{3.482294in}{1.434325in}}%
\pgfpathlineto{\pgfqpoint{3.475856in}{1.447371in}}%
\pgfpathlineto{\pgfqpoint{3.464397in}{1.443359in}}%
\pgfpathlineto{\pgfqpoint{3.452934in}{1.439514in}}%
\pgfpathlineto{\pgfqpoint{3.441467in}{1.435769in}}%
\pgfpathlineto{\pgfqpoint{3.429994in}{1.432055in}}%
\pgfpathlineto{\pgfqpoint{3.418515in}{1.428306in}}%
\pgfpathlineto{\pgfqpoint{3.424941in}{1.414646in}}%
\pgfpathlineto{\pgfqpoint{3.431370in}{1.400886in}}%
\pgfpathlineto{\pgfqpoint{3.437800in}{1.387023in}}%
\pgfpathlineto{\pgfqpoint{3.444232in}{1.373054in}}%
\pgfpathclose%
\pgfusepath{stroke,fill}%
\end{pgfscope}%
\begin{pgfscope}%
\pgfpathrectangle{\pgfqpoint{0.887500in}{0.275000in}}{\pgfqpoint{4.225000in}{4.225000in}}%
\pgfusepath{clip}%
\pgfsetbuttcap%
\pgfsetroundjoin%
\definecolor{currentfill}{rgb}{0.248629,0.278775,0.534556}%
\pgfsetfillcolor{currentfill}%
\pgfsetfillopacity{0.700000}%
\pgfsetlinewidth{0.501875pt}%
\definecolor{currentstroke}{rgb}{1.000000,1.000000,1.000000}%
\pgfsetstrokecolor{currentstroke}%
\pgfsetstrokeopacity{0.500000}%
\pgfsetdash{}{0pt}%
\pgfpathmoveto{\pgfqpoint{3.002201in}{1.583296in}}%
\pgfpathlineto{\pgfqpoint{3.013798in}{1.587392in}}%
\pgfpathlineto{\pgfqpoint{3.025390in}{1.591463in}}%
\pgfpathlineto{\pgfqpoint{3.036976in}{1.595499in}}%
\pgfpathlineto{\pgfqpoint{3.048556in}{1.599490in}}%
\pgfpathlineto{\pgfqpoint{3.060131in}{1.603433in}}%
\pgfpathlineto{\pgfqpoint{3.053785in}{1.615146in}}%
\pgfpathlineto{\pgfqpoint{3.047441in}{1.626811in}}%
\pgfpathlineto{\pgfqpoint{3.041102in}{1.638442in}}%
\pgfpathlineto{\pgfqpoint{3.034765in}{1.650038in}}%
\pgfpathlineto{\pgfqpoint{3.028432in}{1.661596in}}%
\pgfpathlineto{\pgfqpoint{3.016865in}{1.657655in}}%
\pgfpathlineto{\pgfqpoint{3.005292in}{1.653705in}}%
\pgfpathlineto{\pgfqpoint{2.993714in}{1.649748in}}%
\pgfpathlineto{\pgfqpoint{2.982130in}{1.645784in}}%
\pgfpathlineto{\pgfqpoint{2.970540in}{1.641813in}}%
\pgfpathlineto{\pgfqpoint{2.976865in}{1.630202in}}%
\pgfpathlineto{\pgfqpoint{2.983194in}{1.618545in}}%
\pgfpathlineto{\pgfqpoint{2.989526in}{1.606843in}}%
\pgfpathlineto{\pgfqpoint{2.995862in}{1.595096in}}%
\pgfpathclose%
\pgfusepath{stroke,fill}%
\end{pgfscope}%
\begin{pgfscope}%
\pgfpathrectangle{\pgfqpoint{0.887500in}{0.275000in}}{\pgfqpoint{4.225000in}{4.225000in}}%
\pgfusepath{clip}%
\pgfsetbuttcap%
\pgfsetroundjoin%
\definecolor{currentfill}{rgb}{0.180629,0.429975,0.557282}%
\pgfsetfillcolor{currentfill}%
\pgfsetfillopacity{0.700000}%
\pgfsetlinewidth{0.501875pt}%
\definecolor{currentstroke}{rgb}{1.000000,1.000000,1.000000}%
\pgfsetstrokecolor{currentstroke}%
\pgfsetstrokeopacity{0.500000}%
\pgfsetdash{}{0pt}%
\pgfpathmoveto{\pgfqpoint{1.989088in}{1.886492in}}%
\pgfpathlineto{\pgfqpoint{2.000934in}{1.890386in}}%
\pgfpathlineto{\pgfqpoint{2.012773in}{1.894275in}}%
\pgfpathlineto{\pgfqpoint{2.024607in}{1.898159in}}%
\pgfpathlineto{\pgfqpoint{2.036435in}{1.902040in}}%
\pgfpathlineto{\pgfqpoint{2.048258in}{1.905918in}}%
\pgfpathlineto{\pgfqpoint{2.042223in}{1.915844in}}%
\pgfpathlineto{\pgfqpoint{2.036193in}{1.925732in}}%
\pgfpathlineto{\pgfqpoint{2.030168in}{1.935582in}}%
\pgfpathlineto{\pgfqpoint{2.024147in}{1.945394in}}%
\pgfpathlineto{\pgfqpoint{2.018131in}{1.955170in}}%
\pgfpathlineto{\pgfqpoint{2.006318in}{1.951341in}}%
\pgfpathlineto{\pgfqpoint{1.994500in}{1.947510in}}%
\pgfpathlineto{\pgfqpoint{1.982677in}{1.943674in}}%
\pgfpathlineto{\pgfqpoint{1.970847in}{1.939833in}}%
\pgfpathlineto{\pgfqpoint{1.959012in}{1.935987in}}%
\pgfpathlineto{\pgfqpoint{1.965018in}{1.926161in}}%
\pgfpathlineto{\pgfqpoint{1.971029in}{1.916299in}}%
\pgfpathlineto{\pgfqpoint{1.977044in}{1.906400in}}%
\pgfpathlineto{\pgfqpoint{1.983064in}{1.896465in}}%
\pgfpathclose%
\pgfusepath{stroke,fill}%
\end{pgfscope}%
\begin{pgfscope}%
\pgfpathrectangle{\pgfqpoint{0.887500in}{0.275000in}}{\pgfqpoint{4.225000in}{4.225000in}}%
\pgfusepath{clip}%
\pgfsetbuttcap%
\pgfsetroundjoin%
\definecolor{currentfill}{rgb}{0.203063,0.379716,0.553925}%
\pgfsetfillcolor{currentfill}%
\pgfsetfillopacity{0.700000}%
\pgfsetlinewidth{0.501875pt}%
\definecolor{currentstroke}{rgb}{1.000000,1.000000,1.000000}%
\pgfsetstrokecolor{currentstroke}%
\pgfsetstrokeopacity{0.500000}%
\pgfsetdash{}{0pt}%
\pgfpathmoveto{\pgfqpoint{2.405943in}{1.779220in}}%
\pgfpathlineto{\pgfqpoint{2.417686in}{1.783183in}}%
\pgfpathlineto{\pgfqpoint{2.429423in}{1.787146in}}%
\pgfpathlineto{\pgfqpoint{2.441154in}{1.791110in}}%
\pgfpathlineto{\pgfqpoint{2.452880in}{1.795078in}}%
\pgfpathlineto{\pgfqpoint{2.464600in}{1.799050in}}%
\pgfpathlineto{\pgfqpoint{2.458427in}{1.809597in}}%
\pgfpathlineto{\pgfqpoint{2.452260in}{1.820098in}}%
\pgfpathlineto{\pgfqpoint{2.446096in}{1.830553in}}%
\pgfpathlineto{\pgfqpoint{2.439937in}{1.840963in}}%
\pgfpathlineto{\pgfqpoint{2.433782in}{1.851328in}}%
\pgfpathlineto{\pgfqpoint{2.422071in}{1.847399in}}%
\pgfpathlineto{\pgfqpoint{2.410355in}{1.843475in}}%
\pgfpathlineto{\pgfqpoint{2.398633in}{1.839556in}}%
\pgfpathlineto{\pgfqpoint{2.386906in}{1.835637in}}%
\pgfpathlineto{\pgfqpoint{2.375172in}{1.831718in}}%
\pgfpathlineto{\pgfqpoint{2.381318in}{1.821308in}}%
\pgfpathlineto{\pgfqpoint{2.387468in}{1.810853in}}%
\pgfpathlineto{\pgfqpoint{2.393622in}{1.800353in}}%
\pgfpathlineto{\pgfqpoint{2.399780in}{1.789809in}}%
\pgfpathclose%
\pgfusepath{stroke,fill}%
\end{pgfscope}%
\begin{pgfscope}%
\pgfpathrectangle{\pgfqpoint{0.887500in}{0.275000in}}{\pgfqpoint{4.225000in}{4.225000in}}%
\pgfusepath{clip}%
\pgfsetbuttcap%
\pgfsetroundjoin%
\definecolor{currentfill}{rgb}{0.278826,0.175490,0.483397}%
\pgfsetfillcolor{currentfill}%
\pgfsetfillopacity{0.700000}%
\pgfsetlinewidth{0.501875pt}%
\definecolor{currentstroke}{rgb}{1.000000,1.000000,1.000000}%
\pgfsetstrokecolor{currentstroke}%
\pgfsetstrokeopacity{0.500000}%
\pgfsetdash{}{0pt}%
\pgfpathmoveto{\pgfqpoint{3.361021in}{1.407978in}}%
\pgfpathlineto{\pgfqpoint{3.372532in}{1.412183in}}%
\pgfpathlineto{\pgfqpoint{3.384037in}{1.416357in}}%
\pgfpathlineto{\pgfqpoint{3.395536in}{1.420461in}}%
\pgfpathlineto{\pgfqpoint{3.407029in}{1.424454in}}%
\pgfpathlineto{\pgfqpoint{3.418515in}{1.428306in}}%
\pgfpathlineto{\pgfqpoint{3.412090in}{1.441871in}}%
\pgfpathlineto{\pgfqpoint{3.405667in}{1.455343in}}%
\pgfpathlineto{\pgfqpoint{3.399247in}{1.468722in}}%
\pgfpathlineto{\pgfqpoint{3.392828in}{1.482005in}}%
\pgfpathlineto{\pgfqpoint{3.386412in}{1.495192in}}%
\pgfpathlineto{\pgfqpoint{3.374934in}{1.491769in}}%
\pgfpathlineto{\pgfqpoint{3.363451in}{1.488290in}}%
\pgfpathlineto{\pgfqpoint{3.351960in}{1.484747in}}%
\pgfpathlineto{\pgfqpoint{3.340464in}{1.481135in}}%
\pgfpathlineto{\pgfqpoint{3.328961in}{1.477450in}}%
\pgfpathlineto{\pgfqpoint{3.335372in}{1.464208in}}%
\pgfpathlineto{\pgfqpoint{3.341784in}{1.450677in}}%
\pgfpathlineto{\pgfqpoint{3.348197in}{1.436817in}}%
\pgfpathlineto{\pgfqpoint{3.354609in}{1.422593in}}%
\pgfpathclose%
\pgfusepath{stroke,fill}%
\end{pgfscope}%
\begin{pgfscope}%
\pgfpathrectangle{\pgfqpoint{0.887500in}{0.275000in}}{\pgfqpoint{4.225000in}{4.225000in}}%
\pgfusepath{clip}%
\pgfsetbuttcap%
\pgfsetroundjoin%
\definecolor{currentfill}{rgb}{0.257322,0.256130,0.526563}%
\pgfsetfillcolor{currentfill}%
\pgfsetfillopacity{0.700000}%
\pgfsetlinewidth{0.501875pt}%
\definecolor{currentstroke}{rgb}{1.000000,1.000000,1.000000}%
\pgfsetstrokecolor{currentstroke}%
\pgfsetstrokeopacity{0.500000}%
\pgfsetdash{}{0pt}%
\pgfpathmoveto{\pgfqpoint{3.091914in}{1.543507in}}%
\pgfpathlineto{\pgfqpoint{3.103490in}{1.547373in}}%
\pgfpathlineto{\pgfqpoint{3.115061in}{1.551187in}}%
\pgfpathlineto{\pgfqpoint{3.126626in}{1.554966in}}%
\pgfpathlineto{\pgfqpoint{3.138185in}{1.558725in}}%
\pgfpathlineto{\pgfqpoint{3.149739in}{1.562481in}}%
\pgfpathlineto{\pgfqpoint{3.143369in}{1.574680in}}%
\pgfpathlineto{\pgfqpoint{3.137002in}{1.586799in}}%
\pgfpathlineto{\pgfqpoint{3.130639in}{1.598840in}}%
\pgfpathlineto{\pgfqpoint{3.124278in}{1.610804in}}%
\pgfpathlineto{\pgfqpoint{3.117921in}{1.622693in}}%
\pgfpathlineto{\pgfqpoint{3.106374in}{1.618877in}}%
\pgfpathlineto{\pgfqpoint{3.094822in}{1.615050in}}%
\pgfpathlineto{\pgfqpoint{3.083264in}{1.611206in}}%
\pgfpathlineto{\pgfqpoint{3.071701in}{1.607336in}}%
\pgfpathlineto{\pgfqpoint{3.060131in}{1.603433in}}%
\pgfpathlineto{\pgfqpoint{3.066481in}{1.591654in}}%
\pgfpathlineto{\pgfqpoint{3.072835in}{1.579791in}}%
\pgfpathlineto{\pgfqpoint{3.079191in}{1.567824in}}%
\pgfpathlineto{\pgfqpoint{3.085551in}{1.555735in}}%
\pgfpathclose%
\pgfusepath{stroke,fill}%
\end{pgfscope}%
\begin{pgfscope}%
\pgfpathrectangle{\pgfqpoint{0.887500in}{0.275000in}}{\pgfqpoint{4.225000in}{4.225000in}}%
\pgfusepath{clip}%
\pgfsetbuttcap%
\pgfsetroundjoin%
\definecolor{currentfill}{rgb}{0.265145,0.232956,0.516599}%
\pgfsetfillcolor{currentfill}%
\pgfsetfillopacity{0.700000}%
\pgfsetlinewidth{0.501875pt}%
\definecolor{currentstroke}{rgb}{1.000000,1.000000,1.000000}%
\pgfsetstrokecolor{currentstroke}%
\pgfsetstrokeopacity{0.500000}%
\pgfsetdash{}{0pt}%
\pgfpathmoveto{\pgfqpoint{3.181632in}{1.500231in}}%
\pgfpathlineto{\pgfqpoint{3.193187in}{1.504342in}}%
\pgfpathlineto{\pgfqpoint{3.204736in}{1.508433in}}%
\pgfpathlineto{\pgfqpoint{3.216280in}{1.512505in}}%
\pgfpathlineto{\pgfqpoint{3.227819in}{1.516561in}}%
\pgfpathlineto{\pgfqpoint{3.239351in}{1.520602in}}%
\pgfpathlineto{\pgfqpoint{3.232959in}{1.532818in}}%
\pgfpathlineto{\pgfqpoint{3.226569in}{1.545023in}}%
\pgfpathlineto{\pgfqpoint{3.220183in}{1.557232in}}%
\pgfpathlineto{\pgfqpoint{3.213801in}{1.569435in}}%
\pgfpathlineto{\pgfqpoint{3.207421in}{1.581621in}}%
\pgfpathlineto{\pgfqpoint{3.195896in}{1.577725in}}%
\pgfpathlineto{\pgfqpoint{3.184365in}{1.573867in}}%
\pgfpathlineto{\pgfqpoint{3.172828in}{1.570045in}}%
\pgfpathlineto{\pgfqpoint{3.161286in}{1.566251in}}%
\pgfpathlineto{\pgfqpoint{3.149739in}{1.562481in}}%
\pgfpathlineto{\pgfqpoint{3.156111in}{1.550200in}}%
\pgfpathlineto{\pgfqpoint{3.162487in}{1.537836in}}%
\pgfpathlineto{\pgfqpoint{3.168866in}{1.525388in}}%
\pgfpathlineto{\pgfqpoint{3.175247in}{1.512852in}}%
\pgfpathclose%
\pgfusepath{stroke,fill}%
\end{pgfscope}%
\begin{pgfscope}%
\pgfpathrectangle{\pgfqpoint{0.887500in}{0.275000in}}{\pgfqpoint{4.225000in}{4.225000in}}%
\pgfusepath{clip}%
\pgfsetbuttcap%
\pgfsetroundjoin%
\definecolor{currentfill}{rgb}{0.273006,0.204520,0.501721}%
\pgfsetfillcolor{currentfill}%
\pgfsetfillopacity{0.700000}%
\pgfsetlinewidth{0.501875pt}%
\definecolor{currentstroke}{rgb}{1.000000,1.000000,1.000000}%
\pgfsetstrokecolor{currentstroke}%
\pgfsetstrokeopacity{0.500000}%
\pgfsetdash{}{0pt}%
\pgfpathmoveto{\pgfqpoint{3.271354in}{1.457768in}}%
\pgfpathlineto{\pgfqpoint{3.282887in}{1.461882in}}%
\pgfpathlineto{\pgfqpoint{3.294415in}{1.465906in}}%
\pgfpathlineto{\pgfqpoint{3.305937in}{1.469840in}}%
\pgfpathlineto{\pgfqpoint{3.317452in}{1.473686in}}%
\pgfpathlineto{\pgfqpoint{3.328961in}{1.477450in}}%
\pgfpathlineto{\pgfqpoint{3.322551in}{1.490438in}}%
\pgfpathlineto{\pgfqpoint{3.316143in}{1.503210in}}%
\pgfpathlineto{\pgfqpoint{3.309736in}{1.515804in}}%
\pgfpathlineto{\pgfqpoint{3.303332in}{1.528256in}}%
\pgfpathlineto{\pgfqpoint{3.296930in}{1.540605in}}%
\pgfpathlineto{\pgfqpoint{3.285426in}{1.536642in}}%
\pgfpathlineto{\pgfqpoint{3.273916in}{1.532653in}}%
\pgfpathlineto{\pgfqpoint{3.262400in}{1.528646in}}%
\pgfpathlineto{\pgfqpoint{3.250878in}{1.524630in}}%
\pgfpathlineto{\pgfqpoint{3.239351in}{1.520602in}}%
\pgfpathlineto{\pgfqpoint{3.245747in}{1.508331in}}%
\pgfpathlineto{\pgfqpoint{3.252145in}{1.495959in}}%
\pgfpathlineto{\pgfqpoint{3.258546in}{1.483439in}}%
\pgfpathlineto{\pgfqpoint{3.264949in}{1.470724in}}%
\pgfpathclose%
\pgfusepath{stroke,fill}%
\end{pgfscope}%
\begin{pgfscope}%
\pgfpathrectangle{\pgfqpoint{0.887500in}{0.275000in}}{\pgfqpoint{4.225000in}{4.225000in}}%
\pgfusepath{clip}%
\pgfsetbuttcap%
\pgfsetroundjoin%
\definecolor{currentfill}{rgb}{0.210503,0.363727,0.552206}%
\pgfsetfillcolor{currentfill}%
\pgfsetfillopacity{0.700000}%
\pgfsetlinewidth{0.501875pt}%
\definecolor{currentstroke}{rgb}{1.000000,1.000000,1.000000}%
\pgfsetstrokecolor{currentstroke}%
\pgfsetstrokeopacity{0.500000}%
\pgfsetdash{}{0pt}%
\pgfpathmoveto{\pgfqpoint{2.495523in}{1.745625in}}%
\pgfpathlineto{\pgfqpoint{2.507247in}{1.749640in}}%
\pgfpathlineto{\pgfqpoint{2.518964in}{1.753656in}}%
\pgfpathlineto{\pgfqpoint{2.530676in}{1.757672in}}%
\pgfpathlineto{\pgfqpoint{2.542383in}{1.761685in}}%
\pgfpathlineto{\pgfqpoint{2.554083in}{1.765692in}}%
\pgfpathlineto{\pgfqpoint{2.547881in}{1.776433in}}%
\pgfpathlineto{\pgfqpoint{2.541683in}{1.787126in}}%
\pgfpathlineto{\pgfqpoint{2.535489in}{1.797771in}}%
\pgfpathlineto{\pgfqpoint{2.529299in}{1.808369in}}%
\pgfpathlineto{\pgfqpoint{2.523113in}{1.818919in}}%
\pgfpathlineto{\pgfqpoint{2.511422in}{1.814954in}}%
\pgfpathlineto{\pgfqpoint{2.499725in}{1.810982in}}%
\pgfpathlineto{\pgfqpoint{2.488022in}{1.807005in}}%
\pgfpathlineto{\pgfqpoint{2.476314in}{1.803027in}}%
\pgfpathlineto{\pgfqpoint{2.464600in}{1.799050in}}%
\pgfpathlineto{\pgfqpoint{2.470776in}{1.788458in}}%
\pgfpathlineto{\pgfqpoint{2.476956in}{1.777819in}}%
\pgfpathlineto{\pgfqpoint{2.483141in}{1.767134in}}%
\pgfpathlineto{\pgfqpoint{2.489330in}{1.756403in}}%
\pgfpathclose%
\pgfusepath{stroke,fill}%
\end{pgfscope}%
\begin{pgfscope}%
\pgfpathrectangle{\pgfqpoint{0.887500in}{0.275000in}}{\pgfqpoint{4.225000in}{4.225000in}}%
\pgfusepath{clip}%
\pgfsetbuttcap%
\pgfsetroundjoin%
\definecolor{currentfill}{rgb}{0.185556,0.418570,0.556753}%
\pgfsetfillcolor{currentfill}%
\pgfsetfillopacity{0.700000}%
\pgfsetlinewidth{0.501875pt}%
\definecolor{currentstroke}{rgb}{1.000000,1.000000,1.000000}%
\pgfsetstrokecolor{currentstroke}%
\pgfsetstrokeopacity{0.500000}%
\pgfsetdash{}{0pt}%
\pgfpathmoveto{\pgfqpoint{2.078499in}{1.855698in}}%
\pgfpathlineto{\pgfqpoint{2.090326in}{1.859619in}}%
\pgfpathlineto{\pgfqpoint{2.102147in}{1.863538in}}%
\pgfpathlineto{\pgfqpoint{2.113963in}{1.867458in}}%
\pgfpathlineto{\pgfqpoint{2.125772in}{1.871378in}}%
\pgfpathlineto{\pgfqpoint{2.137576in}{1.875298in}}%
\pgfpathlineto{\pgfqpoint{2.131508in}{1.885379in}}%
\pgfpathlineto{\pgfqpoint{2.125445in}{1.895419in}}%
\pgfpathlineto{\pgfqpoint{2.119387in}{1.905419in}}%
\pgfpathlineto{\pgfqpoint{2.113333in}{1.915378in}}%
\pgfpathlineto{\pgfqpoint{2.107284in}{1.925297in}}%
\pgfpathlineto{\pgfqpoint{2.095490in}{1.921420in}}%
\pgfpathlineto{\pgfqpoint{2.083691in}{1.917544in}}%
\pgfpathlineto{\pgfqpoint{2.071886in}{1.913669in}}%
\pgfpathlineto{\pgfqpoint{2.060075in}{1.909794in}}%
\pgfpathlineto{\pgfqpoint{2.048258in}{1.905918in}}%
\pgfpathlineto{\pgfqpoint{2.054297in}{1.895953in}}%
\pgfpathlineto{\pgfqpoint{2.060341in}{1.885949in}}%
\pgfpathlineto{\pgfqpoint{2.066389in}{1.875905in}}%
\pgfpathlineto{\pgfqpoint{2.072442in}{1.865822in}}%
\pgfpathclose%
\pgfusepath{stroke,fill}%
\end{pgfscope}%
\begin{pgfscope}%
\pgfpathrectangle{\pgfqpoint{0.887500in}{0.275000in}}{\pgfqpoint{4.225000in}{4.225000in}}%
\pgfusepath{clip}%
\pgfsetbuttcap%
\pgfsetroundjoin%
\definecolor{currentfill}{rgb}{0.273809,0.031497,0.358853}%
\pgfsetfillcolor{currentfill}%
\pgfsetfillopacity{0.700000}%
\pgfsetlinewidth{0.501875pt}%
\definecolor{currentstroke}{rgb}{1.000000,1.000000,1.000000}%
\pgfsetstrokecolor{currentstroke}%
\pgfsetstrokeopacity{0.500000}%
\pgfsetdash{}{0pt}%
\pgfpathmoveto{\pgfqpoint{3.662010in}{1.174937in}}%
\pgfpathlineto{\pgfqpoint{3.673475in}{1.180655in}}%
\pgfpathlineto{\pgfqpoint{3.684939in}{1.186626in}}%
\pgfpathlineto{\pgfqpoint{3.696403in}{1.192802in}}%
\pgfpathlineto{\pgfqpoint{3.707865in}{1.199142in}}%
\pgfpathlineto{\pgfqpoint{3.719325in}{1.205626in}}%
\pgfpathlineto{\pgfqpoint{3.712827in}{1.218975in}}%
\pgfpathlineto{\pgfqpoint{3.706342in}{1.232914in}}%
\pgfpathlineto{\pgfqpoint{3.699869in}{1.247376in}}%
\pgfpathlineto{\pgfqpoint{3.693406in}{1.262289in}}%
\pgfpathlineto{\pgfqpoint{3.686952in}{1.277584in}}%
\pgfpathlineto{\pgfqpoint{3.675549in}{1.274335in}}%
\pgfpathlineto{\pgfqpoint{3.664137in}{1.270885in}}%
\pgfpathlineto{\pgfqpoint{3.652715in}{1.267177in}}%
\pgfpathlineto{\pgfqpoint{3.641284in}{1.263236in}}%
\pgfpathlineto{\pgfqpoint{3.629844in}{1.259113in}}%
\pgfpathlineto{\pgfqpoint{3.636284in}{1.242950in}}%
\pgfpathlineto{\pgfqpoint{3.642721in}{1.226463in}}%
\pgfpathlineto{\pgfqpoint{3.649155in}{1.209641in}}%
\pgfpathlineto{\pgfqpoint{3.655585in}{1.192470in}}%
\pgfpathclose%
\pgfusepath{stroke,fill}%
\end{pgfscope}%
\begin{pgfscope}%
\pgfpathrectangle{\pgfqpoint{0.887500in}{0.275000in}}{\pgfqpoint{4.225000in}{4.225000in}}%
\pgfusepath{clip}%
\pgfsetbuttcap%
\pgfsetroundjoin%
\definecolor{currentfill}{rgb}{0.218130,0.347432,0.550038}%
\pgfsetfillcolor{currentfill}%
\pgfsetfillopacity{0.700000}%
\pgfsetlinewidth{0.501875pt}%
\definecolor{currentstroke}{rgb}{1.000000,1.000000,1.000000}%
\pgfsetstrokecolor{currentstroke}%
\pgfsetstrokeopacity{0.500000}%
\pgfsetdash{}{0pt}%
\pgfpathmoveto{\pgfqpoint{2.585157in}{1.711261in}}%
\pgfpathlineto{\pgfqpoint{2.596861in}{1.715296in}}%
\pgfpathlineto{\pgfqpoint{2.608560in}{1.719323in}}%
\pgfpathlineto{\pgfqpoint{2.620253in}{1.723340in}}%
\pgfpathlineto{\pgfqpoint{2.631941in}{1.727346in}}%
\pgfpathlineto{\pgfqpoint{2.643623in}{1.731339in}}%
\pgfpathlineto{\pgfqpoint{2.637391in}{1.742285in}}%
\pgfpathlineto{\pgfqpoint{2.631163in}{1.753182in}}%
\pgfpathlineto{\pgfqpoint{2.624940in}{1.764029in}}%
\pgfpathlineto{\pgfqpoint{2.618720in}{1.774825in}}%
\pgfpathlineto{\pgfqpoint{2.612505in}{1.785570in}}%
\pgfpathlineto{\pgfqpoint{2.600831in}{1.781621in}}%
\pgfpathlineto{\pgfqpoint{2.589153in}{1.777657in}}%
\pgfpathlineto{\pgfqpoint{2.577468in}{1.773680in}}%
\pgfpathlineto{\pgfqpoint{2.565779in}{1.769691in}}%
\pgfpathlineto{\pgfqpoint{2.554083in}{1.765692in}}%
\pgfpathlineto{\pgfqpoint{2.560290in}{1.754903in}}%
\pgfpathlineto{\pgfqpoint{2.566501in}{1.744065in}}%
\pgfpathlineto{\pgfqpoint{2.572715in}{1.733178in}}%
\pgfpathlineto{\pgfqpoint{2.578934in}{1.722244in}}%
\pgfpathclose%
\pgfusepath{stroke,fill}%
\end{pgfscope}%
\begin{pgfscope}%
\pgfpathrectangle{\pgfqpoint{0.887500in}{0.275000in}}{\pgfqpoint{4.225000in}{4.225000in}}%
\pgfusepath{clip}%
\pgfsetbuttcap%
\pgfsetroundjoin%
\definecolor{currentfill}{rgb}{0.192357,0.403199,0.555836}%
\pgfsetfillcolor{currentfill}%
\pgfsetfillopacity{0.700000}%
\pgfsetlinewidth{0.501875pt}%
\definecolor{currentstroke}{rgb}{1.000000,1.000000,1.000000}%
\pgfsetstrokecolor{currentstroke}%
\pgfsetstrokeopacity{0.500000}%
\pgfsetdash{}{0pt}%
\pgfpathmoveto{\pgfqpoint{2.167981in}{1.824273in}}%
\pgfpathlineto{\pgfqpoint{2.179789in}{1.828233in}}%
\pgfpathlineto{\pgfqpoint{2.191591in}{1.832192in}}%
\pgfpathlineto{\pgfqpoint{2.203388in}{1.836148in}}%
\pgfpathlineto{\pgfqpoint{2.215179in}{1.840100in}}%
\pgfpathlineto{\pgfqpoint{2.226964in}{1.844046in}}%
\pgfpathlineto{\pgfqpoint{2.220864in}{1.854293in}}%
\pgfpathlineto{\pgfqpoint{2.214769in}{1.864497in}}%
\pgfpathlineto{\pgfqpoint{2.208678in}{1.874659in}}%
\pgfpathlineto{\pgfqpoint{2.202591in}{1.884780in}}%
\pgfpathlineto{\pgfqpoint{2.196509in}{1.894858in}}%
\pgfpathlineto{\pgfqpoint{2.184734in}{1.890957in}}%
\pgfpathlineto{\pgfqpoint{2.172953in}{1.887049in}}%
\pgfpathlineto{\pgfqpoint{2.161166in}{1.883136in}}%
\pgfpathlineto{\pgfqpoint{2.149374in}{1.879218in}}%
\pgfpathlineto{\pgfqpoint{2.137576in}{1.875298in}}%
\pgfpathlineto{\pgfqpoint{2.143648in}{1.865176in}}%
\pgfpathlineto{\pgfqpoint{2.149724in}{1.855012in}}%
\pgfpathlineto{\pgfqpoint{2.155806in}{1.844807in}}%
\pgfpathlineto{\pgfqpoint{2.161891in}{1.834561in}}%
\pgfpathclose%
\pgfusepath{stroke,fill}%
\end{pgfscope}%
\begin{pgfscope}%
\pgfpathrectangle{\pgfqpoint{0.887500in}{0.275000in}}{\pgfqpoint{4.225000in}{4.225000in}}%
\pgfusepath{clip}%
\pgfsetbuttcap%
\pgfsetroundjoin%
\definecolor{currentfill}{rgb}{0.225863,0.330805,0.547314}%
\pgfsetfillcolor{currentfill}%
\pgfsetfillopacity{0.700000}%
\pgfsetlinewidth{0.501875pt}%
\definecolor{currentstroke}{rgb}{1.000000,1.000000,1.000000}%
\pgfsetstrokecolor{currentstroke}%
\pgfsetstrokeopacity{0.500000}%
\pgfsetdash{}{0pt}%
\pgfpathmoveto{\pgfqpoint{2.674841in}{1.675897in}}%
\pgfpathlineto{\pgfqpoint{2.686526in}{1.679913in}}%
\pgfpathlineto{\pgfqpoint{2.698206in}{1.683916in}}%
\pgfpathlineto{\pgfqpoint{2.709880in}{1.687903in}}%
\pgfpathlineto{\pgfqpoint{2.721549in}{1.691874in}}%
\pgfpathlineto{\pgfqpoint{2.733212in}{1.695832in}}%
\pgfpathlineto{\pgfqpoint{2.726952in}{1.706973in}}%
\pgfpathlineto{\pgfqpoint{2.720696in}{1.718068in}}%
\pgfpathlineto{\pgfqpoint{2.714444in}{1.729116in}}%
\pgfpathlineto{\pgfqpoint{2.708195in}{1.740116in}}%
\pgfpathlineto{\pgfqpoint{2.701951in}{1.751066in}}%
\pgfpathlineto{\pgfqpoint{2.690296in}{1.747154in}}%
\pgfpathlineto{\pgfqpoint{2.678636in}{1.743226in}}%
\pgfpathlineto{\pgfqpoint{2.666971in}{1.739281in}}%
\pgfpathlineto{\pgfqpoint{2.655299in}{1.735318in}}%
\pgfpathlineto{\pgfqpoint{2.643623in}{1.731339in}}%
\pgfpathlineto{\pgfqpoint{2.649859in}{1.720344in}}%
\pgfpathlineto{\pgfqpoint{2.656098in}{1.709301in}}%
\pgfpathlineto{\pgfqpoint{2.662342in}{1.698212in}}%
\pgfpathlineto{\pgfqpoint{2.668589in}{1.687077in}}%
\pgfpathclose%
\pgfusepath{stroke,fill}%
\end{pgfscope}%
\begin{pgfscope}%
\pgfpathrectangle{\pgfqpoint{0.887500in}{0.275000in}}{\pgfqpoint{4.225000in}{4.225000in}}%
\pgfusepath{clip}%
\pgfsetbuttcap%
\pgfsetroundjoin%
\definecolor{currentfill}{rgb}{0.280894,0.078907,0.402329}%
\pgfsetfillcolor{currentfill}%
\pgfsetfillopacity{0.700000}%
\pgfsetlinewidth{0.501875pt}%
\definecolor{currentstroke}{rgb}{1.000000,1.000000,1.000000}%
\pgfsetstrokecolor{currentstroke}%
\pgfsetstrokeopacity{0.500000}%
\pgfsetdash{}{0pt}%
\pgfpathmoveto{\pgfqpoint{3.572552in}{1.237510in}}%
\pgfpathlineto{\pgfqpoint{3.584021in}{1.241822in}}%
\pgfpathlineto{\pgfqpoint{3.595485in}{1.246163in}}%
\pgfpathlineto{\pgfqpoint{3.606944in}{1.250526in}}%
\pgfpathlineto{\pgfqpoint{3.618398in}{1.254860in}}%
\pgfpathlineto{\pgfqpoint{3.629844in}{1.259113in}}%
\pgfpathlineto{\pgfqpoint{3.623402in}{1.274967in}}%
\pgfpathlineto{\pgfqpoint{3.616958in}{1.290524in}}%
\pgfpathlineto{\pgfqpoint{3.610511in}{1.305798in}}%
\pgfpathlineto{\pgfqpoint{3.604064in}{1.320802in}}%
\pgfpathlineto{\pgfqpoint{3.597615in}{1.335548in}}%
\pgfpathlineto{\pgfqpoint{3.586166in}{1.331037in}}%
\pgfpathlineto{\pgfqpoint{3.574713in}{1.326514in}}%
\pgfpathlineto{\pgfqpoint{3.563254in}{1.321992in}}%
\pgfpathlineto{\pgfqpoint{3.551790in}{1.317489in}}%
\pgfpathlineto{\pgfqpoint{3.540320in}{1.312991in}}%
\pgfpathlineto{\pgfqpoint{3.546770in}{1.298547in}}%
\pgfpathlineto{\pgfqpoint{3.553219in}{1.283805in}}%
\pgfpathlineto{\pgfqpoint{3.559666in}{1.268738in}}%
\pgfpathlineto{\pgfqpoint{3.566110in}{1.253316in}}%
\pgfpathclose%
\pgfusepath{stroke,fill}%
\end{pgfscope}%
\begin{pgfscope}%
\pgfpathrectangle{\pgfqpoint{0.887500in}{0.275000in}}{\pgfqpoint{4.225000in}{4.225000in}}%
\pgfusepath{clip}%
\pgfsetbuttcap%
\pgfsetroundjoin%
\definecolor{currentfill}{rgb}{0.233603,0.313828,0.543914}%
\pgfsetfillcolor{currentfill}%
\pgfsetfillopacity{0.700000}%
\pgfsetlinewidth{0.501875pt}%
\definecolor{currentstroke}{rgb}{1.000000,1.000000,1.000000}%
\pgfsetstrokecolor{currentstroke}%
\pgfsetstrokeopacity{0.500000}%
\pgfsetdash{}{0pt}%
\pgfpathmoveto{\pgfqpoint{2.764569in}{1.639448in}}%
\pgfpathlineto{\pgfqpoint{2.776235in}{1.643431in}}%
\pgfpathlineto{\pgfqpoint{2.787896in}{1.647407in}}%
\pgfpathlineto{\pgfqpoint{2.799550in}{1.651381in}}%
\pgfpathlineto{\pgfqpoint{2.811200in}{1.655354in}}%
\pgfpathlineto{\pgfqpoint{2.822843in}{1.659331in}}%
\pgfpathlineto{\pgfqpoint{2.816556in}{1.670668in}}%
\pgfpathlineto{\pgfqpoint{2.810272in}{1.681954in}}%
\pgfpathlineto{\pgfqpoint{2.803992in}{1.693192in}}%
\pgfpathlineto{\pgfqpoint{2.797716in}{1.704383in}}%
\pgfpathlineto{\pgfqpoint{2.791444in}{1.715527in}}%
\pgfpathlineto{\pgfqpoint{2.779809in}{1.711589in}}%
\pgfpathlineto{\pgfqpoint{2.768168in}{1.707654in}}%
\pgfpathlineto{\pgfqpoint{2.756522in}{1.703719in}}%
\pgfpathlineto{\pgfqpoint{2.744870in}{1.699779in}}%
\pgfpathlineto{\pgfqpoint{2.733212in}{1.695832in}}%
\pgfpathlineto{\pgfqpoint{2.739476in}{1.684645in}}%
\pgfpathlineto{\pgfqpoint{2.745744in}{1.673413in}}%
\pgfpathlineto{\pgfqpoint{2.752015in}{1.662137in}}%
\pgfpathlineto{\pgfqpoint{2.758290in}{1.650817in}}%
\pgfpathclose%
\pgfusepath{stroke,fill}%
\end{pgfscope}%
\begin{pgfscope}%
\pgfpathrectangle{\pgfqpoint{0.887500in}{0.275000in}}{\pgfqpoint{4.225000in}{4.225000in}}%
\pgfusepath{clip}%
\pgfsetbuttcap%
\pgfsetroundjoin%
\definecolor{currentfill}{rgb}{0.197636,0.391528,0.554969}%
\pgfsetfillcolor{currentfill}%
\pgfsetfillopacity{0.700000}%
\pgfsetlinewidth{0.501875pt}%
\definecolor{currentstroke}{rgb}{1.000000,1.000000,1.000000}%
\pgfsetstrokecolor{currentstroke}%
\pgfsetstrokeopacity{0.500000}%
\pgfsetdash{}{0pt}%
\pgfpathmoveto{\pgfqpoint{2.257530in}{1.792188in}}%
\pgfpathlineto{\pgfqpoint{2.269320in}{1.796171in}}%
\pgfpathlineto{\pgfqpoint{2.281104in}{1.800148in}}%
\pgfpathlineto{\pgfqpoint{2.292882in}{1.804119in}}%
\pgfpathlineto{\pgfqpoint{2.304654in}{1.808084in}}%
\pgfpathlineto{\pgfqpoint{2.316421in}{1.812042in}}%
\pgfpathlineto{\pgfqpoint{2.310290in}{1.822454in}}%
\pgfpathlineto{\pgfqpoint{2.304162in}{1.832824in}}%
\pgfpathlineto{\pgfqpoint{2.298039in}{1.843150in}}%
\pgfpathlineto{\pgfqpoint{2.291921in}{1.853433in}}%
\pgfpathlineto{\pgfqpoint{2.285807in}{1.863673in}}%
\pgfpathlineto{\pgfqpoint{2.274050in}{1.859763in}}%
\pgfpathlineto{\pgfqpoint{2.262287in}{1.855845in}}%
\pgfpathlineto{\pgfqpoint{2.250518in}{1.851920in}}%
\pgfpathlineto{\pgfqpoint{2.238744in}{1.847987in}}%
\pgfpathlineto{\pgfqpoint{2.226964in}{1.844046in}}%
\pgfpathlineto{\pgfqpoint{2.233069in}{1.833758in}}%
\pgfpathlineto{\pgfqpoint{2.239177in}{1.823428in}}%
\pgfpathlineto{\pgfqpoint{2.245291in}{1.813056in}}%
\pgfpathlineto{\pgfqpoint{2.251408in}{1.802643in}}%
\pgfpathclose%
\pgfusepath{stroke,fill}%
\end{pgfscope}%
\begin{pgfscope}%
\pgfpathrectangle{\pgfqpoint{0.887500in}{0.275000in}}{\pgfqpoint{4.225000in}{4.225000in}}%
\pgfusepath{clip}%
\pgfsetbuttcap%
\pgfsetroundjoin%
\definecolor{currentfill}{rgb}{0.283197,0.115680,0.436115}%
\pgfsetfillcolor{currentfill}%
\pgfsetfillopacity{0.700000}%
\pgfsetlinewidth{0.501875pt}%
\definecolor{currentstroke}{rgb}{1.000000,1.000000,1.000000}%
\pgfsetstrokecolor{currentstroke}%
\pgfsetstrokeopacity{0.500000}%
\pgfsetdash{}{0pt}%
\pgfpathmoveto{\pgfqpoint{3.482854in}{1.286791in}}%
\pgfpathlineto{\pgfqpoint{3.494367in}{1.292913in}}%
\pgfpathlineto{\pgfqpoint{3.505869in}{1.298478in}}%
\pgfpathlineto{\pgfqpoint{3.517362in}{1.303601in}}%
\pgfpathlineto{\pgfqpoint{3.528845in}{1.308400in}}%
\pgfpathlineto{\pgfqpoint{3.540320in}{1.312991in}}%
\pgfpathlineto{\pgfqpoint{3.533870in}{1.327166in}}%
\pgfpathlineto{\pgfqpoint{3.527419in}{1.341100in}}%
\pgfpathlineto{\pgfqpoint{3.520969in}{1.354824in}}%
\pgfpathlineto{\pgfqpoint{3.514519in}{1.368364in}}%
\pgfpathlineto{\pgfqpoint{3.508070in}{1.381750in}}%
\pgfpathlineto{\pgfqpoint{3.496602in}{1.377385in}}%
\pgfpathlineto{\pgfqpoint{3.485128in}{1.373015in}}%
\pgfpathlineto{\pgfqpoint{3.473648in}{1.368548in}}%
\pgfpathlineto{\pgfqpoint{3.462161in}{1.363898in}}%
\pgfpathlineto{\pgfqpoint{3.450665in}{1.358975in}}%
\pgfpathlineto{\pgfqpoint{3.457100in}{1.344781in}}%
\pgfpathlineto{\pgfqpoint{3.463537in}{1.330469in}}%
\pgfpathlineto{\pgfqpoint{3.469975in}{1.316036in}}%
\pgfpathlineto{\pgfqpoint{3.476414in}{1.301478in}}%
\pgfpathclose%
\pgfusepath{stroke,fill}%
\end{pgfscope}%
\begin{pgfscope}%
\pgfpathrectangle{\pgfqpoint{0.887500in}{0.275000in}}{\pgfqpoint{4.225000in}{4.225000in}}%
\pgfusepath{clip}%
\pgfsetbuttcap%
\pgfsetroundjoin%
\definecolor{currentfill}{rgb}{0.241237,0.296485,0.539709}%
\pgfsetfillcolor{currentfill}%
\pgfsetfillopacity{0.700000}%
\pgfsetlinewidth{0.501875pt}%
\definecolor{currentstroke}{rgb}{1.000000,1.000000,1.000000}%
\pgfsetstrokecolor{currentstroke}%
\pgfsetstrokeopacity{0.500000}%
\pgfsetdash{}{0pt}%
\pgfpathmoveto{\pgfqpoint{2.854335in}{1.601770in}}%
\pgfpathlineto{\pgfqpoint{2.865981in}{1.605780in}}%
\pgfpathlineto{\pgfqpoint{2.877621in}{1.609794in}}%
\pgfpathlineto{\pgfqpoint{2.889256in}{1.613810in}}%
\pgfpathlineto{\pgfqpoint{2.900885in}{1.617826in}}%
\pgfpathlineto{\pgfqpoint{2.912508in}{1.621840in}}%
\pgfpathlineto{\pgfqpoint{2.906194in}{1.633450in}}%
\pgfpathlineto{\pgfqpoint{2.899884in}{1.644996in}}%
\pgfpathlineto{\pgfqpoint{2.893577in}{1.656480in}}%
\pgfpathlineto{\pgfqpoint{2.887274in}{1.667902in}}%
\pgfpathlineto{\pgfqpoint{2.880975in}{1.679267in}}%
\pgfpathlineto{\pgfqpoint{2.869360in}{1.675279in}}%
\pgfpathlineto{\pgfqpoint{2.857739in}{1.671289in}}%
\pgfpathlineto{\pgfqpoint{2.846113in}{1.667300in}}%
\pgfpathlineto{\pgfqpoint{2.834481in}{1.663314in}}%
\pgfpathlineto{\pgfqpoint{2.822843in}{1.659331in}}%
\pgfpathlineto{\pgfqpoint{2.829134in}{1.647940in}}%
\pgfpathlineto{\pgfqpoint{2.835429in}{1.636491in}}%
\pgfpathlineto{\pgfqpoint{2.841727in}{1.624982in}}%
\pgfpathlineto{\pgfqpoint{2.848030in}{1.613409in}}%
\pgfpathclose%
\pgfusepath{stroke,fill}%
\end{pgfscope}%
\begin{pgfscope}%
\pgfpathrectangle{\pgfqpoint{0.887500in}{0.275000in}}{\pgfqpoint{4.225000in}{4.225000in}}%
\pgfusepath{clip}%
\pgfsetbuttcap%
\pgfsetroundjoin%
\definecolor{currentfill}{rgb}{0.282623,0.140926,0.457517}%
\pgfsetfillcolor{currentfill}%
\pgfsetfillopacity{0.700000}%
\pgfsetlinewidth{0.501875pt}%
\definecolor{currentstroke}{rgb}{1.000000,1.000000,1.000000}%
\pgfsetstrokecolor{currentstroke}%
\pgfsetstrokeopacity{0.500000}%
\pgfsetdash{}{0pt}%
\pgfpathmoveto{\pgfqpoint{3.393078in}{1.330364in}}%
\pgfpathlineto{\pgfqpoint{3.404606in}{1.336196in}}%
\pgfpathlineto{\pgfqpoint{3.416130in}{1.342140in}}%
\pgfpathlineto{\pgfqpoint{3.427649in}{1.348028in}}%
\pgfpathlineto{\pgfqpoint{3.439161in}{1.353689in}}%
\pgfpathlineto{\pgfqpoint{3.450665in}{1.358975in}}%
\pgfpathlineto{\pgfqpoint{3.444232in}{1.373054in}}%
\pgfpathlineto{\pgfqpoint{3.437800in}{1.387023in}}%
\pgfpathlineto{\pgfqpoint{3.431370in}{1.400886in}}%
\pgfpathlineto{\pgfqpoint{3.424941in}{1.414646in}}%
\pgfpathlineto{\pgfqpoint{3.418515in}{1.428306in}}%
\pgfpathlineto{\pgfqpoint{3.407029in}{1.424454in}}%
\pgfpathlineto{\pgfqpoint{3.395536in}{1.420461in}}%
\pgfpathlineto{\pgfqpoint{3.384037in}{1.416357in}}%
\pgfpathlineto{\pgfqpoint{3.372532in}{1.412183in}}%
\pgfpathlineto{\pgfqpoint{3.361021in}{1.407978in}}%
\pgfpathlineto{\pgfqpoint{3.367432in}{1.393002in}}%
\pgfpathlineto{\pgfqpoint{3.373844in}{1.377708in}}%
\pgfpathlineto{\pgfqpoint{3.380255in}{1.362141in}}%
\pgfpathlineto{\pgfqpoint{3.386666in}{1.346345in}}%
\pgfpathclose%
\pgfusepath{stroke,fill}%
\end{pgfscope}%
\begin{pgfscope}%
\pgfpathrectangle{\pgfqpoint{0.887500in}{0.275000in}}{\pgfqpoint{4.225000in}{4.225000in}}%
\pgfusepath{clip}%
\pgfsetbuttcap%
\pgfsetroundjoin%
\definecolor{currentfill}{rgb}{0.250425,0.274290,0.533103}%
\pgfsetfillcolor{currentfill}%
\pgfsetfillopacity{0.700000}%
\pgfsetlinewidth{0.501875pt}%
\definecolor{currentstroke}{rgb}{1.000000,1.000000,1.000000}%
\pgfsetstrokecolor{currentstroke}%
\pgfsetstrokeopacity{0.500000}%
\pgfsetdash{}{0pt}%
\pgfpathmoveto{\pgfqpoint{2.944131in}{1.562757in}}%
\pgfpathlineto{\pgfqpoint{2.955756in}{1.566845in}}%
\pgfpathlineto{\pgfqpoint{2.967376in}{1.570949in}}%
\pgfpathlineto{\pgfqpoint{2.978990in}{1.575064in}}%
\pgfpathlineto{\pgfqpoint{2.990598in}{1.579183in}}%
\pgfpathlineto{\pgfqpoint{3.002201in}{1.583296in}}%
\pgfpathlineto{\pgfqpoint{2.995862in}{1.595096in}}%
\pgfpathlineto{\pgfqpoint{2.989526in}{1.606843in}}%
\pgfpathlineto{\pgfqpoint{2.983194in}{1.618545in}}%
\pgfpathlineto{\pgfqpoint{2.976865in}{1.630202in}}%
\pgfpathlineto{\pgfqpoint{2.970540in}{1.641813in}}%
\pgfpathlineto{\pgfqpoint{2.958945in}{1.637834in}}%
\pgfpathlineto{\pgfqpoint{2.947344in}{1.633847in}}%
\pgfpathlineto{\pgfqpoint{2.935738in}{1.629852in}}%
\pgfpathlineto{\pgfqpoint{2.924126in}{1.625849in}}%
\pgfpathlineto{\pgfqpoint{2.912508in}{1.621840in}}%
\pgfpathlineto{\pgfqpoint{2.918826in}{1.610163in}}%
\pgfpathlineto{\pgfqpoint{2.925147in}{1.598418in}}%
\pgfpathlineto{\pgfqpoint{2.931471in}{1.586602in}}%
\pgfpathlineto{\pgfqpoint{2.937799in}{1.574714in}}%
\pgfpathclose%
\pgfusepath{stroke,fill}%
\end{pgfscope}%
\begin{pgfscope}%
\pgfpathrectangle{\pgfqpoint{0.887500in}{0.275000in}}{\pgfqpoint{4.225000in}{4.225000in}}%
\pgfusepath{clip}%
\pgfsetbuttcap%
\pgfsetroundjoin%
\definecolor{currentfill}{rgb}{0.204903,0.375746,0.553533}%
\pgfsetfillcolor{currentfill}%
\pgfsetfillopacity{0.700000}%
\pgfsetlinewidth{0.501875pt}%
\definecolor{currentstroke}{rgb}{1.000000,1.000000,1.000000}%
\pgfsetstrokecolor{currentstroke}%
\pgfsetstrokeopacity{0.500000}%
\pgfsetdash{}{0pt}%
\pgfpathmoveto{\pgfqpoint{2.347144in}{1.759332in}}%
\pgfpathlineto{\pgfqpoint{2.358915in}{1.763323in}}%
\pgfpathlineto{\pgfqpoint{2.370681in}{1.767307in}}%
\pgfpathlineto{\pgfqpoint{2.382440in}{1.771284in}}%
\pgfpathlineto{\pgfqpoint{2.394194in}{1.775254in}}%
\pgfpathlineto{\pgfqpoint{2.405943in}{1.779220in}}%
\pgfpathlineto{\pgfqpoint{2.399780in}{1.789809in}}%
\pgfpathlineto{\pgfqpoint{2.393622in}{1.800353in}}%
\pgfpathlineto{\pgfqpoint{2.387468in}{1.810853in}}%
\pgfpathlineto{\pgfqpoint{2.381318in}{1.821308in}}%
\pgfpathlineto{\pgfqpoint{2.375172in}{1.831718in}}%
\pgfpathlineto{\pgfqpoint{2.363433in}{1.827796in}}%
\pgfpathlineto{\pgfqpoint{2.351689in}{1.823869in}}%
\pgfpathlineto{\pgfqpoint{2.339938in}{1.819935in}}%
\pgfpathlineto{\pgfqpoint{2.328183in}{1.815992in}}%
\pgfpathlineto{\pgfqpoint{2.316421in}{1.812042in}}%
\pgfpathlineto{\pgfqpoint{2.322557in}{1.801586in}}%
\pgfpathlineto{\pgfqpoint{2.328698in}{1.791087in}}%
\pgfpathlineto{\pgfqpoint{2.334842in}{1.780545in}}%
\pgfpathlineto{\pgfqpoint{2.340991in}{1.769960in}}%
\pgfpathclose%
\pgfusepath{stroke,fill}%
\end{pgfscope}%
\begin{pgfscope}%
\pgfpathrectangle{\pgfqpoint{0.887500in}{0.275000in}}{\pgfqpoint{4.225000in}{4.225000in}}%
\pgfusepath{clip}%
\pgfsetbuttcap%
\pgfsetroundjoin%
\definecolor{currentfill}{rgb}{0.180629,0.429975,0.557282}%
\pgfsetfillcolor{currentfill}%
\pgfsetfillopacity{0.700000}%
\pgfsetlinewidth{0.501875pt}%
\definecolor{currentstroke}{rgb}{1.000000,1.000000,1.000000}%
\pgfsetstrokecolor{currentstroke}%
\pgfsetstrokeopacity{0.500000}%
\pgfsetdash{}{0pt}%
\pgfpathmoveto{\pgfqpoint{1.929778in}{1.866903in}}%
\pgfpathlineto{\pgfqpoint{1.941651in}{1.870839in}}%
\pgfpathlineto{\pgfqpoint{1.953519in}{1.874766in}}%
\pgfpathlineto{\pgfqpoint{1.965381in}{1.878682in}}%
\pgfpathlineto{\pgfqpoint{1.977237in}{1.882591in}}%
\pgfpathlineto{\pgfqpoint{1.989088in}{1.886492in}}%
\pgfpathlineto{\pgfqpoint{1.983064in}{1.896465in}}%
\pgfpathlineto{\pgfqpoint{1.977044in}{1.906400in}}%
\pgfpathlineto{\pgfqpoint{1.971029in}{1.916299in}}%
\pgfpathlineto{\pgfqpoint{1.965018in}{1.926161in}}%
\pgfpathlineto{\pgfqpoint{1.959012in}{1.935987in}}%
\pgfpathlineto{\pgfqpoint{1.947172in}{1.932133in}}%
\pgfpathlineto{\pgfqpoint{1.935325in}{1.928271in}}%
\pgfpathlineto{\pgfqpoint{1.923473in}{1.924400in}}%
\pgfpathlineto{\pgfqpoint{1.911616in}{1.920518in}}%
\pgfpathlineto{\pgfqpoint{1.899753in}{1.916625in}}%
\pgfpathlineto{\pgfqpoint{1.905749in}{1.906752in}}%
\pgfpathlineto{\pgfqpoint{1.911749in}{1.896844in}}%
\pgfpathlineto{\pgfqpoint{1.917754in}{1.886900in}}%
\pgfpathlineto{\pgfqpoint{1.923764in}{1.876920in}}%
\pgfpathclose%
\pgfusepath{stroke,fill}%
\end{pgfscope}%
\begin{pgfscope}%
\pgfpathrectangle{\pgfqpoint{0.887500in}{0.275000in}}{\pgfqpoint{4.225000in}{4.225000in}}%
\pgfusepath{clip}%
\pgfsetbuttcap%
\pgfsetroundjoin%
\definecolor{currentfill}{rgb}{0.278826,0.175490,0.483397}%
\pgfsetfillcolor{currentfill}%
\pgfsetfillopacity{0.700000}%
\pgfsetlinewidth{0.501875pt}%
\definecolor{currentstroke}{rgb}{1.000000,1.000000,1.000000}%
\pgfsetstrokecolor{currentstroke}%
\pgfsetstrokeopacity{0.500000}%
\pgfsetdash{}{0pt}%
\pgfpathmoveto{\pgfqpoint{3.303389in}{1.387762in}}%
\pgfpathlineto{\pgfqpoint{3.314925in}{1.391616in}}%
\pgfpathlineto{\pgfqpoint{3.326457in}{1.395570in}}%
\pgfpathlineto{\pgfqpoint{3.337983in}{1.399632in}}%
\pgfpathlineto{\pgfqpoint{3.349505in}{1.403782in}}%
\pgfpathlineto{\pgfqpoint{3.361021in}{1.407978in}}%
\pgfpathlineto{\pgfqpoint{3.354609in}{1.422593in}}%
\pgfpathlineto{\pgfqpoint{3.348197in}{1.436817in}}%
\pgfpathlineto{\pgfqpoint{3.341784in}{1.450677in}}%
\pgfpathlineto{\pgfqpoint{3.335372in}{1.464208in}}%
\pgfpathlineto{\pgfqpoint{3.328961in}{1.477450in}}%
\pgfpathlineto{\pgfqpoint{3.317452in}{1.473686in}}%
\pgfpathlineto{\pgfqpoint{3.305937in}{1.469840in}}%
\pgfpathlineto{\pgfqpoint{3.294415in}{1.465906in}}%
\pgfpathlineto{\pgfqpoint{3.282887in}{1.461882in}}%
\pgfpathlineto{\pgfqpoint{3.271354in}{1.457768in}}%
\pgfpathlineto{\pgfqpoint{3.277760in}{1.444525in}}%
\pgfpathlineto{\pgfqpoint{3.284167in}{1.430947in}}%
\pgfpathlineto{\pgfqpoint{3.290574in}{1.416989in}}%
\pgfpathlineto{\pgfqpoint{3.296982in}{1.402605in}}%
\pgfpathclose%
\pgfusepath{stroke,fill}%
\end{pgfscope}%
\begin{pgfscope}%
\pgfpathrectangle{\pgfqpoint{0.887500in}{0.275000in}}{\pgfqpoint{4.225000in}{4.225000in}}%
\pgfusepath{clip}%
\pgfsetbuttcap%
\pgfsetroundjoin%
\definecolor{currentfill}{rgb}{0.257322,0.256130,0.526563}%
\pgfsetfillcolor{currentfill}%
\pgfsetfillopacity{0.700000}%
\pgfsetlinewidth{0.501875pt}%
\definecolor{currentstroke}{rgb}{1.000000,1.000000,1.000000}%
\pgfsetstrokecolor{currentstroke}%
\pgfsetstrokeopacity{0.500000}%
\pgfsetdash{}{0pt}%
\pgfpathmoveto{\pgfqpoint{3.033947in}{1.523088in}}%
\pgfpathlineto{\pgfqpoint{3.045551in}{1.527291in}}%
\pgfpathlineto{\pgfqpoint{3.057150in}{1.531452in}}%
\pgfpathlineto{\pgfqpoint{3.068744in}{1.535551in}}%
\pgfpathlineto{\pgfqpoint{3.080332in}{1.539571in}}%
\pgfpathlineto{\pgfqpoint{3.091914in}{1.543507in}}%
\pgfpathlineto{\pgfqpoint{3.085551in}{1.555735in}}%
\pgfpathlineto{\pgfqpoint{3.079191in}{1.567824in}}%
\pgfpathlineto{\pgfqpoint{3.072835in}{1.579791in}}%
\pgfpathlineto{\pgfqpoint{3.066481in}{1.591654in}}%
\pgfpathlineto{\pgfqpoint{3.060131in}{1.603433in}}%
\pgfpathlineto{\pgfqpoint{3.048556in}{1.599490in}}%
\pgfpathlineto{\pgfqpoint{3.036976in}{1.595499in}}%
\pgfpathlineto{\pgfqpoint{3.025390in}{1.591463in}}%
\pgfpathlineto{\pgfqpoint{3.013798in}{1.587392in}}%
\pgfpathlineto{\pgfqpoint{3.002201in}{1.583296in}}%
\pgfpathlineto{\pgfqpoint{3.008543in}{1.571430in}}%
\pgfpathlineto{\pgfqpoint{3.014889in}{1.559488in}}%
\pgfpathlineto{\pgfqpoint{3.021238in}{1.547459in}}%
\pgfpathlineto{\pgfqpoint{3.027591in}{1.535329in}}%
\pgfpathclose%
\pgfusepath{stroke,fill}%
\end{pgfscope}%
\begin{pgfscope}%
\pgfpathrectangle{\pgfqpoint{0.887500in}{0.275000in}}{\pgfqpoint{4.225000in}{4.225000in}}%
\pgfusepath{clip}%
\pgfsetbuttcap%
\pgfsetroundjoin%
\definecolor{currentfill}{rgb}{0.266580,0.228262,0.514349}%
\pgfsetfillcolor{currentfill}%
\pgfsetfillopacity{0.700000}%
\pgfsetlinewidth{0.501875pt}%
\definecolor{currentstroke}{rgb}{1.000000,1.000000,1.000000}%
\pgfsetstrokecolor{currentstroke}%
\pgfsetstrokeopacity{0.500000}%
\pgfsetdash{}{0pt}%
\pgfpathmoveto{\pgfqpoint{3.123773in}{1.479653in}}%
\pgfpathlineto{\pgfqpoint{3.135355in}{1.483737in}}%
\pgfpathlineto{\pgfqpoint{3.146933in}{1.487847in}}%
\pgfpathlineto{\pgfqpoint{3.158505in}{1.491973in}}%
\pgfpathlineto{\pgfqpoint{3.170071in}{1.496104in}}%
\pgfpathlineto{\pgfqpoint{3.181632in}{1.500231in}}%
\pgfpathlineto{\pgfqpoint{3.175247in}{1.512852in}}%
\pgfpathlineto{\pgfqpoint{3.168866in}{1.525388in}}%
\pgfpathlineto{\pgfqpoint{3.162487in}{1.537836in}}%
\pgfpathlineto{\pgfqpoint{3.156111in}{1.550200in}}%
\pgfpathlineto{\pgfqpoint{3.149739in}{1.562481in}}%
\pgfpathlineto{\pgfqpoint{3.138185in}{1.558725in}}%
\pgfpathlineto{\pgfqpoint{3.126626in}{1.554966in}}%
\pgfpathlineto{\pgfqpoint{3.115061in}{1.551187in}}%
\pgfpathlineto{\pgfqpoint{3.103490in}{1.547373in}}%
\pgfpathlineto{\pgfqpoint{3.091914in}{1.543507in}}%
\pgfpathlineto{\pgfqpoint{3.098280in}{1.531119in}}%
\pgfpathlineto{\pgfqpoint{3.104649in}{1.518554in}}%
\pgfpathlineto{\pgfqpoint{3.111021in}{1.505794in}}%
\pgfpathlineto{\pgfqpoint{3.117395in}{1.492820in}}%
\pgfpathclose%
\pgfusepath{stroke,fill}%
\end{pgfscope}%
\begin{pgfscope}%
\pgfpathrectangle{\pgfqpoint{0.887500in}{0.275000in}}{\pgfqpoint{4.225000in}{4.225000in}}%
\pgfusepath{clip}%
\pgfsetbuttcap%
\pgfsetroundjoin%
\definecolor{currentfill}{rgb}{0.273006,0.204520,0.501721}%
\pgfsetfillcolor{currentfill}%
\pgfsetfillopacity{0.700000}%
\pgfsetlinewidth{0.501875pt}%
\definecolor{currentstroke}{rgb}{1.000000,1.000000,1.000000}%
\pgfsetstrokecolor{currentstroke}%
\pgfsetstrokeopacity{0.500000}%
\pgfsetdash{}{0pt}%
\pgfpathmoveto{\pgfqpoint{3.213596in}{1.435897in}}%
\pgfpathlineto{\pgfqpoint{3.225159in}{1.440440in}}%
\pgfpathlineto{\pgfqpoint{3.236716in}{1.444901in}}%
\pgfpathlineto{\pgfqpoint{3.248268in}{1.449277in}}%
\pgfpathlineto{\pgfqpoint{3.259814in}{1.453566in}}%
\pgfpathlineto{\pgfqpoint{3.271354in}{1.457768in}}%
\pgfpathlineto{\pgfqpoint{3.264949in}{1.470724in}}%
\pgfpathlineto{\pgfqpoint{3.258546in}{1.483439in}}%
\pgfpathlineto{\pgfqpoint{3.252145in}{1.495959in}}%
\pgfpathlineto{\pgfqpoint{3.245747in}{1.508331in}}%
\pgfpathlineto{\pgfqpoint{3.239351in}{1.520602in}}%
\pgfpathlineto{\pgfqpoint{3.227819in}{1.516561in}}%
\pgfpathlineto{\pgfqpoint{3.216280in}{1.512505in}}%
\pgfpathlineto{\pgfqpoint{3.204736in}{1.508433in}}%
\pgfpathlineto{\pgfqpoint{3.193187in}{1.504342in}}%
\pgfpathlineto{\pgfqpoint{3.181632in}{1.500231in}}%
\pgfpathlineto{\pgfqpoint{3.188019in}{1.487525in}}%
\pgfpathlineto{\pgfqpoint{3.194409in}{1.474736in}}%
\pgfpathlineto{\pgfqpoint{3.200802in}{1.461867in}}%
\pgfpathlineto{\pgfqpoint{3.207198in}{1.448920in}}%
\pgfpathclose%
\pgfusepath{stroke,fill}%
\end{pgfscope}%
\begin{pgfscope}%
\pgfpathrectangle{\pgfqpoint{0.887500in}{0.275000in}}{\pgfqpoint{4.225000in}{4.225000in}}%
\pgfusepath{clip}%
\pgfsetbuttcap%
\pgfsetroundjoin%
\definecolor{currentfill}{rgb}{0.210503,0.363727,0.552206}%
\pgfsetfillcolor{currentfill}%
\pgfsetfillopacity{0.700000}%
\pgfsetlinewidth{0.501875pt}%
\definecolor{currentstroke}{rgb}{1.000000,1.000000,1.000000}%
\pgfsetstrokecolor{currentstroke}%
\pgfsetstrokeopacity{0.500000}%
\pgfsetdash{}{0pt}%
\pgfpathmoveto{\pgfqpoint{2.436820in}{1.725603in}}%
\pgfpathlineto{\pgfqpoint{2.448572in}{1.729604in}}%
\pgfpathlineto{\pgfqpoint{2.460319in}{1.733605in}}%
\pgfpathlineto{\pgfqpoint{2.472059in}{1.737608in}}%
\pgfpathlineto{\pgfqpoint{2.483794in}{1.741614in}}%
\pgfpathlineto{\pgfqpoint{2.495523in}{1.745625in}}%
\pgfpathlineto{\pgfqpoint{2.489330in}{1.756403in}}%
\pgfpathlineto{\pgfqpoint{2.483141in}{1.767134in}}%
\pgfpathlineto{\pgfqpoint{2.476956in}{1.777819in}}%
\pgfpathlineto{\pgfqpoint{2.470776in}{1.788458in}}%
\pgfpathlineto{\pgfqpoint{2.464600in}{1.799050in}}%
\pgfpathlineto{\pgfqpoint{2.452880in}{1.795078in}}%
\pgfpathlineto{\pgfqpoint{2.441154in}{1.791110in}}%
\pgfpathlineto{\pgfqpoint{2.429423in}{1.787146in}}%
\pgfpathlineto{\pgfqpoint{2.417686in}{1.783183in}}%
\pgfpathlineto{\pgfqpoint{2.405943in}{1.779220in}}%
\pgfpathlineto{\pgfqpoint{2.412110in}{1.768587in}}%
\pgfpathlineto{\pgfqpoint{2.418281in}{1.757908in}}%
\pgfpathlineto{\pgfqpoint{2.424457in}{1.747185in}}%
\pgfpathlineto{\pgfqpoint{2.430636in}{1.736417in}}%
\pgfpathclose%
\pgfusepath{stroke,fill}%
\end{pgfscope}%
\begin{pgfscope}%
\pgfpathrectangle{\pgfqpoint{0.887500in}{0.275000in}}{\pgfqpoint{4.225000in}{4.225000in}}%
\pgfusepath{clip}%
\pgfsetbuttcap%
\pgfsetroundjoin%
\definecolor{currentfill}{rgb}{0.187231,0.414746,0.556547}%
\pgfsetfillcolor{currentfill}%
\pgfsetfillopacity{0.700000}%
\pgfsetlinewidth{0.501875pt}%
\definecolor{currentstroke}{rgb}{1.000000,1.000000,1.000000}%
\pgfsetstrokecolor{currentstroke}%
\pgfsetstrokeopacity{0.500000}%
\pgfsetdash{}{0pt}%
\pgfpathmoveto{\pgfqpoint{2.019279in}{1.836052in}}%
\pgfpathlineto{\pgfqpoint{2.031134in}{1.839990in}}%
\pgfpathlineto{\pgfqpoint{2.042984in}{1.843922in}}%
\pgfpathlineto{\pgfqpoint{2.054828in}{1.847851in}}%
\pgfpathlineto{\pgfqpoint{2.066667in}{1.851776in}}%
\pgfpathlineto{\pgfqpoint{2.078499in}{1.855698in}}%
\pgfpathlineto{\pgfqpoint{2.072442in}{1.865822in}}%
\pgfpathlineto{\pgfqpoint{2.066389in}{1.875905in}}%
\pgfpathlineto{\pgfqpoint{2.060341in}{1.885949in}}%
\pgfpathlineto{\pgfqpoint{2.054297in}{1.895953in}}%
\pgfpathlineto{\pgfqpoint{2.048258in}{1.905918in}}%
\pgfpathlineto{\pgfqpoint{2.036435in}{1.902040in}}%
\pgfpathlineto{\pgfqpoint{2.024607in}{1.898159in}}%
\pgfpathlineto{\pgfqpoint{2.012773in}{1.894275in}}%
\pgfpathlineto{\pgfqpoint{2.000934in}{1.890386in}}%
\pgfpathlineto{\pgfqpoint{1.989088in}{1.886492in}}%
\pgfpathlineto{\pgfqpoint{1.995117in}{1.876481in}}%
\pgfpathlineto{\pgfqpoint{2.001151in}{1.866432in}}%
\pgfpathlineto{\pgfqpoint{2.007189in}{1.856344in}}%
\pgfpathlineto{\pgfqpoint{2.013232in}{1.846218in}}%
\pgfpathclose%
\pgfusepath{stroke,fill}%
\end{pgfscope}%
\begin{pgfscope}%
\pgfpathrectangle{\pgfqpoint{0.887500in}{0.275000in}}{\pgfqpoint{4.225000in}{4.225000in}}%
\pgfusepath{clip}%
\pgfsetbuttcap%
\pgfsetroundjoin%
\definecolor{currentfill}{rgb}{0.272594,0.025563,0.353093}%
\pgfsetfillcolor{currentfill}%
\pgfsetfillopacity{0.700000}%
\pgfsetlinewidth{0.501875pt}%
\definecolor{currentstroke}{rgb}{1.000000,1.000000,1.000000}%
\pgfsetstrokecolor{currentstroke}%
\pgfsetstrokeopacity{0.500000}%
\pgfsetdash{}{0pt}%
\pgfpathmoveto{\pgfqpoint{3.604696in}{1.151745in}}%
\pgfpathlineto{\pgfqpoint{3.616157in}{1.155542in}}%
\pgfpathlineto{\pgfqpoint{3.627619in}{1.159774in}}%
\pgfpathlineto{\pgfqpoint{3.639082in}{1.164449in}}%
\pgfpathlineto{\pgfqpoint{3.650546in}{1.169519in}}%
\pgfpathlineto{\pgfqpoint{3.662010in}{1.174937in}}%
\pgfpathlineto{\pgfqpoint{3.655585in}{1.192470in}}%
\pgfpathlineto{\pgfqpoint{3.649155in}{1.209641in}}%
\pgfpathlineto{\pgfqpoint{3.642721in}{1.226463in}}%
\pgfpathlineto{\pgfqpoint{3.636284in}{1.242950in}}%
\pgfpathlineto{\pgfqpoint{3.629844in}{1.259113in}}%
\pgfpathlineto{\pgfqpoint{3.618398in}{1.254860in}}%
\pgfpathlineto{\pgfqpoint{3.606944in}{1.250526in}}%
\pgfpathlineto{\pgfqpoint{3.595485in}{1.246163in}}%
\pgfpathlineto{\pgfqpoint{3.584021in}{1.241822in}}%
\pgfpathlineto{\pgfqpoint{3.572552in}{1.237510in}}%
\pgfpathlineto{\pgfqpoint{3.578990in}{1.221294in}}%
\pgfpathlineto{\pgfqpoint{3.585424in}{1.204637in}}%
\pgfpathlineto{\pgfqpoint{3.591853in}{1.187512in}}%
\pgfpathlineto{\pgfqpoint{3.598278in}{1.169891in}}%
\pgfpathclose%
\pgfusepath{stroke,fill}%
\end{pgfscope}%
\begin{pgfscope}%
\pgfpathrectangle{\pgfqpoint{0.887500in}{0.275000in}}{\pgfqpoint{4.225000in}{4.225000in}}%
\pgfusepath{clip}%
\pgfsetbuttcap%
\pgfsetroundjoin%
\definecolor{currentfill}{rgb}{0.218130,0.347432,0.550038}%
\pgfsetfillcolor{currentfill}%
\pgfsetfillopacity{0.700000}%
\pgfsetlinewidth{0.501875pt}%
\definecolor{currentstroke}{rgb}{1.000000,1.000000,1.000000}%
\pgfsetstrokecolor{currentstroke}%
\pgfsetstrokeopacity{0.500000}%
\pgfsetdash{}{0pt}%
\pgfpathmoveto{\pgfqpoint{2.526551in}{1.691036in}}%
\pgfpathlineto{\pgfqpoint{2.538283in}{1.695081in}}%
\pgfpathlineto{\pgfqpoint{2.550010in}{1.699128in}}%
\pgfpathlineto{\pgfqpoint{2.561731in}{1.703175in}}%
\pgfpathlineto{\pgfqpoint{2.573447in}{1.707220in}}%
\pgfpathlineto{\pgfqpoint{2.585157in}{1.711261in}}%
\pgfpathlineto{\pgfqpoint{2.578934in}{1.722244in}}%
\pgfpathlineto{\pgfqpoint{2.572715in}{1.733178in}}%
\pgfpathlineto{\pgfqpoint{2.566501in}{1.744065in}}%
\pgfpathlineto{\pgfqpoint{2.560290in}{1.754903in}}%
\pgfpathlineto{\pgfqpoint{2.554083in}{1.765692in}}%
\pgfpathlineto{\pgfqpoint{2.542383in}{1.761685in}}%
\pgfpathlineto{\pgfqpoint{2.530676in}{1.757672in}}%
\pgfpathlineto{\pgfqpoint{2.518964in}{1.753656in}}%
\pgfpathlineto{\pgfqpoint{2.507247in}{1.749640in}}%
\pgfpathlineto{\pgfqpoint{2.495523in}{1.745625in}}%
\pgfpathlineto{\pgfqpoint{2.501720in}{1.734800in}}%
\pgfpathlineto{\pgfqpoint{2.507922in}{1.723929in}}%
\pgfpathlineto{\pgfqpoint{2.514127in}{1.713011in}}%
\pgfpathlineto{\pgfqpoint{2.520337in}{1.702046in}}%
\pgfpathclose%
\pgfusepath{stroke,fill}%
\end{pgfscope}%
\begin{pgfscope}%
\pgfpathrectangle{\pgfqpoint{0.887500in}{0.275000in}}{\pgfqpoint{4.225000in}{4.225000in}}%
\pgfusepath{clip}%
\pgfsetbuttcap%
\pgfsetroundjoin%
\definecolor{currentfill}{rgb}{0.192357,0.403199,0.555836}%
\pgfsetfillcolor{currentfill}%
\pgfsetfillopacity{0.700000}%
\pgfsetlinewidth{0.501875pt}%
\definecolor{currentstroke}{rgb}{1.000000,1.000000,1.000000}%
\pgfsetstrokecolor{currentstroke}%
\pgfsetstrokeopacity{0.500000}%
\pgfsetdash{}{0pt}%
\pgfpathmoveto{\pgfqpoint{2.108855in}{1.804469in}}%
\pgfpathlineto{\pgfqpoint{2.120691in}{1.808431in}}%
\pgfpathlineto{\pgfqpoint{2.132522in}{1.812391in}}%
\pgfpathlineto{\pgfqpoint{2.144348in}{1.816351in}}%
\pgfpathlineto{\pgfqpoint{2.156167in}{1.820312in}}%
\pgfpathlineto{\pgfqpoint{2.167981in}{1.824273in}}%
\pgfpathlineto{\pgfqpoint{2.161891in}{1.834561in}}%
\pgfpathlineto{\pgfqpoint{2.155806in}{1.844807in}}%
\pgfpathlineto{\pgfqpoint{2.149724in}{1.855012in}}%
\pgfpathlineto{\pgfqpoint{2.143648in}{1.865176in}}%
\pgfpathlineto{\pgfqpoint{2.137576in}{1.875298in}}%
\pgfpathlineto{\pgfqpoint{2.125772in}{1.871378in}}%
\pgfpathlineto{\pgfqpoint{2.113963in}{1.867458in}}%
\pgfpathlineto{\pgfqpoint{2.102147in}{1.863538in}}%
\pgfpathlineto{\pgfqpoint{2.090326in}{1.859619in}}%
\pgfpathlineto{\pgfqpoint{2.078499in}{1.855698in}}%
\pgfpathlineto{\pgfqpoint{2.084561in}{1.845534in}}%
\pgfpathlineto{\pgfqpoint{2.090628in}{1.835329in}}%
\pgfpathlineto{\pgfqpoint{2.096699in}{1.825083in}}%
\pgfpathlineto{\pgfqpoint{2.102774in}{1.814796in}}%
\pgfpathclose%
\pgfusepath{stroke,fill}%
\end{pgfscope}%
\begin{pgfscope}%
\pgfpathrectangle{\pgfqpoint{0.887500in}{0.275000in}}{\pgfqpoint{4.225000in}{4.225000in}}%
\pgfusepath{clip}%
\pgfsetbuttcap%
\pgfsetroundjoin%
\definecolor{currentfill}{rgb}{0.225863,0.330805,0.547314}%
\pgfsetfillcolor{currentfill}%
\pgfsetfillopacity{0.700000}%
\pgfsetlinewidth{0.501875pt}%
\definecolor{currentstroke}{rgb}{1.000000,1.000000,1.000000}%
\pgfsetstrokecolor{currentstroke}%
\pgfsetstrokeopacity{0.500000}%
\pgfsetdash{}{0pt}%
\pgfpathmoveto{\pgfqpoint{2.616331in}{1.655663in}}%
\pgfpathlineto{\pgfqpoint{2.628044in}{1.659725in}}%
\pgfpathlineto{\pgfqpoint{2.639752in}{1.663781in}}%
\pgfpathlineto{\pgfqpoint{2.651454in}{1.667829in}}%
\pgfpathlineto{\pgfqpoint{2.663150in}{1.671869in}}%
\pgfpathlineto{\pgfqpoint{2.674841in}{1.675897in}}%
\pgfpathlineto{\pgfqpoint{2.668589in}{1.687077in}}%
\pgfpathlineto{\pgfqpoint{2.662342in}{1.698212in}}%
\pgfpathlineto{\pgfqpoint{2.656098in}{1.709301in}}%
\pgfpathlineto{\pgfqpoint{2.649859in}{1.720344in}}%
\pgfpathlineto{\pgfqpoint{2.643623in}{1.731339in}}%
\pgfpathlineto{\pgfqpoint{2.631941in}{1.727346in}}%
\pgfpathlineto{\pgfqpoint{2.620253in}{1.723340in}}%
\pgfpathlineto{\pgfqpoint{2.608560in}{1.719323in}}%
\pgfpathlineto{\pgfqpoint{2.596861in}{1.715296in}}%
\pgfpathlineto{\pgfqpoint{2.585157in}{1.711261in}}%
\pgfpathlineto{\pgfqpoint{2.591384in}{1.700232in}}%
\pgfpathlineto{\pgfqpoint{2.597614in}{1.689157in}}%
\pgfpathlineto{\pgfqpoint{2.603849in}{1.678037in}}%
\pgfpathlineto{\pgfqpoint{2.610088in}{1.666872in}}%
\pgfpathclose%
\pgfusepath{stroke,fill}%
\end{pgfscope}%
\begin{pgfscope}%
\pgfpathrectangle{\pgfqpoint{0.887500in}{0.275000in}}{\pgfqpoint{4.225000in}{4.225000in}}%
\pgfusepath{clip}%
\pgfsetbuttcap%
\pgfsetroundjoin%
\definecolor{currentfill}{rgb}{0.280267,0.073417,0.397163}%
\pgfsetfillcolor{currentfill}%
\pgfsetfillopacity{0.700000}%
\pgfsetlinewidth{0.501875pt}%
\definecolor{currentstroke}{rgb}{1.000000,1.000000,1.000000}%
\pgfsetstrokecolor{currentstroke}%
\pgfsetstrokeopacity{0.500000}%
\pgfsetdash{}{0pt}%
\pgfpathmoveto{\pgfqpoint{3.515069in}{1.211297in}}%
\pgfpathlineto{\pgfqpoint{3.526589in}{1.217688in}}%
\pgfpathlineto{\pgfqpoint{3.538096in}{1.223346in}}%
\pgfpathlineto{\pgfqpoint{3.549591in}{1.228431in}}%
\pgfpathlineto{\pgfqpoint{3.561076in}{1.233099in}}%
\pgfpathlineto{\pgfqpoint{3.572552in}{1.237510in}}%
\pgfpathlineto{\pgfqpoint{3.566110in}{1.253316in}}%
\pgfpathlineto{\pgfqpoint{3.559666in}{1.268738in}}%
\pgfpathlineto{\pgfqpoint{3.553219in}{1.283805in}}%
\pgfpathlineto{\pgfqpoint{3.546770in}{1.298547in}}%
\pgfpathlineto{\pgfqpoint{3.540320in}{1.312991in}}%
\pgfpathlineto{\pgfqpoint{3.528845in}{1.308400in}}%
\pgfpathlineto{\pgfqpoint{3.517362in}{1.303601in}}%
\pgfpathlineto{\pgfqpoint{3.505869in}{1.298478in}}%
\pgfpathlineto{\pgfqpoint{3.494367in}{1.292913in}}%
\pgfpathlineto{\pgfqpoint{3.482854in}{1.286791in}}%
\pgfpathlineto{\pgfqpoint{3.489295in}{1.271972in}}%
\pgfpathlineto{\pgfqpoint{3.495738in}{1.257017in}}%
\pgfpathlineto{\pgfqpoint{3.502181in}{1.241921in}}%
\pgfpathlineto{\pgfqpoint{3.508624in}{1.226683in}}%
\pgfpathclose%
\pgfusepath{stroke,fill}%
\end{pgfscope}%
\begin{pgfscope}%
\pgfpathrectangle{\pgfqpoint{0.887500in}{0.275000in}}{\pgfqpoint{4.225000in}{4.225000in}}%
\pgfusepath{clip}%
\pgfsetbuttcap%
\pgfsetroundjoin%
\definecolor{currentfill}{rgb}{0.282910,0.105393,0.426902}%
\pgfsetfillcolor{currentfill}%
\pgfsetfillopacity{0.700000}%
\pgfsetlinewidth{0.501875pt}%
\definecolor{currentstroke}{rgb}{1.000000,1.000000,1.000000}%
\pgfsetstrokecolor{currentstroke}%
\pgfsetstrokeopacity{0.500000}%
\pgfsetdash{}{0pt}%
\pgfpathmoveto{\pgfqpoint{3.425158in}{1.249240in}}%
\pgfpathlineto{\pgfqpoint{3.436705in}{1.256838in}}%
\pgfpathlineto{\pgfqpoint{3.448251in}{1.264714in}}%
\pgfpathlineto{\pgfqpoint{3.459794in}{1.272542in}}%
\pgfpathlineto{\pgfqpoint{3.471330in}{1.279997in}}%
\pgfpathlineto{\pgfqpoint{3.482854in}{1.286791in}}%
\pgfpathlineto{\pgfqpoint{3.476414in}{1.301478in}}%
\pgfpathlineto{\pgfqpoint{3.469975in}{1.316036in}}%
\pgfpathlineto{\pgfqpoint{3.463537in}{1.330469in}}%
\pgfpathlineto{\pgfqpoint{3.457100in}{1.344781in}}%
\pgfpathlineto{\pgfqpoint{3.450665in}{1.358975in}}%
\pgfpathlineto{\pgfqpoint{3.439161in}{1.353689in}}%
\pgfpathlineto{\pgfqpoint{3.427649in}{1.348028in}}%
\pgfpathlineto{\pgfqpoint{3.416130in}{1.342140in}}%
\pgfpathlineto{\pgfqpoint{3.404606in}{1.336196in}}%
\pgfpathlineto{\pgfqpoint{3.393078in}{1.330364in}}%
\pgfpathlineto{\pgfqpoint{3.399491in}{1.314243in}}%
\pgfpathlineto{\pgfqpoint{3.405905in}{1.298026in}}%
\pgfpathlineto{\pgfqpoint{3.412320in}{1.281757in}}%
\pgfpathlineto{\pgfqpoint{3.418738in}{1.265480in}}%
\pgfpathclose%
\pgfusepath{stroke,fill}%
\end{pgfscope}%
\begin{pgfscope}%
\pgfpathrectangle{\pgfqpoint{0.887500in}{0.275000in}}{\pgfqpoint{4.225000in}{4.225000in}}%
\pgfusepath{clip}%
\pgfsetbuttcap%
\pgfsetroundjoin%
\definecolor{currentfill}{rgb}{0.233603,0.313828,0.543914}%
\pgfsetfillcolor{currentfill}%
\pgfsetfillopacity{0.700000}%
\pgfsetlinewidth{0.501875pt}%
\definecolor{currentstroke}{rgb}{1.000000,1.000000,1.000000}%
\pgfsetstrokecolor{currentstroke}%
\pgfsetstrokeopacity{0.500000}%
\pgfsetdash{}{0pt}%
\pgfpathmoveto{\pgfqpoint{2.706156in}{1.619346in}}%
\pgfpathlineto{\pgfqpoint{2.717849in}{1.623394in}}%
\pgfpathlineto{\pgfqpoint{2.729538in}{1.627429in}}%
\pgfpathlineto{\pgfqpoint{2.741221in}{1.631450in}}%
\pgfpathlineto{\pgfqpoint{2.752898in}{1.635456in}}%
\pgfpathlineto{\pgfqpoint{2.764569in}{1.639448in}}%
\pgfpathlineto{\pgfqpoint{2.758290in}{1.650817in}}%
\pgfpathlineto{\pgfqpoint{2.752015in}{1.662137in}}%
\pgfpathlineto{\pgfqpoint{2.745744in}{1.673413in}}%
\pgfpathlineto{\pgfqpoint{2.739476in}{1.684645in}}%
\pgfpathlineto{\pgfqpoint{2.733212in}{1.695832in}}%
\pgfpathlineto{\pgfqpoint{2.721549in}{1.691874in}}%
\pgfpathlineto{\pgfqpoint{2.709880in}{1.687903in}}%
\pgfpathlineto{\pgfqpoint{2.698206in}{1.683916in}}%
\pgfpathlineto{\pgfqpoint{2.686526in}{1.679913in}}%
\pgfpathlineto{\pgfqpoint{2.674841in}{1.675897in}}%
\pgfpathlineto{\pgfqpoint{2.681096in}{1.664673in}}%
\pgfpathlineto{\pgfqpoint{2.687355in}{1.653406in}}%
\pgfpathlineto{\pgfqpoint{2.693618in}{1.642097in}}%
\pgfpathlineto{\pgfqpoint{2.699885in}{1.630745in}}%
\pgfpathclose%
\pgfusepath{stroke,fill}%
\end{pgfscope}%
\begin{pgfscope}%
\pgfpathrectangle{\pgfqpoint{0.887500in}{0.275000in}}{\pgfqpoint{4.225000in}{4.225000in}}%
\pgfusepath{clip}%
\pgfsetbuttcap%
\pgfsetroundjoin%
\definecolor{currentfill}{rgb}{0.197636,0.391528,0.554969}%
\pgfsetfillcolor{currentfill}%
\pgfsetfillopacity{0.700000}%
\pgfsetlinewidth{0.501875pt}%
\definecolor{currentstroke}{rgb}{1.000000,1.000000,1.000000}%
\pgfsetstrokecolor{currentstroke}%
\pgfsetstrokeopacity{0.500000}%
\pgfsetdash{}{0pt}%
\pgfpathmoveto{\pgfqpoint{2.198498in}{1.772221in}}%
\pgfpathlineto{\pgfqpoint{2.210316in}{1.776218in}}%
\pgfpathlineto{\pgfqpoint{2.222128in}{1.780215in}}%
\pgfpathlineto{\pgfqpoint{2.233934in}{1.784209in}}%
\pgfpathlineto{\pgfqpoint{2.245735in}{1.788201in}}%
\pgfpathlineto{\pgfqpoint{2.257530in}{1.792188in}}%
\pgfpathlineto{\pgfqpoint{2.251408in}{1.802643in}}%
\pgfpathlineto{\pgfqpoint{2.245291in}{1.813056in}}%
\pgfpathlineto{\pgfqpoint{2.239177in}{1.823428in}}%
\pgfpathlineto{\pgfqpoint{2.233069in}{1.833758in}}%
\pgfpathlineto{\pgfqpoint{2.226964in}{1.844046in}}%
\pgfpathlineto{\pgfqpoint{2.215179in}{1.840100in}}%
\pgfpathlineto{\pgfqpoint{2.203388in}{1.836148in}}%
\pgfpathlineto{\pgfqpoint{2.191591in}{1.832192in}}%
\pgfpathlineto{\pgfqpoint{2.179789in}{1.828233in}}%
\pgfpathlineto{\pgfqpoint{2.167981in}{1.824273in}}%
\pgfpathlineto{\pgfqpoint{2.174076in}{1.813944in}}%
\pgfpathlineto{\pgfqpoint{2.180175in}{1.803574in}}%
\pgfpathlineto{\pgfqpoint{2.186278in}{1.793164in}}%
\pgfpathlineto{\pgfqpoint{2.192386in}{1.782712in}}%
\pgfpathclose%
\pgfusepath{stroke,fill}%
\end{pgfscope}%
\begin{pgfscope}%
\pgfpathrectangle{\pgfqpoint{0.887500in}{0.275000in}}{\pgfqpoint{4.225000in}{4.225000in}}%
\pgfusepath{clip}%
\pgfsetbuttcap%
\pgfsetroundjoin%
\definecolor{currentfill}{rgb}{0.282623,0.140926,0.457517}%
\pgfsetfillcolor{currentfill}%
\pgfsetfillopacity{0.700000}%
\pgfsetlinewidth{0.501875pt}%
\definecolor{currentstroke}{rgb}{1.000000,1.000000,1.000000}%
\pgfsetstrokecolor{currentstroke}%
\pgfsetstrokeopacity{0.500000}%
\pgfsetdash{}{0pt}%
\pgfpathmoveto{\pgfqpoint{3.335425in}{1.308426in}}%
\pgfpathlineto{\pgfqpoint{3.346957in}{1.311498in}}%
\pgfpathlineto{\pgfqpoint{3.358487in}{1.315254in}}%
\pgfpathlineto{\pgfqpoint{3.370018in}{1.319724in}}%
\pgfpathlineto{\pgfqpoint{3.381549in}{1.324817in}}%
\pgfpathlineto{\pgfqpoint{3.393078in}{1.330364in}}%
\pgfpathlineto{\pgfqpoint{3.386666in}{1.346345in}}%
\pgfpathlineto{\pgfqpoint{3.380255in}{1.362141in}}%
\pgfpathlineto{\pgfqpoint{3.373844in}{1.377708in}}%
\pgfpathlineto{\pgfqpoint{3.367432in}{1.393002in}}%
\pgfpathlineto{\pgfqpoint{3.361021in}{1.407978in}}%
\pgfpathlineto{\pgfqpoint{3.349505in}{1.403782in}}%
\pgfpathlineto{\pgfqpoint{3.337983in}{1.399632in}}%
\pgfpathlineto{\pgfqpoint{3.326457in}{1.395570in}}%
\pgfpathlineto{\pgfqpoint{3.314925in}{1.391616in}}%
\pgfpathlineto{\pgfqpoint{3.303389in}{1.387762in}}%
\pgfpathlineto{\pgfqpoint{3.309796in}{1.372500in}}%
\pgfpathlineto{\pgfqpoint{3.316203in}{1.356877in}}%
\pgfpathlineto{\pgfqpoint{3.322610in}{1.340951in}}%
\pgfpathlineto{\pgfqpoint{3.329017in}{1.324781in}}%
\pgfpathclose%
\pgfusepath{stroke,fill}%
\end{pgfscope}%
\begin{pgfscope}%
\pgfpathrectangle{\pgfqpoint{0.887500in}{0.275000in}}{\pgfqpoint{4.225000in}{4.225000in}}%
\pgfusepath{clip}%
\pgfsetbuttcap%
\pgfsetroundjoin%
\definecolor{currentfill}{rgb}{0.243113,0.292092,0.538516}%
\pgfsetfillcolor{currentfill}%
\pgfsetfillopacity{0.700000}%
\pgfsetlinewidth{0.501875pt}%
\definecolor{currentstroke}{rgb}{1.000000,1.000000,1.000000}%
\pgfsetstrokecolor{currentstroke}%
\pgfsetstrokeopacity{0.500000}%
\pgfsetdash{}{0pt}%
\pgfpathmoveto{\pgfqpoint{2.796021in}{1.581756in}}%
\pgfpathlineto{\pgfqpoint{2.807695in}{1.585764in}}%
\pgfpathlineto{\pgfqpoint{2.819364in}{1.589766in}}%
\pgfpathlineto{\pgfqpoint{2.831027in}{1.593765in}}%
\pgfpathlineto{\pgfqpoint{2.842684in}{1.597766in}}%
\pgfpathlineto{\pgfqpoint{2.854335in}{1.601770in}}%
\pgfpathlineto{\pgfqpoint{2.848030in}{1.613409in}}%
\pgfpathlineto{\pgfqpoint{2.841727in}{1.624982in}}%
\pgfpathlineto{\pgfqpoint{2.835429in}{1.636491in}}%
\pgfpathlineto{\pgfqpoint{2.829134in}{1.647940in}}%
\pgfpathlineto{\pgfqpoint{2.822843in}{1.659331in}}%
\pgfpathlineto{\pgfqpoint{2.811200in}{1.655354in}}%
\pgfpathlineto{\pgfqpoint{2.799550in}{1.651381in}}%
\pgfpathlineto{\pgfqpoint{2.787896in}{1.647407in}}%
\pgfpathlineto{\pgfqpoint{2.776235in}{1.643431in}}%
\pgfpathlineto{\pgfqpoint{2.764569in}{1.639448in}}%
\pgfpathlineto{\pgfqpoint{2.770852in}{1.628028in}}%
\pgfpathlineto{\pgfqpoint{2.777139in}{1.616552in}}%
\pgfpathlineto{\pgfqpoint{2.783429in}{1.605017in}}%
\pgfpathlineto{\pgfqpoint{2.789723in}{1.593420in}}%
\pgfpathclose%
\pgfusepath{stroke,fill}%
\end{pgfscope}%
\begin{pgfscope}%
\pgfpathrectangle{\pgfqpoint{0.887500in}{0.275000in}}{\pgfqpoint{4.225000in}{4.225000in}}%
\pgfusepath{clip}%
\pgfsetbuttcap%
\pgfsetroundjoin%
\definecolor{currentfill}{rgb}{0.250425,0.274290,0.533103}%
\pgfsetfillcolor{currentfill}%
\pgfsetfillopacity{0.700000}%
\pgfsetlinewidth{0.501875pt}%
\definecolor{currentstroke}{rgb}{1.000000,1.000000,1.000000}%
\pgfsetstrokecolor{currentstroke}%
\pgfsetstrokeopacity{0.500000}%
\pgfsetdash{}{0pt}%
\pgfpathmoveto{\pgfqpoint{2.885921in}{1.542476in}}%
\pgfpathlineto{\pgfqpoint{2.897574in}{1.546515in}}%
\pgfpathlineto{\pgfqpoint{2.909222in}{1.550561in}}%
\pgfpathlineto{\pgfqpoint{2.920864in}{1.554616in}}%
\pgfpathlineto{\pgfqpoint{2.932500in}{1.558680in}}%
\pgfpathlineto{\pgfqpoint{2.944131in}{1.562757in}}%
\pgfpathlineto{\pgfqpoint{2.937799in}{1.574714in}}%
\pgfpathlineto{\pgfqpoint{2.931471in}{1.586602in}}%
\pgfpathlineto{\pgfqpoint{2.925147in}{1.598418in}}%
\pgfpathlineto{\pgfqpoint{2.918826in}{1.610163in}}%
\pgfpathlineto{\pgfqpoint{2.912508in}{1.621840in}}%
\pgfpathlineto{\pgfqpoint{2.900885in}{1.617826in}}%
\pgfpathlineto{\pgfqpoint{2.889256in}{1.613810in}}%
\pgfpathlineto{\pgfqpoint{2.877621in}{1.609794in}}%
\pgfpathlineto{\pgfqpoint{2.865981in}{1.605780in}}%
\pgfpathlineto{\pgfqpoint{2.854335in}{1.601770in}}%
\pgfpathlineto{\pgfqpoint{2.860645in}{1.590061in}}%
\pgfpathlineto{\pgfqpoint{2.866959in}{1.578279in}}%
\pgfpathlineto{\pgfqpoint{2.873276in}{1.566420in}}%
\pgfpathlineto{\pgfqpoint{2.879596in}{1.554484in}}%
\pgfpathclose%
\pgfusepath{stroke,fill}%
\end{pgfscope}%
\begin{pgfscope}%
\pgfpathrectangle{\pgfqpoint{0.887500in}{0.275000in}}{\pgfqpoint{4.225000in}{4.225000in}}%
\pgfusepath{clip}%
\pgfsetbuttcap%
\pgfsetroundjoin%
\definecolor{currentfill}{rgb}{0.204903,0.375746,0.553533}%
\pgfsetfillcolor{currentfill}%
\pgfsetfillopacity{0.700000}%
\pgfsetlinewidth{0.501875pt}%
\definecolor{currentstroke}{rgb}{1.000000,1.000000,1.000000}%
\pgfsetstrokecolor{currentstroke}%
\pgfsetstrokeopacity{0.500000}%
\pgfsetdash{}{0pt}%
\pgfpathmoveto{\pgfqpoint{2.288205in}{1.739290in}}%
\pgfpathlineto{\pgfqpoint{2.300005in}{1.743308in}}%
\pgfpathlineto{\pgfqpoint{2.311798in}{1.747322in}}%
\pgfpathlineto{\pgfqpoint{2.323586in}{1.751331in}}%
\pgfpathlineto{\pgfqpoint{2.335368in}{1.755334in}}%
\pgfpathlineto{\pgfqpoint{2.347144in}{1.759332in}}%
\pgfpathlineto{\pgfqpoint{2.340991in}{1.769960in}}%
\pgfpathlineto{\pgfqpoint{2.334842in}{1.780545in}}%
\pgfpathlineto{\pgfqpoint{2.328698in}{1.791087in}}%
\pgfpathlineto{\pgfqpoint{2.322557in}{1.801586in}}%
\pgfpathlineto{\pgfqpoint{2.316421in}{1.812042in}}%
\pgfpathlineto{\pgfqpoint{2.304654in}{1.808084in}}%
\pgfpathlineto{\pgfqpoint{2.292882in}{1.804119in}}%
\pgfpathlineto{\pgfqpoint{2.281104in}{1.800148in}}%
\pgfpathlineto{\pgfqpoint{2.269320in}{1.796171in}}%
\pgfpathlineto{\pgfqpoint{2.257530in}{1.792188in}}%
\pgfpathlineto{\pgfqpoint{2.263657in}{1.781692in}}%
\pgfpathlineto{\pgfqpoint{2.269787in}{1.771153in}}%
\pgfpathlineto{\pgfqpoint{2.275922in}{1.760574in}}%
\pgfpathlineto{\pgfqpoint{2.282062in}{1.749953in}}%
\pgfpathclose%
\pgfusepath{stroke,fill}%
\end{pgfscope}%
\begin{pgfscope}%
\pgfpathrectangle{\pgfqpoint{0.887500in}{0.275000in}}{\pgfqpoint{4.225000in}{4.225000in}}%
\pgfusepath{clip}%
\pgfsetbuttcap%
\pgfsetroundjoin%
\definecolor{currentfill}{rgb}{0.278826,0.175490,0.483397}%
\pgfsetfillcolor{currentfill}%
\pgfsetfillopacity{0.700000}%
\pgfsetlinewidth{0.501875pt}%
\definecolor{currentstroke}{rgb}{1.000000,1.000000,1.000000}%
\pgfsetstrokecolor{currentstroke}%
\pgfsetstrokeopacity{0.500000}%
\pgfsetdash{}{0pt}%
\pgfpathmoveto{\pgfqpoint{3.245628in}{1.369706in}}%
\pgfpathlineto{\pgfqpoint{3.257191in}{1.373186in}}%
\pgfpathlineto{\pgfqpoint{3.268749in}{1.376722in}}%
\pgfpathlineto{\pgfqpoint{3.280301in}{1.380325in}}%
\pgfpathlineto{\pgfqpoint{3.291848in}{1.384002in}}%
\pgfpathlineto{\pgfqpoint{3.303389in}{1.387762in}}%
\pgfpathlineto{\pgfqpoint{3.296982in}{1.402605in}}%
\pgfpathlineto{\pgfqpoint{3.290574in}{1.416989in}}%
\pgfpathlineto{\pgfqpoint{3.284167in}{1.430947in}}%
\pgfpathlineto{\pgfqpoint{3.277760in}{1.444525in}}%
\pgfpathlineto{\pgfqpoint{3.271354in}{1.457768in}}%
\pgfpathlineto{\pgfqpoint{3.259814in}{1.453566in}}%
\pgfpathlineto{\pgfqpoint{3.248268in}{1.449277in}}%
\pgfpathlineto{\pgfqpoint{3.236716in}{1.444901in}}%
\pgfpathlineto{\pgfqpoint{3.225159in}{1.440440in}}%
\pgfpathlineto{\pgfqpoint{3.213596in}{1.435897in}}%
\pgfpathlineto{\pgfqpoint{3.219997in}{1.422799in}}%
\pgfpathlineto{\pgfqpoint{3.226401in}{1.409629in}}%
\pgfpathlineto{\pgfqpoint{3.232807in}{1.396388in}}%
\pgfpathlineto{\pgfqpoint{3.239217in}{1.383080in}}%
\pgfpathclose%
\pgfusepath{stroke,fill}%
\end{pgfscope}%
\begin{pgfscope}%
\pgfpathrectangle{\pgfqpoint{0.887500in}{0.275000in}}{\pgfqpoint{4.225000in}{4.225000in}}%
\pgfusepath{clip}%
\pgfsetbuttcap%
\pgfsetroundjoin%
\definecolor{currentfill}{rgb}{0.258965,0.251537,0.524736}%
\pgfsetfillcolor{currentfill}%
\pgfsetfillopacity{0.700000}%
\pgfsetlinewidth{0.501875pt}%
\definecolor{currentstroke}{rgb}{1.000000,1.000000,1.000000}%
\pgfsetstrokecolor{currentstroke}%
\pgfsetstrokeopacity{0.500000}%
\pgfsetdash{}{0pt}%
\pgfpathmoveto{\pgfqpoint{2.975841in}{1.502079in}}%
\pgfpathlineto{\pgfqpoint{2.987473in}{1.506230in}}%
\pgfpathlineto{\pgfqpoint{2.999100in}{1.510414in}}%
\pgfpathlineto{\pgfqpoint{3.010721in}{1.514631in}}%
\pgfpathlineto{\pgfqpoint{3.022337in}{1.518861in}}%
\pgfpathlineto{\pgfqpoint{3.033947in}{1.523088in}}%
\pgfpathlineto{\pgfqpoint{3.027591in}{1.535329in}}%
\pgfpathlineto{\pgfqpoint{3.021238in}{1.547459in}}%
\pgfpathlineto{\pgfqpoint{3.014889in}{1.559488in}}%
\pgfpathlineto{\pgfqpoint{3.008543in}{1.571430in}}%
\pgfpathlineto{\pgfqpoint{3.002201in}{1.583296in}}%
\pgfpathlineto{\pgfqpoint{2.990598in}{1.579183in}}%
\pgfpathlineto{\pgfqpoint{2.978990in}{1.575064in}}%
\pgfpathlineto{\pgfqpoint{2.967376in}{1.570949in}}%
\pgfpathlineto{\pgfqpoint{2.955756in}{1.566845in}}%
\pgfpathlineto{\pgfqpoint{2.944131in}{1.562757in}}%
\pgfpathlineto{\pgfqpoint{2.950466in}{1.550734in}}%
\pgfpathlineto{\pgfqpoint{2.956805in}{1.538652in}}%
\pgfpathlineto{\pgfqpoint{2.963147in}{1.526512in}}%
\pgfpathlineto{\pgfqpoint{2.969492in}{1.514320in}}%
\pgfpathclose%
\pgfusepath{stroke,fill}%
\end{pgfscope}%
\begin{pgfscope}%
\pgfpathrectangle{\pgfqpoint{0.887500in}{0.275000in}}{\pgfqpoint{4.225000in}{4.225000in}}%
\pgfusepath{clip}%
\pgfsetbuttcap%
\pgfsetroundjoin%
\definecolor{currentfill}{rgb}{0.274128,0.199721,0.498911}%
\pgfsetfillcolor{currentfill}%
\pgfsetfillopacity{0.700000}%
\pgfsetlinewidth{0.501875pt}%
\definecolor{currentstroke}{rgb}{1.000000,1.000000,1.000000}%
\pgfsetstrokecolor{currentstroke}%
\pgfsetstrokeopacity{0.500000}%
\pgfsetdash{}{0pt}%
\pgfpathmoveto{\pgfqpoint{3.155702in}{1.413272in}}%
\pgfpathlineto{\pgfqpoint{3.167291in}{1.417621in}}%
\pgfpathlineto{\pgfqpoint{3.178875in}{1.422107in}}%
\pgfpathlineto{\pgfqpoint{3.190454in}{1.426681in}}%
\pgfpathlineto{\pgfqpoint{3.202028in}{1.431294in}}%
\pgfpathlineto{\pgfqpoint{3.213596in}{1.435897in}}%
\pgfpathlineto{\pgfqpoint{3.207198in}{1.448920in}}%
\pgfpathlineto{\pgfqpoint{3.200802in}{1.461867in}}%
\pgfpathlineto{\pgfqpoint{3.194409in}{1.474736in}}%
\pgfpathlineto{\pgfqpoint{3.188019in}{1.487525in}}%
\pgfpathlineto{\pgfqpoint{3.181632in}{1.500231in}}%
\pgfpathlineto{\pgfqpoint{3.170071in}{1.496104in}}%
\pgfpathlineto{\pgfqpoint{3.158505in}{1.491973in}}%
\pgfpathlineto{\pgfqpoint{3.146933in}{1.487847in}}%
\pgfpathlineto{\pgfqpoint{3.135355in}{1.483737in}}%
\pgfpathlineto{\pgfqpoint{3.123773in}{1.479653in}}%
\pgfpathlineto{\pgfqpoint{3.130153in}{1.466359in}}%
\pgfpathlineto{\pgfqpoint{3.136535in}{1.453005in}}%
\pgfpathlineto{\pgfqpoint{3.142921in}{1.439660in}}%
\pgfpathlineto{\pgfqpoint{3.149310in}{1.426393in}}%
\pgfpathclose%
\pgfusepath{stroke,fill}%
\end{pgfscope}%
\begin{pgfscope}%
\pgfpathrectangle{\pgfqpoint{0.887500in}{0.275000in}}{\pgfqpoint{4.225000in}{4.225000in}}%
\pgfusepath{clip}%
\pgfsetbuttcap%
\pgfsetroundjoin%
\definecolor{currentfill}{rgb}{0.266580,0.228262,0.514349}%
\pgfsetfillcolor{currentfill}%
\pgfsetfillopacity{0.700000}%
\pgfsetlinewidth{0.501875pt}%
\definecolor{currentstroke}{rgb}{1.000000,1.000000,1.000000}%
\pgfsetstrokecolor{currentstroke}%
\pgfsetstrokeopacity{0.500000}%
\pgfsetdash{}{0pt}%
\pgfpathmoveto{\pgfqpoint{3.065774in}{1.459836in}}%
\pgfpathlineto{\pgfqpoint{3.077385in}{1.463734in}}%
\pgfpathlineto{\pgfqpoint{3.088991in}{1.467655in}}%
\pgfpathlineto{\pgfqpoint{3.100590in}{1.471609in}}%
\pgfpathlineto{\pgfqpoint{3.112184in}{1.475607in}}%
\pgfpathlineto{\pgfqpoint{3.123773in}{1.479653in}}%
\pgfpathlineto{\pgfqpoint{3.117395in}{1.492820in}}%
\pgfpathlineto{\pgfqpoint{3.111021in}{1.505794in}}%
\pgfpathlineto{\pgfqpoint{3.104649in}{1.518554in}}%
\pgfpathlineto{\pgfqpoint{3.098280in}{1.531119in}}%
\pgfpathlineto{\pgfqpoint{3.091914in}{1.543507in}}%
\pgfpathlineto{\pgfqpoint{3.080332in}{1.539571in}}%
\pgfpathlineto{\pgfqpoint{3.068744in}{1.535551in}}%
\pgfpathlineto{\pgfqpoint{3.057150in}{1.531452in}}%
\pgfpathlineto{\pgfqpoint{3.045551in}{1.527291in}}%
\pgfpathlineto{\pgfqpoint{3.033947in}{1.523088in}}%
\pgfpathlineto{\pgfqpoint{3.040306in}{1.510724in}}%
\pgfpathlineto{\pgfqpoint{3.046668in}{1.498226in}}%
\pgfpathlineto{\pgfqpoint{3.053034in}{1.485582in}}%
\pgfpathlineto{\pgfqpoint{3.059403in}{1.472781in}}%
\pgfpathclose%
\pgfusepath{stroke,fill}%
\end{pgfscope}%
\begin{pgfscope}%
\pgfpathrectangle{\pgfqpoint{0.887500in}{0.275000in}}{\pgfqpoint{4.225000in}{4.225000in}}%
\pgfusepath{clip}%
\pgfsetbuttcap%
\pgfsetroundjoin%
\definecolor{currentfill}{rgb}{0.212395,0.359683,0.551710}%
\pgfsetfillcolor{currentfill}%
\pgfsetfillopacity{0.700000}%
\pgfsetlinewidth{0.501875pt}%
\definecolor{currentstroke}{rgb}{1.000000,1.000000,1.000000}%
\pgfsetstrokecolor{currentstroke}%
\pgfsetstrokeopacity{0.500000}%
\pgfsetdash{}{0pt}%
\pgfpathmoveto{\pgfqpoint{2.377975in}{1.705540in}}%
\pgfpathlineto{\pgfqpoint{2.389755in}{1.709564in}}%
\pgfpathlineto{\pgfqpoint{2.401530in}{1.713582in}}%
\pgfpathlineto{\pgfqpoint{2.413299in}{1.717594in}}%
\pgfpathlineto{\pgfqpoint{2.425062in}{1.721600in}}%
\pgfpathlineto{\pgfqpoint{2.436820in}{1.725603in}}%
\pgfpathlineto{\pgfqpoint{2.430636in}{1.736417in}}%
\pgfpathlineto{\pgfqpoint{2.424457in}{1.747185in}}%
\pgfpathlineto{\pgfqpoint{2.418281in}{1.757908in}}%
\pgfpathlineto{\pgfqpoint{2.412110in}{1.768587in}}%
\pgfpathlineto{\pgfqpoint{2.405943in}{1.779220in}}%
\pgfpathlineto{\pgfqpoint{2.394194in}{1.775254in}}%
\pgfpathlineto{\pgfqpoint{2.382440in}{1.771284in}}%
\pgfpathlineto{\pgfqpoint{2.370681in}{1.767307in}}%
\pgfpathlineto{\pgfqpoint{2.358915in}{1.763323in}}%
\pgfpathlineto{\pgfqpoint{2.347144in}{1.759332in}}%
\pgfpathlineto{\pgfqpoint{2.353302in}{1.748660in}}%
\pgfpathlineto{\pgfqpoint{2.359464in}{1.737945in}}%
\pgfpathlineto{\pgfqpoint{2.365630in}{1.727187in}}%
\pgfpathlineto{\pgfqpoint{2.371800in}{1.716385in}}%
\pgfpathclose%
\pgfusepath{stroke,fill}%
\end{pgfscope}%
\begin{pgfscope}%
\pgfpathrectangle{\pgfqpoint{0.887500in}{0.275000in}}{\pgfqpoint{4.225000in}{4.225000in}}%
\pgfusepath{clip}%
\pgfsetbuttcap%
\pgfsetroundjoin%
\definecolor{currentfill}{rgb}{0.187231,0.414746,0.556547}%
\pgfsetfillcolor{currentfill}%
\pgfsetfillopacity{0.700000}%
\pgfsetlinewidth{0.501875pt}%
\definecolor{currentstroke}{rgb}{1.000000,1.000000,1.000000}%
\pgfsetstrokecolor{currentstroke}%
\pgfsetstrokeopacity{0.500000}%
\pgfsetdash{}{0pt}%
\pgfpathmoveto{\pgfqpoint{1.959918in}{1.816254in}}%
\pgfpathlineto{\pgfqpoint{1.971801in}{1.820231in}}%
\pgfpathlineto{\pgfqpoint{1.983679in}{1.824198in}}%
\pgfpathlineto{\pgfqpoint{1.995551in}{1.828157in}}%
\pgfpathlineto{\pgfqpoint{2.007418in}{1.832107in}}%
\pgfpathlineto{\pgfqpoint{2.019279in}{1.836052in}}%
\pgfpathlineto{\pgfqpoint{2.013232in}{1.846218in}}%
\pgfpathlineto{\pgfqpoint{2.007189in}{1.856344in}}%
\pgfpathlineto{\pgfqpoint{2.001151in}{1.866432in}}%
\pgfpathlineto{\pgfqpoint{1.995117in}{1.876481in}}%
\pgfpathlineto{\pgfqpoint{1.989088in}{1.886492in}}%
\pgfpathlineto{\pgfqpoint{1.977237in}{1.882591in}}%
\pgfpathlineto{\pgfqpoint{1.965381in}{1.878682in}}%
\pgfpathlineto{\pgfqpoint{1.953519in}{1.874766in}}%
\pgfpathlineto{\pgfqpoint{1.941651in}{1.870839in}}%
\pgfpathlineto{\pgfqpoint{1.929778in}{1.866903in}}%
\pgfpathlineto{\pgfqpoint{1.935797in}{1.856849in}}%
\pgfpathlineto{\pgfqpoint{1.941820in}{1.846758in}}%
\pgfpathlineto{\pgfqpoint{1.947848in}{1.836629in}}%
\pgfpathlineto{\pgfqpoint{1.953880in}{1.826461in}}%
\pgfpathclose%
\pgfusepath{stroke,fill}%
\end{pgfscope}%
\begin{pgfscope}%
\pgfpathrectangle{\pgfqpoint{0.887500in}{0.275000in}}{\pgfqpoint{4.225000in}{4.225000in}}%
\pgfusepath{clip}%
\pgfsetbuttcap%
\pgfsetroundjoin%
\definecolor{currentfill}{rgb}{0.272594,0.025563,0.353093}%
\pgfsetfillcolor{currentfill}%
\pgfsetfillopacity{0.700000}%
\pgfsetlinewidth{0.501875pt}%
\definecolor{currentstroke}{rgb}{1.000000,1.000000,1.000000}%
\pgfsetstrokecolor{currentstroke}%
\pgfsetstrokeopacity{0.500000}%
\pgfsetdash{}{0pt}%
\pgfpathmoveto{\pgfqpoint{3.547295in}{1.132033in}}%
\pgfpathlineto{\pgfqpoint{3.558798in}{1.136816in}}%
\pgfpathlineto{\pgfqpoint{3.570286in}{1.140955in}}%
\pgfpathlineto{\pgfqpoint{3.581764in}{1.144674in}}%
\pgfpathlineto{\pgfqpoint{3.593232in}{1.148196in}}%
\pgfpathlineto{\pgfqpoint{3.604696in}{1.151745in}}%
\pgfpathlineto{\pgfqpoint{3.598278in}{1.169891in}}%
\pgfpathlineto{\pgfqpoint{3.591853in}{1.187512in}}%
\pgfpathlineto{\pgfqpoint{3.585424in}{1.204637in}}%
\pgfpathlineto{\pgfqpoint{3.578990in}{1.221294in}}%
\pgfpathlineto{\pgfqpoint{3.572552in}{1.237510in}}%
\pgfpathlineto{\pgfqpoint{3.561076in}{1.233099in}}%
\pgfpathlineto{\pgfqpoint{3.549591in}{1.228431in}}%
\pgfpathlineto{\pgfqpoint{3.538096in}{1.223346in}}%
\pgfpathlineto{\pgfqpoint{3.526589in}{1.217688in}}%
\pgfpathlineto{\pgfqpoint{3.515069in}{1.211297in}}%
\pgfpathlineto{\pgfqpoint{3.521514in}{1.195760in}}%
\pgfpathlineto{\pgfqpoint{3.527959in}{1.180069in}}%
\pgfpathlineto{\pgfqpoint{3.534404in}{1.164220in}}%
\pgfpathlineto{\pgfqpoint{3.540850in}{1.148209in}}%
\pgfpathclose%
\pgfusepath{stroke,fill}%
\end{pgfscope}%
\begin{pgfscope}%
\pgfpathrectangle{\pgfqpoint{0.887500in}{0.275000in}}{\pgfqpoint{4.225000in}{4.225000in}}%
\pgfusepath{clip}%
\pgfsetbuttcap%
\pgfsetroundjoin%
\definecolor{currentfill}{rgb}{0.278791,0.062145,0.386592}%
\pgfsetfillcolor{currentfill}%
\pgfsetfillopacity{0.700000}%
\pgfsetlinewidth{0.501875pt}%
\definecolor{currentstroke}{rgb}{1.000000,1.000000,1.000000}%
\pgfsetstrokecolor{currentstroke}%
\pgfsetstrokeopacity{0.500000}%
\pgfsetdash{}{0pt}%
\pgfpathmoveto{\pgfqpoint{3.457308in}{1.170117in}}%
\pgfpathlineto{\pgfqpoint{3.468867in}{1.178480in}}%
\pgfpathlineto{\pgfqpoint{3.480428in}{1.187204in}}%
\pgfpathlineto{\pgfqpoint{3.491985in}{1.195859in}}%
\pgfpathlineto{\pgfqpoint{3.503534in}{1.204017in}}%
\pgfpathlineto{\pgfqpoint{3.515069in}{1.211297in}}%
\pgfpathlineto{\pgfqpoint{3.508624in}{1.226683in}}%
\pgfpathlineto{\pgfqpoint{3.502181in}{1.241921in}}%
\pgfpathlineto{\pgfqpoint{3.495738in}{1.257017in}}%
\pgfpathlineto{\pgfqpoint{3.489295in}{1.271972in}}%
\pgfpathlineto{\pgfqpoint{3.482854in}{1.286791in}}%
\pgfpathlineto{\pgfqpoint{3.471330in}{1.279997in}}%
\pgfpathlineto{\pgfqpoint{3.459794in}{1.272542in}}%
\pgfpathlineto{\pgfqpoint{3.448251in}{1.264714in}}%
\pgfpathlineto{\pgfqpoint{3.436705in}{1.256838in}}%
\pgfpathlineto{\pgfqpoint{3.425158in}{1.249240in}}%
\pgfpathlineto{\pgfqpoint{3.431580in}{1.233079in}}%
\pgfpathlineto{\pgfqpoint{3.438006in}{1.217042in}}%
\pgfpathlineto{\pgfqpoint{3.444435in}{1.201174in}}%
\pgfpathlineto{\pgfqpoint{3.450869in}{1.185517in}}%
\pgfpathclose%
\pgfusepath{stroke,fill}%
\end{pgfscope}%
\begin{pgfscope}%
\pgfpathrectangle{\pgfqpoint{0.887500in}{0.275000in}}{\pgfqpoint{4.225000in}{4.225000in}}%
\pgfusepath{clip}%
\pgfsetbuttcap%
\pgfsetroundjoin%
\definecolor{currentfill}{rgb}{0.220057,0.343307,0.549413}%
\pgfsetfillcolor{currentfill}%
\pgfsetfillopacity{0.700000}%
\pgfsetlinewidth{0.501875pt}%
\definecolor{currentstroke}{rgb}{1.000000,1.000000,1.000000}%
\pgfsetstrokecolor{currentstroke}%
\pgfsetstrokeopacity{0.500000}%
\pgfsetdash{}{0pt}%
\pgfpathmoveto{\pgfqpoint{2.467802in}{1.670862in}}%
\pgfpathlineto{\pgfqpoint{2.479563in}{1.674893in}}%
\pgfpathlineto{\pgfqpoint{2.491319in}{1.678925in}}%
\pgfpathlineto{\pgfqpoint{2.503069in}{1.682958in}}%
\pgfpathlineto{\pgfqpoint{2.514813in}{1.686995in}}%
\pgfpathlineto{\pgfqpoint{2.526551in}{1.691036in}}%
\pgfpathlineto{\pgfqpoint{2.520337in}{1.702046in}}%
\pgfpathlineto{\pgfqpoint{2.514127in}{1.713011in}}%
\pgfpathlineto{\pgfqpoint{2.507922in}{1.723929in}}%
\pgfpathlineto{\pgfqpoint{2.501720in}{1.734800in}}%
\pgfpathlineto{\pgfqpoint{2.495523in}{1.745625in}}%
\pgfpathlineto{\pgfqpoint{2.483794in}{1.741614in}}%
\pgfpathlineto{\pgfqpoint{2.472059in}{1.737608in}}%
\pgfpathlineto{\pgfqpoint{2.460319in}{1.733605in}}%
\pgfpathlineto{\pgfqpoint{2.448572in}{1.729604in}}%
\pgfpathlineto{\pgfqpoint{2.436820in}{1.725603in}}%
\pgfpathlineto{\pgfqpoint{2.443008in}{1.714745in}}%
\pgfpathlineto{\pgfqpoint{2.449200in}{1.703841in}}%
\pgfpathlineto{\pgfqpoint{2.455397in}{1.692893in}}%
\pgfpathlineto{\pgfqpoint{2.461597in}{1.681899in}}%
\pgfpathclose%
\pgfusepath{stroke,fill}%
\end{pgfscope}%
\begin{pgfscope}%
\pgfpathrectangle{\pgfqpoint{0.887500in}{0.275000in}}{\pgfqpoint{4.225000in}{4.225000in}}%
\pgfusepath{clip}%
\pgfsetbuttcap%
\pgfsetroundjoin%
\definecolor{currentfill}{rgb}{0.282327,0.094955,0.417331}%
\pgfsetfillcolor{currentfill}%
\pgfsetfillopacity{0.700000}%
\pgfsetlinewidth{0.501875pt}%
\definecolor{currentstroke}{rgb}{1.000000,1.000000,1.000000}%
\pgfsetstrokecolor{currentstroke}%
\pgfsetstrokeopacity{0.500000}%
\pgfsetdash{}{0pt}%
\pgfpathmoveto{\pgfqpoint{3.367488in}{1.225916in}}%
\pgfpathlineto{\pgfqpoint{3.379010in}{1.227959in}}%
\pgfpathlineto{\pgfqpoint{3.390538in}{1.231358in}}%
\pgfpathlineto{\pgfqpoint{3.402072in}{1.236176in}}%
\pgfpathlineto{\pgfqpoint{3.413612in}{1.242244in}}%
\pgfpathlineto{\pgfqpoint{3.425158in}{1.249240in}}%
\pgfpathlineto{\pgfqpoint{3.418738in}{1.265480in}}%
\pgfpathlineto{\pgfqpoint{3.412320in}{1.281757in}}%
\pgfpathlineto{\pgfqpoint{3.405905in}{1.298026in}}%
\pgfpathlineto{\pgfqpoint{3.399491in}{1.314243in}}%
\pgfpathlineto{\pgfqpoint{3.393078in}{1.330364in}}%
\pgfpathlineto{\pgfqpoint{3.381549in}{1.324817in}}%
\pgfpathlineto{\pgfqpoint{3.370018in}{1.319724in}}%
\pgfpathlineto{\pgfqpoint{3.358487in}{1.315254in}}%
\pgfpathlineto{\pgfqpoint{3.346957in}{1.311498in}}%
\pgfpathlineto{\pgfqpoint{3.335425in}{1.308426in}}%
\pgfpathlineto{\pgfqpoint{3.341834in}{1.291945in}}%
\pgfpathlineto{\pgfqpoint{3.348244in}{1.275394in}}%
\pgfpathlineto{\pgfqpoint{3.354656in}{1.258834in}}%
\pgfpathlineto{\pgfqpoint{3.361071in}{1.242322in}}%
\pgfpathclose%
\pgfusepath{stroke,fill}%
\end{pgfscope}%
\begin{pgfscope}%
\pgfpathrectangle{\pgfqpoint{0.887500in}{0.275000in}}{\pgfqpoint{4.225000in}{4.225000in}}%
\pgfusepath{clip}%
\pgfsetbuttcap%
\pgfsetroundjoin%
\definecolor{currentfill}{rgb}{0.192357,0.403199,0.555836}%
\pgfsetfillcolor{currentfill}%
\pgfsetfillopacity{0.700000}%
\pgfsetlinewidth{0.501875pt}%
\definecolor{currentstroke}{rgb}{1.000000,1.000000,1.000000}%
\pgfsetstrokecolor{currentstroke}%
\pgfsetstrokeopacity{0.500000}%
\pgfsetdash{}{0pt}%
\pgfpathmoveto{\pgfqpoint{2.049584in}{1.784619in}}%
\pgfpathlineto{\pgfqpoint{2.061450in}{1.788597in}}%
\pgfpathlineto{\pgfqpoint{2.073309in}{1.792570in}}%
\pgfpathlineto{\pgfqpoint{2.085164in}{1.796539in}}%
\pgfpathlineto{\pgfqpoint{2.097012in}{1.800505in}}%
\pgfpathlineto{\pgfqpoint{2.108855in}{1.804469in}}%
\pgfpathlineto{\pgfqpoint{2.102774in}{1.814796in}}%
\pgfpathlineto{\pgfqpoint{2.096699in}{1.825083in}}%
\pgfpathlineto{\pgfqpoint{2.090628in}{1.835329in}}%
\pgfpathlineto{\pgfqpoint{2.084561in}{1.845534in}}%
\pgfpathlineto{\pgfqpoint{2.078499in}{1.855698in}}%
\pgfpathlineto{\pgfqpoint{2.066667in}{1.851776in}}%
\pgfpathlineto{\pgfqpoint{2.054828in}{1.847851in}}%
\pgfpathlineto{\pgfqpoint{2.042984in}{1.843922in}}%
\pgfpathlineto{\pgfqpoint{2.031134in}{1.839990in}}%
\pgfpathlineto{\pgfqpoint{2.019279in}{1.836052in}}%
\pgfpathlineto{\pgfqpoint{2.025331in}{1.825845in}}%
\pgfpathlineto{\pgfqpoint{2.031387in}{1.815599in}}%
\pgfpathlineto{\pgfqpoint{2.037448in}{1.805312in}}%
\pgfpathlineto{\pgfqpoint{2.043514in}{1.794985in}}%
\pgfpathclose%
\pgfusepath{stroke,fill}%
\end{pgfscope}%
\begin{pgfscope}%
\pgfpathrectangle{\pgfqpoint{0.887500in}{0.275000in}}{\pgfqpoint{4.225000in}{4.225000in}}%
\pgfusepath{clip}%
\pgfsetbuttcap%
\pgfsetroundjoin%
\definecolor{currentfill}{rgb}{0.227802,0.326594,0.546532}%
\pgfsetfillcolor{currentfill}%
\pgfsetfillopacity{0.700000}%
\pgfsetlinewidth{0.501875pt}%
\definecolor{currentstroke}{rgb}{1.000000,1.000000,1.000000}%
\pgfsetstrokecolor{currentstroke}%
\pgfsetstrokeopacity{0.500000}%
\pgfsetdash{}{0pt}%
\pgfpathmoveto{\pgfqpoint{2.557680in}{1.635319in}}%
\pgfpathlineto{\pgfqpoint{2.569422in}{1.639387in}}%
\pgfpathlineto{\pgfqpoint{2.581157in}{1.643457in}}%
\pgfpathlineto{\pgfqpoint{2.592887in}{1.647527in}}%
\pgfpathlineto{\pgfqpoint{2.604612in}{1.651596in}}%
\pgfpathlineto{\pgfqpoint{2.616331in}{1.655663in}}%
\pgfpathlineto{\pgfqpoint{2.610088in}{1.666872in}}%
\pgfpathlineto{\pgfqpoint{2.603849in}{1.678037in}}%
\pgfpathlineto{\pgfqpoint{2.597614in}{1.689157in}}%
\pgfpathlineto{\pgfqpoint{2.591384in}{1.700232in}}%
\pgfpathlineto{\pgfqpoint{2.585157in}{1.711261in}}%
\pgfpathlineto{\pgfqpoint{2.573447in}{1.707220in}}%
\pgfpathlineto{\pgfqpoint{2.561731in}{1.703175in}}%
\pgfpathlineto{\pgfqpoint{2.550010in}{1.699128in}}%
\pgfpathlineto{\pgfqpoint{2.538283in}{1.695081in}}%
\pgfpathlineto{\pgfqpoint{2.526551in}{1.691036in}}%
\pgfpathlineto{\pgfqpoint{2.532769in}{1.679980in}}%
\pgfpathlineto{\pgfqpoint{2.538990in}{1.668880in}}%
\pgfpathlineto{\pgfqpoint{2.545216in}{1.657736in}}%
\pgfpathlineto{\pgfqpoint{2.551446in}{1.646549in}}%
\pgfpathclose%
\pgfusepath{stroke,fill}%
\end{pgfscope}%
\begin{pgfscope}%
\pgfpathrectangle{\pgfqpoint{0.887500in}{0.275000in}}{\pgfqpoint{4.225000in}{4.225000in}}%
\pgfusepath{clip}%
\pgfsetbuttcap%
\pgfsetroundjoin%
\definecolor{currentfill}{rgb}{0.235526,0.309527,0.542944}%
\pgfsetfillcolor{currentfill}%
\pgfsetfillopacity{0.700000}%
\pgfsetlinewidth{0.501875pt}%
\definecolor{currentstroke}{rgb}{1.000000,1.000000,1.000000}%
\pgfsetstrokecolor{currentstroke}%
\pgfsetstrokeopacity{0.500000}%
\pgfsetdash{}{0pt}%
\pgfpathmoveto{\pgfqpoint{2.647602in}{1.598980in}}%
\pgfpathlineto{\pgfqpoint{2.659324in}{1.603065in}}%
\pgfpathlineto{\pgfqpoint{2.671041in}{1.607145in}}%
\pgfpathlineto{\pgfqpoint{2.682751in}{1.611220in}}%
\pgfpathlineto{\pgfqpoint{2.694456in}{1.615288in}}%
\pgfpathlineto{\pgfqpoint{2.706156in}{1.619346in}}%
\pgfpathlineto{\pgfqpoint{2.699885in}{1.630745in}}%
\pgfpathlineto{\pgfqpoint{2.693618in}{1.642097in}}%
\pgfpathlineto{\pgfqpoint{2.687355in}{1.653406in}}%
\pgfpathlineto{\pgfqpoint{2.681096in}{1.664673in}}%
\pgfpathlineto{\pgfqpoint{2.674841in}{1.675897in}}%
\pgfpathlineto{\pgfqpoint{2.663150in}{1.671869in}}%
\pgfpathlineto{\pgfqpoint{2.651454in}{1.667829in}}%
\pgfpathlineto{\pgfqpoint{2.639752in}{1.663781in}}%
\pgfpathlineto{\pgfqpoint{2.628044in}{1.659725in}}%
\pgfpathlineto{\pgfqpoint{2.616331in}{1.655663in}}%
\pgfpathlineto{\pgfqpoint{2.622577in}{1.644411in}}%
\pgfpathlineto{\pgfqpoint{2.628828in}{1.633117in}}%
\pgfpathlineto{\pgfqpoint{2.635082in}{1.621781in}}%
\pgfpathlineto{\pgfqpoint{2.641340in}{1.610403in}}%
\pgfpathclose%
\pgfusepath{stroke,fill}%
\end{pgfscope}%
\begin{pgfscope}%
\pgfpathrectangle{\pgfqpoint{0.887500in}{0.275000in}}{\pgfqpoint{4.225000in}{4.225000in}}%
\pgfusepath{clip}%
\pgfsetbuttcap%
\pgfsetroundjoin%
\definecolor{currentfill}{rgb}{0.199430,0.387607,0.554642}%
\pgfsetfillcolor{currentfill}%
\pgfsetfillopacity{0.700000}%
\pgfsetlinewidth{0.501875pt}%
\definecolor{currentstroke}{rgb}{1.000000,1.000000,1.000000}%
\pgfsetstrokecolor{currentstroke}%
\pgfsetstrokeopacity{0.500000}%
\pgfsetdash{}{0pt}%
\pgfpathmoveto{\pgfqpoint{2.139322in}{1.752229in}}%
\pgfpathlineto{\pgfqpoint{2.151169in}{1.756229in}}%
\pgfpathlineto{\pgfqpoint{2.163010in}{1.760227in}}%
\pgfpathlineto{\pgfqpoint{2.174845in}{1.764224in}}%
\pgfpathlineto{\pgfqpoint{2.186674in}{1.768222in}}%
\pgfpathlineto{\pgfqpoint{2.198498in}{1.772221in}}%
\pgfpathlineto{\pgfqpoint{2.192386in}{1.782712in}}%
\pgfpathlineto{\pgfqpoint{2.186278in}{1.793164in}}%
\pgfpathlineto{\pgfqpoint{2.180175in}{1.803574in}}%
\pgfpathlineto{\pgfqpoint{2.174076in}{1.813944in}}%
\pgfpathlineto{\pgfqpoint{2.167981in}{1.824273in}}%
\pgfpathlineto{\pgfqpoint{2.156167in}{1.820312in}}%
\pgfpathlineto{\pgfqpoint{2.144348in}{1.816351in}}%
\pgfpathlineto{\pgfqpoint{2.132522in}{1.812391in}}%
\pgfpathlineto{\pgfqpoint{2.120691in}{1.808431in}}%
\pgfpathlineto{\pgfqpoint{2.108855in}{1.804469in}}%
\pgfpathlineto{\pgfqpoint{2.114939in}{1.794101in}}%
\pgfpathlineto{\pgfqpoint{2.121028in}{1.783693in}}%
\pgfpathlineto{\pgfqpoint{2.127122in}{1.773245in}}%
\pgfpathlineto{\pgfqpoint{2.133220in}{1.762757in}}%
\pgfpathclose%
\pgfusepath{stroke,fill}%
\end{pgfscope}%
\begin{pgfscope}%
\pgfpathrectangle{\pgfqpoint{0.887500in}{0.275000in}}{\pgfqpoint{4.225000in}{4.225000in}}%
\pgfusepath{clip}%
\pgfsetbuttcap%
\pgfsetroundjoin%
\definecolor{currentfill}{rgb}{0.243113,0.292092,0.538516}%
\pgfsetfillcolor{currentfill}%
\pgfsetfillopacity{0.700000}%
\pgfsetlinewidth{0.501875pt}%
\definecolor{currentstroke}{rgb}{1.000000,1.000000,1.000000}%
\pgfsetstrokecolor{currentstroke}%
\pgfsetstrokeopacity{0.500000}%
\pgfsetdash{}{0pt}%
\pgfpathmoveto{\pgfqpoint{2.737566in}{1.561550in}}%
\pgfpathlineto{\pgfqpoint{2.749268in}{1.565615in}}%
\pgfpathlineto{\pgfqpoint{2.760965in}{1.569670in}}%
\pgfpathlineto{\pgfqpoint{2.772656in}{1.573712in}}%
\pgfpathlineto{\pgfqpoint{2.784341in}{1.577740in}}%
\pgfpathlineto{\pgfqpoint{2.796021in}{1.581756in}}%
\pgfpathlineto{\pgfqpoint{2.789723in}{1.593420in}}%
\pgfpathlineto{\pgfqpoint{2.783429in}{1.605017in}}%
\pgfpathlineto{\pgfqpoint{2.777139in}{1.616552in}}%
\pgfpathlineto{\pgfqpoint{2.770852in}{1.628028in}}%
\pgfpathlineto{\pgfqpoint{2.764569in}{1.639448in}}%
\pgfpathlineto{\pgfqpoint{2.752898in}{1.635456in}}%
\pgfpathlineto{\pgfqpoint{2.741221in}{1.631450in}}%
\pgfpathlineto{\pgfqpoint{2.729538in}{1.627429in}}%
\pgfpathlineto{\pgfqpoint{2.717849in}{1.623394in}}%
\pgfpathlineto{\pgfqpoint{2.706156in}{1.619346in}}%
\pgfpathlineto{\pgfqpoint{2.712430in}{1.607899in}}%
\pgfpathlineto{\pgfqpoint{2.718708in}{1.596398in}}%
\pgfpathlineto{\pgfqpoint{2.724990in}{1.584842in}}%
\pgfpathlineto{\pgfqpoint{2.731276in}{1.573227in}}%
\pgfpathclose%
\pgfusepath{stroke,fill}%
\end{pgfscope}%
\begin{pgfscope}%
\pgfpathrectangle{\pgfqpoint{0.887500in}{0.275000in}}{\pgfqpoint{4.225000in}{4.225000in}}%
\pgfusepath{clip}%
\pgfsetbuttcap%
\pgfsetroundjoin%
\definecolor{currentfill}{rgb}{0.282623,0.140926,0.457517}%
\pgfsetfillcolor{currentfill}%
\pgfsetfillopacity{0.700000}%
\pgfsetlinewidth{0.501875pt}%
\definecolor{currentstroke}{rgb}{1.000000,1.000000,1.000000}%
\pgfsetstrokecolor{currentstroke}%
\pgfsetstrokeopacity{0.500000}%
\pgfsetdash{}{0pt}%
\pgfpathmoveto{\pgfqpoint{3.277726in}{1.301980in}}%
\pgfpathlineto{\pgfqpoint{3.289274in}{1.302212in}}%
\pgfpathlineto{\pgfqpoint{3.300817in}{1.302932in}}%
\pgfpathlineto{\pgfqpoint{3.312356in}{1.304182in}}%
\pgfpathlineto{\pgfqpoint{3.323892in}{1.306001in}}%
\pgfpathlineto{\pgfqpoint{3.335425in}{1.308426in}}%
\pgfpathlineto{\pgfqpoint{3.329017in}{1.324781in}}%
\pgfpathlineto{\pgfqpoint{3.322610in}{1.340951in}}%
\pgfpathlineto{\pgfqpoint{3.316203in}{1.356877in}}%
\pgfpathlineto{\pgfqpoint{3.309796in}{1.372500in}}%
\pgfpathlineto{\pgfqpoint{3.303389in}{1.387762in}}%
\pgfpathlineto{\pgfqpoint{3.291848in}{1.384002in}}%
\pgfpathlineto{\pgfqpoint{3.280301in}{1.380325in}}%
\pgfpathlineto{\pgfqpoint{3.268749in}{1.376722in}}%
\pgfpathlineto{\pgfqpoint{3.257191in}{1.373186in}}%
\pgfpathlineto{\pgfqpoint{3.245628in}{1.369706in}}%
\pgfpathlineto{\pgfqpoint{3.252043in}{1.356269in}}%
\pgfpathlineto{\pgfqpoint{3.258459in}{1.342774in}}%
\pgfpathlineto{\pgfqpoint{3.264879in}{1.329225in}}%
\pgfpathlineto{\pgfqpoint{3.271301in}{1.315626in}}%
\pgfpathclose%
\pgfusepath{stroke,fill}%
\end{pgfscope}%
\begin{pgfscope}%
\pgfpathrectangle{\pgfqpoint{0.887500in}{0.275000in}}{\pgfqpoint{4.225000in}{4.225000in}}%
\pgfusepath{clip}%
\pgfsetbuttcap%
\pgfsetroundjoin%
\definecolor{currentfill}{rgb}{0.252194,0.269783,0.531579}%
\pgfsetfillcolor{currentfill}%
\pgfsetfillopacity{0.700000}%
\pgfsetlinewidth{0.501875pt}%
\definecolor{currentstroke}{rgb}{1.000000,1.000000,1.000000}%
\pgfsetstrokecolor{currentstroke}%
\pgfsetstrokeopacity{0.500000}%
\pgfsetdash{}{0pt}%
\pgfpathmoveto{\pgfqpoint{2.827568in}{1.522334in}}%
\pgfpathlineto{\pgfqpoint{2.839250in}{1.526361in}}%
\pgfpathlineto{\pgfqpoint{2.850926in}{1.530388in}}%
\pgfpathlineto{\pgfqpoint{2.862597in}{1.534414in}}%
\pgfpathlineto{\pgfqpoint{2.874262in}{1.538443in}}%
\pgfpathlineto{\pgfqpoint{2.885921in}{1.542476in}}%
\pgfpathlineto{\pgfqpoint{2.879596in}{1.554484in}}%
\pgfpathlineto{\pgfqpoint{2.873276in}{1.566420in}}%
\pgfpathlineto{\pgfqpoint{2.866959in}{1.578279in}}%
\pgfpathlineto{\pgfqpoint{2.860645in}{1.590061in}}%
\pgfpathlineto{\pgfqpoint{2.854335in}{1.601770in}}%
\pgfpathlineto{\pgfqpoint{2.842684in}{1.597766in}}%
\pgfpathlineto{\pgfqpoint{2.831027in}{1.593765in}}%
\pgfpathlineto{\pgfqpoint{2.819364in}{1.589766in}}%
\pgfpathlineto{\pgfqpoint{2.807695in}{1.585764in}}%
\pgfpathlineto{\pgfqpoint{2.796021in}{1.581756in}}%
\pgfpathlineto{\pgfqpoint{2.802323in}{1.570023in}}%
\pgfpathlineto{\pgfqpoint{2.808629in}{1.558215in}}%
\pgfpathlineto{\pgfqpoint{2.814938in}{1.546331in}}%
\pgfpathlineto{\pgfqpoint{2.821251in}{1.534368in}}%
\pgfpathclose%
\pgfusepath{stroke,fill}%
\end{pgfscope}%
\begin{pgfscope}%
\pgfpathrectangle{\pgfqpoint{0.887500in}{0.275000in}}{\pgfqpoint{4.225000in}{4.225000in}}%
\pgfusepath{clip}%
\pgfsetbuttcap%
\pgfsetroundjoin%
\definecolor{currentfill}{rgb}{0.204903,0.375746,0.553533}%
\pgfsetfillcolor{currentfill}%
\pgfsetfillopacity{0.700000}%
\pgfsetlinewidth{0.501875pt}%
\definecolor{currentstroke}{rgb}{1.000000,1.000000,1.000000}%
\pgfsetstrokecolor{currentstroke}%
\pgfsetstrokeopacity{0.500000}%
\pgfsetdash{}{0pt}%
\pgfpathmoveto{\pgfqpoint{2.229125in}{1.719156in}}%
\pgfpathlineto{\pgfqpoint{2.240952in}{1.723186in}}%
\pgfpathlineto{\pgfqpoint{2.252774in}{1.727215in}}%
\pgfpathlineto{\pgfqpoint{2.264590in}{1.731243in}}%
\pgfpathlineto{\pgfqpoint{2.276401in}{1.735268in}}%
\pgfpathlineto{\pgfqpoint{2.288205in}{1.739290in}}%
\pgfpathlineto{\pgfqpoint{2.282062in}{1.749953in}}%
\pgfpathlineto{\pgfqpoint{2.275922in}{1.760574in}}%
\pgfpathlineto{\pgfqpoint{2.269787in}{1.771153in}}%
\pgfpathlineto{\pgfqpoint{2.263657in}{1.781692in}}%
\pgfpathlineto{\pgfqpoint{2.257530in}{1.792188in}}%
\pgfpathlineto{\pgfqpoint{2.245735in}{1.788201in}}%
\pgfpathlineto{\pgfqpoint{2.233934in}{1.784209in}}%
\pgfpathlineto{\pgfqpoint{2.222128in}{1.780215in}}%
\pgfpathlineto{\pgfqpoint{2.210316in}{1.776218in}}%
\pgfpathlineto{\pgfqpoint{2.198498in}{1.772221in}}%
\pgfpathlineto{\pgfqpoint{2.204615in}{1.761688in}}%
\pgfpathlineto{\pgfqpoint{2.210736in}{1.751116in}}%
\pgfpathlineto{\pgfqpoint{2.216861in}{1.740503in}}%
\pgfpathlineto{\pgfqpoint{2.222991in}{1.729849in}}%
\pgfpathclose%
\pgfusepath{stroke,fill}%
\end{pgfscope}%
\begin{pgfscope}%
\pgfpathrectangle{\pgfqpoint{0.887500in}{0.275000in}}{\pgfqpoint{4.225000in}{4.225000in}}%
\pgfusepath{clip}%
\pgfsetbuttcap%
\pgfsetroundjoin%
\definecolor{currentfill}{rgb}{0.258965,0.251537,0.524736}%
\pgfsetfillcolor{currentfill}%
\pgfsetfillopacity{0.700000}%
\pgfsetlinewidth{0.501875pt}%
\definecolor{currentstroke}{rgb}{1.000000,1.000000,1.000000}%
\pgfsetstrokecolor{currentstroke}%
\pgfsetstrokeopacity{0.500000}%
\pgfsetdash{}{0pt}%
\pgfpathmoveto{\pgfqpoint{2.917594in}{1.481675in}}%
\pgfpathlineto{\pgfqpoint{2.929255in}{1.485726in}}%
\pgfpathlineto{\pgfqpoint{2.940910in}{1.489787in}}%
\pgfpathlineto{\pgfqpoint{2.952559in}{1.493863in}}%
\pgfpathlineto{\pgfqpoint{2.964203in}{1.497959in}}%
\pgfpathlineto{\pgfqpoint{2.975841in}{1.502079in}}%
\pgfpathlineto{\pgfqpoint{2.969492in}{1.514320in}}%
\pgfpathlineto{\pgfqpoint{2.963147in}{1.526512in}}%
\pgfpathlineto{\pgfqpoint{2.956805in}{1.538652in}}%
\pgfpathlineto{\pgfqpoint{2.950466in}{1.550734in}}%
\pgfpathlineto{\pgfqpoint{2.944131in}{1.562757in}}%
\pgfpathlineto{\pgfqpoint{2.932500in}{1.558680in}}%
\pgfpathlineto{\pgfqpoint{2.920864in}{1.554616in}}%
\pgfpathlineto{\pgfqpoint{2.909222in}{1.550561in}}%
\pgfpathlineto{\pgfqpoint{2.897574in}{1.546515in}}%
\pgfpathlineto{\pgfqpoint{2.885921in}{1.542476in}}%
\pgfpathlineto{\pgfqpoint{2.892249in}{1.530406in}}%
\pgfpathlineto{\pgfqpoint{2.898580in}{1.518282in}}%
\pgfpathlineto{\pgfqpoint{2.904915in}{1.506113in}}%
\pgfpathlineto{\pgfqpoint{2.911253in}{1.493908in}}%
\pgfpathclose%
\pgfusepath{stroke,fill}%
\end{pgfscope}%
\begin{pgfscope}%
\pgfpathrectangle{\pgfqpoint{0.887500in}{0.275000in}}{\pgfqpoint{4.225000in}{4.225000in}}%
\pgfusepath{clip}%
\pgfsetbuttcap%
\pgfsetroundjoin%
\definecolor{currentfill}{rgb}{0.279574,0.170599,0.479997}%
\pgfsetfillcolor{currentfill}%
\pgfsetfillopacity{0.700000}%
\pgfsetlinewidth{0.501875pt}%
\definecolor{currentstroke}{rgb}{1.000000,1.000000,1.000000}%
\pgfsetstrokecolor{currentstroke}%
\pgfsetstrokeopacity{0.500000}%
\pgfsetdash{}{0pt}%
\pgfpathmoveto{\pgfqpoint{3.187722in}{1.352224in}}%
\pgfpathlineto{\pgfqpoint{3.199316in}{1.355837in}}%
\pgfpathlineto{\pgfqpoint{3.210903in}{1.359360in}}%
\pgfpathlineto{\pgfqpoint{3.222484in}{1.362825in}}%
\pgfpathlineto{\pgfqpoint{3.234059in}{1.366263in}}%
\pgfpathlineto{\pgfqpoint{3.245628in}{1.369706in}}%
\pgfpathlineto{\pgfqpoint{3.239217in}{1.383080in}}%
\pgfpathlineto{\pgfqpoint{3.232807in}{1.396388in}}%
\pgfpathlineto{\pgfqpoint{3.226401in}{1.409629in}}%
\pgfpathlineto{\pgfqpoint{3.219997in}{1.422799in}}%
\pgfpathlineto{\pgfqpoint{3.213596in}{1.435897in}}%
\pgfpathlineto{\pgfqpoint{3.202028in}{1.431294in}}%
\pgfpathlineto{\pgfqpoint{3.190454in}{1.426681in}}%
\pgfpathlineto{\pgfqpoint{3.178875in}{1.422107in}}%
\pgfpathlineto{\pgfqpoint{3.167291in}{1.417621in}}%
\pgfpathlineto{\pgfqpoint{3.155702in}{1.413272in}}%
\pgfpathlineto{\pgfqpoint{3.162097in}{1.400365in}}%
\pgfpathlineto{\pgfqpoint{3.168497in}{1.387740in}}%
\pgfpathlineto{\pgfqpoint{3.174900in}{1.375466in}}%
\pgfpathlineto{\pgfqpoint{3.181309in}{1.363611in}}%
\pgfpathclose%
\pgfusepath{stroke,fill}%
\end{pgfscope}%
\begin{pgfscope}%
\pgfpathrectangle{\pgfqpoint{0.887500in}{0.275000in}}{\pgfqpoint{4.225000in}{4.225000in}}%
\pgfusepath{clip}%
\pgfsetbuttcap%
\pgfsetroundjoin%
\definecolor{currentfill}{rgb}{0.269944,0.014625,0.341379}%
\pgfsetfillcolor{currentfill}%
\pgfsetfillopacity{0.700000}%
\pgfsetlinewidth{0.501875pt}%
\definecolor{currentstroke}{rgb}{1.000000,1.000000,1.000000}%
\pgfsetstrokecolor{currentstroke}%
\pgfsetstrokeopacity{0.500000}%
\pgfsetdash{}{0pt}%
\pgfpathmoveto{\pgfqpoint{3.489596in}{1.098481in}}%
\pgfpathlineto{\pgfqpoint{3.501147in}{1.105471in}}%
\pgfpathlineto{\pgfqpoint{3.512698in}{1.112723in}}%
\pgfpathlineto{\pgfqpoint{3.524243in}{1.119830in}}%
\pgfpathlineto{\pgfqpoint{3.535777in}{1.126385in}}%
\pgfpathlineto{\pgfqpoint{3.547295in}{1.132033in}}%
\pgfpathlineto{\pgfqpoint{3.540850in}{1.148209in}}%
\pgfpathlineto{\pgfqpoint{3.534404in}{1.164220in}}%
\pgfpathlineto{\pgfqpoint{3.527959in}{1.180069in}}%
\pgfpathlineto{\pgfqpoint{3.521514in}{1.195760in}}%
\pgfpathlineto{\pgfqpoint{3.515069in}{1.211297in}}%
\pgfpathlineto{\pgfqpoint{3.503534in}{1.204017in}}%
\pgfpathlineto{\pgfqpoint{3.491985in}{1.195859in}}%
\pgfpathlineto{\pgfqpoint{3.480428in}{1.187204in}}%
\pgfpathlineto{\pgfqpoint{3.468867in}{1.178480in}}%
\pgfpathlineto{\pgfqpoint{3.457308in}{1.170117in}}%
\pgfpathlineto{\pgfqpoint{3.463752in}{1.155015in}}%
\pgfpathlineto{\pgfqpoint{3.470202in}{1.140257in}}%
\pgfpathlineto{\pgfqpoint{3.476659in}{1.125887in}}%
\pgfpathlineto{\pgfqpoint{3.483124in}{1.111947in}}%
\pgfpathclose%
\pgfusepath{stroke,fill}%
\end{pgfscope}%
\begin{pgfscope}%
\pgfpathrectangle{\pgfqpoint{0.887500in}{0.275000in}}{\pgfqpoint{4.225000in}{4.225000in}}%
\pgfusepath{clip}%
\pgfsetbuttcap%
\pgfsetroundjoin%
\definecolor{currentfill}{rgb}{0.274128,0.199721,0.498911}%
\pgfsetfillcolor{currentfill}%
\pgfsetfillopacity{0.700000}%
\pgfsetlinewidth{0.501875pt}%
\definecolor{currentstroke}{rgb}{1.000000,1.000000,1.000000}%
\pgfsetstrokecolor{currentstroke}%
\pgfsetstrokeopacity{0.500000}%
\pgfsetdash{}{0pt}%
\pgfpathmoveto{\pgfqpoint{3.097678in}{1.394498in}}%
\pgfpathlineto{\pgfqpoint{3.109294in}{1.397929in}}%
\pgfpathlineto{\pgfqpoint{3.120904in}{1.401471in}}%
\pgfpathlineto{\pgfqpoint{3.132508in}{1.405181in}}%
\pgfpathlineto{\pgfqpoint{3.144108in}{1.409110in}}%
\pgfpathlineto{\pgfqpoint{3.155702in}{1.413272in}}%
\pgfpathlineto{\pgfqpoint{3.149310in}{1.426393in}}%
\pgfpathlineto{\pgfqpoint{3.142921in}{1.439660in}}%
\pgfpathlineto{\pgfqpoint{3.136535in}{1.453005in}}%
\pgfpathlineto{\pgfqpoint{3.130153in}{1.466359in}}%
\pgfpathlineto{\pgfqpoint{3.123773in}{1.479653in}}%
\pgfpathlineto{\pgfqpoint{3.112184in}{1.475607in}}%
\pgfpathlineto{\pgfqpoint{3.100590in}{1.471609in}}%
\pgfpathlineto{\pgfqpoint{3.088991in}{1.467655in}}%
\pgfpathlineto{\pgfqpoint{3.077385in}{1.463734in}}%
\pgfpathlineto{\pgfqpoint{3.065774in}{1.459836in}}%
\pgfpathlineto{\pgfqpoint{3.072149in}{1.446791in}}%
\pgfpathlineto{\pgfqpoint{3.078527in}{1.433690in}}%
\pgfpathlineto{\pgfqpoint{3.084907in}{1.420578in}}%
\pgfpathlineto{\pgfqpoint{3.091291in}{1.407499in}}%
\pgfpathclose%
\pgfusepath{stroke,fill}%
\end{pgfscope}%
\begin{pgfscope}%
\pgfpathrectangle{\pgfqpoint{0.887500in}{0.275000in}}{\pgfqpoint{4.225000in}{4.225000in}}%
\pgfusepath{clip}%
\pgfsetbuttcap%
\pgfsetroundjoin%
\definecolor{currentfill}{rgb}{0.276022,0.044167,0.370164}%
\pgfsetfillcolor{currentfill}%
\pgfsetfillopacity{0.700000}%
\pgfsetlinewidth{0.501875pt}%
\definecolor{currentstroke}{rgb}{1.000000,1.000000,1.000000}%
\pgfsetstrokecolor{currentstroke}%
\pgfsetstrokeopacity{0.500000}%
\pgfsetdash{}{0pt}%
\pgfpathmoveto{\pgfqpoint{3.399635in}{1.147517in}}%
\pgfpathlineto{\pgfqpoint{3.411149in}{1.148618in}}%
\pgfpathlineto{\pgfqpoint{3.422672in}{1.151474in}}%
\pgfpathlineto{\pgfqpoint{3.434206in}{1.156186in}}%
\pgfpathlineto{\pgfqpoint{3.445753in}{1.162543in}}%
\pgfpathlineto{\pgfqpoint{3.457308in}{1.170117in}}%
\pgfpathlineto{\pgfqpoint{3.450869in}{1.185517in}}%
\pgfpathlineto{\pgfqpoint{3.444435in}{1.201174in}}%
\pgfpathlineto{\pgfqpoint{3.438006in}{1.217042in}}%
\pgfpathlineto{\pgfqpoint{3.431580in}{1.233079in}}%
\pgfpathlineto{\pgfqpoint{3.425158in}{1.249240in}}%
\pgfpathlineto{\pgfqpoint{3.413612in}{1.242244in}}%
\pgfpathlineto{\pgfqpoint{3.402072in}{1.236176in}}%
\pgfpathlineto{\pgfqpoint{3.390538in}{1.231358in}}%
\pgfpathlineto{\pgfqpoint{3.379010in}{1.227959in}}%
\pgfpathlineto{\pgfqpoint{3.367488in}{1.225916in}}%
\pgfpathlineto{\pgfqpoint{3.373908in}{1.209675in}}%
\pgfpathlineto{\pgfqpoint{3.380333in}{1.193656in}}%
\pgfpathlineto{\pgfqpoint{3.386761in}{1.177919in}}%
\pgfpathlineto{\pgfqpoint{3.393195in}{1.162520in}}%
\pgfpathclose%
\pgfusepath{stroke,fill}%
\end{pgfscope}%
\begin{pgfscope}%
\pgfpathrectangle{\pgfqpoint{0.887500in}{0.275000in}}{\pgfqpoint{4.225000in}{4.225000in}}%
\pgfusepath{clip}%
\pgfsetbuttcap%
\pgfsetroundjoin%
\definecolor{currentfill}{rgb}{0.266580,0.228262,0.514349}%
\pgfsetfillcolor{currentfill}%
\pgfsetfillopacity{0.700000}%
\pgfsetlinewidth{0.501875pt}%
\definecolor{currentstroke}{rgb}{1.000000,1.000000,1.000000}%
\pgfsetstrokecolor{currentstroke}%
\pgfsetstrokeopacity{0.500000}%
\pgfsetdash{}{0pt}%
\pgfpathmoveto{\pgfqpoint{3.007634in}{1.440290in}}%
\pgfpathlineto{\pgfqpoint{3.019273in}{1.444236in}}%
\pgfpathlineto{\pgfqpoint{3.030907in}{1.448159in}}%
\pgfpathlineto{\pgfqpoint{3.042535in}{1.452060in}}%
\pgfpathlineto{\pgfqpoint{3.054158in}{1.455948in}}%
\pgfpathlineto{\pgfqpoint{3.065774in}{1.459836in}}%
\pgfpathlineto{\pgfqpoint{3.059403in}{1.472781in}}%
\pgfpathlineto{\pgfqpoint{3.053034in}{1.485582in}}%
\pgfpathlineto{\pgfqpoint{3.046668in}{1.498226in}}%
\pgfpathlineto{\pgfqpoint{3.040306in}{1.510724in}}%
\pgfpathlineto{\pgfqpoint{3.033947in}{1.523088in}}%
\pgfpathlineto{\pgfqpoint{3.022337in}{1.518861in}}%
\pgfpathlineto{\pgfqpoint{3.010721in}{1.514631in}}%
\pgfpathlineto{\pgfqpoint{2.999100in}{1.510414in}}%
\pgfpathlineto{\pgfqpoint{2.987473in}{1.506230in}}%
\pgfpathlineto{\pgfqpoint{2.975841in}{1.502079in}}%
\pgfpathlineto{\pgfqpoint{2.982193in}{1.489795in}}%
\pgfpathlineto{\pgfqpoint{2.988548in}{1.477470in}}%
\pgfpathlineto{\pgfqpoint{2.994907in}{1.465110in}}%
\pgfpathlineto{\pgfqpoint{3.001269in}{1.452717in}}%
\pgfpathclose%
\pgfusepath{stroke,fill}%
\end{pgfscope}%
\begin{pgfscope}%
\pgfpathrectangle{\pgfqpoint{0.887500in}{0.275000in}}{\pgfqpoint{4.225000in}{4.225000in}}%
\pgfusepath{clip}%
\pgfsetbuttcap%
\pgfsetroundjoin%
\definecolor{currentfill}{rgb}{0.212395,0.359683,0.551710}%
\pgfsetfillcolor{currentfill}%
\pgfsetfillopacity{0.700000}%
\pgfsetlinewidth{0.501875pt}%
\definecolor{currentstroke}{rgb}{1.000000,1.000000,1.000000}%
\pgfsetstrokecolor{currentstroke}%
\pgfsetstrokeopacity{0.500000}%
\pgfsetdash{}{0pt}%
\pgfpathmoveto{\pgfqpoint{2.318988in}{1.685348in}}%
\pgfpathlineto{\pgfqpoint{2.330797in}{1.689394in}}%
\pgfpathlineto{\pgfqpoint{2.342600in}{1.693437in}}%
\pgfpathlineto{\pgfqpoint{2.354397in}{1.697476in}}%
\pgfpathlineto{\pgfqpoint{2.366189in}{1.701510in}}%
\pgfpathlineto{\pgfqpoint{2.377975in}{1.705540in}}%
\pgfpathlineto{\pgfqpoint{2.371800in}{1.716385in}}%
\pgfpathlineto{\pgfqpoint{2.365630in}{1.727187in}}%
\pgfpathlineto{\pgfqpoint{2.359464in}{1.737945in}}%
\pgfpathlineto{\pgfqpoint{2.353302in}{1.748660in}}%
\pgfpathlineto{\pgfqpoint{2.347144in}{1.759332in}}%
\pgfpathlineto{\pgfqpoint{2.335368in}{1.755334in}}%
\pgfpathlineto{\pgfqpoint{2.323586in}{1.751331in}}%
\pgfpathlineto{\pgfqpoint{2.311798in}{1.747322in}}%
\pgfpathlineto{\pgfqpoint{2.300005in}{1.743308in}}%
\pgfpathlineto{\pgfqpoint{2.288205in}{1.739290in}}%
\pgfpathlineto{\pgfqpoint{2.294353in}{1.728585in}}%
\pgfpathlineto{\pgfqpoint{2.300506in}{1.717839in}}%
\pgfpathlineto{\pgfqpoint{2.306662in}{1.707051in}}%
\pgfpathlineto{\pgfqpoint{2.312823in}{1.696220in}}%
\pgfpathclose%
\pgfusepath{stroke,fill}%
\end{pgfscope}%
\begin{pgfscope}%
\pgfpathrectangle{\pgfqpoint{0.887500in}{0.275000in}}{\pgfqpoint{4.225000in}{4.225000in}}%
\pgfusepath{clip}%
\pgfsetbuttcap%
\pgfsetroundjoin%
\definecolor{currentfill}{rgb}{0.220057,0.343307,0.549413}%
\pgfsetfillcolor{currentfill}%
\pgfsetfillopacity{0.700000}%
\pgfsetlinewidth{0.501875pt}%
\definecolor{currentstroke}{rgb}{1.000000,1.000000,1.000000}%
\pgfsetstrokecolor{currentstroke}%
\pgfsetstrokeopacity{0.500000}%
\pgfsetdash{}{0pt}%
\pgfpathmoveto{\pgfqpoint{2.408910in}{1.650659in}}%
\pgfpathlineto{\pgfqpoint{2.420700in}{1.654709in}}%
\pgfpathlineto{\pgfqpoint{2.432484in}{1.658754in}}%
\pgfpathlineto{\pgfqpoint{2.444262in}{1.662794in}}%
\pgfpathlineto{\pgfqpoint{2.456035in}{1.666829in}}%
\pgfpathlineto{\pgfqpoint{2.467802in}{1.670862in}}%
\pgfpathlineto{\pgfqpoint{2.461597in}{1.681899in}}%
\pgfpathlineto{\pgfqpoint{2.455397in}{1.692893in}}%
\pgfpathlineto{\pgfqpoint{2.449200in}{1.703841in}}%
\pgfpathlineto{\pgfqpoint{2.443008in}{1.714745in}}%
\pgfpathlineto{\pgfqpoint{2.436820in}{1.725603in}}%
\pgfpathlineto{\pgfqpoint{2.425062in}{1.721600in}}%
\pgfpathlineto{\pgfqpoint{2.413299in}{1.717594in}}%
\pgfpathlineto{\pgfqpoint{2.401530in}{1.713582in}}%
\pgfpathlineto{\pgfqpoint{2.389755in}{1.709564in}}%
\pgfpathlineto{\pgfqpoint{2.377975in}{1.705540in}}%
\pgfpathlineto{\pgfqpoint{2.384153in}{1.694650in}}%
\pgfpathlineto{\pgfqpoint{2.390336in}{1.683718in}}%
\pgfpathlineto{\pgfqpoint{2.396524in}{1.672741in}}%
\pgfpathlineto{\pgfqpoint{2.402715in}{1.661721in}}%
\pgfpathclose%
\pgfusepath{stroke,fill}%
\end{pgfscope}%
\begin{pgfscope}%
\pgfpathrectangle{\pgfqpoint{0.887500in}{0.275000in}}{\pgfqpoint{4.225000in}{4.225000in}}%
\pgfusepath{clip}%
\pgfsetbuttcap%
\pgfsetroundjoin%
\definecolor{currentfill}{rgb}{0.194100,0.399323,0.555565}%
\pgfsetfillcolor{currentfill}%
\pgfsetfillopacity{0.700000}%
\pgfsetlinewidth{0.501875pt}%
\definecolor{currentstroke}{rgb}{1.000000,1.000000,1.000000}%
\pgfsetstrokecolor{currentstroke}%
\pgfsetstrokeopacity{0.500000}%
\pgfsetdash{}{0pt}%
\pgfpathmoveto{\pgfqpoint{1.990172in}{1.764631in}}%
\pgfpathlineto{\pgfqpoint{2.002066in}{1.768644in}}%
\pgfpathlineto{\pgfqpoint{2.013954in}{1.772648in}}%
\pgfpathlineto{\pgfqpoint{2.025836in}{1.776645in}}%
\pgfpathlineto{\pgfqpoint{2.037713in}{1.780635in}}%
\pgfpathlineto{\pgfqpoint{2.049584in}{1.784619in}}%
\pgfpathlineto{\pgfqpoint{2.043514in}{1.794985in}}%
\pgfpathlineto{\pgfqpoint{2.037448in}{1.805312in}}%
\pgfpathlineto{\pgfqpoint{2.031387in}{1.815599in}}%
\pgfpathlineto{\pgfqpoint{2.025331in}{1.825845in}}%
\pgfpathlineto{\pgfqpoint{2.019279in}{1.836052in}}%
\pgfpathlineto{\pgfqpoint{2.007418in}{1.832107in}}%
\pgfpathlineto{\pgfqpoint{1.995551in}{1.828157in}}%
\pgfpathlineto{\pgfqpoint{1.983679in}{1.824198in}}%
\pgfpathlineto{\pgfqpoint{1.971801in}{1.820231in}}%
\pgfpathlineto{\pgfqpoint{1.959918in}{1.816254in}}%
\pgfpathlineto{\pgfqpoint{1.965959in}{1.806009in}}%
\pgfpathlineto{\pgfqpoint{1.972006in}{1.795723in}}%
\pgfpathlineto{\pgfqpoint{1.978057in}{1.785398in}}%
\pgfpathlineto{\pgfqpoint{1.984112in}{1.775034in}}%
\pgfpathclose%
\pgfusepath{stroke,fill}%
\end{pgfscope}%
\begin{pgfscope}%
\pgfpathrectangle{\pgfqpoint{0.887500in}{0.275000in}}{\pgfqpoint{4.225000in}{4.225000in}}%
\pgfusepath{clip}%
\pgfsetbuttcap%
\pgfsetroundjoin%
\definecolor{currentfill}{rgb}{0.227802,0.326594,0.546532}%
\pgfsetfillcolor{currentfill}%
\pgfsetfillopacity{0.700000}%
\pgfsetlinewidth{0.501875pt}%
\definecolor{currentstroke}{rgb}{1.000000,1.000000,1.000000}%
\pgfsetstrokecolor{currentstroke}%
\pgfsetstrokeopacity{0.500000}%
\pgfsetdash{}{0pt}%
\pgfpathmoveto{\pgfqpoint{2.498886in}{1.615033in}}%
\pgfpathlineto{\pgfqpoint{2.510656in}{1.619087in}}%
\pgfpathlineto{\pgfqpoint{2.522421in}{1.623141in}}%
\pgfpathlineto{\pgfqpoint{2.534180in}{1.627197in}}%
\pgfpathlineto{\pgfqpoint{2.545933in}{1.631256in}}%
\pgfpathlineto{\pgfqpoint{2.557680in}{1.635319in}}%
\pgfpathlineto{\pgfqpoint{2.551446in}{1.646549in}}%
\pgfpathlineto{\pgfqpoint{2.545216in}{1.657736in}}%
\pgfpathlineto{\pgfqpoint{2.538990in}{1.668880in}}%
\pgfpathlineto{\pgfqpoint{2.532769in}{1.679980in}}%
\pgfpathlineto{\pgfqpoint{2.526551in}{1.691036in}}%
\pgfpathlineto{\pgfqpoint{2.514813in}{1.686995in}}%
\pgfpathlineto{\pgfqpoint{2.503069in}{1.682958in}}%
\pgfpathlineto{\pgfqpoint{2.491319in}{1.678925in}}%
\pgfpathlineto{\pgfqpoint{2.479563in}{1.674893in}}%
\pgfpathlineto{\pgfqpoint{2.467802in}{1.670862in}}%
\pgfpathlineto{\pgfqpoint{2.474011in}{1.659781in}}%
\pgfpathlineto{\pgfqpoint{2.480223in}{1.648657in}}%
\pgfpathlineto{\pgfqpoint{2.486440in}{1.637490in}}%
\pgfpathlineto{\pgfqpoint{2.492661in}{1.626282in}}%
\pgfpathclose%
\pgfusepath{stroke,fill}%
\end{pgfscope}%
\begin{pgfscope}%
\pgfpathrectangle{\pgfqpoint{0.887500in}{0.275000in}}{\pgfqpoint{4.225000in}{4.225000in}}%
\pgfusepath{clip}%
\pgfsetbuttcap%
\pgfsetroundjoin%
\definecolor{currentfill}{rgb}{0.282656,0.100196,0.422160}%
\pgfsetfillcolor{currentfill}%
\pgfsetfillopacity{0.700000}%
\pgfsetlinewidth{0.501875pt}%
\definecolor{currentstroke}{rgb}{1.000000,1.000000,1.000000}%
\pgfsetstrokecolor{currentstroke}%
\pgfsetstrokeopacity{0.500000}%
\pgfsetdash{}{0pt}%
\pgfpathmoveto{\pgfqpoint{3.309889in}{1.233189in}}%
\pgfpathlineto{\pgfqpoint{3.321412in}{1.229679in}}%
\pgfpathlineto{\pgfqpoint{3.332932in}{1.227110in}}%
\pgfpathlineto{\pgfqpoint{3.344451in}{1.225573in}}%
\pgfpathlineto{\pgfqpoint{3.355969in}{1.225148in}}%
\pgfpathlineto{\pgfqpoint{3.367488in}{1.225916in}}%
\pgfpathlineto{\pgfqpoint{3.361071in}{1.242322in}}%
\pgfpathlineto{\pgfqpoint{3.354656in}{1.258834in}}%
\pgfpathlineto{\pgfqpoint{3.348244in}{1.275394in}}%
\pgfpathlineto{\pgfqpoint{3.341834in}{1.291945in}}%
\pgfpathlineto{\pgfqpoint{3.335425in}{1.308426in}}%
\pgfpathlineto{\pgfqpoint{3.323892in}{1.306001in}}%
\pgfpathlineto{\pgfqpoint{3.312356in}{1.304182in}}%
\pgfpathlineto{\pgfqpoint{3.300817in}{1.302932in}}%
\pgfpathlineto{\pgfqpoint{3.289274in}{1.302212in}}%
\pgfpathlineto{\pgfqpoint{3.277726in}{1.301980in}}%
\pgfpathlineto{\pgfqpoint{3.284153in}{1.288291in}}%
\pgfpathlineto{\pgfqpoint{3.290583in}{1.274564in}}%
\pgfpathlineto{\pgfqpoint{3.297016in}{1.260802in}}%
\pgfpathlineto{\pgfqpoint{3.303451in}{1.247009in}}%
\pgfpathclose%
\pgfusepath{stroke,fill}%
\end{pgfscope}%
\begin{pgfscope}%
\pgfpathrectangle{\pgfqpoint{0.887500in}{0.275000in}}{\pgfqpoint{4.225000in}{4.225000in}}%
\pgfusepath{clip}%
\pgfsetbuttcap%
\pgfsetroundjoin%
\definecolor{currentfill}{rgb}{0.235526,0.309527,0.542944}%
\pgfsetfillcolor{currentfill}%
\pgfsetfillopacity{0.700000}%
\pgfsetlinewidth{0.501875pt}%
\definecolor{currentstroke}{rgb}{1.000000,1.000000,1.000000}%
\pgfsetstrokecolor{currentstroke}%
\pgfsetstrokeopacity{0.500000}%
\pgfsetdash{}{0pt}%
\pgfpathmoveto{\pgfqpoint{2.588908in}{1.578549in}}%
\pgfpathlineto{\pgfqpoint{2.600658in}{1.582632in}}%
\pgfpathlineto{\pgfqpoint{2.612403in}{1.586718in}}%
\pgfpathlineto{\pgfqpoint{2.624142in}{1.590806in}}%
\pgfpathlineto{\pgfqpoint{2.635875in}{1.594894in}}%
\pgfpathlineto{\pgfqpoint{2.647602in}{1.598980in}}%
\pgfpathlineto{\pgfqpoint{2.641340in}{1.610403in}}%
\pgfpathlineto{\pgfqpoint{2.635082in}{1.621781in}}%
\pgfpathlineto{\pgfqpoint{2.628828in}{1.633117in}}%
\pgfpathlineto{\pgfqpoint{2.622577in}{1.644411in}}%
\pgfpathlineto{\pgfqpoint{2.616331in}{1.655663in}}%
\pgfpathlineto{\pgfqpoint{2.604612in}{1.651596in}}%
\pgfpathlineto{\pgfqpoint{2.592887in}{1.647527in}}%
\pgfpathlineto{\pgfqpoint{2.581157in}{1.643457in}}%
\pgfpathlineto{\pgfqpoint{2.569422in}{1.639387in}}%
\pgfpathlineto{\pgfqpoint{2.557680in}{1.635319in}}%
\pgfpathlineto{\pgfqpoint{2.563918in}{1.624048in}}%
\pgfpathlineto{\pgfqpoint{2.570160in}{1.612735in}}%
\pgfpathlineto{\pgfqpoint{2.576405in}{1.601381in}}%
\pgfpathlineto{\pgfqpoint{2.582655in}{1.589986in}}%
\pgfpathclose%
\pgfusepath{stroke,fill}%
\end{pgfscope}%
\begin{pgfscope}%
\pgfpathrectangle{\pgfqpoint{0.887500in}{0.275000in}}{\pgfqpoint{4.225000in}{4.225000in}}%
\pgfusepath{clip}%
\pgfsetbuttcap%
\pgfsetroundjoin%
\definecolor{currentfill}{rgb}{0.199430,0.387607,0.554642}%
\pgfsetfillcolor{currentfill}%
\pgfsetfillopacity{0.700000}%
\pgfsetlinewidth{0.501875pt}%
\definecolor{currentstroke}{rgb}{1.000000,1.000000,1.000000}%
\pgfsetstrokecolor{currentstroke}%
\pgfsetstrokeopacity{0.500000}%
\pgfsetdash{}{0pt}%
\pgfpathmoveto{\pgfqpoint{2.080002in}{1.732195in}}%
\pgfpathlineto{\pgfqpoint{2.091878in}{1.736209in}}%
\pgfpathlineto{\pgfqpoint{2.103747in}{1.740219in}}%
\pgfpathlineto{\pgfqpoint{2.115611in}{1.744225in}}%
\pgfpathlineto{\pgfqpoint{2.127470in}{1.748229in}}%
\pgfpathlineto{\pgfqpoint{2.139322in}{1.752229in}}%
\pgfpathlineto{\pgfqpoint{2.133220in}{1.762757in}}%
\pgfpathlineto{\pgfqpoint{2.127122in}{1.773245in}}%
\pgfpathlineto{\pgfqpoint{2.121028in}{1.783693in}}%
\pgfpathlineto{\pgfqpoint{2.114939in}{1.794101in}}%
\pgfpathlineto{\pgfqpoint{2.108855in}{1.804469in}}%
\pgfpathlineto{\pgfqpoint{2.097012in}{1.800505in}}%
\pgfpathlineto{\pgfqpoint{2.085164in}{1.796539in}}%
\pgfpathlineto{\pgfqpoint{2.073309in}{1.792570in}}%
\pgfpathlineto{\pgfqpoint{2.061450in}{1.788597in}}%
\pgfpathlineto{\pgfqpoint{2.049584in}{1.784619in}}%
\pgfpathlineto{\pgfqpoint{2.055659in}{1.774212in}}%
\pgfpathlineto{\pgfqpoint{2.061738in}{1.763766in}}%
\pgfpathlineto{\pgfqpoint{2.067822in}{1.753281in}}%
\pgfpathlineto{\pgfqpoint{2.073910in}{1.742757in}}%
\pgfpathclose%
\pgfusepath{stroke,fill}%
\end{pgfscope}%
\begin{pgfscope}%
\pgfpathrectangle{\pgfqpoint{0.887500in}{0.275000in}}{\pgfqpoint{4.225000in}{4.225000in}}%
\pgfusepath{clip}%
\pgfsetbuttcap%
\pgfsetroundjoin%
\definecolor{currentfill}{rgb}{0.267004,0.004874,0.329415}%
\pgfsetfillcolor{currentfill}%
\pgfsetfillopacity{0.700000}%
\pgfsetlinewidth{0.501875pt}%
\definecolor{currentstroke}{rgb}{1.000000,1.000000,1.000000}%
\pgfsetstrokecolor{currentstroke}%
\pgfsetstrokeopacity{0.500000}%
\pgfsetdash{}{0pt}%
\pgfpathmoveto{\pgfqpoint{3.431949in}{1.080483in}}%
\pgfpathlineto{\pgfqpoint{3.443460in}{1.081060in}}%
\pgfpathlineto{\pgfqpoint{3.454979in}{1.083157in}}%
\pgfpathlineto{\pgfqpoint{3.466508in}{1.086919in}}%
\pgfpathlineto{\pgfqpoint{3.478049in}{1.092161in}}%
\pgfpathlineto{\pgfqpoint{3.489596in}{1.098481in}}%
\pgfpathlineto{\pgfqpoint{3.483124in}{1.111947in}}%
\pgfpathlineto{\pgfqpoint{3.476659in}{1.125887in}}%
\pgfpathlineto{\pgfqpoint{3.470202in}{1.140257in}}%
\pgfpathlineto{\pgfqpoint{3.463752in}{1.155015in}}%
\pgfpathlineto{\pgfqpoint{3.457308in}{1.170117in}}%
\pgfpathlineto{\pgfqpoint{3.445753in}{1.162543in}}%
\pgfpathlineto{\pgfqpoint{3.434206in}{1.156186in}}%
\pgfpathlineto{\pgfqpoint{3.422672in}{1.151474in}}%
\pgfpathlineto{\pgfqpoint{3.411149in}{1.148618in}}%
\pgfpathlineto{\pgfqpoint{3.399635in}{1.147517in}}%
\pgfpathlineto{\pgfqpoint{3.406082in}{1.132970in}}%
\pgfpathlineto{\pgfqpoint{3.412535in}{1.118934in}}%
\pgfpathlineto{\pgfqpoint{3.418997in}{1.105470in}}%
\pgfpathlineto{\pgfqpoint{3.425468in}{1.092633in}}%
\pgfpathclose%
\pgfusepath{stroke,fill}%
\end{pgfscope}%
\begin{pgfscope}%
\pgfpathrectangle{\pgfqpoint{0.887500in}{0.275000in}}{\pgfqpoint{4.225000in}{4.225000in}}%
\pgfusepath{clip}%
\pgfsetbuttcap%
\pgfsetroundjoin%
\definecolor{currentfill}{rgb}{0.243113,0.292092,0.538516}%
\pgfsetfillcolor{currentfill}%
\pgfsetfillopacity{0.700000}%
\pgfsetlinewidth{0.501875pt}%
\definecolor{currentstroke}{rgb}{1.000000,1.000000,1.000000}%
\pgfsetstrokecolor{currentstroke}%
\pgfsetstrokeopacity{0.500000}%
\pgfsetdash{}{0pt}%
\pgfpathmoveto{\pgfqpoint{2.678971in}{1.541121in}}%
\pgfpathlineto{\pgfqpoint{2.690701in}{1.545214in}}%
\pgfpathlineto{\pgfqpoint{2.702426in}{1.549305in}}%
\pgfpathlineto{\pgfqpoint{2.714145in}{1.553393in}}%
\pgfpathlineto{\pgfqpoint{2.725859in}{1.557475in}}%
\pgfpathlineto{\pgfqpoint{2.737566in}{1.561550in}}%
\pgfpathlineto{\pgfqpoint{2.731276in}{1.573227in}}%
\pgfpathlineto{\pgfqpoint{2.724990in}{1.584842in}}%
\pgfpathlineto{\pgfqpoint{2.718708in}{1.596398in}}%
\pgfpathlineto{\pgfqpoint{2.712430in}{1.607899in}}%
\pgfpathlineto{\pgfqpoint{2.706156in}{1.619346in}}%
\pgfpathlineto{\pgfqpoint{2.694456in}{1.615288in}}%
\pgfpathlineto{\pgfqpoint{2.682751in}{1.611220in}}%
\pgfpathlineto{\pgfqpoint{2.671041in}{1.607145in}}%
\pgfpathlineto{\pgfqpoint{2.659324in}{1.603065in}}%
\pgfpathlineto{\pgfqpoint{2.647602in}{1.598980in}}%
\pgfpathlineto{\pgfqpoint{2.653868in}{1.587511in}}%
\pgfpathlineto{\pgfqpoint{2.660138in}{1.575993in}}%
\pgfpathlineto{\pgfqpoint{2.666412in}{1.564424in}}%
\pgfpathlineto{\pgfqpoint{2.672690in}{1.552801in}}%
\pgfpathclose%
\pgfusepath{stroke,fill}%
\end{pgfscope}%
\begin{pgfscope}%
\pgfpathrectangle{\pgfqpoint{0.887500in}{0.275000in}}{\pgfqpoint{4.225000in}{4.225000in}}%
\pgfusepath{clip}%
\pgfsetbuttcap%
\pgfsetroundjoin%
\definecolor{currentfill}{rgb}{0.252194,0.269783,0.531579}%
\pgfsetfillcolor{currentfill}%
\pgfsetfillopacity{0.700000}%
\pgfsetlinewidth{0.501875pt}%
\definecolor{currentstroke}{rgb}{1.000000,1.000000,1.000000}%
\pgfsetstrokecolor{currentstroke}%
\pgfsetstrokeopacity{0.500000}%
\pgfsetdash{}{0pt}%
\pgfpathmoveto{\pgfqpoint{2.769073in}{1.502129in}}%
\pgfpathlineto{\pgfqpoint{2.780784in}{1.506181in}}%
\pgfpathlineto{\pgfqpoint{2.792488in}{1.510228in}}%
\pgfpathlineto{\pgfqpoint{2.804187in}{1.514269in}}%
\pgfpathlineto{\pgfqpoint{2.815880in}{1.518304in}}%
\pgfpathlineto{\pgfqpoint{2.827568in}{1.522334in}}%
\pgfpathlineto{\pgfqpoint{2.821251in}{1.534368in}}%
\pgfpathlineto{\pgfqpoint{2.814938in}{1.546331in}}%
\pgfpathlineto{\pgfqpoint{2.808629in}{1.558215in}}%
\pgfpathlineto{\pgfqpoint{2.802323in}{1.570023in}}%
\pgfpathlineto{\pgfqpoint{2.796021in}{1.581756in}}%
\pgfpathlineto{\pgfqpoint{2.784341in}{1.577740in}}%
\pgfpathlineto{\pgfqpoint{2.772656in}{1.573712in}}%
\pgfpathlineto{\pgfqpoint{2.760965in}{1.569670in}}%
\pgfpathlineto{\pgfqpoint{2.749268in}{1.565615in}}%
\pgfpathlineto{\pgfqpoint{2.737566in}{1.561550in}}%
\pgfpathlineto{\pgfqpoint{2.743860in}{1.549806in}}%
\pgfpathlineto{\pgfqpoint{2.750158in}{1.537994in}}%
\pgfpathlineto{\pgfqpoint{2.756459in}{1.526110in}}%
\pgfpathlineto{\pgfqpoint{2.762764in}{1.514153in}}%
\pgfpathclose%
\pgfusepath{stroke,fill}%
\end{pgfscope}%
\begin{pgfscope}%
\pgfpathrectangle{\pgfqpoint{0.887500in}{0.275000in}}{\pgfqpoint{4.225000in}{4.225000in}}%
\pgfusepath{clip}%
\pgfsetbuttcap%
\pgfsetroundjoin%
\definecolor{currentfill}{rgb}{0.206756,0.371758,0.553117}%
\pgfsetfillcolor{currentfill}%
\pgfsetfillopacity{0.700000}%
\pgfsetlinewidth{0.501875pt}%
\definecolor{currentstroke}{rgb}{1.000000,1.000000,1.000000}%
\pgfsetstrokecolor{currentstroke}%
\pgfsetstrokeopacity{0.500000}%
\pgfsetdash{}{0pt}%
\pgfpathmoveto{\pgfqpoint{2.169900in}{1.699002in}}%
\pgfpathlineto{\pgfqpoint{2.181756in}{1.703034in}}%
\pgfpathlineto{\pgfqpoint{2.193607in}{1.707065in}}%
\pgfpathlineto{\pgfqpoint{2.205452in}{1.711095in}}%
\pgfpathlineto{\pgfqpoint{2.217291in}{1.715125in}}%
\pgfpathlineto{\pgfqpoint{2.229125in}{1.719156in}}%
\pgfpathlineto{\pgfqpoint{2.222991in}{1.729849in}}%
\pgfpathlineto{\pgfqpoint{2.216861in}{1.740503in}}%
\pgfpathlineto{\pgfqpoint{2.210736in}{1.751116in}}%
\pgfpathlineto{\pgfqpoint{2.204615in}{1.761688in}}%
\pgfpathlineto{\pgfqpoint{2.198498in}{1.772221in}}%
\pgfpathlineto{\pgfqpoint{2.186674in}{1.768222in}}%
\pgfpathlineto{\pgfqpoint{2.174845in}{1.764224in}}%
\pgfpathlineto{\pgfqpoint{2.163010in}{1.760227in}}%
\pgfpathlineto{\pgfqpoint{2.151169in}{1.756229in}}%
\pgfpathlineto{\pgfqpoint{2.139322in}{1.752229in}}%
\pgfpathlineto{\pgfqpoint{2.145429in}{1.741662in}}%
\pgfpathlineto{\pgfqpoint{2.151540in}{1.731056in}}%
\pgfpathlineto{\pgfqpoint{2.157656in}{1.720410in}}%
\pgfpathlineto{\pgfqpoint{2.163776in}{1.709726in}}%
\pgfpathclose%
\pgfusepath{stroke,fill}%
\end{pgfscope}%
\begin{pgfscope}%
\pgfpathrectangle{\pgfqpoint{0.887500in}{0.275000in}}{\pgfqpoint{4.225000in}{4.225000in}}%
\pgfusepath{clip}%
\pgfsetbuttcap%
\pgfsetroundjoin%
\definecolor{currentfill}{rgb}{0.260571,0.246922,0.522828}%
\pgfsetfillcolor{currentfill}%
\pgfsetfillopacity{0.700000}%
\pgfsetlinewidth{0.501875pt}%
\definecolor{currentstroke}{rgb}{1.000000,1.000000,1.000000}%
\pgfsetstrokecolor{currentstroke}%
\pgfsetstrokeopacity{0.500000}%
\pgfsetdash{}{0pt}%
\pgfpathmoveto{\pgfqpoint{2.859205in}{1.461495in}}%
\pgfpathlineto{\pgfqpoint{2.870894in}{1.465523in}}%
\pgfpathlineto{\pgfqpoint{2.882578in}{1.469555in}}%
\pgfpathlineto{\pgfqpoint{2.894255in}{1.473591in}}%
\pgfpathlineto{\pgfqpoint{2.905928in}{1.477631in}}%
\pgfpathlineto{\pgfqpoint{2.917594in}{1.481675in}}%
\pgfpathlineto{\pgfqpoint{2.911253in}{1.493908in}}%
\pgfpathlineto{\pgfqpoint{2.904915in}{1.506113in}}%
\pgfpathlineto{\pgfqpoint{2.898580in}{1.518282in}}%
\pgfpathlineto{\pgfqpoint{2.892249in}{1.530406in}}%
\pgfpathlineto{\pgfqpoint{2.885921in}{1.542476in}}%
\pgfpathlineto{\pgfqpoint{2.874262in}{1.538443in}}%
\pgfpathlineto{\pgfqpoint{2.862597in}{1.534414in}}%
\pgfpathlineto{\pgfqpoint{2.850926in}{1.530388in}}%
\pgfpathlineto{\pgfqpoint{2.839250in}{1.526361in}}%
\pgfpathlineto{\pgfqpoint{2.827568in}{1.522334in}}%
\pgfpathlineto{\pgfqpoint{2.833888in}{1.510240in}}%
\pgfpathlineto{\pgfqpoint{2.840212in}{1.498098in}}%
\pgfpathlineto{\pgfqpoint{2.846540in}{1.485919in}}%
\pgfpathlineto{\pgfqpoint{2.852871in}{1.473714in}}%
\pgfpathclose%
\pgfusepath{stroke,fill}%
\end{pgfscope}%
\begin{pgfscope}%
\pgfpathrectangle{\pgfqpoint{0.887500in}{0.275000in}}{\pgfqpoint{4.225000in}{4.225000in}}%
\pgfusepath{clip}%
\pgfsetbuttcap%
\pgfsetroundjoin%
\definecolor{currentfill}{rgb}{0.279574,0.170599,0.479997}%
\pgfsetfillcolor{currentfill}%
\pgfsetfillopacity{0.700000}%
\pgfsetlinewidth{0.501875pt}%
\definecolor{currentstroke}{rgb}{1.000000,1.000000,1.000000}%
\pgfsetstrokecolor{currentstroke}%
\pgfsetstrokeopacity{0.500000}%
\pgfsetdash{}{0pt}%
\pgfpathmoveto{\pgfqpoint{3.129662in}{1.332224in}}%
\pgfpathlineto{\pgfqpoint{3.141286in}{1.336433in}}%
\pgfpathlineto{\pgfqpoint{3.152904in}{1.340571in}}%
\pgfpathlineto{\pgfqpoint{3.164516in}{1.344601in}}%
\pgfpathlineto{\pgfqpoint{3.176122in}{1.348488in}}%
\pgfpathlineto{\pgfqpoint{3.187722in}{1.352224in}}%
\pgfpathlineto{\pgfqpoint{3.181309in}{1.363611in}}%
\pgfpathlineto{\pgfqpoint{3.174900in}{1.375466in}}%
\pgfpathlineto{\pgfqpoint{3.168497in}{1.387740in}}%
\pgfpathlineto{\pgfqpoint{3.162097in}{1.400365in}}%
\pgfpathlineto{\pgfqpoint{3.155702in}{1.413272in}}%
\pgfpathlineto{\pgfqpoint{3.144108in}{1.409110in}}%
\pgfpathlineto{\pgfqpoint{3.132508in}{1.405181in}}%
\pgfpathlineto{\pgfqpoint{3.120904in}{1.401471in}}%
\pgfpathlineto{\pgfqpoint{3.109294in}{1.397929in}}%
\pgfpathlineto{\pgfqpoint{3.097678in}{1.394498in}}%
\pgfpathlineto{\pgfqpoint{3.104068in}{1.381621in}}%
\pgfpathlineto{\pgfqpoint{3.110461in}{1.368910in}}%
\pgfpathlineto{\pgfqpoint{3.116857in}{1.356412in}}%
\pgfpathlineto{\pgfqpoint{3.123258in}{1.344170in}}%
\pgfpathclose%
\pgfusepath{stroke,fill}%
\end{pgfscope}%
\begin{pgfscope}%
\pgfpathrectangle{\pgfqpoint{0.887500in}{0.275000in}}{\pgfqpoint{4.225000in}{4.225000in}}%
\pgfusepath{clip}%
\pgfsetbuttcap%
\pgfsetroundjoin%
\definecolor{currentfill}{rgb}{0.267968,0.223549,0.512008}%
\pgfsetfillcolor{currentfill}%
\pgfsetfillopacity{0.700000}%
\pgfsetlinewidth{0.501875pt}%
\definecolor{currentstroke}{rgb}{1.000000,1.000000,1.000000}%
\pgfsetstrokecolor{currentstroke}%
\pgfsetstrokeopacity{0.500000}%
\pgfsetdash{}{0pt}%
\pgfpathmoveto{\pgfqpoint{2.949350in}{1.420400in}}%
\pgfpathlineto{\pgfqpoint{2.961018in}{1.424382in}}%
\pgfpathlineto{\pgfqpoint{2.972681in}{1.428367in}}%
\pgfpathlineto{\pgfqpoint{2.984337in}{1.432350in}}%
\pgfpathlineto{\pgfqpoint{2.995988in}{1.436326in}}%
\pgfpathlineto{\pgfqpoint{3.007634in}{1.440290in}}%
\pgfpathlineto{\pgfqpoint{3.001269in}{1.452717in}}%
\pgfpathlineto{\pgfqpoint{2.994907in}{1.465110in}}%
\pgfpathlineto{\pgfqpoint{2.988548in}{1.477470in}}%
\pgfpathlineto{\pgfqpoint{2.982193in}{1.489795in}}%
\pgfpathlineto{\pgfqpoint{2.975841in}{1.502079in}}%
\pgfpathlineto{\pgfqpoint{2.964203in}{1.497959in}}%
\pgfpathlineto{\pgfqpoint{2.952559in}{1.493863in}}%
\pgfpathlineto{\pgfqpoint{2.940910in}{1.489787in}}%
\pgfpathlineto{\pgfqpoint{2.929255in}{1.485726in}}%
\pgfpathlineto{\pgfqpoint{2.917594in}{1.481675in}}%
\pgfpathlineto{\pgfqpoint{2.923939in}{1.469425in}}%
\pgfpathlineto{\pgfqpoint{2.930287in}{1.457165in}}%
\pgfpathlineto{\pgfqpoint{2.936638in}{1.444904in}}%
\pgfpathlineto{\pgfqpoint{2.942992in}{1.432652in}}%
\pgfpathclose%
\pgfusepath{stroke,fill}%
\end{pgfscope}%
\begin{pgfscope}%
\pgfpathrectangle{\pgfqpoint{0.887500in}{0.275000in}}{\pgfqpoint{4.225000in}{4.225000in}}%
\pgfusepath{clip}%
\pgfsetbuttcap%
\pgfsetroundjoin%
\definecolor{currentfill}{rgb}{0.282290,0.145912,0.461510}%
\pgfsetfillcolor{currentfill}%
\pgfsetfillopacity{0.700000}%
\pgfsetlinewidth{0.501875pt}%
\definecolor{currentstroke}{rgb}{1.000000,1.000000,1.000000}%
\pgfsetstrokecolor{currentstroke}%
\pgfsetstrokeopacity{0.500000}%
\pgfsetdash{}{0pt}%
\pgfpathmoveto{\pgfqpoint{3.219864in}{1.300342in}}%
\pgfpathlineto{\pgfqpoint{3.231457in}{1.301651in}}%
\pgfpathlineto{\pgfqpoint{3.243038in}{1.302192in}}%
\pgfpathlineto{\pgfqpoint{3.254610in}{1.302243in}}%
\pgfpathlineto{\pgfqpoint{3.266172in}{1.302080in}}%
\pgfpathlineto{\pgfqpoint{3.277726in}{1.301980in}}%
\pgfpathlineto{\pgfqpoint{3.271301in}{1.315626in}}%
\pgfpathlineto{\pgfqpoint{3.264879in}{1.329225in}}%
\pgfpathlineto{\pgfqpoint{3.258459in}{1.342774in}}%
\pgfpathlineto{\pgfqpoint{3.252043in}{1.356269in}}%
\pgfpathlineto{\pgfqpoint{3.245628in}{1.369706in}}%
\pgfpathlineto{\pgfqpoint{3.234059in}{1.366263in}}%
\pgfpathlineto{\pgfqpoint{3.222484in}{1.362825in}}%
\pgfpathlineto{\pgfqpoint{3.210903in}{1.359360in}}%
\pgfpathlineto{\pgfqpoint{3.199316in}{1.355837in}}%
\pgfpathlineto{\pgfqpoint{3.187722in}{1.352224in}}%
\pgfpathlineto{\pgfqpoint{3.194141in}{1.341264in}}%
\pgfpathlineto{\pgfqpoint{3.200564in}{1.330663in}}%
\pgfpathlineto{\pgfqpoint{3.206993in}{1.320354in}}%
\pgfpathlineto{\pgfqpoint{3.213426in}{1.310269in}}%
\pgfpathclose%
\pgfusepath{stroke,fill}%
\end{pgfscope}%
\begin{pgfscope}%
\pgfpathrectangle{\pgfqpoint{0.887500in}{0.275000in}}{\pgfqpoint{4.225000in}{4.225000in}}%
\pgfusepath{clip}%
\pgfsetbuttcap%
\pgfsetroundjoin%
\definecolor{currentfill}{rgb}{0.274128,0.199721,0.498911}%
\pgfsetfillcolor{currentfill}%
\pgfsetfillopacity{0.700000}%
\pgfsetlinewidth{0.501875pt}%
\definecolor{currentstroke}{rgb}{1.000000,1.000000,1.000000}%
\pgfsetstrokecolor{currentstroke}%
\pgfsetstrokeopacity{0.500000}%
\pgfsetdash{}{0pt}%
\pgfpathmoveto{\pgfqpoint{3.039507in}{1.377190in}}%
\pgfpathlineto{\pgfqpoint{3.051153in}{1.380815in}}%
\pgfpathlineto{\pgfqpoint{3.062793in}{1.384336in}}%
\pgfpathlineto{\pgfqpoint{3.074428in}{1.387757in}}%
\pgfpathlineto{\pgfqpoint{3.086056in}{1.391126in}}%
\pgfpathlineto{\pgfqpoint{3.097678in}{1.394498in}}%
\pgfpathlineto{\pgfqpoint{3.091291in}{1.407499in}}%
\pgfpathlineto{\pgfqpoint{3.084907in}{1.420578in}}%
\pgfpathlineto{\pgfqpoint{3.078527in}{1.433690in}}%
\pgfpathlineto{\pgfqpoint{3.072149in}{1.446791in}}%
\pgfpathlineto{\pgfqpoint{3.065774in}{1.459836in}}%
\pgfpathlineto{\pgfqpoint{3.054158in}{1.455948in}}%
\pgfpathlineto{\pgfqpoint{3.042535in}{1.452060in}}%
\pgfpathlineto{\pgfqpoint{3.030907in}{1.448159in}}%
\pgfpathlineto{\pgfqpoint{3.019273in}{1.444236in}}%
\pgfpathlineto{\pgfqpoint{3.007634in}{1.440290in}}%
\pgfpathlineto{\pgfqpoint{3.014002in}{1.427815in}}%
\pgfpathlineto{\pgfqpoint{3.020373in}{1.415280in}}%
\pgfpathlineto{\pgfqpoint{3.026748in}{1.402673in}}%
\pgfpathlineto{\pgfqpoint{3.033126in}{1.389980in}}%
\pgfpathclose%
\pgfusepath{stroke,fill}%
\end{pgfscope}%
\begin{pgfscope}%
\pgfpathrectangle{\pgfqpoint{0.887500in}{0.275000in}}{\pgfqpoint{4.225000in}{4.225000in}}%
\pgfusepath{clip}%
\pgfsetbuttcap%
\pgfsetroundjoin%
\definecolor{currentfill}{rgb}{0.214298,0.355619,0.551184}%
\pgfsetfillcolor{currentfill}%
\pgfsetfillopacity{0.700000}%
\pgfsetlinewidth{0.501875pt}%
\definecolor{currentstroke}{rgb}{1.000000,1.000000,1.000000}%
\pgfsetstrokecolor{currentstroke}%
\pgfsetstrokeopacity{0.500000}%
\pgfsetdash{}{0pt}%
\pgfpathmoveto{\pgfqpoint{2.259859in}{1.665081in}}%
\pgfpathlineto{\pgfqpoint{2.271697in}{1.669137in}}%
\pgfpathlineto{\pgfqpoint{2.283528in}{1.673192in}}%
\pgfpathlineto{\pgfqpoint{2.295354in}{1.677246in}}%
\pgfpathlineto{\pgfqpoint{2.307174in}{1.681298in}}%
\pgfpathlineto{\pgfqpoint{2.318988in}{1.685348in}}%
\pgfpathlineto{\pgfqpoint{2.312823in}{1.696220in}}%
\pgfpathlineto{\pgfqpoint{2.306662in}{1.707051in}}%
\pgfpathlineto{\pgfqpoint{2.300506in}{1.717839in}}%
\pgfpathlineto{\pgfqpoint{2.294353in}{1.728585in}}%
\pgfpathlineto{\pgfqpoint{2.288205in}{1.739290in}}%
\pgfpathlineto{\pgfqpoint{2.276401in}{1.735268in}}%
\pgfpathlineto{\pgfqpoint{2.264590in}{1.731243in}}%
\pgfpathlineto{\pgfqpoint{2.252774in}{1.727215in}}%
\pgfpathlineto{\pgfqpoint{2.240952in}{1.723186in}}%
\pgfpathlineto{\pgfqpoint{2.229125in}{1.719156in}}%
\pgfpathlineto{\pgfqpoint{2.235263in}{1.708422in}}%
\pgfpathlineto{\pgfqpoint{2.241406in}{1.697647in}}%
\pgfpathlineto{\pgfqpoint{2.247553in}{1.686833in}}%
\pgfpathlineto{\pgfqpoint{2.253704in}{1.675977in}}%
\pgfpathclose%
\pgfusepath{stroke,fill}%
\end{pgfscope}%
\begin{pgfscope}%
\pgfpathrectangle{\pgfqpoint{0.887500in}{0.275000in}}{\pgfqpoint{4.225000in}{4.225000in}}%
\pgfusepath{clip}%
\pgfsetbuttcap%
\pgfsetroundjoin%
\definecolor{currentfill}{rgb}{0.220057,0.343307,0.549413}%
\pgfsetfillcolor{currentfill}%
\pgfsetfillopacity{0.700000}%
\pgfsetlinewidth{0.501875pt}%
\definecolor{currentstroke}{rgb}{1.000000,1.000000,1.000000}%
\pgfsetstrokecolor{currentstroke}%
\pgfsetstrokeopacity{0.500000}%
\pgfsetdash{}{0pt}%
\pgfpathmoveto{\pgfqpoint{2.349877in}{1.630353in}}%
\pgfpathlineto{\pgfqpoint{2.361695in}{1.634420in}}%
\pgfpathlineto{\pgfqpoint{2.373507in}{1.638485in}}%
\pgfpathlineto{\pgfqpoint{2.385314in}{1.642547in}}%
\pgfpathlineto{\pgfqpoint{2.397115in}{1.646605in}}%
\pgfpathlineto{\pgfqpoint{2.408910in}{1.650659in}}%
\pgfpathlineto{\pgfqpoint{2.402715in}{1.661721in}}%
\pgfpathlineto{\pgfqpoint{2.396524in}{1.672741in}}%
\pgfpathlineto{\pgfqpoint{2.390336in}{1.683718in}}%
\pgfpathlineto{\pgfqpoint{2.384153in}{1.694650in}}%
\pgfpathlineto{\pgfqpoint{2.377975in}{1.705540in}}%
\pgfpathlineto{\pgfqpoint{2.366189in}{1.701510in}}%
\pgfpathlineto{\pgfqpoint{2.354397in}{1.697476in}}%
\pgfpathlineto{\pgfqpoint{2.342600in}{1.693437in}}%
\pgfpathlineto{\pgfqpoint{2.330797in}{1.689394in}}%
\pgfpathlineto{\pgfqpoint{2.318988in}{1.685348in}}%
\pgfpathlineto{\pgfqpoint{2.325158in}{1.674433in}}%
\pgfpathlineto{\pgfqpoint{2.331331in}{1.663476in}}%
\pgfpathlineto{\pgfqpoint{2.337509in}{1.652476in}}%
\pgfpathlineto{\pgfqpoint{2.343691in}{1.641435in}}%
\pgfpathclose%
\pgfusepath{stroke,fill}%
\end{pgfscope}%
\begin{pgfscope}%
\pgfpathrectangle{\pgfqpoint{0.887500in}{0.275000in}}{\pgfqpoint{4.225000in}{4.225000in}}%
\pgfusepath{clip}%
\pgfsetbuttcap%
\pgfsetroundjoin%
\definecolor{currentfill}{rgb}{0.277941,0.056324,0.381191}%
\pgfsetfillcolor{currentfill}%
\pgfsetfillopacity{0.700000}%
\pgfsetlinewidth{0.501875pt}%
\definecolor{currentstroke}{rgb}{1.000000,1.000000,1.000000}%
\pgfsetstrokecolor{currentstroke}%
\pgfsetstrokeopacity{0.500000}%
\pgfsetdash{}{0pt}%
\pgfpathmoveto{\pgfqpoint{3.342119in}{1.163827in}}%
\pgfpathlineto{\pgfqpoint{3.353623in}{1.158089in}}%
\pgfpathlineto{\pgfqpoint{3.365124in}{1.153456in}}%
\pgfpathlineto{\pgfqpoint{3.376626in}{1.150065in}}%
\pgfpathlineto{\pgfqpoint{3.388128in}{1.148042in}}%
\pgfpathlineto{\pgfqpoint{3.399635in}{1.147517in}}%
\pgfpathlineto{\pgfqpoint{3.393195in}{1.162520in}}%
\pgfpathlineto{\pgfqpoint{3.386761in}{1.177919in}}%
\pgfpathlineto{\pgfqpoint{3.380333in}{1.193656in}}%
\pgfpathlineto{\pgfqpoint{3.373908in}{1.209675in}}%
\pgfpathlineto{\pgfqpoint{3.367488in}{1.225916in}}%
\pgfpathlineto{\pgfqpoint{3.355969in}{1.225148in}}%
\pgfpathlineto{\pgfqpoint{3.344451in}{1.225573in}}%
\pgfpathlineto{\pgfqpoint{3.332932in}{1.227110in}}%
\pgfpathlineto{\pgfqpoint{3.321412in}{1.229679in}}%
\pgfpathlineto{\pgfqpoint{3.309889in}{1.233189in}}%
\pgfpathlineto{\pgfqpoint{3.316329in}{1.219347in}}%
\pgfpathlineto{\pgfqpoint{3.322773in}{1.205486in}}%
\pgfpathlineto{\pgfqpoint{3.329218in}{1.191609in}}%
\pgfpathlineto{\pgfqpoint{3.335667in}{1.177722in}}%
\pgfpathclose%
\pgfusepath{stroke,fill}%
\end{pgfscope}%
\begin{pgfscope}%
\pgfpathrectangle{\pgfqpoint{0.887500in}{0.275000in}}{\pgfqpoint{4.225000in}{4.225000in}}%
\pgfusepath{clip}%
\pgfsetbuttcap%
\pgfsetroundjoin%
\definecolor{currentfill}{rgb}{0.227802,0.326594,0.546532}%
\pgfsetfillcolor{currentfill}%
\pgfsetfillopacity{0.700000}%
\pgfsetlinewidth{0.501875pt}%
\definecolor{currentstroke}{rgb}{1.000000,1.000000,1.000000}%
\pgfsetstrokecolor{currentstroke}%
\pgfsetstrokeopacity{0.500000}%
\pgfsetdash{}{0pt}%
\pgfpathmoveto{\pgfqpoint{2.439949in}{1.594730in}}%
\pgfpathlineto{\pgfqpoint{2.451748in}{1.598798in}}%
\pgfpathlineto{\pgfqpoint{2.463541in}{1.602862in}}%
\pgfpathlineto{\pgfqpoint{2.475328in}{1.606922in}}%
\pgfpathlineto{\pgfqpoint{2.487110in}{1.610979in}}%
\pgfpathlineto{\pgfqpoint{2.498886in}{1.615033in}}%
\pgfpathlineto{\pgfqpoint{2.492661in}{1.626282in}}%
\pgfpathlineto{\pgfqpoint{2.486440in}{1.637490in}}%
\pgfpathlineto{\pgfqpoint{2.480223in}{1.648657in}}%
\pgfpathlineto{\pgfqpoint{2.474011in}{1.659781in}}%
\pgfpathlineto{\pgfqpoint{2.467802in}{1.670862in}}%
\pgfpathlineto{\pgfqpoint{2.456035in}{1.666829in}}%
\pgfpathlineto{\pgfqpoint{2.444262in}{1.662794in}}%
\pgfpathlineto{\pgfqpoint{2.432484in}{1.658754in}}%
\pgfpathlineto{\pgfqpoint{2.420700in}{1.654709in}}%
\pgfpathlineto{\pgfqpoint{2.408910in}{1.650659in}}%
\pgfpathlineto{\pgfqpoint{2.415110in}{1.639555in}}%
\pgfpathlineto{\pgfqpoint{2.421314in}{1.628409in}}%
\pgfpathlineto{\pgfqpoint{2.427522in}{1.617223in}}%
\pgfpathlineto{\pgfqpoint{2.433733in}{1.605996in}}%
\pgfpathclose%
\pgfusepath{stroke,fill}%
\end{pgfscope}%
\begin{pgfscope}%
\pgfpathrectangle{\pgfqpoint{0.887500in}{0.275000in}}{\pgfqpoint{4.225000in}{4.225000in}}%
\pgfusepath{clip}%
\pgfsetbuttcap%
\pgfsetroundjoin%
\definecolor{currentfill}{rgb}{0.237441,0.305202,0.541921}%
\pgfsetfillcolor{currentfill}%
\pgfsetfillopacity{0.700000}%
\pgfsetlinewidth{0.501875pt}%
\definecolor{currentstroke}{rgb}{1.000000,1.000000,1.000000}%
\pgfsetstrokecolor{currentstroke}%
\pgfsetstrokeopacity{0.500000}%
\pgfsetdash{}{0pt}%
\pgfpathmoveto{\pgfqpoint{2.530070in}{1.558188in}}%
\pgfpathlineto{\pgfqpoint{2.541849in}{1.562256in}}%
\pgfpathlineto{\pgfqpoint{2.553622in}{1.566325in}}%
\pgfpathlineto{\pgfqpoint{2.565390in}{1.570396in}}%
\pgfpathlineto{\pgfqpoint{2.577152in}{1.574470in}}%
\pgfpathlineto{\pgfqpoint{2.588908in}{1.578549in}}%
\pgfpathlineto{\pgfqpoint{2.582655in}{1.589986in}}%
\pgfpathlineto{\pgfqpoint{2.576405in}{1.601381in}}%
\pgfpathlineto{\pgfqpoint{2.570160in}{1.612735in}}%
\pgfpathlineto{\pgfqpoint{2.563918in}{1.624048in}}%
\pgfpathlineto{\pgfqpoint{2.557680in}{1.635319in}}%
\pgfpathlineto{\pgfqpoint{2.545933in}{1.631256in}}%
\pgfpathlineto{\pgfqpoint{2.534180in}{1.627197in}}%
\pgfpathlineto{\pgfqpoint{2.522421in}{1.623141in}}%
\pgfpathlineto{\pgfqpoint{2.510656in}{1.619087in}}%
\pgfpathlineto{\pgfqpoint{2.498886in}{1.615033in}}%
\pgfpathlineto{\pgfqpoint{2.505115in}{1.603743in}}%
\pgfpathlineto{\pgfqpoint{2.511348in}{1.592414in}}%
\pgfpathlineto{\pgfqpoint{2.517585in}{1.581045in}}%
\pgfpathlineto{\pgfqpoint{2.523825in}{1.569637in}}%
\pgfpathclose%
\pgfusepath{stroke,fill}%
\end{pgfscope}%
\begin{pgfscope}%
\pgfpathrectangle{\pgfqpoint{0.887500in}{0.275000in}}{\pgfqpoint{4.225000in}{4.225000in}}%
\pgfusepath{clip}%
\pgfsetbuttcap%
\pgfsetroundjoin%
\definecolor{currentfill}{rgb}{0.201239,0.383670,0.554294}%
\pgfsetfillcolor{currentfill}%
\pgfsetfillopacity{0.700000}%
\pgfsetlinewidth{0.501875pt}%
\definecolor{currentstroke}{rgb}{1.000000,1.000000,1.000000}%
\pgfsetstrokecolor{currentstroke}%
\pgfsetstrokeopacity{0.500000}%
\pgfsetdash{}{0pt}%
\pgfpathmoveto{\pgfqpoint{2.020541in}{1.712035in}}%
\pgfpathlineto{\pgfqpoint{2.032444in}{1.716081in}}%
\pgfpathlineto{\pgfqpoint{2.044342in}{1.720119in}}%
\pgfpathlineto{\pgfqpoint{2.056235in}{1.724150in}}%
\pgfpathlineto{\pgfqpoint{2.068121in}{1.728175in}}%
\pgfpathlineto{\pgfqpoint{2.080002in}{1.732195in}}%
\pgfpathlineto{\pgfqpoint{2.073910in}{1.742757in}}%
\pgfpathlineto{\pgfqpoint{2.067822in}{1.753281in}}%
\pgfpathlineto{\pgfqpoint{2.061738in}{1.763766in}}%
\pgfpathlineto{\pgfqpoint{2.055659in}{1.774212in}}%
\pgfpathlineto{\pgfqpoint{2.049584in}{1.784619in}}%
\pgfpathlineto{\pgfqpoint{2.037713in}{1.780635in}}%
\pgfpathlineto{\pgfqpoint{2.025836in}{1.776645in}}%
\pgfpathlineto{\pgfqpoint{2.013954in}{1.772648in}}%
\pgfpathlineto{\pgfqpoint{2.002066in}{1.768644in}}%
\pgfpathlineto{\pgfqpoint{1.990172in}{1.764631in}}%
\pgfpathlineto{\pgfqpoint{1.996237in}{1.754188in}}%
\pgfpathlineto{\pgfqpoint{2.002306in}{1.743707in}}%
\pgfpathlineto{\pgfqpoint{2.008380in}{1.733188in}}%
\pgfpathlineto{\pgfqpoint{2.014458in}{1.722630in}}%
\pgfpathclose%
\pgfusepath{stroke,fill}%
\end{pgfscope}%
\begin{pgfscope}%
\pgfpathrectangle{\pgfqpoint{0.887500in}{0.275000in}}{\pgfqpoint{4.225000in}{4.225000in}}%
\pgfusepath{clip}%
\pgfsetbuttcap%
\pgfsetroundjoin%
\definecolor{currentfill}{rgb}{0.244972,0.287675,0.537260}%
\pgfsetfillcolor{currentfill}%
\pgfsetfillopacity{0.700000}%
\pgfsetlinewidth{0.501875pt}%
\definecolor{currentstroke}{rgb}{1.000000,1.000000,1.000000}%
\pgfsetstrokecolor{currentstroke}%
\pgfsetstrokeopacity{0.500000}%
\pgfsetdash{}{0pt}%
\pgfpathmoveto{\pgfqpoint{2.620234in}{1.520669in}}%
\pgfpathlineto{\pgfqpoint{2.631993in}{1.524754in}}%
\pgfpathlineto{\pgfqpoint{2.643746in}{1.528843in}}%
\pgfpathlineto{\pgfqpoint{2.655493in}{1.532935in}}%
\pgfpathlineto{\pgfqpoint{2.667235in}{1.537028in}}%
\pgfpathlineto{\pgfqpoint{2.678971in}{1.541121in}}%
\pgfpathlineto{\pgfqpoint{2.672690in}{1.552801in}}%
\pgfpathlineto{\pgfqpoint{2.666412in}{1.564424in}}%
\pgfpathlineto{\pgfqpoint{2.660138in}{1.575993in}}%
\pgfpathlineto{\pgfqpoint{2.653868in}{1.587511in}}%
\pgfpathlineto{\pgfqpoint{2.647602in}{1.598980in}}%
\pgfpathlineto{\pgfqpoint{2.635875in}{1.594894in}}%
\pgfpathlineto{\pgfqpoint{2.624142in}{1.590806in}}%
\pgfpathlineto{\pgfqpoint{2.612403in}{1.586718in}}%
\pgfpathlineto{\pgfqpoint{2.600658in}{1.582632in}}%
\pgfpathlineto{\pgfqpoint{2.588908in}{1.578549in}}%
\pgfpathlineto{\pgfqpoint{2.595165in}{1.567068in}}%
\pgfpathlineto{\pgfqpoint{2.601427in}{1.555541in}}%
\pgfpathlineto{\pgfqpoint{2.607692in}{1.543966in}}%
\pgfpathlineto{\pgfqpoint{2.613961in}{1.532343in}}%
\pgfpathclose%
\pgfusepath{stroke,fill}%
\end{pgfscope}%
\begin{pgfscope}%
\pgfpathrectangle{\pgfqpoint{0.887500in}{0.275000in}}{\pgfqpoint{4.225000in}{4.225000in}}%
\pgfusepath{clip}%
\pgfsetbuttcap%
\pgfsetroundjoin%
\definecolor{currentfill}{rgb}{0.252194,0.269783,0.531579}%
\pgfsetfillcolor{currentfill}%
\pgfsetfillopacity{0.700000}%
\pgfsetlinewidth{0.501875pt}%
\definecolor{currentstroke}{rgb}{1.000000,1.000000,1.000000}%
\pgfsetstrokecolor{currentstroke}%
\pgfsetstrokeopacity{0.500000}%
\pgfsetdash{}{0pt}%
\pgfpathmoveto{\pgfqpoint{2.710437in}{1.481815in}}%
\pgfpathlineto{\pgfqpoint{2.722176in}{1.485883in}}%
\pgfpathlineto{\pgfqpoint{2.733909in}{1.489949in}}%
\pgfpathlineto{\pgfqpoint{2.745636in}{1.494013in}}%
\pgfpathlineto{\pgfqpoint{2.757357in}{1.498073in}}%
\pgfpathlineto{\pgfqpoint{2.769073in}{1.502129in}}%
\pgfpathlineto{\pgfqpoint{2.762764in}{1.514153in}}%
\pgfpathlineto{\pgfqpoint{2.756459in}{1.526110in}}%
\pgfpathlineto{\pgfqpoint{2.750158in}{1.537994in}}%
\pgfpathlineto{\pgfqpoint{2.743860in}{1.549806in}}%
\pgfpathlineto{\pgfqpoint{2.737566in}{1.561550in}}%
\pgfpathlineto{\pgfqpoint{2.725859in}{1.557475in}}%
\pgfpathlineto{\pgfqpoint{2.714145in}{1.553393in}}%
\pgfpathlineto{\pgfqpoint{2.702426in}{1.549305in}}%
\pgfpathlineto{\pgfqpoint{2.690701in}{1.545214in}}%
\pgfpathlineto{\pgfqpoint{2.678971in}{1.541121in}}%
\pgfpathlineto{\pgfqpoint{2.685256in}{1.529384in}}%
\pgfpathlineto{\pgfqpoint{2.691546in}{1.517586in}}%
\pgfpathlineto{\pgfqpoint{2.697839in}{1.505724in}}%
\pgfpathlineto{\pgfqpoint{2.704136in}{1.493799in}}%
\pgfpathclose%
\pgfusepath{stroke,fill}%
\end{pgfscope}%
\begin{pgfscope}%
\pgfpathrectangle{\pgfqpoint{0.887500in}{0.275000in}}{\pgfqpoint{4.225000in}{4.225000in}}%
\pgfusepath{clip}%
\pgfsetbuttcap%
\pgfsetroundjoin%
\definecolor{currentfill}{rgb}{0.206756,0.371758,0.553117}%
\pgfsetfillcolor{currentfill}%
\pgfsetfillopacity{0.700000}%
\pgfsetlinewidth{0.501875pt}%
\definecolor{currentstroke}{rgb}{1.000000,1.000000,1.000000}%
\pgfsetstrokecolor{currentstroke}%
\pgfsetstrokeopacity{0.500000}%
\pgfsetdash{}{0pt}%
\pgfpathmoveto{\pgfqpoint{2.110531in}{1.678808in}}%
\pgfpathlineto{\pgfqpoint{2.122416in}{1.682854in}}%
\pgfpathlineto{\pgfqpoint{2.134296in}{1.686895in}}%
\pgfpathlineto{\pgfqpoint{2.146170in}{1.690934in}}%
\pgfpathlineto{\pgfqpoint{2.158038in}{1.694969in}}%
\pgfpathlineto{\pgfqpoint{2.169900in}{1.699002in}}%
\pgfpathlineto{\pgfqpoint{2.163776in}{1.709726in}}%
\pgfpathlineto{\pgfqpoint{2.157656in}{1.720410in}}%
\pgfpathlineto{\pgfqpoint{2.151540in}{1.731056in}}%
\pgfpathlineto{\pgfqpoint{2.145429in}{1.741662in}}%
\pgfpathlineto{\pgfqpoint{2.139322in}{1.752229in}}%
\pgfpathlineto{\pgfqpoint{2.127470in}{1.748229in}}%
\pgfpathlineto{\pgfqpoint{2.115611in}{1.744225in}}%
\pgfpathlineto{\pgfqpoint{2.103747in}{1.740219in}}%
\pgfpathlineto{\pgfqpoint{2.091878in}{1.736209in}}%
\pgfpathlineto{\pgfqpoint{2.080002in}{1.732195in}}%
\pgfpathlineto{\pgfqpoint{2.086099in}{1.721593in}}%
\pgfpathlineto{\pgfqpoint{2.092201in}{1.710954in}}%
\pgfpathlineto{\pgfqpoint{2.098307in}{1.700276in}}%
\pgfpathlineto{\pgfqpoint{2.104417in}{1.689561in}}%
\pgfpathclose%
\pgfusepath{stroke,fill}%
\end{pgfscope}%
\begin{pgfscope}%
\pgfpathrectangle{\pgfqpoint{0.887500in}{0.275000in}}{\pgfqpoint{4.225000in}{4.225000in}}%
\pgfusepath{clip}%
\pgfsetbuttcap%
\pgfsetroundjoin%
\definecolor{currentfill}{rgb}{0.260571,0.246922,0.522828}%
\pgfsetfillcolor{currentfill}%
\pgfsetfillopacity{0.700000}%
\pgfsetlinewidth{0.501875pt}%
\definecolor{currentstroke}{rgb}{1.000000,1.000000,1.000000}%
\pgfsetstrokecolor{currentstroke}%
\pgfsetstrokeopacity{0.500000}%
\pgfsetdash{}{0pt}%
\pgfpathmoveto{\pgfqpoint{2.800672in}{1.441395in}}%
\pgfpathlineto{\pgfqpoint{2.812390in}{1.445411in}}%
\pgfpathlineto{\pgfqpoint{2.824102in}{1.449429in}}%
\pgfpathlineto{\pgfqpoint{2.835809in}{1.453448in}}%
\pgfpathlineto{\pgfqpoint{2.847510in}{1.457470in}}%
\pgfpathlineto{\pgfqpoint{2.859205in}{1.461495in}}%
\pgfpathlineto{\pgfqpoint{2.852871in}{1.473714in}}%
\pgfpathlineto{\pgfqpoint{2.846540in}{1.485919in}}%
\pgfpathlineto{\pgfqpoint{2.840212in}{1.498098in}}%
\pgfpathlineto{\pgfqpoint{2.833888in}{1.510240in}}%
\pgfpathlineto{\pgfqpoint{2.827568in}{1.522334in}}%
\pgfpathlineto{\pgfqpoint{2.815880in}{1.518304in}}%
\pgfpathlineto{\pgfqpoint{2.804187in}{1.514269in}}%
\pgfpathlineto{\pgfqpoint{2.792488in}{1.510228in}}%
\pgfpathlineto{\pgfqpoint{2.780784in}{1.506181in}}%
\pgfpathlineto{\pgfqpoint{2.769073in}{1.502129in}}%
\pgfpathlineto{\pgfqpoint{2.775386in}{1.490050in}}%
\pgfpathlineto{\pgfqpoint{2.781702in}{1.477926in}}%
\pgfpathlineto{\pgfqpoint{2.788022in}{1.465769in}}%
\pgfpathlineto{\pgfqpoint{2.794345in}{1.453588in}}%
\pgfpathclose%
\pgfusepath{stroke,fill}%
\end{pgfscope}%
\begin{pgfscope}%
\pgfpathrectangle{\pgfqpoint{0.887500in}{0.275000in}}{\pgfqpoint{4.225000in}{4.225000in}}%
\pgfusepath{clip}%
\pgfsetbuttcap%
\pgfsetroundjoin%
\definecolor{currentfill}{rgb}{0.279574,0.170599,0.479997}%
\pgfsetfillcolor{currentfill}%
\pgfsetfillopacity{0.700000}%
\pgfsetlinewidth{0.501875pt}%
\definecolor{currentstroke}{rgb}{1.000000,1.000000,1.000000}%
\pgfsetstrokecolor{currentstroke}%
\pgfsetstrokeopacity{0.500000}%
\pgfsetdash{}{0pt}%
\pgfpathmoveto{\pgfqpoint{3.071458in}{1.311347in}}%
\pgfpathlineto{\pgfqpoint{3.083110in}{1.315404in}}%
\pgfpathlineto{\pgfqpoint{3.094756in}{1.319536in}}%
\pgfpathlineto{\pgfqpoint{3.106397in}{1.323739in}}%
\pgfpathlineto{\pgfqpoint{3.118032in}{1.327981in}}%
\pgfpathlineto{\pgfqpoint{3.129662in}{1.332224in}}%
\pgfpathlineto{\pgfqpoint{3.123258in}{1.344170in}}%
\pgfpathlineto{\pgfqpoint{3.116857in}{1.356412in}}%
\pgfpathlineto{\pgfqpoint{3.110461in}{1.368910in}}%
\pgfpathlineto{\pgfqpoint{3.104068in}{1.381621in}}%
\pgfpathlineto{\pgfqpoint{3.097678in}{1.394498in}}%
\pgfpathlineto{\pgfqpoint{3.086056in}{1.391126in}}%
\pgfpathlineto{\pgfqpoint{3.074428in}{1.387757in}}%
\pgfpathlineto{\pgfqpoint{3.062793in}{1.384336in}}%
\pgfpathlineto{\pgfqpoint{3.051153in}{1.380815in}}%
\pgfpathlineto{\pgfqpoint{3.039507in}{1.377190in}}%
\pgfpathlineto{\pgfqpoint{3.045891in}{1.364289in}}%
\pgfpathlineto{\pgfqpoint{3.052279in}{1.351265in}}%
\pgfpathlineto{\pgfqpoint{3.058669in}{1.338106in}}%
\pgfpathlineto{\pgfqpoint{3.065062in}{1.324798in}}%
\pgfpathclose%
\pgfusepath{stroke,fill}%
\end{pgfscope}%
\begin{pgfscope}%
\pgfpathrectangle{\pgfqpoint{0.887500in}{0.275000in}}{\pgfqpoint{4.225000in}{4.225000in}}%
\pgfusepath{clip}%
\pgfsetbuttcap%
\pgfsetroundjoin%
\definecolor{currentfill}{rgb}{0.267968,0.223549,0.512008}%
\pgfsetfillcolor{currentfill}%
\pgfsetfillopacity{0.700000}%
\pgfsetlinewidth{0.501875pt}%
\definecolor{currentstroke}{rgb}{1.000000,1.000000,1.000000}%
\pgfsetstrokecolor{currentstroke}%
\pgfsetstrokeopacity{0.500000}%
\pgfsetdash{}{0pt}%
\pgfpathmoveto{\pgfqpoint{2.890923in}{1.400557in}}%
\pgfpathlineto{\pgfqpoint{2.902620in}{1.404525in}}%
\pgfpathlineto{\pgfqpoint{2.914311in}{1.408490in}}%
\pgfpathlineto{\pgfqpoint{2.925996in}{1.412456in}}%
\pgfpathlineto{\pgfqpoint{2.937676in}{1.416425in}}%
\pgfpathlineto{\pgfqpoint{2.949350in}{1.420400in}}%
\pgfpathlineto{\pgfqpoint{2.942992in}{1.432652in}}%
\pgfpathlineto{\pgfqpoint{2.936638in}{1.444904in}}%
\pgfpathlineto{\pgfqpoint{2.930287in}{1.457165in}}%
\pgfpathlineto{\pgfqpoint{2.923939in}{1.469425in}}%
\pgfpathlineto{\pgfqpoint{2.917594in}{1.481675in}}%
\pgfpathlineto{\pgfqpoint{2.905928in}{1.477631in}}%
\pgfpathlineto{\pgfqpoint{2.894255in}{1.473591in}}%
\pgfpathlineto{\pgfqpoint{2.882578in}{1.469555in}}%
\pgfpathlineto{\pgfqpoint{2.870894in}{1.465523in}}%
\pgfpathlineto{\pgfqpoint{2.859205in}{1.461495in}}%
\pgfpathlineto{\pgfqpoint{2.865542in}{1.449273in}}%
\pgfpathlineto{\pgfqpoint{2.871882in}{1.437059in}}%
\pgfpathlineto{\pgfqpoint{2.878226in}{1.424864in}}%
\pgfpathlineto{\pgfqpoint{2.884573in}{1.412700in}}%
\pgfpathclose%
\pgfusepath{stroke,fill}%
\end{pgfscope}%
\begin{pgfscope}%
\pgfpathrectangle{\pgfqpoint{0.887500in}{0.275000in}}{\pgfqpoint{4.225000in}{4.225000in}}%
\pgfusepath{clip}%
\pgfsetbuttcap%
\pgfsetroundjoin%
\definecolor{currentfill}{rgb}{0.274128,0.199721,0.498911}%
\pgfsetfillcolor{currentfill}%
\pgfsetfillopacity{0.700000}%
\pgfsetlinewidth{0.501875pt}%
\definecolor{currentstroke}{rgb}{1.000000,1.000000,1.000000}%
\pgfsetstrokecolor{currentstroke}%
\pgfsetstrokeopacity{0.500000}%
\pgfsetdash{}{0pt}%
\pgfpathmoveto{\pgfqpoint{2.981189in}{1.358176in}}%
\pgfpathlineto{\pgfqpoint{2.992864in}{1.362032in}}%
\pgfpathlineto{\pgfqpoint{3.004534in}{1.365881in}}%
\pgfpathlineto{\pgfqpoint{3.016197in}{1.369703in}}%
\pgfpathlineto{\pgfqpoint{3.027855in}{1.373479in}}%
\pgfpathlineto{\pgfqpoint{3.039507in}{1.377190in}}%
\pgfpathlineto{\pgfqpoint{3.033126in}{1.389980in}}%
\pgfpathlineto{\pgfqpoint{3.026748in}{1.402673in}}%
\pgfpathlineto{\pgfqpoint{3.020373in}{1.415280in}}%
\pgfpathlineto{\pgfqpoint{3.014002in}{1.427815in}}%
\pgfpathlineto{\pgfqpoint{3.007634in}{1.440290in}}%
\pgfpathlineto{\pgfqpoint{2.995988in}{1.436326in}}%
\pgfpathlineto{\pgfqpoint{2.984337in}{1.432350in}}%
\pgfpathlineto{\pgfqpoint{2.972681in}{1.428367in}}%
\pgfpathlineto{\pgfqpoint{2.961018in}{1.424382in}}%
\pgfpathlineto{\pgfqpoint{2.949350in}{1.420400in}}%
\pgfpathlineto{\pgfqpoint{2.955711in}{1.408121in}}%
\pgfpathlineto{\pgfqpoint{2.962075in}{1.395787in}}%
\pgfpathlineto{\pgfqpoint{2.968443in}{1.383371in}}%
\pgfpathlineto{\pgfqpoint{2.974814in}{1.370843in}}%
\pgfpathclose%
\pgfusepath{stroke,fill}%
\end{pgfscope}%
\begin{pgfscope}%
\pgfpathrectangle{\pgfqpoint{0.887500in}{0.275000in}}{\pgfqpoint{4.225000in}{4.225000in}}%
\pgfusepath{clip}%
\pgfsetbuttcap%
\pgfsetroundjoin%
\definecolor{currentfill}{rgb}{0.214298,0.355619,0.551184}%
\pgfsetfillcolor{currentfill}%
\pgfsetfillopacity{0.700000}%
\pgfsetlinewidth{0.501875pt}%
\definecolor{currentstroke}{rgb}{1.000000,1.000000,1.000000}%
\pgfsetstrokecolor{currentstroke}%
\pgfsetstrokeopacity{0.500000}%
\pgfsetdash{}{0pt}%
\pgfpathmoveto{\pgfqpoint{2.200586in}{1.644805in}}%
\pgfpathlineto{\pgfqpoint{2.212452in}{1.648860in}}%
\pgfpathlineto{\pgfqpoint{2.224313in}{1.652915in}}%
\pgfpathlineto{\pgfqpoint{2.236167in}{1.656970in}}%
\pgfpathlineto{\pgfqpoint{2.248016in}{1.661025in}}%
\pgfpathlineto{\pgfqpoint{2.259859in}{1.665081in}}%
\pgfpathlineto{\pgfqpoint{2.253704in}{1.675977in}}%
\pgfpathlineto{\pgfqpoint{2.247553in}{1.686833in}}%
\pgfpathlineto{\pgfqpoint{2.241406in}{1.697647in}}%
\pgfpathlineto{\pgfqpoint{2.235263in}{1.708422in}}%
\pgfpathlineto{\pgfqpoint{2.229125in}{1.719156in}}%
\pgfpathlineto{\pgfqpoint{2.217291in}{1.715125in}}%
\pgfpathlineto{\pgfqpoint{2.205452in}{1.711095in}}%
\pgfpathlineto{\pgfqpoint{2.193607in}{1.707065in}}%
\pgfpathlineto{\pgfqpoint{2.181756in}{1.703034in}}%
\pgfpathlineto{\pgfqpoint{2.169900in}{1.699002in}}%
\pgfpathlineto{\pgfqpoint{2.176029in}{1.688240in}}%
\pgfpathlineto{\pgfqpoint{2.182162in}{1.677439in}}%
\pgfpathlineto{\pgfqpoint{2.188299in}{1.666599in}}%
\pgfpathlineto{\pgfqpoint{2.194440in}{1.655721in}}%
\pgfpathclose%
\pgfusepath{stroke,fill}%
\end{pgfscope}%
\begin{pgfscope}%
\pgfpathrectangle{\pgfqpoint{0.887500in}{0.275000in}}{\pgfqpoint{4.225000in}{4.225000in}}%
\pgfusepath{clip}%
\pgfsetbuttcap%
\pgfsetroundjoin%
\definecolor{currentfill}{rgb}{0.283229,0.120777,0.440584}%
\pgfsetfillcolor{currentfill}%
\pgfsetfillopacity{0.700000}%
\pgfsetlinewidth{0.501875pt}%
\definecolor{currentstroke}{rgb}{1.000000,1.000000,1.000000}%
\pgfsetstrokecolor{currentstroke}%
\pgfsetstrokeopacity{0.500000}%
\pgfsetdash{}{0pt}%
\pgfpathmoveto{\pgfqpoint{3.252117in}{1.250701in}}%
\pgfpathlineto{\pgfqpoint{3.263701in}{1.248868in}}%
\pgfpathlineto{\pgfqpoint{3.275268in}{1.245698in}}%
\pgfpathlineto{\pgfqpoint{3.286820in}{1.241694in}}%
\pgfpathlineto{\pgfqpoint{3.298359in}{1.237358in}}%
\pgfpathlineto{\pgfqpoint{3.309889in}{1.233189in}}%
\pgfpathlineto{\pgfqpoint{3.303451in}{1.247009in}}%
\pgfpathlineto{\pgfqpoint{3.297016in}{1.260802in}}%
\pgfpathlineto{\pgfqpoint{3.290583in}{1.274564in}}%
\pgfpathlineto{\pgfqpoint{3.284153in}{1.288291in}}%
\pgfpathlineto{\pgfqpoint{3.277726in}{1.301980in}}%
\pgfpathlineto{\pgfqpoint{3.266172in}{1.302080in}}%
\pgfpathlineto{\pgfqpoint{3.254610in}{1.302243in}}%
\pgfpathlineto{\pgfqpoint{3.243038in}{1.302192in}}%
\pgfpathlineto{\pgfqpoint{3.231457in}{1.301651in}}%
\pgfpathlineto{\pgfqpoint{3.219864in}{1.300342in}}%
\pgfpathlineto{\pgfqpoint{3.226307in}{1.290504in}}%
\pgfpathlineto{\pgfqpoint{3.232754in}{1.280689in}}%
\pgfpathlineto{\pgfqpoint{3.239205in}{1.270828in}}%
\pgfpathlineto{\pgfqpoint{3.245659in}{1.260855in}}%
\pgfpathclose%
\pgfusepath{stroke,fill}%
\end{pgfscope}%
\begin{pgfscope}%
\pgfpathrectangle{\pgfqpoint{0.887500in}{0.275000in}}{\pgfqpoint{4.225000in}{4.225000in}}%
\pgfusepath{clip}%
\pgfsetbuttcap%
\pgfsetroundjoin%
\definecolor{currentfill}{rgb}{0.281887,0.150881,0.465405}%
\pgfsetfillcolor{currentfill}%
\pgfsetfillopacity{0.700000}%
\pgfsetlinewidth{0.501875pt}%
\definecolor{currentstroke}{rgb}{1.000000,1.000000,1.000000}%
\pgfsetstrokecolor{currentstroke}%
\pgfsetstrokeopacity{0.500000}%
\pgfsetdash{}{0pt}%
\pgfpathmoveto{\pgfqpoint{3.161746in}{1.277163in}}%
\pgfpathlineto{\pgfqpoint{3.173386in}{1.283626in}}%
\pgfpathlineto{\pgfqpoint{3.185020in}{1.289451in}}%
\pgfpathlineto{\pgfqpoint{3.196645in}{1.294338in}}%
\pgfpathlineto{\pgfqpoint{3.208260in}{1.297989in}}%
\pgfpathlineto{\pgfqpoint{3.219864in}{1.300342in}}%
\pgfpathlineto{\pgfqpoint{3.213426in}{1.310269in}}%
\pgfpathlineto{\pgfqpoint{3.206993in}{1.320354in}}%
\pgfpathlineto{\pgfqpoint{3.200564in}{1.330663in}}%
\pgfpathlineto{\pgfqpoint{3.194141in}{1.341264in}}%
\pgfpathlineto{\pgfqpoint{3.187722in}{1.352224in}}%
\pgfpathlineto{\pgfqpoint{3.176122in}{1.348488in}}%
\pgfpathlineto{\pgfqpoint{3.164516in}{1.344601in}}%
\pgfpathlineto{\pgfqpoint{3.152904in}{1.340571in}}%
\pgfpathlineto{\pgfqpoint{3.141286in}{1.336433in}}%
\pgfpathlineto{\pgfqpoint{3.129662in}{1.332224in}}%
\pgfpathlineto{\pgfqpoint{3.136070in}{1.320585in}}%
\pgfpathlineto{\pgfqpoint{3.142482in}{1.309255in}}%
\pgfpathlineto{\pgfqpoint{3.148899in}{1.298240in}}%
\pgfpathlineto{\pgfqpoint{3.155320in}{1.287541in}}%
\pgfpathclose%
\pgfusepath{stroke,fill}%
\end{pgfscope}%
\begin{pgfscope}%
\pgfpathrectangle{\pgfqpoint{0.887500in}{0.275000in}}{\pgfqpoint{4.225000in}{4.225000in}}%
\pgfusepath{clip}%
\pgfsetbuttcap%
\pgfsetroundjoin%
\definecolor{currentfill}{rgb}{0.269944,0.014625,0.341379}%
\pgfsetfillcolor{currentfill}%
\pgfsetfillopacity{0.700000}%
\pgfsetlinewidth{0.501875pt}%
\definecolor{currentstroke}{rgb}{1.000000,1.000000,1.000000}%
\pgfsetstrokecolor{currentstroke}%
\pgfsetstrokeopacity{0.500000}%
\pgfsetdash{}{0pt}%
\pgfpathmoveto{\pgfqpoint{3.374419in}{1.094381in}}%
\pgfpathlineto{\pgfqpoint{3.385930in}{1.089937in}}%
\pgfpathlineto{\pgfqpoint{3.397436in}{1.086154in}}%
\pgfpathlineto{\pgfqpoint{3.408940in}{1.083203in}}%
\pgfpathlineto{\pgfqpoint{3.420443in}{1.081255in}}%
\pgfpathlineto{\pgfqpoint{3.431949in}{1.080483in}}%
\pgfpathlineto{\pgfqpoint{3.425468in}{1.092633in}}%
\pgfpathlineto{\pgfqpoint{3.418997in}{1.105470in}}%
\pgfpathlineto{\pgfqpoint{3.412535in}{1.118934in}}%
\pgfpathlineto{\pgfqpoint{3.406082in}{1.132970in}}%
\pgfpathlineto{\pgfqpoint{3.399635in}{1.147517in}}%
\pgfpathlineto{\pgfqpoint{3.388128in}{1.148042in}}%
\pgfpathlineto{\pgfqpoint{3.376626in}{1.150065in}}%
\pgfpathlineto{\pgfqpoint{3.365124in}{1.153456in}}%
\pgfpathlineto{\pgfqpoint{3.353623in}{1.158089in}}%
\pgfpathlineto{\pgfqpoint{3.342119in}{1.163827in}}%
\pgfpathlineto{\pgfqpoint{3.348573in}{1.149929in}}%
\pgfpathlineto{\pgfqpoint{3.355030in}{1.136031in}}%
\pgfpathlineto{\pgfqpoint{3.361490in}{1.122138in}}%
\pgfpathlineto{\pgfqpoint{3.367953in}{1.108254in}}%
\pgfpathclose%
\pgfusepath{stroke,fill}%
\end{pgfscope}%
\begin{pgfscope}%
\pgfpathrectangle{\pgfqpoint{0.887500in}{0.275000in}}{\pgfqpoint{4.225000in}{4.225000in}}%
\pgfusepath{clip}%
\pgfsetbuttcap%
\pgfsetroundjoin%
\definecolor{currentfill}{rgb}{0.221989,0.339161,0.548752}%
\pgfsetfillcolor{currentfill}%
\pgfsetfillopacity{0.700000}%
\pgfsetlinewidth{0.501875pt}%
\definecolor{currentstroke}{rgb}{1.000000,1.000000,1.000000}%
\pgfsetstrokecolor{currentstroke}%
\pgfsetstrokeopacity{0.500000}%
\pgfsetdash{}{0pt}%
\pgfpathmoveto{\pgfqpoint{2.290701in}{1.609995in}}%
\pgfpathlineto{\pgfqpoint{2.302547in}{1.614067in}}%
\pgfpathlineto{\pgfqpoint{2.314388in}{1.618139in}}%
\pgfpathlineto{\pgfqpoint{2.326224in}{1.622212in}}%
\pgfpathlineto{\pgfqpoint{2.338053in}{1.626283in}}%
\pgfpathlineto{\pgfqpoint{2.349877in}{1.630353in}}%
\pgfpathlineto{\pgfqpoint{2.343691in}{1.641435in}}%
\pgfpathlineto{\pgfqpoint{2.337509in}{1.652476in}}%
\pgfpathlineto{\pgfqpoint{2.331331in}{1.663476in}}%
\pgfpathlineto{\pgfqpoint{2.325158in}{1.674433in}}%
\pgfpathlineto{\pgfqpoint{2.318988in}{1.685348in}}%
\pgfpathlineto{\pgfqpoint{2.307174in}{1.681298in}}%
\pgfpathlineto{\pgfqpoint{2.295354in}{1.677246in}}%
\pgfpathlineto{\pgfqpoint{2.283528in}{1.673192in}}%
\pgfpathlineto{\pgfqpoint{2.271697in}{1.669137in}}%
\pgfpathlineto{\pgfqpoint{2.259859in}{1.665081in}}%
\pgfpathlineto{\pgfqpoint{2.266019in}{1.654145in}}%
\pgfpathlineto{\pgfqpoint{2.272183in}{1.643167in}}%
\pgfpathlineto{\pgfqpoint{2.278351in}{1.632149in}}%
\pgfpathlineto{\pgfqpoint{2.284524in}{1.621091in}}%
\pgfpathclose%
\pgfusepath{stroke,fill}%
\end{pgfscope}%
\begin{pgfscope}%
\pgfpathrectangle{\pgfqpoint{0.887500in}{0.275000in}}{\pgfqpoint{4.225000in}{4.225000in}}%
\pgfusepath{clip}%
\pgfsetbuttcap%
\pgfsetroundjoin%
\definecolor{currentfill}{rgb}{0.229739,0.322361,0.545706}%
\pgfsetfillcolor{currentfill}%
\pgfsetfillopacity{0.700000}%
\pgfsetlinewidth{0.501875pt}%
\definecolor{currentstroke}{rgb}{1.000000,1.000000,1.000000}%
\pgfsetstrokecolor{currentstroke}%
\pgfsetstrokeopacity{0.500000}%
\pgfsetdash{}{0pt}%
\pgfpathmoveto{\pgfqpoint{2.380870in}{1.574351in}}%
\pgfpathlineto{\pgfqpoint{2.392697in}{1.578431in}}%
\pgfpathlineto{\pgfqpoint{2.404519in}{1.582509in}}%
\pgfpathlineto{\pgfqpoint{2.416335in}{1.586585in}}%
\pgfpathlineto{\pgfqpoint{2.428145in}{1.590659in}}%
\pgfpathlineto{\pgfqpoint{2.439949in}{1.594730in}}%
\pgfpathlineto{\pgfqpoint{2.433733in}{1.605996in}}%
\pgfpathlineto{\pgfqpoint{2.427522in}{1.617223in}}%
\pgfpathlineto{\pgfqpoint{2.421314in}{1.628409in}}%
\pgfpathlineto{\pgfqpoint{2.415110in}{1.639555in}}%
\pgfpathlineto{\pgfqpoint{2.408910in}{1.650659in}}%
\pgfpathlineto{\pgfqpoint{2.397115in}{1.646605in}}%
\pgfpathlineto{\pgfqpoint{2.385314in}{1.642547in}}%
\pgfpathlineto{\pgfqpoint{2.373507in}{1.638485in}}%
\pgfpathlineto{\pgfqpoint{2.361695in}{1.634420in}}%
\pgfpathlineto{\pgfqpoint{2.349877in}{1.630353in}}%
\pgfpathlineto{\pgfqpoint{2.356067in}{1.619230in}}%
\pgfpathlineto{\pgfqpoint{2.362262in}{1.608068in}}%
\pgfpathlineto{\pgfqpoint{2.368460in}{1.596867in}}%
\pgfpathlineto{\pgfqpoint{2.374663in}{1.585628in}}%
\pgfpathclose%
\pgfusepath{stroke,fill}%
\end{pgfscope}%
\begin{pgfscope}%
\pgfpathrectangle{\pgfqpoint{0.887500in}{0.275000in}}{\pgfqpoint{4.225000in}{4.225000in}}%
\pgfusepath{clip}%
\pgfsetbuttcap%
\pgfsetroundjoin%
\definecolor{currentfill}{rgb}{0.237441,0.305202,0.541921}%
\pgfsetfillcolor{currentfill}%
\pgfsetfillopacity{0.700000}%
\pgfsetlinewidth{0.501875pt}%
\definecolor{currentstroke}{rgb}{1.000000,1.000000,1.000000}%
\pgfsetstrokecolor{currentstroke}%
\pgfsetstrokeopacity{0.500000}%
\pgfsetdash{}{0pt}%
\pgfpathmoveto{\pgfqpoint{2.471088in}{1.537829in}}%
\pgfpathlineto{\pgfqpoint{2.482896in}{1.541906in}}%
\pgfpathlineto{\pgfqpoint{2.494698in}{1.545980in}}%
\pgfpathlineto{\pgfqpoint{2.506494in}{1.550052in}}%
\pgfpathlineto{\pgfqpoint{2.518285in}{1.554121in}}%
\pgfpathlineto{\pgfqpoint{2.530070in}{1.558188in}}%
\pgfpathlineto{\pgfqpoint{2.523825in}{1.569637in}}%
\pgfpathlineto{\pgfqpoint{2.517585in}{1.581045in}}%
\pgfpathlineto{\pgfqpoint{2.511348in}{1.592414in}}%
\pgfpathlineto{\pgfqpoint{2.505115in}{1.603743in}}%
\pgfpathlineto{\pgfqpoint{2.498886in}{1.615033in}}%
\pgfpathlineto{\pgfqpoint{2.487110in}{1.610979in}}%
\pgfpathlineto{\pgfqpoint{2.475328in}{1.606922in}}%
\pgfpathlineto{\pgfqpoint{2.463541in}{1.602862in}}%
\pgfpathlineto{\pgfqpoint{2.451748in}{1.598798in}}%
\pgfpathlineto{\pgfqpoint{2.439949in}{1.594730in}}%
\pgfpathlineto{\pgfqpoint{2.446169in}{1.583426in}}%
\pgfpathlineto{\pgfqpoint{2.452393in}{1.572083in}}%
\pgfpathlineto{\pgfqpoint{2.458621in}{1.560702in}}%
\pgfpathlineto{\pgfqpoint{2.464852in}{1.549285in}}%
\pgfpathclose%
\pgfusepath{stroke,fill}%
\end{pgfscope}%
\begin{pgfscope}%
\pgfpathrectangle{\pgfqpoint{0.887500in}{0.275000in}}{\pgfqpoint{4.225000in}{4.225000in}}%
\pgfusepath{clip}%
\pgfsetbuttcap%
\pgfsetroundjoin%
\definecolor{currentfill}{rgb}{0.244972,0.287675,0.537260}%
\pgfsetfillcolor{currentfill}%
\pgfsetfillopacity{0.700000}%
\pgfsetlinewidth{0.501875pt}%
\definecolor{currentstroke}{rgb}{1.000000,1.000000,1.000000}%
\pgfsetstrokecolor{currentstroke}%
\pgfsetstrokeopacity{0.500000}%
\pgfsetdash{}{0pt}%
\pgfpathmoveto{\pgfqpoint{2.561352in}{1.500305in}}%
\pgfpathlineto{\pgfqpoint{2.573140in}{1.504372in}}%
\pgfpathlineto{\pgfqpoint{2.584922in}{1.508441in}}%
\pgfpathlineto{\pgfqpoint{2.596698in}{1.512512in}}%
\pgfpathlineto{\pgfqpoint{2.608469in}{1.516588in}}%
\pgfpathlineto{\pgfqpoint{2.620234in}{1.520669in}}%
\pgfpathlineto{\pgfqpoint{2.613961in}{1.532343in}}%
\pgfpathlineto{\pgfqpoint{2.607692in}{1.543966in}}%
\pgfpathlineto{\pgfqpoint{2.601427in}{1.555541in}}%
\pgfpathlineto{\pgfqpoint{2.595165in}{1.567068in}}%
\pgfpathlineto{\pgfqpoint{2.588908in}{1.578549in}}%
\pgfpathlineto{\pgfqpoint{2.577152in}{1.574470in}}%
\pgfpathlineto{\pgfqpoint{2.565390in}{1.570396in}}%
\pgfpathlineto{\pgfqpoint{2.553622in}{1.566325in}}%
\pgfpathlineto{\pgfqpoint{2.541849in}{1.562256in}}%
\pgfpathlineto{\pgfqpoint{2.530070in}{1.558188in}}%
\pgfpathlineto{\pgfqpoint{2.536318in}{1.546698in}}%
\pgfpathlineto{\pgfqpoint{2.542571in}{1.535166in}}%
\pgfpathlineto{\pgfqpoint{2.548827in}{1.523590in}}%
\pgfpathlineto{\pgfqpoint{2.555088in}{1.511970in}}%
\pgfpathclose%
\pgfusepath{stroke,fill}%
\end{pgfscope}%
\begin{pgfscope}%
\pgfpathrectangle{\pgfqpoint{0.887500in}{0.275000in}}{\pgfqpoint{4.225000in}{4.225000in}}%
\pgfusepath{clip}%
\pgfsetbuttcap%
\pgfsetroundjoin%
\definecolor{currentfill}{rgb}{0.253935,0.265254,0.529983}%
\pgfsetfillcolor{currentfill}%
\pgfsetfillopacity{0.700000}%
\pgfsetlinewidth{0.501875pt}%
\definecolor{currentstroke}{rgb}{1.000000,1.000000,1.000000}%
\pgfsetstrokecolor{currentstroke}%
\pgfsetstrokeopacity{0.500000}%
\pgfsetdash{}{0pt}%
\pgfpathmoveto{\pgfqpoint{2.651657in}{1.461490in}}%
\pgfpathlineto{\pgfqpoint{2.663424in}{1.465549in}}%
\pgfpathlineto{\pgfqpoint{2.675186in}{1.469613in}}%
\pgfpathlineto{\pgfqpoint{2.686942in}{1.473679in}}%
\pgfpathlineto{\pgfqpoint{2.698692in}{1.477747in}}%
\pgfpathlineto{\pgfqpoint{2.710437in}{1.481815in}}%
\pgfpathlineto{\pgfqpoint{2.704136in}{1.493799in}}%
\pgfpathlineto{\pgfqpoint{2.697839in}{1.505724in}}%
\pgfpathlineto{\pgfqpoint{2.691546in}{1.517586in}}%
\pgfpathlineto{\pgfqpoint{2.685256in}{1.529384in}}%
\pgfpathlineto{\pgfqpoint{2.678971in}{1.541121in}}%
\pgfpathlineto{\pgfqpoint{2.667235in}{1.537028in}}%
\pgfpathlineto{\pgfqpoint{2.655493in}{1.532935in}}%
\pgfpathlineto{\pgfqpoint{2.643746in}{1.528843in}}%
\pgfpathlineto{\pgfqpoint{2.631993in}{1.524754in}}%
\pgfpathlineto{\pgfqpoint{2.620234in}{1.520669in}}%
\pgfpathlineto{\pgfqpoint{2.626510in}{1.508942in}}%
\pgfpathlineto{\pgfqpoint{2.632791in}{1.497162in}}%
\pgfpathlineto{\pgfqpoint{2.639076in}{1.485326in}}%
\pgfpathlineto{\pgfqpoint{2.645365in}{1.473434in}}%
\pgfpathclose%
\pgfusepath{stroke,fill}%
\end{pgfscope}%
\begin{pgfscope}%
\pgfpathrectangle{\pgfqpoint{0.887500in}{0.275000in}}{\pgfqpoint{4.225000in}{4.225000in}}%
\pgfusepath{clip}%
\pgfsetbuttcap%
\pgfsetroundjoin%
\definecolor{currentfill}{rgb}{0.208623,0.367752,0.552675}%
\pgfsetfillcolor{currentfill}%
\pgfsetfillopacity{0.700000}%
\pgfsetlinewidth{0.501875pt}%
\definecolor{currentstroke}{rgb}{1.000000,1.000000,1.000000}%
\pgfsetstrokecolor{currentstroke}%
\pgfsetstrokeopacity{0.500000}%
\pgfsetdash{}{0pt}%
\pgfpathmoveto{\pgfqpoint{2.051020in}{1.658500in}}%
\pgfpathlineto{\pgfqpoint{2.062934in}{1.662574in}}%
\pgfpathlineto{\pgfqpoint{2.074842in}{1.666642in}}%
\pgfpathlineto{\pgfqpoint{2.086744in}{1.670703in}}%
\pgfpathlineto{\pgfqpoint{2.098640in}{1.674758in}}%
\pgfpathlineto{\pgfqpoint{2.110531in}{1.678808in}}%
\pgfpathlineto{\pgfqpoint{2.104417in}{1.689561in}}%
\pgfpathlineto{\pgfqpoint{2.098307in}{1.700276in}}%
\pgfpathlineto{\pgfqpoint{2.092201in}{1.710954in}}%
\pgfpathlineto{\pgfqpoint{2.086099in}{1.721593in}}%
\pgfpathlineto{\pgfqpoint{2.080002in}{1.732195in}}%
\pgfpathlineto{\pgfqpoint{2.068121in}{1.728175in}}%
\pgfpathlineto{\pgfqpoint{2.056235in}{1.724150in}}%
\pgfpathlineto{\pgfqpoint{2.044342in}{1.720119in}}%
\pgfpathlineto{\pgfqpoint{2.032444in}{1.716081in}}%
\pgfpathlineto{\pgfqpoint{2.020541in}{1.712035in}}%
\pgfpathlineto{\pgfqpoint{2.026628in}{1.701402in}}%
\pgfpathlineto{\pgfqpoint{2.032719in}{1.690731in}}%
\pgfpathlineto{\pgfqpoint{2.038815in}{1.680023in}}%
\pgfpathlineto{\pgfqpoint{2.044915in}{1.669279in}}%
\pgfpathclose%
\pgfusepath{stroke,fill}%
\end{pgfscope}%
\begin{pgfscope}%
\pgfpathrectangle{\pgfqpoint{0.887500in}{0.275000in}}{\pgfqpoint{4.225000in}{4.225000in}}%
\pgfusepath{clip}%
\pgfsetbuttcap%
\pgfsetroundjoin%
\definecolor{currentfill}{rgb}{0.262138,0.242286,0.520837}%
\pgfsetfillcolor{currentfill}%
\pgfsetfillopacity{0.700000}%
\pgfsetlinewidth{0.501875pt}%
\definecolor{currentstroke}{rgb}{1.000000,1.000000,1.000000}%
\pgfsetstrokecolor{currentstroke}%
\pgfsetstrokeopacity{0.500000}%
\pgfsetdash{}{0pt}%
\pgfpathmoveto{\pgfqpoint{2.741995in}{1.421309in}}%
\pgfpathlineto{\pgfqpoint{2.753742in}{1.425328in}}%
\pgfpathlineto{\pgfqpoint{2.765483in}{1.429346in}}%
\pgfpathlineto{\pgfqpoint{2.777218in}{1.433363in}}%
\pgfpathlineto{\pgfqpoint{2.788948in}{1.437379in}}%
\pgfpathlineto{\pgfqpoint{2.800672in}{1.441395in}}%
\pgfpathlineto{\pgfqpoint{2.794345in}{1.453588in}}%
\pgfpathlineto{\pgfqpoint{2.788022in}{1.465769in}}%
\pgfpathlineto{\pgfqpoint{2.781702in}{1.477926in}}%
\pgfpathlineto{\pgfqpoint{2.775386in}{1.490050in}}%
\pgfpathlineto{\pgfqpoint{2.769073in}{1.502129in}}%
\pgfpathlineto{\pgfqpoint{2.757357in}{1.498073in}}%
\pgfpathlineto{\pgfqpoint{2.745636in}{1.494013in}}%
\pgfpathlineto{\pgfqpoint{2.733909in}{1.489949in}}%
\pgfpathlineto{\pgfqpoint{2.722176in}{1.485883in}}%
\pgfpathlineto{\pgfqpoint{2.710437in}{1.481815in}}%
\pgfpathlineto{\pgfqpoint{2.716741in}{1.469782in}}%
\pgfpathlineto{\pgfqpoint{2.723050in}{1.457706in}}%
\pgfpathlineto{\pgfqpoint{2.729361in}{1.445597in}}%
\pgfpathlineto{\pgfqpoint{2.735676in}{1.433461in}}%
\pgfpathclose%
\pgfusepath{stroke,fill}%
\end{pgfscope}%
\begin{pgfscope}%
\pgfpathrectangle{\pgfqpoint{0.887500in}{0.275000in}}{\pgfqpoint{4.225000in}{4.225000in}}%
\pgfusepath{clip}%
\pgfsetbuttcap%
\pgfsetroundjoin%
\definecolor{currentfill}{rgb}{0.267968,0.223549,0.512008}%
\pgfsetfillcolor{currentfill}%
\pgfsetfillopacity{0.700000}%
\pgfsetlinewidth{0.501875pt}%
\definecolor{currentstroke}{rgb}{1.000000,1.000000,1.000000}%
\pgfsetstrokecolor{currentstroke}%
\pgfsetstrokeopacity{0.500000}%
\pgfsetdash{}{0pt}%
\pgfpathmoveto{\pgfqpoint{2.832352in}{1.380597in}}%
\pgfpathlineto{\pgfqpoint{2.844077in}{1.384605in}}%
\pgfpathlineto{\pgfqpoint{2.855797in}{1.388607in}}%
\pgfpathlineto{\pgfqpoint{2.867511in}{1.392600in}}%
\pgfpathlineto{\pgfqpoint{2.879220in}{1.396583in}}%
\pgfpathlineto{\pgfqpoint{2.890923in}{1.400557in}}%
\pgfpathlineto{\pgfqpoint{2.884573in}{1.412700in}}%
\pgfpathlineto{\pgfqpoint{2.878226in}{1.424864in}}%
\pgfpathlineto{\pgfqpoint{2.871882in}{1.437059in}}%
\pgfpathlineto{\pgfqpoint{2.865542in}{1.449273in}}%
\pgfpathlineto{\pgfqpoint{2.859205in}{1.461495in}}%
\pgfpathlineto{\pgfqpoint{2.847510in}{1.457470in}}%
\pgfpathlineto{\pgfqpoint{2.835809in}{1.453448in}}%
\pgfpathlineto{\pgfqpoint{2.824102in}{1.449429in}}%
\pgfpathlineto{\pgfqpoint{2.812390in}{1.445411in}}%
\pgfpathlineto{\pgfqpoint{2.800672in}{1.441395in}}%
\pgfpathlineto{\pgfqpoint{2.807001in}{1.429200in}}%
\pgfpathlineto{\pgfqpoint{2.813334in}{1.417014in}}%
\pgfpathlineto{\pgfqpoint{2.819670in}{1.404847in}}%
\pgfpathlineto{\pgfqpoint{2.826009in}{1.392711in}}%
\pgfpathclose%
\pgfusepath{stroke,fill}%
\end{pgfscope}%
\begin{pgfscope}%
\pgfpathrectangle{\pgfqpoint{0.887500in}{0.275000in}}{\pgfqpoint{4.225000in}{4.225000in}}%
\pgfusepath{clip}%
\pgfsetbuttcap%
\pgfsetroundjoin%
\definecolor{currentfill}{rgb}{0.282623,0.140926,0.457517}%
\pgfsetfillcolor{currentfill}%
\pgfsetfillopacity{0.700000}%
\pgfsetlinewidth{0.501875pt}%
\definecolor{currentstroke}{rgb}{1.000000,1.000000,1.000000}%
\pgfsetstrokecolor{currentstroke}%
\pgfsetstrokeopacity{0.500000}%
\pgfsetdash{}{0pt}%
\pgfpathmoveto{\pgfqpoint{3.103484in}{1.245481in}}%
\pgfpathlineto{\pgfqpoint{3.115144in}{1.250940in}}%
\pgfpathlineto{\pgfqpoint{3.126800in}{1.256966in}}%
\pgfpathlineto{\pgfqpoint{3.138452in}{1.263531in}}%
\pgfpathlineto{\pgfqpoint{3.150101in}{1.270364in}}%
\pgfpathlineto{\pgfqpoint{3.161746in}{1.277163in}}%
\pgfpathlineto{\pgfqpoint{3.155320in}{1.287541in}}%
\pgfpathlineto{\pgfqpoint{3.148899in}{1.298240in}}%
\pgfpathlineto{\pgfqpoint{3.142482in}{1.309255in}}%
\pgfpathlineto{\pgfqpoint{3.136070in}{1.320585in}}%
\pgfpathlineto{\pgfqpoint{3.129662in}{1.332224in}}%
\pgfpathlineto{\pgfqpoint{3.118032in}{1.327981in}}%
\pgfpathlineto{\pgfqpoint{3.106397in}{1.323739in}}%
\pgfpathlineto{\pgfqpoint{3.094756in}{1.319536in}}%
\pgfpathlineto{\pgfqpoint{3.083110in}{1.315404in}}%
\pgfpathlineto{\pgfqpoint{3.071458in}{1.311347in}}%
\pgfpathlineto{\pgfqpoint{3.077857in}{1.297840in}}%
\pgfpathlineto{\pgfqpoint{3.084259in}{1.284389in}}%
\pgfpathlineto{\pgfqpoint{3.090664in}{1.271104in}}%
\pgfpathlineto{\pgfqpoint{3.097072in}{1.258098in}}%
\pgfpathclose%
\pgfusepath{stroke,fill}%
\end{pgfscope}%
\begin{pgfscope}%
\pgfpathrectangle{\pgfqpoint{0.887500in}{0.275000in}}{\pgfqpoint{4.225000in}{4.225000in}}%
\pgfusepath{clip}%
\pgfsetbuttcap%
\pgfsetroundjoin%
\definecolor{currentfill}{rgb}{0.279574,0.170599,0.479997}%
\pgfsetfillcolor{currentfill}%
\pgfsetfillopacity{0.700000}%
\pgfsetlinewidth{0.501875pt}%
\definecolor{currentstroke}{rgb}{1.000000,1.000000,1.000000}%
\pgfsetstrokecolor{currentstroke}%
\pgfsetstrokeopacity{0.500000}%
\pgfsetdash{}{0pt}%
\pgfpathmoveto{\pgfqpoint{3.013112in}{1.291785in}}%
\pgfpathlineto{\pgfqpoint{3.024793in}{1.295639in}}%
\pgfpathlineto{\pgfqpoint{3.036468in}{1.299511in}}%
\pgfpathlineto{\pgfqpoint{3.048137in}{1.303412in}}%
\pgfpathlineto{\pgfqpoint{3.059801in}{1.307353in}}%
\pgfpathlineto{\pgfqpoint{3.071458in}{1.311347in}}%
\pgfpathlineto{\pgfqpoint{3.065062in}{1.324798in}}%
\pgfpathlineto{\pgfqpoint{3.058669in}{1.338106in}}%
\pgfpathlineto{\pgfqpoint{3.052279in}{1.351265in}}%
\pgfpathlineto{\pgfqpoint{3.045891in}{1.364289in}}%
\pgfpathlineto{\pgfqpoint{3.039507in}{1.377190in}}%
\pgfpathlineto{\pgfqpoint{3.027855in}{1.373479in}}%
\pgfpathlineto{\pgfqpoint{3.016197in}{1.369703in}}%
\pgfpathlineto{\pgfqpoint{3.004534in}{1.365881in}}%
\pgfpathlineto{\pgfqpoint{2.992864in}{1.362032in}}%
\pgfpathlineto{\pgfqpoint{2.981189in}{1.358176in}}%
\pgfpathlineto{\pgfqpoint{2.987567in}{1.345341in}}%
\pgfpathlineto{\pgfqpoint{2.993948in}{1.332310in}}%
\pgfpathlineto{\pgfqpoint{3.000333in}{1.319056in}}%
\pgfpathlineto{\pgfqpoint{3.006721in}{1.305549in}}%
\pgfpathclose%
\pgfusepath{stroke,fill}%
\end{pgfscope}%
\begin{pgfscope}%
\pgfpathrectangle{\pgfqpoint{0.887500in}{0.275000in}}{\pgfqpoint{4.225000in}{4.225000in}}%
\pgfusepath{clip}%
\pgfsetbuttcap%
\pgfsetroundjoin%
\definecolor{currentfill}{rgb}{0.274128,0.199721,0.498911}%
\pgfsetfillcolor{currentfill}%
\pgfsetfillopacity{0.700000}%
\pgfsetlinewidth{0.501875pt}%
\definecolor{currentstroke}{rgb}{1.000000,1.000000,1.000000}%
\pgfsetstrokecolor{currentstroke}%
\pgfsetstrokeopacity{0.500000}%
\pgfsetdash{}{0pt}%
\pgfpathmoveto{\pgfqpoint{2.922724in}{1.338949in}}%
\pgfpathlineto{\pgfqpoint{2.934429in}{1.342806in}}%
\pgfpathlineto{\pgfqpoint{2.946128in}{1.346651in}}%
\pgfpathlineto{\pgfqpoint{2.957821in}{1.350489in}}%
\pgfpathlineto{\pgfqpoint{2.969508in}{1.354328in}}%
\pgfpathlineto{\pgfqpoint{2.981189in}{1.358176in}}%
\pgfpathlineto{\pgfqpoint{2.974814in}{1.370843in}}%
\pgfpathlineto{\pgfqpoint{2.968443in}{1.383371in}}%
\pgfpathlineto{\pgfqpoint{2.962075in}{1.395787in}}%
\pgfpathlineto{\pgfqpoint{2.955711in}{1.408121in}}%
\pgfpathlineto{\pgfqpoint{2.949350in}{1.420400in}}%
\pgfpathlineto{\pgfqpoint{2.937676in}{1.416425in}}%
\pgfpathlineto{\pgfqpoint{2.925996in}{1.412456in}}%
\pgfpathlineto{\pgfqpoint{2.914311in}{1.408490in}}%
\pgfpathlineto{\pgfqpoint{2.902620in}{1.404525in}}%
\pgfpathlineto{\pgfqpoint{2.890923in}{1.400557in}}%
\pgfpathlineto{\pgfqpoint{2.897276in}{1.388401in}}%
\pgfpathlineto{\pgfqpoint{2.903633in}{1.376198in}}%
\pgfpathlineto{\pgfqpoint{2.909993in}{1.363911in}}%
\pgfpathlineto{\pgfqpoint{2.916357in}{1.351507in}}%
\pgfpathclose%
\pgfusepath{stroke,fill}%
\end{pgfscope}%
\begin{pgfscope}%
\pgfpathrectangle{\pgfqpoint{0.887500in}{0.275000in}}{\pgfqpoint{4.225000in}{4.225000in}}%
\pgfusepath{clip}%
\pgfsetbuttcap%
\pgfsetroundjoin%
\definecolor{currentfill}{rgb}{0.214298,0.355619,0.551184}%
\pgfsetfillcolor{currentfill}%
\pgfsetfillopacity{0.700000}%
\pgfsetlinewidth{0.501875pt}%
\definecolor{currentstroke}{rgb}{1.000000,1.000000,1.000000}%
\pgfsetstrokecolor{currentstroke}%
\pgfsetstrokeopacity{0.500000}%
\pgfsetdash{}{0pt}%
\pgfpathmoveto{\pgfqpoint{2.141169in}{1.624499in}}%
\pgfpathlineto{\pgfqpoint{2.153064in}{1.628566in}}%
\pgfpathlineto{\pgfqpoint{2.164953in}{1.632629in}}%
\pgfpathlineto{\pgfqpoint{2.176836in}{1.636690in}}%
\pgfpathlineto{\pgfqpoint{2.188714in}{1.640748in}}%
\pgfpathlineto{\pgfqpoint{2.200586in}{1.644805in}}%
\pgfpathlineto{\pgfqpoint{2.194440in}{1.655721in}}%
\pgfpathlineto{\pgfqpoint{2.188299in}{1.666599in}}%
\pgfpathlineto{\pgfqpoint{2.182162in}{1.677439in}}%
\pgfpathlineto{\pgfqpoint{2.176029in}{1.688240in}}%
\pgfpathlineto{\pgfqpoint{2.169900in}{1.699002in}}%
\pgfpathlineto{\pgfqpoint{2.158038in}{1.694969in}}%
\pgfpathlineto{\pgfqpoint{2.146170in}{1.690934in}}%
\pgfpathlineto{\pgfqpoint{2.134296in}{1.686895in}}%
\pgfpathlineto{\pgfqpoint{2.122416in}{1.682854in}}%
\pgfpathlineto{\pgfqpoint{2.110531in}{1.678808in}}%
\pgfpathlineto{\pgfqpoint{2.116650in}{1.668019in}}%
\pgfpathlineto{\pgfqpoint{2.122773in}{1.657193in}}%
\pgfpathlineto{\pgfqpoint{2.128901in}{1.646330in}}%
\pgfpathlineto{\pgfqpoint{2.135033in}{1.635433in}}%
\pgfpathclose%
\pgfusepath{stroke,fill}%
\end{pgfscope}%
\begin{pgfscope}%
\pgfpathrectangle{\pgfqpoint{0.887500in}{0.275000in}}{\pgfqpoint{4.225000in}{4.225000in}}%
\pgfusepath{clip}%
\pgfsetbuttcap%
\pgfsetroundjoin%
\definecolor{currentfill}{rgb}{0.282327,0.094955,0.417331}%
\pgfsetfillcolor{currentfill}%
\pgfsetfillopacity{0.700000}%
\pgfsetlinewidth{0.501875pt}%
\definecolor{currentstroke}{rgb}{1.000000,1.000000,1.000000}%
\pgfsetstrokecolor{currentstroke}%
\pgfsetstrokeopacity{0.500000}%
\pgfsetdash{}{0pt}%
\pgfpathmoveto{\pgfqpoint{3.284437in}{1.194865in}}%
\pgfpathlineto{\pgfqpoint{3.296005in}{1.189972in}}%
\pgfpathlineto{\pgfqpoint{3.307555in}{1.183930in}}%
\pgfpathlineto{\pgfqpoint{3.319089in}{1.177232in}}%
\pgfpathlineto{\pgfqpoint{3.330609in}{1.170368in}}%
\pgfpathlineto{\pgfqpoint{3.342119in}{1.163827in}}%
\pgfpathlineto{\pgfqpoint{3.335667in}{1.177722in}}%
\pgfpathlineto{\pgfqpoint{3.329218in}{1.191609in}}%
\pgfpathlineto{\pgfqpoint{3.322773in}{1.205486in}}%
\pgfpathlineto{\pgfqpoint{3.316329in}{1.219347in}}%
\pgfpathlineto{\pgfqpoint{3.309889in}{1.233189in}}%
\pgfpathlineto{\pgfqpoint{3.298359in}{1.237358in}}%
\pgfpathlineto{\pgfqpoint{3.286820in}{1.241694in}}%
\pgfpathlineto{\pgfqpoint{3.275268in}{1.245698in}}%
\pgfpathlineto{\pgfqpoint{3.263701in}{1.248868in}}%
\pgfpathlineto{\pgfqpoint{3.252117in}{1.250701in}}%
\pgfpathlineto{\pgfqpoint{3.258578in}{1.240300in}}%
\pgfpathlineto{\pgfqpoint{3.265041in}{1.229583in}}%
\pgfpathlineto{\pgfqpoint{3.271505in}{1.218483in}}%
\pgfpathlineto{\pgfqpoint{3.277971in}{1.206933in}}%
\pgfpathclose%
\pgfusepath{stroke,fill}%
\end{pgfscope}%
\begin{pgfscope}%
\pgfpathrectangle{\pgfqpoint{0.887500in}{0.275000in}}{\pgfqpoint{4.225000in}{4.225000in}}%
\pgfusepath{clip}%
\pgfsetbuttcap%
\pgfsetroundjoin%
\definecolor{currentfill}{rgb}{0.221989,0.339161,0.548752}%
\pgfsetfillcolor{currentfill}%
\pgfsetfillopacity{0.700000}%
\pgfsetlinewidth{0.501875pt}%
\definecolor{currentstroke}{rgb}{1.000000,1.000000,1.000000}%
\pgfsetstrokecolor{currentstroke}%
\pgfsetstrokeopacity{0.500000}%
\pgfsetdash{}{0pt}%
\pgfpathmoveto{\pgfqpoint{2.231379in}{1.589654in}}%
\pgfpathlineto{\pgfqpoint{2.243255in}{1.593720in}}%
\pgfpathlineto{\pgfqpoint{2.255125in}{1.597786in}}%
\pgfpathlineto{\pgfqpoint{2.266989in}{1.601854in}}%
\pgfpathlineto{\pgfqpoint{2.278848in}{1.605923in}}%
\pgfpathlineto{\pgfqpoint{2.290701in}{1.609995in}}%
\pgfpathlineto{\pgfqpoint{2.284524in}{1.621091in}}%
\pgfpathlineto{\pgfqpoint{2.278351in}{1.632149in}}%
\pgfpathlineto{\pgfqpoint{2.272183in}{1.643167in}}%
\pgfpathlineto{\pgfqpoint{2.266019in}{1.654145in}}%
\pgfpathlineto{\pgfqpoint{2.259859in}{1.665081in}}%
\pgfpathlineto{\pgfqpoint{2.248016in}{1.661025in}}%
\pgfpathlineto{\pgfqpoint{2.236167in}{1.656970in}}%
\pgfpathlineto{\pgfqpoint{2.224313in}{1.652915in}}%
\pgfpathlineto{\pgfqpoint{2.212452in}{1.648860in}}%
\pgfpathlineto{\pgfqpoint{2.200586in}{1.644805in}}%
\pgfpathlineto{\pgfqpoint{2.206736in}{1.633850in}}%
\pgfpathlineto{\pgfqpoint{2.212891in}{1.622857in}}%
\pgfpathlineto{\pgfqpoint{2.219049in}{1.611827in}}%
\pgfpathlineto{\pgfqpoint{2.225212in}{1.600758in}}%
\pgfpathclose%
\pgfusepath{stroke,fill}%
\end{pgfscope}%
\begin{pgfscope}%
\pgfpathrectangle{\pgfqpoint{0.887500in}{0.275000in}}{\pgfqpoint{4.225000in}{4.225000in}}%
\pgfusepath{clip}%
\pgfsetbuttcap%
\pgfsetroundjoin%
\definecolor{currentfill}{rgb}{0.283072,0.130895,0.449241}%
\pgfsetfillcolor{currentfill}%
\pgfsetfillopacity{0.700000}%
\pgfsetlinewidth{0.501875pt}%
\definecolor{currentstroke}{rgb}{1.000000,1.000000,1.000000}%
\pgfsetstrokecolor{currentstroke}%
\pgfsetstrokeopacity{0.500000}%
\pgfsetdash{}{0pt}%
\pgfpathmoveto{\pgfqpoint{3.193953in}{1.230201in}}%
\pgfpathlineto{\pgfqpoint{3.205610in}{1.237636in}}%
\pgfpathlineto{\pgfqpoint{3.217259in}{1.243868in}}%
\pgfpathlineto{\pgfqpoint{3.228895in}{1.248387in}}%
\pgfpathlineto{\pgfqpoint{3.240515in}{1.250695in}}%
\pgfpathlineto{\pgfqpoint{3.252117in}{1.250701in}}%
\pgfpathlineto{\pgfqpoint{3.245659in}{1.260855in}}%
\pgfpathlineto{\pgfqpoint{3.239205in}{1.270828in}}%
\pgfpathlineto{\pgfqpoint{3.232754in}{1.280689in}}%
\pgfpathlineto{\pgfqpoint{3.226307in}{1.290504in}}%
\pgfpathlineto{\pgfqpoint{3.219864in}{1.300342in}}%
\pgfpathlineto{\pgfqpoint{3.208260in}{1.297989in}}%
\pgfpathlineto{\pgfqpoint{3.196645in}{1.294338in}}%
\pgfpathlineto{\pgfqpoint{3.185020in}{1.289451in}}%
\pgfpathlineto{\pgfqpoint{3.173386in}{1.283626in}}%
\pgfpathlineto{\pgfqpoint{3.161746in}{1.277163in}}%
\pgfpathlineto{\pgfqpoint{3.168177in}{1.267109in}}%
\pgfpathlineto{\pgfqpoint{3.174613in}{1.257382in}}%
\pgfpathlineto{\pgfqpoint{3.181054in}{1.247986in}}%
\pgfpathlineto{\pgfqpoint{3.187501in}{1.238924in}}%
\pgfpathclose%
\pgfusepath{stroke,fill}%
\end{pgfscope}%
\begin{pgfscope}%
\pgfpathrectangle{\pgfqpoint{0.887500in}{0.275000in}}{\pgfqpoint{4.225000in}{4.225000in}}%
\pgfusepath{clip}%
\pgfsetbuttcap%
\pgfsetroundjoin%
\definecolor{currentfill}{rgb}{0.229739,0.322361,0.545706}%
\pgfsetfillcolor{currentfill}%
\pgfsetfillopacity{0.700000}%
\pgfsetlinewidth{0.501875pt}%
\definecolor{currentstroke}{rgb}{1.000000,1.000000,1.000000}%
\pgfsetstrokecolor{currentstroke}%
\pgfsetstrokeopacity{0.500000}%
\pgfsetdash{}{0pt}%
\pgfpathmoveto{\pgfqpoint{2.321646in}{1.553953in}}%
\pgfpathlineto{\pgfqpoint{2.333502in}{1.558030in}}%
\pgfpathlineto{\pgfqpoint{2.345353in}{1.562109in}}%
\pgfpathlineto{\pgfqpoint{2.357197in}{1.566190in}}%
\pgfpathlineto{\pgfqpoint{2.369036in}{1.570270in}}%
\pgfpathlineto{\pgfqpoint{2.380870in}{1.574351in}}%
\pgfpathlineto{\pgfqpoint{2.374663in}{1.585628in}}%
\pgfpathlineto{\pgfqpoint{2.368460in}{1.596867in}}%
\pgfpathlineto{\pgfqpoint{2.362262in}{1.608068in}}%
\pgfpathlineto{\pgfqpoint{2.356067in}{1.619230in}}%
\pgfpathlineto{\pgfqpoint{2.349877in}{1.630353in}}%
\pgfpathlineto{\pgfqpoint{2.338053in}{1.626283in}}%
\pgfpathlineto{\pgfqpoint{2.326224in}{1.622212in}}%
\pgfpathlineto{\pgfqpoint{2.314388in}{1.618139in}}%
\pgfpathlineto{\pgfqpoint{2.302547in}{1.614067in}}%
\pgfpathlineto{\pgfqpoint{2.290701in}{1.609995in}}%
\pgfpathlineto{\pgfqpoint{2.296881in}{1.598859in}}%
\pgfpathlineto{\pgfqpoint{2.303066in}{1.587687in}}%
\pgfpathlineto{\pgfqpoint{2.309255in}{1.576478in}}%
\pgfpathlineto{\pgfqpoint{2.315449in}{1.565233in}}%
\pgfpathclose%
\pgfusepath{stroke,fill}%
\end{pgfscope}%
\begin{pgfscope}%
\pgfpathrectangle{\pgfqpoint{0.887500in}{0.275000in}}{\pgfqpoint{4.225000in}{4.225000in}}%
\pgfusepath{clip}%
\pgfsetbuttcap%
\pgfsetroundjoin%
\definecolor{currentfill}{rgb}{0.237441,0.305202,0.541921}%
\pgfsetfillcolor{currentfill}%
\pgfsetfillopacity{0.700000}%
\pgfsetlinewidth{0.501875pt}%
\definecolor{currentstroke}{rgb}{1.000000,1.000000,1.000000}%
\pgfsetstrokecolor{currentstroke}%
\pgfsetstrokeopacity{0.500000}%
\pgfsetdash{}{0pt}%
\pgfpathmoveto{\pgfqpoint{2.411963in}{1.517431in}}%
\pgfpathlineto{\pgfqpoint{2.423799in}{1.521511in}}%
\pgfpathlineto{\pgfqpoint{2.435630in}{1.525591in}}%
\pgfpathlineto{\pgfqpoint{2.447455in}{1.529671in}}%
\pgfpathlineto{\pgfqpoint{2.459275in}{1.533751in}}%
\pgfpathlineto{\pgfqpoint{2.471088in}{1.537829in}}%
\pgfpathlineto{\pgfqpoint{2.464852in}{1.549285in}}%
\pgfpathlineto{\pgfqpoint{2.458621in}{1.560702in}}%
\pgfpathlineto{\pgfqpoint{2.452393in}{1.572083in}}%
\pgfpathlineto{\pgfqpoint{2.446169in}{1.583426in}}%
\pgfpathlineto{\pgfqpoint{2.439949in}{1.594730in}}%
\pgfpathlineto{\pgfqpoint{2.428145in}{1.590659in}}%
\pgfpathlineto{\pgfqpoint{2.416335in}{1.586585in}}%
\pgfpathlineto{\pgfqpoint{2.404519in}{1.582509in}}%
\pgfpathlineto{\pgfqpoint{2.392697in}{1.578431in}}%
\pgfpathlineto{\pgfqpoint{2.380870in}{1.574351in}}%
\pgfpathlineto{\pgfqpoint{2.387080in}{1.563038in}}%
\pgfpathlineto{\pgfqpoint{2.393295in}{1.551689in}}%
\pgfpathlineto{\pgfqpoint{2.399514in}{1.540304in}}%
\pgfpathlineto{\pgfqpoint{2.405736in}{1.528885in}}%
\pgfpathclose%
\pgfusepath{stroke,fill}%
\end{pgfscope}%
\begin{pgfscope}%
\pgfpathrectangle{\pgfqpoint{0.887500in}{0.275000in}}{\pgfqpoint{4.225000in}{4.225000in}}%
\pgfusepath{clip}%
\pgfsetbuttcap%
\pgfsetroundjoin%
\definecolor{currentfill}{rgb}{0.246811,0.283237,0.535941}%
\pgfsetfillcolor{currentfill}%
\pgfsetfillopacity{0.700000}%
\pgfsetlinewidth{0.501875pt}%
\definecolor{currentstroke}{rgb}{1.000000,1.000000,1.000000}%
\pgfsetstrokecolor{currentstroke}%
\pgfsetstrokeopacity{0.500000}%
\pgfsetdash{}{0pt}%
\pgfpathmoveto{\pgfqpoint{2.502326in}{1.479963in}}%
\pgfpathlineto{\pgfqpoint{2.514142in}{1.484034in}}%
\pgfpathlineto{\pgfqpoint{2.525953in}{1.488104in}}%
\pgfpathlineto{\pgfqpoint{2.537759in}{1.492172in}}%
\pgfpathlineto{\pgfqpoint{2.549558in}{1.496238in}}%
\pgfpathlineto{\pgfqpoint{2.561352in}{1.500305in}}%
\pgfpathlineto{\pgfqpoint{2.555088in}{1.511970in}}%
\pgfpathlineto{\pgfqpoint{2.548827in}{1.523590in}}%
\pgfpathlineto{\pgfqpoint{2.542571in}{1.535166in}}%
\pgfpathlineto{\pgfqpoint{2.536318in}{1.546698in}}%
\pgfpathlineto{\pgfqpoint{2.530070in}{1.558188in}}%
\pgfpathlineto{\pgfqpoint{2.518285in}{1.554121in}}%
\pgfpathlineto{\pgfqpoint{2.506494in}{1.550052in}}%
\pgfpathlineto{\pgfqpoint{2.494698in}{1.545980in}}%
\pgfpathlineto{\pgfqpoint{2.482896in}{1.541906in}}%
\pgfpathlineto{\pgfqpoint{2.471088in}{1.537829in}}%
\pgfpathlineto{\pgfqpoint{2.477328in}{1.526335in}}%
\pgfpathlineto{\pgfqpoint{2.483571in}{1.514802in}}%
\pgfpathlineto{\pgfqpoint{2.489819in}{1.503229in}}%
\pgfpathlineto{\pgfqpoint{2.496070in}{1.491617in}}%
\pgfpathclose%
\pgfusepath{stroke,fill}%
\end{pgfscope}%
\begin{pgfscope}%
\pgfpathrectangle{\pgfqpoint{0.887500in}{0.275000in}}{\pgfqpoint{4.225000in}{4.225000in}}%
\pgfusepath{clip}%
\pgfsetbuttcap%
\pgfsetroundjoin%
\definecolor{currentfill}{rgb}{0.253935,0.265254,0.529983}%
\pgfsetfillcolor{currentfill}%
\pgfsetfillopacity{0.700000}%
\pgfsetlinewidth{0.501875pt}%
\definecolor{currentstroke}{rgb}{1.000000,1.000000,1.000000}%
\pgfsetstrokecolor{currentstroke}%
\pgfsetstrokeopacity{0.500000}%
\pgfsetdash{}{0pt}%
\pgfpathmoveto{\pgfqpoint{2.592732in}{1.441265in}}%
\pgfpathlineto{\pgfqpoint{2.604528in}{1.445304in}}%
\pgfpathlineto{\pgfqpoint{2.616319in}{1.449344in}}%
\pgfpathlineto{\pgfqpoint{2.628104in}{1.453388in}}%
\pgfpathlineto{\pgfqpoint{2.639884in}{1.457436in}}%
\pgfpathlineto{\pgfqpoint{2.651657in}{1.461490in}}%
\pgfpathlineto{\pgfqpoint{2.645365in}{1.473434in}}%
\pgfpathlineto{\pgfqpoint{2.639076in}{1.485326in}}%
\pgfpathlineto{\pgfqpoint{2.632791in}{1.497162in}}%
\pgfpathlineto{\pgfqpoint{2.626510in}{1.508942in}}%
\pgfpathlineto{\pgfqpoint{2.620234in}{1.520669in}}%
\pgfpathlineto{\pgfqpoint{2.608469in}{1.516588in}}%
\pgfpathlineto{\pgfqpoint{2.596698in}{1.512512in}}%
\pgfpathlineto{\pgfqpoint{2.584922in}{1.508441in}}%
\pgfpathlineto{\pgfqpoint{2.573140in}{1.504372in}}%
\pgfpathlineto{\pgfqpoint{2.561352in}{1.500305in}}%
\pgfpathlineto{\pgfqpoint{2.567620in}{1.488593in}}%
\pgfpathlineto{\pgfqpoint{2.573892in}{1.476833in}}%
\pgfpathlineto{\pgfqpoint{2.580168in}{1.465025in}}%
\pgfpathlineto{\pgfqpoint{2.586448in}{1.453168in}}%
\pgfpathclose%
\pgfusepath{stroke,fill}%
\end{pgfscope}%
\begin{pgfscope}%
\pgfpathrectangle{\pgfqpoint{0.887500in}{0.275000in}}{\pgfqpoint{4.225000in}{4.225000in}}%
\pgfusepath{clip}%
\pgfsetbuttcap%
\pgfsetroundjoin%
\definecolor{currentfill}{rgb}{0.262138,0.242286,0.520837}%
\pgfsetfillcolor{currentfill}%
\pgfsetfillopacity{0.700000}%
\pgfsetlinewidth{0.501875pt}%
\definecolor{currentstroke}{rgb}{1.000000,1.000000,1.000000}%
\pgfsetstrokecolor{currentstroke}%
\pgfsetstrokeopacity{0.500000}%
\pgfsetdash{}{0pt}%
\pgfpathmoveto{\pgfqpoint{2.683173in}{1.401229in}}%
\pgfpathlineto{\pgfqpoint{2.694949in}{1.405238in}}%
\pgfpathlineto{\pgfqpoint{2.706719in}{1.409252in}}%
\pgfpathlineto{\pgfqpoint{2.718484in}{1.413269in}}%
\pgfpathlineto{\pgfqpoint{2.730242in}{1.417288in}}%
\pgfpathlineto{\pgfqpoint{2.741995in}{1.421309in}}%
\pgfpathlineto{\pgfqpoint{2.735676in}{1.433461in}}%
\pgfpathlineto{\pgfqpoint{2.729361in}{1.445597in}}%
\pgfpathlineto{\pgfqpoint{2.723050in}{1.457706in}}%
\pgfpathlineto{\pgfqpoint{2.716741in}{1.469782in}}%
\pgfpathlineto{\pgfqpoint{2.710437in}{1.481815in}}%
\pgfpathlineto{\pgfqpoint{2.698692in}{1.477747in}}%
\pgfpathlineto{\pgfqpoint{2.686942in}{1.473679in}}%
\pgfpathlineto{\pgfqpoint{2.675186in}{1.469613in}}%
\pgfpathlineto{\pgfqpoint{2.663424in}{1.465549in}}%
\pgfpathlineto{\pgfqpoint{2.651657in}{1.461490in}}%
\pgfpathlineto{\pgfqpoint{2.657953in}{1.449501in}}%
\pgfpathlineto{\pgfqpoint{2.664253in}{1.437474in}}%
\pgfpathlineto{\pgfqpoint{2.670556in}{1.425416in}}%
\pgfpathlineto{\pgfqpoint{2.676863in}{1.413331in}}%
\pgfpathclose%
\pgfusepath{stroke,fill}%
\end{pgfscope}%
\begin{pgfscope}%
\pgfpathrectangle{\pgfqpoint{0.887500in}{0.275000in}}{\pgfqpoint{4.225000in}{4.225000in}}%
\pgfusepath{clip}%
\pgfsetbuttcap%
\pgfsetroundjoin%
\definecolor{currentfill}{rgb}{0.282623,0.140926,0.457517}%
\pgfsetfillcolor{currentfill}%
\pgfsetfillopacity{0.700000}%
\pgfsetlinewidth{0.501875pt}%
\definecolor{currentstroke}{rgb}{1.000000,1.000000,1.000000}%
\pgfsetstrokecolor{currentstroke}%
\pgfsetstrokeopacity{0.500000}%
\pgfsetdash{}{0pt}%
\pgfpathmoveto{\pgfqpoint{3.045111in}{1.223256in}}%
\pgfpathlineto{\pgfqpoint{3.056796in}{1.227353in}}%
\pgfpathlineto{\pgfqpoint{3.068476in}{1.231525in}}%
\pgfpathlineto{\pgfqpoint{3.080151in}{1.235871in}}%
\pgfpathlineto{\pgfqpoint{3.091820in}{1.240491in}}%
\pgfpathlineto{\pgfqpoint{3.103484in}{1.245481in}}%
\pgfpathlineto{\pgfqpoint{3.097072in}{1.258098in}}%
\pgfpathlineto{\pgfqpoint{3.090664in}{1.271104in}}%
\pgfpathlineto{\pgfqpoint{3.084259in}{1.284389in}}%
\pgfpathlineto{\pgfqpoint{3.077857in}{1.297840in}}%
\pgfpathlineto{\pgfqpoint{3.071458in}{1.311347in}}%
\pgfpathlineto{\pgfqpoint{3.059801in}{1.307353in}}%
\pgfpathlineto{\pgfqpoint{3.048137in}{1.303412in}}%
\pgfpathlineto{\pgfqpoint{3.036468in}{1.299511in}}%
\pgfpathlineto{\pgfqpoint{3.024793in}{1.295639in}}%
\pgfpathlineto{\pgfqpoint{3.013112in}{1.291785in}}%
\pgfpathlineto{\pgfqpoint{3.019507in}{1.277864in}}%
\pgfpathlineto{\pgfqpoint{3.025904in}{1.263919in}}%
\pgfpathlineto{\pgfqpoint{3.032303in}{1.250082in}}%
\pgfpathlineto{\pgfqpoint{3.038706in}{1.236483in}}%
\pgfpathclose%
\pgfusepath{stroke,fill}%
\end{pgfscope}%
\begin{pgfscope}%
\pgfpathrectangle{\pgfqpoint{0.887500in}{0.275000in}}{\pgfqpoint{4.225000in}{4.225000in}}%
\pgfusepath{clip}%
\pgfsetbuttcap%
\pgfsetroundjoin%
\definecolor{currentfill}{rgb}{0.269308,0.218818,0.509577}%
\pgfsetfillcolor{currentfill}%
\pgfsetfillopacity{0.700000}%
\pgfsetlinewidth{0.501875pt}%
\definecolor{currentstroke}{rgb}{1.000000,1.000000,1.000000}%
\pgfsetstrokecolor{currentstroke}%
\pgfsetstrokeopacity{0.500000}%
\pgfsetdash{}{0pt}%
\pgfpathmoveto{\pgfqpoint{2.773636in}{1.360554in}}%
\pgfpathlineto{\pgfqpoint{2.785391in}{1.364554in}}%
\pgfpathlineto{\pgfqpoint{2.797140in}{1.368561in}}%
\pgfpathlineto{\pgfqpoint{2.808883in}{1.372572in}}%
\pgfpathlineto{\pgfqpoint{2.820620in}{1.376585in}}%
\pgfpathlineto{\pgfqpoint{2.832352in}{1.380597in}}%
\pgfpathlineto{\pgfqpoint{2.826009in}{1.392711in}}%
\pgfpathlineto{\pgfqpoint{2.819670in}{1.404847in}}%
\pgfpathlineto{\pgfqpoint{2.813334in}{1.417014in}}%
\pgfpathlineto{\pgfqpoint{2.807001in}{1.429200in}}%
\pgfpathlineto{\pgfqpoint{2.800672in}{1.441395in}}%
\pgfpathlineto{\pgfqpoint{2.788948in}{1.437379in}}%
\pgfpathlineto{\pgfqpoint{2.777218in}{1.433363in}}%
\pgfpathlineto{\pgfqpoint{2.765483in}{1.429346in}}%
\pgfpathlineto{\pgfqpoint{2.753742in}{1.425328in}}%
\pgfpathlineto{\pgfqpoint{2.741995in}{1.421309in}}%
\pgfpathlineto{\pgfqpoint{2.748317in}{1.409147in}}%
\pgfpathlineto{\pgfqpoint{2.754642in}{1.396983in}}%
\pgfpathlineto{\pgfqpoint{2.760970in}{1.384827in}}%
\pgfpathlineto{\pgfqpoint{2.767302in}{1.372685in}}%
\pgfpathclose%
\pgfusepath{stroke,fill}%
\end{pgfscope}%
\begin{pgfscope}%
\pgfpathrectangle{\pgfqpoint{0.887500in}{0.275000in}}{\pgfqpoint{4.225000in}{4.225000in}}%
\pgfusepath{clip}%
\pgfsetbuttcap%
\pgfsetroundjoin%
\definecolor{currentfill}{rgb}{0.279574,0.170599,0.479997}%
\pgfsetfillcolor{currentfill}%
\pgfsetfillopacity{0.700000}%
\pgfsetlinewidth{0.501875pt}%
\definecolor{currentstroke}{rgb}{1.000000,1.000000,1.000000}%
\pgfsetstrokecolor{currentstroke}%
\pgfsetstrokeopacity{0.500000}%
\pgfsetdash{}{0pt}%
\pgfpathmoveto{\pgfqpoint{2.954620in}{1.272658in}}%
\pgfpathlineto{\pgfqpoint{2.966330in}{1.276457in}}%
\pgfpathlineto{\pgfqpoint{2.978035in}{1.280272in}}%
\pgfpathlineto{\pgfqpoint{2.989733in}{1.284100in}}%
\pgfpathlineto{\pgfqpoint{3.001426in}{1.287939in}}%
\pgfpathlineto{\pgfqpoint{3.013112in}{1.291785in}}%
\pgfpathlineto{\pgfqpoint{3.006721in}{1.305549in}}%
\pgfpathlineto{\pgfqpoint{3.000333in}{1.319056in}}%
\pgfpathlineto{\pgfqpoint{2.993948in}{1.332310in}}%
\pgfpathlineto{\pgfqpoint{2.987567in}{1.345341in}}%
\pgfpathlineto{\pgfqpoint{2.981189in}{1.358176in}}%
\pgfpathlineto{\pgfqpoint{2.969508in}{1.354328in}}%
\pgfpathlineto{\pgfqpoint{2.957821in}{1.350489in}}%
\pgfpathlineto{\pgfqpoint{2.946128in}{1.346651in}}%
\pgfpathlineto{\pgfqpoint{2.934429in}{1.342806in}}%
\pgfpathlineto{\pgfqpoint{2.922724in}{1.338949in}}%
\pgfpathlineto{\pgfqpoint{2.929096in}{1.326203in}}%
\pgfpathlineto{\pgfqpoint{2.935471in}{1.313234in}}%
\pgfpathlineto{\pgfqpoint{2.941850in}{1.300006in}}%
\pgfpathlineto{\pgfqpoint{2.948233in}{1.286485in}}%
\pgfpathclose%
\pgfusepath{stroke,fill}%
\end{pgfscope}%
\begin{pgfscope}%
\pgfpathrectangle{\pgfqpoint{0.887500in}{0.275000in}}{\pgfqpoint{4.225000in}{4.225000in}}%
\pgfusepath{clip}%
\pgfsetbuttcap%
\pgfsetroundjoin%
\definecolor{currentfill}{rgb}{0.275191,0.194905,0.496005}%
\pgfsetfillcolor{currentfill}%
\pgfsetfillopacity{0.700000}%
\pgfsetlinewidth{0.501875pt}%
\definecolor{currentstroke}{rgb}{1.000000,1.000000,1.000000}%
\pgfsetstrokecolor{currentstroke}%
\pgfsetstrokeopacity{0.500000}%
\pgfsetdash{}{0pt}%
\pgfpathmoveto{\pgfqpoint{2.864115in}{1.319297in}}%
\pgfpathlineto{\pgfqpoint{2.875849in}{1.323275in}}%
\pgfpathlineto{\pgfqpoint{2.887576in}{1.327235in}}%
\pgfpathlineto{\pgfqpoint{2.899298in}{1.331169in}}%
\pgfpathlineto{\pgfqpoint{2.911014in}{1.335073in}}%
\pgfpathlineto{\pgfqpoint{2.922724in}{1.338949in}}%
\pgfpathlineto{\pgfqpoint{2.916357in}{1.351507in}}%
\pgfpathlineto{\pgfqpoint{2.909993in}{1.363911in}}%
\pgfpathlineto{\pgfqpoint{2.903633in}{1.376198in}}%
\pgfpathlineto{\pgfqpoint{2.897276in}{1.388401in}}%
\pgfpathlineto{\pgfqpoint{2.890923in}{1.400557in}}%
\pgfpathlineto{\pgfqpoint{2.879220in}{1.396583in}}%
\pgfpathlineto{\pgfqpoint{2.867511in}{1.392600in}}%
\pgfpathlineto{\pgfqpoint{2.855797in}{1.388607in}}%
\pgfpathlineto{\pgfqpoint{2.844077in}{1.384605in}}%
\pgfpathlineto{\pgfqpoint{2.832352in}{1.380597in}}%
\pgfpathlineto{\pgfqpoint{2.838697in}{1.368475in}}%
\pgfpathlineto{\pgfqpoint{2.845046in}{1.356315in}}%
\pgfpathlineto{\pgfqpoint{2.851399in}{1.344086in}}%
\pgfpathlineto{\pgfqpoint{2.857755in}{1.331757in}}%
\pgfpathclose%
\pgfusepath{stroke,fill}%
\end{pgfscope}%
\begin{pgfscope}%
\pgfpathrectangle{\pgfqpoint{0.887500in}{0.275000in}}{\pgfqpoint{4.225000in}{4.225000in}}%
\pgfusepath{clip}%
\pgfsetbuttcap%
\pgfsetroundjoin%
\definecolor{currentfill}{rgb}{0.216210,0.351535,0.550627}%
\pgfsetfillcolor{currentfill}%
\pgfsetfillopacity{0.700000}%
\pgfsetlinewidth{0.501875pt}%
\definecolor{currentstroke}{rgb}{1.000000,1.000000,1.000000}%
\pgfsetstrokecolor{currentstroke}%
\pgfsetstrokeopacity{0.500000}%
\pgfsetdash{}{0pt}%
\pgfpathmoveto{\pgfqpoint{2.081607in}{1.604099in}}%
\pgfpathlineto{\pgfqpoint{2.093531in}{1.608190in}}%
\pgfpathlineto{\pgfqpoint{2.105449in}{1.612275in}}%
\pgfpathlineto{\pgfqpoint{2.117361in}{1.616354in}}%
\pgfpathlineto{\pgfqpoint{2.129268in}{1.620429in}}%
\pgfpathlineto{\pgfqpoint{2.141169in}{1.624499in}}%
\pgfpathlineto{\pgfqpoint{2.135033in}{1.635433in}}%
\pgfpathlineto{\pgfqpoint{2.128901in}{1.646330in}}%
\pgfpathlineto{\pgfqpoint{2.122773in}{1.657193in}}%
\pgfpathlineto{\pgfqpoint{2.116650in}{1.668019in}}%
\pgfpathlineto{\pgfqpoint{2.110531in}{1.678808in}}%
\pgfpathlineto{\pgfqpoint{2.098640in}{1.674758in}}%
\pgfpathlineto{\pgfqpoint{2.086744in}{1.670703in}}%
\pgfpathlineto{\pgfqpoint{2.074842in}{1.666642in}}%
\pgfpathlineto{\pgfqpoint{2.062934in}{1.662574in}}%
\pgfpathlineto{\pgfqpoint{2.051020in}{1.658500in}}%
\pgfpathlineto{\pgfqpoint{2.057129in}{1.647686in}}%
\pgfpathlineto{\pgfqpoint{2.063242in}{1.636838in}}%
\pgfpathlineto{\pgfqpoint{2.069360in}{1.625957in}}%
\pgfpathlineto{\pgfqpoint{2.075482in}{1.615044in}}%
\pgfpathclose%
\pgfusepath{stroke,fill}%
\end{pgfscope}%
\begin{pgfscope}%
\pgfpathrectangle{\pgfqpoint{0.887500in}{0.275000in}}{\pgfqpoint{4.225000in}{4.225000in}}%
\pgfusepath{clip}%
\pgfsetbuttcap%
\pgfsetroundjoin%
\definecolor{currentfill}{rgb}{0.278791,0.062145,0.386592}%
\pgfsetfillcolor{currentfill}%
\pgfsetfillopacity{0.700000}%
\pgfsetlinewidth{0.501875pt}%
\definecolor{currentstroke}{rgb}{1.000000,1.000000,1.000000}%
\pgfsetstrokecolor{currentstroke}%
\pgfsetstrokeopacity{0.500000}%
\pgfsetdash{}{0pt}%
\pgfpathmoveto{\pgfqpoint{3.316752in}{1.124403in}}%
\pgfpathlineto{\pgfqpoint{3.328302in}{1.117454in}}%
\pgfpathlineto{\pgfqpoint{3.339843in}{1.110950in}}%
\pgfpathlineto{\pgfqpoint{3.351376in}{1.104917in}}%
\pgfpathlineto{\pgfqpoint{3.362901in}{1.099385in}}%
\pgfpathlineto{\pgfqpoint{3.374419in}{1.094381in}}%
\pgfpathlineto{\pgfqpoint{3.367953in}{1.108254in}}%
\pgfpathlineto{\pgfqpoint{3.361490in}{1.122138in}}%
\pgfpathlineto{\pgfqpoint{3.355030in}{1.136031in}}%
\pgfpathlineto{\pgfqpoint{3.348573in}{1.149929in}}%
\pgfpathlineto{\pgfqpoint{3.342119in}{1.163827in}}%
\pgfpathlineto{\pgfqpoint{3.330609in}{1.170368in}}%
\pgfpathlineto{\pgfqpoint{3.319089in}{1.177232in}}%
\pgfpathlineto{\pgfqpoint{3.307555in}{1.183930in}}%
\pgfpathlineto{\pgfqpoint{3.296005in}{1.189972in}}%
\pgfpathlineto{\pgfqpoint{3.284437in}{1.194865in}}%
\pgfpathlineto{\pgfqpoint{3.290903in}{1.182212in}}%
\pgfpathlineto{\pgfqpoint{3.297369in}{1.168907in}}%
\pgfpathlineto{\pgfqpoint{3.303832in}{1.154882in}}%
\pgfpathlineto{\pgfqpoint{3.310294in}{1.140069in}}%
\pgfpathclose%
\pgfusepath{stroke,fill}%
\end{pgfscope}%
\begin{pgfscope}%
\pgfpathrectangle{\pgfqpoint{0.887500in}{0.275000in}}{\pgfqpoint{4.225000in}{4.225000in}}%
\pgfusepath{clip}%
\pgfsetbuttcap%
\pgfsetroundjoin%
\definecolor{currentfill}{rgb}{0.283229,0.120777,0.440584}%
\pgfsetfillcolor{currentfill}%
\pgfsetfillopacity{0.700000}%
\pgfsetlinewidth{0.501875pt}%
\definecolor{currentstroke}{rgb}{1.000000,1.000000,1.000000}%
\pgfsetstrokecolor{currentstroke}%
\pgfsetstrokeopacity{0.500000}%
\pgfsetdash{}{0pt}%
\pgfpathmoveto{\pgfqpoint{3.135611in}{1.192135in}}%
\pgfpathlineto{\pgfqpoint{3.147284in}{1.198525in}}%
\pgfpathlineto{\pgfqpoint{3.158955in}{1.205747in}}%
\pgfpathlineto{\pgfqpoint{3.170624in}{1.213748in}}%
\pgfpathlineto{\pgfqpoint{3.182290in}{1.222068in}}%
\pgfpathlineto{\pgfqpoint{3.193953in}{1.230201in}}%
\pgfpathlineto{\pgfqpoint{3.187501in}{1.238924in}}%
\pgfpathlineto{\pgfqpoint{3.181054in}{1.247986in}}%
\pgfpathlineto{\pgfqpoint{3.174613in}{1.257382in}}%
\pgfpathlineto{\pgfqpoint{3.168177in}{1.267109in}}%
\pgfpathlineto{\pgfqpoint{3.161746in}{1.277163in}}%
\pgfpathlineto{\pgfqpoint{3.150101in}{1.270364in}}%
\pgfpathlineto{\pgfqpoint{3.138452in}{1.263531in}}%
\pgfpathlineto{\pgfqpoint{3.126800in}{1.256966in}}%
\pgfpathlineto{\pgfqpoint{3.115144in}{1.250940in}}%
\pgfpathlineto{\pgfqpoint{3.103484in}{1.245481in}}%
\pgfpathlineto{\pgfqpoint{3.109900in}{1.233365in}}%
\pgfpathlineto{\pgfqpoint{3.116320in}{1.221861in}}%
\pgfpathlineto{\pgfqpoint{3.122745in}{1.211080in}}%
\pgfpathlineto{\pgfqpoint{3.129175in}{1.201135in}}%
\pgfpathclose%
\pgfusepath{stroke,fill}%
\end{pgfscope}%
\begin{pgfscope}%
\pgfpathrectangle{\pgfqpoint{0.887500in}{0.275000in}}{\pgfqpoint{4.225000in}{4.225000in}}%
\pgfusepath{clip}%
\pgfsetbuttcap%
\pgfsetroundjoin%
\definecolor{currentfill}{rgb}{0.223925,0.334994,0.548053}%
\pgfsetfillcolor{currentfill}%
\pgfsetfillopacity{0.700000}%
\pgfsetlinewidth{0.501875pt}%
\definecolor{currentstroke}{rgb}{1.000000,1.000000,1.000000}%
\pgfsetstrokecolor{currentstroke}%
\pgfsetstrokeopacity{0.500000}%
\pgfsetdash{}{0pt}%
\pgfpathmoveto{\pgfqpoint{2.171912in}{1.569318in}}%
\pgfpathlineto{\pgfqpoint{2.183817in}{1.573388in}}%
\pgfpathlineto{\pgfqpoint{2.195716in}{1.577456in}}%
\pgfpathlineto{\pgfqpoint{2.207610in}{1.581522in}}%
\pgfpathlineto{\pgfqpoint{2.219497in}{1.585588in}}%
\pgfpathlineto{\pgfqpoint{2.231379in}{1.589654in}}%
\pgfpathlineto{\pgfqpoint{2.225212in}{1.600758in}}%
\pgfpathlineto{\pgfqpoint{2.219049in}{1.611827in}}%
\pgfpathlineto{\pgfqpoint{2.212891in}{1.622857in}}%
\pgfpathlineto{\pgfqpoint{2.206736in}{1.633850in}}%
\pgfpathlineto{\pgfqpoint{2.200586in}{1.644805in}}%
\pgfpathlineto{\pgfqpoint{2.188714in}{1.640748in}}%
\pgfpathlineto{\pgfqpoint{2.176836in}{1.636690in}}%
\pgfpathlineto{\pgfqpoint{2.164953in}{1.632629in}}%
\pgfpathlineto{\pgfqpoint{2.153064in}{1.628566in}}%
\pgfpathlineto{\pgfqpoint{2.141169in}{1.624499in}}%
\pgfpathlineto{\pgfqpoint{2.147309in}{1.613531in}}%
\pgfpathlineto{\pgfqpoint{2.153453in}{1.602529in}}%
\pgfpathlineto{\pgfqpoint{2.159602in}{1.591492in}}%
\pgfpathlineto{\pgfqpoint{2.165755in}{1.580422in}}%
\pgfpathclose%
\pgfusepath{stroke,fill}%
\end{pgfscope}%
\begin{pgfscope}%
\pgfpathrectangle{\pgfqpoint{0.887500in}{0.275000in}}{\pgfqpoint{4.225000in}{4.225000in}}%
\pgfusepath{clip}%
\pgfsetbuttcap%
\pgfsetroundjoin%
\definecolor{currentfill}{rgb}{0.231674,0.318106,0.544834}%
\pgfsetfillcolor{currentfill}%
\pgfsetfillopacity{0.700000}%
\pgfsetlinewidth{0.501875pt}%
\definecolor{currentstroke}{rgb}{1.000000,1.000000,1.000000}%
\pgfsetstrokecolor{currentstroke}%
\pgfsetstrokeopacity{0.500000}%
\pgfsetdash{}{0pt}%
\pgfpathmoveto{\pgfqpoint{2.262276in}{1.533614in}}%
\pgfpathlineto{\pgfqpoint{2.274161in}{1.537676in}}%
\pgfpathlineto{\pgfqpoint{2.286041in}{1.541740in}}%
\pgfpathlineto{\pgfqpoint{2.297915in}{1.545808in}}%
\pgfpathlineto{\pgfqpoint{2.309784in}{1.549879in}}%
\pgfpathlineto{\pgfqpoint{2.321646in}{1.553953in}}%
\pgfpathlineto{\pgfqpoint{2.315449in}{1.565233in}}%
\pgfpathlineto{\pgfqpoint{2.309255in}{1.576478in}}%
\pgfpathlineto{\pgfqpoint{2.303066in}{1.587687in}}%
\pgfpathlineto{\pgfqpoint{2.296881in}{1.598859in}}%
\pgfpathlineto{\pgfqpoint{2.290701in}{1.609995in}}%
\pgfpathlineto{\pgfqpoint{2.278848in}{1.605923in}}%
\pgfpathlineto{\pgfqpoint{2.266989in}{1.601854in}}%
\pgfpathlineto{\pgfqpoint{2.255125in}{1.597786in}}%
\pgfpathlineto{\pgfqpoint{2.243255in}{1.593720in}}%
\pgfpathlineto{\pgfqpoint{2.231379in}{1.589654in}}%
\pgfpathlineto{\pgfqpoint{2.237550in}{1.578514in}}%
\pgfpathlineto{\pgfqpoint{2.243725in}{1.567339in}}%
\pgfpathlineto{\pgfqpoint{2.249905in}{1.556130in}}%
\pgfpathlineto{\pgfqpoint{2.256088in}{1.544888in}}%
\pgfpathclose%
\pgfusepath{stroke,fill}%
\end{pgfscope}%
\begin{pgfscope}%
\pgfpathrectangle{\pgfqpoint{0.887500in}{0.275000in}}{\pgfqpoint{4.225000in}{4.225000in}}%
\pgfusepath{clip}%
\pgfsetbuttcap%
\pgfsetroundjoin%
\definecolor{currentfill}{rgb}{0.239346,0.300855,0.540844}%
\pgfsetfillcolor{currentfill}%
\pgfsetfillopacity{0.700000}%
\pgfsetlinewidth{0.501875pt}%
\definecolor{currentstroke}{rgb}{1.000000,1.000000,1.000000}%
\pgfsetstrokecolor{currentstroke}%
\pgfsetstrokeopacity{0.500000}%
\pgfsetdash{}{0pt}%
\pgfpathmoveto{\pgfqpoint{2.352692in}{1.497063in}}%
\pgfpathlineto{\pgfqpoint{2.364557in}{1.501131in}}%
\pgfpathlineto{\pgfqpoint{2.376418in}{1.505202in}}%
\pgfpathlineto{\pgfqpoint{2.388272in}{1.509276in}}%
\pgfpathlineto{\pgfqpoint{2.400120in}{1.513353in}}%
\pgfpathlineto{\pgfqpoint{2.411963in}{1.517431in}}%
\pgfpathlineto{\pgfqpoint{2.405736in}{1.528885in}}%
\pgfpathlineto{\pgfqpoint{2.399514in}{1.540304in}}%
\pgfpathlineto{\pgfqpoint{2.393295in}{1.551689in}}%
\pgfpathlineto{\pgfqpoint{2.387080in}{1.563038in}}%
\pgfpathlineto{\pgfqpoint{2.380870in}{1.574351in}}%
\pgfpathlineto{\pgfqpoint{2.369036in}{1.570270in}}%
\pgfpathlineto{\pgfqpoint{2.357197in}{1.566190in}}%
\pgfpathlineto{\pgfqpoint{2.345353in}{1.562109in}}%
\pgfpathlineto{\pgfqpoint{2.333502in}{1.558030in}}%
\pgfpathlineto{\pgfqpoint{2.321646in}{1.553953in}}%
\pgfpathlineto{\pgfqpoint{2.327847in}{1.542640in}}%
\pgfpathlineto{\pgfqpoint{2.334052in}{1.531293in}}%
\pgfpathlineto{\pgfqpoint{2.340261in}{1.519915in}}%
\pgfpathlineto{\pgfqpoint{2.346474in}{1.508505in}}%
\pgfpathclose%
\pgfusepath{stroke,fill}%
\end{pgfscope}%
\begin{pgfscope}%
\pgfpathrectangle{\pgfqpoint{0.887500in}{0.275000in}}{\pgfqpoint{4.225000in}{4.225000in}}%
\pgfusepath{clip}%
\pgfsetbuttcap%
\pgfsetroundjoin%
\definecolor{currentfill}{rgb}{0.246811,0.283237,0.535941}%
\pgfsetfillcolor{currentfill}%
\pgfsetfillopacity{0.700000}%
\pgfsetlinewidth{0.501875pt}%
\definecolor{currentstroke}{rgb}{1.000000,1.000000,1.000000}%
\pgfsetstrokecolor{currentstroke}%
\pgfsetstrokeopacity{0.500000}%
\pgfsetdash{}{0pt}%
\pgfpathmoveto{\pgfqpoint{2.443154in}{1.459631in}}%
\pgfpathlineto{\pgfqpoint{2.455000in}{1.463691in}}%
\pgfpathlineto{\pgfqpoint{2.466840in}{1.467756in}}%
\pgfpathlineto{\pgfqpoint{2.478675in}{1.471823in}}%
\pgfpathlineto{\pgfqpoint{2.490503in}{1.475893in}}%
\pgfpathlineto{\pgfqpoint{2.502326in}{1.479963in}}%
\pgfpathlineto{\pgfqpoint{2.496070in}{1.491617in}}%
\pgfpathlineto{\pgfqpoint{2.489819in}{1.503229in}}%
\pgfpathlineto{\pgfqpoint{2.483571in}{1.514802in}}%
\pgfpathlineto{\pgfqpoint{2.477328in}{1.526335in}}%
\pgfpathlineto{\pgfqpoint{2.471088in}{1.537829in}}%
\pgfpathlineto{\pgfqpoint{2.459275in}{1.533751in}}%
\pgfpathlineto{\pgfqpoint{2.447455in}{1.529671in}}%
\pgfpathlineto{\pgfqpoint{2.435630in}{1.525591in}}%
\pgfpathlineto{\pgfqpoint{2.423799in}{1.521511in}}%
\pgfpathlineto{\pgfqpoint{2.411963in}{1.517431in}}%
\pgfpathlineto{\pgfqpoint{2.418193in}{1.505942in}}%
\pgfpathlineto{\pgfqpoint{2.424427in}{1.494418in}}%
\pgfpathlineto{\pgfqpoint{2.430666in}{1.482858in}}%
\pgfpathlineto{\pgfqpoint{2.436908in}{1.471262in}}%
\pgfpathclose%
\pgfusepath{stroke,fill}%
\end{pgfscope}%
\begin{pgfscope}%
\pgfpathrectangle{\pgfqpoint{0.887500in}{0.275000in}}{\pgfqpoint{4.225000in}{4.225000in}}%
\pgfusepath{clip}%
\pgfsetbuttcap%
\pgfsetroundjoin%
\definecolor{currentfill}{rgb}{0.283197,0.115680,0.436115}%
\pgfsetfillcolor{currentfill}%
\pgfsetfillopacity{0.700000}%
\pgfsetlinewidth{0.501875pt}%
\definecolor{currentstroke}{rgb}{1.000000,1.000000,1.000000}%
\pgfsetstrokecolor{currentstroke}%
\pgfsetstrokeopacity{0.500000}%
\pgfsetdash{}{0pt}%
\pgfpathmoveto{\pgfqpoint{3.226307in}{1.191776in}}%
\pgfpathlineto{\pgfqpoint{3.237964in}{1.195722in}}%
\pgfpathlineto{\pgfqpoint{3.249609in}{1.198349in}}%
\pgfpathlineto{\pgfqpoint{3.261238in}{1.199272in}}%
\pgfpathlineto{\pgfqpoint{3.272848in}{1.198118in}}%
\pgfpathlineto{\pgfqpoint{3.284437in}{1.194865in}}%
\pgfpathlineto{\pgfqpoint{3.277971in}{1.206933in}}%
\pgfpathlineto{\pgfqpoint{3.271505in}{1.218483in}}%
\pgfpathlineto{\pgfqpoint{3.265041in}{1.229583in}}%
\pgfpathlineto{\pgfqpoint{3.258578in}{1.240300in}}%
\pgfpathlineto{\pgfqpoint{3.252117in}{1.250701in}}%
\pgfpathlineto{\pgfqpoint{3.240515in}{1.250695in}}%
\pgfpathlineto{\pgfqpoint{3.228895in}{1.248387in}}%
\pgfpathlineto{\pgfqpoint{3.217259in}{1.243868in}}%
\pgfpathlineto{\pgfqpoint{3.205610in}{1.237636in}}%
\pgfpathlineto{\pgfqpoint{3.193953in}{1.230201in}}%
\pgfpathlineto{\pgfqpoint{3.200412in}{1.221818in}}%
\pgfpathlineto{\pgfqpoint{3.206876in}{1.213781in}}%
\pgfpathlineto{\pgfqpoint{3.213346in}{1.206093in}}%
\pgfpathlineto{\pgfqpoint{3.219823in}{1.198756in}}%
\pgfpathclose%
\pgfusepath{stroke,fill}%
\end{pgfscope}%
\begin{pgfscope}%
\pgfpathrectangle{\pgfqpoint{0.887500in}{0.275000in}}{\pgfqpoint{4.225000in}{4.225000in}}%
\pgfusepath{clip}%
\pgfsetbuttcap%
\pgfsetroundjoin%
\definecolor{currentfill}{rgb}{0.255645,0.260703,0.528312}%
\pgfsetfillcolor{currentfill}%
\pgfsetfillopacity{0.700000}%
\pgfsetlinewidth{0.501875pt}%
\definecolor{currentstroke}{rgb}{1.000000,1.000000,1.000000}%
\pgfsetstrokecolor{currentstroke}%
\pgfsetstrokeopacity{0.500000}%
\pgfsetdash{}{0pt}%
\pgfpathmoveto{\pgfqpoint{2.533661in}{1.421074in}}%
\pgfpathlineto{\pgfqpoint{2.545487in}{1.425113in}}%
\pgfpathlineto{\pgfqpoint{2.557307in}{1.429153in}}%
\pgfpathlineto{\pgfqpoint{2.569121in}{1.433191in}}%
\pgfpathlineto{\pgfqpoint{2.580929in}{1.437228in}}%
\pgfpathlineto{\pgfqpoint{2.592732in}{1.441265in}}%
\pgfpathlineto{\pgfqpoint{2.586448in}{1.453168in}}%
\pgfpathlineto{\pgfqpoint{2.580168in}{1.465025in}}%
\pgfpathlineto{\pgfqpoint{2.573892in}{1.476833in}}%
\pgfpathlineto{\pgfqpoint{2.567620in}{1.488593in}}%
\pgfpathlineto{\pgfqpoint{2.561352in}{1.500305in}}%
\pgfpathlineto{\pgfqpoint{2.549558in}{1.496238in}}%
\pgfpathlineto{\pgfqpoint{2.537759in}{1.492172in}}%
\pgfpathlineto{\pgfqpoint{2.525953in}{1.488104in}}%
\pgfpathlineto{\pgfqpoint{2.514142in}{1.484034in}}%
\pgfpathlineto{\pgfqpoint{2.502326in}{1.479963in}}%
\pgfpathlineto{\pgfqpoint{2.508585in}{1.468269in}}%
\pgfpathlineto{\pgfqpoint{2.514848in}{1.456532in}}%
\pgfpathlineto{\pgfqpoint{2.521115in}{1.444754in}}%
\pgfpathlineto{\pgfqpoint{2.527386in}{1.432933in}}%
\pgfpathclose%
\pgfusepath{stroke,fill}%
\end{pgfscope}%
\begin{pgfscope}%
\pgfpathrectangle{\pgfqpoint{0.887500in}{0.275000in}}{\pgfqpoint{4.225000in}{4.225000in}}%
\pgfusepath{clip}%
\pgfsetbuttcap%
\pgfsetroundjoin%
\definecolor{currentfill}{rgb}{0.262138,0.242286,0.520837}%
\pgfsetfillcolor{currentfill}%
\pgfsetfillopacity{0.700000}%
\pgfsetlinewidth{0.501875pt}%
\definecolor{currentstroke}{rgb}{1.000000,1.000000,1.000000}%
\pgfsetstrokecolor{currentstroke}%
\pgfsetstrokeopacity{0.500000}%
\pgfsetdash{}{0pt}%
\pgfpathmoveto{\pgfqpoint{2.624205in}{1.381266in}}%
\pgfpathlineto{\pgfqpoint{2.636011in}{1.385250in}}%
\pgfpathlineto{\pgfqpoint{2.647810in}{1.389237in}}%
\pgfpathlineto{\pgfqpoint{2.659604in}{1.393229in}}%
\pgfpathlineto{\pgfqpoint{2.671392in}{1.397226in}}%
\pgfpathlineto{\pgfqpoint{2.683173in}{1.401229in}}%
\pgfpathlineto{\pgfqpoint{2.676863in}{1.413331in}}%
\pgfpathlineto{\pgfqpoint{2.670556in}{1.425416in}}%
\pgfpathlineto{\pgfqpoint{2.664253in}{1.437474in}}%
\pgfpathlineto{\pgfqpoint{2.657953in}{1.449501in}}%
\pgfpathlineto{\pgfqpoint{2.651657in}{1.461490in}}%
\pgfpathlineto{\pgfqpoint{2.639884in}{1.457436in}}%
\pgfpathlineto{\pgfqpoint{2.628104in}{1.453388in}}%
\pgfpathlineto{\pgfqpoint{2.616319in}{1.449344in}}%
\pgfpathlineto{\pgfqpoint{2.604528in}{1.445304in}}%
\pgfpathlineto{\pgfqpoint{2.592732in}{1.441265in}}%
\pgfpathlineto{\pgfqpoint{2.599019in}{1.429324in}}%
\pgfpathlineto{\pgfqpoint{2.605310in}{1.417348in}}%
\pgfpathlineto{\pgfqpoint{2.611605in}{1.405342in}}%
\pgfpathlineto{\pgfqpoint{2.617903in}{1.393313in}}%
\pgfpathclose%
\pgfusepath{stroke,fill}%
\end{pgfscope}%
\begin{pgfscope}%
\pgfpathrectangle{\pgfqpoint{0.887500in}{0.275000in}}{\pgfqpoint{4.225000in}{4.225000in}}%
\pgfusepath{clip}%
\pgfsetbuttcap%
\pgfsetroundjoin%
\definecolor{currentfill}{rgb}{0.282884,0.135920,0.453427}%
\pgfsetfillcolor{currentfill}%
\pgfsetfillopacity{0.700000}%
\pgfsetlinewidth{0.501875pt}%
\definecolor{currentstroke}{rgb}{1.000000,1.000000,1.000000}%
\pgfsetstrokecolor{currentstroke}%
\pgfsetstrokeopacity{0.500000}%
\pgfsetdash{}{0pt}%
\pgfpathmoveto{\pgfqpoint{2.986595in}{1.203034in}}%
\pgfpathlineto{\pgfqpoint{2.998310in}{1.206956in}}%
\pgfpathlineto{\pgfqpoint{3.010019in}{1.210971in}}%
\pgfpathlineto{\pgfqpoint{3.021722in}{1.215048in}}%
\pgfpathlineto{\pgfqpoint{3.033419in}{1.219154in}}%
\pgfpathlineto{\pgfqpoint{3.045111in}{1.223256in}}%
\pgfpathlineto{\pgfqpoint{3.038706in}{1.236483in}}%
\pgfpathlineto{\pgfqpoint{3.032303in}{1.250082in}}%
\pgfpathlineto{\pgfqpoint{3.025904in}{1.263919in}}%
\pgfpathlineto{\pgfqpoint{3.019507in}{1.277864in}}%
\pgfpathlineto{\pgfqpoint{3.013112in}{1.291785in}}%
\pgfpathlineto{\pgfqpoint{3.001426in}{1.287939in}}%
\pgfpathlineto{\pgfqpoint{2.989733in}{1.284100in}}%
\pgfpathlineto{\pgfqpoint{2.978035in}{1.280272in}}%
\pgfpathlineto{\pgfqpoint{2.966330in}{1.276457in}}%
\pgfpathlineto{\pgfqpoint{2.954620in}{1.272658in}}%
\pgfpathlineto{\pgfqpoint{2.961010in}{1.258625in}}%
\pgfpathlineto{\pgfqpoint{2.967403in}{1.244516in}}%
\pgfpathlineto{\pgfqpoint{2.973798in}{1.230461in}}%
\pgfpathlineto{\pgfqpoint{2.980196in}{1.216590in}}%
\pgfpathclose%
\pgfusepath{stroke,fill}%
\end{pgfscope}%
\begin{pgfscope}%
\pgfpathrectangle{\pgfqpoint{0.887500in}{0.275000in}}{\pgfqpoint{4.225000in}{4.225000in}}%
\pgfusepath{clip}%
\pgfsetbuttcap%
\pgfsetroundjoin%
\definecolor{currentfill}{rgb}{0.283091,0.110553,0.431554}%
\pgfsetfillcolor{currentfill}%
\pgfsetfillopacity{0.700000}%
\pgfsetlinewidth{0.501875pt}%
\definecolor{currentstroke}{rgb}{1.000000,1.000000,1.000000}%
\pgfsetstrokecolor{currentstroke}%
\pgfsetstrokeopacity{0.500000}%
\pgfsetdash{}{0pt}%
\pgfpathmoveto{\pgfqpoint{3.077186in}{1.167304in}}%
\pgfpathlineto{\pgfqpoint{3.088881in}{1.171833in}}%
\pgfpathlineto{\pgfqpoint{3.100571in}{1.176427in}}%
\pgfpathlineto{\pgfqpoint{3.112256in}{1.181240in}}%
\pgfpathlineto{\pgfqpoint{3.123935in}{1.186424in}}%
\pgfpathlineto{\pgfqpoint{3.135611in}{1.192135in}}%
\pgfpathlineto{\pgfqpoint{3.129175in}{1.201135in}}%
\pgfpathlineto{\pgfqpoint{3.122745in}{1.211080in}}%
\pgfpathlineto{\pgfqpoint{3.116320in}{1.221861in}}%
\pgfpathlineto{\pgfqpoint{3.109900in}{1.233365in}}%
\pgfpathlineto{\pgfqpoint{3.103484in}{1.245481in}}%
\pgfpathlineto{\pgfqpoint{3.091820in}{1.240491in}}%
\pgfpathlineto{\pgfqpoint{3.080151in}{1.235871in}}%
\pgfpathlineto{\pgfqpoint{3.068476in}{1.231525in}}%
\pgfpathlineto{\pgfqpoint{3.056796in}{1.227353in}}%
\pgfpathlineto{\pgfqpoint{3.045111in}{1.223256in}}%
\pgfpathlineto{\pgfqpoint{3.051519in}{1.210533in}}%
\pgfpathlineto{\pgfqpoint{3.057930in}{1.198444in}}%
\pgfpathlineto{\pgfqpoint{3.064345in}{1.187121in}}%
\pgfpathlineto{\pgfqpoint{3.070763in}{1.176697in}}%
\pgfpathclose%
\pgfusepath{stroke,fill}%
\end{pgfscope}%
\begin{pgfscope}%
\pgfpathrectangle{\pgfqpoint{0.887500in}{0.275000in}}{\pgfqpoint{4.225000in}{4.225000in}}%
\pgfusepath{clip}%
\pgfsetbuttcap%
\pgfsetroundjoin%
\definecolor{currentfill}{rgb}{0.269308,0.218818,0.509577}%
\pgfsetfillcolor{currentfill}%
\pgfsetfillopacity{0.700000}%
\pgfsetlinewidth{0.501875pt}%
\definecolor{currentstroke}{rgb}{1.000000,1.000000,1.000000}%
\pgfsetstrokecolor{currentstroke}%
\pgfsetstrokeopacity{0.500000}%
\pgfsetdash{}{0pt}%
\pgfpathmoveto{\pgfqpoint{2.714774in}{1.340654in}}%
\pgfpathlineto{\pgfqpoint{2.726558in}{1.344620in}}%
\pgfpathlineto{\pgfqpoint{2.738337in}{1.348593in}}%
\pgfpathlineto{\pgfqpoint{2.750109in}{1.352573in}}%
\pgfpathlineto{\pgfqpoint{2.761876in}{1.356560in}}%
\pgfpathlineto{\pgfqpoint{2.773636in}{1.360554in}}%
\pgfpathlineto{\pgfqpoint{2.767302in}{1.372685in}}%
\pgfpathlineto{\pgfqpoint{2.760970in}{1.384827in}}%
\pgfpathlineto{\pgfqpoint{2.754642in}{1.396983in}}%
\pgfpathlineto{\pgfqpoint{2.748317in}{1.409147in}}%
\pgfpathlineto{\pgfqpoint{2.741995in}{1.421309in}}%
\pgfpathlineto{\pgfqpoint{2.730242in}{1.417288in}}%
\pgfpathlineto{\pgfqpoint{2.718484in}{1.413269in}}%
\pgfpathlineto{\pgfqpoint{2.706719in}{1.409252in}}%
\pgfpathlineto{\pgfqpoint{2.694949in}{1.405238in}}%
\pgfpathlineto{\pgfqpoint{2.683173in}{1.401229in}}%
\pgfpathlineto{\pgfqpoint{2.689487in}{1.389114in}}%
\pgfpathlineto{\pgfqpoint{2.695804in}{1.376993in}}%
\pgfpathlineto{\pgfqpoint{2.702124in}{1.364874in}}%
\pgfpathlineto{\pgfqpoint{2.708448in}{1.352761in}}%
\pgfpathclose%
\pgfusepath{stroke,fill}%
\end{pgfscope}%
\begin{pgfscope}%
\pgfpathrectangle{\pgfqpoint{0.887500in}{0.275000in}}{\pgfqpoint{4.225000in}{4.225000in}}%
\pgfusepath{clip}%
\pgfsetbuttcap%
\pgfsetroundjoin%
\definecolor{currentfill}{rgb}{0.275191,0.194905,0.496005}%
\pgfsetfillcolor{currentfill}%
\pgfsetfillopacity{0.700000}%
\pgfsetlinewidth{0.501875pt}%
\definecolor{currentstroke}{rgb}{1.000000,1.000000,1.000000}%
\pgfsetstrokecolor{currentstroke}%
\pgfsetstrokeopacity{0.500000}%
\pgfsetdash{}{0pt}%
\pgfpathmoveto{\pgfqpoint{2.805362in}{1.299380in}}%
\pgfpathlineto{\pgfqpoint{2.817124in}{1.303344in}}%
\pgfpathlineto{\pgfqpoint{2.828881in}{1.307323in}}%
\pgfpathlineto{\pgfqpoint{2.840632in}{1.311314in}}%
\pgfpathlineto{\pgfqpoint{2.852376in}{1.315307in}}%
\pgfpathlineto{\pgfqpoint{2.864115in}{1.319297in}}%
\pgfpathlineto{\pgfqpoint{2.857755in}{1.331757in}}%
\pgfpathlineto{\pgfqpoint{2.851399in}{1.344086in}}%
\pgfpathlineto{\pgfqpoint{2.845046in}{1.356315in}}%
\pgfpathlineto{\pgfqpoint{2.838697in}{1.368475in}}%
\pgfpathlineto{\pgfqpoint{2.832352in}{1.380597in}}%
\pgfpathlineto{\pgfqpoint{2.820620in}{1.376585in}}%
\pgfpathlineto{\pgfqpoint{2.808883in}{1.372572in}}%
\pgfpathlineto{\pgfqpoint{2.797140in}{1.368561in}}%
\pgfpathlineto{\pgfqpoint{2.785391in}{1.364554in}}%
\pgfpathlineto{\pgfqpoint{2.773636in}{1.360554in}}%
\pgfpathlineto{\pgfqpoint{2.779974in}{1.348414in}}%
\pgfpathlineto{\pgfqpoint{2.786316in}{1.336247in}}%
\pgfpathlineto{\pgfqpoint{2.792661in}{1.324032in}}%
\pgfpathlineto{\pgfqpoint{2.799009in}{1.311749in}}%
\pgfpathclose%
\pgfusepath{stroke,fill}%
\end{pgfscope}%
\begin{pgfscope}%
\pgfpathrectangle{\pgfqpoint{0.887500in}{0.275000in}}{\pgfqpoint{4.225000in}{4.225000in}}%
\pgfusepath{clip}%
\pgfsetbuttcap%
\pgfsetroundjoin%
\definecolor{currentfill}{rgb}{0.279574,0.170599,0.479997}%
\pgfsetfillcolor{currentfill}%
\pgfsetfillopacity{0.700000}%
\pgfsetlinewidth{0.501875pt}%
\definecolor{currentstroke}{rgb}{1.000000,1.000000,1.000000}%
\pgfsetstrokecolor{currentstroke}%
\pgfsetstrokeopacity{0.500000}%
\pgfsetdash{}{0pt}%
\pgfpathmoveto{\pgfqpoint{2.895978in}{1.253969in}}%
\pgfpathlineto{\pgfqpoint{2.907718in}{1.257669in}}%
\pgfpathlineto{\pgfqpoint{2.919453in}{1.261385in}}%
\pgfpathlineto{\pgfqpoint{2.931181in}{1.265120in}}%
\pgfpathlineto{\pgfqpoint{2.942904in}{1.268878in}}%
\pgfpathlineto{\pgfqpoint{2.954620in}{1.272658in}}%
\pgfpathlineto{\pgfqpoint{2.948233in}{1.286485in}}%
\pgfpathlineto{\pgfqpoint{2.941850in}{1.300006in}}%
\pgfpathlineto{\pgfqpoint{2.935471in}{1.313234in}}%
\pgfpathlineto{\pgfqpoint{2.929096in}{1.326203in}}%
\pgfpathlineto{\pgfqpoint{2.922724in}{1.338949in}}%
\pgfpathlineto{\pgfqpoint{2.911014in}{1.335073in}}%
\pgfpathlineto{\pgfqpoint{2.899298in}{1.331169in}}%
\pgfpathlineto{\pgfqpoint{2.887576in}{1.327235in}}%
\pgfpathlineto{\pgfqpoint{2.875849in}{1.323275in}}%
\pgfpathlineto{\pgfqpoint{2.864115in}{1.319297in}}%
\pgfpathlineto{\pgfqpoint{2.870480in}{1.306675in}}%
\pgfpathlineto{\pgfqpoint{2.876848in}{1.293860in}}%
\pgfpathlineto{\pgfqpoint{2.883221in}{1.280821in}}%
\pgfpathlineto{\pgfqpoint{2.889597in}{1.267528in}}%
\pgfpathclose%
\pgfusepath{stroke,fill}%
\end{pgfscope}%
\begin{pgfscope}%
\pgfpathrectangle{\pgfqpoint{0.887500in}{0.275000in}}{\pgfqpoint{4.225000in}{4.225000in}}%
\pgfusepath{clip}%
\pgfsetbuttcap%
\pgfsetroundjoin%
\definecolor{currentfill}{rgb}{0.223925,0.334994,0.548053}%
\pgfsetfillcolor{currentfill}%
\pgfsetfillopacity{0.700000}%
\pgfsetlinewidth{0.501875pt}%
\definecolor{currentstroke}{rgb}{1.000000,1.000000,1.000000}%
\pgfsetstrokecolor{currentstroke}%
\pgfsetstrokeopacity{0.500000}%
\pgfsetdash{}{0pt}%
\pgfpathmoveto{\pgfqpoint{2.112299in}{1.548930in}}%
\pgfpathlineto{\pgfqpoint{2.124233in}{1.553014in}}%
\pgfpathlineto{\pgfqpoint{2.136162in}{1.557095in}}%
\pgfpathlineto{\pgfqpoint{2.148084in}{1.561172in}}%
\pgfpathlineto{\pgfqpoint{2.160001in}{1.565247in}}%
\pgfpathlineto{\pgfqpoint{2.171912in}{1.569318in}}%
\pgfpathlineto{\pgfqpoint{2.165755in}{1.580422in}}%
\pgfpathlineto{\pgfqpoint{2.159602in}{1.591492in}}%
\pgfpathlineto{\pgfqpoint{2.153453in}{1.602529in}}%
\pgfpathlineto{\pgfqpoint{2.147309in}{1.613531in}}%
\pgfpathlineto{\pgfqpoint{2.141169in}{1.624499in}}%
\pgfpathlineto{\pgfqpoint{2.129268in}{1.620429in}}%
\pgfpathlineto{\pgfqpoint{2.117361in}{1.616354in}}%
\pgfpathlineto{\pgfqpoint{2.105449in}{1.612275in}}%
\pgfpathlineto{\pgfqpoint{2.093531in}{1.608190in}}%
\pgfpathlineto{\pgfqpoint{2.081607in}{1.604099in}}%
\pgfpathlineto{\pgfqpoint{2.087738in}{1.593124in}}%
\pgfpathlineto{\pgfqpoint{2.093872in}{1.582119in}}%
\pgfpathlineto{\pgfqpoint{2.100010in}{1.571084in}}%
\pgfpathlineto{\pgfqpoint{2.106153in}{1.560021in}}%
\pgfpathclose%
\pgfusepath{stroke,fill}%
\end{pgfscope}%
\begin{pgfscope}%
\pgfpathrectangle{\pgfqpoint{0.887500in}{0.275000in}}{\pgfqpoint{4.225000in}{4.225000in}}%
\pgfusepath{clip}%
\pgfsetbuttcap%
\pgfsetroundjoin%
\definecolor{currentfill}{rgb}{0.231674,0.318106,0.544834}%
\pgfsetfillcolor{currentfill}%
\pgfsetfillopacity{0.700000}%
\pgfsetlinewidth{0.501875pt}%
\definecolor{currentstroke}{rgb}{1.000000,1.000000,1.000000}%
\pgfsetstrokecolor{currentstroke}%
\pgfsetstrokeopacity{0.500000}%
\pgfsetdash{}{0pt}%
\pgfpathmoveto{\pgfqpoint{2.202758in}{1.513328in}}%
\pgfpathlineto{\pgfqpoint{2.214673in}{1.517383in}}%
\pgfpathlineto{\pgfqpoint{2.226583in}{1.521439in}}%
\pgfpathlineto{\pgfqpoint{2.238486in}{1.525496in}}%
\pgfpathlineto{\pgfqpoint{2.250384in}{1.529554in}}%
\pgfpathlineto{\pgfqpoint{2.262276in}{1.533614in}}%
\pgfpathlineto{\pgfqpoint{2.256088in}{1.544888in}}%
\pgfpathlineto{\pgfqpoint{2.249905in}{1.556130in}}%
\pgfpathlineto{\pgfqpoint{2.243725in}{1.567339in}}%
\pgfpathlineto{\pgfqpoint{2.237550in}{1.578514in}}%
\pgfpathlineto{\pgfqpoint{2.231379in}{1.589654in}}%
\pgfpathlineto{\pgfqpoint{2.219497in}{1.585588in}}%
\pgfpathlineto{\pgfqpoint{2.207610in}{1.581522in}}%
\pgfpathlineto{\pgfqpoint{2.195716in}{1.577456in}}%
\pgfpathlineto{\pgfqpoint{2.183817in}{1.573388in}}%
\pgfpathlineto{\pgfqpoint{2.171912in}{1.569318in}}%
\pgfpathlineto{\pgfqpoint{2.178073in}{1.558182in}}%
\pgfpathlineto{\pgfqpoint{2.184238in}{1.547015in}}%
\pgfpathlineto{\pgfqpoint{2.190407in}{1.535816in}}%
\pgfpathlineto{\pgfqpoint{2.196581in}{1.524587in}}%
\pgfpathclose%
\pgfusepath{stroke,fill}%
\end{pgfscope}%
\begin{pgfscope}%
\pgfpathrectangle{\pgfqpoint{0.887500in}{0.275000in}}{\pgfqpoint{4.225000in}{4.225000in}}%
\pgfusepath{clip}%
\pgfsetbuttcap%
\pgfsetroundjoin%
\definecolor{currentfill}{rgb}{0.239346,0.300855,0.540844}%
\pgfsetfillcolor{currentfill}%
\pgfsetfillopacity{0.700000}%
\pgfsetlinewidth{0.501875pt}%
\definecolor{currentstroke}{rgb}{1.000000,1.000000,1.000000}%
\pgfsetstrokecolor{currentstroke}%
\pgfsetstrokeopacity{0.500000}%
\pgfsetdash{}{0pt}%
\pgfpathmoveto{\pgfqpoint{2.293273in}{1.476788in}}%
\pgfpathlineto{\pgfqpoint{2.305168in}{1.480835in}}%
\pgfpathlineto{\pgfqpoint{2.317058in}{1.484885in}}%
\pgfpathlineto{\pgfqpoint{2.328942in}{1.488940in}}%
\pgfpathlineto{\pgfqpoint{2.340820in}{1.492999in}}%
\pgfpathlineto{\pgfqpoint{2.352692in}{1.497063in}}%
\pgfpathlineto{\pgfqpoint{2.346474in}{1.508505in}}%
\pgfpathlineto{\pgfqpoint{2.340261in}{1.519915in}}%
\pgfpathlineto{\pgfqpoint{2.334052in}{1.531293in}}%
\pgfpathlineto{\pgfqpoint{2.327847in}{1.542640in}}%
\pgfpathlineto{\pgfqpoint{2.321646in}{1.553953in}}%
\pgfpathlineto{\pgfqpoint{2.309784in}{1.549879in}}%
\pgfpathlineto{\pgfqpoint{2.297915in}{1.545808in}}%
\pgfpathlineto{\pgfqpoint{2.286041in}{1.541740in}}%
\pgfpathlineto{\pgfqpoint{2.274161in}{1.537676in}}%
\pgfpathlineto{\pgfqpoint{2.262276in}{1.533614in}}%
\pgfpathlineto{\pgfqpoint{2.268467in}{1.522308in}}%
\pgfpathlineto{\pgfqpoint{2.274663in}{1.510972in}}%
\pgfpathlineto{\pgfqpoint{2.280862in}{1.499606in}}%
\pgfpathlineto{\pgfqpoint{2.287065in}{1.488212in}}%
\pgfpathclose%
\pgfusepath{stroke,fill}%
\end{pgfscope}%
\begin{pgfscope}%
\pgfpathrectangle{\pgfqpoint{0.887500in}{0.275000in}}{\pgfqpoint{4.225000in}{4.225000in}}%
\pgfusepath{clip}%
\pgfsetbuttcap%
\pgfsetroundjoin%
\definecolor{currentfill}{rgb}{0.283091,0.110553,0.431554}%
\pgfsetfillcolor{currentfill}%
\pgfsetfillopacity{0.700000}%
\pgfsetlinewidth{0.501875pt}%
\definecolor{currentstroke}{rgb}{1.000000,1.000000,1.000000}%
\pgfsetstrokecolor{currentstroke}%
\pgfsetstrokeopacity{0.500000}%
\pgfsetdash{}{0pt}%
\pgfpathmoveto{\pgfqpoint{3.167912in}{1.165252in}}%
\pgfpathlineto{\pgfqpoint{3.179601in}{1.170427in}}%
\pgfpathlineto{\pgfqpoint{3.191285in}{1.175851in}}%
\pgfpathlineto{\pgfqpoint{3.202965in}{1.181457in}}%
\pgfpathlineto{\pgfqpoint{3.214639in}{1.186893in}}%
\pgfpathlineto{\pgfqpoint{3.226307in}{1.191776in}}%
\pgfpathlineto{\pgfqpoint{3.219823in}{1.198756in}}%
\pgfpathlineto{\pgfqpoint{3.213346in}{1.206093in}}%
\pgfpathlineto{\pgfqpoint{3.206876in}{1.213781in}}%
\pgfpathlineto{\pgfqpoint{3.200412in}{1.221818in}}%
\pgfpathlineto{\pgfqpoint{3.193953in}{1.230201in}}%
\pgfpathlineto{\pgfqpoint{3.182290in}{1.222068in}}%
\pgfpathlineto{\pgfqpoint{3.170624in}{1.213748in}}%
\pgfpathlineto{\pgfqpoint{3.158955in}{1.205747in}}%
\pgfpathlineto{\pgfqpoint{3.147284in}{1.198525in}}%
\pgfpathlineto{\pgfqpoint{3.135611in}{1.192135in}}%
\pgfpathlineto{\pgfqpoint{3.142054in}{1.184193in}}%
\pgfpathlineto{\pgfqpoint{3.148505in}{1.177422in}}%
\pgfpathlineto{\pgfqpoint{3.154964in}{1.171933in}}%
\pgfpathlineto{\pgfqpoint{3.161433in}{1.167838in}}%
\pgfpathclose%
\pgfusepath{stroke,fill}%
\end{pgfscope}%
\begin{pgfscope}%
\pgfpathrectangle{\pgfqpoint{0.887500in}{0.275000in}}{\pgfqpoint{4.225000in}{4.225000in}}%
\pgfusepath{clip}%
\pgfsetbuttcap%
\pgfsetroundjoin%
\definecolor{currentfill}{rgb}{0.246811,0.283237,0.535941}%
\pgfsetfillcolor{currentfill}%
\pgfsetfillopacity{0.700000}%
\pgfsetlinewidth{0.501875pt}%
\definecolor{currentstroke}{rgb}{1.000000,1.000000,1.000000}%
\pgfsetstrokecolor{currentstroke}%
\pgfsetstrokeopacity{0.500000}%
\pgfsetdash{}{0pt}%
\pgfpathmoveto{\pgfqpoint{2.383835in}{1.439391in}}%
\pgfpathlineto{\pgfqpoint{2.395711in}{1.443430in}}%
\pgfpathlineto{\pgfqpoint{2.407581in}{1.447474in}}%
\pgfpathlineto{\pgfqpoint{2.419444in}{1.451522in}}%
\pgfpathlineto{\pgfqpoint{2.431302in}{1.455574in}}%
\pgfpathlineto{\pgfqpoint{2.443154in}{1.459631in}}%
\pgfpathlineto{\pgfqpoint{2.436908in}{1.471262in}}%
\pgfpathlineto{\pgfqpoint{2.430666in}{1.482858in}}%
\pgfpathlineto{\pgfqpoint{2.424427in}{1.494418in}}%
\pgfpathlineto{\pgfqpoint{2.418193in}{1.505942in}}%
\pgfpathlineto{\pgfqpoint{2.411963in}{1.517431in}}%
\pgfpathlineto{\pgfqpoint{2.400120in}{1.513353in}}%
\pgfpathlineto{\pgfqpoint{2.388272in}{1.509276in}}%
\pgfpathlineto{\pgfqpoint{2.376418in}{1.505202in}}%
\pgfpathlineto{\pgfqpoint{2.364557in}{1.501131in}}%
\pgfpathlineto{\pgfqpoint{2.352692in}{1.497063in}}%
\pgfpathlineto{\pgfqpoint{2.358912in}{1.485590in}}%
\pgfpathlineto{\pgfqpoint{2.365137in}{1.474086in}}%
\pgfpathlineto{\pgfqpoint{2.371366in}{1.462552in}}%
\pgfpathlineto{\pgfqpoint{2.377599in}{1.450987in}}%
\pgfpathclose%
\pgfusepath{stroke,fill}%
\end{pgfscope}%
\begin{pgfscope}%
\pgfpathrectangle{\pgfqpoint{0.887500in}{0.275000in}}{\pgfqpoint{4.225000in}{4.225000in}}%
\pgfusepath{clip}%
\pgfsetbuttcap%
\pgfsetroundjoin%
\definecolor{currentfill}{rgb}{0.255645,0.260703,0.528312}%
\pgfsetfillcolor{currentfill}%
\pgfsetfillopacity{0.700000}%
\pgfsetlinewidth{0.501875pt}%
\definecolor{currentstroke}{rgb}{1.000000,1.000000,1.000000}%
\pgfsetstrokecolor{currentstroke}%
\pgfsetstrokeopacity{0.500000}%
\pgfsetdash{}{0pt}%
\pgfpathmoveto{\pgfqpoint{2.474443in}{1.400935in}}%
\pgfpathlineto{\pgfqpoint{2.486299in}{1.404950in}}%
\pgfpathlineto{\pgfqpoint{2.498148in}{1.408972in}}%
\pgfpathlineto{\pgfqpoint{2.509992in}{1.413002in}}%
\pgfpathlineto{\pgfqpoint{2.521829in}{1.417036in}}%
\pgfpathlineto{\pgfqpoint{2.533661in}{1.421074in}}%
\pgfpathlineto{\pgfqpoint{2.527386in}{1.432933in}}%
\pgfpathlineto{\pgfqpoint{2.521115in}{1.444754in}}%
\pgfpathlineto{\pgfqpoint{2.514848in}{1.456532in}}%
\pgfpathlineto{\pgfqpoint{2.508585in}{1.468269in}}%
\pgfpathlineto{\pgfqpoint{2.502326in}{1.479963in}}%
\pgfpathlineto{\pgfqpoint{2.490503in}{1.475893in}}%
\pgfpathlineto{\pgfqpoint{2.478675in}{1.471823in}}%
\pgfpathlineto{\pgfqpoint{2.466840in}{1.467756in}}%
\pgfpathlineto{\pgfqpoint{2.455000in}{1.463691in}}%
\pgfpathlineto{\pgfqpoint{2.443154in}{1.459631in}}%
\pgfpathlineto{\pgfqpoint{2.449404in}{1.447963in}}%
\pgfpathlineto{\pgfqpoint{2.455658in}{1.436260in}}%
\pgfpathlineto{\pgfqpoint{2.461916in}{1.424520in}}%
\pgfpathlineto{\pgfqpoint{2.468178in}{1.412744in}}%
\pgfpathclose%
\pgfusepath{stroke,fill}%
\end{pgfscope}%
\begin{pgfscope}%
\pgfpathrectangle{\pgfqpoint{0.887500in}{0.275000in}}{\pgfqpoint{4.225000in}{4.225000in}}%
\pgfusepath{clip}%
\pgfsetbuttcap%
\pgfsetroundjoin%
\definecolor{currentfill}{rgb}{0.263663,0.237631,0.518762}%
\pgfsetfillcolor{currentfill}%
\pgfsetfillopacity{0.700000}%
\pgfsetlinewidth{0.501875pt}%
\definecolor{currentstroke}{rgb}{1.000000,1.000000,1.000000}%
\pgfsetstrokecolor{currentstroke}%
\pgfsetstrokeopacity{0.500000}%
\pgfsetdash{}{0pt}%
\pgfpathmoveto{\pgfqpoint{2.565090in}{1.361351in}}%
\pgfpathlineto{\pgfqpoint{2.576924in}{1.365333in}}%
\pgfpathlineto{\pgfqpoint{2.588753in}{1.369317in}}%
\pgfpathlineto{\pgfqpoint{2.600577in}{1.373300in}}%
\pgfpathlineto{\pgfqpoint{2.612394in}{1.377283in}}%
\pgfpathlineto{\pgfqpoint{2.624205in}{1.381266in}}%
\pgfpathlineto{\pgfqpoint{2.617903in}{1.393313in}}%
\pgfpathlineto{\pgfqpoint{2.611605in}{1.405342in}}%
\pgfpathlineto{\pgfqpoint{2.605310in}{1.417348in}}%
\pgfpathlineto{\pgfqpoint{2.599019in}{1.429324in}}%
\pgfpathlineto{\pgfqpoint{2.592732in}{1.441265in}}%
\pgfpathlineto{\pgfqpoint{2.580929in}{1.437228in}}%
\pgfpathlineto{\pgfqpoint{2.569121in}{1.433191in}}%
\pgfpathlineto{\pgfqpoint{2.557307in}{1.429153in}}%
\pgfpathlineto{\pgfqpoint{2.545487in}{1.425113in}}%
\pgfpathlineto{\pgfqpoint{2.533661in}{1.421074in}}%
\pgfpathlineto{\pgfqpoint{2.539939in}{1.409180in}}%
\pgfpathlineto{\pgfqpoint{2.546222in}{1.397256in}}%
\pgfpathlineto{\pgfqpoint{2.552507in}{1.385307in}}%
\pgfpathlineto{\pgfqpoint{2.558797in}{1.373337in}}%
\pgfpathclose%
\pgfusepath{stroke,fill}%
\end{pgfscope}%
\begin{pgfscope}%
\pgfpathrectangle{\pgfqpoint{0.887500in}{0.275000in}}{\pgfqpoint{4.225000in}{4.225000in}}%
\pgfusepath{clip}%
\pgfsetbuttcap%
\pgfsetroundjoin%
\definecolor{currentfill}{rgb}{0.283091,0.110553,0.431554}%
\pgfsetfillcolor{currentfill}%
\pgfsetfillopacity{0.700000}%
\pgfsetlinewidth{0.501875pt}%
\definecolor{currentstroke}{rgb}{1.000000,1.000000,1.000000}%
\pgfsetstrokecolor{currentstroke}%
\pgfsetstrokeopacity{0.500000}%
\pgfsetdash{}{0pt}%
\pgfpathmoveto{\pgfqpoint{3.018629in}{1.144536in}}%
\pgfpathlineto{\pgfqpoint{3.030352in}{1.148921in}}%
\pgfpathlineto{\pgfqpoint{3.042069in}{1.153456in}}%
\pgfpathlineto{\pgfqpoint{3.053780in}{1.158075in}}%
\pgfpathlineto{\pgfqpoint{3.065486in}{1.162713in}}%
\pgfpathlineto{\pgfqpoint{3.077186in}{1.167304in}}%
\pgfpathlineto{\pgfqpoint{3.070763in}{1.176697in}}%
\pgfpathlineto{\pgfqpoint{3.064345in}{1.187121in}}%
\pgfpathlineto{\pgfqpoint{3.057930in}{1.198444in}}%
\pgfpathlineto{\pgfqpoint{3.051519in}{1.210533in}}%
\pgfpathlineto{\pgfqpoint{3.045111in}{1.223256in}}%
\pgfpathlineto{\pgfqpoint{3.033419in}{1.219154in}}%
\pgfpathlineto{\pgfqpoint{3.021722in}{1.215048in}}%
\pgfpathlineto{\pgfqpoint{3.010019in}{1.210971in}}%
\pgfpathlineto{\pgfqpoint{2.998310in}{1.206956in}}%
\pgfpathlineto{\pgfqpoint{2.986595in}{1.203034in}}%
\pgfpathlineto{\pgfqpoint{2.992998in}{1.189923in}}%
\pgfpathlineto{\pgfqpoint{2.999402in}{1.177388in}}%
\pgfpathlineto{\pgfqpoint{3.005808in}{1.165558in}}%
\pgfpathlineto{\pgfqpoint{3.012217in}{1.154564in}}%
\pgfpathclose%
\pgfusepath{stroke,fill}%
\end{pgfscope}%
\begin{pgfscope}%
\pgfpathrectangle{\pgfqpoint{0.887500in}{0.275000in}}{\pgfqpoint{4.225000in}{4.225000in}}%
\pgfusepath{clip}%
\pgfsetbuttcap%
\pgfsetroundjoin%
\definecolor{currentfill}{rgb}{0.282884,0.135920,0.453427}%
\pgfsetfillcolor{currentfill}%
\pgfsetfillopacity{0.700000}%
\pgfsetlinewidth{0.501875pt}%
\definecolor{currentstroke}{rgb}{1.000000,1.000000,1.000000}%
\pgfsetstrokecolor{currentstroke}%
\pgfsetstrokeopacity{0.500000}%
\pgfsetdash{}{0pt}%
\pgfpathmoveto{\pgfqpoint{2.927926in}{1.185642in}}%
\pgfpathlineto{\pgfqpoint{2.939673in}{1.188836in}}%
\pgfpathlineto{\pgfqpoint{2.951413in}{1.192144in}}%
\pgfpathlineto{\pgfqpoint{2.963147in}{1.195602in}}%
\pgfpathlineto{\pgfqpoint{2.974875in}{1.199239in}}%
\pgfpathlineto{\pgfqpoint{2.986595in}{1.203034in}}%
\pgfpathlineto{\pgfqpoint{2.980196in}{1.216590in}}%
\pgfpathlineto{\pgfqpoint{2.973798in}{1.230461in}}%
\pgfpathlineto{\pgfqpoint{2.967403in}{1.244516in}}%
\pgfpathlineto{\pgfqpoint{2.961010in}{1.258625in}}%
\pgfpathlineto{\pgfqpoint{2.954620in}{1.272658in}}%
\pgfpathlineto{\pgfqpoint{2.942904in}{1.268878in}}%
\pgfpathlineto{\pgfqpoint{2.931181in}{1.265120in}}%
\pgfpathlineto{\pgfqpoint{2.919453in}{1.261385in}}%
\pgfpathlineto{\pgfqpoint{2.907718in}{1.257669in}}%
\pgfpathlineto{\pgfqpoint{2.895978in}{1.253969in}}%
\pgfpathlineto{\pgfqpoint{2.902362in}{1.240228in}}%
\pgfpathlineto{\pgfqpoint{2.908749in}{1.226415in}}%
\pgfpathlineto{\pgfqpoint{2.915139in}{1.212639in}}%
\pgfpathlineto{\pgfqpoint{2.921532in}{1.199012in}}%
\pgfpathclose%
\pgfusepath{stroke,fill}%
\end{pgfscope}%
\begin{pgfscope}%
\pgfpathrectangle{\pgfqpoint{0.887500in}{0.275000in}}{\pgfqpoint{4.225000in}{4.225000in}}%
\pgfusepath{clip}%
\pgfsetbuttcap%
\pgfsetroundjoin%
\definecolor{currentfill}{rgb}{0.269308,0.218818,0.509577}%
\pgfsetfillcolor{currentfill}%
\pgfsetfillopacity{0.700000}%
\pgfsetlinewidth{0.501875pt}%
\definecolor{currentstroke}{rgb}{1.000000,1.000000,1.000000}%
\pgfsetstrokecolor{currentstroke}%
\pgfsetstrokeopacity{0.500000}%
\pgfsetdash{}{0pt}%
\pgfpathmoveto{\pgfqpoint{2.655764in}{1.320929in}}%
\pgfpathlineto{\pgfqpoint{2.667578in}{1.324862in}}%
\pgfpathlineto{\pgfqpoint{2.679386in}{1.328800in}}%
\pgfpathlineto{\pgfqpoint{2.691188in}{1.332745in}}%
\pgfpathlineto{\pgfqpoint{2.702984in}{1.336696in}}%
\pgfpathlineto{\pgfqpoint{2.714774in}{1.340654in}}%
\pgfpathlineto{\pgfqpoint{2.708448in}{1.352761in}}%
\pgfpathlineto{\pgfqpoint{2.702124in}{1.364874in}}%
\pgfpathlineto{\pgfqpoint{2.695804in}{1.376993in}}%
\pgfpathlineto{\pgfqpoint{2.689487in}{1.389114in}}%
\pgfpathlineto{\pgfqpoint{2.683173in}{1.401229in}}%
\pgfpathlineto{\pgfqpoint{2.671392in}{1.397226in}}%
\pgfpathlineto{\pgfqpoint{2.659604in}{1.393229in}}%
\pgfpathlineto{\pgfqpoint{2.647810in}{1.389237in}}%
\pgfpathlineto{\pgfqpoint{2.636011in}{1.385250in}}%
\pgfpathlineto{\pgfqpoint{2.624205in}{1.381266in}}%
\pgfpathlineto{\pgfqpoint{2.630510in}{1.369205in}}%
\pgfpathlineto{\pgfqpoint{2.636819in}{1.357136in}}%
\pgfpathlineto{\pgfqpoint{2.643131in}{1.345065in}}%
\pgfpathlineto{\pgfqpoint{2.649446in}{1.332996in}}%
\pgfpathclose%
\pgfusepath{stroke,fill}%
\end{pgfscope}%
\begin{pgfscope}%
\pgfpathrectangle{\pgfqpoint{0.887500in}{0.275000in}}{\pgfqpoint{4.225000in}{4.225000in}}%
\pgfusepath{clip}%
\pgfsetbuttcap%
\pgfsetroundjoin%
\definecolor{currentfill}{rgb}{0.282656,0.100196,0.422160}%
\pgfsetfillcolor{currentfill}%
\pgfsetfillopacity{0.700000}%
\pgfsetlinewidth{0.501875pt}%
\definecolor{currentstroke}{rgb}{1.000000,1.000000,1.000000}%
\pgfsetstrokecolor{currentstroke}%
\pgfsetstrokeopacity{0.500000}%
\pgfsetdash{}{0pt}%
\pgfpathmoveto{\pgfqpoint{3.258831in}{1.162341in}}%
\pgfpathlineto{\pgfqpoint{3.270443in}{1.155138in}}%
\pgfpathlineto{\pgfqpoint{3.282039in}{1.147413in}}%
\pgfpathlineto{\pgfqpoint{3.293622in}{1.139510in}}%
\pgfpathlineto{\pgfqpoint{3.305192in}{1.131769in}}%
\pgfpathlineto{\pgfqpoint{3.316752in}{1.124403in}}%
\pgfpathlineto{\pgfqpoint{3.310294in}{1.140069in}}%
\pgfpathlineto{\pgfqpoint{3.303832in}{1.154882in}}%
\pgfpathlineto{\pgfqpoint{3.297369in}{1.168907in}}%
\pgfpathlineto{\pgfqpoint{3.290903in}{1.182212in}}%
\pgfpathlineto{\pgfqpoint{3.284437in}{1.194865in}}%
\pgfpathlineto{\pgfqpoint{3.272848in}{1.198118in}}%
\pgfpathlineto{\pgfqpoint{3.261238in}{1.199272in}}%
\pgfpathlineto{\pgfqpoint{3.249609in}{1.198349in}}%
\pgfpathlineto{\pgfqpoint{3.237964in}{1.195722in}}%
\pgfpathlineto{\pgfqpoint{3.226307in}{1.191776in}}%
\pgfpathlineto{\pgfqpoint{3.232797in}{1.185155in}}%
\pgfpathlineto{\pgfqpoint{3.239294in}{1.178897in}}%
\pgfpathlineto{\pgfqpoint{3.245799in}{1.173007in}}%
\pgfpathlineto{\pgfqpoint{3.252311in}{1.167487in}}%
\pgfpathclose%
\pgfusepath{stroke,fill}%
\end{pgfscope}%
\begin{pgfscope}%
\pgfpathrectangle{\pgfqpoint{0.887500in}{0.275000in}}{\pgfqpoint{4.225000in}{4.225000in}}%
\pgfusepath{clip}%
\pgfsetbuttcap%
\pgfsetroundjoin%
\definecolor{currentfill}{rgb}{0.275191,0.194905,0.496005}%
\pgfsetfillcolor{currentfill}%
\pgfsetfillopacity{0.700000}%
\pgfsetlinewidth{0.501875pt}%
\definecolor{currentstroke}{rgb}{1.000000,1.000000,1.000000}%
\pgfsetstrokecolor{currentstroke}%
\pgfsetstrokeopacity{0.500000}%
\pgfsetdash{}{0pt}%
\pgfpathmoveto{\pgfqpoint{2.746460in}{1.279760in}}%
\pgfpathlineto{\pgfqpoint{2.758252in}{1.283662in}}%
\pgfpathlineto{\pgfqpoint{2.770039in}{1.287573in}}%
\pgfpathlineto{\pgfqpoint{2.781819in}{1.291496in}}%
\pgfpathlineto{\pgfqpoint{2.793593in}{1.295431in}}%
\pgfpathlineto{\pgfqpoint{2.805362in}{1.299380in}}%
\pgfpathlineto{\pgfqpoint{2.799009in}{1.311749in}}%
\pgfpathlineto{\pgfqpoint{2.792661in}{1.324032in}}%
\pgfpathlineto{\pgfqpoint{2.786316in}{1.336247in}}%
\pgfpathlineto{\pgfqpoint{2.779974in}{1.348414in}}%
\pgfpathlineto{\pgfqpoint{2.773636in}{1.360554in}}%
\pgfpathlineto{\pgfqpoint{2.761876in}{1.356560in}}%
\pgfpathlineto{\pgfqpoint{2.750109in}{1.352573in}}%
\pgfpathlineto{\pgfqpoint{2.738337in}{1.348593in}}%
\pgfpathlineto{\pgfqpoint{2.726558in}{1.344620in}}%
\pgfpathlineto{\pgfqpoint{2.714774in}{1.340654in}}%
\pgfpathlineto{\pgfqpoint{2.721104in}{1.328540in}}%
\pgfpathlineto{\pgfqpoint{2.727438in}{1.316407in}}%
\pgfpathlineto{\pgfqpoint{2.733775in}{1.304241in}}%
\pgfpathlineto{\pgfqpoint{2.740115in}{1.292029in}}%
\pgfpathclose%
\pgfusepath{stroke,fill}%
\end{pgfscope}%
\begin{pgfscope}%
\pgfpathrectangle{\pgfqpoint{0.887500in}{0.275000in}}{\pgfqpoint{4.225000in}{4.225000in}}%
\pgfusepath{clip}%
\pgfsetbuttcap%
\pgfsetroundjoin%
\definecolor{currentfill}{rgb}{0.280255,0.165693,0.476498}%
\pgfsetfillcolor{currentfill}%
\pgfsetfillopacity{0.700000}%
\pgfsetlinewidth{0.501875pt}%
\definecolor{currentstroke}{rgb}{1.000000,1.000000,1.000000}%
\pgfsetstrokecolor{currentstroke}%
\pgfsetstrokeopacity{0.500000}%
\pgfsetdash{}{0pt}%
\pgfpathmoveto{\pgfqpoint{2.837184in}{1.235551in}}%
\pgfpathlineto{\pgfqpoint{2.848955in}{1.239237in}}%
\pgfpathlineto{\pgfqpoint{2.860720in}{1.242919in}}%
\pgfpathlineto{\pgfqpoint{2.872479in}{1.246598in}}%
\pgfpathlineto{\pgfqpoint{2.884231in}{1.250280in}}%
\pgfpathlineto{\pgfqpoint{2.895978in}{1.253969in}}%
\pgfpathlineto{\pgfqpoint{2.889597in}{1.267528in}}%
\pgfpathlineto{\pgfqpoint{2.883221in}{1.280821in}}%
\pgfpathlineto{\pgfqpoint{2.876848in}{1.293860in}}%
\pgfpathlineto{\pgfqpoint{2.870480in}{1.306675in}}%
\pgfpathlineto{\pgfqpoint{2.864115in}{1.319297in}}%
\pgfpathlineto{\pgfqpoint{2.852376in}{1.315307in}}%
\pgfpathlineto{\pgfqpoint{2.840632in}{1.311314in}}%
\pgfpathlineto{\pgfqpoint{2.828881in}{1.307323in}}%
\pgfpathlineto{\pgfqpoint{2.817124in}{1.303344in}}%
\pgfpathlineto{\pgfqpoint{2.805362in}{1.299380in}}%
\pgfpathlineto{\pgfqpoint{2.811718in}{1.286904in}}%
\pgfpathlineto{\pgfqpoint{2.818079in}{1.274301in}}%
\pgfpathlineto{\pgfqpoint{2.824443in}{1.261552in}}%
\pgfpathlineto{\pgfqpoint{2.830812in}{1.248637in}}%
\pgfpathclose%
\pgfusepath{stroke,fill}%
\end{pgfscope}%
\begin{pgfscope}%
\pgfpathrectangle{\pgfqpoint{0.887500in}{0.275000in}}{\pgfqpoint{4.225000in}{4.225000in}}%
\pgfusepath{clip}%
\pgfsetbuttcap%
\pgfsetroundjoin%
\definecolor{currentfill}{rgb}{0.231674,0.318106,0.544834}%
\pgfsetfillcolor{currentfill}%
\pgfsetfillopacity{0.700000}%
\pgfsetlinewidth{0.501875pt}%
\definecolor{currentstroke}{rgb}{1.000000,1.000000,1.000000}%
\pgfsetstrokecolor{currentstroke}%
\pgfsetstrokeopacity{0.500000}%
\pgfsetdash{}{0pt}%
\pgfpathmoveto{\pgfqpoint{2.143093in}{1.493046in}}%
\pgfpathlineto{\pgfqpoint{2.155038in}{1.497104in}}%
\pgfpathlineto{\pgfqpoint{2.166977in}{1.501161in}}%
\pgfpathlineto{\pgfqpoint{2.178910in}{1.505217in}}%
\pgfpathlineto{\pgfqpoint{2.190837in}{1.509273in}}%
\pgfpathlineto{\pgfqpoint{2.202758in}{1.513328in}}%
\pgfpathlineto{\pgfqpoint{2.196581in}{1.524587in}}%
\pgfpathlineto{\pgfqpoint{2.190407in}{1.535816in}}%
\pgfpathlineto{\pgfqpoint{2.184238in}{1.547015in}}%
\pgfpathlineto{\pgfqpoint{2.178073in}{1.558182in}}%
\pgfpathlineto{\pgfqpoint{2.171912in}{1.569318in}}%
\pgfpathlineto{\pgfqpoint{2.160001in}{1.565247in}}%
\pgfpathlineto{\pgfqpoint{2.148084in}{1.561172in}}%
\pgfpathlineto{\pgfqpoint{2.136162in}{1.557095in}}%
\pgfpathlineto{\pgfqpoint{2.124233in}{1.553014in}}%
\pgfpathlineto{\pgfqpoint{2.112299in}{1.548930in}}%
\pgfpathlineto{\pgfqpoint{2.118450in}{1.537809in}}%
\pgfpathlineto{\pgfqpoint{2.124605in}{1.526661in}}%
\pgfpathlineto{\pgfqpoint{2.130764in}{1.515484in}}%
\pgfpathlineto{\pgfqpoint{2.136926in}{1.504279in}}%
\pgfpathclose%
\pgfusepath{stroke,fill}%
\end{pgfscope}%
\begin{pgfscope}%
\pgfpathrectangle{\pgfqpoint{0.887500in}{0.275000in}}{\pgfqpoint{4.225000in}{4.225000in}}%
\pgfusepath{clip}%
\pgfsetbuttcap%
\pgfsetroundjoin%
\definecolor{currentfill}{rgb}{0.282656,0.100196,0.422160}%
\pgfsetfillcolor{currentfill}%
\pgfsetfillopacity{0.700000}%
\pgfsetlinewidth{0.501875pt}%
\definecolor{currentstroke}{rgb}{1.000000,1.000000,1.000000}%
\pgfsetstrokecolor{currentstroke}%
\pgfsetstrokeopacity{0.500000}%
\pgfsetdash{}{0pt}%
\pgfpathmoveto{\pgfqpoint{3.109395in}{1.140437in}}%
\pgfpathlineto{\pgfqpoint{3.121109in}{1.145513in}}%
\pgfpathlineto{\pgfqpoint{3.132818in}{1.150456in}}%
\pgfpathlineto{\pgfqpoint{3.144521in}{1.155343in}}%
\pgfpathlineto{\pgfqpoint{3.156219in}{1.160249in}}%
\pgfpathlineto{\pgfqpoint{3.167912in}{1.165252in}}%
\pgfpathlineto{\pgfqpoint{3.161433in}{1.167838in}}%
\pgfpathlineto{\pgfqpoint{3.154964in}{1.171933in}}%
\pgfpathlineto{\pgfqpoint{3.148505in}{1.177422in}}%
\pgfpathlineto{\pgfqpoint{3.142054in}{1.184193in}}%
\pgfpathlineto{\pgfqpoint{3.135611in}{1.192135in}}%
\pgfpathlineto{\pgfqpoint{3.123935in}{1.186424in}}%
\pgfpathlineto{\pgfqpoint{3.112256in}{1.181240in}}%
\pgfpathlineto{\pgfqpoint{3.100571in}{1.176427in}}%
\pgfpathlineto{\pgfqpoint{3.088881in}{1.171833in}}%
\pgfpathlineto{\pgfqpoint{3.077186in}{1.167304in}}%
\pgfpathlineto{\pgfqpoint{3.083615in}{1.159074in}}%
\pgfpathlineto{\pgfqpoint{3.090049in}{1.152139in}}%
\pgfpathlineto{\pgfqpoint{3.096490in}{1.146632in}}%
\pgfpathlineto{\pgfqpoint{3.102938in}{1.142687in}}%
\pgfpathclose%
\pgfusepath{stroke,fill}%
\end{pgfscope}%
\begin{pgfscope}%
\pgfpathrectangle{\pgfqpoint{0.887500in}{0.275000in}}{\pgfqpoint{4.225000in}{4.225000in}}%
\pgfusepath{clip}%
\pgfsetbuttcap%
\pgfsetroundjoin%
\definecolor{currentfill}{rgb}{0.241237,0.296485,0.539709}%
\pgfsetfillcolor{currentfill}%
\pgfsetfillopacity{0.700000}%
\pgfsetlinewidth{0.501875pt}%
\definecolor{currentstroke}{rgb}{1.000000,1.000000,1.000000}%
\pgfsetstrokecolor{currentstroke}%
\pgfsetstrokeopacity{0.500000}%
\pgfsetdash{}{0pt}%
\pgfpathmoveto{\pgfqpoint{2.233705in}{1.456603in}}%
\pgfpathlineto{\pgfqpoint{2.245630in}{1.460635in}}%
\pgfpathlineto{\pgfqpoint{2.257550in}{1.464669in}}%
\pgfpathlineto{\pgfqpoint{2.269463in}{1.468706in}}%
\pgfpathlineto{\pgfqpoint{2.281371in}{1.472745in}}%
\pgfpathlineto{\pgfqpoint{2.293273in}{1.476788in}}%
\pgfpathlineto{\pgfqpoint{2.287065in}{1.488212in}}%
\pgfpathlineto{\pgfqpoint{2.280862in}{1.499606in}}%
\pgfpathlineto{\pgfqpoint{2.274663in}{1.510972in}}%
\pgfpathlineto{\pgfqpoint{2.268467in}{1.522308in}}%
\pgfpathlineto{\pgfqpoint{2.262276in}{1.533614in}}%
\pgfpathlineto{\pgfqpoint{2.250384in}{1.529554in}}%
\pgfpathlineto{\pgfqpoint{2.238486in}{1.525496in}}%
\pgfpathlineto{\pgfqpoint{2.226583in}{1.521439in}}%
\pgfpathlineto{\pgfqpoint{2.214673in}{1.517383in}}%
\pgfpathlineto{\pgfqpoint{2.202758in}{1.513328in}}%
\pgfpathlineto{\pgfqpoint{2.208940in}{1.502039in}}%
\pgfpathlineto{\pgfqpoint{2.215125in}{1.490722in}}%
\pgfpathlineto{\pgfqpoint{2.221314in}{1.479376in}}%
\pgfpathlineto{\pgfqpoint{2.227508in}{1.468003in}}%
\pgfpathclose%
\pgfusepath{stroke,fill}%
\end{pgfscope}%
\begin{pgfscope}%
\pgfpathrectangle{\pgfqpoint{0.887500in}{0.275000in}}{\pgfqpoint{4.225000in}{4.225000in}}%
\pgfusepath{clip}%
\pgfsetbuttcap%
\pgfsetroundjoin%
\definecolor{currentfill}{rgb}{0.248629,0.278775,0.534556}%
\pgfsetfillcolor{currentfill}%
\pgfsetfillopacity{0.700000}%
\pgfsetlinewidth{0.501875pt}%
\definecolor{currentstroke}{rgb}{1.000000,1.000000,1.000000}%
\pgfsetstrokecolor{currentstroke}%
\pgfsetstrokeopacity{0.500000}%
\pgfsetdash{}{0pt}%
\pgfpathmoveto{\pgfqpoint{2.324367in}{1.419262in}}%
\pgfpathlineto{\pgfqpoint{2.336273in}{1.423279in}}%
\pgfpathlineto{\pgfqpoint{2.348173in}{1.427301in}}%
\pgfpathlineto{\pgfqpoint{2.360066in}{1.431326in}}%
\pgfpathlineto{\pgfqpoint{2.371954in}{1.435356in}}%
\pgfpathlineto{\pgfqpoint{2.383835in}{1.439391in}}%
\pgfpathlineto{\pgfqpoint{2.377599in}{1.450987in}}%
\pgfpathlineto{\pgfqpoint{2.371366in}{1.462552in}}%
\pgfpathlineto{\pgfqpoint{2.365137in}{1.474086in}}%
\pgfpathlineto{\pgfqpoint{2.358912in}{1.485590in}}%
\pgfpathlineto{\pgfqpoint{2.352692in}{1.497063in}}%
\pgfpathlineto{\pgfqpoint{2.340820in}{1.492999in}}%
\pgfpathlineto{\pgfqpoint{2.328942in}{1.488940in}}%
\pgfpathlineto{\pgfqpoint{2.317058in}{1.484885in}}%
\pgfpathlineto{\pgfqpoint{2.305168in}{1.480835in}}%
\pgfpathlineto{\pgfqpoint{2.293273in}{1.476788in}}%
\pgfpathlineto{\pgfqpoint{2.299484in}{1.465337in}}%
\pgfpathlineto{\pgfqpoint{2.305699in}{1.453858in}}%
\pgfpathlineto{\pgfqpoint{2.311918in}{1.442353in}}%
\pgfpathlineto{\pgfqpoint{2.318141in}{1.430820in}}%
\pgfpathclose%
\pgfusepath{stroke,fill}%
\end{pgfscope}%
\begin{pgfscope}%
\pgfpathrectangle{\pgfqpoint{0.887500in}{0.275000in}}{\pgfqpoint{4.225000in}{4.225000in}}%
\pgfusepath{clip}%
\pgfsetbuttcap%
\pgfsetroundjoin%
\definecolor{currentfill}{rgb}{0.255645,0.260703,0.528312}%
\pgfsetfillcolor{currentfill}%
\pgfsetfillopacity{0.700000}%
\pgfsetlinewidth{0.501875pt}%
\definecolor{currentstroke}{rgb}{1.000000,1.000000,1.000000}%
\pgfsetstrokecolor{currentstroke}%
\pgfsetstrokeopacity{0.500000}%
\pgfsetdash{}{0pt}%
\pgfpathmoveto{\pgfqpoint{2.415076in}{1.380964in}}%
\pgfpathlineto{\pgfqpoint{2.426961in}{1.384947in}}%
\pgfpathlineto{\pgfqpoint{2.438841in}{1.388935in}}%
\pgfpathlineto{\pgfqpoint{2.450714in}{1.392928in}}%
\pgfpathlineto{\pgfqpoint{2.462582in}{1.396928in}}%
\pgfpathlineto{\pgfqpoint{2.474443in}{1.400935in}}%
\pgfpathlineto{\pgfqpoint{2.468178in}{1.412744in}}%
\pgfpathlineto{\pgfqpoint{2.461916in}{1.424520in}}%
\pgfpathlineto{\pgfqpoint{2.455658in}{1.436260in}}%
\pgfpathlineto{\pgfqpoint{2.449404in}{1.447963in}}%
\pgfpathlineto{\pgfqpoint{2.443154in}{1.459631in}}%
\pgfpathlineto{\pgfqpoint{2.431302in}{1.455574in}}%
\pgfpathlineto{\pgfqpoint{2.419444in}{1.451522in}}%
\pgfpathlineto{\pgfqpoint{2.407581in}{1.447474in}}%
\pgfpathlineto{\pgfqpoint{2.395711in}{1.443430in}}%
\pgfpathlineto{\pgfqpoint{2.383835in}{1.439391in}}%
\pgfpathlineto{\pgfqpoint{2.390076in}{1.427765in}}%
\pgfpathlineto{\pgfqpoint{2.396320in}{1.416109in}}%
\pgfpathlineto{\pgfqpoint{2.402568in}{1.404422in}}%
\pgfpathlineto{\pgfqpoint{2.408820in}{1.392707in}}%
\pgfpathclose%
\pgfusepath{stroke,fill}%
\end{pgfscope}%
\begin{pgfscope}%
\pgfpathrectangle{\pgfqpoint{0.887500in}{0.275000in}}{\pgfqpoint{4.225000in}{4.225000in}}%
\pgfusepath{clip}%
\pgfsetbuttcap%
\pgfsetroundjoin%
\definecolor{currentfill}{rgb}{0.263663,0.237631,0.518762}%
\pgfsetfillcolor{currentfill}%
\pgfsetfillopacity{0.700000}%
\pgfsetlinewidth{0.501875pt}%
\definecolor{currentstroke}{rgb}{1.000000,1.000000,1.000000}%
\pgfsetstrokecolor{currentstroke}%
\pgfsetstrokeopacity{0.500000}%
\pgfsetdash{}{0pt}%
\pgfpathmoveto{\pgfqpoint{2.505825in}{1.341533in}}%
\pgfpathlineto{\pgfqpoint{2.517690in}{1.345477in}}%
\pgfpathlineto{\pgfqpoint{2.529549in}{1.349433in}}%
\pgfpathlineto{\pgfqpoint{2.541402in}{1.353399in}}%
\pgfpathlineto{\pgfqpoint{2.553249in}{1.357372in}}%
\pgfpathlineto{\pgfqpoint{2.565090in}{1.361351in}}%
\pgfpathlineto{\pgfqpoint{2.558797in}{1.373337in}}%
\pgfpathlineto{\pgfqpoint{2.552507in}{1.385307in}}%
\pgfpathlineto{\pgfqpoint{2.546222in}{1.397256in}}%
\pgfpathlineto{\pgfqpoint{2.539939in}{1.409180in}}%
\pgfpathlineto{\pgfqpoint{2.533661in}{1.421074in}}%
\pgfpathlineto{\pgfqpoint{2.521829in}{1.417036in}}%
\pgfpathlineto{\pgfqpoint{2.509992in}{1.413002in}}%
\pgfpathlineto{\pgfqpoint{2.498148in}{1.408972in}}%
\pgfpathlineto{\pgfqpoint{2.486299in}{1.404950in}}%
\pgfpathlineto{\pgfqpoint{2.474443in}{1.400935in}}%
\pgfpathlineto{\pgfqpoint{2.480712in}{1.389097in}}%
\pgfpathlineto{\pgfqpoint{2.486985in}{1.377233in}}%
\pgfpathlineto{\pgfqpoint{2.493262in}{1.365349in}}%
\pgfpathlineto{\pgfqpoint{2.499542in}{1.353448in}}%
\pgfpathclose%
\pgfusepath{stroke,fill}%
\end{pgfscope}%
\begin{pgfscope}%
\pgfpathrectangle{\pgfqpoint{0.887500in}{0.275000in}}{\pgfqpoint{4.225000in}{4.225000in}}%
\pgfusepath{clip}%
\pgfsetbuttcap%
\pgfsetroundjoin%
\definecolor{currentfill}{rgb}{0.282910,0.105393,0.426902}%
\pgfsetfillcolor{currentfill}%
\pgfsetfillopacity{0.700000}%
\pgfsetlinewidth{0.501875pt}%
\definecolor{currentstroke}{rgb}{1.000000,1.000000,1.000000}%
\pgfsetstrokecolor{currentstroke}%
\pgfsetstrokeopacity{0.500000}%
\pgfsetdash{}{0pt}%
\pgfpathmoveto{\pgfqpoint{2.959922in}{1.126514in}}%
\pgfpathlineto{\pgfqpoint{2.971677in}{1.129605in}}%
\pgfpathlineto{\pgfqpoint{2.983425in}{1.132902in}}%
\pgfpathlineto{\pgfqpoint{2.995166in}{1.136473in}}%
\pgfpathlineto{\pgfqpoint{3.006901in}{1.140364in}}%
\pgfpathlineto{\pgfqpoint{3.018629in}{1.144536in}}%
\pgfpathlineto{\pgfqpoint{3.012217in}{1.154564in}}%
\pgfpathlineto{\pgfqpoint{3.005808in}{1.165558in}}%
\pgfpathlineto{\pgfqpoint{2.999402in}{1.177388in}}%
\pgfpathlineto{\pgfqpoint{2.992998in}{1.189923in}}%
\pgfpathlineto{\pgfqpoint{2.986595in}{1.203034in}}%
\pgfpathlineto{\pgfqpoint{2.974875in}{1.199239in}}%
\pgfpathlineto{\pgfqpoint{2.963147in}{1.195602in}}%
\pgfpathlineto{\pgfqpoint{2.951413in}{1.192144in}}%
\pgfpathlineto{\pgfqpoint{2.939673in}{1.188836in}}%
\pgfpathlineto{\pgfqpoint{2.927926in}{1.185642in}}%
\pgfpathlineto{\pgfqpoint{2.934322in}{1.172640in}}%
\pgfpathlineto{\pgfqpoint{2.940720in}{1.160117in}}%
\pgfpathlineto{\pgfqpoint{2.947119in}{1.148181in}}%
\pgfpathlineto{\pgfqpoint{2.953520in}{1.136943in}}%
\pgfpathclose%
\pgfusepath{stroke,fill}%
\end{pgfscope}%
\begin{pgfscope}%
\pgfpathrectangle{\pgfqpoint{0.887500in}{0.275000in}}{\pgfqpoint{4.225000in}{4.225000in}}%
\pgfusepath{clip}%
\pgfsetbuttcap%
\pgfsetroundjoin%
\definecolor{currentfill}{rgb}{0.270595,0.214069,0.507052}%
\pgfsetfillcolor{currentfill}%
\pgfsetfillopacity{0.700000}%
\pgfsetlinewidth{0.501875pt}%
\definecolor{currentstroke}{rgb}{1.000000,1.000000,1.000000}%
\pgfsetstrokecolor{currentstroke}%
\pgfsetstrokeopacity{0.500000}%
\pgfsetdash{}{0pt}%
\pgfpathmoveto{\pgfqpoint{2.596604in}{1.301329in}}%
\pgfpathlineto{\pgfqpoint{2.608448in}{1.305240in}}%
\pgfpathlineto{\pgfqpoint{2.620286in}{1.309156in}}%
\pgfpathlineto{\pgfqpoint{2.632118in}{1.313077in}}%
\pgfpathlineto{\pgfqpoint{2.643944in}{1.317001in}}%
\pgfpathlineto{\pgfqpoint{2.655764in}{1.320929in}}%
\pgfpathlineto{\pgfqpoint{2.649446in}{1.332996in}}%
\pgfpathlineto{\pgfqpoint{2.643131in}{1.345065in}}%
\pgfpathlineto{\pgfqpoint{2.636819in}{1.357136in}}%
\pgfpathlineto{\pgfqpoint{2.630510in}{1.369205in}}%
\pgfpathlineto{\pgfqpoint{2.624205in}{1.381266in}}%
\pgfpathlineto{\pgfqpoint{2.612394in}{1.377283in}}%
\pgfpathlineto{\pgfqpoint{2.600577in}{1.373300in}}%
\pgfpathlineto{\pgfqpoint{2.588753in}{1.369317in}}%
\pgfpathlineto{\pgfqpoint{2.576924in}{1.365333in}}%
\pgfpathlineto{\pgfqpoint{2.565090in}{1.361351in}}%
\pgfpathlineto{\pgfqpoint{2.571386in}{1.349353in}}%
\pgfpathlineto{\pgfqpoint{2.577686in}{1.337348in}}%
\pgfpathlineto{\pgfqpoint{2.583989in}{1.325339in}}%
\pgfpathlineto{\pgfqpoint{2.590295in}{1.313333in}}%
\pgfpathclose%
\pgfusepath{stroke,fill}%
\end{pgfscope}%
\begin{pgfscope}%
\pgfpathrectangle{\pgfqpoint{0.887500in}{0.275000in}}{\pgfqpoint{4.225000in}{4.225000in}}%
\pgfusepath{clip}%
\pgfsetbuttcap%
\pgfsetroundjoin%
\definecolor{currentfill}{rgb}{0.282884,0.135920,0.453427}%
\pgfsetfillcolor{currentfill}%
\pgfsetfillopacity{0.700000}%
\pgfsetlinewidth{0.501875pt}%
\definecolor{currentstroke}{rgb}{1.000000,1.000000,1.000000}%
\pgfsetstrokecolor{currentstroke}%
\pgfsetstrokeopacity{0.500000}%
\pgfsetdash{}{0pt}%
\pgfpathmoveto{\pgfqpoint{2.869093in}{1.170061in}}%
\pgfpathlineto{\pgfqpoint{2.880872in}{1.173243in}}%
\pgfpathlineto{\pgfqpoint{2.892645in}{1.176362in}}%
\pgfpathlineto{\pgfqpoint{2.904412in}{1.179443in}}%
\pgfpathlineto{\pgfqpoint{2.916172in}{1.182523in}}%
\pgfpathlineto{\pgfqpoint{2.927926in}{1.185642in}}%
\pgfpathlineto{\pgfqpoint{2.921532in}{1.199012in}}%
\pgfpathlineto{\pgfqpoint{2.915139in}{1.212639in}}%
\pgfpathlineto{\pgfqpoint{2.908749in}{1.226415in}}%
\pgfpathlineto{\pgfqpoint{2.902362in}{1.240228in}}%
\pgfpathlineto{\pgfqpoint{2.895978in}{1.253969in}}%
\pgfpathlineto{\pgfqpoint{2.884231in}{1.250280in}}%
\pgfpathlineto{\pgfqpoint{2.872479in}{1.246598in}}%
\pgfpathlineto{\pgfqpoint{2.860720in}{1.242919in}}%
\pgfpathlineto{\pgfqpoint{2.848955in}{1.239237in}}%
\pgfpathlineto{\pgfqpoint{2.837184in}{1.235551in}}%
\pgfpathlineto{\pgfqpoint{2.843561in}{1.222354in}}%
\pgfpathlineto{\pgfqpoint{2.849940in}{1.209127in}}%
\pgfpathlineto{\pgfqpoint{2.856322in}{1.195949in}}%
\pgfpathlineto{\pgfqpoint{2.862707in}{1.182900in}}%
\pgfpathclose%
\pgfusepath{stroke,fill}%
\end{pgfscope}%
\begin{pgfscope}%
\pgfpathrectangle{\pgfqpoint{0.887500in}{0.275000in}}{\pgfqpoint{4.225000in}{4.225000in}}%
\pgfusepath{clip}%
\pgfsetbuttcap%
\pgfsetroundjoin%
\definecolor{currentfill}{rgb}{0.275191,0.194905,0.496005}%
\pgfsetfillcolor{currentfill}%
\pgfsetfillopacity{0.700000}%
\pgfsetlinewidth{0.501875pt}%
\definecolor{currentstroke}{rgb}{1.000000,1.000000,1.000000}%
\pgfsetstrokecolor{currentstroke}%
\pgfsetstrokeopacity{0.500000}%
\pgfsetdash{}{0pt}%
\pgfpathmoveto{\pgfqpoint{2.687407in}{1.260372in}}%
\pgfpathlineto{\pgfqpoint{2.699230in}{1.264233in}}%
\pgfpathlineto{\pgfqpoint{2.711046in}{1.268102in}}%
\pgfpathlineto{\pgfqpoint{2.722857in}{1.271980in}}%
\pgfpathlineto{\pgfqpoint{2.734661in}{1.275866in}}%
\pgfpathlineto{\pgfqpoint{2.746460in}{1.279760in}}%
\pgfpathlineto{\pgfqpoint{2.740115in}{1.292029in}}%
\pgfpathlineto{\pgfqpoint{2.733775in}{1.304241in}}%
\pgfpathlineto{\pgfqpoint{2.727438in}{1.316407in}}%
\pgfpathlineto{\pgfqpoint{2.721104in}{1.328540in}}%
\pgfpathlineto{\pgfqpoint{2.714774in}{1.340654in}}%
\pgfpathlineto{\pgfqpoint{2.702984in}{1.336696in}}%
\pgfpathlineto{\pgfqpoint{2.691188in}{1.332745in}}%
\pgfpathlineto{\pgfqpoint{2.679386in}{1.328800in}}%
\pgfpathlineto{\pgfqpoint{2.667578in}{1.324862in}}%
\pgfpathlineto{\pgfqpoint{2.655764in}{1.320929in}}%
\pgfpathlineto{\pgfqpoint{2.662086in}{1.308857in}}%
\pgfpathlineto{\pgfqpoint{2.668411in}{1.296773in}}%
\pgfpathlineto{\pgfqpoint{2.674739in}{1.284669in}}%
\pgfpathlineto{\pgfqpoint{2.681071in}{1.272538in}}%
\pgfpathclose%
\pgfusepath{stroke,fill}%
\end{pgfscope}%
\begin{pgfscope}%
\pgfpathrectangle{\pgfqpoint{0.887500in}{0.275000in}}{\pgfqpoint{4.225000in}{4.225000in}}%
\pgfusepath{clip}%
\pgfsetbuttcap%
\pgfsetroundjoin%
\definecolor{currentfill}{rgb}{0.280255,0.165693,0.476498}%
\pgfsetfillcolor{currentfill}%
\pgfsetfillopacity{0.700000}%
\pgfsetlinewidth{0.501875pt}%
\definecolor{currentstroke}{rgb}{1.000000,1.000000,1.000000}%
\pgfsetstrokecolor{currentstroke}%
\pgfsetstrokeopacity{0.500000}%
\pgfsetdash{}{0pt}%
\pgfpathmoveto{\pgfqpoint{2.778240in}{1.217104in}}%
\pgfpathlineto{\pgfqpoint{2.790041in}{1.220789in}}%
\pgfpathlineto{\pgfqpoint{2.801836in}{1.224478in}}%
\pgfpathlineto{\pgfqpoint{2.813625in}{1.228170in}}%
\pgfpathlineto{\pgfqpoint{2.825408in}{1.231861in}}%
\pgfpathlineto{\pgfqpoint{2.837184in}{1.235551in}}%
\pgfpathlineto{\pgfqpoint{2.830812in}{1.248637in}}%
\pgfpathlineto{\pgfqpoint{2.824443in}{1.261552in}}%
\pgfpathlineto{\pgfqpoint{2.818079in}{1.274301in}}%
\pgfpathlineto{\pgfqpoint{2.811718in}{1.286904in}}%
\pgfpathlineto{\pgfqpoint{2.805362in}{1.299380in}}%
\pgfpathlineto{\pgfqpoint{2.793593in}{1.295431in}}%
\pgfpathlineto{\pgfqpoint{2.781819in}{1.291496in}}%
\pgfpathlineto{\pgfqpoint{2.770039in}{1.287573in}}%
\pgfpathlineto{\pgfqpoint{2.758252in}{1.283662in}}%
\pgfpathlineto{\pgfqpoint{2.746460in}{1.279760in}}%
\pgfpathlineto{\pgfqpoint{2.752808in}{1.267420in}}%
\pgfpathlineto{\pgfqpoint{2.759160in}{1.254995in}}%
\pgfpathlineto{\pgfqpoint{2.765516in}{1.242475in}}%
\pgfpathlineto{\pgfqpoint{2.771876in}{1.229845in}}%
\pgfpathclose%
\pgfusepath{stroke,fill}%
\end{pgfscope}%
\begin{pgfscope}%
\pgfpathrectangle{\pgfqpoint{0.887500in}{0.275000in}}{\pgfqpoint{4.225000in}{4.225000in}}%
\pgfusepath{clip}%
\pgfsetbuttcap%
\pgfsetroundjoin%
\definecolor{currentfill}{rgb}{0.282327,0.094955,0.417331}%
\pgfsetfillcolor{currentfill}%
\pgfsetfillopacity{0.700000}%
\pgfsetlinewidth{0.501875pt}%
\definecolor{currentstroke}{rgb}{1.000000,1.000000,1.000000}%
\pgfsetstrokecolor{currentstroke}%
\pgfsetstrokeopacity{0.500000}%
\pgfsetdash{}{0pt}%
\pgfpathmoveto{\pgfqpoint{3.050746in}{1.113464in}}%
\pgfpathlineto{\pgfqpoint{3.062486in}{1.118822in}}%
\pgfpathlineto{\pgfqpoint{3.074221in}{1.124280in}}%
\pgfpathlineto{\pgfqpoint{3.085950in}{1.129756in}}%
\pgfpathlineto{\pgfqpoint{3.097675in}{1.135169in}}%
\pgfpathlineto{\pgfqpoint{3.109395in}{1.140437in}}%
\pgfpathlineto{\pgfqpoint{3.102938in}{1.142687in}}%
\pgfpathlineto{\pgfqpoint{3.096490in}{1.146632in}}%
\pgfpathlineto{\pgfqpoint{3.090049in}{1.152139in}}%
\pgfpathlineto{\pgfqpoint{3.083615in}{1.159074in}}%
\pgfpathlineto{\pgfqpoint{3.077186in}{1.167304in}}%
\pgfpathlineto{\pgfqpoint{3.065486in}{1.162713in}}%
\pgfpathlineto{\pgfqpoint{3.053780in}{1.158075in}}%
\pgfpathlineto{\pgfqpoint{3.042069in}{1.153456in}}%
\pgfpathlineto{\pgfqpoint{3.030352in}{1.148921in}}%
\pgfpathlineto{\pgfqpoint{3.018629in}{1.144536in}}%
\pgfpathlineto{\pgfqpoint{3.025044in}{1.135604in}}%
\pgfpathlineto{\pgfqpoint{3.031463in}{1.127900in}}%
\pgfpathlineto{\pgfqpoint{3.037885in}{1.121554in}}%
\pgfpathlineto{\pgfqpoint{3.044313in}{1.116698in}}%
\pgfpathclose%
\pgfusepath{stroke,fill}%
\end{pgfscope}%
\begin{pgfscope}%
\pgfpathrectangle{\pgfqpoint{0.887500in}{0.275000in}}{\pgfqpoint{4.225000in}{4.225000in}}%
\pgfusepath{clip}%
\pgfsetbuttcap%
\pgfsetroundjoin%
\definecolor{currentfill}{rgb}{0.241237,0.296485,0.539709}%
\pgfsetfillcolor{currentfill}%
\pgfsetfillopacity{0.700000}%
\pgfsetlinewidth{0.501875pt}%
\definecolor{currentstroke}{rgb}{1.000000,1.000000,1.000000}%
\pgfsetstrokecolor{currentstroke}%
\pgfsetstrokeopacity{0.500000}%
\pgfsetdash{}{0pt}%
\pgfpathmoveto{\pgfqpoint{2.173988in}{1.436467in}}%
\pgfpathlineto{\pgfqpoint{2.185943in}{1.440493in}}%
\pgfpathlineto{\pgfqpoint{2.197893in}{1.444518in}}%
\pgfpathlineto{\pgfqpoint{2.209836in}{1.448545in}}%
\pgfpathlineto{\pgfqpoint{2.221774in}{1.452573in}}%
\pgfpathlineto{\pgfqpoint{2.233705in}{1.456603in}}%
\pgfpathlineto{\pgfqpoint{2.227508in}{1.468003in}}%
\pgfpathlineto{\pgfqpoint{2.221314in}{1.479376in}}%
\pgfpathlineto{\pgfqpoint{2.215125in}{1.490722in}}%
\pgfpathlineto{\pgfqpoint{2.208940in}{1.502039in}}%
\pgfpathlineto{\pgfqpoint{2.202758in}{1.513328in}}%
\pgfpathlineto{\pgfqpoint{2.190837in}{1.509273in}}%
\pgfpathlineto{\pgfqpoint{2.178910in}{1.505217in}}%
\pgfpathlineto{\pgfqpoint{2.166977in}{1.501161in}}%
\pgfpathlineto{\pgfqpoint{2.155038in}{1.497104in}}%
\pgfpathlineto{\pgfqpoint{2.143093in}{1.493046in}}%
\pgfpathlineto{\pgfqpoint{2.149264in}{1.481785in}}%
\pgfpathlineto{\pgfqpoint{2.155439in}{1.470497in}}%
\pgfpathlineto{\pgfqpoint{2.161618in}{1.459180in}}%
\pgfpathlineto{\pgfqpoint{2.167801in}{1.447837in}}%
\pgfpathclose%
\pgfusepath{stroke,fill}%
\end{pgfscope}%
\begin{pgfscope}%
\pgfpathrectangle{\pgfqpoint{0.887500in}{0.275000in}}{\pgfqpoint{4.225000in}{4.225000in}}%
\pgfusepath{clip}%
\pgfsetbuttcap%
\pgfsetroundjoin%
\definecolor{currentfill}{rgb}{0.283229,0.120777,0.440584}%
\pgfsetfillcolor{currentfill}%
\pgfsetfillopacity{0.700000}%
\pgfsetlinewidth{0.501875pt}%
\definecolor{currentstroke}{rgb}{1.000000,1.000000,1.000000}%
\pgfsetstrokecolor{currentstroke}%
\pgfsetstrokeopacity{0.500000}%
\pgfsetdash{}{0pt}%
\pgfpathmoveto{\pgfqpoint{3.200508in}{1.178940in}}%
\pgfpathlineto{\pgfqpoint{3.212207in}{1.179056in}}%
\pgfpathlineto{\pgfqpoint{3.223890in}{1.177360in}}%
\pgfpathlineto{\pgfqpoint{3.235555in}{1.173798in}}%
\pgfpathlineto{\pgfqpoint{3.247202in}{1.168676in}}%
\pgfpathlineto{\pgfqpoint{3.258831in}{1.162341in}}%
\pgfpathlineto{\pgfqpoint{3.252311in}{1.167487in}}%
\pgfpathlineto{\pgfqpoint{3.245799in}{1.173007in}}%
\pgfpathlineto{\pgfqpoint{3.239294in}{1.178897in}}%
\pgfpathlineto{\pgfqpoint{3.232797in}{1.185155in}}%
\pgfpathlineto{\pgfqpoint{3.226307in}{1.191776in}}%
\pgfpathlineto{\pgfqpoint{3.214639in}{1.186893in}}%
\pgfpathlineto{\pgfqpoint{3.202965in}{1.181457in}}%
\pgfpathlineto{\pgfqpoint{3.191285in}{1.175851in}}%
\pgfpathlineto{\pgfqpoint{3.179601in}{1.170427in}}%
\pgfpathlineto{\pgfqpoint{3.167912in}{1.165252in}}%
\pgfpathlineto{\pgfqpoint{3.174403in}{1.164287in}}%
\pgfpathlineto{\pgfqpoint{3.180907in}{1.165058in}}%
\pgfpathlineto{\pgfqpoint{3.187425in}{1.167680in}}%
\pgfpathlineto{\pgfqpoint{3.193958in}{1.172268in}}%
\pgfpathclose%
\pgfusepath{stroke,fill}%
\end{pgfscope}%
\begin{pgfscope}%
\pgfpathrectangle{\pgfqpoint{0.887500in}{0.275000in}}{\pgfqpoint{4.225000in}{4.225000in}}%
\pgfusepath{clip}%
\pgfsetbuttcap%
\pgfsetroundjoin%
\definecolor{currentfill}{rgb}{0.248629,0.278775,0.534556}%
\pgfsetfillcolor{currentfill}%
\pgfsetfillopacity{0.700000}%
\pgfsetlinewidth{0.501875pt}%
\definecolor{currentstroke}{rgb}{1.000000,1.000000,1.000000}%
\pgfsetstrokecolor{currentstroke}%
\pgfsetstrokeopacity{0.500000}%
\pgfsetdash{}{0pt}%
\pgfpathmoveto{\pgfqpoint{2.264750in}{1.399226in}}%
\pgfpathlineto{\pgfqpoint{2.276685in}{1.403227in}}%
\pgfpathlineto{\pgfqpoint{2.288615in}{1.407231in}}%
\pgfpathlineto{\pgfqpoint{2.300538in}{1.411238in}}%
\pgfpathlineto{\pgfqpoint{2.312456in}{1.415248in}}%
\pgfpathlineto{\pgfqpoint{2.324367in}{1.419262in}}%
\pgfpathlineto{\pgfqpoint{2.318141in}{1.430820in}}%
\pgfpathlineto{\pgfqpoint{2.311918in}{1.442353in}}%
\pgfpathlineto{\pgfqpoint{2.305699in}{1.453858in}}%
\pgfpathlineto{\pgfqpoint{2.299484in}{1.465337in}}%
\pgfpathlineto{\pgfqpoint{2.293273in}{1.476788in}}%
\pgfpathlineto{\pgfqpoint{2.281371in}{1.472745in}}%
\pgfpathlineto{\pgfqpoint{2.269463in}{1.468706in}}%
\pgfpathlineto{\pgfqpoint{2.257550in}{1.464669in}}%
\pgfpathlineto{\pgfqpoint{2.245630in}{1.460635in}}%
\pgfpathlineto{\pgfqpoint{2.233705in}{1.456603in}}%
\pgfpathlineto{\pgfqpoint{2.239906in}{1.445177in}}%
\pgfpathlineto{\pgfqpoint{2.246111in}{1.433726in}}%
\pgfpathlineto{\pgfqpoint{2.252320in}{1.422250in}}%
\pgfpathlineto{\pgfqpoint{2.258533in}{1.410750in}}%
\pgfpathclose%
\pgfusepath{stroke,fill}%
\end{pgfscope}%
\begin{pgfscope}%
\pgfpathrectangle{\pgfqpoint{0.887500in}{0.275000in}}{\pgfqpoint{4.225000in}{4.225000in}}%
\pgfusepath{clip}%
\pgfsetbuttcap%
\pgfsetroundjoin%
\definecolor{currentfill}{rgb}{0.257322,0.256130,0.526563}%
\pgfsetfillcolor{currentfill}%
\pgfsetfillopacity{0.700000}%
\pgfsetlinewidth{0.501875pt}%
\definecolor{currentstroke}{rgb}{1.000000,1.000000,1.000000}%
\pgfsetstrokecolor{currentstroke}%
\pgfsetstrokeopacity{0.500000}%
\pgfsetdash{}{0pt}%
\pgfpathmoveto{\pgfqpoint{2.355558in}{1.361096in}}%
\pgfpathlineto{\pgfqpoint{2.367473in}{1.365064in}}%
\pgfpathlineto{\pgfqpoint{2.379383in}{1.369035in}}%
\pgfpathlineto{\pgfqpoint{2.391287in}{1.373009in}}%
\pgfpathlineto{\pgfqpoint{2.403184in}{1.376985in}}%
\pgfpathlineto{\pgfqpoint{2.415076in}{1.380964in}}%
\pgfpathlineto{\pgfqpoint{2.408820in}{1.392707in}}%
\pgfpathlineto{\pgfqpoint{2.402568in}{1.404422in}}%
\pgfpathlineto{\pgfqpoint{2.396320in}{1.416109in}}%
\pgfpathlineto{\pgfqpoint{2.390076in}{1.427765in}}%
\pgfpathlineto{\pgfqpoint{2.383835in}{1.439391in}}%
\pgfpathlineto{\pgfqpoint{2.371954in}{1.435356in}}%
\pgfpathlineto{\pgfqpoint{2.360066in}{1.431326in}}%
\pgfpathlineto{\pgfqpoint{2.348173in}{1.427301in}}%
\pgfpathlineto{\pgfqpoint{2.336273in}{1.423279in}}%
\pgfpathlineto{\pgfqpoint{2.324367in}{1.419262in}}%
\pgfpathlineto{\pgfqpoint{2.330598in}{1.407678in}}%
\pgfpathlineto{\pgfqpoint{2.336832in}{1.396068in}}%
\pgfpathlineto{\pgfqpoint{2.343070in}{1.384434in}}%
\pgfpathlineto{\pgfqpoint{2.349312in}{1.372776in}}%
\pgfpathclose%
\pgfusepath{stroke,fill}%
\end{pgfscope}%
\begin{pgfscope}%
\pgfpathrectangle{\pgfqpoint{0.887500in}{0.275000in}}{\pgfqpoint{4.225000in}{4.225000in}}%
\pgfusepath{clip}%
\pgfsetbuttcap%
\pgfsetroundjoin%
\definecolor{currentfill}{rgb}{0.263663,0.237631,0.518762}%
\pgfsetfillcolor{currentfill}%
\pgfsetfillopacity{0.700000}%
\pgfsetlinewidth{0.501875pt}%
\definecolor{currentstroke}{rgb}{1.000000,1.000000,1.000000}%
\pgfsetstrokecolor{currentstroke}%
\pgfsetstrokeopacity{0.500000}%
\pgfsetdash{}{0pt}%
\pgfpathmoveto{\pgfqpoint{2.446409in}{1.321947in}}%
\pgfpathlineto{\pgfqpoint{2.458304in}{1.325852in}}%
\pgfpathlineto{\pgfqpoint{2.470194in}{1.329762in}}%
\pgfpathlineto{\pgfqpoint{2.482077in}{1.333677in}}%
\pgfpathlineto{\pgfqpoint{2.493954in}{1.337600in}}%
\pgfpathlineto{\pgfqpoint{2.505825in}{1.341533in}}%
\pgfpathlineto{\pgfqpoint{2.499542in}{1.353448in}}%
\pgfpathlineto{\pgfqpoint{2.493262in}{1.365349in}}%
\pgfpathlineto{\pgfqpoint{2.486985in}{1.377233in}}%
\pgfpathlineto{\pgfqpoint{2.480712in}{1.389097in}}%
\pgfpathlineto{\pgfqpoint{2.474443in}{1.400935in}}%
\pgfpathlineto{\pgfqpoint{2.462582in}{1.396928in}}%
\pgfpathlineto{\pgfqpoint{2.450714in}{1.392928in}}%
\pgfpathlineto{\pgfqpoint{2.438841in}{1.388935in}}%
\pgfpathlineto{\pgfqpoint{2.426961in}{1.384947in}}%
\pgfpathlineto{\pgfqpoint{2.415076in}{1.380964in}}%
\pgfpathlineto{\pgfqpoint{2.421335in}{1.369197in}}%
\pgfpathlineto{\pgfqpoint{2.427598in}{1.357409in}}%
\pgfpathlineto{\pgfqpoint{2.433865in}{1.345603in}}%
\pgfpathlineto{\pgfqpoint{2.440135in}{1.333781in}}%
\pgfpathclose%
\pgfusepath{stroke,fill}%
\end{pgfscope}%
\begin{pgfscope}%
\pgfpathrectangle{\pgfqpoint{0.887500in}{0.275000in}}{\pgfqpoint{4.225000in}{4.225000in}}%
\pgfusepath{clip}%
\pgfsetbuttcap%
\pgfsetroundjoin%
\definecolor{currentfill}{rgb}{0.270595,0.214069,0.507052}%
\pgfsetfillcolor{currentfill}%
\pgfsetfillopacity{0.700000}%
\pgfsetlinewidth{0.501875pt}%
\definecolor{currentstroke}{rgb}{1.000000,1.000000,1.000000}%
\pgfsetstrokecolor{currentstroke}%
\pgfsetstrokeopacity{0.500000}%
\pgfsetdash{}{0pt}%
\pgfpathmoveto{\pgfqpoint{2.537293in}{1.281896in}}%
\pgfpathlineto{\pgfqpoint{2.549168in}{1.285762in}}%
\pgfpathlineto{\pgfqpoint{2.561036in}{1.289639in}}%
\pgfpathlineto{\pgfqpoint{2.572898in}{1.293527in}}%
\pgfpathlineto{\pgfqpoint{2.584754in}{1.297424in}}%
\pgfpathlineto{\pgfqpoint{2.596604in}{1.301329in}}%
\pgfpathlineto{\pgfqpoint{2.590295in}{1.313333in}}%
\pgfpathlineto{\pgfqpoint{2.583989in}{1.325339in}}%
\pgfpathlineto{\pgfqpoint{2.577686in}{1.337348in}}%
\pgfpathlineto{\pgfqpoint{2.571386in}{1.349353in}}%
\pgfpathlineto{\pgfqpoint{2.565090in}{1.361351in}}%
\pgfpathlineto{\pgfqpoint{2.553249in}{1.357372in}}%
\pgfpathlineto{\pgfqpoint{2.541402in}{1.353399in}}%
\pgfpathlineto{\pgfqpoint{2.529549in}{1.349433in}}%
\pgfpathlineto{\pgfqpoint{2.517690in}{1.345477in}}%
\pgfpathlineto{\pgfqpoint{2.505825in}{1.341533in}}%
\pgfpathlineto{\pgfqpoint{2.512112in}{1.329609in}}%
\pgfpathlineto{\pgfqpoint{2.518403in}{1.317680in}}%
\pgfpathlineto{\pgfqpoint{2.524696in}{1.305749in}}%
\pgfpathlineto{\pgfqpoint{2.530993in}{1.293820in}}%
\pgfpathclose%
\pgfusepath{stroke,fill}%
\end{pgfscope}%
\begin{pgfscope}%
\pgfpathrectangle{\pgfqpoint{0.887500in}{0.275000in}}{\pgfqpoint{4.225000in}{4.225000in}}%
\pgfusepath{clip}%
\pgfsetbuttcap%
\pgfsetroundjoin%
\definecolor{currentfill}{rgb}{0.283091,0.110553,0.431554}%
\pgfsetfillcolor{currentfill}%
\pgfsetfillopacity{0.700000}%
\pgfsetlinewidth{0.501875pt}%
\definecolor{currentstroke}{rgb}{1.000000,1.000000,1.000000}%
\pgfsetstrokecolor{currentstroke}%
\pgfsetstrokeopacity{0.500000}%
\pgfsetdash{}{0pt}%
\pgfpathmoveto{\pgfqpoint{2.901048in}{1.111806in}}%
\pgfpathlineto{\pgfqpoint{2.912836in}{1.114857in}}%
\pgfpathlineto{\pgfqpoint{2.924617in}{1.117800in}}%
\pgfpathlineto{\pgfqpoint{2.936392in}{1.120679in}}%
\pgfpathlineto{\pgfqpoint{2.948160in}{1.123561in}}%
\pgfpathlineto{\pgfqpoint{2.959922in}{1.126514in}}%
\pgfpathlineto{\pgfqpoint{2.953520in}{1.136943in}}%
\pgfpathlineto{\pgfqpoint{2.947119in}{1.148181in}}%
\pgfpathlineto{\pgfqpoint{2.940720in}{1.160117in}}%
\pgfpathlineto{\pgfqpoint{2.934322in}{1.172640in}}%
\pgfpathlineto{\pgfqpoint{2.927926in}{1.185642in}}%
\pgfpathlineto{\pgfqpoint{2.916172in}{1.182523in}}%
\pgfpathlineto{\pgfqpoint{2.904412in}{1.179443in}}%
\pgfpathlineto{\pgfqpoint{2.892645in}{1.176362in}}%
\pgfpathlineto{\pgfqpoint{2.880872in}{1.173243in}}%
\pgfpathlineto{\pgfqpoint{2.869093in}{1.170061in}}%
\pgfpathlineto{\pgfqpoint{2.875481in}{1.157512in}}%
\pgfpathlineto{\pgfqpoint{2.881871in}{1.145331in}}%
\pgfpathlineto{\pgfqpoint{2.888262in}{1.133600in}}%
\pgfpathlineto{\pgfqpoint{2.894655in}{1.122399in}}%
\pgfpathclose%
\pgfusepath{stroke,fill}%
\end{pgfscope}%
\begin{pgfscope}%
\pgfpathrectangle{\pgfqpoint{0.887500in}{0.275000in}}{\pgfqpoint{4.225000in}{4.225000in}}%
\pgfusepath{clip}%
\pgfsetbuttcap%
\pgfsetroundjoin%
\definecolor{currentfill}{rgb}{0.276194,0.190074,0.493001}%
\pgfsetfillcolor{currentfill}%
\pgfsetfillopacity{0.700000}%
\pgfsetlinewidth{0.501875pt}%
\definecolor{currentstroke}{rgb}{1.000000,1.000000,1.000000}%
\pgfsetstrokecolor{currentstroke}%
\pgfsetstrokeopacity{0.500000}%
\pgfsetdash{}{0pt}%
\pgfpathmoveto{\pgfqpoint{2.628203in}{1.241207in}}%
\pgfpathlineto{\pgfqpoint{2.640056in}{1.245021in}}%
\pgfpathlineto{\pgfqpoint{2.651903in}{1.248844in}}%
\pgfpathlineto{\pgfqpoint{2.663744in}{1.252677in}}%
\pgfpathlineto{\pgfqpoint{2.675579in}{1.256520in}}%
\pgfpathlineto{\pgfqpoint{2.687407in}{1.260372in}}%
\pgfpathlineto{\pgfqpoint{2.681071in}{1.272538in}}%
\pgfpathlineto{\pgfqpoint{2.674739in}{1.284669in}}%
\pgfpathlineto{\pgfqpoint{2.668411in}{1.296773in}}%
\pgfpathlineto{\pgfqpoint{2.662086in}{1.308857in}}%
\pgfpathlineto{\pgfqpoint{2.655764in}{1.320929in}}%
\pgfpathlineto{\pgfqpoint{2.643944in}{1.317001in}}%
\pgfpathlineto{\pgfqpoint{2.632118in}{1.313077in}}%
\pgfpathlineto{\pgfqpoint{2.620286in}{1.309156in}}%
\pgfpathlineto{\pgfqpoint{2.608448in}{1.305240in}}%
\pgfpathlineto{\pgfqpoint{2.596604in}{1.301329in}}%
\pgfpathlineto{\pgfqpoint{2.602917in}{1.289323in}}%
\pgfpathlineto{\pgfqpoint{2.609233in}{1.277312in}}%
\pgfpathlineto{\pgfqpoint{2.615553in}{1.265292in}}%
\pgfpathlineto{\pgfqpoint{2.621876in}{1.253258in}}%
\pgfpathclose%
\pgfusepath{stroke,fill}%
\end{pgfscope}%
\begin{pgfscope}%
\pgfpathrectangle{\pgfqpoint{0.887500in}{0.275000in}}{\pgfqpoint{4.225000in}{4.225000in}}%
\pgfusepath{clip}%
\pgfsetbuttcap%
\pgfsetroundjoin%
\definecolor{currentfill}{rgb}{0.282623,0.140926,0.457517}%
\pgfsetfillcolor{currentfill}%
\pgfsetfillopacity{0.700000}%
\pgfsetlinewidth{0.501875pt}%
\definecolor{currentstroke}{rgb}{1.000000,1.000000,1.000000}%
\pgfsetstrokecolor{currentstroke}%
\pgfsetstrokeopacity{0.500000}%
\pgfsetdash{}{0pt}%
\pgfpathmoveto{\pgfqpoint{2.810106in}{1.153533in}}%
\pgfpathlineto{\pgfqpoint{2.821916in}{1.156884in}}%
\pgfpathlineto{\pgfqpoint{2.833719in}{1.160224in}}%
\pgfpathlineto{\pgfqpoint{2.845517in}{1.163541in}}%
\pgfpathlineto{\pgfqpoint{2.857308in}{1.166824in}}%
\pgfpathlineto{\pgfqpoint{2.869093in}{1.170061in}}%
\pgfpathlineto{\pgfqpoint{2.862707in}{1.182900in}}%
\pgfpathlineto{\pgfqpoint{2.856322in}{1.195949in}}%
\pgfpathlineto{\pgfqpoint{2.849940in}{1.209127in}}%
\pgfpathlineto{\pgfqpoint{2.843561in}{1.222354in}}%
\pgfpathlineto{\pgfqpoint{2.837184in}{1.235551in}}%
\pgfpathlineto{\pgfqpoint{2.825408in}{1.231861in}}%
\pgfpathlineto{\pgfqpoint{2.813625in}{1.228170in}}%
\pgfpathlineto{\pgfqpoint{2.801836in}{1.224478in}}%
\pgfpathlineto{\pgfqpoint{2.790041in}{1.220789in}}%
\pgfpathlineto{\pgfqpoint{2.778240in}{1.217104in}}%
\pgfpathlineto{\pgfqpoint{2.784607in}{1.204294in}}%
\pgfpathlineto{\pgfqpoint{2.790978in}{1.191474in}}%
\pgfpathlineto{\pgfqpoint{2.797352in}{1.178702in}}%
\pgfpathlineto{\pgfqpoint{2.803728in}{1.166036in}}%
\pgfpathclose%
\pgfusepath{stroke,fill}%
\end{pgfscope}%
\begin{pgfscope}%
\pgfpathrectangle{\pgfqpoint{0.887500in}{0.275000in}}{\pgfqpoint{4.225000in}{4.225000in}}%
\pgfusepath{clip}%
\pgfsetbuttcap%
\pgfsetroundjoin%
\definecolor{currentfill}{rgb}{0.280255,0.165693,0.476498}%
\pgfsetfillcolor{currentfill}%
\pgfsetfillopacity{0.700000}%
\pgfsetlinewidth{0.501875pt}%
\definecolor{currentstroke}{rgb}{1.000000,1.000000,1.000000}%
\pgfsetstrokecolor{currentstroke}%
\pgfsetstrokeopacity{0.500000}%
\pgfsetdash{}{0pt}%
\pgfpathmoveto{\pgfqpoint{2.719142in}{1.198765in}}%
\pgfpathlineto{\pgfqpoint{2.730974in}{1.202421in}}%
\pgfpathlineto{\pgfqpoint{2.742800in}{1.206082in}}%
\pgfpathlineto{\pgfqpoint{2.754619in}{1.209750in}}%
\pgfpathlineto{\pgfqpoint{2.766433in}{1.213423in}}%
\pgfpathlineto{\pgfqpoint{2.778240in}{1.217104in}}%
\pgfpathlineto{\pgfqpoint{2.771876in}{1.229845in}}%
\pgfpathlineto{\pgfqpoint{2.765516in}{1.242475in}}%
\pgfpathlineto{\pgfqpoint{2.759160in}{1.254995in}}%
\pgfpathlineto{\pgfqpoint{2.752808in}{1.267420in}}%
\pgfpathlineto{\pgfqpoint{2.746460in}{1.279760in}}%
\pgfpathlineto{\pgfqpoint{2.734661in}{1.275866in}}%
\pgfpathlineto{\pgfqpoint{2.722857in}{1.271980in}}%
\pgfpathlineto{\pgfqpoint{2.711046in}{1.268102in}}%
\pgfpathlineto{\pgfqpoint{2.699230in}{1.264233in}}%
\pgfpathlineto{\pgfqpoint{2.687407in}{1.260372in}}%
\pgfpathlineto{\pgfqpoint{2.693747in}{1.248163in}}%
\pgfpathlineto{\pgfqpoint{2.700090in}{1.235905in}}%
\pgfpathlineto{\pgfqpoint{2.706437in}{1.223590in}}%
\pgfpathlineto{\pgfqpoint{2.712788in}{1.211210in}}%
\pgfpathclose%
\pgfusepath{stroke,fill}%
\end{pgfscope}%
\begin{pgfscope}%
\pgfpathrectangle{\pgfqpoint{0.887500in}{0.275000in}}{\pgfqpoint{4.225000in}{4.225000in}}%
\pgfusepath{clip}%
\pgfsetbuttcap%
\pgfsetroundjoin%
\definecolor{currentfill}{rgb}{0.282327,0.094955,0.417331}%
\pgfsetfillcolor{currentfill}%
\pgfsetfillopacity{0.700000}%
\pgfsetlinewidth{0.501875pt}%
\definecolor{currentstroke}{rgb}{1.000000,1.000000,1.000000}%
\pgfsetstrokecolor{currentstroke}%
\pgfsetstrokeopacity{0.500000}%
\pgfsetdash{}{0pt}%
\pgfpathmoveto{\pgfqpoint{2.991966in}{1.090348in}}%
\pgfpathlineto{\pgfqpoint{3.003734in}{1.094456in}}%
\pgfpathlineto{\pgfqpoint{3.015496in}{1.098776in}}%
\pgfpathlineto{\pgfqpoint{3.027251in}{1.103373in}}%
\pgfpathlineto{\pgfqpoint{3.039002in}{1.108287in}}%
\pgfpathlineto{\pgfqpoint{3.050746in}{1.113464in}}%
\pgfpathlineto{\pgfqpoint{3.044313in}{1.116698in}}%
\pgfpathlineto{\pgfqpoint{3.037885in}{1.121554in}}%
\pgfpathlineto{\pgfqpoint{3.031463in}{1.127900in}}%
\pgfpathlineto{\pgfqpoint{3.025044in}{1.135604in}}%
\pgfpathlineto{\pgfqpoint{3.018629in}{1.144536in}}%
\pgfpathlineto{\pgfqpoint{3.006901in}{1.140364in}}%
\pgfpathlineto{\pgfqpoint{2.995166in}{1.136473in}}%
\pgfpathlineto{\pgfqpoint{2.983425in}{1.132902in}}%
\pgfpathlineto{\pgfqpoint{2.971677in}{1.129605in}}%
\pgfpathlineto{\pgfqpoint{2.959922in}{1.126514in}}%
\pgfpathlineto{\pgfqpoint{2.966326in}{1.117002in}}%
\pgfpathlineto{\pgfqpoint{2.972732in}{1.108519in}}%
\pgfpathlineto{\pgfqpoint{2.979141in}{1.101175in}}%
\pgfpathlineto{\pgfqpoint{2.985552in}{1.095081in}}%
\pgfpathclose%
\pgfusepath{stroke,fill}%
\end{pgfscope}%
\begin{pgfscope}%
\pgfpathrectangle{\pgfqpoint{0.887500in}{0.275000in}}{\pgfqpoint{4.225000in}{4.225000in}}%
\pgfusepath{clip}%
\pgfsetbuttcap%
\pgfsetroundjoin%
\definecolor{currentfill}{rgb}{0.283187,0.125848,0.444960}%
\pgfsetfillcolor{currentfill}%
\pgfsetfillopacity{0.700000}%
\pgfsetlinewidth{0.501875pt}%
\definecolor{currentstroke}{rgb}{1.000000,1.000000,1.000000}%
\pgfsetstrokecolor{currentstroke}%
\pgfsetstrokeopacity{0.500000}%
\pgfsetdash{}{0pt}%
\pgfpathmoveto{\pgfqpoint{3.141847in}{1.159359in}}%
\pgfpathlineto{\pgfqpoint{3.153596in}{1.165025in}}%
\pgfpathlineto{\pgfqpoint{3.165338in}{1.170051in}}%
\pgfpathlineto{\pgfqpoint{3.177071in}{1.174203in}}%
\pgfpathlineto{\pgfqpoint{3.188796in}{1.177244in}}%
\pgfpathlineto{\pgfqpoint{3.200508in}{1.178940in}}%
\pgfpathlineto{\pgfqpoint{3.193958in}{1.172268in}}%
\pgfpathlineto{\pgfqpoint{3.187425in}{1.167680in}}%
\pgfpathlineto{\pgfqpoint{3.180907in}{1.165058in}}%
\pgfpathlineto{\pgfqpoint{3.174403in}{1.164287in}}%
\pgfpathlineto{\pgfqpoint{3.167912in}{1.165252in}}%
\pgfpathlineto{\pgfqpoint{3.156219in}{1.160249in}}%
\pgfpathlineto{\pgfqpoint{3.144521in}{1.155343in}}%
\pgfpathlineto{\pgfqpoint{3.132818in}{1.150456in}}%
\pgfpathlineto{\pgfqpoint{3.121109in}{1.145513in}}%
\pgfpathlineto{\pgfqpoint{3.109395in}{1.140437in}}%
\pgfpathlineto{\pgfqpoint{3.115861in}{1.140017in}}%
\pgfpathlineto{\pgfqpoint{3.122338in}{1.141562in}}%
\pgfpathlineto{\pgfqpoint{3.128827in}{1.145209in}}%
\pgfpathlineto{\pgfqpoint{3.135330in}{1.151095in}}%
\pgfpathclose%
\pgfusepath{stroke,fill}%
\end{pgfscope}%
\begin{pgfscope}%
\pgfpathrectangle{\pgfqpoint{0.887500in}{0.275000in}}{\pgfqpoint{4.225000in}{4.225000in}}%
\pgfusepath{clip}%
\pgfsetbuttcap%
\pgfsetroundjoin%
\definecolor{currentfill}{rgb}{0.248629,0.278775,0.534556}%
\pgfsetfillcolor{currentfill}%
\pgfsetfillopacity{0.700000}%
\pgfsetlinewidth{0.501875pt}%
\definecolor{currentstroke}{rgb}{1.000000,1.000000,1.000000}%
\pgfsetstrokecolor{currentstroke}%
\pgfsetstrokeopacity{0.500000}%
\pgfsetdash{}{0pt}%
\pgfpathmoveto{\pgfqpoint{2.204981in}{1.379256in}}%
\pgfpathlineto{\pgfqpoint{2.216947in}{1.383246in}}%
\pgfpathlineto{\pgfqpoint{2.228907in}{1.387238in}}%
\pgfpathlineto{\pgfqpoint{2.240860in}{1.391232in}}%
\pgfpathlineto{\pgfqpoint{2.252808in}{1.395228in}}%
\pgfpathlineto{\pgfqpoint{2.264750in}{1.399226in}}%
\pgfpathlineto{\pgfqpoint{2.258533in}{1.410750in}}%
\pgfpathlineto{\pgfqpoint{2.252320in}{1.422250in}}%
\pgfpathlineto{\pgfqpoint{2.246111in}{1.433726in}}%
\pgfpathlineto{\pgfqpoint{2.239906in}{1.445177in}}%
\pgfpathlineto{\pgfqpoint{2.233705in}{1.456603in}}%
\pgfpathlineto{\pgfqpoint{2.221774in}{1.452573in}}%
\pgfpathlineto{\pgfqpoint{2.209836in}{1.448545in}}%
\pgfpathlineto{\pgfqpoint{2.197893in}{1.444518in}}%
\pgfpathlineto{\pgfqpoint{2.185943in}{1.440493in}}%
\pgfpathlineto{\pgfqpoint{2.173988in}{1.436467in}}%
\pgfpathlineto{\pgfqpoint{2.180179in}{1.425072in}}%
\pgfpathlineto{\pgfqpoint{2.186374in}{1.413653in}}%
\pgfpathlineto{\pgfqpoint{2.192572in}{1.402210in}}%
\pgfpathlineto{\pgfqpoint{2.198775in}{1.390744in}}%
\pgfpathclose%
\pgfusepath{stroke,fill}%
\end{pgfscope}%
\begin{pgfscope}%
\pgfpathrectangle{\pgfqpoint{0.887500in}{0.275000in}}{\pgfqpoint{4.225000in}{4.225000in}}%
\pgfusepath{clip}%
\pgfsetbuttcap%
\pgfsetroundjoin%
\definecolor{currentfill}{rgb}{0.257322,0.256130,0.526563}%
\pgfsetfillcolor{currentfill}%
\pgfsetfillopacity{0.700000}%
\pgfsetlinewidth{0.501875pt}%
\definecolor{currentstroke}{rgb}{1.000000,1.000000,1.000000}%
\pgfsetstrokecolor{currentstroke}%
\pgfsetstrokeopacity{0.500000}%
\pgfsetdash{}{0pt}%
\pgfpathmoveto{\pgfqpoint{2.295890in}{1.341286in}}%
\pgfpathlineto{\pgfqpoint{2.307835in}{1.345244in}}%
\pgfpathlineto{\pgfqpoint{2.319775in}{1.349204in}}%
\pgfpathlineto{\pgfqpoint{2.331709in}{1.353166in}}%
\pgfpathlineto{\pgfqpoint{2.343636in}{1.357130in}}%
\pgfpathlineto{\pgfqpoint{2.355558in}{1.361096in}}%
\pgfpathlineto{\pgfqpoint{2.349312in}{1.372776in}}%
\pgfpathlineto{\pgfqpoint{2.343070in}{1.384434in}}%
\pgfpathlineto{\pgfqpoint{2.336832in}{1.396068in}}%
\pgfpathlineto{\pgfqpoint{2.330598in}{1.407678in}}%
\pgfpathlineto{\pgfqpoint{2.324367in}{1.419262in}}%
\pgfpathlineto{\pgfqpoint{2.312456in}{1.415248in}}%
\pgfpathlineto{\pgfqpoint{2.300538in}{1.411238in}}%
\pgfpathlineto{\pgfqpoint{2.288615in}{1.407231in}}%
\pgfpathlineto{\pgfqpoint{2.276685in}{1.403227in}}%
\pgfpathlineto{\pgfqpoint{2.264750in}{1.399226in}}%
\pgfpathlineto{\pgfqpoint{2.270970in}{1.387680in}}%
\pgfpathlineto{\pgfqpoint{2.277194in}{1.376113in}}%
\pgfpathlineto{\pgfqpoint{2.283422in}{1.364524in}}%
\pgfpathlineto{\pgfqpoint{2.289654in}{1.352914in}}%
\pgfpathclose%
\pgfusepath{stroke,fill}%
\end{pgfscope}%
\begin{pgfscope}%
\pgfpathrectangle{\pgfqpoint{0.887500in}{0.275000in}}{\pgfqpoint{4.225000in}{4.225000in}}%
\pgfusepath{clip}%
\pgfsetbuttcap%
\pgfsetroundjoin%
\definecolor{currentfill}{rgb}{0.263663,0.237631,0.518762}%
\pgfsetfillcolor{currentfill}%
\pgfsetfillopacity{0.700000}%
\pgfsetlinewidth{0.501875pt}%
\definecolor{currentstroke}{rgb}{1.000000,1.000000,1.000000}%
\pgfsetstrokecolor{currentstroke}%
\pgfsetstrokeopacity{0.500000}%
\pgfsetdash{}{0pt}%
\pgfpathmoveto{\pgfqpoint{2.386841in}{1.302436in}}%
\pgfpathlineto{\pgfqpoint{2.398766in}{1.306338in}}%
\pgfpathlineto{\pgfqpoint{2.410686in}{1.310240in}}%
\pgfpathlineto{\pgfqpoint{2.422600in}{1.314143in}}%
\pgfpathlineto{\pgfqpoint{2.434507in}{1.318045in}}%
\pgfpathlineto{\pgfqpoint{2.446409in}{1.321947in}}%
\pgfpathlineto{\pgfqpoint{2.440135in}{1.333781in}}%
\pgfpathlineto{\pgfqpoint{2.433865in}{1.345603in}}%
\pgfpathlineto{\pgfqpoint{2.427598in}{1.357409in}}%
\pgfpathlineto{\pgfqpoint{2.421335in}{1.369197in}}%
\pgfpathlineto{\pgfqpoint{2.415076in}{1.380964in}}%
\pgfpathlineto{\pgfqpoint{2.403184in}{1.376985in}}%
\pgfpathlineto{\pgfqpoint{2.391287in}{1.373009in}}%
\pgfpathlineto{\pgfqpoint{2.379383in}{1.369035in}}%
\pgfpathlineto{\pgfqpoint{2.367473in}{1.365064in}}%
\pgfpathlineto{\pgfqpoint{2.355558in}{1.361096in}}%
\pgfpathlineto{\pgfqpoint{2.361807in}{1.349396in}}%
\pgfpathlineto{\pgfqpoint{2.368060in}{1.337677in}}%
\pgfpathlineto{\pgfqpoint{2.374317in}{1.325943in}}%
\pgfpathlineto{\pgfqpoint{2.380577in}{1.314195in}}%
\pgfpathclose%
\pgfusepath{stroke,fill}%
\end{pgfscope}%
\begin{pgfscope}%
\pgfpathrectangle{\pgfqpoint{0.887500in}{0.275000in}}{\pgfqpoint{4.225000in}{4.225000in}}%
\pgfusepath{clip}%
\pgfsetbuttcap%
\pgfsetroundjoin%
\definecolor{currentfill}{rgb}{0.270595,0.214069,0.507052}%
\pgfsetfillcolor{currentfill}%
\pgfsetfillopacity{0.700000}%
\pgfsetlinewidth{0.501875pt}%
\definecolor{currentstroke}{rgb}{1.000000,1.000000,1.000000}%
\pgfsetstrokecolor{currentstroke}%
\pgfsetstrokeopacity{0.500000}%
\pgfsetdash{}{0pt}%
\pgfpathmoveto{\pgfqpoint{2.477829in}{1.262693in}}%
\pgfpathlineto{\pgfqpoint{2.489734in}{1.266523in}}%
\pgfpathlineto{\pgfqpoint{2.501633in}{1.270356in}}%
\pgfpathlineto{\pgfqpoint{2.513526in}{1.274195in}}%
\pgfpathlineto{\pgfqpoint{2.525413in}{1.278041in}}%
\pgfpathlineto{\pgfqpoint{2.537293in}{1.281896in}}%
\pgfpathlineto{\pgfqpoint{2.530993in}{1.293820in}}%
\pgfpathlineto{\pgfqpoint{2.524696in}{1.305749in}}%
\pgfpathlineto{\pgfqpoint{2.518403in}{1.317680in}}%
\pgfpathlineto{\pgfqpoint{2.512112in}{1.329609in}}%
\pgfpathlineto{\pgfqpoint{2.505825in}{1.341533in}}%
\pgfpathlineto{\pgfqpoint{2.493954in}{1.337600in}}%
\pgfpathlineto{\pgfqpoint{2.482077in}{1.333677in}}%
\pgfpathlineto{\pgfqpoint{2.470194in}{1.329762in}}%
\pgfpathlineto{\pgfqpoint{2.458304in}{1.325852in}}%
\pgfpathlineto{\pgfqpoint{2.446409in}{1.321947in}}%
\pgfpathlineto{\pgfqpoint{2.452686in}{1.310104in}}%
\pgfpathlineto{\pgfqpoint{2.458967in}{1.298254in}}%
\pgfpathlineto{\pgfqpoint{2.465251in}{1.286401in}}%
\pgfpathlineto{\pgfqpoint{2.471538in}{1.274547in}}%
\pgfpathclose%
\pgfusepath{stroke,fill}%
\end{pgfscope}%
\begin{pgfscope}%
\pgfpathrectangle{\pgfqpoint{0.887500in}{0.275000in}}{\pgfqpoint{4.225000in}{4.225000in}}%
\pgfusepath{clip}%
\pgfsetbuttcap%
\pgfsetroundjoin%
\definecolor{currentfill}{rgb}{0.276194,0.190074,0.493001}%
\pgfsetfillcolor{currentfill}%
\pgfsetfillopacity{0.700000}%
\pgfsetlinewidth{0.501875pt}%
\definecolor{currentstroke}{rgb}{1.000000,1.000000,1.000000}%
\pgfsetstrokecolor{currentstroke}%
\pgfsetstrokeopacity{0.500000}%
\pgfsetdash{}{0pt}%
\pgfpathmoveto{\pgfqpoint{2.568845in}{1.222249in}}%
\pgfpathlineto{\pgfqpoint{2.580729in}{1.226028in}}%
\pgfpathlineto{\pgfqpoint{2.592606in}{1.229813in}}%
\pgfpathlineto{\pgfqpoint{2.604478in}{1.233604in}}%
\pgfpathlineto{\pgfqpoint{2.616343in}{1.237402in}}%
\pgfpathlineto{\pgfqpoint{2.628203in}{1.241207in}}%
\pgfpathlineto{\pgfqpoint{2.621876in}{1.253258in}}%
\pgfpathlineto{\pgfqpoint{2.615553in}{1.265292in}}%
\pgfpathlineto{\pgfqpoint{2.609233in}{1.277312in}}%
\pgfpathlineto{\pgfqpoint{2.602917in}{1.289323in}}%
\pgfpathlineto{\pgfqpoint{2.596604in}{1.301329in}}%
\pgfpathlineto{\pgfqpoint{2.584754in}{1.297424in}}%
\pgfpathlineto{\pgfqpoint{2.572898in}{1.293527in}}%
\pgfpathlineto{\pgfqpoint{2.561036in}{1.289639in}}%
\pgfpathlineto{\pgfqpoint{2.549168in}{1.285762in}}%
\pgfpathlineto{\pgfqpoint{2.537293in}{1.281896in}}%
\pgfpathlineto{\pgfqpoint{2.543597in}{1.269973in}}%
\pgfpathlineto{\pgfqpoint{2.549904in}{1.258049in}}%
\pgfpathlineto{\pgfqpoint{2.556214in}{1.246122in}}%
\pgfpathlineto{\pgfqpoint{2.562528in}{1.234189in}}%
\pgfpathclose%
\pgfusepath{stroke,fill}%
\end{pgfscope}%
\begin{pgfscope}%
\pgfpathrectangle{\pgfqpoint{0.887500in}{0.275000in}}{\pgfqpoint{4.225000in}{4.225000in}}%
\pgfusepath{clip}%
\pgfsetbuttcap%
\pgfsetroundjoin%
\definecolor{currentfill}{rgb}{0.283091,0.110553,0.431554}%
\pgfsetfillcolor{currentfill}%
\pgfsetfillopacity{0.700000}%
\pgfsetlinewidth{0.501875pt}%
\definecolor{currentstroke}{rgb}{1.000000,1.000000,1.000000}%
\pgfsetstrokecolor{currentstroke}%
\pgfsetstrokeopacity{0.500000}%
\pgfsetdash{}{0pt}%
\pgfpathmoveto{\pgfqpoint{2.842020in}{1.095489in}}%
\pgfpathlineto{\pgfqpoint{2.853838in}{1.098828in}}%
\pgfpathlineto{\pgfqpoint{2.865650in}{1.102149in}}%
\pgfpathlineto{\pgfqpoint{2.877455in}{1.105433in}}%
\pgfpathlineto{\pgfqpoint{2.889255in}{1.108658in}}%
\pgfpathlineto{\pgfqpoint{2.901048in}{1.111806in}}%
\pgfpathlineto{\pgfqpoint{2.894655in}{1.122399in}}%
\pgfpathlineto{\pgfqpoint{2.888262in}{1.133600in}}%
\pgfpathlineto{\pgfqpoint{2.881871in}{1.145331in}}%
\pgfpathlineto{\pgfqpoint{2.875481in}{1.157512in}}%
\pgfpathlineto{\pgfqpoint{2.869093in}{1.170061in}}%
\pgfpathlineto{\pgfqpoint{2.857308in}{1.166824in}}%
\pgfpathlineto{\pgfqpoint{2.845517in}{1.163541in}}%
\pgfpathlineto{\pgfqpoint{2.833719in}{1.160224in}}%
\pgfpathlineto{\pgfqpoint{2.821916in}{1.156884in}}%
\pgfpathlineto{\pgfqpoint{2.810106in}{1.153533in}}%
\pgfpathlineto{\pgfqpoint{2.816486in}{1.141251in}}%
\pgfpathlineto{\pgfqpoint{2.822867in}{1.129248in}}%
\pgfpathlineto{\pgfqpoint{2.829250in}{1.117581in}}%
\pgfpathlineto{\pgfqpoint{2.835634in}{1.106309in}}%
\pgfpathclose%
\pgfusepath{stroke,fill}%
\end{pgfscope}%
\begin{pgfscope}%
\pgfpathrectangle{\pgfqpoint{0.887500in}{0.275000in}}{\pgfqpoint{4.225000in}{4.225000in}}%
\pgfusepath{clip}%
\pgfsetbuttcap%
\pgfsetroundjoin%
\definecolor{currentfill}{rgb}{0.282623,0.140926,0.457517}%
\pgfsetfillcolor{currentfill}%
\pgfsetfillopacity{0.700000}%
\pgfsetlinewidth{0.501875pt}%
\definecolor{currentstroke}{rgb}{1.000000,1.000000,1.000000}%
\pgfsetstrokecolor{currentstroke}%
\pgfsetstrokeopacity{0.500000}%
\pgfsetdash{}{0pt}%
\pgfpathmoveto{\pgfqpoint{2.750961in}{1.136800in}}%
\pgfpathlineto{\pgfqpoint{2.762803in}{1.140148in}}%
\pgfpathlineto{\pgfqpoint{2.774638in}{1.143491in}}%
\pgfpathlineto{\pgfqpoint{2.786467in}{1.146835in}}%
\pgfpathlineto{\pgfqpoint{2.798290in}{1.150181in}}%
\pgfpathlineto{\pgfqpoint{2.810106in}{1.153533in}}%
\pgfpathlineto{\pgfqpoint{2.803728in}{1.166036in}}%
\pgfpathlineto{\pgfqpoint{2.797352in}{1.178702in}}%
\pgfpathlineto{\pgfqpoint{2.790978in}{1.191474in}}%
\pgfpathlineto{\pgfqpoint{2.784607in}{1.204294in}}%
\pgfpathlineto{\pgfqpoint{2.778240in}{1.217104in}}%
\pgfpathlineto{\pgfqpoint{2.766433in}{1.213423in}}%
\pgfpathlineto{\pgfqpoint{2.754619in}{1.209750in}}%
\pgfpathlineto{\pgfqpoint{2.742800in}{1.206082in}}%
\pgfpathlineto{\pgfqpoint{2.730974in}{1.202421in}}%
\pgfpathlineto{\pgfqpoint{2.719142in}{1.198765in}}%
\pgfpathlineto{\pgfqpoint{2.725500in}{1.186285in}}%
\pgfpathlineto{\pgfqpoint{2.731862in}{1.173809in}}%
\pgfpathlineto{\pgfqpoint{2.738226in}{1.161376in}}%
\pgfpathlineto{\pgfqpoint{2.744592in}{1.149027in}}%
\pgfpathclose%
\pgfusepath{stroke,fill}%
\end{pgfscope}%
\begin{pgfscope}%
\pgfpathrectangle{\pgfqpoint{0.887500in}{0.275000in}}{\pgfqpoint{4.225000in}{4.225000in}}%
\pgfusepath{clip}%
\pgfsetbuttcap%
\pgfsetroundjoin%
\definecolor{currentfill}{rgb}{0.280255,0.165693,0.476498}%
\pgfsetfillcolor{currentfill}%
\pgfsetfillopacity{0.700000}%
\pgfsetlinewidth{0.501875pt}%
\definecolor{currentstroke}{rgb}{1.000000,1.000000,1.000000}%
\pgfsetstrokecolor{currentstroke}%
\pgfsetstrokeopacity{0.500000}%
\pgfsetdash{}{0pt}%
\pgfpathmoveto{\pgfqpoint{2.659890in}{1.180556in}}%
\pgfpathlineto{\pgfqpoint{2.671753in}{1.184191in}}%
\pgfpathlineto{\pgfqpoint{2.683609in}{1.187828in}}%
\pgfpathlineto{\pgfqpoint{2.695460in}{1.191469in}}%
\pgfpathlineto{\pgfqpoint{2.707304in}{1.195115in}}%
\pgfpathlineto{\pgfqpoint{2.719142in}{1.198765in}}%
\pgfpathlineto{\pgfqpoint{2.712788in}{1.211210in}}%
\pgfpathlineto{\pgfqpoint{2.706437in}{1.223590in}}%
\pgfpathlineto{\pgfqpoint{2.700090in}{1.235905in}}%
\pgfpathlineto{\pgfqpoint{2.693747in}{1.248163in}}%
\pgfpathlineto{\pgfqpoint{2.687407in}{1.260372in}}%
\pgfpathlineto{\pgfqpoint{2.675579in}{1.256520in}}%
\pgfpathlineto{\pgfqpoint{2.663744in}{1.252677in}}%
\pgfpathlineto{\pgfqpoint{2.651903in}{1.248844in}}%
\pgfpathlineto{\pgfqpoint{2.640056in}{1.245021in}}%
\pgfpathlineto{\pgfqpoint{2.628203in}{1.241207in}}%
\pgfpathlineto{\pgfqpoint{2.634533in}{1.229135in}}%
\pgfpathlineto{\pgfqpoint{2.640867in}{1.217037in}}%
\pgfpathlineto{\pgfqpoint{2.647204in}{1.204910in}}%
\pgfpathlineto{\pgfqpoint{2.653545in}{1.192749in}}%
\pgfpathclose%
\pgfusepath{stroke,fill}%
\end{pgfscope}%
\begin{pgfscope}%
\pgfpathrectangle{\pgfqpoint{0.887500in}{0.275000in}}{\pgfqpoint{4.225000in}{4.225000in}}%
\pgfusepath{clip}%
\pgfsetbuttcap%
\pgfsetroundjoin%
\definecolor{currentfill}{rgb}{0.282327,0.094955,0.417331}%
\pgfsetfillcolor{currentfill}%
\pgfsetfillopacity{0.700000}%
\pgfsetlinewidth{0.501875pt}%
\definecolor{currentstroke}{rgb}{1.000000,1.000000,1.000000}%
\pgfsetstrokecolor{currentstroke}%
\pgfsetstrokeopacity{0.500000}%
\pgfsetdash{}{0pt}%
\pgfpathmoveto{\pgfqpoint{2.933036in}{1.070788in}}%
\pgfpathlineto{\pgfqpoint{2.944834in}{1.074767in}}%
\pgfpathlineto{\pgfqpoint{2.956626in}{1.078661in}}%
\pgfpathlineto{\pgfqpoint{2.968412in}{1.082514in}}%
\pgfpathlineto{\pgfqpoint{2.980192in}{1.086388in}}%
\pgfpathlineto{\pgfqpoint{2.991966in}{1.090348in}}%
\pgfpathlineto{\pgfqpoint{2.985552in}{1.095081in}}%
\pgfpathlineto{\pgfqpoint{2.979141in}{1.101175in}}%
\pgfpathlineto{\pgfqpoint{2.972732in}{1.108519in}}%
\pgfpathlineto{\pgfqpoint{2.966326in}{1.117002in}}%
\pgfpathlineto{\pgfqpoint{2.959922in}{1.126514in}}%
\pgfpathlineto{\pgfqpoint{2.948160in}{1.123561in}}%
\pgfpathlineto{\pgfqpoint{2.936392in}{1.120679in}}%
\pgfpathlineto{\pgfqpoint{2.924617in}{1.117800in}}%
\pgfpathlineto{\pgfqpoint{2.912836in}{1.114857in}}%
\pgfpathlineto{\pgfqpoint{2.901048in}{1.111806in}}%
\pgfpathlineto{\pgfqpoint{2.907443in}{1.101903in}}%
\pgfpathlineto{\pgfqpoint{2.913839in}{1.092769in}}%
\pgfpathlineto{\pgfqpoint{2.920237in}{1.084485in}}%
\pgfpathlineto{\pgfqpoint{2.926635in}{1.077131in}}%
\pgfpathclose%
\pgfusepath{stroke,fill}%
\end{pgfscope}%
\begin{pgfscope}%
\pgfpathrectangle{\pgfqpoint{0.887500in}{0.275000in}}{\pgfqpoint{4.225000in}{4.225000in}}%
\pgfusepath{clip}%
\pgfsetbuttcap%
\pgfsetroundjoin%
\definecolor{currentfill}{rgb}{0.283197,0.115680,0.436115}%
\pgfsetfillcolor{currentfill}%
\pgfsetfillopacity{0.700000}%
\pgfsetlinewidth{0.501875pt}%
\definecolor{currentstroke}{rgb}{1.000000,1.000000,1.000000}%
\pgfsetstrokecolor{currentstroke}%
\pgfsetstrokeopacity{0.500000}%
\pgfsetdash{}{0pt}%
\pgfpathmoveto{\pgfqpoint{3.083029in}{1.126292in}}%
\pgfpathlineto{\pgfqpoint{3.094802in}{1.133307in}}%
\pgfpathlineto{\pgfqpoint{3.106570in}{1.140182in}}%
\pgfpathlineto{\pgfqpoint{3.118334in}{1.146857in}}%
\pgfpathlineto{\pgfqpoint{3.130093in}{1.153270in}}%
\pgfpathlineto{\pgfqpoint{3.141847in}{1.159359in}}%
\pgfpathlineto{\pgfqpoint{3.135330in}{1.151095in}}%
\pgfpathlineto{\pgfqpoint{3.128827in}{1.145209in}}%
\pgfpathlineto{\pgfqpoint{3.122338in}{1.141562in}}%
\pgfpathlineto{\pgfqpoint{3.115861in}{1.140017in}}%
\pgfpathlineto{\pgfqpoint{3.109395in}{1.140437in}}%
\pgfpathlineto{\pgfqpoint{3.097675in}{1.135169in}}%
\pgfpathlineto{\pgfqpoint{3.085950in}{1.129756in}}%
\pgfpathlineto{\pgfqpoint{3.074221in}{1.124280in}}%
\pgfpathlineto{\pgfqpoint{3.062486in}{1.118822in}}%
\pgfpathlineto{\pgfqpoint{3.050746in}{1.113464in}}%
\pgfpathlineto{\pgfqpoint{3.057186in}{1.111985in}}%
\pgfpathlineto{\pgfqpoint{3.063633in}{1.112393in}}%
\pgfpathlineto{\pgfqpoint{3.070089in}{1.114824in}}%
\pgfpathlineto{\pgfqpoint{3.076553in}{1.119411in}}%
\pgfpathclose%
\pgfusepath{stroke,fill}%
\end{pgfscope}%
\begin{pgfscope}%
\pgfpathrectangle{\pgfqpoint{0.887500in}{0.275000in}}{\pgfqpoint{4.225000in}{4.225000in}}%
\pgfusepath{clip}%
\pgfsetbuttcap%
\pgfsetroundjoin%
\definecolor{currentfill}{rgb}{0.257322,0.256130,0.526563}%
\pgfsetfillcolor{currentfill}%
\pgfsetfillopacity{0.700000}%
\pgfsetlinewidth{0.501875pt}%
\definecolor{currentstroke}{rgb}{1.000000,1.000000,1.000000}%
\pgfsetstrokecolor{currentstroke}%
\pgfsetstrokeopacity{0.500000}%
\pgfsetdash{}{0pt}%
\pgfpathmoveto{\pgfqpoint{2.236070in}{1.321523in}}%
\pgfpathlineto{\pgfqpoint{2.248046in}{1.325473in}}%
\pgfpathlineto{\pgfqpoint{2.260016in}{1.329424in}}%
\pgfpathlineto{\pgfqpoint{2.271980in}{1.333376in}}%
\pgfpathlineto{\pgfqpoint{2.283938in}{1.337331in}}%
\pgfpathlineto{\pgfqpoint{2.295890in}{1.341286in}}%
\pgfpathlineto{\pgfqpoint{2.289654in}{1.352914in}}%
\pgfpathlineto{\pgfqpoint{2.283422in}{1.364524in}}%
\pgfpathlineto{\pgfqpoint{2.277194in}{1.376113in}}%
\pgfpathlineto{\pgfqpoint{2.270970in}{1.387680in}}%
\pgfpathlineto{\pgfqpoint{2.264750in}{1.399226in}}%
\pgfpathlineto{\pgfqpoint{2.252808in}{1.395228in}}%
\pgfpathlineto{\pgfqpoint{2.240860in}{1.391232in}}%
\pgfpathlineto{\pgfqpoint{2.228907in}{1.387238in}}%
\pgfpathlineto{\pgfqpoint{2.216947in}{1.383246in}}%
\pgfpathlineto{\pgfqpoint{2.204981in}{1.379256in}}%
\pgfpathlineto{\pgfqpoint{2.211192in}{1.367748in}}%
\pgfpathlineto{\pgfqpoint{2.217406in}{1.356219in}}%
\pgfpathlineto{\pgfqpoint{2.223623in}{1.344672in}}%
\pgfpathlineto{\pgfqpoint{2.229845in}{1.333106in}}%
\pgfpathclose%
\pgfusepath{stroke,fill}%
\end{pgfscope}%
\begin{pgfscope}%
\pgfpathrectangle{\pgfqpoint{0.887500in}{0.275000in}}{\pgfqpoint{4.225000in}{4.225000in}}%
\pgfusepath{clip}%
\pgfsetbuttcap%
\pgfsetroundjoin%
\definecolor{currentfill}{rgb}{0.265145,0.232956,0.516599}%
\pgfsetfillcolor{currentfill}%
\pgfsetfillopacity{0.700000}%
\pgfsetlinewidth{0.501875pt}%
\definecolor{currentstroke}{rgb}{1.000000,1.000000,1.000000}%
\pgfsetstrokecolor{currentstroke}%
\pgfsetstrokeopacity{0.500000}%
\pgfsetdash{}{0pt}%
\pgfpathmoveto{\pgfqpoint{2.327122in}{1.282927in}}%
\pgfpathlineto{\pgfqpoint{2.339078in}{1.286828in}}%
\pgfpathlineto{\pgfqpoint{2.351027in}{1.290730in}}%
\pgfpathlineto{\pgfqpoint{2.362971in}{1.294632in}}%
\pgfpathlineto{\pgfqpoint{2.374909in}{1.298534in}}%
\pgfpathlineto{\pgfqpoint{2.386841in}{1.302436in}}%
\pgfpathlineto{\pgfqpoint{2.380577in}{1.314195in}}%
\pgfpathlineto{\pgfqpoint{2.374317in}{1.325943in}}%
\pgfpathlineto{\pgfqpoint{2.368060in}{1.337677in}}%
\pgfpathlineto{\pgfqpoint{2.361807in}{1.349396in}}%
\pgfpathlineto{\pgfqpoint{2.355558in}{1.361096in}}%
\pgfpathlineto{\pgfqpoint{2.343636in}{1.357130in}}%
\pgfpathlineto{\pgfqpoint{2.331709in}{1.353166in}}%
\pgfpathlineto{\pgfqpoint{2.319775in}{1.349204in}}%
\pgfpathlineto{\pgfqpoint{2.307835in}{1.345244in}}%
\pgfpathlineto{\pgfqpoint{2.295890in}{1.341286in}}%
\pgfpathlineto{\pgfqpoint{2.302129in}{1.329641in}}%
\pgfpathlineto{\pgfqpoint{2.308372in}{1.317981in}}%
\pgfpathlineto{\pgfqpoint{2.314618in}{1.306307in}}%
\pgfpathlineto{\pgfqpoint{2.320868in}{1.294622in}}%
\pgfpathclose%
\pgfusepath{stroke,fill}%
\end{pgfscope}%
\begin{pgfscope}%
\pgfpathrectangle{\pgfqpoint{0.887500in}{0.275000in}}{\pgfqpoint{4.225000in}{4.225000in}}%
\pgfusepath{clip}%
\pgfsetbuttcap%
\pgfsetroundjoin%
\definecolor{currentfill}{rgb}{0.270595,0.214069,0.507052}%
\pgfsetfillcolor{currentfill}%
\pgfsetfillopacity{0.700000}%
\pgfsetlinewidth{0.501875pt}%
\definecolor{currentstroke}{rgb}{1.000000,1.000000,1.000000}%
\pgfsetstrokecolor{currentstroke}%
\pgfsetstrokeopacity{0.500000}%
\pgfsetdash{}{0pt}%
\pgfpathmoveto{\pgfqpoint{2.418211in}{1.243542in}}%
\pgfpathlineto{\pgfqpoint{2.430147in}{1.247374in}}%
\pgfpathlineto{\pgfqpoint{2.442076in}{1.251205in}}%
\pgfpathlineto{\pgfqpoint{2.454000in}{1.255036in}}%
\pgfpathlineto{\pgfqpoint{2.465917in}{1.258864in}}%
\pgfpathlineto{\pgfqpoint{2.477829in}{1.262693in}}%
\pgfpathlineto{\pgfqpoint{2.471538in}{1.274547in}}%
\pgfpathlineto{\pgfqpoint{2.465251in}{1.286401in}}%
\pgfpathlineto{\pgfqpoint{2.458967in}{1.298254in}}%
\pgfpathlineto{\pgfqpoint{2.452686in}{1.310104in}}%
\pgfpathlineto{\pgfqpoint{2.446409in}{1.321947in}}%
\pgfpathlineto{\pgfqpoint{2.434507in}{1.318045in}}%
\pgfpathlineto{\pgfqpoint{2.422600in}{1.314143in}}%
\pgfpathlineto{\pgfqpoint{2.410686in}{1.310240in}}%
\pgfpathlineto{\pgfqpoint{2.398766in}{1.306338in}}%
\pgfpathlineto{\pgfqpoint{2.386841in}{1.302436in}}%
\pgfpathlineto{\pgfqpoint{2.393108in}{1.290668in}}%
\pgfpathlineto{\pgfqpoint{2.399379in}{1.278892in}}%
\pgfpathlineto{\pgfqpoint{2.405653in}{1.267111in}}%
\pgfpathlineto{\pgfqpoint{2.411930in}{1.255327in}}%
\pgfpathclose%
\pgfusepath{stroke,fill}%
\end{pgfscope}%
\begin{pgfscope}%
\pgfpathrectangle{\pgfqpoint{0.887500in}{0.275000in}}{\pgfqpoint{4.225000in}{4.225000in}}%
\pgfusepath{clip}%
\pgfsetbuttcap%
\pgfsetroundjoin%
\definecolor{currentfill}{rgb}{0.276194,0.190074,0.493001}%
\pgfsetfillcolor{currentfill}%
\pgfsetfillopacity{0.700000}%
\pgfsetlinewidth{0.501875pt}%
\definecolor{currentstroke}{rgb}{1.000000,1.000000,1.000000}%
\pgfsetstrokecolor{currentstroke}%
\pgfsetstrokeopacity{0.500000}%
\pgfsetdash{}{0pt}%
\pgfpathmoveto{\pgfqpoint{2.509333in}{1.203411in}}%
\pgfpathlineto{\pgfqpoint{2.521247in}{1.207174in}}%
\pgfpathlineto{\pgfqpoint{2.533156in}{1.210938in}}%
\pgfpathlineto{\pgfqpoint{2.545058in}{1.214705in}}%
\pgfpathlineto{\pgfqpoint{2.556955in}{1.218475in}}%
\pgfpathlineto{\pgfqpoint{2.568845in}{1.222249in}}%
\pgfpathlineto{\pgfqpoint{2.562528in}{1.234189in}}%
\pgfpathlineto{\pgfqpoint{2.556214in}{1.246122in}}%
\pgfpathlineto{\pgfqpoint{2.549904in}{1.258049in}}%
\pgfpathlineto{\pgfqpoint{2.543597in}{1.269973in}}%
\pgfpathlineto{\pgfqpoint{2.537293in}{1.281896in}}%
\pgfpathlineto{\pgfqpoint{2.525413in}{1.278041in}}%
\pgfpathlineto{\pgfqpoint{2.513526in}{1.274195in}}%
\pgfpathlineto{\pgfqpoint{2.501633in}{1.270356in}}%
\pgfpathlineto{\pgfqpoint{2.489734in}{1.266523in}}%
\pgfpathlineto{\pgfqpoint{2.477829in}{1.262693in}}%
\pgfpathlineto{\pgfqpoint{2.484123in}{1.250840in}}%
\pgfpathlineto{\pgfqpoint{2.490420in}{1.238986in}}%
\pgfpathlineto{\pgfqpoint{2.496721in}{1.227130in}}%
\pgfpathlineto{\pgfqpoint{2.503025in}{1.215272in}}%
\pgfpathclose%
\pgfusepath{stroke,fill}%
\end{pgfscope}%
\begin{pgfscope}%
\pgfpathrectangle{\pgfqpoint{0.887500in}{0.275000in}}{\pgfqpoint{4.225000in}{4.225000in}}%
\pgfusepath{clip}%
\pgfsetbuttcap%
\pgfsetroundjoin%
\definecolor{currentfill}{rgb}{0.280255,0.165693,0.476498}%
\pgfsetfillcolor{currentfill}%
\pgfsetfillopacity{0.700000}%
\pgfsetlinewidth{0.501875pt}%
\definecolor{currentstroke}{rgb}{1.000000,1.000000,1.000000}%
\pgfsetstrokecolor{currentstroke}%
\pgfsetstrokeopacity{0.500000}%
\pgfsetdash{}{0pt}%
\pgfpathmoveto{\pgfqpoint{2.600483in}{1.162354in}}%
\pgfpathlineto{\pgfqpoint{2.612377in}{1.166003in}}%
\pgfpathlineto{\pgfqpoint{2.624264in}{1.169646in}}%
\pgfpathlineto{\pgfqpoint{2.636145in}{1.173285in}}%
\pgfpathlineto{\pgfqpoint{2.648021in}{1.176921in}}%
\pgfpathlineto{\pgfqpoint{2.659890in}{1.180556in}}%
\pgfpathlineto{\pgfqpoint{2.653545in}{1.192749in}}%
\pgfpathlineto{\pgfqpoint{2.647204in}{1.204910in}}%
\pgfpathlineto{\pgfqpoint{2.640867in}{1.217037in}}%
\pgfpathlineto{\pgfqpoint{2.634533in}{1.229135in}}%
\pgfpathlineto{\pgfqpoint{2.628203in}{1.241207in}}%
\pgfpathlineto{\pgfqpoint{2.616343in}{1.237402in}}%
\pgfpathlineto{\pgfqpoint{2.604478in}{1.233604in}}%
\pgfpathlineto{\pgfqpoint{2.592606in}{1.229813in}}%
\pgfpathlineto{\pgfqpoint{2.580729in}{1.226028in}}%
\pgfpathlineto{\pgfqpoint{2.568845in}{1.222249in}}%
\pgfpathlineto{\pgfqpoint{2.575165in}{1.210299in}}%
\pgfpathlineto{\pgfqpoint{2.581489in}{1.198336in}}%
\pgfpathlineto{\pgfqpoint{2.587817in}{1.186359in}}%
\pgfpathlineto{\pgfqpoint{2.594148in}{1.174365in}}%
\pgfpathclose%
\pgfusepath{stroke,fill}%
\end{pgfscope}%
\begin{pgfscope}%
\pgfpathrectangle{\pgfqpoint{0.887500in}{0.275000in}}{\pgfqpoint{4.225000in}{4.225000in}}%
\pgfusepath{clip}%
\pgfsetbuttcap%
\pgfsetroundjoin%
\definecolor{currentfill}{rgb}{0.282623,0.140926,0.457517}%
\pgfsetfillcolor{currentfill}%
\pgfsetfillopacity{0.700000}%
\pgfsetlinewidth{0.501875pt}%
\definecolor{currentstroke}{rgb}{1.000000,1.000000,1.000000}%
\pgfsetstrokecolor{currentstroke}%
\pgfsetstrokeopacity{0.500000}%
\pgfsetdash{}{0pt}%
\pgfpathmoveto{\pgfqpoint{2.691660in}{1.119898in}}%
\pgfpathlineto{\pgfqpoint{2.703533in}{1.123308in}}%
\pgfpathlineto{\pgfqpoint{2.715399in}{1.126702in}}%
\pgfpathlineto{\pgfqpoint{2.727260in}{1.130080in}}%
\pgfpathlineto{\pgfqpoint{2.739114in}{1.133445in}}%
\pgfpathlineto{\pgfqpoint{2.750961in}{1.136800in}}%
\pgfpathlineto{\pgfqpoint{2.744592in}{1.149027in}}%
\pgfpathlineto{\pgfqpoint{2.738226in}{1.161376in}}%
\pgfpathlineto{\pgfqpoint{2.731862in}{1.173809in}}%
\pgfpathlineto{\pgfqpoint{2.725500in}{1.186285in}}%
\pgfpathlineto{\pgfqpoint{2.719142in}{1.198765in}}%
\pgfpathlineto{\pgfqpoint{2.707304in}{1.195115in}}%
\pgfpathlineto{\pgfqpoint{2.695460in}{1.191469in}}%
\pgfpathlineto{\pgfqpoint{2.683609in}{1.187828in}}%
\pgfpathlineto{\pgfqpoint{2.671753in}{1.184191in}}%
\pgfpathlineto{\pgfqpoint{2.659890in}{1.180556in}}%
\pgfpathlineto{\pgfqpoint{2.666238in}{1.168348in}}%
\pgfpathlineto{\pgfqpoint{2.672589in}{1.156153in}}%
\pgfpathlineto{\pgfqpoint{2.678943in}{1.143996in}}%
\pgfpathlineto{\pgfqpoint{2.685300in}{1.131903in}}%
\pgfpathclose%
\pgfusepath{stroke,fill}%
\end{pgfscope}%
\begin{pgfscope}%
\pgfpathrectangle{\pgfqpoint{0.887500in}{0.275000in}}{\pgfqpoint{4.225000in}{4.225000in}}%
\pgfusepath{clip}%
\pgfsetbuttcap%
\pgfsetroundjoin%
\definecolor{currentfill}{rgb}{0.283197,0.115680,0.436115}%
\pgfsetfillcolor{currentfill}%
\pgfsetfillopacity{0.700000}%
\pgfsetlinewidth{0.501875pt}%
\definecolor{currentstroke}{rgb}{1.000000,1.000000,1.000000}%
\pgfsetstrokecolor{currentstroke}%
\pgfsetstrokeopacity{0.500000}%
\pgfsetdash{}{0pt}%
\pgfpathmoveto{\pgfqpoint{2.782833in}{1.078877in}}%
\pgfpathlineto{\pgfqpoint{2.794683in}{1.082194in}}%
\pgfpathlineto{\pgfqpoint{2.806527in}{1.085508in}}%
\pgfpathlineto{\pgfqpoint{2.818364in}{1.088825in}}%
\pgfpathlineto{\pgfqpoint{2.830195in}{1.092151in}}%
\pgfpathlineto{\pgfqpoint{2.842020in}{1.095489in}}%
\pgfpathlineto{\pgfqpoint{2.835634in}{1.106309in}}%
\pgfpathlineto{\pgfqpoint{2.829250in}{1.117581in}}%
\pgfpathlineto{\pgfqpoint{2.822867in}{1.129248in}}%
\pgfpathlineto{\pgfqpoint{2.816486in}{1.141251in}}%
\pgfpathlineto{\pgfqpoint{2.810106in}{1.153533in}}%
\pgfpathlineto{\pgfqpoint{2.798290in}{1.150181in}}%
\pgfpathlineto{\pgfqpoint{2.786467in}{1.146835in}}%
\pgfpathlineto{\pgfqpoint{2.774638in}{1.143491in}}%
\pgfpathlineto{\pgfqpoint{2.762803in}{1.140148in}}%
\pgfpathlineto{\pgfqpoint{2.750961in}{1.136800in}}%
\pgfpathlineto{\pgfqpoint{2.757333in}{1.124735in}}%
\pgfpathlineto{\pgfqpoint{2.763705in}{1.112870in}}%
\pgfpathlineto{\pgfqpoint{2.770080in}{1.101246in}}%
\pgfpathlineto{\pgfqpoint{2.776456in}{1.089902in}}%
\pgfpathclose%
\pgfusepath{stroke,fill}%
\end{pgfscope}%
\begin{pgfscope}%
\pgfpathrectangle{\pgfqpoint{0.887500in}{0.275000in}}{\pgfqpoint{4.225000in}{4.225000in}}%
\pgfusepath{clip}%
\pgfsetbuttcap%
\pgfsetroundjoin%
\definecolor{currentfill}{rgb}{0.282910,0.105393,0.426902}%
\pgfsetfillcolor{currentfill}%
\pgfsetfillopacity{0.700000}%
\pgfsetlinewidth{0.501875pt}%
\definecolor{currentstroke}{rgb}{1.000000,1.000000,1.000000}%
\pgfsetstrokecolor{currentstroke}%
\pgfsetstrokeopacity{0.500000}%
\pgfsetdash{}{0pt}%
\pgfpathmoveto{\pgfqpoint{3.024105in}{1.091025in}}%
\pgfpathlineto{\pgfqpoint{3.035898in}{1.097998in}}%
\pgfpathlineto{\pgfqpoint{3.047687in}{1.105022in}}%
\pgfpathlineto{\pgfqpoint{3.059471in}{1.112094in}}%
\pgfpathlineto{\pgfqpoint{3.071252in}{1.119201in}}%
\pgfpathlineto{\pgfqpoint{3.083029in}{1.126292in}}%
\pgfpathlineto{\pgfqpoint{3.076553in}{1.119411in}}%
\pgfpathlineto{\pgfqpoint{3.070089in}{1.114824in}}%
\pgfpathlineto{\pgfqpoint{3.063633in}{1.112393in}}%
\pgfpathlineto{\pgfqpoint{3.057186in}{1.111985in}}%
\pgfpathlineto{\pgfqpoint{3.050746in}{1.113464in}}%
\pgfpathlineto{\pgfqpoint{3.039002in}{1.108287in}}%
\pgfpathlineto{\pgfqpoint{3.027251in}{1.103373in}}%
\pgfpathlineto{\pgfqpoint{3.015496in}{1.098776in}}%
\pgfpathlineto{\pgfqpoint{3.003734in}{1.094456in}}%
\pgfpathlineto{\pgfqpoint{2.991966in}{1.090348in}}%
\pgfpathlineto{\pgfqpoint{2.998384in}{1.087087in}}%
\pgfpathlineto{\pgfqpoint{3.004806in}{1.085412in}}%
\pgfpathlineto{\pgfqpoint{3.011233in}{1.085434in}}%
\pgfpathlineto{\pgfqpoint{3.017666in}{1.087267in}}%
\pgfpathclose%
\pgfusepath{stroke,fill}%
\end{pgfscope}%
\begin{pgfscope}%
\pgfpathrectangle{\pgfqpoint{0.887500in}{0.275000in}}{\pgfqpoint{4.225000in}{4.225000in}}%
\pgfusepath{clip}%
\pgfsetbuttcap%
\pgfsetroundjoin%
\definecolor{currentfill}{rgb}{0.282327,0.094955,0.417331}%
\pgfsetfillcolor{currentfill}%
\pgfsetfillopacity{0.700000}%
\pgfsetlinewidth{0.501875pt}%
\definecolor{currentstroke}{rgb}{1.000000,1.000000,1.000000}%
\pgfsetstrokecolor{currentstroke}%
\pgfsetstrokeopacity{0.500000}%
\pgfsetdash{}{0pt}%
\pgfpathmoveto{\pgfqpoint{2.873961in}{1.050195in}}%
\pgfpathlineto{\pgfqpoint{2.885788in}{1.054343in}}%
\pgfpathlineto{\pgfqpoint{2.897609in}{1.058495in}}%
\pgfpathlineto{\pgfqpoint{2.909424in}{1.062633in}}%
\pgfpathlineto{\pgfqpoint{2.921233in}{1.066737in}}%
\pgfpathlineto{\pgfqpoint{2.933036in}{1.070788in}}%
\pgfpathlineto{\pgfqpoint{2.926635in}{1.077131in}}%
\pgfpathlineto{\pgfqpoint{2.920237in}{1.084485in}}%
\pgfpathlineto{\pgfqpoint{2.913839in}{1.092769in}}%
\pgfpathlineto{\pgfqpoint{2.907443in}{1.101903in}}%
\pgfpathlineto{\pgfqpoint{2.901048in}{1.111806in}}%
\pgfpathlineto{\pgfqpoint{2.889255in}{1.108658in}}%
\pgfpathlineto{\pgfqpoint{2.877455in}{1.105433in}}%
\pgfpathlineto{\pgfqpoint{2.865650in}{1.102149in}}%
\pgfpathlineto{\pgfqpoint{2.853838in}{1.098828in}}%
\pgfpathlineto{\pgfqpoint{2.842020in}{1.095489in}}%
\pgfpathlineto{\pgfqpoint{2.848406in}{1.085179in}}%
\pgfpathlineto{\pgfqpoint{2.854793in}{1.075436in}}%
\pgfpathlineto{\pgfqpoint{2.861182in}{1.066319in}}%
\pgfpathlineto{\pgfqpoint{2.867571in}{1.057886in}}%
\pgfpathclose%
\pgfusepath{stroke,fill}%
\end{pgfscope}%
\begin{pgfscope}%
\pgfpathrectangle{\pgfqpoint{0.887500in}{0.275000in}}{\pgfqpoint{4.225000in}{4.225000in}}%
\pgfusepath{clip}%
\pgfsetbuttcap%
\pgfsetroundjoin%
\definecolor{currentfill}{rgb}{0.265145,0.232956,0.516599}%
\pgfsetfillcolor{currentfill}%
\pgfsetfillopacity{0.700000}%
\pgfsetlinewidth{0.501875pt}%
\definecolor{currentstroke}{rgb}{1.000000,1.000000,1.000000}%
\pgfsetstrokecolor{currentstroke}%
\pgfsetstrokeopacity{0.500000}%
\pgfsetdash{}{0pt}%
\pgfpathmoveto{\pgfqpoint{2.267251in}{1.263421in}}%
\pgfpathlineto{\pgfqpoint{2.279237in}{1.267321in}}%
\pgfpathlineto{\pgfqpoint{2.291217in}{1.271222in}}%
\pgfpathlineto{\pgfqpoint{2.303191in}{1.275124in}}%
\pgfpathlineto{\pgfqpoint{2.315160in}{1.279025in}}%
\pgfpathlineto{\pgfqpoint{2.327122in}{1.282927in}}%
\pgfpathlineto{\pgfqpoint{2.320868in}{1.294622in}}%
\pgfpathlineto{\pgfqpoint{2.314618in}{1.306307in}}%
\pgfpathlineto{\pgfqpoint{2.308372in}{1.317981in}}%
\pgfpathlineto{\pgfqpoint{2.302129in}{1.329641in}}%
\pgfpathlineto{\pgfqpoint{2.295890in}{1.341286in}}%
\pgfpathlineto{\pgfqpoint{2.283938in}{1.337331in}}%
\pgfpathlineto{\pgfqpoint{2.271980in}{1.333376in}}%
\pgfpathlineto{\pgfqpoint{2.260016in}{1.329424in}}%
\pgfpathlineto{\pgfqpoint{2.248046in}{1.325473in}}%
\pgfpathlineto{\pgfqpoint{2.236070in}{1.321523in}}%
\pgfpathlineto{\pgfqpoint{2.242299in}{1.309926in}}%
\pgfpathlineto{\pgfqpoint{2.248531in}{1.298315in}}%
\pgfpathlineto{\pgfqpoint{2.254768in}{1.286693in}}%
\pgfpathlineto{\pgfqpoint{2.261007in}{1.275061in}}%
\pgfpathclose%
\pgfusepath{stroke,fill}%
\end{pgfscope}%
\begin{pgfscope}%
\pgfpathrectangle{\pgfqpoint{0.887500in}{0.275000in}}{\pgfqpoint{4.225000in}{4.225000in}}%
\pgfusepath{clip}%
\pgfsetbuttcap%
\pgfsetroundjoin%
\definecolor{currentfill}{rgb}{0.270595,0.214069,0.507052}%
\pgfsetfillcolor{currentfill}%
\pgfsetfillopacity{0.700000}%
\pgfsetlinewidth{0.501875pt}%
\definecolor{currentstroke}{rgb}{1.000000,1.000000,1.000000}%
\pgfsetstrokecolor{currentstroke}%
\pgfsetstrokeopacity{0.500000}%
\pgfsetdash{}{0pt}%
\pgfpathmoveto{\pgfqpoint{2.358441in}{1.224364in}}%
\pgfpathlineto{\pgfqpoint{2.370408in}{1.228202in}}%
\pgfpathlineto{\pgfqpoint{2.382368in}{1.232038in}}%
\pgfpathlineto{\pgfqpoint{2.394322in}{1.235874in}}%
\pgfpathlineto{\pgfqpoint{2.406269in}{1.239709in}}%
\pgfpathlineto{\pgfqpoint{2.418211in}{1.243542in}}%
\pgfpathlineto{\pgfqpoint{2.411930in}{1.255327in}}%
\pgfpathlineto{\pgfqpoint{2.405653in}{1.267111in}}%
\pgfpathlineto{\pgfqpoint{2.399379in}{1.278892in}}%
\pgfpathlineto{\pgfqpoint{2.393108in}{1.290668in}}%
\pgfpathlineto{\pgfqpoint{2.386841in}{1.302436in}}%
\pgfpathlineto{\pgfqpoint{2.374909in}{1.298534in}}%
\pgfpathlineto{\pgfqpoint{2.362971in}{1.294632in}}%
\pgfpathlineto{\pgfqpoint{2.351027in}{1.290730in}}%
\pgfpathlineto{\pgfqpoint{2.339078in}{1.286828in}}%
\pgfpathlineto{\pgfqpoint{2.327122in}{1.282927in}}%
\pgfpathlineto{\pgfqpoint{2.333379in}{1.271223in}}%
\pgfpathlineto{\pgfqpoint{2.339639in}{1.259514in}}%
\pgfpathlineto{\pgfqpoint{2.345903in}{1.247799in}}%
\pgfpathlineto{\pgfqpoint{2.352171in}{1.236082in}}%
\pgfpathclose%
\pgfusepath{stroke,fill}%
\end{pgfscope}%
\begin{pgfscope}%
\pgfpathrectangle{\pgfqpoint{0.887500in}{0.275000in}}{\pgfqpoint{4.225000in}{4.225000in}}%
\pgfusepath{clip}%
\pgfsetbuttcap%
\pgfsetroundjoin%
\definecolor{currentfill}{rgb}{0.276194,0.190074,0.493001}%
\pgfsetfillcolor{currentfill}%
\pgfsetfillopacity{0.700000}%
\pgfsetlinewidth{0.501875pt}%
\definecolor{currentstroke}{rgb}{1.000000,1.000000,1.000000}%
\pgfsetstrokecolor{currentstroke}%
\pgfsetstrokeopacity{0.500000}%
\pgfsetdash{}{0pt}%
\pgfpathmoveto{\pgfqpoint{2.449666in}{1.184604in}}%
\pgfpathlineto{\pgfqpoint{2.461612in}{1.188366in}}%
\pgfpathlineto{\pgfqpoint{2.473551in}{1.192127in}}%
\pgfpathlineto{\pgfqpoint{2.485485in}{1.195888in}}%
\pgfpathlineto{\pgfqpoint{2.497412in}{1.199650in}}%
\pgfpathlineto{\pgfqpoint{2.509333in}{1.203411in}}%
\pgfpathlineto{\pgfqpoint{2.503025in}{1.215272in}}%
\pgfpathlineto{\pgfqpoint{2.496721in}{1.227130in}}%
\pgfpathlineto{\pgfqpoint{2.490420in}{1.238986in}}%
\pgfpathlineto{\pgfqpoint{2.484123in}{1.250840in}}%
\pgfpathlineto{\pgfqpoint{2.477829in}{1.262693in}}%
\pgfpathlineto{\pgfqpoint{2.465917in}{1.258864in}}%
\pgfpathlineto{\pgfqpoint{2.454000in}{1.255036in}}%
\pgfpathlineto{\pgfqpoint{2.442076in}{1.251205in}}%
\pgfpathlineto{\pgfqpoint{2.430147in}{1.247374in}}%
\pgfpathlineto{\pgfqpoint{2.418211in}{1.243542in}}%
\pgfpathlineto{\pgfqpoint{2.424495in}{1.231756in}}%
\pgfpathlineto{\pgfqpoint{2.430783in}{1.219968in}}%
\pgfpathlineto{\pgfqpoint{2.437074in}{1.208180in}}%
\pgfpathlineto{\pgfqpoint{2.443369in}{1.196392in}}%
\pgfpathclose%
\pgfusepath{stroke,fill}%
\end{pgfscope}%
\begin{pgfscope}%
\pgfpathrectangle{\pgfqpoint{0.887500in}{0.275000in}}{\pgfqpoint{4.225000in}{4.225000in}}%
\pgfusepath{clip}%
\pgfsetbuttcap%
\pgfsetroundjoin%
\definecolor{currentfill}{rgb}{0.280255,0.165693,0.476498}%
\pgfsetfillcolor{currentfill}%
\pgfsetfillopacity{0.700000}%
\pgfsetlinewidth{0.501875pt}%
\definecolor{currentstroke}{rgb}{1.000000,1.000000,1.000000}%
\pgfsetstrokecolor{currentstroke}%
\pgfsetstrokeopacity{0.500000}%
\pgfsetdash{}{0pt}%
\pgfpathmoveto{\pgfqpoint{2.540922in}{1.144036in}}%
\pgfpathlineto{\pgfqpoint{2.552847in}{1.147706in}}%
\pgfpathlineto{\pgfqpoint{2.564765in}{1.151374in}}%
\pgfpathlineto{\pgfqpoint{2.576677in}{1.155038in}}%
\pgfpathlineto{\pgfqpoint{2.588583in}{1.158699in}}%
\pgfpathlineto{\pgfqpoint{2.600483in}{1.162354in}}%
\pgfpathlineto{\pgfqpoint{2.594148in}{1.174365in}}%
\pgfpathlineto{\pgfqpoint{2.587817in}{1.186359in}}%
\pgfpathlineto{\pgfqpoint{2.581489in}{1.198336in}}%
\pgfpathlineto{\pgfqpoint{2.575165in}{1.210299in}}%
\pgfpathlineto{\pgfqpoint{2.568845in}{1.222249in}}%
\pgfpathlineto{\pgfqpoint{2.556955in}{1.218475in}}%
\pgfpathlineto{\pgfqpoint{2.545058in}{1.214705in}}%
\pgfpathlineto{\pgfqpoint{2.533156in}{1.210938in}}%
\pgfpathlineto{\pgfqpoint{2.521247in}{1.207174in}}%
\pgfpathlineto{\pgfqpoint{2.509333in}{1.203411in}}%
\pgfpathlineto{\pgfqpoint{2.515644in}{1.191546in}}%
\pgfpathlineto{\pgfqpoint{2.521958in}{1.179677in}}%
\pgfpathlineto{\pgfqpoint{2.528276in}{1.167803in}}%
\pgfpathlineto{\pgfqpoint{2.534597in}{1.155922in}}%
\pgfpathclose%
\pgfusepath{stroke,fill}%
\end{pgfscope}%
\begin{pgfscope}%
\pgfpathrectangle{\pgfqpoint{0.887500in}{0.275000in}}{\pgfqpoint{4.225000in}{4.225000in}}%
\pgfusepath{clip}%
\pgfsetbuttcap%
\pgfsetroundjoin%
\definecolor{currentfill}{rgb}{0.282656,0.100196,0.422160}%
\pgfsetfillcolor{currentfill}%
\pgfsetfillopacity{0.700000}%
\pgfsetlinewidth{0.501875pt}%
\definecolor{currentstroke}{rgb}{1.000000,1.000000,1.000000}%
\pgfsetstrokecolor{currentstroke}%
\pgfsetstrokeopacity{0.500000}%
\pgfsetdash{}{0pt}%
\pgfpathmoveto{\pgfqpoint{2.965076in}{1.057075in}}%
\pgfpathlineto{\pgfqpoint{2.976891in}{1.063731in}}%
\pgfpathlineto{\pgfqpoint{2.988701in}{1.070458in}}%
\pgfpathlineto{\pgfqpoint{3.000506in}{1.077251in}}%
\pgfpathlineto{\pgfqpoint{3.012308in}{1.084109in}}%
\pgfpathlineto{\pgfqpoint{3.024105in}{1.091025in}}%
\pgfpathlineto{\pgfqpoint{3.017666in}{1.087267in}}%
\pgfpathlineto{\pgfqpoint{3.011233in}{1.085434in}}%
\pgfpathlineto{\pgfqpoint{3.004806in}{1.085412in}}%
\pgfpathlineto{\pgfqpoint{2.998384in}{1.087087in}}%
\pgfpathlineto{\pgfqpoint{2.991966in}{1.090348in}}%
\pgfpathlineto{\pgfqpoint{2.980192in}{1.086388in}}%
\pgfpathlineto{\pgfqpoint{2.968412in}{1.082514in}}%
\pgfpathlineto{\pgfqpoint{2.956626in}{1.078661in}}%
\pgfpathlineto{\pgfqpoint{2.944834in}{1.074767in}}%
\pgfpathlineto{\pgfqpoint{2.933036in}{1.070788in}}%
\pgfpathlineto{\pgfqpoint{2.939439in}{1.065537in}}%
\pgfpathlineto{\pgfqpoint{2.945844in}{1.061458in}}%
\pgfpathlineto{\pgfqpoint{2.952251in}{1.058634in}}%
\pgfpathlineto{\pgfqpoint{2.958662in}{1.057145in}}%
\pgfpathclose%
\pgfusepath{stroke,fill}%
\end{pgfscope}%
\begin{pgfscope}%
\pgfpathrectangle{\pgfqpoint{0.887500in}{0.275000in}}{\pgfqpoint{4.225000in}{4.225000in}}%
\pgfusepath{clip}%
\pgfsetbuttcap%
\pgfsetroundjoin%
\definecolor{currentfill}{rgb}{0.282623,0.140926,0.457517}%
\pgfsetfillcolor{currentfill}%
\pgfsetfillopacity{0.700000}%
\pgfsetlinewidth{0.501875pt}%
\definecolor{currentstroke}{rgb}{1.000000,1.000000,1.000000}%
\pgfsetstrokecolor{currentstroke}%
\pgfsetstrokeopacity{0.500000}%
\pgfsetdash{}{0pt}%
\pgfpathmoveto{\pgfqpoint{2.632203in}{1.102600in}}%
\pgfpathlineto{\pgfqpoint{2.644107in}{1.106094in}}%
\pgfpathlineto{\pgfqpoint{2.656005in}{1.109570in}}%
\pgfpathlineto{\pgfqpoint{2.667896in}{1.113029in}}%
\pgfpathlineto{\pgfqpoint{2.679781in}{1.116472in}}%
\pgfpathlineto{\pgfqpoint{2.691660in}{1.119898in}}%
\pgfpathlineto{\pgfqpoint{2.685300in}{1.131903in}}%
\pgfpathlineto{\pgfqpoint{2.678943in}{1.143996in}}%
\pgfpathlineto{\pgfqpoint{2.672589in}{1.156153in}}%
\pgfpathlineto{\pgfqpoint{2.666238in}{1.168348in}}%
\pgfpathlineto{\pgfqpoint{2.659890in}{1.180556in}}%
\pgfpathlineto{\pgfqpoint{2.648021in}{1.176921in}}%
\pgfpathlineto{\pgfqpoint{2.636145in}{1.173285in}}%
\pgfpathlineto{\pgfqpoint{2.624264in}{1.169646in}}%
\pgfpathlineto{\pgfqpoint{2.612377in}{1.166003in}}%
\pgfpathlineto{\pgfqpoint{2.600483in}{1.162354in}}%
\pgfpathlineto{\pgfqpoint{2.606821in}{1.150340in}}%
\pgfpathlineto{\pgfqpoint{2.613162in}{1.138341in}}%
\pgfpathlineto{\pgfqpoint{2.619507in}{1.126373in}}%
\pgfpathlineto{\pgfqpoint{2.625854in}{1.114453in}}%
\pgfpathclose%
\pgfusepath{stroke,fill}%
\end{pgfscope}%
\begin{pgfscope}%
\pgfpathrectangle{\pgfqpoint{0.887500in}{0.275000in}}{\pgfqpoint{4.225000in}{4.225000in}}%
\pgfusepath{clip}%
\pgfsetbuttcap%
\pgfsetroundjoin%
\definecolor{currentfill}{rgb}{0.283197,0.115680,0.436115}%
\pgfsetfillcolor{currentfill}%
\pgfsetfillopacity{0.700000}%
\pgfsetlinewidth{0.501875pt}%
\definecolor{currentstroke}{rgb}{1.000000,1.000000,1.000000}%
\pgfsetstrokecolor{currentstroke}%
\pgfsetstrokeopacity{0.500000}%
\pgfsetdash{}{0pt}%
\pgfpathmoveto{\pgfqpoint{2.723489in}{1.062102in}}%
\pgfpathlineto{\pgfqpoint{2.735370in}{1.065492in}}%
\pgfpathlineto{\pgfqpoint{2.747245in}{1.068865in}}%
\pgfpathlineto{\pgfqpoint{2.759114in}{1.072218in}}%
\pgfpathlineto{\pgfqpoint{2.770977in}{1.075554in}}%
\pgfpathlineto{\pgfqpoint{2.782833in}{1.078877in}}%
\pgfpathlineto{\pgfqpoint{2.776456in}{1.089902in}}%
\pgfpathlineto{\pgfqpoint{2.770080in}{1.101246in}}%
\pgfpathlineto{\pgfqpoint{2.763705in}{1.112870in}}%
\pgfpathlineto{\pgfqpoint{2.757333in}{1.124735in}}%
\pgfpathlineto{\pgfqpoint{2.750961in}{1.136800in}}%
\pgfpathlineto{\pgfqpoint{2.739114in}{1.133445in}}%
\pgfpathlineto{\pgfqpoint{2.727260in}{1.130080in}}%
\pgfpathlineto{\pgfqpoint{2.715399in}{1.126702in}}%
\pgfpathlineto{\pgfqpoint{2.703533in}{1.123308in}}%
\pgfpathlineto{\pgfqpoint{2.691660in}{1.119898in}}%
\pgfpathlineto{\pgfqpoint{2.698022in}{1.108007in}}%
\pgfpathlineto{\pgfqpoint{2.704386in}{1.096257in}}%
\pgfpathlineto{\pgfqpoint{2.710751in}{1.084672in}}%
\pgfpathlineto{\pgfqpoint{2.717119in}{1.073279in}}%
\pgfpathclose%
\pgfusepath{stroke,fill}%
\end{pgfscope}%
\begin{pgfscope}%
\pgfpathrectangle{\pgfqpoint{0.887500in}{0.275000in}}{\pgfqpoint{4.225000in}{4.225000in}}%
\pgfusepath{clip}%
\pgfsetbuttcap%
\pgfsetroundjoin%
\definecolor{currentfill}{rgb}{0.282327,0.094955,0.417331}%
\pgfsetfillcolor{currentfill}%
\pgfsetfillopacity{0.700000}%
\pgfsetlinewidth{0.501875pt}%
\definecolor{currentstroke}{rgb}{1.000000,1.000000,1.000000}%
\pgfsetstrokecolor{currentstroke}%
\pgfsetstrokeopacity{0.500000}%
\pgfsetdash{}{0pt}%
\pgfpathmoveto{\pgfqpoint{2.814737in}{1.029918in}}%
\pgfpathlineto{\pgfqpoint{2.826595in}{1.033908in}}%
\pgfpathlineto{\pgfqpoint{2.838445in}{1.037929in}}%
\pgfpathlineto{\pgfqpoint{2.850290in}{1.041983in}}%
\pgfpathlineto{\pgfqpoint{2.862129in}{1.046071in}}%
\pgfpathlineto{\pgfqpoint{2.873961in}{1.050195in}}%
\pgfpathlineto{\pgfqpoint{2.867571in}{1.057886in}}%
\pgfpathlineto{\pgfqpoint{2.861182in}{1.066319in}}%
\pgfpathlineto{\pgfqpoint{2.854793in}{1.075436in}}%
\pgfpathlineto{\pgfqpoint{2.848406in}{1.085179in}}%
\pgfpathlineto{\pgfqpoint{2.842020in}{1.095489in}}%
\pgfpathlineto{\pgfqpoint{2.830195in}{1.092151in}}%
\pgfpathlineto{\pgfqpoint{2.818364in}{1.088825in}}%
\pgfpathlineto{\pgfqpoint{2.806527in}{1.085508in}}%
\pgfpathlineto{\pgfqpoint{2.794683in}{1.082194in}}%
\pgfpathlineto{\pgfqpoint{2.782833in}{1.078877in}}%
\pgfpathlineto{\pgfqpoint{2.789212in}{1.068211in}}%
\pgfpathlineto{\pgfqpoint{2.795592in}{1.057942in}}%
\pgfpathlineto{\pgfqpoint{2.801973in}{1.048111in}}%
\pgfpathlineto{\pgfqpoint{2.808354in}{1.038756in}}%
\pgfpathclose%
\pgfusepath{stroke,fill}%
\end{pgfscope}%
\begin{pgfscope}%
\pgfpathrectangle{\pgfqpoint{0.887500in}{0.275000in}}{\pgfqpoint{4.225000in}{4.225000in}}%
\pgfusepath{clip}%
\pgfsetbuttcap%
\pgfsetroundjoin%
\definecolor{currentfill}{rgb}{0.270595,0.214069,0.507052}%
\pgfsetfillcolor{currentfill}%
\pgfsetfillopacity{0.700000}%
\pgfsetlinewidth{0.501875pt}%
\definecolor{currentstroke}{rgb}{1.000000,1.000000,1.000000}%
\pgfsetstrokecolor{currentstroke}%
\pgfsetstrokeopacity{0.500000}%
\pgfsetdash{}{0pt}%
\pgfpathmoveto{\pgfqpoint{2.298519in}{1.205154in}}%
\pgfpathlineto{\pgfqpoint{2.310515in}{1.208999in}}%
\pgfpathlineto{\pgfqpoint{2.322506in}{1.212842in}}%
\pgfpathlineto{\pgfqpoint{2.334491in}{1.216684in}}%
\pgfpathlineto{\pgfqpoint{2.346469in}{1.220524in}}%
\pgfpathlineto{\pgfqpoint{2.358441in}{1.224364in}}%
\pgfpathlineto{\pgfqpoint{2.352171in}{1.236082in}}%
\pgfpathlineto{\pgfqpoint{2.345903in}{1.247799in}}%
\pgfpathlineto{\pgfqpoint{2.339639in}{1.259514in}}%
\pgfpathlineto{\pgfqpoint{2.333379in}{1.271223in}}%
\pgfpathlineto{\pgfqpoint{2.327122in}{1.282927in}}%
\pgfpathlineto{\pgfqpoint{2.315160in}{1.279025in}}%
\pgfpathlineto{\pgfqpoint{2.303191in}{1.275124in}}%
\pgfpathlineto{\pgfqpoint{2.291217in}{1.271222in}}%
\pgfpathlineto{\pgfqpoint{2.279237in}{1.267321in}}%
\pgfpathlineto{\pgfqpoint{2.267251in}{1.263421in}}%
\pgfpathlineto{\pgfqpoint{2.273497in}{1.251774in}}%
\pgfpathlineto{\pgfqpoint{2.279747in}{1.240122in}}%
\pgfpathlineto{\pgfqpoint{2.286001in}{1.228467in}}%
\pgfpathlineto{\pgfqpoint{2.292258in}{1.216811in}}%
\pgfpathclose%
\pgfusepath{stroke,fill}%
\end{pgfscope}%
\begin{pgfscope}%
\pgfpathrectangle{\pgfqpoint{0.887500in}{0.275000in}}{\pgfqpoint{4.225000in}{4.225000in}}%
\pgfusepath{clip}%
\pgfsetbuttcap%
\pgfsetroundjoin%
\definecolor{currentfill}{rgb}{0.282327,0.094955,0.417331}%
\pgfsetfillcolor{currentfill}%
\pgfsetfillopacity{0.700000}%
\pgfsetlinewidth{0.501875pt}%
\definecolor{currentstroke}{rgb}{1.000000,1.000000,1.000000}%
\pgfsetstrokecolor{currentstroke}%
\pgfsetstrokeopacity{0.500000}%
\pgfsetdash{}{0pt}%
\pgfpathmoveto{\pgfqpoint{2.905935in}{1.024911in}}%
\pgfpathlineto{\pgfqpoint{2.917773in}{1.031190in}}%
\pgfpathlineto{\pgfqpoint{2.929606in}{1.037548in}}%
\pgfpathlineto{\pgfqpoint{2.941434in}{1.043982in}}%
\pgfpathlineto{\pgfqpoint{2.953257in}{1.050492in}}%
\pgfpathlineto{\pgfqpoint{2.965076in}{1.057075in}}%
\pgfpathlineto{\pgfqpoint{2.958662in}{1.057145in}}%
\pgfpathlineto{\pgfqpoint{2.952251in}{1.058634in}}%
\pgfpathlineto{\pgfqpoint{2.945844in}{1.061458in}}%
\pgfpathlineto{\pgfqpoint{2.939439in}{1.065537in}}%
\pgfpathlineto{\pgfqpoint{2.933036in}{1.070788in}}%
\pgfpathlineto{\pgfqpoint{2.921233in}{1.066737in}}%
\pgfpathlineto{\pgfqpoint{2.909424in}{1.062633in}}%
\pgfpathlineto{\pgfqpoint{2.897609in}{1.058495in}}%
\pgfpathlineto{\pgfqpoint{2.885788in}{1.054343in}}%
\pgfpathlineto{\pgfqpoint{2.873961in}{1.050195in}}%
\pgfpathlineto{\pgfqpoint{2.880353in}{1.043304in}}%
\pgfpathlineto{\pgfqpoint{2.886746in}{1.037272in}}%
\pgfpathlineto{\pgfqpoint{2.893140in}{1.032156in}}%
\pgfpathlineto{\pgfqpoint{2.899537in}{1.028016in}}%
\pgfpathclose%
\pgfusepath{stroke,fill}%
\end{pgfscope}%
\begin{pgfscope}%
\pgfpathrectangle{\pgfqpoint{0.887500in}{0.275000in}}{\pgfqpoint{4.225000in}{4.225000in}}%
\pgfusepath{clip}%
\pgfsetbuttcap%
\pgfsetroundjoin%
\definecolor{currentfill}{rgb}{0.276194,0.190074,0.493001}%
\pgfsetfillcolor{currentfill}%
\pgfsetfillopacity{0.700000}%
\pgfsetlinewidth{0.501875pt}%
\definecolor{currentstroke}{rgb}{1.000000,1.000000,1.000000}%
\pgfsetstrokecolor{currentstroke}%
\pgfsetstrokeopacity{0.500000}%
\pgfsetdash{}{0pt}%
\pgfpathmoveto{\pgfqpoint{2.389846in}{1.165777in}}%
\pgfpathlineto{\pgfqpoint{2.401822in}{1.169545in}}%
\pgfpathlineto{\pgfqpoint{2.413793in}{1.173312in}}%
\pgfpathlineto{\pgfqpoint{2.425757in}{1.177077in}}%
\pgfpathlineto{\pgfqpoint{2.437715in}{1.180841in}}%
\pgfpathlineto{\pgfqpoint{2.449666in}{1.184604in}}%
\pgfpathlineto{\pgfqpoint{2.443369in}{1.196392in}}%
\pgfpathlineto{\pgfqpoint{2.437074in}{1.208180in}}%
\pgfpathlineto{\pgfqpoint{2.430783in}{1.219968in}}%
\pgfpathlineto{\pgfqpoint{2.424495in}{1.231756in}}%
\pgfpathlineto{\pgfqpoint{2.418211in}{1.243542in}}%
\pgfpathlineto{\pgfqpoint{2.406269in}{1.239709in}}%
\pgfpathlineto{\pgfqpoint{2.394322in}{1.235874in}}%
\pgfpathlineto{\pgfqpoint{2.382368in}{1.232038in}}%
\pgfpathlineto{\pgfqpoint{2.370408in}{1.228202in}}%
\pgfpathlineto{\pgfqpoint{2.358441in}{1.224364in}}%
\pgfpathlineto{\pgfqpoint{2.364716in}{1.212644in}}%
\pgfpathlineto{\pgfqpoint{2.370993in}{1.200925in}}%
\pgfpathlineto{\pgfqpoint{2.377274in}{1.189207in}}%
\pgfpathlineto{\pgfqpoint{2.383558in}{1.177491in}}%
\pgfpathclose%
\pgfusepath{stroke,fill}%
\end{pgfscope}%
\begin{pgfscope}%
\pgfpathrectangle{\pgfqpoint{0.887500in}{0.275000in}}{\pgfqpoint{4.225000in}{4.225000in}}%
\pgfusepath{clip}%
\pgfsetbuttcap%
\pgfsetroundjoin%
\definecolor{currentfill}{rgb}{0.280255,0.165693,0.476498}%
\pgfsetfillcolor{currentfill}%
\pgfsetfillopacity{0.700000}%
\pgfsetlinewidth{0.501875pt}%
\definecolor{currentstroke}{rgb}{1.000000,1.000000,1.000000}%
\pgfsetstrokecolor{currentstroke}%
\pgfsetstrokeopacity{0.500000}%
\pgfsetdash{}{0pt}%
\pgfpathmoveto{\pgfqpoint{2.481206in}{1.125675in}}%
\pgfpathlineto{\pgfqpoint{2.493162in}{1.129348in}}%
\pgfpathlineto{\pgfqpoint{2.505111in}{1.133021in}}%
\pgfpathlineto{\pgfqpoint{2.517054in}{1.136693in}}%
\pgfpathlineto{\pgfqpoint{2.528991in}{1.140365in}}%
\pgfpathlineto{\pgfqpoint{2.540922in}{1.144036in}}%
\pgfpathlineto{\pgfqpoint{2.534597in}{1.155922in}}%
\pgfpathlineto{\pgfqpoint{2.528276in}{1.167803in}}%
\pgfpathlineto{\pgfqpoint{2.521958in}{1.179677in}}%
\pgfpathlineto{\pgfqpoint{2.515644in}{1.191546in}}%
\pgfpathlineto{\pgfqpoint{2.509333in}{1.203411in}}%
\pgfpathlineto{\pgfqpoint{2.497412in}{1.199650in}}%
\pgfpathlineto{\pgfqpoint{2.485485in}{1.195888in}}%
\pgfpathlineto{\pgfqpoint{2.473551in}{1.192127in}}%
\pgfpathlineto{\pgfqpoint{2.461612in}{1.188366in}}%
\pgfpathlineto{\pgfqpoint{2.449666in}{1.184604in}}%
\pgfpathlineto{\pgfqpoint{2.455968in}{1.172816in}}%
\pgfpathlineto{\pgfqpoint{2.462272in}{1.161028in}}%
\pgfpathlineto{\pgfqpoint{2.468580in}{1.149242in}}%
\pgfpathlineto{\pgfqpoint{2.474891in}{1.137457in}}%
\pgfpathclose%
\pgfusepath{stroke,fill}%
\end{pgfscope}%
\begin{pgfscope}%
\pgfpathrectangle{\pgfqpoint{0.887500in}{0.275000in}}{\pgfqpoint{4.225000in}{4.225000in}}%
\pgfusepath{clip}%
\pgfsetbuttcap%
\pgfsetroundjoin%
\definecolor{currentfill}{rgb}{0.282623,0.140926,0.457517}%
\pgfsetfillcolor{currentfill}%
\pgfsetfillopacity{0.700000}%
\pgfsetlinewidth{0.501875pt}%
\definecolor{currentstroke}{rgb}{1.000000,1.000000,1.000000}%
\pgfsetstrokecolor{currentstroke}%
\pgfsetstrokeopacity{0.500000}%
\pgfsetdash{}{0pt}%
\pgfpathmoveto{\pgfqpoint{2.572593in}{1.084934in}}%
\pgfpathlineto{\pgfqpoint{2.584528in}{1.088487in}}%
\pgfpathlineto{\pgfqpoint{2.596456in}{1.092032in}}%
\pgfpathlineto{\pgfqpoint{2.608378in}{1.095568in}}%
\pgfpathlineto{\pgfqpoint{2.620294in}{1.099091in}}%
\pgfpathlineto{\pgfqpoint{2.632203in}{1.102600in}}%
\pgfpathlineto{\pgfqpoint{2.625854in}{1.114453in}}%
\pgfpathlineto{\pgfqpoint{2.619507in}{1.126373in}}%
\pgfpathlineto{\pgfqpoint{2.613162in}{1.138341in}}%
\pgfpathlineto{\pgfqpoint{2.606821in}{1.150340in}}%
\pgfpathlineto{\pgfqpoint{2.600483in}{1.162354in}}%
\pgfpathlineto{\pgfqpoint{2.588583in}{1.158699in}}%
\pgfpathlineto{\pgfqpoint{2.576677in}{1.155038in}}%
\pgfpathlineto{\pgfqpoint{2.564765in}{1.151374in}}%
\pgfpathlineto{\pgfqpoint{2.552847in}{1.147706in}}%
\pgfpathlineto{\pgfqpoint{2.540922in}{1.144036in}}%
\pgfpathlineto{\pgfqpoint{2.547250in}{1.132156in}}%
\pgfpathlineto{\pgfqpoint{2.553582in}{1.120293in}}%
\pgfpathlineto{\pgfqpoint{2.559916in}{1.108459in}}%
\pgfpathlineto{\pgfqpoint{2.566253in}{1.096669in}}%
\pgfpathclose%
\pgfusepath{stroke,fill}%
\end{pgfscope}%
\begin{pgfscope}%
\pgfpathrectangle{\pgfqpoint{0.887500in}{0.275000in}}{\pgfqpoint{4.225000in}{4.225000in}}%
\pgfusepath{clip}%
\pgfsetbuttcap%
\pgfsetroundjoin%
\definecolor{currentfill}{rgb}{0.283197,0.115680,0.436115}%
\pgfsetfillcolor{currentfill}%
\pgfsetfillopacity{0.700000}%
\pgfsetlinewidth{0.501875pt}%
\definecolor{currentstroke}{rgb}{1.000000,1.000000,1.000000}%
\pgfsetstrokecolor{currentstroke}%
\pgfsetstrokeopacity{0.500000}%
\pgfsetdash{}{0pt}%
\pgfpathmoveto{\pgfqpoint{2.663988in}{1.044933in}}%
\pgfpathlineto{\pgfqpoint{2.675900in}{1.048390in}}%
\pgfpathlineto{\pgfqpoint{2.687807in}{1.051837in}}%
\pgfpathlineto{\pgfqpoint{2.699707in}{1.055272in}}%
\pgfpathlineto{\pgfqpoint{2.711601in}{1.058694in}}%
\pgfpathlineto{\pgfqpoint{2.723489in}{1.062102in}}%
\pgfpathlineto{\pgfqpoint{2.717119in}{1.073279in}}%
\pgfpathlineto{\pgfqpoint{2.710751in}{1.084672in}}%
\pgfpathlineto{\pgfqpoint{2.704386in}{1.096257in}}%
\pgfpathlineto{\pgfqpoint{2.698022in}{1.108007in}}%
\pgfpathlineto{\pgfqpoint{2.691660in}{1.119898in}}%
\pgfpathlineto{\pgfqpoint{2.679781in}{1.116472in}}%
\pgfpathlineto{\pgfqpoint{2.667896in}{1.113029in}}%
\pgfpathlineto{\pgfqpoint{2.656005in}{1.109570in}}%
\pgfpathlineto{\pgfqpoint{2.644107in}{1.106094in}}%
\pgfpathlineto{\pgfqpoint{2.632203in}{1.102600in}}%
\pgfpathlineto{\pgfqpoint{2.638556in}{1.090831in}}%
\pgfpathlineto{\pgfqpoint{2.644911in}{1.079162in}}%
\pgfpathlineto{\pgfqpoint{2.651268in}{1.067611in}}%
\pgfpathlineto{\pgfqpoint{2.657627in}{1.056196in}}%
\pgfpathclose%
\pgfusepath{stroke,fill}%
\end{pgfscope}%
\begin{pgfscope}%
\pgfpathrectangle{\pgfqpoint{0.887500in}{0.275000in}}{\pgfqpoint{4.225000in}{4.225000in}}%
\pgfusepath{clip}%
\pgfsetbuttcap%
\pgfsetroundjoin%
\definecolor{currentfill}{rgb}{0.282327,0.094955,0.417331}%
\pgfsetfillcolor{currentfill}%
\pgfsetfillopacity{0.700000}%
\pgfsetlinewidth{0.501875pt}%
\definecolor{currentstroke}{rgb}{1.000000,1.000000,1.000000}%
\pgfsetstrokecolor{currentstroke}%
\pgfsetstrokeopacity{0.500000}%
\pgfsetdash{}{0pt}%
\pgfpathmoveto{\pgfqpoint{2.755359in}{1.010363in}}%
\pgfpathlineto{\pgfqpoint{2.767247in}{1.014223in}}%
\pgfpathlineto{\pgfqpoint{2.779129in}{1.018110in}}%
\pgfpathlineto{\pgfqpoint{2.791005in}{1.022021in}}%
\pgfpathlineto{\pgfqpoint{2.802874in}{1.025956in}}%
\pgfpathlineto{\pgfqpoint{2.814737in}{1.029918in}}%
\pgfpathlineto{\pgfqpoint{2.808354in}{1.038756in}}%
\pgfpathlineto{\pgfqpoint{2.801973in}{1.048111in}}%
\pgfpathlineto{\pgfqpoint{2.795592in}{1.057942in}}%
\pgfpathlineto{\pgfqpoint{2.789212in}{1.068211in}}%
\pgfpathlineto{\pgfqpoint{2.782833in}{1.078877in}}%
\pgfpathlineto{\pgfqpoint{2.770977in}{1.075554in}}%
\pgfpathlineto{\pgfqpoint{2.759114in}{1.072218in}}%
\pgfpathlineto{\pgfqpoint{2.747245in}{1.068865in}}%
\pgfpathlineto{\pgfqpoint{2.735370in}{1.065492in}}%
\pgfpathlineto{\pgfqpoint{2.723489in}{1.062102in}}%
\pgfpathlineto{\pgfqpoint{2.729860in}{1.051167in}}%
\pgfpathlineto{\pgfqpoint{2.736232in}{1.040500in}}%
\pgfpathlineto{\pgfqpoint{2.742606in}{1.030127in}}%
\pgfpathlineto{\pgfqpoint{2.748982in}{1.020073in}}%
\pgfpathclose%
\pgfusepath{stroke,fill}%
\end{pgfscope}%
\begin{pgfscope}%
\pgfpathrectangle{\pgfqpoint{0.887500in}{0.275000in}}{\pgfqpoint{4.225000in}{4.225000in}}%
\pgfusepath{clip}%
\pgfsetbuttcap%
\pgfsetroundjoin%
\definecolor{currentfill}{rgb}{0.281924,0.089666,0.412415}%
\pgfsetfillcolor{currentfill}%
\pgfsetfillopacity{0.700000}%
\pgfsetlinewidth{0.501875pt}%
\definecolor{currentstroke}{rgb}{1.000000,1.000000,1.000000}%
\pgfsetstrokecolor{currentstroke}%
\pgfsetstrokeopacity{0.500000}%
\pgfsetdash{}{0pt}%
\pgfpathmoveto{\pgfqpoint{2.846669in}{0.994860in}}%
\pgfpathlineto{\pgfqpoint{2.858533in}{1.000662in}}%
\pgfpathlineto{\pgfqpoint{2.870391in}{1.006578in}}%
\pgfpathlineto{\pgfqpoint{2.882244in}{1.012598in}}%
\pgfpathlineto{\pgfqpoint{2.894092in}{1.018712in}}%
\pgfpathlineto{\pgfqpoint{2.905935in}{1.024911in}}%
\pgfpathlineto{\pgfqpoint{2.899537in}{1.028016in}}%
\pgfpathlineto{\pgfqpoint{2.893140in}{1.032156in}}%
\pgfpathlineto{\pgfqpoint{2.886746in}{1.037272in}}%
\pgfpathlineto{\pgfqpoint{2.880353in}{1.043304in}}%
\pgfpathlineto{\pgfqpoint{2.873961in}{1.050195in}}%
\pgfpathlineto{\pgfqpoint{2.862129in}{1.046071in}}%
\pgfpathlineto{\pgfqpoint{2.850290in}{1.041983in}}%
\pgfpathlineto{\pgfqpoint{2.838445in}{1.037929in}}%
\pgfpathlineto{\pgfqpoint{2.826595in}{1.033908in}}%
\pgfpathlineto{\pgfqpoint{2.814737in}{1.029918in}}%
\pgfpathlineto{\pgfqpoint{2.821121in}{1.021636in}}%
\pgfpathlineto{\pgfqpoint{2.827507in}{1.013949in}}%
\pgfpathlineto{\pgfqpoint{2.833893in}{1.006897in}}%
\pgfpathlineto{\pgfqpoint{2.840280in}{1.000521in}}%
\pgfpathclose%
\pgfusepath{stroke,fill}%
\end{pgfscope}%
\begin{pgfscope}%
\pgfpathrectangle{\pgfqpoint{0.887500in}{0.275000in}}{\pgfqpoint{4.225000in}{4.225000in}}%
\pgfusepath{clip}%
\pgfsetbuttcap%
\pgfsetroundjoin%
\definecolor{currentfill}{rgb}{0.276194,0.190074,0.493001}%
\pgfsetfillcolor{currentfill}%
\pgfsetfillopacity{0.700000}%
\pgfsetlinewidth{0.501875pt}%
\definecolor{currentstroke}{rgb}{1.000000,1.000000,1.000000}%
\pgfsetstrokecolor{currentstroke}%
\pgfsetstrokeopacity{0.500000}%
\pgfsetdash{}{0pt}%
\pgfpathmoveto{\pgfqpoint{2.329871in}{1.146903in}}%
\pgfpathlineto{\pgfqpoint{2.341879in}{1.150683in}}%
\pgfpathlineto{\pgfqpoint{2.353880in}{1.154460in}}%
\pgfpathlineto{\pgfqpoint{2.365875in}{1.158235in}}%
\pgfpathlineto{\pgfqpoint{2.377863in}{1.162007in}}%
\pgfpathlineto{\pgfqpoint{2.389846in}{1.165777in}}%
\pgfpathlineto{\pgfqpoint{2.383558in}{1.177491in}}%
\pgfpathlineto{\pgfqpoint{2.377274in}{1.189207in}}%
\pgfpathlineto{\pgfqpoint{2.370993in}{1.200925in}}%
\pgfpathlineto{\pgfqpoint{2.364716in}{1.212644in}}%
\pgfpathlineto{\pgfqpoint{2.358441in}{1.224364in}}%
\pgfpathlineto{\pgfqpoint{2.346469in}{1.220524in}}%
\pgfpathlineto{\pgfqpoint{2.334491in}{1.216684in}}%
\pgfpathlineto{\pgfqpoint{2.322506in}{1.212842in}}%
\pgfpathlineto{\pgfqpoint{2.310515in}{1.208999in}}%
\pgfpathlineto{\pgfqpoint{2.298519in}{1.205154in}}%
\pgfpathlineto{\pgfqpoint{2.304782in}{1.193499in}}%
\pgfpathlineto{\pgfqpoint{2.311050in}{1.181845in}}%
\pgfpathlineto{\pgfqpoint{2.317320in}{1.170194in}}%
\pgfpathlineto{\pgfqpoint{2.323594in}{1.158546in}}%
\pgfpathclose%
\pgfusepath{stroke,fill}%
\end{pgfscope}%
\begin{pgfscope}%
\pgfpathrectangle{\pgfqpoint{0.887500in}{0.275000in}}{\pgfqpoint{4.225000in}{4.225000in}}%
\pgfusepath{clip}%
\pgfsetbuttcap%
\pgfsetroundjoin%
\definecolor{currentfill}{rgb}{0.280255,0.165693,0.476498}%
\pgfsetfillcolor{currentfill}%
\pgfsetfillopacity{0.700000}%
\pgfsetlinewidth{0.501875pt}%
\definecolor{currentstroke}{rgb}{1.000000,1.000000,1.000000}%
\pgfsetstrokecolor{currentstroke}%
\pgfsetstrokeopacity{0.500000}%
\pgfsetdash{}{0pt}%
\pgfpathmoveto{\pgfqpoint{2.421334in}{1.107275in}}%
\pgfpathlineto{\pgfqpoint{2.433321in}{1.110960in}}%
\pgfpathlineto{\pgfqpoint{2.445302in}{1.114642in}}%
\pgfpathlineto{\pgfqpoint{2.457276in}{1.118322in}}%
\pgfpathlineto{\pgfqpoint{2.469244in}{1.121999in}}%
\pgfpathlineto{\pgfqpoint{2.481206in}{1.125675in}}%
\pgfpathlineto{\pgfqpoint{2.474891in}{1.137457in}}%
\pgfpathlineto{\pgfqpoint{2.468580in}{1.149242in}}%
\pgfpathlineto{\pgfqpoint{2.462272in}{1.161028in}}%
\pgfpathlineto{\pgfqpoint{2.455968in}{1.172816in}}%
\pgfpathlineto{\pgfqpoint{2.449666in}{1.184604in}}%
\pgfpathlineto{\pgfqpoint{2.437715in}{1.180841in}}%
\pgfpathlineto{\pgfqpoint{2.425757in}{1.177077in}}%
\pgfpathlineto{\pgfqpoint{2.413793in}{1.173312in}}%
\pgfpathlineto{\pgfqpoint{2.401822in}{1.169545in}}%
\pgfpathlineto{\pgfqpoint{2.389846in}{1.165777in}}%
\pgfpathlineto{\pgfqpoint{2.396137in}{1.154067in}}%
\pgfpathlineto{\pgfqpoint{2.402431in}{1.142360in}}%
\pgfpathlineto{\pgfqpoint{2.408729in}{1.130659in}}%
\pgfpathlineto{\pgfqpoint{2.415030in}{1.118963in}}%
\pgfpathclose%
\pgfusepath{stroke,fill}%
\end{pgfscope}%
\begin{pgfscope}%
\pgfpathrectangle{\pgfqpoint{0.887500in}{0.275000in}}{\pgfqpoint{4.225000in}{4.225000in}}%
\pgfusepath{clip}%
\pgfsetbuttcap%
\pgfsetroundjoin%
\definecolor{currentfill}{rgb}{0.282623,0.140926,0.457517}%
\pgfsetfillcolor{currentfill}%
\pgfsetfillopacity{0.700000}%
\pgfsetlinewidth{0.501875pt}%
\definecolor{currentstroke}{rgb}{1.000000,1.000000,1.000000}%
\pgfsetstrokecolor{currentstroke}%
\pgfsetstrokeopacity{0.500000}%
\pgfsetdash{}{0pt}%
\pgfpathmoveto{\pgfqpoint{2.512826in}{1.067101in}}%
\pgfpathlineto{\pgfqpoint{2.524792in}{1.070675in}}%
\pgfpathlineto{\pgfqpoint{2.536752in}{1.074245in}}%
\pgfpathlineto{\pgfqpoint{2.548705in}{1.077812in}}%
\pgfpathlineto{\pgfqpoint{2.560652in}{1.081375in}}%
\pgfpathlineto{\pgfqpoint{2.572593in}{1.084934in}}%
\pgfpathlineto{\pgfqpoint{2.566253in}{1.096669in}}%
\pgfpathlineto{\pgfqpoint{2.559916in}{1.108459in}}%
\pgfpathlineto{\pgfqpoint{2.553582in}{1.120293in}}%
\pgfpathlineto{\pgfqpoint{2.547250in}{1.132156in}}%
\pgfpathlineto{\pgfqpoint{2.540922in}{1.144036in}}%
\pgfpathlineto{\pgfqpoint{2.528991in}{1.140365in}}%
\pgfpathlineto{\pgfqpoint{2.517054in}{1.136693in}}%
\pgfpathlineto{\pgfqpoint{2.505111in}{1.133021in}}%
\pgfpathlineto{\pgfqpoint{2.493162in}{1.129348in}}%
\pgfpathlineto{\pgfqpoint{2.481206in}{1.125675in}}%
\pgfpathlineto{\pgfqpoint{2.487524in}{1.113903in}}%
\pgfpathlineto{\pgfqpoint{2.493845in}{1.102150in}}%
\pgfpathlineto{\pgfqpoint{2.500169in}{1.090426in}}%
\pgfpathlineto{\pgfqpoint{2.506497in}{1.078740in}}%
\pgfpathclose%
\pgfusepath{stroke,fill}%
\end{pgfscope}%
\begin{pgfscope}%
\pgfpathrectangle{\pgfqpoint{0.887500in}{0.275000in}}{\pgfqpoint{4.225000in}{4.225000in}}%
\pgfusepath{clip}%
\pgfsetbuttcap%
\pgfsetroundjoin%
\definecolor{currentfill}{rgb}{0.283229,0.120777,0.440584}%
\pgfsetfillcolor{currentfill}%
\pgfsetfillopacity{0.700000}%
\pgfsetlinewidth{0.501875pt}%
\definecolor{currentstroke}{rgb}{1.000000,1.000000,1.000000}%
\pgfsetstrokecolor{currentstroke}%
\pgfsetstrokeopacity{0.500000}%
\pgfsetdash{}{0pt}%
\pgfpathmoveto{\pgfqpoint{2.604330in}{1.027521in}}%
\pgfpathlineto{\pgfqpoint{2.616274in}{1.031019in}}%
\pgfpathlineto{\pgfqpoint{2.628212in}{1.034510in}}%
\pgfpathlineto{\pgfqpoint{2.640143in}{1.037992in}}%
\pgfpathlineto{\pgfqpoint{2.652069in}{1.041467in}}%
\pgfpathlineto{\pgfqpoint{2.663988in}{1.044933in}}%
\pgfpathlineto{\pgfqpoint{2.657627in}{1.056196in}}%
\pgfpathlineto{\pgfqpoint{2.651268in}{1.067611in}}%
\pgfpathlineto{\pgfqpoint{2.644911in}{1.079162in}}%
\pgfpathlineto{\pgfqpoint{2.638556in}{1.090831in}}%
\pgfpathlineto{\pgfqpoint{2.632203in}{1.102600in}}%
\pgfpathlineto{\pgfqpoint{2.620294in}{1.099091in}}%
\pgfpathlineto{\pgfqpoint{2.608378in}{1.095568in}}%
\pgfpathlineto{\pgfqpoint{2.596456in}{1.092032in}}%
\pgfpathlineto{\pgfqpoint{2.584528in}{1.088487in}}%
\pgfpathlineto{\pgfqpoint{2.572593in}{1.084934in}}%
\pgfpathlineto{\pgfqpoint{2.578936in}{1.073266in}}%
\pgfpathlineto{\pgfqpoint{2.585281in}{1.061678in}}%
\pgfpathlineto{\pgfqpoint{2.591628in}{1.050183in}}%
\pgfpathlineto{\pgfqpoint{2.597978in}{1.038793in}}%
\pgfpathclose%
\pgfusepath{stroke,fill}%
\end{pgfscope}%
\begin{pgfscope}%
\pgfpathrectangle{\pgfqpoint{0.887500in}{0.275000in}}{\pgfqpoint{4.225000in}{4.225000in}}%
\pgfusepath{clip}%
\pgfsetbuttcap%
\pgfsetroundjoin%
\definecolor{currentfill}{rgb}{0.282327,0.094955,0.417331}%
\pgfsetfillcolor{currentfill}%
\pgfsetfillopacity{0.700000}%
\pgfsetlinewidth{0.501875pt}%
\definecolor{currentstroke}{rgb}{1.000000,1.000000,1.000000}%
\pgfsetstrokecolor{currentstroke}%
\pgfsetstrokeopacity{0.500000}%
\pgfsetdash{}{0pt}%
\pgfpathmoveto{\pgfqpoint{2.695821in}{0.991515in}}%
\pgfpathlineto{\pgfqpoint{2.707742in}{0.995218in}}%
\pgfpathlineto{\pgfqpoint{2.719656in}{0.998956in}}%
\pgfpathlineto{\pgfqpoint{2.731563in}{1.002727in}}%
\pgfpathlineto{\pgfqpoint{2.743464in}{1.006530in}}%
\pgfpathlineto{\pgfqpoint{2.755359in}{1.010363in}}%
\pgfpathlineto{\pgfqpoint{2.748982in}{1.020073in}}%
\pgfpathlineto{\pgfqpoint{2.742606in}{1.030127in}}%
\pgfpathlineto{\pgfqpoint{2.736232in}{1.040500in}}%
\pgfpathlineto{\pgfqpoint{2.729860in}{1.051167in}}%
\pgfpathlineto{\pgfqpoint{2.723489in}{1.062102in}}%
\pgfpathlineto{\pgfqpoint{2.711601in}{1.058694in}}%
\pgfpathlineto{\pgfqpoint{2.699707in}{1.055272in}}%
\pgfpathlineto{\pgfqpoint{2.687807in}{1.051837in}}%
\pgfpathlineto{\pgfqpoint{2.675900in}{1.048390in}}%
\pgfpathlineto{\pgfqpoint{2.663988in}{1.044933in}}%
\pgfpathlineto{\pgfqpoint{2.670351in}{1.033840in}}%
\pgfpathlineto{\pgfqpoint{2.676716in}{1.022935in}}%
\pgfpathlineto{\pgfqpoint{2.683082in}{1.012234in}}%
\pgfpathlineto{\pgfqpoint{2.689451in}{1.001755in}}%
\pgfpathclose%
\pgfusepath{stroke,fill}%
\end{pgfscope}%
\begin{pgfscope}%
\pgfpathrectangle{\pgfqpoint{0.887500in}{0.275000in}}{\pgfqpoint{4.225000in}{4.225000in}}%
\pgfusepath{clip}%
\pgfsetbuttcap%
\pgfsetroundjoin%
\definecolor{currentfill}{rgb}{0.281446,0.084320,0.407414}%
\pgfsetfillcolor{currentfill}%
\pgfsetfillopacity{0.700000}%
\pgfsetlinewidth{0.501875pt}%
\definecolor{currentstroke}{rgb}{1.000000,1.000000,1.000000}%
\pgfsetstrokecolor{currentstroke}%
\pgfsetstrokeopacity{0.500000}%
\pgfsetdash{}{0pt}%
\pgfpathmoveto{\pgfqpoint{2.787263in}{0.967884in}}%
\pgfpathlineto{\pgfqpoint{2.799157in}{0.972988in}}%
\pgfpathlineto{\pgfqpoint{2.811044in}{0.978239in}}%
\pgfpathlineto{\pgfqpoint{2.822925in}{0.983638in}}%
\pgfpathlineto{\pgfqpoint{2.834800in}{0.989182in}}%
\pgfpathlineto{\pgfqpoint{2.846669in}{0.994860in}}%
\pgfpathlineto{\pgfqpoint{2.840280in}{1.000521in}}%
\pgfpathlineto{\pgfqpoint{2.833893in}{1.006897in}}%
\pgfpathlineto{\pgfqpoint{2.827507in}{1.013949in}}%
\pgfpathlineto{\pgfqpoint{2.821121in}{1.021636in}}%
\pgfpathlineto{\pgfqpoint{2.814737in}{1.029918in}}%
\pgfpathlineto{\pgfqpoint{2.802874in}{1.025956in}}%
\pgfpathlineto{\pgfqpoint{2.791005in}{1.022021in}}%
\pgfpathlineto{\pgfqpoint{2.779129in}{1.018110in}}%
\pgfpathlineto{\pgfqpoint{2.767247in}{1.014223in}}%
\pgfpathlineto{\pgfqpoint{2.755359in}{1.010363in}}%
\pgfpathlineto{\pgfqpoint{2.761737in}{1.001024in}}%
\pgfpathlineto{\pgfqpoint{2.768116in}{0.992080in}}%
\pgfpathlineto{\pgfqpoint{2.774497in}{0.983559in}}%
\pgfpathlineto{\pgfqpoint{2.780879in}{0.975485in}}%
\pgfpathclose%
\pgfusepath{stroke,fill}%
\end{pgfscope}%
\begin{pgfscope}%
\pgfpathrectangle{\pgfqpoint{0.887500in}{0.275000in}}{\pgfqpoint{4.225000in}{4.225000in}}%
\pgfusepath{clip}%
\pgfsetbuttcap%
\pgfsetroundjoin%
\definecolor{currentfill}{rgb}{0.280255,0.165693,0.476498}%
\pgfsetfillcolor{currentfill}%
\pgfsetfillopacity{0.700000}%
\pgfsetlinewidth{0.501875pt}%
\definecolor{currentstroke}{rgb}{1.000000,1.000000,1.000000}%
\pgfsetstrokecolor{currentstroke}%
\pgfsetstrokeopacity{0.500000}%
\pgfsetdash{}{0pt}%
\pgfpathmoveto{\pgfqpoint{2.361307in}{1.088789in}}%
\pgfpathlineto{\pgfqpoint{2.373325in}{1.092496in}}%
\pgfpathlineto{\pgfqpoint{2.385337in}{1.096197in}}%
\pgfpathlineto{\pgfqpoint{2.397342in}{1.099894in}}%
\pgfpathlineto{\pgfqpoint{2.409341in}{1.103586in}}%
\pgfpathlineto{\pgfqpoint{2.421334in}{1.107275in}}%
\pgfpathlineto{\pgfqpoint{2.415030in}{1.118963in}}%
\pgfpathlineto{\pgfqpoint{2.408729in}{1.130659in}}%
\pgfpathlineto{\pgfqpoint{2.402431in}{1.142360in}}%
\pgfpathlineto{\pgfqpoint{2.396137in}{1.154067in}}%
\pgfpathlineto{\pgfqpoint{2.389846in}{1.165777in}}%
\pgfpathlineto{\pgfqpoint{2.377863in}{1.162007in}}%
\pgfpathlineto{\pgfqpoint{2.365875in}{1.158235in}}%
\pgfpathlineto{\pgfqpoint{2.353880in}{1.154460in}}%
\pgfpathlineto{\pgfqpoint{2.341879in}{1.150683in}}%
\pgfpathlineto{\pgfqpoint{2.329871in}{1.146903in}}%
\pgfpathlineto{\pgfqpoint{2.336152in}{1.135266in}}%
\pgfpathlineto{\pgfqpoint{2.342436in}{1.123634in}}%
\pgfpathlineto{\pgfqpoint{2.348723in}{1.112011in}}%
\pgfpathlineto{\pgfqpoint{2.355013in}{1.100395in}}%
\pgfpathclose%
\pgfusepath{stroke,fill}%
\end{pgfscope}%
\begin{pgfscope}%
\pgfpathrectangle{\pgfqpoint{0.887500in}{0.275000in}}{\pgfqpoint{4.225000in}{4.225000in}}%
\pgfusepath{clip}%
\pgfsetbuttcap%
\pgfsetroundjoin%
\definecolor{currentfill}{rgb}{0.282290,0.145912,0.461510}%
\pgfsetfillcolor{currentfill}%
\pgfsetfillopacity{0.700000}%
\pgfsetlinewidth{0.501875pt}%
\definecolor{currentstroke}{rgb}{1.000000,1.000000,1.000000}%
\pgfsetstrokecolor{currentstroke}%
\pgfsetstrokeopacity{0.500000}%
\pgfsetdash{}{0pt}%
\pgfpathmoveto{\pgfqpoint{2.452903in}{1.049160in}}%
\pgfpathlineto{\pgfqpoint{2.464900in}{1.052759in}}%
\pgfpathlineto{\pgfqpoint{2.476891in}{1.056352in}}%
\pgfpathlineto{\pgfqpoint{2.488876in}{1.059940in}}%
\pgfpathlineto{\pgfqpoint{2.500854in}{1.063523in}}%
\pgfpathlineto{\pgfqpoint{2.512826in}{1.067101in}}%
\pgfpathlineto{\pgfqpoint{2.506497in}{1.078740in}}%
\pgfpathlineto{\pgfqpoint{2.500169in}{1.090426in}}%
\pgfpathlineto{\pgfqpoint{2.493845in}{1.102150in}}%
\pgfpathlineto{\pgfqpoint{2.487524in}{1.113903in}}%
\pgfpathlineto{\pgfqpoint{2.481206in}{1.125675in}}%
\pgfpathlineto{\pgfqpoint{2.469244in}{1.121999in}}%
\pgfpathlineto{\pgfqpoint{2.457276in}{1.118322in}}%
\pgfpathlineto{\pgfqpoint{2.445302in}{1.114642in}}%
\pgfpathlineto{\pgfqpoint{2.433321in}{1.110960in}}%
\pgfpathlineto{\pgfqpoint{2.421334in}{1.107275in}}%
\pgfpathlineto{\pgfqpoint{2.427642in}{1.095600in}}%
\pgfpathlineto{\pgfqpoint{2.433953in}{1.083944in}}%
\pgfpathlineto{\pgfqpoint{2.440266in}{1.072314in}}%
\pgfpathlineto{\pgfqpoint{2.446583in}{1.060717in}}%
\pgfpathclose%
\pgfusepath{stroke,fill}%
\end{pgfscope}%
\begin{pgfscope}%
\pgfpathrectangle{\pgfqpoint{0.887500in}{0.275000in}}{\pgfqpoint{4.225000in}{4.225000in}}%
\pgfusepath{clip}%
\pgfsetbuttcap%
\pgfsetroundjoin%
\definecolor{currentfill}{rgb}{0.283229,0.120777,0.440584}%
\pgfsetfillcolor{currentfill}%
\pgfsetfillopacity{0.700000}%
\pgfsetlinewidth{0.501875pt}%
\definecolor{currentstroke}{rgb}{1.000000,1.000000,1.000000}%
\pgfsetstrokecolor{currentstroke}%
\pgfsetstrokeopacity{0.500000}%
\pgfsetdash{}{0pt}%
\pgfpathmoveto{\pgfqpoint{2.544515in}{1.009934in}}%
\pgfpathlineto{\pgfqpoint{2.556490in}{1.013464in}}%
\pgfpathlineto{\pgfqpoint{2.568460in}{1.016988in}}%
\pgfpathlineto{\pgfqpoint{2.580423in}{1.020505in}}%
\pgfpathlineto{\pgfqpoint{2.592379in}{1.024016in}}%
\pgfpathlineto{\pgfqpoint{2.604330in}{1.027521in}}%
\pgfpathlineto{\pgfqpoint{2.597978in}{1.038793in}}%
\pgfpathlineto{\pgfqpoint{2.591628in}{1.050183in}}%
\pgfpathlineto{\pgfqpoint{2.585281in}{1.061678in}}%
\pgfpathlineto{\pgfqpoint{2.578936in}{1.073266in}}%
\pgfpathlineto{\pgfqpoint{2.572593in}{1.084934in}}%
\pgfpathlineto{\pgfqpoint{2.560652in}{1.081375in}}%
\pgfpathlineto{\pgfqpoint{2.548705in}{1.077812in}}%
\pgfpathlineto{\pgfqpoint{2.536752in}{1.074245in}}%
\pgfpathlineto{\pgfqpoint{2.524792in}{1.070675in}}%
\pgfpathlineto{\pgfqpoint{2.512826in}{1.067101in}}%
\pgfpathlineto{\pgfqpoint{2.519159in}{1.055519in}}%
\pgfpathlineto{\pgfqpoint{2.525494in}{1.044001in}}%
\pgfpathlineto{\pgfqpoint{2.531832in}{1.032559in}}%
\pgfpathlineto{\pgfqpoint{2.538172in}{1.021200in}}%
\pgfpathclose%
\pgfusepath{stroke,fill}%
\end{pgfscope}%
\begin{pgfscope}%
\pgfpathrectangle{\pgfqpoint{0.887500in}{0.275000in}}{\pgfqpoint{4.225000in}{4.225000in}}%
\pgfusepath{clip}%
\pgfsetbuttcap%
\pgfsetroundjoin%
\definecolor{currentfill}{rgb}{0.282327,0.094955,0.417331}%
\pgfsetfillcolor{currentfill}%
\pgfsetfillopacity{0.700000}%
\pgfsetlinewidth{0.501875pt}%
\definecolor{currentstroke}{rgb}{1.000000,1.000000,1.000000}%
\pgfsetstrokecolor{currentstroke}%
\pgfsetstrokeopacity{0.500000}%
\pgfsetdash{}{0pt}%
\pgfpathmoveto{\pgfqpoint{2.636120in}{0.973364in}}%
\pgfpathlineto{\pgfqpoint{2.648074in}{0.976966in}}%
\pgfpathlineto{\pgfqpoint{2.660020in}{0.980576in}}%
\pgfpathlineto{\pgfqpoint{2.671960in}{0.984200in}}%
\pgfpathlineto{\pgfqpoint{2.683894in}{0.987844in}}%
\pgfpathlineto{\pgfqpoint{2.695821in}{0.991515in}}%
\pgfpathlineto{\pgfqpoint{2.689451in}{1.001755in}}%
\pgfpathlineto{\pgfqpoint{2.683082in}{1.012234in}}%
\pgfpathlineto{\pgfqpoint{2.676716in}{1.022935in}}%
\pgfpathlineto{\pgfqpoint{2.670351in}{1.033840in}}%
\pgfpathlineto{\pgfqpoint{2.663988in}{1.044933in}}%
\pgfpathlineto{\pgfqpoint{2.652069in}{1.041467in}}%
\pgfpathlineto{\pgfqpoint{2.640143in}{1.037992in}}%
\pgfpathlineto{\pgfqpoint{2.628212in}{1.034510in}}%
\pgfpathlineto{\pgfqpoint{2.616274in}{1.031019in}}%
\pgfpathlineto{\pgfqpoint{2.604330in}{1.027521in}}%
\pgfpathlineto{\pgfqpoint{2.610684in}{1.016379in}}%
\pgfpathlineto{\pgfqpoint{2.617040in}{1.005380in}}%
\pgfpathlineto{\pgfqpoint{2.623398in}{0.994536in}}%
\pgfpathlineto{\pgfqpoint{2.629759in}{0.983860in}}%
\pgfpathclose%
\pgfusepath{stroke,fill}%
\end{pgfscope}%
\begin{pgfscope}%
\pgfpathrectangle{\pgfqpoint{0.887500in}{0.275000in}}{\pgfqpoint{4.225000in}{4.225000in}}%
\pgfusepath{clip}%
\pgfsetbuttcap%
\pgfsetroundjoin%
\definecolor{currentfill}{rgb}{0.280894,0.078907,0.402329}%
\pgfsetfillcolor{currentfill}%
\pgfsetfillopacity{0.700000}%
\pgfsetlinewidth{0.501875pt}%
\definecolor{currentstroke}{rgb}{1.000000,1.000000,1.000000}%
\pgfsetstrokecolor{currentstroke}%
\pgfsetstrokeopacity{0.500000}%
\pgfsetdash{}{0pt}%
\pgfpathmoveto{\pgfqpoint{2.727696in}{0.944511in}}%
\pgfpathlineto{\pgfqpoint{2.739623in}{0.948906in}}%
\pgfpathlineto{\pgfqpoint{2.751543in}{0.953439in}}%
\pgfpathlineto{\pgfqpoint{2.763456in}{0.958112in}}%
\pgfpathlineto{\pgfqpoint{2.775363in}{0.962926in}}%
\pgfpathlineto{\pgfqpoint{2.787263in}{0.967884in}}%
\pgfpathlineto{\pgfqpoint{2.780879in}{0.975485in}}%
\pgfpathlineto{\pgfqpoint{2.774497in}{0.983559in}}%
\pgfpathlineto{\pgfqpoint{2.768116in}{0.992080in}}%
\pgfpathlineto{\pgfqpoint{2.761737in}{1.001024in}}%
\pgfpathlineto{\pgfqpoint{2.755359in}{1.010363in}}%
\pgfpathlineto{\pgfqpoint{2.743464in}{1.006530in}}%
\pgfpathlineto{\pgfqpoint{2.731563in}{1.002727in}}%
\pgfpathlineto{\pgfqpoint{2.719656in}{0.998956in}}%
\pgfpathlineto{\pgfqpoint{2.707742in}{0.995218in}}%
\pgfpathlineto{\pgfqpoint{2.695821in}{0.991515in}}%
\pgfpathlineto{\pgfqpoint{2.702193in}{0.981532in}}%
\pgfpathlineto{\pgfqpoint{2.708566in}{0.971822in}}%
\pgfpathlineto{\pgfqpoint{2.714941in}{0.962404in}}%
\pgfpathlineto{\pgfqpoint{2.721318in}{0.953295in}}%
\pgfpathclose%
\pgfusepath{stroke,fill}%
\end{pgfscope}%
\begin{pgfscope}%
\pgfpathrectangle{\pgfqpoint{0.887500in}{0.275000in}}{\pgfqpoint{4.225000in}{4.225000in}}%
\pgfusepath{clip}%
\pgfsetbuttcap%
\pgfsetroundjoin%
\definecolor{currentfill}{rgb}{0.282290,0.145912,0.461510}%
\pgfsetfillcolor{currentfill}%
\pgfsetfillopacity{0.700000}%
\pgfsetlinewidth{0.501875pt}%
\definecolor{currentstroke}{rgb}{1.000000,1.000000,1.000000}%
\pgfsetstrokecolor{currentstroke}%
\pgfsetstrokeopacity{0.500000}%
\pgfsetdash{}{0pt}%
\pgfpathmoveto{\pgfqpoint{2.392822in}{1.031069in}}%
\pgfpathlineto{\pgfqpoint{2.404851in}{1.034701in}}%
\pgfpathlineto{\pgfqpoint{2.416873in}{1.038326in}}%
\pgfpathlineto{\pgfqpoint{2.428889in}{1.041943in}}%
\pgfpathlineto{\pgfqpoint{2.440899in}{1.045555in}}%
\pgfpathlineto{\pgfqpoint{2.452903in}{1.049160in}}%
\pgfpathlineto{\pgfqpoint{2.446583in}{1.060717in}}%
\pgfpathlineto{\pgfqpoint{2.440266in}{1.072314in}}%
\pgfpathlineto{\pgfqpoint{2.433953in}{1.083944in}}%
\pgfpathlineto{\pgfqpoint{2.427642in}{1.095600in}}%
\pgfpathlineto{\pgfqpoint{2.421334in}{1.107275in}}%
\pgfpathlineto{\pgfqpoint{2.409341in}{1.103586in}}%
\pgfpathlineto{\pgfqpoint{2.397342in}{1.099894in}}%
\pgfpathlineto{\pgfqpoint{2.385337in}{1.096197in}}%
\pgfpathlineto{\pgfqpoint{2.373325in}{1.092496in}}%
\pgfpathlineto{\pgfqpoint{2.361307in}{1.088789in}}%
\pgfpathlineto{\pgfqpoint{2.367604in}{1.077198in}}%
\pgfpathlineto{\pgfqpoint{2.373904in}{1.065625in}}%
\pgfpathlineto{\pgfqpoint{2.380207in}{1.054075in}}%
\pgfpathlineto{\pgfqpoint{2.386513in}{1.042555in}}%
\pgfpathclose%
\pgfusepath{stroke,fill}%
\end{pgfscope}%
\begin{pgfscope}%
\pgfpathrectangle{\pgfqpoint{0.887500in}{0.275000in}}{\pgfqpoint{4.225000in}{4.225000in}}%
\pgfusepath{clip}%
\pgfsetbuttcap%
\pgfsetroundjoin%
\definecolor{currentfill}{rgb}{0.283229,0.120777,0.440584}%
\pgfsetfillcolor{currentfill}%
\pgfsetfillopacity{0.700000}%
\pgfsetlinewidth{0.501875pt}%
\definecolor{currentstroke}{rgb}{1.000000,1.000000,1.000000}%
\pgfsetstrokecolor{currentstroke}%
\pgfsetstrokeopacity{0.500000}%
\pgfsetdash{}{0pt}%
\pgfpathmoveto{\pgfqpoint{2.484541in}{0.992196in}}%
\pgfpathlineto{\pgfqpoint{2.496549in}{0.995756in}}%
\pgfpathlineto{\pgfqpoint{2.508550in}{0.999309in}}%
\pgfpathlineto{\pgfqpoint{2.520544in}{1.002857in}}%
\pgfpathlineto{\pgfqpoint{2.532533in}{1.006398in}}%
\pgfpathlineto{\pgfqpoint{2.544515in}{1.009934in}}%
\pgfpathlineto{\pgfqpoint{2.538172in}{1.021200in}}%
\pgfpathlineto{\pgfqpoint{2.531832in}{1.032559in}}%
\pgfpathlineto{\pgfqpoint{2.525494in}{1.044001in}}%
\pgfpathlineto{\pgfqpoint{2.519159in}{1.055519in}}%
\pgfpathlineto{\pgfqpoint{2.512826in}{1.067101in}}%
\pgfpathlineto{\pgfqpoint{2.500854in}{1.063523in}}%
\pgfpathlineto{\pgfqpoint{2.488876in}{1.059940in}}%
\pgfpathlineto{\pgfqpoint{2.476891in}{1.056352in}}%
\pgfpathlineto{\pgfqpoint{2.464900in}{1.052759in}}%
\pgfpathlineto{\pgfqpoint{2.452903in}{1.049160in}}%
\pgfpathlineto{\pgfqpoint{2.459225in}{1.037648in}}%
\pgfpathlineto{\pgfqpoint{2.465550in}{1.026190in}}%
\pgfpathlineto{\pgfqpoint{2.471878in}{1.014790in}}%
\pgfpathlineto{\pgfqpoint{2.478208in}{1.003457in}}%
\pgfpathclose%
\pgfusepath{stroke,fill}%
\end{pgfscope}%
\begin{pgfscope}%
\pgfpathrectangle{\pgfqpoint{0.887500in}{0.275000in}}{\pgfqpoint{4.225000in}{4.225000in}}%
\pgfusepath{clip}%
\pgfsetbuttcap%
\pgfsetroundjoin%
\definecolor{currentfill}{rgb}{0.282656,0.100196,0.422160}%
\pgfsetfillcolor{currentfill}%
\pgfsetfillopacity{0.700000}%
\pgfsetlinewidth{0.501875pt}%
\definecolor{currentstroke}{rgb}{1.000000,1.000000,1.000000}%
\pgfsetstrokecolor{currentstroke}%
\pgfsetstrokeopacity{0.500000}%
\pgfsetdash{}{0pt}%
\pgfpathmoveto{\pgfqpoint{2.576260in}{0.955322in}}%
\pgfpathlineto{\pgfqpoint{2.588245in}{0.958938in}}%
\pgfpathlineto{\pgfqpoint{2.600223in}{0.962552in}}%
\pgfpathlineto{\pgfqpoint{2.612195in}{0.966160in}}%
\pgfpathlineto{\pgfqpoint{2.624161in}{0.969764in}}%
\pgfpathlineto{\pgfqpoint{2.636120in}{0.973364in}}%
\pgfpathlineto{\pgfqpoint{2.629759in}{0.983860in}}%
\pgfpathlineto{\pgfqpoint{2.623398in}{0.994536in}}%
\pgfpathlineto{\pgfqpoint{2.617040in}{1.005380in}}%
\pgfpathlineto{\pgfqpoint{2.610684in}{1.016379in}}%
\pgfpathlineto{\pgfqpoint{2.604330in}{1.027521in}}%
\pgfpathlineto{\pgfqpoint{2.592379in}{1.024016in}}%
\pgfpathlineto{\pgfqpoint{2.580423in}{1.020505in}}%
\pgfpathlineto{\pgfqpoint{2.568460in}{1.016988in}}%
\pgfpathlineto{\pgfqpoint{2.556490in}{1.013464in}}%
\pgfpathlineto{\pgfqpoint{2.544515in}{1.009934in}}%
\pgfpathlineto{\pgfqpoint{2.550859in}{0.998770in}}%
\pgfpathlineto{\pgfqpoint{2.557206in}{0.987718in}}%
\pgfpathlineto{\pgfqpoint{2.563556in}{0.976786in}}%
\pgfpathlineto{\pgfqpoint{2.569907in}{0.965985in}}%
\pgfpathclose%
\pgfusepath{stroke,fill}%
\end{pgfscope}%
\begin{pgfscope}%
\pgfpathrectangle{\pgfqpoint{0.887500in}{0.275000in}}{\pgfqpoint{4.225000in}{4.225000in}}%
\pgfusepath{clip}%
\pgfsetbuttcap%
\pgfsetroundjoin%
\definecolor{currentfill}{rgb}{0.280894,0.078907,0.402329}%
\pgfsetfillcolor{currentfill}%
\pgfsetfillopacity{0.700000}%
\pgfsetlinewidth{0.501875pt}%
\definecolor{currentstroke}{rgb}{1.000000,1.000000,1.000000}%
\pgfsetstrokecolor{currentstroke}%
\pgfsetstrokeopacity{0.500000}%
\pgfsetdash{}{0pt}%
\pgfpathmoveto{\pgfqpoint{2.667957in}{0.924030in}}%
\pgfpathlineto{\pgfqpoint{2.679919in}{0.927993in}}%
\pgfpathlineto{\pgfqpoint{2.691873in}{0.932003in}}%
\pgfpathlineto{\pgfqpoint{2.703821in}{0.936079in}}%
\pgfpathlineto{\pgfqpoint{2.715762in}{0.940242in}}%
\pgfpathlineto{\pgfqpoint{2.727696in}{0.944511in}}%
\pgfpathlineto{\pgfqpoint{2.721318in}{0.953295in}}%
\pgfpathlineto{\pgfqpoint{2.714941in}{0.962404in}}%
\pgfpathlineto{\pgfqpoint{2.708566in}{0.971822in}}%
\pgfpathlineto{\pgfqpoint{2.702193in}{0.981532in}}%
\pgfpathlineto{\pgfqpoint{2.695821in}{0.991515in}}%
\pgfpathlineto{\pgfqpoint{2.683894in}{0.987844in}}%
\pgfpathlineto{\pgfqpoint{2.671960in}{0.984200in}}%
\pgfpathlineto{\pgfqpoint{2.660020in}{0.980576in}}%
\pgfpathlineto{\pgfqpoint{2.648074in}{0.976966in}}%
\pgfpathlineto{\pgfqpoint{2.636120in}{0.973364in}}%
\pgfpathlineto{\pgfqpoint{2.642484in}{0.963061in}}%
\pgfpathlineto{\pgfqpoint{2.648850in}{0.952964in}}%
\pgfpathlineto{\pgfqpoint{2.655217in}{0.943084in}}%
\pgfpathlineto{\pgfqpoint{2.661587in}{0.933436in}}%
\pgfpathclose%
\pgfusepath{stroke,fill}%
\end{pgfscope}%
\begin{pgfscope}%
\pgfpathrectangle{\pgfqpoint{0.887500in}{0.275000in}}{\pgfqpoint{4.225000in}{4.225000in}}%
\pgfusepath{clip}%
\pgfsetbuttcap%
\pgfsetroundjoin%
\definecolor{currentfill}{rgb}{0.283229,0.120777,0.440584}%
\pgfsetfillcolor{currentfill}%
\pgfsetfillopacity{0.700000}%
\pgfsetlinewidth{0.501875pt}%
\definecolor{currentstroke}{rgb}{1.000000,1.000000,1.000000}%
\pgfsetstrokecolor{currentstroke}%
\pgfsetstrokeopacity{0.500000}%
\pgfsetdash{}{0pt}%
\pgfpathmoveto{\pgfqpoint{2.424410in}{0.974317in}}%
\pgfpathlineto{\pgfqpoint{2.436449in}{0.977904in}}%
\pgfpathlineto{\pgfqpoint{2.448481in}{0.981485in}}%
\pgfpathlineto{\pgfqpoint{2.460508in}{0.985061in}}%
\pgfpathlineto{\pgfqpoint{2.472528in}{0.988632in}}%
\pgfpathlineto{\pgfqpoint{2.484541in}{0.992196in}}%
\pgfpathlineto{\pgfqpoint{2.478208in}{1.003457in}}%
\pgfpathlineto{\pgfqpoint{2.471878in}{1.014790in}}%
\pgfpathlineto{\pgfqpoint{2.465550in}{1.026190in}}%
\pgfpathlineto{\pgfqpoint{2.459225in}{1.037648in}}%
\pgfpathlineto{\pgfqpoint{2.452903in}{1.049160in}}%
\pgfpathlineto{\pgfqpoint{2.440899in}{1.045555in}}%
\pgfpathlineto{\pgfqpoint{2.428889in}{1.041943in}}%
\pgfpathlineto{\pgfqpoint{2.416873in}{1.038326in}}%
\pgfpathlineto{\pgfqpoint{2.404851in}{1.034701in}}%
\pgfpathlineto{\pgfqpoint{2.392822in}{1.031069in}}%
\pgfpathlineto{\pgfqpoint{2.399134in}{1.019621in}}%
\pgfpathlineto{\pgfqpoint{2.405449in}{1.008217in}}%
\pgfpathlineto{\pgfqpoint{2.411767in}{0.996862in}}%
\pgfpathlineto{\pgfqpoint{2.418087in}{0.985560in}}%
\pgfpathclose%
\pgfusepath{stroke,fill}%
\end{pgfscope}%
\begin{pgfscope}%
\pgfpathrectangle{\pgfqpoint{0.887500in}{0.275000in}}{\pgfqpoint{4.225000in}{4.225000in}}%
\pgfusepath{clip}%
\pgfsetbuttcap%
\pgfsetroundjoin%
\definecolor{currentfill}{rgb}{0.282656,0.100196,0.422160}%
\pgfsetfillcolor{currentfill}%
\pgfsetfillopacity{0.700000}%
\pgfsetlinewidth{0.501875pt}%
\definecolor{currentstroke}{rgb}{1.000000,1.000000,1.000000}%
\pgfsetstrokecolor{currentstroke}%
\pgfsetstrokeopacity{0.500000}%
\pgfsetdash{}{0pt}%
\pgfpathmoveto{\pgfqpoint{2.516242in}{0.937216in}}%
\pgfpathlineto{\pgfqpoint{2.528258in}{0.940837in}}%
\pgfpathlineto{\pgfqpoint{2.540268in}{0.944459in}}%
\pgfpathlineto{\pgfqpoint{2.552272in}{0.948081in}}%
\pgfpathlineto{\pgfqpoint{2.564269in}{0.951702in}}%
\pgfpathlineto{\pgfqpoint{2.576260in}{0.955322in}}%
\pgfpathlineto{\pgfqpoint{2.569907in}{0.965985in}}%
\pgfpathlineto{\pgfqpoint{2.563556in}{0.976786in}}%
\pgfpathlineto{\pgfqpoint{2.557206in}{0.987718in}}%
\pgfpathlineto{\pgfqpoint{2.550859in}{0.998770in}}%
\pgfpathlineto{\pgfqpoint{2.544515in}{1.009934in}}%
\pgfpathlineto{\pgfqpoint{2.532533in}{1.006398in}}%
\pgfpathlineto{\pgfqpoint{2.520544in}{1.002857in}}%
\pgfpathlineto{\pgfqpoint{2.508550in}{0.999309in}}%
\pgfpathlineto{\pgfqpoint{2.496549in}{0.995756in}}%
\pgfpathlineto{\pgfqpoint{2.484541in}{0.992196in}}%
\pgfpathlineto{\pgfqpoint{2.490877in}{0.981015in}}%
\pgfpathlineto{\pgfqpoint{2.497214in}{0.969920in}}%
\pgfpathlineto{\pgfqpoint{2.503554in}{0.958917in}}%
\pgfpathlineto{\pgfqpoint{2.509897in}{0.948013in}}%
\pgfpathclose%
\pgfusepath{stroke,fill}%
\end{pgfscope}%
\begin{pgfscope}%
\pgfpathrectangle{\pgfqpoint{0.887500in}{0.275000in}}{\pgfqpoint{4.225000in}{4.225000in}}%
\pgfusepath{clip}%
\pgfsetbuttcap%
\pgfsetroundjoin%
\definecolor{currentfill}{rgb}{0.280894,0.078907,0.402329}%
\pgfsetfillcolor{currentfill}%
\pgfsetfillopacity{0.700000}%
\pgfsetlinewidth{0.501875pt}%
\definecolor{currentstroke}{rgb}{1.000000,1.000000,1.000000}%
\pgfsetstrokecolor{currentstroke}%
\pgfsetstrokeopacity{0.500000}%
\pgfsetdash{}{0pt}%
\pgfpathmoveto{\pgfqpoint{2.608057in}{0.904414in}}%
\pgfpathlineto{\pgfqpoint{2.620050in}{0.908324in}}%
\pgfpathlineto{\pgfqpoint{2.632036in}{0.912243in}}%
\pgfpathlineto{\pgfqpoint{2.644016in}{0.916167in}}%
\pgfpathlineto{\pgfqpoint{2.655990in}{0.920094in}}%
\pgfpathlineto{\pgfqpoint{2.667957in}{0.924030in}}%
\pgfpathlineto{\pgfqpoint{2.661587in}{0.933436in}}%
\pgfpathlineto{\pgfqpoint{2.655217in}{0.943084in}}%
\pgfpathlineto{\pgfqpoint{2.648850in}{0.952964in}}%
\pgfpathlineto{\pgfqpoint{2.642484in}{0.963061in}}%
\pgfpathlineto{\pgfqpoint{2.636120in}{0.973364in}}%
\pgfpathlineto{\pgfqpoint{2.624161in}{0.969764in}}%
\pgfpathlineto{\pgfqpoint{2.612195in}{0.966160in}}%
\pgfpathlineto{\pgfqpoint{2.600223in}{0.962552in}}%
\pgfpathlineto{\pgfqpoint{2.588245in}{0.958938in}}%
\pgfpathlineto{\pgfqpoint{2.576260in}{0.955322in}}%
\pgfpathlineto{\pgfqpoint{2.582616in}{0.944807in}}%
\pgfpathlineto{\pgfqpoint{2.588973in}{0.934449in}}%
\pgfpathlineto{\pgfqpoint{2.595333in}{0.924259in}}%
\pgfpathlineto{\pgfqpoint{2.601694in}{0.914244in}}%
\pgfpathclose%
\pgfusepath{stroke,fill}%
\end{pgfscope}%
\begin{pgfscope}%
\pgfpathrectangle{\pgfqpoint{0.887500in}{0.275000in}}{\pgfqpoint{4.225000in}{4.225000in}}%
\pgfusepath{clip}%
\pgfsetbuttcap%
\pgfsetroundjoin%
\definecolor{currentfill}{rgb}{0.282656,0.100196,0.422160}%
\pgfsetfillcolor{currentfill}%
\pgfsetfillopacity{0.700000}%
\pgfsetlinewidth{0.501875pt}%
\definecolor{currentstroke}{rgb}{1.000000,1.000000,1.000000}%
\pgfsetstrokecolor{currentstroke}%
\pgfsetstrokeopacity{0.500000}%
\pgfsetdash{}{0pt}%
\pgfpathmoveto{\pgfqpoint{2.456062in}{0.919153in}}%
\pgfpathlineto{\pgfqpoint{2.468111in}{0.922757in}}%
\pgfpathlineto{\pgfqpoint{2.480153in}{0.926366in}}%
\pgfpathlineto{\pgfqpoint{2.492189in}{0.929980in}}%
\pgfpathlineto{\pgfqpoint{2.504219in}{0.933596in}}%
\pgfpathlineto{\pgfqpoint{2.516242in}{0.937216in}}%
\pgfpathlineto{\pgfqpoint{2.509897in}{0.948013in}}%
\pgfpathlineto{\pgfqpoint{2.503554in}{0.958917in}}%
\pgfpathlineto{\pgfqpoint{2.497214in}{0.969920in}}%
\pgfpathlineto{\pgfqpoint{2.490877in}{0.981015in}}%
\pgfpathlineto{\pgfqpoint{2.484541in}{0.992196in}}%
\pgfpathlineto{\pgfqpoint{2.472528in}{0.988632in}}%
\pgfpathlineto{\pgfqpoint{2.460508in}{0.985061in}}%
\pgfpathlineto{\pgfqpoint{2.448481in}{0.981485in}}%
\pgfpathlineto{\pgfqpoint{2.436449in}{0.977904in}}%
\pgfpathlineto{\pgfqpoint{2.424410in}{0.974317in}}%
\pgfpathlineto{\pgfqpoint{2.430735in}{0.963137in}}%
\pgfpathlineto{\pgfqpoint{2.437063in}{0.952026in}}%
\pgfpathlineto{\pgfqpoint{2.443394in}{0.940988in}}%
\pgfpathlineto{\pgfqpoint{2.449727in}{0.930029in}}%
\pgfpathclose%
\pgfusepath{stroke,fill}%
\end{pgfscope}%
\begin{pgfscope}%
\pgfpathrectangle{\pgfqpoint{0.887500in}{0.275000in}}{\pgfqpoint{4.225000in}{4.225000in}}%
\pgfusepath{clip}%
\pgfsetbuttcap%
\pgfsetroundjoin%
\definecolor{currentfill}{rgb}{0.280894,0.078907,0.402329}%
\pgfsetfillcolor{currentfill}%
\pgfsetfillopacity{0.700000}%
\pgfsetlinewidth{0.501875pt}%
\definecolor{currentstroke}{rgb}{1.000000,1.000000,1.000000}%
\pgfsetstrokecolor{currentstroke}%
\pgfsetstrokeopacity{0.500000}%
\pgfsetdash{}{0pt}%
\pgfpathmoveto{\pgfqpoint{2.547999in}{0.885049in}}%
\pgfpathlineto{\pgfqpoint{2.560023in}{0.888890in}}%
\pgfpathlineto{\pgfqpoint{2.572041in}{0.892749in}}%
\pgfpathlineto{\pgfqpoint{2.584053in}{0.896624in}}%
\pgfpathlineto{\pgfqpoint{2.596058in}{0.900513in}}%
\pgfpathlineto{\pgfqpoint{2.608057in}{0.904414in}}%
\pgfpathlineto{\pgfqpoint{2.601694in}{0.914244in}}%
\pgfpathlineto{\pgfqpoint{2.595333in}{0.924259in}}%
\pgfpathlineto{\pgfqpoint{2.588973in}{0.934449in}}%
\pgfpathlineto{\pgfqpoint{2.582616in}{0.944807in}}%
\pgfpathlineto{\pgfqpoint{2.576260in}{0.955322in}}%
\pgfpathlineto{\pgfqpoint{2.564269in}{0.951702in}}%
\pgfpathlineto{\pgfqpoint{2.552272in}{0.948081in}}%
\pgfpathlineto{\pgfqpoint{2.540268in}{0.944459in}}%
\pgfpathlineto{\pgfqpoint{2.528258in}{0.940837in}}%
\pgfpathlineto{\pgfqpoint{2.516242in}{0.937216in}}%
\pgfpathlineto{\pgfqpoint{2.522589in}{0.926531in}}%
\pgfpathlineto{\pgfqpoint{2.528938in}{0.915965in}}%
\pgfpathlineto{\pgfqpoint{2.535289in}{0.905525in}}%
\pgfpathlineto{\pgfqpoint{2.541643in}{0.895218in}}%
\pgfpathclose%
\pgfusepath{stroke,fill}%
\end{pgfscope}%
\begin{pgfscope}%
\pgfpathrectangle{\pgfqpoint{0.887500in}{0.275000in}}{\pgfqpoint{4.225000in}{4.225000in}}%
\pgfusepath{clip}%
\pgfsetbuttcap%
\pgfsetroundjoin%
\definecolor{currentfill}{rgb}{0.280894,0.078907,0.402329}%
\pgfsetfillcolor{currentfill}%
\pgfsetfillopacity{0.700000}%
\pgfsetlinewidth{0.501875pt}%
\definecolor{currentstroke}{rgb}{1.000000,1.000000,1.000000}%
\pgfsetstrokecolor{currentstroke}%
\pgfsetstrokeopacity{0.500000}%
\pgfsetdash{}{0pt}%
\pgfpathmoveto{\pgfqpoint{2.487775in}{0.866199in}}%
\pgfpathlineto{\pgfqpoint{2.499834in}{0.869915in}}%
\pgfpathlineto{\pgfqpoint{2.511885in}{0.873660in}}%
\pgfpathlineto{\pgfqpoint{2.523929in}{0.877432in}}%
\pgfpathlineto{\pgfqpoint{2.535967in}{0.881229in}}%
\pgfpathlineto{\pgfqpoint{2.547999in}{0.885049in}}%
\pgfpathlineto{\pgfqpoint{2.541643in}{0.895218in}}%
\pgfpathlineto{\pgfqpoint{2.535289in}{0.905525in}}%
\pgfpathlineto{\pgfqpoint{2.528938in}{0.915965in}}%
\pgfpathlineto{\pgfqpoint{2.522589in}{0.926531in}}%
\pgfpathlineto{\pgfqpoint{2.516242in}{0.937216in}}%
\pgfpathlineto{\pgfqpoint{2.504219in}{0.933596in}}%
\pgfpathlineto{\pgfqpoint{2.492189in}{0.929980in}}%
\pgfpathlineto{\pgfqpoint{2.480153in}{0.926366in}}%
\pgfpathlineto{\pgfqpoint{2.468111in}{0.922757in}}%
\pgfpathlineto{\pgfqpoint{2.456062in}{0.919153in}}%
\pgfpathlineto{\pgfqpoint{2.462400in}{0.908366in}}%
\pgfpathlineto{\pgfqpoint{2.468740in}{0.897672in}}%
\pgfpathlineto{\pgfqpoint{2.475083in}{0.887076in}}%
\pgfpathlineto{\pgfqpoint{2.481428in}{0.876583in}}%
\pgfpathclose%
\pgfusepath{stroke,fill}%
\end{pgfscope}%
\begin{pgfscope}%
\pgfpathrectangle{\pgfqpoint{0.887500in}{0.275000in}}{\pgfqpoint{4.225000in}{4.225000in}}%
\pgfusepath{clip}%
\pgfsetbuttcap%
\pgfsetroundjoin%
\definecolor{currentfill}{rgb}{0.000000,0.000000,0.000000}%
\pgfsetfillcolor{currentfill}%
\pgfsetfillopacity{0.800000}%
\pgfsetlinewidth{0.000000pt}%
\definecolor{currentstroke}{rgb}{0.000000,0.000000,0.000000}%
\pgfsetstrokecolor{currentstroke}%
\pgfsetstrokeopacity{0.800000}%
\pgfsetdash{}{0pt}%
\pgfpathmoveto{\pgfqpoint{3.404419in}{3.821112in}}%
\pgfpathcurveto{\pgfqpoint{3.408538in}{3.821112in}}{\pgfqpoint{3.412488in}{3.822748in}}{\pgfqpoint{3.415400in}{3.825660in}}%
\pgfpathcurveto{\pgfqpoint{3.418312in}{3.828572in}}{\pgfqpoint{3.419948in}{3.832522in}}{\pgfqpoint{3.419948in}{3.836640in}}%
\pgfpathcurveto{\pgfqpoint{3.419948in}{3.840758in}}{\pgfqpoint{3.418312in}{3.844708in}}{\pgfqpoint{3.415400in}{3.847620in}}%
\pgfpathcurveto{\pgfqpoint{3.412488in}{3.850532in}}{\pgfqpoint{3.408538in}{3.852168in}}{\pgfqpoint{3.404419in}{3.852168in}}%
\pgfpathcurveto{\pgfqpoint{3.400301in}{3.852168in}}{\pgfqpoint{3.396351in}{3.850532in}}{\pgfqpoint{3.393439in}{3.847620in}}%
\pgfpathcurveto{\pgfqpoint{3.390527in}{3.844708in}}{\pgfqpoint{3.388891in}{3.840758in}}{\pgfqpoint{3.388891in}{3.836640in}}%
\pgfpathcurveto{\pgfqpoint{3.388891in}{3.832522in}}{\pgfqpoint{3.390527in}{3.828572in}}{\pgfqpoint{3.393439in}{3.825660in}}%
\pgfpathcurveto{\pgfqpoint{3.396351in}{3.822748in}}{\pgfqpoint{3.400301in}{3.821112in}}{\pgfqpoint{3.404419in}{3.821112in}}%
\pgfpathclose%
\pgfusepath{fill}%
\end{pgfscope}%
\begin{pgfscope}%
\pgfpathrectangle{\pgfqpoint{0.887500in}{0.275000in}}{\pgfqpoint{4.225000in}{4.225000in}}%
\pgfusepath{clip}%
\pgfsetbuttcap%
\pgfsetroundjoin%
\definecolor{currentfill}{rgb}{0.000000,0.000000,0.000000}%
\pgfsetfillcolor{currentfill}%
\pgfsetfillopacity{0.800000}%
\pgfsetlinewidth{0.000000pt}%
\definecolor{currentstroke}{rgb}{0.000000,0.000000,0.000000}%
\pgfsetstrokecolor{currentstroke}%
\pgfsetstrokeopacity{0.800000}%
\pgfsetdash{}{0pt}%
\pgfpathmoveto{\pgfqpoint{3.575302in}{3.815765in}}%
\pgfpathcurveto{\pgfqpoint{3.579420in}{3.815765in}}{\pgfqpoint{3.583370in}{3.817401in}}{\pgfqpoint{3.586282in}{3.820313in}}%
\pgfpathcurveto{\pgfqpoint{3.589194in}{3.823225in}}{\pgfqpoint{3.590830in}{3.827175in}}{\pgfqpoint{3.590830in}{3.831293in}}%
\pgfpathcurveto{\pgfqpoint{3.590830in}{3.835411in}}{\pgfqpoint{3.589194in}{3.839361in}}{\pgfqpoint{3.586282in}{3.842273in}}%
\pgfpathcurveto{\pgfqpoint{3.583370in}{3.845185in}}{\pgfqpoint{3.579420in}{3.846821in}}{\pgfqpoint{3.575302in}{3.846821in}}%
\pgfpathcurveto{\pgfqpoint{3.571183in}{3.846821in}}{\pgfqpoint{3.567233in}{3.845185in}}{\pgfqpoint{3.564321in}{3.842273in}}%
\pgfpathcurveto{\pgfqpoint{3.561409in}{3.839361in}}{\pgfqpoint{3.559773in}{3.835411in}}{\pgfqpoint{3.559773in}{3.831293in}}%
\pgfpathcurveto{\pgfqpoint{3.559773in}{3.827175in}}{\pgfqpoint{3.561409in}{3.823225in}}{\pgfqpoint{3.564321in}{3.820313in}}%
\pgfpathcurveto{\pgfqpoint{3.567233in}{3.817401in}}{\pgfqpoint{3.571183in}{3.815765in}}{\pgfqpoint{3.575302in}{3.815765in}}%
\pgfpathclose%
\pgfusepath{fill}%
\end{pgfscope}%
\begin{pgfscope}%
\pgfpathrectangle{\pgfqpoint{0.887500in}{0.275000in}}{\pgfqpoint{4.225000in}{4.225000in}}%
\pgfusepath{clip}%
\pgfsetbuttcap%
\pgfsetroundjoin%
\definecolor{currentfill}{rgb}{0.000000,0.000000,0.000000}%
\pgfsetfillcolor{currentfill}%
\pgfsetfillopacity{0.800000}%
\pgfsetlinewidth{0.000000pt}%
\definecolor{currentstroke}{rgb}{0.000000,0.000000,0.000000}%
\pgfsetstrokecolor{currentstroke}%
\pgfsetstrokeopacity{0.800000}%
\pgfsetdash{}{0pt}%
\pgfpathmoveto{\pgfqpoint{3.297346in}{3.794696in}}%
\pgfpathcurveto{\pgfqpoint{3.301465in}{3.794696in}}{\pgfqpoint{3.305415in}{3.796332in}}{\pgfqpoint{3.308327in}{3.799244in}}%
\pgfpathcurveto{\pgfqpoint{3.311239in}{3.802156in}}{\pgfqpoint{3.312875in}{3.806106in}}{\pgfqpoint{3.312875in}{3.810224in}}%
\pgfpathcurveto{\pgfqpoint{3.312875in}{3.814342in}}{\pgfqpoint{3.311239in}{3.818292in}}{\pgfqpoint{3.308327in}{3.821204in}}%
\pgfpathcurveto{\pgfqpoint{3.305415in}{3.824116in}}{\pgfqpoint{3.301465in}{3.825752in}}{\pgfqpoint{3.297346in}{3.825752in}}%
\pgfpathcurveto{\pgfqpoint{3.293228in}{3.825752in}}{\pgfqpoint{3.289278in}{3.824116in}}{\pgfqpoint{3.286366in}{3.821204in}}%
\pgfpathcurveto{\pgfqpoint{3.283454in}{3.818292in}}{\pgfqpoint{3.281818in}{3.814342in}}{\pgfqpoint{3.281818in}{3.810224in}}%
\pgfpathcurveto{\pgfqpoint{3.281818in}{3.806106in}}{\pgfqpoint{3.283454in}{3.802156in}}{\pgfqpoint{3.286366in}{3.799244in}}%
\pgfpathcurveto{\pgfqpoint{3.289278in}{3.796332in}}{\pgfqpoint{3.293228in}{3.794696in}}{\pgfqpoint{3.297346in}{3.794696in}}%
\pgfpathclose%
\pgfusepath{fill}%
\end{pgfscope}%
\begin{pgfscope}%
\pgfpathrectangle{\pgfqpoint{0.887500in}{0.275000in}}{\pgfqpoint{4.225000in}{4.225000in}}%
\pgfusepath{clip}%
\pgfsetbuttcap%
\pgfsetroundjoin%
\definecolor{currentfill}{rgb}{0.000000,0.000000,0.000000}%
\pgfsetfillcolor{currentfill}%
\pgfsetfillopacity{0.800000}%
\pgfsetlinewidth{0.000000pt}%
\definecolor{currentstroke}{rgb}{0.000000,0.000000,0.000000}%
\pgfsetstrokecolor{currentstroke}%
\pgfsetstrokeopacity{0.800000}%
\pgfsetdash{}{0pt}%
\pgfpathmoveto{\pgfqpoint{3.189692in}{3.747568in}}%
\pgfpathcurveto{\pgfqpoint{3.193810in}{3.747568in}}{\pgfqpoint{3.197760in}{3.749204in}}{\pgfqpoint{3.200672in}{3.752116in}}%
\pgfpathcurveto{\pgfqpoint{3.203584in}{3.755028in}}{\pgfqpoint{3.205220in}{3.758978in}}{\pgfqpoint{3.205220in}{3.763096in}}%
\pgfpathcurveto{\pgfqpoint{3.205220in}{3.767214in}}{\pgfqpoint{3.203584in}{3.771164in}}{\pgfqpoint{3.200672in}{3.774076in}}%
\pgfpathcurveto{\pgfqpoint{3.197760in}{3.776988in}}{\pgfqpoint{3.193810in}{3.778624in}}{\pgfqpoint{3.189692in}{3.778624in}}%
\pgfpathcurveto{\pgfqpoint{3.185574in}{3.778624in}}{\pgfqpoint{3.181624in}{3.776988in}}{\pgfqpoint{3.178712in}{3.774076in}}%
\pgfpathcurveto{\pgfqpoint{3.175800in}{3.771164in}}{\pgfqpoint{3.174164in}{3.767214in}}{\pgfqpoint{3.174164in}{3.763096in}}%
\pgfpathcurveto{\pgfqpoint{3.174164in}{3.758978in}}{\pgfqpoint{3.175800in}{3.755028in}}{\pgfqpoint{3.178712in}{3.752116in}}%
\pgfpathcurveto{\pgfqpoint{3.181624in}{3.749204in}}{\pgfqpoint{3.185574in}{3.747568in}}{\pgfqpoint{3.189692in}{3.747568in}}%
\pgfpathclose%
\pgfusepath{fill}%
\end{pgfscope}%
\begin{pgfscope}%
\pgfpathrectangle{\pgfqpoint{0.887500in}{0.275000in}}{\pgfqpoint{4.225000in}{4.225000in}}%
\pgfusepath{clip}%
\pgfsetbuttcap%
\pgfsetroundjoin%
\definecolor{currentfill}{rgb}{0.000000,0.000000,0.000000}%
\pgfsetfillcolor{currentfill}%
\pgfsetfillopacity{0.800000}%
\pgfsetlinewidth{0.000000pt}%
\definecolor{currentstroke}{rgb}{0.000000,0.000000,0.000000}%
\pgfsetstrokecolor{currentstroke}%
\pgfsetstrokeopacity{0.800000}%
\pgfsetdash{}{0pt}%
\pgfpathmoveto{\pgfqpoint{3.468847in}{3.832313in}}%
\pgfpathcurveto{\pgfqpoint{3.472965in}{3.832313in}}{\pgfqpoint{3.476915in}{3.833949in}}{\pgfqpoint{3.479827in}{3.836861in}}%
\pgfpathcurveto{\pgfqpoint{3.482739in}{3.839773in}}{\pgfqpoint{3.484375in}{3.843723in}}{\pgfqpoint{3.484375in}{3.847841in}}%
\pgfpathcurveto{\pgfqpoint{3.484375in}{3.851959in}}{\pgfqpoint{3.482739in}{3.855909in}}{\pgfqpoint{3.479827in}{3.858821in}}%
\pgfpathcurveto{\pgfqpoint{3.476915in}{3.861733in}}{\pgfqpoint{3.472965in}{3.863369in}}{\pgfqpoint{3.468847in}{3.863369in}}%
\pgfpathcurveto{\pgfqpoint{3.464729in}{3.863369in}}{\pgfqpoint{3.460779in}{3.861733in}}{\pgfqpoint{3.457867in}{3.858821in}}%
\pgfpathcurveto{\pgfqpoint{3.454955in}{3.855909in}}{\pgfqpoint{3.453319in}{3.851959in}}{\pgfqpoint{3.453319in}{3.847841in}}%
\pgfpathcurveto{\pgfqpoint{3.453319in}{3.843723in}}{\pgfqpoint{3.454955in}{3.839773in}}{\pgfqpoint{3.457867in}{3.836861in}}%
\pgfpathcurveto{\pgfqpoint{3.460779in}{3.833949in}}{\pgfqpoint{3.464729in}{3.832313in}}{\pgfqpoint{3.468847in}{3.832313in}}%
\pgfpathclose%
\pgfusepath{fill}%
\end{pgfscope}%
\begin{pgfscope}%
\pgfpathrectangle{\pgfqpoint{0.887500in}{0.275000in}}{\pgfqpoint{4.225000in}{4.225000in}}%
\pgfusepath{clip}%
\pgfsetbuttcap%
\pgfsetroundjoin%
\definecolor{currentfill}{rgb}{0.000000,0.000000,0.000000}%
\pgfsetfillcolor{currentfill}%
\pgfsetfillopacity{0.800000}%
\pgfsetlinewidth{0.000000pt}%
\definecolor{currentstroke}{rgb}{0.000000,0.000000,0.000000}%
\pgfsetstrokecolor{currentstroke}%
\pgfsetstrokeopacity{0.800000}%
\pgfsetdash{}{0pt}%
\pgfpathmoveto{\pgfqpoint{3.081623in}{3.691584in}}%
\pgfpathcurveto{\pgfqpoint{3.085741in}{3.691584in}}{\pgfqpoint{3.089691in}{3.693220in}}{\pgfqpoint{3.092603in}{3.696132in}}%
\pgfpathcurveto{\pgfqpoint{3.095515in}{3.699044in}}{\pgfqpoint{3.097151in}{3.702994in}}{\pgfqpoint{3.097151in}{3.707112in}}%
\pgfpathcurveto{\pgfqpoint{3.097151in}{3.711230in}}{\pgfqpoint{3.095515in}{3.715180in}}{\pgfqpoint{3.092603in}{3.718092in}}%
\pgfpathcurveto{\pgfqpoint{3.089691in}{3.721004in}}{\pgfqpoint{3.085741in}{3.722640in}}{\pgfqpoint{3.081623in}{3.722640in}}%
\pgfpathcurveto{\pgfqpoint{3.077505in}{3.722640in}}{\pgfqpoint{3.073555in}{3.721004in}}{\pgfqpoint{3.070643in}{3.718092in}}%
\pgfpathcurveto{\pgfqpoint{3.067731in}{3.715180in}}{\pgfqpoint{3.066094in}{3.711230in}}{\pgfqpoint{3.066094in}{3.707112in}}%
\pgfpathcurveto{\pgfqpoint{3.066094in}{3.702994in}}{\pgfqpoint{3.067731in}{3.699044in}}{\pgfqpoint{3.070643in}{3.696132in}}%
\pgfpathcurveto{\pgfqpoint{3.073555in}{3.693220in}}{\pgfqpoint{3.077505in}{3.691584in}}{\pgfqpoint{3.081623in}{3.691584in}}%
\pgfpathclose%
\pgfusepath{fill}%
\end{pgfscope}%
\begin{pgfscope}%
\pgfpathrectangle{\pgfqpoint{0.887500in}{0.275000in}}{\pgfqpoint{4.225000in}{4.225000in}}%
\pgfusepath{clip}%
\pgfsetbuttcap%
\pgfsetroundjoin%
\definecolor{currentfill}{rgb}{0.000000,0.000000,0.000000}%
\pgfsetfillcolor{currentfill}%
\pgfsetfillopacity{0.800000}%
\pgfsetlinewidth{0.000000pt}%
\definecolor{currentstroke}{rgb}{0.000000,0.000000,0.000000}%
\pgfsetstrokecolor{currentstroke}%
\pgfsetstrokeopacity{0.800000}%
\pgfsetdash{}{0pt}%
\pgfpathmoveto{\pgfqpoint{3.361242in}{3.793561in}}%
\pgfpathcurveto{\pgfqpoint{3.365360in}{3.793561in}}{\pgfqpoint{3.369310in}{3.795197in}}{\pgfqpoint{3.372222in}{3.798109in}}%
\pgfpathcurveto{\pgfqpoint{3.375134in}{3.801021in}}{\pgfqpoint{3.376770in}{3.804971in}}{\pgfqpoint{3.376770in}{3.809089in}}%
\pgfpathcurveto{\pgfqpoint{3.376770in}{3.813207in}}{\pgfqpoint{3.375134in}{3.817157in}}{\pgfqpoint{3.372222in}{3.820069in}}%
\pgfpathcurveto{\pgfqpoint{3.369310in}{3.822981in}}{\pgfqpoint{3.365360in}{3.824617in}}{\pgfqpoint{3.361242in}{3.824617in}}%
\pgfpathcurveto{\pgfqpoint{3.357123in}{3.824617in}}{\pgfqpoint{3.353173in}{3.822981in}}{\pgfqpoint{3.350261in}{3.820069in}}%
\pgfpathcurveto{\pgfqpoint{3.347349in}{3.817157in}}{\pgfqpoint{3.345713in}{3.813207in}}{\pgfqpoint{3.345713in}{3.809089in}}%
\pgfpathcurveto{\pgfqpoint{3.345713in}{3.804971in}}{\pgfqpoint{3.347349in}{3.801021in}}{\pgfqpoint{3.350261in}{3.798109in}}%
\pgfpathcurveto{\pgfqpoint{3.353173in}{3.795197in}}{\pgfqpoint{3.357123in}{3.793561in}}{\pgfqpoint{3.361242in}{3.793561in}}%
\pgfpathclose%
\pgfusepath{fill}%
\end{pgfscope}%
\begin{pgfscope}%
\pgfpathrectangle{\pgfqpoint{0.887500in}{0.275000in}}{\pgfqpoint{4.225000in}{4.225000in}}%
\pgfusepath{clip}%
\pgfsetbuttcap%
\pgfsetroundjoin%
\definecolor{currentfill}{rgb}{0.000000,0.000000,0.000000}%
\pgfsetfillcolor{currentfill}%
\pgfsetfillopacity{0.800000}%
\pgfsetlinewidth{0.000000pt}%
\definecolor{currentstroke}{rgb}{0.000000,0.000000,0.000000}%
\pgfsetstrokecolor{currentstroke}%
\pgfsetstrokeopacity{0.800000}%
\pgfsetdash{}{0pt}%
\pgfpathmoveto{\pgfqpoint{2.973185in}{3.636857in}}%
\pgfpathcurveto{\pgfqpoint{2.977303in}{3.636857in}}{\pgfqpoint{2.981253in}{3.638493in}}{\pgfqpoint{2.984165in}{3.641405in}}%
\pgfpathcurveto{\pgfqpoint{2.987077in}{3.644317in}}{\pgfqpoint{2.988713in}{3.648267in}}{\pgfqpoint{2.988713in}{3.652385in}}%
\pgfpathcurveto{\pgfqpoint{2.988713in}{3.656503in}}{\pgfqpoint{2.987077in}{3.660453in}}{\pgfqpoint{2.984165in}{3.663365in}}%
\pgfpathcurveto{\pgfqpoint{2.981253in}{3.666277in}}{\pgfqpoint{2.977303in}{3.667913in}}{\pgfqpoint{2.973185in}{3.667913in}}%
\pgfpathcurveto{\pgfqpoint{2.969067in}{3.667913in}}{\pgfqpoint{2.965116in}{3.666277in}}{\pgfqpoint{2.962205in}{3.663365in}}%
\pgfpathcurveto{\pgfqpoint{2.959293in}{3.660453in}}{\pgfqpoint{2.957656in}{3.656503in}}{\pgfqpoint{2.957656in}{3.652385in}}%
\pgfpathcurveto{\pgfqpoint{2.957656in}{3.648267in}}{\pgfqpoint{2.959293in}{3.644317in}}{\pgfqpoint{2.962205in}{3.641405in}}%
\pgfpathcurveto{\pgfqpoint{2.965116in}{3.638493in}}{\pgfqpoint{2.969067in}{3.636857in}}{\pgfqpoint{2.973185in}{3.636857in}}%
\pgfpathclose%
\pgfusepath{fill}%
\end{pgfscope}%
\begin{pgfscope}%
\pgfpathrectangle{\pgfqpoint{0.887500in}{0.275000in}}{\pgfqpoint{4.225000in}{4.225000in}}%
\pgfusepath{clip}%
\pgfsetbuttcap%
\pgfsetroundjoin%
\definecolor{currentfill}{rgb}{0.000000,0.000000,0.000000}%
\pgfsetfillcolor{currentfill}%
\pgfsetfillopacity{0.800000}%
\pgfsetlinewidth{0.000000pt}%
\definecolor{currentstroke}{rgb}{0.000000,0.000000,0.000000}%
\pgfsetstrokecolor{currentstroke}%
\pgfsetstrokeopacity{0.800000}%
\pgfsetdash{}{0pt}%
\pgfpathmoveto{\pgfqpoint{3.641122in}{3.848339in}}%
\pgfpathcurveto{\pgfqpoint{3.645240in}{3.848339in}}{\pgfqpoint{3.649190in}{3.849975in}}{\pgfqpoint{3.652102in}{3.852887in}}%
\pgfpathcurveto{\pgfqpoint{3.655014in}{3.855799in}}{\pgfqpoint{3.656650in}{3.859749in}}{\pgfqpoint{3.656650in}{3.863867in}}%
\pgfpathcurveto{\pgfqpoint{3.656650in}{3.867985in}}{\pgfqpoint{3.655014in}{3.871935in}}{\pgfqpoint{3.652102in}{3.874847in}}%
\pgfpathcurveto{\pgfqpoint{3.649190in}{3.877759in}}{\pgfqpoint{3.645240in}{3.879395in}}{\pgfqpoint{3.641122in}{3.879395in}}%
\pgfpathcurveto{\pgfqpoint{3.637003in}{3.879395in}}{\pgfqpoint{3.633053in}{3.877759in}}{\pgfqpoint{3.630141in}{3.874847in}}%
\pgfpathcurveto{\pgfqpoint{3.627229in}{3.871935in}}{\pgfqpoint{3.625593in}{3.867985in}}{\pgfqpoint{3.625593in}{3.863867in}}%
\pgfpathcurveto{\pgfqpoint{3.625593in}{3.859749in}}{\pgfqpoint{3.627229in}{3.855799in}}{\pgfqpoint{3.630141in}{3.852887in}}%
\pgfpathcurveto{\pgfqpoint{3.633053in}{3.849975in}}{\pgfqpoint{3.637003in}{3.848339in}}{\pgfqpoint{3.641122in}{3.848339in}}%
\pgfpathclose%
\pgfusepath{fill}%
\end{pgfscope}%
\begin{pgfscope}%
\pgfpathrectangle{\pgfqpoint{0.887500in}{0.275000in}}{\pgfqpoint{4.225000in}{4.225000in}}%
\pgfusepath{clip}%
\pgfsetbuttcap%
\pgfsetroundjoin%
\definecolor{currentfill}{rgb}{0.000000,0.000000,0.000000}%
\pgfsetfillcolor{currentfill}%
\pgfsetfillopacity{0.800000}%
\pgfsetlinewidth{0.000000pt}%
\definecolor{currentstroke}{rgb}{0.000000,0.000000,0.000000}%
\pgfsetstrokecolor{currentstroke}%
\pgfsetstrokeopacity{0.800000}%
\pgfsetdash{}{0pt}%
\pgfpathmoveto{\pgfqpoint{3.253109in}{3.739773in}}%
\pgfpathcurveto{\pgfqpoint{3.257228in}{3.739773in}}{\pgfqpoint{3.261178in}{3.741409in}}{\pgfqpoint{3.264090in}{3.744321in}}%
\pgfpathcurveto{\pgfqpoint{3.267002in}{3.747233in}}{\pgfqpoint{3.268638in}{3.751183in}}{\pgfqpoint{3.268638in}{3.755301in}}%
\pgfpathcurveto{\pgfqpoint{3.268638in}{3.759419in}}{\pgfqpoint{3.267002in}{3.763369in}}{\pgfqpoint{3.264090in}{3.766281in}}%
\pgfpathcurveto{\pgfqpoint{3.261178in}{3.769193in}}{\pgfqpoint{3.257228in}{3.770829in}}{\pgfqpoint{3.253109in}{3.770829in}}%
\pgfpathcurveto{\pgfqpoint{3.248991in}{3.770829in}}{\pgfqpoint{3.245041in}{3.769193in}}{\pgfqpoint{3.242129in}{3.766281in}}%
\pgfpathcurveto{\pgfqpoint{3.239217in}{3.763369in}}{\pgfqpoint{3.237581in}{3.759419in}}{\pgfqpoint{3.237581in}{3.755301in}}%
\pgfpathcurveto{\pgfqpoint{3.237581in}{3.751183in}}{\pgfqpoint{3.239217in}{3.747233in}}{\pgfqpoint{3.242129in}{3.744321in}}%
\pgfpathcurveto{\pgfqpoint{3.245041in}{3.741409in}}{\pgfqpoint{3.248991in}{3.739773in}}{\pgfqpoint{3.253109in}{3.739773in}}%
\pgfpathclose%
\pgfusepath{fill}%
\end{pgfscope}%
\begin{pgfscope}%
\pgfpathrectangle{\pgfqpoint{0.887500in}{0.275000in}}{\pgfqpoint{4.225000in}{4.225000in}}%
\pgfusepath{clip}%
\pgfsetbuttcap%
\pgfsetroundjoin%
\definecolor{currentfill}{rgb}{0.000000,0.000000,0.000000}%
\pgfsetfillcolor{currentfill}%
\pgfsetfillopacity{0.800000}%
\pgfsetlinewidth{0.000000pt}%
\definecolor{currentstroke}{rgb}{0.000000,0.000000,0.000000}%
\pgfsetstrokecolor{currentstroke}%
\pgfsetstrokeopacity{0.800000}%
\pgfsetdash{}{0pt}%
\pgfpathmoveto{\pgfqpoint{2.864355in}{3.585345in}}%
\pgfpathcurveto{\pgfqpoint{2.868473in}{3.585345in}}{\pgfqpoint{2.872423in}{3.586981in}}{\pgfqpoint{2.875335in}{3.589893in}}%
\pgfpathcurveto{\pgfqpoint{2.878247in}{3.592805in}}{\pgfqpoint{2.879883in}{3.596755in}}{\pgfqpoint{2.879883in}{3.600873in}}%
\pgfpathcurveto{\pgfqpoint{2.879883in}{3.604991in}}{\pgfqpoint{2.878247in}{3.608941in}}{\pgfqpoint{2.875335in}{3.611853in}}%
\pgfpathcurveto{\pgfqpoint{2.872423in}{3.614765in}}{\pgfqpoint{2.868473in}{3.616402in}}{\pgfqpoint{2.864355in}{3.616402in}}%
\pgfpathcurveto{\pgfqpoint{2.860237in}{3.616402in}}{\pgfqpoint{2.856287in}{3.614765in}}{\pgfqpoint{2.853375in}{3.611853in}}%
\pgfpathcurveto{\pgfqpoint{2.850463in}{3.608941in}}{\pgfqpoint{2.848827in}{3.604991in}}{\pgfqpoint{2.848827in}{3.600873in}}%
\pgfpathcurveto{\pgfqpoint{2.848827in}{3.596755in}}{\pgfqpoint{2.850463in}{3.592805in}}{\pgfqpoint{2.853375in}{3.589893in}}%
\pgfpathcurveto{\pgfqpoint{2.856287in}{3.586981in}}{\pgfqpoint{2.860237in}{3.585345in}}{\pgfqpoint{2.864355in}{3.585345in}}%
\pgfpathclose%
\pgfusepath{fill}%
\end{pgfscope}%
\begin{pgfscope}%
\pgfpathrectangle{\pgfqpoint{0.887500in}{0.275000in}}{\pgfqpoint{4.225000in}{4.225000in}}%
\pgfusepath{clip}%
\pgfsetbuttcap%
\pgfsetroundjoin%
\definecolor{currentfill}{rgb}{0.000000,0.000000,0.000000}%
\pgfsetfillcolor{currentfill}%
\pgfsetfillopacity{0.800000}%
\pgfsetlinewidth{0.000000pt}%
\definecolor{currentstroke}{rgb}{0.000000,0.000000,0.000000}%
\pgfsetstrokecolor{currentstroke}%
\pgfsetstrokeopacity{0.800000}%
\pgfsetdash{}{0pt}%
\pgfpathmoveto{\pgfqpoint{3.533908in}{3.836176in}}%
\pgfpathcurveto{\pgfqpoint{3.538026in}{3.836176in}}{\pgfqpoint{3.541976in}{3.837812in}}{\pgfqpoint{3.544888in}{3.840724in}}%
\pgfpathcurveto{\pgfqpoint{3.547800in}{3.843636in}}{\pgfqpoint{3.549436in}{3.847586in}}{\pgfqpoint{3.549436in}{3.851704in}}%
\pgfpathcurveto{\pgfqpoint{3.549436in}{3.855822in}}{\pgfqpoint{3.547800in}{3.859772in}}{\pgfqpoint{3.544888in}{3.862684in}}%
\pgfpathcurveto{\pgfqpoint{3.541976in}{3.865596in}}{\pgfqpoint{3.538026in}{3.867232in}}{\pgfqpoint{3.533908in}{3.867232in}}%
\pgfpathcurveto{\pgfqpoint{3.529790in}{3.867232in}}{\pgfqpoint{3.525840in}{3.865596in}}{\pgfqpoint{3.522928in}{3.862684in}}%
\pgfpathcurveto{\pgfqpoint{3.520016in}{3.859772in}}{\pgfqpoint{3.518379in}{3.855822in}}{\pgfqpoint{3.518379in}{3.851704in}}%
\pgfpathcurveto{\pgfqpoint{3.518379in}{3.847586in}}{\pgfqpoint{3.520016in}{3.843636in}}{\pgfqpoint{3.522928in}{3.840724in}}%
\pgfpathcurveto{\pgfqpoint{3.525840in}{3.837812in}}{\pgfqpoint{3.529790in}{3.836176in}}{\pgfqpoint{3.533908in}{3.836176in}}%
\pgfpathclose%
\pgfusepath{fill}%
\end{pgfscope}%
\begin{pgfscope}%
\pgfpathrectangle{\pgfqpoint{0.887500in}{0.275000in}}{\pgfqpoint{4.225000in}{4.225000in}}%
\pgfusepath{clip}%
\pgfsetbuttcap%
\pgfsetroundjoin%
\definecolor{currentfill}{rgb}{0.000000,0.000000,0.000000}%
\pgfsetfillcolor{currentfill}%
\pgfsetfillopacity{0.800000}%
\pgfsetlinewidth{0.000000pt}%
\definecolor{currentstroke}{rgb}{0.000000,0.000000,0.000000}%
\pgfsetstrokecolor{currentstroke}%
\pgfsetstrokeopacity{0.800000}%
\pgfsetdash{}{0pt}%
\pgfpathmoveto{\pgfqpoint{3.144585in}{3.679642in}}%
\pgfpathcurveto{\pgfqpoint{3.148703in}{3.679642in}}{\pgfqpoint{3.152653in}{3.681278in}}{\pgfqpoint{3.155565in}{3.684190in}}%
\pgfpathcurveto{\pgfqpoint{3.158477in}{3.687102in}}{\pgfqpoint{3.160113in}{3.691052in}}{\pgfqpoint{3.160113in}{3.695171in}}%
\pgfpathcurveto{\pgfqpoint{3.160113in}{3.699289in}}{\pgfqpoint{3.158477in}{3.703239in}}{\pgfqpoint{3.155565in}{3.706151in}}%
\pgfpathcurveto{\pgfqpoint{3.152653in}{3.709063in}}{\pgfqpoint{3.148703in}{3.710699in}}{\pgfqpoint{3.144585in}{3.710699in}}%
\pgfpathcurveto{\pgfqpoint{3.140467in}{3.710699in}}{\pgfqpoint{3.136517in}{3.709063in}}{\pgfqpoint{3.133605in}{3.706151in}}%
\pgfpathcurveto{\pgfqpoint{3.130693in}{3.703239in}}{\pgfqpoint{3.129056in}{3.699289in}}{\pgfqpoint{3.129056in}{3.695171in}}%
\pgfpathcurveto{\pgfqpoint{3.129056in}{3.691052in}}{\pgfqpoint{3.130693in}{3.687102in}}{\pgfqpoint{3.133605in}{3.684190in}}%
\pgfpathcurveto{\pgfqpoint{3.136517in}{3.681278in}}{\pgfqpoint{3.140467in}{3.679642in}}{\pgfqpoint{3.144585in}{3.679642in}}%
\pgfpathclose%
\pgfusepath{fill}%
\end{pgfscope}%
\begin{pgfscope}%
\pgfpathrectangle{\pgfqpoint{0.887500in}{0.275000in}}{\pgfqpoint{4.225000in}{4.225000in}}%
\pgfusepath{clip}%
\pgfsetbuttcap%
\pgfsetroundjoin%
\definecolor{currentfill}{rgb}{0.000000,0.000000,0.000000}%
\pgfsetfillcolor{currentfill}%
\pgfsetfillopacity{0.800000}%
\pgfsetlinewidth{0.000000pt}%
\definecolor{currentstroke}{rgb}{0.000000,0.000000,0.000000}%
\pgfsetstrokecolor{currentstroke}%
\pgfsetstrokeopacity{0.800000}%
\pgfsetdash{}{0pt}%
\pgfpathmoveto{\pgfqpoint{2.755211in}{3.522662in}}%
\pgfpathcurveto{\pgfqpoint{2.759329in}{3.522662in}}{\pgfqpoint{2.763279in}{3.524298in}}{\pgfqpoint{2.766191in}{3.527210in}}%
\pgfpathcurveto{\pgfqpoint{2.769103in}{3.530122in}}{\pgfqpoint{2.770740in}{3.534072in}}{\pgfqpoint{2.770740in}{3.538190in}}%
\pgfpathcurveto{\pgfqpoint{2.770740in}{3.542308in}}{\pgfqpoint{2.769103in}{3.546258in}}{\pgfqpoint{2.766191in}{3.549170in}}%
\pgfpathcurveto{\pgfqpoint{2.763279in}{3.552082in}}{\pgfqpoint{2.759329in}{3.553718in}}{\pgfqpoint{2.755211in}{3.553718in}}%
\pgfpathcurveto{\pgfqpoint{2.751093in}{3.553718in}}{\pgfqpoint{2.747143in}{3.552082in}}{\pgfqpoint{2.744231in}{3.549170in}}%
\pgfpathcurveto{\pgfqpoint{2.741319in}{3.546258in}}{\pgfqpoint{2.739683in}{3.542308in}}{\pgfqpoint{2.739683in}{3.538190in}}%
\pgfpathcurveto{\pgfqpoint{2.739683in}{3.534072in}}{\pgfqpoint{2.741319in}{3.530122in}}{\pgfqpoint{2.744231in}{3.527210in}}%
\pgfpathcurveto{\pgfqpoint{2.747143in}{3.524298in}}{\pgfqpoint{2.751093in}{3.522662in}}{\pgfqpoint{2.755211in}{3.522662in}}%
\pgfpathclose%
\pgfusepath{fill}%
\end{pgfscope}%
\begin{pgfscope}%
\pgfpathrectangle{\pgfqpoint{0.887500in}{0.275000in}}{\pgfqpoint{4.225000in}{4.225000in}}%
\pgfusepath{clip}%
\pgfsetbuttcap%
\pgfsetroundjoin%
\definecolor{currentfill}{rgb}{0.000000,0.000000,0.000000}%
\pgfsetfillcolor{currentfill}%
\pgfsetfillopacity{0.800000}%
\pgfsetlinewidth{0.000000pt}%
\definecolor{currentstroke}{rgb}{0.000000,0.000000,0.000000}%
\pgfsetstrokecolor{currentstroke}%
\pgfsetstrokeopacity{0.800000}%
\pgfsetdash{}{0pt}%
\pgfpathmoveto{\pgfqpoint{2.537228in}{3.294054in}}%
\pgfpathcurveto{\pgfqpoint{2.541346in}{3.294054in}}{\pgfqpoint{2.545296in}{3.295690in}}{\pgfqpoint{2.548208in}{3.298602in}}%
\pgfpathcurveto{\pgfqpoint{2.551120in}{3.301514in}}{\pgfqpoint{2.552756in}{3.305464in}}{\pgfqpoint{2.552756in}{3.309582in}}%
\pgfpathcurveto{\pgfqpoint{2.552756in}{3.313701in}}{\pgfqpoint{2.551120in}{3.317651in}}{\pgfqpoint{2.548208in}{3.320563in}}%
\pgfpathcurveto{\pgfqpoint{2.545296in}{3.323475in}}{\pgfqpoint{2.541346in}{3.325111in}}{\pgfqpoint{2.537228in}{3.325111in}}%
\pgfpathcurveto{\pgfqpoint{2.533109in}{3.325111in}}{\pgfqpoint{2.529159in}{3.323475in}}{\pgfqpoint{2.526247in}{3.320563in}}%
\pgfpathcurveto{\pgfqpoint{2.523335in}{3.317651in}}{\pgfqpoint{2.521699in}{3.313701in}}{\pgfqpoint{2.521699in}{3.309582in}}%
\pgfpathcurveto{\pgfqpoint{2.521699in}{3.305464in}}{\pgfqpoint{2.523335in}{3.301514in}}{\pgfqpoint{2.526247in}{3.298602in}}%
\pgfpathcurveto{\pgfqpoint{2.529159in}{3.295690in}}{\pgfqpoint{2.533109in}{3.294054in}}{\pgfqpoint{2.537228in}{3.294054in}}%
\pgfpathclose%
\pgfusepath{fill}%
\end{pgfscope}%
\begin{pgfscope}%
\pgfpathrectangle{\pgfqpoint{0.887500in}{0.275000in}}{\pgfqpoint{4.225000in}{4.225000in}}%
\pgfusepath{clip}%
\pgfsetbuttcap%
\pgfsetroundjoin%
\definecolor{currentfill}{rgb}{0.000000,0.000000,0.000000}%
\pgfsetfillcolor{currentfill}%
\pgfsetfillopacity{0.800000}%
\pgfsetlinewidth{0.000000pt}%
\definecolor{currentstroke}{rgb}{0.000000,0.000000,0.000000}%
\pgfsetstrokecolor{currentstroke}%
\pgfsetstrokeopacity{0.800000}%
\pgfsetdash{}{0pt}%
\pgfpathmoveto{\pgfqpoint{3.035711in}{3.620686in}}%
\pgfpathcurveto{\pgfqpoint{3.039829in}{3.620686in}}{\pgfqpoint{3.043779in}{3.622322in}}{\pgfqpoint{3.046691in}{3.625234in}}%
\pgfpathcurveto{\pgfqpoint{3.049603in}{3.628146in}}{\pgfqpoint{3.051239in}{3.632096in}}{\pgfqpoint{3.051239in}{3.636215in}}%
\pgfpathcurveto{\pgfqpoint{3.051239in}{3.640333in}}{\pgfqpoint{3.049603in}{3.644283in}}{\pgfqpoint{3.046691in}{3.647195in}}%
\pgfpathcurveto{\pgfqpoint{3.043779in}{3.650107in}}{\pgfqpoint{3.039829in}{3.651743in}}{\pgfqpoint{3.035711in}{3.651743in}}%
\pgfpathcurveto{\pgfqpoint{3.031592in}{3.651743in}}{\pgfqpoint{3.027642in}{3.650107in}}{\pgfqpoint{3.024730in}{3.647195in}}%
\pgfpathcurveto{\pgfqpoint{3.021818in}{3.644283in}}{\pgfqpoint{3.020182in}{3.640333in}}{\pgfqpoint{3.020182in}{3.636215in}}%
\pgfpathcurveto{\pgfqpoint{3.020182in}{3.632096in}}{\pgfqpoint{3.021818in}{3.628146in}}{\pgfqpoint{3.024730in}{3.625234in}}%
\pgfpathcurveto{\pgfqpoint{3.027642in}{3.622322in}}{\pgfqpoint{3.031592in}{3.620686in}}{\pgfqpoint{3.035711in}{3.620686in}}%
\pgfpathclose%
\pgfusepath{fill}%
\end{pgfscope}%
\begin{pgfscope}%
\pgfpathrectangle{\pgfqpoint{0.887500in}{0.275000in}}{\pgfqpoint{4.225000in}{4.225000in}}%
\pgfusepath{clip}%
\pgfsetbuttcap%
\pgfsetroundjoin%
\definecolor{currentfill}{rgb}{0.000000,0.000000,0.000000}%
\pgfsetfillcolor{currentfill}%
\pgfsetfillopacity{0.800000}%
\pgfsetlinewidth{0.000000pt}%
\definecolor{currentstroke}{rgb}{0.000000,0.000000,0.000000}%
\pgfsetstrokecolor{currentstroke}%
\pgfsetstrokeopacity{0.800000}%
\pgfsetdash{}{0pt}%
\pgfpathmoveto{\pgfqpoint{3.425772in}{3.787778in}}%
\pgfpathcurveto{\pgfqpoint{3.429890in}{3.787778in}}{\pgfqpoint{3.433840in}{3.789414in}}{\pgfqpoint{3.436752in}{3.792326in}}%
\pgfpathcurveto{\pgfqpoint{3.439664in}{3.795238in}}{\pgfqpoint{3.441301in}{3.799188in}}{\pgfqpoint{3.441301in}{3.803306in}}%
\pgfpathcurveto{\pgfqpoint{3.441301in}{3.807424in}}{\pgfqpoint{3.439664in}{3.811374in}}{\pgfqpoint{3.436752in}{3.814286in}}%
\pgfpathcurveto{\pgfqpoint{3.433840in}{3.817198in}}{\pgfqpoint{3.429890in}{3.818834in}}{\pgfqpoint{3.425772in}{3.818834in}}%
\pgfpathcurveto{\pgfqpoint{3.421654in}{3.818834in}}{\pgfqpoint{3.417704in}{3.817198in}}{\pgfqpoint{3.414792in}{3.814286in}}%
\pgfpathcurveto{\pgfqpoint{3.411880in}{3.811374in}}{\pgfqpoint{3.410244in}{3.807424in}}{\pgfqpoint{3.410244in}{3.803306in}}%
\pgfpathcurveto{\pgfqpoint{3.410244in}{3.799188in}}{\pgfqpoint{3.411880in}{3.795238in}}{\pgfqpoint{3.414792in}{3.792326in}}%
\pgfpathcurveto{\pgfqpoint{3.417704in}{3.789414in}}{\pgfqpoint{3.421654in}{3.787778in}}{\pgfqpoint{3.425772in}{3.787778in}}%
\pgfpathclose%
\pgfusepath{fill}%
\end{pgfscope}%
\begin{pgfscope}%
\pgfpathrectangle{\pgfqpoint{0.887500in}{0.275000in}}{\pgfqpoint{4.225000in}{4.225000in}}%
\pgfusepath{clip}%
\pgfsetbuttcap%
\pgfsetroundjoin%
\definecolor{currentfill}{rgb}{0.000000,0.000000,0.000000}%
\pgfsetfillcolor{currentfill}%
\pgfsetfillopacity{0.800000}%
\pgfsetlinewidth{0.000000pt}%
\definecolor{currentstroke}{rgb}{0.000000,0.000000,0.000000}%
\pgfsetstrokecolor{currentstroke}%
\pgfsetstrokeopacity{0.800000}%
\pgfsetdash{}{0pt}%
\pgfpathmoveto{\pgfqpoint{2.427166in}{3.242308in}}%
\pgfpathcurveto{\pgfqpoint{2.431284in}{3.242308in}}{\pgfqpoint{2.435234in}{3.243944in}}{\pgfqpoint{2.438146in}{3.246856in}}%
\pgfpathcurveto{\pgfqpoint{2.441058in}{3.249768in}}{\pgfqpoint{2.442694in}{3.253718in}}{\pgfqpoint{2.442694in}{3.257836in}}%
\pgfpathcurveto{\pgfqpoint{2.442694in}{3.261955in}}{\pgfqpoint{2.441058in}{3.265905in}}{\pgfqpoint{2.438146in}{3.268817in}}%
\pgfpathcurveto{\pgfqpoint{2.435234in}{3.271729in}}{\pgfqpoint{2.431284in}{3.273365in}}{\pgfqpoint{2.427166in}{3.273365in}}%
\pgfpathcurveto{\pgfqpoint{2.423048in}{3.273365in}}{\pgfqpoint{2.419098in}{3.271729in}}{\pgfqpoint{2.416186in}{3.268817in}}%
\pgfpathcurveto{\pgfqpoint{2.413274in}{3.265905in}}{\pgfqpoint{2.411638in}{3.261955in}}{\pgfqpoint{2.411638in}{3.257836in}}%
\pgfpathcurveto{\pgfqpoint{2.411638in}{3.253718in}}{\pgfqpoint{2.413274in}{3.249768in}}{\pgfqpoint{2.416186in}{3.246856in}}%
\pgfpathcurveto{\pgfqpoint{2.419098in}{3.243944in}}{\pgfqpoint{2.423048in}{3.242308in}}{\pgfqpoint{2.427166in}{3.242308in}}%
\pgfpathclose%
\pgfusepath{fill}%
\end{pgfscope}%
\begin{pgfscope}%
\pgfpathrectangle{\pgfqpoint{0.887500in}{0.275000in}}{\pgfqpoint{4.225000in}{4.225000in}}%
\pgfusepath{clip}%
\pgfsetbuttcap%
\pgfsetroundjoin%
\definecolor{currentfill}{rgb}{0.000000,0.000000,0.000000}%
\pgfsetfillcolor{currentfill}%
\pgfsetfillopacity{0.800000}%
\pgfsetlinewidth{0.000000pt}%
\definecolor{currentstroke}{rgb}{0.000000,0.000000,0.000000}%
\pgfsetstrokecolor{currentstroke}%
\pgfsetstrokeopacity{0.800000}%
\pgfsetdash{}{0pt}%
\pgfpathmoveto{\pgfqpoint{3.707481in}{3.864433in}}%
\pgfpathcurveto{\pgfqpoint{3.711599in}{3.864433in}}{\pgfqpoint{3.715549in}{3.866069in}}{\pgfqpoint{3.718461in}{3.868981in}}%
\pgfpathcurveto{\pgfqpoint{3.721373in}{3.871893in}}{\pgfqpoint{3.723010in}{3.875843in}}{\pgfqpoint{3.723010in}{3.879961in}}%
\pgfpathcurveto{\pgfqpoint{3.723010in}{3.884079in}}{\pgfqpoint{3.721373in}{3.888029in}}{\pgfqpoint{3.718461in}{3.890941in}}%
\pgfpathcurveto{\pgfqpoint{3.715549in}{3.893853in}}{\pgfqpoint{3.711599in}{3.895489in}}{\pgfqpoint{3.707481in}{3.895489in}}%
\pgfpathcurveto{\pgfqpoint{3.703363in}{3.895489in}}{\pgfqpoint{3.699413in}{3.893853in}}{\pgfqpoint{3.696501in}{3.890941in}}%
\pgfpathcurveto{\pgfqpoint{3.693589in}{3.888029in}}{\pgfqpoint{3.691953in}{3.884079in}}{\pgfqpoint{3.691953in}{3.879961in}}%
\pgfpathcurveto{\pgfqpoint{3.691953in}{3.875843in}}{\pgfqpoint{3.693589in}{3.871893in}}{\pgfqpoint{3.696501in}{3.868981in}}%
\pgfpathcurveto{\pgfqpoint{3.699413in}{3.866069in}}{\pgfqpoint{3.703363in}{3.864433in}}{\pgfqpoint{3.707481in}{3.864433in}}%
\pgfpathclose%
\pgfusepath{fill}%
\end{pgfscope}%
\begin{pgfscope}%
\pgfpathrectangle{\pgfqpoint{0.887500in}{0.275000in}}{\pgfqpoint{4.225000in}{4.225000in}}%
\pgfusepath{clip}%
\pgfsetbuttcap%
\pgfsetroundjoin%
\definecolor{currentfill}{rgb}{0.000000,0.000000,0.000000}%
\pgfsetfillcolor{currentfill}%
\pgfsetfillopacity{0.800000}%
\pgfsetlinewidth{0.000000pt}%
\definecolor{currentstroke}{rgb}{0.000000,0.000000,0.000000}%
\pgfsetstrokecolor{currentstroke}%
\pgfsetstrokeopacity{0.800000}%
\pgfsetdash{}{0pt}%
\pgfpathmoveto{\pgfqpoint{2.926458in}{3.567863in}}%
\pgfpathcurveto{\pgfqpoint{2.930576in}{3.567863in}}{\pgfqpoint{2.934526in}{3.569499in}}{\pgfqpoint{2.937438in}{3.572411in}}%
\pgfpathcurveto{\pgfqpoint{2.940350in}{3.575323in}}{\pgfqpoint{2.941987in}{3.579273in}}{\pgfqpoint{2.941987in}{3.583392in}}%
\pgfpathcurveto{\pgfqpoint{2.941987in}{3.587510in}}{\pgfqpoint{2.940350in}{3.591460in}}{\pgfqpoint{2.937438in}{3.594372in}}%
\pgfpathcurveto{\pgfqpoint{2.934526in}{3.597284in}}{\pgfqpoint{2.930576in}{3.598920in}}{\pgfqpoint{2.926458in}{3.598920in}}%
\pgfpathcurveto{\pgfqpoint{2.922340in}{3.598920in}}{\pgfqpoint{2.918390in}{3.597284in}}{\pgfqpoint{2.915478in}{3.594372in}}%
\pgfpathcurveto{\pgfqpoint{2.912566in}{3.591460in}}{\pgfqpoint{2.910930in}{3.587510in}}{\pgfqpoint{2.910930in}{3.583392in}}%
\pgfpathcurveto{\pgfqpoint{2.910930in}{3.579273in}}{\pgfqpoint{2.912566in}{3.575323in}}{\pgfqpoint{2.915478in}{3.572411in}}%
\pgfpathcurveto{\pgfqpoint{2.918390in}{3.569499in}}{\pgfqpoint{2.922340in}{3.567863in}}{\pgfqpoint{2.926458in}{3.567863in}}%
\pgfpathclose%
\pgfusepath{fill}%
\end{pgfscope}%
\begin{pgfscope}%
\pgfpathrectangle{\pgfqpoint{0.887500in}{0.275000in}}{\pgfqpoint{4.225000in}{4.225000in}}%
\pgfusepath{clip}%
\pgfsetbuttcap%
\pgfsetroundjoin%
\definecolor{currentfill}{rgb}{0.000000,0.000000,0.000000}%
\pgfsetfillcolor{currentfill}%
\pgfsetfillopacity{0.800000}%
\pgfsetlinewidth{0.000000pt}%
\definecolor{currentstroke}{rgb}{0.000000,0.000000,0.000000}%
\pgfsetstrokecolor{currentstroke}%
\pgfsetstrokeopacity{0.800000}%
\pgfsetdash{}{0pt}%
\pgfpathmoveto{\pgfqpoint{3.317155in}{3.727862in}}%
\pgfpathcurveto{\pgfqpoint{3.321273in}{3.727862in}}{\pgfqpoint{3.325223in}{3.729498in}}{\pgfqpoint{3.328135in}{3.732410in}}%
\pgfpathcurveto{\pgfqpoint{3.331047in}{3.735322in}}{\pgfqpoint{3.332683in}{3.739272in}}{\pgfqpoint{3.332683in}{3.743390in}}%
\pgfpathcurveto{\pgfqpoint{3.332683in}{3.747508in}}{\pgfqpoint{3.331047in}{3.751458in}}{\pgfqpoint{3.328135in}{3.754370in}}%
\pgfpathcurveto{\pgfqpoint{3.325223in}{3.757282in}}{\pgfqpoint{3.321273in}{3.758918in}}{\pgfqpoint{3.317155in}{3.758918in}}%
\pgfpathcurveto{\pgfqpoint{3.313037in}{3.758918in}}{\pgfqpoint{3.309087in}{3.757282in}}{\pgfqpoint{3.306175in}{3.754370in}}%
\pgfpathcurveto{\pgfqpoint{3.303263in}{3.751458in}}{\pgfqpoint{3.301626in}{3.747508in}}{\pgfqpoint{3.301626in}{3.743390in}}%
\pgfpathcurveto{\pgfqpoint{3.301626in}{3.739272in}}{\pgfqpoint{3.303263in}{3.735322in}}{\pgfqpoint{3.306175in}{3.732410in}}%
\pgfpathcurveto{\pgfqpoint{3.309087in}{3.729498in}}{\pgfqpoint{3.313037in}{3.727862in}}{\pgfqpoint{3.317155in}{3.727862in}}%
\pgfpathclose%
\pgfusepath{fill}%
\end{pgfscope}%
\begin{pgfscope}%
\pgfpathrectangle{\pgfqpoint{0.887500in}{0.275000in}}{\pgfqpoint{4.225000in}{4.225000in}}%
\pgfusepath{clip}%
\pgfsetbuttcap%
\pgfsetroundjoin%
\definecolor{currentfill}{rgb}{0.000000,0.000000,0.000000}%
\pgfsetfillcolor{currentfill}%
\pgfsetfillopacity{0.800000}%
\pgfsetlinewidth{0.000000pt}%
\definecolor{currentstroke}{rgb}{0.000000,0.000000,0.000000}%
\pgfsetstrokecolor{currentstroke}%
\pgfsetstrokeopacity{0.800000}%
\pgfsetdash{}{0pt}%
\pgfpathmoveto{\pgfqpoint{2.598723in}{3.197825in}}%
\pgfpathcurveto{\pgfqpoint{2.602841in}{3.197825in}}{\pgfqpoint{2.606791in}{3.199461in}}{\pgfqpoint{2.609703in}{3.202373in}}%
\pgfpathcurveto{\pgfqpoint{2.612615in}{3.205285in}}{\pgfqpoint{2.614251in}{3.209235in}}{\pgfqpoint{2.614251in}{3.213353in}}%
\pgfpathcurveto{\pgfqpoint{2.614251in}{3.217471in}}{\pgfqpoint{2.612615in}{3.221421in}}{\pgfqpoint{2.609703in}{3.224333in}}%
\pgfpathcurveto{\pgfqpoint{2.606791in}{3.227245in}}{\pgfqpoint{2.602841in}{3.228881in}}{\pgfqpoint{2.598723in}{3.228881in}}%
\pgfpathcurveto{\pgfqpoint{2.594605in}{3.228881in}}{\pgfqpoint{2.590655in}{3.227245in}}{\pgfqpoint{2.587743in}{3.224333in}}%
\pgfpathcurveto{\pgfqpoint{2.584831in}{3.221421in}}{\pgfqpoint{2.583195in}{3.217471in}}{\pgfqpoint{2.583195in}{3.213353in}}%
\pgfpathcurveto{\pgfqpoint{2.583195in}{3.209235in}}{\pgfqpoint{2.584831in}{3.205285in}}{\pgfqpoint{2.587743in}{3.202373in}}%
\pgfpathcurveto{\pgfqpoint{2.590655in}{3.199461in}}{\pgfqpoint{2.594605in}{3.197825in}}{\pgfqpoint{2.598723in}{3.197825in}}%
\pgfpathclose%
\pgfusepath{fill}%
\end{pgfscope}%
\begin{pgfscope}%
\pgfpathrectangle{\pgfqpoint{0.887500in}{0.275000in}}{\pgfqpoint{4.225000in}{4.225000in}}%
\pgfusepath{clip}%
\pgfsetbuttcap%
\pgfsetroundjoin%
\definecolor{currentfill}{rgb}{0.000000,0.000000,0.000000}%
\pgfsetfillcolor{currentfill}%
\pgfsetfillopacity{0.800000}%
\pgfsetlinewidth{0.000000pt}%
\definecolor{currentstroke}{rgb}{0.000000,0.000000,0.000000}%
\pgfsetstrokecolor{currentstroke}%
\pgfsetstrokeopacity{0.800000}%
\pgfsetdash{}{0pt}%
\pgfpathmoveto{\pgfqpoint{3.599400in}{3.819220in}}%
\pgfpathcurveto{\pgfqpoint{3.603518in}{3.819220in}}{\pgfqpoint{3.607468in}{3.820856in}}{\pgfqpoint{3.610380in}{3.823768in}}%
\pgfpathcurveto{\pgfqpoint{3.613292in}{3.826680in}}{\pgfqpoint{3.614928in}{3.830630in}}{\pgfqpoint{3.614928in}{3.834748in}}%
\pgfpathcurveto{\pgfqpoint{3.614928in}{3.838866in}}{\pgfqpoint{3.613292in}{3.842816in}}{\pgfqpoint{3.610380in}{3.845728in}}%
\pgfpathcurveto{\pgfqpoint{3.607468in}{3.848640in}}{\pgfqpoint{3.603518in}{3.850276in}}{\pgfqpoint{3.599400in}{3.850276in}}%
\pgfpathcurveto{\pgfqpoint{3.595282in}{3.850276in}}{\pgfqpoint{3.591332in}{3.848640in}}{\pgfqpoint{3.588420in}{3.845728in}}%
\pgfpathcurveto{\pgfqpoint{3.585508in}{3.842816in}}{\pgfqpoint{3.583872in}{3.838866in}}{\pgfqpoint{3.583872in}{3.834748in}}%
\pgfpathcurveto{\pgfqpoint{3.583872in}{3.830630in}}{\pgfqpoint{3.585508in}{3.826680in}}{\pgfqpoint{3.588420in}{3.823768in}}%
\pgfpathcurveto{\pgfqpoint{3.591332in}{3.820856in}}{\pgfqpoint{3.595282in}{3.819220in}}{\pgfqpoint{3.599400in}{3.819220in}}%
\pgfpathclose%
\pgfusepath{fill}%
\end{pgfscope}%
\begin{pgfscope}%
\pgfpathrectangle{\pgfqpoint{0.887500in}{0.275000in}}{\pgfqpoint{4.225000in}{4.225000in}}%
\pgfusepath{clip}%
\pgfsetbuttcap%
\pgfsetroundjoin%
\definecolor{currentfill}{rgb}{0.000000,0.000000,0.000000}%
\pgfsetfillcolor{currentfill}%
\pgfsetfillopacity{0.800000}%
\pgfsetlinewidth{0.000000pt}%
\definecolor{currentstroke}{rgb}{0.000000,0.000000,0.000000}%
\pgfsetstrokecolor{currentstroke}%
\pgfsetstrokeopacity{0.800000}%
\pgfsetdash{}{0pt}%
\pgfpathmoveto{\pgfqpoint{2.316250in}{3.214806in}}%
\pgfpathcurveto{\pgfqpoint{2.320369in}{3.214806in}}{\pgfqpoint{2.324319in}{3.216442in}}{\pgfqpoint{2.327231in}{3.219354in}}%
\pgfpathcurveto{\pgfqpoint{2.330143in}{3.222266in}}{\pgfqpoint{2.331779in}{3.226216in}}{\pgfqpoint{2.331779in}{3.230334in}}%
\pgfpathcurveto{\pgfqpoint{2.331779in}{3.234452in}}{\pgfqpoint{2.330143in}{3.238402in}}{\pgfqpoint{2.327231in}{3.241314in}}%
\pgfpathcurveto{\pgfqpoint{2.324319in}{3.244226in}}{\pgfqpoint{2.320369in}{3.245862in}}{\pgfqpoint{2.316250in}{3.245862in}}%
\pgfpathcurveto{\pgfqpoint{2.312132in}{3.245862in}}{\pgfqpoint{2.308182in}{3.244226in}}{\pgfqpoint{2.305270in}{3.241314in}}%
\pgfpathcurveto{\pgfqpoint{2.302358in}{3.238402in}}{\pgfqpoint{2.300722in}{3.234452in}}{\pgfqpoint{2.300722in}{3.230334in}}%
\pgfpathcurveto{\pgfqpoint{2.300722in}{3.226216in}}{\pgfqpoint{2.302358in}{3.222266in}}{\pgfqpoint{2.305270in}{3.219354in}}%
\pgfpathcurveto{\pgfqpoint{2.308182in}{3.216442in}}{\pgfqpoint{2.312132in}{3.214806in}}{\pgfqpoint{2.316250in}{3.214806in}}%
\pgfpathclose%
\pgfusepath{fill}%
\end{pgfscope}%
\begin{pgfscope}%
\pgfpathrectangle{\pgfqpoint{0.887500in}{0.275000in}}{\pgfqpoint{4.225000in}{4.225000in}}%
\pgfusepath{clip}%
\pgfsetbuttcap%
\pgfsetroundjoin%
\definecolor{currentfill}{rgb}{0.000000,0.000000,0.000000}%
\pgfsetfillcolor{currentfill}%
\pgfsetfillopacity{0.800000}%
\pgfsetlinewidth{0.000000pt}%
\definecolor{currentstroke}{rgb}{0.000000,0.000000,0.000000}%
\pgfsetstrokecolor{currentstroke}%
\pgfsetstrokeopacity{0.800000}%
\pgfsetdash{}{0pt}%
\pgfpathmoveto{\pgfqpoint{3.208178in}{3.665691in}}%
\pgfpathcurveto{\pgfqpoint{3.212296in}{3.665691in}}{\pgfqpoint{3.216246in}{3.667327in}}{\pgfqpoint{3.219158in}{3.670239in}}%
\pgfpathcurveto{\pgfqpoint{3.222070in}{3.673151in}}{\pgfqpoint{3.223706in}{3.677101in}}{\pgfqpoint{3.223706in}{3.681219in}}%
\pgfpathcurveto{\pgfqpoint{3.223706in}{3.685337in}}{\pgfqpoint{3.222070in}{3.689287in}}{\pgfqpoint{3.219158in}{3.692199in}}%
\pgfpathcurveto{\pgfqpoint{3.216246in}{3.695111in}}{\pgfqpoint{3.212296in}{3.696747in}}{\pgfqpoint{3.208178in}{3.696747in}}%
\pgfpathcurveto{\pgfqpoint{3.204059in}{3.696747in}}{\pgfqpoint{3.200109in}{3.695111in}}{\pgfqpoint{3.197197in}{3.692199in}}%
\pgfpathcurveto{\pgfqpoint{3.194285in}{3.689287in}}{\pgfqpoint{3.192649in}{3.685337in}}{\pgfqpoint{3.192649in}{3.681219in}}%
\pgfpathcurveto{\pgfqpoint{3.192649in}{3.677101in}}{\pgfqpoint{3.194285in}{3.673151in}}{\pgfqpoint{3.197197in}{3.670239in}}%
\pgfpathcurveto{\pgfqpoint{3.200109in}{3.667327in}}{\pgfqpoint{3.204059in}{3.665691in}}{\pgfqpoint{3.208178in}{3.665691in}}%
\pgfpathclose%
\pgfusepath{fill}%
\end{pgfscope}%
\begin{pgfscope}%
\pgfpathrectangle{\pgfqpoint{0.887500in}{0.275000in}}{\pgfqpoint{4.225000in}{4.225000in}}%
\pgfusepath{clip}%
\pgfsetbuttcap%
\pgfsetroundjoin%
\definecolor{currentfill}{rgb}{0.000000,0.000000,0.000000}%
\pgfsetfillcolor{currentfill}%
\pgfsetfillopacity{0.800000}%
\pgfsetlinewidth{0.000000pt}%
\definecolor{currentstroke}{rgb}{0.000000,0.000000,0.000000}%
\pgfsetstrokecolor{currentstroke}%
\pgfsetstrokeopacity{0.800000}%
\pgfsetdash{}{0pt}%
\pgfpathmoveto{\pgfqpoint{2.816849in}{3.508691in}}%
\pgfpathcurveto{\pgfqpoint{2.820967in}{3.508691in}}{\pgfqpoint{2.824917in}{3.510327in}}{\pgfqpoint{2.827829in}{3.513239in}}%
\pgfpathcurveto{\pgfqpoint{2.830741in}{3.516151in}}{\pgfqpoint{2.832377in}{3.520101in}}{\pgfqpoint{2.832377in}{3.524219in}}%
\pgfpathcurveto{\pgfqpoint{2.832377in}{3.528338in}}{\pgfqpoint{2.830741in}{3.532288in}}{\pgfqpoint{2.827829in}{3.535200in}}%
\pgfpathcurveto{\pgfqpoint{2.824917in}{3.538111in}}{\pgfqpoint{2.820967in}{3.539748in}}{\pgfqpoint{2.816849in}{3.539748in}}%
\pgfpathcurveto{\pgfqpoint{2.812731in}{3.539748in}}{\pgfqpoint{2.808781in}{3.538111in}}{\pgfqpoint{2.805869in}{3.535200in}}%
\pgfpathcurveto{\pgfqpoint{2.802957in}{3.532288in}}{\pgfqpoint{2.801321in}{3.528338in}}{\pgfqpoint{2.801321in}{3.524219in}}%
\pgfpathcurveto{\pgfqpoint{2.801321in}{3.520101in}}{\pgfqpoint{2.802957in}{3.516151in}}{\pgfqpoint{2.805869in}{3.513239in}}%
\pgfpathcurveto{\pgfqpoint{2.808781in}{3.510327in}}{\pgfqpoint{2.812731in}{3.508691in}}{\pgfqpoint{2.816849in}{3.508691in}}%
\pgfpathclose%
\pgfusepath{fill}%
\end{pgfscope}%
\begin{pgfscope}%
\pgfpathrectangle{\pgfqpoint{0.887500in}{0.275000in}}{\pgfqpoint{4.225000in}{4.225000in}}%
\pgfusepath{clip}%
\pgfsetbuttcap%
\pgfsetroundjoin%
\definecolor{currentfill}{rgb}{0.000000,0.000000,0.000000}%
\pgfsetfillcolor{currentfill}%
\pgfsetfillopacity{0.800000}%
\pgfsetlinewidth{0.000000pt}%
\definecolor{currentstroke}{rgb}{0.000000,0.000000,0.000000}%
\pgfsetstrokecolor{currentstroke}%
\pgfsetstrokeopacity{0.800000}%
\pgfsetdash{}{0pt}%
\pgfpathmoveto{\pgfqpoint{2.488382in}{3.149900in}}%
\pgfpathcurveto{\pgfqpoint{2.492500in}{3.149900in}}{\pgfqpoint{2.496451in}{3.151536in}}{\pgfqpoint{2.499362in}{3.154448in}}%
\pgfpathcurveto{\pgfqpoint{2.502274in}{3.157360in}}{\pgfqpoint{2.503911in}{3.161310in}}{\pgfqpoint{2.503911in}{3.165429in}}%
\pgfpathcurveto{\pgfqpoint{2.503911in}{3.169547in}}{\pgfqpoint{2.502274in}{3.173497in}}{\pgfqpoint{2.499362in}{3.176409in}}%
\pgfpathcurveto{\pgfqpoint{2.496451in}{3.179321in}}{\pgfqpoint{2.492500in}{3.180957in}}{\pgfqpoint{2.488382in}{3.180957in}}%
\pgfpathcurveto{\pgfqpoint{2.484264in}{3.180957in}}{\pgfqpoint{2.480314in}{3.179321in}}{\pgfqpoint{2.477402in}{3.176409in}}%
\pgfpathcurveto{\pgfqpoint{2.474490in}{3.173497in}}{\pgfqpoint{2.472854in}{3.169547in}}{\pgfqpoint{2.472854in}{3.165429in}}%
\pgfpathcurveto{\pgfqpoint{2.472854in}{3.161310in}}{\pgfqpoint{2.474490in}{3.157360in}}{\pgfqpoint{2.477402in}{3.154448in}}%
\pgfpathcurveto{\pgfqpoint{2.480314in}{3.151536in}}{\pgfqpoint{2.484264in}{3.149900in}}{\pgfqpoint{2.488382in}{3.149900in}}%
\pgfpathclose%
\pgfusepath{fill}%
\end{pgfscope}%
\begin{pgfscope}%
\pgfpathrectangle{\pgfqpoint{0.887500in}{0.275000in}}{\pgfqpoint{4.225000in}{4.225000in}}%
\pgfusepath{clip}%
\pgfsetbuttcap%
\pgfsetroundjoin%
\definecolor{currentfill}{rgb}{0.000000,0.000000,0.000000}%
\pgfsetfillcolor{currentfill}%
\pgfsetfillopacity{0.800000}%
\pgfsetlinewidth{0.000000pt}%
\definecolor{currentstroke}{rgb}{0.000000,0.000000,0.000000}%
\pgfsetstrokecolor{currentstroke}%
\pgfsetstrokeopacity{0.800000}%
\pgfsetdash{}{0pt}%
\pgfpathmoveto{\pgfqpoint{2.206024in}{3.128829in}}%
\pgfpathcurveto{\pgfqpoint{2.210142in}{3.128829in}}{\pgfqpoint{2.214092in}{3.130465in}}{\pgfqpoint{2.217004in}{3.133377in}}%
\pgfpathcurveto{\pgfqpoint{2.219916in}{3.136289in}}{\pgfqpoint{2.221552in}{3.140239in}}{\pgfqpoint{2.221552in}{3.144358in}}%
\pgfpathcurveto{\pgfqpoint{2.221552in}{3.148476in}}{\pgfqpoint{2.219916in}{3.152426in}}{\pgfqpoint{2.217004in}{3.155338in}}%
\pgfpathcurveto{\pgfqpoint{2.214092in}{3.158250in}}{\pgfqpoint{2.210142in}{3.159886in}}{\pgfqpoint{2.206024in}{3.159886in}}%
\pgfpathcurveto{\pgfqpoint{2.201906in}{3.159886in}}{\pgfqpoint{2.197956in}{3.158250in}}{\pgfqpoint{2.195044in}{3.155338in}}%
\pgfpathcurveto{\pgfqpoint{2.192132in}{3.152426in}}{\pgfqpoint{2.190496in}{3.148476in}}{\pgfqpoint{2.190496in}{3.144358in}}%
\pgfpathcurveto{\pgfqpoint{2.190496in}{3.140239in}}{\pgfqpoint{2.192132in}{3.136289in}}{\pgfqpoint{2.195044in}{3.133377in}}%
\pgfpathcurveto{\pgfqpoint{2.197956in}{3.130465in}}{\pgfqpoint{2.201906in}{3.128829in}}{\pgfqpoint{2.206024in}{3.128829in}}%
\pgfpathclose%
\pgfusepath{fill}%
\end{pgfscope}%
\begin{pgfscope}%
\pgfpathrectangle{\pgfqpoint{0.887500in}{0.275000in}}{\pgfqpoint{4.225000in}{4.225000in}}%
\pgfusepath{clip}%
\pgfsetbuttcap%
\pgfsetroundjoin%
\definecolor{currentfill}{rgb}{0.000000,0.000000,0.000000}%
\pgfsetfillcolor{currentfill}%
\pgfsetfillopacity{0.800000}%
\pgfsetlinewidth{0.000000pt}%
\definecolor{currentstroke}{rgb}{0.000000,0.000000,0.000000}%
\pgfsetstrokecolor{currentstroke}%
\pgfsetstrokeopacity{0.800000}%
\pgfsetdash{}{0pt}%
\pgfpathmoveto{\pgfqpoint{3.490840in}{3.769085in}}%
\pgfpathcurveto{\pgfqpoint{3.494958in}{3.769085in}}{\pgfqpoint{3.498908in}{3.770721in}}{\pgfqpoint{3.501820in}{3.773633in}}%
\pgfpathcurveto{\pgfqpoint{3.504732in}{3.776545in}}{\pgfqpoint{3.506368in}{3.780495in}}{\pgfqpoint{3.506368in}{3.784613in}}%
\pgfpathcurveto{\pgfqpoint{3.506368in}{3.788731in}}{\pgfqpoint{3.504732in}{3.792681in}}{\pgfqpoint{3.501820in}{3.795593in}}%
\pgfpathcurveto{\pgfqpoint{3.498908in}{3.798505in}}{\pgfqpoint{3.494958in}{3.800141in}}{\pgfqpoint{3.490840in}{3.800141in}}%
\pgfpathcurveto{\pgfqpoint{3.486722in}{3.800141in}}{\pgfqpoint{3.482772in}{3.798505in}}{\pgfqpoint{3.479860in}{3.795593in}}%
\pgfpathcurveto{\pgfqpoint{3.476948in}{3.792681in}}{\pgfqpoint{3.475312in}{3.788731in}}{\pgfqpoint{3.475312in}{3.784613in}}%
\pgfpathcurveto{\pgfqpoint{3.475312in}{3.780495in}}{\pgfqpoint{3.476948in}{3.776545in}}{\pgfqpoint{3.479860in}{3.773633in}}%
\pgfpathcurveto{\pgfqpoint{3.482772in}{3.770721in}}{\pgfqpoint{3.486722in}{3.769085in}}{\pgfqpoint{3.490840in}{3.769085in}}%
\pgfpathclose%
\pgfusepath{fill}%
\end{pgfscope}%
\begin{pgfscope}%
\pgfpathrectangle{\pgfqpoint{0.887500in}{0.275000in}}{\pgfqpoint{4.225000in}{4.225000in}}%
\pgfusepath{clip}%
\pgfsetbuttcap%
\pgfsetroundjoin%
\definecolor{currentfill}{rgb}{0.000000,0.000000,0.000000}%
\pgfsetfillcolor{currentfill}%
\pgfsetfillopacity{0.800000}%
\pgfsetlinewidth{0.000000pt}%
\definecolor{currentstroke}{rgb}{0.000000,0.000000,0.000000}%
\pgfsetstrokecolor{currentstroke}%
\pgfsetstrokeopacity{0.800000}%
\pgfsetdash{}{0pt}%
\pgfpathmoveto{\pgfqpoint{3.098863in}{3.608734in}}%
\pgfpathcurveto{\pgfqpoint{3.102981in}{3.608734in}}{\pgfqpoint{3.106931in}{3.610370in}}{\pgfqpoint{3.109843in}{3.613282in}}%
\pgfpathcurveto{\pgfqpoint{3.112755in}{3.616194in}}{\pgfqpoint{3.114391in}{3.620144in}}{\pgfqpoint{3.114391in}{3.624262in}}%
\pgfpathcurveto{\pgfqpoint{3.114391in}{3.628380in}}{\pgfqpoint{3.112755in}{3.632330in}}{\pgfqpoint{3.109843in}{3.635242in}}%
\pgfpathcurveto{\pgfqpoint{3.106931in}{3.638154in}}{\pgfqpoint{3.102981in}{3.639790in}}{\pgfqpoint{3.098863in}{3.639790in}}%
\pgfpathcurveto{\pgfqpoint{3.094745in}{3.639790in}}{\pgfqpoint{3.090795in}{3.638154in}}{\pgfqpoint{3.087883in}{3.635242in}}%
\pgfpathcurveto{\pgfqpoint{3.084971in}{3.632330in}}{\pgfqpoint{3.083335in}{3.628380in}}{\pgfqpoint{3.083335in}{3.624262in}}%
\pgfpathcurveto{\pgfqpoint{3.083335in}{3.620144in}}{\pgfqpoint{3.084971in}{3.616194in}}{\pgfqpoint{3.087883in}{3.613282in}}%
\pgfpathcurveto{\pgfqpoint{3.090795in}{3.610370in}}{\pgfqpoint{3.094745in}{3.608734in}}{\pgfqpoint{3.098863in}{3.608734in}}%
\pgfpathclose%
\pgfusepath{fill}%
\end{pgfscope}%
\begin{pgfscope}%
\pgfpathrectangle{\pgfqpoint{0.887500in}{0.275000in}}{\pgfqpoint{4.225000in}{4.225000in}}%
\pgfusepath{clip}%
\pgfsetbuttcap%
\pgfsetroundjoin%
\definecolor{currentfill}{rgb}{0.000000,0.000000,0.000000}%
\pgfsetfillcolor{currentfill}%
\pgfsetfillopacity{0.800000}%
\pgfsetlinewidth{0.000000pt}%
\definecolor{currentstroke}{rgb}{0.000000,0.000000,0.000000}%
\pgfsetstrokecolor{currentstroke}%
\pgfsetstrokeopacity{0.800000}%
\pgfsetdash{}{0pt}%
\pgfpathmoveto{\pgfqpoint{3.773467in}{3.815682in}}%
\pgfpathcurveto{\pgfqpoint{3.777585in}{3.815682in}}{\pgfqpoint{3.781535in}{3.817318in}}{\pgfqpoint{3.784447in}{3.820230in}}%
\pgfpathcurveto{\pgfqpoint{3.787359in}{3.823142in}}{\pgfqpoint{3.788995in}{3.827092in}}{\pgfqpoint{3.788995in}{3.831210in}}%
\pgfpathcurveto{\pgfqpoint{3.788995in}{3.835329in}}{\pgfqpoint{3.787359in}{3.839279in}}{\pgfqpoint{3.784447in}{3.842191in}}%
\pgfpathcurveto{\pgfqpoint{3.781535in}{3.845102in}}{\pgfqpoint{3.777585in}{3.846739in}}{\pgfqpoint{3.773467in}{3.846739in}}%
\pgfpathcurveto{\pgfqpoint{3.769349in}{3.846739in}}{\pgfqpoint{3.765399in}{3.845102in}}{\pgfqpoint{3.762487in}{3.842191in}}%
\pgfpathcurveto{\pgfqpoint{3.759575in}{3.839279in}}{\pgfqpoint{3.757939in}{3.835329in}}{\pgfqpoint{3.757939in}{3.831210in}}%
\pgfpathcurveto{\pgfqpoint{3.757939in}{3.827092in}}{\pgfqpoint{3.759575in}{3.823142in}}{\pgfqpoint{3.762487in}{3.820230in}}%
\pgfpathcurveto{\pgfqpoint{3.765399in}{3.817318in}}{\pgfqpoint{3.769349in}{3.815682in}}{\pgfqpoint{3.773467in}{3.815682in}}%
\pgfpathclose%
\pgfusepath{fill}%
\end{pgfscope}%
\begin{pgfscope}%
\pgfpathrectangle{\pgfqpoint{0.887500in}{0.275000in}}{\pgfqpoint{4.225000in}{4.225000in}}%
\pgfusepath{clip}%
\pgfsetbuttcap%
\pgfsetroundjoin%
\definecolor{currentfill}{rgb}{0.000000,0.000000,0.000000}%
\pgfsetfillcolor{currentfill}%
\pgfsetfillopacity{0.800000}%
\pgfsetlinewidth{0.000000pt}%
\definecolor{currentstroke}{rgb}{0.000000,0.000000,0.000000}%
\pgfsetstrokecolor{currentstroke}%
\pgfsetstrokeopacity{0.800000}%
\pgfsetdash{}{0pt}%
\pgfpathmoveto{\pgfqpoint{2.094919in}{3.070142in}}%
\pgfpathcurveto{\pgfqpoint{2.099038in}{3.070142in}}{\pgfqpoint{2.102988in}{3.071778in}}{\pgfqpoint{2.105900in}{3.074690in}}%
\pgfpathcurveto{\pgfqpoint{2.108812in}{3.077602in}}{\pgfqpoint{2.110448in}{3.081552in}}{\pgfqpoint{2.110448in}{3.085670in}}%
\pgfpathcurveto{\pgfqpoint{2.110448in}{3.089788in}}{\pgfqpoint{2.108812in}{3.093738in}}{\pgfqpoint{2.105900in}{3.096650in}}%
\pgfpathcurveto{\pgfqpoint{2.102988in}{3.099562in}}{\pgfqpoint{2.099038in}{3.101198in}}{\pgfqpoint{2.094919in}{3.101198in}}%
\pgfpathcurveto{\pgfqpoint{2.090801in}{3.101198in}}{\pgfqpoint{2.086851in}{3.099562in}}{\pgfqpoint{2.083939in}{3.096650in}}%
\pgfpathcurveto{\pgfqpoint{2.081027in}{3.093738in}}{\pgfqpoint{2.079391in}{3.089788in}}{\pgfqpoint{2.079391in}{3.085670in}}%
\pgfpathcurveto{\pgfqpoint{2.079391in}{3.081552in}}{\pgfqpoint{2.081027in}{3.077602in}}{\pgfqpoint{2.083939in}{3.074690in}}%
\pgfpathcurveto{\pgfqpoint{2.086851in}{3.071778in}}{\pgfqpoint{2.090801in}{3.070142in}}{\pgfqpoint{2.094919in}{3.070142in}}%
\pgfpathclose%
\pgfusepath{fill}%
\end{pgfscope}%
\begin{pgfscope}%
\pgfpathrectangle{\pgfqpoint{0.887500in}{0.275000in}}{\pgfqpoint{4.225000in}{4.225000in}}%
\pgfusepath{clip}%
\pgfsetbuttcap%
\pgfsetroundjoin%
\definecolor{currentfill}{rgb}{0.000000,0.000000,0.000000}%
\pgfsetfillcolor{currentfill}%
\pgfsetfillopacity{0.800000}%
\pgfsetlinewidth{0.000000pt}%
\definecolor{currentstroke}{rgb}{0.000000,0.000000,0.000000}%
\pgfsetstrokecolor{currentstroke}%
\pgfsetstrokeopacity{0.800000}%
\pgfsetdash{}{0pt}%
\pgfpathmoveto{\pgfqpoint{2.377329in}{3.118517in}}%
\pgfpathcurveto{\pgfqpoint{2.381447in}{3.118517in}}{\pgfqpoint{2.385397in}{3.120153in}}{\pgfqpoint{2.388309in}{3.123065in}}%
\pgfpathcurveto{\pgfqpoint{2.391221in}{3.125977in}}{\pgfqpoint{2.392857in}{3.129927in}}{\pgfqpoint{2.392857in}{3.134045in}}%
\pgfpathcurveto{\pgfqpoint{2.392857in}{3.138163in}}{\pgfqpoint{2.391221in}{3.142113in}}{\pgfqpoint{2.388309in}{3.145025in}}%
\pgfpathcurveto{\pgfqpoint{2.385397in}{3.147937in}}{\pgfqpoint{2.381447in}{3.149573in}}{\pgfqpoint{2.377329in}{3.149573in}}%
\pgfpathcurveto{\pgfqpoint{2.373211in}{3.149573in}}{\pgfqpoint{2.369261in}{3.147937in}}{\pgfqpoint{2.366349in}{3.145025in}}%
\pgfpathcurveto{\pgfqpoint{2.363437in}{3.142113in}}{\pgfqpoint{2.361801in}{3.138163in}}{\pgfqpoint{2.361801in}{3.134045in}}%
\pgfpathcurveto{\pgfqpoint{2.361801in}{3.129927in}}{\pgfqpoint{2.363437in}{3.125977in}}{\pgfqpoint{2.366349in}{3.123065in}}%
\pgfpathcurveto{\pgfqpoint{2.369261in}{3.120153in}}{\pgfqpoint{2.373211in}{3.118517in}}{\pgfqpoint{2.377329in}{3.118517in}}%
\pgfpathclose%
\pgfusepath{fill}%
\end{pgfscope}%
\begin{pgfscope}%
\pgfpathrectangle{\pgfqpoint{0.887500in}{0.275000in}}{\pgfqpoint{4.225000in}{4.225000in}}%
\pgfusepath{clip}%
\pgfsetbuttcap%
\pgfsetroundjoin%
\definecolor{currentfill}{rgb}{0.000000,0.000000,0.000000}%
\pgfsetfillcolor{currentfill}%
\pgfsetfillopacity{0.800000}%
\pgfsetlinewidth{0.000000pt}%
\definecolor{currentstroke}{rgb}{0.000000,0.000000,0.000000}%
\pgfsetstrokecolor{currentstroke}%
\pgfsetstrokeopacity{0.800000}%
\pgfsetdash{}{0pt}%
\pgfpathmoveto{\pgfqpoint{3.381811in}{3.710828in}}%
\pgfpathcurveto{\pgfqpoint{3.385929in}{3.710828in}}{\pgfqpoint{3.389879in}{3.712464in}}{\pgfqpoint{3.392791in}{3.715376in}}%
\pgfpathcurveto{\pgfqpoint{3.395703in}{3.718288in}}{\pgfqpoint{3.397339in}{3.722238in}}{\pgfqpoint{3.397339in}{3.726356in}}%
\pgfpathcurveto{\pgfqpoint{3.397339in}{3.730474in}}{\pgfqpoint{3.395703in}{3.734424in}}{\pgfqpoint{3.392791in}{3.737336in}}%
\pgfpathcurveto{\pgfqpoint{3.389879in}{3.740248in}}{\pgfqpoint{3.385929in}{3.741884in}}{\pgfqpoint{3.381811in}{3.741884in}}%
\pgfpathcurveto{\pgfqpoint{3.377693in}{3.741884in}}{\pgfqpoint{3.373743in}{3.740248in}}{\pgfqpoint{3.370831in}{3.737336in}}%
\pgfpathcurveto{\pgfqpoint{3.367919in}{3.734424in}}{\pgfqpoint{3.366283in}{3.730474in}}{\pgfqpoint{3.366283in}{3.726356in}}%
\pgfpathcurveto{\pgfqpoint{3.366283in}{3.722238in}}{\pgfqpoint{3.367919in}{3.718288in}}{\pgfqpoint{3.370831in}{3.715376in}}%
\pgfpathcurveto{\pgfqpoint{3.373743in}{3.712464in}}{\pgfqpoint{3.377693in}{3.710828in}}{\pgfqpoint{3.381811in}{3.710828in}}%
\pgfpathclose%
\pgfusepath{fill}%
\end{pgfscope}%
\begin{pgfscope}%
\pgfpathrectangle{\pgfqpoint{0.887500in}{0.275000in}}{\pgfqpoint{4.225000in}{4.225000in}}%
\pgfusepath{clip}%
\pgfsetbuttcap%
\pgfsetroundjoin%
\definecolor{currentfill}{rgb}{0.000000,0.000000,0.000000}%
\pgfsetfillcolor{currentfill}%
\pgfsetfillopacity{0.800000}%
\pgfsetlinewidth{0.000000pt}%
\definecolor{currentstroke}{rgb}{0.000000,0.000000,0.000000}%
\pgfsetstrokecolor{currentstroke}%
\pgfsetstrokeopacity{0.800000}%
\pgfsetdash{}{0pt}%
\pgfpathmoveto{\pgfqpoint{2.989170in}{3.554589in}}%
\pgfpathcurveto{\pgfqpoint{2.993289in}{3.554589in}}{\pgfqpoint{2.997239in}{3.556225in}}{\pgfqpoint{3.000150in}{3.559137in}}%
\pgfpathcurveto{\pgfqpoint{3.003062in}{3.562049in}}{\pgfqpoint{3.004699in}{3.565999in}}{\pgfqpoint{3.004699in}{3.570118in}}%
\pgfpathcurveto{\pgfqpoint{3.004699in}{3.574236in}}{\pgfqpoint{3.003062in}{3.578186in}}{\pgfqpoint{3.000150in}{3.581098in}}%
\pgfpathcurveto{\pgfqpoint{2.997239in}{3.584010in}}{\pgfqpoint{2.993289in}{3.585646in}}{\pgfqpoint{2.989170in}{3.585646in}}%
\pgfpathcurveto{\pgfqpoint{2.985052in}{3.585646in}}{\pgfqpoint{2.981102in}{3.584010in}}{\pgfqpoint{2.978190in}{3.581098in}}%
\pgfpathcurveto{\pgfqpoint{2.975278in}{3.578186in}}{\pgfqpoint{2.973642in}{3.574236in}}{\pgfqpoint{2.973642in}{3.570118in}}%
\pgfpathcurveto{\pgfqpoint{2.973642in}{3.565999in}}{\pgfqpoint{2.975278in}{3.562049in}}{\pgfqpoint{2.978190in}{3.559137in}}%
\pgfpathcurveto{\pgfqpoint{2.981102in}{3.556225in}}{\pgfqpoint{2.985052in}{3.554589in}}{\pgfqpoint{2.989170in}{3.554589in}}%
\pgfpathclose%
\pgfusepath{fill}%
\end{pgfscope}%
\begin{pgfscope}%
\pgfpathrectangle{\pgfqpoint{0.887500in}{0.275000in}}{\pgfqpoint{4.225000in}{4.225000in}}%
\pgfusepath{clip}%
\pgfsetbuttcap%
\pgfsetroundjoin%
\definecolor{currentfill}{rgb}{0.000000,0.000000,0.000000}%
\pgfsetfillcolor{currentfill}%
\pgfsetfillopacity{0.800000}%
\pgfsetlinewidth{0.000000pt}%
\definecolor{currentstroke}{rgb}{0.000000,0.000000,0.000000}%
\pgfsetstrokecolor{currentstroke}%
\pgfsetstrokeopacity{0.800000}%
\pgfsetdash{}{0pt}%
\pgfpathmoveto{\pgfqpoint{1.984253in}{2.980823in}}%
\pgfpathcurveto{\pgfqpoint{1.988371in}{2.980823in}}{\pgfqpoint{1.992321in}{2.982459in}}{\pgfqpoint{1.995233in}{2.985371in}}%
\pgfpathcurveto{\pgfqpoint{1.998145in}{2.988283in}}{\pgfqpoint{1.999781in}{2.992233in}}{\pgfqpoint{1.999781in}{2.996351in}}%
\pgfpathcurveto{\pgfqpoint{1.999781in}{3.000469in}}{\pgfqpoint{1.998145in}{3.004419in}}{\pgfqpoint{1.995233in}{3.007331in}}%
\pgfpathcurveto{\pgfqpoint{1.992321in}{3.010243in}}{\pgfqpoint{1.988371in}{3.011880in}}{\pgfqpoint{1.984253in}{3.011880in}}%
\pgfpathcurveto{\pgfqpoint{1.980135in}{3.011880in}}{\pgfqpoint{1.976185in}{3.010243in}}{\pgfqpoint{1.973273in}{3.007331in}}%
\pgfpathcurveto{\pgfqpoint{1.970361in}{3.004419in}}{\pgfqpoint{1.968725in}{3.000469in}}{\pgfqpoint{1.968725in}{2.996351in}}%
\pgfpathcurveto{\pgfqpoint{1.968725in}{2.992233in}}{\pgfqpoint{1.970361in}{2.988283in}}{\pgfqpoint{1.973273in}{2.985371in}}%
\pgfpathcurveto{\pgfqpoint{1.976185in}{2.982459in}}{\pgfqpoint{1.980135in}{2.980823in}}{\pgfqpoint{1.984253in}{2.980823in}}%
\pgfpathclose%
\pgfusepath{fill}%
\end{pgfscope}%
\begin{pgfscope}%
\pgfpathrectangle{\pgfqpoint{0.887500in}{0.275000in}}{\pgfqpoint{4.225000in}{4.225000in}}%
\pgfusepath{clip}%
\pgfsetbuttcap%
\pgfsetroundjoin%
\definecolor{currentfill}{rgb}{0.000000,0.000000,0.000000}%
\pgfsetfillcolor{currentfill}%
\pgfsetfillopacity{0.800000}%
\pgfsetlinewidth{0.000000pt}%
\definecolor{currentstroke}{rgb}{0.000000,0.000000,0.000000}%
\pgfsetstrokecolor{currentstroke}%
\pgfsetstrokeopacity{0.800000}%
\pgfsetdash{}{0pt}%
\pgfpathmoveto{\pgfqpoint{3.665065in}{3.770097in}}%
\pgfpathcurveto{\pgfqpoint{3.669183in}{3.770097in}}{\pgfqpoint{3.673133in}{3.771734in}}{\pgfqpoint{3.676045in}{3.774645in}}%
\pgfpathcurveto{\pgfqpoint{3.678957in}{3.777557in}}{\pgfqpoint{3.680593in}{3.781507in}}{\pgfqpoint{3.680593in}{3.785626in}}%
\pgfpathcurveto{\pgfqpoint{3.680593in}{3.789744in}}{\pgfqpoint{3.678957in}{3.793694in}}{\pgfqpoint{3.676045in}{3.796606in}}%
\pgfpathcurveto{\pgfqpoint{3.673133in}{3.799518in}}{\pgfqpoint{3.669183in}{3.801154in}}{\pgfqpoint{3.665065in}{3.801154in}}%
\pgfpathcurveto{\pgfqpoint{3.660947in}{3.801154in}}{\pgfqpoint{3.656997in}{3.799518in}}{\pgfqpoint{3.654085in}{3.796606in}}%
\pgfpathcurveto{\pgfqpoint{3.651173in}{3.793694in}}{\pgfqpoint{3.649537in}{3.789744in}}{\pgfqpoint{3.649537in}{3.785626in}}%
\pgfpathcurveto{\pgfqpoint{3.649537in}{3.781507in}}{\pgfqpoint{3.651173in}{3.777557in}}{\pgfqpoint{3.654085in}{3.774645in}}%
\pgfpathcurveto{\pgfqpoint{3.656997in}{3.771734in}}{\pgfqpoint{3.660947in}{3.770097in}}{\pgfqpoint{3.665065in}{3.770097in}}%
\pgfpathclose%
\pgfusepath{fill}%
\end{pgfscope}%
\begin{pgfscope}%
\pgfpathrectangle{\pgfqpoint{0.887500in}{0.275000in}}{\pgfqpoint{4.225000in}{4.225000in}}%
\pgfusepath{clip}%
\pgfsetbuttcap%
\pgfsetroundjoin%
\definecolor{currentfill}{rgb}{0.000000,0.000000,0.000000}%
\pgfsetfillcolor{currentfill}%
\pgfsetfillopacity{0.800000}%
\pgfsetlinewidth{0.000000pt}%
\definecolor{currentstroke}{rgb}{0.000000,0.000000,0.000000}%
\pgfsetstrokecolor{currentstroke}%
\pgfsetstrokeopacity{0.800000}%
\pgfsetdash{}{0pt}%
\pgfpathmoveto{\pgfqpoint{2.549739in}{3.076274in}}%
\pgfpathcurveto{\pgfqpoint{2.553857in}{3.076274in}}{\pgfqpoint{2.557807in}{3.077910in}}{\pgfqpoint{2.560719in}{3.080822in}}%
\pgfpathcurveto{\pgfqpoint{2.563631in}{3.083734in}}{\pgfqpoint{2.565268in}{3.087684in}}{\pgfqpoint{2.565268in}{3.091802in}}%
\pgfpathcurveto{\pgfqpoint{2.565268in}{3.095921in}}{\pgfqpoint{2.563631in}{3.099871in}}{\pgfqpoint{2.560719in}{3.102783in}}%
\pgfpathcurveto{\pgfqpoint{2.557807in}{3.105695in}}{\pgfqpoint{2.553857in}{3.107331in}}{\pgfqpoint{2.549739in}{3.107331in}}%
\pgfpathcurveto{\pgfqpoint{2.545621in}{3.107331in}}{\pgfqpoint{2.541671in}{3.105695in}}{\pgfqpoint{2.538759in}{3.102783in}}%
\pgfpathcurveto{\pgfqpoint{2.535847in}{3.099871in}}{\pgfqpoint{2.534211in}{3.095921in}}{\pgfqpoint{2.534211in}{3.091802in}}%
\pgfpathcurveto{\pgfqpoint{2.534211in}{3.087684in}}{\pgfqpoint{2.535847in}{3.083734in}}{\pgfqpoint{2.538759in}{3.080822in}}%
\pgfpathcurveto{\pgfqpoint{2.541671in}{3.077910in}}{\pgfqpoint{2.545621in}{3.076274in}}{\pgfqpoint{2.549739in}{3.076274in}}%
\pgfpathclose%
\pgfusepath{fill}%
\end{pgfscope}%
\begin{pgfscope}%
\pgfpathrectangle{\pgfqpoint{0.887500in}{0.275000in}}{\pgfqpoint{4.225000in}{4.225000in}}%
\pgfusepath{clip}%
\pgfsetbuttcap%
\pgfsetroundjoin%
\definecolor{currentfill}{rgb}{0.000000,0.000000,0.000000}%
\pgfsetfillcolor{currentfill}%
\pgfsetfillopacity{0.800000}%
\pgfsetlinewidth{0.000000pt}%
\definecolor{currentstroke}{rgb}{0.000000,0.000000,0.000000}%
\pgfsetstrokecolor{currentstroke}%
\pgfsetstrokeopacity{0.800000}%
\pgfsetdash{}{0pt}%
\pgfpathmoveto{\pgfqpoint{2.266565in}{3.045507in}}%
\pgfpathcurveto{\pgfqpoint{2.270683in}{3.045507in}}{\pgfqpoint{2.274633in}{3.047143in}}{\pgfqpoint{2.277545in}{3.050055in}}%
\pgfpathcurveto{\pgfqpoint{2.280457in}{3.052967in}}{\pgfqpoint{2.282094in}{3.056917in}}{\pgfqpoint{2.282094in}{3.061035in}}%
\pgfpathcurveto{\pgfqpoint{2.282094in}{3.065153in}}{\pgfqpoint{2.280457in}{3.069103in}}{\pgfqpoint{2.277545in}{3.072015in}}%
\pgfpathcurveto{\pgfqpoint{2.274633in}{3.074927in}}{\pgfqpoint{2.270683in}{3.076563in}}{\pgfqpoint{2.266565in}{3.076563in}}%
\pgfpathcurveto{\pgfqpoint{2.262447in}{3.076563in}}{\pgfqpoint{2.258497in}{3.074927in}}{\pgfqpoint{2.255585in}{3.072015in}}%
\pgfpathcurveto{\pgfqpoint{2.252673in}{3.069103in}}{\pgfqpoint{2.251037in}{3.065153in}}{\pgfqpoint{2.251037in}{3.061035in}}%
\pgfpathcurveto{\pgfqpoint{2.251037in}{3.056917in}}{\pgfqpoint{2.252673in}{3.052967in}}{\pgfqpoint{2.255585in}{3.050055in}}%
\pgfpathcurveto{\pgfqpoint{2.258497in}{3.047143in}}{\pgfqpoint{2.262447in}{3.045507in}}{\pgfqpoint{2.266565in}{3.045507in}}%
\pgfpathclose%
\pgfusepath{fill}%
\end{pgfscope}%
\begin{pgfscope}%
\pgfpathrectangle{\pgfqpoint{0.887500in}{0.275000in}}{\pgfqpoint{4.225000in}{4.225000in}}%
\pgfusepath{clip}%
\pgfsetbuttcap%
\pgfsetroundjoin%
\definecolor{currentfill}{rgb}{0.000000,0.000000,0.000000}%
\pgfsetfillcolor{currentfill}%
\pgfsetfillopacity{0.800000}%
\pgfsetlinewidth{0.000000pt}%
\definecolor{currentstroke}{rgb}{0.000000,0.000000,0.000000}%
\pgfsetstrokecolor{currentstroke}%
\pgfsetstrokeopacity{0.800000}%
\pgfsetdash{}{0pt}%
\pgfpathmoveto{\pgfqpoint{3.272399in}{3.649097in}}%
\pgfpathcurveto{\pgfqpoint{3.276517in}{3.649097in}}{\pgfqpoint{3.280467in}{3.650734in}}{\pgfqpoint{3.283379in}{3.653646in}}%
\pgfpathcurveto{\pgfqpoint{3.286291in}{3.656558in}}{\pgfqpoint{3.287927in}{3.660508in}}{\pgfqpoint{3.287927in}{3.664626in}}%
\pgfpathcurveto{\pgfqpoint{3.287927in}{3.668744in}}{\pgfqpoint{3.286291in}{3.672694in}}{\pgfqpoint{3.283379in}{3.675606in}}%
\pgfpathcurveto{\pgfqpoint{3.280467in}{3.678518in}}{\pgfqpoint{3.276517in}{3.680154in}}{\pgfqpoint{3.272399in}{3.680154in}}%
\pgfpathcurveto{\pgfqpoint{3.268281in}{3.680154in}}{\pgfqpoint{3.264331in}{3.678518in}}{\pgfqpoint{3.261419in}{3.675606in}}%
\pgfpathcurveto{\pgfqpoint{3.258507in}{3.672694in}}{\pgfqpoint{3.256871in}{3.668744in}}{\pgfqpoint{3.256871in}{3.664626in}}%
\pgfpathcurveto{\pgfqpoint{3.256871in}{3.660508in}}{\pgfqpoint{3.258507in}{3.656558in}}{\pgfqpoint{3.261419in}{3.653646in}}%
\pgfpathcurveto{\pgfqpoint{3.264331in}{3.650734in}}{\pgfqpoint{3.268281in}{3.649097in}}{\pgfqpoint{3.272399in}{3.649097in}}%
\pgfpathclose%
\pgfusepath{fill}%
\end{pgfscope}%
\begin{pgfscope}%
\pgfpathrectangle{\pgfqpoint{0.887500in}{0.275000in}}{\pgfqpoint{4.225000in}{4.225000in}}%
\pgfusepath{clip}%
\pgfsetbuttcap%
\pgfsetroundjoin%
\definecolor{currentfill}{rgb}{0.000000,0.000000,0.000000}%
\pgfsetfillcolor{currentfill}%
\pgfsetfillopacity{0.800000}%
\pgfsetlinewidth{0.000000pt}%
\definecolor{currentstroke}{rgb}{0.000000,0.000000,0.000000}%
\pgfsetstrokecolor{currentstroke}%
\pgfsetstrokeopacity{0.800000}%
\pgfsetdash{}{0pt}%
\pgfpathmoveto{\pgfqpoint{2.879085in}{3.500423in}}%
\pgfpathcurveto{\pgfqpoint{2.883203in}{3.500423in}}{\pgfqpoint{2.887153in}{3.502059in}}{\pgfqpoint{2.890065in}{3.504971in}}%
\pgfpathcurveto{\pgfqpoint{2.892977in}{3.507883in}}{\pgfqpoint{2.894613in}{3.511833in}}{\pgfqpoint{2.894613in}{3.515951in}}%
\pgfpathcurveto{\pgfqpoint{2.894613in}{3.520070in}}{\pgfqpoint{2.892977in}{3.524020in}}{\pgfqpoint{2.890065in}{3.526932in}}%
\pgfpathcurveto{\pgfqpoint{2.887153in}{3.529844in}}{\pgfqpoint{2.883203in}{3.531480in}}{\pgfqpoint{2.879085in}{3.531480in}}%
\pgfpathcurveto{\pgfqpoint{2.874967in}{3.531480in}}{\pgfqpoint{2.871017in}{3.529844in}}{\pgfqpoint{2.868105in}{3.526932in}}%
\pgfpathcurveto{\pgfqpoint{2.865193in}{3.524020in}}{\pgfqpoint{2.863557in}{3.520070in}}{\pgfqpoint{2.863557in}{3.515951in}}%
\pgfpathcurveto{\pgfqpoint{2.863557in}{3.511833in}}{\pgfqpoint{2.865193in}{3.507883in}}{\pgfqpoint{2.868105in}{3.504971in}}%
\pgfpathcurveto{\pgfqpoint{2.871017in}{3.502059in}}{\pgfqpoint{2.874967in}{3.500423in}}{\pgfqpoint{2.879085in}{3.500423in}}%
\pgfpathclose%
\pgfusepath{fill}%
\end{pgfscope}%
\begin{pgfscope}%
\pgfpathrectangle{\pgfqpoint{0.887500in}{0.275000in}}{\pgfqpoint{4.225000in}{4.225000in}}%
\pgfusepath{clip}%
\pgfsetbuttcap%
\pgfsetroundjoin%
\definecolor{currentfill}{rgb}{0.000000,0.000000,0.000000}%
\pgfsetfillcolor{currentfill}%
\pgfsetfillopacity{0.800000}%
\pgfsetlinewidth{0.000000pt}%
\definecolor{currentstroke}{rgb}{0.000000,0.000000,0.000000}%
\pgfsetstrokecolor{currentstroke}%
\pgfsetstrokeopacity{0.800000}%
\pgfsetdash{}{0pt}%
\pgfpathmoveto{\pgfqpoint{2.155627in}{2.966870in}}%
\pgfpathcurveto{\pgfqpoint{2.159745in}{2.966870in}}{\pgfqpoint{2.163695in}{2.968506in}}{\pgfqpoint{2.166607in}{2.971418in}}%
\pgfpathcurveto{\pgfqpoint{2.169519in}{2.974330in}}{\pgfqpoint{2.171155in}{2.978280in}}{\pgfqpoint{2.171155in}{2.982398in}}%
\pgfpathcurveto{\pgfqpoint{2.171155in}{2.986517in}}{\pgfqpoint{2.169519in}{2.990467in}}{\pgfqpoint{2.166607in}{2.993379in}}%
\pgfpathcurveto{\pgfqpoint{2.163695in}{2.996291in}}{\pgfqpoint{2.159745in}{2.997927in}}{\pgfqpoint{2.155627in}{2.997927in}}%
\pgfpathcurveto{\pgfqpoint{2.151509in}{2.997927in}}{\pgfqpoint{2.147559in}{2.996291in}}{\pgfqpoint{2.144647in}{2.993379in}}%
\pgfpathcurveto{\pgfqpoint{2.141735in}{2.990467in}}{\pgfqpoint{2.140099in}{2.986517in}}{\pgfqpoint{2.140099in}{2.982398in}}%
\pgfpathcurveto{\pgfqpoint{2.140099in}{2.978280in}}{\pgfqpoint{2.141735in}{2.974330in}}{\pgfqpoint{2.144647in}{2.971418in}}%
\pgfpathcurveto{\pgfqpoint{2.147559in}{2.968506in}}{\pgfqpoint{2.151509in}{2.966870in}}{\pgfqpoint{2.155627in}{2.966870in}}%
\pgfpathclose%
\pgfusepath{fill}%
\end{pgfscope}%
\begin{pgfscope}%
\pgfpathrectangle{\pgfqpoint{0.887500in}{0.275000in}}{\pgfqpoint{4.225000in}{4.225000in}}%
\pgfusepath{clip}%
\pgfsetbuttcap%
\pgfsetroundjoin%
\definecolor{currentfill}{rgb}{0.000000,0.000000,0.000000}%
\pgfsetfillcolor{currentfill}%
\pgfsetfillopacity{0.800000}%
\pgfsetlinewidth{0.000000pt}%
\definecolor{currentstroke}{rgb}{0.000000,0.000000,0.000000}%
\pgfsetstrokecolor{currentstroke}%
\pgfsetstrokeopacity{0.800000}%
\pgfsetdash{}{0pt}%
\pgfpathmoveto{\pgfqpoint{3.839543in}{3.742465in}}%
\pgfpathcurveto{\pgfqpoint{3.843661in}{3.742465in}}{\pgfqpoint{3.847611in}{3.744101in}}{\pgfqpoint{3.850523in}{3.747013in}}%
\pgfpathcurveto{\pgfqpoint{3.853435in}{3.749925in}}{\pgfqpoint{3.855071in}{3.753875in}}{\pgfqpoint{3.855071in}{3.757993in}}%
\pgfpathcurveto{\pgfqpoint{3.855071in}{3.762111in}}{\pgfqpoint{3.853435in}{3.766061in}}{\pgfqpoint{3.850523in}{3.768973in}}%
\pgfpathcurveto{\pgfqpoint{3.847611in}{3.771885in}}{\pgfqpoint{3.843661in}{3.773522in}}{\pgfqpoint{3.839543in}{3.773522in}}%
\pgfpathcurveto{\pgfqpoint{3.835425in}{3.773522in}}{\pgfqpoint{3.831475in}{3.771885in}}{\pgfqpoint{3.828563in}{3.768973in}}%
\pgfpathcurveto{\pgfqpoint{3.825651in}{3.766061in}}{\pgfqpoint{3.824015in}{3.762111in}}{\pgfqpoint{3.824015in}{3.757993in}}%
\pgfpathcurveto{\pgfqpoint{3.824015in}{3.753875in}}{\pgfqpoint{3.825651in}{3.749925in}}{\pgfqpoint{3.828563in}{3.747013in}}%
\pgfpathcurveto{\pgfqpoint{3.831475in}{3.744101in}}{\pgfqpoint{3.835425in}{3.742465in}}{\pgfqpoint{3.839543in}{3.742465in}}%
\pgfpathclose%
\pgfusepath{fill}%
\end{pgfscope}%
\begin{pgfscope}%
\pgfpathrectangle{\pgfqpoint{0.887500in}{0.275000in}}{\pgfqpoint{4.225000in}{4.225000in}}%
\pgfusepath{clip}%
\pgfsetbuttcap%
\pgfsetroundjoin%
\definecolor{currentfill}{rgb}{0.000000,0.000000,0.000000}%
\pgfsetfillcolor{currentfill}%
\pgfsetfillopacity{0.800000}%
\pgfsetlinewidth{0.000000pt}%
\definecolor{currentstroke}{rgb}{0.000000,0.000000,0.000000}%
\pgfsetstrokecolor{currentstroke}%
\pgfsetstrokeopacity{0.800000}%
\pgfsetdash{}{0pt}%
\pgfpathmoveto{\pgfqpoint{1.871779in}{2.945467in}}%
\pgfpathcurveto{\pgfqpoint{1.875897in}{2.945467in}}{\pgfqpoint{1.879847in}{2.947104in}}{\pgfqpoint{1.882759in}{2.950016in}}%
\pgfpathcurveto{\pgfqpoint{1.885671in}{2.952928in}}{\pgfqpoint{1.887307in}{2.956878in}}{\pgfqpoint{1.887307in}{2.960996in}}%
\pgfpathcurveto{\pgfqpoint{1.887307in}{2.965114in}}{\pgfqpoint{1.885671in}{2.969064in}}{\pgfqpoint{1.882759in}{2.971976in}}%
\pgfpathcurveto{\pgfqpoint{1.879847in}{2.974888in}}{\pgfqpoint{1.875897in}{2.976524in}}{\pgfqpoint{1.871779in}{2.976524in}}%
\pgfpathcurveto{\pgfqpoint{1.867661in}{2.976524in}}{\pgfqpoint{1.863711in}{2.974888in}}{\pgfqpoint{1.860799in}{2.971976in}}%
\pgfpathcurveto{\pgfqpoint{1.857887in}{2.969064in}}{\pgfqpoint{1.856251in}{2.965114in}}{\pgfqpoint{1.856251in}{2.960996in}}%
\pgfpathcurveto{\pgfqpoint{1.856251in}{2.956878in}}{\pgfqpoint{1.857887in}{2.952928in}}{\pgfqpoint{1.860799in}{2.950016in}}%
\pgfpathcurveto{\pgfqpoint{1.863711in}{2.947104in}}{\pgfqpoint{1.867661in}{2.945467in}}{\pgfqpoint{1.871779in}{2.945467in}}%
\pgfpathclose%
\pgfusepath{fill}%
\end{pgfscope}%
\begin{pgfscope}%
\pgfpathrectangle{\pgfqpoint{0.887500in}{0.275000in}}{\pgfqpoint{4.225000in}{4.225000in}}%
\pgfusepath{clip}%
\pgfsetbuttcap%
\pgfsetroundjoin%
\definecolor{currentfill}{rgb}{0.000000,0.000000,0.000000}%
\pgfsetfillcolor{currentfill}%
\pgfsetfillopacity{0.800000}%
\pgfsetlinewidth{0.000000pt}%
\definecolor{currentstroke}{rgb}{0.000000,0.000000,0.000000}%
\pgfsetstrokecolor{currentstroke}%
\pgfsetstrokeopacity{0.800000}%
\pgfsetdash{}{0pt}%
\pgfpathmoveto{\pgfqpoint{3.556189in}{3.720786in}}%
\pgfpathcurveto{\pgfqpoint{3.560307in}{3.720786in}}{\pgfqpoint{3.564257in}{3.722422in}}{\pgfqpoint{3.567169in}{3.725334in}}%
\pgfpathcurveto{\pgfqpoint{3.570081in}{3.728246in}}{\pgfqpoint{3.571717in}{3.732196in}}{\pgfqpoint{3.571717in}{3.736314in}}%
\pgfpathcurveto{\pgfqpoint{3.571717in}{3.740432in}}{\pgfqpoint{3.570081in}{3.744382in}}{\pgfqpoint{3.567169in}{3.747294in}}%
\pgfpathcurveto{\pgfqpoint{3.564257in}{3.750206in}}{\pgfqpoint{3.560307in}{3.751842in}}{\pgfqpoint{3.556189in}{3.751842in}}%
\pgfpathcurveto{\pgfqpoint{3.552070in}{3.751842in}}{\pgfqpoint{3.548120in}{3.750206in}}{\pgfqpoint{3.545208in}{3.747294in}}%
\pgfpathcurveto{\pgfqpoint{3.542296in}{3.744382in}}{\pgfqpoint{3.540660in}{3.740432in}}{\pgfqpoint{3.540660in}{3.736314in}}%
\pgfpathcurveto{\pgfqpoint{3.540660in}{3.732196in}}{\pgfqpoint{3.542296in}{3.728246in}}{\pgfqpoint{3.545208in}{3.725334in}}%
\pgfpathcurveto{\pgfqpoint{3.548120in}{3.722422in}}{\pgfqpoint{3.552070in}{3.720786in}}{\pgfqpoint{3.556189in}{3.720786in}}%
\pgfpathclose%
\pgfusepath{fill}%
\end{pgfscope}%
\begin{pgfscope}%
\pgfpathrectangle{\pgfqpoint{0.887500in}{0.275000in}}{\pgfqpoint{4.225000in}{4.225000in}}%
\pgfusepath{clip}%
\pgfsetbuttcap%
\pgfsetroundjoin%
\definecolor{currentfill}{rgb}{0.000000,0.000000,0.000000}%
\pgfsetfillcolor{currentfill}%
\pgfsetfillopacity{0.800000}%
\pgfsetlinewidth{0.000000pt}%
\definecolor{currentstroke}{rgb}{0.000000,0.000000,0.000000}%
\pgfsetstrokecolor{currentstroke}%
\pgfsetstrokeopacity{0.800000}%
\pgfsetdash{}{0pt}%
\pgfpathmoveto{\pgfqpoint{2.438340in}{3.050304in}}%
\pgfpathcurveto{\pgfqpoint{2.442458in}{3.050304in}}{\pgfqpoint{2.446408in}{3.051940in}}{\pgfqpoint{2.449320in}{3.054852in}}%
\pgfpathcurveto{\pgfqpoint{2.452232in}{3.057764in}}{\pgfqpoint{2.453869in}{3.061714in}}{\pgfqpoint{2.453869in}{3.065832in}}%
\pgfpathcurveto{\pgfqpoint{2.453869in}{3.069950in}}{\pgfqpoint{2.452232in}{3.073900in}}{\pgfqpoint{2.449320in}{3.076812in}}%
\pgfpathcurveto{\pgfqpoint{2.446408in}{3.079724in}}{\pgfqpoint{2.442458in}{3.081361in}}{\pgfqpoint{2.438340in}{3.081361in}}%
\pgfpathcurveto{\pgfqpoint{2.434222in}{3.081361in}}{\pgfqpoint{2.430272in}{3.079724in}}{\pgfqpoint{2.427360in}{3.076812in}}%
\pgfpathcurveto{\pgfqpoint{2.424448in}{3.073900in}}{\pgfqpoint{2.422812in}{3.069950in}}{\pgfqpoint{2.422812in}{3.065832in}}%
\pgfpathcurveto{\pgfqpoint{2.422812in}{3.061714in}}{\pgfqpoint{2.424448in}{3.057764in}}{\pgfqpoint{2.427360in}{3.054852in}}%
\pgfpathcurveto{\pgfqpoint{2.430272in}{3.051940in}}{\pgfqpoint{2.434222in}{3.050304in}}{\pgfqpoint{2.438340in}{3.050304in}}%
\pgfpathclose%
\pgfusepath{fill}%
\end{pgfscope}%
\begin{pgfscope}%
\pgfpathrectangle{\pgfqpoint{0.887500in}{0.275000in}}{\pgfqpoint{4.225000in}{4.225000in}}%
\pgfusepath{clip}%
\pgfsetbuttcap%
\pgfsetroundjoin%
\definecolor{currentfill}{rgb}{0.000000,0.000000,0.000000}%
\pgfsetfillcolor{currentfill}%
\pgfsetfillopacity{0.800000}%
\pgfsetlinewidth{0.000000pt}%
\definecolor{currentstroke}{rgb}{0.000000,0.000000,0.000000}%
\pgfsetstrokecolor{currentstroke}%
\pgfsetstrokeopacity{0.800000}%
\pgfsetdash{}{0pt}%
\pgfpathmoveto{\pgfqpoint{3.162670in}{3.599852in}}%
\pgfpathcurveto{\pgfqpoint{3.166788in}{3.599852in}}{\pgfqpoint{3.170738in}{3.601489in}}{\pgfqpoint{3.173650in}{3.604401in}}%
\pgfpathcurveto{\pgfqpoint{3.176562in}{3.607313in}}{\pgfqpoint{3.178198in}{3.611263in}}{\pgfqpoint{3.178198in}{3.615381in}}%
\pgfpathcurveto{\pgfqpoint{3.178198in}{3.619499in}}{\pgfqpoint{3.176562in}{3.623449in}}{\pgfqpoint{3.173650in}{3.626361in}}%
\pgfpathcurveto{\pgfqpoint{3.170738in}{3.629273in}}{\pgfqpoint{3.166788in}{3.630909in}}{\pgfqpoint{3.162670in}{3.630909in}}%
\pgfpathcurveto{\pgfqpoint{3.158552in}{3.630909in}}{\pgfqpoint{3.154602in}{3.629273in}}{\pgfqpoint{3.151690in}{3.626361in}}%
\pgfpathcurveto{\pgfqpoint{3.148778in}{3.623449in}}{\pgfqpoint{3.147141in}{3.619499in}}{\pgfqpoint{3.147141in}{3.615381in}}%
\pgfpathcurveto{\pgfqpoint{3.147141in}{3.611263in}}{\pgfqpoint{3.148778in}{3.607313in}}{\pgfqpoint{3.151690in}{3.604401in}}%
\pgfpathcurveto{\pgfqpoint{3.154602in}{3.601489in}}{\pgfqpoint{3.158552in}{3.599852in}}{\pgfqpoint{3.162670in}{3.599852in}}%
\pgfpathclose%
\pgfusepath{fill}%
\end{pgfscope}%
\begin{pgfscope}%
\pgfpathrectangle{\pgfqpoint{0.887500in}{0.275000in}}{\pgfqpoint{4.225000in}{4.225000in}}%
\pgfusepath{clip}%
\pgfsetbuttcap%
\pgfsetroundjoin%
\definecolor{currentfill}{rgb}{0.000000,0.000000,0.000000}%
\pgfsetfillcolor{currentfill}%
\pgfsetfillopacity{0.800000}%
\pgfsetlinewidth{0.000000pt}%
\definecolor{currentstroke}{rgb}{0.000000,0.000000,0.000000}%
\pgfsetstrokecolor{currentstroke}%
\pgfsetstrokeopacity{0.800000}%
\pgfsetdash{}{0pt}%
\pgfpathmoveto{\pgfqpoint{2.044090in}{2.901104in}}%
\pgfpathcurveto{\pgfqpoint{2.048209in}{2.901104in}}{\pgfqpoint{2.052159in}{2.902741in}}{\pgfqpoint{2.055071in}{2.905653in}}%
\pgfpathcurveto{\pgfqpoint{2.057983in}{2.908565in}}{\pgfqpoint{2.059619in}{2.912515in}}{\pgfqpoint{2.059619in}{2.916633in}}%
\pgfpathcurveto{\pgfqpoint{2.059619in}{2.920751in}}{\pgfqpoint{2.057983in}{2.924701in}}{\pgfqpoint{2.055071in}{2.927613in}}%
\pgfpathcurveto{\pgfqpoint{2.052159in}{2.930525in}}{\pgfqpoint{2.048209in}{2.932161in}}{\pgfqpoint{2.044090in}{2.932161in}}%
\pgfpathcurveto{\pgfqpoint{2.039972in}{2.932161in}}{\pgfqpoint{2.036022in}{2.930525in}}{\pgfqpoint{2.033110in}{2.927613in}}%
\pgfpathcurveto{\pgfqpoint{2.030198in}{2.924701in}}{\pgfqpoint{2.028562in}{2.920751in}}{\pgfqpoint{2.028562in}{2.916633in}}%
\pgfpathcurveto{\pgfqpoint{2.028562in}{2.912515in}}{\pgfqpoint{2.030198in}{2.908565in}}{\pgfqpoint{2.033110in}{2.905653in}}%
\pgfpathcurveto{\pgfqpoint{2.036022in}{2.902741in}}{\pgfqpoint{2.039972in}{2.901104in}}{\pgfqpoint{2.044090in}{2.901104in}}%
\pgfpathclose%
\pgfusepath{fill}%
\end{pgfscope}%
\begin{pgfscope}%
\pgfpathrectangle{\pgfqpoint{0.887500in}{0.275000in}}{\pgfqpoint{4.225000in}{4.225000in}}%
\pgfusepath{clip}%
\pgfsetbuttcap%
\pgfsetroundjoin%
\definecolor{currentfill}{rgb}{0.000000,0.000000,0.000000}%
\pgfsetfillcolor{currentfill}%
\pgfsetfillopacity{0.800000}%
\pgfsetlinewidth{0.000000pt}%
\definecolor{currentstroke}{rgb}{0.000000,0.000000,0.000000}%
\pgfsetstrokecolor{currentstroke}%
\pgfsetstrokeopacity{0.800000}%
\pgfsetdash{}{0pt}%
\pgfpathmoveto{\pgfqpoint{3.730872in}{3.695878in}}%
\pgfpathcurveto{\pgfqpoint{3.734990in}{3.695878in}}{\pgfqpoint{3.738940in}{3.697514in}}{\pgfqpoint{3.741852in}{3.700426in}}%
\pgfpathcurveto{\pgfqpoint{3.744764in}{3.703338in}}{\pgfqpoint{3.746400in}{3.707288in}}{\pgfqpoint{3.746400in}{3.711406in}}%
\pgfpathcurveto{\pgfqpoint{3.746400in}{3.715525in}}{\pgfqpoint{3.744764in}{3.719475in}}{\pgfqpoint{3.741852in}{3.722387in}}%
\pgfpathcurveto{\pgfqpoint{3.738940in}{3.725299in}}{\pgfqpoint{3.734990in}{3.726935in}}{\pgfqpoint{3.730872in}{3.726935in}}%
\pgfpathcurveto{\pgfqpoint{3.726753in}{3.726935in}}{\pgfqpoint{3.722803in}{3.725299in}}{\pgfqpoint{3.719891in}{3.722387in}}%
\pgfpathcurveto{\pgfqpoint{3.716979in}{3.719475in}}{\pgfqpoint{3.715343in}{3.715525in}}{\pgfqpoint{3.715343in}{3.711406in}}%
\pgfpathcurveto{\pgfqpoint{3.715343in}{3.707288in}}{\pgfqpoint{3.716979in}{3.703338in}}{\pgfqpoint{3.719891in}{3.700426in}}%
\pgfpathcurveto{\pgfqpoint{3.722803in}{3.697514in}}{\pgfqpoint{3.726753in}{3.695878in}}{\pgfqpoint{3.730872in}{3.695878in}}%
\pgfpathclose%
\pgfusepath{fill}%
\end{pgfscope}%
\begin{pgfscope}%
\pgfpathrectangle{\pgfqpoint{0.887500in}{0.275000in}}{\pgfqpoint{4.225000in}{4.225000in}}%
\pgfusepath{clip}%
\pgfsetbuttcap%
\pgfsetroundjoin%
\definecolor{currentfill}{rgb}{0.000000,0.000000,0.000000}%
\pgfsetfillcolor{currentfill}%
\pgfsetfillopacity{0.800000}%
\pgfsetlinewidth{0.000000pt}%
\definecolor{currentstroke}{rgb}{0.000000,0.000000,0.000000}%
\pgfsetstrokecolor{currentstroke}%
\pgfsetstrokeopacity{0.800000}%
\pgfsetdash{}{0pt}%
\pgfpathmoveto{\pgfqpoint{3.446869in}{3.668258in}}%
\pgfpathcurveto{\pgfqpoint{3.450987in}{3.668258in}}{\pgfqpoint{3.454937in}{3.669894in}}{\pgfqpoint{3.457849in}{3.672806in}}%
\pgfpathcurveto{\pgfqpoint{3.460761in}{3.675718in}}{\pgfqpoint{3.462397in}{3.679668in}}{\pgfqpoint{3.462397in}{3.683786in}}%
\pgfpathcurveto{\pgfqpoint{3.462397in}{3.687904in}}{\pgfqpoint{3.460761in}{3.691854in}}{\pgfqpoint{3.457849in}{3.694766in}}%
\pgfpathcurveto{\pgfqpoint{3.454937in}{3.697678in}}{\pgfqpoint{3.450987in}{3.699314in}}{\pgfqpoint{3.446869in}{3.699314in}}%
\pgfpathcurveto{\pgfqpoint{3.442750in}{3.699314in}}{\pgfqpoint{3.438800in}{3.697678in}}{\pgfqpoint{3.435888in}{3.694766in}}%
\pgfpathcurveto{\pgfqpoint{3.432977in}{3.691854in}}{\pgfqpoint{3.431340in}{3.687904in}}{\pgfqpoint{3.431340in}{3.683786in}}%
\pgfpathcurveto{\pgfqpoint{3.431340in}{3.679668in}}{\pgfqpoint{3.432977in}{3.675718in}}{\pgfqpoint{3.435888in}{3.672806in}}%
\pgfpathcurveto{\pgfqpoint{3.438800in}{3.669894in}}{\pgfqpoint{3.442750in}{3.668258in}}{\pgfqpoint{3.446869in}{3.668258in}}%
\pgfpathclose%
\pgfusepath{fill}%
\end{pgfscope}%
\begin{pgfscope}%
\pgfpathrectangle{\pgfqpoint{0.887500in}{0.275000in}}{\pgfqpoint{4.225000in}{4.225000in}}%
\pgfusepath{clip}%
\pgfsetbuttcap%
\pgfsetroundjoin%
\definecolor{currentfill}{rgb}{0.000000,0.000000,0.000000}%
\pgfsetfillcolor{currentfill}%
\pgfsetfillopacity{0.800000}%
\pgfsetlinewidth{0.000000pt}%
\definecolor{currentstroke}{rgb}{0.000000,0.000000,0.000000}%
\pgfsetstrokecolor{currentstroke}%
\pgfsetstrokeopacity{0.800000}%
\pgfsetdash{}{0pt}%
\pgfpathmoveto{\pgfqpoint{2.611567in}{2.999984in}}%
\pgfpathcurveto{\pgfqpoint{2.615685in}{2.999984in}}{\pgfqpoint{2.619635in}{3.001621in}}{\pgfqpoint{2.622547in}{3.004533in}}%
\pgfpathcurveto{\pgfqpoint{2.625459in}{3.007444in}}{\pgfqpoint{2.627095in}{3.011395in}}{\pgfqpoint{2.627095in}{3.015513in}}%
\pgfpathcurveto{\pgfqpoint{2.627095in}{3.019631in}}{\pgfqpoint{2.625459in}{3.023581in}}{\pgfqpoint{2.622547in}{3.026493in}}%
\pgfpathcurveto{\pgfqpoint{2.619635in}{3.029405in}}{\pgfqpoint{2.615685in}{3.031041in}}{\pgfqpoint{2.611567in}{3.031041in}}%
\pgfpathcurveto{\pgfqpoint{2.607448in}{3.031041in}}{\pgfqpoint{2.603498in}{3.029405in}}{\pgfqpoint{2.600586in}{3.026493in}}%
\pgfpathcurveto{\pgfqpoint{2.597674in}{3.023581in}}{\pgfqpoint{2.596038in}{3.019631in}}{\pgfqpoint{2.596038in}{3.015513in}}%
\pgfpathcurveto{\pgfqpoint{2.596038in}{3.011395in}}{\pgfqpoint{2.597674in}{3.007444in}}{\pgfqpoint{2.600586in}{3.004533in}}%
\pgfpathcurveto{\pgfqpoint{2.603498in}{3.001621in}}{\pgfqpoint{2.607448in}{2.999984in}}{\pgfqpoint{2.611567in}{2.999984in}}%
\pgfpathclose%
\pgfusepath{fill}%
\end{pgfscope}%
\begin{pgfscope}%
\pgfpathrectangle{\pgfqpoint{0.887500in}{0.275000in}}{\pgfqpoint{4.225000in}{4.225000in}}%
\pgfusepath{clip}%
\pgfsetbuttcap%
\pgfsetroundjoin%
\definecolor{currentfill}{rgb}{0.000000,0.000000,0.000000}%
\pgfsetfillcolor{currentfill}%
\pgfsetfillopacity{0.800000}%
\pgfsetlinewidth{0.000000pt}%
\definecolor{currentstroke}{rgb}{0.000000,0.000000,0.000000}%
\pgfsetstrokecolor{currentstroke}%
\pgfsetstrokeopacity{0.800000}%
\pgfsetdash{}{0pt}%
\pgfpathmoveto{\pgfqpoint{2.327390in}{2.968675in}}%
\pgfpathcurveto{\pgfqpoint{2.331508in}{2.968675in}}{\pgfqpoint{2.335458in}{2.970311in}}{\pgfqpoint{2.338370in}{2.973223in}}%
\pgfpathcurveto{\pgfqpoint{2.341282in}{2.976135in}}{\pgfqpoint{2.342918in}{2.980085in}}{\pgfqpoint{2.342918in}{2.984203in}}%
\pgfpathcurveto{\pgfqpoint{2.342918in}{2.988321in}}{\pgfqpoint{2.341282in}{2.992271in}}{\pgfqpoint{2.338370in}{2.995183in}}%
\pgfpathcurveto{\pgfqpoint{2.335458in}{2.998095in}}{\pgfqpoint{2.331508in}{2.999731in}}{\pgfqpoint{2.327390in}{2.999731in}}%
\pgfpathcurveto{\pgfqpoint{2.323271in}{2.999731in}}{\pgfqpoint{2.319321in}{2.998095in}}{\pgfqpoint{2.316409in}{2.995183in}}%
\pgfpathcurveto{\pgfqpoint{2.313497in}{2.992271in}}{\pgfqpoint{2.311861in}{2.988321in}}{\pgfqpoint{2.311861in}{2.984203in}}%
\pgfpathcurveto{\pgfqpoint{2.311861in}{2.980085in}}{\pgfqpoint{2.313497in}{2.976135in}}{\pgfqpoint{2.316409in}{2.973223in}}%
\pgfpathcurveto{\pgfqpoint{2.319321in}{2.970311in}}{\pgfqpoint{2.323271in}{2.968675in}}{\pgfqpoint{2.327390in}{2.968675in}}%
\pgfpathclose%
\pgfusepath{fill}%
\end{pgfscope}%
\begin{pgfscope}%
\pgfpathrectangle{\pgfqpoint{0.887500in}{0.275000in}}{\pgfqpoint{4.225000in}{4.225000in}}%
\pgfusepath{clip}%
\pgfsetbuttcap%
\pgfsetroundjoin%
\definecolor{currentfill}{rgb}{0.000000,0.000000,0.000000}%
\pgfsetfillcolor{currentfill}%
\pgfsetfillopacity{0.800000}%
\pgfsetlinewidth{0.000000pt}%
\definecolor{currentstroke}{rgb}{0.000000,0.000000,0.000000}%
\pgfsetstrokecolor{currentstroke}%
\pgfsetstrokeopacity{0.800000}%
\pgfsetdash{}{0pt}%
\pgfpathmoveto{\pgfqpoint{1.758771in}{2.910545in}}%
\pgfpathcurveto{\pgfqpoint{1.762889in}{2.910545in}}{\pgfqpoint{1.766839in}{2.912181in}}{\pgfqpoint{1.769751in}{2.915093in}}%
\pgfpathcurveto{\pgfqpoint{1.772663in}{2.918005in}}{\pgfqpoint{1.774299in}{2.921955in}}{\pgfqpoint{1.774299in}{2.926073in}}%
\pgfpathcurveto{\pgfqpoint{1.774299in}{2.930191in}}{\pgfqpoint{1.772663in}{2.934141in}}{\pgfqpoint{1.769751in}{2.937053in}}%
\pgfpathcurveto{\pgfqpoint{1.766839in}{2.939965in}}{\pgfqpoint{1.762889in}{2.941601in}}{\pgfqpoint{1.758771in}{2.941601in}}%
\pgfpathcurveto{\pgfqpoint{1.754652in}{2.941601in}}{\pgfqpoint{1.750702in}{2.939965in}}{\pgfqpoint{1.747790in}{2.937053in}}%
\pgfpathcurveto{\pgfqpoint{1.744878in}{2.934141in}}{\pgfqpoint{1.743242in}{2.930191in}}{\pgfqpoint{1.743242in}{2.926073in}}%
\pgfpathcurveto{\pgfqpoint{1.743242in}{2.921955in}}{\pgfqpoint{1.744878in}{2.918005in}}{\pgfqpoint{1.747790in}{2.915093in}}%
\pgfpathcurveto{\pgfqpoint{1.750702in}{2.912181in}}{\pgfqpoint{1.754652in}{2.910545in}}{\pgfqpoint{1.758771in}{2.910545in}}%
\pgfpathclose%
\pgfusepath{fill}%
\end{pgfscope}%
\begin{pgfscope}%
\pgfpathrectangle{\pgfqpoint{0.887500in}{0.275000in}}{\pgfqpoint{4.225000in}{4.225000in}}%
\pgfusepath{clip}%
\pgfsetbuttcap%
\pgfsetroundjoin%
\definecolor{currentfill}{rgb}{0.000000,0.000000,0.000000}%
\pgfsetfillcolor{currentfill}%
\pgfsetfillopacity{0.800000}%
\pgfsetlinewidth{0.000000pt}%
\definecolor{currentstroke}{rgb}{0.000000,0.000000,0.000000}%
\pgfsetstrokecolor{currentstroke}%
\pgfsetstrokeopacity{0.800000}%
\pgfsetdash{}{0pt}%
\pgfpathmoveto{\pgfqpoint{2.216864in}{2.854597in}}%
\pgfpathcurveto{\pgfqpoint{2.220982in}{2.854597in}}{\pgfqpoint{2.224933in}{2.856233in}}{\pgfqpoint{2.227844in}{2.859145in}}%
\pgfpathcurveto{\pgfqpoint{2.230756in}{2.862057in}}{\pgfqpoint{2.232393in}{2.866007in}}{\pgfqpoint{2.232393in}{2.870125in}}%
\pgfpathcurveto{\pgfqpoint{2.232393in}{2.874244in}}{\pgfqpoint{2.230756in}{2.878194in}}{\pgfqpoint{2.227844in}{2.881106in}}%
\pgfpathcurveto{\pgfqpoint{2.224933in}{2.884018in}}{\pgfqpoint{2.220982in}{2.885654in}}{\pgfqpoint{2.216864in}{2.885654in}}%
\pgfpathcurveto{\pgfqpoint{2.212746in}{2.885654in}}{\pgfqpoint{2.208796in}{2.884018in}}{\pgfqpoint{2.205884in}{2.881106in}}%
\pgfpathcurveto{\pgfqpoint{2.202972in}{2.878194in}}{\pgfqpoint{2.201336in}{2.874244in}}{\pgfqpoint{2.201336in}{2.870125in}}%
\pgfpathcurveto{\pgfqpoint{2.201336in}{2.866007in}}{\pgfqpoint{2.202972in}{2.862057in}}{\pgfqpoint{2.205884in}{2.859145in}}%
\pgfpathcurveto{\pgfqpoint{2.208796in}{2.856233in}}{\pgfqpoint{2.212746in}{2.854597in}}{\pgfqpoint{2.216864in}{2.854597in}}%
\pgfpathclose%
\pgfusepath{fill}%
\end{pgfscope}%
\begin{pgfscope}%
\pgfpathrectangle{\pgfqpoint{0.887500in}{0.275000in}}{\pgfqpoint{4.225000in}{4.225000in}}%
\pgfusepath{clip}%
\pgfsetbuttcap%
\pgfsetroundjoin%
\definecolor{currentfill}{rgb}{0.000000,0.000000,0.000000}%
\pgfsetfillcolor{currentfill}%
\pgfsetfillopacity{0.800000}%
\pgfsetlinewidth{0.000000pt}%
\definecolor{currentstroke}{rgb}{0.000000,0.000000,0.000000}%
\pgfsetstrokecolor{currentstroke}%
\pgfsetstrokeopacity{0.800000}%
\pgfsetdash{}{0pt}%
\pgfpathmoveto{\pgfqpoint{3.905608in}{3.645339in}}%
\pgfpathcurveto{\pgfqpoint{3.909726in}{3.645339in}}{\pgfqpoint{3.913676in}{3.646976in}}{\pgfqpoint{3.916588in}{3.649888in}}%
\pgfpathcurveto{\pgfqpoint{3.919500in}{3.652800in}}{\pgfqpoint{3.921136in}{3.656750in}}{\pgfqpoint{3.921136in}{3.660868in}}%
\pgfpathcurveto{\pgfqpoint{3.921136in}{3.664986in}}{\pgfqpoint{3.919500in}{3.668936in}}{\pgfqpoint{3.916588in}{3.671848in}}%
\pgfpathcurveto{\pgfqpoint{3.913676in}{3.674760in}}{\pgfqpoint{3.909726in}{3.676396in}}{\pgfqpoint{3.905608in}{3.676396in}}%
\pgfpathcurveto{\pgfqpoint{3.901490in}{3.676396in}}{\pgfqpoint{3.897540in}{3.674760in}}{\pgfqpoint{3.894628in}{3.671848in}}%
\pgfpathcurveto{\pgfqpoint{3.891716in}{3.668936in}}{\pgfqpoint{3.890079in}{3.664986in}}{\pgfqpoint{3.890079in}{3.660868in}}%
\pgfpathcurveto{\pgfqpoint{3.890079in}{3.656750in}}{\pgfqpoint{3.891716in}{3.652800in}}{\pgfqpoint{3.894628in}{3.649888in}}%
\pgfpathcurveto{\pgfqpoint{3.897540in}{3.646976in}}{\pgfqpoint{3.901490in}{3.645339in}}{\pgfqpoint{3.905608in}{3.645339in}}%
\pgfpathclose%
\pgfusepath{fill}%
\end{pgfscope}%
\begin{pgfscope}%
\pgfpathrectangle{\pgfqpoint{0.887500in}{0.275000in}}{\pgfqpoint{4.225000in}{4.225000in}}%
\pgfusepath{clip}%
\pgfsetbuttcap%
\pgfsetroundjoin%
\definecolor{currentfill}{rgb}{0.000000,0.000000,0.000000}%
\pgfsetfillcolor{currentfill}%
\pgfsetfillopacity{0.800000}%
\pgfsetlinewidth{0.000000pt}%
\definecolor{currentstroke}{rgb}{0.000000,0.000000,0.000000}%
\pgfsetstrokecolor{currentstroke}%
\pgfsetstrokeopacity{0.800000}%
\pgfsetdash{}{0pt}%
\pgfpathmoveto{\pgfqpoint{3.052521in}{3.544202in}}%
\pgfpathcurveto{\pgfqpoint{3.056639in}{3.544202in}}{\pgfqpoint{3.060589in}{3.545838in}}{\pgfqpoint{3.063501in}{3.548750in}}%
\pgfpathcurveto{\pgfqpoint{3.066413in}{3.551662in}}{\pgfqpoint{3.068049in}{3.555612in}}{\pgfqpoint{3.068049in}{3.559730in}}%
\pgfpathcurveto{\pgfqpoint{3.068049in}{3.563848in}}{\pgfqpoint{3.066413in}{3.567798in}}{\pgfqpoint{3.063501in}{3.570710in}}%
\pgfpathcurveto{\pgfqpoint{3.060589in}{3.573622in}}{\pgfqpoint{3.056639in}{3.575258in}}{\pgfqpoint{3.052521in}{3.575258in}}%
\pgfpathcurveto{\pgfqpoint{3.048403in}{3.575258in}}{\pgfqpoint{3.044453in}{3.573622in}}{\pgfqpoint{3.041541in}{3.570710in}}%
\pgfpathcurveto{\pgfqpoint{3.038629in}{3.567798in}}{\pgfqpoint{3.036993in}{3.563848in}}{\pgfqpoint{3.036993in}{3.559730in}}%
\pgfpathcurveto{\pgfqpoint{3.036993in}{3.555612in}}{\pgfqpoint{3.038629in}{3.551662in}}{\pgfqpoint{3.041541in}{3.548750in}}%
\pgfpathcurveto{\pgfqpoint{3.044453in}{3.545838in}}{\pgfqpoint{3.048403in}{3.544202in}}{\pgfqpoint{3.052521in}{3.544202in}}%
\pgfpathclose%
\pgfusepath{fill}%
\end{pgfscope}%
\begin{pgfscope}%
\pgfpathrectangle{\pgfqpoint{0.887500in}{0.275000in}}{\pgfqpoint{4.225000in}{4.225000in}}%
\pgfusepath{clip}%
\pgfsetbuttcap%
\pgfsetroundjoin%
\definecolor{currentfill}{rgb}{0.000000,0.000000,0.000000}%
\pgfsetfillcolor{currentfill}%
\pgfsetfillopacity{0.800000}%
\pgfsetlinewidth{0.000000pt}%
\definecolor{currentstroke}{rgb}{0.000000,0.000000,0.000000}%
\pgfsetstrokecolor{currentstroke}%
\pgfsetstrokeopacity{0.800000}%
\pgfsetdash{}{0pt}%
\pgfpathmoveto{\pgfqpoint{1.931337in}{2.866206in}}%
\pgfpathcurveto{\pgfqpoint{1.935455in}{2.866206in}}{\pgfqpoint{1.939405in}{2.867843in}}{\pgfqpoint{1.942317in}{2.870754in}}%
\pgfpathcurveto{\pgfqpoint{1.945229in}{2.873666in}}{\pgfqpoint{1.946865in}{2.877616in}}{\pgfqpoint{1.946865in}{2.881735in}}%
\pgfpathcurveto{\pgfqpoint{1.946865in}{2.885853in}}{\pgfqpoint{1.945229in}{2.889803in}}{\pgfqpoint{1.942317in}{2.892715in}}%
\pgfpathcurveto{\pgfqpoint{1.939405in}{2.895627in}}{\pgfqpoint{1.935455in}{2.897263in}}{\pgfqpoint{1.931337in}{2.897263in}}%
\pgfpathcurveto{\pgfqpoint{1.927219in}{2.897263in}}{\pgfqpoint{1.923269in}{2.895627in}}{\pgfqpoint{1.920357in}{2.892715in}}%
\pgfpathcurveto{\pgfqpoint{1.917445in}{2.889803in}}{\pgfqpoint{1.915809in}{2.885853in}}{\pgfqpoint{1.915809in}{2.881735in}}%
\pgfpathcurveto{\pgfqpoint{1.915809in}{2.877616in}}{\pgfqpoint{1.917445in}{2.873666in}}{\pgfqpoint{1.920357in}{2.870754in}}%
\pgfpathcurveto{\pgfqpoint{1.923269in}{2.867843in}}{\pgfqpoint{1.927219in}{2.866206in}}{\pgfqpoint{1.931337in}{2.866206in}}%
\pgfpathclose%
\pgfusepath{fill}%
\end{pgfscope}%
\begin{pgfscope}%
\pgfpathrectangle{\pgfqpoint{0.887500in}{0.275000in}}{\pgfqpoint{4.225000in}{4.225000in}}%
\pgfusepath{clip}%
\pgfsetbuttcap%
\pgfsetroundjoin%
\definecolor{currentfill}{rgb}{0.000000,0.000000,0.000000}%
\pgfsetfillcolor{currentfill}%
\pgfsetfillopacity{0.800000}%
\pgfsetlinewidth{0.000000pt}%
\definecolor{currentstroke}{rgb}{0.000000,0.000000,0.000000}%
\pgfsetstrokecolor{currentstroke}%
\pgfsetstrokeopacity{0.800000}%
\pgfsetdash{}{0pt}%
\pgfpathmoveto{\pgfqpoint{3.337117in}{3.611036in}}%
\pgfpathcurveto{\pgfqpoint{3.341235in}{3.611036in}}{\pgfqpoint{3.345185in}{3.612672in}}{\pgfqpoint{3.348097in}{3.615584in}}%
\pgfpathcurveto{\pgfqpoint{3.351009in}{3.618496in}}{\pgfqpoint{3.352645in}{3.622446in}}{\pgfqpoint{3.352645in}{3.626564in}}%
\pgfpathcurveto{\pgfqpoint{3.352645in}{3.630682in}}{\pgfqpoint{3.351009in}{3.634632in}}{\pgfqpoint{3.348097in}{3.637544in}}%
\pgfpathcurveto{\pgfqpoint{3.345185in}{3.640456in}}{\pgfqpoint{3.341235in}{3.642092in}}{\pgfqpoint{3.337117in}{3.642092in}}%
\pgfpathcurveto{\pgfqpoint{3.332999in}{3.642092in}}{\pgfqpoint{3.329049in}{3.640456in}}{\pgfqpoint{3.326137in}{3.637544in}}%
\pgfpathcurveto{\pgfqpoint{3.323225in}{3.634632in}}{\pgfqpoint{3.321589in}{3.630682in}}{\pgfqpoint{3.321589in}{3.626564in}}%
\pgfpathcurveto{\pgfqpoint{3.321589in}{3.622446in}}{\pgfqpoint{3.323225in}{3.618496in}}{\pgfqpoint{3.326137in}{3.615584in}}%
\pgfpathcurveto{\pgfqpoint{3.329049in}{3.612672in}}{\pgfqpoint{3.332999in}{3.611036in}}{\pgfqpoint{3.337117in}{3.611036in}}%
\pgfpathclose%
\pgfusepath{fill}%
\end{pgfscope}%
\begin{pgfscope}%
\pgfpathrectangle{\pgfqpoint{0.887500in}{0.275000in}}{\pgfqpoint{4.225000in}{4.225000in}}%
\pgfusepath{clip}%
\pgfsetbuttcap%
\pgfsetroundjoin%
\definecolor{currentfill}{rgb}{0.000000,0.000000,0.000000}%
\pgfsetfillcolor{currentfill}%
\pgfsetfillopacity{0.800000}%
\pgfsetlinewidth{0.000000pt}%
\definecolor{currentstroke}{rgb}{0.000000,0.000000,0.000000}%
\pgfsetstrokecolor{currentstroke}%
\pgfsetstrokeopacity{0.800000}%
\pgfsetdash{}{0pt}%
\pgfpathmoveto{\pgfqpoint{2.500399in}{2.939480in}}%
\pgfpathcurveto{\pgfqpoint{2.504517in}{2.939480in}}{\pgfqpoint{2.508467in}{2.941116in}}{\pgfqpoint{2.511379in}{2.944028in}}%
\pgfpathcurveto{\pgfqpoint{2.514291in}{2.946940in}}{\pgfqpoint{2.515927in}{2.950890in}}{\pgfqpoint{2.515927in}{2.955008in}}%
\pgfpathcurveto{\pgfqpoint{2.515927in}{2.959126in}}{\pgfqpoint{2.514291in}{2.963076in}}{\pgfqpoint{2.511379in}{2.965988in}}%
\pgfpathcurveto{\pgfqpoint{2.508467in}{2.968900in}}{\pgfqpoint{2.504517in}{2.970536in}}{\pgfqpoint{2.500399in}{2.970536in}}%
\pgfpathcurveto{\pgfqpoint{2.496281in}{2.970536in}}{\pgfqpoint{2.492331in}{2.968900in}}{\pgfqpoint{2.489419in}{2.965988in}}%
\pgfpathcurveto{\pgfqpoint{2.486507in}{2.963076in}}{\pgfqpoint{2.484871in}{2.959126in}}{\pgfqpoint{2.484871in}{2.955008in}}%
\pgfpathcurveto{\pgfqpoint{2.484871in}{2.950890in}}{\pgfqpoint{2.486507in}{2.946940in}}{\pgfqpoint{2.489419in}{2.944028in}}%
\pgfpathcurveto{\pgfqpoint{2.492331in}{2.941116in}}{\pgfqpoint{2.496281in}{2.939480in}}{\pgfqpoint{2.500399in}{2.939480in}}%
\pgfpathclose%
\pgfusepath{fill}%
\end{pgfscope}%
\begin{pgfscope}%
\pgfpathrectangle{\pgfqpoint{0.887500in}{0.275000in}}{\pgfqpoint{4.225000in}{4.225000in}}%
\pgfusepath{clip}%
\pgfsetbuttcap%
\pgfsetroundjoin%
\definecolor{currentfill}{rgb}{0.000000,0.000000,0.000000}%
\pgfsetfillcolor{currentfill}%
\pgfsetfillopacity{0.800000}%
\pgfsetlinewidth{0.000000pt}%
\definecolor{currentstroke}{rgb}{0.000000,0.000000,0.000000}%
\pgfsetstrokecolor{currentstroke}%
\pgfsetstrokeopacity{0.800000}%
\pgfsetdash{}{0pt}%
\pgfpathmoveto{\pgfqpoint{3.621721in}{3.645555in}}%
\pgfpathcurveto{\pgfqpoint{3.625839in}{3.645555in}}{\pgfqpoint{3.629789in}{3.647191in}}{\pgfqpoint{3.632701in}{3.650103in}}%
\pgfpathcurveto{\pgfqpoint{3.635613in}{3.653015in}}{\pgfqpoint{3.637249in}{3.656965in}}{\pgfqpoint{3.637249in}{3.661083in}}%
\pgfpathcurveto{\pgfqpoint{3.637249in}{3.665202in}}{\pgfqpoint{3.635613in}{3.669152in}}{\pgfqpoint{3.632701in}{3.672064in}}%
\pgfpathcurveto{\pgfqpoint{3.629789in}{3.674976in}}{\pgfqpoint{3.625839in}{3.676612in}}{\pgfqpoint{3.621721in}{3.676612in}}%
\pgfpathcurveto{\pgfqpoint{3.617603in}{3.676612in}}{\pgfqpoint{3.613653in}{3.674976in}}{\pgfqpoint{3.610741in}{3.672064in}}%
\pgfpathcurveto{\pgfqpoint{3.607829in}{3.669152in}}{\pgfqpoint{3.606193in}{3.665202in}}{\pgfqpoint{3.606193in}{3.661083in}}%
\pgfpathcurveto{\pgfqpoint{3.606193in}{3.656965in}}{\pgfqpoint{3.607829in}{3.653015in}}{\pgfqpoint{3.610741in}{3.650103in}}%
\pgfpathcurveto{\pgfqpoint{3.613653in}{3.647191in}}{\pgfqpoint{3.617603in}{3.645555in}}{\pgfqpoint{3.621721in}{3.645555in}}%
\pgfpathclose%
\pgfusepath{fill}%
\end{pgfscope}%
\begin{pgfscope}%
\pgfpathrectangle{\pgfqpoint{0.887500in}{0.275000in}}{\pgfqpoint{4.225000in}{4.225000in}}%
\pgfusepath{clip}%
\pgfsetbuttcap%
\pgfsetroundjoin%
\definecolor{currentfill}{rgb}{0.000000,0.000000,0.000000}%
\pgfsetfillcolor{currentfill}%
\pgfsetfillopacity{0.800000}%
\pgfsetlinewidth{0.000000pt}%
\definecolor{currentstroke}{rgb}{0.000000,0.000000,0.000000}%
\pgfsetstrokecolor{currentstroke}%
\pgfsetstrokeopacity{0.800000}%
\pgfsetdash{}{0pt}%
\pgfpathmoveto{\pgfqpoint{1.645189in}{2.876906in}}%
\pgfpathcurveto{\pgfqpoint{1.649307in}{2.876906in}}{\pgfqpoint{1.653257in}{2.878543in}}{\pgfqpoint{1.656169in}{2.881454in}}%
\pgfpathcurveto{\pgfqpoint{1.659081in}{2.884366in}}{\pgfqpoint{1.660717in}{2.888316in}}{\pgfqpoint{1.660717in}{2.892435in}}%
\pgfpathcurveto{\pgfqpoint{1.660717in}{2.896553in}}{\pgfqpoint{1.659081in}{2.900503in}}{\pgfqpoint{1.656169in}{2.903415in}}%
\pgfpathcurveto{\pgfqpoint{1.653257in}{2.906327in}}{\pgfqpoint{1.649307in}{2.907963in}}{\pgfqpoint{1.645189in}{2.907963in}}%
\pgfpathcurveto{\pgfqpoint{1.641070in}{2.907963in}}{\pgfqpoint{1.637120in}{2.906327in}}{\pgfqpoint{1.634208in}{2.903415in}}%
\pgfpathcurveto{\pgfqpoint{1.631296in}{2.900503in}}{\pgfqpoint{1.629660in}{2.896553in}}{\pgfqpoint{1.629660in}{2.892435in}}%
\pgfpathcurveto{\pgfqpoint{1.629660in}{2.888316in}}{\pgfqpoint{1.631296in}{2.884366in}}{\pgfqpoint{1.634208in}{2.881454in}}%
\pgfpathcurveto{\pgfqpoint{1.637120in}{2.878543in}}{\pgfqpoint{1.641070in}{2.876906in}}{\pgfqpoint{1.645189in}{2.876906in}}%
\pgfpathclose%
\pgfusepath{fill}%
\end{pgfscope}%
\begin{pgfscope}%
\pgfpathrectangle{\pgfqpoint{0.887500in}{0.275000in}}{\pgfqpoint{4.225000in}{4.225000in}}%
\pgfusepath{clip}%
\pgfsetbuttcap%
\pgfsetroundjoin%
\definecolor{currentfill}{rgb}{0.000000,0.000000,0.000000}%
\pgfsetfillcolor{currentfill}%
\pgfsetfillopacity{0.800000}%
\pgfsetlinewidth{0.000000pt}%
\definecolor{currentstroke}{rgb}{0.000000,0.000000,0.000000}%
\pgfsetstrokecolor{currentstroke}%
\pgfsetstrokeopacity{0.800000}%
\pgfsetdash{}{0pt}%
\pgfpathmoveto{\pgfqpoint{2.104371in}{2.819890in}}%
\pgfpathcurveto{\pgfqpoint{2.108489in}{2.819890in}}{\pgfqpoint{2.112439in}{2.821527in}}{\pgfqpoint{2.115351in}{2.824439in}}%
\pgfpathcurveto{\pgfqpoint{2.118263in}{2.827351in}}{\pgfqpoint{2.119899in}{2.831301in}}{\pgfqpoint{2.119899in}{2.835419in}}%
\pgfpathcurveto{\pgfqpoint{2.119899in}{2.839537in}}{\pgfqpoint{2.118263in}{2.843487in}}{\pgfqpoint{2.115351in}{2.846399in}}%
\pgfpathcurveto{\pgfqpoint{2.112439in}{2.849311in}}{\pgfqpoint{2.108489in}{2.850947in}}{\pgfqpoint{2.104371in}{2.850947in}}%
\pgfpathcurveto{\pgfqpoint{2.100253in}{2.850947in}}{\pgfqpoint{2.096303in}{2.849311in}}{\pgfqpoint{2.093391in}{2.846399in}}%
\pgfpathcurveto{\pgfqpoint{2.090479in}{2.843487in}}{\pgfqpoint{2.088843in}{2.839537in}}{\pgfqpoint{2.088843in}{2.835419in}}%
\pgfpathcurveto{\pgfqpoint{2.088843in}{2.831301in}}{\pgfqpoint{2.090479in}{2.827351in}}{\pgfqpoint{2.093391in}{2.824439in}}%
\pgfpathcurveto{\pgfqpoint{2.096303in}{2.821527in}}{\pgfqpoint{2.100253in}{2.819890in}}{\pgfqpoint{2.104371in}{2.819890in}}%
\pgfpathclose%
\pgfusepath{fill}%
\end{pgfscope}%
\begin{pgfscope}%
\pgfpathrectangle{\pgfqpoint{0.887500in}{0.275000in}}{\pgfqpoint{4.225000in}{4.225000in}}%
\pgfusepath{clip}%
\pgfsetbuttcap%
\pgfsetroundjoin%
\definecolor{currentfill}{rgb}{0.000000,0.000000,0.000000}%
\pgfsetfillcolor{currentfill}%
\pgfsetfillopacity{0.800000}%
\pgfsetlinewidth{0.000000pt}%
\definecolor{currentstroke}{rgb}{0.000000,0.000000,0.000000}%
\pgfsetstrokecolor{currentstroke}%
\pgfsetstrokeopacity{0.800000}%
\pgfsetdash{}{0pt}%
\pgfpathmoveto{\pgfqpoint{3.796763in}{3.599805in}}%
\pgfpathcurveto{\pgfqpoint{3.800881in}{3.599805in}}{\pgfqpoint{3.804831in}{3.601441in}}{\pgfqpoint{3.807743in}{3.604353in}}%
\pgfpathcurveto{\pgfqpoint{3.810655in}{3.607265in}}{\pgfqpoint{3.812292in}{3.611215in}}{\pgfqpoint{3.812292in}{3.615333in}}%
\pgfpathcurveto{\pgfqpoint{3.812292in}{3.619452in}}{\pgfqpoint{3.810655in}{3.623402in}}{\pgfqpoint{3.807743in}{3.626314in}}%
\pgfpathcurveto{\pgfqpoint{3.804831in}{3.629226in}}{\pgfqpoint{3.800881in}{3.630862in}}{\pgfqpoint{3.796763in}{3.630862in}}%
\pgfpathcurveto{\pgfqpoint{3.792645in}{3.630862in}}{\pgfqpoint{3.788695in}{3.629226in}}{\pgfqpoint{3.785783in}{3.626314in}}%
\pgfpathcurveto{\pgfqpoint{3.782871in}{3.623402in}}{\pgfqpoint{3.781235in}{3.619452in}}{\pgfqpoint{3.781235in}{3.615333in}}%
\pgfpathcurveto{\pgfqpoint{3.781235in}{3.611215in}}{\pgfqpoint{3.782871in}{3.607265in}}{\pgfqpoint{3.785783in}{3.604353in}}%
\pgfpathcurveto{\pgfqpoint{3.788695in}{3.601441in}}{\pgfqpoint{3.792645in}{3.599805in}}{\pgfqpoint{3.796763in}{3.599805in}}%
\pgfpathclose%
\pgfusepath{fill}%
\end{pgfscope}%
\begin{pgfscope}%
\pgfpathrectangle{\pgfqpoint{0.887500in}{0.275000in}}{\pgfqpoint{4.225000in}{4.225000in}}%
\pgfusepath{clip}%
\pgfsetbuttcap%
\pgfsetroundjoin%
\definecolor{currentfill}{rgb}{0.000000,0.000000,0.000000}%
\pgfsetfillcolor{currentfill}%
\pgfsetfillopacity{0.800000}%
\pgfsetlinewidth{0.000000pt}%
\definecolor{currentstroke}{rgb}{0.000000,0.000000,0.000000}%
\pgfsetstrokecolor{currentstroke}%
\pgfsetstrokeopacity{0.800000}%
\pgfsetdash{}{0pt}%
\pgfpathmoveto{\pgfqpoint{2.784674in}{3.021589in}}%
\pgfpathcurveto{\pgfqpoint{2.788792in}{3.021589in}}{\pgfqpoint{2.792742in}{3.023225in}}{\pgfqpoint{2.795654in}{3.026137in}}%
\pgfpathcurveto{\pgfqpoint{2.798566in}{3.029049in}}{\pgfqpoint{2.800202in}{3.032999in}}{\pgfqpoint{2.800202in}{3.037117in}}%
\pgfpathcurveto{\pgfqpoint{2.800202in}{3.041236in}}{\pgfqpoint{2.798566in}{3.045186in}}{\pgfqpoint{2.795654in}{3.048098in}}%
\pgfpathcurveto{\pgfqpoint{2.792742in}{3.051010in}}{\pgfqpoint{2.788792in}{3.052646in}}{\pgfqpoint{2.784674in}{3.052646in}}%
\pgfpathcurveto{\pgfqpoint{2.780556in}{3.052646in}}{\pgfqpoint{2.776606in}{3.051010in}}{\pgfqpoint{2.773694in}{3.048098in}}%
\pgfpathcurveto{\pgfqpoint{2.770782in}{3.045186in}}{\pgfqpoint{2.769146in}{3.041236in}}{\pgfqpoint{2.769146in}{3.037117in}}%
\pgfpathcurveto{\pgfqpoint{2.769146in}{3.032999in}}{\pgfqpoint{2.770782in}{3.029049in}}{\pgfqpoint{2.773694in}{3.026137in}}%
\pgfpathcurveto{\pgfqpoint{2.776606in}{3.023225in}}{\pgfqpoint{2.780556in}{3.021589in}}{\pgfqpoint{2.784674in}{3.021589in}}%
\pgfpathclose%
\pgfusepath{fill}%
\end{pgfscope}%
\begin{pgfscope}%
\pgfpathrectangle{\pgfqpoint{0.887500in}{0.275000in}}{\pgfqpoint{4.225000in}{4.225000in}}%
\pgfusepath{clip}%
\pgfsetbuttcap%
\pgfsetroundjoin%
\definecolor{currentfill}{rgb}{0.000000,0.000000,0.000000}%
\pgfsetfillcolor{currentfill}%
\pgfsetfillopacity{0.800000}%
\pgfsetlinewidth{0.000000pt}%
\definecolor{currentstroke}{rgb}{0.000000,0.000000,0.000000}%
\pgfsetstrokecolor{currentstroke}%
\pgfsetstrokeopacity{0.800000}%
\pgfsetdash{}{0pt}%
\pgfpathmoveto{\pgfqpoint{2.941926in}{3.508748in}}%
\pgfpathcurveto{\pgfqpoint{2.946044in}{3.508748in}}{\pgfqpoint{2.949994in}{3.510384in}}{\pgfqpoint{2.952906in}{3.513296in}}%
\pgfpathcurveto{\pgfqpoint{2.955818in}{3.516208in}}{\pgfqpoint{2.957454in}{3.520158in}}{\pgfqpoint{2.957454in}{3.524276in}}%
\pgfpathcurveto{\pgfqpoint{2.957454in}{3.528394in}}{\pgfqpoint{2.955818in}{3.532344in}}{\pgfqpoint{2.952906in}{3.535256in}}%
\pgfpathcurveto{\pgfqpoint{2.949994in}{3.538168in}}{\pgfqpoint{2.946044in}{3.539805in}}{\pgfqpoint{2.941926in}{3.539805in}}%
\pgfpathcurveto{\pgfqpoint{2.937808in}{3.539805in}}{\pgfqpoint{2.933858in}{3.538168in}}{\pgfqpoint{2.930946in}{3.535256in}}%
\pgfpathcurveto{\pgfqpoint{2.928034in}{3.532344in}}{\pgfqpoint{2.926398in}{3.528394in}}{\pgfqpoint{2.926398in}{3.524276in}}%
\pgfpathcurveto{\pgfqpoint{2.926398in}{3.520158in}}{\pgfqpoint{2.928034in}{3.516208in}}{\pgfqpoint{2.930946in}{3.513296in}}%
\pgfpathcurveto{\pgfqpoint{2.933858in}{3.510384in}}{\pgfqpoint{2.937808in}{3.508748in}}{\pgfqpoint{2.941926in}{3.508748in}}%
\pgfpathclose%
\pgfusepath{fill}%
\end{pgfscope}%
\begin{pgfscope}%
\pgfpathrectangle{\pgfqpoint{0.887500in}{0.275000in}}{\pgfqpoint{4.225000in}{4.225000in}}%
\pgfusepath{clip}%
\pgfsetbuttcap%
\pgfsetroundjoin%
\definecolor{currentfill}{rgb}{0.000000,0.000000,0.000000}%
\pgfsetfillcolor{currentfill}%
\pgfsetfillopacity{0.800000}%
\pgfsetlinewidth{0.000000pt}%
\definecolor{currentstroke}{rgb}{0.000000,0.000000,0.000000}%
\pgfsetstrokecolor{currentstroke}%
\pgfsetstrokeopacity{0.800000}%
\pgfsetdash{}{0pt}%
\pgfpathmoveto{\pgfqpoint{3.971516in}{3.522872in}}%
\pgfpathcurveto{\pgfqpoint{3.975634in}{3.522872in}}{\pgfqpoint{3.979584in}{3.524508in}}{\pgfqpoint{3.982496in}{3.527420in}}%
\pgfpathcurveto{\pgfqpoint{3.985408in}{3.530332in}}{\pgfqpoint{3.987045in}{3.534282in}}{\pgfqpoint{3.987045in}{3.538400in}}%
\pgfpathcurveto{\pgfqpoint{3.987045in}{3.542518in}}{\pgfqpoint{3.985408in}{3.546468in}}{\pgfqpoint{3.982496in}{3.549380in}}%
\pgfpathcurveto{\pgfqpoint{3.979584in}{3.552292in}}{\pgfqpoint{3.975634in}{3.553928in}}{\pgfqpoint{3.971516in}{3.553928in}}%
\pgfpathcurveto{\pgfqpoint{3.967398in}{3.553928in}}{\pgfqpoint{3.963448in}{3.552292in}}{\pgfqpoint{3.960536in}{3.549380in}}%
\pgfpathcurveto{\pgfqpoint{3.957624in}{3.546468in}}{\pgfqpoint{3.955988in}{3.542518in}}{\pgfqpoint{3.955988in}{3.538400in}}%
\pgfpathcurveto{\pgfqpoint{3.955988in}{3.534282in}}{\pgfqpoint{3.957624in}{3.530332in}}{\pgfqpoint{3.960536in}{3.527420in}}%
\pgfpathcurveto{\pgfqpoint{3.963448in}{3.524508in}}{\pgfqpoint{3.967398in}{3.522872in}}{\pgfqpoint{3.971516in}{3.522872in}}%
\pgfpathclose%
\pgfusepath{fill}%
\end{pgfscope}%
\begin{pgfscope}%
\pgfpathrectangle{\pgfqpoint{0.887500in}{0.275000in}}{\pgfqpoint{4.225000in}{4.225000in}}%
\pgfusepath{clip}%
\pgfsetbuttcap%
\pgfsetroundjoin%
\definecolor{currentfill}{rgb}{0.000000,0.000000,0.000000}%
\pgfsetfillcolor{currentfill}%
\pgfsetfillopacity{0.800000}%
\pgfsetlinewidth{0.000000pt}%
\definecolor{currentstroke}{rgb}{0.000000,0.000000,0.000000}%
\pgfsetstrokecolor{currentstroke}%
\pgfsetstrokeopacity{0.800000}%
\pgfsetdash{}{0pt}%
\pgfpathmoveto{\pgfqpoint{2.388865in}{2.878433in}}%
\pgfpathcurveto{\pgfqpoint{2.392983in}{2.878433in}}{\pgfqpoint{2.396933in}{2.880069in}}{\pgfqpoint{2.399845in}{2.882981in}}%
\pgfpathcurveto{\pgfqpoint{2.402757in}{2.885893in}}{\pgfqpoint{2.404393in}{2.889843in}}{\pgfqpoint{2.404393in}{2.893961in}}%
\pgfpathcurveto{\pgfqpoint{2.404393in}{2.898079in}}{\pgfqpoint{2.402757in}{2.902029in}}{\pgfqpoint{2.399845in}{2.904941in}}%
\pgfpathcurveto{\pgfqpoint{2.396933in}{2.907853in}}{\pgfqpoint{2.392983in}{2.909489in}}{\pgfqpoint{2.388865in}{2.909489in}}%
\pgfpathcurveto{\pgfqpoint{2.384747in}{2.909489in}}{\pgfqpoint{2.380797in}{2.907853in}}{\pgfqpoint{2.377885in}{2.904941in}}%
\pgfpathcurveto{\pgfqpoint{2.374973in}{2.902029in}}{\pgfqpoint{2.373337in}{2.898079in}}{\pgfqpoint{2.373337in}{2.893961in}}%
\pgfpathcurveto{\pgfqpoint{2.373337in}{2.889843in}}{\pgfqpoint{2.374973in}{2.885893in}}{\pgfqpoint{2.377885in}{2.882981in}}%
\pgfpathcurveto{\pgfqpoint{2.380797in}{2.880069in}}{\pgfqpoint{2.384747in}{2.878433in}}{\pgfqpoint{2.388865in}{2.878433in}}%
\pgfpathclose%
\pgfusepath{fill}%
\end{pgfscope}%
\begin{pgfscope}%
\pgfpathrectangle{\pgfqpoint{0.887500in}{0.275000in}}{\pgfqpoint{4.225000in}{4.225000in}}%
\pgfusepath{clip}%
\pgfsetbuttcap%
\pgfsetroundjoin%
\definecolor{currentfill}{rgb}{0.000000,0.000000,0.000000}%
\pgfsetfillcolor{currentfill}%
\pgfsetfillopacity{0.800000}%
\pgfsetlinewidth{0.000000pt}%
\definecolor{currentstroke}{rgb}{0.000000,0.000000,0.000000}%
\pgfsetstrokecolor{currentstroke}%
\pgfsetstrokeopacity{0.800000}%
\pgfsetdash{}{0pt}%
\pgfpathmoveto{\pgfqpoint{3.512170in}{3.596168in}}%
\pgfpathcurveto{\pgfqpoint{3.516288in}{3.596168in}}{\pgfqpoint{3.520238in}{3.597805in}}{\pgfqpoint{3.523150in}{3.600717in}}%
\pgfpathcurveto{\pgfqpoint{3.526062in}{3.603629in}}{\pgfqpoint{3.527698in}{3.607579in}}{\pgfqpoint{3.527698in}{3.611697in}}%
\pgfpathcurveto{\pgfqpoint{3.527698in}{3.615815in}}{\pgfqpoint{3.526062in}{3.619765in}}{\pgfqpoint{3.523150in}{3.622677in}}%
\pgfpathcurveto{\pgfqpoint{3.520238in}{3.625589in}}{\pgfqpoint{3.516288in}{3.627225in}}{\pgfqpoint{3.512170in}{3.627225in}}%
\pgfpathcurveto{\pgfqpoint{3.508052in}{3.627225in}}{\pgfqpoint{3.504102in}{3.625589in}}{\pgfqpoint{3.501190in}{3.622677in}}%
\pgfpathcurveto{\pgfqpoint{3.498278in}{3.619765in}}{\pgfqpoint{3.496642in}{3.615815in}}{\pgfqpoint{3.496642in}{3.611697in}}%
\pgfpathcurveto{\pgfqpoint{3.496642in}{3.607579in}}{\pgfqpoint{3.498278in}{3.603629in}}{\pgfqpoint{3.501190in}{3.600717in}}%
\pgfpathcurveto{\pgfqpoint{3.504102in}{3.597805in}}{\pgfqpoint{3.508052in}{3.596168in}}{\pgfqpoint{3.512170in}{3.596168in}}%
\pgfpathclose%
\pgfusepath{fill}%
\end{pgfscope}%
\begin{pgfscope}%
\pgfpathrectangle{\pgfqpoint{0.887500in}{0.275000in}}{\pgfqpoint{4.225000in}{4.225000in}}%
\pgfusepath{clip}%
\pgfsetbuttcap%
\pgfsetroundjoin%
\definecolor{currentfill}{rgb}{0.000000,0.000000,0.000000}%
\pgfsetfillcolor{currentfill}%
\pgfsetfillopacity{0.800000}%
\pgfsetlinewidth{0.000000pt}%
\definecolor{currentstroke}{rgb}{0.000000,0.000000,0.000000}%
\pgfsetstrokecolor{currentstroke}%
\pgfsetstrokeopacity{0.800000}%
\pgfsetdash{}{0pt}%
\pgfpathmoveto{\pgfqpoint{3.227048in}{3.568340in}}%
\pgfpathcurveto{\pgfqpoint{3.231166in}{3.568340in}}{\pgfqpoint{3.235116in}{3.569976in}}{\pgfqpoint{3.238028in}{3.572888in}}%
\pgfpathcurveto{\pgfqpoint{3.240940in}{3.575800in}}{\pgfqpoint{3.242576in}{3.579750in}}{\pgfqpoint{3.242576in}{3.583868in}}%
\pgfpathcurveto{\pgfqpoint{3.242576in}{3.587986in}}{\pgfqpoint{3.240940in}{3.591936in}}{\pgfqpoint{3.238028in}{3.594848in}}%
\pgfpathcurveto{\pgfqpoint{3.235116in}{3.597760in}}{\pgfqpoint{3.231166in}{3.599396in}}{\pgfqpoint{3.227048in}{3.599396in}}%
\pgfpathcurveto{\pgfqpoint{3.222930in}{3.599396in}}{\pgfqpoint{3.218980in}{3.597760in}}{\pgfqpoint{3.216068in}{3.594848in}}%
\pgfpathcurveto{\pgfqpoint{3.213156in}{3.591936in}}{\pgfqpoint{3.211519in}{3.587986in}}{\pgfqpoint{3.211519in}{3.583868in}}%
\pgfpathcurveto{\pgfqpoint{3.211519in}{3.579750in}}{\pgfqpoint{3.213156in}{3.575800in}}{\pgfqpoint{3.216068in}{3.572888in}}%
\pgfpathcurveto{\pgfqpoint{3.218980in}{3.569976in}}{\pgfqpoint{3.222930in}{3.568340in}}{\pgfqpoint{3.227048in}{3.568340in}}%
\pgfpathclose%
\pgfusepath{fill}%
\end{pgfscope}%
\begin{pgfscope}%
\pgfpathrectangle{\pgfqpoint{0.887500in}{0.275000in}}{\pgfqpoint{4.225000in}{4.225000in}}%
\pgfusepath{clip}%
\pgfsetbuttcap%
\pgfsetroundjoin%
\definecolor{currentfill}{rgb}{0.000000,0.000000,0.000000}%
\pgfsetfillcolor{currentfill}%
\pgfsetfillopacity{0.800000}%
\pgfsetlinewidth{0.000000pt}%
\definecolor{currentstroke}{rgb}{0.000000,0.000000,0.000000}%
\pgfsetstrokecolor{currentstroke}%
\pgfsetstrokeopacity{0.800000}%
\pgfsetdash{}{0pt}%
\pgfpathmoveto{\pgfqpoint{2.277848in}{2.771681in}}%
\pgfpathcurveto{\pgfqpoint{2.281966in}{2.771681in}}{\pgfqpoint{2.285916in}{2.773317in}}{\pgfqpoint{2.288828in}{2.776229in}}%
\pgfpathcurveto{\pgfqpoint{2.291740in}{2.779141in}}{\pgfqpoint{2.293377in}{2.783091in}}{\pgfqpoint{2.293377in}{2.787209in}}%
\pgfpathcurveto{\pgfqpoint{2.293377in}{2.791328in}}{\pgfqpoint{2.291740in}{2.795278in}}{\pgfqpoint{2.288828in}{2.798190in}}%
\pgfpathcurveto{\pgfqpoint{2.285916in}{2.801102in}}{\pgfqpoint{2.281966in}{2.802738in}}{\pgfqpoint{2.277848in}{2.802738in}}%
\pgfpathcurveto{\pgfqpoint{2.273730in}{2.802738in}}{\pgfqpoint{2.269780in}{2.801102in}}{\pgfqpoint{2.266868in}{2.798190in}}%
\pgfpathcurveto{\pgfqpoint{2.263956in}{2.795278in}}{\pgfqpoint{2.262320in}{2.791328in}}{\pgfqpoint{2.262320in}{2.787209in}}%
\pgfpathcurveto{\pgfqpoint{2.262320in}{2.783091in}}{\pgfqpoint{2.263956in}{2.779141in}}{\pgfqpoint{2.266868in}{2.776229in}}%
\pgfpathcurveto{\pgfqpoint{2.269780in}{2.773317in}}{\pgfqpoint{2.273730in}{2.771681in}}{\pgfqpoint{2.277848in}{2.771681in}}%
\pgfpathclose%
\pgfusepath{fill}%
\end{pgfscope}%
\begin{pgfscope}%
\pgfpathrectangle{\pgfqpoint{0.887500in}{0.275000in}}{\pgfqpoint{4.225000in}{4.225000in}}%
\pgfusepath{clip}%
\pgfsetbuttcap%
\pgfsetroundjoin%
\definecolor{currentfill}{rgb}{0.000000,0.000000,0.000000}%
\pgfsetfillcolor{currentfill}%
\pgfsetfillopacity{0.800000}%
\pgfsetlinewidth{0.000000pt}%
\definecolor{currentstroke}{rgb}{0.000000,0.000000,0.000000}%
\pgfsetstrokecolor{currentstroke}%
\pgfsetstrokeopacity{0.800000}%
\pgfsetdash{}{0pt}%
\pgfpathmoveto{\pgfqpoint{1.818032in}{2.832082in}}%
\pgfpathcurveto{\pgfqpoint{1.822150in}{2.832082in}}{\pgfqpoint{1.826100in}{2.833719in}}{\pgfqpoint{1.829012in}{2.836631in}}%
\pgfpathcurveto{\pgfqpoint{1.831924in}{2.839542in}}{\pgfqpoint{1.833560in}{2.843493in}}{\pgfqpoint{1.833560in}{2.847611in}}%
\pgfpathcurveto{\pgfqpoint{1.833560in}{2.851729in}}{\pgfqpoint{1.831924in}{2.855679in}}{\pgfqpoint{1.829012in}{2.858591in}}%
\pgfpathcurveto{\pgfqpoint{1.826100in}{2.861503in}}{\pgfqpoint{1.822150in}{2.863139in}}{\pgfqpoint{1.818032in}{2.863139in}}%
\pgfpathcurveto{\pgfqpoint{1.813914in}{2.863139in}}{\pgfqpoint{1.809964in}{2.861503in}}{\pgfqpoint{1.807052in}{2.858591in}}%
\pgfpathcurveto{\pgfqpoint{1.804140in}{2.855679in}}{\pgfqpoint{1.802504in}{2.851729in}}{\pgfqpoint{1.802504in}{2.847611in}}%
\pgfpathcurveto{\pgfqpoint{1.802504in}{2.843493in}}{\pgfqpoint{1.804140in}{2.839542in}}{\pgfqpoint{1.807052in}{2.836631in}}%
\pgfpathcurveto{\pgfqpoint{1.809964in}{2.833719in}}{\pgfqpoint{1.813914in}{2.832082in}}{\pgfqpoint{1.818032in}{2.832082in}}%
\pgfpathclose%
\pgfusepath{fill}%
\end{pgfscope}%
\begin{pgfscope}%
\pgfpathrectangle{\pgfqpoint{0.887500in}{0.275000in}}{\pgfqpoint{4.225000in}{4.225000in}}%
\pgfusepath{clip}%
\pgfsetbuttcap%
\pgfsetroundjoin%
\definecolor{currentfill}{rgb}{0.000000,0.000000,0.000000}%
\pgfsetfillcolor{currentfill}%
\pgfsetfillopacity{0.800000}%
\pgfsetlinewidth{0.000000pt}%
\definecolor{currentstroke}{rgb}{0.000000,0.000000,0.000000}%
\pgfsetstrokecolor{currentstroke}%
\pgfsetstrokeopacity{0.800000}%
\pgfsetdash{}{0pt}%
\pgfpathmoveto{\pgfqpoint{2.673586in}{2.949591in}}%
\pgfpathcurveto{\pgfqpoint{2.677704in}{2.949591in}}{\pgfqpoint{2.681654in}{2.951227in}}{\pgfqpoint{2.684566in}{2.954139in}}%
\pgfpathcurveto{\pgfqpoint{2.687478in}{2.957051in}}{\pgfqpoint{2.689114in}{2.961001in}}{\pgfqpoint{2.689114in}{2.965119in}}%
\pgfpathcurveto{\pgfqpoint{2.689114in}{2.969237in}}{\pgfqpoint{2.687478in}{2.973187in}}{\pgfqpoint{2.684566in}{2.976099in}}%
\pgfpathcurveto{\pgfqpoint{2.681654in}{2.979011in}}{\pgfqpoint{2.677704in}{2.980647in}}{\pgfqpoint{2.673586in}{2.980647in}}%
\pgfpathcurveto{\pgfqpoint{2.669468in}{2.980647in}}{\pgfqpoint{2.665518in}{2.979011in}}{\pgfqpoint{2.662606in}{2.976099in}}%
\pgfpathcurveto{\pgfqpoint{2.659694in}{2.973187in}}{\pgfqpoint{2.658058in}{2.969237in}}{\pgfqpoint{2.658058in}{2.965119in}}%
\pgfpathcurveto{\pgfqpoint{2.658058in}{2.961001in}}{\pgfqpoint{2.659694in}{2.957051in}}{\pgfqpoint{2.662606in}{2.954139in}}%
\pgfpathcurveto{\pgfqpoint{2.665518in}{2.951227in}}{\pgfqpoint{2.669468in}{2.949591in}}{\pgfqpoint{2.673586in}{2.949591in}}%
\pgfpathclose%
\pgfusepath{fill}%
\end{pgfscope}%
\begin{pgfscope}%
\pgfpathrectangle{\pgfqpoint{0.887500in}{0.275000in}}{\pgfqpoint{4.225000in}{4.225000in}}%
\pgfusepath{clip}%
\pgfsetbuttcap%
\pgfsetroundjoin%
\definecolor{currentfill}{rgb}{0.000000,0.000000,0.000000}%
\pgfsetfillcolor{currentfill}%
\pgfsetfillopacity{0.800000}%
\pgfsetlinewidth{0.000000pt}%
\definecolor{currentstroke}{rgb}{0.000000,0.000000,0.000000}%
\pgfsetstrokecolor{currentstroke}%
\pgfsetstrokeopacity{0.800000}%
\pgfsetdash{}{0pt}%
\pgfpathmoveto{\pgfqpoint{3.687435in}{3.551226in}}%
\pgfpathcurveto{\pgfqpoint{3.691553in}{3.551226in}}{\pgfqpoint{3.695503in}{3.552862in}}{\pgfqpoint{3.698415in}{3.555774in}}%
\pgfpathcurveto{\pgfqpoint{3.701327in}{3.558686in}}{\pgfqpoint{3.702963in}{3.562636in}}{\pgfqpoint{3.702963in}{3.566754in}}%
\pgfpathcurveto{\pgfqpoint{3.702963in}{3.570873in}}{\pgfqpoint{3.701327in}{3.574823in}}{\pgfqpoint{3.698415in}{3.577735in}}%
\pgfpathcurveto{\pgfqpoint{3.695503in}{3.580646in}}{\pgfqpoint{3.691553in}{3.582283in}}{\pgfqpoint{3.687435in}{3.582283in}}%
\pgfpathcurveto{\pgfqpoint{3.683317in}{3.582283in}}{\pgfqpoint{3.679367in}{3.580646in}}{\pgfqpoint{3.676455in}{3.577735in}}%
\pgfpathcurveto{\pgfqpoint{3.673543in}{3.574823in}}{\pgfqpoint{3.671907in}{3.570873in}}{\pgfqpoint{3.671907in}{3.566754in}}%
\pgfpathcurveto{\pgfqpoint{3.671907in}{3.562636in}}{\pgfqpoint{3.673543in}{3.558686in}}{\pgfqpoint{3.676455in}{3.555774in}}%
\pgfpathcurveto{\pgfqpoint{3.679367in}{3.552862in}}{\pgfqpoint{3.683317in}{3.551226in}}{\pgfqpoint{3.687435in}{3.551226in}}%
\pgfpathclose%
\pgfusepath{fill}%
\end{pgfscope}%
\begin{pgfscope}%
\pgfpathrectangle{\pgfqpoint{0.887500in}{0.275000in}}{\pgfqpoint{4.225000in}{4.225000in}}%
\pgfusepath{clip}%
\pgfsetbuttcap%
\pgfsetroundjoin%
\definecolor{currentfill}{rgb}{0.000000,0.000000,0.000000}%
\pgfsetfillcolor{currentfill}%
\pgfsetfillopacity{0.800000}%
\pgfsetlinewidth{0.000000pt}%
\definecolor{currentstroke}{rgb}{0.000000,0.000000,0.000000}%
\pgfsetstrokecolor{currentstroke}%
\pgfsetstrokeopacity{0.800000}%
\pgfsetdash{}{0pt}%
\pgfpathmoveto{\pgfqpoint{1.531062in}{2.843365in}}%
\pgfpathcurveto{\pgfqpoint{1.535180in}{2.843365in}}{\pgfqpoint{1.539130in}{2.845002in}}{\pgfqpoint{1.542042in}{2.847914in}}%
\pgfpathcurveto{\pgfqpoint{1.544954in}{2.850826in}}{\pgfqpoint{1.546590in}{2.854776in}}{\pgfqpoint{1.546590in}{2.858894in}}%
\pgfpathcurveto{\pgfqpoint{1.546590in}{2.863012in}}{\pgfqpoint{1.544954in}{2.866962in}}{\pgfqpoint{1.542042in}{2.869874in}}%
\pgfpathcurveto{\pgfqpoint{1.539130in}{2.872786in}}{\pgfqpoint{1.535180in}{2.874422in}}{\pgfqpoint{1.531062in}{2.874422in}}%
\pgfpathcurveto{\pgfqpoint{1.526944in}{2.874422in}}{\pgfqpoint{1.522994in}{2.872786in}}{\pgfqpoint{1.520082in}{2.869874in}}%
\pgfpathcurveto{\pgfqpoint{1.517170in}{2.866962in}}{\pgfqpoint{1.515534in}{2.863012in}}{\pgfqpoint{1.515534in}{2.858894in}}%
\pgfpathcurveto{\pgfqpoint{1.515534in}{2.854776in}}{\pgfqpoint{1.517170in}{2.850826in}}{\pgfqpoint{1.520082in}{2.847914in}}%
\pgfpathcurveto{\pgfqpoint{1.522994in}{2.845002in}}{\pgfqpoint{1.526944in}{2.843365in}}{\pgfqpoint{1.531062in}{2.843365in}}%
\pgfpathclose%
\pgfusepath{fill}%
\end{pgfscope}%
\begin{pgfscope}%
\pgfpathrectangle{\pgfqpoint{0.887500in}{0.275000in}}{\pgfqpoint{4.225000in}{4.225000in}}%
\pgfusepath{clip}%
\pgfsetbuttcap%
\pgfsetroundjoin%
\definecolor{currentfill}{rgb}{0.000000,0.000000,0.000000}%
\pgfsetfillcolor{currentfill}%
\pgfsetfillopacity{0.800000}%
\pgfsetlinewidth{0.000000pt}%
\definecolor{currentstroke}{rgb}{0.000000,0.000000,0.000000}%
\pgfsetstrokecolor{currentstroke}%
\pgfsetstrokeopacity{0.800000}%
\pgfsetdash{}{0pt}%
\pgfpathmoveto{\pgfqpoint{1.991343in}{2.785447in}}%
\pgfpathcurveto{\pgfqpoint{1.995461in}{2.785447in}}{\pgfqpoint{1.999411in}{2.787084in}}{\pgfqpoint{2.002323in}{2.789995in}}%
\pgfpathcurveto{\pgfqpoint{2.005235in}{2.792907in}}{\pgfqpoint{2.006871in}{2.796857in}}{\pgfqpoint{2.006871in}{2.800976in}}%
\pgfpathcurveto{\pgfqpoint{2.006871in}{2.805094in}}{\pgfqpoint{2.005235in}{2.809044in}}{\pgfqpoint{2.002323in}{2.811956in}}%
\pgfpathcurveto{\pgfqpoint{1.999411in}{2.814868in}}{\pgfqpoint{1.995461in}{2.816504in}}{\pgfqpoint{1.991343in}{2.816504in}}%
\pgfpathcurveto{\pgfqpoint{1.987224in}{2.816504in}}{\pgfqpoint{1.983274in}{2.814868in}}{\pgfqpoint{1.980362in}{2.811956in}}%
\pgfpathcurveto{\pgfqpoint{1.977450in}{2.809044in}}{\pgfqpoint{1.975814in}{2.805094in}}{\pgfqpoint{1.975814in}{2.800976in}}%
\pgfpathcurveto{\pgfqpoint{1.975814in}{2.796857in}}{\pgfqpoint{1.977450in}{2.792907in}}{\pgfqpoint{1.980362in}{2.789995in}}%
\pgfpathcurveto{\pgfqpoint{1.983274in}{2.787084in}}{\pgfqpoint{1.987224in}{2.785447in}}{\pgfqpoint{1.991343in}{2.785447in}}%
\pgfpathclose%
\pgfusepath{fill}%
\end{pgfscope}%
\begin{pgfscope}%
\pgfpathrectangle{\pgfqpoint{0.887500in}{0.275000in}}{\pgfqpoint{4.225000in}{4.225000in}}%
\pgfusepath{clip}%
\pgfsetbuttcap%
\pgfsetroundjoin%
\definecolor{currentfill}{rgb}{0.000000,0.000000,0.000000}%
\pgfsetfillcolor{currentfill}%
\pgfsetfillopacity{0.800000}%
\pgfsetlinewidth{0.000000pt}%
\definecolor{currentstroke}{rgb}{0.000000,0.000000,0.000000}%
\pgfsetstrokecolor{currentstroke}%
\pgfsetstrokeopacity{0.800000}%
\pgfsetdash{}{0pt}%
\pgfpathmoveto{\pgfqpoint{3.862610in}{3.480443in}}%
\pgfpathcurveto{\pgfqpoint{3.866728in}{3.480443in}}{\pgfqpoint{3.870678in}{3.482079in}}{\pgfqpoint{3.873590in}{3.484991in}}%
\pgfpathcurveto{\pgfqpoint{3.876502in}{3.487903in}}{\pgfqpoint{3.878138in}{3.491853in}}{\pgfqpoint{3.878138in}{3.495971in}}%
\pgfpathcurveto{\pgfqpoint{3.878138in}{3.500089in}}{\pgfqpoint{3.876502in}{3.504039in}}{\pgfqpoint{3.873590in}{3.506951in}}%
\pgfpathcurveto{\pgfqpoint{3.870678in}{3.509863in}}{\pgfqpoint{3.866728in}{3.511499in}}{\pgfqpoint{3.862610in}{3.511499in}}%
\pgfpathcurveto{\pgfqpoint{3.858491in}{3.511499in}}{\pgfqpoint{3.854541in}{3.509863in}}{\pgfqpoint{3.851629in}{3.506951in}}%
\pgfpathcurveto{\pgfqpoint{3.848717in}{3.504039in}}{\pgfqpoint{3.847081in}{3.500089in}}{\pgfqpoint{3.847081in}{3.495971in}}%
\pgfpathcurveto{\pgfqpoint{3.847081in}{3.491853in}}{\pgfqpoint{3.848717in}{3.487903in}}{\pgfqpoint{3.851629in}{3.484991in}}%
\pgfpathcurveto{\pgfqpoint{3.854541in}{3.482079in}}{\pgfqpoint{3.858491in}{3.480443in}}{\pgfqpoint{3.862610in}{3.480443in}}%
\pgfpathclose%
\pgfusepath{fill}%
\end{pgfscope}%
\begin{pgfscope}%
\pgfpathrectangle{\pgfqpoint{0.887500in}{0.275000in}}{\pgfqpoint{4.225000in}{4.225000in}}%
\pgfusepath{clip}%
\pgfsetbuttcap%
\pgfsetroundjoin%
\definecolor{currentfill}{rgb}{0.000000,0.000000,0.000000}%
\pgfsetfillcolor{currentfill}%
\pgfsetfillopacity{0.800000}%
\pgfsetlinewidth{0.000000pt}%
\definecolor{currentstroke}{rgb}{0.000000,0.000000,0.000000}%
\pgfsetstrokecolor{currentstroke}%
\pgfsetstrokeopacity{0.800000}%
\pgfsetdash{}{0pt}%
\pgfpathmoveto{\pgfqpoint{2.830837in}{3.471488in}}%
\pgfpathcurveto{\pgfqpoint{2.834955in}{3.471488in}}{\pgfqpoint{2.838905in}{3.473124in}}{\pgfqpoint{2.841817in}{3.476036in}}%
\pgfpathcurveto{\pgfqpoint{2.844729in}{3.478948in}}{\pgfqpoint{2.846365in}{3.482898in}}{\pgfqpoint{2.846365in}{3.487017in}}%
\pgfpathcurveto{\pgfqpoint{2.846365in}{3.491135in}}{\pgfqpoint{2.844729in}{3.495085in}}{\pgfqpoint{2.841817in}{3.497997in}}%
\pgfpathcurveto{\pgfqpoint{2.838905in}{3.500909in}}{\pgfqpoint{2.834955in}{3.502545in}}{\pgfqpoint{2.830837in}{3.502545in}}%
\pgfpathcurveto{\pgfqpoint{2.826719in}{3.502545in}}{\pgfqpoint{2.822769in}{3.500909in}}{\pgfqpoint{2.819857in}{3.497997in}}%
\pgfpathcurveto{\pgfqpoint{2.816945in}{3.495085in}}{\pgfqpoint{2.815309in}{3.491135in}}{\pgfqpoint{2.815309in}{3.487017in}}%
\pgfpathcurveto{\pgfqpoint{2.815309in}{3.482898in}}{\pgfqpoint{2.816945in}{3.478948in}}{\pgfqpoint{2.819857in}{3.476036in}}%
\pgfpathcurveto{\pgfqpoint{2.822769in}{3.473124in}}{\pgfqpoint{2.826719in}{3.471488in}}{\pgfqpoint{2.830837in}{3.471488in}}%
\pgfpathclose%
\pgfusepath{fill}%
\end{pgfscope}%
\begin{pgfscope}%
\pgfpathrectangle{\pgfqpoint{0.887500in}{0.275000in}}{\pgfqpoint{4.225000in}{4.225000in}}%
\pgfusepath{clip}%
\pgfsetbuttcap%
\pgfsetroundjoin%
\definecolor{currentfill}{rgb}{0.000000,0.000000,0.000000}%
\pgfsetfillcolor{currentfill}%
\pgfsetfillopacity{0.800000}%
\pgfsetlinewidth{0.000000pt}%
\definecolor{currentstroke}{rgb}{0.000000,0.000000,0.000000}%
\pgfsetstrokecolor{currentstroke}%
\pgfsetstrokeopacity{0.800000}%
\pgfsetdash{}{0pt}%
\pgfpathmoveto{\pgfqpoint{4.037573in}{3.392722in}}%
\pgfpathcurveto{\pgfqpoint{4.041692in}{3.392722in}}{\pgfqpoint{4.045642in}{3.394358in}}{\pgfqpoint{4.048554in}{3.397270in}}%
\pgfpathcurveto{\pgfqpoint{4.051466in}{3.400182in}}{\pgfqpoint{4.053102in}{3.404132in}}{\pgfqpoint{4.053102in}{3.408250in}}%
\pgfpathcurveto{\pgfqpoint{4.053102in}{3.412368in}}{\pgfqpoint{4.051466in}{3.416318in}}{\pgfqpoint{4.048554in}{3.419230in}}%
\pgfpathcurveto{\pgfqpoint{4.045642in}{3.422142in}}{\pgfqpoint{4.041692in}{3.423778in}}{\pgfqpoint{4.037573in}{3.423778in}}%
\pgfpathcurveto{\pgfqpoint{4.033455in}{3.423778in}}{\pgfqpoint{4.029505in}{3.422142in}}{\pgfqpoint{4.026593in}{3.419230in}}%
\pgfpathcurveto{\pgfqpoint{4.023681in}{3.416318in}}{\pgfqpoint{4.022045in}{3.412368in}}{\pgfqpoint{4.022045in}{3.408250in}}%
\pgfpathcurveto{\pgfqpoint{4.022045in}{3.404132in}}{\pgfqpoint{4.023681in}{3.400182in}}{\pgfqpoint{4.026593in}{3.397270in}}%
\pgfpathcurveto{\pgfqpoint{4.029505in}{3.394358in}}{\pgfqpoint{4.033455in}{3.392722in}}{\pgfqpoint{4.037573in}{3.392722in}}%
\pgfpathclose%
\pgfusepath{fill}%
\end{pgfscope}%
\begin{pgfscope}%
\pgfpathrectangle{\pgfqpoint{0.887500in}{0.275000in}}{\pgfqpoint{4.225000in}{4.225000in}}%
\pgfusepath{clip}%
\pgfsetbuttcap%
\pgfsetroundjoin%
\definecolor{currentfill}{rgb}{0.000000,0.000000,0.000000}%
\pgfsetfillcolor{currentfill}%
\pgfsetfillopacity{0.800000}%
\pgfsetlinewidth{0.000000pt}%
\definecolor{currentstroke}{rgb}{0.000000,0.000000,0.000000}%
\pgfsetstrokecolor{currentstroke}%
\pgfsetstrokeopacity{0.800000}%
\pgfsetdash{}{0pt}%
\pgfpathmoveto{\pgfqpoint{3.402209in}{3.547979in}}%
\pgfpathcurveto{\pgfqpoint{3.406327in}{3.547979in}}{\pgfqpoint{3.410277in}{3.549616in}}{\pgfqpoint{3.413189in}{3.552528in}}%
\pgfpathcurveto{\pgfqpoint{3.416101in}{3.555440in}}{\pgfqpoint{3.417737in}{3.559390in}}{\pgfqpoint{3.417737in}{3.563508in}}%
\pgfpathcurveto{\pgfqpoint{3.417737in}{3.567626in}}{\pgfqpoint{3.416101in}{3.571576in}}{\pgfqpoint{3.413189in}{3.574488in}}%
\pgfpathcurveto{\pgfqpoint{3.410277in}{3.577400in}}{\pgfqpoint{3.406327in}{3.579036in}}{\pgfqpoint{3.402209in}{3.579036in}}%
\pgfpathcurveto{\pgfqpoint{3.398090in}{3.579036in}}{\pgfqpoint{3.394140in}{3.577400in}}{\pgfqpoint{3.391228in}{3.574488in}}%
\pgfpathcurveto{\pgfqpoint{3.388316in}{3.571576in}}{\pgfqpoint{3.386680in}{3.567626in}}{\pgfqpoint{3.386680in}{3.563508in}}%
\pgfpathcurveto{\pgfqpoint{3.386680in}{3.559390in}}{\pgfqpoint{3.388316in}{3.555440in}}{\pgfqpoint{3.391228in}{3.552528in}}%
\pgfpathcurveto{\pgfqpoint{3.394140in}{3.549616in}}{\pgfqpoint{3.398090in}{3.547979in}}{\pgfqpoint{3.402209in}{3.547979in}}%
\pgfpathclose%
\pgfusepath{fill}%
\end{pgfscope}%
\begin{pgfscope}%
\pgfpathrectangle{\pgfqpoint{0.887500in}{0.275000in}}{\pgfqpoint{4.225000in}{4.225000in}}%
\pgfusepath{clip}%
\pgfsetbuttcap%
\pgfsetroundjoin%
\definecolor{currentfill}{rgb}{0.000000,0.000000,0.000000}%
\pgfsetfillcolor{currentfill}%
\pgfsetfillopacity{0.800000}%
\pgfsetlinewidth{0.000000pt}%
\definecolor{currentstroke}{rgb}{0.000000,0.000000,0.000000}%
\pgfsetstrokecolor{currentstroke}%
\pgfsetstrokeopacity{0.800000}%
\pgfsetdash{}{0pt}%
\pgfpathmoveto{\pgfqpoint{2.562176in}{2.878330in}}%
\pgfpathcurveto{\pgfqpoint{2.566295in}{2.878330in}}{\pgfqpoint{2.570245in}{2.879966in}}{\pgfqpoint{2.573157in}{2.882878in}}%
\pgfpathcurveto{\pgfqpoint{2.576069in}{2.885790in}}{\pgfqpoint{2.577705in}{2.889740in}}{\pgfqpoint{2.577705in}{2.893858in}}%
\pgfpathcurveto{\pgfqpoint{2.577705in}{2.897976in}}{\pgfqpoint{2.576069in}{2.901927in}}{\pgfqpoint{2.573157in}{2.904838in}}%
\pgfpathcurveto{\pgfqpoint{2.570245in}{2.907750in}}{\pgfqpoint{2.566295in}{2.909387in}}{\pgfqpoint{2.562176in}{2.909387in}}%
\pgfpathcurveto{\pgfqpoint{2.558058in}{2.909387in}}{\pgfqpoint{2.554108in}{2.907750in}}{\pgfqpoint{2.551196in}{2.904838in}}%
\pgfpathcurveto{\pgfqpoint{2.548284in}{2.901927in}}{\pgfqpoint{2.546648in}{2.897976in}}{\pgfqpoint{2.546648in}{2.893858in}}%
\pgfpathcurveto{\pgfqpoint{2.546648in}{2.889740in}}{\pgfqpoint{2.548284in}{2.885790in}}{\pgfqpoint{2.551196in}{2.882878in}}%
\pgfpathcurveto{\pgfqpoint{2.554108in}{2.879966in}}{\pgfqpoint{2.558058in}{2.878330in}}{\pgfqpoint{2.562176in}{2.878330in}}%
\pgfpathclose%
\pgfusepath{fill}%
\end{pgfscope}%
\begin{pgfscope}%
\pgfpathrectangle{\pgfqpoint{0.887500in}{0.275000in}}{\pgfqpoint{4.225000in}{4.225000in}}%
\pgfusepath{clip}%
\pgfsetbuttcap%
\pgfsetroundjoin%
\definecolor{currentfill}{rgb}{0.000000,0.000000,0.000000}%
\pgfsetfillcolor{currentfill}%
\pgfsetfillopacity{0.800000}%
\pgfsetlinewidth{0.000000pt}%
\definecolor{currentstroke}{rgb}{0.000000,0.000000,0.000000}%
\pgfsetstrokecolor{currentstroke}%
\pgfsetstrokeopacity{0.800000}%
\pgfsetdash{}{0pt}%
\pgfpathmoveto{\pgfqpoint{2.451123in}{2.762645in}}%
\pgfpathcurveto{\pgfqpoint{2.455241in}{2.762645in}}{\pgfqpoint{2.459191in}{2.764282in}}{\pgfqpoint{2.462103in}{2.767193in}}%
\pgfpathcurveto{\pgfqpoint{2.465015in}{2.770105in}}{\pgfqpoint{2.466651in}{2.774055in}}{\pgfqpoint{2.466651in}{2.778174in}}%
\pgfpathcurveto{\pgfqpoint{2.466651in}{2.782292in}}{\pgfqpoint{2.465015in}{2.786242in}}{\pgfqpoint{2.462103in}{2.789154in}}%
\pgfpathcurveto{\pgfqpoint{2.459191in}{2.792066in}}{\pgfqpoint{2.455241in}{2.793702in}}{\pgfqpoint{2.451123in}{2.793702in}}%
\pgfpathcurveto{\pgfqpoint{2.447005in}{2.793702in}}{\pgfqpoint{2.443055in}{2.792066in}}{\pgfqpoint{2.440143in}{2.789154in}}%
\pgfpathcurveto{\pgfqpoint{2.437231in}{2.786242in}}{\pgfqpoint{2.435595in}{2.782292in}}{\pgfqpoint{2.435595in}{2.778174in}}%
\pgfpathcurveto{\pgfqpoint{2.435595in}{2.774055in}}{\pgfqpoint{2.437231in}{2.770105in}}{\pgfqpoint{2.440143in}{2.767193in}}%
\pgfpathcurveto{\pgfqpoint{2.443055in}{2.764282in}}{\pgfqpoint{2.447005in}{2.762645in}}{\pgfqpoint{2.451123in}{2.762645in}}%
\pgfpathclose%
\pgfusepath{fill}%
\end{pgfscope}%
\begin{pgfscope}%
\pgfpathrectangle{\pgfqpoint{0.887500in}{0.275000in}}{\pgfqpoint{4.225000in}{4.225000in}}%
\pgfusepath{clip}%
\pgfsetbuttcap%
\pgfsetroundjoin%
\definecolor{currentfill}{rgb}{0.000000,0.000000,0.000000}%
\pgfsetfillcolor{currentfill}%
\pgfsetfillopacity{0.800000}%
\pgfsetlinewidth{0.000000pt}%
\definecolor{currentstroke}{rgb}{0.000000,0.000000,0.000000}%
\pgfsetstrokecolor{currentstroke}%
\pgfsetstrokeopacity{0.800000}%
\pgfsetdash{}{0pt}%
\pgfpathmoveto{\pgfqpoint{3.116534in}{3.536396in}}%
\pgfpathcurveto{\pgfqpoint{3.120652in}{3.536396in}}{\pgfqpoint{3.124602in}{3.538032in}}{\pgfqpoint{3.127514in}{3.540944in}}%
\pgfpathcurveto{\pgfqpoint{3.130426in}{3.543856in}}{\pgfqpoint{3.132062in}{3.547806in}}{\pgfqpoint{3.132062in}{3.551924in}}%
\pgfpathcurveto{\pgfqpoint{3.132062in}{3.556042in}}{\pgfqpoint{3.130426in}{3.559992in}}{\pgfqpoint{3.127514in}{3.562904in}}%
\pgfpathcurveto{\pgfqpoint{3.124602in}{3.565816in}}{\pgfqpoint{3.120652in}{3.567452in}}{\pgfqpoint{3.116534in}{3.567452in}}%
\pgfpathcurveto{\pgfqpoint{3.112416in}{3.567452in}}{\pgfqpoint{3.108466in}{3.565816in}}{\pgfqpoint{3.105554in}{3.562904in}}%
\pgfpathcurveto{\pgfqpoint{3.102642in}{3.559992in}}{\pgfqpoint{3.101006in}{3.556042in}}{\pgfqpoint{3.101006in}{3.551924in}}%
\pgfpathcurveto{\pgfqpoint{3.101006in}{3.547806in}}{\pgfqpoint{3.102642in}{3.543856in}}{\pgfqpoint{3.105554in}{3.540944in}}%
\pgfpathcurveto{\pgfqpoint{3.108466in}{3.538032in}}{\pgfqpoint{3.112416in}{3.536396in}}{\pgfqpoint{3.116534in}{3.536396in}}%
\pgfpathclose%
\pgfusepath{fill}%
\end{pgfscope}%
\begin{pgfscope}%
\pgfpathrectangle{\pgfqpoint{0.887500in}{0.275000in}}{\pgfqpoint{4.225000in}{4.225000in}}%
\pgfusepath{clip}%
\pgfsetbuttcap%
\pgfsetroundjoin%
\definecolor{currentfill}{rgb}{0.000000,0.000000,0.000000}%
\pgfsetfillcolor{currentfill}%
\pgfsetfillopacity{0.800000}%
\pgfsetlinewidth{0.000000pt}%
\definecolor{currentstroke}{rgb}{0.000000,0.000000,0.000000}%
\pgfsetstrokecolor{currentstroke}%
\pgfsetstrokeopacity{0.800000}%
\pgfsetdash{}{0pt}%
\pgfpathmoveto{\pgfqpoint{2.165092in}{2.737341in}}%
\pgfpathcurveto{\pgfqpoint{2.169210in}{2.737341in}}{\pgfqpoint{2.173160in}{2.738977in}}{\pgfqpoint{2.176072in}{2.741889in}}%
\pgfpathcurveto{\pgfqpoint{2.178984in}{2.744801in}}{\pgfqpoint{2.180620in}{2.748751in}}{\pgfqpoint{2.180620in}{2.752869in}}%
\pgfpathcurveto{\pgfqpoint{2.180620in}{2.756988in}}{\pgfqpoint{2.178984in}{2.760938in}}{\pgfqpoint{2.176072in}{2.763850in}}%
\pgfpathcurveto{\pgfqpoint{2.173160in}{2.766762in}}{\pgfqpoint{2.169210in}{2.768398in}}{\pgfqpoint{2.165092in}{2.768398in}}%
\pgfpathcurveto{\pgfqpoint{2.160974in}{2.768398in}}{\pgfqpoint{2.157024in}{2.766762in}}{\pgfqpoint{2.154112in}{2.763850in}}%
\pgfpathcurveto{\pgfqpoint{2.151200in}{2.760938in}}{\pgfqpoint{2.149564in}{2.756988in}}{\pgfqpoint{2.149564in}{2.752869in}}%
\pgfpathcurveto{\pgfqpoint{2.149564in}{2.748751in}}{\pgfqpoint{2.151200in}{2.744801in}}{\pgfqpoint{2.154112in}{2.741889in}}%
\pgfpathcurveto{\pgfqpoint{2.157024in}{2.738977in}}{\pgfqpoint{2.160974in}{2.737341in}}{\pgfqpoint{2.165092in}{2.737341in}}%
\pgfpathclose%
\pgfusepath{fill}%
\end{pgfscope}%
\begin{pgfscope}%
\pgfpathrectangle{\pgfqpoint{0.887500in}{0.275000in}}{\pgfqpoint{4.225000in}{4.225000in}}%
\pgfusepath{clip}%
\pgfsetbuttcap%
\pgfsetroundjoin%
\definecolor{currentfill}{rgb}{0.000000,0.000000,0.000000}%
\pgfsetfillcolor{currentfill}%
\pgfsetfillopacity{0.800000}%
\pgfsetlinewidth{0.000000pt}%
\definecolor{currentstroke}{rgb}{0.000000,0.000000,0.000000}%
\pgfsetstrokecolor{currentstroke}%
\pgfsetstrokeopacity{0.800000}%
\pgfsetdash{}{0pt}%
\pgfpathmoveto{\pgfqpoint{1.704160in}{2.798821in}}%
\pgfpathcurveto{\pgfqpoint{1.708278in}{2.798821in}}{\pgfqpoint{1.712228in}{2.800457in}}{\pgfqpoint{1.715140in}{2.803369in}}%
\pgfpathcurveto{\pgfqpoint{1.718052in}{2.806281in}}{\pgfqpoint{1.719689in}{2.810231in}}{\pgfqpoint{1.719689in}{2.814349in}}%
\pgfpathcurveto{\pgfqpoint{1.719689in}{2.818467in}}{\pgfqpoint{1.718052in}{2.822417in}}{\pgfqpoint{1.715140in}{2.825329in}}%
\pgfpathcurveto{\pgfqpoint{1.712228in}{2.828241in}}{\pgfqpoint{1.708278in}{2.829877in}}{\pgfqpoint{1.704160in}{2.829877in}}%
\pgfpathcurveto{\pgfqpoint{1.700042in}{2.829877in}}{\pgfqpoint{1.696092in}{2.828241in}}{\pgfqpoint{1.693180in}{2.825329in}}%
\pgfpathcurveto{\pgfqpoint{1.690268in}{2.822417in}}{\pgfqpoint{1.688632in}{2.818467in}}{\pgfqpoint{1.688632in}{2.814349in}}%
\pgfpathcurveto{\pgfqpoint{1.688632in}{2.810231in}}{\pgfqpoint{1.690268in}{2.806281in}}{\pgfqpoint{1.693180in}{2.803369in}}%
\pgfpathcurveto{\pgfqpoint{1.696092in}{2.800457in}}{\pgfqpoint{1.700042in}{2.798821in}}{\pgfqpoint{1.704160in}{2.798821in}}%
\pgfpathclose%
\pgfusepath{fill}%
\end{pgfscope}%
\begin{pgfscope}%
\pgfpathrectangle{\pgfqpoint{0.887500in}{0.275000in}}{\pgfqpoint{4.225000in}{4.225000in}}%
\pgfusepath{clip}%
\pgfsetbuttcap%
\pgfsetroundjoin%
\definecolor{currentfill}{rgb}{0.000000,0.000000,0.000000}%
\pgfsetfillcolor{currentfill}%
\pgfsetfillopacity{0.800000}%
\pgfsetlinewidth{0.000000pt}%
\definecolor{currentstroke}{rgb}{0.000000,0.000000,0.000000}%
\pgfsetstrokecolor{currentstroke}%
\pgfsetstrokeopacity{0.800000}%
\pgfsetdash{}{0pt}%
\pgfpathmoveto{\pgfqpoint{2.339268in}{2.687509in}}%
\pgfpathcurveto{\pgfqpoint{2.343386in}{2.687509in}}{\pgfqpoint{2.347336in}{2.689145in}}{\pgfqpoint{2.350248in}{2.692057in}}%
\pgfpathcurveto{\pgfqpoint{2.353160in}{2.694969in}}{\pgfqpoint{2.354797in}{2.698919in}}{\pgfqpoint{2.354797in}{2.703037in}}%
\pgfpathcurveto{\pgfqpoint{2.354797in}{2.707155in}}{\pgfqpoint{2.353160in}{2.711105in}}{\pgfqpoint{2.350248in}{2.714017in}}%
\pgfpathcurveto{\pgfqpoint{2.347336in}{2.716929in}}{\pgfqpoint{2.343386in}{2.718565in}}{\pgfqpoint{2.339268in}{2.718565in}}%
\pgfpathcurveto{\pgfqpoint{2.335150in}{2.718565in}}{\pgfqpoint{2.331200in}{2.716929in}}{\pgfqpoint{2.328288in}{2.714017in}}%
\pgfpathcurveto{\pgfqpoint{2.325376in}{2.711105in}}{\pgfqpoint{2.323740in}{2.707155in}}{\pgfqpoint{2.323740in}{2.703037in}}%
\pgfpathcurveto{\pgfqpoint{2.323740in}{2.698919in}}{\pgfqpoint{2.325376in}{2.694969in}}{\pgfqpoint{2.328288in}{2.692057in}}%
\pgfpathcurveto{\pgfqpoint{2.331200in}{2.689145in}}{\pgfqpoint{2.335150in}{2.687509in}}{\pgfqpoint{2.339268in}{2.687509in}}%
\pgfpathclose%
\pgfusepath{fill}%
\end{pgfscope}%
\begin{pgfscope}%
\pgfpathrectangle{\pgfqpoint{0.887500in}{0.275000in}}{\pgfqpoint{4.225000in}{4.225000in}}%
\pgfusepath{clip}%
\pgfsetbuttcap%
\pgfsetroundjoin%
\definecolor{currentfill}{rgb}{0.000000,0.000000,0.000000}%
\pgfsetfillcolor{currentfill}%
\pgfsetfillopacity{0.800000}%
\pgfsetlinewidth{0.000000pt}%
\definecolor{currentstroke}{rgb}{0.000000,0.000000,0.000000}%
\pgfsetstrokecolor{currentstroke}%
\pgfsetstrokeopacity{0.800000}%
\pgfsetdash{}{0pt}%
\pgfpathmoveto{\pgfqpoint{1.416375in}{2.810178in}}%
\pgfpathcurveto{\pgfqpoint{1.420493in}{2.810178in}}{\pgfqpoint{1.424443in}{2.811814in}}{\pgfqpoint{1.427355in}{2.814726in}}%
\pgfpathcurveto{\pgfqpoint{1.430267in}{2.817638in}}{\pgfqpoint{1.431903in}{2.821588in}}{\pgfqpoint{1.431903in}{2.825706in}}%
\pgfpathcurveto{\pgfqpoint{1.431903in}{2.829824in}}{\pgfqpoint{1.430267in}{2.833774in}}{\pgfqpoint{1.427355in}{2.836686in}}%
\pgfpathcurveto{\pgfqpoint{1.424443in}{2.839598in}}{\pgfqpoint{1.420493in}{2.841234in}}{\pgfqpoint{1.416375in}{2.841234in}}%
\pgfpathcurveto{\pgfqpoint{1.412257in}{2.841234in}}{\pgfqpoint{1.408307in}{2.839598in}}{\pgfqpoint{1.405395in}{2.836686in}}%
\pgfpathcurveto{\pgfqpoint{1.402483in}{2.833774in}}{\pgfqpoint{1.400847in}{2.829824in}}{\pgfqpoint{1.400847in}{2.825706in}}%
\pgfpathcurveto{\pgfqpoint{1.400847in}{2.821588in}}{\pgfqpoint{1.402483in}{2.817638in}}{\pgfqpoint{1.405395in}{2.814726in}}%
\pgfpathcurveto{\pgfqpoint{1.408307in}{2.811814in}}{\pgfqpoint{1.412257in}{2.810178in}}{\pgfqpoint{1.416375in}{2.810178in}}%
\pgfpathclose%
\pgfusepath{fill}%
\end{pgfscope}%
\begin{pgfscope}%
\pgfpathrectangle{\pgfqpoint{0.887500in}{0.275000in}}{\pgfqpoint{4.225000in}{4.225000in}}%
\pgfusepath{clip}%
\pgfsetbuttcap%
\pgfsetroundjoin%
\definecolor{currentfill}{rgb}{0.000000,0.000000,0.000000}%
\pgfsetfillcolor{currentfill}%
\pgfsetfillopacity{0.800000}%
\pgfsetlinewidth{0.000000pt}%
\definecolor{currentstroke}{rgb}{0.000000,0.000000,0.000000}%
\pgfsetstrokecolor{currentstroke}%
\pgfsetstrokeopacity{0.800000}%
\pgfsetdash{}{0pt}%
\pgfpathmoveto{\pgfqpoint{1.877753in}{2.751865in}}%
\pgfpathcurveto{\pgfqpoint{1.881871in}{2.751865in}}{\pgfqpoint{1.885821in}{2.753501in}}{\pgfqpoint{1.888733in}{2.756413in}}%
\pgfpathcurveto{\pgfqpoint{1.891645in}{2.759325in}}{\pgfqpoint{1.893282in}{2.763275in}}{\pgfqpoint{1.893282in}{2.767393in}}%
\pgfpathcurveto{\pgfqpoint{1.893282in}{2.771512in}}{\pgfqpoint{1.891645in}{2.775462in}}{\pgfqpoint{1.888733in}{2.778374in}}%
\pgfpathcurveto{\pgfqpoint{1.885821in}{2.781286in}}{\pgfqpoint{1.881871in}{2.782922in}}{\pgfqpoint{1.877753in}{2.782922in}}%
\pgfpathcurveto{\pgfqpoint{1.873635in}{2.782922in}}{\pgfqpoint{1.869685in}{2.781286in}}{\pgfqpoint{1.866773in}{2.778374in}}%
\pgfpathcurveto{\pgfqpoint{1.863861in}{2.775462in}}{\pgfqpoint{1.862225in}{2.771512in}}{\pgfqpoint{1.862225in}{2.767393in}}%
\pgfpathcurveto{\pgfqpoint{1.862225in}{2.763275in}}{\pgfqpoint{1.863861in}{2.759325in}}{\pgfqpoint{1.866773in}{2.756413in}}%
\pgfpathcurveto{\pgfqpoint{1.869685in}{2.753501in}}{\pgfqpoint{1.873635in}{2.751865in}}{\pgfqpoint{1.877753in}{2.751865in}}%
\pgfpathclose%
\pgfusepath{fill}%
\end{pgfscope}%
\begin{pgfscope}%
\pgfpathrectangle{\pgfqpoint{0.887500in}{0.275000in}}{\pgfqpoint{4.225000in}{4.225000in}}%
\pgfusepath{clip}%
\pgfsetbuttcap%
\pgfsetroundjoin%
\definecolor{currentfill}{rgb}{0.000000,0.000000,0.000000}%
\pgfsetfillcolor{currentfill}%
\pgfsetfillopacity{0.800000}%
\pgfsetlinewidth{0.000000pt}%
\definecolor{currentstroke}{rgb}{0.000000,0.000000,0.000000}%
\pgfsetstrokecolor{currentstroke}%
\pgfsetstrokeopacity{0.800000}%
\pgfsetdash{}{0pt}%
\pgfpathmoveto{\pgfqpoint{3.753229in}{3.436968in}}%
\pgfpathcurveto{\pgfqpoint{3.757347in}{3.436968in}}{\pgfqpoint{3.761297in}{3.438604in}}{\pgfqpoint{3.764209in}{3.441516in}}%
\pgfpathcurveto{\pgfqpoint{3.767121in}{3.444428in}}{\pgfqpoint{3.768757in}{3.448378in}}{\pgfqpoint{3.768757in}{3.452497in}}%
\pgfpathcurveto{\pgfqpoint{3.768757in}{3.456615in}}{\pgfqpoint{3.767121in}{3.460565in}}{\pgfqpoint{3.764209in}{3.463477in}}%
\pgfpathcurveto{\pgfqpoint{3.761297in}{3.466389in}}{\pgfqpoint{3.757347in}{3.468025in}}{\pgfqpoint{3.753229in}{3.468025in}}%
\pgfpathcurveto{\pgfqpoint{3.749110in}{3.468025in}}{\pgfqpoint{3.745160in}{3.466389in}}{\pgfqpoint{3.742249in}{3.463477in}}%
\pgfpathcurveto{\pgfqpoint{3.739337in}{3.460565in}}{\pgfqpoint{3.737700in}{3.456615in}}{\pgfqpoint{3.737700in}{3.452497in}}%
\pgfpathcurveto{\pgfqpoint{3.737700in}{3.448378in}}{\pgfqpoint{3.739337in}{3.444428in}}{\pgfqpoint{3.742249in}{3.441516in}}%
\pgfpathcurveto{\pgfqpoint{3.745160in}{3.438604in}}{\pgfqpoint{3.749110in}{3.436968in}}{\pgfqpoint{3.753229in}{3.436968in}}%
\pgfpathclose%
\pgfusepath{fill}%
\end{pgfscope}%
\begin{pgfscope}%
\pgfpathrectangle{\pgfqpoint{0.887500in}{0.275000in}}{\pgfqpoint{4.225000in}{4.225000in}}%
\pgfusepath{clip}%
\pgfsetbuttcap%
\pgfsetroundjoin%
\definecolor{currentfill}{rgb}{0.000000,0.000000,0.000000}%
\pgfsetfillcolor{currentfill}%
\pgfsetfillopacity{0.800000}%
\pgfsetlinewidth{0.000000pt}%
\definecolor{currentstroke}{rgb}{0.000000,0.000000,0.000000}%
\pgfsetstrokecolor{currentstroke}%
\pgfsetstrokeopacity{0.800000}%
\pgfsetdash{}{0pt}%
\pgfpathmoveto{\pgfqpoint{3.928616in}{3.352518in}}%
\pgfpathcurveto{\pgfqpoint{3.932734in}{3.352518in}}{\pgfqpoint{3.936684in}{3.354154in}}{\pgfqpoint{3.939596in}{3.357066in}}%
\pgfpathcurveto{\pgfqpoint{3.942508in}{3.359978in}}{\pgfqpoint{3.944144in}{3.363928in}}{\pgfqpoint{3.944144in}{3.368046in}}%
\pgfpathcurveto{\pgfqpoint{3.944144in}{3.372165in}}{\pgfqpoint{3.942508in}{3.376115in}}{\pgfqpoint{3.939596in}{3.379027in}}%
\pgfpathcurveto{\pgfqpoint{3.936684in}{3.381939in}}{\pgfqpoint{3.932734in}{3.383575in}}{\pgfqpoint{3.928616in}{3.383575in}}%
\pgfpathcurveto{\pgfqpoint{3.924498in}{3.383575in}}{\pgfqpoint{3.920548in}{3.381939in}}{\pgfqpoint{3.917636in}{3.379027in}}%
\pgfpathcurveto{\pgfqpoint{3.914724in}{3.376115in}}{\pgfqpoint{3.913088in}{3.372165in}}{\pgfqpoint{3.913088in}{3.368046in}}%
\pgfpathcurveto{\pgfqpoint{3.913088in}{3.363928in}}{\pgfqpoint{3.914724in}{3.359978in}}{\pgfqpoint{3.917636in}{3.357066in}}%
\pgfpathcurveto{\pgfqpoint{3.920548in}{3.354154in}}{\pgfqpoint{3.924498in}{3.352518in}}{\pgfqpoint{3.928616in}{3.352518in}}%
\pgfpathclose%
\pgfusepath{fill}%
\end{pgfscope}%
\begin{pgfscope}%
\pgfpathrectangle{\pgfqpoint{0.887500in}{0.275000in}}{\pgfqpoint{4.225000in}{4.225000in}}%
\pgfusepath{clip}%
\pgfsetbuttcap%
\pgfsetroundjoin%
\definecolor{currentfill}{rgb}{0.000000,0.000000,0.000000}%
\pgfsetfillcolor{currentfill}%
\pgfsetfillopacity{0.800000}%
\pgfsetlinewidth{0.000000pt}%
\definecolor{currentstroke}{rgb}{0.000000,0.000000,0.000000}%
\pgfsetstrokecolor{currentstroke}%
\pgfsetstrokeopacity{0.800000}%
\pgfsetdash{}{0pt}%
\pgfpathmoveto{\pgfqpoint{4.103857in}{3.259222in}}%
\pgfpathcurveto{\pgfqpoint{4.107975in}{3.259222in}}{\pgfqpoint{4.111925in}{3.260858in}}{\pgfqpoint{4.114837in}{3.263770in}}%
\pgfpathcurveto{\pgfqpoint{4.117749in}{3.266682in}}{\pgfqpoint{4.119385in}{3.270632in}}{\pgfqpoint{4.119385in}{3.274750in}}%
\pgfpathcurveto{\pgfqpoint{4.119385in}{3.278869in}}{\pgfqpoint{4.117749in}{3.282819in}}{\pgfqpoint{4.114837in}{3.285731in}}%
\pgfpathcurveto{\pgfqpoint{4.111925in}{3.288643in}}{\pgfqpoint{4.107975in}{3.290279in}}{\pgfqpoint{4.103857in}{3.290279in}}%
\pgfpathcurveto{\pgfqpoint{4.099739in}{3.290279in}}{\pgfqpoint{4.095789in}{3.288643in}}{\pgfqpoint{4.092877in}{3.285731in}}%
\pgfpathcurveto{\pgfqpoint{4.089965in}{3.282819in}}{\pgfqpoint{4.088329in}{3.278869in}}{\pgfqpoint{4.088329in}{3.274750in}}%
\pgfpathcurveto{\pgfqpoint{4.088329in}{3.270632in}}{\pgfqpoint{4.089965in}{3.266682in}}{\pgfqpoint{4.092877in}{3.263770in}}%
\pgfpathcurveto{\pgfqpoint{4.095789in}{3.260858in}}{\pgfqpoint{4.099739in}{3.259222in}}{\pgfqpoint{4.103857in}{3.259222in}}%
\pgfpathclose%
\pgfusepath{fill}%
\end{pgfscope}%
\begin{pgfscope}%
\pgfpathrectangle{\pgfqpoint{0.887500in}{0.275000in}}{\pgfqpoint{4.225000in}{4.225000in}}%
\pgfusepath{clip}%
\pgfsetbuttcap%
\pgfsetroundjoin%
\definecolor{currentfill}{rgb}{0.000000,0.000000,0.000000}%
\pgfsetfillcolor{currentfill}%
\pgfsetfillopacity{0.800000}%
\pgfsetlinewidth{0.000000pt}%
\definecolor{currentstroke}{rgb}{0.000000,0.000000,0.000000}%
\pgfsetstrokecolor{currentstroke}%
\pgfsetstrokeopacity{0.800000}%
\pgfsetdash{}{0pt}%
\pgfpathmoveto{\pgfqpoint{2.847484in}{2.995569in}}%
\pgfpathcurveto{\pgfqpoint{2.851602in}{2.995569in}}{\pgfqpoint{2.855552in}{2.997206in}}{\pgfqpoint{2.858464in}{3.000117in}}%
\pgfpathcurveto{\pgfqpoint{2.861376in}{3.003029in}}{\pgfqpoint{2.863013in}{3.006979in}}{\pgfqpoint{2.863013in}{3.011098in}}%
\pgfpathcurveto{\pgfqpoint{2.863013in}{3.015216in}}{\pgfqpoint{2.861376in}{3.019166in}}{\pgfqpoint{2.858464in}{3.022078in}}%
\pgfpathcurveto{\pgfqpoint{2.855552in}{3.024990in}}{\pgfqpoint{2.851602in}{3.026626in}}{\pgfqpoint{2.847484in}{3.026626in}}%
\pgfpathcurveto{\pgfqpoint{2.843366in}{3.026626in}}{\pgfqpoint{2.839416in}{3.024990in}}{\pgfqpoint{2.836504in}{3.022078in}}%
\pgfpathcurveto{\pgfqpoint{2.833592in}{3.019166in}}{\pgfqpoint{2.831956in}{3.015216in}}{\pgfqpoint{2.831956in}{3.011098in}}%
\pgfpathcurveto{\pgfqpoint{2.831956in}{3.006979in}}{\pgfqpoint{2.833592in}{3.003029in}}{\pgfqpoint{2.836504in}{3.000117in}}%
\pgfpathcurveto{\pgfqpoint{2.839416in}{2.997206in}}{\pgfqpoint{2.843366in}{2.995569in}}{\pgfqpoint{2.847484in}{2.995569in}}%
\pgfpathclose%
\pgfusepath{fill}%
\end{pgfscope}%
\begin{pgfscope}%
\pgfpathrectangle{\pgfqpoint{0.887500in}{0.275000in}}{\pgfqpoint{4.225000in}{4.225000in}}%
\pgfusepath{clip}%
\pgfsetbuttcap%
\pgfsetroundjoin%
\definecolor{currentfill}{rgb}{0.000000,0.000000,0.000000}%
\pgfsetfillcolor{currentfill}%
\pgfsetfillopacity{0.800000}%
\pgfsetlinewidth{0.000000pt}%
\definecolor{currentstroke}{rgb}{0.000000,0.000000,0.000000}%
\pgfsetstrokecolor{currentstroke}%
\pgfsetstrokeopacity{0.800000}%
\pgfsetdash{}{0pt}%
\pgfpathmoveto{\pgfqpoint{2.736314in}{2.876517in}}%
\pgfpathcurveto{\pgfqpoint{2.740433in}{2.876517in}}{\pgfqpoint{2.744383in}{2.878153in}}{\pgfqpoint{2.747295in}{2.881065in}}%
\pgfpathcurveto{\pgfqpoint{2.750207in}{2.883977in}}{\pgfqpoint{2.751843in}{2.887927in}}{\pgfqpoint{2.751843in}{2.892045in}}%
\pgfpathcurveto{\pgfqpoint{2.751843in}{2.896163in}}{\pgfqpoint{2.750207in}{2.900113in}}{\pgfqpoint{2.747295in}{2.903025in}}%
\pgfpathcurveto{\pgfqpoint{2.744383in}{2.905937in}}{\pgfqpoint{2.740433in}{2.907573in}}{\pgfqpoint{2.736314in}{2.907573in}}%
\pgfpathcurveto{\pgfqpoint{2.732196in}{2.907573in}}{\pgfqpoint{2.728246in}{2.905937in}}{\pgfqpoint{2.725334in}{2.903025in}}%
\pgfpathcurveto{\pgfqpoint{2.722422in}{2.900113in}}{\pgfqpoint{2.720786in}{2.896163in}}{\pgfqpoint{2.720786in}{2.892045in}}%
\pgfpathcurveto{\pgfqpoint{2.720786in}{2.887927in}}{\pgfqpoint{2.722422in}{2.883977in}}{\pgfqpoint{2.725334in}{2.881065in}}%
\pgfpathcurveto{\pgfqpoint{2.728246in}{2.878153in}}{\pgfqpoint{2.732196in}{2.876517in}}{\pgfqpoint{2.736314in}{2.876517in}}%
\pgfpathclose%
\pgfusepath{fill}%
\end{pgfscope}%
\begin{pgfscope}%
\pgfpathrectangle{\pgfqpoint{0.887500in}{0.275000in}}{\pgfqpoint{4.225000in}{4.225000in}}%
\pgfusepath{clip}%
\pgfsetbuttcap%
\pgfsetroundjoin%
\definecolor{currentfill}{rgb}{0.000000,0.000000,0.000000}%
\pgfsetfillcolor{currentfill}%
\pgfsetfillopacity{0.800000}%
\pgfsetlinewidth{0.000000pt}%
\definecolor{currentstroke}{rgb}{0.000000,0.000000,0.000000}%
\pgfsetstrokecolor{currentstroke}%
\pgfsetstrokeopacity{0.800000}%
\pgfsetdash{}{0pt}%
\pgfpathmoveto{\pgfqpoint{3.005522in}{3.484408in}}%
\pgfpathcurveto{\pgfqpoint{3.009640in}{3.484408in}}{\pgfqpoint{3.013590in}{3.486044in}}{\pgfqpoint{3.016502in}{3.488956in}}%
\pgfpathcurveto{\pgfqpoint{3.019414in}{3.491868in}}{\pgfqpoint{3.021050in}{3.495818in}}{\pgfqpoint{3.021050in}{3.499936in}}%
\pgfpathcurveto{\pgfqpoint{3.021050in}{3.504054in}}{\pgfqpoint{3.019414in}{3.508004in}}{\pgfqpoint{3.016502in}{3.510916in}}%
\pgfpathcurveto{\pgfqpoint{3.013590in}{3.513828in}}{\pgfqpoint{3.009640in}{3.515464in}}{\pgfqpoint{3.005522in}{3.515464in}}%
\pgfpathcurveto{\pgfqpoint{3.001404in}{3.515464in}}{\pgfqpoint{2.997454in}{3.513828in}}{\pgfqpoint{2.994542in}{3.510916in}}%
\pgfpathcurveto{\pgfqpoint{2.991630in}{3.508004in}}{\pgfqpoint{2.989994in}{3.504054in}}{\pgfqpoint{2.989994in}{3.499936in}}%
\pgfpathcurveto{\pgfqpoint{2.989994in}{3.495818in}}{\pgfqpoint{2.991630in}{3.491868in}}{\pgfqpoint{2.994542in}{3.488956in}}%
\pgfpathcurveto{\pgfqpoint{2.997454in}{3.486044in}}{\pgfqpoint{3.001404in}{3.484408in}}{\pgfqpoint{3.005522in}{3.484408in}}%
\pgfpathclose%
\pgfusepath{fill}%
\end{pgfscope}%
\begin{pgfscope}%
\pgfpathrectangle{\pgfqpoint{0.887500in}{0.275000in}}{\pgfqpoint{4.225000in}{4.225000in}}%
\pgfusepath{clip}%
\pgfsetbuttcap%
\pgfsetroundjoin%
\definecolor{currentfill}{rgb}{0.000000,0.000000,0.000000}%
\pgfsetfillcolor{currentfill}%
\pgfsetfillopacity{0.800000}%
\pgfsetlinewidth{0.000000pt}%
\definecolor{currentstroke}{rgb}{0.000000,0.000000,0.000000}%
\pgfsetstrokecolor{currentstroke}%
\pgfsetstrokeopacity{0.800000}%
\pgfsetdash{}{0pt}%
\pgfpathmoveto{\pgfqpoint{3.291939in}{3.520970in}}%
\pgfpathcurveto{\pgfqpoint{3.296057in}{3.520970in}}{\pgfqpoint{3.300007in}{3.522606in}}{\pgfqpoint{3.302919in}{3.525518in}}%
\pgfpathcurveto{\pgfqpoint{3.305831in}{3.528430in}}{\pgfqpoint{3.307467in}{3.532380in}}{\pgfqpoint{3.307467in}{3.536498in}}%
\pgfpathcurveto{\pgfqpoint{3.307467in}{3.540616in}}{\pgfqpoint{3.305831in}{3.544566in}}{\pgfqpoint{3.302919in}{3.547478in}}%
\pgfpathcurveto{\pgfqpoint{3.300007in}{3.550390in}}{\pgfqpoint{3.296057in}{3.552026in}}{\pgfqpoint{3.291939in}{3.552026in}}%
\pgfpathcurveto{\pgfqpoint{3.287821in}{3.552026in}}{\pgfqpoint{3.283871in}{3.550390in}}{\pgfqpoint{3.280959in}{3.547478in}}%
\pgfpathcurveto{\pgfqpoint{3.278047in}{3.544566in}}{\pgfqpoint{3.276410in}{3.540616in}}{\pgfqpoint{3.276410in}{3.536498in}}%
\pgfpathcurveto{\pgfqpoint{3.276410in}{3.532380in}}{\pgfqpoint{3.278047in}{3.528430in}}{\pgfqpoint{3.280959in}{3.525518in}}%
\pgfpathcurveto{\pgfqpoint{3.283871in}{3.522606in}}{\pgfqpoint{3.287821in}{3.520970in}}{\pgfqpoint{3.291939in}{3.520970in}}%
\pgfpathclose%
\pgfusepath{fill}%
\end{pgfscope}%
\begin{pgfscope}%
\pgfpathrectangle{\pgfqpoint{0.887500in}{0.275000in}}{\pgfqpoint{4.225000in}{4.225000in}}%
\pgfusepath{clip}%
\pgfsetbuttcap%
\pgfsetroundjoin%
\definecolor{currentfill}{rgb}{0.000000,0.000000,0.000000}%
\pgfsetfillcolor{currentfill}%
\pgfsetfillopacity{0.800000}%
\pgfsetlinewidth{0.000000pt}%
\definecolor{currentstroke}{rgb}{0.000000,0.000000,0.000000}%
\pgfsetstrokecolor{currentstroke}%
\pgfsetstrokeopacity{0.800000}%
\pgfsetdash{}{0pt}%
\pgfpathmoveto{\pgfqpoint{2.625059in}{2.760565in}}%
\pgfpathcurveto{\pgfqpoint{2.629177in}{2.760565in}}{\pgfqpoint{2.633127in}{2.762201in}}{\pgfqpoint{2.636039in}{2.765113in}}%
\pgfpathcurveto{\pgfqpoint{2.638951in}{2.768025in}}{\pgfqpoint{2.640587in}{2.771975in}}{\pgfqpoint{2.640587in}{2.776093in}}%
\pgfpathcurveto{\pgfqpoint{2.640587in}{2.780211in}}{\pgfqpoint{2.638951in}{2.784161in}}{\pgfqpoint{2.636039in}{2.787073in}}%
\pgfpathcurveto{\pgfqpoint{2.633127in}{2.789985in}}{\pgfqpoint{2.629177in}{2.791621in}}{\pgfqpoint{2.625059in}{2.791621in}}%
\pgfpathcurveto{\pgfqpoint{2.620941in}{2.791621in}}{\pgfqpoint{2.616991in}{2.789985in}}{\pgfqpoint{2.614079in}{2.787073in}}%
\pgfpathcurveto{\pgfqpoint{2.611167in}{2.784161in}}{\pgfqpoint{2.609531in}{2.780211in}}{\pgfqpoint{2.609531in}{2.776093in}}%
\pgfpathcurveto{\pgfqpoint{2.609531in}{2.771975in}}{\pgfqpoint{2.611167in}{2.768025in}}{\pgfqpoint{2.614079in}{2.765113in}}%
\pgfpathcurveto{\pgfqpoint{2.616991in}{2.762201in}}{\pgfqpoint{2.620941in}{2.760565in}}{\pgfqpoint{2.625059in}{2.760565in}}%
\pgfpathclose%
\pgfusepath{fill}%
\end{pgfscope}%
\begin{pgfscope}%
\pgfpathrectangle{\pgfqpoint{0.887500in}{0.275000in}}{\pgfqpoint{4.225000in}{4.225000in}}%
\pgfusepath{clip}%
\pgfsetbuttcap%
\pgfsetroundjoin%
\definecolor{currentfill}{rgb}{0.000000,0.000000,0.000000}%
\pgfsetfillcolor{currentfill}%
\pgfsetfillopacity{0.800000}%
\pgfsetlinewidth{0.000000pt}%
\definecolor{currentstroke}{rgb}{0.000000,0.000000,0.000000}%
\pgfsetstrokecolor{currentstroke}%
\pgfsetstrokeopacity{0.800000}%
\pgfsetdash{}{0pt}%
\pgfpathmoveto{\pgfqpoint{2.513852in}{2.635950in}}%
\pgfpathcurveto{\pgfqpoint{2.517970in}{2.635950in}}{\pgfqpoint{2.521920in}{2.637586in}}{\pgfqpoint{2.524832in}{2.640498in}}%
\pgfpathcurveto{\pgfqpoint{2.527744in}{2.643410in}}{\pgfqpoint{2.529380in}{2.647360in}}{\pgfqpoint{2.529380in}{2.651478in}}%
\pgfpathcurveto{\pgfqpoint{2.529380in}{2.655596in}}{\pgfqpoint{2.527744in}{2.659546in}}{\pgfqpoint{2.524832in}{2.662458in}}%
\pgfpathcurveto{\pgfqpoint{2.521920in}{2.665370in}}{\pgfqpoint{2.517970in}{2.667006in}}{\pgfqpoint{2.513852in}{2.667006in}}%
\pgfpathcurveto{\pgfqpoint{2.509734in}{2.667006in}}{\pgfqpoint{2.505784in}{2.665370in}}{\pgfqpoint{2.502872in}{2.662458in}}%
\pgfpathcurveto{\pgfqpoint{2.499960in}{2.659546in}}{\pgfqpoint{2.498324in}{2.655596in}}{\pgfqpoint{2.498324in}{2.651478in}}%
\pgfpathcurveto{\pgfqpoint{2.498324in}{2.647360in}}{\pgfqpoint{2.499960in}{2.643410in}}{\pgfqpoint{2.502872in}{2.640498in}}%
\pgfpathcurveto{\pgfqpoint{2.505784in}{2.637586in}}{\pgfqpoint{2.509734in}{2.635950in}}{\pgfqpoint{2.513852in}{2.635950in}}%
\pgfpathclose%
\pgfusepath{fill}%
\end{pgfscope}%
\begin{pgfscope}%
\pgfpathrectangle{\pgfqpoint{0.887500in}{0.275000in}}{\pgfqpoint{4.225000in}{4.225000in}}%
\pgfusepath{clip}%
\pgfsetbuttcap%
\pgfsetroundjoin%
\definecolor{currentfill}{rgb}{0.000000,0.000000,0.000000}%
\pgfsetfillcolor{currentfill}%
\pgfsetfillopacity{0.800000}%
\pgfsetlinewidth{0.000000pt}%
\definecolor{currentstroke}{rgb}{0.000000,0.000000,0.000000}%
\pgfsetstrokecolor{currentstroke}%
\pgfsetstrokeopacity{0.800000}%
\pgfsetdash{}{0pt}%
\pgfpathmoveto{\pgfqpoint{2.051791in}{2.703351in}}%
\pgfpathcurveto{\pgfqpoint{2.055909in}{2.703351in}}{\pgfqpoint{2.059859in}{2.704987in}}{\pgfqpoint{2.062771in}{2.707899in}}%
\pgfpathcurveto{\pgfqpoint{2.065683in}{2.710811in}}{\pgfqpoint{2.067320in}{2.714761in}}{\pgfqpoint{2.067320in}{2.718879in}}%
\pgfpathcurveto{\pgfqpoint{2.067320in}{2.722997in}}{\pgfqpoint{2.065683in}{2.726947in}}{\pgfqpoint{2.062771in}{2.729859in}}%
\pgfpathcurveto{\pgfqpoint{2.059859in}{2.732771in}}{\pgfqpoint{2.055909in}{2.734407in}}{\pgfqpoint{2.051791in}{2.734407in}}%
\pgfpathcurveto{\pgfqpoint{2.047673in}{2.734407in}}{\pgfqpoint{2.043723in}{2.732771in}}{\pgfqpoint{2.040811in}{2.729859in}}%
\pgfpathcurveto{\pgfqpoint{2.037899in}{2.726947in}}{\pgfqpoint{2.036263in}{2.722997in}}{\pgfqpoint{2.036263in}{2.718879in}}%
\pgfpathcurveto{\pgfqpoint{2.036263in}{2.714761in}}{\pgfqpoint{2.037899in}{2.710811in}}{\pgfqpoint{2.040811in}{2.707899in}}%
\pgfpathcurveto{\pgfqpoint{2.043723in}{2.704987in}}{\pgfqpoint{2.047673in}{2.703351in}}{\pgfqpoint{2.051791in}{2.703351in}}%
\pgfpathclose%
\pgfusepath{fill}%
\end{pgfscope}%
\begin{pgfscope}%
\pgfpathrectangle{\pgfqpoint{0.887500in}{0.275000in}}{\pgfqpoint{4.225000in}{4.225000in}}%
\pgfusepath{clip}%
\pgfsetbuttcap%
\pgfsetroundjoin%
\definecolor{currentfill}{rgb}{0.000000,0.000000,0.000000}%
\pgfsetfillcolor{currentfill}%
\pgfsetfillopacity{0.800000}%
\pgfsetlinewidth{0.000000pt}%
\definecolor{currentstroke}{rgb}{0.000000,0.000000,0.000000}%
\pgfsetstrokecolor{currentstroke}%
\pgfsetstrokeopacity{0.800000}%
\pgfsetdash{}{0pt}%
\pgfpathmoveto{\pgfqpoint{1.589731in}{2.765828in}}%
\pgfpathcurveto{\pgfqpoint{1.593849in}{2.765828in}}{\pgfqpoint{1.597799in}{2.767464in}}{\pgfqpoint{1.600711in}{2.770376in}}%
\pgfpathcurveto{\pgfqpoint{1.603623in}{2.773288in}}{\pgfqpoint{1.605259in}{2.777238in}}{\pgfqpoint{1.605259in}{2.781356in}}%
\pgfpathcurveto{\pgfqpoint{1.605259in}{2.785474in}}{\pgfqpoint{1.603623in}{2.789424in}}{\pgfqpoint{1.600711in}{2.792336in}}%
\pgfpathcurveto{\pgfqpoint{1.597799in}{2.795248in}}{\pgfqpoint{1.593849in}{2.796884in}}{\pgfqpoint{1.589731in}{2.796884in}}%
\pgfpathcurveto{\pgfqpoint{1.585613in}{2.796884in}}{\pgfqpoint{1.581663in}{2.795248in}}{\pgfqpoint{1.578751in}{2.792336in}}%
\pgfpathcurveto{\pgfqpoint{1.575839in}{2.789424in}}{\pgfqpoint{1.574203in}{2.785474in}}{\pgfqpoint{1.574203in}{2.781356in}}%
\pgfpathcurveto{\pgfqpoint{1.574203in}{2.777238in}}{\pgfqpoint{1.575839in}{2.773288in}}{\pgfqpoint{1.578751in}{2.770376in}}%
\pgfpathcurveto{\pgfqpoint{1.581663in}{2.767464in}}{\pgfqpoint{1.585613in}{2.765828in}}{\pgfqpoint{1.589731in}{2.765828in}}%
\pgfpathclose%
\pgfusepath{fill}%
\end{pgfscope}%
\begin{pgfscope}%
\pgfpathrectangle{\pgfqpoint{0.887500in}{0.275000in}}{\pgfqpoint{4.225000in}{4.225000in}}%
\pgfusepath{clip}%
\pgfsetbuttcap%
\pgfsetroundjoin%
\definecolor{currentfill}{rgb}{0.000000,0.000000,0.000000}%
\pgfsetfillcolor{currentfill}%
\pgfsetfillopacity{0.800000}%
\pgfsetlinewidth{0.000000pt}%
\definecolor{currentstroke}{rgb}{0.000000,0.000000,0.000000}%
\pgfsetstrokecolor{currentstroke}%
\pgfsetstrokeopacity{0.800000}%
\pgfsetdash{}{0pt}%
\pgfpathmoveto{\pgfqpoint{3.467799in}{3.483387in}}%
\pgfpathcurveto{\pgfqpoint{3.471917in}{3.483387in}}{\pgfqpoint{3.475867in}{3.485023in}}{\pgfqpoint{3.478779in}{3.487935in}}%
\pgfpathcurveto{\pgfqpoint{3.481691in}{3.490847in}}{\pgfqpoint{3.483327in}{3.494797in}}{\pgfqpoint{3.483327in}{3.498915in}}%
\pgfpathcurveto{\pgfqpoint{3.483327in}{3.503033in}}{\pgfqpoint{3.481691in}{3.506983in}}{\pgfqpoint{3.478779in}{3.509895in}}%
\pgfpathcurveto{\pgfqpoint{3.475867in}{3.512807in}}{\pgfqpoint{3.471917in}{3.514443in}}{\pgfqpoint{3.467799in}{3.514443in}}%
\pgfpathcurveto{\pgfqpoint{3.463681in}{3.514443in}}{\pgfqpoint{3.459730in}{3.512807in}}{\pgfqpoint{3.456819in}{3.509895in}}%
\pgfpathcurveto{\pgfqpoint{3.453907in}{3.506983in}}{\pgfqpoint{3.452270in}{3.503033in}}{\pgfqpoint{3.452270in}{3.498915in}}%
\pgfpathcurveto{\pgfqpoint{3.452270in}{3.494797in}}{\pgfqpoint{3.453907in}{3.490847in}}{\pgfqpoint{3.456819in}{3.487935in}}%
\pgfpathcurveto{\pgfqpoint{3.459730in}{3.485023in}}{\pgfqpoint{3.463681in}{3.483387in}}{\pgfqpoint{3.467799in}{3.483387in}}%
\pgfpathclose%
\pgfusepath{fill}%
\end{pgfscope}%
\begin{pgfscope}%
\pgfpathrectangle{\pgfqpoint{0.887500in}{0.275000in}}{\pgfqpoint{4.225000in}{4.225000in}}%
\pgfusepath{clip}%
\pgfsetbuttcap%
\pgfsetroundjoin%
\definecolor{currentfill}{rgb}{0.000000,0.000000,0.000000}%
\pgfsetfillcolor{currentfill}%
\pgfsetfillopacity{0.800000}%
\pgfsetlinewidth{0.000000pt}%
\definecolor{currentstroke}{rgb}{0.000000,0.000000,0.000000}%
\pgfsetstrokecolor{currentstroke}%
\pgfsetstrokeopacity{0.800000}%
\pgfsetdash{}{0pt}%
\pgfpathmoveto{\pgfqpoint{2.226253in}{2.653447in}}%
\pgfpathcurveto{\pgfqpoint{2.230372in}{2.653447in}}{\pgfqpoint{2.234322in}{2.655084in}}{\pgfqpoint{2.237234in}{2.657996in}}%
\pgfpathcurveto{\pgfqpoint{2.240146in}{2.660908in}}{\pgfqpoint{2.241782in}{2.664858in}}{\pgfqpoint{2.241782in}{2.668976in}}%
\pgfpathcurveto{\pgfqpoint{2.241782in}{2.673094in}}{\pgfqpoint{2.240146in}{2.677044in}}{\pgfqpoint{2.237234in}{2.679956in}}%
\pgfpathcurveto{\pgfqpoint{2.234322in}{2.682868in}}{\pgfqpoint{2.230372in}{2.684504in}}{\pgfqpoint{2.226253in}{2.684504in}}%
\pgfpathcurveto{\pgfqpoint{2.222135in}{2.684504in}}{\pgfqpoint{2.218185in}{2.682868in}}{\pgfqpoint{2.215273in}{2.679956in}}%
\pgfpathcurveto{\pgfqpoint{2.212361in}{2.677044in}}{\pgfqpoint{2.210725in}{2.673094in}}{\pgfqpoint{2.210725in}{2.668976in}}%
\pgfpathcurveto{\pgfqpoint{2.210725in}{2.664858in}}{\pgfqpoint{2.212361in}{2.660908in}}{\pgfqpoint{2.215273in}{2.657996in}}%
\pgfpathcurveto{\pgfqpoint{2.218185in}{2.655084in}}{\pgfqpoint{2.222135in}{2.653447in}}{\pgfqpoint{2.226253in}{2.653447in}}%
\pgfpathclose%
\pgfusepath{fill}%
\end{pgfscope}%
\begin{pgfscope}%
\pgfpathrectangle{\pgfqpoint{0.887500in}{0.275000in}}{\pgfqpoint{4.225000in}{4.225000in}}%
\pgfusepath{clip}%
\pgfsetbuttcap%
\pgfsetroundjoin%
\definecolor{currentfill}{rgb}{0.000000,0.000000,0.000000}%
\pgfsetfillcolor{currentfill}%
\pgfsetfillopacity{0.800000}%
\pgfsetlinewidth{0.000000pt}%
\definecolor{currentstroke}{rgb}{0.000000,0.000000,0.000000}%
\pgfsetstrokecolor{currentstroke}%
\pgfsetstrokeopacity{0.800000}%
\pgfsetdash{}{0pt}%
\pgfpathmoveto{\pgfqpoint{1.301171in}{2.776148in}}%
\pgfpathcurveto{\pgfqpoint{1.305289in}{2.776148in}}{\pgfqpoint{1.309239in}{2.777785in}}{\pgfqpoint{1.312151in}{2.780697in}}%
\pgfpathcurveto{\pgfqpoint{1.315063in}{2.783609in}}{\pgfqpoint{1.316699in}{2.787559in}}{\pgfqpoint{1.316699in}{2.791677in}}%
\pgfpathcurveto{\pgfqpoint{1.316699in}{2.795795in}}{\pgfqpoint{1.315063in}{2.799745in}}{\pgfqpoint{1.312151in}{2.802657in}}%
\pgfpathcurveto{\pgfqpoint{1.309239in}{2.805569in}}{\pgfqpoint{1.305289in}{2.807205in}}{\pgfqpoint{1.301171in}{2.807205in}}%
\pgfpathcurveto{\pgfqpoint{1.297053in}{2.807205in}}{\pgfqpoint{1.293103in}{2.805569in}}{\pgfqpoint{1.290191in}{2.802657in}}%
\pgfpathcurveto{\pgfqpoint{1.287279in}{2.799745in}}{\pgfqpoint{1.285643in}{2.795795in}}{\pgfqpoint{1.285643in}{2.791677in}}%
\pgfpathcurveto{\pgfqpoint{1.285643in}{2.787559in}}{\pgfqpoint{1.287279in}{2.783609in}}{\pgfqpoint{1.290191in}{2.780697in}}%
\pgfpathcurveto{\pgfqpoint{1.293103in}{2.777785in}}{\pgfqpoint{1.297053in}{2.776148in}}{\pgfqpoint{1.301171in}{2.776148in}}%
\pgfpathclose%
\pgfusepath{fill}%
\end{pgfscope}%
\begin{pgfscope}%
\pgfpathrectangle{\pgfqpoint{0.887500in}{0.275000in}}{\pgfqpoint{4.225000in}{4.225000in}}%
\pgfusepath{clip}%
\pgfsetbuttcap%
\pgfsetroundjoin%
\definecolor{currentfill}{rgb}{0.000000,0.000000,0.000000}%
\pgfsetfillcolor{currentfill}%
\pgfsetfillopacity{0.800000}%
\pgfsetlinewidth{0.000000pt}%
\definecolor{currentstroke}{rgb}{0.000000,0.000000,0.000000}%
\pgfsetstrokecolor{currentstroke}%
\pgfsetstrokeopacity{0.800000}%
\pgfsetdash{}{0pt}%
\pgfpathmoveto{\pgfqpoint{1.763596in}{2.718977in}}%
\pgfpathcurveto{\pgfqpoint{1.767714in}{2.718977in}}{\pgfqpoint{1.771664in}{2.720613in}}{\pgfqpoint{1.774576in}{2.723525in}}%
\pgfpathcurveto{\pgfqpoint{1.777488in}{2.726437in}}{\pgfqpoint{1.779124in}{2.730387in}}{\pgfqpoint{1.779124in}{2.734505in}}%
\pgfpathcurveto{\pgfqpoint{1.779124in}{2.738623in}}{\pgfqpoint{1.777488in}{2.742573in}}{\pgfqpoint{1.774576in}{2.745485in}}%
\pgfpathcurveto{\pgfqpoint{1.771664in}{2.748397in}}{\pgfqpoint{1.767714in}{2.750033in}}{\pgfqpoint{1.763596in}{2.750033in}}%
\pgfpathcurveto{\pgfqpoint{1.759478in}{2.750033in}}{\pgfqpoint{1.755528in}{2.748397in}}{\pgfqpoint{1.752616in}{2.745485in}}%
\pgfpathcurveto{\pgfqpoint{1.749704in}{2.742573in}}{\pgfqpoint{1.748068in}{2.738623in}}{\pgfqpoint{1.748068in}{2.734505in}}%
\pgfpathcurveto{\pgfqpoint{1.748068in}{2.730387in}}{\pgfqpoint{1.749704in}{2.726437in}}{\pgfqpoint{1.752616in}{2.723525in}}%
\pgfpathcurveto{\pgfqpoint{1.755528in}{2.720613in}}{\pgfqpoint{1.759478in}{2.718977in}}{\pgfqpoint{1.763596in}{2.718977in}}%
\pgfpathclose%
\pgfusepath{fill}%
\end{pgfscope}%
\begin{pgfscope}%
\pgfpathrectangle{\pgfqpoint{0.887500in}{0.275000in}}{\pgfqpoint{4.225000in}{4.225000in}}%
\pgfusepath{clip}%
\pgfsetbuttcap%
\pgfsetroundjoin%
\definecolor{currentfill}{rgb}{0.000000,0.000000,0.000000}%
\pgfsetfillcolor{currentfill}%
\pgfsetfillopacity{0.800000}%
\pgfsetlinewidth{0.000000pt}%
\definecolor{currentstroke}{rgb}{0.000000,0.000000,0.000000}%
\pgfsetstrokecolor{currentstroke}%
\pgfsetstrokeopacity{0.800000}%
\pgfsetdash{}{0pt}%
\pgfpathmoveto{\pgfqpoint{3.643572in}{3.405897in}}%
\pgfpathcurveto{\pgfqpoint{3.647690in}{3.405897in}}{\pgfqpoint{3.651640in}{3.407533in}}{\pgfqpoint{3.654552in}{3.410445in}}%
\pgfpathcurveto{\pgfqpoint{3.657464in}{3.413357in}}{\pgfqpoint{3.659100in}{3.417307in}}{\pgfqpoint{3.659100in}{3.421425in}}%
\pgfpathcurveto{\pgfqpoint{3.659100in}{3.425543in}}{\pgfqpoint{3.657464in}{3.429493in}}{\pgfqpoint{3.654552in}{3.432405in}}%
\pgfpathcurveto{\pgfqpoint{3.651640in}{3.435317in}}{\pgfqpoint{3.647690in}{3.436953in}}{\pgfqpoint{3.643572in}{3.436953in}}%
\pgfpathcurveto{\pgfqpoint{3.639454in}{3.436953in}}{\pgfqpoint{3.635504in}{3.435317in}}{\pgfqpoint{3.632592in}{3.432405in}}%
\pgfpathcurveto{\pgfqpoint{3.629680in}{3.429493in}}{\pgfqpoint{3.628044in}{3.425543in}}{\pgfqpoint{3.628044in}{3.421425in}}%
\pgfpathcurveto{\pgfqpoint{3.628044in}{3.417307in}}{\pgfqpoint{3.629680in}{3.413357in}}{\pgfqpoint{3.632592in}{3.410445in}}%
\pgfpathcurveto{\pgfqpoint{3.635504in}{3.407533in}}{\pgfqpoint{3.639454in}{3.405897in}}{\pgfqpoint{3.643572in}{3.405897in}}%
\pgfpathclose%
\pgfusepath{fill}%
\end{pgfscope}%
\begin{pgfscope}%
\pgfpathrectangle{\pgfqpoint{0.887500in}{0.275000in}}{\pgfqpoint{4.225000in}{4.225000in}}%
\pgfusepath{clip}%
\pgfsetbuttcap%
\pgfsetroundjoin%
\definecolor{currentfill}{rgb}{0.000000,0.000000,0.000000}%
\pgfsetfillcolor{currentfill}%
\pgfsetfillopacity{0.800000}%
\pgfsetlinewidth{0.000000pt}%
\definecolor{currentstroke}{rgb}{0.000000,0.000000,0.000000}%
\pgfsetstrokecolor{currentstroke}%
\pgfsetstrokeopacity{0.800000}%
\pgfsetdash{}{0pt}%
\pgfpathmoveto{\pgfqpoint{4.170684in}{3.134278in}}%
\pgfpathcurveto{\pgfqpoint{4.174803in}{3.134278in}}{\pgfqpoint{4.178753in}{3.135915in}}{\pgfqpoint{4.181665in}{3.138827in}}%
\pgfpathcurveto{\pgfqpoint{4.184577in}{3.141739in}}{\pgfqpoint{4.186213in}{3.145689in}}{\pgfqpoint{4.186213in}{3.149807in}}%
\pgfpathcurveto{\pgfqpoint{4.186213in}{3.153925in}}{\pgfqpoint{4.184577in}{3.157875in}}{\pgfqpoint{4.181665in}{3.160787in}}%
\pgfpathcurveto{\pgfqpoint{4.178753in}{3.163699in}}{\pgfqpoint{4.174803in}{3.165335in}}{\pgfqpoint{4.170684in}{3.165335in}}%
\pgfpathcurveto{\pgfqpoint{4.166566in}{3.165335in}}{\pgfqpoint{4.162616in}{3.163699in}}{\pgfqpoint{4.159704in}{3.160787in}}%
\pgfpathcurveto{\pgfqpoint{4.156792in}{3.157875in}}{\pgfqpoint{4.155156in}{3.153925in}}{\pgfqpoint{4.155156in}{3.149807in}}%
\pgfpathcurveto{\pgfqpoint{4.155156in}{3.145689in}}{\pgfqpoint{4.156792in}{3.141739in}}{\pgfqpoint{4.159704in}{3.138827in}}%
\pgfpathcurveto{\pgfqpoint{4.162616in}{3.135915in}}{\pgfqpoint{4.166566in}{3.134278in}}{\pgfqpoint{4.170684in}{3.134278in}}%
\pgfpathclose%
\pgfusepath{fill}%
\end{pgfscope}%
\begin{pgfscope}%
\pgfpathrectangle{\pgfqpoint{0.887500in}{0.275000in}}{\pgfqpoint{4.225000in}{4.225000in}}%
\pgfusepath{clip}%
\pgfsetbuttcap%
\pgfsetroundjoin%
\definecolor{currentfill}{rgb}{0.000000,0.000000,0.000000}%
\pgfsetfillcolor{currentfill}%
\pgfsetfillopacity{0.800000}%
\pgfsetlinewidth{0.000000pt}%
\definecolor{currentstroke}{rgb}{0.000000,0.000000,0.000000}%
\pgfsetstrokecolor{currentstroke}%
\pgfsetstrokeopacity{0.800000}%
\pgfsetdash{}{0pt}%
\pgfpathmoveto{\pgfqpoint{3.819401in}{3.323663in}}%
\pgfpathcurveto{\pgfqpoint{3.823519in}{3.323663in}}{\pgfqpoint{3.827469in}{3.325299in}}{\pgfqpoint{3.830381in}{3.328211in}}%
\pgfpathcurveto{\pgfqpoint{3.833293in}{3.331123in}}{\pgfqpoint{3.834930in}{3.335073in}}{\pgfqpoint{3.834930in}{3.339191in}}%
\pgfpathcurveto{\pgfqpoint{3.834930in}{3.343309in}}{\pgfqpoint{3.833293in}{3.347259in}}{\pgfqpoint{3.830381in}{3.350171in}}%
\pgfpathcurveto{\pgfqpoint{3.827469in}{3.353083in}}{\pgfqpoint{3.823519in}{3.354719in}}{\pgfqpoint{3.819401in}{3.354719in}}%
\pgfpathcurveto{\pgfqpoint{3.815283in}{3.354719in}}{\pgfqpoint{3.811333in}{3.353083in}}{\pgfqpoint{3.808421in}{3.350171in}}%
\pgfpathcurveto{\pgfqpoint{3.805509in}{3.347259in}}{\pgfqpoint{3.803873in}{3.343309in}}{\pgfqpoint{3.803873in}{3.339191in}}%
\pgfpathcurveto{\pgfqpoint{3.803873in}{3.335073in}}{\pgfqpoint{3.805509in}{3.331123in}}{\pgfqpoint{3.808421in}{3.328211in}}%
\pgfpathcurveto{\pgfqpoint{3.811333in}{3.325299in}}{\pgfqpoint{3.815283in}{3.323663in}}{\pgfqpoint{3.819401in}{3.323663in}}%
\pgfpathclose%
\pgfusepath{fill}%
\end{pgfscope}%
\begin{pgfscope}%
\pgfpathrectangle{\pgfqpoint{0.887500in}{0.275000in}}{\pgfqpoint{4.225000in}{4.225000in}}%
\pgfusepath{clip}%
\pgfsetbuttcap%
\pgfsetroundjoin%
\definecolor{currentfill}{rgb}{0.000000,0.000000,0.000000}%
\pgfsetfillcolor{currentfill}%
\pgfsetfillopacity{0.800000}%
\pgfsetlinewidth{0.000000pt}%
\definecolor{currentstroke}{rgb}{0.000000,0.000000,0.000000}%
\pgfsetstrokecolor{currentstroke}%
\pgfsetstrokeopacity{0.800000}%
\pgfsetdash{}{0pt}%
\pgfpathmoveto{\pgfqpoint{3.995119in}{3.232126in}}%
\pgfpathcurveto{\pgfqpoint{3.999237in}{3.232126in}}{\pgfqpoint{4.003187in}{3.233762in}}{\pgfqpoint{4.006099in}{3.236674in}}%
\pgfpathcurveto{\pgfqpoint{4.009011in}{3.239586in}}{\pgfqpoint{4.010647in}{3.243536in}}{\pgfqpoint{4.010647in}{3.247654in}}%
\pgfpathcurveto{\pgfqpoint{4.010647in}{3.251772in}}{\pgfqpoint{4.009011in}{3.255722in}}{\pgfqpoint{4.006099in}{3.258634in}}%
\pgfpathcurveto{\pgfqpoint{4.003187in}{3.261546in}}{\pgfqpoint{3.999237in}{3.263182in}}{\pgfqpoint{3.995119in}{3.263182in}}%
\pgfpathcurveto{\pgfqpoint{3.991001in}{3.263182in}}{\pgfqpoint{3.987051in}{3.261546in}}{\pgfqpoint{3.984139in}{3.258634in}}%
\pgfpathcurveto{\pgfqpoint{3.981227in}{3.255722in}}{\pgfqpoint{3.979590in}{3.251772in}}{\pgfqpoint{3.979590in}{3.247654in}}%
\pgfpathcurveto{\pgfqpoint{3.979590in}{3.243536in}}{\pgfqpoint{3.981227in}{3.239586in}}{\pgfqpoint{3.984139in}{3.236674in}}%
\pgfpathcurveto{\pgfqpoint{3.987051in}{3.233762in}}{\pgfqpoint{3.991001in}{3.232126in}}{\pgfqpoint{3.995119in}{3.232126in}}%
\pgfpathclose%
\pgfusepath{fill}%
\end{pgfscope}%
\begin{pgfscope}%
\pgfpathrectangle{\pgfqpoint{0.887500in}{0.275000in}}{\pgfqpoint{4.225000in}{4.225000in}}%
\pgfusepath{clip}%
\pgfsetbuttcap%
\pgfsetroundjoin%
\definecolor{currentfill}{rgb}{0.000000,0.000000,0.000000}%
\pgfsetfillcolor{currentfill}%
\pgfsetfillopacity{0.800000}%
\pgfsetlinewidth{0.000000pt}%
\definecolor{currentstroke}{rgb}{0.000000,0.000000,0.000000}%
\pgfsetstrokecolor{currentstroke}%
\pgfsetstrokeopacity{0.800000}%
\pgfsetdash{}{0pt}%
\pgfpathmoveto{\pgfqpoint{3.181088in}{3.483815in}}%
\pgfpathcurveto{\pgfqpoint{3.185206in}{3.483815in}}{\pgfqpoint{3.189156in}{3.485451in}}{\pgfqpoint{3.192068in}{3.488363in}}%
\pgfpathcurveto{\pgfqpoint{3.194980in}{3.491275in}}{\pgfqpoint{3.196616in}{3.495225in}}{\pgfqpoint{3.196616in}{3.499343in}}%
\pgfpathcurveto{\pgfqpoint{3.196616in}{3.503461in}}{\pgfqpoint{3.194980in}{3.507411in}}{\pgfqpoint{3.192068in}{3.510323in}}%
\pgfpathcurveto{\pgfqpoint{3.189156in}{3.513235in}}{\pgfqpoint{3.185206in}{3.514871in}}{\pgfqpoint{3.181088in}{3.514871in}}%
\pgfpathcurveto{\pgfqpoint{3.176969in}{3.514871in}}{\pgfqpoint{3.173019in}{3.513235in}}{\pgfqpoint{3.170107in}{3.510323in}}%
\pgfpathcurveto{\pgfqpoint{3.167196in}{3.507411in}}{\pgfqpoint{3.165559in}{3.503461in}}{\pgfqpoint{3.165559in}{3.499343in}}%
\pgfpathcurveto{\pgfqpoint{3.165559in}{3.495225in}}{\pgfqpoint{3.167196in}{3.491275in}}{\pgfqpoint{3.170107in}{3.488363in}}%
\pgfpathcurveto{\pgfqpoint{3.173019in}{3.485451in}}{\pgfqpoint{3.176969in}{3.483815in}}{\pgfqpoint{3.181088in}{3.483815in}}%
\pgfpathclose%
\pgfusepath{fill}%
\end{pgfscope}%
\begin{pgfscope}%
\pgfpathrectangle{\pgfqpoint{0.887500in}{0.275000in}}{\pgfqpoint{4.225000in}{4.225000in}}%
\pgfusepath{clip}%
\pgfsetbuttcap%
\pgfsetroundjoin%
\definecolor{currentfill}{rgb}{0.000000,0.000000,0.000000}%
\pgfsetfillcolor{currentfill}%
\pgfsetfillopacity{0.800000}%
\pgfsetlinewidth{0.000000pt}%
\definecolor{currentstroke}{rgb}{0.000000,0.000000,0.000000}%
\pgfsetstrokecolor{currentstroke}%
\pgfsetstrokeopacity{0.800000}%
\pgfsetdash{}{0pt}%
\pgfpathmoveto{\pgfqpoint{2.799777in}{2.761724in}}%
\pgfpathcurveto{\pgfqpoint{2.803895in}{2.761724in}}{\pgfqpoint{2.807845in}{2.763360in}}{\pgfqpoint{2.810757in}{2.766272in}}%
\pgfpathcurveto{\pgfqpoint{2.813669in}{2.769184in}}{\pgfqpoint{2.815306in}{2.773134in}}{\pgfqpoint{2.815306in}{2.777252in}}%
\pgfpathcurveto{\pgfqpoint{2.815306in}{2.781370in}}{\pgfqpoint{2.813669in}{2.785320in}}{\pgfqpoint{2.810757in}{2.788232in}}%
\pgfpathcurveto{\pgfqpoint{2.807845in}{2.791144in}}{\pgfqpoint{2.803895in}{2.792780in}}{\pgfqpoint{2.799777in}{2.792780in}}%
\pgfpathcurveto{\pgfqpoint{2.795659in}{2.792780in}}{\pgfqpoint{2.791709in}{2.791144in}}{\pgfqpoint{2.788797in}{2.788232in}}%
\pgfpathcurveto{\pgfqpoint{2.785885in}{2.785320in}}{\pgfqpoint{2.784249in}{2.781370in}}{\pgfqpoint{2.784249in}{2.777252in}}%
\pgfpathcurveto{\pgfqpoint{2.784249in}{2.773134in}}{\pgfqpoint{2.785885in}{2.769184in}}{\pgfqpoint{2.788797in}{2.766272in}}%
\pgfpathcurveto{\pgfqpoint{2.791709in}{2.763360in}}{\pgfqpoint{2.795659in}{2.761724in}}{\pgfqpoint{2.799777in}{2.761724in}}%
\pgfpathclose%
\pgfusepath{fill}%
\end{pgfscope}%
\begin{pgfscope}%
\pgfpathrectangle{\pgfqpoint{0.887500in}{0.275000in}}{\pgfqpoint{4.225000in}{4.225000in}}%
\pgfusepath{clip}%
\pgfsetbuttcap%
\pgfsetroundjoin%
\definecolor{currentfill}{rgb}{0.000000,0.000000,0.000000}%
\pgfsetfillcolor{currentfill}%
\pgfsetfillopacity{0.800000}%
\pgfsetlinewidth{0.000000pt}%
\definecolor{currentstroke}{rgb}{0.000000,0.000000,0.000000}%
\pgfsetstrokecolor{currentstroke}%
\pgfsetstrokeopacity{0.800000}%
\pgfsetdash{}{0pt}%
\pgfpathmoveto{\pgfqpoint{2.401128in}{2.601900in}}%
\pgfpathcurveto{\pgfqpoint{2.405247in}{2.601900in}}{\pgfqpoint{2.409197in}{2.603536in}}{\pgfqpoint{2.412109in}{2.606448in}}%
\pgfpathcurveto{\pgfqpoint{2.415020in}{2.609360in}}{\pgfqpoint{2.416657in}{2.613310in}}{\pgfqpoint{2.416657in}{2.617428in}}%
\pgfpathcurveto{\pgfqpoint{2.416657in}{2.621547in}}{\pgfqpoint{2.415020in}{2.625497in}}{\pgfqpoint{2.412109in}{2.628409in}}%
\pgfpathcurveto{\pgfqpoint{2.409197in}{2.631320in}}{\pgfqpoint{2.405247in}{2.632957in}}{\pgfqpoint{2.401128in}{2.632957in}}%
\pgfpathcurveto{\pgfqpoint{2.397010in}{2.632957in}}{\pgfqpoint{2.393060in}{2.631320in}}{\pgfqpoint{2.390148in}{2.628409in}}%
\pgfpathcurveto{\pgfqpoint{2.387236in}{2.625497in}}{\pgfqpoint{2.385600in}{2.621547in}}{\pgfqpoint{2.385600in}{2.617428in}}%
\pgfpathcurveto{\pgfqpoint{2.385600in}{2.613310in}}{\pgfqpoint{2.387236in}{2.609360in}}{\pgfqpoint{2.390148in}{2.606448in}}%
\pgfpathcurveto{\pgfqpoint{2.393060in}{2.603536in}}{\pgfqpoint{2.397010in}{2.601900in}}{\pgfqpoint{2.401128in}{2.601900in}}%
\pgfpathclose%
\pgfusepath{fill}%
\end{pgfscope}%
\begin{pgfscope}%
\pgfpathrectangle{\pgfqpoint{0.887500in}{0.275000in}}{\pgfqpoint{4.225000in}{4.225000in}}%
\pgfusepath{clip}%
\pgfsetbuttcap%
\pgfsetroundjoin%
\definecolor{currentfill}{rgb}{0.000000,0.000000,0.000000}%
\pgfsetfillcolor{currentfill}%
\pgfsetfillopacity{0.800000}%
\pgfsetlinewidth{0.000000pt}%
\definecolor{currentstroke}{rgb}{0.000000,0.000000,0.000000}%
\pgfsetstrokecolor{currentstroke}%
\pgfsetstrokeopacity{0.800000}%
\pgfsetdash{}{0pt}%
\pgfpathmoveto{\pgfqpoint{2.893948in}{3.468310in}}%
\pgfpathcurveto{\pgfqpoint{2.898066in}{3.468310in}}{\pgfqpoint{2.902016in}{3.469946in}}{\pgfqpoint{2.904928in}{3.472858in}}%
\pgfpathcurveto{\pgfqpoint{2.907840in}{3.475770in}}{\pgfqpoint{2.909476in}{3.479720in}}{\pgfqpoint{2.909476in}{3.483838in}}%
\pgfpathcurveto{\pgfqpoint{2.909476in}{3.487956in}}{\pgfqpoint{2.907840in}{3.491906in}}{\pgfqpoint{2.904928in}{3.494818in}}%
\pgfpathcurveto{\pgfqpoint{2.902016in}{3.497730in}}{\pgfqpoint{2.898066in}{3.499367in}}{\pgfqpoint{2.893948in}{3.499367in}}%
\pgfpathcurveto{\pgfqpoint{2.889829in}{3.499367in}}{\pgfqpoint{2.885879in}{3.497730in}}{\pgfqpoint{2.882967in}{3.494818in}}%
\pgfpathcurveto{\pgfqpoint{2.880055in}{3.491906in}}{\pgfqpoint{2.878419in}{3.487956in}}{\pgfqpoint{2.878419in}{3.483838in}}%
\pgfpathcurveto{\pgfqpoint{2.878419in}{3.479720in}}{\pgfqpoint{2.880055in}{3.475770in}}{\pgfqpoint{2.882967in}{3.472858in}}%
\pgfpathcurveto{\pgfqpoint{2.885879in}{3.469946in}}{\pgfqpoint{2.889829in}{3.468310in}}{\pgfqpoint{2.893948in}{3.468310in}}%
\pgfpathclose%
\pgfusepath{fill}%
\end{pgfscope}%
\begin{pgfscope}%
\pgfpathrectangle{\pgfqpoint{0.887500in}{0.275000in}}{\pgfqpoint{4.225000in}{4.225000in}}%
\pgfusepath{clip}%
\pgfsetbuttcap%
\pgfsetroundjoin%
\definecolor{currentfill}{rgb}{0.000000,0.000000,0.000000}%
\pgfsetfillcolor{currentfill}%
\pgfsetfillopacity{0.800000}%
\pgfsetlinewidth{0.000000pt}%
\definecolor{currentstroke}{rgb}{0.000000,0.000000,0.000000}%
\pgfsetstrokecolor{currentstroke}%
\pgfsetstrokeopacity{0.800000}%
\pgfsetdash{}{0pt}%
\pgfpathmoveto{\pgfqpoint{1.937924in}{2.670223in}}%
\pgfpathcurveto{\pgfqpoint{1.942042in}{2.670223in}}{\pgfqpoint{1.945992in}{2.671859in}}{\pgfqpoint{1.948904in}{2.674771in}}%
\pgfpathcurveto{\pgfqpoint{1.951816in}{2.677683in}}{\pgfqpoint{1.953452in}{2.681633in}}{\pgfqpoint{1.953452in}{2.685751in}}%
\pgfpathcurveto{\pgfqpoint{1.953452in}{2.689869in}}{\pgfqpoint{1.951816in}{2.693819in}}{\pgfqpoint{1.948904in}{2.696731in}}%
\pgfpathcurveto{\pgfqpoint{1.945992in}{2.699643in}}{\pgfqpoint{1.942042in}{2.701279in}}{\pgfqpoint{1.937924in}{2.701279in}}%
\pgfpathcurveto{\pgfqpoint{1.933806in}{2.701279in}}{\pgfqpoint{1.929856in}{2.699643in}}{\pgfqpoint{1.926944in}{2.696731in}}%
\pgfpathcurveto{\pgfqpoint{1.924032in}{2.693819in}}{\pgfqpoint{1.922396in}{2.689869in}}{\pgfqpoint{1.922396in}{2.685751in}}%
\pgfpathcurveto{\pgfqpoint{1.922396in}{2.681633in}}{\pgfqpoint{1.924032in}{2.677683in}}{\pgfqpoint{1.926944in}{2.674771in}}%
\pgfpathcurveto{\pgfqpoint{1.929856in}{2.671859in}}{\pgfqpoint{1.933806in}{2.670223in}}{\pgfqpoint{1.937924in}{2.670223in}}%
\pgfpathclose%
\pgfusepath{fill}%
\end{pgfscope}%
\begin{pgfscope}%
\pgfpathrectangle{\pgfqpoint{0.887500in}{0.275000in}}{\pgfqpoint{4.225000in}{4.225000in}}%
\pgfusepath{clip}%
\pgfsetbuttcap%
\pgfsetroundjoin%
\definecolor{currentfill}{rgb}{0.000000,0.000000,0.000000}%
\pgfsetfillcolor{currentfill}%
\pgfsetfillopacity{0.800000}%
\pgfsetlinewidth{0.000000pt}%
\definecolor{currentstroke}{rgb}{0.000000,0.000000,0.000000}%
\pgfsetstrokecolor{currentstroke}%
\pgfsetstrokeopacity{0.800000}%
\pgfsetdash{}{0pt}%
\pgfpathmoveto{\pgfqpoint{1.474740in}{2.733018in}}%
\pgfpathcurveto{\pgfqpoint{1.478858in}{2.733018in}}{\pgfqpoint{1.482808in}{2.734654in}}{\pgfqpoint{1.485720in}{2.737566in}}%
\pgfpathcurveto{\pgfqpoint{1.488632in}{2.740478in}}{\pgfqpoint{1.490268in}{2.744428in}}{\pgfqpoint{1.490268in}{2.748546in}}%
\pgfpathcurveto{\pgfqpoint{1.490268in}{2.752665in}}{\pgfqpoint{1.488632in}{2.756615in}}{\pgfqpoint{1.485720in}{2.759527in}}%
\pgfpathcurveto{\pgfqpoint{1.482808in}{2.762439in}}{\pgfqpoint{1.478858in}{2.764075in}}{\pgfqpoint{1.474740in}{2.764075in}}%
\pgfpathcurveto{\pgfqpoint{1.470622in}{2.764075in}}{\pgfqpoint{1.466672in}{2.762439in}}{\pgfqpoint{1.463760in}{2.759527in}}%
\pgfpathcurveto{\pgfqpoint{1.460848in}{2.756615in}}{\pgfqpoint{1.459212in}{2.752665in}}{\pgfqpoint{1.459212in}{2.748546in}}%
\pgfpathcurveto{\pgfqpoint{1.459212in}{2.744428in}}{\pgfqpoint{1.460848in}{2.740478in}}{\pgfqpoint{1.463760in}{2.737566in}}%
\pgfpathcurveto{\pgfqpoint{1.466672in}{2.734654in}}{\pgfqpoint{1.470622in}{2.733018in}}{\pgfqpoint{1.474740in}{2.733018in}}%
\pgfpathclose%
\pgfusepath{fill}%
\end{pgfscope}%
\begin{pgfscope}%
\pgfpathrectangle{\pgfqpoint{0.887500in}{0.275000in}}{\pgfqpoint{4.225000in}{4.225000in}}%
\pgfusepath{clip}%
\pgfsetbuttcap%
\pgfsetroundjoin%
\definecolor{currentfill}{rgb}{0.000000,0.000000,0.000000}%
\pgfsetfillcolor{currentfill}%
\pgfsetfillopacity{0.800000}%
\pgfsetlinewidth{0.000000pt}%
\definecolor{currentstroke}{rgb}{0.000000,0.000000,0.000000}%
\pgfsetstrokecolor{currentstroke}%
\pgfsetstrokeopacity{0.800000}%
\pgfsetdash{}{0pt}%
\pgfpathmoveto{\pgfqpoint{2.688082in}{2.662573in}}%
\pgfpathcurveto{\pgfqpoint{2.692200in}{2.662573in}}{\pgfqpoint{2.696150in}{2.664209in}}{\pgfqpoint{2.699062in}{2.667121in}}%
\pgfpathcurveto{\pgfqpoint{2.701974in}{2.670033in}}{\pgfqpoint{2.703610in}{2.673983in}}{\pgfqpoint{2.703610in}{2.678101in}}%
\pgfpathcurveto{\pgfqpoint{2.703610in}{2.682219in}}{\pgfqpoint{2.701974in}{2.686170in}}{\pgfqpoint{2.699062in}{2.689081in}}%
\pgfpathcurveto{\pgfqpoint{2.696150in}{2.691993in}}{\pgfqpoint{2.692200in}{2.693630in}}{\pgfqpoint{2.688082in}{2.693630in}}%
\pgfpathcurveto{\pgfqpoint{2.683964in}{2.693630in}}{\pgfqpoint{2.680014in}{2.691993in}}{\pgfqpoint{2.677102in}{2.689081in}}%
\pgfpathcurveto{\pgfqpoint{2.674190in}{2.686170in}}{\pgfqpoint{2.672554in}{2.682219in}}{\pgfqpoint{2.672554in}{2.678101in}}%
\pgfpathcurveto{\pgfqpoint{2.672554in}{2.673983in}}{\pgfqpoint{2.674190in}{2.670033in}}{\pgfqpoint{2.677102in}{2.667121in}}%
\pgfpathcurveto{\pgfqpoint{2.680014in}{2.664209in}}{\pgfqpoint{2.683964in}{2.662573in}}{\pgfqpoint{2.688082in}{2.662573in}}%
\pgfpathclose%
\pgfusepath{fill}%
\end{pgfscope}%
\begin{pgfscope}%
\pgfpathrectangle{\pgfqpoint{0.887500in}{0.275000in}}{\pgfqpoint{4.225000in}{4.225000in}}%
\pgfusepath{clip}%
\pgfsetbuttcap%
\pgfsetroundjoin%
\definecolor{currentfill}{rgb}{0.000000,0.000000,0.000000}%
\pgfsetfillcolor{currentfill}%
\pgfsetfillopacity{0.800000}%
\pgfsetlinewidth{0.000000pt}%
\definecolor{currentstroke}{rgb}{0.000000,0.000000,0.000000}%
\pgfsetstrokecolor{currentstroke}%
\pgfsetstrokeopacity{0.800000}%
\pgfsetdash{}{0pt}%
\pgfpathmoveto{\pgfqpoint{3.357193in}{3.446530in}}%
\pgfpathcurveto{\pgfqpoint{3.361311in}{3.446530in}}{\pgfqpoint{3.365261in}{3.448167in}}{\pgfqpoint{3.368173in}{3.451079in}}%
\pgfpathcurveto{\pgfqpoint{3.371085in}{3.453990in}}{\pgfqpoint{3.372721in}{3.457940in}}{\pgfqpoint{3.372721in}{3.462059in}}%
\pgfpathcurveto{\pgfqpoint{3.372721in}{3.466177in}}{\pgfqpoint{3.371085in}{3.470127in}}{\pgfqpoint{3.368173in}{3.473039in}}%
\pgfpathcurveto{\pgfqpoint{3.365261in}{3.475951in}}{\pgfqpoint{3.361311in}{3.477587in}}{\pgfqpoint{3.357193in}{3.477587in}}%
\pgfpathcurveto{\pgfqpoint{3.353074in}{3.477587in}}{\pgfqpoint{3.349124in}{3.475951in}}{\pgfqpoint{3.346213in}{3.473039in}}%
\pgfpathcurveto{\pgfqpoint{3.343301in}{3.470127in}}{\pgfqpoint{3.341664in}{3.466177in}}{\pgfqpoint{3.341664in}{3.462059in}}%
\pgfpathcurveto{\pgfqpoint{3.341664in}{3.457940in}}{\pgfqpoint{3.343301in}{3.453990in}}{\pgfqpoint{3.346213in}{3.451079in}}%
\pgfpathcurveto{\pgfqpoint{3.349124in}{3.448167in}}{\pgfqpoint{3.353074in}{3.446530in}}{\pgfqpoint{3.357193in}{3.446530in}}%
\pgfpathclose%
\pgfusepath{fill}%
\end{pgfscope}%
\begin{pgfscope}%
\pgfpathrectangle{\pgfqpoint{0.887500in}{0.275000in}}{\pgfqpoint{4.225000in}{4.225000in}}%
\pgfusepath{clip}%
\pgfsetbuttcap%
\pgfsetroundjoin%
\definecolor{currentfill}{rgb}{0.000000,0.000000,0.000000}%
\pgfsetfillcolor{currentfill}%
\pgfsetfillopacity{0.800000}%
\pgfsetlinewidth{0.000000pt}%
\definecolor{currentstroke}{rgb}{0.000000,0.000000,0.000000}%
\pgfsetstrokecolor{currentstroke}%
\pgfsetstrokeopacity{0.800000}%
\pgfsetdash{}{0pt}%
\pgfpathmoveto{\pgfqpoint{2.576395in}{2.548880in}}%
\pgfpathcurveto{\pgfqpoint{2.580513in}{2.548880in}}{\pgfqpoint{2.584463in}{2.550516in}}{\pgfqpoint{2.587375in}{2.553428in}}%
\pgfpathcurveto{\pgfqpoint{2.590287in}{2.556340in}}{\pgfqpoint{2.591923in}{2.560290in}}{\pgfqpoint{2.591923in}{2.564408in}}%
\pgfpathcurveto{\pgfqpoint{2.591923in}{2.568526in}}{\pgfqpoint{2.590287in}{2.572476in}}{\pgfqpoint{2.587375in}{2.575388in}}%
\pgfpathcurveto{\pgfqpoint{2.584463in}{2.578300in}}{\pgfqpoint{2.580513in}{2.579936in}}{\pgfqpoint{2.576395in}{2.579936in}}%
\pgfpathcurveto{\pgfqpoint{2.572277in}{2.579936in}}{\pgfqpoint{2.568327in}{2.578300in}}{\pgfqpoint{2.565415in}{2.575388in}}%
\pgfpathcurveto{\pgfqpoint{2.562503in}{2.572476in}}{\pgfqpoint{2.560867in}{2.568526in}}{\pgfqpoint{2.560867in}{2.564408in}}%
\pgfpathcurveto{\pgfqpoint{2.560867in}{2.560290in}}{\pgfqpoint{2.562503in}{2.556340in}}{\pgfqpoint{2.565415in}{2.553428in}}%
\pgfpathcurveto{\pgfqpoint{2.568327in}{2.550516in}}{\pgfqpoint{2.572277in}{2.548880in}}{\pgfqpoint{2.576395in}{2.548880in}}%
\pgfpathclose%
\pgfusepath{fill}%
\end{pgfscope}%
\begin{pgfscope}%
\pgfpathrectangle{\pgfqpoint{0.887500in}{0.275000in}}{\pgfqpoint{4.225000in}{4.225000in}}%
\pgfusepath{clip}%
\pgfsetbuttcap%
\pgfsetroundjoin%
\definecolor{currentfill}{rgb}{0.000000,0.000000,0.000000}%
\pgfsetfillcolor{currentfill}%
\pgfsetfillopacity{0.800000}%
\pgfsetlinewidth{0.000000pt}%
\definecolor{currentstroke}{rgb}{0.000000,0.000000,0.000000}%
\pgfsetstrokecolor{currentstroke}%
\pgfsetstrokeopacity{0.800000}%
\pgfsetdash{}{0pt}%
\pgfpathmoveto{\pgfqpoint{2.112689in}{2.619736in}}%
\pgfpathcurveto{\pgfqpoint{2.116807in}{2.619736in}}{\pgfqpoint{2.120757in}{2.621372in}}{\pgfqpoint{2.123669in}{2.624284in}}%
\pgfpathcurveto{\pgfqpoint{2.126581in}{2.627196in}}{\pgfqpoint{2.128217in}{2.631146in}}{\pgfqpoint{2.128217in}{2.635265in}}%
\pgfpathcurveto{\pgfqpoint{2.128217in}{2.639383in}}{\pgfqpoint{2.126581in}{2.643333in}}{\pgfqpoint{2.123669in}{2.646245in}}%
\pgfpathcurveto{\pgfqpoint{2.120757in}{2.649157in}}{\pgfqpoint{2.116807in}{2.650793in}}{\pgfqpoint{2.112689in}{2.650793in}}%
\pgfpathcurveto{\pgfqpoint{2.108571in}{2.650793in}}{\pgfqpoint{2.104621in}{2.649157in}}{\pgfqpoint{2.101709in}{2.646245in}}%
\pgfpathcurveto{\pgfqpoint{2.098797in}{2.643333in}}{\pgfqpoint{2.097161in}{2.639383in}}{\pgfqpoint{2.097161in}{2.635265in}}%
\pgfpathcurveto{\pgfqpoint{2.097161in}{2.631146in}}{\pgfqpoint{2.098797in}{2.627196in}}{\pgfqpoint{2.101709in}{2.624284in}}%
\pgfpathcurveto{\pgfqpoint{2.104621in}{2.621372in}}{\pgfqpoint{2.108571in}{2.619736in}}{\pgfqpoint{2.112689in}{2.619736in}}%
\pgfpathclose%
\pgfusepath{fill}%
\end{pgfscope}%
\begin{pgfscope}%
\pgfpathrectangle{\pgfqpoint{0.887500in}{0.275000in}}{\pgfqpoint{4.225000in}{4.225000in}}%
\pgfusepath{clip}%
\pgfsetbuttcap%
\pgfsetroundjoin%
\definecolor{currentfill}{rgb}{0.000000,0.000000,0.000000}%
\pgfsetfillcolor{currentfill}%
\pgfsetfillopacity{0.800000}%
\pgfsetlinewidth{0.000000pt}%
\definecolor{currentstroke}{rgb}{0.000000,0.000000,0.000000}%
\pgfsetstrokecolor{currentstroke}%
\pgfsetstrokeopacity{0.800000}%
\pgfsetdash{}{0pt}%
\pgfpathmoveto{\pgfqpoint{3.533371in}{3.373284in}}%
\pgfpathcurveto{\pgfqpoint{3.537489in}{3.373284in}}{\pgfqpoint{3.541439in}{3.374920in}}{\pgfqpoint{3.544351in}{3.377832in}}%
\pgfpathcurveto{\pgfqpoint{3.547263in}{3.380744in}}{\pgfqpoint{3.548899in}{3.384694in}}{\pgfqpoint{3.548899in}{3.388812in}}%
\pgfpathcurveto{\pgfqpoint{3.548899in}{3.392931in}}{\pgfqpoint{3.547263in}{3.396881in}}{\pgfqpoint{3.544351in}{3.399793in}}%
\pgfpathcurveto{\pgfqpoint{3.541439in}{3.402705in}}{\pgfqpoint{3.537489in}{3.404341in}}{\pgfqpoint{3.533371in}{3.404341in}}%
\pgfpathcurveto{\pgfqpoint{3.529253in}{3.404341in}}{\pgfqpoint{3.525303in}{3.402705in}}{\pgfqpoint{3.522391in}{3.399793in}}%
\pgfpathcurveto{\pgfqpoint{3.519479in}{3.396881in}}{\pgfqpoint{3.517843in}{3.392931in}}{\pgfqpoint{3.517843in}{3.388812in}}%
\pgfpathcurveto{\pgfqpoint{3.517843in}{3.384694in}}{\pgfqpoint{3.519479in}{3.380744in}}{\pgfqpoint{3.522391in}{3.377832in}}%
\pgfpathcurveto{\pgfqpoint{3.525303in}{3.374920in}}{\pgfqpoint{3.529253in}{3.373284in}}{\pgfqpoint{3.533371in}{3.373284in}}%
\pgfpathclose%
\pgfusepath{fill}%
\end{pgfscope}%
\begin{pgfscope}%
\pgfpathrectangle{\pgfqpoint{0.887500in}{0.275000in}}{\pgfqpoint{4.225000in}{4.225000in}}%
\pgfusepath{clip}%
\pgfsetbuttcap%
\pgfsetroundjoin%
\definecolor{currentfill}{rgb}{0.000000,0.000000,0.000000}%
\pgfsetfillcolor{currentfill}%
\pgfsetfillopacity{0.800000}%
\pgfsetlinewidth{0.000000pt}%
\definecolor{currentstroke}{rgb}{0.000000,0.000000,0.000000}%
\pgfsetstrokecolor{currentstroke}%
\pgfsetstrokeopacity{0.800000}%
\pgfsetdash{}{0pt}%
\pgfpathmoveto{\pgfqpoint{1.648878in}{2.686272in}}%
\pgfpathcurveto{\pgfqpoint{1.652996in}{2.686272in}}{\pgfqpoint{1.656946in}{2.687908in}}{\pgfqpoint{1.659858in}{2.690820in}}%
\pgfpathcurveto{\pgfqpoint{1.662770in}{2.693732in}}{\pgfqpoint{1.664406in}{2.697682in}}{\pgfqpoint{1.664406in}{2.701801in}}%
\pgfpathcurveto{\pgfqpoint{1.664406in}{2.705919in}}{\pgfqpoint{1.662770in}{2.709869in}}{\pgfqpoint{1.659858in}{2.712781in}}%
\pgfpathcurveto{\pgfqpoint{1.656946in}{2.715693in}}{\pgfqpoint{1.652996in}{2.717329in}}{\pgfqpoint{1.648878in}{2.717329in}}%
\pgfpathcurveto{\pgfqpoint{1.644760in}{2.717329in}}{\pgfqpoint{1.640810in}{2.715693in}}{\pgfqpoint{1.637898in}{2.712781in}}%
\pgfpathcurveto{\pgfqpoint{1.634986in}{2.709869in}}{\pgfqpoint{1.633350in}{2.705919in}}{\pgfqpoint{1.633350in}{2.701801in}}%
\pgfpathcurveto{\pgfqpoint{1.633350in}{2.697682in}}{\pgfqpoint{1.634986in}{2.693732in}}{\pgfqpoint{1.637898in}{2.690820in}}%
\pgfpathcurveto{\pgfqpoint{1.640810in}{2.687908in}}{\pgfqpoint{1.644760in}{2.686272in}}{\pgfqpoint{1.648878in}{2.686272in}}%
\pgfpathclose%
\pgfusepath{fill}%
\end{pgfscope}%
\begin{pgfscope}%
\pgfpathrectangle{\pgfqpoint{0.887500in}{0.275000in}}{\pgfqpoint{4.225000in}{4.225000in}}%
\pgfusepath{clip}%
\pgfsetbuttcap%
\pgfsetroundjoin%
\definecolor{currentfill}{rgb}{0.000000,0.000000,0.000000}%
\pgfsetfillcolor{currentfill}%
\pgfsetfillopacity{0.800000}%
\pgfsetlinewidth{0.000000pt}%
\definecolor{currentstroke}{rgb}{0.000000,0.000000,0.000000}%
\pgfsetstrokecolor{currentstroke}%
\pgfsetstrokeopacity{0.800000}%
\pgfsetdash{}{0pt}%
\pgfpathmoveto{\pgfqpoint{4.237631in}{3.001747in}}%
\pgfpathcurveto{\pgfqpoint{4.241749in}{3.001747in}}{\pgfqpoint{4.245699in}{3.003383in}}{\pgfqpoint{4.248611in}{3.006295in}}%
\pgfpathcurveto{\pgfqpoint{4.251523in}{3.009207in}}{\pgfqpoint{4.253159in}{3.013157in}}{\pgfqpoint{4.253159in}{3.017276in}}%
\pgfpathcurveto{\pgfqpoint{4.253159in}{3.021394in}}{\pgfqpoint{4.251523in}{3.025344in}}{\pgfqpoint{4.248611in}{3.028256in}}%
\pgfpathcurveto{\pgfqpoint{4.245699in}{3.031168in}}{\pgfqpoint{4.241749in}{3.032804in}}{\pgfqpoint{4.237631in}{3.032804in}}%
\pgfpathcurveto{\pgfqpoint{4.233513in}{3.032804in}}{\pgfqpoint{4.229563in}{3.031168in}}{\pgfqpoint{4.226651in}{3.028256in}}%
\pgfpathcurveto{\pgfqpoint{4.223739in}{3.025344in}}{\pgfqpoint{4.222103in}{3.021394in}}{\pgfqpoint{4.222103in}{3.017276in}}%
\pgfpathcurveto{\pgfqpoint{4.222103in}{3.013157in}}{\pgfqpoint{4.223739in}{3.009207in}}{\pgfqpoint{4.226651in}{3.006295in}}%
\pgfpathcurveto{\pgfqpoint{4.229563in}{3.003383in}}{\pgfqpoint{4.233513in}{3.001747in}}{\pgfqpoint{4.237631in}{3.001747in}}%
\pgfpathclose%
\pgfusepath{fill}%
\end{pgfscope}%
\begin{pgfscope}%
\pgfpathrectangle{\pgfqpoint{0.887500in}{0.275000in}}{\pgfqpoint{4.225000in}{4.225000in}}%
\pgfusepath{clip}%
\pgfsetbuttcap%
\pgfsetroundjoin%
\definecolor{currentfill}{rgb}{0.000000,0.000000,0.000000}%
\pgfsetfillcolor{currentfill}%
\pgfsetfillopacity{0.800000}%
\pgfsetlinewidth{0.000000pt}%
\definecolor{currentstroke}{rgb}{0.000000,0.000000,0.000000}%
\pgfsetstrokecolor{currentstroke}%
\pgfsetstrokeopacity{0.800000}%
\pgfsetdash{}{0pt}%
\pgfpathmoveto{\pgfqpoint{4.061736in}{3.102222in}}%
\pgfpathcurveto{\pgfqpoint{4.065854in}{3.102222in}}{\pgfqpoint{4.069804in}{3.103858in}}{\pgfqpoint{4.072716in}{3.106770in}}%
\pgfpathcurveto{\pgfqpoint{4.075628in}{3.109682in}}{\pgfqpoint{4.077264in}{3.113632in}}{\pgfqpoint{4.077264in}{3.117750in}}%
\pgfpathcurveto{\pgfqpoint{4.077264in}{3.121868in}}{\pgfqpoint{4.075628in}{3.125818in}}{\pgfqpoint{4.072716in}{3.128730in}}%
\pgfpathcurveto{\pgfqpoint{4.069804in}{3.131642in}}{\pgfqpoint{4.065854in}{3.133278in}}{\pgfqpoint{4.061736in}{3.133278in}}%
\pgfpathcurveto{\pgfqpoint{4.057618in}{3.133278in}}{\pgfqpoint{4.053668in}{3.131642in}}{\pgfqpoint{4.050756in}{3.128730in}}%
\pgfpathcurveto{\pgfqpoint{4.047844in}{3.125818in}}{\pgfqpoint{4.046208in}{3.121868in}}{\pgfqpoint{4.046208in}{3.117750in}}%
\pgfpathcurveto{\pgfqpoint{4.046208in}{3.113632in}}{\pgfqpoint{4.047844in}{3.109682in}}{\pgfqpoint{4.050756in}{3.106770in}}%
\pgfpathcurveto{\pgfqpoint{4.053668in}{3.103858in}}{\pgfqpoint{4.057618in}{3.102222in}}{\pgfqpoint{4.061736in}{3.102222in}}%
\pgfpathclose%
\pgfusepath{fill}%
\end{pgfscope}%
\begin{pgfscope}%
\pgfpathrectangle{\pgfqpoint{0.887500in}{0.275000in}}{\pgfqpoint{4.225000in}{4.225000in}}%
\pgfusepath{clip}%
\pgfsetbuttcap%
\pgfsetroundjoin%
\definecolor{currentfill}{rgb}{0.000000,0.000000,0.000000}%
\pgfsetfillcolor{currentfill}%
\pgfsetfillopacity{0.800000}%
\pgfsetlinewidth{0.000000pt}%
\definecolor{currentstroke}{rgb}{0.000000,0.000000,0.000000}%
\pgfsetstrokecolor{currentstroke}%
\pgfsetstrokeopacity{0.800000}%
\pgfsetdash{}{0pt}%
\pgfpathmoveto{\pgfqpoint{3.709599in}{3.291714in}}%
\pgfpathcurveto{\pgfqpoint{3.713718in}{3.291714in}}{\pgfqpoint{3.717668in}{3.293350in}}{\pgfqpoint{3.720580in}{3.296262in}}%
\pgfpathcurveto{\pgfqpoint{3.723491in}{3.299174in}}{\pgfqpoint{3.725128in}{3.303124in}}{\pgfqpoint{3.725128in}{3.307242in}}%
\pgfpathcurveto{\pgfqpoint{3.725128in}{3.311360in}}{\pgfqpoint{3.723491in}{3.315310in}}{\pgfqpoint{3.720580in}{3.318222in}}%
\pgfpathcurveto{\pgfqpoint{3.717668in}{3.321134in}}{\pgfqpoint{3.713718in}{3.322770in}}{\pgfqpoint{3.709599in}{3.322770in}}%
\pgfpathcurveto{\pgfqpoint{3.705481in}{3.322770in}}{\pgfqpoint{3.701531in}{3.321134in}}{\pgfqpoint{3.698619in}{3.318222in}}%
\pgfpathcurveto{\pgfqpoint{3.695707in}{3.315310in}}{\pgfqpoint{3.694071in}{3.311360in}}{\pgfqpoint{3.694071in}{3.307242in}}%
\pgfpathcurveto{\pgfqpoint{3.694071in}{3.303124in}}{\pgfqpoint{3.695707in}{3.299174in}}{\pgfqpoint{3.698619in}{3.296262in}}%
\pgfpathcurveto{\pgfqpoint{3.701531in}{3.293350in}}{\pgfqpoint{3.705481in}{3.291714in}}{\pgfqpoint{3.709599in}{3.291714in}}%
\pgfpathclose%
\pgfusepath{fill}%
\end{pgfscope}%
\begin{pgfscope}%
\pgfpathrectangle{\pgfqpoint{0.887500in}{0.275000in}}{\pgfqpoint{4.225000in}{4.225000in}}%
\pgfusepath{clip}%
\pgfsetbuttcap%
\pgfsetroundjoin%
\definecolor{currentfill}{rgb}{0.000000,0.000000,0.000000}%
\pgfsetfillcolor{currentfill}%
\pgfsetfillopacity{0.800000}%
\pgfsetlinewidth{0.000000pt}%
\definecolor{currentstroke}{rgb}{0.000000,0.000000,0.000000}%
\pgfsetstrokecolor{currentstroke}%
\pgfsetstrokeopacity{0.800000}%
\pgfsetdash{}{0pt}%
\pgfpathmoveto{\pgfqpoint{3.885738in}{3.200301in}}%
\pgfpathcurveto{\pgfqpoint{3.889856in}{3.200301in}}{\pgfqpoint{3.893806in}{3.201937in}}{\pgfqpoint{3.896718in}{3.204849in}}%
\pgfpathcurveto{\pgfqpoint{3.899630in}{3.207761in}}{\pgfqpoint{3.901266in}{3.211711in}}{\pgfqpoint{3.901266in}{3.215829in}}%
\pgfpathcurveto{\pgfqpoint{3.901266in}{3.219948in}}{\pgfqpoint{3.899630in}{3.223898in}}{\pgfqpoint{3.896718in}{3.226810in}}%
\pgfpathcurveto{\pgfqpoint{3.893806in}{3.229721in}}{\pgfqpoint{3.889856in}{3.231358in}}{\pgfqpoint{3.885738in}{3.231358in}}%
\pgfpathcurveto{\pgfqpoint{3.881620in}{3.231358in}}{\pgfqpoint{3.877670in}{3.229721in}}{\pgfqpoint{3.874758in}{3.226810in}}%
\pgfpathcurveto{\pgfqpoint{3.871846in}{3.223898in}}{\pgfqpoint{3.870210in}{3.219948in}}{\pgfqpoint{3.870210in}{3.215829in}}%
\pgfpathcurveto{\pgfqpoint{3.870210in}{3.211711in}}{\pgfqpoint{3.871846in}{3.207761in}}{\pgfqpoint{3.874758in}{3.204849in}}%
\pgfpathcurveto{\pgfqpoint{3.877670in}{3.201937in}}{\pgfqpoint{3.881620in}{3.200301in}}{\pgfqpoint{3.885738in}{3.200301in}}%
\pgfpathclose%
\pgfusepath{fill}%
\end{pgfscope}%
\begin{pgfscope}%
\pgfpathrectangle{\pgfqpoint{0.887500in}{0.275000in}}{\pgfqpoint{4.225000in}{4.225000in}}%
\pgfusepath{clip}%
\pgfsetbuttcap%
\pgfsetroundjoin%
\definecolor{currentfill}{rgb}{0.000000,0.000000,0.000000}%
\pgfsetfillcolor{currentfill}%
\pgfsetfillopacity{0.800000}%
\pgfsetlinewidth{0.000000pt}%
\definecolor{currentstroke}{rgb}{0.000000,0.000000,0.000000}%
\pgfsetstrokecolor{currentstroke}%
\pgfsetstrokeopacity{0.800000}%
\pgfsetdash{}{0pt}%
\pgfpathmoveto{\pgfqpoint{2.287859in}{2.568115in}}%
\pgfpathcurveto{\pgfqpoint{2.291977in}{2.568115in}}{\pgfqpoint{2.295927in}{2.569751in}}{\pgfqpoint{2.298839in}{2.572663in}}%
\pgfpathcurveto{\pgfqpoint{2.301751in}{2.575575in}}{\pgfqpoint{2.303387in}{2.579525in}}{\pgfqpoint{2.303387in}{2.583643in}}%
\pgfpathcurveto{\pgfqpoint{2.303387in}{2.587761in}}{\pgfqpoint{2.301751in}{2.591711in}}{\pgfqpoint{2.298839in}{2.594623in}}%
\pgfpathcurveto{\pgfqpoint{2.295927in}{2.597535in}}{\pgfqpoint{2.291977in}{2.599171in}}{\pgfqpoint{2.287859in}{2.599171in}}%
\pgfpathcurveto{\pgfqpoint{2.283741in}{2.599171in}}{\pgfqpoint{2.279791in}{2.597535in}}{\pgfqpoint{2.276879in}{2.594623in}}%
\pgfpathcurveto{\pgfqpoint{2.273967in}{2.591711in}}{\pgfqpoint{2.272330in}{2.587761in}}{\pgfqpoint{2.272330in}{2.583643in}}%
\pgfpathcurveto{\pgfqpoint{2.272330in}{2.579525in}}{\pgfqpoint{2.273967in}{2.575575in}}{\pgfqpoint{2.276879in}{2.572663in}}%
\pgfpathcurveto{\pgfqpoint{2.279791in}{2.569751in}}{\pgfqpoint{2.283741in}{2.568115in}}{\pgfqpoint{2.287859in}{2.568115in}}%
\pgfpathclose%
\pgfusepath{fill}%
\end{pgfscope}%
\begin{pgfscope}%
\pgfpathrectangle{\pgfqpoint{0.887500in}{0.275000in}}{\pgfqpoint{4.225000in}{4.225000in}}%
\pgfusepath{clip}%
\pgfsetbuttcap%
\pgfsetroundjoin%
\definecolor{currentfill}{rgb}{0.000000,0.000000,0.000000}%
\pgfsetfillcolor{currentfill}%
\pgfsetfillopacity{0.800000}%
\pgfsetlinewidth{0.000000pt}%
\definecolor{currentstroke}{rgb}{0.000000,0.000000,0.000000}%
\pgfsetstrokecolor{currentstroke}%
\pgfsetstrokeopacity{0.800000}%
\pgfsetdash{}{0pt}%
\pgfpathmoveto{\pgfqpoint{1.823493in}{2.637449in}}%
\pgfpathcurveto{\pgfqpoint{1.827611in}{2.637449in}}{\pgfqpoint{1.831562in}{2.639085in}}{\pgfqpoint{1.834473in}{2.641997in}}%
\pgfpathcurveto{\pgfqpoint{1.837385in}{2.644909in}}{\pgfqpoint{1.839022in}{2.648859in}}{\pgfqpoint{1.839022in}{2.652977in}}%
\pgfpathcurveto{\pgfqpoint{1.839022in}{2.657095in}}{\pgfqpoint{1.837385in}{2.661045in}}{\pgfqpoint{1.834473in}{2.663957in}}%
\pgfpathcurveto{\pgfqpoint{1.831562in}{2.666869in}}{\pgfqpoint{1.827611in}{2.668505in}}{\pgfqpoint{1.823493in}{2.668505in}}%
\pgfpathcurveto{\pgfqpoint{1.819375in}{2.668505in}}{\pgfqpoint{1.815425in}{2.666869in}}{\pgfqpoint{1.812513in}{2.663957in}}%
\pgfpathcurveto{\pgfqpoint{1.809601in}{2.661045in}}{\pgfqpoint{1.807965in}{2.657095in}}{\pgfqpoint{1.807965in}{2.652977in}}%
\pgfpathcurveto{\pgfqpoint{1.807965in}{2.648859in}}{\pgfqpoint{1.809601in}{2.644909in}}{\pgfqpoint{1.812513in}{2.641997in}}%
\pgfpathcurveto{\pgfqpoint{1.815425in}{2.639085in}}{\pgfqpoint{1.819375in}{2.637449in}}{\pgfqpoint{1.823493in}{2.637449in}}%
\pgfpathclose%
\pgfusepath{fill}%
\end{pgfscope}%
\begin{pgfscope}%
\pgfpathrectangle{\pgfqpoint{0.887500in}{0.275000in}}{\pgfqpoint{4.225000in}{4.225000in}}%
\pgfusepath{clip}%
\pgfsetbuttcap%
\pgfsetroundjoin%
\definecolor{currentfill}{rgb}{0.000000,0.000000,0.000000}%
\pgfsetfillcolor{currentfill}%
\pgfsetfillopacity{0.800000}%
\pgfsetlinewidth{0.000000pt}%
\definecolor{currentstroke}{rgb}{0.000000,0.000000,0.000000}%
\pgfsetstrokecolor{currentstroke}%
\pgfsetstrokeopacity{0.800000}%
\pgfsetdash{}{0pt}%
\pgfpathmoveto{\pgfqpoint{1.359217in}{2.699537in}}%
\pgfpathcurveto{\pgfqpoint{1.363335in}{2.699537in}}{\pgfqpoint{1.367285in}{2.701173in}}{\pgfqpoint{1.370197in}{2.704085in}}%
\pgfpathcurveto{\pgfqpoint{1.373109in}{2.706997in}}{\pgfqpoint{1.374745in}{2.710947in}}{\pgfqpoint{1.374745in}{2.715065in}}%
\pgfpathcurveto{\pgfqpoint{1.374745in}{2.719183in}}{\pgfqpoint{1.373109in}{2.723133in}}{\pgfqpoint{1.370197in}{2.726045in}}%
\pgfpathcurveto{\pgfqpoint{1.367285in}{2.728957in}}{\pgfqpoint{1.363335in}{2.730594in}}{\pgfqpoint{1.359217in}{2.730594in}}%
\pgfpathcurveto{\pgfqpoint{1.355099in}{2.730594in}}{\pgfqpoint{1.351149in}{2.728957in}}{\pgfqpoint{1.348237in}{2.726045in}}%
\pgfpathcurveto{\pgfqpoint{1.345325in}{2.723133in}}{\pgfqpoint{1.343689in}{2.719183in}}{\pgfqpoint{1.343689in}{2.715065in}}%
\pgfpathcurveto{\pgfqpoint{1.343689in}{2.710947in}}{\pgfqpoint{1.345325in}{2.706997in}}{\pgfqpoint{1.348237in}{2.704085in}}%
\pgfpathcurveto{\pgfqpoint{1.351149in}{2.701173in}}{\pgfqpoint{1.355099in}{2.699537in}}{\pgfqpoint{1.359217in}{2.699537in}}%
\pgfpathclose%
\pgfusepath{fill}%
\end{pgfscope}%
\begin{pgfscope}%
\pgfpathrectangle{\pgfqpoint{0.887500in}{0.275000in}}{\pgfqpoint{4.225000in}{4.225000in}}%
\pgfusepath{clip}%
\pgfsetbuttcap%
\pgfsetroundjoin%
\definecolor{currentfill}{rgb}{0.000000,0.000000,0.000000}%
\pgfsetfillcolor{currentfill}%
\pgfsetfillopacity{0.800000}%
\pgfsetlinewidth{0.000000pt}%
\definecolor{currentstroke}{rgb}{0.000000,0.000000,0.000000}%
\pgfsetstrokecolor{currentstroke}%
\pgfsetstrokeopacity{0.800000}%
\pgfsetdash{}{0pt}%
\pgfpathmoveto{\pgfqpoint{2.910752in}{3.013093in}}%
\pgfpathcurveto{\pgfqpoint{2.914870in}{3.013093in}}{\pgfqpoint{2.918820in}{3.014729in}}{\pgfqpoint{2.921732in}{3.017641in}}%
\pgfpathcurveto{\pgfqpoint{2.924644in}{3.020553in}}{\pgfqpoint{2.926280in}{3.024503in}}{\pgfqpoint{2.926280in}{3.028621in}}%
\pgfpathcurveto{\pgfqpoint{2.926280in}{3.032739in}}{\pgfqpoint{2.924644in}{3.036689in}}{\pgfqpoint{2.921732in}{3.039601in}}%
\pgfpathcurveto{\pgfqpoint{2.918820in}{3.042513in}}{\pgfqpoint{2.914870in}{3.044149in}}{\pgfqpoint{2.910752in}{3.044149in}}%
\pgfpathcurveto{\pgfqpoint{2.906633in}{3.044149in}}{\pgfqpoint{2.902683in}{3.042513in}}{\pgfqpoint{2.899771in}{3.039601in}}%
\pgfpathcurveto{\pgfqpoint{2.896859in}{3.036689in}}{\pgfqpoint{2.895223in}{3.032739in}}{\pgfqpoint{2.895223in}{3.028621in}}%
\pgfpathcurveto{\pgfqpoint{2.895223in}{3.024503in}}{\pgfqpoint{2.896859in}{3.020553in}}{\pgfqpoint{2.899771in}{3.017641in}}%
\pgfpathcurveto{\pgfqpoint{2.902683in}{3.014729in}}{\pgfqpoint{2.906633in}{3.013093in}}{\pgfqpoint{2.910752in}{3.013093in}}%
\pgfpathclose%
\pgfusepath{fill}%
\end{pgfscope}%
\begin{pgfscope}%
\pgfpathrectangle{\pgfqpoint{0.887500in}{0.275000in}}{\pgfqpoint{4.225000in}{4.225000in}}%
\pgfusepath{clip}%
\pgfsetbuttcap%
\pgfsetroundjoin%
\definecolor{currentfill}{rgb}{0.000000,0.000000,0.000000}%
\pgfsetfillcolor{currentfill}%
\pgfsetfillopacity{0.800000}%
\pgfsetlinewidth{0.000000pt}%
\definecolor{currentstroke}{rgb}{0.000000,0.000000,0.000000}%
\pgfsetstrokecolor{currentstroke}%
\pgfsetstrokeopacity{0.800000}%
\pgfsetdash{}{0pt}%
\pgfpathmoveto{\pgfqpoint{3.246059in}{3.404015in}}%
\pgfpathcurveto{\pgfqpoint{3.250177in}{3.404015in}}{\pgfqpoint{3.254127in}{3.405651in}}{\pgfqpoint{3.257039in}{3.408563in}}%
\pgfpathcurveto{\pgfqpoint{3.259951in}{3.411475in}}{\pgfqpoint{3.261587in}{3.415425in}}{\pgfqpoint{3.261587in}{3.419543in}}%
\pgfpathcurveto{\pgfqpoint{3.261587in}{3.423661in}}{\pgfqpoint{3.259951in}{3.427611in}}{\pgfqpoint{3.257039in}{3.430523in}}%
\pgfpathcurveto{\pgfqpoint{3.254127in}{3.433435in}}{\pgfqpoint{3.250177in}{3.435072in}}{\pgfqpoint{3.246059in}{3.435072in}}%
\pgfpathcurveto{\pgfqpoint{3.241941in}{3.435072in}}{\pgfqpoint{3.237991in}{3.433435in}}{\pgfqpoint{3.235079in}{3.430523in}}%
\pgfpathcurveto{\pgfqpoint{3.232167in}{3.427611in}}{\pgfqpoint{3.230531in}{3.423661in}}{\pgfqpoint{3.230531in}{3.419543in}}%
\pgfpathcurveto{\pgfqpoint{3.230531in}{3.415425in}}{\pgfqpoint{3.232167in}{3.411475in}}{\pgfqpoint{3.235079in}{3.408563in}}%
\pgfpathcurveto{\pgfqpoint{3.237991in}{3.405651in}}{\pgfqpoint{3.241941in}{3.404015in}}{\pgfqpoint{3.246059in}{3.404015in}}%
\pgfpathclose%
\pgfusepath{fill}%
\end{pgfscope}%
\begin{pgfscope}%
\pgfpathrectangle{\pgfqpoint{0.887500in}{0.275000in}}{\pgfqpoint{4.225000in}{4.225000in}}%
\pgfusepath{clip}%
\pgfsetbuttcap%
\pgfsetroundjoin%
\definecolor{currentfill}{rgb}{0.000000,0.000000,0.000000}%
\pgfsetfillcolor{currentfill}%
\pgfsetfillopacity{0.800000}%
\pgfsetlinewidth{0.000000pt}%
\definecolor{currentstroke}{rgb}{0.000000,0.000000,0.000000}%
\pgfsetstrokecolor{currentstroke}%
\pgfsetstrokeopacity{0.800000}%
\pgfsetdash{}{0pt}%
\pgfpathmoveto{\pgfqpoint{2.463424in}{2.515188in}}%
\pgfpathcurveto{\pgfqpoint{2.467542in}{2.515188in}}{\pgfqpoint{2.471492in}{2.516824in}}{\pgfqpoint{2.474404in}{2.519736in}}%
\pgfpathcurveto{\pgfqpoint{2.477316in}{2.522648in}}{\pgfqpoint{2.478952in}{2.526598in}}{\pgfqpoint{2.478952in}{2.530716in}}%
\pgfpathcurveto{\pgfqpoint{2.478952in}{2.534834in}}{\pgfqpoint{2.477316in}{2.538784in}}{\pgfqpoint{2.474404in}{2.541696in}}%
\pgfpathcurveto{\pgfqpoint{2.471492in}{2.544608in}}{\pgfqpoint{2.467542in}{2.546244in}}{\pgfqpoint{2.463424in}{2.546244in}}%
\pgfpathcurveto{\pgfqpoint{2.459306in}{2.546244in}}{\pgfqpoint{2.455356in}{2.544608in}}{\pgfqpoint{2.452444in}{2.541696in}}%
\pgfpathcurveto{\pgfqpoint{2.449532in}{2.538784in}}{\pgfqpoint{2.447896in}{2.534834in}}{\pgfqpoint{2.447896in}{2.530716in}}%
\pgfpathcurveto{\pgfqpoint{2.447896in}{2.526598in}}{\pgfqpoint{2.449532in}{2.522648in}}{\pgfqpoint{2.452444in}{2.519736in}}%
\pgfpathcurveto{\pgfqpoint{2.455356in}{2.516824in}}{\pgfqpoint{2.459306in}{2.515188in}}{\pgfqpoint{2.463424in}{2.515188in}}%
\pgfpathclose%
\pgfusepath{fill}%
\end{pgfscope}%
\begin{pgfscope}%
\pgfpathrectangle{\pgfqpoint{0.887500in}{0.275000in}}{\pgfqpoint{4.225000in}{4.225000in}}%
\pgfusepath{clip}%
\pgfsetbuttcap%
\pgfsetroundjoin%
\definecolor{currentfill}{rgb}{0.000000,0.000000,0.000000}%
\pgfsetfillcolor{currentfill}%
\pgfsetfillopacity{0.800000}%
\pgfsetlinewidth{0.000000pt}%
\definecolor{currentstroke}{rgb}{0.000000,0.000000,0.000000}%
\pgfsetstrokecolor{currentstroke}%
\pgfsetstrokeopacity{0.800000}%
\pgfsetdash{}{0pt}%
\pgfpathmoveto{\pgfqpoint{2.751594in}{2.551925in}}%
\pgfpathcurveto{\pgfqpoint{2.755712in}{2.551925in}}{\pgfqpoint{2.759662in}{2.553561in}}{\pgfqpoint{2.762574in}{2.556473in}}%
\pgfpathcurveto{\pgfqpoint{2.765486in}{2.559385in}}{\pgfqpoint{2.767122in}{2.563335in}}{\pgfqpoint{2.767122in}{2.567453in}}%
\pgfpathcurveto{\pgfqpoint{2.767122in}{2.571571in}}{\pgfqpoint{2.765486in}{2.575522in}}{\pgfqpoint{2.762574in}{2.578433in}}%
\pgfpathcurveto{\pgfqpoint{2.759662in}{2.581345in}}{\pgfqpoint{2.755712in}{2.582982in}}{\pgfqpoint{2.751594in}{2.582982in}}%
\pgfpathcurveto{\pgfqpoint{2.747476in}{2.582982in}}{\pgfqpoint{2.743526in}{2.581345in}}{\pgfqpoint{2.740614in}{2.578433in}}%
\pgfpathcurveto{\pgfqpoint{2.737702in}{2.575522in}}{\pgfqpoint{2.736066in}{2.571571in}}{\pgfqpoint{2.736066in}{2.567453in}}%
\pgfpathcurveto{\pgfqpoint{2.736066in}{2.563335in}}{\pgfqpoint{2.737702in}{2.559385in}}{\pgfqpoint{2.740614in}{2.556473in}}%
\pgfpathcurveto{\pgfqpoint{2.743526in}{2.553561in}}{\pgfqpoint{2.747476in}{2.551925in}}{\pgfqpoint{2.751594in}{2.551925in}}%
\pgfpathclose%
\pgfusepath{fill}%
\end{pgfscope}%
\begin{pgfscope}%
\pgfpathrectangle{\pgfqpoint{0.887500in}{0.275000in}}{\pgfqpoint{4.225000in}{4.225000in}}%
\pgfusepath{clip}%
\pgfsetbuttcap%
\pgfsetroundjoin%
\definecolor{currentfill}{rgb}{0.000000,0.000000,0.000000}%
\pgfsetfillcolor{currentfill}%
\pgfsetfillopacity{0.800000}%
\pgfsetlinewidth{0.000000pt}%
\definecolor{currentstroke}{rgb}{0.000000,0.000000,0.000000}%
\pgfsetstrokecolor{currentstroke}%
\pgfsetstrokeopacity{0.800000}%
\pgfsetdash{}{0pt}%
\pgfpathmoveto{\pgfqpoint{4.304520in}{2.857087in}}%
\pgfpathcurveto{\pgfqpoint{4.308638in}{2.857087in}}{\pgfqpoint{4.312588in}{2.858723in}}{\pgfqpoint{4.315500in}{2.861635in}}%
\pgfpathcurveto{\pgfqpoint{4.318412in}{2.864547in}}{\pgfqpoint{4.320048in}{2.868497in}}{\pgfqpoint{4.320048in}{2.872615in}}%
\pgfpathcurveto{\pgfqpoint{4.320048in}{2.876733in}}{\pgfqpoint{4.318412in}{2.880683in}}{\pgfqpoint{4.315500in}{2.883595in}}%
\pgfpathcurveto{\pgfqpoint{4.312588in}{2.886507in}}{\pgfqpoint{4.308638in}{2.888143in}}{\pgfqpoint{4.304520in}{2.888143in}}%
\pgfpathcurveto{\pgfqpoint{4.300401in}{2.888143in}}{\pgfqpoint{4.296451in}{2.886507in}}{\pgfqpoint{4.293539in}{2.883595in}}%
\pgfpathcurveto{\pgfqpoint{4.290627in}{2.880683in}}{\pgfqpoint{4.288991in}{2.876733in}}{\pgfqpoint{4.288991in}{2.872615in}}%
\pgfpathcurveto{\pgfqpoint{4.288991in}{2.868497in}}{\pgfqpoint{4.290627in}{2.864547in}}{\pgfqpoint{4.293539in}{2.861635in}}%
\pgfpathcurveto{\pgfqpoint{4.296451in}{2.858723in}}{\pgfqpoint{4.300401in}{2.857087in}}{\pgfqpoint{4.304520in}{2.857087in}}%
\pgfpathclose%
\pgfusepath{fill}%
\end{pgfscope}%
\begin{pgfscope}%
\pgfpathrectangle{\pgfqpoint{0.887500in}{0.275000in}}{\pgfqpoint{4.225000in}{4.225000in}}%
\pgfusepath{clip}%
\pgfsetbuttcap%
\pgfsetroundjoin%
\definecolor{currentfill}{rgb}{0.000000,0.000000,0.000000}%
\pgfsetfillcolor{currentfill}%
\pgfsetfillopacity{0.800000}%
\pgfsetlinewidth{0.000000pt}%
\definecolor{currentstroke}{rgb}{0.000000,0.000000,0.000000}%
\pgfsetstrokecolor{currentstroke}%
\pgfsetstrokeopacity{0.800000}%
\pgfsetdash{}{0pt}%
\pgfpathmoveto{\pgfqpoint{1.998557in}{2.586718in}}%
\pgfpathcurveto{\pgfqpoint{2.002675in}{2.586718in}}{\pgfqpoint{2.006625in}{2.588354in}}{\pgfqpoint{2.009537in}{2.591266in}}%
\pgfpathcurveto{\pgfqpoint{2.012449in}{2.594178in}}{\pgfqpoint{2.014085in}{2.598128in}}{\pgfqpoint{2.014085in}{2.602246in}}%
\pgfpathcurveto{\pgfqpoint{2.014085in}{2.606365in}}{\pgfqpoint{2.012449in}{2.610315in}}{\pgfqpoint{2.009537in}{2.613226in}}%
\pgfpathcurveto{\pgfqpoint{2.006625in}{2.616138in}}{\pgfqpoint{2.002675in}{2.617775in}}{\pgfqpoint{1.998557in}{2.617775in}}%
\pgfpathcurveto{\pgfqpoint{1.994439in}{2.617775in}}{\pgfqpoint{1.990489in}{2.616138in}}{\pgfqpoint{1.987577in}{2.613226in}}%
\pgfpathcurveto{\pgfqpoint{1.984665in}{2.610315in}}{\pgfqpoint{1.983029in}{2.606365in}}{\pgfqpoint{1.983029in}{2.602246in}}%
\pgfpathcurveto{\pgfqpoint{1.983029in}{2.598128in}}{\pgfqpoint{1.984665in}{2.594178in}}{\pgfqpoint{1.987577in}{2.591266in}}%
\pgfpathcurveto{\pgfqpoint{1.990489in}{2.588354in}}{\pgfqpoint{1.994439in}{2.586718in}}{\pgfqpoint{1.998557in}{2.586718in}}%
\pgfpathclose%
\pgfusepath{fill}%
\end{pgfscope}%
\begin{pgfscope}%
\pgfpathrectangle{\pgfqpoint{0.887500in}{0.275000in}}{\pgfqpoint{4.225000in}{4.225000in}}%
\pgfusepath{clip}%
\pgfsetbuttcap%
\pgfsetroundjoin%
\definecolor{currentfill}{rgb}{0.000000,0.000000,0.000000}%
\pgfsetfillcolor{currentfill}%
\pgfsetfillopacity{0.800000}%
\pgfsetlinewidth{0.000000pt}%
\definecolor{currentstroke}{rgb}{0.000000,0.000000,0.000000}%
\pgfsetstrokecolor{currentstroke}%
\pgfsetstrokeopacity{0.800000}%
\pgfsetdash{}{0pt}%
\pgfpathmoveto{\pgfqpoint{3.422618in}{3.337556in}}%
\pgfpathcurveto{\pgfqpoint{3.426737in}{3.337556in}}{\pgfqpoint{3.430687in}{3.339192in}}{\pgfqpoint{3.433599in}{3.342104in}}%
\pgfpathcurveto{\pgfqpoint{3.436511in}{3.345016in}}{\pgfqpoint{3.438147in}{3.348966in}}{\pgfqpoint{3.438147in}{3.353084in}}%
\pgfpathcurveto{\pgfqpoint{3.438147in}{3.357202in}}{\pgfqpoint{3.436511in}{3.361152in}}{\pgfqpoint{3.433599in}{3.364064in}}%
\pgfpathcurveto{\pgfqpoint{3.430687in}{3.366976in}}{\pgfqpoint{3.426737in}{3.368612in}}{\pgfqpoint{3.422618in}{3.368612in}}%
\pgfpathcurveto{\pgfqpoint{3.418500in}{3.368612in}}{\pgfqpoint{3.414550in}{3.366976in}}{\pgfqpoint{3.411638in}{3.364064in}}%
\pgfpathcurveto{\pgfqpoint{3.408726in}{3.361152in}}{\pgfqpoint{3.407090in}{3.357202in}}{\pgfqpoint{3.407090in}{3.353084in}}%
\pgfpathcurveto{\pgfqpoint{3.407090in}{3.348966in}}{\pgfqpoint{3.408726in}{3.345016in}}{\pgfqpoint{3.411638in}{3.342104in}}%
\pgfpathcurveto{\pgfqpoint{3.414550in}{3.339192in}}{\pgfqpoint{3.418500in}{3.337556in}}{\pgfqpoint{3.422618in}{3.337556in}}%
\pgfpathclose%
\pgfusepath{fill}%
\end{pgfscope}%
\begin{pgfscope}%
\pgfpathrectangle{\pgfqpoint{0.887500in}{0.275000in}}{\pgfqpoint{4.225000in}{4.225000in}}%
\pgfusepath{clip}%
\pgfsetbuttcap%
\pgfsetroundjoin%
\definecolor{currentfill}{rgb}{0.000000,0.000000,0.000000}%
\pgfsetfillcolor{currentfill}%
\pgfsetfillopacity{0.800000}%
\pgfsetlinewidth{0.000000pt}%
\definecolor{currentstroke}{rgb}{0.000000,0.000000,0.000000}%
\pgfsetstrokecolor{currentstroke}%
\pgfsetstrokeopacity{0.800000}%
\pgfsetdash{}{0pt}%
\pgfpathmoveto{\pgfqpoint{1.533592in}{2.653752in}}%
\pgfpathcurveto{\pgfqpoint{1.537710in}{2.653752in}}{\pgfqpoint{1.541660in}{2.655388in}}{\pgfqpoint{1.544572in}{2.658300in}}%
\pgfpathcurveto{\pgfqpoint{1.547484in}{2.661212in}}{\pgfqpoint{1.549120in}{2.665162in}}{\pgfqpoint{1.549120in}{2.669280in}}%
\pgfpathcurveto{\pgfqpoint{1.549120in}{2.673398in}}{\pgfqpoint{1.547484in}{2.677348in}}{\pgfqpoint{1.544572in}{2.680260in}}%
\pgfpathcurveto{\pgfqpoint{1.541660in}{2.683172in}}{\pgfqpoint{1.537710in}{2.684808in}}{\pgfqpoint{1.533592in}{2.684808in}}%
\pgfpathcurveto{\pgfqpoint{1.529474in}{2.684808in}}{\pgfqpoint{1.525524in}{2.683172in}}{\pgfqpoint{1.522612in}{2.680260in}}%
\pgfpathcurveto{\pgfqpoint{1.519700in}{2.677348in}}{\pgfqpoint{1.518064in}{2.673398in}}{\pgfqpoint{1.518064in}{2.669280in}}%
\pgfpathcurveto{\pgfqpoint{1.518064in}{2.665162in}}{\pgfqpoint{1.519700in}{2.661212in}}{\pgfqpoint{1.522612in}{2.658300in}}%
\pgfpathcurveto{\pgfqpoint{1.525524in}{2.655388in}}{\pgfqpoint{1.529474in}{2.653752in}}{\pgfqpoint{1.533592in}{2.653752in}}%
\pgfpathclose%
\pgfusepath{fill}%
\end{pgfscope}%
\begin{pgfscope}%
\pgfpathrectangle{\pgfqpoint{0.887500in}{0.275000in}}{\pgfqpoint{4.225000in}{4.225000in}}%
\pgfusepath{clip}%
\pgfsetbuttcap%
\pgfsetroundjoin%
\definecolor{currentfill}{rgb}{0.000000,0.000000,0.000000}%
\pgfsetfillcolor{currentfill}%
\pgfsetfillopacity{0.800000}%
\pgfsetlinewidth{0.000000pt}%
\definecolor{currentstroke}{rgb}{0.000000,0.000000,0.000000}%
\pgfsetstrokecolor{currentstroke}%
\pgfsetstrokeopacity{0.800000}%
\pgfsetdash{}{0pt}%
\pgfpathmoveto{\pgfqpoint{4.128547in}{2.967400in}}%
\pgfpathcurveto{\pgfqpoint{4.132665in}{2.967400in}}{\pgfqpoint{4.136615in}{2.969036in}}{\pgfqpoint{4.139527in}{2.971948in}}%
\pgfpathcurveto{\pgfqpoint{4.142439in}{2.974860in}}{\pgfqpoint{4.144075in}{2.978810in}}{\pgfqpoint{4.144075in}{2.982928in}}%
\pgfpathcurveto{\pgfqpoint{4.144075in}{2.987046in}}{\pgfqpoint{4.142439in}{2.990996in}}{\pgfqpoint{4.139527in}{2.993908in}}%
\pgfpathcurveto{\pgfqpoint{4.136615in}{2.996820in}}{\pgfqpoint{4.132665in}{2.998456in}}{\pgfqpoint{4.128547in}{2.998456in}}%
\pgfpathcurveto{\pgfqpoint{4.124429in}{2.998456in}}{\pgfqpoint{4.120479in}{2.996820in}}{\pgfqpoint{4.117567in}{2.993908in}}%
\pgfpathcurveto{\pgfqpoint{4.114655in}{2.990996in}}{\pgfqpoint{4.113019in}{2.987046in}}{\pgfqpoint{4.113019in}{2.982928in}}%
\pgfpathcurveto{\pgfqpoint{4.113019in}{2.978810in}}{\pgfqpoint{4.114655in}{2.974860in}}{\pgfqpoint{4.117567in}{2.971948in}}%
\pgfpathcurveto{\pgfqpoint{4.120479in}{2.969036in}}{\pgfqpoint{4.124429in}{2.967400in}}{\pgfqpoint{4.128547in}{2.967400in}}%
\pgfpathclose%
\pgfusepath{fill}%
\end{pgfscope}%
\begin{pgfscope}%
\pgfpathrectangle{\pgfqpoint{0.887500in}{0.275000in}}{\pgfqpoint{4.225000in}{4.225000in}}%
\pgfusepath{clip}%
\pgfsetbuttcap%
\pgfsetroundjoin%
\definecolor{currentfill}{rgb}{0.000000,0.000000,0.000000}%
\pgfsetfillcolor{currentfill}%
\pgfsetfillopacity{0.800000}%
\pgfsetlinewidth{0.000000pt}%
\definecolor{currentstroke}{rgb}{0.000000,0.000000,0.000000}%
\pgfsetstrokecolor{currentstroke}%
\pgfsetstrokeopacity{0.800000}%
\pgfsetdash{}{0pt}%
\pgfpathmoveto{\pgfqpoint{3.599238in}{3.257263in}}%
\pgfpathcurveto{\pgfqpoint{3.603356in}{3.257263in}}{\pgfqpoint{3.607306in}{3.258899in}}{\pgfqpoint{3.610218in}{3.261811in}}%
\pgfpathcurveto{\pgfqpoint{3.613130in}{3.264723in}}{\pgfqpoint{3.614767in}{3.268673in}}{\pgfqpoint{3.614767in}{3.272791in}}%
\pgfpathcurveto{\pgfqpoint{3.614767in}{3.276910in}}{\pgfqpoint{3.613130in}{3.280860in}}{\pgfqpoint{3.610218in}{3.283772in}}%
\pgfpathcurveto{\pgfqpoint{3.607306in}{3.286683in}}{\pgfqpoint{3.603356in}{3.288320in}}{\pgfqpoint{3.599238in}{3.288320in}}%
\pgfpathcurveto{\pgfqpoint{3.595120in}{3.288320in}}{\pgfqpoint{3.591170in}{3.286683in}}{\pgfqpoint{3.588258in}{3.283772in}}%
\pgfpathcurveto{\pgfqpoint{3.585346in}{3.280860in}}{\pgfqpoint{3.583710in}{3.276910in}}{\pgfqpoint{3.583710in}{3.272791in}}%
\pgfpathcurveto{\pgfqpoint{3.583710in}{3.268673in}}{\pgfqpoint{3.585346in}{3.264723in}}{\pgfqpoint{3.588258in}{3.261811in}}%
\pgfpathcurveto{\pgfqpoint{3.591170in}{3.258899in}}{\pgfqpoint{3.595120in}{3.257263in}}{\pgfqpoint{3.599238in}{3.257263in}}%
\pgfpathclose%
\pgfusepath{fill}%
\end{pgfscope}%
\begin{pgfscope}%
\pgfpathrectangle{\pgfqpoint{0.887500in}{0.275000in}}{\pgfqpoint{4.225000in}{4.225000in}}%
\pgfusepath{clip}%
\pgfsetbuttcap%
\pgfsetroundjoin%
\definecolor{currentfill}{rgb}{0.000000,0.000000,0.000000}%
\pgfsetfillcolor{currentfill}%
\pgfsetfillopacity{0.800000}%
\pgfsetlinewidth{0.000000pt}%
\definecolor{currentstroke}{rgb}{0.000000,0.000000,0.000000}%
\pgfsetstrokecolor{currentstroke}%
\pgfsetstrokeopacity{0.800000}%
\pgfsetdash{}{0pt}%
\pgfpathmoveto{\pgfqpoint{3.775771in}{3.165027in}}%
\pgfpathcurveto{\pgfqpoint{3.779889in}{3.165027in}}{\pgfqpoint{3.783839in}{3.166663in}}{\pgfqpoint{3.786751in}{3.169575in}}%
\pgfpathcurveto{\pgfqpoint{3.789663in}{3.172487in}}{\pgfqpoint{3.791299in}{3.176437in}}{\pgfqpoint{3.791299in}{3.180555in}}%
\pgfpathcurveto{\pgfqpoint{3.791299in}{3.184673in}}{\pgfqpoint{3.789663in}{3.188623in}}{\pgfqpoint{3.786751in}{3.191535in}}%
\pgfpathcurveto{\pgfqpoint{3.783839in}{3.194447in}}{\pgfqpoint{3.779889in}{3.196083in}}{\pgfqpoint{3.775771in}{3.196083in}}%
\pgfpathcurveto{\pgfqpoint{3.771653in}{3.196083in}}{\pgfqpoint{3.767703in}{3.194447in}}{\pgfqpoint{3.764791in}{3.191535in}}%
\pgfpathcurveto{\pgfqpoint{3.761879in}{3.188623in}}{\pgfqpoint{3.760243in}{3.184673in}}{\pgfqpoint{3.760243in}{3.180555in}}%
\pgfpathcurveto{\pgfqpoint{3.760243in}{3.176437in}}{\pgfqpoint{3.761879in}{3.172487in}}{\pgfqpoint{3.764791in}{3.169575in}}%
\pgfpathcurveto{\pgfqpoint{3.767703in}{3.166663in}}{\pgfqpoint{3.771653in}{3.165027in}}{\pgfqpoint{3.775771in}{3.165027in}}%
\pgfpathclose%
\pgfusepath{fill}%
\end{pgfscope}%
\begin{pgfscope}%
\pgfpathrectangle{\pgfqpoint{0.887500in}{0.275000in}}{\pgfqpoint{4.225000in}{4.225000in}}%
\pgfusepath{clip}%
\pgfsetbuttcap%
\pgfsetroundjoin%
\definecolor{currentfill}{rgb}{0.000000,0.000000,0.000000}%
\pgfsetfillcolor{currentfill}%
\pgfsetfillopacity{0.800000}%
\pgfsetlinewidth{0.000000pt}%
\definecolor{currentstroke}{rgb}{0.000000,0.000000,0.000000}%
\pgfsetstrokecolor{currentstroke}%
\pgfsetstrokeopacity{0.800000}%
\pgfsetdash{}{0pt}%
\pgfpathmoveto{\pgfqpoint{3.952312in}{3.072260in}}%
\pgfpathcurveto{\pgfqpoint{3.956431in}{3.072260in}}{\pgfqpoint{3.960381in}{3.073896in}}{\pgfqpoint{3.963293in}{3.076808in}}%
\pgfpathcurveto{\pgfqpoint{3.966205in}{3.079720in}}{\pgfqpoint{3.967841in}{3.083670in}}{\pgfqpoint{3.967841in}{3.087789in}}%
\pgfpathcurveto{\pgfqpoint{3.967841in}{3.091907in}}{\pgfqpoint{3.966205in}{3.095857in}}{\pgfqpoint{3.963293in}{3.098769in}}%
\pgfpathcurveto{\pgfqpoint{3.960381in}{3.101681in}}{\pgfqpoint{3.956431in}{3.103317in}}{\pgfqpoint{3.952312in}{3.103317in}}%
\pgfpathcurveto{\pgfqpoint{3.948194in}{3.103317in}}{\pgfqpoint{3.944244in}{3.101681in}}{\pgfqpoint{3.941332in}{3.098769in}}%
\pgfpathcurveto{\pgfqpoint{3.938420in}{3.095857in}}{\pgfqpoint{3.936784in}{3.091907in}}{\pgfqpoint{3.936784in}{3.087789in}}%
\pgfpathcurveto{\pgfqpoint{3.936784in}{3.083670in}}{\pgfqpoint{3.938420in}{3.079720in}}{\pgfqpoint{3.941332in}{3.076808in}}%
\pgfpathcurveto{\pgfqpoint{3.944244in}{3.073896in}}{\pgfqpoint{3.948194in}{3.072260in}}{\pgfqpoint{3.952312in}{3.072260in}}%
\pgfpathclose%
\pgfusepath{fill}%
\end{pgfscope}%
\begin{pgfscope}%
\pgfpathrectangle{\pgfqpoint{0.887500in}{0.275000in}}{\pgfqpoint{4.225000in}{4.225000in}}%
\pgfusepath{clip}%
\pgfsetbuttcap%
\pgfsetroundjoin%
\definecolor{currentfill}{rgb}{0.000000,0.000000,0.000000}%
\pgfsetfillcolor{currentfill}%
\pgfsetfillopacity{0.800000}%
\pgfsetlinewidth{0.000000pt}%
\definecolor{currentstroke}{rgb}{0.000000,0.000000,0.000000}%
\pgfsetstrokecolor{currentstroke}%
\pgfsetstrokeopacity{0.800000}%
\pgfsetdash{}{0pt}%
\pgfpathmoveto{\pgfqpoint{2.957855in}{3.420085in}}%
\pgfpathcurveto{\pgfqpoint{2.961973in}{3.420085in}}{\pgfqpoint{2.965923in}{3.421721in}}{\pgfqpoint{2.968835in}{3.424633in}}%
\pgfpathcurveto{\pgfqpoint{2.971747in}{3.427545in}}{\pgfqpoint{2.973383in}{3.431495in}}{\pgfqpoint{2.973383in}{3.435613in}}%
\pgfpathcurveto{\pgfqpoint{2.973383in}{3.439732in}}{\pgfqpoint{2.971747in}{3.443682in}}{\pgfqpoint{2.968835in}{3.446594in}}%
\pgfpathcurveto{\pgfqpoint{2.965923in}{3.449506in}}{\pgfqpoint{2.961973in}{3.451142in}}{\pgfqpoint{2.957855in}{3.451142in}}%
\pgfpathcurveto{\pgfqpoint{2.953737in}{3.451142in}}{\pgfqpoint{2.949787in}{3.449506in}}{\pgfqpoint{2.946875in}{3.446594in}}%
\pgfpathcurveto{\pgfqpoint{2.943963in}{3.443682in}}{\pgfqpoint{2.942327in}{3.439732in}}{\pgfqpoint{2.942327in}{3.435613in}}%
\pgfpathcurveto{\pgfqpoint{2.942327in}{3.431495in}}{\pgfqpoint{2.943963in}{3.427545in}}{\pgfqpoint{2.946875in}{3.424633in}}%
\pgfpathcurveto{\pgfqpoint{2.949787in}{3.421721in}}{\pgfqpoint{2.953737in}{3.420085in}}{\pgfqpoint{2.957855in}{3.420085in}}%
\pgfpathclose%
\pgfusepath{fill}%
\end{pgfscope}%
\begin{pgfscope}%
\pgfpathrectangle{\pgfqpoint{0.887500in}{0.275000in}}{\pgfqpoint{4.225000in}{4.225000in}}%
\pgfusepath{clip}%
\pgfsetbuttcap%
\pgfsetroundjoin%
\definecolor{currentfill}{rgb}{0.000000,0.000000,0.000000}%
\pgfsetfillcolor{currentfill}%
\pgfsetfillopacity{0.800000}%
\pgfsetlinewidth{0.000000pt}%
\definecolor{currentstroke}{rgb}{0.000000,0.000000,0.000000}%
\pgfsetstrokecolor{currentstroke}%
\pgfsetstrokeopacity{0.800000}%
\pgfsetdash{}{0pt}%
\pgfpathmoveto{\pgfqpoint{2.639376in}{2.460445in}}%
\pgfpathcurveto{\pgfqpoint{2.643494in}{2.460445in}}{\pgfqpoint{2.647444in}{2.462081in}}{\pgfqpoint{2.650356in}{2.464993in}}%
\pgfpathcurveto{\pgfqpoint{2.653268in}{2.467905in}}{\pgfqpoint{2.654904in}{2.471855in}}{\pgfqpoint{2.654904in}{2.475973in}}%
\pgfpathcurveto{\pgfqpoint{2.654904in}{2.480091in}}{\pgfqpoint{2.653268in}{2.484041in}}{\pgfqpoint{2.650356in}{2.486953in}}%
\pgfpathcurveto{\pgfqpoint{2.647444in}{2.489865in}}{\pgfqpoint{2.643494in}{2.491501in}}{\pgfqpoint{2.639376in}{2.491501in}}%
\pgfpathcurveto{\pgfqpoint{2.635258in}{2.491501in}}{\pgfqpoint{2.631308in}{2.489865in}}{\pgfqpoint{2.628396in}{2.486953in}}%
\pgfpathcurveto{\pgfqpoint{2.625484in}{2.484041in}}{\pgfqpoint{2.623848in}{2.480091in}}{\pgfqpoint{2.623848in}{2.475973in}}%
\pgfpathcurveto{\pgfqpoint{2.623848in}{2.471855in}}{\pgfqpoint{2.625484in}{2.467905in}}{\pgfqpoint{2.628396in}{2.464993in}}%
\pgfpathcurveto{\pgfqpoint{2.631308in}{2.462081in}}{\pgfqpoint{2.635258in}{2.460445in}}{\pgfqpoint{2.639376in}{2.460445in}}%
\pgfpathclose%
\pgfusepath{fill}%
\end{pgfscope}%
\begin{pgfscope}%
\pgfpathrectangle{\pgfqpoint{0.887500in}{0.275000in}}{\pgfqpoint{4.225000in}{4.225000in}}%
\pgfusepath{clip}%
\pgfsetbuttcap%
\pgfsetroundjoin%
\definecolor{currentfill}{rgb}{0.000000,0.000000,0.000000}%
\pgfsetfillcolor{currentfill}%
\pgfsetfillopacity{0.800000}%
\pgfsetlinewidth{0.000000pt}%
\definecolor{currentstroke}{rgb}{0.000000,0.000000,0.000000}%
\pgfsetstrokecolor{currentstroke}%
\pgfsetstrokeopacity{0.800000}%
\pgfsetdash{}{0pt}%
\pgfpathmoveto{\pgfqpoint{2.863215in}{2.722201in}}%
\pgfpathcurveto{\pgfqpoint{2.867333in}{2.722201in}}{\pgfqpoint{2.871283in}{2.723837in}}{\pgfqpoint{2.874195in}{2.726749in}}%
\pgfpathcurveto{\pgfqpoint{2.877107in}{2.729661in}}{\pgfqpoint{2.878743in}{2.733611in}}{\pgfqpoint{2.878743in}{2.737729in}}%
\pgfpathcurveto{\pgfqpoint{2.878743in}{2.741847in}}{\pgfqpoint{2.877107in}{2.745797in}}{\pgfqpoint{2.874195in}{2.748709in}}%
\pgfpathcurveto{\pgfqpoint{2.871283in}{2.751621in}}{\pgfqpoint{2.867333in}{2.753257in}}{\pgfqpoint{2.863215in}{2.753257in}}%
\pgfpathcurveto{\pgfqpoint{2.859097in}{2.753257in}}{\pgfqpoint{2.855147in}{2.751621in}}{\pgfqpoint{2.852235in}{2.748709in}}%
\pgfpathcurveto{\pgfqpoint{2.849323in}{2.745797in}}{\pgfqpoint{2.847686in}{2.741847in}}{\pgfqpoint{2.847686in}{2.737729in}}%
\pgfpathcurveto{\pgfqpoint{2.847686in}{2.733611in}}{\pgfqpoint{2.849323in}{2.729661in}}{\pgfqpoint{2.852235in}{2.726749in}}%
\pgfpathcurveto{\pgfqpoint{2.855147in}{2.723837in}}{\pgfqpoint{2.859097in}{2.722201in}}{\pgfqpoint{2.863215in}{2.722201in}}%
\pgfpathclose%
\pgfusepath{fill}%
\end{pgfscope}%
\begin{pgfscope}%
\pgfpathrectangle{\pgfqpoint{0.887500in}{0.275000in}}{\pgfqpoint{4.225000in}{4.225000in}}%
\pgfusepath{clip}%
\pgfsetbuttcap%
\pgfsetroundjoin%
\definecolor{currentfill}{rgb}{0.000000,0.000000,0.000000}%
\pgfsetfillcolor{currentfill}%
\pgfsetfillopacity{0.800000}%
\pgfsetlinewidth{0.000000pt}%
\definecolor{currentstroke}{rgb}{0.000000,0.000000,0.000000}%
\pgfsetstrokecolor{currentstroke}%
\pgfsetstrokeopacity{0.800000}%
\pgfsetdash{}{0pt}%
\pgfpathmoveto{\pgfqpoint{2.174034in}{2.534680in}}%
\pgfpathcurveto{\pgfqpoint{2.178152in}{2.534680in}}{\pgfqpoint{2.182102in}{2.536316in}}{\pgfqpoint{2.185014in}{2.539228in}}%
\pgfpathcurveto{\pgfqpoint{2.187926in}{2.542140in}}{\pgfqpoint{2.189562in}{2.546090in}}{\pgfqpoint{2.189562in}{2.550208in}}%
\pgfpathcurveto{\pgfqpoint{2.189562in}{2.554326in}}{\pgfqpoint{2.187926in}{2.558276in}}{\pgfqpoint{2.185014in}{2.561188in}}%
\pgfpathcurveto{\pgfqpoint{2.182102in}{2.564100in}}{\pgfqpoint{2.178152in}{2.565736in}}{\pgfqpoint{2.174034in}{2.565736in}}%
\pgfpathcurveto{\pgfqpoint{2.169915in}{2.565736in}}{\pgfqpoint{2.165965in}{2.564100in}}{\pgfqpoint{2.163053in}{2.561188in}}%
\pgfpathcurveto{\pgfqpoint{2.160141in}{2.558276in}}{\pgfqpoint{2.158505in}{2.554326in}}{\pgfqpoint{2.158505in}{2.550208in}}%
\pgfpathcurveto{\pgfqpoint{2.158505in}{2.546090in}}{\pgfqpoint{2.160141in}{2.542140in}}{\pgfqpoint{2.163053in}{2.539228in}}%
\pgfpathcurveto{\pgfqpoint{2.165965in}{2.536316in}}{\pgfqpoint{2.169915in}{2.534680in}}{\pgfqpoint{2.174034in}{2.534680in}}%
\pgfpathclose%
\pgfusepath{fill}%
\end{pgfscope}%
\begin{pgfscope}%
\pgfpathrectangle{\pgfqpoint{0.887500in}{0.275000in}}{\pgfqpoint{4.225000in}{4.225000in}}%
\pgfusepath{clip}%
\pgfsetbuttcap%
\pgfsetroundjoin%
\definecolor{currentfill}{rgb}{0.000000,0.000000,0.000000}%
\pgfsetfillcolor{currentfill}%
\pgfsetfillopacity{0.800000}%
\pgfsetlinewidth{0.000000pt}%
\definecolor{currentstroke}{rgb}{0.000000,0.000000,0.000000}%
\pgfsetstrokecolor{currentstroke}%
\pgfsetstrokeopacity{0.800000}%
\pgfsetdash{}{0pt}%
\pgfpathmoveto{\pgfqpoint{1.708493in}{2.604943in}}%
\pgfpathcurveto{\pgfqpoint{1.712611in}{2.604943in}}{\pgfqpoint{1.716561in}{2.606580in}}{\pgfqpoint{1.719473in}{2.609492in}}%
\pgfpathcurveto{\pgfqpoint{1.722385in}{2.612404in}}{\pgfqpoint{1.724021in}{2.616354in}}{\pgfqpoint{1.724021in}{2.620472in}}%
\pgfpathcurveto{\pgfqpoint{1.724021in}{2.624590in}}{\pgfqpoint{1.722385in}{2.628540in}}{\pgfqpoint{1.719473in}{2.631452in}}%
\pgfpathcurveto{\pgfqpoint{1.716561in}{2.634364in}}{\pgfqpoint{1.712611in}{2.636000in}}{\pgfqpoint{1.708493in}{2.636000in}}%
\pgfpathcurveto{\pgfqpoint{1.704375in}{2.636000in}}{\pgfqpoint{1.700425in}{2.634364in}}{\pgfqpoint{1.697513in}{2.631452in}}%
\pgfpathcurveto{\pgfqpoint{1.694601in}{2.628540in}}{\pgfqpoint{1.692965in}{2.624590in}}{\pgfqpoint{1.692965in}{2.620472in}}%
\pgfpathcurveto{\pgfqpoint{1.692965in}{2.616354in}}{\pgfqpoint{1.694601in}{2.612404in}}{\pgfqpoint{1.697513in}{2.609492in}}%
\pgfpathcurveto{\pgfqpoint{1.700425in}{2.606580in}}{\pgfqpoint{1.704375in}{2.604943in}}{\pgfqpoint{1.708493in}{2.604943in}}%
\pgfpathclose%
\pgfusepath{fill}%
\end{pgfscope}%
\begin{pgfscope}%
\pgfpathrectangle{\pgfqpoint{0.887500in}{0.275000in}}{\pgfqpoint{4.225000in}{4.225000in}}%
\pgfusepath{clip}%
\pgfsetbuttcap%
\pgfsetroundjoin%
\definecolor{currentfill}{rgb}{0.000000,0.000000,0.000000}%
\pgfsetfillcolor{currentfill}%
\pgfsetfillopacity{0.800000}%
\pgfsetlinewidth{0.000000pt}%
\definecolor{currentstroke}{rgb}{0.000000,0.000000,0.000000}%
\pgfsetstrokecolor{currentstroke}%
\pgfsetstrokeopacity{0.800000}%
\pgfsetdash{}{0pt}%
\pgfpathmoveto{\pgfqpoint{2.815693in}{2.403802in}}%
\pgfpathcurveto{\pgfqpoint{2.819811in}{2.403802in}}{\pgfqpoint{2.823761in}{2.405438in}}{\pgfqpoint{2.826673in}{2.408350in}}%
\pgfpathcurveto{\pgfqpoint{2.829585in}{2.411262in}}{\pgfqpoint{2.831221in}{2.415212in}}{\pgfqpoint{2.831221in}{2.419330in}}%
\pgfpathcurveto{\pgfqpoint{2.831221in}{2.423449in}}{\pgfqpoint{2.829585in}{2.427399in}}{\pgfqpoint{2.826673in}{2.430311in}}%
\pgfpathcurveto{\pgfqpoint{2.823761in}{2.433223in}}{\pgfqpoint{2.819811in}{2.434859in}}{\pgfqpoint{2.815693in}{2.434859in}}%
\pgfpathcurveto{\pgfqpoint{2.811575in}{2.434859in}}{\pgfqpoint{2.807625in}{2.433223in}}{\pgfqpoint{2.804713in}{2.430311in}}%
\pgfpathcurveto{\pgfqpoint{2.801801in}{2.427399in}}{\pgfqpoint{2.800164in}{2.423449in}}{\pgfqpoint{2.800164in}{2.419330in}}%
\pgfpathcurveto{\pgfqpoint{2.800164in}{2.415212in}}{\pgfqpoint{2.801801in}{2.411262in}}{\pgfqpoint{2.804713in}{2.408350in}}%
\pgfpathcurveto{\pgfqpoint{2.807625in}{2.405438in}}{\pgfqpoint{2.811575in}{2.403802in}}{\pgfqpoint{2.815693in}{2.403802in}}%
\pgfpathclose%
\pgfusepath{fill}%
\end{pgfscope}%
\begin{pgfscope}%
\pgfpathrectangle{\pgfqpoint{0.887500in}{0.275000in}}{\pgfqpoint{4.225000in}{4.225000in}}%
\pgfusepath{clip}%
\pgfsetbuttcap%
\pgfsetroundjoin%
\definecolor{currentfill}{rgb}{0.000000,0.000000,0.000000}%
\pgfsetfillcolor{currentfill}%
\pgfsetfillopacity{0.800000}%
\pgfsetlinewidth{0.000000pt}%
\definecolor{currentstroke}{rgb}{0.000000,0.000000,0.000000}%
\pgfsetstrokecolor{currentstroke}%
\pgfsetstrokeopacity{0.800000}%
\pgfsetdash{}{0pt}%
\pgfpathmoveto{\pgfqpoint{4.371587in}{2.709279in}}%
\pgfpathcurveto{\pgfqpoint{4.375705in}{2.709279in}}{\pgfqpoint{4.379655in}{2.710915in}}{\pgfqpoint{4.382567in}{2.713827in}}%
\pgfpathcurveto{\pgfqpoint{4.385479in}{2.716739in}}{\pgfqpoint{4.387116in}{2.720689in}}{\pgfqpoint{4.387116in}{2.724807in}}%
\pgfpathcurveto{\pgfqpoint{4.387116in}{2.728925in}}{\pgfqpoint{4.385479in}{2.732875in}}{\pgfqpoint{4.382567in}{2.735787in}}%
\pgfpathcurveto{\pgfqpoint{4.379655in}{2.738699in}}{\pgfqpoint{4.375705in}{2.740335in}}{\pgfqpoint{4.371587in}{2.740335in}}%
\pgfpathcurveto{\pgfqpoint{4.367469in}{2.740335in}}{\pgfqpoint{4.363519in}{2.738699in}}{\pgfqpoint{4.360607in}{2.735787in}}%
\pgfpathcurveto{\pgfqpoint{4.357695in}{2.732875in}}{\pgfqpoint{4.356059in}{2.728925in}}{\pgfqpoint{4.356059in}{2.724807in}}%
\pgfpathcurveto{\pgfqpoint{4.356059in}{2.720689in}}{\pgfqpoint{4.357695in}{2.716739in}}{\pgfqpoint{4.360607in}{2.713827in}}%
\pgfpathcurveto{\pgfqpoint{4.363519in}{2.710915in}}{\pgfqpoint{4.367469in}{2.709279in}}{\pgfqpoint{4.371587in}{2.709279in}}%
\pgfpathclose%
\pgfusepath{fill}%
\end{pgfscope}%
\begin{pgfscope}%
\pgfpathrectangle{\pgfqpoint{0.887500in}{0.275000in}}{\pgfqpoint{4.225000in}{4.225000in}}%
\pgfusepath{clip}%
\pgfsetbuttcap%
\pgfsetroundjoin%
\definecolor{currentfill}{rgb}{0.000000,0.000000,0.000000}%
\pgfsetfillcolor{currentfill}%
\pgfsetfillopacity{0.800000}%
\pgfsetlinewidth{0.000000pt}%
\definecolor{currentstroke}{rgb}{0.000000,0.000000,0.000000}%
\pgfsetstrokecolor{currentstroke}%
\pgfsetstrokeopacity{0.800000}%
\pgfsetdash{}{0pt}%
\pgfpathmoveto{\pgfqpoint{3.134486in}{3.380087in}}%
\pgfpathcurveto{\pgfqpoint{3.138604in}{3.380087in}}{\pgfqpoint{3.142554in}{3.381723in}}{\pgfqpoint{3.145466in}{3.384635in}}%
\pgfpathcurveto{\pgfqpoint{3.148378in}{3.387547in}}{\pgfqpoint{3.150015in}{3.391497in}}{\pgfqpoint{3.150015in}{3.395615in}}%
\pgfpathcurveto{\pgfqpoint{3.150015in}{3.399733in}}{\pgfqpoint{3.148378in}{3.403683in}}{\pgfqpoint{3.145466in}{3.406595in}}%
\pgfpathcurveto{\pgfqpoint{3.142554in}{3.409507in}}{\pgfqpoint{3.138604in}{3.411143in}}{\pgfqpoint{3.134486in}{3.411143in}}%
\pgfpathcurveto{\pgfqpoint{3.130368in}{3.411143in}}{\pgfqpoint{3.126418in}{3.409507in}}{\pgfqpoint{3.123506in}{3.406595in}}%
\pgfpathcurveto{\pgfqpoint{3.120594in}{3.403683in}}{\pgfqpoint{3.118958in}{3.399733in}}{\pgfqpoint{3.118958in}{3.395615in}}%
\pgfpathcurveto{\pgfqpoint{3.118958in}{3.391497in}}{\pgfqpoint{3.120594in}{3.387547in}}{\pgfqpoint{3.123506in}{3.384635in}}%
\pgfpathcurveto{\pgfqpoint{3.126418in}{3.381723in}}{\pgfqpoint{3.130368in}{3.380087in}}{\pgfqpoint{3.134486in}{3.380087in}}%
\pgfpathclose%
\pgfusepath{fill}%
\end{pgfscope}%
\begin{pgfscope}%
\pgfpathrectangle{\pgfqpoint{0.887500in}{0.275000in}}{\pgfqpoint{4.225000in}{4.225000in}}%
\pgfusepath{clip}%
\pgfsetbuttcap%
\pgfsetroundjoin%
\definecolor{currentfill}{rgb}{0.000000,0.000000,0.000000}%
\pgfsetfillcolor{currentfill}%
\pgfsetfillopacity{0.800000}%
\pgfsetlinewidth{0.000000pt}%
\definecolor{currentstroke}{rgb}{0.000000,0.000000,0.000000}%
\pgfsetstrokecolor{currentstroke}%
\pgfsetstrokeopacity{0.800000}%
\pgfsetdash{}{0pt}%
\pgfpathmoveto{\pgfqpoint{2.349905in}{2.481504in}}%
\pgfpathcurveto{\pgfqpoint{2.354023in}{2.481504in}}{\pgfqpoint{2.357974in}{2.483140in}}{\pgfqpoint{2.360885in}{2.486052in}}%
\pgfpathcurveto{\pgfqpoint{2.363797in}{2.488964in}}{\pgfqpoint{2.365434in}{2.492914in}}{\pgfqpoint{2.365434in}{2.497032in}}%
\pgfpathcurveto{\pgfqpoint{2.365434in}{2.501151in}}{\pgfqpoint{2.363797in}{2.505101in}}{\pgfqpoint{2.360885in}{2.508013in}}%
\pgfpathcurveto{\pgfqpoint{2.357974in}{2.510925in}}{\pgfqpoint{2.354023in}{2.512561in}}{\pgfqpoint{2.349905in}{2.512561in}}%
\pgfpathcurveto{\pgfqpoint{2.345787in}{2.512561in}}{\pgfqpoint{2.341837in}{2.510925in}}{\pgfqpoint{2.338925in}{2.508013in}}%
\pgfpathcurveto{\pgfqpoint{2.336013in}{2.505101in}}{\pgfqpoint{2.334377in}{2.501151in}}{\pgfqpoint{2.334377in}{2.497032in}}%
\pgfpathcurveto{\pgfqpoint{2.334377in}{2.492914in}}{\pgfqpoint{2.336013in}{2.488964in}}{\pgfqpoint{2.338925in}{2.486052in}}%
\pgfpathcurveto{\pgfqpoint{2.341837in}{2.483140in}}{\pgfqpoint{2.345787in}{2.481504in}}{\pgfqpoint{2.349905in}{2.481504in}}%
\pgfpathclose%
\pgfusepath{fill}%
\end{pgfscope}%
\begin{pgfscope}%
\pgfpathrectangle{\pgfqpoint{0.887500in}{0.275000in}}{\pgfqpoint{4.225000in}{4.225000in}}%
\pgfusepath{clip}%
\pgfsetbuttcap%
\pgfsetroundjoin%
\definecolor{currentfill}{rgb}{0.000000,0.000000,0.000000}%
\pgfsetfillcolor{currentfill}%
\pgfsetfillopacity{0.800000}%
\pgfsetlinewidth{0.000000pt}%
\definecolor{currentstroke}{rgb}{0.000000,0.000000,0.000000}%
\pgfsetstrokecolor{currentstroke}%
\pgfsetstrokeopacity{0.800000}%
\pgfsetdash{}{0pt}%
\pgfpathmoveto{\pgfqpoint{4.195369in}{2.821984in}}%
\pgfpathcurveto{\pgfqpoint{4.199487in}{2.821984in}}{\pgfqpoint{4.203437in}{2.823620in}}{\pgfqpoint{4.206349in}{2.826532in}}%
\pgfpathcurveto{\pgfqpoint{4.209261in}{2.829444in}}{\pgfqpoint{4.210897in}{2.833394in}}{\pgfqpoint{4.210897in}{2.837512in}}%
\pgfpathcurveto{\pgfqpoint{4.210897in}{2.841631in}}{\pgfqpoint{4.209261in}{2.845581in}}{\pgfqpoint{4.206349in}{2.848493in}}%
\pgfpathcurveto{\pgfqpoint{4.203437in}{2.851405in}}{\pgfqpoint{4.199487in}{2.853041in}}{\pgfqpoint{4.195369in}{2.853041in}}%
\pgfpathcurveto{\pgfqpoint{4.191251in}{2.853041in}}{\pgfqpoint{4.187301in}{2.851405in}}{\pgfqpoint{4.184389in}{2.848493in}}%
\pgfpathcurveto{\pgfqpoint{4.181477in}{2.845581in}}{\pgfqpoint{4.179841in}{2.841631in}}{\pgfqpoint{4.179841in}{2.837512in}}%
\pgfpathcurveto{\pgfqpoint{4.179841in}{2.833394in}}{\pgfqpoint{4.181477in}{2.829444in}}{\pgfqpoint{4.184389in}{2.826532in}}%
\pgfpathcurveto{\pgfqpoint{4.187301in}{2.823620in}}{\pgfqpoint{4.191251in}{2.821984in}}{\pgfqpoint{4.195369in}{2.821984in}}%
\pgfpathclose%
\pgfusepath{fill}%
\end{pgfscope}%
\begin{pgfscope}%
\pgfpathrectangle{\pgfqpoint{0.887500in}{0.275000in}}{\pgfqpoint{4.225000in}{4.225000in}}%
\pgfusepath{clip}%
\pgfsetbuttcap%
\pgfsetroundjoin%
\definecolor{currentfill}{rgb}{0.000000,0.000000,0.000000}%
\pgfsetfillcolor{currentfill}%
\pgfsetfillopacity{0.800000}%
\pgfsetlinewidth{0.000000pt}%
\definecolor{currentstroke}{rgb}{0.000000,0.000000,0.000000}%
\pgfsetstrokecolor{currentstroke}%
\pgfsetstrokeopacity{0.800000}%
\pgfsetdash{}{0pt}%
\pgfpathmoveto{\pgfqpoint{3.311354in}{3.301662in}}%
\pgfpathcurveto{\pgfqpoint{3.315472in}{3.301662in}}{\pgfqpoint{3.319422in}{3.303298in}}{\pgfqpoint{3.322334in}{3.306210in}}%
\pgfpathcurveto{\pgfqpoint{3.325246in}{3.309122in}}{\pgfqpoint{3.326882in}{3.313072in}}{\pgfqpoint{3.326882in}{3.317190in}}%
\pgfpathcurveto{\pgfqpoint{3.326882in}{3.321308in}}{\pgfqpoint{3.325246in}{3.325258in}}{\pgfqpoint{3.322334in}{3.328170in}}%
\pgfpathcurveto{\pgfqpoint{3.319422in}{3.331082in}}{\pgfqpoint{3.315472in}{3.332718in}}{\pgfqpoint{3.311354in}{3.332718in}}%
\pgfpathcurveto{\pgfqpoint{3.307236in}{3.332718in}}{\pgfqpoint{3.303286in}{3.331082in}}{\pgfqpoint{3.300374in}{3.328170in}}%
\pgfpathcurveto{\pgfqpoint{3.297462in}{3.325258in}}{\pgfqpoint{3.295826in}{3.321308in}}{\pgfqpoint{3.295826in}{3.317190in}}%
\pgfpathcurveto{\pgfqpoint{3.295826in}{3.313072in}}{\pgfqpoint{3.297462in}{3.309122in}}{\pgfqpoint{3.300374in}{3.306210in}}%
\pgfpathcurveto{\pgfqpoint{3.303286in}{3.303298in}}{\pgfqpoint{3.307236in}{3.301662in}}{\pgfqpoint{3.311354in}{3.301662in}}%
\pgfpathclose%
\pgfusepath{fill}%
\end{pgfscope}%
\begin{pgfscope}%
\pgfpathrectangle{\pgfqpoint{0.887500in}{0.275000in}}{\pgfqpoint{4.225000in}{4.225000in}}%
\pgfusepath{clip}%
\pgfsetbuttcap%
\pgfsetroundjoin%
\definecolor{currentfill}{rgb}{0.000000,0.000000,0.000000}%
\pgfsetfillcolor{currentfill}%
\pgfsetfillopacity{0.800000}%
\pgfsetlinewidth{0.000000pt}%
\definecolor{currentstroke}{rgb}{0.000000,0.000000,0.000000}%
\pgfsetstrokecolor{currentstroke}%
\pgfsetstrokeopacity{0.800000}%
\pgfsetdash{}{0pt}%
\pgfpathmoveto{\pgfqpoint{1.883859in}{2.553968in}}%
\pgfpathcurveto{\pgfqpoint{1.887977in}{2.553968in}}{\pgfqpoint{1.891927in}{2.555604in}}{\pgfqpoint{1.894839in}{2.558516in}}%
\pgfpathcurveto{\pgfqpoint{1.897751in}{2.561428in}}{\pgfqpoint{1.899387in}{2.565378in}}{\pgfqpoint{1.899387in}{2.569496in}}%
\pgfpathcurveto{\pgfqpoint{1.899387in}{2.573614in}}{\pgfqpoint{1.897751in}{2.577564in}}{\pgfqpoint{1.894839in}{2.580476in}}%
\pgfpathcurveto{\pgfqpoint{1.891927in}{2.583388in}}{\pgfqpoint{1.887977in}{2.585024in}}{\pgfqpoint{1.883859in}{2.585024in}}%
\pgfpathcurveto{\pgfqpoint{1.879741in}{2.585024in}}{\pgfqpoint{1.875791in}{2.583388in}}{\pgfqpoint{1.872879in}{2.580476in}}%
\pgfpathcurveto{\pgfqpoint{1.869967in}{2.577564in}}{\pgfqpoint{1.868331in}{2.573614in}}{\pgfqpoint{1.868331in}{2.569496in}}%
\pgfpathcurveto{\pgfqpoint{1.868331in}{2.565378in}}{\pgfqpoint{1.869967in}{2.561428in}}{\pgfqpoint{1.872879in}{2.558516in}}%
\pgfpathcurveto{\pgfqpoint{1.875791in}{2.555604in}}{\pgfqpoint{1.879741in}{2.553968in}}{\pgfqpoint{1.883859in}{2.553968in}}%
\pgfpathclose%
\pgfusepath{fill}%
\end{pgfscope}%
\begin{pgfscope}%
\pgfpathrectangle{\pgfqpoint{0.887500in}{0.275000in}}{\pgfqpoint{4.225000in}{4.225000in}}%
\pgfusepath{clip}%
\pgfsetbuttcap%
\pgfsetroundjoin%
\definecolor{currentfill}{rgb}{0.000000,0.000000,0.000000}%
\pgfsetfillcolor{currentfill}%
\pgfsetfillopacity{0.800000}%
\pgfsetlinewidth{0.000000pt}%
\definecolor{currentstroke}{rgb}{0.000000,0.000000,0.000000}%
\pgfsetstrokecolor{currentstroke}%
\pgfsetstrokeopacity{0.800000}%
\pgfsetdash{}{0pt}%
\pgfpathmoveto{\pgfqpoint{4.018930in}{2.932198in}}%
\pgfpathcurveto{\pgfqpoint{4.023048in}{2.932198in}}{\pgfqpoint{4.026998in}{2.933834in}}{\pgfqpoint{4.029910in}{2.936746in}}%
\pgfpathcurveto{\pgfqpoint{4.032822in}{2.939658in}}{\pgfqpoint{4.034458in}{2.943608in}}{\pgfqpoint{4.034458in}{2.947727in}}%
\pgfpathcurveto{\pgfqpoint{4.034458in}{2.951845in}}{\pgfqpoint{4.032822in}{2.955795in}}{\pgfqpoint{4.029910in}{2.958707in}}%
\pgfpathcurveto{\pgfqpoint{4.026998in}{2.961619in}}{\pgfqpoint{4.023048in}{2.963255in}}{\pgfqpoint{4.018930in}{2.963255in}}%
\pgfpathcurveto{\pgfqpoint{4.014812in}{2.963255in}}{\pgfqpoint{4.010862in}{2.961619in}}{\pgfqpoint{4.007950in}{2.958707in}}%
\pgfpathcurveto{\pgfqpoint{4.005038in}{2.955795in}}{\pgfqpoint{4.003402in}{2.951845in}}{\pgfqpoint{4.003402in}{2.947727in}}%
\pgfpathcurveto{\pgfqpoint{4.003402in}{2.943608in}}{\pgfqpoint{4.005038in}{2.939658in}}{\pgfqpoint{4.007950in}{2.936746in}}%
\pgfpathcurveto{\pgfqpoint{4.010862in}{2.933834in}}{\pgfqpoint{4.014812in}{2.932198in}}{\pgfqpoint{4.018930in}{2.932198in}}%
\pgfpathclose%
\pgfusepath{fill}%
\end{pgfscope}%
\begin{pgfscope}%
\pgfpathrectangle{\pgfqpoint{0.887500in}{0.275000in}}{\pgfqpoint{4.225000in}{4.225000in}}%
\pgfusepath{clip}%
\pgfsetbuttcap%
\pgfsetroundjoin%
\definecolor{currentfill}{rgb}{0.000000,0.000000,0.000000}%
\pgfsetfillcolor{currentfill}%
\pgfsetfillopacity{0.800000}%
\pgfsetlinewidth{0.000000pt}%
\definecolor{currentstroke}{rgb}{0.000000,0.000000,0.000000}%
\pgfsetstrokecolor{currentstroke}%
\pgfsetstrokeopacity{0.800000}%
\pgfsetdash{}{0pt}%
\pgfpathmoveto{\pgfqpoint{1.417769in}{2.620473in}}%
\pgfpathcurveto{\pgfqpoint{1.421887in}{2.620473in}}{\pgfqpoint{1.425837in}{2.622109in}}{\pgfqpoint{1.428749in}{2.625021in}}%
\pgfpathcurveto{\pgfqpoint{1.431661in}{2.627933in}}{\pgfqpoint{1.433297in}{2.631883in}}{\pgfqpoint{1.433297in}{2.636001in}}%
\pgfpathcurveto{\pgfqpoint{1.433297in}{2.640119in}}{\pgfqpoint{1.431661in}{2.644069in}}{\pgfqpoint{1.428749in}{2.646981in}}%
\pgfpathcurveto{\pgfqpoint{1.425837in}{2.649893in}}{\pgfqpoint{1.421887in}{2.651529in}}{\pgfqpoint{1.417769in}{2.651529in}}%
\pgfpathcurveto{\pgfqpoint{1.413651in}{2.651529in}}{\pgfqpoint{1.409701in}{2.649893in}}{\pgfqpoint{1.406789in}{2.646981in}}%
\pgfpathcurveto{\pgfqpoint{1.403877in}{2.644069in}}{\pgfqpoint{1.402241in}{2.640119in}}{\pgfqpoint{1.402241in}{2.636001in}}%
\pgfpathcurveto{\pgfqpoint{1.402241in}{2.631883in}}{\pgfqpoint{1.403877in}{2.627933in}}{\pgfqpoint{1.406789in}{2.625021in}}%
\pgfpathcurveto{\pgfqpoint{1.409701in}{2.622109in}}{\pgfqpoint{1.413651in}{2.620473in}}{\pgfqpoint{1.417769in}{2.620473in}}%
\pgfpathclose%
\pgfusepath{fill}%
\end{pgfscope}%
\begin{pgfscope}%
\pgfpathrectangle{\pgfqpoint{0.887500in}{0.275000in}}{\pgfqpoint{4.225000in}{4.225000in}}%
\pgfusepath{clip}%
\pgfsetbuttcap%
\pgfsetroundjoin%
\definecolor{currentfill}{rgb}{0.000000,0.000000,0.000000}%
\pgfsetfillcolor{currentfill}%
\pgfsetfillopacity{0.800000}%
\pgfsetlinewidth{0.000000pt}%
\definecolor{currentstroke}{rgb}{0.000000,0.000000,0.000000}%
\pgfsetstrokecolor{currentstroke}%
\pgfsetstrokeopacity{0.800000}%
\pgfsetdash{}{0pt}%
\pgfpathmoveto{\pgfqpoint{3.488338in}{3.220736in}}%
\pgfpathcurveto{\pgfqpoint{3.492456in}{3.220736in}}{\pgfqpoint{3.496406in}{3.222372in}}{\pgfqpoint{3.499318in}{3.225284in}}%
\pgfpathcurveto{\pgfqpoint{3.502230in}{3.228196in}}{\pgfqpoint{3.503866in}{3.232146in}}{\pgfqpoint{3.503866in}{3.236264in}}%
\pgfpathcurveto{\pgfqpoint{3.503866in}{3.240382in}}{\pgfqpoint{3.502230in}{3.244332in}}{\pgfqpoint{3.499318in}{3.247244in}}%
\pgfpathcurveto{\pgfqpoint{3.496406in}{3.250156in}}{\pgfqpoint{3.492456in}{3.251792in}}{\pgfqpoint{3.488338in}{3.251792in}}%
\pgfpathcurveto{\pgfqpoint{3.484220in}{3.251792in}}{\pgfqpoint{3.480270in}{3.250156in}}{\pgfqpoint{3.477358in}{3.247244in}}%
\pgfpathcurveto{\pgfqpoint{3.474446in}{3.244332in}}{\pgfqpoint{3.472810in}{3.240382in}}{\pgfqpoint{3.472810in}{3.236264in}}%
\pgfpathcurveto{\pgfqpoint{3.472810in}{3.232146in}}{\pgfqpoint{3.474446in}{3.228196in}}{\pgfqpoint{3.477358in}{3.225284in}}%
\pgfpathcurveto{\pgfqpoint{3.480270in}{3.222372in}}{\pgfqpoint{3.484220in}{3.220736in}}{\pgfqpoint{3.488338in}{3.220736in}}%
\pgfpathclose%
\pgfusepath{fill}%
\end{pgfscope}%
\begin{pgfscope}%
\pgfpathrectangle{\pgfqpoint{0.887500in}{0.275000in}}{\pgfqpoint{4.225000in}{4.225000in}}%
\pgfusepath{clip}%
\pgfsetbuttcap%
\pgfsetroundjoin%
\definecolor{currentfill}{rgb}{0.000000,0.000000,0.000000}%
\pgfsetfillcolor{currentfill}%
\pgfsetfillopacity{0.800000}%
\pgfsetlinewidth{0.000000pt}%
\definecolor{currentstroke}{rgb}{0.000000,0.000000,0.000000}%
\pgfsetstrokecolor{currentstroke}%
\pgfsetstrokeopacity{0.800000}%
\pgfsetdash{}{0pt}%
\pgfpathmoveto{\pgfqpoint{3.842263in}{3.037816in}}%
\pgfpathcurveto{\pgfqpoint{3.846381in}{3.037816in}}{\pgfqpoint{3.850331in}{3.039453in}}{\pgfqpoint{3.853243in}{3.042365in}}%
\pgfpathcurveto{\pgfqpoint{3.856155in}{3.045276in}}{\pgfqpoint{3.857791in}{3.049227in}}{\pgfqpoint{3.857791in}{3.053345in}}%
\pgfpathcurveto{\pgfqpoint{3.857791in}{3.057463in}}{\pgfqpoint{3.856155in}{3.061413in}}{\pgfqpoint{3.853243in}{3.064325in}}%
\pgfpathcurveto{\pgfqpoint{3.850331in}{3.067237in}}{\pgfqpoint{3.846381in}{3.068873in}}{\pgfqpoint{3.842263in}{3.068873in}}%
\pgfpathcurveto{\pgfqpoint{3.838145in}{3.068873in}}{\pgfqpoint{3.834195in}{3.067237in}}{\pgfqpoint{3.831283in}{3.064325in}}%
\pgfpathcurveto{\pgfqpoint{3.828371in}{3.061413in}}{\pgfqpoint{3.826735in}{3.057463in}}{\pgfqpoint{3.826735in}{3.053345in}}%
\pgfpathcurveto{\pgfqpoint{3.826735in}{3.049227in}}{\pgfqpoint{3.828371in}{3.045276in}}{\pgfqpoint{3.831283in}{3.042365in}}%
\pgfpathcurveto{\pgfqpoint{3.834195in}{3.039453in}}{\pgfqpoint{3.838145in}{3.037816in}}{\pgfqpoint{3.842263in}{3.037816in}}%
\pgfpathclose%
\pgfusepath{fill}%
\end{pgfscope}%
\begin{pgfscope}%
\pgfpathrectangle{\pgfqpoint{0.887500in}{0.275000in}}{\pgfqpoint{4.225000in}{4.225000in}}%
\pgfusepath{clip}%
\pgfsetbuttcap%
\pgfsetroundjoin%
\definecolor{currentfill}{rgb}{0.000000,0.000000,0.000000}%
\pgfsetfillcolor{currentfill}%
\pgfsetfillopacity{0.800000}%
\pgfsetlinewidth{0.000000pt}%
\definecolor{currentstroke}{rgb}{0.000000,0.000000,0.000000}%
\pgfsetstrokecolor{currentstroke}%
\pgfsetstrokeopacity{0.800000}%
\pgfsetdash{}{0pt}%
\pgfpathmoveto{\pgfqpoint{3.665339in}{3.132802in}}%
\pgfpathcurveto{\pgfqpoint{3.669457in}{3.132802in}}{\pgfqpoint{3.673407in}{3.134438in}}{\pgfqpoint{3.676319in}{3.137350in}}%
\pgfpathcurveto{\pgfqpoint{3.679231in}{3.140262in}}{\pgfqpoint{3.680867in}{3.144212in}}{\pgfqpoint{3.680867in}{3.148330in}}%
\pgfpathcurveto{\pgfqpoint{3.680867in}{3.152448in}}{\pgfqpoint{3.679231in}{3.156398in}}{\pgfqpoint{3.676319in}{3.159310in}}%
\pgfpathcurveto{\pgfqpoint{3.673407in}{3.162222in}}{\pgfqpoint{3.669457in}{3.163859in}}{\pgfqpoint{3.665339in}{3.163859in}}%
\pgfpathcurveto{\pgfqpoint{3.661221in}{3.163859in}}{\pgfqpoint{3.657271in}{3.162222in}}{\pgfqpoint{3.654359in}{3.159310in}}%
\pgfpathcurveto{\pgfqpoint{3.651447in}{3.156398in}}{\pgfqpoint{3.649811in}{3.152448in}}{\pgfqpoint{3.649811in}{3.148330in}}%
\pgfpathcurveto{\pgfqpoint{3.649811in}{3.144212in}}{\pgfqpoint{3.651447in}{3.140262in}}{\pgfqpoint{3.654359in}{3.137350in}}%
\pgfpathcurveto{\pgfqpoint{3.657271in}{3.134438in}}{\pgfqpoint{3.661221in}{3.132802in}}{\pgfqpoint{3.665339in}{3.132802in}}%
\pgfpathclose%
\pgfusepath{fill}%
\end{pgfscope}%
\begin{pgfscope}%
\pgfpathrectangle{\pgfqpoint{0.887500in}{0.275000in}}{\pgfqpoint{4.225000in}{4.225000in}}%
\pgfusepath{clip}%
\pgfsetbuttcap%
\pgfsetroundjoin%
\definecolor{currentfill}{rgb}{0.000000,0.000000,0.000000}%
\pgfsetfillcolor{currentfill}%
\pgfsetfillopacity{0.800000}%
\pgfsetlinewidth{0.000000pt}%
\definecolor{currentstroke}{rgb}{0.000000,0.000000,0.000000}%
\pgfsetstrokecolor{currentstroke}%
\pgfsetstrokeopacity{0.800000}%
\pgfsetdash{}{0pt}%
\pgfpathmoveto{\pgfqpoint{2.526163in}{2.426936in}}%
\pgfpathcurveto{\pgfqpoint{2.530281in}{2.426936in}}{\pgfqpoint{2.534232in}{2.428573in}}{\pgfqpoint{2.537143in}{2.431485in}}%
\pgfpathcurveto{\pgfqpoint{2.540055in}{2.434397in}}{\pgfqpoint{2.541692in}{2.438347in}}{\pgfqpoint{2.541692in}{2.442465in}}%
\pgfpathcurveto{\pgfqpoint{2.541692in}{2.446583in}}{\pgfqpoint{2.540055in}{2.450533in}}{\pgfqpoint{2.537143in}{2.453445in}}%
\pgfpathcurveto{\pgfqpoint{2.534232in}{2.456357in}}{\pgfqpoint{2.530281in}{2.457993in}}{\pgfqpoint{2.526163in}{2.457993in}}%
\pgfpathcurveto{\pgfqpoint{2.522045in}{2.457993in}}{\pgfqpoint{2.518095in}{2.456357in}}{\pgfqpoint{2.515183in}{2.453445in}}%
\pgfpathcurveto{\pgfqpoint{2.512271in}{2.450533in}}{\pgfqpoint{2.510635in}{2.446583in}}{\pgfqpoint{2.510635in}{2.442465in}}%
\pgfpathcurveto{\pgfqpoint{2.510635in}{2.438347in}}{\pgfqpoint{2.512271in}{2.434397in}}{\pgfqpoint{2.515183in}{2.431485in}}%
\pgfpathcurveto{\pgfqpoint{2.518095in}{2.428573in}}{\pgfqpoint{2.522045in}{2.426936in}}{\pgfqpoint{2.526163in}{2.426936in}}%
\pgfpathclose%
\pgfusepath{fill}%
\end{pgfscope}%
\begin{pgfscope}%
\pgfpathrectangle{\pgfqpoint{0.887500in}{0.275000in}}{\pgfqpoint{4.225000in}{4.225000in}}%
\pgfusepath{clip}%
\pgfsetbuttcap%
\pgfsetroundjoin%
\definecolor{currentfill}{rgb}{0.000000,0.000000,0.000000}%
\pgfsetfillcolor{currentfill}%
\pgfsetfillopacity{0.800000}%
\pgfsetlinewidth{0.000000pt}%
\definecolor{currentstroke}{rgb}{0.000000,0.000000,0.000000}%
\pgfsetstrokecolor{currentstroke}%
\pgfsetstrokeopacity{0.800000}%
\pgfsetdash{}{0pt}%
\pgfpathmoveto{\pgfqpoint{2.059640in}{2.501767in}}%
\pgfpathcurveto{\pgfqpoint{2.063758in}{2.501767in}}{\pgfqpoint{2.067708in}{2.503403in}}{\pgfqpoint{2.070620in}{2.506315in}}%
\pgfpathcurveto{\pgfqpoint{2.073532in}{2.509227in}}{\pgfqpoint{2.075168in}{2.513177in}}{\pgfqpoint{2.075168in}{2.517295in}}%
\pgfpathcurveto{\pgfqpoint{2.075168in}{2.521414in}}{\pgfqpoint{2.073532in}{2.525364in}}{\pgfqpoint{2.070620in}{2.528276in}}%
\pgfpathcurveto{\pgfqpoint{2.067708in}{2.531188in}}{\pgfqpoint{2.063758in}{2.532824in}}{\pgfqpoint{2.059640in}{2.532824in}}%
\pgfpathcurveto{\pgfqpoint{2.055522in}{2.532824in}}{\pgfqpoint{2.051572in}{2.531188in}}{\pgfqpoint{2.048660in}{2.528276in}}%
\pgfpathcurveto{\pgfqpoint{2.045748in}{2.525364in}}{\pgfqpoint{2.044112in}{2.521414in}}{\pgfqpoint{2.044112in}{2.517295in}}%
\pgfpathcurveto{\pgfqpoint{2.044112in}{2.513177in}}{\pgfqpoint{2.045748in}{2.509227in}}{\pgfqpoint{2.048660in}{2.506315in}}%
\pgfpathcurveto{\pgfqpoint{2.051572in}{2.503403in}}{\pgfqpoint{2.055522in}{2.501767in}}{\pgfqpoint{2.059640in}{2.501767in}}%
\pgfpathclose%
\pgfusepath{fill}%
\end{pgfscope}%
\begin{pgfscope}%
\pgfpathrectangle{\pgfqpoint{0.887500in}{0.275000in}}{\pgfqpoint{4.225000in}{4.225000in}}%
\pgfusepath{clip}%
\pgfsetbuttcap%
\pgfsetroundjoin%
\definecolor{currentfill}{rgb}{0.000000,0.000000,0.000000}%
\pgfsetfillcolor{currentfill}%
\pgfsetfillopacity{0.800000}%
\pgfsetlinewidth{0.000000pt}%
\definecolor{currentstroke}{rgb}{0.000000,0.000000,0.000000}%
\pgfsetstrokecolor{currentstroke}%
\pgfsetstrokeopacity{0.800000}%
\pgfsetdash{}{0pt}%
\pgfpathmoveto{\pgfqpoint{4.438867in}{2.559620in}}%
\pgfpathcurveto{\pgfqpoint{4.442985in}{2.559620in}}{\pgfqpoint{4.446935in}{2.561256in}}{\pgfqpoint{4.449847in}{2.564168in}}%
\pgfpathcurveto{\pgfqpoint{4.452759in}{2.567080in}}{\pgfqpoint{4.454395in}{2.571030in}}{\pgfqpoint{4.454395in}{2.575149in}}%
\pgfpathcurveto{\pgfqpoint{4.454395in}{2.579267in}}{\pgfqpoint{4.452759in}{2.583217in}}{\pgfqpoint{4.449847in}{2.586129in}}%
\pgfpathcurveto{\pgfqpoint{4.446935in}{2.589041in}}{\pgfqpoint{4.442985in}{2.590677in}}{\pgfqpoint{4.438867in}{2.590677in}}%
\pgfpathcurveto{\pgfqpoint{4.434748in}{2.590677in}}{\pgfqpoint{4.430798in}{2.589041in}}{\pgfqpoint{4.427886in}{2.586129in}}%
\pgfpathcurveto{\pgfqpoint{4.424974in}{2.583217in}}{\pgfqpoint{4.423338in}{2.579267in}}{\pgfqpoint{4.423338in}{2.575149in}}%
\pgfpathcurveto{\pgfqpoint{4.423338in}{2.571030in}}{\pgfqpoint{4.424974in}{2.567080in}}{\pgfqpoint{4.427886in}{2.564168in}}%
\pgfpathcurveto{\pgfqpoint{4.430798in}{2.561256in}}{\pgfqpoint{4.434748in}{2.559620in}}{\pgfqpoint{4.438867in}{2.559620in}}%
\pgfpathclose%
\pgfusepath{fill}%
\end{pgfscope}%
\begin{pgfscope}%
\pgfpathrectangle{\pgfqpoint{0.887500in}{0.275000in}}{\pgfqpoint{4.225000in}{4.225000in}}%
\pgfusepath{clip}%
\pgfsetbuttcap%
\pgfsetroundjoin%
\definecolor{currentfill}{rgb}{0.000000,0.000000,0.000000}%
\pgfsetfillcolor{currentfill}%
\pgfsetfillopacity{0.800000}%
\pgfsetlinewidth{0.000000pt}%
\definecolor{currentstroke}{rgb}{0.000000,0.000000,0.000000}%
\pgfsetstrokecolor{currentstroke}%
\pgfsetstrokeopacity{0.800000}%
\pgfsetdash{}{0pt}%
\pgfpathmoveto{\pgfqpoint{1.592922in}{2.572537in}}%
\pgfpathcurveto{\pgfqpoint{1.597040in}{2.572537in}}{\pgfqpoint{1.600990in}{2.574173in}}{\pgfqpoint{1.603902in}{2.577085in}}%
\pgfpathcurveto{\pgfqpoint{1.606814in}{2.579997in}}{\pgfqpoint{1.608451in}{2.583947in}}{\pgfqpoint{1.608451in}{2.588065in}}%
\pgfpathcurveto{\pgfqpoint{1.608451in}{2.592183in}}{\pgfqpoint{1.606814in}{2.596133in}}{\pgfqpoint{1.603902in}{2.599045in}}%
\pgfpathcurveto{\pgfqpoint{1.600990in}{2.601957in}}{\pgfqpoint{1.597040in}{2.603593in}}{\pgfqpoint{1.592922in}{2.603593in}}%
\pgfpathcurveto{\pgfqpoint{1.588804in}{2.603593in}}{\pgfqpoint{1.584854in}{2.601957in}}{\pgfqpoint{1.581942in}{2.599045in}}%
\pgfpathcurveto{\pgfqpoint{1.579030in}{2.596133in}}{\pgfqpoint{1.577394in}{2.592183in}}{\pgfqpoint{1.577394in}{2.588065in}}%
\pgfpathcurveto{\pgfqpoint{1.577394in}{2.583947in}}{\pgfqpoint{1.579030in}{2.579997in}}{\pgfqpoint{1.581942in}{2.577085in}}%
\pgfpathcurveto{\pgfqpoint{1.584854in}{2.574173in}}{\pgfqpoint{1.588804in}{2.572537in}}{\pgfqpoint{1.592922in}{2.572537in}}%
\pgfpathclose%
\pgfusepath{fill}%
\end{pgfscope}%
\begin{pgfscope}%
\pgfpathrectangle{\pgfqpoint{0.887500in}{0.275000in}}{\pgfqpoint{4.225000in}{4.225000in}}%
\pgfusepath{clip}%
\pgfsetbuttcap%
\pgfsetroundjoin%
\definecolor{currentfill}{rgb}{0.000000,0.000000,0.000000}%
\pgfsetfillcolor{currentfill}%
\pgfsetfillopacity{0.800000}%
\pgfsetlinewidth{0.000000pt}%
\definecolor{currentstroke}{rgb}{0.000000,0.000000,0.000000}%
\pgfsetstrokecolor{currentstroke}%
\pgfsetstrokeopacity{0.800000}%
\pgfsetdash{}{0pt}%
\pgfpathmoveto{\pgfqpoint{2.702795in}{2.370551in}}%
\pgfpathcurveto{\pgfqpoint{2.706914in}{2.370551in}}{\pgfqpoint{2.710864in}{2.372187in}}{\pgfqpoint{2.713776in}{2.375099in}}%
\pgfpathcurveto{\pgfqpoint{2.716688in}{2.378011in}}{\pgfqpoint{2.718324in}{2.381961in}}{\pgfqpoint{2.718324in}{2.386079in}}%
\pgfpathcurveto{\pgfqpoint{2.718324in}{2.390197in}}{\pgfqpoint{2.716688in}{2.394147in}}{\pgfqpoint{2.713776in}{2.397059in}}%
\pgfpathcurveto{\pgfqpoint{2.710864in}{2.399971in}}{\pgfqpoint{2.706914in}{2.401607in}}{\pgfqpoint{2.702795in}{2.401607in}}%
\pgfpathcurveto{\pgfqpoint{2.698677in}{2.401607in}}{\pgfqpoint{2.694727in}{2.399971in}}{\pgfqpoint{2.691815in}{2.397059in}}%
\pgfpathcurveto{\pgfqpoint{2.688903in}{2.394147in}}{\pgfqpoint{2.687267in}{2.390197in}}{\pgfqpoint{2.687267in}{2.386079in}}%
\pgfpathcurveto{\pgfqpoint{2.687267in}{2.381961in}}{\pgfqpoint{2.688903in}{2.378011in}}{\pgfqpoint{2.691815in}{2.375099in}}%
\pgfpathcurveto{\pgfqpoint{2.694727in}{2.372187in}}{\pgfqpoint{2.698677in}{2.370551in}}{\pgfqpoint{2.702795in}{2.370551in}}%
\pgfpathclose%
\pgfusepath{fill}%
\end{pgfscope}%
\begin{pgfscope}%
\pgfpathrectangle{\pgfqpoint{0.887500in}{0.275000in}}{\pgfqpoint{4.225000in}{4.225000in}}%
\pgfusepath{clip}%
\pgfsetbuttcap%
\pgfsetroundjoin%
\definecolor{currentfill}{rgb}{0.000000,0.000000,0.000000}%
\pgfsetfillcolor{currentfill}%
\pgfsetfillopacity{0.800000}%
\pgfsetlinewidth{0.000000pt}%
\definecolor{currentstroke}{rgb}{0.000000,0.000000,0.000000}%
\pgfsetstrokecolor{currentstroke}%
\pgfsetstrokeopacity{0.800000}%
\pgfsetdash{}{0pt}%
\pgfpathmoveto{\pgfqpoint{4.262431in}{2.675222in}}%
\pgfpathcurveto{\pgfqpoint{4.266549in}{2.675222in}}{\pgfqpoint{4.270499in}{2.676858in}}{\pgfqpoint{4.273411in}{2.679770in}}%
\pgfpathcurveto{\pgfqpoint{4.276323in}{2.682682in}}{\pgfqpoint{4.277959in}{2.686632in}}{\pgfqpoint{4.277959in}{2.690750in}}%
\pgfpathcurveto{\pgfqpoint{4.277959in}{2.694868in}}{\pgfqpoint{4.276323in}{2.698818in}}{\pgfqpoint{4.273411in}{2.701730in}}%
\pgfpathcurveto{\pgfqpoint{4.270499in}{2.704642in}}{\pgfqpoint{4.266549in}{2.706279in}}{\pgfqpoint{4.262431in}{2.706279in}}%
\pgfpathcurveto{\pgfqpoint{4.258313in}{2.706279in}}{\pgfqpoint{4.254363in}{2.704642in}}{\pgfqpoint{4.251451in}{2.701730in}}%
\pgfpathcurveto{\pgfqpoint{4.248539in}{2.698818in}}{\pgfqpoint{4.246903in}{2.694868in}}{\pgfqpoint{4.246903in}{2.690750in}}%
\pgfpathcurveto{\pgfqpoint{4.246903in}{2.686632in}}{\pgfqpoint{4.248539in}{2.682682in}}{\pgfqpoint{4.251451in}{2.679770in}}%
\pgfpathcurveto{\pgfqpoint{4.254363in}{2.676858in}}{\pgfqpoint{4.258313in}{2.675222in}}{\pgfqpoint{4.262431in}{2.675222in}}%
\pgfpathclose%
\pgfusepath{fill}%
\end{pgfscope}%
\begin{pgfscope}%
\pgfpathrectangle{\pgfqpoint{0.887500in}{0.275000in}}{\pgfqpoint{4.225000in}{4.225000in}}%
\pgfusepath{clip}%
\pgfsetbuttcap%
\pgfsetroundjoin%
\definecolor{currentfill}{rgb}{0.000000,0.000000,0.000000}%
\pgfsetfillcolor{currentfill}%
\pgfsetfillopacity{0.800000}%
\pgfsetlinewidth{0.000000pt}%
\definecolor{currentstroke}{rgb}{0.000000,0.000000,0.000000}%
\pgfsetstrokecolor{currentstroke}%
\pgfsetstrokeopacity{0.800000}%
\pgfsetdash{}{0pt}%
\pgfpathmoveto{\pgfqpoint{3.022341in}{3.345010in}}%
\pgfpathcurveto{\pgfqpoint{3.026459in}{3.345010in}}{\pgfqpoint{3.030409in}{3.346646in}}{\pgfqpoint{3.033321in}{3.349558in}}%
\pgfpathcurveto{\pgfqpoint{3.036233in}{3.352470in}}{\pgfqpoint{3.037869in}{3.356420in}}{\pgfqpoint{3.037869in}{3.360538in}}%
\pgfpathcurveto{\pgfqpoint{3.037869in}{3.364656in}}{\pgfqpoint{3.036233in}{3.368606in}}{\pgfqpoint{3.033321in}{3.371518in}}%
\pgfpathcurveto{\pgfqpoint{3.030409in}{3.374430in}}{\pgfqpoint{3.026459in}{3.376066in}}{\pgfqpoint{3.022341in}{3.376066in}}%
\pgfpathcurveto{\pgfqpoint{3.018223in}{3.376066in}}{\pgfqpoint{3.014273in}{3.374430in}}{\pgfqpoint{3.011361in}{3.371518in}}%
\pgfpathcurveto{\pgfqpoint{3.008449in}{3.368606in}}{\pgfqpoint{3.006812in}{3.364656in}}{\pgfqpoint{3.006812in}{3.360538in}}%
\pgfpathcurveto{\pgfqpoint{3.006812in}{3.356420in}}{\pgfqpoint{3.008449in}{3.352470in}}{\pgfqpoint{3.011361in}{3.349558in}}%
\pgfpathcurveto{\pgfqpoint{3.014273in}{3.346646in}}{\pgfqpoint{3.018223in}{3.345010in}}{\pgfqpoint{3.022341in}{3.345010in}}%
\pgfpathclose%
\pgfusepath{fill}%
\end{pgfscope}%
\begin{pgfscope}%
\pgfpathrectangle{\pgfqpoint{0.887500in}{0.275000in}}{\pgfqpoint{4.225000in}{4.225000in}}%
\pgfusepath{clip}%
\pgfsetbuttcap%
\pgfsetroundjoin%
\definecolor{currentfill}{rgb}{0.000000,0.000000,0.000000}%
\pgfsetfillcolor{currentfill}%
\pgfsetfillopacity{0.800000}%
\pgfsetlinewidth{0.000000pt}%
\definecolor{currentstroke}{rgb}{0.000000,0.000000,0.000000}%
\pgfsetstrokecolor{currentstroke}%
\pgfsetstrokeopacity{0.800000}%
\pgfsetdash{}{0pt}%
\pgfpathmoveto{\pgfqpoint{2.235828in}{2.448085in}}%
\pgfpathcurveto{\pgfqpoint{2.239946in}{2.448085in}}{\pgfqpoint{2.243896in}{2.449721in}}{\pgfqpoint{2.246808in}{2.452633in}}%
\pgfpathcurveto{\pgfqpoint{2.249720in}{2.455545in}}{\pgfqpoint{2.251356in}{2.459495in}}{\pgfqpoint{2.251356in}{2.463614in}}%
\pgfpathcurveto{\pgfqpoint{2.251356in}{2.467732in}}{\pgfqpoint{2.249720in}{2.471682in}}{\pgfqpoint{2.246808in}{2.474594in}}%
\pgfpathcurveto{\pgfqpoint{2.243896in}{2.477506in}}{\pgfqpoint{2.239946in}{2.479142in}}{\pgfqpoint{2.235828in}{2.479142in}}%
\pgfpathcurveto{\pgfqpoint{2.231710in}{2.479142in}}{\pgfqpoint{2.227760in}{2.477506in}}{\pgfqpoint{2.224848in}{2.474594in}}%
\pgfpathcurveto{\pgfqpoint{2.221936in}{2.471682in}}{\pgfqpoint{2.220300in}{2.467732in}}{\pgfqpoint{2.220300in}{2.463614in}}%
\pgfpathcurveto{\pgfqpoint{2.220300in}{2.459495in}}{\pgfqpoint{2.221936in}{2.455545in}}{\pgfqpoint{2.224848in}{2.452633in}}%
\pgfpathcurveto{\pgfqpoint{2.227760in}{2.449721in}}{\pgfqpoint{2.231710in}{2.448085in}}{\pgfqpoint{2.235828in}{2.448085in}}%
\pgfpathclose%
\pgfusepath{fill}%
\end{pgfscope}%
\begin{pgfscope}%
\pgfpathrectangle{\pgfqpoint{0.887500in}{0.275000in}}{\pgfqpoint{4.225000in}{4.225000in}}%
\pgfusepath{clip}%
\pgfsetbuttcap%
\pgfsetroundjoin%
\definecolor{currentfill}{rgb}{0.000000,0.000000,0.000000}%
\pgfsetfillcolor{currentfill}%
\pgfsetfillopacity{0.800000}%
\pgfsetlinewidth{0.000000pt}%
\definecolor{currentstroke}{rgb}{0.000000,0.000000,0.000000}%
\pgfsetstrokecolor{currentstroke}%
\pgfsetstrokeopacity{0.800000}%
\pgfsetdash{}{0pt}%
\pgfpathmoveto{\pgfqpoint{4.085725in}{2.787579in}}%
\pgfpathcurveto{\pgfqpoint{4.089843in}{2.787579in}}{\pgfqpoint{4.093793in}{2.789215in}}{\pgfqpoint{4.096705in}{2.792127in}}%
\pgfpathcurveto{\pgfqpoint{4.099617in}{2.795039in}}{\pgfqpoint{4.101253in}{2.798989in}}{\pgfqpoint{4.101253in}{2.803107in}}%
\pgfpathcurveto{\pgfqpoint{4.101253in}{2.807225in}}{\pgfqpoint{4.099617in}{2.811175in}}{\pgfqpoint{4.096705in}{2.814087in}}%
\pgfpathcurveto{\pgfqpoint{4.093793in}{2.816999in}}{\pgfqpoint{4.089843in}{2.818635in}}{\pgfqpoint{4.085725in}{2.818635in}}%
\pgfpathcurveto{\pgfqpoint{4.081607in}{2.818635in}}{\pgfqpoint{4.077657in}{2.816999in}}{\pgfqpoint{4.074745in}{2.814087in}}%
\pgfpathcurveto{\pgfqpoint{4.071833in}{2.811175in}}{\pgfqpoint{4.070197in}{2.807225in}}{\pgfqpoint{4.070197in}{2.803107in}}%
\pgfpathcurveto{\pgfqpoint{4.070197in}{2.798989in}}{\pgfqpoint{4.071833in}{2.795039in}}{\pgfqpoint{4.074745in}{2.792127in}}%
\pgfpathcurveto{\pgfqpoint{4.077657in}{2.789215in}}{\pgfqpoint{4.081607in}{2.787579in}}{\pgfqpoint{4.085725in}{2.787579in}}%
\pgfpathclose%
\pgfusepath{fill}%
\end{pgfscope}%
\begin{pgfscope}%
\pgfpathrectangle{\pgfqpoint{0.887500in}{0.275000in}}{\pgfqpoint{4.225000in}{4.225000in}}%
\pgfusepath{clip}%
\pgfsetbuttcap%
\pgfsetroundjoin%
\definecolor{currentfill}{rgb}{0.000000,0.000000,0.000000}%
\pgfsetfillcolor{currentfill}%
\pgfsetfillopacity{0.800000}%
\pgfsetlinewidth{0.000000pt}%
\definecolor{currentstroke}{rgb}{0.000000,0.000000,0.000000}%
\pgfsetstrokecolor{currentstroke}%
\pgfsetstrokeopacity{0.800000}%
\pgfsetdash{}{0pt}%
\pgfpathmoveto{\pgfqpoint{2.974820in}{3.013288in}}%
\pgfpathcurveto{\pgfqpoint{2.978938in}{3.013288in}}{\pgfqpoint{2.982888in}{3.014924in}}{\pgfqpoint{2.985800in}{3.017836in}}%
\pgfpathcurveto{\pgfqpoint{2.988712in}{3.020748in}}{\pgfqpoint{2.990348in}{3.024698in}}{\pgfqpoint{2.990348in}{3.028816in}}%
\pgfpathcurveto{\pgfqpoint{2.990348in}{3.032934in}}{\pgfqpoint{2.988712in}{3.036884in}}{\pgfqpoint{2.985800in}{3.039796in}}%
\pgfpathcurveto{\pgfqpoint{2.982888in}{3.042708in}}{\pgfqpoint{2.978938in}{3.044344in}}{\pgfqpoint{2.974820in}{3.044344in}}%
\pgfpathcurveto{\pgfqpoint{2.970701in}{3.044344in}}{\pgfqpoint{2.966751in}{3.042708in}}{\pgfqpoint{2.963839in}{3.039796in}}%
\pgfpathcurveto{\pgfqpoint{2.960928in}{3.036884in}}{\pgfqpoint{2.959291in}{3.032934in}}{\pgfqpoint{2.959291in}{3.028816in}}%
\pgfpathcurveto{\pgfqpoint{2.959291in}{3.024698in}}{\pgfqpoint{2.960928in}{3.020748in}}{\pgfqpoint{2.963839in}{3.017836in}}%
\pgfpathcurveto{\pgfqpoint{2.966751in}{3.014924in}}{\pgfqpoint{2.970701in}{3.013288in}}{\pgfqpoint{2.974820in}{3.013288in}}%
\pgfpathclose%
\pgfusepath{fill}%
\end{pgfscope}%
\begin{pgfscope}%
\pgfpathrectangle{\pgfqpoint{0.887500in}{0.275000in}}{\pgfqpoint{4.225000in}{4.225000in}}%
\pgfusepath{clip}%
\pgfsetbuttcap%
\pgfsetroundjoin%
\definecolor{currentfill}{rgb}{0.000000,0.000000,0.000000}%
\pgfsetfillcolor{currentfill}%
\pgfsetfillopacity{0.800000}%
\pgfsetlinewidth{0.000000pt}%
\definecolor{currentstroke}{rgb}{0.000000,0.000000,0.000000}%
\pgfsetstrokecolor{currentstroke}%
\pgfsetstrokeopacity{0.800000}%
\pgfsetdash{}{0pt}%
\pgfpathmoveto{\pgfqpoint{1.768590in}{2.521401in}}%
\pgfpathcurveto{\pgfqpoint{1.772708in}{2.521401in}}{\pgfqpoint{1.776658in}{2.523037in}}{\pgfqpoint{1.779570in}{2.525949in}}%
\pgfpathcurveto{\pgfqpoint{1.782482in}{2.528861in}}{\pgfqpoint{1.784118in}{2.532811in}}{\pgfqpoint{1.784118in}{2.536929in}}%
\pgfpathcurveto{\pgfqpoint{1.784118in}{2.541047in}}{\pgfqpoint{1.782482in}{2.544997in}}{\pgfqpoint{1.779570in}{2.547909in}}%
\pgfpathcurveto{\pgfqpoint{1.776658in}{2.550821in}}{\pgfqpoint{1.772708in}{2.552457in}}{\pgfqpoint{1.768590in}{2.552457in}}%
\pgfpathcurveto{\pgfqpoint{1.764472in}{2.552457in}}{\pgfqpoint{1.760522in}{2.550821in}}{\pgfqpoint{1.757610in}{2.547909in}}%
\pgfpathcurveto{\pgfqpoint{1.754698in}{2.544997in}}{\pgfqpoint{1.753062in}{2.541047in}}{\pgfqpoint{1.753062in}{2.536929in}}%
\pgfpathcurveto{\pgfqpoint{1.753062in}{2.532811in}}{\pgfqpoint{1.754698in}{2.528861in}}{\pgfqpoint{1.757610in}{2.525949in}}%
\pgfpathcurveto{\pgfqpoint{1.760522in}{2.523037in}}{\pgfqpoint{1.764472in}{2.521401in}}{\pgfqpoint{1.768590in}{2.521401in}}%
\pgfpathclose%
\pgfusepath{fill}%
\end{pgfscope}%
\begin{pgfscope}%
\pgfpathrectangle{\pgfqpoint{0.887500in}{0.275000in}}{\pgfqpoint{4.225000in}{4.225000in}}%
\pgfusepath{clip}%
\pgfsetbuttcap%
\pgfsetroundjoin%
\definecolor{currentfill}{rgb}{0.000000,0.000000,0.000000}%
\pgfsetfillcolor{currentfill}%
\pgfsetfillopacity{0.800000}%
\pgfsetlinewidth{0.000000pt}%
\definecolor{currentstroke}{rgb}{0.000000,0.000000,0.000000}%
\pgfsetstrokecolor{currentstroke}%
\pgfsetstrokeopacity{0.800000}%
\pgfsetdash{}{0pt}%
\pgfpathmoveto{\pgfqpoint{3.908781in}{2.896138in}}%
\pgfpathcurveto{\pgfqpoint{3.912899in}{2.896138in}}{\pgfqpoint{3.916849in}{2.897775in}}{\pgfqpoint{3.919761in}{2.900687in}}%
\pgfpathcurveto{\pgfqpoint{3.922673in}{2.903599in}}{\pgfqpoint{3.924310in}{2.907549in}}{\pgfqpoint{3.924310in}{2.911667in}}%
\pgfpathcurveto{\pgfqpoint{3.924310in}{2.915785in}}{\pgfqpoint{3.922673in}{2.919735in}}{\pgfqpoint{3.919761in}{2.922647in}}%
\pgfpathcurveto{\pgfqpoint{3.916849in}{2.925559in}}{\pgfqpoint{3.912899in}{2.927195in}}{\pgfqpoint{3.908781in}{2.927195in}}%
\pgfpathcurveto{\pgfqpoint{3.904663in}{2.927195in}}{\pgfqpoint{3.900713in}{2.925559in}}{\pgfqpoint{3.897801in}{2.922647in}}%
\pgfpathcurveto{\pgfqpoint{3.894889in}{2.919735in}}{\pgfqpoint{3.893253in}{2.915785in}}{\pgfqpoint{3.893253in}{2.911667in}}%
\pgfpathcurveto{\pgfqpoint{3.893253in}{2.907549in}}{\pgfqpoint{3.894889in}{2.903599in}}{\pgfqpoint{3.897801in}{2.900687in}}%
\pgfpathcurveto{\pgfqpoint{3.900713in}{2.897775in}}{\pgfqpoint{3.904663in}{2.896138in}}{\pgfqpoint{3.908781in}{2.896138in}}%
\pgfpathclose%
\pgfusepath{fill}%
\end{pgfscope}%
\begin{pgfscope}%
\pgfpathrectangle{\pgfqpoint{0.887500in}{0.275000in}}{\pgfqpoint{4.225000in}{4.225000in}}%
\pgfusepath{clip}%
\pgfsetbuttcap%
\pgfsetroundjoin%
\definecolor{currentfill}{rgb}{0.000000,0.000000,0.000000}%
\pgfsetfillcolor{currentfill}%
\pgfsetfillopacity{0.800000}%
\pgfsetlinewidth{0.000000pt}%
\definecolor{currentstroke}{rgb}{0.000000,0.000000,0.000000}%
\pgfsetstrokecolor{currentstroke}%
\pgfsetstrokeopacity{0.800000}%
\pgfsetdash{}{0pt}%
\pgfpathmoveto{\pgfqpoint{2.879783in}{2.311666in}}%
\pgfpathcurveto{\pgfqpoint{2.883901in}{2.311666in}}{\pgfqpoint{2.887851in}{2.313303in}}{\pgfqpoint{2.890763in}{2.316215in}}%
\pgfpathcurveto{\pgfqpoint{2.893675in}{2.319127in}}{\pgfqpoint{2.895311in}{2.323077in}}{\pgfqpoint{2.895311in}{2.327195in}}%
\pgfpathcurveto{\pgfqpoint{2.895311in}{2.331313in}}{\pgfqpoint{2.893675in}{2.335263in}}{\pgfqpoint{2.890763in}{2.338175in}}%
\pgfpathcurveto{\pgfqpoint{2.887851in}{2.341087in}}{\pgfqpoint{2.883901in}{2.342723in}}{\pgfqpoint{2.879783in}{2.342723in}}%
\pgfpathcurveto{\pgfqpoint{2.875665in}{2.342723in}}{\pgfqpoint{2.871715in}{2.341087in}}{\pgfqpoint{2.868803in}{2.338175in}}%
\pgfpathcurveto{\pgfqpoint{2.865891in}{2.335263in}}{\pgfqpoint{2.864255in}{2.331313in}}{\pgfqpoint{2.864255in}{2.327195in}}%
\pgfpathcurveto{\pgfqpoint{2.864255in}{2.323077in}}{\pgfqpoint{2.865891in}{2.319127in}}{\pgfqpoint{2.868803in}{2.316215in}}%
\pgfpathcurveto{\pgfqpoint{2.871715in}{2.313303in}}{\pgfqpoint{2.875665in}{2.311666in}}{\pgfqpoint{2.879783in}{2.311666in}}%
\pgfpathclose%
\pgfusepath{fill}%
\end{pgfscope}%
\begin{pgfscope}%
\pgfpathrectangle{\pgfqpoint{0.887500in}{0.275000in}}{\pgfqpoint{4.225000in}{4.225000in}}%
\pgfusepath{clip}%
\pgfsetbuttcap%
\pgfsetroundjoin%
\definecolor{currentfill}{rgb}{0.000000,0.000000,0.000000}%
\pgfsetfillcolor{currentfill}%
\pgfsetfillopacity{0.800000}%
\pgfsetlinewidth{0.000000pt}%
\definecolor{currentstroke}{rgb}{0.000000,0.000000,0.000000}%
\pgfsetstrokecolor{currentstroke}%
\pgfsetstrokeopacity{0.800000}%
\pgfsetdash{}{0pt}%
\pgfpathmoveto{\pgfqpoint{3.731672in}{3.002081in}}%
\pgfpathcurveto{\pgfqpoint{3.735791in}{3.002081in}}{\pgfqpoint{3.739741in}{3.003717in}}{\pgfqpoint{3.742653in}{3.006629in}}%
\pgfpathcurveto{\pgfqpoint{3.745565in}{3.009541in}}{\pgfqpoint{3.747201in}{3.013491in}}{\pgfqpoint{3.747201in}{3.017609in}}%
\pgfpathcurveto{\pgfqpoint{3.747201in}{3.021727in}}{\pgfqpoint{3.745565in}{3.025677in}}{\pgfqpoint{3.742653in}{3.028589in}}%
\pgfpathcurveto{\pgfqpoint{3.739741in}{3.031501in}}{\pgfqpoint{3.735791in}{3.033137in}}{\pgfqpoint{3.731672in}{3.033137in}}%
\pgfpathcurveto{\pgfqpoint{3.727554in}{3.033137in}}{\pgfqpoint{3.723604in}{3.031501in}}{\pgfqpoint{3.720692in}{3.028589in}}%
\pgfpathcurveto{\pgfqpoint{3.717780in}{3.025677in}}{\pgfqpoint{3.716144in}{3.021727in}}{\pgfqpoint{3.716144in}{3.017609in}}%
\pgfpathcurveto{\pgfqpoint{3.716144in}{3.013491in}}{\pgfqpoint{3.717780in}{3.009541in}}{\pgfqpoint{3.720692in}{3.006629in}}%
\pgfpathcurveto{\pgfqpoint{3.723604in}{3.003717in}}{\pgfqpoint{3.727554in}{3.002081in}}{\pgfqpoint{3.731672in}{3.002081in}}%
\pgfpathclose%
\pgfusepath{fill}%
\end{pgfscope}%
\begin{pgfscope}%
\pgfpathrectangle{\pgfqpoint{0.887500in}{0.275000in}}{\pgfqpoint{4.225000in}{4.225000in}}%
\pgfusepath{clip}%
\pgfsetbuttcap%
\pgfsetroundjoin%
\definecolor{currentfill}{rgb}{0.000000,0.000000,0.000000}%
\pgfsetfillcolor{currentfill}%
\pgfsetfillopacity{0.800000}%
\pgfsetlinewidth{0.000000pt}%
\definecolor{currentstroke}{rgb}{0.000000,0.000000,0.000000}%
\pgfsetstrokecolor{currentstroke}%
\pgfsetstrokeopacity{0.800000}%
\pgfsetdash{}{0pt}%
\pgfpathmoveto{\pgfqpoint{3.377023in}{3.196248in}}%
\pgfpathcurveto{\pgfqpoint{3.381141in}{3.196248in}}{\pgfqpoint{3.385091in}{3.197884in}}{\pgfqpoint{3.388003in}{3.200796in}}%
\pgfpathcurveto{\pgfqpoint{3.390915in}{3.203708in}}{\pgfqpoint{3.392551in}{3.207658in}}{\pgfqpoint{3.392551in}{3.211776in}}%
\pgfpathcurveto{\pgfqpoint{3.392551in}{3.215895in}}{\pgfqpoint{3.390915in}{3.219845in}}{\pgfqpoint{3.388003in}{3.222757in}}%
\pgfpathcurveto{\pgfqpoint{3.385091in}{3.225668in}}{\pgfqpoint{3.381141in}{3.227305in}}{\pgfqpoint{3.377023in}{3.227305in}}%
\pgfpathcurveto{\pgfqpoint{3.372905in}{3.227305in}}{\pgfqpoint{3.368955in}{3.225668in}}{\pgfqpoint{3.366043in}{3.222757in}}%
\pgfpathcurveto{\pgfqpoint{3.363131in}{3.219845in}}{\pgfqpoint{3.361495in}{3.215895in}}{\pgfqpoint{3.361495in}{3.211776in}}%
\pgfpathcurveto{\pgfqpoint{3.361495in}{3.207658in}}{\pgfqpoint{3.363131in}{3.203708in}}{\pgfqpoint{3.366043in}{3.200796in}}%
\pgfpathcurveto{\pgfqpoint{3.368955in}{3.197884in}}{\pgfqpoint{3.372905in}{3.196248in}}{\pgfqpoint{3.377023in}{3.196248in}}%
\pgfpathclose%
\pgfusepath{fill}%
\end{pgfscope}%
\begin{pgfscope}%
\pgfpathrectangle{\pgfqpoint{0.887500in}{0.275000in}}{\pgfqpoint{4.225000in}{4.225000in}}%
\pgfusepath{clip}%
\pgfsetbuttcap%
\pgfsetroundjoin%
\definecolor{currentfill}{rgb}{0.000000,0.000000,0.000000}%
\pgfsetfillcolor{currentfill}%
\pgfsetfillopacity{0.800000}%
\pgfsetlinewidth{0.000000pt}%
\definecolor{currentstroke}{rgb}{0.000000,0.000000,0.000000}%
\pgfsetstrokecolor{currentstroke}%
\pgfsetstrokeopacity{0.800000}%
\pgfsetdash{}{0pt}%
\pgfpathmoveto{\pgfqpoint{3.554398in}{3.102510in}}%
\pgfpathcurveto{\pgfqpoint{3.558516in}{3.102510in}}{\pgfqpoint{3.562466in}{3.104146in}}{\pgfqpoint{3.565378in}{3.107058in}}%
\pgfpathcurveto{\pgfqpoint{3.568290in}{3.109970in}}{\pgfqpoint{3.569926in}{3.113920in}}{\pgfqpoint{3.569926in}{3.118038in}}%
\pgfpathcurveto{\pgfqpoint{3.569926in}{3.122156in}}{\pgfqpoint{3.568290in}{3.126106in}}{\pgfqpoint{3.565378in}{3.129018in}}%
\pgfpathcurveto{\pgfqpoint{3.562466in}{3.131930in}}{\pgfqpoint{3.558516in}{3.133566in}}{\pgfqpoint{3.554398in}{3.133566in}}%
\pgfpathcurveto{\pgfqpoint{3.550280in}{3.133566in}}{\pgfqpoint{3.546330in}{3.131930in}}{\pgfqpoint{3.543418in}{3.129018in}}%
\pgfpathcurveto{\pgfqpoint{3.540506in}{3.126106in}}{\pgfqpoint{3.538870in}{3.122156in}}{\pgfqpoint{3.538870in}{3.118038in}}%
\pgfpathcurveto{\pgfqpoint{3.538870in}{3.113920in}}{\pgfqpoint{3.540506in}{3.109970in}}{\pgfqpoint{3.543418in}{3.107058in}}%
\pgfpathcurveto{\pgfqpoint{3.546330in}{3.104146in}}{\pgfqpoint{3.550280in}{3.102510in}}{\pgfqpoint{3.554398in}{3.102510in}}%
\pgfpathclose%
\pgfusepath{fill}%
\end{pgfscope}%
\begin{pgfscope}%
\pgfpathrectangle{\pgfqpoint{0.887500in}{0.275000in}}{\pgfqpoint{4.225000in}{4.225000in}}%
\pgfusepath{clip}%
\pgfsetbuttcap%
\pgfsetroundjoin%
\definecolor{currentfill}{rgb}{0.000000,0.000000,0.000000}%
\pgfsetfillcolor{currentfill}%
\pgfsetfillopacity{0.800000}%
\pgfsetlinewidth{0.000000pt}%
\definecolor{currentstroke}{rgb}{0.000000,0.000000,0.000000}%
\pgfsetstrokecolor{currentstroke}%
\pgfsetstrokeopacity{0.800000}%
\pgfsetdash{}{0pt}%
\pgfpathmoveto{\pgfqpoint{4.506481in}{2.411623in}}%
\pgfpathcurveto{\pgfqpoint{4.510599in}{2.411623in}}{\pgfqpoint{4.514549in}{2.413259in}}{\pgfqpoint{4.517461in}{2.416171in}}%
\pgfpathcurveto{\pgfqpoint{4.520373in}{2.419083in}}{\pgfqpoint{4.522009in}{2.423033in}}{\pgfqpoint{4.522009in}{2.427151in}}%
\pgfpathcurveto{\pgfqpoint{4.522009in}{2.431269in}}{\pgfqpoint{4.520373in}{2.435219in}}{\pgfqpoint{4.517461in}{2.438131in}}%
\pgfpathcurveto{\pgfqpoint{4.514549in}{2.441043in}}{\pgfqpoint{4.510599in}{2.442679in}}{\pgfqpoint{4.506481in}{2.442679in}}%
\pgfpathcurveto{\pgfqpoint{4.502363in}{2.442679in}}{\pgfqpoint{4.498413in}{2.441043in}}{\pgfqpoint{4.495501in}{2.438131in}}%
\pgfpathcurveto{\pgfqpoint{4.492589in}{2.435219in}}{\pgfqpoint{4.490952in}{2.431269in}}{\pgfqpoint{4.490952in}{2.427151in}}%
\pgfpathcurveto{\pgfqpoint{4.490952in}{2.423033in}}{\pgfqpoint{4.492589in}{2.419083in}}{\pgfqpoint{4.495501in}{2.416171in}}%
\pgfpathcurveto{\pgfqpoint{4.498413in}{2.413259in}}{\pgfqpoint{4.502363in}{2.411623in}}{\pgfqpoint{4.506481in}{2.411623in}}%
\pgfpathclose%
\pgfusepath{fill}%
\end{pgfscope}%
\begin{pgfscope}%
\pgfpathrectangle{\pgfqpoint{0.887500in}{0.275000in}}{\pgfqpoint{4.225000in}{4.225000in}}%
\pgfusepath{clip}%
\pgfsetbuttcap%
\pgfsetroundjoin%
\definecolor{currentfill}{rgb}{0.000000,0.000000,0.000000}%
\pgfsetfillcolor{currentfill}%
\pgfsetfillopacity{0.800000}%
\pgfsetlinewidth{0.000000pt}%
\definecolor{currentstroke}{rgb}{0.000000,0.000000,0.000000}%
\pgfsetstrokecolor{currentstroke}%
\pgfsetstrokeopacity{0.800000}%
\pgfsetdash{}{0pt}%
\pgfpathmoveto{\pgfqpoint{2.412400in}{2.393265in}}%
\pgfpathcurveto{\pgfqpoint{2.416518in}{2.393265in}}{\pgfqpoint{2.420468in}{2.394901in}}{\pgfqpoint{2.423380in}{2.397813in}}%
\pgfpathcurveto{\pgfqpoint{2.426292in}{2.400725in}}{\pgfqpoint{2.427928in}{2.404675in}}{\pgfqpoint{2.427928in}{2.408793in}}%
\pgfpathcurveto{\pgfqpoint{2.427928in}{2.412912in}}{\pgfqpoint{2.426292in}{2.416862in}}{\pgfqpoint{2.423380in}{2.419773in}}%
\pgfpathcurveto{\pgfqpoint{2.420468in}{2.422685in}}{\pgfqpoint{2.416518in}{2.424322in}}{\pgfqpoint{2.412400in}{2.424322in}}%
\pgfpathcurveto{\pgfqpoint{2.408282in}{2.424322in}}{\pgfqpoint{2.404332in}{2.422685in}}{\pgfqpoint{2.401420in}{2.419773in}}%
\pgfpathcurveto{\pgfqpoint{2.398508in}{2.416862in}}{\pgfqpoint{2.396872in}{2.412912in}}{\pgfqpoint{2.396872in}{2.408793in}}%
\pgfpathcurveto{\pgfqpoint{2.396872in}{2.404675in}}{\pgfqpoint{2.398508in}{2.400725in}}{\pgfqpoint{2.401420in}{2.397813in}}%
\pgfpathcurveto{\pgfqpoint{2.404332in}{2.394901in}}{\pgfqpoint{2.408282in}{2.393265in}}{\pgfqpoint{2.412400in}{2.393265in}}%
\pgfpathclose%
\pgfusepath{fill}%
\end{pgfscope}%
\begin{pgfscope}%
\pgfpathrectangle{\pgfqpoint{0.887500in}{0.275000in}}{\pgfqpoint{4.225000in}{4.225000in}}%
\pgfusepath{clip}%
\pgfsetbuttcap%
\pgfsetroundjoin%
\definecolor{currentfill}{rgb}{0.000000,0.000000,0.000000}%
\pgfsetfillcolor{currentfill}%
\pgfsetfillopacity{0.800000}%
\pgfsetlinewidth{0.000000pt}%
\definecolor{currentstroke}{rgb}{0.000000,0.000000,0.000000}%
\pgfsetstrokecolor{currentstroke}%
\pgfsetstrokeopacity{0.800000}%
\pgfsetdash{}{0pt}%
\pgfpathmoveto{\pgfqpoint{1.944683in}{2.468865in}}%
\pgfpathcurveto{\pgfqpoint{1.948801in}{2.468865in}}{\pgfqpoint{1.952751in}{2.470502in}}{\pgfqpoint{1.955663in}{2.473413in}}%
\pgfpathcurveto{\pgfqpoint{1.958575in}{2.476325in}}{\pgfqpoint{1.960211in}{2.480275in}}{\pgfqpoint{1.960211in}{2.484394in}}%
\pgfpathcurveto{\pgfqpoint{1.960211in}{2.488512in}}{\pgfqpoint{1.958575in}{2.492462in}}{\pgfqpoint{1.955663in}{2.495374in}}%
\pgfpathcurveto{\pgfqpoint{1.952751in}{2.498286in}}{\pgfqpoint{1.948801in}{2.499922in}}{\pgfqpoint{1.944683in}{2.499922in}}%
\pgfpathcurveto{\pgfqpoint{1.940565in}{2.499922in}}{\pgfqpoint{1.936615in}{2.498286in}}{\pgfqpoint{1.933703in}{2.495374in}}%
\pgfpathcurveto{\pgfqpoint{1.930791in}{2.492462in}}{\pgfqpoint{1.929155in}{2.488512in}}{\pgfqpoint{1.929155in}{2.484394in}}%
\pgfpathcurveto{\pgfqpoint{1.929155in}{2.480275in}}{\pgfqpoint{1.930791in}{2.476325in}}{\pgfqpoint{1.933703in}{2.473413in}}%
\pgfpathcurveto{\pgfqpoint{1.936615in}{2.470502in}}{\pgfqpoint{1.940565in}{2.468865in}}{\pgfqpoint{1.944683in}{2.468865in}}%
\pgfpathclose%
\pgfusepath{fill}%
\end{pgfscope}%
\begin{pgfscope}%
\pgfpathrectangle{\pgfqpoint{0.887500in}{0.275000in}}{\pgfqpoint{4.225000in}{4.225000in}}%
\pgfusepath{clip}%
\pgfsetbuttcap%
\pgfsetroundjoin%
\definecolor{currentfill}{rgb}{0.000000,0.000000,0.000000}%
\pgfsetfillcolor{currentfill}%
\pgfsetfillopacity{0.800000}%
\pgfsetlinewidth{0.000000pt}%
\definecolor{currentstroke}{rgb}{0.000000,0.000000,0.000000}%
\pgfsetstrokecolor{currentstroke}%
\pgfsetstrokeopacity{0.800000}%
\pgfsetdash{}{0pt}%
\pgfpathmoveto{\pgfqpoint{4.329700in}{2.526170in}}%
\pgfpathcurveto{\pgfqpoint{4.333818in}{2.526170in}}{\pgfqpoint{4.337768in}{2.527806in}}{\pgfqpoint{4.340680in}{2.530718in}}%
\pgfpathcurveto{\pgfqpoint{4.343592in}{2.533630in}}{\pgfqpoint{4.345228in}{2.537580in}}{\pgfqpoint{4.345228in}{2.541698in}}%
\pgfpathcurveto{\pgfqpoint{4.345228in}{2.545816in}}{\pgfqpoint{4.343592in}{2.549766in}}{\pgfqpoint{4.340680in}{2.552678in}}%
\pgfpathcurveto{\pgfqpoint{4.337768in}{2.555590in}}{\pgfqpoint{4.333818in}{2.557226in}}{\pgfqpoint{4.329700in}{2.557226in}}%
\pgfpathcurveto{\pgfqpoint{4.325581in}{2.557226in}}{\pgfqpoint{4.321631in}{2.555590in}}{\pgfqpoint{4.318719in}{2.552678in}}%
\pgfpathcurveto{\pgfqpoint{4.315807in}{2.549766in}}{\pgfqpoint{4.314171in}{2.545816in}}{\pgfqpoint{4.314171in}{2.541698in}}%
\pgfpathcurveto{\pgfqpoint{4.314171in}{2.537580in}}{\pgfqpoint{4.315807in}{2.533630in}}{\pgfqpoint{4.318719in}{2.530718in}}%
\pgfpathcurveto{\pgfqpoint{4.321631in}{2.527806in}}{\pgfqpoint{4.325581in}{2.526170in}}{\pgfqpoint{4.329700in}{2.526170in}}%
\pgfpathclose%
\pgfusepath{fill}%
\end{pgfscope}%
\begin{pgfscope}%
\pgfpathrectangle{\pgfqpoint{0.887500in}{0.275000in}}{\pgfqpoint{4.225000in}{4.225000in}}%
\pgfusepath{clip}%
\pgfsetbuttcap%
\pgfsetroundjoin%
\definecolor{currentfill}{rgb}{0.000000,0.000000,0.000000}%
\pgfsetfillcolor{currentfill}%
\pgfsetfillopacity{0.800000}%
\pgfsetlinewidth{0.000000pt}%
\definecolor{currentstroke}{rgb}{0.000000,0.000000,0.000000}%
\pgfsetstrokecolor{currentstroke}%
\pgfsetstrokeopacity{0.800000}%
\pgfsetdash{}{0pt}%
\pgfpathmoveto{\pgfqpoint{1.476812in}{2.539284in}}%
\pgfpathcurveto{\pgfqpoint{1.480930in}{2.539284in}}{\pgfqpoint{1.484880in}{2.540920in}}{\pgfqpoint{1.487792in}{2.543832in}}%
\pgfpathcurveto{\pgfqpoint{1.490704in}{2.546744in}}{\pgfqpoint{1.492340in}{2.550694in}}{\pgfqpoint{1.492340in}{2.554812in}}%
\pgfpathcurveto{\pgfqpoint{1.492340in}{2.558931in}}{\pgfqpoint{1.490704in}{2.562881in}}{\pgfqpoint{1.487792in}{2.565793in}}%
\pgfpathcurveto{\pgfqpoint{1.484880in}{2.568705in}}{\pgfqpoint{1.480930in}{2.570341in}}{\pgfqpoint{1.476812in}{2.570341in}}%
\pgfpathcurveto{\pgfqpoint{1.472694in}{2.570341in}}{\pgfqpoint{1.468744in}{2.568705in}}{\pgfqpoint{1.465832in}{2.565793in}}%
\pgfpathcurveto{\pgfqpoint{1.462920in}{2.562881in}}{\pgfqpoint{1.461283in}{2.558931in}}{\pgfqpoint{1.461283in}{2.554812in}}%
\pgfpathcurveto{\pgfqpoint{1.461283in}{2.550694in}}{\pgfqpoint{1.462920in}{2.546744in}}{\pgfqpoint{1.465832in}{2.543832in}}%
\pgfpathcurveto{\pgfqpoint{1.468744in}{2.540920in}}{\pgfqpoint{1.472694in}{2.539284in}}{\pgfqpoint{1.476812in}{2.539284in}}%
\pgfpathclose%
\pgfusepath{fill}%
\end{pgfscope}%
\begin{pgfscope}%
\pgfpathrectangle{\pgfqpoint{0.887500in}{0.275000in}}{\pgfqpoint{4.225000in}{4.225000in}}%
\pgfusepath{clip}%
\pgfsetbuttcap%
\pgfsetroundjoin%
\definecolor{currentfill}{rgb}{0.000000,0.000000,0.000000}%
\pgfsetfillcolor{currentfill}%
\pgfsetfillopacity{0.800000}%
\pgfsetlinewidth{0.000000pt}%
\definecolor{currentstroke}{rgb}{0.000000,0.000000,0.000000}%
\pgfsetstrokecolor{currentstroke}%
\pgfsetstrokeopacity{0.800000}%
\pgfsetdash{}{0pt}%
\pgfpathmoveto{\pgfqpoint{4.152749in}{2.641001in}}%
\pgfpathcurveto{\pgfqpoint{4.156867in}{2.641001in}}{\pgfqpoint{4.160817in}{2.642638in}}{\pgfqpoint{4.163729in}{2.645549in}}%
\pgfpathcurveto{\pgfqpoint{4.166641in}{2.648461in}}{\pgfqpoint{4.168278in}{2.652411in}}{\pgfqpoint{4.168278in}{2.656530in}}%
\pgfpathcurveto{\pgfqpoint{4.168278in}{2.660648in}}{\pgfqpoint{4.166641in}{2.664598in}}{\pgfqpoint{4.163729in}{2.667510in}}%
\pgfpathcurveto{\pgfqpoint{4.160817in}{2.670422in}}{\pgfqpoint{4.156867in}{2.672058in}}{\pgfqpoint{4.152749in}{2.672058in}}%
\pgfpathcurveto{\pgfqpoint{4.148631in}{2.672058in}}{\pgfqpoint{4.144681in}{2.670422in}}{\pgfqpoint{4.141769in}{2.667510in}}%
\pgfpathcurveto{\pgfqpoint{4.138857in}{2.664598in}}{\pgfqpoint{4.137221in}{2.660648in}}{\pgfqpoint{4.137221in}{2.656530in}}%
\pgfpathcurveto{\pgfqpoint{4.137221in}{2.652411in}}{\pgfqpoint{4.138857in}{2.648461in}}{\pgfqpoint{4.141769in}{2.645549in}}%
\pgfpathcurveto{\pgfqpoint{4.144681in}{2.642638in}}{\pgfqpoint{4.148631in}{2.641001in}}{\pgfqpoint{4.152749in}{2.641001in}}%
\pgfpathclose%
\pgfusepath{fill}%
\end{pgfscope}%
\begin{pgfscope}%
\pgfpathrectangle{\pgfqpoint{0.887500in}{0.275000in}}{\pgfqpoint{4.225000in}{4.225000in}}%
\pgfusepath{clip}%
\pgfsetbuttcap%
\pgfsetroundjoin%
\definecolor{currentfill}{rgb}{0.000000,0.000000,0.000000}%
\pgfsetfillcolor{currentfill}%
\pgfsetfillopacity{0.800000}%
\pgfsetlinewidth{0.000000pt}%
\definecolor{currentstroke}{rgb}{0.000000,0.000000,0.000000}%
\pgfsetstrokecolor{currentstroke}%
\pgfsetstrokeopacity{0.800000}%
\pgfsetdash{}{0pt}%
\pgfpathmoveto{\pgfqpoint{2.589345in}{2.337222in}}%
\pgfpathcurveto{\pgfqpoint{2.593463in}{2.337222in}}{\pgfqpoint{2.597413in}{2.338858in}}{\pgfqpoint{2.600325in}{2.341770in}}%
\pgfpathcurveto{\pgfqpoint{2.603237in}{2.344682in}}{\pgfqpoint{2.604873in}{2.348632in}}{\pgfqpoint{2.604873in}{2.352750in}}%
\pgfpathcurveto{\pgfqpoint{2.604873in}{2.356868in}}{\pgfqpoint{2.603237in}{2.360818in}}{\pgfqpoint{2.600325in}{2.363730in}}%
\pgfpathcurveto{\pgfqpoint{2.597413in}{2.366642in}}{\pgfqpoint{2.593463in}{2.368279in}}{\pgfqpoint{2.589345in}{2.368279in}}%
\pgfpathcurveto{\pgfqpoint{2.585227in}{2.368279in}}{\pgfqpoint{2.581277in}{2.366642in}}{\pgfqpoint{2.578365in}{2.363730in}}%
\pgfpathcurveto{\pgfqpoint{2.575453in}{2.360818in}}{\pgfqpoint{2.573817in}{2.356868in}}{\pgfqpoint{2.573817in}{2.352750in}}%
\pgfpathcurveto{\pgfqpoint{2.573817in}{2.348632in}}{\pgfqpoint{2.575453in}{2.344682in}}{\pgfqpoint{2.578365in}{2.341770in}}%
\pgfpathcurveto{\pgfqpoint{2.581277in}{2.338858in}}{\pgfqpoint{2.585227in}{2.337222in}}{\pgfqpoint{2.589345in}{2.337222in}}%
\pgfpathclose%
\pgfusepath{fill}%
\end{pgfscope}%
\begin{pgfscope}%
\pgfpathrectangle{\pgfqpoint{0.887500in}{0.275000in}}{\pgfqpoint{4.225000in}{4.225000in}}%
\pgfusepath{clip}%
\pgfsetbuttcap%
\pgfsetroundjoin%
\definecolor{currentfill}{rgb}{0.000000,0.000000,0.000000}%
\pgfsetfillcolor{currentfill}%
\pgfsetfillopacity{0.800000}%
\pgfsetlinewidth{0.000000pt}%
\definecolor{currentstroke}{rgb}{0.000000,0.000000,0.000000}%
\pgfsetstrokecolor{currentstroke}%
\pgfsetstrokeopacity{0.800000}%
\pgfsetdash{}{0pt}%
\pgfpathmoveto{\pgfqpoint{2.927053in}{2.739931in}}%
\pgfpathcurveto{\pgfqpoint{2.931171in}{2.739931in}}{\pgfqpoint{2.935121in}{2.741567in}}{\pgfqpoint{2.938033in}{2.744479in}}%
\pgfpathcurveto{\pgfqpoint{2.940945in}{2.747391in}}{\pgfqpoint{2.942582in}{2.751341in}}{\pgfqpoint{2.942582in}{2.755459in}}%
\pgfpathcurveto{\pgfqpoint{2.942582in}{2.759577in}}{\pgfqpoint{2.940945in}{2.763527in}}{\pgfqpoint{2.938033in}{2.766439in}}%
\pgfpathcurveto{\pgfqpoint{2.935121in}{2.769351in}}{\pgfqpoint{2.931171in}{2.770987in}}{\pgfqpoint{2.927053in}{2.770987in}}%
\pgfpathcurveto{\pgfqpoint{2.922935in}{2.770987in}}{\pgfqpoint{2.918985in}{2.769351in}}{\pgfqpoint{2.916073in}{2.766439in}}%
\pgfpathcurveto{\pgfqpoint{2.913161in}{2.763527in}}{\pgfqpoint{2.911525in}{2.759577in}}{\pgfqpoint{2.911525in}{2.755459in}}%
\pgfpathcurveto{\pgfqpoint{2.911525in}{2.751341in}}{\pgfqpoint{2.913161in}{2.747391in}}{\pgfqpoint{2.916073in}{2.744479in}}%
\pgfpathcurveto{\pgfqpoint{2.918985in}{2.741567in}}{\pgfqpoint{2.922935in}{2.739931in}}{\pgfqpoint{2.927053in}{2.739931in}}%
\pgfpathclose%
\pgfusepath{fill}%
\end{pgfscope}%
\begin{pgfscope}%
\pgfpathrectangle{\pgfqpoint{0.887500in}{0.275000in}}{\pgfqpoint{4.225000in}{4.225000in}}%
\pgfusepath{clip}%
\pgfsetbuttcap%
\pgfsetroundjoin%
\definecolor{currentfill}{rgb}{0.000000,0.000000,0.000000}%
\pgfsetfillcolor{currentfill}%
\pgfsetfillopacity{0.800000}%
\pgfsetlinewidth{0.000000pt}%
\definecolor{currentstroke}{rgb}{0.000000,0.000000,0.000000}%
\pgfsetstrokecolor{currentstroke}%
\pgfsetstrokeopacity{0.800000}%
\pgfsetdash{}{0pt}%
\pgfpathmoveto{\pgfqpoint{3.975562in}{2.753268in}}%
\pgfpathcurveto{\pgfqpoint{3.979680in}{2.753268in}}{\pgfqpoint{3.983630in}{2.754904in}}{\pgfqpoint{3.986542in}{2.757816in}}%
\pgfpathcurveto{\pgfqpoint{3.989454in}{2.760728in}}{\pgfqpoint{3.991090in}{2.764678in}}{\pgfqpoint{3.991090in}{2.768796in}}%
\pgfpathcurveto{\pgfqpoint{3.991090in}{2.772914in}}{\pgfqpoint{3.989454in}{2.776864in}}{\pgfqpoint{3.986542in}{2.779776in}}%
\pgfpathcurveto{\pgfqpoint{3.983630in}{2.782688in}}{\pgfqpoint{3.979680in}{2.784324in}}{\pgfqpoint{3.975562in}{2.784324in}}%
\pgfpathcurveto{\pgfqpoint{3.971444in}{2.784324in}}{\pgfqpoint{3.967494in}{2.782688in}}{\pgfqpoint{3.964582in}{2.779776in}}%
\pgfpathcurveto{\pgfqpoint{3.961670in}{2.776864in}}{\pgfqpoint{3.960033in}{2.772914in}}{\pgfqpoint{3.960033in}{2.768796in}}%
\pgfpathcurveto{\pgfqpoint{3.960033in}{2.764678in}}{\pgfqpoint{3.961670in}{2.760728in}}{\pgfqpoint{3.964582in}{2.757816in}}%
\pgfpathcurveto{\pgfqpoint{3.967494in}{2.754904in}}{\pgfqpoint{3.971444in}{2.753268in}}{\pgfqpoint{3.975562in}{2.753268in}}%
\pgfpathclose%
\pgfusepath{fill}%
\end{pgfscope}%
\begin{pgfscope}%
\pgfpathrectangle{\pgfqpoint{0.887500in}{0.275000in}}{\pgfqpoint{4.225000in}{4.225000in}}%
\pgfusepath{clip}%
\pgfsetbuttcap%
\pgfsetroundjoin%
\definecolor{currentfill}{rgb}{0.000000,0.000000,0.000000}%
\pgfsetfillcolor{currentfill}%
\pgfsetfillopacity{0.800000}%
\pgfsetlinewidth{0.000000pt}%
\definecolor{currentstroke}{rgb}{0.000000,0.000000,0.000000}%
\pgfsetstrokecolor{currentstroke}%
\pgfsetstrokeopacity{0.800000}%
\pgfsetdash{}{0pt}%
\pgfpathmoveto{\pgfqpoint{2.121182in}{2.415018in}}%
\pgfpathcurveto{\pgfqpoint{2.125301in}{2.415018in}}{\pgfqpoint{2.129251in}{2.416654in}}{\pgfqpoint{2.132162in}{2.419566in}}%
\pgfpathcurveto{\pgfqpoint{2.135074in}{2.422478in}}{\pgfqpoint{2.136711in}{2.426428in}}{\pgfqpoint{2.136711in}{2.430546in}}%
\pgfpathcurveto{\pgfqpoint{2.136711in}{2.434664in}}{\pgfqpoint{2.135074in}{2.438614in}}{\pgfqpoint{2.132162in}{2.441526in}}%
\pgfpathcurveto{\pgfqpoint{2.129251in}{2.444438in}}{\pgfqpoint{2.125301in}{2.446074in}}{\pgfqpoint{2.121182in}{2.446074in}}%
\pgfpathcurveto{\pgfqpoint{2.117064in}{2.446074in}}{\pgfqpoint{2.113114in}{2.444438in}}{\pgfqpoint{2.110202in}{2.441526in}}%
\pgfpathcurveto{\pgfqpoint{2.107290in}{2.438614in}}{\pgfqpoint{2.105654in}{2.434664in}}{\pgfqpoint{2.105654in}{2.430546in}}%
\pgfpathcurveto{\pgfqpoint{2.105654in}{2.426428in}}{\pgfqpoint{2.107290in}{2.422478in}}{\pgfqpoint{2.110202in}{2.419566in}}%
\pgfpathcurveto{\pgfqpoint{2.113114in}{2.416654in}}{\pgfqpoint{2.117064in}{2.415018in}}{\pgfqpoint{2.121182in}{2.415018in}}%
\pgfpathclose%
\pgfusepath{fill}%
\end{pgfscope}%
\begin{pgfscope}%
\pgfpathrectangle{\pgfqpoint{0.887500in}{0.275000in}}{\pgfqpoint{4.225000in}{4.225000in}}%
\pgfusepath{clip}%
\pgfsetbuttcap%
\pgfsetroundjoin%
\definecolor{currentfill}{rgb}{0.000000,0.000000,0.000000}%
\pgfsetfillcolor{currentfill}%
\pgfsetfillopacity{0.800000}%
\pgfsetlinewidth{0.000000pt}%
\definecolor{currentstroke}{rgb}{0.000000,0.000000,0.000000}%
\pgfsetstrokecolor{currentstroke}%
\pgfsetstrokeopacity{0.800000}%
\pgfsetdash{}{0pt}%
\pgfpathmoveto{\pgfqpoint{3.087281in}{3.239433in}}%
\pgfpathcurveto{\pgfqpoint{3.091400in}{3.239433in}}{\pgfqpoint{3.095350in}{3.241069in}}{\pgfqpoint{3.098262in}{3.243981in}}%
\pgfpathcurveto{\pgfqpoint{3.101173in}{3.246893in}}{\pgfqpoint{3.102810in}{3.250843in}}{\pgfqpoint{3.102810in}{3.254961in}}%
\pgfpathcurveto{\pgfqpoint{3.102810in}{3.259079in}}{\pgfqpoint{3.101173in}{3.263029in}}{\pgfqpoint{3.098262in}{3.265941in}}%
\pgfpathcurveto{\pgfqpoint{3.095350in}{3.268853in}}{\pgfqpoint{3.091400in}{3.270489in}}{\pgfqpoint{3.087281in}{3.270489in}}%
\pgfpathcurveto{\pgfqpoint{3.083163in}{3.270489in}}{\pgfqpoint{3.079213in}{3.268853in}}{\pgfqpoint{3.076301in}{3.265941in}}%
\pgfpathcurveto{\pgfqpoint{3.073389in}{3.263029in}}{\pgfqpoint{3.071753in}{3.259079in}}{\pgfqpoint{3.071753in}{3.254961in}}%
\pgfpathcurveto{\pgfqpoint{3.071753in}{3.250843in}}{\pgfqpoint{3.073389in}{3.246893in}}{\pgfqpoint{3.076301in}{3.243981in}}%
\pgfpathcurveto{\pgfqpoint{3.079213in}{3.241069in}}{\pgfqpoint{3.083163in}{3.239433in}}{\pgfqpoint{3.087281in}{3.239433in}}%
\pgfpathclose%
\pgfusepath{fill}%
\end{pgfscope}%
\begin{pgfscope}%
\pgfpathrectangle{\pgfqpoint{0.887500in}{0.275000in}}{\pgfqpoint{4.225000in}{4.225000in}}%
\pgfusepath{clip}%
\pgfsetbuttcap%
\pgfsetroundjoin%
\definecolor{currentfill}{rgb}{0.000000,0.000000,0.000000}%
\pgfsetfillcolor{currentfill}%
\pgfsetfillopacity{0.800000}%
\pgfsetlinewidth{0.000000pt}%
\definecolor{currentstroke}{rgb}{0.000000,0.000000,0.000000}%
\pgfsetstrokecolor{currentstroke}%
\pgfsetstrokeopacity{0.800000}%
\pgfsetdash{}{0pt}%
\pgfpathmoveto{\pgfqpoint{3.798151in}{2.861816in}}%
\pgfpathcurveto{\pgfqpoint{3.802270in}{2.861816in}}{\pgfqpoint{3.806220in}{2.863452in}}{\pgfqpoint{3.809132in}{2.866364in}}%
\pgfpathcurveto{\pgfqpoint{3.812043in}{2.869276in}}{\pgfqpoint{3.813680in}{2.873226in}}{\pgfqpoint{3.813680in}{2.877345in}}%
\pgfpathcurveto{\pgfqpoint{3.813680in}{2.881463in}}{\pgfqpoint{3.812043in}{2.885413in}}{\pgfqpoint{3.809132in}{2.888325in}}%
\pgfpathcurveto{\pgfqpoint{3.806220in}{2.891237in}}{\pgfqpoint{3.802270in}{2.892873in}}{\pgfqpoint{3.798151in}{2.892873in}}%
\pgfpathcurveto{\pgfqpoint{3.794033in}{2.892873in}}{\pgfqpoint{3.790083in}{2.891237in}}{\pgfqpoint{3.787171in}{2.888325in}}%
\pgfpathcurveto{\pgfqpoint{3.784259in}{2.885413in}}{\pgfqpoint{3.782623in}{2.881463in}}{\pgfqpoint{3.782623in}{2.877345in}}%
\pgfpathcurveto{\pgfqpoint{3.782623in}{2.873226in}}{\pgfqpoint{3.784259in}{2.869276in}}{\pgfqpoint{3.787171in}{2.866364in}}%
\pgfpathcurveto{\pgfqpoint{3.790083in}{2.863452in}}{\pgfqpoint{3.794033in}{2.861816in}}{\pgfqpoint{3.798151in}{2.861816in}}%
\pgfpathclose%
\pgfusepath{fill}%
\end{pgfscope}%
\begin{pgfscope}%
\pgfpathrectangle{\pgfqpoint{0.887500in}{0.275000in}}{\pgfqpoint{4.225000in}{4.225000in}}%
\pgfusepath{clip}%
\pgfsetbuttcap%
\pgfsetroundjoin%
\definecolor{currentfill}{rgb}{0.000000,0.000000,0.000000}%
\pgfsetfillcolor{currentfill}%
\pgfsetfillopacity{0.800000}%
\pgfsetlinewidth{0.000000pt}%
\definecolor{currentstroke}{rgb}{0.000000,0.000000,0.000000}%
\pgfsetstrokecolor{currentstroke}%
\pgfsetstrokeopacity{0.800000}%
\pgfsetdash{}{0pt}%
\pgfpathmoveto{\pgfqpoint{1.652755in}{2.488673in}}%
\pgfpathcurveto{\pgfqpoint{1.656873in}{2.488673in}}{\pgfqpoint{1.660823in}{2.490310in}}{\pgfqpoint{1.663735in}{2.493221in}}%
\pgfpathcurveto{\pgfqpoint{1.666647in}{2.496133in}}{\pgfqpoint{1.668283in}{2.500083in}}{\pgfqpoint{1.668283in}{2.504202in}}%
\pgfpathcurveto{\pgfqpoint{1.668283in}{2.508320in}}{\pgfqpoint{1.666647in}{2.512270in}}{\pgfqpoint{1.663735in}{2.515182in}}%
\pgfpathcurveto{\pgfqpoint{1.660823in}{2.518094in}}{\pgfqpoint{1.656873in}{2.519730in}}{\pgfqpoint{1.652755in}{2.519730in}}%
\pgfpathcurveto{\pgfqpoint{1.648637in}{2.519730in}}{\pgfqpoint{1.644687in}{2.518094in}}{\pgfqpoint{1.641775in}{2.515182in}}%
\pgfpathcurveto{\pgfqpoint{1.638863in}{2.512270in}}{\pgfqpoint{1.637227in}{2.508320in}}{\pgfqpoint{1.637227in}{2.504202in}}%
\pgfpathcurveto{\pgfqpoint{1.637227in}{2.500083in}}{\pgfqpoint{1.638863in}{2.496133in}}{\pgfqpoint{1.641775in}{2.493221in}}%
\pgfpathcurveto{\pgfqpoint{1.644687in}{2.490310in}}{\pgfqpoint{1.648637in}{2.488673in}}{\pgfqpoint{1.652755in}{2.488673in}}%
\pgfpathclose%
\pgfusepath{fill}%
\end{pgfscope}%
\begin{pgfscope}%
\pgfpathrectangle{\pgfqpoint{0.887500in}{0.275000in}}{\pgfqpoint{4.225000in}{4.225000in}}%
\pgfusepath{clip}%
\pgfsetbuttcap%
\pgfsetroundjoin%
\definecolor{currentfill}{rgb}{0.000000,0.000000,0.000000}%
\pgfsetfillcolor{currentfill}%
\pgfsetfillopacity{0.800000}%
\pgfsetlinewidth{0.000000pt}%
\definecolor{currentstroke}{rgb}{0.000000,0.000000,0.000000}%
\pgfsetstrokecolor{currentstroke}%
\pgfsetstrokeopacity{0.800000}%
\pgfsetdash{}{0pt}%
\pgfpathmoveto{\pgfqpoint{2.766658in}{2.278676in}}%
\pgfpathcurveto{\pgfqpoint{2.770777in}{2.278676in}}{\pgfqpoint{2.774727in}{2.280312in}}{\pgfqpoint{2.777638in}{2.283224in}}%
\pgfpathcurveto{\pgfqpoint{2.780550in}{2.286136in}}{\pgfqpoint{2.782187in}{2.290086in}}{\pgfqpoint{2.782187in}{2.294204in}}%
\pgfpathcurveto{\pgfqpoint{2.782187in}{2.298322in}}{\pgfqpoint{2.780550in}{2.302272in}}{\pgfqpoint{2.777638in}{2.305184in}}%
\pgfpathcurveto{\pgfqpoint{2.774727in}{2.308096in}}{\pgfqpoint{2.770777in}{2.309732in}}{\pgfqpoint{2.766658in}{2.309732in}}%
\pgfpathcurveto{\pgfqpoint{2.762540in}{2.309732in}}{\pgfqpoint{2.758590in}{2.308096in}}{\pgfqpoint{2.755678in}{2.305184in}}%
\pgfpathcurveto{\pgfqpoint{2.752766in}{2.302272in}}{\pgfqpoint{2.751130in}{2.298322in}}{\pgfqpoint{2.751130in}{2.294204in}}%
\pgfpathcurveto{\pgfqpoint{2.751130in}{2.290086in}}{\pgfqpoint{2.752766in}{2.286136in}}{\pgfqpoint{2.755678in}{2.283224in}}%
\pgfpathcurveto{\pgfqpoint{2.758590in}{2.280312in}}{\pgfqpoint{2.762540in}{2.278676in}}{\pgfqpoint{2.766658in}{2.278676in}}%
\pgfpathclose%
\pgfusepath{fill}%
\end{pgfscope}%
\begin{pgfscope}%
\pgfpathrectangle{\pgfqpoint{0.887500in}{0.275000in}}{\pgfqpoint{4.225000in}{4.225000in}}%
\pgfusepath{clip}%
\pgfsetbuttcap%
\pgfsetroundjoin%
\definecolor{currentfill}{rgb}{0.000000,0.000000,0.000000}%
\pgfsetfillcolor{currentfill}%
\pgfsetfillopacity{0.800000}%
\pgfsetlinewidth{0.000000pt}%
\definecolor{currentstroke}{rgb}{0.000000,0.000000,0.000000}%
\pgfsetstrokecolor{currentstroke}%
\pgfsetstrokeopacity{0.800000}%
\pgfsetdash{}{0pt}%
\pgfpathmoveto{\pgfqpoint{4.574413in}{2.264502in}}%
\pgfpathcurveto{\pgfqpoint{4.578532in}{2.264502in}}{\pgfqpoint{4.582482in}{2.266138in}}{\pgfqpoint{4.585394in}{2.269050in}}%
\pgfpathcurveto{\pgfqpoint{4.588306in}{2.271962in}}{\pgfqpoint{4.589942in}{2.275912in}}{\pgfqpoint{4.589942in}{2.280030in}}%
\pgfpathcurveto{\pgfqpoint{4.589942in}{2.284149in}}{\pgfqpoint{4.588306in}{2.288099in}}{\pgfqpoint{4.585394in}{2.291011in}}%
\pgfpathcurveto{\pgfqpoint{4.582482in}{2.293922in}}{\pgfqpoint{4.578532in}{2.295559in}}{\pgfqpoint{4.574413in}{2.295559in}}%
\pgfpathcurveto{\pgfqpoint{4.570295in}{2.295559in}}{\pgfqpoint{4.566345in}{2.293922in}}{\pgfqpoint{4.563433in}{2.291011in}}%
\pgfpathcurveto{\pgfqpoint{4.560521in}{2.288099in}}{\pgfqpoint{4.558885in}{2.284149in}}{\pgfqpoint{4.558885in}{2.280030in}}%
\pgfpathcurveto{\pgfqpoint{4.558885in}{2.275912in}}{\pgfqpoint{4.560521in}{2.271962in}}{\pgfqpoint{4.563433in}{2.269050in}}%
\pgfpathcurveto{\pgfqpoint{4.566345in}{2.266138in}}{\pgfqpoint{4.570295in}{2.264502in}}{\pgfqpoint{4.574413in}{2.264502in}}%
\pgfpathclose%
\pgfusepath{fill}%
\end{pgfscope}%
\begin{pgfscope}%
\pgfpathrectangle{\pgfqpoint{0.887500in}{0.275000in}}{\pgfqpoint{4.225000in}{4.225000in}}%
\pgfusepath{clip}%
\pgfsetbuttcap%
\pgfsetroundjoin%
\definecolor{currentfill}{rgb}{0.000000,0.000000,0.000000}%
\pgfsetfillcolor{currentfill}%
\pgfsetfillopacity{0.800000}%
\pgfsetlinewidth{0.000000pt}%
\definecolor{currentstroke}{rgb}{0.000000,0.000000,0.000000}%
\pgfsetstrokecolor{currentstroke}%
\pgfsetstrokeopacity{0.800000}%
\pgfsetdash{}{0pt}%
\pgfpathmoveto{\pgfqpoint{3.620566in}{2.966438in}}%
\pgfpathcurveto{\pgfqpoint{3.624685in}{2.966438in}}{\pgfqpoint{3.628635in}{2.968074in}}{\pgfqpoint{3.631547in}{2.970986in}}%
\pgfpathcurveto{\pgfqpoint{3.634458in}{2.973898in}}{\pgfqpoint{3.636095in}{2.977848in}}{\pgfqpoint{3.636095in}{2.981966in}}%
\pgfpathcurveto{\pgfqpoint{3.636095in}{2.986084in}}{\pgfqpoint{3.634458in}{2.990034in}}{\pgfqpoint{3.631547in}{2.992946in}}%
\pgfpathcurveto{\pgfqpoint{3.628635in}{2.995858in}}{\pgfqpoint{3.624685in}{2.997495in}}{\pgfqpoint{3.620566in}{2.997495in}}%
\pgfpathcurveto{\pgfqpoint{3.616448in}{2.997495in}}{\pgfqpoint{3.612498in}{2.995858in}}{\pgfqpoint{3.609586in}{2.992946in}}%
\pgfpathcurveto{\pgfqpoint{3.606674in}{2.990034in}}{\pgfqpoint{3.605038in}{2.986084in}}{\pgfqpoint{3.605038in}{2.981966in}}%
\pgfpathcurveto{\pgfqpoint{3.605038in}{2.977848in}}{\pgfqpoint{3.606674in}{2.973898in}}{\pgfqpoint{3.609586in}{2.970986in}}%
\pgfpathcurveto{\pgfqpoint{3.612498in}{2.968074in}}{\pgfqpoint{3.616448in}{2.966438in}}{\pgfqpoint{3.620566in}{2.966438in}}%
\pgfpathclose%
\pgfusepath{fill}%
\end{pgfscope}%
\begin{pgfscope}%
\pgfpathrectangle{\pgfqpoint{0.887500in}{0.275000in}}{\pgfqpoint{4.225000in}{4.225000in}}%
\pgfusepath{clip}%
\pgfsetbuttcap%
\pgfsetroundjoin%
\definecolor{currentfill}{rgb}{0.000000,0.000000,0.000000}%
\pgfsetfillcolor{currentfill}%
\pgfsetfillopacity{0.800000}%
\pgfsetlinewidth{0.000000pt}%
\definecolor{currentstroke}{rgb}{0.000000,0.000000,0.000000}%
\pgfsetstrokecolor{currentstroke}%
\pgfsetstrokeopacity{0.800000}%
\pgfsetdash{}{0pt}%
\pgfpathmoveto{\pgfqpoint{3.265069in}{3.160710in}}%
\pgfpathcurveto{\pgfqpoint{3.269187in}{3.160710in}}{\pgfqpoint{3.273137in}{3.162346in}}{\pgfqpoint{3.276049in}{3.165258in}}%
\pgfpathcurveto{\pgfqpoint{3.278961in}{3.168170in}}{\pgfqpoint{3.280597in}{3.172120in}}{\pgfqpoint{3.280597in}{3.176238in}}%
\pgfpathcurveto{\pgfqpoint{3.280597in}{3.180356in}}{\pgfqpoint{3.278961in}{3.184306in}}{\pgfqpoint{3.276049in}{3.187218in}}%
\pgfpathcurveto{\pgfqpoint{3.273137in}{3.190130in}}{\pgfqpoint{3.269187in}{3.191766in}}{\pgfqpoint{3.265069in}{3.191766in}}%
\pgfpathcurveto{\pgfqpoint{3.260951in}{3.191766in}}{\pgfqpoint{3.257001in}{3.190130in}}{\pgfqpoint{3.254089in}{3.187218in}}%
\pgfpathcurveto{\pgfqpoint{3.251177in}{3.184306in}}{\pgfqpoint{3.249541in}{3.180356in}}{\pgfqpoint{3.249541in}{3.176238in}}%
\pgfpathcurveto{\pgfqpoint{3.249541in}{3.172120in}}{\pgfqpoint{3.251177in}{3.168170in}}{\pgfqpoint{3.254089in}{3.165258in}}%
\pgfpathcurveto{\pgfqpoint{3.257001in}{3.162346in}}{\pgfqpoint{3.260951in}{3.160710in}}{\pgfqpoint{3.265069in}{3.160710in}}%
\pgfpathclose%
\pgfusepath{fill}%
\end{pgfscope}%
\begin{pgfscope}%
\pgfpathrectangle{\pgfqpoint{0.887500in}{0.275000in}}{\pgfqpoint{4.225000in}{4.225000in}}%
\pgfusepath{clip}%
\pgfsetbuttcap%
\pgfsetroundjoin%
\definecolor{currentfill}{rgb}{0.000000,0.000000,0.000000}%
\pgfsetfillcolor{currentfill}%
\pgfsetfillopacity{0.800000}%
\pgfsetlinewidth{0.000000pt}%
\definecolor{currentstroke}{rgb}{0.000000,0.000000,0.000000}%
\pgfsetstrokecolor{currentstroke}%
\pgfsetstrokeopacity{0.800000}%
\pgfsetdash{}{0pt}%
\pgfpathmoveto{\pgfqpoint{3.442866in}{3.067709in}}%
\pgfpathcurveto{\pgfqpoint{3.446984in}{3.067709in}}{\pgfqpoint{3.450934in}{3.069345in}}{\pgfqpoint{3.453846in}{3.072257in}}%
\pgfpathcurveto{\pgfqpoint{3.456758in}{3.075169in}}{\pgfqpoint{3.458394in}{3.079119in}}{\pgfqpoint{3.458394in}{3.083237in}}%
\pgfpathcurveto{\pgfqpoint{3.458394in}{3.087356in}}{\pgfqpoint{3.456758in}{3.091306in}}{\pgfqpoint{3.453846in}{3.094218in}}%
\pgfpathcurveto{\pgfqpoint{3.450934in}{3.097129in}}{\pgfqpoint{3.446984in}{3.098766in}}{\pgfqpoint{3.442866in}{3.098766in}}%
\pgfpathcurveto{\pgfqpoint{3.438747in}{3.098766in}}{\pgfqpoint{3.434797in}{3.097129in}}{\pgfqpoint{3.431885in}{3.094218in}}%
\pgfpathcurveto{\pgfqpoint{3.428974in}{3.091306in}}{\pgfqpoint{3.427337in}{3.087356in}}{\pgfqpoint{3.427337in}{3.083237in}}%
\pgfpathcurveto{\pgfqpoint{3.427337in}{3.079119in}}{\pgfqpoint{3.428974in}{3.075169in}}{\pgfqpoint{3.431885in}{3.072257in}}%
\pgfpathcurveto{\pgfqpoint{3.434797in}{3.069345in}}{\pgfqpoint{3.438747in}{3.067709in}}{\pgfqpoint{3.442866in}{3.067709in}}%
\pgfpathclose%
\pgfusepath{fill}%
\end{pgfscope}%
\begin{pgfscope}%
\pgfpathrectangle{\pgfqpoint{0.887500in}{0.275000in}}{\pgfqpoint{4.225000in}{4.225000in}}%
\pgfusepath{clip}%
\pgfsetbuttcap%
\pgfsetroundjoin%
\definecolor{currentfill}{rgb}{0.000000,0.000000,0.000000}%
\pgfsetfillcolor{currentfill}%
\pgfsetfillopacity{0.800000}%
\pgfsetlinewidth{0.000000pt}%
\definecolor{currentstroke}{rgb}{0.000000,0.000000,0.000000}%
\pgfsetstrokecolor{currentstroke}%
\pgfsetstrokeopacity{0.800000}%
\pgfsetdash{}{0pt}%
\pgfpathmoveto{\pgfqpoint{4.397202in}{2.375861in}}%
\pgfpathcurveto{\pgfqpoint{4.401320in}{2.375861in}}{\pgfqpoint{4.405270in}{2.377498in}}{\pgfqpoint{4.408182in}{2.380410in}}%
\pgfpathcurveto{\pgfqpoint{4.411094in}{2.383322in}}{\pgfqpoint{4.412731in}{2.387272in}}{\pgfqpoint{4.412731in}{2.391390in}}%
\pgfpathcurveto{\pgfqpoint{4.412731in}{2.395508in}}{\pgfqpoint{4.411094in}{2.399458in}}{\pgfqpoint{4.408182in}{2.402370in}}%
\pgfpathcurveto{\pgfqpoint{4.405270in}{2.405282in}}{\pgfqpoint{4.401320in}{2.406918in}}{\pgfqpoint{4.397202in}{2.406918in}}%
\pgfpathcurveto{\pgfqpoint{4.393084in}{2.406918in}}{\pgfqpoint{4.389134in}{2.405282in}}{\pgfqpoint{4.386222in}{2.402370in}}%
\pgfpathcurveto{\pgfqpoint{4.383310in}{2.399458in}}{\pgfqpoint{4.381674in}{2.395508in}}{\pgfqpoint{4.381674in}{2.391390in}}%
\pgfpathcurveto{\pgfqpoint{4.381674in}{2.387272in}}{\pgfqpoint{4.383310in}{2.383322in}}{\pgfqpoint{4.386222in}{2.380410in}}%
\pgfpathcurveto{\pgfqpoint{4.389134in}{2.377498in}}{\pgfqpoint{4.393084in}{2.375861in}}{\pgfqpoint{4.397202in}{2.375861in}}%
\pgfpathclose%
\pgfusepath{fill}%
\end{pgfscope}%
\begin{pgfscope}%
\pgfpathrectangle{\pgfqpoint{0.887500in}{0.275000in}}{\pgfqpoint{4.225000in}{4.225000in}}%
\pgfusepath{clip}%
\pgfsetbuttcap%
\pgfsetroundjoin%
\definecolor{currentfill}{rgb}{0.000000,0.000000,0.000000}%
\pgfsetfillcolor{currentfill}%
\pgfsetfillopacity{0.800000}%
\pgfsetlinewidth{0.000000pt}%
\definecolor{currentstroke}{rgb}{0.000000,0.000000,0.000000}%
\pgfsetstrokecolor{currentstroke}%
\pgfsetstrokeopacity{0.800000}%
\pgfsetdash{}{0pt}%
\pgfpathmoveto{\pgfqpoint{2.298074in}{2.359858in}}%
\pgfpathcurveto{\pgfqpoint{2.302192in}{2.359858in}}{\pgfqpoint{2.306142in}{2.361494in}}{\pgfqpoint{2.309054in}{2.364406in}}%
\pgfpathcurveto{\pgfqpoint{2.311966in}{2.367318in}}{\pgfqpoint{2.313602in}{2.371268in}}{\pgfqpoint{2.313602in}{2.375386in}}%
\pgfpathcurveto{\pgfqpoint{2.313602in}{2.379504in}}{\pgfqpoint{2.311966in}{2.383454in}}{\pgfqpoint{2.309054in}{2.386366in}}%
\pgfpathcurveto{\pgfqpoint{2.306142in}{2.389278in}}{\pgfqpoint{2.302192in}{2.390914in}}{\pgfqpoint{2.298074in}{2.390914in}}%
\pgfpathcurveto{\pgfqpoint{2.293956in}{2.390914in}}{\pgfqpoint{2.290006in}{2.389278in}}{\pgfqpoint{2.287094in}{2.386366in}}%
\pgfpathcurveto{\pgfqpoint{2.284182in}{2.383454in}}{\pgfqpoint{2.282546in}{2.379504in}}{\pgfqpoint{2.282546in}{2.375386in}}%
\pgfpathcurveto{\pgfqpoint{2.282546in}{2.371268in}}{\pgfqpoint{2.284182in}{2.367318in}}{\pgfqpoint{2.287094in}{2.364406in}}%
\pgfpathcurveto{\pgfqpoint{2.290006in}{2.361494in}}{\pgfqpoint{2.293956in}{2.359858in}}{\pgfqpoint{2.298074in}{2.359858in}}%
\pgfpathclose%
\pgfusepath{fill}%
\end{pgfscope}%
\begin{pgfscope}%
\pgfpathrectangle{\pgfqpoint{0.887500in}{0.275000in}}{\pgfqpoint{4.225000in}{4.225000in}}%
\pgfusepath{clip}%
\pgfsetbuttcap%
\pgfsetroundjoin%
\definecolor{currentfill}{rgb}{0.000000,0.000000,0.000000}%
\pgfsetfillcolor{currentfill}%
\pgfsetfillopacity{0.800000}%
\pgfsetlinewidth{0.000000pt}%
\definecolor{currentstroke}{rgb}{0.000000,0.000000,0.000000}%
\pgfsetstrokecolor{currentstroke}%
\pgfsetstrokeopacity{0.800000}%
\pgfsetdash{}{0pt}%
\pgfpathmoveto{\pgfqpoint{2.944310in}{2.217117in}}%
\pgfpathcurveto{\pgfqpoint{2.948428in}{2.217117in}}{\pgfqpoint{2.952378in}{2.218753in}}{\pgfqpoint{2.955290in}{2.221665in}}%
\pgfpathcurveto{\pgfqpoint{2.958202in}{2.224577in}}{\pgfqpoint{2.959838in}{2.228527in}}{\pgfqpoint{2.959838in}{2.232645in}}%
\pgfpathcurveto{\pgfqpoint{2.959838in}{2.236764in}}{\pgfqpoint{2.958202in}{2.240714in}}{\pgfqpoint{2.955290in}{2.243626in}}%
\pgfpathcurveto{\pgfqpoint{2.952378in}{2.246538in}}{\pgfqpoint{2.948428in}{2.248174in}}{\pgfqpoint{2.944310in}{2.248174in}}%
\pgfpathcurveto{\pgfqpoint{2.940192in}{2.248174in}}{\pgfqpoint{2.936242in}{2.246538in}}{\pgfqpoint{2.933330in}{2.243626in}}%
\pgfpathcurveto{\pgfqpoint{2.930418in}{2.240714in}}{\pgfqpoint{2.928782in}{2.236764in}}{\pgfqpoint{2.928782in}{2.232645in}}%
\pgfpathcurveto{\pgfqpoint{2.928782in}{2.228527in}}{\pgfqpoint{2.930418in}{2.224577in}}{\pgfqpoint{2.933330in}{2.221665in}}%
\pgfpathcurveto{\pgfqpoint{2.936242in}{2.218753in}}{\pgfqpoint{2.940192in}{2.217117in}}{\pgfqpoint{2.944310in}{2.217117in}}%
\pgfpathclose%
\pgfusepath{fill}%
\end{pgfscope}%
\begin{pgfscope}%
\pgfpathrectangle{\pgfqpoint{0.887500in}{0.275000in}}{\pgfqpoint{4.225000in}{4.225000in}}%
\pgfusepath{clip}%
\pgfsetbuttcap%
\pgfsetroundjoin%
\definecolor{currentfill}{rgb}{0.000000,0.000000,0.000000}%
\pgfsetfillcolor{currentfill}%
\pgfsetfillopacity{0.800000}%
\pgfsetlinewidth{0.000000pt}%
\definecolor{currentstroke}{rgb}{0.000000,0.000000,0.000000}%
\pgfsetstrokecolor{currentstroke}%
\pgfsetstrokeopacity{0.800000}%
\pgfsetdash{}{0pt}%
\pgfpathmoveto{\pgfqpoint{1.829156in}{2.435974in}}%
\pgfpathcurveto{\pgfqpoint{1.833274in}{2.435974in}}{\pgfqpoint{1.837224in}{2.437610in}}{\pgfqpoint{1.840136in}{2.440522in}}%
\pgfpathcurveto{\pgfqpoint{1.843048in}{2.443434in}}{\pgfqpoint{1.844684in}{2.447384in}}{\pgfqpoint{1.844684in}{2.451502in}}%
\pgfpathcurveto{\pgfqpoint{1.844684in}{2.455620in}}{\pgfqpoint{1.843048in}{2.459570in}}{\pgfqpoint{1.840136in}{2.462482in}}%
\pgfpathcurveto{\pgfqpoint{1.837224in}{2.465394in}}{\pgfqpoint{1.833274in}{2.467030in}}{\pgfqpoint{1.829156in}{2.467030in}}%
\pgfpathcurveto{\pgfqpoint{1.825038in}{2.467030in}}{\pgfqpoint{1.821088in}{2.465394in}}{\pgfqpoint{1.818176in}{2.462482in}}%
\pgfpathcurveto{\pgfqpoint{1.815264in}{2.459570in}}{\pgfqpoint{1.813628in}{2.455620in}}{\pgfqpoint{1.813628in}{2.451502in}}%
\pgfpathcurveto{\pgfqpoint{1.813628in}{2.447384in}}{\pgfqpoint{1.815264in}{2.443434in}}{\pgfqpoint{1.818176in}{2.440522in}}%
\pgfpathcurveto{\pgfqpoint{1.821088in}{2.437610in}}{\pgfqpoint{1.825038in}{2.435974in}}{\pgfqpoint{1.829156in}{2.435974in}}%
\pgfpathclose%
\pgfusepath{fill}%
\end{pgfscope}%
\begin{pgfscope}%
\pgfpathrectangle{\pgfqpoint{0.887500in}{0.275000in}}{\pgfqpoint{4.225000in}{4.225000in}}%
\pgfusepath{clip}%
\pgfsetbuttcap%
\pgfsetroundjoin%
\definecolor{currentfill}{rgb}{0.000000,0.000000,0.000000}%
\pgfsetfillcolor{currentfill}%
\pgfsetfillopacity{0.800000}%
\pgfsetlinewidth{0.000000pt}%
\definecolor{currentstroke}{rgb}{0.000000,0.000000,0.000000}%
\pgfsetstrokecolor{currentstroke}%
\pgfsetstrokeopacity{0.800000}%
\pgfsetdash{}{0pt}%
\pgfpathmoveto{\pgfqpoint{4.220011in}{2.492900in}}%
\pgfpathcurveto{\pgfqpoint{4.224129in}{2.492900in}}{\pgfqpoint{4.228079in}{2.494536in}}{\pgfqpoint{4.230991in}{2.497448in}}%
\pgfpathcurveto{\pgfqpoint{4.233903in}{2.500360in}}{\pgfqpoint{4.235539in}{2.504310in}}{\pgfqpoint{4.235539in}{2.508428in}}%
\pgfpathcurveto{\pgfqpoint{4.235539in}{2.512546in}}{\pgfqpoint{4.233903in}{2.516496in}}{\pgfqpoint{4.230991in}{2.519408in}}%
\pgfpathcurveto{\pgfqpoint{4.228079in}{2.522320in}}{\pgfqpoint{4.224129in}{2.523956in}}{\pgfqpoint{4.220011in}{2.523956in}}%
\pgfpathcurveto{\pgfqpoint{4.215892in}{2.523956in}}{\pgfqpoint{4.211942in}{2.522320in}}{\pgfqpoint{4.209030in}{2.519408in}}%
\pgfpathcurveto{\pgfqpoint{4.206118in}{2.516496in}}{\pgfqpoint{4.204482in}{2.512546in}}{\pgfqpoint{4.204482in}{2.508428in}}%
\pgfpathcurveto{\pgfqpoint{4.204482in}{2.504310in}}{\pgfqpoint{4.206118in}{2.500360in}}{\pgfqpoint{4.209030in}{2.497448in}}%
\pgfpathcurveto{\pgfqpoint{4.211942in}{2.494536in}}{\pgfqpoint{4.215892in}{2.492900in}}{\pgfqpoint{4.220011in}{2.492900in}}%
\pgfpathclose%
\pgfusepath{fill}%
\end{pgfscope}%
\begin{pgfscope}%
\pgfpathrectangle{\pgfqpoint{0.887500in}{0.275000in}}{\pgfqpoint{4.225000in}{4.225000in}}%
\pgfusepath{clip}%
\pgfsetbuttcap%
\pgfsetroundjoin%
\definecolor{currentfill}{rgb}{0.000000,0.000000,0.000000}%
\pgfsetfillcolor{currentfill}%
\pgfsetfillopacity{0.800000}%
\pgfsetlinewidth{0.000000pt}%
\definecolor{currentstroke}{rgb}{0.000000,0.000000,0.000000}%
\pgfsetstrokecolor{currentstroke}%
\pgfsetstrokeopacity{0.800000}%
\pgfsetdash{}{0pt}%
\pgfpathmoveto{\pgfqpoint{2.991943in}{2.594560in}}%
\pgfpathcurveto{\pgfqpoint{2.996061in}{2.594560in}}{\pgfqpoint{3.000011in}{2.596196in}}{\pgfqpoint{3.002923in}{2.599108in}}%
\pgfpathcurveto{\pgfqpoint{3.005835in}{2.602020in}}{\pgfqpoint{3.007471in}{2.605970in}}{\pgfqpoint{3.007471in}{2.610088in}}%
\pgfpathcurveto{\pgfqpoint{3.007471in}{2.614207in}}{\pgfqpoint{3.005835in}{2.618157in}}{\pgfqpoint{3.002923in}{2.621069in}}%
\pgfpathcurveto{\pgfqpoint{3.000011in}{2.623981in}}{\pgfqpoint{2.996061in}{2.625617in}}{\pgfqpoint{2.991943in}{2.625617in}}%
\pgfpathcurveto{\pgfqpoint{2.987825in}{2.625617in}}{\pgfqpoint{2.983875in}{2.623981in}}{\pgfqpoint{2.980963in}{2.621069in}}%
\pgfpathcurveto{\pgfqpoint{2.978051in}{2.618157in}}{\pgfqpoint{2.976415in}{2.614207in}}{\pgfqpoint{2.976415in}{2.610088in}}%
\pgfpathcurveto{\pgfqpoint{2.976415in}{2.605970in}}{\pgfqpoint{2.978051in}{2.602020in}}{\pgfqpoint{2.980963in}{2.599108in}}%
\pgfpathcurveto{\pgfqpoint{2.983875in}{2.596196in}}{\pgfqpoint{2.987825in}{2.594560in}}{\pgfqpoint{2.991943in}{2.594560in}}%
\pgfpathclose%
\pgfusepath{fill}%
\end{pgfscope}%
\begin{pgfscope}%
\pgfpathrectangle{\pgfqpoint{0.887500in}{0.275000in}}{\pgfqpoint{4.225000in}{4.225000in}}%
\pgfusepath{clip}%
\pgfsetbuttcap%
\pgfsetroundjoin%
\definecolor{currentfill}{rgb}{0.000000,0.000000,0.000000}%
\pgfsetfillcolor{currentfill}%
\pgfsetfillopacity{0.800000}%
\pgfsetlinewidth{0.000000pt}%
\definecolor{currentstroke}{rgb}{0.000000,0.000000,0.000000}%
\pgfsetstrokecolor{currentstroke}%
\pgfsetstrokeopacity{0.800000}%
\pgfsetdash{}{0pt}%
\pgfpathmoveto{\pgfqpoint{4.042543in}{2.606788in}}%
\pgfpathcurveto{\pgfqpoint{4.046661in}{2.606788in}}{\pgfqpoint{4.050611in}{2.608424in}}{\pgfqpoint{4.053523in}{2.611336in}}%
\pgfpathcurveto{\pgfqpoint{4.056435in}{2.614248in}}{\pgfqpoint{4.058071in}{2.618198in}}{\pgfqpoint{4.058071in}{2.622316in}}%
\pgfpathcurveto{\pgfqpoint{4.058071in}{2.626435in}}{\pgfqpoint{4.056435in}{2.630385in}}{\pgfqpoint{4.053523in}{2.633297in}}%
\pgfpathcurveto{\pgfqpoint{4.050611in}{2.636208in}}{\pgfqpoint{4.046661in}{2.637845in}}{\pgfqpoint{4.042543in}{2.637845in}}%
\pgfpathcurveto{\pgfqpoint{4.038425in}{2.637845in}}{\pgfqpoint{4.034475in}{2.636208in}}{\pgfqpoint{4.031563in}{2.633297in}}%
\pgfpathcurveto{\pgfqpoint{4.028651in}{2.630385in}}{\pgfqpoint{4.027015in}{2.626435in}}{\pgfqpoint{4.027015in}{2.622316in}}%
\pgfpathcurveto{\pgfqpoint{4.027015in}{2.618198in}}{\pgfqpoint{4.028651in}{2.614248in}}{\pgfqpoint{4.031563in}{2.611336in}}%
\pgfpathcurveto{\pgfqpoint{4.034475in}{2.608424in}}{\pgfqpoint{4.038425in}{2.606788in}}{\pgfqpoint{4.042543in}{2.606788in}}%
\pgfpathclose%
\pgfusepath{fill}%
\end{pgfscope}%
\begin{pgfscope}%
\pgfpathrectangle{\pgfqpoint{0.887500in}{0.275000in}}{\pgfqpoint{4.225000in}{4.225000in}}%
\pgfusepath{clip}%
\pgfsetbuttcap%
\pgfsetroundjoin%
\definecolor{currentfill}{rgb}{0.000000,0.000000,0.000000}%
\pgfsetfillcolor{currentfill}%
\pgfsetfillopacity{0.800000}%
\pgfsetlinewidth{0.000000pt}%
\definecolor{currentstroke}{rgb}{0.000000,0.000000,0.000000}%
\pgfsetstrokecolor{currentstroke}%
\pgfsetstrokeopacity{0.800000}%
\pgfsetdash{}{0pt}%
\pgfpathmoveto{\pgfqpoint{2.475343in}{2.303387in}}%
\pgfpathcurveto{\pgfqpoint{2.479461in}{2.303387in}}{\pgfqpoint{2.483411in}{2.305024in}}{\pgfqpoint{2.486323in}{2.307936in}}%
\pgfpathcurveto{\pgfqpoint{2.489235in}{2.310847in}}{\pgfqpoint{2.490871in}{2.314798in}}{\pgfqpoint{2.490871in}{2.318916in}}%
\pgfpathcurveto{\pgfqpoint{2.490871in}{2.323034in}}{\pgfqpoint{2.489235in}{2.326984in}}{\pgfqpoint{2.486323in}{2.329896in}}%
\pgfpathcurveto{\pgfqpoint{2.483411in}{2.332808in}}{\pgfqpoint{2.479461in}{2.334444in}}{\pgfqpoint{2.475343in}{2.334444in}}%
\pgfpathcurveto{\pgfqpoint{2.471225in}{2.334444in}}{\pgfqpoint{2.467275in}{2.332808in}}{\pgfqpoint{2.464363in}{2.329896in}}%
\pgfpathcurveto{\pgfqpoint{2.461451in}{2.326984in}}{\pgfqpoint{2.459815in}{2.323034in}}{\pgfqpoint{2.459815in}{2.318916in}}%
\pgfpathcurveto{\pgfqpoint{2.459815in}{2.314798in}}{\pgfqpoint{2.461451in}{2.310847in}}{\pgfqpoint{2.464363in}{2.307936in}}%
\pgfpathcurveto{\pgfqpoint{2.467275in}{2.305024in}}{\pgfqpoint{2.471225in}{2.303387in}}{\pgfqpoint{2.475343in}{2.303387in}}%
\pgfpathclose%
\pgfusepath{fill}%
\end{pgfscope}%
\begin{pgfscope}%
\pgfpathrectangle{\pgfqpoint{0.887500in}{0.275000in}}{\pgfqpoint{4.225000in}{4.225000in}}%
\pgfusepath{clip}%
\pgfsetbuttcap%
\pgfsetroundjoin%
\definecolor{currentfill}{rgb}{0.000000,0.000000,0.000000}%
\pgfsetfillcolor{currentfill}%
\pgfsetfillopacity{0.800000}%
\pgfsetlinewidth{0.000000pt}%
\definecolor{currentstroke}{rgb}{0.000000,0.000000,0.000000}%
\pgfsetstrokecolor{currentstroke}%
\pgfsetstrokeopacity{0.800000}%
\pgfsetdash{}{0pt}%
\pgfpathmoveto{\pgfqpoint{3.864869in}{2.718792in}}%
\pgfpathcurveto{\pgfqpoint{3.868987in}{2.718792in}}{\pgfqpoint{3.872937in}{2.720428in}}{\pgfqpoint{3.875849in}{2.723340in}}%
\pgfpathcurveto{\pgfqpoint{3.878761in}{2.726252in}}{\pgfqpoint{3.880397in}{2.730202in}}{\pgfqpoint{3.880397in}{2.734320in}}%
\pgfpathcurveto{\pgfqpoint{3.880397in}{2.738438in}}{\pgfqpoint{3.878761in}{2.742388in}}{\pgfqpoint{3.875849in}{2.745300in}}%
\pgfpathcurveto{\pgfqpoint{3.872937in}{2.748212in}}{\pgfqpoint{3.868987in}{2.749849in}}{\pgfqpoint{3.864869in}{2.749849in}}%
\pgfpathcurveto{\pgfqpoint{3.860751in}{2.749849in}}{\pgfqpoint{3.856801in}{2.748212in}}{\pgfqpoint{3.853889in}{2.745300in}}%
\pgfpathcurveto{\pgfqpoint{3.850977in}{2.742388in}}{\pgfqpoint{3.849341in}{2.738438in}}{\pgfqpoint{3.849341in}{2.734320in}}%
\pgfpathcurveto{\pgfqpoint{3.849341in}{2.730202in}}{\pgfqpoint{3.850977in}{2.726252in}}{\pgfqpoint{3.853889in}{2.723340in}}%
\pgfpathcurveto{\pgfqpoint{3.856801in}{2.720428in}}{\pgfqpoint{3.860751in}{2.718792in}}{\pgfqpoint{3.864869in}{2.718792in}}%
\pgfpathclose%
\pgfusepath{fill}%
\end{pgfscope}%
\begin{pgfscope}%
\pgfpathrectangle{\pgfqpoint{0.887500in}{0.275000in}}{\pgfqpoint{4.225000in}{4.225000in}}%
\pgfusepath{clip}%
\pgfsetbuttcap%
\pgfsetroundjoin%
\definecolor{currentfill}{rgb}{0.000000,0.000000,0.000000}%
\pgfsetfillcolor{currentfill}%
\pgfsetfillopacity{0.800000}%
\pgfsetlinewidth{0.000000pt}%
\definecolor{currentstroke}{rgb}{0.000000,0.000000,0.000000}%
\pgfsetstrokecolor{currentstroke}%
\pgfsetstrokeopacity{0.800000}%
\pgfsetdash{}{0pt}%
\pgfpathmoveto{\pgfqpoint{4.642447in}{2.112621in}}%
\pgfpathcurveto{\pgfqpoint{4.646565in}{2.112621in}}{\pgfqpoint{4.650515in}{2.114257in}}{\pgfqpoint{4.653427in}{2.117169in}}%
\pgfpathcurveto{\pgfqpoint{4.656339in}{2.120081in}}{\pgfqpoint{4.657975in}{2.124031in}}{\pgfqpoint{4.657975in}{2.128149in}}%
\pgfpathcurveto{\pgfqpoint{4.657975in}{2.132267in}}{\pgfqpoint{4.656339in}{2.136217in}}{\pgfqpoint{4.653427in}{2.139129in}}%
\pgfpathcurveto{\pgfqpoint{4.650515in}{2.142041in}}{\pgfqpoint{4.646565in}{2.143677in}}{\pgfqpoint{4.642447in}{2.143677in}}%
\pgfpathcurveto{\pgfqpoint{4.638329in}{2.143677in}}{\pgfqpoint{4.634379in}{2.142041in}}{\pgfqpoint{4.631467in}{2.139129in}}%
\pgfpathcurveto{\pgfqpoint{4.628555in}{2.136217in}}{\pgfqpoint{4.626919in}{2.132267in}}{\pgfqpoint{4.626919in}{2.128149in}}%
\pgfpathcurveto{\pgfqpoint{4.626919in}{2.124031in}}{\pgfqpoint{4.628555in}{2.120081in}}{\pgfqpoint{4.631467in}{2.117169in}}%
\pgfpathcurveto{\pgfqpoint{4.634379in}{2.114257in}}{\pgfqpoint{4.638329in}{2.112621in}}{\pgfqpoint{4.642447in}{2.112621in}}%
\pgfpathclose%
\pgfusepath{fill}%
\end{pgfscope}%
\begin{pgfscope}%
\pgfpathrectangle{\pgfqpoint{0.887500in}{0.275000in}}{\pgfqpoint{4.225000in}{4.225000in}}%
\pgfusepath{clip}%
\pgfsetbuttcap%
\pgfsetroundjoin%
\definecolor{currentfill}{rgb}{0.000000,0.000000,0.000000}%
\pgfsetfillcolor{currentfill}%
\pgfsetfillopacity{0.800000}%
\pgfsetlinewidth{0.000000pt}%
\definecolor{currentstroke}{rgb}{0.000000,0.000000,0.000000}%
\pgfsetstrokecolor{currentstroke}%
\pgfsetstrokeopacity{0.800000}%
\pgfsetdash{}{0pt}%
\pgfpathmoveto{\pgfqpoint{2.005973in}{2.381787in}}%
\pgfpathcurveto{\pgfqpoint{2.010092in}{2.381787in}}{\pgfqpoint{2.014042in}{2.383423in}}{\pgfqpoint{2.016954in}{2.386335in}}%
\pgfpathcurveto{\pgfqpoint{2.019865in}{2.389247in}}{\pgfqpoint{2.021502in}{2.393197in}}{\pgfqpoint{2.021502in}{2.397316in}}%
\pgfpathcurveto{\pgfqpoint{2.021502in}{2.401434in}}{\pgfqpoint{2.019865in}{2.405384in}}{\pgfqpoint{2.016954in}{2.408296in}}%
\pgfpathcurveto{\pgfqpoint{2.014042in}{2.411208in}}{\pgfqpoint{2.010092in}{2.412844in}}{\pgfqpoint{2.005973in}{2.412844in}}%
\pgfpathcurveto{\pgfqpoint{2.001855in}{2.412844in}}{\pgfqpoint{1.997905in}{2.411208in}}{\pgfqpoint{1.994993in}{2.408296in}}%
\pgfpathcurveto{\pgfqpoint{1.992081in}{2.405384in}}{\pgfqpoint{1.990445in}{2.401434in}}{\pgfqpoint{1.990445in}{2.397316in}}%
\pgfpathcurveto{\pgfqpoint{1.990445in}{2.393197in}}{\pgfqpoint{1.992081in}{2.389247in}}{\pgfqpoint{1.994993in}{2.386335in}}%
\pgfpathcurveto{\pgfqpoint{1.997905in}{2.383423in}}{\pgfqpoint{2.001855in}{2.381787in}}{\pgfqpoint{2.005973in}{2.381787in}}%
\pgfpathclose%
\pgfusepath{fill}%
\end{pgfscope}%
\begin{pgfscope}%
\pgfpathrectangle{\pgfqpoint{0.887500in}{0.275000in}}{\pgfqpoint{4.225000in}{4.225000in}}%
\pgfusepath{clip}%
\pgfsetbuttcap%
\pgfsetroundjoin%
\definecolor{currentfill}{rgb}{0.000000,0.000000,0.000000}%
\pgfsetfillcolor{currentfill}%
\pgfsetfillopacity{0.800000}%
\pgfsetlinewidth{0.000000pt}%
\definecolor{currentstroke}{rgb}{0.000000,0.000000,0.000000}%
\pgfsetstrokecolor{currentstroke}%
\pgfsetstrokeopacity{0.800000}%
\pgfsetdash{}{0pt}%
\pgfpathmoveto{\pgfqpoint{3.152555in}{3.110923in}}%
\pgfpathcurveto{\pgfqpoint{3.156673in}{3.110923in}}{\pgfqpoint{3.160623in}{3.112559in}}{\pgfqpoint{3.163535in}{3.115471in}}%
\pgfpathcurveto{\pgfqpoint{3.166447in}{3.118383in}}{\pgfqpoint{3.168083in}{3.122333in}}{\pgfqpoint{3.168083in}{3.126452in}}%
\pgfpathcurveto{\pgfqpoint{3.168083in}{3.130570in}}{\pgfqpoint{3.166447in}{3.134520in}}{\pgfqpoint{3.163535in}{3.137432in}}%
\pgfpathcurveto{\pgfqpoint{3.160623in}{3.140344in}}{\pgfqpoint{3.156673in}{3.141980in}}{\pgfqpoint{3.152555in}{3.141980in}}%
\pgfpathcurveto{\pgfqpoint{3.148436in}{3.141980in}}{\pgfqpoint{3.144486in}{3.140344in}}{\pgfqpoint{3.141574in}{3.137432in}}%
\pgfpathcurveto{\pgfqpoint{3.138662in}{3.134520in}}{\pgfqpoint{3.137026in}{3.130570in}}{\pgfqpoint{3.137026in}{3.126452in}}%
\pgfpathcurveto{\pgfqpoint{3.137026in}{3.122333in}}{\pgfqpoint{3.138662in}{3.118383in}}{\pgfqpoint{3.141574in}{3.115471in}}%
\pgfpathcurveto{\pgfqpoint{3.144486in}{3.112559in}}{\pgfqpoint{3.148436in}{3.110923in}}{\pgfqpoint{3.152555in}{3.110923in}}%
\pgfpathclose%
\pgfusepath{fill}%
\end{pgfscope}%
\begin{pgfscope}%
\pgfpathrectangle{\pgfqpoint{0.887500in}{0.275000in}}{\pgfqpoint{4.225000in}{4.225000in}}%
\pgfusepath{clip}%
\pgfsetbuttcap%
\pgfsetroundjoin%
\definecolor{currentfill}{rgb}{0.000000,0.000000,0.000000}%
\pgfsetfillcolor{currentfill}%
\pgfsetfillopacity{0.800000}%
\pgfsetlinewidth{0.000000pt}%
\definecolor{currentstroke}{rgb}{0.000000,0.000000,0.000000}%
\pgfsetstrokecolor{currentstroke}%
\pgfsetstrokeopacity{0.800000}%
\pgfsetdash{}{0pt}%
\pgfpathmoveto{\pgfqpoint{3.686993in}{2.827503in}}%
\pgfpathcurveto{\pgfqpoint{3.691112in}{2.827503in}}{\pgfqpoint{3.695062in}{2.829139in}}{\pgfqpoint{3.697974in}{2.832051in}}%
\pgfpathcurveto{\pgfqpoint{3.700886in}{2.834963in}}{\pgfqpoint{3.702522in}{2.838913in}}{\pgfqpoint{3.702522in}{2.843031in}}%
\pgfpathcurveto{\pgfqpoint{3.702522in}{2.847150in}}{\pgfqpoint{3.700886in}{2.851100in}}{\pgfqpoint{3.697974in}{2.854012in}}%
\pgfpathcurveto{\pgfqpoint{3.695062in}{2.856924in}}{\pgfqpoint{3.691112in}{2.858560in}}{\pgfqpoint{3.686993in}{2.858560in}}%
\pgfpathcurveto{\pgfqpoint{3.682875in}{2.858560in}}{\pgfqpoint{3.678925in}{2.856924in}}{\pgfqpoint{3.676013in}{2.854012in}}%
\pgfpathcurveto{\pgfqpoint{3.673101in}{2.851100in}}{\pgfqpoint{3.671465in}{2.847150in}}{\pgfqpoint{3.671465in}{2.843031in}}%
\pgfpathcurveto{\pgfqpoint{3.671465in}{2.838913in}}{\pgfqpoint{3.673101in}{2.834963in}}{\pgfqpoint{3.676013in}{2.832051in}}%
\pgfpathcurveto{\pgfqpoint{3.678925in}{2.829139in}}{\pgfqpoint{3.682875in}{2.827503in}}{\pgfqpoint{3.686993in}{2.827503in}}%
\pgfpathclose%
\pgfusepath{fill}%
\end{pgfscope}%
\begin{pgfscope}%
\pgfpathrectangle{\pgfqpoint{0.887500in}{0.275000in}}{\pgfqpoint{4.225000in}{4.225000in}}%
\pgfusepath{clip}%
\pgfsetbuttcap%
\pgfsetroundjoin%
\definecolor{currentfill}{rgb}{0.000000,0.000000,0.000000}%
\pgfsetfillcolor{currentfill}%
\pgfsetfillopacity{0.800000}%
\pgfsetlinewidth{0.000000pt}%
\definecolor{currentstroke}{rgb}{0.000000,0.000000,0.000000}%
\pgfsetstrokecolor{currentstroke}%
\pgfsetstrokeopacity{0.800000}%
\pgfsetdash{}{0pt}%
\pgfpathmoveto{\pgfqpoint{1.536372in}{2.455183in}}%
\pgfpathcurveto{\pgfqpoint{1.540491in}{2.455183in}}{\pgfqpoint{1.544441in}{2.456819in}}{\pgfqpoint{1.547353in}{2.459731in}}%
\pgfpathcurveto{\pgfqpoint{1.550265in}{2.462643in}}{\pgfqpoint{1.551901in}{2.466593in}}{\pgfqpoint{1.551901in}{2.470711in}}%
\pgfpathcurveto{\pgfqpoint{1.551901in}{2.474829in}}{\pgfqpoint{1.550265in}{2.478779in}}{\pgfqpoint{1.547353in}{2.481691in}}%
\pgfpathcurveto{\pgfqpoint{1.544441in}{2.484603in}}{\pgfqpoint{1.540491in}{2.486239in}}{\pgfqpoint{1.536372in}{2.486239in}}%
\pgfpathcurveto{\pgfqpoint{1.532254in}{2.486239in}}{\pgfqpoint{1.528304in}{2.484603in}}{\pgfqpoint{1.525392in}{2.481691in}}%
\pgfpathcurveto{\pgfqpoint{1.522480in}{2.478779in}}{\pgfqpoint{1.520844in}{2.474829in}}{\pgfqpoint{1.520844in}{2.470711in}}%
\pgfpathcurveto{\pgfqpoint{1.520844in}{2.466593in}}{\pgfqpoint{1.522480in}{2.462643in}}{\pgfqpoint{1.525392in}{2.459731in}}%
\pgfpathcurveto{\pgfqpoint{1.528304in}{2.456819in}}{\pgfqpoint{1.532254in}{2.455183in}}{\pgfqpoint{1.536372in}{2.455183in}}%
\pgfpathclose%
\pgfusepath{fill}%
\end{pgfscope}%
\begin{pgfscope}%
\pgfpathrectangle{\pgfqpoint{0.887500in}{0.275000in}}{\pgfqpoint{4.225000in}{4.225000in}}%
\pgfusepath{clip}%
\pgfsetbuttcap%
\pgfsetroundjoin%
\definecolor{currentfill}{rgb}{0.000000,0.000000,0.000000}%
\pgfsetfillcolor{currentfill}%
\pgfsetfillopacity{0.800000}%
\pgfsetlinewidth{0.000000pt}%
\definecolor{currentstroke}{rgb}{0.000000,0.000000,0.000000}%
\pgfsetstrokecolor{currentstroke}%
\pgfsetstrokeopacity{0.800000}%
\pgfsetdash{}{0pt}%
\pgfpathmoveto{\pgfqpoint{2.652977in}{2.245351in}}%
\pgfpathcurveto{\pgfqpoint{2.657095in}{2.245351in}}{\pgfqpoint{2.661045in}{2.246987in}}{\pgfqpoint{2.663957in}{2.249899in}}%
\pgfpathcurveto{\pgfqpoint{2.666869in}{2.252811in}}{\pgfqpoint{2.668505in}{2.256761in}}{\pgfqpoint{2.668505in}{2.260879in}}%
\pgfpathcurveto{\pgfqpoint{2.668505in}{2.264997in}}{\pgfqpoint{2.666869in}{2.268947in}}{\pgfqpoint{2.663957in}{2.271859in}}%
\pgfpathcurveto{\pgfqpoint{2.661045in}{2.274771in}}{\pgfqpoint{2.657095in}{2.276407in}}{\pgfqpoint{2.652977in}{2.276407in}}%
\pgfpathcurveto{\pgfqpoint{2.648859in}{2.276407in}}{\pgfqpoint{2.644909in}{2.274771in}}{\pgfqpoint{2.641997in}{2.271859in}}%
\pgfpathcurveto{\pgfqpoint{2.639085in}{2.268947in}}{\pgfqpoint{2.637449in}{2.264997in}}{\pgfqpoint{2.637449in}{2.260879in}}%
\pgfpathcurveto{\pgfqpoint{2.637449in}{2.256761in}}{\pgfqpoint{2.639085in}{2.252811in}}{\pgfqpoint{2.641997in}{2.249899in}}%
\pgfpathcurveto{\pgfqpoint{2.644909in}{2.246987in}}{\pgfqpoint{2.648859in}{2.245351in}}{\pgfqpoint{2.652977in}{2.245351in}}%
\pgfpathclose%
\pgfusepath{fill}%
\end{pgfscope}%
\begin{pgfscope}%
\pgfpathrectangle{\pgfqpoint{0.887500in}{0.275000in}}{\pgfqpoint{4.225000in}{4.225000in}}%
\pgfusepath{clip}%
\pgfsetbuttcap%
\pgfsetroundjoin%
\definecolor{currentfill}{rgb}{0.000000,0.000000,0.000000}%
\pgfsetfillcolor{currentfill}%
\pgfsetfillopacity{0.800000}%
\pgfsetlinewidth{0.000000pt}%
\definecolor{currentstroke}{rgb}{0.000000,0.000000,0.000000}%
\pgfsetstrokecolor{currentstroke}%
\pgfsetstrokeopacity{0.800000}%
\pgfsetdash{}{0pt}%
\pgfpathmoveto{\pgfqpoint{4.465114in}{2.229259in}}%
\pgfpathcurveto{\pgfqpoint{4.469233in}{2.229259in}}{\pgfqpoint{4.473183in}{2.230895in}}{\pgfqpoint{4.476095in}{2.233807in}}%
\pgfpathcurveto{\pgfqpoint{4.479006in}{2.236719in}}{\pgfqpoint{4.480643in}{2.240669in}}{\pgfqpoint{4.480643in}{2.244787in}}%
\pgfpathcurveto{\pgfqpoint{4.480643in}{2.248905in}}{\pgfqpoint{4.479006in}{2.252855in}}{\pgfqpoint{4.476095in}{2.255767in}}%
\pgfpathcurveto{\pgfqpoint{4.473183in}{2.258679in}}{\pgfqpoint{4.469233in}{2.260315in}}{\pgfqpoint{4.465114in}{2.260315in}}%
\pgfpathcurveto{\pgfqpoint{4.460996in}{2.260315in}}{\pgfqpoint{4.457046in}{2.258679in}}{\pgfqpoint{4.454134in}{2.255767in}}%
\pgfpathcurveto{\pgfqpoint{4.451222in}{2.252855in}}{\pgfqpoint{4.449586in}{2.248905in}}{\pgfqpoint{4.449586in}{2.244787in}}%
\pgfpathcurveto{\pgfqpoint{4.449586in}{2.240669in}}{\pgfqpoint{4.451222in}{2.236719in}}{\pgfqpoint{4.454134in}{2.233807in}}%
\pgfpathcurveto{\pgfqpoint{4.457046in}{2.230895in}}{\pgfqpoint{4.460996in}{2.229259in}}{\pgfqpoint{4.465114in}{2.229259in}}%
\pgfpathclose%
\pgfusepath{fill}%
\end{pgfscope}%
\begin{pgfscope}%
\pgfpathrectangle{\pgfqpoint{0.887500in}{0.275000in}}{\pgfqpoint{4.225000in}{4.225000in}}%
\pgfusepath{clip}%
\pgfsetbuttcap%
\pgfsetroundjoin%
\definecolor{currentfill}{rgb}{0.000000,0.000000,0.000000}%
\pgfsetfillcolor{currentfill}%
\pgfsetfillopacity{0.800000}%
\pgfsetlinewidth{0.000000pt}%
\definecolor{currentstroke}{rgb}{0.000000,0.000000,0.000000}%
\pgfsetstrokecolor{currentstroke}%
\pgfsetstrokeopacity{0.800000}%
\pgfsetdash{}{0pt}%
\pgfpathmoveto{\pgfqpoint{3.508960in}{2.932804in}}%
\pgfpathcurveto{\pgfqpoint{3.513079in}{2.932804in}}{\pgfqpoint{3.517029in}{2.934440in}}{\pgfqpoint{3.519941in}{2.937352in}}%
\pgfpathcurveto{\pgfqpoint{3.522853in}{2.940264in}}{\pgfqpoint{3.524489in}{2.944214in}}{\pgfqpoint{3.524489in}{2.948332in}}%
\pgfpathcurveto{\pgfqpoint{3.524489in}{2.952450in}}{\pgfqpoint{3.522853in}{2.956400in}}{\pgfqpoint{3.519941in}{2.959312in}}%
\pgfpathcurveto{\pgfqpoint{3.517029in}{2.962224in}}{\pgfqpoint{3.513079in}{2.963860in}}{\pgfqpoint{3.508960in}{2.963860in}}%
\pgfpathcurveto{\pgfqpoint{3.504842in}{2.963860in}}{\pgfqpoint{3.500892in}{2.962224in}}{\pgfqpoint{3.497980in}{2.959312in}}%
\pgfpathcurveto{\pgfqpoint{3.495068in}{2.956400in}}{\pgfqpoint{3.493432in}{2.952450in}}{\pgfqpoint{3.493432in}{2.948332in}}%
\pgfpathcurveto{\pgfqpoint{3.493432in}{2.944214in}}{\pgfqpoint{3.495068in}{2.940264in}}{\pgfqpoint{3.497980in}{2.937352in}}%
\pgfpathcurveto{\pgfqpoint{3.500892in}{2.934440in}}{\pgfqpoint{3.504842in}{2.932804in}}{\pgfqpoint{3.508960in}{2.932804in}}%
\pgfpathclose%
\pgfusepath{fill}%
\end{pgfscope}%
\begin{pgfscope}%
\pgfpathrectangle{\pgfqpoint{0.887500in}{0.275000in}}{\pgfqpoint{4.225000in}{4.225000in}}%
\pgfusepath{clip}%
\pgfsetbuttcap%
\pgfsetroundjoin%
\definecolor{currentfill}{rgb}{0.000000,0.000000,0.000000}%
\pgfsetfillcolor{currentfill}%
\pgfsetfillopacity{0.800000}%
\pgfsetlinewidth{0.000000pt}%
\definecolor{currentstroke}{rgb}{0.000000,0.000000,0.000000}%
\pgfsetstrokecolor{currentstroke}%
\pgfsetstrokeopacity{0.800000}%
\pgfsetdash{}{0pt}%
\pgfpathmoveto{\pgfqpoint{3.039618in}{2.993942in}}%
\pgfpathcurveto{\pgfqpoint{3.043737in}{2.993942in}}{\pgfqpoint{3.047687in}{2.995578in}}{\pgfqpoint{3.050598in}{2.998490in}}%
\pgfpathcurveto{\pgfqpoint{3.053510in}{3.001402in}}{\pgfqpoint{3.055147in}{3.005352in}}{\pgfqpoint{3.055147in}{3.009470in}}%
\pgfpathcurveto{\pgfqpoint{3.055147in}{3.013588in}}{\pgfqpoint{3.053510in}{3.017539in}}{\pgfqpoint{3.050598in}{3.020450in}}%
\pgfpathcurveto{\pgfqpoint{3.047687in}{3.023362in}}{\pgfqpoint{3.043737in}{3.024999in}}{\pgfqpoint{3.039618in}{3.024999in}}%
\pgfpathcurveto{\pgfqpoint{3.035500in}{3.024999in}}{\pgfqpoint{3.031550in}{3.023362in}}{\pgfqpoint{3.028638in}{3.020450in}}%
\pgfpathcurveto{\pgfqpoint{3.025726in}{3.017539in}}{\pgfqpoint{3.024090in}{3.013588in}}{\pgfqpoint{3.024090in}{3.009470in}}%
\pgfpathcurveto{\pgfqpoint{3.024090in}{3.005352in}}{\pgfqpoint{3.025726in}{3.001402in}}{\pgfqpoint{3.028638in}{2.998490in}}%
\pgfpathcurveto{\pgfqpoint{3.031550in}{2.995578in}}{\pgfqpoint{3.035500in}{2.993942in}}{\pgfqpoint{3.039618in}{2.993942in}}%
\pgfpathclose%
\pgfusepath{fill}%
\end{pgfscope}%
\begin{pgfscope}%
\pgfpathrectangle{\pgfqpoint{0.887500in}{0.275000in}}{\pgfqpoint{4.225000in}{4.225000in}}%
\pgfusepath{clip}%
\pgfsetbuttcap%
\pgfsetroundjoin%
\definecolor{currentfill}{rgb}{0.000000,0.000000,0.000000}%
\pgfsetfillcolor{currentfill}%
\pgfsetfillopacity{0.800000}%
\pgfsetlinewidth{0.000000pt}%
\definecolor{currentstroke}{rgb}{0.000000,0.000000,0.000000}%
\pgfsetstrokecolor{currentstroke}%
\pgfsetstrokeopacity{0.800000}%
\pgfsetdash{}{0pt}%
\pgfpathmoveto{\pgfqpoint{3.330798in}{3.031957in}}%
\pgfpathcurveto{\pgfqpoint{3.334916in}{3.031957in}}{\pgfqpoint{3.338866in}{3.033593in}}{\pgfqpoint{3.341778in}{3.036505in}}%
\pgfpathcurveto{\pgfqpoint{3.344690in}{3.039417in}}{\pgfqpoint{3.346326in}{3.043367in}}{\pgfqpoint{3.346326in}{3.047486in}}%
\pgfpathcurveto{\pgfqpoint{3.346326in}{3.051604in}}{\pgfqpoint{3.344690in}{3.055554in}}{\pgfqpoint{3.341778in}{3.058466in}}%
\pgfpathcurveto{\pgfqpoint{3.338866in}{3.061378in}}{\pgfqpoint{3.334916in}{3.063014in}}{\pgfqpoint{3.330798in}{3.063014in}}%
\pgfpathcurveto{\pgfqpoint{3.326680in}{3.063014in}}{\pgfqpoint{3.322730in}{3.061378in}}{\pgfqpoint{3.319818in}{3.058466in}}%
\pgfpathcurveto{\pgfqpoint{3.316906in}{3.055554in}}{\pgfqpoint{3.315269in}{3.051604in}}{\pgfqpoint{3.315269in}{3.047486in}}%
\pgfpathcurveto{\pgfqpoint{3.315269in}{3.043367in}}{\pgfqpoint{3.316906in}{3.039417in}}{\pgfqpoint{3.319818in}{3.036505in}}%
\pgfpathcurveto{\pgfqpoint{3.322730in}{3.033593in}}{\pgfqpoint{3.326680in}{3.031957in}}{\pgfqpoint{3.330798in}{3.031957in}}%
\pgfpathclose%
\pgfusepath{fill}%
\end{pgfscope}%
\begin{pgfscope}%
\pgfpathrectangle{\pgfqpoint{0.887500in}{0.275000in}}{\pgfqpoint{4.225000in}{4.225000in}}%
\pgfusepath{clip}%
\pgfsetbuttcap%
\pgfsetroundjoin%
\definecolor{currentfill}{rgb}{0.000000,0.000000,0.000000}%
\pgfsetfillcolor{currentfill}%
\pgfsetfillopacity{0.800000}%
\pgfsetlinewidth{0.000000pt}%
\definecolor{currentstroke}{rgb}{0.000000,0.000000,0.000000}%
\pgfsetstrokecolor{currentstroke}%
\pgfsetstrokeopacity{0.800000}%
\pgfsetdash{}{0pt}%
\pgfpathmoveto{\pgfqpoint{4.287470in}{2.342160in}}%
\pgfpathcurveto{\pgfqpoint{4.291588in}{2.342160in}}{\pgfqpoint{4.295538in}{2.343796in}}{\pgfqpoint{4.298450in}{2.346708in}}%
\pgfpathcurveto{\pgfqpoint{4.301362in}{2.349620in}}{\pgfqpoint{4.302998in}{2.353570in}}{\pgfqpoint{4.302998in}{2.357688in}}%
\pgfpathcurveto{\pgfqpoint{4.302998in}{2.361806in}}{\pgfqpoint{4.301362in}{2.365756in}}{\pgfqpoint{4.298450in}{2.368668in}}%
\pgfpathcurveto{\pgfqpoint{4.295538in}{2.371580in}}{\pgfqpoint{4.291588in}{2.373217in}}{\pgfqpoint{4.287470in}{2.373217in}}%
\pgfpathcurveto{\pgfqpoint{4.283352in}{2.373217in}}{\pgfqpoint{4.279402in}{2.371580in}}{\pgfqpoint{4.276490in}{2.368668in}}%
\pgfpathcurveto{\pgfqpoint{4.273578in}{2.365756in}}{\pgfqpoint{4.271942in}{2.361806in}}{\pgfqpoint{4.271942in}{2.357688in}}%
\pgfpathcurveto{\pgfqpoint{4.271942in}{2.353570in}}{\pgfqpoint{4.273578in}{2.349620in}}{\pgfqpoint{4.276490in}{2.346708in}}%
\pgfpathcurveto{\pgfqpoint{4.279402in}{2.343796in}}{\pgfqpoint{4.283352in}{2.342160in}}{\pgfqpoint{4.287470in}{2.342160in}}%
\pgfpathclose%
\pgfusepath{fill}%
\end{pgfscope}%
\begin{pgfscope}%
\pgfpathrectangle{\pgfqpoint{0.887500in}{0.275000in}}{\pgfqpoint{4.225000in}{4.225000in}}%
\pgfusepath{clip}%
\pgfsetbuttcap%
\pgfsetroundjoin%
\definecolor{currentfill}{rgb}{0.000000,0.000000,0.000000}%
\pgfsetfillcolor{currentfill}%
\pgfsetfillopacity{0.800000}%
\pgfsetlinewidth{0.000000pt}%
\definecolor{currentstroke}{rgb}{0.000000,0.000000,0.000000}%
\pgfsetstrokecolor{currentstroke}%
\pgfsetstrokeopacity{0.800000}%
\pgfsetdash{}{0pt}%
\pgfpathmoveto{\pgfqpoint{2.183187in}{2.326287in}}%
\pgfpathcurveto{\pgfqpoint{2.187305in}{2.326287in}}{\pgfqpoint{2.191255in}{2.327923in}}{\pgfqpoint{2.194167in}{2.330835in}}%
\pgfpathcurveto{\pgfqpoint{2.197079in}{2.333747in}}{\pgfqpoint{2.198715in}{2.337697in}}{\pgfqpoint{2.198715in}{2.341815in}}%
\pgfpathcurveto{\pgfqpoint{2.198715in}{2.345933in}}{\pgfqpoint{2.197079in}{2.349883in}}{\pgfqpoint{2.194167in}{2.352795in}}%
\pgfpathcurveto{\pgfqpoint{2.191255in}{2.355707in}}{\pgfqpoint{2.187305in}{2.357343in}}{\pgfqpoint{2.183187in}{2.357343in}}%
\pgfpathcurveto{\pgfqpoint{2.179069in}{2.357343in}}{\pgfqpoint{2.175119in}{2.355707in}}{\pgfqpoint{2.172207in}{2.352795in}}%
\pgfpathcurveto{\pgfqpoint{2.169295in}{2.349883in}}{\pgfqpoint{2.167658in}{2.345933in}}{\pgfqpoint{2.167658in}{2.341815in}}%
\pgfpathcurveto{\pgfqpoint{2.167658in}{2.337697in}}{\pgfqpoint{2.169295in}{2.333747in}}{\pgfqpoint{2.172207in}{2.330835in}}%
\pgfpathcurveto{\pgfqpoint{2.175119in}{2.327923in}}{\pgfqpoint{2.179069in}{2.326287in}}{\pgfqpoint{2.183187in}{2.326287in}}%
\pgfpathclose%
\pgfusepath{fill}%
\end{pgfscope}%
\begin{pgfscope}%
\pgfpathrectangle{\pgfqpoint{0.887500in}{0.275000in}}{\pgfqpoint{4.225000in}{4.225000in}}%
\pgfusepath{clip}%
\pgfsetbuttcap%
\pgfsetroundjoin%
\definecolor{currentfill}{rgb}{0.000000,0.000000,0.000000}%
\pgfsetfillcolor{currentfill}%
\pgfsetfillopacity{0.800000}%
\pgfsetlinewidth{0.000000pt}%
\definecolor{currentstroke}{rgb}{0.000000,0.000000,0.000000}%
\pgfsetstrokecolor{currentstroke}%
\pgfsetstrokeopacity{0.800000}%
\pgfsetdash{}{0pt}%
\pgfpathmoveto{\pgfqpoint{2.830965in}{2.184296in}}%
\pgfpathcurveto{\pgfqpoint{2.835083in}{2.184296in}}{\pgfqpoint{2.839033in}{2.185932in}}{\pgfqpoint{2.841945in}{2.188844in}}%
\pgfpathcurveto{\pgfqpoint{2.844857in}{2.191756in}}{\pgfqpoint{2.846493in}{2.195706in}}{\pgfqpoint{2.846493in}{2.199824in}}%
\pgfpathcurveto{\pgfqpoint{2.846493in}{2.203942in}}{\pgfqpoint{2.844857in}{2.207892in}}{\pgfqpoint{2.841945in}{2.210804in}}%
\pgfpathcurveto{\pgfqpoint{2.839033in}{2.213716in}}{\pgfqpoint{2.835083in}{2.215352in}}{\pgfqpoint{2.830965in}{2.215352in}}%
\pgfpathcurveto{\pgfqpoint{2.826847in}{2.215352in}}{\pgfqpoint{2.822897in}{2.213716in}}{\pgfqpoint{2.819985in}{2.210804in}}%
\pgfpathcurveto{\pgfqpoint{2.817073in}{2.207892in}}{\pgfqpoint{2.815437in}{2.203942in}}{\pgfqpoint{2.815437in}{2.199824in}}%
\pgfpathcurveto{\pgfqpoint{2.815437in}{2.195706in}}{\pgfqpoint{2.817073in}{2.191756in}}{\pgfqpoint{2.819985in}{2.188844in}}%
\pgfpathcurveto{\pgfqpoint{2.822897in}{2.185932in}}{\pgfqpoint{2.826847in}{2.184296in}}{\pgfqpoint{2.830965in}{2.184296in}}%
\pgfpathclose%
\pgfusepath{fill}%
\end{pgfscope}%
\begin{pgfscope}%
\pgfpathrectangle{\pgfqpoint{0.887500in}{0.275000in}}{\pgfqpoint{4.225000in}{4.225000in}}%
\pgfusepath{clip}%
\pgfsetbuttcap%
\pgfsetroundjoin%
\definecolor{currentfill}{rgb}{0.000000,0.000000,0.000000}%
\pgfsetfillcolor{currentfill}%
\pgfsetfillopacity{0.800000}%
\pgfsetlinewidth{0.000000pt}%
\definecolor{currentstroke}{rgb}{0.000000,0.000000,0.000000}%
\pgfsetstrokecolor{currentstroke}%
\pgfsetstrokeopacity{0.800000}%
\pgfsetdash{}{0pt}%
\pgfpathmoveto{\pgfqpoint{1.713062in}{2.402834in}}%
\pgfpathcurveto{\pgfqpoint{1.717180in}{2.402834in}}{\pgfqpoint{1.721130in}{2.404470in}}{\pgfqpoint{1.724042in}{2.407382in}}%
\pgfpathcurveto{\pgfqpoint{1.726954in}{2.410294in}}{\pgfqpoint{1.728591in}{2.414244in}}{\pgfqpoint{1.728591in}{2.418362in}}%
\pgfpathcurveto{\pgfqpoint{1.728591in}{2.422480in}}{\pgfqpoint{1.726954in}{2.426430in}}{\pgfqpoint{1.724042in}{2.429342in}}%
\pgfpathcurveto{\pgfqpoint{1.721130in}{2.432254in}}{\pgfqpoint{1.717180in}{2.433891in}}{\pgfqpoint{1.713062in}{2.433891in}}%
\pgfpathcurveto{\pgfqpoint{1.708944in}{2.433891in}}{\pgfqpoint{1.704994in}{2.432254in}}{\pgfqpoint{1.702082in}{2.429342in}}%
\pgfpathcurveto{\pgfqpoint{1.699170in}{2.426430in}}{\pgfqpoint{1.697534in}{2.422480in}}{\pgfqpoint{1.697534in}{2.418362in}}%
\pgfpathcurveto{\pgfqpoint{1.697534in}{2.414244in}}{\pgfqpoint{1.699170in}{2.410294in}}{\pgfqpoint{1.702082in}{2.407382in}}%
\pgfpathcurveto{\pgfqpoint{1.704994in}{2.404470in}}{\pgfqpoint{1.708944in}{2.402834in}}{\pgfqpoint{1.713062in}{2.402834in}}%
\pgfpathclose%
\pgfusepath{fill}%
\end{pgfscope}%
\begin{pgfscope}%
\pgfpathrectangle{\pgfqpoint{0.887500in}{0.275000in}}{\pgfqpoint{4.225000in}{4.225000in}}%
\pgfusepath{clip}%
\pgfsetbuttcap%
\pgfsetroundjoin%
\definecolor{currentfill}{rgb}{0.000000,0.000000,0.000000}%
\pgfsetfillcolor{currentfill}%
\pgfsetfillopacity{0.800000}%
\pgfsetlinewidth{0.000000pt}%
\definecolor{currentstroke}{rgb}{0.000000,0.000000,0.000000}%
\pgfsetstrokecolor{currentstroke}%
\pgfsetstrokeopacity{0.800000}%
\pgfsetdash{}{0pt}%
\pgfpathmoveto{\pgfqpoint{4.109779in}{2.459295in}}%
\pgfpathcurveto{\pgfqpoint{4.113897in}{2.459295in}}{\pgfqpoint{4.117847in}{2.460931in}}{\pgfqpoint{4.120759in}{2.463843in}}%
\pgfpathcurveto{\pgfqpoint{4.123671in}{2.466755in}}{\pgfqpoint{4.125307in}{2.470705in}}{\pgfqpoint{4.125307in}{2.474823in}}%
\pgfpathcurveto{\pgfqpoint{4.125307in}{2.478941in}}{\pgfqpoint{4.123671in}{2.482891in}}{\pgfqpoint{4.120759in}{2.485803in}}%
\pgfpathcurveto{\pgfqpoint{4.117847in}{2.488715in}}{\pgfqpoint{4.113897in}{2.490351in}}{\pgfqpoint{4.109779in}{2.490351in}}%
\pgfpathcurveto{\pgfqpoint{4.105661in}{2.490351in}}{\pgfqpoint{4.101711in}{2.488715in}}{\pgfqpoint{4.098799in}{2.485803in}}%
\pgfpathcurveto{\pgfqpoint{4.095887in}{2.482891in}}{\pgfqpoint{4.094251in}{2.478941in}}{\pgfqpoint{4.094251in}{2.474823in}}%
\pgfpathcurveto{\pgfqpoint{4.094251in}{2.470705in}}{\pgfqpoint{4.095887in}{2.466755in}}{\pgfqpoint{4.098799in}{2.463843in}}%
\pgfpathcurveto{\pgfqpoint{4.101711in}{2.460931in}}{\pgfqpoint{4.105661in}{2.459295in}}{\pgfqpoint{4.109779in}{2.459295in}}%
\pgfpathclose%
\pgfusepath{fill}%
\end{pgfscope}%
\begin{pgfscope}%
\pgfpathrectangle{\pgfqpoint{0.887500in}{0.275000in}}{\pgfqpoint{4.225000in}{4.225000in}}%
\pgfusepath{clip}%
\pgfsetbuttcap%
\pgfsetroundjoin%
\definecolor{currentfill}{rgb}{0.000000,0.000000,0.000000}%
\pgfsetfillcolor{currentfill}%
\pgfsetfillopacity{0.800000}%
\pgfsetlinewidth{0.000000pt}%
\definecolor{currentstroke}{rgb}{0.000000,0.000000,0.000000}%
\pgfsetstrokecolor{currentstroke}%
\pgfsetstrokeopacity{0.800000}%
\pgfsetdash{}{0pt}%
\pgfpathmoveto{\pgfqpoint{4.710743in}{1.960338in}}%
\pgfpathcurveto{\pgfqpoint{4.714862in}{1.960338in}}{\pgfqpoint{4.718812in}{1.961974in}}{\pgfqpoint{4.721724in}{1.964886in}}%
\pgfpathcurveto{\pgfqpoint{4.724636in}{1.967798in}}{\pgfqpoint{4.726272in}{1.971748in}}{\pgfqpoint{4.726272in}{1.975867in}}%
\pgfpathcurveto{\pgfqpoint{4.726272in}{1.979985in}}{\pgfqpoint{4.724636in}{1.983935in}}{\pgfqpoint{4.721724in}{1.986847in}}%
\pgfpathcurveto{\pgfqpoint{4.718812in}{1.989759in}}{\pgfqpoint{4.714862in}{1.991395in}}{\pgfqpoint{4.710743in}{1.991395in}}%
\pgfpathcurveto{\pgfqpoint{4.706625in}{1.991395in}}{\pgfqpoint{4.702675in}{1.989759in}}{\pgfqpoint{4.699763in}{1.986847in}}%
\pgfpathcurveto{\pgfqpoint{4.696851in}{1.983935in}}{\pgfqpoint{4.695215in}{1.979985in}}{\pgfqpoint{4.695215in}{1.975867in}}%
\pgfpathcurveto{\pgfqpoint{4.695215in}{1.971748in}}{\pgfqpoint{4.696851in}{1.967798in}}{\pgfqpoint{4.699763in}{1.964886in}}%
\pgfpathcurveto{\pgfqpoint{4.702675in}{1.961974in}}{\pgfqpoint{4.706625in}{1.960338in}}{\pgfqpoint{4.710743in}{1.960338in}}%
\pgfpathclose%
\pgfusepath{fill}%
\end{pgfscope}%
\begin{pgfscope}%
\pgfpathrectangle{\pgfqpoint{0.887500in}{0.275000in}}{\pgfqpoint{4.225000in}{4.225000in}}%
\pgfusepath{clip}%
\pgfsetbuttcap%
\pgfsetroundjoin%
\definecolor{currentfill}{rgb}{0.000000,0.000000,0.000000}%
\pgfsetfillcolor{currentfill}%
\pgfsetfillopacity{0.800000}%
\pgfsetlinewidth{0.000000pt}%
\definecolor{currentstroke}{rgb}{0.000000,0.000000,0.000000}%
\pgfsetstrokecolor{currentstroke}%
\pgfsetstrokeopacity{0.800000}%
\pgfsetdash{}{0pt}%
\pgfpathmoveto{\pgfqpoint{3.931823in}{2.573360in}}%
\pgfpathcurveto{\pgfqpoint{3.935941in}{2.573360in}}{\pgfqpoint{3.939891in}{2.574996in}}{\pgfqpoint{3.942803in}{2.577908in}}%
\pgfpathcurveto{\pgfqpoint{3.945715in}{2.580820in}}{\pgfqpoint{3.947352in}{2.584770in}}{\pgfqpoint{3.947352in}{2.588888in}}%
\pgfpathcurveto{\pgfqpoint{3.947352in}{2.593006in}}{\pgfqpoint{3.945715in}{2.596956in}}{\pgfqpoint{3.942803in}{2.599868in}}%
\pgfpathcurveto{\pgfqpoint{3.939891in}{2.602780in}}{\pgfqpoint{3.935941in}{2.604416in}}{\pgfqpoint{3.931823in}{2.604416in}}%
\pgfpathcurveto{\pgfqpoint{3.927705in}{2.604416in}}{\pgfqpoint{3.923755in}{2.602780in}}{\pgfqpoint{3.920843in}{2.599868in}}%
\pgfpathcurveto{\pgfqpoint{3.917931in}{2.596956in}}{\pgfqpoint{3.916295in}{2.593006in}}{\pgfqpoint{3.916295in}{2.588888in}}%
\pgfpathcurveto{\pgfqpoint{3.916295in}{2.584770in}}{\pgfqpoint{3.917931in}{2.580820in}}{\pgfqpoint{3.920843in}{2.577908in}}%
\pgfpathcurveto{\pgfqpoint{3.923755in}{2.574996in}}{\pgfqpoint{3.927705in}{2.573360in}}{\pgfqpoint{3.931823in}{2.573360in}}%
\pgfpathclose%
\pgfusepath{fill}%
\end{pgfscope}%
\begin{pgfscope}%
\pgfpathrectangle{\pgfqpoint{0.887500in}{0.275000in}}{\pgfqpoint{4.225000in}{4.225000in}}%
\pgfusepath{clip}%
\pgfsetbuttcap%
\pgfsetroundjoin%
\definecolor{currentfill}{rgb}{0.000000,0.000000,0.000000}%
\pgfsetfillcolor{currentfill}%
\pgfsetfillopacity{0.800000}%
\pgfsetlinewidth{0.000000pt}%
\definecolor{currentstroke}{rgb}{0.000000,0.000000,0.000000}%
\pgfsetstrokecolor{currentstroke}%
\pgfsetstrokeopacity{0.800000}%
\pgfsetdash{}{0pt}%
\pgfpathmoveto{\pgfqpoint{2.360778in}{2.269644in}}%
\pgfpathcurveto{\pgfqpoint{2.364896in}{2.269644in}}{\pgfqpoint{2.368846in}{2.271280in}}{\pgfqpoint{2.371758in}{2.274192in}}%
\pgfpathcurveto{\pgfqpoint{2.374670in}{2.277104in}}{\pgfqpoint{2.376306in}{2.281054in}}{\pgfqpoint{2.376306in}{2.285172in}}%
\pgfpathcurveto{\pgfqpoint{2.376306in}{2.289291in}}{\pgfqpoint{2.374670in}{2.293241in}}{\pgfqpoint{2.371758in}{2.296153in}}%
\pgfpathcurveto{\pgfqpoint{2.368846in}{2.299065in}}{\pgfqpoint{2.364896in}{2.300701in}}{\pgfqpoint{2.360778in}{2.300701in}}%
\pgfpathcurveto{\pgfqpoint{2.356660in}{2.300701in}}{\pgfqpoint{2.352710in}{2.299065in}}{\pgfqpoint{2.349798in}{2.296153in}}%
\pgfpathcurveto{\pgfqpoint{2.346886in}{2.293241in}}{\pgfqpoint{2.345250in}{2.289291in}}{\pgfqpoint{2.345250in}{2.285172in}}%
\pgfpathcurveto{\pgfqpoint{2.345250in}{2.281054in}}{\pgfqpoint{2.346886in}{2.277104in}}{\pgfqpoint{2.349798in}{2.274192in}}%
\pgfpathcurveto{\pgfqpoint{2.352710in}{2.271280in}}{\pgfqpoint{2.356660in}{2.269644in}}{\pgfqpoint{2.360778in}{2.269644in}}%
\pgfpathclose%
\pgfusepath{fill}%
\end{pgfscope}%
\begin{pgfscope}%
\pgfpathrectangle{\pgfqpoint{0.887500in}{0.275000in}}{\pgfqpoint{4.225000in}{4.225000in}}%
\pgfusepath{clip}%
\pgfsetbuttcap%
\pgfsetroundjoin%
\definecolor{currentfill}{rgb}{0.000000,0.000000,0.000000}%
\pgfsetfillcolor{currentfill}%
\pgfsetfillopacity{0.800000}%
\pgfsetlinewidth{0.000000pt}%
\definecolor{currentstroke}{rgb}{0.000000,0.000000,0.000000}%
\pgfsetstrokecolor{currentstroke}%
\pgfsetstrokeopacity{0.800000}%
\pgfsetdash{}{0pt}%
\pgfpathmoveto{\pgfqpoint{3.009270in}{2.120143in}}%
\pgfpathcurveto{\pgfqpoint{3.013388in}{2.120143in}}{\pgfqpoint{3.017338in}{2.121779in}}{\pgfqpoint{3.020250in}{2.124691in}}%
\pgfpathcurveto{\pgfqpoint{3.023162in}{2.127603in}}{\pgfqpoint{3.024798in}{2.131553in}}{\pgfqpoint{3.024798in}{2.135671in}}%
\pgfpathcurveto{\pgfqpoint{3.024798in}{2.139790in}}{\pgfqpoint{3.023162in}{2.143740in}}{\pgfqpoint{3.020250in}{2.146652in}}%
\pgfpathcurveto{\pgfqpoint{3.017338in}{2.149564in}}{\pgfqpoint{3.013388in}{2.151200in}}{\pgfqpoint{3.009270in}{2.151200in}}%
\pgfpathcurveto{\pgfqpoint{3.005151in}{2.151200in}}{\pgfqpoint{3.001201in}{2.149564in}}{\pgfqpoint{2.998289in}{2.146652in}}%
\pgfpathcurveto{\pgfqpoint{2.995377in}{2.143740in}}{\pgfqpoint{2.993741in}{2.139790in}}{\pgfqpoint{2.993741in}{2.135671in}}%
\pgfpathcurveto{\pgfqpoint{2.993741in}{2.131553in}}{\pgfqpoint{2.995377in}{2.127603in}}{\pgfqpoint{2.998289in}{2.124691in}}%
\pgfpathcurveto{\pgfqpoint{3.001201in}{2.121779in}}{\pgfqpoint{3.005151in}{2.120143in}}{\pgfqpoint{3.009270in}{2.120143in}}%
\pgfpathclose%
\pgfusepath{fill}%
\end{pgfscope}%
\begin{pgfscope}%
\pgfpathrectangle{\pgfqpoint{0.887500in}{0.275000in}}{\pgfqpoint{4.225000in}{4.225000in}}%
\pgfusepath{clip}%
\pgfsetbuttcap%
\pgfsetroundjoin%
\definecolor{currentfill}{rgb}{0.000000,0.000000,0.000000}%
\pgfsetfillcolor{currentfill}%
\pgfsetfillopacity{0.800000}%
\pgfsetlinewidth{0.000000pt}%
\definecolor{currentstroke}{rgb}{0.000000,0.000000,0.000000}%
\pgfsetstrokecolor{currentstroke}%
\pgfsetstrokeopacity{0.800000}%
\pgfsetdash{}{0pt}%
\pgfpathmoveto{\pgfqpoint{3.753656in}{2.684930in}}%
\pgfpathcurveto{\pgfqpoint{3.757774in}{2.684930in}}{\pgfqpoint{3.761724in}{2.686566in}}{\pgfqpoint{3.764636in}{2.689478in}}%
\pgfpathcurveto{\pgfqpoint{3.767548in}{2.692390in}}{\pgfqpoint{3.769184in}{2.696340in}}{\pgfqpoint{3.769184in}{2.700458in}}%
\pgfpathcurveto{\pgfqpoint{3.769184in}{2.704577in}}{\pgfqpoint{3.767548in}{2.708527in}}{\pgfqpoint{3.764636in}{2.711439in}}%
\pgfpathcurveto{\pgfqpoint{3.761724in}{2.714351in}}{\pgfqpoint{3.757774in}{2.715987in}}{\pgfqpoint{3.753656in}{2.715987in}}%
\pgfpathcurveto{\pgfqpoint{3.749538in}{2.715987in}}{\pgfqpoint{3.745588in}{2.714351in}}{\pgfqpoint{3.742676in}{2.711439in}}%
\pgfpathcurveto{\pgfqpoint{3.739764in}{2.708527in}}{\pgfqpoint{3.738128in}{2.704577in}}{\pgfqpoint{3.738128in}{2.700458in}}%
\pgfpathcurveto{\pgfqpoint{3.738128in}{2.696340in}}{\pgfqpoint{3.739764in}{2.692390in}}{\pgfqpoint{3.742676in}{2.689478in}}%
\pgfpathcurveto{\pgfqpoint{3.745588in}{2.686566in}}{\pgfqpoint{3.749538in}{2.684930in}}{\pgfqpoint{3.753656in}{2.684930in}}%
\pgfpathclose%
\pgfusepath{fill}%
\end{pgfscope}%
\begin{pgfscope}%
\pgfpathrectangle{\pgfqpoint{0.887500in}{0.275000in}}{\pgfqpoint{4.225000in}{4.225000in}}%
\pgfusepath{clip}%
\pgfsetbuttcap%
\pgfsetroundjoin%
\definecolor{currentfill}{rgb}{0.000000,0.000000,0.000000}%
\pgfsetfillcolor{currentfill}%
\pgfsetfillopacity{0.800000}%
\pgfsetlinewidth{0.000000pt}%
\definecolor{currentstroke}{rgb}{0.000000,0.000000,0.000000}%
\pgfsetstrokecolor{currentstroke}%
\pgfsetstrokeopacity{0.800000}%
\pgfsetdash{}{0pt}%
\pgfpathmoveto{\pgfqpoint{4.533191in}{2.079163in}}%
\pgfpathcurveto{\pgfqpoint{4.537309in}{2.079163in}}{\pgfqpoint{4.541259in}{2.080799in}}{\pgfqpoint{4.544171in}{2.083711in}}%
\pgfpathcurveto{\pgfqpoint{4.547083in}{2.086623in}}{\pgfqpoint{4.548719in}{2.090573in}}{\pgfqpoint{4.548719in}{2.094691in}}%
\pgfpathcurveto{\pgfqpoint{4.548719in}{2.098809in}}{\pgfqpoint{4.547083in}{2.102759in}}{\pgfqpoint{4.544171in}{2.105671in}}%
\pgfpathcurveto{\pgfqpoint{4.541259in}{2.108583in}}{\pgfqpoint{4.537309in}{2.110219in}}{\pgfqpoint{4.533191in}{2.110219in}}%
\pgfpathcurveto{\pgfqpoint{4.529072in}{2.110219in}}{\pgfqpoint{4.525122in}{2.108583in}}{\pgfqpoint{4.522210in}{2.105671in}}%
\pgfpathcurveto{\pgfqpoint{4.519298in}{2.102759in}}{\pgfqpoint{4.517662in}{2.098809in}}{\pgfqpoint{4.517662in}{2.094691in}}%
\pgfpathcurveto{\pgfqpoint{4.517662in}{2.090573in}}{\pgfqpoint{4.519298in}{2.086623in}}{\pgfqpoint{4.522210in}{2.083711in}}%
\pgfpathcurveto{\pgfqpoint{4.525122in}{2.080799in}}{\pgfqpoint{4.529072in}{2.079163in}}{\pgfqpoint{4.533191in}{2.079163in}}%
\pgfpathclose%
\pgfusepath{fill}%
\end{pgfscope}%
\begin{pgfscope}%
\pgfpathrectangle{\pgfqpoint{0.887500in}{0.275000in}}{\pgfqpoint{4.225000in}{4.225000in}}%
\pgfusepath{clip}%
\pgfsetbuttcap%
\pgfsetroundjoin%
\definecolor{currentfill}{rgb}{0.000000,0.000000,0.000000}%
\pgfsetfillcolor{currentfill}%
\pgfsetfillopacity{0.800000}%
\pgfsetlinewidth{0.000000pt}%
\definecolor{currentstroke}{rgb}{0.000000,0.000000,0.000000}%
\pgfsetstrokecolor{currentstroke}%
\pgfsetstrokeopacity{0.800000}%
\pgfsetdash{}{0pt}%
\pgfpathmoveto{\pgfqpoint{1.890199in}{2.348307in}}%
\pgfpathcurveto{\pgfqpoint{1.894318in}{2.348307in}}{\pgfqpoint{1.898268in}{2.349944in}}{\pgfqpoint{1.901180in}{2.352856in}}%
\pgfpathcurveto{\pgfqpoint{1.904092in}{2.355767in}}{\pgfqpoint{1.905728in}{2.359718in}}{\pgfqpoint{1.905728in}{2.363836in}}%
\pgfpathcurveto{\pgfqpoint{1.905728in}{2.367954in}}{\pgfqpoint{1.904092in}{2.371904in}}{\pgfqpoint{1.901180in}{2.374816in}}%
\pgfpathcurveto{\pgfqpoint{1.898268in}{2.377728in}}{\pgfqpoint{1.894318in}{2.379364in}}{\pgfqpoint{1.890199in}{2.379364in}}%
\pgfpathcurveto{\pgfqpoint{1.886081in}{2.379364in}}{\pgfqpoint{1.882131in}{2.377728in}}{\pgfqpoint{1.879219in}{2.374816in}}%
\pgfpathcurveto{\pgfqpoint{1.876307in}{2.371904in}}{\pgfqpoint{1.874671in}{2.367954in}}{\pgfqpoint{1.874671in}{2.363836in}}%
\pgfpathcurveto{\pgfqpoint{1.874671in}{2.359718in}}{\pgfqpoint{1.876307in}{2.355767in}}{\pgfqpoint{1.879219in}{2.352856in}}%
\pgfpathcurveto{\pgfqpoint{1.882131in}{2.349944in}}{\pgfqpoint{1.886081in}{2.348307in}}{\pgfqpoint{1.890199in}{2.348307in}}%
\pgfpathclose%
\pgfusepath{fill}%
\end{pgfscope}%
\begin{pgfscope}%
\pgfpathrectangle{\pgfqpoint{0.887500in}{0.275000in}}{\pgfqpoint{4.225000in}{4.225000in}}%
\pgfusepath{clip}%
\pgfsetbuttcap%
\pgfsetroundjoin%
\definecolor{currentfill}{rgb}{0.000000,0.000000,0.000000}%
\pgfsetfillcolor{currentfill}%
\pgfsetfillopacity{0.800000}%
\pgfsetlinewidth{0.000000pt}%
\definecolor{currentstroke}{rgb}{0.000000,0.000000,0.000000}%
\pgfsetstrokecolor{currentstroke}%
\pgfsetstrokeopacity{0.800000}%
\pgfsetdash{}{0pt}%
\pgfpathmoveto{\pgfqpoint{3.057095in}{2.493251in}}%
\pgfpathcurveto{\pgfqpoint{3.061213in}{2.493251in}}{\pgfqpoint{3.065163in}{2.494887in}}{\pgfqpoint{3.068075in}{2.497799in}}%
\pgfpathcurveto{\pgfqpoint{3.070987in}{2.500711in}}{\pgfqpoint{3.072623in}{2.504661in}}{\pgfqpoint{3.072623in}{2.508779in}}%
\pgfpathcurveto{\pgfqpoint{3.072623in}{2.512898in}}{\pgfqpoint{3.070987in}{2.516848in}}{\pgfqpoint{3.068075in}{2.519760in}}%
\pgfpathcurveto{\pgfqpoint{3.065163in}{2.522672in}}{\pgfqpoint{3.061213in}{2.524308in}}{\pgfqpoint{3.057095in}{2.524308in}}%
\pgfpathcurveto{\pgfqpoint{3.052976in}{2.524308in}}{\pgfqpoint{3.049026in}{2.522672in}}{\pgfqpoint{3.046114in}{2.519760in}}%
\pgfpathcurveto{\pgfqpoint{3.043202in}{2.516848in}}{\pgfqpoint{3.041566in}{2.512898in}}{\pgfqpoint{3.041566in}{2.508779in}}%
\pgfpathcurveto{\pgfqpoint{3.041566in}{2.504661in}}{\pgfqpoint{3.043202in}{2.500711in}}{\pgfqpoint{3.046114in}{2.497799in}}%
\pgfpathcurveto{\pgfqpoint{3.049026in}{2.494887in}}{\pgfqpoint{3.052976in}{2.493251in}}{\pgfqpoint{3.057095in}{2.493251in}}%
\pgfpathclose%
\pgfusepath{fill}%
\end{pgfscope}%
\begin{pgfscope}%
\pgfpathrectangle{\pgfqpoint{0.887500in}{0.275000in}}{\pgfqpoint{4.225000in}{4.225000in}}%
\pgfusepath{clip}%
\pgfsetbuttcap%
\pgfsetroundjoin%
\definecolor{currentfill}{rgb}{0.000000,0.000000,0.000000}%
\pgfsetfillcolor{currentfill}%
\pgfsetfillopacity{0.800000}%
\pgfsetlinewidth{0.000000pt}%
\definecolor{currentstroke}{rgb}{0.000000,0.000000,0.000000}%
\pgfsetstrokecolor{currentstroke}%
\pgfsetstrokeopacity{0.800000}%
\pgfsetdash{}{0pt}%
\pgfpathmoveto{\pgfqpoint{3.575296in}{2.792677in}}%
\pgfpathcurveto{\pgfqpoint{3.579414in}{2.792677in}}{\pgfqpoint{3.583364in}{2.794313in}}{\pgfqpoint{3.586276in}{2.797225in}}%
\pgfpathcurveto{\pgfqpoint{3.589188in}{2.800137in}}{\pgfqpoint{3.590824in}{2.804087in}}{\pgfqpoint{3.590824in}{2.808205in}}%
\pgfpathcurveto{\pgfqpoint{3.590824in}{2.812323in}}{\pgfqpoint{3.589188in}{2.816273in}}{\pgfqpoint{3.586276in}{2.819185in}}%
\pgfpathcurveto{\pgfqpoint{3.583364in}{2.822097in}}{\pgfqpoint{3.579414in}{2.823733in}}{\pgfqpoint{3.575296in}{2.823733in}}%
\pgfpathcurveto{\pgfqpoint{3.571177in}{2.823733in}}{\pgfqpoint{3.567227in}{2.822097in}}{\pgfqpoint{3.564315in}{2.819185in}}%
\pgfpathcurveto{\pgfqpoint{3.561403in}{2.816273in}}{\pgfqpoint{3.559767in}{2.812323in}}{\pgfqpoint{3.559767in}{2.808205in}}%
\pgfpathcurveto{\pgfqpoint{3.559767in}{2.804087in}}{\pgfqpoint{3.561403in}{2.800137in}}{\pgfqpoint{3.564315in}{2.797225in}}%
\pgfpathcurveto{\pgfqpoint{3.567227in}{2.794313in}}{\pgfqpoint{3.571177in}{2.792677in}}{\pgfqpoint{3.575296in}{2.792677in}}%
\pgfpathclose%
\pgfusepath{fill}%
\end{pgfscope}%
\begin{pgfscope}%
\pgfpathrectangle{\pgfqpoint{0.887500in}{0.275000in}}{\pgfqpoint{4.225000in}{4.225000in}}%
\pgfusepath{clip}%
\pgfsetbuttcap%
\pgfsetroundjoin%
\definecolor{currentfill}{rgb}{0.000000,0.000000,0.000000}%
\pgfsetfillcolor{currentfill}%
\pgfsetfillopacity{0.800000}%
\pgfsetlinewidth{0.000000pt}%
\definecolor{currentstroke}{rgb}{0.000000,0.000000,0.000000}%
\pgfsetstrokecolor{currentstroke}%
\pgfsetstrokeopacity{0.800000}%
\pgfsetdash{}{0pt}%
\pgfpathmoveto{\pgfqpoint{3.218164in}{2.986860in}}%
\pgfpathcurveto{\pgfqpoint{3.222282in}{2.986860in}}{\pgfqpoint{3.226232in}{2.988496in}}{\pgfqpoint{3.229144in}{2.991408in}}%
\pgfpathcurveto{\pgfqpoint{3.232056in}{2.994320in}}{\pgfqpoint{3.233692in}{2.998270in}}{\pgfqpoint{3.233692in}{3.002388in}}%
\pgfpathcurveto{\pgfqpoint{3.233692in}{3.006506in}}{\pgfqpoint{3.232056in}{3.010456in}}{\pgfqpoint{3.229144in}{3.013368in}}%
\pgfpathcurveto{\pgfqpoint{3.226232in}{3.016280in}}{\pgfqpoint{3.222282in}{3.017916in}}{\pgfqpoint{3.218164in}{3.017916in}}%
\pgfpathcurveto{\pgfqpoint{3.214046in}{3.017916in}}{\pgfqpoint{3.210096in}{3.016280in}}{\pgfqpoint{3.207184in}{3.013368in}}%
\pgfpathcurveto{\pgfqpoint{3.204272in}{3.010456in}}{\pgfqpoint{3.202635in}{3.006506in}}{\pgfqpoint{3.202635in}{3.002388in}}%
\pgfpathcurveto{\pgfqpoint{3.202635in}{2.998270in}}{\pgfqpoint{3.204272in}{2.994320in}}{\pgfqpoint{3.207184in}{2.991408in}}%
\pgfpathcurveto{\pgfqpoint{3.210096in}{2.988496in}}{\pgfqpoint{3.214046in}{2.986860in}}{\pgfqpoint{3.218164in}{2.986860in}}%
\pgfpathclose%
\pgfusepath{fill}%
\end{pgfscope}%
\begin{pgfscope}%
\pgfpathrectangle{\pgfqpoint{0.887500in}{0.275000in}}{\pgfqpoint{4.225000in}{4.225000in}}%
\pgfusepath{clip}%
\pgfsetbuttcap%
\pgfsetroundjoin%
\definecolor{currentfill}{rgb}{0.000000,0.000000,0.000000}%
\pgfsetfillcolor{currentfill}%
\pgfsetfillopacity{0.800000}%
\pgfsetlinewidth{0.000000pt}%
\definecolor{currentstroke}{rgb}{0.000000,0.000000,0.000000}%
\pgfsetstrokecolor{currentstroke}%
\pgfsetstrokeopacity{0.800000}%
\pgfsetdash{}{0pt}%
\pgfpathmoveto{\pgfqpoint{4.355048in}{2.186650in}}%
\pgfpathcurveto{\pgfqpoint{4.359166in}{2.186650in}}{\pgfqpoint{4.363116in}{2.188286in}}{\pgfqpoint{4.366028in}{2.191198in}}%
\pgfpathcurveto{\pgfqpoint{4.368940in}{2.194110in}}{\pgfqpoint{4.370576in}{2.198060in}}{\pgfqpoint{4.370576in}{2.202178in}}%
\pgfpathcurveto{\pgfqpoint{4.370576in}{2.206296in}}{\pgfqpoint{4.368940in}{2.210246in}}{\pgfqpoint{4.366028in}{2.213158in}}%
\pgfpathcurveto{\pgfqpoint{4.363116in}{2.216070in}}{\pgfqpoint{4.359166in}{2.217706in}}{\pgfqpoint{4.355048in}{2.217706in}}%
\pgfpathcurveto{\pgfqpoint{4.350930in}{2.217706in}}{\pgfqpoint{4.346980in}{2.216070in}}{\pgfqpoint{4.344068in}{2.213158in}}%
\pgfpathcurveto{\pgfqpoint{4.341156in}{2.210246in}}{\pgfqpoint{4.339520in}{2.206296in}}{\pgfqpoint{4.339520in}{2.202178in}}%
\pgfpathcurveto{\pgfqpoint{4.339520in}{2.198060in}}{\pgfqpoint{4.341156in}{2.194110in}}{\pgfqpoint{4.344068in}{2.191198in}}%
\pgfpathcurveto{\pgfqpoint{4.346980in}{2.188286in}}{\pgfqpoint{4.350930in}{2.186650in}}{\pgfqpoint{4.355048in}{2.186650in}}%
\pgfpathclose%
\pgfusepath{fill}%
\end{pgfscope}%
\begin{pgfscope}%
\pgfpathrectangle{\pgfqpoint{0.887500in}{0.275000in}}{\pgfqpoint{4.225000in}{4.225000in}}%
\pgfusepath{clip}%
\pgfsetbuttcap%
\pgfsetroundjoin%
\definecolor{currentfill}{rgb}{0.000000,0.000000,0.000000}%
\pgfsetfillcolor{currentfill}%
\pgfsetfillopacity{0.800000}%
\pgfsetlinewidth{0.000000pt}%
\definecolor{currentstroke}{rgb}{0.000000,0.000000,0.000000}%
\pgfsetstrokecolor{currentstroke}%
\pgfsetstrokeopacity{0.800000}%
\pgfsetdash{}{0pt}%
\pgfpathmoveto{\pgfqpoint{3.396777in}{2.894560in}}%
\pgfpathcurveto{\pgfqpoint{3.400895in}{2.894560in}}{\pgfqpoint{3.404845in}{2.896196in}}{\pgfqpoint{3.407757in}{2.899108in}}%
\pgfpathcurveto{\pgfqpoint{3.410669in}{2.902020in}}{\pgfqpoint{3.412305in}{2.905970in}}{\pgfqpoint{3.412305in}{2.910088in}}%
\pgfpathcurveto{\pgfqpoint{3.412305in}{2.914206in}}{\pgfqpoint{3.410669in}{2.918156in}}{\pgfqpoint{3.407757in}{2.921068in}}%
\pgfpathcurveto{\pgfqpoint{3.404845in}{2.923980in}}{\pgfqpoint{3.400895in}{2.925616in}}{\pgfqpoint{3.396777in}{2.925616in}}%
\pgfpathcurveto{\pgfqpoint{3.392658in}{2.925616in}}{\pgfqpoint{3.388708in}{2.923980in}}{\pgfqpoint{3.385796in}{2.921068in}}%
\pgfpathcurveto{\pgfqpoint{3.382884in}{2.918156in}}{\pgfqpoint{3.381248in}{2.914206in}}{\pgfqpoint{3.381248in}{2.910088in}}%
\pgfpathcurveto{\pgfqpoint{3.381248in}{2.905970in}}{\pgfqpoint{3.382884in}{2.902020in}}{\pgfqpoint{3.385796in}{2.899108in}}%
\pgfpathcurveto{\pgfqpoint{3.388708in}{2.896196in}}{\pgfqpoint{3.392658in}{2.894560in}}{\pgfqpoint{3.396777in}{2.894560in}}%
\pgfpathclose%
\pgfusepath{fill}%
\end{pgfscope}%
\begin{pgfscope}%
\pgfpathrectangle{\pgfqpoint{0.887500in}{0.275000in}}{\pgfqpoint{4.225000in}{4.225000in}}%
\pgfusepath{clip}%
\pgfsetbuttcap%
\pgfsetroundjoin%
\definecolor{currentfill}{rgb}{0.000000,0.000000,0.000000}%
\pgfsetfillcolor{currentfill}%
\pgfsetfillopacity{0.800000}%
\pgfsetlinewidth{0.000000pt}%
\definecolor{currentstroke}{rgb}{0.000000,0.000000,0.000000}%
\pgfsetstrokecolor{currentstroke}%
\pgfsetstrokeopacity{0.800000}%
\pgfsetdash{}{0pt}%
\pgfpathmoveto{\pgfqpoint{3.104956in}{2.864572in}}%
\pgfpathcurveto{\pgfqpoint{3.109074in}{2.864572in}}{\pgfqpoint{3.113024in}{2.866208in}}{\pgfqpoint{3.115936in}{2.869120in}}%
\pgfpathcurveto{\pgfqpoint{3.118848in}{2.872032in}}{\pgfqpoint{3.120484in}{2.875982in}}{\pgfqpoint{3.120484in}{2.880100in}}%
\pgfpathcurveto{\pgfqpoint{3.120484in}{2.884219in}}{\pgfqpoint{3.118848in}{2.888169in}}{\pgfqpoint{3.115936in}{2.891081in}}%
\pgfpathcurveto{\pgfqpoint{3.113024in}{2.893992in}}{\pgfqpoint{3.109074in}{2.895629in}}{\pgfqpoint{3.104956in}{2.895629in}}%
\pgfpathcurveto{\pgfqpoint{3.100838in}{2.895629in}}{\pgfqpoint{3.096888in}{2.893992in}}{\pgfqpoint{3.093976in}{2.891081in}}%
\pgfpathcurveto{\pgfqpoint{3.091064in}{2.888169in}}{\pgfqpoint{3.089428in}{2.884219in}}{\pgfqpoint{3.089428in}{2.880100in}}%
\pgfpathcurveto{\pgfqpoint{3.089428in}{2.875982in}}{\pgfqpoint{3.091064in}{2.872032in}}{\pgfqpoint{3.093976in}{2.869120in}}%
\pgfpathcurveto{\pgfqpoint{3.096888in}{2.866208in}}{\pgfqpoint{3.100838in}{2.864572in}}{\pgfqpoint{3.104956in}{2.864572in}}%
\pgfpathclose%
\pgfusepath{fill}%
\end{pgfscope}%
\begin{pgfscope}%
\pgfpathrectangle{\pgfqpoint{0.887500in}{0.275000in}}{\pgfqpoint{4.225000in}{4.225000in}}%
\pgfusepath{clip}%
\pgfsetbuttcap%
\pgfsetroundjoin%
\definecolor{currentfill}{rgb}{0.000000,0.000000,0.000000}%
\pgfsetfillcolor{currentfill}%
\pgfsetfillopacity{0.800000}%
\pgfsetlinewidth{0.000000pt}%
\definecolor{currentstroke}{rgb}{0.000000,0.000000,0.000000}%
\pgfsetstrokecolor{currentstroke}%
\pgfsetstrokeopacity{0.800000}%
\pgfsetdash{}{0pt}%
\pgfpathmoveto{\pgfqpoint{2.538741in}{2.211347in}}%
\pgfpathcurveto{\pgfqpoint{2.542859in}{2.211347in}}{\pgfqpoint{2.546809in}{2.212983in}}{\pgfqpoint{2.549721in}{2.215895in}}%
\pgfpathcurveto{\pgfqpoint{2.552633in}{2.218807in}}{\pgfqpoint{2.554269in}{2.222757in}}{\pgfqpoint{2.554269in}{2.226875in}}%
\pgfpathcurveto{\pgfqpoint{2.554269in}{2.230993in}}{\pgfqpoint{2.552633in}{2.234943in}}{\pgfqpoint{2.549721in}{2.237855in}}%
\pgfpathcurveto{\pgfqpoint{2.546809in}{2.240767in}}{\pgfqpoint{2.542859in}{2.242403in}}{\pgfqpoint{2.538741in}{2.242403in}}%
\pgfpathcurveto{\pgfqpoint{2.534622in}{2.242403in}}{\pgfqpoint{2.530672in}{2.240767in}}{\pgfqpoint{2.527760in}{2.237855in}}%
\pgfpathcurveto{\pgfqpoint{2.524848in}{2.234943in}}{\pgfqpoint{2.523212in}{2.230993in}}{\pgfqpoint{2.523212in}{2.226875in}}%
\pgfpathcurveto{\pgfqpoint{2.523212in}{2.222757in}}{\pgfqpoint{2.524848in}{2.218807in}}{\pgfqpoint{2.527760in}{2.215895in}}%
\pgfpathcurveto{\pgfqpoint{2.530672in}{2.212983in}}{\pgfqpoint{2.534622in}{2.211347in}}{\pgfqpoint{2.538741in}{2.211347in}}%
\pgfpathclose%
\pgfusepath{fill}%
\end{pgfscope}%
\begin{pgfscope}%
\pgfpathrectangle{\pgfqpoint{0.887500in}{0.275000in}}{\pgfqpoint{4.225000in}{4.225000in}}%
\pgfusepath{clip}%
\pgfsetbuttcap%
\pgfsetroundjoin%
\definecolor{currentfill}{rgb}{0.000000,0.000000,0.000000}%
\pgfsetfillcolor{currentfill}%
\pgfsetfillopacity{0.800000}%
\pgfsetlinewidth{0.000000pt}%
\definecolor{currentstroke}{rgb}{0.000000,0.000000,0.000000}%
\pgfsetstrokecolor{currentstroke}%
\pgfsetstrokeopacity{0.800000}%
\pgfsetdash{}{0pt}%
\pgfpathmoveto{\pgfqpoint{4.177219in}{2.308981in}}%
\pgfpathcurveto{\pgfqpoint{4.181337in}{2.308981in}}{\pgfqpoint{4.185287in}{2.310617in}}{\pgfqpoint{4.188199in}{2.313529in}}%
\pgfpathcurveto{\pgfqpoint{4.191111in}{2.316441in}}{\pgfqpoint{4.192747in}{2.320391in}}{\pgfqpoint{4.192747in}{2.324509in}}%
\pgfpathcurveto{\pgfqpoint{4.192747in}{2.328627in}}{\pgfqpoint{4.191111in}{2.332577in}}{\pgfqpoint{4.188199in}{2.335489in}}%
\pgfpathcurveto{\pgfqpoint{4.185287in}{2.338401in}}{\pgfqpoint{4.181337in}{2.340037in}}{\pgfqpoint{4.177219in}{2.340037in}}%
\pgfpathcurveto{\pgfqpoint{4.173100in}{2.340037in}}{\pgfqpoint{4.169150in}{2.338401in}}{\pgfqpoint{4.166238in}{2.335489in}}%
\pgfpathcurveto{\pgfqpoint{4.163326in}{2.332577in}}{\pgfqpoint{4.161690in}{2.328627in}}{\pgfqpoint{4.161690in}{2.324509in}}%
\pgfpathcurveto{\pgfqpoint{4.161690in}{2.320391in}}{\pgfqpoint{4.163326in}{2.316441in}}{\pgfqpoint{4.166238in}{2.313529in}}%
\pgfpathcurveto{\pgfqpoint{4.169150in}{2.310617in}}{\pgfqpoint{4.173100in}{2.308981in}}{\pgfqpoint{4.177219in}{2.308981in}}%
\pgfpathclose%
\pgfusepath{fill}%
\end{pgfscope}%
\begin{pgfscope}%
\pgfpathrectangle{\pgfqpoint{0.887500in}{0.275000in}}{\pgfqpoint{4.225000in}{4.225000in}}%
\pgfusepath{clip}%
\pgfsetbuttcap%
\pgfsetroundjoin%
\definecolor{currentfill}{rgb}{0.000000,0.000000,0.000000}%
\pgfsetfillcolor{currentfill}%
\pgfsetfillopacity{0.800000}%
\pgfsetlinewidth{0.000000pt}%
\definecolor{currentstroke}{rgb}{0.000000,0.000000,0.000000}%
\pgfsetstrokecolor{currentstroke}%
\pgfsetstrokeopacity{0.800000}%
\pgfsetdash{}{0pt}%
\pgfpathmoveto{\pgfqpoint{2.067736in}{2.292464in}}%
\pgfpathcurveto{\pgfqpoint{2.071854in}{2.292464in}}{\pgfqpoint{2.075804in}{2.294100in}}{\pgfqpoint{2.078716in}{2.297012in}}%
\pgfpathcurveto{\pgfqpoint{2.081628in}{2.299924in}}{\pgfqpoint{2.083264in}{2.303874in}}{\pgfqpoint{2.083264in}{2.307992in}}%
\pgfpathcurveto{\pgfqpoint{2.083264in}{2.312111in}}{\pgfqpoint{2.081628in}{2.316061in}}{\pgfqpoint{2.078716in}{2.318972in}}%
\pgfpathcurveto{\pgfqpoint{2.075804in}{2.321884in}}{\pgfqpoint{2.071854in}{2.323521in}}{\pgfqpoint{2.067736in}{2.323521in}}%
\pgfpathcurveto{\pgfqpoint{2.063618in}{2.323521in}}{\pgfqpoint{2.059668in}{2.321884in}}{\pgfqpoint{2.056756in}{2.318972in}}%
\pgfpathcurveto{\pgfqpoint{2.053844in}{2.316061in}}{\pgfqpoint{2.052208in}{2.312111in}}{\pgfqpoint{2.052208in}{2.307992in}}%
\pgfpathcurveto{\pgfqpoint{2.052208in}{2.303874in}}{\pgfqpoint{2.053844in}{2.299924in}}{\pgfqpoint{2.056756in}{2.297012in}}%
\pgfpathcurveto{\pgfqpoint{2.059668in}{2.294100in}}{\pgfqpoint{2.063618in}{2.292464in}}{\pgfqpoint{2.067736in}{2.292464in}}%
\pgfpathclose%
\pgfusepath{fill}%
\end{pgfscope}%
\begin{pgfscope}%
\pgfpathrectangle{\pgfqpoint{0.887500in}{0.275000in}}{\pgfqpoint{4.225000in}{4.225000in}}%
\pgfusepath{clip}%
\pgfsetbuttcap%
\pgfsetroundjoin%
\definecolor{currentfill}{rgb}{0.000000,0.000000,0.000000}%
\pgfsetfillcolor{currentfill}%
\pgfsetfillopacity{0.800000}%
\pgfsetlinewidth{0.000000pt}%
\definecolor{currentstroke}{rgb}{0.000000,0.000000,0.000000}%
\pgfsetstrokecolor{currentstroke}%
\pgfsetstrokeopacity{0.800000}%
\pgfsetdash{}{0pt}%
\pgfpathmoveto{\pgfqpoint{4.779234in}{1.806038in}}%
\pgfpathcurveto{\pgfqpoint{4.783352in}{1.806038in}}{\pgfqpoint{4.787302in}{1.807674in}}{\pgfqpoint{4.790214in}{1.810586in}}%
\pgfpathcurveto{\pgfqpoint{4.793126in}{1.813498in}}{\pgfqpoint{4.794762in}{1.817448in}}{\pgfqpoint{4.794762in}{1.821566in}}%
\pgfpathcurveto{\pgfqpoint{4.794762in}{1.825684in}}{\pgfqpoint{4.793126in}{1.829634in}}{\pgfqpoint{4.790214in}{1.832546in}}%
\pgfpathcurveto{\pgfqpoint{4.787302in}{1.835458in}}{\pgfqpoint{4.783352in}{1.837094in}}{\pgfqpoint{4.779234in}{1.837094in}}%
\pgfpathcurveto{\pgfqpoint{4.775116in}{1.837094in}}{\pgfqpoint{4.771166in}{1.835458in}}{\pgfqpoint{4.768254in}{1.832546in}}%
\pgfpathcurveto{\pgfqpoint{4.765342in}{1.829634in}}{\pgfqpoint{4.763706in}{1.825684in}}{\pgfqpoint{4.763706in}{1.821566in}}%
\pgfpathcurveto{\pgfqpoint{4.763706in}{1.817448in}}{\pgfqpoint{4.765342in}{1.813498in}}{\pgfqpoint{4.768254in}{1.810586in}}%
\pgfpathcurveto{\pgfqpoint{4.771166in}{1.807674in}}{\pgfqpoint{4.775116in}{1.806038in}}{\pgfqpoint{4.779234in}{1.806038in}}%
\pgfpathclose%
\pgfusepath{fill}%
\end{pgfscope}%
\begin{pgfscope}%
\pgfpathrectangle{\pgfqpoint{0.887500in}{0.275000in}}{\pgfqpoint{4.225000in}{4.225000in}}%
\pgfusepath{clip}%
\pgfsetbuttcap%
\pgfsetroundjoin%
\definecolor{currentfill}{rgb}{0.000000,0.000000,0.000000}%
\pgfsetfillcolor{currentfill}%
\pgfsetfillopacity{0.800000}%
\pgfsetlinewidth{0.000000pt}%
\definecolor{currentstroke}{rgb}{0.000000,0.000000,0.000000}%
\pgfsetstrokecolor{currentstroke}%
\pgfsetstrokeopacity{0.800000}%
\pgfsetdash{}{0pt}%
\pgfpathmoveto{\pgfqpoint{2.717061in}{2.150712in}}%
\pgfpathcurveto{\pgfqpoint{2.721179in}{2.150712in}}{\pgfqpoint{2.725129in}{2.152348in}}{\pgfqpoint{2.728041in}{2.155260in}}%
\pgfpathcurveto{\pgfqpoint{2.730953in}{2.158172in}}{\pgfqpoint{2.732589in}{2.162122in}}{\pgfqpoint{2.732589in}{2.166240in}}%
\pgfpathcurveto{\pgfqpoint{2.732589in}{2.170358in}}{\pgfqpoint{2.730953in}{2.174308in}}{\pgfqpoint{2.728041in}{2.177220in}}%
\pgfpathcurveto{\pgfqpoint{2.725129in}{2.180132in}}{\pgfqpoint{2.721179in}{2.181768in}}{\pgfqpoint{2.717061in}{2.181768in}}%
\pgfpathcurveto{\pgfqpoint{2.712943in}{2.181768in}}{\pgfqpoint{2.708993in}{2.180132in}}{\pgfqpoint{2.706081in}{2.177220in}}%
\pgfpathcurveto{\pgfqpoint{2.703169in}{2.174308in}}{\pgfqpoint{2.701533in}{2.170358in}}{\pgfqpoint{2.701533in}{2.166240in}}%
\pgfpathcurveto{\pgfqpoint{2.701533in}{2.162122in}}{\pgfqpoint{2.703169in}{2.158172in}}{\pgfqpoint{2.706081in}{2.155260in}}%
\pgfpathcurveto{\pgfqpoint{2.708993in}{2.152348in}}{\pgfqpoint{2.712943in}{2.150712in}}{\pgfqpoint{2.717061in}{2.150712in}}%
\pgfpathclose%
\pgfusepath{fill}%
\end{pgfscope}%
\begin{pgfscope}%
\pgfpathrectangle{\pgfqpoint{0.887500in}{0.275000in}}{\pgfqpoint{4.225000in}{4.225000in}}%
\pgfusepath{clip}%
\pgfsetbuttcap%
\pgfsetroundjoin%
\definecolor{currentfill}{rgb}{0.000000,0.000000,0.000000}%
\pgfsetfillcolor{currentfill}%
\pgfsetfillopacity{0.800000}%
\pgfsetlinewidth{0.000000pt}%
\definecolor{currentstroke}{rgb}{0.000000,0.000000,0.000000}%
\pgfsetstrokecolor{currentstroke}%
\pgfsetstrokeopacity{0.800000}%
\pgfsetdash{}{0pt}%
\pgfpathmoveto{\pgfqpoint{1.596415in}{2.369014in}}%
\pgfpathcurveto{\pgfqpoint{1.600533in}{2.369014in}}{\pgfqpoint{1.604483in}{2.370650in}}{\pgfqpoint{1.607395in}{2.373562in}}%
\pgfpathcurveto{\pgfqpoint{1.610307in}{2.376474in}}{\pgfqpoint{1.611943in}{2.380424in}}{\pgfqpoint{1.611943in}{2.384542in}}%
\pgfpathcurveto{\pgfqpoint{1.611943in}{2.388660in}}{\pgfqpoint{1.610307in}{2.392610in}}{\pgfqpoint{1.607395in}{2.395522in}}%
\pgfpathcurveto{\pgfqpoint{1.604483in}{2.398434in}}{\pgfqpoint{1.600533in}{2.400070in}}{\pgfqpoint{1.596415in}{2.400070in}}%
\pgfpathcurveto{\pgfqpoint{1.592297in}{2.400070in}}{\pgfqpoint{1.588347in}{2.398434in}}{\pgfqpoint{1.585435in}{2.395522in}}%
\pgfpathcurveto{\pgfqpoint{1.582523in}{2.392610in}}{\pgfqpoint{1.580887in}{2.388660in}}{\pgfqpoint{1.580887in}{2.384542in}}%
\pgfpathcurveto{\pgfqpoint{1.580887in}{2.380424in}}{\pgfqpoint{1.582523in}{2.376474in}}{\pgfqpoint{1.585435in}{2.373562in}}%
\pgfpathcurveto{\pgfqpoint{1.588347in}{2.370650in}}{\pgfqpoint{1.592297in}{2.369014in}}{\pgfqpoint{1.596415in}{2.369014in}}%
\pgfpathclose%
\pgfusepath{fill}%
\end{pgfscope}%
\begin{pgfscope}%
\pgfpathrectangle{\pgfqpoint{0.887500in}{0.275000in}}{\pgfqpoint{4.225000in}{4.225000in}}%
\pgfusepath{clip}%
\pgfsetbuttcap%
\pgfsetroundjoin%
\definecolor{currentfill}{rgb}{0.000000,0.000000,0.000000}%
\pgfsetfillcolor{currentfill}%
\pgfsetfillopacity{0.800000}%
\pgfsetlinewidth{0.000000pt}%
\definecolor{currentstroke}{rgb}{0.000000,0.000000,0.000000}%
\pgfsetstrokecolor{currentstroke}%
\pgfsetstrokeopacity{0.800000}%
\pgfsetdash{}{0pt}%
\pgfpathmoveto{\pgfqpoint{3.999009in}{2.425611in}}%
\pgfpathcurveto{\pgfqpoint{4.003127in}{2.425611in}}{\pgfqpoint{4.007077in}{2.427247in}}{\pgfqpoint{4.009989in}{2.430159in}}%
\pgfpathcurveto{\pgfqpoint{4.012901in}{2.433071in}}{\pgfqpoint{4.014537in}{2.437021in}}{\pgfqpoint{4.014537in}{2.441139in}}%
\pgfpathcurveto{\pgfqpoint{4.014537in}{2.445257in}}{\pgfqpoint{4.012901in}{2.449207in}}{\pgfqpoint{4.009989in}{2.452119in}}%
\pgfpathcurveto{\pgfqpoint{4.007077in}{2.455031in}}{\pgfqpoint{4.003127in}{2.456667in}}{\pgfqpoint{3.999009in}{2.456667in}}%
\pgfpathcurveto{\pgfqpoint{3.994891in}{2.456667in}}{\pgfqpoint{3.990941in}{2.455031in}}{\pgfqpoint{3.988029in}{2.452119in}}%
\pgfpathcurveto{\pgfqpoint{3.985117in}{2.449207in}}{\pgfqpoint{3.983480in}{2.445257in}}{\pgfqpoint{3.983480in}{2.441139in}}%
\pgfpathcurveto{\pgfqpoint{3.983480in}{2.437021in}}{\pgfqpoint{3.985117in}{2.433071in}}{\pgfqpoint{3.988029in}{2.430159in}}%
\pgfpathcurveto{\pgfqpoint{3.990941in}{2.427247in}}{\pgfqpoint{3.994891in}{2.425611in}}{\pgfqpoint{3.999009in}{2.425611in}}%
\pgfpathclose%
\pgfusepath{fill}%
\end{pgfscope}%
\begin{pgfscope}%
\pgfpathrectangle{\pgfqpoint{0.887500in}{0.275000in}}{\pgfqpoint{4.225000in}{4.225000in}}%
\pgfusepath{clip}%
\pgfsetbuttcap%
\pgfsetroundjoin%
\definecolor{currentfill}{rgb}{0.000000,0.000000,0.000000}%
\pgfsetfillcolor{currentfill}%
\pgfsetfillopacity{0.800000}%
\pgfsetlinewidth{0.000000pt}%
\definecolor{currentstroke}{rgb}{0.000000,0.000000,0.000000}%
\pgfsetstrokecolor{currentstroke}%
\pgfsetstrokeopacity{0.800000}%
\pgfsetdash{}{0pt}%
\pgfpathmoveto{\pgfqpoint{4.601474in}{1.927036in}}%
\pgfpathcurveto{\pgfqpoint{4.605592in}{1.927036in}}{\pgfqpoint{4.609542in}{1.928672in}}{\pgfqpoint{4.612454in}{1.931584in}}%
\pgfpathcurveto{\pgfqpoint{4.615366in}{1.934496in}}{\pgfqpoint{4.617002in}{1.938446in}}{\pgfqpoint{4.617002in}{1.942564in}}%
\pgfpathcurveto{\pgfqpoint{4.617002in}{1.946682in}}{\pgfqpoint{4.615366in}{1.950632in}}{\pgfqpoint{4.612454in}{1.953544in}}%
\pgfpathcurveto{\pgfqpoint{4.609542in}{1.956456in}}{\pgfqpoint{4.605592in}{1.958092in}}{\pgfqpoint{4.601474in}{1.958092in}}%
\pgfpathcurveto{\pgfqpoint{4.597356in}{1.958092in}}{\pgfqpoint{4.593406in}{1.956456in}}{\pgfqpoint{4.590494in}{1.953544in}}%
\pgfpathcurveto{\pgfqpoint{4.587582in}{1.950632in}}{\pgfqpoint{4.585946in}{1.946682in}}{\pgfqpoint{4.585946in}{1.942564in}}%
\pgfpathcurveto{\pgfqpoint{4.585946in}{1.938446in}}{\pgfqpoint{4.587582in}{1.934496in}}{\pgfqpoint{4.590494in}{1.931584in}}%
\pgfpathcurveto{\pgfqpoint{4.593406in}{1.928672in}}{\pgfqpoint{4.597356in}{1.927036in}}{\pgfqpoint{4.601474in}{1.927036in}}%
\pgfpathclose%
\pgfusepath{fill}%
\end{pgfscope}%
\begin{pgfscope}%
\pgfpathrectangle{\pgfqpoint{0.887500in}{0.275000in}}{\pgfqpoint{4.225000in}{4.225000in}}%
\pgfusepath{clip}%
\pgfsetbuttcap%
\pgfsetroundjoin%
\definecolor{currentfill}{rgb}{0.000000,0.000000,0.000000}%
\pgfsetfillcolor{currentfill}%
\pgfsetfillopacity{0.800000}%
\pgfsetlinewidth{0.000000pt}%
\definecolor{currentstroke}{rgb}{0.000000,0.000000,0.000000}%
\pgfsetstrokecolor{currentstroke}%
\pgfsetstrokeopacity{0.800000}%
\pgfsetdash{}{0pt}%
\pgfpathmoveto{\pgfqpoint{3.170501in}{2.695958in}}%
\pgfpathcurveto{\pgfqpoint{3.174619in}{2.695958in}}{\pgfqpoint{3.178569in}{2.697594in}}{\pgfqpoint{3.181481in}{2.700506in}}%
\pgfpathcurveto{\pgfqpoint{3.184393in}{2.703418in}}{\pgfqpoint{3.186029in}{2.707368in}}{\pgfqpoint{3.186029in}{2.711486in}}%
\pgfpathcurveto{\pgfqpoint{3.186029in}{2.715604in}}{\pgfqpoint{3.184393in}{2.719554in}}{\pgfqpoint{3.181481in}{2.722466in}}%
\pgfpathcurveto{\pgfqpoint{3.178569in}{2.725378in}}{\pgfqpoint{3.174619in}{2.727014in}}{\pgfqpoint{3.170501in}{2.727014in}}%
\pgfpathcurveto{\pgfqpoint{3.166383in}{2.727014in}}{\pgfqpoint{3.162433in}{2.725378in}}{\pgfqpoint{3.159521in}{2.722466in}}%
\pgfpathcurveto{\pgfqpoint{3.156609in}{2.719554in}}{\pgfqpoint{3.154973in}{2.715604in}}{\pgfqpoint{3.154973in}{2.711486in}}%
\pgfpathcurveto{\pgfqpoint{3.154973in}{2.707368in}}{\pgfqpoint{3.156609in}{2.703418in}}{\pgfqpoint{3.159521in}{2.700506in}}%
\pgfpathcurveto{\pgfqpoint{3.162433in}{2.697594in}}{\pgfqpoint{3.166383in}{2.695958in}}{\pgfqpoint{3.170501in}{2.695958in}}%
\pgfpathclose%
\pgfusepath{fill}%
\end{pgfscope}%
\begin{pgfscope}%
\pgfpathrectangle{\pgfqpoint{0.887500in}{0.275000in}}{\pgfqpoint{4.225000in}{4.225000in}}%
\pgfusepath{clip}%
\pgfsetbuttcap%
\pgfsetroundjoin%
\definecolor{currentfill}{rgb}{0.000000,0.000000,0.000000}%
\pgfsetfillcolor{currentfill}%
\pgfsetfillopacity{0.800000}%
\pgfsetlinewidth{0.000000pt}%
\definecolor{currentstroke}{rgb}{0.000000,0.000000,0.000000}%
\pgfsetstrokecolor{currentstroke}%
\pgfsetstrokeopacity{0.800000}%
\pgfsetdash{}{0pt}%
\pgfpathmoveto{\pgfqpoint{3.820552in}{2.539334in}}%
\pgfpathcurveto{\pgfqpoint{3.824671in}{2.539334in}}{\pgfqpoint{3.828621in}{2.540971in}}{\pgfqpoint{3.831533in}{2.543883in}}%
\pgfpathcurveto{\pgfqpoint{3.834445in}{2.546795in}}{\pgfqpoint{3.836081in}{2.550745in}}{\pgfqpoint{3.836081in}{2.554863in}}%
\pgfpathcurveto{\pgfqpoint{3.836081in}{2.558981in}}{\pgfqpoint{3.834445in}{2.562931in}}{\pgfqpoint{3.831533in}{2.565843in}}%
\pgfpathcurveto{\pgfqpoint{3.828621in}{2.568755in}}{\pgfqpoint{3.824671in}{2.570391in}}{\pgfqpoint{3.820552in}{2.570391in}}%
\pgfpathcurveto{\pgfqpoint{3.816434in}{2.570391in}}{\pgfqpoint{3.812484in}{2.568755in}}{\pgfqpoint{3.809572in}{2.565843in}}%
\pgfpathcurveto{\pgfqpoint{3.806660in}{2.562931in}}{\pgfqpoint{3.805024in}{2.558981in}}{\pgfqpoint{3.805024in}{2.554863in}}%
\pgfpathcurveto{\pgfqpoint{3.805024in}{2.550745in}}{\pgfqpoint{3.806660in}{2.546795in}}{\pgfqpoint{3.809572in}{2.543883in}}%
\pgfpathcurveto{\pgfqpoint{3.812484in}{2.540971in}}{\pgfqpoint{3.816434in}{2.539334in}}{\pgfqpoint{3.820552in}{2.539334in}}%
\pgfpathclose%
\pgfusepath{fill}%
\end{pgfscope}%
\begin{pgfscope}%
\pgfpathrectangle{\pgfqpoint{0.887500in}{0.275000in}}{\pgfqpoint{4.225000in}{4.225000in}}%
\pgfusepath{clip}%
\pgfsetbuttcap%
\pgfsetroundjoin%
\definecolor{currentfill}{rgb}{0.000000,0.000000,0.000000}%
\pgfsetfillcolor{currentfill}%
\pgfsetfillopacity{0.800000}%
\pgfsetlinewidth{0.000000pt}%
\definecolor{currentstroke}{rgb}{0.000000,0.000000,0.000000}%
\pgfsetstrokecolor{currentstroke}%
\pgfsetstrokeopacity{0.800000}%
\pgfsetdash{}{0pt}%
\pgfpathmoveto{\pgfqpoint{2.245655in}{2.235391in}}%
\pgfpathcurveto{\pgfqpoint{2.249773in}{2.235391in}}{\pgfqpoint{2.253723in}{2.237027in}}{\pgfqpoint{2.256635in}{2.239939in}}%
\pgfpathcurveto{\pgfqpoint{2.259547in}{2.242851in}}{\pgfqpoint{2.261183in}{2.246801in}}{\pgfqpoint{2.261183in}{2.250919in}}%
\pgfpathcurveto{\pgfqpoint{2.261183in}{2.255038in}}{\pgfqpoint{2.259547in}{2.258988in}}{\pgfqpoint{2.256635in}{2.261899in}}%
\pgfpathcurveto{\pgfqpoint{2.253723in}{2.264811in}}{\pgfqpoint{2.249773in}{2.266448in}}{\pgfqpoint{2.245655in}{2.266448in}}%
\pgfpathcurveto{\pgfqpoint{2.241537in}{2.266448in}}{\pgfqpoint{2.237587in}{2.264811in}}{\pgfqpoint{2.234675in}{2.261899in}}%
\pgfpathcurveto{\pgfqpoint{2.231763in}{2.258988in}}{\pgfqpoint{2.230127in}{2.255038in}}{\pgfqpoint{2.230127in}{2.250919in}}%
\pgfpathcurveto{\pgfqpoint{2.230127in}{2.246801in}}{\pgfqpoint{2.231763in}{2.242851in}}{\pgfqpoint{2.234675in}{2.239939in}}%
\pgfpathcurveto{\pgfqpoint{2.237587in}{2.237027in}}{\pgfqpoint{2.241537in}{2.235391in}}{\pgfqpoint{2.245655in}{2.235391in}}%
\pgfpathclose%
\pgfusepath{fill}%
\end{pgfscope}%
\begin{pgfscope}%
\pgfpathrectangle{\pgfqpoint{0.887500in}{0.275000in}}{\pgfqpoint{4.225000in}{4.225000in}}%
\pgfusepath{clip}%
\pgfsetbuttcap%
\pgfsetroundjoin%
\definecolor{currentfill}{rgb}{0.000000,0.000000,0.000000}%
\pgfsetfillcolor{currentfill}%
\pgfsetfillopacity{0.800000}%
\pgfsetlinewidth{0.000000pt}%
\definecolor{currentstroke}{rgb}{0.000000,0.000000,0.000000}%
\pgfsetstrokecolor{currentstroke}%
\pgfsetstrokeopacity{0.800000}%
\pgfsetdash{}{0pt}%
\pgfpathmoveto{\pgfqpoint{2.895714in}{2.086801in}}%
\pgfpathcurveto{\pgfqpoint{2.899832in}{2.086801in}}{\pgfqpoint{2.903782in}{2.088437in}}{\pgfqpoint{2.906694in}{2.091349in}}%
\pgfpathcurveto{\pgfqpoint{2.909606in}{2.094261in}}{\pgfqpoint{2.911242in}{2.098211in}}{\pgfqpoint{2.911242in}{2.102329in}}%
\pgfpathcurveto{\pgfqpoint{2.911242in}{2.106447in}}{\pgfqpoint{2.909606in}{2.110397in}}{\pgfqpoint{2.906694in}{2.113309in}}%
\pgfpathcurveto{\pgfqpoint{2.903782in}{2.116221in}}{\pgfqpoint{2.899832in}{2.117857in}}{\pgfqpoint{2.895714in}{2.117857in}}%
\pgfpathcurveto{\pgfqpoint{2.891596in}{2.117857in}}{\pgfqpoint{2.887646in}{2.116221in}}{\pgfqpoint{2.884734in}{2.113309in}}%
\pgfpathcurveto{\pgfqpoint{2.881822in}{2.110397in}}{\pgfqpoint{2.880186in}{2.106447in}}{\pgfqpoint{2.880186in}{2.102329in}}%
\pgfpathcurveto{\pgfqpoint{2.880186in}{2.098211in}}{\pgfqpoint{2.881822in}{2.094261in}}{\pgfqpoint{2.884734in}{2.091349in}}%
\pgfpathcurveto{\pgfqpoint{2.887646in}{2.088437in}}{\pgfqpoint{2.891596in}{2.086801in}}{\pgfqpoint{2.895714in}{2.086801in}}%
\pgfpathclose%
\pgfusepath{fill}%
\end{pgfscope}%
\begin{pgfscope}%
\pgfpathrectangle{\pgfqpoint{0.887500in}{0.275000in}}{\pgfqpoint{4.225000in}{4.225000in}}%
\pgfusepath{clip}%
\pgfsetbuttcap%
\pgfsetroundjoin%
\definecolor{currentfill}{rgb}{0.000000,0.000000,0.000000}%
\pgfsetfillcolor{currentfill}%
\pgfsetfillopacity{0.800000}%
\pgfsetlinewidth{0.000000pt}%
\definecolor{currentstroke}{rgb}{0.000000,0.000000,0.000000}%
\pgfsetstrokecolor{currentstroke}%
\pgfsetstrokeopacity{0.800000}%
\pgfsetdash{}{0pt}%
\pgfpathmoveto{\pgfqpoint{3.283992in}{2.841739in}}%
\pgfpathcurveto{\pgfqpoint{3.288110in}{2.841739in}}{\pgfqpoint{3.292060in}{2.843375in}}{\pgfqpoint{3.294972in}{2.846287in}}%
\pgfpathcurveto{\pgfqpoint{3.297884in}{2.849199in}}{\pgfqpoint{3.299520in}{2.853149in}}{\pgfqpoint{3.299520in}{2.857267in}}%
\pgfpathcurveto{\pgfqpoint{3.299520in}{2.861385in}}{\pgfqpoint{3.297884in}{2.865335in}}{\pgfqpoint{3.294972in}{2.868247in}}%
\pgfpathcurveto{\pgfqpoint{3.292060in}{2.871159in}}{\pgfqpoint{3.288110in}{2.872795in}}{\pgfqpoint{3.283992in}{2.872795in}}%
\pgfpathcurveto{\pgfqpoint{3.279873in}{2.872795in}}{\pgfqpoint{3.275923in}{2.871159in}}{\pgfqpoint{3.273011in}{2.868247in}}%
\pgfpathcurveto{\pgfqpoint{3.270099in}{2.865335in}}{\pgfqpoint{3.268463in}{2.861385in}}{\pgfqpoint{3.268463in}{2.857267in}}%
\pgfpathcurveto{\pgfqpoint{3.268463in}{2.853149in}}{\pgfqpoint{3.270099in}{2.849199in}}{\pgfqpoint{3.273011in}{2.846287in}}%
\pgfpathcurveto{\pgfqpoint{3.275923in}{2.843375in}}{\pgfqpoint{3.279873in}{2.841739in}}{\pgfqpoint{3.283992in}{2.841739in}}%
\pgfpathclose%
\pgfusepath{fill}%
\end{pgfscope}%
\begin{pgfscope}%
\pgfpathrectangle{\pgfqpoint{0.887500in}{0.275000in}}{\pgfqpoint{4.225000in}{4.225000in}}%
\pgfusepath{clip}%
\pgfsetbuttcap%
\pgfsetroundjoin%
\definecolor{currentfill}{rgb}{0.000000,0.000000,0.000000}%
\pgfsetfillcolor{currentfill}%
\pgfsetfillopacity{0.800000}%
\pgfsetlinewidth{0.000000pt}%
\definecolor{currentstroke}{rgb}{0.000000,0.000000,0.000000}%
\pgfsetstrokecolor{currentstroke}%
\pgfsetstrokeopacity{0.800000}%
\pgfsetdash{}{0pt}%
\pgfpathmoveto{\pgfqpoint{3.641890in}{2.650122in}}%
\pgfpathcurveto{\pgfqpoint{3.646008in}{2.650122in}}{\pgfqpoint{3.649958in}{2.651758in}}{\pgfqpoint{3.652870in}{2.654670in}}%
\pgfpathcurveto{\pgfqpoint{3.655782in}{2.657582in}}{\pgfqpoint{3.657418in}{2.661532in}}{\pgfqpoint{3.657418in}{2.665650in}}%
\pgfpathcurveto{\pgfqpoint{3.657418in}{2.669768in}}{\pgfqpoint{3.655782in}{2.673718in}}{\pgfqpoint{3.652870in}{2.676630in}}%
\pgfpathcurveto{\pgfqpoint{3.649958in}{2.679542in}}{\pgfqpoint{3.646008in}{2.681178in}}{\pgfqpoint{3.641890in}{2.681178in}}%
\pgfpathcurveto{\pgfqpoint{3.637772in}{2.681178in}}{\pgfqpoint{3.633822in}{2.679542in}}{\pgfqpoint{3.630910in}{2.676630in}}%
\pgfpathcurveto{\pgfqpoint{3.627998in}{2.673718in}}{\pgfqpoint{3.626362in}{2.669768in}}{\pgfqpoint{3.626362in}{2.665650in}}%
\pgfpathcurveto{\pgfqpoint{3.626362in}{2.661532in}}{\pgfqpoint{3.627998in}{2.657582in}}{\pgfqpoint{3.630910in}{2.654670in}}%
\pgfpathcurveto{\pgfqpoint{3.633822in}{2.651758in}}{\pgfqpoint{3.637772in}{2.650122in}}{\pgfqpoint{3.641890in}{2.650122in}}%
\pgfpathclose%
\pgfusepath{fill}%
\end{pgfscope}%
\begin{pgfscope}%
\pgfpathrectangle{\pgfqpoint{0.887500in}{0.275000in}}{\pgfqpoint{4.225000in}{4.225000in}}%
\pgfusepath{clip}%
\pgfsetbuttcap%
\pgfsetroundjoin%
\definecolor{currentfill}{rgb}{0.000000,0.000000,0.000000}%
\pgfsetfillcolor{currentfill}%
\pgfsetfillopacity{0.800000}%
\pgfsetlinewidth{0.000000pt}%
\definecolor{currentstroke}{rgb}{0.000000,0.000000,0.000000}%
\pgfsetstrokecolor{currentstroke}%
\pgfsetstrokeopacity{0.800000}%
\pgfsetdash{}{0pt}%
\pgfpathmoveto{\pgfqpoint{4.423379in}{2.045197in}}%
\pgfpathcurveto{\pgfqpoint{4.427498in}{2.045197in}}{\pgfqpoint{4.431448in}{2.046833in}}{\pgfqpoint{4.434360in}{2.049745in}}%
\pgfpathcurveto{\pgfqpoint{4.437271in}{2.052657in}}{\pgfqpoint{4.438908in}{2.056607in}}{\pgfqpoint{4.438908in}{2.060725in}}%
\pgfpathcurveto{\pgfqpoint{4.438908in}{2.064843in}}{\pgfqpoint{4.437271in}{2.068793in}}{\pgfqpoint{4.434360in}{2.071705in}}%
\pgfpathcurveto{\pgfqpoint{4.431448in}{2.074617in}}{\pgfqpoint{4.427498in}{2.076253in}}{\pgfqpoint{4.423379in}{2.076253in}}%
\pgfpathcurveto{\pgfqpoint{4.419261in}{2.076253in}}{\pgfqpoint{4.415311in}{2.074617in}}{\pgfqpoint{4.412399in}{2.071705in}}%
\pgfpathcurveto{\pgfqpoint{4.409487in}{2.068793in}}{\pgfqpoint{4.407851in}{2.064843in}}{\pgfqpoint{4.407851in}{2.060725in}}%
\pgfpathcurveto{\pgfqpoint{4.407851in}{2.056607in}}{\pgfqpoint{4.409487in}{2.052657in}}{\pgfqpoint{4.412399in}{2.049745in}}%
\pgfpathcurveto{\pgfqpoint{4.415311in}{2.046833in}}{\pgfqpoint{4.419261in}{2.045197in}}{\pgfqpoint{4.423379in}{2.045197in}}%
\pgfpathclose%
\pgfusepath{fill}%
\end{pgfscope}%
\begin{pgfscope}%
\pgfpathrectangle{\pgfqpoint{0.887500in}{0.275000in}}{\pgfqpoint{4.225000in}{4.225000in}}%
\pgfusepath{clip}%
\pgfsetbuttcap%
\pgfsetroundjoin%
\definecolor{currentfill}{rgb}{0.000000,0.000000,0.000000}%
\pgfsetfillcolor{currentfill}%
\pgfsetfillopacity{0.800000}%
\pgfsetlinewidth{0.000000pt}%
\definecolor{currentstroke}{rgb}{0.000000,0.000000,0.000000}%
\pgfsetstrokecolor{currentstroke}%
\pgfsetstrokeopacity{0.800000}%
\pgfsetdash{}{0pt}%
\pgfpathmoveto{\pgfqpoint{1.773857in}{2.314576in}}%
\pgfpathcurveto{\pgfqpoint{1.777975in}{2.314576in}}{\pgfqpoint{1.781925in}{2.316212in}}{\pgfqpoint{1.784837in}{2.319124in}}%
\pgfpathcurveto{\pgfqpoint{1.787749in}{2.322036in}}{\pgfqpoint{1.789385in}{2.325986in}}{\pgfqpoint{1.789385in}{2.330105in}}%
\pgfpathcurveto{\pgfqpoint{1.789385in}{2.334223in}}{\pgfqpoint{1.787749in}{2.338173in}}{\pgfqpoint{1.784837in}{2.341085in}}%
\pgfpathcurveto{\pgfqpoint{1.781925in}{2.343997in}}{\pgfqpoint{1.777975in}{2.345633in}}{\pgfqpoint{1.773857in}{2.345633in}}%
\pgfpathcurveto{\pgfqpoint{1.769739in}{2.345633in}}{\pgfqpoint{1.765789in}{2.343997in}}{\pgfqpoint{1.762877in}{2.341085in}}%
\pgfpathcurveto{\pgfqpoint{1.759965in}{2.338173in}}{\pgfqpoint{1.758329in}{2.334223in}}{\pgfqpoint{1.758329in}{2.330105in}}%
\pgfpathcurveto{\pgfqpoint{1.758329in}{2.325986in}}{\pgfqpoint{1.759965in}{2.322036in}}{\pgfqpoint{1.762877in}{2.319124in}}%
\pgfpathcurveto{\pgfqpoint{1.765789in}{2.316212in}}{\pgfqpoint{1.769739in}{2.314576in}}{\pgfqpoint{1.773857in}{2.314576in}}%
\pgfpathclose%
\pgfusepath{fill}%
\end{pgfscope}%
\begin{pgfscope}%
\pgfpathrectangle{\pgfqpoint{0.887500in}{0.275000in}}{\pgfqpoint{4.225000in}{4.225000in}}%
\pgfusepath{clip}%
\pgfsetbuttcap%
\pgfsetroundjoin%
\definecolor{currentfill}{rgb}{0.000000,0.000000,0.000000}%
\pgfsetfillcolor{currentfill}%
\pgfsetfillopacity{0.800000}%
\pgfsetlinewidth{0.000000pt}%
\definecolor{currentstroke}{rgb}{0.000000,0.000000,0.000000}%
\pgfsetstrokecolor{currentstroke}%
\pgfsetstrokeopacity{0.800000}%
\pgfsetdash{}{0pt}%
\pgfpathmoveto{\pgfqpoint{3.463034in}{2.755071in}}%
\pgfpathcurveto{\pgfqpoint{3.467152in}{2.755071in}}{\pgfqpoint{3.471102in}{2.756707in}}{\pgfqpoint{3.474014in}{2.759619in}}%
\pgfpathcurveto{\pgfqpoint{3.476926in}{2.762531in}}{\pgfqpoint{3.478562in}{2.766481in}}{\pgfqpoint{3.478562in}{2.770599in}}%
\pgfpathcurveto{\pgfqpoint{3.478562in}{2.774717in}}{\pgfqpoint{3.476926in}{2.778667in}}{\pgfqpoint{3.474014in}{2.781579in}}%
\pgfpathcurveto{\pgfqpoint{3.471102in}{2.784491in}}{\pgfqpoint{3.467152in}{2.786127in}}{\pgfqpoint{3.463034in}{2.786127in}}%
\pgfpathcurveto{\pgfqpoint{3.458915in}{2.786127in}}{\pgfqpoint{3.454965in}{2.784491in}}{\pgfqpoint{3.452053in}{2.781579in}}%
\pgfpathcurveto{\pgfqpoint{3.449141in}{2.778667in}}{\pgfqpoint{3.447505in}{2.774717in}}{\pgfqpoint{3.447505in}{2.770599in}}%
\pgfpathcurveto{\pgfqpoint{3.447505in}{2.766481in}}{\pgfqpoint{3.449141in}{2.762531in}}{\pgfqpoint{3.452053in}{2.759619in}}%
\pgfpathcurveto{\pgfqpoint{3.454965in}{2.756707in}}{\pgfqpoint{3.458915in}{2.755071in}}{\pgfqpoint{3.463034in}{2.755071in}}%
\pgfpathclose%
\pgfusepath{fill}%
\end{pgfscope}%
\begin{pgfscope}%
\pgfpathrectangle{\pgfqpoint{0.887500in}{0.275000in}}{\pgfqpoint{4.225000in}{4.225000in}}%
\pgfusepath{clip}%
\pgfsetbuttcap%
\pgfsetroundjoin%
\definecolor{currentfill}{rgb}{0.000000,0.000000,0.000000}%
\pgfsetfillcolor{currentfill}%
\pgfsetfillopacity{0.800000}%
\pgfsetlinewidth{0.000000pt}%
\definecolor{currentstroke}{rgb}{0.000000,0.000000,0.000000}%
\pgfsetstrokecolor{currentstroke}%
\pgfsetstrokeopacity{0.800000}%
\pgfsetdash{}{0pt}%
\pgfpathmoveto{\pgfqpoint{3.122611in}{2.363089in}}%
\pgfpathcurveto{\pgfqpoint{3.126729in}{2.363089in}}{\pgfqpoint{3.130679in}{2.364725in}}{\pgfqpoint{3.133591in}{2.367637in}}%
\pgfpathcurveto{\pgfqpoint{3.136503in}{2.370549in}}{\pgfqpoint{3.138139in}{2.374499in}}{\pgfqpoint{3.138139in}{2.378617in}}%
\pgfpathcurveto{\pgfqpoint{3.138139in}{2.382735in}}{\pgfqpoint{3.136503in}{2.386685in}}{\pgfqpoint{3.133591in}{2.389597in}}%
\pgfpathcurveto{\pgfqpoint{3.130679in}{2.392509in}}{\pgfqpoint{3.126729in}{2.394145in}}{\pgfqpoint{3.122611in}{2.394145in}}%
\pgfpathcurveto{\pgfqpoint{3.118493in}{2.394145in}}{\pgfqpoint{3.114543in}{2.392509in}}{\pgfqpoint{3.111631in}{2.389597in}}%
\pgfpathcurveto{\pgfqpoint{3.108719in}{2.386685in}}{\pgfqpoint{3.107083in}{2.382735in}}{\pgfqpoint{3.107083in}{2.378617in}}%
\pgfpathcurveto{\pgfqpoint{3.107083in}{2.374499in}}{\pgfqpoint{3.108719in}{2.370549in}}{\pgfqpoint{3.111631in}{2.367637in}}%
\pgfpathcurveto{\pgfqpoint{3.114543in}{2.364725in}}{\pgfqpoint{3.118493in}{2.363089in}}{\pgfqpoint{3.122611in}{2.363089in}}%
\pgfpathclose%
\pgfusepath{fill}%
\end{pgfscope}%
\begin{pgfscope}%
\pgfpathrectangle{\pgfqpoint{0.887500in}{0.275000in}}{\pgfqpoint{4.225000in}{4.225000in}}%
\pgfusepath{clip}%
\pgfsetbuttcap%
\pgfsetroundjoin%
\definecolor{currentfill}{rgb}{0.000000,0.000000,0.000000}%
\pgfsetfillcolor{currentfill}%
\pgfsetfillopacity{0.800000}%
\pgfsetlinewidth{0.000000pt}%
\definecolor{currentstroke}{rgb}{0.000000,0.000000,0.000000}%
\pgfsetstrokecolor{currentstroke}%
\pgfsetstrokeopacity{0.800000}%
\pgfsetdash{}{0pt}%
\pgfpathmoveto{\pgfqpoint{3.074656in}{2.019366in}}%
\pgfpathcurveto{\pgfqpoint{3.078774in}{2.019366in}}{\pgfqpoint{3.082724in}{2.021002in}}{\pgfqpoint{3.085636in}{2.023914in}}%
\pgfpathcurveto{\pgfqpoint{3.088548in}{2.026826in}}{\pgfqpoint{3.090184in}{2.030776in}}{\pgfqpoint{3.090184in}{2.034894in}}%
\pgfpathcurveto{\pgfqpoint{3.090184in}{2.039012in}}{\pgfqpoint{3.088548in}{2.042962in}}{\pgfqpoint{3.085636in}{2.045874in}}%
\pgfpathcurveto{\pgfqpoint{3.082724in}{2.048786in}}{\pgfqpoint{3.078774in}{2.050422in}}{\pgfqpoint{3.074656in}{2.050422in}}%
\pgfpathcurveto{\pgfqpoint{3.070538in}{2.050422in}}{\pgfqpoint{3.066588in}{2.048786in}}{\pgfqpoint{3.063676in}{2.045874in}}%
\pgfpathcurveto{\pgfqpoint{3.060764in}{2.042962in}}{\pgfqpoint{3.059128in}{2.039012in}}{\pgfqpoint{3.059128in}{2.034894in}}%
\pgfpathcurveto{\pgfqpoint{3.059128in}{2.030776in}}{\pgfqpoint{3.060764in}{2.026826in}}{\pgfqpoint{3.063676in}{2.023914in}}%
\pgfpathcurveto{\pgfqpoint{3.066588in}{2.021002in}}{\pgfqpoint{3.070538in}{2.019366in}}{\pgfqpoint{3.074656in}{2.019366in}}%
\pgfpathclose%
\pgfusepath{fill}%
\end{pgfscope}%
\begin{pgfscope}%
\pgfpathrectangle{\pgfqpoint{0.887500in}{0.275000in}}{\pgfqpoint{4.225000in}{4.225000in}}%
\pgfusepath{clip}%
\pgfsetbuttcap%
\pgfsetroundjoin%
\definecolor{currentfill}{rgb}{0.000000,0.000000,0.000000}%
\pgfsetfillcolor{currentfill}%
\pgfsetfillopacity{0.800000}%
\pgfsetlinewidth{0.000000pt}%
\definecolor{currentstroke}{rgb}{0.000000,0.000000,0.000000}%
\pgfsetstrokecolor{currentstroke}%
\pgfsetstrokeopacity{0.800000}%
\pgfsetdash{}{0pt}%
\pgfpathmoveto{\pgfqpoint{2.423945in}{2.176918in}}%
\pgfpathcurveto{\pgfqpoint{2.428064in}{2.176918in}}{\pgfqpoint{2.432014in}{2.178554in}}{\pgfqpoint{2.434926in}{2.181466in}}%
\pgfpathcurveto{\pgfqpoint{2.437838in}{2.184378in}}{\pgfqpoint{2.439474in}{2.188328in}}{\pgfqpoint{2.439474in}{2.192446in}}%
\pgfpathcurveto{\pgfqpoint{2.439474in}{2.196564in}}{\pgfqpoint{2.437838in}{2.200514in}}{\pgfqpoint{2.434926in}{2.203426in}}%
\pgfpathcurveto{\pgfqpoint{2.432014in}{2.206338in}}{\pgfqpoint{2.428064in}{2.207974in}}{\pgfqpoint{2.423945in}{2.207974in}}%
\pgfpathcurveto{\pgfqpoint{2.419827in}{2.207974in}}{\pgfqpoint{2.415877in}{2.206338in}}{\pgfqpoint{2.412965in}{2.203426in}}%
\pgfpathcurveto{\pgfqpoint{2.410053in}{2.200514in}}{\pgfqpoint{2.408417in}{2.196564in}}{\pgfqpoint{2.408417in}{2.192446in}}%
\pgfpathcurveto{\pgfqpoint{2.408417in}{2.188328in}}{\pgfqpoint{2.410053in}{2.184378in}}{\pgfqpoint{2.412965in}{2.181466in}}%
\pgfpathcurveto{\pgfqpoint{2.415877in}{2.178554in}}{\pgfqpoint{2.419827in}{2.176918in}}{\pgfqpoint{2.423945in}{2.176918in}}%
\pgfpathclose%
\pgfusepath{fill}%
\end{pgfscope}%
\begin{pgfscope}%
\pgfpathrectangle{\pgfqpoint{0.887500in}{0.275000in}}{\pgfqpoint{4.225000in}{4.225000in}}%
\pgfusepath{clip}%
\pgfsetbuttcap%
\pgfsetroundjoin%
\definecolor{currentfill}{rgb}{0.000000,0.000000,0.000000}%
\pgfsetfillcolor{currentfill}%
\pgfsetfillopacity{0.800000}%
\pgfsetlinewidth{0.000000pt}%
\definecolor{currentstroke}{rgb}{0.000000,0.000000,0.000000}%
\pgfsetstrokecolor{currentstroke}%
\pgfsetstrokeopacity{0.800000}%
\pgfsetdash{}{0pt}%
\pgfpathmoveto{\pgfqpoint{4.066409in}{2.275207in}}%
\pgfpathcurveto{\pgfqpoint{4.070527in}{2.275207in}}{\pgfqpoint{4.074477in}{2.276843in}}{\pgfqpoint{4.077389in}{2.279755in}}%
\pgfpathcurveto{\pgfqpoint{4.080301in}{2.282667in}}{\pgfqpoint{4.081937in}{2.286617in}}{\pgfqpoint{4.081937in}{2.290735in}}%
\pgfpathcurveto{\pgfqpoint{4.081937in}{2.294853in}}{\pgfqpoint{4.080301in}{2.298803in}}{\pgfqpoint{4.077389in}{2.301715in}}%
\pgfpathcurveto{\pgfqpoint{4.074477in}{2.304627in}}{\pgfqpoint{4.070527in}{2.306263in}}{\pgfqpoint{4.066409in}{2.306263in}}%
\pgfpathcurveto{\pgfqpoint{4.062291in}{2.306263in}}{\pgfqpoint{4.058341in}{2.304627in}}{\pgfqpoint{4.055429in}{2.301715in}}%
\pgfpathcurveto{\pgfqpoint{4.052517in}{2.298803in}}{\pgfqpoint{4.050881in}{2.294853in}}{\pgfqpoint{4.050881in}{2.290735in}}%
\pgfpathcurveto{\pgfqpoint{4.050881in}{2.286617in}}{\pgfqpoint{4.052517in}{2.282667in}}{\pgfqpoint{4.055429in}{2.279755in}}%
\pgfpathcurveto{\pgfqpoint{4.058341in}{2.276843in}}{\pgfqpoint{4.062291in}{2.275207in}}{\pgfqpoint{4.066409in}{2.275207in}}%
\pgfpathclose%
\pgfusepath{fill}%
\end{pgfscope}%
\begin{pgfscope}%
\pgfpathrectangle{\pgfqpoint{0.887500in}{0.275000in}}{\pgfqpoint{4.225000in}{4.225000in}}%
\pgfusepath{clip}%
\pgfsetbuttcap%
\pgfsetroundjoin%
\definecolor{currentfill}{rgb}{0.000000,0.000000,0.000000}%
\pgfsetfillcolor{currentfill}%
\pgfsetfillopacity{0.800000}%
\pgfsetlinewidth{0.000000pt}%
\definecolor{currentstroke}{rgb}{0.000000,0.000000,0.000000}%
\pgfsetstrokecolor{currentstroke}%
\pgfsetstrokeopacity{0.800000}%
\pgfsetdash{}{0pt}%
\pgfpathmoveto{\pgfqpoint{1.951721in}{2.258303in}}%
\pgfpathcurveto{\pgfqpoint{1.955839in}{2.258303in}}{\pgfqpoint{1.959789in}{2.259939in}}{\pgfqpoint{1.962701in}{2.262851in}}%
\pgfpathcurveto{\pgfqpoint{1.965613in}{2.265763in}}{\pgfqpoint{1.967249in}{2.269713in}}{\pgfqpoint{1.967249in}{2.273831in}}%
\pgfpathcurveto{\pgfqpoint{1.967249in}{2.277949in}}{\pgfqpoint{1.965613in}{2.281899in}}{\pgfqpoint{1.962701in}{2.284811in}}%
\pgfpathcurveto{\pgfqpoint{1.959789in}{2.287723in}}{\pgfqpoint{1.955839in}{2.289359in}}{\pgfqpoint{1.951721in}{2.289359in}}%
\pgfpathcurveto{\pgfqpoint{1.947603in}{2.289359in}}{\pgfqpoint{1.943653in}{2.287723in}}{\pgfqpoint{1.940741in}{2.284811in}}%
\pgfpathcurveto{\pgfqpoint{1.937829in}{2.281899in}}{\pgfqpoint{1.936193in}{2.277949in}}{\pgfqpoint{1.936193in}{2.273831in}}%
\pgfpathcurveto{\pgfqpoint{1.936193in}{2.269713in}}{\pgfqpoint{1.937829in}{2.265763in}}{\pgfqpoint{1.940741in}{2.262851in}}%
\pgfpathcurveto{\pgfqpoint{1.943653in}{2.259939in}}{\pgfqpoint{1.947603in}{2.258303in}}{\pgfqpoint{1.951721in}{2.258303in}}%
\pgfpathclose%
\pgfusepath{fill}%
\end{pgfscope}%
\begin{pgfscope}%
\pgfpathrectangle{\pgfqpoint{0.887500in}{0.275000in}}{\pgfqpoint{4.225000in}{4.225000in}}%
\pgfusepath{clip}%
\pgfsetbuttcap%
\pgfsetroundjoin%
\definecolor{currentfill}{rgb}{0.000000,0.000000,0.000000}%
\pgfsetfillcolor{currentfill}%
\pgfsetfillopacity{0.800000}%
\pgfsetlinewidth{0.000000pt}%
\definecolor{currentstroke}{rgb}{0.000000,0.000000,0.000000}%
\pgfsetstrokecolor{currentstroke}%
\pgfsetstrokeopacity{0.800000}%
\pgfsetdash{}{0pt}%
\pgfpathmoveto{\pgfqpoint{4.669958in}{1.772882in}}%
\pgfpathcurveto{\pgfqpoint{4.674076in}{1.772882in}}{\pgfqpoint{4.678026in}{1.774518in}}{\pgfqpoint{4.680938in}{1.777430in}}%
\pgfpathcurveto{\pgfqpoint{4.683850in}{1.780342in}}{\pgfqpoint{4.685486in}{1.784292in}}{\pgfqpoint{4.685486in}{1.788411in}}%
\pgfpathcurveto{\pgfqpoint{4.685486in}{1.792529in}}{\pgfqpoint{4.683850in}{1.796479in}}{\pgfqpoint{4.680938in}{1.799391in}}%
\pgfpathcurveto{\pgfqpoint{4.678026in}{1.802303in}}{\pgfqpoint{4.674076in}{1.803939in}}{\pgfqpoint{4.669958in}{1.803939in}}%
\pgfpathcurveto{\pgfqpoint{4.665840in}{1.803939in}}{\pgfqpoint{4.661890in}{1.802303in}}{\pgfqpoint{4.658978in}{1.799391in}}%
\pgfpathcurveto{\pgfqpoint{4.656066in}{1.796479in}}{\pgfqpoint{4.654429in}{1.792529in}}{\pgfqpoint{4.654429in}{1.788411in}}%
\pgfpathcurveto{\pgfqpoint{4.654429in}{1.784292in}}{\pgfqpoint{4.656066in}{1.780342in}}{\pgfqpoint{4.658978in}{1.777430in}}%
\pgfpathcurveto{\pgfqpoint{4.661890in}{1.774518in}}{\pgfqpoint{4.665840in}{1.772882in}}{\pgfqpoint{4.669958in}{1.772882in}}%
\pgfpathclose%
\pgfusepath{fill}%
\end{pgfscope}%
\begin{pgfscope}%
\pgfpathrectangle{\pgfqpoint{0.887500in}{0.275000in}}{\pgfqpoint{4.225000in}{4.225000in}}%
\pgfusepath{clip}%
\pgfsetbuttcap%
\pgfsetroundjoin%
\definecolor{currentfill}{rgb}{0.000000,0.000000,0.000000}%
\pgfsetfillcolor{currentfill}%
\pgfsetfillopacity{0.800000}%
\pgfsetlinewidth{0.000000pt}%
\definecolor{currentstroke}{rgb}{0.000000,0.000000,0.000000}%
\pgfsetstrokecolor{currentstroke}%
\pgfsetstrokeopacity{0.800000}%
\pgfsetdash{}{0pt}%
\pgfpathmoveto{\pgfqpoint{2.602600in}{2.116274in}}%
\pgfpathcurveto{\pgfqpoint{2.606718in}{2.116274in}}{\pgfqpoint{2.610668in}{2.117910in}}{\pgfqpoint{2.613580in}{2.120822in}}%
\pgfpathcurveto{\pgfqpoint{2.616492in}{2.123734in}}{\pgfqpoint{2.618128in}{2.127684in}}{\pgfqpoint{2.618128in}{2.131803in}}%
\pgfpathcurveto{\pgfqpoint{2.618128in}{2.135921in}}{\pgfqpoint{2.616492in}{2.139871in}}{\pgfqpoint{2.613580in}{2.142783in}}%
\pgfpathcurveto{\pgfqpoint{2.610668in}{2.145695in}}{\pgfqpoint{2.606718in}{2.147331in}}{\pgfqpoint{2.602600in}{2.147331in}}%
\pgfpathcurveto{\pgfqpoint{2.598481in}{2.147331in}}{\pgfqpoint{2.594531in}{2.145695in}}{\pgfqpoint{2.591620in}{2.142783in}}%
\pgfpathcurveto{\pgfqpoint{2.588708in}{2.139871in}}{\pgfqpoint{2.587071in}{2.135921in}}{\pgfqpoint{2.587071in}{2.131803in}}%
\pgfpathcurveto{\pgfqpoint{2.587071in}{2.127684in}}{\pgfqpoint{2.588708in}{2.123734in}}{\pgfqpoint{2.591620in}{2.120822in}}%
\pgfpathcurveto{\pgfqpoint{2.594531in}{2.117910in}}{\pgfqpoint{2.598481in}{2.116274in}}{\pgfqpoint{2.602600in}{2.116274in}}%
\pgfpathclose%
\pgfusepath{fill}%
\end{pgfscope}%
\begin{pgfscope}%
\pgfpathrectangle{\pgfqpoint{0.887500in}{0.275000in}}{\pgfqpoint{4.225000in}{4.225000in}}%
\pgfusepath{clip}%
\pgfsetbuttcap%
\pgfsetroundjoin%
\definecolor{currentfill}{rgb}{0.000000,0.000000,0.000000}%
\pgfsetfillcolor{currentfill}%
\pgfsetfillopacity{0.800000}%
\pgfsetlinewidth{0.000000pt}%
\definecolor{currentstroke}{rgb}{0.000000,0.000000,0.000000}%
\pgfsetstrokecolor{currentstroke}%
\pgfsetstrokeopacity{0.800000}%
\pgfsetdash{}{0pt}%
\pgfpathmoveto{\pgfqpoint{3.887689in}{2.391588in}}%
\pgfpathcurveto{\pgfqpoint{3.891808in}{2.391588in}}{\pgfqpoint{3.895758in}{2.393224in}}{\pgfqpoint{3.898670in}{2.396136in}}%
\pgfpathcurveto{\pgfqpoint{3.901582in}{2.399048in}}{\pgfqpoint{3.903218in}{2.402998in}}{\pgfqpoint{3.903218in}{2.407117in}}%
\pgfpathcurveto{\pgfqpoint{3.903218in}{2.411235in}}{\pgfqpoint{3.901582in}{2.415185in}}{\pgfqpoint{3.898670in}{2.418097in}}%
\pgfpathcurveto{\pgfqpoint{3.895758in}{2.421009in}}{\pgfqpoint{3.891808in}{2.422645in}}{\pgfqpoint{3.887689in}{2.422645in}}%
\pgfpathcurveto{\pgfqpoint{3.883571in}{2.422645in}}{\pgfqpoint{3.879621in}{2.421009in}}{\pgfqpoint{3.876709in}{2.418097in}}%
\pgfpathcurveto{\pgfqpoint{3.873797in}{2.415185in}}{\pgfqpoint{3.872161in}{2.411235in}}{\pgfqpoint{3.872161in}{2.407117in}}%
\pgfpathcurveto{\pgfqpoint{3.872161in}{2.402998in}}{\pgfqpoint{3.873797in}{2.399048in}}{\pgfqpoint{3.876709in}{2.396136in}}%
\pgfpathcurveto{\pgfqpoint{3.879621in}{2.393224in}}{\pgfqpoint{3.883571in}{2.391588in}}{\pgfqpoint{3.887689in}{2.391588in}}%
\pgfpathclose%
\pgfusepath{fill}%
\end{pgfscope}%
\begin{pgfscope}%
\pgfpathrectangle{\pgfqpoint{0.887500in}{0.275000in}}{\pgfqpoint{4.225000in}{4.225000in}}%
\pgfusepath{clip}%
\pgfsetbuttcap%
\pgfsetroundjoin%
\definecolor{currentfill}{rgb}{0.000000,0.000000,0.000000}%
\pgfsetfillcolor{currentfill}%
\pgfsetfillopacity{0.800000}%
\pgfsetlinewidth{0.000000pt}%
\definecolor{currentstroke}{rgb}{0.000000,0.000000,0.000000}%
\pgfsetstrokecolor{currentstroke}%
\pgfsetstrokeopacity{0.800000}%
\pgfsetdash{}{0pt}%
\pgfpathmoveto{\pgfqpoint{3.708697in}{2.502716in}}%
\pgfpathcurveto{\pgfqpoint{3.712815in}{2.502716in}}{\pgfqpoint{3.716765in}{2.504352in}}{\pgfqpoint{3.719677in}{2.507264in}}%
\pgfpathcurveto{\pgfqpoint{3.722589in}{2.510176in}}{\pgfqpoint{3.724225in}{2.514126in}}{\pgfqpoint{3.724225in}{2.518244in}}%
\pgfpathcurveto{\pgfqpoint{3.724225in}{2.522362in}}{\pgfqpoint{3.722589in}{2.526312in}}{\pgfqpoint{3.719677in}{2.529224in}}%
\pgfpathcurveto{\pgfqpoint{3.716765in}{2.532136in}}{\pgfqpoint{3.712815in}{2.533772in}}{\pgfqpoint{3.708697in}{2.533772in}}%
\pgfpathcurveto{\pgfqpoint{3.704579in}{2.533772in}}{\pgfqpoint{3.700629in}{2.532136in}}{\pgfqpoint{3.697717in}{2.529224in}}%
\pgfpathcurveto{\pgfqpoint{3.694805in}{2.526312in}}{\pgfqpoint{3.693169in}{2.522362in}}{\pgfqpoint{3.693169in}{2.518244in}}%
\pgfpathcurveto{\pgfqpoint{3.693169in}{2.514126in}}{\pgfqpoint{3.694805in}{2.510176in}}{\pgfqpoint{3.697717in}{2.507264in}}%
\pgfpathcurveto{\pgfqpoint{3.700629in}{2.504352in}}{\pgfqpoint{3.704579in}{2.502716in}}{\pgfqpoint{3.708697in}{2.502716in}}%
\pgfpathclose%
\pgfusepath{fill}%
\end{pgfscope}%
\begin{pgfscope}%
\pgfpathrectangle{\pgfqpoint{0.887500in}{0.275000in}}{\pgfqpoint{4.225000in}{4.225000in}}%
\pgfusepath{clip}%
\pgfsetbuttcap%
\pgfsetroundjoin%
\definecolor{currentfill}{rgb}{0.000000,0.000000,0.000000}%
\pgfsetfillcolor{currentfill}%
\pgfsetfillopacity{0.800000}%
\pgfsetlinewidth{0.000000pt}%
\definecolor{currentstroke}{rgb}{0.000000,0.000000,0.000000}%
\pgfsetstrokecolor{currentstroke}%
\pgfsetstrokeopacity{0.800000}%
\pgfsetdash{}{0pt}%
\pgfpathmoveto{\pgfqpoint{4.491679in}{1.894164in}}%
\pgfpathcurveto{\pgfqpoint{4.495797in}{1.894164in}}{\pgfqpoint{4.499747in}{1.895800in}}{\pgfqpoint{4.502659in}{1.898712in}}%
\pgfpathcurveto{\pgfqpoint{4.505571in}{1.901624in}}{\pgfqpoint{4.507207in}{1.905574in}}{\pgfqpoint{4.507207in}{1.909692in}}%
\pgfpathcurveto{\pgfqpoint{4.507207in}{1.913810in}}{\pgfqpoint{4.505571in}{1.917760in}}{\pgfqpoint{4.502659in}{1.920672in}}%
\pgfpathcurveto{\pgfqpoint{4.499747in}{1.923584in}}{\pgfqpoint{4.495797in}{1.925220in}}{\pgfqpoint{4.491679in}{1.925220in}}%
\pgfpathcurveto{\pgfqpoint{4.487561in}{1.925220in}}{\pgfqpoint{4.483611in}{1.923584in}}{\pgfqpoint{4.480699in}{1.920672in}}%
\pgfpathcurveto{\pgfqpoint{4.477787in}{1.917760in}}{\pgfqpoint{4.476151in}{1.913810in}}{\pgfqpoint{4.476151in}{1.909692in}}%
\pgfpathcurveto{\pgfqpoint{4.476151in}{1.905574in}}{\pgfqpoint{4.477787in}{1.901624in}}{\pgfqpoint{4.480699in}{1.898712in}}%
\pgfpathcurveto{\pgfqpoint{4.483611in}{1.895800in}}{\pgfqpoint{4.487561in}{1.894164in}}{\pgfqpoint{4.491679in}{1.894164in}}%
\pgfpathclose%
\pgfusepath{fill}%
\end{pgfscope}%
\begin{pgfscope}%
\pgfpathrectangle{\pgfqpoint{0.887500in}{0.275000in}}{\pgfqpoint{4.225000in}{4.225000in}}%
\pgfusepath{clip}%
\pgfsetbuttcap%
\pgfsetroundjoin%
\definecolor{currentfill}{rgb}{0.000000,0.000000,0.000000}%
\pgfsetfillcolor{currentfill}%
\pgfsetfillopacity{0.800000}%
\pgfsetlinewidth{0.000000pt}%
\definecolor{currentstroke}{rgb}{0.000000,0.000000,0.000000}%
\pgfsetstrokecolor{currentstroke}%
\pgfsetstrokeopacity{0.800000}%
\pgfsetdash{}{0pt}%
\pgfpathmoveto{\pgfqpoint{2.129972in}{2.200711in}}%
\pgfpathcurveto{\pgfqpoint{2.134090in}{2.200711in}}{\pgfqpoint{2.138040in}{2.202348in}}{\pgfqpoint{2.140952in}{2.205260in}}%
\pgfpathcurveto{\pgfqpoint{2.143864in}{2.208172in}}{\pgfqpoint{2.145500in}{2.212122in}}{\pgfqpoint{2.145500in}{2.216240in}}%
\pgfpathcurveto{\pgfqpoint{2.145500in}{2.220358in}}{\pgfqpoint{2.143864in}{2.224308in}}{\pgfqpoint{2.140952in}{2.227220in}}%
\pgfpathcurveto{\pgfqpoint{2.138040in}{2.230132in}}{\pgfqpoint{2.134090in}{2.231768in}}{\pgfqpoint{2.129972in}{2.231768in}}%
\pgfpathcurveto{\pgfqpoint{2.125854in}{2.231768in}}{\pgfqpoint{2.121904in}{2.230132in}}{\pgfqpoint{2.118992in}{2.227220in}}%
\pgfpathcurveto{\pgfqpoint{2.116080in}{2.224308in}}{\pgfqpoint{2.114444in}{2.220358in}}{\pgfqpoint{2.114444in}{2.216240in}}%
\pgfpathcurveto{\pgfqpoint{2.114444in}{2.212122in}}{\pgfqpoint{2.116080in}{2.208172in}}{\pgfqpoint{2.118992in}{2.205260in}}%
\pgfpathcurveto{\pgfqpoint{2.121904in}{2.202348in}}{\pgfqpoint{2.125854in}{2.200711in}}{\pgfqpoint{2.129972in}{2.200711in}}%
\pgfpathclose%
\pgfusepath{fill}%
\end{pgfscope}%
\begin{pgfscope}%
\pgfpathrectangle{\pgfqpoint{0.887500in}{0.275000in}}{\pgfqpoint{4.225000in}{4.225000in}}%
\pgfusepath{clip}%
\pgfsetbuttcap%
\pgfsetroundjoin%
\definecolor{currentfill}{rgb}{0.000000,0.000000,0.000000}%
\pgfsetfillcolor{currentfill}%
\pgfsetfillopacity{0.800000}%
\pgfsetlinewidth{0.000000pt}%
\definecolor{currentstroke}{rgb}{0.000000,0.000000,0.000000}%
\pgfsetstrokecolor{currentstroke}%
\pgfsetstrokeopacity{0.800000}%
\pgfsetdash{}{0pt}%
\pgfpathmoveto{\pgfqpoint{3.350143in}{2.703430in}}%
\pgfpathcurveto{\pgfqpoint{3.354261in}{2.703430in}}{\pgfqpoint{3.358211in}{2.705066in}}{\pgfqpoint{3.361123in}{2.707978in}}%
\pgfpathcurveto{\pgfqpoint{3.364035in}{2.710890in}}{\pgfqpoint{3.365671in}{2.714840in}}{\pgfqpoint{3.365671in}{2.718958in}}%
\pgfpathcurveto{\pgfqpoint{3.365671in}{2.723076in}}{\pgfqpoint{3.364035in}{2.727026in}}{\pgfqpoint{3.361123in}{2.729938in}}%
\pgfpathcurveto{\pgfqpoint{3.358211in}{2.732850in}}{\pgfqpoint{3.354261in}{2.734486in}}{\pgfqpoint{3.350143in}{2.734486in}}%
\pgfpathcurveto{\pgfqpoint{3.346025in}{2.734486in}}{\pgfqpoint{3.342075in}{2.732850in}}{\pgfqpoint{3.339163in}{2.729938in}}%
\pgfpathcurveto{\pgfqpoint{3.336251in}{2.727026in}}{\pgfqpoint{3.334615in}{2.723076in}}{\pgfqpoint{3.334615in}{2.718958in}}%
\pgfpathcurveto{\pgfqpoint{3.334615in}{2.714840in}}{\pgfqpoint{3.336251in}{2.710890in}}{\pgfqpoint{3.339163in}{2.707978in}}%
\pgfpathcurveto{\pgfqpoint{3.342075in}{2.705066in}}{\pgfqpoint{3.346025in}{2.703430in}}{\pgfqpoint{3.350143in}{2.703430in}}%
\pgfpathclose%
\pgfusepath{fill}%
\end{pgfscope}%
\begin{pgfscope}%
\pgfpathrectangle{\pgfqpoint{0.887500in}{0.275000in}}{\pgfqpoint{4.225000in}{4.225000in}}%
\pgfusepath{clip}%
\pgfsetbuttcap%
\pgfsetroundjoin%
\definecolor{currentfill}{rgb}{0.000000,0.000000,0.000000}%
\pgfsetfillcolor{currentfill}%
\pgfsetfillopacity{0.800000}%
\pgfsetlinewidth{0.000000pt}%
\definecolor{currentstroke}{rgb}{0.000000,0.000000,0.000000}%
\pgfsetstrokecolor{currentstroke}%
\pgfsetstrokeopacity{0.800000}%
\pgfsetdash{}{0pt}%
\pgfpathmoveto{\pgfqpoint{3.529511in}{2.609146in}}%
\pgfpathcurveto{\pgfqpoint{3.533630in}{2.609146in}}{\pgfqpoint{3.537580in}{2.610782in}}{\pgfqpoint{3.540492in}{2.613694in}}%
\pgfpathcurveto{\pgfqpoint{3.543404in}{2.616606in}}{\pgfqpoint{3.545040in}{2.620556in}}{\pgfqpoint{3.545040in}{2.624674in}}%
\pgfpathcurveto{\pgfqpoint{3.545040in}{2.628792in}}{\pgfqpoint{3.543404in}{2.632742in}}{\pgfqpoint{3.540492in}{2.635654in}}%
\pgfpathcurveto{\pgfqpoint{3.537580in}{2.638566in}}{\pgfqpoint{3.533630in}{2.640203in}}{\pgfqpoint{3.529511in}{2.640203in}}%
\pgfpathcurveto{\pgfqpoint{3.525393in}{2.640203in}}{\pgfqpoint{3.521443in}{2.638566in}}{\pgfqpoint{3.518531in}{2.635654in}}%
\pgfpathcurveto{\pgfqpoint{3.515619in}{2.632742in}}{\pgfqpoint{3.513983in}{2.628792in}}{\pgfqpoint{3.513983in}{2.624674in}}%
\pgfpathcurveto{\pgfqpoint{3.513983in}{2.620556in}}{\pgfqpoint{3.515619in}{2.616606in}}{\pgfqpoint{3.518531in}{2.613694in}}%
\pgfpathcurveto{\pgfqpoint{3.521443in}{2.610782in}}{\pgfqpoint{3.525393in}{2.609146in}}{\pgfqpoint{3.529511in}{2.609146in}}%
\pgfpathclose%
\pgfusepath{fill}%
\end{pgfscope}%
\begin{pgfscope}%
\pgfpathrectangle{\pgfqpoint{0.887500in}{0.275000in}}{\pgfqpoint{4.225000in}{4.225000in}}%
\pgfusepath{clip}%
\pgfsetbuttcap%
\pgfsetroundjoin%
\definecolor{currentfill}{rgb}{0.000000,0.000000,0.000000}%
\pgfsetfillcolor{currentfill}%
\pgfsetfillopacity{0.800000}%
\pgfsetlinewidth{0.000000pt}%
\definecolor{currentstroke}{rgb}{0.000000,0.000000,0.000000}%
\pgfsetstrokecolor{currentstroke}%
\pgfsetstrokeopacity{0.800000}%
\pgfsetdash{}{0pt}%
\pgfpathmoveto{\pgfqpoint{2.781597in}{2.052521in}}%
\pgfpathcurveto{\pgfqpoint{2.785715in}{2.052521in}}{\pgfqpoint{2.789665in}{2.054157in}}{\pgfqpoint{2.792577in}{2.057069in}}%
\pgfpathcurveto{\pgfqpoint{2.795489in}{2.059981in}}{\pgfqpoint{2.797125in}{2.063931in}}{\pgfqpoint{2.797125in}{2.068049in}}%
\pgfpathcurveto{\pgfqpoint{2.797125in}{2.072167in}}{\pgfqpoint{2.795489in}{2.076117in}}{\pgfqpoint{2.792577in}{2.079029in}}%
\pgfpathcurveto{\pgfqpoint{2.789665in}{2.081941in}}{\pgfqpoint{2.785715in}{2.083577in}}{\pgfqpoint{2.781597in}{2.083577in}}%
\pgfpathcurveto{\pgfqpoint{2.777479in}{2.083577in}}{\pgfqpoint{2.773529in}{2.081941in}}{\pgfqpoint{2.770617in}{2.079029in}}%
\pgfpathcurveto{\pgfqpoint{2.767705in}{2.076117in}}{\pgfqpoint{2.766069in}{2.072167in}}{\pgfqpoint{2.766069in}{2.068049in}}%
\pgfpathcurveto{\pgfqpoint{2.766069in}{2.063931in}}{\pgfqpoint{2.767705in}{2.059981in}}{\pgfqpoint{2.770617in}{2.057069in}}%
\pgfpathcurveto{\pgfqpoint{2.773529in}{2.054157in}}{\pgfqpoint{2.777479in}{2.052521in}}{\pgfqpoint{2.781597in}{2.052521in}}%
\pgfpathclose%
\pgfusepath{fill}%
\end{pgfscope}%
\begin{pgfscope}%
\pgfpathrectangle{\pgfqpoint{0.887500in}{0.275000in}}{\pgfqpoint{4.225000in}{4.225000in}}%
\pgfusepath{clip}%
\pgfsetbuttcap%
\pgfsetroundjoin%
\definecolor{currentfill}{rgb}{0.000000,0.000000,0.000000}%
\pgfsetfillcolor{currentfill}%
\pgfsetfillopacity{0.800000}%
\pgfsetlinewidth{0.000000pt}%
\definecolor{currentstroke}{rgb}{0.000000,0.000000,0.000000}%
\pgfsetstrokecolor{currentstroke}%
\pgfsetstrokeopacity{0.800000}%
\pgfsetdash{}{0pt}%
\pgfpathmoveto{\pgfqpoint{3.236471in}{2.578735in}}%
\pgfpathcurveto{\pgfqpoint{3.240589in}{2.578735in}}{\pgfqpoint{3.244539in}{2.580371in}}{\pgfqpoint{3.247451in}{2.583283in}}%
\pgfpathcurveto{\pgfqpoint{3.250363in}{2.586195in}}{\pgfqpoint{3.252000in}{2.590145in}}{\pgfqpoint{3.252000in}{2.594263in}}%
\pgfpathcurveto{\pgfqpoint{3.252000in}{2.598381in}}{\pgfqpoint{3.250363in}{2.602331in}}{\pgfqpoint{3.247451in}{2.605243in}}%
\pgfpathcurveto{\pgfqpoint{3.244539in}{2.608155in}}{\pgfqpoint{3.240589in}{2.609791in}}{\pgfqpoint{3.236471in}{2.609791in}}%
\pgfpathcurveto{\pgfqpoint{3.232353in}{2.609791in}}{\pgfqpoint{3.228403in}{2.608155in}}{\pgfqpoint{3.225491in}{2.605243in}}%
\pgfpathcurveto{\pgfqpoint{3.222579in}{2.602331in}}{\pgfqpoint{3.220943in}{2.598381in}}{\pgfqpoint{3.220943in}{2.594263in}}%
\pgfpathcurveto{\pgfqpoint{3.220943in}{2.590145in}}{\pgfqpoint{3.222579in}{2.586195in}}{\pgfqpoint{3.225491in}{2.583283in}}%
\pgfpathcurveto{\pgfqpoint{3.228403in}{2.580371in}}{\pgfqpoint{3.232353in}{2.578735in}}{\pgfqpoint{3.236471in}{2.578735in}}%
\pgfpathclose%
\pgfusepath{fill}%
\end{pgfscope}%
\begin{pgfscope}%
\pgfpathrectangle{\pgfqpoint{0.887500in}{0.275000in}}{\pgfqpoint{4.225000in}{4.225000in}}%
\pgfusepath{clip}%
\pgfsetbuttcap%
\pgfsetroundjoin%
\definecolor{currentfill}{rgb}{0.000000,0.000000,0.000000}%
\pgfsetfillcolor{currentfill}%
\pgfsetfillopacity{0.800000}%
\pgfsetlinewidth{0.000000pt}%
\definecolor{currentstroke}{rgb}{0.000000,0.000000,0.000000}%
\pgfsetstrokecolor{currentstroke}%
\pgfsetstrokeopacity{0.800000}%
\pgfsetdash{}{0pt}%
\pgfpathmoveto{\pgfqpoint{4.312946in}{2.008664in}}%
\pgfpathcurveto{\pgfqpoint{4.317064in}{2.008664in}}{\pgfqpoint{4.321014in}{2.010300in}}{\pgfqpoint{4.323926in}{2.013212in}}%
\pgfpathcurveto{\pgfqpoint{4.326838in}{2.016124in}}{\pgfqpoint{4.328474in}{2.020074in}}{\pgfqpoint{4.328474in}{2.024192in}}%
\pgfpathcurveto{\pgfqpoint{4.328474in}{2.028310in}}{\pgfqpoint{4.326838in}{2.032260in}}{\pgfqpoint{4.323926in}{2.035172in}}%
\pgfpathcurveto{\pgfqpoint{4.321014in}{2.038084in}}{\pgfqpoint{4.317064in}{2.039721in}}{\pgfqpoint{4.312946in}{2.039721in}}%
\pgfpathcurveto{\pgfqpoint{4.308828in}{2.039721in}}{\pgfqpoint{4.304878in}{2.038084in}}{\pgfqpoint{4.301966in}{2.035172in}}%
\pgfpathcurveto{\pgfqpoint{4.299054in}{2.032260in}}{\pgfqpoint{4.297418in}{2.028310in}}{\pgfqpoint{4.297418in}{2.024192in}}%
\pgfpathcurveto{\pgfqpoint{4.297418in}{2.020074in}}{\pgfqpoint{4.299054in}{2.016124in}}{\pgfqpoint{4.301966in}{2.013212in}}%
\pgfpathcurveto{\pgfqpoint{4.304878in}{2.010300in}}{\pgfqpoint{4.308828in}{2.008664in}}{\pgfqpoint{4.312946in}{2.008664in}}%
\pgfpathclose%
\pgfusepath{fill}%
\end{pgfscope}%
\begin{pgfscope}%
\pgfpathrectangle{\pgfqpoint{0.887500in}{0.275000in}}{\pgfqpoint{4.225000in}{4.225000in}}%
\pgfusepath{clip}%
\pgfsetbuttcap%
\pgfsetroundjoin%
\definecolor{currentfill}{rgb}{0.000000,0.000000,0.000000}%
\pgfsetfillcolor{currentfill}%
\pgfsetfillopacity{0.800000}%
\pgfsetlinewidth{0.000000pt}%
\definecolor{currentstroke}{rgb}{0.000000,0.000000,0.000000}%
\pgfsetstrokecolor{currentstroke}%
\pgfsetstrokeopacity{0.800000}%
\pgfsetdash{}{0pt}%
\pgfpathmoveto{\pgfqpoint{1.656958in}{2.280161in}}%
\pgfpathcurveto{\pgfqpoint{1.661077in}{2.280161in}}{\pgfqpoint{1.665027in}{2.281797in}}{\pgfqpoint{1.667939in}{2.284709in}}%
\pgfpathcurveto{\pgfqpoint{1.670851in}{2.287621in}}{\pgfqpoint{1.672487in}{2.291571in}}{\pgfqpoint{1.672487in}{2.295689in}}%
\pgfpathcurveto{\pgfqpoint{1.672487in}{2.299807in}}{\pgfqpoint{1.670851in}{2.303757in}}{\pgfqpoint{1.667939in}{2.306669in}}%
\pgfpathcurveto{\pgfqpoint{1.665027in}{2.309581in}}{\pgfqpoint{1.661077in}{2.311218in}}{\pgfqpoint{1.656958in}{2.311218in}}%
\pgfpathcurveto{\pgfqpoint{1.652840in}{2.311218in}}{\pgfqpoint{1.648890in}{2.309581in}}{\pgfqpoint{1.645978in}{2.306669in}}%
\pgfpathcurveto{\pgfqpoint{1.643066in}{2.303757in}}{\pgfqpoint{1.641430in}{2.299807in}}{\pgfqpoint{1.641430in}{2.295689in}}%
\pgfpathcurveto{\pgfqpoint{1.641430in}{2.291571in}}{\pgfqpoint{1.643066in}{2.287621in}}{\pgfqpoint{1.645978in}{2.284709in}}%
\pgfpathcurveto{\pgfqpoint{1.648890in}{2.281797in}}{\pgfqpoint{1.652840in}{2.280161in}}{\pgfqpoint{1.656958in}{2.280161in}}%
\pgfpathclose%
\pgfusepath{fill}%
\end{pgfscope}%
\begin{pgfscope}%
\pgfpathrectangle{\pgfqpoint{0.887500in}{0.275000in}}{\pgfqpoint{4.225000in}{4.225000in}}%
\pgfusepath{clip}%
\pgfsetbuttcap%
\pgfsetroundjoin%
\definecolor{currentfill}{rgb}{0.000000,0.000000,0.000000}%
\pgfsetfillcolor{currentfill}%
\pgfsetfillopacity{0.800000}%
\pgfsetlinewidth{0.000000pt}%
\definecolor{currentstroke}{rgb}{0.000000,0.000000,0.000000}%
\pgfsetstrokecolor{currentstroke}%
\pgfsetstrokeopacity{0.800000}%
\pgfsetdash{}{0pt}%
\pgfpathmoveto{\pgfqpoint{3.188464in}{2.237027in}}%
\pgfpathcurveto{\pgfqpoint{3.192582in}{2.237027in}}{\pgfqpoint{3.196532in}{2.238663in}}{\pgfqpoint{3.199444in}{2.241575in}}%
\pgfpathcurveto{\pgfqpoint{3.202356in}{2.244487in}}{\pgfqpoint{3.203993in}{2.248437in}}{\pgfqpoint{3.203993in}{2.252556in}}%
\pgfpathcurveto{\pgfqpoint{3.203993in}{2.256674in}}{\pgfqpoint{3.202356in}{2.260624in}}{\pgfqpoint{3.199444in}{2.263536in}}%
\pgfpathcurveto{\pgfqpoint{3.196532in}{2.266448in}}{\pgfqpoint{3.192582in}{2.268084in}}{\pgfqpoint{3.188464in}{2.268084in}}%
\pgfpathcurveto{\pgfqpoint{3.184346in}{2.268084in}}{\pgfqpoint{3.180396in}{2.266448in}}{\pgfqpoint{3.177484in}{2.263536in}}%
\pgfpathcurveto{\pgfqpoint{3.174572in}{2.260624in}}{\pgfqpoint{3.172936in}{2.256674in}}{\pgfqpoint{3.172936in}{2.252556in}}%
\pgfpathcurveto{\pgfqpoint{3.172936in}{2.248437in}}{\pgfqpoint{3.174572in}{2.244487in}}{\pgfqpoint{3.177484in}{2.241575in}}%
\pgfpathcurveto{\pgfqpoint{3.180396in}{2.238663in}}{\pgfqpoint{3.184346in}{2.237027in}}{\pgfqpoint{3.188464in}{2.237027in}}%
\pgfpathclose%
\pgfusepath{fill}%
\end{pgfscope}%
\begin{pgfscope}%
\pgfpathrectangle{\pgfqpoint{0.887500in}{0.275000in}}{\pgfqpoint{4.225000in}{4.225000in}}%
\pgfusepath{clip}%
\pgfsetbuttcap%
\pgfsetroundjoin%
\definecolor{currentfill}{rgb}{0.000000,0.000000,0.000000}%
\pgfsetfillcolor{currentfill}%
\pgfsetfillopacity{0.800000}%
\pgfsetlinewidth{0.000000pt}%
\definecolor{currentstroke}{rgb}{0.000000,0.000000,0.000000}%
\pgfsetstrokecolor{currentstroke}%
\pgfsetstrokeopacity{0.800000}%
\pgfsetdash{}{0pt}%
\pgfpathmoveto{\pgfqpoint{4.134055in}{2.123616in}}%
\pgfpathcurveto{\pgfqpoint{4.138173in}{2.123616in}}{\pgfqpoint{4.142123in}{2.125252in}}{\pgfqpoint{4.145035in}{2.128164in}}%
\pgfpathcurveto{\pgfqpoint{4.147947in}{2.131076in}}{\pgfqpoint{4.149583in}{2.135026in}}{\pgfqpoint{4.149583in}{2.139144in}}%
\pgfpathcurveto{\pgfqpoint{4.149583in}{2.143262in}}{\pgfqpoint{4.147947in}{2.147212in}}{\pgfqpoint{4.145035in}{2.150124in}}%
\pgfpathcurveto{\pgfqpoint{4.142123in}{2.153036in}}{\pgfqpoint{4.138173in}{2.154672in}}{\pgfqpoint{4.134055in}{2.154672in}}%
\pgfpathcurveto{\pgfqpoint{4.129937in}{2.154672in}}{\pgfqpoint{4.125987in}{2.153036in}}{\pgfqpoint{4.123075in}{2.150124in}}%
\pgfpathcurveto{\pgfqpoint{4.120163in}{2.147212in}}{\pgfqpoint{4.118527in}{2.143262in}}{\pgfqpoint{4.118527in}{2.139144in}}%
\pgfpathcurveto{\pgfqpoint{4.118527in}{2.135026in}}{\pgfqpoint{4.120163in}{2.131076in}}{\pgfqpoint{4.123075in}{2.128164in}}%
\pgfpathcurveto{\pgfqpoint{4.125987in}{2.125252in}}{\pgfqpoint{4.129937in}{2.123616in}}{\pgfqpoint{4.134055in}{2.123616in}}%
\pgfpathclose%
\pgfusepath{fill}%
\end{pgfscope}%
\begin{pgfscope}%
\pgfpathrectangle{\pgfqpoint{0.887500in}{0.275000in}}{\pgfqpoint{4.225000in}{4.225000in}}%
\pgfusepath{clip}%
\pgfsetbuttcap%
\pgfsetroundjoin%
\definecolor{currentfill}{rgb}{0.000000,0.000000,0.000000}%
\pgfsetfillcolor{currentfill}%
\pgfsetfillopacity{0.800000}%
\pgfsetlinewidth{0.000000pt}%
\definecolor{currentstroke}{rgb}{0.000000,0.000000,0.000000}%
\pgfsetstrokecolor{currentstroke}%
\pgfsetstrokeopacity{0.800000}%
\pgfsetdash{}{0pt}%
\pgfpathmoveto{\pgfqpoint{2.308596in}{2.141718in}}%
\pgfpathcurveto{\pgfqpoint{2.312715in}{2.141718in}}{\pgfqpoint{2.316665in}{2.143354in}}{\pgfqpoint{2.319577in}{2.146266in}}%
\pgfpathcurveto{\pgfqpoint{2.322488in}{2.149178in}}{\pgfqpoint{2.324125in}{2.153128in}}{\pgfqpoint{2.324125in}{2.157246in}}%
\pgfpathcurveto{\pgfqpoint{2.324125in}{2.161364in}}{\pgfqpoint{2.322488in}{2.165314in}}{\pgfqpoint{2.319577in}{2.168226in}}%
\pgfpathcurveto{\pgfqpoint{2.316665in}{2.171138in}}{\pgfqpoint{2.312715in}{2.172774in}}{\pgfqpoint{2.308596in}{2.172774in}}%
\pgfpathcurveto{\pgfqpoint{2.304478in}{2.172774in}}{\pgfqpoint{2.300528in}{2.171138in}}{\pgfqpoint{2.297616in}{2.168226in}}%
\pgfpathcurveto{\pgfqpoint{2.294704in}{2.165314in}}{\pgfqpoint{2.293068in}{2.161364in}}{\pgfqpoint{2.293068in}{2.157246in}}%
\pgfpathcurveto{\pgfqpoint{2.293068in}{2.153128in}}{\pgfqpoint{2.294704in}{2.149178in}}{\pgfqpoint{2.297616in}{2.146266in}}%
\pgfpathcurveto{\pgfqpoint{2.300528in}{2.143354in}}{\pgfqpoint{2.304478in}{2.141718in}}{\pgfqpoint{2.308596in}{2.141718in}}%
\pgfpathclose%
\pgfusepath{fill}%
\end{pgfscope}%
\begin{pgfscope}%
\pgfpathrectangle{\pgfqpoint{0.887500in}{0.275000in}}{\pgfqpoint{4.225000in}{4.225000in}}%
\pgfusepath{clip}%
\pgfsetbuttcap%
\pgfsetroundjoin%
\definecolor{currentfill}{rgb}{0.000000,0.000000,0.000000}%
\pgfsetfillcolor{currentfill}%
\pgfsetfillopacity{0.800000}%
\pgfsetlinewidth{0.000000pt}%
\definecolor{currentstroke}{rgb}{0.000000,0.000000,0.000000}%
\pgfsetstrokecolor{currentstroke}%
\pgfsetstrokeopacity{0.800000}%
\pgfsetdash{}{0pt}%
\pgfpathmoveto{\pgfqpoint{2.960899in}{1.985151in}}%
\pgfpathcurveto{\pgfqpoint{2.965017in}{1.985151in}}{\pgfqpoint{2.968967in}{1.986787in}}{\pgfqpoint{2.971879in}{1.989699in}}%
\pgfpathcurveto{\pgfqpoint{2.974791in}{1.992611in}}{\pgfqpoint{2.976427in}{1.996561in}}{\pgfqpoint{2.976427in}{2.000679in}}%
\pgfpathcurveto{\pgfqpoint{2.976427in}{2.004797in}}{\pgfqpoint{2.974791in}{2.008747in}}{\pgfqpoint{2.971879in}{2.011659in}}%
\pgfpathcurveto{\pgfqpoint{2.968967in}{2.014571in}}{\pgfqpoint{2.965017in}{2.016207in}}{\pgfqpoint{2.960899in}{2.016207in}}%
\pgfpathcurveto{\pgfqpoint{2.956780in}{2.016207in}}{\pgfqpoint{2.952830in}{2.014571in}}{\pgfqpoint{2.949918in}{2.011659in}}%
\pgfpathcurveto{\pgfqpoint{2.947006in}{2.008747in}}{\pgfqpoint{2.945370in}{2.004797in}}{\pgfqpoint{2.945370in}{2.000679in}}%
\pgfpathcurveto{\pgfqpoint{2.945370in}{1.996561in}}{\pgfqpoint{2.947006in}{1.992611in}}{\pgfqpoint{2.949918in}{1.989699in}}%
\pgfpathcurveto{\pgfqpoint{2.952830in}{1.986787in}}{\pgfqpoint{2.956780in}{1.985151in}}{\pgfqpoint{2.960899in}{1.985151in}}%
\pgfpathclose%
\pgfusepath{fill}%
\end{pgfscope}%
\begin{pgfscope}%
\pgfpathrectangle{\pgfqpoint{0.887500in}{0.275000in}}{\pgfqpoint{4.225000in}{4.225000in}}%
\pgfusepath{clip}%
\pgfsetbuttcap%
\pgfsetroundjoin%
\definecolor{currentfill}{rgb}{0.000000,0.000000,0.000000}%
\pgfsetfillcolor{currentfill}%
\pgfsetfillopacity{0.800000}%
\pgfsetlinewidth{0.000000pt}%
\definecolor{currentstroke}{rgb}{0.000000,0.000000,0.000000}%
\pgfsetstrokecolor{currentstroke}%
\pgfsetstrokeopacity{0.800000}%
\pgfsetdash{}{0pt}%
\pgfpathmoveto{\pgfqpoint{3.302322in}{2.382585in}}%
\pgfpathcurveto{\pgfqpoint{3.306440in}{2.382585in}}{\pgfqpoint{3.310390in}{2.384221in}}{\pgfqpoint{3.313302in}{2.387133in}}%
\pgfpathcurveto{\pgfqpoint{3.316214in}{2.390045in}}{\pgfqpoint{3.317850in}{2.393995in}}{\pgfqpoint{3.317850in}{2.398113in}}%
\pgfpathcurveto{\pgfqpoint{3.317850in}{2.402231in}}{\pgfqpoint{3.316214in}{2.406181in}}{\pgfqpoint{3.313302in}{2.409093in}}%
\pgfpathcurveto{\pgfqpoint{3.310390in}{2.412005in}}{\pgfqpoint{3.306440in}{2.413642in}}{\pgfqpoint{3.302322in}{2.413642in}}%
\pgfpathcurveto{\pgfqpoint{3.298204in}{2.413642in}}{\pgfqpoint{3.294254in}{2.412005in}}{\pgfqpoint{3.291342in}{2.409093in}}%
\pgfpathcurveto{\pgfqpoint{3.288430in}{2.406181in}}{\pgfqpoint{3.286794in}{2.402231in}}{\pgfqpoint{3.286794in}{2.398113in}}%
\pgfpathcurveto{\pgfqpoint{3.286794in}{2.393995in}}{\pgfqpoint{3.288430in}{2.390045in}}{\pgfqpoint{3.291342in}{2.387133in}}%
\pgfpathcurveto{\pgfqpoint{3.294254in}{2.384221in}}{\pgfqpoint{3.298204in}{2.382585in}}{\pgfqpoint{3.302322in}{2.382585in}}%
\pgfpathclose%
\pgfusepath{fill}%
\end{pgfscope}%
\begin{pgfscope}%
\pgfpathrectangle{\pgfqpoint{0.887500in}{0.275000in}}{\pgfqpoint{4.225000in}{4.225000in}}%
\pgfusepath{clip}%
\pgfsetbuttcap%
\pgfsetroundjoin%
\definecolor{currentfill}{rgb}{0.000000,0.000000,0.000000}%
\pgfsetfillcolor{currentfill}%
\pgfsetfillopacity{0.800000}%
\pgfsetlinewidth{0.000000pt}%
\definecolor{currentstroke}{rgb}{0.000000,0.000000,0.000000}%
\pgfsetstrokecolor{currentstroke}%
\pgfsetstrokeopacity{0.800000}%
\pgfsetdash{}{0pt}%
\pgfpathmoveto{\pgfqpoint{3.955016in}{2.239713in}}%
\pgfpathcurveto{\pgfqpoint{3.959134in}{2.239713in}}{\pgfqpoint{3.963084in}{2.241349in}}{\pgfqpoint{3.965996in}{2.244261in}}%
\pgfpathcurveto{\pgfqpoint{3.968908in}{2.247173in}}{\pgfqpoint{3.970544in}{2.251123in}}{\pgfqpoint{3.970544in}{2.255241in}}%
\pgfpathcurveto{\pgfqpoint{3.970544in}{2.259359in}}{\pgfqpoint{3.968908in}{2.263309in}}{\pgfqpoint{3.965996in}{2.266221in}}%
\pgfpathcurveto{\pgfqpoint{3.963084in}{2.269133in}}{\pgfqpoint{3.959134in}{2.270769in}}{\pgfqpoint{3.955016in}{2.270769in}}%
\pgfpathcurveto{\pgfqpoint{3.950897in}{2.270769in}}{\pgfqpoint{3.946947in}{2.269133in}}{\pgfqpoint{3.944035in}{2.266221in}}%
\pgfpathcurveto{\pgfqpoint{3.941123in}{2.263309in}}{\pgfqpoint{3.939487in}{2.259359in}}{\pgfqpoint{3.939487in}{2.255241in}}%
\pgfpathcurveto{\pgfqpoint{3.939487in}{2.251123in}}{\pgfqpoint{3.941123in}{2.247173in}}{\pgfqpoint{3.944035in}{2.244261in}}%
\pgfpathcurveto{\pgfqpoint{3.946947in}{2.241349in}}{\pgfqpoint{3.950897in}{2.239713in}}{\pgfqpoint{3.955016in}{2.239713in}}%
\pgfpathclose%
\pgfusepath{fill}%
\end{pgfscope}%
\begin{pgfscope}%
\pgfpathrectangle{\pgfqpoint{0.887500in}{0.275000in}}{\pgfqpoint{4.225000in}{4.225000in}}%
\pgfusepath{clip}%
\pgfsetbuttcap%
\pgfsetroundjoin%
\definecolor{currentfill}{rgb}{0.000000,0.000000,0.000000}%
\pgfsetfillcolor{currentfill}%
\pgfsetfillopacity{0.800000}%
\pgfsetlinewidth{0.000000pt}%
\definecolor{currentstroke}{rgb}{0.000000,0.000000,0.000000}%
\pgfsetstrokecolor{currentstroke}%
\pgfsetstrokeopacity{0.800000}%
\pgfsetdash{}{0pt}%
\pgfpathmoveto{\pgfqpoint{1.835142in}{2.223714in}}%
\pgfpathcurveto{\pgfqpoint{1.839260in}{2.223714in}}{\pgfqpoint{1.843210in}{2.225350in}}{\pgfqpoint{1.846122in}{2.228262in}}%
\pgfpathcurveto{\pgfqpoint{1.849034in}{2.231174in}}{\pgfqpoint{1.850670in}{2.235124in}}{\pgfqpoint{1.850670in}{2.239243in}}%
\pgfpathcurveto{\pgfqpoint{1.850670in}{2.243361in}}{\pgfqpoint{1.849034in}{2.247311in}}{\pgfqpoint{1.846122in}{2.250223in}}%
\pgfpathcurveto{\pgfqpoint{1.843210in}{2.253135in}}{\pgfqpoint{1.839260in}{2.254771in}}{\pgfqpoint{1.835142in}{2.254771in}}%
\pgfpathcurveto{\pgfqpoint{1.831024in}{2.254771in}}{\pgfqpoint{1.827074in}{2.253135in}}{\pgfqpoint{1.824162in}{2.250223in}}%
\pgfpathcurveto{\pgfqpoint{1.821250in}{2.247311in}}{\pgfqpoint{1.819613in}{2.243361in}}{\pgfqpoint{1.819613in}{2.239243in}}%
\pgfpathcurveto{\pgfqpoint{1.819613in}{2.235124in}}{\pgfqpoint{1.821250in}{2.231174in}}{\pgfqpoint{1.824162in}{2.228262in}}%
\pgfpathcurveto{\pgfqpoint{1.827074in}{2.225350in}}{\pgfqpoint{1.831024in}{2.223714in}}{\pgfqpoint{1.835142in}{2.223714in}}%
\pgfpathclose%
\pgfusepath{fill}%
\end{pgfscope}%
\begin{pgfscope}%
\pgfpathrectangle{\pgfqpoint{0.887500in}{0.275000in}}{\pgfqpoint{4.225000in}{4.225000in}}%
\pgfusepath{clip}%
\pgfsetbuttcap%
\pgfsetroundjoin%
\definecolor{currentfill}{rgb}{0.000000,0.000000,0.000000}%
\pgfsetfillcolor{currentfill}%
\pgfsetfillopacity{0.800000}%
\pgfsetlinewidth{0.000000pt}%
\definecolor{currentstroke}{rgb}{0.000000,0.000000,0.000000}%
\pgfsetstrokecolor{currentstroke}%
\pgfsetstrokeopacity{0.800000}%
\pgfsetdash{}{0pt}%
\pgfpathmoveto{\pgfqpoint{3.140456in}{1.914089in}}%
\pgfpathcurveto{\pgfqpoint{3.144575in}{1.914089in}}{\pgfqpoint{3.148525in}{1.915725in}}{\pgfqpoint{3.151437in}{1.918637in}}%
\pgfpathcurveto{\pgfqpoint{3.154349in}{1.921549in}}{\pgfqpoint{3.155985in}{1.925499in}}{\pgfqpoint{3.155985in}{1.929617in}}%
\pgfpathcurveto{\pgfqpoint{3.155985in}{1.933735in}}{\pgfqpoint{3.154349in}{1.937685in}}{\pgfqpoint{3.151437in}{1.940597in}}%
\pgfpathcurveto{\pgfqpoint{3.148525in}{1.943509in}}{\pgfqpoint{3.144575in}{1.945145in}}{\pgfqpoint{3.140456in}{1.945145in}}%
\pgfpathcurveto{\pgfqpoint{3.136338in}{1.945145in}}{\pgfqpoint{3.132388in}{1.943509in}}{\pgfqpoint{3.129476in}{1.940597in}}%
\pgfpathcurveto{\pgfqpoint{3.126564in}{1.937685in}}{\pgfqpoint{3.124928in}{1.933735in}}{\pgfqpoint{3.124928in}{1.929617in}}%
\pgfpathcurveto{\pgfqpoint{3.124928in}{1.925499in}}{\pgfqpoint{3.126564in}{1.921549in}}{\pgfqpoint{3.129476in}{1.918637in}}%
\pgfpathcurveto{\pgfqpoint{3.132388in}{1.915725in}}{\pgfqpoint{3.136338in}{1.914089in}}{\pgfqpoint{3.140456in}{1.914089in}}%
\pgfpathclose%
\pgfusepath{fill}%
\end{pgfscope}%
\begin{pgfscope}%
\pgfpathrectangle{\pgfqpoint{0.887500in}{0.275000in}}{\pgfqpoint{4.225000in}{4.225000in}}%
\pgfusepath{clip}%
\pgfsetbuttcap%
\pgfsetroundjoin%
\definecolor{currentfill}{rgb}{0.000000,0.000000,0.000000}%
\pgfsetfillcolor{currentfill}%
\pgfsetfillopacity{0.800000}%
\pgfsetlinewidth{0.000000pt}%
\definecolor{currentstroke}{rgb}{0.000000,0.000000,0.000000}%
\pgfsetstrokecolor{currentstroke}%
\pgfsetstrokeopacity{0.800000}%
\pgfsetdash{}{0pt}%
\pgfpathmoveto{\pgfqpoint{3.416351in}{2.538152in}}%
\pgfpathcurveto{\pgfqpoint{3.420469in}{2.538152in}}{\pgfqpoint{3.424419in}{2.539788in}}{\pgfqpoint{3.427331in}{2.542700in}}%
\pgfpathcurveto{\pgfqpoint{3.430243in}{2.545612in}}{\pgfqpoint{3.431879in}{2.549562in}}{\pgfqpoint{3.431879in}{2.553680in}}%
\pgfpathcurveto{\pgfqpoint{3.431879in}{2.557799in}}{\pgfqpoint{3.430243in}{2.561749in}}{\pgfqpoint{3.427331in}{2.564661in}}%
\pgfpathcurveto{\pgfqpoint{3.424419in}{2.567573in}}{\pgfqpoint{3.420469in}{2.569209in}}{\pgfqpoint{3.416351in}{2.569209in}}%
\pgfpathcurveto{\pgfqpoint{3.412233in}{2.569209in}}{\pgfqpoint{3.408283in}{2.567573in}}{\pgfqpoint{3.405371in}{2.564661in}}%
\pgfpathcurveto{\pgfqpoint{3.402459in}{2.561749in}}{\pgfqpoint{3.400823in}{2.557799in}}{\pgfqpoint{3.400823in}{2.553680in}}%
\pgfpathcurveto{\pgfqpoint{3.400823in}{2.549562in}}{\pgfqpoint{3.402459in}{2.545612in}}{\pgfqpoint{3.405371in}{2.542700in}}%
\pgfpathcurveto{\pgfqpoint{3.408283in}{2.539788in}}{\pgfqpoint{3.412233in}{2.538152in}}{\pgfqpoint{3.416351in}{2.538152in}}%
\pgfpathclose%
\pgfusepath{fill}%
\end{pgfscope}%
\begin{pgfscope}%
\pgfpathrectangle{\pgfqpoint{0.887500in}{0.275000in}}{\pgfqpoint{4.225000in}{4.225000in}}%
\pgfusepath{clip}%
\pgfsetbuttcap%
\pgfsetroundjoin%
\definecolor{currentfill}{rgb}{0.000000,0.000000,0.000000}%
\pgfsetfillcolor{currentfill}%
\pgfsetfillopacity{0.800000}%
\pgfsetlinewidth{0.000000pt}%
\definecolor{currentstroke}{rgb}{0.000000,0.000000,0.000000}%
\pgfsetstrokecolor{currentstroke}%
\pgfsetstrokeopacity{0.800000}%
\pgfsetdash{}{0pt}%
\pgfpathmoveto{\pgfqpoint{3.254366in}{2.052623in}}%
\pgfpathcurveto{\pgfqpoint{3.258485in}{2.052623in}}{\pgfqpoint{3.262435in}{2.054259in}}{\pgfqpoint{3.265347in}{2.057171in}}%
\pgfpathcurveto{\pgfqpoint{3.268258in}{2.060083in}}{\pgfqpoint{3.269895in}{2.064033in}}{\pgfqpoint{3.269895in}{2.068151in}}%
\pgfpathcurveto{\pgfqpoint{3.269895in}{2.072270in}}{\pgfqpoint{3.268258in}{2.076220in}}{\pgfqpoint{3.265347in}{2.079132in}}%
\pgfpathcurveto{\pgfqpoint{3.262435in}{2.082044in}}{\pgfqpoint{3.258485in}{2.083680in}}{\pgfqpoint{3.254366in}{2.083680in}}%
\pgfpathcurveto{\pgfqpoint{3.250248in}{2.083680in}}{\pgfqpoint{3.246298in}{2.082044in}}{\pgfqpoint{3.243386in}{2.079132in}}%
\pgfpathcurveto{\pgfqpoint{3.240474in}{2.076220in}}{\pgfqpoint{3.238838in}{2.072270in}}{\pgfqpoint{3.238838in}{2.068151in}}%
\pgfpathcurveto{\pgfqpoint{3.238838in}{2.064033in}}{\pgfqpoint{3.240474in}{2.060083in}}{\pgfqpoint{3.243386in}{2.057171in}}%
\pgfpathcurveto{\pgfqpoint{3.246298in}{2.054259in}}{\pgfqpoint{3.250248in}{2.052623in}}{\pgfqpoint{3.254366in}{2.052623in}}%
\pgfpathclose%
\pgfusepath{fill}%
\end{pgfscope}%
\begin{pgfscope}%
\pgfpathrectangle{\pgfqpoint{0.887500in}{0.275000in}}{\pgfqpoint{4.225000in}{4.225000in}}%
\pgfusepath{clip}%
\pgfsetbuttcap%
\pgfsetroundjoin%
\definecolor{currentfill}{rgb}{0.000000,0.000000,0.000000}%
\pgfsetfillcolor{currentfill}%
\pgfsetfillopacity{0.800000}%
\pgfsetlinewidth{0.000000pt}%
\definecolor{currentstroke}{rgb}{0.000000,0.000000,0.000000}%
\pgfsetstrokecolor{currentstroke}%
\pgfsetstrokeopacity{0.800000}%
\pgfsetdash{}{0pt}%
\pgfpathmoveto{\pgfqpoint{2.487583in}{2.080980in}}%
\pgfpathcurveto{\pgfqpoint{2.491701in}{2.080980in}}{\pgfqpoint{2.495651in}{2.082616in}}{\pgfqpoint{2.498563in}{2.085528in}}%
\pgfpathcurveto{\pgfqpoint{2.501475in}{2.088440in}}{\pgfqpoint{2.503111in}{2.092390in}}{\pgfqpoint{2.503111in}{2.096508in}}%
\pgfpathcurveto{\pgfqpoint{2.503111in}{2.100626in}}{\pgfqpoint{2.501475in}{2.104576in}}{\pgfqpoint{2.498563in}{2.107488in}}%
\pgfpathcurveto{\pgfqpoint{2.495651in}{2.110400in}}{\pgfqpoint{2.491701in}{2.112036in}}{\pgfqpoint{2.487583in}{2.112036in}}%
\pgfpathcurveto{\pgfqpoint{2.483465in}{2.112036in}}{\pgfqpoint{2.479515in}{2.110400in}}{\pgfqpoint{2.476603in}{2.107488in}}%
\pgfpathcurveto{\pgfqpoint{2.473691in}{2.104576in}}{\pgfqpoint{2.472055in}{2.100626in}}{\pgfqpoint{2.472055in}{2.096508in}}%
\pgfpathcurveto{\pgfqpoint{2.472055in}{2.092390in}}{\pgfqpoint{2.473691in}{2.088440in}}{\pgfqpoint{2.476603in}{2.085528in}}%
\pgfpathcurveto{\pgfqpoint{2.479515in}{2.082616in}}{\pgfqpoint{2.483465in}{2.080980in}}{\pgfqpoint{2.487583in}{2.080980in}}%
\pgfpathclose%
\pgfusepath{fill}%
\end{pgfscope}%
\begin{pgfscope}%
\pgfpathrectangle{\pgfqpoint{0.887500in}{0.275000in}}{\pgfqpoint{4.225000in}{4.225000in}}%
\pgfusepath{clip}%
\pgfsetbuttcap%
\pgfsetroundjoin%
\definecolor{currentfill}{rgb}{0.000000,0.000000,0.000000}%
\pgfsetfillcolor{currentfill}%
\pgfsetfillopacity{0.800000}%
\pgfsetlinewidth{0.000000pt}%
\definecolor{currentstroke}{rgb}{0.000000,0.000000,0.000000}%
\pgfsetstrokecolor{currentstroke}%
\pgfsetstrokeopacity{0.800000}%
\pgfsetdash{}{0pt}%
\pgfpathmoveto{\pgfqpoint{4.560176in}{1.740752in}}%
\pgfpathcurveto{\pgfqpoint{4.564294in}{1.740752in}}{\pgfqpoint{4.568244in}{1.742388in}}{\pgfqpoint{4.571156in}{1.745300in}}%
\pgfpathcurveto{\pgfqpoint{4.574068in}{1.748212in}}{\pgfqpoint{4.575704in}{1.752162in}}{\pgfqpoint{4.575704in}{1.756280in}}%
\pgfpathcurveto{\pgfqpoint{4.575704in}{1.760398in}}{\pgfqpoint{4.574068in}{1.764348in}}{\pgfqpoint{4.571156in}{1.767260in}}%
\pgfpathcurveto{\pgfqpoint{4.568244in}{1.770172in}}{\pgfqpoint{4.564294in}{1.771808in}}{\pgfqpoint{4.560176in}{1.771808in}}%
\pgfpathcurveto{\pgfqpoint{4.556058in}{1.771808in}}{\pgfqpoint{4.552108in}{1.770172in}}{\pgfqpoint{4.549196in}{1.767260in}}%
\pgfpathcurveto{\pgfqpoint{4.546284in}{1.764348in}}{\pgfqpoint{4.544647in}{1.760398in}}{\pgfqpoint{4.544647in}{1.756280in}}%
\pgfpathcurveto{\pgfqpoint{4.544647in}{1.752162in}}{\pgfqpoint{4.546284in}{1.748212in}}{\pgfqpoint{4.549196in}{1.745300in}}%
\pgfpathcurveto{\pgfqpoint{4.552108in}{1.742388in}}{\pgfqpoint{4.556058in}{1.740752in}}{\pgfqpoint{4.560176in}{1.740752in}}%
\pgfpathclose%
\pgfusepath{fill}%
\end{pgfscope}%
\begin{pgfscope}%
\pgfpathrectangle{\pgfqpoint{0.887500in}{0.275000in}}{\pgfqpoint{4.225000in}{4.225000in}}%
\pgfusepath{clip}%
\pgfsetbuttcap%
\pgfsetroundjoin%
\definecolor{currentfill}{rgb}{0.000000,0.000000,0.000000}%
\pgfsetfillcolor{currentfill}%
\pgfsetfillopacity{0.800000}%
\pgfsetlinewidth{0.000000pt}%
\definecolor{currentstroke}{rgb}{0.000000,0.000000,0.000000}%
\pgfsetstrokecolor{currentstroke}%
\pgfsetstrokeopacity{0.800000}%
\pgfsetdash{}{0pt}%
\pgfpathmoveto{\pgfqpoint{3.775768in}{2.354456in}}%
\pgfpathcurveto{\pgfqpoint{3.779886in}{2.354456in}}{\pgfqpoint{3.783836in}{2.356092in}}{\pgfqpoint{3.786748in}{2.359004in}}%
\pgfpathcurveto{\pgfqpoint{3.789660in}{2.361916in}}{\pgfqpoint{3.791296in}{2.365866in}}{\pgfqpoint{3.791296in}{2.369984in}}%
\pgfpathcurveto{\pgfqpoint{3.791296in}{2.374103in}}{\pgfqpoint{3.789660in}{2.378053in}}{\pgfqpoint{3.786748in}{2.380965in}}%
\pgfpathcurveto{\pgfqpoint{3.783836in}{2.383877in}}{\pgfqpoint{3.779886in}{2.385513in}}{\pgfqpoint{3.775768in}{2.385513in}}%
\pgfpathcurveto{\pgfqpoint{3.771650in}{2.385513in}}{\pgfqpoint{3.767700in}{2.383877in}}{\pgfqpoint{3.764788in}{2.380965in}}%
\pgfpathcurveto{\pgfqpoint{3.761876in}{2.378053in}}{\pgfqpoint{3.760240in}{2.374103in}}{\pgfqpoint{3.760240in}{2.369984in}}%
\pgfpathcurveto{\pgfqpoint{3.760240in}{2.365866in}}{\pgfqpoint{3.761876in}{2.361916in}}{\pgfqpoint{3.764788in}{2.359004in}}%
\pgfpathcurveto{\pgfqpoint{3.767700in}{2.356092in}}{\pgfqpoint{3.771650in}{2.354456in}}{\pgfqpoint{3.775768in}{2.354456in}}%
\pgfpathclose%
\pgfusepath{fill}%
\end{pgfscope}%
\begin{pgfscope}%
\pgfpathrectangle{\pgfqpoint{0.887500in}{0.275000in}}{\pgfqpoint{4.225000in}{4.225000in}}%
\pgfusepath{clip}%
\pgfsetbuttcap%
\pgfsetroundjoin%
\definecolor{currentfill}{rgb}{0.000000,0.000000,0.000000}%
\pgfsetfillcolor{currentfill}%
\pgfsetfillopacity{0.800000}%
\pgfsetlinewidth{0.000000pt}%
\definecolor{currentstroke}{rgb}{0.000000,0.000000,0.000000}%
\pgfsetstrokecolor{currentstroke}%
\pgfsetstrokeopacity{0.800000}%
\pgfsetdash{}{0pt}%
\pgfpathmoveto{\pgfqpoint{3.596200in}{2.458111in}}%
\pgfpathcurveto{\pgfqpoint{3.600318in}{2.458111in}}{\pgfqpoint{3.604268in}{2.459747in}}{\pgfqpoint{3.607180in}{2.462659in}}%
\pgfpathcurveto{\pgfqpoint{3.610092in}{2.465571in}}{\pgfqpoint{3.611728in}{2.469521in}}{\pgfqpoint{3.611728in}{2.473640in}}%
\pgfpathcurveto{\pgfqpoint{3.611728in}{2.477758in}}{\pgfqpoint{3.610092in}{2.481708in}}{\pgfqpoint{3.607180in}{2.484620in}}%
\pgfpathcurveto{\pgfqpoint{3.604268in}{2.487532in}}{\pgfqpoint{3.600318in}{2.489168in}}{\pgfqpoint{3.596200in}{2.489168in}}%
\pgfpathcurveto{\pgfqpoint{3.592082in}{2.489168in}}{\pgfqpoint{3.588132in}{2.487532in}}{\pgfqpoint{3.585220in}{2.484620in}}%
\pgfpathcurveto{\pgfqpoint{3.582308in}{2.481708in}}{\pgfqpoint{3.580672in}{2.477758in}}{\pgfqpoint{3.580672in}{2.473640in}}%
\pgfpathcurveto{\pgfqpoint{3.580672in}{2.469521in}}{\pgfqpoint{3.582308in}{2.465571in}}{\pgfqpoint{3.585220in}{2.462659in}}%
\pgfpathcurveto{\pgfqpoint{3.588132in}{2.459747in}}{\pgfqpoint{3.592082in}{2.458111in}}{\pgfqpoint{3.596200in}{2.458111in}}%
\pgfpathclose%
\pgfusepath{fill}%
\end{pgfscope}%
\begin{pgfscope}%
\pgfpathrectangle{\pgfqpoint{0.887500in}{0.275000in}}{\pgfqpoint{4.225000in}{4.225000in}}%
\pgfusepath{clip}%
\pgfsetbuttcap%
\pgfsetroundjoin%
\definecolor{currentfill}{rgb}{0.000000,0.000000,0.000000}%
\pgfsetfillcolor{currentfill}%
\pgfsetfillopacity{0.800000}%
\pgfsetlinewidth{0.000000pt}%
\definecolor{currentstroke}{rgb}{0.000000,0.000000,0.000000}%
\pgfsetstrokecolor{currentstroke}%
\pgfsetstrokeopacity{0.800000}%
\pgfsetdash{}{0pt}%
\pgfpathmoveto{\pgfqpoint{3.320247in}{1.843106in}}%
\pgfpathcurveto{\pgfqpoint{3.324365in}{1.843106in}}{\pgfqpoint{3.328315in}{1.844742in}}{\pgfqpoint{3.331227in}{1.847654in}}%
\pgfpathcurveto{\pgfqpoint{3.334139in}{1.850566in}}{\pgfqpoint{3.335775in}{1.854516in}}{\pgfqpoint{3.335775in}{1.858634in}}%
\pgfpathcurveto{\pgfqpoint{3.335775in}{1.862752in}}{\pgfqpoint{3.334139in}{1.866702in}}{\pgfqpoint{3.331227in}{1.869614in}}%
\pgfpathcurveto{\pgfqpoint{3.328315in}{1.872526in}}{\pgfqpoint{3.324365in}{1.874162in}}{\pgfqpoint{3.320247in}{1.874162in}}%
\pgfpathcurveto{\pgfqpoint{3.316129in}{1.874162in}}{\pgfqpoint{3.312179in}{1.872526in}}{\pgfqpoint{3.309267in}{1.869614in}}%
\pgfpathcurveto{\pgfqpoint{3.306355in}{1.866702in}}{\pgfqpoint{3.304719in}{1.862752in}}{\pgfqpoint{3.304719in}{1.858634in}}%
\pgfpathcurveto{\pgfqpoint{3.304719in}{1.854516in}}{\pgfqpoint{3.306355in}{1.850566in}}{\pgfqpoint{3.309267in}{1.847654in}}%
\pgfpathcurveto{\pgfqpoint{3.312179in}{1.844742in}}{\pgfqpoint{3.316129in}{1.843106in}}{\pgfqpoint{3.320247in}{1.843106in}}%
\pgfpathclose%
\pgfusepath{fill}%
\end{pgfscope}%
\begin{pgfscope}%
\pgfpathrectangle{\pgfqpoint{0.887500in}{0.275000in}}{\pgfqpoint{4.225000in}{4.225000in}}%
\pgfusepath{clip}%
\pgfsetbuttcap%
\pgfsetroundjoin%
\definecolor{currentfill}{rgb}{0.000000,0.000000,0.000000}%
\pgfsetfillcolor{currentfill}%
\pgfsetfillopacity{0.800000}%
\pgfsetlinewidth{0.000000pt}%
\definecolor{currentstroke}{rgb}{0.000000,0.000000,0.000000}%
\pgfsetstrokecolor{currentstroke}%
\pgfsetstrokeopacity{0.800000}%
\pgfsetdash{}{0pt}%
\pgfpathmoveto{\pgfqpoint{2.013729in}{2.165516in}}%
\pgfpathcurveto{\pgfqpoint{2.017847in}{2.165516in}}{\pgfqpoint{2.021797in}{2.167153in}}{\pgfqpoint{2.024709in}{2.170065in}}%
\pgfpathcurveto{\pgfqpoint{2.027621in}{2.172977in}}{\pgfqpoint{2.029257in}{2.176927in}}{\pgfqpoint{2.029257in}{2.181045in}}%
\pgfpathcurveto{\pgfqpoint{2.029257in}{2.185163in}}{\pgfqpoint{2.027621in}{2.189113in}}{\pgfqpoint{2.024709in}{2.192025in}}%
\pgfpathcurveto{\pgfqpoint{2.021797in}{2.194937in}}{\pgfqpoint{2.017847in}{2.196573in}}{\pgfqpoint{2.013729in}{2.196573in}}%
\pgfpathcurveto{\pgfqpoint{2.009611in}{2.196573in}}{\pgfqpoint{2.005661in}{2.194937in}}{\pgfqpoint{2.002749in}{2.192025in}}%
\pgfpathcurveto{\pgfqpoint{1.999837in}{2.189113in}}{\pgfqpoint{1.998200in}{2.185163in}}{\pgfqpoint{1.998200in}{2.181045in}}%
\pgfpathcurveto{\pgfqpoint{1.998200in}{2.176927in}}{\pgfqpoint{1.999837in}{2.172977in}}{\pgfqpoint{2.002749in}{2.170065in}}%
\pgfpathcurveto{\pgfqpoint{2.005661in}{2.167153in}}{\pgfqpoint{2.009611in}{2.165516in}}{\pgfqpoint{2.013729in}{2.165516in}}%
\pgfpathclose%
\pgfusepath{fill}%
\end{pgfscope}%
\begin{pgfscope}%
\pgfpathrectangle{\pgfqpoint{0.887500in}{0.275000in}}{\pgfqpoint{4.225000in}{4.225000in}}%
\pgfusepath{clip}%
\pgfsetbuttcap%
\pgfsetroundjoin%
\definecolor{currentfill}{rgb}{0.000000,0.000000,0.000000}%
\pgfsetfillcolor{currentfill}%
\pgfsetfillopacity{0.800000}%
\pgfsetlinewidth{0.000000pt}%
\definecolor{currentstroke}{rgb}{0.000000,0.000000,0.000000}%
\pgfsetstrokecolor{currentstroke}%
\pgfsetstrokeopacity{0.800000}%
\pgfsetdash{}{0pt}%
\pgfpathmoveto{\pgfqpoint{4.381326in}{1.861040in}}%
\pgfpathcurveto{\pgfqpoint{4.385444in}{1.861040in}}{\pgfqpoint{4.389394in}{1.862676in}}{\pgfqpoint{4.392306in}{1.865588in}}%
\pgfpathcurveto{\pgfqpoint{4.395218in}{1.868500in}}{\pgfqpoint{4.396854in}{1.872450in}}{\pgfqpoint{4.396854in}{1.876568in}}%
\pgfpathcurveto{\pgfqpoint{4.396854in}{1.880686in}}{\pgfqpoint{4.395218in}{1.884636in}}{\pgfqpoint{4.392306in}{1.887548in}}%
\pgfpathcurveto{\pgfqpoint{4.389394in}{1.890460in}}{\pgfqpoint{4.385444in}{1.892097in}}{\pgfqpoint{4.381326in}{1.892097in}}%
\pgfpathcurveto{\pgfqpoint{4.377208in}{1.892097in}}{\pgfqpoint{4.373258in}{1.890460in}}{\pgfqpoint{4.370346in}{1.887548in}}%
\pgfpathcurveto{\pgfqpoint{4.367434in}{1.884636in}}{\pgfqpoint{4.365798in}{1.880686in}}{\pgfqpoint{4.365798in}{1.876568in}}%
\pgfpathcurveto{\pgfqpoint{4.365798in}{1.872450in}}{\pgfqpoint{4.367434in}{1.868500in}}{\pgfqpoint{4.370346in}{1.865588in}}%
\pgfpathcurveto{\pgfqpoint{4.373258in}{1.862676in}}{\pgfqpoint{4.377208in}{1.861040in}}{\pgfqpoint{4.381326in}{1.861040in}}%
\pgfpathclose%
\pgfusepath{fill}%
\end{pgfscope}%
\begin{pgfscope}%
\pgfpathrectangle{\pgfqpoint{0.887500in}{0.275000in}}{\pgfqpoint{4.225000in}{4.225000in}}%
\pgfusepath{clip}%
\pgfsetbuttcap%
\pgfsetroundjoin%
\definecolor{currentfill}{rgb}{0.000000,0.000000,0.000000}%
\pgfsetfillcolor{currentfill}%
\pgfsetfillopacity{0.800000}%
\pgfsetlinewidth{0.000000pt}%
\definecolor{currentstroke}{rgb}{0.000000,0.000000,0.000000}%
\pgfsetstrokecolor{currentstroke}%
\pgfsetstrokeopacity{0.800000}%
\pgfsetdash{}{0pt}%
\pgfpathmoveto{\pgfqpoint{4.201736in}{1.963722in}}%
\pgfpathcurveto{\pgfqpoint{4.205854in}{1.963722in}}{\pgfqpoint{4.209804in}{1.965358in}}{\pgfqpoint{4.212716in}{1.968270in}}%
\pgfpathcurveto{\pgfqpoint{4.215628in}{1.971182in}}{\pgfqpoint{4.217264in}{1.975132in}}{\pgfqpoint{4.217264in}{1.979250in}}%
\pgfpathcurveto{\pgfqpoint{4.217264in}{1.983368in}}{\pgfqpoint{4.215628in}{1.987318in}}{\pgfqpoint{4.212716in}{1.990230in}}%
\pgfpathcurveto{\pgfqpoint{4.209804in}{1.993142in}}{\pgfqpoint{4.205854in}{1.994778in}}{\pgfqpoint{4.201736in}{1.994778in}}%
\pgfpathcurveto{\pgfqpoint{4.197618in}{1.994778in}}{\pgfqpoint{4.193668in}{1.993142in}}{\pgfqpoint{4.190756in}{1.990230in}}%
\pgfpathcurveto{\pgfqpoint{4.187844in}{1.987318in}}{\pgfqpoint{4.186208in}{1.983368in}}{\pgfqpoint{4.186208in}{1.979250in}}%
\pgfpathcurveto{\pgfqpoint{4.186208in}{1.975132in}}{\pgfqpoint{4.187844in}{1.971182in}}{\pgfqpoint{4.190756in}{1.968270in}}%
\pgfpathcurveto{\pgfqpoint{4.193668in}{1.965358in}}{\pgfqpoint{4.197618in}{1.963722in}}{\pgfqpoint{4.201736in}{1.963722in}}%
\pgfpathclose%
\pgfusepath{fill}%
\end{pgfscope}%
\begin{pgfscope}%
\pgfpathrectangle{\pgfqpoint{0.887500in}{0.275000in}}{\pgfqpoint{4.225000in}{4.225000in}}%
\pgfusepath{clip}%
\pgfsetbuttcap%
\pgfsetroundjoin%
\definecolor{currentfill}{rgb}{0.000000,0.000000,0.000000}%
\pgfsetfillcolor{currentfill}%
\pgfsetfillopacity{0.800000}%
\pgfsetlinewidth{0.000000pt}%
\definecolor{currentstroke}{rgb}{0.000000,0.000000,0.000000}%
\pgfsetstrokecolor{currentstroke}%
\pgfsetstrokeopacity{0.800000}%
\pgfsetdash{}{0pt}%
\pgfpathmoveto{\pgfqpoint{2.666922in}{2.017298in}}%
\pgfpathcurveto{\pgfqpoint{2.671040in}{2.017298in}}{\pgfqpoint{2.674990in}{2.018934in}}{\pgfqpoint{2.677902in}{2.021846in}}%
\pgfpathcurveto{\pgfqpoint{2.680814in}{2.024758in}}{\pgfqpoint{2.682450in}{2.028708in}}{\pgfqpoint{2.682450in}{2.032826in}}%
\pgfpathcurveto{\pgfqpoint{2.682450in}{2.036944in}}{\pgfqpoint{2.680814in}{2.040894in}}{\pgfqpoint{2.677902in}{2.043806in}}%
\pgfpathcurveto{\pgfqpoint{2.674990in}{2.046718in}}{\pgfqpoint{2.671040in}{2.048354in}}{\pgfqpoint{2.666922in}{2.048354in}}%
\pgfpathcurveto{\pgfqpoint{2.662804in}{2.048354in}}{\pgfqpoint{2.658854in}{2.046718in}}{\pgfqpoint{2.655942in}{2.043806in}}%
\pgfpathcurveto{\pgfqpoint{2.653030in}{2.040894in}}{\pgfqpoint{2.651394in}{2.036944in}}{\pgfqpoint{2.651394in}{2.032826in}}%
\pgfpathcurveto{\pgfqpoint{2.651394in}{2.028708in}}{\pgfqpoint{2.653030in}{2.024758in}}{\pgfqpoint{2.655942in}{2.021846in}}%
\pgfpathcurveto{\pgfqpoint{2.658854in}{2.018934in}}{\pgfqpoint{2.662804in}{2.017298in}}{\pgfqpoint{2.666922in}{2.017298in}}%
\pgfpathclose%
\pgfusepath{fill}%
\end{pgfscope}%
\begin{pgfscope}%
\pgfpathrectangle{\pgfqpoint{0.887500in}{0.275000in}}{\pgfqpoint{4.225000in}{4.225000in}}%
\pgfusepath{clip}%
\pgfsetbuttcap%
\pgfsetroundjoin%
\definecolor{currentfill}{rgb}{0.000000,0.000000,0.000000}%
\pgfsetfillcolor{currentfill}%
\pgfsetfillopacity{0.800000}%
\pgfsetlinewidth{0.000000pt}%
\definecolor{currentstroke}{rgb}{0.000000,0.000000,0.000000}%
\pgfsetstrokecolor{currentstroke}%
\pgfsetstrokeopacity{0.800000}%
\pgfsetdash{}{0pt}%
\pgfpathmoveto{\pgfqpoint{3.368648in}{2.233046in}}%
\pgfpathcurveto{\pgfqpoint{3.372767in}{2.233046in}}{\pgfqpoint{3.376717in}{2.234682in}}{\pgfqpoint{3.379629in}{2.237594in}}%
\pgfpathcurveto{\pgfqpoint{3.382541in}{2.240506in}}{\pgfqpoint{3.384177in}{2.244456in}}{\pgfqpoint{3.384177in}{2.248574in}}%
\pgfpathcurveto{\pgfqpoint{3.384177in}{2.252693in}}{\pgfqpoint{3.382541in}{2.256643in}}{\pgfqpoint{3.379629in}{2.259555in}}%
\pgfpathcurveto{\pgfqpoint{3.376717in}{2.262467in}}{\pgfqpoint{3.372767in}{2.264103in}}{\pgfqpoint{3.368648in}{2.264103in}}%
\pgfpathcurveto{\pgfqpoint{3.364530in}{2.264103in}}{\pgfqpoint{3.360580in}{2.262467in}}{\pgfqpoint{3.357668in}{2.259555in}}%
\pgfpathcurveto{\pgfqpoint{3.354756in}{2.256643in}}{\pgfqpoint{3.353120in}{2.252693in}}{\pgfqpoint{3.353120in}{2.248574in}}%
\pgfpathcurveto{\pgfqpoint{3.353120in}{2.244456in}}{\pgfqpoint{3.354756in}{2.240506in}}{\pgfqpoint{3.357668in}{2.237594in}}%
\pgfpathcurveto{\pgfqpoint{3.360580in}{2.234682in}}{\pgfqpoint{3.364530in}{2.233046in}}{\pgfqpoint{3.368648in}{2.233046in}}%
\pgfpathclose%
\pgfusepath{fill}%
\end{pgfscope}%
\begin{pgfscope}%
\pgfpathrectangle{\pgfqpoint{0.887500in}{0.275000in}}{\pgfqpoint{4.225000in}{4.225000in}}%
\pgfusepath{clip}%
\pgfsetbuttcap%
\pgfsetroundjoin%
\definecolor{currentfill}{rgb}{0.000000,0.000000,0.000000}%
\pgfsetfillcolor{currentfill}%
\pgfsetfillopacity{0.800000}%
\pgfsetlinewidth{0.000000pt}%
\definecolor{currentstroke}{rgb}{0.000000,0.000000,0.000000}%
\pgfsetstrokecolor{currentstroke}%
\pgfsetstrokeopacity{0.800000}%
\pgfsetdash{}{0pt}%
\pgfpathmoveto{\pgfqpoint{4.022607in}{2.087511in}}%
\pgfpathcurveto{\pgfqpoint{4.026725in}{2.087511in}}{\pgfqpoint{4.030675in}{2.089148in}}{\pgfqpoint{4.033587in}{2.092060in}}%
\pgfpathcurveto{\pgfqpoint{4.036499in}{2.094972in}}{\pgfqpoint{4.038135in}{2.098922in}}{\pgfqpoint{4.038135in}{2.103040in}}%
\pgfpathcurveto{\pgfqpoint{4.038135in}{2.107158in}}{\pgfqpoint{4.036499in}{2.111108in}}{\pgfqpoint{4.033587in}{2.114020in}}%
\pgfpathcurveto{\pgfqpoint{4.030675in}{2.116932in}}{\pgfqpoint{4.026725in}{2.118568in}}{\pgfqpoint{4.022607in}{2.118568in}}%
\pgfpathcurveto{\pgfqpoint{4.018489in}{2.118568in}}{\pgfqpoint{4.014539in}{2.116932in}}{\pgfqpoint{4.011627in}{2.114020in}}%
\pgfpathcurveto{\pgfqpoint{4.008715in}{2.111108in}}{\pgfqpoint{4.007079in}{2.107158in}}{\pgfqpoint{4.007079in}{2.103040in}}%
\pgfpathcurveto{\pgfqpoint{4.007079in}{2.098922in}}{\pgfqpoint{4.008715in}{2.094972in}}{\pgfqpoint{4.011627in}{2.092060in}}%
\pgfpathcurveto{\pgfqpoint{4.014539in}{2.089148in}}{\pgfqpoint{4.018489in}{2.087511in}}{\pgfqpoint{4.022607in}{2.087511in}}%
\pgfpathclose%
\pgfusepath{fill}%
\end{pgfscope}%
\begin{pgfscope}%
\pgfpathrectangle{\pgfqpoint{0.887500in}{0.275000in}}{\pgfqpoint{4.225000in}{4.225000in}}%
\pgfusepath{clip}%
\pgfsetbuttcap%
\pgfsetroundjoin%
\definecolor{currentfill}{rgb}{0.000000,0.000000,0.000000}%
\pgfsetfillcolor{currentfill}%
\pgfsetfillopacity{0.800000}%
\pgfsetlinewidth{0.000000pt}%
\definecolor{currentstroke}{rgb}{0.000000,0.000000,0.000000}%
\pgfsetstrokecolor{currentstroke}%
\pgfsetstrokeopacity{0.800000}%
\pgfsetdash{}{0pt}%
\pgfpathmoveto{\pgfqpoint{2.192689in}{2.106000in}}%
\pgfpathcurveto{\pgfqpoint{2.196807in}{2.106000in}}{\pgfqpoint{2.200757in}{2.107636in}}{\pgfqpoint{2.203669in}{2.110548in}}%
\pgfpathcurveto{\pgfqpoint{2.206581in}{2.113460in}}{\pgfqpoint{2.208217in}{2.117410in}}{\pgfqpoint{2.208217in}{2.121528in}}%
\pgfpathcurveto{\pgfqpoint{2.208217in}{2.125646in}}{\pgfqpoint{2.206581in}{2.129597in}}{\pgfqpoint{2.203669in}{2.132508in}}%
\pgfpathcurveto{\pgfqpoint{2.200757in}{2.135420in}}{\pgfqpoint{2.196807in}{2.137057in}}{\pgfqpoint{2.192689in}{2.137057in}}%
\pgfpathcurveto{\pgfqpoint{2.188571in}{2.137057in}}{\pgfqpoint{2.184621in}{2.135420in}}{\pgfqpoint{2.181709in}{2.132508in}}%
\pgfpathcurveto{\pgfqpoint{2.178797in}{2.129597in}}{\pgfqpoint{2.177161in}{2.125646in}}{\pgfqpoint{2.177161in}{2.121528in}}%
\pgfpathcurveto{\pgfqpoint{2.177161in}{2.117410in}}{\pgfqpoint{2.178797in}{2.113460in}}{\pgfqpoint{2.181709in}{2.110548in}}%
\pgfpathcurveto{\pgfqpoint{2.184621in}{2.107636in}}{\pgfqpoint{2.188571in}{2.106000in}}{\pgfqpoint{2.192689in}{2.106000in}}%
\pgfpathclose%
\pgfusepath{fill}%
\end{pgfscope}%
\begin{pgfscope}%
\pgfpathrectangle{\pgfqpoint{0.887500in}{0.275000in}}{\pgfqpoint{4.225000in}{4.225000in}}%
\pgfusepath{clip}%
\pgfsetbuttcap%
\pgfsetroundjoin%
\definecolor{currentfill}{rgb}{0.000000,0.000000,0.000000}%
\pgfsetfillcolor{currentfill}%
\pgfsetfillopacity{0.800000}%
\pgfsetlinewidth{0.000000pt}%
\definecolor{currentstroke}{rgb}{0.000000,0.000000,0.000000}%
\pgfsetstrokecolor{currentstroke}%
\pgfsetstrokeopacity{0.800000}%
\pgfsetdash{}{0pt}%
\pgfpathmoveto{\pgfqpoint{2.846580in}{1.949822in}}%
\pgfpathcurveto{\pgfqpoint{2.850698in}{1.949822in}}{\pgfqpoint{2.854648in}{1.951458in}}{\pgfqpoint{2.857560in}{1.954370in}}%
\pgfpathcurveto{\pgfqpoint{2.860472in}{1.957282in}}{\pgfqpoint{2.862108in}{1.961232in}}{\pgfqpoint{2.862108in}{1.965350in}}%
\pgfpathcurveto{\pgfqpoint{2.862108in}{1.969468in}}{\pgfqpoint{2.860472in}{1.973418in}}{\pgfqpoint{2.857560in}{1.976330in}}%
\pgfpathcurveto{\pgfqpoint{2.854648in}{1.979242in}}{\pgfqpoint{2.850698in}{1.980878in}}{\pgfqpoint{2.846580in}{1.980878in}}%
\pgfpathcurveto{\pgfqpoint{2.842462in}{1.980878in}}{\pgfqpoint{2.838512in}{1.979242in}}{\pgfqpoint{2.835600in}{1.976330in}}%
\pgfpathcurveto{\pgfqpoint{2.832688in}{1.973418in}}{\pgfqpoint{2.831052in}{1.969468in}}{\pgfqpoint{2.831052in}{1.965350in}}%
\pgfpathcurveto{\pgfqpoint{2.831052in}{1.961232in}}{\pgfqpoint{2.832688in}{1.957282in}}{\pgfqpoint{2.835600in}{1.954370in}}%
\pgfpathcurveto{\pgfqpoint{2.838512in}{1.951458in}}{\pgfqpoint{2.842462in}{1.949822in}}{\pgfqpoint{2.846580in}{1.949822in}}%
\pgfpathclose%
\pgfusepath{fill}%
\end{pgfscope}%
\begin{pgfscope}%
\pgfpathrectangle{\pgfqpoint{0.887500in}{0.275000in}}{\pgfqpoint{4.225000in}{4.225000in}}%
\pgfusepath{clip}%
\pgfsetbuttcap%
\pgfsetroundjoin%
\definecolor{currentfill}{rgb}{0.000000,0.000000,0.000000}%
\pgfsetfillcolor{currentfill}%
\pgfsetfillopacity{0.800000}%
\pgfsetlinewidth{0.000000pt}%
\definecolor{currentstroke}{rgb}{0.000000,0.000000,0.000000}%
\pgfsetstrokecolor{currentstroke}%
\pgfsetstrokeopacity{0.800000}%
\pgfsetdash{}{0pt}%
\pgfpathmoveto{\pgfqpoint{3.482949in}{2.389179in}}%
\pgfpathcurveto{\pgfqpoint{3.487067in}{2.389179in}}{\pgfqpoint{3.491017in}{2.390815in}}{\pgfqpoint{3.493929in}{2.393727in}}%
\pgfpathcurveto{\pgfqpoint{3.496841in}{2.396639in}}{\pgfqpoint{3.498477in}{2.400589in}}{\pgfqpoint{3.498477in}{2.404707in}}%
\pgfpathcurveto{\pgfqpoint{3.498477in}{2.408825in}}{\pgfqpoint{3.496841in}{2.412775in}}{\pgfqpoint{3.493929in}{2.415687in}}%
\pgfpathcurveto{\pgfqpoint{3.491017in}{2.418599in}}{\pgfqpoint{3.487067in}{2.420235in}}{\pgfqpoint{3.482949in}{2.420235in}}%
\pgfpathcurveto{\pgfqpoint{3.478830in}{2.420235in}}{\pgfqpoint{3.474880in}{2.418599in}}{\pgfqpoint{3.471968in}{2.415687in}}%
\pgfpathcurveto{\pgfqpoint{3.469056in}{2.412775in}}{\pgfqpoint{3.467420in}{2.408825in}}{\pgfqpoint{3.467420in}{2.404707in}}%
\pgfpathcurveto{\pgfqpoint{3.467420in}{2.400589in}}{\pgfqpoint{3.469056in}{2.396639in}}{\pgfqpoint{3.471968in}{2.393727in}}%
\pgfpathcurveto{\pgfqpoint{3.474880in}{2.390815in}}{\pgfqpoint{3.478830in}{2.389179in}}{\pgfqpoint{3.482949in}{2.389179in}}%
\pgfpathclose%
\pgfusepath{fill}%
\end{pgfscope}%
\begin{pgfscope}%
\pgfpathrectangle{\pgfqpoint{0.887500in}{0.275000in}}{\pgfqpoint{4.225000in}{4.225000in}}%
\pgfusepath{clip}%
\pgfsetbuttcap%
\pgfsetroundjoin%
\definecolor{currentfill}{rgb}{0.000000,0.000000,0.000000}%
\pgfsetfillcolor{currentfill}%
\pgfsetfillopacity{0.800000}%
\pgfsetlinewidth{0.000000pt}%
\definecolor{currentstroke}{rgb}{0.000000,0.000000,0.000000}%
\pgfsetstrokecolor{currentstroke}%
\pgfsetstrokeopacity{0.800000}%
\pgfsetdash{}{0pt}%
\pgfpathmoveto{\pgfqpoint{3.662998in}{2.296794in}}%
\pgfpathcurveto{\pgfqpoint{3.667116in}{2.296794in}}{\pgfqpoint{3.671066in}{2.298431in}}{\pgfqpoint{3.673978in}{2.301343in}}%
\pgfpathcurveto{\pgfqpoint{3.676890in}{2.304254in}}{\pgfqpoint{3.678526in}{2.308205in}}{\pgfqpoint{3.678526in}{2.312323in}}%
\pgfpathcurveto{\pgfqpoint{3.678526in}{2.316441in}}{\pgfqpoint{3.676890in}{2.320391in}}{\pgfqpoint{3.673978in}{2.323303in}}%
\pgfpathcurveto{\pgfqpoint{3.671066in}{2.326215in}}{\pgfqpoint{3.667116in}{2.327851in}}{\pgfqpoint{3.662998in}{2.327851in}}%
\pgfpathcurveto{\pgfqpoint{3.658880in}{2.327851in}}{\pgfqpoint{3.654930in}{2.326215in}}{\pgfqpoint{3.652018in}{2.323303in}}%
\pgfpathcurveto{\pgfqpoint{3.649106in}{2.320391in}}{\pgfqpoint{3.647470in}{2.316441in}}{\pgfqpoint{3.647470in}{2.312323in}}%
\pgfpathcurveto{\pgfqpoint{3.647470in}{2.308205in}}{\pgfqpoint{3.649106in}{2.304254in}}{\pgfqpoint{3.652018in}{2.301343in}}%
\pgfpathcurveto{\pgfqpoint{3.654930in}{2.298431in}}{\pgfqpoint{3.658880in}{2.296794in}}{\pgfqpoint{3.662998in}{2.296794in}}%
\pgfpathclose%
\pgfusepath{fill}%
\end{pgfscope}%
\begin{pgfscope}%
\pgfpathrectangle{\pgfqpoint{0.887500in}{0.275000in}}{\pgfqpoint{4.225000in}{4.225000in}}%
\pgfusepath{clip}%
\pgfsetbuttcap%
\pgfsetroundjoin%
\definecolor{currentfill}{rgb}{0.000000,0.000000,0.000000}%
\pgfsetfillcolor{currentfill}%
\pgfsetfillopacity{0.800000}%
\pgfsetlinewidth{0.000000pt}%
\definecolor{currentstroke}{rgb}{0.000000,0.000000,0.000000}%
\pgfsetstrokecolor{currentstroke}%
\pgfsetstrokeopacity{0.800000}%
\pgfsetdash{}{0pt}%
\pgfpathmoveto{\pgfqpoint{1.718008in}{2.188351in}}%
\pgfpathcurveto{\pgfqpoint{1.722126in}{2.188351in}}{\pgfqpoint{1.726076in}{2.189987in}}{\pgfqpoint{1.728988in}{2.192899in}}%
\pgfpathcurveto{\pgfqpoint{1.731900in}{2.195811in}}{\pgfqpoint{1.733536in}{2.199761in}}{\pgfqpoint{1.733536in}{2.203879in}}%
\pgfpathcurveto{\pgfqpoint{1.733536in}{2.207997in}}{\pgfqpoint{1.731900in}{2.211947in}}{\pgfqpoint{1.728988in}{2.214859in}}%
\pgfpathcurveto{\pgfqpoint{1.726076in}{2.217771in}}{\pgfqpoint{1.722126in}{2.219407in}}{\pgfqpoint{1.718008in}{2.219407in}}%
\pgfpathcurveto{\pgfqpoint{1.713890in}{2.219407in}}{\pgfqpoint{1.709940in}{2.217771in}}{\pgfqpoint{1.707028in}{2.214859in}}%
\pgfpathcurveto{\pgfqpoint{1.704116in}{2.211947in}}{\pgfqpoint{1.702480in}{2.207997in}}{\pgfqpoint{1.702480in}{2.203879in}}%
\pgfpathcurveto{\pgfqpoint{1.702480in}{2.199761in}}{\pgfqpoint{1.704116in}{2.195811in}}{\pgfqpoint{1.707028in}{2.192899in}}%
\pgfpathcurveto{\pgfqpoint{1.709940in}{2.189987in}}{\pgfqpoint{1.713890in}{2.188351in}}{\pgfqpoint{1.718008in}{2.188351in}}%
\pgfpathclose%
\pgfusepath{fill}%
\end{pgfscope}%
\begin{pgfscope}%
\pgfpathrectangle{\pgfqpoint{0.887500in}{0.275000in}}{\pgfqpoint{4.225000in}{4.225000in}}%
\pgfusepath{clip}%
\pgfsetbuttcap%
\pgfsetroundjoin%
\definecolor{currentfill}{rgb}{0.000000,0.000000,0.000000}%
\pgfsetfillcolor{currentfill}%
\pgfsetfillopacity{0.800000}%
\pgfsetlinewidth{0.000000pt}%
\definecolor{currentstroke}{rgb}{0.000000,0.000000,0.000000}%
\pgfsetstrokecolor{currentstroke}%
\pgfsetstrokeopacity{0.800000}%
\pgfsetdash{}{0pt}%
\pgfpathmoveto{\pgfqpoint{3.026507in}{1.878476in}}%
\pgfpathcurveto{\pgfqpoint{3.030625in}{1.878476in}}{\pgfqpoint{3.034575in}{1.880112in}}{\pgfqpoint{3.037487in}{1.883024in}}%
\pgfpathcurveto{\pgfqpoint{3.040399in}{1.885936in}}{\pgfqpoint{3.042035in}{1.889886in}}{\pgfqpoint{3.042035in}{1.894004in}}%
\pgfpathcurveto{\pgfqpoint{3.042035in}{1.898122in}}{\pgfqpoint{3.040399in}{1.902072in}}{\pgfqpoint{3.037487in}{1.904984in}}%
\pgfpathcurveto{\pgfqpoint{3.034575in}{1.907896in}}{\pgfqpoint{3.030625in}{1.909532in}}{\pgfqpoint{3.026507in}{1.909532in}}%
\pgfpathcurveto{\pgfqpoint{3.022389in}{1.909532in}}{\pgfqpoint{3.018439in}{1.907896in}}{\pgfqpoint{3.015527in}{1.904984in}}%
\pgfpathcurveto{\pgfqpoint{3.012615in}{1.902072in}}{\pgfqpoint{3.010979in}{1.898122in}}{\pgfqpoint{3.010979in}{1.894004in}}%
\pgfpathcurveto{\pgfqpoint{3.010979in}{1.889886in}}{\pgfqpoint{3.012615in}{1.885936in}}{\pgfqpoint{3.015527in}{1.883024in}}%
\pgfpathcurveto{\pgfqpoint{3.018439in}{1.880112in}}{\pgfqpoint{3.022389in}{1.878476in}}{\pgfqpoint{3.026507in}{1.878476in}}%
\pgfpathclose%
\pgfusepath{fill}%
\end{pgfscope}%
\begin{pgfscope}%
\pgfpathrectangle{\pgfqpoint{0.887500in}{0.275000in}}{\pgfqpoint{4.225000in}{4.225000in}}%
\pgfusepath{clip}%
\pgfsetbuttcap%
\pgfsetroundjoin%
\definecolor{currentfill}{rgb}{0.000000,0.000000,0.000000}%
\pgfsetfillcolor{currentfill}%
\pgfsetfillopacity{0.800000}%
\pgfsetlinewidth{0.000000pt}%
\definecolor{currentstroke}{rgb}{0.000000,0.000000,0.000000}%
\pgfsetstrokecolor{currentstroke}%
\pgfsetstrokeopacity{0.800000}%
\pgfsetdash{}{0pt}%
\pgfpathmoveto{\pgfqpoint{2.372014in}{2.044737in}}%
\pgfpathcurveto{\pgfqpoint{2.376132in}{2.044737in}}{\pgfqpoint{2.380082in}{2.046373in}}{\pgfqpoint{2.382994in}{2.049285in}}%
\pgfpathcurveto{\pgfqpoint{2.385906in}{2.052197in}}{\pgfqpoint{2.387542in}{2.056147in}}{\pgfqpoint{2.387542in}{2.060265in}}%
\pgfpathcurveto{\pgfqpoint{2.387542in}{2.064383in}}{\pgfqpoint{2.385906in}{2.068333in}}{\pgfqpoint{2.382994in}{2.071245in}}%
\pgfpathcurveto{\pgfqpoint{2.380082in}{2.074157in}}{\pgfqpoint{2.376132in}{2.075793in}}{\pgfqpoint{2.372014in}{2.075793in}}%
\pgfpathcurveto{\pgfqpoint{2.367896in}{2.075793in}}{\pgfqpoint{2.363946in}{2.074157in}}{\pgfqpoint{2.361034in}{2.071245in}}%
\pgfpathcurveto{\pgfqpoint{2.358122in}{2.068333in}}{\pgfqpoint{2.356486in}{2.064383in}}{\pgfqpoint{2.356486in}{2.060265in}}%
\pgfpathcurveto{\pgfqpoint{2.356486in}{2.056147in}}{\pgfqpoint{2.358122in}{2.052197in}}{\pgfqpoint{2.361034in}{2.049285in}}%
\pgfpathcurveto{\pgfqpoint{2.363946in}{2.046373in}}{\pgfqpoint{2.367896in}{2.044737in}}{\pgfqpoint{2.372014in}{2.044737in}}%
\pgfpathclose%
\pgfusepath{fill}%
\end{pgfscope}%
\begin{pgfscope}%
\pgfpathrectangle{\pgfqpoint{0.887500in}{0.275000in}}{\pgfqpoint{4.225000in}{4.225000in}}%
\pgfusepath{clip}%
\pgfsetbuttcap%
\pgfsetroundjoin%
\definecolor{currentfill}{rgb}{0.000000,0.000000,0.000000}%
\pgfsetfillcolor{currentfill}%
\pgfsetfillopacity{0.800000}%
\pgfsetlinewidth{0.000000pt}%
\definecolor{currentstroke}{rgb}{0.000000,0.000000,0.000000}%
\pgfsetstrokecolor{currentstroke}%
\pgfsetstrokeopacity{0.800000}%
\pgfsetdash{}{0pt}%
\pgfpathmoveto{\pgfqpoint{3.435058in}{2.064129in}}%
\pgfpathcurveto{\pgfqpoint{3.439176in}{2.064129in}}{\pgfqpoint{3.443126in}{2.065765in}}{\pgfqpoint{3.446038in}{2.068677in}}%
\pgfpathcurveto{\pgfqpoint{3.448950in}{2.071589in}}{\pgfqpoint{3.450586in}{2.075539in}}{\pgfqpoint{3.450586in}{2.079658in}}%
\pgfpathcurveto{\pgfqpoint{3.450586in}{2.083776in}}{\pgfqpoint{3.448950in}{2.087726in}}{\pgfqpoint{3.446038in}{2.090638in}}%
\pgfpathcurveto{\pgfqpoint{3.443126in}{2.093550in}}{\pgfqpoint{3.439176in}{2.095186in}}{\pgfqpoint{3.435058in}{2.095186in}}%
\pgfpathcurveto{\pgfqpoint{3.430940in}{2.095186in}}{\pgfqpoint{3.426990in}{2.093550in}}{\pgfqpoint{3.424078in}{2.090638in}}%
\pgfpathcurveto{\pgfqpoint{3.421166in}{2.087726in}}{\pgfqpoint{3.419530in}{2.083776in}}{\pgfqpoint{3.419530in}{2.079658in}}%
\pgfpathcurveto{\pgfqpoint{3.419530in}{2.075539in}}{\pgfqpoint{3.421166in}{2.071589in}}{\pgfqpoint{3.424078in}{2.068677in}}%
\pgfpathcurveto{\pgfqpoint{3.426990in}{2.065765in}}{\pgfqpoint{3.430940in}{2.064129in}}{\pgfqpoint{3.435058in}{2.064129in}}%
\pgfpathclose%
\pgfusepath{fill}%
\end{pgfscope}%
\begin{pgfscope}%
\pgfpathrectangle{\pgfqpoint{0.887500in}{0.275000in}}{\pgfqpoint{4.225000in}{4.225000in}}%
\pgfusepath{clip}%
\pgfsetbuttcap%
\pgfsetroundjoin%
\definecolor{currentfill}{rgb}{0.000000,0.000000,0.000000}%
\pgfsetfillcolor{currentfill}%
\pgfsetfillopacity{0.800000}%
\pgfsetlinewidth{0.000000pt}%
\definecolor{currentstroke}{rgb}{0.000000,0.000000,0.000000}%
\pgfsetstrokecolor{currentstroke}%
\pgfsetstrokeopacity{0.800000}%
\pgfsetdash{}{0pt}%
\pgfpathmoveto{\pgfqpoint{3.206651in}{1.803359in}}%
\pgfpathcurveto{\pgfqpoint{3.210770in}{1.803359in}}{\pgfqpoint{3.214720in}{1.804995in}}{\pgfqpoint{3.217632in}{1.807907in}}%
\pgfpathcurveto{\pgfqpoint{3.220544in}{1.810819in}}{\pgfqpoint{3.222180in}{1.814769in}}{\pgfqpoint{3.222180in}{1.818887in}}%
\pgfpathcurveto{\pgfqpoint{3.222180in}{1.823006in}}{\pgfqpoint{3.220544in}{1.826956in}}{\pgfqpoint{3.217632in}{1.829868in}}%
\pgfpathcurveto{\pgfqpoint{3.214720in}{1.832779in}}{\pgfqpoint{3.210770in}{1.834416in}}{\pgfqpoint{3.206651in}{1.834416in}}%
\pgfpathcurveto{\pgfqpoint{3.202533in}{1.834416in}}{\pgfqpoint{3.198583in}{1.832779in}}{\pgfqpoint{3.195671in}{1.829868in}}%
\pgfpathcurveto{\pgfqpoint{3.192759in}{1.826956in}}{\pgfqpoint{3.191123in}{1.823006in}}{\pgfqpoint{3.191123in}{1.818887in}}%
\pgfpathcurveto{\pgfqpoint{3.191123in}{1.814769in}}{\pgfqpoint{3.192759in}{1.810819in}}{\pgfqpoint{3.195671in}{1.807907in}}%
\pgfpathcurveto{\pgfqpoint{3.198583in}{1.804995in}}{\pgfqpoint{3.202533in}{1.803359in}}{\pgfqpoint{3.206651in}{1.803359in}}%
\pgfpathclose%
\pgfusepath{fill}%
\end{pgfscope}%
\begin{pgfscope}%
\pgfpathrectangle{\pgfqpoint{0.887500in}{0.275000in}}{\pgfqpoint{4.225000in}{4.225000in}}%
\pgfusepath{clip}%
\pgfsetbuttcap%
\pgfsetroundjoin%
\definecolor{currentfill}{rgb}{0.000000,0.000000,0.000000}%
\pgfsetfillcolor{currentfill}%
\pgfsetfillopacity{0.800000}%
\pgfsetlinewidth{0.000000pt}%
\definecolor{currentstroke}{rgb}{0.000000,0.000000,0.000000}%
\pgfsetstrokecolor{currentstroke}%
\pgfsetstrokeopacity{0.800000}%
\pgfsetdash{}{0pt}%
\pgfpathmoveto{\pgfqpoint{3.549307in}{2.199209in}}%
\pgfpathcurveto{\pgfqpoint{3.553425in}{2.199209in}}{\pgfqpoint{3.557375in}{2.200845in}}{\pgfqpoint{3.560287in}{2.203757in}}%
\pgfpathcurveto{\pgfqpoint{3.563199in}{2.206669in}}{\pgfqpoint{3.564835in}{2.210619in}}{\pgfqpoint{3.564835in}{2.214737in}}%
\pgfpathcurveto{\pgfqpoint{3.564835in}{2.218855in}}{\pgfqpoint{3.563199in}{2.222805in}}{\pgfqpoint{3.560287in}{2.225717in}}%
\pgfpathcurveto{\pgfqpoint{3.557375in}{2.228629in}}{\pgfqpoint{3.553425in}{2.230265in}}{\pgfqpoint{3.549307in}{2.230265in}}%
\pgfpathcurveto{\pgfqpoint{3.545189in}{2.230265in}}{\pgfqpoint{3.541239in}{2.228629in}}{\pgfqpoint{3.538327in}{2.225717in}}%
\pgfpathcurveto{\pgfqpoint{3.535415in}{2.222805in}}{\pgfqpoint{3.533779in}{2.218855in}}{\pgfqpoint{3.533779in}{2.214737in}}%
\pgfpathcurveto{\pgfqpoint{3.533779in}{2.210619in}}{\pgfqpoint{3.535415in}{2.206669in}}{\pgfqpoint{3.538327in}{2.203757in}}%
\pgfpathcurveto{\pgfqpoint{3.541239in}{2.200845in}}{\pgfqpoint{3.545189in}{2.199209in}}{\pgfqpoint{3.549307in}{2.199209in}}%
\pgfpathclose%
\pgfusepath{fill}%
\end{pgfscope}%
\begin{pgfscope}%
\pgfpathrectangle{\pgfqpoint{0.887500in}{0.275000in}}{\pgfqpoint{4.225000in}{4.225000in}}%
\pgfusepath{clip}%
\pgfsetbuttcap%
\pgfsetroundjoin%
\definecolor{currentfill}{rgb}{0.000000,0.000000,0.000000}%
\pgfsetfillcolor{currentfill}%
\pgfsetfillopacity{0.800000}%
\pgfsetlinewidth{0.000000pt}%
\definecolor{currentstroke}{rgb}{0.000000,0.000000,0.000000}%
\pgfsetstrokecolor{currentstroke}%
\pgfsetstrokeopacity{0.800000}%
\pgfsetdash{}{0pt}%
\pgfpathmoveto{\pgfqpoint{4.090042in}{1.918924in}}%
\pgfpathcurveto{\pgfqpoint{4.094160in}{1.918924in}}{\pgfqpoint{4.098110in}{1.920560in}}{\pgfqpoint{4.101022in}{1.923472in}}%
\pgfpathcurveto{\pgfqpoint{4.103934in}{1.926384in}}{\pgfqpoint{4.105570in}{1.930334in}}{\pgfqpoint{4.105570in}{1.934452in}}%
\pgfpathcurveto{\pgfqpoint{4.105570in}{1.938570in}}{\pgfqpoint{4.103934in}{1.942520in}}{\pgfqpoint{4.101022in}{1.945432in}}%
\pgfpathcurveto{\pgfqpoint{4.098110in}{1.948344in}}{\pgfqpoint{4.094160in}{1.949980in}}{\pgfqpoint{4.090042in}{1.949980in}}%
\pgfpathcurveto{\pgfqpoint{4.085923in}{1.949980in}}{\pgfqpoint{4.081973in}{1.948344in}}{\pgfqpoint{4.079061in}{1.945432in}}%
\pgfpathcurveto{\pgfqpoint{4.076149in}{1.942520in}}{\pgfqpoint{4.074513in}{1.938570in}}{\pgfqpoint{4.074513in}{1.934452in}}%
\pgfpathcurveto{\pgfqpoint{4.074513in}{1.930334in}}{\pgfqpoint{4.076149in}{1.926384in}}{\pgfqpoint{4.079061in}{1.923472in}}%
\pgfpathcurveto{\pgfqpoint{4.081973in}{1.920560in}}{\pgfqpoint{4.085923in}{1.918924in}}{\pgfqpoint{4.090042in}{1.918924in}}%
\pgfpathclose%
\pgfusepath{fill}%
\end{pgfscope}%
\begin{pgfscope}%
\pgfpathrectangle{\pgfqpoint{0.887500in}{0.275000in}}{\pgfqpoint{4.225000in}{4.225000in}}%
\pgfusepath{clip}%
\pgfsetbuttcap%
\pgfsetroundjoin%
\definecolor{currentfill}{rgb}{0.000000,0.000000,0.000000}%
\pgfsetfillcolor{currentfill}%
\pgfsetfillopacity{0.800000}%
\pgfsetlinewidth{0.000000pt}%
\definecolor{currentstroke}{rgb}{0.000000,0.000000,0.000000}%
\pgfsetstrokecolor{currentstroke}%
\pgfsetstrokeopacity{0.800000}%
\pgfsetdash{}{0pt}%
\pgfpathmoveto{\pgfqpoint{1.896919in}{2.129976in}}%
\pgfpathcurveto{\pgfqpoint{1.901037in}{2.129976in}}{\pgfqpoint{1.904987in}{2.131612in}}{\pgfqpoint{1.907899in}{2.134524in}}%
\pgfpathcurveto{\pgfqpoint{1.910811in}{2.137436in}}{\pgfqpoint{1.912448in}{2.141386in}}{\pgfqpoint{1.912448in}{2.145504in}}%
\pgfpathcurveto{\pgfqpoint{1.912448in}{2.149622in}}{\pgfqpoint{1.910811in}{2.153572in}}{\pgfqpoint{1.907899in}{2.156484in}}%
\pgfpathcurveto{\pgfqpoint{1.904987in}{2.159396in}}{\pgfqpoint{1.901037in}{2.161032in}}{\pgfqpoint{1.896919in}{2.161032in}}%
\pgfpathcurveto{\pgfqpoint{1.892801in}{2.161032in}}{\pgfqpoint{1.888851in}{2.159396in}}{\pgfqpoint{1.885939in}{2.156484in}}%
\pgfpathcurveto{\pgfqpoint{1.883027in}{2.153572in}}{\pgfqpoint{1.881391in}{2.149622in}}{\pgfqpoint{1.881391in}{2.145504in}}%
\pgfpathcurveto{\pgfqpoint{1.881391in}{2.141386in}}{\pgfqpoint{1.883027in}{2.137436in}}{\pgfqpoint{1.885939in}{2.134524in}}%
\pgfpathcurveto{\pgfqpoint{1.888851in}{2.131612in}}{\pgfqpoint{1.892801in}{2.129976in}}{\pgfqpoint{1.896919in}{2.129976in}}%
\pgfpathclose%
\pgfusepath{fill}%
\end{pgfscope}%
\begin{pgfscope}%
\pgfpathrectangle{\pgfqpoint{0.887500in}{0.275000in}}{\pgfqpoint{4.225000in}{4.225000in}}%
\pgfusepath{clip}%
\pgfsetbuttcap%
\pgfsetroundjoin%
\definecolor{currentfill}{rgb}{0.000000,0.000000,0.000000}%
\pgfsetfillcolor{currentfill}%
\pgfsetfillopacity{0.800000}%
\pgfsetlinewidth{0.000000pt}%
\definecolor{currentstroke}{rgb}{0.000000,0.000000,0.000000}%
\pgfsetstrokecolor{currentstroke}%
\pgfsetstrokeopacity{0.800000}%
\pgfsetdash{}{0pt}%
\pgfpathmoveto{\pgfqpoint{4.270340in}{1.825351in}}%
\pgfpathcurveto{\pgfqpoint{4.274459in}{1.825351in}}{\pgfqpoint{4.278409in}{1.826987in}}{\pgfqpoint{4.281321in}{1.829899in}}%
\pgfpathcurveto{\pgfqpoint{4.284233in}{1.832811in}}{\pgfqpoint{4.285869in}{1.836761in}}{\pgfqpoint{4.285869in}{1.840879in}}%
\pgfpathcurveto{\pgfqpoint{4.285869in}{1.844998in}}{\pgfqpoint{4.284233in}{1.848948in}}{\pgfqpoint{4.281321in}{1.851860in}}%
\pgfpathcurveto{\pgfqpoint{4.278409in}{1.854772in}}{\pgfqpoint{4.274459in}{1.856408in}}{\pgfqpoint{4.270340in}{1.856408in}}%
\pgfpathcurveto{\pgfqpoint{4.266222in}{1.856408in}}{\pgfqpoint{4.262272in}{1.854772in}}{\pgfqpoint{4.259360in}{1.851860in}}%
\pgfpathcurveto{\pgfqpoint{4.256448in}{1.848948in}}{\pgfqpoint{4.254812in}{1.844998in}}{\pgfqpoint{4.254812in}{1.840879in}}%
\pgfpathcurveto{\pgfqpoint{4.254812in}{1.836761in}}{\pgfqpoint{4.256448in}{1.832811in}}{\pgfqpoint{4.259360in}{1.829899in}}%
\pgfpathcurveto{\pgfqpoint{4.262272in}{1.826987in}}{\pgfqpoint{4.266222in}{1.825351in}}{\pgfqpoint{4.270340in}{1.825351in}}%
\pgfpathclose%
\pgfusepath{fill}%
\end{pgfscope}%
\begin{pgfscope}%
\pgfpathrectangle{\pgfqpoint{0.887500in}{0.275000in}}{\pgfqpoint{4.225000in}{4.225000in}}%
\pgfusepath{clip}%
\pgfsetbuttcap%
\pgfsetroundjoin%
\definecolor{currentfill}{rgb}{0.000000,0.000000,0.000000}%
\pgfsetfillcolor{currentfill}%
\pgfsetfillopacity{0.800000}%
\pgfsetlinewidth{0.000000pt}%
\definecolor{currentstroke}{rgb}{0.000000,0.000000,0.000000}%
\pgfsetstrokecolor{currentstroke}%
\pgfsetstrokeopacity{0.800000}%
\pgfsetdash{}{0pt}%
\pgfpathmoveto{\pgfqpoint{3.386948in}{1.722693in}}%
\pgfpathcurveto{\pgfqpoint{3.391066in}{1.722693in}}{\pgfqpoint{3.395016in}{1.724329in}}{\pgfqpoint{3.397928in}{1.727241in}}%
\pgfpathcurveto{\pgfqpoint{3.400840in}{1.730153in}}{\pgfqpoint{3.402476in}{1.734103in}}{\pgfqpoint{3.402476in}{1.738221in}}%
\pgfpathcurveto{\pgfqpoint{3.402476in}{1.742339in}}{\pgfqpoint{3.400840in}{1.746289in}}{\pgfqpoint{3.397928in}{1.749201in}}%
\pgfpathcurveto{\pgfqpoint{3.395016in}{1.752113in}}{\pgfqpoint{3.391066in}{1.753749in}}{\pgfqpoint{3.386948in}{1.753749in}}%
\pgfpathcurveto{\pgfqpoint{3.382830in}{1.753749in}}{\pgfqpoint{3.378880in}{1.752113in}}{\pgfqpoint{3.375968in}{1.749201in}}%
\pgfpathcurveto{\pgfqpoint{3.373056in}{1.746289in}}{\pgfqpoint{3.371419in}{1.742339in}}{\pgfqpoint{3.371419in}{1.738221in}}%
\pgfpathcurveto{\pgfqpoint{3.371419in}{1.734103in}}{\pgfqpoint{3.373056in}{1.730153in}}{\pgfqpoint{3.375968in}{1.727241in}}%
\pgfpathcurveto{\pgfqpoint{3.378880in}{1.724329in}}{\pgfqpoint{3.382830in}{1.722693in}}{\pgfqpoint{3.386948in}{1.722693in}}%
\pgfpathclose%
\pgfusepath{fill}%
\end{pgfscope}%
\begin{pgfscope}%
\pgfpathrectangle{\pgfqpoint{0.887500in}{0.275000in}}{\pgfqpoint{4.225000in}{4.225000in}}%
\pgfusepath{clip}%
\pgfsetbuttcap%
\pgfsetroundjoin%
\definecolor{currentfill}{rgb}{0.000000,0.000000,0.000000}%
\pgfsetfillcolor{currentfill}%
\pgfsetfillopacity{0.800000}%
\pgfsetlinewidth{0.000000pt}%
\definecolor{currentstroke}{rgb}{0.000000,0.000000,0.000000}%
\pgfsetstrokecolor{currentstroke}%
\pgfsetstrokeopacity{0.800000}%
\pgfsetdash{}{0pt}%
\pgfpathmoveto{\pgfqpoint{3.501245in}{1.857553in}}%
\pgfpathcurveto{\pgfqpoint{3.505364in}{1.857553in}}{\pgfqpoint{3.509314in}{1.859189in}}{\pgfqpoint{3.512226in}{1.862101in}}%
\pgfpathcurveto{\pgfqpoint{3.515138in}{1.865013in}}{\pgfqpoint{3.516774in}{1.868963in}}{\pgfqpoint{3.516774in}{1.873081in}}%
\pgfpathcurveto{\pgfqpoint{3.516774in}{1.877199in}}{\pgfqpoint{3.515138in}{1.881149in}}{\pgfqpoint{3.512226in}{1.884061in}}%
\pgfpathcurveto{\pgfqpoint{3.509314in}{1.886973in}}{\pgfqpoint{3.505364in}{1.888610in}}{\pgfqpoint{3.501245in}{1.888610in}}%
\pgfpathcurveto{\pgfqpoint{3.497127in}{1.888610in}}{\pgfqpoint{3.493177in}{1.886973in}}{\pgfqpoint{3.490265in}{1.884061in}}%
\pgfpathcurveto{\pgfqpoint{3.487353in}{1.881149in}}{\pgfqpoint{3.485717in}{1.877199in}}{\pgfqpoint{3.485717in}{1.873081in}}%
\pgfpathcurveto{\pgfqpoint{3.485717in}{1.868963in}}{\pgfqpoint{3.487353in}{1.865013in}}{\pgfqpoint{3.490265in}{1.862101in}}%
\pgfpathcurveto{\pgfqpoint{3.493177in}{1.859189in}}{\pgfqpoint{3.497127in}{1.857553in}}{\pgfqpoint{3.501245in}{1.857553in}}%
\pgfpathclose%
\pgfusepath{fill}%
\end{pgfscope}%
\begin{pgfscope}%
\pgfpathrectangle{\pgfqpoint{0.887500in}{0.275000in}}{\pgfqpoint{4.225000in}{4.225000in}}%
\pgfusepath{clip}%
\pgfsetbuttcap%
\pgfsetroundjoin%
\definecolor{currentfill}{rgb}{0.000000,0.000000,0.000000}%
\pgfsetfillcolor{currentfill}%
\pgfsetfillopacity{0.800000}%
\pgfsetlinewidth{0.000000pt}%
\definecolor{currentstroke}{rgb}{0.000000,0.000000,0.000000}%
\pgfsetstrokecolor{currentstroke}%
\pgfsetstrokeopacity{0.800000}%
\pgfsetdash{}{0pt}%
\pgfpathmoveto{\pgfqpoint{2.551693in}{1.980955in}}%
\pgfpathcurveto{\pgfqpoint{2.555811in}{1.980955in}}{\pgfqpoint{2.559761in}{1.982592in}}{\pgfqpoint{2.562673in}{1.985504in}}%
\pgfpathcurveto{\pgfqpoint{2.565585in}{1.988416in}}{\pgfqpoint{2.567221in}{1.992366in}}{\pgfqpoint{2.567221in}{1.996484in}}%
\pgfpathcurveto{\pgfqpoint{2.567221in}{2.000602in}}{\pgfqpoint{2.565585in}{2.004552in}}{\pgfqpoint{2.562673in}{2.007464in}}%
\pgfpathcurveto{\pgfqpoint{2.559761in}{2.010376in}}{\pgfqpoint{2.555811in}{2.012012in}}{\pgfqpoint{2.551693in}{2.012012in}}%
\pgfpathcurveto{\pgfqpoint{2.547575in}{2.012012in}}{\pgfqpoint{2.543625in}{2.010376in}}{\pgfqpoint{2.540713in}{2.007464in}}%
\pgfpathcurveto{\pgfqpoint{2.537801in}{2.004552in}}{\pgfqpoint{2.536165in}{2.000602in}}{\pgfqpoint{2.536165in}{1.996484in}}%
\pgfpathcurveto{\pgfqpoint{2.536165in}{1.992366in}}{\pgfqpoint{2.537801in}{1.988416in}}{\pgfqpoint{2.540713in}{1.985504in}}%
\pgfpathcurveto{\pgfqpoint{2.543625in}{1.982592in}}{\pgfqpoint{2.547575in}{1.980955in}}{\pgfqpoint{2.551693in}{1.980955in}}%
\pgfpathclose%
\pgfusepath{fill}%
\end{pgfscope}%
\begin{pgfscope}%
\pgfpathrectangle{\pgfqpoint{0.887500in}{0.275000in}}{\pgfqpoint{4.225000in}{4.225000in}}%
\pgfusepath{clip}%
\pgfsetbuttcap%
\pgfsetroundjoin%
\definecolor{currentfill}{rgb}{0.000000,0.000000,0.000000}%
\pgfsetfillcolor{currentfill}%
\pgfsetfillopacity{0.800000}%
\pgfsetlinewidth{0.000000pt}%
\definecolor{currentstroke}{rgb}{0.000000,0.000000,0.000000}%
\pgfsetstrokecolor{currentstroke}%
\pgfsetstrokeopacity{0.800000}%
\pgfsetdash{}{0pt}%
\pgfpathmoveto{\pgfqpoint{3.567302in}{1.635141in}}%
\pgfpathcurveto{\pgfqpoint{3.571420in}{1.635141in}}{\pgfqpoint{3.575370in}{1.636777in}}{\pgfqpoint{3.578282in}{1.639689in}}%
\pgfpathcurveto{\pgfqpoint{3.581194in}{1.642601in}}{\pgfqpoint{3.582830in}{1.646551in}}{\pgfqpoint{3.582830in}{1.650669in}}%
\pgfpathcurveto{\pgfqpoint{3.582830in}{1.654787in}}{\pgfqpoint{3.581194in}{1.658737in}}{\pgfqpoint{3.578282in}{1.661649in}}%
\pgfpathcurveto{\pgfqpoint{3.575370in}{1.664561in}}{\pgfqpoint{3.571420in}{1.666197in}}{\pgfqpoint{3.567302in}{1.666197in}}%
\pgfpathcurveto{\pgfqpoint{3.563184in}{1.666197in}}{\pgfqpoint{3.559234in}{1.664561in}}{\pgfqpoint{3.556322in}{1.661649in}}%
\pgfpathcurveto{\pgfqpoint{3.553410in}{1.658737in}}{\pgfqpoint{3.551774in}{1.654787in}}{\pgfqpoint{3.551774in}{1.650669in}}%
\pgfpathcurveto{\pgfqpoint{3.551774in}{1.646551in}}{\pgfqpoint{3.553410in}{1.642601in}}{\pgfqpoint{3.556322in}{1.639689in}}%
\pgfpathcurveto{\pgfqpoint{3.559234in}{1.636777in}}{\pgfqpoint{3.563184in}{1.635141in}}{\pgfqpoint{3.567302in}{1.635141in}}%
\pgfpathclose%
\pgfusepath{fill}%
\end{pgfscope}%
\begin{pgfscope}%
\pgfpathrectangle{\pgfqpoint{0.887500in}{0.275000in}}{\pgfqpoint{4.225000in}{4.225000in}}%
\pgfusepath{clip}%
\pgfsetbuttcap%
\pgfsetroundjoin%
\definecolor{currentfill}{rgb}{0.000000,0.000000,0.000000}%
\pgfsetfillcolor{currentfill}%
\pgfsetfillopacity{0.800000}%
\pgfsetlinewidth{0.000000pt}%
\definecolor{currentstroke}{rgb}{0.000000,0.000000,0.000000}%
\pgfsetstrokecolor{currentstroke}%
\pgfsetstrokeopacity{0.800000}%
\pgfsetdash{}{0pt}%
\pgfpathmoveto{\pgfqpoint{3.910294in}{2.036497in}}%
\pgfpathcurveto{\pgfqpoint{3.914412in}{2.036497in}}{\pgfqpoint{3.918362in}{2.038133in}}{\pgfqpoint{3.921274in}{2.041045in}}%
\pgfpathcurveto{\pgfqpoint{3.924186in}{2.043957in}}{\pgfqpoint{3.925822in}{2.047907in}}{\pgfqpoint{3.925822in}{2.052025in}}%
\pgfpathcurveto{\pgfqpoint{3.925822in}{2.056143in}}{\pgfqpoint{3.924186in}{2.060093in}}{\pgfqpoint{3.921274in}{2.063005in}}%
\pgfpathcurveto{\pgfqpoint{3.918362in}{2.065917in}}{\pgfqpoint{3.914412in}{2.067553in}}{\pgfqpoint{3.910294in}{2.067553in}}%
\pgfpathcurveto{\pgfqpoint{3.906175in}{2.067553in}}{\pgfqpoint{3.902225in}{2.065917in}}{\pgfqpoint{3.899314in}{2.063005in}}%
\pgfpathcurveto{\pgfqpoint{3.896402in}{2.060093in}}{\pgfqpoint{3.894765in}{2.056143in}}{\pgfqpoint{3.894765in}{2.052025in}}%
\pgfpathcurveto{\pgfqpoint{3.894765in}{2.047907in}}{\pgfqpoint{3.896402in}{2.043957in}}{\pgfqpoint{3.899314in}{2.041045in}}%
\pgfpathcurveto{\pgfqpoint{3.902225in}{2.038133in}}{\pgfqpoint{3.906175in}{2.036497in}}{\pgfqpoint{3.910294in}{2.036497in}}%
\pgfpathclose%
\pgfusepath{fill}%
\end{pgfscope}%
\begin{pgfscope}%
\pgfpathrectangle{\pgfqpoint{0.887500in}{0.275000in}}{\pgfqpoint{4.225000in}{4.225000in}}%
\pgfusepath{clip}%
\pgfsetbuttcap%
\pgfsetroundjoin%
\definecolor{currentfill}{rgb}{0.000000,0.000000,0.000000}%
\pgfsetfillcolor{currentfill}%
\pgfsetfillopacity{0.800000}%
\pgfsetlinewidth{0.000000pt}%
\definecolor{currentstroke}{rgb}{0.000000,0.000000,0.000000}%
\pgfsetstrokecolor{currentstroke}%
\pgfsetstrokeopacity{0.800000}%
\pgfsetdash{}{0pt}%
\pgfpathmoveto{\pgfqpoint{3.730176in}{2.143914in}}%
\pgfpathcurveto{\pgfqpoint{3.734295in}{2.143914in}}{\pgfqpoint{3.738245in}{2.145550in}}{\pgfqpoint{3.741157in}{2.148462in}}%
\pgfpathcurveto{\pgfqpoint{3.744069in}{2.151374in}}{\pgfqpoint{3.745705in}{2.155324in}}{\pgfqpoint{3.745705in}{2.159442in}}%
\pgfpathcurveto{\pgfqpoint{3.745705in}{2.163560in}}{\pgfqpoint{3.744069in}{2.167510in}}{\pgfqpoint{3.741157in}{2.170422in}}%
\pgfpathcurveto{\pgfqpoint{3.738245in}{2.173334in}}{\pgfqpoint{3.734295in}{2.174970in}}{\pgfqpoint{3.730176in}{2.174970in}}%
\pgfpathcurveto{\pgfqpoint{3.726058in}{2.174970in}}{\pgfqpoint{3.722108in}{2.173334in}}{\pgfqpoint{3.719196in}{2.170422in}}%
\pgfpathcurveto{\pgfqpoint{3.716284in}{2.167510in}}{\pgfqpoint{3.714648in}{2.163560in}}{\pgfqpoint{3.714648in}{2.159442in}}%
\pgfpathcurveto{\pgfqpoint{3.714648in}{2.155324in}}{\pgfqpoint{3.716284in}{2.151374in}}{\pgfqpoint{3.719196in}{2.148462in}}%
\pgfpathcurveto{\pgfqpoint{3.722108in}{2.145550in}}{\pgfqpoint{3.726058in}{2.143914in}}{\pgfqpoint{3.730176in}{2.143914in}}%
\pgfpathclose%
\pgfusepath{fill}%
\end{pgfscope}%
\begin{pgfscope}%
\pgfpathrectangle{\pgfqpoint{0.887500in}{0.275000in}}{\pgfqpoint{4.225000in}{4.225000in}}%
\pgfusepath{clip}%
\pgfsetbuttcap%
\pgfsetroundjoin%
\definecolor{currentfill}{rgb}{0.000000,0.000000,0.000000}%
\pgfsetfillcolor{currentfill}%
\pgfsetfillopacity{0.800000}%
\pgfsetlinewidth{0.000000pt}%
\definecolor{currentstroke}{rgb}{0.000000,0.000000,0.000000}%
\pgfsetstrokecolor{currentstroke}%
\pgfsetstrokeopacity{0.800000}%
\pgfsetdash{}{0pt}%
\pgfpathmoveto{\pgfqpoint{3.615893in}{2.015757in}}%
\pgfpathcurveto{\pgfqpoint{3.620011in}{2.015757in}}{\pgfqpoint{3.623961in}{2.017393in}}{\pgfqpoint{3.626873in}{2.020305in}}%
\pgfpathcurveto{\pgfqpoint{3.629785in}{2.023217in}}{\pgfqpoint{3.631422in}{2.027167in}}{\pgfqpoint{3.631422in}{2.031285in}}%
\pgfpathcurveto{\pgfqpoint{3.631422in}{2.035403in}}{\pgfqpoint{3.629785in}{2.039353in}}{\pgfqpoint{3.626873in}{2.042265in}}%
\pgfpathcurveto{\pgfqpoint{3.623961in}{2.045177in}}{\pgfqpoint{3.620011in}{2.046813in}}{\pgfqpoint{3.615893in}{2.046813in}}%
\pgfpathcurveto{\pgfqpoint{3.611775in}{2.046813in}}{\pgfqpoint{3.607825in}{2.045177in}}{\pgfqpoint{3.604913in}{2.042265in}}%
\pgfpathcurveto{\pgfqpoint{3.602001in}{2.039353in}}{\pgfqpoint{3.600365in}{2.035403in}}{\pgfqpoint{3.600365in}{2.031285in}}%
\pgfpathcurveto{\pgfqpoint{3.600365in}{2.027167in}}{\pgfqpoint{3.602001in}{2.023217in}}{\pgfqpoint{3.604913in}{2.020305in}}%
\pgfpathcurveto{\pgfqpoint{3.607825in}{2.017393in}}{\pgfqpoint{3.611775in}{2.015757in}}{\pgfqpoint{3.615893in}{2.015757in}}%
\pgfpathclose%
\pgfusepath{fill}%
\end{pgfscope}%
\begin{pgfscope}%
\pgfpathrectangle{\pgfqpoint{0.887500in}{0.275000in}}{\pgfqpoint{4.225000in}{4.225000in}}%
\pgfusepath{clip}%
\pgfsetbuttcap%
\pgfsetroundjoin%
\definecolor{currentfill}{rgb}{0.000000,0.000000,0.000000}%
\pgfsetfillcolor{currentfill}%
\pgfsetfillopacity{0.800000}%
\pgfsetlinewidth{0.000000pt}%
\definecolor{currentstroke}{rgb}{0.000000,0.000000,0.000000}%
\pgfsetstrokecolor{currentstroke}%
\pgfsetstrokeopacity{0.800000}%
\pgfsetdash{}{0pt}%
\pgfpathmoveto{\pgfqpoint{2.076222in}{2.069763in}}%
\pgfpathcurveto{\pgfqpoint{2.080340in}{2.069763in}}{\pgfqpoint{2.084290in}{2.071399in}}{\pgfqpoint{2.087202in}{2.074311in}}%
\pgfpathcurveto{\pgfqpoint{2.090114in}{2.077223in}}{\pgfqpoint{2.091750in}{2.081173in}}{\pgfqpoint{2.091750in}{2.085291in}}%
\pgfpathcurveto{\pgfqpoint{2.091750in}{2.089409in}}{\pgfqpoint{2.090114in}{2.093359in}}{\pgfqpoint{2.087202in}{2.096271in}}%
\pgfpathcurveto{\pgfqpoint{2.084290in}{2.099183in}}{\pgfqpoint{2.080340in}{2.100819in}}{\pgfqpoint{2.076222in}{2.100819in}}%
\pgfpathcurveto{\pgfqpoint{2.072104in}{2.100819in}}{\pgfqpoint{2.068154in}{2.099183in}}{\pgfqpoint{2.065242in}{2.096271in}}%
\pgfpathcurveto{\pgfqpoint{2.062330in}{2.093359in}}{\pgfqpoint{2.060693in}{2.089409in}}{\pgfqpoint{2.060693in}{2.085291in}}%
\pgfpathcurveto{\pgfqpoint{2.060693in}{2.081173in}}{\pgfqpoint{2.062330in}{2.077223in}}{\pgfqpoint{2.065242in}{2.074311in}}%
\pgfpathcurveto{\pgfqpoint{2.068154in}{2.071399in}}{\pgfqpoint{2.072104in}{2.069763in}}{\pgfqpoint{2.076222in}{2.069763in}}%
\pgfpathclose%
\pgfusepath{fill}%
\end{pgfscope}%
\begin{pgfscope}%
\pgfpathrectangle{\pgfqpoint{0.887500in}{0.275000in}}{\pgfqpoint{4.225000in}{4.225000in}}%
\pgfusepath{clip}%
\pgfsetbuttcap%
\pgfsetroundjoin%
\definecolor{currentfill}{rgb}{0.000000,0.000000,0.000000}%
\pgfsetfillcolor{currentfill}%
\pgfsetfillopacity{0.800000}%
\pgfsetlinewidth{0.000000pt}%
\definecolor{currentstroke}{rgb}{0.000000,0.000000,0.000000}%
\pgfsetstrokecolor{currentstroke}%
\pgfsetstrokeopacity{0.800000}%
\pgfsetdash{}{0pt}%
\pgfpathmoveto{\pgfqpoint{2.731703in}{1.913459in}}%
\pgfpathcurveto{\pgfqpoint{2.735821in}{1.913459in}}{\pgfqpoint{2.739772in}{1.915095in}}{\pgfqpoint{2.742683in}{1.918007in}}%
\pgfpathcurveto{\pgfqpoint{2.745595in}{1.920919in}}{\pgfqpoint{2.747232in}{1.924869in}}{\pgfqpoint{2.747232in}{1.928987in}}%
\pgfpathcurveto{\pgfqpoint{2.747232in}{1.933106in}}{\pgfqpoint{2.745595in}{1.937056in}}{\pgfqpoint{2.742683in}{1.939968in}}%
\pgfpathcurveto{\pgfqpoint{2.739772in}{1.942880in}}{\pgfqpoint{2.735821in}{1.944516in}}{\pgfqpoint{2.731703in}{1.944516in}}%
\pgfpathcurveto{\pgfqpoint{2.727585in}{1.944516in}}{\pgfqpoint{2.723635in}{1.942880in}}{\pgfqpoint{2.720723in}{1.939968in}}%
\pgfpathcurveto{\pgfqpoint{2.717811in}{1.937056in}}{\pgfqpoint{2.716175in}{1.933106in}}{\pgfqpoint{2.716175in}{1.928987in}}%
\pgfpathcurveto{\pgfqpoint{2.716175in}{1.924869in}}{\pgfqpoint{2.717811in}{1.920919in}}{\pgfqpoint{2.720723in}{1.918007in}}%
\pgfpathcurveto{\pgfqpoint{2.723635in}{1.915095in}}{\pgfqpoint{2.727585in}{1.913459in}}{\pgfqpoint{2.731703in}{1.913459in}}%
\pgfpathclose%
\pgfusepath{fill}%
\end{pgfscope}%
\begin{pgfscope}%
\pgfpathrectangle{\pgfqpoint{0.887500in}{0.275000in}}{\pgfqpoint{4.225000in}{4.225000in}}%
\pgfusepath{clip}%
\pgfsetbuttcap%
\pgfsetroundjoin%
\definecolor{currentfill}{rgb}{0.000000,0.000000,0.000000}%
\pgfsetfillcolor{currentfill}%
\pgfsetfillopacity{0.800000}%
\pgfsetlinewidth{0.000000pt}%
\definecolor{currentstroke}{rgb}{0.000000,0.000000,0.000000}%
\pgfsetstrokecolor{currentstroke}%
\pgfsetstrokeopacity{0.800000}%
\pgfsetdash{}{0pt}%
\pgfpathmoveto{\pgfqpoint{3.682363in}{1.815071in}}%
\pgfpathcurveto{\pgfqpoint{3.686481in}{1.815071in}}{\pgfqpoint{3.690431in}{1.816707in}}{\pgfqpoint{3.693343in}{1.819619in}}%
\pgfpathcurveto{\pgfqpoint{3.696255in}{1.822531in}}{\pgfqpoint{3.697891in}{1.826481in}}{\pgfqpoint{3.697891in}{1.830599in}}%
\pgfpathcurveto{\pgfqpoint{3.697891in}{1.834717in}}{\pgfqpoint{3.696255in}{1.838667in}}{\pgfqpoint{3.693343in}{1.841579in}}%
\pgfpathcurveto{\pgfqpoint{3.690431in}{1.844491in}}{\pgfqpoint{3.686481in}{1.846127in}}{\pgfqpoint{3.682363in}{1.846127in}}%
\pgfpathcurveto{\pgfqpoint{3.678245in}{1.846127in}}{\pgfqpoint{3.674295in}{1.844491in}}{\pgfqpoint{3.671383in}{1.841579in}}%
\pgfpathcurveto{\pgfqpoint{3.668471in}{1.838667in}}{\pgfqpoint{3.666835in}{1.834717in}}{\pgfqpoint{3.666835in}{1.830599in}}%
\pgfpathcurveto{\pgfqpoint{3.666835in}{1.826481in}}{\pgfqpoint{3.668471in}{1.822531in}}{\pgfqpoint{3.671383in}{1.819619in}}%
\pgfpathcurveto{\pgfqpoint{3.674295in}{1.816707in}}{\pgfqpoint{3.678245in}{1.815071in}}{\pgfqpoint{3.682363in}{1.815071in}}%
\pgfpathclose%
\pgfusepath{fill}%
\end{pgfscope}%
\begin{pgfscope}%
\pgfpathrectangle{\pgfqpoint{0.887500in}{0.275000in}}{\pgfqpoint{4.225000in}{4.225000in}}%
\pgfusepath{clip}%
\pgfsetbuttcap%
\pgfsetroundjoin%
\definecolor{currentfill}{rgb}{0.000000,0.000000,0.000000}%
\pgfsetfillcolor{currentfill}%
\pgfsetfillopacity{0.800000}%
\pgfsetlinewidth{0.000000pt}%
\definecolor{currentstroke}{rgb}{0.000000,0.000000,0.000000}%
\pgfsetstrokecolor{currentstroke}%
\pgfsetstrokeopacity{0.800000}%
\pgfsetdash{}{0pt}%
\pgfpathmoveto{\pgfqpoint{2.911996in}{1.842087in}}%
\pgfpathcurveto{\pgfqpoint{2.916115in}{1.842087in}}{\pgfqpoint{2.920065in}{1.843723in}}{\pgfqpoint{2.922977in}{1.846635in}}%
\pgfpathcurveto{\pgfqpoint{2.925889in}{1.849547in}}{\pgfqpoint{2.927525in}{1.853497in}}{\pgfqpoint{2.927525in}{1.857615in}}%
\pgfpathcurveto{\pgfqpoint{2.927525in}{1.861733in}}{\pgfqpoint{2.925889in}{1.865683in}}{\pgfqpoint{2.922977in}{1.868595in}}%
\pgfpathcurveto{\pgfqpoint{2.920065in}{1.871507in}}{\pgfqpoint{2.916115in}{1.873143in}}{\pgfqpoint{2.911996in}{1.873143in}}%
\pgfpathcurveto{\pgfqpoint{2.907878in}{1.873143in}}{\pgfqpoint{2.903928in}{1.871507in}}{\pgfqpoint{2.901016in}{1.868595in}}%
\pgfpathcurveto{\pgfqpoint{2.898104in}{1.865683in}}{\pgfqpoint{2.896468in}{1.861733in}}{\pgfqpoint{2.896468in}{1.857615in}}%
\pgfpathcurveto{\pgfqpoint{2.896468in}{1.853497in}}{\pgfqpoint{2.898104in}{1.849547in}}{\pgfqpoint{2.901016in}{1.846635in}}%
\pgfpathcurveto{\pgfqpoint{2.903928in}{1.843723in}}{\pgfqpoint{2.907878in}{1.842087in}}{\pgfqpoint{2.911996in}{1.842087in}}%
\pgfpathclose%
\pgfusepath{fill}%
\end{pgfscope}%
\begin{pgfscope}%
\pgfpathrectangle{\pgfqpoint{0.887500in}{0.275000in}}{\pgfqpoint{4.225000in}{4.225000in}}%
\pgfusepath{clip}%
\pgfsetbuttcap%
\pgfsetroundjoin%
\definecolor{currentfill}{rgb}{0.000000,0.000000,0.000000}%
\pgfsetfillcolor{currentfill}%
\pgfsetfillopacity{0.800000}%
\pgfsetlinewidth{0.000000pt}%
\definecolor{currentstroke}{rgb}{0.000000,0.000000,0.000000}%
\pgfsetstrokecolor{currentstroke}%
\pgfsetstrokeopacity{0.800000}%
\pgfsetdash{}{0pt}%
\pgfpathmoveto{\pgfqpoint{3.796951in}{1.955471in}}%
\pgfpathcurveto{\pgfqpoint{3.801070in}{1.955471in}}{\pgfqpoint{3.805020in}{1.957107in}}{\pgfqpoint{3.807932in}{1.960019in}}%
\pgfpathcurveto{\pgfqpoint{3.810844in}{1.962931in}}{\pgfqpoint{3.812480in}{1.966881in}}{\pgfqpoint{3.812480in}{1.970999in}}%
\pgfpathcurveto{\pgfqpoint{3.812480in}{1.975117in}}{\pgfqpoint{3.810844in}{1.979067in}}{\pgfqpoint{3.807932in}{1.981979in}}%
\pgfpathcurveto{\pgfqpoint{3.805020in}{1.984891in}}{\pgfqpoint{3.801070in}{1.986527in}}{\pgfqpoint{3.796951in}{1.986527in}}%
\pgfpathcurveto{\pgfqpoint{3.792833in}{1.986527in}}{\pgfqpoint{3.788883in}{1.984891in}}{\pgfqpoint{3.785971in}{1.981979in}}%
\pgfpathcurveto{\pgfqpoint{3.783059in}{1.979067in}}{\pgfqpoint{3.781423in}{1.975117in}}{\pgfqpoint{3.781423in}{1.970999in}}%
\pgfpathcurveto{\pgfqpoint{3.781423in}{1.966881in}}{\pgfqpoint{3.783059in}{1.962931in}}{\pgfqpoint{3.785971in}{1.960019in}}%
\pgfpathcurveto{\pgfqpoint{3.788883in}{1.957107in}}{\pgfqpoint{3.792833in}{1.955471in}}{\pgfqpoint{3.796951in}{1.955471in}}%
\pgfpathclose%
\pgfusepath{fill}%
\end{pgfscope}%
\begin{pgfscope}%
\pgfpathrectangle{\pgfqpoint{0.887500in}{0.275000in}}{\pgfqpoint{4.225000in}{4.225000in}}%
\pgfusepath{clip}%
\pgfsetbuttcap%
\pgfsetroundjoin%
\definecolor{currentfill}{rgb}{0.000000,0.000000,0.000000}%
\pgfsetfillcolor{currentfill}%
\pgfsetfillopacity{0.800000}%
\pgfsetlinewidth{0.000000pt}%
\definecolor{currentstroke}{rgb}{0.000000,0.000000,0.000000}%
\pgfsetstrokecolor{currentstroke}%
\pgfsetstrokeopacity{0.800000}%
\pgfsetdash{}{0pt}%
\pgfpathmoveto{\pgfqpoint{2.255889in}{2.007886in}}%
\pgfpathcurveto{\pgfqpoint{2.260007in}{2.007886in}}{\pgfqpoint{2.263957in}{2.009522in}}{\pgfqpoint{2.266869in}{2.012434in}}%
\pgfpathcurveto{\pgfqpoint{2.269781in}{2.015346in}}{\pgfqpoint{2.271417in}{2.019296in}}{\pgfqpoint{2.271417in}{2.023414in}}%
\pgfpathcurveto{\pgfqpoint{2.271417in}{2.027532in}}{\pgfqpoint{2.269781in}{2.031482in}}{\pgfqpoint{2.266869in}{2.034394in}}%
\pgfpathcurveto{\pgfqpoint{2.263957in}{2.037306in}}{\pgfqpoint{2.260007in}{2.038942in}}{\pgfqpoint{2.255889in}{2.038942in}}%
\pgfpathcurveto{\pgfqpoint{2.251771in}{2.038942in}}{\pgfqpoint{2.247821in}{2.037306in}}{\pgfqpoint{2.244909in}{2.034394in}}%
\pgfpathcurveto{\pgfqpoint{2.241997in}{2.031482in}}{\pgfqpoint{2.240361in}{2.027532in}}{\pgfqpoint{2.240361in}{2.023414in}}%
\pgfpathcurveto{\pgfqpoint{2.240361in}{2.019296in}}{\pgfqpoint{2.241997in}{2.015346in}}{\pgfqpoint{2.244909in}{2.012434in}}%
\pgfpathcurveto{\pgfqpoint{2.247821in}{2.009522in}}{\pgfqpoint{2.251771in}{2.007886in}}{\pgfqpoint{2.255889in}{2.007886in}}%
\pgfpathclose%
\pgfusepath{fill}%
\end{pgfscope}%
\begin{pgfscope}%
\pgfpathrectangle{\pgfqpoint{0.887500in}{0.275000in}}{\pgfqpoint{4.225000in}{4.225000in}}%
\pgfusepath{clip}%
\pgfsetbuttcap%
\pgfsetroundjoin%
\definecolor{currentfill}{rgb}{0.000000,0.000000,0.000000}%
\pgfsetfillcolor{currentfill}%
\pgfsetfillopacity{0.800000}%
\pgfsetlinewidth{0.000000pt}%
\definecolor{currentstroke}{rgb}{0.000000,0.000000,0.000000}%
\pgfsetstrokecolor{currentstroke}%
\pgfsetstrokeopacity{0.800000}%
\pgfsetdash{}{0pt}%
\pgfpathmoveto{\pgfqpoint{3.977560in}{1.861729in}}%
\pgfpathcurveto{\pgfqpoint{3.981678in}{1.861729in}}{\pgfqpoint{3.985628in}{1.863365in}}{\pgfqpoint{3.988540in}{1.866277in}}%
\pgfpathcurveto{\pgfqpoint{3.991452in}{1.869189in}}{\pgfqpoint{3.993088in}{1.873139in}}{\pgfqpoint{3.993088in}{1.877257in}}%
\pgfpathcurveto{\pgfqpoint{3.993088in}{1.881375in}}{\pgfqpoint{3.991452in}{1.885325in}}{\pgfqpoint{3.988540in}{1.888237in}}%
\pgfpathcurveto{\pgfqpoint{3.985628in}{1.891149in}}{\pgfqpoint{3.981678in}{1.892785in}}{\pgfqpoint{3.977560in}{1.892785in}}%
\pgfpathcurveto{\pgfqpoint{3.973442in}{1.892785in}}{\pgfqpoint{3.969492in}{1.891149in}}{\pgfqpoint{3.966580in}{1.888237in}}%
\pgfpathcurveto{\pgfqpoint{3.963668in}{1.885325in}}{\pgfqpoint{3.962032in}{1.881375in}}{\pgfqpoint{3.962032in}{1.877257in}}%
\pgfpathcurveto{\pgfqpoint{3.962032in}{1.873139in}}{\pgfqpoint{3.963668in}{1.869189in}}{\pgfqpoint{3.966580in}{1.866277in}}%
\pgfpathcurveto{\pgfqpoint{3.969492in}{1.863365in}}{\pgfqpoint{3.973442in}{1.861729in}}{\pgfqpoint{3.977560in}{1.861729in}}%
\pgfpathclose%
\pgfusepath{fill}%
\end{pgfscope}%
\begin{pgfscope}%
\pgfpathrectangle{\pgfqpoint{0.887500in}{0.275000in}}{\pgfqpoint{4.225000in}{4.225000in}}%
\pgfusepath{clip}%
\pgfsetbuttcap%
\pgfsetroundjoin%
\definecolor{currentfill}{rgb}{0.000000,0.000000,0.000000}%
\pgfsetfillcolor{currentfill}%
\pgfsetfillopacity{0.800000}%
\pgfsetlinewidth{0.000000pt}%
\definecolor{currentstroke}{rgb}{0.000000,0.000000,0.000000}%
\pgfsetstrokecolor{currentstroke}%
\pgfsetstrokeopacity{0.800000}%
\pgfsetdash{}{0pt}%
\pgfpathmoveto{\pgfqpoint{4.338816in}{1.672834in}}%
\pgfpathcurveto{\pgfqpoint{4.342934in}{1.672834in}}{\pgfqpoint{4.346884in}{1.674470in}}{\pgfqpoint{4.349796in}{1.677382in}}%
\pgfpathcurveto{\pgfqpoint{4.352708in}{1.680294in}}{\pgfqpoint{4.354344in}{1.684244in}}{\pgfqpoint{4.354344in}{1.688362in}}%
\pgfpathcurveto{\pgfqpoint{4.354344in}{1.692480in}}{\pgfqpoint{4.352708in}{1.696430in}}{\pgfqpoint{4.349796in}{1.699342in}}%
\pgfpathcurveto{\pgfqpoint{4.346884in}{1.702254in}}{\pgfqpoint{4.342934in}{1.703890in}}{\pgfqpoint{4.338816in}{1.703890in}}%
\pgfpathcurveto{\pgfqpoint{4.334698in}{1.703890in}}{\pgfqpoint{4.330747in}{1.702254in}}{\pgfqpoint{4.327836in}{1.699342in}}%
\pgfpathcurveto{\pgfqpoint{4.324924in}{1.696430in}}{\pgfqpoint{4.323287in}{1.692480in}}{\pgfqpoint{4.323287in}{1.688362in}}%
\pgfpathcurveto{\pgfqpoint{4.323287in}{1.684244in}}{\pgfqpoint{4.324924in}{1.680294in}}{\pgfqpoint{4.327836in}{1.677382in}}%
\pgfpathcurveto{\pgfqpoint{4.330747in}{1.674470in}}{\pgfqpoint{4.334698in}{1.672834in}}{\pgfqpoint{4.338816in}{1.672834in}}%
\pgfpathclose%
\pgfusepath{fill}%
\end{pgfscope}%
\begin{pgfscope}%
\pgfpathrectangle{\pgfqpoint{0.887500in}{0.275000in}}{\pgfqpoint{4.225000in}{4.225000in}}%
\pgfusepath{clip}%
\pgfsetbuttcap%
\pgfsetroundjoin%
\definecolor{currentfill}{rgb}{0.000000,0.000000,0.000000}%
\pgfsetfillcolor{currentfill}%
\pgfsetfillopacity{0.800000}%
\pgfsetlinewidth{0.000000pt}%
\definecolor{currentstroke}{rgb}{0.000000,0.000000,0.000000}%
\pgfsetstrokecolor{currentstroke}%
\pgfsetstrokeopacity{0.800000}%
\pgfsetdash{}{0pt}%
\pgfpathmoveto{\pgfqpoint{3.748806in}{1.607268in}}%
\pgfpathcurveto{\pgfqpoint{3.752925in}{1.607268in}}{\pgfqpoint{3.756875in}{1.608904in}}{\pgfqpoint{3.759787in}{1.611816in}}%
\pgfpathcurveto{\pgfqpoint{3.762698in}{1.614728in}}{\pgfqpoint{3.764335in}{1.618678in}}{\pgfqpoint{3.764335in}{1.622797in}}%
\pgfpathcurveto{\pgfqpoint{3.764335in}{1.626915in}}{\pgfqpoint{3.762698in}{1.630865in}}{\pgfqpoint{3.759787in}{1.633777in}}%
\pgfpathcurveto{\pgfqpoint{3.756875in}{1.636689in}}{\pgfqpoint{3.752925in}{1.638325in}}{\pgfqpoint{3.748806in}{1.638325in}}%
\pgfpathcurveto{\pgfqpoint{3.744688in}{1.638325in}}{\pgfqpoint{3.740738in}{1.636689in}}{\pgfqpoint{3.737826in}{1.633777in}}%
\pgfpathcurveto{\pgfqpoint{3.734914in}{1.630865in}}{\pgfqpoint{3.733278in}{1.626915in}}{\pgfqpoint{3.733278in}{1.622797in}}%
\pgfpathcurveto{\pgfqpoint{3.733278in}{1.618678in}}{\pgfqpoint{3.734914in}{1.614728in}}{\pgfqpoint{3.737826in}{1.611816in}}%
\pgfpathcurveto{\pgfqpoint{3.740738in}{1.608904in}}{\pgfqpoint{3.744688in}{1.607268in}}{\pgfqpoint{3.748806in}{1.607268in}}%
\pgfpathclose%
\pgfusepath{fill}%
\end{pgfscope}%
\begin{pgfscope}%
\pgfpathrectangle{\pgfqpoint{0.887500in}{0.275000in}}{\pgfqpoint{4.225000in}{4.225000in}}%
\pgfusepath{clip}%
\pgfsetbuttcap%
\pgfsetroundjoin%
\definecolor{currentfill}{rgb}{0.000000,0.000000,0.000000}%
\pgfsetfillcolor{currentfill}%
\pgfsetfillopacity{0.800000}%
\pgfsetlinewidth{0.000000pt}%
\definecolor{currentstroke}{rgb}{0.000000,0.000000,0.000000}%
\pgfsetstrokecolor{currentstroke}%
\pgfsetstrokeopacity{0.800000}%
\pgfsetdash{}{0pt}%
\pgfpathmoveto{\pgfqpoint{3.092520in}{1.766508in}}%
\pgfpathcurveto{\pgfqpoint{3.096638in}{1.766508in}}{\pgfqpoint{3.100588in}{1.768144in}}{\pgfqpoint{3.103500in}{1.771056in}}%
\pgfpathcurveto{\pgfqpoint{3.106412in}{1.773968in}}{\pgfqpoint{3.108049in}{1.777918in}}{\pgfqpoint{3.108049in}{1.782037in}}%
\pgfpathcurveto{\pgfqpoint{3.108049in}{1.786155in}}{\pgfqpoint{3.106412in}{1.790105in}}{\pgfqpoint{3.103500in}{1.793017in}}%
\pgfpathcurveto{\pgfqpoint{3.100588in}{1.795929in}}{\pgfqpoint{3.096638in}{1.797565in}}{\pgfqpoint{3.092520in}{1.797565in}}%
\pgfpathcurveto{\pgfqpoint{3.088402in}{1.797565in}}{\pgfqpoint{3.084452in}{1.795929in}}{\pgfqpoint{3.081540in}{1.793017in}}%
\pgfpathcurveto{\pgfqpoint{3.078628in}{1.790105in}}{\pgfqpoint{3.076992in}{1.786155in}}{\pgfqpoint{3.076992in}{1.782037in}}%
\pgfpathcurveto{\pgfqpoint{3.076992in}{1.777918in}}{\pgfqpoint{3.078628in}{1.773968in}}{\pgfqpoint{3.081540in}{1.771056in}}%
\pgfpathcurveto{\pgfqpoint{3.084452in}{1.768144in}}{\pgfqpoint{3.088402in}{1.766508in}}{\pgfqpoint{3.092520in}{1.766508in}}%
\pgfpathclose%
\pgfusepath{fill}%
\end{pgfscope}%
\begin{pgfscope}%
\pgfpathrectangle{\pgfqpoint{0.887500in}{0.275000in}}{\pgfqpoint{4.225000in}{4.225000in}}%
\pgfusepath{clip}%
\pgfsetbuttcap%
\pgfsetroundjoin%
\definecolor{currentfill}{rgb}{0.000000,0.000000,0.000000}%
\pgfsetfillcolor{currentfill}%
\pgfsetfillopacity{0.800000}%
\pgfsetlinewidth{0.000000pt}%
\definecolor{currentstroke}{rgb}{0.000000,0.000000,0.000000}%
\pgfsetstrokecolor{currentstroke}%
\pgfsetstrokeopacity{0.800000}%
\pgfsetdash{}{0pt}%
\pgfpathmoveto{\pgfqpoint{4.158582in}{1.781597in}}%
\pgfpathcurveto{\pgfqpoint{4.162700in}{1.781597in}}{\pgfqpoint{4.166650in}{1.783234in}}{\pgfqpoint{4.169562in}{1.786146in}}%
\pgfpathcurveto{\pgfqpoint{4.172474in}{1.789058in}}{\pgfqpoint{4.174110in}{1.793008in}}{\pgfqpoint{4.174110in}{1.797126in}}%
\pgfpathcurveto{\pgfqpoint{4.174110in}{1.801244in}}{\pgfqpoint{4.172474in}{1.805194in}}{\pgfqpoint{4.169562in}{1.808106in}}%
\pgfpathcurveto{\pgfqpoint{4.166650in}{1.811018in}}{\pgfqpoint{4.162700in}{1.812654in}}{\pgfqpoint{4.158582in}{1.812654in}}%
\pgfpathcurveto{\pgfqpoint{4.154464in}{1.812654in}}{\pgfqpoint{4.150514in}{1.811018in}}{\pgfqpoint{4.147602in}{1.808106in}}%
\pgfpathcurveto{\pgfqpoint{4.144690in}{1.805194in}}{\pgfqpoint{4.143054in}{1.801244in}}{\pgfqpoint{4.143054in}{1.797126in}}%
\pgfpathcurveto{\pgfqpoint{4.143054in}{1.793008in}}{\pgfqpoint{4.144690in}{1.789058in}}{\pgfqpoint{4.147602in}{1.786146in}}%
\pgfpathcurveto{\pgfqpoint{4.150514in}{1.783234in}}{\pgfqpoint{4.154464in}{1.781597in}}{\pgfqpoint{4.158582in}{1.781597in}}%
\pgfpathclose%
\pgfusepath{fill}%
\end{pgfscope}%
\begin{pgfscope}%
\pgfpathrectangle{\pgfqpoint{0.887500in}{0.275000in}}{\pgfqpoint{4.225000in}{4.225000in}}%
\pgfusepath{clip}%
\pgfsetbuttcap%
\pgfsetroundjoin%
\definecolor{currentfill}{rgb}{0.000000,0.000000,0.000000}%
\pgfsetfillcolor{currentfill}%
\pgfsetfillopacity{0.800000}%
\pgfsetlinewidth{0.000000pt}%
\definecolor{currentstroke}{rgb}{0.000000,0.000000,0.000000}%
\pgfsetstrokecolor{currentstroke}%
\pgfsetstrokeopacity{0.800000}%
\pgfsetdash{}{0pt}%
\pgfpathmoveto{\pgfqpoint{3.863494in}{1.748814in}}%
\pgfpathcurveto{\pgfqpoint{3.867613in}{1.748814in}}{\pgfqpoint{3.871563in}{1.750450in}}{\pgfqpoint{3.874475in}{1.753362in}}%
\pgfpathcurveto{\pgfqpoint{3.877387in}{1.756274in}}{\pgfqpoint{3.879023in}{1.760224in}}{\pgfqpoint{3.879023in}{1.764343in}}%
\pgfpathcurveto{\pgfqpoint{3.879023in}{1.768461in}}{\pgfqpoint{3.877387in}{1.772411in}}{\pgfqpoint{3.874475in}{1.775323in}}%
\pgfpathcurveto{\pgfqpoint{3.871563in}{1.778235in}}{\pgfqpoint{3.867613in}{1.779871in}}{\pgfqpoint{3.863494in}{1.779871in}}%
\pgfpathcurveto{\pgfqpoint{3.859376in}{1.779871in}}{\pgfqpoint{3.855426in}{1.778235in}}{\pgfqpoint{3.852514in}{1.775323in}}%
\pgfpathcurveto{\pgfqpoint{3.849602in}{1.772411in}}{\pgfqpoint{3.847966in}{1.768461in}}{\pgfqpoint{3.847966in}{1.764343in}}%
\pgfpathcurveto{\pgfqpoint{3.847966in}{1.760224in}}{\pgfqpoint{3.849602in}{1.756274in}}{\pgfqpoint{3.852514in}{1.753362in}}%
\pgfpathcurveto{\pgfqpoint{3.855426in}{1.750450in}}{\pgfqpoint{3.859376in}{1.748814in}}{\pgfqpoint{3.863494in}{1.748814in}}%
\pgfpathclose%
\pgfusepath{fill}%
\end{pgfscope}%
\begin{pgfscope}%
\pgfpathrectangle{\pgfqpoint{0.887500in}{0.275000in}}{\pgfqpoint{4.225000in}{4.225000in}}%
\pgfusepath{clip}%
\pgfsetbuttcap%
\pgfsetroundjoin%
\definecolor{currentfill}{rgb}{0.000000,0.000000,0.000000}%
\pgfsetfillcolor{currentfill}%
\pgfsetfillopacity{0.800000}%
\pgfsetlinewidth{0.000000pt}%
\definecolor{currentstroke}{rgb}{0.000000,0.000000,0.000000}%
\pgfsetstrokecolor{currentstroke}%
\pgfsetstrokeopacity{0.800000}%
\pgfsetdash{}{0pt}%
\pgfpathmoveto{\pgfqpoint{1.779558in}{2.093569in}}%
\pgfpathcurveto{\pgfqpoint{1.783676in}{2.093569in}}{\pgfqpoint{1.787626in}{2.095205in}}{\pgfqpoint{1.790538in}{2.098117in}}%
\pgfpathcurveto{\pgfqpoint{1.793450in}{2.101029in}}{\pgfqpoint{1.795086in}{2.104979in}}{\pgfqpoint{1.795086in}{2.109097in}}%
\pgfpathcurveto{\pgfqpoint{1.795086in}{2.113216in}}{\pgfqpoint{1.793450in}{2.117166in}}{\pgfqpoint{1.790538in}{2.120078in}}%
\pgfpathcurveto{\pgfqpoint{1.787626in}{2.122990in}}{\pgfqpoint{1.783676in}{2.124626in}}{\pgfqpoint{1.779558in}{2.124626in}}%
\pgfpathcurveto{\pgfqpoint{1.775440in}{2.124626in}}{\pgfqpoint{1.771490in}{2.122990in}}{\pgfqpoint{1.768578in}{2.120078in}}%
\pgfpathcurveto{\pgfqpoint{1.765666in}{2.117166in}}{\pgfqpoint{1.764030in}{2.113216in}}{\pgfqpoint{1.764030in}{2.109097in}}%
\pgfpathcurveto{\pgfqpoint{1.764030in}{2.104979in}}{\pgfqpoint{1.765666in}{2.101029in}}{\pgfqpoint{1.768578in}{2.098117in}}%
\pgfpathcurveto{\pgfqpoint{1.771490in}{2.095205in}}{\pgfqpoint{1.775440in}{2.093569in}}{\pgfqpoint{1.779558in}{2.093569in}}%
\pgfpathclose%
\pgfusepath{fill}%
\end{pgfscope}%
\begin{pgfscope}%
\pgfpathrectangle{\pgfqpoint{0.887500in}{0.275000in}}{\pgfqpoint{4.225000in}{4.225000in}}%
\pgfusepath{clip}%
\pgfsetbuttcap%
\pgfsetroundjoin%
\definecolor{currentfill}{rgb}{0.000000,0.000000,0.000000}%
\pgfsetfillcolor{currentfill}%
\pgfsetfillopacity{0.800000}%
\pgfsetlinewidth{0.000000pt}%
\definecolor{currentstroke}{rgb}{0.000000,0.000000,0.000000}%
\pgfsetstrokecolor{currentstroke}%
\pgfsetstrokeopacity{0.800000}%
\pgfsetdash{}{0pt}%
\pgfpathmoveto{\pgfqpoint{3.273213in}{1.685971in}}%
\pgfpathcurveto{\pgfqpoint{3.277332in}{1.685971in}}{\pgfqpoint{3.281282in}{1.687608in}}{\pgfqpoint{3.284194in}{1.690520in}}%
\pgfpathcurveto{\pgfqpoint{3.287106in}{1.693432in}}{\pgfqpoint{3.288742in}{1.697382in}}{\pgfqpoint{3.288742in}{1.701500in}}%
\pgfpathcurveto{\pgfqpoint{3.288742in}{1.705618in}}{\pgfqpoint{3.287106in}{1.709568in}}{\pgfqpoint{3.284194in}{1.712480in}}%
\pgfpathcurveto{\pgfqpoint{3.281282in}{1.715392in}}{\pgfqpoint{3.277332in}{1.717028in}}{\pgfqpoint{3.273213in}{1.717028in}}%
\pgfpathcurveto{\pgfqpoint{3.269095in}{1.717028in}}{\pgfqpoint{3.265145in}{1.715392in}}{\pgfqpoint{3.262233in}{1.712480in}}%
\pgfpathcurveto{\pgfqpoint{3.259321in}{1.709568in}}{\pgfqpoint{3.257685in}{1.705618in}}{\pgfqpoint{3.257685in}{1.701500in}}%
\pgfpathcurveto{\pgfqpoint{3.257685in}{1.697382in}}{\pgfqpoint{3.259321in}{1.693432in}}{\pgfqpoint{3.262233in}{1.690520in}}%
\pgfpathcurveto{\pgfqpoint{3.265145in}{1.687608in}}{\pgfqpoint{3.269095in}{1.685971in}}{\pgfqpoint{3.273213in}{1.685971in}}%
\pgfpathclose%
\pgfusepath{fill}%
\end{pgfscope}%
\begin{pgfscope}%
\pgfpathrectangle{\pgfqpoint{0.887500in}{0.275000in}}{\pgfqpoint{4.225000in}{4.225000in}}%
\pgfusepath{clip}%
\pgfsetbuttcap%
\pgfsetroundjoin%
\definecolor{currentfill}{rgb}{0.000000,0.000000,0.000000}%
\pgfsetfillcolor{currentfill}%
\pgfsetfillopacity{0.800000}%
\pgfsetlinewidth{0.000000pt}%
\definecolor{currentstroke}{rgb}{0.000000,0.000000,0.000000}%
\pgfsetstrokecolor{currentstroke}%
\pgfsetstrokeopacity{0.800000}%
\pgfsetdash{}{0pt}%
\pgfpathmoveto{\pgfqpoint{2.435912in}{1.943660in}}%
\pgfpathcurveto{\pgfqpoint{2.440030in}{1.943660in}}{\pgfqpoint{2.443981in}{1.945296in}}{\pgfqpoint{2.446892in}{1.948208in}}%
\pgfpathcurveto{\pgfqpoint{2.449804in}{1.951120in}}{\pgfqpoint{2.451441in}{1.955070in}}{\pgfqpoint{2.451441in}{1.959188in}}%
\pgfpathcurveto{\pgfqpoint{2.451441in}{1.963306in}}{\pgfqpoint{2.449804in}{1.967256in}}{\pgfqpoint{2.446892in}{1.970168in}}%
\pgfpathcurveto{\pgfqpoint{2.443981in}{1.973080in}}{\pgfqpoint{2.440030in}{1.974717in}}{\pgfqpoint{2.435912in}{1.974717in}}%
\pgfpathcurveto{\pgfqpoint{2.431794in}{1.974717in}}{\pgfqpoint{2.427844in}{1.973080in}}{\pgfqpoint{2.424932in}{1.970168in}}%
\pgfpathcurveto{\pgfqpoint{2.422020in}{1.967256in}}{\pgfqpoint{2.420384in}{1.963306in}}{\pgfqpoint{2.420384in}{1.959188in}}%
\pgfpathcurveto{\pgfqpoint{2.420384in}{1.955070in}}{\pgfqpoint{2.422020in}{1.951120in}}{\pgfqpoint{2.424932in}{1.948208in}}%
\pgfpathcurveto{\pgfqpoint{2.427844in}{1.945296in}}{\pgfqpoint{2.431794in}{1.943660in}}{\pgfqpoint{2.435912in}{1.943660in}}%
\pgfpathclose%
\pgfusepath{fill}%
\end{pgfscope}%
\begin{pgfscope}%
\pgfpathrectangle{\pgfqpoint{0.887500in}{0.275000in}}{\pgfqpoint{4.225000in}{4.225000in}}%
\pgfusepath{clip}%
\pgfsetbuttcap%
\pgfsetroundjoin%
\definecolor{currentfill}{rgb}{0.000000,0.000000,0.000000}%
\pgfsetfillcolor{currentfill}%
\pgfsetfillopacity{0.800000}%
\pgfsetlinewidth{0.000000pt}%
\definecolor{currentstroke}{rgb}{0.000000,0.000000,0.000000}%
\pgfsetstrokecolor{currentstroke}%
\pgfsetstrokeopacity{0.800000}%
\pgfsetdash{}{0pt}%
\pgfpathmoveto{\pgfqpoint{3.815402in}{1.403316in}}%
\pgfpathcurveto{\pgfqpoint{3.819521in}{1.403316in}}{\pgfqpoint{3.823471in}{1.404952in}}{\pgfqpoint{3.826383in}{1.407864in}}%
\pgfpathcurveto{\pgfqpoint{3.829295in}{1.410776in}}{\pgfqpoint{3.830931in}{1.414726in}}{\pgfqpoint{3.830931in}{1.418844in}}%
\pgfpathcurveto{\pgfqpoint{3.830931in}{1.422962in}}{\pgfqpoint{3.829295in}{1.426912in}}{\pgfqpoint{3.826383in}{1.429824in}}%
\pgfpathcurveto{\pgfqpoint{3.823471in}{1.432736in}}{\pgfqpoint{3.819521in}{1.434372in}}{\pgfqpoint{3.815402in}{1.434372in}}%
\pgfpathcurveto{\pgfqpoint{3.811284in}{1.434372in}}{\pgfqpoint{3.807334in}{1.432736in}}{\pgfqpoint{3.804422in}{1.429824in}}%
\pgfpathcurveto{\pgfqpoint{3.801510in}{1.426912in}}{\pgfqpoint{3.799874in}{1.422962in}}{\pgfqpoint{3.799874in}{1.418844in}}%
\pgfpathcurveto{\pgfqpoint{3.799874in}{1.414726in}}{\pgfqpoint{3.801510in}{1.410776in}}{\pgfqpoint{3.804422in}{1.407864in}}%
\pgfpathcurveto{\pgfqpoint{3.807334in}{1.404952in}}{\pgfqpoint{3.811284in}{1.403316in}}{\pgfqpoint{3.815402in}{1.403316in}}%
\pgfpathclose%
\pgfusepath{fill}%
\end{pgfscope}%
\begin{pgfscope}%
\pgfpathrectangle{\pgfqpoint{0.887500in}{0.275000in}}{\pgfqpoint{4.225000in}{4.225000in}}%
\pgfusepath{clip}%
\pgfsetbuttcap%
\pgfsetroundjoin%
\definecolor{currentfill}{rgb}{0.000000,0.000000,0.000000}%
\pgfsetfillcolor{currentfill}%
\pgfsetfillopacity{0.800000}%
\pgfsetlinewidth{0.000000pt}%
\definecolor{currentstroke}{rgb}{0.000000,0.000000,0.000000}%
\pgfsetstrokecolor{currentstroke}%
\pgfsetstrokeopacity{0.800000}%
\pgfsetdash{}{0pt}%
\pgfpathmoveto{\pgfqpoint{3.453987in}{1.598451in}}%
\pgfpathcurveto{\pgfqpoint{3.458105in}{1.598451in}}{\pgfqpoint{3.462055in}{1.600088in}}{\pgfqpoint{3.464967in}{1.603000in}}%
\pgfpathcurveto{\pgfqpoint{3.467879in}{1.605912in}}{\pgfqpoint{3.469515in}{1.609862in}}{\pgfqpoint{3.469515in}{1.613980in}}%
\pgfpathcurveto{\pgfqpoint{3.469515in}{1.618098in}}{\pgfqpoint{3.467879in}{1.622048in}}{\pgfqpoint{3.464967in}{1.624960in}}%
\pgfpathcurveto{\pgfqpoint{3.462055in}{1.627872in}}{\pgfqpoint{3.458105in}{1.629508in}}{\pgfqpoint{3.453987in}{1.629508in}}%
\pgfpathcurveto{\pgfqpoint{3.449869in}{1.629508in}}{\pgfqpoint{3.445919in}{1.627872in}}{\pgfqpoint{3.443007in}{1.624960in}}%
\pgfpathcurveto{\pgfqpoint{3.440095in}{1.622048in}}{\pgfqpoint{3.438459in}{1.618098in}}{\pgfqpoint{3.438459in}{1.613980in}}%
\pgfpathcurveto{\pgfqpoint{3.438459in}{1.609862in}}{\pgfqpoint{3.440095in}{1.605912in}}{\pgfqpoint{3.443007in}{1.603000in}}%
\pgfpathcurveto{\pgfqpoint{3.445919in}{1.600088in}}{\pgfqpoint{3.449869in}{1.598451in}}{\pgfqpoint{3.453987in}{1.598451in}}%
\pgfpathclose%
\pgfusepath{fill}%
\end{pgfscope}%
\begin{pgfscope}%
\pgfpathrectangle{\pgfqpoint{0.887500in}{0.275000in}}{\pgfqpoint{4.225000in}{4.225000in}}%
\pgfusepath{clip}%
\pgfsetbuttcap%
\pgfsetroundjoin%
\definecolor{currentfill}{rgb}{0.000000,0.000000,0.000000}%
\pgfsetfillcolor{currentfill}%
\pgfsetfillopacity{0.800000}%
\pgfsetlinewidth{0.000000pt}%
\definecolor{currentstroke}{rgb}{0.000000,0.000000,0.000000}%
\pgfsetstrokecolor{currentstroke}%
\pgfsetstrokeopacity{0.800000}%
\pgfsetdash{}{0pt}%
\pgfpathmoveto{\pgfqpoint{4.044973in}{1.684936in}}%
\pgfpathcurveto{\pgfqpoint{4.049092in}{1.684936in}}{\pgfqpoint{4.053042in}{1.686572in}}{\pgfqpoint{4.055954in}{1.689484in}}%
\pgfpathcurveto{\pgfqpoint{4.058866in}{1.692396in}}{\pgfqpoint{4.060502in}{1.696346in}}{\pgfqpoint{4.060502in}{1.700464in}}%
\pgfpathcurveto{\pgfqpoint{4.060502in}{1.704582in}}{\pgfqpoint{4.058866in}{1.708532in}}{\pgfqpoint{4.055954in}{1.711444in}}%
\pgfpathcurveto{\pgfqpoint{4.053042in}{1.714356in}}{\pgfqpoint{4.049092in}{1.715992in}}{\pgfqpoint{4.044973in}{1.715992in}}%
\pgfpathcurveto{\pgfqpoint{4.040855in}{1.715992in}}{\pgfqpoint{4.036905in}{1.714356in}}{\pgfqpoint{4.033993in}{1.711444in}}%
\pgfpathcurveto{\pgfqpoint{4.031081in}{1.708532in}}{\pgfqpoint{4.029445in}{1.704582in}}{\pgfqpoint{4.029445in}{1.700464in}}%
\pgfpathcurveto{\pgfqpoint{4.029445in}{1.696346in}}{\pgfqpoint{4.031081in}{1.692396in}}{\pgfqpoint{4.033993in}{1.689484in}}%
\pgfpathcurveto{\pgfqpoint{4.036905in}{1.686572in}}{\pgfqpoint{4.040855in}{1.684936in}}{\pgfqpoint{4.044973in}{1.684936in}}%
\pgfpathclose%
\pgfusepath{fill}%
\end{pgfscope}%
\begin{pgfscope}%
\pgfpathrectangle{\pgfqpoint{0.887500in}{0.275000in}}{\pgfqpoint{4.225000in}{4.225000in}}%
\pgfusepath{clip}%
\pgfsetbuttcap%
\pgfsetroundjoin%
\definecolor{currentfill}{rgb}{0.000000,0.000000,0.000000}%
\pgfsetfillcolor{currentfill}%
\pgfsetfillopacity{0.800000}%
\pgfsetlinewidth{0.000000pt}%
\definecolor{currentstroke}{rgb}{0.000000,0.000000,0.000000}%
\pgfsetstrokecolor{currentstroke}%
\pgfsetstrokeopacity{0.800000}%
\pgfsetdash{}{0pt}%
\pgfpathmoveto{\pgfqpoint{1.959194in}{2.033002in}}%
\pgfpathcurveto{\pgfqpoint{1.963312in}{2.033002in}}{\pgfqpoint{1.967262in}{2.034638in}}{\pgfqpoint{1.970174in}{2.037550in}}%
\pgfpathcurveto{\pgfqpoint{1.973086in}{2.040462in}}{\pgfqpoint{1.974722in}{2.044412in}}{\pgfqpoint{1.974722in}{2.048530in}}%
\pgfpathcurveto{\pgfqpoint{1.974722in}{2.052648in}}{\pgfqpoint{1.973086in}{2.056598in}}{\pgfqpoint{1.970174in}{2.059510in}}%
\pgfpathcurveto{\pgfqpoint{1.967262in}{2.062422in}}{\pgfqpoint{1.963312in}{2.064059in}}{\pgfqpoint{1.959194in}{2.064059in}}%
\pgfpathcurveto{\pgfqpoint{1.955076in}{2.064059in}}{\pgfqpoint{1.951126in}{2.062422in}}{\pgfqpoint{1.948214in}{2.059510in}}%
\pgfpathcurveto{\pgfqpoint{1.945302in}{2.056598in}}{\pgfqpoint{1.943666in}{2.052648in}}{\pgfqpoint{1.943666in}{2.048530in}}%
\pgfpathcurveto{\pgfqpoint{1.943666in}{2.044412in}}{\pgfqpoint{1.945302in}{2.040462in}}{\pgfqpoint{1.948214in}{2.037550in}}%
\pgfpathcurveto{\pgfqpoint{1.951126in}{2.034638in}}{\pgfqpoint{1.955076in}{2.033002in}}{\pgfqpoint{1.959194in}{2.033002in}}%
\pgfpathclose%
\pgfusepath{fill}%
\end{pgfscope}%
\begin{pgfscope}%
\pgfpathrectangle{\pgfqpoint{0.887500in}{0.275000in}}{\pgfqpoint{4.225000in}{4.225000in}}%
\pgfusepath{clip}%
\pgfsetbuttcap%
\pgfsetroundjoin%
\definecolor{currentfill}{rgb}{0.000000,0.000000,0.000000}%
\pgfsetfillcolor{currentfill}%
\pgfsetfillopacity{0.800000}%
\pgfsetlinewidth{0.000000pt}%
\definecolor{currentstroke}{rgb}{0.000000,0.000000,0.000000}%
\pgfsetstrokecolor{currentstroke}%
\pgfsetstrokeopacity{0.800000}%
\pgfsetdash{}{0pt}%
\pgfpathmoveto{\pgfqpoint{2.616274in}{1.875972in}}%
\pgfpathcurveto{\pgfqpoint{2.620392in}{1.875972in}}{\pgfqpoint{2.624342in}{1.877608in}}{\pgfqpoint{2.627254in}{1.880520in}}%
\pgfpathcurveto{\pgfqpoint{2.630166in}{1.883432in}}{\pgfqpoint{2.631802in}{1.887382in}}{\pgfqpoint{2.631802in}{1.891500in}}%
\pgfpathcurveto{\pgfqpoint{2.631802in}{1.895618in}}{\pgfqpoint{2.630166in}{1.899568in}}{\pgfqpoint{2.627254in}{1.902480in}}%
\pgfpathcurveto{\pgfqpoint{2.624342in}{1.905392in}}{\pgfqpoint{2.620392in}{1.907028in}}{\pgfqpoint{2.616274in}{1.907028in}}%
\pgfpathcurveto{\pgfqpoint{2.612155in}{1.907028in}}{\pgfqpoint{2.608205in}{1.905392in}}{\pgfqpoint{2.605293in}{1.902480in}}%
\pgfpathcurveto{\pgfqpoint{2.602381in}{1.899568in}}{\pgfqpoint{2.600745in}{1.895618in}}{\pgfqpoint{2.600745in}{1.891500in}}%
\pgfpathcurveto{\pgfqpoint{2.600745in}{1.887382in}}{\pgfqpoint{2.602381in}{1.883432in}}{\pgfqpoint{2.605293in}{1.880520in}}%
\pgfpathcurveto{\pgfqpoint{2.608205in}{1.877608in}}{\pgfqpoint{2.612155in}{1.875972in}}{\pgfqpoint{2.616274in}{1.875972in}}%
\pgfpathclose%
\pgfusepath{fill}%
\end{pgfscope}%
\begin{pgfscope}%
\pgfpathrectangle{\pgfqpoint{0.887500in}{0.275000in}}{\pgfqpoint{4.225000in}{4.225000in}}%
\pgfusepath{clip}%
\pgfsetbuttcap%
\pgfsetroundjoin%
\definecolor{currentfill}{rgb}{0.000000,0.000000,0.000000}%
\pgfsetfillcolor{currentfill}%
\pgfsetfillopacity{0.800000}%
\pgfsetlinewidth{0.000000pt}%
\definecolor{currentstroke}{rgb}{0.000000,0.000000,0.000000}%
\pgfsetstrokecolor{currentstroke}%
\pgfsetstrokeopacity{0.800000}%
\pgfsetdash{}{0pt}%
\pgfpathmoveto{\pgfqpoint{3.930707in}{1.568196in}}%
\pgfpathcurveto{\pgfqpoint{3.934825in}{1.568196in}}{\pgfqpoint{3.938775in}{1.569832in}}{\pgfqpoint{3.941687in}{1.572744in}}%
\pgfpathcurveto{\pgfqpoint{3.944599in}{1.575656in}}{\pgfqpoint{3.946235in}{1.579606in}}{\pgfqpoint{3.946235in}{1.583724in}}%
\pgfpathcurveto{\pgfqpoint{3.946235in}{1.587842in}}{\pgfqpoint{3.944599in}{1.591792in}}{\pgfqpoint{3.941687in}{1.594704in}}%
\pgfpathcurveto{\pgfqpoint{3.938775in}{1.597616in}}{\pgfqpoint{3.934825in}{1.599252in}}{\pgfqpoint{3.930707in}{1.599252in}}%
\pgfpathcurveto{\pgfqpoint{3.926589in}{1.599252in}}{\pgfqpoint{3.922639in}{1.597616in}}{\pgfqpoint{3.919727in}{1.594704in}}%
\pgfpathcurveto{\pgfqpoint{3.916815in}{1.591792in}}{\pgfqpoint{3.915179in}{1.587842in}}{\pgfqpoint{3.915179in}{1.583724in}}%
\pgfpathcurveto{\pgfqpoint{3.915179in}{1.579606in}}{\pgfqpoint{3.916815in}{1.575656in}}{\pgfqpoint{3.919727in}{1.572744in}}%
\pgfpathcurveto{\pgfqpoint{3.922639in}{1.569832in}}{\pgfqpoint{3.926589in}{1.568196in}}{\pgfqpoint{3.930707in}{1.568196in}}%
\pgfpathclose%
\pgfusepath{fill}%
\end{pgfscope}%
\begin{pgfscope}%
\pgfpathrectangle{\pgfqpoint{0.887500in}{0.275000in}}{\pgfqpoint{4.225000in}{4.225000in}}%
\pgfusepath{clip}%
\pgfsetbuttcap%
\pgfsetroundjoin%
\definecolor{currentfill}{rgb}{0.000000,0.000000,0.000000}%
\pgfsetfillcolor{currentfill}%
\pgfsetfillopacity{0.800000}%
\pgfsetlinewidth{0.000000pt}%
\definecolor{currentstroke}{rgb}{0.000000,0.000000,0.000000}%
\pgfsetstrokecolor{currentstroke}%
\pgfsetstrokeopacity{0.800000}%
\pgfsetdash{}{0pt}%
\pgfpathmoveto{\pgfqpoint{2.796930in}{1.804401in}}%
\pgfpathcurveto{\pgfqpoint{2.801048in}{1.804401in}}{\pgfqpoint{2.804998in}{1.806037in}}{\pgfqpoint{2.807910in}{1.808949in}}%
\pgfpathcurveto{\pgfqpoint{2.810822in}{1.811861in}}{\pgfqpoint{2.812458in}{1.815811in}}{\pgfqpoint{2.812458in}{1.819929in}}%
\pgfpathcurveto{\pgfqpoint{2.812458in}{1.824048in}}{\pgfqpoint{2.810822in}{1.827998in}}{\pgfqpoint{2.807910in}{1.830910in}}%
\pgfpathcurveto{\pgfqpoint{2.804998in}{1.833822in}}{\pgfqpoint{2.801048in}{1.835458in}}{\pgfqpoint{2.796930in}{1.835458in}}%
\pgfpathcurveto{\pgfqpoint{2.792812in}{1.835458in}}{\pgfqpoint{2.788862in}{1.833822in}}{\pgfqpoint{2.785950in}{1.830910in}}%
\pgfpathcurveto{\pgfqpoint{2.783038in}{1.827998in}}{\pgfqpoint{2.781402in}{1.824048in}}{\pgfqpoint{2.781402in}{1.819929in}}%
\pgfpathcurveto{\pgfqpoint{2.781402in}{1.815811in}}{\pgfqpoint{2.783038in}{1.811861in}}{\pgfqpoint{2.785950in}{1.808949in}}%
\pgfpathcurveto{\pgfqpoint{2.788862in}{1.806037in}}{\pgfqpoint{2.792812in}{1.804401in}}{\pgfqpoint{2.796930in}{1.804401in}}%
\pgfpathclose%
\pgfusepath{fill}%
\end{pgfscope}%
\begin{pgfscope}%
\pgfpathrectangle{\pgfqpoint{0.887500in}{0.275000in}}{\pgfqpoint{4.225000in}{4.225000in}}%
\pgfusepath{clip}%
\pgfsetbuttcap%
\pgfsetroundjoin%
\definecolor{currentfill}{rgb}{0.000000,0.000000,0.000000}%
\pgfsetfillcolor{currentfill}%
\pgfsetfillopacity{0.800000}%
\pgfsetlinewidth{0.000000pt}%
\definecolor{currentstroke}{rgb}{0.000000,0.000000,0.000000}%
\pgfsetstrokecolor{currentstroke}%
\pgfsetstrokeopacity{0.800000}%
\pgfsetdash{}{0pt}%
\pgfpathmoveto{\pgfqpoint{3.997469in}{1.361789in}}%
\pgfpathcurveto{\pgfqpoint{4.001587in}{1.361789in}}{\pgfqpoint{4.005537in}{1.363425in}}{\pgfqpoint{4.008449in}{1.366337in}}%
\pgfpathcurveto{\pgfqpoint{4.011361in}{1.369249in}}{\pgfqpoint{4.012997in}{1.373199in}}{\pgfqpoint{4.012997in}{1.377317in}}%
\pgfpathcurveto{\pgfqpoint{4.012997in}{1.381435in}}{\pgfqpoint{4.011361in}{1.385385in}}{\pgfqpoint{4.008449in}{1.388297in}}%
\pgfpathcurveto{\pgfqpoint{4.005537in}{1.391209in}}{\pgfqpoint{4.001587in}{1.392846in}}{\pgfqpoint{3.997469in}{1.392846in}}%
\pgfpathcurveto{\pgfqpoint{3.993351in}{1.392846in}}{\pgfqpoint{3.989401in}{1.391209in}}{\pgfqpoint{3.986489in}{1.388297in}}%
\pgfpathcurveto{\pgfqpoint{3.983577in}{1.385385in}}{\pgfqpoint{3.981941in}{1.381435in}}{\pgfqpoint{3.981941in}{1.377317in}}%
\pgfpathcurveto{\pgfqpoint{3.981941in}{1.373199in}}{\pgfqpoint{3.983577in}{1.369249in}}{\pgfqpoint{3.986489in}{1.366337in}}%
\pgfpathcurveto{\pgfqpoint{3.989401in}{1.363425in}}{\pgfqpoint{3.993351in}{1.361789in}}{\pgfqpoint{3.997469in}{1.361789in}}%
\pgfpathclose%
\pgfusepath{fill}%
\end{pgfscope}%
\begin{pgfscope}%
\pgfpathrectangle{\pgfqpoint{0.887500in}{0.275000in}}{\pgfqpoint{4.225000in}{4.225000in}}%
\pgfusepath{clip}%
\pgfsetbuttcap%
\pgfsetroundjoin%
\definecolor{currentfill}{rgb}{0.000000,0.000000,0.000000}%
\pgfsetfillcolor{currentfill}%
\pgfsetfillopacity{0.800000}%
\pgfsetlinewidth{0.000000pt}%
\definecolor{currentstroke}{rgb}{0.000000,0.000000,0.000000}%
\pgfsetstrokecolor{currentstroke}%
\pgfsetstrokeopacity{0.800000}%
\pgfsetdash{}{0pt}%
\pgfpathmoveto{\pgfqpoint{2.139208in}{1.970424in}}%
\pgfpathcurveto{\pgfqpoint{2.143326in}{1.970424in}}{\pgfqpoint{2.147276in}{1.972060in}}{\pgfqpoint{2.150188in}{1.974972in}}%
\pgfpathcurveto{\pgfqpoint{2.153100in}{1.977884in}}{\pgfqpoint{2.154736in}{1.981834in}}{\pgfqpoint{2.154736in}{1.985952in}}%
\pgfpathcurveto{\pgfqpoint{2.154736in}{1.990070in}}{\pgfqpoint{2.153100in}{1.994020in}}{\pgfqpoint{2.150188in}{1.996932in}}%
\pgfpathcurveto{\pgfqpoint{2.147276in}{1.999844in}}{\pgfqpoint{2.143326in}{2.001480in}}{\pgfqpoint{2.139208in}{2.001480in}}%
\pgfpathcurveto{\pgfqpoint{2.135090in}{2.001480in}}{\pgfqpoint{2.131140in}{1.999844in}}{\pgfqpoint{2.128228in}{1.996932in}}%
\pgfpathcurveto{\pgfqpoint{2.125316in}{1.994020in}}{\pgfqpoint{2.123680in}{1.990070in}}{\pgfqpoint{2.123680in}{1.985952in}}%
\pgfpathcurveto{\pgfqpoint{2.123680in}{1.981834in}}{\pgfqpoint{2.125316in}{1.977884in}}{\pgfqpoint{2.128228in}{1.974972in}}%
\pgfpathcurveto{\pgfqpoint{2.131140in}{1.972060in}}{\pgfqpoint{2.135090in}{1.970424in}}{\pgfqpoint{2.139208in}{1.970424in}}%
\pgfpathclose%
\pgfusepath{fill}%
\end{pgfscope}%
\begin{pgfscope}%
\pgfpathrectangle{\pgfqpoint{0.887500in}{0.275000in}}{\pgfqpoint{4.225000in}{4.225000in}}%
\pgfusepath{clip}%
\pgfsetbuttcap%
\pgfsetroundjoin%
\definecolor{currentfill}{rgb}{0.000000,0.000000,0.000000}%
\pgfsetfillcolor{currentfill}%
\pgfsetfillopacity{0.800000}%
\pgfsetlinewidth{0.000000pt}%
\definecolor{currentstroke}{rgb}{0.000000,0.000000,0.000000}%
\pgfsetstrokecolor{currentstroke}%
\pgfsetstrokeopacity{0.800000}%
\pgfsetdash{}{0pt}%
\pgfpathmoveto{\pgfqpoint{2.977830in}{1.728618in}}%
\pgfpathcurveto{\pgfqpoint{2.981948in}{1.728618in}}{\pgfqpoint{2.985898in}{1.730255in}}{\pgfqpoint{2.988810in}{1.733167in}}%
\pgfpathcurveto{\pgfqpoint{2.991722in}{1.736079in}}{\pgfqpoint{2.993359in}{1.740029in}}{\pgfqpoint{2.993359in}{1.744147in}}%
\pgfpathcurveto{\pgfqpoint{2.993359in}{1.748265in}}{\pgfqpoint{2.991722in}{1.752215in}}{\pgfqpoint{2.988810in}{1.755127in}}%
\pgfpathcurveto{\pgfqpoint{2.985898in}{1.758039in}}{\pgfqpoint{2.981948in}{1.759675in}}{\pgfqpoint{2.977830in}{1.759675in}}%
\pgfpathcurveto{\pgfqpoint{2.973712in}{1.759675in}}{\pgfqpoint{2.969762in}{1.758039in}}{\pgfqpoint{2.966850in}{1.755127in}}%
\pgfpathcurveto{\pgfqpoint{2.963938in}{1.752215in}}{\pgfqpoint{2.962302in}{1.748265in}}{\pgfqpoint{2.962302in}{1.744147in}}%
\pgfpathcurveto{\pgfqpoint{2.962302in}{1.740029in}}{\pgfqpoint{2.963938in}{1.736079in}}{\pgfqpoint{2.966850in}{1.733167in}}%
\pgfpathcurveto{\pgfqpoint{2.969762in}{1.730255in}}{\pgfqpoint{2.973712in}{1.728618in}}{\pgfqpoint{2.977830in}{1.728618in}}%
\pgfpathclose%
\pgfusepath{fill}%
\end{pgfscope}%
\begin{pgfscope}%
\pgfpathrectangle{\pgfqpoint{0.887500in}{0.275000in}}{\pgfqpoint{4.225000in}{4.225000in}}%
\pgfusepath{clip}%
\pgfsetbuttcap%
\pgfsetroundjoin%
\definecolor{currentfill}{rgb}{0.000000,0.000000,0.000000}%
\pgfsetfillcolor{currentfill}%
\pgfsetfillopacity{0.800000}%
\pgfsetlinewidth{0.000000pt}%
\definecolor{currentstroke}{rgb}{0.000000,0.000000,0.000000}%
\pgfsetstrokecolor{currentstroke}%
\pgfsetstrokeopacity{0.800000}%
\pgfsetdash{}{0pt}%
\pgfpathmoveto{\pgfqpoint{3.158916in}{1.648041in}}%
\pgfpathcurveto{\pgfqpoint{3.163034in}{1.648041in}}{\pgfqpoint{3.166984in}{1.649677in}}{\pgfqpoint{3.169896in}{1.652589in}}%
\pgfpathcurveto{\pgfqpoint{3.172808in}{1.655501in}}{\pgfqpoint{3.174444in}{1.659451in}}{\pgfqpoint{3.174444in}{1.663569in}}%
\pgfpathcurveto{\pgfqpoint{3.174444in}{1.667687in}}{\pgfqpoint{3.172808in}{1.671637in}}{\pgfqpoint{3.169896in}{1.674549in}}%
\pgfpathcurveto{\pgfqpoint{3.166984in}{1.677461in}}{\pgfqpoint{3.163034in}{1.679097in}}{\pgfqpoint{3.158916in}{1.679097in}}%
\pgfpathcurveto{\pgfqpoint{3.154798in}{1.679097in}}{\pgfqpoint{3.150848in}{1.677461in}}{\pgfqpoint{3.147936in}{1.674549in}}%
\pgfpathcurveto{\pgfqpoint{3.145024in}{1.671637in}}{\pgfqpoint{3.143387in}{1.667687in}}{\pgfqpoint{3.143387in}{1.663569in}}%
\pgfpathcurveto{\pgfqpoint{3.143387in}{1.659451in}}{\pgfqpoint{3.145024in}{1.655501in}}{\pgfqpoint{3.147936in}{1.652589in}}%
\pgfpathcurveto{\pgfqpoint{3.150848in}{1.649677in}}{\pgfqpoint{3.154798in}{1.648041in}}{\pgfqpoint{3.158916in}{1.648041in}}%
\pgfpathclose%
\pgfusepath{fill}%
\end{pgfscope}%
\begin{pgfscope}%
\pgfpathrectangle{\pgfqpoint{0.887500in}{0.275000in}}{\pgfqpoint{4.225000in}{4.225000in}}%
\pgfusepath{clip}%
\pgfsetbuttcap%
\pgfsetroundjoin%
\definecolor{currentfill}{rgb}{0.000000,0.000000,0.000000}%
\pgfsetfillcolor{currentfill}%
\pgfsetfillopacity{0.800000}%
\pgfsetlinewidth{0.000000pt}%
\definecolor{currentstroke}{rgb}{0.000000,0.000000,0.000000}%
\pgfsetstrokecolor{currentstroke}%
\pgfsetstrokeopacity{0.800000}%
\pgfsetdash{}{0pt}%
\pgfpathmoveto{\pgfqpoint{2.319581in}{1.905493in}}%
\pgfpathcurveto{\pgfqpoint{2.323699in}{1.905493in}}{\pgfqpoint{2.327649in}{1.907129in}}{\pgfqpoint{2.330561in}{1.910041in}}%
\pgfpathcurveto{\pgfqpoint{2.333473in}{1.912953in}}{\pgfqpoint{2.335109in}{1.916903in}}{\pgfqpoint{2.335109in}{1.921021in}}%
\pgfpathcurveto{\pgfqpoint{2.335109in}{1.925139in}}{\pgfqpoint{2.333473in}{1.929090in}}{\pgfqpoint{2.330561in}{1.932001in}}%
\pgfpathcurveto{\pgfqpoint{2.327649in}{1.934913in}}{\pgfqpoint{2.323699in}{1.936550in}}{\pgfqpoint{2.319581in}{1.936550in}}%
\pgfpathcurveto{\pgfqpoint{2.315463in}{1.936550in}}{\pgfqpoint{2.311513in}{1.934913in}}{\pgfqpoint{2.308601in}{1.932001in}}%
\pgfpathcurveto{\pgfqpoint{2.305689in}{1.929090in}}{\pgfqpoint{2.304053in}{1.925139in}}{\pgfqpoint{2.304053in}{1.921021in}}%
\pgfpathcurveto{\pgfqpoint{2.304053in}{1.916903in}}{\pgfqpoint{2.305689in}{1.912953in}}{\pgfqpoint{2.308601in}{1.910041in}}%
\pgfpathcurveto{\pgfqpoint{2.311513in}{1.907129in}}{\pgfqpoint{2.315463in}{1.905493in}}{\pgfqpoint{2.319581in}{1.905493in}}%
\pgfpathclose%
\pgfusepath{fill}%
\end{pgfscope}%
\begin{pgfscope}%
\pgfpathrectangle{\pgfqpoint{0.887500in}{0.275000in}}{\pgfqpoint{4.225000in}{4.225000in}}%
\pgfusepath{clip}%
\pgfsetbuttcap%
\pgfsetroundjoin%
\definecolor{currentfill}{rgb}{0.000000,0.000000,0.000000}%
\pgfsetfillcolor{currentfill}%
\pgfsetfillopacity{0.800000}%
\pgfsetlinewidth{0.000000pt}%
\definecolor{currentstroke}{rgb}{0.000000,0.000000,0.000000}%
\pgfsetstrokecolor{currentstroke}%
\pgfsetstrokeopacity{0.800000}%
\pgfsetdash{}{0pt}%
\pgfpathmoveto{\pgfqpoint{3.521255in}{1.462364in}}%
\pgfpathcurveto{\pgfqpoint{3.525373in}{1.462364in}}{\pgfqpoint{3.529323in}{1.464000in}}{\pgfqpoint{3.532235in}{1.466912in}}%
\pgfpathcurveto{\pgfqpoint{3.535147in}{1.469824in}}{\pgfqpoint{3.536784in}{1.473774in}}{\pgfqpoint{3.536784in}{1.477892in}}%
\pgfpathcurveto{\pgfqpoint{3.536784in}{1.482010in}}{\pgfqpoint{3.535147in}{1.485960in}}{\pgfqpoint{3.532235in}{1.488872in}}%
\pgfpathcurveto{\pgfqpoint{3.529323in}{1.491784in}}{\pgfqpoint{3.525373in}{1.493420in}}{\pgfqpoint{3.521255in}{1.493420in}}%
\pgfpathcurveto{\pgfqpoint{3.517137in}{1.493420in}}{\pgfqpoint{3.513187in}{1.491784in}}{\pgfqpoint{3.510275in}{1.488872in}}%
\pgfpathcurveto{\pgfqpoint{3.507363in}{1.485960in}}{\pgfqpoint{3.505727in}{1.482010in}}{\pgfqpoint{3.505727in}{1.477892in}}%
\pgfpathcurveto{\pgfqpoint{3.505727in}{1.473774in}}{\pgfqpoint{3.507363in}{1.469824in}}{\pgfqpoint{3.510275in}{1.466912in}}%
\pgfpathcurveto{\pgfqpoint{3.513187in}{1.464000in}}{\pgfqpoint{3.517137in}{1.462364in}}{\pgfqpoint{3.521255in}{1.462364in}}%
\pgfpathclose%
\pgfusepath{fill}%
\end{pgfscope}%
\begin{pgfscope}%
\pgfpathrectangle{\pgfqpoint{0.887500in}{0.275000in}}{\pgfqpoint{4.225000in}{4.225000in}}%
\pgfusepath{clip}%
\pgfsetbuttcap%
\pgfsetroundjoin%
\definecolor{currentfill}{rgb}{0.000000,0.000000,0.000000}%
\pgfsetfillcolor{currentfill}%
\pgfsetfillopacity{0.800000}%
\pgfsetlinewidth{0.000000pt}%
\definecolor{currentstroke}{rgb}{0.000000,0.000000,0.000000}%
\pgfsetstrokecolor{currentstroke}%
\pgfsetstrokeopacity{0.800000}%
\pgfsetdash{}{0pt}%
\pgfpathmoveto{\pgfqpoint{3.702427in}{1.366049in}}%
\pgfpathcurveto{\pgfqpoint{3.706545in}{1.366049in}}{\pgfqpoint{3.710495in}{1.367685in}}{\pgfqpoint{3.713407in}{1.370597in}}%
\pgfpathcurveto{\pgfqpoint{3.716319in}{1.373509in}}{\pgfqpoint{3.717955in}{1.377459in}}{\pgfqpoint{3.717955in}{1.381577in}}%
\pgfpathcurveto{\pgfqpoint{3.717955in}{1.385696in}}{\pgfqpoint{3.716319in}{1.389646in}}{\pgfqpoint{3.713407in}{1.392558in}}%
\pgfpathcurveto{\pgfqpoint{3.710495in}{1.395469in}}{\pgfqpoint{3.706545in}{1.397106in}}{\pgfqpoint{3.702427in}{1.397106in}}%
\pgfpathcurveto{\pgfqpoint{3.698309in}{1.397106in}}{\pgfqpoint{3.694359in}{1.395469in}}{\pgfqpoint{3.691447in}{1.392558in}}%
\pgfpathcurveto{\pgfqpoint{3.688535in}{1.389646in}}{\pgfqpoint{3.686899in}{1.385696in}}{\pgfqpoint{3.686899in}{1.381577in}}%
\pgfpathcurveto{\pgfqpoint{3.686899in}{1.377459in}}{\pgfqpoint{3.688535in}{1.373509in}}{\pgfqpoint{3.691447in}{1.370597in}}%
\pgfpathcurveto{\pgfqpoint{3.694359in}{1.367685in}}{\pgfqpoint{3.698309in}{1.366049in}}{\pgfqpoint{3.702427in}{1.366049in}}%
\pgfpathclose%
\pgfusepath{fill}%
\end{pgfscope}%
\begin{pgfscope}%
\pgfpathrectangle{\pgfqpoint{0.887500in}{0.275000in}}{\pgfqpoint{4.225000in}{4.225000in}}%
\pgfusepath{clip}%
\pgfsetbuttcap%
\pgfsetroundjoin%
\definecolor{currentfill}{rgb}{0.000000,0.000000,0.000000}%
\pgfsetfillcolor{currentfill}%
\pgfsetfillopacity{0.800000}%
\pgfsetlinewidth{0.000000pt}%
\definecolor{currentstroke}{rgb}{0.000000,0.000000,0.000000}%
\pgfsetstrokecolor{currentstroke}%
\pgfsetstrokeopacity{0.800000}%
\pgfsetdash{}{0pt}%
\pgfpathmoveto{\pgfqpoint{3.340103in}{1.560384in}}%
\pgfpathcurveto{\pgfqpoint{3.344222in}{1.560384in}}{\pgfqpoint{3.348172in}{1.562020in}}{\pgfqpoint{3.351084in}{1.564932in}}%
\pgfpathcurveto{\pgfqpoint{3.353996in}{1.567844in}}{\pgfqpoint{3.355632in}{1.571794in}}{\pgfqpoint{3.355632in}{1.575912in}}%
\pgfpathcurveto{\pgfqpoint{3.355632in}{1.580030in}}{\pgfqpoint{3.353996in}{1.583980in}}{\pgfqpoint{3.351084in}{1.586892in}}%
\pgfpathcurveto{\pgfqpoint{3.348172in}{1.589804in}}{\pgfqpoint{3.344222in}{1.591440in}}{\pgfqpoint{3.340103in}{1.591440in}}%
\pgfpathcurveto{\pgfqpoint{3.335985in}{1.591440in}}{\pgfqpoint{3.332035in}{1.589804in}}{\pgfqpoint{3.329123in}{1.586892in}}%
\pgfpathcurveto{\pgfqpoint{3.326211in}{1.583980in}}{\pgfqpoint{3.324575in}{1.580030in}}{\pgfqpoint{3.324575in}{1.575912in}}%
\pgfpathcurveto{\pgfqpoint{3.324575in}{1.571794in}}{\pgfqpoint{3.326211in}{1.567844in}}{\pgfqpoint{3.329123in}{1.564932in}}%
\pgfpathcurveto{\pgfqpoint{3.332035in}{1.562020in}}{\pgfqpoint{3.335985in}{1.560384in}}{\pgfqpoint{3.340103in}{1.560384in}}%
\pgfpathclose%
\pgfusepath{fill}%
\end{pgfscope}%
\begin{pgfscope}%
\pgfpathrectangle{\pgfqpoint{0.887500in}{0.275000in}}{\pgfqpoint{4.225000in}{4.225000in}}%
\pgfusepath{clip}%
\pgfsetbuttcap%
\pgfsetroundjoin%
\definecolor{currentfill}{rgb}{0.000000,0.000000,0.000000}%
\pgfsetfillcolor{currentfill}%
\pgfsetfillopacity{0.800000}%
\pgfsetlinewidth{0.000000pt}%
\definecolor{currentstroke}{rgb}{0.000000,0.000000,0.000000}%
\pgfsetstrokecolor{currentstroke}%
\pgfsetstrokeopacity{0.800000}%
\pgfsetdash{}{0pt}%
\pgfpathmoveto{\pgfqpoint{1.841613in}{1.995456in}}%
\pgfpathcurveto{\pgfqpoint{1.845731in}{1.995456in}}{\pgfqpoint{1.849681in}{1.997092in}}{\pgfqpoint{1.852593in}{2.000004in}}%
\pgfpathcurveto{\pgfqpoint{1.855505in}{2.002916in}}{\pgfqpoint{1.857141in}{2.006866in}}{\pgfqpoint{1.857141in}{2.010985in}}%
\pgfpathcurveto{\pgfqpoint{1.857141in}{2.015103in}}{\pgfqpoint{1.855505in}{2.019053in}}{\pgfqpoint{1.852593in}{2.021965in}}%
\pgfpathcurveto{\pgfqpoint{1.849681in}{2.024877in}}{\pgfqpoint{1.845731in}{2.026513in}}{\pgfqpoint{1.841613in}{2.026513in}}%
\pgfpathcurveto{\pgfqpoint{1.837495in}{2.026513in}}{\pgfqpoint{1.833545in}{2.024877in}}{\pgfqpoint{1.830633in}{2.021965in}}%
\pgfpathcurveto{\pgfqpoint{1.827721in}{2.019053in}}{\pgfqpoint{1.826085in}{2.015103in}}{\pgfqpoint{1.826085in}{2.010985in}}%
\pgfpathcurveto{\pgfqpoint{1.826085in}{2.006866in}}{\pgfqpoint{1.827721in}{2.002916in}}{\pgfqpoint{1.830633in}{2.000004in}}%
\pgfpathcurveto{\pgfqpoint{1.833545in}{1.997092in}}{\pgfqpoint{1.837495in}{1.995456in}}{\pgfqpoint{1.841613in}{1.995456in}}%
\pgfpathclose%
\pgfusepath{fill}%
\end{pgfscope}%
\begin{pgfscope}%
\pgfpathrectangle{\pgfqpoint{0.887500in}{0.275000in}}{\pgfqpoint{4.225000in}{4.225000in}}%
\pgfusepath{clip}%
\pgfsetbuttcap%
\pgfsetroundjoin%
\definecolor{currentfill}{rgb}{0.000000,0.000000,0.000000}%
\pgfsetfillcolor{currentfill}%
\pgfsetfillopacity{0.800000}%
\pgfsetlinewidth{0.000000pt}%
\definecolor{currentstroke}{rgb}{0.000000,0.000000,0.000000}%
\pgfsetstrokecolor{currentstroke}%
\pgfsetstrokeopacity{0.800000}%
\pgfsetdash{}{0pt}%
\pgfpathmoveto{\pgfqpoint{2.500292in}{1.837526in}}%
\pgfpathcurveto{\pgfqpoint{2.504411in}{1.837526in}}{\pgfqpoint{2.508361in}{1.839163in}}{\pgfqpoint{2.511273in}{1.842074in}}%
\pgfpathcurveto{\pgfqpoint{2.514185in}{1.844986in}}{\pgfqpoint{2.515821in}{1.848936in}}{\pgfqpoint{2.515821in}{1.853055in}}%
\pgfpathcurveto{\pgfqpoint{2.515821in}{1.857173in}}{\pgfqpoint{2.514185in}{1.861123in}}{\pgfqpoint{2.511273in}{1.864035in}}%
\pgfpathcurveto{\pgfqpoint{2.508361in}{1.866947in}}{\pgfqpoint{2.504411in}{1.868583in}}{\pgfqpoint{2.500292in}{1.868583in}}%
\pgfpathcurveto{\pgfqpoint{2.496174in}{1.868583in}}{\pgfqpoint{2.492224in}{1.866947in}}{\pgfqpoint{2.489312in}{1.864035in}}%
\pgfpathcurveto{\pgfqpoint{2.486400in}{1.861123in}}{\pgfqpoint{2.484764in}{1.857173in}}{\pgfqpoint{2.484764in}{1.853055in}}%
\pgfpathcurveto{\pgfqpoint{2.484764in}{1.848936in}}{\pgfqpoint{2.486400in}{1.844986in}}{\pgfqpoint{2.489312in}{1.842074in}}%
\pgfpathcurveto{\pgfqpoint{2.492224in}{1.839163in}}{\pgfqpoint{2.496174in}{1.837526in}}{\pgfqpoint{2.500292in}{1.837526in}}%
\pgfpathclose%
\pgfusepath{fill}%
\end{pgfscope}%
\begin{pgfscope}%
\pgfpathrectangle{\pgfqpoint{0.887500in}{0.275000in}}{\pgfqpoint{4.225000in}{4.225000in}}%
\pgfusepath{clip}%
\pgfsetbuttcap%
\pgfsetroundjoin%
\definecolor{currentfill}{rgb}{0.000000,0.000000,0.000000}%
\pgfsetfillcolor{currentfill}%
\pgfsetfillopacity{0.800000}%
\pgfsetlinewidth{0.000000pt}%
\definecolor{currentstroke}{rgb}{0.000000,0.000000,0.000000}%
\pgfsetstrokecolor{currentstroke}%
\pgfsetstrokeopacity{0.800000}%
\pgfsetdash{}{0pt}%
\pgfpathmoveto{\pgfqpoint{2.681310in}{1.765758in}}%
\pgfpathcurveto{\pgfqpoint{2.685428in}{1.765758in}}{\pgfqpoint{2.689378in}{1.767394in}}{\pgfqpoint{2.692290in}{1.770306in}}%
\pgfpathcurveto{\pgfqpoint{2.695202in}{1.773218in}}{\pgfqpoint{2.696838in}{1.777168in}}{\pgfqpoint{2.696838in}{1.781286in}}%
\pgfpathcurveto{\pgfqpoint{2.696838in}{1.785404in}}{\pgfqpoint{2.695202in}{1.789354in}}{\pgfqpoint{2.692290in}{1.792266in}}%
\pgfpathcurveto{\pgfqpoint{2.689378in}{1.795178in}}{\pgfqpoint{2.685428in}{1.796814in}}{\pgfqpoint{2.681310in}{1.796814in}}%
\pgfpathcurveto{\pgfqpoint{2.677192in}{1.796814in}}{\pgfqpoint{2.673242in}{1.795178in}}{\pgfqpoint{2.670330in}{1.792266in}}%
\pgfpathcurveto{\pgfqpoint{2.667418in}{1.789354in}}{\pgfqpoint{2.665782in}{1.785404in}}{\pgfqpoint{2.665782in}{1.781286in}}%
\pgfpathcurveto{\pgfqpoint{2.665782in}{1.777168in}}{\pgfqpoint{2.667418in}{1.773218in}}{\pgfqpoint{2.670330in}{1.770306in}}%
\pgfpathcurveto{\pgfqpoint{2.673242in}{1.767394in}}{\pgfqpoint{2.677192in}{1.765758in}}{\pgfqpoint{2.681310in}{1.765758in}}%
\pgfpathclose%
\pgfusepath{fill}%
\end{pgfscope}%
\begin{pgfscope}%
\pgfpathrectangle{\pgfqpoint{0.887500in}{0.275000in}}{\pgfqpoint{4.225000in}{4.225000in}}%
\pgfusepath{clip}%
\pgfsetbuttcap%
\pgfsetroundjoin%
\definecolor{currentfill}{rgb}{0.000000,0.000000,0.000000}%
\pgfsetfillcolor{currentfill}%
\pgfsetfillopacity{0.800000}%
\pgfsetlinewidth{0.000000pt}%
\definecolor{currentstroke}{rgb}{0.000000,0.000000,0.000000}%
\pgfsetstrokecolor{currentstroke}%
\pgfsetstrokeopacity{0.800000}%
\pgfsetdash{}{0pt}%
\pgfpathmoveto{\pgfqpoint{2.021968in}{1.932433in}}%
\pgfpathcurveto{\pgfqpoint{2.026086in}{1.932433in}}{\pgfqpoint{2.030036in}{1.934070in}}{\pgfqpoint{2.032948in}{1.936982in}}%
\pgfpathcurveto{\pgfqpoint{2.035860in}{1.939894in}}{\pgfqpoint{2.037496in}{1.943844in}}{\pgfqpoint{2.037496in}{1.947962in}}%
\pgfpathcurveto{\pgfqpoint{2.037496in}{1.952080in}}{\pgfqpoint{2.035860in}{1.956030in}}{\pgfqpoint{2.032948in}{1.958942in}}%
\pgfpathcurveto{\pgfqpoint{2.030036in}{1.961854in}}{\pgfqpoint{2.026086in}{1.963490in}}{\pgfqpoint{2.021968in}{1.963490in}}%
\pgfpathcurveto{\pgfqpoint{2.017850in}{1.963490in}}{\pgfqpoint{2.013900in}{1.961854in}}{\pgfqpoint{2.010988in}{1.958942in}}%
\pgfpathcurveto{\pgfqpoint{2.008076in}{1.956030in}}{\pgfqpoint{2.006440in}{1.952080in}}{\pgfqpoint{2.006440in}{1.947962in}}%
\pgfpathcurveto{\pgfqpoint{2.006440in}{1.943844in}}{\pgfqpoint{2.008076in}{1.939894in}}{\pgfqpoint{2.010988in}{1.936982in}}%
\pgfpathcurveto{\pgfqpoint{2.013900in}{1.934070in}}{\pgfqpoint{2.017850in}{1.932433in}}{\pgfqpoint{2.021968in}{1.932433in}}%
\pgfpathclose%
\pgfusepath{fill}%
\end{pgfscope}%
\begin{pgfscope}%
\pgfpathrectangle{\pgfqpoint{0.887500in}{0.275000in}}{\pgfqpoint{4.225000in}{4.225000in}}%
\pgfusepath{clip}%
\pgfsetbuttcap%
\pgfsetroundjoin%
\definecolor{currentfill}{rgb}{0.000000,0.000000,0.000000}%
\pgfsetfillcolor{currentfill}%
\pgfsetfillopacity{0.800000}%
\pgfsetlinewidth{0.000000pt}%
\definecolor{currentstroke}{rgb}{0.000000,0.000000,0.000000}%
\pgfsetstrokecolor{currentstroke}%
\pgfsetstrokeopacity{0.800000}%
\pgfsetdash{}{0pt}%
\pgfpathmoveto{\pgfqpoint{2.862585in}{1.689684in}}%
\pgfpathcurveto{\pgfqpoint{2.866703in}{1.689684in}}{\pgfqpoint{2.870653in}{1.691321in}}{\pgfqpoint{2.873565in}{1.694233in}}%
\pgfpathcurveto{\pgfqpoint{2.876477in}{1.697145in}}{\pgfqpoint{2.878113in}{1.701095in}}{\pgfqpoint{2.878113in}{1.705213in}}%
\pgfpathcurveto{\pgfqpoint{2.878113in}{1.709331in}}{\pgfqpoint{2.876477in}{1.713281in}}{\pgfqpoint{2.873565in}{1.716193in}}%
\pgfpathcurveto{\pgfqpoint{2.870653in}{1.719105in}}{\pgfqpoint{2.866703in}{1.720741in}}{\pgfqpoint{2.862585in}{1.720741in}}%
\pgfpathcurveto{\pgfqpoint{2.858467in}{1.720741in}}{\pgfqpoint{2.854517in}{1.719105in}}{\pgfqpoint{2.851605in}{1.716193in}}%
\pgfpathcurveto{\pgfqpoint{2.848693in}{1.713281in}}{\pgfqpoint{2.847057in}{1.709331in}}{\pgfqpoint{2.847057in}{1.705213in}}%
\pgfpathcurveto{\pgfqpoint{2.847057in}{1.701095in}}{\pgfqpoint{2.848693in}{1.697145in}}{\pgfqpoint{2.851605in}{1.694233in}}%
\pgfpathcurveto{\pgfqpoint{2.854517in}{1.691321in}}{\pgfqpoint{2.858467in}{1.689684in}}{\pgfqpoint{2.862585in}{1.689684in}}%
\pgfpathclose%
\pgfusepath{fill}%
\end{pgfscope}%
\begin{pgfscope}%
\pgfpathrectangle{\pgfqpoint{0.887500in}{0.275000in}}{\pgfqpoint{4.225000in}{4.225000in}}%
\pgfusepath{clip}%
\pgfsetbuttcap%
\pgfsetroundjoin%
\definecolor{currentfill}{rgb}{0.000000,0.000000,0.000000}%
\pgfsetfillcolor{currentfill}%
\pgfsetfillopacity{0.800000}%
\pgfsetlinewidth{0.000000pt}%
\definecolor{currentstroke}{rgb}{0.000000,0.000000,0.000000}%
\pgfsetstrokecolor{currentstroke}%
\pgfsetstrokeopacity{0.800000}%
\pgfsetdash{}{0pt}%
\pgfpathmoveto{\pgfqpoint{3.044059in}{1.608896in}}%
\pgfpathcurveto{\pgfqpoint{3.048178in}{1.608896in}}{\pgfqpoint{3.052128in}{1.610532in}}{\pgfqpoint{3.055040in}{1.613444in}}%
\pgfpathcurveto{\pgfqpoint{3.057951in}{1.616356in}}{\pgfqpoint{3.059588in}{1.620306in}}{\pgfqpoint{3.059588in}{1.624424in}}%
\pgfpathcurveto{\pgfqpoint{3.059588in}{1.628542in}}{\pgfqpoint{3.057951in}{1.632492in}}{\pgfqpoint{3.055040in}{1.635404in}}%
\pgfpathcurveto{\pgfqpoint{3.052128in}{1.638316in}}{\pgfqpoint{3.048178in}{1.639952in}}{\pgfqpoint{3.044059in}{1.639952in}}%
\pgfpathcurveto{\pgfqpoint{3.039941in}{1.639952in}}{\pgfqpoint{3.035991in}{1.638316in}}{\pgfqpoint{3.033079in}{1.635404in}}%
\pgfpathcurveto{\pgfqpoint{3.030167in}{1.632492in}}{\pgfqpoint{3.028531in}{1.628542in}}{\pgfqpoint{3.028531in}{1.624424in}}%
\pgfpathcurveto{\pgfqpoint{3.028531in}{1.620306in}}{\pgfqpoint{3.030167in}{1.616356in}}{\pgfqpoint{3.033079in}{1.613444in}}%
\pgfpathcurveto{\pgfqpoint{3.035991in}{1.610532in}}{\pgfqpoint{3.039941in}{1.608896in}}{\pgfqpoint{3.044059in}{1.608896in}}%
\pgfpathclose%
\pgfusepath{fill}%
\end{pgfscope}%
\begin{pgfscope}%
\pgfpathrectangle{\pgfqpoint{0.887500in}{0.275000in}}{\pgfqpoint{4.225000in}{4.225000in}}%
\pgfusepath{clip}%
\pgfsetbuttcap%
\pgfsetroundjoin%
\definecolor{currentfill}{rgb}{0.000000,0.000000,0.000000}%
\pgfsetfillcolor{currentfill}%
\pgfsetfillopacity{0.800000}%
\pgfsetlinewidth{0.000000pt}%
\definecolor{currentstroke}{rgb}{0.000000,0.000000,0.000000}%
\pgfsetstrokecolor{currentstroke}%
\pgfsetstrokeopacity{0.800000}%
\pgfsetdash{}{0pt}%
\pgfpathmoveto{\pgfqpoint{3.770297in}{1.220232in}}%
\pgfpathcurveto{\pgfqpoint{3.774415in}{1.220232in}}{\pgfqpoint{3.778365in}{1.221868in}}{\pgfqpoint{3.781277in}{1.224780in}}%
\pgfpathcurveto{\pgfqpoint{3.784189in}{1.227692in}}{\pgfqpoint{3.785825in}{1.231642in}}{\pgfqpoint{3.785825in}{1.235760in}}%
\pgfpathcurveto{\pgfqpoint{3.785825in}{1.239878in}}{\pgfqpoint{3.784189in}{1.243828in}}{\pgfqpoint{3.781277in}{1.246740in}}%
\pgfpathcurveto{\pgfqpoint{3.778365in}{1.249652in}}{\pgfqpoint{3.774415in}{1.251288in}}{\pgfqpoint{3.770297in}{1.251288in}}%
\pgfpathcurveto{\pgfqpoint{3.766179in}{1.251288in}}{\pgfqpoint{3.762229in}{1.249652in}}{\pgfqpoint{3.759317in}{1.246740in}}%
\pgfpathcurveto{\pgfqpoint{3.756405in}{1.243828in}}{\pgfqpoint{3.754769in}{1.239878in}}{\pgfqpoint{3.754769in}{1.235760in}}%
\pgfpathcurveto{\pgfqpoint{3.754769in}{1.231642in}}{\pgfqpoint{3.756405in}{1.227692in}}{\pgfqpoint{3.759317in}{1.224780in}}%
\pgfpathcurveto{\pgfqpoint{3.762229in}{1.221868in}}{\pgfqpoint{3.766179in}{1.220232in}}{\pgfqpoint{3.770297in}{1.220232in}}%
\pgfpathclose%
\pgfusepath{fill}%
\end{pgfscope}%
\begin{pgfscope}%
\pgfpathrectangle{\pgfqpoint{0.887500in}{0.275000in}}{\pgfqpoint{4.225000in}{4.225000in}}%
\pgfusepath{clip}%
\pgfsetbuttcap%
\pgfsetroundjoin%
\definecolor{currentfill}{rgb}{0.000000,0.000000,0.000000}%
\pgfsetfillcolor{currentfill}%
\pgfsetfillopacity{0.800000}%
\pgfsetlinewidth{0.000000pt}%
\definecolor{currentstroke}{rgb}{0.000000,0.000000,0.000000}%
\pgfsetstrokecolor{currentstroke}%
\pgfsetstrokeopacity{0.800000}%
\pgfsetdash{}{0pt}%
\pgfpathmoveto{\pgfqpoint{2.202693in}{1.866796in}}%
\pgfpathcurveto{\pgfqpoint{2.206812in}{1.866796in}}{\pgfqpoint{2.210762in}{1.868432in}}{\pgfqpoint{2.213674in}{1.871344in}}%
\pgfpathcurveto{\pgfqpoint{2.216586in}{1.874256in}}{\pgfqpoint{2.218222in}{1.878206in}}{\pgfqpoint{2.218222in}{1.882324in}}%
\pgfpathcurveto{\pgfqpoint{2.218222in}{1.886442in}}{\pgfqpoint{2.216586in}{1.890392in}}{\pgfqpoint{2.213674in}{1.893304in}}%
\pgfpathcurveto{\pgfqpoint{2.210762in}{1.896216in}}{\pgfqpoint{2.206812in}{1.897852in}}{\pgfqpoint{2.202693in}{1.897852in}}%
\pgfpathcurveto{\pgfqpoint{2.198575in}{1.897852in}}{\pgfqpoint{2.194625in}{1.896216in}}{\pgfqpoint{2.191713in}{1.893304in}}%
\pgfpathcurveto{\pgfqpoint{2.188801in}{1.890392in}}{\pgfqpoint{2.187165in}{1.886442in}}{\pgfqpoint{2.187165in}{1.882324in}}%
\pgfpathcurveto{\pgfqpoint{2.187165in}{1.878206in}}{\pgfqpoint{2.188801in}{1.874256in}}{\pgfqpoint{2.191713in}{1.871344in}}%
\pgfpathcurveto{\pgfqpoint{2.194625in}{1.868432in}}{\pgfqpoint{2.198575in}{1.866796in}}{\pgfqpoint{2.202693in}{1.866796in}}%
\pgfpathclose%
\pgfusepath{fill}%
\end{pgfscope}%
\begin{pgfscope}%
\pgfpathrectangle{\pgfqpoint{0.887500in}{0.275000in}}{\pgfqpoint{4.225000in}{4.225000in}}%
\pgfusepath{clip}%
\pgfsetbuttcap%
\pgfsetroundjoin%
\definecolor{currentfill}{rgb}{0.000000,0.000000,0.000000}%
\pgfsetfillcolor{currentfill}%
\pgfsetfillopacity{0.800000}%
\pgfsetlinewidth{0.000000pt}%
\definecolor{currentstroke}{rgb}{0.000000,0.000000,0.000000}%
\pgfsetstrokecolor{currentstroke}%
\pgfsetstrokeopacity{0.800000}%
\pgfsetdash{}{0pt}%
\pgfpathmoveto{\pgfqpoint{3.225659in}{1.521445in}}%
\pgfpathcurveto{\pgfqpoint{3.229777in}{1.521445in}}{\pgfqpoint{3.233727in}{1.523081in}}{\pgfqpoint{3.236639in}{1.525993in}}%
\pgfpathcurveto{\pgfqpoint{3.239551in}{1.528905in}}{\pgfqpoint{3.241188in}{1.532855in}}{\pgfqpoint{3.241188in}{1.536973in}}%
\pgfpathcurveto{\pgfqpoint{3.241188in}{1.541091in}}{\pgfqpoint{3.239551in}{1.545041in}}{\pgfqpoint{3.236639in}{1.547953in}}%
\pgfpathcurveto{\pgfqpoint{3.233727in}{1.550865in}}{\pgfqpoint{3.229777in}{1.552501in}}{\pgfqpoint{3.225659in}{1.552501in}}%
\pgfpathcurveto{\pgfqpoint{3.221541in}{1.552501in}}{\pgfqpoint{3.217591in}{1.550865in}}{\pgfqpoint{3.214679in}{1.547953in}}%
\pgfpathcurveto{\pgfqpoint{3.211767in}{1.545041in}}{\pgfqpoint{3.210131in}{1.541091in}}{\pgfqpoint{3.210131in}{1.536973in}}%
\pgfpathcurveto{\pgfqpoint{3.210131in}{1.532855in}}{\pgfqpoint{3.211767in}{1.528905in}}{\pgfqpoint{3.214679in}{1.525993in}}%
\pgfpathcurveto{\pgfqpoint{3.217591in}{1.523081in}}{\pgfqpoint{3.221541in}{1.521445in}}{\pgfqpoint{3.225659in}{1.521445in}}%
\pgfpathclose%
\pgfusepath{fill}%
\end{pgfscope}%
\begin{pgfscope}%
\pgfpathrectangle{\pgfqpoint{0.887500in}{0.275000in}}{\pgfqpoint{4.225000in}{4.225000in}}%
\pgfusepath{clip}%
\pgfsetbuttcap%
\pgfsetroundjoin%
\definecolor{currentfill}{rgb}{0.000000,0.000000,0.000000}%
\pgfsetfillcolor{currentfill}%
\pgfsetfillopacity{0.800000}%
\pgfsetlinewidth{0.000000pt}%
\definecolor{currentstroke}{rgb}{0.000000,0.000000,0.000000}%
\pgfsetstrokecolor{currentstroke}%
\pgfsetstrokeopacity{0.800000}%
\pgfsetdash{}{0pt}%
\pgfpathmoveto{\pgfqpoint{3.407263in}{1.424127in}}%
\pgfpathcurveto{\pgfqpoint{3.411381in}{1.424127in}}{\pgfqpoint{3.415331in}{1.425763in}}{\pgfqpoint{3.418243in}{1.428675in}}%
\pgfpathcurveto{\pgfqpoint{3.421155in}{1.431587in}}{\pgfqpoint{3.422791in}{1.435537in}}{\pgfqpoint{3.422791in}{1.439655in}}%
\pgfpathcurveto{\pgfqpoint{3.422791in}{1.443773in}}{\pgfqpoint{3.421155in}{1.447723in}}{\pgfqpoint{3.418243in}{1.450635in}}%
\pgfpathcurveto{\pgfqpoint{3.415331in}{1.453547in}}{\pgfqpoint{3.411381in}{1.455183in}}{\pgfqpoint{3.407263in}{1.455183in}}%
\pgfpathcurveto{\pgfqpoint{3.403144in}{1.455183in}}{\pgfqpoint{3.399194in}{1.453547in}}{\pgfqpoint{3.396282in}{1.450635in}}%
\pgfpathcurveto{\pgfqpoint{3.393370in}{1.447723in}}{\pgfqpoint{3.391734in}{1.443773in}}{\pgfqpoint{3.391734in}{1.439655in}}%
\pgfpathcurveto{\pgfqpoint{3.391734in}{1.435537in}}{\pgfqpoint{3.393370in}{1.431587in}}{\pgfqpoint{3.396282in}{1.428675in}}%
\pgfpathcurveto{\pgfqpoint{3.399194in}{1.425763in}}{\pgfqpoint{3.403144in}{1.424127in}}{\pgfqpoint{3.407263in}{1.424127in}}%
\pgfpathclose%
\pgfusepath{fill}%
\end{pgfscope}%
\begin{pgfscope}%
\pgfpathrectangle{\pgfqpoint{0.887500in}{0.275000in}}{\pgfqpoint{4.225000in}{4.225000in}}%
\pgfusepath{clip}%
\pgfsetbuttcap%
\pgfsetroundjoin%
\definecolor{currentfill}{rgb}{0.000000,0.000000,0.000000}%
\pgfsetfillcolor{currentfill}%
\pgfsetfillopacity{0.800000}%
\pgfsetlinewidth{0.000000pt}%
\definecolor{currentstroke}{rgb}{0.000000,0.000000,0.000000}%
\pgfsetstrokecolor{currentstroke}%
\pgfsetstrokeopacity{0.800000}%
\pgfsetdash{}{0pt}%
\pgfpathmoveto{\pgfqpoint{3.588875in}{1.327490in}}%
\pgfpathcurveto{\pgfqpoint{3.592993in}{1.327490in}}{\pgfqpoint{3.596943in}{1.329126in}}{\pgfqpoint{3.599855in}{1.332038in}}%
\pgfpathcurveto{\pgfqpoint{3.602767in}{1.334950in}}{\pgfqpoint{3.604404in}{1.338900in}}{\pgfqpoint{3.604404in}{1.343018in}}%
\pgfpathcurveto{\pgfqpoint{3.604404in}{1.347136in}}{\pgfqpoint{3.602767in}{1.351086in}}{\pgfqpoint{3.599855in}{1.353998in}}%
\pgfpathcurveto{\pgfqpoint{3.596943in}{1.356910in}}{\pgfqpoint{3.592993in}{1.358547in}}{\pgfqpoint{3.588875in}{1.358547in}}%
\pgfpathcurveto{\pgfqpoint{3.584757in}{1.358547in}}{\pgfqpoint{3.580807in}{1.356910in}}{\pgfqpoint{3.577895in}{1.353998in}}%
\pgfpathcurveto{\pgfqpoint{3.574983in}{1.351086in}}{\pgfqpoint{3.573347in}{1.347136in}}{\pgfqpoint{3.573347in}{1.343018in}}%
\pgfpathcurveto{\pgfqpoint{3.573347in}{1.338900in}}{\pgfqpoint{3.574983in}{1.334950in}}{\pgfqpoint{3.577895in}{1.332038in}}%
\pgfpathcurveto{\pgfqpoint{3.580807in}{1.329126in}}{\pgfqpoint{3.584757in}{1.327490in}}{\pgfqpoint{3.588875in}{1.327490in}}%
\pgfpathclose%
\pgfusepath{fill}%
\end{pgfscope}%
\begin{pgfscope}%
\pgfpathrectangle{\pgfqpoint{0.887500in}{0.275000in}}{\pgfqpoint{4.225000in}{4.225000in}}%
\pgfusepath{clip}%
\pgfsetbuttcap%
\pgfsetroundjoin%
\definecolor{currentfill}{rgb}{0.000000,0.000000,0.000000}%
\pgfsetfillcolor{currentfill}%
\pgfsetfillopacity{0.800000}%
\pgfsetlinewidth{0.000000pt}%
\definecolor{currentstroke}{rgb}{0.000000,0.000000,0.000000}%
\pgfsetstrokecolor{currentstroke}%
\pgfsetstrokeopacity{0.800000}%
\pgfsetdash{}{0pt}%
\pgfpathmoveto{\pgfqpoint{2.383760in}{1.798291in}}%
\pgfpathcurveto{\pgfqpoint{2.387878in}{1.798291in}}{\pgfqpoint{2.391828in}{1.799927in}}{\pgfqpoint{2.394740in}{1.802839in}}%
\pgfpathcurveto{\pgfqpoint{2.397652in}{1.805751in}}{\pgfqpoint{2.399288in}{1.809701in}}{\pgfqpoint{2.399288in}{1.813819in}}%
\pgfpathcurveto{\pgfqpoint{2.399288in}{1.817937in}}{\pgfqpoint{2.397652in}{1.821887in}}{\pgfqpoint{2.394740in}{1.824799in}}%
\pgfpathcurveto{\pgfqpoint{2.391828in}{1.827711in}}{\pgfqpoint{2.387878in}{1.829347in}}{\pgfqpoint{2.383760in}{1.829347in}}%
\pgfpathcurveto{\pgfqpoint{2.379642in}{1.829347in}}{\pgfqpoint{2.375692in}{1.827711in}}{\pgfqpoint{2.372780in}{1.824799in}}%
\pgfpathcurveto{\pgfqpoint{2.369868in}{1.821887in}}{\pgfqpoint{2.368232in}{1.817937in}}{\pgfqpoint{2.368232in}{1.813819in}}%
\pgfpathcurveto{\pgfqpoint{2.368232in}{1.809701in}}{\pgfqpoint{2.369868in}{1.805751in}}{\pgfqpoint{2.372780in}{1.802839in}}%
\pgfpathcurveto{\pgfqpoint{2.375692in}{1.799927in}}{\pgfqpoint{2.379642in}{1.798291in}}{\pgfqpoint{2.383760in}{1.798291in}}%
\pgfpathclose%
\pgfusepath{fill}%
\end{pgfscope}%
\begin{pgfscope}%
\pgfpathrectangle{\pgfqpoint{0.887500in}{0.275000in}}{\pgfqpoint{4.225000in}{4.225000in}}%
\pgfusepath{clip}%
\pgfsetbuttcap%
\pgfsetroundjoin%
\definecolor{currentfill}{rgb}{0.000000,0.000000,0.000000}%
\pgfsetfillcolor{currentfill}%
\pgfsetfillopacity{0.800000}%
\pgfsetlinewidth{0.000000pt}%
\definecolor{currentstroke}{rgb}{0.000000,0.000000,0.000000}%
\pgfsetstrokecolor{currentstroke}%
\pgfsetstrokeopacity{0.800000}%
\pgfsetdash{}{0pt}%
\pgfpathmoveto{\pgfqpoint{2.565138in}{1.726237in}}%
\pgfpathcurveto{\pgfqpoint{2.569256in}{1.726237in}}{\pgfqpoint{2.573206in}{1.727874in}}{\pgfqpoint{2.576118in}{1.730786in}}%
\pgfpathcurveto{\pgfqpoint{2.579030in}{1.733698in}}{\pgfqpoint{2.580666in}{1.737648in}}{\pgfqpoint{2.580666in}{1.741766in}}%
\pgfpathcurveto{\pgfqpoint{2.580666in}{1.745884in}}{\pgfqpoint{2.579030in}{1.749834in}}{\pgfqpoint{2.576118in}{1.752746in}}%
\pgfpathcurveto{\pgfqpoint{2.573206in}{1.755658in}}{\pgfqpoint{2.569256in}{1.757294in}}{\pgfqpoint{2.565138in}{1.757294in}}%
\pgfpathcurveto{\pgfqpoint{2.561020in}{1.757294in}}{\pgfqpoint{2.557070in}{1.755658in}}{\pgfqpoint{2.554158in}{1.752746in}}%
\pgfpathcurveto{\pgfqpoint{2.551246in}{1.749834in}}{\pgfqpoint{2.549610in}{1.745884in}}{\pgfqpoint{2.549610in}{1.741766in}}%
\pgfpathcurveto{\pgfqpoint{2.549610in}{1.737648in}}{\pgfqpoint{2.551246in}{1.733698in}}{\pgfqpoint{2.554158in}{1.730786in}}%
\pgfpathcurveto{\pgfqpoint{2.557070in}{1.727874in}}{\pgfqpoint{2.561020in}{1.726237in}}{\pgfqpoint{2.565138in}{1.726237in}}%
\pgfpathclose%
\pgfusepath{fill}%
\end{pgfscope}%
\begin{pgfscope}%
\pgfpathrectangle{\pgfqpoint{0.887500in}{0.275000in}}{\pgfqpoint{4.225000in}{4.225000in}}%
\pgfusepath{clip}%
\pgfsetbuttcap%
\pgfsetroundjoin%
\definecolor{currentfill}{rgb}{0.000000,0.000000,0.000000}%
\pgfsetfillcolor{currentfill}%
\pgfsetfillopacity{0.800000}%
\pgfsetlinewidth{0.000000pt}%
\definecolor{currentstroke}{rgb}{0.000000,0.000000,0.000000}%
\pgfsetstrokecolor{currentstroke}%
\pgfsetstrokeopacity{0.800000}%
\pgfsetdash{}{0pt}%
\pgfpathmoveto{\pgfqpoint{1.904171in}{1.893826in}}%
\pgfpathcurveto{\pgfqpoint{1.908289in}{1.893826in}}{\pgfqpoint{1.912239in}{1.895462in}}{\pgfqpoint{1.915151in}{1.898374in}}%
\pgfpathcurveto{\pgfqpoint{1.918063in}{1.901286in}}{\pgfqpoint{1.919699in}{1.905236in}}{\pgfqpoint{1.919699in}{1.909354in}}%
\pgfpathcurveto{\pgfqpoint{1.919699in}{1.913472in}}{\pgfqpoint{1.918063in}{1.917422in}}{\pgfqpoint{1.915151in}{1.920334in}}%
\pgfpathcurveto{\pgfqpoint{1.912239in}{1.923246in}}{\pgfqpoint{1.908289in}{1.924882in}}{\pgfqpoint{1.904171in}{1.924882in}}%
\pgfpathcurveto{\pgfqpoint{1.900053in}{1.924882in}}{\pgfqpoint{1.896103in}{1.923246in}}{\pgfqpoint{1.893191in}{1.920334in}}%
\pgfpathcurveto{\pgfqpoint{1.890279in}{1.917422in}}{\pgfqpoint{1.888642in}{1.913472in}}{\pgfqpoint{1.888642in}{1.909354in}}%
\pgfpathcurveto{\pgfqpoint{1.888642in}{1.905236in}}{\pgfqpoint{1.890279in}{1.901286in}}{\pgfqpoint{1.893191in}{1.898374in}}%
\pgfpathcurveto{\pgfqpoint{1.896103in}{1.895462in}}{\pgfqpoint{1.900053in}{1.893826in}}{\pgfqpoint{1.904171in}{1.893826in}}%
\pgfpathclose%
\pgfusepath{fill}%
\end{pgfscope}%
\begin{pgfscope}%
\pgfpathrectangle{\pgfqpoint{0.887500in}{0.275000in}}{\pgfqpoint{4.225000in}{4.225000in}}%
\pgfusepath{clip}%
\pgfsetbuttcap%
\pgfsetroundjoin%
\definecolor{currentfill}{rgb}{0.000000,0.000000,0.000000}%
\pgfsetfillcolor{currentfill}%
\pgfsetfillopacity{0.800000}%
\pgfsetlinewidth{0.000000pt}%
\definecolor{currentstroke}{rgb}{0.000000,0.000000,0.000000}%
\pgfsetstrokecolor{currentstroke}%
\pgfsetstrokeopacity{0.800000}%
\pgfsetdash{}{0pt}%
\pgfpathmoveto{\pgfqpoint{3.656240in}{1.154204in}}%
\pgfpathcurveto{\pgfqpoint{3.660358in}{1.154204in}}{\pgfqpoint{3.664308in}{1.155841in}}{\pgfqpoint{3.667220in}{1.158753in}}%
\pgfpathcurveto{\pgfqpoint{3.670132in}{1.161664in}}{\pgfqpoint{3.671768in}{1.165615in}}{\pgfqpoint{3.671768in}{1.169733in}}%
\pgfpathcurveto{\pgfqpoint{3.671768in}{1.173851in}}{\pgfqpoint{3.670132in}{1.177801in}}{\pgfqpoint{3.667220in}{1.180713in}}%
\pgfpathcurveto{\pgfqpoint{3.664308in}{1.183625in}}{\pgfqpoint{3.660358in}{1.185261in}}{\pgfqpoint{3.656240in}{1.185261in}}%
\pgfpathcurveto{\pgfqpoint{3.652122in}{1.185261in}}{\pgfqpoint{3.648172in}{1.183625in}}{\pgfqpoint{3.645260in}{1.180713in}}%
\pgfpathcurveto{\pgfqpoint{3.642348in}{1.177801in}}{\pgfqpoint{3.640711in}{1.173851in}}{\pgfqpoint{3.640711in}{1.169733in}}%
\pgfpathcurveto{\pgfqpoint{3.640711in}{1.165615in}}{\pgfqpoint{3.642348in}{1.161664in}}{\pgfqpoint{3.645260in}{1.158753in}}%
\pgfpathcurveto{\pgfqpoint{3.648172in}{1.155841in}}{\pgfqpoint{3.652122in}{1.154204in}}{\pgfqpoint{3.656240in}{1.154204in}}%
\pgfpathclose%
\pgfusepath{fill}%
\end{pgfscope}%
\begin{pgfscope}%
\pgfpathrectangle{\pgfqpoint{0.887500in}{0.275000in}}{\pgfqpoint{4.225000in}{4.225000in}}%
\pgfusepath{clip}%
\pgfsetbuttcap%
\pgfsetroundjoin%
\definecolor{currentfill}{rgb}{0.000000,0.000000,0.000000}%
\pgfsetfillcolor{currentfill}%
\pgfsetfillopacity{0.800000}%
\pgfsetlinewidth{0.000000pt}%
\definecolor{currentstroke}{rgb}{0.000000,0.000000,0.000000}%
\pgfsetstrokecolor{currentstroke}%
\pgfsetstrokeopacity{0.800000}%
\pgfsetdash{}{0pt}%
\pgfpathmoveto{\pgfqpoint{2.746786in}{1.650046in}}%
\pgfpathcurveto{\pgfqpoint{2.750904in}{1.650046in}}{\pgfqpoint{2.754854in}{1.651682in}}{\pgfqpoint{2.757766in}{1.654594in}}%
\pgfpathcurveto{\pgfqpoint{2.760678in}{1.657506in}}{\pgfqpoint{2.762314in}{1.661456in}}{\pgfqpoint{2.762314in}{1.665574in}}%
\pgfpathcurveto{\pgfqpoint{2.762314in}{1.669692in}}{\pgfqpoint{2.760678in}{1.673642in}}{\pgfqpoint{2.757766in}{1.676554in}}%
\pgfpathcurveto{\pgfqpoint{2.754854in}{1.679466in}}{\pgfqpoint{2.750904in}{1.681102in}}{\pgfqpoint{2.746786in}{1.681102in}}%
\pgfpathcurveto{\pgfqpoint{2.742667in}{1.681102in}}{\pgfqpoint{2.738717in}{1.679466in}}{\pgfqpoint{2.735806in}{1.676554in}}%
\pgfpathcurveto{\pgfqpoint{2.732894in}{1.673642in}}{\pgfqpoint{2.731257in}{1.669692in}}{\pgfqpoint{2.731257in}{1.665574in}}%
\pgfpathcurveto{\pgfqpoint{2.731257in}{1.661456in}}{\pgfqpoint{2.732894in}{1.657506in}}{\pgfqpoint{2.735806in}{1.654594in}}%
\pgfpathcurveto{\pgfqpoint{2.738717in}{1.651682in}}{\pgfqpoint{2.742667in}{1.650046in}}{\pgfqpoint{2.746786in}{1.650046in}}%
\pgfpathclose%
\pgfusepath{fill}%
\end{pgfscope}%
\begin{pgfscope}%
\pgfpathrectangle{\pgfqpoint{0.887500in}{0.275000in}}{\pgfqpoint{4.225000in}{4.225000in}}%
\pgfusepath{clip}%
\pgfsetbuttcap%
\pgfsetroundjoin%
\definecolor{currentfill}{rgb}{0.000000,0.000000,0.000000}%
\pgfsetfillcolor{currentfill}%
\pgfsetfillopacity{0.800000}%
\pgfsetlinewidth{0.000000pt}%
\definecolor{currentstroke}{rgb}{0.000000,0.000000,0.000000}%
\pgfsetstrokecolor{currentstroke}%
\pgfsetstrokeopacity{0.800000}%
\pgfsetdash{}{0pt}%
\pgfpathmoveto{\pgfqpoint{3.474621in}{1.275438in}}%
\pgfpathcurveto{\pgfqpoint{3.478739in}{1.275438in}}{\pgfqpoint{3.482689in}{1.277074in}}{\pgfqpoint{3.485601in}{1.279986in}}%
\pgfpathcurveto{\pgfqpoint{3.488513in}{1.282898in}}{\pgfqpoint{3.490149in}{1.286848in}}{\pgfqpoint{3.490149in}{1.290967in}}%
\pgfpathcurveto{\pgfqpoint{3.490149in}{1.295085in}}{\pgfqpoint{3.488513in}{1.299035in}}{\pgfqpoint{3.485601in}{1.301947in}}%
\pgfpathcurveto{\pgfqpoint{3.482689in}{1.304859in}}{\pgfqpoint{3.478739in}{1.306495in}}{\pgfqpoint{3.474621in}{1.306495in}}%
\pgfpathcurveto{\pgfqpoint{3.470503in}{1.306495in}}{\pgfqpoint{3.466553in}{1.304859in}}{\pgfqpoint{3.463641in}{1.301947in}}%
\pgfpathcurveto{\pgfqpoint{3.460729in}{1.299035in}}{\pgfqpoint{3.459093in}{1.295085in}}{\pgfqpoint{3.459093in}{1.290967in}}%
\pgfpathcurveto{\pgfqpoint{3.459093in}{1.286848in}}{\pgfqpoint{3.460729in}{1.282898in}}{\pgfqpoint{3.463641in}{1.279986in}}%
\pgfpathcurveto{\pgfqpoint{3.466553in}{1.277074in}}{\pgfqpoint{3.470503in}{1.275438in}}{\pgfqpoint{3.474621in}{1.275438in}}%
\pgfpathclose%
\pgfusepath{fill}%
\end{pgfscope}%
\begin{pgfscope}%
\pgfpathrectangle{\pgfqpoint{0.887500in}{0.275000in}}{\pgfqpoint{4.225000in}{4.225000in}}%
\pgfusepath{clip}%
\pgfsetbuttcap%
\pgfsetroundjoin%
\definecolor{currentfill}{rgb}{0.000000,0.000000,0.000000}%
\pgfsetfillcolor{currentfill}%
\pgfsetfillopacity{0.800000}%
\pgfsetlinewidth{0.000000pt}%
\definecolor{currentstroke}{rgb}{0.000000,0.000000,0.000000}%
\pgfsetstrokecolor{currentstroke}%
\pgfsetstrokeopacity{0.800000}%
\pgfsetdash{}{0pt}%
\pgfpathmoveto{\pgfqpoint{2.928648in}{1.568959in}}%
\pgfpathcurveto{\pgfqpoint{2.932766in}{1.568959in}}{\pgfqpoint{2.936716in}{1.570595in}}{\pgfqpoint{2.939628in}{1.573507in}}%
\pgfpathcurveto{\pgfqpoint{2.942540in}{1.576419in}}{\pgfqpoint{2.944177in}{1.580369in}}{\pgfqpoint{2.944177in}{1.584487in}}%
\pgfpathcurveto{\pgfqpoint{2.944177in}{1.588606in}}{\pgfqpoint{2.942540in}{1.592556in}}{\pgfqpoint{2.939628in}{1.595468in}}%
\pgfpathcurveto{\pgfqpoint{2.936716in}{1.598380in}}{\pgfqpoint{2.932766in}{1.600016in}}{\pgfqpoint{2.928648in}{1.600016in}}%
\pgfpathcurveto{\pgfqpoint{2.924530in}{1.600016in}}{\pgfqpoint{2.920580in}{1.598380in}}{\pgfqpoint{2.917668in}{1.595468in}}%
\pgfpathcurveto{\pgfqpoint{2.914756in}{1.592556in}}{\pgfqpoint{2.913120in}{1.588606in}}{\pgfqpoint{2.913120in}{1.584487in}}%
\pgfpathcurveto{\pgfqpoint{2.913120in}{1.580369in}}{\pgfqpoint{2.914756in}{1.576419in}}{\pgfqpoint{2.917668in}{1.573507in}}%
\pgfpathcurveto{\pgfqpoint{2.920580in}{1.570595in}}{\pgfqpoint{2.924530in}{1.568959in}}{\pgfqpoint{2.928648in}{1.568959in}}%
\pgfpathclose%
\pgfusepath{fill}%
\end{pgfscope}%
\begin{pgfscope}%
\pgfpathrectangle{\pgfqpoint{0.887500in}{0.275000in}}{\pgfqpoint{4.225000in}{4.225000in}}%
\pgfusepath{clip}%
\pgfsetbuttcap%
\pgfsetroundjoin%
\definecolor{currentfill}{rgb}{0.000000,0.000000,0.000000}%
\pgfsetfillcolor{currentfill}%
\pgfsetfillopacity{0.800000}%
\pgfsetlinewidth{0.000000pt}%
\definecolor{currentstroke}{rgb}{0.000000,0.000000,0.000000}%
\pgfsetstrokecolor{currentstroke}%
\pgfsetstrokeopacity{0.800000}%
\pgfsetdash{}{0pt}%
\pgfpathmoveto{\pgfqpoint{2.085242in}{1.827825in}}%
\pgfpathcurveto{\pgfqpoint{2.089361in}{1.827825in}}{\pgfqpoint{2.093311in}{1.829461in}}{\pgfqpoint{2.096223in}{1.832373in}}%
\pgfpathcurveto{\pgfqpoint{2.099135in}{1.835285in}}{\pgfqpoint{2.100771in}{1.839235in}}{\pgfqpoint{2.100771in}{1.843353in}}%
\pgfpathcurveto{\pgfqpoint{2.100771in}{1.847471in}}{\pgfqpoint{2.099135in}{1.851421in}}{\pgfqpoint{2.096223in}{1.854333in}}%
\pgfpathcurveto{\pgfqpoint{2.093311in}{1.857245in}}{\pgfqpoint{2.089361in}{1.858881in}}{\pgfqpoint{2.085242in}{1.858881in}}%
\pgfpathcurveto{\pgfqpoint{2.081124in}{1.858881in}}{\pgfqpoint{2.077174in}{1.857245in}}{\pgfqpoint{2.074262in}{1.854333in}}%
\pgfpathcurveto{\pgfqpoint{2.071350in}{1.851421in}}{\pgfqpoint{2.069714in}{1.847471in}}{\pgfqpoint{2.069714in}{1.843353in}}%
\pgfpathcurveto{\pgfqpoint{2.069714in}{1.839235in}}{\pgfqpoint{2.071350in}{1.835285in}}{\pgfqpoint{2.074262in}{1.832373in}}%
\pgfpathcurveto{\pgfqpoint{2.077174in}{1.829461in}}{\pgfqpoint{2.081124in}{1.827825in}}{\pgfqpoint{2.085242in}{1.827825in}}%
\pgfpathclose%
\pgfusepath{fill}%
\end{pgfscope}%
\begin{pgfscope}%
\pgfpathrectangle{\pgfqpoint{0.887500in}{0.275000in}}{\pgfqpoint{4.225000in}{4.225000in}}%
\pgfusepath{clip}%
\pgfsetbuttcap%
\pgfsetroundjoin%
\definecolor{currentfill}{rgb}{0.000000,0.000000,0.000000}%
\pgfsetfillcolor{currentfill}%
\pgfsetfillopacity{0.800000}%
\pgfsetlinewidth{0.000000pt}%
\definecolor{currentstroke}{rgb}{0.000000,0.000000,0.000000}%
\pgfsetstrokecolor{currentstroke}%
\pgfsetstrokeopacity{0.800000}%
\pgfsetdash{}{0pt}%
\pgfpathmoveto{\pgfqpoint{3.110657in}{1.481545in}}%
\pgfpathcurveto{\pgfqpoint{3.114775in}{1.481545in}}{\pgfqpoint{3.118725in}{1.483181in}}{\pgfqpoint{3.121637in}{1.486093in}}%
\pgfpathcurveto{\pgfqpoint{3.124549in}{1.489005in}}{\pgfqpoint{3.126185in}{1.492955in}}{\pgfqpoint{3.126185in}{1.497073in}}%
\pgfpathcurveto{\pgfqpoint{3.126185in}{1.501191in}}{\pgfqpoint{3.124549in}{1.505141in}}{\pgfqpoint{3.121637in}{1.508053in}}%
\pgfpathcurveto{\pgfqpoint{3.118725in}{1.510965in}}{\pgfqpoint{3.114775in}{1.512601in}}{\pgfqpoint{3.110657in}{1.512601in}}%
\pgfpathcurveto{\pgfqpoint{3.106538in}{1.512601in}}{\pgfqpoint{3.102588in}{1.510965in}}{\pgfqpoint{3.099676in}{1.508053in}}%
\pgfpathcurveto{\pgfqpoint{3.096764in}{1.505141in}}{\pgfqpoint{3.095128in}{1.501191in}}{\pgfqpoint{3.095128in}{1.497073in}}%
\pgfpathcurveto{\pgfqpoint{3.095128in}{1.492955in}}{\pgfqpoint{3.096764in}{1.489005in}}{\pgfqpoint{3.099676in}{1.486093in}}%
\pgfpathcurveto{\pgfqpoint{3.102588in}{1.483181in}}{\pgfqpoint{3.106538in}{1.481545in}}{\pgfqpoint{3.110657in}{1.481545in}}%
\pgfpathclose%
\pgfusepath{fill}%
\end{pgfscope}%
\begin{pgfscope}%
\pgfpathrectangle{\pgfqpoint{0.887500in}{0.275000in}}{\pgfqpoint{4.225000in}{4.225000in}}%
\pgfusepath{clip}%
\pgfsetbuttcap%
\pgfsetroundjoin%
\definecolor{currentfill}{rgb}{0.000000,0.000000,0.000000}%
\pgfsetfillcolor{currentfill}%
\pgfsetfillopacity{0.800000}%
\pgfsetlinewidth{0.000000pt}%
\definecolor{currentstroke}{rgb}{0.000000,0.000000,0.000000}%
\pgfsetstrokecolor{currentstroke}%
\pgfsetstrokeopacity{0.800000}%
\pgfsetdash{}{0pt}%
\pgfpathmoveto{\pgfqpoint{3.292706in}{1.385274in}}%
\pgfpathcurveto{\pgfqpoint{3.296824in}{1.385274in}}{\pgfqpoint{3.300775in}{1.386910in}}{\pgfqpoint{3.303686in}{1.389822in}}%
\pgfpathcurveto{\pgfqpoint{3.306598in}{1.392734in}}{\pgfqpoint{3.308235in}{1.396684in}}{\pgfqpoint{3.308235in}{1.400802in}}%
\pgfpathcurveto{\pgfqpoint{3.308235in}{1.404921in}}{\pgfqpoint{3.306598in}{1.408871in}}{\pgfqpoint{3.303686in}{1.411783in}}%
\pgfpathcurveto{\pgfqpoint{3.300775in}{1.414695in}}{\pgfqpoint{3.296824in}{1.416331in}}{\pgfqpoint{3.292706in}{1.416331in}}%
\pgfpathcurveto{\pgfqpoint{3.288588in}{1.416331in}}{\pgfqpoint{3.284638in}{1.414695in}}{\pgfqpoint{3.281726in}{1.411783in}}%
\pgfpathcurveto{\pgfqpoint{3.278814in}{1.408871in}}{\pgfqpoint{3.277178in}{1.404921in}}{\pgfqpoint{3.277178in}{1.400802in}}%
\pgfpathcurveto{\pgfqpoint{3.277178in}{1.396684in}}{\pgfqpoint{3.278814in}{1.392734in}}{\pgfqpoint{3.281726in}{1.389822in}}%
\pgfpathcurveto{\pgfqpoint{3.284638in}{1.386910in}}{\pgfqpoint{3.288588in}{1.385274in}}{\pgfqpoint{3.292706in}{1.385274in}}%
\pgfpathclose%
\pgfusepath{fill}%
\end{pgfscope}%
\begin{pgfscope}%
\pgfpathrectangle{\pgfqpoint{0.887500in}{0.275000in}}{\pgfqpoint{4.225000in}{4.225000in}}%
\pgfusepath{clip}%
\pgfsetbuttcap%
\pgfsetroundjoin%
\definecolor{currentfill}{rgb}{0.000000,0.000000,0.000000}%
\pgfsetfillcolor{currentfill}%
\pgfsetfillopacity{0.800000}%
\pgfsetlinewidth{0.000000pt}%
\definecolor{currentstroke}{rgb}{0.000000,0.000000,0.000000}%
\pgfsetstrokecolor{currentstroke}%
\pgfsetstrokeopacity{0.800000}%
\pgfsetdash{}{0pt}%
\pgfpathmoveto{\pgfqpoint{2.266670in}{1.758607in}}%
\pgfpathcurveto{\pgfqpoint{2.270788in}{1.758607in}}{\pgfqpoint{2.274738in}{1.760243in}}{\pgfqpoint{2.277650in}{1.763155in}}%
\pgfpathcurveto{\pgfqpoint{2.280562in}{1.766067in}}{\pgfqpoint{2.282199in}{1.770017in}}{\pgfqpoint{2.282199in}{1.774135in}}%
\pgfpathcurveto{\pgfqpoint{2.282199in}{1.778253in}}{\pgfqpoint{2.280562in}{1.782203in}}{\pgfqpoint{2.277650in}{1.785115in}}%
\pgfpathcurveto{\pgfqpoint{2.274738in}{1.788027in}}{\pgfqpoint{2.270788in}{1.789663in}}{\pgfqpoint{2.266670in}{1.789663in}}%
\pgfpathcurveto{\pgfqpoint{2.262552in}{1.789663in}}{\pgfqpoint{2.258602in}{1.788027in}}{\pgfqpoint{2.255690in}{1.785115in}}%
\pgfpathcurveto{\pgfqpoint{2.252778in}{1.782203in}}{\pgfqpoint{2.251142in}{1.778253in}}{\pgfqpoint{2.251142in}{1.774135in}}%
\pgfpathcurveto{\pgfqpoint{2.251142in}{1.770017in}}{\pgfqpoint{2.252778in}{1.766067in}}{\pgfqpoint{2.255690in}{1.763155in}}%
\pgfpathcurveto{\pgfqpoint{2.258602in}{1.760243in}}{\pgfqpoint{2.262552in}{1.758607in}}{\pgfqpoint{2.266670in}{1.758607in}}%
\pgfpathclose%
\pgfusepath{fill}%
\end{pgfscope}%
\begin{pgfscope}%
\pgfpathrectangle{\pgfqpoint{0.887500in}{0.275000in}}{\pgfqpoint{4.225000in}{4.225000in}}%
\pgfusepath{clip}%
\pgfsetbuttcap%
\pgfsetroundjoin%
\definecolor{currentfill}{rgb}{0.000000,0.000000,0.000000}%
\pgfsetfillcolor{currentfill}%
\pgfsetfillopacity{0.800000}%
\pgfsetlinewidth{0.000000pt}%
\definecolor{currentstroke}{rgb}{0.000000,0.000000,0.000000}%
\pgfsetstrokecolor{currentstroke}%
\pgfsetstrokeopacity{0.800000}%
\pgfsetdash{}{0pt}%
\pgfpathmoveto{\pgfqpoint{2.448411in}{1.686181in}}%
\pgfpathcurveto{\pgfqpoint{2.452530in}{1.686181in}}{\pgfqpoint{2.456480in}{1.687818in}}{\pgfqpoint{2.459392in}{1.690730in}}%
\pgfpathcurveto{\pgfqpoint{2.462304in}{1.693642in}}{\pgfqpoint{2.463940in}{1.697592in}}{\pgfqpoint{2.463940in}{1.701710in}}%
\pgfpathcurveto{\pgfqpoint{2.463940in}{1.705828in}}{\pgfqpoint{2.462304in}{1.709778in}}{\pgfqpoint{2.459392in}{1.712690in}}%
\pgfpathcurveto{\pgfqpoint{2.456480in}{1.715602in}}{\pgfqpoint{2.452530in}{1.717238in}}{\pgfqpoint{2.448411in}{1.717238in}}%
\pgfpathcurveto{\pgfqpoint{2.444293in}{1.717238in}}{\pgfqpoint{2.440343in}{1.715602in}}{\pgfqpoint{2.437431in}{1.712690in}}%
\pgfpathcurveto{\pgfqpoint{2.434519in}{1.709778in}}{\pgfqpoint{2.432883in}{1.705828in}}{\pgfqpoint{2.432883in}{1.701710in}}%
\pgfpathcurveto{\pgfqpoint{2.432883in}{1.697592in}}{\pgfqpoint{2.434519in}{1.693642in}}{\pgfqpoint{2.437431in}{1.690730in}}%
\pgfpathcurveto{\pgfqpoint{2.440343in}{1.687818in}}{\pgfqpoint{2.444293in}{1.686181in}}{\pgfqpoint{2.448411in}{1.686181in}}%
\pgfpathclose%
\pgfusepath{fill}%
\end{pgfscope}%
\begin{pgfscope}%
\pgfpathrectangle{\pgfqpoint{0.887500in}{0.275000in}}{\pgfqpoint{4.225000in}{4.225000in}}%
\pgfusepath{clip}%
\pgfsetbuttcap%
\pgfsetroundjoin%
\definecolor{currentfill}{rgb}{0.000000,0.000000,0.000000}%
\pgfsetfillcolor{currentfill}%
\pgfsetfillopacity{0.800000}%
\pgfsetlinewidth{0.000000pt}%
\definecolor{currentstroke}{rgb}{0.000000,0.000000,0.000000}%
\pgfsetstrokecolor{currentstroke}%
\pgfsetstrokeopacity{0.800000}%
\pgfsetdash{}{0pt}%
\pgfpathmoveto{\pgfqpoint{3.359752in}{1.206623in}}%
\pgfpathcurveto{\pgfqpoint{3.363871in}{1.206623in}}{\pgfqpoint{3.367821in}{1.208260in}}{\pgfqpoint{3.370733in}{1.211172in}}%
\pgfpathcurveto{\pgfqpoint{3.373645in}{1.214083in}}{\pgfqpoint{3.375281in}{1.218034in}}{\pgfqpoint{3.375281in}{1.222152in}}%
\pgfpathcurveto{\pgfqpoint{3.375281in}{1.226270in}}{\pgfqpoint{3.373645in}{1.230220in}}{\pgfqpoint{3.370733in}{1.233132in}}%
\pgfpathcurveto{\pgfqpoint{3.367821in}{1.236044in}}{\pgfqpoint{3.363871in}{1.237680in}}{\pgfqpoint{3.359752in}{1.237680in}}%
\pgfpathcurveto{\pgfqpoint{3.355634in}{1.237680in}}{\pgfqpoint{3.351684in}{1.236044in}}{\pgfqpoint{3.348772in}{1.233132in}}%
\pgfpathcurveto{\pgfqpoint{3.345860in}{1.230220in}}{\pgfqpoint{3.344224in}{1.226270in}}{\pgfqpoint{3.344224in}{1.222152in}}%
\pgfpathcurveto{\pgfqpoint{3.344224in}{1.218034in}}{\pgfqpoint{3.345860in}{1.214083in}}{\pgfqpoint{3.348772in}{1.211172in}}%
\pgfpathcurveto{\pgfqpoint{3.351684in}{1.208260in}}{\pgfqpoint{3.355634in}{1.206623in}}{\pgfqpoint{3.359752in}{1.206623in}}%
\pgfpathclose%
\pgfusepath{fill}%
\end{pgfscope}%
\begin{pgfscope}%
\pgfpathrectangle{\pgfqpoint{0.887500in}{0.275000in}}{\pgfqpoint{4.225000in}{4.225000in}}%
\pgfusepath{clip}%
\pgfsetbuttcap%
\pgfsetroundjoin%
\definecolor{currentfill}{rgb}{0.000000,0.000000,0.000000}%
\pgfsetfillcolor{currentfill}%
\pgfsetfillopacity{0.800000}%
\pgfsetlinewidth{0.000000pt}%
\definecolor{currentstroke}{rgb}{0.000000,0.000000,0.000000}%
\pgfsetstrokecolor{currentstroke}%
\pgfsetstrokeopacity{0.800000}%
\pgfsetdash{}{0pt}%
\pgfpathmoveto{\pgfqpoint{3.542117in}{1.114291in}}%
\pgfpathcurveto{\pgfqpoint{3.546235in}{1.114291in}}{\pgfqpoint{3.550185in}{1.115927in}}{\pgfqpoint{3.553097in}{1.118839in}}%
\pgfpathcurveto{\pgfqpoint{3.556009in}{1.121751in}}{\pgfqpoint{3.557645in}{1.125701in}}{\pgfqpoint{3.557645in}{1.129819in}}%
\pgfpathcurveto{\pgfqpoint{3.557645in}{1.133937in}}{\pgfqpoint{3.556009in}{1.137887in}}{\pgfqpoint{3.553097in}{1.140799in}}%
\pgfpathcurveto{\pgfqpoint{3.550185in}{1.143711in}}{\pgfqpoint{3.546235in}{1.145347in}}{\pgfqpoint{3.542117in}{1.145347in}}%
\pgfpathcurveto{\pgfqpoint{3.537998in}{1.145347in}}{\pgfqpoint{3.534048in}{1.143711in}}{\pgfqpoint{3.531136in}{1.140799in}}%
\pgfpathcurveto{\pgfqpoint{3.528224in}{1.137887in}}{\pgfqpoint{3.526588in}{1.133937in}}{\pgfqpoint{3.526588in}{1.129819in}}%
\pgfpathcurveto{\pgfqpoint{3.526588in}{1.125701in}}{\pgfqpoint{3.528224in}{1.121751in}}{\pgfqpoint{3.531136in}{1.118839in}}%
\pgfpathcurveto{\pgfqpoint{3.534048in}{1.115927in}}{\pgfqpoint{3.537998in}{1.114291in}}{\pgfqpoint{3.542117in}{1.114291in}}%
\pgfpathclose%
\pgfusepath{fill}%
\end{pgfscope}%
\begin{pgfscope}%
\pgfpathrectangle{\pgfqpoint{0.887500in}{0.275000in}}{\pgfqpoint{4.225000in}{4.225000in}}%
\pgfusepath{clip}%
\pgfsetbuttcap%
\pgfsetroundjoin%
\definecolor{currentfill}{rgb}{0.000000,0.000000,0.000000}%
\pgfsetfillcolor{currentfill}%
\pgfsetfillopacity{0.800000}%
\pgfsetlinewidth{0.000000pt}%
\definecolor{currentstroke}{rgb}{0.000000,0.000000,0.000000}%
\pgfsetstrokecolor{currentstroke}%
\pgfsetstrokeopacity{0.800000}%
\pgfsetdash{}{0pt}%
\pgfpathmoveto{\pgfqpoint{2.630431in}{1.609784in}}%
\pgfpathcurveto{\pgfqpoint{2.634550in}{1.609784in}}{\pgfqpoint{2.638500in}{1.611421in}}{\pgfqpoint{2.641412in}{1.614333in}}%
\pgfpathcurveto{\pgfqpoint{2.644324in}{1.617244in}}{\pgfqpoint{2.645960in}{1.621195in}}{\pgfqpoint{2.645960in}{1.625313in}}%
\pgfpathcurveto{\pgfqpoint{2.645960in}{1.629431in}}{\pgfqpoint{2.644324in}{1.633381in}}{\pgfqpoint{2.641412in}{1.636293in}}%
\pgfpathcurveto{\pgfqpoint{2.638500in}{1.639205in}}{\pgfqpoint{2.634550in}{1.640841in}}{\pgfqpoint{2.630431in}{1.640841in}}%
\pgfpathcurveto{\pgfqpoint{2.626313in}{1.640841in}}{\pgfqpoint{2.622363in}{1.639205in}}{\pgfqpoint{2.619451in}{1.636293in}}%
\pgfpathcurveto{\pgfqpoint{2.616539in}{1.633381in}}{\pgfqpoint{2.614903in}{1.629431in}}{\pgfqpoint{2.614903in}{1.625313in}}%
\pgfpathcurveto{\pgfqpoint{2.614903in}{1.621195in}}{\pgfqpoint{2.616539in}{1.617244in}}{\pgfqpoint{2.619451in}{1.614333in}}%
\pgfpathcurveto{\pgfqpoint{2.622363in}{1.611421in}}{\pgfqpoint{2.626313in}{1.609784in}}{\pgfqpoint{2.630431in}{1.609784in}}%
\pgfpathclose%
\pgfusepath{fill}%
\end{pgfscope}%
\begin{pgfscope}%
\pgfpathrectangle{\pgfqpoint{0.887500in}{0.275000in}}{\pgfqpoint{4.225000in}{4.225000in}}%
\pgfusepath{clip}%
\pgfsetbuttcap%
\pgfsetroundjoin%
\definecolor{currentfill}{rgb}{0.000000,0.000000,0.000000}%
\pgfsetfillcolor{currentfill}%
\pgfsetfillopacity{0.800000}%
\pgfsetlinewidth{0.000000pt}%
\definecolor{currentstroke}{rgb}{0.000000,0.000000,0.000000}%
\pgfsetstrokecolor{currentstroke}%
\pgfsetstrokeopacity{0.800000}%
\pgfsetdash{}{0pt}%
\pgfpathmoveto{\pgfqpoint{2.812680in}{1.528744in}}%
\pgfpathcurveto{\pgfqpoint{2.816799in}{1.528744in}}{\pgfqpoint{2.820749in}{1.530380in}}{\pgfqpoint{2.823661in}{1.533292in}}%
\pgfpathcurveto{\pgfqpoint{2.826573in}{1.536204in}}{\pgfqpoint{2.828209in}{1.540154in}}{\pgfqpoint{2.828209in}{1.544272in}}%
\pgfpathcurveto{\pgfqpoint{2.828209in}{1.548390in}}{\pgfqpoint{2.826573in}{1.552340in}}{\pgfqpoint{2.823661in}{1.555252in}}%
\pgfpathcurveto{\pgfqpoint{2.820749in}{1.558164in}}{\pgfqpoint{2.816799in}{1.559800in}}{\pgfqpoint{2.812680in}{1.559800in}}%
\pgfpathcurveto{\pgfqpoint{2.808562in}{1.559800in}}{\pgfqpoint{2.804612in}{1.558164in}}{\pgfqpoint{2.801700in}{1.555252in}}%
\pgfpathcurveto{\pgfqpoint{2.798788in}{1.552340in}}{\pgfqpoint{2.797152in}{1.548390in}}{\pgfqpoint{2.797152in}{1.544272in}}%
\pgfpathcurveto{\pgfqpoint{2.797152in}{1.540154in}}{\pgfqpoint{2.798788in}{1.536204in}}{\pgfqpoint{2.801700in}{1.533292in}}%
\pgfpathcurveto{\pgfqpoint{2.804612in}{1.530380in}}{\pgfqpoint{2.808562in}{1.528744in}}{\pgfqpoint{2.812680in}{1.528744in}}%
\pgfpathclose%
\pgfusepath{fill}%
\end{pgfscope}%
\begin{pgfscope}%
\pgfpathrectangle{\pgfqpoint{0.887500in}{0.275000in}}{\pgfqpoint{4.225000in}{4.225000in}}%
\pgfusepath{clip}%
\pgfsetbuttcap%
\pgfsetroundjoin%
\definecolor{currentfill}{rgb}{0.000000,0.000000,0.000000}%
\pgfsetfillcolor{currentfill}%
\pgfsetfillopacity{0.800000}%
\pgfsetlinewidth{0.000000pt}%
\definecolor{currentstroke}{rgb}{0.000000,0.000000,0.000000}%
\pgfsetstrokecolor{currentstroke}%
\pgfsetstrokeopacity{0.800000}%
\pgfsetdash{}{0pt}%
\pgfpathmoveto{\pgfqpoint{1.967232in}{1.788318in}}%
\pgfpathcurveto{\pgfqpoint{1.971350in}{1.788318in}}{\pgfqpoint{1.975300in}{1.789955in}}{\pgfqpoint{1.978212in}{1.792867in}}%
\pgfpathcurveto{\pgfqpoint{1.981124in}{1.795778in}}{\pgfqpoint{1.982760in}{1.799729in}}{\pgfqpoint{1.982760in}{1.803847in}}%
\pgfpathcurveto{\pgfqpoint{1.982760in}{1.807965in}}{\pgfqpoint{1.981124in}{1.811915in}}{\pgfqpoint{1.978212in}{1.814827in}}%
\pgfpathcurveto{\pgfqpoint{1.975300in}{1.817739in}}{\pgfqpoint{1.971350in}{1.819375in}}{\pgfqpoint{1.967232in}{1.819375in}}%
\pgfpathcurveto{\pgfqpoint{1.963114in}{1.819375in}}{\pgfqpoint{1.959164in}{1.817739in}}{\pgfqpoint{1.956252in}{1.814827in}}%
\pgfpathcurveto{\pgfqpoint{1.953340in}{1.811915in}}{\pgfqpoint{1.951704in}{1.807965in}}{\pgfqpoint{1.951704in}{1.803847in}}%
\pgfpathcurveto{\pgfqpoint{1.951704in}{1.799729in}}{\pgfqpoint{1.953340in}{1.795778in}}{\pgfqpoint{1.956252in}{1.792867in}}%
\pgfpathcurveto{\pgfqpoint{1.959164in}{1.789955in}}{\pgfqpoint{1.963114in}{1.788318in}}{\pgfqpoint{1.967232in}{1.788318in}}%
\pgfpathclose%
\pgfusepath{fill}%
\end{pgfscope}%
\begin{pgfscope}%
\pgfpathrectangle{\pgfqpoint{0.887500in}{0.275000in}}{\pgfqpoint{4.225000in}{4.225000in}}%
\pgfusepath{clip}%
\pgfsetbuttcap%
\pgfsetroundjoin%
\definecolor{currentfill}{rgb}{0.000000,0.000000,0.000000}%
\pgfsetfillcolor{currentfill}%
\pgfsetfillopacity{0.800000}%
\pgfsetlinewidth{0.000000pt}%
\definecolor{currentstroke}{rgb}{0.000000,0.000000,0.000000}%
\pgfsetstrokecolor{currentstroke}%
\pgfsetstrokeopacity{0.800000}%
\pgfsetdash{}{0pt}%
\pgfpathmoveto{\pgfqpoint{2.995097in}{1.441451in}}%
\pgfpathcurveto{\pgfqpoint{2.999215in}{1.441451in}}{\pgfqpoint{3.003165in}{1.443088in}}{\pgfqpoint{3.006077in}{1.446000in}}%
\pgfpathcurveto{\pgfqpoint{3.008989in}{1.448911in}}{\pgfqpoint{3.010625in}{1.452861in}}{\pgfqpoint{3.010625in}{1.456980in}}%
\pgfpathcurveto{\pgfqpoint{3.010625in}{1.461098in}}{\pgfqpoint{3.008989in}{1.465048in}}{\pgfqpoint{3.006077in}{1.467960in}}%
\pgfpathcurveto{\pgfqpoint{3.003165in}{1.470872in}}{\pgfqpoint{2.999215in}{1.472508in}}{\pgfqpoint{2.995097in}{1.472508in}}%
\pgfpathcurveto{\pgfqpoint{2.990979in}{1.472508in}}{\pgfqpoint{2.987029in}{1.470872in}}{\pgfqpoint{2.984117in}{1.467960in}}%
\pgfpathcurveto{\pgfqpoint{2.981205in}{1.465048in}}{\pgfqpoint{2.979569in}{1.461098in}}{\pgfqpoint{2.979569in}{1.456980in}}%
\pgfpathcurveto{\pgfqpoint{2.979569in}{1.452861in}}{\pgfqpoint{2.981205in}{1.448911in}}{\pgfqpoint{2.984117in}{1.446000in}}%
\pgfpathcurveto{\pgfqpoint{2.987029in}{1.443088in}}{\pgfqpoint{2.990979in}{1.441451in}}{\pgfqpoint{2.995097in}{1.441451in}}%
\pgfpathclose%
\pgfusepath{fill}%
\end{pgfscope}%
\begin{pgfscope}%
\pgfpathrectangle{\pgfqpoint{0.887500in}{0.275000in}}{\pgfqpoint{4.225000in}{4.225000in}}%
\pgfusepath{clip}%
\pgfsetbuttcap%
\pgfsetroundjoin%
\definecolor{currentfill}{rgb}{0.000000,0.000000,0.000000}%
\pgfsetfillcolor{currentfill}%
\pgfsetfillopacity{0.800000}%
\pgfsetlinewidth{0.000000pt}%
\definecolor{currentstroke}{rgb}{0.000000,0.000000,0.000000}%
\pgfsetstrokecolor{currentstroke}%
\pgfsetstrokeopacity{0.800000}%
\pgfsetdash{}{0pt}%
\pgfpathmoveto{\pgfqpoint{3.177588in}{1.346231in}}%
\pgfpathcurveto{\pgfqpoint{3.181706in}{1.346231in}}{\pgfqpoint{3.185656in}{1.347867in}}{\pgfqpoint{3.188568in}{1.350779in}}%
\pgfpathcurveto{\pgfqpoint{3.191480in}{1.353691in}}{\pgfqpoint{3.193116in}{1.357641in}}{\pgfqpoint{3.193116in}{1.361759in}}%
\pgfpathcurveto{\pgfqpoint{3.193116in}{1.365877in}}{\pgfqpoint{3.191480in}{1.369827in}}{\pgfqpoint{3.188568in}{1.372739in}}%
\pgfpathcurveto{\pgfqpoint{3.185656in}{1.375651in}}{\pgfqpoint{3.181706in}{1.377287in}}{\pgfqpoint{3.177588in}{1.377287in}}%
\pgfpathcurveto{\pgfqpoint{3.173469in}{1.377287in}}{\pgfqpoint{3.169519in}{1.375651in}}{\pgfqpoint{3.166607in}{1.372739in}}%
\pgfpathcurveto{\pgfqpoint{3.163696in}{1.369827in}}{\pgfqpoint{3.162059in}{1.365877in}}{\pgfqpoint{3.162059in}{1.361759in}}%
\pgfpathcurveto{\pgfqpoint{3.162059in}{1.357641in}}{\pgfqpoint{3.163696in}{1.353691in}}{\pgfqpoint{3.166607in}{1.350779in}}%
\pgfpathcurveto{\pgfqpoint{3.169519in}{1.347867in}}{\pgfqpoint{3.173469in}{1.346231in}}{\pgfqpoint{3.177588in}{1.346231in}}%
\pgfpathclose%
\pgfusepath{fill}%
\end{pgfscope}%
\begin{pgfscope}%
\pgfpathrectangle{\pgfqpoint{0.887500in}{0.275000in}}{\pgfqpoint{4.225000in}{4.225000in}}%
\pgfusepath{clip}%
\pgfsetbuttcap%
\pgfsetroundjoin%
\definecolor{currentfill}{rgb}{0.000000,0.000000,0.000000}%
\pgfsetfillcolor{currentfill}%
\pgfsetfillopacity{0.800000}%
\pgfsetlinewidth{0.000000pt}%
\definecolor{currentstroke}{rgb}{0.000000,0.000000,0.000000}%
\pgfsetstrokecolor{currentstroke}%
\pgfsetstrokeopacity{0.800000}%
\pgfsetdash{}{0pt}%
\pgfpathmoveto{\pgfqpoint{2.149014in}{1.718818in}}%
\pgfpathcurveto{\pgfqpoint{2.153132in}{1.718818in}}{\pgfqpoint{2.157082in}{1.720454in}}{\pgfqpoint{2.159994in}{1.723366in}}%
\pgfpathcurveto{\pgfqpoint{2.162906in}{1.726278in}}{\pgfqpoint{2.164542in}{1.730228in}}{\pgfqpoint{2.164542in}{1.734346in}}%
\pgfpathcurveto{\pgfqpoint{2.164542in}{1.738464in}}{\pgfqpoint{2.162906in}{1.742414in}}{\pgfqpoint{2.159994in}{1.745326in}}%
\pgfpathcurveto{\pgfqpoint{2.157082in}{1.748238in}}{\pgfqpoint{2.153132in}{1.749874in}}{\pgfqpoint{2.149014in}{1.749874in}}%
\pgfpathcurveto{\pgfqpoint{2.144896in}{1.749874in}}{\pgfqpoint{2.140946in}{1.748238in}}{\pgfqpoint{2.138034in}{1.745326in}}%
\pgfpathcurveto{\pgfqpoint{2.135122in}{1.742414in}}{\pgfqpoint{2.133486in}{1.738464in}}{\pgfqpoint{2.133486in}{1.734346in}}%
\pgfpathcurveto{\pgfqpoint{2.133486in}{1.730228in}}{\pgfqpoint{2.135122in}{1.726278in}}{\pgfqpoint{2.138034in}{1.723366in}}%
\pgfpathcurveto{\pgfqpoint{2.140946in}{1.720454in}}{\pgfqpoint{2.144896in}{1.718818in}}{\pgfqpoint{2.149014in}{1.718818in}}%
\pgfpathclose%
\pgfusepath{fill}%
\end{pgfscope}%
\begin{pgfscope}%
\pgfpathrectangle{\pgfqpoint{0.887500in}{0.275000in}}{\pgfqpoint{4.225000in}{4.225000in}}%
\pgfusepath{clip}%
\pgfsetbuttcap%
\pgfsetroundjoin%
\definecolor{currentfill}{rgb}{0.000000,0.000000,0.000000}%
\pgfsetfillcolor{currentfill}%
\pgfsetfillopacity{0.800000}%
\pgfsetlinewidth{0.000000pt}%
\definecolor{currentstroke}{rgb}{0.000000,0.000000,0.000000}%
\pgfsetstrokecolor{currentstroke}%
\pgfsetstrokeopacity{0.800000}%
\pgfsetdash{}{0pt}%
\pgfpathmoveto{\pgfqpoint{3.427215in}{1.051542in}}%
\pgfpathcurveto{\pgfqpoint{3.431333in}{1.051542in}}{\pgfqpoint{3.435283in}{1.053178in}}{\pgfqpoint{3.438195in}{1.056090in}}%
\pgfpathcurveto{\pgfqpoint{3.441107in}{1.059002in}}{\pgfqpoint{3.442743in}{1.062952in}}{\pgfqpoint{3.442743in}{1.067070in}}%
\pgfpathcurveto{\pgfqpoint{3.442743in}{1.071188in}}{\pgfqpoint{3.441107in}{1.075138in}}{\pgfqpoint{3.438195in}{1.078050in}}%
\pgfpathcurveto{\pgfqpoint{3.435283in}{1.080962in}}{\pgfqpoint{3.431333in}{1.082599in}}{\pgfqpoint{3.427215in}{1.082599in}}%
\pgfpathcurveto{\pgfqpoint{3.423097in}{1.082599in}}{\pgfqpoint{3.419147in}{1.080962in}}{\pgfqpoint{3.416235in}{1.078050in}}%
\pgfpathcurveto{\pgfqpoint{3.413323in}{1.075138in}}{\pgfqpoint{3.411687in}{1.071188in}}{\pgfqpoint{3.411687in}{1.067070in}}%
\pgfpathcurveto{\pgfqpoint{3.411687in}{1.062952in}}{\pgfqpoint{3.413323in}{1.059002in}}{\pgfqpoint{3.416235in}{1.056090in}}%
\pgfpathcurveto{\pgfqpoint{3.419147in}{1.053178in}}{\pgfqpoint{3.423097in}{1.051542in}}{\pgfqpoint{3.427215in}{1.051542in}}%
\pgfpathclose%
\pgfusepath{fill}%
\end{pgfscope}%
\begin{pgfscope}%
\pgfpathrectangle{\pgfqpoint{0.887500in}{0.275000in}}{\pgfqpoint{4.225000in}{4.225000in}}%
\pgfusepath{clip}%
\pgfsetbuttcap%
\pgfsetroundjoin%
\definecolor{currentfill}{rgb}{0.000000,0.000000,0.000000}%
\pgfsetfillcolor{currentfill}%
\pgfsetfillopacity{0.800000}%
\pgfsetlinewidth{0.000000pt}%
\definecolor{currentstroke}{rgb}{0.000000,0.000000,0.000000}%
\pgfsetstrokecolor{currentstroke}%
\pgfsetstrokeopacity{0.800000}%
\pgfsetdash{}{0pt}%
\pgfpathmoveto{\pgfqpoint{2.331124in}{1.645846in}}%
\pgfpathcurveto{\pgfqpoint{2.335242in}{1.645846in}}{\pgfqpoint{2.339192in}{1.647482in}}{\pgfqpoint{2.342104in}{1.650394in}}%
\pgfpathcurveto{\pgfqpoint{2.345016in}{1.653306in}}{\pgfqpoint{2.346652in}{1.657256in}}{\pgfqpoint{2.346652in}{1.661374in}}%
\pgfpathcurveto{\pgfqpoint{2.346652in}{1.665492in}}{\pgfqpoint{2.345016in}{1.669442in}}{\pgfqpoint{2.342104in}{1.672354in}}%
\pgfpathcurveto{\pgfqpoint{2.339192in}{1.675266in}}{\pgfqpoint{2.335242in}{1.676903in}}{\pgfqpoint{2.331124in}{1.676903in}}%
\pgfpathcurveto{\pgfqpoint{2.327006in}{1.676903in}}{\pgfqpoint{2.323056in}{1.675266in}}{\pgfqpoint{2.320144in}{1.672354in}}%
\pgfpathcurveto{\pgfqpoint{2.317232in}{1.669442in}}{\pgfqpoint{2.315596in}{1.665492in}}{\pgfqpoint{2.315596in}{1.661374in}}%
\pgfpathcurveto{\pgfqpoint{2.315596in}{1.657256in}}{\pgfqpoint{2.317232in}{1.653306in}}{\pgfqpoint{2.320144in}{1.650394in}}%
\pgfpathcurveto{\pgfqpoint{2.323056in}{1.647482in}}{\pgfqpoint{2.327006in}{1.645846in}}{\pgfqpoint{2.331124in}{1.645846in}}%
\pgfpathclose%
\pgfusepath{fill}%
\end{pgfscope}%
\begin{pgfscope}%
\pgfpathrectangle{\pgfqpoint{0.887500in}{0.275000in}}{\pgfqpoint{4.225000in}{4.225000in}}%
\pgfusepath{clip}%
\pgfsetbuttcap%
\pgfsetroundjoin%
\definecolor{currentfill}{rgb}{0.000000,0.000000,0.000000}%
\pgfsetfillcolor{currentfill}%
\pgfsetfillopacity{0.800000}%
\pgfsetlinewidth{0.000000pt}%
\definecolor{currentstroke}{rgb}{0.000000,0.000000,0.000000}%
\pgfsetstrokecolor{currentstroke}%
\pgfsetstrokeopacity{0.800000}%
\pgfsetdash{}{0pt}%
\pgfpathmoveto{\pgfqpoint{2.513518in}{1.569243in}}%
\pgfpathcurveto{\pgfqpoint{2.517636in}{1.569243in}}{\pgfqpoint{2.521586in}{1.570879in}}{\pgfqpoint{2.524498in}{1.573791in}}%
\pgfpathcurveto{\pgfqpoint{2.527410in}{1.576703in}}{\pgfqpoint{2.529046in}{1.580653in}}{\pgfqpoint{2.529046in}{1.584771in}}%
\pgfpathcurveto{\pgfqpoint{2.529046in}{1.588889in}}{\pgfqpoint{2.527410in}{1.592839in}}{\pgfqpoint{2.524498in}{1.595751in}}%
\pgfpathcurveto{\pgfqpoint{2.521586in}{1.598663in}}{\pgfqpoint{2.517636in}{1.600299in}}{\pgfqpoint{2.513518in}{1.600299in}}%
\pgfpathcurveto{\pgfqpoint{2.509400in}{1.600299in}}{\pgfqpoint{2.505449in}{1.598663in}}{\pgfqpoint{2.502538in}{1.595751in}}%
\pgfpathcurveto{\pgfqpoint{2.499626in}{1.592839in}}{\pgfqpoint{2.497989in}{1.588889in}}{\pgfqpoint{2.497989in}{1.584771in}}%
\pgfpathcurveto{\pgfqpoint{2.497989in}{1.580653in}}{\pgfqpoint{2.499626in}{1.576703in}}{\pgfqpoint{2.502538in}{1.573791in}}%
\pgfpathcurveto{\pgfqpoint{2.505449in}{1.570879in}}{\pgfqpoint{2.509400in}{1.569243in}}{\pgfqpoint{2.513518in}{1.569243in}}%
\pgfpathclose%
\pgfusepath{fill}%
\end{pgfscope}%
\begin{pgfscope}%
\pgfpathrectangle{\pgfqpoint{0.887500in}{0.275000in}}{\pgfqpoint{4.225000in}{4.225000in}}%
\pgfusepath{clip}%
\pgfsetbuttcap%
\pgfsetroundjoin%
\definecolor{currentfill}{rgb}{0.000000,0.000000,0.000000}%
\pgfsetfillcolor{currentfill}%
\pgfsetfillopacity{0.800000}%
\pgfsetlinewidth{0.000000pt}%
\definecolor{currentstroke}{rgb}{0.000000,0.000000,0.000000}%
\pgfsetstrokecolor{currentstroke}%
\pgfsetstrokeopacity{0.800000}%
\pgfsetdash{}{0pt}%
\pgfpathmoveto{\pgfqpoint{2.696153in}{1.488248in}}%
\pgfpathcurveto{\pgfqpoint{2.700271in}{1.488248in}}{\pgfqpoint{2.704221in}{1.489884in}}{\pgfqpoint{2.707133in}{1.492796in}}%
\pgfpathcurveto{\pgfqpoint{2.710045in}{1.495708in}}{\pgfqpoint{2.711681in}{1.499658in}}{\pgfqpoint{2.711681in}{1.503776in}}%
\pgfpathcurveto{\pgfqpoint{2.711681in}{1.507894in}}{\pgfqpoint{2.710045in}{1.511844in}}{\pgfqpoint{2.707133in}{1.514756in}}%
\pgfpathcurveto{\pgfqpoint{2.704221in}{1.517668in}}{\pgfqpoint{2.700271in}{1.519304in}}{\pgfqpoint{2.696153in}{1.519304in}}%
\pgfpathcurveto{\pgfqpoint{2.692035in}{1.519304in}}{\pgfqpoint{2.688084in}{1.517668in}}{\pgfqpoint{2.685173in}{1.514756in}}%
\pgfpathcurveto{\pgfqpoint{2.682261in}{1.511844in}}{\pgfqpoint{2.680624in}{1.507894in}}{\pgfqpoint{2.680624in}{1.503776in}}%
\pgfpathcurveto{\pgfqpoint{2.680624in}{1.499658in}}{\pgfqpoint{2.682261in}{1.495708in}}{\pgfqpoint{2.685173in}{1.492796in}}%
\pgfpathcurveto{\pgfqpoint{2.688084in}{1.489884in}}{\pgfqpoint{2.692035in}{1.488248in}}{\pgfqpoint{2.696153in}{1.488248in}}%
\pgfpathclose%
\pgfusepath{fill}%
\end{pgfscope}%
\begin{pgfscope}%
\pgfpathrectangle{\pgfqpoint{0.887500in}{0.275000in}}{\pgfqpoint{4.225000in}{4.225000in}}%
\pgfusepath{clip}%
\pgfsetbuttcap%
\pgfsetroundjoin%
\definecolor{currentfill}{rgb}{0.000000,0.000000,0.000000}%
\pgfsetfillcolor{currentfill}%
\pgfsetfillopacity{0.800000}%
\pgfsetlinewidth{0.000000pt}%
\definecolor{currentstroke}{rgb}{0.000000,0.000000,0.000000}%
\pgfsetstrokecolor{currentstroke}%
\pgfsetstrokeopacity{0.800000}%
\pgfsetdash{}{0pt}%
\pgfpathmoveto{\pgfqpoint{2.878975in}{1.401421in}}%
\pgfpathcurveto{\pgfqpoint{2.883093in}{1.401421in}}{\pgfqpoint{2.887043in}{1.403057in}}{\pgfqpoint{2.889955in}{1.405969in}}%
\pgfpathcurveto{\pgfqpoint{2.892867in}{1.408881in}}{\pgfqpoint{2.894503in}{1.412831in}}{\pgfqpoint{2.894503in}{1.416950in}}%
\pgfpathcurveto{\pgfqpoint{2.894503in}{1.421068in}}{\pgfqpoint{2.892867in}{1.425018in}}{\pgfqpoint{2.889955in}{1.427930in}}%
\pgfpathcurveto{\pgfqpoint{2.887043in}{1.430842in}}{\pgfqpoint{2.883093in}{1.432478in}}{\pgfqpoint{2.878975in}{1.432478in}}%
\pgfpathcurveto{\pgfqpoint{2.874857in}{1.432478in}}{\pgfqpoint{2.870907in}{1.430842in}}{\pgfqpoint{2.867995in}{1.427930in}}%
\pgfpathcurveto{\pgfqpoint{2.865083in}{1.425018in}}{\pgfqpoint{2.863447in}{1.421068in}}{\pgfqpoint{2.863447in}{1.416950in}}%
\pgfpathcurveto{\pgfqpoint{2.863447in}{1.412831in}}{\pgfqpoint{2.865083in}{1.408881in}}{\pgfqpoint{2.867995in}{1.405969in}}%
\pgfpathcurveto{\pgfqpoint{2.870907in}{1.403057in}}{\pgfqpoint{2.874857in}{1.401421in}}{\pgfqpoint{2.878975in}{1.401421in}}%
\pgfpathclose%
\pgfusepath{fill}%
\end{pgfscope}%
\begin{pgfscope}%
\pgfpathrectangle{\pgfqpoint{0.887500in}{0.275000in}}{\pgfqpoint{4.225000in}{4.225000in}}%
\pgfusepath{clip}%
\pgfsetbuttcap%
\pgfsetroundjoin%
\definecolor{currentfill}{rgb}{0.000000,0.000000,0.000000}%
\pgfsetfillcolor{currentfill}%
\pgfsetfillopacity{0.800000}%
\pgfsetlinewidth{0.000000pt}%
\definecolor{currentstroke}{rgb}{0.000000,0.000000,0.000000}%
\pgfsetstrokecolor{currentstroke}%
\pgfsetstrokeopacity{0.800000}%
\pgfsetdash{}{0pt}%
\pgfpathmoveto{\pgfqpoint{3.061902in}{1.307596in}}%
\pgfpathcurveto{\pgfqpoint{3.066020in}{1.307596in}}{\pgfqpoint{3.069971in}{1.309232in}}{\pgfqpoint{3.072882in}{1.312144in}}%
\pgfpathcurveto{\pgfqpoint{3.075794in}{1.315056in}}{\pgfqpoint{3.077431in}{1.319006in}}{\pgfqpoint{3.077431in}{1.323124in}}%
\pgfpathcurveto{\pgfqpoint{3.077431in}{1.327242in}}{\pgfqpoint{3.075794in}{1.331192in}}{\pgfqpoint{3.072882in}{1.334104in}}%
\pgfpathcurveto{\pgfqpoint{3.069971in}{1.337016in}}{\pgfqpoint{3.066020in}{1.338652in}}{\pgfqpoint{3.061902in}{1.338652in}}%
\pgfpathcurveto{\pgfqpoint{3.057784in}{1.338652in}}{\pgfqpoint{3.053834in}{1.337016in}}{\pgfqpoint{3.050922in}{1.334104in}}%
\pgfpathcurveto{\pgfqpoint{3.048010in}{1.331192in}}{\pgfqpoint{3.046374in}{1.327242in}}{\pgfqpoint{3.046374in}{1.323124in}}%
\pgfpathcurveto{\pgfqpoint{3.046374in}{1.319006in}}{\pgfqpoint{3.048010in}{1.315056in}}{\pgfqpoint{3.050922in}{1.312144in}}%
\pgfpathcurveto{\pgfqpoint{3.053834in}{1.309232in}}{\pgfqpoint{3.057784in}{1.307596in}}{\pgfqpoint{3.061902in}{1.307596in}}%
\pgfpathclose%
\pgfusepath{fill}%
\end{pgfscope}%
\begin{pgfscope}%
\pgfpathrectangle{\pgfqpoint{0.887500in}{0.275000in}}{\pgfqpoint{4.225000in}{4.225000in}}%
\pgfusepath{clip}%
\pgfsetbuttcap%
\pgfsetroundjoin%
\definecolor{currentfill}{rgb}{0.000000,0.000000,0.000000}%
\pgfsetfillcolor{currentfill}%
\pgfsetfillopacity{0.800000}%
\pgfsetlinewidth{0.000000pt}%
\definecolor{currentstroke}{rgb}{0.000000,0.000000,0.000000}%
\pgfsetstrokecolor{currentstroke}%
\pgfsetstrokeopacity{0.800000}%
\pgfsetdash{}{0pt}%
\pgfpathmoveto{\pgfqpoint{2.030795in}{1.678576in}}%
\pgfpathcurveto{\pgfqpoint{2.034913in}{1.678576in}}{\pgfqpoint{2.038863in}{1.680213in}}{\pgfqpoint{2.041775in}{1.683125in}}%
\pgfpathcurveto{\pgfqpoint{2.044687in}{1.686037in}}{\pgfqpoint{2.046323in}{1.689987in}}{\pgfqpoint{2.046323in}{1.694105in}}%
\pgfpathcurveto{\pgfqpoint{2.046323in}{1.698223in}}{\pgfqpoint{2.044687in}{1.702173in}}{\pgfqpoint{2.041775in}{1.705085in}}%
\pgfpathcurveto{\pgfqpoint{2.038863in}{1.707997in}}{\pgfqpoint{2.034913in}{1.709633in}}{\pgfqpoint{2.030795in}{1.709633in}}%
\pgfpathcurveto{\pgfqpoint{2.026677in}{1.709633in}}{\pgfqpoint{2.022727in}{1.707997in}}{\pgfqpoint{2.019815in}{1.705085in}}%
\pgfpathcurveto{\pgfqpoint{2.016903in}{1.702173in}}{\pgfqpoint{2.015267in}{1.698223in}}{\pgfqpoint{2.015267in}{1.694105in}}%
\pgfpathcurveto{\pgfqpoint{2.015267in}{1.689987in}}{\pgfqpoint{2.016903in}{1.686037in}}{\pgfqpoint{2.019815in}{1.683125in}}%
\pgfpathcurveto{\pgfqpoint{2.022727in}{1.680213in}}{\pgfqpoint{2.026677in}{1.678576in}}{\pgfqpoint{2.030795in}{1.678576in}}%
\pgfpathclose%
\pgfusepath{fill}%
\end{pgfscope}%
\begin{pgfscope}%
\pgfpathrectangle{\pgfqpoint{0.887500in}{0.275000in}}{\pgfqpoint{4.225000in}{4.225000in}}%
\pgfusepath{clip}%
\pgfsetbuttcap%
\pgfsetroundjoin%
\definecolor{currentfill}{rgb}{0.000000,0.000000,0.000000}%
\pgfsetfillcolor{currentfill}%
\pgfsetfillopacity{0.800000}%
\pgfsetlinewidth{0.000000pt}%
\definecolor{currentstroke}{rgb}{0.000000,0.000000,0.000000}%
\pgfsetstrokecolor{currentstroke}%
\pgfsetstrokeopacity{0.800000}%
\pgfsetdash{}{0pt}%
\pgfpathmoveto{\pgfqpoint{3.245028in}{1.246121in}}%
\pgfpathcurveto{\pgfqpoint{3.249147in}{1.246121in}}{\pgfqpoint{3.253097in}{1.247757in}}{\pgfqpoint{3.256009in}{1.250669in}}%
\pgfpathcurveto{\pgfqpoint{3.258920in}{1.253581in}}{\pgfqpoint{3.260557in}{1.257531in}}{\pgfqpoint{3.260557in}{1.261649in}}%
\pgfpathcurveto{\pgfqpoint{3.260557in}{1.265768in}}{\pgfqpoint{3.258920in}{1.269718in}}{\pgfqpoint{3.256009in}{1.272630in}}%
\pgfpathcurveto{\pgfqpoint{3.253097in}{1.275542in}}{\pgfqpoint{3.249147in}{1.277178in}}{\pgfqpoint{3.245028in}{1.277178in}}%
\pgfpathcurveto{\pgfqpoint{3.240910in}{1.277178in}}{\pgfqpoint{3.236960in}{1.275542in}}{\pgfqpoint{3.234048in}{1.272630in}}%
\pgfpathcurveto{\pgfqpoint{3.231136in}{1.269718in}}{\pgfqpoint{3.229500in}{1.265768in}}{\pgfqpoint{3.229500in}{1.261649in}}%
\pgfpathcurveto{\pgfqpoint{3.229500in}{1.257531in}}{\pgfqpoint{3.231136in}{1.253581in}}{\pgfqpoint{3.234048in}{1.250669in}}%
\pgfpathcurveto{\pgfqpoint{3.236960in}{1.247757in}}{\pgfqpoint{3.240910in}{1.246121in}}{\pgfqpoint{3.245028in}{1.246121in}}%
\pgfpathclose%
\pgfusepath{fill}%
\end{pgfscope}%
\begin{pgfscope}%
\pgfpathrectangle{\pgfqpoint{0.887500in}{0.275000in}}{\pgfqpoint{4.225000in}{4.225000in}}%
\pgfusepath{clip}%
\pgfsetbuttcap%
\pgfsetroundjoin%
\definecolor{currentfill}{rgb}{0.000000,0.000000,0.000000}%
\pgfsetfillcolor{currentfill}%
\pgfsetfillopacity{0.800000}%
\pgfsetlinewidth{0.000000pt}%
\definecolor{currentstroke}{rgb}{0.000000,0.000000,0.000000}%
\pgfsetstrokecolor{currentstroke}%
\pgfsetstrokeopacity{0.800000}%
\pgfsetdash{}{0pt}%
\pgfpathmoveto{\pgfqpoint{2.213267in}{1.605489in}}%
\pgfpathcurveto{\pgfqpoint{2.217385in}{1.605489in}}{\pgfqpoint{2.221335in}{1.607125in}}{\pgfqpoint{2.224247in}{1.610037in}}%
\pgfpathcurveto{\pgfqpoint{2.227159in}{1.612949in}}{\pgfqpoint{2.228795in}{1.616899in}}{\pgfqpoint{2.228795in}{1.621017in}}%
\pgfpathcurveto{\pgfqpoint{2.228795in}{1.625136in}}{\pgfqpoint{2.227159in}{1.629086in}}{\pgfqpoint{2.224247in}{1.631998in}}%
\pgfpathcurveto{\pgfqpoint{2.221335in}{1.634909in}}{\pgfqpoint{2.217385in}{1.636546in}}{\pgfqpoint{2.213267in}{1.636546in}}%
\pgfpathcurveto{\pgfqpoint{2.209149in}{1.636546in}}{\pgfqpoint{2.205199in}{1.634909in}}{\pgfqpoint{2.202287in}{1.631998in}}%
\pgfpathcurveto{\pgfqpoint{2.199375in}{1.629086in}}{\pgfqpoint{2.197739in}{1.625136in}}{\pgfqpoint{2.197739in}{1.621017in}}%
\pgfpathcurveto{\pgfqpoint{2.197739in}{1.616899in}}{\pgfqpoint{2.199375in}{1.612949in}}{\pgfqpoint{2.202287in}{1.610037in}}%
\pgfpathcurveto{\pgfqpoint{2.205199in}{1.607125in}}{\pgfqpoint{2.209149in}{1.605489in}}{\pgfqpoint{2.213267in}{1.605489in}}%
\pgfpathclose%
\pgfusepath{fill}%
\end{pgfscope}%
\begin{pgfscope}%
\pgfpathrectangle{\pgfqpoint{0.887500in}{0.275000in}}{\pgfqpoint{4.225000in}{4.225000in}}%
\pgfusepath{clip}%
\pgfsetbuttcap%
\pgfsetroundjoin%
\definecolor{currentfill}{rgb}{0.000000,0.000000,0.000000}%
\pgfsetfillcolor{currentfill}%
\pgfsetfillopacity{0.800000}%
\pgfsetlinewidth{0.000000pt}%
\definecolor{currentstroke}{rgb}{0.000000,0.000000,0.000000}%
\pgfsetstrokecolor{currentstroke}%
\pgfsetstrokeopacity{0.800000}%
\pgfsetdash{}{0pt}%
\pgfpathmoveto{\pgfqpoint{2.396038in}{1.528592in}}%
\pgfpathcurveto{\pgfqpoint{2.400156in}{1.528592in}}{\pgfqpoint{2.404106in}{1.530228in}}{\pgfqpoint{2.407018in}{1.533140in}}%
\pgfpathcurveto{\pgfqpoint{2.409930in}{1.536052in}}{\pgfqpoint{2.411566in}{1.540002in}}{\pgfqpoint{2.411566in}{1.544120in}}%
\pgfpathcurveto{\pgfqpoint{2.411566in}{1.548238in}}{\pgfqpoint{2.409930in}{1.552188in}}{\pgfqpoint{2.407018in}{1.555100in}}%
\pgfpathcurveto{\pgfqpoint{2.404106in}{1.558012in}}{\pgfqpoint{2.400156in}{1.559648in}}{\pgfqpoint{2.396038in}{1.559648in}}%
\pgfpathcurveto{\pgfqpoint{2.391920in}{1.559648in}}{\pgfqpoint{2.387970in}{1.558012in}}{\pgfqpoint{2.385058in}{1.555100in}}%
\pgfpathcurveto{\pgfqpoint{2.382146in}{1.552188in}}{\pgfqpoint{2.380510in}{1.548238in}}{\pgfqpoint{2.380510in}{1.544120in}}%
\pgfpathcurveto{\pgfqpoint{2.380510in}{1.540002in}}{\pgfqpoint{2.382146in}{1.536052in}}{\pgfqpoint{2.385058in}{1.533140in}}%
\pgfpathcurveto{\pgfqpoint{2.387970in}{1.530228in}}{\pgfqpoint{2.391920in}{1.528592in}}{\pgfqpoint{2.396038in}{1.528592in}}%
\pgfpathclose%
\pgfusepath{fill}%
\end{pgfscope}%
\begin{pgfscope}%
\pgfpathrectangle{\pgfqpoint{0.887500in}{0.275000in}}{\pgfqpoint{4.225000in}{4.225000in}}%
\pgfusepath{clip}%
\pgfsetbuttcap%
\pgfsetroundjoin%
\definecolor{currentfill}{rgb}{0.000000,0.000000,0.000000}%
\pgfsetfillcolor{currentfill}%
\pgfsetfillopacity{0.800000}%
\pgfsetlinewidth{0.000000pt}%
\definecolor{currentstroke}{rgb}{0.000000,0.000000,0.000000}%
\pgfsetstrokecolor{currentstroke}%
\pgfsetstrokeopacity{0.800000}%
\pgfsetdash{}{0pt}%
\pgfpathmoveto{\pgfqpoint{2.579058in}{1.447728in}}%
\pgfpathcurveto{\pgfqpoint{2.583177in}{1.447728in}}{\pgfqpoint{2.587127in}{1.449365in}}{\pgfqpoint{2.590039in}{1.452277in}}%
\pgfpathcurveto{\pgfqpoint{2.592950in}{1.455189in}}{\pgfqpoint{2.594587in}{1.459139in}}{\pgfqpoint{2.594587in}{1.463257in}}%
\pgfpathcurveto{\pgfqpoint{2.594587in}{1.467375in}}{\pgfqpoint{2.592950in}{1.471325in}}{\pgfqpoint{2.590039in}{1.474237in}}%
\pgfpathcurveto{\pgfqpoint{2.587127in}{1.477149in}}{\pgfqpoint{2.583177in}{1.478785in}}{\pgfqpoint{2.579058in}{1.478785in}}%
\pgfpathcurveto{\pgfqpoint{2.574940in}{1.478785in}}{\pgfqpoint{2.570990in}{1.477149in}}{\pgfqpoint{2.568078in}{1.474237in}}%
\pgfpathcurveto{\pgfqpoint{2.565166in}{1.471325in}}{\pgfqpoint{2.563530in}{1.467375in}}{\pgfqpoint{2.563530in}{1.463257in}}%
\pgfpathcurveto{\pgfqpoint{2.563530in}{1.459139in}}{\pgfqpoint{2.565166in}{1.455189in}}{\pgfqpoint{2.568078in}{1.452277in}}%
\pgfpathcurveto{\pgfqpoint{2.570990in}{1.449365in}}{\pgfqpoint{2.574940in}{1.447728in}}{\pgfqpoint{2.579058in}{1.447728in}}%
\pgfpathclose%
\pgfusepath{fill}%
\end{pgfscope}%
\begin{pgfscope}%
\pgfpathrectangle{\pgfqpoint{0.887500in}{0.275000in}}{\pgfqpoint{4.225000in}{4.225000in}}%
\pgfusepath{clip}%
\pgfsetbuttcap%
\pgfsetroundjoin%
\definecolor{currentfill}{rgb}{0.000000,0.000000,0.000000}%
\pgfsetfillcolor{currentfill}%
\pgfsetfillopacity{0.800000}%
\pgfsetlinewidth{0.000000pt}%
\definecolor{currentstroke}{rgb}{0.000000,0.000000,0.000000}%
\pgfsetstrokecolor{currentstroke}%
\pgfsetstrokeopacity{0.800000}%
\pgfsetdash{}{0pt}%
\pgfpathmoveto{\pgfqpoint{2.762284in}{1.361540in}}%
\pgfpathcurveto{\pgfqpoint{2.766402in}{1.361540in}}{\pgfqpoint{2.770352in}{1.363176in}}{\pgfqpoint{2.773264in}{1.366088in}}%
\pgfpathcurveto{\pgfqpoint{2.776176in}{1.369000in}}{\pgfqpoint{2.777812in}{1.372950in}}{\pgfqpoint{2.777812in}{1.377068in}}%
\pgfpathcurveto{\pgfqpoint{2.777812in}{1.381187in}}{\pgfqpoint{2.776176in}{1.385137in}}{\pgfqpoint{2.773264in}{1.388048in}}%
\pgfpathcurveto{\pgfqpoint{2.770352in}{1.390960in}}{\pgfqpoint{2.766402in}{1.392597in}}{\pgfqpoint{2.762284in}{1.392597in}}%
\pgfpathcurveto{\pgfqpoint{2.758165in}{1.392597in}}{\pgfqpoint{2.754215in}{1.390960in}}{\pgfqpoint{2.751303in}{1.388048in}}%
\pgfpathcurveto{\pgfqpoint{2.748391in}{1.385137in}}{\pgfqpoint{2.746755in}{1.381187in}}{\pgfqpoint{2.746755in}{1.377068in}}%
\pgfpathcurveto{\pgfqpoint{2.746755in}{1.372950in}}{\pgfqpoint{2.748391in}{1.369000in}}{\pgfqpoint{2.751303in}{1.366088in}}%
\pgfpathcurveto{\pgfqpoint{2.754215in}{1.363176in}}{\pgfqpoint{2.758165in}{1.361540in}}{\pgfqpoint{2.762284in}{1.361540in}}%
\pgfpathclose%
\pgfusepath{fill}%
\end{pgfscope}%
\begin{pgfscope}%
\pgfpathrectangle{\pgfqpoint{0.887500in}{0.275000in}}{\pgfqpoint{4.225000in}{4.225000in}}%
\pgfusepath{clip}%
\pgfsetbuttcap%
\pgfsetroundjoin%
\definecolor{currentfill}{rgb}{0.000000,0.000000,0.000000}%
\pgfsetfillcolor{currentfill}%
\pgfsetfillopacity{0.800000}%
\pgfsetlinewidth{0.000000pt}%
\definecolor{currentstroke}{rgb}{0.000000,0.000000,0.000000}%
\pgfsetstrokecolor{currentstroke}%
\pgfsetstrokeopacity{0.800000}%
\pgfsetdash{}{0pt}%
\pgfpathmoveto{\pgfqpoint{2.945641in}{1.269370in}}%
\pgfpathcurveto{\pgfqpoint{2.949759in}{1.269370in}}{\pgfqpoint{2.953709in}{1.271006in}}{\pgfqpoint{2.956621in}{1.273918in}}%
\pgfpathcurveto{\pgfqpoint{2.959533in}{1.276830in}}{\pgfqpoint{2.961169in}{1.280780in}}{\pgfqpoint{2.961169in}{1.284898in}}%
\pgfpathcurveto{\pgfqpoint{2.961169in}{1.289016in}}{\pgfqpoint{2.959533in}{1.292967in}}{\pgfqpoint{2.956621in}{1.295878in}}%
\pgfpathcurveto{\pgfqpoint{2.953709in}{1.298790in}}{\pgfqpoint{2.949759in}{1.300427in}}{\pgfqpoint{2.945641in}{1.300427in}}%
\pgfpathcurveto{\pgfqpoint{2.941523in}{1.300427in}}{\pgfqpoint{2.937573in}{1.298790in}}{\pgfqpoint{2.934661in}{1.295878in}}%
\pgfpathcurveto{\pgfqpoint{2.931749in}{1.292967in}}{\pgfqpoint{2.930113in}{1.289016in}}{\pgfqpoint{2.930113in}{1.284898in}}%
\pgfpathcurveto{\pgfqpoint{2.930113in}{1.280780in}}{\pgfqpoint{2.931749in}{1.276830in}}{\pgfqpoint{2.934661in}{1.273918in}}%
\pgfpathcurveto{\pgfqpoint{2.937573in}{1.271006in}}{\pgfqpoint{2.941523in}{1.269370in}}{\pgfqpoint{2.945641in}{1.269370in}}%
\pgfpathclose%
\pgfusepath{fill}%
\end{pgfscope}%
\begin{pgfscope}%
\pgfpathrectangle{\pgfqpoint{0.887500in}{0.275000in}}{\pgfqpoint{4.225000in}{4.225000in}}%
\pgfusepath{clip}%
\pgfsetbuttcap%
\pgfsetroundjoin%
\definecolor{currentfill}{rgb}{0.000000,0.000000,0.000000}%
\pgfsetfillcolor{currentfill}%
\pgfsetfillopacity{0.800000}%
\pgfsetlinewidth{0.000000pt}%
\definecolor{currentstroke}{rgb}{0.000000,0.000000,0.000000}%
\pgfsetstrokecolor{currentstroke}%
\pgfsetstrokeopacity{0.800000}%
\pgfsetdash{}{0pt}%
\pgfpathmoveto{\pgfqpoint{3.129059in}{1.178628in}}%
\pgfpathcurveto{\pgfqpoint{3.133177in}{1.178628in}}{\pgfqpoint{3.137127in}{1.180264in}}{\pgfqpoint{3.140039in}{1.183176in}}%
\pgfpathcurveto{\pgfqpoint{3.142951in}{1.186088in}}{\pgfqpoint{3.144587in}{1.190038in}}{\pgfqpoint{3.144587in}{1.194156in}}%
\pgfpathcurveto{\pgfqpoint{3.144587in}{1.198274in}}{\pgfqpoint{3.142951in}{1.202224in}}{\pgfqpoint{3.140039in}{1.205136in}}%
\pgfpathcurveto{\pgfqpoint{3.137127in}{1.208048in}}{\pgfqpoint{3.133177in}{1.209684in}}{\pgfqpoint{3.129059in}{1.209684in}}%
\pgfpathcurveto{\pgfqpoint{3.124941in}{1.209684in}}{\pgfqpoint{3.120991in}{1.208048in}}{\pgfqpoint{3.118079in}{1.205136in}}%
\pgfpathcurveto{\pgfqpoint{3.115167in}{1.202224in}}{\pgfqpoint{3.113531in}{1.198274in}}{\pgfqpoint{3.113531in}{1.194156in}}%
\pgfpathcurveto{\pgfqpoint{3.113531in}{1.190038in}}{\pgfqpoint{3.115167in}{1.186088in}}{\pgfqpoint{3.118079in}{1.183176in}}%
\pgfpathcurveto{\pgfqpoint{3.120991in}{1.180264in}}{\pgfqpoint{3.124941in}{1.178628in}}{\pgfqpoint{3.129059in}{1.178628in}}%
\pgfpathclose%
\pgfusepath{fill}%
\end{pgfscope}%
\begin{pgfscope}%
\pgfpathrectangle{\pgfqpoint{0.887500in}{0.275000in}}{\pgfqpoint{4.225000in}{4.225000in}}%
\pgfusepath{clip}%
\pgfsetbuttcap%
\pgfsetroundjoin%
\definecolor{currentfill}{rgb}{0.000000,0.000000,0.000000}%
\pgfsetfillcolor{currentfill}%
\pgfsetfillopacity{0.800000}%
\pgfsetlinewidth{0.000000pt}%
\definecolor{currentstroke}{rgb}{0.000000,0.000000,0.000000}%
\pgfsetstrokecolor{currentstroke}%
\pgfsetstrokeopacity{0.800000}%
\pgfsetdash{}{0pt}%
\pgfpathmoveto{\pgfqpoint{2.094841in}{1.564850in}}%
\pgfpathcurveto{\pgfqpoint{2.098959in}{1.564850in}}{\pgfqpoint{2.102909in}{1.566486in}}{\pgfqpoint{2.105821in}{1.569398in}}%
\pgfpathcurveto{\pgfqpoint{2.108733in}{1.572310in}}{\pgfqpoint{2.110369in}{1.576260in}}{\pgfqpoint{2.110369in}{1.580378in}}%
\pgfpathcurveto{\pgfqpoint{2.110369in}{1.584496in}}{\pgfqpoint{2.108733in}{1.588446in}}{\pgfqpoint{2.105821in}{1.591358in}}%
\pgfpathcurveto{\pgfqpoint{2.102909in}{1.594270in}}{\pgfqpoint{2.098959in}{1.595907in}}{\pgfqpoint{2.094841in}{1.595907in}}%
\pgfpathcurveto{\pgfqpoint{2.090723in}{1.595907in}}{\pgfqpoint{2.086773in}{1.594270in}}{\pgfqpoint{2.083861in}{1.591358in}}%
\pgfpathcurveto{\pgfqpoint{2.080949in}{1.588446in}}{\pgfqpoint{2.079313in}{1.584496in}}{\pgfqpoint{2.079313in}{1.580378in}}%
\pgfpathcurveto{\pgfqpoint{2.079313in}{1.576260in}}{\pgfqpoint{2.080949in}{1.572310in}}{\pgfqpoint{2.083861in}{1.569398in}}%
\pgfpathcurveto{\pgfqpoint{2.086773in}{1.566486in}}{\pgfqpoint{2.090723in}{1.564850in}}{\pgfqpoint{2.094841in}{1.564850in}}%
\pgfpathclose%
\pgfusepath{fill}%
\end{pgfscope}%
\begin{pgfscope}%
\pgfpathrectangle{\pgfqpoint{0.887500in}{0.275000in}}{\pgfqpoint{4.225000in}{4.225000in}}%
\pgfusepath{clip}%
\pgfsetbuttcap%
\pgfsetroundjoin%
\definecolor{currentfill}{rgb}{0.000000,0.000000,0.000000}%
\pgfsetfillcolor{currentfill}%
\pgfsetfillopacity{0.800000}%
\pgfsetlinewidth{0.000000pt}%
\definecolor{currentstroke}{rgb}{0.000000,0.000000,0.000000}%
\pgfsetstrokecolor{currentstroke}%
\pgfsetstrokeopacity{0.800000}%
\pgfsetdash{}{0pt}%
\pgfpathmoveto{\pgfqpoint{2.277982in}{1.488090in}}%
\pgfpathcurveto{\pgfqpoint{2.282100in}{1.488090in}}{\pgfqpoint{2.286051in}{1.489726in}}{\pgfqpoint{2.288962in}{1.492638in}}%
\pgfpathcurveto{\pgfqpoint{2.291874in}{1.495550in}}{\pgfqpoint{2.293511in}{1.499500in}}{\pgfqpoint{2.293511in}{1.503619in}}%
\pgfpathcurveto{\pgfqpoint{2.293511in}{1.507737in}}{\pgfqpoint{2.291874in}{1.511687in}}{\pgfqpoint{2.288962in}{1.514599in}}%
\pgfpathcurveto{\pgfqpoint{2.286051in}{1.517511in}}{\pgfqpoint{2.282100in}{1.519147in}}{\pgfqpoint{2.277982in}{1.519147in}}%
\pgfpathcurveto{\pgfqpoint{2.273864in}{1.519147in}}{\pgfqpoint{2.269914in}{1.517511in}}{\pgfqpoint{2.267002in}{1.514599in}}%
\pgfpathcurveto{\pgfqpoint{2.264090in}{1.511687in}}{\pgfqpoint{2.262454in}{1.507737in}}{\pgfqpoint{2.262454in}{1.503619in}}%
\pgfpathcurveto{\pgfqpoint{2.262454in}{1.499500in}}{\pgfqpoint{2.264090in}{1.495550in}}{\pgfqpoint{2.267002in}{1.492638in}}%
\pgfpathcurveto{\pgfqpoint{2.269914in}{1.489726in}}{\pgfqpoint{2.273864in}{1.488090in}}{\pgfqpoint{2.277982in}{1.488090in}}%
\pgfpathclose%
\pgfusepath{fill}%
\end{pgfscope}%
\begin{pgfscope}%
\pgfpathrectangle{\pgfqpoint{0.887500in}{0.275000in}}{\pgfqpoint{4.225000in}{4.225000in}}%
\pgfusepath{clip}%
\pgfsetbuttcap%
\pgfsetroundjoin%
\definecolor{currentfill}{rgb}{0.000000,0.000000,0.000000}%
\pgfsetfillcolor{currentfill}%
\pgfsetfillopacity{0.800000}%
\pgfsetlinewidth{0.000000pt}%
\definecolor{currentstroke}{rgb}{0.000000,0.000000,0.000000}%
\pgfsetstrokecolor{currentstroke}%
\pgfsetstrokeopacity{0.800000}%
\pgfsetdash{}{0pt}%
\pgfpathmoveto{\pgfqpoint{2.461389in}{1.407357in}}%
\pgfpathcurveto{\pgfqpoint{2.465507in}{1.407357in}}{\pgfqpoint{2.469457in}{1.408994in}}{\pgfqpoint{2.472369in}{1.411905in}}%
\pgfpathcurveto{\pgfqpoint{2.475281in}{1.414817in}}{\pgfqpoint{2.476918in}{1.418767in}}{\pgfqpoint{2.476918in}{1.422886in}}%
\pgfpathcurveto{\pgfqpoint{2.476918in}{1.427004in}}{\pgfqpoint{2.475281in}{1.430954in}}{\pgfqpoint{2.472369in}{1.433866in}}%
\pgfpathcurveto{\pgfqpoint{2.469457in}{1.436778in}}{\pgfqpoint{2.465507in}{1.438414in}}{\pgfqpoint{2.461389in}{1.438414in}}%
\pgfpathcurveto{\pgfqpoint{2.457271in}{1.438414in}}{\pgfqpoint{2.453321in}{1.436778in}}{\pgfqpoint{2.450409in}{1.433866in}}%
\pgfpathcurveto{\pgfqpoint{2.447497in}{1.430954in}}{\pgfqpoint{2.445861in}{1.427004in}}{\pgfqpoint{2.445861in}{1.422886in}}%
\pgfpathcurveto{\pgfqpoint{2.445861in}{1.418767in}}{\pgfqpoint{2.447497in}{1.414817in}}{\pgfqpoint{2.450409in}{1.411905in}}%
\pgfpathcurveto{\pgfqpoint{2.453321in}{1.408994in}}{\pgfqpoint{2.457271in}{1.407357in}}{\pgfqpoint{2.461389in}{1.407357in}}%
\pgfpathclose%
\pgfusepath{fill}%
\end{pgfscope}%
\begin{pgfscope}%
\pgfpathrectangle{\pgfqpoint{0.887500in}{0.275000in}}{\pgfqpoint{4.225000in}{4.225000in}}%
\pgfusepath{clip}%
\pgfsetbuttcap%
\pgfsetroundjoin%
\definecolor{currentfill}{rgb}{0.000000,0.000000,0.000000}%
\pgfsetfillcolor{currentfill}%
\pgfsetfillopacity{0.800000}%
\pgfsetlinewidth{0.000000pt}%
\definecolor{currentstroke}{rgb}{0.000000,0.000000,0.000000}%
\pgfsetstrokecolor{currentstroke}%
\pgfsetstrokeopacity{0.800000}%
\pgfsetdash{}{0pt}%
\pgfpathmoveto{\pgfqpoint{2.645014in}{1.321980in}}%
\pgfpathcurveto{\pgfqpoint{2.649132in}{1.321980in}}{\pgfqpoint{2.653082in}{1.323617in}}{\pgfqpoint{2.655994in}{1.326529in}}%
\pgfpathcurveto{\pgfqpoint{2.658906in}{1.329441in}}{\pgfqpoint{2.660542in}{1.333391in}}{\pgfqpoint{2.660542in}{1.337509in}}%
\pgfpathcurveto{\pgfqpoint{2.660542in}{1.341627in}}{\pgfqpoint{2.658906in}{1.345577in}}{\pgfqpoint{2.655994in}{1.348489in}}%
\pgfpathcurveto{\pgfqpoint{2.653082in}{1.351401in}}{\pgfqpoint{2.649132in}{1.353037in}}{\pgfqpoint{2.645014in}{1.353037in}}%
\pgfpathcurveto{\pgfqpoint{2.640896in}{1.353037in}}{\pgfqpoint{2.636946in}{1.351401in}}{\pgfqpoint{2.634034in}{1.348489in}}%
\pgfpathcurveto{\pgfqpoint{2.631122in}{1.345577in}}{\pgfqpoint{2.629486in}{1.341627in}}{\pgfqpoint{2.629486in}{1.337509in}}%
\pgfpathcurveto{\pgfqpoint{2.629486in}{1.333391in}}{\pgfqpoint{2.631122in}{1.329441in}}{\pgfqpoint{2.634034in}{1.326529in}}%
\pgfpathcurveto{\pgfqpoint{2.636946in}{1.323617in}}{\pgfqpoint{2.640896in}{1.321980in}}{\pgfqpoint{2.645014in}{1.321980in}}%
\pgfpathclose%
\pgfusepath{fill}%
\end{pgfscope}%
\begin{pgfscope}%
\pgfpathrectangle{\pgfqpoint{0.887500in}{0.275000in}}{\pgfqpoint{4.225000in}{4.225000in}}%
\pgfusepath{clip}%
\pgfsetbuttcap%
\pgfsetroundjoin%
\definecolor{currentfill}{rgb}{0.000000,0.000000,0.000000}%
\pgfsetfillcolor{currentfill}%
\pgfsetfillopacity{0.800000}%
\pgfsetlinewidth{0.000000pt}%
\definecolor{currentstroke}{rgb}{0.000000,0.000000,0.000000}%
\pgfsetstrokecolor{currentstroke}%
\pgfsetstrokeopacity{0.800000}%
\pgfsetdash{}{0pt}%
\pgfpathmoveto{\pgfqpoint{3.012662in}{1.131051in}}%
\pgfpathcurveto{\pgfqpoint{3.016780in}{1.131051in}}{\pgfqpoint{3.020730in}{1.132687in}}{\pgfqpoint{3.023642in}{1.135599in}}%
\pgfpathcurveto{\pgfqpoint{3.026554in}{1.138511in}}{\pgfqpoint{3.028190in}{1.142461in}}{\pgfqpoint{3.028190in}{1.146579in}}%
\pgfpathcurveto{\pgfqpoint{3.028190in}{1.150697in}}{\pgfqpoint{3.026554in}{1.154647in}}{\pgfqpoint{3.023642in}{1.157559in}}%
\pgfpathcurveto{\pgfqpoint{3.020730in}{1.160471in}}{\pgfqpoint{3.016780in}{1.162107in}}{\pgfqpoint{3.012662in}{1.162107in}}%
\pgfpathcurveto{\pgfqpoint{3.008544in}{1.162107in}}{\pgfqpoint{3.004594in}{1.160471in}}{\pgfqpoint{3.001682in}{1.157559in}}%
\pgfpathcurveto{\pgfqpoint{2.998770in}{1.154647in}}{\pgfqpoint{2.997134in}{1.150697in}}{\pgfqpoint{2.997134in}{1.146579in}}%
\pgfpathcurveto{\pgfqpoint{2.997134in}{1.142461in}}{\pgfqpoint{2.998770in}{1.138511in}}{\pgfqpoint{3.001682in}{1.135599in}}%
\pgfpathcurveto{\pgfqpoint{3.004594in}{1.132687in}}{\pgfqpoint{3.008544in}{1.131051in}}{\pgfqpoint{3.012662in}{1.131051in}}%
\pgfpathclose%
\pgfusepath{fill}%
\end{pgfscope}%
\begin{pgfscope}%
\pgfpathrectangle{\pgfqpoint{0.887500in}{0.275000in}}{\pgfqpoint{4.225000in}{4.225000in}}%
\pgfusepath{clip}%
\pgfsetbuttcap%
\pgfsetroundjoin%
\definecolor{currentfill}{rgb}{0.000000,0.000000,0.000000}%
\pgfsetfillcolor{currentfill}%
\pgfsetfillopacity{0.800000}%
\pgfsetlinewidth{0.000000pt}%
\definecolor{currentstroke}{rgb}{0.000000,0.000000,0.000000}%
\pgfsetstrokecolor{currentstroke}%
\pgfsetstrokeopacity{0.800000}%
\pgfsetdash{}{0pt}%
\pgfpathmoveto{\pgfqpoint{2.828792in}{1.231727in}}%
\pgfpathcurveto{\pgfqpoint{2.832910in}{1.231727in}}{\pgfqpoint{2.836860in}{1.233363in}}{\pgfqpoint{2.839772in}{1.236275in}}%
\pgfpathcurveto{\pgfqpoint{2.842684in}{1.239187in}}{\pgfqpoint{2.844321in}{1.243137in}}{\pgfqpoint{2.844321in}{1.247255in}}%
\pgfpathcurveto{\pgfqpoint{2.844321in}{1.251373in}}{\pgfqpoint{2.842684in}{1.255324in}}{\pgfqpoint{2.839772in}{1.258235in}}%
\pgfpathcurveto{\pgfqpoint{2.836860in}{1.261147in}}{\pgfqpoint{2.832910in}{1.262784in}}{\pgfqpoint{2.828792in}{1.262784in}}%
\pgfpathcurveto{\pgfqpoint{2.824674in}{1.262784in}}{\pgfqpoint{2.820724in}{1.261147in}}{\pgfqpoint{2.817812in}{1.258235in}}%
\pgfpathcurveto{\pgfqpoint{2.814900in}{1.255324in}}{\pgfqpoint{2.813264in}{1.251373in}}{\pgfqpoint{2.813264in}{1.247255in}}%
\pgfpathcurveto{\pgfqpoint{2.813264in}{1.243137in}}{\pgfqpoint{2.814900in}{1.239187in}}{\pgfqpoint{2.817812in}{1.236275in}}%
\pgfpathcurveto{\pgfqpoint{2.820724in}{1.233363in}}{\pgfqpoint{2.824674in}{1.231727in}}{\pgfqpoint{2.828792in}{1.231727in}}%
\pgfpathclose%
\pgfusepath{fill}%
\end{pgfscope}%
\begin{pgfscope}%
\pgfpathrectangle{\pgfqpoint{0.887500in}{0.275000in}}{\pgfqpoint{4.225000in}{4.225000in}}%
\pgfusepath{clip}%
\pgfsetbuttcap%
\pgfsetroundjoin%
\definecolor{currentfill}{rgb}{0.000000,0.000000,0.000000}%
\pgfsetfillcolor{currentfill}%
\pgfsetfillopacity{0.800000}%
\pgfsetlinewidth{0.000000pt}%
\definecolor{currentstroke}{rgb}{0.000000,0.000000,0.000000}%
\pgfsetstrokecolor{currentstroke}%
\pgfsetstrokeopacity{0.800000}%
\pgfsetdash{}{0pt}%
\pgfpathmoveto{\pgfqpoint{2.159341in}{1.447824in}}%
\pgfpathcurveto{\pgfqpoint{2.163459in}{1.447824in}}{\pgfqpoint{2.167409in}{1.449461in}}{\pgfqpoint{2.170321in}{1.452373in}}%
\pgfpathcurveto{\pgfqpoint{2.173233in}{1.455285in}}{\pgfqpoint{2.174870in}{1.459235in}}{\pgfqpoint{2.174870in}{1.463353in}}%
\pgfpathcurveto{\pgfqpoint{2.174870in}{1.467471in}}{\pgfqpoint{2.173233in}{1.471421in}}{\pgfqpoint{2.170321in}{1.474333in}}%
\pgfpathcurveto{\pgfqpoint{2.167409in}{1.477245in}}{\pgfqpoint{2.163459in}{1.478881in}}{\pgfqpoint{2.159341in}{1.478881in}}%
\pgfpathcurveto{\pgfqpoint{2.155223in}{1.478881in}}{\pgfqpoint{2.151273in}{1.477245in}}{\pgfqpoint{2.148361in}{1.474333in}}%
\pgfpathcurveto{\pgfqpoint{2.145449in}{1.471421in}}{\pgfqpoint{2.143813in}{1.467471in}}{\pgfqpoint{2.143813in}{1.463353in}}%
\pgfpathcurveto{\pgfqpoint{2.143813in}{1.459235in}}{\pgfqpoint{2.145449in}{1.455285in}}{\pgfqpoint{2.148361in}{1.452373in}}%
\pgfpathcurveto{\pgfqpoint{2.151273in}{1.449461in}}{\pgfqpoint{2.155223in}{1.447824in}}{\pgfqpoint{2.159341in}{1.447824in}}%
\pgfpathclose%
\pgfusepath{fill}%
\end{pgfscope}%
\begin{pgfscope}%
\pgfpathrectangle{\pgfqpoint{0.887500in}{0.275000in}}{\pgfqpoint{4.225000in}{4.225000in}}%
\pgfusepath{clip}%
\pgfsetbuttcap%
\pgfsetroundjoin%
\definecolor{currentfill}{rgb}{0.000000,0.000000,0.000000}%
\pgfsetfillcolor{currentfill}%
\pgfsetfillopacity{0.800000}%
\pgfsetlinewidth{0.000000pt}%
\definecolor{currentstroke}{rgb}{0.000000,0.000000,0.000000}%
\pgfsetstrokecolor{currentstroke}%
\pgfsetstrokeopacity{0.800000}%
\pgfsetdash{}{0pt}%
\pgfpathmoveto{\pgfqpoint{2.343131in}{1.367481in}}%
\pgfpathcurveto{\pgfqpoint{2.347250in}{1.367481in}}{\pgfqpoint{2.351200in}{1.369117in}}{\pgfqpoint{2.354112in}{1.372029in}}%
\pgfpathcurveto{\pgfqpoint{2.357024in}{1.374941in}}{\pgfqpoint{2.358660in}{1.378891in}}{\pgfqpoint{2.358660in}{1.383009in}}%
\pgfpathcurveto{\pgfqpoint{2.358660in}{1.387127in}}{\pgfqpoint{2.357024in}{1.391077in}}{\pgfqpoint{2.354112in}{1.393989in}}%
\pgfpathcurveto{\pgfqpoint{2.351200in}{1.396901in}}{\pgfqpoint{2.347250in}{1.398537in}}{\pgfqpoint{2.343131in}{1.398537in}}%
\pgfpathcurveto{\pgfqpoint{2.339013in}{1.398537in}}{\pgfqpoint{2.335063in}{1.396901in}}{\pgfqpoint{2.332151in}{1.393989in}}%
\pgfpathcurveto{\pgfqpoint{2.329239in}{1.391077in}}{\pgfqpoint{2.327603in}{1.387127in}}{\pgfqpoint{2.327603in}{1.383009in}}%
\pgfpathcurveto{\pgfqpoint{2.327603in}{1.378891in}}{\pgfqpoint{2.329239in}{1.374941in}}{\pgfqpoint{2.332151in}{1.372029in}}%
\pgfpathcurveto{\pgfqpoint{2.335063in}{1.369117in}}{\pgfqpoint{2.339013in}{1.367481in}}{\pgfqpoint{2.343131in}{1.367481in}}%
\pgfpathclose%
\pgfusepath{fill}%
\end{pgfscope}%
\begin{pgfscope}%
\pgfpathrectangle{\pgfqpoint{0.887500in}{0.275000in}}{\pgfqpoint{4.225000in}{4.225000in}}%
\pgfusepath{clip}%
\pgfsetbuttcap%
\pgfsetroundjoin%
\definecolor{currentfill}{rgb}{0.000000,0.000000,0.000000}%
\pgfsetfillcolor{currentfill}%
\pgfsetfillopacity{0.800000}%
\pgfsetlinewidth{0.000000pt}%
\definecolor{currentstroke}{rgb}{0.000000,0.000000,0.000000}%
\pgfsetstrokecolor{currentstroke}%
\pgfsetstrokeopacity{0.800000}%
\pgfsetdash{}{0pt}%
\pgfpathmoveto{\pgfqpoint{2.527154in}{1.282916in}}%
\pgfpathcurveto{\pgfqpoint{2.531272in}{1.282916in}}{\pgfqpoint{2.535222in}{1.284552in}}{\pgfqpoint{2.538134in}{1.287464in}}%
\pgfpathcurveto{\pgfqpoint{2.541046in}{1.290376in}}{\pgfqpoint{2.542682in}{1.294326in}}{\pgfqpoint{2.542682in}{1.298444in}}%
\pgfpathcurveto{\pgfqpoint{2.542682in}{1.302562in}}{\pgfqpoint{2.541046in}{1.306512in}}{\pgfqpoint{2.538134in}{1.309424in}}%
\pgfpathcurveto{\pgfqpoint{2.535222in}{1.312336in}}{\pgfqpoint{2.531272in}{1.313972in}}{\pgfqpoint{2.527154in}{1.313972in}}%
\pgfpathcurveto{\pgfqpoint{2.523036in}{1.313972in}}{\pgfqpoint{2.519086in}{1.312336in}}{\pgfqpoint{2.516174in}{1.309424in}}%
\pgfpathcurveto{\pgfqpoint{2.513262in}{1.306512in}}{\pgfqpoint{2.511626in}{1.302562in}}{\pgfqpoint{2.511626in}{1.298444in}}%
\pgfpathcurveto{\pgfqpoint{2.511626in}{1.294326in}}{\pgfqpoint{2.513262in}{1.290376in}}{\pgfqpoint{2.516174in}{1.287464in}}%
\pgfpathcurveto{\pgfqpoint{2.519086in}{1.284552in}}{\pgfqpoint{2.523036in}{1.282916in}}{\pgfqpoint{2.527154in}{1.282916in}}%
\pgfpathclose%
\pgfusepath{fill}%
\end{pgfscope}%
\begin{pgfscope}%
\pgfpathrectangle{\pgfqpoint{0.887500in}{0.275000in}}{\pgfqpoint{4.225000in}{4.225000in}}%
\pgfusepath{clip}%
\pgfsetbuttcap%
\pgfsetroundjoin%
\definecolor{currentfill}{rgb}{0.000000,0.000000,0.000000}%
\pgfsetfillcolor{currentfill}%
\pgfsetfillopacity{0.800000}%
\pgfsetlinewidth{0.000000pt}%
\definecolor{currentstroke}{rgb}{0.000000,0.000000,0.000000}%
\pgfsetstrokecolor{currentstroke}%
\pgfsetstrokeopacity{0.800000}%
\pgfsetdash{}{0pt}%
\pgfpathmoveto{\pgfqpoint{2.711347in}{1.194497in}}%
\pgfpathcurveto{\pgfqpoint{2.715465in}{1.194497in}}{\pgfqpoint{2.719415in}{1.196133in}}{\pgfqpoint{2.722327in}{1.199045in}}%
\pgfpathcurveto{\pgfqpoint{2.725239in}{1.201957in}}{\pgfqpoint{2.726875in}{1.205907in}}{\pgfqpoint{2.726875in}{1.210025in}}%
\pgfpathcurveto{\pgfqpoint{2.726875in}{1.214143in}}{\pgfqpoint{2.725239in}{1.218093in}}{\pgfqpoint{2.722327in}{1.221005in}}%
\pgfpathcurveto{\pgfqpoint{2.719415in}{1.223917in}}{\pgfqpoint{2.715465in}{1.225553in}}{\pgfqpoint{2.711347in}{1.225553in}}%
\pgfpathcurveto{\pgfqpoint{2.707228in}{1.225553in}}{\pgfqpoint{2.703278in}{1.223917in}}{\pgfqpoint{2.700366in}{1.221005in}}%
\pgfpathcurveto{\pgfqpoint{2.697455in}{1.218093in}}{\pgfqpoint{2.695818in}{1.214143in}}{\pgfqpoint{2.695818in}{1.210025in}}%
\pgfpathcurveto{\pgfqpoint{2.695818in}{1.205907in}}{\pgfqpoint{2.697455in}{1.201957in}}{\pgfqpoint{2.700366in}{1.199045in}}%
\pgfpathcurveto{\pgfqpoint{2.703278in}{1.196133in}}{\pgfqpoint{2.707228in}{1.194497in}}{\pgfqpoint{2.711347in}{1.194497in}}%
\pgfpathclose%
\pgfusepath{fill}%
\end{pgfscope}%
\begin{pgfscope}%
\pgfpathrectangle{\pgfqpoint{0.887500in}{0.275000in}}{\pgfqpoint{4.225000in}{4.225000in}}%
\pgfusepath{clip}%
\pgfsetbuttcap%
\pgfsetroundjoin%
\definecolor{currentfill}{rgb}{0.000000,0.000000,0.000000}%
\pgfsetfillcolor{currentfill}%
\pgfsetfillopacity{0.800000}%
\pgfsetlinewidth{0.000000pt}%
\definecolor{currentstroke}{rgb}{0.000000,0.000000,0.000000}%
\pgfsetstrokecolor{currentstroke}%
\pgfsetstrokeopacity{0.800000}%
\pgfsetdash{}{0pt}%
\pgfpathmoveto{\pgfqpoint{2.895662in}{1.100528in}}%
\pgfpathcurveto{\pgfqpoint{2.899780in}{1.100528in}}{\pgfqpoint{2.903730in}{1.102164in}}{\pgfqpoint{2.906642in}{1.105076in}}%
\pgfpathcurveto{\pgfqpoint{2.909554in}{1.107988in}}{\pgfqpoint{2.911190in}{1.111938in}}{\pgfqpoint{2.911190in}{1.116056in}}%
\pgfpathcurveto{\pgfqpoint{2.911190in}{1.120174in}}{\pgfqpoint{2.909554in}{1.124124in}}{\pgfqpoint{2.906642in}{1.127036in}}%
\pgfpathcurveto{\pgfqpoint{2.903730in}{1.129948in}}{\pgfqpoint{2.899780in}{1.131584in}}{\pgfqpoint{2.895662in}{1.131584in}}%
\pgfpathcurveto{\pgfqpoint{2.891544in}{1.131584in}}{\pgfqpoint{2.887594in}{1.129948in}}{\pgfqpoint{2.884682in}{1.127036in}}%
\pgfpathcurveto{\pgfqpoint{2.881770in}{1.124124in}}{\pgfqpoint{2.880134in}{1.120174in}}{\pgfqpoint{2.880134in}{1.116056in}}%
\pgfpathcurveto{\pgfqpoint{2.880134in}{1.111938in}}{\pgfqpoint{2.881770in}{1.107988in}}{\pgfqpoint{2.884682in}{1.105076in}}%
\pgfpathcurveto{\pgfqpoint{2.887594in}{1.102164in}}{\pgfqpoint{2.891544in}{1.100528in}}{\pgfqpoint{2.895662in}{1.100528in}}%
\pgfpathclose%
\pgfusepath{fill}%
\end{pgfscope}%
\begin{pgfscope}%
\pgfpathrectangle{\pgfqpoint{0.887500in}{0.275000in}}{\pgfqpoint{4.225000in}{4.225000in}}%
\pgfusepath{clip}%
\pgfsetbuttcap%
\pgfsetroundjoin%
\definecolor{currentfill}{rgb}{0.000000,0.000000,0.000000}%
\pgfsetfillcolor{currentfill}%
\pgfsetfillopacity{0.800000}%
\pgfsetlinewidth{0.000000pt}%
\definecolor{currentstroke}{rgb}{0.000000,0.000000,0.000000}%
\pgfsetstrokecolor{currentstroke}%
\pgfsetstrokeopacity{0.800000}%
\pgfsetdash{}{0pt}%
\pgfpathmoveto{\pgfqpoint{3.197041in}{1.175543in}}%
\pgfpathcurveto{\pgfqpoint{3.201160in}{1.175543in}}{\pgfqpoint{3.205110in}{1.177179in}}{\pgfqpoint{3.208022in}{1.180091in}}%
\pgfpathcurveto{\pgfqpoint{3.210933in}{1.183003in}}{\pgfqpoint{3.212570in}{1.186953in}}{\pgfqpoint{3.212570in}{1.191071in}}%
\pgfpathcurveto{\pgfqpoint{3.212570in}{1.195189in}}{\pgfqpoint{3.210933in}{1.199139in}}{\pgfqpoint{3.208022in}{1.202051in}}%
\pgfpathcurveto{\pgfqpoint{3.205110in}{1.204963in}}{\pgfqpoint{3.201160in}{1.206599in}}{\pgfqpoint{3.197041in}{1.206599in}}%
\pgfpathcurveto{\pgfqpoint{3.192923in}{1.206599in}}{\pgfqpoint{3.188973in}{1.204963in}}{\pgfqpoint{3.186061in}{1.202051in}}%
\pgfpathcurveto{\pgfqpoint{3.183149in}{1.199139in}}{\pgfqpoint{3.181513in}{1.195189in}}{\pgfqpoint{3.181513in}{1.191071in}}%
\pgfpathcurveto{\pgfqpoint{3.181513in}{1.186953in}}{\pgfqpoint{3.183149in}{1.183003in}}{\pgfqpoint{3.186061in}{1.180091in}}%
\pgfpathcurveto{\pgfqpoint{3.188973in}{1.177179in}}{\pgfqpoint{3.192923in}{1.175543in}}{\pgfqpoint{3.197041in}{1.175543in}}%
\pgfpathclose%
\pgfusepath{fill}%
\end{pgfscope}%
\begin{pgfscope}%
\pgfpathrectangle{\pgfqpoint{0.887500in}{0.275000in}}{\pgfqpoint{4.225000in}{4.225000in}}%
\pgfusepath{clip}%
\pgfsetbuttcap%
\pgfsetroundjoin%
\definecolor{currentfill}{rgb}{0.000000,0.000000,0.000000}%
\pgfsetfillcolor{currentfill}%
\pgfsetfillopacity{0.800000}%
\pgfsetlinewidth{0.000000pt}%
\definecolor{currentstroke}{rgb}{0.000000,0.000000,0.000000}%
\pgfsetstrokecolor{currentstroke}%
\pgfsetstrokeopacity{0.800000}%
\pgfsetdash{}{0pt}%
\pgfpathmoveto{\pgfqpoint{2.224278in}{1.327927in}}%
\pgfpathcurveto{\pgfqpoint{2.228396in}{1.327927in}}{\pgfqpoint{2.232346in}{1.329563in}}{\pgfqpoint{2.235258in}{1.332475in}}%
\pgfpathcurveto{\pgfqpoint{2.238170in}{1.335387in}}{\pgfqpoint{2.239806in}{1.339337in}}{\pgfqpoint{2.239806in}{1.343455in}}%
\pgfpathcurveto{\pgfqpoint{2.239806in}{1.347573in}}{\pgfqpoint{2.238170in}{1.351523in}}{\pgfqpoint{2.235258in}{1.354435in}}%
\pgfpathcurveto{\pgfqpoint{2.232346in}{1.357347in}}{\pgfqpoint{2.228396in}{1.358983in}}{\pgfqpoint{2.224278in}{1.358983in}}%
\pgfpathcurveto{\pgfqpoint{2.220160in}{1.358983in}}{\pgfqpoint{2.216210in}{1.357347in}}{\pgfqpoint{2.213298in}{1.354435in}}%
\pgfpathcurveto{\pgfqpoint{2.210386in}{1.351523in}}{\pgfqpoint{2.208750in}{1.347573in}}{\pgfqpoint{2.208750in}{1.343455in}}%
\pgfpathcurveto{\pgfqpoint{2.208750in}{1.339337in}}{\pgfqpoint{2.210386in}{1.335387in}}{\pgfqpoint{2.213298in}{1.332475in}}%
\pgfpathcurveto{\pgfqpoint{2.216210in}{1.329563in}}{\pgfqpoint{2.220160in}{1.327927in}}{\pgfqpoint{2.224278in}{1.327927in}}%
\pgfpathclose%
\pgfusepath{fill}%
\end{pgfscope}%
\begin{pgfscope}%
\pgfpathrectangle{\pgfqpoint{0.887500in}{0.275000in}}{\pgfqpoint{4.225000in}{4.225000in}}%
\pgfusepath{clip}%
\pgfsetbuttcap%
\pgfsetroundjoin%
\definecolor{currentfill}{rgb}{0.000000,0.000000,0.000000}%
\pgfsetfillcolor{currentfill}%
\pgfsetfillopacity{0.800000}%
\pgfsetlinewidth{0.000000pt}%
\definecolor{currentstroke}{rgb}{0.000000,0.000000,0.000000}%
\pgfsetstrokecolor{currentstroke}%
\pgfsetstrokeopacity{0.800000}%
\pgfsetdash{}{0pt}%
\pgfpathmoveto{\pgfqpoint{2.408689in}{1.244607in}}%
\pgfpathcurveto{\pgfqpoint{2.412807in}{1.244607in}}{\pgfqpoint{2.416757in}{1.246243in}}{\pgfqpoint{2.419669in}{1.249155in}}%
\pgfpathcurveto{\pgfqpoint{2.422581in}{1.252067in}}{\pgfqpoint{2.424217in}{1.256017in}}{\pgfqpoint{2.424217in}{1.260135in}}%
\pgfpathcurveto{\pgfqpoint{2.424217in}{1.264253in}}{\pgfqpoint{2.422581in}{1.268203in}}{\pgfqpoint{2.419669in}{1.271115in}}%
\pgfpathcurveto{\pgfqpoint{2.416757in}{1.274027in}}{\pgfqpoint{2.412807in}{1.275663in}}{\pgfqpoint{2.408689in}{1.275663in}}%
\pgfpathcurveto{\pgfqpoint{2.404571in}{1.275663in}}{\pgfqpoint{2.400621in}{1.274027in}}{\pgfqpoint{2.397709in}{1.271115in}}%
\pgfpathcurveto{\pgfqpoint{2.394797in}{1.268203in}}{\pgfqpoint{2.393161in}{1.264253in}}{\pgfqpoint{2.393161in}{1.260135in}}%
\pgfpathcurveto{\pgfqpoint{2.393161in}{1.256017in}}{\pgfqpoint{2.394797in}{1.252067in}}{\pgfqpoint{2.397709in}{1.249155in}}%
\pgfpathcurveto{\pgfqpoint{2.400621in}{1.246243in}}{\pgfqpoint{2.404571in}{1.244607in}}{\pgfqpoint{2.408689in}{1.244607in}}%
\pgfpathclose%
\pgfusepath{fill}%
\end{pgfscope}%
\begin{pgfscope}%
\pgfpathrectangle{\pgfqpoint{0.887500in}{0.275000in}}{\pgfqpoint{4.225000in}{4.225000in}}%
\pgfusepath{clip}%
\pgfsetbuttcap%
\pgfsetroundjoin%
\definecolor{currentfill}{rgb}{0.000000,0.000000,0.000000}%
\pgfsetfillcolor{currentfill}%
\pgfsetfillopacity{0.800000}%
\pgfsetlinewidth{0.000000pt}%
\definecolor{currentstroke}{rgb}{0.000000,0.000000,0.000000}%
\pgfsetstrokecolor{currentstroke}%
\pgfsetstrokeopacity{0.800000}%
\pgfsetdash{}{0pt}%
\pgfpathmoveto{\pgfqpoint{2.593292in}{1.157853in}}%
\pgfpathcurveto{\pgfqpoint{2.597410in}{1.157853in}}{\pgfqpoint{2.601360in}{1.159489in}}{\pgfqpoint{2.604272in}{1.162401in}}%
\pgfpathcurveto{\pgfqpoint{2.607184in}{1.165313in}}{\pgfqpoint{2.608820in}{1.169263in}}{\pgfqpoint{2.608820in}{1.173381in}}%
\pgfpathcurveto{\pgfqpoint{2.608820in}{1.177499in}}{\pgfqpoint{2.607184in}{1.181449in}}{\pgfqpoint{2.604272in}{1.184361in}}%
\pgfpathcurveto{\pgfqpoint{2.601360in}{1.187273in}}{\pgfqpoint{2.597410in}{1.188909in}}{\pgfqpoint{2.593292in}{1.188909in}}%
\pgfpathcurveto{\pgfqpoint{2.589174in}{1.188909in}}{\pgfqpoint{2.585224in}{1.187273in}}{\pgfqpoint{2.582312in}{1.184361in}}%
\pgfpathcurveto{\pgfqpoint{2.579400in}{1.181449in}}{\pgfqpoint{2.577763in}{1.177499in}}{\pgfqpoint{2.577763in}{1.173381in}}%
\pgfpathcurveto{\pgfqpoint{2.577763in}{1.169263in}}{\pgfqpoint{2.579400in}{1.165313in}}{\pgfqpoint{2.582312in}{1.162401in}}%
\pgfpathcurveto{\pgfqpoint{2.585224in}{1.159489in}}{\pgfqpoint{2.589174in}{1.157853in}}{\pgfqpoint{2.593292in}{1.157853in}}%
\pgfpathclose%
\pgfusepath{fill}%
\end{pgfscope}%
\begin{pgfscope}%
\pgfpathrectangle{\pgfqpoint{0.887500in}{0.275000in}}{\pgfqpoint{4.225000in}{4.225000in}}%
\pgfusepath{clip}%
\pgfsetbuttcap%
\pgfsetroundjoin%
\definecolor{currentfill}{rgb}{0.000000,0.000000,0.000000}%
\pgfsetfillcolor{currentfill}%
\pgfsetfillopacity{0.800000}%
\pgfsetlinewidth{0.000000pt}%
\definecolor{currentstroke}{rgb}{0.000000,0.000000,0.000000}%
\pgfsetstrokecolor{currentstroke}%
\pgfsetstrokeopacity{0.800000}%
\pgfsetdash{}{0pt}%
\pgfpathmoveto{\pgfqpoint{2.778033in}{1.068201in}}%
\pgfpathcurveto{\pgfqpoint{2.782151in}{1.068201in}}{\pgfqpoint{2.786101in}{1.069837in}}{\pgfqpoint{2.789013in}{1.072749in}}%
\pgfpathcurveto{\pgfqpoint{2.791925in}{1.075661in}}{\pgfqpoint{2.793561in}{1.079611in}}{\pgfqpoint{2.793561in}{1.083729in}}%
\pgfpathcurveto{\pgfqpoint{2.793561in}{1.087847in}}{\pgfqpoint{2.791925in}{1.091797in}}{\pgfqpoint{2.789013in}{1.094709in}}%
\pgfpathcurveto{\pgfqpoint{2.786101in}{1.097621in}}{\pgfqpoint{2.782151in}{1.099257in}}{\pgfqpoint{2.778033in}{1.099257in}}%
\pgfpathcurveto{\pgfqpoint{2.773915in}{1.099257in}}{\pgfqpoint{2.769965in}{1.097621in}}{\pgfqpoint{2.767053in}{1.094709in}}%
\pgfpathcurveto{\pgfqpoint{2.764141in}{1.091797in}}{\pgfqpoint{2.762505in}{1.087847in}}{\pgfqpoint{2.762505in}{1.083729in}}%
\pgfpathcurveto{\pgfqpoint{2.762505in}{1.079611in}}{\pgfqpoint{2.764141in}{1.075661in}}{\pgfqpoint{2.767053in}{1.072749in}}%
\pgfpathcurveto{\pgfqpoint{2.769965in}{1.069837in}}{\pgfqpoint{2.773915in}{1.068201in}}{\pgfqpoint{2.778033in}{1.068201in}}%
\pgfpathclose%
\pgfusepath{fill}%
\end{pgfscope}%
\begin{pgfscope}%
\pgfpathrectangle{\pgfqpoint{0.887500in}{0.275000in}}{\pgfqpoint{4.225000in}{4.225000in}}%
\pgfusepath{clip}%
\pgfsetbuttcap%
\pgfsetroundjoin%
\definecolor{currentfill}{rgb}{0.000000,0.000000,0.000000}%
\pgfsetfillcolor{currentfill}%
\pgfsetfillopacity{0.800000}%
\pgfsetlinewidth{0.000000pt}%
\definecolor{currentstroke}{rgb}{0.000000,0.000000,0.000000}%
\pgfsetstrokecolor{currentstroke}%
\pgfsetstrokeopacity{0.800000}%
\pgfsetdash{}{0pt}%
\pgfpathmoveto{\pgfqpoint{3.080085in}{1.108693in}}%
\pgfpathcurveto{\pgfqpoint{3.084203in}{1.108693in}}{\pgfqpoint{3.088153in}{1.110329in}}{\pgfqpoint{3.091065in}{1.113241in}}%
\pgfpathcurveto{\pgfqpoint{3.093977in}{1.116153in}}{\pgfqpoint{3.095613in}{1.120103in}}{\pgfqpoint{3.095613in}{1.124221in}}%
\pgfpathcurveto{\pgfqpoint{3.095613in}{1.128339in}}{\pgfqpoint{3.093977in}{1.132289in}}{\pgfqpoint{3.091065in}{1.135201in}}%
\pgfpathcurveto{\pgfqpoint{3.088153in}{1.138113in}}{\pgfqpoint{3.084203in}{1.139749in}}{\pgfqpoint{3.080085in}{1.139749in}}%
\pgfpathcurveto{\pgfqpoint{3.075967in}{1.139749in}}{\pgfqpoint{3.072017in}{1.138113in}}{\pgfqpoint{3.069105in}{1.135201in}}%
\pgfpathcurveto{\pgfqpoint{3.066193in}{1.132289in}}{\pgfqpoint{3.064557in}{1.128339in}}{\pgfqpoint{3.064557in}{1.124221in}}%
\pgfpathcurveto{\pgfqpoint{3.064557in}{1.120103in}}{\pgfqpoint{3.066193in}{1.116153in}}{\pgfqpoint{3.069105in}{1.113241in}}%
\pgfpathcurveto{\pgfqpoint{3.072017in}{1.110329in}}{\pgfqpoint{3.075967in}{1.108693in}}{\pgfqpoint{3.080085in}{1.108693in}}%
\pgfpathclose%
\pgfusepath{fill}%
\end{pgfscope}%
\begin{pgfscope}%
\pgfpathrectangle{\pgfqpoint{0.887500in}{0.275000in}}{\pgfqpoint{4.225000in}{4.225000in}}%
\pgfusepath{clip}%
\pgfsetbuttcap%
\pgfsetroundjoin%
\definecolor{currentfill}{rgb}{0.000000,0.000000,0.000000}%
\pgfsetfillcolor{currentfill}%
\pgfsetfillopacity{0.800000}%
\pgfsetlinewidth{0.000000pt}%
\definecolor{currentstroke}{rgb}{0.000000,0.000000,0.000000}%
\pgfsetstrokecolor{currentstroke}%
\pgfsetstrokeopacity{0.800000}%
\pgfsetdash{}{0pt}%
\pgfpathmoveto{\pgfqpoint{2.962714in}{1.039879in}}%
\pgfpathcurveto{\pgfqpoint{2.966832in}{1.039879in}}{\pgfqpoint{2.970782in}{1.041515in}}{\pgfqpoint{2.973694in}{1.044427in}}%
\pgfpathcurveto{\pgfqpoint{2.976606in}{1.047339in}}{\pgfqpoint{2.978242in}{1.051289in}}{\pgfqpoint{2.978242in}{1.055407in}}%
\pgfpathcurveto{\pgfqpoint{2.978242in}{1.059526in}}{\pgfqpoint{2.976606in}{1.063476in}}{\pgfqpoint{2.973694in}{1.066388in}}%
\pgfpathcurveto{\pgfqpoint{2.970782in}{1.069299in}}{\pgfqpoint{2.966832in}{1.070936in}}{\pgfqpoint{2.962714in}{1.070936in}}%
\pgfpathcurveto{\pgfqpoint{2.958596in}{1.070936in}}{\pgfqpoint{2.954646in}{1.069299in}}{\pgfqpoint{2.951734in}{1.066388in}}%
\pgfpathcurveto{\pgfqpoint{2.948822in}{1.063476in}}{\pgfqpoint{2.947186in}{1.059526in}}{\pgfqpoint{2.947186in}{1.055407in}}%
\pgfpathcurveto{\pgfqpoint{2.947186in}{1.051289in}}{\pgfqpoint{2.948822in}{1.047339in}}{\pgfqpoint{2.951734in}{1.044427in}}%
\pgfpathcurveto{\pgfqpoint{2.954646in}{1.041515in}}{\pgfqpoint{2.958596in}{1.039879in}}{\pgfqpoint{2.962714in}{1.039879in}}%
\pgfpathclose%
\pgfusepath{fill}%
\end{pgfscope}%
\begin{pgfscope}%
\pgfpathrectangle{\pgfqpoint{0.887500in}{0.275000in}}{\pgfqpoint{4.225000in}{4.225000in}}%
\pgfusepath{clip}%
\pgfsetbuttcap%
\pgfsetroundjoin%
\definecolor{currentfill}{rgb}{0.000000,0.000000,0.000000}%
\pgfsetfillcolor{currentfill}%
\pgfsetfillopacity{0.800000}%
\pgfsetlinewidth{0.000000pt}%
\definecolor{currentstroke}{rgb}{0.000000,0.000000,0.000000}%
\pgfsetstrokecolor{currentstroke}%
\pgfsetstrokeopacity{0.800000}%
\pgfsetdash{}{0pt}%
\pgfpathmoveto{\pgfqpoint{2.289623in}{1.206191in}}%
\pgfpathcurveto{\pgfqpoint{2.293741in}{1.206191in}}{\pgfqpoint{2.297691in}{1.207827in}}{\pgfqpoint{2.300603in}{1.210739in}}%
\pgfpathcurveto{\pgfqpoint{2.303515in}{1.213651in}}{\pgfqpoint{2.305151in}{1.217601in}}{\pgfqpoint{2.305151in}{1.221719in}}%
\pgfpathcurveto{\pgfqpoint{2.305151in}{1.225837in}}{\pgfqpoint{2.303515in}{1.229787in}}{\pgfqpoint{2.300603in}{1.232699in}}%
\pgfpathcurveto{\pgfqpoint{2.297691in}{1.235611in}}{\pgfqpoint{2.293741in}{1.237247in}}{\pgfqpoint{2.289623in}{1.237247in}}%
\pgfpathcurveto{\pgfqpoint{2.285505in}{1.237247in}}{\pgfqpoint{2.281555in}{1.235611in}}{\pgfqpoint{2.278643in}{1.232699in}}%
\pgfpathcurveto{\pgfqpoint{2.275731in}{1.229787in}}{\pgfqpoint{2.274095in}{1.225837in}}{\pgfqpoint{2.274095in}{1.221719in}}%
\pgfpathcurveto{\pgfqpoint{2.274095in}{1.217601in}}{\pgfqpoint{2.275731in}{1.213651in}}{\pgfqpoint{2.278643in}{1.210739in}}%
\pgfpathcurveto{\pgfqpoint{2.281555in}{1.207827in}}{\pgfqpoint{2.285505in}{1.206191in}}{\pgfqpoint{2.289623in}{1.206191in}}%
\pgfpathclose%
\pgfusepath{fill}%
\end{pgfscope}%
\begin{pgfscope}%
\pgfpathrectangle{\pgfqpoint{0.887500in}{0.275000in}}{\pgfqpoint{4.225000in}{4.225000in}}%
\pgfusepath{clip}%
\pgfsetbuttcap%
\pgfsetroundjoin%
\definecolor{currentfill}{rgb}{0.000000,0.000000,0.000000}%
\pgfsetfillcolor{currentfill}%
\pgfsetfillopacity{0.800000}%
\pgfsetlinewidth{0.000000pt}%
\definecolor{currentstroke}{rgb}{0.000000,0.000000,0.000000}%
\pgfsetstrokecolor{currentstroke}%
\pgfsetstrokeopacity{0.800000}%
\pgfsetdash{}{0pt}%
\pgfpathmoveto{\pgfqpoint{2.474625in}{1.121193in}}%
\pgfpathcurveto{\pgfqpoint{2.478743in}{1.121193in}}{\pgfqpoint{2.482693in}{1.122829in}}{\pgfqpoint{2.485605in}{1.125741in}}%
\pgfpathcurveto{\pgfqpoint{2.488517in}{1.128653in}}{\pgfqpoint{2.490153in}{1.132603in}}{\pgfqpoint{2.490153in}{1.136721in}}%
\pgfpathcurveto{\pgfqpoint{2.490153in}{1.140839in}}{\pgfqpoint{2.488517in}{1.144789in}}{\pgfqpoint{2.485605in}{1.147701in}}%
\pgfpathcurveto{\pgfqpoint{2.482693in}{1.150613in}}{\pgfqpoint{2.478743in}{1.152249in}}{\pgfqpoint{2.474625in}{1.152249in}}%
\pgfpathcurveto{\pgfqpoint{2.470507in}{1.152249in}}{\pgfqpoint{2.466557in}{1.150613in}}{\pgfqpoint{2.463645in}{1.147701in}}%
\pgfpathcurveto{\pgfqpoint{2.460733in}{1.144789in}}{\pgfqpoint{2.459097in}{1.140839in}}{\pgfqpoint{2.459097in}{1.136721in}}%
\pgfpathcurveto{\pgfqpoint{2.459097in}{1.132603in}}{\pgfqpoint{2.460733in}{1.128653in}}{\pgfqpoint{2.463645in}{1.125741in}}%
\pgfpathcurveto{\pgfqpoint{2.466557in}{1.122829in}}{\pgfqpoint{2.470507in}{1.121193in}}{\pgfqpoint{2.474625in}{1.121193in}}%
\pgfpathclose%
\pgfusepath{fill}%
\end{pgfscope}%
\begin{pgfscope}%
\pgfpathrectangle{\pgfqpoint{0.887500in}{0.275000in}}{\pgfqpoint{4.225000in}{4.225000in}}%
\pgfusepath{clip}%
\pgfsetbuttcap%
\pgfsetroundjoin%
\definecolor{currentfill}{rgb}{0.000000,0.000000,0.000000}%
\pgfsetfillcolor{currentfill}%
\pgfsetfillopacity{0.800000}%
\pgfsetlinewidth{0.000000pt}%
\definecolor{currentstroke}{rgb}{0.000000,0.000000,0.000000}%
\pgfsetstrokecolor{currentstroke}%
\pgfsetstrokeopacity{0.800000}%
\pgfsetdash{}{0pt}%
\pgfpathmoveto{\pgfqpoint{2.659782in}{1.034406in}}%
\pgfpathcurveto{\pgfqpoint{2.663901in}{1.034406in}}{\pgfqpoint{2.667851in}{1.036042in}}{\pgfqpoint{2.670763in}{1.038954in}}%
\pgfpathcurveto{\pgfqpoint{2.673675in}{1.041866in}}{\pgfqpoint{2.675311in}{1.045816in}}{\pgfqpoint{2.675311in}{1.049934in}}%
\pgfpathcurveto{\pgfqpoint{2.675311in}{1.054052in}}{\pgfqpoint{2.673675in}{1.058002in}}{\pgfqpoint{2.670763in}{1.060914in}}%
\pgfpathcurveto{\pgfqpoint{2.667851in}{1.063826in}}{\pgfqpoint{2.663901in}{1.065462in}}{\pgfqpoint{2.659782in}{1.065462in}}%
\pgfpathcurveto{\pgfqpoint{2.655664in}{1.065462in}}{\pgfqpoint{2.651714in}{1.063826in}}{\pgfqpoint{2.648802in}{1.060914in}}%
\pgfpathcurveto{\pgfqpoint{2.645890in}{1.058002in}}{\pgfqpoint{2.644254in}{1.054052in}}{\pgfqpoint{2.644254in}{1.049934in}}%
\pgfpathcurveto{\pgfqpoint{2.644254in}{1.045816in}}{\pgfqpoint{2.645890in}{1.041866in}}{\pgfqpoint{2.648802in}{1.038954in}}%
\pgfpathcurveto{\pgfqpoint{2.651714in}{1.036042in}}{\pgfqpoint{2.655664in}{1.034406in}}{\pgfqpoint{2.659782in}{1.034406in}}%
\pgfpathclose%
\pgfusepath{fill}%
\end{pgfscope}%
\begin{pgfscope}%
\pgfpathrectangle{\pgfqpoint{0.887500in}{0.275000in}}{\pgfqpoint{4.225000in}{4.225000in}}%
\pgfusepath{clip}%
\pgfsetbuttcap%
\pgfsetroundjoin%
\definecolor{currentfill}{rgb}{0.000000,0.000000,0.000000}%
\pgfsetfillcolor{currentfill}%
\pgfsetfillopacity{0.800000}%
\pgfsetlinewidth{0.000000pt}%
\definecolor{currentstroke}{rgb}{0.000000,0.000000,0.000000}%
\pgfsetstrokecolor{currentstroke}%
\pgfsetstrokeopacity{0.800000}%
\pgfsetdash{}{0pt}%
\pgfpathmoveto{\pgfqpoint{2.844898in}{0.976860in}}%
\pgfpathcurveto{\pgfqpoint{2.849017in}{0.976860in}}{\pgfqpoint{2.852967in}{0.978497in}}{\pgfqpoint{2.855879in}{0.981408in}}%
\pgfpathcurveto{\pgfqpoint{2.858791in}{0.984320in}}{\pgfqpoint{2.860427in}{0.988270in}}{\pgfqpoint{2.860427in}{0.992389in}}%
\pgfpathcurveto{\pgfqpoint{2.860427in}{0.996507in}}{\pgfqpoint{2.858791in}{1.000457in}}{\pgfqpoint{2.855879in}{1.003369in}}%
\pgfpathcurveto{\pgfqpoint{2.852967in}{1.006281in}}{\pgfqpoint{2.849017in}{1.007917in}}{\pgfqpoint{2.844898in}{1.007917in}}%
\pgfpathcurveto{\pgfqpoint{2.840780in}{1.007917in}}{\pgfqpoint{2.836830in}{1.006281in}}{\pgfqpoint{2.833918in}{1.003369in}}%
\pgfpathcurveto{\pgfqpoint{2.831006in}{1.000457in}}{\pgfqpoint{2.829370in}{0.996507in}}{\pgfqpoint{2.829370in}{0.992389in}}%
\pgfpathcurveto{\pgfqpoint{2.829370in}{0.988270in}}{\pgfqpoint{2.831006in}{0.984320in}}{\pgfqpoint{2.833918in}{0.981408in}}%
\pgfpathcurveto{\pgfqpoint{2.836830in}{0.978497in}}{\pgfqpoint{2.840780in}{0.976860in}}{\pgfqpoint{2.844898in}{0.976860in}}%
\pgfpathclose%
\pgfusepath{fill}%
\end{pgfscope}%
\begin{pgfscope}%
\pgfpathrectangle{\pgfqpoint{0.887500in}{0.275000in}}{\pgfqpoint{4.225000in}{4.225000in}}%
\pgfusepath{clip}%
\pgfsetbuttcap%
\pgfsetroundjoin%
\definecolor{currentfill}{rgb}{0.000000,0.000000,0.000000}%
\pgfsetfillcolor{currentfill}%
\pgfsetfillopacity{0.800000}%
\pgfsetlinewidth{0.000000pt}%
\definecolor{currentstroke}{rgb}{0.000000,0.000000,0.000000}%
\pgfsetstrokecolor{currentstroke}%
\pgfsetstrokeopacity{0.800000}%
\pgfsetdash{}{0pt}%
\pgfpathmoveto{\pgfqpoint{2.355344in}{1.084256in}}%
\pgfpathcurveto{\pgfqpoint{2.359462in}{1.084256in}}{\pgfqpoint{2.363413in}{1.085892in}}{\pgfqpoint{2.366324in}{1.088804in}}%
\pgfpathcurveto{\pgfqpoint{2.369236in}{1.091716in}}{\pgfqpoint{2.370873in}{1.095666in}}{\pgfqpoint{2.370873in}{1.099784in}}%
\pgfpathcurveto{\pgfqpoint{2.370873in}{1.103902in}}{\pgfqpoint{2.369236in}{1.107852in}}{\pgfqpoint{2.366324in}{1.110764in}}%
\pgfpathcurveto{\pgfqpoint{2.363413in}{1.113676in}}{\pgfqpoint{2.359462in}{1.115312in}}{\pgfqpoint{2.355344in}{1.115312in}}%
\pgfpathcurveto{\pgfqpoint{2.351226in}{1.115312in}}{\pgfqpoint{2.347276in}{1.113676in}}{\pgfqpoint{2.344364in}{1.110764in}}%
\pgfpathcurveto{\pgfqpoint{2.341452in}{1.107852in}}{\pgfqpoint{2.339816in}{1.103902in}}{\pgfqpoint{2.339816in}{1.099784in}}%
\pgfpathcurveto{\pgfqpoint{2.339816in}{1.095666in}}{\pgfqpoint{2.341452in}{1.091716in}}{\pgfqpoint{2.344364in}{1.088804in}}%
\pgfpathcurveto{\pgfqpoint{2.347276in}{1.085892in}}{\pgfqpoint{2.351226in}{1.084256in}}{\pgfqpoint{2.355344in}{1.084256in}}%
\pgfpathclose%
\pgfusepath{fill}%
\end{pgfscope}%
\begin{pgfscope}%
\pgfpathrectangle{\pgfqpoint{0.887500in}{0.275000in}}{\pgfqpoint{4.225000in}{4.225000in}}%
\pgfusepath{clip}%
\pgfsetbuttcap%
\pgfsetroundjoin%
\definecolor{currentfill}{rgb}{0.000000,0.000000,0.000000}%
\pgfsetfillcolor{currentfill}%
\pgfsetfillopacity{0.800000}%
\pgfsetlinewidth{0.000000pt}%
\definecolor{currentstroke}{rgb}{0.000000,0.000000,0.000000}%
\pgfsetstrokecolor{currentstroke}%
\pgfsetstrokeopacity{0.800000}%
\pgfsetdash{}{0pt}%
\pgfpathmoveto{\pgfqpoint{2.540910in}{0.999652in}}%
\pgfpathcurveto{\pgfqpoint{2.545028in}{0.999652in}}{\pgfqpoint{2.548978in}{1.001288in}}{\pgfqpoint{2.551890in}{1.004200in}}%
\pgfpathcurveto{\pgfqpoint{2.554802in}{1.007112in}}{\pgfqpoint{2.556438in}{1.011062in}}{\pgfqpoint{2.556438in}{1.015180in}}%
\pgfpathcurveto{\pgfqpoint{2.556438in}{1.019298in}}{\pgfqpoint{2.554802in}{1.023248in}}{\pgfqpoint{2.551890in}{1.026160in}}%
\pgfpathcurveto{\pgfqpoint{2.548978in}{1.029072in}}{\pgfqpoint{2.545028in}{1.030708in}}{\pgfqpoint{2.540910in}{1.030708in}}%
\pgfpathcurveto{\pgfqpoint{2.536792in}{1.030708in}}{\pgfqpoint{2.532842in}{1.029072in}}{\pgfqpoint{2.529930in}{1.026160in}}%
\pgfpathcurveto{\pgfqpoint{2.527018in}{1.023248in}}{\pgfqpoint{2.525381in}{1.019298in}}{\pgfqpoint{2.525381in}{1.015180in}}%
\pgfpathcurveto{\pgfqpoint{2.525381in}{1.011062in}}{\pgfqpoint{2.527018in}{1.007112in}}{\pgfqpoint{2.529930in}{1.004200in}}%
\pgfpathcurveto{\pgfqpoint{2.532842in}{1.001288in}}{\pgfqpoint{2.536792in}{0.999652in}}{\pgfqpoint{2.540910in}{0.999652in}}%
\pgfpathclose%
\pgfusepath{fill}%
\end{pgfscope}%
\begin{pgfscope}%
\pgfpathrectangle{\pgfqpoint{0.887500in}{0.275000in}}{\pgfqpoint{4.225000in}{4.225000in}}%
\pgfusepath{clip}%
\pgfsetbuttcap%
\pgfsetroundjoin%
\definecolor{currentfill}{rgb}{0.000000,0.000000,0.000000}%
\pgfsetfillcolor{currentfill}%
\pgfsetfillopacity{0.800000}%
\pgfsetlinewidth{0.000000pt}%
\definecolor{currentstroke}{rgb}{0.000000,0.000000,0.000000}%
\pgfsetstrokecolor{currentstroke}%
\pgfsetstrokeopacity{0.800000}%
\pgfsetdash{}{0pt}%
\pgfpathmoveto{\pgfqpoint{2.726506in}{0.928184in}}%
\pgfpathcurveto{\pgfqpoint{2.730624in}{0.928184in}}{\pgfqpoint{2.734574in}{0.929820in}}{\pgfqpoint{2.737486in}{0.932732in}}%
\pgfpathcurveto{\pgfqpoint{2.740398in}{0.935644in}}{\pgfqpoint{2.742034in}{0.939594in}}{\pgfqpoint{2.742034in}{0.943713in}}%
\pgfpathcurveto{\pgfqpoint{2.742034in}{0.947831in}}{\pgfqpoint{2.740398in}{0.951781in}}{\pgfqpoint{2.737486in}{0.954693in}}%
\pgfpathcurveto{\pgfqpoint{2.734574in}{0.957605in}}{\pgfqpoint{2.730624in}{0.959241in}}{\pgfqpoint{2.726506in}{0.959241in}}%
\pgfpathcurveto{\pgfqpoint{2.722388in}{0.959241in}}{\pgfqpoint{2.718438in}{0.957605in}}{\pgfqpoint{2.715526in}{0.954693in}}%
\pgfpathcurveto{\pgfqpoint{2.712614in}{0.951781in}}{\pgfqpoint{2.710978in}{0.947831in}}{\pgfqpoint{2.710978in}{0.943713in}}%
\pgfpathcurveto{\pgfqpoint{2.710978in}{0.939594in}}{\pgfqpoint{2.712614in}{0.935644in}}{\pgfqpoint{2.715526in}{0.932732in}}%
\pgfpathcurveto{\pgfqpoint{2.718438in}{0.929820in}}{\pgfqpoint{2.722388in}{0.928184in}}{\pgfqpoint{2.726506in}{0.928184in}}%
\pgfpathclose%
\pgfusepath{fill}%
\end{pgfscope}%
\begin{pgfscope}%
\pgfpathrectangle{\pgfqpoint{0.887500in}{0.275000in}}{\pgfqpoint{4.225000in}{4.225000in}}%
\pgfusepath{clip}%
\pgfsetbuttcap%
\pgfsetroundjoin%
\definecolor{currentfill}{rgb}{0.000000,0.000000,0.000000}%
\pgfsetfillcolor{currentfill}%
\pgfsetfillopacity{0.800000}%
\pgfsetlinewidth{0.000000pt}%
\definecolor{currentstroke}{rgb}{0.000000,0.000000,0.000000}%
\pgfsetstrokecolor{currentstroke}%
\pgfsetstrokeopacity{0.800000}%
\pgfsetdash{}{0pt}%
\pgfpathmoveto{\pgfqpoint{2.421414in}{0.964107in}}%
\pgfpathcurveto{\pgfqpoint{2.425532in}{0.964107in}}{\pgfqpoint{2.429482in}{0.965743in}}{\pgfqpoint{2.432394in}{0.968655in}}%
\pgfpathcurveto{\pgfqpoint{2.435306in}{0.971567in}}{\pgfqpoint{2.436943in}{0.975517in}}{\pgfqpoint{2.436943in}{0.979635in}}%
\pgfpathcurveto{\pgfqpoint{2.436943in}{0.983753in}}{\pgfqpoint{2.435306in}{0.987703in}}{\pgfqpoint{2.432394in}{0.990615in}}%
\pgfpathcurveto{\pgfqpoint{2.429482in}{0.993527in}}{\pgfqpoint{2.425532in}{0.995163in}}{\pgfqpoint{2.421414in}{0.995163in}}%
\pgfpathcurveto{\pgfqpoint{2.417296in}{0.995163in}}{\pgfqpoint{2.413346in}{0.993527in}}{\pgfqpoint{2.410434in}{0.990615in}}%
\pgfpathcurveto{\pgfqpoint{2.407522in}{0.987703in}}{\pgfqpoint{2.405886in}{0.983753in}}{\pgfqpoint{2.405886in}{0.979635in}}%
\pgfpathcurveto{\pgfqpoint{2.405886in}{0.975517in}}{\pgfqpoint{2.407522in}{0.971567in}}{\pgfqpoint{2.410434in}{0.968655in}}%
\pgfpathcurveto{\pgfqpoint{2.413346in}{0.965743in}}{\pgfqpoint{2.417296in}{0.964107in}}{\pgfqpoint{2.421414in}{0.964107in}}%
\pgfpathclose%
\pgfusepath{fill}%
\end{pgfscope}%
\begin{pgfscope}%
\pgfpathrectangle{\pgfqpoint{0.887500in}{0.275000in}}{\pgfqpoint{4.225000in}{4.225000in}}%
\pgfusepath{clip}%
\pgfsetbuttcap%
\pgfsetroundjoin%
\definecolor{currentfill}{rgb}{0.000000,0.000000,0.000000}%
\pgfsetfillcolor{currentfill}%
\pgfsetfillopacity{0.800000}%
\pgfsetlinewidth{0.000000pt}%
\definecolor{currentstroke}{rgb}{0.000000,0.000000,0.000000}%
\pgfsetstrokecolor{currentstroke}%
\pgfsetstrokeopacity{0.800000}%
\pgfsetdash{}{0pt}%
\pgfpathmoveto{\pgfqpoint{2.607461in}{0.888360in}}%
\pgfpathcurveto{\pgfqpoint{2.611580in}{0.888360in}}{\pgfqpoint{2.615530in}{0.889996in}}{\pgfqpoint{2.618441in}{0.892908in}}%
\pgfpathcurveto{\pgfqpoint{2.621353in}{0.895820in}}{\pgfqpoint{2.622990in}{0.899770in}}{\pgfqpoint{2.622990in}{0.903888in}}%
\pgfpathcurveto{\pgfqpoint{2.622990in}{0.908006in}}{\pgfqpoint{2.621353in}{0.911956in}}{\pgfqpoint{2.618441in}{0.914868in}}%
\pgfpathcurveto{\pgfqpoint{2.615530in}{0.917780in}}{\pgfqpoint{2.611580in}{0.919416in}}{\pgfqpoint{2.607461in}{0.919416in}}%
\pgfpathcurveto{\pgfqpoint{2.603343in}{0.919416in}}{\pgfqpoint{2.599393in}{0.917780in}}{\pgfqpoint{2.596481in}{0.914868in}}%
\pgfpathcurveto{\pgfqpoint{2.593569in}{0.911956in}}{\pgfqpoint{2.591933in}{0.908006in}}{\pgfqpoint{2.591933in}{0.903888in}}%
\pgfpathcurveto{\pgfqpoint{2.591933in}{0.899770in}}{\pgfqpoint{2.593569in}{0.895820in}}{\pgfqpoint{2.596481in}{0.892908in}}%
\pgfpathcurveto{\pgfqpoint{2.599393in}{0.889996in}}{\pgfqpoint{2.603343in}{0.888360in}}{\pgfqpoint{2.607461in}{0.888360in}}%
\pgfpathclose%
\pgfusepath{fill}%
\end{pgfscope}%
\begin{pgfscope}%
\pgfpathrectangle{\pgfqpoint{0.887500in}{0.275000in}}{\pgfqpoint{4.225000in}{4.225000in}}%
\pgfusepath{clip}%
\pgfsetbuttcap%
\pgfsetroundjoin%
\definecolor{currentfill}{rgb}{0.000000,0.000000,0.000000}%
\pgfsetfillcolor{currentfill}%
\pgfsetfillopacity{0.800000}%
\pgfsetlinewidth{0.000000pt}%
\definecolor{currentstroke}{rgb}{0.000000,0.000000,0.000000}%
\pgfsetstrokecolor{currentstroke}%
\pgfsetstrokeopacity{0.800000}%
\pgfsetdash{}{0pt}%
\pgfpathmoveto{\pgfqpoint{2.487775in}{0.850671in}}%
\pgfpathcurveto{\pgfqpoint{2.491894in}{0.850671in}}{\pgfqpoint{2.495844in}{0.852307in}}{\pgfqpoint{2.498756in}{0.855219in}}%
\pgfpathcurveto{\pgfqpoint{2.501667in}{0.858131in}}{\pgfqpoint{2.503304in}{0.862081in}}{\pgfqpoint{2.503304in}{0.866199in}}%
\pgfpathcurveto{\pgfqpoint{2.503304in}{0.870317in}}{\pgfqpoint{2.501667in}{0.874267in}}{\pgfqpoint{2.498756in}{0.877179in}}%
\pgfpathcurveto{\pgfqpoint{2.495844in}{0.880091in}}{\pgfqpoint{2.491894in}{0.881727in}}{\pgfqpoint{2.487775in}{0.881727in}}%
\pgfpathcurveto{\pgfqpoint{2.483657in}{0.881727in}}{\pgfqpoint{2.479707in}{0.880091in}}{\pgfqpoint{2.476795in}{0.877179in}}%
\pgfpathcurveto{\pgfqpoint{2.473883in}{0.874267in}}{\pgfqpoint{2.472247in}{0.870317in}}{\pgfqpoint{2.472247in}{0.866199in}}%
\pgfpathcurveto{\pgfqpoint{2.472247in}{0.862081in}}{\pgfqpoint{2.473883in}{0.858131in}}{\pgfqpoint{2.476795in}{0.855219in}}%
\pgfpathcurveto{\pgfqpoint{2.479707in}{0.852307in}}{\pgfqpoint{2.483657in}{0.850671in}}{\pgfqpoint{2.487775in}{0.850671in}}%
\pgfpathclose%
\pgfusepath{fill}%
\end{pgfscope}%
\begin{pgfscope}%
\definecolor{textcolor}{rgb}{0.150000,0.150000,0.150000}%
\pgfsetstrokecolor{textcolor}%
\pgfsetfillcolor{textcolor}%
\pgftext[x=3.000000in,y=4.583333in,,base]{\color{textcolor}\sffamily\fontsize{12.000000}{14.400000}\selectfont Interpolated Model of \(\displaystyle C_L\) for SD7032 Airfoil}%
\end{pgfscope}%
\begin{pgfscope}%
\pgfsetbuttcap%
\pgfsetmiterjoin%
\definecolor{currentfill}{rgb}{0.917647,0.917647,0.949020}%
\pgfsetfillcolor{currentfill}%
\pgfsetfillopacity{0.800000}%
\pgfsetlinewidth{1.003750pt}%
\definecolor{currentstroke}{rgb}{0.800000,0.800000,0.800000}%
\pgfsetstrokecolor{currentstroke}%
\pgfsetstrokeopacity{0.800000}%
\pgfsetdash{}{0pt}%
\pgfpathmoveto{\pgfqpoint{3.837111in}{4.164806in}}%
\pgfpathlineto{\pgfqpoint{5.005556in}{4.164806in}}%
\pgfpathquadraticcurveto{\pgfqpoint{5.036111in}{4.164806in}}{\pgfqpoint{5.036111in}{4.195361in}}%
\pgfpathlineto{\pgfqpoint{5.036111in}{4.393056in}}%
\pgfpathquadraticcurveto{\pgfqpoint{5.036111in}{4.423611in}}{\pgfqpoint{5.005556in}{4.423611in}}%
\pgfpathlineto{\pgfqpoint{3.837111in}{4.423611in}}%
\pgfpathquadraticcurveto{\pgfqpoint{3.806556in}{4.423611in}}{\pgfqpoint{3.806556in}{4.393056in}}%
\pgfpathlineto{\pgfqpoint{3.806556in}{4.195361in}}%
\pgfpathquadraticcurveto{\pgfqpoint{3.806556in}{4.164806in}}{\pgfqpoint{3.837111in}{4.164806in}}%
\pgfpathclose%
\pgfusepath{stroke,fill}%
\end{pgfscope}%
\begin{pgfscope}%
\pgfsetbuttcap%
\pgfsetroundjoin%
\definecolor{currentfill}{rgb}{0.000000,0.000000,0.000000}%
\pgfsetfillcolor{currentfill}%
\pgfsetfillopacity{0.800000}%
\pgfsetlinewidth{0.000000pt}%
\definecolor{currentstroke}{rgb}{0.000000,0.000000,0.000000}%
\pgfsetstrokecolor{currentstroke}%
\pgfsetstrokeopacity{0.800000}%
\pgfsetdash{}{0pt}%
\pgfsys@defobject{currentmarker}{\pgfqpoint{-0.015528in}{-0.015528in}}{\pgfqpoint{0.015528in}{0.015528in}}{%
\pgfpathmoveto{\pgfqpoint{0.000000in}{-0.015528in}}%
\pgfpathcurveto{\pgfqpoint{0.004118in}{-0.015528in}}{\pgfqpoint{0.008068in}{-0.013892in}}{\pgfqpoint{0.010980in}{-0.010980in}}%
\pgfpathcurveto{\pgfqpoint{0.013892in}{-0.008068in}}{\pgfqpoint{0.015528in}{-0.004118in}}{\pgfqpoint{0.015528in}{0.000000in}}%
\pgfpathcurveto{\pgfqpoint{0.015528in}{0.004118in}}{\pgfqpoint{0.013892in}{0.008068in}}{\pgfqpoint{0.010980in}{0.010980in}}%
\pgfpathcurveto{\pgfqpoint{0.008068in}{0.013892in}}{\pgfqpoint{0.004118in}{0.015528in}}{\pgfqpoint{0.000000in}{0.015528in}}%
\pgfpathcurveto{\pgfqpoint{-0.004118in}{0.015528in}}{\pgfqpoint{-0.008068in}{0.013892in}}{\pgfqpoint{-0.010980in}{0.010980in}}%
\pgfpathcurveto{\pgfqpoint{-0.013892in}{0.008068in}}{\pgfqpoint{-0.015528in}{0.004118in}}{\pgfqpoint{-0.015528in}{0.000000in}}%
\pgfpathcurveto{\pgfqpoint{-0.015528in}{-0.004118in}}{\pgfqpoint{-0.013892in}{-0.008068in}}{\pgfqpoint{-0.010980in}{-0.010980in}}%
\pgfpathcurveto{\pgfqpoint{-0.008068in}{-0.013892in}}{\pgfqpoint{-0.004118in}{-0.015528in}}{\pgfqpoint{0.000000in}{-0.015528in}}%
\pgfpathclose%
\pgfusepath{fill}%
}%
\begin{pgfscope}%
\pgfsys@transformshift{4.020444in}{4.295660in}%
\pgfsys@useobject{currentmarker}{}%
\end{pgfscope}%
\end{pgfscope}%
\begin{pgfscope}%
\definecolor{textcolor}{rgb}{0.150000,0.150000,0.150000}%
\pgfsetstrokecolor{textcolor}%
\pgfsetfillcolor{textcolor}%
\pgftext[x=4.295444in,y=4.255556in,left,base]{\color{textcolor}\sffamily\fontsize{11.000000}{13.200000}\selectfont XFoil Data}%
\end{pgfscope}%
\end{pgfpicture}%
\makeatother%
\endgroup%
}
    \caption{An interpolated model for airfoil $C_L$ from a multidimensional dataset computed by XFoil \cite{xfoil}.}
    \label{fig:interpolated-xfoil-3d}
\end{figure}

This lookup-table approach to surrogate modeling is quite convenient to use when compared to the fitting approach described in Section \ref{sect:fitting}. With a high-quality dataset, little-to-no engineering effort is need to create a surrogate model; by contrast, the fitting approach requires engineering intuition about the type of analytical model that best describes the data.

However, this interpolation approach is not without drawbacks. First, the b-spline interpolators implemented in AeroSandbox have no ability to reject noise - surrogate models will always pass exactly through input data points (as these form the knots of the spline). Splines also do not preserve monotonicity of the dataset: on datasets with low signal-to-noise ratios\footnote{In particular, this is an issue on some experimental datasets}, this can lead to wild oscillations as the interpolator attempts to model the random noise. This can even occur on datasets with moderate noise if the input space is not approprately sampled\footnote{Often, typical linear full-factorial sampling is quite a poor choice; using Chebyshev nodes often produces far superior results for a given number of data points. A fuller description is given in Chapter 13 of \cite{koch2019}.}, a numerical problem known as Runge's phenomenon. Noisy datasets can also be effectively used if they are pre-processed by smoothing with several passes of a Laplacian (i.e. "heat equation") kernel. This process, sometimes referred to as the application of a \textit{Gaussian blur}, yields more well-behaved interpolators at the cost of accuracy to the original dataset.

The lookup-table approach also necessitates the use of structured (i.e. gridded in $n$ dimensions) data, rather than unstructured (i.e. "point cloud") data. This is because the interpolation consists of a collection of piecewise polynomials, and these polynomials must be linked back to specific knot points (drawn from the dataset) to determine their shape. With unstructured data, it is quite difficult to know which knot points should be used for a given polynomial.

Most synthetic datasets (e.g. those from expensive, high-fidelity computational analysis) tend to be structured, so this is not a problem. Conveniently, these datasets also tend to lend themselves well to interpolation as they generally have no random noise.

On the other hand, many experimental datasets tend to be unstructured, which prohibits the use of AeroSandbox interpopation as-is. Hence, fitting is generally preferred. Fitting also tends to be a superior choice for experimental datasets due to their better noise rejection.

In the event that interpolation is desired on an unstructured dataset, an approach that has been proven successful is to interpolate the unstructured dataset onto a structured grid using a radial basis function (RBF) interpolator, at which point the AeroSandbox spline interpolator in \mintinline{python}{asb.InterpolatedModel} can be used.