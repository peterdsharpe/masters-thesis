
\title{AeroSandbox: A Differentiable Framework for Aircraft Design Optimization}

\author{Peter Sharpe}
%\prevdegrees{B.S., Washington University in St. Louis (2014)}
% If you wish to list your previous degrees on the cover page, use the 
% previous degrees command:
%       \prevdegrees{A.A., Harvard University (1985)}
% You can use the \\ command to list multiple previous degrees
%       \prevdegrees{B.S., University of California (1978) \\
%                    S.M., Massachusetts Institute of Technology (1981)}
\department{Department of Aeronautics and Astronautics}

% If the thesis is for two degrees simultaneously, list them both
% separated by \and like this:
% \degree{Doctor of Philosophy \and Master of Science}
\degree{Master of Science in Aeronautics and Astronautics}

% As of the 2007-08 academic year, valid degree months are September, 
% February, or June.  The default is June.
\degreemonth{September}
\degreeyear{2021}
\thesisdate{August 17, 2021}

%% By default, the thesis will be copyrighted to MIT.  If you need to copyright
%% the thesis to yourself, just specify the `vi' documentclass option.  If for
%% some reason you want to exactly specify the copyright notice text, you can
%% use the \copyrightnoticetext command.  
%\copyrightnoticetext{\copyright IBM, 1990.  Do not open till Xmas.}

% If there is more than one supervisor, use the \supervisor command
% once for each.
\supervisor{R. John Hansman}{T. Wilson Professor in Aeronautics}

% This is the department committee chairman, not the thesis committee
% chairman.  You should replace this with your Department's Committee
% Chairman.
\chairman{Jonathan How}{Professor, Aeronautics and Astronautics\\Chair, Graduate Program Committee}

% Make the titlepage based on the above information.  If you need
% something special and can't use the standard form, you can specify
% the exact text of the titlepage yourself.  Put it in a titlepage
% environment and leave blank lines where you want vertical space.
% The spaces will be adjusted to fill the entire page.  The dotted
% lines for the signatures are made with the \signature command.
\maketitle

% The abstractpage environment sets up everything on the page except
% the text itself.  The title and other header material are put at the
% top of the page, and the supervisors are listed at the bottom.  A
% new page is begun both before and after.  Of course, an abstract may
% be more than one page itself.  If you need more control over the
% format of the page, you can use the abstract environment, which puts
% the word "Abstract" at the beginning and single spaces its text.

%% You can either \input (*not* \include) your abstract file, or you can put
%% the text of the abstract directly between the \begin{abstractpage} and
%% \end{abstractpage} commands.

% First copy: start a new page, and save the page number.
\cleardoublepage
% Uncomment the next line if you do NOT want a page number on your
% abstract and acknowledgments pages.
% \pagestyle{empty}
\setcounter{savepage}{\thepage}
\begin{abstractpage}
% $Log: abstract.tex,v $
% Revision 1.1  93/05/14  14:56:25  starflt
% Initial revision
% 
% Revision 1.1  90/05/04  10:41:01  lwvanels
% Initial revision
% 
%
%% The text of your abstract and nothing else (other than comments) goes here.
%% It will be single-spaced and the rest of the text that is supposed to go on
%% the abstract page will be generated by the abstractpage environment.  This
%% file should be \input (not \include 'd) from cover.tex.
Lorem ipsum dolor sit amet, consectetur adipiscing elit. Nam quis neque et erat laoreet finibus at ac leo. Curabitur pellentesque, diam quis dignissim finibus, enim dui feugiat leo, nec porttitor sapien mi ac felis. Nam aliquam pretium nibh, quis dapibus dolor gravida sit amet. Cras porttitor dui quis elementum pulvinar. Nulla id pulvinar massa. Nullam ut diam non lorem venenatis faucibus. Vivamus lacus ante, pellentesque vitae nisl sit amet, bibendum facilisis purus.

\end{abstractpage}

% Additional copy: start a new page, and reset the page number.  This way,
% the second copy of the abstract is not counted as separate pages.
% Uncomment the next 6 lines if you need two copies of the abstract
% page.
% \setcounter{page}{\thesavepage}
% \begin{abstractpage}
% % $Log: abstract.tex,v $
% Revision 1.1  93/05/14  14:56:25  starflt
% Initial revision
% 
% Revision 1.1  90/05/04  10:41:01  lwvanels
% Initial revision
% 
%
%% The text of your abstract and nothing else (other than comments) goes here.
%% It will be single-spaced and the rest of the text that is supposed to go on
%% the abstract page will be generated by the abstractpage environment.  This
%% file should be \input (not \include 'd) from cover.tex.
Lorem ipsum dolor sit amet, consectetur adipiscing elit. Nam quis neque et erat laoreet finibus at ac leo. Curabitur pellentesque, diam quis dignissim finibus, enim dui feugiat leo, nec porttitor sapien mi ac felis. Nam aliquam pretium nibh, quis dapibus dolor gravida sit amet. Cras porttitor dui quis elementum pulvinar. Nulla id pulvinar massa. Nullam ut diam non lorem venenatis faucibus. Vivamus lacus ante, pellentesque vitae nisl sit amet, bibendum facilisis purus.

% \end{abstractpage}

\cleardoublepage

\section*{Acknowledgments}

This thesis and my associated research endeavours would not have been possible without the support of so many people, and for that I am so thankful.

My advisor, Professor John Hansman, is the most fiercely supportive advocate I could have asked for. Our Socratic-method discussions of various engineering design questions has helped me grow so much as a researcher, and I am always impressed by his depth of insight across such a wide number of aerospace disciplines. Whether we are hanging out in lab, flying his 152 around Mt. Wachusett, or having a BBQ in his backyard, I am very thankful for Prof. Hansman's mentorship.

I would also like to thank Professor Mark Drela for various insightful conversations over the years. His contributions to aerodynamics and flight vehicle design are unparalleled, and the two classes I've taken with him have proven to be exceptionally useful.

I'm also so thankful for my lively labmates in ICAT - their friendliness and warmth could melt any Boston winter. In particular, the motley crew of 35-217 has been my cornerstone through thick and thin: Matt Vernacchia, Kelly Mathesius (and the delicious carrot cakes she has baked for my birthdays), and Maddie Jansson; as well as venerated previous resident Tony Tao (Gone but never forgotten!).

To so many new MIT friends and colleagues - Ara Mahseredijan, Chris Courtin, Annick Dewald, Julia Gaubatz, Trevor Long, Chelsea Onyeador, Charles Dawson, Katie Carroll, Sydney Dolan, Grace Wijaya, Kelvin Leung, and many more: thank you for all of the good times, inspiration, encouragement, and trips to the Muddy.

To dear friends made many long years ago - Christophe Foyer, Cameron Urban, Austin Stover (and many more from the Wash. U. Design/Build/Fly team), Nathan Duncan, Kevin Hainline, and John Duffy: as the saying goes, "Of oil, wine, and friends, the oldest is the best."

Special thanks to Marta Manzin for her unwavering kindness and optimism, her inspiring maker spirit, free Italian lessons, and so much more. Grazie mille, Marta! To the other residents of SSDH - Marcus, Andrea, Banti, Sadun, and Allie - thank you for welcoming me into your home. Your company has made weathering the pandemic lockdown so much more lively.

Thanks to the East Campus community and especially all of my residents on Putz, my home.

Thank you to my mom, dad, and sister Julie for their support, love, and teaching from my earliest days. This work and so much more would not have been possible without you.

On a more technical note, this thesis would not be possible without the open-source scientific computing community. To the maintainers of NumPy/SciPy, CasADi, IPOPT, Python, and the entire underlying stacks thereof, thank you for your selfless contributions (collectively, millions of person-hours) to scientific research everywhere.

% High school/college teachers

% MIT License and why

% NDSEG


%%%%%%%%%%%%%%%%%%%%%%%%%%%%%%%%%%%%%%%%%%%%%%%%%%%%%%%%%%%%%%%%%%%%%%
% -*-latex-*-
