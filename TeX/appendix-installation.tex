\chapter{Installation Instructions and Basic Usage}
\label{chapter:installation}

AeroSandbox is very lightweight, and it can be installed on any machine running a modern operating system (Windows, Mac, or Linux) in just a few minutes using the following simple steps.


\section{Installing Python}

AeroSandbox is a Python package, so an installation of Python is required. AeroSandbox supports Python 3.7 and above, although the latest release is always recommended.

Python can be downloaded and installed from the \url{Python.org} website. However, is is recommended to instead install Python through the \textit{Anaconda Distribution}, a distribution of Python that is popular in the scientific computing community and comes pre-packaged with many performance-optimized libraries \cite{anaconda}. The Anaconda Distribution is free and open-source for individual use and can be downloaded from the \url{Anaconda.com} website.

When installing, it is recommended that you add the Anaconda Python executable (and all other executables in its \texttt{/bin/} binaries folder) to the system \texttt{PATH}. This can be done through a checkbox in the installer, or it can be performed manually after installation.

Finally, correct installation can be verified by opening up a new terminal\footnote{On Windows, Command Prompt is one such terminal.}. If Python has been installed correctly and added to the PATH, you will see a similar printout upon giving the command \mintinline{bash}{python}:

\begin{minted}[breaklines]{text}
C:\Users\User>python
Python 3.8.8 | packaged by conda-forge | (default, Feb 20 2021, 15:50:08) [MSC v.1916 64 bit (AMD64)] on win32
Type "help", "copyright", "credits" or "license" for more information.
>>>
\end{minted}

To return to the terminal, you can give the \mintinline{python}{quit()} command.


\section{Installing AeroSandbox}

Once Python has been installed, installation of AeroSandbox is very straightforward. Open up a new terminal window and give the following command:
\begin{minted}{text}
pip install aerosandbox
\end{minted}

Assuming the machine has a working internet connection, this will download AeroSandbox from the Python Package Index (PyPI) and install it. This concludes a normal installation.


\section{Basic Usage}

In order to use AeroSandbox, use a text editor or Python IDE\footnote{PyCharm and VSCode being two popular choices} to create a Python script file\footnote{\texttt{.py} file extension}. Code the optimization problem that you would like to solve; an example is given in Listing \ref{listing:constrained-rosen}. After saving the file, call the Python interpreter on this file by giving the following terminal command:
\begin{minted}{text}
python "path/to/my/file.py"
\end{minted}

Alternatively, Python IDEs often have a "Build" or "Run File" command that can execute this code via the Python interpreter as well.


\section{Developer Installation}

For developers, the open-source AeroSandbox GitHub repository is available at \url{https://github.com/peterdsharpe/AeroSandbox}.

Any bugs and installation problems should also be reported via issue tickets at this GitHub repository.


\section{Versioning}

The version of AeroSandbox used in the code listings presented throughout this thesis is v3.0.17. Because of AeroSandbox's semantic versioning, compatibility is expected for all future AeroSandbox v3.x.x versions.

The version of a user installation can be checked at any time with the following Python code:

\begin{minted}{python}
import aerosandbox as asb
print(asb.__version__)
\end{minted}

Please report any compatibility problems at the aforementioned GitHub repository.